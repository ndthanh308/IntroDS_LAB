\documentclass[10pt,twocolumn,letterpaper]{article}
\usepackage{iccv}
\usepackage{times}
\usepackage{epsfig}
\usepackage{graphicx}
\usepackage{csquotes}
\usepackage{verbatim}
\usepackage{color,soul}
\usepackage{epsfig}
\usepackage{amsmath}
\usepackage{amssymb}
\usepackage{datetime2}
\usepackage{bm}
\usepackage{microtype}      % microtypography
\usepackage{dcolumn}        % Align table columns on decimal point
\usepackage{url}            % simple URL typesetting
\usepackage{booktabs}       % professional-quality tables
\usepackage{amsfonts}       % blackboard math symbols
\usepackage{nicefrac}       % compact symbols for 1/2, etc.
\usepackage{enumitem}
\usepackage{float}
\usepackage{stfloats}
\usepackage{algorithm}
\usepackage[noend]{algpseudocode}
\usepackage[numbers,sort,compress]{natbib}
\usepackage{amsthm}
\usepackage{subfiles}
\usepackage{adjustbox}
\usepackage{subcaption}
\usepackage{makecell}
\usepackage[normalem]{ulem}
\usepackage{caption}
% \usepackage{nopageno}
\usepackage[flushmargin,para]{footmisc}
% \usepackage[pagebackref,breaklinks,colorlinks]{hyperref} % please import last
\usepackage[table]{xcolor}
\usepackage{pifont}% http://ctan.org/pkg/pifont
\usepackage{array}
\usepackage{arydshln} 
\usepackage{tabularx}
\usepackage{booktabs}

\usepackage{multirow}
\newcommand{\PreserveBackslash}[1]{\let\temp=\\#1\let\\=\temp}
\newcolumntype{C}[1]{>{\PreserveBackslash\centering}p{#1}}
\newcolumntype{R}[1]{>{\PreserveBackslash\raggedleft}p{#1}}
\newcolumntype{L}[1]{>{\PreserveBackslash\raggedright}p{#1}}
\newcommand{\cmark}{\ding{51}}%
\newcommand{\xmark}{\ding{55}}%

\newcommand{\methodname}{JOTR\xspace}
\definecolor{lightgray}{gray}{0.97}
\definecolor{lightblue}{rgb}{0.93,0.95,1.0}
% Include other packages here, before hyperref.

% If you comment hyperref and then uncomment it, you should delete
% egpaper.aux before re-running latex.  (Or just hit 'q' on the first latex
% run, let it finish, and you should be clear).
\usepackage[pagebackref=true,breaklinks=true,letterpaper=true,colorlinks,bookmarks=false]{hyperref}
% \iccvfinalcopy

\usepackage[capitalize]{cleveref}
\crefname{section}{Sec.}{Secs.}
\Crefname{section}{Section}{Sections}
\Crefname{table}{Table}{Tables}
\crefname{table}{Tab.}{Tabs.}

\iccvfinalcopy % *** Uncomment this line for the final submission

\def\iccvPaperID{2019} % *** Enter the ICCV Paper ID here
\def\httilde{\mbox{\tt\raisebox{-.5ex}{\symbol{126}}}}

% Pages are numbered in submission mode, and unnumbered in camera-ready
% \ificcvfinal\pagestyle{empty}\fi

\begin{document}
\title{JOTR: 3D Joint Contrastive Learning with Transformers for \\ Occluded Human Mesh Recovery}
\author{
		Jiahao Li$^{1,2*}$, Zongxin Yang$^{1}$, Xiaohan Wang$^1$, Jianxin Ma$^2$, Chang Zhou$^2$, Yi Yang$^1$ \\
		$^1$ ReLER, CCAI, Zhejiang University \space \space \space $^2$ DAMO Academy, Alibaba Group \\
	}
\twocolumn[{%
		\renewcommand\twocolumn[1][]{#1}%
		\maketitle
		\begin{center}
			\newcommand{\teaserwidth}{\textwidth}
			\vspace{-6mm}
			\centerline{
				% Figure removed
			}
			\vspace{-0.1in}
			\captionof{figure}{
                (a) Current methods for human mesh recovery could be classified into two categories: AvgFeatReg~\cite{SPIN,HMR} and JointFeatReg~\cite{ROMP,zhang2021pymaf,3dCrwodNet}, which focus on improving 2D alignment by using average 2D features (feat.) for regression and employing sampled 2D joint feature for regression, respectively.
                In contrast, our proposed novel techniques (\ie, FusionFeatReg), fuse 2D and 3D features for regression to enhance both 2D and 3D alignment. (b) Moreover, to provide global supervision for the entire 3D space, we introduce a 3D joint contrastive learning method, which stands in contrast to previous approaches that solely apply 3D joints as local supervision.
			}
			\vspace{-4mm}
			\label{fig:fig_1}
		\end{center}%
}]
% \maketitle

%%%%%%%%% TITLE
% Remove page # from the first page of camera-ready.
% \ificcvfinal\thispagestyle{empty}\fi
\renewcommand{\thefootnote}{*}
\footnotetext{This work was done during Jiahao's internship at Alibaba.}
\begin{abstract}
Graph Neural Networks (GNNs) have proven to be effective in processing and learning from graph-structured data.
However, previous works mainly focused on understanding single graph inputs while many real-world applications require pair-wise analysis for graph-structured data (e.g., scene graph matching, code searching, and drug-drug interaction prediction).
To this end, recent works have shifted their focus to learning the interaction between pairs of graphs.
Despite their improved performance, these works were still limited in that the interactions were considered at the node-level, resulting in high computational costs and suboptimal performance.
To address this issue, we propose a novel and efficient graph-level approach for extracting interaction representations using co-attention in graph pooling. 
Our method, Co-Attention Graph Pooling (CAGPool), exhibits competitive performance relative to existing methods in both classification and regression tasks using real-world datasets, while maintaining lower computational complexity.

\end{abstract}

\section{Introduction}
Deep learning models have been widely used in many applications.
For example, BERT~\citep{devlin_bert_2019}, GPT-3~\citep{brown_language_2020}, and T5~\citep{raffel_exploring_2020} achieved state-of-the-art~(SOTA) results on different natural language processing~(NLP) tasks. 
For computer vision~(CV), Transformer-like models such as ViT~\citep{dosovitskiy_image_2021} and Swin Transformer~\citep{liu_swin_2021} deliver excellent accuracy performance upon multiple tasks. 


At the same time, training deep learning models has been a critical problem troubling the community due to the long training time, especially for those large models with billions of parameters~\citep{brown_language_2020}. 
In order to enhance the training efficiency, researchers propose some manually designed parallel training strategies~\citep{narayanan_efficient_2021,shazeer_mesh-tensorflow_2018,xu_gspmd_2021}. 
However, selecting, tuning, and combining these strategies require extensive domain knowledge in deep learning models and hardware environments. With the increasing diversity of modern hardware architectures~\cite{flynn_very_1966,flynn_computer_1972} and the rapid development of deep learning models, these manually designed approaches are bringing heavier burdens to developers. 
Hence, \emph{automatic parallelism} is introduced to automate the parallel strategy searching for training models.


There are two main categories of parallelism in deep learning models: inter-layer parallelism~\citep{huang_gpipe_2019,narayanan_pipedream_2019,narayanan_memory-efficient_2021,fan_dapple_2021,li_chimera_2021,lepikhin_gshard_2021,du_glam_2022,fedus_switch_2022} and intra-layer parallelism~\citep{li_pytorch_2020,narayanan_efficient_2021,rasley_deepspeed_2020,fairscale_authors_fairscale_2021}. 
Inter-layer parallelism partitions the model into disjoint sets on different devices without slicing tensors. 
Alternatively, intra-layer parallelism partitions tensors in a layer along one or more axes and distributes them across different devices.


Current automatic parallelism techniques focus on optimizing strategies within these two categories. However, they treat these two categories separately. 
Some methods~\citep{zhao_vpipe_2022,jia_exploring_2018,cai_tensoropt_2022,wang_supporting_2019,jia_beyond_2019,schaarschmidt_automap_2021,liu_colossal-auto_2023} overlook potential opportunities for inter- or intra-layer parallelism, the others optimize inter- and intra-layer parallelism hierarchically and sequentially~\citep{narayanan_pipedream_2019,fan_dapple_2021,he_pipetransformer_2021,tarnawski_efficient_2020,tarnawski_piper_2021,zheng_alpa_2022}. 
As a result, current automatic parallelism techniques often fail to achieve the global optima and instead become trapped in local optima. 
Therefore, a unified inter- and intra-layer approach is needed to enhance the effectiveness of automatic parallelism.


This paper aims to find the optimal parallelism strategy while simultaneously considering inter- and intra-layer parallelism. 
It enables us to search in a more extensive strategy space where the globally optimal solution lurk. 
However, unifying inter- and intra-layer parallelism in automatic parallelism brings us two challenges. 
Firstly, to adopt a unified perspective on the inter- and intra-layer automatic parallelism, we should not formalize them with separate formulations as prior works. Therefore, how can we express these parallelism strategies in a unified formulation? 
Secondly, previous methods take a long time to obtain the solution with a limited strategy space. Therefore, how can we ensure that the best solution can be obtained in a reasonable time while expanding the strategy space?


To solve the above challenges, we propose UniAP. For the first challenge, UniAP adopts the mixed integer quadratic programming~(MIQP)~\citep{lazimy_mixed_1982} to search for the globally optimal parallel strategy automatically. 
It unifies the inter- and intra-layer automatic parallelism in a single MIQP formulation. 
For the second challenge, our complexity analysis and experimental results show that UniAP can obtain the globally optimal solution in a significantly shorter time.


The contributions of this paper are summarized as follows: 
\begin{itemize}
    \item We propose UniAP, the first framework to unify inter- and intra-layer automatic parallelism in model training.
    \item The optimal parallel strategies discovered by UniAP exhibit scalability on training throughput and strategy searching time.
    \item The experimental results show that UniAP speeds up model training on four Transformer-like models by up to 1.70$\times$ and reduces the strategy searching time by up to 16$\times$, compared with the SOTA method.
\end{itemize}

% \section{Preliminaries}
% \paragraph{Input feature attribution methods.}
% Consider a linear model $f(x) = w_1 x_1 + w_2 x_2$. To explain which feature is more important for predicting the value of f(x), we can compare their coefficients. If $w_1 = 1000$ and $w_2 = 0.01$, we can say that $x_1$ would be weighed more than $x_2$. This type of explanation assumes that the values of $x_1$ and $x_2$ are of the same order. This is true in the case of most inputs to the neural network models, for example image pixels. 
% Gradients are the general way of discussing the coefficient with respect to a particular feature to discuss its importance.
% \textbf{Element-wise product of gradient into input} \textsc{grad $\odot$ input} \cite{Shrikumar2016NotJA}, provides global importance about the input feature in the model's output. 
% \cite{} have used it show the feature importance in attention models. 
% , as compared to just the gradient.
% details of computing the attribution with math 
% Assume $\mathbf{x}$ is a real-valued input feature vector (for any modality). For discrete inputs, real-valued vector obtained after passing the feature through a look-up embedding.

% , but there is no clear superior attribution technique over another. 

% Instead of considering attributions over pixels, \textbf{XRAI} \cite{Kapishnikov2019XRAIBA} computes the effective attributions of integrated gradients over overly segmented image. The image is segmented based on similarity such as color, which makes the segment boundaries align with the edges. The segmentation is done at multiple scales to obtain a set of overlapping image segments.
% Assume that attribution mask over an image $I$ of size ${H\times W}$ is $A$ of the same size. 
% Using graph-based segmentations over multiple scale parameters, we obtain a set of segments $\mathcal{S}$. 
% Let a pixel be indexed by $i$ in the original image. For a segment $s$, the gain can be calculated by $g_s = \sum_{i \in s\backslash M} \frac{A_i}{area(s\backslash M)}$. 
% The segment with maximum gain is selected as  attribution to update the XRAI saliency set $\mathcal{M}$.
% The process is repeated with the remaining segments until the area of the mask set is equal to that of the image. 
% While this method seems to produce slightly better visual attributions over other variants of IG, it is sensitive to the size of segmentation scales and dilation factor. We consider  $XRAI(\cdot)$ to denote this attribution method for visual attribution analysis in \S \ref{subsec:visual_attr}.   
% which create grainy regions. 
% However, this method depends on the size of segmentation scales selected for computation. Further, dilation added to the final attribution masks to include edges may depict an inflated version of model's actual feature importance. 
% In this work, $XRAI(\cdot)$ denotes that this attribution method is applied.
%  \vspace{-0.5em}
\section{Related Work}
%  \vspace{-0.3em}
\label{sec:related_work}
\paragraph{Interpretability and explainability } Recent work in multimodal explainability in autonomous vehicles \cite{gilpin-2021-multimodal} uses symbolic explanations to debug and process outputs out of sub-components.
In contrast, we address the challenge of post-hoc multimodal interpretability for any existing end-to-end trained differentiable policies. \textsc{grad $\odot$ input}~\cite{Shrikumar2016NotJA},  a simple and modality-agnostic attribution that works on par with recent methods~\cite{Ancona2017AUV}. We use this method to compute multimodal attribution at inputs to the fusion layer to weigh how each modality contributes to the decision-making. 
% as it has been shown to work at par compared to the recent gradient-based attribution techniques~\cite{Ancona2017AUV}.
While \textsc{grad $\odot$ input} is a modality-agnostic starting point for attributions, 
it is not easy to understand, especially for images. Among recent works to improve visual attribution  \cite{Smilkov2017SmoothGradRN, Simonyan2014DeepIC, ig, sturmfels2020visualizing, Xu_2020_CVPR, Kapishnikov2021GuidedIG, Kapishnikov2019XRAIBA},  we use XRAI~\cite{Kapishnikov2019XRAIBA} for vision-specific analysis as it produces visually intuitive attributions by relying on regions, not individual pixels. 
% \cite{Smilkov2017SmoothGradRN} proposed ways to visually sharpen these vanilla gradient-based attributions. ~\cite{Simonyan2014DeepIC}  applying Gaussian noise perturbations over averaged over a sufficient number of samples.
% describe IG
% IG \cite{ig} and path methods have been studied as a cost-sharing method called Aumann-Shapley. 
% Attribution based on IG preserves axiomatic properties like \textit{sensitivity} and \textit{implementation invariance}.
% While IG aggregate the gradients on sampling inputs on a straight line between the baseline and the input, there are several paths possible in higher dimensional spaces and corresponding different attribution.
% Recent works build on IG to obtain more visually intuitive attributions, like SHAP Deep Explainer~\cite{sturmfels2020visualizing}, Blur IG ~\cite{Xu_2020_CVPR}, Guided IG~\cite{Kapishnikov2021GuidedIG} and XRAI~\cite{Kapishnikov2019XRAIBA}. Qualitatively, XRAI showed visually intuitive attributions by relying on regions and not individual pixels.  
% Interpretability using gradient-based attribution techniques is quite similar to adversarial attacks \cite{Goodfellow2015ExplainingAH} and adversarial training for robustness \cite{Bai2021RecentAI}, as both fundamentally rely on gradient of the input feature with respect to the output. 
% Do we need a figure to show the difference in attributions with just gradient vs gradxinput? 
\vspace{-0.8em}
\paragraph{Language-driven task benchmarks}

There are many benchmarks to study an agent’s ability to follow natural language instructions \cite{ALFRED20, padmakumar2022teach, gu2022vision,  mahmoudieh2022zero}. 
% While most existing settings apply only to either navigation \cite{} or manipulation \cite{}, 
% we conside one of the benchmarks which handles both, that is,
% navigation (Anderson et al., 2018; Chen
% et al., 2019), object manipulation (Misra et al.,
% 2017; Zhu et al., 2017) and embodied reasoning
% (Das et al., 2018a; Gordon et al., 2018). 
ALFRED \cite{ALFRED20} serves as a suitable testbed for this analysis as these tasks require both high reasoning for navigation and manipulation. ALFRED dataset provides visual demonstrations collected through PDDL planning in 3D Unity household environments and natural language description of the high-level goal and low-level instructions annotated by MTurkers. 
The benchmarks provide evaluation metrics for the overall task goal completion success rate (SR) and those weighted by the expert's path length (PLWSR)
% over seen and unseen tasks
and have reported a huge gap in the performance of learning algorithms and humans at these tasks. 
% ALFRED  is a benchmarking environment that provides natural language instructions annotated by MTurkers on egocentric visual sequences of actions taken for everyday household tasks. As ALFRED is a simulated environment on Unity3D game engine, the visual demonstrations are collected based on PDDL planning. 

\vspace{-0.8em}
\paragraph{End-to-end Learned Policies} We investigate the end-to-end learned policies for the task, such that, the gradient can be attributed at a task level. While we do not discuss modular yet differentiable policies like \cite{min2021film} \cite{DBLP:journals/corr/ZhouC15}, tying the gradient across multiple modular learned components is a direction for future work.
% as 
% tying the gradient among modular learned components in future work. 
In our work, we consider the checkpoints of policies trained on the ALFRED dataset. Broadly, these policies are of two types: (a) sequence-to-sequence models, that are, the one proposed with ALFRED dataset (Baseline) \cite{ALFRED20} and Modular Object-Centric Approach (MOCA) \cite{Singh2021FactorizingPA}, (b) transformer-based models, that are Episodic Transformers (ET) \cite{pashevich2021episodic}, and Hierarchical Tasks via Unified Transformers (HiTUT) \cite{Zhang2021HierarchicalTL}. Refer Table~\ref{tab:policiesarch} to compare architectural details \footnote{Previous action is modeled with learned embedding look-up in all these policies.}.
% \textbf{Seq2Seq(Baseline)} \cite{ALFRED20} is a single-stream Seq-to-Seq model with progress monitoring, processing the visual frames through  a frozen ResNet-18 encoder, language through bi-LSTM and soft attention and fusion of the latent visual, language and previous action encodings through an LSTM layer.
%%%% The visual frames are encoded by a frozen ResNet-18 encoder. The language instruction tokens are processed with a bi-LSTM and soft attention. The latent encodings for visual, language and previous action are passed through an LSTM.
% \textbf{MOCA} \cite{Singh2021FactorizingPA} presents a factorized model into two, i.e. interactive perception and action policy. The inputs to the action policy model are language encoding from bi-LSTM, visual embedding from a pretrained ResNet-18, and previous action embedding; all concatenated as input to an LSTM with residual connection.
% \textbf{Episodic Transformers} \cite{pashevich2021episodic} proposes a transformer architecture that encodes the language instructions and the sequence of visual observations and actions to predict subsequent actions per visual frame. Visual observations are processed through pretrained ResNet-50, language tokens passed through a transformer encoder pre-trained with synthetic language targets,  and action are encoded by embedding look-up. 

% Please add the following required packages to your document preamble:
% \usepackage{booktabs}
% Please add the following required packages to your document preamble:
% \usepackage{booktabs}
% Please add the following required packages to your document preamble:
% \usepackage{booktabs}
\begin{table}[t]
\centering
%  \vspace{-1em}
\caption{Policies trained on ALFRED Dataset and their architectures for each modality}
\label{tab:policiesarch}
\begin{tabular}{@{}llll@{}}
\toprule
Policies & Visual                                                                       & Language                       & Fusion                                                                   \\ \midrule
Baseline \cite{ALFRED20} & Frozen ResNet-18                                                             & Learned Embedding, Bi-LSTM     & LSTM                                                                     \\
MOCA \cite{Singh2021FactorizingPA}    & \begin{tabular}[c]{@{}l@{}}Frozen ResNet-18\\ + Dynamic Filters\end{tabular} & Learned Embedding, Bi-LSTM     & \begin{tabular}[c]{@{}l@{}}LSTM with \\ residual connection\end{tabular} \\
ET \cite{pashevich2021episodic}      & Frozen ResNet-50                                                             & Learned Embedding, Transformer & Transformer Encoder                                                      \\
HiTUT \cite{Zhang2021HierarchicalTL}   & Frozen MaskRCNN                                                              & Learned Embedding, FC, LN      & Transformer Encoder                                                  \\ \bottomrule
\end{tabular}
\vspace{-0.2em}
\end{table}
% EmBERT




 
%  provide spurious 
%  explanations and 
%  may not 
%  In cases where the attribution may 
%  this method depends on the underlying attribution methods such as IG. 

\section{Methodology}
\label{sec:method}

\subsection{Overview}
\label{sec:method_fmwk}

As shown in~\cref{fig:method_fmwk}, the proposed unsupervised MOT framework is trained with the widely-used contrastive learning technique~\cite{chen2020simple,he2020momentum}. 
\lk{Specifically, for multi-object tracking}, objects within the tracklet ($\boldsymbol{k}_{+}$) should be pulled together and different tracklets ($\boldsymbol{k}_{-}$) should be separated. It can be mathematically formulated as:

\begin{equation}
% \begin{split}
    \mathcal{L}_{cl}( \boldsymbol{q}; \boldsymbol{k}_{+}; \boldsymbol{k}_{-} )= 
    - \log \frac{\exp(\boldsymbol{q} \cdot \boldsymbol{k}_{+} / \epsilon)}{\sum_{i}\exp(\boldsymbol{q} \cdot \boldsymbol{k}_{i} / \epsilon)}
  \label{eq:method_nce}
% \end{split}  
\end{equation}

\noindent where $\mathcal{L}_{cl}$ denotes the InfoNCE~\cite{oord2018representation} loss function, and $\epsilon$ is the temperature hyper-parameter~\cite{wu2018unsupervised}. 
Within a video, following the unsupervised tracking fashion~\cite{liu2022online,shuai2022id}, the positive and negative keys mainly come from two sources, \ie pseudo-labeled historical frame and self-augmented frame. 

\lk{However, two issues occur: (1) the uncertainty reduces the accuracy of pseudo-tracklets and (2) the randomly augmented samples fail to learn the inter-frame consistency.} 
We argue the above issues are not independent. 
\lk{By leveraging the uncertainty in turn,} the accurate pseudo-tracklets can guide the qualified positive and negative augmentations.

To address these two issues, we propose an uncertainty-aware pseudo-tracklet labeling strategy in \cref{sec:method_uoap}, which integrates a verification-and-rectification mechanism into the tracklet generation. Our method significantly improves the accuracy of pseudo-identities, especially in long-term interval. 
Then we propose a tracklet-guided augmentation strategy in \cref{sec:method_ada_aug}, which brings the temporary information into spatial augmentation. The augmented samples simulates the objects' motion. A hierarchical uncertainty-based sampling strategy is proposed for hard sample mining. More details are described in the following section.


\subsection{Uncertainty-aware Tracklet-Labeling}
\label{sec:method_uoap}

Accurate pseudo tracklet is critical in \liuk{learning feature consistency}. 
However, without manual annotation, \lk{the aggravated uncertainty makes} the tracklet-labeling a huge challenge due to various interference factors, including similar appearance among objects, frequent object cross and occlusions, \etc. 
\lk{In fact, the uncertainty can also be leveraged to improve the pseudo-accuracy in turn.}
In this section, we propose an \textbf{U}ncertainty-aware \textbf{T}racklet-\textbf{L}abeling (\textbf{UTL}) strategy for better pseudo-tracklets.

Given an input video sequence $V = \{I^{1}, I^{2}, \cdots, I^{N}\}$, each frame $I^{t}$ is annotated with the bounding boxes $B^{t} = \{b_{1}^{t}, b_{2}^{t}, \cdots, b_{M^{t}}^{t}\}$ of $M^{t}$ objects in $t_{th}$ frame, where $b_{i}^{t} = (cx_{i}^{t}, cy_{i}^{t}, w_{i}^{t}, h_{i}^{t})$ is the center coordinate and shape of the $i_{th}$ object $o_{i}^{t}$. As shown in~\cref{fig:method_fmwk}, \mywork~generates accurate pseudo-tracklets in four main steps:

1) \textbf{Association}. For a certain object $o_{i}^{t}$ in frame $I^{t}$, the $\ell_2$-normalized representation $\boldsymbol{f}_{i}^{t}$ can be expressed as $\boldsymbol{f}_{i}^{t} = {\phi}(I^{t}, b_{i}^{t})$, 
% \begin{equation}
%   \boldsymbol{f}_{i}^{t} = {\phi}(I^{t}, b_{i}^{t})
%   % / {\Vert {\phi}(I^{t}, b_{i}^{t}) \Vert}_{2}
%   \label{eq:method_feat}
% \end{equation}
where the embedding encoder is denoted as $\phi$.

To associate the objects in frame $I^{t}$ with the objects or trajectories in previous $I^{t \minus 1}$, a similarity matrix is constructed with their appearance embeddings:

\begin{equation}
  \boldsymbol{C} \in \mathbb{R}^{M^{t} \times M^{t \minus 1}}, \;
  c_{i,j} = {\boldsymbol{f}_{i}^{t}} \cdot  \boldsymbol{f}_{j}^{t \minus 1}
  \label{eq:method_matrix}
\end{equation}

\noindent where $c_{i,j}$ represents the cosine similarity between the $i_{th}$ object in frame $I^{t}$ and the $j_{th}$ object (or trajectory) in frame $I^{t \minus 1}$. Then the Hungarian algorithm~\cite{kuhn1955hungarian} is adopted to generate the identity association results.

2) \textbf{Verification}. However, the appearance representations are sometimes unreliable, especially in the unsupervised scenario. To solve this issue, an uncertainty metric is proposed to evaluate the association after the first stage.

% For an object $o_{i}^{t}$ in frame $I^{t}$, the similarities against the $M^{t \minus 1}$ objects in the previous frame can be expressed as:

% \begin{equation}
%   \boldsymbol{s}_{i} = \boldsymbol{C}_{i} = [c_{i,1}, c_{i,2}, \cdots, c_{i,M^{t \minus 1}}]^T
%   \label{eq:method_svec}
% \end{equation}

% Inspired by margin-based OOD detection~\cite{hendrycks2016baseline}, we assume that the assignment ($o_{i}^{t} \!\sim\! o_{j}^{t \minus 1}$) in the association stage is not convincing under the following circumstances:

% \begin{itemize}
%     \setlength{\itemsep}{0pt}
%     \item The assigned similarity between $o_{i}^{t}$ and $o_{j}^{t \minus 1}$ is relatively low (\ie, $c_{i,j} < m_1$).
%     \item The second-highest similarity with others ($c_{i,j_{2}}$) is close to the assigned $o_{j}^{t \minus 1}$ (\ie, $c_{i,j} - c_{i,j_{2}} < m_2$).
% \end{itemize}

% Based on these assumptions, we define an association-level uncertainty metric, which is formulated as:



Object association can be viewed as multi-category classification.
And confidence-score has been proved efficient and effective on detecting mis-classified examples~\cite{hendrycks2016baseline}.
Inspired by this, we propose to detect the mis-associated objects through the similarity-scores.


Given an object $o_{i}^{t}$ associated with $o_{j}^{t \minus 1}$ in the previous frame based on \cref{eq:method_matrix}, the association ($o_{i}^{t} \!\sim\! o_{j}^{t \minus 1}$) is unconvincing in two cases: 
1) the assigned similarity $c_{i,j}$ is relatively low (\eg, partial occlusion or motion blur) and 
2) there are other objects whose similarities are close to the assigned $c_{i,j}$ (\eg, similar appearance or indistinguishable embedding).
It can be formulated as:

\begin{equation}
  c_{i,j} < m_1; \quad c_{i,j_{2}} > c_{i,j} - m_2
  \label{eq:method_margin}
\end{equation}


\noindent 
where $m_1,m_2$ are constant margins. Only the second-highest similarity with others ($c_{i,j_{2}}$) is considered for simplicity.
In an ideal association, $c_{i,j}$ should be close to 1 and $c_{i,j_{2}}$ close to 0.
We thus proposed to estimate the association \lk{risk} as:

% \sigma_{i,j} = - \left( 
% \log c_{i,j} + \log \left( 1 - c_{i,j_{2}} \right)
% + \overline{\log \left( 1 - c_{i,l} \right) }
% \right)  
\begin{equation}
  \sigma_{i,j} = - \log c_{i,j} - \log \left( 1 - c_{i,j_{2}} \right)
  \label{eq:method_energy}
\end{equation}

Detailed derivation process refers to the supplementary materials.
Combining with \cref{eq:method_margin} and \cref{eq:method_energy} , an adaptive threshold is proposed:

\begin{equation}
  % \gamma_{i,j} = -\log \left( 1 + m_2 - c_{i,j} \right) -\log m_1 \left( 1 - m_3 \right)
  \gamma_{i,j} =  -\log m_1 - \log \left( 1 + m_2 - c_{i,j} \right)
  \label{eq:method_border}
\end{equation}

As shown in~\cref{fig:method_verify}, when the \lk{risk} $\sigma_{i,j}$ is higher than the threshold $\gamma_{i,j}$, the assignment ($o_{i}^{t} \!\sim\! o_{j}^{t \minus 1}$) should be re-considered. 
\lk{The \textbf{association uncertainty} is quantified as:}

\begin{equation}
  \delta_{i,j} = \sigma_{i,j} - \gamma_{i,j}
  \label{eq:method_uncertain}
\end{equation}

The results are not sensitive to the exact margins. We set $m_1 = 0.5$ and $m_2 = 0.05$ for a slightly better performance.
% More experimental details are shown in the supplementary materials.

The uncertain pairs after the verification stage and unmatched objects after the association stage are gathered as uncertain candidates for the rectification stage.


3) \textbf{Rectification}. 
The rectification stage is performed among the uncertain candidate. The similarities between two adjacent frames are no longer convincing.
% due to irregular motion, severe occlusion, and so on. 
More information should be taken into account, including motion \lk{estimation} and appearance \lk{variation} within a tracklet. 
% Specifically, intersection-over-union (IoU)~\cite{bewley2016simple} is the widely-used motion metric. At the same time, the tracklet embeddings can provide complementary appearance information.

For the uncertain candidates, \mywork~constructs another similarity matrix for the secondary rectification. 
First, \lk{the motion constraints should be relaxed}, so the association shares overlap \lk{higher than} $\beta$ 
% in adjacent frames 
\lk{are preserved}. 
Second, \lk{the appearance should not vary extremely fast}, so we adopt the averaged similarity between object $o_{i}^{t}$ and tracklet $trk_{j} = \{o_{j}^{t \minus K}, \cdots, o_{j}^{t \minus 1}\}$ within previous $K$ frames. 
In this stage, we solve the sub-problem of global identity assignments, which can be formulated as:

\begin{equation}
\begin{split}
  \boldsymbol{C}^\prime \in \mathbb{R}^{{M^{t}}^\prime \times {M^{t \minus 1}}^\prime} & \\
  c^\prime_{i,j} = \left( \frac{1}{K} \sum_{\hat{t} = t \minus K}^{t \minus 1} {\boldsymbol{f}_{i}^{t}} \cdot  \boldsymbol{f}_{j}^{\hat{t}} \right) 
            \times \mathbb{I} & \left( \text{IoU} \left( b_{i}^{t}, b_{j}^{t \minus 1} \right) > \beta \right) 
  \label{eq:method_recti}
\end{split}
\end{equation}

\noindent where $\mathbb{I}(*)$ is the indicator function. Then the match set is updated based on the Hungarian algorithm.

\lk{
\textit{Remark.} Our core contribution is the uncertainty-based verification mechanism, rather than the specific rectification, which shall be adjusted in practice. Empirically we set $\beta=0.1$ and $K=5$.
}

% Figure environment removed

4) \textbf{Propagation}. The pseudo-tracklets are propagated frame-by-frame. As shown in~\cref{fig:method_reidacc}, our strategy brings \lk{consistently} accurate pseudo-identities, \lk{\eg, reaching 97\% accuracy across 100 frames}.
% The pseudo-tracklets are progressively updated during the training process.
The long-term intra-tracklet consistency is successfully maintained.
% by the accurate pseudo-identities.

It is worth mentioning that the \lk{verification and rectification} stages can be naturally applied to the inference process to boost the performance, \lk{which does not conflict with existing association methods}.

\subsection{Tracklet-Guided Augmentation}
\label{sec:method_ada_aug}

The accurate pseudo-tracklets can guide the sample augmentation in the unsupervised MOT framework.
To learn the \liuk{inter-frame consistency}~\cite{chen2020simple,zhang2021fairmot}, good training samples should be diverse and \liuk{temporal-aware}. 
However, as illustrated in~\cref{fig:method_ada_aug}, existing methods usually treat augmentation and multi-object tracking as two isolated tasks, leading to ineffective augmentations. 
Instead, this paper utilizes the tracklet's spatial displacements to guide the augmentation process. 
According to a properly selected anchor pair, the proposed strategy makes the augmented frames aligned to the historical frames, simulating realistic tracklet movements. The proposed method concurrently focuses on the hard negative samples.
Details \lk{of the \textbf{T}racklet-\textbf{G}uided \textbf{A}ugmentation (TGA)} are given below.

% Figure environment removed

We introduce the temporal information into spatial transformation. 
First, given a current frame $I^{t}$ with $M^{t}$ objects, we select a source-anchor object $o_{a}^{t}$, whose bounding box is denoted as $b_{a}^{t} = (cx_{a}^{t}, cy_{a}^{t}, w_{a}^{t}, h_{a}^{t})$. Then, we choose a target-anchor $o_{a}^{t \minus \tau}$ in $(t \minus \tau)_{th}$  historical frame from the pseudo-tracklet $trk_{a} = \{o_{a}^{t_0}, o_{a}^{t_1}, \cdots, o_{a}^{t}\}$. 
Finally, to augment the current $I^{t}$ to align with historical $I^{t \minus \tau}$,  a tracklet-guided affine transformation can be expressed as:

\begin{equation}
  \begin{bmatrix}
      x^{t \minus \tau} \\ y^{t \minus \tau} \\ 1
  \end{bmatrix}
  =
  \boldsymbol{M}_{t}^{t \minus \tau}
  \begin{bmatrix}
      x^{t} \\ y^{t} \\ 1
  \end{bmatrix}
  =
  \begin{bmatrix}
      m_{11} & m_{12} & m_{13} \\
      m_{21} & m_{22} & m_{23} \\
      0      & 0      & 1
  \end{bmatrix}
  \begin{bmatrix}
      x^{t} \\ y^{t} \\ 1
  \end{bmatrix}
  \label{eq:method_affine}
\end{equation}

\noindent where $x^*,y^*$ are spatial coordinates, and $\boldsymbol{M}_{t}^{t \minus \tau}$ can be solved by direct linear transform (DLT) algorithm ~\cite{detone2016deep}. 
% with the corner locations of the anchor pair $(o_{a}^{t} \!\sim\! o_{a}^{t \minus \tau})$. 
Then an augmented frame $\tilde{I}^{t}$ is generated based on the tracklet-guided affine transformation with perspective jitter, which can be expressed as $\tilde{I}^{t} = \mathcal{T}\left(I^{t}, M_{t}^{t \minus \tau} \right)$.
% \begin{equation}
%   \tilde{I}^{t} = \mathcal{T}\left(I^{t}, M_{t}^{t \minus \tau} \right)
%   \label{eq:method_aug}
% \end{equation}

Intuitively, a proper anchor-selection is vitally important for our augmentation strategy. 

First, the identity accuracy of anchor pair $(o_{a}^{t} \!\sim\! o_{a}^{t \minus \tau})$ is important. In other words, the consistency of anchor tracklet $trk_{a}$ should be guaranteed. We thus design a tracklet-level uncertain metric based on the propagated association-level uncertainty defined in \cref{eq:method_uncertain}, which is formulated as:

\begin{equation}
  \Omega_{i} = \frac{1}{n} \sum_{s=t_0}^{t} \exp (\delta_{i}^{s})
  % \Omega_{i} = \sqrt[n]{\sigma_{i}^{t_0} \cdot \sigma_{i}^{t_1} \cdots \sigma_{i}^{t}}
  \label{eq:method_tenergy}
\end{equation}

\noindent where $\Omega_{i}$ represents the uncertainty of tracklet $trk_{i}$, \lk{and $n$ is the tracklet length}.
An uncertainty-based sampling strategy is designed to select the source anchor $o_{a}^{t}$ (along with the anchor $trk_{a}$) from the $M^{t}$ objects in frame $I^{t}$, which can be formulated as:

\begin{equation}
  p\left(a=i \mid t \right) 
  % = softmax\left(-\Omega_{i}\right)
  = \frac{\exp{\left(-\Omega_{i}\right)}}{\sum_{\hat{i}=1}^{M^{t}}\exp{\left(-\Omega_{\hat{i}}\right)}}
  \label{eq:method_sel_an_src}
\end{equation}

\noindent where $p\left(a=i \mid t \right)$ represents the probability to choose the $i_{th}$ tracklet $trk_{i}$ as the anchor $trk_{a}$.
The uncertain tracklet with high $\Omega$ is less likely to be selected, avoiding dramatic augmentations from erroneous pseudo-tracklets.

Second, hard negative samples matters in discriminablity learning. We tend to choose an indistinguishable (or, high uncertain) target anchor $o_{a}^{t \minus \tau}$ along the tracklet $trk_{i}$. The selection probability can be formulated as:

\begin{equation}
  p\left(\pi=t \minus \tau \mid a \right) 
  = \frac{\exp{\left(\delta_{a}^{t \minus \tau}\right)}}{\sum_{\hat{\tau}=t_0}^{t-1}\exp{\left(\delta_{a}^{t-\hat{\tau}}\right)}}
  \label{eq:method_sel_an_tgt}
\end{equation}

\lk{A visualization example are displayed in the supplementary material to illustrate the hierarchical sampling process.}

Compared with conventional random transformation, the proposed tracklet-guided augmentation is well-directed and tracking-related. 
\lk{Together with accurate pseudo-tracklets, \mywork~successfully improves the inter-frame consistency, as shown in \cref{fig:method_disc_vis}. }


% Figure environment removed

% \subsection{Momentum Memory Dictionary}
% \label{sec:method_md}


%To reuse the encoded samples from the intermediate mini-batches, we maintain a queue for each video in the memory dictionary by enqueueing the $M^{t}$ objects in the current frame and removing the oldest samples.
%In representation learning, high-quality negative samples play an essential role~\cite{chen2020simple,he2020momentum}. However, existing unsupervised trackers only take negative samples from adjacent frames, augmented frames, and the current frame itself. The lack of negative sample diversity prevents trackers from learning discriminative representations. \mywork~adopts a momentum dictionary mechanism to alleviate this problem.

%As shown in~\cref{fig:method_fmwk}, we build a memory dictionary for each \textit{independent} video input during training. Given an input image $I^{t}$ from video $V$, we randomly sample a number of negative object samples from other videos in the memory dictionary, so as to enrich the negative sample diversity. To reuse the encoded samples from the intermediate mini-batches, we maintain a queue for each video in the memory dictionary by enqueueing the $M^{t}$ objects in the current frame and removing the oldest samples.
\begin{table*}[t]
  \vspace{-2mm}
    \centering
    \footnotesize
    \setlength{\tabcolsep}{1pt} % 
    % \rowcolors{3}{}{lightgray}
    \renewcommand{\arraystretch}{1.15} % Default value: 1, controls row space
    % \begin{tabular}{l|C{0.8cm} C{1.55cm} C{0.55cm}|C{0.8cm} C{1.55cm} C{0.55cm} |C{0.8cm} C{1.55cm} C{0.55cm} | C{0.8cm} C{1.55cm} C{0.55cm}}
      \begin{tabularx}{\linewidth}{>{\raggedright\arraybackslash}p{2.1cm}|>{\centering\arraybackslash}X>{\centering\arraybackslash}p{1.55cm}>{\centering\arraybackslash}X | >{\centering\arraybackslash}X>{\centering\arraybackslash}p{1.55cm} >{\centering\arraybackslash}X | >{\centering\arraybackslash}X>{\centering\arraybackslash}p{1.55cm}>{\centering\arraybackslash}X | >{\centering\arraybackslash}X>{\centering\arraybackslash}p{1.55cm}>{\centering\arraybackslash}X}
    \Xhline{3\arrayrulewidth}
   
    \multirow{2}{*}{\textbf{Method}} & \multicolumn{3}{c|}{\textbf{3DPW-OC} } & \multicolumn{3}{c|}{\textbf{3DOH} } & \multicolumn{3}{c|}{\textbf{3DPW-PC }} & \multicolumn{3}{c}{\textbf{3DPW-Crowd }}   \\
    
      & \textbf{MPJPE$\downarrow$} & \textbf{PA-MPJPE$\downarrow$} & \textbf{PVE$\downarrow$} & \textbf{MPJPE$\downarrow$} & \textbf{PA-MPJPE$\downarrow$} & \textbf{PVE$\downarrow$} & \textbf{MPJPE$\downarrow$} & \textbf{PA-MPJPE$\downarrow$} & \textbf{PVE$\downarrow$}& \textbf{MPJPE$\downarrow$} & \textbf{PA-MPJPE$\downarrow$} & \textbf{PVE$\downarrow$}\\
    \hline
    I2L-MeshNet~\cite{i2L_meshNet}  & 92.0 & 61.4 & 129.5 & - & - & - & 117.3 & 80.0 & 160.2 & 115.7 & 73.5 & 162.0   \\
    SPIN~\cite{SPIN} & 95.5 & 60.7 & 121.4 & 110.5 & 71.6 & 124.2 & 122.1 & 77.5 & 159.8  & 121.2 & 69.9 & 144.1   \\
    PyMaf~\cite{zhang2021pymaf}  & 89.6 & 59.1 & 113.7 & 101.6 & 67.7 & 116.6 & 117.5 & 74.5 & 154.6 & 115.7 & 66.4 & 147.5   \\
    ROMP~\cite{ROMP}  & 91.0 & 62.0  & -  & - & - & - & 98.7 & 69.0 & - & 104.8 & 63.9 & 127.8   \\ 
    OCHMR~\cite{OCHMR}  & 112.2 & 75.2 &  145.9 & - & - & - & - & - & - & - & - & -  \\
    PARE*~\cite{PARE} & 83.5 & 57.0 & 101.5  & 109.0 & 63.8 & 117.4 &  96.8 & 64.5 & 122.4  &  94.9 & 57.5 & 117.6  \\
    3DCrowdNet~\cite{3dCrwodNet}  & 83.5 & 57.1 & 101.5 & 102.8 & 61.6 & 111.8 & 90.9 & 64.4 & 114.8 & 85.8 & 55.8 & 108.5  \\
    
    \hline
    Ours & \textbf{75.7} & \textbf{52.2}  & \textbf{92.6} & \textbf{98.7} & \textbf{59.3} & \textbf{104.8}& \textbf{86.5} &\textbf{58.3} & \textbf{109.7} & \textbf{82.4} & \textbf{52.0} &\textbf{103.4}\\
    \Xhline{3\arrayrulewidth}
    \end{tabularx}
  % \end{tabular}
    \vspace{-3mm}
    \caption{Comparisons to the state-of-the-art methods under severe occlusion. The units for mean joint and vertex errors are in mm.  PARE* use a HRNet-32 backbone, others are with ResNet-50. }
    \label{table:oc_sota}
    \vspace{-5mm}
\end{table*}


\begin{table*}[t]
\vspace{-3mm}
    \setlength\tabcolsep{1.5mm}
    % \hspace{-0.5mm}
    \parbox{0.47\linewidth}{
    %\begin{minipage}[b]{.5\linewidth}
    \centering
        \footnotesize
        \vspace{2.5mm}
        \renewcommand{\arraystretch}{1.15} % Default value: 1, controls row space
        % \begin{tabular}{l|C{1.4cm} C{1.6cm} C{1.4cm}}
        \begin{tabularx}{\linewidth}{>{\raggedright\arraybackslash}p{2.4cm} |>{\centering\arraybackslash}X >{\centering\arraybackslash}p{1.6cm} >{\centering\arraybackslash}X}
            \Xhline{3\arrayrulewidth}
            \textbf{Method}  &\textbf{MPJPE$\downarrow$} & \textbf{PA-MPJPE$\downarrow$} &  \textbf{PVE$\downarrow$}\\
            \hline 
            HMR~\cite{HMR}  & 130.0 & 76.7 & - \\
            Kanazawa \etal~\cite{kanazawa2019learning}  & 116.5 & 72.6 & 139.3 \\
            % Arnab \etal~\cite{arnab2019exploiting}  & - & 72.2 & - \\
            GCMR~\cite{GraphCMR}  & - & 70.2 & - \\
            DSD-SATN~\cite{sun2019human}  & - & 69.5 & -\\
            SPIN~\cite{SPIN}  & 96.9 & 59.2 & 116.4\\
            
            %\hline
            I2L-MeshNet~\cite{i2L_meshNet}  & 93.2 & 58.6 & 136.5\\
            PyMAF~\cite{zhang2021pymaf}  & 92.8 & 58.9 & 110.1 \\
            OCHMR~\cite{OCHMR} & 89.7 & 58.3 & 107.1 \\
            EFT~\cite{joo2021exemplar_EFT}  & - & 54.2 & -\\
            ROMP~\cite{ROMP}  & 89.3 & 53.5 & 105.6\\
            PARE~\cite{PARE} & 82.9 & 52.3 & 99.7 \\
            3DCrowdNet~\cite{3dCrwodNet} & 81.7 & 51.2 & 98.3 \\
            \hline
            Ours  & \textbf{76.4}  & \textbf{48.7} & \textbf{92.6}\\
                \Xhline{3\arrayrulewidth}
        % \end{tabular}
    \end{tabularx}
        \vspace{-2mm}
        \caption{Comparisons to the state-of-the-art methods on standard 3DPW \cite{3dpw} test split.}
        \label{table:3dpw}}
    %\end{minipage}
    \hspace{8.7mm}
    %\begin{minipage}[b]{.38\linewidth}
    \parbox{.47\linewidth}{
    \centering
    \footnotesize
    \vspace{1mm}
    \renewcommand{\arraystretch}{1.15} % Default value: 1, controls row space
    \begin{tabularx}{\linewidth}{>{\raggedright\arraybackslash}p{2.2cm}| >{\centering\arraybackslash}X >{\centering\arraybackslash}X >{\centering\arraybackslash}X >{\centering\arraybackslash}X | >{\centering\arraybackslash}X }
    % \begin{tabular}{l|C{0.85cm} C{0.85cm} C{0.85cm} C{0.85cm} | C{0.7cm} }
        \Xhline{3\arrayrulewidth}
        \textbf{Method}  &\textbf{Haggl.} & \textbf{Mafia} &  \textbf{Ultim.} & \textbf{Pizza} & \textbf{Mean} \\
        \hline 
        Zanfir~\etal~\cite{zanfir2018monocular} & 140.0 & 156.9 & 150.7 & 156.0 & 153.4 \\
        Zanfir\etal~\cite{zanfir2018deep} & 141.4 & 152.3 & 145.0 & 162.5 & 150.3 \\
        Jiang~\etal~\cite{jiang2020multiperson} & 129.6 & 133.5 & 153.0 & 156.7 & 143.2 \\
        ROMP~\cite{ROMP} & 111.8 & 129.0 & 148.5 & 149.1 & 134.6 \\
        SPIN~\cite{SPIN} & 124.3 & 132.4 & 150.4 & 153.5 & 133.1 \\
        OCHMR~\cite{OCHMR} & 115.5 & 123.7 & 142.6 & 150.6 & 133.1 \\
        REMIPS~\cite{fieraru2021remips} & 121.6 & 137.1 & 146.4 & 148.0 & 138.3 \\
        3DCrowdNet~\cite{3dCrwodNet} & 109.6 & 135.9 & 129.8 & 135.6 & 127.6 \\
        BEV*~\cite{BEV} & 110.3 & 125.6 & 150.7 & 131.7 & 127.9 \\
        \hline
        Ours & \textbf{99.9} & \textbf{113.5} & \textbf{115.7} & \textbf{123.6} & \textbf{114.7}\\
            \Xhline{3\arrayrulewidth}
    \end{tabularx}
    % \end{tabular}
    \vspace{-1mm}
    \caption{Comparison on CMU-Panoptic \cite{CMUpanoptic}. The numbers denote MPJPE. For a fair comparison, we apply BEV* model that is not fine-tuned on AGORA \cite{patel2021agora} (\ie, a synthetic 3D dataset.).}  \label{table:cmu}}
%\end{minipage}%
\vspace{-4mm}
\end{table*}\vspace{-1mm} 
\begin{table}[t]
\setcounter{table}{0}
\caption{Ablation study on \textit{Oxford-IIIT Pet INP} dataset. (a) Ablation of losses. (b) Ablation of architectural choices: (i) remove context information $I_C$ and $s_c$, (ii) remove $C_e$, (iii) decode all strokes parameters with a single-step decoder}
\vspace{3mm}
\setlength{\tabcolsep}{1.0pt}
\footnotesize
    \begin{minipage}[t]{\columnwidth}
    \captionsetup{labelformat=empty} 
    \centering
        \resizebox{0.97\linewidth}{!}{
        \begin{tabular}{lcccccccccc}
        \toprule
         & $\mathcal{L}_{\beta\mhyphen\mathrm{VAE}}$ & $\mathcal{L}_{\mathrm{col}}$ & $\mathcal{L}_{\mathrm{col}}^{reg}$ & $\mathcal{L}_{\mathrm{dist}}^{reg}$ & L2$\downarrow$ & FSD$\downarrow$ & FVD$\downarrow$ & WD$\downarrow$ & DTW$\downarrow$ & LPIPS$\uparrow$\\
        \midrule
         &\cmark &\xmark & \xmark & \xmark &  0.155 & 1.29 & 13.2 & \textbf{0.031} & \textbf{0.851} & 0.038 \\
         &\cmark &\cmark & \xmark & \xmark & 0.057 & 2.05 & 7.31 & 0.033 & 0.910 &  0.029 \\
         &\cmark &\cmark & \cmark & \xmark & \textbf{0.039} & 5.17 & 6.77 & 0.035 & 0.942 & 0.030 \\
         &\cmark &\cmark & \xmark & \cmark & 0.091 & \textbf{1.16} & 9.51 & \textbf{0.031} & 0.893 & \textbf{0.039} \\
        \midrule
        Full &\cmark &\cmark&\cmark&\cmark& 0.042 &	1.51 &	\textbf{6.72} & 0.032 &	0.893 & 0.030 \\
    
        \bottomrule
        \end{tabular}}
        \vspace{1.5mm}
        \caption{(a)}
        \label{tab:ablation_losses}
    \end{minipage}
    \begin{minipage}[t]{.40\textwidth}
    \captionsetup{labelformat=empty} 
    \centering
        \resizebox{0.99\linewidth}{!}{
        \begin{tabular}{lcccccc}
        \toprule
        Var. & L2$\downarrow$ & FSD$\downarrow$ & FVD$\downarrow$ & WD$\downarrow$ & DTW$\downarrow$ & LPIPS$\uparrow$\\
        \midrule
        (i) &  0.059 &	11.9 & 23.3 & 0.058 & 1.13 & \textbf{0.037} \\
        (ii) &  0.048 & 1.87 & 7.78 & 0.036 & 0.915 &  0.027 \\
        (iii) &  0.049 &	1.70 & 7.83 & 0.033 &	\textbf{0.893} &	0.028 \\
        %(iv) & 0.043 & 1.73 & 6.77 & 0.033 & 0.910 &	0.029 \\
        \midrule
        
        Full &  \textbf{0.042} & \textbf{1.51} & \textbf{6.72} & \textbf{0.032} & \textbf{0.893} & 0.030  \\
        \bottomrule
        \vspace{-0.0mm}
        \end{tabular}}
        \vspace{1.0mm}
        \caption{(b)}
    \label{tab:ablation_architecture}
    \end{minipage}
\vspace{-2mm}
%(iv) concatenate $z$ and $c$ in the channel dimension.}
\label{tab:ablation}
\end{table}

\section{Experiments}
\label{sec:experiments}


\noindent\textbf{Implementation Detail.}
This proposed \methodname is validated using an end-to-end training approach on the ResNet-50~\cite{resnet} backbone. Following 3DCrowdNet~\cite{3dCrwodNet}, we initialize ResNet from Xiao~\etal~\cite{xiao2018simple} for fast convergence. We use AdamW optimizer~\cite{AdamW} with a batch size of 256 and weight decay of $10^{-4}$. The initial learning rate is  $10^{-4}$. 
The ResNet-50 backbone takes a $256 \times 256$ image as input and produces image features with size of $2048 \times 8 \times 8 $. We build the 3D features with size of $256 \times 8 \times 8 \times 8$ and 2D features with size of $256 \times 8 \times 8$. 
For joint-to-non-joint contrast, we sample $50$ anchor joints in each mini-batch, which are paired with $1024$ positive and $2048$  and negative samples. For joint-to-joint contrast, we sample $50$ anchor joints in each mini-batch, which are paired with $128$ positive and $512$ and negative samples. Both contrastive losses are set to a temperature of $0.07$.
More details can be found in the supplementary material. 

\noindent\textbf{Training.} Following the settings of previous work~\cite{HMR,SPIN,3dCrwodNet}, our approach is trained on a mixture of data from several datasets with 3D and 2D annotations, including Human3.6M~\cite{ionescu_h36m}, MuCo-3DHP~\cite{muco}, MSCOCO~\cite{coco},  and CrowdPose~\cite{li2018crowdpose}. Only the training sets are used, following the standard split protocols. For the 2D datasets, we also utilize their pseudo ground-truth SMPL parameters~\cite{Moon_2022_CVPRW_NeuralAnnot} for training. 

\noindent\textbf{Evaluation.} The 3DPW~\cite{3dpw} test split, 3DOH~\cite{CVPE_2020_OOH} test split, 3DPW-PC~\cite{ROMP,3dpw}, 3DPW-OC~\cite{CVPE_2020_OOH,3dpw}, 3DPW-Crowd~\cite{3dCrwodNet,3dpw} and CMU-Panoptic~\cite{CMUpanoptic} datasets are used for evaluation.
3DPW-PC and 3DPW-Crowd are the \textit{person-person} occlusion subset of 3DPW, 3DPW-OC is the \textit{person-object} occlusion subset of 3DPW and 3DOH is another \textit{person-object} occlusion specific dataset.
We adopt per-vertex error (PVE) in mm to evaluate the 3D mesh error. 
We employ Procrustes-aligned mean per joint position error (PA-MPJPE) in mm and mean per joint position error (MPJPE) in mm to evaluate the 3D pose accuracy. As for CMU-Panoptic, we only report  mean per joint position error (MPJPE) in mm following previous work~\cite{jiang2020multiperson, 3dCrwodNet, ROMP}.

\subsection{Comparison to the State-of-the-Art on Occlusion Benchmark}
\noindent\textbf{3DPW-OC~\cite{3dpw,CVPE_2020_OOH}} is a person-object occlusion subset of 3DPW and contains 20243 persons.~\cref{table:oc_sota} shows our method  achieve a new state-of-the-art performance on 3DPW-OC.

\noindent\textbf{3DOH~\cite{CVPE_2020_OOH}} is a person-object occlusion-specific dataset and contains 1290 persons in testing set, which incorporates a greater extent of occlusions than 3DPW-OC. 
For a fair comparison, we initialize PARE with weights that are not trained on the 3DOH training set, resulting in different performances from the results reported in~\cite{PARE}.
\cref{table:oc_sota} shows our method surpasses all the competitors with $59.3$  (PA-MPJPE). 

\noindent\textbf{3DPW-PC~\cite{3dpw,ROMP}} is a multi-person subset of 3DPW and contains 2218 persons' annotations under person-person occlusion.~\cref{table:oc_sota} shows our method surpasses all the competitors with $58.8$  (PA-MPJPE).

\noindent\textbf{3DPW-Crowd~\cite{3dpw, 3dCrwodNet}} is a person crowded  subset of 3DPW and contains 1923 persons.  We slightly surpass previous state-of-the-art as shown in~\cref{table:oc_sota}.
 
\noindent\textbf{CMU-Panoptic~\cite{CMUpanoptic}} is a dataset with multi-person indoor scenes. We follow previous methods~\cite{3dCrwodNet,jiang2020multiperson} applying 4 scenes for evaluation without using any data from training set. 
\cref{table:cmu}, shows that our method outperforms previous 3D human pose estimation methods on CMU-Panoptic, which means our model also works well for indoor and daily life scenes.
\vspace{-2mm}
\subsection{Comparison to the State-of-the-Art on Standard Benchmark}
\vspace{-2mm}
\noindent\textbf{3DPW~\cite{3dpw}} is the latest large-scale benchmark for 3D human mesh recovery. We do not use the training set and report performance on its test split which contains 60 videos and 3D annotations of 35515 persons. As shown in~\cref{table:3dpw}, Our method achieves  state-of-the-art  results among previous approaches.
 The results demonstrate the robustness of \methodname to a variety of in-the-wild scenarios.
\vspace{-2mm}
\subsection{Analysis.}
\vspace{-2mm}
In this section, we analyze the main components of \methodname and evaluate their impact on the mesh recovery performance.
More details and ablation studies can be found in the supplementary material.

\noindent\textbf{Utilization of 2D and 3D features:}
\cref{tab:2d_3d} demonstrates that the incorporation of 3D features is beneficial for mesh recovery performance. For the utilization of 2D features, flatting shows better performance than sampling, which supports our hypothesis that sampling joint features in obscured regions could have a negative impact.
For 3D features, we do not conduct experiments for flatting 3D features due to memory limitations. Moreover, we believe that 3D joint feature sampling is adequate for alleviating occlusion problems by attending to the accurate depth.
\cref{fig:_weight} shows the attention weights in the last refining layer. The query tokens significantly pay more attention to 3D features, which validates the usefulness of our fusion framework.

\noindent\textbf{Validation of coarse-to-fine regression:}
We validate the accuracy of intermediate predictions of fusion transformer in~\cref{tab:coarse_to_fine}, which shows the coarse-to-fine regression process in \methodname.

\noindent\textbf{Decoupling SMPL query:}
\methodname performance improvement is observed in~\cref{tab:smpl_token} by decoupling SMPL query from joint query. In the experiment without decoupling, we employ mean pooling on the decoder's output and regress SMPL parameters through MPLs. Decoupling SMPL query is presumed to enhance performance by reducing interference in executing other tasks (\eg, joint localization) during SMPL parameter regression.


% Figure environment removed
% Figure environment removed

% Figure environment removed


\noindent\textbf{3D Joint contrastive learning:}
The impact of 3D joint contrastive learning on the performance of \methodname is presented in~\cref{tab:contrastive}. Both joint-to-non-joint and joint-to-joint contrastive losses result in improved performance, with the former being more effective as it incorporates global supervision for the entire 3D space. 
Our contrastive losses also lead to more compact and well-separated 
learned joint embeddings, as shown in~\cref{fig:joint_contrast}. 
This indicates that our network can generate more discriminative 3D features, producing semantically clear 3D spaces and promising results. 


We proposed a machine-learning based method to approximate diagonal as well as non-diagonal elements of the Hessian of a molecule. The representation used is specific for every internal coordinates, and takes explicitly into account the bond order, which is sensible because a single point DFT calculation is computationally considerably less expensive that the explicit calculation of the Hessian.
We trained our ML model on a relatively small dataset (subset of QM7) of less than 7000 molecules. The Hessian was computed at the B3LYP/cc-pVDZ level of theory. 
The agreement between ML and DFT was satisfactory. In particular, the calculated MAPE for bond stretching force constant was below 2\%, and was particularly small for bonds involving hydrogen atoms because they point outwards and are less affected by the chemical environment. The MAPE for bending and torsion was of 5\% and 10\%, respectively. 
From the ML model trained on QM7 we were also able to predict the Hessian of some molecules representative of the QM9 dataset. The Hessian predicted in internal coordinates was then transformed into the mass-weighted Cartesian Hessian, the diagonalization of which yields the harmonic vibrational frequencies and normal modes, that can be compared to the ones calculated  explicitly from DFT.

High frequency vibrations and normal modes were predicted accurately, while lower frequency ones were not. This behaviour is analogous to the IR spectroscopy theory, where stretchings and bendings can be identified accurately, while torsion and delocalized vibrations are more difficult to be interpreted.

The approximate Hessian obtained with ML is computational inexpensive and can be used as an initial Hessian guess for geometry optimization, or in the context of alchemical geometry relaxation \cite{Domenichini2020,domenichini2022alchemical, shiraogawa2022exploration,shiraogawa2023optimization}. 
A good starting Hessian may speed up the convergence of the geometrical optimization. An in detail assessment of the performance of the ML Hessian proposed is not yet provided, but should carefully take into account many parameters on which the optimization depends, \textit{e.g.} the type of molecule, the initial geometry, the optimization algorithm, and the Hessian update scheme.



{\small
\bibliographystyle{ieee_fullname}
\bibliography{main}
}

\end{document}