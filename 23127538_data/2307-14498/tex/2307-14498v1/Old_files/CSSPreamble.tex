%%% ============================================================================================================= %%%
%%%                                  The lastest updated date: November 11th, 2004                                    %%%
%%% ============================================================================================================= %%%
%%% Update Log: 11/16/2004, add the \RomanNumber and \romannumber for capital and small Roman numbers
%%%                         add \rb for table raising box
%%% Update Log: 05/08/2004, add the matlab and simulink definition in the end for reference
%%% Update Log: 04/26/2004, add the math operator ``dist'', update the definition for theorem-like enviroments
%%% Update Log: 03/04/2004, use package 'upgreek' for constants, e.g., $\uppi$ = 3.1415926.... (ISO standard)
%%% Update Log: 09/12/2004, update for solving the conflicts between amsthm and cssconf
%%% ============================================================================================================= %%%

\usepackage{latexsym, amssymb, amsmath}
%\usepackage[dvips]{graphicx} % added for inserting the pictures
%% \usepackage[dvips]{color}
%% \usepackage{amsthm}
%% \usepackage{amscd}% amscd for "commutative diagram"
\usepackage{enumerate}
%% \usepackage{url}
%\usepackage{upref} % make all ref's are upright fonttype, no use in this paper % zzz: removed by Dr. Gray and Dr. Gonzalez
%% \usepackage{upgreek} % use upright greek to represent constant values, i.e., $\uppi$ = 3.14, ...
%\usepackage{flushend} % align at the last lines of every page
\usepackage{mathrsfs} % added by Hong for Script Capital Letters by using "$\mathscr{}$"
%% \usepackage{easybmat} % A simple package for writing block matrices with equal column widths or equal rows heights or both, with various kinds of rules between rows and columns.
%% \usepackage{txfonts}%, pxfonts
%% \usepackage{euscript} %, amsfonts}
%% \usepackage{times}
%% \usepackage{color}
%\usepackage{stfloats}
%% \usepackage{float}
%% Packages added by Dr. Gonzalez
%\usepackage{indentfirst}    %Automatically indents the beginning of the paragraph following a Latex section
\usepackage{cite}
%%% ============================================================================================================= %%%

% Hong's setting for figures for this paper
%\setlength{\textfloatsep}{.3em}
%\setlength{\intextsep}{.5em}

% Reduce Space Between Paragraphs
%\setlength{\parskip}{1.5ex} %plus 1pt minus 1pt}

% Reduce Space Above and Below Equations
%\setlength{\jot}{-0.7pt} % eqnarray extra space
%\setlength{\abovedisplayskip}{-1pt}% plus 1pt minus 1pt} % extra space above eqn
%\setlength{\abovedisplayshortskip}{-1pt}% plus 1pt minus 1pt} % extra space above eqn
%\setlength{\belowdisplayskip}{-1pt}% plus 1pt minus 1pt} % extra space below eqn
%\setlength{\belowdisplayshortskip}{-1pt}% plus 1pt minus 1pt} % extra space below eqn
%\setlength{\topsep}{0.2pt}

% Insert Breaks Between a Long Formula
%\allowdisplaybreaks[3] % added by Hong for allowing page breaks among a large formula


%%% ============================================================================================================= %%%
% new theorems environment
%\swapnumbers
%% \theoremstyle{plain} % default
\newtheorem{prop}{Proposition} % [section]
\newtheorem{proposition}[prop]{Proposition} % [section]
\newtheorem{lem}{Lemma} % [section]
\newtheorem{lemma}[lem]{Lemma} % [section]
\newtheorem{thm}{Theorem} % [section]
\newtheorem{theorem}[thm]{Theorem} % [section]
\newtheorem{cor}{Corollary} % [section]
\newtheorem{corollary}[cor]{Corollary} % [section]
%\newtheorem{exer}{Exercise}[section]
%\newtheorem{test}{Test}[section]

%% \theoremstyle{definition}
\newtheorem{defn}{Definition} % [section]
\newtheorem{defi}[defn]{Definition} % [section]
\newtheorem{definition}[defn]{Definition} % [section]
\newtheorem{conj}{Conjecture} % [section]
\newtheorem{exmp}{Example} % [section]
\newtheorem{example}[exmp]{Example} % [section]
\newtheorem{exam}[exmp]{Example} % [section]
\newtheorem{exam*}{Example}

%% \theoremstyle{remark}
%% \newtheorem*{rmk}{Remark}
%% \newtheorem*{remark}{Remark}
% %\newtheorem*{note}{Note}
% \newtheorem{case}{Case}

% \newtheoremstyle{model}% name
%   {3pt}%      Space above, empty = `usual value'
%   {3pt}%      Space below
%   {\itshape}% Body font
%   {}%         Indent amount (empty = no indent, \parindent = para indent)
%   {\textnormal}% Thm head font
%   {.}%        Punctuation after thm head
%   {.5em}%     Space after thm head: " " = normal interword space;
%         %       \newline = linebreak
%   {}% Thm head spec
%\renewcommand {\thmhead}[3]{\thmname{#1}\thmnumber{#2}\thmnote{\textnormal{[\emph{#3}]}}}

% \theoremstyle{model}
% \newtheorem*{mdl}{Rollback Interference Model}


%%% ============================================================================================================= %%%
%% Generates a bibliography with 'References' centered above the column, changed by Hong
%\def\custombibliography#1{
% \normalsize
%% The part was commented by Hong
%% \begin{center}
%% {\Large \bf{References}}
%% \end{center}
%\section*{\centering References}
% \list
% {[\arabic{enumi}]}{\settowidth\labelwidth{[#1]}\leftmargin\labelwidth
% \setlength{\itemsep}{.1em}
% \advance\leftmargin\labelsep
% \usecounter{enumi}}
% \def\newblock{\hskip .11em plus .33em minus -.07em}
% \sloppy
% \sfcode`\.=1000\relax}
%\let\endthebibliography=\endlist


%%% ============================================================================================================= %%%
% Gray's Abbreviations
\def\L2{{\cal L}_2}
\def\begar{\begin{array}}
\def\endar{\end{array}}
\def\begce{\begin{center}}
\def\endce{\end{center}}
\def\begco{\begin{cor}}
\def\endco{\end{cor}}
\def\begde{\begin{defn}}
\def\endde{\end{defn}}
\def\begdes{\begin{description}}
\def\enddes{\end{description}}
\def\begdi{\begin{displaymath}}
\def\enddi{\end{displaymath}}
\def\begdis{\begin{eqnarray*}}
\def\enddis{\end{eqnarray*}}
\def\begen{\begin{enumerate}}
\def\enden{\end{enumerate}}
\def\begeq{\begin{equation}}
\def\endeq{\end{equation}}
\def\begeqa{\begin{eqnarray}}
\def\endeqa{\end{eqnarray}}
\def\begex{\begin{exmp}}
\def\endex{\end{exmp}}
\def\begfig{\begin{fig}}
\def\endfig{\end{fig}}
\def\begit{\begin{itemize}}
\def\endit{\end{itemize}}
\def\begle{\begin{lem}}
\def\endle{\end{lem}}
\def\begpro{\begin{prop}} % added by Hong
\def\endpro{\end{prop}} % added by Hong
\def\begth{\begin{thm}}
\def\endth{\end{thm}}
\def\begre{\noindent{\em Remark}:\ } % changed by Hong
\def\endre{\\}
\def\begres{\noindent{\em Remarks}:\begin{enumerate}}
\def\endres{\end{enumerate} \par}
\def\begpr{\begin{proof}} % changed by Hong
\def\endpr{\end{proof}}
%\def\begpr{\noindent{\em Proof}\ : } % changed by Hong
%\def\endpr{\hspace*{0.05in}\bull\vspace*{0.15in}\\}
%\def\beglar{\left[\begin{array}}
%\def\endrar{\end{array}\right]}


%%% ============================================================================================================= %%%
% Gray's New Commands
\newcommand\bull{\vrule height .9ex width .8ex depth -.1ex } % square bullet
\newcommand\bul{$\bullet\;\;\;$}
\newcommand\re{\rm I\! R}
%\renewcommand\qed{\hfill$\Box \Box \Box$}
%\newcommand\mathbfsub#1{\hbox{\mathversion{bold}$\scriptstyle #1$}} % Stopped by Hong; Define better command \rv
%\newcommand\rref[1]{(\ref{#1})} % Stopped by Hong; Use AMS LaTeX command \eqref
\newcommand\spacebox[1]{\raisebox{-6pt}[7pt][0pt]{#1}} % puts extra space around items in a table
\newcommand\cdcout[1]{} % reduce it to 6 pages
%\newcommand\cdcout[1]{#1} % restores the CDC 7 page version


%%% ============================================================================================================= %%%
% Hong's Abbreviations for Left and Right Delimiters
\newcommand{\Lb}{\left [}          % Left bracket
\newcommand{\Rb}{\right ]}         % Right bracket
\newcommand{\LB}{\left \{}         % Left brace
\newcommand{\RB}{\right \}}        % Right brace
\newcommand{\Ld}{\left .}          % Left default
\newcommand{\Rd}{\right .}         % Right default
\newcommand{\Lp}{\left (}          % Left parenthese
\newcommand{\Rp}{\right )}         % Right parenthese
\newcommand{\Lv}{\left |}          % Left vertical line
\newcommand{\Rv}{\right |}         % Right vertical line
\newcommand{\LV}{\left \|}         % Left vertical double line
\newcommand{\RV}{\right \|}        % Right vertical double line


%%% ============================================================================================================= %%%
% Hong's New Commands and Declarations for Linear Algebra and Matrix Analysis
\DeclareMathOperator{\trace}{tr}   % Define a matrix' trace operator, sometimes
\DeclareMathOperator{\diag}{diag}  % Define the diagonal matrix operator
\DeclareMathOperator{\rank}{rank}  % Define the rank operator of a matrix
\DeclareMathOperator{\vect}{vec}   % Define the function to reshape a matrix into a vector column-wise.
% \DeclareMathOperator{\col}{col}    % Define the column operator, e.g., $cy = \col \LB y_i, \dots, y_n \RB$
\newcommand{\T}{^\mathrm{T}}       % Transposition operator of the matrix or vector
\newcommand{\h}{^\mathrm{H}}       % Hermitian Transposition of a matrix or vector


%%% ============================================================================================================= %%%
% Hong's New Commands and Declarations for Random Variables, Probability and Stochastic Process
\newcommand{\rv}[1]{\boldsymbol{#1}} % Use italic boldface to indicate the Random Variables
\DeclareMathOperator{\Var}{Var}      % Operator for variance of a RV
\DeclareMathOperator{\Cov}{Cov}      % Operator for covariance of a RV
\DeclareMathOperator*{\Lim}{l.i.m.}  % Operator for convergence in the MS sense

%%% ============================================================================================================= %%%
% Hong's New Commands and Declarations for Mathematical Analysis and Other Areas
\DeclareMathOperator{\sgn}{sgn}      % Operator for sign function
\DeclareMathOperator*{\argmax}{\arg\max}
\DeclareMathOperator{\dist}{dist}

%%% ============================================================================================================= %%%
%% New Commands for Sets or Spaces
\newcommand{\N}{\mathbb N} % for the set of natural
\newcommand{\C}{\mathbb C} % for the set of complex
\newcommand{\Q}{\mathbb Q} % for the set of rational
\newcommand{\R}{\mathbb R} % for the set of real
\newcommand{\Z}{\mathbb Z} % for the set of integer

%%% ============================================================================================================= %%%
% International Typesetting Standards
\newcommand{\me}{\mathrm{e}} % for math e
\newcommand{\mi}{\mathrm{i}} % for math i
\newcommand{\mj}{\mathrm{j}} % added by Hong for engineering j
\newcommand{\dif}{\,\mathrm{d}}% for differential

%%% ============================================================================================================= %%%
%% Hong's abbreviations of sets or spaces for this paper
\newcommand{\MRn}{\mathbb M(\R^{n})}
\newcommand{\Hn}{\mathbb H^n}
\newcommand{\Hnp}{\mathbb H^{n+}}
\newcommand{\BHn}{\mathbb B(\Hn)}

%%% ============================================================================================================= %%%
%% Hong's other settings just for this paper
%% \renewcommand{\labelenumi}{\textup{(\alph{enumi})}}
%% \setlength{\topsep}{-1pt}
%%\setlength{\itemsep}{-5pt}

%%% ============================================================================================================= %%%
%% Hong's other commands
\newcommand{\matlab}{\textsc{MATLAB}\textsuperscript{\textregistered}} % for the mathworks MATLAB registered product
\newcommand{\simulink}{Simulink\textsuperscript{\textregistered}} % for the mathworks Simulink registered product

%%% ============================================================================================================= %%%
%% Hong's Capital/Small Roman Numbers
\newcommand{\RomanNumber}[1]{\uppercase\expandafter{\romannumeral #1}}
\newcommand{\romannumber}[1]{\lowercase\expandafter{\romannumeral #1}}

%%% ============================================================================================================= %%%
%% Hong's Fonts definitions

\DeclareMathAlphabet{\mathpzc}{OT1}{pzc}{m}{it}

%% Table Raising box
\newcommand{\rb}[1]{\raisebox{1.5ex}[0pt]{#1}}

%%% ============================================================================================================= %%%
%% Dr. Gonzalez's Group definitions
\def\1{\rv 1} %Indicator
\def\I{{\mathcal I}}    %Index set
\def\E{{\mathbb E}}    %PDP state space
\def\bE{{\mathcal E}}    %Borel set
\def\x{{\rv\chi}}    %hybrid PDP state
\def\B{{\mathcal B}}    %Index set


%\renewcommand{\baselinestretch}{0.94}