\documentclass[10pt]{amsart}
%%\usepackage[subtle]{savetrees}
%\usepackage[margin=2cm]{geometry}
\usepackage{tikz,amsmath, amssymb,bm,color, amsthm,amsfonts}
\usetikzlibrary{positioning, calc,chains,fit,shapes}
%\usetikzlibrary{circuits.logic.US,circuits.logic.IEC,fit}
\usepackage{enumerate}
\usepackage{comment}
\usepackage{tikz}
\usepackage{graphics}
%\usepackage[cm]{fullpage}
\usepackage{longtable}
\usepackage{mdframed}
\usepackage{caption}
\usepackage{subcaption}
\usepackage{slashbox}
\usepackage{url}
\usepackage{framed}
\usepackage{array}
\usepackage{tabu}
\usepackage{lscape}
\usepackage{multirow}
\usepackage{ulem}
\usepackage{multicol}
\usepackage{placeins}
\usepackage{cite}
\usepackage{enumitem}
\usepackage{mathtools}
%\usepackage[numbers]{natbib}
%\usepackage{mathtools}
%\usepackage{authblk}

\mdfsetup{skipabove=2pt,skipbelow=2pt}
%\setlenght {\marginparwidth }{2cm}
%\usepackage{todonotes}

%\usepackage{floatrow}
%\usepackage{adjustbox}
%\setlength{\extrarowheight}{.05ex}
%\renewcommand\thesubfigure{\roman{subfigure}}


%\newtheorem{theorem}{Theorem}[section]
%\newtheorem{lemma}[theorem]{Lemma}
%\newtheorem{observation}[theorem]{Observation}
%\newtheorem{corollary}[theorem]{Corollary}
%\newtheorem{proposition}[theorem]{Proposition}
%\newtheorem{definition}[theorem]{Definition}
\newtheorem{construction}{Construction}
%\newtheorem{conjecture}{Conjecture}
%\newtheorem{remark}[theorem]{Remark}

\newcommand{\pname}[1]{\textnormal{\textsc{#1}}}
\newcommand{\cclass}[1]{\textnormal{\textsf{#1}}}
\newcommand{\nog}{nine} % no of members in the gang!
\newcommand{\nogd}{nineteen} % no of members in the gang - for deletion/completion
\newcommand{\nogl}{eighteen} % no of members in the larger gang - for editing
\newcommand{\nogld}{thirty eight} % no of members in the larger gang - for deletion/completion
\newcommand{\diffnog}{ten} %
%\newcommand{\dominatedby}{dominated by} %
%\newcommand{\dominatingset}{dominating set} %
%\newcommand{\dominates}{dominates} %
\newcommand{\simulates}{simulates} %
\newcommand{\baseset}{base} %
\newcommand{\issimulatedby}{is simulated by} %

\newcommand{\StarSAT}{\pname{8-SAT$_{\geq 6}$}}
\newcommand{\FSAT}{\pname{4-SAT$_{\geq 2}$}}
\newcommand{\FISAT}{\pname{5-SAT$_{\geq 3}$}}
\newcommand{\SIXSAT}{\pname{6-SAT$_{\geq 4}$}}
\newcommand{\ESAT}{\pname{8-SAT$_{\geq 6}$}}
\newcommand{\KSAT}{\pname{$k$-SAT$_{\geq {k-2}}$}}
\newcommand{\KSATO}{\pname{$k$-SAT}}
\newcommand{\ESATO}{\pname{8-SAT}}
\newcommand{\FSATO}{\pname{4-SAT}}
\newcommand{\FISATO}{\pname{5-SAT}}
\newcommand{\TSAT}{\pname{3-SAT}}
\newcommand{\HED}{\pname{${H}$-free Edge Deletion}}
\newcommand{\AEE}{\pname{${A}$-free Edge Editing}}
\newcommand{\AED}{\pname{${A}$-free Edge Deletion}}
\newcommand{\TSED}{\pname{$t$-star-free Edge Deletion}}
\newcommand{\ATSED}{\pname{Annotated $t$-star-free Edge Deletion}}
\newcommand{\AFSED}{\pname{Annotated $4$-star-free Edge Deletion}}
\newcommand{\FSED}{\pname{$4$-star-free Edge Deletion}}
\newcommand{\FVSED}{\pname{$5$-star-free Edge Deletion}}
\newcommand{\HEE}{\pname{${H}$-free Edge Editing}}
\newcommand{\HEC}{\pname{${H}$-free Edge Completion}}
\newcommand{\HDEE}{\pname{${H'}$-free Edge Editing}}
\newcommand{\HDDEE}{\pname{${H''}$-free Edge Editing}}
\newcommand{\HDED}{\pname{${H'}$-free Edge Deletion}}
\newcommand{\HDEC}{\pname{${H'}$-free Edge Completion}}
\newcommand{\HBEE}{\pname{${\overline{H}}$-free Edge Editing}}
\newcommand{\HBED}{\pname{${\overline{H}}$-free Edge Deletion}}
\newcommand{\HBEC}{\pname{${\overline{H}}$-free Edge Completion}}
\newcommand{\HOEDCE}{\pname{${H_1}$-free Edge Deletion(Completion/Editing)}}
\newcommand{\HEDCE}{\pname{${H}$-free Edge Deletion(Completion/Editing)}}
\newcommand{\HEEDC}{\pname{${H}$-free Edge Editing(Deletion/Completion)}}
\newcommand{\HDEEDC}{\pname{${H'}$-free Edge Editing(Deletion/Completion)}}
\newcommand{\BFED}{\pname{Bow-free Edge Deletion}}
\newcommand{\ABFED}{\pname{Annotated Bow-free Edge Deletion}}
\newcommand{\DTIS}{\pname{Distance-3 Independent Set}}
\newcommand{\SVC}{\pname{Strong Vertex Cover}}
\newcommand{\CLIQUE}{\pname{Clique}}
\newcommand{\IS}{\pname{Independent Set}}
\newcommand{\PFS}{\pname{Propagational-$f$ Satisfiability}}
\newcommand{\RHED}{\pname{Restricted ${H}$-free Edge Deletion}}
\newcommand{\RHEC}{\pname{Restricted ${H}$-free Edge Completion}}
\newcommand{\RHDED}{\pname{Restricted ${H'}$-free Edge Deletion}}
\newcommand{\RHDEC}{\pname{Restricted ${H'}$-free Edge Completion}}
\newcommand{\RHEE}{\pname{Restricted ${H}$-free Edge Editing}}
\newcommand{\PH}{$\cclass{NP} \subseteq \cclass{coNP/poly}$}
\newcommand{\NOPH}{$\cclass{NP} \not\subseteq \cclass{coNP/poly}$}
\newcommand{\LG}{\mathcal{W}}
\newcommand{\LGD}{\mathcal{W}'}
\newcommand{\LGDD}{\mathcal{W}''}


%\let\oldvee\vee
\renewcommand\vee{\boxtimes}

\newcommand\addvmargin[1]{
  \node[fit=(current bounding box),inner ysep=#1,inner xsep=0]{};
}
\setlength{\fboxrule}{0pt}

\newcommand{\defstage}[2]{% PGD Version
  \hfill\\\smallskip\noindent%
  \begin{tabularx}{\textwidth}{|l X|}%
    \hline%
    \multicolumn{2}{|l|}{\textbf{#1}}\\%
    &#2\\\hline%
  \end{tabularx}%
%  \smallskip%
}
\setlength\extrarowheight{15pt}

\newcounter{rowcntr}[table]
\renewcommand{\therowcntr}{\thetable.\arabic{rowcntr}}

% A new columntype to apply automatic stepping
\newcolumntype{N}{>{\refstepcounter{rowcntr}\therowcntr}c}

% Reset the rowcntr counter at each new tabular
\AtBeginEnvironment{longtabu}{\setcounter{rowcntr}{0}}

\newcounter{rowcntra}[table]
\renewcommand{\therowcntra}{\arabic{rowcntra}}

% A new columntype to apply automatic stepping
\newcolumntype{M}{>{\refstepcounter{rowcntra}\therowcntra}c}

% Reset the rowcntr counter at each new tabular
\AtBeginEnvironment{tabular}{\setcounter{rowcntra}{0}}

\newcommand{\NPC}{NP-Complete}


\newcommand{\highlight}[1]{\textcolor{blue}{#1}}
\newcommand{\dhanya}[1]{\textcolor{blue}{dhanya: #1}}


%\newcommand{\XCD1}[1]{\pname{$\chi_{cd}$\ensuremath{(#1)}}}
\newcommand{\XCD}{\pname{$\chi_{cd}$}}
\newcommand{\SC}{\pname{$\omega_{s}$}}

\newcommand{\CDC}{\textsc{CD-coloring}}
\newcommand{\SCP}{\textsc{Separated-Cluster}}
\newcommand{\TD}{\textsc{Total Domination}}
\newcommand{\ISP}{\textsc{Independent Set}}
\newcommand{\CC}{\textsc{Clique Cover}}
\newcommand{\TETHS}{Further, the problem cannot be solved in time \ensuremath{2^{o(|V(G)|)}}, unless the ETH fails}
%\usetikzlibrary{positioning,chains,shapes,calc}
\usetikzlibrary{fit}
\thispagestyle{empty}
\usetikzlibrary{
  graphs,
  graphs.standard
}
%\usepackage[utf8]{inputenc}
\usepackage{txfonts}
\usepackage[T1]{fontenc}
\usepackage{fancyhdr}
%\usepackage{newtxtext}
%\let\textfrac\undefined
%\usepackage{gfsartemisia-euler}

\usepackage{amsmath,amsthm,amssymb,hyperref, mdframed}
\usepackage{mathrsfs} 
\usepackage{tikz}
\usepackage[all]{xy}
\usetikzlibrary{decorations.markings}
\usetikzlibrary{backgrounds,shapes}
\usepackage{subfig}
\usetikzlibrary{shapes,arrows}
\usetikzlibrary{shapes.geometric, arrows}
\usetikzlibrary{decorations.markings}
\usepackage{subfig}
\usepackage{tabularx}
\usepackage{enumitem}
\usepackage{amsmath,amsthm,amscd}
\usepackage{amssymb,amsfonts}
\usepackage{fancyhdr} % This should be set AFTER setting up the page geometry
\usepackage{pgf,tikz}
\usepackage{caption}
\usepackage{tikz-cd}
\usepackage{tikzscale}
\usetikzlibrary{patterns}
\usepackage{rotating}
\usepackage{xcolor}
\usepackage{lscape}
\usepackage{array} 
\usepackage{tabularx}
\usepackage{multirow}
\usepackage{multicol}
\usepackage{booktabs} % To thicken table lines
\usepackage{caption}
%\captionsetup[table]{position=bottom} 
%\usepackage{enumerate}
\usepackage{xspace}
\usepackage{xfrac}
\usepackage{blindtext}
\usepackage[ruled,vlined]{algorithm2e}
%\usepackage{showlabels}

\usetikzlibrary{decorations.markings}
\usetikzlibrary{patterns}
\usetikzlibrary{calc}
\usetikzlibrary{shapes.geometric}

%\usepackage{fdsymbol}

\usepackage{color}
\usepackage{hyperref}
\hypersetup{
    colorlinks=true, %set true if you want colored links
    linktoc=all,     %set to all if you want both sections and subsections linked
    linkcolor=blue,  %choose some color if you want links to stand out
    citecolor=blue,
    filecolor=blue,
    urlcolor=blue
}


\setcounter{secnumdepth}{4}
\setcounter{tocdepth}{1}


%\newcommand{\cm}[1]{{\tt \textcolor{blue}{#1}} }

\definecolor{aliceblue}{rgb}{0.9, 0.95, 1.0}

\usepackage{geometry}
\geometry{a4paper,top=2.5cm,bottom=2.5cm,left=3cm,right=3cm,heightrounded,bindingoffset=0mm}

\numberwithin{equation}{section}


\newcommand\RR{{\mathcal R}}
\newcommand\RX{{\mathcal X}}
\newcommand\Z{{\mathbb Z}}
\newcommand\ziz{{\Z\oplus i\,\Z}}

\newcommand\R{{\mathbb R}}
\newcommand\Q{{\mathbb Q}}
\DeclareMathOperator\Aut{Aut}
\DeclareMathOperator\Out{Out}
\DeclareMathOperator\Mod{Mod}
\DeclareMathOperator\Inn{Inn}
\DeclareMathOperator\Hom{Hom}
\DeclareMathOperator\PGL{PGL}
\DeclareMathOperator\GL{GL}
\newcommand{\boundary}{\partial}
\newcommand{\hhat}{\widehat}
\newcommand{\C}{{\mathbb C}}
\newcommand{\HH}{{\mathbb H}}
\newcommand{\mspht}{\mathcal{M S}\mathit{ph}_{0,n}(\vartheta)}
\newcommand{\msph}{\mathcal{M S}\mathit{ph}_{0,n}}


\newcommand{\G}{{\Gamma}}
\newcommand{\s}{{\Sigma}}
\newcommand{\Te}{Teichm\"{u}ller }
\newcommand{\PSL}{{PSL_2 (\mathbb{C})}}
\newcommand{\glc}{{\mathrm{GL}(2,\mathbb{C})}}
\newcommand{\pslc}{{\mathrm{PSL}(2,\mathbb{C})}}
\newcommand{\slr}{{\mathrm{SL}(2,\mathbb{R})}}
\newcommand{\pslr}{{\mathrm{PSL}(2,\mathbb{R})}}
\newcommand{\psu}{{\mathrm{PSU}(2)}}
\newcommand{\sorr}{\mathrm{SO}(2,\R)}
\newcommand{\spzz}{\mathrm{Sp}(2,\Z)}
\newcommand{\spz}{\mathrm{Sp}(2g,\Z)}
\newcommand{\glplus}{\mathrm{GL}^+(2,\mathbb{R})}

\newcommand{\slc}{{\mathrm{SL}_2 (\mathbb{C})}}
\newcommand{\slnc}{{SL_n (\mathbb{C})}}
\newcommand{\Gr}{{\mathcal G}}
\newcommand{\affc}{\text{Aff}(\C)}
\newcommand{\pgk}{\mathcal{P}_g(k)} 
\newcommand{\cpp}{\mathbb{C}\mathrm{P}^1}
\newcommand{\cp}{\mathbb{C}\mathrm{\mathbf{P}}^1}
\newcommand{\rp}{\mathbb{R}\mathrm{\mathrm{P}}^1}
\newcommand{\hyp}{\mathbb{H}}
\newcommand{\modul}{\textnormal{Mod}(S_{g,n})}


\newcommand{\cext}{\mathcal{C}_{\textnormal{ext}}}



\newcommand{\homolz}{\text{H}_1(X,\mathbb{Z})}
\newcommand{\homolzz}{\textnormal{H}_1(X,\mathbb{Z})}

\newcommand{\shomolzn}{\textnormal{H}_1(S_{g,n},\mathbb{Z})}
\newcommand{\shomolzo}{\textnormal{H}_1(S_{g,1},\mathbb{Z})}
\newcommand{\shomolzt}{\textnormal{H}_1(S_{g,2},\mathbb{Z})}
\newcommand{\shomolzon}{\textnormal{H}_1(S_{1,\,n},\mathbb{Z})}
\newcommand{\shomolzoo}{\textnormal{H}_1(S_{1,1},\mathbb{Z})}
\newcommand{\shomolznps}{\textnormal{H}_1(S_{0,n},\mathbb{Z})}
\newcommand{\shomolz}{\textnormal{H}_1(S_{g},\mathbb{Z})}

\makeatletter
\def\namedlabel#1#2{\begingroup
   \def\@currentlabel{#2}%
   \label{#1}\endgroup
}
\makeatother

\theoremstyle{plain}                    
\newtheorem{thm}{Theorem}[section]
\newtheorem*{hthm}{Haupt's Theorem}
\newtheorem*{mthm}{Main Theorem}
\newtheorem{thma}{Theorem}
\newtheorem{propa}[thma]{Proposition}
\newtheorem{cora}[thma]{Corollary}

\renewcommand{\thethma}{\Alph{thma}}
\newtheorem{thmaa}{Theorem}
\renewcommand{\thethmaa}{\Alph{thmaa}.2}
\newtheorem{lem}[thm]{Lemma}
\newtheorem{prop}[thm]{Proposition}
\newtheorem{cor}[thm]{Corollary}
\newtheorem*{thmnn}{Theorem}
\newtheorem*{propnn}{Proposition \ref{aiffd}}
\newtheorem{conj}{Conjecture}	

\theoremstyle{definition}
\newtheorem{defn}[thm]{Definition}
\newtheorem*{defnn}{Definition}
\newtheorem{ex}[thm]{Example}
\newtheorem*{qs}{Question}
\newtheorem{ques}[thm]{Question}
\newtheorem{rmk}[thm]{Remark}
\newtheorem{py}[thm]{Properties}

%\theoremstyle{remark}



\newtheorem{claim}[thm]{Claim} 
\newcommand{\cm}[1]{\textcolor{blue}{#1}}
\newcommand*{\bigchi}{\mbox{\large$\chi$}}% big chi

\newcommand*{\defeq}{\mathrel{\vcenter{\baselineskip0.5ex \lineskiplimit0pt
                     \hbox{\scriptsize.}\hbox{\scriptsize.}}}%
                     =}





\begin{document}

\title[Simple loop conjecture for discrete representations in $\pslr$]{Simple loop conjecture for discrete representations in $\pslr$}

\author{Gianluca Faraco}
\address[Gianluca Faraco]{Dipartimento di Matematica e Applicazioni U5, Universita` degli Studi di Milano-Bicocca, Via Cozzi 55, 20125 Milano, Italy}
\email{gianluca.faraco@unimib.it}


\author[Subhojoy Gupta]{Subhojoy Gupta}
\address[Subhojoy Gupta]{Department of Mathematics, Indian Institute of Science, Bangalore, India}
\email{subhojoy@iisc.ac.in}

 
\keywords{Discrete representations, Fuchsian representations}
\subjclass[2010]{57M50}%
\date{\today}
\dedicatory{}

\begin{abstract}
We show that the Simple Loop Conjecture holds for any representation $\rho\colon\pi_1(S)\longrightarrow \pslr$ that is discrete but not faithful. That is, we show the existence of a simple closed curve in the kernel of such a representation. 
\end{abstract}

\maketitle
%\tableofcontents

\section{Introduction}\label{sec:intro}

\bigskip

\noindent Let $S$ be a closed orientable surface of genus $g\ge2$. Let $\rho\colon\pi_1(S)\longrightarrow \pslr$ be a representation; recall that $\rho$ is said to be \textit{discrete} if the image of $\rho$ is a discrete subgroup of $\pslr$ and \textit{faithful} if $\rho$ is injective. In this short note we shall prove: 

\begin{mthm}\namedlabel{thm:mainthm}{Main Theorem}
Let $\rho\colon\pi_1(S)\longrightarrow \pslr$ be a discrete representation which is not faithful. Then  $\textnormal{ker}(\,\rho\,)$ contains a simple closed curve.
\end{mthm}

\smallskip

\noindent Originally, the so-called Simple Loop Conjecture stated that if a map $f\colon S\longrightarrow \Sigma$ between closed surfaces induces a non-injective map of fundamental groups $f_*\colon\pi_1(S)\to\pi_1(\Sigma)$, then there exists a simple closed curve in $\textnormal{ker}(f_*)$. In \cite{GD}, Gabai provided a positive answer to this conjecture, see Theorem \ref{thm:gabai}. 

\smallskip

\noindent Shortly thereafter, in \cite{HJ} Hass proved that the Simple Loop Conjecture holds for maps from a closed surface $S$ to a Seifert-fibered $3$-manifold $M$. A longstanding question has been whether result still holds if the target is replaced by any orientable $3$-manifold; although this is known for graph-manifolds, see \cite{RW}, the case of a \textit{hyperbolic} $3$-manifold is still open. Recently, counterexamples to the Simple loop conjecture in dimension higher than $3$ has been found in \cite{Fara}. At the level of fundamental groups, the open question for hyperbolic $3$-manifolds could be stated as follows:

\begin{ques} Let $\rho\colon\pi_1(S)\longrightarrow \pslc$ be a discrete co-compact representation which is non-injective. Then does $\textnormal{ker}(\,\rho\,)$ always contain a simple closed curve?
\end{ques} 

\noindent If the assumption of discreteness is dropped, then it is known that the Simple Loop Conjecture fails -- counterexamples were first provided by Cooper and Manning in \cite{CM}, and this answered a question of Minsky in \cite[Question 5.3]{Minsky}. More counterexamples were described in the papers of Louder \cite{Louder} and Calegari \cite{Calegari}. For representations into $\pslr$, which is the focus of this article, explicit counterexamples were constructed by Mann in \cite{MaKat}; our theorem implies that these would necessarily be indiscrete. 

\smallskip

\noindent Another motivation for understanding the case of surface group representations into $\pslr$ is the following question of Bowditch, stated here for a closed surface $S$.

\begin{ques}\cite[Question C]{BBH}
Suppose $\rho\colon\pi_1(S)\longrightarrow \pslr$ is a non-elementary representation that is not both discrete and faithful. Then does there always exist a simple closed curve whose image is a non-hyperbolic element of $\pslr$? 
\end{ques}

%\subsection{What triggers the question?} In \cite[Question C]{BBH}, Bowditch states the following question regarding surface group representations in $\pslr$:

%\begin{quote}
 %   \textit{Suppose $\rho\colon\pi_1(S)\longrightarrow \pslr$ is a non-elementary representation such that the image of every non-peripheral simple closed curve is hyperbolic. Then must $\rho$ be Fuchsian, \textit{i.e.} discrete and faithful?}
%\end{quote}

\noindent Bowditch's question has been answered in the affirmative for surfaces of genus two by March\'e and Wolff in \cite{MW} but it is currently open for higher genus surfaces. Our result provides an affirmative answer in the special case of discrete but non-faithful representations. 

%The interests for this question arises also for its deep connection with another long-standing conjecture by Goldman, see \cite{GO97} which claims that the mapping class group action on certain components of the $\pslr-$character variety is ergodic. \cm{write better?}


%\subsection{Simple loop conjecture for representations in $\pslr$} Originally, the so-called simple loop conjecture states that if a map $f\colon S\longrightarrow \Sigma$ of closed surfaces induces a non-injective mapping of fundamental groups $f_*\colon\pi_1(S)\to\pi_1(\Sigma)$, then there exists a simple closed curve in $\textnormal{ker}(f_*)$. In \cite{GD}, Gabai provided a positive answer to this conjecture. Subsequently, several generalisations of of the simple loop conjecture has been formulated. It has been conjectured that the result still holds if the target is replaced by the fundamental group of an orientable $3$-manifold and, although special cases have been proved along the years, see \cite{HJ, RW}, the general hyperbolic case is still open. In \cite{MY}, asked whether the same conjecture holds with $\textnormal{PSL}(2,\mathbb C)$ as the target group. In this case a negative answer has been given by Cooper and Manning in \cite{CM}. Subsequently, in \cite{MaKat} Mann showed the existence of uncountably many surface group non-faithful representation $\rho\colon\pi_1(S)\longrightarrow \pslr$ that kill only no simple closed curves; that is no simple closed curve lies in the kernel of $\rho$. A key point of Mann's result is that all representations are \textit{indiscrete}. It is natural to wonder whether the simple loop conjecture for discrete representations in $\pslr$ holds. Our main Theorem \ref{thm:mainthm} aims to answer this question in positive.

\subsection*{Structure of the paper} The paper is organised as follows. In Section \S\ref{sec:fuchgroup} we recall basic  facts about surface group representations into $\pslr$ for the reader's convenience, in order to make the exposition self-contained. Sections \S\ref{sec:phr}, \S\ref{sec:crysrep} and \S\ref{sec:repnoel} are devoted to prove our main result in the following cases respectively: purely hyperbolic representations, representations with cocompact image having torsion elements, which we call \textit{crystallographic} and, finally, all remaining discrete representations, with necessarily non-cocompact image.  Our proof relies on Gabai's result that we mentioned above and results from Edmonds' work concerning maps between surfaces in \cite{EA}. 

\subsection{Acknowledgements} GF is grateful to the University of Luxembourg, in which part of the present project has been developed, for the very nice hospitality. He is also grateful to Stefano Francaviglia for his interest and good conversations about this project. Both authors would like to thank Siddhartha Gadgil for his insightful comments and suggestions. 



\bigskip


\section{Fuchsian groups and surface group representations}\label{sec:fuchgroup}

\noindent In this section we mainly recall basic and well-known facts about Fuchsian groups, i.e. discrete subgroups of $\pslr$. Good references for a full treatment of this topic are \cite{BP}, \cite{GO88} and \cite{KA}. We shall then introduce surface group representations into $\pslr$ which are discrete, and prove our main result in the case that the discrete represetation is elementary. 

\subsection{The Lie group $\pslr$} The group $\pslr$ is defined as the quotient of the Lie group $\slr$ with its center $\{\pm I\}$ and it identifies with the group of orientation-preserving isometries of $\hyp$, the upper half-plane model of the hyperbolic plane, in which the elements acts by M\"obius transformations
\begin{equation} \pslr \times \hyp \longrightarrow \hyp, \quad \left(\begin{array}{cc} 
a & b\\ c & d \\ \end{array} \right)\cdot z \longmapsto \dfrac{az+b}{cz+d}
\end{equation}
\noindent In $\pslr$, we mainly distinguish four types of elements according to the nature of the solutions of the equation
\begin{equation}\label{fp}
g(z)=\dfrac{az+b}{cz+d}=z.
\end{equation}

\SetLabelAlign{center}{\null\hfill\textbf{#1}\hfill\null}
\begin{itemize}[leftmargin=1.4em, labelwidth=1.5em, align=center, itemsep=\parskip]
\item[ \textnormal{1.}] \textit{The Identity}: Here the equation \eqref{fp} reduces to the tautological identity $z=z$ and, of course, every point is fixed.
\item[ \textnormal{2.}] \textit{Elliptics}: If the equation \eqref{fp} has two complex (but not real) solutions, then these are conjugated because the coefficients of \eqref{fp} are  real. Only one of these solutions lies in $\hyp$ and $g$ acts as a rotation around it.
\item[ \textnormal{3.}] \textit{Parabolics}: In the case the equation \eqref{fp} has only one real solution of multiplicity two. In this case, $g$ has no fixed points in $\hyp$ but it has only one fixed point on the boundary at infinity $\partial_\infty\hyp\cong\rp$. One can think of the action of $g$ as a rotation with a fixed point in $\partial_\infty\hyp$ and the orbits are given by the horocircles based at that fixed point. 
\item[ \textnormal{4.}] \textit{Hyperbolics}: In the case the equation \eqref{fp} has two real solutions, the map $g$ has no fixed point in $\hyp$ and have two fixed points on the boundary at infinity. There is a unique geodesic line (the axis of $g$) with these as its endpoints, that is invariant under the action of $g$.
\end{itemize} 

\noindent As the equation \eqref{fp} defines a polynomial of degree two with real coefficients, the list above is exhaustive and no other cases occur.


\smallskip

\subsection{Fuchsian groups} We now recall a few special classes of subgroups of $\pslr$ along with their properties which will be relevant for us in the sequel.

\begin{defn}
A \textit{Fuchsian group} is a discrete subgroup $\Gamma$ of $\pslr$; so a group of orientation-preserving isometries acting properly discontinuously on $\hyp$.
\end{defn}

\begin{defn}\label{defn:elem} 
A subgroup $\Gamma$ of $\pslr$ is said to be \textit{elementary} if there is a finite $\Gamma-$orbit in $\hyp\,\cup\,\rp$; otherwise $\Gamma$ is non-elementary.
\end{defn}


\noindent A discrete group may very well be elementary and hence it is worthwhile to have a proper description of them. In order to find all such groups we first consider the case when $\Gamma$ fixes  a single point, say $p\in\hyp$. Recall that the fixed points of elliptic elements are not in $\rp$ whereas the fixed points of parabolic and hyperbolic transformations of $\Gamma$ are in $\rp$. It is possible to show that an elementary $\Gamma$ cannot contain parabolic and hyperbolic elements at the same time because they would necessarily have a common fixed point, in which case it cannot be discrete -- see for instance \cite[Theorem 5.1.2]{BA}. We then deduce that, in the case there is a single global fixed point, $\Gamma$ can only contain elements of one type. Further considerations lead to the following result; for a proof the reader is invited to consult \cite[Theorem 2.4.3]{KA}.

\begin{prop}\label{prop:elgroups}
Any elementary Fuchsian group is either cyclic or it is conjugate to a subgroup generated by $g(z)=\lambda z$, with $\lambda>1$, and $h(z)=-\frac1z$.
\end{prop}

\noindent In the sequel we shall make use of Proposition \ref{prop:elgroups} in order to show that \ref{thm:mainthm} follows easily for elementary representations, \textit{i.e.} discrete representations into $\pslr$ whose image is an elementary subgroup, see Definition \ref{defn:elrep}. Henceforth, in Sections \S\ref{sec:phr}, \S\ref{sec:crysrep} and \S\ref{sec:repnoel} we shall have the blanket assumption that the image of the representations are non-elementary. The following proposition provides a useful characterisation of discrete subgroups of $\pslr$ that are non-elementary . For a list of equivalent characterisations see \cite[Section \S 8.4]{BA}.

\begin{prop}\label{notelgroup}
A non-elementary subgroup $\Gamma$ of $\pslr$ is discrete if and only if each elliptic element (if any) of $\Gamma$ has finite order.
\end{prop}


\begin{proof}
We provide here an outline of the proof for the reader's convenience; for details we refer to \cite[Section \S 8.4]{BA}. One implication is trivial: if $\Gamma$ is discrete, any convergent sequence of elements of $\Gamma$ is eventually constant. However, if $\Gamma$ contains an elliptic of infinite order then we may find a sequence of transformations converging to the identity which is not eventually constant, reaching a contradiction. The opposite implication relies on Selberg's Lemma: For any pair of elements, say $f,g\in\Gamma$, consider the group $\langle f,\,g\rangle$ contains a finite index subgroup, say $\Gamma_o$ without torsion elements; \textit{i.e.} without elliptics, see \cite{SA}. It can be shown that any non-elementary subgroup of $\pslr$ with no elliptics is necessarily discrete (see \cite[Theorem 8.3.1]{BA}). Hence $\Gamma_o$ is discrete and hence, being of finite index in $\langle f,\,g\rangle$, the latter group is also discrete. Finally, one can show that a non-elementary group $\Gamma$ is discrete if and only if $\langle f,\,g\rangle$ for any pair $f,g\in\Gamma$, (see \cite[Theorem 5.4.2]{BA}), using the so-called J\o rgensen inequality (see \cite[Theorem 5.4.1]{BA}). This completes the proof.
\end{proof}

\smallskip

%However, there may be limit points on the boundary at infinity $\rp$. Let $\Lambda(\Gamma)$ be the limit set of $\Gamma$, \textit{i.e.} the set of limit points of $\Gamma\cdot z$ for $z\in\hyp$. Then $\Lambda(\Gamma)\subset\rp$. The limit set may be empty, or may contain one or two points if $\Gamma$ is elementary, or may contain an infinite number, see \cite[Theorem 3.4.6]{KA}. In the latter case, there are two types of Fuchsian groups.

%\begin{itemize}
 %   \item[1.] A Fuchsian group of the \textit{first type} is a group for which the limit set is the entire $\rp$. This happens if the quotient space $\hyp/\Gamma$ has finite volume.
    
  %  \smallskip
  %  \item[2.] Otherwise, a Fuchsian group is said to be of the \textit{second type}. One can in fact show that the limit set is a perfect set that is nowhere dense on $\rp$, i.e. a Cantor set.
%\end{itemize}


\noindent Let $\Gamma$ be a Fuchsian group. Being $\Gamma$ discrete, its action on the hyperbolic plane is properly discontinuous; such an action admits a fundamental domain. Moreover, note that in this paper, we deal only with finitely generated Fuchsian groups, since the fundamental group $\pi_1(S)$ is a finitely generated. For such a Fuchsian group $\Gamma$ its Dirichlet domain $\mathcal D(\Gamma)$ is a fundamental domain that enjoys nice properties like a local finiteness which guarantees that $\hyp/\Gamma\cong\mathcal D(\Gamma)/\Gamma$, see \cite[Section \S9]{BA}. A Fuchsian group $\Gamma$ is said to be \textit{cocompact} if the identification space $\hyp/\Gamma$ is compact. Of course, if the Dirichlet domain $\mathcal D(\Gamma)$ is compact, then the identification space is compact and $\Gamma$ does not contain any parabolic elements (see \cite[Theorem 4.2.1]{KA}). On the other hand, if $\mathcal D(\Gamma)$ is not compact then neither is $\hyp/\Gamma$. Moreover, if $\mathcal D(\Gamma)$ has finite hyperbolic area but it is not compact, then $\Gamma$ contains  parabolic elements (see \cite[Theorem 4.2.2]{KA}).

%Being finitely generated implies they are also geometrically finite \cite[Theorem 4.6.1]{KA}. In our setup, if the Dirichlet domain associated to a Fuchsian group $\Gamma$ has finite hyperbolic area, then $\Gamma$ is of the first kind, see \cite[Theorem 4.5.2]{KA}. Vice versa, if $\Gamma$ is of the second kind, then it cannot be cocompact.

%\begin{rmk}\label{rmk:noparcons}
%An upshot of the preceding discussion is the following fact: If a Fuchsian group $\Gamma$ with Dirichlet domain $\mathcal D(\Gamma)$ does not contain  parabolic elements then $\Gamma$ is cocompact (hence of the first kind) or $\mathcal D(\Gamma)$ has infinite hyperbolic area and $\Gamma$ is of the second kind. In what follows, this observation shall play a role in distingishing the cocompact case from the non-cocompact one.
%\end{rmk}


\subsection{Surface group representations}\label{ssec:sgr} Let $S$ be a closed genus $g$ surface of hyperbolic type, \textit{i.e.} $2-2g<0$, and let $\pi_1(S)$ denote its fundamental group. A \textit{surface group representation}, or often in this paper, simply a \textit{representation}, is a homomorphism $\rho:\pi_1(S)\longrightarrow \pslr$. 

\smallskip 

\noindent For a surface $S$, let $\mathcal R_g=\textnormal{Hom}\big(\,\pi_1(S),\,\pslr\,\big)$ denote the representation space. Each representation is uniquely determined by the images of $2g$ generators of $\pi_1(S)$, which satisfy the relation given by the products of commutators. The geometry and topology of this space of representations  has been studied extensively in last decades. In his work \cite{GO88}, Goldman showed that the representation space exhibits $4g-3$ connected components which are distinguish by the \textit{Euler number}\,; a topological invariant $\mathcal E(\rho)$ naturally attached to a representation $\rho$. In brief, the Euler number measures the failure for the existence of a global section of the $\rp-$bundle over $S$ with holonomy $\rho$, see \cite{GO88} and \cite{MI}. The Euler number has to obey the so-called Milnor-Wood inequality; \textit{i.e.} $|\,\mathcal E(\rho)\,|\le 2g-2$. The \textit{maximal components} of $\mathcal R_g$ are those connected components of representations that satisfy $\mathcal E(\rho)=\pm2g-2$. These are two diffeomorphic copies of the Teichm\"uller space of $S$. We now introduce a class of representations which is of particular relevance for us.
 
\begin{defn}[Discrete representation]\label{defn:fuch}
A representation $\rho\colon\pi_1(S)\longrightarrow \pslr$ is said to be \textit{discrete} if the subgroup $\text{Im}(\,\rho\,)=\Gamma$ of $\pslr$ is a Fuchsian group. Furthermore, a discrete representation $\rho$ is said to be \textit{Fuchsian} if, in addition, it is a faithful representation.
\end{defn}

\noindent Maximal components of the representation space $\mathcal R_g$ comprises all Fuchsian representations, which are in addition faithful representations. On the other hand,  in each \textit{non}-maximal connected component,  discrete representations are somewhat rare as they form a closed subset which is a non-empty and nowhere dense, see \cite[Proposition 1.3]{FW}.

\smallskip 

\noindent Recalling Definition \ref{defn:elem}, we can also define:

\begin{defn}[Elementary representation]\label{defn:elrep} 
A representation $\rho\colon\pi_1(S)\longrightarrow \pslr$ is said to be \textit{elementary} if the subgroup $\text{Im}(\,\rho\,)=\Gamma$ is an elementary subgroup of $\pslr$.
\end{defn}

\noindent In the next subsection we focus on this type of representations and, from Section \S\ref{sec:phr} all the representations will be supposed non-elementary.


\smallskip

\subsection{\ref{thm:mainthm} for elementary representations} In the present subsection we aim to prove the Simple Loop Conjecture for elementary representations, see Proposition \ref{prop:mainele} below. We shall make use of the following terminology.

\begin{defn}[Handle, handle-generators]\label{handle} On surface $S$ of genus $g\ge2$, a \textit{handle} is an embedded subsurface $\Sigma$ that is homeomorphic to a punctured tours, and a \textit{handle-generator} is a simple closed curve; \textit{i.e.} one of the generators of $\text{H}_1(\Sigma, \mathbb{Z})$. A \textit{pair of handle-generators} for a handle will refer to a pair of simple closed curves $\{\alpha, \beta\}$ that generate $\text{H}_1(\Sigma, \mathbb{Z})$; in particular, $\alpha$ and $\beta$ intersect once. 
\end{defn} 

\noindent Note that, according to the definition just given, for a surface $S$ of hyperbolic type its fundamental group is generated by $g$ suitable pairs of handle generators $\{\alpha_1,\beta_1,\dots,\alpha_g,\beta_g\}$. We define such a set of generators as \textit{system of handle generators}.

\begin{prop}\label{prop:mainele}
Let $S$ be a surface of genus $g\ge2$ and let $\rho\colon\pi_1(S)\longrightarrow\pslr$ be an elementary representation. Then $\textnormal{ker}(\rho)$ contains a simple closed curve.
\end{prop}

\noindent Before moving on to the proof of Proposition \ref{prop:mainele}, a few preliminaries are in order. Let $\rho$ be an elementary representation and suppose, up to conjugation, that
\begin{equation}
    \Gamma=\textnormal{Im}(\,\rho\,)=\,\langle\, g,\,h\,\rangle;
\end{equation} where $g,\,h\in\pslr$ are as in Proposition \ref{prop:elgroups}. We may observe that $\Gamma$ is a subgroup of the larger group of transformations defined as
\begin{equation}
    \left\{\, z\mapsto\,\lambda\,z^{\varepsilon}\,|\, \lambda \in \mathbb R^*\,\,\textnormal{ and }\,\, \varepsilon=\pm1\,\right\}.
\end{equation}
\noindent Since $\Gamma$ is generated by $g(z)=\lambda\,z$ and $h(z)=-z^{-1}$, there is a handle generator $\alpha_i$ such that $\rho(\alpha_i)(z)=\lambda\,z$ and there is another generator $\beta_j$ such that $\rho(\beta_j)(z)=-z^{-1}$. In principle, $\alpha_i$ and $\beta_j$ may belong to different handles. Observe that the map  $\varepsilon:\Gamma\longrightarrow \{\pm1\}\cong\Z_2$ is a homomorphism. We now invoke \cite[Theorem 1.2]{EA1} to claim the following

\begin{prop}
The action of $\textnormal{Aut}^+\left(\pi_1(S)\right)$ on the set $\textnormal{Epi}(\pi_1(S),\,\Z_n)$ is transitive.
\end{prop}

\noindent As a consequence, we claim the existence of a basis $\{\alpha_1,\beta_1,\dots,\alpha_g,\beta_g\}$ of handle generators such that
\begin{equation}\label{eq:redform}
    \varepsilon(\,\rho(\alpha_1)\,)=-1 \,\quad\,\text{ and }\,\quad\, \varepsilon(\,\rho(\delta)\,)=1
\end{equation}
for all $\delta\in\{\beta_1,\dots,\alpha_g,\beta_g\}$. 

\begin{proof}[Proof of Proposition \ref{prop:mainele}]
Let $\rho$ be an elementary representation and let $\Gamma$ be its image. By definition it is an elementary group. According to Proposition \ref{prop:elgroups}, the group $\Gamma$ is either cyclic or conjugated to a group of the form $\left\langle\, g,\,h\,\right\rangle$ with $g(z)=\lambda z$ and $h(z)=-1/z$. In the first case, the group $\Gamma$ is abelian and the commutator $\gamma=[\alpha,\,\beta]$ of any pair of handle generators lies in the kernel of $\rho$. Since $\gamma$ is a simple closed curve we are done. Therefore assume to be in the second case. According to the discussion above, there is a system of handle generators $\{\alpha_1,\beta_1,\dots,\alpha_g,\beta_g\}$ that satisfies the reduced form above, see \eqref{eq:redform}. For all $i\ge2$ it follows that $\rho\big(\,[\alpha_i,\,\beta_i]\,\big)=I$. Since the commutator of a pair of handle generators is a simple closed separating curve we get the desired result.
\end{proof}

\noindent It remains to deal with the case of non-elementary and discrete representations in $\pslr$. %Before moving to consider this case, we briefly introduce hyperbolic structures on surfaces.

\smallskip


\section{Purely hyperbolic cocompact representations}\label{sec:phr}

\noindent In the present section we prove \ref{thm:mainthm} for a particular class of discrete representations, in which case it reduces to Gabai's theorem on the Simple Loop Conjecture. 

\begin{defn}\label{def:phr} A Fuchsian group $\Gamma$ is said to be \textit{purely hyperbolic} if it contains only hyperbolic elements and the identity. A discrete representation $\rho\colon\pi_1(S)\longrightarrow \pslr$ is said to be \textit{purely hyperbolic} if its image is a purely hyperbolic Fuchsian group.
\end{defn}

\noindent This class of representations clearly includes Fuchsian representations, \textit{i.e.} faithful and discrete representations in $\pslr$. However, for the \ref{thm:mainthm} we need to focus on non-faithful representations, and also by virtue of the discussion in \S2.4, only on those that are \textit{non-elementary}. In addition, in this section we shall only consider the case when the image of the purely hyperbolic representation $\rho$ is cocompact. 

\smallskip

\noindent Let $\rho\colon\pi_1(S)\longrightarrow \pslr$ be a non-elementary and purely hyperbolic cocompact representation. Let $\Gamma$ be the image of $\rho$. Since $\Gamma$ is purely hyperbolic, its action on $\hyp$ is free and properly discontinuous and the identification space $\hyp/\Gamma\cong\mathcal D(\Gamma)\cong\Sigma$ is a closed surface of hyperbolic type equipped with a complete hyperbolic structure with Fuchsian holonomy representation $\sigma\colon\pi_1(\Sigma)\longrightarrow \pslr$. Since $\sigma$ is Fuchsian, it is by definition an injective representation and hence $\pi_1(\Sigma)$ identifies with $\Gamma$. Therefore, there is a well-defined epimorphism $f_*\colon\pi_1(S)\longrightarrow \pi_1(\Sigma)$ such that $\rho=\sigma\circ f_*$. Since surfaces are $K(\pi_1,\,1)$ spaces, the map $f$ is induced by a map $f\colon S\longrightarrow \Sigma$ which is unique up to homotopy. If $\rho$ is faithful, hence a Fuchsian representation, then no simple closed curve is mapped to the identity because $\textnormal{ker}(\rho)$ is trivial. Note that, in this case we would have $\pi_1(S)\cong\Gamma\cong\pi_1(\Sigma)$; in particular $S\cong\Sigma$. However since $\rho$ is a non-faithful representation, we can invoke the Theorem of Gabai mentioned in \S\ref{sec:intro}, see \cite{GD}:

\begin{thm}\label{thm:gabai} 
If $f\colon S\to \Sigma$ is a map  between closed connected surfaces such that $f_*\colon \pi_1(S)\to \pi_1(\Sigma)$ is not injective, then there exists an essential simple closed curve $\gamma\subset S$ such that $f(\gamma)$ is homotopically trivial. 
\end{thm}

\noindent In other words, if $f_*$ is not injective, then there exists an essential (non-contractible) simple closed curve $\gamma$ in $\textnormal{ker}(f_*)$. Since the representation $\sigma$ is Fuchsian (in particular, injective)  it follows that $f_*\colon\pi_1(S)\longrightarrow \pi_1(\Sigma)$ is \textit{not} injective. There is an essential simple closed curve $\gamma\in\pi_1(S)$ such that $f_*(\gamma)=1_{\Sigma}$; as a consequence
\begin{equation}
    \rho(\gamma)=\sigma\circ f_*(\gamma)=1_S
\end{equation}
and we have found a simple closed curve in $\textnormal{ker}(\rho)$. Therefore the \ref{thm:mainthm} holds for purely hyperbolic cocompact representations.

\smallskip

\section{Crystallographic representations}\label{sec:crysrep}

\noindent In the present section we still consider cocompact Fuchsian subgroups $\Gamma$ of $\pslr$ and we drop the condition on $\Gamma$ to be purely hyperbolic. In other words, we now allow a Fuchsian group to have elliptic transformations. Recall that according to Proposition \ref{notelgroup} every such elliptic transformation has finite order. 

\begin{defn}
A group $\Gamma<\pslr$ is said to be \textit{crystallographic} if it a cocompact Fuchsian group; that is $\Gamma$ acts properly discontinuously on $\hyp$ and the identification space $\hyp/\Gamma$ is compact. A discrete representation $\rho\colon\pi_1(S)\longrightarrow \pslr$ is said to be \textit{crystallographic} if its image is a crystallographic group.
\end{defn}


\noindent A crystallographic group in the above sense has a presentation of the form
\begin{equation}\label{eq:crysgroup}
    \Gamma(g,\,m_1,\dots,m_k)=\left \langle\, a_1,b_1,\dots,a_g,b_g,c_1,\dots,c_k\,|\, c_i^{m_i}=1,\,[a_1,b_1]\cdots[a_g,b_g]c_1\cdots c_k=1 \right\rangle
\end{equation} 
where $g\ge0$ and $k\ge3$ if $g=0$, such that
\begin{equation}
    2-2g - \sum_{i=1}^k \left(1-\frac{1}{m_i} \right) <0.
\end{equation}
\noindent For instance, all triangle groups are crystallographic groups, see \cite{JW} for a nice exposition of the topic. In this section, we prove our \ref{thm:mainthm} for crystallographic representations. We begin with the following

\begin{rmk}
For a crystallographic group $\Gamma$, the quotient space $\Sigma=\mathbb H/\Gamma$ carries an orbifold hyperbolic structure on $S$, with cone points $\{p_1,\dots,p_k\}$ of angles $2\pi/m_i$. Its holonomy representation is a faithful representation $\chi\colon\pi_1^{\textnormal{orb}}(\Sigma)\longrightarrow \pslr$, where the domain, known as \textit{orbifold fundamental group}, is defined as: 
\begin{equation}\label{eq:orb} 
    \pi_1^{\textnormal{orb}}(\Sigma)\cong\left \langle\, \alpha_1,\beta_1,\dots,\alpha_g,\beta_g,\gamma_1,\dots,\gamma_k\,|\, \gamma_i^{m_i}=1,\,[\alpha_1,\beta_1]\cdots[\alpha_g,\beta_g]\gamma_1\cdots\gamma_k=1 \right\rangle
\end{equation}
with $\{\alpha_1,\beta_1,\dots,\alpha_g,\beta_g\}$ be a set of handle generators and $\gamma_i$ is a simple loop around $p_i$ for $i=1,\dots,k$. We denote by $\pi_{\textnormal{orb}}\colon\pi_1^{\textnormal{orb}}(\Sigma)\longrightarrow \pi_1(\Sigma)$ the natural forgetful map on the fundamental group of the underlying topological surface.
\end{rmk}

\smallskip

\noindent Let $S$ be a closed oriented surface of genus $g\geq 2$ and let $\rho\colon\pi_1(S) \to \pslr$ be a representation whose image $\Gamma$ is a co-compact discrete subgroup of $\pslr$ with torsion.  Let $\Sigma = \HH/\Gamma$ be the corresponding hyperbolic orbifold. There is a continuous $\rho$-equivariant map $\widetilde{f}\colon\widetilde{S} \longrightarrow \HH$ that descends to a continuous map $f\colon S \to \Sigma$. Here $f_*\colon \pi_1(S)\longrightarrow \pi_1(\Sigma)$ does not coincide with $\rho$ because $\Gamma=\pi_1^{\textnormal{orb}}(\Sigma)$, and in fact $f_\ast = \pi_{\text{orb}} \circ \rho$. Gabai's Theorem ensures the existence of a simple closed curve, say $\gamma$, such that $f(\gamma)$ is null-homotopic in the target surface $\Sigma$. In order to show that $\gamma\in\text{ker}(\rho)$ we need to show that the (possibly immersed) disk $D$ that $f(\gamma)$ bounds does not include any cone point of the orbifold $\Sigma$.

\smallskip

%\noindent Let $\gamma$ be a simple closed curve provided by Gabai's Theorem. If $\gamma$ is non-separating then there exists a path $\delta\subset \Sigma$ joining two distinct points such that $f(\gamma)=\delta\,\cup\,-\delta$. In particular this is a trivial closed loop in $\Gamma$ and we are done in this case. Assume $\gamma$ is separating in $S$. If $f(\gamma)$ does not enclose any cone point then $f(\gamma)$ is trivial in $\Gamma=\pi_1^{\text{orb}}(\Sigma)$. In particular $\gamma \in\text{ker}(\rho)$ and there is nothing to show in this case. On the other hand, if $f(\gamma)$ encloses a cone point we cannot directly infer that $\gamma\in\text{ker}(\rho)$. 

\begin{rmk}\label{rmk:enclose}
Let $\gamma$ be a simple closed curve provided by Gabai's Theorem. Since $f(\gamma)$ is homotopic to a point,  $f(\gamma)$ is the boundary of an immersed disk $D$ (we shall describe this as ``$f(\gamma)$ encloses $D$"). If $D$ is disjoint from the set of cone-points then $f(\gamma)$ is trivial in $\Gamma=\pi_1^{\text{orb}}(\Sigma)$. In particular, in this case $\gamma \in\text{ker}(\rho)$ and we are done. On the other hand, if $f(\gamma)$ encloses a cone-point we cannot directly infer that $\gamma\in\text{ker}(\rho)$. 
\end{rmk}

\smallskip

\noindent In the light of Remark \ref{rmk:enclose} above, we proceed as follows. Let $p$ be any cone point, let $D$ be a closed disc centered at $p$ and consider the preimage $f^{-1}(D)=U_1\,\cup\,\cdots\,\cup\,U_k$. In principle, some components may have positive genus. Up to perturbing $f$ a little bit with a homotopy, we may assume two different connected components have disjoint boundaries. Let $U$ be any connected component of $f^{-1}(D)$ and consider the restriction $f_{|U}\colon U\longrightarrow D$. According to \cite[pag. 116]{EA} we give the following

\begin{defn}\label{def:allowable}
$f_{|U}$ is an \textit{allowable map} if $f_{|\partial U}\colon\partial U\longrightarrow \partial D$ is a covering.
\end{defn}

\noindent The mapping $f$ defined above need not to be an allowable map when restricted to any preimage of $D$. We show how to modify $f$ in such a way the restriction is allowable in the sense of Definition \ref{def:allowable} above. More precisely we prove the following

\begin{lem}\label{lem:makeall}
We may assume $f_{|U}$ is an allowable map for any connected component $U$ of $f^{-1}(D)$.
\end{lem}

\begin{proof}
Let $U$ be any connected component of $f^{-1}(D)$ and let $\partial U$ be its boundary. Let $A$ be a closed collar containing $\partial U$ in its interior and denote by $\delta_l$ and $\delta_r$ the left and right boundary of $A$, where left and right clearly depends on the orientation of $\gamma$ -- here we do not need to rely on any particular orientation so lets fix one of them. Let $A_l$ denote the annulus having $\delta_l$ and $\partial U$ as the boundary components. In the same fashion let $A_r$ denote the annulus having $\delta_r$ and $\partial U$ as the boundary components. Clearly $\delta_l\cong \mathbb S^1$ and we can regard $f_{|\,\delta_l}$ as an element in $\textnormal{Map}(\mathbb S^1,\,\mathbb S^1)$ with non-zero degree $\textnormal{deg}(f_{|\,\delta_l})=n$. Therefore $f_{|\,\delta_l}\simeq g$ where $g\colon\mathbb S^1\to\mathbb S^1$ is the covering map $z\longmapsto z^n$. There is a homotopy
\begin{equation}
    \mathcal H_l\colon[-1,\,0]\,\times\,\mathbb S^1\longrightarrow \mathbb S^1
\end{equation}
such that $\mathcal H_l\,(-1,\,p)=f_{|\,\delta_l}(p)$ and $\mathcal H_l\,(0,\,p)=g(p)=p^n$. On the other hand, we can also regard $f_{|\,\delta_r}$ as an element in $\textnormal{Map}(\mathbb S^1,\,\mathbb S^1)$ of the same degree $n$ and hence $f_{|\,\delta_r}\simeq g$. In particular there is a homotopy 
\begin{equation}
    \mathcal H_r\colon[0,\,1]\,\times\,\mathbb S^1\longrightarrow \mathbb S^1
\end{equation}
such that $\mathcal H_r\,(0,\,p)=g(p)=p^n$ and $\mathcal H_r\,(1,\,p)=f_{|\,\delta_r}(p)$. By identifying $A$ with $[-1,\,1]\times\mathbb S^1$, these two homotopies can be glued to obtain a continuous function $\mathcal H\colon A\longrightarrow f(A)\subset \Sigma$. Since $f\equiv \mathcal H$ on the set $\{\delta_l,\,\delta_r\}=\partial A$ by construction, we can replace $f_{|A}$ with the function $\mathcal H$. With a small abuse of notation let us still denote the new function with $f$ and observe it lifts to a $\rho-$equivariant mapping $\widetilde S\longrightarrow \mathbb H$. By construction we have that $\partial U=\{\,0\,\}\times \mathbb S^1\subset A$ and $\mathcal H(0,\,-\,)=g$; therefore $f_{|\partial U}\colon\partial U\longrightarrow\partial D$ is a covering map and hence $f_{|U}$ is allowable.
\end{proof}

\noindent We now invoke the following result, see \cite[Theorem 3.1]{EA}.

\begin{thm}\label{thm:edmbond}
Let $h\colon M \longrightarrow N$ be an allowable map of nonzero degree between connected, compact, oriented surfaces. Then there is a pinch map $\pi\colon M\longrightarrow Q$ and a branched covering $\beta\colon Q\longrightarrow N$ such that $h\simeq \beta\circ \pi$ \textnormal{rel} $\partial M$.
\end{thm}

\noindent According \cite{EA}, recall that a mapping $f\colon S\to \Sigma$ is called \textit{pinch map} if there is a compact and connected sub-manifold, say $S'\subset S$, with boundary consisting of a single simple closed curve in the interior of $S$, such that $\Sigma\cong S\,/\,S'$, the quotient of $S$ with $S'$ identified to a point, such that $f$ is the quotient map. Note that this is equivalent to say that $\deg(\,f\,)=1$ and there are two simple closed non-contractible curves, say $\alpha$ and $\beta$, meeting transversally at a single point such that $f(\alpha)$ and $f(\beta)$ are both contractible in $\Sigma$.

\medskip

\noindent We apply this latter result to our mapping $f_{|U}\colon U\longrightarrow D$ to claim that, up to homotopy, we may assume it as a composition of a pinching map with a branched covering. We shall now explain how the case when $U$ has positive genus is easier to handle.

\begin{rmk}
First, recall that \textit{winding number} of a closed curve in a disk around a given point is an integer representing the total number of times that curve travels counterclockwise around the point itself. In our case, if $p$ is a cone point of magnitude $\frac{2\pi}{m}$, then a simple closed curve $\delta\subset U\subset S$ belongs to $\textnormal{ker}(\rho)$ if the winding number of its image via $f_{|U}$ is an integral multiple of $m$.
\end{rmk}

\noindent Assume $U$ has positive genus and let $\alpha$ be a handle generator. Then $f_{|U}(\alpha)$ is a loop in $D$ and it is necessarily null-homotopic because $D$ is a disk. Recall that $p$ is the chosen  cone-point. If the winding number $w\left( f_{|U}(\alpha),\,p\right)=k\,m$ with $k\in\mathbb Z$, then $f_{|U}(\alpha)$ is trivial in $\Gamma$ and hence $\alpha\in \text{ker}(\rho)$. In the case the winding number of $f_{|U}(\alpha)$ is not a multiple of $m$ we consider a handle generator $\beta$ such that $i(\alpha,\,\beta)=1$; where $i(\,\cdot,\,\cdot\,)$ denotes as usual the intersection number. Once again, since $D$ is a disk, $f_{|U}(\beta)$ is a null-homotopic loop. If the winding number $w\left( f_{|U}(\beta),\,p\right)=k\,m$, then $f_{|U}(\beta)$ is trivial in $\Gamma$ and hence $\beta\in \text{ker}(\rho)$. Even if this is not the case, it always follows that $\delta=[\alpha,\,\beta]$ is a closed separating curve in $U\subset S$ and $\delta\in \text{ker}(\rho)$ because the winding number of $f_{|U}(\delta)$ around $p$ is zero. 

\smallskip

\noindent Henceforth, assume $U$ has genus zero and $f_{|U}\colon U\longrightarrow D$ is a branched covering of degree $\deg(f_{|U})\leq \deg(f)$ where the equality holds if $U=f^{-1}(D)$. Note that we are assuming that $f_{|U}$ does not admit a pinch map as a factor, as otherwise we will be done. If $p$ is a regular value then it has exactly $\deg(f_{|U})$ preimages and they all are regular points. Otherwise, if $p$ is a branched value we perturb $f_{|U}$ a little by changing it by a homotopy if necessary. So we may always assume $p$ to be a regular value for $f_{|U}$. 

\medskip

\noindent Note that what we have discussed so far applies to any connected component $U$ of $f^{-1}(D)$. Two possible situations may happen. Either:
\begin{itemize}
    \item[1.] There is a cone point $p\in\Sigma$ and a closed disk $D$ containing $p$ (and no other cone points) such that at least one connected component of $f^{-1}(D)$ has positive genus, or
    \item[2.] for every cone point $p$ and closed disk $D$ around $p$ all the connected components of $f^{-1}(D)$ have genus $0$ (together with some boundary components).
\end{itemize}

\noindent In the first case we have already proved the existence of some simple closed curve $\gamma\in\textnormal{ker}(\rho)$. Therefore let us assume to be in the second case. Let $p_1,\dots,p_\ell$ be the cone points of $\Sigma$ and let $D_i$ denote a small closed disk around $p_i$ that contains no cone points other than $p_i$. We apply Lemma \ref{lem:makeall} to all boundary components of $f^{-1}(D_i)$ for all $i=1,\dots,\ell$ so that we may assume the restriction of $f$ to each connected component of the cut surface 
\begin{equation}\label{eq:cut1}
    S\setminus \bigcup_{i=1}^\ell f^{-1}(\partial D_i)
\end{equation}
is an allowable map. We next apply Edmonds' Theorem \ref{thm:edmbond} to the restriction of $f$ to every connected component of the cut surface. Note that, in principle, the surface 
\begin{equation}\label{eq:cut2}
    S'=S\setminus \bigcup_{i=1}^\ell f^{-1}(D_i^\circ)
\end{equation}
may very well be disconnected. The restriction of $f$ to any connected component, say $U$, of $S'$ is a map $f\vert_U :U \to \Sigma \setminus (D_1 \cup D_2 \cup\cdots\cup D_\ell)$. If it admits a pinch then we can find a simple closed curve in $\textnormal{ker}(\rho)$ because the  target surface has all the disks $D_i$ removed, so the image of the curve is null-homotopic in the $\ell$-holed surface, and consequently also in $\Sigma$. If the restriction of $f$ to every connected component does not admit a pinching map, that restriction is a branched covering for each component. Moreover, by the allowability of these restrictions, each map also restricts to a standard covering map on each boundary circle. Gluing these maps back along the boundary components, we can thus assume that the entire map $f\colon S\longrightarrow \Sigma$ is a branched covering. The assumption that $f$ is a branched covering will hold in the remaining part of the argument. 

\smallskip

\noindent Now let $\gamma\subset S$ be a simple closed curve provided by Gabai's Theorem, %We now need to distinguish two cases depending on whether $\gamma$ is separating or not separating. Let us assume first $\gamma$ to be separating. 
\textit{i.e.} $f(\gamma)$ is null-homotopic in $\Sigma$. In fact, Gabai's proof shows that we could choose $f$, in its the homotopy class, such that $f(\gamma)$ it bounds a disk, say $D$, in $\Sigma$. See also the modification of $\gamma$ sketched in page 9. Let $S_o$ be a subsurface bounded by some components of $f^{-1}\big(\partial D\big)$ such that the restriction $f_{\vert S_o}\colon S_o\longrightarrow D$ is a branched covering and $\chi(S_o)\le 0$. This exists since the entire map $f$ is a branched covering, and at least one component of $f^{-1}(D)$ is not a disk otherwise $\gamma$ would be not essential. Riemann-Hurwitz formula implies that 
\begin{equation}
    \chi(S_o) = \deg(\,f\,)\cdot\chi(D) - \beta
\end{equation}

\noindent where $\beta$ is a measure of the "total branching".  Since $\chi (S_o) \le0$ and $\chi(D) =1$, we obtain that $\beta\ge \deg(\,f\,)$. Note that if $D$ contains no orbifold points (i.e.\ cone-points), then for any simple closed curve $\alpha$ in $S_o$, the image $f(\alpha)$ is trivial in the orbifold fundamental group, and we are done. 

\smallskip

\noindent Suppose now that $D$ contains $k\geq 1$ orbifold points. By construction no branch point of $f$ coincides with a cone-point in $\Sigma$. Remove small disks around each orbifold point from $D$, to get a $k$-holed disk $D_o$ which has Euler characteristic $\chi(D_o) = 1-k$. Since the map $f_{|S_o}$ is a $\deg(\,f\,)$-fold covering map away from the branch-points, each such disk has exactly $\deg(\,f\,)$ preimages in $S_o$. Removing them also, we obtain a surface-with-boundary $S_o'$ having Euler characteristic $\chi(S_o) - k\deg(\,f\,)$, and a branched covering $f_{\vert_{S_o^\prime}} \colon S_o^\prime \to D_o$.  We now apply Riemann-Hurwitz's Theorem to this branched cover and we get $\chi(S_o^\prime) =  (1-k)\,\deg(\,f\,) - \beta$. On the other hand, since $\beta\ge\deg(\,f\,)$, we conclude that $\chi(S_o^\prime) \le -k\deg(\,f\,)$. Hence $\chi(S^\prime) \le \big(\chi(D_o) -1\big)\deg(\,f\,)$. In this case we need to invoke the following result of Edmonds, see \cite[Theorem 4.5]{EA}

\begin{thm}\label{edm4.5}
Let $f\colon S\longrightarrow \Sigma$ be an allowable map between connected, compact, orientable surfaces. Suppose in addition that $\chi(S)\le |\deg(\,f\,)|\,\Big(\,\chi(\Sigma)-1\,\Big)$. Then there is a non-contractible simple closed curve $\gamma\subset S$ such that $f(\gamma)$ is null-homotopic.
\end{thm}

\noindent Note from the preceding discussion that the hypotheses of the above theorem are satisfied. So, by applying that Theorem, there is a simple closed curve $\alpha$ on $S_o^\prime$ whose image under $f$ is null-homotopic in $D_o$. Since the target surface has all the orbifold points removed, such a null-homotopy avoids them, and $f(\alpha)$ is null-homotopic in the orbifold fundamental group. Therefore \ref{thm:mainthm} holds for crystallographic representations.

\section{Non-cocompact representations}\label{sec:repnoel}

\noindent It remains to handle the case of discrete representations $\rho\colon\pi_1(S)\longrightarrow \pslr$ with non-cocompact image $\text{Im}(\rho)=\Gamma$; \textit{i.e.} $\Gamma$ is discrete but $\mathbb{H}/\Gamma$ is not compact. We shall refer to such representations as \textit{non-cocompact} representations. 

\smallskip

\noindent Let $\rho\colon\pi_1(S)\longrightarrow \pslr$ be a non-cocompact representation and let $\mathcal D(\Gamma)$ be the Dirichlet domain of $\Gamma$. Note that $\pi_1(S)$ is a finitely generated group and hence so is the image $\Gamma$. Since $\Gamma$ is a Fuchsian group but non-cocompact, $\mathcal D(\Gamma)$ is not compact and $\mathbb H/\Gamma\cong \Sigma_{g,n}$, where $\Sigma_{g,\,n}$ is a finite type surface of genus $g$ and $n\ge1$ punctures of hyperbolic type; \textit{i.e.} $2-2g-n<0$. 

\smallskip

\noindent Assume first that $\Gamma$ is torsion-free. There exists a $\rho-$equivariant map $\widetilde f\colon \widetilde S\longrightarrow \mathbb H$ which induces a map $f\colon S\longrightarrow \Sigma_{g,n}$. In this case $\pi_1(\Sigma_{g,n})\cong\Gamma$, and $f_\ast:\pi_1(S) \to \Gamma$ is surjective and coincides with $\rho$. Since $\Sigma_{g,n}$ is a punctured surface, it is homotopically equivalent to a bouquet $T$ of $2g+n-1$ circles, \textit{i.e.} a wedge of $2g+n-1$ curves at some point $t_o\in T$. In other words, for a point $s_o\in\Sigma$ there is a retraction $r\colon \Sigma_{g,n}\longrightarrow T$ mapping $s_o$ to $t_o$ satisfying $r\circ j=\text{id}_T$ where  $j\colon T\longrightarrow \Sigma_{g,n}$ is the inclusion map. Since $j\circ r\simeq \text{id}_{\Sigma_{g,\,n}}$ it follows from functoriality properties that $j_*\colon\pi_1(T)\longrightarrow\pi_1(\Sigma_{g,n})$ and $r_*\colon\pi_1(\Sigma_{g,n})\longrightarrow\pi_1(T)$ are inverses of each other and hence $\pi_1(T)\cong\pi_1(\Sigma_{g,n})\cong F_{2g+n-1}$, which is the free group with $2g+n-1$ generators.  

\smallskip

\noindent Consider the map $r\circ f\colon S\longrightarrow T$. We invoke the following technical result from \cite[Lemma 3.2]{JaW}.

\begin{lem}\label{lem:jlem}
Let $S$ be a closed surface. Suppose $g$ is a map from $S$ into $T$, a wedge at $t_o$ of $k$ simple closed curves $T_1,\dots,T_k$ and $g(s_o) = t_o$. If $g_*$ is an epimorphism, then there is a PL map $h\colon S\to T$ so that:
\begin{itemize}
    \item[1.] $h\simeq g$ \textnormal{rel} $\{s_o\}$;
    \item[2.] for each $i$, $1\le i\le k$, there is a point $t_i\in T_i-\{t_o\}$ so that $h^{-1}(t_i)$ is a single simple closed curve $\gamma_i$ in $S$ and $g$ is transverse with respect to $\{t_1,\dots,t_k\}$; and
    \item[3.] $S-\{\gamma_i\}_{1\le i\le k}$ is connected.
\end{itemize}
\end{lem}

\noindent Here \textit{transverse} refers to a technical condition which shall not be relevant for us,  see \cite[pag. 367]{JaW}. A straightforward consequence of this lemma is that $r\circ f$ maps a simple closed curve $\gamma$ to a contractible (in fact a constant) loop in $T$ and hence $\gamma \in \text{ker}(r_*\circ f_*)$. On the other hand, being $r_*$ an isomorphism, it follows that $\gamma\in \text{ker}(f_*)$. Since $f_*=\rho$ in this case, the desired result follows.

\smallskip

\noindent Let us finally assume $\Gamma$ has torsion elements, that is elliptic transformations. We can argue exactly as above but, in this case, $f_\ast = \pi_{\textnormal{orb}} \circ \rho$ and hence $f_*$ no longer coincides with $\rho$; recall that $\pi_{\textnormal{orb}}$ is the projection from $\Gamma$ to the fundamental group of the surface $\mathbb{H}/\Gamma$ that kills the torsion elements.  So from the existence of a simple closed curve $\gamma\in \text{ker}(f_*)$ we cannot immediately infer the existence of some simple closed curve in $\textnormal{ker}(\,\rho\,)$. 

\smallskip

\noindent We proceed as follows. The following argument is  an adaption of that used in Section \S\ref{sec:crysrep}; therefore we just summarise it in brief and highlight the main differences. Recall that, since $\Gamma$ has both torsion and parabolics elements, the quotient space $\Sigma=\mathbb H/\Gamma$ is a non-compact space and has cone-points as well as cusps or funnels. Even in this case, for a cone point, say $p$, we consider a closed disk $D$ centered at $p$ and let $f^{-1}(D)=U_1\,\cup\,\cdots\,\cup\,U_k$. If any one of the preimages $U_i$ of $D$ has positive genus then we can easily infer the existence of a simple closed essential curve in $\textnormal{ker}(\rho)$ as already argued in Section \S\ref{sec:crysrep}. Let $\{p_1,\dots,p_\ell\}$ be the cone points of $\Sigma$ and let us assume that for every $p_i$ and closed disk $D_i$ around $p_i$, all the connected components of $f^{-1}(D_i)$ have genus $0$. Let $D_{\infty}$ be an closed set containing all the punctures corresponding to cusps and funnels whose interior is homeomorphic to an $n$-punctured disk for some $n\ge1$. Once again we apply Lemma \ref{lem:makeall} to all the boundary components of $f^{-1}(D_i)$ for $i=1,\dots,\ell,\infty$. Thus we can also assume that the restriction of $f$ to any connected component of 
\begin{equation}
    S\setminus \left(\,\bigcup_{i=1}^\ell f^{-1}(\partial D_i) \cup f^{-1}(\partial D_\infty)\,\right)
\end{equation}
\noindent is an allowable map. Next we apply Edmonds' Theorem \ref{thm:edmbond} to each connected component of $f^{-1}(D_i)$ as well as to every connected component of the cut surface 
\begin{equation}
    S'=S\setminus \left(\,\bigcup_{i=1}^\ell f^{-1}(D_i^\circ)\cup f^{-1}(D^\circ_\infty) \,\right).
\end{equation}
\noindent Note that these cut surfaces differ from those defined in \eqref{eq:cut1} and \eqref{eq:cut2} since we also remove the pre-images of $D_\infty$. In the case the restriction of $f$ to any connected component, say $U$, of $S'$ admits a pinch then we are done. Even in this case, since $f(U)$ avoids every disk, we can find a simple closed curve, say $\gamma\in \textnormal{ker}(\rho)$ -- recall that, in order to find an essential curve $\gamma\in\textnormal{ker}(\rho)$ it is sufficient to show that $f(\gamma)$ does not enclose any cone point. As before, if the restriction of $f$ does not have a pinching map as a factor for any connected component of $S'$, then we can assume that the restriction of $f$ to $S\setminus f^{-1}(D_\infty)$ is a branched covering onto its image. Let $\gamma$ be a  simple closed curve provided by Jaco's Lemma above, see Lemma \ref{lem:jlem}. Note that we can choose the neighborhood $D_\infty$ above so that the contractible loop $f(\gamma)$ is not contained in $D_\infty$. Moreover, by changing $\gamma$ by a homotopy, still keeping it a simple curve,  we can ensure that $f(\gamma)$ bounds an embedded disk $D$. We sketch this modification briefly: away from the branch-points of $f$, the homotopy is simply the lift of the homotopy of the contractible arc $f(\gamma)$ to an embedded curve, which exists by an innermost-disk argument. If such an innermost-disk, say $E$, contains a branch-point, the homotopy is chosen to collapse the boundary of $E$ to that branch-point, and the corresponding lift in the domain surface will be an arc that passes through a ramification point. If $E$ contains multiple branch-points, then we first modify the image curve by a homotopy of the boundary arc so that $E$ is replaced by a collection of disks, each containing a single branch-point, thus reducing to the case already handled.

\smallskip

\noindent Now let $D\subset \Sigma\setminus D_\infty$ be the disk bounded $f(\gamma)$. If $D$ does not contain any cone point in its interior then $\gamma\in\textnormal{ker}(\rho)$. If $D$ contains cone points, then we proceed as in Section \S\ref{sec:crysrep}: applying Theorem \ref{edm4.5} we find a simple closed curve $\gamma$ in $f^{-1}(D)$ so that its image $f(\gamma)$ is null-homotopic in $D\setminus\{\,\text{cone points in }\,D\}$. This concludes the proof of the non-compact case and indeed the proof of our \ref{thm:mainthm}.



\bibliographystyle{amsalpha}
\bibliography{slcpslr}

\end{document}