\section{Introduction}



% \chen{ios is hard to analyze, different data collection, different analysis method, show our contri in iOS, add more features}
% \chen{1. Do you want to try ubicomp? It seems very relevant to their topic. 2. May also consider the industry track. 3 We now build the relationship with Wechat, not sure if there are any on-device models on their mini-program, maybe some future collaboration. 4. Please refer to Xuanzhe Liu's writing style if submitted to SE/Ubicomp.}
Deep Learning (DL) models have become increasingly common in applications (apps) for smartphones in recent years~\cite{xu2019first}.
They play a vital role in the functionality features of some apps, like image classification~\cite{lu2007survey, wang2017residual, haralick1973textural}, face detection~\cite{hjelmaas2001face, rowley1998neural}, speech recognition~\cite{reddy1976speech, gaikwad2010review, povey2011kaldi}, etc.
DL models in apps are now employed in two ways: on-cloud models and on-device models~\cite{googlecloud}.
On-cloud models are stored on the remote server, triggered by instructions and data from apps via the Internet, and return results to local apps.
However, this method is heavily sensitive to the Internet's condition and may expose user privacy~\cite{el2014literature}.
Consequently, on-device models, which are put locally on the smartphone, have become a feasible option for an increasing number of apps.
% Due to the benefits of integrating DL models into smartphone apps,
% many DL frameworks have been released to support on-device models, including TensorFlow Lite (TF Lite) from Google~\cite{tflite}, Core ML from Apple~\cite{coreml}, Caffe2 from Meta~\cite{caffe2}, etc. 
With this trend, industry and academia are starting to focus on how DL models are being used in smartphone apps, including framework distribution, model attributes, potential security issues, etc~\cite{huang2021robustness, huang2022smart, madry2017towards, ali2017same,sang2023beyond}.

Researchers have systematically studied how real-world Android apps use on-device models~\cite{huang2021robustness, huang2022smart, xu2019first}.
% The adversarial attack is one of the greatest threats to the security of DL models that pose a significant risk~\cite{madry2017towards}.
The Android platform is an open-source ecosystem~\cite{android}.
The on-device model in \emph{.tflite} format, which is most used in Android apps, is easily accessible to third parties for the structure and the trained weights of the models~\cite{tfLite, xu2019first}.
Due to the transparency of the Android platform and the on-device models inside, on-device models on Android smartphones have been proven to be extremely vulnerable, and adversarial examples generated by common adversarial attacks can easily fool on-device models~\cite{huang2021robustness, huang2022smart}.
Attacking the models inside those apps will be disastrous for users because many mobile apps with on-device models are used for critical tasks like user authentication, medical monitoring, and driving assistance~\cite{ren2020adversarial, zhang2021deep,wang2023energy}.
Android and iOS are the two most-used platforms for smartphones~\cite{phoneSystem}.
Due to the similarity of the functionalities of the same app on the Android and iOS platforms, the same vulnerabilities may exist on both platforms~\cite{gronli2014mobile, garg2021comparative}.
However, no research has yet been done to study the state of on-device models in iOS apps.

According to our observations, there are two reasons that prevent the study of the on-device models in iOS apps.
First, Apple does not actively distribute the source files of iOS apps with third parties, and the iOS ecosystem is closed-source~\cite{appleStore}. 
The data collection and acquisition process for the iOS app is much more challenging than the Android APK files.
Second, the most used iOS-specific DL framework Core ML~\cite{coreML}, provided by Apple~\cite{apple}, is not open source either, making it challenging to study the on-device model of this framework~\cite{CoreMLConvert}.
Core ML on-device models are gray-box models that do not share trained weights with third parties and provide extra model security protections for trained weights to prevent model hacking~\cite{CoreMLConvert}.
The most effective adversarial attacks against on-device models are all white-box attacks~\cite{ren2020adversarial, kurakin2018adversarial}, which require knowing the model's structure and weights prior to the attack.
Traditional adversarial attack techniques~\cite{goodfellow2014explaining, carlini2017towards, croce2020minimally, he2019towards} perform poorly when attacking Core ML models.
These factors lead many followers to accept that iOS on-device models are by nature more 'secure' than models on Android.
Therefore, there is a lack of systematic study on characteristics and potential security flaws in on-device models on iOS.


In this paper, we present the first systematic empirical study on how real-world iOS apps exploit on-device DL models and its potential security issues.
We seek to answer three research questions: (1) what are the characteristics of on-device models on iOS; 
(2) Why developers use different models for one app on iOS and Android; 
(3) How robust are on-device models on iOS against adversarial attacks.



To answer these three questions, we first propose a pipeline for iOS app source file collection.
We collect and make public the first dataset of iOS app source files\footnote{\href{https://github.com/huhanGitHub/iOS-App-database}{{iOS-App-database}}}, which contains 2907 iOS apps from Apple App Store~\cite{appleStore} and a list of apps containing on-device models.
Following related on-device model detection approaches~\cite{xu2019first, huang2021robustness, huang2022smart}, we identify 334 iOS apps with 1883 on-device models among all the collected iOS apps.

To understand the current use of on-device models in iOS apps, we analyse the current characteristics of on-device models in 334 apps in RQ1, including their framework, model developers, model size, model functionality, and model structure.
We discover that the same app regularly adopts different DL frameworks and on-device models across Android and iOS.

To understand why developers choose different DL models for the same app on different platforms, we manually explore the iOS-Android app pairs with different on-device models in RQ2.
We summarise reasons for model changes, including platform compatibility, model usability, etc., to provide developers insights when developing mirror apps between Android and iOS.
After analysing the characteristics of the current on-device model of iOS and the similarities and differences with models on Android, we investigate the current security concerns of models on iOS.

In RQ3, we first propose a more general and effective approach to attack white-box on-device models. 
Then, based on the results of RQ1 and RQ2, we propose a new method for performing adversarial attacks against gray-box Core ML models.
We evaluate the robustness of white-box models and Core ML models against adversarial attacks in our approaches.
The experimental results reveal the effectiveness of our approaches compared with the recently proposed baseline ModelAttacker~\cite{huang2021robustness}. 
All selected 10 gray-box on-device models of Core ML are successfully attacked by our approach with an average success rate of 75\%.
% Although the iOS platform is often considered safer than the Android platform, our experimental findings reveal that, like the on-device model on Android, the on-device models on iOS are still very vulnerable and face significant security threats.
Finally, we successfully use the on-device model vulnerability to fool real-world iOS apps.

In summary, our contributions are as follows:
\begin{itemize}
    \item This is the first systematic study of on-device models on iOS. 
    We collected and analysed the characteristics of the on-device models in iOS apps.
    Our research can help developers and researchers understand the current state of the on-device models in apps.
    
    \item We design and implement pipelines for analyzing on-device models on iOS and comparing on-device models between iOS and Android. Our findings show differences and motivations for selecting DL models for IOS and Android apps.
    
    \item We present a new adversarial attack approach for iOS gray-box on-device models. This is the first study on how to attack on-device models of Core ML. 
    Our method achieves an average success rate of 75\% when applying adversarial attacks on gray-box on-device models of Core ML.
    
\end{itemize}



% With the efforts of DL frameworks and the increasing computing capability of smartphones, 
% the performance of on-device models has increased greatly, but the model's security has not received sufficient consideration.



