
\section{approach overview}
We design and implement a workflow to explore our research goals.
% The pipeline can (1) crawl iOS apps from Apple App Store~\cite{appleStore} and identify all on-device models in crawled iOS apps;
% (2) match iOS apps' Android counterparts in Google Play;
% (3) evaluate the robustness of on-device models against adversarial attacks on iOS.
As shown in Figure~\ref{fig:workflow}, the first step of the workflow is to crawl iOS apps from Apple App Store.
This is achieved by simulating real people downloading iOS apps on iPhone emulators using the IPA tool~\cite{ipaTool}.
% These apps' entire relevant metadata, including the app category, ratings, reviews, and developers, is likewise scraped from the Apple App Store.
Following the guideline for iOS reverse engineering~\cite{owasp}, we recompile and extract non-code resources from all crawled iOS IPA files to detect on-device models.
We study these apps and on-device models in RQ1.
Second, we match identified iOS apps with their Android mirror apps on Google Play~\cite{googleplay}.
We study how and why developers choose different on-device models across Android and iOS.
Third, we propose a new approach for employing adversarial attacks to gray-box Core ML models on iOS.
We employ our proposed adversarial attack approach to evaluate the robustness of gray-box on-device models on iOS in RQ3.
Finally, we select attacked models to identify the point in apps where models are invoked and then manually input the adversarial examples to validate our method's effectiveness in directly attacking real-world iOS apps.

We will discuss the detail of crawling apps, detecting on-device models, paring Android counterparts, and attacking models in the following sections.

% Figure environment removed