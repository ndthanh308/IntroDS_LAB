%%
%% This is file `sample-sigconf.tex',
%% generated with the docstrip utility.
%%
%% The original source files were:
%%
%% samples.dtx  (with options: `sigconf')
%% 
%% IMPORTANT NOTICE:
%% 
%% For the copyright see the source file.
%% 
%% Any modified versions of this file must be renamed
%% with new filenames distinct from sample-sigconf.tex.
%% 
%% For distribution of the original source see the terms
%% for copying and modification in the file samples.dtx.
%% 
%% This generated file may be distributed as long as the
%% original source files, as listed above, are part of the
%% same distribution. (The sources need not necessarily be
%% in the same archive or directory.)
%%
 %% Commands for TeXCount
%TC:macro \cite [option:text,text]
%TC:macro \citep [option:text,text]
%TC:macro \citet [option:text,text]
%TC:envir table 0 1
%TC:envir table* 0 1
%TC:envir tabular [ignore] word
%TC:envir displaymath 0 word
%TC:envir math 0 word
%TC:envir comment 0 0
%%
%%
%% The first command in your LaTeX source must be the \documentclass command.
\documentclass[acmsmall]{acmart}
%% NOTE that a single column version is required for 
%% submission and peer review. This can be done by changing
%% the \doucmentclass[...]{acmart} in this template to 
%% \documentclass[manuscript,screen]{acmart}
%% 
%% To ensure 100% compatibility, please check the white list of
%% approved LaTeX packages to be used with the Master Article Template at
%% https://www.acm.org/publications/taps/whitelist-of-latex-packages 
%% before creating your document. The white list page provides 
%% information on how to submit additional LaTeX packages for 
%% review and adoption.
%% Fonts used in the template cannot be substituted; margin 
%% adjustments are not allowed.

\usepackage{tcolorbox}
\usepackage{lineno}
\tcbuselibrary{skins, breakable, theorems}
\newtcolorbox[auto counter]{summary}[1][]{
        title={\bfseries Summary},enhanced,
	coltitle=black,
	top=0.17in,
	attach boxed title to top left=
	{xshift=1.5em,yshift=-\tcboxedtitleheight/2},
        boxed title style={size=small,colback=lightgray},#1}

%%
%% \BibTeX command to typeset BibTeX logo in the docs
\AtBeginDocument{%
  \providecommand\BibTeX{{%
    \normalfont B\kern-0.5em{\scshape i\kern-0.25em b}\kern-0.8em\TeX}}}

%% Rights management information.  This information is sent to you
%% when you complete the rights form.  These commands have SAMPLE
%% values in them; it is your responsibility as an author to replace
%% the commands and values with those provided to you when you
%% complete the rights form.
\setcopyright{acmcopyright}
\copyrightyear{2023}
\acmYear{2023}
\acmDOI{10.1145/1122445.1122456}

%% These commands are for a PROCEEDINGS abstract or paper.
\acmConference[Conference acronym 'XX]{Make sure to enter the correct
  conference title from your rights confirmation emai}{June 03--05,
  2018}{Woodstock, NY}
%
%  Uncomment \acmBooktitle if th title of the proceedings is different
%  from ``Proceedings of ...''!
%
%\acmBooktitle{Woodstock '18: ACM Symposium on Neural Gaze Detection,
%  June 03--05, 2018, Woodstock, NY} 
\acmPrice{15.00}
\acmISBN{978-1-4503-XXXX-X/18/06}

%%
%% These commands are for a JOURNAL article.
\acmJournal{TOSEM}
\acmVolume{X}
\acmNumber{Y}
\acmArticle{Z}
\acmMonth{2}

%%
%% Submission ID.
%% Use this when submitting an article to a sponsored event. You'll
%% receive a unique submission ID from the organizers
%% of the event, and this ID should be used as the parameter to this command.
%%\acmSubmissionID{123-A56-BU3}

%%
%% For managing citations, it is recommended to use bibliography
%% files in BibTeX format.
%%
%% You can then either use BibTeX with the ACM-Reference-Format style,
%% or BibLaTeX with the acmnumeric or acmauthoryear sytles, that include
%% support for advanced citation of software artefact from the
%% biblatex-software package, also separately available on CTAN.
%%
%% Look at the sample-*-biblatex.tex files for templates showcasing
%% the biblatex styles.
%%

%%
%% The majority of ACM publications use numbered citations and
%% references.  The command \citestyle{authoryear} switches to the
%% "author year" style.
%%
%% If you are preparing content for an event
%% sponsored by ACM SIGGRAPH, you must use the "author year" style of
%% citations and references.
%% Uncommenting
%% the next command will enable that style.
%%\citestyle{acmauthoryear}

%%
%% end of the preamble, start of the body of the document source.



\newcommand{\old}[1]{\textcolor{red}{{#1}}}
\newcommand{\todo}[1]{\textcolor{blue}{{#1}}}
\newcommand{\todored}[1]{\textcolor{red}{{#1}}}
\newcommand{\todomagenta}[1]{\textcolor{magenta}{{#1}}}

% Commands & definitions

\def\Gammaqq{\ensuremath{\Gamma_{q\bar{q}}}\xspace}
\def\Gammahad{\ensuremath{\Gamma_{hadrons}}\xspace}

\def\Kgamma{\ensuremath{E_{\gamma}}\xspace}
\def\Kcut{\ensuremath{E_{\gamma}^{cut}}\xspace}
\def\acolcut{\ensuremath{\sin{(\Psi_{acol})}^{cut}}\xspace}

\def\Kreco{\ensuremath{K_{reco}}\xspace}
\def\kaonness{\ensuremath{\Delta_{\dEdx-K}}\xspace}
\def\pionness{\ensuremath{\Delta_{\dEdx-\pi}}\xspace}
\def\protonness{\ensuremath{\Delta_{\dEdx-p}}\xspace}
\def\epsilonhad{\ensuremath{\epsilon_{had}}\xspace}
\def\epsilonb{\ensuremath{\epsilon_{b}}\xspace}
\def\epsilonc{\ensuremath{\epsilon_{c}}\xspace}
\def\epsilonuds{\ensuremath{\epsilon_{uds}}\xspace}
\def\epsilonb2{\ensuremath{\epsilon^{2}_{b}}\xspace}
\def\epsilonc2{\ensuremath{\epsilon^{2}_{c}}\xspace}
\def\epsilonuds2{\ensuremath{\epsilon^{2}_{uds}}\xspace}

\def\costheta{\ensuremath{\cos \theta}\xspace}
\def\costhetab{\ensuremath{\cos \theta_{b}}\xspace}
\def\costhetaq{\ensuremath{\cos \theta_{q}}\xspace}
\def\sinthetaq{\ensuremath{\sin \theta_{q}}\xspace}
\def\costhetasq{\ensuremath{\cos^2 \theta_{b}}\xspace}
\def\sinthetasq{\ensuremath{\sin^2 \theta_{b}}\xspace}
\def\costhetaj{\ensuremath{\cos \theta_{j}}\xspace}
\def\costhetac{\ensuremath{\cos \theta_{c}}\xspace}

\def\fb{fb\ensuremath{^{-1}}\xspace}
\def\mum{\textmu\ensuremath{m}\xspace}
%\def\eL{\ensuremath{e_L^{\mbox{\scriptsize -}}}\xspace}
%\def\eR{\ensuremath{e_R^{\mbox{\scriptsize -}}}\xspace}
%\def\pL{\ensuremath{e_L^{\mbox{\scriptsize +}}}\xspace}
%\def\pR{\ensuremath{e_R^{\mbox{\scriptsize +}}}\xspace}

\def\eL{\ensuremath{e_L^{-}}\xspace}
\def\eR{\ensuremath{e_R^{-}}\xspace}
\def\pL{\ensuremath{e_L^{+}}\xspace}
\def\pR{\ensuremath{e_R^{+}}\xspace}

\def\b{\ensuremath{b}\xspace}
\def\bbar{\ensuremath{\overline{b}}\xspace}
\def\bbbar{\ensuremath{b}\ensuremath{\overline{b}}\xspace}
\def\qqbar{\ensuremath{q}\ensuremath{\overline{q}}\xspace}
\def\ccbar{\ensuremath{c}\ensuremath{\overline{c}}\xspace}
\def\ttbar{\ensuremath{t}\ensuremath{\overline{t}}\xspace}
\def\eebbbar{\ensuremath{e^{-}e^{+}\rightarrow b\bar{b}}\xspace}%\ensuremath{e^{\mbox{\scriptsize +}}}\ensuremath{\rightarrow}\ensuremath{b}\ensuremath{\overline{b}}\xspace}
\def\eeccbar{\ensuremath{e^{-}e^{+}\rightarrow c\bar{c}}\xspace}
\def\eeqqbar{\ensuremath{e^{-}e^{+}\rightarrow q\bar{q}}\xspace}
\def\ee{\ensuremath{e^{-}e^{+}}\xspace}
\def\eeZ{\ensuremath{e^{-}e^{+}\rightarrow Z}\xspace}
\def\eeZqqbar{\ensuremath{e^{-}e^{+}\rightarrow Z\rightarrow q\bar{q}}\xspace}
\def\eeZgammaqqbar{\ensuremath{e^{-}e^{+}\rightarrow Z \gamma \rightarrow q\bar{q} \gamma}\xspace}

\def\eebb{\ensuremath{e^{-}e^{+}\rightarrow b\bar{b}}\xspace}%\ensuremath{e^{\mbox{\scriptsize +}}}\ensuremath{\rightarrow}\ensuremath{b}\ensuremath{\overline{b}}\xspace}
\def\eecc{\ensuremath{e^{-}e^{+}\rightarrow c\bar{c}}\xspace}
\def\eeqq{\ensuremath{e^{-}e^{+}\rightarrow q\bar{q}}\xspace}
\def\eeZqq{\ensuremath{e^{-}e^{+}\rightarrow Z\rightarrow q\bar{q}}\xspace}
\def\eeZgammaqq{\ensuremath{e^{-}e^{+}\rightarrow Z \gamma \rightarrow q\bar{q} \gamma}\xspace}

%\def\eecc{\ensuremath{e^{\mbox{\scriptsize -}}}\ensuremath{e^{\mbox{\scriptsize +}}}\ensuremath{\rightarrow}\ensuremath{c}\ensuremath{\overline{c}}\xspace}
%\def\eeqq{\ensuremath{e^{\mbox{\scriptsize -}}}\ensuremath{e^{\mbox{\scriptsize +}}}\ensuremath{\rightarrow}\ensuremath{q}\ensuremath{\overline{q}}\xspace}
%\def\ee{\ensuremath{e^{\mbox{\scriptsize -}}}\ensuremath{e^{\mbox{\scriptsize +}}}\xspace}
%\def\eeZ{\ensuremath{e^{\mbox{\scriptsize -}}}\ensuremath{e^{\mbox{\scriptsize +}}}\ensuremath{\rightarrow}\ensuremath{Z}\xspace}
%\def\eeZqq{\ensuremath{e^{\mbox{\scriptsize -}}}\ensuremath{e^{\mbox{\scriptsize +}}}\ensuremath{\rightarrow}\ensuremath{Z}\ensuremath{\rightarrow}\ensuremath{q}\ensuremath{\overline{q}}\xspace}
%\def\eLpRbb{\ensuremath{e_L^{\mbox{\scriptsize -}}}\ensuremath{e_R^{\mbox{\scriptsize +}}}\ensuremath{\rightarrow}\ensuremath{b}\ensuremath{\overline{b}}\xspace}
%\def\eRpLbb{\ensuremath{e_R^{\mbox{\scriptsize -}}}\ensuremath{e_L^{\mbox{\scriptsize +}}}\ensuremath{\rightarrow}\ensuremath{b}\ensuremath{\overline{b}}\xspace}
%\def\eLpRcc{\ensuremath{e_L^{\mbox{\scriptsize -}}}\ensuremath{e_R^{\mbox{\scriptsize +}}}\ensuremath{\rightarrow}\ensuremath{c}\ensuremath{\overline{c}}\xspace}
%\def\eRpLcc{\ensuremath{e_R^{\mbox{\scriptsize -}}}\ensuremath{e_L^{\mbox{\scriptsize +}}}\ensuremath{\rightarrow}\ensuremath{c}\ensuremath{\overline{c}}\xspace}
%\def\eLpRqq{\ensuremath{e_L^{\mbox{\scriptsize -}}}\ensuremath{e_R^{\mbox{\scriptsize +}}}\ensuremath{\rightarrow}\ensuremath{q}\ensuremath{\overline{q}}\xspace}
%\def\eRpLqq{\ensuremath{e_R^{\mbox{\scriptsize -}}}\ensuremath{e_L^{\mbox{\scriptsize +}}}\ensuremath{\rightarrow}\ensuremath{q}\ensuremath{\overline{q}}\xspace}
\def\cme{\ensuremath{c.m.e.}\xspace}
\def\eLpR{\ensuremath{e_L^{-}e_R^{+}}\xspace}
\def\eRpL{\ensuremath{e_R^{-}e_L^{+}}\xspace}
\def\eLpRqq{\ensuremath{e_L^{-}e_R^{+}\rightarrow q\bar{q}}\xspace}
\def\eRpLqq{\ensuremath{e_R^{-}e_L^{+}}\rightarrow q\bar{q}\xspace}


%\def\eLpR{\ensuremath{e_L}\ensuremath{p_R}\xspace}
%\def\eRpL{\ensuremath{e_R}\ensuremath{p_L}\xspace}
\def\dEdx{\ensuremath{dE/\/dx}\xspace}
\def\dNdx{\ensuremath{dN/\/dx}\xspace}
\def\Afbb{\ensuremath{A^{b\bar{b}}_{FB}}\xspace}
\def\AFBb{\ensuremath{A^{b\bar{b}}_{FB}}\xspace}
\def\Afb{\ensuremath{A_{FB}}\xspace}
\def\AFB{\ensuremath{A_{FB}}\xspace}

\def\ALR{\ensuremath{A_{LR}}\xspace}
\def\Rb{\ensuremath{R_{b}}\xspace}
\def\Rc{\ensuremath{R_{c}}\xspace}
\def\Rq{\ensuremath{R_{q}}\xspace}
\def\Rqp{\ensuremath{R_{q\prime}}\xspace}
\def\Ruds{\ensuremath{R_{uds}}\xspace}
\def\Rbcostheta{\ensuremath{R_{b}(|cos\theta_{b}|)}\xspace}
\def\Rccostheta{\ensuremath{R_{c}(|cos\theta_{c}|)}\xspace}
\def\Rqcostheta{\ensuremath{R_{q}(|cos\theta_{q}|)}\xspace}
\def\Rudscostheta{\ensuremath{R_{uds}(|cos\theta_{uds}|)}\xspace}
\def\Afbc{\ensuremath{A^{c\bar{c}}_{FB}}\xspace}
\def\Afbq{\ensuremath{A^{q\bar{q}}_{FB}}\xspace}
\def\AFBc{\ensuremath{A^{c\bar{c}}_{FB}}\xspace}
\def\AFBq{\ensuremath{A^{q\bar{q}}_{FB}}\xspace}
\def\eett{\ensuremath{e^{\mbox{\scriptsize +}}}\ensuremath{e^{\mbox{\scriptsize -}}}\ensuremath{\rightarrow}\ensuremath{t}\ensuremath{\overline{t}}\xspace}
\def\bquark{\ensuremath{b}-quark\xspace}
\def\bjet{\ensuremath{b}-jet\xspace}
\def\bjets{\ensuremath{b}-jets\xspace}
\def\btagging{\ensuremath{b}-tagging\xspace}
\def\btag{\ensuremath{b_{tag}}\xspace}
\def\cquark{\ensuremath{c}-quark\xspace}
\def\cjet{\ensuremath{c}-jet\xspace}
\def\cjets{\ensuremath{c}-jets\xspace}
\def\ctagging{\ensuremath{c}-tagging\xspace}
\def\ctag{\ensuremath{c_{tag}}\xspace}
\def\udsjet{\ensuremath{uds}-jet\xspace}
\def\udsjets{\ensuremath{uds}-jets\xspace}
\def\tquark{\ensuremath{t}-quark\xspace}
\def\Zboson{\ensuremath{Z}-boson\xspace}
\def\Zpole{\ensuremath{Z}-pole\xspace}
\def\Zprime{\ensuremath{Z^{\prime}}\xspace}
\def\Zbb{\ensuremath{Z_{b\bar{b}}}\xspace}
\def\ZbRbR{\ensuremath{Z_{b_{R}\bar{b}_R}}\xspace}
\def\ZbLbL{\ensuremath{Z_{b_{L}\bar{b}_L}}\xspace}
\def\Bc{\ensuremath{Vtx}-method\xspace}
\def\Kc{\ensuremath{K}-method\xspace}
\def\BcKc{\ensuremath{Vtx/K}-method\xspace}
\def\BcKcsame{\ensuremath{Vtx/K_{same\,jet}}-method\xspace}
\def\BcBc{\ensuremath{Vtx/Vtx}-method\xspace}
\def\KcKc{\ensuremath{K/K}-method\xspace}

\def\Pb{\ensuremath{P_{chg.}}\xspace}
\def\Qb{\ensuremath{Q_{chg.}}\xspace}
%\def\PbB{\ensuremath{P_{chg.,Vtx}}\xspace}
%\def\PbK{\ensuremath{P_{chg.,K}}\xspace}
%\def\QbB{\ensuremath{q_{chg.,Vtx}}\xspace}
%\def\QbK{\ensuremath{q_{chg.,K}}\xspace}

\def\PbB{\ensuremath{P_{chg.,M_{1}}}\xspace}
\def\PbK{\ensuremath{P_{chg.,M_{2}}}\xspace}
\def\QbB{\ensuremath{Q_{chg.,M_{1}}}\xspace}
\def\QbK{\ensuremath{Q_{chg.,M_{2}}}\xspace}

\def\Pbi{\ensuremath{P_{chg.,i}}\xspace}
\def\Pbj{\ensuremath{P_{chg.,j}}\xspace}
\def\Qbi{\ensuremath{Q_{chg.,i}}\xspace}
\def\Qbj{\ensuremath{Q_{chg.,j}}\xspace}


\def\pmp{\ensuremath{+-}\xspace}
\def\mpp{\ensuremath{-+}\xspace}
\def\pp{\ensuremath{++}\xspace}
\def\mm{\ensuremath{--}\xspace}

\newcommand\blankpage{%
    \null
    \thispagestyle{empty}%
    \addtocounter{page}{-1}%
    \newpage}

\begin{document}

%%
%% The "title" command has an optional parameter,
%% allowing the author to define a "short title" to be used in page headers.
\title{A First Look at On-device Models in iOS Apps}

%%
%% The "author" command and its associated commands are used to define
%% the authors and their affiliations.
%% Of note is the shared affiliation of the first two authors, and the
%% "authornote" and "authornotemark" commands
%% used to denote shared contribution to the research.
\author{Han Hu}
\email{han.hu@monash.edu}
\affiliation{%
  \institution{Monash University}
  \streetaddress{Wellington Road}
  \city{Clayton}
  \state{Victoria}
  \country{Australia}
  \postcode{3800}
}
\author{Yujin Huang}
\email{yujin.huang@monash.edu}
\affiliation{%
  \institution{Monash University}
  \streetaddress{Wellington Road}
  \city{Clayton}
  \state{Victoria}
  \country{Australia}
  \postcode{3800}
}

\author{Qiuyuan Chen}
\email{joeqychen@tencent.com}
\affiliation{%
 \institution{Tencent}
 \streetaddress{Nanshan}
 \city{Shenzhen}
 \state{Guangdong}
 \country{China}}

\author{Terry Yue zhuo}
\email{terry.zhuo@monash.edu}
\affiliation{%
  \institution{Monash University}
  \streetaddress{Wellington Road}
  \city{Clayton}
  \state{Victoria}
  \country{Australia}
  \postcode{3800}
}
  \author{Chunyang Chen}
\email{chunyang.chen@monash.edu}
\authornote{Corresponding author}
\affiliation{%
  \institution{Monash University}
  \streetaddress{Wellington Road}
  \city{Clayton}
  \state{Victoria}
  \country{Australia}
  \postcode{3800}
}
  


%%
%% By default, the full list of authors will be used in the page
%% headers. Often, this list is too long, and will overlap
%% other information printed in the page headers. This command allows
%% the author to define a more concise list
%% of authors' names for this purpose.
% \renewcommand{\shortauthors}{Trovato and Tobin, et al.}
\newcommand{\chen}[1]{\textcolor{red}{#1}}

%%
%% The abstract is a short summary of the work to be presented in the
%% article.
\begin{abstract}
 Powered by the rising popularity of deep learning techniques on smartphones, on-device deep learning models are being used in vital fields like finance, social media, and driving assistance.
 Because of the transparency of the Android platform and the on-device models inside, on-device models on Android smartphones have been proven to be extremely vulnerable.
 However, due to the challenge in accessing and analysing iOS app files, despite iOS being a mobile platform as popular as Android, there are no relevant works on on-device models in iOS apps.
 Since the functionalities of the same app on Android and iOS platforms are similar, the same vulnerabilities may exist on both platforms.
 In this paper, we present the first empirical study about on-device models in iOS apps, including their adoption of deep learning frameworks, structure, functionality, and potential security issues. 
 We study why current developers use different on-device models for one app between iOS and Android.
 We propose a more general attack against white-box models that does not rely on pre-trained models and a new adversarial attack approach based on our findings to target iOS's gray-box on-device models.
 Our results show the effectiveness of our approaches.
 Finally, we successfully exploit the vulnerabilities of on-device models to attack real-world iOS apps.
\end{abstract}

%%
%% The code below is generated by the tool at http://dl.acm.org/ccs.cfm.
%% Please copy and paste the code instead of the example below.
%%
\begin{CCSXML}
<ccs2012>
   <concept>
       <concept_id>10002978.10003022</concept_id>
       <concept_desc>Security and privacy~Software and application security</concept_desc>
       <concept_significance>500</concept_significance>
       </concept>
   <concept>
       <concept_id>10011007</concept_id>
       <concept_desc>Software and its engineering</concept_desc>
       <concept_significance>500</concept_significance>
       </concept>
 </ccs2012>
\end{CCSXML}

\ccsdesc[500]{Security and privacy~Software and application security}
\ccsdesc[500]{Software and its engineering}

% \ccsdesc[500]{Computer systems organization~Embedded systems}
% \ccsdesc[300]{Computer systems organization~Redundancy}
% \ccsdesc{Computer systems organization~Robotics}
% \ccsdesc[100]{Networks~Network reliability}

%%
%% Keywords. The author(s) should pick words that accurately describe
%% the work being presented. Separate the keywords with commas.
\keywords{on-device models, iOS, adversarial attack, mobile, iPhone}


%%
%% This command processes the author and affiliation and title
%% information and builds the first part of the formatted document.
\maketitle

\section{Introduction}
Current quantum hardware is unable to carry out universal quantum computations due to the buildup of errors that occur during the computation. 
The magnitude of the individual error is currently above the value that the Threshold Theorem requires in order to kick-start quantum error correction and fault-tolerant quantum computation~\cite[Section 10.6]{nielsen_chuang_2010}. 
Although the experimentally achieved fidelity rates are promising and the error bounds are inching closer to the required threshold, we will have to work for the foreseeable future with quantum hardware with errors that build-up during the computation.  This implies that we can only do a limited number of steps before the output of the computation has become completely uncorrelated with the intended one.

For fault-tolerant quantum computing, we repeat four steps: 
1) We apply a number of single and two-qubit quantum gates, in parallel whenever possible; 
2) We perform a syndrome measurement on a subset of the qubits; 
3) We perform fast classical computations to determine which errors have occurred and how to correct them; 
and, 4) We apply correction terms based on the classical computations.
We then repeat these four steps with a next sequence of gates. 
These four steps are essential to fault-tolerant quantum computing. 


The starting point of this work is to use the four steps outlined above, not to carry out error correction and fault-tolerant computation, but to enhance short, constant-depth, {\em uncorrected} quantum circuits that perform single qubit gates and {\em nearest-neighbor} two qubit gates. 
Since in the long run we will have to implement error-correction and fault-tolerant computation anyhow, and this is done by such a four-step process, why not make other use of this architecture? Moreover, on some of the quantum hardware platforms, these operations are already in place.
Embracing this idea we naturally arrive at the question: what is the computational power of \textit{low-depth} quantum-classical circuits organized as in the four steps outlined above? 
We thus investigate circuits that execute a small, ideally constant, number of stages, where at each stage we may apply, in parallel, single qubit gates and {\em nearest-neighbor} two qubit gates, followed by measurements, followed by low-depth classical computations of which the outcome can control quantum gates in later stages. 
It is not clear, at first, whether such circuits, especially with constant depth, can do anything remotely useful. 
But we will see that this is indeed the case: many quantum computations can be done by such circuits in constant depth. 
By parallelizing quantum computations in this way, we improve the overall computational capabilities of these circuits, as we do not incur errors on qubits that are idle, simply because qubits are not idle for a very long time. 
Furthermore, reducing the depth of quantum circuits, at the cost of increasing width, allows the circuit to be run faster even if errors occur.

The first usage of such a four-step layout, not to do error correction, but to perform computations, can be found in the paradigm of measurement-based quantum computing~\cite{gottesman1999demonstrating,raussendorf2001one,jozsa2006introduction,clark2007generalised}: 
A universal form of quantum computing where a quantum state is prepared and operations are performed by measuring qubits in different bases, depending on previous measurements and intermediate measurements.

\citeauthor{PhamSvore2013} were the first to formalize the four-step protocol for performing computations~\cite{PhamSvore2013}. They included specific hardware topologies by considering two-dimensional graphs for imposing constraints on qubit interactions. In their model, they develop circuits for particularly useful multi-qubit gates, including specifying costs in the width, number of qubits, depth, number of concurrent time steps, size, and total number of non-Identity operations.
As a result, they find an algorithm that factors integers in polylogarithmic depth.
\citeauthor{Browne:2011} showed that the main tool in the work by \citeauthor{PhamSvore2013}, the fan-out gate, can also be replaced by additional log-depth classical computations in the measurement-based quantum computing setting~\cite{Browne:2011}.

More recently, \citeauthor{Cirac:2021} introduced a scheme to implement unitary operations involving quantum circuits combined with Local Operations and Classical Communication ($\mathsf{LOCC}$) channels: $\mathsf{LOCC}$-assisted quantum circuits~\cite{Cirac:2021}. Similarly to the four-step scheme we just described, they allow for a short depth circuit to be run on the qubits, followed by one round of $\mathsf{LOCC}$, in which ancilla qubits are measured and local unitaries are applied based on the measurement outcomes. They show that in this model any 1D transitionally invariant matrix-product state (MPS) with fixed bond dimension is in the same phase of matter as the trivial state. Similar ideas can be found in~\cite{TVV_NonAbelianTopologicalOrder_2022, tantivasadakarn2021long}.

In this work, we introduce a new model, called \textit{Local Alternating Quantum-Classical Computations} ($\LAQCC$). In this model we alternate between running quantum circuits (constrained by locality), ending in the measurement of a subset of qubits, and fast classical computations based on the measurement results. The outcome of the classical computations are then used to control future quantum circuits. We allow for flexibility in this model, by giving different constraints to the power of both the quantum circuits and the classical circuits as well as the number of alternations between them. 
Most attention will be given to $\LAQCC$ containing quantum circuits of constant depth, classical circuits of logarithmic depth and at most a constant number of alternations between them. 
Any circuit constructed in this model is considered to be of constant depth. 
We restrict ourselves to logarithmic depth classical computations, as this is the first natural and non-trivial extension beyond constant-depth classical computations. 
Constant-depth classical computations do however also have an equivalent constant-depth quantum implementation.

The definition of $\LAQCC$ sharpens the original definition of \citeauthor{PhamSvore2013} by adding constraints to the intermediate classical computations. This allows us to bound the power of $\LAQCC$ from above. 

The main result of \citeauthor{Cirac:2021}, that 1D translational invariant MPS with fixed bond dimension can be prepared by $\mathsf{LOCC}$-assisted circuits, relies on local symmetries of the MPS. These symmetries allow them to prepare local states (on a constant number of qubits) and glue them together by doing one round of the appropriate entangling measurement and corrections, after which they run a round of local unitaries to get the desired result. This general scheme for preparing states that exhibit an MPS description with the appropriate local symmetries requires only geometrically local unitaries and one round of measurement and corrections an therefore is accessible in $\LAQCC$. Studying different local symmetries, known as Symmetry Protected Topological (SPT) phases of matter, to find measurement-based constant depth circuits for states is a broad ongoing field of research~\cite{TVV_NonAbelianTopologicalOrder_2022, tantivasadakarn2021long, smith2023deterministic}. 
All these schemes have a $\LAQCC$ implementation.

%$\LAQCC$-circuits also exist for general schemes of preparing local states, based on the local tensors, and gluing them together using one round of entangled measurement and corrections, based on the local symmetry. 
%The main result of \citeauthor{Cirac:2021}, that 1D translational invariant MPS with fixed bond dimension can be prepared by $\mathsf{LOCC}$-assisted circuits, relies heavily on local symmetries of the MPS and as a result also has an equivalent $\LAQCC$ implementation. 
%The corrections applied after the measurement round are local unitaries depending on the local symmetries of the MPS. 

 

%This general scheme of preparing local states, based on the local tensors, and gluing it together by doing one round of entangled measurement and corrections, based on the local symmetry, is accessible in $\LAQCC$.
Note however that \citeauthor{Cirac:2021} also suggest a circuit for the $W$-state.
This circuit uses sequentially and dependent measurement-based corrections of the ancilla qubits. 
These dependent measurements translate to sequential alternations between the quantum and classical circuits and therefore increase the total depth to linear depth, exceeding the constant-depth constraints imposed by $\LAQCC$-circuits. 

We study the power of the $\LAQCC$ model with respect to state preparation, showing that even with only constant quantum-depth and logarithmic classical depth it remains possible to prepare states with long-range entanglement.
Another surprising result is that it is unlikely that $\LAQCC$ circuits are classically simulatable. We show that any instantaneous quantum polynomial-time (IQP) circuit~\cite{Bremner2010,Shepherd2009} has an $\LAQCC$ implementation.
Classical simulation of IQP circuits implies the collapse of the polynomial hierarchy to the third level, which is not believed to be true~\cite{Bremner2017}. Therefore, we expect that $\LAQCC$ circuits are unlikely to be classically simulatable. We bound the power of $\LAQCC$ by showing that it is contained in $\QNC^1$, the class of polynomial-size, log-depth circuits.

Next, we also study the power that intermediate classical calculations can add to quantum computations, by considering a new model that alternates between polynomially many polynomial-depth quantum circuits and unbounded classical computations
We study this model by doing a complexity theoretical analysis, where we draw inspiration from the notions of complexity given by \citeauthor{RosenthalYuen:2022}, \citeauthor{MetgerYuen:2023}, and \citeauthor{Aaronson:2004}.
All three complexity notions are based on the notion of state preparation, instead of more traditional definition of complexity such as the decidability of a computational problem. 
The first two consider classes based on sequences of quantum states preparable by a polynomial-sized quantum circuit, where the circuits are uniformly generated by a computational class, for instance, the class $\mathsf{PSPACE}$, which results in the complexity class $\mathsf{StatePSPACE}$~\cite{RosenthalYuen:2022,MetgerYuen:2023}.
The third notion considers a relative complexity, where the complexity is measured between two given states, and is measured by the number of gates, from a given gate-set, required to transform one state in another state~\cite{Aaronson:2004}. 
For our definition of state preparation complexity, we drop the uniformity constraint from~\cite{RosenthalYuen:2022,MetgerYuen:2023} and define a class as $\mathsf{StateX}$, which refers to states preparable by circuits of type $\mathsf{X}$. 
As an example, if $\mathsf{X} = \QNC^0$, this results in the class $\mathsf{StateQNC^0}$, which is the set of states preparable from the $\ket{0}^n$ state by poly-size constant-depth circuits. 
This notion is similar to the relative complexity from~\cite{Aaronson:2004}, where one state is the  $\ket{0}^n$ state and instead of counting the number of gates we consider the set of states preparable by a fixed number of gates. Using this notion of complexity we show that any state preparable by an $\LAQCC^*$ circuit is also preparable by a $\mathsf{PostQPoly}$ circuit, the class of circuits of polynomial depth with an additional post-selection gate. 

All Clifford circuits have a constant-depth $\LAQCC$ implementation, implying that any stabilizer state can be implemented by a constant-depth $\LAQCC$ circuit, see Section~\ref{sec:clifford_circuits} for a proof of this statement. 
Efficient circuits for stabilizer states have been known already through measurement-based quantum computing. Therefore this paper focuses on the preparation of non-stabilizer states, and as a surprising result we find novel constant-depth protocols for four very natural classes of non-stabilizer states.
Despite the extensive research into these four classes of non-stabilizer states and the many applications of them, no efficient constant- or low-depth state preparation protocols are known yet. We specifically consider these four classes as they are all often used as initial states in other algorithms.

The first state is a uniform superposition over an arbitrary number of states. 
This state finds applications in many quantum algorithms, as they often start with a uniform superposition over multiple states. 
This superposition is often achieved by applying Hadamard gates to every qubit due to its simplicity to prepare. 
Yet, the analysis of many algorithms, such as Shor's algorithm~\cite{Shor:1997}, would benefit from a different initial superposition. 
The circuit to prepare the uniform superposition over an arbitrary number of states uses an exact version of Grover search as a subroutine, that turns a probabilistic circuit, with a known constant probability of success, into a deterministic circuit. 
We use the circuit for preparing a uniform superposition over an arbitrary number of states as a subroutine in the next two quantum state preparation protocols. 

The second state is the $W$-state, the uniform superposition over all computational basis states of Hamming-weight~$1$, a natural long-ranged entangled state that displays a fundamentally nonequivalent type of entanglement from the Greenberger–Horne–Zeilinger state~\cite{WState:2000}, for which $\LAQCC$-type constant-depth circuits were previously known~\cite{PhamSvore2013, Cirac:2021}. 
The $W$-state is often used as benchmark for new quantum hardware~\cite{Haffner2005,Neeley2010,GarciaPerez:2021}. 
A novel way to prepare the $W$-state therefore gives a new way to benchmark different quantum devices with each other. 
A circuit for preparing the $W$-state was given in~\cite{Cirac:2021}, but this implementation requires sequentially alternating measurements followed by local unitaries, which in the $\LAQCC$ model is not considered to be of constant depth. 
We improve this protocol by giving an $\LAQCC$ implementation of the $W$-state, based on a compress-uncompress method that links the one-hot and binary encoding of integers.

The third state considered is the Dicke state, a generalization of the $W$-state, a superposition over all computational basis states with Hamming-weight $k$~\cite{Dicke:1954}. 
Dicke states have relevance in various practical settings.
For instance, for quantum game theory~\cite{zdemir2007}, quantum storage~\cite{Bacon_Compress:2006,Plesch:2010}, quantum error correction~\cite{ouyang2014permutation}, quantum metrology~\cite{toth2012multipartite}, and quantum networking~\cite{prevedel2009experimental}. 
Dicke states have been used as a starting state for variational optimization algorithms, most notably Quantum Alternating Operator Ansatz (QAOA)~\cite{Hadfield2019}, to find solutions to problems such as Maximum k-vertex Cover~\cite{Brandhofer2022,cook2020quantum}.
The ground states of physical Hamiltonians describing one-dimensional chains tend to show a resemblance to Dicke states such as states resulting from the Bethe ansatz, making them an ideal starting state when investigating the ground state behavior of these Hamiltonians~\cite{TDL_BetheAnsatzDerivation:2010,B_ExcitedStateQuantumPhaseTransitions:2013,DickeTransitions:2021}. 
For instance, the algorithm by \citeauthor{van2021preparing}, who give an algorithm to prepare the Bethe ansatz eigenstates of the spin-1/2 XXZ spin chain, starts by first preparing a Dicke state~\cite{van2021preparing}. 
A Dicke-state preparation protocol based on the compress-uncompress methodology used in the $W$-state furthermore finds applications in entanglement distillation, where the entanglement of a large state is concentrated on only a few qubits. 
Efficient deterministic circuits for preparing Dicke states have been proposed by \citeauthor{bartschi2019deterministic}~\cite{bartschi2019deterministic, bartschi2022deterministic_short_depth}. 
They provide a quantum circuit of depth $\mathO(k \log(\frac{n}{k}))$, allowing arbitrary connectivity, to prepare a Dicke state, which they conjecture to be optimal when $k$ is constant. 
In this work, we provide a constant-depth $\LAQCC$ circuit below their conjectured bound already for constant $k$. 
However, this does not directly disprove their conjecture, as we allow for intermediate measurements and classical computations. 
More significantly, we even construct constant-depth $\LAQCC$ circuits for $k = \mathO(\sqrt{n})$ greatly improving their bound.
This construction extends the compress-uncompress method for the $W$-state combined with additional subroutines. 

We continue with a log-depth state preparation protocol for the Dicke-state for arbitrary $k$. 
This protocol implements an efficient transformation between the factoradic number representation and the combinatorial number representation of a positive integer. 
The combinatorial number representation relates directly to the Dicke state. 
The provided efficient transformation between number representation systems might be of independent interest. 

We conclude by modifying our protocol for preparing a Dicke-state to a protocol that prepares quantum many-body scar states in constant-depth. 
These states have low entanglement and longer coherence times than states with similar energy density.
These characteristics make many-body scar states interesting to analyze and relevant within physics.
Many-body scar states appear for instance in the AKLT model~\cite{AKLT:1987,MRBAR:2018,MRB:2018} and different spin models~\cite{SI:2019,MOBFR:2020}.
Known methods for preparing these states have polynomial-depth~\cite{Gustafson:2023}, whereas our circuit has constant depth. 

% We conclude by studying the power that intermediate classical calculations can add to quantum computations. 
% In this study, we define a new model that relaxes constant-depth quantum circuits to polynomial depth quantum circuits, log-depth classical calculations to unbounded classical computations and a constant number of alternations to a polynomial number of alternations. 
% We call this model $\LAQCC^*$. 
% We study this model by doing a complexity theoretical analysis, where we draw inspiration from the notions of complexity given by \citeauthor{RosenthalYuen:2022}, \citeauthor{MetgerYuen:2023}, and \citeauthor{Aaronson:2004}.
% All three complexity notions are based on the notion of state preparation, instead of more traditional definition of complexity such as the decidability of a computational problem. 
% The first two consider classes based on sequences of quantum states preparable by a polynomial-sized quantum circuit, where the circuits are uniformly generated by a computational class, for instance, the class $\mathsf{PSPACE}$, which results in the complexity class $\mathsf{StatePSPACE}$~\cite{RosenthalYuen:2022,MetgerYuen:2023}.
% The third notion considers a relative complexity, where the complexity is measured between two given states, and is measured by the number of gates, from a given gate-set, required to transform one state in another state~\cite{Aaronson:2004}. 
% For our definition of state preparation complexity, we drop the uniformity constraint from~\cite{RosenthalYuen:2022,MetgerYuen:2023} and define a class as $\mathsf{StateX}$, which refers to states preparable by circuits of type $\mathsf{X}$. 
% As an example, if $\mathsf{X} = \QNC^0$, this results in the class $\mathsf{StateQNC^0}$, which is the set of states preparable from the $\ket{0}^n$ state by poly-size constant-depth circuits. 
% This notion is similar to the relative complexity from~\cite{Aaronson:2004}, where one state is the  $\ket{0}^n$ state and instead of counting the number of gates we consider the set of states preparable by a fixed number of gates. Using this notion of complexity we show that any state preparable by an $\LAQCC^*$ circuit is also preparable by a $\mathsf{PostQPoly}$ circuit, the class of circuits of polynomial depth with an additional post-selection gate. 

\paragraph{Summary of results}
\begin{itemize}
    \item We give a new definition of a computational model that captures the power of the four step process: applying a constant number of layers of one- and two-qubit gates; performing a syndrome measurement; perform a fast classical computation determining corrections; apply corrections. We call this model \emph{Local Alternating Quantum Classical Computations}, or $\LAQCC$ for short. In this model we bound the allowed quantum operations, intermediate classical calculations, and number of rounds separately. In Section~\ref{sec:LAQCC_model} we define this model and give a list of operations based on results from literature contained in this computational model. In some of these operations we explicitly use that we allow for multiple, but at most constant, rounds  of corrections.
    \item  We show show that there exist $\LAQCC$ circuits that can not be weakly simulated in Section~\ref{sec:IQP_in_LAQCC}. We further show that for every $\LAQCC$ circuit there exists a $\QNC^1$ circuit simulating it perfectly, in Section~\ref{sec:LAQCC_in_QNC1}.
    \item We introduce a new type computational complexity for preparing states and show that the extension of $\LAQCC$ where we allow a polynomial number of rounds and unbounded classical computation, is contained in $\mathsf{PostQPoly}$, the class of polynomial circuits with post-selection, in Section~\ref{sec:Complexity results}.
    \item We show a protocol to prepare the uniform superposition state of size $q$ in $\LAQCC$ using $\mathO(\ceil{\log_2(q)}^2)$ qubits in Section~\ref{sec:superposition_modulo_q}. 
    \item We show a protocol to prepare the $W_n$ state in $\LAQCC$ using $\mathO(n\log(n))$ qubits in Section~\ref{sec:W_state_in_LAQCC}.
    \item We show two ways of preparing the Dicke-$(n,k)$ state. The first method is in $\LAQCC$, works up to $k = \mathO(\sqrt{n})$, uses $\mathO(n^2\log(n))$ qubits, and is found in Section~\ref{sec:dicke:small_k}. The second method is in $\LAQCC\text{-}\mathsf{LOG}$ (an extension of $\LAQCC$ allowing for logarithmic number of alterations instead of constant), works for any $k$, uses $\mathO(\text{poly}(n))$ qubits, and is found in Section~\ref{sec:Dicke_in_LAQCC_LOG}. 
    \item We extend on our $\LAQCC$ method of generating Dicke-$(n,k)$ states for $k = \mathO(\sqrt{n})$ and show a protocol to generate many-body scar states for a particular Hamiltonian in $\LAQCC$ (Section~\ref{sec:many_body_scar}). 
\end{itemize}
Summarized in a table, we provide the following state generation protocols:
\begin{table}[htb]
\centering
\begin{tabular}{l|l|l|l}
\textbf{State description} & \textbf{Width} & \textbf{Depth} & \textbf{Implementation}\\
\hline 
Uniform superposition mod $q$: $\frac{1}{\sqrt{q}} \sum_{i = 0}^{q-1}\ket{i}$ & $\mathO(\ceil{\log^2 q})$ & $\mathO(1)$ & Section~\ref{sec:superposition_modulo_q}\\

$W$-state: $\frac{1}{\sqrt{n}}\sum_{i = 0}^{n-1}\ket{e_i}$ & $\mathO(n \log n)$ & $\mathO(1)$ & Section~\ref{sec:W_state_in_LAQCC}\\

Dicke-$(n,k)$, $k = \mathO(\sqrt{n})$: $\binom{n}{k}^{-1/2}\sum_{x \in \{0,1\}^n: |x| = k} \ket{x}$ &  $\mathO(n^2\log n)$ & $\mathO(1)$ 
&Section~\ref{sec:dicke:small_k}\\

Dicke-$(n,k)$: $\binom{n}{k}^{-1/2}\sum_{x \in \{0,1\}^n: |x| = k} \ket{x}$ & $\mathO(\text{poly}(n))$ & $\mathO(\log n)$ &Section~\ref{sec:Dicke_in_LAQCC_LOG}\\

QMBS: $\ket{S_k} = \frac{1}{k! \sqrt{\mathcal N(n,k)}}(Q^\dagger)^k \ket{\Omega}$ &  $\mathO(n^2\log n)$ & $\mathO(1)$  &  Section~\ref{sec:many_body_scar}
\end{tabular}
\caption{Summary of state preparation protocols given in this paper.}
\label{tab:sate_prep}
\end{table}
In the entry for the quantum many-body scar state $Q$ denotes the raising operator and $\mathcal N(n,k)=\binom{n-k-1}{k}$. 
Section~\ref{sec:many_body_scar} will provide more details on the variables and the implementation. 

\paragraph{Organization of the paper}
\noindent We first introduce relevant preliminaries in Section~\ref{sec:preliminaries}. 
In Section~\ref{sec:LAQCC_model} we formally define the class of Local Alternating Quantum-Classical Computations ($\LAQCC$). We also show that any Clifford circuit can be implemented in constant depth $\LAQCC$ (a result based on a result from measurement-based quantum computing~\cite{jozsa2006introduction}). 
This result allows us to give many useful multi-qubit gates and routines in Section~\ref{sec:gates_created_in_LAQCC}. 
Beyond that we show that constant depth $\LAQCC$ circuits are contained in $\QNC^1$ and that any $\mathsf{IQP}$ circuit has an $\LAQCC$ implementation.
We conclude this section with an analysis of a more powerful instantiation of $\LAQCC$ and show an inclusion with respect to the class $\mathsf{PostQPoly}$, which is the class of circuits of polynomial depth with one additional post-selection gate. 
In Section~\ref{sec:state_prep_in_LAQCC} we give $\LAQCC$ circuit implementations for preparing the uniform superposition over an arbitrary number of states, the $W$-state and the Dicke state up to $k = \mathO(\sqrt{n})$. We furthermore give a log-depth circuit implementation for preparing the Dicke state for any $k$. We conclude by showing a $\LAQCC$ circuit for generating many body scar states of a particular type of Hamiltonian.


% \vspacebeforesection
\section{Background}
\label{sec:background}

In this section, we provide the necessary background information to ensure a comprehensive understanding of the attack described in this paper. We start with a description of the Distributed Hash Table (DHT) used by IPFS, followed by its content resolution mechanisms. We also detail techniques for network size estimation, necessary for our attack detection and mitigation mechanisms.

\vspacebeforesection
\subsection{IPFS DHT}
\label{sec:kad_dht}

We review the features of the Kademlia DHT~\cite{maymounkov2002kademlia} and its \texttt{libp2p} implementation~\cite{libp2p_github} that are the most relevant to our attack.
To participate in the DHT, each peer generates a public/private key pair and derives an identity $\peerid \in \{0,1\}^{256}$ as the hash of its public key.
Ideally, each peer generates a random key pair and, therefore, peer IDs are distributed uniformly and independently over the space $\{0,1\}^{256}$.
While honest nodes follow this rule, malicious nodes may generate and choose from an arbitrary number of key pairs.
Each peer maintains a routing table consisting of $m=256$ buckets.
The $i$-th bucket contains the addresses of up to $k=20$ peers whose peer IDs share a common prefix of exactly $i$ bits with the peer's own peer ID. 

%
A new participant node joins the IPFS network by contacting one of the hardcoded bootstrap nodes. This bootstrap node provides the new node with some initial peers allowing it to join the DHT. The new node uses this information to perform a walk through the DHT towards its own peer ID.
The walk allows to: \textit{(i)}~make sure that there is no other node in the network with the same ID; \textit{(ii)}~discover new peers and fill the newcomer's DHT routing table. At the same time, the newcomer establishes \bitswap~\cite{de2021accelerating} connections to a subset of encountered peers (usually around 300 of them). The core role of the \bitswap protocol is to enable bilateral content transfer and to play the role of a cache for recently-accessed content.

The main DHT operation $\Call{GetClosestPeers}{\key}$ returns the $k=20$ closest peers to $\key$. 
%
In Kademlia, the distance between two keys $x$ and $y$ in the key space is given by $x \oplus y \in \{0,...,2^{256}-1\}$, where $\oplus$ denotes the bitwise XOR operation on the keys; the resulting binary string is interpreted as an integer.
%
When a client wants to find the peers with IDs closest to $\key$, it sends a request to the $\alpha=3$ peers in its routing table whose peer IDs are closest to $\key$. Each of these peers returns the $k$ closest peers to $\key$ in its own routing table and the addresses of these peers. 
%
The client again sends a request to the $\alpha$ peers closest to $\key$, among peers in its routing table and those whose addresses it just received. This process repeats until the client does not find any more peers closer to $\key$.
Due to network churn and imperfect routing tables, we observed in our experiments that successive calls to $\Call{GetClosestPeers}{\key}$ do not always return the same set of $k=20$ peers (we provide more details in \Cref{sec:evaluation}, \Cref{fig:20closest}). This is an important limitation affecting our attack.

\vspacebeforesection
\subsection{Content Resolution in IPFS}
\label{sec:ipfs}

IPFS is a content-centric network.
It allows its participant to request files without specifying their location. 
%
Content is indexed by content IDs $\cid \in \{0,1\}^{256}$ that are derived from a hash of that content.
Both peer IDs and CIDs are used as keys in the DHT.
Each node can play the role of a \provider, \downloader, or \resolver. 
The process of content advertisement and resolution is illustrated in \Cref{fig:add_get_provider}.

%
When a \provider wishes to publish content with a given $\cid$ on IPFS, it creates a \emph{provider record} that contains $cid$ and the \provider's address.
During a $\Call{Provide}{\cid}$ operation, the \provider first uses $\Call{GetClosestPeers}{\cid}$ to locate the $k=20$ peers with their peer IDs closest to $\cid$, 
%
and then sends them a $\mathsf{PutProvider}$ message including the provider record (\Cref{fig:add_get_provider}(a)).
We call the peers that hold provider records for $\cid$ the \emph{resolvers} for $\cid$.

Each CID can have several \providers. In fact, by default, each IPFS client becomes a provider for each piece of content it downloads for a fixed amount of time (12h, 24h, or 48h depending on the client version or custom configuration). As a result, the system provides an auto-scaling feature with supply automatically rising with demand.

%
When a \downloader wishes to fetch a piece of content, it first sends a request to all its \bitswap peers. If none of them has the content, the \downloader uses the DHT-based resolution system. We stress that the \bitswap protocol plays the supporting role of a cache in the dissemination of popular files. However, the mechanism does not provide reliable content resolution, in particular for new or less popular content. %

When \bitswap unstructured search fails, the \downloader resolves $\cid$ using $\Call{FindProviders}{\cid}$. This operation uses a DHT walk identical to that of $\Call{GetClosestPeers}{\cid}$ to find $k$ \resolvers but also queries encountered nodes for a provider record for $\cid$ (\Cref{fig:add_get_provider}(b)). The process terminates when either 20 \providers have been found, or all \resolvers have been asked. Querying all encountered nodes (\ie, not only the designated \resolvers) is useful because some of the encountered nodes may have a provider record in their cache.
%

Upon receiving a provider record, the client connects to the address specified in the provider record to retrieve the actual content (\Cref{fig:add_get_provider}(c)).
Provider records are not authenticated, and therefore malicious \providers may respond with incorrect provider records (or may not respond at all). However, the integrity of the content is preserved because the hash of the retrieved content can be verified against its $\cid$.
%


%

\input{img/add_get_provider.tex}

\vspacebeforesection
\subsection{Network Size Estimator}
\label{sec:netsize}

The number of nodes in a decentralized system is generally unknown due to the avoidance of centralized membership management.
This number is nonetheless useful for optimizations, deciding on individual node configurations, or security mechanisms.
Various methods were proposed for the decentralized estimation of unstructured and structured networks~\cite{eli-sohl-dht-size-estimation,kostoulas2005decentralized, manku2003symphony}.
We use in this work a mechanism developed initially by Protocol Labs as part of a mechanism for decreasing the latency of publishing content in IPFS~\cite{network-size-estimation-notion,network-size-estimation-github-pr}.

%
%
%
%
%
%
%
%
%
%

Each node in the DHT refreshes its routing table periodically (every $10$ minutes in \texttt{libp2p}). 
For this, the node samples $m$ random keys (one for each bucket of its routing table)
%
and queries the DHT to obtain the $k=20$ closest peer IDs to each key.
Using these, the node then computes the average distance between each one of these keys $\key_j$ for $j=1,\dots,m$ and their $i$-th closest peer ID for $i=1,...,k$ (with $m=256$ and $k=20$).
\begin{equation}
    \label{equ:avg-dist}
    \overline{D}_i = \frac{1}{m} \sum_{j=1}^m \operatorname{dist}(\key_j, \peerid_{j}^{(i)})
\end{equation}
where $\peerid_{j}^{(i)}$ is the $i$-th closest peer ID to $\key_j$.
With $N$ peers in the DHT and peer IDs uniformly distributed in the hash space, the expected distance between a $\key$ and its $i$-th closest peer ID is $\frac{2^{256}i}{N+1}$. The node then runs a least square regression to compute the value of $N$ for which the expected distances best fit the empirical average distances, \ie,
\begin{equation}
    \label{equ:netsize-least-squares}
    \hat{N} = \arg\min_{N} \sum_{i=1}^k \left(\overline{D}_i - \frac{2^{256}i}{N+1}\right)^2.
\end{equation}
The resulting estimate $\hat{N}$ can be computed in closed form.
%

When a node starts running, it must perform DHT queries for a few random keys to initialize its network size estimate. 
Since a larger number of queries will result in higher accuracy, making more queries than what is needed to initialize one's routing table is recommended.
Thereafter, keeping the estimate up-to-date does not require any excess DHT queries beyond what is already used for refreshing the routing table as this is done frequently (every 10 minutes).

While the network size estimate has a stochastic variance resulting from the probability distribution of the honest peer IDs, it is hard for an attacker to bias the estimate significantly. Since the estimator uses the density of peer IDs around keys chosen uniformly at random, the adversary would require numerous Sybil nodes (on the order of the whole network size) to significantly affect the peer ID density around those keys.


\section{approach overview}
We design and implement a workflow to explore our research goals.
% The pipeline can (1) crawl iOS apps from Apple App Store~\cite{appleStore} and identify all on-device models in crawled iOS apps;
% (2) match iOS apps' Android counterparts in Google Play;
% (3) evaluate the robustness of on-device models against adversarial attacks on iOS.
As shown in Figure~\ref{fig:workflow}, the first step of the workflow is to crawl iOS apps from Apple App Store.
This is achieved by simulating real people downloading iOS apps on iPhone emulators using the IPA tool~\cite{ipaTool}.
% These apps' entire relevant metadata, including the app category, ratings, reviews, and developers, is likewise scraped from the Apple App Store.
Following the guideline for iOS reverse engineering~\cite{owasp}, we recompile and extract non-code resources from all crawled iOS IPA files to detect on-device models.
We study these apps and on-device models in RQ1.
Second, we match identified iOS apps with their Android mirror apps on Google Play~\cite{googleplay}.
We study how and why developers choose different on-device models across Android and iOS.
Third, we propose a new approach for employing adversarial attacks to gray-box Core ML models on iOS.
We employ our proposed adversarial attack approach to evaluate the robustness of gray-box on-device models on iOS in RQ3.
Finally, we select attacked models to identify the point in apps where models are invoked and then manually input the adversarial examples to validate our method's effectiveness in directly attacking real-world iOS apps.

We will discuss the detail of crawling apps, detecting on-device models, paring Android counterparts, and attacking models in the following sections.

% Figure environment removed
\section{RQ1: what are the characteristics of on-device models in iOS apps}


\subsection{Motivation}
As with the study of on-device models in Android~\cite{huang2021robustness, huang2022smart, xu2019first}, this RQ focuses on on-device models' existing characteristics in iOS apps.
To comprehend the current trends, characteristics, and experiences of on-device models employed by iOS developers, we concentrate on DL framework selection, model quantity, model size, model type, model functionality and model developer for iOS app models.
The exploration of this RQ can support our future investigation into the causes of the current trends and the potential security vulnerabilities associated with the current trends.

This is the first study of the on-device model in iOS, to offer potential insights to future researchers, we begin by presenting our pipeline for data collection in the iOS platform, including how to acquire iOS app files from the Apple App Store~\cite{appleStore} and how to recognise on-device models in iOS apps.

\subsection{iOS App Collection}
\label{sec:appCollection}

\subsubsection{How to Select Apps with On-device Models}
First, we download all 2312 top-rated free apps across 25 categories which are all publicly available on the Apple App Store~\cite{topRatedApp}.

Second, we review the literature on common application scenarios of DL models in mobile apps~\cite{gcloud, xu2019first} and identify 16 prevalent applications based on our own expertise and observations. 
These applications include photo enhancement, object detection, image classification, face detection, image segmentation, optical character recognition (OCR) text recognition, augmented reality, barcode scanning, language identification, smart replies, translation, sound recognition, recommendations, movement tracking, video segmentation, and gesture recognition.
To identify relevant apps, we conduct a search on the iOS app store.
We utilize the identified application scenarios as search keywords and manually sift through the search results to isolate apps that exhibit these features. 
Our search yields 205 apps that align with the aforementioned application scenarios in this way.

Third, one app is usually available on both Android and iOS platforms~\cite{ali2017same}.
Due to their similar functionality, iOS apps may contain on-device models if on-device models are discovered in their Android counterparts.
We discover 423 Android apps that contain on-device models out of a total of 26,346 Android apps crawled from Google Play~\cite{googleplay}.
To match their iOS counterparts, we start by finding corresponding iOS apps with the same app names and developers.
However, the name of the same app may be slightly different on different platforms~\cite{ali2017same}.
For instance, the \emph{Tubi} app's official name on Android is \emph{Tubi-Movies \& TV Shows} whereas it is \emph{Tubi-Watch Movies \& TV Shows} on iOS.
Therefore, we manually locate iOS equivalents for Android apps that cannot be found with the same app name and developer.
In this way, we collects 390 iOS apps.

Until September 30, 2022, we collected a total of 2,907 iOS apps by using previous three approaches.

\subsubsection{How to Get Source Files of iOS Apps}
After logging in with an apple id, we use automated python scripts to imitate the user downloading all free apps from the apple store to the iPhone emulator. Then, we utilise the IPA~\cite{ipaTool} tool to extract all of the downloaded apps' source files.
We make all our collected iOS apps publicly available for more researchers to study\footnote{\href{https://github.com/huhanGitHub/iOS-App-database}{iOS-App-database}}.


\subsection{On-device Model Collection}


\begin{table*}[htbp]
\vspace{-0.5cm}
\setlength{\abovecaptionskip}{10pt} 
\setlength{\belowcaptionskip}{10pt}
\caption{An overview of popular deep learning frameworks and their smartphone support at the time of writing (Sep. 2022).}
\begin{adjustbox}{ width=\textwidth,center}
\centering
\begin{tabular}{|lcccccc|}
    \noalign{\hrule height 1pt}
\textbf{Framework} & \textbf{Owner} & \textbf{Mobile Platform} &
\textbf{Mobile API} & \textbf{Is Open-source} & \textbf{Supported Model Format} & \textbf{Support Training} \\ 

    \noalign{\hrule height 1pt}

ONNX~\cite{ONNX} & ONNX & Android \& iOS & Java, C, C++, OC & \cmark & Protobuf (\emph{.pb, .onnx}), Numpy (\emph{.npz}) & \cmark\\ 

TF Lite~\cite{tfLite} & Google & Android \& iOS & Java, Python, C++, OC, Swift & \cmark & FlatBuffers (\emph{.tflite}) & \cmark\\ 

Caffe~\cite{caffe} & Berkeley & Android \& iOS & C++ & \cmark & customized, json (\emph{.caffemodel, .prototxt}), json, YAML & \cmark\\ 

Caffe2~\cite{caffe2} & Facebook & Android \& iOS & C, C++ & \cmark & ProtoBuf (\emph{.pb}) & \cmark\\ 

MxNet~\cite{mxnet} & Apache Incubator & Android \&  iOS & C, C++ & \cmark & customized, json (\emph{.json, .params}) & \cmark\\ 

DeepLearning4J~\cite{dl4j}  & Skymind & Android \& iOS & Java & \cmark & customized (\emph{.zip}) & \cmark\\ 

ncnn~\cite{ncnn} & Tencent & Android \& iOS & C++ & \cmark & customized (\emph{.params, .bin}) & \xmark\\ 

OpenCV~\cite{opencv} & OpenCV Team & Android \& iOS & C, C++ & \cmark & TensorFlow, Caffe, Darknet, ONNX, Torch, PyTorch & \cmark\\ 

FeatherCNN~\cite{feathercnn} & Tencent & Android \& iOS & C++ & \cmark & customized (\emph{.feathermodel}) & \xmark\\ 

Paddle-Lite~\cite{pdlite} & Baidu & Android \& iOS \& Kirin  & Java, C++ & \cmark & parambase (\emph{.pdmodel, .pdparams, .pdopt}), TensorFlow, Caffe, ONNX, PyTorch & \xmark\\ 

MNN~\cite{mnn} & Alibaba & Android \& iOS & Python, C++ & \cmark & TensorFlow, Caffe, Darknet, ONNX & \cmark\\ 

MACE~\cite{mace} & XiaoMi & ARM-based \& Android \& iOS & C++ & \cmark & customized (\emph{.pb, .yml, .a}), TensorFlow, Caffe,  ONNX & \xmark\\

CoreML~\cite{coreML} & Apple & iOS \& iPadOS \& watchOS & OC,Swift & \xmark & customized, ProtoBuf (\emph{.proto, .mlmodel}), TensorFlow, Caffe,  ONNX, PyTorch & \cmark\\ 

PyTorch Mobile~\cite{pytorch-mobile} & Facebook & Android \& iOS & Java, OC, Swift & \cmark & customized, pickle (\emph{.ckpt, .pkl, .pt, .pth, .ptl}), ONNX & \xmark\\ 

Bender~\cite{bender} & Xmartlabs & iOS & Swift & \cmark & Tensorflow & \xmark\\ 

    \noalign{\hrule height 1pt}
\end{tabular}
\end{adjustbox}
\label{tab:overview_of_frameworks}
\vspace{-0.5cm}
\end{table*}


Inspired by related on-device model detection approaches~\cite{xu2019first, huang2021robustness, huang2022smart}, we identify the on-device model by matching specific suffix patterns of the model files in reverse-engineered iOS app source files.

We first investigate popular DL frameworks today that enable the deployment of iOS on-device models.
Following the framework selection process in related works~\cite{xu2019first, huang2021robustness}, we commence our investigation by searching Google and GitHub to identify popular DL frameworks.
We collect relevant information on the features of these frameworks from their official documentation, such as the supported model formats, the suffix patterns of the on-device model supported, open-source status, etc.
We also adopt a similar set of evaluation criteria employed in prior studies~\cite{xu2019first, huang2021robustness} to determine the suitability of the identified frameworks. 
Specifically, frameworks that have not been actively maintained for more than two years and have garnered minimal attention, such as fewer than 100 open-source projects on GitHub, are excluded from consideration in this study.
Table~\ref{tab:overview_of_frameworks} shows the overview of this investigation.
We delete frameworks with fewer than 100 GitHub stars and no updates in the past two years~\cite{xu2019first}.
As illustrated in Table~\ref{tab:overview_of_frameworks}, the columns \emph{Framework}, \emph{Owner}, \emph{Mobile Platform}, \emph{Mobile API} and \emph{Supported Model Format} represent the names, owners, supported mobile platform, supported mobile API and supported model format of DL frameworks, respectively.  
We refer to the suffix patterns in the column \emph{Supported Model Format} to detect on-device models.
The columns \emph{Is Open-source} and \emph{Supported Training} indicate whether the framework is open source and whether the framework's on-device model supports training. 
There are 15 popular DL frameworks support deploying on-device models on iOS now.
9 out of 15 framework models support training, while the remaining models only support prediction using trained models.


To ensure the quality of the model and remove false positive cases, we evaluate the quality of each detected model by loading the model and performing predictions on randomly generated data.
Finally, we discover 1,883 valid on-device models among 2,907 iOS apps after eliminating all non-predictive models.
These 1,883 on-device models belong to 334 iOS apps, of which 190 apps are from top-rated free apps, 40 are from related application scenarios searches, and 104 are from matching Android-iOS app pairs.



\subsection{Characteristics of Apps with On-device Models}

% \chen{1. Are the absolute number comparable? Or use relative percentage? 2. We are analyzing IOS models in this RQ, why also put Android here? Maybe move it to RQ2?}

We study the characteristics of existing on-device models on the iOS platform from two aspects. 
The first aspect is to look at the features of apps containing on-device models, including the category of apps in Section~\ref{sec:appCate}, the number of models contained in each app in Section~\ref{sec:appNum}, the size of the model as a percentage of the app's size in Section~\ref{sec:modelSize} and the developers of DL apps in Section~\ref{sec:appDev}.
Apps dominant portion of smartphones and platforms~\cite{android, apple}.
The first perspective enables us to understand current app-level trends in the employment of on-device models on the iOS platform.
The second aspect is to look at the features of on-device model itself, including the type of current models in Section~\ref{sec:modelType}, the functionality of current models in Section~\ref{sec:modelFunction} and the adoption of DL frameworks in Section~\ref{sec:frameCompare}.
The second perspective allows us to gain insight into the model-level characteristics, laying the foundation for the subsequent research questions.

\subsubsection{How On-device Models are Distributed among App Categories?}
\label{sec:appCate}
We explore the categories of  334 DL model apps.
The top five categories of apps are Photo \& Video (44), Shopping (33), Social Networking (22), Health \& Fitness (20), and Travel (17).
Note that most on-device models are still used in computer vision-related scenarios, like OCR text detection, object recognition, and face detection, even though these apps fall outside of Photo \& Video category.
% We summarize the application scenarios of DL models in apps outside the Photo \& Video category and find that OCR text detection, object recognition, and face detection are the most prevalent.
The findings show that computer vision-related scenarios still predominate the industrial applications of deep learning.
% Deep learning on-device models are most commonly applied in the disciplines of face beauty, virtual reality, and face detection.


\subsubsection{How Many On-device Models are there in One App?}
\label{sec:appNum}
Figure~\ref{fig:modelNum} shows the boxplot of the number of on-device models in one app on iOS.
The mean is 5.54, and the median is 2, indicating that the average number of models in a single app is 5.54 and that half of DL model apps have less than 2 models.
The outliers in the box plot demonstrate that 14 apps own more than 20 DL models.
These outliers are all photo/video processing apps, like Gradient (95), Video Star (51), Facetune2 Editor (160), etc.
These apps provide a multitude of image-processing features, such as skin smoothing, eye enlargement, beauty, etc.
To fulfill these features, developers typically employ multiple pre-trained on-device models.
Hence, these apps contain more on-device models.
\begin{comment}
Among apps with fewer than two models, the majority of apps use object detection models like \emph{SSDOcr.mlmodelc} and \emph{rpn\_text\_detector.tflite} to detect OCR text or QR code.
These apps fall in the Tool, Social Networking, and Lifestyle categories rather than Photo \& Video.
\end{comment}

% Figure environment removed



\subsubsection{What is the Size of On-device Models?}
\label{sec:modelSize}
Given the limited computing and storage resources of mobile phones, the size of the model is a crucial indicator of its suitability for deployment on such devices. Models with large sizes may cause performance and storage issues, while smaller models may be more practical for use on mobile devices~\cite{onCloud, xu2019first}.
The average model size of detected 1,883 on-device models is 0.45MB.
Figure~\ref{fig:modelSize} demonstrates the boxplot of the percentages of the app's size that is occupied by the app's models (Total model sizes/App size).
Results show that the present app's model size is barely average of 1.7\% of the app size.
The apps with the highest proportion are \emph{SoundLab Audio Editor} 46.26\%, \emph{iScanner} 20.92\% and \emph{Carl: Plant Identification} 18.77\%.
The average proportion of on-device models is small, but outliers in Figure~\ref{fig:modelSize} illustrates that the proportion of model size exceeds 10\% in all apps where deep learning plays a crucial role, such as \emph{B612}'s 17.63\%, \emph{EPIK Photo Editor}'s 14.37\%, and \emph{VITA Video Editor}'s 15.72\%.
Even though the average proportion of model size is quite small, the proportions of models in machine learning-related apps are considerable.
Large models in apps will increase the cost of development and maintenance and may have a negative effect on the user experience if they slow down the app~\cite{ballard2007designing}.
Optimizing and downsizing the on-device models is an urgent and promising direction.

\subsubsection{Who Develops On-device Model Apps?}
\label{sec:appDev}
The iOS app store provides a details web page for each app, which includes relevant information about the app's developer. 
In this study, we crawl these web pages and collect the developer information for 334 apps.
334 iOS apps are developed by 297 different companies or developers. 
\emph{Google LLC} has the most apps that employ on-device models, with 10, followed by \emph{Microsoft Corporation} with 5 and \emph{Meta Platforms} with 4.
We notice that Google's apps, such as \emph{Google Street View}, \emph{Google Family Link}, and \emph{Google Home}, typically employ their own frameworks: TensorFlow and TF Lite.
Correspondingly, Microsoft, Meta, and the majority of small- and medium-sized companies, such as Zoom, Netflix, and SHEIN, commonly employ the third-party frameworks Core ML and TF Lite in their apps, such as \emph{Microsoft Office}, \emph{Microsoft OneDrive}, \emph{Meta Business Suite}, and \emph{Oculus}.
Owing to the accumulation of technology and vast data,
large tech companies are the most prolific developers of on-device models.


\subsubsection{What Types are Current On-device Models?}
\label{sec:modelType}
Among 1,883 on-device models, 1,561 (82.9\%) models are CNN models, which are primarily used for image and video classification and detection, 163 (8.7\%) models are RNN models, which are primarily used for text and sound classification, and remaining 159 (8.4\%) models fail to identify the model structures.
The results are consistent with the model functionality analysis in Section~\ref{sec:modelFunction}, as the majority of on-device models in current iOS apps are used in the computer vision domain, and hence the majority of models are of type CNN.


\subsubsection{What are On-device Models Used for in iOS Apps?}
\label{sec:modelFunction}
To figure out what on-device models are used for in iOS apps, we propose a strict pipeline to analyze the functionalities of current on-device models.

\input{tables/tab_figure}

We classify the functionality of a model in three steps.
According to a recent study~\cite{huang2021robustness}, over 60\% of existing DL models in industrial apps are structurally comparable to existing pre-trained models from TensorFlow Hub~\cite{tfhub}.
So, we first manually check whether the model is fine-tuned from typical model architectures, such as ResNet~\cite{he2016deep}, Mobilenet~\cite{howard2017mobilenets}, YOLO~\cite{redmon2016you}, etc.
Typical model architectures are used for specific tasks, such as YOLO for detection, ResNet for classification, and DeepLab~\cite{chen2017deeplab} for segmentation.
The model's functionality could be identified by its classical model structure.
Following the related work proposed by Huang et al.~\cite{huang2021robustness},  we assess whether a given model has been fine-tuned from one of the models in TensorFlow Hub by comparing the structure and parameter similarity between the two models. 
Specifically, we first extract the structure information of the model in terms of layer names, shapes, and data types, and represent it as a string sequence. 
Similarly, we convert a set of pre-trained models into string sequences. 
We then compute the similarity between the model sequence and the pre-trained model sequences. 
Based on this similarity, we can determine whether the model's structure is similar to that of any pre-trained models. 
We follow related works~\cite{huang2022smart, sim2019investigation} to set the threshold of similarity to 80\%.
If the model structure matches a pre-trained model, we compute the parameter similarity between the two models to verify whether the model has been fine-tuned.
Second, we infer the use of models by analyzing the output layer of the mode.
Models with a single activation function, such as \emph{Softmax} and \emph{Sigmoid}, are typically employed for classification tasks\cite{activationFunction}.
Due to the necessity of locating the object's bounding box, the detection task's output layer of models contains regression operations~\cite{szegedy2013deep}.
After the previous two steps, we can identify the model's functionality at a high level.
In the final third step, we follow the approach proposed by Xu et al.~\cite{xu2019first} to infer the model's precise usage scenario by analyzing the semantic data in the app, including the app description, app content, the model name, the layer name, detected labels, and other app-related information.
Finally, we run the app to validate our inferences.
For instance, if the model name is \emph{face\_detect.tflite}, we first confirm that it is a detection model by analyzing the last layer of the model. Next, we infer from the model name and the app's documentation that the model may have a face detection function. Since face detection requires the use of the camera and pictures, we then search for all places in the app that use the camera and pictures for content recognition to verify if face detection exists. If we find face detection in any of these places, we assume that the model's function is face detection. However, if the face detection function is not found, we assign the model's function to object detection at a high level.


Given the amount of effort, we randomly sample 68 (20\%) apps in the 334 iOS apps to study their model functionalities.
We discover 420 (22\%) on-device models among these 68 apps.
Table~\ref{tab:modelUsage} shows the results of our model functionality analysis among these 420 models.
% We refer to related works~\cite{huang2021robustness, xu2019first} and Google Cloud to classify model functionalities.
332 (79\%) on-device models are used in the computer vision field.
The most widely used scenario is Photo Beauty (68), followed by Object Detection (48) and Image Classification (46).
In the natural language processing field, we discover that Language identification (20) is the scenario that uses on-device models the most.
On-device models like \emph{tflite\_langid.tflite} are frequently used by apps like \emph{Camera Translator: Translate +}, \emph{Think Dirty - Shop Clean}, and \emph{Wizz - Make new friends} to distinguish the types of input languages.
Sound recognition (10) is the most widely-used task in the audio field.
In addition, on-device models are also frequently utilized for recommendations (17), particularly in planning apps like \emph{Structured-Daily Planner}, \emph{Motivation-Daily quotes}, and \emph{I am-Daily Affirmations}, where developers frequently make use of the trained On-device model to provide users with appropriate recommendations.

According to the results, computer vision-related domains continue to be the most popular application areas for on-device models. Despite the widespread usage of deep learning techniques in machine translation, developers currently prefer to employ on-cloud deep learning models or third-party APIs for translation and on-device models to identify the type of input language.


\begin{comment}

\subsubsection{App support languages}
\han{optional}
The language an app supports is an important factor reflecting the distribution of app users.
325 of the 340 apps are available in English, followed by 213 in Spanish, 211 in French, 205 in German, and 181 in Chinese.
Developers in almost any country will provide English support for their apps, and English users are by far the largest user base for these apps.

\end{comment}



\begin{comment}
\subsubsection{How the model's effect would influence app ratings and popularity?}

\han{optional}
To explore if and how the effect of the on-device models would influence app ratings and popularity, we conduct a pilot study to analyze app reviews and ratings qualitatively.

% We first select DL model apps and collect all on-device models in these apps.
We first follow the steps in Section~\ref{sec:modelFunction} to figure out the functionalities of detected on-device models in iOS apps.
Second, we collect the top 3 positive and 3 negative reviews of these apps on Apple App Store.
To figure out the impact of on-device models for app ratings and popularity, we manually assess whether these reviews are relevant to the model's functioning after reading them. 
If people praise certain app features as being closely related to the model's functionalities, we consider the model to have a positive influence on the app and vice versa. 

34 apps (10\%) and 201 (10.67\%) on-device models in these apps are randomly selected for this study.
We collect the top 3 positive and negative reviews of these apps from the app store and invite 3 volunteers to read them to see if the content is relevant to the models.
Note that a single review may mention multiple app features, such as the app's price and device compatibility issues, at the same time.
In 102 positive reviews, 51 (50\%) are considered to be relevant to the on-device model functionalities.
On-device model-related functionalities are the most commonly mentioned features in positive reviews, ranking above free (42), usability (23), etc.
For example, in reviews of the app \emph{YouCam Makeup: Selfie Editor}, the highest-ranked reviews by users state \emph{Can do animal makeup too!} and \emph{it helps me take so Many great photos and make u look more beautiful}.
These two reviews clearly show that it is the effect of the DL model that makes the app so well received by users.
In contrast, only 10 out of 102 negative reviews mention model-related features, far below non-free (81), frequent updates (35), device compatibility issues (26), etc.
The results demonstrate that the functionality of the on-device model has been one of the primary reasons why users appreciate an app.
A solid DL model could improve the user experience, entice and retain more users, and increase the app's popularity. 
Deep learning technology has been the major competitive advantage of these apps.

% We find that the fact that the majority of DL apps are photo editing and beauty apps that are well-liked by young users is one of the main reasons for the high amount of reviews on DL model apps.
% Young users are more active and sensitive to upgrades to app functionality, therefore they are more likely to provide app reviews.
% Additionally, we find that features linked to on-device models, such as face beauty effect, good photo capturing, and simple photo editing, are often cited by users in their comments.


\end{comment}


\subsubsection{What is the Adoption of DL Frameworks in iOS Apps? How does it Compare to Models in Android Apps?}
\label{sec:frameCompare}

We investigate the adoption of DL frameworks in 334 iOS apps with on-device models and compare it with their Android counterparts.
1,744 and 1,642 on-device models are validated successfully for the frameworks on 334 iOS and Android apps, respectively.


\begin{comment}
% Figure environment removed
\end{comment}



Figure~\ref{fig:dlDis} demonstrates the distribution of DL frameworks in 334 iOS apps and their comparisons on Android.
There is a significant difference between the distribution of DL frameworks in Android and iOS.
The most widely-used framework in iOS is Core ML, with 638 (36.58\%), followed by TF Lite, with 400 (22.94\%), and TensorFlow 335 (19.21\%).
The first three frameworks combined make up over 75\% of the market for industrial apps.
Android apps usually employ TensorFlow and TF Lite as DL frameworks, accounting for 789 (48.05\%) and 342 (20.83\%) out of 1,642 DL models.
Compared to iOS apps, Android apps have more customized DL models (276, 16.81\%). 
One possible explanation is that Core ML and TF Lite are optimized for their respective platforms and exhibit better compatibility, resulting in lower development costs. Additionally, both frameworks have a large community of users and developers who provide support and resources, making them more accessible and easier to use.
Moreover, the popularity of these frameworks may also be attributed to the companies behind them. Apple and Google are well-established and widely recognized tech giants, and their frameworks may be perceived as more reliable and trustworthy compared to other third-party alternatives.

Moreover, we find that TF Lite and TensorFlow remain the second and third most used frameworks in the iOS platform, with a share of 22.94\% and 19.21\%, respectively. This suggests that the continued popularity of TF Lite and TensorFlow in the iOS platform is not solely based on their ownership by Google but may also be due to other factors.
One possible reason for their popularity is that TensorFlow has been in the market longer than Core ML and has established itself in the deep learning community. Developers may have more experience and expertise in using TensorFlow, making it easier for them to integrate it into their iOS apps. Furthermore, TensorFlow has a larger user base, providing developers with a larger community for support and resources.
Another reason for the popularity of TensorFlow is its wide range of functionalities, including support for natural language processing and speech recognition, in addition to computer vision tasks. This versatility means that TensorFlow may be preferred by developers who require support for a wider range of tasks. Furthermore, TensorFlow may have cross-platform compatibility, allowing developers to use the same model across multiple platforms, including iOS and Android.
% We observe that models, the majority of which have the suffix "model", could not be assigned to a specific framework.
We discover that these customized models are primarily derived from three sources: customized DL SDK, like SenseTime~\cite{sstimeSDK}, traditional machine learning frameworks like XGBoost~\cite{xgboost} and compressed/obfuscated DL models.
% There are still some developers who employ their own deep learning technologies rather than well-known frameworks for the stability of their apps.
Given that Android is considered to be less safe than iOS, more Android developers choose to compress or encrypt the model's parameters and structure to improve the security of on-device models~\cite{huang2021robustness}.
% Other frameworks have much less coverage than TF Lite, TensorFlow, and Core ML.
Compared to Android apps, iOS apps are more likely to utilize Caffe.
However, possibly due to the development expense, many iOS apps, which employ the Caffe framework, such as \emph{NETGEAR Nighthawk - WiFi App} and \emph{Stash: Invest \& Build Wealth}, use the more popular frameworks TF Lite and TensorFlow in their corresponding Android apps, rather than Caffe.
A tiny number of Android and iOS apps use other frameworks like NCNN~\cite{ncnn}, Paddle Lite~\cite{pdlite}, etc.
% We discover that uncommon frameworks are used by less popular apps.
% Popular apps now all use Core ML, TF Lite, TensorFlow, and other mainstream frameworks.
Mainstream apps tend to employ popular and stable DL frameworks like Core ML, TF Lite, and TensorFlow.

On the developer side, reusing the same framework technology when migrating apps across platforms can reduce development costs and improve development efficiency.
On the model side, popular deep learning frameworks support both Android and IOS platforms, and on-device models are intentionally designed to be highly re-usable~\cite{coreML, TensorFlow, xgboost, caffe2, mxnet}.
Despite the framework's efforts for cross-platform reuse architecture, current developers have not adopted it to its fullest extent.




\begin{summary}{}{}
On-device models are utilized extensively across all categories of apps for various functions, especially photo beauty and object detection.
Framework vendors should consider optimizing the model size and quantity in apps.
Large tech companies prefer to on-device models in their apps.
Current on-device models are mostly used in the field of computer vision, hence the majority of models are of the CNN variety.
Developers are more likely to utilize DL frameworks provided by Android or iOS operating system owners like Core ML, TF Lite and TensorFlow.
% Core ML has become the most popular framework for using on-device models on iOS apps.
The consistency between the same app on the Android and iOS platforms utilising the DL Framework is glaringly lacking.
% \chen{Still do not see interesting findings for RQ1.}
\end{summary}








\section{RQ2: Why developers use different models for one app on iOS and Android?}
% \chen{One big issue with this work is that there is too much text especially in the later part, while few figure/table, which makes it really hard for reviewers to understand}
% \chen{How do you know which app version is migrating from which app version? i.e., ios -> android or android -> ios?}
To conquer the market, one mobile app can be available on both Android and iOS~\cite{ali2017same}.
% Reusing the same DL frameworks and models in different platforms can reduce the engineering efforts and expenses of companies.
% A trained DL model is also easily reusable~\cite{coreML, TensorFlow, xgboost}, and retraining a DL model demands a substantial amount of high-quality data and expense~\cite{trainDLPrice}.
As discussed in Section~\ref{sec:frameCompare}, considering the cost of development associated with retraining a model, most developers should accept reusing models when developing mirror apps for other platforms.
However, we discovered notable differences of DL framework selection between Android and iOS apps in Section~\ref{sec:frameCompare}.
To investigate the impact of platform differences on the selection of DL frameworks by developers, we analyze the sharing and replacement of on-device models in iOS-Android app pairs and conclude why this occurs.


In this study, we first provide the method for comparing DL models to locate the same DL models between two platforms in Section~\ref{sec:modelCompare}.
We investigate current models' cross-platform selection in Section~\ref{sec:modelChange}.
Then we design a pipeline to investigate how current developers replace and share on-device models and analyze the underlying reasons for adopting different DL models in Section~\ref{sec:factorStudy}.
Through the study of RQ2, we explain how and why current developers use different DL models to support other developers in developing Android and iOS mirror apps.
We also provide suggestions for subsequent cross-platform optimization directions for the current DL frameworks from a developer's perspective.
It establish the groundwork for RQ3 to further analyse if current cross-platform usage approaches of on-device models pose any security threats.


\subsection{Model Comparison}
\label{sec:modelCompare}
We first match the same models between the two platforms.
Model structure and trained model parameters are the primary distinguishing characteristics of DL models~\cite{goodfellow2016deep}.
So, following related works~\cite{huang2021robustness, huang2022smart}, we identify reused on-device models by comparing their structure similarity (SS) and trained parameter similarity (PS).
Two model pairs are considered the same only if the structure and parameters are identical, as denoted by SS and PS values of 1.


\subsubsection{How to Compare Models' Structure Similarity?}
% The same model often has highly similar names on different platforms, so we first screened out models with highly similar names from different platforms of the same app.

We convert an on-device model to a sequence of elements, with each element constructed by the shape, data type, and name of each layer from the model~\cite{huang2021robustness, huang2022smart}. 
Given model $M_1$ and model $M_2$, their structural similarity is the longest common subsequence between the two models.
It is calculated by
\begin{equation}
    SS(M_1, \ M_2) = \frac{2 * L_{l}}{L_{M_1} + L_{M_2}}
\end{equation}
where $L_{l}$ is the length of the longest common subsequence in two models, and $L_{M_1}$ and $L_{M_2}$ are the numbers of two models' layers. 
Note that two layer elements are matched only if their attributes are totally the same.
The similarity score ranges from 0 to 1, and the higher the value, the greater the structural similarity between the two models.

\subsubsection{How to Compare Models' Parameter Similarity?}
\label{sec:ps}
After comparing the model structure similarity, we extract the trained parameters in each layer to compare the trained parameter similarity.
We count the number of elements with the same parameter in the longest common subsequence as their parameter similarity~\cite{goodfellow2016deep, huang2021robustness, huang2022smart}:
\begin{equation}
    PS(M_1, \ M_2) = \frac{2 * N_{l}}{L_{M_1} + L_{M_2}}
\end{equation}
where $N_{l}$ is the number of the same parameter elements in their longest common subsequence.

\subsubsection{How to Compare Gray-box Models?}
\label{sec:ps2}
As shown in Figure~\ref{fig:dlDis}, the gray-box model of Core ML framework is the most frequent model in iOS apps, accounting for more than one-third of the total.
For better platform compatibility, Apple encourage developers to convert the trained on-device model into the format of Core ML framework~\cite{CoreMLConvert}, such as converting the format from TF Lite in Android to Core ML in iOS.
Due to the same structure and parameters, the converted on-device and original models are considered equal.
However, it is hard for third parties to extract the trained parameters of Core ML models~\cite{coreML}.
To compare the Core ML model with the white-box model, we perform the following procedures.

% Core ML facilitates the conversion of certain TF Lite and TensorFlow models to Core ML format~\cite{CoreMLConvert}. 
We attempt to convert all eligible models to the Core ML format and compare the structure, weight and metadata of the converted models to verify if they are consistent.
Faced with models which cannot be converted to Core ML format, we first compare the structural similarity of the models and select two models with the same structure for further comparison.
We prepare a set of input data to feed to both models for prediction.
Two models are considered the same if their outputs are exactly the same.


\subsection{Model Selection on Two Platforms}
\label{sec:modelChange}
To acquire a comprehensive understanding of the present differences in DL model selection between iOS and Android developers, we first analyse quantitatively, among 334 matched iOS-Android app pairs, how many apps adopt distinct on-device models on Android and iOS.
We follow steps in Section~\ref{sec:modelCompare} to automatically and manually compare all models in 334 iOS-Android app pairs.

\begin{comment}
\subsubsection{Procedure}
Following model comparison steps in Section~\ref{sec:modelCompare},
we first automatically compare the structural similarity of on-device models in turn in iOS and Android apps.
All model pairs with the exact same structure are collected for further parameter comparison.
By applying parameter comparison steps in Sections~\ref{sec:ps} and~\ref{sec:ps2}, we manually and automatically compare the parameter similarities of all models with the same structure.
\end{comment}

\subsubsection{How Many Apps Adopt Different Models?}
In 334 corresponding Android apps, 125 (37.43\%) use completely distinct on-device models from that comparable iOS apps.
68 (20.36\%) out of 334 Android apps do not use on-device models.
Only 22 (6.59\%) out of 334 Android apps have on-device models that are all the same as their iOS counterparts.
21 (6.29\%) Android apps share models in part with their corresponding iOS apps.
These iOS-Android app pairs reuse a portion of on-device models across both platforms.
Our finding shows that most apps adopt different on-device models and frameworks on different platform apps.

\subsubsection{How Many Models are Shared between iOS and Android?}
\label{sec:ratio}
We match 332 (17.63\% ) the same on-device model pairs out of 1,883 on-device models on iOS.
We find 270 (81.33\%) out of 332 model pairs are exactly the same, including structure, trained parameters, and DL framework.
62 (18.67\%) out of 332 model pairs have the same structure and parameters but different DL frameworks.
On Android, these 62 on-device models are in TF Lite framework format, whereas on iOS, it is converted to gray-box Core ML model.
% We analyze the functionalities of these 282 app pairs.

% In the preceding sections, we study the variations in on-device models used by the same app on the iOS and android platforms. 
% In this part, we analyze the on-device models that are reused by the same app on both platforms.



\subsection{Reasons for Model Selection}
\label{sec:factorStudy}

% Figure environment removed

We provide a strict pipeline to study how current developers replace and share on-device models between Android and iOS and the underlying motives.
Considering the tremendous workload, we manually study a statistically representative random sample of 62 (20\%) out of 312 iOS-Android app pairs for a detailed study.
We find 226 on-device models with identified scenarios and functionalities in 62 iOS apps.

\subsubsection{Study Pipeline}
Figure~\ref{fig:rq2} shows the study pipeline of the effecting factors.
First, we identify the functionalities and usage scenarios of on-device models in selected iOS apps.
Second, we attempt to locate these use scenarios and functionalities within Android counterparts.
In this step, we follow the methodologies employed by recent related works~\cite{xu2019first, chen2021my, huang2022smart} to analyze the source code and semantic information in our investigated mobile apps.
For Android apps that can accurately find the code that calls the model, we instrumented the method at which the model is called and then dynamically run the app to dynamically analyze the app usage scenario where the model is called.
In cases where the calling code of the model could not be located, we manually analyze the semantic information of the model's functionality, file names, code comments, method names, and the user interface in the app to infer the possible scenarios in which the model could be used in the app. Finally, we validate our inferences by running the app and verifying whether the inferred scenarios encompassed the model's functionalities.
If we cannot find comparable functionality or usage scenarios in Android apps, we conclude that the original functionality of the on-device model on iOS has been discarded in their Android counterparts.
If we find the corresponding functionality and model, we further compare whether the model is the same.
If comparable functionality and usage scenarios can be found, but no corresponding on-device model, the role of the model on the iOS app is regarded as having been replaced in Android counterparts.

There are now three alternatives for on-device models: using other on-device models for the same task, invoking on-cloud models~\cite{gcloud, onCloud}, and implementing custom techniques to obfuscate models.
We first check if there are other models in the app that provide the required app functionality.
According to guidelines of the current frameworks for invoking on-cloud models~\cite{onCloud}, developers need pre-configure the cloud-related SDK libraries in the configuration files of the Android apps before invoking on-cloud models in the code.
For example, developers must pre-configure \emph{play-services-mlkit-text-recognition} in the configure file to invoke on-cloud text recognition services.
So, we first check if there are configured cloud-related SDK libraries in the configuration files of the Android app.
Then, we statically analyze whether apps invoke on-cloud models via application programming interfaces (APIs) provided by DL frameworks~\cite{mlKit} in decompiled app source files.
% We also check whether related libraries, such as \emph{libtensorflowlite\_jni.so} and \emph{libopencv\_java4.so} 
After failing to find the relevant on-device or on-cloud models, we inspect the decompiled Android apps to manually identify alternatives and customized DL models.
We search for two file types: the lib file of the DL frameworks associated with the model's functionality and the processed DL model file. 
Apps that employ deep learning techniques should include related libs~\cite{xu2019first, mlKit, pytorch-mobile}.
For example, the library file \emph{libtensorflowlite\_jni.so} is included in the app which uses TensorFlow.
Therefore, we used the presence of these files as an indicator of the usage of DL frameworks in the app.
We filter the obfuscated model file by analyzing the semantic information associated with the model, such as the names of the obfuscated and source models, and the names of the code files that call the model. 
If both the library file and the obfuscated model file are identified, we infer that the app's developer used a customized technique to obfuscate the model.


We invited three industry developers with at least one year of experience in Android and iOS development to conduct this study.
Each of the three participants follow the above steps to inspect all selected iOS-Android app pairs to explore the causes of change.
After the initial inspection, three volunteers have a discussion and merge conflicts.
They clarify their findings, scope boundaries among categories, and misunderstandings in this step.
Finally, they iterate to revise study results and discuss with each other until a consensus is reached.




\begin{table*}[htbp]
\vspace{-0.5cm}
\setlength{\abovecaptionskip}{10pt} 
\setlength{\belowcaptionskip}{10pt}
    \caption{Results of model sharing and replacement in iOS-Android app pairs}
\begin{adjustbox}{width=\textwidth,center}
    \centering
    \begin{tabular}{|c|c|c|cc|c|}
    \noalign{\hrule height 1pt}
        \multirow{2}*{\textbf{Alternative}} & \multirow{2}*{\textbf{\#Number (\%)}} & \multirow{2}*{\textbf{Example App}} & \multicolumn{2}{c|}{\textbf{\underline {iOS Platform}}} & \textbf{\underline{Android Platform}}  \\ 
        
        ~ & ~ & ~ & \textbf{Example Model} & \textbf{Model Function} & \textbf{Function Reservation} \\ 
        
        \noalign{\hrule height 1pt}
        
        \multirow{3}*{Select other model} & \multirow{3}*{58 (25.66\%)} &
         Hair Color Changer & hair\_segmentation.tflite & hair segmentation & Yes \\ 
        
        & & Rock Identifier: Stone ID & object-detection-centernet.mlmodelc & object detection & Yes \\ 
        
        & & Hello Pal:Talk to the World & M\_SenseME\_Face\_Video\_5.3.3.model & face detection & Yes \\ 
        \noalign{\hrule height 1pt}
        \multirow{3}*{Remove functionality} & \multirow{3}*{55 (\%24.34\%)} &
        Strike: Bitcoin \& Payments & inference\_graph.tflite & object detection & No \\ 
        
        & & AfterShip Package Tracker & SSDOcr.mlmodelc & OCR text detection & No \\ 
        
        & & adidas CONFIRMED & FindFour.mlmodelc & object detection & No \\ 
        
        \noalign{\hrule height 1pt}
        \multirow{2}*{Share On-device model} & \multirow{2}*{45 (\%19.91\%)} & \multirow{2}*{TikTok} & tt\_face\_attribute\_age.model & age prediction & Yes \\ 
        
        & & & tt\_face.model & face detection & Yes \\
        
        \noalign{\hrule height 1pt}
        \multirow{2}*{Select On-cloud model} & \multirow{2}*{34 (15.04\%)} &
        Homebase Employee Scheduling & SSDOcr.mlmodelc & OCR text detection & Yes \\ 
        
        & & Postmates - Food Delivery & DocScanCSCModel.tflite & object detection & Yes \\

        \noalign{\hrule height 1pt}
        
        \multirow{3}*{Obfuscate/Encryption model} & \multirow{3}*{24 (10.62\%)}
        & \multirow{3}*{EnhanceFox - AI Photo Enhancer} & EVBodyML100S16FP16.mlmodelc & \multirow{3}*{photo enhancement} & Yes \\
        & ~ & & facesr\_380000\_pt.mlmodelc & ~ & Yes \\ 
        & ~ & & RealESRGANv2-animevideo.mlmodelc & ~ & Yes \\ 
        
         \noalign{\hrule height 1pt}
        Unknown & 10 (4.42\%) & TickPick: No Fee Tickets & SSDOcr.mlmodelc & OCR text detection & Yes \\ 
    
    \hline
    \end{tabular}
\end{adjustbox}
    \label{tab:notuse}
    \vspace{-0.5cm}
\end{table*}

\subsubsection{Results}
Table~\ref{tab:notuse} shows the investigation results of model sharing and replacement for 226 models in iOS-Android app pairs.
The column \emph{Alternative} represents the alternative method adopted by the corresponding Android app.
The \emph{unknown} in column \emph{Alternative} represents we cannot determine how this model is replaced now.
The column \emph{Number (\%)} represents the number and percentage of models are treated in this way.
The column \emph{Example App} shows some representative apps which are treated in this way. 
The column \emph{Example Models} and \emph{Model Functions} represent the detected on-device models in iOS apps and their functionalities.
The column \emph{Function Reservation} shows whether the identical function is reserved in the corresponding Android apps. 
% The column \emph{Alternative methods} represents the alternative method adopted by the corresponding Android app.

58 (25.66\%) out of 226 models choose alternative on-device models in corresponding Android apps.
The app \emph{Hello Pal} in Table~\ref{tab:notuse} employs a third-party SDK from SenseTime for face detection on iOS, but the Android app leverages Google's face detection technology.
% Additionally, we discover that some apps of large companies, like ByteDance, will eventually adopt their own technical solutions to replace the third-party DL framework that was previously used and to create and customize their own internal DL models.
Because of the difference in platforms, the same app's can select more appropriate frameworks~\cite{nguyen2019machine}.
Similar to choosing on-cloud models, we discover developers, particularly those from small and medium-sized developing teams, tend to select DL frameworks like Core ML and TF Lite that are more practical, simple to use, and more affinity with mobile systems~\cite{chooseDL, findDL}.


In corresponding Android apps, 55 (24.34\%) of the 226 models are eliminated. 
We find that the functionalities of these eliminated models are not fundamental to the apps, so their removals have no impact on the apps' primary business.
For example, the \emph{inference\_graph.tflite} model in Table~\ref{tab:notuse}, which is used for inference in the \emph{Strike:Bitcoin \& Payments} app, is removed in the Android counterpart, but no effect for this app's business.
According to the survey report~\cite{diffTeam}, Android and iOS apps are frequently developed by different teams in one software business due to their varied technical routes.
Thus, Android versions may use various technical solutions, like as DL frameworks, resulting in small variations in the auxiliary functions of the same app on different platforms.

45 (19.91\%) of the 226 models are found to be shared between Android and iOS apps.
We notice that the functionalities of these re-used models are always customized by the apps' features and plays vital parts in their apps.
TikTok, for instance, reuses two on-device models: \emph{tt\_face\_attribute\_age.model}, which is used for age prediction, and \emph{tt\_face\.model}, which is used for face detection in both Android and iOS apps.
 
34 (15.04\%) out of 226 models are replaced with on-cloud DL models.
Different from Core ML on iOS, Google's ML Kit~\cite{mlKit} offers practical and convenient on-cloud DL model solutions. 
Android developers can activate on-cloud deep learning models for text, face, and barcode recognition with a few lines of code.
Note that differences in hardware and software configurations can significantly affect the performance of on-device models, particularly on the fragmented Android platform. IPhones generally have similar hardware and software configurations, but Android devices are plagued with fragmentation issues, causing compatibility concerns for app developers across various Android devices. Certain Android devices with suboptimal hardware conditions may severely impact the efficiency of on-device models, rendering them unsuitable for use. In such scenarios, on-cloud models are a more viable option, given that they have lower hardware and software requirements for phones and are better suited for the heavily fragmented Android platform. Additionally, while on-cloud models may have network requirements, they can be more cost-effective in terms of engineering efforts and deployment, thus making them a more appealing choice for some Android developers.
For example, as shown in Table~\ref{tab:notuse}, \emph{Homebase Employee Scheduling} and \emph{Postmates - Food Delivery} use on-cloud models provided by ML Kit to detect objects in Android apps to replace on-device models \emph{SSDOcr.mlmodelc} and \emph{DocScanCSCModel.tflite} in iOS apps.

24 (10.62\%) out of 226 models in the Android app received extra protections, including weights compression and model obfuscation.
It is tough for us to obtain extra details on the obscured model.
Android apps are widely perceived as being less secure than iOS apps.
Therefore, developers have increased security measures for Android apps.
To prevent the model from being readily taken or attacked,
\emph{EnhanceFox - AI Photo Enhancer} in Table~\ref{tab:notuse} obfuscates the DL models in Android apps, that could have a positive impact on user privacy and corporate assets.

10 (4.42\%) out of 226 models cannot currently be identified for replacement in Android apps.

\subsubsection{Suggestions}
\label{sec:suggestions}
We offer suggestions for both app developers and DL framework providers, which may support future development in this field.

To retain current DL framework users and appeal to a broader user base, framework providers should optimize the framework in two areas. 
First, to encourage the use of deep learning frameworks in multiple platforms, it is essential to make the cross-platform migration of the model more convenient and concise, while improving the cross-platform compatibility of the model. 
Our analysis reveals that despite the claims of deep learning frameworks to support both Android and iOS platforms, only a small percentage of models, i.e., 17.63\% out of 1,883 models, are shared between the two platforms.
Currently, the migration of a deep learning model from Android to iOS platforms involves converting the model from its original format to a format compatible with the iOS platform, while ensuring that the functionality and accuracy of the model remain unchanged. This conversion process may require modifications to be made to the model's architecture, code, and parameters, while considering the differences in hardware and software configurations between the two platforms.
Developers may find the engineering efforts involved in using the current framework's cross-platform migration model approach relatively high, as compared to using a more convenient approach specific to the current platform to achieve the same deep learning functionality, such as the on-cloud model. Therefore, the proportion of current developers reusing models between Android and iOS is not high.
Second, the ease of use of the framework should be improved by providing a comparable and accessible solution for utilizing deep learning functionalities on the iOS platform as on the Android platform, along with robust cross-platform support.
Deep learning features, such as text recognition, QR code recognition, face recognition, and object recognition, are more accessible to developers in Android apps compared to iOS. 
Due to cost-saving advantages, Android developers are more inclined to use pre-existing deep learning techniques provided by frameworks rather than reuse their own models.
In cases where on-device models are reused between the two platforms, they are often customized by app developers to provide tailored functionalities, which cannot be provided by DL frameworks.
In conclusion, cross-platform compatibility and ease of use should become the key direction of future framework optimization.

From the developer's perspective, it is essential to understand that differences in hardware and software between Android and iOS platforms can significantly affect the accuracy and efficiency of the model. Furthermore, the availability of specific hardware components, such as GPUs or accelerometers, can vary between the two platforms, influencing model selection and usage. Therefore, developers should carefully consider these factors and choose a more suitable framework when developing mirror apps on different platforms.
Moreover, developers should prioritize the cross-platform compatibility of app-customized models by pre-defining cross-platform interfaces and security safeguards for these models. This can mitigate the engineering efforts and security concerns associated with developing mirror apps for other platforms in the future. In summary, selecting a more appropriate DL framework for the app's target users and designing interfaces and security for custom models in advance can ease the developer's engineering efforts and better maintain the consistency of one app on different platforms.


 

% TF Lite and Core ML are deep learning frameworks provided by Google and Apple respectively, and are optimized for their respective platforms, Android and iOS, with better platform compatibility compared to other frameworks~\cite{tfLite, coreML}. 
% The differences in platform compatibility may affect the efficiency of the model~\cite{CoreMLConvert}. 




\begin{comment}
In conclusion, current developers reuse app-customized models due to the significant engineering cost required to train a model.
For the remaining models, current developers prefer selecting a more convenient DL framework or cloud model, or even removing them directly.
% considering the cost of development, platform compatibility, security and the importance of the model's corresponding functionality, current developers always choose a more convenient DL framework or cloud model instead, or even remove it directly.
Therefore, while developing apps, developers should prioritize the cross-platform compatibility of the app-customized models. 
By pre-defining cross-platform interfaces and security safeguards for these models, developers may alleviate the engineering effort and security concerns associated with developing mirror apps for other platforms in the future.
\end{comment}

\begin{summary}[]
Due to the different ecosystems of the Android and iOS platforms, only a tiny number of Android apps fully reuse the on-device models of iOS apps. 
% \chen{So developers have on-device models on iOS apps and then develop Android apps? How do you know that? By analyzing the app development history?}
25.66\% of iOS apps' on-device models are replaced in Android by selecting more convenient models.
17.63\% of on-device models, which provide customized functionalities for apps, are shared by apps across Android and iOS.
% Existing app developers will pick a more convenient \chen{Compared with what?} alternative to these on-device models on the present platform for on-device models that do not impact the app's fundamental functionality.
Existing DL framework vendors should simplify developers' use of fundamental DL models and improve cross-platform interoperability and security for developers' on-device models to acquire market share.
\end{summary}


% Developers are gradually building their own DL solutions, or choosing more convenient DL solutions provided by Google and Apple, so the market for other third-party DL frameworks is gradually decreasing.
% Our findings also 

\section{RQ3: How robust are on-device models on iOS against adversarial attacks?}

\subsection{Motivation}
RQ1 demonstrates that on-device models are widely used and serve fundamental roles in iOS apps.
We discover in RQ2 that while most apps will not utilise the exact same on-device model for iOS and Android, the model that serves as the app's basic functionality will be shared between the two platforms. 
Most of the shared models are exposed directly in the source files, and only a very few developers choose to provide additional protection to the models.
Shared models include the white-box model and the gray-box model.
% Considering the widespread use of the on-device model on the iOS platform, it is necessary to evaluate the security of the on-device model on the iOS platform.

White-box on-device models on Android are proved to be vulnerable to adversarial attacks~\cite{huang2021robustness, huang2022smart, karim2020adversarial}.
Due to the closed-source ecosystem and the inability of current methods to attack the gray-box model~\cite{vivek2018gray, ebrahimi2017hotflip}, attacking iOS apps is considered more challenging than attacking Android apps~\cite{garg2021comparative, dehling2015exploring, fredrikson2015model}, resulting in the lack of methods that focus on exploiting iOS-specific on-device models.
However, some studies have demonstrated that there are concurrent cross-platform issues in iOS and Android apps~\cite{ahmad2013comparison, gronli2014mobile, aljedaani2019comparison, garg2021comparative}.
Regardless of the fact that investigating iOS security issues is regarded as being more challenging, given the widespread use of on-device models on iOS, we study how robust these models are against adversarial attacks.

In this RQ, we first present our methodology of attacking on-device models in Section~\ref{sec:attackPipe}, including white-box and iOS-specific gray-box models.
Then, we carry out experiments to evaluate and compare the effectiveness of our approach in Section~\ref{sec:atkEva}.
% We also perform experiments to evaluate the performance of our proposed method in attacking iOS-specific on-device models.
Finally, we successfully attacked the relevant functionality in real-world iOS apps by taking advantage of the flaws of the on-device models in Section~\ref{sec:realworld}.

\subsection{Methodology of Model Attacking}
\label{sec:attackPipe}

% Figure environment removed

According to the empirical study in Section~\ref{sec:frameCompare}, there are two categories of on-device models inside current iOS apps:
(1) White-box Model: both the trained network weights and model structure can be extracted, for example, TF Lite and TensorFlow models; (2) Gray-box Model: only the model structure can be obtained (Core ML models).
Figure~\ref{fig:rq3} shows the methodology of our model attacking.
Regarding white-box models, we propose a more general attack approach for models of TF Lite and TensorFlow in this study.
Current approaches perform poorly when attacking gray-box on-device models of Core ML~\cite{huang2021robustness, huang2022smart, karim2020adversarial, rauber2020fast}.
Regarding gray-box models, we use the Core ML models' white-box counterpart as a bridge to propose an more efficient pipeline for attacking Core ML models.
Our approach is also applicable to white-box and gray-box models that do not fit into these three frameworks.


\begin{comment}
Different from current attack approaches~\cite{huang2022smart, karim2020adversarial, huang2021robustness, kurakin2018adversarial} for white-box on-device models of TF Lite and TensorFlow, we propose a new attack approach that can attack any models of these two frameworks, not just fine-tuning models.
% Models of TF Lite and TensorFlow have been shown to be vulnerable to adversarial attacks.
Unlike models of TF Lite and TensorFlow frameworks, the models of the iOS platform-specific Core ML framework are gray-box models, and third parties can only extract the structure of the model without trained weights.
% Based on our discovery in Section~\ref{sec:frameCompare} that apps reuse some on-device models in both Android and iOS, we propose our attack pipeline for Core ML models. 
First, we locate the white-box counterpart of the Core ML model in the corresponding App.
Second, we produce adversarial examples by attacking the white-box counterpart, and then use these adversarial examples to directly attack the iOS app's gray-box model.
\end{comment}

% Existing on-device models inside iOS can be divided into two categories:
% (1) White Box Model: both the trained network weights and model structure can be extracted (TF Lite and TensorFlow models); (2) Grey Box Model: only the model structure can be obtained (Core ML models).


% \chen{I do not see the relation between white box and grey box, and why do you mention that. Please give an overview of the whole attack pipeline in the first paragraph.}

\subsubsection{How to Attack White-box Models?}
\label{sec:attackWhite}
The attacking approaches proposed by Huang et al.~\cite{huang2021robustness, huang2022smart} require identifying the on-device model's pre-trained model for producing adversarial examples, and then using these adversarial examples to attack the on-device model.
Such approaches can only be used for fine-tuning models and requires the identification of the corresponding pre-trained model.
The fundamental reason why TF-based models cannot be directly attacked is that it is hard to continue training the models in these formats and unable to update the parameters in the model by backpropagation.
Therefore, Huang et al.~\cite{huang2021robustness, huang2022smart} choose to first identify the pre-trained model from TensorFlowHub~\cite{tfhub} which could perform backpropagation to generate adversarial examples.

Our approach does not rely on pre-trained models to generate adversarial examples and also supports direct attacks on models without fine-tuning.
Step 2 (\emph{Convert TF-based models to trainable model}) in Figure~\ref{fig:rq3} shows the pipeline of this approach.
In our approach, given an on-device model, we first retrieve its parameters and structure information and then use them to reproduce the model in a trainable format, such as the \emph{.pt} from PyTorch.
Expressly, we feed the extracted parameters to our reconstructed model structure layer by layer.
After converting to a trainable model, as shown in step 3 (\emph{Attack trainable model}) in Figure~\ref{fig:rq3}, we employ adversarial attacks on the reproduced trainable model to generate adversarial examples, which will be used to attack the original on-device model in the corresponding app.
% We also employ this attack approach against white-box on-device models in Android apps \yujin{This sentence is not clear to me}.
Figure~\ref{fig:conversionExp} shows a concrete example of converting an on-device model into a trainable model format, namely\emph{pt} of PyTorch.
We first reproduce all the neural network structures of this model. Then we load the trained weight dictionary (\emph{parameter\_dict}) of the original on-device model, and feed the weights of the original model for our reproduced network structure according to the shape and name of each network layer. Finally, we save the replicated model in trainable format (\emph{Model.pt}).
We generate adversarial examples based on the training of our replicated trainable models (\emph{Model.pt}). 
we then use these adversarial examples to attack the original on-device model (\emph{Model.tflite}).

% Figure environment removed


\subsubsection{How to Attack Gray-box Models?}
On-device models of the Core ML framework do not share trained parameters with third parties and may compress or encrypt model parameters~\cite{modelEncry}, resulting in the inability to retrieve the parameters of the trained model.
This type of gray-box model is tough to be successfully attacked by current approaches without knowing trained parameters.
% Core ML framework encourages developers to convert other frameworks' trained on-device models to Core ML framework format for use in iOS apps~\cite{CoreMLConvert}.
Steps 1, 2, 3 and 4 in Figure~\ref{fig:rq3} demonstrate our attack pipeline for gray-box models.
In Section~\ref{sec:modelChange}, we find that some iOS and Android apps share on-device models and that some on-device models of Core ML framework are converted from models in Android apps~\cite{CoreMLConvert}.
Therefore, following steps in Section~\ref{sec:ps2}, we first locate the matching on-device model before conversion in the corresponding Android app (Step 1 in Figure~\ref{fig:rq3}).
We retrieve the model's parameters and reconstruct it in a trainable format (Step 2 in Figure~\ref{fig:rq3}). 
Then, we attack the reconstructed model of its counterpart to generate adversarial examples for targeting the original on-device model of Core ML (Step 3 and 4 in Figure~\ref{fig:rq3}).


\subsection{Evaluation of Attacking}
\label{sec:atkEva}
The effectiveness of our proposed approach to attack the white-box model is evaluated first, followed by the effectiveness of our proposed method to attack the gray-box model.



\subsubsection{Dataset}
\begin{table*}[htbp]
  	\vspace{-0.8cm}
\setlength{\abovecaptionskip}{0pt} 
\setlength{\belowcaptionskip}{10pt}
  \centering
  \caption{Details of the selected 10 models}
  \begin{adjustbox}{width=\textwidth,center}
    \begin{tabular}{|c|c|c|c|c|}
    \noalign{\hrule height 1pt}
    \textbf{ID} & \textbf{App Name} &\textbf{Model in iOS} & \textbf{Model in Android} & \textbf{Model Function} \\
    \noalign{\hrule height 1pt}
    1     & hp smart & doc\_classification.mlmodelc & doc\_classification.tflite & document classification \\
    2     & paypal - send, shop, manage & ObamModel.mlmodelc & obamModel.tflite & image classification \\
    3     & merlin bird id by cornell lab & geo\_v18.mlmodelc & geo\_v17.tflite & bird classification   \\
    4     & seek by inaturalist  & optimized\_model.mlmodelc & optimized\_model.tflite & identify plants and animals in pictures\\
    5     & smart bird id & NABirdsImageClassifier.mlmodelc & BirdImageClassifier.tflite & identify birds in pictures\\
    6     & sticker maker studio & deeplabv3\_mnv2\_pascal\_trainval.mlmodelc & deeplabv3\_mnv2\_pascal\_trainval.tflite & image segmentation \\
    7     & scentbird & FindFour.mlmodelc & findfour.tflite & detect box in pictures \\
    8     & scentbird & FourRecognize.mlmodelc & fourrecognize.tflite & detect box in pictures\\
    9     & gradient: celebrity look alike & gender\_nn.mlmodelc & gender\_nn.tflite & identify gender in pictures\\
    10    & Hoop - Make new friends & Nudity.mlmodelc & optimized\_nudity\_graph.tflite & identify nudity in pictures\\
    \noalign{\hrule height 1pt}
    \end{tabular}%
    \end{adjustbox}
  \label{tab:dataset}%
  	\vspace{-0.2cm}
\end{table*}%





% 

\begin{table*}[htbp]
  \centering
  \caption{Attack Results}
    \begin{adjustbox}{width=0.7\textwidth,center}
\begin{tabular}{|cc|cccccccccc|c|} 
\noalign{\hrule height 1pt}
Attack                      & Epsilon & M1   & M2   & M3   & M4   & M5   & M6   & M7   & M8   & M9   & M10  & Average  \\ 
\noalign{\hrule height 1pt}
Pointwise Attack            & 3.5     & 0.6  & 0.6  & 0.5  & 0.7  & 0.7  & 0.7  & 0.8  & 0.6  & 0.8  & 0.7  & 0.67     \\
Boundary Attack             & 5.5     & 0.5  & 0.7  & 0.8  & 0.7  & 0.6  & 0.8  & 0.5  & 0.6  & 0.8  & 0.8  & 0.68     \\
DDN Attack                  & 0.8     & 0.8  & 0.4  & 0.6  & 0.7  & 0.6  & 0.7  & 0.7  & 0.7  & 0.8  & 0.5  & 0.65     \\
DeepFool Attack             & 1.7     & 0.5  & 0.6  & 0.6  & 0.9  & 0.8  & 0.8  & 0.4  & 0.7  & 0.9  & 0.5  & 0.67     \\
FGSM                        & 0.02    & 0.6  & 0.7  & 0.8  & 0.7  & 0.7  & 0.6  & 0.5  & 0.7  & 0.8  & 0.7  & 0.68     \\
BIM                         & 2.5     & 0.6  & 0.6  & 0.9  & 0.5  & 0.2  & 0.8  & 0.7  & 0.8  & 0.9  & 0.6  & 0.66     \\
PGD                         & 8       & 0.5  & 0.5  & 0.7  & 0.8  & 0.5  & 0.5  & 0.7  & 0.8  & 0.6  & 0.7  & 0.63     \\
C\&W Attack                 & 0.6     & 0.7  & 0.6  & 0.5  & 0.8  & 0.5  & 0.7  & 0.5  & 0.6  & 0.9  & 0.6  & 0.64     \\
Newton Fool Attack          & 9.5     & 0.6  & 0.7  & 0.7  & 0.7  & 0.8  & 0.7  & 0.6  & 0.7  & 0.8  & 0.7  & 0.70     \\
Clipping-Aware Noise Attack & 20      & 0.8  & 0.9  & 0.9  & 0.8  & 0.9  & 0.8  & 0.8  & 0.9  & 0.8  & 0.9  & \textbf{0.85}     \\ 
\noalign{\hrule height 1pt}
Average (Models)            &         & 0.62 & 0.63 & 0.70 & 0.73 & 0.73 & 0.71 & 0.62 & 0.71 & \textbf{0.81} & 0.67 & 0.68     \\
\noalign{\hrule height 1pt}
\end{tabular}
    \end{adjustbox}
  \label{tab:results}%
\end{table*}%
We evaluate the performance of our proposed attack pipeline in real-world industrial iOS apps from the Apple App Store.
We select 10 representative on-device models of the Core ML framework for testing our approach.
These 10 models all have clear usage scenarios in iOS apps and serve core functions.
Table~\ref{tab:dataset} shows the detail of selected 10 on-device models.
The column \emph{App Name} represents the apps to which the on-device models belong. 
The column \emph{Model in iOS} represents the gray-box model in iOS apps.
The column \emph{Model in Android} represents gray-box models' located white-box counterparts in the Android app.
The column \emph{Model Function} represents the identified model function in apps.
In our present analysis of commonly utilized apps, we have only identified the reuse of image-based on-device models between Android and iOS platforms. 
Therefore, the selected models are all image-based.
However, the advent of large language models (LLMs) suggests the potential for expansion of this phenomenon to other domain models such as natural language processing (NLP), speech recognition, and more. 
As of now, these models are not yet prevalent in cross-platform reuse. 
Nevertheless, our methodology exhibits a high degree of flexibility and adaptability, making it well-suited to accommodate such future developments. 
We will direct our research efforts toward exploring the security implications of these emergent domains and their corresponding large models in the future.

\subsubsection{Baseline}
To demonstrate the effectiveness of our newly proposed method to attack the white-box model, we use the ModelAttacker proposed by Huang et al.~\cite{huang2021robustness, huang2022smart} as the baseline to carry out a control experiment.
ModelAttacker hacks DL models using adversarial attacks by first identifying highly similar pre-trained models from TensorFlow Hub~\cite{tensorflowHub}.
Then, it attacks the identified pre-train model to produce adversarial examples for the original model.


\subsubsection{Experimental Setup}
% Following the pipeline in Section~\ref{sec:ps2} to match pair on-device models of the Core ML framework, 
We first identify Android white-box counterparts of Core ML models (as shown in column \emph{Model in Android}).
Second, following steps in Section~\ref{sec:attackWhite}, we convert these on-device models into trainable models (\emph{.pt}).
We find 10 random images for each model in Table~\ref{tab:dataset} by referring to their identified functionalities as the original inputs~\cite{huang2021robustness, huang2022smart}.
To ensure the validity of the input images selected for our experiments, we employ a two-step collection process. 
In the first step, we identify target images matching the function of the model and the app's specific usage scenario. 
For instance, for model 5 (\emph{NABirdsImageClassifier.mlmodelc}) in Table~\ref{tab:dataset}, which is used for bird identification, we selected various bird images as input for the model. 
In order to ensure the diversity of the input test set, we will deliberately select images that encompass distinct object categories, backgrounds, and lighting conditions. This approach will ensure that our test set is a true reflection of the variability encountered in real-world images, and will help to establish the efficacy of our method in practical settings.
In the second step, we manually input the selected experimental images directly into the corresponding application scenes of the model in the app to verify the model's correct recognition of these images. Following this process, we obtained a set of valid input images. 

In this study, we follow the approach of Huang et al.~\cite{huang2022smart, huang2021robustness} and randomly select 20 images for each on-device model. To ensure the diversity of the selected images, we partition the 20 images into two subsets such that each subset consists of dissimilar images.
To quantitatively evaluate the diversity of the new dataset, we use Maximum Mean Discrepancy (MMD)~\cite{borgwardt2006integrating} as a metric. MMD is a measure of the discrepancy between two probability distributions, frequently used in machine learning to compare the similarity between datasets. MMD compares the means of the two datasets in a reproducing kernel Hilbert space (RKHS) and quantifies their differences. Higher MMD values indicate greater dissimilarity between two datasets or samples.
Based on prior research~\cite{dziugaite2015training, yan2017mind}, we require that the MMD values of the two datasets exceed 0.6 to ensure their diversity. If the MMD values fall below this threshold, we re-collect images until the requirements are satisfied. 


Finally, we use selected images as input to attack the converted trainable models to generate adversarial examples and use generated adversarial examples to evaluate if they will be misclassified by original on-device models on Android and iOS.

In the control experiment, we first find the most similar pre-train models from TensorFlowHub for each model in the column \emph{Model in Android} by following the experiment setup of the baseline.
Then, we use selected 10 images as input to attack the pre-train models to generate adversarial examples.
Finally, we use the generated adversarial examples to attack the original iOS on-device models.
The baseline requires locating pre-trained models with greater than 80\% structural similarity~\cite{huang2021robustness}.
However, we cannot find such models on TensorFlowHub for model 9 (\emph{gender\_nn.tflite}) and model 10 (\emph{optimized\_nudity\_graph.tflite}).
So when comparing the attack success rate of the baseline, we only consider the success rate of  models 1 to 8.

\subsubsection{Evaluation Metrics}
Successful adversarial examples must meet two requirements:  (1) make the model misclassify (2) changes made in the original image of the input cannot be noticed by humans.
% After successfully causing models to misclassify adversarial examples and
Following the same evaluation steps with related works~\cite{huang2021robustness, huang2022smart, karim2020adversarial}, we invite three PhD students with experience in adversarial attacks to manually evaluate whether the modifications made to adversarial example images are too subtle to be noticed.
The attack is deemed successful if the three volunteers think the modifications to the image cannot be easily recognised.
Otherwise, the attack is considered failed.
we count the number of examples that successfully misclassify the model among these 10 input adversarial examples and count the success rate of different types of adversarial attacks.

\subsubsection{Adversarial Attack Algorithms}

The whole process of adversarial attack can be summarized as
\begin{equation}
    x^{'}_{adv} = x + \epsilon * attack_i(\nabla J (\theta,\ x,\ y_{true}))
\end{equation}
where $y_{true}$ represents the original label or class of input image $x$, $\epsilon$ represents a multiplier to ensure the perturbations between input images and adversarial images are small and its value is empirically determined in the experiments to produce unnoticeable image perturbations, $attack_i$ represents the $i_{th}$ type of adversarial attack, $\theta$ represents parameters, and $J$ is the loss.

We select 10 representative attacks: Pointwise (PW) Attack~\cite{yang2019adversarial}, Boundary (BD) Attack~\cite{croce2020minimally}, DDN Attack~\cite{rony2019decoupling}, DeepFool (DF) Attack~\cite{moosavi2016deepfool}, FGSM Attack~\cite{goodfellow2014explaining}, BIM~\cite{kurakin2018adversarial}, PGD~\cite{he2019towards}, C\&W Attack~\cite{carlini2017towards}, Newton Fool (NF) Attack~\cite{jang2017objective} and Clipping-Aware Noise (CAN) Attack~\cite{rauber2020fast} as all these attacks have been proved effective in many tasks and widely-used to evaluate the robustness of on-device models~\cite{huang2021robustness, huang2022smart, karim2020adversarial}.


\subsubsection{Results}
\begin{table*}[htbp]
\vspace{-0.8cm}
\setlength{\abovecaptionskip}{0pt} 
\setlength{\belowcaptionskip}{10pt}
  \centering
  \caption{Attack results of Core ML models on iOS and their counterparts on Android}
    \begin{adjustbox}{width=\textwidth, center}
    \begin{tabular}{|cc|ccc|ccc|ccc|ccc|ccc|ccc|ccc|ccc|ccc|ccc|ccc|}
    \noalign{\hrule height 1pt}
    \textbf{ATK} & \textbf{Ep} & \multicolumn{3}{c|}{\textbf{M1}} & \multicolumn{3}{c|}{\textbf{M2}} & \multicolumn{3}{c|}{\textbf{M3}} & \multicolumn{3}{c|}{\textbf{M4}} & \multicolumn{3}{c|}{\textbf{M5}} & \multicolumn{3}{c|}{\textbf{M6}} & \multicolumn{3}{c|}{\textbf{M7}} & \multicolumn{3}{c|}{\textbf{M8}} & \multicolumn{3}{c|}{\textbf{M9}} & \multicolumn{3}{c|}{\textbf{M10}} & \multicolumn{3}{c|}{\textbf{Ave (ATKs)}} \\
    \noalign{\hrule height 1pt}
     & & \textbf{M} & \textbf{A} & \textbf{I} &  \textbf{M} &\textbf{A} & \textbf{I} &  \textbf{M}&  \textbf{A} & \textbf{I} & \textbf{M}&  \textbf{A} & \textbf{I} & \textbf{M}&  \textbf{A} & \textbf{I} & \textbf{M}&  \textbf{A} & \textbf{I} & \textbf{M}&  \textbf{A} & \textbf{I} & \textbf{M}&  \textbf{A} & \textbf{I} & \textbf{M}&  \textbf{A} & \textbf{I} & \textbf{M}&  \textbf{A} & \textbf{I} & \textbf{M}&  \textbf{A} & \textbf{I} \\
    
    \hline
    
    PW & 3.5  & 0.2  & 0.6  & {0.5}   & 0.4  & {0.7}  & {0.7}   & 0.3  & {0.6}  & {0.6}  & 0.5   & 0.7  & 0.7   & 0.6  & 0.8  & 0.7   & 0.4  & {0.8}  & {0.8}    & 0.6 & 0.8  & 0.8    & 0.3 & 0.6  & 0.6    & - & 0.8  & {0.7}    & - & 0.8  & 0.7    & 0.41 & {0.72}  & {0.68} \\
    
    BD & 5.5    & 0.3 & {0.6}  & {0.6}   & {0.4} & 0.7  & 0.7   & 0.4 & 0.8  & 0.8   & 0.5 & 0.7  & 0.7   & 0.2 & {0.7}  & {0.7}  & 0.6 & 0.8  & 0.8   & 0.3 & {0.6}  & 0.5   & 0.3 & {0.5}  & {0.5}  & - & 0.8  & 0.8   & - & 0.8  & 0.8   & {0.28} & {0.70}  & {0.69}\\
    
    DDN & 0.8  & 0.4 & 0.8 & {0.7}  & 0.1 & {0.6} & {0.6}    & 0.2 & 0.7 & 0.6    & 0.2 & 0.7 & 0.7   & 0.3 & 0.6 & 0.6   & 0.3 & {0.8} & {0.8}  & 0.2 & 0.7 & 0.7   & {0.2} & 0.7 & 0.7    & - & 0.8 & 0.8   & - & 0.7 & 0.5  & {0.24} & {0.71}  & {0.67}\\
    
    DF & 1.7  & 0.1 & {0.6} & {0.6}  & 0.2 & 0.6 & 0.6    & 0.4 & {0.7} & {0.7}    & 0.3 & {0.8} & {0.8}   & 0.3 & 0.8 & 0.8   & 0.2 & {0.7} & {0.7}  & 0.3 & {0.5} & {0.5}    & 0.3 & 0.7 & 0.7    & - & 0.9 & 0.9   & - & {0.6} & {0.6}  & 0.26 & {0.69}  & {0.69} \\
    
    FGSM  & 0.02 & 0.3 & 0.6 & 0.6  & 0.4 & {0.6} & {0.6}   & 0.5 & 0.8 & 0.8    & 0.3 & 0.7 & 0.7   & 0.4 & 0.7 & 0.7   & 0.2 & {0.8} & {0.8}  & 0.3 & {0.7} & {0.7}    & 0.3 & 0.7  & 0.7    & - & 0.8 & 0.8   & - & 0.8 & 0.7    & 0.34 & {0.72}  & {0.71}\\
    
    BIM & 2.5    & 0.3 & {0.7} & {0.7}   & 0.2 & {0.7} & {0.7}    & 0.5 & 0.9 & 0.9    & 0.2 & {0.7} & {0.7}   & 0.1 & {0.6} & {0.6}   & 0.6 & {0.7} & {0.7}  & 0.3 & 0.7 & 0.7    & 0.4 & {0.7} & {0.7}    & - & 0.9 & 0.9   & - & 0.8 & 0.6   & 0.35 & {0.70}  & {0.68}\\
    
    PGD & 8   & 0.3 & {0.6} & {0.5}  & 0.2 & {0.6} & {0.5}    & 0.6 & 0.9 & 0.8    & 0.5 & {0.7} & {0.7}    & 0.2 & {0.6} & 0.5   & 0.1 & 0.5 & 0.5  & 0.3 & 0.7 & 0.7    & 0.4 & 0.8 & 0.8    & - & {0.7} & {0.7}   & - & 0.7 & 0.7  & 0.33 & {0.68}  & {0.67}\\
    
    C\&W & 0.2 & 0.5 & 0.7 & 0.7  & 0.2 & 0.6 & 0.6    & 0.1 & {0.8} & {0.8}    & 0.6 & 0.8 & 0.8   & 0.4 & {0.8} & {0.7}   & 0.4 & 0.7 & 0.7  & 0.2 & 0.5 & 0.5    & 0.2 & 0.6 & 0.6    & - & 0.9 & 0.9   & - & {0.7} & {0.7}    & 0.36 & {0.71}  & {0.70}\\

    NF & 9.5   & 0.4 & {0.7} & {0.7} &   0.3 & 0.7 & 0.7    & 0.3 & 0.7 & 0.7    & 0.4 & 0.7 & 0.6  & 0.5 & {0.8} & 0.8   & 0.4 & 0.7 & {0.6}  & 0.2 & {0.7} & {0.7}    & 0.3 & {0.6} & {0.6}    & - & 0.8 & 0.8   & - & 0.7 & 0.7 &   0.35 & {0.71}  & {0.69}\\
    
    CAN & 20   & 0.5 & 0.8 & 0.8  & 0.4 & 0.9 & 0.9    & {0.5} & 0.9 & 0.9    & 0.4 & 0.8 & 0.8   & 0.6 & 0.9 & 0.9    & 0.5 & 0.8 & {0.7}  & 0.5 & 0.8 & 0.8    & 0.6 & 0.9 & 0.9    & - & 0.8 & 0.8   & - & 0.9 & 0.9 & {0.49} & \textbf{0.85}  & {\textbf{0.84}}\\
    \noalign{\hrule height 1pt}
    Ave &   ~   & 0.33 & {0.67} & {0.64}  & {0.28} & {0.67} & {0.66}    & {0.38} & {0.78} & {0.76}    & 0.39 & {0.73} & {0.72}   & 0.36 & {0.73} & {0.70}  & 0.37 & {0.73} & {0.72}  & 0.34 & {0.67} & {0.66}     & {0.33} & {0.68} & {0.68}   & - & \textbf{0.82} & {\textbf{0.81}}   & - & {0.76} & {0.70} & {0.34} & {\textbf{0.72}}  & {0.70} \\
    \noalign{\hrule height 1pt}
    \end{tabular}%
    \end{adjustbox}
  \label{tab:results3}%
  \vspace{-0.3cm}
\end{table*}%


Table~\ref{tab:results3} shows the adversarial attack results in the dataset.
The column \emph{ATK} shows the type of adversarial attacks.
The column \emph{Ep} denotes the empirical value of the $\epsilon$ that provides the best performance of the current attack in the experiment.
The columns from \emph{M1} to \emph{M10} represent models with ids 1 to 10 in Table~\ref{tab:dataset}.
The subcolumns \emph{M} represent the attack results of the baseline ModelAttacker.
The subcolumns \emph{A} and \emph{I} represent the attack results of our approach on Android and iOS models, respectively.

As shown in the column \emph{Ave (ATKs)}, the average success rate of ModelAttacker's attack is 0.34, which is much lower than that of our approach 0.72 on white-box models and 0.70 on gray-box Core ML models.
The baseline for producing adversarial examples by attacking the corresponding pre-training model has limits and is only applicable for fine-tuning on-device models.
Our proposed attack method is demonstrably more effective and is applicable to non-pre-trained models like \emph{M9} and \emph{M10}.

As shown in the row \emph{Ave}, all of these 10 Core ML models are effectively attacked by our approach, with a success rate of at least 64\%.
The average success rate of attacks against Android equivalents is 0.72 (shown in column \emph{Ave(ATKs)}, row \emph{Ave}), which is slightly higher than the success rate of attacks against the original Core ML models, which is 0.70.
The extremely close success rate implies that gray-box Core ML models are vulnerable to adversarial examples generated by targeting their white-box counterparts, demonstrating the effectiveness of our cross-platform attack approach.

We investigate why there is a slight distinction in the success rates of attacking Android and iOS models.
There are two reasons: (1) Different frameworks may optimise the model differently. Consequently, the models may be updated slightly during the framework conversion process, resulting in subtle variations in model effects.
(2) Android app and iOS app version updates are not synchronised, so the model versions inside the app are not consistent, such as models \emph{geo\_v18.mlmodelc} and \emph{geo\_v17.tflite} in app \emph{merlin bird id by cornell lab}.
Different versions of the model may have a slightly different effect.

The model \emph{M9} (\emph{gender\_nn.mlmodelc}) is the most vulnerable with the highest average success rate of is 81\%,  since \emph{gender\_nn.mlmodelc} has the simplest structure, consisting of several residual blocks.
The Clipping-Aware Noise Attack has the most significant attack effect (85\% average success rate), whereas the success rates of the other 9 attacks are much lower (68\% - 72\%).
We find that the Clipping-Aware Noise Attack generates adversarial images with a narrower range of perturbations that are more likely to survive manual recognition and so have a higher success rate.
% The attack results demonstrate that although the current Core ML grey-box models are tough to hack, its white box counterparts can be leveraged as a bridge to attack it.

\subsubsection{Results Discussion}

Our experiments conclusively demonstrate that the black-box models present in iOS apps, which can match white-box counterparts, are inherently insecure. 
As outlined in Section~\ref{sec:ratio}, we discover that 332 (17.63\%) out of the total 1883 models are shared across both iOS and Android platforms and that these models are frequently reused in critical functionalities of the app. 
This finding highlights a significant ratio that cannot be ignored. 
Given the insecurities present in these models, both academia and industry must pay close attention to the potential unforeseen consequences and take necessary steps to prevent them.

In recent studies, researchers have analyzed the robustness of on-device models in Android apps against adversarial attacks. Our findings indicate that our proposed approach of converting the on-device model into a trainable format and generating adversarial examples through its own training is more effective than common adversarial attacks or attacking the corresponding pre-trained model before attacking the on-device model. Our approach can increase the success rate of attacks from about 45\% to 76\% when attacking white-box models.
In the work of Deng et al.~\cite{deng2022understanding} and Huang et al.~\cite{huang2021robustness}, C\&W and inversion attacks are found to be the most effective countermeasures against white-box models. However, we find that the Clipping-Aware Noise (CAN) attack, which has been recently proposed, has the highest success rate of the attack so far. Compared to C\&W and inversion attacks, CAN attack is relatively new and thus should receive more attention in terms of prevention.

In our black-box model attack experiments, we observed that compared to the NES algorithm~\cite{ilyas2018black} and ModelAttacker, our approach demonstrates a higher success rate in attacking partial black-box models with potential security threats.
Despite the comparatively closed and secure nature of the iOS system, we have shown that on-device models in Android apps are vulnerable to similar threats as those present in iOS apps. To mitigate this issue, app developers should apply the same model protection measures used in the Android platform, such as model obfuscation and encryption, to models in the iOS platform.
Furthermore, DL framework providers could enhance the security of models and attract more users by providing a cross-platform security mechanism for models that use their framework.

In above related studies and our work, it is worth noting that the on-device models examined in our experiments primarily focus on image-related domains and hardly cover other domains, such as natural language processing (NLP) and speech recognition. This is due to the fact that on-device models related to computer vision are currently the most prevalent type of DL models utilized by real-world applications, as discussed in Section~\ref{sec:modelFunction}. 
The models identified in our experiments that are reused in both platforms are primarily image-related. 
From the perspective of attack methods, investigating a broader range of model types would provide a more comprehensive understanding of how different types of models perform in the face of adversarial attacks. Thus, in future work, we plan to investigate the security vulnerabilities of all types of models against adversarial attacks.


% \textcolor{red}{xxx} show the physical attack results of physical attacking the app \emph{seek by inaturalist}.
% The input modified image can also cause the app to misclassify the species of the bird. 

\subsubsection{Examples of Attacking Real-world iOS Apps}
\label{sec:realworld}
After locating the usage scenarios within iOS apps, we input the generated adversarial examples to the iOS apps to attack real-world iOS apps, as illustrated in the step 5 of Figure~\ref{fig:rq3}.
Figure~\ref{fig:phAt1} shows the attack results of the app \emph{gradient: celebrity look alike}.
The adversarial example image directly cause the app to misclassify the gender of the image from male to female.
Figure~\ref{fig:phAt2} shows the attack results of the app \emph{smart bird id}.
The adversarial example image cause the app to misclassify the bird type of the image from \emph{Australia Magpie} to \emph{Satin Bowerbird}.

\input{tables/fig_realworldAttack}

% % Figure environment removed

% % Figure environment removed



\begin{summary}
% Most on-device models, including models of Core ML, TF Lite, and TensorFlow in current iOS apps, are vulnerable to adversarial attacks.
Due to the reuse of models between platforms and apps, gray-box models of iOS-specific Core ML are also vulnerable to adversarial attacks via our approach.
The discovered vulnerabilities of on-device models inside iOS apps could disable some functionalities of real-world iOS apps.
Our findings reveal that the supposedly more secure iOS platform also has potential model security concerns, driving developers and DL frameworks to develop cross-platform model security defences against the possibility of cross-platform attacks.
\end{summary}



\section{Discussion}
% \chen{I see that there are many DL models in HuggingFace, any potential insight?}
In this section, we discuss some implications following our previous observation and experiments.
% \chen{Too many shallow analyses and please merge them to get deeper analysis. Please back up your statement with examples/statistics/conclusions from prior sections.}

\subsection{Implication on Employing and Deploying On-device Models on iOS}

\subsubsection{Optimize NLP On-device Model Deployment.} In RQ1, most on-device models perform CV-related tasks, while the NLP-related on-device models are less common. This can be attributed to: (1) traditional/classical NLP models like LSTM and RNN are less efficient than Large Language Models (LLMs), however the development of LLMs for edge devices is hard. (2) NLP models inevitably come with dictionaries/word embedding layers, which are heavy and not easy to optimize. (3) Employing NLP models on edge devices should address the language challenge as a DL app may provide services to people from different countries.
Hence, this can be a future research direction as NLP is another crucial part for on-device deep learning models.

\subsubsection{Optimize Between-app On-device Model Development}
For developing a DL app in iOS and Android platform, DL framework vendors should optimize their products to facilitate the cross-platform on-device model development. Based on our findings in RQ2, potential solutions can be: (1) Implementing platform-specific optimisation to ensure that a well-trained model can be adapted for proprietary use on several platforms.
(2) offering unified APIs makes the development of on-device models for different platforms indistinguishable. (3) enhancing framework compatibility for various platforms.

\begin{comment}
\subsubsection{Differentiate between the app's essential and non-essential functionalities}
Before developing or migrating an app, it is necessary for developers to explicitly identify between the core and non-core functionalities \chen{??}.
With the awareness of the app's essential and non-essential functionalities, developers can choose a more lightweight deep learning framework or on-cloud model for the present platform for non-essential features to avoid having too many on-device models, which can lead to bloated app sizes and even affect the app's core functionality.


\subsubsection{Convert on-device models to Core ML framework format}
Core ml is currently the most popular deep learning framework in iOS apps.
Some developers use on-device models in formats such as TF lite without converting them to the specified core ml format in iOS apps.
The on-device model of Core ML is more secure and has a lesser likelihood of being successfully attacked than other white-box models. \chen{This is the implication??}
\end{comment}


\subsubsection{Optimize In-app On-device Model Collaboration}

In RQ1, we explore \emph{How many on-device models are there in one app?} in Section~\ref{sec:appNum} and \emph{What is the size of on-device models?} in Section~\ref{sec:modelSize}.
Current DL frameworks generally correspond to a single model for a single feature, and extensive usage of deep learning in iOS apps has resulted in an excessive and bloated on-device models for some apps.
Appropriately improving the management and scheduling of models can significantly enhance the app's utilisation of models.
A vital optimization direction is to optimize the management for all on-device models in an app.

From the DL framework vendor side, many current downstream tasks are based on the same type of pretrain models~\cite{huang2021robustness, tensorflowHub}.
Frameworks can attempt to slice and dice models and extract sliced model for usage in numerous tasks. 
Frameworks dynamically combine shared pretrain and fine-tuning model slices when on-device models are used. 
This increases the security of models by making it more difficult for an attacker to locate specific models. 
On the other hand, it can reduce the number and size of models within the app and optimise model utilisation efficiency.

% \yujin{What about changing this sub-section title to "Optimize in-app on-device model collaboration"?}

\subsubsection{Prevent App Users from Locating to a Specific On-device Model using App Semantics}
We discover in RQ1 that locating the scenario of the model's usage in the app by the model's functionality is typically straightforward.
Although it is difficult to directly attack an iOS app, after discovering the model's usage scenario in the app, it is possible to use the model's vulnerability to disable the related functionality in order to further attack the app.
To prevent possible security issues, app developers should avoid leaving semantics that can be used to infer connections between models and functionalities.


\subsection{Implication for the Security of On-devices Models}

\subsubsection{Cross-platform Protection for Your Own On-device Models}
Although there are differences between the Android and iOS platforms, our research demonstrates that the same security vulnerabilities exist in the on-device model on both platforms.
In RQ3, we generated adversarial examples using the on-device model on the Android app and experimentally verified that these adversarial examples are equally valid for the Core ML model in iOS.
It is also possible to directly deactivate a portion of the iOS app's functionality.

From the app developer side, app developers should provide uniform protections when deploying on-device models in multiple platforms.
Also, the unified management of all on-device models within an app facilitates the reuse and deployment of uniform security measures for these models.
% Due to the high cost of training a model, DL model reuse has become more and more common.
% We discover in RQ2 that 19.04\% on-device models which serves as essential functionalities in apps are shared across Android and iOS.

\subsubsection{Obfuscate On-device Models}
It is difficult to attack obfuscated on-device models with current ways of attacking on-device models.
However, we found in RQ2 that the percentage of obfuscated on-device models in current iOS apps is low (10.63\%).
Current iOS app developers and deep learning frameworks can design more methods to obfuscate the model to increase the model's security.


% \subsubsection{Encrypt on-device models can be another implication for protecting on-device models.}

% \subsubsection{Adversarial train on-device models can be another implication for protecting on-device models.}

\subsubsection{Authenticate the Users of On-device Models}
Regarding DL app developers, the utilization of their on-device models can be secured by mandating user authentication and the provision of a unique token. From the perspective of model providers, it is essential to establish secure license distribution mechanisms from the server side and implement effective license management procedures on the client side prior to the utilization of their models in order to ensure security.



\section{Threats to Validity}
We discuss the threats to validity, including internal threats and external threats in the section.

\subsection{Internal Threats}

\subsubsection{False Negatives of Detecting On-device Models}

App developers employ various techniques to safeguard on-device models from theft, such as encryption and obfuscation. 
In our study, we discover that some developers may slice the models and compile them into binaries, making it more challenging for third parties to detect them by semantic pattern matching. 
This may cause our model detection method to miss some models, resulting in false negatives. 
The current limitations of dynamic code analysis tools for iOS apps also hinder a more in-depth investigation of this area.
% To mitigate this issue, instrumenting the app and dynamically detecting and analyzing the app source code can help in detecting encryption and obfuscation models. 
% However, the current limitations of dynamic code analysis tools for iOS apps hinder a more in-depth investigation of this area.
In this paper, we chose to invite experienced developers to manually search for encrypted or obfuscated on-device models in apps to mitigate this issue.
% However, due to the extensive workload involved (2,907 iOS apps), we can only invite three experienced participants in RQ2 to manually inspect source files of part of collected apps to detect possible false negatives.
% To mitigate this issue, we initially summarize the semantic patterns that the on-device model might contain and automatically detect the on-device model in all apps.
% Finally, we invite experienced developers to manually search and validate encrypted or obfuscated models in apps.

As a future direction, we plan to conduct a comprehensive analysis of the iOS app's code, including calling the model API, loading the model API, dynamically downloading the model API, and detecting the DL model from the code pattern level, to address the limitations of our current study.




\subsubsection{Bias for Model Functionality and Use Scenarios Inference}

We infer possible functions of the model in Section~\ref{sec:modelFunction} by leveraging semantic information of the app and the model. However, obtaining a comprehensive understanding of the usage of each model in an iOS app may require identifying the calling model at the code level and dynamically running the app to detect its usage. Current limitations of static and dynamic analysis tools for iOS apps hinder our ability to analyze models from the code level effectively as we do for Android apps. 
To mitigate possible inaccuracies in our functional inferences, we dynamically run the iOS app to verify our inferences. However, we cannot fully guarantee the accuracy of all inferences, resulting in potential bias in our current approach. 
In future work, we plan to explore alternative methods for analyzing the code of on-device models in iOS apps. This aims to reduce potential bias and improve the accuracy of our functional inferences.



\subsubsection{Impact of Remote Attacks on On-device Models in iOS Apps}

One potential threat to the validity of our study is the assumption that attackers have access to the targeted iOS devices and can perform the attacks locally. This may not be a realistic scenario in some cases, especially if the attack is conducted remotely.
Our study did not investigate the potential impact of remote attacks on on-device models in iOS apps, which is a limitation. Remote attacks may have different characteristics and may require different techniques than local attacks, and their effectiveness may depend on the network environment and other factors. Therefore, the results of our study may not fully reflect the security risks faced by on-device models in real-world scenarios.
However, we note that our proposed attack can still be effective in some scenarios where users need to capture inputs from the real-world, and the attacker can place adversarial inputs beforehand to cause misclassification. We also acknowledge that further research is needed to explore the impact of remote attacks on on-device models and to develop effective defenses against such attacks.


\subsubsection{Limitations on the scope of the attack method in RQ3}

Our attack method can only be applied to black-box models that have white-box counterparts, which represent 17.63\% of the 1,883 models discovered in iOS apps. We have only identified image-based models in real-world iOS apps, and lack experiments with models from other domains such as natural language processing and speech recognition. While our primary objective is to assess the robustness of current on-device models against adversarial attacks, our focus on image-based models limits the generalizability of our findings. 
It's worth noting that our proposed attack method can be easily migrated to other types of models, such as NLP and speech recognition. 
To address these limitations, we plan to design and analyze the effectiveness of our approach on more types of models algorithmically in future work. To expand the range of black-box models that we can attack, we plan to try to migrate the adversarial example to attack black-box models with similar structures and parameters.



\subsection{External Threats}
\subsubsection{Dynamic Analysis of On-device Models}
In this study, we primarily use static analysis to examine on-device models and code in Android and iOS apps. Although useful, static analysis cannot capture the state of on-device models after the app is launched. Dynamic analysis, on the other hand, can provide more in-depth insights into the current state of on-device models, including memory usage, call time analysis, and dynamic data security. Additionally, it can help collect more on-device models for obfuscated or encrypted models by analyzing the code of the app loading the model and dumping the model from the memory where the app is running. However, current dynamic analysis tools for iOS apps have limitations that prevent effective analysis of on-device models. As a result, we invite three experienced participants to manually analyze the running app to identify the app's functions and use scenarios that might call the on-device model. However, this approach proved less effective than direct dynamic analysis.

Therefore, we think that the development of better dynamic analysis tools for iOS apps is crucial for future research. These tools will enable more effective analysis of on-device models and facilitate further investigations on dynamic security risks, model attacks, and protection.



\subsubsection{Impact of Model Properties on Model Selection}

Model accuracy, interpretability, and explainability are important considerations when selecting on-device models. Accuracy is commonly used to evaluate model performance, while interpretability and explainability are crucial for building trust in the model and understanding how it makes predictions. However, balancing these factors can be challenging, as accuracy may come at the expense of interpretability and explainability, and vice versa. Moreover, the impact of these factors may vary across different application domains. While interpretability and explainability may be more critical in domains like healthcare and finance, accuracy may be the primary consideration in image or speech recognition. Despite their significance, the impact of these factors on developer model selection is often overlooked. 
The scope of this paper focuses on the impact of the differences between iOS and Android platforms on on-device model selection, so we have not delved into the impact of model properties on model selection in this paper. Still, it is a crucial area for future investigation.


\subsubsection{Impact of Platform Hardware and Software Configurations on Model Selection}

The differences in hardware and software configurations across platforms can influence the performance and compatibility of the on-device models, which may affect the model selection process. While this paper briefly touches upon the impact of hardware and software differences on model selection in Section~\ref{sec:suggestions} of RQ2, no detailed examination of how specific hardware and software factors affect the selection process is conducted. Thus, in future work, we aim to investigate the effects of hardware and software-specific factors on the model selection process.


\subsubsection{Impact of User Behavior and Preference on Model Selection}

% External threats to validity include the potential impact of user behavior and preference on model selection. 
Different apps have varying target users, the preferred model for one user may not necessarily be suitable for another. 
User preference may prioritize model accuracy or interpretability over computational resources or inference time. Moreover, users may have different concerns about data privacy and ethics, which can affect the selection and usage of specific models. 
Our study focuses on the perspective of app developers and DL framework providers when selecting on-device models and does not consider user behavior and preference. 
% Although the potential influences are acknowledged, this study did not analyze them in detail. 
Future research will investigate the impact of user behavior and preference on model selection and usage to provide a more comprehensive understanding of on-device model adoption.



\section{Related work on REG in context}\label{sec:litreview}

Deciding about the form of a referring expression and determining its content are two different steps of the Referring Expression Generation (REG) task \citep{comreg2019}. In the current article, our focus is on the first step, namely determining the form of a referring expression. We will expand thiw work to the second task, namelz the content realisation of the REs in future work.
 

\subsection{Different RF categories} \label{subsec:refcateories}

As \citet{kibrik2016referential} put it, the \textit{basic} and binary referential choice is between the choice of a pronoun versus a variety of NPs. Studies addressing pronominalization, such as \citet{mccoy1999generating,poesio2004centering} and \citet{henschel2000pronominalization}, often focus on this binary distinction. More recent studies 
have looked at a wider range of referring expression (RE) types. For instance, the GREC shared tasks \citep{belz2009generating} exploited four RE types, namely pronoun, proper name, common noun and covert (empty) reference. \citet{kibrik2016referential} focused on a three-way distinction between the choice of a pronoun, proper name and common noun; while \citet{castro-ferreira-etal-2016-towards} classified REs into five categories of pronoun, proper name, common noun, demonstrative NP and empty reference.

\subsection{Different approaches to REG in context} \label{subsec:regapproach} 


Different methods are used to predict the referential choice in context. Rule-based approaches, such as \citet{passonneau1996using}, \citet{mccoy1999generating}, \citet{henschel2000pronominalization} and \citet{krahmer2002efficient}, employ different algorithms to predict RF choice-taking, for instance, centering rules or salience-based accounts into consideration. The GREC shared-task challenges, as one of the first systematic studies on the generation of REs in context, introduced new
feature-based Machine Learning (ML) solutions to this task (e.g. \citet{greenbacker-mccoy-2009-udel, hendrickx-etal-2008-cnts, bohnet-2008-g, favre-bohnet-2009-icsi}, among others). 
Following these shared tasks, \citet{kibrik2016referential} trained decision trees and regression models on the WSJ MoRA, a corpus of Wall Street Journal articles, using a large number of factors. In a more recent feature-based study, \citet{castro-ferreira-etal-2016-towards} trained Naive Bayes and Recurrent Neural Network (RNN) algorithms on the VaREG corpus, taking individual differences in the generation of REs into account. Over the past few years, deep learning approaches have been pre-dominantly used for an end-to-end generation of REs in context, predicting type and content of expressions altogether \citep{castro-ferreira-etal-2018-neuralreg,cao-cheung-2019-referring,cunha-etal-2020-referring,same-etal-2022-non}. \citet{chen-etal-2021-neural-referential} have used pre-trained language models for the choice of RF, but they only use the benchmark NLG dataset, namely WebNLG \citep{gardent-etal-2017-webnlg, castro-ferreira-etal-2018-enriching}, in their study and only use \bert. 


\section{Conclusion and Future Work}
In this work, I design corruption-robust algorithms for the Lipschitz contextual search problem. I present the \emph{agnostic checking} technique and demonstrate its effectiveness in designing corruption-robust algorithms. There are several open problems for future research. First, in the algorithm I propose for pricing loss, the schedule for agnostic checks is fixed upfront. Can the learner design an adaptive checking schedule for the pricing loss? Second, this work assumes the learner has knowledge of the Lipschitz constant $L$. Can the learner design efficient no-regret algorithms without knowledge of $L$? 
%%
%% The next two lines define the bibliography style to be used, and
%% the bibliography file.
\bibliographystyle{ACM-Reference-Format}
\bibliography{reference}

\clearpage
\appendix
% \begin{comment}
\section{System Architecture}
\label{appendix:architecture}
\system has a novel modularized system architecture with three key components: 
\emph{StreamManager}, 
\emph{TxnManager} and \emph{TxnScheduler}. 
These components are instantiated in each thread locally.
The execution outline of \system is presented in Algorithm~\ref{alg:algo}.
Transactional stream processing is continuous and potentially never ends (Line 1$\sim$8).
The dependency resolution and execution of state transactions are separated into two non-overlapping phases by punctuations~\cite{Tucker:2003:EPS:776752.776780} (Line 2 and 5), which guarantees that no subsequent input event will have a smaller timestamp. 
Effectively, a batch of state transactions is collected during the first phase, and processed during the second phase.

In the first phase (i.e., stream processing phase), 
the \emph{StreamManager} conducts preprocessing for every input event ($e$). Similar to some prior works~\cite{tstream}, state transactions may be issued but not immediately processed during preprocessing (Line 3).
The \emph{pre\_processing} and \emph{post\_processing} functions are exposed as APIs to users.
The \emph{TxnManager} handles dependency resolution (Line 4) among state transactions and insert decomposed operations to construct a \tpg. We discuss the detailed two-phase \tpg construction process in Section~\ref{subsec:construction}.

In the second phase  (i.e., transaction processing phase), 
the \emph{TxnManager} is first involved again to refine (Line 6) the constructed \tpg with further dependency resolution.
The \emph{TxnScheduler} 
schedules operations for concurrent execution based on the constructed \tpg according to the three dimensions of scheduling decisions (Line 7). 
In particular, a scheduling decision model $M$ is instantiated based on the constructed \tpg (Line 14).
\textbf{\circled{1}} Guided by $M$, execution threads adopt an exploration strategy (Section~\ref{subsec:explore}) to explore the constructed \tpg for operations available to be scheduled constrained by dependencies. 
\textbf{\circled{2}} 
During exploration, one or multiple operations may be treated as the 
% basic 
unit of scheduling (Section~\ref{subsec:granularity}). 
Subsequently, \textbf{\circled{3}} every thread executes operation(s) in the unit of scheduling with various abort handling mechanisms (Section~\ref{subsec:abort_handling}).
Only when state transactions are processed (i.e., committed or aborted) can the associated input events be postprocessed (Line 8) by the \emph{StreamManager} based on transaction processing results.
\end{comment}

\begin{comment}
\begin{algorithm}
\footnotesize
    \KwData{$e$ \tcp{Input event}}
    \KwData{$txn_{ts}$ \tcp{State transaction}}
    \KwData{$G$ \tcp{The currently constructed TPG}}
    \While{!finish processing of input streams}{
        \eIf(\tcp*[h]{Phase 1}){\text{$e$ is not a $punctuation$}}{
                $txn_{ts}$ $\gets$ PRE\_Processing($e$)\;
                \textbf{TPG\_Construction}($G$, $txn_{ts}$)\; 
          }(\tcp*[h]{Phase 2}){
                \textbf{TPG\_Refinement}($G$)\; 
                \textbf{TXN\_Scheduling}($G$)\; 
                POST\_Processing()\;
          }
    }
    
    \SetKwFunction{FMain}{TPG\_Construction}
    \SetKwProg{Fn}{Function}{:}{}
    \Fn{\FMain{$G$, $txn_{ts}$}}{
        $O_{1..k}$ $\gets$ \textbf{Partition} $txn_{ts}$\;
        \ForEach{\text{operation $O_{i}$ $\in$ $O_{1..k}$}}{
            \textbf{Identify} its \ld\;
            $G$ $\gets$ $G$ + $O_{i}$ \;
        }
    }
    \SetKwFunction{FMain}{TPG\_Refinement}
    \SetKwProg{Fn}{Function}{:}{}
    \Fn{\FMain{$G$}}{
        \ForEach{\text{vertex $e_{i}$ $\in$ $G$}}{
            \textbf{Identify} its \td, \pd\;
        }
    }
    
    \SetKwFunction{FMain}{TXN\_Scheduling}
    \SetKwProg{Fn}{Function}{:}{}
    \Fn{\FMain{$G$}}{
        $M$ $\gets$ Instantiated with $G$;\tcp{A decision model}
        \While{!finish scheduling of $G$
        }{
          \textbf{\circled{2}} $Scheduling Unit$ $\gets$ \textbf{\circled{1}} \emph{Explore}($G$, $M$)\; 
            \textbf{\circled{3}} \emph{Execute with Abort Handling} ($Scheduling Unit$)\; 
        }
    }
  \caption{Execution Outline of \system}
  \label{alg:algo}
\end{algorithm}
\end{comment}

\end{document}
\endinput
%%
%% End of file `sample-sigconf.tex'.
