\documentclass[reprint, nofootinbib, aps, superscriptaddress]{revtex4-2}

\usepackage{graphicx}
\graphicspath{{./figures/}}
\usepackage[caption=false]{subfig}
\usepackage{hyperref}

\usepackage{amsmath,amsfonts,amssymb,physics,bbm}
\usepackage{qcircuit}

\usepackage[boxed]{algorithm}
\usepackage{algorithmic}
\renewcommand{\algorithmicrequire}{\textbf{Input:}}
\renewcommand{\algorithmicensure}{\textbf{Output:}}

\bibliographystyle{apsrev4-2}

\newtheorem{Definition}{Definition}
\newtheorem{Theorem}{Theorem}
\newtheorem{Lemma}{Lemma}
\newtheorem{Corollary}{Corollary}

\newcommand{\Cl}{\mathcal{C}\ell}
\newcommand{\Herm}{\text{Herm}}

\begin{document}

\title{Simulation of quantum computation with magic states via Jordan-Wigner transformations}

\author{Michael Zurel}
\thanks{These authors contributed equally.}
\affiliation{Department of Physics \& Astronomy, University of British Columbia, Vancouver, Canada}
\affiliation{Stewart Blusson Quantum Matter Institute, University of British Columbia, Vancouver, Canada}

\author{Lawrence Z. Cohen}
\thanks{These authors contributed equally.}
\affiliation{Centre for Engineered Quantum Systems, School of Physics, University of Sydney, Sydney, Australia}

\author{Robert Raussendorf}
\affiliation{Department of Physics \& Astronomy, University of British Columbia, Vancouver, Canada}
\affiliation{Stewart Blusson Quantum Matter Institute, University of British Columbia, Vancouver, Canada}

\date{\today}

\begin{abstract}
    Negativity in certain quasiprobability representations is a necessary condition for a quantum computational advantage. Here we define a new quasiprobability representation exhibiting this property with respect to quantum computations in the magic state model. It is based on generalized Jordan-Wigner transformations and it has a close connection to the probability representation of universal quantum computation based on the $\Lambda$ polytopes. For each number of qubits it defines a polytope contained in the $\Lambda$ polytope with some shared vertices. It leads to an efficient classical simulation algorithm for magic state quantum circuits for which the input state is positively represented, and it outperforms previous representations in terms of the states that can be positively represented.
\end{abstract}

\maketitle

\section{Introduction}

Quasiprobability representations have long played a crucial role in physics bridging the gap between classical and quantum~\cite{Wigner1932}. Often with such representations, negativity serves as an indicator of genuinely quantum behaviour, with the fragment of quantum theory that is positively represented behaving more classically~\cite{KenfackZyczkowski2004}.

Gross' discrete Wigner function~\cite{Gross2006,Gross2008}---a quasiprobability representation for systems of odd-dimensional qudits---has been particularly useful in describing quantum computations in the magic state model. In fact, Veitch et al.~\cite{VeitchEmerson2012} showed that negativity in this representation is a necessary condition for a quantum computational advantage over classical computation. This result is obtained by defining an efficient classical simulation algorithm for magic state quantum computations that applies whenever the input state of the quantum circuit has a nonnegative representation. In the last decade, many other quasiprobability representations have been defined which also exhibit this property~\cite{DelfosseRaussendorf2015,PashayanBartlett2015,HowardCampbell2017,KociaLove2017,RallKretschmer2019,RaussendorfZurel2020,SeddonCampbell2021}.

This viewpoint relating negativity and quantum computational advantage was disrupted in Refs.~\cite{ZurelRaussendorf2020,ZurelHeimendahl2021} where a fully probabilistic model describing universal quantum computation was defined. In this model, all quantum states are represented by a probability distribution over a finite set of hidden states and all computational dynamics are represented by stochastic state update rules---no negative probabilities are required. However, this circumvention of negativity comes at a cost: it can no longer be guaranteed that the update rules are efficiently computable classically. Thus, although this model yields a classical simulation algorithm for universal quantum computation, the simulation is likely inefficient in general (as would be expected for any classical simulation of universal quantum computation).

The ultimate question for these classical simulation methods based quasiprobability representations is: \emph{Where runs the line between efficient and inefficient classical simulation, and which physical or mathematical property determines it?} For qubits, so far it is known that this dividing line lies somewhere between the quantum computations covered by the so-called CNC construction~\cite{RaussendorfZurel2020}, and the universal quantum computations described by the $\Lambda$ polytopes~\cite{ZurelRaussendorf2020} (see Fig.~\ref{Figure:LambdaHierarchyCartoon}).

% Figure environment removed

In this work, we enlarge the known region of efficient classical simulability inside the $\Lambda$-polytopes. Specifically, we define a new quasiprobability representation intermediate between those of the CNC construction and the $\Lambda$-polytopes. This model has efficiently computable update rules, and it yields an efficient classical simulation algorithm for quantum comptutations on the subset of input states that are positively represented. It includes the CNC construction, and previous methods such as those based on quasimixtures of stabilizer states~\cite{HowardCampbell2017}. Thus, we effectively push out the boundary of quantum computations which can be efficiently simulated classically.

The model is based on generalized Jordan-Wigner transformations~\cite{ChapmanFlammia2020}, and it's conception was influenced by the surprising connection to Majorana fermions first realized in the CNC construction~\cite{RaussendorfZurel2020}. It was also partially inspired by the techniques of mapping to free fermions in the simulation of other computational models such as matchgate circuits~\cite{JozsaMiyake2008}, which have recently received renewed interest~\cite{MocherlaBrowne2023,CudbyStrelchuk2023,DiasKoenig2023,ReardonSmithKorzekwa2023}.

\medskip

The remainder of this paper is structured as follows. We begin in Section~\ref{Section:Preliminaries} by introducing some notation and definitions. In Section~\ref{Section:PhaseSpaceDescription} we define the new quasiprobability representation of quantum computation with magic states. In Section~\ref{Section:ExtendedClassicalSimulation} we describe the behaviour of the generalized phase space over which this representation is defined with respect to the dynamics of quantum computation with magic states, namely, Clifford gates and Pauli measurements, and we define a classical simulation algorithm for quantum computation with magic states that applies whenever the input state of the quantum circuit is positively represented. We also define a monotone for the resource theory of stabilizer quantum computation~\cite{VeitchEmerson2014} in the case where the state is not positively represented. In Section~\ref{Section:NewLambdaVertices} we elucidate the relationship between this model and the $\Lambda$ polytopes. Finally, we conclude with a discussion of these results in Section~\ref{Section:Discussion}.


\section{Preliminaries}\label{Section:Preliminaries}

The setting of this paper is quantum computation with magic states (QCM) on systems of qubits~\cite{BravyiKitaev2005}.  In this model, computation proceeds through the application of a sequence of Clifford gates and Pauli measurements on an initially prepared ``magic'' input state. For example, Fig.~\ref{Figure:Magic state circuit} shows the standard implementation of a $T$ gate in this model. In general, the input can be any nonstabilizer quantum state, but for universal quantum computation it suffices to consider input states formed from multiple copies of a single-qubit nonstabilizer state~\cite{Reichardt2009}. The output of the computation is derived from the outcomes of the measurements.

% Figure environment removed

\subsection{Notation}

Before proceeding we need to introduce some notation. The single-qubit Pauli group is the group generated by the Pauli operators $\langle X,Y,Z\rangle$. More generally, the $n$-qubit Pauli group $\mathcal{P}$ is the group of Pauli operators acting on $n$ qubits. It is constructed from tensor products of single-qubit Pauli operators. Quotienting out overall phases we have $\mathcal{P}/\mathcal{Z}(\mathcal{P})\cong\mathbb{Z}_2^{2n}$ and we can fix a phase convention for the Pauli operators to be
\begin{equation}
    T_a=i^{-\braket{a_z}{a_x}}Z(a_z)X(a_x)
\end{equation}
for each $a=(a_z,a_x)\in\mathbb{Z}_2^n\times\mathbb{Z}_2^n=:E$, where $Z(a_z)=\bigotimes_{k=1}^{n}Z^{a_z[k]}$, $X(a_x)=\bigotimes_{k=1}^{n}X^{a_x[k]}$, and the inner product $\braket{a_z}{a_x}$ is computed modulo $4$.  The symplectic product $[\cdot,\cdot]:E\times E\rightarrow\mathbb{Z}_2$ defined as
\begin{equation*}
    [a,b]=\braket{a_z}{b_x}+\braket{a_x}{b_z},\;\forall a,b\in E
\end{equation*}
tracks the commutator of the Pauli operators as $T_aT_b=(-1)^{[a,b]}T_bT_a$. We define a function $\beta$ that tracks the signs that get picked up by composing pairs of commuting Pauli operators compose as
\begin{equation}\label{Equation:BetaDefinition}
    T_aT_b=(-1)^{\beta(a,b)}T_{a+b}.
\end{equation}

A projector onto the eigenspace of a set of pair-wise commuting Pauli observables $I\subset E$ corresponding to eigenvalues $\{(-1)^{r(a)}\;|\;a\in I\}$ is given by
\begin{equation}
	\Pi_I^r:=\frac{1}{|I|}\sum\limits_{a\in I}(-1)^{r(a)}T_a.
\end{equation}
The set $I\subset E$ and the function $r:I\rightarrow\mathbb{Z}_2$ must satisfy a set of consistency conditions in order for $\Pi_I^r$ to be a valid projector. First, $I$ must be a closed subspace of $E$. Furthermore, in order for the observables in $I$ to be simultaneously measurable they must commute, and so $I$ must be an isotropic subspace---a subspace on which the symplectic form vanishes. We must also have $(-1)^{r(0)}T_0=1$, or equivalently with the phase convention chosen above, $r(0)=0$. Lastly, for each $a,b\in I$, from the relation $(-1)^{r(a)}T_a\cdot(-1)^{r(b)}T_b=(-1)^{r(a+b)}T_{a+b}$ we must have
\begin{equation*}
	r(a)+r(b)+r(a+b)\equiv\beta(a,b)\mod2.
\end{equation*}
In the case of a single Pauli measurement $a\in E$ yielding measurement outcome $s\in\mathbb{Z}_2$, the corresponding projector is $\Pi_a^s=(1+(-1)^sT_a)/2$.

A stabilizer state on $n$-qubits is an eigenstate of set of $n$ independent and pair-wise commuting Pauli operators~\cite{Gottesman1997,Gottesman1998}. In the case $I\subset E$ is a maximal isotropic subspace, $\Pi_I^r$ is a projector onto a stabilizer state. We denote by $\mathcal{S}$ the set of $n$-qubit stabilizer states.

The gates of QCM are drawn from the Clifford group---the normalizer of the Pauli group in the unitary group up to overall phases: $\Cl:=\mathcal{N}(\mathcal{P})/U(1)$. The Clifford group acts on the Pauli group as
\begin{equation*}
    g(T_a)=(-1)^{\Phi_g(a)}T_{S_ga},\forall g\in\Cl,\forall a\in E,
\end{equation*}
where for each $g\in\Cl$, $S_g$ is a symplectic map on $E$ (i.e. a linear map that preserves the symplectic product), and $\Phi_g$ is a function defined through this relation to track the signs that get picked up in the Clifford group action.

The functions $\beta$ and $\Phi_g$ above have a cohomological interpretation defined in Ref.~\cite{OkayRaussendorf2017} that relates them to proofs of contextuality. This was further elucidated in Ref.~\cite{RaussendorfFeldmann2021} which relates them to the phenomenology of quantum computation with magic states.

In the following we use the notation $\mathcal{O}$ for a general subset of $E$ (labels for a subset of Pauli operators), and $\mathcal{O}^*$ for the set $\mathcal{O}\setminus\{0\}$ (the nonidentity Pauli operators in $\mathcal{O}$). We denote by $\Herm(\mathcal{H})$ the space of Hermitian operators on Hilbert space $\mathcal{H}$, and unless otherwise specified, $\mathcal{H}=(\mathbb{C}^2)^{\otimes n}$ is the $n$-qubit Hilbert space. $\Herm_1(\mathcal{H})$ is the affine subspace of $\Herm(\mathcal{H})$ consisting of operators with unit trace and $\Herm_1^{\succeq0}(\mathcal{H})$ is the subset of $\Herm_1(\mathcal{H})$ consisting of positive semidefinite operators. $\Herm_1^{\succeq0}(\mathcal{H})$ contains density operators representing physical quantum states.


\subsection{Previous quasiprobability representations}

\subsubsection*{The CNC construction}

Recently, a quasiprobability representation for QCM was defined based on noncontextual sets of Pauli observables~\cite{RaussendorfZurel2020}, this is the so-called CNC construction.  Therein, phase space points are identified with pairs $(\Omega,\gamma)$, where $\Omega\subset E$ is a subset of Pauli operators and $\gamma:\Omega\rightarrow\mathbb{Z}_2$ is a function on $\Omega$ satisfying two conditions:\footnote{``CNC'' is for Closed and Non-Contextual after these conditions.}
\begin{enumerate}
    \item \emph{Closure under inference.} For all $a,b\in\Omega$, $$[a,b]=0\Longrightarrow a+b\in\Omega$$
    \item \emph{Noncontextuality.\footnote{Implicit in the condition on $\gamma$ there is a constraint on the set $\Omega$, namely that such a noncontextual value assignment function exists. Because of Mermin square-style proofs of contextuality~\cite{Mermin1993}, this is not guaranteed for general subsets of $E$.}} $\gamma$ is a noncontextual value assignment function on $\Omega$. I.e. a function $\gamma:\Omega\rightarrow\mathbb{Z}_2$ such that $(-1)^{\gamma(0)}T_0=1$ and $\forall a,b\in\Omega$ with $[a,b]=0$, we have
    \begin{equation}\label{Equation:NoncontextualityCondition}
    	\gamma(a)+\gamma(b)+\gamma(a+b)\equiv\beta(a,b)\mod2.
    \end{equation}
\end{enumerate}

Then the phase space point operator associated to phase space point $(\Omega,\gamma)$ is defined as
\begin{equation}\label{Equation:CNCOperators}
    A_\Omega^\gamma=\frac{1}{2^n}\sum\limits_{b\in\Omega}(-1)^{\gamma(b)}T_b.
\end{equation}
These operators span the space of Hermitian operators on Hilbert space $\mathcal{H}=(\mathbb{C}^2)^{\otimes n}$ and so any $n$-qubit quantum state $\rho$ can be expanded in these operators as
\begin{equation}\label{Equation:CNCStateRepresentation}
    \rho=\sum\limits_{(\Omega,\gamma)\in\mathcal{V}}W_\rho(\Omega,\gamma)A_\Omega^\gamma.
\end{equation}

It was shown in Ref.~\cite[\S IV]{RaussendorfZurel2020} that the admissible pairs $(\Omega,\gamma)$ could be characterized as follows. Let $I \subset E$ be an isotropic subspace and $\tilde{\Omega} \subset E$ be a subset of Pauli operators such that $\left[a, b\right] = 1$ for all $a, b\in \tilde{\Omega}$. Then the sets $\Omega$ can be expressed as $\Omega = \bigcup_{a \in \tilde{\Omega}} \langle a, I \rangle.$ The signs $\gamma$ on $\Omega$ can be chosen freely on $\tilde{\Omega}$ and on a basis of $I$, and then the signs on the rest of $\Omega$ are determined by eq.~(\ref{Equation:NoncontextualityCondition}).

In simpler terms, the phase point operators have the form
\begin{equation}\label{Equation:CNCOperatorAlt}
    A_\Omega^\gamma=g(A_{\tilde{\Omega}}^{\tilde{\gamma}}\otimes\ket{\sigma}\bra{\sigma})g^\dagger
\end{equation}
where $g\in\Cl$ is a Clifford group element, all elements of $\tilde{\Omega}$ pair-wise anticommute, the signs $\tilde{\gamma}$ on $\tilde{\Omega}^*$ can be chosen freely, and $\ket{\sigma}$ is a stabilizer state. Interestingly, a set of pair-wise anticommuting Pauli operators has the same algebraic structure as a set of Majorana fermion operators and so, ignoring the stabilizer state tail in Eq.~(\ref{Equation:CNCOperatorAlt}), we can view the operator $A_{\tilde{\Omega}}^{\tilde{\gamma}}$ as linear combinations of Majorana operators.

A phase space point operators of the form eq.~(\ref{Equation:CNCOperators}) maps deterministically to another phase space point operators under any Clifford gate, and it maps to a probabilistic combination of such operators under Pauli measurements (see Ref.~\cite[\S5]{RaussendorfZurel2020} for details). Further, these updates can be computed efficiently on a classical computer. As a result, the representation of eq.~(\ref{Equation:CNCStateRepresentation}) yields an efficient classical simulation algorithm for QCM that applies whenever the input state of the quantum circuit has a nonnegative representation.

\subsubsection*{The \texorpdfstring{$\Lambda$}{Lambda} polytopes}

In Refs.~\cite{ZurelRaussendorf2020,ZurelHeimendahl2021}, a \emph{probability representation} (or a hidden variable model) of quantum computation with magic states was defined. It has the same structure as a quasiprobability representation, except for the fact that no negativity is needed.

This representation can be defined for qudits of any Hilbert space dimension, but we state only the multi-qubit version here. The state space of the model is based on the set
\begin{equation}
	\Lambda = \left\{ X\in\Herm_1(\mathcal{H})\;|\;\Tr(\ket{\sigma}\bra{\sigma}X)\ge0\;\forall\ket{\sigma}\in\mathcal{S} \right\}
\end{equation}
where $\mathcal{S}$ denotes the set of pure $n$-qubit stabilizer states. For each number $n$ of qubits, $\Lambda$ is a bounded polytope with a finite number of vertices~\cite{ZurelHeimendahl2021}. We denote the vertices of $\Lambda$ by $\{A_\alpha\;|\;\alpha\in\mathcal{V}\}$ where $\mathcal{V}$ is an index set. Then the model is defined by the following theorem.
\begin{Theorem}[Ref.~\cite{ZurelRaussendorf2020}; Theorem 1]\label{Theorem:HVM}
	For any number of qubits $n$,
	\begin{enumerate}
		\item Any $n$-qubit quantum state $\rho\in\Herm_1^{\succeq0}(\mathcal{H})$ can be decomposed as
		\begin{equation}
			\rho=\sum\limits_{\alpha\in\mathcal{V}}p_\rho(\alpha)A_\alpha,
		\end{equation}
		with $p_\rho(\alpha)\ge0$ for all $\alpha\in\mathcal{V}$, and $\sum_\alpha p_\rho(\alpha)=1$.
		\item For any $A_\alpha,\;\alpha\in\mathcal{V}$, and any Clifford gate $g\in\Cl$, $gA_\alpha g^\dagger$ is a vertex of $\Lambda$. This defines an action of the Clifford group on $\mathcal{V}$ as $gA_\alpha g^\dagger=:A_{g\cdot\alpha}$ where $g\cdot\alpha\in\mathcal{V}$.
		\item For any $A_\alpha,\;\alpha\in\mathcal{V}$ and any Pauli projector $\Pi_a^s$ we have
		\begin{equation*}
			\Pi_a^sA_\alpha\Pi_a^s=\sum\limits_{\beta\in\mathcal{V}}q_{\alpha,a}(\beta,s)A_\beta.
		\end{equation*}
		with $q_{\alpha,a}(a,s)\ge0$ for all $\beta\in\mathcal{V}$ and $s\in\mathbb{Z}_2$, and $\sum_{\beta,s}q_{\alpha,a}(\beta,s)=1$.
	\end{enumerate}
\end{Theorem}

This theorem describes a hidden variable model for QCM in which (i)~states are represented by probability distributions $p_\rho$ over $\mathcal{V}$, and (ii)~Clifford gates and Pauli measurements are represented by stochastic maps $g\cdot\alpha$ and $q_{\alpha,a}$. In this model, no negative values are needed in the representation of states, gates, or measurements---it is a true probability representation, but we can no longer guarantee that the updates under Clifford gates and Pauli measurements are efficiently computable classically. Thus, although this representation does give a classical simulation algorithm for any magic state quantum circuit, the simulation is inefficient in general. Analyzing the efficiency of simulation using the $\Lambda$ polytope requires a characterization of it's vertices.

To date, only the $\Lambda$ polytopes on one and two qubits have been fully characterized. In addition, some vertices of $\Lambda$ are known for every qubit number. For example, it is known that the CNC-type phase point operators associated to maximal CNC sets are vertices of the $\Lambda$ polytopes~\cite{Heimendahl2019,ZurelRaussendorf2020}.


\section{Multiqubit phase space from Jordan-Wigner transformations}\label{Section:PhaseSpaceDescription}

Below we define a new quasiprobability representation intermediate between the CNC construction and the full $\Lambda$ polytope model based on newly characterized vertices of $\Lambda$ for any number of qubits. This new model can positively represent more states than the CNC construction (though not all quantum states as in the case of $\Lambda$), and it maintains the property of efficiently computable state update rules for the dynamics of QCM.

\subsection{Line graphs and Jordan-Wigner transformations}\label{Section:LineGraph}

% Figure environment removed

The generalized phase space we define will consist of operators which can be described by quadratic polynomials of Majorana fermion operators. We motivate this approach by first considering the earlier CNC construction. As alluded to above, on $n$-qubits a maximal CNC set (without stabilizer state tensor factors) is given by $2n+1$ pair-wise anti-commuting Pauli operators~\cite[\S IV]{RaussendorfZurel2020}. Such a set of Pauli operators satisfy the anti-commutation relations
\begin{equation*}
     \{ \gamma_i, \gamma_j \} := \gamma_i \gamma_j + \gamma_j \gamma_i = 2\delta_{ij}
\end{equation*}
which are exactly the relations satisfied by a set of Majorana operators. Thus, under a Jordan-Wigner-like transformation, the maximal CNC operators are given by linear combinations of Majorana operators. Furthermore, there is a large body of work studying which operators can be described via quadratic combinations of Majorana operators. In particular, recently it has been shown that the operators describable in such a way can be identified by the structure of the graphs describing their anti-commutation relations~\cite{ChapmanFlammia2020}.

Given a graph $R = (\mathfrak{V}, \mathfrak{E})$, the \emph{line graph} of $R$, $L(R) = (\mathfrak{E}, \mathfrak{E}')$ is the graph whose vertex set is the edge set of $R$, and whose edge set is $\mathfrak{E}' = \left\{ (e_1, e_2) \in \mathfrak{E} \times \mathfrak{E} | e_1 \cap e_2 \neq \emptyset \right\}$; that is two vertices in $L(R)$ are neighbours if and only if the corresponding edges in $R$ share a vertex. Given a line graph $G \cong L(R)$, we refer to the graph $R$ as the \emph{root graph} of $L(R)$. See Figure~\ref{Figure:LineGraphExample} for an example. We will also use the notion of \emph{twin vertices}. Two vertices $u, v \in \mathfrak{V}$ are twin vertices if for every vertex $w \in \mathfrak{V}$, $(u, w) \in \mathfrak{E}$ if and only if $(v, w) \in \mathfrak{E}$.

Let $\mathcal{O} \subset E$ be a subset of the Pauli operators. We define the \emph{frustration graph} of $\mathcal{O}$, $F \left( \mathcal{O} \right) = \left( \mathcal{O}, \mathfrak{E} \right)$ as the graph whose vertices are identified with elements of $\mathcal{O}$, and with edges drawn between $a, b \in  \mathcal{O}$ if and only if $\left[ a, b \right] = 1$.

We now show that every line graph, $L(R)$, can be realized as the frustration graph of some set of Pauli operators $\mathcal{O}$~\cite{ChapmanFlammia2020} (see Fig.~\ref{Figure:frustration graph specific} for an example). To do this, we first construct a set of $|R|$ pair-wise anti-commuting Pauli operators. For even $|R|$, we can find such a set by taking the standard Jordan-Wigner transformation of Majorana fermion operators~\cite{JordanWigner1928}
\begin{align}
    C_{2j-1} &= \bigotimes_{k=1}^{j-1} Z_k \otimes X_j \\
    C_{2j} &= \bigotimes_{k=1}^{j-1} Z_k \otimes Y_j
\end{align}
for $j = 1, \ldots, |R|/2$. For odd $|R|$, we use the Jordan-Wigner transformation on $|R|-1$ qubits and also include the operator
\begin{equation}
    C_{|R|} = \bigotimes_{k=1}^{(|R|-1)/2} Z_k.
\end{equation}
Now we identify each vertex $r$ in $R$ with an operator $C_r$. We can then identify each edge $(r_j, r_k)$ in $R$, and consequently each vertex in $L(R)$, with the Pauli operator $\pm i C_{r_1} C_{r_2}$. Two Pauli operators $i C_{r_1} C_{r_2}$ and $i C_{r_3} C_{r_4}$ anti-commute if and only if they share exactly one Majorana operator, but this is equivalent to the associated edges in $R$ sharing a vertex. Thus, the frustration graph of this set of quadratic Majorana operators is exactly the line graph $L(R)$, as desired. Note that this is not necessarily the most efficient way to realize each line graph as a frustration graph, though it does work for any line graph $L(R)$.

% Figure environment removed


\subsection{Definition of the generalized phase space}

In this work we are interested in operators with the following structure:
\begin{Definition}\label{Definition:LineGraphOperator}
    A Hermitian operator expanded in the Pauli basis as
    \begin{equation}\label{Equation:LineGraphOperator}
        A_{\mathcal{O}}^{\vec{c}} = \frac{1}{2^n} \left( 1 + \sum_{b \in \mathcal{O}^*} c_b T_b \right)
    \end{equation}
    is a line graph operator if the frustration graph of $\mathcal{O}^*$ is a line graph.
\end{Definition}

We actually wish to be a bit more permissive than requiring the frustration graphs to be exactly line graphs. In particular, we also want to include operators that have been obtained by tensoring on stabilizer states and conjugating by a Clifford gate, as in Eq.~(\Ref{Equation:CNCOperatorAlt}). In this case, the frustration graph will only be a line graph up to twin vertices. Suppose we take an operator which is the tensor product of a projector onto an $m$-qubit stabilizer state, $\ket{\sigma}\bra{\sigma}$, and an $n-m$-qubit operator, $A_{\mathcal{O}}^{\vec{c}}$, whose frustration graph is a line graph. Then for each non-trivial Pauli operator in $A_{\mathcal{O}}^{\vec{c}}$ there are $2^m$ associated vertices in the frustration graph of $A_{\mathcal{O}}^{\vec{c}}\otimes\ket{\sigma}\bra{\sigma}$. These vertices will all be pair-wise twin vertices. Thus if we can partition a graph into sets of equally sized pair-wise twin vertices then the associated operator will have a stabilizer tail. It is thus possible to include stabilizer states in this formalism by collapsing sets of twin vertices.

In order for operators with the above structure to correctly reproduce the quantum mechanical predictions for sequences of Pauli measurements, they must be in the $\Lambda$ polytopes. Otherwise, we could easily construct sequences of Pauli measurements which give nonsensical results when performed on these operators. Without any more constraints, this defines an infinite set of operators for every number of qubits since the coefficients in eq.~(\ref{Equation:LineGraphOperator}) can vary continuously.

For the purpose of defining the generalized phase space we can focus our attention on a finite subset of these operators. We choose this set in the following way. Let $\mathcal{O}$ be a set of Pauli operators such that the frustration graph of $\mathcal{O}^*$ is a line graph. The operators of the form eq.~(\ref{Equation:LineGraphOperator}) with support $\mathcal{O}$ form an affine subspace of $\Herm_1(\mathcal{H})$. Projecting the polytope $\Lambda$ onto this affine subspace defines a new polytope with a finite number of vertices, and when interpreted as a polytope embedded in the same $4^n-1$-dimensional space as $\Lambda$ with the Pauli  coefficients of $E\setminus\mathcal{O}$ set to zero, it is contained in $\Lambda$. This projected polytope also has a finite number of vertices. We take the generalized phase point operators to be operators of this form, possibly with a stabilizer state tensor factor added and a conjugation by a Clifford operator. This augmentation of the line graph operators is natural for our purposes since tensoring on stabilizer states and conjugating by Clifford operators does not increase the complexity of classical simulation~\cite{OkayRaussendorf2021}.

To summarize, the set of phase space point operators corresponding to our generalized phase space $\mathcal{V}$ is the set of operators that can be constructed by the following procedure:
\begin{enumerate}
    \item Start by defining a support $\tilde{\mathcal{O}}\subset E$ such that the frustration graph of $\tilde{\mathcal{O}}^*$ is a line graph,
    \item project $\Lambda$ onto the affine subspace spanned by the operators in $\tilde{\mathcal{O}}^*$, choose an extreme point $A_{\tilde{\mathcal{O}}}^{\tilde{\vec{c}}}$ of this projected polytope,
    \item choose a stabilizer state $\ket{\sigma}\in\mathcal{S}$ and a Clifford gate $g\in\Cl$, return $A_{\mathcal{O}}^{\vec{c}}:=g(A_{\tilde{\mathcal{O}}}^{\tilde{\vec{c}}}\otimes\ket{\sigma}\bra{\sigma})g^\dagger$ as the phase point operator.
\end{enumerate}
Such an operator can be uniquely labeled by its support $\mathcal{O}$ in the Pauli basis and the corresponding Pauli basis coefficients $\vec{c}$.

With these conditions defining the generalized phase space $\mathcal{V}$, any state $\rho$ can be decomposed as
\begin{equation}\label{Equation:LineDecomp}
    \rho = \sum_{(\mathcal{O}, \vec{c}) \in \mathcal{V}} W_\rho (\mathcal{O}, \vec{c}) A_{\mathcal{O}}^{\vec{c}}.
\end{equation}
This is the representation of states in the model. Note that since the phase point operators are over-complete, this representation is not unique. It is generally preferable to choose a representation that minimizes the amount of negativity as measured by the $1$-norm of the coefficients $W_\rho(\mathcal{O},\vec{c})$. This can be obtained through linear programming.


\section{Extended classical simulation}\label{Section:ExtendedClassicalSimulation}

Let $A_\mathcal{O}^{\vec{c}}$ be an operator of the form eq.~(\ref{Equation:LineGraphOperator}) where the frustration graph of the Pauli operators $\mathcal{O}^*$ is a line graph. Further assume that the coefficients $\vec{c}$ are chosen so that $A_\mathcal{O}^{\vec{c}}$ is in $\Lambda$. The goal of this section is to show that the set of operators with these properties is closed under the dynamics of quantum computation with magic states---Clifford gates and Pauli measurements.

Note that if we can establish that the line graph operators themselves are closed under conjugation by Clifford group elements and Pauli measurements, then clearly the generalized phase space defined in Section~\ref{Section:PhaseSpaceDescription} is also closed under these operations.

\subsection{Closure under Clifford gates}\label{Section:CliffordUpdateRules}

First we consider the action of Clifford gates on operators of the form eq.~(\ref{Equation:LineGraphOperator}). For any Clifford operation $g\in\Cl$, we have
\begin{align*}
    g(A_\mathcal{O}^{\vec{c}})=&\frac{1}{2^n}\sum\limits_{b\in\mathcal{O}^*}c_bg(T_b)\\
    =&\frac{1}{2^n}\sum\limits_{b\in\mathcal{O}^*}c_b(-1)^{\Phi_g(b)}T_{S_gb}=:A_{g\cdot\mathcal{O}}^{g\cdot\vec{c}}.
\end{align*}
This defines an action of $g$ on the support $\mathcal{O}$ as
\begin{equation*}
    g\cdot\mathcal{O}=\{S_gb\;|\;b\in\mathcal{O}\},
\end{equation*}
and on the coefficients $\vec{c}$ as
\begin{equation*}
    (g\cdot \vec{c})_{S_gb}=c_b(-1)^{\Phi_g(b)}.
\end{equation*}
Since $\Lambda$ is closed under Clifford operations~\cite{ZurelRaussendorf2020}, if $A_\mathcal{O}^{\vec{c}}$ is in $\Lambda$ then $A_{g\cdot\mathcal{O}}^{g\cdot\vec{c}}$ is also in $\Lambda$.

Since Clifford operations preserve commutation relations of Pauli operators, the frustration graph of $g\cdot\mathcal{O}^*$ is isomorphic to that of $\mathcal{O}^*$. Thus, if the frustration graph of $\mathcal{O}^*$ is a line graph then so is the frustration graph of $g\cdot\mathcal{O}^*$.

\subsection{Closure under Pauli measurements}\label{Section:PauliUpdateRules}

Now consider the action of the Pauli projector $\Pi_a^s$ corresponding to a Pauli measurement $T_a,\;a\in E$, yielding measurement outcome $(-1)^s$. We want to compute the projection of the ``state'' $A_\mathcal{O}^{\vec{c}}$. 

The only terms in $A_\mathcal{O}^{\vec{c}}$ that remain after projecting with $\Pi^s_a$ are those that commute with $T_a$. Let $T_b$ be any Pauli operator in $\mathcal{O}$ that anti-commutes with $T_a$. Let $J_1=\{a,b\}^\perp \cap \mathcal{O}$, i.e., $J_1$ is the subset of Paulis in $\mathcal{O}$ that commute with $a$ and $b$, and let $J_2$ be the subset of $\{a\}^\perp \cap \mathcal{O}$ that anti-commute with $b$. Then we can write

\begin{align*}
    \Pi^s_a A_\mathcal{O}^{\vec{c}} \Pi^s_a &= \frac{1}{2^n} \Pi^s_a \left( \sum_{j \in J_1} c_j T_j + \sum_{j \in J_2} c_j T_j \right) \\
    &= \frac{1}{2^{n+1}} \left( A_\sigma + (-1)^sT_a A_\sigma \right) \\
    &= \frac{1}{2^n} \Pi_a^s A_\sigma
\end{align*}
where
\begin{align}\label{Equation:update}
    A_\sigma &= \sum_{j \in J_1} c_j T_j + (-1)^sT_a\sum_{j \in J_2} c_j T_j.
\end{align}

Now note that the Pauli elements non-trivially acted on by $A_\sigma$ are distinct from those non-trivially acted on by $T_a A_\sigma$ since those in $A_\sigma$ commute with $T_b$ while those in $T_a A_\sigma$ anti-commute with $T_b$. Furthermore, for each $T_j \in A_\sigma$ there is a corresponding element $T_a T_j \in T_a A_\sigma$ such that for any $T_k \in \Pi^s_a A_\mathcal{O}^{\vec{c}} \Pi^s_a$ we have $[T_j, T_k] = 0 \iff [T_a T_j, T_k] = 0$. Thus $A_\sigma$ and $T_a A_\sigma$ form sets of twin vertices in the frustration graph of $\Pi^s_a A_\mathcal{O}^{\vec{c}} \Pi^s_a$.

Now we only need to show that the frustration graph of $A_\sigma$ is a line graph. Since all the elements of $T_{J_1}$ commute with $T_a$ their commutation relations with element in $T_a T_{J_2}$ is the same as their commutation relations with elements in $T_{J_2}$. Hence the frustration graph of $A_\sigma$ is an induced subgraph of the frustration graph of $\mathcal{O}$. Since the property of being a line graph is an induced-subgraph-hereditary property, the frustration graph of $A_\sigma$ is a line graph.

We also need to show closure of operators of the form $\Pi_{I_1}^{s_1} \ldots \Pi_{I_m}^{s_m} A^{\vec{c}}_{\mathcal{O}}$ under Pauli measurements. These are the operators that are line graph operators up to twin vertices. We can appeal to the circuit reduction techniques in Refs~\cite{BravyiSmolin2016,PeresGalvao2021} and in Theorem~3 of Ref.~\cite{OkayRaussendorf2021} to reduce a sequence of Pauli measurements on $n$ qubits to a sequence of exactly $n$ commuting Pauli measurements. Then a measurement of $\Pi_{I_{m+1}}^{s_{m+1}}$ will commute with the operator $\Pi_{I_1}^{s_1} \ldots \Pi_{I_m}^{s_m}$ and we can apply the techniques in the previous paragraph to update the phase space point operator. %\footnote{Note that the statement of Theorem~3 of Ref.~\cite{OkayRaussendorf2021} is about Pauli measurements performed on operators of the form $X\otimes\ket{\sigma}\bra{\sigma}$ where $X$ is a vertex of $\Lambda$. Here we apply the same statement to the case where $X$ is not a vertex of $\Lambda$ but rather some operator contained in $\Lambda$. This causes no problems because the fact that $X$ is a vertex is never used in the proof of Theorem~3, and so the proof is identical for our (slightly) more general case.}

\subsection{Classical simulation algorithm}\label{Section:ClassicalSimulation}

As shown in Sections~\ref{Section:CliffordUpdateRules} and~\ref{Section:PauliUpdateRules}, the generalized phase space $\mathcal{V}$ over which the quasiprobability representation is define is closed under Clifford gates and Pauli measurements---the two dynamical operations in the model of quantum computation with magic states. This fact allows us to define a classical simulation algorithm for quantum computation with magic states that applies whenever the representation of the input state is nonnegative, Algorithm~\ref{Algorithm:ClassicalSimulation}.

The proof of correctness for this algorithm is analogous to the proof of Theorem~5 of Ref.~\cite{RaussendorfZurel2020} or the proof of Theorem~2 of Ref.~\cite{ZurelRaussendorf2020}. The proof of efficiency for the state updates follows from the fact that each set $\mathcal{O}$ consists of $O(n^2)$ Pauli operators when neglecting the stabilizer tail. Clifford updates require determining the action of $g \in \Cl$ on each element of $\mathcal{O}$, which can be computed in polynomial time. Update under Pauli measurements first requires calculating $\text{Tr}(\Pi_a^s A_{(\mathcal{O},\vec{c})})$, which can be determiend efficiently as $A_{(\mathcal{O},\vec{c})})$ only has $O(n^2)$ Pauli operators with non-zero coefficients. We then need to find the projection of $O(n^2)$ Pauli operators onto $T_a$, which can be done efficiently via the Gottesman-Knill theorem. Lastly we need to update the sequence of gates and measurements, which can be done efficiently via Theorem~3 of Ref.~\cite{OkayRaussendorf2021}.

In general a state $\rho$ will not admit a decomposition such that $W_\rho (\mathcal{O}, \vec{c}) \geq 0$ for all $(\mathcal{O}, \vec{c})$. Consequently we define the robustness of a state as
\begin{equation*}
    \mathfrak{R}(\rho) := \min\limits_{W}\left\{||W||_1\;\bigg|\;\rho=\sum\limits_{(\mathcal{O},\vec{c})\in\mathcal{V}}W(\mathcal{O},\vec{c})A_{\mathcal{O}}^{\vec{c}}\right\}.
\end{equation*}
Since line graph operators are preserved under Clifford gates and Pauli measurement the robustness is a monotone. The complexity of the simulation algorithm in the presence of negativity can then be related to the robustness~\cite{PashayanBartlett2015}. This generalized robustness is bounded above by the phase-space robustness of the CNC construction~\cite{RaussendorfZurel2020} and the robustness of magic~\cite{HowardCampbell2017}.

\begin{algorithm}[H]
	\begin{algorithmic}[1]
		\REQUIRE $p_{\rho_{0}},\text{ sequence of gates and measurements } \mathcal{T}$
		\STATE sample a point $(O, \vec{c})\in\mathcal{V}$ according to the probability distribution $p_{\rho_{0}}$
		\WHILE{end of circuit has not been reached}
		      \IF{a Clifford gate $g\in\Cl$ is encountered}
		          \STATE update $(O, \vec{c})\leftarrow g\cdot(O, \vec{c})$
		      \ENDIF
		      \IF{a Pauli measurement $T_a,\;a\in E$ is encountered}
                \STATE choose outcome $s$ with probability $\text{Tr}(\Pi_a^s A_{(\mathcal{O},\vec{c})})$
                \STATE choose a Pauli operator $T_b$ such that $[a,b] \neq 0$
                \STATE update $A_{(O, \vec{c})} \leftarrow \Pi_a^s A_\sigma$ as in Eq.~\ref{Equation:update}
                \STATE update $\mathcal{T}$ according to Refs.~\cite{BravyiSmolin2016,PeresGalvao2021}
		      \ENDIF
		\ENDWHILE
	\end{algorithmic}\caption{One run of the classical simulation of quantum computation with magic states based on the quasiprobability representation defined in Sections~\ref{Section:PhaseSpaceDescription} and the update rules described in Sections~\ref{Section:CliffordUpdateRules} and~\ref{Section:PauliUpdateRules}.  The algorithm provides samples from the joint probability distribution of the Pauli measurements in a quantum circuit consisting of Clifford gates and Pauli measurements applied to an input state $\rho$ such that $W_\rho \geq 0$.}
    \label{Algorithm:ClassicalSimulation}
\end{algorithm}


\section{New vertices of the \texorpdfstring{$\Lambda$}{Lambda} polytopes}\label{Section:NewLambdaVertices}

The generalized phase space of the present model is defined in part by looking at projections of $\Lambda$. Depending on the subspace on which we project, the projected polytope may share vertices with $\Lambda$. In this section we show that this is indeed the case. In particular, we show that, for any number $n$ of qubits, projecting $\Lambda$ onto the space of operators with support $\mathcal{O}$ for which the frustration graph of $\mathcal{O}^*$ is the line graph $L(K_{2n+1})$ gives a polytope which shares vertices with $\Lambda$. This is the content of Theorem~\ref{Theorem:NewLineGraphVertices} below. In this case we can determine the coefficients of the vertices in the Pauli basis as well, thus we obtain a complete characterization of new families of vertices of the $\Lambda$ polytopes for every number of qubits. By polar duality~\cite{ZurelHeimendahl2021}, we also obtain a complete characterization of new families of facets of the stabilizer polytope for every number of qubits.

This is the result of the following theorem.
\begin{Theorem}\label{Theorem:NewLineGraphVertices}
	Define the operator
	\begin{equation}\label{Equation:NewLineGraphVertices}
		A_{\mathcal{O}}^{\eta} = \frac{1}{2^n} \left( 1 + \frac{1}{n} \sum_{b\in\mathcal{O}^*} (-1)^{\eta(b)} T_b \right)
	\end{equation}
	where the frustration graph of $\mathcal{O}^*$ is $L(K_{2n+1})$. There exist choices for the signs $\eta:\mathcal{O}^*\rightarrow\mathbb{Z}_2$ such that the operators $A_\mathcal{O}^\eta$ of the form eq.~(\ref{Equation:NewLineGraphVertices}) are vertices of $\Lambda$.
\end{Theorem}
The rest of this section is devoted to the proof of this Theorem.

\emph{Proof of Theorem~\ref{Theorem:NewLineGraphVertices}.} By Theorem~18.1 of Ref.~\cite{Chvatal1983}, to prove that an operator $A_{\mathcal{O}}^\eta$ is a vertex of $\Lambda$, it suffices to show (1) that $A_{\mathcal{O}}^\eta$ is in $\Lambda$ and (2) that the set of projectors onto stabilizer states which are orthogonal to $A_{\mathcal{O}}^\eta$ with respect to the Hilbert-Schmidt inner product has rank $4^n-1$ when viewed as vectors of Pauli basis coefficients.

To start we can directly compute the inner product $\Tr(\Pi_I^rA_{\mathcal{O}}^\eta)$ for any choice of signs $\eta$ and any projector onto a stabilizer state $\Pi_I^r$:
\begin{align*}
    \Tr(\Pi_I^rA_{\mathcal{O}}^\eta)=&\frac{1}{2^n}+\frac{1}{n2^{2n}}\sum\limits_{a\in I}\sum\limits_{b\in \mathcal{O}^*}(-1)^{r(a)+\eta(b)}\Tr(T_aT_b)\\
    =&\frac{1}{2^n}+\frac{1}{n2^n}\sum\limits_{a\in I\cap\mathcal{O}^*}(-1)^{r(a)+\eta(a)}.
\end{align*}
Since the largest independent set in $L(K_{2n+1})$ has size $n$, this inner product is always nonnegative and so $A_{\mathcal{O}}^{\eta}\in\Lambda$. Also, the inner product is zero if and only if $|I\cap\mathcal{O}^*|=n$ and $r(a)\ne\eta(a)$ for all $a\in I\cap\mathcal{O}^*$. All that remains is to show that the signs $\eta$ can be chosen so that the set of stabilizer states for which this inner product is zero has rank $4^n-1$.

% Figure environment removed

Consider the bipartite graph $G$ constructed as follows: $G$ has a vertex for each Pauli observable $a\in E\setminus\{0\}$, a vertex for each isotropic subspace $I\subset E$ such that $|I\cap\mathcal{O}^*|=n$, and an edge connecting a vertex $a\in E$ to an isotropic subspace $I\subset E$ if $a\in I$. A sketch of this graph is shown in Fig.~\ref{Figure:New vertices proof - Bipartite graph}. We need to show that there exists a choice of signs for the edges of this graph such that each off-diagonal block of the corresponding signed adjacency matrix has rank $4^n-1$.

Using a graph theoretic result proven in Appendix~\ref{Appendix:GraphTheoryResults} (see in particular Corollary~\ref{Corollary2}), to establish this it suffices to show that the graph $G$ has a matching of size $4^n-1$. By K\H{o}nig's theorem, this is true if and only if the minimum vertex cover of $G$ has order $4^n-1$. One vertex cover of order $4^n-1$ is obtained by taking all vertices on the Pauli operator side of the bipartition. Now we must show that a smaller vertex cover cannot be obtained by removing vertices from the Pauli side and adding fewer vertices on the isotropic subspace side. To do this we compute the degree of each vertex in the graph. If we can show that the degree of each vertex on the left hand side of the graph in Fig.~\ref{Figure:New vertices proof - Bipartite graph} is larger than the degree of every vertex on the right hand side of the graph, then the minimum vertex degree has size $4^n-1$ and the result is proved.

Each isotropic subspace $I\subset E$ has order $2^n$, one of these elements is $0$, and so the degree of each vertex on the right hand side is $2^n-1$.

Now we need to count the number of isotropic subspaces $I\subset E$ containing each observable $a\in E\setminus\{0\}$. Here we have many cases. Up to an overall phase, each Pauli observable $T_a$ can be written as a product of some subset of the $2n+1$ pair-wise anticommuting observables $C_1,C_2,\dots,C_{2n+1}$ which generate $\mathcal{O}$ by taking pair-wise products. Since the product of all $2n+1$ of these Pauli operators is proportional to the identity, this representation is not unique. Each Pauli observable $a$ will have exactly two factorizations of the form $T_a\propto C_{\mu_1}C_{\mu_2}\cdots C_{\mu_k}$, one where $k$ is odd and one where $k$ is even. Therefore, without loss of generality we can restrict our attention to the factorizations where $k$ is even.

Suppose $T_{a}\propto C_{\mu_1}C_{\mu_2}\cdots C_{\mu_{2m}}$. In order for $I\subset E$ to contain $a$, $I$ must contain a set of $m$ generators, each of which is proportional to a product of a pair of operators $C_{\mu_1},C_{\mu_2},\dots$. Once these $m$ generators are specified, the remaining $n-m$ generators of $I$ must be chosen from products of pairs of operators from the remaining operators $\{C_1,C_2,\dots\}\setminus\{C_{\mu_1},C_{\mu_2},...\}$. Therefore, the number of isotropic subspaces $I$ containing $a$ is
\begin{widetext}
\begin{align}
    f(n,m)=&\left[\frac{1}{m!}\prod\limits_{j=0}^{m-1}\begin{pmatrix}2m-2j\\2\end{pmatrix}\right]\cdot\frac{1}{(n-m)!}\prod\limits_{k=0}^{n-m-1}\begin{pmatrix}2n+1-2m-2k\\2\end{pmatrix}\label{Equation:vertex degree function}\\
    =&\frac{1}{m!(n-m)!2^n}\left[\prod\limits_{j=1}^{2m}j\right]\cdot\left[\prod\limits_{j=1}^{2n+1-2m}j\right]\nonumber\\
    =&\frac{(2m)!(2n-2m)!}{m!(n-m)!2^n}\cdot(2n+1-2m).\nonumber
\end{align}
\end{widetext}
To show that $f(n,m)\ge2^n-1$ for all $n$, $m$, it suffices to show that $\log_2(2^nf(n,m))\ge2n$ for all $m$ and all sufficiently large $n$. Then the remaining cases of small $n$ can be checked directly.

Using Stirling's approximation
\begin{equation*}
    \sqrt{2\pi n}\left(\frac{n}{e}\right)e^{\frac{1}{12n+1}}<n!<\sqrt{2\pi n}\left(\frac{n}{e}\right)^ne^{\frac{1}{12n}}.
\end{equation*}
this can be lower bounded as
\begin{align*}
	\log_2(2^nf(n,m))\ge&m\log_2(m)+(n-m)\log_2(n-m)\\
	&+(2-\log_2(e))n-\frac{\log_2(e)}{6}.
\end{align*}
For details, see Appendix~\ref{Appendix:LowerBoundOnF}. This last expression is minimized for each $n$ at $m=n/2$. Therefore,
\begin{align*}
    \log_2(2^nf(n,m))\ge(2-\log_2(e))n + n\log_2(n/2) - \frac{\log_2(e)}{6}.
\end{align*}
For $n\ge6$ this is larger than $2n$. For the remaining cases $n=1,2,3,4,5$ it can be checked directly that $f(n,m)\ge2^n-1$. These cases are shown in Table~\ref{Table:New vertex proof - counting function small n}. Therefore, $f(n,m)\ge2^n-1$ for all $n$,$m$. This completes the proof.$\Box$

\begin{table}
    \centering
    \begin{tabular}{|c||c|c|c|c|c||c|}
        \hline
        $f(n,m)\;\;n\backslash m$ & 1 & 2 & 3 & 4 & 5 & $2^n-1$\\
        \hline\hline
        1 &   1 &     &     &     &     & 1 \\
        \hline
        2 &   3 &   3 &     &     &     & 3 \\
        \hline
        3 &  15 &   9 &  15 &     &     & 7 \\
        \hline
        4 & 105 &  45 &  45 & 105 &     & 15 \\
        \hline
        5 & 945 & 315 & 225 & 315 & 945 & 31 \\
        \hline
    \end{tabular}
    \caption{The function $f(n,m)$ of eq.~(\ref{Equation:vertex degree function}) evaluated for all $m$ and for $1\le n\le5$. This table, together with the calculation above shows that $f(n,m)\ge2^n-1$ for all $n$ and $m$.}
    \label{Table:New vertex proof - counting function small n}
\end{table}


\section{Discussion}\label{Section:Discussion}

In this work, we presented a novel quasiprobability representation of quantum computation with magic states based on generalized Jordan-Wigner transformations. We demonstrated that this representation has efficiently computable update rules with respect to Clifford gates and Pauli measurements. Moreover, it extends previous representations including those based on quasiprobabilistic decompositions in projectors onto stabilizer states~\cite{HowardCampbell2017}, and the CNC construction~\cite{RaussendorfZurel2020}. By leveraging this new construction we can efficiently simulate magic state quantum circuits on a larger class of input states than was previously known, thus pushing back the boundary between efficiently classically simulable, and potentially advantageous quantum circuits.

This model has a close connection to the probabilistic model of quantum computation based on the $\Lambda$ polytopes. Namely, for each number $n$ of qubits, it defines a new polytope contained inside of the $\Lambda$ polytope which shares some vertices with the $\Lambda$ polytope. These vertices include the previously known CNC vertices~\cite{RaussendorfZurel2020,ZurelRaussendorf2020,Heimendahl2019}, as well as some new infinite families of vertices as shown by Theorem~\ref{Theorem:NewLineGraphVertices}.

\paragraph*{Outlook.} Our results provide several avenues for future research. One possible direction is to look for more families of vertices of the $\Lambda$ polytopes which also have the line graph structure like those described in Theorem~\ref{Theorem:NewLineGraphVertices}.

A more speculative idea is to look at the potential application of these results for magic state distillation protocols. One of our results is a complete characterization of a new family of vertices of the $\Lambda$ polytope. By polar duality~\cite{ZurelHeimendahl2021} this also gives a full characterization of new families of facets of the stabilizer polytopes for every number of qubits. Facets of the two-qubit stabilizer polytope have in the past been linked to magic state distillation. The CNC type facets~\cite{RaussendorfZurel2020,Heimendahl2019}, as well as the new facets defined by Theorem~\ref{Theorem:NewLineGraphVertices} could be used in a similar way for any number of qubits.

\section*{Acknowledgements}
\noindent M.Z. and R.R. are funded by the National Science and Engineering Research Council of Canada (NSERC), in part through the Canada First Research Excellence Fund, Quantum Materials and Future Technologies Program, and the Alliance International program. This work is also supported by the Horizon Europe project FoQaCiA, a European-Canadian collaboration [NSERC funding reference number 569582-2021]. L.Z.C. acknowledges support from the Australian Research Council through project number DP220101771 and the Centre of Excellence in Engineered Quantum Systems (EQUS) project number CE170100009. We thank Stephen Bartlett for fruitful discussions.

\bibliography{refs}

\appendix
\onecolumngrid


\section{Graph theory results}\label{Appendix:GraphTheoryResults}
In the proof of Theorem~\ref{Theorem:NewLineGraphVertices}, we use some concepts from graph theory. A signed graph $G^\sigma$ is a undirected graph $G$ together with a sign function $\sigma:\text{Edge}(G)\rightarrow\mathbb{Z}_2$ that assigns to each edge $e$ of $G$ a sign $(-1)^{\sigma(e)}$. We say a graph has full rank if its adjacency matrix has full rank. The perrank of a graph $G$ is the order of the largest subgraph of $G$ which is a disjoint union of copies of $K_2$ and cycles. If $G$ has order $n$ we say $G$ has full perrank if $perrank(G)=n$.

We use the following Lemma from Ref.~\cite{AkbariNahvi2020}
\begin{Lemma}
    Let $G$ be a graph. Then there exists a sign function $\sigma$ for $G$ so that $G^\sigma$ has full rank if and only if $G$ has full perrank.
\end{Lemma}
In a bipartite graph, all cycles are even, so edges can be removed from any cycle of a bipartite graph to give copies of $K_2$ which cover the same vertices. Therefore, for bipartite graphs the existence of a $\{1,2\}$-factor of order $n$ implies the existence of a matching of size $n$. Thus we get the following corollaries
\begin{Corollary}
    Let $G$ be a bipartite graph with adjacency matrix
    \begin{equation}\label{Equation:Bipartite adjacency matrix}
        A = \begin{bmatrix}0&B\\B^T&0\end{bmatrix}.
    \end{equation}
    where $B$ is square. Then there exists a choice of signs $\sigma$ for $G$ so that $B$ has full rank if and only if there exists a perfect matching of $G^\sigma$.
\end{Corollary}

\begin{Corollary}\label{Corollary2}
    Let $G$ be a bipartite graph with adjacency matrix of the form eq.~\ref{Equation:Bipartite adjacency matrix} where $B$ is a (not necessarily square) $m\times n$ matrix with $m\le n$.  Then there exists a choice of signs $\sigma$ for $G$ so that $B$ has full rank if and only if there exists a matching of order $m$.
\end{Corollary}


\section{Lower bound on \texorpdfstring{$f(n,m)$}{f(n,m)}}\label{Appendix:LowerBoundOnF}
Here we prove the lower bound on $f(n,m)$ used in the proof of Theorem~\ref{Theorem:NewLineGraphVertices}, namely,
\begin{equation*}
	\log_2(2^nf(n,m))\ge m\log_2(m) + (n-m)\log_2(n-m) + (2-\log_2(e))n - \frac{\log_2(e)}{6}.
\end{equation*}
This proceeds by using the exact form of Stirling's formula:
\begin{align*}
	\log_2(2^nf(n,m))=&\log_2((2m)!)+\log_2((2n-2m)!)-\log_2(m!)-\log_2((n-m)!)+\log_2(2n+1-2m)\\
	\ge & \frac{\log_2(e)}{24m+1}     +\frac{1}{2}\log_2(2m)     -2m\log_2(e)     + 2m\log_2(2m)\\
	&+\frac{\log_2(e)}{24(n-m)+1} +\frac{1}{2}\log_2(2(n-m)) -2(n-m)\log_2(e) + 2(n-m)\log_2(2(n-m))\\
	&-\frac{\log_2(e)}{12m+1}     -\frac{1}{2}\log_2(m)      +m\log_2(e)      + m\log_2(m)\\
	&-\frac{\log_2(e)}{12(n-m)+1} -\frac{1}{2}\log_2(n-m)    +(n-m)\log_2(e)  + (n-m)\log_2(n-m) + \log_2(2n+1-2m)\\
	= & m\log_2(m) + (n-m)\log_2(n-m) + 2n - nlog_2(e) + \log_2(2n+1-2m)\\
	&+\log_2(e)\left[\frac{1}{24m+1}+\frac{1}{24(n-m)+1}-\frac{1}{12m}-\frac{1}{12(n-m)}\right]\\
	\ge & m\log_2(m) + (n-m)\log_2(n-m) + (2-\log_2(2n+1-2m)) + \log_2(e)\left[\frac{2}{24n+1}-\frac{1}{6}\right]\\
	\ge & m\log_2(m) + (n-m)\log_2(n-m) + (2-\log_2(e))n - \frac{\log_2(e)}{6}.
\end{align*}


\section{Lower bound on generalized robustness}

\begin{Lemma}\label{Lemma:RobustnessLowerBound}
    For all the vertices described by Theorem~\ref{Theorem:NewLineGraphVertices}, on any number of qubits $n$, the robustness $\mathfrak{R}(A_{\mathcal{O}}^{\eta})$ and the robustness of magic $\mathfrak{R}_S(A_{\mathcal{O}}^\eta)$, satisfy the inequalities

    \begin{equation}
        \mathfrak{R}_S(A_{\mathcal{O}}^\eta) \leq n^2 + \frac{n}{2} + 1
    \end{equation}
\end{Lemma}

\emph{Proof of Lemma~\ref{Lemma:RobustnessLowerBound}} Let $a, b \in \mathbb{R}^d$ and let $c \in \mathbb{R}^d$ collinear to $a$ and $b$ but not in the convex hull of $a$ and $b$. Then the negativity of $c$ relative to $a$ and $b$ is

\[ ||c||_1 = \frac{||a-c||_2 + ||b-c||_2}{||a-b||_2} \]

Let $c$ be a vertex of $\Lambda_n$ as given in Theorem~\ref{Theorem:NewLineGraphVertices}(in Cartesian coordinates). Let $a$ be the point on the stabiliser facet $1 + c \cdot x = 0$ closest to $c$ and let $b$ be the origin, these three points are all collinear. Then we have

\[ ||a-c||_2 = \frac{\frac{n(2n+1)}{4}+1}{\left( \frac{n(2n+1)}{4}\right)^{1/2}} = \left( \frac{n(2n+1)}{4}\right)^{1/2} + \left( \frac{n(2n+1)}{4}\right)^{-1/2} \]
\[ ||b-c||_2 = \left( \frac{n(2n+1)}{4}\right)^{1/2}\]
\[ ||a-b||_2 = \left( \frac{n(2n+1)}{4}\right)^{-1/2} \]

Thus

\[ \mathfrak{R}_S(c) \leq ||c||_1 = \frac{n(2n+1)}{2}+1 \]

\end{document}
