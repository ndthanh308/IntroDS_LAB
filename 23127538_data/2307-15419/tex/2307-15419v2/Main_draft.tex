\documentclass[11pt,a4paper]{article}
\pdfoutput=1
\usepackage{mysty}
\usepackage{amsfonts}
\usepackage{amsmath}
\usepackage{cancel}
\usepackage{graphicx}
\usepackage{hyperref}
\usepackage{multirow}
\usepackage{makecell}
\usepackage[table]{xcolor}
\usepackage{xspace}
\usepackage{placeins}
\usepackage{lmodern}
\usepackage[left=1.85in, right=-0.2in,top=1.8in,bottom=-0.2in]{geometry}

\definecolor{colorDG}{HTML}{008000}
\newcommand{\ak}[1]{{\color{brown}  #1}}
\newcommand{\vm}[1]{{\color{colorDG}  #1}}
\newcommand{\tp}[1]{{\color{red}  #1}}
\newcommand{\TPcom}[1]{{\color{brown}\bf  #1}}
\newcommand{\RF}[1]{{\color{blue}  #1}}%Reply Referee
\newcommand{\n}{(NEW)}
\newcommand{\sigbr}{$\sigma \cdot {\cal B}$\xspace}
\newcommand{\HEPfit}{\texttt{HEPfit}\xspace}
\newcommand{\nl}{{\mathnormal l}}
\newcommand{\rn}{{\tt r}}
\newcommand{\Mn}{{\tt M}}
\newcommand{\R}{\text{Re}}
\newcommand{\I}{\text{Im}}
\newcommand{\an}{\text{and}}
\newcommand{\wi}{\text{with}}
\newcolumntype{P}[1]{>{\centering\arraybackslash}p{#1}}

\hypersetup{pdfstartview=FitV,colorlinks=true,linkcolor=blue,citecolor=red,filecolor=black,urlcolor=blue}

\definecolor{color1}{HTML}{00dd00}
\definecolor{color2}{HTML}{dd0000}

\definecolor{color8TeV}{HTML}{ddddff}
\definecolor{color13TeV}{HTML}{eeee99}
\definecolor{color2TeV}{HTML}{9DCEBF}



\setcounter{tocdepth}{1}

\allowdisplaybreaks

\begin{document}

\title{Updated global fit of the aligned two-Higgs-doublet model with heavy scalars}

\author[a]{Anirban Karan,}
\emailAdd{kanirban@ific.uv.es} 


\author[b]{V\'ictor Miralles,}
\emailAdd{victor.miralles@roma1.infn.it}


\author[a]{Antonio Pich}
\emailAdd{antonio.pich@ific.uv.es}

\affiliation[a]{Institut de F\'isica Corpuscular (CSIC-UV), Parque Cient\'ifico, Catedr\'atico Jos\'e Beltr\'an 2, E-46980 Paterna, Spain}


\affiliation[b]{INFN, Sezione di Roma, Piazzale A. Moro 2, I-00185 Roma, Italy}

\abstract{An updated global fit on the parameter-space of the Aligned Two-Higgs-Doublet model has been performed with the help of the open-source package {\tt HEPfit}, assuming the Standard-Model Higgs to be the lightest scalar. No new sources of CP violation, other than the phase in the CKM matrix of the Standard Model, have been considered. A similar global fit was previously performed in Ref. \cite{Eberhardt:2020dat} with a slightly different set of parameters. Our updated fit incorporates improved analyses of the theoretical constraints required for perturbative unitarity and boundedness of the scalar potential from below, additional flavour observables and updated data on direct searches of heavy scalars at the LHC, Higgs signal strengths and electroweak precision observables. Although not included in the main fit, the implications of the CDF measurement of the $W^\pm$ mass are also discussed.


}

\preprint{{\raggedleft IFIC/23-30 \par}}

%\setcounter{tocdepth}{1}

\maketitle
\vspace*{0mm}
%%%%%%%%%%%%%%%%%%%%%%%%%%%%%%%%%%%%%%%%%%%%%%%%%%%%%%%%%
\section{Introduction}
\label{sec:Intro}

With the discovery of the Higgs boson \cite{ATLAS:2012yve,CMS:2012qbp}, the Standard Model (SM) has become a well-established theory describing the interactions among elementary particles in a very elegant way. Nonetheless, there are ample evidences indicating that the SM cannot explain all aspects of nature, and hence the existence of physics beyond the Standard Model (BSM) is indispensable. 
 In order to explain such BSM phenomena, the SM is usually enlarged 
 with some additional particles or gauge groups that respect all its fundamental principles. 
 While the augmentation of the SM with extra generations of quarks and leptons receives severe experimental constraints from unitarity-triangle data \cite{UTfit:2022hsi} or $Z$-boson branching fractions \cite{ALEPH:2005ab}, extensions with additional $SU(2)_L$ scalar doublets do not suffer from such stringent bounds.\footnote{In generic scalar extensions the tree-level value of the $\rho$ parameter shifts from unity, depending on the hypercharge and weak isospin of the extra scalars \cite{Diaz-Cruz:2003kcx}. However, it remains equal to 1
with additional doublets having the same hypercharge as the SM Higgs doublet.}
	
Among such scalar extensions, the simplest one is the two-Higgs-doublet model (THDM),  where one additional scalar doublet with the same quantum numbers than the SM Higgs doublet is appended \cite{Branco:2011iw,Gunion:1989we,Ivanov:2017dad}. In addition to the three needed Goldstone bosons, this model includes three neutral and one pair of charged scalars. Such rich scalar spectrum opens
various interesting possibilities such as new sources of CP violation \cite{Gunion:2005ja, Wu:1994ja, Keus:2015hva,Chen:2017com,Iguro:2019zlc}, axion-like phenomenology \cite{Kim:1986ax, Espriu:2015mfa, Celis:2014zaa}, dark matter aspects \cite{LopezHonorez:2006gr,Belyaev:2016lok,Tsai:2019eqi}, neutrino mass generation \cite{Ma:2006km, Hirsch:2013ola}, electroweak baryogenesis \cite{Turok:1990zg, Cline:2011mm,Fuyuto:2015jha}, stability of the scalar potential till the Planck scale \cite{Ferreira:2015rha,Das:2015mwa,Schuh:2018hig}, etc. Moreover, the THDM can also be thought of as a low-energy effective field theory framework for several models with larger symmetry groups (like supersymmetry).
	
One major shortcoming of the most general THDM is the emergence of tree-level flavour-changing neutral currents (FCNC), which are observed to be tightly constrained experimentally.
This problem is usually avoided by imposing
%%% different types of 
discrete $\mathcal Z_2$ symmetries 
so that each type of right-handed fermions couples to one scalar doublet only \cite{Glashow:1976nt,Paschos:1976ay}.
However, the absence of tree-level FCNC can be guaranteed 
%%%by imposing a much weaker condition: 
with a much weaker requirement: the alignment of Yukawa couplings in the flavour space, so that the interactions of the two scalar doublets acquire the same flavour structure \cite{Pich:2009sp,Pich:2010ic,Penuelas:2017ikk,Manohar:2006ga}. This provides a more generic theoretical framework, known as  
the `Aligned Two-Higgs Doublet model' (ATHDM), where flavour violation is minimal \cite{Chivukula:1987py,DAmbrosio:2002vsn} and (highly-suppressed) FCNC only appear at higher perturbative orders \cite{Pich:2009sp,Pich:2010ic,Ferreira:2010xe,Jung:2010ik,Braeuninger:2010td,Bijnens:2011gd,Li:2014fea,Abbas:2015cua,Botella:2015yfa,Penuelas:2017ikk,Gori:2017qwg}. In this model, all the fermion-scalar interactions become proportional to the masses of the corresponding fermions, leading to a
%%%to 
quite compelling 
phenomenology in  high-energy colliders \cite{ Abbas:2015cua, Celis:2013ixa, Celis:2013rcs} as well as in low-energy flavour experiments \cite{Jung:2010ik,Jung:2010ab,Jung:2012vu}. The ATHDM constitutes a generic platform for THDMs; all $\mathcal{Z}_2$-symmetric THDM scenarios can be explored as special cases of the ATHDM \cite{Pich:2009sp}. In addition, the ATHDM can accommodate new sources of CP violation beyond the CKM matrix in both the scalar and Yukawa sectors \cite{Pich:2009sp}.
	
The parameter space of the THDM has been extensively scrutinized in the literature, considering LHC data, LEP data, flavour bounds and theoretical constraints \cite{Eberhardt:2020dat,Jung:2010ik,Celis:2013rcs,Abbas:2015cua,Celis:2013ixa,Ilisie:2014hea,Chowdhury:2017aav, Haller:2018nnx, Cacchio:2016qyh,Chowdhury:2015yja,Eberhardt:2013uba,Wang:2013sha,Botella:2015hoa,Craig:2015jba,Bernon:2014nxa,Bernon:2015qea,Ilnicka:2015jba,Bernon:2015wef,Belusca-Maito:2016dqe,Dercks:2018wch,Ilnicka:2018def,Sanyal:2019xcp,Herrero-Garcia:2019mcy,Karmakar:2019vnq,Chen:2019pkq,Arco:2020ucn,Botella:2018gzy,Botella:2020xzf,Botella:2022rte,Botella:2023tiw,Athron:2021auq, Connell:2023jqq, Karan:2023xze}. A global fit of the ATHDM with no new sources of CP violation (and a slightly different set of parameters) was performed in Ref.~\cite{Eberhardt:2020dat}, taking into account various theoretical and experimental bounds. In this paper we reanalyse the global fit of this scenario in great detail, assuming that the SM Higgs boson is the lightest scalar. The fits are performed with the help of the open-source code {\tt HEPfit}  \cite{DeBlas:2019ehy}. Compared with the previous study, we have updated the Higgs signal strengths and added direct searches for all the scalars from the most recent LHC data. Additionally, we have improved the analysis of theoretical constraints by including \textit{necessary and sufficient} conditions for boundedness of the scalar potential from below (i.e. the potential never tends to negative infinity), and incorporated the branching fractions for semileptonic decays of $B$, $D$ and $K$ mesons in the flavour sector too. We have also analysed the 
%%%possible 
implications of the CDF measurement of the $W^\pm$ mass \cite{CDF:2022hxs}, %%%separately, 
although we have not included it in our global fits in view of its currently unresolved discrepancy with other measurements. A similar conservative attitude has been taken for
the muon $g-2$ anomaly \cite{Aoyama:2020ynm}, in view of the current controversy with lattice \cite{Borsanyi:2020mff,Ce:2022kxy,ExtendedTwistedMass:2022jpw} and $\tau$-decay \cite{Davier:2010fmf,Pich:2013lsa} data; although not included in the global fit, we have also studied the ensuing constraints on the parameter space of the ATHDM.
 
The paper is organized in the following way. Next section (Section \ref{sec:Model}) briefly describes the ATHDM scenario. The set up for the global fit is discussed in Section \ref{sec:set_up}. While in Section \ref{sec:constraints}, we illustrate different theoretical as well as experimental constraints that are considered for this study, in the subsequent section (Section \ref{sec:results}) we present the numerical results from the fits. Finally we conclude in Section \ref{sec:conclusion}. All the data and various other details used in the global fit are incorporated in the Appendix section.


\section{The ATHDM model}

\label{sec:Model}

\subsection{Scalar Sector}

We extend the SM with a second complex scalar doublet having the same hypercharge as the SM scalar doublet, i.e. $Y=1/2$. %%% During the spontaneous breaking of the electroweak symmetry, 
The neutral components of both scalar doublets may acquire non-zero complex vacuum expectation values (vev); nonetheless, with a suitable $SU(2)_L\otimes U(1)_Y$ global transformation, one can always rotate the basis of the scalar space such that only the neutral component of the first doublet acquires a non-zero (real) vev.
Working in this so-called Higgs-basis, we can write down the two doublet scalar fields 
%%% after symmetry breaking 
as:
\begin{equation}
\Phi_1=\frac{1}{\sqrt 2}\begin{pmatrix}
\sqrt 2\;G^+\\
S_1+v+i\, G^0
\end{pmatrix}\, ,\qquad\qquad \Phi_2=\frac{1}{\sqrt 2}\begin{pmatrix}
\sqrt 2\;H^+\\
S_2+i\, S_3
\end{pmatrix}\, ,
\end{equation} 
where $\Phi_1$ gets the vev  $v=246$ GeV. The components $G^\pm$ and $G^0$ act as Goldstone bosons, providing the masses to the $W^\pm$ and $Z$ bosons. Thus, we are left with one pair of charged scalars $H^\pm$, two neutral scalars $S_{1,2}$ and one neutral pseudoscalar $S_3$. The three neutral particles $S_j$ mix with each other through an orthogonal transformation to produce the mass eigenstates $\varphi_i^0$.
%%%$h, H\text{ and }A$. 
The explicit form of this orthogonal transformation depends on the scalar potential. 
%%% It is important to mention that if CP is not conserved in the scalar potential, 
In general, if the scalar potential is not invariant under the CP symmetry,
all the three $S_j$ fields mix together giving no definite CP to the mass eigenstates. 
%%% However, for a CP-symmetric scalar potential, $S_3$ does not mix with $S_{1,2}$ resulting in:


The most general scalar potential, allowed by the SM gauge symmetry, takes the form:
\begin{align}
\label{eq:pot}
V&=\mu_1\,\Phi_1^\dagger \Phi_1+\mu_2\,\Phi_2^\dagger \Phi_2+ \Big[\mu_3\,\Phi_1^\dagger \Phi_2+h.c.\Big]+\frac{\lambda_1}{2}\,(\Phi_1^\dagger \Phi_1)^2+\frac{\lambda_2}{2}\,(\Phi_2^\dagger \Phi_2)^2+\lambda_3\,(\Phi_1^\dagger \Phi_1)(\Phi_2^\dagger \Phi_2)\nonumber\\
&+\lambda_4\,(\Phi_1^\dagger \Phi_2)(\Phi_2^\dagger \Phi_1)+\Big[\Big(\frac{\lambda_5}{2}\,\Phi_1^\dagger \Phi_2+\lambda_6 \,\Phi_1^\dagger \Phi_1 +\lambda_7 \,\Phi_2^\dagger \Phi_2\Big)(\Phi_1^\dagger \Phi_2)+ \mathrm{h.c.}\Big]\, ,
\end{align}
where $\mu_3,\lambda_5,\lambda_6$ and $\lambda_7$ are complex parameters and the rest are real. Minimizing the potential with respect to the neutral fields at their corresponding minima, one obtains:
\begin{equation}
v^2=-\frac{2\mu_1}{\lambda_1}=-\frac{2\mu_3}{\lambda_6}\, ,
\end{equation}
implying that $\mu_1$ and $\mu_3$ are not independent quantities. Redefining the phase of $\Phi_2$, 
%%% with one phase 
one can make any one of the three parameters $\lambda_5,\lambda_6$ or $\lambda_7$ real. Therefore, 
%%% to describe the generic potential of the THDM, one needs eleven real independent  
the scalar potential involves eleven independent real
parameters: $\mu_2,\; v,\; \lambda_{1,2,3,4},\; |\lambda_{5,6,7}|$ and the two relative phases between $\lambda_{5,6,7}$. 

The quadratic terms of the potential, 
%%%will give rise to the scalar masses, these terms 
which give rise to the scalar masses, can be written as:
\begin{equation}
    V_M = \left(\mu_2+\frac{1}{2}\lambda_3 v^2\right) H^+H^-+\frac{1}{2}\begin{pmatrix} S_1 & S_2 & S_3 \end{pmatrix} \mathcal{M} \begin{pmatrix} S_1 \\ S_2 \\ S_3 \end{pmatrix}\, ,
\end{equation}
with   
\begin{equation}
\mathcal{M}=\begin{pmatrix} v^2\, \lambda_1  && v^2\, \rm{Re}(\lambda_6) && -v^2\, \rm{Im}(\lambda_6) \\
    v^2\, \rm{Re}(\lambda_6) && \mu_2+\frac{1}{2}v^2\{\lambda_3+\lambda_4+\rm{Re}(\lambda_5)\} && -\frac{1}{2}v^2\, \rm{Im}(\lambda_5) \\
    -v^2\, \rm{Im}(\lambda_6) && -\frac{1}{2}v^2\, \rm{Im}(\lambda_5) && \mu_2+\frac{1}{2}v^2\{\lambda_3 + \lambda_4 - \rm{Re}(\lambda_5) \} \end{pmatrix}\, .
    \label{eq:massmatrix}
\end{equation}

In order to obtain the physical mass eigenstates,
we need to diagonalise the neutral mass matrix~\eqref{eq:massmatrix}. Imposing CP conservation in the scalar sector, all the parameters in our potential become real, reducing to nine the number of independent inputs.
Moreover, in this case the mass eigenstates have definite CP: we get two CP-even ($h$, $H$) and one CP-odd ($A$) fields.
From the mass matrix, we explicitly see that in this case $S_3=A$ does not mix with the other scalars
%%% , as mentioned before, 
and we only need to diagonalise a $2\times 2$ matrix.
%%%in order to find the scalar masses. 
The squared masses are then given by:
\begin{equation}
M_{H^\pm}^2=\mu_2+\frac{\lambda_3}{2}\,v^2\, ,\qquad M_{h,H}^2=\frac{1}{2}\,(\Sigma\mp\Delta),\qquad M_A^2=M_{H^\pm}^2+\frac{v^2}{2}\,(\lambda_4-\lambda_5)\, ,
\end{equation}
with
\begin{equation}
\Sigma=M_{H^\pm}^2+\Big(\lambda_1+\frac{\lambda_4}{2}+\frac{\lambda_5}{2}\Big)\,v^2\qquad \text{ and }\qquad \Delta=\sqrt{\big(\Sigma-2\lambda_1 v^2\big)^2+4\,\lambda_6^2\,v^4}\, .
\end{equation}
%
Note also that, since the trace of the mass matrix must be invariant, the scalar masses must satisfy the following relation:
\begin{equation}
    M_{h}^2+M_{H}^2+M_{A}^2=2 M_{H^\pm}^2+v^2(\lambda_1+\lambda_4)\, ,
    \label{eq:tracematrix}
\end{equation}
which can also be easily confirmed by substituting the expressions of the masses.
%
The mixing between the two CP-even neutral scalars,
\begin{equation}
\begin{pmatrix}
h\\H
\end{pmatrix}=\begin{pmatrix}
\cos\tilde{\alpha}& \sin\tilde{\alpha}\\ -\sin\tilde{\alpha}&\cos\tilde{\alpha}
\end{pmatrix}\begin{pmatrix}
S_1\\S_2
\end{pmatrix} 
%%%\qquad\qquad \text{and}\qquad\qquad A=S_3
\, ,
\end{equation}
%
%%%The mixing angle of the CP-even neutral scalars, i.e. $\tilde\alpha$, can be written as:
is given by:
\begin{equation}
\tan\tilde\alpha=\frac{M_h^2 -v^2\,\lambda_1}{v^2\,\lambda_6}=\frac{v^2\,\lambda_6}{v^2\,\lambda_1-M_H^2}\, .
\label{eq:mix_ang_lam}
\end{equation}


 Using the above equations,  we can trade five of the nine parameters of the potential ($\mu_2$, $v$ and $\lambda_{1-7}$) by the four scalar masses ($M_{H^\pm},\,M_{h},\,M_{H},\,M_{A}$) and the mixing angle $\tilde{\alpha}$, which are more physical. Clearly $\lambda_2$ must be kept since it does not relate with any of the new parameters. Since $\mu_2$ and $\lambda_3$ always appear in the same combination ($M_{H^\pm}$), we must keep one of them which we choose to be $\lambda_3$. From Eq.~\eqref{eq:mix_ang_lam} we can relate $\lambda_1$ and $\lambda_6$ with $\tilde{\alpha}$, $M_h^2$, $M_H^2$ and $v$. Then we can use Eq.~\eqref{eq:tracematrix} to obtain $\lambda_4$ and $\Sigma$ to obtain $\lambda_5$. With this procedure we
 %%%will then 
 trade $\mu_2$, $\lambda_1$, $\lambda_4$, $\lambda_5$ and $\lambda_6$ by the four masses and the mixing angle. Our choice for the nine input parameters 
 %%% will therefore be 
is then: $v,\,M_{H^\pm},\,M_{h},\,M_{H},\,M_{A},\,\tilde{\alpha},\,\lambda_2,\,\lambda_3$ and $\lambda_7$. 
 %%% The relation of the parameters of the potential with these ones is given by
The remaining parameters of the scalar potential can be easily obtained in terms of these inputs:
\begin{align}
\nonumber   & \mu_2=M_{H^\pm}^2-\frac{\lambda_3}{2}v^2\, ,\qquad \lambda_1=\frac{M_h^2+M_H^2\tan^2{\tilde{\alpha}}}{v^2(1+\tan^2{\tilde{\alpha}})}\, ,\qquad \lambda_6=\frac{(M_h^2-M_H^2)\tan{\tilde{\alpha}}}{v^2(1+\tan^2{\tilde{\alpha}})}\, ,\\
    \lambda_4=&\frac{1}{v^2}\left( M_h^2+M_A^2-2M_{H^\pm}^2+\frac{M_H^2-M_h^2}{1+\tan^2{\tilde{\alpha}}}\right) ,\qquad \lambda_5=\frac{1}{v^2}\left( \frac{M_H^2+M_h^2\tan^2{\tilde{\alpha}}}{1+\tan^2{\tilde{\alpha}}}-M_A^2\right) .
\end{align}



While considering interactions of neutral scalars with gauge bosons, $S_1$ plays the role of the SM Higgs boson. This implies that
the couplings of the scalar mass eigenstates with a pair of gauge bosons are given by:
\begin{equation}
g_{hVV}=\cos\tilde{\alpha}\;g_{hVV}^{SM}\, , \qquad\qquad g_{HVV}=-\sin\tilde{\alpha}\;g_{hVV}^{SM}\, , \qquad\qquad g_{AVV}=0\, ,
\label{eq:Higgs_weak_boson_couplings}
\end{equation}
where $VV\equiv W^+W^-, ZZ$.

\subsection{Yukawa Sector}
%%%After electroweak symmetry breaking, 
The interactions of the fermion mass eigenstates with the scalar fields read:
\begin{align}
-\mathcal L_Y\, =\, &\Big(1+\frac{S_1}{v}\Big)\left\{\bar u_L\,M_u\,u_R+\bar d_L\,M_d\,d_R+\bar\nl_L\, M_\nl\,\nl_R\right\}\nonumber\\
&+\frac{1}{v}\, (S_2+i S_3) \left\{\bar u_L\,Y_u\,u_R+\bar d_L\,Y_d\,d_R+\bar\nl_L\, Y_\nl\,\nl_R\right\}\\
&+\frac{\sqrt 2}{v}\, H^+ \left\{\bar u_L\,V\,Y_d\,d_R-\bar u_R \,Y_u^\dagger\,V\,d_L+\bar\nu_L\, Y_\nl\,\nl_R\right\} + \mathrm{h.c.}\, ,\nonumber
\end{align}
where the generation indices are suppressed and the subscripts $L,R$ denote the usual left and right chiral fields. Here, $M_f\;(f\equiv u,d,\nl)$ are diagonal mass matrices for the up-type quark, down-type quarks and charged leptons, respectively, 
originated through the Yukawa interactions with the doublet $\Phi_1$.
$Y_f$ are the Yukawa matrices parametrizing the fermionic couplings with the second doublet $\Phi_2$, and $V$ is the usual CKM matrix required for the quark mixing. In general, 
$Y_f$ could be arbitrary $3\times3$ complex matrices, which leads to unwanted FCNCs at tree level. This can be easily avoided
imposing the alignment condition of $M_f$ and $Y_f$ in flavour space \cite{Pich:2009sp,Pich:2010ic}, i.e.
\begin{equation}\label{eq:Yalignment}
Y_u=\varsigma^*_u\,M_u \qquad\qquad \text{and} \qquad\qquad Y_{d,\nl}=\varsigma_{d,\nl}\,M_{d,\nl}\, ,
\end{equation}
where the Yukawa alignment parameters $\varsigma_f$ could be arbitrary complex numbers. In terms of the scalar and fermion mass eigenstates, the Yukawa Lagrangian takes then the form:
\begin{equation}
-\mathcal L_Y=\sum_{i,f}\Big(\frac{y_f^{\varphi^0_i}}{v}\Big)\,\varphi^0_i\,\Big[\bar f M_f \mathcal{P}_R f\Big]+\Big(\frac{\sqrt 2}{v}\Big) H^+\,\Big[\bar u\,\big\{\varsigma_d V M_d \mathcal{P}_R-\varsigma_u M_u^\dagger V\mathcal{P}_L\big\}\, d+\varsigma_\nl\, \bar \nu M_\nl \mathcal P_R \nl\Big] + \mathrm{h.c.}\, ,
\end{equation}
where $\mathcal P_{L,R}$ are chirality projection operators
and $\varphi_i^0$ 
%%%\,(\equiv h,H,A)$ 
are the scalar mass eigenstates.

In the following, we will assume a CP-conserving potential and that there are no additional sources of CP violation beyond the CKM phase, i.e. we will only consider real alignment parameters. The Yukawa couplings of the neutral scalars are then given by:
%%%and their corresponding Yukawa couplings $(y_f^{\varphi^0_i})$ are given by:
\begin{eqnarray}
&y_{u}^H=-\sin\tilde\alpha+\varsigma_{u}\,\cos \tilde\alpha\, , \qquad\quad y_{u}^h=\cos\tilde\alpha+\varsigma_{u}\,\sin \tilde\alpha\, , \qquad\quad
y_{u}^A= -i\varsigma_{u}\, ,&
\label{eq:Higgs_yuk_up}
\\
&y_{d,\nl}^H=-\sin\tilde\alpha+\varsigma_{d,\nl}\,\cos \tilde\alpha\, , \quad\quad\; \;y_{d,\nl}^h=\cos\tilde\alpha+\varsigma_{d,\nl}\,\sin \tilde\alpha\, , \quad\quad\;  
y_{d,\nl}^A= i\varsigma_{d,\nl}\,.&
\label{eq:Higgs_yuk_down}
\end{eqnarray}

%%%\RF{It is worth mentioning that all the alignment parameters $\varsigma_f$ become real when no CP-violation beyond the phase in CKM matrix is assumed.} 

The usual THDM scenarios based on $\mathcal{Z}_2$ symmetries are just particular cases of the more general ATHDM framework; they can be retrieved by imposing $\mu_3=\lambda_6=\lambda_7=0$ along with the following conditions: 
\begin{align}
\label{eq:THDM_types}
&\text{Type I:\;\;} \varsigma_{u}=\varsigma_d=\varsigma_\nl=\cot\beta,\quad \text{Type II:\;\;} \varsigma_{u}=-\frac{1}{\varsigma_d}=-\frac{1}{\varsigma_\nl}=\cot\beta\, ,\quad 
\text{Inert:\;\;}\varsigma_{u}=\varsigma_d=\varsigma_\nl=0\, ,
\nonumber\\
&\text{Type X:\;\;} \varsigma_{u}=\varsigma_d=-\frac{1}{\varsigma_\nl}=\cot\beta 
\qquad \text{and}\qquad
\text{Type Y:\;\;} \varsigma_{u}=-\frac{1}{\varsigma_d}=\varsigma_\nl=\cot\beta \, .
\end{align}
%
The alignment requirement (\ref{eq:Yalignment}) remains stable under renormalisation when it is protected by $\mathcal{Z}_2$ symmetries \cite{Ferreira:2010xe}. Otherwise, higher-order quantum corrections create a misalignment of $M_f$ and $Y_f$ that generates loop-suppressed FCNC effects. However, the special Yukawa structure of the ATHDM strongly constrain the possible FCNC interactions, making those effects numerically small \cite{Pich:2009sp,Pich:2010ic,Jung:2010ik}. Assuming exact alignment at some high-energy scale (even at the Planck mass), the small misalignment generated by running down to low energies remains well below the current experimental limits \cite{Braeuninger:2010td,Bijnens:2011gd,Penuelas:2017ikk,Gori:2017qwg}.



\section{Fit set up}

\label{sec:set_up}
Our numerical analyses have been performed with the open-source {\tt HEPfit} package \cite{DeBlas:2019ehy}. This code has been widely used,
due to its efficiency and versatility that allows making global fits, both within the SM \cite{deBlas:2021wap} or in general BSM extensions such as the SM effective field theory \cite{Durieux:2019rbz,Miralles:2021dyw} or particular models of new physics  \cite{Eberhardt:2020dat,Eberhardt:2021ebh}, as it is the case of this paper. 
{\tt HEPfit} works within the Bayesian statistics framework and, therefore, we need to choose carefully the priors of our variables. We have a total of ten new degrees of freedom with respect to the SM:
the physical masses of the additional scalars ($M_{H^\pm},\,M_{H}$ and $M_{A}$), the mixing angle of the CP-even neutral scalars ($\tilde{\alpha}$), three quartic couplings of the potential ($\lambda_2$, $\lambda_3$ and $\lambda_7$) and the three Yukawa alignment parameters ($\varsigma_u$, $\varsigma_d$ and $\varsigma_\nl$). Since our fits include all the available information, we have used uniform distributions as priors. In general, the priors are chosen to cover the region of the parameters that is physically relevant. For the mixing angle we have chosen the prior in such a way that at least the 5$\sigma$ region of the posterior probability is contained within the selected range. The quartic couplings are mainly constrained from theory assumptions, which impose a hard cut in the values that these parameters can take. In this case we have chosen a prior wide enough to include all points allowed by the theory constraints. The priors of the Yukawa couplings were also set within the limits allowed by perturbative constraints. 

There is a huge freedom when choosing ranges for the scalar masses. In this analysis we are assuming that the SM Higgs is the lightest scalar, i.e. that 
%%%the masses of 
$M_H$, $M_A$ and $M_{H^\pm}$ are larger than 125 GeV. 
%%% It is worth mentioning that 
The complementary scenario where the SM Higgs is not the lightest scalar  is phenomenologically very intriguing and capable of displaying very distinct collider signatures at the LHC \cite{Bernon:2014nxa,Bernon:2015qea,Bernon:2015wef}. However, these signatures and the overall phenomenology %%% might be very pronounced depending 
are strongly dependent on the assumed hierarchy of scalar masses
(which BSM particles are considered to be lighter than the SM Higgs). Moreover, a detailed analysis of the different possible scenarios requires the inclusion of additional experimental data from LEP and flavour factories, and a proper theoretical treatment of hadronic resonances in the mass region below 10 GeV.
%Scrutiny of this scenario requires consideration of additional experimental data from LEP and Belle involving more care in the theoretical side while dealing with resonances in the mass region of below 10 GeV. 
These intricacies for the light BSM particles are beyond the scope of this paper and therefore, here we focus only on the heavy BSM scenario.\footnote{In a forthcoming publication we will soon address the possibility of lighter scalars.} 

Higher scalar masses are obviously preferred by the data because no clear deviations from the SM have been observed so far.
Taking into account that many direct searches have been studying mass ranges up to 1~TeV, this number seems a reasonable higher cut-off for our global analysis. However, we will also provide some results in which the highest value of the scalar masses have been set to 1.5~TeV, so that we can get a feeling on how much our results depend on the priors. Finally, we would also like to comment that, instead of taking the masses as fundamental parameters, using the masses squared could also be a well-justified choice. However, as can be seen in Ref.~\cite{Eberhardt:2020dat}, when using a uniform distribution for the masses squared the results depend much more on the ranges of masses analysed. Therefore, we have decided to 
impose our priors linearly on the scalar mass parameters.
The chosen priors can be found in Table~\ref{tab:priors}.


\begin{table}[htb]
\begin{center}
\begin{tabular}{|P{1.1cm}|P{1.1cm}|P{1.1cm}|P{1.1cm}|P{1.1cm}|P{1.1cm}|P{1.1cm}|P{1.1cm}|P{1.1cm}|P{1.1cm}|P{1.1cm}|P{1.1cm}|}
\hline
\multicolumn{12}{|c|}{Priors} \\
\hline
\hline
\multicolumn{4}{|c}{$M_{H^\pm} \subset$ [0.125, 1.0\, (1.5)] TeV} & \multicolumn{4}{|c|}{$M_{H} \subset$  [0.125, 1.0\, (1.5)] TeV} & \multicolumn{4}{c|}{$M_{A} \subset$  [0.125, 1.0\, (1.5)] TeV} \\
\hline
\multicolumn{4}{|c}{$\lambda_2 \subset$ [0, 11]} & \multicolumn{4}{|c|}{$\lambda_3 \subset$ [-3, 17]}  & \multicolumn{4}{c|}{$\lambda_7 \subset$ [-5, 5]}  \\
\hline
\multicolumn{3}{|P{3.3 cm}}{$\tilde{\alpha} \subset$ [-0.16, 0.16]} & \multicolumn{3}{|P{3.3 cm}|}{$\varsigma_u \subset$ [-1.5, 1.5]} & \multicolumn{3}{P{3.3 cm}|}{$\varsigma_d \subset$ [-50, 50]} & \multicolumn{3}{P{3.3 cm}|}{$\varsigma_\nl \subset$ [-100, 100]} \\  
\hline
\end{tabular}
\caption{Priors chosen for the new-physics parameters.}
\label{tab:priors}
\end{center}
\end{table}


\section{Fit constraints}
\label{sec:constraints}


We aim to use as much information as we can in order to constrain the parameter space of the ATHDM. We have combined in this work the whole set of theoretical constraints with all relevant experimental limits coming from LHC direct  searches, Higgs data, electroweak precision data, and flavour observables. 

\subsection{Theoretical considerations}

Regarding theoretical constraints, the two conditions that are usually demanded are: a scalar potential bounded from below and perturbative unitarity of the $S$ matrix. The requirement that the scalar potential is bounded from below indicates that it should not go to large negative values, which would make it unstable, for any configuration of the fields. 
For this purpose, it is useful to recast the scalar potential $V$ in the following Minkowskian form \cite{Ivanov:2006yq}:
\begin{equation}
\label{eq:sc_pot_2}
     V=-\,\Mn_\mu\,{\rn}^\mu + \frac{1}{2}\,\Lambda^{\mu}_{\phantom{\mu}\nu}\, \rn_\mu\,\rn^\nu\, ,
\end{equation}
with
\begin{eqnarray}
\quad &\Mn_\mu\, =\, \Big(-\frac{\mu_1+\mu_2}{2},\,-\,\R\,\mu_3, \,\I\,\mu_3,\, -\,\frac{\mu_1-\mu_2}{2}\Big)\, , &
 \nonumber\\  
 &\rn^\mu\, =\, \Big(|\Phi_1|^2+|\Phi_2|^2, \, 2\,\R (\Phi_1^\dagger\Phi_2), \, 2\,\I (\Phi_1^\dagger\Phi_2), \, |\Phi_1|^2-|\Phi_2|^2\Big)\, , 
\end{eqnarray}
and
 \begin{equation}
 \Lambda^{\mu}_{\phantom{\mu}\nu}\, =\,\frac{1}{2}\,\begin{pmatrix}
     \;\frac{1}{2}(\lambda_1+\lambda_2)+\lambda_3 && \R (\lambda_6+\lambda_7) && -\,\I (\lambda_6+\lambda_7) && \frac{1}{2}(\lambda_1-\lambda_2)\\
     - \R (\lambda_6+\lambda_7) && -\lambda_4-\R \lambda_5&& \I \lambda_5 && -\, \R(\lambda_6-\lambda_7)\\
     \I (\lambda_6+\lambda_7) && \I \lambda_5&& -\lambda_4+\R \lambda_5 &&  \I(\lambda_6-\lambda_7)\\
     -\frac{1}{2}(\lambda_1-\lambda_2)&& -\R (\lambda_6-\lambda_7) && \I (\lambda_6-\lambda_7)&&-\frac{1}{2}(\lambda_1+\lambda_2)+\lambda_3\;\;
 \end{pmatrix}\, .
\end{equation}
Diagonalisation of the mixed-symmetric matrix\footnote{Actually $\Lambda_{\mu\nu}$ is a symmetric matrix and can be diagonalized by a $SO(1,3)$ transformation.} $\Lambda_{\phantom{\mu}\nu}^{\mu}$ produces one ``timelike'' ($\Lambda_0$) and three ``spacelike'' ($\Lambda_{1,2,3}$) eigenvalues.\footnote{The attributes ``timelike'' and ``spacelike'' are related to the corresponding eigenvectors.} The scalar potential remains bounded from below 
%%% only
when the following two conditions are satisfied \cite{Ivanov:2006yq,Ivanov:2015nea}:
\begin{enumerate}
    \item All the eigenvalues are real,
    \item $\Lambda_0>0$ with $\Lambda_0>\Lambda_i$ $\forall\;i\in \{1,2,3\}$.
\end{enumerate}
Several \textit{necessary} conditions for the scalar potential to be bounded from below have been listed in Ref.~\cite{Bahl:2022lio}. However, they are comprehended in the two above-mentioned \textit{necessary and sufficient} conditions.

One additional constraint could be imposed on the quartic couplings by demanding that the vacuum of the scalar potential is a stable neutral minimum.\footnote{This requirement is a bit more restrictive. It can be relaxed by allowing a meta-stable vacuum with a transition time to the true vacuum larger than the age of the universe.} For this purpose one defines the discriminant of the matrix $\xi\, {\mathbb I_4}-\Lambda^{\mu}_{\phantom{\mu}\nu}$ as:
\begin{equation}
    D=-\prod_{k=0}^{3}(\xi-\Lambda_k) \qquad \wi \qquad  \xi=\frac{M_{H^\pm}^2}{v^2}\,,
\end{equation}
 where the Lagrange multiplier $\xi$ is determined by minimizing the scalar potential $V$ with the constraint\footnote{The condition $\rn^\mu\, \rn_\mu=0$ ensures the vacuum to be a charge-neutral minimum. It has been proved that neutral and charge-breaking vacua cannot coexist in any THDM \cite{Ferreira:2004yd,Barroso:2013awa}.}  $\rn^\mu\, \rn_\mu=0$. The condition for a global minimum is found to be \cite{Ivanov:2015nea}: $D>0$, or
$D<0$ with $\xi>\Lambda_0$.
In our analysis we have included this condition, requiring a stable neutral minimum. However, removing this constraint does not change our results significantly.


The unitarity of the $S$ matrix ensures that, in order to conserve the total probability,  scattering amplitudes do not grow monotonically with energy. Since unitarity emerges from the basic formulation of QFT, it is bound to hold in a complete renormalisable theory while dealing with the full $S$ matrix. However, the unitarity bound does not need to be satisfied by the perturbative calculation of the $S$ matrix to any particular order. The stronger requirement of \textit{perturbative unitarity} enforces the unitarity of the $S$ matrix to be obeyed at every order of perturbation theory. It indirectly indicates that the couplings, in terms of which the perturbative expansion is performed, are not very large and one can safely neglect higher-order contributions. Therefore, imposition of perturbative unitarity on all the $2\to2$ scattering amplitudes involving the scalars 
(both the massive ones and the Goldstone bosons, which account for the $W_L^\pm$ and $Z_L$)
will restrict the quartic couplings $\lambda_i$, ensuring that the perturbative expansion of the $S$ matrix does not diverge at high energies. 
\textit{Tree-level unitarity} is enforced on the $2\to2$ scattering matrix of scalars by demanding that:
\begin{equation}
\label{eq:pert}
    (a_j^{0})^2\leq \frac{1}{4} \qquad\text{with}\qquad (\mathbf{a_0})_{i,f}=\frac{1}{16\pi s}\int_{-s}^{0} dt \;\mathcal M_{i\to f}(s,t)\, ,
\end{equation}
where $\mathbf{a_0}$ is the matrix of tree-level partial-wave amplitudes and $a_j^{0}$  the corresponding eigenvalues in the $j^{th}$ partial wave.
Nevertheless, while considering the scattering of scalars at very high energy, only the $S$-wave amplitude with $j=0$ becomes most significant at tree level. 
Constructing two-scalar scattering states with definite hypercharge and weak isospin $(Y,I)$ and grouping the ones with the same set of quantum numbers,
%%%Depending on the total charge, hypercharge and weak isospin of the scattering states, 
the 25-dimensional\footnote{For any THDM, there are fourteen neutral, eight single-charged and three doubly-charged two-body scalar scattering states possible.} matrix $\mathbf{a_0}$ can be expressed in a block-diagonal form:
\begin{eqnarray}
&\mathbf{a_0^{++}}=\frac{1}{16\pi}\,  X_{(1,1)}\; ,\quad 
\mathbf{a_0^+}=\frac{1}{16\pi}\, \text{diag}\, [X_{(0,1)},\, X_{(1,0)},\, X_{(1,1)}]\; ,&\nonumber\\
&\mathbf{a_0^0}=\frac{1}{16\pi}\, \text{diag}\, [X_{(0,0)},\, X_{(0,1)},\, X_{(1,1)},\, X_{(1,1)}]\; ,&   
\end{eqnarray}
where the superscript indicates the total charge of the initial or final states. Thus the scattering matrix  $\mathbf{a_0}$ is comprised of the following four submatrices \cite{Ginzburg:2005dt,Bahl:2022lio}:
\begin{eqnarray}
\label{eq:pert1}
    &X_{(1,0)}=\lambda_3-\lambda_4\, ,\qquad X_{(1,1)}=\begin{pmatrix}\lambda_1&&\lambda_5&&\sqrt 2 \lambda_6\\
    \lambda_5^*&&\lambda_2&&\sqrt 2 \lambda_7^*\\
    \sqrt 2\lambda_6^*&&\sqrt 2 \lambda_7^*&&\lambda_3+\lambda_4\end{pmatrix} , \qquad X_{(0,1)}=\begin{pmatrix}
        \lambda_1&&\lambda_4&&\lambda_6&&\lambda_6^*\\   \lambda_4&&\lambda_2&&\lambda_7&&\lambda_7^*\\
       \lambda_6^*&&\lambda_7^*&&\lambda_3&&\lambda_5^*\\  
       \lambda_6&&\lambda_7&&\lambda_5&&\lambda_3\\  
    \end{pmatrix},&\nonumber\\
    &X_{(0,0)}=\begin{pmatrix}
        3\lambda_1&&2\lambda_3+\lambda_4&&3\lambda_6&&3\lambda_6^*\\   
        2\lambda_3+\lambda_4&&3\lambda_2&&3\lambda_7&&3\lambda_7^*\\
       3\lambda_6^*&&3\lambda_7^*&&\lambda_3+2\lambda_4&&3\lambda_5^*\\  
       3\lambda_6&&3\lambda_7&&3\lambda_5&&\lambda_3+2\lambda_4\\  
    \end{pmatrix}.&   
\end{eqnarray}
The perturbative-unitarity condition~(\ref{eq:pert}) can be traded in terms of the eigenvalues ($e_i$) of the above four submatrices by demanding:
\begin{equation}
\label{eq:pert2}
    |e_i|< 8\pi.
\end{equation}

%One could also consider the scattering amplitudes of $W$ and $Z$ bosons, which are dominated at high energies by their longitudinal components. For example, imposing perturbative unitarity on the high-energy behaviour of $W_L^+ W_L^- \to W_L^+ W_L^-$ leads to the following inequality \cite{Logan:2022uus}:
%\begin{equation}
%    \sin^2 \tilde \alpha \,\leq\, \frac{4\pi v^2 - m_h^2}{m_H^2- m_h^2}\, .
%\end{equation}
%However, due to the \textit{equivalence theorem} \cite{Cornwall:1974km}, the high-energy scattering amplitudes of $W_L$ and $Z_L$, which show monotonic growths with energy, can be estimated by replacing them with the corresponding Goldstone bosons. Therefore they are already taken into account in Eq.~(\ref{eq:pert2}).

On the other hand, considering the coupling of fermions to the charged scalars, we vary the value of $\varsigma_f$ in the perturbative range satisfying $\sqrt{2}\,|\varsigma_f| m_f/v<1$. A more detailed study on the perturbativity of the alignment parameters could be performed based on the procedure mentioned in Ref. \cite{Allwicher:2021rtd}, which would extend the range of $\varsigma_f$ a bit more. However, for our analysis it is sufficient to consider the above inequality.

\subsection{Direct searches}
\label{sec:direct_searches}

Many searches of additional scalars have been developed at the LHC. In order to use those results we have compared the theoretical prediction of the cross-section times branching ratio, $\sigma\cdot \mathcal{B}$, for several processes with the exclusion limits obtained by the CMS and ATLAS experimental collaborations. 

The experimental data are provided in the form of tables, which compile the values of the 95\% upper limits on $\sigma\cdot \mathcal{B}$, as a function of the resonance mass. In these tables, \texttt{HEPfit} performs a linear interpolation if needed.
In order to compare the theoretical results with the experimental data, we define the
ratio of the theoretical prediction over the experimental upper limit. To this ratio \tp{we} %%% will
assign a Gaussian distribution (restricted to positive values) centered at 0 such that the
value 1 is excluded with a 95\% probability.

All the channels included in our fit can be found in Appendix \ref{sec:data_included}. In Tab.~\ref{tab:directsearches4} we show the channels included for the charged scalars, while those of the neutral scalars are shown in Tabs.~\ref{tab:directsearches3}, \ref{tab:directsearches1} and \ref{tab:directsearches2}. Specifically, in Tab.~\ref{tab:directsearches3} are shown the decays into neutral scalars (including the SM Higgs boson); in Tab.~\ref{tab:directsearches1} the decays into fermions, $\gamma\gamma$ and $Z\gamma$; and in Tab.~\ref{tab:directsearches2} the decays into weak gauge bosons. When the limits on $\sigma\cdot \mathcal{B}$ provided in the experimental papers consider also a subsequent decay of the SM particles, we show this final decay with parenthesis. When the limits on $\sigma\cdot \mathcal{B}$ are provided directly on the decay width of the NP particles to SM particles, but using particular decays of the produced SM particles, we show such SM decays with square brackets.%Parenthesis indicate an specific final state and square brackets that limits are quoted on the primary final state, measured through the second final state.


\subsection{Higgs data}

The presence of additional scalars generates relevant effects on the production and decay of the SM Higgs. 
%%%\tp{through} gluon fusion (ggF), 
%as well as \tp{its decay to $2gamma$}, since these processes \tp{occur at the one-loop} level in the SM (and we can have additional particles entering in the loops). 
The one-loop $h\to 2\gamma$ amplitude receives additional contributions from the charged scalar.
Furthermore, due to the mixing among the CP-even scalars, the coupling of the SM Higgs with the weak bosons is modified as shown in Eq.~(\ref{eq:Higgs_weak_boson_couplings}), which %%%produces a modification on 
changes the Higgs production through vector boson fusion (VBF) and the associated production with vector bosons (Vh). % Furthermore, also due to the mixing, all 
The decays of the Higgs boson to fermions are also sensitive to the scalar mixing, as shown in Eqs.~(\ref{eq:Higgs_yuk_up}) and (\ref{eq:Higgs_yuk_down}), which also modifies the Higgs production through gluon fusion (ggF) and its associated production with $t\bar{t}$ pairs (tth). 

The production and the subsequent decay of the SM Higgs boson have been measured (or bounds have been set) at the LHC for the most relevant production modes (ggF, VBF, Vh and tth) and decay channels ($c\bar{c}$, $b\bar{b}$, $\gamma\gamma$, $\mu^+\mu^-$, $\tau^+\tau^-$, $WW$, $Z\gamma$ and $ZZ$).
These data are parametrised in terms of the Higgs signal strengths, which are defined as the measured cross section times branching ratio for a given production and decay Higgs channel, in units of the SM prediction. Table \ref{Tab:HiggsStrengths}, in Appendix \ref{sec:data_included}, compiles the experimental papers from which we have taken the values of the different Higgs signal strengths relevant to our analysis.



\subsection{Electroweak precision observables}

The presence of additional scalars generates contributions to the oblique parameters (also known as Peskin–Takeuchi parameters \cite{Peskin:1990zt, Peskin:1991sw}) $S$, $T$ and $U$ \cite{Haber:2010bw}. The experimental values of these parameters are obtained from global fits of electroweak precision data, using observables directly measured (mainly) at LEP and SLC. Among those observables, 
we highlight the ratio $R_b\equiv \Gamma(Z\rightarrow b\bar{b})/\Gamma(Z\rightarrow \rm{hadrons})$ \cite{Haber:1999zh,Degrassi:2010ne}, which is also affected by the additional scalars. Using the values of the oblique parameters from the PDG would be inconsistent in our study because those values do not take into account the NP contamination on $R_b$. 
Moreover, since we include $R_b$ directly in our global fit of the ATHDM, the information from this observable cannot be employed to determine the oblique parameters.
%Indeed, we will also include $R_b$ in our global fit of the ATHDM, so it is clear that we cannot use it also to predict the oblique parameters.


In order to obtain non-contaminated values for the oblique parameters, we have repeated the electroweak fit removing $R_b$ from the list of fitted measurements, obtaining the values of $S$, $T$ and $U$ that will be included as inputs to our analysis. Those values are summarised in Tab.~\ref{tab:STU} of Appendix \ref{sec:STU_fit}. 

The values of the oblique parameters are also highly dependent on the value of the $W$-boson mass ($M_W$). For our baseline results on these parameters we have used the value of $M_W$ quoted by the PDG 2022 \cite{ParticleDataGroup:2022pth}. However, in April of 2022 the CDF collaboration
announced a controversial new measurement
%%%claimed to measure a controversial new value 
of $M_W$ \cite{CDF:2022hxs}, which is incompatible with the SM prediction by 7$\sigma$. Since there is not yet consensus in the community, we have not included this measurement in our main global fit but, instead, we will show how much the results change with the values for the oblique parameters obtained incorporating this new measurement in a global electroweak fit performed with \texttt{HEPfit} \cite{deBlas:2016ojx,deBlas:2022hdk}.\footnote{Note that in these references $R_b$ is also included as an input but we have checked that repeating the fit of Ref.~\cite{deBlas:2022hdk} removing $R_b$ leads to very similar results. Therefore, we decided to use the values quoted in the original paper.}

Finally, we have also noticed that using the three oblique parameters $S$, $T$ and $U$ or fitting only $S$ and $T$ (assuming $U$ to be negligible) gives very similar results. Therefore, we use only $S$ and $T$, since it is well known that the THDM contributions to $U$ are highly suppressed \cite{Haber:2010bw}.



\subsection{Flavour observables}
\label{sec:flavour}
The parameter space of the ATHDM can also be constrained from flavour observables, since the new scalars generate relevant contributions to many of them.
%%% these processes. 
However, in order to be able to determine these contributions, we first need to know the numerical values of the CKM parameters. For this work we have adopted the Wolfenstein parametrisation \cite{Wolfenstein:1983yz}, so we need to provide the corresponding values as inputs in our analysis.
The world averages quoted in the PDG 2022 \cite{ParticleDataGroup:2022pth} originate in SM fits from the CKMfitter \cite{Charles:2004jd} and UTfit \cite{UTfit:2006vpt,UTfit:2022hsi} collaborations, which make use of several flavour transitions that could be affected by the additional scalars.
The UTfit collaboration also provides the values of the Wolfenstein parameters removing the loop processes from the fit \cite{UTfit:2005lis}, which gives the correct values for models in which the tree-level effects of NP are negligible. Nevertheless, for some (small) regions of the parameter space in the ATHDM there could still be some contamination to some of the tree-level processes used to determine the Wolfenstein parameters. For this reason we have decided to repeat this CKM fit, using only processes that are not contaminated by the additional scalars. Details on this fit, as well as the resulting numerical values for the CKM parameters, can be found in Appendix \ref{sec:CKM_fit}. Note also that we use as observables for our ATHDM global fit some processes that are usually taken into account in the CKM fit, like the pseudoscalar-meson leptonic decays, so it is clear that we should remove these processes from our CKM determination.

We will consider all relevant flavour observables that constrain the CP-conserving ATHDM, including the contributions to loop processes 
like the neutral-meson mixing of $B_s$ ($\Delta M_{B_s}$) \cite{Jung:2010ik,Chang:2015rva}, the weak radiative decay $B\rightarrow X_s \gamma$ \cite{Bobeth:1999ww,Misiak:2006ab,Misiak:2006zs,Jung:2010ik,Jung:2010ab,Jung:2012vu,Hermann:2012fc, Misiak:2015xwa, Misiak:2017woa, Misiak:2020vlo} and the rare weak leptonic decay $B_s\rightarrow \mu^+\mu^-$ \cite{Li:2014fea,Arnan:2017lxi}. Besides these loop processes, we have also included some relevant tree-level transitions like the leptonic decays of heavy pseudoscalar mesons ($B\rightarrow \tau \nu$, $D_{(s)}\rightarrow \mu \nu$ and $D_{(s)}\rightarrow \tau \nu$), as well as the ratios of the leptonic  decays of light pseudoscalar mesons ($\Gamma(K\rightarrow \mu\nu)/\Gamma(\pi\rightarrow \mu\nu)$) and the similar tau decays ($\Gamma(\tau\rightarrow K\nu)/\Gamma(\tau\rightarrow \pi\nu)$) \cite{Jung:2010ik}. 

We have not included in our main fit the anomalous magnetic moment of the muon, $(g-2)_\mu$ \cite{Ilisie:2015tra},
 because there is no full consensus in the community on the SM prediction of this observable. The most recent lattice computations of the hadronic vacuum polarisation \cite{Borsanyi:2020mff,Ce:2022kxy,ExtendedTwistedMass:2022jpw} and the estimates from $\tau$ decay
\cite{Davier:2010fmf,Pich:2013lsa} seem to find a SM result much closer to the experimental value \cite{Muong-2:2021ojo,Muong-2:2023cdq} than the dispersive $e^+e^-$ prediction \cite{Aoyama:2020ynm}, and additional hints of a possible dispersive underestimate are suggested by a recent QCD analysis of the Adler function \cite{Davier:2023hhn} and the CMD3 data \cite{CMD-3:2023alj}.
Nevertheless, this observable is included in our code and we will show the individual constraints obtained with it, comparing those results with the ones obtained using all the other flavour observables that we have mentioned here. 


\section{Fit results}
\label{sec:results}

Here we provide the results for all the fits we have performed in this analysis. We discuss first in several subsections the limits obtained using only some subsets of observables, in order to show the relevance that each observable has in the global fit. 
%%% Later, we will discuss 
Afterwards, we present
the global fit, which provides the main results of this 
%%% analysis. 
work. We
%%% Finally, we decided to 
show also how the results of our global fit would change if
%%% once we include 
the new CDF measurement of $M_W$ \cite{CDF:2022hxs} is included.



\subsection{Theoretical bounds}

The theoretical considerations generate constraints on the parameters of the scalar potential.
As shown in Section \ref{sec:Model}, we have a total of nine degrees of freedom in the CP-conserving potential.
%%%, after EWSB. Two of these parameters are 
The scalar vacuum expectation value and $M_h$
%%% the SM Higgs mass
%%%, whose values 
are fixed by the measurement of the Fermi constant in muon decay \cite{ParticleDataGroup:2022pth} and the Higgs mass measurement at the LHC \cite{ParticleDataGroup:2022pth}. We then have a total of seven degrees of freedom which could be possibly constrained with the theory assumptions. As mentioned before, we 
%%% have decided to 
trade some of these scalar potential parameters 
with more physical inputs. Our chosen parameters are the masses of the new scalars, the mixing angle among the CP-even scalars, and the three quartic couplings $\lambda_2$, $\lambda_3$ and $\lambda_7$. 

The theoretical constraints on the parameters of the potential can be translated into limits on the scalar mass splittings.
Fig.~\ref{fig:theo_masses} displays the resulting  constraints on the correlation among the different masses and the correlation of the charged scalar mass with the mixing angle. Note that in this case we are showing the 100\% probability region, since all the points that satisfy these constraints are equally valid. In general, we observe that for scalar mass values below 700 GeV, the constraints on the mass splittings and
the mixing angle are rather weak. This can be easily understood, since for low masses the mass differences cannot be large enough to reach the theoretical bounds.
%%% relatively lenient. 
However, above this threshold, especially beyond 1 TeV, the
constraints become considerably more stringent. 
Since differences of masses squared are proportional to combinations of quartic couplings  times the vev squared, the naive estimate $\sqrt{4\pi} v\approx 870$ GeV gives indeed a good idea of the scale where these limits become relevant.
%Note that the theory considerations set constraints on the parameters of the potential which can be translated in limits on the mass splitting. When the masses are below 700 GeV their mass difference cannot get high enough to make these limits relevant. 
%Indeed the mass difference on the square masses is proportional to some combination of some quartic couplings ($\lambda$) couplings times vev square, doing an (extremely rough) estimate setting these combinations of couplings to 4$\pi$ we get $4\pi\sqrt{v}\approx870$ GeV, which gives an idea of the order of magnitude the mass difference must be to make these limits relevant. }
%In general we can see how for values of the masses below 700 GeV the constraints on the mass splitting and the mixing angle are quite loose, but above those values, and specially above 1 TeV, the constraints start to become quite relevant.

% Figure environment removed

% Figure environment removed




\FloatBarrier

The constraints on the quartic couplings are shown in Fig.~\ref{fig:theo_lambdas}, where we can see that negative values of $\lambda_2$ are forbidden while slightly negative values of $\lambda_3$ could be allowed, provided they are
% but only for values 
very close to zero. Finally, $\lambda_7$ is clearly constrained to be smaller than 3 in modulus.

%\newpage



\subsection{Experimental bounds}


Fig.~\ref{fig:sig_str} shows that the measured Higgs signal strengths generate tight constraints in the 
$\tilde\alpha - \varsigma_f$ planes.
%%% mixing angle and the Yukawa alignment parameters planes. 
At 99.7\% probability some allowed regions, at the corners of the central and right plots of Fig.~\ref{fig:sig_str}, appear really far away from the SM solution. 
These regions correspond to down-type and/or lepton Yukawas of opposite sign to the SM Higgs couplings.
%%% solutions flipped-sign Yukawas in which the sign of the Yukawa couplings of the SM Higgs is inverted. Indeed, 
Since these observables are sensitive only to the modulus of the Yukawa couplings, this could be a possible solution, although following a bayesian approach the prior probability of this kind of solutions would be quite small. 
No such regions appear for up-type quarks because the relative sign between the top Yukawa and the Higgs coupling to the gauge bosons gets constrained by the $h\to 2\gamma$ decay width and the range of allowed $|\varsigma_u|$ values is much smaller.


% Figure environment removed


% Figure environment removed


The oblique parameters ($S$ and $T$) generate significant constraints on the mass splitting of the additional scalars. These constraints can be found in Fig.~\ref{fig:ST}, where we
compare the results obtained when $S$ and $T$ are fitted with the PDG 2022 value of $M_W$ (PDG-$M_W$) with those
%%%have included the individual fit using the $S$ and $T$ values obtained with the PDG 2022 value of $M_W$ (PDG-$M_W$) as well as the one 
using the modified $S$ and $T$ values emerging from the world average~\cite{deBlas:2022hdk} after including the CDF $M_W$ (CDF-$M_W$) measurement \cite{CDF:2022hxs}. 
The allowed regions using the PDG value of $M_W$ are compatible with small or even zero mass splittings. However, once the CDF value is included in the fit, a non-zero mass difference between the new scalars is definitely required to explain the deviation from the SM prediction of $M_W$. This scalar mass difference could be accommodated within this model, since we are able to generate the needed contribution to $M_W$, even when we impose agreement with all other observables in combination, as we will discuss in Section~\ref{sec:CDF-MW}.


% Figure environment removed

% Figure environment removed




% Figure environment removed


The flavour observables and $R_b$ generate
constraints on the Yukawa alignment parameters ($\varsigma_u$, $\varsigma_d$, $\varsigma_\nl$) and the mass of the charged Higgs $M_{H^\pm}$.
%%%\footnote{The neutral scalars are only relevant for the anomalous magnetic moment, which has not been included in this fit as discussed in Section \ref{sec:flavour}.}
Fig.~\ref{fig:flav_plot_couplings} displays the correlations among $\varsigma_u$, $\varsigma_d$ and $\varsigma_\nl$, while in Fig.~\ref{fig:flav_plot_couplings_vs_mass} we show the correlation of those parameters with $M_{H^\pm}$. 
The dominant NP contribution to the $B\to X_s\gamma$ amplitude is proportional to the product $\varsigma_u\varsigma_d$, which explains the strong correlation between these two alignment parameters observed in the left panel of Fig.~\ref{fig:flav_plot_couplings}.
The first two panels in Fig.~\ref{fig:flav_plot_couplings_vs_mass} exhibit also the presence of two additional bands emerging from the central 
$B\to X_s\gamma$ allowed region; they correspond to solutions with a NP contribution equal to minus two times the SM amplitude.
%
The strongest constraint on the $\varsigma_u-M_{H^\pm}$ plane comes from meson mixing, although $R_b$, $B\rightarrow X_s\gamma$ and $B_s\rightarrow \mu^+\mu^-$ also help in constraining this plane.
%\footnote{ \RF{Note also how the allowed regions for $B\rightarrow X_s\gamma$ presents a particular shape with two additional bands besides the usual central triangular shape. Actually, observables are proportional to the square of the amplitudes as well as the square of effective Wilson coefficient $|C_{eff}|^2$ which can be expressed as $|C_{SM}+C_{NP}|^2$. The additional regions in $B\rightarrow X_s\gamma$ correspond to the solution in which the NP contribution is minus two times the SM, so that we recover the usual SM value. For the actual expressions of Wilson coefficients see Ref. \cite{Degrassi:2010ne}.}}
In the $\varsigma_d-M_{H^\pm}$ plane the tree-level 
leptonic and semileptonic decays of pseudoscalar mesons (and tau decays into pseudoscalars) generate the strongest limits
%processes\footnote{Leptonic and semileptonic decays of pseudoscalar mesons (and tau decays into pseudoscalars), look at Section \ref{sec:flavour} for more details.} are producing the strongest constraints 
for low values of the scalar masses. However, the tree-level NP contributions rapidly drop when increasing the mass of the charged scalar, so for larger mass values the loop processes dominate. $B\rightarrow X_s\gamma$ provides the strongest limits for intermediate values of $M_{H^\pm}$ around 500 GeV,
%%% provide the dominant constraints. 
%%% For intermediate mass values, of around 500 GeV, the constraints from $B\rightarrow X_s\gamma$ become the dominant ones 
while for heavier masses $B_s\rightarrow \mu^+\mu^-$ produces the most relevant constraints because
it also gets contributions from neutral scalar exchanges that are sizeable at large $|\varsigma_d|$.
In the $\varsigma_\nl-M_{H^\pm}$ plane the constraints are very poor, except at very low values of $M_{H^\pm}$ where useful limits are obtained from the tree-level processes.
As shown in the first two panels of Fig.~\ref{fig:flav_plot_couplings_vs_mass}, small values of $M_{H^\pm}$ are allowed when $\varsigma_{u,d}\to 0$, but when marginalising over these two alignment parameters their probability becomes very small, which explains the lower bound on the charged scalar mass observed in the third panel.
%In fact, $B_s\rightarrow \mu^+\mu^-$ obtains some non-negligible contributions (which depend on the alignment parameters) due to the misalignment of SM Higgs and the CP-even scalar, i.e. non-zero $\tilde\alpha$, at higher masses of BSM particles (see Ref. \cite{Li:2014fea}). It should be noted that these mass parameters and $\tilde\alpha$ are independent variables in our parametrization. On the contrary, NP contributions to $B\rightarrow X_s\gamma$ becomes negligible at higher mass of BSM particle obeying the decoupling theorem. Therefore, $B_s\rightarrow \mu^+\mu^-$ provides stronger constraints at higher masses.} 
%In the $\varsigma_\nl-m_{H^\pm}$ plane most of the observables produce very poor constraints, except the tree-level processes for low values of the masses. 
%%%The constraints on the correlations of $\varsigma_u$, $\varsigma_d$ and $\varsigma_\nl$ among themselves are dominated by $B\rightarrow X_s\gamma$, $\Delta M_{B_s}$ and $B_s\rightarrow \mu^+\mu^-$. 
%\RF{It is also worth mentioning that when comparing Figs.~\ref{fig:flav_plot_couplings} and \ref{fig:flav_plot_couplings_vs_mass} the maximum and minimum values allowed for the different parameters are slightly different among the different correlations. }
Notice that the maximum and minimum values allowed for the different parameters are slightly different in Figs.~\ref{fig:flav_plot_couplings} and \ref{fig:flav_plot_couplings_vs_mass}; this is just a consequence of having non-Gaussian distributions.
%%%Note that this is just a consequence of having a non-Gaussian distributions and, in any case, the differences are not substantial. }







Finally, in Fig.~\ref{fig:flav_gm2} we compare the constraints on $\varsigma_\nl$ obtained from the combination of all flavour observables with the results needed to explain the anomalous magnetic moment of the muon. Since we are only considering scalar masses above 125~GeV, quite large values of the leptonic alignment parameter are required, but
there is, indeed, room to satisfy all the flavour constraints in combination \tp{with} $(g-2)_\mu$.
Nevertheless, we decided to exclude this observable from our global fit 
because its SM prediction is currently under debate, as discussed in Section \ref{sec:flavour}.



\FloatBarrier









\subsection{Global Fits}

\label{sec:globa_fits_results}

\begin{table}[h!]
\begin{center}
\begin{tabular}{|P{1.1cm}|P{1.1cm}|P{1.1cm}|P{1.1cm}|P{1.1cm}|P{1.1cm}|P{1.1cm}|P{1.1cm}|P{1.1cm}|P{1.1cm}|P{1.1cm}|P{1.1cm}|}
\hline
\multicolumn{12}{|c|}{\bf Marginalised Individual results} \\
\hline
\hline
\multicolumn{12}{|c|}{ \it Masses up to 1 TeV} \\
\hline
\multicolumn{4}{|P{4.4 cm}}{$M_{H^\pm} \ge$ 390 GeV} & \multicolumn{4}{|P{4.4 cm}|}{$M_{H} \ge$ 410 GeV} & \multicolumn{4}{P{4.4 cm}|}{$M_{A} \ge$ 370 GeV} \\
\hline
\multicolumn{4}{|P{4.4 cm}}{$\lambda_2$: $3.2 \pm 1.9$} & \multicolumn{4}{|P{4.4 cm}|}{$\lambda_3$: $5.9 \pm 3.5$}  & \multicolumn{4}{P{4.4 cm}|}{$\lambda_7:$ $0.0 \pm 1.1$}  \\
\hline
\multicolumn{3}{|P{3.8 cm}}{$\tilde{\alpha}:$ $(0.05 \pm 21.0)\cdot10^{-3}$} & \multicolumn{3}{|P{3.3 cm}|}{$\varsigma_u:$ $0.006 \pm 0.257$} & \multicolumn{3}{P{3.3 cm}|}{$\varsigma_d:$ $0.12 \pm 4.12$} & \multicolumn{3}{P{3.3 cm}|}{$\varsigma_\nl:$ $-0.39 \pm 11.69$} \\  
\hline
\hline
\multicolumn{12}{|c|}{ \it Masses up to 1.5 TeV} \\
\hline
\multicolumn{4}{|P{4.4 cm}}{$M_{H^\pm} \ge$ 480 GeV} & \multicolumn{4}{|P{4.4 cm}|}{$M_{H} \ge$ 490 GeV} & \multicolumn{4}{P{4.4 cm}|}{$M_{A} \ge$ 480 GeV} \\
\hline
\multicolumn{4}{|P{4.4 cm}}{$\lambda_2$: $3.2 \pm 1.9$} & \multicolumn{4}{|P{4.4 cm}|}{$\lambda_3$: $5.9 \pm 3.8$}  & \multicolumn{4}{P{4.4 cm}|}{$\lambda_7:$ $0.0 \pm 1.2$}  \\
\hline
\multicolumn{3}{|P{3.8 cm}}{$\tilde{\alpha}:$ $(0.8 \pm 16.8)\cdot10^{-3}$} & \multicolumn{3}{|P{3.3 cm}|}{$\varsigma_u:$ $-0.011 \pm 0.407$} & \multicolumn{3}{P{3.3 cm}|}{$\varsigma_d:$ $-0.096 \pm 6.22$} & \multicolumn{3}{P{3.3 cm}|}{$\varsigma_\nl:$ $-1.18 \pm 17.54$} \\  
\hline
\end{tabular}
\caption{Marginalised individual results. The mass limits are at 95\% probability while for the others we show the mean value and the square root of the variance. }
\label{tab:marginalised_results}
\end{center}
\end{table}


In this section we provide the results emerging from the global fit, using all the experimental observables and theoretical considerations at the same time. As mentioned in Section~\ref{sec:set_up}, we have considered masses up to 1 and 1.5 TeV, in such a way that we can also explicitly see the dependence on the adopted priors. The marginalised probabilities, given
%%% can be found 
in Tab.~\ref{tab:marginalised_results},
exhibit indeed
%%% . From this table we can see 
some dependence on the priors, specially regarding the mass limits. This behaviour is reasonable since the likelihood is maximised for higher values of the masses; 
%%% Then, the values with 
smaller masses would always be disfavoured, moving the 95\% region further away when we allow for higher mass values. The allowed ranges of the Yukawa alignment parameters are also wider when heavier masses are allowed because the scalar contributions to the flavour observables decrease with increasing mass values and, therefore, higher values of 
$\varsigma_{u,d,\nl}$ become possible,
%%%we allow for higher mass values. This behaviour is also well understood since the higher the masses are, the smaller their contribution to the flavour observables is and, therefore, the higher the $\varsigma_{u,d,\nl}$ can get\RF{, 
as can be observed in Fig.~\ref{fig:flav_plot_couplings_vs_mass}. 
In contrast, the preferred region of the mixing angle shrinks when we scan over heavier masses because the theoretical constraints, shown in Fig.~\ref{fig:theo_masses}, become more severe for higher values of the scalar masses.



%%% However, the mixing angle behaves in the contrary manner, i.e. the preferred region of the mixing angle shrinks when we scan over higher mass values.}
%This behaviour is also seen in Fig.~\ref{fig:flav_plot_couplings_vs_mass}. The mixing angle experiences the opposite behaviour, though.
%\RF{This behaviour can be well understood looking at Fig.~\ref{fig:theo_masses}. Here we can see how the theoretical assumptions impose tight constraint on the mixing angle for high values of the scalar masses. Indeed, the higher the mass, the more stringent the limits are. Since the shown marginalised distribution is obtained from integrating over the allowed ranges of all the other variables, when we scan over a range that includes higher mass values (which highly restrict the mixing angle) the distribution will prefer smaller values of the mixing angle. }
%In this case, as can be seen in Fig.~\ref{fig:theo_masses}, the theoretical considerations impose tight constraint on the mixing angle for high values of the scalar masses. 
%\RF{Since the shown marginalised distribution is obtained from integrating over all the other variables, when we scan up to higher mass ranges the distribution would prefer much more smaller values of the mixing angle}, in this case over a wider mass range, we obtain a smaller allowed range.
%when we marginalise over a wider mass range we obtain a smaller allowed range. 


% Figure environment removed

% Figure environment removed

% Figure environment removed

%Another interesting region to look at is 
The correlations among the scalar masses are
%This correlation is 
shown in Fig.~\ref{fig:All_Masses}, 
which exhibits diagonal bands enforced by the theoretical constraints and the oblique parameters. The bounds on the masses get significantly relaxed, getting close to disappear, 
%%% where we can see how 
in the limit of zero mass splittings between the charged and the neutral scalars.
%%%the constraints on the masses get significantly relaxed, getting close to disappear. 
This is due to the reduction on the constraints from the oblique parameters, which makes these points much more favoured. 
%%% We can also look at 
The correlations among the mass splittings are shown in Fig.~\ref{fig:Delta_mass_vs_delta_mass}. 
In the middle plot one observes that the two mass differences between the charged scalar and the neutral CP-even and CP-odd bosons cannot be simultaneously large; when one increases the other must decrease.
%%% Here in the middle plot we can see how when the mass difference among the CP-odd and the charged scalar \RF{increases} the mass splitting between the CP-even scalar and the charged scalar \RF{needs to decrease in order to satisfy the constraints. Complementarily, when the mass splitting between the CP-even scalar and the charged scalar increases, the mass splitting between the CP-odd and the charged scalar must decrease}. 
From the left and right plots we can also see that when the mass splitting among the neutral scalars gets large the mass splitting among the charged scalar and one of the neutral scalars must go to zero.  Here again, 
this behaviour is generated by
%%% the behaviour is explained from 
the oblique parameters, whose NP contribution tends to zero when the mass splitting between any neutral scalar and the charged scalar vanishes. The last interesting plots regarding the masses are their correlations with
% among them and both 
the Yukawa alignment parameters and the mixing angle, shown in Fig.~\ref{fig:Masses_vs_alpha_and_couplings}. Here we can see how the higher the masses get, the higher the Yukawa alignment parameters can be. Nevertheless, for masses above 700 GeV, smaller values of the mixing angle are preferred due to the theoretical constraints, as explained before. Finally, for very small values of the Yukawa alignment parameters or the mixing angle the limits on the masses also start to disappear, as expected.



The correlations of the Yukawa alignment parameters 
among themselves and with the mixing angle are shown in Figs.~\ref{fig:Couplings} and \ref{fig:Couplings_vs_alpha}, respectively. In the first one we observe that the regions get wider when higher masses are allowed, since all experimental constraints become softer,
as also shown in Fig.~\ref{fig:Masses_vs_alpha_and_couplings}. 
%We can also see here how 
Moreover, once one of the Yukawa alignment parameters tends to high values, the others get highly suppressed. Indeed, for masses up to 1 TeV, in order to obtain values of $\varsigma_d$ or $\varsigma_\nl$ of order 10, $\varsigma_u$ must be of order 0.1, or smaller.  We observe a similar behaviour looking at the correlations of the Yukawa alignment parameters with the mixing angle. The higher the mixing angle, the smaller the Yukawa alignment parameters must be; and the higher the allowed range of masses, the higher the Yukawa alignment parameters can be. However, as can also be seen in Fig.~\ref{fig:Masses_vs_alpha_and_couplings}, the mixing angle does not grow with the masses, 
and slightly smaller values of $\tilde\alpha$ are preferred when heavier masses are allowed in the fit.
%%% but tends also to smaller values when the masses take very high values. This is why in the fit in which we allow higher mass values smaller regions of the mixing angle are preferred. 
Nevertheless, 
%%%in this same Fig.~\ref{fig:Masses_vs_alpha_and_couplings} we can see how 
performing a fit with a wider mass region does not significantly change the shape of the correlations between the Yukawa alignment parameters and the mixing angle with respect to the masses of the scalars. 
%%%In the plots in which we integrate 
Integrating
over the masses, like in Figs.~\ref{fig:Couplings} and \ref{fig:Couplings_vs_alpha}, 
%%% we again see how 
the shape stays basically unchanged and just rescales when a different mass range is covered. Indeed, 
%%%the fact of 
performing the fits for these two different mass ranges allows us to get an idea of the shape of the multidimensional correlation of the masses against the other parameters which, unfortunately, cannot be printed in a two dimensional plot. 

In Fig.~\ref{fig:Couplings} we also show the restrictions (Eq. \eqref{eq:THDM_types}) that each $\mathcal{Z}_2$ THDM would impose on the Yukawa alignment couplings. 
In these models all alignment parameters are related
%%% As can be noticed from Eq. \eqref{eq:THDM_types}, the alignment parameters ($\varsigma_f^{}$) in these models are not independent to each other; rather they are related 
through an angle $\beta$,  whose tangent measures the ratio of the vevs  of the  two scalar fields in the basis where the discrete $\mathcal{Z}_2$ symmetry is imposed. 
Therefore, each $\mathcal{Z}_2$-symmetric model corresponds to
specific curves (either straight lines or hyperbolas)
in the $\varsigma_{f_1}^{}-\varsigma_{f_2}^{}$ planes, %%%these $\mathcal{Z}_2$-symmetric models trace different curves, 
each point on which represents different values of $\tan \beta$. 
%%%It is evident from Eq. \eqref{eq:THDM_types} that these curves generally take the shape of a straight line or a rectangular hyperbola, whereas for inert THDM they become isolated point at the origin. 
The curves in green, cyan, orange and black depict the Type-I, Type-II, Type-X and Type-Y THDMs, respectively,
whereas the inert THDM corresponds to an isolated point at the origin.
The first panel in Fig.~\ref{fig:Couplings} indicates that the type-II and type-Y models have some tension with the (mainly flavour) data. 
%%% From this comparison we can clearly see how, if we take an agnostic assumption about these models, the ones that are more compatible with data will be the Type-I and Type-X, since the the Type-II and Type-Y present some tension (mainly with flavour data). 
Note, however, that this is just a qualitative comparison. In order to analyse the parameter space of any of these particular models, we would need to change the priors of our analysis by considering $\tan \beta$ as the free parameter instead of the alignment parameters along with the constraint $\mu_3=\lambda_6=\lambda_7=0$.
% since the Yukawa alignment parameters would not be free anymore (as we have assumed for our analysis). 
A comprehensive investigation of the parameter space for each of these models, though valuable, falls outside the scope of this study. 
%%%We refer the interested reader to other global fit analysis studying these models \cite{Chowdhury:2017aav, Haller:2018nnx}. }
We refer the interested reader to the more recent specific analyses in Refs. \cite{Chowdhury:2017aav, Haller:2018nnx}.

% Figure environment removed
%
% Figure environment removed
%
% Figure environment removed

\FloatBarrier

Finally, in Fig.~\ref{fig:Lambdas} we plot the correlations among the quartic couplings of the scalar potential. These couplings are mainly constrained by the theoretical bounds (perturbative unitarity, boundness from below and vacuum stability) and therefore their constraints do not depend on the mass range studied. Indeed, these limits are approximately the same as those of Fig.~\ref{fig:theo_lambdas},
in which only the theoretical considerations have been taken into account.



\subsection{CDF $W$ mass measurement}
\label{sec:CDF-MW}

As mentioned before, last year the CDF collaboration released a new measurement of the mass of the $W$ boson \cite{CDF:2022hxs} with a 7$\sigma$ tension with the SM prediction. Since there is still a 
controversy in the community, we decided not to include this last measurement in our main analysis but we provide here the results of a global fit including that information. In particular, we take the values of the oblique parameters ($S$ and $T$) from Ref.~\cite{deBlas:2022hdk}. Using that data as an input, we obtain a posterior value for $M_W=80.4178 \pm 0.0091$ GeV which is compatible with the CDF measurement ($M_W^{\mathrm{CDF}}=80.433 \pm 0.009$) within a 95\% probability. The additional scalars of the ATHDM would then be able to explain this result, if confirmed.\footnote{Of course we are not able to explain the 4$\sigma$ incompatibility between the ATLAS \cite{ATLAS:2017rzl,ATLAS:2023fsi} and CDF  \cite{CDF:2022hxs} measurements.} As can be seen in Figs.~\ref{fig:All_Masses_CDF} and \ref{fig:Delta_mass_vs_delta_mass_CDF}, in order to generate this additional contribution to $M_W$ we need a mass difference among the charged scalar and the neutral scalars of, at least, few tens of GeVs. Besides this mass difference, the results for the other parameters are very similar to the ones obtained in the baseline fit, as expected. 

% Figure environment removed
%
% Figure environment removed

\FloatBarrier




\section{Conclusion}
\label{sec:conclusion}

In this work we have performed an extensive phenomenological analysis of the CP-conserving Aligned Two-Higgs-Doublet model, using the \texttt{HEPfit} package. These results update the previous work of Ref.~\cite{Eberhardt:2020dat},\footnote{In this work we have adopted as baseline the same code used in Ref.~\cite{Eberhardt:2020dat}, but we have also included some relevant modifications. We thank here the developers of the previous version who are not signing this paper, O.~Eberhardt and A.~Pe\~nuelas.} although we use here a slightly different set of parameters. The observed SM Higgs boson at a mass of 125 GeV has been assumed to be the lightest scalar particle 
and we have, therefore, focused our study in heavy new-physics scalars.
More details on the region of the parameter space that we have scanned
%%% decided to scan 
can be found in Section~\ref{sec:set_up}.


We have incorporated the theoretical requirements of perturbative unitarity, boundedness of the scalar potential from below and vacuum stability,
%%% included all the constraints coming from theoretical considerations, perturbative unitarity and vacuum stability,
and the most relevant experimental constraints, including direct searches at the LHC, Higgs signal strengths, electroweak precision observables and flavour data.
The main results of the global fit, combining all theoretical and experimental constraints at the same time,
are shown in Section \ref{sec:globa_fits_results}.
%%% where we show the results of the global fit, combining all the theoretical and experimental constraints at the same time. 

We have a total of ten new-physics parameters to fit. Three of them are the Yukawa alignment parameters for up-type quarks ($\varsigma_u$), down-type quarks ($\varsigma_d$) and leptons ($\varsigma_\nl$). The other seven come from the scalar potential, where we have chosen the three heavy scalar masses ($M_H$, $M_A$ and $M_{H^\pm}$), the mixing angle among the neutral scalars ($\tilde{\alpha}$), and three quartic couplings ($\lambda_2$, $\lambda_3$ and $\lambda_7$). The only relevant constraints on these quartic couplings come from theoretical considerations. The mixing angle is mainly constrained by the Higgs signal strengths and lower limits on the masses of the new scalars are obtained from direct searches and flavour observables (which constrain the mass of the charged scalar). The mass difference among the charged scalar and (at least one of) the neutral scalars is highly constrained by the oblique parameters. The theoretical requirements also provide relevant constraints on the mass splittings and on the mixing angle for high values of the charged scalar mass. Finally, the Yukawa alignment parameters are mainly constrained by the direct searches and flavour observables.

The marginalised probabilities for all the parameters can be found in Tab.~\ref{tab:marginalised_results}. In this table we show results for a fit in which the scan on the masses is done up to 1 TeV and one in which the scan is done up to 1.5 TeV. 
These ranges of masses have been chosen according to the ranges testable at the LHC.
Obviously, the results differ according to the range considered, since higher masses are always favoured. 
%The range of masses used has been chosen according to the ranges testable at the LHC. 
Note also that the lower bounds on the masses are highly correlated with the values of the Yukawa alignment parameters. If we set these parameters to sufficiently low values, the constraints on the masses vanish. For masses up to 1 TeV, values of $|\varsigma_u|>0.7$ , $|\varsigma_d|>12$,  $|\varsigma_\nl|>30$ and  $|\tilde{\alpha}|>0.06$ lay outside the 95\% probability region. 



Besides performing this global fit with the state of the art data, we have also studied 
%%% some other observables and 
two additional measurements that, at the moment, deviate from the SM prediction: the muon $(g-2)$ \cite{Muong-2:2021ojo,Muong-2:2023cdq} and the new measurement of $M_W$ from the CDF collaboration \cite{CDF:2022hxs}. In both cases we have not included them in the global fit because they are still controversial, although for different reasons.
%%% decided not to include them in the global fit since, although for different reasons, they are still controversial. 
Nevertheless, we have shown that the ATHDM has enough flexibility to accommodate both measurements within its parameter space, while being still compatible with all other constraints.

\section*{Acknowledgement}

We thank the developers of the previous global fit to this model, Otto Eberhardt and Ana Pe\~nuelas.
We also thank Luca Silvestrini, Ayan Paul and Antonio Coutinho for some useful discussions and technical support regarding {\tt HEPfit}. 
This work has been supported in part by Generalitat Valenciana (Grant PROMETEO/2021/071) and by
MCIN/AEI/10.13039/501100011033 (Grant No. PID2020-114473GB-I00). The work of VM has been supported by the Italian Ministry of Research (MUR) under
the grant PRIN20172LNEEZ.

 
\appendix

\section{Data Included}

\label{sec:data_included}

The following tables compile all experimental references included in the fit for the direct searches and the Higgs signal strengths.

%%%%%%%%%%%%%%%%%%%%%%%%%%%%%%%%%%%%%%%%%%%%%%%%%%%%%
%%%%%%%%%%%%%%%%%%%%%%%%%%%%%%%%%%%%%%%%%%%%%%%%%%%%%
%%%%%%%%%%%%%%%%%%%%%%%%%%%%%%%%%%%%%%%%%%%%%%%%%%%%%


\vspace*{8mm}
\begin{table}[h!]
	{\small
		\begin{center}
			\begin{tabular}{|c |c| c c c| c c c|}
				\hline
				\multicolumn{1}{|c}{\textbf{Production}} & \textbf{Decay} & \textbf{Reference} & \textbf{$L$} & \textbf{$\sqrt{s}$} & \textbf{Reference} & \textbf{$L$} & \textbf{$\sqrt{s}$}\\[3pt]
				\multicolumn{1}{|c}{}& & & \textbf{[fb$^{-1}$]} & \textbf{[TeV]} & & \textbf{[fb$^{-1}$]} & \textbf{[TeV]} \\[3pt]
				\cline{3-8}
				
				\multicolumn{1}{|c}{}& &\multicolumn{3}{c|}{\textbf{ATLAS}}&\multicolumn{3}{c|}{\textbf{CMS}}\\
				\hline
				\multirow{ 15}{*}{\makecell{ ggF \\ \\ VBF \\ \\ Vh \\ \\ tth } }&  \multirow{ 1}{*}{$c \bar{c}$}  & \cite{ATLAS:2021zwx} & 139 & 13 & \cite{CMS:2019hve} & 35.9 & 13 \\
				%\hline
				\cline{2-8}
				
				
				& \multirow{ 2}{*}{$b \bar{b}$}  & \cite{ATLAS:2014vuz,ATLAS:2015utn} & 4.7/20.3 & 7/8 & \cite{CMS:2013poe,CMS:2014tll} & 5.1/18.9 & 7/8 \\
				&  & \cite{ATLAS:2020bhl,ATLAS:2020fcp,ParticleDataGroup:2022pth}  &  126/139  & 13 & \cite{CMS:2016mmc,CMS:2018nsn,ParticleDataGroup:2022pth}   & 2.3/ 41.3 & 13 \\\cline{2-8}
				
				
				
				& \multirow{ 2}{*}{$ \gamma\gamma$} & \multirow{ 1}{*}{\cite{ ATLAS:2014cnc}}   & \multirow{ 1}{*}{4.5/20.3}  & \multirow{ 1}{*}{7/8} & \multirow{ 1}{*}{\cite{CMS:2014afl}} & \multirow{ 1}{*}{5.1/19.7} & \multirow{ 1}{*}{7/8}\\
				&  & \multirow{ 1}{*}{\cite{ATLAS:2020pvn}}& \multirow{ 1}{*}{139} & \multirow{ 1}{*}{13}  & \multirow{ 1}{*}{ \cite{CMS:2021kom}} & \multirow{ 1}{*}{137} & \multirow{ 1}{*}{13} \\\cline{2-8}
				
				
				& \multirow{ 2}{*}{$\mu^+ \mu^-$}  & \cite{ ATLAS:2016neq} & 5/20& 7/8 & \cite{ ATLAS:2016neq} & 5/20& 7/8 \\
				&  & \cite{ATLAS:2020fzp}  &  139  & 13 & \cite{CMS:2020xwi}   & 137 & 13 \\\cline{2-8}
				&  \multirow{ 2}{*}{$\tau^+ \tau^-$ } & \cite{ATLAS:2015xst}& 4.5/20.3 & 7/8 & \cite{CMS:2014wdm}  & 4.9/19.7 & 7/8 \\
				&   &\cite{ATLAS:2020qdt} & 139 & 13&  \cite{CMS:2020gsy}  & 137 &13 \\\cline{2-8}
				&   \multirow{ 2}{*}{$WW$} & \cite{ ATLAS:2014aga, ATLAS:2015muc}  & 25, 4.5/20.3 & 7/8 & \cite{CMS:2013zmy} & 4.9/19.4& 7/8 \\
				&  & \cite{ATLAS:2020qdt} &  139  & 13 &  \cite{CMS:2020gsy} &  137 & 13 \\\cline{2-8}
				&  \multirow{ 2}{*}{$Z\gamma$ } & \cite{ATLAS:2015egz} & 4.7/20.3 & 7/8 &\cite{CMS:2013rmy}  & 5/19.6 & 7/8 \\
				&  & \cite{ATLAS:2020qcv} & 139 & 13 & \cite{CMS:2018myz}  & 35.9 & 13 \\\cline{2-8}
				&   \multirow{ 2}{*}{$ZZ$} & \cite{ATLAS:2014kct}&  4.5/20.3& 7/8 & \cite{CMS:2014fzn} & 5.1/19.7 & 7/8 \\
				&  & \cite{ATLAS:2020rej} & 139 & 13 &\cite{CMS:2020gsy} & 137 & 13 \\\hline
				
			\end{tabular}
			\caption{Higgs signal strengths measured by ATLAS and CMS. Note that not all the decays have been measured for all the production channels. }
			\label{Tab:HiggsStrengths}
		\end{center}
	}
\end{table}



%%%%%%%%%%%%%%%%%%%%% 
%%%%%   Charged Higgs searches   %%
%%%%%%%%%%%%%%%%%%%%% 


\begin{table}[htb]
\begin{center}
\begin{tabular}{| l | l | l c | c | c |}
\hline
\textbf{Label} &\textbf{Channel} & \multicolumn{2}{| l |}{\textbf{Experiment}} & \textbf{Mass range} & ${\cal L}$ \\
&&&& \textbf{[TeV]} & \textbf{[fb$^{-1}$]} \\[1pt]
\hline
\cellcolor{color8TeV} $A_{8}^{\tau\nu}$ & $pp\to H^\pm \to \tau^\pm \nu $ & ATLAS &\cite{ATLAS:2014otc} & [0.18;1] & 19.5\\
\hline
\cellcolor{color8TeV} $C_{8}^{\tau\nu}$  & $pp\to H^{\RF \pm} \to \tau^{\RF \pm} \nu $ & CMS &\cite{CMS:2015lsf}& [0.18;0.6] & 19.7\\
\hline
  \cellcolor{color13TeV}  $A_{13}^{\tau\nu}$  & \multirow{2}{*}{$pp\to H^{\pm} \to \tau^\pm \nu $ }& \multirow{1}{*}{ATLAS} &  \cite{ATLAS:2018gfm} & [0.09;2] & 36.1\\

\cellcolor{color13TeV} $C_{13}^{\tau\nu}$ & & \multirow{1}{*}{CMS} &    \cite{CMS:2019bfg} & [0.08;3]  & 35.9 \\
\hline
\hline
\cellcolor{color8TeV} $A_{8}^{tb}$ & $pp\to H^\pm \to t b $ & ATLAS & \cite{ATLAS:2015nkq} & [0.2;0.6] & 20.3\\
\hline
\cellcolor{color8TeV} $C_{8}^{tb}$ & $pp\to H^{\RF \pm} \to t b $ & CMS & \cite{CMS:2015lsf} & [0.18;0.6] & 19.7\\
\hline
\cellcolor{color13TeV} $A_{13}^{tb}$ & \multirow{2}{*}{$pp\to H^\pm \to tb $} & \multirow{1}{*}{ATLAS} &   \cite{ATLAS:2021upq} & [0.2;2] & 139 \\\cline{3-6}
\cellcolor{color13TeV} $C_{13}^{tb}$ &  & CMS & \cite{CMS:2020imj} & [0.2;3] & 35.9\\
\hline
\end{tabular}
\caption{Direct searches for charged  scalars.}
\label{tab:directsearches4}
\end{center}
\end{table}

















\begin{table}[htb]
\centering
{\renewcommand{\arraystretch}{1.0}
\resizebox{\textwidth}{!}{%
\begin{tabular}{| l | l | l c | c | c |}
\hline
\textbf{Label} &\textbf{Channel} & \multicolumn{2}{| l |}{\textbf{Experiment}} & \textbf{Mass range} & ${\cal L}$ \\
&&&& \textbf{[TeV]} & \textbf{[fb$^{-1}$]} \\[1pt]
\hline
\hline
% hh
\cellcolor{color8TeV} $A_{8}^{hh}$ &$gg\to \varphi_i^0 \to hh$ & ATLAS &\cite{ATLAS:2015sxd} & [0.26;1] & 20.3\\
\hline
\cellcolor{color8TeV} $C_{8}^{4b}$  &$pp\to  \varphi_i^0 \to hh \to (bb) (bb)$ & CMS &\cite{CMS:2015jal} & [0.27;1.1] & 17.9\\
\hline
\cellcolor{color8TeV} $C_{8}^{2\gamma2b}$  &$pp\to  \varphi_i^0 \to hh \to (bb) (\gamma \gamma)$ & CMS & \cite{CMS:2016cma} & [0.260;1.1] & 19.7\\
\hline
\cellcolor{color8TeV} $C_{8g}^{2b2\tau}$ &$gg\to  \varphi_i^0 \to hh \to (bb) (\tau\tau)$ & CMS & \cite{CMS:2015uzk} & [0.26;0.35] & 19.7\\
\hline
\cellcolor{color8TeV} $C_{8}^{2b2\tau}$  &$pp\to  \varphi_i^0 \to hh [\to (bb) (\tau\tau)]$ & CMS & \cite{CMS:2017yfv} & [0.35;1] & 18.3\\
\hline
\cellcolor{color13TeV} $A_{13}^{4b}$& \multirow{3}{*}{$pp \to  \varphi_i^0 \to hh \to (bb) (bb)$} & ATLAS & \cite{ATLAS:2022hwc} & [0.251;5] & 139\\
\cellcolor{color13TeV} $C_{13,1}^{4b}$  & &  CMS & \cite{CMS:2018qmt} & [0.26;1.2] & 35.9\\
\cellcolor{color13TeV} $C_{13,2}^{4b}$  & &  CMS & \cite{CMS:2021qvd} & [1;3] & 138\\
\hline
\cellcolor{color13TeV} $C_{13}^{4W}$& \multirow{1}{*}{$pp \to  \varphi_i^0 \to hh \to (WW)(WW)/(WW)(\tau\tau)/(\tau\tau)(\tau\tau)$} & CMS & \cite{CMS:2022kdx} & [0.25;1]  & 138\\
\hline
\cellcolor{color13TeV} $A_{13}^{2\gamma2b}$  & $pp \to  \varphi_i^0 \to hh [\to (bb) (\gamma \gamma)]$ & ATLAS &  \cite{ATLAS:2021ifb} & [0.251;1] & 139 \\
\cellcolor{color13TeV} $C_{13}^{2\gamma2b}$  & $pp \to  \varphi_i^0 \to hh \to (bb) (\gamma \gamma)$ & CMS & \cite{CMS:2018tla} & [0.25;0.9] & 35.9\\
\hline
\cellcolor{color13TeV} $A_{13,1}^{2b2\tau}$ & \multirow{2}{*}{$pp \to \varphi_i^0 \to hh \to (bb) (\tau \tau)$} & ATLAS &   \cite{ATLAS:2021fet} & [0.251;1.6] & 139\\
\cellcolor{color13TeV} $A_{13,1}^{2b2\tau}$ &  & ATLAS &  \cite{ATLAS:2020azv} & [1;3] & 139\\
\cellcolor{color13TeV} $C_{13,1}^{2b2\tau}$ & \multirow{2}{*}{$pp \to  \varphi_i^0 \to hh [\to (bb) (\tau \tau)]$} & CMS & \cite{CMS:2017hea} & [0.25;0.9] & 35.9\\
\cellcolor{color13TeV} $C_{13,2}^{2b2\tau}$ &  & CMS & \cite{CMS:2018kaz} & [0.9;4] & 35.9\\

\hline
\cellcolor{color13TeV} $C_{13}^{2b2V}$ & $pp \to  \varphi_i^0 \to hh \to (bb) (VV\to \ell \nu \ell \nu)$ & CMS & \cite{CMS:2017rpp} & [0.26;0.9] & 35.9\\
\hline
\cellcolor{color13TeV} $C_{13}^{2b2W}$ & $pp \to  \varphi_i^0 \to hh \to (bb) (WW\to q \bar{q} \ell \nu)$ & CMS & \cite{CMS:2019noi} & [0.8;3.5] & 35.9\\
\hline
\cellcolor{color13TeV} $C_{13}^{2b2Z}$ & $pp \to  \varphi_i^0 \to hh \to (bb) (ZZ \to \ell \ell j j)$ & CMS & \cite{CMS:2020jeo} & [0.26;1] & 35.9\\
\hline
\cellcolor{color13TeV} $C_{13}^{2b2Z}$ & $pp \to  \varphi_i^0 \to hh \to (bb) (ZZ \to \ell \ell \nu \nu)$ & CMS & \cite{CMS:2020jeo} & [0.26;1] & 35.9\\
\hline
\cellcolor{color13TeV} $A_{13}^{2b2W}$ & $pp \to  \varphi_i^0 \to hh [\to (bb) (WW)]$ & ATLAS &  \cite{ATLAS:2018fpd} & [0.5;3] & 36.1\\

\hline
\cellcolor{color13TeV} $C_{13}^{2b2W}$ & $pp \to  \varphi_i^0 \to hh [\to (bb) (WW/\tau\tau\to (q\bar{q}/\ell\nu) \ell\nu)]$ & CMS & \cite{CMS:2021roc} & [0.8;4.5] & 138\\
\hline
\cellcolor{color13TeV} $A_{13}^{2\gamma2W}$  & $gg \to  \varphi_i^0 \to hh \to (\gamma \gamma) (WW)$ & ATLAS & \cite{ATLAS:2018hqk} & [0.26;0.5] & 36.1\\ 
\hline
\hline
% hZ
\cellcolor{color8TeV} $A_{8}^{bbZ}$ &$gg\to  \varphi_i^0 \to hZ \to (bb) Z$ & ATLAS & \cite{ATLAS:2015kpj} & [0.22;1] & 20.3\\
\hline
\cellcolor{color8TeV} $C_{8}^{2b2\ell}$ &$gg\to  \varphi_i^0 \to hZ \to (bb) (\ell \ell)$ & CMS &\cite{CMS:2015flt} & [0.225;0.6] &19.7\\
\hline
\cellcolor{color8TeV} $A_{8}^{\tau\tau Z}$ &$gg\to  \varphi_i^0 \to hZ \to (\tau\tau) Z$ & ATLAS & \cite{ATLAS:2015kpj} & [0.22;1] & 20.3\\
\hline
\cellcolor{color8TeV} $C_{8}^{2\tau2\ell}$  &$gg\to  \varphi_i^0 \to hZ \to (\tau\tau) (\ell \ell)$ & CMS & \cite{CMS:2015uzk} & [0.22;0.35] & 19.7\\
\hline
\cellcolor{color13TeV} $A_{13}^{bbZ}$  & \multirow{5}{*}{$gg\to  \varphi_i^0 \to hZ \to (bb) Z$} & ATLAS & \cite{ATLAS:2017xel} & [0.2;2] & 36.1\\
\cellcolor{color13TeV} $C_{13,1}^{bbZ}$  & & CMS & \cite{CMS:2019qcx} & [0.22;0.8] & 35.9\\
\cellcolor{color13TeV} $C_{13,2}^{bbZ}$  & & CMS &  \cite{CMS:2018ljc} & [0.8;2] & 35.9\\
\hline
\cellcolor{color13TeV} $C_{13,3}^{bbZ}$  & $gg\to  \varphi_i^0 \to (h \to \tau\tau) (Z \to \ell \ell)$ & CMS & \cite{CMS:2019kca} & [0.22;0.4] & 35.9\\
\hline
\cellcolor{color13TeV} $A_{13b}^{bbZ}$  & \multirow{3}{*}{$bb\to  \varphi_i^0 \to hZ \to (bb) Z$} & ATLAS & \cite{ATLAS:2017xel} & [0.2;2] & 36.1\\
\cellcolor{color13TeV} $C_{13b,1}^{bbZ}$  & & CMS & \cite{CMS:2019qcx} & [0.22;0.8] & 35.9\\
\cellcolor{color13TeV} $C_{13b,2}^{bbZ}$  & & CMS & \cite{CMS:2018ljc} & [0.8;2] & 35.9\\
\hline
\hline
%
\cellcolor{color8TeV} $C_{8, 1}^{\varphi_2^0 Z}$& $pp\to \varphi_3^0  \to \varphi_2^0 Z \to (bb) (\ell\ell)$ & CMS & \cite{CMS:2016xnc} & [0.04;1] & 19.8\\
\hline
\cellcolor{color8TeV} $C_{8, 2}^{\varphi_2^0 Z}$& $pp\to \varphi_3^0 \to \varphi_2^0 Z \to (\tau \tau) (\ell\ell)$ & CMS & \cite{CMS:2016xnc} & [0.05;1] & 19.8\\
\hline
\cellcolor{color13TeV} $A_{13}^{\varphi^0 Z}$& $gg\to \varphi_{3}^0 \to \varphi_2^0 Z \to (bb) Z$ & ATLAS &  \cite{ATLAS:2020gxx} & [0.13/0.23;0.7/0.8]  & 139\\
\hline
\cellcolor{color13TeV} $A_{13b}^{\varphi^0 Z}$& $bb\to \varphi_{3}^0 \to \varphi_2^0 Z \to (bb) Z$ & ATLAS &  \cite{ATLAS:2020gxx} & [0.13/0.23;0.7/0.8]  & 139\\
\hline
\end{tabular}
}
}
\caption{Direct searches for neutral heavy scalars, $\varphi_i^0 = H, A$, with final states including the SM Higgs boson or other neutral scalars. $\varphi_3$ denotes the heaviest scalar,  $V = W,Z$, $\ell = e, \mu$. The parenthesis show the final decay of the SM particles produced from the NP particles. The square brackets are used when the values of $\sigma\cdot \mathcal{B}$ are shown in terms of the primary decay (i.e. the NP particle decay) but a particular decay channel of the SM particles is used to obtain those values. See Section~\ref{sec:direct_searches} for more details.
}
\label{tab:directsearches3}
\end{table}

























\begin{table}[htb]
	\centering
	{\renewcommand{\arraystretch}{0.8}
		\resizebox{0.88\textwidth}{!}{%
			\begin{tabular}{|l|l|lc|c|c|}
				\hline
				\textbf{Label} &\textbf{Channel} & \multicolumn{2}{| l |}{\textbf{Experiment}} & \textbf{Mass range} & ${\cal L}$ \\
				&&&& \textbf{[TeV]} & \textbf{[fb$^{-1}$]} \\[1pt]
				\hline
				\hline
				% tt

				\cellcolor{color13TeV} $A_{13b}^{tt}$  & $bb \to \varphi_i^0 \to tt$ & ATLAS & \cite{ATLAS:2016btu} & [0.4;1] & 13.2\\
				\hline
				\hline
				
				\cellcolor{color13TeV}$C_{13t}^{tt}$  & $tt/tW/tq \to \varphi_i^0 \to tt$ & CMS & \cite{CMS:2019rvj} & [0.35;0.65] & 137\\
				\hline
				\cellcolor{color13TeV}$A_{13t}^{tt}$  & $tt \to \varphi_i^0 \to tt$ & ATLAS & \cite{ATLAS:2022ohr} & [0.4;1] & 137\\
				\hline
				\hline
				
				
				% bb
				\cellcolor{color8TeV} $C_{8b}^{bb}$ &$bb \to \varphi_i^0 \to bb$ & CMS & \cite{CMS:2015grx} & [0.1;0.9] & 19.7\\
				\hline
				\cellcolor{color8TeV} $C_{8}^{bb}$  &$gg \to \varphi_i^0\to bb$ & CMS & \cite{CMS:2018kcg} & [0.33;1.2] & 19.7\\
				\hline
				\cellcolor{color13TeV} $C_{13}^{bb}$ & $pp \to \varphi_i^0\to bb$ & CMS & \cite{CMS:2016ncz} & [0.55;1.2] & 2.69\\
				\hline
				\cellcolor{color13TeV} $C_{13b}^{bb}$ & $bb \to \varphi_i^0\to bb$ & CMS & \cite{CMS:2018hir} & [0.3;1.3] & 35.7\\
				\hline\hline

				
				\cellcolor{color13TeV}$A_{13}^{bb}$ & $pp \to \varphi_i^0\to bb$  ($\geq 1$ b-jet)& ATLAS & \cite{ATLAS:2018tfk} & [1.4;6.6] & 36.1\\
				\hline
				
				\cellcolor{color13TeV}$A_{13}^{bb}$ & $pp \to \varphi_i^0\to bb$ & ATLAS & \cite{ATLAS:2018tfk} & [0.6;1.25] & 24.3\\
				\hline
				
				\cellcolor{color13TeV}$A_{13}^{bb}$ & $pp \to \varphi_i^0\to bb$ & ATLAS & \cite{ATLAS:2018tfk} & [1.25;6.2] & 36.1\\
				\hline

				\cellcolor{color13TeV}$A_{13b}^{bb}$ & $bb \to \varphi_i^0\to bb$ & ATLAS & \cite{ATLAS:2019tpq} & [0.45;1.4] & 27.8\\
				\hline
				\hline
				
				\cellcolor{color13TeV}$C_{13}^{bb}$ & $pp \to \varphi_i^0\to bb$ & CMS & \cite{CMS:2018pwl} & [0.05;0.35] & 35.9\\
				\hline
				\hline

				
				%mumu
				\cellcolor{color8TeV} $C_{8b}^{\mu\mu}$ &$bb\to \varphi_i^0 \to \mu\mu$ & CMS &\cite{CMS:2015ooa} & [0.12;0.5] & 19.3 \\
				\hline
				\cellcolor{color8TeV} $C_{8}^{\mu\mu}$ &$gg\to \varphi_i^0 \to \mu\mu$ & CMS &\cite{CMS:2015ooa} & [0.12;0.5] & 19.3 \\
				\hline
				
				\cellcolor{color13TeV} $C_{13b}^{\mu\mu}$ &$bb\to \varphi_i^0 \to \mu\mu$ & CMS &\cite{CMS:2019mij} & [0.14;1] & 35.9 \\
				\hline
				\cellcolor{color13TeV} $C_{13}^{\mu\mu}$ &$gg\to \varphi_i^0 \to \mu\mu$ & CMS &\cite{CMS:2019mij} & [0.14;1] & 35.9 \\
				\hline
				
				\cellcolor{color13TeV} $A_{13b}^{\mu\mu}$ &$bb\to \varphi_i^0 \to \mu\mu$ & ATLAS &\cite{ATLAS:2019odt} & [0.2;1] & 36.1 \\
				\hline
				\cellcolor{color13TeV} $A_{13}^{\mu\mu}$ &$gg\to \varphi_i^0 \to \mu\mu$ & ATLAS &\cite{ATLAS:2019odt} & [0.2;1] & 36.1 \\
				\hline\hline
				
				
				
				% tautau
				\cellcolor{color8TeV} $A_{8}^{\tau\tau}$ &\multirow{2}{*}{$gg\to \varphi_i^0 \to \tau\tau$} & ATLAS &\cite{ATLAS:2014vhc} & [0.09;1] & 20 \\
				\cellcolor{color8TeV} $C_{8}^{\tau\tau}$  & & CMS &\cite{CMS:2015mca} &  [0.09;1]  &19.7 \\
				\hline
				\cellcolor{color8TeV} $A_{8b}^{\tau\tau}$  &\multirow{2}{*}{$bb\to \varphi_i^0 \to \tau\tau$} & ATLAS &\cite{ATLAS:2014vhc} & [0.09;1] & 20 \\
				\cellcolor{color8TeV} $C_{8b}^{\tau\tau}$ & & CMS & \cite{CMS:2015mca}& [0.09;1] & 19.7 \\
				\hline
				
				\cellcolor{color13TeV} $A_{13}^{\tau\tau}$ &$gg \to \varphi_i^0\to \tau \tau$ & ATLAS & \cite{ATLAS:2016ivh} & [0.2;1.2] & 3.2\\
				\cellcolor{color13TeV} $A_{13b}^{\tau\tau}$ &$bb \to \varphi_i^0\to \tau \tau$ & ATLAS & \cite{ATLAS:2016ivh} & [0.2;1.2] & 3.2\\
				\hline\hline
				
				
				\cellcolor{color13TeV} $A_{13}^{\tau\tau}$ &$gg \to \varphi_i^0\to \tau \tau$ & ATLAS & \cite{ATLAS:2020zms} & [0.2;2.5] & 139\\
				\cellcolor{color13TeV} $A_{13b}^{\tau\tau}$ &$bb \to \varphi_i^0\to \tau \tau$ & ATLAS & \cite{ATLAS:2020zms} & [0.2;2.5] & 139\\
				\hline\hline
				
				\cellcolor{color13TeV} $C_{13}^{\tau\tau}$ &$gg \to \varphi_i^0\to \tau \tau$ & CMS & \cite{CMS:2022rbd} & [0.06;3.5] & 138\\
				\cellcolor{color13TeV} $C_{13b}^{\tau\tau}$ &$bb \to \varphi_i^0\to \tau \tau$ & CMS & \cite{CMS:2022rbd} & [0.06;3.5] & 138\\
				\hline\hline
				
				\cellcolor{color13TeV} $A_{13}^{\tau\tau}$ &\multirow{2}{*}{$gg \to \varphi_i^0\to \tau \tau$} & ATLAS & \cite{ATLAS:2017eiz} & [0.2;2.25] & 36.1\\[-1pt]
				\cellcolor{color13TeV} $C_{13}^{\tau\tau}$ &  & CMS & \cite{CMS:2018rmh} & [0.09;3.2] & 35.9\\
				\hline
				\cellcolor{color13TeV} $A_{13b}^{\tau\tau}$ &\multirow{2}{*}{$bb \to \varphi_i^0\to \tau \tau$} & ATLAS & \cite{ATLAS:2017eiz} & [0.2;2.25] & 36.1\\[-1pt]
				\cellcolor{color13TeV} $C_{13b}^{\tau\tau}$ & & CMS & \cite{CMS:2018rmh} & [0.09;3.2] & 35.9\\
				\hline
				\hline
				\cellcolor{color8TeV} $A_{8}^{\gamma\gamma}$&$gg\to \varphi_i^0 \to \gamma\gamma$ & ATLAS &\cite{ATLAS:2014jdv} & [0.065;0.6] & 20.3 \\
				\hline
				\cellcolor{color13TeV} $C_{13}^{\gamma\gamma}$ & $gg \to \varphi_i^0\to \gamma \gamma$ & CMS & \cite{CMS:2016kgr} & [0.5;4] & 35.9\\ 
				\hline
				
				\cellcolor{color13TeV} $C_{13}^{\gamma\gamma}$ & $gg \to \varphi_i^0\to \gamma \gamma$ & CMS & \cite{CMS:2018dqv} & [0.5;5] & 35.9\\ 
				\cellcolor{color13TeV} $A_{13}^{\gamma\gamma}$ & $pp \to \varphi_i^0\to \gamma \gamma$ & ATLAS & \cite{ATLAS:2021uiz} & [0.15;3] & 139\\ 
				\hline
				
				\hline
				
				
				
				\cellcolor{color8TeV} $A_{8}^{Z\gamma}$  &\multirow{2}{*}{$pp\to \varphi_i^0 \to Z\gamma \to (\ell \ell) \gamma$} & ATLAS & \cite{ATLAS:2014lfk} & [0.2;1.6] & 20.3 \\
				\cellcolor{color8TeV} $C_{8}^{Z\gamma}$ & & CMS & \cite{CMS:2016all} & [0.2;1.2] & 19.7 \\
				\hline
				
				
				\cellcolor{color13TeV} $C_{13}^{\ell\ell\gamma}$  & $pp \to \varphi_i^0\to Z \gamma [\to (\ell \ell) \gamma ]$ & CMS & \cite{CMS:2017dyb} & [0.35;4] & 35.9\\
				\cellcolor{color13TeV} $C_{13}^{qq\gamma}$  & $pp \to \varphi_i^0\to Z \gamma [\to (qq) \gamma ]$ & CMS & \cite{CMS:2017dyb} & [0.35;4] & 35.9\\
				\cellcolor{color13TeV} $C_{13}^{Z\gamma}$  & $pp \to \varphi_i^0\to Z \gamma [\to (\ell \ell \,\&\, qq) \gamma ]$ & CMS & \cite{CMS:2017dyb} & [0.35;4] & 35.9\\
				\hline
				
				
				\cellcolor{color13TeV} $A_{13}^{\ell\ell\gamma}$  & $gg \to \varphi_i^0\to Z \gamma [\to (\ell \ell) \gamma ]$ & ATLAS & \cite{ATLAS:2017zdf} & [0.25;2.4] & 36.1\\
				\hline
				\cellcolor{color13TeV} $A_{13}^{qq\gamma}$  & $gg \to \varphi_i^0\to Z \gamma [\to (qq) \gamma ]$ & ATLAS & \cite{ATLAS:2018sxj} & [1;6.8] & 36.1\\
				\hline
				\cellcolor{color13TeV} $C_{8+13}^{Z\gamma}$  & $gg \to \varphi_i^0\to Z \gamma$ & CMS & \cite{CMS:2017dyb} & [0.35;4] & 35.9\\
				\hline
			\end{tabular}
		}
	}
	\caption{Direct searches for neutral heavy scalars, $\varphi_i^0 = H, A$, with quarks, leptons ($\ell= e,\mu$), photons and $Z\gamma$ final states. The parenthesis show the final decay of the SM particles produced from the NP particles. The square brackets are used when the values of $\sigma\cdot \mathcal{B}$ are shown in terms of the primary decay (i.e. the NP particle decay) but a particular decay channel of the SM particles is used to obtain those values. See Section~\ref{sec:direct_searches} for more details.}
	\label{tab:directsearches1}
\end{table}






\begin{table}[htb]
	\centering
	{\renewcommand{\arraystretch}{1.}
		\resizebox{\textwidth}{!}{%
			\begin{tabular}{|l|l|lc|c|c|}
				\hline
				\textbf{Label} &\textbf{Channel} & \multicolumn{2}{ l| }{\textbf{Experiment}} & \textbf{Mass range} & ${\cal L}$ \\
				&&&& \textbf{[TeV]} & \textbf{[fb$^{-1}$]} \\[1pt]
				\hline
				% ZZ
				\cellcolor{color8TeV} $A_{8}^{ZZ}$  &$gg\to \varphi_i^0\to ZZ$ & ATLAS & \cite{ATLAS:2015pre}& [0.14;1] & 20.3 \\
				\hline
				\cellcolor{color8TeV} $A_{8V}^{ZZ}$  &$VV \to \varphi_i^0\to ZZ$ & ATLAS & \cite{ATLAS:2015pre}& [0.14;1] & 20.3 \\
				\hline
				\cellcolor{color13TeV} $A_{13}^{2\ell2L}$  & $gg\to \varphi_i^0 \to ZZ [\to (\ell \ell) (\ell \ell, \nu \nu)]$ & ATLAS & \cite{ATLAS:2017tlw} & [0.2;1.2] & 36.1\\
				\hline
				\cellcolor{color13TeV} $A_{13V}^{2\ell2L}$  & $VV\to \varphi_i^0\to ZZ [\to (\ell \ell) (\ell \ell, \nu \nu)]$ & ATLAS & \cite{ATLAS:2017tlw} & [0.2;1.2] & 36.1\\
				\hline
				
				
				\cellcolor{color13TeV} $A_{13}^{2\ell2L}$  & $gg\to \varphi_i^0 \to ZZ [\to (\ell \ell) (\ell \ell, \nu \nu)]$ & ATLAS & \cite{ATLAS:2020tlo} & [0.2;2] & 139\\
				\hline
				\cellcolor{color13TeV} $A_{13V}^{2\ell2L}$  & $VV\to \varphi_i^0\to ZZ [\to (\ell \ell) (\ell \ell, \nu \nu)]$ & ATLAS & \cite{ATLAS:2020tlo} & [0.2;2] & 139\\
				\hline
				
				
				\cellcolor{color13TeV} $A_{13}^{2L2q}$ & $gg\to \varphi_i^0\to ZZ [\to (\ell \ell, \nu \nu) (qq)]$ & ATLAS & \cite{ATLAS:2017otj} & [0.3;3] & 36.1\\
				\hline
				\cellcolor{color13TeV} $A_{13V}^{2L2q}$ & $VV\to \varphi_i^0\to ZZ [\to (\ell \ell, \nu \nu) (qq)]$ & ATLAS & \cite{ATLAS:2017otj} & [0.3;3] & 36.1\\
				\hline
				\cellcolor{color13TeV} $C_{13}^{2\ell2X}$ & $pp\to \varphi_i^0\to ZZ [\to (\ell \ell) (qq,\nu\nu,\ell\ell)]$ & CMS & \cite{CMS:2018amk} & [0.13;3] & 35.9\\
				\hline
				\cellcolor{color13TeV} $C_{13}^{2q2\nu}$ & $pp\to \varphi_i^0\to ZZ [\to (qq)(\nu\nu)]$ & CMS & \cite{CMS:2018ygj} & [1;4] & 35.9\\
				\hline
				\hline
				% WW
				\cellcolor{color8TeV} $A_{8}^{WW}$  &$gg\to \varphi_i^0\to WW$ & ATLAS &\cite{ATLAS:2015iie}& [0.3;1.5] & 20.3 \\			
				\hline
				\cellcolor{color8TeV} $A_{8V}^{WW}$ &$VV \to \varphi_i^0\to WW$ & ATLAS & \cite{ATLAS:2015iie}& [0.3;1.5] & 20.3 \\
				
				\hline
				\hline
				\cellcolor{color13TeV} $C_{13V}^{WW}$ &$VV \to \varphi_i^0\to WW$ & CMS & \cite{CMS:2019bnu}& [0.2;3] & 35.9 \\
				\hline
				\hline
				
				\cellcolor{color13TeV} $A_{13}^{2(\ell\nu)}$  & $gg\to \varphi_i^0\to WW [\to (e \nu) (\mu \nu)]$ & ATLAS & \cite{ATLAS:2017uhp} & [0.2;4] & 36.1\\
				\hline
				\cellcolor{color13TeV} $A_{13V}^{2(\ell\nu)}$  & $VV\to \varphi_i^0\to WW [\to (e \nu) (\mu \nu)]$ & ATLAS & \cite{ATLAS:2017uhp} & [0.2;3] & 36.1\\
				\hline
				\cellcolor{color13TeV} $C_{13}^{2(\ell\nu)}$ & $(gg\!+\!VV)\to \varphi_i^0\to WW \to (\ell \nu) (\ell \nu)$ & CMS & \cite{CMS:2016jpd} & [0.2;1] & 2.3\\
				\hline
				\cellcolor{color13TeV} $A_{13}^{\ell\nu2q}$ & $gg\to \varphi_i^0\to WW[\to (\ell \nu) (qq)]$ & ATLAS & \cite{ATLAS:2017jag} & [0.3;3] & 36.1\\
				\hline
				\cellcolor{color13TeV} $A_{13V}^{\ell\nu2q}$ & $VV\to \varphi_i^0\to WW[\to (\ell \nu) (qq)]$ & ATLAS & \cite{ATLAS:2017jag} & [0.3;3] & 36.1\\
				\hline
				\cellcolor{color13TeV} $C_{13}^{\ell\nu2q}$ & $pp\to \varphi_i^0\to WW[\to (\ell \nu) (qq)]$ & CMS & \cite{CMS:2018dff} & [1;4.4] & 35.9\\
				\hline
				\hline
				%VV
				\cellcolor{color8TeV} $C_{8}^{VV}$  & $pp \to \varphi_i^0\to VV$ & CMS & \cite{CMS:2015hra} & [0.145;1] & 24.8 \\
				\cellcolor{color13TeV} $A_{13}^{4q}$  & $pp\to \varphi_i^0\to VV [\to (qq) (qq)]$ & ATLAS & \cite{ATLAS:2017zuf} & [1.2;3] & 36.7\\
				\hline
				
			\cellcolor{color13TeV} $A_{13}^{VV}$  & $pp \to \varphi_i^0\to VV$ & ATLAS & \cite{ATLAS:2018sbw} & [0.3;3] & 36.1 \\	
			
			\cellcolor{color13TeV} $A_{13V}^{VV}$  & $VV \to \varphi_i^0\to VV$ & ATLAS & \cite{ATLAS:2018sbw} & [0.3;3] & 36.1 \\	
			\hline\hline
			
			\cellcolor{color13TeV} $A_{13}^{VV}$  & $gg \to \varphi_i^0\to VV$ & ATLAS & \cite{ATLAS:2020fry} & [0.2;5.2] & 139 \\	
			
			\cellcolor{color13TeV} $A_{13V}^{VV}$  & $VV \to \varphi_i^0\to VV$ & ATLAS & \cite{ATLAS:2020fry} & [0.2;5.2] & 139 \\	
			\hline
			\hline
			
			\cellcolor{color13TeV} $C_{13}^{WW}$  & $gg \to \varphi_i^0\to WW$ & CMS & \cite{CMS:2021klu} & [1,4.5] & 137 \\	
			
			\cellcolor{color13TeV} $C_{13V}^{WW}$  & $VV \to \varphi_i^0\to WW$ & CMS & \cite{CMS:2021klu} & [1;4.5] & 137 \\	
			\hline
			\end{tabular}
		}
	}
	\caption{Direct searches for neutral heavy scalars, $\varphi_i^0 = H, A$, with vector-boson final states. $V = W,Z$, $\ell = e, \mu$. The parenthesis show the final decay of the SM particles produced from the NP particles. The square brackets are used when the values of $\sigma\cdot \mathcal{B}$ are shown in terms of the primary decay (i.e. the NP particle decay) but a particular decay channel of the SM particles is used to obtain those values. See Section~\ref{sec:direct_searches} for more details.}
	\label{tab:directsearches2}
\end{table}







\FloatBarrier







%%%%%%%%%%%%%%%%%%%%% 
%%%%%   Standard Model Inputs   %%
%%%%%%%%%%%%%%%%%%%%% 

\section{Non-contaminated Standard Model inputs}


In this work we have included as inputs the entries of the CKM matrix, in the Wolfenstein parametrisation \cite{Wolfenstein:1983yz}, and the oblique parameters \cite{Peskin:1990zt, Peskin:1991sw}. 
The experimental values of these parameters quoted by the Particle Data Group \cite{ParticleDataGroup:2022pth}  have been obtained through a SM fit of multiple observables, neglecting any new-physics contributions. Unfortunately, this assumption is not satisfied in our model for all the observables included in the fits, and we need to repeat those fits removing the problematic observables. 

\subsection{Non-contaminated CKM fit}
\label{sec:CKM_fit}

We extract the Wolfenstein parameters from the measured values of the CKM matrix elements (or ratios among them). 

\subsubsection{Determination of $|V_{ud}|$}

For $|V_{ud}|$ we use directly the value quoted by the PDG \cite{ParticleDataGroup:2022pth},
\begin{equation}
    |V_{ud}|=0.97373\pm 0.00031,
\end{equation}
since this value comes from superallowed $0^+\rightarrow 0^+$ nuclear $\beta$ decays \cite{Hardy:2020qwl} with completely negligible contamination from our additional scalars.

\subsubsection{Determination of $|V_{us}|$}

The determination of $|V_{us}|$ in the PDG \cite{ParticleDataGroup:2022pth} is based on semileptonic decays of kaons, averaging the electronic and muonic channel, and, independently, on the pure leptonic decay of kaons and pions to muons which provide the ratio $|V_{us}/V_{ud}|$. 
The scalar contribution to the semileptonic decays is highly suppressed by the light lepton masses. Therefore, as long as the mass of the kaons is much higher than the mass of the decaying lepton we can safely neglect the NP contribution in the semileptonic decays. This assumption holds for the semileptonic decay to electrons but it is not fulfilled for the muonic channel.  
 Therefore, for the semileptonic decays we only consider the decays to electrons, including the $K_L$, $K_S$ and $K^+$ decays, which give the average value 
 $|V_{us}f_+^K(0)|=0.21626\pm 0.00040$ \cite{Seng:2021nar}.
 Taking the form factor average.
% The form factor average 
$f_+(0)=0.9698\pm 0.0017$,  from $N_f= 2 + 1 + 1$ lattice QCD calculations \cite{FlavourLatticeAveragingGroupFLAG:2021npn}, gives $|V_{us}|=0.2230\pm 0.0006$.

As mentioned above, the alternative determination of $|V_{us}|$ is based on the leptonic kaon  decay. In this case, the SM contribution is helicity suppressed, making the scalar contribution relatively much higher. In general, the charged-scalar contribution to the leptonic decay of a pseudoscalar-meson  takes the form \cite{Jung:2010ik}
\begin{equation}
    \Gamma(P^+_{ij}\rightarrow l^+ \RF{\nu}_l) = \Gamma^{\rm SM}(P^+_{ij}\rightarrow l^+ \RF{\nu}_l)\; |1-\Delta_{ij}|^2\, ,
\end{equation}
with $i,j$ the flavour indices corresponding to the valence quarks of the meson, the new-physics correction
\begin{equation}
    \Delta_{ij}=\left(\frac{m_{P^+_{ij}}}{M_{H^\pm}}\right)^2 \varsigma_l^*\;\frac{\varsigma_u\, m_{ui}+\varsigma_d\, m_{dj}}{m_{ui}+ m_{dj}}\, ,
\end{equation}
and the SM contribution being related with the CKM matrix element by
\begin{equation}
    \Gamma^{\rm SM}(P^+_{ij}\rightarrow l^+ \RF{\nu}_l) = G_F^2\, m_l^2\, f^2_P\,|V_{ij}|^2 \, \frac{m_{P^+_{ij}}}{8\pi}\left(1-\frac{m_l^2}{m_{P^+_{ij}}}\right)^2
    \left(1+\delta_{\rm em}^{M\ell 2}\right)\, ,
\end{equation}
where $f_P$ is the meson decay constant and $\delta_{\rm em}^{M\ell 2}$ the electromagnetic radiative corrections.

In particular, in order to determine $|V_{us}|$, the ratio among the kaon and pion decay widths into muons is used, which gets a scalar contribution that is dominated by $2\Delta_{us}\approx 2\varsigma^*_l \varsigma_d m_K^2/M_{H^\pm}$. Since $\varsigma^*_l$ and $\varsigma_d$ can, in general, reach quite high values, we cannot neglect the scalar contribution in this case and, therefore, we cannot utilize this for the $|V_{us}|$ determination. 



%note that the determination of |Vus|, derived from the average of measurements involving leptonic decays (|Vus| = 0.2252 ± 0.0005), exhibits an incompatibility of more than two standard deviations with the determination derived from the averaging of measurements involving semileptonic decays (|Vus| = 0.2231 ± 0.0006)


Note that the determination of $|V_{us}|$ obtained from the average of measurements involving leptonic decays ($|V_{us}|= 0.2252\pm 0.0005$) is not compatible by more than two standard deviations with the determination derived from the average of measurements involving semileptonic decays ($|V_{us}|= 0.2231\pm 0.0006$). When these two determinations are combined in the PDG average, the total uncertainty of the mean value is multiplied by a factor of two ($|V_{us}|= 0.2243\pm 0.0008$) in order to account for the discrepancy among them.
In our case, we could attribute the discrepancy to the new-physics effects. However, using only the determination from semileptonic decays generates a deviation from unitarity in the first row of the CKM matrix above the three-sigma level, when combined with $|V_{ud}|$ from nuclear $\beta$ decays. This `Cabibbo anomaly' is already present in the PDG  average, although at a slightly lower level because the leptonic kaon decay pushes
the central value of $|V_{us}|$ to a more favorable direction and the uncertainty is increased. In order to relax the deviation from unitarity below the three-sigma level, and also motivated by the PDG procedure, we follow a more conservative approach increasing also the uncertainty on $|V_{us}|$ by a factor of two (as well as the PDG does). The final value used in our fits is then 
\begin{equation}
    |V_{us}|=0.2230\pm 0.0012\, .
\end{equation}



\subsubsection{Determination of $|V_{cd}|$}

The PDG \cite{ParticleDataGroup:2022pth} average is obtained from semileptonic decays of $D$ mesons to light leptons, leptonic $D$ decays to muons and taus, and from neutrino scattering data.
The scalar contribution to the leptonic decay can be sizable and, therefore, we will only use the data coming from semileptonic decays
($|V_{cd}|=0.2330\pm0.0136$)
and the neutrino scattering data
($|V_{cd}|=0.230\pm0.011$). 
Averaging these two values we obtain
\begin{equation}
    |V_{cd}|=0.231\pm0.009\, .
\end{equation}
Note, however, that the uncertainty is significantly higher than the one of $|V_{us}|$, so the impact of this measurement on the CKM-fit is basically negligible.

\subsubsection{Determination of $|V_{cs}|$}

Similarly to the previous case, the determination of $|V_{cs}|$ is obtained from measurements of semileptonic decays of $D$ mesons and the leptonic decay of $D_s$, provided that the form factors are obtained from lattice QCD computations. We have dismissed the determination from leptonic decays and we have used only the one coming from semileptonic decays:
\begin{equation}
    |V_{cs}|=0.972\pm 0.007\, .
\end{equation}
As happens for $|V_{cd}|$, we have a much higher uncertainty compared to the light-quark data ($|V_{ud}|$) so, again, this observable could be neglected from our fit leaving the results unchanged. 

\subsubsection{Determinations of $|V_{cb}|$ and $|V_{ub}|$}

The methods used to determine $|V_{cb}|$ and $|V_{ub}|$ in the PDG do not receive sizable contributions from our additional scalars. Note that, as before, the leptonic decay of the $B$ mesons would be affected by the NP but these processes are not used for the current world average due to their \tp{large} uncertainty. The values quoted by the PDG \cite{ParticleDataGroup:2022pth} are
\begin{equation}
   |V_{cb}|=(40.8 \pm 1.4)\times 10^{-3},
\end{equation}
and
\begin{equation}
   |V_{ub}|=(3.82 \pm 0.20)\times 10^{-3}.
\end{equation}

\subsubsection{Determination of $|V_{td}/V_{ts}|$}

Finally, we will use the determination of $|V_{td}/V_{ts}|$, which is obtained from measurements of $B^0_{(s)}-\bar{B}^0_{(s)}$ meson mixing. Obviously, the additional scalars will contribute to this mixing but, once the ratio of the $B_d$ and $B_s$ transitions is taken, the new-physics contribution is highly suppressed since it is only present through
SU(3)-breaking effects. We adopt the value for $|V_{td}/V_{ts}|$ quoted by the PDG \cite{ParticleDataGroup:2022pth}:
\begin{equation}
   |V_{td}/V_{ts}|=(0.207 \pm 0.001 \pm 0.003).
\end{equation}

\subsubsection{CKM fit result}
Using as inputs all the measurements mentioned in the previous sections, we obtain the values of Tab.~\ref{tab:WolfResult} for the Wolfenstein parameters. 




\begin{table}[h!]
\centering
\begin{tabular}{c|c|c|c|c|c}
    \multirow{2}{*}{ } & \multirow{2}{*}{Value} & \multicolumn{4}{c}{Correlation}  \\\cline{3-6}
    & & $ \lambda $ & $A$ & $\overline{\rho}$ & $ \overline{\eta}$\\ \hline
    $\lambda$ & $0.2249 \pm 0.0009$        & 1 & $-0.22$ & 0.16 & $-0.13$\\
    $A$ & $0.806  \pm 0.028$               & $-0.22$  & 1  & $-0.37$  & $-0.49$\\
    $ \overline{\rho}$ & $0.173\pm 0.016$  & 0.16 & $-0.37$ & 1 & 0.41  \\
    $ \overline{\eta}$  & $0.368\pm 0.024$ & $-0.13$ & $-0.49$ & 0.41 & 1 \\
\end{tabular}

\caption{Wolfenstein parameters obtained from a fit to
the CKM entries in Section \ref{sec:CKM_fit}.}
\label{tab:WolfResult}
\end{table}



\FloatBarrier


\subsection{Non-contaminated fit to the oblique parameters}
\label{sec:STU_fit}

For the determination of the oblique parameters a global fit of the EW observables is performed, removing the contribution from $R_b$. We use the same inputs as described in Ref.~\cite{deBlas:2022hdk}, but we remove $R_b$ and use the PDG value for $M_W$.


\begin{table}[h!]
\begin{center}
\begin{tabular}{c|c|c|c|c||c|c|c}
    \multirow{2}{*}{ } & \multirow{2}{*}{Value} & \multicolumn{3}{c||}{Correlation} & \multirow{2}{*}{Value} & \multicolumn{2}{c}{Correlation} \\\cline{3-5}\cline{7-8}
    & & $ S $ & $ T $ & $ U $ & & $S$ & $T$ \\ \hline
    $ S $ & 0.005 $\pm$ 0.096 & 1.00  &  0.91  &  -0.62 & 0.024 $\pm$ 0.076 & 1 & 0.91\\
     $ T $ & 0.042 $\pm$ 0.118 & 0.91  &  1.00  &  -0.84 & 0.075 $\pm$ 0.063 & 0.91 & 1\\
     $ U $ & 0.030 $\pm$ 0.091 & -0.62 &  -0.84 &  1.00 & & &\\
\end{tabular}
\end{center}
\caption{ Results for the fit of the oblique parameters $S$, $T$ and $U$, excluding the information from $R_b$.}
\label{tab:STU}
\end{table}

\FloatBarrier


%\bibliographystyle{mybst}
%\bibliography{References_ATHDM}


\begin{thebibliography}{100}
	
	\bibitem{Eberhardt:2020dat}
	O.~Eberhardt, A.~Pe\~nuelas Mart\'\i{}nez and A.~Pich, \emph{{Global fits in
			the Aligned Two-Higgs-Doublet model}},
	\href{https://doi.org/10.1007/JHEP05(2021)005}{\emph{JHEP} {\bfseries 05}
		(2021) 005} [\href{https://arxiv.org/abs/2012.09200}{{\ttfamily
			2012.09200}}].
	
	\bibitem{ATLAS:2012yve}
	{\scshape ATLAS} collaboration, \emph{{Observation of a new particle in the
			search for the Standard Model Higgs boson with the ATLAS detector at the
			LHC}}, \href{https://doi.org/10.1016/j.physletb.2012.08.020}{\emph{Phys.
			Lett. B} {\bfseries 716} (2012) 1}
	[\href{https://arxiv.org/abs/1207.7214}{{\ttfamily 1207.7214}}].
	
	\bibitem{CMS:2012qbp}
	{\scshape CMS} collaboration, \emph{{Observation of a New Boson at a Mass of
			125 GeV with the CMS Experiment at the LHC}},
	\href{https://doi.org/10.1016/j.physletb.2012.08.021}{\emph{Phys. Lett. B}
		{\bfseries 716} (2012) 30} [\href{https://arxiv.org/abs/1207.7235}{{\ttfamily
			1207.7235}}].
	
	\bibitem{UTfit:2022hsi}
	{\scshape UTfit} collaboration, \emph{{New UTfit Analysis of the Unitarity
			Triangle in the Cabibbo-Kobayashi-Maskawa scheme}},
	\href{https://doi.org/10.1007/s12210-023-01137-5}{\emph{Rend. Lincei Sci.
			Fis. Nat.} {\bfseries 34} (2023) 37}
	[\href{https://arxiv.org/abs/2212.03894}{{\ttfamily 2212.03894}}].
	
	\bibitem{ALEPH:2005ab}
	{\scshape ALEPH, DELPHI, L3, OPAL, SLD, LEP Electroweak Working Group, SLD
		Electroweak Group, SLD Heavy Flavour Group} collaboration, \emph{{Precision
			electroweak measurements on the $Z$ resonance}},
	\href{https://doi.org/10.1016/j.physrep.2005.12.006}{\emph{Phys. Rept.}
		{\bfseries 427} (2006) 257}
	[\href{https://arxiv.org/abs/hep-ex/0509008}{{\ttfamily hep-ex/0509008}}].
	
	\bibitem{Diaz-Cruz:2003kcx}
	J.L.~Diaz-Cruz and D.A.~Lopez-Falcon, \emph{{Probing the mechanism of EWSB with
			a rho parameter defined in terms of Higgs couplings}},
	\href{https://doi.org/10.1016/j.physletb.2003.06.047}{\emph{Phys. Lett. B}
		{\bfseries 568} (2003) 245}
	[\href{https://arxiv.org/abs/hep-ph/0304212}{{\ttfamily hep-ph/0304212}}].
	
	\bibitem{Branco:2011iw}
	G.C.~Branco, P.M.~Ferreira, L.~Lavoura, M.N.~Rebelo, M.~Sher and J.P.~Silva,
	\emph{{Theory and phenomenology of two-Higgs-doublet models}},
	\href{https://doi.org/10.1016/j.physrep.2012.02.002}{\emph{Phys. Rept.}
		{\bfseries 516} (2012) 1} [\href{https://arxiv.org/abs/1106.0034}{{\ttfamily
			1106.0034}}].
	
	\bibitem{Gunion:1989we}
	J.F.~Gunion, H.E.~Haber, G.L.~Kane and S.~Dawson, \emph{{The Higgs Hunter's
			Guide}}, {\emph{Front. Phys.} {\bfseries 80} (2000) 1}.
	
	\bibitem{Ivanov:2017dad}
	I.P.~Ivanov, \emph{{Building and testing models with extended Higgs sectors}},
	\href{https://doi.org/10.1016/j.ppnp.2017.03.001}{\emph{Prog. Part. Nucl.
			Phys.} {\bfseries 95} (2017) 160}
	[\href{https://arxiv.org/abs/1702.03776}{{\ttfamily 1702.03776}}].



\bibitem{Gunion:2005ja}
J.F.~Gunion and H.E.~Haber, \emph{{Conditions for CP-violation in the general
  two-Higgs-doublet model}},
  \href{https://doi.org/10.1103/PhysRevD.72.095002}{\emph{Phys. Rev. D}
  {\bfseries 72} (2005) 095002}
  [\href{https://arxiv.org/abs/hep-ph/0506227}{{\ttfamily hep-ph/0506227}}].

\bibitem{Wu:1994ja}
Y.L.~Wu and L.~Wolfenstein, \emph{{Sources of CP violation in the two Higgs
  doublet model}},
  \href{https://doi.org/10.1103/PhysRevLett.73.1762}{\emph{Phys. Rev. Lett.}
  {\bfseries 73} (1994) 1762}
  [\href{https://arxiv.org/abs/hep-ph/9409421}{{\ttfamily hep-ph/9409421}}].

\bibitem{Keus:2015hva}
V.~Keus, S.F.~King, S.~Moretti and K.~Yagyu, \emph{{CP Violating
  Two-Higgs-Doublet Model: Constraints and LHC Predictions}},
  \href{https://doi.org/10.1007/JHEP04(2016)048}{\emph{JHEP} {\bfseries 04}
  (2016) 048} [\href{https://arxiv.org/abs/1510.04028}{{\ttfamily
  1510.04028}}].

\bibitem{Chen:2017com}
C.-Y.~Chen, H.-L.~Li and M.~Ramsey-Musolf, \emph{{CP-Violation in the Two Higgs
  Doublet Model: from the LHC to EDMs}},
  \href{https://doi.org/10.1103/PhysRevD.97.015020}{\emph{Phys. Rev. D}
  {\bfseries 97} (2018) 015020}
  [\href{https://arxiv.org/abs/1708.00435}{{\ttfamily 1708.00435}}].

\bibitem{Iguro:2019zlc}
S.~Iguro and Y.~Omura, \emph{{The direct CP violation in a general two Higgs
  doublet model}}, \href{https://doi.org/10.1007/JHEP08(2019)098}{\emph{JHEP}
  {\bfseries 08} (2019) 098}
  [\href{https://arxiv.org/abs/1905.11778}{{\ttfamily 1905.11778}}].

\bibitem{Kim:1986ax}
J.E.~Kim, \emph{{Light Pseudoscalars, Particle Physics and Cosmology}},
  \href{https://doi.org/10.1016/0370-1573(87)90017-2}{\emph{Phys. Rept.}
  {\bfseries 150} (1987) 1}.

\bibitem{Espriu:2015mfa}
D.~Espriu, F.~Mescia and A.~Renau, \emph{{Axion-Higgs interplay in the two
  Higgs-doublet model}},
  \href{https://doi.org/10.1103/PhysRevD.92.095013}{\emph{Phys. Rev. D}
  {\bfseries 92} (2015) 095013}
  [\href{https://arxiv.org/abs/1503.02953}{{\ttfamily 1503.02953}}].

\bibitem{Celis:2014zaa}
A.~Celis, J.~Fuentes-Mart\'\i{}n and H.~Ser\^odio, \emph{{Effective Aligned
  2HDM with a DFSZ-like invisible axion}},
  \href{https://doi.org/10.1016/j.physletb.2014.08.032}{\emph{Phys. Lett. B}
  {\bfseries 737} (2014) 185}
  [\href{https://arxiv.org/abs/1407.0971}{{\ttfamily 1407.0971}}].

\bibitem{LopezHonorez:2006gr}
L.~Lopez~Honorez, E.~Nezri, J.F.~Oliver and M.H.G.~Tytgat, \emph{{The Inert
  Doublet Model: An Archetype for Dark Matter}},
  \href{https://doi.org/10.1088/1475-7516/2007/02/028}{\emph{JCAP} {\bfseries
  02} (2007) 028} [\href{https://arxiv.org/abs/hep-ph/0612275}{{\ttfamily
  hep-ph/0612275}}].

\bibitem{Belyaev:2016lok}
A.~Belyaev, G.~Cacciapaglia, I.P.~Ivanov, F.~Rojas-Abatte and M.~Thomas,
  \emph{{Anatomy of the Inert Two Higgs Doublet Model in the light of the LHC
  and non-LHC Dark Matter Searches}},
  \href{https://doi.org/10.1103/PhysRevD.97.035011}{\emph{Phys. Rev. D}
  {\bfseries 97} (2018) 035011}
  [\href{https://arxiv.org/abs/1612.00511}{{\ttfamily 1612.00511}}].

\bibitem{Tsai:2019eqi}
Y.-L.S.~Tsai, V.Q.~Tran and C.-T.~Lu, \emph{{Confronting dark matter
  co-annihilation of Inert two Higgs Doublet Model with a compressed mass
  spectrum}}, \href{https://doi.org/10.1007/JHEP06(2020)033}{\emph{JHEP}
  {\bfseries 06} (2020) 033}
  [\href{https://arxiv.org/abs/1912.08875}{{\ttfamily 1912.08875}}].

\bibitem{Ma:2006km}
E.~Ma, \emph{{Verifiable radiative seesaw mechanism of neutrino mass and dark
  matter}}, \href{https://doi.org/10.1103/PhysRevD.73.077301}{\emph{Phys. Rev.
  D} {\bfseries 73} (2006) 077301}
  [\href{https://arxiv.org/abs/hep-ph/0601225}{{\ttfamily hep-ph/0601225}}].

\bibitem{Hirsch:2013ola}
M.~Hirsch, R.A.~Lineros, S.~Morisi, J.~Palacio, N.~Rojas and J.W.F.~Valle,
  \emph{{WIMP dark matter as radiative neutrino mass messenger}},
  \href{https://doi.org/10.1007/JHEP10(2013)149}{\emph{JHEP} {\bfseries 10}
  (2013) 149} [\href{https://arxiv.org/abs/1307.8134}{{\ttfamily 1307.8134}}].

\bibitem{Turok:1990zg}
N.~Turok and J.~Zadrozny, \emph{{Electroweak baryogenesis in the two doublet
  model}}, \href{https://doi.org/10.1016/0550-3213(91)90356-3}{\emph{Nucl.
  Phys. B} {\bfseries 358} (1991) 471}.

\bibitem{Cline:2011mm}
J.M.~Cline, K.~Kainulainen and M.~Trott, \emph{{Electroweak Baryogenesis in Two
  Higgs Doublet Models and B meson anomalies}},
  \href{https://doi.org/10.1007/JHEP11(2011)089}{\emph{JHEP} {\bfseries 11}
  (2011) 089} [\href{https://arxiv.org/abs/1107.3559}{{\ttfamily 1107.3559}}].

\bibitem{Fuyuto:2015jha}
K.~Fuyuto and E.~Senaha, \emph{{Sphaleron and critical bubble in the scale
  invariant two Higgs doublet model}},
  \href{https://doi.org/10.1016/j.physletb.2015.05.061}{\emph{Phys. Lett. B}
  {\bfseries 747} (2015) 152}
  [\href{https://arxiv.org/abs/1504.04291}{{\ttfamily 1504.04291}}].

\bibitem{Ferreira:2015rha}
P.~Ferreira, H.E.~Haber and E.~Santos, \emph{{Preserving the validity of the
  Two-Higgs Doublet Model up to the Planck scale}},
  \href{https://doi.org/10.1103/PhysRevD.92.033003}{\emph{Phys. Rev. D}
  {\bfseries 92} (2015) 033003}
  [\href{https://arxiv.org/abs/1505.04001}{{\ttfamily 1505.04001}}].

\bibitem{Das:2015mwa}
D.~Das and I.~Saha, \emph{{Search for a stable alignment limit in
  two-Higgs-doublet models}},
  \href{https://doi.org/10.1103/PhysRevD.91.095024}{\emph{Phys. Rev. D}
  {\bfseries 91} (2015) 095024}
  [\href{https://arxiv.org/abs/1503.02135}{{\ttfamily 1503.02135}}].

\bibitem{Schuh:2018hig}
P.~Schuh, \emph{{Vacuum Stability of Asymptotically Safe Two Higgs Doublet
  Models}}, \href{https://doi.org/10.1140/epjc/s10052-019-7426-8}{\emph{Eur.
  Phys. J. C} {\bfseries 79} (2019) 909}
  [\href{https://arxiv.org/abs/1810.07664}{{\ttfamily 1810.07664}}].



    
 
	\bibitem{Glashow:1976nt}
	S.L.~Glashow and S.~Weinberg, \emph{{Natural Conservation Laws for Neutral
			Currents}}, \href{https://doi.org/10.1103/PhysRevD.15.1958}{\emph{Phys. Rev.
			D} {\bfseries 15} (1977) 1958}.
	
	\bibitem{Paschos:1976ay}
	E.A.~Paschos, \emph{{Diagonal Neutral Currents}},
	\href{https://doi.org/10.1103/PhysRevD.15.1966}{\emph{Phys. Rev. D}
		{\bfseries 15} (1977) 1966}.
	
	\bibitem{Pich:2009sp}
	A.~Pich and P.~Tuzon, \emph{{Yukawa Alignment in the Two-Higgs-Doublet Model}},
	\href{https://doi.org/10.1103/PhysRevD.80.091702}{\emph{Phys. Rev. D}
		{\bfseries 80} (2009) 091702}
	[\href{https://arxiv.org/abs/0908.1554}{{\ttfamily 0908.1554}}].
	
	\bibitem{Pich:2010ic}
	A.~Pich, \emph{{Flavour constraints on multi-Higgs-doublet models: Yukawa
			alignment}},
	\href{https://doi.org/10.1016/j.nuclphysbps.2010.12.030}{\emph{Nucl. Phys. B
			Proc. Suppl.} {\bfseries 209} (2010) 182}
	[\href{https://arxiv.org/abs/1010.5217}{{\ttfamily 1010.5217}}].
	
	\bibitem{Penuelas:2017ikk}
	A.~Pe\~nuelas and A.~Pich, \emph{{Flavour alignment in multi-Higgs-doublet
			models}}, \href{https://doi.org/10.1007/JHEP12(2017)084}{\emph{JHEP}
		{\bfseries 12} (2017) 084}
	[\href{https://arxiv.org/abs/1710.02040}{{\ttfamily 1710.02040}}].
	
	\bibitem{Manohar:2006ga}
	A.V.~Manohar and M.B.~Wise, \emph{{Flavor changing neutral currents, an
			extended scalar sector, and the Higgs production rate at the CERN LHC}},
	\href{https://doi.org/10.1103/PhysRevD.74.035009}{\emph{Phys. Rev. D}
		{\bfseries 74} (2006) 035009}
	[\href{https://arxiv.org/abs/hep-ph/0606172}{{\ttfamily hep-ph/0606172}}].
	
	\bibitem{Chivukula:1987py}
	R.S.~Chivukula and H.~Georgi, \emph{{Composite Technicolor Standard Model}},
	\href{https://doi.org/10.1016/0370-2693(87)90713-1}{\emph{Phys. Lett. B}
		{\bfseries 188} (1987) 99}.
	
	\bibitem{DAmbrosio:2002vsn}
	G.~D'Ambrosio, G.F.~Giudice, G.~Isidori and A.~Strumia, \emph{{Minimal flavor
			violation: An Effective field theory approach}},
	\href{https://doi.org/10.1016/S0550-3213(02)00836-2}{\emph{Nucl. Phys. B}
		{\bfseries 645} (2002) 155}
	[\href{https://arxiv.org/abs/hep-ph/0207036}{{\ttfamily hep-ph/0207036}}].
	
	\bibitem{Ferreira:2010xe}
	P.M.~Ferreira, L.~Lavoura and J.P.~Silva, \emph{{Renormalization-group
			constraints on Yukawa alignment in multi-Higgs-doublet models}},
	\href{https://doi.org/10.1016/j.physletb.2010.04.033}{\emph{Phys. Lett. B}
		{\bfseries 688} (2010) 341}
	[\href{https://arxiv.org/abs/1001.2561}{{\ttfamily 1001.2561}}].
	
	\bibitem{Jung:2010ik}
	M.~Jung, A.~Pich and P.~Tuzon, \emph{{Charged-Higgs phenomenology in the
			Aligned two-Higgs-doublet model}},
	\href{https://doi.org/10.1007/JHEP11(2010)003}{\emph{JHEP} {\bfseries 11}
		(2010) 003} [\href{https://arxiv.org/abs/1006.0470}{{\ttfamily 1006.0470}}].
	
	\bibitem{Braeuninger:2010td}
	C.B.~Braeuninger, A.~Ibarra and C.~Simonetto, \emph{{Radiatively induced
			flavour violation in the general two-Higgs doublet model with Yukawa
			alignment}},
	\href{https://doi.org/10.1016/j.physletb.2010.07.039}{\emph{Phys. Lett. B}
		{\bfseries 692} (2010) 189}
	[\href{https://arxiv.org/abs/1005.5706}{{\ttfamily 1005.5706}}].
	
	\bibitem{Bijnens:2011gd}
	J.~Bijnens, J.~Lu and J.~Rathsman, \emph{{Constraining General Two Higgs
			Doublet Models by the Evolution of Yukawa Couplings}},
	\href{https://doi.org/10.1007/JHEP05(2012)118}{\emph{JHEP} {\bfseries 05}
		(2012) 118} [\href{https://arxiv.org/abs/1111.5760}{{\ttfamily 1111.5760}}].
	
	\bibitem{Li:2014fea}
	X.-Q.~Li, J.~Lu and A.~Pich, \emph{{$B_{s,d}^0 \to \ell^+\ell^-$ Decays in the
			Aligned Two-Higgs-Doublet Model}},
	\href{https://doi.org/10.1007/JHEP06(2014)022}{\emph{JHEP} {\bfseries 06}
		(2014) 022} [\href{https://arxiv.org/abs/1404.5865}{{\ttfamily 1404.5865}}].
	
	
	
	\bibitem{Botella:2015yfa}
	F.J.~Botella, G.C.~Branco, A.M.~Coutinho, M.N.~Rebelo and J.I.~Silva-Marcos,
	\emph{{Natural Quasi-Alignment with two Higgs Doublets and RGE Stability}},
	\href{https://doi.org/10.1140/epjc/s10052-015-3487-5}{\emph{Eur. Phys. J. C}
		{\bfseries 75} (2015) 286}
	[\href{https://arxiv.org/abs/1501.07435}{{\ttfamily 1501.07435}}].
	
	\bibitem{Gori:2017qwg}
	S.~Gori, H.E.~Haber and E.~Santos, \emph{{High scale flavor alignment in
			two-Higgs doublet models and its phenomenology}},
	\href{https://doi.org/10.1007/JHEP06(2017)110}{\emph{JHEP} {\bfseries 06}
		(2017) 110} [\href{https://arxiv.org/abs/1703.05873}{{\ttfamily
			1703.05873}}].

   \bibitem{Abbas:2015cua}
	G.~Abbas, A.~Celis, X.-Q.~Li, J.~Lu and A.~Pich, \emph{{Flavour-changing top
			decays in the aligned two-Higgs-doublet model}},
	\href{https://doi.org/10.1007/JHEP06(2015)005}{\emph{JHEP} {\bfseries 06}
		(2015) 005} [\href{https://arxiv.org/abs/1503.06423}{{\ttfamily
			1503.06423}}].

   
	
	\bibitem{Celis:2013rcs}
	A.~Celis, V.~Ilisie and A.~Pich, \emph{{LHC constraints on two-Higgs doublet
			models}}, \href{https://doi.org/10.1007/JHEP07(2013)053}{\emph{JHEP}
		{\bfseries 07} (2013) 053} [\href{https://arxiv.org/abs/1302.4022}{{\ttfamily
			1302.4022}}].
	
	\bibitem{Celis:2013ixa}
	A.~Celis, V.~Ilisie and A.~Pich, \emph{{Towards a general analysis of LHC data
			within two-Higgs-doublet models}},
	\href{https://doi.org/10.1007/JHEP12(2013)095}{\emph{JHEP} {\bfseries 12}
		(2013) 095} [\href{https://arxiv.org/abs/1310.7941}{{\ttfamily 1310.7941}}].


       \bibitem{Jung:2010ab}
	M.~Jung, A.~Pich and P.~Tuzon, \emph{{The B -\ensuremath{>} Xs gamma Rate and
			CP Asymmetry within the Aligned Two-Higgs-Doublet Model}},
	\href{https://doi.org/10.1103/PhysRevD.83.074011}{\emph{Phys. Rev. D}
		{\bfseries 83} (2011) 074011}
	[\href{https://arxiv.org/abs/1011.5154}{{\ttfamily 1011.5154}}].
	
	\bibitem{Jung:2012vu}
	M.~Jung, X.-Q.~Li and A.~Pich, \emph{{Exclusive radiative B-meson decays within
			the aligned two-Higgs-doublet model}},
	\href{https://doi.org/10.1007/JHEP10(2012)063}{\emph{JHEP} {\bfseries 10}
		(2012) 063} [\href{https://arxiv.org/abs/1208.1251}{{\ttfamily 1208.1251}}].

 
	\bibitem{Ilisie:2014hea}
	V.~Ilisie and A.~Pich, \emph{{Low-mass fermiophobic charged Higgs phenomenology
			in two-Higgs-doublet models}},
	\href{https://doi.org/10.1007/JHEP09(2014)089}{\emph{JHEP} {\bfseries 09}
		(2014) 089} [\href{https://arxiv.org/abs/1405.6639}{{\ttfamily 1405.6639}}].


 
	\bibitem{Chowdhury:2017aav}
	D.~Chowdhury and O.~Eberhardt, \emph{{Update of Global Two-Higgs-Doublet Model
			Fits}}, \href{https://doi.org/10.1007/JHEP05(2018)161}{\emph{JHEP} {\bfseries
			05} (2018) 161} [\href{https://arxiv.org/abs/1711.02095}{{\ttfamily
			1711.02095}}].

   \bibitem{Haller:2018nnx}
J.~Haller, A.~Hoecker, R.~Kogler, K.~M\"onig, T.~Peiffer and J.~Stelzer,
  \emph{{Update of the global electroweak fit and constraints on
  two-Higgs-doublet models}},
  \href{https://doi.org/10.1140/epjc/s10052-018-6131-3}{\emph{Eur. Phys. J. C}
  {\bfseries 78} (2018) 675}
  [\href{https://arxiv.org/abs/1803.01853}{{\ttfamily 1803.01853}}].

	
	\bibitem{Cacchio:2016qyh}
	V.~Cacchio, D.~Chowdhury, O.~Eberhardt and C.W.~Murphy, \emph{{Next-to-leading
			order unitarity fits in Two-Higgs-Doublet models with soft $\mathbb{Z}_2$
			breaking}}, \href{https://doi.org/10.1007/JHEP11(2016)026}{\emph{JHEP}
		{\bfseries 11} (2016) 026}
	[\href{https://arxiv.org/abs/1609.01290}{{\ttfamily 1609.01290}}].
	
	\bibitem{Chowdhury:2015yja}
	D.~Chowdhury and O.~Eberhardt, \emph{{Global fits of the two-loop renormalized
			Two-Higgs-Doublet model with soft Z$_{2}$ breaking}},
	\href{https://doi.org/10.1007/JHEP11(2015)052}{\emph{JHEP} {\bfseries 11}
		(2015) 052} [\href{https://arxiv.org/abs/1503.08216}{{\ttfamily
			1503.08216}}].
	
	\bibitem{Eberhardt:2013uba}
	O.~Eberhardt, U.~Nierste and M.~Wiebusch, \emph{{Status of the
			two-Higgs-doublet model of type II}},
	\href{https://doi.org/10.1007/JHEP07(2013)118}{\emph{JHEP} {\bfseries 07}
		(2013) 118} [\href{https://arxiv.org/abs/1305.1649}{{\ttfamily 1305.1649}}].
	
	\bibitem{Wang:2013sha}
	L.~Wang and X.-F.~Han, \emph{{Status of the aligned two-Higgs-doublet model
			confronted with the Higgs data}},
	\href{https://doi.org/10.1007/JHEP04(2014)128}{\emph{JHEP} {\bfseries 04}
		(2014) 128} [\href{https://arxiv.org/abs/1312.4759}{{\ttfamily 1312.4759}}].
	
	\bibitem{Botella:2015hoa}
	F.J.~Botella, G.C.~Branco, M.~Nebot and M.N.~Rebelo, \emph{{Flavour Changing
			Higgs Couplings in a Class of Two Higgs Doublet Models}},
	\href{https://doi.org/10.1140/epjc/s10052-016-3993-0}{\emph{Eur. Phys. J. C}
		{\bfseries 76} (2016) 161}
	[\href{https://arxiv.org/abs/1508.05101}{{\ttfamily 1508.05101}}].

\bibitem{Craig:2015jba}
	N.~Craig, F.~D'Eramo, P.~Draper, S.~Thomas and H.~Zhang, \emph{{The Hunt for
			the Rest of the Higgs Bosons}},
	\href{https://doi.org/10.1007/JHEP06(2015)137}{\emph{JHEP} {\bfseries 06}
		(2015) 137} [\href{https://arxiv.org/abs/1504.04630}{{\ttfamily
			1504.04630}}].


\bibitem{Bernon:2014nxa}
J.~Bernon, J.F.~Gunion, Y.~Jiang and S.~Kraml, \emph{{Light Higgs bosons in
  Two-Higgs-Doublet Models}},
  \href{https://doi.org/10.1103/PhysRevD.91.075019}{\emph{Phys. Rev. D}
  {\bfseries 91} (2015) 075019}
  [\href{https://arxiv.org/abs/1412.3385}{{\ttfamily 1412.3385}}].

\bibitem{Bernon:2015qea}
J.~Bernon, J.F.~Gunion, H.E.~Haber, Y.~Jiang and S.~Kraml, \emph{{Scrutinizing
  the alignment limit in two-Higgs-doublet models: m$_h$=125 GeV}},
  \href{https://doi.org/10.1103/PhysRevD.92.075004}{\emph{Phys. Rev. D}
  {\bfseries 92} (2015) 075004}
  [\href{https://arxiv.org/abs/1507.00933}{{\ttfamily 1507.00933}}].

\bibitem{Bernon:2015wef}
J.~Bernon, J.F.~Gunion, H.E.~Haber, Y.~Jiang and S.~Kraml, \emph{{Scrutinizing
  the alignment limit in two-Higgs-doublet models. II. m$_H$=125 GeV}},
  \href{https://doi.org/10.1103/PhysRevD.93.035027}{\emph{Phys. Rev. D}
  {\bfseries 93} (2016) 035027}
  [\href{https://arxiv.org/abs/1511.03682}{{\ttfamily 1511.03682}}].



  



 
	
	
	
	\bibitem{Ilnicka:2015jba}
	A.~Ilnicka, M.~Krawczyk and T.~Robens, \emph{{Inert Doublet Model in light of
			LHC Run I and astrophysical data}},
	\href{https://doi.org/10.1103/PhysRevD.93.055026}{\emph{Phys. Rev. D}
		{\bfseries 93} (2016) 055026}
	[\href{https://arxiv.org/abs/1508.01671}{{\ttfamily 1508.01671}}].
	
	\bibitem{Belusca-Maito:2016dqe}
	H.~B\'elusca-Ma\"\i{}to, A.~Falkowski, D.~Fontes, J.C.~Rom\~ao and
	J.a.P.~Silva, \emph{{Higgs EFT for 2HDM and beyond}},
	\href{https://doi.org/10.1140/epjc/s10052-017-4745-5}{\emph{Eur. Phys. J. C}
		{\bfseries 77} (2017) 176}
	[\href{https://arxiv.org/abs/1611.01112}{{\ttfamily 1611.01112}}].
	
	\bibitem{Dercks:2018wch}
	D.~Dercks and T.~Robens, \emph{{Constraining the Inert Doublet Model using
			Vector Boson Fusion}},
	\href{https://doi.org/10.1140/epjc/s10052-019-7436-6}{\emph{Eur. Phys. J. C}
		{\bfseries 79} (2019) 924}
	[\href{https://arxiv.org/abs/1812.07913}{{\ttfamily 1812.07913}}].
	
	\bibitem{Ilnicka:2018def}
	A.~Ilnicka, T.~Robens and T.~Stefaniak, \emph{{Constraining Extended Scalar
			Sectors at the LHC and beyond}},
	\href{https://doi.org/10.1142/S0217732318300070}{\emph{Mod. Phys. Lett. A}
		{\bfseries 33} (2018) 1830007}
	[\href{https://arxiv.org/abs/1803.03594}{{\ttfamily 1803.03594}}].
	
	\bibitem{Sanyal:2019xcp}
	P.~Sanyal, \emph{{Limits on the Charged Higgs Parameters in the Two Higgs
			Doublet Model using CMS $\sqrt{s}=13$ TeV Results}},
	\href{https://doi.org/10.1140/epjc/s10052-019-7431-y}{\emph{Eur. Phys. J. C}
		{\bfseries 79} (2019) 913}
	[\href{https://arxiv.org/abs/1906.02520}{{\ttfamily 1906.02520}}].
	
	\bibitem{Herrero-Garcia:2019mcy}
	J.~Herrero-Garcia, M.~Nebot, F.~Rajec, M.~White and A.G.~Williams, \emph{{Higgs
			Quark Flavor Violation: Simplified Models and Status of General
			Two-Higgs-Doublet Model}},
	\href{https://doi.org/10.1007/JHEP02(2020)147}{\emph{JHEP} {\bfseries 02}
		(2020) 147} [\href{https://arxiv.org/abs/1907.05900}{{\ttfamily
			1907.05900}}].
	
	\bibitem{Karmakar:2019vnq}
	S.~Karmakar and S.~Rakshit, \emph{{Relaxed constraints on the heavy scalar
			masses in 2HDM}},
	\href{https://doi.org/10.1103/PhysRevD.100.055016}{\emph{Phys. Rev. D}
		{\bfseries 100} (2019) 055016}
	[\href{https://arxiv.org/abs/1901.11361}{{\ttfamily 1901.11361}}].
	
	\bibitem{Chen:2019pkq}
	N.~Chen, T.~Han, S.~Li, S.~Su, W.~Su and Y.~Wu, \emph{{Type-I 2HDM under the
			Higgs and Electroweak Precision Measurements}},
	\href{https://doi.org/10.1007/JHEP08(2020)131}{\emph{JHEP} {\bfseries 08}
		(2020) 131} [\href{https://arxiv.org/abs/1912.01431}{{\ttfamily
			1912.01431}}].
	
	\bibitem{Arco:2020ucn}
	F.~Arco, S.~Heinemeyer and M.J.~Herrero, \emph{{Exploring sizable triple Higgs
			couplings in the 2HDM}},
	\href{https://doi.org/10.1140/epjc/s10052-020-8406-8}{\emph{Eur. Phys. J. C}
		{\bfseries 80} (2020) 884}
	[\href{https://arxiv.org/abs/2005.10576}{{\ttfamily 2005.10576}}].
	
	\bibitem{Botella:2018gzy}
	F.J.~Botella, F.~Cornet-Gomez and M.~Nebot, \emph{{Flavor conservation in
			two-Higgs-doublet models}},
	\href{https://doi.org/10.1103/PhysRevD.98.035046}{\emph{Phys. Rev. D}
		{\bfseries 98} (2018) 035046}
	[\href{https://arxiv.org/abs/1803.08521}{{\ttfamily 1803.08521}}].
	
	\bibitem{Botella:2020xzf}
	F.J.~Botella, F.~Cornet-Gomez and M.~Nebot, \emph{{Electron and muon $g-2$
			anomalies in general flavour conserving two Higgs doublets models}},
	\href{https://doi.org/10.1103/PhysRevD.102.035023}{\emph{Phys. Rev. D}
		{\bfseries 102} (2020) 035023}
	[\href{https://arxiv.org/abs/2006.01934}{{\ttfamily 2006.01934}}].
	
	\bibitem{Botella:2022rte}
	F.J.~Botella, F.~Cornet-Gomez, C.~Mir\'o and M.~Nebot, \emph{{Muon and electron
			$g-2$ anomalies in a flavor conserving 2HDM with an oblique view on the CDF
			$M_W$ value}},
	\href{https://doi.org/10.1140/epjc/s10052-022-10893-x}{\emph{Eur. Phys. J. C}
		{\bfseries 82} (2022) 915}
	[\href{https://arxiv.org/abs/2205.01115}{{\ttfamily 2205.01115}}].
	
	\bibitem{Botella:2023tiw}
	F.J.~Botella, F.~Cornet-Gomez, C.~Mir\'o and M.~Nebot, \emph{{New Physics hints
			from $\tau$ scalar interactions and $(g-2)_{e,\mu}$}},
	\href{https://arxiv.org/abs/2302.05471}{{\ttfamily 2302.05471}}.
	
	\bibitem{Athron:2021auq}
	P.~Athron, C.~Balazs, T.E.~Gonzalo, D.~Jacob, F.~Mahmoudi and C.~Sierra,
	\emph{{Likelihood analysis of the flavour anomalies and g \textendash{} 2 in
			the general two Higgs doublet model}},
	\href{https://doi.org/10.1007/JHEP01(2022)037}{\emph{JHEP} {\bfseries 01}
		(2022) 037} [\href{https://arxiv.org/abs/2111.10464}{{\ttfamily
			2111.10464}}].
	
	\bibitem{Connell:2023jqq}
	J.~M.~Connell, P.~Ferreira and H.~E.~Haber,
	\emph{{Accommodating hints of new heavy scalars in the framework of the flavor-aligned two-Higgs-doublet model}},
	\href{https://doi.org/10.1103/PhysRevD.108.055031}{\emph{Phys. Rev. D} {\bfseries 108} (2023) 055031}
	[\href{https://arxiv.org/abs/2302.13697} {{\ttfamily 2302.13697}}].
	
	\bibitem{Karan:2023xze}
	A.~Karan, V.~Miralles and A.~Pich,
	\emph{{Aligned two Higgs doublet model and the global fits}},
	\href{https://doi.org/10.22323/1.449.0053}{PoS EPS-HEP2023 (2024) 053}
	[\href{https://arxiv.org/abs/2312.00514} {{\ttfamily 2312.00514}}].
	
	\bibitem{DeBlas:2019ehy}
	J.~De~Blas et~al., \emph{{$\texttt{HEPfit}$: a code for the combination of
			indirect and direct constraints on high energy physics models}},
	\href{https://doi.org/10.1140/epjc/s10052-020-7904-z}{\emph{Eur. Phys. J. C}
		{\bfseries 80} (2020) 456}
	[\href{https://arxiv.org/abs/1910.14012}{{\ttfamily 1910.14012}}].

\bibitem{CDF:2022hxs}
	{\scshape CDF} collaboration, \emph{{High-precision measurement of the $W$
			boson mass with the CDF II detector}},
	\href{https://doi.org/10.1126/science.abk1781}{\emph{Science} {\bfseries 376}
		(2022) 170}.

\bibitem{Aoyama:2020ynm}
	T.~Aoyama et~al., \emph{{The anomalous magnetic moment of the muon in the
			Standard Model}},
	\href{https://doi.org/10.1016/j.physrep.2020.07.006}{\emph{Phys. Rept.}
		{\bfseries 887} (2020) 1} [\href{https://arxiv.org/abs/2006.04822}{{\ttfamily
			2006.04822}}].
	
  
\bibitem{Borsanyi:2020mff}
	S.~Borsanyi et~al., \emph{{Leading hadronic contribution to the muon magnetic
			moment from lattice QCD}},
	\href{https://doi.org/10.1038/s41586-021-03418-1}{\emph{Nature} {\bfseries
			593} (2021) 51} [\href{https://arxiv.org/abs/2002.12347}{{\ttfamily
			2002.12347}}].
	
	\bibitem{Ce:2022kxy}
	M.~C\`e et~al., \emph{{Window observable for the hadronic vacuum polarization
			contribution to the muon g-2 from lattice QCD}},
	\href{https://doi.org/10.1103/PhysRevD.106.114502}{\emph{Phys. Rev. D}
		{\bfseries 106} (2022) 114502}
	[\href{https://arxiv.org/abs/2206.06582}{{\ttfamily 2206.06582}}].
	
	\bibitem{ExtendedTwistedMass:2022jpw}
	{\scshape Extended Twisted Mass} collaboration, \emph{{Lattice calculation of
			the short and intermediate time-distance hadronic vacuum polarization
			contributions to the muon magnetic moment using twisted-mass fermions}},
	\href{https://doi.org/10.1103/PhysRevD.107.074506}{\emph{Phys. Rev. D}
		{\bfseries 107} (2023) 074506}
	[\href{https://arxiv.org/abs/2206.15084}{{\ttfamily 2206.15084}}].

\bibitem{Davier:2010fmf}
M.~Davier, A.~Hoecker, G.~Lopez Castro, B.~Malaescu, X.~H.~Mo, G.~Toledo Sanchez, P.~Wang, C.~Z.~Yuan and Z.~Zhang,
\emph{{The Discrepancy Between $\tau$ and $e^+e^-$ Spectral Functions Revisited and the Consequences for the Muon Magnetic Anomaly}},
\href{https://doi.org/10.1140/epjc/s10052-009-1219-4}
{\emph{Eur. Phys. J. C} {\bfseries 66} (2010) 127-136}
[\href{https://arxiv.org/abs/0906.5443}{{\ttfamily 0906.5443}}].

 
	\bibitem{Pich:2013lsa}
	A.~Pich, \emph{{Precision Tau Physics}},
	\href{https://doi.org/10.1016/j.ppnp.2013.11.002}{\emph{Prog. Part. Nucl.
			Phys.} {\bfseries 75} (2014) 41}
	[\href{https://arxiv.org/abs/1310.7922}{{\ttfamily 1310.7922}}].







 
	\bibitem{deBlas:2021wap}
	J.~de~Blas, M.~Ciuchini, E.~Franco, A.~Goncalves, S.~Mishima, M.~Pierini
	et~al., \emph{{Global analysis of electroweak data in the Standard Model}},
	\href{https://doi.org/10.1103/PhysRevD.106.033003}{\emph{Phys. Rev. D}
		{\bfseries 106} (2022) 033003}
	[\href{https://arxiv.org/abs/2112.07274}{{\ttfamily 2112.07274}}].
	
	\bibitem{Durieux:2019rbz}
	G.~Durieux, A.~Irles, V.~Miralles, A.~Pe\~nuelas, R.~P\"oschl, M.~Perell\'o
	et~al., \emph{{The electro-weak couplings of the top and bottom quarks
			\textemdash{} Global fit and future prospects}},
	\href{https://doi.org/10.1007/JHEP12(2019)098}{\emph{JHEP} {\bfseries 12}
		(2019) 98} [\href{https://arxiv.org/abs/1907.10619}{{\ttfamily 1907.10619}}].
	
	\bibitem{Miralles:2021dyw}
	V.~Miralles, M.M.~L\'opez, M.M.~Ll\'acer, A.~Pe\~nuelas, M.~Perell\'o and
	M.~Vos, \emph{{The top quark electro-weak couplings after LHC Run 2}},
	\href{https://doi.org/10.1007/JHEP02(2022)032}{\emph{JHEP} {\bfseries 02}
		(2022) 032} [\href{https://arxiv.org/abs/2107.13917}{{\ttfamily
			2107.13917}}].
	
	\bibitem{Eberhardt:2021ebh}
	O.~Eberhardt, V.~Miralles and A.~Pich, \emph{{Constraints on coloured scalars
			from global fits}},
	\href{https://doi.org/10.1007/JHEP10(2021)123}{\emph{JHEP} {\bfseries 10}
		(2021) 123} [\href{https://arxiv.org/abs/2106.12235}{{\ttfamily
			2106.12235}}].
	
	\bibitem{Ivanov:2006yq}
	I.P.~Ivanov, \emph{{Minkowski space structure of the Higgs potential in 2HDM}},
	\href{https://doi.org/10.1103/PhysRevD.75.035001}{\emph{Phys. Rev. D}
		{\bfseries 75} (2007) 035001}
	[\href{https://arxiv.org/abs/hep-ph/0609018}{{\ttfamily hep-ph/0609018}}].
	
	\bibitem{Ivanov:2015nea}
	I.P.~Ivanov and J.P.~Silva, \emph{{Tree-level metastability bounds for the most
			general two Higgs doublet model}},
	\href{https://doi.org/10.1103/PhysRevD.92.055017}{\emph{Phys. Rev. D}
		{\bfseries 92} (2015) 055017}
	[\href{https://arxiv.org/abs/1507.05100}{{\ttfamily 1507.05100}}].
	
	\bibitem{Bahl:2022lio}
	H.~Bahl, M.~Carena, N.M.~Coyle, A.~Ireland and C.E.M.~Wagner, \emph{{New tools
			for dissecting the general 2HDM}},
	\href{https://doi.org/10.1007/JHEP03(2023)165}{\emph{JHEP} {\bfseries 03}
		(2023) 165} [\href{https://arxiv.org/abs/2210.00024}{{\ttfamily
			2210.00024}}].
	
	\bibitem{Ferreira:2004yd}
	P.M.~Ferreira, R.~Santos and A.~Barroso, \emph{{Stability of the tree-level
			vacuum in two Higgs doublet models against charge or CP spontaneous
			violation}},
	\href{https://doi.org/10.1016/j.physletb.2004.10.022}{\emph{Phys. Lett. B}
		{\bfseries 603} (2004) 219}
	[\href{https://arxiv.org/abs/hep-ph/0406231}{{\ttfamily hep-ph/0406231}}].
	
	\bibitem{Barroso:2013awa}
	A.~Barroso, P.M.~Ferreira, I.P.~Ivanov and R.~Santos, \emph{{Metastability
			bounds on the two Higgs doublet model}},
	\href{https://doi.org/10.1007/JHEP06(2013)045}{\emph{JHEP} {\bfseries 06}
		(2013) 045} [\href{https://arxiv.org/abs/1303.5098}{{\ttfamily 1303.5098}}].
	
	\bibitem{Ginzburg:2005dt}
	I.F.~Ginzburg and I.P.~Ivanov, \emph{{Tree-level unitarity constraints in the
			most general 2HDM}},
	\href{https://doi.org/10.1103/PhysRevD.72.115010}{\emph{Phys. Rev. D}
		{\bfseries 72} (2005) 115010}
	[\href{https://arxiv.org/abs/hep-ph/0508020}{{\ttfamily hep-ph/0508020}}].
	
%	\bibitem{Logan:2022uus}
%	H.E.~Logan, \emph{{Lectures on perturbative unitarity and decoupling in Higgs physics}},  \href{https://arxiv.org/abs/2207.01064}{{\ttfamily 2207.01064}}.
	
%	\bibitem{Cornwall:1974km}
%	J.M.~Cornwall, D.N.~Levin and G.~Tiktopoulos, \emph{{Derivation of Gauge			Invariance from High-Energy Unitarity Bounds on the s Matrix}},
%	\href{https://doi.org/10.1103/PhysRevD.10.1145}{\emph{Phys. Rev. D}	{\bfseries 10} (1974) 1145}.
	
	\bibitem{Allwicher:2021rtd}
	L.~Allwicher, P.~Arnan, D.~Barducci and M.~Nardecchia, \emph{{Perturbative
			unitarity constraints on generic Yukawa interactions}},
	\href{https://doi.org/10.1007/JHEP10(2021)129}{\emph{JHEP} {\bfseries 10}
		(2021) 129} [\href{https://arxiv.org/abs/2108.00013}{{\ttfamily
			2108.00013}}].
	
	\bibitem{Peskin:1990zt}
	M.E.~Peskin and T.~Takeuchi, \emph{{A New constraint on a strongly interacting
			Higgs sector}}, \href{https://doi.org/10.1103/PhysRevLett.65.964}{\emph{Phys.
			Rev. Lett.} {\bfseries 65} (1990) 964}.
	
	\bibitem{Peskin:1991sw}
	M.E.~Peskin and T.~Takeuchi, \emph{{Estimation of oblique electroweak
			corrections}}, \href{https://doi.org/10.1103/PhysRevD.46.381}{\emph{Phys.
			Rev. D} {\bfseries 46} (1992) 381}.
	
	\bibitem{Haber:2010bw}
	H.E.~Haber and D.~O'Neil, \emph{{Basis-independent methods for the
			two-Higgs-doublet model III: The CP-conserving limit, custodial symmetry, and
			the oblique parameters S, T, U}},
	\href{https://doi.org/10.1103/PhysRevD.83.055017}{\emph{Phys. Rev. D}
		{\bfseries 83} (2011) 055017}
	[\href{https://arxiv.org/abs/1011.6188}{{\ttfamily 1011.6188}}].
	
	\bibitem{Haber:1999zh}
	H.E.~Haber and H.E.~Logan, \emph{{Radiative corrections to the Z b anti-b
			vertex and constraints on extended Higgs sectors}},
	\href{https://doi.org/10.1103/PhysRevD.62.015011}{\emph{Phys. Rev. D}
		{\bfseries 62} (2000) 015011}
	[\href{https://arxiv.org/abs/hep-ph/9909335}{{\ttfamily hep-ph/9909335}}].
	
	\bibitem{Degrassi:2010ne}
	G.~Degrassi and P.~Slavich, \emph{{QCD Corrections in two-Higgs-doublet
			extensions of the Standard Model with Minimal Flavor Violation}},
	\href{https://doi.org/10.1103/PhysRevD.81.075001}{\emph{Phys. Rev. D}
		{\bfseries 81} (2010) 075001}
	[\href{https://arxiv.org/abs/1002.1071}{{\ttfamily 1002.1071}}].
	
	\bibitem{ParticleDataGroup:2022pth}
	{\scshape Particle Data Group} collaboration, \emph{{Review of Particle
			Physics}}, \href{https://doi.org/10.1093/ptep/ptac097}{\emph{PTEP} {\bfseries
			2022} (2022) 083C01}.
	
	
	
	\bibitem{deBlas:2016ojx}
	J.~de~Blas, M.~Ciuchini, E.~Franco, S.~Mishima, M.~Pierini, L.~Reina et~al.,
	\emph{{Electroweak precision observables and Higgs-boson signal strengths in
			the Standard Model and beyond: present and future}},
	\href{https://doi.org/10.1007/JHEP12(2016)135}{\emph{JHEP} {\bfseries 12}
		(2016) 135} [\href{https://arxiv.org/abs/1608.01509}{{\ttfamily
			1608.01509}}].
	
	\bibitem{deBlas:2022hdk}
	J.~de~Blas, M.~Pierini, L.~Reina and L.~Silvestrini, \emph{{Impact of the
			Recent Measurements of the Top-Quark and W-Boson Masses on Electroweak
			Precision Fits}},
	\href{https://doi.org/10.1103/PhysRevLett.129.271801}{\emph{Phys. Rev. Lett.}
		{\bfseries 129} (2022) 271801}
	[\href{https://arxiv.org/abs/2204.04204}{{\ttfamily 2204.04204}}].
	
	\bibitem{Wolfenstein:1983yz}
	L.~Wolfenstein, \emph{{Parametrization of the Kobayashi-Maskawa Matrix}},
	\href{https://doi.org/10.1103/PhysRevLett.51.1945}{\emph{Phys. Rev. Lett.}
		{\bfseries 51} (1983) 1945}.
	
	\bibitem{Charles:2004jd}
	{\scshape CKMfitter Group} collaboration, \emph{{CP violation and the CKM
			matrix: Assessing the impact of the asymmetric $B$ factories}},
	\href{https://doi.org/10.1140/epjc/s2005-02169-1}{\emph{Eur. Phys. J. C}
		{\bfseries 41} (2005) 1}
	[\href{https://arxiv.org/abs/hep-ph/0406184}{{\ttfamily hep-ph/0406184}}].
	
	\bibitem{UTfit:2006vpt}
	{\scshape UTfit} collaboration, \emph{{The Unitarity Triangle Fit in the
			Standard Model and Hadronic Parameters from Lattice QCD: A Reappraisal after
			the Measurements of Delta m(s) and BR(B ---\ensuremath{>} tau nu(tau))}},
	\href{https://doi.org/10.1088/1126-6708/2006/10/081}{\emph{JHEP} {\bfseries
			10} (2006) 081} [\href{https://arxiv.org/abs/hep-ph/0606167}{{\ttfamily
			hep-ph/0606167}}].
	
	\bibitem{UTfit:2005lis}
	{\scshape UTfit} collaboration, \emph{{The UTfit collaboration report on the
			status of the unitarity triangle beyond the standard model. I.
			Model-independent analysis and minimal flavor violation}},
	\href{https://doi.org/10.1088/1126-6708/2006/03/080}{\emph{JHEP} {\bfseries
			03} (2006) 080} [\href{https://arxiv.org/abs/hep-ph/0509219}{{\ttfamily
			hep-ph/0509219}}].
	
	\bibitem{Chang:2015rva}
	Q.~Chang, P.-F.~Li and X.-Q.~Li, \emph{{${B_s^0}$ \textendash{} ${\bar{B}}_s^0$
			mixing within minimal flavor-violating two-Higgs-doublet models}},
	\href{https://doi.org/10.1140/epjc/s10052-015-3813-y}{\emph{Eur. Phys. J. C}
		{\bfseries 75} (2015) 594}
	[\href{https://arxiv.org/abs/1505.03650}{{\ttfamily 1505.03650}}].
	
	\bibitem{Bobeth:1999ww}
	C.~Bobeth, M.~Misiak and J.~Urban, \emph{{Matching conditions for $b \to s
			\gamma$ and $b \to s gluon$ in extensions of the standard model}},
	\href{https://doi.org/10.1016/S0550-3213(99)00688-4}{\emph{Nucl. Phys. B}
		{\bfseries 567} (2000) 153}
	[\href{https://arxiv.org/abs/hep-ph/9904413}{{\ttfamily hep-ph/9904413}}].
	
	\bibitem{Misiak:2006ab}
	M.~Misiak and M.~Steinhauser, \emph{{NNLO QCD corrections to the anti-B
			---\ensuremath{>} X(s) gamma matrix elements using interpolation in m(c)}},
	\href{https://doi.org/10.1016/j.nuclphysb.2006.11.027}{\emph{Nucl. Phys. B}
		{\bfseries 764} (2007) 62}
	[\href{https://arxiv.org/abs/hep-ph/0609241}{{\ttfamily hep-ph/0609241}}].
	
	\bibitem{Misiak:2006zs}
	M.~Misiak et~al., \emph{{Estimate of $\mathcal{B} (\bar B \to X_s \gamma)$ at
			$O(\alpha_s^2)$}},
	\href{https://doi.org/10.1103/PhysRevLett.98.022002}{\emph{Phys. Rev. Lett.}
		{\bfseries 98} (2007) 022002}
	[\href{https://arxiv.org/abs/hep-ph/0609232}{{\ttfamily hep-ph/0609232}}].
	
	
	
	\bibitem{Hermann:2012fc}
	T.~Hermann, M.~Misiak and M.~Steinhauser, \emph{{$\bar{B}\to X_s \gamma$ in the
			Two Higgs Doublet Model up to Next-to-Next-to-Leading Order in QCD}},
	\href{https://doi.org/10.1007/JHEP11(2012)036}{\emph{JHEP} {\bfseries 11}
		(2012) 036} [\href{https://arxiv.org/abs/1208.2788}{{\ttfamily 1208.2788}}].
	
	\bibitem{Misiak:2015xwa}
	M.~Misiak et~al., \emph{{Updated NNLO QCD predictions for the weak radiative
			B-meson decays}},
	\href{https://doi.org/10.1103/PhysRevLett.114.221801}{\emph{Phys. Rev. Lett.}
		{\bfseries 114} (2015) 221801}
	[\href{https://arxiv.org/abs/1503.01789}{{\ttfamily 1503.01789}}].
	
	\bibitem{Misiak:2017woa}
	M.~Misiak, A.~Rehman and M.~Steinhauser, \emph{{NNLO QCD counterterm
			contributions to $\bar B \to X_{s\gamma}$ for the physical value of $m_c$}},
	\href{https://doi.org/10.1016/j.physletb.2017.05.008}{\emph{Phys. Lett. B}
		{\bfseries 770} (2017) 431}
	[\href{https://arxiv.org/abs/1702.07674}{{\ttfamily 1702.07674}}].
	
	\bibitem{Misiak:2020vlo}
	M.~Misiak, A.~Rehman and M.~Steinhauser, \emph{{Towards $ \overline{B}\to
			{X}_s\gamma $ at the NNLO in QCD without interpolation in m$_{c}$}},
	\href{https://doi.org/10.1007/JHEP06(2020)175}{\emph{JHEP} {\bfseries 06}
		(2020) 175} [\href{https://arxiv.org/abs/2002.01548}{{\ttfamily
			2002.01548}}].
	
	\bibitem{Arnan:2017lxi}
	P.~Arnan, D.~Be\v{c}irevi\'c, F.~Mescia and O.~Sumensari, \emph{{Two Higgs
			doublet models and $b\rightarrow s$ exclusive decays}},
	\href{https://doi.org/10.1140/epjc/s10052-017-5370-z}{\emph{Eur. Phys. J. C}
		{\bfseries 77} (2017) 796}
	[\href{https://arxiv.org/abs/1703.03426}{{\ttfamily 1703.03426}}].

 \bibitem{Ilisie:2015tra}
V.~Ilisie, \emph{{New Barr-Zee contributions to $\mathbf{(g-2)_\mu}$ in two-Higgs-doublet models}},
\href{doi:10.1007/JHEP04(2015)077}{\emph{JHEP} {\bfseries 04}  (2015) 077}
[\href{https://arxiv.org/abs/1502.04199}{{\ttfamily 1502.04199}}].
		
	\bibitem{Muong-2:2021ojo}
	{\scshape Muon g-2} collaboration, \emph{{Measurement of the Positive Muon
			Anomalous Magnetic Moment to 0.46 ppm}},
	\href{https://doi.org/10.1103/PhysRevLett.126.141801}{\emph{Phys. Rev. Lett.}
		{\bfseries 126} (2021) 141801}
	[\href{https://arxiv.org/abs/2104.03281}{{\ttfamily 2104.03281}}].

\bibitem{Muong-2:2023cdq}
{\scshape Muon g-2} collaboration, \emph{{Measurement of the Positive Muon Anomalous Magnetic Moment to 0.20~ppm}},
\href{https://doi.org/10.1103/PhysRevLett.131.161802}
{\emph{Phys. Rev. Lett.} {\bfseries 131} (2023) 161802}
[\href{https://arxiv.org/abs/2308.06230}{{\ttfamily 2308.06230}}].

	
	\bibitem{Davier:2023hhn}
	M.~Davier, D.~D\'\i{}az-Calder\'on, B.~Malaescu, A.~Pich,
	A.~Rodr\'\i{}guez-S\'anchez and Z.~Zhang, \emph{{The Euclidean Adler function
			and its interplay with $ \Delta {\alpha}_{\textrm{QED}}^{\textrm{had}} $ and
			\ensuremath{\alpha}$_{s}$}},
	\href{https://doi.org/10.1007/JHEP04(2023)067}{\emph{JHEP} {\bfseries 04}
		(2023) 067} [\href{https://arxiv.org/abs/2302.01359}{{\ttfamily
			2302.01359}}].
	
	\bibitem{CMD-3:2023alj}
	{\scshape CMD-3} collaboration, \emph{{Measurement of the $e^+e^-\to\pi^+\pi^-$
			cross section from threshold to 1.2 GeV with the CMD-3 detector}},
	\href{https://arxiv.org/abs/2302.08834}{{\ttfamily 2302.08834}}.
	
	\bibitem{ATLAS:2017rzl}
	{\scshape ATLAS} collaboration, \emph{{Measurement of the $W$-boson mass in pp
			collisions at $\sqrt{s}=7$ TeV with the ATLAS detector}},
	\href{https://doi.org/10.1140/epjc/s10052-017-5475-4}{\emph{Eur. Phys. J. C}
		{\bfseries 78} (2018) 110}
	[\href{https://arxiv.org/abs/1701.07240}{{\ttfamily 1701.07240}}].
	
	\bibitem{ATLAS:2023fsi}
	{\scshape ATLAS} collaboration, \emph{{Improved W boson Mass Measurement using
			7 TeV Proton-Proton Collisions with the ATLAS Detector}},
	{\emph{ATLAS-CONF-2023-004} (2023) }.
	
	\bibitem{ATLAS:2021zwx}
	{\scshape ATLAS} collaboration, \emph{{Direct constraint on the Higgs-charm
			coupling from a search for Higgs boson decays to charm quarks with the ATLAS
			detector}}, {\emph{ATLAS-CONF-2021-021} (2021) }.
	
	\bibitem{CMS:2019hve}
	{\scshape CMS} collaboration, \emph{{A search for the standard model Higgs
			boson decaying to charm quarks}},
	\href{https://doi.org/10.1007/JHEP03(2020)131}{\emph{JHEP} {\bfseries 03}
		(2020) 131} [\href{https://arxiv.org/abs/1912.01662}{{\ttfamily
			1912.01662}}].
	
	\bibitem{ATLAS:2014vuz}
	{\scshape ATLAS} collaboration, \emph{{Search for the $b\bar{b}$ decay of the
			Standard Model Higgs boson in associated $(W/Z)H$ production with the ATLAS
			detector}}, \href{https://doi.org/10.1007/JHEP01(2015)069}{\emph{JHEP}
		{\bfseries 01} (2015) 069} [\href{https://arxiv.org/abs/1409.6212}{{\ttfamily
			1409.6212}}].
	
	\bibitem{ATLAS:2015utn}
	{\scshape ATLAS} collaboration, \emph{{Search for the Standard Model Higgs
			boson produced in association with top quarks and decaying into $b\bar{b}$ in
			pp collisions at $\sqrt{s}$ = 8 TeV with the ATLAS detector}},
	\href{https://doi.org/10.1140/epjc/s10052-015-3543-1}{\emph{Eur. Phys. J. C}
		{\bfseries 75} (2015) 349}
	[\href{https://arxiv.org/abs/1503.05066}{{\ttfamily 1503.05066}}].
	
	\bibitem{CMS:2013poe}
	{\scshape CMS} collaboration, \emph{{Search for the Standard Model Higgs Boson
			Produced in Association with a W or a Z Boson and Decaying to Bottom
			Quarks}}, \href{https://doi.org/10.1103/PhysRevD.89.012003}{\emph{Phys. Rev.
			D} {\bfseries 89} (2014) 012003}
	[\href{https://arxiv.org/abs/1310.3687}{{\ttfamily 1310.3687}}].
	
	\bibitem{CMS:2014tll}
	{\scshape CMS} collaboration, \emph{{Search for the associated production of
			the Higgs boson with a top-quark pair}},
	\href{https://doi.org/10.1007/JHEP09(2014)087}{\emph{JHEP} {\bfseries 09}
		(2014) 087} [\href{https://arxiv.org/abs/1408.1682}{{\ttfamily 1408.1682}}].
	
	\bibitem{ATLAS:2020bhl}
	{\scshape ATLAS} collaboration, \emph{{Measurements of Higgs bosons decaying to
			bottom quarks from vector boson fusion production with the ATLAS experiment
			at $\sqrt{s}=13\,\text {TeV}$}},
	\href{https://doi.org/10.1140/epjc/s10052-021-09192-8}{\emph{Eur. Phys. J. C}
		{\bfseries 81} (2021) 537}
	[\href{https://arxiv.org/abs/2011.08280}{{\ttfamily 2011.08280}}].
	
	\bibitem{ATLAS:2020fcp}
	{\scshape ATLAS} collaboration, \emph{{Measurements of $WH$ and $ZH$ production
			in the $H \rightarrow b\bar{b}$ decay channel in $pp$ collisions at 13 TeV
			with the ATLAS detector}},
	\href{https://doi.org/10.1140/epjc/s10052-020-08677-2}{\emph{Eur. Phys. J. C}
		{\bfseries 81} (2021) 178}
	[\href{https://arxiv.org/abs/2007.02873}{{\ttfamily 2007.02873}}].
	
	\bibitem{CMS:2016mmc}
	{\scshape CMS} collaboration, \emph{{VBF H to bb using the 2015 data sample}},
	{\emph{CMS-PAS-HIG-16-003} (2016) }.
	
	\bibitem{CMS:2018nsn}
	{\scshape CMS} collaboration, \emph{{Observation of Higgs boson decay to bottom
			quarks}}, \href{https://doi.org/10.1103/PhysRevLett.121.121801}{\emph{Phys.
			Rev. Lett.} {\bfseries 121} (2018) 121801}
	[\href{https://arxiv.org/abs/1808.08242}{{\ttfamily 1808.08242}}].
	
	\bibitem{ATLAS:2014cnc}
	{\scshape ATLAS} collaboration, \emph{{Measurement of Higgs boson production in
			the diphoton decay channel in pp collisions at center-of-mass energies of 7
			and 8 TeV with the ATLAS detector}},
	\href{https://doi.org/10.1103/PhysRevD.90.112015}{\emph{Phys. Rev. D}
		{\bfseries 90} (2014) 112015}
	[\href{https://arxiv.org/abs/1408.7084}{{\ttfamily 1408.7084}}].
	
	\bibitem{CMS:2014afl}
	{\scshape CMS} collaboration, \emph{{Observation of the Diphoton Decay of the
			Higgs Boson and Measurement of Its Properties}},
	\href{https://doi.org/10.1140/epjc/s10052-014-3076-z}{\emph{Eur. Phys. J. C}
		{\bfseries 74} (2014) 3076}
	[\href{https://arxiv.org/abs/1407.0558}{{\ttfamily 1407.0558}}].
	
	\bibitem{ATLAS:2020pvn}
	{\scshape ATLAS} collaboration, \emph{{Measurement of the properties of Higgs
			boson production at $\sqrt{s}$=13 TeV in the $H\to \gamma\gamma$ channel
			using 139 fb$^{-1}$ of $pp$ collision data with the ATLAS experiment}},
	{\emph{ATLAS-CONF-2020-026} (2020) }.
	
	\bibitem{CMS:2021kom}
	{\scshape CMS} collaboration, \emph{{Measurements of Higgs boson production
			cross sections and couplings in the diphoton decay channel at $
			\sqrt{\mathrm{s}} $ = 13 TeV}},
	\href{https://doi.org/10.1007/JHEP07(2021)027}{\emph{JHEP} {\bfseries 07}
		(2021) 027} [\href{https://arxiv.org/abs/2103.06956}{{\ttfamily
			2103.06956}}].
	
	\bibitem{ATLAS:2016neq}
	{\scshape ATLAS, CMS} collaboration, \emph{{Measurements of the Higgs boson
			production and decay rates and constraints on its couplings from a combined
			ATLAS and CMS analysis of the LHC pp collision data at $ \sqrt{s}=7 $ and 8
			TeV}}, \href{https://doi.org/10.1007/JHEP08(2016)045}{\emph{JHEP} {\bfseries
			08} (2016) 045} [\href{https://arxiv.org/abs/1606.02266}{{\ttfamily
			1606.02266}}].
	
	\bibitem{ATLAS:2020fzp}
	{\scshape ATLAS} collaboration, \emph{{A search for the dimuon decay of the
			Standard Model Higgs boson with the ATLAS detector}},
	\href{https://doi.org/10.1016/j.physletb.2020.135980}{\emph{Phys. Lett. B}
		{\bfseries 812} (2021) 135980}
	[\href{https://arxiv.org/abs/2007.07830}{{\ttfamily 2007.07830}}].
	
	\bibitem{CMS:2020xwi}
	{\scshape CMS} collaboration, \emph{{Evidence for Higgs boson decay to a pair
			of muons}}, \href{https://doi.org/10.1007/JHEP01(2021)148}{\emph{JHEP}
		{\bfseries 01} (2021) 148}
	[\href{https://arxiv.org/abs/2009.04363}{{\ttfamily 2009.04363}}].
	
	\bibitem{ATLAS:2015xst}
	{\scshape ATLAS} collaboration, \emph{{Evidence for the Higgs-boson Yukawa
			coupling to tau leptons with the ATLAS detector}},
	\href{https://doi.org/10.1007/JHEP04(2015)117}{\emph{JHEP} {\bfseries 04}
		(2015) 117} [\href{https://arxiv.org/abs/1501.04943}{{\ttfamily
			1501.04943}}].
	
	\bibitem{CMS:2014wdm}
	{\scshape CMS} collaboration, \emph{{Evidence for the 125 GeV Higgs boson
			decaying to a pair of $\tau$ leptons}},
	\href{https://doi.org/10.1007/JHEP05(2014)104}{\emph{JHEP} {\bfseries 05}
		(2014) 104} [\href{https://arxiv.org/abs/1401.5041}{{\ttfamily 1401.5041}}].
	
	\bibitem{ATLAS:2020qdt}
	{\scshape ATLAS} collaboration, \emph{{A combination of measurements of Higgs
			boson production and decay using up to $139$ fb$^{-1}$ of proton--proton
			collision data at $\sqrt{s}=$ 13 TeV collected with the ATLAS experiment}},
	{\emph{ATLAS-CONF-2020-027} (2020) }.
	
	\bibitem{CMS:2020gsy}
	{\scshape CMS} collaboration, \emph{{Combined Higgs boson production and decay
			measurements with up to 137 fb$^{-1}$ of proton-proton collision data at
			$\sqrt s$ = 13 TeV}}, {\emph{CMS-PAS-HIG-19-005} (2020) }.
	
	\bibitem{ATLAS:2014aga}
	{\scshape ATLAS} collaboration, \emph{{Observation and measurement of Higgs
			boson decays to WW$^*$ with the ATLAS detector}},
	\href{https://doi.org/10.1103/PhysRevD.92.012006}{\emph{Phys. Rev. D}
		{\bfseries 92} (2015) 012006}
	[\href{https://arxiv.org/abs/1412.2641}{{\ttfamily 1412.2641}}].
	
	\bibitem{ATLAS:2015muc}
	{\scshape ATLAS} collaboration, \emph{{Study of (W/Z)H production and Higgs
			boson couplings using $H \rightarrow WW^{\ast}$ decays with the ATLAS
			detector}}, \href{https://doi.org/10.1007/JHEP08(2015)137}{\emph{JHEP}
		{\bfseries 08} (2015) 137}
	[\href{https://arxiv.org/abs/1506.06641}{{\ttfamily 1506.06641}}].
	
	\bibitem{CMS:2013zmy}
	{\scshape CMS} collaboration, \emph{{Measurement of Higgs Boson Production and
			Properties in the WW Decay Channel with Leptonic Final States}},
	\href{https://doi.org/10.1007/JHEP01(2014)096}{\emph{JHEP} {\bfseries 01}
		(2014) 096} [\href{https://arxiv.org/abs/1312.1129}{{\ttfamily 1312.1129}}].
	
	\bibitem{ATLAS:2015egz}
	{\scshape ATLAS} collaboration, \emph{{Measurements of the Higgs boson
			production and decay rates and coupling strengths using pp collision data at
			$\sqrt{s}=7$ and 8 TeV in the ATLAS experiment}},
	\href{https://doi.org/10.1140/epjc/s10052-015-3769-y}{\emph{Eur. Phys. J. C}
		{\bfseries 76} (2016) 6} [\href{https://arxiv.org/abs/1507.04548}{{\ttfamily
			1507.04548}}].
	
	\bibitem{CMS:2013rmy}
	{\scshape CMS} collaboration, \emph{{Search for a Higgs Boson Decaying into a Z
			and a Photon in $pp$ Collisions at $\sqrt{s}$ = 7 and 8 TeV}},
	\href{https://doi.org/10.1016/j.physletb.2013.09.057}{\emph{Phys. Lett. B}
		{\bfseries 726} (2013) 587}
	[\href{https://arxiv.org/abs/1307.5515}{{\ttfamily 1307.5515}}].
	
	\bibitem{ATLAS:2020qcv}
	{\scshape ATLAS} collaboration, \emph{{A search for the $Z\gamma$ decay mode of
			the Higgs boson in $pp$ collisions at $\sqrt{s}$ = 13 TeV with the ATLAS
			detector}}, \href{https://doi.org/10.1016/j.physletb.2020.135754}{\emph{Phys.
			Lett. B} {\bfseries 809} (2020) 135754}
	[\href{https://arxiv.org/abs/2005.05382}{{\ttfamily 2005.05382}}].
	
	\bibitem{CMS:2018myz}
	{\scshape CMS} collaboration, \emph{{Search for the decay of a Higgs boson in
			the $\ell\ell\gamma$ channel in proton-proton collisions at $\sqrt{s} =$ 13
			TeV}}, \href{https://doi.org/10.1007/JHEP11(2018)152}{\emph{JHEP} {\bfseries
			11} (2018) 152} [\href{https://arxiv.org/abs/1806.05996}{{\ttfamily
			1806.05996}}].
	
	\bibitem{ATLAS:2014kct}
	{\scshape ATLAS} collaboration, \emph{{Measurements of Higgs boson production
			and couplings in the four-lepton channel in pp collisions at center-of-mass
			energies of 7 and 8 TeV with the ATLAS detector}},
	\href{https://doi.org/10.1103/PhysRevD.91.012006}{\emph{Phys. Rev. D}
		{\bfseries 91} (2015) 012006}
	[\href{https://arxiv.org/abs/1408.5191}{{\ttfamily 1408.5191}}].
	
	\bibitem{CMS:2014fzn}
	{\scshape CMS} collaboration, \emph{{Precise determination of the mass of the
			Higgs boson and tests of compatibility of its couplings with the standard
			model predictions using proton collisions at 7 and 8 $\,\text {TeV}$}},
	\href{https://doi.org/10.1140/epjc/s10052-015-3351-7}{\emph{Eur. Phys. J. C}
		{\bfseries 75} (2015) 212} [\href{https://arxiv.org/abs/1412.8662}{{\ttfamily
			1412.8662}}].
	
	\bibitem{ATLAS:2020rej}
	{\scshape ATLAS} collaboration, \emph{{Higgs boson production cross-section
			measurements and their EFT interpretation in the $4\ell $ decay channel at
			$\sqrt{s}=$13 TeV with the ATLAS detector}},
	\href{https://doi.org/10.1140/epjc/s10052-020-8227-9}{\emph{Eur. Phys. J. C}
		{\bfseries 80} (2020) 957}
	[\href{https://arxiv.org/abs/2004.03447}{{\ttfamily 2004.03447}}].
	
	\bibitem{ATLAS:2014otc}
	{\scshape ATLAS} collaboration, \emph{{Search for charged Higgs bosons decaying
			via $H^{\pm} \rightarrow \tau^{\pm}\nu$ in fully hadronic final states using
			$pp$ collision data at $\sqrt{s} = 8$ TeV with the ATLAS detector}},
	\href{https://doi.org/10.1007/JHEP03(2015)088}{\emph{JHEP} {\bfseries 03}
		(2015) 088} [\href{https://arxiv.org/abs/1412.6663}{{\ttfamily 1412.6663}}].
	
	\bibitem{CMS:2015lsf}
	{\scshape CMS} collaboration, \emph{{Search for a charged Higgs boson in pp
			collisions at $ \sqrt{s}=8 $ TeV}},
	\href{https://doi.org/10.1007/JHEP11(2015)018}{\emph{JHEP} {\bfseries 11}
		(2015) 018} [\href{https://arxiv.org/abs/1508.07774}{{\ttfamily
			1508.07774}}].
	
	\bibitem{ATLAS:2018gfm}
	{\scshape ATLAS} collaboration, \emph{{Search for charged Higgs bosons decaying
			via $H^{\pm} \to \tau^{\pm}\nu_{\tau}$ in the $\tau$+jets and $\tau$+lepton
			final states with 36 fb$^{-1}$ of $pp$ collision data recorded at $\sqrt{s} =
			13$ TeV with the ATLAS experiment}},
	\href{https://doi.org/10.1007/JHEP09(2018)139}{\emph{JHEP} {\bfseries 09}
		(2018) 139} [\href{https://arxiv.org/abs/1807.07915}{{\ttfamily
			1807.07915}}].
	
	\bibitem{CMS:2019bfg}
	{\scshape CMS} collaboration, \emph{{Search for charged Higgs bosons in the
			H$^{\pm}$ $\to$ $\tau^{\pm}\nu_\tau$ decay channel in proton-proton
			collisions at $\sqrt{s} =$ 13 TeV}},
	\href{https://doi.org/10.1007/JHEP07(2019)142}{\emph{JHEP} {\bfseries 07}
		(2019) 142} [\href{https://arxiv.org/abs/1903.04560}{{\ttfamily
			1903.04560}}].
	
	\bibitem{ATLAS:2015nkq}
	{\scshape ATLAS} collaboration, \emph{{Search for charged Higgs bosons in the
			$H^{\pm} \rightarrow tb$ decay channel in $pp$ collisions at $\sqrt{s}=8 $
			TeV using the ATLAS detector}},
	\href{https://doi.org/10.1007/JHEP03(2016)127}{\emph{JHEP} {\bfseries 03}
		(2016) 127} [\href{https://arxiv.org/abs/1512.03704}{{\ttfamily
			1512.03704}}].
	
	\bibitem{ATLAS:2021upq}
	{\scshape ATLAS} collaboration, \emph{{Search for charged Higgs bosons decaying
			into a top quark and a bottom quark at $ \sqrt{\mathrm{s}} $ = 13 TeV with
			the ATLAS detector}},
	\href{https://doi.org/10.1007/JHEP06(2021)145}{\emph{JHEP} {\bfseries 06}
		(2021) 145} [\href{https://arxiv.org/abs/2102.10076}{{\ttfamily
			2102.10076}}].
	
	\bibitem{CMS:2020imj}
	{\scshape CMS} collaboration, \emph{{Search for charged Higgs bosons decaying
			into a top and a bottom quark in the all-jet final state of pp collisions at
			$ \sqrt{s} $ = 13 TeV}},
	\href{https://doi.org/10.1007/JHEP07(2020)126}{\emph{JHEP} {\bfseries 07}
		(2020) 126} [\href{https://arxiv.org/abs/2001.07763}{{\ttfamily
			2001.07763}}].
	
	\bibitem{ATLAS:2015sxd}
	{\scshape ATLAS} collaboration, \emph{{Searches for Higgs boson pair production
			in the $hh\to bb\tau\tau, \gamma\gamma WW^*, \gamma\gamma bb, bbbb$ channels
			with the ATLAS detector}},
	\href{https://doi.org/10.1103/PhysRevD.92.092004}{\emph{Phys. Rev. D}
		{\bfseries 92} (2015) 092004}
	[\href{https://arxiv.org/abs/1509.04670}{{\ttfamily 1509.04670}}].
	
	\bibitem{CMS:2015jal}
	{\scshape CMS} collaboration, \emph{{Search for resonant pair production of
			Higgs bosons decaying to two bottom quark\textendash{}antiquark pairs in
			proton\textendash{}proton collisions at 8 TeV}},
	\href{https://doi.org/10.1016/j.physletb.2015.08.047}{\emph{Phys. Lett. B}
		{\bfseries 749} (2015) 560}
	[\href{https://arxiv.org/abs/1503.04114}{{\ttfamily 1503.04114}}].
	
	\bibitem{CMS:2016cma}
	{\scshape CMS} collaboration, \emph{{Search for two Higgs bosons in final
			states containing two photons and two bottom quarks in proton-proton
			collisions at 8 TeV}},
	\href{https://doi.org/10.1103/PhysRevD.94.052012}{\emph{Phys. Rev. D}
		{\bfseries 94} (2016) 052012}
	[\href{https://arxiv.org/abs/1603.06896}{{\ttfamily 1603.06896}}].
	
	\bibitem{CMS:2015uzk}
	{\scshape CMS} collaboration, \emph{{Searches for a heavy scalar boson H
			decaying to a pair of 125 GeV Higgs bosons hh or for a heavy pseudoscalar
			boson A decaying to Zh, in the final states with $h \to \tau \tau$}},
	\href{https://doi.org/10.1016/j.physletb.2016.01.056}{\emph{Phys. Lett. B}
		{\bfseries 755} (2016) 217}
	[\href{https://arxiv.org/abs/1510.01181}{{\ttfamily 1510.01181}}].
	
	\bibitem{CMS:2017yfv}
	{\scshape CMS} collaboration, \emph{{Search for Higgs boson pair production in
			the $bb\tau\tau$ final state in proton-proton collisions at
			$\sqrt{(}s)=8\text{ }\text{ }\mathrm{TeV}$}},
	\href{https://doi.org/10.1103/PhysRevD.96.072004}{\emph{Phys. Rev. D}
		{\bfseries 96} (2017) 072004}
	[\href{https://arxiv.org/abs/1707.00350}{{\ttfamily 1707.00350}}].
	
	\bibitem{ATLAS:2022hwc}
	{\scshape ATLAS} collaboration, \emph{{Search for resonant pair production of
			Higgs bosons in the $b\bar{b}b\bar{b}$ final state using $pp$ collisions at
			$\sqrt{s}$ = 13 TeV with the ATLAS detector}},
	\href{https://doi.org/10.1103/PhysRevD.105.092002}{\emph{Phys. Rev. D}
		{\bfseries 105} (2022) 092002}
	[\href{https://arxiv.org/abs/2202.07288}{{\ttfamily 2202.07288}}].
	
	\bibitem{CMS:2018qmt}
	{\scshape CMS} collaboration, \emph{{Search for resonant pair production of
			Higgs bosons decaying to bottom quark-antiquark pairs in proton-proton
			collisions at 13 TeV}},
	\href{https://doi.org/10.1007/JHEP08(2018)152}{\emph{JHEP} {\bfseries 08}
		(2018) 152} [\href{https://arxiv.org/abs/1806.03548}{{\ttfamily
			1806.03548}}].
	
	\bibitem{CMS:2021qvd}
	{\scshape CMS} collaboration, \emph{{Search for resonant Higgs boson pair
			production in four b quark final state using large-area jets in proton-proton
			collisions at $\sqrt{s}=13~\mathrm{TeV}$}}, {\emph{CMS-PAS-B2G-20-004} (2021)
	}.
	
	\bibitem{CMS:2022kdx}
	{\scshape CMS} collaboration, \emph{{Search for Higgs boson pairs decaying to
			WWWW, WW$\tau\tau$, and $\tau\tau\tau\tau$ in proton-proton collisions at
			$\sqrt{s}$ = 13 TeV}},  \href{https://arxiv.org/abs/2206.10268}{{\ttfamily
			2206.10268}}.
	
	\bibitem{ATLAS:2021ifb}
	{\scshape ATLAS} collaboration, \emph{{Search for Higgs boson pair production
			in the two bottom quarks plus two photons final state in $pp$ collisions at
			$\sqrt{s}=13$ TeV with the ATLAS detector}},
	\href{https://arxiv.org/abs/2112.11876}{{\ttfamily 2112.11876}}.
	
	\bibitem{CMS:2018tla}
	{\scshape CMS} collaboration, \emph{{Search for Higgs boson pair production in
			the $\gamma\gamma\mathrm{b\overline{b}}$ final state in pp collisions at
			$\sqrt{s}=$ 13 TeV}},
	\href{https://doi.org/10.1016/j.physletb.2018.10.056}{\emph{Phys. Lett. B}
		{\bfseries 788} (2019) 7} [\href{https://arxiv.org/abs/1806.00408}{{\ttfamily
			1806.00408}}].
	
	\bibitem{ATLAS:2021fet}
	{\scshape ATLAS} collaboration, \emph{{Search for resonant and non-resonant
			Higgs boson pair production in the $b\bar b\tau^+\tau^-$ decay channel using
			13 TeV $pp$ collision data from the ATLAS detector}},
	{\emph{ATLAS-CONF-2021-030} (2021) }.
	
	\bibitem{ATLAS:2020azv}
	{\scshape ATLAS} collaboration, \emph{{Reconstruction and identification of
			boosted di-$\tau$ systems in a search for Higgs boson pairs using 13 TeV
			proton-proton collision data in ATLAS}},
	\href{https://doi.org/10.1007/JHEP11(2020)163}{\emph{JHEP} {\bfseries 11}
		(2020) 163} [\href{https://arxiv.org/abs/2007.14811}{{\ttfamily
			2007.14811}}].
	
	\bibitem{CMS:2017hea}
	{\scshape CMS} collaboration, \emph{{Search for Higgs boson pair production in
			events with two bottom quarks and two tau leptons in
			proton\textendash{}proton collisions at $\sqrt s$ =13TeV}},
	\href{https://doi.org/10.1016/j.physletb.2018.01.001}{\emph{Phys. Lett. B}
		{\bfseries 778} (2018) 101}
	[\href{https://arxiv.org/abs/1707.02909}{{\ttfamily 1707.02909}}].
	
	\bibitem{CMS:2018kaz}
	{\scshape CMS} collaboration, \emph{{Search for heavy resonances decaying into
			two Higgs bosons or into a Higgs boson and a W or Z boson in proton-proton
			collisions at 13 TeV}},
	\href{https://doi.org/10.1007/JHEP01(2019)051}{\emph{JHEP} {\bfseries 01}
		(2019) 051} [\href{https://arxiv.org/abs/1808.01365}{{\ttfamily
			1808.01365}}].
	
	\bibitem{CMS:2017rpp}
	{\scshape CMS} collaboration, \emph{{Search for resonant and nonresonant Higgs
			boson pair production in the $ \mathrm{b}\overline{\mathrm{b}}\mathit{\ell
				\nu \ell \nu } $ final state in proton-proton collisions at $ \sqrt{s}=13 $
			TeV}}, \href{https://doi.org/10.1007/JHEP01(2018)054}{\emph{JHEP} {\bfseries
			01} (2018) 054} [\href{https://arxiv.org/abs/1708.04188}{{\ttfamily
			1708.04188}}].
	
	\bibitem{CMS:2019noi}
	{\scshape CMS} collaboration, \emph{{Search for resonances decaying to a pair
			of Higgs bosons in the $\mathrm{b\overline{b}q\overline{q}'}\ell\nu$ final
			state in proton-proton collisions at $\sqrt{s}=$ 13 TeV}},
	\href{https://doi.org/10.1007/JHEP10(2019)125}{\emph{JHEP} {\bfseries 10}
		(2019) 125} [\href{https://arxiv.org/abs/1904.04193}{{\ttfamily
			1904.04193}}].
	
	\bibitem{CMS:2020jeo}
	{\scshape CMS} collaboration, \emph{{Search for resonant pair production of
			Higgs bosons in the $bbZZ$ channel in proton-proton collisions at $\sqrt{s}=$
			13 TeV}}, \href{https://doi.org/10.1103/PhysRevD.102.032003}{\emph{Phys. Rev.
			D} {\bfseries 102} (2020) 032003}
	[\href{https://arxiv.org/abs/2006.06391}{{\ttfamily 2006.06391}}].
	
	\bibitem{ATLAS:2018fpd}
	{\scshape ATLAS} collaboration, \emph{{Search for Higgs boson pair production
			in the $b\bar{b}WW^{*}$ decay mode at $\sqrt{s}=13$ TeV with the ATLAS
			detector}}, \href{https://doi.org/10.1007/JHEP04(2019)092}{\emph{JHEP}
		{\bfseries 04} (2019) 092}
	[\href{https://arxiv.org/abs/1811.04671}{{\ttfamily 1811.04671}}].
	
	\bibitem{CMS:2021roc}
	{\scshape CMS} collaboration, \emph{{Search for heavy resonances decaying to a
			pair of Lorentz-boosted Higgs bosons in final states with leptons and a
			bottom quark pair at $ \sqrt{s} $= 13 TeV}},
	\href{https://doi.org/10.1007/JHEP05(2022)005}{\emph{JHEP} {\bfseries 05}
		(2022) 005} [\href{https://arxiv.org/abs/2112.03161}{{\ttfamily
			2112.03161}}].
	
	\bibitem{ATLAS:2018hqk}
	{\scshape ATLAS} collaboration, \emph{{Search for Higgs boson pair production
			in the $\gamma\gamma WW^{*}$ channel using $pp$ collision data recorded at
			$\sqrt{s} = 13$ TeV with the ATLAS detector}},
	\href{https://doi.org/10.1140/epjc/s10052-018-6457-x}{\emph{Eur. Phys. J. C}
		{\bfseries 78} (2018) 1007}
	[\href{https://arxiv.org/abs/1807.08567}{{\ttfamily 1807.08567}}].
	
	\bibitem{ATLAS:2015kpj}
	{\scshape ATLAS} collaboration, \emph{{Search for a CP-odd Higgs boson decaying
			to Zh in pp collisions at $\sqrt{s} = 8$ TeV with the ATLAS detector}},
	\href{https://doi.org/10.1016/j.physletb.2015.03.054}{\emph{Phys. Lett. B}
		{\bfseries 744} (2015) 163}
	[\href{https://arxiv.org/abs/1502.04478}{{\ttfamily 1502.04478}}].
	
	\bibitem{CMS:2015flt}
	{\scshape CMS} collaboration, \emph{{Search for a pseudoscalar boson decaying
			into a Z boson and the 125 GeV Higgs boson in $\ell^+\ell^-b\overline{b}$
			final states}},
	\href{https://doi.org/10.1016/j.physletb.2015.07.010}{\emph{Phys. Lett. B}
		{\bfseries 748} (2015) 221}
	[\href{https://arxiv.org/abs/1504.04710}{{\ttfamily 1504.04710}}].
	
	\bibitem{ATLAS:2017xel}
	{\scshape ATLAS} collaboration, \emph{{Search for heavy resonances decaying
			into a $W$ or $Z$ boson and a Higgs boson in final states with leptons and
			$b$-jets in 36 fb$^{-1}$ of $\sqrt s = 13$ TeV $pp$ collisions with the ATLAS
			detector}}, \href{https://doi.org/10.1007/JHEP03(2018)174}{\emph{JHEP}
		{\bfseries 03} (2018) 174}
	[\href{https://arxiv.org/abs/1712.06518}{{\ttfamily 1712.06518}}].
	
	\bibitem{CMS:2019qcx}
	{\scshape CMS} collaboration, \emph{{Search for a heavy pseudoscalar boson
			decaying to a Z and a Higgs boson at $\sqrt{s} =$ 13 TeV}},
	\href{https://doi.org/10.1140/epjc/s10052-019-7058-z}{\emph{Eur. Phys. J. C}
		{\bfseries 79} (2019) 564}
	[\href{https://arxiv.org/abs/1903.00941}{{\ttfamily 1903.00941}}].
	
	\bibitem{CMS:2018ljc}
	{\scshape CMS} collaboration, \emph{{Search for heavy resonances decaying into
			a vector boson and a Higgs boson in final states with charged leptons,
			neutrinos and b quarks at $ \sqrt{s}=13 $ TeV}},
	\href{https://doi.org/10.1007/JHEP11(2018)172}{\emph{JHEP} {\bfseries 11}
		(2018) 172} [\href{https://arxiv.org/abs/1807.02826}{{\ttfamily
			1807.02826}}].
	
	\bibitem{CMS:2019kca}
	{\scshape CMS} collaboration, \emph{{Search for a heavy pseudoscalar Higgs
			boson decaying into a 125 GeV Higgs boson and a Z boson in final states with
			two tau and two light leptons at $\sqrt{s}=$ 13 TeV}},
	\href{https://doi.org/10.1007/JHEP03(2020)065}{\emph{JHEP} {\bfseries 03}
		(2020) 065} [\href{https://arxiv.org/abs/1910.11634}{{\ttfamily
			1910.11634}}].
	
	\bibitem{CMS:2016xnc}
	{\scshape CMS} collaboration, \emph{{Search for neutral resonances decaying
			into a Z boson and a pair of b jets or $\tau$ leptons}},
	\href{https://doi.org/10.1016/j.physletb.2016.05.087}{\emph{Phys. Lett. B}
		{\bfseries 759} (2016) 369}
	[\href{https://arxiv.org/abs/1603.02991}{{\ttfamily 1603.02991}}].
	
	\bibitem{ATLAS:2020gxx}
	{\scshape ATLAS} collaboration, \emph{{Search for a heavy Higgs boson decaying
			into a Z boson and another heavy Higgs boson in the $\ell \ell bb$ and $\ell
			\ell WW$ final states in $pp$ collisions at $\sqrt{s}=13$ $\text {TeV}$ with
			the ATLAS detector}},
	\href{https://doi.org/10.1140/epjc/s10052-021-09117-5}{\emph{Eur. Phys. J. C}
		{\bfseries 81} (2021) 396}
	[\href{https://arxiv.org/abs/2011.05639}{{\ttfamily 2011.05639}}].
	
	\bibitem{ATLAS:2016btu}
	{\scshape ATLAS} collaboration, \emph{{Search for new phenomena in $t\bar{t}$
			final states with additional heavy-flavour jets in $pp$ collisions at
			$\sqrt{s}=13$ TeV with the ATLAS detector}}, {\emph{ATLAS-CONF-2016-104}
		(2016) }.
	
	\bibitem{CMS:2019rvj}
	{\scshape CMS} collaboration, \emph{{Search for production of four top quarks
			in final states with same-sign or multiple leptons in proton-proton
			collisions at $\sqrt{s}=$ 13 TeV}},
	\href{https://doi.org/10.1140/epjc/s10052-019-7593-7}{\emph{Eur. Phys. J. C}
		{\bfseries 80} (2020) 75} [\href{https://arxiv.org/abs/1908.06463}{{\ttfamily
			1908.06463}}].
	
	\bibitem{ATLAS:2022ohr}
	{\scshape ATLAS} collaboration, \emph{{Search for $t\bar tH/A \rightarrow t\bar
			tt\bar t$ production in the multilepton final state in proton-proton
			collisions at $\sqrt{s} = 13$ TeV with the ATLAS detector}},
	{\emph{ATLAS-CONF-2022-008} (2022) }.
	
	\bibitem{CMS:2015grx}
	{\scshape CMS} collaboration, \emph{{Search for neutral MSSM Higgs bosons
			decaying into a pair of bottom quarks}},
	\href{https://doi.org/10.1007/JHEP11(2015)071}{\emph{JHEP} {\bfseries 11}
		(2015) 071} [\href{https://arxiv.org/abs/1506.08329}{{\ttfamily
			1506.08329}}].
	
	\bibitem{CMS:2018kcg}
	{\scshape CMS} collaboration, \emph{{Search for narrow resonances in the
			b-tagged dijet mass spectrum in proton-proton collisions at $\sqrt{s} =$ 8
			TeV}}, \href{https://doi.org/10.1103/PhysRevLett.120.201801}{\emph{Phys. Rev.
			Lett.} {\bfseries 120} (2018) 201801}
	[\href{https://arxiv.org/abs/1802.06149}{{\ttfamily 1802.06149}}].
	
	\bibitem{CMS:2016ncz}
	{\scshape CMS} collaboration, \emph{{Search for a narrow heavy decaying to
			bottom quark pairs in the 13 TeV data sample}}, {\emph{CMS-PAS-HIG-16-025}
		(2016) }.
	
	\bibitem{CMS:2018hir}
	{\scshape CMS} collaboration, \emph{{Search for beyond the standard model Higgs
			bosons decaying into a $\mathrm{b\overline{b}}$ pair in pp collisions at
			$\sqrt{s} =$ 13 TeV}},
	\href{https://doi.org/10.1007/JHEP08(2018)113}{\emph{JHEP} {\bfseries 08}
		(2018) 113} [\href{https://arxiv.org/abs/1805.12191}{{\ttfamily
			1805.12191}}].
	
	\bibitem{ATLAS:2018tfk}
	{\scshape ATLAS} collaboration, \emph{{Search for resonances in the mass
			distribution of jet pairs with one or two jets identified as $b$-jets in
			proton-proton collisions at $\sqrt{s}=13$ TeV with the ATLAS detector}},
	\href{https://doi.org/10.1103/PhysRevD.98.032016}{\emph{Phys. Rev. D}
		{\bfseries 98} (2018) 032016}
	[\href{https://arxiv.org/abs/1805.09299}{{\ttfamily 1805.09299}}].
	
	\bibitem{ATLAS:2019tpq}
	{\scshape ATLAS} collaboration, \emph{{Search for heavy neutral Higgs bosons
			produced in association with $b$-quarks and decaying into $b$-quarks at
			$\sqrt{s}=13$ TeV with the ATLAS detector}},
	\href{https://doi.org/10.1103/PhysRevD.102.032004}{\emph{Phys. Rev. D}
		{\bfseries 102} (2020) 032004}
	[\href{https://arxiv.org/abs/1907.02749}{{\ttfamily 1907.02749}}].
	
	\bibitem{CMS:2018pwl}
	{\scshape CMS} collaboration, \emph{{Search for low-mass resonances decaying
			into bottom quark-antiquark pairs in proton-proton collisions at $\sqrt{s} =$
			13 TeV}}, \href{https://doi.org/10.1103/PhysRevD.99.012005}{\emph{Phys. Rev.
			D} {\bfseries 99} (2019) 012005}
	[\href{https://arxiv.org/abs/1810.11822}{{\ttfamily 1810.11822}}].
	
	\bibitem{CMS:2015ooa}
	{\scshape CMS} collaboration, \emph{{Search for neutral MSSM Higgs bosons
			decaying to $\mu^{+} \mu^{-}$ in pp collisions at $ \sqrt{s} =$ 7 and 8
			TeV}}, \href{https://doi.org/10.1016/j.physletb.2015.11.042}{\emph{Phys.
			Lett. B} {\bfseries 752} (2016) 221}
	[\href{https://arxiv.org/abs/1508.01437}{{\ttfamily 1508.01437}}].
	
	\bibitem{CMS:2019mij}
	{\scshape CMS} collaboration, \emph{{Search for MSSM Higgs bosons decaying to
			\ensuremath{\mu} + \ensuremath{\mu} \ensuremath{-} in proton-proton
			collisions at s=13TeV}},
	\href{https://doi.org/10.1016/j.physletb.2019.134992}{\emph{Phys. Lett. B}
		{\bfseries 798} (2019) 134992}
	[\href{https://arxiv.org/abs/1907.03152}{{\ttfamily 1907.03152}}].
	
	\bibitem{ATLAS:2019odt}
	{\scshape ATLAS} collaboration, \emph{{Search for scalar resonances decaying
			into $\mu^{+}\mu^{-}$ in events with and without $b$-tagged jets produced in
			proton-proton collisions at $\sqrt{s}=13$ TeV with the ATLAS detector}},
	\href{https://doi.org/10.1007/JHEP07(2019)117}{\emph{JHEP} {\bfseries 07}
		(2019) 117} [\href{https://arxiv.org/abs/1901.08144}{{\ttfamily
			1901.08144}}].
	
	\bibitem{ATLAS:2014vhc}
	{\scshape ATLAS} collaboration, \emph{{Search for neutral Higgs bosons of the
			minimal supersymmetric standard model in pp collisions at $\sqrt{s}$ = 8 TeV
			with the ATLAS detector}},
	\href{https://doi.org/10.1007/JHEP11(2014)056}{\emph{JHEP} {\bfseries 11}
		(2014) 056} [\href{https://arxiv.org/abs/1409.6064}{{\ttfamily 1409.6064}}].
	
	\bibitem{CMS:2015mca}
	{\scshape CMS} collaboration, \emph{{Search for additional neutral Higgs bosons
			decaying to a pair of tau leptons in $pp$ collisions at $\sqrt{s}$ = 7 and 8
			TeV}}, {\emph{CMS-PAS-HIG-14-029} (2015) }.
	
	\bibitem{ATLAS:2016ivh}
	{\scshape ATLAS} collaboration, \emph{{Search for Minimal Supersymmetric
			Standard Model Higgs bosons $H/A$ and for a $Z^{\prime}$ boson in the $\tau
			\tau$ final state produced in $pp$ collisions at $\sqrt{s}=13$ TeV with the
			ATLAS Detector}},
	\href{https://doi.org/10.1140/epjc/s10052-016-4400-6}{\emph{Eur. Phys. J. C}
		{\bfseries 76} (2016) 585}
	[\href{https://arxiv.org/abs/1608.00890}{{\ttfamily 1608.00890}}].
	
	\bibitem{ATLAS:2020zms}
	{\scshape ATLAS} collaboration, \emph{{Search for heavy Higgs bosons decaying
			into two tau leptons with the ATLAS detector using $pp$ collisions at
			$\sqrt{s}=13$ TeV}},
	\href{https://doi.org/10.1103/PhysRevLett.125.051801}{\emph{Phys. Rev. Lett.}
		{\bfseries 125} (2020) 051801}
	[\href{https://arxiv.org/abs/2002.12223}{{\ttfamily 2002.12223}}].
	
	\bibitem{CMS:2022rbd}
	{\scshape CMS} collaboration, \emph{{Searches for additional Higgs bosons and
			vector leptoquarks in $\tau\tau$ final states in proton-proton collisions at
			$\sqrt{s}=13~\mathrm{TeV}$}}, {\emph{CMS-PAS-HIG-21-001} (2022) }.
	
	\bibitem{ATLAS:2017eiz}
	{\scshape ATLAS} collaboration, \emph{{Search for additional heavy neutral
			Higgs and gauge bosons in the ditau final state produced in 36 fb$^{-1}$ of
			pp collisions at $ \sqrt{s}=13 $ TeV with the ATLAS detector}},
	\href{https://doi.org/10.1007/JHEP01(2018)055}{\emph{JHEP} {\bfseries 01}
		(2018) 055} [\href{https://arxiv.org/abs/1709.07242}{{\ttfamily
			1709.07242}}].
	
	\bibitem{CMS:2018rmh}
	{\scshape CMS} collaboration, \emph{{Search for additional neutral MSSM Higgs
			bosons in the $\tau\tau$ final state in proton-proton collisions at
			$\sqrt{s}=$ 13 TeV}},
	\href{https://doi.org/10.1007/JHEP09(2018)007}{\emph{JHEP} {\bfseries 09}
		(2018) 007} [\href{https://arxiv.org/abs/1803.06553}{{\ttfamily
			1803.06553}}].
	
	\bibitem{ATLAS:2014jdv}
	{\scshape ATLAS} collaboration, \emph{{Search for Scalar Diphoton Resonances in
			the Mass Range $65-600$ GeV with the ATLAS Detector in $pp$ Collision Data at
			$\sqrt{s}$ = 8 $TeV$}},
	\href{https://doi.org/10.1103/PhysRevLett.113.171801}{\emph{Phys. Rev. Lett.}
		{\bfseries 113} (2014) 171801}
	[\href{https://arxiv.org/abs/1407.6583}{{\ttfamily 1407.6583}}].
	
	\bibitem{CMS:2016kgr}
	{\scshape CMS} collaboration, \emph{{Search for high-mass diphoton resonances
			in proton\textendash{}proton collisions at 13 TeV and combination with 8 TeV
			search}}, \href{https://doi.org/10.1016/j.physletb.2017.01.027}{\emph{Phys.
			Lett. B} {\bfseries 767} (2017) 147}
	[\href{https://arxiv.org/abs/1609.02507}{{\ttfamily 1609.02507}}].
	
	\bibitem{CMS:2018dqv}
	{\scshape CMS} collaboration, \emph{{Search for physics beyond the standard
			model in high-mass diphoton events from proton-proton collisions at $\sqrt{s}
			=$ 13 TeV}}, \href{https://doi.org/10.1103/PhysRevD.98.092001}{\emph{Phys.
			Rev. D} {\bfseries 98} (2018) 092001}
	[\href{https://arxiv.org/abs/1809.00327}{{\ttfamily 1809.00327}}].
	
	\bibitem{ATLAS:2021uiz}
	{\scshape ATLAS} collaboration, \emph{{Search for resonances decaying into
			photon pairs in 139 fb$^{-1}$ of $pp$ collisions at $\sqrt {s}$=13 TeV with
			the ATLAS detector}},
	\href{https://doi.org/10.1016/j.physletb.2021.136651}{\emph{Phys. Lett. B}
		{\bfseries 822} (2021) 136651}
	[\href{https://arxiv.org/abs/2102.13405}{{\ttfamily 2102.13405}}].
	
	\bibitem{ATLAS:2014lfk}
	{\scshape ATLAS} collaboration, \emph{{Search for new resonances in $W\gamma$
			and $Z\gamma$ final states in $pp$ collisions at $\sqrt s=8$ TeV with the
			ATLAS detector}},
	\href{https://doi.org/10.1016/j.physletb.2014.10.002}{\emph{Phys. Lett. B}
		{\bfseries 738} (2014) 428}
	[\href{https://arxiv.org/abs/1407.8150}{{\ttfamily 1407.8150}}].
	
	\bibitem{CMS:2016all}
	{\scshape CMS} collaboration, \emph{{Search for scalar resonances in the
			200--1200 GeV mass range decaying into a Z and a photon in pp collisions at
			$\sqrt{s}=8~\mathrm{TeV}$}}, {\emph{CMS-PAS-HIG-16-014} (2016) }.
	
	\bibitem{CMS:2017dyb}
	{\scshape CMS} collaboration, \emph{{Search for Z$\gamma$ resonances using
			leptonic and hadronic final states in proton-proton collisions at $\sqrt{s}=$
			13 TeV}}, \href{https://doi.org/10.1007/JHEP09(2018)148}{\emph{JHEP}
		{\bfseries 09} (2018) 148}
	[\href{https://arxiv.org/abs/1712.03143}{{\ttfamily 1712.03143}}].
	
	\bibitem{ATLAS:2017zdf}
	{\scshape ATLAS} collaboration, \emph{{Searches for the $Z\gamma$ decay mode of
			the Higgs boson and for new high-mass resonances in $pp$ collisions at
			$\sqrt{s} = 13$ TeV with the ATLAS detector}},
	\href{https://doi.org/10.1007/JHEP10(2017)112}{\emph{JHEP} {\bfseries 10}
		(2017) 112} [\href{https://arxiv.org/abs/1708.00212}{{\ttfamily
			1708.00212}}].
	
	\bibitem{ATLAS:2018sxj}
	{\scshape ATLAS} collaboration, \emph{{Search for heavy resonances decaying to
			a photon and a hadronically decaying $Z/W/H$ boson in $pp$ collisions at
			$\sqrt{s}=13$ $\mathrm{TeV}$ with the ATLAS detector}},
	\href{https://doi.org/10.1103/PhysRevD.98.032015}{\emph{Phys. Rev. D}
		{\bfseries 98} (2018) 032015}
	[\href{https://arxiv.org/abs/1805.01908}{{\ttfamily 1805.01908}}].
	
	\bibitem{ATLAS:2015pre}
	{\scshape ATLAS} collaboration, \emph{{Search for an additional, heavy Higgs
			boson in the $H\rightarrow ZZ$ decay channel at $\sqrt{s} = 8\;\text{ TeV }$
			in $pp$ collision data with the ATLAS detector}},
	\href{https://doi.org/10.1140/epjc/s10052-015-3820-z}{\emph{Eur. Phys. J. C}
		{\bfseries 76} (2016) 45} [\href{https://arxiv.org/abs/1507.05930}{{\ttfamily
			1507.05930}}].
	
	\bibitem{ATLAS:2017tlw}
	{\scshape ATLAS} collaboration, \emph{{Search for heavy ZZ resonances in the
			$\ell ^+\ell ^-\ell ^+\ell ^-$ and $\ell ^+\ell ^-\nu \bar{\nu }$ final
			states using proton\textendash{}proton collisions at $\sqrt{s}= 13$ $\text
			{TeV}$ with the ATLAS detector}},
	\href{https://doi.org/10.1140/epjc/s10052-018-5686-3}{\emph{Eur. Phys. J. C}
		{\bfseries 78} (2018) 293}
	[\href{https://arxiv.org/abs/1712.06386}{{\ttfamily 1712.06386}}].
	
	\bibitem{ATLAS:2020tlo}
	{\scshape ATLAS} collaboration, \emph{{Search for heavy resonances decaying
			into a pair of Z bosons in the $\ell ^+\ell ^-\ell '^+\ell '^-$ and $\ell
			^+\ell ^-\nu {{\bar{\nu }}}$ final states using 139 $\mathrm {fb}^{-1}$ of
			proton\textendash{}proton collisions at $\sqrt{s} = 13\,$TeV with the ATLAS
			detector}}, \href{https://doi.org/10.1140/epjc/s10052-021-09013-y}{\emph{Eur.
			Phys. J. C} {\bfseries 81} (2021) 332}
	[\href{https://arxiv.org/abs/2009.14791}{{\ttfamily 2009.14791}}].
	
	\bibitem{ATLAS:2017otj}
	{\scshape ATLAS} collaboration, \emph{{Searches for heavy $ZZ$ and $ZW$
			resonances in the $\ell\ell qq$ and $\nu\nu qq$ final states in $pp$
			collisions at $\sqrt{s}=13$ TeV with the ATLAS detector}},
	\href{https://doi.org/10.1007/JHEP03(2018)009}{\emph{JHEP} {\bfseries 03}
		(2018) 009} [\href{https://arxiv.org/abs/1708.09638}{{\ttfamily
			1708.09638}}].
	
	\bibitem{CMS:2018amk}
	{\scshape CMS} collaboration, \emph{{Search for a new scalar resonance decaying
			to a pair of Z bosons in proton-proton collisions at $\sqrt{s}=13 $ TeV}},
	\href{https://doi.org/10.1007/JHEP06(2018)127}{\emph{JHEP} {\bfseries 06}
		(2018) 127} [\href{https://arxiv.org/abs/1804.01939}{{\ttfamily
			1804.01939}}].
	
	\bibitem{CMS:2018ygj}
	{\scshape CMS} collaboration, \emph{{Search for a heavy resonance decaying into
			a Z boson and a vector boson in the $ \nu
			\overline{\nu}\mathrm{q}\overline{\mathrm{q}} $ final state}},
	\href{https://doi.org/10.1007/JHEP07(2018)075}{\emph{JHEP} {\bfseries 07}
		(2018) 075} [\href{https://arxiv.org/abs/1803.03838}{{\ttfamily
			1803.03838}}].
	
	\bibitem{ATLAS:2015iie}
	{\scshape ATLAS} collaboration, \emph{{Search for a high-mass Higgs boson
			decaying to a $W$ boson pair in $pp$ collisions at $\sqrt{s} = 8$ TeV with
			the ATLAS detector}},
	\href{https://doi.org/10.1007/JHEP01(2016)032}{\emph{JHEP} {\bfseries 01}
		(2016) 032} [\href{https://arxiv.org/abs/1509.00389}{{\ttfamily
			1509.00389}}].
	
	\bibitem{CMS:2019bnu}
	{\scshape CMS} collaboration, \emph{{Search for a heavy Higgs boson decaying to
			a pair of W bosons in proton-proton collisions at $\sqrt{s} =$ 13 TeV}},
	\href{https://doi.org/10.1007/JHEP03(2020)034}{\emph{JHEP} {\bfseries 03}
		(2020) 034} [\href{https://arxiv.org/abs/1912.01594}{{\ttfamily
			1912.01594}}].
	
	\bibitem{ATLAS:2017uhp}
	{\scshape ATLAS} collaboration, \emph{{Search for heavy resonances decaying
			into $WW$ in the $e\nu\mu\nu$ final state in $pp$ collisions at $\sqrt{s}=13$
			TeV with the ATLAS detector}},
	\href{https://doi.org/10.1140/epjc/s10052-017-5491-4}{\emph{Eur. Phys. J. C}
		{\bfseries 78} (2018) 24} [\href{https://arxiv.org/abs/1710.01123}{{\ttfamily
			1710.01123}}].
	
	\bibitem{CMS:2016jpd}
	{\scshape CMS} collaboration, \emph{{Search for high mass Higgs to WW with
			fully leptonic decays using 2015 data}}, {\emph{CMS-PAS-HIG-16-023} (2016) }.
	
	\bibitem{ATLAS:2017jag}
	{\scshape ATLAS} collaboration, \emph{{Search for $WW/WZ$ resonance production
			in $\ell \nu qq$ final states in $pp$ collisions at $\sqrt{s} =$ 13 TeV with
			the ATLAS detector}},
	\href{https://doi.org/10.1007/JHEP03(2018)042}{\emph{JHEP} {\bfseries 03}
		(2018) 042} [\href{https://arxiv.org/abs/1710.07235}{{\ttfamily
			1710.07235}}].
	
	\bibitem{CMS:2018dff}
	{\scshape CMS} collaboration, \emph{{Search for a heavy resonance decaying to a
			pair of vector bosons in the lepton plus merged jet final state at $
			\sqrt{s}=13 $ TeV}},
	\href{https://doi.org/10.1007/JHEP05(2018)088}{\emph{JHEP} {\bfseries 05}
		(2018) 088} [\href{https://arxiv.org/abs/1802.09407}{{\ttfamily
			1802.09407}}].
	
	\bibitem{CMS:2015hra}
	{\scshape CMS} collaboration, \emph{{Search for a Higgs boson in the mass range
			from 145 to 1000 GeV decaying to a pair of W or Z bosons}},
	\href{https://doi.org/10.1007/JHEP10(2015)144}{\emph{JHEP} {\bfseries 10}
		(2015) 144} [\href{https://arxiv.org/abs/1504.00936}{{\ttfamily
			1504.00936}}].
	
	\bibitem{ATLAS:2017zuf}
	{\scshape ATLAS} collaboration, \emph{{Search for diboson resonances with
			boson-tagged jets in $pp$ collisions at $\sqrt{s}=13$ TeV with the ATLAS
			detector}}, \href{https://doi.org/10.1016/j.physletb.2017.12.011}{\emph{Phys.
			Lett. B} {\bfseries 777} (2018) 91}
	[\href{https://arxiv.org/abs/1708.04445}{{\ttfamily 1708.04445}}].
	
	\bibitem{ATLAS:2018sbw}
	{\scshape ATLAS} collaboration, \emph{{Combination of searches for heavy
			resonances decaying into bosonic and leptonic final states using 36 fb$^{-1}$
			of proton-proton collision data at $\sqrt{s} = 13$ TeV with the ATLAS
			detector}}, \href{https://doi.org/10.1103/PhysRevD.98.052008}{\emph{Phys.
			Rev. D} {\bfseries 98} (2018) 052008}
	[\href{https://arxiv.org/abs/1808.02380}{{\ttfamily 1808.02380}}].
	
	\bibitem{ATLAS:2020fry}
	{\scshape ATLAS} collaboration, \emph{{Search for heavy diboson resonances in
			semileptonic final states in pp collisions at $\sqrt{s}=13$ TeV with the
			ATLAS detector}},
	\href{https://doi.org/10.1140/epjc/s10052-020-08554-y}{\emph{Eur. Phys. J. C}
		{\bfseries 80} (2020) 1165}
	[\href{https://arxiv.org/abs/2004.14636}{{\ttfamily 2004.14636}}].
	
	\bibitem{CMS:2021klu}
	{\scshape CMS} collaboration, \emph{{Search for heavy resonances decaying to
			WW, WZ, or WH boson pairs in the lepton plus merged jet final state in
			proton-proton collisions at $\sqrt{s}$ = 13 TeV}},
	\href{https://doi.org/10.1103/PhysRevD.105.032008}{\emph{Phys. Rev. D}
		{\bfseries 105} (2022) 032008}
	[\href{https://arxiv.org/abs/2109.06055}{{\ttfamily 2109.06055}}].
	
	\bibitem{Hardy:2020qwl}
	J.C.~Hardy and I.S.~Towner, \emph{{Superallowed $0^+ \to 0^+$ nuclear $\beta$
			decays: 2020 critical survey, with implications for V$_{ud}$ and CKM
			unitarity}}, \href{https://doi.org/10.1103/PhysRevC.102.045501}{\emph{Phys.
			Rev. C} {\bfseries 102} (2020) 045501}.
	
	\bibitem{Seng:2021nar}
	C.-Y.~Seng, D.~Galviz, W.J.~Marciano and U.-G.~Mei\ss{}ner, \emph{{Update on
			$|Vus|$ and $|Vus/Vud|$ from semileptonic kaon and pion decays}},
	\href{https://doi.org/10.1103/PhysRevD.105.013005}{\emph{Phys. Rev. D}
		{\bfseries 105} (2022) 013005}
	[\href{https://arxiv.org/abs/2107.14708}{{\ttfamily 2107.14708}}].
	
	\bibitem{FlavourLatticeAveragingGroupFLAG:2021npn}
	{\scshape Flavour Lattice Averaging Group (FLAG)} collaboration, \emph{{FLAG
			Review 2021}},
	\href{https://doi.org/10.1140/epjc/s10052-022-10536-1}{\emph{Eur. Phys. J. C}
		{\bfseries 82} (2022) 869}
	[\href{https://arxiv.org/abs/2111.09849}{{\ttfamily 2111.09849}}].


 
\end{thebibliography}

\end{document}

