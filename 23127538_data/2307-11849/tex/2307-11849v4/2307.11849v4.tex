\documentclass{amsart}
%Edited on Feb12, 25 by Shabnam. To be submitted to arXiv

\usepackage{hyperref}
\usepackage{array}
\usepackage{amssymb, latexsym, url}
\usepackage{amsthm, epsfig}
\usepackage{xcolor}
\usepackage{mathtools}

\DeclareMathOperator{\Aut}{\mathrm Aut}
\DeclareMathOperator{\CM}{\mathrm CM}
\DeclareMathOperator{\End}{\mathrm End}
\DeclareMathOperator{\GL}{\mathrm GL}
\DeclareMathOperator{\Hom}{\mathrm Hom}
\DeclareMathOperator{\closure}{\mathrm closure}
\DeclareMathOperator{\Norm}{\mathrm Norm}
\DeclareMathOperator{\Null}{\mathrm Null}
\DeclareMathOperator{\Imm}{\mathrm Im}
\DeclareMathOperator{\rank}{\mathrm rank}
\DeclareMathOperator{\Reg}{\mathrm Reg}
\DeclareMathOperator{\sgn}{\mathrm sgn}
\DeclareMathOperator{\supp}{\mathrm supp}
\DeclareMathOperator{\Tor}{\mathrm Tor}
\DeclareMathOperator{\trace}{\mathrm trace}
\DeclareMathOperator{\Vol}{\mathrm Vol}
\DeclareMathOperator{\ord}{\mathrm ord}

\begin{document}
 \bibliographystyle{plain}

 \newtheorem{theorem}{Theorem}[section]
 \newtheorem{lemma}{Lemma}[section]
 \newtheorem{corollary}{Corollary}[section]
 \newtheorem{conjecture}{Conjecture}[section]
 \newtheorem{question}{Question}[section]
 
 \newcommand\house[1]{%
     \begingroup\setlength\arraycolsep{0pt}
     \begin{array}[t]{@{}c@{} | c |@{}c@{}}
     \firsthline
     &\;#1\;{}&
     \end{array}
     \endgroup
 }
 \newcommand{\mc}{\mathcal}
 \newcommand{\mbb}{\mathbb}
 \newcommand{\fA}{\mathfrak A}
 \newcommand{\B}{\mc B}
 \newcommand{\fB}{\mathfrak B}
 \newcommand{\cC}{\mc C}
 \newcommand{\D}{\mc D}
 \newcommand{\E}{\mc E}
 \newcommand{\F}{\mc F}
 \newcommand{\G}{\mc G}
 \newcommand{\hH}{\mc H}
 \newcommand{\fG}{\mathfrak G}
 \newcommand{\fI}{\mathfrak I}
 \newcommand{\I}{\mc I}
 \newcommand{\J}{\mc J}
 \newcommand{\K}{\mc K}
 \newcommand{\lL}{\mc L}
 \newcommand{\M}{\mc M}
 \newcommand{\fM}{\mathfrak M}
 \newcommand{\pp}{\mc P}
 \newcommand{\fP}{\mathfrak P}
 \newcommand{\fR}{\mathfrak R}
 \newcommand{\rR}{\mc R}
 \newcommand{\fS}{\mathfrak S}
 \newcommand{\sS}{\mc S}
 \newcommand{\U}{\mc U}
 \newcommand{\uU}{\mathfrak U}
 \newcommand{\X}{\mc X}
 \newcommand{\Y}{\mc Y}
 \newcommand{\A}{\mathbb{A}}
 \newcommand{\C}{\mathbb{C}}
 \newcommand{\pP}{\mathbb{P}}
 \newcommand{\Q}{\mathbb Q}
 \newcommand{\R}{\mathbb R}
 \newcommand{\T}{\mathbb T}
 \newcommand{\Z}{\mathbb{Z}}
 \newcommand{\ahat}{\widehat\alpha}
 \newcommand{\bhat}{\widehat\beta}
 \newcommand{\fhat}{\widehat f}
 \newcommand{\ghat}{\widehat g}
 \newcommand{\hhat}{\widehat h}
 \newcommand{\wsigma}{\widetilde{\sigma}}
 \newcommand{\wtau}{\widetilde{\tau}}
 \newcommand{\p}{\boldsymbol{\varphi}}
 \newcommand{\h}{\tfrac{1}{2}}
 \newcommand{\hh}{\frac{1}{2}}
 \newcommand{\ba}{\boldsymbol a}
 \newcommand{\bb}{\boldsymbol b}
 \newcommand{\be}{\boldsymbol e}
 \newcommand{\bm}{\boldsymbol m}
 \newcommand{\bn}{\boldsymbol n}
 \newcommand{\bu}{\boldsymbol u}
 \newcommand{\bv}{\boldsymbol v}
 \newcommand{\bw}{\boldsymbol w}
 \newcommand{\bx}{\boldsymbol x}
 \newcommand{\bwy}{\boldsymbol y}
 \newcommand{\bL}{\boldsymbol L}
 \newcommand{\bta}{\boldsymbol \beta}
 \newcommand{\bet}{\boldsymbol \eta}
 \newcommand{\bxi}{\boldsymbol \xi}
 \newcommand{\bo}{\boldsymbol 0}
 \newcommand{\bid}{\boldsymbol 1}
 \newcommand{\ep}{\varepsilon}
 \newcommand{\vphi}{\varphi}
  \newcommand{\eeta}{\mu}
 \newcommand{\dlambda}{\text{\rm d}\lambda}
 \newcommand{\dbeta}{\text{\rm d}\beta}
 \newcommand{\dmu}{\text{\rm d}\mu}
 \newcommand{\dr}{\text{\rm d}r}
 \newcommand{\du}{\text{\rm d}u}
 \newcommand{\dv}{\text{\rm d}v}
 \newcommand{\dt}{\text{\rm d}t}
 \newcommand{\dw}{\text{\rm d}w}
 \newcommand{\dx}{\text{\rm d}x}
 \newcommand{\dy}{\text{\rm d}y}
 \newcommand{\dxi}{\text{\rm d}\xi}
 \newcommand{\oQ}{\overline{\Q}}
 \newcommand{\oq}{\oQ^{\times}}
 \newcommand{\oQt}{\oQ^{\times}/\Tor\bigl(\oQ^{\times}\bigr)}
 \newcommand{\ot}{\Tor\bigl(\oQ^{\times}\bigr)}

\def\housealp{{%
    \setbox2=\hbox{$\alpha$}
    \vrule height \dimexpr\ht2+1.75pt width .4pt depth \dp2\relax
    \vrule height 6.05pt width \dimexpr\wd2+2pt depth -5.65pt
    \llap{$\alpha$\kern1pt}
    \vrule height \dimexpr\ht2+1.75pt width .4pt depth \dp2\relax
}}
\def\houseeta{{%
    \setbox2=\hbox{$\eeta$}
    \vrule height \dimexpr\ht2+1.75pt width .4pt depth \dp2\relax
    \vrule height 6.05pt width \dimexpr\wd2+2pt depth -5.65pt
    \llap{$\eeta$\kern1pt}
    \vrule height \dimexpr\ht2+1.75pt width .4pt depth \dp2\relax
}}

\title[heights of generators]
{A note on generators\\of number fields, II}
\author{Shabnam Akhtari, Jeffrey~D.~Vaaler and Martin Widmer}
\subjclass[2010]{11H06, 11R29, 11R56}
\keywords{height, integral generators, $\CM$-fields, roots of unity}
\address{Department of Mathematics, Pennsylvania State University, University Park, PA 16802 USA}
\email{akhtari@psu.edu}

\address{Department of Mathematics, University of Texas, Austin, TX 78712 USA}
\email{vaaler@math.utexas.edu}

\address{Graz University of Technology, Institute of Analysis and Number Theory, Steyrergasse 30/II, 8010 Graz, Austria}
\email{martin.widmer@tugraz.at}



\allowdisplaybreaks
\numberwithin{equation}{section}

%%%%%%%%%%%%%%%%%%%%%%%%%%%%%%%%%%%%%%%%%%%%%%%%%%%%%%%%%%%%%%
\begin{abstract}  Let $K$ be an algebraic number field and $H$ the absolute Weil height.  Write $c_K$ for a certain positive constant that is 
an invariant of $K$.  We consider the question: does $K$ contain an algebraic integer $\alpha$ such that both $K = \Q(\alpha)$ and 
$H(\alpha) \le c_K$?  If $K$ has a real embedding then a positive answer was established in previous work.  Here we obtain a positive 
answer if $\Tor\bigl(K^{\times}\bigr) \not= \{\pm 1\}$, and so $K$ has only complex embeddings.  We also show that if the answer is negative, 
then $K$ is totally complex, $\Tor\bigl(K^{\times}\bigr) = \{\pm 1\}$, and $K$ is a Galois extension of its maximal totally real subfield.
Further, we show that if $\mu \in O_K$ is not totally real, then there exists $\alpha$ in $O_K$ with $K = \Q(\alpha)$ and 
$H(\alpha) \le H(\mu)\thinspace c_K$. 
\end{abstract}

\maketitle
%%%%%%%%%%%%%%%%%%%%%%%%%%%%%%%%%%%%%%%%%%%%%%%%%%%%%%%%%%%%%%%%%%%%%%%%

\section{Introduction}%%section 1

Let $K$ be an algebraic number field of degree $d = [K : \Q]$, and let $\Delta_K$ be the discriminant of $K$.  We define the positive constant
\begin{equation}\label{early3}
c_K = \biggl(\frac{2}{\pi}\biggr)^{s/d} \bigl|\Delta_K\bigr|^{1/2d},
\end{equation}  
where $s$ is the number of complex places of $K$.  We write $K^{\times}$ for the multiplicative group of nonzero elements of $K$, and
\begin{equation}\label{early7}
H : K^{\times} \rightarrow [1, \infty)
\end{equation}  
for the absolute, multiplicative Weil height defined in (\ref{extra261}).

In \cite[Question 2]{ruppert1998} W.~Ruppert asked the following question:

\begin{question} {\sc [Ruppert, 1998]}\label{con1}  Does there exist a positive constant $A = A(d)$ such that if $K$ is an algebraic 
number field of degree $d$ over $\Q$, then there exists an element $\alpha$ in $K$ such that 
\begin{equation*}\label{early12}
K = \Q(\alpha),\quad\text{and}\quad H(\alpha) \le A \bigl|\Delta_K\bigr|^{1/2d}\thinspace ?
\end{equation*} 
\end{question}

Ruppert stated his question using the naive height.  However, it follows from elementary inequalities for heights that the variant we 
have stated here is equivalent to the question originally asked by Ruppert.  In \cite[Proposition 2]{ruppert1998} Ruppert obtained a positive 
answer to his question when $[K : \Q] = 2$.  He also proved that if $K$ is a real quadratic extension of $\Q$, then the generator 
$\alpha$ can be selected from the ring $O_K$ of algebraic integers in $K$.  In \cite{vaaler2013} the second and third named authors provided the 
following partial answer to Ruppert's question:

\begin{theorem}\label{thmearly1}  Assume that $K$ has an embedding into $\R$.  Then there exists an algebraic integer $\alpha$ in $O_K$
such that
\begin{equation}\label{early16}
K = \Q(\alpha),\quad\text{and}\quad H(\alpha) \le c_K.
\end{equation}
\end{theorem}

In Theorem \ref{thmearly1} the generator $\alpha$ is an algebraic integer, a requirement that was {\it not} stated in Ruppert's question, 
while the height of $\alpha$ is bounded in a manner that was anticipated in Ruppert's question. Hence Theorem \ref{thmearly1} generalizes 
Ruppert's earlier result to number fields $K$ that have at least one real embedding. 

In this note we establish a positive answer to Ruppert's question for a new class of number fields. 

\begin{theorem}\label{thmearly2}  Assume that $K$ is a number field, and $\mu$ is an algebraic integer in $O_K$ that is not totally
real.  Then there exists an algebraic integer $\alpha$ in $O_K$ such that 
\begin{equation}\label{early22}
K = \Q(\alpha),\quad\text{and}\quad H(\alpha) \le H(\mu)\thinspace c_K.
\end{equation}
\end{theorem}

Let $\Tor\bigl(K^{\times}\bigr)$ denote the torsion subgroup of the multiplicative group $K^{\times}$.  This group is known to be a finite, cyclic group 
of even order $2 q_K$, where $q_K$ is a positive divisor of $\Delta_K$.  The following simple corollary illustrates how Theorem \ref{thmearly2}
can be applied.

\begin{corollary}\label{corearly1}  Assume that $K$ is a number field such that 
\begin{equation}\label{early24}
\Tor\bigl(K^{\times}\bigr) \neq  \{\pm 1\}.
\end{equation}
Then there exists $\alpha$ in $O_K$ such that 
\begin{equation}\label{early27}
K = \Q(\alpha), \quad \text{and}\quad H(\alpha) \le c_K.
\end{equation}
\end{corollary}

To derive Corollary \ref{corearly1} from Theorem \ref{thmearly2} we select $\mu$ to be an element of the group $\Tor\bigl(K^{\times}\bigr)$
having order greater than or equal to $3$.   It follows that $\mu$ is a complex (and not real) root of unity and therefore $H(\mu) = 1$.  In this 
example $\mu$ has no real conjugates over $\Q$ and therefore the field $K$ containing $\mu$ is totally complex.

Another implication of  Theorem \ref{thmearly2} is an affirmative answer to Question \ref{con1} (with $A = 2$, say) whenever the field 
$K$ contains an algebraic integer of the form $n^{1/m}$ with integers $m \geq n \geq 3$.

The constructive method used in the proof of Theorem \ref{thmearly2} enables us to obtain a result in which we identify a class of 
number fields which might not have a small integral generator. 

\begin{theorem}\label{thmearly3}   Let $1 \le \hH$ and assume that $K$ is a number field such that
\begin{equation}\label{early35}
\hH c_K < \min\big\{H(\alpha) : \text{$\alpha \in O_K$ and $K = \Q(\alpha)$}\big\}.
\end{equation}
Let $F \subseteq K$ be the maximal, totally real subfield of $K$.  Then $K$ is totally complex, $K/F$ is Galois, and every $\mu \in O_K$ 
with $H(\mu) \leq \hH$ is contained in $F$.
\end{theorem}

For a number field  $K$ that satisfies the hypotheses of Theorem \ref{thmearly3}, a representation for the Galois group $\Aut(K/F)$ is 
provided in (\ref{extra755}).  We also note that the restriction of complex conjugation to $K$ belongs to $\Aut(K/F)$ and is an automorphism
of order $2$.  This implies that $K$ is a quadratic extension of the unique real number field $k = K \cap \R$ and $F \subseteq k \subseteq K$. 
The number field $k$ is the maximal real subfield of the totally complex number field $K$.  A more general form of these remarks follow from 
(\ref{extra719}) and (\ref{extra726}) in the proof of Theorem \ref{thmearly3}.

Taking $\hH=1$ in Theorem \ref{thmearly3}, we conclude that if $K$ is a number field such that
\begin{equation}\label{early48}
c_K < \min\big\{H(\alpha) : \text{$\alpha \in O_K$ and $K = \Q(\alpha)$}\big\},
\end{equation}
then $K$ is totally complex, $K$ is Galois over its maximal totally real subfield $F$,  and $\Tor\bigl(K^{\times}\bigr) =  \{\pm 1\}$.
Next we record some examples of fields $K$ that satisfy the inequality (\ref{early35}) in Theorem \ref{thmearly3} with $\hH=1$.

Let $m$ be a squarefree, negative integer and let $L = \Q\bigl(\sqrt{m}\bigr)$ be the imaginary quadratic field generated by $\sqrt{m}$.  
An integral basis for the ring $O_L$ is well known (see \cite[Theorem 7.1.1]{alaca2004}).  We also recall
(see \cite[Theorem 7.1.2]{alaca2004}) that
\begin{equation}\label{early62}
\Delta_L = \begin{cases}   m&    \text{if $m \equiv 1 \pmod 4$},\\
				       4m&    \text{if $m \not\equiv 1 \pmod 4$}.\end{cases}	
\end{equation}
It is now a simple matter to minimize the height over elements of $O_L$ that generate the field $L$.  If $m \equiv 1 \pmod 4$ we find that
\begin{equation}\label{early69}
\min\big\{H(\alpha) : \text{$\alpha \in O_L$ and $L = \Q(\alpha)$}\big\} = \h \bigl(1 + |\Delta_L|\bigr)^{\hh},
\end{equation}
and if $m \not\equiv 1 \pmod 4$ then 
\begin{equation}\label{early76}
\min\big\{H(\alpha) : \text{$\alpha \in O_L$ and $L = \Q(\alpha)$}\big\} = \h |\Delta_L|^{\hh}.
\end{equation}

Among the imaginary quadratic fields $L$, only the fields 
\begin{equation*}\label{early83}
L = \Q\bigl(\sqrt{-3})\quad \text{and}\quad L = \Q\bigl(\sqrt{-1}\bigr)
\end{equation*}
satisfy the condition $\Tor\bigl(L^{\times}\bigr) \not= \{\pm 1\}$.  These fields are both generated by a root of unity and therefore the minimal height
of an algebraic integer that generates the field is $1$.  This conclusion also follows from (\ref{early62}), (\ref{early69}), and (\ref{early76}).  

If $L = \Q\bigl(\sqrt{m}\bigr)$ satisfies $\Tor\bigl(L^{\times}\bigr) = \{\pm 1\}$ then it follows from our previous remarks that the minimum height of
an integral generator is greater than $1$.  As the value of the minimal height is given by (\ref{early69}) or by (\ref{early76}), it is easy to verify that
\begin{equation}\label{early90}
c_L = \biggl(\frac{2}{\pi}\biggr)^{1/2} \bigl|\Delta_L\bigr|^{1/4}< \min\big\{H(\alpha) : \text{$\alpha \in O_L$ and $L = \Q(\alpha)$}\big\}
\end{equation}  
in both cases.  This shows that for imaginary quadratic fields $L$ the inequality (\ref{early35}) with $\hH=1$ holds if and only if $L$ satisfies
$\Tor\bigl(L^{\times}\bigr) = \{\pm 1\}$. 

While for quadratic number fields $K$, the inequality \eqref{early48} holds if and only if $K$ is totally complex  Galois extension of $\Q$  and $\Tor\bigl(K^{\times}\bigr) =  \{\pm 1\}$,  for number fields of degree greater than $2$,  this equivalence no longer holds.
The following is an example of a totally complex quartic number field $K$ that is a Galois  extension of a totally real number field,  with 
$\Tor\bigl(K^{\times}\bigr) = \{\pm 1\}$ for which (\ref{early48}) does not hold.  Let $\alpha$ be an algebraic integer that satisfies
\begin{equation*}\label{early97}
\alpha^2= \sqrt{3} - 2 < 0,
\end{equation*}
and $K=\Q(\alpha)$.  The quartic field $K$ has discriminant (see \cite[Theorem 1]{huardspearmanwilliams1995})
\begin{equation*}\label{early104}
\Delta_K=2^8 3^2.
\end{equation*}
On the other hand,
\begin{equation*}\label{early111}
H(\alpha)=H(\sqrt{3}-2)^{1/2}=(\sqrt{3}+2)^{1/4} < (2/\pi)^{1/2}(2^83^2)^{1/8} = c_K.
\end{equation*}
Assume that $\Tor\bigl(K^{\times}\bigr) \neq \{\pm 1\}$. Since $2$ and $3$ are the only primes that ramify in $K$ we conclude that 
\begin{equation*}\label{early118}
\h \bigl(\sqrt{-3}-1\bigr) \in K\quad \text{or}\quad \sqrt{-1} \in K.
\end{equation*}
But since $\sqrt{3}\in K$,  we get  $\sqrt{-1} \in K$ in either case, and thus also
$\sqrt{2-\sqrt{3}}\in K$. However, the algebraic number $\sqrt{2-\sqrt{3}}$ is totally real of degree $4$, and therefore it cannot belong to the totally 
complex quartic field $K$.  This contradiction confirms that $\Tor\bigl(K^{\times}\bigr) = \{\pm 1\}$.
 
We recall that $K$ is a $\CM$-field if $K$ has only complex embeddings and there exists a totally real subfield $k \subseteq K$ such that $K/k$ is 
a quadratic extension.  There are well known characterizations of $\CM$-fields due to Blanksby and Loxton \cite{blanksby1978} and
Shimura \cite[Proposition 5.11 ]{shimura1971}.  It follows that if $K$ is a $\CM$-field that satisfies the hypotheses of Theorem \ref{thmearly3} 
then the maximal totally real subfield $F$ that occurs in the statement of Theorem \ref{thmearly3} is equal to $k$.  

Next we give a family of examples of $\CM$-fields of degree $d = 2N \geq 4$ that satisfy \eqref{early48}.  First we pick  a totally 
real field $F$ of degree $N$. Let $\{\omega_1, \ldots, \omega_N\}$ be a $\Z$-basis of $O_F$.  Let $n$ be a positive, squarefree integer, 
set $M=\Q(\sqrt{-n})$, and let $1, \xi$ be a  $\Z$-basis of $O_M$.  Assume that $\Delta_F$ and $\Delta_M$ are coprime, and set 
$K=FM=F(\sqrt{-n})$.  This implies that 
\begin{equation*}\label{early128}
\{\omega_1, \dots, \omega_N, \omega_1\xi, \dots, \omega_N\xi\}
\end{equation*}
is a  $\Z$-basis of $O_K$, and
\begin{equation}\label{discW}
|\Delta_K|= \Delta_F^2 |\Delta_M|^N\leq \Delta_F^2(4n)^N.
\end{equation}
In particular, every $\alpha\in O_K$ with $K=\Q(\alpha)$ can be written in the form 
\begin{equation*}\label{early158}
\alpha=\tfrac{1}{2} \bigl(\omega+\omega'\sqrt{-n}\bigr),
\end{equation*}
where $\omega$ and $\omega' \not= 0$ are in $O_F$. 
Using that $F$ is totally real, and writing $\tau$ for an embedding in $\Hom(K, \C)$, we find
\begin{equation*}\label{early165}
\begin{split}
H(\alpha) &\geq \prod_{\tau \in \Hom(K, \C)} |\tau(\alpha)|^{1/d} = \h \prod_{\tau \in \Hom(K, \C)} \bigl|\tau(\omega)+\tau(\omega')\tau(\sqrt{-n})\bigr|^{1/d}\\
                &\geq \h \sqrt{n}\thinspace \prod_{\tau \in \Hom(K, \C)}\bigl|\tau(\omega')\bigr|^{1/d}\geq \h \sqrt{n}.
\end{split}
\end{equation*}
Hence $K$ satisfies (\ref{early35}) with $\hH=1$ for all sufficiently large $n$.

%%%%%%%%%%%%%%%%%%%%section 2%%%%%%%%%%%%%%%%%%%%%%%%%%%%%%%%%%%%%%%%%%%%%%%%
\section{Lemmas}

We assume that $K$ is an algebraic number field of degree $d = [K : \Q]$.  And we use
\begin{equation*}\label{extra249}
|\ | : \C \rightarrow [0, \infty)\quad\text{and}\quad \rho : \C \rightarrow \C,
\end{equation*}
for the Hermitian absolute value on $\C$ and complex conjugation, respectively.
At each place $v$ of $K$ we write $K_v$ for the completion of $K$ with respect to an absolute value from $v$.  Then $\|\ \|_v$ is the unique absolute 
value from $v$ that extends either the usual Euclidean absolute value on $\Q_{\infty}$ or the unque $p$-adic absolute value on $\Q_p$.  We write
$d = [K : \Q]$ for the degree of $K$ and $d_v = [K_v : \Q_v]$ for the local degree at the place $v$.  Then we define a second absolute value from $v$ by
\begin{equation}\label{extra258}
|\ |_v = \|\ \|_v^{d_v/d}.
\end{equation}
It follows that the absolute, multiplicative Weil height of $\alpha$ in $K^{\times}$ is the map (\ref{early7}) defined by 
(see \cite[section 1.5.7]{bombieri2006})
\begin{equation}\label{extra261}
H(\alpha) = \prod_v \max\big\{1, |\alpha|_v\big\}.
\end{equation}

We also write $W_{\infty}(K/\Q)$ for the set of archimedean places of $K$.  If $v$ is a real place of $K$ then $\sigma_v : K \rightarrow \C$ 
is the embedding associated to $v$ by
\begin{equation}\label{extra265}
\|\beta\|_v = \bigl|\sigma_v(\beta)\bigr|\quad\text{for each $\beta$ in $K$}.
\end{equation}
If $w$ is a complex place of $K$ then
\begin{equation*}\label{extra267}
\sigma_w : K \rightarrow \C\quad\text{and}\quad \rho \sigma_w : K \rightarrow \C
\end{equation*}
are the two distinct embeddings associated to $w$ by the identity
\begin{equation}\label{extra274}
\|\beta\|_w = \bigl|\sigma_w(\beta)\bigr| = \bigl|\rho \sigma_w(\beta)\bigr| \quad\text{for each $\beta$ in $K$}.
\end{equation}
We write more simply $\Hom(K, \C)$ for the collection of all $d$ distinct embeddings of $K$ into $\C$.
If $\alpha \not= 0$ belongs to the ring $O_K$ of algebraic integers in $K$ then from (\ref{extra258}), (\ref{extra265}), and (\ref{extra274}), 
we obtain the identity
\begin{equation}\label{extra283}
H(\alpha)^d = \prod_{v | \infty} \max\big\{1, \|\alpha\|_v^{d_v}\big\} = \prod_{\tau \in \Hom(K, \C)} \max\big\{1, |\tau \alpha|\big\}.
\end{equation}

The following result is a version of Minkowski's basic theorem on lattice points in convex bodies.  A proof is given in \cite[Theorem 5.3]{neukirch1999}.

\begin{theorem}\label{thmhaar3}  Let $\fA \subseteq O_K$ be a nonzero integral ideal, and for each embedding $\tau$ in $\Hom(K, \C)$ let
$b(\tau)$ be a positive real number.  Assume that $b(\rho \tau) = b(\tau)$ for each $\tau$ in $\Hom(K, \C)$, and
\begin{equation}\label{extra310}
(c_K)^d \bigl[O_K : \fA\bigr] < \prod_{\tau \in \Hom(K, \C)} b(\tau),
\end{equation} 
where $c_K$ is defined in {\rm (\ref{early3})}.  Then there exists $\xi \not= 0$ in $\fA$ such that
\begin{equation}\label{extra326}
|\tau \xi| < b(\tau)\quad\text{for all $\tau$ in $\Hom(K, \C)$}.
\end{equation}
\end{theorem}

The next lemma is a simple application of Minkowski's theorem but restricted to the case $\fA = O_K$.

\begin{lemma}\label{lemhaar1}  Assume that the field $K$ has at least two archimedean places.  Let $w$ be an archimedean place of $K$.  
Then there exists a nonzero algebraic integer $\xi^{(w)}$ in $O_K$ such that
\begin{equation}\label{extra338}
0 < \bigl|\xi^{(w)}\bigr|_v < 1\quad\text{if $v | \infty$ and $v \not= w$,}
\end{equation}
and
\begin{equation}\label{extra345}
1 < \bigl|\xi^{(w)}\bigr|_w = H\bigl(\xi^{(w)}\bigr) \le  c_K.
\end{equation}
\end{lemma}

\begin{proof}  Let $0 < \ep$.  If $v \not= w$ is a real place of $K$ and 
\begin{equation*}\label{extra348}
\sigma_v : K \rightarrow \C 
\end{equation*}
is the associated embedding, we set
\begin{equation}\label{extra355}
b\bigl(\sigma_v\bigr) = 1.
\end{equation}  
Similarly, if $v \not= w$ is a complex place of $K$ and the associated embeddings are
\begin{equation*}\label{extra362}
\sigma_v : K \rightarrow \C\quad\text{and}\quad \rho \sigma_v : K \rightarrow \C,
\end{equation*}
we set
\begin{equation}\label{extra369}
b\bigl(\sigma_v\bigr) = b\bigl(\rho \sigma_v\bigr) = 1.
\end{equation}
If $w$ is a real place we define
\begin{equation}\label{extra376}
b\bigl(\sigma_w\bigr) = (1 + \ep)^d (c_K)^d,
\end{equation}
and if $w$ is a complex place we define
\begin{equation*}\label{extra383}
b\bigl(\sigma_w\bigr) = b\bigl(\rho \sigma_w\bigr) = \big\{(1 + \ep)^d (c_K)^d \big\}^{\hh}.
\end{equation*}
Then it follows in all cases that
\begin{equation}\label{extra390}
\prod_{\tau \in \Hom(K, \C)} b(\tau) = (1 + \ep)^d (c_K)^d.
\end{equation}

The identity (\ref{extra390}) implies that the positive real numbers $b(\tau)$ satisfy the hypothesis (\ref{extra310}) in the statement of 
Theorem \ref{thmhaar3}.  Then it follows from the conclusion of Theorem \ref{thmhaar3} that there exists an algebraic integer $\xi^{(w)} \not= 0$ 
in $O_K$ such that 
\begin{equation*}\label{extra420}
\bigl|\tau \xi^{(w)}\bigr| <  b(\tau)\quad\text{for all $\tau$ in $\Hom(K, \C)$}.
\end{equation*}
To obtain a bound for the Weil height of $\xi^{(w)}$ we apply (\ref{extra283}), (\ref{extra326}), the definitions (\ref{extra355}), (\ref{extra369}), and
(\ref{extra376}) given to the numbers $b(\sigma_v)$ and $b(\sigma_w)$, and the identity (\ref{extra390}).  We find that
\begin{equation*}\label{extra427}
\begin{split}
H\bigl(\xi^{(w)}\bigr)^d &= \prod_{\tau \in \Hom(K, \C)} \max\big\{1, \bigl|\tau \xi^{(w)}\bigr|\big\} < \prod_{\tau \in \Hom(K, \C)} b(\tau)\\
                                     &= (1 + \ep)^d (c_K)^d,
\end{split}
\end{equation*}
and therefore
\begin{equation*}\label{extra434}
H\bigl(\xi^{(w)}\bigr) < (1 + \ep) c_K.
\end{equation*}

We have shown that for every positive value of $\ep$ there exists a nonzero algebraic integer $\xi^{(w)}$ in $O_K$ such that
\begin{equation}\label{extra452}
0 < \bigl|\xi^{(w)}\bigr|_v < 1\quad\text{if $v | \infty$ and $v \not= w$,}
\end{equation}
and
\begin{equation}\label{extra459}
1 < \bigl|\xi^{(w)}\bigr|_w = H\bigl(\xi^{(w)}\bigr) \le (1 + \ep) c_K.
\end{equation}
It follows from Northcott's theorem \cite{northcott1949} that the set of algebraic integers $\xi^{(w)}$ in $O_K$ that satisfy (\ref{extra452}) and
(\ref{extra459}) with $\ep = 1$ is finite.  And we have shown that this finite set is not empty for every positive value of $\ep$.  Therefore the 
statement of the lemma follows.
\end{proof}

\begin{lemma}\label{lemhaar2}  Assume that the field $K$ is totally complex and has at least two archimedean places.
Let $w$ be an archimedean place of $K$ and let $\xi^{(w)}$ be a nonzero element of $O_K$ that satisfies the inequalities {\rm (\ref{extra338})}
and {\rm (\ref{extra345})} in the statement of {\rm Lemma \ref{lemhaar1}}.  Write
\begin{equation}\label{extra498}
\sigma_w : K \rightarrow \C\quad \text{and}\quad \rho \sigma_w : K \rightarrow \C
\end{equation}
for the embeddings of $K$ into $\C$ that satisfy the identity {\rm (\ref{extra274})}.  Assume that the field
\begin{equation}\label{extra505}
k^{(w)} = \Q\bigl(\xi^{(w)}\bigr)
\end{equation}
is a proper subfield of $K$.   Then we have 
\begin{equation}\label{extra510}
\bigl[K : k^{(w)}\bigr] = 2.  
\end{equation}
Moreover, the subfield $k^{(w)}$ satisfies 
\begin{equation}\label{extra516}
\sigma_w\bigl(k^{(w)}\bigr) \subseteq \R,\quad\text{and}\quad \sigma_w\bigl(k^{(w)}\bigr) = \sigma_w(K) \cap \R.
\end{equation}
And the restriction of complex conjugation
\begin{equation}\label{extra523}
\rho : \sigma_w(K) \rightarrow \sigma_w(K)
\end{equation}
is an automorphism that fixes the subfield $\sigma_w\bigl(k^{(w)}\bigr)$.
\end{lemma}

\begin{proof}  Let $w$ be an archimedean place of $K$, and let $x$ be the unique archimedean place of the proper subfield $k^{(w)}$ that 
satisfies $w | x$.  Write $K_w$ for the completion of $K$ at the place $w$ and $k_x^{(w)}$ for the completion of $k^{(w)}$ at the place $x$.  
Then it follows from (\ref{extra345}) that
\begin{equation*}\label{extra537}
1 < \big\|\xi^{(w)}\big\|_w = \big\|\xi^{(w)}\big\|_x.
\end{equation*}
If $v \not= w$ is a second archimedean place of $K$ that also satisfies $v | x$, we find that
\begin{equation}\label{extra544}
1 < \big\|\xi^{(w)}\big\|_v = \big\|\xi^{(w)}\big\|_x.
\end{equation}
However, (\ref{extra338}) implies that
\begin{equation*}\label{extra551}
\big\|\xi^{(w)}\big\|_v < 1,
\end{equation*}
and this contradicts (\ref{extra544}).  We conclude that $w$ is the {\it unique} place of $K$ that satisfies $w | x$.  Because 
the global degree of the extension $K/k^{(w)}$ is the sum of local degrees over the completion $k_x^{(w)}$, we conclude that
\begin{equation}\label{extra558}
2 \le \bigl[K : k^{(w)}\bigr] = \bigl[K_w : k_x^{(w)}\bigr] \le [\C : \R] = 2.
\end{equation}
This verifies (\ref{extra510}).  

It also follows from the equality in (\ref{extra558}) that the completion $k_x^{(w)}$ at the unique place $x$ such that $w | x$ is isomorphic 
to $\R$.  We conclude that
\begin{equation}\label{extra565}
\sigma_w\bigl(k^{(w)}\bigr) \subseteq \R,\quad\text{and}\quad \sigma_w\bigl(k^{(w)}\bigr) \subseteq \sigma_w(K) \cap \R \subseteq \sigma_w(K).
\end{equation}
Because $K/k^{(w)}$ is a quadratic extension it follows from the two containments on the right of (\ref{extra565}) that either
\begin{equation}\label{extra572}
\sigma_w\bigl(k^{(w)}\bigr) = \sigma_w(K) \cap \R\quad\text{or}\quad \sigma_w(K) \cap \R = \sigma_w(K).
\end{equation}
As $K$ is totally complex the equality on the right of (\ref{extra572}) is clearly impossible, and we conclude that the equality on the left of 
(\ref{extra572}) must hold.  This verifies the equality on the right of (\ref{extra516}).  It also shows that the automorphism $\rho$ in 
(\ref{extra523}) fixes the subfield $\sigma_w\bigl(k^{(w)}\bigr)$.
\end{proof}

In the statement of Lemma \ref{lemhaar2} the map $\rho : \C \rightarrow \C$ is restricted in (\ref{extra523}) to the various embeddings of $K$ 
into $\C$.   We record a variant in which $K$ is fixed and the group $\Aut\bigl(K/k^{(w)}\bigr)$ is identified.

\begin{corollary}\label{corhaar1}  Let $K$ be an algebraic number field that satisfies the hypotheses of {\rm Lemma \ref{lemhaar2}}.  For
each archimedean place $w$ of $K$ let 
\begin{equation}\label{extra603}
k^{(w)} = \Q\bigl(\xi^{(w)}\bigr)
\end{equation}
be the subfield defined in {\rm (\ref{extra505})}.  Then $K/k^{(w)}$ is a Galois extension of order $2$ and
\begin{equation}\label{extra610}
\Aut\bigl(K/k^{(w)}\bigr) = \langle \sigma_w^{-1} \rho \sigma_w\rangle.
\end{equation}
\end{corollary}

\begin{proof}  It follows from (\ref{extra523}) that
\begin{equation}\label{extra617}
\sigma_w^{-1} \rho \sigma_w : K \rightarrow K
\end{equation}
is an automorphism that fixes the subfield $k^{(w)} \subseteq K$.  As $K/k^{(w)}$ is an extension of degree $2$ it is Galois and 
$\Aut\bigl(K/k^{(w)}\bigr)$ is generated by the nontrivial automorphism (\ref{extra617}).
\end{proof}

Next we prove an elementary lemma from Galois theory.

\begin{lemma}\label{lemhaar3}  Let $L$ be an algebraic number field and let
\begin{equation}\label{extra190}
\big\{\ell_1, \ell_2, \cdots , \ell_J\big\}
\end{equation}
be a finite collection of subfields of $L$.  Assume that $L/\ell_j$ is a Galois extension for each $j = 1, 2, \dots , J$.  If
\begin{equation}\label{extra197}
F = \ell_1 \cap \ell_2 \cap \cdots \cap \ell_J
\end{equation}
then $L/F$ is a Galois extension.  Moreover, let
\begin{equation*}\label{extra204}
H_j = \Aut(L/\ell_j) \subseteq \Aut(L/F)\quad\text{for $j = 1, 2, \dots , J$,}
\end{equation*}  
be the subgroups attached to the extensions $L/\ell_j$.  And let
\begin{equation}\label{extra211}
G = \langle H_1, H_2, \cdots , H_J\rangle \subseteq \Aut(L/F)
\end{equation}
be the group generated by the collection of subgroups $H_1, H_2, \dots , H_J$.  Then we have
\begin{equation}\label{extra218}
G = \Aut(L/F).
\end{equation}
\end{lemma}

\begin{proof}  Let $\alpha$ in $L$ satisfy $L = F(\alpha)$.  Then let $P(x)$ be the monic polynomial  
\begin{equation*}\label{extra225}
P(x) = \prod_{\gamma \in G} \bigl(x - \gamma(\alpha)\bigr).
\end{equation*}
It follows that $P(\alpha) = 0$. Since $G \subseteq \Aut(L/F)$ each root $\gamma(\alpha)$ belongs to $L$.  Assume that $\gamma_1$ and 
$\gamma_2$ are distinct automorphisms in $G$ and satisfy 
\begin{equation*}\label{extra232}
\gamma_1(\alpha) = \gamma_2(\alpha).  
\end{equation*}
As $\gamma_1$ and $\gamma_2$ agree on $F$ and agree at $\alpha$, they also agree on the field $F(\alpha) = L$.  This contradicts
our assumption that $\gamma_1$ and $\gamma_2$ are distinct, and it follows that $P(x)$ is separable.

Let $\vphi$ be an automorphism in $G$.  Then we have
\begin{equation}\label{extra239}
\begin{split}
\vphi\bigl(P(x)\bigr) &= \prod_{\gamma \in G} \bigl(x - \vphi\bigl(\gamma(\alpha)\bigr)\bigr)\\ 
			      &= \prod_{\gamma \in G} \bigl(x - \gamma(\alpha)\bigr) = P(x).
\end{split}
\end{equation}
From (\ref{extra239}) we conclude that each coefficient of $P(x)$ is fixed by the automorphisms in each of the groups $H_j$ for 
$j = 1, 2, \dots , J$.  Therefore each coefficient belongs to the field
\begin{equation*}\label{extra245}
\ell_1 \cap \ell_2 \cap \cdots \cap \ell_J = F.
\end{equation*}
This shows that $P(x)$ is a separable polynomial in $F[x]$.  As $P(x)$ splits in $L$ it follows that $L/F$ is
a Galois extension and $\Aut(L/F)$ is given by (\ref{extra218}).
\end{proof}

%%%%%%%%%%%%%%%%%%%%%%%%%%%%%%%%%%%%%%%%%%%%%%%%%%%%%%%%%%%%%%%%%%%%
\section{Proof of Theorem \ref{thmearly2}}%%section 3

If $K$ has an embedding into $\R$ then the inequality (\ref{early22}) follows from (\ref{early16}) in the statement of Theorem \ref{thmearly1}.  
Therefore we assume throughout the remainder of the proof that $K$ is totally complex.  

If $K$ is a complex quadratic field then we have $K = \Q(\mu)$ because $\mu$ is totally complex.  It follows that (\ref{early22}) holds with
$\alpha = \mu$. 

For the remainder of the proof we assume that $K$ is totally complex and $K$ has at least two archimedean places.  Let $w$ be an archimedean 
place of $K$ such that $\sigma_w(\mu)$ is complex and not real.  

Let $\xi^{(w)}$ be a nonzero algebraic integer in $O_K$ that satisfies the inequalities (\ref{extra338}) and (\ref{extra345}) in Lemma \ref{lemhaar1}.   
We set $\alpha = \mu \xi^{(w)}$ and define the subfield
\begin{equation}\label{temp30}
\ell = \Q(\alpha) = \Q\bigl(\mu \xi^{(w)}\bigr) \subseteq K.
\end{equation}
If there is equality in the containment (\ref{temp30}) then using (\ref{extra345}) we find that
\begin{equation}\label{temp34}
H(\alpha) = H\bigl(\mu \xi^{(w)}\bigr) \le H(\mu) H\bigl(\xi^{(w)}\bigr) \le H(\mu) c_K.
\end{equation}
This verifies the inequality in (\ref{early22}).

To complete the proof we assume that
\begin{equation*}\label{temp41}
\ell = \Q(\alpha) = \Q\bigl(\mu \xi^{(w)}\bigr) \subseteq K
\end{equation*}
is a proper subfield of $K$.   Let $y$ be the unique archimedean place of $\ell$ such that $w | y$.   If $v \not= w$ is an archimedean place of 
$K$ such that $v | y$ then we would have both
\begin{equation*}\label{temp46}
1 <  \big\|\xi^{(w)}\big\|_w = \big\|\xi^{(w)}\big\|_y
\end{equation*}
and
\begin{equation}\label{temp51}
1 <  \big\|\xi^{(w)}\big\|_v = \big\|\xi^{(w)}\big\|_y.
\end{equation}
However, (\ref{temp51}) contradicts the inequality (\ref{extra338}) and we conclude that $w$ is the unique place of $K$ that satisfies $w | y$.  
Because $\sigma_y : \ell \rightarrow \C$ is the restriction of $\sigma_w : K \rightarrow \C$ to the subfield $\ell$, we find that
\begin{equation*}\label{temp55}
\sigma_y\bigl(\mu \xi^{(w)}\bigr) = \sigma_w\bigl(\mu \xi^{(w)}\bigr) = \sigma_w(\mu) \sigma_w\bigr(\xi^{(w)}\bigr).
\end{equation*}
The place $w$ was selected so that $\sigma_w(\mu)$ is complex and not real.  And it follows from (\ref{extra510}) and (\ref{extra516})
that $\sigma_w\bigl(\xi^{(w)}\bigr)$ is real.  Therefore $\sigma_y\bigl(\mu \xi^{(w)}\bigr)$ is complex and not real.  We conclude that the 
completion $\ell_y$ at the place $y$ is isomorphic to $\C$ and not isomorphic to $\R$.  As the global degree $[K : \ell]$ is the 
sum of local degrees over the archimedean place $y$ of $\ell$, we find that
\begin{equation*}\label{temp57}
[K : \ell] = [K_w : \ell_y] = [\C : \C] = 1.
\end{equation*}
We have shown $K = \ell$, and therefore the inequality (\ref{temp34}) applies in all cases.  This proves Theorem \ref{thmearly2}.

%%%%%%%%%%%%%%%%%%%%%%%%%%%%%%%%%%%%%%%%%%%%%%%%%%%%%%%%%%%%%%%%%%%
\section{Proof of Theorem \ref{thmearly3}}%%section 4

If $K$ has a real embedding then it follows from Theorem \ref{thmearly1} that the inequality (\ref{early35}) is false.  

Now suppose that $K$ has only complex embeddings and there exists $\mu \in O_K$ such that $H(\mu) \leq \hH$ and $\mu \notin F$.  
It follows that $\mu$ is not totally real.  And Theorem \ref{thmearly2} implies that there exists $\alpha \not= 0$ in $O_K$ such that 
\begin{equation}\label{extra711}
K = \Q(\alpha),\quad\text{and}\quad H(\alpha) \le H(\mu) c_K \le \hH c_K.
\end{equation}
But the inequality (\ref{extra711}) contradicts the hypothesis (\ref{early35}) in the statement of Theorem \ref{thmearly3}.  We conclude that the
hypothesis (\ref{early35}) implies that $K$ is totally complex and that every $\mu \in O_K$ with $H(\mu) \leq \hH$ lies in $F$.  
It remains to prove that (\ref{early35}) also implies that $K/F$ is a Galois extension.

If $K$ is totally complex and has exactly one archimedean place then $K$ is an imaginary quadratic extension of $\Q$.  In this case it is easy to
prove that $F = \Q$ and $K/\Q$ is Galois.

For the remainder of the proof we restrict our attention to fields $K$ that are totally complex, that satisfy 
\begin{equation*}\label{extra714}
\{\alpha\in O_K : H(\alpha) \leq \hH\} \subseteq F,
\end{equation*}
that satisfy the inequality (\ref{early35}), and that have at least two archimedean places.  Therefore $K$ satisfies the hypotheses of 
Lemma \ref{lemhaar1} and Lemma \ref{lemhaar2}.  For each archimedean place $w$ of $K$ we write $\xi^{(w)}$ for algebraic integers that 
satisfy the inequalities (\ref{extra338}) and (\ref{extra345}) in the statement of Lemma \ref{lemhaar1}.  Then we define the subfields
\begin{equation}\label{extra719}
k^{(w)} = \Q\bigl(\xi^{(w)}\bigr)
\end{equation}
as in (\ref{extra505}) and (\ref{extra603}).  It follows from Lemma \ref{lemhaar2} and Corollary \ref{corhaar1} that $K/k^{(w)}$ is a 
Galois extension of order $2$ and
\begin{equation}\label{extra726}
\Aut\bigl(K/k^{(w)}\bigr) = \langle \sigma_w^{-1} \rho \sigma_w\rangle.
\end{equation}

Next we define the subfield
\begin{equation}\label{extra748}
\widetilde{F} = \bigcap_{w | \infty} k^{(w)} \subseteq K.
\end{equation}
Then it follows from Lemma \ref{lemhaar3} that $K/\widetilde{F}$ is a Galois extension and
\begin{equation}\label{extra755}
\Aut(K/\widetilde{F}) = \langle \sigma_w^{-1} \rho \sigma_w : \text{$w | \infty$ is a place of $K$}\rangle.
\end{equation}
If $\alpha \not= 0$ belongs to $\widetilde{F}$ then it follows from (\ref{extra748}) that $\alpha$ belongs to $k^{(w)}$ at each archimedean place $w$ of $K$. 
And it follows from the containment on the left of (\ref{extra516}) that
\begin{equation*}\label{extra762}
\sigma_w(\alpha) \subseteq \sigma_w\bigl(k^{(w)}\bigr) \subseteq \R
\end{equation*}
at each archimedean place $w$ of $K$.  We conclude that $\widetilde{F}$ is totally real.

Finally, let $\beta \not= 0$ be an element of $K$ that is totally real.  Then at each archimedean place $w$ of $K$ we have both 
$\sigma_w(\beta) \subseteq \sigma_w(K)$ and $\sigma_w(\beta) \subseteq \R$.  Applying the identity on the right of (\ref{extra516}) we find that
\begin{equation}\label{extra769}
\sigma_w(\beta) \subseteq \sigma_w(K) \cap \R = \sigma_w\bigl(k^{(w)}\bigr)
\end{equation}  
at each archimedean place $w$ of $K$.  We conclude from (\ref{extra769}) that
\begin{equation*}\label{extra776}
\beta \in \bigcap_{w | \infty} k^{(w)} = \widetilde{F},
\end{equation*} 
and therefore $\widetilde{F} = F$ is the maximal totally real subfield in $K$.

%%%%%%%%%%%%%%%%%%%%%%%%%%%%%%%%%%%%%%%%%%%%%%%%%%%%%%%%%%%%%%%%%%%
\section*{Acknowledgements}

We thank Dr. Michael Mossinghoff for insightful  conversation about this project. We are grateful to Professor Hendrik Lenstra for pointing out an error in an earlier version of this manuscript. 


The authors acknowledge support from the Max Planck Institute for Mathematics in Bonn and support from the  Institute for 
Advanced Study in  Princeton.

Shabnam Akhtari's research is partially supported by the National Science Foundation Awards DMS-2001281 and DMS-2327098.


\begin{thebibliography}{99}

\bibitem{alaca2004}
	{{\c S}.~Alaca and K.~S.~Williams}, Introductory Algebraic Number Theory, Cambridge U. Press, New York,  2004.
	

\bibitem{blanksby1978}
	 {P.~E.~Blanksby and J.~H.~Loxton}, A note on the characterization of {CM}-fields,
	 {J. Austral. Math. Soc. (Series A)}
	26:   26--30, 1978.
	
	\bibitem{bombieri2006} {E. Bombieri and W. Gubler}, {Heights in Diophantine Geometry},
	{Cambridge U. Press},  {New York},
		2006.
	
	
	\bibitem{huardspearmanwilliams1995}
	 {J.~G.~Huard and B.~K.~Spearman and K.~S.~Williams}, {Integral bases for quartic fields with  quadratic subfields},
	 {J. Number Theory},
	51:  {87--102},1995.
	
	
	\bibitem{neukirch1999}
	 {J.~Neukirch}, {Algebraic Number Theory},
	{Springer-Verlag}, {New York},
	1999.
	
	
	\bibitem{ruppert1998}
	 {W.~Ruppert},  {Small generators of number fields}
	{Manuscripta Math.},
	96(1):	{17--22},
	 {1998}.
	
	
	\bibitem{shimura1971}
	 {G.~Shimura}, {Introduction to the Arithmetic Theory of Automorphic Functions}
	{Princeton University Press}, Princeton, NJ, 
	1971.
	
	\bibitem{vaaler2013}
	
	 {J.~D.~Vaaler and M.~Widmer}, {A note on generators of number fields}, In
	 {Diophantine Methods, Lattices and the Arithmetic Theory of Quadratic Forms}, Vol. {587} of  {Contemp. Math.}
	{201--211},
	 {Amer. Math. Soc.},  {Providence, RI} 2013.
	
\end{thebibliography}
\end{document}

 