\begin{table*}[!tb]
\centering
\footnotesize
\begin{tabular}{p{0.22\textwidth} p{0.78\textwidth}}
\toprule
\textbf{Term} & \textbf{Definition} \\
\midrule
\textit{Partisan animosity} & {Negative thoughts,
feelings and behaviours towards a political out-group.}\\[0.1cm]
\textit{Anti-democratic attitudes} & {Eight variables from the political science literature, combined into a measure in ~\citet{voelkel2023megastudy} to describe willingness to engage in good faith in the democratic process: (1)~partisan animosity, (2)~support for undemocratic practices, (3)~support for partisan violence, (4)~support for undemocratic candidates, (5)~opposition to bipartisanship, (6)~social distrust, (7)~social distance, and (8)~biased evaluation of politicized facts.}\\[0.1cm]
\textit{Democratic attitude model} & {Our AI model that leverages zero-shot prompts to a large language model to rate the impact of social media posts on anti-democratic attitudes. The model takes social media posts as input and generates scores for each of the eight anti-democratic attitudes. These scores estimate the extent to which the post promotes each of the eight anti-democratic attitudes, and are summed to a single democratic attitude score.}\\[0.1cm]
\textit{Democratic attitude feed} & {A re-ranked social media feed that uses the democratic attitude model to perform downranking, generate content warnings, or perform content removal-and-replacement.}\\[0.1cm]
\textit{Societal objective function} & {Our method that integrates established and vetted social scientific constructs into objective functions by translating from constructs to manual codebooks to algorithmic ranking using zero-shot prompting with LLMs.}\\[0.1cm]
\bottomrule
\end{tabular}
\vspace{-0.01in}
\caption{A summary of key terms used throughout the paper.}
\label{tab:term_bank}
\end{table*}