\section{Study 1: The Impact of Manually Labeled Democratic Attitude Feeds}

In Study 1, we examined the impact of manually-generated democratic attitude feeds on the partisan animosity of US partisans. By experimenting with a manual version of our ranking algorithm, we aim to understand the effect of democratic attitude feeds given a hypothetical ``perfect'' AI that reflects the initial social science construct via expert annotation. 
% \hgv{I thought you were using a simulated model? if so, I don't think it is considered wizard-of-oz}

\subsection{Method}
\subsubsection{Experimental Design}
We created a social media feed named \textit{PolitiFeed} that consists of a wide variety of real-world political posts, ranging from posts with high anti-democratic attitudes and low anti-democratic attitudes. We conducted a between-subjects design among US partisans ($N$ = 1,380) in March 2023 with seven conditions, respectively the downranking, content warning, removal, engagement-based, ideologically balanced, chronological feed, and null control. We randomly assigned participants to one of the seven conditions, as shown in Figure~\ref{fig:condition}. In different treatment conditions, participants were exposed to scrollable social media feeds with the same inventory content but different ranking methods. In the control condition, participants were exposed to a chronological social media feed. Participants in each feed condition were asked to read 60 political posts sourced from the inventory dataset, as 60 posts roughly replicate two full loads of the Twitter timeline. In the null control condition, participants were not exposed to any feeds. Participants spent on average 339.39 seconds on viewing the feed and 631.31 seconds on answering the questionnaire. 

\subsubsection{Participants} 
To detect a main effect of condition and interactions with people's partisan affiliation, a priori power analysis using G*Power determined that a total sample of at least 1,369 participants would be needed to achieve 80\% power if $\alpha$ = .05 and effect size $f$ = .10 were assumed to be the minimum effect size of interest. We recruited 1,427 participants in March 2023 using CloudResearch Connect, an online participant pool that aggregates multiple market research platforms \cite{litman2017turkprime}. 
Participants were all from the United States and were required to be over 18 years old. We also filtered out participants who indicated they were non-partisans. Participants who identified as ``True Independents'' who leaned neither toward the Democratic nor Republican parties were filtered out of the study before participants were randomly assigned to conditions. After ruling out people who failed the embedded attention check question ($n$ = 26), were under the age of 18 ($n$ = 1), used duplicate IP addresses ($n$ = 15), gave incomplete answers ($n$ = 1), or spent less than two minutes on the survey ($n$ = 4), 1,380 participants remained in the data analysis.

\subsubsection{Platform Development}

In order to review the effects of ranking political content in realistic social media environments, we created the PolitiFeed web page that resembles modern social media feeds and used real Facebook posts from political candidates as content for this mock social media feed. This webpage was developed using Flask, a Python web framework, and included study filtering and attention checks as well as the mock social media feed. The site was deployed on Heroku, a cloud-based deployment platform, for easy accessibility and stability when handling a significant influx of subjects. The data was recorded via Qualtrics and included participants' responses and engagement with the political content presented on the web page. 

We designed PolitiFeed to resemble modern social media platforms to emulate users' behavior and experiences on their own social media feeds and better measure the impacts of alternative ranking methods. Additionally, we intentionally designed the website logo ``PolitiFeed'' with a gradient color (both blue and red) to limit potential political bias. An example of the website interface is shown in Figure \ref{fig:engagement}.

% Figure environment removed

\subsubsection{Measurements}
Here, we provide an overview of our Study 1 measures.

\begin{itemize}
    \item \textbf{Partisan Animosity}. Adapting from \cite{voelkel2023megastudy}, we measured partisan animosity using an instrument called the ``feeling thermometer'' ($M$ = 74.67, $SD$ = 22.26). We asked participants to rate their feelings towards both Democrats and Republicans on a 101-point scale. Participants were told, ``Ratings between 50 degrees and 100 degrees mean that you feel favorable and warm toward them. Ratings between 0 degrees and 50 degrees mean that you don't feel favorable toward them and that you don't care too much for them. You would rate them at the 50 degree mark if you don't feel particularly warm or cold toward them.'' We then reverse coded participants' rating towards outpartisans so that higher scores indicated higher partisan animosity.

    \item \textbf{Support for Undemocratic Practices}. Adapting from \cite{voelkel2023megastudy}, we asked participants to rate their support for undemocratic practices (Cronbach's $\alpha$  = .76, $M$ = 15.62, $SD$ = 18.51) on a 101-point scale by indicating the extent to which they disagree or agree with each of the following statements:
        \begin{itemize}
            \item (Republicans/Democrats) should reduce the number of polling stations in areas that support (Democrats/Republicans).
            \item (Republican/Democratic) governors should ignore unfavorable court rulings by (Democrat/Republican) -appointed judges.
            \item (Republican/Democratic) governors should prosecute journalists who accuse (Republican/Democratic) politicians of misconduct without revealing sources.
            \item (Republicans/Democrats) should not accept the results of elections if they lose.
        \end{itemize}
    
   \item \textbf{Perceived Threat to Freedom of Speech}. Adapting from previous research \cite{dillard2005nature, moyer2010explaining}, four items were included to assess the degree to which one perceives a threat to freedom in response to the ranking methods. Participants were asked to indicate how much they agree or disagree with the following statements based on a 7-point scale (1 = strongly disagree, 7 = strongly agree): ``The social media platform I just used threatened users’ freedom to express on social media,'' ``The social media platform I just used tried to manipulate users on the platform,'' ``The social media platform I just used tried to pressure users on the platform,'' and ``The social media platform I just used tried to make a decision for users.'' The four items were highly correlated and could be formed into a reliable index (Cronbach's $\alpha$ = .93, $M$ = 2.79, $SD$ = 1.57).

    \item \textbf{Feed-level Satisfaction}. We adapted feed-level measures from \citet{jannach2016recommendations} and a survey conducted by Facebook in 2019. After exposure to the holistic social media feed, participants will be asked to rate three items: ``Is PolitiFeed you just read worth your time?'', ``Are you satisfied with PolitiFeed?'', and ``Is PolitiFeed you just read helpful for users to find relevant items?'' on 7-point scales from 1 (not at all) to 7 (very). The three items were highly correlated and could be formed into a reliable index (Cronbach's $\alpha$  = .91, $M$ = 4.74, $SD$ = 1.95).
       
   \item \textbf{Time Spent on the Social Media Feed}. We measured engagement level by recording participants' time elapsed on the social media feed ($M$ = 414.51 seconds).
      
    \item \textbf{Post-survey Measurements}. We collected demographic information from participants, such as education, income, gender, and social media use. 

\end{itemize}

\subsubsection{Data Analysis Plan}

We used multiple generalized linear models (GLMs) with post hoc tests using Bonferroni correction to compare the effects of different ranking methods on participants’ partisan animosity, engagement level, and feed-level satisfaction.

\subsection{Results}


\subsubsection{Democratic attitude feeds significantly reduced partisan animosity}

\textbf{H1} was mostly supported: it predicted that partisans exposed to the downranking, content warning, and removal conditions will reduce their partisan animosity compared to partisans exposed to the engagement feed and chronological feed. We visualize the results in Figure \ref{fig:animosity}. A generalized linear model (GLM) was used to test the effects of different feeds and party affiliation on partisan animosity. There was a significant main effect of feed condition, \textit{F}(6, 1366) = 2.68, $p$ = .01, \begin{math}\eta_{\text{p}}^{2}\end{math} = .01, and a significant main effect of party, $F$(1, 1366) = 33.93, $p$ < .001, \begin{math}\eta_{\text{p}}^{2}\end{math} = .02 on partisan animosity. There was no significant interaction effect between condition and party, $F$ (6, 1366) = 1.04, $p$ = .40, \begin{math}\eta_{\text{p}}^{2}\end{math} = .01, which means the effect of the feed condition on partisan animosity did not differ by partisanship. 
% \msb{I rewrote the rest of this paragraph. The details are in the table. Notice how this is more focused on the narrative.} 
We report a full set of pairwise comparisons between conditions in Table \ref{tab:pairwise}.
Notably, the removal and downranking conditions resulted in significantly less partisan animosity than the traditional engagement-based feed ($d = -.20$, $p$ = .04, and $d = -.25$, $p$ = .02). The effect of the content warning condition was only marginal ($d = -.18$, $p = .10$) --- as seen in Figure~\ref{fig:animosity}, it was roughly as effective as the other democratic attitude conditions for Democrats, but far less effective for Republicans. The three democratic attitude feeds had no measurable difference from the chronological condition (all $p >.05$). 
In summary: while the content warning feed backfired with conservatives, the removal and downranking feeds decreased partisan animosity compared to the engagement feed.

% Figure environment removed

% Please add the following required packages to your document preamble:
% \usepackage{booktabs}
\begin{table}[tb]
\resizebox{0.65\textwidth}{!}{%
\begin{tabular}{@{}lccccccc@{}}
\toprule
\textbf{Condition Comparisons} & \multicolumn{1}{c}{\textbf{Mean Diff.}} & \multicolumn{1}{c}{\textit{\textbf{SE}}} & \multicolumn{1}{c}{\textit{\textbf{p}}} & \multicolumn{1}{c}{\textbf{Cohen's $d$}} \\ \midrule
Content warning/Engagement                         & -3.21                                        & 1.97                                     & .10                                     & -.18                                            \\
Removal/Engagement                                 & -4.48                                        & 2.12                                     & .04*                                    & -.20                                            \\
Downranking/Engagement                                & -4.97                                        & 2.11                                     & .02*                                   & -.25                                            \\
Null/Engagement                                    & -8.07                                        & 2.10                                     & \textless{}.001***                      & -.40                                            \\
Chronological/Engagement                           & -4.82                                        & 2.10                                     & .02*                                    & -.23                                            \\
Ideology/Engagement                                & -0.95                                        & 2.08                                     & .65                                     & -.06                                            \\
Content warning/Chronological                      & 1.69                                         & 2.17                                     & .44                                     & .07                                             \\
Removal/Chronological                              & .39                                         & 2.32                                     & .87                                     & .03                                             \\
Downranking/Chronological                             & -.17                                        & 2.31                                     & .94                                     & -.02                                            \\
Null/Chronological                                 & -3.23                                        & 2.29                                     & .16                                     & -.15                                            \\
Ideology/Chronological                             & 3.85                                         & 2.28                                     & .09†                                    & .17                                             \\
Content warning/Ideology                           & -2.12                                        & 2.15                                     & .33                                     & -.11                                            \\
Removal/Ideology                                   & -3.49                                        & 2.30                                     & .13                                     & -.13                                            \\
Downranking/Ideology                                  & -4.00                                        & 2.29                                     & .08†                                    & -.18                                            \\
Null/ideology                                      & -7.05                                        & 2.27                                     & .002**                                  & -.32                                            \\
Content warning/Null                               & 4.98                                         & 2.16                                     & .02*                                    & .23                                             \\
Removal/Null                                       & 3.73                                         & 2.31                                     & .11                                     & -.18                                            \\
Downranking/Null                                      & 3.02                                         & 2.30                                     & .19                                     & .14                                             \\
Content warning/Downranking                           & 1.95                                         & 2.18                                     & .37                                     & .08                                             \\
Removal/Downranking                                   & .58                                          & 2.34                                     & .80                                     & .05                                             \\
Content warning/Removal                            & 1.02                                         & 2.20                                     & .64                                     & .03                                             \\ \bottomrule
\end{tabular}
}
\vspace{0.05in}
\caption{Pairwise Comparison of Partisan Animosity (Note: †$p$ < .10, *$p$ < .05, **$p$ < .01 *** $p$ < .001). For each row, the mean difference is calculated by subtracting the partisan animosity of the second condition from that of the first condition.}
\label{tab:pairwise}
% \vspace{-0.4768in}
\end{table}

\textbf{H2}, which tested the impact on support for undemocratic practices, was not supported; however, this is consistent with prior work, which found that support for undemocratic practices is more resistant to short-term interventions than partisan animosity~\cite{voelkel2023megastudy}. A generalized linear model (GLM) was used to test the effects of different feeds and party affiliation on support for undemocratic practices. There was no significant interaction between feed condition and party, $F$ (6, 1366) = 1.11, $p$ = .36,  partial \begin{math}\eta_{\text{p}}^{2}\end{math} = .01 on support for undemocratic practices, and no main effect of feed condition, $F$ (6, 1366) = .93, $p$ =.47, partial \begin{math}\eta_{\text{p}}^{2}\end{math} = .004. There was a significant main effect of party affiliation, $F$ (1, 1366) = 59.64, $p$ < .001, partial \begin{math}\eta_{\text{p}}^{2}\end{math} = .04 on support for undemocratic practices. Specifically, Republicans ($M$=20.52, $SD$= 21.0) had significantly higher support for undemocratic practices than Democrats ($M$=12.73, $SD$ = 16.10), $p$ < .001, Cohen’s $d$ = .42. Pairwise comparisons showed no significant differences across different feeds. 

\subsubsection{Democratic attitude feeds did not compromise feed-level satisfaction and engagement}

The downranking feed did not compromise participants' satisfaction with the feed. There was a marginal significant interaction between feed condition and party, $F$ (5, 1163) = 2.14, $p$=.058,  partial \begin{math}\eta_{\text{p}}^{2}\end{math} = .01 on feel-level satisfaction. There was a significant main effect of feed condition, $F$ (5, 1163) = 3.15, $p$ = .008, partial \begin{math}\eta_{\text{p}}^{2}\end{math} = .013 on feed-level satisfaction, but no main effect of party affiliation, $F$ (1, 1163) = .02, $p$=.89, partial \begin{math}\eta_{\text{p}}^{2}\end{math} = .000. In fact, Democrats exposed to the downranking and removal-and-replace feed even reported greater satisfaction with the feed compared to those exposed to the feed ranked by engagement (downranking vs. engagement: $p$ = .04, Cohen’s $d$ = .25; removal vs. engagement: $p$ = .005, Cohen’s $d$ = .35). For Republicans, there was no significant difference between their satisfaction with the downranking and the engagement-based feed, $p$ =.79. Republicans in the removal feed showed significantly higher level of satisfaction compared to those in the engagement feed, $p$ = .039, Cohen’s $d$ = .36, as shown in Figure \ref{fig:satisfaction}. 
When comparing to the chronological feed, partisans exposed to the removal feed had significantly higher feed-level satisfaction compared to those exposed to the chronological feed, $p$ = .017, Cohen’s $d$ =.13, but there was no significant difference between the downranking and chronological feed, $p$ = .84.

In addition, we found no significant difference between the time spent on feed when partisans were assigned to a downranking feed ($M$=488.52) and the engagement-based feed ($M$=400.24), $p$ = .30, Cohen’s $d$ = .11 (in seconds). Detailed means and $SD$s are reported in the Appendix ~\ref{tab:mean}.


% Figure environment removed

\subsubsection{Content warning feed prompted freedom of speech concerns}
\textbf{H3}, testing threats to freedom of speech, was supported: the removal and downranking feeds did not prompt freedom of speech threats in participants, but the content warning feed did. A generalized linear model (GLM) was used to test the effects of different feeds and party affiliation on partisan animosity. There was a significant interaction effect between condition and party, $F$ (5, 1163) = 2.13, $p$ =.06,  partial \begin{math}\eta_{\text{p}}^{2}\end{math} = .01. There was a significant main effect of the feed condition, $F$ (5, 1163) = 16.86, $p$ <.001,  partial \begin{math}\eta_{\text{p}}^{2}\end{math} = .07, and a significant main effect of party, $F$ (5, 1163) = 8.25, $p$ =.004,  partial \begin{math}\eta_{\text{p}}^{2}\end{math} = .01 on perceived threat to freedom of speech. Pairwise comparisons showed that partisans exposed to the content warning feed perceived a significantly higher level of threat to freedom of speech compared to partisans exposed to other feeds, $p$ <.001, with a range of sizable effects (vs. chronological: Cohen’s $d$ = .62; vs. downranking: Cohen’s $d$ =.75; vs. engagement: Cohen’s $d$ =.72; vs. removal-and-replace: Cohen’s $d$ =.77; vs. ideology: Cohen’s $d$ =.57). We suggest that this threat was likely the reason that the content warning condition was less effective than the other democratic attitude feeds in reducing partisan animosity.

\subsubsection{Summary}
Conditions that utilized manual annotations to craft democratic attitude feeds---the downranking feed and remove-and-replace feed---significantly reduced partisan animosity without reducing engagement and satisfaction, and without or raising freedom of speech concerns. Content warnings, using the same manual annotations, did prompt freedom of speech concerns amongst conservatives, and had no overall effect on reducing partisan animosity. None of the approaches impacted support for undemocratic practices.