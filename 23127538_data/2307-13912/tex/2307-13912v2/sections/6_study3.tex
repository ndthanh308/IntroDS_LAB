\section{Study 3: Replication Using The Democratic Attitude Model}
Closing the loop, we wished to test whether a feed using the automated democratic attitude model labels replicated the effect on partisan animosity that we observed in Study 1 with the manual labels. In Study 3, we conducted a pre-registered replication study ($N$ = 558) \footnote{A priori power analysis using G*Power determined that a total sample of at least 432 participants would be needed to achieve 80\% power if $\alpha$ = .05 and effect size $f$ = .10 were assumed to be the minimum effect size of interest.} among US partisans to compare the algorithmic downranking feed to a manual downranking feed and to a control feed ranked by engagement. Participants were recruited from CloudResearch Connect. Those who participated in Study 1 were ruled out from the study pool. The experimental procedure and measurements remained the same as Study 1.

% elapsed time on the feed (209s) Qualtrics 642.08

\subsection{Experimental Design and Hypothesis}
In Study 3, we conducted a between-subjects study design online experiment with three conditions in June 2023. The \textit{manual} condition was the downranking feed from Study 1 that utilized manual expert labels. The \textit{algorithm} condition was a downranking feed using the output from the democratic attitude model using GPT-4 instead of manual labels. The \textit{engagement} control condition was identical to Study 1. We predicted that partisans exposed to the algorithmic feed would reduce partisan animosity compared to partisans in the engagement feed\footnote{We only wanted to replicate the effect on partisan animosity as we did not find a significant effect on support for undemocratic practices in our Study 1 (H2).}:

\begin{enumerate}
    \item[\textbf{H4:}] Partisans exposed to the (a) manual downranking and (b) algorithmic downranking feed on social media will reduce partisan animosity compared to partisans in the engagement feed.
\end{enumerate}

% \hgv{why do you only replicate H1? not H2? or H3?}

% Based on Study 1 and Study 2 Results, \textbf{H4} predicted that partisans exposed to the (a) manual downrank and (b) GPT-4 downrank feed on social media will reduce partisan animosity compared to partisans in the engagement feed condition.

% \msb{I don't understand this sentence, can you rewrite or clarify? Is it important to have this sentence here? The one after it seems more important.} Since there are many ties (i.e., posts with same ranking scores), we tried to match the GPT-4 ranking distribution with the manual ranking when the scores are the same. 

\subsection{Results}

\subsubsection{Both algorithmic and manual downranked feeds significantly reduced partisan animosity.}
A general linear model (GLM) was used to test the effects of different feeds and party affiliation on the dependent variable, partisan animosity. Both feed condition and party affiliation were entered into the model as two independent variables. \textbf{H4} was supported (Figure~\ref{fig:animositygpt}): the automated, GPT-4 based democratic attitude feed reduced partisan animosity without impacting time on the site. There was a significant main effect of condition, $F$(2,557) = 3.77, $p$ = .024, and a significant main effect of party, $F$(1,557) = 53.31, $p$ <.001, on  partisan animosity. There was no significant interaction effect between party and condition, $F$(2,557) = .52, $p$ = .60. 

Multiple pairwise comparisons using Bonferroni correction showed that partisans exposed to both the manual ($M$ = 68.96, $SD$ = 23.33) and the algorithmic ($M$ = 69.32, $SD$ = 22.59) downranked feeds displayed significantly lower partisan animosity compared to those exposed to the engagement feed ($M$ = 74.82, $SD$ = 19.73), manual vs. engagement: $p$ = .02, $d$ = -.27, algorithmic vs. engagement: $p$ = .036, $d$ = -.25. Same as in Study 1, there was no significant difference of time spent on the feed across conditions, $F$(1,557) = .26, $p$ = .77, \begin{math}\eta_{\text{p}}^{2}\end{math} = .001 (Manual: $M$ = 310.16, $SD$ = 256.96; GPT-4: $M$ = 326.16, $SD$ = 255.01; Engagement-based: $M$ = 318.47, $SD$ = 275.05), which indicates that both democratic attitude feeds did not compromise time spent on the feed.

% Figure environment removed


\subsubsection{Additional analyses on strong vs. weak partisans}
As in Study 1, we conducted additional exploratory analyses (not pre-registered) to examine different impacts of feed conditions on strong vs. weak partisans. We again found a significant difference between the downranking and engagement conditions for weak partisans (manual vs. engagement: $d = -.44$, $p$ = .05; algorithmic vs. engagement: $d = -.26$, $p$ = .08), but not for strong partisans (manual vs. engagement: $d = -.18$, $p$ = .81; algorithmic vs. engagement: $d = -.14$, $p$ = .45).

\subsection{Summary}
In summary, Study 3 replicated the original result using algorithmic labeling rather than manual annotation. The effect size between the downranking feeds and the engagement feed on reducing partisan animosity (Manual vs. Engagement: $d$ = -.27; GPT-4 vs. Engagement: $d$ = -.25), is also the same as Study 1. Additional analyses suggested that the significant differences between the downranking and engagement conditions come from weak partisans and not strong partisans.