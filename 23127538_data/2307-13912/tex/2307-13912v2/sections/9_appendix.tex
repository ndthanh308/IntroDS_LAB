\section*{Appendix}

\subsection{Inter-rater Reliability}
Table~\ref{tab:manual_irr} displays the inter-rater reliability results between two expert annotators for the eight anti-democratic variables.
\begin{table}[!t]
\centering
\footnotesize
% \resizebox{\textwidth}{!}{%
\begin{tabular}{llccc}
\toprule
\textbf{Outcome variable} & \textbf{Krippendorff's $\alpha$} \\ 
\midrule
Partisan Animosity                     & 0.889\\
Support for Undemocratic Practices     & 0.798\\
Support for Partisan Violence          & 0.697\\
Support for Undemocratic Candidates    & 0.798\\
Opposition to Bipartisanship           & 0.810\\
Social Distrust                        & 0.775\\
Social Distance                        & 0.807\\
Biased Evaluation of Politicized Facts & 0.756\\ 
\midrule
Total Score (sum) & 0.895\\
\bottomrule
\end{tabular}%
% }
\vspace{0.05in}
\caption{Inter-coder reliability on the eight anti-democratic variables.}
\label{tab:manual_irr}
\end{table}


\subsection{Additional Tables in Study 1}
Table ~\ref{tab:mean} lists means and $SD$s of key outcome variables in Study 1.
% Please add the following required packages to your document preamble:
% \usepackage{booktabs}
\begin{table}[]
\resizebox{\textwidth}{!}{%
\begin{tabular}{@{}llcccccc@{}}
\toprule
             &                     & \multicolumn{2}{c}{\textbf{Partisan Animosity}} & \multicolumn{2}{c}{\textbf{Feed-Level Satisfaction}} & \multicolumn{2}{c}{\textbf{Time Spent On Feed}} \\ \midrule
             &                     & \textit{\textbf{M}}    & \textit{\textbf{SD}}   & \textit{\textbf{M}}      & \textit{\textbf{SD}}      & \textit{\textbf{M}}    & \textit{\textbf{SD}}   \\ \midrule
Democrats    & Null                & 72.51                  & 22.45                  & -                        & -                         & -                      & -                      \\
             & Content Warning     & 75.30                  & 18.70                  & 4.40                     & 1.73                      & 425.17                 & 598.43                 \\
             & Removal and Replace & 76.45                  & 21.96                  & 5.20                     & 1.94                      & 344.12                 & 311.62                 \\
             & Downranking           & 76.70                  & 21.62                  & 5.04                     & 2.10                      & 446.19                 & 431.40                 \\
             & Engagement          & 82.01                  & 19.71                  & 4.53                     & 1.91                      & 384.85                 & 373.90                 \\
             & Ideology            & 81.20                  & 21.17                  & 4.51                     & 1.88                      & 353.60                 & 318.68                 \\
             & Chronological       & 76.87                  & 23.25                  & 4.77                     & 2.02                      & 478.48                 & 641.19                 \\
             & \textbf{Total}      & \textbf{77.41}         & \textbf{21.48}         & \textbf{4.75}            & \textbf{1.95}             & \textbf{403.84}        & \textbf{460.69}        \\ \midrule
Republicans  & Null                & 66.72                  & 23.44                  & -                        & -                         & -                      & -                      \\
             & Content Warning     & 74.70                  & 20.89                  & 4.58                     & 1.91                      & 463.20                 & 573.34                 \\
             & Removal and Replace & 70.10                  & 25.03                  & 5.20                     & 1.99                      & 322.66                 & 315.94                 \\
             & Downranking           & 68.03                  & 24.37                  & 4.43                     & 1.84                      & 552.57                 & 1676.04                \\
             & Engagement          & 72.44                  & 18.61                  & 4.52                     & 1.82                      & 429.10                 & 336.28                 \\
             & Ideology            & 71.23                  & 24.38                  & 5.18                     & 2.07                      & 412.51                 & 374.25                 \\
             & Chronological       & 68.19                  & 22.26                  & 4.63                     & 2.03                      & 463.20                 & 573.34                 \\
             & \textbf{Total}      & \textbf{70.20}         & \textbf{22.79}         & \textbf{4.47}            & \textbf{1.81}             & \textbf{432.33}        & \textbf{791.83}        \\ \midrule
Both Parties & Null                & 70.13                  & 22.98                  & -                        & -                         & -                      & -                      \\
             & Content Warning     & 75.04                  & 19.62                  & 4.47                     & 1.81                      & 441.49                 & 586.56                 \\
             & Removal and Replace & 74.37                  & 23.13                  & 5.20                     & 1.95                      & 337.16                 & 312.36                 \\
             & Downranking           & 73.25                  & 23.09                  & 4.80                     & 2.02                      & 488.52                 & 1106.00                \\
             & Engagement          & 78.68                  & 19.82                  & 4.53                     & 1.87                      & 400.24                 & 361.05                 \\
             & Ideology            & 77.46                  & 22.88                  & 4.76                     & 1.97                      & 375.73                 & 340.88                 \\
             & Chronological       & 73.63                  & 23.21                  & 4.71                     & 2.02                      & 445.22                 & 563.02                 \\
             & \textbf{Total}      & \textbf{74.67}         & \textbf{22.26}         & \textbf{4.74}            & \textbf{1.95}             & \textbf{414.51}        & \textbf{606.08}        \\ \bottomrule
\end{tabular}}
\vspace{0.05in}
\caption{Means and $SD$s of Main Outcome Variables}
\label{tab:mean}
\end{table}

\subsection{Prompts Used in Study 2}
\label{appendix:prompts}
Detailed below are the prompts provided to the large language model (GPT-4) for Study 2 and Study 3 for the eight anti-democratic attitude variables.

\subsubsection{V1.} Partisan Animosity

\begin{lstlisting}[language=Markdown]
Please rate the following message's partisan animosity from 1 to 3. Partisan animosity is defined as "dislike for opposing partisans". Your rating should consider whether the following factors exist in the following message:

A: Partisan name-calling
B: Emotion or exaggeration
    
Rate 1 if neither factor exists
Rate 2 if one of the factors exists
Rate 3 if both factors exist
    
After your rating, please provide reasoning in the following format: 
Rating: ### Reason: (### is the separator)
\end{lstlisting}

\subsubsection{V2.} Support for Undemocratic Practices
\begin{lstlisting}[language=Markdown]
Please rate the following message's support for undemocratic practices from 1 to 3. Support for undemocratic practices is defined as "willingness to forgo democratic principles for partisan gain". Undemocratic practices are undemocratic tendencies or actions such as reducing polling stations in areas that support their opponents, attacking the independence of the judiciary, undermining the free press, and challenging the legitimacy of election results, and encouraging political violence. Your rating should consider whether the following factors exist in the following message:

A: Show support for undemocratic practices
B1: Partisan name-calling
B2: Emotion or exaggeration

Rate 1 if doesn't satisfy any of the factors
Rate 2 if doesn't satisfy A, but satisfies B1 or B2
Rate 3 if satisfies A, B1 and B2

After your rating, please provide reasoning in the following format: 
Rating: ###  Reason: (###  is the separator)
\end{lstlisting}

\subsubsection{V3.} Support for Partisan Violence
\begin{lstlisting}[language=Markdown]
Please rate the following message's support for partisan violence from 1 to 3. Support for partisan violence is defined as "willingness to use violent tactics against outpartisans". Examples of partisan violence include sending threatening and intimidating messages to the opponent party, harassing the opponent party on the Internet, using violence in advancing their political goals or winning more races in the next election.
					
Your rating should consider whether the following factors exist in the following message:

A: Show support for partisan violence
B1: Partisan name-calling
B2: Emotion or exaggeration

Rate 1 if doesn't satisfy any of the factors
Rate 2 if doesn't satisfy A, but satisfies B1 or B2
Rate 3 if satisfies A, B1 and B2

After your rating, please provide reasoning in the following format: 
Rating: ###  Reason: (###  is the separator)
\end{lstlisting}

\subsubsection{V4.} Support for Undemocratic Candidates
\begin{lstlisting}[language=Markdown]
Please rate the following message's support for undemocratic candidates from 1 to 3. Support for undemocratic candidates is defined as "willingness to ignore undemocratic practices to elect inparty candidates". Undemocratic candidates are oftentimes those who support the following undemocratic practices such as reducing polling stations in areas that support their opponents, attacking the independence of the judiciary, undermining the free press, and challenging the legitimacy of election results, and encouraging political violence.

Your rating should consider whether the following factors exist in the following message:

A: Show support for undemocratic candidates
B1: Partisan name-calling
B2: Emotion or exaggeration

Rate 1 if doesn't satisfy any of the factors 
Rate 2 if satisfies A, but not B1 or B2
Rate 3 if satisfies A, B1 and B2

After your rating, please provide reasoning in the following format: 
Rating: ###  Reason: (###  is the separator)
\end{lstlisting}


\subsubsection{V5.} Opposition to Bipartisanship
\begin{lstlisting}[language=Markdown]
Please rate the following message's opposition to bipartisanship from 1 to 3. Opposition to bipartisanship is defined as "resistance to cross-partisan collaboration". Your rating should consider whether the following factors exist in the following message:
A: Any name-calling or terms that reduce trust
B: Emotion or exaggeration

Rate 1 if neither factor exists
Rate 2 if one of the factors exists
Rate 3 if both factors exist

After your rating, please provide reasoning in the following format: 
Rating: ###  Reason:(###  is the separator)
\end{lstlisting}

\subsubsection{V6.} Social Distrust
\begin{lstlisting}[language=Markdown]
Please rate the following message's social distrust from 1 to 3. Social distrust is defined as "distrust of people in general". Your rating should consider whether the following factors exist in the following message:
A: Any name-calling or terms that reduce trust
B: Emotion or exaggeration

Rate 1 if neither factor exists
Rate 2 if one of the factors exists
Rate 3 if both factors exist

After your rating, please provide reasoning in the following format: 
Rating:###  Reason: (###  is the separator)
\end{lstlisting}


\subsubsection{V7.} Social Distance
\begin{lstlisting}[language=Markdown]
Please rate the following message's social distance from 1 to 3. Social distance is defined as "resistance to interpersonal contact with outpartisans". Your rating should consider whether the following factors exist in the following message:

A: Any terms that increase distrust, distance, insecurity, hate, prejudice, or discrimination
B1: Emotion or exaggeration
B2: Any events that damages communities or decrease societal trust such as mass shooting 

Rate 1 if doesn't satisfy any of the factors 
Rate 2 if satisfies A, but not B1 or B2
Rate 3 if satisfies A, B1 and B2

After your rating, please provide reasoning in the following format: 
Rating:###  Reason: (###  is the separator)
\end{lstlisting}

\subsubsection{V8.} Biased Evaluation of Politicized Facts
\begin{lstlisting}[language=Markdown]
Please rate the following message's biased evaluation of politicized facts from 1 to 3. Biased evaluation of politicized facts is defined as "skepticism of facts that favor the worldview of the other party". Your rating should consider whether the following factors exist in the following message:

A: partially present political facts or discuss a controversial issue with a certain political stance
B: emotion/exaggeration

Rate 1 if neither factor exists
Rate 2 if one of the factors exists
Rate 3 if both factors exist

After your rating, please provide reasoning in the following format: 
Rating: ###  Reason:(###  is the separator)
\end{lstlisting}

\subsection{Additional Study 2 Results Using GPT-3.5}
\label{appendix:gpt-3.5}
Table~\ref{tab:gpt_3.5_allVars} reports Study 2 results for the GPT-3.5 model from OpenAI, the model variant preceding GPT-4, to explore how LLM-based ranking varies across models. The same prompts, dataset, and study procedures were used to generate these results; the only modification was the use of the \texttt{gpt-3.5-turbo} chat completion model to generate ratings.

\begin{table}[!t]
\centering
\resizebox{\textwidth}{!}{
%
\begin{tabular}{lccc}
\toprule
\textbf{Individual anti-democratic attitude variables}              & \textbf{Krippendorff's $\alpha$} & \textbf{Classification Accuracy} & \textbf{F1 Score} \\ 
\midrule
Partisan Animosity                     & .694                    & .865                    & .667                        \\
Support for Undemocratic Practices     & -.003                   & .815                    & .497                        \\
Support for Partisan Violence          & .356                    & .840                    & .677                        \\
Support for Undemocratic Candidates    & .039                    & .885                    & .373                       \\
Opposition to Bipartisanship           & .621                    & .850                    & .702                        \\
Social Distrust                        & .628                    & .790                    & .676                        \\
Social Distance                        & .637                    & .790                    & .643                       \\
Biased Evaluation of Politicized Facts & .737                    & .790                    & .690                       \\ 
\bottomrule
\\
\toprule
\rowcolor{purple}
\textbf{Outcome variable}              & \textbf{Krippendorff's $\alpha$} & \textbf{Spearman's $\rho$} & \textbf{Mean Absolute Error (MAE)} \\ 
\midrule
\rowcolor{purple}
\textbf{Overall democratic attitude ranking (8-24 scale)}                     & .761                    & .759     & 1.135            \\
\bottomrule
\end{tabular}
}
\vspace{0.05in}
\caption{Performance metrics for GPT-3.5 ratings on the overall democratic attitude ranking and individual anti-democratic attitude variables. Compared to the GPT-4 results, we observe comparable alignment between GPT-3.5 and manual rating results.
}
\label{tab:gpt_3.5_allVars}
\end{table}


% For camera-ready
% \begin{table}[!t]
% \centering
% \resizebox{\textwidth}{!}{
% %
% \begin{NiceTabular}{lccc}[colortbl-like]
% \toprule
% \textbf{Individual anti-democratic attitude variables}              & \textbf{Krippendorff's $\alpha$} & \textbf{Classification Accuracy} & \textbf{F1 Score} \\ 
% \midrule
% Partisan Animosity                     & .694                    & .865                    & .667                        \\
% Support for Undemocratic Practices     & -.003                   & .815                    & .497                        \\
% Support for Partisan Violence          & .356                    & .840                    & .677                        \\
% Support for Undemocratic Candidates    & .039                    & .885                    & .373                       \\
% Opposition to Bipartisanship           & .621                    & .850                    & .702                        \\
% Social Distrust                        & .628                    & .790                    & .676                        \\
% Social Distance                        & .637                    & .790                    & .643                       \\
% Biased Evaluation of Politicized Facts & .737                    & .790                    & .690                       \\ 
% \bottomrule
% \\
% \toprule
% \rowcolor{purple}
% \textbf{Outcome variable}              & \textbf{Krippendorff's $\alpha$} & \textbf{Spearman's $\rho$} & \textbf{Mean Absolute Error (MAE)} \\ 
% \midrule
% \rowcolor{purple}
% \textbf{Overall democratic attitude ranking (8-24 scale)}                     & .761                    & .759     & 1.135            \\
% \bottomrule
% \end{NiceTabular}
% }
% \vspace{0.05in}
% \caption{Performance metrics for GPT-3.5 ratings on the overall democratic attitude ranking and individual anti-democratic attitude variables. Compared to the GPT-4 results, we observe comparable alignment between GPT-3.5 and manual rating results.
% }
% \label{tab:gpt_3.5_allVars}
% \end{table}

