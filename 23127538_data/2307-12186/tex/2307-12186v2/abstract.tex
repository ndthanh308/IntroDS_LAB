\section*{ABSTRACT}

Computational models help decision makers understand epidemic dynamics to optimize  public health interventions. Agent-based simulation of disease spread in synthetic populations allows us  to compare and contrast different effects across identical populations or to investigate the effect of interventions keeping every other factor constant between ``digital twins''. FRED (A Framework for Reconstructing Epidemiological Dynamics) is an agent-based modeling system with a geo-spatial perspective using a synthetic population that is constructed based on the U.S. census data. In this paper, we show how Gaussian process regression can be used on FRED-synthesized data to infer the differing spatial dispersion of the epidemic dynamics for two disease conditions that start from the same initial conditions and spread among identical populations. Our results showcase the utility of agent-based simulation frameworks such as FRED for inferring differences between conditions where controlling for all confounding factors for such comparisons is next to impossible without synthetic data.