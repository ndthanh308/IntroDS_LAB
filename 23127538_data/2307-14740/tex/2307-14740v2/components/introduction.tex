\section{Introduction}

Electronic Design Automation (EDA) tools such as KiCad~\cite{kicad-official}, Cadence~\cite{cadence-official}, and Altium Designer~\cite{altium-official} provide powerful capabilities for schematic capture, PCB layout, and verification. Despite their robustness, these tools remain notoriously difficult to master due to steep learning curves, intricate user interfaces, and a lack of intuitive, task-driven guidance. For novice designers in particular, identifying the right features or plugins often requires sifting through fragmented documentation or community forums, which severely limits productivity and accessibility. Furthermore, these resources are often static, outdated, or written for expert audiences, offering little assistance when users encounter nuanced, context-specific challenges during design iterations. The lack of an adaptive, interactive support mechanism means that users must engage in time-consuming trial-and-error, leading to frustration and reduced design throughput.

Recent advances in large language models (LLMs) have transformed how users engage with complex software systems. Architecturally, LLMs are typically based on the Transformer framework~\cite{vaswani2017attention}, which enables attention-based modeling of long-range dependencies in text. Pretrained on large-scale corpora and fine-tuned for downstream tasks, modern LLMs such as GPT-4.1~\cite{gpt4-openai}, Claude 3 Opus~\cite{claude3-anthropic}, Gemini 2.5 Pro~\cite{gemini25-deepmind}, and open-source systems like Meta's LLaMA-3~\cite{llama3-grattafiori2024} and Alibaba's Qwen2.5~\cite{qwen25-alibaba} exhibit remarkable capabilities in few-shot learning, tool use, and contextual reasoning. These models maintain conversational state across multiple turns, support function calling, and exhibit emergent capabilities such as code generation, retrieval augmentation, and planning. As a result, they are increasingly being integrated into real-world workflows through APIs, plugins, and autonomous agents.

Complementing these advances, retrieval-augmented generation (RAG) techniques~\cite{rag-lewis2020} have become essential in improving LLM effectiveness in knowledge-intensive tasks. RAG pipelines dynamically retrieve relevant external content—such as documentation, plugin descriptions, or design tutorials—based on user queries, and condition the language model's output on this evidence. This hybrid framework helps mitigate context window limitations and ensures generated responses are grounded in factual, domain-specific knowledge. In the context of EDA software, such augmentation is particularly beneficial, as it enables accurate and context-aware guidance drawn directly from official manuals or plugin repositories.

LLM-based interfaces have demonstrated success in domains such as software development (e.g., Codex~\cite{codex-chen2021}), general AI orchestration (e.g., HuggingGPT~\cite{hugginggpt-shen2023}), and multimodal agents. However, their application in domain-specific engineering tools—such as EDA software—remains underexplored. Unlike general-purpose tasks, EDA workflows involve intricate procedural logic, structured design hierarchies, and tightly-coupled UI actions, which pose unique challenges for general LLM deployment.

In this work, we introduce \textbf{SmartonAI}, an AI-assisted interaction system for EDA software, initially integrated with KiCad~\cite{kicad-official}. SmartonAI leverages the natural language understanding and reasoning capabilities of LLMs to transform vague or high-level user requests into concrete design operations. The system is composed of two complementary components: (1) a \textbf{Chat Plugin}, which supports multi-turn dialogue for decomposing tasks and retrieving relevant documentation, and (2) a \textbf{OneCommandLine Plugin}, which enables intelligent plugin recommendation and parameterized execution based on user goals. Both components integrate retrieval-augmented generation~\cite{rag-lewis2020} to enable precise document grounding and plugin-specific assistance.

SmartonAI pioneers the application of LLMs and RAG pipelines in EDA workflows, bridging the gap between user intent and low-level design execution. By combining language-based guidance with context-aware automation, SmartonAI lowers the barrier to entry for PCB design, improves user efficiency, and offers a blueprint for extending AI-assisted interfaces to other complex, domain-specific software ecosystems.
