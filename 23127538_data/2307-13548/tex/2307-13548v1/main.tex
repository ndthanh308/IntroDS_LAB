%% This is file `sample-sigconf.tex',
%% generated with the docstrip utility.
%%
%% The original source files were:
%%
%%
%% samples.dtx  (with options: `sigconf')
%% 
%% IMPORTANT NOTICE:
%% 
%% For the copyright see the source file.
%% 
%% Any modified versions of this file must be renamed
%% with new filenames distinct from sample-sigconf.tex.
%% 
%% For distribution of the original source see the terms
%% for copying and modification in the file samples.dtx.
%% 
%% This generated file may be distributed as long as the
%% original source files, as listed above, are part of the
%% same distribution. (The sources need not necessarily be
%% in the same archive or directory.)
%%
%% Commands for TeXCount
%TC:macro \cite [option:text,text]
%TC:macro \citep [option:text,text]
%TC:macro \citet [option:text,text]
%TC:envir table 0 1
%TC:envir table* 0 1
%TC:envir tabular [ignore] word
%TC:envir displaymath 0 word
%TC:envir math 0 word
%TC:envir comment 0 0
%%
%%
%% The first command in your LaTeX source must be the \documentclass command.
\documentclass[sigconf,noacm]{acmart}
%\documentclass[sigconf,anonymous,review,balance=false]{acmart}
%\documentclass[sigconf, anonymous, review]{acmart}
%% NOTE that a single column version may be required for 
%% submission and peer review. This can be done by changing
%% the \doucmentclass[...]{acmart} in this template to 
%% \documentclass[manuscript,screen]{acmart}
%% 
%% To ensure 100% compatibility, please check the white list of
%% approved LaTeX packages to be used with the Master Article Template at
%% https://www.acm.org/publications/taps/whitelist-of-latex-packages 
%% before creating your document. The white list page provides 
%% information on how to submit additional LaTeX packages for 
%% review and adoption.
%% Fonts used in the template cannot be substituted; margin 
%% adjustments are not allowed.
%%
%%
%% \BibTeX command to typeset BibTeX logo in the docs
\usepackage[ruled]{algorithm2e}
\usepackage{multirow}
\usepackage{adjustbox}

\AtBeginDocument{%
  \providecommand\BibTeX{{%
    \normalfont B\kern-0.5em{\scshape i\kern-0.25em b}\kern-0.8em\TeX}}}

%% Rights management information.  This information is sent to you
%% when you complete the rights form.  These commands have SAMPLE
%% values in them; it is your responsibility as an author to replace
%% the commands and values with those provided to you when you
%% complete the rights form.
% \setcopyright{acmcopyright}
% \copyrightyear{2023}
% \acmYear{2023}
% \acmDOI{XXXXXXX.XXXXXXX}

%% These commands are for a PROCEEDINGS abstract or paper.
% \acmConference[Conference acronym 'XX]{Make sure to enter the correct
%   conference title from your rights confirmation emai}{June 03--05,
%   2018}{Woodstock, NY}
%
%  Uncomment \acmBooktitle if th title of the proceedings is different
%  from ``Proceedings of ...''!
%
%\acmBooktitle{Woodstock '18: ACM Symposium on Neural Gaze Detection,
%  June 03--05, 2018, Woodstock, NY} 
% \acmPrice{15.00}
% \acmISBN{978-1-4503-XXXX-X/18/06}


%%
%% Submission ID.
%% Use this when submitting an article to a sponsored event. You'll
%% receive a unique submission ID from the organizers
%% of the event, and this ID should be used as the parameter to this command.
%%\acmSubmissionID{123-A56-BU3}

%%
%% For managing citations, it is recommended to use bibliography
%% files in BibTeX format.
%%
%% You can then either use BibTeX with the ACM-Reference-Format style,
%% or BibLaTeX with the acmnumeric or acmauthoryear sytles, that include
%% support for advanced citation of software artefact from the
%% biblatex-software package, also separately available on CTAN.
%%
%% Look at the sample-*-biblatex.tex files for templates showcasing
%% the biblatex styles.
%%

%%
%% The majority of ACM publications use numbered citations and
%% references.  The command \citestyle{authoryear} switches to the
%% "author year" style.
%%
%% If you are preparing content for an event
%% sponsored by ACM SIGGRAPH, you must use the "author year" style of
%% citations and references.
%% Uncommenting
%% the next command will enable that style.
%%\citestyle{acmauthoryear}

%%
%% end of the preamble, start of the body of the document source.
\usepackage[utf8]{inputenc}
%\usepackage{amsmath}
%\usepackage{amssymb}
\usepackage{graphicx}
%\usepackage{biblatex} %Imports biblatex package
%\addbibresource{biblio.bib} %Import the bibliography file
\usepackage{caption}
\usepackage{subcaption}
\usepackage{xcolor}
\usepackage{algpseudocode}
%\usepackage{algorithmic}
%\usepackage{algorithm}
%\usepackage{comment}
%\usepackage{authblk}
%\usepackage{graphbox,graphicx}
\usepackage{float}

\DeclareMathOperator{\oP}{P}   

\newcommand{\melek}[1]{\textcolor{violet}{#1}}
\newcommand{\ayse}[1]{\textcolor{orange}{#1}}
\newcommand{\oualid}[1]{\textcolor{blue}{#1}}
\newcommand{\javi}[1]{\textcolor{brown}{#1}}

%%%%%%%%%%%%%%%%%%%%%%%%%%%%%%%%
%%%%%% SPECIFIC COMMANDS %%%%%%%
%%%%%%%%%%%%%%%%%%%%%%%%%%%%%%%%

\newcommand{\M}{\mathcal{M}}
\newcommand{\D}{\mathbb{D}}
\renewcommand{\S}{\mathcal{R}}
\newcommand{\overS}{\overline{\S}}
\newcommand{\X}{\mathcal{X}}
\newcommand{\NX}{\N^{|\X|}}
\newcommand{\DX}{\D_\X}
\newcommand{\DP}[2]{{#1}\text{-DP}_{#2}}
\DeclareMathOperator{\Range}{Range}
\DeclareMathOperator{\dist}{dist}
\DeclareMathOperator{\diam}{diam}
\DeclareMathOperator{\V}{V}
\DeclareMathOperator{\Prob}{P}
\DeclareMathOperator*{\argmax}{arg\,max}


%\settopmatter{printacmref=false}
\settopmatter{printacmref=false}
% remove copyright note
\renewcommand\footnotetextcopyrightpermission[1]{}
\pagestyle{plain}
%Copyright removal
 \makeatletter
 \def\runningfoot{\def\@runningfoot{}}
 \def\firstfoot{\def\@firstfoot{}}
 \makeatother 
 \makeatletter
 \renewcommand\@formatdoi[1]{\ignorespaces}
 \makeatother

% \copyrightyear{}
% \acmYear{}
% \acmConference[]{}{}{}
% \acmBooktitle{}
% \acmDOI{}
% \acmISBN{}

\copyrightyear{}
\acmYear{2023}
\acmConference[ACM CCS]{ACM CCS}{2023}{Denmark}
\acmBooktitle{Anonymous Conference}
\acmDOI{}
\acmISBN{}

\begin{document}

%%
%% The "title" command has an optional parameter,
%% allowing the author to define a "short title" to be used in page headers.
\title{Node Injection Link Stealing Attack}

%%
%% The "author" command and its associated commands are used to define
%% the authors and their affiliations.
%% Of note is the shared affiliation of the first two authors, and the
%% "authornote" and "authornotemark" commands
%% used to denote shared contribution to the research.
\author{Oualid Zari}
\email{oualid.zari@eurecom.fr}
%\orcid{1234-5678-9012}
\affiliation{%
  \institution{Eurecom}
  \city{Biot}
  \country{France}
}

\author{Javier Parra-Arnau}
\email{javi.parra-arnau@kit.edu}

\affiliation{%
  \institution{Karlsruhe Institute of Technology}
  \city{Karlsruhe}
  \country{Germany}
}
\affiliation{%
  \institution{Universitat Politecnica de Catalunya}
  \city{Barcelona}
  \country{Spain}
}


\author{Ayşe Ünsal}
\email{ayse.unsal@eurecom.fr}
%\orcid{1234-5678-9012}
\affiliation{%
  \institution{Eurecom}
  \city{Biot}
  \country{France}
}
\author{Melek Önen}
\email{melek.onen@eurecom.fr}
%\orcid{1234-5678-9012}
\affiliation{%
  \institution{EURECOM}
  \city{Sophia-Antipolis}
  \country{France}
}


%%
%% By default, the full list of authors will be used in the page
%% headers. Often, this list is too long, and will overlap
%% other information printed in the page headers. This command allows
%% the author to define a more concise list
%% of authors' names for this purpose.

%\renewcommand{\shortauthors}{Trovato and Tobin, et al.}



%%
%% The abstract is a short summary of the work to be presented in the
%% article.
\begin{abstract}
%\newcommand{\ayse}[1]{\textcolor{orange}{#1}}
In this paper, we present a stealthy and effective attack that exposes privacy vulnerabilities in Graph Neural Networks (GNNs) by inferring private links within graph-structured data. Focusing on the inductive setting where new nodes join the graph and an API is used to query predictions, we investigate the potential leakage of private edge information. We also propose methods to preserve privacy while maintaining model utility. Our attack demonstrates superior performance in inferring the links compared to the state of the art. Furthermore, we examine the application of differential privacy (DP) mechanisms to mitigate the impact of our proposed attack, we analyze the trade-off between privacy preservation and model utility. Our work highlights the privacy vulnerabilities inherent in GNNs, underscoring the importance of developing robust privacy-preserving mechanisms for their application.
\end{abstract}

%%
%% The code below is generated by the tool at http://dl.acm.org/ccs.cfm.
%% Please copy and paste the code instead of the example below.
%%
% \begin{CCSXML}
% <ccs2012>
%  <concept>
%   <concept_id>10010520.10010553.10010562</concept_id>
%   <concept_desc>Computer systems organization~Embedded systems</concept_desc>
%   <concept_significance>500</concept_significance>
%  </concept>
%  <concept>
%   <concept_id>10010520.10010575.10010755</concept_id>
%   <concept_desc>Computer systems organization~Redundancy</concept_desc>
%   <concept_significance>300</concept_significance>
%  </concept>
%  <concept>
%   <concept_id>10010520.10010553.10010554</concept_id>
%   <concept_desc>Computer systems organization~Robotics</concept_desc>
%   <concept_significance>100</concept_significance>
%  </concept>
%  <concept>
%   <concept_id>10003033.10003083.10003095</concept_id>
%   <concept_desc>Networks~Network reliability</concept_desc>
%   <concept_significance>100</concept_significance>
%  </concept>
% </ccs2012>
% \end{CCSXML}

% \ccsdesc[500]{Computer systems organization~Embedded systems}
% \ccsdesc[300]{Computer systems organization~Redundancy}
% \ccsdesc{Computer systems organization~Robotics}
% \ccsdesc[100]{Networks~Network reliability}

%%
%% Keywords. The author(s) should pick words that accurately describe
%% the work being presented. Separate the keywords with commas.
\keywords{graph neural networks, privacy attacks, link inference, link stealing, differential privacy}
%\melek{Melek}, \ayse{Ayse}, \oualid{Oualid}, \javi{Javi}.


%% A "teaser" image appears between the author and affiliation
%% information and the body of the document, and typically spans the
%% page.


% \received{20 February 2007}
% \received[revised]{12 March 2009}
% \received[accepted]{5 June 2009}

%%
%% This command processes the author and affiliation and title
%% information and builds the first part of the formatted document.
%\settopmatter{printacmref=false, printcopyright=false}

\maketitle
\pagestyle{empty}

\section{Introduction}
Current quantum hardware is unable to carry out universal quantum computations due to the buildup of errors that occur during the computation. 
The magnitude of the individual error is currently above the value that the Threshold Theorem requires in order to kick-start quantum error correction and fault-tolerant quantum computation~\cite[Section 10.6]{nielsen_chuang_2010}. 
Although the experimentally achieved fidelity rates are promising and the error bounds are inching closer to the required threshold, we will have to work for the foreseeable future with quantum hardware with errors that build-up during the computation.  This implies that we can only do a limited number of steps before the output of the computation has become completely uncorrelated with the intended one.

For fault-tolerant quantum computing, we repeat four steps: 
1) We apply a number of single and two-qubit quantum gates, in parallel whenever possible; 
2) We perform a syndrome measurement on a subset of the qubits; 
3) We perform fast classical computations to determine which errors have occurred and how to correct them; 
and, 4) We apply correction terms based on the classical computations.
We then repeat these four steps with a next sequence of gates. 
These four steps are essential to fault-tolerant quantum computing. 


The starting point of this work is to use the four steps outlined above, not to carry out error correction and fault-tolerant computation, but to enhance short, constant-depth, {\em uncorrected} quantum circuits that perform single qubit gates and {\em nearest-neighbor} two qubit gates. 
Since in the long run we will have to implement error-correction and fault-tolerant computation anyhow, and this is done by such a four-step process, why not make other use of this architecture? Moreover, on some of the quantum hardware platforms, these operations are already in place.
Embracing this idea we naturally arrive at the question: what is the computational power of \textit{low-depth} quantum-classical circuits organized as in the four steps outlined above? 
We thus investigate circuits that execute a small, ideally constant, number of stages, where at each stage we may apply, in parallel, single qubit gates and {\em nearest-neighbor} two qubit gates, followed by measurements, followed by low-depth classical computations of which the outcome can control quantum gates in later stages. 
It is not clear, at first, whether such circuits, especially with constant depth, can do anything remotely useful. 
But we will see that this is indeed the case: many quantum computations can be done by such circuits in constant depth. 
By parallelizing quantum computations in this way, we improve the overall computational capabilities of these circuits, as we do not incur errors on qubits that are idle, simply because qubits are not idle for a very long time. 
Furthermore, reducing the depth of quantum circuits, at the cost of increasing width, allows the circuit to be run faster even if errors occur.

The first usage of such a four-step layout, not to do error correction, but to perform computations, can be found in the paradigm of measurement-based quantum computing~\cite{gottesman1999demonstrating,raussendorf2001one,jozsa2006introduction,clark2007generalised}: 
A universal form of quantum computing where a quantum state is prepared and operations are performed by measuring qubits in different bases, depending on previous measurements and intermediate measurements.

\citeauthor{PhamSvore2013} were the first to formalize the four-step protocol for performing computations~\cite{PhamSvore2013}. They included specific hardware topologies by considering two-dimensional graphs for imposing constraints on qubit interactions. In their model, they develop circuits for particularly useful multi-qubit gates, including specifying costs in the width, number of qubits, depth, number of concurrent time steps, size, and total number of non-Identity operations.
As a result, they find an algorithm that factors integers in polylogarithmic depth.
\citeauthor{Browne:2011} showed that the main tool in the work by \citeauthor{PhamSvore2013}, the fan-out gate, can also be replaced by additional log-depth classical computations in the measurement-based quantum computing setting~\cite{Browne:2011}.

More recently, \citeauthor{Cirac:2021} introduced a scheme to implement unitary operations involving quantum circuits combined with Local Operations and Classical Communication ($\mathsf{LOCC}$) channels: $\mathsf{LOCC}$-assisted quantum circuits~\cite{Cirac:2021}. Similarly to the four-step scheme we just described, they allow for a short depth circuit to be run on the qubits, followed by one round of $\mathsf{LOCC}$, in which ancilla qubits are measured and local unitaries are applied based on the measurement outcomes. They show that in this model any 1D transitionally invariant matrix-product state (MPS) with fixed bond dimension is in the same phase of matter as the trivial state. Similar ideas can be found in~\cite{TVV_NonAbelianTopologicalOrder_2022, tantivasadakarn2021long}.

In this work, we introduce a new model, called \textit{Local Alternating Quantum-Classical Computations} ($\LAQCC$). In this model we alternate between running quantum circuits (constrained by locality), ending in the measurement of a subset of qubits, and fast classical computations based on the measurement results. The outcome of the classical computations are then used to control future quantum circuits. We allow for flexibility in this model, by giving different constraints to the power of both the quantum circuits and the classical circuits as well as the number of alternations between them. 
Most attention will be given to $\LAQCC$ containing quantum circuits of constant depth, classical circuits of logarithmic depth and at most a constant number of alternations between them. 
Any circuit constructed in this model is considered to be of constant depth. 
We restrict ourselves to logarithmic depth classical computations, as this is the first natural and non-trivial extension beyond constant-depth classical computations. 
Constant-depth classical computations do however also have an equivalent constant-depth quantum implementation.

The definition of $\LAQCC$ sharpens the original definition of \citeauthor{PhamSvore2013} by adding constraints to the intermediate classical computations. This allows us to bound the power of $\LAQCC$ from above. 

The main result of \citeauthor{Cirac:2021}, that 1D translational invariant MPS with fixed bond dimension can be prepared by $\mathsf{LOCC}$-assisted circuits, relies on local symmetries of the MPS. These symmetries allow them to prepare local states (on a constant number of qubits) and glue them together by doing one round of the appropriate entangling measurement and corrections, after which they run a round of local unitaries to get the desired result. This general scheme for preparing states that exhibit an MPS description with the appropriate local symmetries requires only geometrically local unitaries and one round of measurement and corrections an therefore is accessible in $\LAQCC$. Studying different local symmetries, known as Symmetry Protected Topological (SPT) phases of matter, to find measurement-based constant depth circuits for states is a broad ongoing field of research~\cite{TVV_NonAbelianTopologicalOrder_2022, tantivasadakarn2021long, smith2023deterministic}. 
All these schemes have a $\LAQCC$ implementation.

%$\LAQCC$-circuits also exist for general schemes of preparing local states, based on the local tensors, and gluing them together using one round of entangled measurement and corrections, based on the local symmetry. 
%The main result of \citeauthor{Cirac:2021}, that 1D translational invariant MPS with fixed bond dimension can be prepared by $\mathsf{LOCC}$-assisted circuits, relies heavily on local symmetries of the MPS and as a result also has an equivalent $\LAQCC$ implementation. 
%The corrections applied after the measurement round are local unitaries depending on the local symmetries of the MPS. 

 

%This general scheme of preparing local states, based on the local tensors, and gluing it together by doing one round of entangled measurement and corrections, based on the local symmetry, is accessible in $\LAQCC$.
Note however that \citeauthor{Cirac:2021} also suggest a circuit for the $W$-state.
This circuit uses sequentially and dependent measurement-based corrections of the ancilla qubits. 
These dependent measurements translate to sequential alternations between the quantum and classical circuits and therefore increase the total depth to linear depth, exceeding the constant-depth constraints imposed by $\LAQCC$-circuits. 

We study the power of the $\LAQCC$ model with respect to state preparation, showing that even with only constant quantum-depth and logarithmic classical depth it remains possible to prepare states with long-range entanglement.
Another surprising result is that it is unlikely that $\LAQCC$ circuits are classically simulatable. We show that any instantaneous quantum polynomial-time (IQP) circuit~\cite{Bremner2010,Shepherd2009} has an $\LAQCC$ implementation.
Classical simulation of IQP circuits implies the collapse of the polynomial hierarchy to the third level, which is not believed to be true~\cite{Bremner2017}. Therefore, we expect that $\LAQCC$ circuits are unlikely to be classically simulatable. We bound the power of $\LAQCC$ by showing that it is contained in $\QNC^1$, the class of polynomial-size, log-depth circuits.

Next, we also study the power that intermediate classical calculations can add to quantum computations, by considering a new model that alternates between polynomially many polynomial-depth quantum circuits and unbounded classical computations
We study this model by doing a complexity theoretical analysis, where we draw inspiration from the notions of complexity given by \citeauthor{RosenthalYuen:2022}, \citeauthor{MetgerYuen:2023}, and \citeauthor{Aaronson:2004}.
All three complexity notions are based on the notion of state preparation, instead of more traditional definition of complexity such as the decidability of a computational problem. 
The first two consider classes based on sequences of quantum states preparable by a polynomial-sized quantum circuit, where the circuits are uniformly generated by a computational class, for instance, the class $\mathsf{PSPACE}$, which results in the complexity class $\mathsf{StatePSPACE}$~\cite{RosenthalYuen:2022,MetgerYuen:2023}.
The third notion considers a relative complexity, where the complexity is measured between two given states, and is measured by the number of gates, from a given gate-set, required to transform one state in another state~\cite{Aaronson:2004}. 
For our definition of state preparation complexity, we drop the uniformity constraint from~\cite{RosenthalYuen:2022,MetgerYuen:2023} and define a class as $\mathsf{StateX}$, which refers to states preparable by circuits of type $\mathsf{X}$. 
As an example, if $\mathsf{X} = \QNC^0$, this results in the class $\mathsf{StateQNC^0}$, which is the set of states preparable from the $\ket{0}^n$ state by poly-size constant-depth circuits. 
This notion is similar to the relative complexity from~\cite{Aaronson:2004}, where one state is the  $\ket{0}^n$ state and instead of counting the number of gates we consider the set of states preparable by a fixed number of gates. Using this notion of complexity we show that any state preparable by an $\LAQCC^*$ circuit is also preparable by a $\mathsf{PostQPoly}$ circuit, the class of circuits of polynomial depth with an additional post-selection gate. 

All Clifford circuits have a constant-depth $\LAQCC$ implementation, implying that any stabilizer state can be implemented by a constant-depth $\LAQCC$ circuit, see Section~\ref{sec:clifford_circuits} for a proof of this statement. 
Efficient circuits for stabilizer states have been known already through measurement-based quantum computing. Therefore this paper focuses on the preparation of non-stabilizer states, and as a surprising result we find novel constant-depth protocols for four very natural classes of non-stabilizer states.
Despite the extensive research into these four classes of non-stabilizer states and the many applications of them, no efficient constant- or low-depth state preparation protocols are known yet. We specifically consider these four classes as they are all often used as initial states in other algorithms.

The first state is a uniform superposition over an arbitrary number of states. 
This state finds applications in many quantum algorithms, as they often start with a uniform superposition over multiple states. 
This superposition is often achieved by applying Hadamard gates to every qubit due to its simplicity to prepare. 
Yet, the analysis of many algorithms, such as Shor's algorithm~\cite{Shor:1997}, would benefit from a different initial superposition. 
The circuit to prepare the uniform superposition over an arbitrary number of states uses an exact version of Grover search as a subroutine, that turns a probabilistic circuit, with a known constant probability of success, into a deterministic circuit. 
We use the circuit for preparing a uniform superposition over an arbitrary number of states as a subroutine in the next two quantum state preparation protocols. 

The second state is the $W$-state, the uniform superposition over all computational basis states of Hamming-weight~$1$, a natural long-ranged entangled state that displays a fundamentally nonequivalent type of entanglement from the Greenberger–Horne–Zeilinger state~\cite{WState:2000}, for which $\LAQCC$-type constant-depth circuits were previously known~\cite{PhamSvore2013, Cirac:2021}. 
The $W$-state is often used as benchmark for new quantum hardware~\cite{Haffner2005,Neeley2010,GarciaPerez:2021}. 
A novel way to prepare the $W$-state therefore gives a new way to benchmark different quantum devices with each other. 
A circuit for preparing the $W$-state was given in~\cite{Cirac:2021}, but this implementation requires sequentially alternating measurements followed by local unitaries, which in the $\LAQCC$ model is not considered to be of constant depth. 
We improve this protocol by giving an $\LAQCC$ implementation of the $W$-state, based on a compress-uncompress method that links the one-hot and binary encoding of integers.

The third state considered is the Dicke state, a generalization of the $W$-state, a superposition over all computational basis states with Hamming-weight $k$~\cite{Dicke:1954}. 
Dicke states have relevance in various practical settings.
For instance, for quantum game theory~\cite{zdemir2007}, quantum storage~\cite{Bacon_Compress:2006,Plesch:2010}, quantum error correction~\cite{ouyang2014permutation}, quantum metrology~\cite{toth2012multipartite}, and quantum networking~\cite{prevedel2009experimental}. 
Dicke states have been used as a starting state for variational optimization algorithms, most notably Quantum Alternating Operator Ansatz (QAOA)~\cite{Hadfield2019}, to find solutions to problems such as Maximum k-vertex Cover~\cite{Brandhofer2022,cook2020quantum}.
The ground states of physical Hamiltonians describing one-dimensional chains tend to show a resemblance to Dicke states such as states resulting from the Bethe ansatz, making them an ideal starting state when investigating the ground state behavior of these Hamiltonians~\cite{TDL_BetheAnsatzDerivation:2010,B_ExcitedStateQuantumPhaseTransitions:2013,DickeTransitions:2021}. 
For instance, the algorithm by \citeauthor{van2021preparing}, who give an algorithm to prepare the Bethe ansatz eigenstates of the spin-1/2 XXZ spin chain, starts by first preparing a Dicke state~\cite{van2021preparing}. 
A Dicke-state preparation protocol based on the compress-uncompress methodology used in the $W$-state furthermore finds applications in entanglement distillation, where the entanglement of a large state is concentrated on only a few qubits. 
Efficient deterministic circuits for preparing Dicke states have been proposed by \citeauthor{bartschi2019deterministic}~\cite{bartschi2019deterministic, bartschi2022deterministic_short_depth}. 
They provide a quantum circuit of depth $\mathO(k \log(\frac{n}{k}))$, allowing arbitrary connectivity, to prepare a Dicke state, which they conjecture to be optimal when $k$ is constant. 
In this work, we provide a constant-depth $\LAQCC$ circuit below their conjectured bound already for constant $k$. 
However, this does not directly disprove their conjecture, as we allow for intermediate measurements and classical computations. 
More significantly, we even construct constant-depth $\LAQCC$ circuits for $k = \mathO(\sqrt{n})$ greatly improving their bound.
This construction extends the compress-uncompress method for the $W$-state combined with additional subroutines. 

We continue with a log-depth state preparation protocol for the Dicke-state for arbitrary $k$. 
This protocol implements an efficient transformation between the factoradic number representation and the combinatorial number representation of a positive integer. 
The combinatorial number representation relates directly to the Dicke state. 
The provided efficient transformation between number representation systems might be of independent interest. 

We conclude by modifying our protocol for preparing a Dicke-state to a protocol that prepares quantum many-body scar states in constant-depth. 
These states have low entanglement and longer coherence times than states with similar energy density.
These characteristics make many-body scar states interesting to analyze and relevant within physics.
Many-body scar states appear for instance in the AKLT model~\cite{AKLT:1987,MRBAR:2018,MRB:2018} and different spin models~\cite{SI:2019,MOBFR:2020}.
Known methods for preparing these states have polynomial-depth~\cite{Gustafson:2023}, whereas our circuit has constant depth. 

% We conclude by studying the power that intermediate classical calculations can add to quantum computations. 
% In this study, we define a new model that relaxes constant-depth quantum circuits to polynomial depth quantum circuits, log-depth classical calculations to unbounded classical computations and a constant number of alternations to a polynomial number of alternations. 
% We call this model $\LAQCC^*$. 
% We study this model by doing a complexity theoretical analysis, where we draw inspiration from the notions of complexity given by \citeauthor{RosenthalYuen:2022}, \citeauthor{MetgerYuen:2023}, and \citeauthor{Aaronson:2004}.
% All three complexity notions are based on the notion of state preparation, instead of more traditional definition of complexity such as the decidability of a computational problem. 
% The first two consider classes based on sequences of quantum states preparable by a polynomial-sized quantum circuit, where the circuits are uniformly generated by a computational class, for instance, the class $\mathsf{PSPACE}$, which results in the complexity class $\mathsf{StatePSPACE}$~\cite{RosenthalYuen:2022,MetgerYuen:2023}.
% The third notion considers a relative complexity, where the complexity is measured between two given states, and is measured by the number of gates, from a given gate-set, required to transform one state in another state~\cite{Aaronson:2004}. 
% For our definition of state preparation complexity, we drop the uniformity constraint from~\cite{RosenthalYuen:2022,MetgerYuen:2023} and define a class as $\mathsf{StateX}$, which refers to states preparable by circuits of type $\mathsf{X}$. 
% As an example, if $\mathsf{X} = \QNC^0$, this results in the class $\mathsf{StateQNC^0}$, which is the set of states preparable from the $\ket{0}^n$ state by poly-size constant-depth circuits. 
% This notion is similar to the relative complexity from~\cite{Aaronson:2004}, where one state is the  $\ket{0}^n$ state and instead of counting the number of gates we consider the set of states preparable by a fixed number of gates. Using this notion of complexity we show that any state preparable by an $\LAQCC^*$ circuit is also preparable by a $\mathsf{PostQPoly}$ circuit, the class of circuits of polynomial depth with an additional post-selection gate. 

\paragraph{Summary of results}
\begin{itemize}
    \item We give a new definition of a computational model that captures the power of the four step process: applying a constant number of layers of one- and two-qubit gates; performing a syndrome measurement; perform a fast classical computation determining corrections; apply corrections. We call this model \emph{Local Alternating Quantum Classical Computations}, or $\LAQCC$ for short. In this model we bound the allowed quantum operations, intermediate classical calculations, and number of rounds separately. In Section~\ref{sec:LAQCC_model} we define this model and give a list of operations based on results from literature contained in this computational model. In some of these operations we explicitly use that we allow for multiple, but at most constant, rounds  of corrections.
    \item  We show show that there exist $\LAQCC$ circuits that can not be weakly simulated in Section~\ref{sec:IQP_in_LAQCC}. We further show that for every $\LAQCC$ circuit there exists a $\QNC^1$ circuit simulating it perfectly, in Section~\ref{sec:LAQCC_in_QNC1}.
    \item We introduce a new type computational complexity for preparing states and show that the extension of $\LAQCC$ where we allow a polynomial number of rounds and unbounded classical computation, is contained in $\mathsf{PostQPoly}$, the class of polynomial circuits with post-selection, in Section~\ref{sec:Complexity results}.
    \item We show a protocol to prepare the uniform superposition state of size $q$ in $\LAQCC$ using $\mathO(\ceil{\log_2(q)}^2)$ qubits in Section~\ref{sec:superposition_modulo_q}. 
    \item We show a protocol to prepare the $W_n$ state in $\LAQCC$ using $\mathO(n\log(n))$ qubits in Section~\ref{sec:W_state_in_LAQCC}.
    \item We show two ways of preparing the Dicke-$(n,k)$ state. The first method is in $\LAQCC$, works up to $k = \mathO(\sqrt{n})$, uses $\mathO(n^2\log(n))$ qubits, and is found in Section~\ref{sec:dicke:small_k}. The second method is in $\LAQCC\text{-}\mathsf{LOG}$ (an extension of $\LAQCC$ allowing for logarithmic number of alterations instead of constant), works for any $k$, uses $\mathO(\text{poly}(n))$ qubits, and is found in Section~\ref{sec:Dicke_in_LAQCC_LOG}. 
    \item We extend on our $\LAQCC$ method of generating Dicke-$(n,k)$ states for $k = \mathO(\sqrt{n})$ and show a protocol to generate many-body scar states for a particular Hamiltonian in $\LAQCC$ (Section~\ref{sec:many_body_scar}). 
\end{itemize}
Summarized in a table, we provide the following state generation protocols:
\begin{table}[htb]
\centering
\begin{tabular}{l|l|l|l}
\textbf{State description} & \textbf{Width} & \textbf{Depth} & \textbf{Implementation}\\
\hline 
Uniform superposition mod $q$: $\frac{1}{\sqrt{q}} \sum_{i = 0}^{q-1}\ket{i}$ & $\mathO(\ceil{\log^2 q})$ & $\mathO(1)$ & Section~\ref{sec:superposition_modulo_q}\\

$W$-state: $\frac{1}{\sqrt{n}}\sum_{i = 0}^{n-1}\ket{e_i}$ & $\mathO(n \log n)$ & $\mathO(1)$ & Section~\ref{sec:W_state_in_LAQCC}\\

Dicke-$(n,k)$, $k = \mathO(\sqrt{n})$: $\binom{n}{k}^{-1/2}\sum_{x \in \{0,1\}^n: |x| = k} \ket{x}$ &  $\mathO(n^2\log n)$ & $\mathO(1)$ 
&Section~\ref{sec:dicke:small_k}\\

Dicke-$(n,k)$: $\binom{n}{k}^{-1/2}\sum_{x \in \{0,1\}^n: |x| = k} \ket{x}$ & $\mathO(\text{poly}(n))$ & $\mathO(\log n)$ &Section~\ref{sec:Dicke_in_LAQCC_LOG}\\

QMBS: $\ket{S_k} = \frac{1}{k! \sqrt{\mathcal N(n,k)}}(Q^\dagger)^k \ket{\Omega}$ &  $\mathO(n^2\log n)$ & $\mathO(1)$  &  Section~\ref{sec:many_body_scar}
\end{tabular}
\caption{Summary of state preparation protocols given in this paper.}
\label{tab:sate_prep}
\end{table}
In the entry for the quantum many-body scar state $Q$ denotes the raising operator and $\mathcal N(n,k)=\binom{n-k-1}{k}$. 
Section~\ref{sec:many_body_scar} will provide more details on the variables and the implementation. 

\paragraph{Organization of the paper}
\noindent We first introduce relevant preliminaries in Section~\ref{sec:preliminaries}. 
In Section~\ref{sec:LAQCC_model} we formally define the class of Local Alternating Quantum-Classical Computations ($\LAQCC$). We also show that any Clifford circuit can be implemented in constant depth $\LAQCC$ (a result based on a result from measurement-based quantum computing~\cite{jozsa2006introduction}). 
This result allows us to give many useful multi-qubit gates and routines in Section~\ref{sec:gates_created_in_LAQCC}. 
Beyond that we show that constant depth $\LAQCC$ circuits are contained in $\QNC^1$ and that any $\mathsf{IQP}$ circuit has an $\LAQCC$ implementation.
We conclude this section with an analysis of a more powerful instantiation of $\LAQCC$ and show an inclusion with respect to the class $\mathsf{PostQPoly}$, which is the class of circuits of polynomial depth with one additional post-selection gate. 
In Section~\ref{sec:state_prep_in_LAQCC} we give $\LAQCC$ circuit implementations for preparing the uniform superposition over an arbitrary number of states, the $W$-state and the Dicke state up to $k = \mathO(\sqrt{n})$. We furthermore give a log-depth circuit implementation for preparing the Dicke state for any $k$. We conclude by showing a $\LAQCC$ circuit for generating many body scar states of a particular type of Hamiltonian.



\vspacebeforesection
\section{Background}
\label{sec:background}

In this section, we provide the necessary background information to ensure a comprehensive understanding of the attack described in this paper. We start with a description of the Distributed Hash Table (DHT) used by IPFS, followed by its content resolution mechanisms. We also detail techniques for network size estimation, necessary for our attack detection and mitigation mechanisms.

\vspacebeforesection
\subsection{IPFS DHT}
\label{sec:kad_dht}

We review the features of the Kademlia DHT~\cite{maymounkov2002kademlia} and its \texttt{libp2p} implementation~\cite{libp2p_github} that are the most relevant to our attack.
To participate in the DHT, each peer generates a public/private key pair and derives an identity $\peerid \in \{0,1\}^{256}$ as the hash of its public key.
Ideally, each peer generates a random key pair and, therefore, peer IDs are distributed uniformly and independently over the space $\{0,1\}^{256}$.
While honest nodes follow this rule, malicious nodes may generate and choose from an arbitrary number of key pairs.
Each peer maintains a routing table consisting of $m=256$ buckets.
The $i$-th bucket contains the addresses of up to $k=20$ peers whose peer IDs share a common prefix of exactly $i$ bits with the peer's own peer ID. 

%
A new participant node joins the IPFS network by contacting one of the hardcoded bootstrap nodes. This bootstrap node provides the new node with some initial peers allowing it to join the DHT. The new node uses this information to perform a walk through the DHT towards its own peer ID.
The walk allows to: \textit{(i)}~make sure that there is no other node in the network with the same ID; \textit{(ii)}~discover new peers and fill the newcomer's DHT routing table. At the same time, the newcomer establishes \bitswap~\cite{de2021accelerating} connections to a subset of encountered peers (usually around 300 of them). The core role of the \bitswap protocol is to enable bilateral content transfer and to play the role of a cache for recently-accessed content.

The main DHT operation $\Call{GetClosestPeers}{\key}$ returns the $k=20$ closest peers to $\key$. 
%
In Kademlia, the distance between two keys $x$ and $y$ in the key space is given by $x \oplus y \in \{0,...,2^{256}-1\}$, where $\oplus$ denotes the bitwise XOR operation on the keys; the resulting binary string is interpreted as an integer.
%
When a client wants to find the peers with IDs closest to $\key$, it sends a request to the $\alpha=3$ peers in its routing table whose peer IDs are closest to $\key$. Each of these peers returns the $k$ closest peers to $\key$ in its own routing table and the addresses of these peers. 
%
The client again sends a request to the $\alpha$ peers closest to $\key$, among peers in its routing table and those whose addresses it just received. This process repeats until the client does not find any more peers closer to $\key$.
Due to network churn and imperfect routing tables, we observed in our experiments that successive calls to $\Call{GetClosestPeers}{\key}$ do not always return the same set of $k=20$ peers (we provide more details in \Cref{sec:evaluation}, \Cref{fig:20closest}). This is an important limitation affecting our attack.

\vspacebeforesection
\subsection{Content Resolution in IPFS}
\label{sec:ipfs}

IPFS is a content-centric network.
It allows its participant to request files without specifying their location. 
%
Content is indexed by content IDs $\cid \in \{0,1\}^{256}$ that are derived from a hash of that content.
Both peer IDs and CIDs are used as keys in the DHT.
Each node can play the role of a \provider, \downloader, or \resolver. 
The process of content advertisement and resolution is illustrated in \Cref{fig:add_get_provider}.

%
When a \provider wishes to publish content with a given $\cid$ on IPFS, it creates a \emph{provider record} that contains $cid$ and the \provider's address.
During a $\Call{Provide}{\cid}$ operation, the \provider first uses $\Call{GetClosestPeers}{\cid}$ to locate the $k=20$ peers with their peer IDs closest to $\cid$, 
%
and then sends them a $\mathsf{PutProvider}$ message including the provider record (\Cref{fig:add_get_provider}(a)).
We call the peers that hold provider records for $\cid$ the \emph{resolvers} for $\cid$.

Each CID can have several \providers. In fact, by default, each IPFS client becomes a provider for each piece of content it downloads for a fixed amount of time (12h, 24h, or 48h depending on the client version or custom configuration). As a result, the system provides an auto-scaling feature with supply automatically rising with demand.

%
When a \downloader wishes to fetch a piece of content, it first sends a request to all its \bitswap peers. If none of them has the content, the \downloader uses the DHT-based resolution system. We stress that the \bitswap protocol plays the supporting role of a cache in the dissemination of popular files. However, the mechanism does not provide reliable content resolution, in particular for new or less popular content. %

When \bitswap unstructured search fails, the \downloader resolves $\cid$ using $\Call{FindProviders}{\cid}$. This operation uses a DHT walk identical to that of $\Call{GetClosestPeers}{\cid}$ to find $k$ \resolvers but also queries encountered nodes for a provider record for $\cid$ (\Cref{fig:add_get_provider}(b)). The process terminates when either 20 \providers have been found, or all \resolvers have been asked. Querying all encountered nodes (\ie, not only the designated \resolvers) is useful because some of the encountered nodes may have a provider record in their cache.
%

Upon receiving a provider record, the client connects to the address specified in the provider record to retrieve the actual content (\Cref{fig:add_get_provider}(c)).
Provider records are not authenticated, and therefore malicious \providers may respond with incorrect provider records (or may not respond at all). However, the integrity of the content is preserved because the hash of the retrieved content can be verified against its $\cid$.
%


%

\input{img/add_get_provider.tex}

\vspacebeforesection
\subsection{Network Size Estimator}
\label{sec:netsize}

The number of nodes in a decentralized system is generally unknown due to the avoidance of centralized membership management.
This number is nonetheless useful for optimizations, deciding on individual node configurations, or security mechanisms.
Various methods were proposed for the decentralized estimation of unstructured and structured networks~\cite{eli-sohl-dht-size-estimation,kostoulas2005decentralized, manku2003symphony}.
We use in this work a mechanism developed initially by Protocol Labs as part of a mechanism for decreasing the latency of publishing content in IPFS~\cite{network-size-estimation-notion,network-size-estimation-github-pr}.

%
%
%
%
%
%
%
%
%
%

Each node in the DHT refreshes its routing table periodically (every $10$ minutes in \texttt{libp2p}). 
For this, the node samples $m$ random keys (one for each bucket of its routing table)
%
and queries the DHT to obtain the $k=20$ closest peer IDs to each key.
Using these, the node then computes the average distance between each one of these keys $\key_j$ for $j=1,\dots,m$ and their $i$-th closest peer ID for $i=1,...,k$ (with $m=256$ and $k=20$).
\begin{equation}
    \label{equ:avg-dist}
    \overline{D}_i = \frac{1}{m} \sum_{j=1}^m \operatorname{dist}(\key_j, \peerid_{j}^{(i)})
\end{equation}
where $\peerid_{j}^{(i)}$ is the $i$-th closest peer ID to $\key_j$.
With $N$ peers in the DHT and peer IDs uniformly distributed in the hash space, the expected distance between a $\key$ and its $i$-th closest peer ID is $\frac{2^{256}i}{N+1}$. The node then runs a least square regression to compute the value of $N$ for which the expected distances best fit the empirical average distances, \ie,
\begin{equation}
    \label{equ:netsize-least-squares}
    \hat{N} = \arg\min_{N} \sum_{i=1}^k \left(\overline{D}_i - \frac{2^{256}i}{N+1}\right)^2.
\end{equation}
The resulting estimate $\hat{N}$ can be computed in closed form.
%

When a node starts running, it must perform DHT queries for a few random keys to initialize its network size estimate. 
Since a larger number of queries will result in higher accuracy, making more queries than what is needed to initialize one's routing table is recommended.
Thereafter, keeping the estimate up-to-date does not require any excess DHT queries beyond what is already used for refreshing the routing table as this is done frequently (every 10 minutes).

While the network size estimate has a stochastic variance resulting from the probability distribution of the honest peer IDs, it is hard for an attacker to bias the estimate significantly. Since the estimator uses the density of peer IDs around keys chosen uniformly at random, the adversary would require numerous Sybil nodes (on the order of the whole network size) to significantly affect the peer ID density around those keys.

\section{Related Work}
%\subsection{Cost Volume based Deep Stereo Matching}
%Stereo matching is a typical problem that has been studied for decades and a well-known four-step pipeline \cite{scharstein2002taxonomy} has been established, where cost volume construction is an indispensable step. Current state-of-the-art stereo matching methods are all cost volume based methods and they can be categorized into two types. Typically, a cost volume is a 4D tensor of height, width, disparity, and features. The first category just uses a full correlation to generate a single-feature cost volume. Such methods are usually efficient but lose much information because of the decimation of feature channels. Many previous work, including Dispnet \cite{dispnet}, MADNet \cite{madnet}, IResNet \cite{iresnet} and AANet \cite{aanet}, belong to this category. The second category usually uses concatenation \cite{gcnet} or group-wise correlation \cite{gwcnet} to generate a multi-feature 4D cost volume. Such a method can achieve better performance while requiring higher computational complexity and memory consumption. Actually, a majority of the top-performing networks in public leaderboards belong to this category, such as GANet \cite{ganet}, CSPN \cite{cspn} and ACFNet \cite{acfnet}. These methods generally employ multiple 3D convolution layers to constantly regularize the 4D cost volume and then apply softmax over the disparity dimension to produce a discrete disparity probability distribution. The final predicted disparity is obtained by softly weighting indices according to their probability, which is also called soft argmin in GCNet \cite{gcnet}. However, soft argmin leaves the output susceptible to multi-modal disparity probability distributions. ACFNet \cite{acfnet} observes this problem and proposes to directly supervise the cost volume with unimodal ground truth distributions. In contrast, we define an uncertainty estimation to quantify the degree to which the cost volume tends to be multi-modal distribution, higher implies the higher possibility of estimation error.

\subsection{Multi-scale Cost Volume based Stereo Matching}
Cost volume construction is an indispensable step in the well-known four-step pipeline for stereo matching \cite{scharstein2002taxonomy, pamisurvey1, pamisurvey2}. Typically, current state-of-the-art stereo matching methods can be categorized into two types of cost volume-based methods, where the cost volume is a 4D tensor of height, width, disparity, and features. The first category usually uses the single-feature 3D cost volume generated by full correlation, which is efficient while losing much information due to the decimation of feature channels. Many real-time methods, such as Dispnet \cite{dispnet}, MADNet \cite{madnet, madnet_pami} and AANet \cite{aanet}, belongs to the category. Moreover, two-stage refinement \cite{mcvmfc} and pyramidal towers \cite{madnet} are commonly applied in the single-feature cost volume based network to construct multi-scale cost volume. The second category usually uses the multi-feature 4D cost volume generated by concatenation \cite{gcnet} or group-wise correlation \cite{gwcnet}, which can achieve better performance with higher computational complexity and memory consumption. Most top-performing networks, including GANet \cite{ganet}, CSPN \cite{cspn} and ACFNet \cite{acfnet} belong to this category. 
% In these methods, the 4D cost volume is constantly regularized by multiple 3D convolution layers and then a discrete disparity probability distribution can be produced by softmax. Next, the final predicted disparity can be obtained by softly weighting indices according to their probability \cite{gcnet}. However, such output is susceptible to multimodal disparity probability distributions and ACFNet \cite{acfnet} gives a solution by directly supervising the cost volume with unimodal ground truth distributions to alleviate this problem. 
Recently, to alleviate the high computational complexity and memory consumption when employing multi-feature 4D cost volumes, \cite{cvpmvsnet, cascade, uscnet} propose to use cascade cost volume representation in multi-view stereo. These methods usually first predict an initial disparity at the coarsest resolution of the image and then gradually refine the disparity by narrowing down the disparity search space. More closely related to our approach is Casstereo \cite{cascade}, which first extended such representation to stereo matching. It selected to uniform sample a pre-defined range to generate the next stage’s disparity search range. Instead, we employ pixel-level uncertainty estimation to adaptively adjust the next stage disparity searching range and generate pseudo-labels for subsequent domain adaptation. Our method also shares similarities with UCSNet \cite{uscnet}, which constructs uncertainty-aware cost volume in multi-view stereo while it doesn’t employ uncertainty estimation to generate pseudo-labels.

%\subsection{Multi-scale Cost Volume based Deep Stereo Matching} 
% \subsection{Multi-scale Cost Volume based Stereo Matching} 
%Multi-scale cost volume firstly was applied in the single-feature cost volume based network with the form of two-stage refinement \cite{mcvmfc} and pyramidal towers \cite{madnet}. Recently, cascade cost volume representation \cite{cvpmvsnet, cascade, uscnet} was proposed in multi-view stereo to alleviate the high computational complexity and memory consumption when employing multi-feature 4D cost volumes. These methods generally predict an initial disparity at the coarsest resolution of the image. Then, they will narrow down the disparity search space and gradually refine the disparity. More closely related to our approach is Casstereo \cite{cascade}, which first extended such representation to stereo matching. It selected to uniform sample a pre-defined range to generate the next stage’s disparity search range. Instead, we employ uncertainty estimation to adaptively adjust the next stage pixel-level disparity searching range and push the next stage's cost volume to be predominantly unimodal.

% The single-feature cost volume based network with the form of two-stage refinement \cite{mcvmfc} and pyramidal towers \cite{madnet} first employ multi-scale cost volume for stereo matching. Recently, to alleviate the high computational complexity and memory consumption when employing multi-feature 4D cost volumes, \cite{cvpmvsnet, cascade, uscnet} propose to use cascade cost volume representation in multi-view stereo, which generally predict an initial disparity at the coarsest resolution of the image. Then, the disparity search space is narrowed down and the disparity is gradually refined. More closely related to our approach is Casstereo \cite{cascade}, which first extended such representation to stereo matching. It selected to uniform sample a pre-defined range to generate the next stage’s disparity search range. Instead, we employ uncertainty estimation to adaptively adjust the next stage pixel-level disparity searching range and push the next stage's cost volume to be predominantly unimodal.

% Figure environment removed

\subsection{Robust Stereo Matching} 
There exist three categories of generalization definitions for robust stereo matching. 1) Cross-domain Generalization: the network’s ability to perform well on unseen scenes (cannot see the image pairs of the target domain in advance). Towards this end, Jia et al \cite{sungeneralizaiton} propose to incorporate scene geometry priors into an end-to-end network. Zhang et al \cite{dsmnet} introduce a domain normalization and a trainable non-local graph-based filter to construct a domain-invariant stereo matching network. 2) Adapt Generalization: the network’s ability to adapt pre-trained models to the new domain with unlabeled target data. Previous work usually pre-trains the models on synthetic data and then adapts it to new target domains with Graph Laplacian regularization \cite{zoom}, non-adversarial progressive color transfer \cite{adastereo}, and Knowledge Reverse Distillation \cite{aohnet}. More closely related to our approach are \cite{aohnet, unsuperviseddomainadaptation} in stereo matching and Monoresmatch \cite{monoresmatch} in monocular depth estimation, which also proposes to generate a pseudo-label for domain adaptation. However, these methods all select to employ classical stereo matching methods \cite{sgm} alongside with confidence estimators, e.g., left-right consistency check to generate pseudo-labels. That is all these methods need an independent method to generate corresponding pseudo-labels. Instead, the proposed method is an end-to-end network that can generate the predicted disparity map, corresponding uncertainty map and pseudo-labels jointly, which is a more simple, yet efficient way. 
% Instead, our proposed method can employ pixel-level and area-level uncertainty estimation to self-distill the predicted disparity maps of our pre-training model and generate sparse while reliable pseudo-labels to align the domain gap, which is a more simple, yet efficient way. 
3) Joint Generalization: the network’s ability to perform well on a variety of datasets with the same model parameters. MCV-MFC \cite{mcvmfc} introduces a two-stage finetuning scheme to achieve a good trade-off between generalization and fitting capability on multiple datasets. However, it doesn’t touch the inner difference between diverse datasets, e.g, the unbalanced disparity distribution. To further address this problem, we propose a cascade cost volume to adaptively the next stage disparity searching space, where the pixel-level uncertainty estimation is at the core.

% \subsection{Monocular Depth Estimation}
% Monocular depth estimation aims to estimate depth values from a single image, instead of stereo images or multiple frames in a video. This problem is ill-posed because of the ambiguity of object sizes. However, humans could estimate the depth from a single image with prior knowledge of the scenes. Recently, learning based methods were explored to learn depth values by supervised or unsupervised learning. Eigen et al. first employed Convolutional Neural Networks (CNN) to predict depth in a coarse-to-fine manner and further improved its performance by multi-task learning. Liu et al. presented deep convolutional neural fields model by combining deep model with continuous CRF. Li et al. [22] refined deep CNN outputs with a hierarchical CRF. Multi-scale continuous CRF was formulated into a deep sequential network by Xu et al. [45] to refine depth estimation. Unsupervised methods tried to train monocular depth estimation with stereo
% image pairs or image sequences and test on single images. Garg et al. [9] used novel image view synthesis loss to train a depth estimation network in an unsupervised way. Godard et al. [11] introduced left-right consistency regularization to improve the performance of view synthesis loss. Recently, some work also propose to use the stereo matching network as a proxy to learn depth from synthetic data or directly employ traditional stereo matching methods to distill proxies labels from the target domain, which proves the feasibility of distilling stereo matching networks to learn monocular depth estimation.



\section{Node Injection Link Stealing Attack\label{sec:attack}}

GNNs are prone to various privacy attacks that usually aim at learning as much information as possible about their underlying graph structure. GNNs inherit the potential attacks against standard neural networks such as membership inference attacks \cite{MIA_GNN, MIA_Jiayuan}, whereby the goal of the adversary is to ascertain whether a sample is included in the training dataset or not.

As introduced earlier, in this paper, we focus on a particular attack named as \textit{link stealing attack}, where an adversary without access to the adjacency matrix aims to learn whether a particular edge exists or not.

In this section, we first introduce the threat model to characterize the adversary's background knowledge. Then, we propose our node injection link stealing attack that takes advantage of the dynamicity of GNNs.


\subsection{Threat model}
\subsubsection{Environment}
As mentioned in the previous section, we consider a GNN application in which a server has already trained the GNN using a specific dataset and offers access to this GNN through a black-box API. In this context, the black-box API is an interface provided by the server that enables users to interact with the pre-trained GNN model without directly accessing its internal components, such as the model architecture, parameters, or graph structure.
Users can submit prediction queries using node IDs. If a new node needs to be added to the graph, users can employ a \textit{connect} query to attach the node to the graph before querying its prediction based on its ID.
The API processes input data into output predictions, ensuring that the model's underlying computations remain hidden from the user. Users can query this GNN for the purpose of node classification. Hence, the query consists of the queried node's ID and the output of this query is the vector of prediction scores for this particular node. The users do not have the knowledge of edges of this graph. Hence the only information that a user knows is the set of nodes' ids.

\subsubsection{Adversary's goal and knowledge}

We consider an adversary, $\mathcal{A}$, who assumes the role of a GNN user. Her objective is to determine the neighbors of a specific \textit{target node}, $v_t$, selected from a set of \textit{target nodes}, $V_{\mathcal{A}}$, within the graph. This is done based on the GNN's predictions for the node set $V_{\mathcal{A}}$. In simpler terms, $\mathcal{A}$ aims to identify the neighbors of the target node $v_t$ that are included in the target set nodes $V_{\mathcal{A}}$.

We should note that if the adversary aims to identify all the links within the graph, then the set of target nodes $V_{\mathcal{A}}$ becomes the set containing all the nodes of the graph $V$. To achieve this, the adversary may need to perform multiple node injections, targeting different nodes from the graph each time. However, the practicality of such an approach is debatable. The adversary's selection of target nodes reflects her background knowledge about these nodes. For instance, in the context of social networks, the adversary's background knowledge could include information such as users' interests. This information can guide the adversary in selecting target nodes $V_{\mathcal{A}}$ that are more likely to be connected. In our attack scenario, we choose the target nodes uniformly at random.

The adversary $\mathcal{A}$ is able to obtain the predictions of the target nodes $V_{\mathcal{A}}$ by sending the server their corresponding IDs through the provided API.
In addition, the adversary $\mathcal{A}$ is able to use the \textit{connect} query to connect a node $v_m$ to a target node $v_t$. In general, we assume that the adversary does not have access to the features of the nodes in the graph, with the exception of certain attack strategies described in Sec.~\ref{subsection:malicious features strategies}.

\subsection{Node injection link stealing attack}
\label{sec:node-injection-attack}
In this section, we formally define our NILS attack that, unlike existing link-stealing attacks, exploits the dynamic nature of the underlying GNN. Indeed, adversary $\mathcal{A}$ can \textit{connect} new nodes and further query the prediction scores of a set of nodes $V_{\mathcal{A}}$ in the graph. While adding this new node $v_m$, $\mathcal{A}$ can choose which existing node $v_t$ it actually connects to and hence try to discover its neighbors. More formally:

\begin{enumerate}
\item $\mathcal{A}$ first queries the prediction scores of the target nodes $V_{\mathcal{A}}$
and receives the corresponding prediction matrix $P$ of the target nodes $V_{\mathcal{A}}$.
    \item $\mathcal{A}$ generates malicious features of a malicious node $v_m$ based on the obtained prediction matrix $P$ (see Sec.~\ref{sec:strategies} for further details on this step).
    \item Next, $\mathcal{A}$ sends a \emph{connect} query to inject the malicious node $v_m$. The query has the following parameters: the features $x_m$ of the new node, and the ID of the target node $v_t$ the adversary wishes to connect $v_m$ to.
    \item The server adds this \text{malicious} node $v_m$ to the graph and links it to the target node $v_t$.
    \item $\mathcal{A}$ queries back the server for new prediction matrix $P'$ of the target nodes $V_{\mathcal{A}}$ and obtains it.
    \item With access to $P$ and $P'$, $\mathcal{A}$ computes the $L_1$ distance between $P(v)$ and $P'(v)$ of each node $v$ in $V_{\mathcal{A}}$.
    % Next, $\mathcal{A}$ can infer the actual links of $v_t$ by computing the $L_1$ distance between the prediction scores $P$ and $P'$ of each node $v$  in $V_{\mathcal{A}}$ before and after the injection.
    A significant change in the prediction scores of a node $v$ indicates a high probability of being a neighbor to $v_t$. If the difference exceeds a threshold $R$, the adversary infers that node $v$ is a neighbor of $v_t$.
    % Indeed, there is a high chance that the prediction scores of $v_t$'s neighbors have significantly been modified. Hence, for a given node, if this difference is large enough, there is a high chance that this particular node $v$ is a neighbor of $v_t$.
\end{enumerate}

The decision threshold $R$ is determined through an extensive parameter tuning process, aiming for an optimal trade-off between precision and recall in identifying the true neighbors of the target node. This balance is represented by the $F_1$ score. We evaluate various candidate values of $R$, selecting the one that yields the highest $F_1$ score as the optimal threshold. The results reported in our study are based on this optimal value of $R$.

 This attack strategy is depicted in Figure \ref{fig:attack_strategy} and outlined in Algorithm \ref{alg:prob_vecs_attack}.
 
% Figure environment removed

\begin{algorithm}
\caption{Node Injection Link Stealing Attack}
\label{alg:prob_vecs_attack}
\textbf{Input:} set of nodes $V_{\mathcal{A}}$ and target node $v_t$. \\
\textbf{Output:} the identified neighbors of $v_t$ by the adversary.\\

$P$ = GNN($V_{\mathcal{A}}$, $X_{V_{\mathcal{A}}}$) \Comment{Step 1}\\
Generate malicious features $x_m$ of node $v_m$ \Comment{Step 2}\\
Connect node $v_m$ to $v_t$. \Comment{Step 3-4}\\
$P'$ = GNN($V_{\mathcal{A}} \cup v_m$ , $X_{V_{\mathcal{A}}} \cup x_m$) \Comment{Step 5}\\
\For{each node $v$ in $V_{\mathcal{A}}$}{ 
$D(v)$ = $\lVert P(v) - P'(v) \rVert_1$ \Comment{Step 6} \\ 
\If{$D(v) \geq R$}{
      $v$ is a neighbor of $v_t$ \\}
\Else{
      $v$ is not a neighbor of $v_t$ \\}
}
\end{algorithm}

 \subsection{Strategies for malicious node's features\label{subsection:malicious features strategies}}
 \label{sec:strategies}
In order to evaluate how the injection of the malicious node $v_m$ influences the predictions of the GNN, we study five strategies to generate the malicious node's features $x_m$. This helps us to assess the success of our attack. These five strategies are designed with varying degrees of sparsity and stealthiness,
%\javi{[after reading the whole subsection, I'm not sure about what we mean by stealthiness. I think the key point here is whether the adversary knows the features of ]}
enabling us to explore their effectiveness in altering the model's predictions. We define the proposed strategies as follow:

\begin{enumerate}
    \item \textbf{All-ones strategy}: Generates a dense feature vector for the malicious node, containing all ones, as shown in the equation below:
    \begin{equation*}
        x_m = \mathbf{1}.
    \end{equation*}
    This strategy potentially causes significant changes in predictions but may be less stealthy due to its dense feature vector.
    
    \item \textbf{All-zeros strategy}: Creates a sparse feature vector for the malicious node, containing all zeros, as shown in the equation below:
    \begin{equation*}
        x_m = \mathbf{0}.
    \end{equation*}
    This approach may subtly alter the output of the GNN, leading to smaller changes in predictions, while offering increased stealthiness.
    
    \item \textbf{Identity strategy}: Introduces a malicious node with a feature vector identical to the target node's feature vector, as shown below:
    \begin{equation*}
        x_m = x_t.
    \end{equation*}
    This strategy causes confusion in the model's predictions for neighboring nodes and has variable stealthiness based on the similarity between injected and target nodes.
    For this strategy, we assume that $\mathcal{A}$ knows the features of the target node $x_t$.
    
    \item \textbf{Max attributes strategy}: This method creates a malicious node feature vector by computing the element-wise maximum of each attribute in the target nodes' feature matrix.
    Specifically, it considers only nodes from classes different from the target node's class, as shown below:
    \begin{equation*}
    x_{m,k} = \max_{i \in V_{\mathcal{A}}, \text{ with } C_i \neq C_t} X_{i,k}, \quad \text{for} \quad k = 1, \ldots, d.
    \end{equation*}
    Here, $C_i$ represents the class of node $i$, and $C_t$ is the class of the target node.
    This strategy potentially causes significant changes in predictions but may be less stealthy due to exaggerated features. We assume in this strategy that the adversary has access to the features of the set of target nodes $V_{\mathcal{A}}$ and also to their predicted classes by the GNN. The predicted classes are accessible to the adversary after step 1 in Algorithm \ref{alg:prob_vecs_attack}.
    
    \item \textbf{Class representative strategy}: This approach generates a malicious node feature vector by selecting the feature vector of the node with the highest confidence score for a specific class, different from the target node's class, as shown below:
    \begin{equation*}
    x_m = x_{i^*} \text{ with } i^* = \argmax_{\substack{ i \in V_{\mathcal{A}}, \\ C_i \neq C_t}} p_{i,j}.
    \end{equation*}
    In this equation, $x_m$ is the malicious node feature vector, $i^*$ is the node index with the highest confidence score for a specific class different from the target node's class, $V_{\mathcal{A}}$ is the set of target nodes, $C_i$ represents the class of node $i$, and $C_t$ is the class of the target node. This strategy leverages the model's predictions to alter the neighbors of the target node predictions, potentially offering increased stealthiness.
\end{enumerate}
Additionally, we introduce the \textit{so-called} LinkTeller \textbf{Influence} strategy as an alternative to the original method in \cite{linkteller} incorporating their feature perturbation strategy.
This strategy entails perturbing the features of the target node by adding a small real value $\delta$, as shown below:
\begin{equation*}
x_m = x_t + \mathbf{\delta}.
\end{equation*}
We assess the performance of the Influence strategy in comparison to other strategies, aiming to determine whether the attack performance gains are attributable to node injection or the crafting of malicious features. It is worth noting, however, that the Influence strategy may be easily detected if the feature $x_t$ has a discrete nature, given that $x_m$ is real-valued.

\section{Evaluation of The Attack\label{sec:attack_eval}}
In this section, we present the evaluation results of our proposed attack. First, we introduce our experimental setup. Then, we provide a detailed analysis of the performance of our attack on various datasets, discussing its effectiveness and limitations.
\subsection{Experimental setup}
\subsubsection{Datasets}
In order to evaluate the effectiveness of our attack, we conducted experiments on various real-world datasets previously utilized in related research. We include the Flickr \cite{ZengZSKP20_Flickr} dataset, where nodes represent images uploaded to the Flickr platform. Edges connect nodes if the images share common properties like geographic location, gallery, or user comments. Node features contain word representations. Additionally, we utilize two Twitch datasets (TWITCH-FR and TWITCH-RU)\cite{rozemberczki2021twitch} to evaluate NILS. We use Twitch-ES to train the GNNs as done previously in \cite{linkteller} for the inductive setting. Twitch datasets \cite{rozemberczki2021twitch} illustrate follow relationships between users on the Twitch streaming platform. The objective of these datasets is to perform binary classification to determine if a streamer uses explicit language, using features such as users' preferred games, location, and streaming habits.

Furthermore, for the transductive setting, where the training and testing of the GNNs occur on the same graph, we incorporate three citation network datasets \cite{citation_datasets}, Cora, Citeseer, and Pubmed. These datasets capture citation relationships among scientific publications across various fields. The classification task of these datasets involves predicting the topic of publications based on their textual features. While Cora and Citeseer encompass general scientific publications, Pubmed is dedicated to biomedical publications. By employing these datasets in our evaluation, we aim to demonstrate the effectiveness of our proposed attack in both inductive and transductive settings, as well as across a range application domains.

\subsubsection{Models}

In our study, we follow LinkTeller's approach to training the models and selecting hyperparameters \cite{linkteller}. In LinkTeller \cite{linkteller}, the authors trained Graph Convolutional Networks (GCNs) using various configurations and hyperparameters, which encompassed normalization techniques applied to the adjacency matrix, the number of hidden layers, input and output units, and dropout rates. In order to identify the optimal set of hyperparameters, the authors employed a grid search strategy, systematically exploring combinations of hyperparameters and evaluating their performance on a validation set.
The search space for hyperparameters and the formulae for different normalization techniques were provided in \cite[Appendix F]{linkteller}. After obtaining the best set of hyperparameters, the authors trained the GCN models to minimize the cross-entropy loss for the intended tasks.

In our experiments, we adhere to the same methodology as in LinkTeller \cite{linkteller}, ensuring consistency across the studies. By utilizing the same training procedures and hyperparameter tuning strategies, we aim to provide a comprehensive understanding of the attack performance across different layer configurations (two, three, and four layers) while maintaining consistency.

\subsubsection{Evaluation of attack performance}
In accordance with the evaluation methodology presented in the LinkTeller paper \cite{linkteller}, we employ precision, recall, and the $F_1$ score as our primary evaluation metrics. These metrics are particularly suitable for addressing the imbalanced binary classification problem at hand, in which the minority class (i.e., connected nodes) is of central interest. We primarily select the set target nodes $V_{\mathcal{A}}$, such that  $|V_{\mathcal{A}}|=500$, using a uniform random sampling approach. Furthermore, following the baseline \cite{linkteller} study's example, we explore scenarios where target nodes exhibit either low or high degrees. A comprehensive discussion of the sampling strategy can be found in \cite[Section V.D.]{linkteller}. We report the results averaged over three runs with different random seeds along with the standard deviation.
% training
% hyperparameters
% GNN architecture
% Evaluation methodology of the attack
\subsection{Analysis of strategies for malicious node’s features}
In this section, we analyze the impact of different strategies, as defined in Section \ref{subsection:malicious features strategies}, for generating the features $x_m$ of the malicious node $v_m$ on the success of our attack.

The success rates of these strategies, as shown in Table \ref{tab:adv_strategies}, reveal that the All-ones, Max attributes, and Class representative strategies are the most effective in causing significant changes in the predictions of the target node's neighbors. These results suggest that injecting nodes with high-valued or class-specific features can effectively disrupt the model's output predictions.

Conversely, the All-zeros, and Identity strategies exhibit relatively lower success rates, as shown in Table \ref{tab:adv_strategies}. While these strategies offer certain benefits in terms of stealthiness, their impact on the graph structure and predictions is less pronounced, highlighting a trade-off between attack effectiveness and stealthiness.

Concerning the Influence strategy, our NILS method exhibits a modest improvement over the LinkTeller baseline for the Twitch-FR dataset, as illustrated in Table \ref{tab:adv_strategies}. This suggests that the node injection property of our NILS attack is effective in this context. However, for the Twitch-RU dataset, NILS underperforms in comparison to the LinkTeller baseline. The most significant improvement is observed in the Flickr dataset, where the node injection property of NILS considerably increases the $F_1$ score from $0.32 \pm 0.13$ of LinkTeller to $0.89 \pm 0.10$. This outcome highlights the advantage of NILS attack's node injection method within the Influence strategy, particularly when compared to the LinkTeller attack, which employs the Influence strategy without node injection. 

These findings underscore the importance of considering both the effectiveness and stealthiness of malicious feature generation strategies when devising link inference attacks on GNNs.

\begin{table}[h]

\begin{adjustbox}{width=\columnwidth,center}
\centering


\begin{tabular}{lccc}
\toprule
Method & Twitch-FR & Twitch-RU & Flickr \\
\midrule
Class Rep. & $0.94 \pm 0.01$ & $0.83 \pm 0.06$ & $0.96 \pm 0.06$ \\
Max Attr.  & $0.99 \pm 0.00$  & $0.98 \pm 0.02$ & $\boldsymbol{1.00 \pm 0.00}$ \\
All-ones   & $\boldsymbol{0.99 \pm 0.00}$ & $\boldsymbol{0.97 \pm 0.01}$ & $0.99 \pm 0.02$ \\
All-zeros  & $0.58 \pm 0.02$  & $0.48 \pm 0.01$  & $0.71 \pm 0.07$ \\
Identity   & $0.81 \pm 0.02$  & $0.69 \pm 0.01$  & $0.95 \pm 0.07$ \\
Influence NILS  & $0.81 \pm 0.02$  & $0.70 \pm 0.01$  & $0.89 \pm 0.10$ \\
Influence LinkTeller \cite{linkteller}   & $0.80 \pm 0.02$  & $0.74 \pm 0.01$  & $0.32 \pm 0.13$ \\
\bottomrule
\end{tabular}
\end{adjustbox}
\caption{$F_1$ scores and standard deviations for different attack methods and datasets.}
\label{tab:adv_strategies}
\end{table}
\subsection{Comparison with the baselines}

In this study, we conducted experiments to evaluate the performance of our proposed NILS attack in comparison to the LinkTeller attack using the same experimental setup. Our focus is on analyzing the optimal attacks for both approaches, which involved accurately estimating the number of neighbors of the target set nodes. The results, summarized in Table \ref{tab:comp_LT}, demonstrate that our attack outperforms LinkTeller on both Twitch datasets (TWITCH-FR and TWITCH-RU). Furthermore, our method exhibits a substantial improvement over LinkTeller on the Flickr dataset, achieving nearly double the precision and recall values. Notably, our attack demonstrates stable performance across varying node degrees, with only a marginal decrease in effectiveness for high-degree target nodes. This can be attributed to the smaller influence that each neighboring node has on the aggregation of the GCN layer when the target node degree is high. Overall, our proposed NILS attack demonstrates consistently a superior performance compared to the LinkTeller attack.

We further compare our attack with link-stealing attacks introduced in \cite{he2021stealing}, where the authors' various attack strategies rely on different types of background knowledge available to the adversary, such as node attributes and shadow datasets. Specifically, in their Attack-2, the adversary has access to both the features and prediction scores of the nodes. Utilizing this information, the adversary creates two types of attacks: LSA2-attr and LSA2-post. LSA2-attr calculates distances between node attributes, while LSA2-post computes distances between node prediction scores (posteriors). It is important to highlight that these two attacks align closely with our threat model, as both assume that the adversary has access to the features and prediction scores of the target node. This similarity in assumptions renders these attacks particularly relevant for comparison with our proposed NILS attack. The attacks are executed under the transductive setting, where training and inference occur on the same graph. As shown in Table \ref{tab:LST_comp}, our proposed NILS attack outperforms the LSA2-post and LSA2-attr attacks constructed in \cite{he2021stealing}. However, our attack performance is nearly equivalent to that of LinkTeller. These results demonstrate that NILS attack maintains effectiveness under the transductive setting, just as in the inductive setting.
% \usepackage{multirow}
\begin{table*}[]
%\begin{adjustbox}{width=\columnwidth,center}
\begin{tabular}{cccccccc}
\toprule
\multirow{2}{*}{Dataset} &
  \multirow{2}{*}{Method} &
  \multicolumn{2}{c}{low} &
  \multicolumn{2}{c}{uncontrained} &
  \multicolumn{2}{c}{high} \\ \cline{3-8} 
 &
   &
  precision &
  recall &
  precision &
  recall &
  precision &
  recall \\ \hline
\multirow{2}{*}{TWITCH-FR} &
  NILS (Ours) &
  $100.0 \pm \scriptstyle 0.0$ &
  $100.0 \pm \scriptstyle 0.0$ &
  $99.13 \pm \scriptstyle 0.8$ &
  $99.57 \pm \scriptstyle 0.35$ &
  $99.91 \pm \scriptstyle 2.6$ &
  $100.0\pm \scriptstyle 0.0$ \\
 &
  LinkTeller &
  $92.5 \pm \scriptstyle 5.4$ &
  $92.5 \pm \scriptstyle 5.4$ &
  $84.1 \pm \scriptstyle 3.7$ &
  $78.2 \pm \scriptstyle 1.9$ &
  $83.2 \pm \scriptstyle 1.4$ &
  $80.6 \pm \scriptstyle 6.7$ \\ \hline
\multirow{2}{*}{TWITCH-RU} &
  NILS (Ours) &
  $100.0 \pm \scriptstyle 0.0$ &
  $100.0 \pm \scriptstyle 0.0$ &
  $96.45 \pm \scriptstyle 0.4 $ &
  $ 98.34\pm \scriptstyle 0.7$ &
  $99.77 \pm \scriptstyle 0.1$ &
  $ 99.37\pm \scriptstyle 0.1$ \\
 &
  LinkTeller &
  $78.8 \pm \scriptstyle 1.9$ &
  $ 92.6 \pm \scriptstyle 5.5 $ &
  $ 71.8\pm \scriptstyle 2.2$ &
  $78.5 \pm \scriptstyle 2.4$ &
  $ 89.7\pm \scriptstyle 1.7 $ &
  $65.7 \pm \scriptstyle 3.9 $ \\ \hline
\multirow{2}{*}{Flickr} &
  NILS (Ours) &
  $100.0\pm \scriptstyle 0.0$ &
  $100.0\pm \scriptstyle 0.0$ &
  $99.11\pm \scriptstyle 1.7$ &
  $95.83\pm \scriptstyle 5.0$ &
  $93.72\pm \scriptstyle 3.1$ &
  $78.9\pm \scriptstyle 1.9 $ \\
 &
  LinkTeller &
  $51.0 \pm \scriptstyle 7.0$ &
  $53.3\pm \scriptstyle 4.7$ &
  $33.8\pm \scriptstyle 13.3$ &
  $32.1\pm \scriptstyle 13.3$ &
  $18.2\pm \scriptstyle 4.5$ &
  $18.5\pm \scriptstyle 6.1$ \\ \hline
\end{tabular}
%\end{adjustbox}
\caption{Comparative performance of our proposed attack NILS and LinkTeller across three datasets (TWITCH-FR, TWITCH-RU, and Flickr) under low, unconstrained, and high constraint settings. The results are presented in terms of precision and recall with corresponding standard deviations}
\label{tab:comp_LT}
\end{table*}

\begin{table}[]
\centering
\begin{adjustbox}{width=\columnwidth,center}
\begin{tabular}{ccccccc}
\toprule
\multirow{2}{*}{Method} &
  \multicolumn{2}{c}{Cora} &
  \multicolumn{2}{c}{Citeseer} &
  \multicolumn{2}{c}{Pubmed} \\ \cline{2-7} 
 &
  precision &
  recall &
  precision &
  recall &
  precision &
  recall \\ \midrule
NILS (Ours) &
  $ 99.7\pm \scriptstyle 0.2$ &
  $ 99.6\pm \scriptstyle 0.3 $ &
  $97.4 \pm \scriptstyle 0.2 $ &
  $98.2 \pm \scriptstyle 0.1$ &
  $ 99.7\pm \scriptstyle 0.0 $ &
  $100.0 \pm \scriptstyle 0.0 $ \\
LinkTeller &
  $99.5 \pm \scriptstyle 0.1 $ &
  $ 99.5\pm \scriptstyle 0.1$ &
  $99.7 \pm \scriptstyle 0.0$ &
  $99.7 \pm \scriptstyle 0.0$ &
  $99.7 \pm \scriptstyle 0.0$ &
  $99.7 \pm \scriptstyle 0.0$ \\
LSA2-post &
  $ 86.7 \pm \scriptstyle 0.2 $ &
  $ 86.7\pm \scriptstyle 0.2$ &
  $ 90.1 \pm \scriptstyle 0.2$ &
  $ 90.1 \pm \scriptstyle 0.2$ &
  $ 78.8\pm \scriptstyle 0.1$ &
  $ 78.8\pm \scriptstyle 0.1$ \\
LSA2-attr &
  $73.6 \pm \scriptstyle 0.1$ &
  $73.6 \pm \scriptstyle 0.1$ &
  $80.9 \pm \scriptstyle 0.1$ &
  $80.9 \pm \scriptstyle 0.1$ &
  $ 82.4\pm \scriptstyle0.1 $ &
  $ 82.4\pm \scriptstyle0.1 $ \\
\bottomrule
  
\end{tabular}
\end{adjustbox}
\caption{Comparative performance of NILS attack with LinkTeller \cite{linkteller} and link-stealing attacks in \cite{he2021stealing} across three datasets (Cora, Citeseer, and Pubmed).}
\label{tab:LST_comp}
\end{table}

\subsection{Depth of the GNN}
In this section, we examine the impact of increasing the depth of GNN on the success rate of the attack for the Twitch-Fr dataset. Our findings illustrated in Figure \ref{fig:depth_imact_LT} indicate that as the depth of the GNN increases, the attack's success rate decreases, which can be attributed to the dilution of the injected poisoning node's influence within the target node's neighborhood. As the GNN depth increases, the model aggregates information from a larger neighborhood, encompassing nodes that are $k-$hops away from the target node. Consequently, the injected malicious node's features become one among many contributing factors in the aggregated information, leading to a dilution of its influence. This reduction in the injected node's impact on the aggregated information diminishes the overall effectiveness of the attack, making it less successful in altering the predictions of the target node's neighbors.

In comparison with LinkTeller \cite{linkteller}, as shown in Table \ref{tab:depth_imact_LT}, NILS outperforms LinkTeller \cite{linkteller} across various GCN depths. Specifically, for Twitch-FR dataset, NILS demonstrates higher precision and recall values when the GCN depth is 3 (precision: $85.06 \pm \scriptstyle 1.2$, recall: $81.56 \pm \scriptstyle 1.2$) compared to the LinkTeller method (precision: $50.01 \pm \scriptstyle 5.1$, recall: $46.6 \pm \scriptstyle 5.0$). Notably, NILS consistently outperforms LinkTeller even when comparing the attack performance of LinkTeller with a GCN depth of 2 and NILS with a GCN depth of 3. Specifically, for Twitch-FR dataset, NILS demonstrates higher precision and recall values at a GCN depth of 3 (precision: $85.06 \pm \scriptstyle 1.2$, recall: $81.56 \pm \scriptstyle 1.2$) compared to the LinkTeller method with a GCN depth of 2 (precision: $84.1 \pm \scriptstyle 3.7$, recall: $78.2 \pm \scriptstyle 1.9$). These results highlight the effectiveness of our node injection strategy, as it consistently outperforms the LinkTeller method across different depths of the GCN.

% Figure environment removed

%\ayse{isn't it better to use NILS instead of Ours in the table?}
\begin{table}[]
\begin{adjustbox}{width=\columnwidth,center}
\begin{tabular}{cccccc}
\toprule
\multirow{2}{*}{Dataset} &
  \multirow{2}{*}{Method} &
  \multicolumn{2}{c}{Depth-2} &
  \multicolumn{2}{c}{Depth-3} \\ \cline{3-6} 
 &
   &
  precision &
  recall &
  precision &
  recall \\ \midrule
\multirow{2}{*}{TWITCH-FR} & NILS (Ours) & $99.13 \pm \scriptstyle 0.8$ & $99.57 \pm \scriptstyle 0.35$ & $85.06\pm \scriptstyle 1.2$   & $81.56 \pm \scriptstyle 1.2$ \\
 &
  LinkTeller &
  $84.1 \pm \scriptstyle 3.7$ &
  $78.2 \pm \scriptstyle 1.9$ &
  $50.1 \pm \scriptstyle 5.1$ &
  $46.6 \pm \scriptstyle 5.0$ \\ \midrule
\multirow{2}{*}{TWITCH-RU} & NILS (Ours) & $96.45 \pm \scriptstyle 0.4$ & $98.34 \pm \scriptstyle 0.7$  & $78.78 \pm \scriptstyle 3.8 $ & $ 76.35\pm \scriptstyle 9.3$ \\
 &
  LinkTeller &
  $71.8\pm \scriptstyle 2.2$ &
  $78.5 \pm \scriptstyle 2.4 $ &
  $45.7\pm \scriptstyle 2.2$ &
  $50.0 \pm \scriptstyle 2.8$ \\
\bottomrule
\end{tabular}
\end{adjustbox}
  \caption{Success rates of the attack for different depths in comparison with LinkTeller \cite{linkteller}. We use the all-ones strategy and Twitch-FR dataset.}
  \label{tab:depth_imact_LT}
\end{table}


% \section{Defense\label{sec:defense}}

This section introduces the basic notions of DP in the context of GNNs. As a reminder, the goal is to protect the privacy of the graph, in the sense of preventing an adversary from discovering whether, in a given graph, there is a link between two nodes. With this aim, we need to define the neighbouring relation of graphs and further revise the definition of DP.

\subsection{DP for graphs}
Recall from Sec.~\ref{sec:background:DP} that the notion of neighborhood of DP was defined originally for microdata, and, %\ayse{omit}that
accordingly, two databases are said to be neighbors if they differ just in one record.
In the context of graphs at hand, however, this notion must be adapted since two graphs may differ with respect to either one edge or one node. 


In the literature, we find two attempts~\cite{kossinets2006empirical,hay2009accurate} to adapt DP to graphs. %, taking into account these two possibilities. %depending on whether neighboring graphs differ by at most one edge or one node. 
Before examining them, recall from Sec.~\ref{sec:GNNsOverview} that a graph $\mathcal{G} =(V,E)$ is represented with an adjacency matrix $A$, whereby $A_{ij}=1$ if there is a link between node $i$ and node $j$, and $A_{ij} = 0$ otherwise (where $i,j \in \{1,\ldots,|V|\}$). 

\begin{definition}[\textbf{Edge-level adjacent graphs} \cite{kossinets2006empirical}]
$\mathcal{G}$ and $\mathcal{G'}$ are considered \emph{edge-level adjacent graphs} if one can be obtained from the other by removing a single edge. In other words, $\mathcal{G}$ and $\mathcal{G'}$ differ by at most one edge. Hence, their adjacency matrices differ by one element only.
\end{definition}
Accordingly, an edge-level DP mechanism is defined as follows:
\begin{definition}[\textbf{$(\varepsilon, \delta)$-Edge-level differential privacy}]
	\label{def:edgedp}
	A randomized mechanism $\mathcal{M}$ satisfies \textbf{$(\varepsilon,\delta)$-edge-level DP} with $\varepsilon,\delta \geqslant 0$ if, for all pairs of
	edge-level adjacent graphs $\mathcal{G},\mathcal{G}'$ and for all measurable $\mathcal{O}\subseteq \Range(\M)$,
	
	$$\oP\{\mathcal{M}(\mathcal{G})\in \mathcal{O}\} \leqslant e^{\varepsilon} \oP\{\mathcal{M}(\mathcal{G}')\in \mathcal{O}\} + \delta.$$
\end{definition}


%\subsubsection{Node-level differential privacy for graphs}
\begin{definition}[\textbf{Node-level adjacent graphs} \cite{hay2009accurate}]
\label{def:node-level}
$\mathcal{G}$ and $\mathcal{G'}$ are said to be \emph{node-level adjacent graphs} if one can be obtained from the other by removing a single node and all of its incident edges.
\end{definition}
Node-level DP is defined analogously as follows:
\begin{definition}[\textbf{$(\varepsilon, \delta)$-Node-level differential privacy}]
	\label{def:nodedp}
	A randomized mechanism $\mathcal{M}$ satisfies \textbf{$(\varepsilon,\delta)$-node-level DP} with $\varepsilon,\delta \geqslant 0$ if, for all pairs of node-level adjacent graphs $\mathcal{G},\mathcal{G}'$ and for all measurable $\mathcal{O}\subseteq\Range(\mathcal{M})$, the following inequality holds:
	
	$$\oP\{\mathcal{M}(\mathcal{G})\in \mathcal{O}\} \leqslant e^{\varepsilon} \oP\{\mathcal{M}(\mathcal{G}')\in \mathcal{O}\} + \delta$$
\end{definition}


\subsection{One-node-one-edge-level DP}
The adversary defined in Sec.~\ref{sec:node-injection-attack} adds a malicious node to a graph and connects it to a target node through a \emph{single} edge. Countering such an adversary with a node-level DP mechanism (see Definition~\ref{def:node-level}) is clearly not a suitable choice in terms of model accuracy since node-level DP targets a stronger adversary. Trying to hide the presence or absence of one node and \emph{all} of its incident edges intuitively would increase the scale of the noise to be added (for example to the original adjacency matrix), and incur more data inaccuracy than necessary.
Motivated by this, we define a new notion of neighboring graphs (and the corresponding DP mechanism), which is designed to specifically counter the adversary proposed in this work. 
\begin{definition}[\textbf{One-node-one-edge-level adjacent graphs}]
\label{def:1n1e-graphs}
$\mathcal{G}$ and $\mathcal{G'}$ are considered \emph{one-node-one-edge-level adjacent graphs} if one can be obtained from the other by adding a single node with one edge only.
\end{definition}
Note that, as in node-level adjacent graphs (Definition~\ref{def:node-level}), the adjacency matrices of two neighboring graphs (in the sense of one-node-one-edge) differs by one row and one column only, but unlike node-level, the difference in $L_1-$norm between the adjacency matrices is always one.
Based on Definition~\ref{def:1n1e-graphs}, a one-node-one-edge-level DP is defined as follows:
\begin{definition}[\textbf{$(\varepsilon, \delta)$-One-node-one-edge-level differential privacy}]
	\label{def:nodedp}
	A randomized mechanism $\mathcal{M}$ satisfies \textbf{$(\varepsilon,\delta)$-one-node-one-edge-level DP} with $\varepsilon,\delta \geqslant 0$ if, for all pairs of one-node-one-edge-level adjacent graphs $\mathcal{G},\mathcal{G}'$ and for all measurable $\mathcal{O}\subseteq \Range(\mathcal{M})$,
the following holds:
	$$\oP\{\mathcal{M}(\mathcal{G})\in \mathcal{O}\} \leqslant e^{\varepsilon} \oP\{\mathcal{M}(\mathcal{G}')\in \mathcal{O}\} + \delta$$
\end{definition}


\subsection{Countermeasures for our attack}


In this section, we describe one DP-based strategy, namely the LapGraph mechanism, which was introduced in \cite{linkteller}. A much simpler defense approach against any privacy attack to GNNs would of course be output perturbation~\cite{outputperturbation}, whereby the very same output of the GNN prediction is perturbed with some DP mechanism (e.g., the classical Laplace mechanism). While this solution is straightforward to implement and indeed can be used to satisfy the one-node-one-edge-level DP notion, unfortunately, it would significantly deteriorate the accuracy of the GNN output. 
It is easy to see that the $L_1$-global sensitivity of a prediction matrix for the set of nodes $V_{\mathcal{A}}$ is as large as $2\,|V_{\mathcal{A}}|$, which makes us rule out output perturbation. 

To defend against the newly proposed attack, similar to \cite{linkteller}, we propose to apply the LapGraph algorithm, which consists in perturbing the adjacency matrix using the Laplace mechanism and binarizing it by replacing the top-$N$ largest values by 1 and the remaining values by 0. Here, $N$ represents the estimated number of edges in the graph, which is also computed using the Laplace mechanism. % \ayse{omit}in a DP manner. 

By leveraging the post-processing property of DP\footnote{The post-processing of DP allows arbitrary data-independent transformations to DP outputs without affecting their privacy guarantee~\cite{postprocessing}.}, the edge information remains protected even if the adversary observes the predictions generated by the GNN. Furthermore, 
 
each time a user connects a new node, a new adjacency matrix is generated following the same LapGraph mechanism, accumulating this way the privacy budget by the sequential composition property of DP \cite{dwork2014algorithmic}.

Although the LapGraph mechanism was proposed to meet edge-level DP, it is not difficult to show that the mechanism can also be used to satisfy one-node-one-edge-level DP. 
For this, let $f_A$ be the query function returning the adjacency matrix of a graph $G$. Unlike edge-level neighborhood, the corresponding matrices $A, A'$ of two one-node-one-edge neighboring graphs $G,G'$ have different dimensions, namely, %\ayse{this is not a sentence, should be attached to the previous one}
either $A\in \mathbb{R}^{n \times n}$ and $A'\in \mathbb{R}^{(n+1) \times (n+1)}$, or $A\in \mathbb{R}^{(n+1) \times (n+1)}$ and $A'\in \mathbb{R}^{n \times n}$. Without loss of generality, we assume the former case, where $A$ and $A'$ represent the adjacency matrices \emph{before} and \emph{after} the new node is connected to $G$ (resulting in $G'$). We shall also assume that the new node corresponds to the $(n+1)$-th row and, for symmetry, to the $(n+1)$-th column of $A'$. Precisely, since any adjacency matrix is symmetric by definition, the computation of the sensitivity of $f_A$ only requires the upper or lower triangular matrix of $A$.

To enable the subtraction operation $A-A'$ implicit in the definition of the global sensitivity (see Definition~\ref{def:GS}), we append one zero-row and one-zero column to $A$ and denote the resulting matrix by $\bar{A}\in \mathbb{R}^{(n+1) \times (n+1)}$.
As in the case of $A'$, we assume that the appended row and column are in the %\ayse{parantheses}
$(n+1)$-th position of $\bar{A}$.

To compute the sensitivity of $f_A$ for the notion of one-node-one-edge neighboring graphs, we just need to note that the $(n+1)$-th columns (or rows, if we consider the lower triangular of the adjacency matrix) of $\bar{A}$ and $A'$ always differ in one element. The reason is because one-node-one-edge neighboring graphs differ in only one edge. As a result,
$$\|\bar{A}-A'\|_1=1$$
for any pair of neighboring graphs,
which yields an $L_1$-global sensitivity of 1, as in the original LapGraph mechanism intended for edge-level DP. The fact that the two sensitivities coincide implies that the LapGraph version utilized in this work will provide stronger protection for the same level of utility, compared to the original LapGraph.
The reason for that is because, while the scale of the Laplace noise will be the same for a same $\varepsilon$, one-node-one-edge guarantees indistinguishability between any pair of graphs differing not only in one edge and but also in one node. 
% Figure environment removed


\subsection{LapGraph evaluation}
In this section, we evaluate the effectiveness of LapGraph \cite{linkteller} in reducing the success of  NILS attack while ensuring our one-node-one-edge-level DP notion. We also investigate the utility of GCN models trained with LapGraph protection.

\subsubsection{Evaluation setup:} We use the same training hyperparameters and normalization techniques as in the vanilla case, where DP is not applied. Initially, we protect the training graph with LapGraph. Following that, we apply LapGraph each time the graph changes due to node injection by the adversary.
In line with the setup in \cite{linkteller}, we compute the $F_1$ score for our NILS attack as well as the classification task's $F_1$ score for the GCN. This allows us to measure LapGraph protection along with the GCN utility across various privacy budgets $\varepsilon$. We report the results averaged over 5 runs with different random seeds for LapGraph.

\subsubsection{Evaluation results:} Figure \ref{fig:lapgraph_effectiveness} presents the $F_1$ score of the attack for various $\varepsilon$ values. We observe that applying LapGraph reduces the effectiveness of NILS. The $F_1$ score becomes almost zero when the privacy budget $\varepsilon$ is small. However, for large $\varepsilon$, LapGraph provides moderate protection, but the attack's $F_1$ score remains significantly lower than in the non-private case where DP is not applied.

For comparison, in the LinkTeller \cite{linkteller} attack, where LapGraph is applied only once to ensure edge-level DP, LapGraph offers limited protection when $\varepsilon$ is large, allowing LinkTeller to achieve a success rate nearly as high in the non-private case. Conversely, in our scenario, where LapGraph is also applied after the adversary's node injection, LapGraph provides stronger protection. The application of LapGraph during inference makes it more challenging for the adversary to distinguish between the target node's neighbors and non-neighbors, as the prediction scores of all target nodes change after each inference query.
Consequently, the distances between the prediction scores $P$ and $P'$, before and after the node injection, become noisier due to LapGraph's application following the node injection.

To provide insights about the privacy-utility tradeoff of LapGraph, we present in Figure \ref{fig:lapgraph_utility} the utility of the GCNs for different values of the privacy budget. We observe that the utility increases when $\varepsilon$ increases, as expected. Large values of $\varepsilon \geq 7$ give a better utility close to that in the non-private vanilla case. Therefore, carefully choosing an $\varepsilon$ will give fairly good utility and a certain level of protection against NILS attack.



% Figure environment removed


% \section{DP Evaluation\label{sec:dp_eval}}

- Write that output perturbation deteriorates the utility of the GNN?
- Hence, we use LapGraph

\section{Defense\label{sec:defense}}

This section introduces the basic notions of DP in the context of GNNs. As a reminder, the goal is to protect the privacy of the graph, in the sense of preventing an adversary from discovering whether, in a given graph, there is a link between two nodes. With this aim, we need to define the neighbouring relation of graphs and further revise the definition of DP.

\subsection{DP for graphs}
Recall from Sec.~\ref{sec:background:DP} that the notion of neighborhood of DP was defined originally for microdata, and, %\ayse{omit}that
accordingly, two databases are said to be neighbors if they differ just in one record.
In the context of graphs at hand, however, this notion must be adapted since two graphs may differ with respect to either one edge or one node. 


In the literature, we find two attempts~\cite{kossinets2006empirical,hay2009accurate} to adapt DP to graphs. %, taking into account these two possibilities. %depending on whether neighboring graphs differ by at most one edge or one node. 
Before examining them, recall from Sec.~\ref{sec:GNNsOverview} that a graph $\mathcal{G} =(V,E)$ is represented with an adjacency matrix $A$, whereby $A_{ij}=1$ if there is a link between node $i$ and node $j$, and $A_{ij} = 0$ otherwise (where $i,j \in \{1,\ldots,|V|\}$). 

\begin{definition}[\textbf{Edge-level adjacent graphs} \cite{kossinets2006empirical}]
$\mathcal{G}$ and $\mathcal{G'}$ are considered \emph{edge-level adjacent graphs} if one can be obtained from the other by removing a single edge. In other words, $\mathcal{G}$ and $\mathcal{G'}$ differ by at most one edge. Hence, their adjacency matrices differ by one element only.
\end{definition}
Accordingly, an edge-level DP mechanism is defined as follows:
\begin{definition}[\textbf{$(\varepsilon, \delta)$-Edge-level differential privacy}]
	\label{def:edgedp}
	A randomized mechanism $\mathcal{M}$ satisfies \textbf{$(\varepsilon,\delta)$-edge-level DP} with $\varepsilon,\delta \geqslant 0$ if, for all pairs of
	edge-level adjacent graphs $\mathcal{G},\mathcal{G}'$ and for all measurable $\mathcal{O}\subseteq \Range(\M)$,
	
	$$\oP\{\mathcal{M}(\mathcal{G})\in \mathcal{O}\} \leqslant e^{\varepsilon} \oP\{\mathcal{M}(\mathcal{G}')\in \mathcal{O}\} + \delta.$$
\end{definition}


%\subsubsection{Node-level differential privacy for graphs}
\begin{definition}[\textbf{Node-level adjacent graphs} \cite{hay2009accurate}]
\label{def:node-level}
$\mathcal{G}$ and $\mathcal{G'}$ are said to be \emph{node-level adjacent graphs} if one can be obtained from the other by removing a single node and all of its incident edges.
\end{definition}
Node-level DP is defined analogously as follows:
\begin{definition}[\textbf{$(\varepsilon, \delta)$-Node-level differential privacy}]
	\label{def:nodedp}
	A randomized mechanism $\mathcal{M}$ satisfies \textbf{$(\varepsilon,\delta)$-node-level DP} with $\varepsilon,\delta \geqslant 0$ if, for all pairs of node-level adjacent graphs $\mathcal{G},\mathcal{G}'$ and for all measurable $\mathcal{O}\subseteq\Range(\mathcal{M})$, the following inequality holds:
	
	$$\oP\{\mathcal{M}(\mathcal{G})\in \mathcal{O}\} \leqslant e^{\varepsilon} \oP\{\mathcal{M}(\mathcal{G}')\in \mathcal{O}\} + \delta$$
\end{definition}


\subsection{One-node-one-edge-level DP}
The adversary defined in Sec.~\ref{sec:node-injection-attack} adds a malicious node to a graph and connects it to a target node through a \emph{single} edge. Countering such an adversary with a node-level DP mechanism (see Definition~\ref{def:node-level}) is clearly not a suitable choice in terms of model accuracy since node-level DP targets a stronger adversary. Trying to hide the presence or absence of one node and \emph{all} of its incident edges intuitively would increase the scale of the noise to be added (for example to the original adjacency matrix), and incur more data inaccuracy than necessary.
Motivated by this, we define a new notion of neighboring graphs (and the corresponding DP mechanism), which is designed to specifically counter the adversary proposed in this work. 
\begin{definition}[\textbf{One-node-one-edge-level adjacent graphs}]
\label{def:1n1e-graphs}
$\mathcal{G}$ and $\mathcal{G'}$ are considered \emph{one-node-one-edge-level adjacent graphs} if one can be obtained from the other by adding a single node with one edge only.
\end{definition}
Note that, as in node-level adjacent graphs (Definition~\ref{def:node-level}), the adjacency matrices of two neighboring graphs (in the sense of one-node-one-edge) differs by one row and one column only, but unlike node-level, the difference in $L_1-$norm between the adjacency matrices is always one.
Based on Definition~\ref{def:1n1e-graphs}, a one-node-one-edge-level DP is defined as follows:
\begin{definition}[\textbf{$(\varepsilon, \delta)$-One-node-one-edge-level differential privacy}]
	\label{def:nodedp}
	A randomized mechanism $\mathcal{M}$ satisfies \textbf{$(\varepsilon,\delta)$-one-node-one-edge-level DP} with $\varepsilon,\delta \geqslant 0$ if, for all pairs of one-node-one-edge-level adjacent graphs $\mathcal{G},\mathcal{G}'$ and for all measurable $\mathcal{O}\subseteq \Range(\mathcal{M})$,
the following holds:
	$$\oP\{\mathcal{M}(\mathcal{G})\in \mathcal{O}\} \leqslant e^{\varepsilon} \oP\{\mathcal{M}(\mathcal{G}')\in \mathcal{O}\} + \delta$$
\end{definition}


\subsection{Countermeasures for our attack}


In this section, we describe one DP-based strategy, namely the LapGraph mechanism, which was introduced in \cite{linkteller}. A much simpler defense approach against any privacy attack to GNNs would of course be output perturbation~\cite{outputperturbation}, whereby the very same output of the GNN prediction is perturbed with some DP mechanism (e.g., the classical Laplace mechanism). While this solution is straightforward to implement and indeed can be used to satisfy the one-node-one-edge-level DP notion, unfortunately, it would significantly deteriorate the accuracy of the GNN output. 
It is easy to see that the $L_1$-global sensitivity of a prediction matrix for the set of nodes $V_{\mathcal{A}}$ is as large as $2\,|V_{\mathcal{A}}|$, which makes us rule out output perturbation. 

To defend against the newly proposed attack, similar to \cite{linkteller}, we propose to apply the LapGraph algorithm, which consists in perturbing the adjacency matrix using the Laplace mechanism and binarizing it by replacing the top-$N$ largest values by 1 and the remaining values by 0. Here, $N$ represents the estimated number of edges in the graph, which is also computed using the Laplace mechanism. % \ayse{omit}in a DP manner. 

By leveraging the post-processing property of DP\footnote{The post-processing of DP allows arbitrary data-independent transformations to DP outputs without affecting their privacy guarantee~\cite{postprocessing}.}, the edge information remains protected even if the adversary observes the predictions generated by the GNN. Furthermore, 
 
each time a user connects a new node, a new adjacency matrix is generated following the same LapGraph mechanism, accumulating this way the privacy budget by the sequential composition property of DP \cite{dwork2014algorithmic}.

Although the LapGraph mechanism was proposed to meet edge-level DP, it is not difficult to show that the mechanism can also be used to satisfy one-node-one-edge-level DP. 
For this, let $f_A$ be the query function returning the adjacency matrix of a graph $G$. Unlike edge-level neighborhood, the corresponding matrices $A, A'$ of two one-node-one-edge neighboring graphs $G,G'$ have different dimensions, namely, %\ayse{this is not a sentence, should be attached to the previous one}
either $A\in \mathbb{R}^{n \times n}$ and $A'\in \mathbb{R}^{(n+1) \times (n+1)}$, or $A\in \mathbb{R}^{(n+1) \times (n+1)}$ and $A'\in \mathbb{R}^{n \times n}$. Without loss of generality, we assume the former case, where $A$ and $A'$ represent the adjacency matrices \emph{before} and \emph{after} the new node is connected to $G$ (resulting in $G'$). We shall also assume that the new node corresponds to the $(n+1)$-th row and, for symmetry, to the $(n+1)$-th column of $A'$. Precisely, since any adjacency matrix is symmetric by definition, the computation of the sensitivity of $f_A$ only requires the upper or lower triangular matrix of $A$.

To enable the subtraction operation $A-A'$ implicit in the definition of the global sensitivity (see Definition~\ref{def:GS}), we append one zero-row and one-zero column to $A$ and denote the resulting matrix by $\bar{A}\in \mathbb{R}^{(n+1) \times (n+1)}$.
As in the case of $A'$, we assume that the appended row and column are in the %\ayse{parantheses}
$(n+1)$-th position of $\bar{A}$.

To compute the sensitivity of $f_A$ for the notion of one-node-one-edge neighboring graphs, we just need to note that the $(n+1)$-th columns (or rows, if we consider the lower triangular of the adjacency matrix) of $\bar{A}$ and $A'$ always differ in one element. The reason is because one-node-one-edge neighboring graphs differ in only one edge. As a result,
$$\|\bar{A}-A'\|_1=1$$
for any pair of neighboring graphs,
which yields an $L_1$-global sensitivity of 1, as in the original LapGraph mechanism intended for edge-level DP. The fact that the two sensitivities coincide implies that the LapGraph version utilized in this work will provide stronger protection for the same level of utility, compared to the original LapGraph.
The reason for that is because, while the scale of the Laplace noise will be the same for a same $\varepsilon$, one-node-one-edge guarantees indistinguishability between any pair of graphs differing not only in one edge and but also in one node. 
% Figure environment removed


\subsection{LapGraph evaluation}
In this section, we evaluate the effectiveness of LapGraph \cite{linkteller} in reducing the success of  NILS attack while ensuring our one-node-one-edge-level DP notion. We also investigate the utility of GCN models trained with LapGraph protection.

\subsubsection{Evaluation setup:} We use the same training hyperparameters and normalization techniques as in the vanilla case, where DP is not applied. Initially, we protect the training graph with LapGraph. Following that, we apply LapGraph each time the graph changes due to node injection by the adversary.
In line with the setup in \cite{linkteller}, we compute the $F_1$ score for our NILS attack as well as the classification task's $F_1$ score for the GCN. This allows us to measure LapGraph protection along with the GCN utility across various privacy budgets $\varepsilon$. We report the results averaged over 5 runs with different random seeds for LapGraph.

\subsubsection{Evaluation results:} Figure \ref{fig:lapgraph_effectiveness} presents the $F_1$ score of the attack for various $\varepsilon$ values. We observe that applying LapGraph reduces the effectiveness of NILS. The $F_1$ score becomes almost zero when the privacy budget $\varepsilon$ is small. However, for large $\varepsilon$, LapGraph provides moderate protection, but the attack's $F_1$ score remains significantly lower than in the non-private case where DP is not applied.

For comparison, in the LinkTeller \cite{linkteller} attack, where LapGraph is applied only once to ensure edge-level DP, LapGraph offers limited protection when $\varepsilon$ is large, allowing LinkTeller to achieve a success rate nearly as high in the non-private case. Conversely, in our scenario, where LapGraph is also applied after the adversary's node injection, LapGraph provides stronger protection. The application of LapGraph during inference makes it more challenging for the adversary to distinguish between the target node's neighbors and non-neighbors, as the prediction scores of all target nodes change after each inference query.
Consequently, the distances between the prediction scores $P$ and $P'$, before and after the node injection, become noisier due to LapGraph's application following the node injection.

To provide insights about the privacy-utility tradeoff of LapGraph, we present in Figure \ref{fig:lapgraph_utility} the utility of the GCNs for different values of the privacy budget. We observe that the utility increases when $\varepsilon$ increases, as expected. Large values of $\varepsilon \geq 7$ give a better utility close to that in the non-private vanilla case. Therefore, carefully choosing an $\varepsilon$ will give fairly good utility and a certain level of protection against NILS attack.



% Figure environment removed



\section{Conclusion and Future Work}
In this work, I design corruption-robust algorithms for the Lipschitz contextual search problem. I present the \emph{agnostic checking} technique and demonstrate its effectiveness in designing corruption-robust algorithms. There are several open problems for future research. First, in the algorithm I propose for pricing loss, the schedule for agnostic checks is fixed upfront. Can the learner design an adaptive checking schedule for the pricing loss? Second, this work assumes the learner has knowledge of the Lipschitz constant $L$. Can the learner design efficient no-regret algorithms without knowledge of $L$? 

%\clearpage
\newpage
\bibliographystyle{ACM-Reference-Format}
\bibliography{sample-base}
%\bibliography{sample-base}



\end{document}
\endinput
%%
%% End of file `sample-sigconf.tex'.
