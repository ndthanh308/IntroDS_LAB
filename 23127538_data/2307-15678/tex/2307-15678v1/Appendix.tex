\newpage

\section{Appendix}

In the following, we start by presented additional experimental results then present an examination of the datasets we have considered.



\subsection{Additional results}

\begin{table*}[h]
	\caption{Results for real IT monitoring datasets for $\gamma_{max}=15$.  We report the F1-score. } \label{tab:results_real_gm_15_2}
	\centering
	% \resizebox{\columnwidth}{!}{%
	\begin{tabular}{r|c|c|c|c|c|c|c}
		\hline 
		& MoM 1 & MoM 2 &  Ingestion & Web 1 & Web 2 & Antivirus 1 & Antivirus 2\\ \hline
		GCMVL& $0.0$ & $0.0$ & $0.2$  & $0.29$ & $0.0$    &$0.08$ & $0.0$\\
        Dynotears& $0.26$ & $0.2$ & $0.14$& $0.24$ & $\textbf{0.34}$     &$0.19$ & $0.25$\\
		PCMCI$^+$ & $\textbf{0.4}$ & $0.0$  & $0.0$ &  $0.22$ & $0.31$     &$0.1$ & $0.13$\\
		PCGCE & $0.0$ & $0.12$ & $0.12$ &  $\textbf{0.3}$ &   $0.27$      &$0.27$ & $\textbf{0.26}$\\
		VLiNGAM& $0.0$ & $0.0$ & $0.19$ &  $0.24$  & $0.17$     &$0.19$ & $0.16$\\
  		TiMINo& $0.0$ & $0.17$ & $0.18$ & $0.0$ &  $0.13$     &$0.0$ & $0.07$\\
  		NBCB-w& $\textbf{0.4}$ & $0.0$ & $0.13$&  $0.18$
 &    $0.23$     &$0.13$ & $0.19$\\
            NBCB-e& $0.13$ & $\textbf{0.29}$ & $\textbf{0.27}$& $0.19$ & $0.22$     &$0.22$ & $0.15$\\
            CBNB-w& $\textbf{0.4}$ & $0.0$ & $0.15$&  $0.22$ &  $0.29$ &$0.2$ & $0.19$\\
            CBNB-e& $0.0$ & $0.24$ & $0.13$&  $0.23$ &  $0.33$ &$\textbf{0.28}$ & $0.22$
	\end{tabular} 
	% }
\end{table*}


\begin{table*}[h]
	\caption{Results for real IT monitoring datasets for $\gamma_{max}=10$.  We report the F1-score. } \label{tab:results_real_gm_10}
	\centering
	% \resizebox{\columnwidth}{!}{%
	\begin{tabular}{r|c|c|c|c|c|c|c}
		\hline 
		& MoM 1 & MoM 2 & Ingestion & Web 1 & Web 2 & Antivirus 1 & Antivirus 2\\ \hline
        GCMVL& $0.0$ & $0.0 $  & $0.0$ & $\textbf{0.32}$  & $0.0$ & $0.09$ & $0.0$\\
        Dynotears& $\textbf{0.36}$ & $0.14$& $0.14$  & $0.22$ & $\textbf{0.39}$ & $0.18$ & $0.22$ \\ 
        PCMCI$^+$ & $0.0$ & $0.0$ & $0.0$ & $0.22$  & $0.31$ & $0.07$ & $0.14$\\
        PCGCE & $0.0$ & $0.0$ & $0.11$ & $0.27$  & $0.24$ &$ \textbf{0.33}$ & $0.27$\\
        VLiNGAM& $0.27$ & $0.09$ & $\textbf{0.27}$ & $0.22$ & $0.18$ & $0.19$ & $0.16$\\
        TiMINo& $0.0$ & $0.17$ & $0.17$ & $0.0$  & $0.0$ & $0.06$ &  $0.06$\\
        NBCB-w& $0.15$ & $0.0$& $0.13$  & $0.19$ & $0.23$ & $0.15$& $0.25$\\
        NBCB-e& $0.13$ & $\textbf{0.2}$& $0.18$ & $0.25$  & $0.22$ & $0.26$ & $0.21$\\
        CBNB-w& $0.15$ & $0.0$ & $0.16$&  $0.22$ &  $0.29$ &$0.2$ & $0.21$\\
        CBNB-e& $0.0$ & $0.12$ & $0.11$&  $0.21$ &  $0.26$ &$\textbf{0.33}$ & $\textbf{0.29}$

	\end{tabular} 
	% }
\end{table*}



\begin{table*}[h]
	\caption{Results for real IT monitoring datasets for $\gamma_{max}=5$.  We report the F1-score.} \label{tab:results_real_gm_5}
	\centering
	% \resizebox{\columnwidth}{!}{%
	\begin{tabular}{r|c|c|c|c|c|c|c}
		\hline 
		& MoM 1 & MoM 2 & Ingestion & Web 1  & Web 2 & Antivirus 1 & Antivirus 2\\ \hline
		GCMVL& $0.0$ & $0.0$  & $0.0$  & $0.19$    & $0.0$ & $0.08$ & $0.0$\\
        Dynotears& $0.27$ & $\textbf{0.21}$& $0.14$  & $0.22$    & $0.3$ & $0.18$ & $0.17$\\
		PCMCI$^+$ & $0.0$  & $0.15$ & $0.0$  & $0.17$    & $0.32$ & $0.04$ & $0.11$\\
		PCGCE & $\textbf{0.31}$ & $0.0$ & $0.22$   & $0.21$ & $0.34$ & $0.3$ & $0.36$\\
		VLiNGAM& $0.0$ & $0.19$ & $\textbf{0.25}$  & $0.23$  & $0.2$ & $0.18$ & $0.18$\\
  		TiMINo& $0.0$ & $0.0$ & $0.18$  & $0.0$  & $0.0$  & $0.0$ & $0.0$\\
		NBCB-w& $0.0$ & $0.12$& $0.13$  & $0.2$ & $0.23$ & $0.13$ & $0.3$\\
		NBCB-e& $0.27$ & $0.0$& $0.11$  & $\textbf{0.24}$  & $\textbf{0.42}$ & $0.29$ & $\textbf{0.38}$\\
        CBNB-w& $0.0$ & $0.13$ & $0.15$&  $\textbf{0.24}$ &  $0.29$ &$0.18$ & $0.18$\\
        CBNB-e& $\textbf{0.31}$ & $0.0$ & $0.13$&  $0.15$ &  $0.38$ &$\textbf{0.33}$ & $0.27$
	\end{tabular}
	% }
\end{table*}



\begin{table*}[t]
	\caption{Results for real IT monitoring datasets for $\gamma_{max}=3$.  We report the F1-score. } \label{tab:results_real_gm_3}
	\centering
	% \resizebox{\columnwidth}{!}{%
	\begin{tabular}{r|c|c|c|c|c|c|c}
		\hline 
		& MoM 1 & MoM 2 &  Ingestion & Web 1 & Web 2 & Antivirus 1 & Antivirus 2\\ \hline
		GCMVL& $0.0$ & $0.0$ & $0.14$  & $0.2$  & $0.0$    &$0.08$ & $0.0$\\
        Dynotears& $0.14$ & $\textbf{0.3}$ & $0.14$&  $0.23$  & $0.3$     &$0.18$ & $0.19$\\
		PCMCI$^+$ & $0.0$ & $0.0$  & $0.0$ & $0.23$  & $0.3$     &$0.04$ & $0.11$\\
		PCGCE & $\textbf{0.15}$ & $0.0$ & $0.22$ &  $0.22$
 &   $0.15$      & $0.3$ & $\textbf{0.45}$\\
		VLiNGAM& $0.0$ & $0.0$ & $\textbf{0.38}$ & $\textbf{0.29}$
  & $0.18$     &$0.15$ & $0.22$\\
  		TiMINo& $0.0$ & $0.17$ & $0.18$ & $0.0$ &  $0.0$     &$0.0$ & $0.0$\\
  		NBCB-w& $0.0$ & $0.0$ & $0.15$&  $0.23$
 &    $0.3$     &$0.14$ & $0.24$\\
            NBCB-e& $0.14$ & $0.0$ & $0.22$&  $0.19$
   & $\textbf{0.42}$     &$\textbf{0.31}$ & $\textbf{0.45}$\\
       CBNB-w& $0.0$ & $0.0$ & $0.16$&  $0.23$ &  $0.3$ &$0.17$ & $0.16$\\
        CBNB-e& $0.0$ & $0.0$ & $0.13$&  $0.22$ &  $0.29$ &$\textbf{0.31}$ & $0.38$
	\end{tabular} 
	% }
\end{table*}

Tables \ref{tab:results_real_gm_15_2}, \ref{tab:results_real_gm_10}, \ref{tab:results_real_gm_5}, and \ref{tab:results_real_gm_3} present the F1-scores for each method using different values of $\gamma_{max}$ (15, 10, 5, and 3, respectively). Among these methods, GCMVL performs poorly on all datasets, except for Web 1 dataset where it achieves the highest F1-scores of $0.32$, when $\gamma_{max} =$ $10$.  Dynotears demonstrates stable performance across various datasets when $\gamma_{max}$ is varied. It achieves the highest F1-scores on the Web 2 dataset with a large values of $\gamma_{max}$, and on the MoM 2 dataset with a small values of $\gamma_{max}$.

PCMCI$^{+}$ exhibits poor performance on the MoM, Ingestion, and Antivirus datasets, except for the MoM 1 dataset when $\gamma_{max}$ is set to $15$, where it achieves an F1-score of $0.4$. However, PCMCI$^{+}$ shows better performance on the Web datasets. 
PCGCE achieves the highest F1-score on the MoM 1 dataset when $\gamma_{max}=3$ and $\gamma_{max}=5$ however  it F1-score drops to zero for $\gamma_{max}=10$ and $15$. For the MoM 2 dataset, PCGCE has almost always a zero F1-score and for the Ingestion dataset it has relatively a low performance.
However, when it comes to the Web and Antivirus datasets, PCGCE consistently exhibits good performance across all values of $\gamma_{max}$. 
VLiNGAM consistently achieves high F1-scores on the Ingestion dataset for all values of $\gamma_{max}$ except when $\gamma_{max}=3$, and it performs better on the Web and Antivirus datasets compared to the MoM datasets. 
TiMINo performs poorly on the majority of the datasets, but it demonstrates stable performance on the Ingestion dataset regardless of the value of $\gamma_{max}$. It shows a similar conclusion on the MoM 2 dataset, except when $\gamma_{max}$ is set to $5$.
NBCB-w and CBNB-w achieve the best F1-score of $0.4$ on the MoM 1 dataset when $\gamma_{max}$ is set to $15$, and it generally performs better on the Web and Antivirus datasets compared to the other datasets. 
NBCB-e tends to achieve the highest F1-scores in most cases. It achieves the highest F1-scores on the MoM 2 and Ingestion datasets when $\gamma_{max}$ is large, and on the Web and Antivirus datasets when $\gamma_{max}$ is smaller, meanwhile, it should be noted that as $\gamma_{max}$ increases, its performance remains comparative on these datasets.
Similarly, CBNB-e has the best F1-score in Antivirus 1 when $\gamma_{max}$ is set to $10$ and $15$ and in Antivirus 2 when $\gamma_{max}$ is set to $10$.  

In summary, it appears that there is no single method that works well for all datasets. If the value of $\gamma_{max}$ is unknown, NBCB-e and PCGCE are the recommended choice for the Antivirus datasets, as they consistently performs well across these datasets for all values of $\gamma_{max}$. Similarly, NBCB-e, PCGCE and PCMCI$^+$ are the recommended choice for the Web datasets (Dynotears was excluded because as shown in Figures~\ref{fig:web_infer_graph1} and \ref{fig:web_infer_graph2}, it gives almost a fully connected graph for the Web datasets).
For the Ingestion dataset, VLiNGAM is the best choice. Lastly, Dynotears is a better option for the MoM 1 and MoM 2 datasets due to its stability across different values of $\gamma_{max}$. 

However, it is important to note that the best performance achieved ($0.45$ in Antivirus 2) is far from being satisfactory for real world application.


% An alternative nonparametric method called convergence cross mapping (CCM) was proposed by \cite{Sugihara_2012} to detect causal relationships in ecological dynamic systems. 
% CCM is a pairwise method that infers that $X$ causes $Y$ if lagged time series of $Y$ alone can be used to estimate $X$, and the precision of the estimate increase with the amount of data. 
% It is defined for deterministic nonlinear systems. 
% \url{https://github.com/nickc1/skccm/}


% \begin{table*}[h]
% 	\caption{Results for real IT monitoring datasets.  We report the mean and the standard deviation of the F1 score. Non-linear methods $\gamma_{max}=15$} \label{tab:results_real}
% 	\centering
% 	% \resizebox{\columnwidth}{!}{%
% 	\begin{tabular}{r|c|c|c|c}
% 		\hline 
% 		& MoM 1 & MoM 2 & Ingestion & Web Activity\\ \hline
% 		CCM & $0.33$ & $\textbf{0.29}$ & $0.26$ & $0.25$ \\
% 		PCMCI$^+$ & $0.33 \pm 0.04$  & $0.21 \pm 0.01$ & $\textbf{1}$ \\
% 		PCGCE & $0.36 \pm 0.03$ & $0.2 \pm 0.01$ & $\textbf{1}$  \\
% 		NBCB-w& $0.4 \pm 0.03$ & $0.26 \pm 0.01$& $\textbf{1}$ \\
% 		NBCB-e& $0.4 \pm 0.03$ & $0.26 \pm 0.01$& $\textbf{1}$ \\
% 	\end{tabular} 
% 	% }
% \end{table*}


%\newpage
%
%\subsection{Inferred graphs}


In Figures~\ref{fig:controlled_infer_graph_1},\ref{fig:controlled_infer_graph_2},\ref{fig:storm_infer_graph}, \ref{fig:web_infer_graph1}, \ref{fig:web_infer_graph2}, \ref{fig:antivirus_infer_graph1} and \ref{fig:antivirus_infer_graph2} we also give the the inferred graphs that correspond to the results in Table~\ref{tab:results_real_gm_15_1} (where $\gamma_{max}$ is set using the 15 seconds rule for the MoM datasets and using the 15 minutes results for the rest of the datasets).
In general, we can say that there is a lot of false positives and that Dynotears tend to give a fully connected graph while constraint-based and hybrid based methods tend to give sparse graphs. 


% Figure environment removed








% Figure environment removed






% Figure environment removed













% Figure environment removed












% Figure environment removed



% Figure environment removed










% Figure environment removed














\newpage

\subsection{Data examination}


Examination and visualization of time series is useful to observe trends, patterns, and dependencies in the data. By analyzing the data beforehand, we can identify potential behavior change, seasonality, sleeping time series, missing values or other time-dependent effects that may influence the outcomes we are interested in. Abnormal behavior, sleeping time series and misaligned data are a common occurrence in Monitoring data. 





\subsubsection{MoM datasets}

Since MoM dataset was created in a controlled environment, the time series are aligned because sampling is uniformly collected at every second. There are also no missing values in this dataset, and no completely or partially sleeping time series. 


% Figure environment removed

% Figure environment removed





\subsubsection{Ingestion activity dataset}


As mentioned before, all the data are aligned for this dataset with sampling of $1$ minute. Moreover, it contained no missing values upon inspection. Figure~\ref{fig:ingestion_raw_data} contains a clear example of behavior change (highlighted in red). This is particularly interesting because the behavior change occurs in all time series approximately in the same region. There are no completely sleeping time series in this dataset, PMDB and RTMB are partially sleeping. 


% Figure environment removed


\subsubsection{Web activity dataset}



Upon examination of the $10$ time series, it was observed that the timestamps were not exactly aligned. It is noteworthy to mention that there were no sleeping time series observed in this dataset. However, NPP, NetIn and NetOut are partially sleeping. In terms of sampling, all time series had a sampling of $1$ minute. To align all the time series and make them of the same sampling, all the time series were resampled to $5$ minutes using either Strategy 1 or Strategy 2. Upon resampling, RamH and CpuG contained missing values, the maximum number of missing values was $1$ for both. The missing values were filled using simple linear interpolation of Pandas dataframes. It is important to mention that there were missing values in the raw data, but when sampled to a longer sampling the number of missing values were reduced. Afterwards, they were interpolated. 


% Figure environment removed


% Figure environment removed


% Figure environment removed



\subsubsection{Antivirus activity dataset}


This dataset contained $13$ time series in total, and the timestamps were not exactly aligned. Moreover, the raw data contained missing values. There were no completely sleeping time series observed in this dataset, but RP, CUP,RV,CUV and MUP were partially sleeping. In terms of sampling, ChIE, T and ChP had an original sampling of 5 minutes and the rest were $1$ minute. To align all the time series and make them of the same sampling, all the time series were resampled to $5$ minutes using either Strategy 1 or Strategy 2. Upon resampling, four metrics contained missing values. CUGV had $5$ missing values with at most $1$ consecutive missing value. CUV, ChP and MUP had $219$ missing values. However, there were at most $2$ consecutive missing values in these time series, so no large block of missing values was observed. The missing values were filled using simple linear interpolation of Pandas data frames. 

% Figure environment removed


% Figure environment removed


% Figure environment removed


\begin{table*}[ht]
	\caption{Summary of different datasets.} \label{tab:data_summary}
\centering
\begin{tabular}{r|c|c|c|c|c|c}
    \hline 
    & MoM  &  Ingestion  & Web & Antivirus\\ \hline
    Number sleeping time series after pre-processing& $0$ & $0$ & $0$ & $0$\\
    Number of partially sleeping time series after pre-processing & 0 & $2$ & $3$ & $5$ \\
    Sampling rate(s) before pre-processing & $1$ sec & $1$ min & $1$ min & $1$ \& $5$ mins \\
    Contained missing values before pre-processing & No & No & Yes & Yes\\
    Resampled after pre-processing & $1$ sec & $1$ min & $5$ mins & $5$ mins \\
    Number of time series with missing values after resampling & $0$ & $0$ & $2$ & $4$\\
\end{tabular}
\end{table*}
