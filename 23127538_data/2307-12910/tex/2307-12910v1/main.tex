%%%%%%%%%%%%%%%%%%%%%%%%%%%%%%%%%%%
%This is the LaTeX ARTICLE template for RSC journals
%Copyright The Royal Society of Chemistry 2016
%%%%%%%%%%%%%%%%%%%%%%%%%%%%%%%%%%%

\documentclass[twoside,twocolumn,9pt]{article}
\usepackage{extsizes}
\usepackage[super,sort&compress,comma]{natbib}
\usepackage[version=3]{mhchem}
\usepackage[left=1.5cm, right=1.5cm, top=1.785cm, bottom=2.0cm]{geometry}
\usepackage{balance}
\usepackage{mathptmx}
\usepackage{sectsty}
\usepackage{graphicx}
\usepackage{lastpage}
\usepackage[format=plain,justification=justified,singlelinecheck=false,font={stretch=1.125,small,sf},labelfont=bf,labelsep=space]{caption}
\usepackage{float}
\usepackage{fancyhdr}
\usepackage{fnpos}
\usepackage[english]{babel}
\addto{\captionsenglish}{%
  \renewcommand{\refname}{Notes and references}
}
\usepackage{array}
\usepackage{droidsans}
\usepackage{charter}
\usepackage[T1]{fontenc}
\usepackage[usenames,dvipsnames]{xcolor}
\usepackage{setspace}
\usepackage[compact]{titlesec}
\usepackage{hyperref}
%\usepackage{ulem}
\usepackage{amsmath}
%%%Please don't disable any packages in the preamble, as this may cause the template to display incorrectly.%%%

%\usepackage{epstopdf}%This line makes .eps figures into .pdf - please comment out if not required.

%custom-used packages
\usepackage{physics}
\usepackage{wasysym}
\usepackage{xcolor}
\usepackage{amsmath}
\usepackage{amssymb}
\usepackage{comment}
\usepackage{bm}
%\usepackage{mathrsfs}

\definecolor{cream}{RGB}{222,217,201}

\begin{document}

\pagestyle{fancy}
\thispagestyle{plain}
\fancypagestyle{plain}{
    %%%HEADER%%%
    \renewcommand{\headrulewidth}{0pt}
}
%%%END OF HEADER%%%

%%%PAGE SETUP - Please do not change any commands within this section%%%
\makeFNbottom
\makeatletter
\renewcommand\LARGE{\@setfontsize\LARGE{15pt}{17}}
\renewcommand\Large{\@setfontsize\Large{12pt}{14}}
\renewcommand\large{\@setfontsize\large{10pt}{12}}
\renewcommand\footnotesize{\@setfontsize\footnotesize{7pt}{10}}
\makeatother

\renewcommand{\thefootnote}{\fnsymbol{footnote}}
\renewcommand\footnoterule{\vspace*{1pt}%
    \color{cream}\hrule width 3.5in height 0.4pt \color{black}\vspace*{5pt}}
\setcounter{secnumdepth}{5}

\makeatletter
\renewcommand\@biblabel[1]{#1}
\renewcommand\@makefntext[1]%
{\noindent\makebox[0pt][r]{\@thefnmark\,}#1}
\makeatother
\renewcommand{\figurename}{\small{Fig.}~}
\sectionfont{\sffamily\Large}
\subsectionfont{\normalsize}
\subsubsectionfont{\bf}
\setstretch{1.125} %In particular, please do not alter this line.
\setlength{\skip\footins}{0.8cm}
\setlength{\footnotesep}{0.25cm}
\setlength{\jot}{10pt}
\titlespacing*{\section}{0pt}{4pt}{4pt}
\titlespacing*{\subsection}{0pt}{15pt}{1pt}
%%%END OF PAGE SETUP%%%
%\end{}
%%%FOOTER%%%
\fancyfoot{}
\fancyfoot[LO,RE]{\vspace{-7.1pt}% Figure removed}
\fancyfoot[CO]{\vspace{-7.1pt}\hspace{13.2cm}% Figure removed}
\fancyfoot[CE]{\vspace{-7.2pt}\hspace{-14.2cm}% Figure removed}
\fancyfoot[RO]{\footnotesize{\sffamily{1--\pageref{LastPage} ~\textbar  \hspace{2pt}\thepage}}}
\fancyfoot[LE]{\footnotesize{\sffamily{\thepage~\textbar\hspace{3.45cm} 1--\pageref{LastPage}}}}
\fancyhead{}
\renewcommand{\headrulewidth}{0pt}
\renewcommand{\footrulewidth}{0pt}
\setlength{\arrayrulewidth}{1pt}
\setlength{\columnsep}{6.5mm}
\setlength\bibsep{1pt}
%%%END OF FOOTER%%%

%%%FIGURE SETUP - please do not change any commands within this section%%%
\makeatletter
\newlength{\figrulesep}
\setlength{\figrulesep}{0.5\textfloatsep}

\newcommand{\topfigrule}{\vspace*{-1pt}%
    \noindent{\color{cream}\rule[-\figrulesep]{\columnwidth}{1.5pt}} }

\newcommand{\botfigrule}{\vspace*{-2pt}%
    \noindent{\color{cream}\rule[\figrulesep]{\columnwidth}{1.5pt}} }

\newcommand{\dblfigrule}{\vspace*{-1pt}%
    \noindent{\color{cream}\rule[-\figrulesep]{\textwidth}{1.5pt}} }

\makeatother
%%%END OF FIGURE SETUP%%%

%%%TITLE, AUTHORS AND ABSTRACT%%%
\twocolumn[
    \begin{@twocolumnfalse}
        %{% Figure removed\hfill\raisebox{0pt}[0pt][0pt]{% Figure removed}\\[1ex]
        %% Figure removed}\par
        \vspace{1em}
        \sffamily
        \begin{tabular}{m{4.5cm} p{13.5cm} }
            % Figure removed
            & \noindent\LARGE{\textbf{Flow states of two dimensional active gels driven by external shear}
            %\textbf{Effect of externally imposed shear flow on an active gel in a straight channel}
            }\\
            \vspace{0.3cm} & \vspace{0.3cm}\\
            & \noindent\large{Wan Luo$^{\ast}$\textit{$^{ab}$}, Aparna Baskaran\textit{$^{c}$}, Robert A. Pelcovits\textit{$^{de}$}, and Thomas R. Powers\textit{$^{abde\dag}$}}\\
            % Figure removed & \noindent\normalsize{Using a minimal hydrodynamic model, we theoretically and computationally study active gels in straight and annular two-dimensional channels subject to an externally imposed shear. The gels are isotropic in the absence of externally- or activity-driven shear, but have nematic order that increases with shear rate. 
            %straight channel with the bottom wall oscillatory at a fixed velocity along the channel direction. 
             Using the finite element method, we determine the possible flow states for a range of activities and shear rates. Linear stability analysis of an unconfined gel in a straight channel shows that an externally imposed shear flow can stabilize an extensile fluid that would be unstable to spontaneous flow in the absence of the shear flow,
              and destabilize a contractile fluid that would be stable against spontaneous flow in the absence of shear flow. These results are in rough agreement with the stability boundaries between the base shear flow state and the nonlinear flow states that we find numerically for a confined active gel.  For extensile fluids, we find three kinds of nonlinear flow states in the range of parameters we study: unidirectional flows, oscillatory flows, and  dancing flows. To highlight the activity-driven spontaneous component of the nonlinear flows, we characterize these states by the average volumetric flow rate and the wall stress. For contractile fluids, we only find the linear shear flow and a nonlinear unidirectional flow in the range of parameters that we studied. For large magnitudes of the activity, the unidirectional contractile flow develops a boundary layer. Our analysis of annular channels shows how curvature of the streamlines in the base flow affects the transitions among flow states.
            }\\
        \end{tabular}
    \end{@twocolumnfalse} \vspace{0.6cm}
]
%%%END OF TITLE, AUTHORS AND ABSTRACT%%%

%%%FONT SETUP - please do not change any commands within this section
\renewcommand*\rmdefault{bch}\normalfont\upshape
\rmfamily
\section*{}
\vspace{-1cm}


%%%FOOTNOTES%%%
\footnotetext{\textit{$^{\ast}$~Email: Wan\_Luo@brown.edu}}
\footnotetext{\textit{$^{\dag}$~Email: Thomas\_Powers@brown.edu}}
\footnotetext{\textit{$^{a}$~School of Engineering, Brown University, Providence, RI 02912, USA.}}
\footnotetext{\textit{$^{b}$~Center for Fluid Mechanics, Brown University, Providence, RI 02912, USA.}}
\footnotetext{\textit{$^{c}$~Martin Fisher School of Physics, Brandeis University, Waltham, MA 02453 USA.}}
\footnotetext{\textit{$^{d}$~Department of Physics, Brown University, Providence, RI 02912, USA.}}
\footnotetext{\textit{$^{e}$~Brown Theoretical Physics Center, Brown University, Providence, RI 02912, USA.}}


%Please use \dag to cite the ESI in the main text of the article.
%If you article does not have ESI please remove the the \dag symbol from the title and the footnotetext below.
%\footnotetext{\dag~Electronic Supplementary Information (ESI) available: [details of any supplementary information available should be included here]. See DOI: 10.1039/cXsm00000x/}
%additional addresses can be cited as above using the lower-case letters, c, d, e... If all authors are from the same address, no letter is required
%\footnotetext{\ddag~Additional footnotes to the title and authors can be included \textit{e.g.}\ `Present address:' or `These authors contributed equally to this work' as above using the symbols: \ddag, \textsection, and \P. Please place the appropriate symbol next to the author's name and include a \texttt{\textbackslash footnotetext} entry in the the correct place in the list.}


%%%END OF FOOTNOTES%%%


\newcommand\trp[1]{#1}
\newcommand\rap[1]{#1}
\newcommand\wl[1]{#1}
 %\newcommand{\trp}[1]{{\textcolor{blue}{#1}}}
 \newcommand{\trpb}[1]{{\textcolor{blue}{#1}}}
 %\newcommand{\rap}[1]{{\textcolor{red}{#1}}}
 %\newcommand{\wl}[1]{{\textcolor{orange}{#1}}}
% \newcommand{\btheta}{\mbox{\boldmath$\theta$}}
% \newcommand{\bTheta}{\mbox{\boldmath$\Theta$}}
\newcommand{\abeds}[1]{{\textcolor{black}{#1}}}







% body of paper here - Use proper section commands
% References should be done using the \cite, \ref, and \label commands
%%%%%%%%%%%%%%%%%%%%%%%%%%%%%%%%%


%%%%%%%%%%%%%%%%%%%%%%%%%%%%%%%%%


\section{Introduction}
\label{sec:introduction}






%3 previous studies use confinement to add an element of control; microfluidics~\cite{norton2018insensitivity,opathalage2019self, samui2021flow,hardouin2019reconfigurable}; discus Chandrakar here
%4 us: externally applied shear via moving walls; contrast with Samui~\cite{samui2021flow}
%5 distinction between movers--self-propelled particles---and shakers---reconstituted cytsol; we focus on the latter.
%6 active gel vs active nematic
%7 contrast with inertia-driven instabilities of Newtonian flows
%7 outline paper

The defining property of an active fluid is that energy is added to the system at the  small length scales of the particles that make up the fluid, instead of at the large length scales of the bounding walls or inlets of the system.\cite{marchetti2013hydrodynamics}
Commonly studied examples include cytoplasm~\cite{GoldsteinTuvalvandeMeent2008} or its reconstituted components,\cite{NedelecSurreyMaggsLeibler1997,sanchez2012spontaneous,Alvarado2017} collections of swimming microorganisms,\cite{RiedelKruseHoward2005,Koch2011,Saintillan2013} and model two-dimensional  layers of cells.\cite{Duclos_etal2016}
%Active matter storage and dissipate the energy from a microscopic scale of the constituent particles, distinct from the conventional low Reynolds number flows consuming energy at a level of macroscopic structure scale. \cite{sanchez2012spontaneous,marchetti2013hydrodynamics} 
%The individual energy can produce a self-propulsion of the active particles %by internally applying a stress field on the surrounding
%medium and locally driving the system far from a equilibrium %state.\cite{marchetti2013hydrodynamics,bowick2022symmetry} Through these long-%range interactions, 
The interplay of the energy injected at small scales and the interactions among the constituent particles lead to nonequilibrium collective behavior, including spontaneous coherent flows,\cite{woodhouse2012spontaneous,lushi2014fluid,wu2017transition} sustained oscillations,\cite{marchetti2013hydrodynamics,samui2021flow} active turbulence,\cite{wensink2012meso,dombrowski2004self,dunkel2013fluid} and two-dimensional\cite{sanchez2012spontaneous} or three-dimensional~\cite{simha2002hydrodynamic,vcopar2019topology}
topological defects in active liquid crystalline fluids.  These phenomena suggest that active fluids may be used for novel microfluidics applications, including fluids that pump themselves or mix themselves. Since these applications require a degree of control over active fluids, recent investigations have studied how confinement of active fluids affects flows and the formation of defects.\cite{Arajo2023,norton2018insensitivity,opathalage2019self, samui2021flow,hardouin2019reconfigurable}
In this paper, we build on %and extend 
these investigations by studying the flow states of an active gel in a channel with moving boundaries to see how an imposed shear affects the possible flow states and the transitions among them.  

By `active gel' we mean a model liquid crystal which tends to the isotropic phase away from boundaries with strong anchoring conditions and in the absence of shear flow. The motionless, isotropic state of an unbounded two-dimensional active gel is unstable to spontaneous flow and nematic ordering above a critical activity.\cite{Soni2018,Santhosh_etal2020} Recent numerical calculations have identified the spontaneous flow states in straight three-dimensional~\cite{VargheseBaskaranHaganBaskaran2020,chandrakar2020confinement} and two-dimensional channels~\cite{VargheseBaskaranHaganBaskaran2020,samui2021flow,chandragiri2020flow} with stationary walls. In a two dimensional channel with no-torque anchoring conditions at the  walls, the critical activity for spontaneous flow increases as the channel width decreases.\cite{VargheseBaskaranHaganBaskaran2020} Thus, confinement is stabilizing, as has been found in other related situations.\cite{wioland2016directed} For a given value of the activity parameter, new flow states emerge as the channel width increases, with the flow progressing through unidirectional, undulating (also known as `oscillatory',\cite{samui2021flow}) and dancing flow states.\cite{shendruk2017dancing,VargheseBaskaranHaganBaskaran2020} A similar sequence of flow states is found for fixed channel width and increasing activity.\cite{VargheseBaskaranHaganBaskaran2020}


Our work is  motivated by the experimental observation that imposed shear can prevent~\cite{chandrakar2020confinement} the spontaneous instability of a solution~\cite{wu2017transition} of microtubule bundles and kinesin motors in the presence of the molecular fuel ATP.  Instead of a motionless state, our base state is the state of simple shear in which the flow field is given by the solution to the Stokes equation for our straight or annular channel geometry. Working at fixed channel width, we find that increasing the activity leads to a sequence of flow states which are reminiscent of the ones seen in the case of no external shear, but with some important new elements. For example, the imposed shear rate can be stabilizing in the same sense that confinement is stabilizing: for an extensile active gel, we find that the critical activity for the imposed simple shear flow to develop a spontaneous flow component increases with the imposed shear rate. A similar result was %{pointed} \trp{out via} %through 
established using 
linear stability analysis of a polar system by Muhuri, Rao, and Ramaswamy.\cite{MuhuriRaoRamaswamy2007} Here we give a more systematic treatment of this problem for the apolar case, revealing that the imposed shear also leads to oscillatory behavior in the unstable modes. For a contractile active gel, we find that shear is \textit{destabilizing}. Earlier work has also examined the rheology of active nematics and gels, showing that polar active particles have a nonmonotonic stress-strain relation at high activity,\cite{GiomiLiverpoolMarchetti2010} and illuminating the nature of shear banding in apolar active gels.\cite{Fielding2011}  Our work extends these investigations to the case of an annular channel, illustrating the role of the curvature of the streamlines of the base flow. 

%TP: DO WE SEE S-SHAPED CURVES IN THE WALL STESS VS SHEAR RATE CURVE? 

%% Figure environment removed


%Previous studies showed that confinement of active fluids generates distinct flow and defect states, ranging from net coherent states to active turbulent states, which have the potential for novel microfluidics applications.\cite{norton2018insensitivity,opathalage2019self, samui2021flow,hardouin2019reconfigurable}
%Much of the experimental and theoretical work measures or computes the hydrodynamics of active matter in channel-confined geometries in the absence of any external driving. Samui et al. \cite{samui2021flow} and Wiolandet al. \cite{wioland2016directed} numerically observed that confinement can stabilize two-dimensional channel fluid. Also, experimental results by Hardouin et al. indicate that active stresses destabilize the order with topological defects.\cite{hardouin2019reconfigurable} 
%Additionally, the appearance of a net flow in active nematics and the transition of extensile rodlike flows from turbulence to coherence depend on the aspect ratio of the channel cross section.\cite{chandragiri2020flowrgheseBaskaranHaganBaskaran2020}
%Chandrakar et al. experimentally and theoretically investigated the relation of the wavelength of an activity driven hydrodynamic instability and its growth rate to the channel dimensions for a extensile fluid~\cite{chandrakar2020confinement}.



%An understanding of the transport of active fluids subject to externally driven flows is not complete. Early studies showed that polar active particles have a nonmonotonic stress-strain relation at high  activity.\cite{GiomiLiverpoolMarchetti2010} A more recent work showed shear-thickening properties of a contractile active nematic subject to a Poiseuille flow. \cite{MackayTonerMorozovMarenduzzo2020}
%Muhuri, Rao, and Ramaswamy \cite{MuhuriRaoRamaswamy2007} recently showed that imposing a sufficiently high shear rate on polar systems can suppress the bend instability. 

%In this work we further develop the effect of the externally imposed shear flow on active matter in an low-concentration isotropic phase (active gels) in a two dimensional straight channel. One example of active gels is a suspension composed of %filamentous microtubules, clusters of kinesin molecular motors, and depleting polymers
%extensile microtubule bundles and ATP-fueled kinesin molecular motors\cite{wu2017transition}. 
Our paper begins with %theoretically describe the active gels by 
a minimal hydrodynamic model for active gels. 
We then study the linear stability of an active gel in a straight channel subject to a uniform shear flow imposed by a moving plate. In the stable region, the linear rheology, orientational order, and the shear stress exerted by the active fluids on the moving boundary are analytically calculated for the state of uniform shear. Then we turn to the other flow states %Nonlinear spontaneous flow states are numerically 
using the finite element method to characterize the flow transitions %both 
for the extensile and contractile fluids. Next, we turn to an annular channel and carry out similar analytical and numerical studies to assess the effects of the curvature of the boundaries.


%Furthermore, a key element of theories of active matter is the active component of the stress, which drives the novel phenomena mentioned above. One of the big questions in the field of active matter is, “What is the value of the active stress for a particular system?” To answer this question, we propose to determine the active stress for active motor-microtubule gels by computing the critical imposed shear rate for flow transitions.




\section{Minimal hydrodynamic model}
\label{model}
%For the hydrodynamic and stability problems we consider here, effective continuum theories are a more suitable choice compared to continuum kinetic theory approaches.\cite{SaintillanShelley2007,SaintillanShelley2008,SaintillanShelley2013,Saintillan2018}
We use a well-\trp{studied} %established 
continuum hydrodynamic model
for nematic liquid crystals~\cite{olmsted1992isotropic,toth2002hydrodynamics} to describe apolar microtubules, adding a term corresponding to non-equilibrium active forces as was done in the "minimal" model \trp{used} %introduced 
by Varghese et al.\cite{VargheseBaskaranHaganBaskaran2020}
%We study the apolar microtubules bundles which have a symmetry upon rotating by 180$^\circ$ to swap the head and tail.
In two dimensions, the orientational order of %two-dimensional 
apolar active matter is described by a traceless, symmetric tensor---the %same 
tensor order parameter that is used in the theory of nematic liquid crystals---$Q_{ij}=S( 2 n_i n_j-\delta_{ij})$, with $i,j=x,y$.~\cite{deGennesProst} %\wl{in a two dimensional Cartesian coordinate or $i,j=r,\theta$ in a polar coordinate}, which 
%The tensor $Q_{ij}$ is %identical to 
%the tensor order parameter used in the theory of nematic liquid crystals~\cite{deGennesProst}. 
The unit vector $\mathbf{n}(\mathbf{x})$ is the director at position $\mathbf{x}$ and the scalar order parameter $S$ represents the degree of alignment.
 %Can write the tensor order parameter in terms of the director and a magnitude $S$, which we denote as the scalar order parameter:
%Note that one of the eigenvectors of $Q_{\alpha\beta}$ is $n_\alpha$ with eigenvalue $S(d-1)$. Therefore,  if we know $Q_{\alpha\beta}$, we can construct the director field and the scalar order parameter field.  
The equilibrium state of the microtubule  bundles is governed by a Landau-Ginzburg free energy density,
\begin{equation}
\begin{aligned}
\mathcal{F}&=\frac{K}{2}\partial_i Q_{jk}\partial_i Q_{jk}+\frac{A}{2}Q_{ij}Q_{ij}+\frac{C}{4}\left(Q_{ij}Q_{ij}\right)^2,
\end{aligned}
\label{LG}
\end{equation}
%\frac{B}{3}Q_{ij}Q_{jk}Q_{jk}+
where repeated indices are summed over. % $x$ and $y$.
%Depending on the sign (and magnitude) of $A$, the minimizing state is either isotropic ($S=0$) or nematic ($S\neq0$). 
The single Frank elastic constant $K$ penalizes gradients of $Q_{ij}$.  Since we focus on a low concentration isotropic phase, %of active matter, 
$A$ will be positive to guarantee that the minimizing state is disordered.
In two dimensions there is no term cubic in  $Q_{ij}$, and the isotropic-nematic transition is continuous. In the isotropic phase we consider in this paper, the term proportional to $C$ can be neglected, as was done in
%In the nematic phase, this constant leads to a Frank elastic constant $K$ proportional to $S^2L$ that penalizes gradients in the director $\mathbf{n}$. 
%We have simplified the presentation by only including one of the possible elastic constants.
previous studies of two-dimensional and three-dimensional channel flow.\cite{VargheseBaskaranHaganBaskaran2020,chandrakar2020confinement}
%Relaxation to equilibrium is driven by a tensor molecular field, which is given by a variational derivative of the Landau-Ginzburg free energy. the molecular field is given by  $H_{\alpha\beta}=-AQ_{\alpha\beta}+L\nabla^2Q_{\alpha\beta}$ which is the entropic force that drives the order to zero when the isotropic state is perturbed by a flow or an external force.
%The liquid crystalline molecular field $H_{\alpha\beta}=-AQ_{\alpha\beta}+L\nabla^2Q_{\alpha\beta}$ is given by a variational derivative of the Landau-Ginzburg free energy.
%It is the entropic force that governs the relaxation of the orientational order from a perturbation imposed by an external shear flow.

%SINCE WE JUST PRESENT THE MODEL AND REFERENCE VARGHESE, DO WE REALLY NEED THE MOLECULAR FIELD SENTENCE?

A minimal hydrodynamic model for incompressible flow in two dimensions is given by\cite{VargheseBaskaranHaganBaskaran2020}
\begin{eqnarray}
0&=&\boldmath{\nabla}\cdot\mathbf{v} \label{incompress}\\
0&=&-\boldsymbol{\nabla}p+\eta\nabla^2\mathbf{v}-a\nabla\cdot\textsf{Q}\label{veqn}\\
0&=&-\nu\big(\partial_t\textsf{Q}+ \textbf{v} \cdot \nabla \textsf{Q}+\textsf{Q} \cdot \Omega-\Omega \cdot \textsf{Q}\bigr)-A\textsf{Q}\nonumber+K\nabla^2\mathsf{Q} \\
&+&2 \lambda\nu \mathsf{E},\label{Qeqn}
\end{eqnarray}
%+\mathsf{Q}\cdot\mathsf{E}+\mathsf{E}\cdot\mathsf{Q}-\mathrm{tr}(\mathsf{Q}\cdot\mathsf{E})\mathsf{I}
where %$\rm{tr}$ is the trace of the tensor, 
$\eta$ is the shear viscosity, $\nu$ is the rotational viscosity, $p$ is pressure, $(\mathbf{v}\cdot\grad{\mathsf{Q}})_{ij}=v_k \partial_k Q_{ij}$,  $\mathsf{E}=(\nabla \mathbf{v}+(\nabla \mathbf{v})^{\rm{T}})/2$ is the strain rate tensor, $\mathsf{\Omega}=(\nabla \mathbf{v}-(\nabla \mathbf{v})^{\rm{T}})/2$ 
[i.e. $\Omega_{ij}=(\partial_j v_i-\partial_i v_j)/2$]
is the vorticity tensor, and $a$ is the strength of the activity. A positive value of $a$ corresponds to extensile particles, and a negative value of $a$ corresponds to contractile particles. The shape parameter $\lambda$ is positive for prolate particles and negative for oblate particles; $\lambda=1$ corresponds to needle-like particles. %. When we present our numerical results we will consider needle-like particles with $\lambda=1$, but in the next section we present some analytical results for general values of $\lambda$. 
Note that in three dimensions there will be additional nonlinear terms proportional to $\lambda$ appearing in eqn~(\ref{Qeqn}).

We disregard inertial effects because the Reynolds number of the typical active flows we study is small. In this minimal hydrodynamic model, passive backflow effects are neglected and the order parameter field $\mathsf{Q}$ only affects the flow through the active stress $-a\mathsf{Q}$. 
% Figure environment removed
%It is straightforward to add to our theory an advection-diffusion equation for the concentration of the bundles.
The active time scale which results from the competition between viscosity and activity is given by $\eta/|a|$.
%Note that the shear viscosity $\eta$ and the activity parameter $a$ define a natural characteristic time scale, %$\tau_\mathrm{active}=\eta/|a|$. 
%I DON'T SEE WHY WE NEED TAU-ACTIVE; WE NEVER USE IT. 
From the dynamical equation for $\mathsf{Q}$, eqn~(\ref{Qeqn}), it is apparent that the relaxation time $\tau$ for distortions away from the equilibrium isotropic %liquid crystal 
state is $\tau=\nu/A$. Likewise, $\sqrt{K/A}$ is a correlation length for the liquid crystalline order, which we write in nondimensional form as $\ell=\sqrt{K/A}/W$\rap{, where $W$ is the width of the straight or annular channel}.
%Likewise, $\ell_\mathrm{active}=\sqrt{L/|a|}$ defines an active length scale, and $\ell=\sqrt{L/A}$ is a correlation length for the liquid crystalline order. SHOULD WE DELETE THE PREVIOUS SENTENCE? WE USE THE CHANNEL WIDTH AS THE UNIT OF LENGTH, SO IT'S CONFUSING TO MENTION THE OTHER LENGTH SCALES UNLESS WE USE THEM LATER. 
The factor $\lambda\nu$ characterizes the flow birefringence of a passive ($a=0$) liquid crystal.\cite{DeGennes1969}
When weak shear $\dot{\gamma}\ll1/\tau$ is applied to a nematic liquid crystal in the isotropic state, the rods align such that $A\mathsf{Q}\approx2\lambda\nu\mathsf{E}$, which implies that the scalar order parameter is proportional to the shear rate: $S\propto\dot{\gamma}\tau$.


%We use the width of the channel $W$ as the unit for length and use the relaxation time of the liquid crystal $\nu/A$ as the unit for time. It is well-known that the motionless isotropic state is unstable to spontaneous shear flow and the development of orientational order in an infinite domain when $a\lambda >\eta A /\nu$~(e.g.~\cite{Soni2018}). Thus, it is natural to define the dimensionless activity coefficient as $\bar{a}=\lambda\nu a/(\eta A)$. Also, we introduce the dimensionless correlation length $\ell=\sqrt{K/A W^2}$, which is the dimensionless characteristic length scale for the variation of the order parameter. Since we are interested in the small Frank elasticity case, $\ell=0.1$ in the rest of the article.
%Note that in two dimensions, the eigenvalues of $\textsf{Q}$ are $\pm S$ with corresponding eigenvectors $\mathbf{n}=(\cos\theta,\sin\theta)$ and $\mathbf{n}^\perp=(-\sin\theta,\cos\theta)$.
%The eigenvector of $\textsf{Q}$ corresponding to the eigenvalue $S$ is the director $\mathbf{n}$. Denote by $S_0$ the value of the scalar order parameter on the disk.

%We impose no-slip boundary conditions on the velocity. For the order parameter, there are various possibilities to consider. The simplest from the theoretical point of view is the natural boundary condition of zero generalized torque, i.e. $\boldmath{\nabla}\mathsf{Q} \cdot \mathbf{n}=0$ at the particle surface where $\mathbf{n}$ is the normal vector of the surface. In this case, the order parameter vanishes if the activity is below the critical value and the particle is motionless. We can also demand planar or homeotropic boundary conditions, in which the director is parallel or perpendicular to the particle surface, respectively, and the scalar order parameter takes some prescribed value. One could also impose an anchoring potential, but we will not do so here.






\section{Straight channel: start-up problem and linear stability analysis}
\label{stability}

%\sout{Before considering a straight channel of finite width,} 
\abeds{Let us begin by} %\sout{we} 
review\abeds{ing} the linear stability analysis of an unbounded two-dimensional active gel \cite{hatwalne04}. 
%\sout{Linear stability analysis of the minimal hydrodynamic model shows that} 
An isotropic ($\mathsf{Q}=0$),  motionless ($\mathbf{v}=0$) gel is unstable to shear flow and nematic ordering when the effective shear viscosity ($\eta_\mathrm{eff}\equiv\eta-a \lambda \tau$) vanishes, which occurs for a critical activity $a_\mathrm{c}=\eta/(\lambda\tau)$.\cite{Soni2018,Santhosh_etal2020} %\rap{I think we are missing a factor of $\lambda$ in the denominator of $a_\mathrm{c}$ and in the $a\tau$ term in $\eta_{eff}$} 
The form of the effective shear viscosity shows that extensile particles tend to reduce the shear viscosity, whereas contractile particles tend to increase it. %\sout{, in agreement with Hatwalne et al. .}
In the  unstable state of the unconfined geometry, the pattern of alignment of the bundles follows a sine wave, appearing like a bent filament, or like the nematic configuration of bend.\cite{deGennesProst}
%A\eta/(\lambda\nu)$
%Note that the critical condition is when $\tau_\mathrm{active}\approx\tau_\mathrm{LC}$, assuming $\lambda\nu\approx\eta$.%
%In an  and thus this instability is usually called the ``bend instability''.\cite{MuhuriRaoRamaswamy2007} %When the active gel is confined to a two-dimensional channel (see fig. \ref{fig:eg}), the isotropic motionless state is unstable to the unidirectional flow state at the critical activity $a_\mathrm{c}$.


%The base state isotropic ($\mathsf{Q}=0$) and motionless ($\mathbf{v}=0$) is subject to a small perturbation with uniform nonzero $\mathsf{Q}$ and constant shear, we linearize the governing equations and eliminate $\mathsf{Q}$ from eqn~(\ref{veqn})

Next, \abeds{let us} consider %linear stability of 
an active gel confined to an infinite straight channel of %finite 
width $W$ and subject to a steady uniform shear flow $\mathbf{v}_0=\dot{\gamma}(W-y)\hat{\mathbf{x}}$ as shown in Fig. \ref{fig:eg}. We assume no-slip boundary conditions on the channel walls for the velocity field, and Neumann conditions, ($\partial_i Q_{jk}=0$) or ``zero-torque conditions"  for the order parameter field on the walls. 
%IT LOOKS LIKE THE NEXT SECTION HAS W SO SHOULD THE NEXT SENTENCE BE DELETED? We use the width of channel $W$ as our unit of length. 
Given the parallel planar channel walls and zero-torque  boundary conditions, the nematic order parameter is uniform and divergenceless for the imposed uniform shear flow. In our hydrodynamic model, activity only appears in eqn (\ref{veqn}), and thus, when activity is below the critical value for the instability, the order parameter field is unaffected by the activity. 

% Figure environment removed
%% Figure environment removed
Before considering the stability of simple shear flow, we solve the startup problem, assuming an initially stationary isotropic gel with activity below the critical value (to be deduced below). Since the Reynolds number is assumed to be small, the flow immediately assumes its steady-state value $\mathbf{v}_0$. But the order parameter field attains its steady-state value only after a time comparable to the liquid crystal relaxation time $\tau$.~\cite{KriegerDiasPowers2015} Given the boundary conditions on the order parameter, we may assume that $\mathsf{Q}$ is uniform in space. Since $\mathsf{Q}$ is uniform, the divergence of the active stress vanishes and the flow remains simple shear as the order-parameter field evolves. The order parameter equations~(eqn (\ref{Qeqn})) reduce to 
\begin{eqnarray}
\partial_t Q_{xx}&=&-\frac{1}{\tau}Q_{xx}-\dot{\gamma} Q_{xy}\\
\partial_t Q_{xy}&=&\dot{\gamma} Q_{xx}-\frac{1}{\tau}Q_{xy}-\lambda\dot{\gamma}.
\end{eqnarray}
Assuming $\mathsf{Q}(t=0)=0$,  we find
\begin{eqnarray}
    Q_{xx}&=&Q^{(0)}_{xx}\left[1-\mathrm{e}^{-t/\tau}\cos\left(\dot{\gamma} t \right)\right]
+Q^{(0)}_{xy}\mathrm{e}^{-t/\tau}\sin\left(\dot{\gamma} t \right)\label{Qxxstartup}\\
Q_{xy}&=&Q^{(0)}_{xy}\left[1-\mathrm{e}^{-t/\tau}\cos\left(\dot{\gamma} t \right)\right]
-Q^{(0)}_{xx}\mathrm{e}^{-t/\tau}\sin\left(\dot{\gamma} t \right),\label{Qxystartup}
\end{eqnarray}
where the steady-state order parameter tensor $\mathsf{Q_0}$ is given by
%The nematic tensor $\textsf{Q}_0$ in the base state is then only a function of $\dot{\gamma}$ and substituting  $\bf{v_0}$ into eqn~(\ref{Qeqn}) we find,
\begin{eqnarray}
Q^{(0)}_{xx}&=&\frac{\lambda\dot{\gamma}^2\tau^2}{1+\dot{\gamma}^2\tau^2},\label{Qxx0}\\
Q^{(0)}_{xy}&=&-\frac{\lambda\dot{\gamma} \tau}{1+\dot{\gamma}^2\tau^2}.\label{Qxy0}
\end{eqnarray}
The order parameter rises to its steady state, with oscillations that become apparent when the shear rate is greater than the relaxation rate $1/\tau$. These oscillations are reminiscent of the oscillations observed~\cite{GuJamiesonWang1993} in the apparent viscosity during the startup flow of 8CB, a director-tumbling nematogen.\cite{larson1999} In simple shear, the director of a tumbling nematic makes a complete revolution, like a rod undergoing a Jeffery orbit in shear flow.\cite{larson1999}  In our case, as long as $\tau$ is finite, the directors oscillate about their final steady state. Fig.~\ref{fig:oscQ} shows the %scalar order parameter $S=\sqrt{Q_{xx}^2+Q_{xy}^2}$ and the 
director angle $\phi=\arctan[Q_{xy}/(S+Q_{xx})]$ (measured counterclockwise from the $x$-axis) as a function of time. %\trp{If we don't state the sense of positive $\phi$, then I think the arrow on the angle $\phi$ in Fig. 2 is potentially confusing, since it might imply the sense of increasing $\phi$. Should we take the arrow out?}
%section{Linear viscous flows}
%In fact, this base state is also the state for values of activity less than the critical value $a_c$, because in this range of activity the divergence of $\textsf{Q}$ is zero and thus the velocity field is a simple shear flow state. Therefore, 

The steady-state scalar order parameter and the director angle %of the director with the $x$ axis 
are given by
\begin{eqnarray}
S&=&%\sqrt{Q_{0,xx}^2+Q_{0,xy}^2}=
\frac{\lambda\dot{\gamma} 
\tau}{\sqrt{1+\dot{\gamma}^2\tau^2}}\label{eq:S}\\ %,\quad a<a_c,\label{eq:S}\\
\phi %&=&\arctan(\frac{Q_{0,xy}}{S+Q_{0,xx}})\nonumber\\
&=&-\arctan \left(\frac{1}{\sqrt{1+\dot{\gamma}^2 \tau^2}+\dot{\gamma}\tau}\right ).\label{eq:phi} %\quad a<a_c.
\end{eqnarray}
Equations~(\ref{eq:S}) and (\ref{eq:phi}) show that in steady state, the flow aligns the nematic director at a nonzero angle with the horizontal streamlines, with a degree of order that increases with increasing shear rate. %As Fig.~\ref{fig:steadyQ} shows, 
At low shear rates, $\dot{\gamma}\tau \ll 1$, the bundles are oriented at an angle of %$-45^\circ$ 
$\phi=-\pi/4$ with the streamlines, and the order is weak ($S\ll1)$. At high shear rates, the bundles tend to align parallel to the streamlines, and $S\approx\lambda$. For needle-like particles, with $\lambda\approx 1$, the order is strong in the limit of high shear rate.
%the effective shear viscosity $\eta_{\rm{eff}}$ is $\eta-\lambda \nu a/A$ for weakly active systems we study. Activity lowers the linear bulk viscosity of extensile gels and increases the viscosity of contractile gels. 
%In the regime of linear rheology,  given the effect of activity on the shear viscosity ($\eta_\mathrm{eff}=\eta-\lambda\nu a/A$), it is natural to predict that the shear stress imposed by active flow on the bottom wall increases with increasing activity in contractile flows and decreases with increasing activity in  extensile flows, consistent with the results of Giomi. \cite{giomi2010sheared}) This is verified by an analytical calculation of 
The shear stress on the moving plate in the stable region is 
\begin{equation}
\sigma_{\rm{W}}=-\eta \dot{\gamma}-a Q^{(0)}_{xy}=\dot{\gamma}\left(-\eta+\frac{a\lambda \tau}{1+\dot{\gamma}^2\tau^2}\right),\quad a<a_c. \label{wallshear}   
\end{equation}
%In the figure \ref{fig:wallshear}, 
From eqn~(\ref{wallshear}), it is easy to see the wall shear stress increases linearly with activity but the dependence on the imposed shear is not linear when the activity is below the critical value.
%and the change of the wall shear stress at low externally imposed shear is more significant than at high external shear.
%This has been validated with numerical results.

%% Figure environment removed


To analyze the %linear 
stability 
of %this 
the base configuration %characterized by 
with flow rate $\mathbf{v}_0$ \abeds{and the confinement $W$} %\sout{and tensor order parameter $\mathsf{Q}_0$}
, 
%in order to avoid the convective term $\mathbf{v} \cdot \nabla \textsf{Q}$ 
we consider %a $y$ dependent convective term, where the perturbation in the velocity is along $x$, but all variations are along $y$. 
a perturbation that is independent of $x$, \abeds{the channel axis}. \footnote{A more general assumption would be to suppose the perturbation depends on both $x$ and $y$, but here we forbid $x$-dependence to simplify the analysis. The more general analysis using pseudospectral methods will be reported elsewhere.}  
Thus, $\mathbf{v}=\mathbf{v}_0+ \mathbf{v}_1$ and $\mathsf{Q}=\mathsf{Q}_0+ \mathsf{Q}_1$ , with the perturbations
\begin{eqnarray}
  \mathbf{v}_1&=& %\hat{v}_x v_n(y)\exp\left(\beta t\right) \hat{\mathbf x},\label{dimensionv}\\
  v_x \sin(n\pi y/W)\exp\left(\beta t\right) \hat{\mathbf x},\label{dimensionv}\\
  \mathsf{Q}_1&=&\begin{pmatrix} \mathcal{Q}_{xx} & \mathcal{Q}_{xy}\\ \mathcal{Q}_{xy} & -\mathcal{Q}_{xx}\end{pmatrix} \cos(n\pi y/W)\exp\left(\beta t\right),
\end{eqnarray}
where $v_x$, $\mathcal{Q}_{xx}$, and $\mathcal{Q}_{xy}$ are constants, $n$ is a nonzero positive integer, and $\beta$ is the growth rate of the perturbation.
%$v_n(y)=\sin(n \pi y /W)$ and $Q_n(y)=\cos \left (n\pi y /W\right )$ with . 
With these assumptions, the $x$ component of the force equation eqn~(\ref{veqn}) implies 
\begin{equation}
{v}_x=\frac{a \mathcal{Q}_{xy} W}{n \pi \eta}.\label{vxitoQxy}
\end{equation}
%Using linear stability analysis, we obtain the growth rates of the perturbation:
Using eqn (\ref{vxitoQxy}) in the linearized equations for $\mathsf{Q}_1$ yields
%\begin{eqnarray}
%\beta_1&=&\frac{a-\sqrt{a^2-4 a \dot{\gamma}^2 \eta  \tau  \left(\dot{\gamma}^2 \tau ^2+1\right)-4 \dot{\gamma}^2 \eta ^2 \left(\dot{\gamma}^2 \tau ^2+1\right)^2}}{2 \left(\dot{\gamma}^2 \eta  \tau ^2+\eta \right)}\nonumber\\
%&-&\frac{\pi ^2 K n^2}{\nu W^2}-\frac{1}{\tau }\\
%\beta_2&=&\frac{a+\sqrt{a^2-4 a \dot{\gamma}^2 \eta  \tau  %\left(\dot{\gamma}^2 \tau ^2+1\right)-4 \dot{\gamma}^2 \eta ^2 %\left(\dot{\gamma}^2 \tau ^2+1\right)^2}}{2 \left(\dot{\gamma}^2 \eta  \tau %^2+\eta \right)}\nonumber\\
%&-&\frac{\pi ^2 K n^2}{\nu W^2}-\frac{1}{\tau }
%\end{eqnarray}
\begin{eqnarray}
\beta_\pm&=&-\frac{\trp{1}}{\tau}\left(1+\frac{\pi ^2 K n^2}{A W^2}\right)+\frac{\lambda a}{2\eta(1+\dot{\gamma}^2  \tau ^2)}
\nonumber\\
&\pm&
\sqrt{\left[\frac{\lambda a}{2\eta(1+\dot{\gamma}^2\tau^2)}\right]^2-\dot{\gamma}^2\left(1+\frac{\lambda a\tau/\eta}{1+\dot{\gamma}^2\tau^2}\right)}.\label{betapm}
%\frac{\sqrt{a^2-4 a \dot{\gamma}^2 \eta  \tau  \left(\dot{\gamma}^2 \tau ^2+1\right)-4 \dot{\gamma}^2 \eta ^2 \left(\dot{\gamma}^2 \tau ^2+1\right)^2}}{2 \eta\left(\dot{\gamma}^2  \tau ^2+1 \right)}
\end{eqnarray}
%In the limit case when there is no external shear, the two growth rates can be simplified to 
%\begin{eqnarray}
%\beta_1(\dot{\gamma}=0)&=&-\frac{\pi ^2 K n^2}{\nu  W^2}-\frac{1}{\tau },\\
%\beta_2(\dot{\gamma}=0)&=&\frac{a}{\eta }-\frac{\pi ^2 K n^2}{\nu  W^2}-\frac{1}{\tau }.
%\end{eqnarray}
%and thus $\beta_2(\dot{\gamma}=0)-\beta_1(\dot{\gamma}=0)=a/\eta$, the magnitude of which is the inverse of the active time scale.
%(RP: IS THE POINT OF THIS LAST RESULT TO RECOVER THE $A_C = ETA/TAU$ RESULT? IF SO, WHY DOES THE DIFFERENCE OF BETA'S MATTER? BETA1 IS NEGATIVE DEFINITE.)
%(WL: NO. THE EQUATIONS 15 AND 16 ARE ONLY FOR GAMMADOT=0.) (RP: I'M STILL CONFUSED. THE RESULT  $A_C = ETA/TAU$ IS VALID ONLY FOR ZERO SHEAR. I STILL DON'T UNDERSTAND WHAT THE DIFFERENCE OF THE GROWTH RATES TELLS US)
%We define the dimensionless activity $\bar{a}=a\tau/\eta$ and dimensionless growth rate $\bar{\dot{\gamma}}=\dot{\gamma} \tau$. WE DON'T SEEM TO USE THESE BARRED QUANTITIES LATER ON. SHOULD THEY BE ELIMINATED?
There are two modes. In the limit of a passive fluid, $a=0$, the modes collapse to a single mode corresponding to oscillations of the order parameter as it decays to its equilibrium value \abeds{given by eqn (\ref{eq:S})}: $\beta_\pm=-\lambda/\tau[1+\pi^2K/(AW^2)]\pm\mathrm{i}\dot{\gamma}$. Note the similarity between these damped oscillations and the damped oscillations in the startup problem, eqns~(\ref{Qxxstartup}) and (\ref{Qxystartup}).
A nonzero activity makes the two modes distinct.
In the limit of zero shear rate,  $\beta_-$ is negative and independent of activity even if $a\neq0$, and corresponds to the decay of the scalar order parameter of a passive isotropic nematic when it is perturbed from the isotropic value $S=0$. The other mode corresponds to the spontaneous flow and ordering of an active isotropic nematic when  $a>a_\mathrm{c}=[1+\pi^2K/(AW^2)]\eta/(\lambda\tau)$. Note that the confining channel walls raise the critical activity above the previously quoted critical value for unbounded space. The elastic constant $K$ only enters the growth rate if the channel width is finite.
%Note that if we neglect Frank elasticity ($K=0$), the growth rates %will not depend on 
%are independent of $W$.
%, and the finite width of the channel has no effect on the instability.
% EVEN IF K=0, W STILL AFFECTS THE INSTABILITY VIA THE RELATION BETWEEN v_x and Q_xy
% Figure environment removed

In general, the critical activity for instability depends on the shear rate, and is found by determining when $\Re(\beta_+)=0$ for $n=1$. The modes are oscillatory when the square root in eqn~(\ref{betapm}) is imaginary, or when $a_-<a<a_+$, where
\begin{equation}
a_\pm=\frac{2\eta\dot{\gamma}}{\lambda}(1+\dot{\gamma}^2\tau^2)\left(\dot{\gamma}\tau\pm\sqrt{1+\dot{\gamma}^2\tau^2}\right).
\end{equation}
When $a_-<a<a_+$, the critical curve $\mathrm{Re}[\beta_+(n=1)]=0$ in the $\dot{\gamma}$-$a$ plane is given by 
\begin{equation}
a_\mathrm{1c}=2\frac{\eta}{\trp{\lambda}\tau}\left(1+ \dot{\gamma}^2 \tau^2 \right)\left(1+\pi^2\ell^2\right),
 \end{equation}
where $\ell$ is the dimensionless correlation length defined in the previous section. 
When $a<a_-$ or $a>a_+$, the growth rate is purely real, and the critical curve $\beta_+(n=1)=0$ is given by 
\begin{equation}
a_{2\rm{c}}=\frac{\eta}{\tau}\frac{\left(1+\dot{\gamma}^2 \tau^2\right)\left[(1+\pi^2\ell^2)^2+\dot{\gamma}^2\tau^2\right]}{\lambda(1+\pi^2\ell^2-\dot{\gamma}^2 \tau^2)}\label{a2c}
\end{equation}
Note that $a_{2\mathrm{c}}>0$ for $\sqrt{1+\pi^2\ell^2}>\dot{\gamma}\tau$, and $a_{2\mathrm{c}}<0$ for $\sqrt{1+\pi^2\ell^2}<\dot{\gamma}\tau$.

%The critical activity $a_c$ for the instability is determined by the solution of $\Re(\beta)=0$, where $\Re(\beta)$ is the larger of $\rm{Re}(\beta_\pm)$.  Since the real parts of both growth rates need to be negative in the stable regions and the relative values of the growth rates depend on $\dot{\gamma}$ and $a$, the boundaries of stability may need to be given in a piecewise fashion. Indeed, for the extensile case ($a>0$), we find
%\begin{eqnarray}
%a_{\rm{c}}&=&\frac{\eta}{\tau}\frac{\left(1+\dot{\gamma}^2 \tau^2\right)^2}{1-\dot{\gamma}^2 \tau^2},\qquad \dot{\gamma} \leq \frac{1}{\sqrt{3} \tau},\\
%a_c&=&2\frac{\eta}{\tau}\left (1+ \dot{\gamma}^2 \tau^2 \right %),\qquad \dot{\gamma} > \frac{1}{\sqrt{3} \tau}
%\end{eqnarray}
%while for the contractile case ($a<0$), the critical activity is given by the simpler result,
% \begin{equation}
%a_{\rm{c}}=-\frac{\eta}{\tau}\frac{\left(\dot{\gamma}^2 \tau^2+1%\right)^2}{ \dot{\gamma}^2 \tau^2-1}.
% \end{equation}

The stability boundaries are plotted in  Fig.~\ref{fig:linearstability} for the case of $\ell=0$ (zero Frank elasticity). The region of oscillatory growth rates, $a_-<a<a_+$, is the region between the dashed lines. The stable region is the shaded blue region between the solid blue curves, whereas the unstable regions are the white regions. Note that the upper stability boundary is given by $a_{1\mathrm{c}}$ in the oscillatory region, and $a_{2\mathrm{c}}$ in the non-oscillatory region. The lower stability boundary lies wholly in the non-oscillatory region, and is therefore given by $a_{2\mathrm{c}}$. Since the upper stability boundary near $\dot{\gamma}=0$  increases with shear rate, our results are in agreement with Muhuri et al.,\cite{MuhuriRaoRamaswamy2007} who found that shear counteracts the instability for extensile particles.
Surprisingly, we also find that shear can be \textit{destablilizing}  for contractile active particles if the magnitude of the activity is large enough. 




%For values of shear rate and activity in the blue region, between the curves defined by $a_{1\mathrm{c}}$ and $a_{2\mathrm{c}$, the simple shear flow is linearly stable.  %in Fig.~\ref{fig:linearstability} 
%represents the linearly stable region. %for zero Frank elasticity. 
%The upper white region is the linearly unstable region for extensile fluids and the lower white region is for contractile fluids. In these unstable regions, the perturbations grow with time. For nonzero shear rate, we indeed find, in agreement with Muhuri et al.,\cite{MuhuriRaoRamaswamy2007} that shear counteracts the instability for extensile particles. Suprisingly, we %surprisingly 
%see that for contractile particles, an externally induced shear flow can be destabilizing for sufficiently large shear rate when the magnitude of the activity is large enough, and the externally imposed shear rate is greater than the liquid crystal relaxation rate $1/\tau$. In the next section, we will also numerically consider the effect of the nonlinearity in the model on the stability of both extensile and contractile fluids and compare with the instability boundaries in Fig. \ref{fig:linearstability}.


%We also observe %an 
%oscillatory states when $\rm{Im}(\beta)\neq0$, i.e. when $a_{\mathrm{osc}-}<a<a_{\mathrm{osc+}}$, where
%\begin{equation}
%a_{\mathrm{osc}\pm}=2\eta\dot{\gamma}\left(1+\dot{\gamma}^2\tau^2\right)\left(\dot{\gamma}\tau\pm\sqrt{1+\dot{\gamma}^2\tau^2}\right),
%\end{equation}
%\begin{eqnarray}
% a&>&-2 \dot{\gamma} \eta \left((\dot{\gamma} \tau)^2+1\right) \left(\sqrt{(\dot{\gamma} \tau)^2+1}-\dot{\gamma} \tau\right)\\
%a&<&2 \dot{\gamma} \eta \left(\left((\dot{\gamma} \tau)^2+1\right)^{3/2}+(\dot{\gamma} \tau)^3+\dot{\gamma} \tau\right),    % 
% \end{eqnarray}
%which gives the regions between two yellow lines shown in Fig.~\ref{fig:linearstability}. When oscillations occur in the stable region, it is a damped oscillatory mode in which the oscillation will vanish with time. But the oscillation in the unstable region is undamped and the flow cannot be steady. This linearly oscillatory mode will also be compared with our numerical results in the next section.


\section{Straight channel: nonlinear spontaneous flows}\label{sec:nonlinear}

%\rap{
%Note that fig 7 is being cited before 5 and 6 so 7 should be moved. 
%Also, we need a discussion of how we suddenly have a linear stability analysis at nonzero $\ell$. There should also be some mention of this in the previous section even if the analysis of the previous section remains as is, i.e. $\ell=0$}

% CHANGE 'AVERAGE FLOW RATE' TO 'FLUX' AND DEFINE FLUX LITTLE Q. 
\trp{The linear analysis of the previous section predicts that simple shear flow with uniform nematic order is stable as long as the activity and externally imposed shear rate lie in the shaded region of Fig.~\ref{fig:linearstability}. However, there may be transitions to flow states that are not captured by linear stability analysis, and furthermore, the linear equations cannot describe the fully-developed flow states.} 
%When activity is below the critical value $a_c$, there is no net flow driven by the activity, and thus the system exhibits normal viscous flow. But when $a>a_c$, there will be spontaneous flow which breaks the uniformity of the external imposed shear rate and the order parameter field. In this section, 
\trp{Thus,} we explore the activity-induced flow states and the transitions between them by numerically solving the full nonlinear equations, eqns (\ref{incompress})--(\ref{Qeqn}). 
We use the open source finite element software FEniCS\cite{Loggwells2010,LoggEtal_10_2012,AlnaesEtal2014} to solve the nonlinear equations, %in which the 
\trp{employing a} backwards Euler scheme %is employed 
to solve for the time dependence. 
We characterize 
\trp{t}he flow states by the 
\trp{spontaneous volumetric flow rate as well as the wall shear stress.}
%velocity field and the oscillation of the shear stress imposed by the active flow on the moving wall. 

The system is initialized with a small value of the nematic order parameter $S$, appropriate for an isotropic state. For sufficiently small values of the external shear, the direction of the activity-induced flow for $a>a_c$ %will 
depends on the %direction of the weak 
\trp{configuration of the} nematic order. 
\trp{We can achieve positive flow---flow in the same direction the bottom wall moves---or negative flow---flow against the direction the bottom wall moves---by imposing appropriate initial conditions on the directors. These conditions will be described below for the extensile and contractile cases.}
%\wl{To achieve both positive active flow and negative active flow, where "positive" and "negative" mean that the activity-induced spontaneous component of the flow is in the same or opposite direction as the moving wall, respectively, we impose two different kinds of initial conditions according to the steady states of the positive and negative unidirectional flows in the extensile and contractile flows separately.}
% To \trp{achieve right-moving or positive active flow,}
% %allow for the activity induced flow to point either right or left, 
% we impose \trp{initial splay conditions on the order parameter tensor with the directors converging to the right, as in Fig.~\ref{fig:ext_unid}b. Likewise, to get negative flow, we impose initial condition in which the directors converge  to the left, as in Fig.~\ref{fig:ext_unid}d.}
%either 
%right splay-like (similar to a.2 of Fig. \ref{fig:ext_unid}) or left splay-like (similar to b.2 of Fig. \ref{fig:ext_unid}) director fields with some random fluctuations. 
\trp{The initial director fields also have small} random fluctuations.
Because we are neglecting inertial effects, we do not need to initialize the velocity field, which \trp{is}
%will be 
determined from eqns (\ref{incompress})-(\ref{Qeqn}). %To simulate an infinitely long straight channel, 
\trp{Instead of attempting to simulate a very long channel,} we use periodic boundary conditions on the left and right boundaries of the channel. 
The length \trp{$L$} of the channel is chosen to be five times the width $W$\trp{;} %which 
we found \trp{this length} to be the longest channel \trp{length}
we could simulate in a reasonable amount of \trp{computing} time.  %We define a dimensionless correlation length of the orientation of the liquid crystals $\ell=\sqrt{K/A}/W$ for simplification. 
%While we are interested in comparing our computational results with the analytic linear stability results (carried out for zero Frank elasticity), 
%We carry out our 
\trp{We focus on situations in which the width $W$ of the channel is large compared to the correlation length $\sqrt{K/A}$ of the liquid crystal.} \trp{Therefore, our} simulations \trp{are carried out} with a small value of the Frank elasticity, $K/A=0.01 W^2$ (i.e. $\ell=0.1$)\trp{.} %\trp{We found that smaller values of $\ell$ led} to %avoid potential 
%numerical instabilities. 
\trp{In our numerical calculations,} %We choose 
$W$ %for 
\trp{is} the unit of length, $\tau$ %for 
\trp{is} the unit of time, and $\eta/\tau$ \trp{is} %for 
the unit of pressure. We also define the dimensionless activity $\alpha=a{\lambda}\tau/\eta$,
%we don't include lambda in the definition of alpha
%We 
\trp{and} restrict our simulations to the %flow 
\trp{case} of needle-like particles, %i.e., 
$\lambda=1$.%\rap{We might introduce these dimensionless quantities earlier, since they appear in many of the figures} 
% Figure environment removed

% Figure environment removed

\subsection{Extensile fluids}
%\subsection{Extensile flows}
%% Figure environment removed

For extensile fluids, we find three types of flow states when the activity is above the critical value $a_c$: unidirectional, oscillatory, and dancing\trp{. These states are}  similar to three of the states found by Samui et al., \cite{samui2021flow} who studied an active nematic fluid confined to a channel in the absence of external shear. \trp{T}hese authors also found an active turbulent state at high activity, \trp{which we do not explore here}. \trp{The unidirectional flow is steady, consisting of a superposition of spontaneous flow and simple shear flow. The oscillatory flow is unsteady, with a pattern of flow and order that translates at a constant velocity along the channel, which makes the spatially-averaged wall stress constant in time. The dancing flow is truly unsteady, with a spatially-averaged wall stress that oscillates in time. These states will be described in more detail below.}
Fig.~\ref{fig:extensile} %illustrates 
\trp{shows the phase diagram for} %the 
flow states for \trp{dimensionless} activity in the range $0 \leq \alpha \lesssim 2.5$ and shear rate in the range $0 \leq \dot{\gamma}\tau \lesssim 1$. 
%\wl{To offer both opportunities to reach a positive or negative active flow for extensile fluids, we impose two initial condition in which the directors converge to the right and left, as b and d in Fig.~\ref{fig:ext_unid}.}
\trp{To get positive spontaneous flow, we imposed initial conditions with the directors converging to the right, as in Fig.~\ref{fig:ext_unid}b. To get negative spontaneous flow, we imposed initial conditions with the directors converging to the left, as in Fig.~\ref{fig:ext_unid}d.}
We ran %the 
\trp{each} simulation until \trp{either all transients died out, or} $t=600\tau$, \trp{whichever came first}. \trp{The final state could either be a steady state or a state with regular periodic behaviour. Then we classified the states as follows}. 
\trp{The simple shear and unidirectional flow states} generally emerge at times $t\trp{<} 600\tau$. \trp{Both states are steady with negligible $y$-component of velocity, and these two flow states are easily distinguished since simple shear has the standard linear flow profile $v_x=\dot{\gamma}(W-y)$, whereas unidirectional flow has a spontaneous flow component added to the linear flow.} 
%The steady simple shear (similar to the passive simple shear flow) and unidirectional (negligible values of $v_y$) states generally emerge at times $t\leq 600\tau$. 
\trp{If there is a nonzero $y$-component of the velocity at the end of the simulation, we}
%If a steady state has not been realized by \trp{the} end of the simulation, we check the value of $v_y$, the $y$ component of the velocity. If it is not negligible, we c
check %the 
\trp{for} oscillation\trp{s} %of 
in the average wall stress, %in the $x$ direction $\bar{\sigma}_W$ %, eqn~(\ref{wallshear}) EQUATION 9 IS ONLY FOR STABLE FLOWS.
\trp{$\bar{\sigma}_\mathrm{w}=\int_0^{L}\mathrm{d}x\sigma_{xy}(x,y=0)/L$},
for times in the range $550\tau$--$600\tau$. Negligible %values of wall shear oscillation 
\trp{oscillation in the average wall stress} impl\trp{ies} an oscillatory flow state, while non-negligible values imply a dancing state. 
% All but three of the points shown in Fig.~\ref{fig:extensile} %approximately 
% reached a steady or %dynamically stable 
% \trp{regular periodic} state by $t=600\tau$\trp{, or came very close to doing so. In these latter cases, we spot-checked a few examples to make sure that running for longer time led to convergence to a final state.}
% The three %points 
% \trp{cases that did not converge,} even \trp{for} runs as long as %least 
% $t=1000\tau$, %if not greater,
%  are located at \trp{$(\dot{\gamma}\tau,\alpha)=$} (0.0, 1.1), (0.3,1.4) and (0.7, 1.6), % in the $\dot{\gamma}\tau$-$a\tau/\eta$ plane, 
%  all in the vicinity of transitions between different flow states. THESE THREE CASES CONVERGE BUT NEED MUCH LONGER TIME TO DEVELOPER TO THEIR FINAL STATE. FOR EXAMPLE (0.0,1.1) NEEDS TIME LONGER THAN T=1500 TO BE A FULLY UNIDIRECTIONAL STATE; (0.7,1.6) NEEDS TIME T=1200 TO BE A FULLY OSCILLATORY STATE. \trp{So instead of marking these states as special, why don't we just mention that some states near the transition boundary need more time to fully develop?} I AGREE. 
 Most of the points shown in Fig.~\ref{fig:extensile}
reached a steady or %dynamically stable 
\trp{regular periodic} state by $t=600\tau$, or came very close to doing so.
%\wl{But few of cases in the vicinity of transitions between different flow states need much longer time to develop to their fully steady or regular periodic states, which we used the solid markers to denote.}
\trp{But a few cases near transitions between flow states needed much longer to fully develop.} %; these case are denoted with solid markers.}

%in the figures \ref{fig:extensile}, \ref{fig:extensilewall}, \ref{fig:extensileflux}, \ref{fig:contractile}, \ref{fig:cont_flux} and \ref{fig:cont_stress} 

%If a steady state has not been realized by \trp{the} end of the simulation, we check the value of $v_y$, the $y$ component of the velocity. If it is not negligible, we check the oscillation of the average wall shear in the $x$ direction $\bar{\sigma}_W$ %, eqn~(\ref{wallshear}) EQUATION 9 IS ONLY FOR STABLE FLOWS.
%for times in the range $550\tau$--$600\tau$. Negligible values of wall shear oscillation imply an oscillatory flow state, while nonnegligible values imply a dancing state. All but three of the points shown in Fig.~\ref{fig:extensile} approximately reached a steady or dynamically stable state by $t=600\tau$. The three points %even runs as long as least $t=1000\tau$, if not greater,
 %are located at (0.0, 1.1), (0.3,1.4) and (0.7, 1.6) in the $\dot{\gamma}\tau$-$a\tau/\eta$ plane, all in the vicinity of transitions between different flow states. 
 

The \trp{limit of stability for the simple shear flow states} %stability boundary of the numerical results 
in the Fig. \ref{fig:extensile} is the boundary between the %simple shear flow state (black crosses) and other spontaneous flow states (other symbols). 
\trp{region with black crosses and the regions with other symbols.} 
We observe \trp{that} the numerical \trp{limit of} stability %boundary 
\trp{for simple shear flow} matches very well
with the \trp{prediction of} linear stability analysis (filled blue region), but only for the transition from the simple shear to unidirectional flow\trp{, $\dot{\gamma}\tau\lesssim0.3$}. The disagreement between the linear stability boundary and the transition from simple shear flow to oscillatory flow %is 
\trp{may be} due to \trp{our neglect of the possibility that the perturbation could depend on $x$ as well as $y$.} %that our linear analysis only allows one direction perturbation.
%The disagreement between the stability boundary and the result of the linear stability analysis (filled blue region) arises from our assumption of the y-dependent convective term only valid for small shear cases. \rap{these sentences need some work}
In the \trp{region of simple shear flow (black crosses in Fig.~\ref{fig:extensile}), our numerical results show that the wall stress decreases with activity, in agreement with eqn~ \ref{wallshear}. Fig.~\ref{fig:extensilewall} shows the numerically computed wall stress, normalized by the passive (viscous) stress. When the flow state is simple shear, activity reduces the total wall stress in proportion to the activity, in accord with the general understanding that extensile particles with activity reduces the effective viscosity.~\cite{hatwalne04}}





% Figure environment removed

%In the stable region (simple shear flow state in \ref{fig:extensile}), although the velocity is a simple shear flow, the average wall shear normalized by the passive flows %is 
%linearly decreases with activity due to the active force, as shown in the Fig. \ref{fig:extensilewall}, which agrees with eqn~ \ref{wallshear}. Since in the range Fig. \ref{fig:extensilewall} shows, the normalized wall shear is always smaller than the passive case, the active extensile flow helps the right moving of the wall.

%WL: POINT ((0.0,1.1) AT $t=4000\tau$ CAN BE OBSERVED AS A SPONTANEOUS UNIDIRECTIONAL BUT NOT TOTALLY REACH A STEADY STATE. THE 'AVERAGE FLOW RATE' IS SILL INCREASING. SO IT MAY BE MORE SAFE TO NOT HAVE THE LAST SENTENCE. RP: I'VE REMOVED THE SENTENCE.

%IF NOT, THE CALCULATION WILL STOP AT 600 (BECAUSE OF THE CALCULATION TIME LIMIT). THERE ARE TWO POSSIBILITIES: 1. THE SYSTEM CAN BE STEADY LONGER THAN 600 (THE TIME NEEDS TO BE STEADY TOO LONG. THIS SITUATION IS FEW ON THE BOUNDARIES OF THE DIFFERENT FLOW STATE IN FIG 4, E.G. GAMMGA=0,A=1.1); 2. THE SYSTEM IS UNSTEADY (OSCILLATORY FLOW OR DANCING FLOW). THIS IS THE SITUATION FOR MOST OF CASES. FOR MOST OF CASES 600 IS LONG ENOUGH TO BE A FINAL STATE (THE OSCILLATIONS ARE 'STEADY'). THE FLOW STATES ARE DETERMINED BY: FIRST IF THE FLOW FIELD IS THE SAME AS THE STOKES FLOW AT 600, IT IS VISCOUS IN FIG 4 (THIS SITUATION MAY BE THE POSSIBILITY 1. THE SYSTEM MAY HAVE BULK FLOW LONGER THAN 600, BUT SINCE THE TIME TO SEE THAT IS TOO LONG, IT MAY BE NOT SUCH IMPORTANT FOR US TO FIND); SECOND, IF IT IS NOT VISCOUS AND THE UY IS NEGLIGIBLE AT T=600, IT IS SPONTANEOUS UNIDIRECTIONAL; THIRD, IF UY IS NOT NEGLIGIBLE AT T=600, I WILL CHECK THE OSCILLATION OF WALL SHEAR FROM 550 TO 600. IF THERE IS NOT A OSCILLATION OF WALL SHEAR, IT IS OSCILLATORY FLOW, OTHERWISE IT IS A DANCING FLOW.

% Figure environment removed
% % Figure environment removed

\textbf{Unidirectional flow.} When the externally imposed shear is \trp{in the range $0\le\dot{\gamma}\tau\lesssim0.3$, and the dimensionless activity is in a relatively narrow band near $\alpha\approx1$ (Fig.~\ref{fig:extensile}),}
%When the externally imposed shear is small and activity is somewhat larger than the critical value $a_c$, 
activity creates a steady %spontaneous 
unidirectional flow along the $x$-axis~\trp{(Fig.~\ref{fig:ext_unid})}. \trp{The activity-induced component spontaneously breaks the left-right symmetry of the channel, with the actual direction of the active flow component determined not by the imposed external shear but instead by the initial conditions of the directors, as described above.}
\trp{Since the total shear rate vanishes at the value of $y$ at which the flow rate has an extremum, the scalar parameter vanishes at this same value of $y$ (Fig.~\ref{fig:ext_unid}).}
\trp{Fig.~\ref{fig:extvx_u0}} shows the flow profile subtracting off the imposed shear flow for fixed activity and various values of $\dot{\gamma}$ for both the left-moving and right-moving spontaneous flows.
%From Fig.~\ref{fig:extvx_u0} %in the Supplementary Information, we see 
It indicates that the \trp{spontaneous active component of the flow depends on $\dot{\gamma}$; in other words, the}
total flow is not %a 
simply a %linear 
superposition of the passive %flow subject to an externally imposed shear rate $\dot{\gamma}$ 
\trp{shear flow $v_x=\dot{\gamma}(W-y)$} and \trp{the spontaneous flow at} \emph{zero} 
%(RP: SHOULD WE ITALICIZE "ZERO"? I ASK BECAUSE WE JUST SAID THAT POSITIVE AND NEGATIVE SPONTANEOUS FLOWS HAVE THE SAME MAGNITUDE, SO IT'S THEN POSSIBLY CONFUSING TO SAY THAT THE TOTAL FLOW IS NOT A SUPERPOSITION OF PASSIVE PLUS ACTIVE. IT SEEMS THAT EXTERNAL SHEAR IS IMPORTANT. AM I CORRECT? ALSO IS MY STATEMENT ABOUT "FLOW DUE TO ACTIVITY NOT BEING BIASED BY EXT SHEAR" CORRECT?)  (WL: I FEEL THIS SENTENCE MAY BE MISUNDERSTOOD AS "THE FLOW DUE TO ACTIVITY IS THE SAME AT DIFFERENT EXTERNAL SHEAR".)
externally imposed shear. 

%Fig.~\ref{fig:extvx_a_spon} shows higher activity creating faster spontaneous flow which is also indicated in Fig.~\ref{fig:extensileflux}.
% Figure environment removed

% Figure environment removed

\trp{To better characterize these flows, we subtract the passive volumetric flow rate  from the total volumetric flow rate to get the dimensionless activity-induced volumetric flow rate (per unit channel width), 
\begin{equation}
    q^\mathrm{active}\equiv\left(\int_0^W\mathrm{d}y v_x-\frac{\dot{\gamma}W^2}{2}\right)\frac{\tau}{W^2},
\end{equation}
shown in Fig.~\ref{fig:extensileflux}. This quantity serves as an order parameter describing the transitions among the various flow states. Fig.~\ref{fig:extensileflux} shows} that the activity-driven %flux 
\trp{flow rate has the same} magnitude \trp{for the left-moving and right-moving flows, and also that the amplitude of the unidirectional flows increases as the activity increases.} 

\trp{Examining Fig.~\ref{fig:extensilewall} for the case of $\dot{\gamma}\tau=0.2$ reveals that reduction of the \wl{normalized} wall stress with increasing activity ceases at the onset of the unidirectional flow, and the \wl{normalized} wall stress at $y=0$ starts to increase slightly as activity increases further. The active component of the wall stress at $y=0$ in the unidirectional flow has the opposite sign compared to that of the simple shear flow, as can be seen from the opposite orientation of the directors near the wall $y=0$ in Fig.~\ref{fig:ext_unid}b and Fig.~\ref{fig:eg}. Also, the active component of the flow changes the sign of the flow gradient near the wall, as can be seen from Fig.~\ref{fig:extvx_u0}. These two effects together lead to the rise in \wl{the normalized} wall stress at the onset of unidirectional flow.}

%We also %explored  \trp{validated our finite-element calculations of} the unidirectional flow states \trp{from the governing partial differential equations} by \trp{numerically} solving \trp{the} ordinary differential equations (ODEs) \trp{that result from the}
%with assumption that the flow \trp{is steady and} only depends on $y$. \trp{The agreement between the finite-element results and the numerical solution of the ODEs for the steady-state unidirectional flows is excellent.}
%direction. 
%The ODE results of the unidirectional flows shown on Fig.~\ref{fig:extensile} match the results obtained by solving the full partial differential equations (PDEs) well but these ODE results cannot determine the state transitions obtained from the PDEs.

%\trp{, with a direction that depends on the initial conditions of the director as described above} %($v_y=0$). 

%For sufficiently small external shear, the direction of the spontaneous activity-induced flow depends on the initial conditions of the orientational order parameter, i.e., whether the director has a positive or negative $x$ component. 
%Thus, the unidirectional spontaneous flow can be in either the $+x$ or $-x$ directions corresponding to the positive spontaneous flow and negative spontaneous flow patterns shown in Fig.~\ref{fig:ext_unid}. 
%Thus activity can create a spontaneous flow in either the $+x$ or $-x$ directions (corresponding to the positive and negative spontaneous flow shown in Fig.~\ref{fig:ext_unid}) to assist or resist passive Stokes flow induced by the external shear.

%Fig.~\ref{fig:extvx_u0} shows that, in the positive spontaneous flow case, the slope of the velocity at $y=W$ is negative and fixed at a specific activity. Increasing the externally imposed shear rate moves the velocity peak from $y=W/2$ (when $\dot{\gamma}=0$) to a value near $y=0$. %In the negative spontaneous flow the slope of the velocity at $y=0$ is negative and fixed at a specific activity. But the increasing externally imposed shear moves the velocity valley close to $y=W$ from $y=W/2$. 
%The velocity profile of the negative and positive spontaneous flows are symmetric about the point ($W/2,\dot{\gamma}\tau/2$).  

 %In Fig.~\ref{fig:extensileflux} we observe that the activity-driven flux (which is the flux $q$ of the active case minus the flux $q^{\rm {passive}}$ of the passive $a=0$ case in the positive and negative flows are equal in magnitude, i.e., the unidirectional flow due to activity is not biased by the direction of the external shear. This implies that activity can create a coherent flow when subject to a fixed external shear which is not too large. We also explored the unidirectional flow states by solving ordinary differential equations (ODEs) with assumption that the flow only depends on $y$ direction. The ODE results of the unidirectional flows shown on Fig.~\ref{fig:extensile} match the results obtained by solving the full partial differential equations (PDEs) well but these ODE results cannot determine the state transitions obtained from the PDEs.
 
 


%The two unidirectional flow states in the figure \ref{fig:extensileflowstate} are a superposition of the corresponding spontaneous flow and Newtonian flow. 
%The nonzero shear rate of the unidirectional spontaneous flow causes a splay wall-like alignment of the bundles. The alignment vanishes when the shear rate is zero.
%The splay-like structure arises because the shear rate has a different align on either side of the zero shear rate line.



%% Figure environment removed



\textbf{Oscillatory flow.} \trp{Our phase diagram of flow states shows that for $\dot{\gamma}\tau\lesssim0.3$, there is a transition with increasing activity from the unidirectional flow states to two-dimensional oscillatory flows (Fig.~\ref{fig:ext_trans}). 
When $\dot{\gamma}\tau\gtrsim0.3$, the simple shear states transition directly to two-dimensional oscillatory flows as activity increases.}
%Further increasing \trp{the} activity leads to flow \trp{with components} in both the $x$ and $y$ directions. 
%The flow transitions from a steady unidirectional flow to an unsteady oscillatory undulating flow: 
\trp{Although the oscillatory flow states are unsteady, with the velocity and order parameter taking the form of a  traveling wave,}  
the flow pattern and order parameter configuration rigidly translate in the $x$ direction %at a velocity 
\trp{with wave speed} $v_{\mathrm T}$. \trp{In other words, in the frame moving relative to the channel walls with speed $v_\mathrm{T},$ the streamlines meander in space but are steady. Likewise, the configuration of the order parameter tensor is steady in this frame. 
Because we use periodic boundary conditions, the flow field and orientational order parameter must have period in $x$ equal to the channel length $L$. But these fields could also have a shorter period, which must evenly divide the total channel length. Since we use a channel length $L=5W$, the possible wavelengths for a  periodic configuration are $5W$, $5W/2$, $5W/3,$ .... Different wavelengths are selected in the dynamical final state depending on 
%the seed for the pseudo-random number generator used to impose random fluctuations imposed in the initial conditions for the director
\wl{the initial state of the nematic directors}, as well as the value of the activity and the imposed shear.}
%In general, the wavelength grows with decreas\trp{ing} %of 
%activity and %with the 
%increas\trp{ing} %of 
%external shear. 
Because it is difficult to determine the relationship between the 
%seeding
\wl{random fluctuations imposed on the initial directors} and the wavelength that is finally selected, we did not make a systematic study of all the possible wavelengths.
%we focus our study on the case of the wavelength $5W/4$. 
%while the streamlines undulate. 
It is natural to worry that the steady translation of the flow field and order parameter pattern could be an artifact of the periodic boundary conditions. In Sec.~\ref{nonlinear annular}, we study an annular geometry as a single domain without the need for periodic boundary conditions. 
%\wl{In the annular channel geometry case that we will study in Sec.~\ref{nonlinear annular} where periodic boundary conditions are not necessary, 
Since we also observe an oscillatory flow state with constant angular wave speed in that situation, %which suggests that 
we are confident the constant wave speed $v_T$ we see in the straight channel is not an artifact of the period boundary conditions.
%Another limitation results from our use of periodic boundary conditions and our choice of the length of the channel, namely five times its width $W$. For the flow states with nonzero $v_y$, the wavelength of the flow patterns can only be $5/5W, 5/4W, 5/3W, 5/2W$ or $5/1W$. Also, reseeding the pseudo-random number generator for the random fluctuations in the initial conditions may generate different wavelengths of directors which results in a different flux and wall shear stress even when given the same activity and external shear rate. But in general the wavelength grows with the decrease of activity and with the increase of external shear.

We measured the %flux 
\trp{volumetric flux for} %at 
times in the range $t=550$--$600\tau$, which is %the range at which 
\trp{when} the system is generally in its final dynamically stable state. In the final state, the %flux 
\trp{volumetric flow rate} and wall shear stress of the oscillatory flows are %still 
constant.
 %The total flow fields shown in the figure \ref{fig:ext_trans} are the superposition of the oscillatory undulating flow driven by the activity and a Newtonian flow. 
 For small externally imposed shear (e.g. \wl{$\dot{\gamma}\tau \leq 0.2$} in Fig.~\ref{fig:extensile}), the spontaneous activity-induced flows can be either positive or negative\trp{, depending on the form of the splay in the initial conditions for nematic order, as for the unidirectional flows.} See Figs.~\ref{fig:ext_trans}a--d (movies are in the SI).
 %(one example is shown in Fig.~\ref{fig:ext_trans}a and b), depending on the initial direction of the weak nematic order imposed at $t=0$. 
 For positive spontaneous flow, the streamlines undulate, but the externally imposed shear breaks the up-down symmetry of the waves with respect to the horizontal centerline of the channel. The velocity at the valleys of the waves is higher than at the peaks. For negative spontaneous flow, since the activity-induced flow is opposite %in direction 
 to the \trp{direction of the} externally imposed shear \trp{flow}, the flow more easily forms circular streamlines. Thus, for $\dot{\gamma}\neq0$, the absolute value of the activity-driven flux of negative spontaneous flows 
 %\wl{in nonzero external shear cases} 
 is slight\trp{ly} smaller than the flux for the positive spontaneous flows, \trp{as can be seen by looking very closely at Fig.~\ref{fig:extensileflux}}. 
 %For given values of activity and imposed shear, the %flux 
 %\wl{the active contribution to} \trp{volumetric flow rate} of an oscillatory flow with a shorter wavelength is \wl{larger} than one with a longer wavelength, because it is easier to close \trp{the} streamlines in a shorter wavelength flow.
%\wl{Although the flow pattern looks very different for different wavelengths, the volumetric flow rate is weakly dependent on the wavelength in the straight channel.}

 The direction of the spontaneous flow not only %distinguishes 
 \trp{determines} the shape of the streamlines, %and 
 but also determines the direction of translation of the total flow pattern, %i.e.,  
 including the passive viscous flow.  
 %RP: IS THE LAST PHRASE I JUST ADDED CORRECT?.  WL: YES.
 %RP: I DON'T UNDERSTAND THE NEXT SENTENCE. HOW COULD THE TRANSLATION DIRECTION BE ANYTHING OTHER THAN THE DIRECTION OF THE SPONTANEOUS FLOW?
%WL: BECAUSE I THINK THE TRANSLATION OF FLOW PATTER AND THE VELOCITY FIELD ARE TWO SEPARATE THINGS. MY THOUGHT MAY BE WRONG.
%RP: THE TRANSLATION IS EITHER ALONG X OR -X. THE VELOCITY FIELD IS TWO DIMENSIONAL SO I DON'T UNDERSTAND THE COMPARISON
 For positive spontaneous flow, the total flow pattern translates in the $+x$ direction, while for the negative spontaneous flow case, it translates in the $-x$ direction. The activity-driven %flux 
 \trp{volumetric flow rate} is nonzero but generally decreases with increasing activity as shown in Fig.~\ref{fig:extensileflux}. 
 %From 
 Fig.~\ref{fig:trans_speed} %we see that the 
 shows that the %translati\trp{on} 
 wave speed $v_{\rm{T}}$ %of the flow pattern depends on the activity. 
% Moreover, Fig.~\ref{fig:trans_speed}b shows the absolute value of the translation speed of the the 
pattern is faster than the flux, and the difference between these two quantities decreases with the growth of the activity.

 

 


 
 
We now turn to 
%the direct transition to the oscillatory state from the simple shear state without an intermediate unidirectional state, which occurs f}or
\wl{larger externally imposed shear (e.g. $\dot{\gamma}\tau \geq 0.3$ in Fig.~\ref{fig:extensile})}. 
%, we never observe a unidirectional flow state, when the activity is above the critical value $a_c$. Instead, the flow changes from viscous directly to oscillatory when the bend instability appears. 
%Also, for sufficiently large externally imposed shear rate and relatively small activity, 
\trp{In this case,} only the positive spontaneous flow appears; the symmetry is broken by the flow imposed by the external shear. The activity-driven \trp{volumetric flow rate is} %fluxes are 
zero %in the cases when externally imposed shear rate is large 
because %it 
\trp{the imposed shear rate} is large enough to close the streamlines. Interestingly, \trp{our numerical results indicate} %Interestingly, from Fig. \ref{fig:trans_speed}a, we observe 
that the %oscillatory 
\trp{wave} speed is equal to the %flux of the passive case.
\trp{average volumetric flow rate of simple shear, $v_\mathrm{T}=\dot{\gamma}W/2$.}
 
% For all oscillatory flow states, there are $+1/2$ topological defects aligned horizontally above and below the centerline of the channel but \rap{not clear} since the degree of alignment decays from the core of the defect (see Fig.~\ref{fig:ext_trans} b, d and f). The director orientation patterns don't change with time but rigidly translate with the same speed and direction as flow pattern.
 %WE SHOULD SAY MORE ABOUT THE DEFECTS. 
 
%Note that the Landau-Ginzburg free energy density eqn~(\ref{LG}) does not restrict the scalar order parameter $S$ to be less than 1 and $S$ can exceed 1 for large activity.  
%WL: I CALCULATED BOTH SPLAY+NOISE AND ONLY NOISE TWO KINDS OF INITIAL CONDITIONS. ALL PLOTS IN THE PAPER ARE RESULTS FOR SPLAY+NOISE. I DIDN'T CHANGE THE SEED OF NOISE IN THE SPLAY+NOISE CASE BECAUSE LEFT AND RIGHT SPLAY ALREADY HAS A GOOD CONTROL OF THE SPONTANEOUS FLOW DIRECTION. BUT I TRIED THE DIFFERENT SEEDS IN THE ONLY NOISE CASE. THE ONLY NOISE CASE IS SENSITIVE TO THE SEEDS AND HAS MUCH RANDOMNESS TO BE DIFFERENT WAVELENGTH.
% RP: THE NEXT PARAGRAPH DOESN'T HAVE A FIGURE, CORRECT? IT'S ONLY FIG 11 THAT SHOWS WAVELENGTHS AND THAT'S A FIGURE ABOUT OSCILLATION FREQUENCY. THE PRESENT DISCUSSION ABOUT WAVELENGTH IS HARD TO FOLLOW GIVEN THE ABSENCE OF A FIGURE. WL: CORRECT, WE DON'T' HAVE PLOT FOR NEXT PARAGRAPH. IT IS VERY HARD TO COLLECT ALL WAVELENGTHS BECAUSE WE HAVE DIFFICULTY TO CONTROL THE RANDOMNESS OF THE WAVELENGTH IN THE ONLY NOISE CASE.



 
%Because $v_x$ begins to oscillates in the oscillatory flow state, we measure the frequency $f_v$ of this oscillation in the time range $480 \lesssim t \lesssim 600\tau$. From Fig.~\ref{fig:osci_ux}a, we observe that $f_v$ hardly depends on activity by numerically solving nonlinear equations when the externally shear rate is large enough. In this range, for the same wavelength, $f_v$ increases approximately linearly  with increasing external shear rate (approximately consistent with the results of the linear stability analysis). Additionally, in our simulations we found that in the numerical results, the longer the wavelength, the slower the rate of increase of the oscillation frequency $f_v$. Fig.~\ref{fig:osci_ux}b shows a zoomed in region of the cases subject to a small external shear flow. It indicates that $f_v$ depends on activity and for one specific set of values of the activity and $\dot{\gamma}$ there also exist two different frequencies corresponding to different initial conditions of the alignment, when $\dot{\gamma}$ is small.
 % Figure environment removed

%% Figure environment removed
 
\textbf{Dancing flow.}
%Also, sometimes we find dynamical final states that are close to a pattern with a wavelength less than $5W$, but have a period of $5W$ (Fig.~\ref{fig:osci_like}).  
At higher activity\trp{, the flow field and tensor order parameter field become unsteady in any frame, and we find states {(Fig. \ref{fig:ext_dan}; movies are in the SI)} analogous to the dancing flows found by {Shendruk et al.~\cite{shendruk2017dancing}} and Samui et al.~\cite{samui2021flow} in their study of active nematic flow in a two-dimensional channel}.
%showing the velocity field and order parameter field at $t=600\tau$), (will attach movies in the supplementary material), the shape of the streamlines starts to change periodically with time.  
%periodically due to the change in the velocity gradient in the $y$ direction. 
%This is a typical dancing flow \cite{shendruk2017dancing} where the velocity and director field are periodic in the $x$ direction. 
The volumetric flow rate of dancing flow is still constant with time. Additionally, in the range we study ($\alpha \leq 2.5)$, %the time-averaged flow rate is the same as the passive flow.}
%, \trpb{but near the transition from oscillatory to dancing states}.}\wl{THE OSCILLATORY-LIKE FLOW DOES NOT HAVE TO APPEAR NEAR THE TRANSITION.}  %appendix Fig.~\ref{fig:osci_like}.
%Analysis the oscillation frequency with the imposed shear! I don't think the frequency of the oscillation of the wall shear stress is the same thing as the oscillatory mode in the linear stability analysis because the wall shear stress I calculated is a average value along the channel length, so the oscillation only reflect the change of the velocity in $y$ direction. But the perturbation is in $x$ direction. So I think it is more reasonable to compare the the oscillation of $u_x$! one more plot
%\trp{Like the oscillatory states at higher shear rate, there is no active contribution to the volumetric flow rate; the average flow rate for the dancing flows is the same as the flow rate of simple shear flow.}
when activity %(<2.5. up to 3 there is some variation) 
is large enough to dynamically close all streamlines for the part of the flow that is activity-driven, the total flux is the same as in the passive case. 
\trp{As in the case of the oscillatory flows, sometimes we find multiple states at the same values of parameters. For example, n}oise in the initial conditions may cause the system to exhibit oscillatory-like states  \trp{in the region of the phase diagram where dancing flows are also found.}

\trp{Given a director configuration $\hat{\mathbf{n}}=\cos\phi\hat{\mathbf{x}}+\sin\phi\hat{\mathbf{y}}$, we may define the topological charge inside a closed loop by computing $\int\mathrm{d}\phi=2\pi m$ around the loop, where $m$ is the charge. Applying this definition to the configuration in Fig. \ref{fig:ext_dan}b may be problematic because the order parameter $S$ vanishes not just in small cores but in extended two-dimensional regions. If the loop drawn to encircle a potential topological defect crosses a region where $S$ vanishes, the angle $\phi$ and the topological charge are ill-defined. Nevertheless, we can simply look at the director configuration of dancing flow and see that there are parts of the configuration around the regions of small $S$ near the center of the channel that closely approximate the director field of $+1/2$ defects.}
The %re are 
$+1/2$ defects appear %ing 
in pairs\trp{,} and the two defect cores move with undulations \trp{of the flow} in opposite directions %(left and right) 
leading to \trp{the pairs} exchang\trp{ing} %of pairs, 
\trp{partners with the pair to the immediate left and immediate right,} consistent with the Ceilidh dance observed by Shendruk ~\cite{shendruk2017dancing} and Samui \cite{samui2021flow}.
%This is the most typical case. However, we also observe a short-wavelength wave-like dancing flow when $\dot{\gamma}=0$ (cases denoted by black square symbols in Fig.~\ref{fig:extensileflux}). The flow pattern and defect pattern look like the oscillatory flows' but in fact they are not because the patterns not only rigid translate in $x$ direction but also change with time. For this wave-like dancing case, the wavelength is $W$. Since the streamlines are not closed, the activity-driven flux is non-zero for this dancing flow.

% Figure environment removed


% Figure environment removed
% Figure environment removed

The \trp{spatially averaged} shear stress imposed by the active flow on the moving wall also oscillates in time. 
The average wall shear stress no longer decreases linearly with activity in the spontaneous flow region. 
%In the range Fig. \ref{fig:extensilewall} shows the wall shear also helps us to find the transitions of states, i.e. the wall shear shows turning points at these transitions. But we also observe some turning points that are not associated with transitions. They appears due to the change of the wavelength of the flow between the two neighbors. %%THIS NEEDS TO BE EXPLAINED BETTER


%ARE WE GOING TO SAY SOMETHING ABOUT TURBULENT FLOW? WE MENTIONED IT AT THE BEGINNING OF SECTION 5.1.
%WL: we need to calculate higher activity to see turbulence. 

\subsection{Contractile fluids}
\trp{Negative activity corresponds to contractile particles. When the activity is sufficiently negative and the shear rate is large enough,} $\dot{\gamma}\tau>\sqrt{1+\pi^2\ell^2}$ \trp{, we observe unidirectional flow states in our finite-element calculations. The stability boundary that we find in our numerical calculations is}
\trp{c}onsistent with the \trp{results of our} linear stability analysis (Fig.~\ref{fig:contractile}).
%,  we also observe in the finite element method calculations that for large activity contractile fluids can also exhibit a spontaneous flow when $\dot{\gamma}\tau>\sqrt{1+\pi^2\ell^2}$. From Fig.~\ref{fig:contractile}, we observe good agreement of the linear stability boundary (between the filled blue region and blank region) and the numerical stability boundary (between black cross and red circle) since we only observe unidirectional flow states in the unstable region we calculated which implies that our $y$-dependent perturbation assumption is reasonable.
%According to the alignment of the steady state of the spontaneous contractile flows a.2 and b.2 in the Fig. \ref{fig:con_unid}, we give two initial conditions of order parameter as showed in the Fig. \ref{fig:cont_init} in order to allow the activity induced flow to point eight right or left.
%in order to allow the spontaneous component of the flow to point either right or left.}
%\wl{In the steady state of the positive active flow, the directors align with the top plate and gradually tilt to the bottom plate (\ref{fig:con_unid}b). In contrast, the directors align with the bottom plate and gradually tilt to the top plate in the steady negative active flow (\ref{fig:con_unid}d).}
%\wl{Referring to an example of the alignment for the positive and negative active flows in the contractile case Fig. \ref{fig:con_unid}b and d, we accordingly impose two initial director conditions with small fluctuations: (1) directors align in $x$ direction when $y>W/2$ and align $-45^{\circ}$ to $x$ direction when $y<W/2$; (2) directors align in $x$ direction when $y<W/2$ and align $-45^{\circ}$ to x direction when $y>W/2$.}
\trp{As in the extensile case, we get {both} positive %or 
\trp{and} negative flows, \trp{depending on whether the initial configuration of the directors bends downward as in Fig.~\ref{fig:con_unid}b, or upward as in Fig. \ref{fig:con_unid}d.}}
%depending on the initial conditions for the nematic director} (Fig.~\ref{fig:contractile}). 
\trp{After transients have died out, the active component of the volumetric flow rate is equal in magnitude for the positive and negative flows, and the amplitude of the flow rate increases as the magnitude of the activity increases.} \trp{It is well-known that contractile elongated particles in a shear flow enhance the shear viscosity.~\cite{hatwalne04} Thus, the wall stress (normalized by passive stress) increases linearly with the magnitude of the activity when the flow is simple shear, according to eqn~\ref{wallshear}. %This linear increase with $|\alpha|$ (or decrease with $\alpha$) is plotted in Fig.~\ref{fig:cont_stress}.
When the flow transitions to unidirectional flow, we also find that the \wl{normalized} wall \wl{stress} increases linearly with the magnitude of the activity, however with a slightly smaller \wl{absolute value of} slope.}
The figures showing the dependence of the active component of flow rate and the dependence of the \wl{normalized} wall \wl{stress} on activity are in the SI.
%We analyze the flow state when the active flows become steady.
%In the range of activity and $\dot{\gamma}$ shown in Fig.~\ref{fig:contractile}, all cases exhibit both positive and negative unidirectional spontaneous flows, determined by the initial conditions of nematic alignment (see Fig.~\ref{fig:con_unid} for examples of positive and negative unidirectional flow).
%Similarly, in contractile unidirectional flow, the activity-driven flux in the positive and negative flows are equal in magnitude. From Fig.~\ref{fig:cont_flux} we see that with increasing magnitude of the activity, the magnitude of the activity-driven flux increases which implies that contractile active particles also have the ability to create the coherent flow. 

When the magnitude of the activity becomes large, we observe a boundary layer in the flow velocity. Since we found only  steady-state unidirectional flow states for contractile activity, it is computationally more efficient to reduce the governing partial differential equations to ordinary differential equations [see eqns~(\ref{vxstdy}--\ref{Qxystdy}) below] and solve them using the bvp5c solver of MATLAB.\cite{kierzenka2008bvp}
 Fig.~\ref{fig:cont_u0} shows the active component of the flow %profile 
 for the positive and negative spontaneous flows of contractile \trp{gels.} %cases. \trpb{I'M CONFUSED BY THE WORD 'NET' HERE---DOES IT MEAN YOU'VE SUBTRACTED OFF THE BASE SHEAR FLOW?} \wl{YES}
When the absolute value of the activity is large, we observe that the spontaneous component of the flow approaches simple shear flow, with a boundary layer of \wl{dimensionless} thickness $\ell_\delta$ near one of the walls, which we define as the displacement boundary layer thickness~\cite{KunduCohen2008} % HOW DOES THE BOUNDARY LAYER THICKNESS SCALE WITH $|\alpha|$?}
 %velocity location to the nearer boundary. 
 \begin{equation}
     \ell_{\delta}\equiv\frac{\int_0^W \mathrm{d} y \left(\dot{\gamma}_0 y-(v_x-v_x^{\rm{passive}})\right) }{\int_0^W \mathrm{d} y \dot{\gamma}_0 y},
 \end{equation}
 where $\dot{\gamma}_0=\mathrm{d} (v_x-v_x^{\rm{passive}})/{\mathrm{d}y}$ at $y=0$ for positive spontaneous flow. The boundary layer thickness is the same for positive and negative spontaneous flow.
 %Since the active component of the volumetric flow rate for the negative spontaneous flow is equal to the positive spontaneous flow in magnitude, the boundary layer thickness is also the same as the positive spontaneous flow except that the wall near the boundary layer is at $y=0$ in the negative spontaneous flow instead of $y=1$.}
 Fig.~\ref{fig:cont_u0} shows that the peak flow speed of the active component is higher and the boundary layer is thinner for larger magnitudes of the activity. {From Fig. \ref{fig:Blayer},} we find that $\ell_\delta\propto|\alpha-\alpha_c|^\zeta$, where $\zeta$ is close to \wl{$-0.5$}, but %depends on 
 \wl{its magnitude} increases with $\dot{\gamma}\tau$. This dependence will be studied in another publication.
 
 %In order to efficiently find the thickness of the boundary layer, we use 'bvp5c' solver \cite{kierzenka2008bvp} in MATLAB to solve ordinary differential equations (ODEs) with an assumption that the flow only depends on $y$ direction, and we show the relation of the boundary layer to the absolute value of the dimensionless activity in Fig.~\ref{fig:Blayer}. The exponents of $1/(\alpha-\alpha_c)$ that $\ell_{\delta}$ scales as gradually increase with the growth of $\dot{\gamma}\tau$ in the range of Fig.~\ref{fig:Blayer}%Fig.~\ref{fig:cont_u0}b shows increasing the rate of the external shear also makes the velocity is higher but makes the boundary layer is thicker.
 %\trpb{I WOULD NOT SAY THERE IS A BOUNDARY LAYER IN FIGURE 15B, BUT IT IS INTERESTING THAT THE CURVES SEEM TO APPROACH A LIMITING CURVE WHEN GAMMA TAU GOES TO 2.4. WHAT HAPPENS AT BIGGER GAMMA?.} \wl{2.4 IS THE BIGGEST VALUE WE HAVE NOW. I AM RUNNING A CASE OF GAMMADOT=10. I WILL LET YOU KNOW WHEN IT FINISHES}
 %I'M NOT SURE WHAT INSIGHTS WE CAN OFFER HERE.}
%\wl{I AM CALCULATING THE LARGE ACTIVITY IN THE CONTRACTILE CASE TO SEE 'BOUNDRADRY LAYER'}
%that in contractile fluids, the slopes of the velocity are negative on the top and bottom walls of the channel. But in the positive spontaneous flow case, the slopes of the velocity are fixed at $y=0$ at a specific activity and increasing externally imposed shear moves the velocity turning point close to $y=W$. In the negative spontaneous flow case, the slopes are the same at $y=W$ at a specific activity and increasing the externally imposed shear moves the velocity turning point close to $y=0$.
 
 
 %In agreement with the absence of velocity in $y$ direction in the contractile flows in the range we calculated, we did not observe any topological defects in the nematic director field.
 
%From  eqn~(\ref{wallshear}), we know that the slope of normalized wall shear is $1-a\tau/(\eta(1+\dot{\gamma}^2\tau^2))$ for the simple shear flow state. In the finite element calculation, we also observe that in the unidirectional flow state the normalized average wall shear still has an approximately linear dependence on activity but the slope is different from the simple shear flow (see Fig. \ref{fig:cont_stress}). Since the wall shear in the contractile case is always larger than the passive case in the range shown in the Fig. \ref{fig:cont_stress}, the active contractile flow resist the motion of the wall in our calculations. Similar to the unidirectional flow state in the extensile case, the external shear could weaken the effect of the activity on wall shear.

%WL: THE INITIAL CONDITION IN THE CONTRACTILE CASE IS A LITTLE DIFFERENT FROM THE EXTENSILE CASE. IT IS NOT AN UP-DOWN SYMMETRY SPLAY-LIKE ALIGNMENT. INSTEAD, ACCORDING TO THE A.2 AND B.2 IN THE FIG. \ref{fig:cont_unid}, I USED THE TWO INITIAL CONDITIONS SHOWED IN THE FIG \ref{fig:cont_init}. I HAVEN'T FOUND A GOOD WORD TO DESCRIBE IT.

%% Figure environment removed




% % Figure environment removed

% % Figure environment removed
% % Figure environment removed
% Figure environment removed

% Figure environment removed

% % Figure environment removed

% % Figure environment removed

%\trp{Park the $\dot{\gamma}=0$ case here for now.}
\subsection{Weakly nonlinear analysis for $\dot{\gamma}=0$.}
\label{weaknonlinear}
To conclude this section, we turn to a weakly nonlinear analysis of the spontaneous steady unidirectional flow near the transition from the motionless isotropic state.\cite{OhmShelley2022} We {continue to assume $\lambda=1$ and} only consider the case of zero shear rate, $\dot{\gamma}=0$, leaving the case of nonzero $\dot{\gamma}$ for another publication. Assuming that the velocity field, order parameter tensor, and pressure depend only on the coordinate $y$, the dimensionless governing equations are 
\begin{eqnarray}
v_x''-\alpha Q_{xy}'&=&0\label{vxstdy}\\
-p'+\alpha Q_{xx}'&=&0\label{vystdy}\\
\ell^2Q_{xx}^{\prime\prime}-Q_{xx}+v_x' Q_{xy}&=&0\label{Qxxstdy}\\
\ell^2Q_{xy}^{\prime\prime}-Q_{xy}-v_x'Q_{xx}+%\lambda
v_x'&=&0\label{Qxystdy}
\end{eqnarray}
with no-slip boundary conditions $v_x(0)=v_x(1)=0$ and  no-torque (Neumann) boundary conditions $Q_{ij}'(0)=Q_{ij}'(1)=0$. The prime denotes a derivative with respect to $y$. We already saw in Sec.~\ref{stability} that the motionless, distorted state at zero imposed shear rate is unstable when $\alpha>\alpha_\mathrm{c}$, where $\alpha_c=(1+\pi^2\ell^2)%/\lambda
$ is the dimensionless critical activity. Here we study the spontaneous flow and weak ordering for $\alpha=\alpha_\mathrm{c}+\delta\alpha$, with $\delta\alpha>0$. Assuming the balance $Q_{xx}\approx v_x^\prime Q_{xy}$ in eqn~(\ref{Qxxstdy}) suggests that to leading order, $v_x=\mathcal{O}(\delta\alpha^{1/2})$, $Q_{xy}=\mathcal{O}(\delta\alpha^{1/2})$, and $Q_{xx}=\mathcal{O}(\delta\alpha)$. Thus, we expand in powers of $\delta\alpha^{1/2}$:
\begin{eqnarray}
    v_x&=&\delta\alpha^{1/2}v_x^{(1)}+\delta\alpha v_x^{(2)}+\delta\alpha^{3/2} v_x^{(3)}+\dots\\
    Q_{ij}&=&\delta\alpha^{1/2}Q_{ij}^{(1)}+\delta\alpha Q_{ij}^{(2)}+\delta\alpha^{3/2} Q_{ij}^{(3)}+\dots. %\\
   % Q_{1ij}\quad Q^\prime_{1ij}\quad Q^{(1)\prime}_{ij}
\end{eqnarray}
At $\mathcal{O}(\delta\alpha^{1/2})$, we find the steady versions of the linearized equations we used in Sec.~\ref{stability} to solve for the growth rate,
%the same linearized equations we used in Sec~\ref{stability} to determine $\alpha_\mathrm{c}$ when the growth rate is purely real. 
\begin{eqnarray}
    v_x^{(1)\prime\prime}-\alpha_\mathrm{c}Q^{(1)\prime}_{xy}&=&0\label{v1eqn}\\
    \ell^2Q^{(1)\prime\prime}_{xx}-Q^{(1)}_{xx}&=&0\label{Q1xxeqn}\\
    \ell^2Q^{(1)\prime\prime}_{xy}-Q^{(1)}_{xy}+%\lambda
    v_x^{(1)\prime}&=&0.\label{Q1xyeqn}
\end{eqnarray}
The Neumann boundary conditions on $Q_{ij}$ together with eqn~(\ref{Q1xxeqn}) imply that $Q^{(1)}_{xx}(y)=0$. Integrating eqn~(\ref{v1eqn}) yields $v_x^{(1)\prime}-\alpha_\mathrm{c}Q^{(1)}_{xy}=\sigma^{(1)}$, where  $\sigma^{(1)}$ is a constant. Eliminating  $v_x^{(1)}$ from eqn~(\ref{Q1xyeqn}) leads to 
\begin{equation}
    \ell^2Q_{xy}^{(1)\prime\prime}+(%\lambda
    \alpha_\mathrm{c}-1)Q^{(1)}_{xy}=-%\lambda
    \sigma^{(1)}.
\end{equation} 
To get a solution for $Q^{(1)}_{xy}$ that satisfies the Neumann boundary conditions, 
we must have 
\begin{eqnarray}
Q^{(1)}_{xy}&=&c_1\cos(\sqrt{%\lambda
\alpha_\mathrm{c}-1}y/\ell)-\frac{%\lambda
\sigma^{(1)}}{%\lambda
\alpha_\mathrm{c}-1}\\
&=&c_1\cos\pi y-\frac{%\lambda
\sigma^{(1)}}{%\lambda
\alpha_\mathrm{c}-1}, 
\end{eqnarray}
%or $\alpha_\mathrm{c}=(1+\pi^2\ell^2)/\lambda$, which is the dimensionless version of what we found for the critical activity at zero imposed shear rate in sec~\ref{stability}. 
Using eqn~(\ref{v1eqn}) and the no-slip boundary conditions implies $\sigma^{(1)}=0$ and $v^{(1)}=(c_1\alpha_\mathrm{c}/\pi)\sin\pi y$.
%Using the boundary conditions, these equations have solutions
%\begin{eqnarray}
%Q^{(1)}_{xx}&=&0\\
%Q^{(1)}_{xy}&=&c_1\cos\pi y\\
%v^{(1)}&=&c_1\frac{1+\pi^2\ell^2}{\lambda\pi}\sin\pi y,
%\end{eqnarray}
%where $c_1$ is a parameter to be determined by our analysis. 
Note that to leading order, $v$ and $Q_{xy}$ are $\mathcal{O}(\delta\alpha^{1/2})$, but $Q_{xx}$ is at most $\mathcal{O}(\delta\alpha).$ At the next order, the equations are
\begin{eqnarray}
    v_x^{(2)\prime\prime}-(1+\pi^2\ell^2)Q_{xy}^{(2)\prime}&=&0\\
    -\ell^2Q_{xy}^{(2)\prime\prime}+Q_{xy}^{(2)}-%\lambda
    v_x^{(2)\prime}&=&0\\
    -\ell^2Q_{xx}^{(2)\prime\prime}+Q_{xx}^{(2)}&=&c_1^2{(1+\pi^2\ell^2)}\cos^2\pi y,
\end{eqnarray}
with solutions 
\begin{eqnarray}
    Q_{xx}^{(2)}&=&c_1^2\frac{1+\pi^2\ell^2}{2%\lambda
    }\left(1+
    \frac{\cos2\pi y}{1+4\pi^2\ell^2}\right)\\
    Q_{xy}^{(2)}&=&c_2\cos\pi y\\
    v_x^{(2)}&=&c_2\frac{1+\pi^2\ell^2}{%\lambda
    \pi}\sin\pi y,
\end{eqnarray}
where $c_2$ is a constant.

To determine $c_1$, we must expand to $\mathcal{O}(\delta\alpha^{3/2})$:
\begin{eqnarray}    v_x^{(3)\prime\prime}-{(1+\pi^2\ell^2)}Q^{(3)\prime}_{xy}&=&-c_1\pi\sin(\pi y)\label{v3eqn}\\
    \ell^2Q_{xy}^{(3)\prime\prime}-Q_{xy}^{(3)}+%\lambda 
    v^{(3)\prime}&=&
    c_1^3C_0\left[\left(\frac{3}{2}+4\pi^2\ell^2\right)\cos\pi y\right.\nonumber\\
    &+&\left.\frac{1}{2}\cos3\pi y\right]\label{Qxy3},
\end{eqnarray}
where $C_0=(1+\pi^2\ell^2)^2/[2(1+4\pi^2\ell^2)%\lambda^2
]$.
%with no-slip boundary conditions for $v^{(3)}$ and no-torque boundary conditions for $Q^{(3)}_{xy}$ at $y=0$ and $y=1$. 
Integrating eqn~(\ref{v3eqn}) yields
\begin{equation}
    v_x^{(3)\prime}={(1+\pi^2\ell^2)}Q^{(3)}_{xy}
+c_1\cos\pi y+\sigma^{(3)},\label{stress3}\end{equation}
where the constant $\sigma^{(3)}$ appears in the expansion of the stress, $\sigma=v_x^\prime-\alpha Q_{xy}=\delta\alpha^{1/2}\sigma^{(1)}+\delta\alpha\sigma^{(2)}+\delta\alpha^{3/2}\sigma^{(3)}+\dots$. The solutions we have already found at lower order imply that $\sigma^{(1)}=\sigma^{(2)}=0$. The no-slip boundary conditions on $v_x^{(3)}$ also imply that $\sigma^{(3)}=0$. Thus, the stress vanishes not only at the critical value of the activity, but also as $\alpha$ is increased above $\alpha_\mathrm{c}$. Our numerical computations give the same result just above the critical activity. Using eqn~(\ref{stress3}) to eliminate $v^{(3)}$ from eqn~(\ref{Qxy3}) yields
\begin{equation}
\ell^2Q_{xy}^{(3)\prime\prime}+\pi^2\ell^2Q_{xy}^{(3)}=C_1\cos\pi y+C_2\cos3\pi y,\label{Qxy3eq}
\end{equation}
where
\begin{eqnarray}
    C_1&=&\frac{c_1^3(1+\pi^2\ell^2)^\trp{2}(3+8\pi^2\ell^2)}{4(1+4\pi^2\ell^2)%\lambda^2
    }-c_1%\lambda
    \\
    C_2&=&\frac{c_1^3(1+\pi^2\ell^2)^2}{4(1+4\pi^2\ell^2)%\lambda^2
    }.
\end{eqnarray}

To find $c_1$, we use the Fredholm alternative,~\cite{StakgoldHolst2011} which implies that the right-hand side of eqn~(\ref{Qxy3eq}) must be orthogonal to the solution of the corresponding homogeneous equation. Thus, $C_1=0$, and
\begin{eqnarray}
Q_{xx}&=&\frac{2\delta\alpha}{\alpha_\mathrm{c}}\frac{%\lambda
(1+4\pi^2\ell^2)}{3+8\pi^2\ell^2}\left(1+\frac{\cos2\pi y}{1+4\pi\ell^2}\right)+\mathcal{O}(\delta\alpha^{3/2})\\
Q_{xy}&=&\pm\frac{2\delta\alpha^{1/2}}{\alpha_\mathrm{c}}\sqrt{\frac{%\lambda
(1+4\pi^2\ell^2)}{3+8\pi^2\ell^2}}\cos\pi y+\mathcal{O}(\delta\alpha)\\
v_x&=&\pm\frac{2\delta\alpha^{1/2}}{\pi}\sqrt{\frac{%\lambda
(1+4\pi^2\ell^2)}{3+8\pi^2\ell^2}}\sin\pi y+\mathcal{O}(\delta\alpha),
\end{eqnarray}
where the two signs for $v_x$ and $Q_{xy}$ correspond to the two different spontaneous directions of flow, and the corresponding orientation of the directors. These analytical solutions agree well with our numerical solutions for the spontaneous unidirectional flow state with activity just above the critical activity. 
%\rap{why is v proportional to cosine and not sine?}
%\end{comment}







\section{Annular channel: nonlinear spontaneous flows}
\label{nonlinear annular}

\trp{In our work on the straight channel, we saw that simple shear flow led to a spatially uniform order parameter field $\mathsf{Q}$ when the activity is less than a critical value. Uniform  $\mathsf{Q}$ leads to zero active force on the fluid. In contrast, if the shear rate in the flow is spatially nonuniform, the alignment and degree of ordering of the directors will also be spatially nonuniform, leading to an active force. This situation arises in the case of curved boundaries---as in an annular channel---for any nonzero value of the activity, no matter how small.}
%\wl{The orientation is spatially uniform when $a<a_c$ due to the geometric symmetry between the top and bottom plates in the straight channel. But if the two boundaries are curved to different extents, this uniformity will vanish.} 
%Previous studies have considered the effects of confinement of active matter by curved boundaries in the absence of external shear.\cite{norton2018insensitivity,opathalage2019self,saintillan2018rheology} %\rap{reference Norton?} 
{Previous} \trp{theoretical} {studies involving curved boundaries}
\trp{have focused on the case of motionless walls. For example, Woodhouse and Goldstein found spontaneous circular flow in a circular chamber,\cite{woodhouse2012spontaneous}  and Norton et al. showed that the nature of topological defects in the director field is determined by the flow rather than the director anchoring conditions at the wall of a circular chamber.\cite{norton2018insensitivity}} 

% Figure environment removed
%observed a motionless circular confinement allows active suspensions to exhibit a spontaneous circulating flow and different topological defect states when above a critical activity. \cite{woodhouse2012spontaneous,norton2018insensitivity} 
%Motivated by these, we are interested in how the curvature affects the flow states driven by an external shear.
%Curved boundaries can disrupt the orientation of active matter resulting in the presence of topological defects. \cite{ellis2018curvature} 
%Norton et al. found changing the circular confinement and activity can obtain three topological states (A topologically minimal state, a circulating defect state, and a turbulent state) for an extensile active nematic. \cite{norton2018insensitivity} 
%Norton et al. found that while boundary conditions and the geometry of the confining container change the topological states of an extensile active nematic,  the dynamics are insensitive to these conditions. \cite{norton2018insensitivity} 
%Increasing the curvature of the bounding surface increases the range of activity that allows for spontaneous flows. \cite{bell2022active} 
%\trp{MAKE A DISTINCTION BETWEEN SURFACE CURVATURE WITH NO BOUNDARIES AND OUR CASE?}
%Instead of considering a surface curvature without boundaries, 
%Here we focus on an annular geometry and how the curvature of the channel affects the spontaneous flow states.
%Based on our analysis of the straight channel the previous sections, 
\trp{In this section,} we introduce curvature by considering %\trpb{the} Taylor-Couette geometry, i.e., 
the flow states of a two-dimensional active gel in %an 
\trp{the Taylor-Couette geometry of an} annular channel between two concentric circular boundaries of radius $R$ and $R+W$. We impose external shear %on the annular channel 
by rotating the inner boundary with \trp{steady} angular frequency $\omega$, leaving the outer boundary stationary. %In this section, we will first analytically calculate the steady state of the active flow when $\omega$ is very small and then present numerical results for the flow transitions of both extensile and contractile fluids using the full nonlinear equations of Sec.~\ref{model}.
Stokes flow in this geometry, \trp{known as} Couette flow, % in  polar %($r-\theta$) 
%coordinates consists of terms proportional to  $r$ and $1/r$. With no-slip boundary conditions ($v_{\theta}(r=R)=\omega R$ and $v_{\theta}(r=R+W)=0$), the flow velocity 
is  given by~\cite{landauFM}
\begin{equation}
    %v_{\theta}=\frac{R^2\left (-r^2+(R+W)^2\right )\omega}{r W (2R+W)}.
    v_\theta = \frac{\omega R^2}{(2R+W)W}\left[\frac{(R+W)^2}{r}-r\right],\label{TCstokes}
\end{equation}
\trp{where $r$ is the radial polar coordinate. The second term of eqn~(\ref{TCstokes}) corresponds to rigid body rotation and does not lead to any strain rate, but the first term leads to a nonuniform strain rate, and thus induces a nonuniform order parameter field and an active force on the fluid for \textit{any} nonzero value of the activity.} To study the nonlinear flow states of active flows in the annular channel, we again employ the finite element software FEniCS to solve the the full nonlinear equations, eqns~(\ref{incompress})-(\ref{Qeqn}). %The inner radii of inner circle and outer circle are $R_1$ and $R_2$. The width of the channel is $W(=R_2-R_1)$. The curved shear is externally imposed by rotating the inner circle with the angular velocity $\omega$. 
We set $\ell=0.1$, $\lambda=1$ and $R/W=1$.

\subsection{Extensile fluids}
We begin our discussion of the flow states in the annulus with extensile active gels, $\alpha>0$. As in the case of the straight channel, we give the initial director field some splay to induce counterclockwise or clockwise spontaneous flow, with the flow direction depending on the sense of the splay. For example, splay with the rods converging as we move counterclockwise around the annulus (Fig.~~\ref{fig:disk_states}d) leads to counterclockwise active flow (Fig.~\ref{fig:disk_states}c).
%
%For extensile fluids, we use two splay-like initial conditions for the director field in order to allow the spontaneous component of the flow to rotate either counter-clockwise (similar to d in Fig.~\ref{fig:disk_states}) or clockwise with small fluctuations. 
For the activities we used, we find the same kinds of active flow states as in the straight channel: Couette-like states which have no radial component of flow and are the analogs of the unidirectional states in the straight channel (Figs.~\ref{fig:disk_states}a--d), %(along the $\theta$ direction), 
oscillatory states (Figs.~\ref{fig:disk_states}e and f), and dancing states (Figs.~\ref{fig:disk_states}g and h). 
%Examples of these states are shown in Fig. \ref{fig:disk_states}. %in the range $1.0 \leq a \lesssim 2.5 \eta/\tau$ and $0 \leq \omega \lesssim1 /\tau$. 
We run the simulations until $t=600\tau$, and characterize the flow states as we did in the case of the straight channel (Sec.~\ref{sec:nonlinear}).
%except we consider the behavior of $v_r$ instead of $v_y$. 
%First we check whether the active system achieves a steady dynamical state before the end of the simulation. %(Couette flow and spontaneous flow states \rap{I DON'T UNDERSTAND THE POINT OF THIS REFERENCE TO STOKES AND SPONTANEOUS, ARE THESE ALL OF THE STATES SHOWN?)}. 
%If a steady state is not reached, then we check the absolute value of the velocity component $v_r$ at the end of the simulation. 
For the %spontaneous unidirectional 
Couette-like flows, we distinguish %the two Couette-like 
two flow states by checking whether the maximum velocity is at the moving wall or in the interior of the annulus. If the flow is fastest on the wall, we label it a ``Couette-like 1'' flow state; otherwise the label is ``Couette-like 2''. If the transverse component of the velocity $v_r$ is nonnegligible, %(>1e-4), %\rap{(how does "nonnegligible" with a < sign?} 
we check whether the torque exerted by the total flow on the inner boundary oscillates during the time interval $550\tau$-$600\tau$. If it oscillates, then the state is dancing, otherwise it is oscillatory. 
% The Stokes-like flow and unidirectional spontaneous flow states generally become steady before $t\leq 600\tau$. If a steady state has not been realized by end of the simulation, we check the value of $v_r$. If it is not negligible, we check the oscillation of torque (wall shear) on the rotating boundary for times in the range $550\tau-600\tau$. Negligible values of torque oscillation imply an oscillatory flow state, while nonnegligible values imply a dancing state. 
 %We use solid symbols in Fig.~\ref{fig:disk_stability} to  mark the few points that have not reached steady or regular periodic behavior by $t=600\tau$. 
%All but three points $(\dot{\gamma}\tau,\alpha)=$((0.0, 1.1), (0.6,1.7) and (0.7, 1.8)) in the $\dot{\gamma}\tau$-$a\tau$ plane shown in Fig. \ref{fig:disk_stability} approximately reached a steady or 
%dynamically stable 
%\trp{regular periodic}
%state by the end of the simulation. 
{There are a few flow states
%points %These exceptional points are 
%located at 
near transitions %between different flow states that %longer times are needed to equilibrate
that need a longer time to equilibrate. %longer than $t=600\tau$.} 
We also find multiple %states or 
solutions for particular values of $\omega\tau$ and $\alpha$  for the oscillatory and dancing flows.
%In this section, for each given activity and shear rate, we only display one possible solution with one random fluctuation 
{Fig.~\ref{fig:disk_stability} shows flow transitions in the annular channel in range of $0 \leq \alpha \lesssim 2.5$ and $0 \leq \dot{\gamma}\tau \lesssim 1$.}
The transition from Couette-like to oscillatory flow states is relatively robust, with the transition states showing little dependence on the initial conditions. However, {comparing with} the case of the straight channel, the states observed in the transition from oscillatory to dancing flow are {more sensitive} to %the noise of 
the choice of initial conditions.%, which is less robust 
%compare to the case of the straight channel.

% Figure environment removed

In the case of a straight channel, our numerical calculations always yielded the Newtonian simple shear state solution as long as the magnitude of the activity was small enough. The situation is different for the annular channel: our numerical calculations only yield the Newtonian Couette flow state solution (eqn (\ref{TCstokes})) when the activity vanishes. 
As emphasized earlier, any nonzero value of activity leads to active force and an active component of the flow because the order parameter field is nonuniform for nonzero wall rotation speed $\omega$. 
{%This is consistent Green's proof \cite{green2017geometry} that active nematics confined on a surface of nonvanishing Gaussian curvature must generate a thresholdless spontaneous flow.
Green, Toner and Vitelli examined a similar phenomena for active nematics in which a surface of nonvanishing Gaussian curvature generates a spontaneous flow at arbitrarily low values of the activity parameter.\cite{green2017geometry}}
As long as $\omega\tau$ is sufficiently small, the flow profile varies continuously between the Couette, Couette-like 1, and Couette-like 2 states as the activity increases (Fig.~\ref{fig:disk_extvx_a}). Note that the flow velocity increases with activity for a given imposed rotation rate, as expected because extensile activity reduces the effective shear viscosity.\cite{hatwalne04}} Also, the change from the Newtonian Couette flow profile is small as long as the activity is modest, $\alpha\lesssim0.9$ (Fig.~\ref{fig:disk_extvx_a}), which we examine in Sec.~\ref{annular low shear}.
%A Couette flow state is not seen in Fig. \ref{fig:disk_stability} once $\omega$ and $a$ are nonzero, a result which is different from the straight channel geometry where the simple shear flow state persists until the activity or external shear rate are sufficiently large. 
%The unidirectional flow state of the annular channel includes Couette-like 1 and Couette-like 2 flow states in which the velocity points in the angular direction as shown in Figs.~\ref{fig:disk_states} a and c. Fig. \ref{fig:disk_extvx_a} shows velocity profiles for Couette, Couette-like 1 and  2 flows at increasing values of $\alpha$ starting from zero at a fixed $\omega\tau=0.2$. With the increase of $\alpha$, the velocity gradually increases from passive case and the increase amplitude becomes larger and larger.
%In contrast to the case of a straight channel, the nonzero curvature of the two boundaries of the annular channel leads to a nonzero divergence of $\mathsf{Q}$ for all values of $\alpha$.
%when $\alpha<1$ which is the critical activity of an active gel without confinement to be unstable to spontaneous flow. 
%Thus, the active force always has an effect on the velocity field, and for all values of activity, the velocity differs from Couette flow, unlike the straight channel case. 
%From Fig. \ref{fig:disk_extvx_a}
%\rap{WL: I will update this figure to include the Couette flow}

% Figure environment removed

The oscillatory flow in an annular channel (Figs.~\ref{fig:disk_states}e and f; movies are in SI) is similar to the oscillatory flow in a straight channel. The flow and order parameter patterns are steady in  a frame that rotates at constant speed,
%the radial direction but translates in the $\theta$ direction. 
and the average volumetric flow rate ($\int\mathrm{d}rv_{\theta}/W$) is %still 
constant in time. Since we solve the equations in the annular domain without applying periodic boundary conditions, and still see steady rotation of the flow pattern and order parameter pattern, we can be confident that the constant wave speed we saw in the case of the oscillatory flows in the straight channel is  not
%Since the annular channel geometry does not require period boundary condition yet we still observe waves in the oscillatory and dancing flow patterns, this gives us confidence that our results for these flow states in the straight channel case are not 
an artifact of the periodic boundary condition.


%But u
In the dancing state, the flow and order parameter patterns periodically change in %with 
time, %in the radial direction, 
similar to the case of the straight channel. % case.
Unlike the straight channel, the %flux 
volumetric flow rate of the dancing flow state (Figs.~\ref{fig:disk_states}g and h; movies are in SI) in the annular case is %no longer 
not constant in time.
This time dependence arises because the difference in curvatures of the inner and outer boundaries of the annulus breaks the reflection symmetry of the boundaries of the straight channel that relates the dancing flow at the top wall to the dancing flow at the bottom wall.
%, in part because %.  
%However, 
%the curvature of the annular channel varies with $r$ and the average flow rates of dancing flows periodically change with time. W
Also, as in the straight channel, we observe %the 
moving pairs of $+1/2$ defect-like patterns \rap{with an} exchange \rap{of} partners in the annular dancing flow. {In the straight channel, the defect pairs are mirror images of each other (see Fig~\ref{fig:ext_dan}b), but in the annulus, the different curvatures of the two boundaries spoils this symmetry.}
\wl{Joshi et al. also found similar oscillatory and dancing flow states for active nematics by changing the curvatures of the annular channel without external shear.~\cite{joshi2023disks}} 
%An isolated dancing state point (1.0, 2.3) in the $\dot{\gamma}\tau-a\tau$ plane in Fig. \ref{fig:disk_stability} appears due to the fact that the flow pattern at this point is not perfectly periodic in the annular channel. Thus, the flux is not a constant and the torque oscillates.

% Figure environment removed
Fig.~\ref{fig:disk_flux} shows the active component of the average flow rate (defined as before as the average flow rate of the total flow minus the average flow rate of the $a=0$ case) for the various flow states we studied in the annular channel. For the case of zero 
applied shear %shear case 
($\omega\tau=0$), there are %still 
positive and negative spontaneous flows when the activity %is above the 
exceeds a critical value. % to be stable against spontaneous flow. 
But for %nonzero external shear 
$\omega\tau\neq0$, %the flux 
the flow rate has no bifurcation: it continuously increases from zero as the activity increases from zero.
%and 
%the Newtonian
%has no critical activity.
%In extensile flow,
%\rap{I'M CONFUSED BY THIS WORDING. WHY ARE WE COMPARING THE FLUX OF EXTENSILE FLOW TO "EXTENSILE FLOW"? WHAT DOES "LIKE IN EXTENSILE FLOW" REFER TO?} 
%we do not see a transition from simple shear flow to the unidirectional flow state for the nonzero shear case when the channel is curved. Instead, 
Another striking difference with the straight channel is that for nonzero rotation rates of the inner curved wall,
we only observe positive spontaneous flows %in 
(Fig.~\ref{fig:disk_flux}), %for the annular channel, 
even when we \trp{attempt to reverse the direction of flow by altering the initial conditions of the directors.}
%\wl{impose the two splay-like initial conditions of $Q$.}
%which could in principle lead to positive and a negative spontaneous flows respectively
%\rap{What does respectively refer to her? Which initial conditions of Q?}
%\wl{here what I wanted to say is 'we give both initial conditions of Q. One reaches the positive spontaneous flow and the other reaches the negative spontaneous flow.'}
This rectification arises %is 
because in the curved channel, the non-uniform alignment of the directors arising from the applied shear leads to spontaneous flow with the same rotation sense as the rotating wall.
%different curvatures of the two boundaries biases the direction of spontaneous flow. 
Furthermore, since the wave translation direction corresponds to the direction of the spontaneous component of the flow, the oscillatory flow patterns all translate in the $+\theta$ direction when $\omega\tau\neq0$. %in the nonzero shear cases.
%This is different from the straight channel case in which both positive and negative spontaneous flow appears when the external shear is small. 
%Similar to what we find at low shear in Sec.~\ref{annular low shear}, when the shear $\omega$ is relatively larger, we still only find activity-induced flux in a single direction in all nonzero shear in the annular geometry.
%Additionally, we only find activity-induced flux in a single direction except the zero shear case in the comparable large shear $\omega$, which is an extended results of the small shear in Sec.~\ref{annular low shear}.\rap{I DON'T UNDERSTAND THE LAST PART OF THE PREVIOUS SENTENCE. PLEASE REWRITE} 
{Another difference from the straight channel case is that the active contribution to the average flow rate does not disappear in the annular channel for larger shear rate.}

%Like the straight channel case, the value of the average volumetric flow rate of the oscillatory and dancing flows may be different when the wavelength of the period configuration is different. 
%probably controlled by the noise present in the initial conditions of the director field. 
%Fig.~\ref{fig:disk_flux} just show a possible average volumetric flow rate given by one random noise present in the initial conditions of the director field.

Fig.~\ref{fig:disk_torque} shows the torque exerted by the %of extensile fluids 
active fluid on the inner boundary, normalized by the wall torque in %of 
the passive case. The relation of the wall torque to the activity is very similar to the relation of the wall stress to the activity in the straight channel case, i.e. the normalized wall torque decreases with increasing activity for the Couette-like 1 flow state. %\trp{Since the active stress $\sigma_{ij}=-a Q_{ij}$, the linear dependence of the normalized wall stress on activity indicates that the order parameter tensor $Q_{ij}$ depends weakly on activity in the Couette-like 1 states.}
%It is interesting that even with the effects of curvature, the normalized torque still depends linearly on %changes with 
%$\alpha$ for Couette-like 1 flows.
%. %, similar to the relation of the wall shear to the activity in the straight channel case. 

The change in slope in the active-flow rate vs. $\alpha$ curve in Fig.~\ref{fig:disk_flux}, or the normalized wall torque vs. $\alpha$ in Fig.~\ref{fig:disk_torque} indicates the transition from the Couette-like flow state to the oscillatory flow state.
%and \ref{fig:disk_torque} clearly show the transitions from Couette-like 1 or 2 flow to the oscillatory flows.
%of flow states or the difference of the wavelength of configuration between the two neighbors.
As noted earlier, sometimes our numerical approach finds oscillatory patterns of different wavelengths for the same values of the parameters\wl{, which would likely result in values of the volumetric flow rate and wall torque different from those shown in Fig.~\ref{fig:disk_flux} and Fig.~\ref{fig:disk_torque}.} Some of the variation in the normalized torque in the oscillatory and dancing flow regimes in Fig.~\ref{fig:disk_torque} arises from abrupt changes in wavelength as $\alpha$ was varied.
%\wl{The bumps in Fig.~\ref{fig:disk_torque} not located at the transitions between flow states are due to the difference of the wavelength of the configuration between the neighbors.}

We compare the wall torque and wall stress of annular and straight channels in Fig.~\ref{fig:disk_torque_omega} to show the effect of curvature on the wall stress as a function of external shear in the range of $0 < \alpha\leq 1$.
The normalized wall torque and wall stress are close to each other for small external shear rate and both increase with external shear rate, but the increase is larger in the annular channel, i.e. normalized wall torque is closer in value to the passive case. Thus, the curvature of the channel reduces the effect of activity on the wall with increasing external shear.
%THE CRITICAL RADIUS TO BE UNSTABLE? NO




% % Figure environment removed

% Figure environment removed


% Figure environment removed

% Figure environment removed




\subsection{Contractile fluids}

We studied contractile active fluids 
in a two-dimensional annulus with the parameters in the range $-16 \leq \alpha < 0$ and $0 < \omega \tau \leq 2.4$. When $\alpha<0$, we only found Couette-like states with no radial component of the flow. Since contractile activity is effectively shear thickening,\cite{hatwalne04}  the effect of the activity is always to reduce the flow relative to Newtonian Couette flow (Fig.~\ref{fig:disk_convx_a}). As in the extensile case, the direction that the active component of the flow travels around the annulus is independent of the initial conditions, but unlike the extensile case, the active component of flow is negative (against the direction imposed by the externally applied shear). The magnitude of the negative flow is always less than the magnitude of the externally imposed Couette flow; therefore, the total flow never reverses. In this sense, the contractile annular flows are Couette-like 1 states rather than Couette-like 2 states.
In accord with the effective shear-thickenening  of contractile active fluids, the total torque on the inner boundary is always greater than the hydrodynamic torque in  Couette flow (SI Fig.~S4).
%For contractile fluids, 
%we only observe the Couette-like 1 flow state %\rap{DOES ANNULAR FLOW STATE MEAN UNIDIRECTIONAL? 
%WL:YES}  
%in the range $-16 \leq \alpha < 0$ and $0 < \omega \tau \leq 2.4$ where the externally imposed shear flow could destabilize the contractile fluids to create spontaneous unidirectional flow in the straight channel. 
%\wl{We similarly impose two opposite initial conditions of order parameter with small fluctuations like in the straight channel case: (1) the directors are parallel to $\theta$ direction when $r<R+W/2$ and are at $-45^{\circ}$ to $\theta$ direction when $r>R+W/2$; 
%(2) the directors are parallel to $\theta$ direction when $r>R+W/2$ and are at $-45^{\circ}$ to $\theta$ direction when $r<R+W/2$.}
%Fig.~\ref{fig:disk_convx_a} shows examples of the velocity profiles and the active component of velocity profiles for $\alpha$ from zero to $-16$ and $\omega\tau=1.6$. Unlike the Couette-like 1 flows  extensile fluids, the active component of velocity for contractile fluids is always negative even when we impose the two kinds of initial director conditions.
%which initially guide the spontaneous component of the flow to clockwise and counter-clockwise directions. 
%With increasing activity, the total flow is slower but always larger than zero, i.e. the absolute value of the active component of the velocity cannot be larger than the value of the passive flow.

%The active component of the average volumetric flow rate is shown in Fig. \ref{fig:disk_cont_fluxM}a. 
%\wl{By analyzing the active component of the average volumetric flow rate, we observe }there is no bifurcation and thus no critical transition from the Couette flow state to the spontaneous annular flow state in the active case due to the introduction of curvature.
%Unlike the case of straight channel case, 
%In agreement with the observation from Fig. \ref{fig:disk_convx_a}, the activity component of the average volumetric flow rate is always in the opposite direction of the rotation of the inner disk, irrespective of the two kinds of the initial director conditions.  %(similar to our initial condition of contractile fluids in the straight channel). 
%\wl{The absolute value of the spontaneous component grows with the increase of absolute $\alpha$. But the increment become smaller and smaller.}
%We see that when the shear is small (e.g. $\omega=0.1$), there is a reversal in the sense of rotation of the unidirectional flow state. We suspect that this may occur at the critical activity of a hidden transition of the annular case. \rap{DOES HIDDEN TRANSITION MEAN SOMETHING NOT SEEN IN THE EARLIER PHASE DIAGRAMS? WL: I just found the turning points of torque are close to the transition from Couette flow to Couette-like flow but not exactly. The activity of the turning points are a little larger than the activity of the transition between Couette and Couette-like}
%From Fig.~\ref{fig:disk_cont_fluxM}b, 
%We also observe a continuous but not linear increase of the normalized wall torque with the increase of absolute $\alpha$. Thus, active flow helps the rotation of the inner disk, similar to the straight channel. But the relationship of the normalized wall torque to the $\alpha$ is not linear and there is no turning due to no flow transition. \wl{The figures of the dependence of the active component of flow rate and the dependence of the wall torque on activity are shown in the SI. \trpb{Give section or figure number}}
%
%\wl{question: why normalized wall torque is linear to activity in extensile fluids for Couette-like 1 state (Fig.~\ref{fig:disk_torque}) but not linear in contractile fluids (Fig.~\ref{fig:disk_cont_fluxM}b?} 
%\trpb{We can discuss---is the $Q_{ij}$ more strongly dependent on activity for the contractile case?}


% % Figure environment removed

% % Figure environment removed

% % Figure environment removed

% Figure environment removed

 % of the streamlines. \rap{WHERE DO WE SHOW THE CURVATURE OF THE STREAMLINES?} 
%Also, when the curvature is small enough (e.g. $R/W=50$), the normalized torque in the annular channel is the same as the normalized wall shear in the straight channel case.


% \begin{equation}
%     M_W=-\frac{4 \pi  \omega  \left(2 R^2 (\ell^2-\xi^2 )  K_2\left(\frac{R}{\xi }\right)+\xi  R^3 K_1\left(\frac{R}{\xi }\right)+2 \xi ^2 R^2 K_2\left(\frac{R}{\xi }\right)\right)}{\xi  R \left(\left(4 \ell^2-4 \xi ^2+R^2\right) K_1\left(\frac{R}{\xi }\right)+2 \xi  R K_2\left(\frac{R}{\xi }\right)\right)}.
% \end{equation}
    %WHY THERE ARE TWO LINES CROSS? BECAUSE IT CAN'T WORK FOR SMALL R.
% Figure environment removed


\section{Annular channel: Linear analysis of curvature at low shear rate}\label{annular low shear}

In this section we study the limit in which the flow in the annular channel is slow enough that the induced order is small, $S\ll1$. For slow enough flow, it is valid to neglect the nonlinear terms in eqn (\ref{Qeqn}).
%Here 
%we analyze the steady linearized equations for flow and order in 
%steady rheology of flow in 
%an annular channel. 
On the one hand, %it 
this analysis %can 
offers a theoretical explanation of some of the observations in Sec.~\ref{nonlinear annular}; on the other hand, it gives some insight into the role of the curvature of the boundaries, which we did not vary in the previous section. 
For convenience, here we restate the modified Stokes equation (eqn~(\ref{veqn})) in dimensionless form, along with the dimensionless form of the steady linearized equation for $\mathsf{Q}$:
\begin{eqnarray}
0&=&-\boldsymbol{\nabla}p+\nabla^2\mathbf{v}-{\frac{\alpha}{\lambda}}\nabla\cdot\textsf{Q}\label{v2eqn}\\
0&=&-\textsf{Q}+\ell^2\nabla^2\mathsf{Q} 
+2 \lambda \mathsf{E},\label{Q2eqn}
\end{eqnarray}
As in our numerical calculations, we use the width $W$ of the channel as the unit of length.
Since we seek to study the Couette-like flow state, we assume $\mathbf{v}=v_\theta(r)\hat{\bm\theta}$. Note that this flow is incompressible. We also suppose that $p$, $Q_{rr}$, and $Q_{r\theta}$ are functions of radius only. With these assumptions, the $rr$ component of eqn~(\ref{Q2eqn}) is homogeneous, which together with the Neumann boundary conditions $\partial_rQ_{rr}=0$ at $r=R/W$ and $r=R/W+1$ implies $Q_{rr}=0$. Since the radial component of the modified Stokes equation(eqn~(\ref{v2eqn}) with $Q_{rr}=0$ implies that the pressure gradient vanishes, we take $p=0$.

To solve for the velocity and order parameter fields, we take the divergence of eqn~(\ref{Q2eqn}) and combine with eqn (\ref{v2eqn}) with $p=0$ to find
\begin{equation}
    \nabla^2\left(\nabla^2-\frac{1}{\xi^2}\right)\mathbf{v}=0,
\end{equation}
where $\xi^2={\ell^2}/(1-\alpha)$. To focus our attention on the Couette-like states only, we restrict our analysis to $\alpha<1$ in this section. Thus, 
\begin{eqnarray}
    v_\theta=c_1 r+c_2/r+c_3 I_1(r/\xi)+c_4 K_1(r/\xi),\label{vc1234}
\end{eqnarray}
where the $c_i$ are constants to be determined, and $I_1(x)$ and $K_1(x)$ are modified Bessel functions. Inserting the velocity field eqn (\ref{vc1234}) into the $r\theta$ component of eqn~(\ref{Q2eqn}),
\begin{equation}
    0=\ell^2 \left(Q_{r\theta}''+\frac{1}{r}Q_{r\theta}'-\frac{4}{r^2}Q_{r\theta}\right)-Q_{r\theta}+\lambda\left(v_\theta^\prime-\frac{v_\theta}{r}\right),\label{Qrthetav}
\end{equation}
yields
\begin{eqnarray}
    &&\ell^2\left(Q_{r\theta}''+\frac{1}{r}Q_{r\theta}'-\frac{4}{r^2}Q_{r\theta}\right)-Q_{r\theta}
    \nonumber\\
    &=&\frac{\lambda}{\xi}
    \left[\frac{2c_2\xi}{r^2}-c_3 I_2(r/\xi)+c_4 K_2(r/\xi)\right],
    \label{Qrthetav2}
\end{eqnarray}
which has general solution
\begin{eqnarray}
    Q_{r\theta}&=&c_5 I_2(r/\ell)+c_6 K_2(r/\ell)-2c_2\lambda/r^2\nonumber\\
    &-&c_3\frac{\lambda\xi}{\ell^2-\xi^2} I_2(r/\xi)+c_4\frac{\lambda \xi}{\ell^2-\xi^2}K_2(r/\xi).
\end{eqnarray}
Inserting this solution into the modified Stokes equation~[eqn~(\ref{v2eqn})] shows that $c_5=c_6=0$. The rest of the integration constants are determined by the no-slip boundary conditions on the (dimensionless) velocity, $v_\theta(R/W)=\omega \tau R/W$ and $v_\theta(R/W+1)=0$, and the Neumann boundary conditions on the order parameter field $Q_{rr}^\prime(R/W)=Q_{rr}^\prime(R/W+1)=0$.
%helps to study the effect of different curvature at low shear. 
%We impose %incorporate 
%the incompressibility condition eqn~(\ref{incompress}) with a stream function $\psi$, where $\mathbf{v}=\boldsymbol{\nabla}\times\left[\psi(r,\phi)\hat{\mathbf{z}}\right]$, and take the curl of eqn~(\ref{veqn}) to eliminate the pressure. Since we focus on Couette %-like 1 
%flow states %in the region of 
%for $\alpha<1$, we assume that $\psi$ and $Q_{ij}$ only depend on the radial coordinate $r$.
%, and that $v_r=0$. 
%\rap{Regarding eqns 20-22: a) in (20) I think we should replace the superscripts (4) and (3) by 4 primes and 3 primes to be consistent with the primes elsewhere. b) Should the last two terms of (22) have a tau in front? c) where does the psi-prime/r term in (22) come from? Isn't it the last term of (4) that yields the psi terms? The only psi term I get is psi-doubleprime.}
%\textcolor{blue}{(a)replaced; (b)$\tau$ is time unit and the equations are dimensionless. So $\tau$ was cancelled in the dimensionless forms (c) it comes from strain rate tensor}
%\rap{I also don't understand why the laplacian of Q (the ell terms in (21) and (22)) has a term without a derivative of Q. Did something go wrong in Mathematica when using cylindrical coordinates? Eqn (20) may also be wrong in the curl of the laplacian of v} \textcolor{blue}{detailed in Mathematica file}
%With these assumptions, the curl of nondimensionalized 
%\trp{(WE SHOULD STATE EXPLICITLY HERE WHAT LENGTH IS THE UNIT LENGTH---W OR R)} 
%eqn~(\ref{veqn}) in terms of the unit length $W$ in polar coordinates $(r,\theta)$  can be written as
%\begin{eqnarray}
%0&=&-\frac{r^3 \psi ^{''''}(r)+2 r^2 \psi %^{'''}(r)-r \psi ''(r)+\psi '(r)}{r^3}\nonumber\\
%&-&\frac{\alpha}{\lambda} \left(Q_{r\theta}''(r)+%\frac{3 Q_{r\theta}'(r)}{r}\right)\label{vtheta}.
%\end{eqnarray}
%\begin{eqnarray}
%0&=&\psi^{''''}+\frac{2}{r} \psi ^{'''}-\frac{1}{r}\psi^{''}+%\frac{1}{r}\psi'+\frac{\alpha}{\lambda} \left(Q_{r\theta}''+\frac{3}{r} Q_{r\theta}'\right)\label{vtheta}.
%\end{eqnarray}
%For small $\omega\tau$, the linearized, nondimensionalized \trp{form of} eqn~(\ref{Qeqn}) in polar coordinates is given by
%\begin{eqnarray}
%0&=&\frac{\ell^2 \left[r^2 Q_{rr}''(r)+r Q_{rr}'(r)-4 Q_{rr}(r)\right]}{r^2}-Q_{rr}(r)\label{Qrr},\\
%0&=&\frac{\ell^2 \left(r \left(r Q_{r\theta}''(r)+Q_{r\theta}'(r)\right)-4 Q_{r\theta}(r)\right)}{r^2}-Q_{r\theta}(r)\nonumber\\
%&+&\lambda\left[\frac{\psi '(r)}{r}-\psi''(r)\right]\label{Qrtheta}.
%\end{eqnarray}
%\begin{eqnarray}
%0&=&\ell^2 \left(r^2 Q_{rr}''+\frac{1}{r} Q_{rr}'-4 Q_{rr}\right)-Q_{rr},\label{Qrr}\\
%0&=&\ell^2 \left(Q_{r\theta}''+\frac{1}{r}Q_{r\theta}'-\frac{4}{r^2}Q_{r\theta}\right)-Q_{r\theta}+\lambda\left(\frac{1}{r}\psi '-\psi''\right)\label{Qrtheta}.
%\end{eqnarray}
%From eqns~(\ref{vtheta}- \ref{Qrtheta}), it is easy to see that $Q_{r\theta}$ and $\psi$  are independent of $Q_{rr}$.
%With the Neumann boundary condition for $Q_{rr}$ applied to eqn~(\ref{Qrr}) we find that $Q_{rr}=0$.
%Using the standard approach \cite{taylor1923viii} 
%for solving Stokes flow in an annular channel and imposing no-slip and Neumann boundary conditions, the solutions of eqns~(\ref{vtheta}) and (\ref{Qrtheta}) will be of the form
%\begin{eqnarray}
%\psi&=& c_{a1} \log (r)+c_{a2} K_0\left(\frac{r}{\xi }\right)+c_{a3} I_0\left(\frac{r}{\xi }\right)+c_{a7} r^2,\label{psiform}\\
%Q_{r\theta}&=&\frac{c_{a4}}{r^2}+c_{a5} K_2\left(\frac{r}{\xi }\right)+c_{a6} I_2\left(\frac{r}{\xi }\right)+c_{a8},\label{Qform}
%\end{eqnarray}
%where $\xi^2=\frac{\ell^2}{1-\alpha}$, ($\alpha<1$), and $I_n$ and $K_n$ are respectively Bessel functions of the first and second kind. The coefficients $c_i, i=1,...,8$ are determined by inserting eqns~(\ref{psiform}) and (\ref{Qform}) into eqns~(\ref{vtheta}) and (\ref{Qrtheta}) and using no slip boundary conditions for $\bf{v}$ and Neumann boundary conditions for $Q$. 
The %general 
complete formulas are too %quite complex, 
complicated to display, but we plot the velocity in Fig.~\ref{fig:linear_disk_R} for various ratios of $R/W$ for a representative contractile case (top panel) and extensile case (bottom panel). In both cases, the flow velocity approaches a linear profile as $R/W$ becomes large, as expected, since in that limit the curvature of the annulus becomes unimportant, and the flow approaches simple shear flow. For the contractile case, Fig.~\ref{fig:linear_disk_R}a, the velocity profile is close to the Newtonian result, with the agreement between the two cases getting better as $R/W$ increases. For the extensile case, the velocity curves for different values of $R/W$ get closer to each other as $\alpha$ increases, becoming very close to the linear profile around $\alpha=0.885$. Above this value of activity, the order of the curves reverses, with the linear curve lying below all the other curves. When $\alpha$ gets very close to unity and $R/W$ is small, the maximum velocity is not at the wall, i.e. the flow continuously changes from the Couette-like 1 state to the Couette-like 2 state [Fig.~\ref{fig:linear_disk_R}b]. Fig.~\ref{fig:linear_disk_M} shows the total torque $M=2\pi R^2\sigma_{r\theta}$ on the circle $r=R$ as a function of $R/W$. Note that the limit of a straight channel is almost obtained when $R$ becomes comparable to $W$. The torque for a contractile fluid is higher than the passive value since contractile fluids effectively increase the shear viscosity. Likewise, the torque for an extensile fluid is less than the passive value since extensile fluids are shear thinning. The torque approaches the passive value when $R\ll W$. Note that since we use $W$ as the unit of length, the limit $R\ll W$ corresponds to making the inner cylinder of vanishing thickness. When $R<\ell$, the term $\ell^2\nabla^2\mathsf{Q}$ dominates eqn~(\ref{Q2eqn}), and therefore $\mathsf{Q}\rightarrow0$. In this limit, the active force vanishes, and flow is Couette flow.




%so here we show only the limiting case where \wl{the inner circle is negligible compared to the width of the channel,
%the outer boundary is taken to infinity 
%i.e., $R/W\to 0$:} %In that case we find 

% \begin{eqnarray}
% c_1&=&-\frac{R^4 \omega  \left(K_1\left(\frac{R}{\xi %}\right)+K_3\left(\frac{R}{\xi }\right)\right)}{8 \ell^2 %K_1\left(\frac{R}{\xi }\right)+R^2 K_1\left(\frac{R}{\xi %}\right)+R^2 K_3\left(\frac{R}{\xi }\right)-8 \xi ^2 %K_1\left(\frac{R}{\xi }\right)},\\
% c_2&=&c_3 (\ell^2-\xi^2),\\
% c_3&=&\frac{8 \lambda  \xi  R \omega }{8 \ell^2 %K_1\left(\frac{R}{\xi }\right)+R^2 K_1\left(\frac{R}{\xi %}\right)+R^2 K_3\left(\frac{R}{\xi }\right)-8 \xi ^2 K_1\left(\frac{R}{\xi }\right)},\\
% c_4&=&2 c_1.
% \end{eqnarray}

%\begin{eqnarray}
%Q_{r\theta}&=&-\frac{2  \lambda \omega \tau \left(-4 \xi  r^2 \left(\frac{R}{W} \right) K_2\left(\frac{r}{\xi }\right)+\left(\frac{R}{W}\right)^4 \left(K_1\left(\frac{R}{\xi W}\right)+ K_3\left(\frac{R}{\xi W}\right)\right)\right)}{r^2 \left(\left(8 \ell^2-8 \xi ^2+\left(\frac{R}{W}\right)^2\right) K_1\left(\frac{R}{\xi W}\right)+\left(\frac{R}{W}\right)^2 K_3\left(\frac{R}{\xi W}\right)\right)},\label{sol:Qrtheta}\\
%v_{\theta}&=&\frac{\omega  \tau \left(8 r \frac{R}{W} (\ell^2-\xi^2 ) K_1\left(\frac{r}{\xi }\right)+\left(\frac{R}{W}\right)^4 \left(K_1\left(\frac{R}{\xi W}\right)+K_3\left(\frac{R}{\xi W}\right)\right)\right)}{r^2 \left(\left(8 \ell^2-8 \xi ^2+\left(\frac{R}{W}\right)^2\right) K_1\left(\frac{R}{\xi W}\right)+\left(\frac{R}{W}\right)^2 K_3\left(\frac{R}{\xi W}\right)\right)}.\label{sol:vtheta}
%\end{eqnarray}
%Eqns~(\ref{sol:Qrtheta}) and (\ref{sol:vtheta}) indicate that the order parameter and the velocity are both proportional to $\omega$. %when $a<\eta/\tau$ in this unconfined case and we only consider a $\theta$ direction perturbation of velocity.\rap{WHY DO WE NEED TO REPEAT THAT WE ARE BELOW CRITICAL ACTIVITY AND THAT V HAS ONLY A THETA COMPONENT? 
%\rap{IN FIG 20, CONFINEMENT IS PRESENT (FINITE W) SO I DON'T SEE WHAT WE ARE LEARNING FROM THE INFINITE W RESULTS.}  
It is informative to find the velocity and the order parameter field in the limit $R\gg W$, where the curvature of the annulus is small. Rather than taking the limit of the formulas used to make Figs.~\ref{fig:linear_disk_R} and \ref{fig:linear_disk_M}, it is simplest to solve the equations directly using regular perturbation theory in powers of $W/R$. 
Reinstating the dimensions and writing $r=R+y$, we find
\begin{eqnarray}
    v_\theta&=&\omega R\left(1-\frac{y}{W}\right) -\frac{\omega y}{2}\left(1-\frac{y}{W}\right)\nonumber\\
    &+&\frac{2\alpha\ell^2\omega W}{1-\alpha}\left[\frac{1-\cosh\left[{(1-2y/W)/}{\xi}\right]}{\cosh[{1}/(2\xi)]}\right]\label{vthetaasympt}\\
    Q_{r\theta}&=&-\frac{\lambda \omega \tau R}{W}\nonumber\\
    &+&\frac{\lambda\omega\tau}{2}\left[4y/W-3+4\xi\frac{\sinh\left[{(1-2y/W)/}{\xi}\right]}{\cosh[{1}/(2\xi)]}\right].\label{Qrthetaasympt}
\end{eqnarray}
The first terms of eqns~(\ref{vthetaasympt}) and (\ref{Qrthetaasympt}) correspond to the velocity and order parameter field, respectively, of a straight channel with an infinitesimal imposed shear rate $\dot{\gamma}=\omega R/W$. The remaining terms are the corrections due to the nonzero curvature of the annular channel. Unlike our weakly analysis of the active flow in the straight channel (Sec.~\ref{weaknonlinear}), which had spontaneous flow in either direction, here we see that the component of flow driven by the activity has a definite sign, and is the same direction as externally imposed flow for extensile fluids.

%We recover the results of the straight channel case in the limit where the radius of the disk is infinitely large (i.e. $R\rightarrow\infty$). In that limit,  the Bessel functions $K_n(R/\xi)$ %appearing in eqns~\ref{psiform} and \ref{Qform} 
%vanish and the solution 
%\wl{is given by asymptotic analysis as
%\begin{eqnarray}
%Q_{r\theta}=- \frac{\omega R\tau }{W} \frac{2 R \left(\frac{R}{W}+1\right)^2}{(2 \frac{R}{W}+1) \left(4 \ell^2+(\frac{R}{W}+\frac{y}{W})^2\right)} \to -\frac{\omega R\tau }{W} \label{Q_R_inf}\\
%    v_{\theta}= \frac{\omega R\tau }{W}\left(1-\frac{y}{W}\right) \frac{ (2 \frac{R}{W}+1+\frac{y}{W}) \frac{R}{W} }{(2 \frac{R}{W}+1) (\frac{R}{W}+\frac{y}{W})} \to \frac{\omega R\tau }{W}\left(1-\frac{y}{W}\right)\label{v_R_inf}.
%\end{eqnarray}
%In this limit, the solution can be simplified to a straight channel case. Eqn.\ref{v_R_inf} %equals to the steady uniform shear flow in the straight channel. Eqn. \ref{Q_R_inf} is not %the same as Eqn. \ref{Qxy0} because we only consider linear terms in this section.
%}

%Different from Sec.~\ref{weaknonlinear}, we find the general solution for the eqns~\ref{psiform} and \ref{Qform}
%velocity (eqn~(\ref{sol:vtheta})) and order parameter (eqn~(\ref{sol:Qrtheta})) 
%is unique for a given $\alpha$ and $\omega\tau$. This analytically explains why we only observe one direction for the active component of each Couette-like 1 flow in Sec.~\ref{nonlinear annular}.
%We also validate that the analytical solution matches with the numerical solution in Sec.~\ref{nonlinear annular} well for small $\omega\tau$ given the same parameters ($R=1$, $\lambda=1$ and $\ell=0.1$).
%We compare the general analytical solution of linear rheology to the numerical solution in Sec.~\ref{nonlinear annular} with the same parameters $R=1$, $\lambda=1$ and $\ell=0.1$. Fig.~\ref{fig:disk_ext_flux_com} shows the dependence of the normalized spontaneous component of the average flow rate on $\alpha$. The black line shows the analytical results and it agrees with thenumerical result only for small $\omega\tau$, consistent with our assumption of small $\omega$ in the linear analysis. Since we know from Sec.~\ref{nonlinear annular} that the normalized wall torque on the inner boundary is linearly dependent on $\omega\tau$ when $\alpha<1$, we compare the dependence of the wall torque on external shear by plotting Fig.~\ref{fig:disk_ext_M_com}. Again, we see that the linear analysis only works for small $\omega\tau$.

% % Figure environment removed

% % Figure environment removed


% The velocity of the active gel increases with the strength of the activity and the increase is more significant in the vicinity of the critical activity $\alpha_c$. The effect of activity on contractile flow is small. The difference in the orientation of the active particles on the two boundaries decreases with increasing activity. 

%Using the analytical solution of linear rheology, we can effectively explore the influence of different curvatures on Couette-like 1 flows by varying the ratio $R/W$ with $\ell=0.1$ and $\lambda=1$. %the larger ratio corresponds to the small curvature. 
%Since the velocity in the linear analysis of small $\omega\tau$ is linearly dependent on $\omega\tau$, we normalize the fluid velocity by the velocity on the inner boundary ($\omega R$) in Fig. \ref{fig:linear_disk_R} to more clearly see the effect of curvature. Parts (a) and (b) of the figure show examples of contractile and extensile flows respectively. We observe that for both contractile and extensile flows, increasing the value of the curvature (i.e., decreasing $R/W$) increases the magnitude of the spontaneous flows. But the spontaneous component of velocity is positive in extensile fluids while negative in contractile fluids. 
%In the opposite limit when $R/W$ is sufficiently large, 
%In Fig. \ref{fig:linear_disk_R}, the curvature is negligible and the flow is basically the same as the simple shear case when $R/W \geq 50$.
%Using the $\psi$ and $\textsf{Q}$ solutions, 
%We also analytically calculate the wall torque shown in \ref{fig:linear_disk_M}. %Part (a) shows 
%that activity resists the rotation of the inner disk in the contractile case but reinforces the rotation in the extensile case. We also find that this effect of activity is reduced by increasing the curvature. 
%The effect of activity on the wall torque is reduced by increasing the curvature
%and the influence of curvature is significant for small $R/W$ but slight when $R>W$.
% In contrast to the straight channel case,  Figs. \ref{fig:linear_disk_a}a indicates that activity-induced flux (the total flux minus the simple shear flux) points in only one direction. For extensile flow, the activity-induced flow points along the external shear direction and velocity increases with increasing activity. For contractile flow, the activity-induced flow is opposite to the shear direction but the magnitude is much smaller.
% % Figure environment removed

% % Figure environment removed



\section{Summary}
We investigated the stability and flow states of the active gel confined in a channel subject to a external shear.
An externally imposed shear flow can stabilize an extensile fluid that would be unstable to spontaneous flow when there is no external shear flow, and destabilize a contractile fluid that would be stable against spontaneous flow when there is no external shear flow. In accordance with previous simulations \cite{samui2021flow,VargheseBaskaranHaganBaskaran2020} \rap{carried out in the absence of external shear,} we find three kinds of nonlinear flow states in the range of parameters we study: unidirectional flows, oscillatory flows, and dancing flows for extensile fluids. The unidirectional flow observed in the straight channel can have a spontaneous active component which is either positive---in the same direction as the moving wall---or negative---in the opposite direction of the moving wall.
%has possibility to be both the positive or negative spontaneous flow, and the 
The oscillatory flow states also have two possible directions for the spontaneous active component when the externally imposed shear rate is small. For greater imposed shear rates, the spontaneous flow direction will be the same as the moving wall. For contractile gels, we only observe %the two directions of spontaneous 
unidirectional flow states in the range of parameters that we studied. These unidirectional flows can have positive or negative spontaneous active components.
In the analysis of the the wall stress caused by the active flow on the moving boundary, the extensile flow helps the motion of the moving boundary, while the contractile flow resists the motion. Moreover, the external shear flow can weaken this effect of activity on the motion.

Our analysis of the curvature shows there are three main differences between
%different points for the annular channel case from the straight channel case. 
the flows states for the straight channel and the annular channel. First 
%and also most obviously, 
in the annular channel, there is no critical activity for the system to be stable against the spontaneous flow given a nonzero external shear. Second, we only observe one direction of spontaneous flow: positive for extensile gels, but negative for contractile gels.
%(which is same direction with the external shear in the extensile flow but opposite to the shear in the contractile flow) even given different initial conditions.
Last, the average volumetric flow rate of the annular case oscillates with time for the dancing flow state, while it is steady %fixed with time 
in the straight channel.  %case. Besides, 
Also, we find increasing the curvature of the streamlines weakens the dependence of the wall stress on activity.

%Although our work theoretically analyzes the nonlinear flow properties of active gels, %there are
%it still some limitations. 
Our work suggests several directions for future study. An obvious extension is to work in three dimensions, allowing both the directors and velocity vectors to point out of the plane and vary in both directions across a channel. Also, it would be natural to study the effect of aligning flows induced by a pressure gradient rather than a moving wall, since Poiseuille-like flow may be easier to study experimentally.


%For example, we neglect the fluctuations of the directors out of the plane. The theoretical and numerical study of three-dimensional problems are worth to do in the future work. Besides, our continuum hydrodynamic model cannot fully realistically describe some experimental materials, such as that for some passive fluids there is actually no simple shear flow.
%\trp{discuss limitations such as neglect of fluctuations of the directors out of the plane}

% \textbf{Discuss}~\cite{PengBrady2022}

% apply to other geometries and base flows. 
% More recently, A more recent work showed shear-thickening properties of a contractile active nematic subject to a Poiseuille flow. \cite{MackayTonerMorozovMarenduzzo2020}

% \section{Appendices}


% %% Figure environment removed

% Fig.~\ref{fig:force_occi2} shows the oscillation frequency $f_S$ of the \trp{spatially-averaged} wall shear on the moving plate for all cases with wavelength equal to $1.25W$. Part (a) shows that $f_s$ increases  almost linearly with the growth of activity, and part (b) shows that $f_s$ decreases with external shear. %\trp{WHAT ARE WE TO MAKE OF THIS?}
% %For the dancing flow the average value of the oscillation of the wall shear and approximately linear increases with activity.

% %% Figure environment removed

% % Figure environment removed


% \subsection{Spontaneous velocity }
% % Figure environment removed






% \subsection{Weak formulation of the equations}
% Introducing scalar test function $p^*$, vector test function $\mathbf{v}^*$ and tensor test function $\textsf{Q}^*$, the Green formulae with the symmetry of Q and the properties of the inner production ($:$) give
% \begin{align}
% &\int_{\Omega} \nabla^2 \mathbf{v} \cdot \mathbf{v}^* =-\int_{\Omega} \nabla \mathbf{v} : \nabla \mathbf{v}^* +\int_{\partial \Omega} \frac{\partial \mathbf{v}}{\partial \mathbf{n}} \cdot \mathbf{v}^* ,\label{lapv} \\
% &\int_{\Omega}\nabla^2 \mathsf{Q} : \mathsf{Q}^*=-\int_{\Omega}\nabla \mathsf{Q} : \nabla \textsf{Q}^*+\int_{\partial \Omega} \frac{\partial \textsf{Q}}{\partial \mathbf{n}}: \textsf{Q}^*\label{lapQ}\\
% &\int_{\Omega} \nabla p \cdot \mathbf{v}^* =-\int_{\Omega} p \nabla \cdot \mathbf{v}^* +\int_{\partial \Omega} p \mathbf{n} \cdot \mathbf{v}^*.\label{gradp}\\
% &\int_{\Omega} (\nabla \cdot \mathsf{Q})\cdot \mathbf{v}^* =-\int_{\Omega} \mathsf{Q} : \nabla \mathbf{v}^*+\int_{\partial \Omega}\mathbf{v}^*\cdot(\mathsf{Q} \cdot \mathbf{n}). \label{divQ}
% \end{align}
% %where $:$ denotes inner production and our convention is consistent with "grad" and "div" expressions in fenics.
% %Note: the gradient in eqns \ref{lapv}, \ref{lapQ}, \ref{gradp} and \ref{divQ} originally are in convention of $\nabla_{i,:}=\partial_{i}$ ("nablagrad" expression in fenics). But since $Q$ is symmetry and due to the property of inner product, it equals to the convention of $\nabla_{:,i}=\partial_{i}$ ("grad" expression in fenics).

% Thus, the weak form of the dimensionless equations become
% \begin{eqnarray}
% 0&=&\int_{\Omega} p^* \nabla \cdot \mathbf{v},\label{imcompweakform}\\
% 0&=&\int_{\Omega} (-\boldsymbol{\nabla}p+\nabla^2\mathbf{v}-\frac{\alpha }{\lambda} \nabla\cdot\textsf{Q})\cdot \mathbf{v}^*\nonumber\\
% &=&\int_{\Omega} p \nabla \cdot \textbf{v}^*-\int_{\partial \Omega} p \mathbf{v}^* \cdot \mathbf{n} -\int_{\Omega} \nabla \textbf{v} : \nabla \textbf{v*} + \int_{\partial \Omega} \frac{\partial \textbf{v}}{\partial  \textbf{n}} \cdot \textbf{v}^*\nonumber\\
% &-&\frac{\alpha}{\lambda} \bigl(\int_{\partial \Omega}(\mathsf{Q} \cdot \mathbf{n}) \cdot \mathbf{v}^* -\int_{\Omega} \mathsf{Q} : \nabla \mathbf{v}^*),\label{vbeforeBC}\\
% 0&=&\int_{\Omega} \Biggl(-\partial_t \mathsf{Q} -\textbf{v}\cdot \nabla \textsf{Q}-\textsf{Q} \cdot \Omega+\Omega \cdot \textsf{Q}
% - \mathsf{Q}+\ell^2\nabla^2 \mathsf{Q}+ 2\lambda\mathsf{E} \Biggr) : \textsf{Q}^*.\nonumber\\
% &=&\int_{\Omega} \Biggl(-\frac{\textsf{Q}-\textsf{Q}^n}{\Delta t} -\textbf{v} \cdot \nabla \textsf{Q}-\textsf{Q} \cdot \Omega+\Omega \cdot \textsf{Q}-  \mathsf{Q}+ 2\lambda\mathsf{E}\nonumber\\
% &+&\ell^2\nabla^2 \mathsf{Q} \Biggr) : \textsf{Q}^*\nonumber\\
% &=&\int_{\Omega} \frac{\Biggl(-\textsf{Q}+\textsf{Q}^n + \Delta t \biggl(  -\textbf{v} \cdot \nabla \textsf{Q}-\textsf{Q} \cdot \Omega+\Omega \cdot \mathsf{Q} 
% - \mathsf{Q}+2\lambda\mathsf{E} \biggr) \Biggr)}{\Delta t}: \textsf{Q}^*\nonumber\\
% &+&\ell^2\bigl(-\int_{\Omega}\nabla \mathsf{Q} : \nabla \textsf{Q}^*+\int_{\partial \Omega} \frac{\partial \textsf{Q}}{\partial \mathbf{n}}: \textsf{Q}^*\bigr),\label{QbeforeBC}
% \end{eqnarray}
% where $\mathsf{Q}^n$ is in past and $\mathsf{Q}$ is in current. Here, we use a simple backward to differentiate the time-dependence.

% %Note the gradient in $\mathbf{v}\cdot\nabla Q$ is in convention of $\nabla_{i,:}=\partial_{i}$. So this term can be written as "grad(Q)*u" or "n*nablagrad(Q)" in fenics.


% Given Neumann boundary conditions of $\textsf{Q}$ ( $\frac{\partial \textsf{Q}}{\partial \mathbf{n}}=0$) and Dirichlet boundary conditions of $\mathbf{v}$ ($\mathbf{v}^*$=0), the eqns \ref{vbeforeBC} and \ref{QbeforeBC} become
% \begin{eqnarray}
% 0&=&\int_{\Omega} -p \nabla \cdot \textbf{v}^* +\int_{\Omega} \nabla \textbf{v} : \textbf{v*}-\frac{\alpha}{\lambda} \int_{\Omega}  \textsf{Q} : \nabla  \textbf{v}^*,\label{veqnweakform}\\
% 0&=&\int_{\Omega} \Biggl((1+\Delta t)\textsf{Q}-\textsf{Q}^n - \Delta t \Bigl(  -\textbf{v} \cdot \nabla \textsf{Q}-\textsf{Q} \cdot \Omega+\Omega \cdot \mathsf{Q}+2\lambda\mathsf{E} \Bigr) \Biggr) : \textsf{Q}^* \nonumber\\
%  &+& \Delta t \ell^2\int_{\Omega}\nabla \mathsf{Q} : \nabla \textsf{Q}^*.\label{Qeqnweakform}
% \end{eqnarray}

% \subsection{Dimensionless equations}
% \begin{eqnarray}
% 0&=&\nabla \cdot \mathbf{v}\label{incompdimless}\\
% 0&=&-\boldsymbol{\nabla}p+\nabla^2\mathbf{v}-\frac{\bar{a}}{\lambda}\nabla\cdot\textsf{Q},\label{veqnfulldimless}\\
% 0&=&-\partial_t\textsf{Q}-\mathbf{v}\cdot \nabla \textsf{Q}-\textsf{Q} \cdot \Omega+\Omega \cdot \textsf{Q}
% -\mathsf{Q}+\ell^2\nabla^2 \mathsf{Q}\nonumber\\
% &+&\lambda\Bigl(2\mathsf{E}+\mathsf{Q}\cdot\mathsf{E}+\mathsf{E}\cdot\mathsf{Q}-\frac{2}{d}\mathrm{tr}(\mathsf{Q}\cdot\mathsf{E})\mathsf{I}\Bigr).\label{Qeqnfulldimless}
% \end{eqnarray}

% \section{Linearized results for steady states in the stable region}
% Assume velocity and order parameter tensor $\mathbf{v}=[v_x[y],0]$ and $\mathsf{Q}=[[Q_{xx}[y],Q_{xy}[y]],[Q_{xy}[y],-Q_{xx}[y]]]$. Also, assume there is no pressure gradient. The linearized equations, eqns (\ref{veqn}) and (\ref{Qeqn}) can be simplified to
% \begin{eqnarray}
% 0&=&\eta  {v_x}''(y)-a Q_{xy}'(y)\label{veqnx}\\
% 0&=&a Q_{xx}'(y)\label{veqny}\\
% 0&=&K Q_{xx}''(y)-A Q_{xx}(y)\label{Qeqnxx}\\
% 0&=&-A Q_{xy}(y)+2 \lambda  \nu  {v_x}'(y)+K Q_{xy}''(y)\label{Qeqnxy}.
% \end{eqnarray}
% With the no-slip and zero ordering boundary conditions, from eqns (\ref{veqny}) and (\ref{Qeqnxx}), we get $Q_{xx}=0$. Solving eqns (\ref{veqnx}) and (\ref{Qeqnxy}), we get
% \begin{eqnarray}
% {Q_{xy}}(y)&=& -\frac{\lambda  \nu  {u_0}}{A W}\label{Qxy_linear}\\
% {v_x}(y) &=& {u_0}(1-\frac{y}{W})
% \end{eqnarray}
% So the shear stress is \begin{equation}
%     \sigma_{xy}=\eta (\nabla \times \mathbf{v})_{xy}-a Q_{xy}= \eta(-1+\frac{a \lambda \nu}{A \eta})\frac{u_0}{W}=\eta(-1+\bar{a})\frac{u_0}{W}.
% \end{equation}
% Normalized active shear stress with the Newtonian case,
% \begin{equation}
%     \bar{\sigma}_{xy}= (1-\bar{a}).
% \end{equation}
% Since the linearized equations only work for the weak order case ($Q_{xy}$ is small, and thus $u_0$ is small), this normalized stress only works for the small $u_0$.

% If $u_0$ is not small, \ref{Qxy_linear} is not valid. So in the sub-critical case, the normalized stress is given by
% \begin{equation}
%     \bar{\sigma}_{xy}= 1+(Q_{xy}/u_0) W a.
% \end{equation}
% W is the unit length. So the relationship between activity and normalized shear should be the linear and its slope should be $Q_{xy}/u_0$.



% \section{Relaxation time of the linearized model in absence of Frank free energy}
% Assume $\mathbf{v}=[v_x(y,t),0]$ and $\mathsf{Q}=[[0,Q_{xy}(y,t)],[Q_{xy}(y,t),0]]$.
% Solving the linearized \ref{Qeqn} we can get
% \begin{equation}
%     \frac{\partial v_x}{\partial y}=\frac{2(A Q_{xy}+\nu \frac{\partial Q_{xy}}{\partial t})}{\lambda \nu}, \label{relaxationv}
% \end{equation}
% Substituting eqn(\ref{relaxationv}) into eqn(\ref{veqn}),
% \begin{equation}
%     \frac{\partial^2 Q_{xy}}{\partial y^2}+\frac{\nu}{A} \frac{\partial}{\partial t} \frac{\partial^2 Q_{xy}}{\partial y^2}=\frac{a \lambda \nu}{ A d\eta} \frac{\partial^2 Q_{xy}}{\partial y^2}
% \end{equation}
% So 
% \begin{equation}
%  \frac{\partial^2 Q_{xy}}{\partial y^2}=\frac{\partial^2 Q_{xy}}{\partial y^2}(t=0) \rm{exp}(A/\nu (\bar{a}-1))
% \end{equation}
% The relaxation time of an linearized active fluid in absence of Frank free energy is $\eta\nu/(A(\eta-a\tau))$.

% \section{Solutions of pure liquid crystal between two circles}
% For a steady state, the dimensionless equation of pure liquid crystal is
% \begin{equation}
%     -\textsf{Q}+\ell^2 \nabla^2 \textsf{Q}=0
% \end{equation}
% With a hometropic boundary condition on the inner circle, the solution of order parameter is
% \begin{eqnarray}
%     Q_{rr}&=&\frac{I_2(\frac{R_2}{\ell})K_2(\frac{r}{\ell})-K_2(\frac{R_2}{\ell})I_2(\frac{r}{\ell})}{I_2(\frac{R_2}{\ell})K_2(\frac{1}{\ell})-K_2(\frac{R_2}{\ell})I_2(\frac{1}{\ell})}\\
%     Q_{r\phi}&=&0.
% \end{eqnarray}
% For a tangential case, the formulas are the same but the sign of $Q_{rr}$ is opposite.

% \section{Linear instability analysis without externally imposed shear}
% %The critical activity for the motionless isotropic state in absence of the Frank elasticity to be unstable to spontaneous shear flow and the development of orientational order when $\bar{a}>1$. 

% %In this motionless, a small orientational order is given as initial values in the whole region. Initial $\mathsf{Q}_{xx}$ is $0.169$ and $\mathsf{Q}_{xy}$ is given by a random number between $-0.001$ to $0.001$. 

% %{\color{red}I find the spontaneous flow goes to the $-x$ direction when initial $Q_{xy}$ is positive and goes to the $+x$ direction while the initial $Q_{xy}$ is negative.}


% First we review the instability of an active isotropic gel confined between two lines for the case of Neumann boundary conditions on the order parameter tensor $\partial_y Q_{ij}(y=0$ and $y=W)=0$. Linearizing the equations, we have $\boldsymbol{\nabla}\cdot\mathbf{v}=0$ and
% \begin{eqnarray}
% \eta\nabla^2\mathbf{v}-\boldsymbol{\nabla}p-a\boldsymbol{\nabla}\cdot\mathsf{Q}&=&0\label{flineqn}\\
% \nu\partial_t\mathsf{Q}
% +\left(A-K\nabla^2\right)\mathsf{Q}-2\lambda\nu\mathsf{E}&=&0.\label{dimensionQ}
% \end{eqnarray}
% Choose the $x$ axis to be along the direction of the perturbing flow. To satisfy the no-slip boundary condition and incomprehensibility, we suppose
% \begin{equation}
% v_x=v_n(y)\exp\left[\beta t\right]\label{dimensionvF}
% \end{equation}
% where $v_n(y)=\hat{v}_x\cos(n\pi y/W)$ for $n$ an odd integer, and $v_n(y)=\hat{v}_x\sin(n\pi y/W)$ for $n$ even. To enforce the Neumann conditions on $\mathsf{Q}$ we require $\mathsf{Q}=\hat{\mathsf{Q}}\sin(n\pi y/W)\exp[\beta t ]$ for $n$ odd and $\mathsf{Q}=\hat{\mathsf{Q}}\cos(n\pi y/W)\exp[\sigma t ]$ for $n$ even. To simplify the discussion, in the remainder of this section we will only display the formulas for $n$ odd. The formulas for $n$ even are similar, although some of the signs are opposite. With these assumptions, the $x$ component of the force equation, eqn(\ref{flineqn}) implies $\hat{Q}_{xz}=0$ and 
% \begin{equation}
% \hat{v}_x=-\frac{a \hat{Q}_{xy}}{\eta(n \pi/W)}.\label{vxandQxy}
% \end{equation}
% % Assuming $p=[p_1\sin(n\pi y/h)+p_2\cos(n\pi y/h)]\exp[\sigma ]$, the $y$ component of Eq.~(\ref{flineqn}) implies $p_1=-a\hat{Q}_{yy}$.

% Only the  $xx$ and $xy$ equations for $\mathsf{Q}$ are nontrivial. $Q_{xx}$ relax with rate 
% \begin{equation}
% \beta_1=-\frac{A+K n^2\pi^2/W^2}{\nu}.
% \end{equation}
% This mode corresponds to the simple relaxation of the order parameter, with no backflow: $\hat{v}_x=0$.
% The other mode has relaxation rate
% \begin{equation}
% \beta_2=\beta_1+\frac{\lambda a}{\eta}=-\frac{A+K n^2\pi^2/W^2}{\nu}+\frac{\lambda a}{\eta}.
% \end{equation}
% with $\hat{v}_x$ related to $\hat{Q}_{xy}$ via Eq.~(\ref{vxandQxy}). 

% %Btw, the relaxation time is 
% %\begin{equation}
% %\tau=\frac{\nu}{A}\frac{1}{1+\ell^2 n^2\pi^2-\lambda a \nu/(2\eta 5A)}=\frac{\nu}{A}\frac{1}{1+\ell^2 n^2\pi^2-\bar{a}}.
% %\end{equation}

% Confinement suppresses the activity-driven instability to shear and spontaneous flow, with the critical activity given by 
% \begin{equation}
% a_c^{\rm{motionless}}=\frac{(A+K n^2\pi^2/W^2)\eta}{\lambda\nu}.
% \end{equation}
%  The critical value for an active isotropic gel confined between two lines depends on the correlation length of the ordering (when considering the Frank elasticity). With the increase of correlation length, the activity to spontaneous shear flow is larger. For the first order linear stability, the dimensionless critical value is 
% \begin{equation}
% \bar{a}_c^{\rm{motionless}}=\left ( 1+\pi^2 \ell^2\right).
% \end{equation}
% This is consistent with the results of Varghese et al.\cite{VargheseBaskaranHaganBaskaran2020}


% \section{Linear instability of disordered state on a disk (no outer boundary) {\color{blue} need check}}
% The radius of the disk is $R_1=1$. Choose the $\phi$ direction to be along the direction of the perturbing flow. In order to find the formation of disturbing variables, we first review the analytical solutions to eqns(\ref{dimensionQ}) and (\ref{dimensionvF}) for a rotating disk with no-slip boundary condition, $Q_{rr}(r=1)=-S_0$ and $Q_{r\phi}=0$ on the disk. The stream function is
% \begin{equation}
%     \psi=\frac{2 a \lambda  K \nu  K_0\left(r \xi \right)+A \log (r) (2 A \eta -a \lambda  \nu ) K_2\left(\xi\right)}{A (a \lambda  \nu -2 A \eta ) K_2\left(\xi \right)+a \lambda  K \nu  \sqrt{\frac{4 A \eta -2 a \lambda  \nu }{\eta  K}} K_1\left(\xi \right)},
% \end{equation}
% where $\xi=\sqrt{\frac{A}{K}-\frac{a \lambda  \nu }{2 K \eta }}$.
% \begin{equation}
%     v_\phi=\frac{A (a \lambda  \nu -2 A \eta ) K_2\left(\xi \right)+a \lambda  K \nu  r \sqrt{\frac{4 A \eta -2 a \lambda  \nu }{\eta  K}} K_1\left(r \xi \right)}{A r (a \lambda  \nu -2 A \eta ) K_2\left(\xi\right)+a \lambda  K \nu  r \sqrt{\frac{4 A \eta -2 a \lambda  \nu }{\eta  K}} K_1\left(\xi \right)}.
% \end{equation}
% \begin{equation}
%     Q_{rr}=-\frac{S_0 K_2\left(\sqrt{\frac{A}{K}} r\right)}{2 K_2\left(\sqrt{\frac{A}{K}}\right)}.
% \end{equation}
% \begin{equation}
%     Q_{r\phi}=\frac{\lambda  \nu  (a \lambda  \nu -2 A \eta ) \left(r^2 K_2\left(r \xi \right)-K_2\left(\xi \right)\right)}{r^2 \left(A (a \lambda  \nu -2 A \eta ) K_2\left(\xi \right)+a \lambda  K \nu  \sqrt{\frac{4 A \eta -2 a \lambda  \nu }{\eta  K}} K_1\left(\xi \right)\right)}.
% \end{equation}

% For the motionless case, we suppose the formation of disturbing are the first or second Bessel function instead of modified Bessel function due to the change of velocity boundary condition on the disk.
% \begin{equation}
% v_{\phi}=\hat{v}_{\phi} J_1(r)\exp\left[\beta t\right]\ \mathrm{or} \ \hat{v}_{\phi} Y_1(r)\exp\left[\beta t\right]
% \end{equation}
% and
% \begin{equation}
% \mathsf{Q}=\hat{\mathsf{Q}} J_2(r) \exp[\beta t ]\ \mathrm{or} \ \hat{\mathsf{Q}} Y_2(r) \exp[\beta t ].
% \end{equation}
% With these assumptions, the $x$ component of the force equation, eqn(\ref{flineqn}) implies $\hat{Q}_{xz}=0$ and 
% \begin{equation}
% \hat{v}_{\phi}=-\frac{a \hat{Q}_{r\phi}}{\eta }.
% \end{equation}

% $Q_{rr}$ relax with rate
% \begin{equation}
%     \beta_1=-\frac{A+K}{\nu }.
% \end{equation}

% $Q_{r\phi}$ relax with rate
% \begin{equation}
%     \beta_2=\beta_1+\frac{a \lambda }{2 \eta }=-\frac{A+K}{\nu }+\frac{\lambda a}{2 \eta }.
% \end{equation}
% The relation of $\beta_1$ and $\beta_2$ is consistent with the channel case. The relaxation time is 
% \begin{equation}
% \tau=\frac{\nu}{A}\frac{1}{1+(K/A)^2-\lambda a \nu/(2\eta A)}=\frac{\nu}{A}\frac{1}{1+(K/A)^2-\bar{a}}.
% \end{equation}
% When neglecting the Frank elasticity, the relaxation time is same with the channel case. 

% The critical activity of the disk geometry is given by
% \begin{equation}
%   a_c^{\rm{motionless}}=\frac{2 \eta  (A+K)}{\lambda  \nu }.
% \end{equation}
% \begin{equation}
%   \bar{a}_c^{\rm{motionless}}= 1+K/A.
% \end{equation}
%  Again the Frank elasticity makes the system more stable and larger correlation length makes the system more stable.

% \section{Form for a biharmonic function in cylindrical polar coordinates}

% In polar coordinates, the general form for a biharmonic is given by the Michell solution
% ~\cite{Michell1899}:
% \begin{eqnarray}
% \chi&=&A_0 r^2+B_0r^2\log r+C_0\log r\nonumber\\
% &+& \left( A_1 r+B_1/r+C_1 r^3+D_1 r\log r\right)\cos\phi\nonumber\\
% &+& \left( E_1 r+F_1/r+G_1 r^3+H_1 r\log r\right)\sin\phi\nonumber\\
% &+&\sum_{m=2}^\infty\left( A_m r^m+B_mr^{-m}+C_m r^{m+2}+D_m r^{2-m}\right)\cos m\phi\nonumber\\
% &+&\sum_{m=2}^\infty\left( E_m r^m+F_mr^{-m}+G_m r^{m+2}+H_m r^{2-m}\right)\sin m\phi,\nonumber\\
% \end{eqnarray}
% where we have only retained terms that are periodic in $\phi$. 

% \section{Stokes flow in the region between two concentric spheres} 
% \label{spheresHB}
% Here we recall from Ref.~\cite{happel_brenner1965} the expression for the force on a motionless sphere of radius $R_1$ in a fluid obeying Stokes equations with boundary condition $\mathbf{v}=\mathbf{V}$ at $r=R_2$. For a fluid with viscosity $\eta$, the force is given by $F=6\pi\eta V K$, where
% \begin{equation}
% K=\frac{4R_2(R_1^4+R_1^3R_2+R_1^2R_2^2+R_1R_2^3+R_2^4)}{(R_2-R_1)^3(4R_1^2+7R_1R_2+4R_2^2)}.
% \end{equation}
% \newline

%\section{Vector calculus in spin-weight components}
%
%For completeness we review the formulas for various vector derivatives in spin weight components~\cite{Torres2003}. We define the spherical basis as 
%\begin{eqnarray}
%\hat{\mathbf{e}}_\pm&=&\pm\frac{1}{\sqrt{2}}\left(\hat{\boldsymbol{\theta}}\mp\mathrm{i}\hat{\boldsymbol{\phi}}\right)\\
%\hat{\mathbf{e}}_0&=&-\sqrt{2}\hat{\mathbf{r}}.
%%\hat{\mathbf {e}}_\pm&=&\pm\frac{1}q{\sqrt{2}}\left(\boldsymbol{\theta}\right)\\
%%\hat{\mathbf {e}}_0&=&-\sqrt\hat{\mathbf{r}}
%\end{eqnarray}
%Under a a change of basis by rotation of the vectors $\hat{\boldsymbol{\theta}}$ and $\hat{\boldsymbol{\phi}}$ by $-\alpha$ about $\hat{\mathbf{r}}$, these vectors transform as
%\begin{equation}
%\hat{\mathbf{e}}_\pm^\prime=\mathrm{e}^{\mp\mathrm{i}\alpha}\hat{\mathbf{e}}_\pm.
%\end{equation}
%We say that $\hat{\mathbf{e}}_\pm$ has spin weight $\mp1$, and $\hat{\mathbf{e}}_0$ has spin weight 0. Since the vectors $\hat{\mathbf{e}}_\pm$ are complex, we find the components of a vector by dotting with the complex conjugates of the basis vectors: $v_{\pm 1}=\hat{\mathbf{e}}_\pm^*\cdot\mathbf{v}$. (Note that $\hat{\mathbf{e}}_0\cdot \hat{\mathbf{e}}_0=2$.) The vector $\mathbf{v}$ is invariant under the change of basis. Thus, the components $v_{\pm1}=\pm1/\sqrt{2}(v_\theta\pm \mathrm{i}v_\phi)$ have spin weight $\pm 1$, and $v_0=-v_r/\sqrt{2}$ has spin weight $0$. 
%
%In the same way we calculate the spin weight components of a traceless symmetric second rank tensor to find
%\begin{eqnarray}
%Q_0&=&\frac{1}{4}\hat{\mathbf{e}}_0\cdot\mathsf{Q}\cdot\hat{\mathbf{e}}_0=\frac{1}{2}Q_{rr}\label{Q0def}\\
%Q_{\pm1}&=&\frac{1}{2}\hat{\mathbf{e}}_\pm^*\cdot\mathsf{Q}\cdot\hat{\mathbf{e}}_0=\mp\frac{1}{2}(Q_{r\theta}\pm\mathrm{i}Q_{r\phi})\label{Qpm1def}\\
%Q_{\pm2}&=&\hat{\mathbf{e}}_\pm^*\cdot\mathsf{Q}\cdot\hat{\mathbf{e}}_\pm^*=\frac{1}{2}(Q_{\theta\theta}-Q_{\phi\phi}\pm2\mathrm{i}Q_{\theta\phi})\label{Qpm2def1}\\
%&=&\frac{1}{2}(Q_{rr}+2Q_{\theta\theta}\pm2\mathrm{i}Q_{\theta\phi}).
%\end{eqnarray}
%
%The gradient along the direction $\hat{\mathbf{e}}_\pm$ is given by 
%\begin{equation}
%\hat{\mathbf{e}}^*_\pm\cdot\boldsymbol{\nabla}=\pm\frac{1}{r\sqrt{2}}\left(\partial_\theta\pm\frac{\mathrm{i}}{\sin\theta}\partial_\phi\right),
%\end{equation}
%which raises or lowers by one the spin weight of whatever object it acts, and therefore can be considered as an operator with spin weight $+1$ for $+$ and $-1$ for $-$. For an object $f$ with spin weight $s$, it is convenient to define the raising and lowering operators
%\begin{eqnarray}
%\eth f&=&-\sin^s\theta\left(\partial_\theta+\frac{\mathrm{i}}{\sin\theta}\partial_\phi\right)\left[\sin^{-s}\theta f\right]\label{ethdef}\\
%\bar{\eth} f&=&-\sin^{-s}\theta\left(\partial_\theta-\frac{\mathrm{i}}{\sin\theta}\partial_\phi\right)\left[\sin^{s}\theta f\right] ,\label{ethbardef}
%\end{eqnarray}
%where the symbol $\eth$ is pronounced ``eth." If $f$  has spin weight $s$, then
%\begin{equation}
%\left(\bar{\eth}\eth-\eth\bar{\eth}\right)f=2sf,
%\end{equation}
%which is a convenient relation for some of our calculations.
%
%We can now write the strain rate $\mathsf{E}$ in terms of spin weighted components. Using the standard definition of the strain rate in spherical polar coordinates~\cite{landauFM} and the relations~(\ref{Q0def}--\ref{Qpm2def1}) and (\ref{ethdef}, \ref{ethbardef}), we find
%\begin{eqnarray}
%\mathsf{E}_{+2}&=&-\frac{1}{r\sqrt{2}}\eth v_{+1}\\
%\mathsf{E}_{+1}&=&-\frac{1}{2r\sqrt{2}}\left[r^2\partial_r(v_{+1}/r)+\eth v_0\right]\\
%\mathsf{E}_0&=&-\frac{1}{\sqrt{2}}\partial_r v_0\\
%\mathsf{E}_{-1}&=&-\frac{1}{2r\sqrt{2}}\left[r^2\partial_r(v_{-1}/r)-\bar{\eth} v_0\right]\\
%\mathsf{E}_{-2}&=&\frac{1}{r\sqrt{2}}\bar{\eth}v_{-1}
%\end{eqnarray}
%
%
%Using the formula for the gradient in spherical polar coordinates and the definitions we have just introduced, we find that
%\begin{equation}
%\boldsymbol{\nabla}f=-\frac{1}{\sqrt{2}}\left(\hat{\mathbf{e}}_0\partial_r f+\hat{\mathbf{e}}_+\frac{1}r\eth f-\hat{\mathbf{e}}_-\frac{1}r\bar{\eth}f\right).
%\end{equation}
%Likewise, the divergence and curl of a vector $\mathbf{v}$ are given by 
%\begin{equation}
%\boldsymbol{\nabla}\cdot\mathbf{v}=-\frac{1}{\sqrt{2}}\left[\frac{2}{r^2}\partial_r(r^2v_0)+\frac{1}{r}\left(\bar{\eth}v_{+1}-\eth v_{-1}\right)\right],
%\end{equation}
%and
%\begin{eqnarray}
%\boldsymbol{\nabla}\times\mathbf{v}&=&-\frac{\mathrm{i}}{2r}\hat{\mathbf{e}}_0\left(\eth v_{-1}+\bar{\eth}v_{+1}\right)\nonumber\\
%&+&\frac{\mathrm{i}}{r}\hat{\mathbf{e}}_+\left[\partial_r\left(r v_{+1}\right)-\eth v_0\right]\nonumber\\
%&-&\frac{\mathrm{i}}{r}\hat{\mathbf{e}}_-\left[\partial_r\left(r v_{-1}\right)+\bar{\eth} v_0\right].\label{curlvspher}
%\end{eqnarray}
%Thus, the Laplacian of a scalar function is
%\begin{equation}
%\Delta f=\frac{1}{r^2}\partial_r(r^2\partial_r f)+\frac{1}{r^2}\bar{\eth}\eth f.
%\end{equation}
%We also need the Laplacian of a vector:
%\begin{eqnarray}
%\Delta\mathbf{v}&=&\hat{\mathbf{e}}_+\left[\frac{1}{r}\partial^2_r(r v_{+1})+\frac{1}{r^2}\eth\bar{\eth} v_{+1}+\frac{2}{r^2}{\eth}v_0\right]\nonumber\\
%&+&\hat{\mathbf{e}}_-\left[\frac{1}{r}\partial^2_r(r v_{-1})+\frac{1}{r^2}\bar{\eth}\eth v_{-1}-\frac{2}{r^2}\bar{\eth}v_0\right]\nonumber\\
%&+&\hat{\mathbf{e}}_0\left[\partial_r\frac{1}{r^2}\partial_r(r^2v_0)+\frac{1}{r^2}\bar{\eth}\eth v_0+\frac{1}{r^2}\eth v_{-1}\right.\nonumber\\
%&-&\left.\frac{1}{r^2}\bar{\eth}v_{+1}\right]
%\end{eqnarray}
%
%
%The divergence of a symmetric traceless second-rank tensor is given by~\cite{TorresRojas1993} 
%\begin{equation}
%\left(\boldsymbol{\nabla}\cdot\mathsf{Q}\right)_s=\frac{1}{\sqrt{2}}\left[-\frac{1}{r}\bar{\eth}Q_{s+1}-\frac{2}{r^3}\partial_r(r^3Q_s)+\frac{1}{r}\eth Q_{s-1}\right].\label{divQspher}
%\end{equation}
%We need the curl of the divergence of a traceless symmetric second rank tensor, which we can calculate using Eqs.~(\ref{curlvspher}) and~(\ref{divQspher}):
%\begin{widetext}
%\begin{eqnarray}
%\left[\boldsymbol{\nabla}\times\left(\boldsymbol{\nabla}\cdot\mathsf{Q}\right)\right]_{+1}&=&-\frac{\mathrm{i}}{\sqrt{2}r}\left[\partial_r\bar{\eth}Q_{+2}+2\partial_r\frac{1}{r^2}\partial_r \left(r^3Q_{+1}\right)-\frac{1}{r}\eth\bar{\eth}Q_{+1}-\frac{3}{r^2}\partial_r\left(r^2\eth Q_0\right)+\frac{1}{r}\eth\eth Q_{-1}\right]\\
%\left[\boldsymbol{\nabla}\times\left(\boldsymbol{\nabla}\cdot\mathsf{Q}\right)\right]_{0}&=&\frac{\mathrm{i}}{\sqrt{2}r^2}\left\{\frac{1}{2}\bar{\eth}\bar{\eth}Q_{+2}+\frac{1}{r^2}\partial_r\left[r^3(\eth Q_{-1}+\bar{\eth}Q_{+1})\right]-\frac{1}{2}\eth\eth Q_{-2}\right\}\\
%\left[\boldsymbol{\nabla}\times\left(\boldsymbol{\nabla}\cdot\mathsf{Q}\right)\right]_{-1}&=&-\frac{\mathrm{i}}{\sqrt{2}r}\left[\partial_r{\eth}Q_{-2}-2\partial_r\frac{1}{r^2}\partial_r \left(r^3Q_{-1}\right)+\frac{1}{r}\bar{\eth}{\eth}Q_{-1}-\frac{3}{r^2}\partial_r\left(r^2\bar{\eth} Q_0\right)-\frac{1}{r}\bar{\eth}\bar{\eth} Q_{1}\right].
%\end{eqnarray}
%\end{widetext}
%
%Finally, we write out the spin-weight components of the Laplacian of a traceless symmetric second-rank tensor~\cite{TorresRojas1993}:
%\begin{widetext}
%\begin{eqnarray}
%(\Delta\mathsf{Q})_{+2}&=&\frac{1}{r^2}\partial_r\left(r^2\partial_rQ_{+2}\right)+\frac{1}{r^2}\eth\bar{\eth}Q_{+2}+\frac{4}{r^2}\eth Q_{+1}\\
%(\Delta\mathsf{Q})_{+1}&=&\frac{1}{r^2}\partial_r\left(r^2\partial_rQ_{+1}\right)-\frac{4}{r^2}Q_{+1}+\frac{1}{r^2}\eth\bar{\eth}Q_{+1}-\frac{1}{r^2}\bar{\eth}Q_{+2}+\frac{3}{r^2}\eth Q_0\\
%(\Delta\mathsf{Q})_{0}&=&\frac{1}{r^2}\partial_r\left(r^2\partial_rQ_{0}\right)-\frac{6}{r^2}Q_0+\frac{1}{r}\eth\bar{\eth}Q_0+\frac{2}{r^2}\left(\eth Q_{-1}-\bar{\eth}Q_{+1}\right)\\
%(\Delta\mathsf{Q})_{-1}&=&\frac{1}{r^2}\partial_r\left(r^2\partial_rQ_{-1}\right)-\frac{4}{r^2}Q_{-1}+\frac{1}{r^2}\bar{\eth}\eth Q_{-1}+\frac{1}{r^2}{\eth}Q_{-2}-\frac{3}{r^2}\bar{\eth} Q_0\\
%(\Delta\mathsf{Q})_{-2}&=&\frac{1}{r^2}\partial_r\left(r^2\partial_rQ_{-2}\right)+\frac{1}{r^2}\bar{\eth}{\eth}Q_{-2}-\frac{4}{r^2}\bar{\eth }Q_{-1}
%\end{eqnarray}
%\end{widetext}
%
%Next we define the spin $s$ spherical harmonics ${}_s Y_{lm}$ via
%\begin{equation}
%{}_s Y_{lm}(\theta,\phi)=[(l-s)!/(l+s)!]^{1/2}\eth^sY_{lm}(\theta,\phi)
%\end{equation}
%for integer $s$, $l$, $m$ and $0\le s\le l$, and 
%\begin{equation}
%{}_s Y_{lm}(\theta,\phi)=[(l+s)!/(l-s)!]^{1/2}(-)^s\bar{\eth}^{-s}Y_{lm}(\theta,\phi)
%\end{equation}
%for $-l\le s\le 0$. From the definition we see that spin $s$ spherical harmonics have spin weight $s$, and we can show that ~\cite{Goldberg_etal1967}:
%\begin{eqnarray}
%{}_sY_{lm}^*&=&(-)^{m+s}{}_{-s}Y_{lm}\\
%\eth({}_s Y_{lm})&=&[(l-s)(l+s+1)]^{1/2} {}_{s+1}Y_{lm}\\
%\bar{\eth}({}_s Y_{lm})&=&-[(l+s)(l-s+1)]^{1/2} {}_{s-1}Y_{lm}\\
%\bar{\eth}\eth({}_sY_{lm})&=&-(l-s)(l+s+1){}_sY_{lm}.
%\end{eqnarray}

% \section{Weak formulation of the equations}
% According to the Green formulae, we have
% \begin{align}
% &-\int_{\Omega} \Delta \mathbf{v} \cdot \mathbf{v}^* =\int_{\Omega} \nabla \mathbf{v} \cdot \nabla \mathbf{v}^* -\int_{\partial \Omega} \frac{\partial \mathbf{v}}{\partial \mathbf{n}} \cdot \mathbf{v}^* , \\
% &\int_{\Omega}\nabla^2 \mathsf{Q} \cdot \mathsf{Q}^*=-\int_{\Omega}\nabla \mathsf{Q} \cdot \nabla \textsf{Q}^*+\int_{\partial \Omega} \frac{\partial \textsf{Q}}{\partial \mathbf{n}}\cdot \textsf{Q}^*\\
% &\int_{\Omega} \nabla p \cdot \mathbf{v}^* =-\int_{\Omega} p \nabla \cdot \mathbf{v}^* +\int_{\partial \Omega} p \mathbf{n} \cdot \mathbf{v}^*.\\
% &\int_{\Omega} (\nabla \cdot \mathsf{Q}) \cdot \mathbf{v}^* =\int_{\partial \Omega}(\mathsf{Q} \cdot \mathbf{n}) \cdot \mathbf{v}^* -\int_{\Omega} \mathsf{Q} : \nabla \mathbf{v}^*
% \end{align}

% Introducing scalar test function $p^*$, vector test function $\mathbf{v}^*$ and  tensor test function $\textsf{Q}^*$, the weak form of the dimensionless equations become
% \begin{eqnarray}
% 0&=&\int_{\Omega} p^* \nabla \cdot \mathbf{v},\\
% 0&=&\int_{\Omega} (-\boldsymbol{\nabla}p+\nabla^2\mathbf{v}-\frac{\bar{a} }{\lambda} \nabla\cdot\textsf{Q})\cdot \mathbf{v}^*\nonumber\\
% &=&\int_{\Omega} p \nabla \cdot \textbf{v}^*-\int_{\partial \Omega} p \mathbf{v}^* \cdot \mathbf{n} \nonumber\\
% &-&\int_{\Omega} \nabla \textbf{v} \cdot \nabla \textbf{v*} + \int_{\partial \Omega} \frac{\partial \textbf{v}}{\partial  \textbf{n}} \cdot \textbf{v}^*\nonumber\\
% &-&\frac{\bar{a}}{\lambda} \bigl(\int_{\partial \Omega}(\mathsf{Q} \cdot \mathbf{n}) \cdot \mathbf{v}^* -\int_{\Omega} \mathsf{Q} : \nabla \mathbf{v}^*),\\
% 0&=&\int_{\Omega} \Biggl(-\frac{\textsf{Q}-\textsf{Q}^n}{\Delta t} -\mathbf{v}\cdot \nabla \textsf{Q}-\textsf{Q} \cdot \Omega+\Omega \cdot \textsf{Q}\nonumber\\
% &-& \Bigl(1+\frac{C}{A}\mathrm{tr}(\mathsf{Q}^2)\Bigr) \mathsf{Q}+\ell^2\nabla^2 \mathsf{Q}\nonumber\\
% &+& \lambda\Bigl(2\mathsf{E}+\mathsf{Q}\cdot\mathsf{E}+\mathsf{E}\cdot\mathsf{Q}-\frac{2}{d}\mathrm{tr}(\mathsf{Q}\cdot\mathsf{E})\mathsf{I}\Bigr) \Biggr) \cdot \textsf{Q}^*.\nonumber\\
% 0&=&\int_{\Omega} \Biggl(-\textsf{Q}+\textsf{Q}^n + \Delta t \biggl(  -\mathbf{v}\cdot \nabla \textsf{Q}-\textsf{Q} \cdot \Omega+\Omega \cdot \mathsf{Q} \nonumber\\
% &-&\Bigl(1+\frac{C}{A}\rm{tr}(\mathsf{Q}^2)\Bigr) \mathsf{Q}\nonumber\\
% &+&\lambda \Bigl( 2\mathsf{E}+\mathsf{Q}\cdot\mathsf{E}+\mathsf{E}\cdot\mathsf{Q}-\frac{2}{d}\mathrm{tr}(\mathsf{Q}\cdot\mathsf{E})\mathsf{I}\Bigr) \biggr) \Biggr) \cdot \textsf{Q}^* \nonumber\\ 
% &+&\Delta t \ell^2\bigl(-\int_{\Omega}\nabla \mathsf{Q} \cdot \nabla \textsf{Q}^*+\int_{\partial \Omega} \frac{\partial \textsf{Q}}{\partial \mathbf{n}}\cdot \textsf{Q}^*\bigr).
% \end{eqnarray}
% Given Neumann boundary conditions on $\textsf{Q}$ ( $\frac{\partial \textsf{Q}^*}{\partial \mathbf{n}}=0$) and Dirichlet boundary conditions on $\mathbf{v}$ ($\mathbf{v}^*$=0), the equations become
% \begin{eqnarray}
% 0&=&\int_{\Omega} p^* \nabla \cdot \mathbf{v},\\
% 0&=&\int_{\Omega} p \nabla \cdot \textbf{v}^*-\int_{\Omega} \nabla \textbf{v} \cdot \nabla \textbf{v*}+\frac{\bar{a}}{\lambda} \int_{\Omega}  \textsf{Q} : \nabla  \textbf{v}^*,\\
% 0&=&\int_{\Omega} \Biggl(-\textsf{Q}+\textsf{Q}^n + \Delta t \Bigl(  -\mathbf{v}\cdot \nabla \textsf{Q}-\textsf{Q} \cdot \Omega+\Omega \cdot \mathsf{Q} \nonumber\\
%  &-&\bigl(1+\frac{C}{A}\rm{tr}(\mathsf{Q}^2)\bigr) \mathsf{Q}\nonumber\\
%  &+&\lambda\bigl(2\mathsf{E}+\mathsf{Q}\cdot\mathsf{E}+\mathsf{E}\cdot\mathsf{Q}-\frac{2}{d}\mathrm{tr}(\mathsf{Q}\cdot\mathsf{E})\mathsf{I}\bigr) \Bigr) \Biggr) \cdot \textsf{Q}^* \nonumber\\
% &-&\Delta t \ell^2\int_{\Omega}\nabla \mathsf{Q} \cdot \nabla \textsf{Q}^*.
% \end{eqnarray}
% When neglecting the forth power term in the free energy, they become
% \begin{eqnarray}
% 0&=&\int_{\Omega} p^* \nabla \cdot \mathbf{v},\label{imcompweakform}\\
% 0&=&\int_{\Omega} p \nabla \cdot \textbf{v}^*-\int_{\Omega} \nabla \textbf{v} \cdot \nabla \textbf{v*}+\frac{\bar{a}}{\lambda} \int_{\Omega}  \textsf{Q} : \nabla  \textbf{v}^*,\label{veqnweakform}\\
% 0&=&\int_{\Omega} \Biggl(-(1+\Delta t)\textsf{Q}+\textsf{Q}^n + \Delta t \Bigl(  -\mathbf{v}\cdot \nabla \textsf{Q}-\textsf{Q} \cdot \Omega+\Omega \cdot \mathsf{Q} \nonumber\\
% &+&\lambda\bigl(2\mathsf{E}+\mathsf{Q}\cdot\mathsf{E}+\mathsf{E}\cdot\mathsf{Q}-\frac{2}{d}\mathrm{tr}(\mathsf{Q}\cdot\mathsf{E})\mathsf{I}\bigr) \Bigr) \Biggr) \cdot \textsf{Q}^* \nonumber\\
% &-& \Delta t \ell^2\int_{\Omega}\nabla \mathsf{Q} \cdot \nabla \textsf{Q}^*.\label{Qeqnweakform}
% \end{eqnarray}


%To conclude this section, we turn to a weakly nonlinear analysis of the states that occur when the simple shear state is unstable. We fix the value of the shear rate and look for a steady state of unidirectional flow $\mathbf{v}=v_x(y)\hat{\mathbf{x}}$ for $a=a_\mathrm{c}(\dot{\gamma})+\delta a$, where $\delta a/a_\mathrm{c}$ is a small parameter, with $a_\mathrm{c}=a_{1,\mathrm{c}}$ or $a_{2,\mathrm{c}}$. To simplify the presentation, we assume $\lambda=1$ for the rest of this section. For unidirectional steady flow, the stress $\sigma_{xy}=\eta\partial_y v_x-a Q_{xy}$ is constant and uniform. As we describe below, our numerics indicate that the unidirectional flow states have $\sigma=0$. Thus, using the prime to denote derivatives with respect to $y$, we may use $\eta\ v_x^\prime=a Q_{xy}$ to eliminate the velocity field from the equations for $\mathsf{Q}$, yielding
%\begin{eqnarray}
%\ell^2Q_{xx}^{\prime\prime}-Q_{xx}+(a\tau/\eta) Q_{xy}^2&=&0\label{Qxxsubeq}\\
%\ell^2Q_{xy}^{\prime\prime}-Q_{xy}+(a\tau/\eta) Q_{xy}-(a\tau/\eta) %Q_{xx}Q_{xy}&=&0.
%\end{eqnarray}
%Writing $\delta\mathsf{Q}=\mathsf{Q}-\mathsf{Q}^{(0)}$, eqn~(\ref{Qxxsubeq}) suggests the balance $\delta Q_{xy}$
%\rap{the velocity $v$ here looks like the viscosity introduced in Sec. II. While the primes indicate that it's not a viscosity, can't we use a different notation?}





\section*{Acknowledgements} This work was supported in part by the National Science Foundation through Grant Nos. MRSEC DMR-2011846, CBET-2227361, and  PHY-1748958. We are grateful to Jesse Ault, Kenny Breuer, Guillaume Duclos, Hamid Karani, Jasper Chen, Alexander Morozov, and Pranay Sampat for helpful discussions. We also thank the Center for Computation and Visualization (CCV) at Brown university for use of high performance computing facilities. 


\bibliography{newrefs}
\bibliographystyle{rsc}

%merlin.mbs apsrev4-1.bst 2010-07-25 4.21a (PWD, AO, DPC) hacked
%Control: key (0)
%Control: author (8) initials jnrlst
%Control: editor formatted (1) identically to author
%Control: production of article title (-1) disabled
%Control: page (0) single
%Control: year (1) truncated
%Control: production of eprint (0) enabled


\end{document}









