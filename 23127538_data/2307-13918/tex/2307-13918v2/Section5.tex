\section{Discussion}\label{sec:discussion}

\paragraph{Sensitivity analysis.}
The typical tool for understanding 1D hemodynamics models is variance-based sensitivity analysis~\citep[VBSA,][]{melis2017bayesian, piccioli2022effect,schafer2022uncertainty}.
% Several studies~\citep{melis2017bayesian, piccioli2022effect,schafer2022uncertainty} based on variance-based sensitivity analysis~(VBSA) have made essential contributions to a better understanding of 1D hemodynamics models.
In \cite{melis2017bayesian}, authors perform the analysis on a learnt surrogate of the simulator; \cite{schafer2022uncertainty} relies on polynomial chaos expansion; \cite{piccioli2022effect} studies the effects of various cardiac parameters. 
By contrast, Section~\ref{sec:in-silico} shows that SBI-based uncertainty analysis supports the study of a richer set of uncertainty properties compared to VBSA. For instance, SBI quantifies uncertainty for individual measurements and not only on the population level (highlighted in \figref{fig:npe_vs_laplace}).
% Section~3\ref{sec:in-silico} has shown that SBI-based uncertainty analyses similarly support studying properties of 1D hemodynamics models. 
% Compared to these works, a distinctive attribute of SBI is the ability to quantify uncertainty for individual measurements and not only on the population level (highlighted in \figref{fig:npe_vs_laplace}). 
Furthermore, multi-dimensional VBSA poses computational challenges and is thus not studied in the literature, whereas SBI directly addresses this challenge by learning a joint posterior distribution. Finally, the ambiguity of inverse solutions, which SBI highlights with multi-modal posterior distributions, may invalidate conclusions drawn from uni-dimensional VBSAs.

% our analysis In contrast to  uni-dimensional statistics. 
% We have introduced the SBI methodology to analyse CV simulations, enabling multi-dimensional and individual-level uncertainty analyses. 
% Our results in \figref{fig:npe_vs_laplace} emphasised the need to go beyond these classical sensitivity and uncertainty analyses as neglecting the true complexity of the statistical model may lead to conclusions that are inconsistent with the forward model considered.

\paragraph{ML-based inversion of CV simulators.}
% Another line of work~\citep{chakshu2021towards, jin2021estimating, bikia2021estimation, ipar2021blood, bonnemain2021deep} use ML to inverse CV simulations. 
ML methods can reveal complex relationships between biomarkers and biosignals that are hidden to VBSAs, especially in the presence of salient nuisance effects, as demonstrated in \cite{chakshu2021towards, jin2021estimating, bikia2021estimation, ipar2021blood, bonnemain2021deep}. 
Nevertheless, ML-based sensitivity analyses~(MLBSAs) that do not rely on an effective representation of uncertainty have limitations similar to the ones highlighted for VBSAs in Section~\ref{sec:in-silico}. For instance, they cannot reveal the ambiguity of inverse solutions, as pointed out in \cite{chakshu2021towards}. 
In addition to providing a better understanding of biomarker predictability, MLBSA often aims to address data scarcity, a common challenge for ML in CV applications, with simulated data~\citep{chakshu2021towards, jin2021estimating, bikia2021estimation, ipar2021blood, bonnemain2021deep}. However, as illustrated in Section~\ref{sec:in-vivo}, model misspecification hinders the direct transfer of predictors learned in-silico to real data. Nevertheless, SBI may reveal more general relationships from the simulations (such as dependencies between biomarkers or sub-population identification), which can inform data collection and design of ML algorithms well-suited for the intended CV application.
% In this context, hybrid learning strategies~\citep{takeishi2021physics, yin2021augmenting, wehenkel2022robust}, combining in-silico and in-vivo data, are promising directions that should help deploy such models in production.
% Similarly, SBI leverages ML for revealing sophisticated relationships. Compared to these works, however, SBI offers a more contrasted picture that reveal insights at the level of individual measurements, e.g., ambiguous inverse solutions with multi-modality or dependencies.
% However, offers a more contrasted picture that considers multi-modality, dependencies between parameters and individualised uncertainties. 
% However, similarly to studies relying on VBSAs, these results only quantify identifiability at the population-level and independently for each parameter considered.
% For instance, \cite{bonnemain2021deep} studies left ventricle~(LV) systolic functions in the presence of LV assist device with 0D models; \cite{ipar2021blood} considers various biomarkers related to arterial age; \cite{chakshu2021towards}
% relies on CV simulators to address the challenge of data scarcity for learning an ML-based point predictor of latent biomarkers from biosignals. Similarly to standard sensitivity analyses, these studies inform on the potential identifiability 
% Although NPE is not more than an elaborated ML algorithm, it targets an accurate representation of uncertainty. It aims to help the application of the scientific loop on CV simulation, which contrasts with these ML approaches. While small misspecification, unnoticeable to a human observer, can dramatically hurt generalisation to in-vivo data, insights revealed by uncertainty analyses of the CV model have a better chance to generalise to the real world.
% These insights may provide recommendations for gathering real-world data, potentially supporting the training of ML models that work in the real world.
% In this context, hybrid learning strategies~\citep{takeishi2021physics, yin2021augmenting, wehenkel2022robust}, combining in-silico and in-vivo data, are promising directions that should help deploy such models in production.

\paragraph{Next generation of CV simulators.}
Many research projects build upon the versatility and computational efficiency of full-body 1D hemodynamics, relating biosignals to a growing number of biomarkers~\citep{mejia2022effects, buoso2021personalising, coccarelli2021framework}.
Extensions of these models regularly introduce additional parameters, hence additional sources of uncertainty, which further necessitate a probabilistic perspective on the associated inverse problems.
Another avenue for extending CV models is the joint analysis of biosignals relying on distinct physical processes, e.g., PPG and electrocardiograms. In this context, while combining point estimators consistently is complicated, posterior estimators of each model can be combined easily into a joint posterior distribution. Indeed, assuming a conditional independence  $\mathcal{X} \perp \mathcal{Y} \mid \Phi$ between the two forward models $p(x\mid \phi)$ and $p(y\mid \phi)$ allows rewriting the posterior distribution given the two measurements as the product of each posterior distribution: $p(\phi \mid x, y) \propto p(\phi \mid x) p(\phi \mid y)$.
% \paragraph{SBI and the scientific loop.}
% The main advantage of SBI is its ability to port the rigour and practicality of statistical analyses to complex CV simulators. 
% As Section~4\ref{sec:in-vivo} mentions, statistical inference plays two roles in scientific enquiry. First, it reveals implicit hypotheses that follow from the model, guiding real-world experiments that aim to confirm or reject these implications. Second, analysing the gathered real-world data requires another statistical inference tied to the model considered. Our discussion also highlighted the importance of coping with model misspecification. Contrary to what one might think, the growing complexity of CV models does not especially result in lighter misspecification but can increase it. In this context, SBI is particularly well-positioned to address these challenges as it may directly extend robust Bayesian and frequentist inference strategies~\citep{berger1994overview, huang2023learning, cherief2020mmd} to simulators. SBI should thus provide long-term support to research in CV health.
