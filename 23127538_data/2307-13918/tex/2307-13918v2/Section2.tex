\section{Background on Hemodynamics and SBI}\label{sec:SBI-4-CV}
% \section{Background}\label{sec:SBI-4-CV}

% The increasing complexity of cardiovascular models, particularly whole-body 1D cardiovascular models, accurately captures a wide range of configurations and heterogeneous populations. However, this complexity necessitates a reevaluation of traditional sensitivity analyses, as not all parameters are identifiable or relevant biomarkers. Similar challenges have been addressed in fields like evolutionary biology, particle physics and astrophysics, which have employed statistical inference methods to handle nuisance effects. In particular, SBI, an emerging field of ML, provides tools to perform statistical inference of models defined as simulators. We now provide additional context to the reader unfamiliar with the SBI methodology and unveil details on the 1D CV model considered. 
\subsection{Inverting Whole-body 1D Cardiovascular Simulations}\label{sec:background_CV}

We define a simulator as a forward generative process $g: \Theta \rightarrow \mathcal{X}$ that inputs a vector of parameters $\mathbf{\theta} \in \Theta$ and returns a simulation $\mathbf{x} \in \mathcal{X}$. In scientific contexts, simulators encode complex generative processes influenced by many exogenous factors represented by the parameters $\mathbf{\theta}$. As a consequence, scientific simulators often depend on a large number of parameters and are usually stochastic. In practice, we split the parameters $\mathbf{\theta}=(\phi, \psi) \in\Theta = \Phi \times \Psi$ into variables of direct interest $\phi \in \Phi$ and nuisance parameters $\psi \in \Psi$ that are necessary to run the simulations but are not of direct interest for the downstream task. 
% We distinguish these two subsets of variables and split a parameter into $\theta=(\phi, \psi) \in\Theta = \Phi \times \Psi$.
% : parameters $\theta$ live in the cartesian product of parameters of interest $\phi$ and nuisance parameter $\psi$.

We rely in this work on the simulator from \cite{charlton2019modeling}, which describes the hemodynamics in the $116$ largest human arteries, using the principle of mass and momentum conservation in blood circulation. The model's parameters describe the blood out-flowing the left ventricle, uni-dimensional physical properties of each artery, and a lumped-element model of the vascular beds. Each individual simulation involves solving the corresponding partial differential equations with appropriate numerical schemes~\citep{Melis2017}. In addition, we reduce the gap between simulated and real-world data with a stochastic measurement model. The resulting simulations, shown in \appref{app:sample_generation}, can faithfully describe the heterogeneity of real-world data, as encountered when considering various individuals, biosignals, or measurement scenarios. Section~\ref{sec:model} further motivates and describes the model.

Using this model, we assess the identifiability of the parameters $\phi \in \Phi$ of physiological interest, from a given measurement $\mathbf{x} \in \mathcal{X}$. In the following, we also use the terms biomarkers for physiological parameters of interest and biosignals for measurements.
The biomarkers considered are the heart rate~\citep[HR,][]{kannel1987heart}, the left ventricular ejection time~\citep[LVET,][]{alhakak2021significance}, the average diameter of the arteries~\citep[Diameter,][]{patel2005cardiovascular}, the pulse wave velocity~\citep[PWV,][]{sutton2005elevated}, and the systemic vascular resistance~\citep[SVR,][]{cotter2003role}. These biomarkers are chosen because of their relevance to assess CV health as supported by the provided references~\citep{kannel1987heart, alhakak2021significance, patel2005cardiovascular, sutton2005elevated, cotter2003role}. We consider biosignals that are commonly collected in intensive care units (ICUs) or in medical studies: the arterial pressure waveform (APW) at the radial artery and the photoplethysmograms (PPGs) at the digital and the radial arteries. 

% The effect of nuisance parameters $\psi$ is typically marginalised out~\citep{taper2011evidence}. The measurements considered in this study are signals commonly collected in intensive care units (ICUs): the arterial pressure waveform (APW) at the radial artery and the photoplethysmograms (PPGs) at the digital and the radial arteries. 

We consider a population aged $25$ to $75$ with several free parameters $\theta$ that model heterogeneous cardiac and arterial properties. 
% The population heterogeneity together with the fact that all parameters are considered unknown at inference time, lead to challenging inverse problems. 
In this context, a consistent representation of the solution's uncertainty is key, in order to capture \textbf{1.} the effect of nuisance parameters, responsible for the forward model stochasticity; \textbf{2.} the symmetries of the forward model, leading to non-unique inverse solutions; \textbf{3.} the lack of sensitivity, magnifying small output uncertainty into high input uncertainty; and \textbf{4.} the heterogeneity of the population considered, leading to distinct uncertainty profiles.
% \textcolor{red}{A short paragrapg that says answering these questions requires more than variance-based sensitivity analyses. We must handle properly: 1) the effect of nuisance parameters 2) the fact that we want granularity (not population-level) answers 3) That we need faithful representation of uncertainty (e.g., bi-modality) as these models are complex.}
 % Typically, the literature provides an answer to this question on a population level~\citep{piccioli2022effect, melis2017bayesian, schafer2022uncertainty}. However, a negative answer at the population level is blind to positive answers for a sub-population. We are thus interested in addressing the question of identifiability precisely, per individual measurement and distinguish identifiable sub-populations. 

% Our experiments consider the following set of physiological quantities relevant for the study of cardiovascular health: the heart rate~\citep[HR,][]{kannel1987heart}, the left ventricular ejection time~\citep[LVET,][]{alhakak2021significance}, the average diameter of the arteries~\citep[Diameter,][]{patel2005cardiovascular}, the pulse wave velocity~\citep[PWV,][]{sutton2005elevated}, and the systemic vascular resistance~\citep[SVR,][]{cotter2003role}. We consider the effect of all other simulation parameters, denoted $\psi$, as a nuisance~\citep{taper2011evidence} that we want to marginalise out. The measurements considered in this study are signals commonly collected in intensive care units (ICUs): the APW at the radial artery and the PPGs at the digital and the radial arteries. 


% Statistical inference provides a rigorous framework to study such research questions with CV simulators. These models implicitly define likelihood functions $p(\mathbf{x}\mid \phi)$ from which only sampling is possible. We follow the prior distribution $p(\phi)$ from \cite{charlton2019modeling} that defines a population of individuals aged $25$ to $75$, and leverage SBI algorithms to approximate the posterior $p(\phi \mid \mathbf{x}) \propto p(\phi) p(\mathbf{x} \mid \phi)$. With this surrogate model, we conduct real-time statistical analyses that align with the implicit likelihood of the considered CV simulator.


\subsection{Simulation-based Inference~(SBI)} 
% Over the last years, SBI has become an active subject area in machine learning~\citep{lueckmann2021benchmarking, hermans2021averting, glockler2022variational} with applications in astrophysics~\citep{delaunoy2020lightning, dax2021real, wagner2023images}, particle physics~\citep{brehmer2021simulation}, robotics~\citep{marlier2023simulation}, and many others~\citep{luckmann2022simulation, hashemi2022simulation, tolley2023methods, avecilla2022neural}. 
% With the increasing complexity of CV models, there arises a pressing need for a methodology that can effectively handle nuisance effects and the non-uniqueness of solutions to the inverse problems considered. 
SBI~\citep{cranmer2020frontier} has established itself as an essential tool in various domains of science that rely on complex simulations, e.g., in astrophysics~\citep{delaunoy2020lightning, dax2021real, wagner2023images}, particle physics~\citep{brehmer2021simulation}, neuroscience~\citep{luckmann2022simulation, linhart2022neural},  robotics~\citep{marlier2023simulation}, and many others~\citep{hashemi2022simulation, tolley2023methods, avecilla2022neural}. 
SBI extends statistical inference to statistical models defined implicitly from a simulator, such as the one defined in the previous section. By design, SBI methods aim to provide a consistent representation of uncertainty as demanded by the four requirements listed in the previous paragraph.

While possible, applying statistical inference to simulators is challenged by the absence of a tractable likelihood function. Abstracting all sources of randomness within the nuisance parameters $\psi$, the likelihood is implicitly defined as
$p(\mathbf{x} \mid \phi) \propto \int \mathbb{1}_{\mathbf{x}}\bigl( g\left(\psi, \phi \right) \bigr) p(\psi) d \psi, $ where $\mathbb{1}_{\mathbf{x}}$ is the indicator function for the singleton $\{\mathbf{x}\}$. Considering that only forward simulation is possible, as is the case for the simulator considered in this work, the integral is intractable.

% Given that most simulators only allow sampling from the corresponding distribution $p(\mathbf{x} \mid \phi)$, but not evaluating it, as the latter operation requires nuisance parameters marginalisation, which is intractable in most cases. 

% The success of SBI lies in its ability to to port the rigour of statistical analyses to stochastic simulators by leveraging modern ML methods. 

% The apparent intractability of the likelihood function makes statistical inference over such models challenging. Nevertheless, 
SBI algorithms leverage modern machine learning methods to tackle inference in this likelihood-free setting~\citep{lueckmann2021benchmarking, hermans2021averting, glockler2022variational}. SBI algorithms are broadly categorized as \textit{Bayesian} vs \textit{frequentist} and \textit{amortized} vs \textit{online} methods. In contrast to frequentist methods that only assume a domain for the parameter values, Bayesian methods rely on a prior distribution that encodes a-priori belief and target the posterior distribution, modeling the updated belief on the true parameter value after making an observation. As the prior distribution gets uninformative or the number of observations grows, both methods eventually yield the same inference results, the remaining difference being the interpretation of probabilities as a representation of belief or of intrinsic stochasticity. While online methods focus on a particular instance of the inference problem, amortized methods directly target a surrogate of the likelihood~\citep{vandegar2021neural}, the likelihood ratio~\citep{cranmer2015approximating, hermans2020likelihood}, or the posterior~\citep{lueckmann2017flexible,papamakarios2016fast} that is valid for all possible observations. 

Amortized methods are particularly appealing when repeated inference is required, as new observations can be processed efficiently after an offline simulation and training phase. In this work, we also perform repeated inference over the population of interest and have access to a properly defined prior distribution. Thus, we rely on neural posterior estimation~\citep[NPE,][]{lueckmann2017flexible}, a \textit{Bayesian} and \textit{amortized} method, which learns a surrogate of the posterior distribution $p(\phi \mid \mathbf{x})$ with conditional density estimation. Section~\ref{sec:SBI} details the NPE algorithm and Bayesian uncertainty analysis.

 % Although our experiments focus on 1D simulations, the methodology presented here should benefit the analysis of all categories of cardiovascular models and will provide similar conclusions when models are consistent. Section~4\ref{sec:model} provides additional details on 1D hemodynamics models and motivate it over 0D or 3D models.




% We are interested in finding the best way to analyse whole-body CV simulations. We demonstrate that, SBI, as a mature field of machine learning with impactful applications in physics and many other fields, 
% This section provides a background on the application of simulation-based inference for mining CV simulations and enabling precision medicine.
% In Section~3\ref{sec:in-silico}, we analyse a virtual population of healthy individuals aged $25$ to $75$ from \cite{charlton2019modeling} with SBI~\cite{cranmer2020frontier}. Then, we study the transfer of in-silico results to in-vivo, in Section~3\ref{sec:in-vivo}. 



% Figure environment removed