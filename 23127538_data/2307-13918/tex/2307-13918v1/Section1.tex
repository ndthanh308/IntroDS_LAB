\section{Introduction}
%% CV sims are important and useful
A fine-grained understanding of the human cardiovascular~(CV) system is crucial to mitigating CV diseases. CV models, starting with those of William Harvey~\citep{ribatti2009william}, have seen tremendous progress over the past decades, going from paper calculations~\citep{altschule1938effects, patterson1914regulation} to comprehensive simulators~\citep{updegrove2017simvascular, melis2017gaussian, charlton2019modeling, alastruey2023arterial} that leverage advances in scientific computing. 
Such simulators now provide a personalized description of many aspects of the CV system. They can, for instance, describe cardiac function with 3D models~\citep{baillargeon2014living}, or even simulate hemodynamics in the entire human arterial system~\citep{melis2017gaussian, charlton2019modeling, alastruey2023arterial}. These advances in CV modeling support the development of \textit{personalized} monitoring and treatment of CV diseases, ushering in a new era of precision medicine~\citep{ashley2016towards}.

While whole-body 1D hemodynamics simulators~\citep{melis2017gaussian, charlton2019modeling} establish a clear path from latent physiological variables to measurable biosignals,
% and play an essential role in the study of CV disease~\citep{alastruey2023arterial}.
their use for scientific inquiry, or precision medicine, necessitates solving the corresponding \textit{inverse} problem of inferring latent biomarkers from measurable biosignals. However, this inversion is challenging as the forward model is often specified as a computationally expensive black-box simulator~\citep{manganotti2022modeling}. To add to the complexity, the numerous interactions between parameters lead to convoluted symmetries and ambiguous inverse solutions~\citep{quick2001infinite, nolte2022inverse}. 

Recent works have studied these inverse problems with variance-based sensitivity analysis, highlighting which biomarkers have the most decisive influence on measured biosignals~\cite{melis2017bayesian,schafer2022uncertainty,piccioli2022effect}.
In parallel, machine learning approaches, relying on sophisticated patterns for predicting biomarkers from biosignals, have gained popularity~\citep{chakshu2021towards, jin2021estimating, bikia2021estimation, ipar2021blood,bonnemain2021deep}. While these approaches provide an essential step towards a better understanding of the inverse problem, they do not address the challenges caused by the non-deterministic and multi-modal nature of inverse solutions, as substantiated in Section~3\ref{sec:in-silico}.

%% SBI extends the information that can be extracted from the inverse problem and answer questions previous methods could not
Motivated by breakthroughs in simulation-based inference~\citep[SBI,][]{cranmer2020frontier, tejero2020sbi}, which has addressed similar challenges in other scientific fields, we go beyond producing point-estimates for such inverse problems and consider instead a distributional perspective supported by neural posterior estimation~\citep{lueckmann2017flexible}. 
As a result, the SBI methodology provides a \textit{consistent, multi-dimensional} and, \textit{individualized} representation of uncertainty and naturally handles ambiguous inverse solutions, as showcased in \figref{fig:two_populations}.



% We start this paper with a condensed background (Section~\ref{sec:SBI-4-CV}) that introduces the reader to the CV model considered and SBI. In Section~\ref{sec:results}, we provide an intuitive overview of the various results we can obtain with SBI.  Section~\ref{sec:methods} provides a more detailed technical review of both the CV model and the SBI implementation choices made. Finally, Section~\ref{sec:discussion} concludes with a thorough discussion on the projected impact of SBI to CV modelling.



% Figure environment removed