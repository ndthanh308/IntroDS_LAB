%%%%%%%% ICML 2023 EXAMPLE LATEX SUBMISSION FILE %%%%%%%%%%%%%%%%%

\documentclass{article}

% Recommended, but optional, packages for figures and better typesetting:
\usepackage{microtype}
\usepackage{graphicx}
% \usepackage{subfigure}
\usepackage{booktabs} % for professional tables

% hyperref makes hyperlinks in the resulting PDF.
% If your build breaks (sometimes temporarily if a hyperlink spans a page)
% please comment out the following usepackage line and replace
% \usepackage{icml2023} with \usepackage[nohyperref]{icml2023} above.
\usepackage{hyperref}

% \usepackage{caption}
% \usepackage{xcolor}
% % \bibliographystyle{plainnat}
% \usepackage{wrapfig, blindtext}

% \usepackage{algorithm}
% \usepackage{algcompatible}


\newcommand\figref{Figure~\ref}
\newcommand\tabref{Table~\ref}
\newcommand\appref{Appendix~\ref}
\newcommand\hypref{A\ref}
\newcommand\secref{Section~\ref}
% Attempt to make hyperref and algorithmic work together better:
\newcommand{\theHalgorithm}{\arabic{algorithm}}

% Use the following line for the initial blind version submitted for review:
\usepackage[accepted]{icml2023}

% If accepted, instead use the following line for the camera-ready submission:
% \usepackage[accepted]{icml2023}

% For theorems and such
\usepackage{amsmath}
\usepackage{amssymb}
\usepackage{mathtools}
\usepackage{amsthm}
% \let\subcaption\relax
% \let\subfloat\relax
\usepackage{subcaption}

% if you use cleveref..
\usepackage[capitalize,noabbrev]{cleveref}

%%%%%%%%%%%%%%%%%%%%%%%%%%%%%%%%
% THEOREMS
%%%%%%%%%%%%%%%%%%%%%%%%%%%%%%%%
\theoremstyle{plain}
\newtheorem{theorem}{Theorem}[section]
\newtheorem{proposition}[theorem]{Proposition}
\newtheorem{lemma}[theorem]{Lemma}
\newtheorem{corollary}[theorem]{Corollary}
\theoremstyle{definition}
\newtheorem{definition}[theorem]{Definition}
\newtheorem{assumption}[theorem]{Assumption}
\theoremstyle{remark}
\newtheorem{remark}[theorem]{Remark}

% Todonotes is useful during development; simply uncomment the next line
%    and comment out the line below the next line to turn off comments
%\usepackage[disable,textsize=tiny]{todonotes}
% \usepackage[textsize=tiny]{todonotes}


% The \icmltitle you define below is probably too long as a header.
% Therefore, a short form for the running title is supplied here:
\icmltitlerunning{Simulation-based Inference for Cardiovascular Models}

\begin{document}

\twocolumn[
\icmltitle{Simulation-based Inference for Cardiovascular Models}
% Comments Ozan: 
% is it useful to show this "interesting" behavior but then say the simulation does not reflect real data?
% Can we somehow learn why inference on real data does not work from simulation inference, e.g. multi-modality?
% focuse more on the fact that these models are a standard and used in many areas (and the distinction between 3D and 1D)



% Use letters for affiliations, numbers to show equal authorship (if applicable) and to indicate the corresponding author

% It is OKAY to include author information, even for blind
% submissions: the style file will automatically remove it for you
% unless you've provided the [accepted] option to the icml2023
% package.

% List of affiliations: The first argument should be a (short)
% identifier you will use later to specify author affiliations
% Academic affiliations should list Department, University, City, Region, Country
% Industry affiliations should list Company, City, Region, Country

% You can specify symbols, otherwise they are numbered in order.
% Ideally, you should not use this facility. Affiliations will be numbered
% in order of appearance and this is the preferred way.
\icmlsetsymbol{equal}{*}

\begin{icmlauthorlist}
\icmlauthor{Antoine Wehenkel}{comp}
\icmlauthor{Jens Behrmann}{comp}
\icmlauthor{Andrew C.~Miller} {comp}
\icmlauthor{Guillermo Sapiro} {comp}
\icmlauthor{Ozan Sener}{comp}
\icmlauthor{Marco Cuturi} {comp}
\icmlauthor{Jörn-Henrik Jacobsen} {comp}
%\icmlauthor{}{sch}
%\icmlauthor{}{sch}
\end{icmlauthorlist}

\icmlaffiliation{comp}{Apple}

\icmlcorrespondingauthor{Antoine Wehenkel}{awehenkel@apple.com}

% You may provide any keywords that you
% find helpful for describing your paper; these are used to populate
% the "keywords" metadata in the PDF but will not be shown in the document
\icmlkeywords{Bayesian Inference, Cardiovascular Simulations, Hemodynamics, Machine Learning, Simulation-based Inference}

\vskip 0.3in
]

% this must go after the closing bracket ] following \twocolumn[ ...

% This command actually creates the footnote in the first column
% listing the affiliations and the copyright notice.
% The command takes one argument, which is text to display at the start of the footnote.
% The \icmlEqualContribution command is standard text for equal contribution.
% Remove it (just {}) if you do not need this facility.

%\printAffiliationsAndNotice{}  % leave blank if no need to mention equal contribution
\printAffiliationsAndNotice{\icmlEqualContribution} % otherwise use the standard text.

\begin{abstract}
Over the past decades, hemodynamics simulators have steadily evolved and have become tools of choice for studying cardiovascular systems in-silico. 
While such tools are routinely used to simulate whole-body hemodynamics from physiological parameters, solving the corresponding inverse problem  of mapping waveforms back to plausible physiological parameters remains both promising and challenging.
Motivated by advances in simulation-based inference (SBI), we cast this inverse problem as statistical inference.
In contrast to alternative approaches, SBI provides \textit{posterior distributions} for the parameters of interest, providing a \textit{multi-dimensional} representation of uncertainty for \textit{individual} measurements. 
We showcase this ability by performing an in-silico uncertainty analysis of five biomarkers of clinical interest comparing several measurement modalities. Beyond the corroboration of known facts, such as the feasibility of estimating heart rate, our study highlights the potential of estimating new biomarkers from standard-of-care measurements. SBI reveals practically relevant findings that cannot be captured by standard sensitivity analyses, such as the existence of sub-populations for which parameter estimation exhibits distinct uncertainty regimes. Finally, we study the gap between in-vivo and in-silico with the MIMIC-III waveform database and critically discuss how cardiovascular simulations can inform real-world data analysis. 
\end{abstract}


 %!TEX root = jsac_v6.tex
\section{Introduction} 
Communication theorists are always on the lookout for new technologies to improve the speed and reliability of wireless communications. Chief among the technologies that blossomed into major advances is the multiple antenna technology, whose latest implementation is Massive MIMO (multiple-input multiple-output)~\cite{marzetta2010noncooperative, massivemimobook}. Inspired by its potential benefits~\cite{sanguinettiTCOM2020}, new research directions are taking place under different names~\cite{BJORNSON20193}, e.g., Holographic MIMO~\cite{Huang2020} and large intelligent surfaces~\cite{Rusek2018}. Particularly, the former concept refers to an array (possibly \emph{electromagnetically large}, i.e., compared to the wavelength) with a massive number of closely spaced antennas whose electromagnetic interactions inevitably results into mutual coupling~\cite{balanis}. Although few exceptions exist, e.g., \cite{7831497,6843218,8350292,9048753,9838533}, the vast majority of the MIMO literature has entirely neglected mutual coupling since it is all about using (possibly \emph{physically large}) arrays with \emph{half-wavelength} antenna spacing~\cite{massivemimobook}. Another major caveat of the classical MIMO literature (in general) is that it mostly relies on the abstractions of signal processing and information theories, which are not always consistent with the physical context of the underlying system.  
Fortunately, there exists a thin, but solid, literature that can be used to overcome these limitations~\cite{Janaswamy2002,Svantesson_ICASSP2001,Wallace2004,Nossek2010,Nossek2014} but its development has been relatively slow due to the less tractable analysis.
% Unfortunately, its important insights and messages are easily missed since they are a minor fraction compared to the vast literature that neglects it. 
% The primary objectives of this paper are two fold: $i$) to help (among others) in bringing to the attention of MIMO communications theorists such line of research; and $ii$) to use it for understanding
% the effects of having closely spaced antennas in the uplink and downlink of multi-user
% Holographic MIMO communications. To the best of authors knowledge, a comprehensive treatment in this direction is currently missing in the relevant literature.

% \subsection{Motivation and relevant literature} 
 The first attempts in this direction can be found in~\cite{Janaswamy2002,Svantesson_ICASSP2001,Wallace2004}. Particularly, in~\cite{Wallace2004} the authors derived the model of a single-user MIMO communication system as an electrical network described by scattering matrices. This allows to account for the mutual coupling between transmit and/or receive antennas. A matching network was also introduced at the receiver to maximize the power transfer from the source to the loads. The framework developed in~\cite{Wallace2004} is also among the first to connect the physical power to the abstract concept used in signal and information theories. An alternative framework is developed in~\cite{Nossek2010,Nossek2014} based on the multiport communication theory. This involves a circuit theoretic approach where the inputs and outputs of the multiple antenna communication system are associated with ports of a multi-port black-box, described by impedance matrices. Notice that the two frameworks above are almost equivalent (except for constructed special cases) and the multiport communication theory has been used in the MIMO literature to study several aspects. 
 %The authors in \cite{Wallace2004} found the use of scattering matrices more convenient for capacity computation. \textcolor{red}
 {For example, in~\cite{yordanov2009arrays,Ivrlac2009ICC,Laas2020} the transmit/receive array gain is evaluated (with and without matching networks) for uniform linear and circular arrays. The diversity gain is investigated in \cite{ivrlavc2011diversity}, while the effects of the antenna separation on the mutual information of two Hertzian dipoles are analyzed in~\cite{Nossek2014}. The multiport communication theory is also used in~\cite{Laas2020_Reciprocity} for studying the uplink/downlink reciprocity and mutual information of multi-user MIMO systems. More recently, \cite{Bamelak_2023} used it to investigate the impact of mutual coupling in the channel estimation of single-user MIMO communications.}    

 The main objectives of this paper are two fold: $i$) to use the multiport communication theory to derive physically consistent uplink and downlink models for multi-user Holographic MIMO communications with linear processing; and $ii$) to use the developed models to answer the following question: \textit{what are the spectral efficiency advantages (if any) of having closely spaced antennas?} To answer this question, we first consider a simple uplink scenario with two side-by-side half-wavelength dipoles at the base station (BS), two user equipments (UEs) and single path line-of-sight propagation. In this context, we show
both analytically and numerically that the channel gain, interference gain and spectral efficiency depend strongly on
the directions from which the UE signals are received and on the array matching network used at the BS. Advantages can be obtained only with impedance matching (e.g., \cite{Volodymyr_2022}) and under certain conditions, which may not be met in practical
systems. In these cases, the gains may be marginal or even non-existent.
The internal losses within the dipole antennas are also shown to significantly impact the
spectral efficiency as the spacing reduces. Numerical results are then used to show that similar conclusions hold true in more practical scenarios with an arbitrary number of UEs and an arbitrary number of dipole antennas at the BS. Particularly, the analysis is conducted in the following two cases: $i$) the number of dipoles is fixed as we vary their spacing; $i$) the array size is fixed as we vary the dipole spacing. In the latter case, it turns out that the spectral efficiency increases s the antenna distance reduces. However, this comes from the larger energy that is collected by the larger number of dipoles, not from the mutual coupling. Interestingly, the spectral efficiency tends to increase less and less as the size of the antenna array increases (compared to the wavelength). 

Although most of the analysis focuses on the uplink, we also investigate the downlink. Particular attention is given to the uplink and downlink duality in the presence of different matching networks. {Specifically, we show that the downlink and uplink channels are reciprocal up to a linear transformation. In line with \cite{Laas2020_Reciprocity}, the ordinary channel reciprocity (i.e., no linear transformation) holds true only if full matching networks (that are hard to implement in arrays with many antennas) are employed at both sides.} Numerical results are used to quantify the spectral efficiency loss when the linear transformation is not applied.     

 


% The first studies on the effect of mutual coupling on the capacity of single-user MIMO systems are dated back to the beginning of  2000s~\cite{Janaswamy2002,Svantesson_ICASSP2001,Wallace2004}. Particularly,~\cite{Wallace2004} represented the MIMO communication system as an electrical network described by scattering matrices that account for mutual coupling between transmit and/or receive antennas. A matching network was also introduced at the receiver to maximize the power transfer from the source to the loads. The framework developed in~\cite{Wallace2004} is also among the first to connect abstract mathematical parameters (such as transmit power) with physical quantities. %Indeed, the analysis and design of MIMO systems in the communications society have historically evolved around the abstractions of information theory. The latter serves well as the mathematical theory of communication but it contains no provision that makes sure its abstractions are consistent with the physical laws that govern any system.
% % As a matter of fact, there was a gap between the mathematical models adopted by the communication and information theorists and the physics of electromagnetic propagation (of which mutual coupling is only one aspect). 
% To better understand this point, consider a narrowband communication system equipped with $M$ antennas at the receiver and $N$ antennas at the source. This is commonly described by the following information-theoretic discrete-time input-output relation:%~\cite{TseBook}:
% \begin{equation}\label{eq:MIMO_channel}
% {\bf y} = {\bf H}{\bf x} + {\bf n}
% \end{equation} 
% where ${\bf y}\in \mathbb{C}^{M}$ and ${\bf x}\in \mathbb{C}^{N}$ denote the received and transmitted signal vectors, respectively. The vector ${\bf x}$ must satisfy $\mathbb{E}\{{{\bf x}^{\Htran}{\bf x}}\}\le P_{\rm T}$ to constrain the total transmit power. Also, ${\bf n}\sim \mathcal{N}_\mathbb{C}({\bf 0}, {\bf R}_n)$ is the additive Gaussian noise and ${\bf H}\in \mathbb{C}^{M \times N}$ is the MIMO channel matrix. The input-output relationship given in \eqref{eq:MIMO_channel} provides a \textit{mathematical} model in which the various building blocks are not necessarily related to \textit{physical} quantities and phenomena (among which mutual coupling is one of them). This is the point where the multiport communication theory developed by Ivrla\v{c} and Nossek comes into the play~\cite{Nossek2010,Nossek2014}. This framework involves a circuit theoretic approach where the inputs and outputs of the communication system are associated with ports of a multi- port black-box. It establishes an interface between the physical and information worlds, and, thus, it ensures the applicability of mathematical theories of communication to the physical world.


% They are pure numbers and do not obey any physical law. On the other hand, transmission of information is always connected with some physical process, and hence the relevant equations must be consistent with the fundamental principles of physics, such as, for example, the law of conservation of energy. The Multiport Communication Theory (MCT), developed by Ivrla\v{c} and Nossek~\cite{Nossek2010,Nossek2014}, provides a description of the communication link in terms of voltages and currents which are physical quantities related to the transmission process. Since the publication of \cite{Nossek2010}, a lot of papers have been devoted...      

% \subsection{Contribution}
% The primary objective of this paper is that of analyzing the spectral efficiency of multi-user holographic MIMO communications, characterized by a large number of closely spaced antennas at the base station (BS). We limit our study to uniform linear arrays (ULAs) of half-wavelength dipoles, but the provided framework can be used to extend the results to different array configurations and different types of antennas. Both the downlink and uplink directions are analyzed, and we assume that linear precoding (in the downlink) or linear combining (in the uplink) techniques are used at the BS. In particular, we focus on \textit{maximum ratio} (MR) and \textit{minimum mean square error} (MMSE) schemes, for both transmission and reception. 

% The analysis is conducted on the basis of the Multiport Communication Theory (MCT) developed by Ivrla\v{c} and Nossek~\cite{Nossek2014,Nossek2010}. For convenience, the main results of MCT are discussed in the first part of this paper. A key ingredient of MCT is represented by the impedance matrix that accounts for the mutual coupling between the transmit and/or receive antennas (\textit{intra-array} and \textit{inter-array coupling}). The computation of this matrix derives from electromagnetic principles, which are discussed in a dedicated section where it is studied the impact of the antenna spacing on the spatial correlation properties of signal and noise.

% The important question we aim to answer is the following: \textit{in a multi-user holographic MIMO  communication system, what are the effects of the mutual coupling on the spectral efficiency? In particular, what happens when the antennas are packed close to each other?} In order to answer this question we start with a case study with two single-antenna users and a BS with two antennas, and we provide analytical expressions for the array gain and the interference gain, which help understanding the impact of the mutual coupling and the antenna spacing on the spectral efficiency. Most of the conclusions that can be drawn in this simplified scenario are maintained when we consider a more practical situation with an arbitrary number of users and an arbitrary number of antennas at the BS. In particular, in the general scenario the effects of varying the antenna spacing (in conjunction with other system parameters, such as the transmit/receive matching networks) have been analyzed in two cases of significant interest: in one case, we consider a fixed number of antennas and we vary their distances; in the other, we fix the array size and we vary the number of antennas, reducing or increasing the spacing. It turns out that the behavior of the spectral efficiency is quite different in the two cases. \textcolor{red}{Forse si può anticipare qualche conclusione}

% Though most of the paper is devoted to the study of the system performance in the uplink direction, we also investigate the duality between the uplink and downlink channels. It turns out that such a duality exists depending on the particular matching networks employed at the transmit and receive sides. In this respect, we analyze (by simulation) the impact of the antenna spacing and the mutual coupling on the downlink spectral efficiency, when it is erroneously assumed that the downlink and the uplink vector channels coincide (within a scale factor).     


% The primary objective of this paper is that of giving an answer to the following question: in an holographic MIMO communication system, characterized by a large number of antennas (possibly close to each other), what are the effects of the antenna spacing on the system performance? In particular, does mutual coupling have a positive impact on the communication process? In order to answer this question we consider a practical scenario in which single-antenna users communicate with a base station (BS) endowed with multiple antennas. We limit our study to uniform linear arrays (ULAs) of half-wavelength dipoles, but the provided framework can be used to extend the results to different array configurations and different types of antennas. Both the downlink and uplink directions are analyzed, and we assume that linear precoding (in the downlink) or linear combining (in the uplink) techniques are used at the BS. In particular, we focus on \textit{maximum ratio} (MR) and \textit{minimum mean square error} (MMSE) schemes, for both transmission and reception. The system performance is assessed in terms of spectral efficiency (SE). 



% \subsection{Paper outline and reproducible research}
The remainder of this paper is organized as follows. In Section~\ref{sec:system_model}, we review the Multiport Communication Theory from~\cite{Nossek2014}. In Section~III, we show how to compute the mutual coupling impedance matrix when a uniform linear array made of half-wavelength dipoles is used at both sides. In Section IV, the uplink and downlink signal models for Holographic MIMO communications are derived on the basis of the
multiport communication model. The concept of uplink and downlink duality is also discussed. To showcase what is the impact of mutual coupling, a simple case study with two dipole antennas and two UEs is considered in Section V. The analysis is then extended in Section VI to more realistic scenarios with multiple antennas, multiple UEs and arrays of varying or fixed aperture. Conclusions are drawn in Section VI. 

\textit{Reproducible research:} The Matlab code used to obtain the simulation results will be made available upon completion of the review process.

% We use $\rank({\bf A})$ to denote the rank of matrix $\bf {A}$. For a given vector ${\bf p}$, $\hat {\bf p}$ is a unit vector along its direction and $||{\bf p}||$ denotes its magnitude. $\nabla \times$ denotes the curl operation, $\delta(x)$ the Dirac delta function, and $\floor*{x}$ the greatest integer less than or equal to $x$. $\delta(\cdot)$ denotes the Dirac delta function.


\section{Background on Hemodynamics and SBI}\label{sec:SBI-4-CV}
% \section{Background}\label{sec:SBI-4-CV}

% The increasing complexity of cardiovascular models, particularly whole-body 1D cardiovascular models, accurately captures a wide range of configurations and heterogeneous populations. However, this complexity necessitates a reevaluation of traditional sensitivity analyses, as not all parameters are identifiable or relevant biomarkers. Similar challenges have been addressed in fields like evolutionary biology, particle physics and astrophysics, which have employed statistical inference methods to handle nuisance effects. In particular, SBI, an emerging field of ML, provides tools to perform statistical inference of models defined as simulators. We now provide additional context to the reader unfamiliar with the SBI methodology and unveil details on the 1D CV model considered. 
\subsection{Inverting Whole-body 1D Cardiovascular Simulations}\label{sec:background_CV}

We define a simulator as a forward generative process $g: \Theta \rightarrow \mathcal{X}$ that inputs a vector of parameters $\mathbf{\theta} \in \Theta$ and returns a simulation $\mathbf{x} \in \mathcal{X}$. In scientific contexts, simulators encode complex generative processes influenced by many exogenous factors represented by the parameters $\mathbf{\theta}$. As a consequence, scientific simulators often depend on a large number of parameters and are usually stochastic. In practice, we split the parameters $\mathbf{\theta}=(\phi, \psi) \in\Theta = \Phi \times \Psi$ into variables of direct interest $\phi \in \Phi$ and nuisance parameters $\psi \in \Psi$ that are necessary to run the simulations but are not of direct interest for the downstream task. 
% We distinguish these two subsets of variables and split a parameter into $\theta=(\phi, \psi) \in\Theta = \Phi \times \Psi$.
% : parameters $\theta$ live in the cartesian product of parameters of interest $\phi$ and nuisance parameter $\psi$.

We rely in this work on the simulator from \cite{charlton2019modeling}, which describes the hemodynamics in the $116$ largest human arteries, using the principle of mass and momentum conservation in blood circulation. The model's parameters describe the blood out-flowing the left ventricle, uni-dimensional physical properties of each artery, and a lumped-element model of the vascular beds. Each individual simulation involves solving the corresponding partial differential equations with appropriate numerical schemes~\citep{Melis2017}. In addition, we reduce the gap between simulated and real-world data with a stochastic measurement model. The resulting simulations, shown in \appref{app:sample_generation}, can faithfully describe the heterogeneity of real-world data, as encountered when considering various individuals, biosignals, or measurement scenarios. Section~\ref{sec:model} further motivates and describes the model.

Using this model, we assess the identifiability of the parameters $\phi \in \Phi$ of physiological interest, from a given measurement $\mathbf{x} \in \mathcal{X}$. In the following, we also use the terms biomarkers for physiological parameters of interest and biosignals for measurements.
The biomarkers considered are the heart rate~\citep[HR,][]{kannel1987heart}, the left ventricular ejection time~\citep[LVET,][]{alhakak2021significance}, the average diameter of the arteries~\citep[Diameter,][]{patel2005cardiovascular}, the pulse wave velocity~\citep[PWV,][]{sutton2005elevated}, and the systemic vascular resistance~\citep[SVR,][]{cotter2003role}. These biomarkers are chosen because of their relevance to assess CV health as supported by the provided references~\citep{kannel1987heart, alhakak2021significance, patel2005cardiovascular, sutton2005elevated, cotter2003role}. We consider biosignals that are commonly collected in intensive care units (ICUs) or in medical studies: the arterial pressure waveform (APW) at the radial artery and the photoplethysmograms (PPGs) at the digital and the radial arteries. 

% The effect of nuisance parameters $\psi$ is typically marginalised out~\citep{taper2011evidence}. The measurements considered in this study are signals commonly collected in intensive care units (ICUs): the arterial pressure waveform (APW) at the radial artery and the photoplethysmograms (PPGs) at the digital and the radial arteries. 

We consider a population aged $25$ to $75$ with several free parameters $\theta$ that model heterogeneous cardiac and arterial properties. 
% The population heterogeneity together with the fact that all parameters are considered unknown at inference time, lead to challenging inverse problems. 
In this context, a consistent representation of the solution's uncertainty is key, in order to capture \textbf{1.} the effect of nuisance parameters, responsible for the forward model stochasticity; \textbf{2.} the symmetries of the forward model, leading to non-unique inverse solutions; \textbf{3.} the lack of sensitivity, magnifying small output uncertainty into high input uncertainty; and \textbf{4.} the heterogeneity of the population considered, leading to distinct uncertainty profiles.
% \textcolor{red}{A short paragrapg that says answering these questions requires more than variance-based sensitivity analyses. We must handle properly: 1) the effect of nuisance parameters 2) the fact that we want granularity (not population-level) answers 3) That we need faithful representation of uncertainty (e.g., bi-modality) as these models are complex.}
 % Typically, the literature provides an answer to this question on a population level~\citep{piccioli2022effect, melis2017bayesian, schafer2022uncertainty}. However, a negative answer at the population level is blind to positive answers for a sub-population. We are thus interested in addressing the question of identifiability precisely, per individual measurement and distinguish identifiable sub-populations. 

% Our experiments consider the following set of physiological quantities relevant for the study of cardiovascular health: the heart rate~\citep[HR,][]{kannel1987heart}, the left ventricular ejection time~\citep[LVET,][]{alhakak2021significance}, the average diameter of the arteries~\citep[Diameter,][]{patel2005cardiovascular}, the pulse wave velocity~\citep[PWV,][]{sutton2005elevated}, and the systemic vascular resistance~\citep[SVR,][]{cotter2003role}. We consider the effect of all other simulation parameters, denoted $\psi$, as a nuisance~\citep{taper2011evidence} that we want to marginalise out. The measurements considered in this study are signals commonly collected in intensive care units (ICUs): the APW at the radial artery and the PPGs at the digital and the radial arteries. 


% Statistical inference provides a rigorous framework to study such research questions with CV simulators. These models implicitly define likelihood functions $p(\mathbf{x}\mid \phi)$ from which only sampling is possible. We follow the prior distribution $p(\phi)$ from \cite{charlton2019modeling} that defines a population of individuals aged $25$ to $75$, and leverage SBI algorithms to approximate the posterior $p(\phi \mid \mathbf{x}) \propto p(\phi) p(\mathbf{x} \mid \phi)$. With this surrogate model, we conduct real-time statistical analyses that align with the implicit likelihood of the considered CV simulator.


\subsection{Simulation-based Inference~(SBI)} 
% Over the last years, SBI has become an active subject area in machine learning~\citep{lueckmann2021benchmarking, hermans2021averting, glockler2022variational} with applications in astrophysics~\citep{delaunoy2020lightning, dax2021real, wagner2023images}, particle physics~\citep{brehmer2021simulation}, robotics~\citep{marlier2023simulation}, and many others~\citep{luckmann2022simulation, hashemi2022simulation, tolley2023methods, avecilla2022neural}. 
% With the increasing complexity of CV models, there arises a pressing need for a methodology that can effectively handle nuisance effects and the non-uniqueness of solutions to the inverse problems considered. 
SBI~\citep{cranmer2020frontier} has established itself as an essential tool in various domains of science that rely on complex simulations, e.g., in astrophysics~\citep{delaunoy2020lightning, dax2021real, wagner2023images}, particle physics~\citep{brehmer2021simulation}, neuroscience~\citep{luckmann2022simulation, linhart2022neural},  robotics~\citep{marlier2023simulation}, and many others~\citep{hashemi2022simulation, tolley2023methods, avecilla2022neural}. 
SBI extends statistical inference to statistical models defined implicitly from a simulator, such as the one defined in the previous section. By design, SBI methods aim to provide a consistent representation of uncertainty as demanded by the four requirements listed in the previous paragraph.

While possible, applying statistical inference to simulators is challenged by the absence of a tractable likelihood function. Abstracting all sources of randomness within the nuisance parameters $\psi$, the likelihood is implicitly defined as
$p(\mathbf{x} \mid \phi) \propto \int \mathbb{1}_{\mathbf{x}}\bigl( g\left(\psi, \phi \right) \bigr) p(\psi) d \psi, $ where $\mathbb{1}_{\mathbf{x}}$ is the indicator function for the singleton $\{\mathbf{x}\}$. Considering that only forward simulation is possible, as is the case for the simulator considered in this work, the integral is intractable.

% Given that most simulators only allow sampling from the corresponding distribution $p(\mathbf{x} \mid \phi)$, but not evaluating it, as the latter operation requires nuisance parameters marginalisation, which is intractable in most cases. 

% The success of SBI lies in its ability to to port the rigour of statistical analyses to stochastic simulators by leveraging modern ML methods. 

% The apparent intractability of the likelihood function makes statistical inference over such models challenging. Nevertheless, 
SBI algorithms leverage modern machine learning methods to tackle inference in this likelihood-free setting~\citep{lueckmann2021benchmarking, hermans2021averting, glockler2022variational}. SBI algorithms are broadly categorized as \textit{Bayesian} vs \textit{frequentist} and \textit{amortized} vs \textit{online} methods. In contrast to frequentist methods that only assume a domain for the parameter values, Bayesian methods rely on a prior distribution that encodes a-priori belief and target the posterior distribution, modeling the updated belief on the true parameter value after making an observation. As the prior distribution gets uninformative or the number of observations grows, both methods eventually yield the same inference results, the remaining difference being the interpretation of probabilities as a representation of belief or of intrinsic stochasticity. While online methods focus on a particular instance of the inference problem, amortized methods directly target a surrogate of the likelihood~\citep{vandegar2021neural}, the likelihood ratio~\citep{cranmer2015approximating, hermans2020likelihood}, or the posterior~\citep{lueckmann2017flexible,papamakarios2016fast} that is valid for all possible observations. 

Amortized methods are particularly appealing when repeated inference is required, as new observations can be processed efficiently after an offline simulation and training phase. In this work, we also perform repeated inference over the population of interest and have access to a properly defined prior distribution. Thus, we rely on neural posterior estimation~\citep[NPE,][]{lueckmann2017flexible}, a \textit{Bayesian} and \textit{amortized} method, which learns a surrogate of the posterior distribution $p(\phi \mid \mathbf{x})$ with conditional density estimation. Section~\ref{sec:SBI} details the NPE algorithm and Bayesian uncertainty analysis.

 % Although our experiments focus on 1D simulations, the methodology presented here should benefit the analysis of all categories of cardiovascular models and will provide similar conclusions when models are consistent. Section~4\ref{sec:model} provides additional details on 1D hemodynamics models and motivate it over 0D or 3D models.




% We are interested in finding the best way to analyse whole-body CV simulations. We demonstrate that, SBI, as a mature field of machine learning with impactful applications in physics and many other fields, 
% This section provides a background on the application of simulation-based inference for mining CV simulations and enabling precision medicine.
% In Section~3\ref{sec:in-silico}, we analyse a virtual population of healthy individuals aged $25$ to $75$ from \cite{charlton2019modeling} with SBI~\cite{cranmer2020frontier}. Then, we study the transfer of in-silico results to in-vivo, in Section~3\ref{sec:in-vivo}. 



% Figure environment removed

\section{Results}\label{sec:results}
% In this section, we analyse the posterior distributions obtained with SBI. We rely on the neural posterior estimation~\citep[NPE,][]{papamakarios2016fast, lueckmann2017flexible} algorithm to learn a normalizing-flow (NF) based conditional density estimator for each measurement of interest. Compared to other SBI algorithms, NPE is straightforward to train and directly enables evaluating and sampling from the distribution $p(\phi \mid \mathbf{x})$.  
Our experiments consider both in-silico and in-vivo scenarios.  
We split the simulation dataset from \cite{charlton2019modeling} into train ($70\%$), validation ($10\%$), and test sets ($20\%$) at random. All results reported are on the test set for the NPE models that maximize the validation likelihood, error bars report the standard deviation over five training instances. The dataset considered assumed various dependencies between age and most parameters. We prevent age to confound our analysis by explicitly conditioning both posterior and prior distributions on age, and averaging out age from our results. 
% Section~\ref{sec:methods} details further the NPE algorithm and the test metrics.

\subsection{In-silico analysis}\label{sec:in-silico}
% exemplified by successful inference of heart rate without the need of any real training data. However, there remains a non-negligible gap with real data which needs to be closed to allow accurate estimation of the remaining parameters of interest.
\paragraph{SBI enables comprehensive population-level uncertainty analyses.}

Figure~\ref{fig:identifiability_analysis} shows the average size of credible intervals extracted from various posterior distributions of the parameters of interest. With these results, we can compare the identifiability of several parameters given various measurement modalities and levels of noise. We can even inspect the information content of multiple measurement modalities at the same time, as shown in orange, for the Digital PPG and Radial APW. 
% A few examples of the simulated waveforms are provided in \appref{app:sample_generation}. 
To study the robustness of inverse solutions, we consider an additive Gaussian noise model with five noise levels. As a prerequisite for a meaningful analysis, we need to consider inference that is well statistically calibrated, which is the case here as observed in \appref{app:calib_mae}. 
% By construction, for a given measurement modality and noise level, the size of credible intervals~(SCI) with a credibility level equal to $95\%$ are larger than for $68\%$. 
Assuming consistent calibrations, observing the size of intervals as a function of SNR and comparing them to the intervals of the prior distribution quantifies how much information a measurement carries about the biomarkers, as discussed in \appref{app:MI_identifiability}. Section~\ref{sec:metrics} further motivates SCI as an effective metric to study the identifiability of parameters from posterior distributions.

% Overall, there is no unique best measurement. 
Unsurprisingly, the HR is easily identified from all measurements, except for very high levels of noise.
% As the observations are 8-second waveforms it is no surprise that. 
Overall, uncertainty about all parameters reduces significantly as the noise level decreases. This observation indicates that the measurements carry information about all parameters considered which is consistent with the findings of other studies~\citep{melis2017bayesian, charlton2019modeling, charlton2022assessing}. The results also highlight that each measurement has its unique information content. For instance, the digital PPG reveals more about SVR and PWV than the radial PPG. However, it is the opposite for the Diameter for which the Radial PPG is the most informative measurement. 

These results highlight that, similarly to standard sensitivity analyses, SBI enables interpretable assessment of the predictability of biomarkers from biosignals, in-silico, while having additional properties exemplified in subsequent experiments.
% In contrast to traditional sensitivity analyses, SBI directly provide
% and what they indicate about the ability to predict a (set of) parameter(s) given a measurement.

% Our analysis relies on a surrogate and not the true posterior distributions. Thus, some observations can be inconsistent. Theoretically, the size of credible intervals should decrease monotonically as a function of SNR and be upper bounded by the prior. Both aspects are sometimes violated in our results, e.g., in \figref{fig:identifiability_analysis} for HR at high noise levels and for the inference of SVR given the radial APW. Although improving this consistency is possible, e.g. by increasing the size of the training set or improving the conditional density estimators' inductive bias, this is not essential to obtain valuable insights from the simulators. For instance, let us assume the studied surrogate models are calibrated -- they usually are, as shown in \appref{app:calibration}. The measurement modality that provides the tightest credible intervals is the one that allows for the easiest extraction of information about the parameter, based on easily identifiable patterns for the considered class of models rather than numerical artefacts. This suggests that if similar patterns exist in real-world data and provide information about the parameter, a similar phenomenon can be expected, which would be aligned with conclusions drawn in-silico. Furthermore, selecting models on validation performance enforces that the posterior model relies on statistics that generalise to the validation set, rather than intricate numerical artefacts of the simulation. Analyses based on such statistics are also more likely to transfer to real-world data.

% Figure environment removed

\paragraph{SBI enables per-individual uncertainty quantification.}
\figref{fig:npe_vs_laplace} compares the estimation of uncertainty provided by NPE and Laplace's approximation~\citep{MacKay2003Information} around the expectation of the posterior distribution, which is representative of the underlying assumptions made in variance-based sensitivity analyses~(VBSAs). Similarly to VBSAs, Laplace's approximation models uncertainty through a second-order statistic over the population considered.
% \figref{fig:npe_vs_laplace} compares the estimation of uncertainty provided by the posterior distribution obtained with NPE and the uncertainty extracted by Laplace's approximation around the expectation of the posterior distribution. 
Compared to Laplace's approximation, NPE yields tighter and better calibrated credibility intervals. Laplace's intervals tend to be overconfident for measurements that lead to multi-modal posterior distributions, and they are underconfident when the posterior is uni-modal. 
% In comparison, NPE is better calibrated. 
% for all cases although not perfectly, which is expected as NPE is an approximation of the true posterior. 
Furthermore, a point estimator, even with Laplace's uncertainty estimation, will likely assign high density to low-density areas for certain observations, especially if the true posterior is multi-modal. Such inconsistent quantification of uncertainty may mislead downstream decisions. \appref{app:add_exp} showcases the $5$D posterior distributions corresponding to two test examples, which exhibit distinct uncertainty profiles and support further the necessity to use an expressive and observation-dependent quantification of uncertainty.

\figref{fig:two_populations} sketches the use of SBI to study the relationship between the digital PPG and the SVR and LVET. The figure highlights distinctive aspects of posterior distributions within the population studied for which we tested multi-modality~\citep{hartigan1985dip}. While the uncertainty about the value of SVR and LVET can be reduced substantially for approximately half of the test population, for the other half, the posterior is multi-modal. In addition, there remains a strong dependence between the two parameters for this second half. In practice, while a point estimator would be reasonable for the first sub-population, it would be a poor guess for the multi-modal sub-population. Multi-modality indicates that only specific sub-regions of the parameter space are credible, which may suffice to inform certain downstream tasks, e.g., detecting high risk zones, which does not necessarily require knowing the parameter's value exactly. The possibility to perform such fine-grained analysis is only possible with an accurate quantification of uncertainty at the individual level, as offered by SBI.

These results demonstrate that a consistent, multi-dimensional, and individualized representation of uncertainty, as obtained with NPE, yields essential insights from the hemodynamics simulator that are left unnoticed by VBSAs. Multi-modality and dependencies, as observed in \figref{fig:two_populations}, highlight the presence of symmetries in the forward model and individualized uncertainty profiles enable us to stratify the population based on this criterion.


\subsection{In-vivo analysis}\label{sec:in-vivo}
Models are never a perfect representation of real-world data~\citep{box1976science}. Misspecification, as it becomes more significant, hampers the practical relevance of insights extracted from a model~\citep {white1982maximum}. Thus, it is necessary to understand model misspecification to reason about the real world confidently~\citep{geweke2012prediction, box1976science}. Overcoming misspecification is most effectively achieved by identifying conclusions that are independent of the most critical sources of misspecification rather than aiming for perfect models, which do not exist. For instance, in this work, the simulated waveforms strongly depend on the shape of the boundary inflow condition at the aorta, which is approximated with a simplistic and idealized five-parameter descriptions, misspecified for most practical cases. Nevertheless, this description represents accurately the relationship between the length of a beat and the HR. Thus, insights that rely mainly on the beat length of the inflow waveform generalize well to real-world data. However, considering more complex aspects of the model, which reveal more unexpected and intricate relationships between parameters and simulated observations, increases misspecification and identifying definitive conclusions becomes challenging. We further discuss the problem of misspecification in Bayesian inference in Section~\ref{sec:misspecification_Bayesian} 

\paragraph{MIMIC-III results.}
In Figure~\ref{fig:MIMIC_results}, we assess the performance of surrogate posterior distributions learned from 1D simulations in predicting HR and LVET using 8-second waveforms from the MIMIC-III dataset~\citep{johnson2016mimic}. Examples of such waveforms are showcased in \appref{app:sample_generation}. As the posterior distributions are uni-modal for the LVET and HR (see, e.g., \appref{app:add_exp}), we focus on point estimates obtained by taking the expectation of the posterior distributions. While we can accurately determine HR by counting the number of beats, assessing LVET is more challenging as there is no gold standard method for obtaining it from PPG or APW. To address this, we use electrocardiograms from MIMIC-III and standard digital signal processing techniques to estimate LVET~\citep{dehkordi2019comparison, alhakak2021significance}. Although this estimation method is not perfect, it serves as a baseline for comparison. We evaluate the mean absolute error (MAE) and correlation between the point estimates and the labels. We must carefully take into account that HR and LVET are negatively correlated in healthy populations, and hence in the prior distribution considered in-silico. We prevent this potentially spurious correlation from corrupting our analysis by explicitly conditioning both prior and posterior distributions on HR and then averaging out the effect of HR.

We reduce prior misspecification (see discussion in Section~\ref{sec:misspecification_Bayesian}) by restricting our analysis to segments that fall within the support of the prior distribution, specifically $\text{HR} \in [ 60, 90]$ and  $\text{LVET} \in [ 230, 330]$. This filtering leaves 547 patients and one measurement per patient. In Figure~\ref{fig:MIMIC_results}, we observe successful transfer of posterior distributions to real-world data for HR but not for LVET. The MAE of LVET approaches that of the prior distribution, indicating limited improvement. However, the remaining correlation between the predicted and real LVET values suggest a partial transfer of information.

The uncertainty analysis in Figure~\ref{fig:identifiability_analysis} aligns with the in-vivo results, showing that HR estimation performs steadily well if SNR is higher than 5dB. At the same time, in-silico and in-vivo results are contradictory for the LVET. In-vivo, the best transfer occur at high noise levels, suggesting that the LVET effect is significantly misrepresented. Investigating and alleviating this misspecification with the appropriate modifications to the model might be crucial to successfully transferring findings from in-silico to in-vivo. This iterative process of \textbf{1.} model analysis, \textbf{2.} real-world experimentation, \textbf{3.} comparison with observations, and \textbf{4.} model refinement; exemplifies the scientific method. In-silico and in-vivo experiments demonstrate that SBI facilitates more scrutiny in applying the scientific loop to cardiovascular models relying on numerical simulations, in extracting scientific hypotheses from the model (step \textbf{1.}); and comparing theoretical predictions and real-world data (step \textbf{3.}).
% Our results demonstrate the value of SBI for mining scientific hypotheses from in-silico cardiovascular models. 
Furthermore, the posterior distributions obtained with SBI provide a multi-dimensional representation of uncertainty and enable to study insights both at the population and the individual level. These analyses provide insights that go beyond what uni-dimensional and population-aggregated uncertainty and identifiability analyses would conclude. 
% Finally, in Section~3\ref{sec:in-vivo}, results on real-world measurements support a critical discussion of how in-silico analyses can provide insights that generalise to the real world. 

% In summary, by carefully examining the posterior distributions learned from simulations onto real-world data, we can identify aspects of the model that effectively transfer in-vivo and those that require further refinement. If necessary, in-silico analyses may guide the resolution of such misspecification by informing when and which parameters are hinted identifiable by the misspecified model. The next step is to gather the corresponding set of real-world labelled data and resolve the misspecification on this set, e.g. by modifying the noise model. A consecutive uncertainty analysis will tell if these parameters are still identifiable under the new and better-specified model.


% By stratifying the population based on expected levels of error using the posterior distribution, we can validate the transferability of the learned posterior distributions. Additionally, an important aspect left for future work is to propose new noise models that produces posterior distributions whose uncertainty is well-calibrated on real-world signals.

% Figure environment removed


\section{Discussion}\label{sec:discussion}

\paragraph{Sensitivity analysis.}
The typical tool for understanding 1D hemodynamics models is variance-based sensitivity analysis~\citep[VBSA,][]{melis2017bayesian, piccioli2022effect,schafer2022uncertainty}.
% Several studies~\citep{melis2017bayesian, piccioli2022effect,schafer2022uncertainty} based on variance-based sensitivity analysis~(VBSA) have made essential contributions to a better understanding of 1D hemodynamics models.
In \cite{melis2017bayesian}, authors perform the analysis on a learnt surrogate of the simulator; \cite{schafer2022uncertainty} relies on polynomial chaos expansion; \cite{piccioli2022effect} studies the effects of various cardiac parameters. 
By contrast, Section~\ref{sec:in-silico} shows that SBI-based uncertainty analysis supports the study of a richer set of uncertainty properties compared to VBSA. For instance, SBI quantifies uncertainty for individual measurements and not only on the population level (highlighted in \figref{fig:npe_vs_laplace}).
% Section~3\ref{sec:in-silico} has shown that SBI-based uncertainty analyses similarly support studying properties of 1D hemodynamics models. 
% Compared to these works, a distinctive attribute of SBI is the ability to quantify uncertainty for individual measurements and not only on the population level (highlighted in \figref{fig:npe_vs_laplace}). 
Furthermore, multi-dimensional VBSA poses computational challenges and is thus not studied in the literature, whereas SBI directly addresses this challenge by learning a joint posterior distribution. Finally, the ambiguity of inverse solutions, which SBI highlights with multi-modal posterior distributions, may invalidate conclusions drawn from uni-dimensional VBSAs.

% our analysis In contrast to  uni-dimensional statistics. 
% We have introduced the SBI methodology to analyse CV simulations, enabling multi-dimensional and individual-level uncertainty analyses. 
% Our results in \figref{fig:npe_vs_laplace} emphasised the need to go beyond these classical sensitivity and uncertainty analyses as neglecting the true complexity of the statistical model may lead to conclusions that are inconsistent with the forward model considered.

\paragraph{ML-based inversion of CV simulators.}
% Another line of work~\citep{chakshu2021towards, jin2021estimating, bikia2021estimation, ipar2021blood, bonnemain2021deep} use ML to inverse CV simulations. 
ML methods can reveal complex relationships between biomarkers and biosignals that are hidden to VBSAs, especially in the presence of salient nuisance effects, as demonstrated in \cite{chakshu2021towards, jin2021estimating, bikia2021estimation, ipar2021blood, bonnemain2021deep}. 
Nevertheless, ML-based sensitivity analyses~(MLBSAs) that do not rely on an effective representation of uncertainty have limitations similar to the ones highlighted for VBSAs in Section~\ref{sec:in-silico}. For instance, they cannot reveal the ambiguity of inverse solutions, as pointed out in \cite{chakshu2021towards}. 
In addition to providing a better understanding of biomarker predictability, MLBSA often aims to address data scarcity, a common challenge for ML in CV applications, with simulated data~\citep{chakshu2021towards, jin2021estimating, bikia2021estimation, ipar2021blood, bonnemain2021deep}. However, as illustrated in Section~\ref{sec:in-vivo}, model misspecification hinders the direct transfer of predictors learned in-silico to real data. Nevertheless, SBI may reveal more general relationships from the simulations (such as dependencies between biomarkers or sub-population identification), which can inform data collection and design of ML algorithms well-suited for the intended CV application.
% In this context, hybrid learning strategies~\citep{takeishi2021physics, yin2021augmenting, wehenkel2022robust}, combining in-silico and in-vivo data, are promising directions that should help deploy such models in production.
% Similarly, SBI leverages ML for revealing sophisticated relationships. Compared to these works, however, SBI offers a more contrasted picture that reveal insights at the level of individual measurements, e.g., ambiguous inverse solutions with multi-modality or dependencies.
% However, offers a more contrasted picture that considers multi-modality, dependencies between parameters and individualised uncertainties. 
% However, similarly to studies relying on VBSAs, these results only quantify identifiability at the population-level and independently for each parameter considered.
% For instance, \cite{bonnemain2021deep} studies left ventricle~(LV) systolic functions in the presence of LV assist device with 0D models; \cite{ipar2021blood} considers various biomarkers related to arterial age; \cite{chakshu2021towards}
% relies on CV simulators to address the challenge of data scarcity for learning an ML-based point predictor of latent biomarkers from biosignals. Similarly to standard sensitivity analyses, these studies inform on the potential identifiability 
% Although NPE is not more than an elaborated ML algorithm, it targets an accurate representation of uncertainty. It aims to help the application of the scientific loop on CV simulation, which contrasts with these ML approaches. While small misspecification, unnoticeable to a human observer, can dramatically hurt generalisation to in-vivo data, insights revealed by uncertainty analyses of the CV model have a better chance to generalise to the real world.
% These insights may provide recommendations for gathering real-world data, potentially supporting the training of ML models that work in the real world.
% In this context, hybrid learning strategies~\citep{takeishi2021physics, yin2021augmenting, wehenkel2022robust}, combining in-silico and in-vivo data, are promising directions that should help deploy such models in production.

\paragraph{Next generation of CV simulators.}
Many research projects build upon the versatility and computational efficiency of full-body 1D hemodynamics, relating biosignals to a growing number of biomarkers~\citep{mejia2022effects, buoso2021personalising, coccarelli2021framework}.
Extensions of these models regularly introduce additional parameters, hence additional sources of uncertainty, which further necessitate a probabilistic perspective on the associated inverse problems.
Another avenue for extending CV models is the joint analysis of biosignals relying on distinct physical processes, e.g., PPG and electrocardiograms. In this context, while combining point estimators consistently is complicated, posterior estimators of each model can be combined easily into a joint posterior distribution. Indeed, assuming a conditional independence  $\mathcal{X} \perp \mathcal{Y} \mid \Phi$ between the two forward models $p(x\mid \phi)$ and $p(y\mid \phi)$ allows rewriting the posterior distribution given the two measurements as the product of each posterior distribution: $p(\phi \mid x, y) \propto p(\phi \mid x) p(\phi \mid y)$.
% \paragraph{SBI and the scientific loop.}
% The main advantage of SBI is its ability to port the rigour and practicality of statistical analyses to complex CV simulators. 
% As Section~4\ref{sec:in-vivo} mentions, statistical inference plays two roles in scientific enquiry. First, it reveals implicit hypotheses that follow from the model, guiding real-world experiments that aim to confirm or reject these implications. Second, analysing the gathered real-world data requires another statistical inference tied to the model considered. Our discussion also highlighted the importance of coping with model misspecification. Contrary to what one might think, the growing complexity of CV models does not especially result in lighter misspecification but can increase it. In this context, SBI is particularly well-positioned to address these challenges as it may directly extend robust Bayesian and frequentist inference strategies~\citep{berger1994overview, huang2023learning, cherief2020mmd} to simulators. SBI should thus provide long-term support to research in CV health.


%!TEX root = jsac_v6.tex


\vspace{-0.3cm}
\section{Numerical Analysis}
The analysis presented above highlights that the mutual coupling effects resulting from closely spaced antennas can potentially provide benefits to the uplink spectral efficiency in single-user and multi-user Holographic MIMO systems, depending on the specific propagation conditions. However, it is important to note that these conditions may not be met in practical network scenarios, and therefore the average gains may be marginal or even non-existent. The analysis focused on a simplified uplink case study with two antennas, two UEs, and single-path LoS propagation. Next, the numerical analysis is expanded to more realistic scenarios, including a larger number of antennas, multiple UEs, and different propagation conditions. Additionally, the analysis considers the case of densely packed antennas in a space-constrained form factor. By exploring these scenarios, a more comprehensive understanding of the benefits of mutual coupling in Holographic MIMO systems can be obtained.

% We assume that the BS is located at a height of $10$~m. The communication takes places over a bandwidth of $B = 20$~MHz, with the total receiver noise power σ2 = −87 dBm. Each UE transmits with power pk = 20 dBm ∀k. We assume a carrier frequency of f0 = 28GHz such that λ = 10.71mm, NH = 62, and NV = 42, to focus on a 5G hot-spot scenario. When relevant, throughout the letter, we also consider higher carrier frequencies that cover future use cases and scenarios.


The key parameters of the system are those reported in Table~\ref{tab:array_parameters}. Unless otherwise stated, we consider a scenario with single-path LoS propagation, where the BS is positioned at a height of $10$~m. The azimuth angle of each UE is randomly distributed within the sector $[-\pi/2, \pi/2]$ while the elevation angle depends on the distance from the BS. The UEs are randomly dropped at a minimum distance of $15$~m and a maximum distance of $150$~m from the BS, and they transmit with the same power. The results are obtained by averaging over $1000$ UE drops. Due to space limitations, our main emphasis is on the uplink but we put a specific focus on addressing the duality issue in the downlink. Moreover, we still focus on line-of-sight propagation but notice that similar results can be obtained with different channel models, e.g., based on stochastic approaches. The Matlab code that will be made available upon completion of the review process can be used to generate the omitted results.
% Figure environment removed

\subsection{Fixing the Number of Antennas while Varying Array Size}

% \textcolor{red}{Uniformare le label dell'asse Y nelle figure 10.a e 10.b. In particolare, fare attenzione al fatto che, anche in altre figure, alcune volte l'efficienza spettrale viene indicata in [bit/s/Hz] altre volte in [bit/s/Hz/cell].}

Fig.~\ref{fig:SE_vs_dH_K10_M16,64} illustrates the average SE per UE in the uplink as a function of $d_H/\lambda$ for two different antenna configurations: $M_{\rm BS}=16$ and $M_{\rm BS}=64$, with a fixed number of UEs, $K=10$. The results show that, \textit{when the number of antennas is fixed, reducing the antenna spacing generally has a negative impact on the average SE}. Optimal performance is observed for $d_H/\lambda > 0.5$, where the advantages of mutual coupling 
% \textcolor{red}{Cosa ci aspettiamo in assenza di coupling? È realmente significativo il coupling quando $d_H = \lambda$? Dalla figura 3 sembrerebbe che per $d_H = \lambda$ non ci sia accoppiamento.}  \textcolor{blue}{Una minima differenza rispetto a $\lambda/2$ c'e'.} 
and holographic MIMO become more pronounced. In this range, employing a full matching network yields only a marginal gain compared to the self-impedance matching design. However, a significant decrease in SE occurs when no matching network is utilized. Additionally, as expected, increasing the number of antennas results in higher SE due to enhanced interference rejection capabilities. Similar conclusions can be drawn from Figure~\ref{fig:SE_vs_dH_M32_K8,24}, where the number of antennas is fixed at $M_{\rm BS}=32$, while the number of UEs is varied between $K=10$ and $K=30$.
% \textcolor{red}{La figura 10.b la farei solamente con la full matching network, per K = 8, 16, 24. Direi che nei casi SI matching network o senza matching network le conclusioni sono le stesse della figura 10.a.} \textcolor{blue}{Per me va bene. Mi sembrava pero interessante far vedere che all'aumentare di $K$ il gap con SI matching e senza matching diminuisce. Nel senso che sono meno importanti quando $K$ si avvicina ad $M_{\rm BS}$.}
The curves in Fig.~\ref{fig:SE_vs_dH_K10_M16,64} were obtained with $R_{\rm d} =10^{-3} R_{\rm r}$. Numerical results (not reported for space limitations) show that the SE worsens as $R_{\rm d}$ increases but similar behaviors can be observed. We also notice that the effect of $R_{\rm d}$ is more significant for $d_H < \lambda/2$ while a marginal impact is observed for large antenna spacings (i.e, $d_H > \lambda/2$).

% Figure environment removed

% Fig.~\ref{fig:SE_vs_dH_K10_M16,64} illustrates the average SE per UE as a function of $d_H/\lambda$ for two different numbers of antennas, $M_{\rm BS}=16$ and $M_{\rm BS}=64$, with a fixed number of UEs, $K=10$. The results confirm that reducing the antenna spacing generally has a negative impact on the average SE. The best performance is observed for $d_H/\lambda > 0.5$, where the benefits of mutual coupling and holographic MIMO are more pronounced. In this regime, the use of a full matching network provides only a marginal gain compared to the self-impedance matching design. However, a significant loss in SE is incurred when no matching network is employed. Furthermore, as expected, an increase in the number of antennas leads to higher spectral efficiency due to improved interference rejection capabilities. Similar conclusions can be drawn from Fig.~\ref{fig:SE_vs_dH_M32_K10,30} where the number of antennas is fixed to $M_{\rm BS}=32$ while the number of UEs is $K=10$ or $30$ 


% Fig.~\ref{fig:SE_vs_dH_K10_M16,64} shows the average SE per UE as a function of $d_H/\lambda$ for $M_{\rm BS}=16$ and $M_{\rm BS}=64$. The number of UEs is $K=10$. The results confirm that reducing the antenna spacing has on average a negative effect on the spectral efficiency. The best performance is obtained for $d_H/\lambda > 0.5$, so that a full matching network provides a negligible gain compared to the self-impedance design. A considerable loss is incurred if no matching network is used. As expected, an increase in the number of antennas results in a larger SE due to a better capability to reject the interference. Similar conclusions can be drawn from Fig.~\ref{fig:SE_vs_dH_M32_K10,30} where the number of antennas is fixed to $M_{\rm BS}=32$ while the number of UEs is $K=10$ or $30$.   


\subsection{Fixing the Array Size while Varying the Number of Antennas}

Fig.~\ref{fig:SE_vs_dH_Lfixed} illustrates the average SE per UE in the uplink as a function of $d_H/\lambda$ for a fixed array size $L_H = 6 \lambda$. Both MR and MMSE receivers exhibit similar SE behaviors. Notably, when a full matching network is employed, SE increases as $d_H/\lambda$ decreases due to the augmented number of antennas $M_{\rm BS}=L_H/d_H + 1$. This increase in antennas contributes to higher array gain and improved interference rejection. However, without a matching network or with a self-impedance matching network, the optimal performance is achieved when $d_H/\lambda \approx 0.4$. Going below this value may result in a decrease in SE. It is important to emphasize that the SE improvement observed when reducing $d_H$ with a full matching network cannot be attributed to antenna coupling. This can be clearly seen in Figure~\ref{fig:ChannelGain_vs_dH_Lfixed}, which displays the channel gain per antenna using MMSE, corresponding to the same simulation scenario as in Figure~\ref{fig:SE_vs_dH_Lfixed}. The results demonstrate that bringing the antennas closer to each other in Holographic MIMO systems can have adverse effects on the channel gain, even when employing a full matching strategy. This is evident from the declining trend of the channel gain as the antenna spacing decreases.



Fig.~\ref{SE_vs_dH_K10_LH61224} shows the average SE per UE as a function of $d_H/\lambda$ for three different values of the array size $L_H$. The number of UEs is $K=10$ and an MMSE combiner is employed, with either full noise or SI matching networks. As can be seen from the results, the behavior is the same irrespective of the array size. We only observe that, moving the antennas close to each other, the gain reduces as $L_H$ increases. In particular, when $L_H=6 \lambda$ and a full matching network is used, for $d_H=\lambda/10$ the average SE is about $4 \operatorname{bit/s/Hz}$ and drops to about $2 \operatorname{bit/s/Hz}$ for $d_H=\lambda$, with a ratio of 2 between the two values. On the other hand, when $L_H=24 \lambda$ the ratio decreases to about $6/4.5 \approx 1.33$. Essentially, this means that the positive effects of packing the antennas close to each other is greater for smaller antenna sizes.   

% Fig.~\ref{fig:SE_vs_dH_Lfixed} shows SE as a function of $d_H/\lambda$ for a fixed array size $L_H = 6 \lambda$. As can be seen, the spectral efficiency has similar behaviors with both MR and MMSE receivers. In particular, we observe that with a full matching network SE increases as $d_H/\lambda$ decreases, due to the increase in the number of antennas $M_{\rm BS}=L_H/d_H + 1$. This has beneficial effects on the the array gain (which increases with $M_{\rm BS}$) and interference rejection. On the other hand, without matching network or with a self-impedance matching network the best performance is attained for $d_H/\lambda \approx 0.4$, and decreasing $d_H/\lambda$ below this value may result in some losses in terms of spectral efficiency.

% It is worth pointing out that the increase in SE when $d_H$ is reduced, observed with a full matching network, cannot be attributed to the coupling between the antennas. This can be appreciated with the aid of Fig.~\ref{fig:ChannelGain_vs_dH} where we report the channel gain per antenna, with MMSE, in the same simulation scenario of Fig.~\ref{fig:SE_vs_dH_Lfixed}. Indeed, the results demonstrate that moving the antennas close to each other in Holographic MIMO systems can have detrimental effects on the channel gain, even when using a full matching strategy. This is evident from the decreasing trend of the channel gain as the antenna spacing decreases. 

% Figure environment removed


% \subsection{Impact of Array and System Parameters}
% The results of Fig. ... have been obtained with $\zeta = R_d /R_r =10^{-3}$. To show the impact of $R_d$ on the system performance, in Fig. 11a and 11.b we report the spectral efficiency, for different values of $\zeta$. The number of UEs is $K=10$. In particular, Fig. 11a shows SE as a function of $d_H/\lambda$ for a fixed number of antennas, namely $M_{\rm BS}=32$, whereas in Fig. 11b the array size is $L_H=6/\lambda$ and the number of antennas vary according to $M_{\rm BS} = L_H/d_H+1$. The results of ...


% \subsection{Impact of Dissipation Resistance}
% To assess the effects of the dissipation resistance on the system performance, in Fig.~\ref{fig:SE_vs_dH_5zeta} we plot the average sum SE as a function of the antenna spacing for different values of $\zeta=R_{\rm d}/R_{\rm r}$.\textcolor{blue}{e' stata ottenuta con $M_{\rm BS}=32$ e $K=10$.} \textcolor{red}{Riportare i valori dei parametri con cui sono state ottenute le curve.} As expected, performance worsens as $R_{\rm d}$ increases. However, as can be seen the effects of $R_{\rm d}$ are more significant for $d_H < \lambda/2$ while a marginal impact is observed for large antenna spacings (i.e, $d_H > \lambda/2$).


% The spatial correlation matrix of UE $k$ is modelled according to~\eqref{eq:spatial-correlation} , e.g.,~\cite[Sec. 2.6]{massivemimobook}, with $f_k(\theta, \phi)$ being the joint probability density function of the azimuth angle $\theta$ and elevation angle $\phi$. We assume that $\theta = \theta_k + \delta_\theta$, $\phi = \phi_k + \delta_\phi$ where $\delta_\theta$ $\delta_\phi$ are random deviations from the nominal angles $\theta_k$ and $\phi_k$, uniformly distributed with standard deviations $\sigma^2_{\theta}$ and $\sigma_{\phi}$.




% % Figure environment removed


\subsection{Impact of uplink and downlink duality}
We now consider the downlink with MMSE precoding, and with either a full or an SI matching network. From Table~\ref{tab:duality} it is seen that, with a full matching network, ${\bf h}_k^{\rm ul}$ and ${\bf h}_k^{\rm dl}$ differ only for a scaling factor. Accordingly, it is correct to design the MMSE precoder in downlink by using the measured value of ${\bf h}_k^{\rm ul}$ in uplink. On the other hand, when an SI matching network is employed, the uplink-downlink channel duality requires to apply a linear transformation to ${\bf h}_k^{\rm ul}$. A performance loss is incurred if this is not done.



Fig.~\ref{fig:SE_vs_dH_M32K8_DownLink} shows the average SE per UE in the same setup of Fig.~\ref{fig:SE_vs_dH_M32_K8,24}, i.e., with $M_{\rm BS} = 32$ antennas and $K = 8 $ UEs. We see that, with a full matched network, the performance in uplink and downlink is the same. As for the SI matching design, two different cases have been considered. In the first case, the MMSE precoder is computed by using ${\bf h}_k^{\rm ul}$ instead of ${\bf h}_k^{\rm dl} = \dfrac {\alpha_{\rm dl}} {\alpha_{\rm ul}} {\bf B}_{\rm dl}^{-\Ttran/2}{\bf A}_{\rm dl,ul}{\bf h}_k^{\rm ul}$, as indicated in Table~\ref{tab:duality}. In the second case, the MMSE precoder is correctly computed taking the matrix ${\bf B}_{\rm dl}^{-\Ttran/2}{\bf A}_{\rm dl,ul}$ into account. Comparing the results in Fig.~\ref{fig:SE_vs_dH_M32K8_DownLink} with those in 
Fig.~\ref{fig:SE_vs_dH_M32_K8,24}, we see that in the latter case the average SE is the same in uplink and downlink, while a considerable loss is observed in the former case (thicker line), especially at low values of $d_H/\lambda$. The same conclusions can be drawn from Fig.~\ref{fig:SE_vs_dH_Lh6K10_DownLink}, obtained in the simulation setting of Fig.~\ref{SE_vs_dH_K10_LH61224}, which shows the average SE per UE for a fixed size $L_H = 6 \lambda$ of the array and $K=10$.


% Figure environment removed


\section{Conclusion}

% We have demonstrated that SBI enables fine-grained uncertainty analysis and comparison of the relationship between various biomarkers and biosignals. 
We have introduced a simulation-based inference methodology to analyze complex cardiovascular simulators. 
% Our results highlight that the simulation-based inference methodology is able to extract insights consistent with variance-based sensitivity analysis. 
Our results show that our simulation-based inference method yields additional insights about 1D hemodynamics models, beyond the commonly-used VBSA and MLBSA techniques. This is done by considering the complete posterior distribution, which provides a consistent and multi-dimensional quantification of uncertainty for individual measurements.
% Furthermore, we show how simulation-based inference goes beyond variance-based sensitivity analysis owing to its fine-grained representation of uncertainty. 
This uncertainty representation enables us to recognize ambiguous inverse solutions, study the heterogeneity of sensitivity in the population considered, and understand dependencies between biomarkers in the inverse problem. Supported by results on real-world data, we have discussed the challenge of model misspecification in scientific inquiry and how to tackle it in the context of full-body hemodynamics. 
In summary, simulation-based inference enables scientists to address inverse problems in CV models, accounting for complex forward model dynamics and individualized uncertainty. Our work provides foundations for a more effective use of CV simulations for scientific inquiry and personalized medicine.

% In summary, simulation-based inference allows scientists to study inverse problems arising from CV models with a methodology that acknowledges the forward model complexity and enables an individualised representation of uncertainty. These two aspects are necessary to fully leverage CV simulations for scientific enquiry and personalised medicine.

% facilitates iterations over the scientific loop and, thus, the discovery of real-world insights from CV simulations.
% The rigour and versatility of the SBI framework should eventually enhance the scientific and real-world impact of complex cardiovascular simulations.


\bibliography{main}
\bibliographystyle{icml2023}
% \bibliography{biblio}

\clearpage

\appendix
\onecolumn

% This is samplepaper.tex, a sample chapter demonstrating the
% LLNCS macro package for Springer Computer Science proceedings;
% Version 2.20 of 2017/10/04
%
\documentclass[runningheads]{llncs}
%
\usepackage{graphicx}
% Used for displaying a sample figure. If possible, figure files should
% be included in EPS format.
%
% If you use the hyperref package, please uncomment the following line
% to display URLs in blue roman font according to Springer's eBook style:
% \renewcommand\UrlFont{\color{blue}\rmfamily}

\begin{document}
%
\title{Multi-View Vertebra Localization and Identification from CT Images \\ Supplementary Material}
%
%\titlerunning{Abbreviated paper title}
% If the paper title is too long for the running head, you can set
% an abbreviated paper title here
%
\author{Paper ID: 534}
%
\authorrunning{Paper ID: 534}
% First names are abbreviated in the running head.
% If there are more than two authors, 'et al.' is used.
%
\institute{}
%
\maketitle              % typeset the header of the contribution

\begin{table}
\centering
\caption{Evaluation results on a large-scale in-house dataset collected from the practical clinics with 500 CT scans divided into 300 for training, 100 for testing, and 100 for validation. We train the model on the training dataset, and further evaluate it on the test and validation dataset with $K$ set to 10.}
\begin{tabular}{l|ll|ll}
\hline
     & \multicolumn{2}{l|}{Test dataset}              & \multicolumn{2}{l}{Validation dataset}         \\ \cline{2-5} 
     & \multicolumn{1}{l|}{Id-Rate(\%)} & L-Error(mm) & \multicolumn{1}{l|}{Id-Rate(\%)} & L-Error(mm) \\ \hline
Cer. & \multicolumn{1}{l|}{99.67}       & 1.31       & \multicolumn{1}{l|}{99.55}       & 1.51       \\
Tho. & \multicolumn{1}{l|}{98.24}       & 1.34       & \multicolumn{1}{l|}{99.00}       & 1.48       \\
Lum. & \multicolumn{1}{l|}{99.31}       & 1.35       & \multicolumn{1}{l|}{99.54}       & 1.50       \\ 
All  & \multicolumn{1}{l|}{98.62}       & 1.34       & \multicolumn{1}{l|}{99.04}       & 1.49       \\ \hline
\end{tabular}
\end{table}

% Figure environment removed


\end{document}


\end{document}


% This document was modified from the file originally made available by
% Pat Langley and Andrea Danyluk for ICML-2K. This version was created
% by Iain Murray in 2018, and modified by Alexandre Bouchard in
% 2019 and 2021 and by Csaba Szepesvari, Gang Niu and Sivan Sabato in 2022.
% Modified again in 2023 by Sivan Sabato and Jonathan Scarlett.
% Previous contributors include Dan Roy, Lise Getoor and Tobias
% Scheffer, which was slightly modified from the 2010 version by
% Thorsten Joachims & Johannes Fuernkranz, slightly modified from the
% 2009 version by Kiri Wagstaff and Sam Roweis's 2008 version, which is
% slightly modified from Prasad Tadepalli's 2007 version which is a
% lightly changed version of the previous year's version by Andrew
% Moore, which was in turn edited from those of Kristian Kersting and
% Codrina Lauth. Alex Smola contributed to the algorithmic style files.
