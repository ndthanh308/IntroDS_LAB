%%%%%%%%%%%%%%%%%%%%%%%%%%%%%%%%%%%%%%%%%%%%%%%%%%%%%%%%%%%%%%%%%%%%%
%% This is a (brief) model paper using the achemso class
%% The document class accepts keyval options, which should include
%% the target journal and optionally the manuscript type. 
%%%%%%%%%%%%%%%%%%%%%%%%%%%%%%%%%%%%%%%%%%%%%%%%%%%%%%%%%%%%%%%%%%%%%
\documentclass[journal=jacsat,manuscript=article]{achemso}

%%%%%%%%%%%%%%%%%%%%%%%%%%%%%%%%%%%%%%%%%%%%%%%%%%%%%%%%%%%%%%%%%%%%%
%% Place any additional packages needed here.  Only include packages
%% which are essential, to avoid problems later. Do NOT use any
%% packages which require e-TeX (for example etoolbox): the e-TeX
%% extensions are not currently available on the ACS conversion
%% servers.
%%%%%%%%%%%%%%%%%%%%%%%%%%%%%%%%%%%%%%%%%%%%%%%%%%%%%%%%%%%%%%%%%%%%%
\usepackage[version=3]{mhchem} % Formula subscripts using \ce{}
\usepackage{braket}
\usepackage{mathtools}
\usepackage{amsthm, amssymb, amsfonts}
\usepackage{cancel}
\usepackage{array}
\usepackage{listings}
\usepackage{float}
\usepackage{bm}
\usepackage{graphicx}
\usepackage{threeparttable}


% \usepackage{graphicx}
% \usepackage{multicol}
% \usepackage{multirow}

% \usepackage{simplewick}
% \usepackage{subcaption}
% \usepackage{wrapfig}

% \usepackage{fancyhdr}

% \usepackage{afterpage}
% \usetikzlibrary{fadings}

% \usepackage{doi}

% \usepackage{hyperref}

% \usepackage{babel}

% \usepackage[left=2cm,right=2cm,top=2.0cm,bottom=2.0cm]{geometry}
% \usepackage[utf8]{inputenc}


\setlength{\unitlength}{1cm}

\usepackage{tikz}


\newcommand{\finfo}[2]{f_{#1#2}}
\newcommand{\f}[2]{f^{#1}_{#2}}
\newcommand{\fetOperateur}[2]{f^{#1}_{#2}\left\lbrace \adag #1 
 \aoper #2  \right\rbrace}
\newcommand{\Vinfo}[4]{\left\langle #1#2 \right| \left| #3#4 \right\rangle }
\newcommand{\Vinfochem}[4]{\left(  #1#2 \right| \! \left| #3#4 \right) }
\newcommand{\Vinfochemspin}[4]{\left[  #1#2 \right| \! \left| #3#4 \right]_{spin} }
\newcommand{\V}[4]{V^{#1#2}_{#3#4}}
\newcommand{\VetOperateur}[4]{V^{#1#2}_{#3#4}\left\lbrace \adag #1 
\adag #2 \aoper #4 \aoper #3 \right\rbrace }

\newcommand{\Lr}{\mathcal{L}}
\newcommand{\Rr}{\mathcal{R}}
\newcommand{\Gr}{\mathcal{G}}
\newcommand{\Hcurl}{\mathcal{H}}
\newcommand{\continuum}{\mathbf{c}}


\newcommand{\Finfo}[4]{\mathcal{F}_{#1#2}[#3,#4]}
\newcommand{\Ftinfo}[4]{\tilde{\mathcal{F}}_{#1#2}[#3,#4]}
\newcommand{\Winfo}[8]{\mathcal{W}_{#1#2#3#4}[#5,#6,#7,#8]}
\newcommand{\Wtinfo}[8]{\tilde{\mathcal{W}}_{#1#2#3#4}[#5,#6,#7,#8]}
\newcommand{\Ginfo}[2]{\mathcal{G}_{#1#2}}
\newcommand{\Gtinfo}[2]{\tilde{\mathcal{G}}_{#1#2}}
\newcommand{\taupinfo}[4]{\tau_{#1#2#3#4}}
\newcommand{\tauptinfo}[4]{\tilde{\tau}_{#1#2#3#4}}

\newcommand{\F}[2]{\mathcal{F}^{#1}_{#2}}
\newcommand{\Ft}[2]{\tilde{\mathcal{F}}^{#1}_{#2}}
\newcommand{\Fb}[2]{\bar{\mathcal{F}}^{#1}_{#2}}
\newcommand{\W}[4]{\mathcal{W}^{#1#2}_{#3#4}}
\newcommand{\Wt}[4]{\tilde{\mathcal{W}}^{#1#2}_{#3#4}}
\newcommand{\Wb}[4]{\bar{\mathcal{W}}^{#1#2}_{#3#4}}
%\newcommand{\G}[2]{\mathcal{G}^{#1}_{#2}}
\newcommand{\Gt}[2]{\tilde{\mathcal{G}}^{#1}_{#2}}
\newcommand{\taup}[4]{\tau^{#1#2}_{#3#4}}
\newcommand{\taupt}[4]{\tilde{\tau}^{#1#2}_{#3#4}}

%Unknown Tensor
\newcommand{\TensorS}[2]{T^{#1}_{#2}}
\newcommand{\TensorD}[4]{T^{#1#2}_{#3#4}}
\newcommand{\TensorT}[6]{T^{#1#2#3}_{#4#5#6}}
\newcommand{\TensorQ}[8]{T^{#1#2#3#4}_{#5#6#7#8}}

\newcommand{\ts}[2]{t^{#1}_{#2}}
\newcommand{\td}[4]{t^{#1#2}_{#3#4}}
\newcommand{\tXs}[2]{{\overset{{\textsc{x}}}{t^{#1}_{#2}}}}
\newcommand{\tXd}[4]{{\overset{{\textsc{x}}}{t^{#1#2}_{#3#4}}}}
\newcommand{\tYs}[2]{{\overset{{\textsc{y}}}{t^{#1}_{#2}}}}
\newcommand{\tYd}[4]{{\overset{{\textsc{y}}}{t^{#1#2}_{#3#4}}}}


\newcommand{\Ys}[2]{Y^{#1}_{#2}}
\newcommand{\Yd}[4]{Y^{#1#2}_{#3#4}}
%\newcommand{\tri}[6]{t^{#1#2#3}_{#4#5#6}}

\newcommand{\lambdas}[2]{\lambda^{#1}_{#2}}
\newcommand{\lambdad}[4]{\lambda^{#1#2}_{#3#4}}
\newcommand{\Lambdas}[2]{\Lambda^{#1}_{#2}}
\newcommand{\Lambdad}[4]{\Lambda^{#1#2}_{#3#4}}
\newcommand{\sigmas}[2]{\sigma^{#1}_{#2}}
\newcommand{\sigmad}[4]{\sigma^{#1#2}_{#3#4}}
\newcommand{\ro}{r_{0}}
\newcommand{\rs}[2]{r^{#1}_{#2}}
\newcommand{\rd}[4]{r^{#1#2}_{#3#4}}
\newcommand{\ls}[2]{l^{#1}_{#2}}
\newcommand{\ld}[4]{l^{#1#2}_{#3#4}}
\newcommand{\rhos}[2]{\rho^{#1}_{#2}}
\newcommand{\rhod}[4]{\rho^{#1#2}_{#3#4}}

\newcommand{\gammas}[2]{\gamma^{#1}_{#2}}
\newcommand{\gammasb}[2]{\bar{\gamma}^{#1}_{#2}}
\newcommand{\gammad}[4]{\gamma^{#1#2}_{#3#4}}
\newcommand{\gammadb}[4]{\bar{\gamma}^{#1#2}_{#3#4}}

\newcommand{\Gammad}[4]{\Gamma^{#1#2}_{#3#4}}

\newcommand{\Ds}[2]{D^{#1}_{#2}}
\newcommand{\Dd}[4]{D^{#1#2}_{#3#4}}

\newcommand{\SUMs}[1]{\sum\limits_{#1}}
\newcommand{\SUMd}[2]{\sum\limits_{#1#2}}
\newcommand{\SUMt}[3]{\sum\limits_{#1#2#3}}
\newcommand{\SUMq}[4]{\sum\limits_{#1#2#3#4}}
\newcommand{\Pm}[2]{P_-(#1#2)}
\newcommand{\Pmm}[2]{P_{-#1#2}}
\newcommand{\dirac}[2]{\delta_{#1#2}}
\newcommand{\undemi}{\frac{1}{2}}
\newcommand{\unquart}{\frac{1}{4}}
\newcommand{\unhuit}{\frac{1}{8}}
\newcommand{\unseize}{\frac{1}{16}}

\newcommand{\CCdeux}{\ {}_{CC2}}

\newcommand{\sadag}[1]{#1^\dag}
\newcommand{\adag}[1]{\hat{a}_#1^\dag}
\newcommand{\aoper}[1]{\hat{a}_#1}

\newcommand{\braPhizero}{\left\langle \Phi_0 \right| }
\newcommand{\ketPhizero}{\left| \Phi_0 \right\rangle }

\newcommand{\braPhiia}{\left\langle \Phi_i^a \right| }
\newcommand{\ketPhiia}{\left|  \Phi_i^a \right\rangle  }

\newcommand{\braPhiijab}{\left\langle \Phi_{ij}^{ab} \right| }
\newcommand{\ketPhiijab}{\left| \Phi_{ij}^{ab}\right\rangle  }
\newcommand{\braHF}{\left\langle \mathrm{HF} \right|}
\newcommand{\ketHF}{\left| \mathrm{HF} \right\rangle }
\newcommand{\braCC}{\left\langle \mathrm{CC} \right|}
\newcommand{\ketCC}{\left| \mathrm{CC} \right\rangle }

\newcommand{\bramui}{\left\langle \mu_{i} \right|}


\newcommand{\Lagrange}{ \mathscr{L} }
\newcommand{\Quasi}{ \mathscr{Q} }

\newcommand{\PhismatrEOM}[2]{ \Phi_{\mathrm{#2}  
	}^{\mathrm{#1} }}
\newcommand{\PhidmatrEOM}[4]{
	\Phi_{\mathrm{#3#4}}^{\mathrm{#1#2}}}

\newcommand{\namediag}[2]{#1#2}
\newcommand{\namediagIP}[2]{#1#2${}^{\mathrm{IP}} $}
\newcommand{\namediagEA}[2]{#1#2${}^{\mathrm{EA}} $}
\newcommand{\namediagequa}[1]{{}_{\ \{#1\}}}
\newcommand{\namediagequaIP}[1]{{}_{\ \{#1^{\mathrm{IP}}\}}}
\newcommand{\namediagequaEA}[1]{{}_{\ \{#1^{\mathrm{EA}}\}}}

\newcommand{\equivalencesimple}[2]{\{#1#2\}}
\newcommand{\equivalencedouble}[4]{\{\{#1#2\};\{#3#4\}\}}
\newcommand{\equivalencetriple}[6]{\{\{#1#2#3\};\{#4#5#6\}\}}
\newcommand{\equivalencequadruple}[8]{\{\{#1#2#3#4\};\{#5#6#7#8\}\}}

\newcommand{\as}[2]{a^{#1}_{#2}}
\newcommand{\ad}[4]{a^{#1#2}_{#3#4}}
\newcommand{\us}[2]{u^{#1}_{#2}}
\newcommand{\ud}[4]{u^{#1#2}_{#3#4}}
\newcommand{\vs}[2]{v^{#1}_{#2}}
\newcommand{\vd}[4]{v^{#1#2}_{#3#4}}
\newcommand{\Gdeltas}[2]{\Delta^{#1}_{#2}}
\newcommand{\Gdeltad}[4]{\Delta^{#1#2}_{#3#4}}

\newcommand{\MATone}[2]{{\mathrm{MAT}^1}_#2^#1}
\newcommand{\MATtwo}[2]{{\mathrm{MAT}^2}_#2^#1}
\newcommand{\Auxs}[2]{\mathrm{Aux}^{#1}_{#2}}
\newcommand{\Auxd}[4]{\mathrm{Aux}^{#1#2}_{#3#4}}

\newcommand{\MAToneIP}[2]{{\mathrm{MAT}^{\mathrm{IP}1}}_#2^#1}
\newcommand{\MATtwoIP}[2]{{\mathrm{MAT}^{\mathrm{IP}2}}_#2^#1}


\newcommand{\taudown}[4]{{\overset{{\tiny \APLup}}{\tau}}{}^{#1#2}_{#3#4}}
\newcommand{\tauno}[4]{{\overset{{\tiny \APLbox}}{\tau}}{}^{#1#2}_{#3#4}}
\newcommand{\tauboth}[4]{{\overset{{\tiny \APLup \APLdown}}{\tau}}{}^{#1#2}_{#3#4}}
\newcommand{\tauUp}[4]{{\overset{{\tiny\APLdown}}{\tau}}{}^{#1#2}_{#3#4}}

\newcommand{\xis}[2]{\xi^{#1}_{#2}}
\newcommand{\xid}[4]{\xi^{#1#2}_{#3#4}}


\newcommand{\EcritureFortran}[1]{{\noindent {\scriptsize #1 }}}

\newcommand{\DiracProgram}{$ \mathcal{D} $irac}

\newcommand{\light}{\mathrm{c}}
\newcommand{\qe}{\mathrm{e}}
\newcommand{\hbardroit}{\text{\textcrh}}

\newcommand{\XtwoC}{${}^2$DC}
\newcommand{\XtwoCmmf}{${}^2$DC$^{\text{M}}$}
\newcommand{\XtwoCmmfGaunt}{${}^2$DCG$^{\text{M}}$}


\newcommand{\CoutComp}[1]{$ \mathcal{O}\left( N^{#1}\right)  $}
\newcommand{\CoutOV}[2]{$\left(n_{occ}^{#1}n_{vir}^{#2}\right)$}

\newcommand{\Exatensor}{ExaTensor}
\newcommand{\Exacorr}{ExaCorr}
\newcommand{\Talsh}{TAL-SH}

\newcommand{\NULNUL}{\textcolor{red}{Des faits pas du baratin}}

\newcommand{\ordreH}[1]{^{\left( #1 \right) }}

\newcommand{\etal}{\emph{et al.}}

%\renewcommand*{\glstextformat}[1]{\emph{#1}}

\newcommand{\cmark}{\ding{51}}%
\newcommand{\xmark}{\ding{55}}%


\newcommand{\LH}[1]{\textcolor{orange}{*\todo[size=\tiny,
		color=white,linecolor=black,fancyline]{#1}}}
	
\newcommand{\ASPG}[1]{\textcolor{red}{*\todo[size=\tiny,
		color=white,linecolor=black,fancyline]{#1}}}
		
\newcommand{\RM}[1]{\textcolor{green}{RM : #1}}
\newcommand{\TJ}[1]{\textcolor{blue}{TJ : #1}}

\newcommand{\Mycite}[1]{\citeauthor{#1}~\cite{#1}}



\newcommand{\ip}{IP}%
\newcommand{\MBPTip}{MBPT(2)-IP}%
\newcommand{\pip}{P-IP}%
\newcommand{\pMBPTip}{P-\MBPTip}%

\newcommand{\iptab}{IP}%
\newcommand{\MBPTiptab}{PT(2)}%
\newcommand{\piptab}{P-IP}%
\newcommand{\pMBPTiptab}{P-\MBPTiptab}%

\newcommand{\ea}{EA}%
\newcommand{\MBPTea}{MBPT(2)-EA}%
\newcommand{\pea}{P-EA}%
\newcommand{\pMBPTea}{P-\MBPTea}%

\newcommand{\eatab}{EA}%
\newcommand{\MBPTeatab}{PT(2)}%
\newcommand{\peatab}{P-EA}%
\newcommand{\pMBPTeatab}{P-\MBPTeatab}%

\newcommand{\ee}{EE}%
\newcommand{\MBPTee}{MBPT(2)-EE}%
\newcommand{\pee}{P-EE}%
\newcommand{\pMBPTee}{P-\MBPTee}%

\newcommand{\eetab}{EE}%
\newcommand{\MBPTeetab}{PT(2)-EE}%
\newcommand{\peetab}{P-EE}%
\newcommand{\pMBPTeetab}{P-\MBPTeetab}%




\newcommand{\HamilDCG}{${}^{2}$DCG$^{M}$}%

%%%% Pour les niveaux des PES

\newcommand{\trenteetun}{X$^{3/2} $}
\newcommand{\trentedeux}{A$^{3/2} $}
\newcommand{\onze}{X$^{1/2} $}
\newcommand{\douze}{A$^{1/2} $}
\newcommand{\treize}{\textbf{OnAEnleverCetteSolution}}

\newcommand{\TransiUn}{\onze-\trenteetun}
\newcommand{\TransiDeux}{\trentedeux-\trenteetun}
\newcommand{\TransiTrois}{\douze-\trenteetun}
\newcommand{\TransiQuatre}{\treize-\trenteetun}

\newcommand{\fullip}{full-IP}

\newcommand{\dyallavtz}{dyall-av3$\zeta$}
\newcommand{\dyallacvtz}{dyall-acv3$\zeta$}
\newcommand{\inftycorr}{$[\infty \SI{}{\hartree}]$}
\newcommand{\corrlim}[2]{$\left[\num{#1};\ \num{#2} \right]$\SI{}{\hartree}}




% \usepackage{xcolor}
% \usepackage{rancy}

\newcommand{\revAG}[1]{{\textcolor{blue}{ASPG: #1}}}
\newcommand{\revXY}[1]{{\textcolor{magenta}{XY: #1}}}
\newcommand{\revLH}[1]{{\textcolor{green}{LH: #1}}}
\newcommand{\revLV}[1]{{\textcolor{cyan}{LV: #1}}}
\newcommand{\revSP}[1]{{\textcolor{pink}{SP: #1}}}



%%%%%%%%%%%%%%%%%%%%%%%%%%%%%%%%%%%%%%%%%%%%%%%%%%%%%%%%%%%%%%%%%%%%%
%% If issues arise when submitting your manuscript, you may want to
%% un-comment the next line.  This provides information on the
%% version of every file you have used.
%%%%%%%%%%%%%%%%%%%%%%%%%%%%%%%%%%%%%%%%%%%%%%%%%%%%%%%%%%%%%%%%%%%%%
%%\listfiles

%%%%%%%%%%%%%%%%%%%%%%%%%%%%%%%%%%%%%%%%%%%%%%%%%%%%%%%%%%%%%%%%%%%%%
%% Place any additional macros here.  Please use \newcommand* where
%% possible, and avoid layout-changing macros (which are not used
%% when typesetting).
%%%%%%%%%%%%%%%%%%%%%%%%%%%%%%%%%%%%%%%%%%%%%%%%%%%%%%%%%%%%%%%%%%%%%
\newcommand*\mycommand[1]{\texttt{\emph{#1}}}

%%%%%%%%%%%%%%%%%%%%%%%%%%%%%%%%%%%%%%%%%%%%%%%%%%%%%%%%%%%%%%%%%%%%%
%% Meta-data block
%% ---------------
%% Each author should be given as a separate \author command.
%%
%% Corresponding authors should have an e-mail given after the author
%% name as an \email command. Phone and fax numbers can be given
%% using \phone and \fax, respectively; this information is optional.
%%
%% The affiliation of authors is given after the authors; each
%% \affiliation command applies to all preceding authors not already
%% assigned an affiliation.
%%
%% The affiliation takes an option argument for the short name.  This
%% will typically be something like "University of Somewhere".
%%
%% The \altaffiliation macro should be used for new address, etc.
%% On the other hand, \alsoaffiliation is used on a per author basis
%% when authors are associated with multiple institutions.
%%%%%%%%%%%%%%%%%%%%%%%%%%%%%%%%%%%%%%%%%%%%%%%%%%%%%%%%%%%%%%%%%%%%%
\author{Xiang Yuan}
% \altaffiliation{A shared footnote}
\affiliation[PhLAM]{Univ. Lille, CNRS, UMR 8523 - PhLAM - Physique des Lasers Atomes et Molécules, F-59000 Lille, France}
\email{xiang.yuan@univ-lille.fr}\alsoaffiliation[Vrije Universiteit Amsterdam]
{Department of Chemistry and Pharmaceutical Sciences, Faculty of Science, Vrije Universiteit Amsterdam, 1081 HV Amsterdam, The Netherlands}


\author{Loïc Halbert}
% \altaffiliation{A shared footnote}
% \phone{+123 (0)123 4445556}
% \fax{+123 (0)123 4445557}
\email{loic.halbert@univ-lille.fr}
\affiliation[PhLAM]{Univ. Lille, CNRS, UMR 8523 - PhLAM - Physique des Lasers Atomes et Molécules, F-59000 Lille, France}

\author{Johann Pototschnig}
\affiliation[Vrije Universiteit Amsterdam]
{Department of Chemistry and Pharmaceutical Sciences, Faculty of Science, Vrije Universiteit Amsterdam, 1081 HV Amsterdam, The Netherlands}

\author{Anastasios Papadopoulos}
% \altaffiliation{A shared footnote}
\affiliation[Vrije Universiteit Amsterdam]
{Department of Chemistry and Pharmaceutical Sciences, Faculty of Science, Vrije Universiteit Amsterdam, 1081 HV Amsterdam, The Netherlands}

\author{Sonia Coriani}
\email{soco@kemi.dtu.dk}
\affiliation{DTU Chemistry -- Department of Chemistry, Technical University of Denmark, DK-2800 Kongens Lyngby, Denmark}

\author{Lucas Visscher}
\email{l.visscher@ vu.nl}
\affiliation[Vrije Universiteit Amsterdam]
{Department of Chemistry and Pharmaceutical Sciences, Faculty of Science, Vrije Universiteit Amsterdam, 1081 HV Amsterdam, The Netherlands}

\author{André Severo Pereira Gomes}
\email{andre.gomes@univ-lille.fr}
\affiliation[PhLAM]{Univ. Lille, CNRS, UMR 8523 - PhLAM - Physique des Lasers Atomes et Molécules, F-59000 Lille, France}

%%%%%%%%%%%%%%%%%%%%%%%%%%%%%%%%%%%%%%%%%%%%%%%%%%%%%%%%%%%%%%%%%%%%%
%% The document title should be given as usual. Some journals require
%% a running title from the author: this should be supplied as an
%% optional argument to \title.
%%%%%%%%%%%%%%%%%%%%%%%%%%%%%%%%%%%%%%%%%%%%%%%%%%%%%%%%%%%%%%%%%%%%%
\title[An \textsf{achemso} demo]
  {Supplementary information: Formulation and Implementation of frequency-dependent linear response properties with Relativistic Coupled-Cluster theory for GPU-accelerated computer architectures}

%%%%%%%%%%%%%%%%%%%%%%%%%%%%%%%%%%%%%%%%%%%%%%%%%%%%%%%%%%%%%%%%%%%%%
%% Some journals require a list of abbreviations or keywords to be
%% supplied. These should be set up here, and will be printed after
%% the title and author information, if needed.
%%%%%%%%%%%%%%%%%%%%%%%%%%%%%%%%%%%%%%%%%%%%%%%%%%%%%%%%%%%%%%%%%%%%%
\abbreviations{IR,NMR,UV}
\keywords{American Chemical Society, \LaTeX}

%%%%%%%%%%%%%%%%%%%%%%%%%%%%%%%%%%%%%%%%%%%%%%%%%%%%%%%%%%%%%%%%%%%%%
%% The manuscript does not need to include \maketitle, which is
%% executed automatically.
%%%%%%%%%%%%%%%%%%%%%%%%%%%%%%%%%%%%%%%%%%%%%%%%%%%%%%%%%%%%%%%%%%%%%
\begin{document}

\section{Working equations for CCSD linear response}

In what follows $a,b,c,.. $ will indicate {particle lines}, $ i,j,k,... $ {hole lines}, and $ p,q,r,s,... $ general indexes~\cite{crawford2007introduction}. In all equations below we use Einstein notation. Furthermore we define 
\begin{itemize}
    \item $P$ as a permutation operator, with : $P_{-pq} f\left(\dots pq \dots\right)= f\left(\dots pq \dots\right) - f\left(\dots qp \dots\right)$;
    \item $X_{q}^{p} = \bra{p} X \ket{q}$ are matrix elements of property operator $X$ ;
    \item $V_{rs}^{pq} = \bra{pq}\ket{rs}$ are antisymmetrized two-electron integrals, and $f_{q}^{p} = \bra{p} f \ket{q}$ Fock matrix elements;
    \item $\boldsymbol{\lambda}$ denotes ground-state CC Lagrange multipliers, and is therefore equivalent to $\bar{\mathbf{t}}^{(0)}$, and we have $\bra{\Lambda} = \bra{R} + \sum_\mu \lambda_\mu \bra{{\mu}}e^{-T_0} = \bra{R} + \sum_\mu \lambda_\mu \bra{\bar{\mu}} \equiv \bra{R} + \sum_\mu \bar{t}^{(0)}_\mu \bra{\bar{\mu}}$
    \item $\mathbf{r}$ and $\mathbf{l}$ denote, depending on context, (trial) vectors associated to the solution of right and left-hand EOMCC or response equations.
\end{itemize}


\subsection{Linear response}

%\subsubsection{Vector $\boldsymbol{\xi}^{X}$}

\noindent The $\boldsymbol{\xi}^{X}$ vector is defined as\cite{Christiansen1998}
\begin{align}
\xi^{X}_{\mu} = \bra{\mu}X\ketCC \label{xi-equation}
\end{align}
and the programmable expressions for its elements are given by:
\begin{align}
    \xi_{i}^{a} &= + X_{i}^{a} + X_{e}^{a}t_{i}^{e}- X_{i}^{m}t_{m}^{a} - (X_{e}^{m}t_{i}^{e})t_{m}^{a} + X_{e}^{m}t_{im}^{ae}  \\
%
    \xi_{ij}^{ab}&= + P_{-ab}X_{f}^{b}t_{ij}^{af} - P_{-ij}X_{j}^{m}t_{im}^{ab}-P_{-ij}(X_{e}^{m}t_{i}^{e})t_{mj}^{ab} - P_{-ab}(X_{f}^{m}t_{m}^{a})t_{ij}^{fb}
\end{align}

%\subsubsection{Vector $\boldsymbol{\eta}^{X}$}

\noindent The $\boldsymbol{\eta}^{X}$ vector is defined as~\cite{Christiansen1998}
\begin{align}
\eta_\mu^{Y}&=\bra{\Lambda}[Y,\hat{\tau}_{\mu}]\ketCC \label{eta-equation}
\end{align}
and the programmable expressions for its elements are given by
\begin{align}
        \eta_{i}^{a}=& +X_{i}^{a}+X_{e}^{a}\lambda_{i}^{e}-X_{i}^{m}\lambda_{m}^{a}-X_{m}^{a}(t_{e}^{m}\lambda_{i}^{e})-(X_{i}^{e}t_{e}^{m})\lambda_{m}^{a}-\frac{1}{2}(t_{fe}^{mn}\lambda_{mi}^{fe})X_{n}^{a}\nonumber \\
        &-\frac{1}{2}(t_{fe}^{nm}\lambda_{nm}^{fa})X_{i}^{e} \\
%
%%
%%
% \begin{align}
%         ^{EOM}\eta_{i}^{a}=&X_{i}^{a}+X_{e}^{a}\lambda_{i}^{e}-X_{i}^{m}\lambda_{m}^{a}-X_{m}^{a}(t_{e}^{m}\lambda_{i}^{e})-(X_{i}^{e}t_{e}^{m})\lambda_{m}^{a}+(t_{m}^{f}X_{f}^{e})\lambda_{ie}^{am}-(t_{n}^{e}X_{m}^{n})\lambda_{ie}^{am}+\lambda_{ie}^{am}X_{m}^{e}\nonumber\\
%         &-\frac{1}{2}(t_{fe}^{mn}\lambda_{mi}^{fe})X_{n}^{a}-\frac{1}{2}(t_{fe}^{nm}\lambda_{nm}^{fa})X_{i}^{e}+(t_{mn}^{ef}X_{f}^{n})\lambda_{ie}^{am}-(t_{n}^{e}(t_{m}^{f}X_{f}^{n}))\lambda_{ie}^{am}
% \end{align} 
%%
%%
%%\end{align}
%%
%%\begin{align}
        \eta_{ij}^{ab}=&+P_{-ab}P_{-ij}\lambda_{i}^{a}X_{j}^{b}-P_{-ij}\lambda_{im}^{ab}X_{j}^{m}+P_{-ab}\lambda_{ij}^{ae}X_{e}^{b}-P_{-ij}(t_{m}^{e}X_{j}^{e})\lambda_{im}^{ab}\nonumber\\
        &-P_{-ab}(t_{m}^{e}X_{m}^{b})\lambda_{ij}^{ae}
\end{align}
whereas the elements of the $^{\textrm{EOM}}\boldsymbol{\eta}$ vector~\cite{faber2018resonant} are
\begin{align}
    ^{\textrm{EOM}}\eta_{i}^{a} &= +\eta_{i}^{a} +  \lambda_{ij}^{ab}\xi_{j}^{b} %\\
%    ^{\textrm{EOM}}\eta_{ij}^{ab} &= +\eta_{ij}^{ab} 
\end{align}


%\subsubsection{Hessian matrix and its contraction with response vectors}

\noindent The CC Hessian ($\mathbf{F}$) is defined as\cite{Christiansen1998}
\begin{align}
  F_{\mu \nu}=&\left< \Lambda \left | \left[ \left[ H_0,\hat{\tau}_{\mu} \right],\hat{\tau}_{\nu} \right] \right|\mathrm{CC}\right>
\end{align}
and the programmable expressions for the matrix elements for its contraction with a response vector $\tXs{}{}$, $\left( \tXs {}{} F \right) $ are given by
 \begin{align}
 \displaybreak[1]
\left( \tXs {}{} F \right)^{k}_{c} = & +    \V kica . \tXs ai  -\lambdas ic  \F ka . \tXs ai  -\lambdas ka  \F ic . \tXs ai  + \lambdas ke   \W eica . \tXs ai  
- \lambdas mc \W kima . \tXs ai -  \lambdas ma  \W ikmc . \tXs ai + \lambdas ie \W ekac . \tXs ai  \nonumber\\
&
-  \undemi \lambdas ic \V kjab . \tXd abij   - \undemi \lambdas ka   \V ijcb . \tXd abij + \lambdas ia  \V jkbc . \tXd abij   -  \lambdad mkae    \W iemc . \tXs ai - \lambdad imec  \W ekam .  \tXs ai  + \undemi \lambdad kief \W efca . \tXs ai  \nonumber \\
&+ \undemi \lambdad mnca  \W kimn . \tXs ai 
%\displaybreak[1]
  - \undemi  \left( \lambdad mnec \td efmn \right) \V kifa   \tXs ai  - \undemi \left(\lambdad mkef \td efmn \right) \V nica    . \tXs ai  - \undemi  \left(\lambdad mnea \td efmn \right)  \V ikfc    \tXs ai \nonumber \\
%\displaybreak[1]
& - \undemi \left(\lambdad mief \td efmn \right)  \V nkac  . \tXs ai  
- \undemi \lambdad ijac \F kb . \tXd abij  - \undemi \lambdad ikab  \F jc . \tXd abij  
-  \lambdad ikeb   \W ejac . \tXd abij  + \lambdad mjac   \W ikmb . \tXd abij  \nonumber \\
& -\unquart \lambdad ijec   \W ekab . \tXd abij  + \unquart \lambdad mkab   \W ijmc . \tXd  abij  + \undemi \lambdad ijae  \W ekbc . \tXd abij  - \undemi \lambdad imab  \W jkmc . \tXd abij  \\
\displaybreak[1]
&\nonumber\\
\left( \tXs {}{} F \right)^{kl}_{cd} =&+ \Pmm cd \Pmm kl \lambdas kc   \V lida . \tXs ai - \Pmm kl \lambdas ka  \V ilcd . \tXs ai  - \Pmm cd\lambdas ic  \V klad . \tXs ai  -\Pmm kl \lambdad kicd \F la . \tXs ai  - \Pmm cd  \lambdad klca  \F id . \tXs ai \nonumber \\
& + \Pmm cd \lambdad klce  \W eida . \tXs ai  - \Pmm kl \lambdad kmcd  \W lima . \tXs ai 
 + \Pmm cd \Pmm kl \lambdad mlca    \W kimd . \tXs ai  + \lambdad micd  \W klma . \tXs ai  \nonumber \\
 &-  \Pmm cd \Pmm kl \lambdad kied  \W elca . \tXs ai  - \lambdad klea \W eicd . \tXs ai    -\undemi \Pmm cd \lambdad klca  \V ijdb . \tXd abij   -\undemi \Pmm kl\lambdad kicd  \V ljab . \tXd abij  +\unquart \lambdad ijcd  \V klab . \tXd abij \nonumber \\
 & + \unquart \lambdad klab  \V ijcd  . \tXd abij  +\Pmm kl \Pmm cd \lambdad kjcb  \V lida . \tXd abij  -\undemi \Pmm kl \lambdad ikab \V jlcd . \tXd abij -\undemi \Pmm cd \lambdad ijac \V klbd . \tXd abij 
\end{align}


\subsection{EOM-EE $\sigma$-Vectors and intermediates }

% \noindent
% \begin{equation}
% % \begin{split}
% \label{sigma_sin-ee}
%     \sigma_{i}^{a} = \sum_{e}\bar{F}_{e}^{a}r_{i}^{e}-\sum_{m}F_{i}^{m}r_{m}^{a}+\sum_{me}\bar{F}_{e}^{m}r_{mi}^{ea}+\sum_{me}W_{ei}^{ma}r_{m}^{e} + \frac{1}{2}\sum_{m,ef}W_{ef}^{am}r_{im}^{ef}+\frac{1}{2}\sum_{mn,e}W_{ie}^{mn}r_{mn}^{ea}
% % \end{split}
% \end{equation}

The programmable expressions for the elements for the EOM-EE right $^{R}\sigma$ and left $^{L}\sigma$ vectors are given by:
\begin{align}
%\displaybreak
    ^{R}\sigma_{i}^{a} = &     \F ae r_{i}^{e}    -\F mi r_{m}^{a}+   \F me r_{mi}^{ea}    +\W ei ma  r_{m}^{e}     +\frac{1}{2} \W am ef r_{im}^{ef}
    +\frac{1}{2} \W mn ie r_{mn}^{ea} \label{sigma_sin-ee} \\
    ^{R}\sigma_{ij}^{ab} = &- \Pmm ab  \W mbij \rs am +\Pmm ij \W abej \rs ei + \Pmm ab (\W bmfe \rs em )\td afij -P_{-ij}(\W nmje \rs em )\td abin \nonumber  \\
    &+P_{-ab}\F be \rd aeij - \Pmm ij \F mj \rd abim +\frac{1}{2} \W mnij \rd abmn  + \Pmm ab \Pmm ij \W mbej \rd aeim \nonumber  \\
    &-\Pmm ab \frac{1}{2}(\V nmfe \rd eamn )\td fbij -\Pmm ij \frac{1}{2}(\V nmfe \rd feim )\td bajn +\frac{1}{2} \W abef \rd efij \label{sigma_double-ee}\\
    &\nonumber \\
     ^{L}\sigma_{a}^{i} =& \ls ie \F ea - \ls ma \F im + \frac{1}{2} \ld imef \W efam -\undemi \ld mnae \W iemn  - G_{e}^{f} \W eifa -  G_{m}^{n} \W mina +\ls me \W ieam \label{sigma_sin-ee-left} \\
    ^{L}\sigma_{ab}^{ij}=& \Pmm ab \ld ijae \F eb -\Pmm ij \ld imab \F jm + \undemi \ld mnab \W ijmn + \Pmm ij \Pmm ab \ld imae \W jebm + \Pmm ab \V ijae G_{b}^{e} - \ls ma \W ijmb  \nonumber \\
    & -\Pmm ij \V imab  G_{m}^{j} + \Pmm ij \ls ie \V ejab + \Pmm ij \Pmm ab \ls ia \F jb + \undemi \ld ijef \W efam \label{sigma_double-ee-left}
 \end{align}

\clearpage%\subsection{Intermediates}
%
The programmable expressions for the elements of the intermediates $\F {}{},\ \W{}{}{}{} $  used above are given by:
\begin{align}
 	\F im = & f_{m}^{i} +f_{e}^{i}t_{m}^{e}+V_{me}^{in}t_{n}^{e}+\frac{1}{2}V_{ef}^{in}\tau_{mn}^{ef} 	\label{F_im} \\
	\F ae = & f_{e}^{a} -f_{a}^{m}t_{m}^{e}+V_{fa}^{me}t_{m}^{f}-\frac{1}{2}V_{af}^{mn}\tau_{mn}^{ef} 	\label{F_ea}\\
	\F me = & f_{e}^{m} + V_{ef}^{mn}t_{n}^{f}	\label{F_me} \\
	\W ijmn = & V_{mn}^{ij} + P_{-mn}V_{en}^{ij}t_{m}^{e}+\frac{1}{2}V_{ef}^{ij}\tau_{mn}^{ef} 	\label{W_ijmn} \\
	%\displaybreak
    \W mbej = & V_{ej}^{mb} + V_{ef}^{mb}t_{j}^{f}-V_{ej}^{mn}t_{n}^{b}-V_{ef}^{mn}(t_{jn}^{fb}+t_{j}^{f}t_{n}^{b})
	\label{W_mbej}\\
	\W iemn = & V_{mn}^{ie} + \F if t_{mn}^{ef}- \W iomn t_{o}^{e}+\frac{1}{2}V_{fg}^{ie}\tau_{mn}^{fg} \nonumber \\
	&+P_{-mn}\bar{W}_{fn}^{ie}t_{m}^{f}+P_{-mn}V_{mf}^{io}t_{no}^{ef} 	\label{W_iemn} \\
	\W efam = &V_{am}^{ef} + P_{-ef}V_{ag}^{en}t_{mn}^{gf} \nonumber\\
	&+\W efag t_{m}^{g}+\F na t_{mn}^{ef}+\frac{1}{2}V_{am}^{no}\tau_{no}^{ef}-P_{-ef}\bar{W}_{am}^{nf}t_{n}^{e}
	\label{W_efam}\\
  %\displaybreak
	\W efab = &V_{ab}^{ef} - P_{-ef}V_{ab}^{mf}t_{m}^{e}+\frac{1}{2}V_{ab}^{mn}\tau_{mn}^{ef}
	\label{W_efab}\\
	\bar{W}_{ej}^{mb} =& V_{ej}^{mb}-V_{ef}^{mn}t_{nj}^{bf}
	\label{barW_mbej}\\
	\W mnie =& V_{ie}^{mn} + t_{i}^{f}V_{fe}^{mn}
    \label{W_mnie}\\
	\W amef = & V_{ef}^{am} - V_{ef}^{nm}t_{n}^{a}
	\label{W_amef}\\
	G_{a}^{e} = & -\frac{1}{2}l_{af}^{mn}t_{mn}^{ef}
	\label{G_ea}\\
	G_{m}^{i} = &\frac{1}{2}l_{ef}^{in}t_{mn}^{ef}
	\label{G_mi}\\
	\tau^{ab}_{ij} =& \td abij + \undemi \Pmm ab \Pmm ij \ts ai \ts bj
	\label{tau}
\end{align}

\clearpage

\section{Comparison between LRCC with DIRAC and Dalton}

% Figure environment removed


\clearpage


% \begin{table}[H]
%     \begin{threeparttable}
            
%         \center
%         \caption{Nonrelativistic calculation on Isotropic and anisotropic spin-spin coupling J(Hz) for HF, HCl, and HBr }\label{tab:spsp}

%     \setlength{\tabcolsep}{3.5mm}{
%         \begin{tabular}{cccccc}
%              &  HF-LR$_{iso}$ & B3LYP-LR$_{iso}$& CC-CI-LR$_{iso}$ &CC-CC-LR$_{iso}$ &CFOUR$_{iso}$     \\
%             \hline
%             $^{1}$HF$^{19}$ & 560.2324&377.5294 & 458.55 &457.6892 &457.6715  \\
%             $^{1}$HCl$^{35}$ &33.1722 &23.2295 &36.9018 &36.6414 &36.6414 \\
%             $^{1}$HBr$^{79}$ &32.6909 &-5.6679 &92.9897 &90.5146 &90.5144\\
%             % H$_{2}$O$^{17}$\tnote{a} &-87.6031 &-64.1766 &-68.7492 &-68.8131 &-68.8869\\
%             \hline
%              &  HF-LR$_{aniso}$ & B3LYP-LR$_{aniso}$ & CC-CI-LR$_{iso}$ & CC-CC-LR$_{aniso}$ &     \\
%             \hline
%             $^{1}$HF$^{19}$ & 28.8305&71.7803 & -42.4747&-39.4953 &\\
%             $^{1}$HCl$^{35}$ &70.3046 &59.0412 &42.8696 &43.7373 &\\
%             $^{1}$HBr$^{79}$ &489.1606 &393.8655 &304.9419 &310.8541 &\\
%             % $^{1}$H$_{2}$O$^{17}$\tnote{a} &23.5821 &13.5156 &23.1081 &22.9865 &\\
%             \hline
%         \end{tabular}}

%         % \begin{tablenotes}
%         %     \item[a]  H-O contribution
%         % \end{tablenotes}
%         \end{threeparttable}
% \end{table}



    % \begin{table}[H]
    % \begin{threeparttable}
        

    %     \centering
    %     \caption{Isotropic spin-spin constant of $^{1}$HF$^{19}$, HCl$^{35}$, HBr$^{79}$ and the corresponding contribution of DSO, PSO, SD, and FC in nonrelativistic Hartree-Fock and Coupled-Cluster linear response }
    %     \setlength{\tabcolsep}{6.0mm}{
    %     \begin{tabular}{cccccc}
    %         & FC & SD  & PSO & DSO & Total \\
    %     \hline
    %         \multicolumn{6}{c}{$^{1}$HF$^{19}$}\\
    %     \hline
    %     HF\tnote{a}&386.5258  &-9.3722 &183.4624  &0.4917 &561.1076  \\
    %     CC-CC\tnote{b}&282.9942  &0.8801 &173.2108  &0.5863 &457.6715  \\
    %     \hline
    %         \multicolumn{6}{c}{$^{1}$HCl$^{35}$}\\
    %     \hline
    %     HF\tnote{a}&19.4153  &-0.1216 &13.9246  &-0.0014 &33.2169  \\
    %     CC-CC\tnote{b}&21.9502  &0.5894 &14.0962  &0.0055 &36.6415  \\
    %     \hline
    %         \multicolumn{6}{c}{$^{1}$HBr$^{79}$}\\
    %     \hline
    %     HF\tnote{a}&-22.6955  &-2.2799  &57.6236  &-0.0146 &32.6336  \\
    %     CC-CC\tnote{b}&29.0662  &1.6774 &59.7715  &-0.0007 &90.5144  \\
    %     \hline
    %     \end{tabular}}

    %     \begin{tablenotes}
    %         \item[a] Nonrelativistic Hartree-Fock linear response calculated by DALTON program
    %         \item[b] Nonrelativistic Coupled-Cluster linear response calculated by CFOUR program
    %     \end{tablenotes}
    %     \label{tab:solvent of h2se}
    %     \end{threeparttable}
    % \end{table}



    % \begin{table}[]
    %     \centering
    %     \caption{The excitation energy of the first triplet excited state for $^{1}$HF$^{19}$, $^{1}$HCl$^{35}$, and $^{1}$HBr$^{79}$}
    %     \setlength{\tabcolsep}{7.0mm}{
    %     \begin{tabular}{cccc}
    %          &$^{1}$HF$^{19}$ &$^{1}$HCl$^{35}$ &$^{1}$HBr$^{79}$ \\
    %         \hline
    %         HF     &0.4004  &0.2719 &0.2281 \\
    %         CC-CC  &0.3648  &0.2706 &0.2308 \\
    %         \hline
    %     \end{tabular}}
        
    %     \label{tab:my_label}
    % \end{table}

\section{Indirect spin-spin couplings}

We present here the results for the Indirect spin-spin coupling $J$ for the hydrogen halides.

        \begin{table}[H]
        \begin{threeparttable}
            
            \center
            \caption{Isotropic and anisotropic spin-spin coupling J(Hz) for HX(X=F, Cl, Br, I)}\label{tab:spsp}

        \setlength{\tabcolsep}{8.0mm}{
            \begin{tabular}{ccccc}
            Models&  $^{1}$HF$^{19}$ &$^{1}$HCl$^{35}$ &$^{1}$HBr$^{79}$ &$^{1}$HI$^{127}$ \\
            \hline
             \multicolumn{5}{c}{Isotropic}\\
            \hline
            NR-HF     &560.2324 &33.1722 &32.6909 &-2.1729\\
            NR-B3LYP  &377.5294 &23.2295 &-5.6679 &\\
            NR-CC-CI  &458.5500 &36.9018 &92.9897 &\\
            NR-CC-CC  &457.6892 &36.6414 &90.5146 &\\
            NR-CC-CC\tnote{a}&457.6715 &36.6414 &90.5144 &\\
            X2C-HF    &559.7091 &32.0803 &-13.6915 &-201.2263\\
            X2C-B3LYP &375.7986 &22.3179 &-35.3065 &-138.7448\\
            X2C-CC-CI &457.7350 &36.4102 &71.9471 &8.4426\\
            X2C-CC-CC &456.8450 &36.1321 &68.7287 &1.8104\\
            DC-HF     &559.3715 &31.9899 &-14.6151 &-203.2970\\
            \hline
            \multicolumn{5}{c}{Anisotropic}\\
            \hline
            NR-HF     &28.8305 &70.3046 &489.1606 &672.0842\\
            NR-B3LYP  &71.7803 &59.0412 &393.8655 &\\
            NR-CC-CI  &-42.4747 &42.8696 &304.9419 &\\
            NR-CC-CC  &-39.4953 &43.7373 &310.8541 &\\
            X2C-HF    &29.2373 &70.9774 &508.9765 &739.8355\\
            X2C-B3LYP &72.9026 &59.3847 &394.2909 &486.5593\\
            X2C-CC-CI &-41.3592 &43.4599 &322.0869 &466.6810\\
            X2C-CC-CC &-38.3627 &44.3369 &328.3281 &475.1557\\
            DC-HF     &29.3720 &71.0300 &509.5169 &740.8219\\
            \hline
            
            \end{tabular}}

            \begin{tablenotes}
                \item[a] Calculations were performed using the CFOUR program
            \end{tablenotes}

        \end{threeparttable}
        \end{table}

        \clearpage

        \begin{table}[H]
          \center
            \caption{Difference in the linear response contribution to the spin-spin coupling constant $K_{HX}$ between CC-CI and CC-CC ($\Delta K^\text{LR} = K^\text{LR}(\text{CC-CI}) - K^\text{LR}(\text{CC-CC})$, in a.\ u.) for the HX(X=F, Cl, Br, I) systems, broken down into the $xx, yy, zz$ components. Apart from the absolute values, we also provide the difference per correlated electron ($\Delta K^\text{LR}_e = \Delta K^\text{LR} / N_e$, with $N_e = 10, 18, 36$ and $54$ across the series)}\label{tab:spsp-ccci-vs-cccc}

        \setlength{\tabcolsep}{6.0mm}{
            \begin{tabular}{cccccc}
                  & &             &\multicolumn{3}{c}{components} \\
\cline{4-6}
         &   System & Hamiltonian & ${zz}$ & ${xx}$  & ${yy}$ \\
            \hline
$\Delta K^\text{LR}$
&	HF	&	NR &	-0.0277	&	0.0456	&	0.0456	\\
&		&	X2C	&	-0.0273	&	0.0465	&	0.0465	\\
&	HCl	&	NR	&	-0.0751	&	0.1298	&	0.1298	\\
&		&	X2C	&	-0.0724	&	0.1347	&	0.1347	\\
&	HBr	&	NR	&	-0.1351	&	0.4097	&	0.4097	\\
&		&	X2C	&	-0.0860	&	0.4834	&	0.4834	\\
&	HI	&	X2C	&	0.1130	&	1.0875	&	1.0875	\\            \hline
$\Delta K^\text{LR}_e$ 
&	HF	&	NR	&	-0.0028	&	0.0046	&	0.0046	\\
&		&	X2C	&	-0.0027	&	0.0046	&	0.0046	\\
&	HCl	&	NR	&	-0.0042	&	0.0072	&	0.0072	\\
&		&	X2C	&	-0.0040	&	0.0075	&	0.0075	\\
&	HBr	&	NR	&	-0.0038	&	0.0114	&	0.0114	\\
&		&	X2C	&	-0.0024	&	0.0134	&	0.0134	\\
&	HI	&	X2C	&	0.0021	&	0.0201	&	0.0201	\\
\hline           
            \end{tabular}}

        \end{table}

        \clearpage
        
        
\section{Optical rotation}


\begin{table}[H]
            \center
            \caption{Optical rotation Test of H$_{2}$S$_{2}$ using uncontracted basis set}\label{tab:H2S2opt}

        \setlength{\tabcolsep}{7.5mm}{
            \begin{tabular}{ccc}
            Method    & G Tensor & Optical rotation\\
            \hline
            CC(X2C) & -0.10732 &-271.1273\\
            CC(X2C, virtual to 100 a.u.) & -0.10757&-271.7746 \\
            CC(LEVY-LEBLOND) &-0.10029&-253.3778\\
            CC(DALTON) &-0.10029&-253.3856\\
            \hline
            \end{tabular}}
\end{table}




        \begin{table}[H]
            \center
            \caption{Optical rotation (a.u.) of Hydrogen peroxide series (H$_{2}$Y$_{2}$) with a frequency corresponding to the sodium D-line (589.29 nm, 0.077319 a.u.) calculated with X2C and LEVY-LEBLOND Hamiltonian}\label{tab:opt-rot-h2y2}

        \setlength{\tabcolsep}{5.5mm}{
            \begin{tabular}{ccccc}
            Method    & H$_{2}$O$_{2}$ & H$_{2}$S$_{2}$& H$_{2}$Se$_{2}$ & H$_{2}$Te$_{2}$\\
            \hline
            HF(LEVY-LEBLOND)    &-93.2588  &-124.5926 &-205.8141  &-93.6808 \\
            HF(X2C)             &-93.0240  &-136.7733 &-22.3730   &2007.9020 \\
            B3LYP(LEVY-LEBLOND) &-172.7622 &-295.4117 &-1098.4716 &-18269 \\
            B3LYP(X2C)          &-173.8253 &-320.9461 &-1418.7842 &4075 \\
            CC(LEVY-LEBLOND)    &-181.5585 &-253.3779 &-386.6950  &-263.5968 \\
            CC(X2C)             &-182.4976 &-271.1273 &-1906.1708 &218.8335 \\
            \hline
            \end{tabular}}
        \end{table}


\clearpage

\bibliography{CC_Lin_Rsp}
        
\end{document}
