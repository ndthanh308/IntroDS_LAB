%%%%%%%%%%%%%%%%%%%%%%%%%%%%%%%%%%%%%%%%%%%%%%%%%%%%%%%%%%%%%%%%%%%%%
%% This is a (brief) model paper using the achemso class
%% The document class accepts keyval options, which should include
%% the target journal and optionally the manuscript type. 
%%%%%%%%%%%%%%%%%%%%%%%%%%%%%%%%%%%%%%%%%%%%%%%%%%%%%%%%%%%%%%%%%%%%%
\documentclass[journal=jctcce,manuscript=article]{achemso}

%%%%%%%%%%%%%%%%%%%%%%%%%%%%%%%%%%%%%%%%%%%%%%%%%%%%%%%%%%%%%%%%%%%%%
%% Place any additional packages needed here.  Only include packages
%% which are essential, to avoid problems later. Do NOT use any
%% packages which require e-TeX (for example etoolbox): the e-TeX
%% extensions are not currently available on the ACS conversion
%% servers.
%%%%%%%%%%%%%%%%%%%%%%%%%%%%%%%%%%%%%%%%%%%%%%%%%%%%%%%%%%%%%%%%%%%%%
\usepackage[version=3]{mhchem} % Formula subscripts using \ce{}
\usepackage{braket}
\usepackage{mathtools}
\usepackage{amsthm, amssymb, amsfonts}
\usepackage{cancel}
\usepackage{array}
\usepackage{listings}
\usepackage{float}
\usepackage{bm}
\usepackage{graphicx}
\usepackage{threeparttable}
\usepackage[hidelinks]{hyperref}


% \usepackage{graphicx}
% \usepackage{multicol}
% \usepackage{multirow}

% \usepackage{simplewick}
% \usepackage{subcaption}
% \usepackage{wrapfig}

% \usepackage{fancyhdr}

% \usepackage{afterpage}
% \usetikzlibrary{fadings}

% \usepackage{doi}

% \usepackage{hyperref}

% \usepackage{babel}

% \usepackage[left=2cm,right=2cm,top=2.0cm,bottom=2.0cm]{geometry}
% \usepackage[utf8]{inputenc}


\setlength{\unitlength}{1cm}

\usepackage{tikz}

%%%%%%%%%%%% %%%%%%%%%%%% %%%%%%%%%%%% %%%%%%%%%%%%
%%% sucking up Ecriture_EOM.tex here
%%%\input{Ecriture_EOM.tex}


\newcommand{\finfo}[2]{f_{#1#2}}
\newcommand{\f}[2]{f^{#1}_{#2}}
\newcommand{\fetOperateur}[2]{f^{#1}_{#2}\left\lbrace \adag #1 
 \aoper #2  \right\rbrace}
\newcommand{\Vinfo}[4]{\left\langle #1#2 \right| \left| #3#4 \right\rangle }
\newcommand{\Vinfochem}[4]{\left(  #1#2 \right| \! \left| #3#4 \right) }
\newcommand{\Vinfochemspin}[4]{\left[  #1#2 \right| \! \left| #3#4 \right]_{spin} }
\newcommand{\V}[4]{V^{#1#2}_{#3#4}}
\newcommand{\VetOperateur}[4]{V^{#1#2}_{#3#4}\left\lbrace \adag #1 
\adag #2 \aoper #4 \aoper #3 \right\rbrace }

\newcommand{\Lr}{\mathcal{L}}
\newcommand{\Rr}{\mathcal{R}}
\newcommand{\Gr}{\mathcal{G}}
\newcommand{\Hcurl}{\mathcal{H}}
\newcommand{\continuum}{\mathbf{c}}


\newcommand{\Finfo}[4]{\mathcal{F}_{#1#2}[#3,#4]}
\newcommand{\Ftinfo}[4]{\tilde{\mathcal{F}}_{#1#2}[#3,#4]}
\newcommand{\Winfo}[8]{\mathcal{W}_{#1#2#3#4}[#5,#6,#7,#8]}
\newcommand{\Wtinfo}[8]{\tilde{\mathcal{W}}_{#1#2#3#4}[#5,#6,#7,#8]}
\newcommand{\Ginfo}[2]{\mathcal{G}_{#1#2}}
\newcommand{\Gtinfo}[2]{\tilde{\mathcal{G}}_{#1#2}}
\newcommand{\taupinfo}[4]{\tau_{#1#2#3#4}}
\newcommand{\tauptinfo}[4]{\tilde{\tau}_{#1#2#3#4}}

\newcommand{\F}[2]{\mathcal{F}^{#1}_{#2}}
\newcommand{\Ft}[2]{\tilde{\mathcal{F}}^{#1}_{#2}}
\newcommand{\Fb}[2]{\bar{\mathcal{F}}^{#1}_{#2}}
\newcommand{\W}[4]{\mathcal{W}^{#1#2}_{#3#4}}
\newcommand{\Wt}[4]{\tilde{\mathcal{W}}^{#1#2}_{#3#4}}
\newcommand{\Wb}[4]{\bar{\mathcal{W}}^{#1#2}_{#3#4}}
%\newcommand{\G}[2]{\mathcal{G}^{#1}_{#2}}
\newcommand{\Gt}[2]{\tilde{\mathcal{G}}^{#1}_{#2}}
\newcommand{\taup}[4]{\tau^{#1#2}_{#3#4}}
\newcommand{\taupt}[4]{\tilde{\tau}^{#1#2}_{#3#4}}

%Unknown Tensor
\newcommand{\TensorS}[2]{T^{#1}_{#2}}
\newcommand{\TensorD}[4]{T^{#1#2}_{#3#4}}
\newcommand{\TensorT}[6]{T^{#1#2#3}_{#4#5#6}}
\newcommand{\TensorQ}[8]{T^{#1#2#3#4}_{#5#6#7#8}}

\newcommand{\ts}[2]{t^{#1}_{#2}}
\newcommand{\td}[4]{t^{#1#2}_{#3#4}}
\newcommand{\tXs}[2]{{\overset{{\textsc{x}}}{t^{#1}_{#2}}}}
\newcommand{\tXd}[4]{{\overset{{\textsc{x}}}{t^{#1#2}_{#3#4}}}}
\newcommand{\tYs}[2]{{\overset{{\textsc{y}}}{t^{#1}_{#2}}}}
\newcommand{\tYd}[4]{{\overset{{\textsc{y}}}{t^{#1#2}_{#3#4}}}}


\newcommand{\Ys}[2]{Y^{#1}_{#2}}
\newcommand{\Yd}[4]{Y^{#1#2}_{#3#4}}
%\newcommand{\tri}[6]{t^{#1#2#3}_{#4#5#6}}

\newcommand{\lambdas}[2]{\lambda^{#1}_{#2}}
\newcommand{\lambdad}[4]{\lambda^{#1#2}_{#3#4}}
\newcommand{\Lambdas}[2]{\Lambda^{#1}_{#2}}
\newcommand{\Lambdad}[4]{\Lambda^{#1#2}_{#3#4}}
\newcommand{\sigmas}[2]{\sigma^{#1}_{#2}}
\newcommand{\sigmad}[4]{\sigma^{#1#2}_{#3#4}}
\newcommand{\ro}{r_{0}}
\newcommand{\rs}[2]{r^{#1}_{#2}}
\newcommand{\rd}[4]{r^{#1#2}_{#3#4}}
\newcommand{\ls}[2]{l^{#1}_{#2}}
\newcommand{\ld}[4]{l^{#1#2}_{#3#4}}
\newcommand{\rhos}[2]{\rho^{#1}_{#2}}
\newcommand{\rhod}[4]{\rho^{#1#2}_{#3#4}}

\newcommand{\gammas}[2]{\gamma^{#1}_{#2}}
\newcommand{\gammasb}[2]{\bar{\gamma}^{#1}_{#2}}
\newcommand{\gammad}[4]{\gamma^{#1#2}_{#3#4}}
\newcommand{\gammadb}[4]{\bar{\gamma}^{#1#2}_{#3#4}}

\newcommand{\Gammad}[4]{\Gamma^{#1#2}_{#3#4}}

\newcommand{\Ds}[2]{D^{#1}_{#2}}
\newcommand{\Dd}[4]{D^{#1#2}_{#3#4}}

\newcommand{\SUMs}[1]{\sum\limits_{#1}}
\newcommand{\SUMd}[2]{\sum\limits_{#1#2}}
\newcommand{\SUMt}[3]{\sum\limits_{#1#2#3}}
\newcommand{\SUMq}[4]{\sum\limits_{#1#2#3#4}}
\newcommand{\Pm}[2]{P_-(#1#2)}
\newcommand{\Pmm}[2]{P_{-#1#2}}
\newcommand{\dirac}[2]{\delta_{#1#2}}
\newcommand{\undemi}{\frac{1}{2}}
\newcommand{\unquart}{\frac{1}{4}}
\newcommand{\unhuit}{\frac{1}{8}}
\newcommand{\unseize}{\frac{1}{16}}

\newcommand{\CCdeux}{\ {}_{CC2}}

\newcommand{\sadag}[1]{#1^\dag}
\newcommand{\adag}[1]{\hat{a}_#1^\dag}
\newcommand{\aoper}[1]{\hat{a}_#1}

\newcommand{\braPhizero}{\left\langle \Phi_0 \right| }
\newcommand{\ketPhizero}{\left| \Phi_0 \right\rangle }

\newcommand{\braPhiia}{\left\langle \Phi_i^a \right| }
\newcommand{\ketPhiia}{\left|  \Phi_i^a \right\rangle  }

\newcommand{\braPhiijab}{\left\langle \Phi_{ij}^{ab} \right| }
\newcommand{\ketPhiijab}{\left| \Phi_{ij}^{ab}\right\rangle  }
\newcommand{\braHF}{\left\langle \mathrm{HF} \right|}
\newcommand{\ketHF}{\left| \mathrm{HF} \right\rangle }
\newcommand{\braCC}{\left\langle \mathrm{CC} \right|}
\newcommand{\ketCC}{\left| \mathrm{CC} \right\rangle }

\newcommand{\bramui}{\left\langle \mu_{i} \right|}


\newcommand{\Lagrange}{ \mathscr{L} }
\newcommand{\Quasi}{ \mathscr{Q} }

\newcommand{\PhismatrEOM}[2]{ \Phi_{\mathrm{#2}  
	}^{\mathrm{#1} }}
\newcommand{\PhidmatrEOM}[4]{
	\Phi_{\mathrm{#3#4}}^{\mathrm{#1#2}}}

\newcommand{\namediag}[2]{#1#2}
\newcommand{\namediagIP}[2]{#1#2${}^{\mathrm{IP}} $}
\newcommand{\namediagEA}[2]{#1#2${}^{\mathrm{EA}} $}
\newcommand{\namediagequa}[1]{{}_{\ \{#1\}}}
\newcommand{\namediagequaIP}[1]{{}_{\ \{#1^{\mathrm{IP}}\}}}
\newcommand{\namediagequaEA}[1]{{}_{\ \{#1^{\mathrm{EA}}\}}}

\newcommand{\equivalencesimple}[2]{\{#1#2\}}
\newcommand{\equivalencedouble}[4]{\{\{#1#2\};\{#3#4\}\}}
\newcommand{\equivalencetriple}[6]{\{\{#1#2#3\};\{#4#5#6\}\}}
\newcommand{\equivalencequadruple}[8]{\{\{#1#2#3#4\};\{#5#6#7#8\}\}}

\newcommand{\as}[2]{a^{#1}_{#2}}
\newcommand{\ad}[4]{a^{#1#2}_{#3#4}}
\newcommand{\us}[2]{u^{#1}_{#2}}
\newcommand{\ud}[4]{u^{#1#2}_{#3#4}}
\newcommand{\vs}[2]{v^{#1}_{#2}}
\newcommand{\vd}[4]{v^{#1#2}_{#3#4}}
\newcommand{\Gdeltas}[2]{\Delta^{#1}_{#2}}
\newcommand{\Gdeltad}[4]{\Delta^{#1#2}_{#3#4}}

\newcommand{\MATone}[2]{{\mathrm{MAT}^1}_#2^#1}
\newcommand{\MATtwo}[2]{{\mathrm{MAT}^2}_#2^#1}
\newcommand{\Auxs}[2]{\mathrm{Aux}^{#1}_{#2}}
\newcommand{\Auxd}[4]{\mathrm{Aux}^{#1#2}_{#3#4}}

\newcommand{\MAToneIP}[2]{{\mathrm{MAT}^{\mathrm{IP}1}}_#2^#1}
\newcommand{\MATtwoIP}[2]{{\mathrm{MAT}^{\mathrm{IP}2}}_#2^#1}


\newcommand{\taudown}[4]{{\overset{{\tiny \APLup}}{\tau}}{}^{#1#2}_{#3#4}}
\newcommand{\tauno}[4]{{\overset{{\tiny \APLbox}}{\tau}}{}^{#1#2}_{#3#4}}
\newcommand{\tauboth}[4]{{\overset{{\tiny \APLup \APLdown}}{\tau}}{}^{#1#2}_{#3#4}}
\newcommand{\tauUp}[4]{{\overset{{\tiny\APLdown}}{\tau}}{}^{#1#2}_{#3#4}}

\newcommand{\xis}[2]{\xi^{#1}_{#2}}
\newcommand{\xid}[4]{\xi^{#1#2}_{#3#4}}


\newcommand{\EcritureFortran}[1]{{\noindent {\scriptsize #1 }}}

\newcommand{\DiracProgram}{$ \mathcal{D} $irac}

\newcommand{\light}{\mathrm{c}}
\newcommand{\qe}{\mathrm{e}}
\newcommand{\hbardroit}{\text{\textcrh}}

\newcommand{\XtwoC}{${}^2$DC}
\newcommand{\XtwoCmmf}{${}^2$DC$^{\text{M}}$}
\newcommand{\XtwoCmmfGaunt}{${}^2$DCG$^{\text{M}}$}


\newcommand{\CoutComp}[1]{$ \mathcal{O}\left( N^{#1}\right)  $}
\newcommand{\CoutOV}[2]{$\left(n_{occ}^{#1}n_{vir}^{#2}\right)$}

\newcommand{\Exatensor}{ExaTensor}
\newcommand{\Exacorr}{ExaCorr}
\newcommand{\Talsh}{TAL-SH}

\newcommand{\NULNUL}{\textcolor{red}{Des faits pas du baratin}}

\newcommand{\ordreH}[1]{^{\left( #1 \right) }}

\newcommand{\etal}{\emph{et al.}}

%\renewcommand*{\glstextformat}[1]{\emph{#1}}

\newcommand{\cmark}{\ding{51}}%
\newcommand{\xmark}{\ding{55}}%


\newcommand{\LH}[1]{\textcolor{orange}{*\todo[size=\tiny,
		color=white,linecolor=black,fancyline]{#1}}}
	
\newcommand{\ASPG}[1]{\textcolor{red}{*\todo[size=\tiny,
		color=white,linecolor=black,fancyline]{#1}}}
		
\newcommand{\RM}[1]{\textcolor{green}{RM : #1}}
\newcommand{\TJ}[1]{\textcolor{blue}{TJ : #1}}

\newcommand{\Mycite}[1]{\citeauthor{#1}~\cite{#1}}



\newcommand{\ip}{IP}%
\newcommand{\MBPTip}{MBPT(2)-IP}%
\newcommand{\pip}{P-IP}%
\newcommand{\pMBPTip}{P-\MBPTip}%

\newcommand{\iptab}{IP}%
\newcommand{\MBPTiptab}{PT(2)}%
\newcommand{\piptab}{P-IP}%
\newcommand{\pMBPTiptab}{P-\MBPTiptab}%

\newcommand{\ea}{EA}%
\newcommand{\MBPTea}{MBPT(2)-EA}%
\newcommand{\pea}{P-EA}%
\newcommand{\pMBPTea}{P-\MBPTea}%

\newcommand{\eatab}{EA}%
\newcommand{\MBPTeatab}{PT(2)}%
\newcommand{\peatab}{P-EA}%
\newcommand{\pMBPTeatab}{P-\MBPTeatab}%

\newcommand{\ee}{EE}%
\newcommand{\MBPTee}{MBPT(2)-EE}%
\newcommand{\pee}{P-EE}%
\newcommand{\pMBPTee}{P-\MBPTee}%

\newcommand{\eetab}{EE}%
\newcommand{\MBPTeetab}{PT(2)-EE}%
\newcommand{\peetab}{P-EE}%
\newcommand{\pMBPTeetab}{P-\MBPTeetab}%




\newcommand{\HamilDCG}{${}^{2}$DCG$^{M}$}%

%%%% Pour les niveaux des PES

\newcommand{\trenteetun}{X$^{3/2} $}
\newcommand{\trentedeux}{A$^{3/2} $}
\newcommand{\onze}{X$^{1/2} $}
\newcommand{\douze}{A$^{1/2} $}
\newcommand{\treize}{\textbf{OnAEnleverCetteSolution}}

\newcommand{\TransiUn}{\onze-\trenteetun}
\newcommand{\TransiDeux}{\trentedeux-\trenteetun}
\newcommand{\TransiTrois}{\douze-\trenteetun}
\newcommand{\TransiQuatre}{\treize-\trenteetun}

\newcommand{\fullip}{full-IP}

\newcommand{\dyallavtz}{dyall-av3$\zeta$}
\newcommand{\dyallacvtz}{dyall-acv3$\zeta$}
\newcommand{\inftycorr}{$[\infty \SI{}{\hartree}]$}
\newcommand{\corrlim}[2]{$\left[\num{#1};\ \num{#2} \right]$\SI{}{\hartree}}





%%%
%%%%%%%%%%%% %%%%%%%%%%%% %%%%%%%%%%%% %%%%%%%%%%%%


% \usepackage{xcolor}
% \usepackage{rancy}

\newcommand{\revAG}[1]{{\textcolor{blue}{ASPG: #1}}}
\newcommand{\revXY}[1]{{\textcolor{magenta}{XY: #1}}}
\newcommand{\revLH}[1]{{\textcolor{teal}{LH: #1}}}
\newcommand{\revLV}[1]{{\textcolor{cyan}{LV: #1}}}
\newcommand{\revSP}[1]{{\textcolor{pink}{SP: #1}}}
\newcommand{\revS}[1]{{\color{orange}{SC: #1}}}


%%%%%%%%%%%%%%%%%%%%%%%%%%%%%%%%%%%%%%%%%%%%%%%%%%%%%%%%%%%%%%%%%%%%%
%% If issues arise when submitting your manuscript, you may want to
%% un-comment the next line.  This provides information on the
%% version of every file you have used.
%%%%%%%%%%%%%%%%%%%%%%%%%%%%%%%%%%%%%%%%%%%%%%%%%%%%%%%%%%%%%%%%%%%%%
%%\listfiles

%%%%%%%%%%%%%%%%%%%%%%%%%%%%%%%%%%%%%%%%%%%%%%%%%%%%%%%%%%%%%%%%%%%%%
%% Place any additional macros here.  Please use \newcommand* where
%% possible, and avoid layout-changing macros (which are not used
%% when typesetting).
%%%%%%%%%%%%%%%%%%%%%%%%%%%%%%%%%%%%%%%%%%%%%%%%%%%%%%%%%%%%%%%%%%%%%
\newcommand*\mycommand[1]{\texttt{\emph{#1}}}

%%%%%%%%%%%%%%%%%%%%%%%%%%%%%%%%%%%%%%%%%%%%%%%%%%%%%%%%%%%%%%%%%%%%%
%% Meta-data block
%% ---------------
%% Each author should be given as a separate \author command.
%%
%% Corresponding authors should have an e-mail given after the author
%% name as an \email command. Phone and fax numbers can be given
%% using \phone and \fax, respectively; this information is optional.
%%
%% The affiliation of authors is given after the authors; each
%% \affiliation command applies to all preceding authors not already
%% assigned an affiliation.
%%
%% The affiliation takes an option argument for the short name.  This
%% will typically be something like "University of Somewhere".
%%
%% The \altaffiliation macro should be used for new address, etc.
%% On the other hand, \alsoaffiliation is used on a per author basis
%% when authors are associated with multiple institutions.
%%%%%%%%%%%%%%%%%%%%%%%%%%%%%%%%%%%%%%%%%%%%%%%%%%%%%%%%%%%%%%%%%%%%%
\author{Xiang Yuan}
% \altaffiliation{A shared footnote}
\affiliation[PhLAM]{Univ. Lille, CNRS, UMR 8523 - PhLAM - Physique des Lasers Atomes et Molécules, F-59000 Lille, France}
%\email{xiang.yuan@univ-lille.fr}
\alsoaffiliation[Vrije Universiteit Amsterdam]
{Department of Chemistry and Pharmaceutical Sciences, Faculty of Science, Vrije Universiteit Amsterdam, 1081 HV Amsterdam, The Netherlands}


\author{Loïc Halbert}
% \altaffiliation{A shared footnote}
% \phone{+123 (0)123 4445556}
% \fax{+123 (0)123 4445557}
%\email{loic.halbert@univ-lille.fr}
\affiliation[PhLAM]{Univ. Lille, CNRS, UMR 8523 - PhLAM - Physique des Lasers Atomes et Molécules, F-59000 Lille, France}

\author{Johann Valentin Pototschnig}
\affiliation[Vrije Universiteit Amsterdam]
{Department of Chemistry and Pharmaceutical Sciences, Faculty of Science, Vrije Universiteit Amsterdam, 1081 HV Amsterdam, The Netherlands}

\author{Anastasios Papadopoulos}
% \altaffiliation{A shared footnote}
\affiliation[Vrije Universiteit Amsterdam]
{Department of Chemistry and Pharmaceutical Sciences, Faculty of Science, Vrije Universiteit Amsterdam, 1081 HV Amsterdam, The Netherlands}

\author{Sonia Coriani}
%\email{soco@kemi.dtu.dk}
\affiliation{DTU Chemistry -- Department of Chemistry, Technical University of Denmark, DK-2800 Kongens Lyngby, Denmark}

\author{Lucas Visscher}
%\email{l.visscher@ vu.nl}
\affiliation[Vrije Universiteit Amsterdam]
{Department of Chemistry and Pharmaceutical Sciences, Faculty of Science, Vrije Universiteit Amsterdam, 1081 HV Amsterdam, The Netherlands}

\author{André Severo Pereira Gomes}
\email{andre.gomes@univ-lille.fr}
\affiliation[PhLAM]{Univ. Lille, CNRS, UMR 8523 - PhLAM - Physique des Lasers Atomes et Molécules, F-59000 Lille, France}

%%%%%%%%%%%%%%%%%%%%%%%%%%%%%%%%%%%%%%%%%%%%%%%%%%%%%%%%%%%%%%%%%%%%%
%% The document title should be given as usual. Some journals require
%% a running title from the author: this should be supplied as an
%% optional argument to \title.
%%%%%%%%%%%%%%%%%%%%%%%%%%%%%%%%%%%%%%%%%%%%%%%%%%%%%%%%%%%%%%%%%%%%%
\title[An \textsf{achemso} demo]
  {Formulation and Implementation of Frequency-Dependent Linear Response Properties with Relativistic Coupled 
  Cluster Theory for GPU-accelerated Computer Architectures}

%%%%%%%%%%%%%%%%%%%%%%%%%%%%%%%%%%%%%%%%%%%%%%%%%%%%%%%%%%%%%%%%%%%%%
%% Some journals require a list of abbreviations or keywords to be
%% supplied. These should be set up here, and will be printed after
%% the title and author information, if needed.
%%%%%%%%%%%%%%%%%%%%%%%%%%%%%%%%%%%%%%%%%%%%%%%%%%%%%%%%%%%%%%%%%%%%%
\abbreviations{IR,NMR,UV}
\keywords{American Chemical Society, \LaTeX}

%%%%%%%%%%%%%%%%%%%%%%%%%%%%%%%%%%%%%%%%%%%%%%%%%%%%%%%%%%%%%%%%%%%%%
%% The manuscript does not need to include \maketitle, which is
%% executed automatically.
%%%%%%%%%%%%%%%%%%%%%%%%%%%%%%%%%%%%%%%%%%%%%%%%%%%%%%%%%%%%%%%%%%%%%
\begin{document}

%%%%%%%%%%%%%%%%%%%%%%%%%%%%%%%%%%%%%%%%%%%%%%%%%%%%%%%%%%%%%%%%%%%%%
%% The abstract environment will automatically gobble the contents
%% if an abstract is not used by the target journal.
%%%%%%%%%%%%%%%%%%%%%%%%%%%%%%%%%%%%%%%%%%%%%%%%%%%%%%%%%%%%%%%%%%%%%
\begin{abstract}
We present the development and implementation of the relativistic coupled cluster linear response theory (CC-LR)  
%capable of determining 
which allows the determination of
molecular properties arising from time-dependent or time-independent electric, magnetic, or mixed electric-magnetic perturbations (within a common gauge origin for the magnetic properties), as well as to take into account the finite lifetime of excited states in the framework of damped response theory. We showcase our implementation, which is capable to offload the computationally intensive tensor contractions characteristic of coupled cluster theory onto graphical processing units (GPUs), in the calculation of: \textit{(a)} frequency-(in)dependent dipole-dipole polarizabilities of IIB atoms and selected diatomic molecules, with a particular emphasis on the calculation of valence absorption cross-sections for the I$_2$ molecule;\textit{(b)} indirect spin-spin coupling constants for benchmark systems such as the hydrogen halides (HX, X = F-I) as well the H$_2$Se-H$_2$O dimer as a prototypical system containing hydrogen bonds; and 
\textit{(c)} optical rotations at the sodium D line for hydrogen peroxide analogues (H$_{2}$Y$_{2}$,  Y=O, S, Se, Te). Thanks to this implementation, we are able show the similarities in performance--but often the significant discrepancies--between CC-LR and approximate methods such as density functional theory (DFT). Comparing standard CC response theory with the flavor based upon the equation of motion formalism, we find that, for valence properties such as polarizabilities, the two frameworks yield very similar results across the periodic table as found elsewhere in the literature; for properties that probe the core region such as spin-spin couplings, on the other hand, we show a progressive differentiation between the two as relativistic effects become more important. Our results also suggest that as one goes down the periodic table it may become increasingly difficult to measure pure optical rotation at the sodium D line, due to the appearance of absorbing states.   
\end{abstract}

%%%%%%%%%%%%%%%%%%%%%%%%%%%%%%%%%%%%%%%%%%%%%%%%%%%%%%%%%%%%%%%%%%%%%
%% Start the main part of the manuscript here.
%%%%%%%%%%%%%%%%%%%%%%%%%%%%%%%%%%%%%%%%%%%%%%%%%%%%%%%%%%%%%%%%%%%%%
\section{Introduction}

The fundamental molecular properties, that are 
connected to the response of a system to external perturbations such as electric or magnetic fields, are central to the study of linear and non-linear optics \cite{barron2009molecular,bishop_molecular_1990,papadopoulos_non-linear_2006,cronstrand_multi-photon_2005}. It is widely acknowledged that molecules containing heavy elements, that is, those found towards the lower parts of the periodic table, have a plethora of applications. For instance, by manipulating the molecular polarizability, researchers can design materials with advanced optical properties for use in photovoltaic devices and glasses, such as bismuth oxide-based materials\cite{komatsu2020review}. Another important example is the utilization of optical activity to design Lanthanide complexes as chiral probes for biological processes\cite{carr2012lanthanide}. A detailed understanding of the physical phenomena behind these properties at the atomic or molecular level is very important to tune them or to provide insight for the development of new materials and novel applications.


In quantum mechanics, molecular properties can be derived via perturbation theory, or more specifically, through the response theory formalism, which 
in general lines 
identify molecular properties from the derivatives of the energy (or an equivalent quantity) with respect to the external perturbations. The genesis of modern response theory may perhaps be traced back to the introduction by~\citet{langhoff1972aspects} in 1972, of a formalism that allowed both time-dependent and time-independent perturbations to be taken into account analytically, 
i.e., without employing finite-difference (finite-field) approaches, which are numerically straightforward (but computationally expensive) and only applicable to the time-independent case. 
Among the properties one can calculate, those 
related to the linear response~\cite{norman_perspective_2011,helgaker_recent_2012} of the systems are particularly interesting since they give rise to e.g.\ the polarizability and optical activity, and can provide us with information on electronically excited states.

The current formulations of response properties may be categorized into those employing either Ehrenfest theorem~\cite{dalgaard1980time,olsen_linear_1985} 
or quasi-energy approaches\cite{rice1991calculation,sasagane1993higher,Christiansen1998}. Although response theory based on exact wave functions can provide the expressions for molecular properties directly, practical applications require the use of approximate models such as Hartree-Fock (HF) and density functional theory (DFT), and many other wave function based approaches such as multi-configuration self-consistent field (MCSCF), configuration interaction (CI), coupled cluster (CC) to name just a few (see~\citet{helgaker_recent_2012} for a comprehensive survey). To date response theory has achieved great success in dealing with a wide variety of molecular properties, and treating both small and large-scale systems \cite{pawlowski_molecular_2015,helgaker_recent_2012,norman_perspective_2011,norman_principles_2018,albota1998design,macak2000electronic}. Here, the availability of analytic derivatives approaches has proven to be important for efficient calculations, particularly for large-scale molecule simulations. 

However, while most formulations (and implementations) of response theory mentioned above are based on non-relativistic quantum mechanics, it is now widely recognized that when dealing with molecules containing heavy elements, relativistic effects must also be taken into account\cite{shee_analytic_2016,saue_relativistic_2011,vicha_relativistic_2020,bolvin_alternative_2006}. In addition, heavy elements also have more electrons than their lighter counterparts, which can bring about subtler effects due to electron correlation that may significantly impact the molecular properties. In the domain of relativistic quantum chemistry, the linear-response function based on approximate models including HF~\cite{saue_linear_2003,visscher_4-component_1997}, DFT~\cite{saue2002four,aquino_electric_2010}, and Second-Order-Polarization-Propagator Approximation (SOPPA)\cite{schnack-petersen_second-order-polarization-propagator-approximation_2020} has been well-established. Due to its modest computational cost, DFT has become the most widely used approach for correlated electronic structure theory, even though it is not possible to systematically improve the quality of calculations with currently available density functional approximations~\cite{burke2012perspective}. Due to that, depending on the property of interest, DFT results may deviate strongly from experimental or accurate theoretical models for relativistic electronic structure calculations, even for closed-shell species around the ground-state equilibrium structure~\cite{kervazo_accurate_2019,sunaga_towards_2021}. An alternative to DFT is found in CC theory, which is considered as a “gold standard”~\cite{crawford2007introduction,bartlett2007coupled} among electronic structure methods due to its ability to yield results that approach chemical accuracy for both correlation energies and properties. 

To date, there are various CC linear-response (CC-LR) implementations based on standard models such as CC2~\cite{Christiansen1995}, CCSD\cite{Christiansen1998,pawlowski_molecular_2015} and CC3~\cite{hald_calculation_2003}. These approaches have been shown to achieve good agreement with experimental values for both electric and magnetic molecular properties~\cite{crawford2019reduced,krylov2008equation,khani_uv_2019,gauss_coupledcluster_1995,christiansen_integral-direct_1998,ruud_optical_2002}. We also note the emergence in recent years of response theory implementations based on the equation-of-motion coupled cluster (EOM-CC) model~\cite{coriani_molecular_2016,nanda_communication_2018,alessio_equation--motion_2021,andersen2022cherry,andersen2022probing,faber2018resonant}, which are appealing due to their simpler programmable expressions while yielding exactly the same excitation energies as CC-LR, and nearly equivalent numerical results for response properties. In the time-dependent framework, as pointed out by~\citet{coriani_molecular_2016}, the EOM-CC response is equivalent to the combination of an exponential parametrization for the ground-state wavefunction, and a linear parametrization for the time-dependent wavefunction (which these authors refer to as a CC-CI type wave function), as opposed to the CC-LR case, which employs exponential parametrizations for both time-dependent and time-independent wavefunctions (referred to as CC-CC type wavefunctions). 

A significant downside of these implementations, however, is that they are available only for non-relativistic or rather approximate relativistic Hamiltonians. As such, they are not generally suitable for treating molecular systems containing heavy elements. In this manuscript, we aim to bridge this gap and present the implementation and pilot applications of CC-LR and EOM-CC models in combination with relativistic Hamiltonians, as part of the ExaCorr~\cite{pototschnig_implementation_2021} module of the DIRAC program~\cite{saue2020dirac}. One feature of ExaCorr is its ability, through the use of the ExaTENSOR\cite{lyakh_domainspecific_2019} library, to carry out distributed tensor operations with offloading to graphical processing units (GPUs)--which have been shown to be ideally suited to accelerate coupled cluster calculations due to the latter's substantial floating-point operation and memory-intensive nature~\cite{deprince2011coupled,calvin2020many,hohenstein2021gpu,pototschnig_implementation_2021,hillers2023massively}. In the work detailed here we take advantage of GPU offloading and thread-level parallelism, and will discuss the currently ongoing work to enable large-scale parallel calculations in a subsequent publication.


Apart from discussing our implementation, we showcase its generality and versatility by examining examples of three distinct classes of molecular properties: those involving purely electrical perturbations, purely magnetic perturbations, and mixed electric and magnetic perturbations. 

As an example of the first class, we take the electric dipole polarizabilities because of their significance in a wide range of applications and because they provide valuable insights into the properties and behavior of molecules. For example, materials with high dipole polarizabilities and dielectric constant are used in the polymers that are needed for high-energy-density capacitors\cite{thakur2016recent}, while materials with low dipole polarizabilities\cite{volksen2010low} are used as insulators in electrical devices. For optical spectroscopies, in the calculation of resonant processes such as electronic excitations it is important, from both a practical and physical points of view, to account for the finite excited-state lifetimes in the calculation of response functions, since these will relate to the broadening in the measured spectra. The damped coupled cluster response theory has in recent years emerged as a very effective tool for incorporating such effects in simulating the spectroscopy of complex molecules\cite{norman2001near,norman2005nonlinear,coriani2012coupled,coriani2012lanczos,kauczor2013communication}. In this manuscript, we demonstrate our ability to calculate damped response functions, as we can handle perturbing external fields with either real or complex frequencies. 


We consider indirect nuclear spin-spin coupling constants as a representative of the second class. Indirect nuclear spin-spin coupling constants 
%CHECK THIS https://doi.org/10.1016/j.pnmrs.2008.02.002
manifest themselves in Nuclear Magnetic Resonance (NMR) spectroscopy, which alongside optical spectroscopies is another invaluable tool in chemistry. As a substantial fraction of the atoms in the periodic table is NMR-active, the technique can very often be used to provide critical information about their chemical environment\cite{helgaker1999ab,helgaker2008quantum,vaara2007theory} in a non-destructive way. Regarding computational analysis, apart from the fact that theoretical calculations are extremely useful to interpret experimental signals, it has been demonstrated that it is essential to account for relativistic effects already for elements around the third row of the periodic table\cite{visscher1999full,franzke2021nmr,franzke2023reducing,aucar2018foundations,liu2017handbook}. Magnetic properties are often challenging to calculate, due to the dependence of the results on the gauge origin of an external magnetic field for incomplete bases sets. However, the indirect spin-spin coupling is expressed as the second derivative of the electronic energy with respect to the internal magnetic fields caused by nuclear spins, so that the gauge-origin issue does not arise. 


Optical rotation is taken as an example of the third class. Studying optical rotation is of significant interest for several reasons. First and foremost, optical rotation measurements can provide information about the chiral nature of molecules. This is particularly important in the pharmaceutical industry, as many drugs are chiral and their properties can vary depending on their handedness\cite{nguyen2006chiral}. In addition to its applications in the pharmaceutical industry, studying optical rotations can also provide insights into the electronic and structural properties of molecules. Optical rotations are influenced by a variety of factors, including the electronic structure of the molecule, the molecular geometry, and the surrounding environment.  Moreover, in materials science, the optical properties of materials can be used to design and develop new materials\cite{he2008multiphoton}. For this property the gauge-origin issue mentioned above also arises\cite{helgaker_recent_2012}. In subsequent work we will explore approaches to ensure gauge-invariance for  coupled-cluster calculations of optical rotation~\cite{Pedersen2004,Caricato2020,Parsons2023}, but we note that for the small, symmetric molecules studied here the use of a common gauge origin yields sufficiently accurate results to allow for a comparison of different electronic structure approaches~\cite{Ruud2002,Ruud2003,Crawford2009}, which is our goal here.



The manuscript is organized as follows: In  
Section~\ref{Theory},
response theory and its corresponding parametrization for time-dependent coupled cluster wave-function are summarized. 
In Section~\ref{Implementation}, we described the details of the implementation. 
Section~\ref{ComputDetails} is devoted to the details of the computations we used to test the implementation. The calculations are presented and discussed in Section~\ref{Results}. Finally, a brief summary is given in Section~\ref{Conclusions}.


\section{Theory}
\label{Theory}

We base the theory on the 
time-averaged quasienergy formalism, which we briefly summarize below, and refer the reader to the landmark paper by~\citet{Christiansen1998} for a detailed discussion on it, as well as other more recent works~\cite{coriani_molecular_2016,pawlowski_molecular_2015,norman2018principles}.

\subsection{Response functions based on time-average quasienergy}

We aim to solve the time-dependent wave equation 
\begin{equation}
 i \frac{\partial}{\partial t} |\Psi(t)\rangle = H |\Psi(t) \rangle
\end{equation}
where ${H}$ is the total electronic Hamiltonian 
\begin{equation}
H = H_0 + V(t),
\end{equation}
composed of $H_0$, which represents the time-independent electronic Hamiltonian (e.g.\ the Dirac-Coulomb Hamiltonian, the eXact 2-component Hamiltonian (X2C), the Levy-Leblond Hamiltonian, etc., see~\cite{saue_relativistic_2011,saue2020dirac} and references therein), and $V(t)$ representing a sum of $N$ perturbations that are periodic in time with frequencies $\omega_k$
\begin{equation}
V(t) = \sum_{k=1}^N \left[ (e^{i\omega_k t} + e^{-i\omega_k t}) \sum_\beta \epsilon_\beta (\omega_k) {X}_\beta \right]
\end{equation}
expressed in terms of a one-body operator ${X}_\beta$ 
and the associated frequency-dependent perturbation strength $\epsilon_\beta (\omega_k)$. 
In the present study, ${X}_\beta$
corresponds, for instance, to the $x$-component of the electric dipole operator $\hat{\mu}_x$, or 
 to the $y$-component of the magnetic dipole operator, $\hat{m}_y$, etc.

We write the time-dependent wavefunction $\ket{\Psi(t)}$ as a product of a time-dependent phase and a time-dependent wavefunction $\ket{0(t)}$,
\begin{equation}
\ket{\Psi(t)} = e^{-iF(t)} \ket{0(t)}
\end{equation}
and assume that $\ket{0(t)}$ will reduce to the time-independent
$\ket{0}$ at the time-independent limit (obtained by setting all $\omega_k = 0$). With this choice, one can
define a Quasienergy $Q(t)$
\begin{equation}
Q(t) = \dot{F}(t) =\bra{0(t)} H - i \frac{\partial}{\partial t} \ket{0(t)}
\end{equation}
that, in the absence of perturbation, will reduce to the time-independent energy $E_0$.  According to the time-averaged time-dependent Hellmann-Feynman theorem\cite{Christiansen1998,langhoff1972aspects}, by defining a time-averaged quasienergy $\{Q\}_{T}$
(over the period $T$)
    \begin{equation}
        \{Q\}_{T}=\frac{1}{T}\int_{-T/2}^{T/2} \bra{0(t)}(\hat{H}-i\frac{\partial}{\partial t})\ket{0(t)} dt,
    \end{equation}
and making it stationary to changes in $\ket{0(t)}$,
\begin{equation}
    \delta\{Q\}_{T} = 0, \label{quasienergy-stationary}
\end{equation}
we arrive at a definition of time-dependent response properties as derivatives of $\{Q\}_{T}$
\begin{equation}
\{Q(t)\}_T = E_0 + \sum_x \langle X \rangle \epsilon_x(0) + \frac{1}{2} \sum_{x,y,k} \langle\langle X; Y\rangle\rangle_{\omega_k} \epsilon_y(\omega_k)\delta(\omega_0 + \omega_k)
\end{equation}
where
    \begin{equation}
        \langle X \rangle=\frac{d \{Q\}_{T}}{d\epsilon_{x}(0)}
    \end{equation}
corresponds to an expectation value and
    \begin{equation}
        \langle\langle X; Y \rangle\rangle_{\omega_{k}}=\frac{d^{2} \{Q\}_{T}}{d\epsilon_{x}(\omega_{0})d\epsilon_{y}(\omega_{k})}, \qquad \omega_0 = -\omega_k
    \end{equation}
    
\noindent to linear response properties. Thus, the quasienergy notion leads to a smooth transition from the time-dependent to the static case and provides a connection between time-dependent response properties and derivatives of the time-averaged quasienergy. 
In order to solve 
Eq.~\eqref{quasienergy-stationary} and determine the response functions, one proceeds by choosing a suitable parametrization for $\ket{0(t)}$, and subsequently carries out an expansion of it in terms of orders of perturbation so that the quasienergy itself can be decomposed into contributions arising from different orders of perturbation, e.g.
\begin{equation}
\{Q\}_T = \{Q^{(0)}\}_T + \{Q^{(1)}\}_T + \{Q^{(2)}\}_T + \ldots
\end{equation}
so that\begin{equation}
        \langle\langle X; Y \rangle\rangle_{\omega_{k}}=\frac{d^{2} \{Q^{(2)}\}_{T}}{d\epsilon_{x}(\omega_{0})d\epsilon_{y}(\omega_{k})}
    \end{equation}

An appealing feature of the quasienergy formalism is that, as it will be discussed below, both variational or non-variational parametrizations of $|0(t) \rangle$ can be treated on the same footing, as long as the stationarity conditions are fulfilled. 

\subsection{Parametrization of the time-dependent wave-function}

In the following, we shall be concerned with wavefunctions based on an exponential parametrization of the ground state wavefunction such as the coupled cluster expansion,
\begin{equation}
    \ket{0} = e^{T_0}\ket{R} = \ketCC
\end{equation}
in which $\ket{R}$ denotes the reference state, typically the Hartree-Fock wavefunction, and $T_0$ is the time-independent cluster operator, here restricted to single ($\nu_1$) and double ($\nu_2$) excitations
\begin{equation}
T_0 =  T_1 + T_2 = \sum_{\nu_1} t_{\nu_1} \hat{\tau}_{\nu_{1}} + \sum_{\nu_2} t_{\nu_2} \hat{\tau}_{\nu_{2}} = \sum_{ai} t_i^a \{a^\dagger_a a_i\} + \frac{1}{4}  \sum_{abij} t^{ab}_{ij}  \{a^\dagger_a a^\dagger_b a_ j a_i\}
\end{equation}
with  $a,b$ indicating {particle lines} and $i,j$ hole lines, respectively~\cite{crawford2007introduction}, and $\nu_1, \nu_2$ representing excited configurations with respect to the reference ($\nu_1 \leftrightarrow \ket{\nu^a_i}= \{a^\dagger_a a_i\}\ket{R}, \nu_2 \leftrightarrow\ket{\nu^{ab}_{ij}} = \{a^\dagger_a a^\dagger_b a_ j a_i\}\ket{R}$); in the following, we shall sometimes omit explicit excitation ranks and particle/hole labels and instead employ the shorthand notation $\mu, \nu$ to denote excited determinants.

As suggested by~\citet{pawlowski_molecular_2015}, the time-dependent wave-function $\ket{0(t)}$ can be parametrized in a general manner as :
\begin{equation}
    \ket{0(t,\epsilon_{\beta})} = e^{B_{0}} e^{B(t,\epsilon_{\beta})}\ket{R},
\end{equation}
where $e^{B_{0}}$ and $e^{B(t,\epsilon_{\beta})}$ define the parametrization of the time-independent and time-dependent wavefunctions associated with perturbation $X_{\beta}$ with perturbation strength $\epsilon_{\beta}$, respectively. In the case of coupled cluster wavefunctions, $B_{0} = T_0$,
and the choice to be made is that of the parametrization of the time-dependent part. If the exponential parametrization is retained, we have the CC-CC model (more commonly known as LR-CC), whereas for a linearized version we have the CC-CI model (also referred to as EOM-CC)
\begin{equation}
e^{B(t,X)} \simeq 1 + B(t,\epsilon_{\beta}) = 1 +  \sum_{ai} t_i^a (t,\epsilon_{\beta}) \{a^\dagger_a a_i\} + \frac{1}{4}  \sum_{abij} t^{ab}_{ij}(t,\epsilon_{\beta})  \{a^\dagger_a a^\dagger_b a_ j a_i\}.
\end{equation} 


    
\subsection{The coupled cluster linear response function}

As in the time-independent case, the non-variational nature of the coupled cluster method requires that we define a second-order quasienergy Lagrangian 
    \begin{equation}
        \{L^{(2)}\}_{T} = \{Q^{(2)}\}_{T} + \sum_{\mu}\bar{t}^{(0)}_{\mu}\{\bra{\bar{\mu}}e^{-B^{(1)}(t,\epsilon_{\beta})}(H-i\frac{\partial}{\partial t})e^{B^{(1)}(t,\epsilon)}\ketCC\}_{T}
    \end{equation}
in order to obtain the linear response functions. Here, $\bra{\bar{\mu}} \equiv \bra{\mu} e^{-T_0}$ and $\bar{t}^{(0)}_{\mu}$ are the Lagrange multipliers for the ground-state, obtained solving the linear system
\begin{equation}
\bar{\mathbf{t}}^{(0)}\mathbf{A}=-\boldsymbol{\eta},
\end{equation}
in which the matrix $\mathbf{A}$ is the Jacobian matrix. We note that $\mathbf{A}$ is strictly equivalent to the normal-ordered similarly transformed Hamiltonian $\bar{H}_N$
\begin{equation}
        A_{\mu\nu} \equiv (\bar{H}_N)_{\mu\nu}
        = \left[\exp(-\hat{T}_0)\hat{H}_0\exp(\hat{T}_0) - \bra{\mathrm{HF}}\hat{H}_0\ket{\mathrm{HF}}\right].
\label{RHS rsp equation}
\end{equation}
In the following, we shall use the two terms interchangeably, and for brevity drop the subscript $N$ in $\bar{H}_N$. 



The linear response functions  are expressed as
    \begin{equation}
    \label{CC-CC response function}
    \begin{split}
        ^{\textrm{CC-CC}}\langle\langle X; Y \rangle\rangle_{\omega_{k}} = \frac{1}{2}C^{\pm\omega}P(X(\omega_0),Y(\omega_k))\left[\boldsymbol{\eta}^{X}+\frac{1}{2}\mathbf{F}\mathbf{t}^{X}(\omega_0)\right]\mathbf{t}^{Y}(\omega_k)
    \end{split}
    \end{equation}
for CC-CC~\cite{Christiansen1998,pawlowski_molecular_2015,coriani_molecular_2016} and    
    \begin{equation}
    \label{CC-CI response function}
        ^{\textrm{CC-CI}}\langle\langle X; Y \rangle\rangle_{\omega_{k}} =\frac{1}{2}C^{\pm\omega}P(X(\omega_0),Y(\omega_k))\left[^{EOM}\boldsymbol{\eta}^{X}\mathbf{t}^{Y}(\omega_k)-\sum_{\mu}\bar{t}_{\mu}^{(0)}t_{\mu}^{X}(\omega_0)\sum_{\nu}\bar{t}_{\nu}^{(0)}\xi^{Y}_{\nu}\right]
    \end{equation}
for CC-CI~\cite{pawlowski_molecular_2015,coriani_molecular_2016,faber2018resonant}. In the equations above, $P(X(\omega_0),Y(\omega_k))$ acts to permute perturbations $X$ and $Y$, and 
\begin{equation}
    C^{\pm\omega}f^{XY}(\omega_0,\omega_k) = f^{XY}(\omega_0,\omega_k) + f^{XY}(-\omega_0,-\omega_k)^*
\end{equation} symmetrizes the response functions with respect to simultaneous complex conjugation and inversion of the sign of the frequencies~\cite{Christiansen1998}.

To evaluate the linear response function, we need to obtain the frequency (in-)dependent first-order perturbed amplitudes $\mathbf{t}^{Y}$ by solving the corresponding first-order right-hand side response equations~\cite{Christiansen1998}:
    \begin{equation}
    \label{LR response equation}
        (\bar{\mathbf{H}}-\omega_{k}\mathbf{I})\mathbf{t}^{Y}=-\boldsymbol{\xi}^{Y}.
    \end{equation}
 with $\mathbf{I}$ as the identity matrix.

Because of the equivalence between $\mathbf{A}$ and $\bar{\mathbf{H}}$, Eq.~\eqref{LR response equation} is the same for the CC-CC and CC-CI models, and the poles of the response functions  will occur at the same 
places in the two formulations. This is in line with the fact that excitation energies for CC-LR and EOM-CC are the eigenvalues of $\mathbf{A}$ or $\bar{\mathbf{H}}$ respectively. 

Here, we use the same definitions for matrices $\boldsymbol{\eta}^{Y}, \boldsymbol{\xi}^{Y}$ and $\mathbf{F}$ (the coupled cluster Hessian) as done by ~\citet{Christiansen1998}, and note that in the case of CC-CI, $\boldsymbol{\eta}^{Y}$ is replaced by ${^{\textrm{EOM}}}\boldsymbol{\eta}^{Y}$ as defined by \citet{faber2018resonant}. The detailed working equations used in our implementation are listed in the supplementary material.


Finally, due to the fact that ExaCorr was originally designed for treating systems without symmetry and that in such a case the relativistic wave functions are complex-valued, complex algebra is used throughout. This makes the implementation of damped coupled cluster response theory relatively straightforward; it suffices, in the computation of the response function of interest (say ${\alpha_{yy}}(\omega_0;\omega_k)$), to set the imaginary component of the perturbing frequency $\omega_k$ to a particular inverse lifetime $\gamma$
\begin{equation}
\omega_k \equiv \omega_k + i0 \rightarrow \omega_k + i\gamma
\end{equation}
when solving the response equation\cite{norman2001near,norman2005nonlinear,coriani2012coupled}:
%
    \begin{equation}
    \label{LR response equation_cpp}
        (\bar{\mathbf{H}}-(\omega_{k}+i\gamma)\mathbf{I})\mathbf{t}^{Y}(\omega_{k}+i\gamma)=-\boldsymbol{\xi}^{Y},
    \end{equation}
subject to the condition that $(\omega_k + i\gamma)+(-\omega_0 -i\gamma) = 0$. We note that, while we can in principle use a different value of $\gamma$ for each $\omega_k$, in practice we will follow common usage and keep this value constant for a range of frequencies for which we shall calculate a particular response function. With that, the absorption cross-section for dipole transitions can be determined by the imaginary part of the complex dipole electric polarizability\cite{boyd2020nonlinear} :
    \begin{equation}
        \sigma(\omega) = \frac{4\pi \omega}{c} \mathrm{Im}[\bar{\alpha}(\omega)]
        \label{eq:abs_cpp}
    \end{equation}


\section{Implementation}
\label{Implementation}

The above-mentioned algorithm has been implemented in the development version of the relativistic quantum chemistry package DIRAC\cite{saue2020dirac}  as a part of the ExaCorr code~\cite{pototschnig_implementation_2021}. Currently, the implementation allows for calculations to be carried out only using a single-node configuration. The implementation of multi-node is currently in progress and will be reported in forthcoming works. We can summarize the main computational tasks in the following four steps :
\begin{enumerate}
    \item Solve closed-shell ground state CCSD equations to obtain the $\bf{t}_1, \bf{t}_2$ amplitudes. 
    \item With $\bf{t}_1, \bf{t}_2$, construct the one and two-body intermediates, that are necessary for building the $\bar{\mathbf{H}}$ and linear response functions.
    \item Solve the linear response equation in the full single-double excitation space to obtain the first-order perturbed amplitudes for each operator with different frequencies. To avoid the explicit construction of large matrix $\bar{\mathbf{H}}$, an iterative solver is employed. 
    \item Construct the response function by combing the first-order perturbed amplitudes and the property integrals in the molecular orbital (MO) basis set. 
\end{enumerate}

The first step is carried out within a Kramers-unrestricted formalism\cite{visscher1996formulation} and has been extensively discussed in prior work\cite{pototschnig_implementation_2021}. 

The intermediates in the second step contain two parts: The first part is property-independent and is utilized to construct the $\boldsymbol{\sigma}$-vectors, which are the projection of $\bar{\mathbf{H}}$ in the trial vector space. These intermediates were previously discussed in the implementation of the relativistic EOM-CC 
in the RelCCSD module~\cite{shee2018equation}. We have included a rewritten version of the  $\boldsymbol{\sigma}$-vectors for EOM-CC for excitation energies (EOM-EE) in the supplementary material  (for completeness, expressions for the left EOM-EE $\boldsymbol{\sigma}$-vectors  are also given), due to our use of full tensors in this implementation. We have also corrected misprints identified in the expressions given by~\citet{shee2018equation} (the previously implemented expressions were verified and found to be correct).



In deriving the working equations, we note that for the matrix $\mathbf{F}$ it is not possible to obtain its matrix elements diagrammatically \cite{shavitt2009many}, due to the number of unconnected hole/particle-lines. However, $\mathbf{F}$ is never used by itself but rather as the vector-matrix product  $\mathbf{t}^{X}\mathbf{F}$, in analogy to the $\boldsymbol{\sigma}$-vector expressions for the eigenvalue and response equations. Apart from being readily expressed diagrammaticaly, dealing with the vector-matrix product is computationally advantageous as it reduces storage requirements.   

Moreover, as~\citet{faber2018resonant} suggested, we can avoid computing the \textbf{F} matrix in CC-CI implementation by computing the $^{\textrm{EOM}}\boldsymbol{\eta}^{X}$, which is easily accomplished by modifying the existing $\boldsymbol{\eta}^{X}$ routine (see working equation in the supplementary material).

In the current implementation, all property-related intermediates are computed on the MO basis, but as the property integrals are first generated in the atomic orbital (AO) basis in DIRAC, it is necessary to transform all desired property integrals from AO to MO prior to the response calculations. All the intermediates and the property integrals in MO basis are therefore stored as TAL-SH tensors, so that they can be efficiently employed in constructing the elements of $\boldsymbol{\xi}^{X}, \boldsymbol{\eta}^{X}$ and $^{\textrm{EOM}}\boldsymbol{\eta}^{X}$.

In the next subsection, we will focus on discussing the third step, which involves solving the first-order response equation. This is accomplished by modifying the Davidson schemes utilized for evaluating the eigenvalues and eigenvectors of $\bar{\mathbf{H}}$ in EOM-CC implementation. 


\subsection{Davidson scheme for solving first-order response equation}

Due to the extremely large dimension of $\bar{\mathbf{H}}$ in practical calculations, solving the response equations (e.g., Eq.~\eqref{LR response equation}) to obtain the perturbed amplitudes calls for the use of iterative procedures~\cite{er1975iterativecalculationof,hirao1982generalization}, since directly inverting $(\bar{\mathbf{H}} - \omega_k\mathbf{I})$
in the full single and double excitation space to is not feasible in all but the simplest cases. 
In this work we have opted to follow the scheme outlined by~\citet{olsen1988solution}, with adjustments due to the fact that in $\bar{\mathbf{H}}$ is non-symmetric, so that a common framework for solving both linear systems and eigenvalue equations can be put in place. 

To this end, we have reimplemented and generalized the Davidson solver code outlined by~\citet{shee2018equation}, so that all matrix/vector operations are now expressed in terms of tensor operations, involving the tensor datatypes available in the GPU-accelerated tensor operation frameworks used in ExaCorr~\cite{pototschnig_implementation_2021}. In doing so, we have conserved the features previously implemented for the solution of eigenvalue equations (multi-root solutions, root-following, etc), and added the ability to solve right-hand side and left-hand side linear systems (though for linear response we will only make use of the righ-hand side solutions).  

The iterative solver workflow for the solution of linear systems therefore consists of the following steps, which are summarized in figure~\ref{fig:davidson}:
\begin{enumerate}
    \item Choose an orthonormal vector as the initial guess for the trial vector space $\{\mathbf{b}\}$ where $\mathbf{t}=\mathbf{b}\boldsymbol{\beta}^\prime$ (note that $\mathbf{t}$ and $\mathbf{b}$ are TAL-SH tensors and $\boldsymbol{\beta}^\prime$ is a Fortran array). By default, we start with pivoted unit trial vectors (see below for details);
    \item Construct the reduced subspace matrix  $\mathbf{G}^\prime$ and column vector  $\mathbf{C'}$ by projecting the $\mathbf{G} = (\bar{\mathbf{H}} - \omega_k\mathbf{I})$ matrix $\bar{\mathbf{H}}$ and property gradient vector $\mathbf{C} = \boldsymbol{\xi}^{Y}$ onto the current trial vector space $\{\mathbf{b}_i, i = 1,\ldots L\}$, respectively. The $\bar{\mathbf{H}}\mathbf{b}$ products are obtained with the EOM-EE $\boldsymbol{\sigma}$-vector routines;
    \begin{equation}
    \mathbf{G}\mathbf{t}^Y=-\boldsymbol{\xi}^Y \implies \mathbf{b}^\dagger \mathbf{G}\mathbf{b}\boldsymbol{\beta}^\prime=\mathbf{b}^\dagger\mathbf{C} \implies  \mathbf{G}^\prime\boldsymbol{\beta}^\prime = \mathbf{C}^\prime
    \end{equation}
    \item Evaluate the residual vector ($\mathbf{r}^k$) and preconditioner ($\mathbf{p}^k$) 
    \begin{align*}        
    \mathbf{r}^k&= (\boldsymbol{\sigma}- \omega_k\mathbf{b})\boldsymbol{\beta}^\prime - \mathbf{C} \\
    \mathbf{p}^k &= (\omega_k-\bar{H}_{||})^{-1}
    \end{align*}
    the latter being utilized to improve convergence~\cite{saue_linear_2003}; compute the norm of $\mathbf{r}^k$ and compare it to the threshold defined by the user ;
    \item If the norm of $\mathbf{r}^k$ exceeds the threshold, it indicates that the calculation has not converged. In this case, we construct the correction vector $\boldsymbol{\epsilon}^k = \mathbf{p}^k\mathbf{r}^k$ and orthonormalize it to the existing trial vector using a modified Gram-Schmidt procedure, in order to generate the new trial vectors $\mathbf{b}^{k}$, adding it to $\{\mathbf{b}_i, i=1,\ldots,L,L_k\}$. Using the newly generated trial vectors, repeat step 2 until the norm of the residual vector becomes smaller than the threshold;
    \item Once the norm of the error vector falls below the threshold, it indicates that the calculation has converged. At this point, the Davidson routine stops, and the final solution vector is obtained from $\mathbf{b}$ and $\boldsymbol{\beta}^\prime$ as a TAL-SH tensor, which is subsequently used to calculate the response function of interest.
\end{enumerate}


% Figure environment removed


One particular difference between our implementation and the one by~\citet{shee2018equation} is that the ExaTENSOR and TAL-SH libraries, for reasons of scalability and generality, do not enforce triangularity or the (anti)symmetry of tensors with respect to exchange of pairs of indices. Consequently, and in contrast to the prior implementation, beyond rank-2 tensors antisymmetry needs to be enforced in order to ensure that at all times we satisfy the underlying fermionic nature of the problem.

For example, in the generation of trial vectors for the ${\bf{t}}^x_2$ amplitudes (or ${\bf{r}}_2$ in EOM-CC), this means that we pick out a unique element $\ket{^{ab}_{ij}}$, where $a>b$ and $i>j$, generate the permutations and antisymmetrize them. During Davidson iterations we also ensure that the trial vectors remain antisymmetric during the Gram-Schmidt orthonormalization process, as we found that if explicit antisymmetrization is not carried out, numerical noise may lead to loss of the antisymmetry in new vectors during iterations.

With respect to the choice of starting vectors, differently from the eigenvalue case in which the pivoting was done on the basis of the value of the diagonal of $\bar{\mathbf{H}}$ (see~\citet{shee2018equation} for details), for linear systems, the pivoting is done on the basis of the magnitude of the property gradient $\boldsymbol{\xi}^{X}$ vector elements (from highest to lowest), in order to avoid selecting initial vectors with zero norm. 

\section{Computational details}
\label{ComputDetails}
All coupled cluster linear-response calculations were carried out with a development version of the DIRAC code\cite{saue2020dirac,DIRAC23}, employing the uncontracted singly-augmented valence double zeta Dyall basis set s-aug-dyall.v2z for the heavy elements (Zn~\cite{dyall2009relativistic}, Cd~\cite{dyall2009relativistic}, Hg~\cite{dyall2009relativistic}, Cs~\cite{dyall2009relativistic}, I~\cite{dyall2022diffuse,dyall2006relativistic},  Te~\cite{dyall2022diffuse,dyall2006relativistic}), and a similar uncontracted Dunning basis set aug-cc-pVDZ for the light elements (H~\cite{kendall1992electron}, Li~\cite{prascher2011gaussian}, Na~\cite{prascher2011gaussian}, K~\cite{hill2017gaussian}, F~\cite{kendall1992electron}, Cl~\cite{woon1993gaussian},  O~\cite{kendall1992electron}, S~\cite{woon1993gaussian}, Se~\cite{wilson1999gaussian}, Br~\cite{wilson1999gaussian}). In most calculations, we utilized the exact two-component (X2C)\cite{iliavs2007infinite} relativistic Hamiltonian, but to show the effect of relativity explicitly we also provide results using the non-relativistic Hamiltonian\cite{levy1967nonrelativistic,visscher2000approximate} (as activated by the \texttt{.Levy-Leblond} keyword). To study the effect of electron correlation, we performed linear-response calculations based on mean-field methods like Hartree-Fock (HF)\cite{saue2003linear} as well as density-functional theory\cite{salek2005linear} (especially with the B3LYP\cite{becke1993new} density functional approximation). The relativistic and non-relativistic calculations have been carried out with the Gaussian type\cite{visscher1997dirac} and  point charge nucleus model, respectively. 

In our calculations, we have profited from the components of an ongoing implementation in ExaCorr of the Cholesky-decomposition\cite{beebe1977simplifications,koch2003reduced,aquilante2007low} approach to reduce the memory footprint of our calculations in the step to transform two-electron integrals from AO to MO basis, by avoiding the storage in memory of the whole AO basis two-electron integral tensor. The Cholesky vectors (generated with a  conservative threshold of 10$^{-9}$, as to retain most of them) are then used to explicitly form all six two-electron integral classes needed by the coupled cluster method. In a subsequent publication~\cite{Pototschnig_CD_2023}  we shall address the use of Cholesky vectors explicitly in the coupled cluster implementation of ground and excited state properties. 

The molecular structures employed in all calculations are taken from the literature: In case of the diatomic molecules, from~\citet{huber1979constants} for HX(X=F, Cl, Br, I), I$_{2}$, ICl, from \citet{hessel1971experimental} for NaLi, and from \citet{ferber2008ground} for CsK. The internuclear distances employed are thus H-F(0.91680 \AA), H-Cl(1.27455 \AA), H-Br(1.41443 \AA), H-I(1.60916 \AA), Cl-I(2.32087 \AA), I-I (2.6663 \AA), Li-Na(2.81 \AA), and K-Cs(4.285 \AA). For the chiral molecules H$_{2}$Y$_{2}$(Y=O, S, Se, Te), the Y-Y bond length, H-Y bond length, and H-Y-Y bond angle are taken from Table I of~\citet{laerdahl1999fully} and the dihedral angle is kept fixed at 45 degrees. 


%In the calculations, 
The size of the correlated virtual spinor spaces in the coupled cluster 
calculations is truncated by discarding spinors with energies above 5 a.u. For the IIB atoms, we correlate both semi-core and valence electrons for Zn(3d,4s), Cd(4d,5s), Hg(5d,6s), respectively. In the polarizability and optical rotation calculations of molecular systems, we correlate only valence electrons. In the spin-spin coupling calculations, which are known to be more sensitive to core relaxation and correlation, we correlate all occupied and virtual orbitals. 

All optical rotation calculations (HF, DFT, CC) employed a common gauge origin, set to the origin of the coordinate system, chosen at the midpoint of the bond between the two chalcogen atoms, which nearly coincides with the systems' center of mass. The atomic coordinates for each system under consideration as well as further details on the calculations (position of center of mass etc.) are provided respectively as XYZ and output files in the dataset associated with this manuscript (see ``Supporting Information Available'').

\section{Sample applications}
\label{Results}


\subsection{Polarizability of IIB atoms}
We begin the discussion by analyzing the obtained results for the polarizability of the Zn, Cd, and Hg atoms and present in Table \ref{tab:staticIIB} the static polarizability of these atoms, calculated by different methods. A comparison of the first three rows shows the growing influence of relativity on the static polarizability from Zn to Hg. For example, the relativistic HF value for Hg (44.82 a.u.) is nearly half of its nonrelativistic counterpart (81.05 a.u.). This is mainly due to the strong relativistic contraction of the $6s$-shell.

In contrast, the effect of electron correlation at CCSD level is rather constant for these elements ( $-$10.15 a.u. for Zn, $-$15.39 a.u, and $-$9.57 a.u, for Hg). 

For Zn, electron correlation primarily accounts for the discrepancy between HF and the experimental results. However, for Cd and especially Hg, the inclusion of relativity is crucial. The above-mentioned contraction of the valence s-shell reduces the magnitude of the polarizability, whereas spin-orbit coupling (SOC) becomes increasingly important by enabling spin-forbidden transitions. We will discuss this consequence of relativity in greater depth when looking at the frequency-dependence of the polarizability in the next section. 

An error of approximately 1-2 a.u. 
remains between our relativistic CCSD results and the experimental values. To locate the source of this error, we performed CCSD(T) calculations using the finite-field method, since the analytic gradient is not yet available for CCSD(T) in DIRAC. In these CCSD(T) finite-field calculations, we used an external electric strength of 0.005 a.u, which is sufficiently large to avoid numerical issues and small enough to remain in the linear regime. 

From a comparison between the results of 3rd and 4th rows it is evident that the inclusion of the triple excitations indeed enhances the accuracy: from 95.99\% to 99.84\% for Zn, from 98.44\% to 99.71\% for Cd, and from 96.20\% to 97.95\% for Hg, respectively. In the 5th row, we also performed a calculation in which all virtual orbitals were used (so without energy truncation), but this did not  significantly affect the results. Upon using the valence triple-zeta basis set s-aug-dyall.v3z in the 6th row, the Hg results improve and come close to the experimental error bar.   

    \begin{table}[H]
        \begin{threeparttable}
            \center
            \caption{\label{tab:staticIIB}Static polarizability (a.u.) of IIB atoms calculated with the X2C Hamiltonian}
            \setlength{\tabcolsep}{5.5mm}
            {
            \begin{tabular}{cccc}
            & Zn    & Cd   & Hg \\
            \hline
            NR-HF\tnote{a}  & 53.88 & 76.01& 81.05\\
            HF  & 50.57 & 63.64& 44.82\\
            CCSD & 40.42 & 48.25& 35.25\\
            CCSD(T) & 38.86 & 47.64& 34.62\\
            CCSD(T) (all virtual dz)& 38.80 & 47.69& 34.66\\
            CCSD(T) (tz)& 38.86 & 46.64& 34.27\\
            Exp& 38.8$\pm$0.80\cite{goebel1996theoretical} & 47.5$\pm$2\cite{hohm2022dipole}& 33.91$\pm$0.34\cite{goebel1996dipole} \\
            \hline    
            \end{tabular}
            }
            \begin{tablenotes}
                \item [a] Nonrelativistic calculation with the Levy-Leblond Hamiltonian
            \end{tablenotes}
        \end{threeparttable}
    \end{table}

We now turn to the frequency-dependence of the polarizability and look at the effect of SOC. In Figure \ref{fig:fre-dep-iib}, the frequency-range from 0.0 to 0.30 a.u. is displayed. The  singularities at the frequencies of spin-allowed transition $^1S_0 \rightarrow ^1P_1 (ns\rightarrow np)$ dominate these curves, while the non-relativistically spin-forbidden transition to the $^3P$ state is clearly visible for Hg and, after zooming in on the transition energy, also already for Zn. While calculating the polarizability~\cite{kauczor2013communication}
\begin{equation}
    \alpha_{\alpha\beta}(\omega) = - \sum_{n} \left[ \frac{\bra{0}\hat{\mu}_{\alpha}\ket{n}\bra{n}\hat{\mu}_{\beta}\ket{0}}{E_{n} - \omega} + \frac{\bra{0}\hat{\mu}_{\beta}\ket{n}\bra{n}\hat{\mu}_{\alpha}\ket{0}}{E_{n}+\omega} \right]
    \label{fre-den-pol}
\end{equation}
over a range of frequencies $\{\omega\}$ will yield all excitation energies $E_n$ comprised in the associated energy range. However, in this particular case it is more efficient to directly solve for the poles by diagonalizing $\bar{\mathbf{H}}$. To check the correctness of the implemented solvers, we therefore compared the linear response and EOM-EE results employing the same Hamiltonian and basis set. The resulting excitation energies are depicted in Fig \ref{fig:fre-dep-iib} with red lines and do indeed precisely align with the pole locations in the polarizability curves. 

Looking at the low-lying parity-allowed ($ns \rightarrow mp$) transitions in the studied frequency-range, for Zn and Cd we find two $ns \rightarrow (n+1)p$ transitions (A and B, respectively spin-forbidden ${}^{1}S_{0} \rightarrow {}^{3}P^{o}_{1}$ and spin-allowed ${}^{1}S_{0}\rightarrow{}^{1}P^{o}_{1}$ transitions) and two $ns \rightarrow (n+2)p$ transitions (C and D, similarly spin-forbidden and spin-allowed transitions). For Hg on the other hand, only A and B are within the studied frequency-range, with C and D coming at higher energies and therefore not observed.

On the right side of Fig~\ref{fig:fre-dep-iib}, we also show a simulated spectrum of the first spin-allowed transition $^{1}S_{0} \to {^{1}}P^{o}_{1}$ by calculating the damped linear response function for both CC-CC and CC-CI models. While CC-CI is an approximation of the CC-CC model, we note that the CC-CI curve exhibits a shape very similar shape to that of CC-CC curves, in that both are Lorentzian-type line shapes and share the exact same peak location since they solve the same response equation as demonstrated in the Eq.~\eqref{LR response equation}. The CC-CI model spectrum shows only a minor difference in the peak height with a relative error of about 1\%. To verify the implementation of the complex polarizability, we pay particular attention to the peak value of the spectrum. The Lorentzian line shape function can be represented as
\begin{equation}
    L=\frac{1}{1+(\frac{\omega-\omega_{0}}{\gamma/2})^{2}},
\end{equation}
and compared with the complex polarizability~\cite{kauczor2013communication} around the pole of the B atomic transitions we are investigating, the stationary point in the curves should be well approximated by the norm of the transition dipole moment divided by $\gamma$,
\begin{align}
    \mathrm{Im} \left(\alpha_{\alpha\beta}(\omega)\right) 
 &= \mathrm{Im} \left(- \sum_{n} \left[ \frac{\bra{0}\hat{\mu}_{\alpha}\ket{n}\bra{n}\hat{\mu}_{\beta}\ket{0}}{E_{n} - \omega -i\gamma} + \frac{\bra{0}\hat{\mu}_{\beta}\ket{n}\bra{n}\hat{\mu}_{\alpha}\ket{0}}{E_{n}+\omega+i\gamma} \right]\right)
 \nonumber \\
    \label{complex-fre-den-pol} 
    &\approx \mathrm{Im}\left(\frac{\bra{0}\hat{\mu}_{\alpha}\ket{n}\bra{n}\hat{\mu}_{\beta}\ket{0}}{i\gamma}\right) 
\end{align}
In the current work,  we set the imaginary component of the frequency $\gamma$ as 0.01 a.u. for all three atoms. Even though the EOMCC transition dipole moment is not yet available in DIRAC, 
we can still compare the intensity ratios (Zn:Cd:Hg) between our results (1.39:1.55:1.0) and the values derived from the experimental lifetimes\cite{lurio1964lifetime_Zn,lurio1964lifetime,pinnington1988lifetime} (1.48:1.54:1.0). It is noteworthy that our results qualitatively mirror the experimental trend. The small difference in ratios likely stems from the omission of higher-order correlation and the quality of the basis set used. 

% Figure environment removed




\subsection{Polarizability of Molecules}
As our implementation is mainly intended for molecular systems, we will now look at results for molecular polarizabilities which may have up to three distinct values upon diagonalizing the polarizability tensor. For diatomic and other symmetric molecules it is sufficient to consider the mean dipole polarizability $\alpha(\omega)$ and the anisotropy $\Delta\alpha(\omega)$:
    \begin{align}
        \alpha(\omega) = & \frac{1}{3}\left(\alpha_{zz}(\omega)+\alpha_{xx}(\omega)+\alpha_{yy}(\omega)\right)\\
        \Delta\alpha(\omega) = & \alpha_{zz}(\omega)-\frac{1}{2}\left(\alpha_{xx}(\omega)+\alpha_{yy}(\omega)\right),
    \end{align}

\noindent where $z$ is the molecular symmetry axis.
In Table \ref{tab:static polarizability of diatomics}, we list the static mean and anisotropic polarizability of hydrogen halides and alkali-metal diatomic molecules assessed by HF, B3LYP, and CC models with both relativistic (X2C) and nonrelativistic Hamiltonians, and the corresponding experimental values as well. Unless otherwise specified, 'CC' refers to 'CC-CC-LR'.

The HF results deviate from the experimental value for both the mean and anisotropic polarizability and the impact of the relativistic effect increases as we move from lighter to heavier molecules. For example, the relativistic correction is nearly zero for hydrogen fluoride but amounts to 1.3 a.u. for I$_{2}$. In the case of the CsK molecule, the relativistic correction at the HF level is 31 a.u, emphasizing the necessity of  considering the relativistic effect in the calculation of heavy elements. For this molecule, the effect of relativity may again be rationalized in terms of contraction of the outermost valence $s$-orbitals, in particular that of Cs, which reduces the polarizability, similar to what we observed in the Hg atom. 

Apart from relativity, another source of discrepancy between HF and  experiment lies in the importance of electron correlation. Electron correlation is modeled in DFT by the B3LYP functional, and explicitly calculated in the CC models. From the results, it is evident that in both cases the electron correlation and the relativistic correction are not additive. The B3LYP calculations yield  much better values than HF for both the relativistic and nonrelativistic Hamiltonian. The relativistic correction on the B3LYP model is similar in magnitude as found for HF, but with a different sign. For instance, the relativistic correction of anisotropic polarizability for CsK for HF is +9 a.u while it is $-11$ a.u. in B3LYP. 
%For the halide molecules, 
For the halides, 
the B3LYP calculations yield values that are close to the CC results and arewithin or only slightly outside the experimental error bars for both isotropic and anisotropic polarizability. However, for the alkali-metal diatomic molecules NaLi and CsK, the B3LYP values significantly deviate from the experimental value.


The CC results are close to the experimental data for both halide and alkali-metal molecules. We have also tried using the triple-zeta basis set for CC on three light hydrogen halide molecules (HF, HCl HBr) to reduce the error and indeed observe an improvement of CC values which then fall within the experiment error bar for the isotropic polarizability. Getting the smaller anisotropic polarizability agree with experimental data is more demanding on the model and may require addition of more diffuse functions and/or the inclusion of the triple excitations. 

One may note for the anisotropic polarizability of HI the considerable deviation of all three theoretical values (around 2 a.u.) from the experimental value of 11.4 a.u. that was determined in 1940 by Denbigh\cite{denbigh1940polarisabilities}. Curiously, this value appears to have not been re-evaluated since then, while the isotropic polarizabilities of HCl and HBr, that were also reported by Denbigh, were later estimated to be significantly lower by Kumar and Meath\cite{kumar1985integrated}. The anisotropy of HBr that was given as 6.14 a.u. by Denbigh was adjusted to just 1.7 a.u. by Pinkham and Jones\cite{pinkham2008extracting}, but we could not find a similar re-evaluation of the anistropic polarizability of HI on basis of experimental data in the literature. This discrepancy between theory and the old experimental value for the anisotropy was also noted  in theoretical work of~\citet{maroulis2000dipole} and~\citet{iliavs2003electric}. \citeauthor{iliavs2003electric} used relativistic CCSD(T) and included vibrational corrections on both dipole moment and static polarizability and found their results to be significantly lower than the experimental value: their anisotropic polarizability was 2.33$\pm$0.05 a.u, which agrees well with the current relativistic CC linear response value of 2.51 a.u. While their suggestion that also the experimental value of the dipole moment could be inaccurate could not be not sustained\cite{van2004theoretical, Li_HIReference_2013} we agree that the discrepancy between theory and experiment for the anisotropic polarizability is likely due to an inaccuracy in the experimental value. Nonetheless, it would be nice to put more firm error bars on the theoretical value as well by employing a larger basis set, including g and h functions. This was not feasible with our current implementation due to  memory constraints related to the use of a single compute node.


        \begin{table}[H]
        \begin{threeparttable}        
            \center
            \caption{Static dipole polarizability (a.u.) of diatomic molecules}\label{tab:static polarizability of diatomics}

        \setlength{\tabcolsep}{3.0mm}{
            \begin{tabular}{ccccccccc}
                &HF\tnote{a}  &HF\tnote{b}& B3LYP\tnote{a} & B3LYP\tnote{b}  &  CC\tnote{a} & CC\tnote{b}  & CC\tnote{c} & Exp \\
            \hline
            \multicolumn{9}{c}{Mean dipole polarizability}\\
            \hline
            HF  &4.40 &4.40&5.11 &5.12   &5.04 &5.05 &5.52 &5.60$\pm$0.10\cite{kumar1985integrated}\\
            HCl &15.51 &15.54&16.34 &16.38  &16.06 &16.09 &17.14 &17.39$\pm$0.20\cite{kumar1985integrated}\\
            HBr &21.86 &21.90&22.85 &22.94 &22.52 &22.58 &24.02 &23.74$\pm$0.50\cite{kumar1985integrated}\\
            HI  &33.62 &33.50&34.63 &34.69 &34.36 &34.30 & &35.30$\pm$0.50\cite{cuthbertson1914info}\\
            ICl &46.48 &46.52&47.53 &47.66 &47.48 &47.59 && 43.8$\pm$4.4\cite{swift1988dispersion}\\
            I$_{2}$ &67.90 &69.20&68.72 &69.92 &68.81 &69.72 & &69.7$\pm$1.8\cite{maroulis1997electrooptical}\\
            NaLi &231 &230 &210 &209 &240 &240 &239 &263$\pm$20\cite{antoine1999static}\\
            CsK &723 &692 &581 &548  &637\tnote{d} & 611\tnote{d}& &601$\pm$44\cite{tarnovsky1993measurements}\\
            \hline
            \multicolumn{9}{c}{Anisotropic dipole polarizability}\\
            \hline
            HF  &1.79 &1.79 &1.91 &1.91  &1.96 &1.96 &1.45 &1.62\cite{muenter1972polarizability}\\
            HCl &2.35 &2.34 &2.18 &2.16  &2.39 &2.38 &1.85 &2.10\cite{bridge1966polarization} \\
            HBr &2.43 &2.45 &1.98 &1.92  &2.35 &2.30 &2.02 &1.7\cite{pinkham2008extracting}\\
            HI  &2.66 &2.81 &2.09 &2.00  &2.60 &2.51  & &11.4\cite{denbigh1940polarisabilities}\\
            ICl &26.14 &26.96 &24.27 &24.63  &25.30 &25.82 & & \\
            I$_{2}$ &44.92 &49.01  &41.31 &43.75 &42.88 &45.87 & &45.1$\pm$ 2.3\cite{maroulis1997electrooptical}\\
            NaLi &92 &92 &109 &109 &154 &154 &149 &\\
            CsK &353  &362 &400 &389  &510\tnote{d} &499\tnote{d} & &\\
            \hline
            \end{tabular}}
            \begin{tablenotes}
                \item [a] Nonrelativistic calculation using the Levy-Leblond Hamiltonian
                \item [b] Relativistic calculation using the X2C Hamiltonian
                \item [c] Using diffuse Triple-zeta basis set
                \item [d] Correlate both 6s and 5p electrons of Cs
            \end{tablenotes}
        \end{threeparttable}
        \end{table}


    We now turn our attention to the frequency-dependent polarizability and focus on the I$_{2}$ molecule given the extensive experimental research on this molecule and the abundance of experimental spectral data that can be used to validate theoretical models. In Table \ref{tab:fre-dep-pol of i2}, we present the computed frequency-dependent polarizability for three theoretical methods alongside the experimental values\cite{maroulis1997electrooptical} measured by \citeauthor{maroulis1997electrooptical} at three frequencies. Like the experimental values, the values computed with CC at these three frequencies are quite close to each other and we find reasonable agreement with the CC values slightly underestimating the experimental data. The HF and B3LYP results deviate rather strongly from the experimental results for the first two frequencies which can be rationalized as being caused by an error in the position of the pole close to the first two laser frequencies, that is computed at a too low energy with HF and DFT(B3LYP). Due to the selection rules for this transition to the B$^3\Pi_{0^+u}$ state, this then leads to a negative value of the parallel (zz-)component of the polarizability for HF and B3LYP, while the perpendicular (xx-)component is not affected and has a similar value for HF, B3LYP and CC.

    \begin{table}[H]
        \centering
        \caption{Frequency dependent polarizability (a.u.) of I$_{2}$ molecule}
        \setlength{\tabcolsep}{12.0 mm}{
        \begin{tabular}{cccc}
             &$\alpha_{zz}(\omega)$  &$\alpha_{xx}(\omega)$ &$\alpha(\omega)$ \\
        \hline
            \multicolumn{4}{c}{$\omega_{1}$=15798 cm$^{-1}$}\\
        \hline
        HF&152.0  &55.0  &87.4  \\
        B3LYP&$-$10.7  &58.7  &35.6  \\
        CC&114.8  &55.8  &75.5  \\
        Exp\cite{maroulis1997electrooptical}&  &  &86.8$\pm$2.2  \\
        \hline
            \multicolumn{4}{c}{$\omega_{1}$=16832.3 cm$^{-1}$}\\
        \hline
        HF&$-$97.3  &56.0  &4.9  \\
        B3LYP&75.4  &62.0  &66.5  \\
        CC&124.0  &56.8  &79.2  \\
        Exp\cite{maroulis1997electrooptical}&  &  & 93.6$\pm$3.4  \\
        \hline
            \multicolumn{4}{c}{$\omega_{1}$=30756.9 cm$^{-1}$}\\
        \hline
        HF&117.9  &55.3  &  76.2\\
        B3LYP&114.5  &61.0  &78.8  \\
        CC&113.9  &59.9  &77.9  \\
        Exp\cite{maroulis1997electrooptical}&  &  &95.3$\pm$1.9  \\
        \hline
        \end{tabular}}
        
        \label{tab:fre-dep-pol of i2}
    \end{table}

    
    Rather than looking at the values for just these three frequencies, two of which are close to the X$\rightarrow$B transition, it is more illustrative to apply 
    Eq.~\eqref{eq:abs_cpp} and plot simulated absorption cross-section curves. We scan the wavelength ranging from 400 nm to 700 nm and set the imaginary component of the complex frequency ($\gamma$) to 0.005 a.u, which corresponds to the experimental lifetime of the B$^{3}\Pi_{0+}$ state. As selection rules are different for the transitions to the B and C states we may thereby identify the $zz$-component of the complex dipole electric polarizability as being (primarily) due to the B state, while the $xx$-component is due to the C state. This facilitates the comparison to the experimental analysis that was carried out by \citeauthor{tellinghuisen2011least}\cite{tellinghuisen2011least}. The resulting curves for three models, NR-HF(green lines), X2C-HF(red lines), and X2C-CC(blue lines), are depicted in Fig.~\ref{fig:i2_cpp} and clearly show the effect of SOC. The NR computed curves are entirely due to the weaker transition X$^{1}\Sigma_{0+}$ to C$^{1}\Pi_{1}$ and severely underestimate the absorption cross-section. With SOC, this transition becomes a shoulder on the dominant X$^{1}\Sigma_{0+}$ to B$^{3}\Pi_{0+}$ transition. Comparison with the measured curves (black lines) from the work of \citeauthor{tellinghuisen2011least}\cite{tellinghuisen2011least} shows a quite good agreement for the height of the dominant peak that is slightly red-shifted compared to the experimental transition. The dominant peak in the X2C-HF exhibits a severe red-shift, which clarifies the error seen in the frequency-dependent polarizability in Table \ref{tab:fre-dep-pol of i2}. Errors of the HF model are also large at other frequencies. For example, at the frequency of 500 nm, the experimental cross-section is 2.24 $10^{-18}$cm$^{-2}$, while the X2C-CC value is 1.51 $10^{-18}$cm$^{-2}$ and the X2C-HF value is only 0.51 $10^{-18}$cm$^{-2}$. For the peak values, the X2C-HF result 2.95 $10^{-18}$cm$^{-2}$ is, however, quite close to the X2C-CC value 2.98 $10^{-18}$cm$^{-2}$ suggesting that the value of the transition dipole moment is similar in both models under the current calculation conditions.

Regarding the spin-allowed transition from X$^{1}\Sigma_{0+}$ to C$^{1}\Pi_{1}$ state, displayed by the dotted line in Fig \ref{fig:i2_cpp}, we observe the X2C-CC model to agree well with the experimental analysis of \citeauthor{tellinghuisen2011least}\cite{tellinghuisen2011least}. The discrepancy in pole location is around 20 nm and the difference in the peak value is minor at 0.02 $10^{-18}$cm$^{-2}$. 

    
% Figure environment removed


In the supplemental materials, we provide a calculation of the spectrum of BH molecule which was used to verify the correctness of our implementation. We compare our damped CC-LR calculation with the broadening of coupled cluster transition dipole moment computed by the DALTON program\cite{aidas2014d,christiansen1998integral}. and find good agreement.




\subsection{Spin-spin coupling }

In the previous section, we investigated electric properties. In the current section, we show a calculation of the indirect nuclear spin-spin coupling constant as an illustrative example of the use of our implementation for a magnetic property. The coupling constant $K_{KL}$ can be related to the experimentally observed coupling $J_{KL}$ between the nuclear spins of atoms $K$ and $L$ via
%
\begin{equation}
    J_{KL} = \frac{1}{2\pi}\gamma_{K}\gamma_{L}K_{KL}
\end{equation}

\noindent where $\gamma_{K}$ is the gyromagnetic ratio of nucleus $K$.
The $K_{KL}$ tensor can in a relativistic framework be expressed in terms of linear response functions with respect to the hyperfine operator $\hat{h}_{K}^{\mathrm{hfs}}$:
\begin{equation}
    K_{KL,\mu\nu} =  \frac{\partial^{2}}{\partial m_{K;\mu}\partial m_{L;\nu}}\langle\langle \hat{h}_{K}^{\mathrm{hfs}}; \hat{h}_{L}^{\mathrm{hfs}} \rangle\rangle_{\omega_{k1}, \omega_{k2}}
    \label{spsp_rel_lr}
\end{equation}
\begin{equation}
    \hat{h}_{K}^{\mathrm{hfs}} =  -\sum_{i}\textbf{m}_{K}\cdot\frac{-1}{4\pi\epsilon_{0}c^{2}}\frac{\textbf{r}_{iK}\times ec \pmb{\alpha}}{r_{iK}^{3}}
\end{equation}

In the non-relativistic framework, it is common to formulate $K$ in terms of three distinct 
contributions: diamagnetic spin-orbit coupling (DSO), paramagnetic spin-orbit (PSO), and the Fermi contact-Spin dipolar (FC-SD) term. Of these, the first term can be computed as an expectation value, whereas the second and third require the use of response theory. Moreover, the PSO term involves only singlet excitations, whereas the FC-SD term couples a singlet ground state to triplet excited states due to the triplet nature of the Fermi contact and spin-dipolar operators. An explicit sum-over-states form of the contributions to $K_{KL}$ in the nonrelativistic framework is:

\begin{equation}
\begin{split}
    K_{KL} &= \frac{\alpha^{4}}{2}
    \bra{0}\frac{\textbf{r}_{K}^{T}\textbf{r}_{L}I-\textbf{r}_{K}\textbf{r}_{L}^{T}}{r_{K}^{3}r_{L}^{3}}\ket{0}-2\alpha^{4}\sum_{n_{S}}\frac{\bra{0}r_{K}^{-3}I_{K}\ket{n_{S}}\bra{n_{S}}r_{L}^{-3}I_{L}\ket{0}}{E_{n_{S}}-E_{0}} \\
    &
    -2\alpha^{4}\sum_{n_{T}}\frac{\bra{0}\frac{8\pi}{3}\delta(r_{L})\textbf{s}+\frac{3\textbf{r}_{L}\textbf{r}_{L}^{T}-r_{L}^{2}I_{3}}{r_{L}^{5}}\textbf{s}\ket{n_{T}}\bra{n_{T}}\frac{8\pi}{3}\delta(r_{L})\textbf{s}^{T}+\frac{3\textbf{r}_{L}\textbf{r}_{L}^{T}-r_{L}^{2}I_{3}}{r_{L}^{5}}\textbf{s}^{T}\ket{0}}{E_{n_{T}}-E_{0}}
\end{split}
\label{spsp_nonrel_lr}
\end{equation}

As discussed in Reference~\citenum{aucar1999origin}, the PSO and FC-SD response functions can in the relativistic framework of Eq. \eqref{spsp_rel_lr} be identified as orbital responses involving rotations amongst positive energy orbitals. The DSO contribution, on the other hand, comes from the rotations between positive and negative energy orbitals and can in a sequence of approximations be brought into an expectation value form that is identical to the non-relativistic expression and is then called the Sternheim approximation\cite{sternheim1962second}. 
Therefore, in relativistic calculations, there are two ways to obtain the diamagnetic terms: one by including electron-positron rotations explicitly in the response calculation or by making use of the Sternheim approximation. 

In contrast to the Sternheim approximation, in which a numerically very stable expectation value is computed, the formally more rigorous response approach is quite sensitive to the quality of sampling of the positronic orbital space in a finite basis\cite{aucar1999origin,visscher1999full}. This is why in the current study, we compute the diamagnetic terms as an expectation value. An important modification as compared to the original application in 4-component theory is the use of the X2C transformation, in which all operators are first transformed to a 2C representation. This affects both the diagonal DSO operator, for which the $SS$ part of its matrix representations is folded into the $LL$ part, as well as the off-diagonal PSO and FC-SD operators that couple the small and large parts of the positive energy orbitals. These matrix transformations are carried out automatically by the DIRAC code once the X2C Hamiltonian is activated.


In Table \ref{tab:spsp}, we list the resulting reduced isotropic and anisotropic spin-spin coupling constants of HX(X=F, Cl, Br, I) computed by HF, B3LYP, CC-CI, and CC-CC models with both nonrelativistic and relativistic Hamiltonians. As is well-known, relativistic effects are very important for magnetic properties and we see the expected increase of their magnitude upon descending 
%down in (descending is moving down :-) 
%in 
the periodic table from hydrogen fluoride to hydrogen iodide. To benchmark the quality of the X2C transformation, we also carried out four-component Dirac-Coulomb(DC) HF calculations with default approximation for the all small two-electron integrals\cite{visscher1997approximate} and see that the X2C values match the DC results very well for all molecules.  

At the Hartree-Fock level, the isotropic constants generally exhibit a downward trend from HF to HI, while the anisotropic values typically show an upward trend for both relativistic and nonrelativistic calculations. After including electron correlation, these trends are qualitatively the same although the precise values change considerably, especially for HBr. To verify our CC implementation, we also utilize the CFOUR program\cite{matthews2020coupled} for nonrelativistic CC response and find our CC-CC models with the nonrelativistic Hamiltonian to reproduce the CFOUR values for all three light molecules. 

Although the nonrelativistic calculation is useful for analysis, we cannot ignore relativistic effects for heavy molecules. For example, the relativistic correction at the coupled cluster level for HBr is around 25\% and slightly smaller than that with Hartree-Fock. We also performed DFT calculations, the results obtained with B3LYP functionals are quite far from both the HF and the CC results. As there are no suitable experimental values to compare with one cannot assess rigorously the performance of the methods, but the large discrepancy between the commonly used B3LYP DFT and CC makes these systems of interest for future benchmarking with converged CC expansion (we deem both our employed basis set as well as excitation level not yet suitable for this purpose).  


Looking at the two ways of carrying out CC response calculations, we observe minor variances between the CC-CI and CC-CC, which appear to become more pronounced for the heavier elements. It is known that LR-CC transition moments are size-extensive whereas EOM-CC ones are not ~\cite{kobayashi1994calculation,Koch1994,Sekino1994,Sekino1999,caricato2009difference,Perera2010,Nanda2015,coriani_molecular_2016,faber2018resonant}, though in these comparisons it was found the numerical differences between LR-CC and EOM-CC were rather small for a single molecule. Numerical studies have been primarily concerned with light molecules and properties within the valence domain, like the electric transition dipole moment\cite{caricato2009difference,Nanda2015,faber2018resonant}, and the dipole polarizability\cite{kobayashi1994calculation,Sekino1999}, and our results for polarizabilities are in line with these findings. A notable exception is the work of~\citet{Sekino1999}, which have investigated spin-spin couplings for ethane and found a difference of 0.05\% between EOM-CC and LR-CC for $J_{CC}$. This value is comparable to our difference of about 0.19\% for the HF molecule.

If the lack of size extensivity in EOM-CC transition moments is a significant source of discrepancies, one would expect the difference between CC-CI and CC-CC to grow as the number of electrons correlated increases across the HX series, but the difference per correlated electron to remain roughy constant. Our analysis of the $zz, xx$ and $yy$ components of the linear response contribution to $K_{HX}$ (see supplementary information) provides some evidence this is the case, as differences (in absolute value) for each component fall between 0.004 and 0.02 a.u.\ for all molecules. There are some differences between Hamiltonians for HBr and HI, but these are of smaller magnitude than those due to non-extensivity. However, we believe the sample size is not large enough for definitive conclusions, and in future investigations we intent to revisit this issue for a broader range of molecules.


               \begin{table}[H]
        \begin{threeparttable}
            
            \center
            \caption{Isotropic and anisotropic reduced spin-spin coupling K(10$ ^{19}$ m$^{-2}$ kg s$^{-2}$ A$^{-2}$) for HX(X=F, Cl, Br, I)}\label{tab:spsp}

        \setlength{\tabcolsep}{8.0mm}{
            \begin{tabular}{ccccc}
            Models&  $^{1}$HF$^{19}$ &$^{1}$HCl$^{35}$ &$^{1}$HBr$^{79}$ &$^{1}$HI$^{127}$ \\
            \hline
             \multicolumn{5}{c}{Isotropic}\\
            \hline
            NR-HF     &49.5486 &28.1528 &10.8253 &-0.8979\\
            NR-B3LYP  &33.3898 &19.7146 &-1.8769 &\\
            NR-CC-CI  &40.5554 &31.3181 &30.7926 &\\
            NR-CC-CC  &40.4794 &31.0971 &29.9730 &\\
            NR-CC-CC\tnote{a}     &40.4778 &31.0970 &29.9729 &\\
            X2C-HF    &49.5023 &27.2261 &-4.5338 &-83.1522\\
            X2C-B3LYP &33.2367 &18.9409 &-11.6914 &-57.3316\\
            X2C-CC-CI &40.4834 &30.9008 &23.8246 &3.4887\\
            X2C-CC-CC &40.4047 &30.6448 &22.7588 &0.7481\\
            DC-HF     &49.4725 &27.1494 &-4.8396 &-84.0079\\
            \hline
            \multicolumn{5}{c}{Anisotropic}\\
            \hline
            NR-HF     &2.5499 &59.6666 &161.9806 &277.7237\\
            NR-B3LYP  &6.3484 &50.1075 &130.4249 &\\
            NR-CC-CI  &-3.7566 &36.3828 &100.9785 &\\
            NR-CC-CC  &-3.4931 &37.1193 &102.9362 &\\
            X2C-HF    &2.5858 &60.2375 &168.5425 &305.7204\\
            X2C-B3LYP &6.4477 &50.3990 &130.5655 &201.0597\\
            X2C-CC-CI &-3.6579 &36.8838 &106.6559 &192.8454\\
            X2C-CC-CC &-3.3929 &37.6281 &108.7226 &196.3474\\
            DC-HF     &2.5978 &60.2822 &168.7214 &306.1280\\
            \hline
            
            \end{tabular}}

            \begin{tablenotes}
                \item[a] Calculations were performed using the CFOUR program
            \end{tablenotes}

        \end{threeparttable}
        \end{table}



        

As most experimental work is carried out in the condensed phase, we wanted to go beyond isolated diatomic molecules, and provide a sample investigatation of solvent effects. For this purpose we chose the solvent shift on the spin-spin coupling constant $^{1}$H$_{b}$-${^{34}}$Se in the the H$_2$Se-H$_2$O dimer. The supermolecular structure is taken from the work of \citeauthor{olejniczak2017calculation}\cite{olejniczak2017calculation} and displayed on Fig.\ref{fig:h2se-h2o}. It can readily be seen from Table \ref{tab:solvent of h2se} that all calculations show the solvent effect on the Se-H$_{b}$ coupling for the bond involved in the hydrogen bond to be quite substantial. However, the shifts $\Delta J$ in the correlated models have a different magnitude than that at the HF level. For example, the shifts of Se-H$_{b}$ are 19.5403 Hz and 18.6294 Hz for CC-CC and B3LYP, respectively, while they are almost twice as large at 38.6752 Hz for HF. Although, the shifts of DFT are quite close to those computed with CC, the absolute $J_{iso}^{super}$ and $J_{iso}$ deviate a lot. Comparing with BLYP and B3LYP values, we find the addition of exact exchange to the DFT to have a significant effect, with the hybrid DFT B3LYP results being closer to the CC values.



        % Figure environment removed

        

    \begin{table}[H]
    \begin{threeparttable}
        

        \centering
        \caption{Isotropic and anisotropic indirect spin-spin coupling ($J_{iso}$ and $J_{aniso}$ in Hz) for isolated H$_{2}$Se subsystem, ($J_{iso}^{super}$ and $J_{aniso}^{super}$ in Hz) for H$_{2}$Se subsystem in H$_{2}$Se-H$_{2}$O, and the shifts ($\Delta J$, in Hz) for the isolated ("ME") H$_{2}$Se molecules in the presence of H$_{2}$O}
        \setlength{\tabcolsep}{5.0mm}{
        \begin{tabular}{ccccccc}
        Models  & $J_{iso}^{}$ & $J_{iso}^{super}$ & $\Delta J_{iso}^{ME}$  & $J_{aniso}^{}$ &$J_{aniso}^{super}$ &$\Delta J_{ianso}^{ME}$ \\
        \hline
            \multicolumn{7}{c}{$^{1}$H$_{b}$-Se$^{34}$}\\
        \hline
        HF\tnote{a} &90.4910 &121.9997   &31.5087  &273.0378 &273.1880  &1.1502 \\
        HF&53.4904 & 92.1656    &38.6752  &344.8134 &346.3105  &1.4971 \\
        BLYP&-24.7643  &-9.7263   &15.0380  &258.2390 &264.7545  &6.5155\\
        B3LYP&-8.0744  &10.5550  &18.6294  &265.9582 &271.5174  &5.5592 \\
        CC-CI&66.6432 &85.8755  &19.2303  &215.4586 &218.7408  &3.2821 \\
        CC-CC&65.9553 &85.4956  &19.5403  &219.8036 &223.0023  &3.1987 \\
        
        \hline
            \multicolumn{7}{c}{$^{1}$H$_{b}$-H$^{1}$}\\
        \hline
        HF\tnote{a}&-17.1178 &-18.4485   &-1.3307  &43.4045 &42.9924  &-0.4121 \\
        HF&-16.3628  &-17.6498   &-1.2870  &43.5594 &43.1685  &-0.3936 \\
        BLYP&-7.4933 &-7.8587   &-0.3654  &41.2533 &40.9028  &-0.3505 \\
        B3LYP&-8.1160  &-8.5625 &-0.4465  &41.3009 &40.9406  &-0.3603 \\
        CC-CI&-9.8335 &-10.2246   &-0.3911 &39.9376 &39.5455  &-0.3921 \\
        CC-CC&-10.2487 &-10.6565  &-0.4078 &40.1031  &39.7072  &-0.3959 \\
        \hline
            \multicolumn{7}{c}{$^{1}$H-Se$^{34}$}\\
        \hline
        HF\tnote{a}&92.1711 &92.6673   &0.4962 &269.0086 &266.7415  &-2.2671 \\
        HF&55.6961 &57.0164   &1.3203  &341.4340 &336.7568  &-4.6772 \\
        BLYP&-22.7734 &-19.3135  &3.4599 &256.3827 &253.9397  &-2.4430 \\
        B3LYP&-6.0583 &-3.0526  &3.0057  &263.9494 &261.4449  &-2.5045 \\
        CC-CI&68.1164 &70.0653  &1.9489  &213.7819 &212.0822  &-1.6998 \\
        CC-CC&67.4464 &69.4103  &1.9639  &218.0737 &216.3287  &-1.7450 \\
        \hline
        \end{tabular}}

        \begin{tablenotes}
            \item[a]  Nonrelativistic calculation with the Levy-Leblond Hamiltonian
        \end{tablenotes}
        \label{tab:solvent of h2se}
        \end{threeparttable}
    \end{table}

    


\subsection{Optical rotation}

Finally, we consider both electric and magnetic fields, by looking at optical rotation (in the length gauge and for a common gauge origin) for the archetypical chiral molecules H$_{2}$Y$_{2}$(Y=O, S, Se, Te). At the frequency of the sodium D-line (in 589.29 nm), which is the most common experimental setup, the specific optical rotation $[\alpha]_{D}^{25}$ in unit [${^\circ}$ dm$^{-1}$(g/mol)$^{-1}$] is given by the equations  

    \begin{equation}
        [\alpha]_{D}^{25}=-228\cdot10^{-30}\frac{\pi^{2}N a_{0}^{4}\omega}{3M}\sum_{\alpha}G'_{\alpha\alpha}
    \end{equation}

    
    \begin{equation}
        G'_{\alpha\beta}(-\omega;\omega)= -\mathrm{Im}\langle\langle \hat{\mu}_{\alpha}; \hat{m}_{\beta} \rangle\rangle_{\omega}
    \end{equation}

\noindent where $M$ is the molecular mass in g mol$^{-1}$, $N$ is the number density, and $\mu_{\alpha}$ and $m_{\beta}$ are the electric and magnetic dipole operator, respectively. 

In Fig \ref{fig:opt_rot}, we display the results for HF, B3LYP, and CC for both the nonrelativistic and X2C Hamiltonian. First, to verify our implementation, we performed the calculation on H$_{2}$S$_{2}$  with the DALTON program with the same basis set. The resulting data are available in the supplemental information and show good agreement, confirming the correctness of the implementation. To benchmark the influence of the truncation on the virtual orbital space, we furthermore performed a calculation in which we truncated the virtual orbital space with an energy threshold of 100 a.u. instead of the otherwise used value on 5 a.u. and found that results match up to 99\%. This is similar to the tendency observed in the electric dipole polarizability, as expected as both optical rotation and electric dipole polarizability are predominantly determined by the valence electrons and do not require core-like high-energy virtuals. 

Figure~\ref{fig:opt_rot} shows that for the lighter molecules, H$_{2}$O$_{2}$ and H$_{2}$S$_{2}$, the B3LYP and CC values are nearly twice as large than those of the HF. While the relativistic effect is negligible for H$_{2}$O$_{2}$, with a correction of less than 1\%, it cannot be neglected for H$_{2}$S$_{2}$, where it rises to 10\%. The impact of the relativistic effect is present for all models, but correlation and relativistic effects are again not additive. For instance, we find a relativistic HF correction of -12 [${^\circ}$ dm$^{-1}$(g/mol)$^{-1}$], while for B3LYP and CC, these corrections are -26 [${^\circ}$ dm$^{-1}$(g/mol)$^{-1}$] and -18 [${^\circ}$ dm$^{-1}$(g/mol)$^{-1}$], respectively.
For the heavier molecules H$_{2}$Se$_{2}$ and H$_{2}$Te$_{2}$, values computed for the sodium D-line frequency become exceedingly large as these molecules have an excitation that is almost at resonance with this frequency. To better understand this phenomenon, we have therefore calculated the excitation energy of the first five states for these two molecules. The resulting values are compiled and presented in Table \ref{tab: H2Y2exc}. 



        % Figure environment removed
        


The computed excitation values show that in case of H$_{2}$Se$_{2}$, the relativistic CC value is significantly larger than the nonrelativistic CC because the employed frequency is quite close to the resonance frequency of the second excited state in the relativistic calculation (0.0789 a.u.), whereas it is distant from all excited states in the nonrelativistic CC calculation. For the B3LYP computations, we see that the frequency is close to the fourth excited state in both relativistic and nonrelativistic scenarios (0.0863 and 0.0889 respectively). This proximity results in large values being obtained from both calculations.


When we examine the H$_{2}$Te$_{2}$ molecule, we find the relativistic effect to be substantial for all three models and even reversing the sign of the optical rotation. For example, the nonrelativistic CC value is -263.59 [${^\circ}$ dm$^{-1}$(g/mol)$^{-1}$], but the relativistic CC is 218.83 [${^\circ}$ dm$^{-1}$(g/mol)$^{-1}$]. Besides reversing the sign, with HF also the magnitude of the optical rotation is very different in the relativistic and nonrelativistic cases. This is because the first three excited states, while being close to a transition, are triplets and hence do not contribute to the optical rotation that is in the nonrelativistic case solely due to the fourth state that is a singlet. In the relativistic case SOC makes these transitions allowed, which combined with their proximity to the sodium D-line leads to a much larger optical rotation of opposite sign than computed non-relativistically. The B3LYP values are large in both the relativistic and the NR case because the frequency is then close to the singlet state (0.0721 a.u. and 0.0777 a.u. respectively). To avoid artifacts due to the proximity of poles and the associated infinity of the real frequency-dependent response function, it is probably opportune to consider the lifetime of the excited state and use damped response theory like shown for the complex polarizability for I$_{2}$.



        \begin{table}[H]
        \begin{threeparttable}
            
        
            \center
            \caption{Excitation energy (a.u.) of the first five states for H$_{2}$Se$_{2}$ and H$_{2}$Te$_{2}$}.\label{tab: H2Y2exc}

        \setlength{\tabcolsep}{3.5mm}{
            \begin{tabular}{cccccccc}
            &State& HF &HF\tnote{a} & B3LYP &B3LYP\tnote{a} &CC  & CC\tnote{a}\\
            \hline
            H$_{2}$Se$_{2}$&1&0.0671&0.0719 &0.0639 &0.0661 &0.0789 &0.0812 \\
                           &2&0.0672&0.0719 &0.0639 &0.0661 &0.0789 &0.1016\\
                           &3&0.0699&0.0719 &0.0642 &0.0661 &0.0792 &0.1309 \\
                           &4&0.0860&0.0954 &0.0863 &0.0889 &0.0988 &0.1478 \\
                           &5&0.0911&0.1083 &0.1098 &0.1121 &0.1285 &0.1561 \\
            \hline
            H$_{2}$Te$_{2}$&1&0.0517 &0.0635 &0.0542 &0.0584 &0.0672 &0.0721 \\
                           &2&0.0517 &0.0635 &0.0543 &0.0584 &0.0673 &0.0898\\
                           &3&0.0605 &0.0635 &0.0558 &0.0584 &0.0689 &0.1143 \\
                           &4&0.0697 &0.0924 &0.0721 &0.0777 &0.0835 &0.1292 \\
                           &5&0.0873 &0.0957 &0.0931 &0.0968 &0.1098 &0.1367 \\
            \hline
            \end{tabular}}
                \begin{tablenotes}
                    \item[a]  Nonrelativistic calculation with the Levy-Leblond Hamiltonian
                \end{tablenotes}
            \end{threeparttable}

        \end{table}




\section{Conclusion}
\label{Conclusions}
In this work, we describe the formulation and implementation of the relativistic coupled cluster linear response method for static and frequency-dependent molecular property calculations, which can accurately treat both relativistic and electronic correlation effects. 
This implementation was accomplished in the GPU-accelerated coupled cluster module of the DIRAC program leveraging a framework designed to handle similar transformed Hamiltonian in subspace. This framework aids in solving both eigenvalue and linear system problems. The current code is capable of calculating excitation energies within the EOM-CCSD framework and computing the linear response function for both CC-CI and CC-CC type wave-function models. 

We have validated the implementation by assessing purely electric properties such as static and frequency-dependent polarizability for Group IIB atoms (Zn, Cd, Hg) and several diatomic molecules. Compared to 
previous Hartree-Fock linear response calculations, our current linear response calculation based on the relativistic coupled cluster approach offers  a notably improved accuracy. This enhancement is particularly evident in terms of relativistic corrections and correlation, bringing our results  closer to the experimental data. 

In this study, we also tested the indirect spin-spin coupling constant---a purely magnetic property--- for the hydrogen halide series HX(X=F, Cl, Br, I). Validation was done by reproducing the results obtained by other programs such as DALTON and CFOUR using a nonrelativistic Hamiltonian. We extended our study to explore the impact of solvent effect on the H$_{2}$Se-H$_{2}$O complex systems. Both correlation and relativistic corrections were found to have pronounced effects on the solvent shift. While CC and DFT gave  similar magnitudes for the shifts in solvent effect, the absolute spin-spin coupling constants differed significantly. This finding calls for caution when employing DFT for such calculations.

Lastly, we computed the optical rotation--- an electric and magnetic mixed property--- for chiral molecules H$_{2}$Y$_{2}$ (Y=O, S, Se, Te) at the wave-length of sodium D-line (589.29 nm). Our exploration revealed potential challenges when using this frequency for heavy molecules. We analyzed the poles of the response function by calculating the excitation energy, and advise caution when using sodium D-line for these heavier molecules in future investigations. 

A distinguishing aspect of our implementation is its use of complex algebra, which facilitates a straightforward extension of real to complex frequencies for the evaluation of the damped linear response function. We used this feature to simulate the spectrum of I$_{2}$ through the assessment of the absorption cross-section.

As a final point and perspective,  it is worth noting that our current implementation relies on the single-code tensor operation library TAL-SH. While efficient, this library is limited to using the memory capacity of a single node. Therefore, a natural 
development
is to extend the current code for the EOM-CCSD energy and linear response to use a library suited for distributed memory computing architectures, such as the ExaTENSOR library already employed for the CC energy evaluation, but still lacks some features needed in the Davidson diagonalization procedure. After resolving these issues we are optimistic that we can eliminate the single-node memory limitations and enable treatment of larger systems.

%%%%%%%%%%%%%%%%%%%%%%%%%%%%%%%%%%%%%%%%%%%%%%%%%%%%%%%%%%%%%%%%%%%%%
%% The "Acknowledgement" section can be given in all manuscript
%% classes.  This should be given within the "acknowledgement"
%% environment, which will make the correct section or running title.
%%%%%%%%%%%%%%%%%%%%%%%%%%%%%%%%%%%%%%%%%%%%%%%%%%%%%%%%%%%%%%%%%%%%%
\begin{acknowledgement}

This research used resources of the Oak Ridge Leadership Computing Facility, which is a DOE Office of Science User Facility supported under Contract DE-AC05-00OR22725 (allocations CHM160, CHM191 and CHP109). XY, LH and ASPG acknowledge funding from projects Labex CaPPA (Grant No. ANR-11-LABX-0005-01) and CompRIXS (Grant Nos. ANR-19CE29-0019 and DFG JA 2329/6-1), the I-SITE ULNE project OVERSEE and MESONM International Associated Laboratory (LAI) (Grant No. ANR-16-IDEX-0004), as well support from the French national supercomputing facilities (Grant Nos. DARI A0130801859, A0110801859, project grand challenge AdAstra GDA2210).
SC acknowledges funding from the Independent Research Fund Denmark--Natural Sciences, Research Project 2 - grant no. 7014-00258B.

\end{acknowledgement}

%%%%%%%%%%%%%%%%%%%%%%%%%%%%%%%%%%%%%%%%%%%%%%%%%%%%%%%%%%%%%%%%%%%%%
%% The same is true for Supporting Information, which should use the
%% suppinfo environment.
%%%%%%%%%%%%%%%%%%%%%%%%%%%%%%%%%%%%%%%%%%%%%%%%%%%%%%%%%%%%%%%%%%%%%


\begin{suppinfo}

The data (input/output) corresponding to the calculations of this paper are available at the Zenodo repository under DOI: \href{http://doi.org/10.5281/zenodo.8136133}{10.5281/zenodo.8136133}.

The Supporting Information is available free of charge on the \href{http://pubs.acs.org}{ACS Publications website} at DOI: \href{}{XXX}. Working equations for CCSD linear response theory and EOM-EE sigma vectors and intermediates, comparison for the BH molecule of the damped response LR-CC results obtained with DIRAC and standard LR-CC results obtained with DALTON, Indirect spin-spin coupling constants ($J$) for the hydrogen halide molecule, additional comparisons of methods for optical rotation.

\end{suppinfo}


%%%%%%%%%%%%%%%%%%%%%%%%%%%%%%%%%%%%%%%%%%%%%%%%%%%%%%%%%%%%%%%%%%%%%
%% The appropriate \bibliography command should be placed here.
%% Notice that the class file automatically sets \bibliographystyle
%% and also names the section correctly.
%%%%%%%%%%%%%%%%%%%%%%%%%%%%%%%%%%%%%%%%%%%%%%%%%%%%%%%%%%%%%%%%%%%%%
\bibliography{CC_Lin_Rsp}

\end{document}
