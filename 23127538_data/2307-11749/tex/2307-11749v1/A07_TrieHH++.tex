\section{TrieHH++}
\label{appendix:TrieHH++}
Based on Lemma 3 of~\cite{cormode2022sample}, $\triehhp$ achieves $(\epsilon, \delta)$ differential privacy when sampling rate $p_s = \alpha (1-e^{-\epsilon})$ where $ 0 < \alpha \le 1$ and $\epsilon < 1$ for $\delta = e^{-C_{\alpha}\theta}$, where $C_{\alpha}= \ln{1/\alpha}-1/(1+\alpha)$. The $\epsilon$ here is for one iteration. We used advanced composition in theorem 3.4 of~\cite{Kairouz:2017} to find the optimal $\epsilon$ per iteration which gives us $\epsagg$ of 1. We set the $\alpha$ parameter so that $\delta = 10^{-6}$. $\triehhp$ provide the analysis that shows the trade off between sampling and threshold values. We change the threshold $\theta$ from 10 to 20 and its effect on sampling rate are reported in Table~\ref{tab:TrieHH++} and~\ref{tab:TrieHH++large}.
\begin{table}[]
    \centering
    \resizebox{\columnwidth}{!}{%
    \begin{tabular}{|c|c|c|c|c|c|c|c|c|c|c|c|c|c|}
    \hline
        $\epsagg$ & \multicolumn{4}{|c|}{$1$} & \multicolumn{4}{|c|}{$0.5$} & \multicolumn{4}{|c|}{$0.25$} \\ \hline
        $T$ & $12$ & $6$  & $4$ & $3$  & $12$ & $6$  & $4$ & $3$& $12$ & $6$  & $4$ & $3$\\ \hline
         Sampling Rate  & $0.0071$ & $0.0129$ & $0.0193$& $0.0255$ & $0.0032$ & $0.0067$ & $0.0102$& $0.0138$   &$0.0016$ & $0.0034$ & $0.0053$& $0.0071$  \\ \hline
    \end{tabular}
    }
    \caption{Number of rounds effect on the sampling rate of $\triehhp$ ($\delta=10^{-6}$, $\numusers = 1.6\times10^6, \theta = 10$)}
    \label{tab:TrieHH++}
\end{table}


\begin{table}[]
    \centering
    \resizebox{\columnwidth}{!}{%
    \begin{tabular}{|c|c|c|c|c|c|c|c|c|c|c|c|c|c|}
    \hline
        $\epsagg$ & \multicolumn{4}{|c|}{$1$} & \multicolumn{4}{|c|}{$0.5$} & \multicolumn{4}{|c|}{$0.25$} \\ \hline
        $T$ & $12$ & $6$  & $4$ & $3$  & $12$ & $6$  & $4$ & $3$& $12$ & $6$  & $4$ & $3$\\ \hline
         Sampling Rate  & $0.0153
$ & $0.0305$ & $0.0449$  & $0.0589$ & $0.0078$ & $0.0159$ & $0.0239$&  $0.0319$   
         &$0.0039$ & $0.0082$ & $0.0123$& $0.0166$  \\ \hline
    \end{tabular}
    }
    \caption{Number of rounds effect on the sampling rate of $\triehhp$ ($\delta=10^{-6}$, $\numusers = 1.6\times10^6, \theta = 20$)}
    \label{tab:TrieHH++large}
\end{table}

\subsection{Single Data Point Setting for $\triehhp$}

In this section we discuss the effect of number of iterations on the utility of a single data point setting for $\triehhp$~\cite{cormode2022sample}. In Figure~\ref{fig:triehh+single}, we show the effect of different segmentation on the utility of the algorithm for a single data point per \device setting. For these experiments we used $\epsagg = 1$ and sampling rates are set based on table~\ref{tab:TrieHH++}. To evaluate the effect of different segmentation in this part we used the $\complimit = 10^7$ on the dimension. The number of heavy hitters detected by $\triehhp$ algorithm, when the number of iterations are 12 (1 char), 6 (2 char), 4 (3 char), 3 (4 char) are $[124, 223, 314, 137]$. Similar to $\triehhg$ having larger segments help with the algorithm since less number of iterations are required and consequently sampling rate becomes larger. However, by increasing the segment size to a certain point, the utility drops. The reason is by enlarging the segment length the number of prefixes that can be kept in each iteration reduces because of the dimension constraint. Thus, for the comparisons with $\ouralgorithm$ we used the best configurations which is having 4 iterations. 

% Figure environment removed

\subsection{Multiple Data Points Setting for $\triehhp$}

For our evaluation, we used the same binary encoding we described before. The total number of heavy hitters detected by $\triehhp$ algorithm, when the number of iterations are 12 (1 char), 6 (2 char), 4 (3 char), 3 (4 char), are $[714, 455, 417, 90]$ respectively. As shown in Figure~\ref{fig:triehh+multi}, in this setting, 12 iterations shows the best utility. 

To further improve the utility of $\triehhp$, we used unweighted sampling in another set of experiments. This sampling scheme leads to finding $[704, 610, 484, 102]$ heavy hitters when having 12 (1 char), 6 (2 char), 4 (3 char), 3 (4 char) iterations respectively. Figure ~\ref{fig:triehh+multicounts1} and ~\ref{fig:triehh+multitotalunweighted} shows the loss and marginal discovered counts based on the unweighted sampling scheme. As demonstrated in these figures, unweighted scheme is able to find more heavy hitters and cause less utility degradation. Also using both sampling schemes, 12 iterations shows the highest utility. Thus, for the comparisons with $\ouralgorithm$ we use unweighted sampling, and 12 iterations for $\triehhp$ and we refer to it as $\opttriehhp$.

 % Figure environment removed

 % Figure environment removed