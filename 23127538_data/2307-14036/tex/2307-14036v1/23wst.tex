\documentclass[a4paper,UKenglish,cleveref,autoref,thm-restate,pdfa]
{lipics-v2021}

%\pdfoutput=1
\hideLIPIcs
\nolinenumbers

\usepackage{mathtools,stackengine}
\usepackage{stmaryrd}
\usepackage{ebproof}
\RequirePackage{pifont}
\usepackage{tikz}
\tikzstyle{sd}[6]=[sibling distance=#1mm]
\tikzstyle{ld}[5.5]=[level distance=#1mm]
\tikzstyle{mis}[20]=[minimum size=#1mm]
\tikzstyle{is}[3]=[inner sep=#1pt]
\tikzstyle{inner}=[mis=1.8,is=0,draw,fill=white,shape=circle]
\tikzstyle{leaf}=[mis=1.8,is=0,fill=black,shape=circle]
\tikzstyle{stage}=[scale=1]

\newcommand{\btkz}{\begin{tikzpicture}}
\newcommand{\etkz}{\end{tikzpicture}}
\newcommand{\seq}[2][n]{{#2_1},\dots,{#2_{#1}}}
\newcommand{\hh}[1][.3]{\hspace{#1mm}}
\newcommand{\SET}[1]{\{\hh#1\hh\}}
\newcommand{\m}[1]{\mathsf{#1}}
\newcommand{\mc}[1]{\mathcal{#1}}
\newcommand{\mr}[1]{\mathrel{#1}}
\newcommand{\xA}{\mc{A}}
\newcommand{\xB}{\mc{B}}
\newcommand{\xF}{\mc{F}}
\newcommand{\xG}{\mc{G}}
\newcommand{\xH}{\mc{H}}
\newcommand{\xM}{\mc{M}}
\newcommand{\xT}[1]{\mc{T}(\SET{#1})}
\newcommand{\xO}{\mc{O}}
\newcommand{\xR}{\mc{R}}
\newcommand{\xS}{\mc{S}}
\newcommand{\R}{\rightarrow}
\newcommand{\MC}[2][0]{\makebox[#1mm]{#2}}
\newcommand{\ML}[2][0]{\makebox[#1mm][l]{#2}}
\newcommand{\MR}[2][0]{\makebox[#1mm][r]{#2}}

\newcommand{\AC}{\m{AC}}
\newcommand{\Fun}{{\mathcal{F}\mathsf{un}}}
\newcommand{\NN}{\mathbb{N}}
\newcommand{\OO}{\mathbb{O}}
\newcommand{\iN}{\hh\in\hh}
\newcommand{\gN}{>_\m{N}}
\newcommand{\GN}{~\gN~}
\newcommand{\gO}{>_\m{O}}
\newcommand{\ngO}{\not\gO}
\newcommand{\GO}{~\gO~}
\newcommand{\geN}{\geqslant_\m{N}}
\newcommand{\GeN}{~\geN~}
\newcommand{\geO}{\geqslant_\m{O}}
\newcommand{\GeO}{~\geO~}
\newcommand{\U}{\text{--}\hh}
\newcommand{\MAX}[1]{\m{max}(#1)}
\newcommand{\mul}{\mathsf{mul}}
\newcommand{\MUL}[1][>]{\mr{#1_\m{mul}}}

\newcommand{\si}{\mathbin{|}} % siblings
\newcommand{\h}{\m{h}} % head
\newcommand{\I}{\m{i}} % internal node
\newcommand{\hs}[1][.3]{\hspace{#1mm}}
\newcommand{\Oplus}{\hs\oplus\hs}
\newcommand{\plus}{\hs+\hs}
\newcommand{\Cdot}{\hs\cdot\hs}
\newcommand{\xTH}{\mc{T}_{\xH}}
\newcommand{\xD}{\mc{D}}
\newcommand{\xV}{\mc{V}}
\newcommand{\EVALA}[2][\alpha]{[#1]_\xA(#2)}
\newcommand{\lab}[2][\alpha]{\m{lab}_#1(#2)}
\newcommand{\xFl}{\xF_\m{lab}}
\newcommand{\xTl}{\mc{T}_\m{lab}}
\newcommand{\TAl}{\TA_\m{lab}}
\newcommand{\xHl}{\xH_\m{lab}}
\newcommand{\xRl}{\xR_\m{lab}}
\newcommand{\ACl}{\AC_{\hh\m{lab}}}
\newcommand{\DEC}{\xD\m{ec}}

\newcommand{\Rb}[1][]{\R_{#1}}
\newcommand{\Rab}[2][]{\R_{#1}^{#2}}
\newcommand{\RHAC}[1][*]{\Rab[\xH/\AC]{#1}}
\newcommand{\RHnAC}[2][*]{\Rab[\xH_#2/\AC]{#1}}
\newcommand{\RRAC}[1][n]{\Rb[\xR_#1/\AC]}
\newcommand{\nR}[1]{\,\xrightarrow{\,\MC[2]{$\scriptstyle #1$}\,}\,}
\newcommand{\tf}[1]{{\triangledown_{\!#1}}}
\newcommand{\TF}[2][]{{\triangledown_{\!#1}}(#2)}
\newcommand{\RT}[1]{\m{root}(#1)}
%\newcommand{\acrpo}{{\m{acrpo}}}
\newcommand{\acmpo}{{\m{acmpo}}}
\newcommand{\ACMPO}[1][>]{#1_{\m{acmpo}}}
\newcommand{\ACMPOm}[1][>]{#1_{\m{acmpo}}^\mul}
\newcommand{\ACm}{=_\AC^\mul}
\newcommand{\superterm}{\trianglerighteqslant}
\newcommand{\prsuperterm}{\rhd}
\newcommand{\prsubterm}{\mr{\vartriangleleft}}
\newcommand{\Emb}{{\mathcal{E}\m{mb}}}

%\newcommand{\NH}[1]{\textcolor{red}{#1}}
\newcommand{\NH}[1]{#1}
\newcommand{\AM}[1]{\textcolor{magenta}{#1}}

\bibliographystyle{plainurl}

\title{Hydra Battles and AC Termination, Revisited}

\author{Nao Hirokawa}
{School of Information Science, JAIST, Japan}
{hirokawa@jaist.ac.jp}
{https://orcid.org/0000-0002-8499-0501}
{JSPS KAKENHI Grant Numbers 22K11900}

\author{Aart Middeldorp}
{Department of Computer Science, University of Innsbruck, Austria}
{aart.middeldorp@uibk.ac.at}
{https://orcid.org/0000-0001-7366-8464}
{}

\authorrunning{N. Hirokawa and A. Middeldorp}

\Copyright{Nao Hirokawa and Aart Middeldorp}

\ccsdesc[300]{Theory of computation~Equational logic and rewriting}
\ccsdesc[500]{Theory of computation~Rewrite systems}
\ccsdesc[500]{Theory of computation~Computability}

\keywords{battle of Hercules and Hydra, term rewriting, termination}

\category{}

\acknowledgements{}

\nolinenumbers

%Editor-only macros:: begin (do not touch as author)
\EventEditors{John Q. Open and Joan R. Access}
\EventNoEds{2}
\EventLongTitle{42nd Conference on Very Important Topics (CVIT 2016)}
\EventShortTitle{CVIT 2016}
\EventAcronym{CVIT}
\EventYear{2016}
\EventDate{December 24--27, 2016}
\EventLocation{Little Whinging, United Kingdom}
\EventLogo{}
\SeriesVolume{42}
\ArticleNo{23}
%%%%%%%%%%%%%%%%%%%%%%%%%%%%%%%%%%%%%%%%%%%%%%%%%%%%%%

\begin{document}

\maketitle

\begin{abstract}
We present a termination proof for the Battle of Hercules
and Hydra represented as a rewrite system with AC symbols.
Our proof employs type introduction in connection with
many-sorted semantic labeling for AC rewriting and AC-RPO.
\end{abstract}

\section{Introduction}
\label{sec:introduction}

In the previous edition of WST we presented a faithful encoding of the
Battle of Hercules and Hydra as a rewrite system with AC symbols,
together with a new AC termination criterion based on weakly monotone
algebras. Later we recognized that this criterion is not applicable to
our encoding. A new proof, based on type introduction and
many-sorted semantic labeling for AC rewriting and
Rubio's AC-RPO~\cite{R02}, appeared
at FSCD~\cite{HM23}. In this note we present a summary of the latter.
As a matter of fact, we do not need the full power of AC-RPO. Inspired
by Steinbach's AC-KBO~\cite{S90}, we define a much weakened version of
AC-RPO and show that it is powerful enough for our purpose.

\begin{definition}
To represent Hydras, we use a signature containing a constant
symbol $\h$ representing a head, a binary symbol $\si$ for siblings, and
a unary function symbol $\I$ representing the internal nodes. We use infix
notation for $\si$ and declare it to be an \textup{AC} symbol.
\end{definition}

\begin{definition}
The \textup{TRS} $\xH$ consists of the following 14 rewrite rules:
\begin{align*}
\m{A}(n,\I(\h)) &\nR{1} \m{A}(\m{s}(n),\h) &
\m{D}(n,\I(\I(x))) &\nR{8} \I(\m{D}(n,\I(x))) \\
\m{A}(n,\I(\h \si x)) &\nR{2} \m{A}(\m{s}(n),\I(x)) &
\m{D}(n,\I(\I(x) \si y)) &\nR{9} \I(\m{D}(n,\I(x)) \si y) \\
\m{A}(n,\I(x)) &\nR{3} \m{B}(n,\m{D}(\m{s}(n),\I(x))) &
\m{D}(n,\I(\I(\h \si x) \si y)) &\nR{10} \I(\m{C}(n,\I(x)) \si y) \\
\m{C}(\m{0},x) &\nR{4} \m{E}(x) &
\m{D}(n,\I(\I(\h \si x))) &\nR{11} \I(\m{C}(n,\I(x))) \\
\m{C}(\m{s}(n),x) &\nR{5} x \si \m{C}(n,x) &
\m{D}(n,\I(\I(\h) \si y)) &\nR{12} \I(\m{C}(n,\h) \si y) \\
\I(\m{E}(x) \si y) &\nR{6} \m{E}(\I(x \si y)) &
\m{D}(n,\I(\I(\h))) &\nR{13} \I(\m{C}(n,\h)) \\
\I(\m{E}(x)) &\nR{7} \m{E}(\I(x)) &
\m{B}(n,\m{E}(x)) &\nR{14} \m{A}(\m{s}(n),x)
\tag*{\lipicsEnd}
\end{align*}
\end{definition}

The Battle is started with the term $\m{A}(\m{0},t)$ where $t$ is the term
representation of the initial Hydra. Rule 1 takes care of the dying Hydra
\btkz[ld=7,level 2/.style={sd},rotate=90]
\node[inner] at (0,0) {}
 child { node[leaf] {} };
\etkz\,. Rule 2 cuts a head without grandparent node, and so no
copying takes place. Due to the power of AC matching, the removed
head need not be the leftmost one. With rule 3, the search for locating a
head with grandparent node starts. The search is performed with the
auxiliary symbol $\m{D}$ and involves rules 8--13. When the head
to be cut is located (in rules 10--13), copying begins with the
auxiliary symbol $\m{C}$ and rules 4 and 5. The end of the copying phase
is signaled with $\m{E}$, which travels upwards with rules 6 and 7.
Finally, rule 14 creates the next stage of the Battle. Note that we make
extensive use of AC matching to simplify the search process.

\begin{theorem}
\label{thm:simulation}
If $H$ and $H'$ are the encodings in $\xT{\h,\I,\si}$ %\setminus \SET{\h}$
of successive Hydras in an arbitrary battle then
$\m{A}(n,H) \RHAC[+] \m{A}(\m{s}(n),H')$ for some $n \in \xT{\m{0},\m{s}}$.
\qed
\end{theorem}

\begin{theorem}
\label{thm:converse}
Let $H, H' \in \xT{\h,\I,\si}$ be encodings of Hydras and let
$n \in \xT{\m{0},\m{s}}$ be the encoding of a natural number. If
$\m{A}(n,H) \RHAC \m{A}(\m{s}(n),H')$ then $H$ and $H'$ are successive
Hydras in a battle.
\qed
\end{theorem}

The remaining part of this note is devoted to proving the termination
of $\xH/\AC$. We exploit the fact that the TRS $\xH$ can be seen as a TRS
over the many-sorted signature
\begin{align*}
\h &: \m{O} &
\I, \m{E} &: \m{O} \to \m{O} &
{\si} &: \m{O} \times \m{O} \to \m{O} &
\m{A}, \m{B} &: \m{N} \times \m{O} \to \m{S} \\
\m{0} &: \m{N} &
\m{s} &: \m{N} \to \m{N} &&&
\m{C}, \m{D} &: \m{N} \times \m{O} \to \m{O}
\end{align*}
where $\m{N}$, $\m{O}$ and $\m{S}$ are sort symbols. 
The type introduction technique~\cite[Corollary~3.9]{MO00}
guarantees that AC termination of $\xH$ follows from AC termination
on well-sorted terms.

\begin{theorem}
\label{thm:persistency}
A non-collapsing \textup{TRS} over a many-sorted signature is
\textup{AC} terminating if and only if the corresponding \textup{TRS} over
the unsorted version of the signature is \textup{AC} terminating.
\end{theorem}

\section{Many-Sorted Semantic Labeling modulo AC}
\label{sec:semantic labeling}

The mutual dependence between the function symbols $\m{A}$ and $\m{B}$
in rules 3 and 14 of $\xH$ makes proving termination of $\xH/\AC$ a
non-trivial task. We use the technique of semantic labeling 
(Zantema~\cite{Z95}) to resolve the dependence by labeling both
$\m{A}$ and $\m{B}$ by the ordinal value of the Hydra encoded in their
second arguments. Semantic labeling for rewriting modulo has been
investigated in \cite{OMG00}. We need, however, a version for many-sorted
rewriting since the distinction between ordinals and natural numbers is
essential for the effectiveness of semantic labeling for $\xH/\AC$.

Before introducing semantic labeling, we recall some basic
semantic definitions. An \emph{algebra} $\xA$ for an $\xS$-sorted
signature $\xF$
is a pair $(\SET{S_\xA}_{S \in \xS},\SET{f_\xA}_{f \in \xF})$, where
each $S_\xA$ is a non-empty set, called the
\emph{carrier of sort $S$}, and
each $f_\xA$ is a function of type
$f : (S_1)_\xA \times \cdots \times (S_n)_\xA \to S_\xA$, called the
\emph{interpretation function} of
$f : S_1 \times \cdots \times S_n \to S$.
A mapping that associates each variable of sort $S$ to an element in
$S_\xA$ is called an \emph{assignment}. 
We write $\xA^\xV$ for the set of all assignments.
Given an assignment $\alpha \in \xA^\xV$, the \emph{interpretation} of a
term $t$ is inductively defined as follows:
%NH to save space
%\begin{align*}
%\EVALA{t} = \begin{cases}
%\alpha(t) &\text{if $t$ is a variable} \\
%f_\xA(\EVALA{t_1},\dots,\EVALA{t_n}) &\text{if $t = f(\seq{t})$}
%\end{cases}
%\end{align*}
$\EVALA{t} = \alpha(t)$ if $t$ is a variable, and 
$\EVALA{t} = f_\xA(\EVALA{t_1},\dots,\EVALA{t_n})$ if $t = f(\seq{t})$.
Let $\xA = (\SET{S_\xA}_{S \in \xS},\SET{f_\xA}_{f \in \xF})$ be an
$\xS$-sorted $\xF$-algebra. We assume that each carrier set $S_\xA$
is equipped with a well-founded order $>_S$ such that the interpretation
functions are weakly monotone in all argument positions, and call
$(\xA,\SET{>_S}_{S \in \xS})$ a weakly monotone many-sorted algebra.
Given terms $s$ and $t$ of sort $S$, we write $s \geqslant_\xA t$ 
($s =_\xA t$) if
$[\alpha]_\xA(s) \geqslant_S [\alpha]_\xA(t)$
($[\alpha]_\xA(s) =_S [\alpha]_\xA(t)$)
holds for all $\alpha \in \xA^\xV$.

A labeling $L$ for $\xF$ consists of sets of labels
$L_f \subseteq S_\xA$ for every $f : S_1 \times \cdots \times S_n \to S$.
The labeled signature $\xFl$
consists of function symbols $f_a : S_1 \times \cdots \times S_n \to S$
for every function symbol $f : S_1 \times \cdots \times S_n \to S$ in
$\xF$ and label $a \in L_f$ together with all
function symbols $f \in \xF$ such that $L_f = \varnothing$. A
\emph{labeling} $(L,\m{lab})$ for $(\xA,\SET{>_S}_{S \in \xS})$ consists
of a labeling $L$ for the signature $\xF$ together with a mapping
$\m{lab}_f\colon (S_1)_A \times \cdots \times (S_n)_A \to L_f$ for
every function symbol $f : S_1 \times \cdots \times S_n \to S$ in
$\xF$ with $L_f \neq \varnothing$.
We call $(L,\m{lab})$ \emph{weakly monotone} if all its labeling
functions $\m{lab}_f$ are weakly monotone in all coordinates.
The mapping $\m{lab}_f$ determines the label of the root symbol $f$ of
a term $f(\seq{t})$, based on the values of its arguments $\seq{t}$.
Formally, for every assignment $\alpha \in \xA^\xV$ we define a
mapping $\m{lab}_\alpha$ inductively as follows:
\begin{gather*}
\lab{t} = \begin{cases}
t &\text{if $t \in \xV$} \\
f(\lab{t_1},\dots,\lab{t_n})
&\text{if $t = f(\seq{t})$ and $L_f = \varnothing$} \\
f_a(\lab{t_1},\dots,\lab{t_n})
&\text{if $t = f(\seq{t})$ and $L_f \neq \varnothing$}
\end{cases}
\end{gather*}
where $a$ denotes the label $\m{lab}_f(\EVALA{t_1},\dotsc,\EVALA{t_n})$.
Note that $\lab{t}$ and $t$ have the same sort.
Given a TRS $\xR$ over a (many-sorted) signature $\xF$, we define the
\emph{labeled} TRS $\xRl$ over the signature $\xFl$ as follows:
\begin{gather*}
\xRl = \SET{\lab{\ell} \R \lab{r} \mid
\text{$\ell \R r \in \xR$ and $\alpha \in A^\xV$}}
\end{gather*}
Since the AC symbol $\si$ in the encoding of the Hydra battle is
a constructor, there is no need to label it. Hence we assume
for simplicity that $L_f = \varnothing$ for every AC symbol $f \in \xF$.
The TRS $\DEC$ consists of all rewrite rules
\begin{gather*}
f_a(\seq{x}) \R f_b(\seq{x})
\end{gather*}
with $f : S_1 \times \cdots \times S_n \to S$ a function symbol in $\xF$,
$a, b \in L_f$ such that $a >_S b$, and pairwise different variables
$\seq{x}$.
A weakly monotone algebra $(\xA,>)$ is a \emph{quasi-model} of $\xR/\AC$
if $\ell \geqslant_{\xA} r$ for all rewrite rules $\ell \R r$ in $\xR$ 
and $\ell =_\xA r$ for all equations $\ell \approx r$ in $\AC$.

\begin{theorem}
\label{thm:semantic-labeling}
Let $\xR/\AC$ be a \textup{TRS} over a many-sorted signature $\xF$,
$(\xA,\SET{>_S}_{S \in \xS})$ a quasi-model of $\xR/\AC$ with a
weakly monotone labeling $(L,\m{lab})$. If $(\xRl \cup \DEC) / \AC$
is terminating then $\xR/\AC$ is terminating.
\qed
\end{theorem}

After these preliminaries, we are ready to put many-sorted semantic
labeling to the test.
Consider the many-sorted algebra $\xA$ with carriers $\NN$ for sort
$\m{N}$ and $\OO$, the set of ordinal numbers smaller than
$\epsilon_0$, for sorts $\m{O}$ and $\m{S}$ and
the following interpretation functions:
%NH to save space
%\begin{align*} 
%\m{0}_\xA &= \h_\xA = 1 & \m{s}_\xA(n) &= n+1 &
%\I_\xA(x) &= \omega^x \\
%x \si_\xA y &= x \oplus y & \m{E}_\xA(x) &= x+1 &
%\m{C}_\xA(n,x) &= x \cdot n + 1 \\
%\m{A}_\xA(n,x) &= \ML{$\m{B}_\xA(n,x) = \m{D}_\xA(n,x) = x$}
%\end{align*}
\begin{align*} 
\m{0}_\xA &= 1
& \m{s}_\xA(n) &= n+1
& \I_\xA(x) &= \omega^x
& \m{B}_\xA(n,x) &= x
& \m{D}_\xA(n,x) &= x
\\
\h_\xA &= 1
& x \si_\xA y &= x \oplus y
& \m{E}_\xA(x) &= x+1
& \m{A}_\xA(n,x) &= x
& \m{C}_\xA(n,x) &= x \cdot n + 1 
\end{align*}
Here $\oplus$ denotes natural addition on ordinals, which is
strictly monotone in both arguments.

\begin{lemma}
The algebra $(\xA,\SET{>_\m{O},>_\m{N}})$ is a quasi-model of $\xH/\AC$.
\qed
\end{lemma}

We now label $\m{A}$ and $\m{B}$ by the value of their second argument.
Let $L_\m{A} = L_\m{B} = \OO$ and $L_f = \varnothing$ for the other
function symbols $f$, and define $\m{lab}$ as follows:
\begin{gather*}
\m{lab}_{\m{A}}(n,x) = \m{lab}_{\m{B}}(n,x) = x
\end{gather*}
The labeling $(L,\m{lab})$
results in the infinite rewrite system $\xHl \cup \DEC$ with
$\xHl$ consisting of the rewrite rules
% NH might be useful:
%\allowdisplaybreaks
\begin{align*}
\m{A}_\omega(n,\I(\h)) &\nR{1} \m{A}_1(\m{s}(n),\h) &
\m{D}(n,\I(\I(x))) &\nR{8} \I(\m{D}(n,\I(x))) \\
\m{A}_{\omega^{v+1}}(n,\I(\h \si x)) &\nR{2}
\m{A}_{\omega^v}(\m{s}(n),\I(x)) &
\m{D}(n,\I(\I(x) \si y)) &\nR{9} \I(\m{D}(n,\I(x)) \si y) \\
\m{A}_{\omega^v}(n,\I(x)) &\nR{3}
\m{B}_{\omega^v}(n,\m{D}(\m{s}(n),\I(x))) &
\m{D}(n,\I(\I(\h \si x) \si y)) &\nR{10} \I(\m{C}(n,\I(x)) \si y) \\
\m{C}(\m{0},x) &\nR{4} \m{E}(x) &
\m{D}(n,\I(\I(\h \si x))) &\nR{11} \I(\m{C}(n,\I(x))) \\
\m{C}(\m{s}(n),x) &\nR{5} x \si \m{C}(n,x) &
\m{D}(n,\I(\I(\h) \si y)) &\nR{12} \I(\m{C}(n,\h) \si y) \\
\I(\m{E}(x) \si y) &\nR{6} \m{E}(\I(x \si y)) &
\m{D}(n,\I(\I(\h))) &\nR{13} \I(\m{C}(n,\h)) \\
\I(\m{E}(x)) &\nR{7} \m{E}(\I(x)) &
\m{B}_{v+1}(n,\m{E}(x)) &\nR{14} \m{A}_v(\m{s}(n),x)
\end{align*}
for all $v \in \OO$ and $\DEC$ consisting of the rewrite rules
\begin{align*}
\m{A}_v(n,x) &\,\R\, \m{A}_w(n,x) &
\m{B}_v(n,x) &\,\R\, \m{B}_w(n,x)
\end{align*}
for all $v, w \in \OO$ with $v > w$.
According to \cref{thm:semantic-labeling}, the AC termination of $\xH$
on many-sorted terms follows from the AC termination of $\xHl \cup \DEC$.

\begin{corollary}
\label{cor:termination H labeled => H}
If $\xHl \cup \DEC$ is \textup{AC} terminating then $\xH$ is
\textup{AC} terminating on sorted terms.
\qed
\end{corollary}

\section{AC-MPO}

In order to show AC termination of $\xHl \cup \DEC$ we use a simplified
version of AC-RPO. Let $\xF_\AC$ be the set of AC symbols in $\xF$.
Given a non-variable term $t = f(\seq{t})$, the multiset $\TF{t}$ of
its arguments is defined inductively as follows:
\begin{align*}
\TF{t} &= \TF[f]{t_1} \uplus \cdots \uplus \TF[f]{t_n} \\
\TF[f]{t} &= \begin{cases}
\TF[f]{t_1} \uplus \TF[f]{t_2} &\text{if $t = f(t_1,t_2)$ and
$f \in \xF_\AC$} \\
\SET{t} &\text{otherwise}
\end{cases}
\end{align*}
For example, if $+$ is an AC symbol, we have
$\TF[+]{\m{a} + (\m{b} + x)} = \SET{\m{a},\m{b},x}$. If $f$ is a non-AC
symbol, we have $\TF{f(\seq{t})} = \SET{\seq{t}}$.

\begin{definition}
\label{def:acrpo}
Let $>$ be a precedence. We define $\ACMPO$
inductively as follows: $s \ACMPO t$ if
$s \notin \xV$
and one of the following conditions holds:
\begin{enumerate}
\item
$\TF{s} \ACMPOm[\geqslant] \SET{t}$,
%$s_i \geqslant_\acmpo t$ for some $1 \leqslant i \leqslant n$,
\smallskip
\item
$\RT{s} > \RT{t}$ and $\SET{s} \ACMPOm \TF{t}$,
%$t = g(\seq[m]{t})$, $f > g$, and $s >_\acmpo t_j$ for 
%all $1 \leqslant j \leqslant m$,
\smallskip
\item
$\RT{s} = \RT{t}$ and $\TF{s} \ACMPOm \TF{t}$.
%$t = f(\seq{t})$, and $\tf{}(s) >_\acmpo^\mul \tf{}(t)$.
\smallskip
\end{enumerate}
Here $\ACMPO[\geqslant]$ is the union of $\ACMPO$ and $=_\AC$. Moreover,
$=_\AC$ is used instead of $=$ in the definition of multiset extension.
\end{definition}

Note that if there are no AC symbols, the above definition reduces to the
original recursive path order of Dershowitz~\cite{D82}, nowadays known as
the \emph{multiset path order}. Hence the simplified AC-RPO will be
called AC-MPO.

We assume that AC symbols are minimal in a given precedence. Without the
assumption, the relation $\ACMPO$ is not closed under contexts. To see
this, consider the example of \cite[Section~3]{R02}: Let $>$ be the
precedence with ${+} > \m{c}$, where $+$ is an AC symbol. The relation
$\m{a} + \m{a} \ACMPO \m{c}$ holds, but
$\m{a} + (\m{a} + \m{a}) \ACMPO \m{a} + \m{c}$ does not.
Note that due to the minimality requirement, different AC symbols are
incomparable. In fact, if the AC symbols $\times$ and $+$ satisfy
${\times} > {+}$ then $x \times y \ACMPO x + y$ but not
$z \times (x \times y) \ACMPO z \times (x + y)$.

\begin{theorem}
\label{thm:acrpo}
If \textup{AC} symbols are minimal in the precedence $>$ then
$\ACMPO$ is an incremental \textup{AC}-compatible rewrite order with
the subterm property.
\qed
\end{theorem}

As a consequence, $\ACMPO$ is an AC-compatible reduction order when
the underlying signature is finite. This also holds for
infinite signatures, provided the precedence $>$
is well-founded and there are only finitely many AC symbols.
This extension is important because the signature of $\xHl$ is
infinite. Below, we will formally prove the correctness of the extension,
by adopting the approach of \cite{MZ97}.

A strict order $>$ on a set $A$ is a \emph{partial well-order} if for
every infinite sequence $a_0, a_1, \dots$ of elements in $A$ there
exist indices $i$ and $j$ such that $i < j$ and $a_i \leqslant a_j$.
Well-founded total orders (\emph{well-orders}) are partial well-orders.
Given a partial well-order $>$ on $\xF$, the \emph{embedding} TRS
$\Emb(\xF,>)$ consists of the rules $f(\seq{x}) \to x_i$ for
every $n$-ary function symbol and $1 \leqslant i \leqslant n$, together
with the rules
$f(\seq{x}) \to g(x_{i_1},\dots,x_{i_m})$
for all function symbols $f$ and $g$ with arities $m$ and $n$ such
that $f > g$, and indices
$1 \leqslant i_1 < i_2 < \cdots < i_m \leqslant n$.
Here $\seq{x}$ are pairwise distinct variables.

\begin{theorem}[\textnormal{\cite[Theorem~5.3]{MZ97}}]
\label{thm:MZ97}
A rewrite order $>$ is well-founded if 
$\Emb(\xF,{\sqsupset}) \subseteq {>}$ for some partial well-order
$\sqsupset$.
\qed
\end{theorem}

\begin{theorem}
\label{thm:acrpo infinite}
Consider a signature $\xF$ with only finitely many \textup{AC} symbols
that are minimal in a given well-founded precedence $>$. The relation
$\ACMPO$ is an \textup{AC}-compatible reduction order.
\end{theorem}

\begin{proof}
We only need to show 
well-foundedness of $>_\acmpo$ because the other properties follow by
\cref{thm:acrpo}.
Let $\sqsupset$ be an arbitrary partial well-order that
contains $>$ and in which AC symbols are minimal.
% AM if the extension is a well-order then there can be only one AC symbol
%Trivially, $\sqsupset$ is a partial well-order.
The
%NH \acrpo -> \acmpo several times
inclusion $\Emb(\xF,{\sqsupset}) \subseteq {\sqsupset_{\NH{\acmpo}}}$
is easily verified. Hence the well-foundedness of $\sqsupset_{\NH{\acmpo}}$
is obtained from \cref{thm:MZ97}. Since ${>} \subseteq {\sqsupset}$, the
%NH AC-RPO -> AC-MPO
incrementality of \NH{AC-MPO} yields
${>_{\NH{\acmpo}}} \subseteq {\sqsupset_{\NH{\acmpo}}}$. It follows that 
$>_{\NH{\acmpo}}$ is well-founded.
\end{proof}

We show the termination of $\xHl \cup \DEC$ by AC-RPO. To this end,
we consider the following precedence $>$ on the labeled signature:
\begin{alignat*}{2}
\m{A}_v &> \m{A}_w &\quad&
\text{for all $v, w \in \OO$ with $v > w$} \\
\m{B}_v &> \m{B}_w &\quad&
\text{for all $v, w \in \OO$ with $v > w$} 
\\
\m{B}_{v+1} &> \m{A}_v > \m{B}_v &&
\text{for all $v \in \OO$} \\
\m{B}_0 &> \ML{$\m{s} > \m{D} > \m{C} > \I > \m{E} > \si$}
\end{alignat*}
Note that $>$ is well-founded and the only AC symbol $\si$ is minimal.
In order to ease the compatibility
verification we employ the following simple criterion.

\begin{lemma}
\label{lem:root}
Let $\ell \to r$ be a rewrite rule and let $>$ be a precedence.
If $\RT{\ell} > g$ for all function symbols $g$ in $r$ then
$\ell \ACMPO r$.
\qed
\end{lemma}

\begin{theorem}
\label{thm:termination of H labeled}
$\xHl \cup \DEC \,\subseteq\, {>_{\NH{\acmpo}}}$
\end{theorem}

\begin{proof}
\cref{lem:root} applies to all rules of $\xHl \cup \DEC$, except 5\,--\,9.
We consider rule 6 here; the other rewrite rules are handled in a similar
fashion. 
Since case (1) of \cref{def:acrpo} yields $\m{E}(x) \ACMPO x$, we have
\(
\TF{\m{E}(x) \si y} = \SET{\m{E}(x),y} \ACMPOm \SET{x,y} = \TF{x \si y}
\).
Thus $\m{E}(x) \si y \ACMPO x \si y$ follows by case (3).
Using case (3) again, we obtain $\I(\m{E}(x) \si y) \ACMPO \I(x \si y)$.
Because of $\I > \m{E}$, the desired orientation $\I(\m{E}(x) \si y)
\ACMPO \I(x \si y)$ is concluded by case (2).
\end{proof}

\begin{theorem}
\label{thm:termination H labeled}
The \textup{TRS} $\xHl \cup \DEC$ is \textup{AC} terminating.
\qed
\end{theorem}

Hence, the TRS $\xH/\AC$ is terminating.

\section{Concluding Remarks}

We presented a termination proof of the recent encoding of the
Battle of Hydra and Hercules as a TRS with AC symbols. Compared to
\cite{HM23}, the new ingredient in this paper is a much weakened version
of AC-RPO. This might ease future formalization efforts.

In the previous edition of WST we presented a termination proof of the
TRS $\xH$ using a new AC termination criterion based on weakly monotone
algebras, which we recall here.

\begin{theorem}
\label{thm:AC termination}
A \textup{TRS} $\xR$ over a finite many-sorted signature $\xF$ is
\textup{AC} terminating if there exists a totally ordered simple monotone
$\xF$-algebra $(\xA,>)$ such that $\xR \subseteq {>_\xA}$, 
$\AC \subseteq {=_\xA}$, and $f_\xA$ is strictly monotone for all
\textup{AC} symbols $f$.
\end{theorem}

Here an $\xS$-sorted $\xF$-algebra 
$\xA = (\SET{S_\xA}_{S \iN \xS},\SET{f_\xA}_{f \iN \xF})$ equipped
with a strict order $>$ on the union of all carriers $S_\xA$
is \emph{simple monotone} if
every carrier $S_\xA$ is non-empty,
$(S_i)_\xA \subseteq S_\xA$ for all
$f : S_1 \times \cdots \times S_n \to S$ in $\xF$ and
$1 \leqslant i \leqslant n$, and
every interpretation function $f_\xA$ is weakly monotone and simple. The
latter amounts to the requirement
$f_\xA(a_1,\dots,a_i,\dots,a_n) \geqslant a_i$
for all $1 \leqslant i \leqslant n$ and
$(\seq{a}) \in (S_1)_\xA \times \cdots \times (S_n)_\xA$.

When applying \cref{thm:AC termination} to the \emph{many-sorted} TRS $\xH$
we used the following interpretation for the symbol $\m{D}$:
\begin{align*}
\m{D}_\xA((n_1,n_2,n_3),(x_1,x_2,x_3)) &=
(n_1 + x_1, n_2 + x_2, n_2 + n_3 + x_2 + x_3)
\end{align*}
with $(n_1,n_2,n_3) \in (\NN \setminus \SET{0,1}) \times \NN \times \NN$
and $(x_1,x_2,x_3) \in (\OO \setminus \SET{0,1}) \times \NN \times \NN$.
This is however not weakly monotone as
$(1,0,0) > (0,0,1)$ but
\begin{align*}
\m{D}_\xA((1,0,0),(\omega,0,0)) &= (\omega,0,0) &
\m{D}_\xA((0,0,1),(\omega,0,0)) &= (\omega,0,1)
\end{align*}
with $(\omega,0,0) < (\omega,0,1)$. Hence the termination proof in
last year's WST paper is wrong. The problem is that the lexicographic
product of weakly monotone orders is in general not weakly monotone
(\cite[Example~26]{ZWM15}).
Note that the second and third components were introduced to orient
rules 6\,--\,9. In the new proof AC-MPO orients the rules by taking the
precedence $\m{D} > \m{C} > \I > \m{E} > \si$.

%\bibliography{references}
\bibliography{references-short}

\end{document}
