



\documentclass[journal]{IEEEtran}
%
% If IEEEtran.cls has not been installed into the LaTeX system files,
% manually specify the path to it like:
% \documentclass[journal]{../sty/IEEEtran}












% *** CITATION PACKAGES ***
%
%\usepackage{cite}
% cite.sty was written by Donald Arseneau
% V1.6 and later of IEEEtran pre-defines the format of the cite.sty package
% \cite{} output to follow that of the IEEE. Loading the cite package will
% result in citation numbers being automatically sorted and properly
% "compressed/ranged". e.g., [1], [9], [2], [7], [5], [6] without using
% cite.sty will become [1], [2], [5]--[7], [9] using cite.sty. cite.sty's
% \cite will automatically add leading space, if needed. Use cite.sty's
% noadjust option (cite.sty V3.8 and later) if you want to turn this off
% such as if a citation ever needs to be enclosed in parenthesis.
% cite.sty is already installed on most LaTeX systems. Be sure and use
% version 5.0 (2009-03-20) and later if using hyperref.sty.
% The latest version can be obtained at:
% http://www.ctan.org/pkg/cite
% The documentation is contained in the cite.sty file itself.


\usepackage[numbers,sort&compress]{natbib}

\usepackage{algorithm}
\usepackage{algorithmic}

\usepackage{pgf}%figure
\usepackage{tikz}
\usetikzlibrary{arrows, decorations.pathmorphing, backgrounds, positioning, fit, petri, automata}
\definecolor{yellow1}{rgb}{1,0.8,0.2}

\usepackage{makecell}

\usepackage{authblk}%作者标注
\usepackage[T1]{fontenc}
\usepackage{inputenc}

\usepackage{amsmath,amsthm,amsfonts,amssymb,amsbsy,amsopn,amstext}
\usepackage{graphicx,color}
\usepackage{mathbbol,mathrsfs}
\usepackage{float,ccaption}
\usepackage{indentfirst}
\usepackage{subfigure}

\usepackage{booktabs}
\usepackage{threeparttable}
\usepackage{multirow}
\usepackage{diagbox}



% *** GRAPHICS RELATED PACKAGES ***
%
\ifCLASSINFOpdf
  % \usepackage[pdftex]{graphicx}
  % declare the path(s) where your graphic files are
  % \graphicspath{{../pdf/}{../jpeg/}}
  % and their extensions so you won't have to specify these with
  % every instance of \includegraphics
  % \DeclareGraphicsExtensions{.pdf,.jpeg,.png}
\else
  % or other class option (dvipsone, dvipdf, if not using dvips). graphicx
  % will default to the driver specified in the system graphics.cfg if no
  % driver is specified.
  % \usepackage[dvips]{graphicx}
  % declare the path(s) where your graphic files are
  % \graphicspath{{../eps/}}
  % and their extensions so you won't have to specify these with
  % every instance of \includegraphics
  % \DeclareGraphicsExtensions{.eps}
\fi
% graphicx was written by David Carlisle and Sebastian Rahtz. It is
% required if you want graphics, photos, etc. graphicx.sty is already
% installed on most LaTeX systems. The latest version and documentation
% can be obtained at: 
% http://www.ctan.org/pkg/graphicx
% Another good source of documentation is "Using Imported Graphics in
% LaTeX2e" by Keith Reckdahl which can be found at:
% http://www.ctan.org/pkg/epslatex
%
% latex, and pdflatex in dvi mode, support graphics in encapsulated
% postscript (.eps) format. pdflatex in pdf mode supports graphics
% in .pdf, .jpeg, .png and .mps (metapost) formats. Users should ensure
% that all non-photo figures use a vector format (.eps, .pdf, .mps) and
% not a bitmapped formats (.jpeg, .png). The IEEE frowns on bitmapped formats
% which can result in "jaggedy"/blurry rendering of lines and letters as
% well as large increases in file sizes.
%
% You can find documentation about the pdfTeX application at:
% http://www.tug.org/applications/pdftex





% *** MATH PACKAGES ***
%
%\usepackage{amsmath}
% A popular package from the American Mathematical Society that provides
% many useful and powerful commands for dealing with mathematics.
%
% Note that the amsmath package sets \interdisplaylinepenalty to 10000
% thus preventing page breaks from occurring within multiline equations. Use:
%\interdisplaylinepenalty=2500
% after loading amsmath to restore such page breaks as IEEEtran.cls normally
% does. amsmath.sty is already installed on most LaTeX systems. The latest
% version and documentation can be obtained at:
% http://www.ctan.org/pkg/amsmath





% *** SPECIALIZED LIST PACKAGES ***
%
%\usepackage{algorithmic}
% algorithmic.sty was written by Peter Williams and Rogerio Brito.
% This package provides an algorithmic environment fo describing algorithms.
% You can use the algorithmic environment in-text or within a figure
% environment to provide for a floating algorithm. Do NOT use the algorithm
% floating environment provided by algorithm.sty (by the same authors) or
% algorithm2e.sty (by Christophe Fiorio) as the IEEE does not use dedicated
% algorithm float types and packages that provide these will not provide
% correct IEEE style captions. The latest version and documentation of
% algorithmic.sty can be obtained at:
% http://www.ctan.org/pkg/algorithms
% Also of interest may be the (relatively newer and more customizable)
% algorithmicx.sty package by Szasz Janos:
% http://www.ctan.org/pkg/algorithmicx




% *** ALIGNMENT PACKAGES ***
%
%\usepackage{array}
% Frank Mittelbach's and David Carlisle's array.sty patches and improves
% the standard LaTeX2e array and tabular environments to provide better
% appearance and additional user controls. As the default LaTeX2e table
% generation code is lacking to the point of almost being broken with
% respect to the quality of the end results, all users are strongly
% advised to use an enhanced (at the very least that provided by array.sty)
% set of table tools. array.sty is already installed on most systems. The
% latest version and documentation can be obtained at:
% http://www.ctan.org/pkg/array


% IEEEtran contains the IEEEeqnarray family of commands that can be used to
% generate multiline equations as well as matrices, tables, etc., of high
% quality.




% *** SUBFIGURE PACKAGES ***
%\ifCLASSOPTIONcompsoc
%  \usepackage[caption=false,font=normalsize,labelfont=sf,textfont=sf]{subfig}
%\else
%  \usepackage[caption=false,font=footnotesize]{subfig}
%\fi
% subfig.sty, written by Steven Douglas Cochran, is the modern replacement
% for subfigure.sty, the latter of which is no longer maintained and is
% incompatible with some LaTeX packages including fixltx2e. However,
% subfig.sty requires and automatically loads Axel Sommerfeldt's caption.sty
% which will override IEEEtran.cls' handling of captions and this will result
% in non-IEEE style figure/table captions. To prevent this problem, be sure
% and invoke subfig.sty's "caption=false" package option (available since
% subfig.sty version 1.3, 2005/06/28) as this is will preserve IEEEtran.cls
% handling of captions.
% Note that the Computer Society format requires a larger sans serif font
% than the serif footnote size font used in traditional IEEE formatting
% and thus the need to invoke different subfig.sty package options depending
% on whether compsoc mode has been enabled.
%
% The latest version and documentation of subfig.sty can be obtained at:
% http://www.ctan.org/pkg/subfig




% *** FLOAT PACKAGES ***
%
%\usepackage{fixltx2e}
% fixltx2e, the successor to the earlier fix2col.sty, was written by
% Frank Mittelbach and David Carlisle. This package corrects a few problems
% in the LaTeX2e kernel, the most notable of which is that in current
% LaTeX2e releases, the ordering of single and double column floats is not
% guaranteed to be preserved. Thus, an unpatched LaTeX2e can allow a
% single column figure to be placed prior to an earlier double column
% figure.
% Be aware that LaTeX2e kernels dated 2015 and later have fixltx2e.sty's
% corrections already built into the system in which case a warning will
% be issued if an attempt is made to load fixltx2e.sty as it is no longer
% needed.
% The latest version and documentation can be found at:
% http://www.ctan.org/pkg/fixltx2e


%\usepackage{stfloats}
% stfloats.sty was written by Sigitas Tolusis. This package gives LaTeX2e
% the ability to do double column floats at the bottom of the page as well
% as the top. (e.g., "% Figure environment removed
The following theorem establishes the convergence rate of VRA-GT method in the mean square sense.
\begin{thm}\label{thm:VRA-GT conv-1}
Suppose that (a) Assumptions \ref{ass:function}-\ref{ass:matrix} hold, (b) $f(x)$ is $\mu$-strongly convex and there exists a constant $\sigma$ such that $\mathbb{E}\left[\|\zeta_{i,t}\|^2\right]\le \sigma^2$, $\mathbb{E}\left[\|\xi_{i,t}\|^2\right]\le \sigma^2$, (c) $\gamma<1$, $\eta_k=\frac{a_1}{k^\eta}$, $\beta_k=\frac{a_2}{k^\beta}$, $\alpha_k=\frac{a_3}{k^\alpha}$, where $a_1,a_2,a_3\in (0,1]$, $\eta,\alpha,\beta\in(0.5,1)$ and $\alpha,\beta$ satisfy $\alpha>\frac{1+\beta}{2}$.
Then
\begin{equation}\label{ie-4}
\mathbb{E}\left[v_{k+1}\right]\le\mathcal{O}\left(\frac{1}{k^{\min\{2\beta-\alpha,\beta+\eta-\alpha\}}}\right),
\end{equation}
where $v_{k+1}=\left\|\mathbf{y}_{k+1}^{'}-v\bar{y}_{k+1}^{'}\right\|_C^2+\left\|\mathbf{x}_{k+1}-\mathbf{1}\bar{x}_{k+1}\right\|_R^2+c_*\left\|\bar{x}_{k+1}-x^*\right\|^2$.
\end{thm}
\begin{proof}
Under the conditions (b) and (c), Theorem \ref{thm:VRA-conv} implies
$$\mathbb{E}\left[\|\omega_k\|^2\right]\le c_\eta\eta_t$$
for some $c_\eta>0$.
Taking exception on both sides of (\ref{ie-3}) and substituting above relation into it,
\begin{align*}
&\mathbb{E}\left[v_{k+1}\right]\notag\\
&\le(1+q_k)\mathbb{E}\left[v_k\right]-p_k\left(\left\|\mathbf{y}_{k}^{'}-v\bar{y}_{k}^{'}\right\|_C^2+\left\|\mathbf{x}_{k}-\mathbf{1}\bar{x}_{k}\right\|_R^2\right.\notag\\
&\quad+f(\bar{x}_k)- f(x^*)\bigg)\notag\\
&\quad+\left(c_1+c_2+\frac{2\|u\|^2}{n^2}c_*\right)\beta_k^2\mathbb{E}\left[\left\|\xi_k^R\right\|^2\right]\notag\\
&\quad +\left(4c_3\left(c_1\alpha_k^2+\frac{\alpha_k^2c_2}{\beta_k\rho_R}+\frac{2\alpha_k^2\|u\|^2}{n^2}c_*\right)\right.\notag\\
&\quad\left.+3c_*\frac{c_3\|u\|^2}{n^2}\beta_k\right)\sum_{t=1}^{k}\rho_\gamma^{k-t}c_\eta\eta_t\notag\\
&\le(1+q_k)\mathbb{E}\left[v_k\right]-p_k\left(\left\|\mathbf{y}_{k}^{'}-v\bar{y}_{k}^{'}\right\|_C^2+\left\|\mathbf{x}_{k}-\mathbf{1}\bar{x}_{k}\right\|_R^2\right.\notag\\
&\quad+\mu\mathbb{E}\left[\left\|\bar{x}_{k+1}-x^*\right\|^2\right]\bigg)\notag\\
&\quad+\left(c_1+c_2+\frac{2\|u\|^2}{n^2}c_*\right)\beta_k^2n\sigma^2\notag\\
&\quad +\left(4c_3\left(c_1\alpha_k^2+\frac{\alpha_k^2c_2}{\beta_k\rho_R}+\frac{2\alpha_k^2\|u\|^2}{n^2}c_*\right)\right.\notag\\
&\quad\left.+3c_*\frac{c_3\|u\|^2}{n^2}\beta_k\right)\mathcal{O}\left(\eta_k\right)\notag\\
&\le\left(1+q_k-p_k\min\left\{1,\frac{\mu}{c_*}\right\}\right)\mathbb{E}\left[v_k\right]\notag\\
&\quad+\left(c_1+c_2+\frac{2\|u\|^2}{n^2}c_*\right)\beta_k^2n\sigma^2\notag\\
&\quad +\left(4c_3\left(c_1\alpha_k^2+\frac{\alpha_k^2c_2}{\beta_k\rho_R}+\frac{2\alpha_k^2\|u\|^2}{n^2}c_*\right)\right.\notag\\
&\quad\left.+3c_*\frac{c_3\|u\|^2}{n^2}\beta_k\right)\mathcal{O}\left(\eta_k\right),
\end{align*}
where the second inequality follows from the strong convexity of $f(x)$, the fact $\mathbb{E}\left[\|\xi_{i,t}\|^2\right]\le \sigma^2$ and the relation $\sum_{t=1}^{k}\rho_\gamma^{k-t}c_\eta\eta_t\le \mathcal{O}\left(\eta_k\right)$ (\cite[Lemma 3 in Appendix A]{zhao2020asymptotic}). By condition (c), 
\begin{equation*}
1+q_k-p_k\min\left\{1,\frac{\mu}{c_*}\right\}\le 1-\mathcal{O}\left(\frac{1}{k^\alpha}\right)
\end{equation*}
and then
\begin{align*}
&\mathbb{E}\left[v_{k+1}\right]\notag\\
&\le\left(1-\mathcal{O}\left(\frac{1}{k^\alpha}\right)\right)\mathbb{E}\left[v_k\right]+\mathcal{O}\left(\frac{1}{k^{2\beta}}+\frac{1}{k^{\beta+\eta}}\right).
\end{align*}
Applying \cite[Lemma 5 in Chapter 2]{polyak1987Introduction}  on above relation, we arrive (\ref{ie-4}).
\end{proof}

Theorem \ref{thm:VRA-GT conv-1} establishes the convergence rate of VRA-GT method when the variance of the information-sharing noise is bounded and the objective function is strongly convex. Particularly, setting $\eta=\beta=1-5/8\epsilon$ and  $\alpha=1-0.25\epsilon$,  the convergence rate of VRA-GT is  $\mathcal{O}\left(\frac{1}{k^{1-\epsilon}}\right)$, where $\epsilon$ can close to zero infinitely. Moreover, Theorem \ref{thm:VRA-GT conv-1} may complement the convergence rate result of arriving in the optimal solution's neighborhood \cite{Sri2011async,pu2020robust,Chen2022Priv}.  

\section{Experimental Results}\label{sec:num-exm}



In this section, we perform a simulation study to illustrate our theoretic findings on the convergence properties of VRA-GT method.  Consider the ridge regression problem \cite{wangGT2022}:
\begin{equation}\label{sim-pro}
\min_{x\in\mathbb{R}^d} ~f(x)=\sum_{j=1}^n\left\|M_j x-v_j\right\|^2+r\|x\|^2,\\
\end{equation}
where $f_j(x)\define\left\|w_j^\intercal x-v_j\right\|^2+r\|x\|^2$ is the objective function of agent $j$, $M_j\in \mathbb{R}^{d_1\times d}$ is  the measurement matrix, $v_j\in \mathbb{R}^{d_1}$  is a noisy measurement, $r$ is the regularization parameter.
%where $\gamma=1$ is a penalty parameter.
In problem (\ref{sim-pro}),
each agent $ i\in\mathcal{V}$ has access to sample $(M_i,v_i)$ given by the linear model
$v_i=M_i^\intercal \tilde{x}+\nu_i,$
where  $\nu_i $ is the measurement noise and $\tilde{x}$ is  an unknown parameter. 

In this experiment, the settings of predetermined parameters and network topology follow from \cite{wangGT2022}. We make $r=0.05$, $d_1=3$, $d=2$, $M_i$ is generated from a uniform distribution in the unit $\mathbb{R}^{d_1\times d}$ space, $\nu_i$ follows an i.i.d. Gaussian process
with zero mean and unit variance, $\tilde{x}$ is evenly located in $[1,~10]^d$ for $\forall i\in\mathcal{V}$.  The directed graph $\mathcal{G}$ made up of 100 agents is generated by adding random links to a ring network, where a directed link exists between any two nonadjacent nodes with a probability $p=0.3$. 
For $\forall i\in \mathcal{V}$,
$\mathcal{G}_\mathbf{R}=\mathcal{G}_\mathbf{C}=\mathcal{G}$ and
\begin{equation*}
\begin{aligned}
&\mathbf{R}_{ij}=\left\{
\begin{aligned}
&\frac{1}{|\mathcal{N}_{\mathbf{R},i}^{\text{in}}|+1},\quad j\in \mathcal{N}_{\mathbf{R},i}^{\text{in}},\\
&1-\sum_{j\in\mathcal{N}_{\mathbf{R},i}^{\text{in}}}\mathbf{R}_{ij},\quad j=i,
\end{aligned}\right.
\\
&\mathbf{C}_{ji}=\left\{
\begin{aligned}
&\frac{1}{|\mathcal{N}_{\mathbf{C},i}^{\text{out}}|+1},\quad j\in\mathcal{N}_{\mathbf{C},i}^{\text{out}},\\
&1-\sum_{j\in\mathcal{N}_{\mathbf{C},i}^{\text{out}}}\mathbf{C}_{ji},\quad j=i,
\end{aligned}\right.
\end{aligned}
\end{equation*}
where $|\mathcal{N}_{\mathbf{R},i}^{\text{in}}|$ and $|\mathcal{N}_{\mathbf{C},i}^{\text{out}}|$ are the  cardinality of $\mathcal{N}_{\mathbf{R},i}^{\text{in}}$ and $\mathcal{N}_{\mathbf{C},i}^{\text{out}}$.





We run VRA-GT and the algorithms proposed in \cite{pu2020robust,wangGT2022} for 100 times and calculate the average as well as the variance of the optimization error $\sum_{j=1}^n \left\|x_{i,k}-x^*\right\|^2$ as a function of the iteration index $k$. We set  $\gamma=0.8$, $\beta_{k}=\frac{0.1}{1+k^{0.6}}$ and $\alpha_k=\frac{0.1}{1+k^{0.9}}$ for VRA-GT, $\gamma=0.5$, $\eta=0.01$ and $\alpha=0.01$ for the algorithms proposed in \cite{pu2020robust} (i.e. Robust push pull method),  $\gamma_k=\frac{1}{1+k^{0.7}}$ and $\lambda_k=\frac{1}{1+0.7k^{0.9}}$ for the methods proposed in \cite{wangGT2022}.  Their performance under Gaussian sharing-information noise with variance $\sigma^2_\xi=\sigma^2_\zeta=1,25,50$ are depicted in Figure \ref{fig-1}, where the solid curve, dot curve, dash-dot carve and dashed curve display the evaluation of VAR-GT method and the methods proposed in \cite{pu2020robust,wangGT2022} respectively.

In summary, the displayed algorithms in Figure \ref{fig-1} are all robust to the information-sharing noise with different variance (i.e. 1, 25, 50) and can converge to optimal solution with different accuracy. In the initial stage of iteration,  R-Push-Pull method has more faster convergence rate due to the stepsize and factors added on coupling weight being constant, especially for the noise with smaller variance as shown in Figure \ref{fig-1} (a). With the iterations increasing, VRA-GT and the proposed methods in \cite{wangGT2022} have preferable optimization accuracy than R-Push-Pull method for the noise with different variance levels, which can be attributed to the noise suppressing effect of decreasing factors. Obviously,  VRA-GT has the best convergence performance among the displayed methods in Figure \ref{fig-1} since VRA provides more accurate gradient-tracking result by reducing the gradient-estimation noise variance.
%{RL_d_grad_e_6-eps-converted-to}






 





\section*{Acknowledgment}

The authors thank Professor Yongqiang Wang for the discussions on the proof of Theorem 2. The research is supported by National Key R$\&$D Program of China No. 2022YFA1004000, the NSFC \#11971090 and  Fundamental Research Funds for the Central Universities   DUT22LAB301.
% if have a single appendix:
%\appendix[Proof of the Zonklar Equations]
% or
%\appendix  % for no appendix heading
% do not use \section anymore after \appendix, only \section*
% is possibly needed

% use appendices with more than one appendix
% then use \section to start each appendix
% you must declare a \section before using any
% \subsection or using \label (\appendices by itself
% starts a section numbered zero.)
%
\bibliographystyle{IEEEtran}
\bibliography{mybib}


\end{document}


