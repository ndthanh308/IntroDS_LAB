\documentclass[10pt,twocolumn,letterpaper]{article}

\usepackage{iccv}
\usepackage{times}
\usepackage{epsfig}
\usepackage{graphicx}
\usepackage{amsmath}
\usepackage{amssymb}
\usepackage{epsfig}
\usepackage{graphicx}
\usepackage{amsmath}
\usepackage{amssymb}
\usepackage{algorithm}
\usepackage{algorithmic}
\usepackage[switch]{lineno}
\usepackage{subcaption}
\usepackage{makecell, multirow, tabularx}
\usepackage{booktabs}
\usepackage{cite}
\usepackage{color}
\usepackage{amsmath}
\definecolor{hollywoodcerise}{rgb}{0.96, 0.0, 0.63}
\definecolor{lasallegreen}{rgb}{0.03, 0.47, 0.19}
\definecolor{hanpurple}{rgb}{0.32, 0.09, 0.98}
\definecolor{green(pigment)}{rgb}{0.0, 0.65, 0.31}
\usepackage{amssymb}% http://ctan.org/pkg/amssymb
\usepackage{pifont}% http://ctan.org/pkg/pifont
\newcommand{\cmark}{\ding{51}}%
\newcommand{\xmark}{\ding{55}}%
%\usepackage{hyperref}
\usepackage{amsmath}
% Include other packages here, before hyperref.
\newcommand*{\affmark}[1][*]{\textsuperscript{#1}}
\definecolor{hollywoodcerise}{rgb}{0.96, 0.0, 0.63}
\definecolor{lasallegreen}{rgb}{0.03, 0.47, 0.19}
\definecolor{hanpurple}{rgb}{0.32, 0.09, 0.98}
\definecolor{green(pigment)}{rgb}{0.0, 0.65, 0.31}
% If you comment hyperref and then uncomment it, you should delete
% egpaper.aux before re-running latex.  (Or just hit 'q' on the first latex
% run, let it finish, and you should be clear).
% \usepackage[pagebackref=true,breaklinks=true,letterpaper=true,colorlinks,bookmarks=false]{hyperref}
% \usepackage{hyperref}
% \usepackage[breaklinks=true,bookmarks=false]{hyperref}
% \hypersetup{colorlinks,linkcolor={red},citecolor={hanpurple},urlcolor={red}}  
% Include other packages here, before hyperref.
% If you comment hyperref and then uncomment it, you should delete
% egpaper.aux before re-running latex.  (Or just hit 'q' on the first latex
% run, let it finish, and you should be clear).
\iccvfinalcopy % *** Uncomment this line for the final submission
\def\iccvPaperID{****} % *** Enter the ICCV Paper ID here
\def\httilde{\mbox{\tt\raisebox{-.5ex}{\symbol{126}}}}

% Pages are numbered in submission mode, and unnumbered in camera-ready
\ificcvfinal\pagestyle{empty}\fi

\begin{document}

%%%%%%%%% TITLE
\title{A Good Student is Cooperative and Reliable: CNN-Transformer Collaborative Learning for Semantic Segmentation}

\author{Jinjing Zhu$^{1}$
\quad
Yunhao Luo$^{3}$
\quad
Xu Zheng$^{1}$
\quad
Hao Wang $^{4}$
\quad
Lin Wang$^{1,2}$ \thanks{Corresponding author}
\and
\affmark[1] AI Thrust, HKUST(GZ)\quad
\affmark[2] Dept. of CSE, HKUST \quad
\affmark[3] Brown University\quad
\affmark[4] Alibaba Cloud, Alibaba Group\\
\quad
{\tt\footnotesize zhujinjing.hkust@gmail.com, devinluo27@gmail.com, zhengxu128@gmail.com,
cashenry@126.com, linwang@ust.hk}}


% Remove page # from the first page of camera-ready.
\ificcvfinal\thispagestyle{empty}\fi
\twocolumn[{
\renewcommand\twocolumn[1][]{#1}%
\captionsetup{font=small}
\maketitle
% Remove page # from the first page of camera-ready.
\ificcvfinal\thispagestyle{empty}\fi
\captionsetup{font=small}
\begin{center}
\vspace{-19pt}
    \centering
    % Figure removed
\vspace{-7pt}
    \captionof{figure}{(a) Our CNN-ViT collaborative learning framework can learn the compact ViT-models and CNN-based models simultaneously while achieving the SoTA segmentation performance than prior methods. (b) We propose the first online KD framework to collaboratively learn compact CNN-based and ViT-based models by selecting and exchanging reliable knowledge between them. }
\label{fig:Introduction}
\vspace{-5pt}
\end{center}}] 
%%%%%%%%% ABSTRACT
\begin{abstract}
\vspace{-6pt}
   In this paper, we strive to answer the question `\textit{how to collaboratively learn convolutional neural network (CNN)-based and vision transformer (ViT)-based models by selecting and exchanging the reliable knowledge between them for semantic segmentation?}' Accordingly, we propose an online knowledge distillation (KD) framework that can simultaneously learn compact yet effective CNN-based and ViT-based models with two key technical breakthroughs to take full advantage of CNNs and ViT while compensating their limitations. Firstly, we propose heterogeneous feature distillation (\textbf{HFD}) to improve students' consistency in low-layer feature space by mimicking heterogeneous features between CNNs and ViT. Secondly, to facilitate the two students to learn reliable knowledge from each other, we propose bidirectional selective distillation (\textbf{BSD}) that can dynamically transfer selective knowledge. This is achieved by 1) region-wise BSD determining the directions of knowledge transferred between the corresponding regions in the feature space and 2) pixel-wise BSD discerning which of the prediction knowledge to be transferred in the logit space. Extensive experiments on three benchmark datasets demonstrate that our proposed framework outperforms the state-of-the-art online distillation methods by a large margin, and shows its efficacy in learning collaboratively between ViT-based and CNN-based models.
\end{abstract}
\section{Introduction}
Semantic segmentation \cite{LongSD15, XieWYAAL21, ChenPKMY18} is a crucial and challenging vision task, which aims to predict a category label for each pixel in the input image. Although the state-of-the-art (SoTA) segmentation methods have achieved remarkable performance, they often require prohibitive computational costs. This limits their applications to resource-limited scenarios, \eg, autonomous driving \cite{FengHRHGTWD21}. Consequently, growing attention has been paid to model compression aiming at obtaining more compact networks. It can be roughly divided into quantization \cite{WuLWHC16, bs-2103-13630, EsserMACAABMMBN16}, pruning \cite{CaiAYYX22, LouizosWK18, abs-1803-05729}, and knowledge distillation (KD)~\cite{WangY22,SonNCH21, ParkCJKH21}. The standard KD paradigm aims to learn a compact yet effective student model under the guidance of a high-capacity teacher model. For instance, CD \cite{ShuLGYS21} proposes a channel-wise KD approach by normalizing the activation map of each channel. IFVD \cite{WangZJBX20} characterizes the intra-class feature variation (IFV) and makes the student model mimic the IFV of the teacher model. 

Recently, vision transformer (ViT) achieves comparable or even better performance than that of CNNs thanks to the computing paradigm, \eg, multi-head self-attention (MHSA). For instance, PVT~\cite{WangX0FSLL0021, WangXLFSLLLS22} and Swin Transformer \cite{abs-2111-09883, DongBCZYYCG22} extract the pyramid features from the high-resolution images and achieve SoTA performance on various benchmarks. To minimize the model complexity, SegFormer~\cite{XieWYAAL21} proposes a hierarchically structured transformer encoder to learn a simple yet efficient ViT-based model. 

In this paper, we strive to collaboratively learn \textit{compact} yet \textit{effective} CNN-based and ViT-based models for semantic segmentation. Intuitively, we explore an online KD paradigm for this goal. Existing online KD methods for classification employ a `Dual-Student' framework (without the pre-trained model) by enabling the students to learn from each other in a one-stage learning manner \cite{AnilPPODH18, SongC18, ZhangXHL18, GuoWWYLHL20}. For example, Deep Mutual Learning (DML) \cite{ZhangXHL18} proposes to make the CNN-based students teach each other in the training process. KDCL~\cite{GuoWWYLHL20} enables the students with different capacities to learn collaboratively to generate reliable soft supervision and boost their classification performance. However, naively applying these CNN-based KD methods is less effective and even leads to performance drops (see Fig.~\ref{fig:Introduction} (a)).
The reasons are that: 1) 
% Online KD ignores the performance gap between CNN and ViT [ViT and CNN difference]. For instance, CNNs are hardcoded to attend only locally and ViT does not learn to attend locally in earlier layers;
The discrepancies in the feature and prediction space between CNNs and ViT caused by the distinct computing paradigms make it challenging to perform online KD.
2)
% These methods only transfer knowledge in the logits space, which lacks effective and reliable guidance in the feature space because ViT [].
These methods only transfer knowledge in the logit space while more reliable and informative knowledge does exist in the \textit{feature space}.
3) There are considerable model size gap and learning capacity gap between CNNs and ViT. 
Intuitively, we ask a question: \textit{`how to collaboratively learn CNN-based and ViT-based models by selecting and exchanging the reliable knowledge between them for semantic segmentation?'
% How to explore a collaborative learning strategy by selecting or exchanging the reliable knowledge between the ViT-based and CNN-based students for semantic segmentation?'
}

In light of this, we propose, to the best of our knowledge, the \textbf{first} online KD strategy 
to further push the limit of CNNs and ViT for semantic segmentation (See Fig. \ref{fig:Introduction} (b)). Our method enjoys two key technical breakthroughs.
Firstly, we propose heterogeneous feature distillation (\textbf{HFD}) to make the students learn the heterogeneous features from each other for complementary knowledge in the low-layer feature space. Concretely, the ViT-based student takes the low-level features from the CNN-based student as guidance and vice versa.
% perform self-attention in ViT-based student with the lower-layer features of the CNN-based student as input. [goal of this module, the why do we propose, []
% Then, we align the output with low-layer features of ViT-based students and can enable CNN-based student to learn global feature representations. 
% Similarly,  we perform convolution in CNN-based student with the lower-layer features of the ViT-based student as input. 
%Then we align the output with low-layer features of CNN-based student. 
Then, consistency between the low-layer features of CNN-based and ViT-based students is imposed to encourage them to compensate for their limitations.
Secondly, to transfer reliable knowledge between CNNs and ViT, we propose a bidirectional selective distillation (\textbf{BSD}) module that selectively distills the reliable region-wise and pixel-wise knowledge.
Specifically, the region-wise distillation dynamically transfers reliable knowledge of regions in the feature space by determining the directions of transferring knowledge. Similarly, pixel-wise distillation discerns which of the prediction knowledge to be transferred in the logit space. Note that these bidirectional distillation approaches are both guided by the cross entropy between predictions and ground-truth (GT) labels. 

In summary, our main contributions are four-fold: (\textbf{I}) We introduce the \textit{first} online collaborative learning strategy to collaboratively learn compact ViT-based and CNN-based models for semantic segmentation. (\textbf{II}) We propose HFD to facilitate CNNs and ViT learning global and local feature representations correspondingly. (\textbf{III}) We propose BSD to distill knowledge between ViT and CNNs in the feature and logit spaces. (\textbf{IV}) Our proposed method consistently achieves new state-of-the-art performance on three benchmark datasets for semantic segmentation. 

% \begin{itemize}
%     \item We introduce the \textit{first} online collaborative learning strategy to collaboratively learn compact ViT-based and CNN-based models for semantic segmentation.
%     \item We propose HFD to facilitate CNNs and ViT learning global and local representations, correspondingly.
%     \item We propose DSD module to distill knowledge between ViT and CNNs in the feature and logit spaces.
%     \item Our proposed method consistently achieves new state-of-the-art performance on three benchmark datasets for semantic segmentation. 
% \end{itemize}




% [describe the KD methods for CNN segmentation model compression]


% [Introduce transofmer and point out our motivation]
% Recently, transformer has been demonstrated. We explore the possibility of collaboratively transformer and CNN for segmentation. A naive approach is to directly apply the existing mutual learning methods for classification, \eg, ~\cite{ZhangXHL18} to our problem. However 

% [introduce the DML-related collaborative learning methods for classification. however, naively applying these methods or loss functions leads to some critical problems: 1) 2)  

% They are less effective for CNN-VIT segmentation (See Table.~\ref{})]

% Point out the specific RQ? To this  end, we propose a novel 

% [point out the research problem/challenges?]

% [introduce our method, and highlight the technical breakthroughs]
% (1) what is the first contribution?

% (2) what is the second ?

% (3) ?

% [summarize the overall performance]
% [describe the experimental results]

% KD approaches aim to improve the performance of a compact student network under the guidance of a high-capacity teacher network. However, in the traditional offline knowledge \cite{HintonVD15} distillation framework, the teacher is pre-trained first and then fixed, meaning that the knowledge can only be transferred from the teacher to the student in a more complex two-phase training procedure. 



% Online counterparts \cite {AnilPPODH18, SongC18, ZhangXHL18, GuoWWYLHL20} address this limitation by treating all networks as students and enabling students to gain extra knowledge from each other in single-stage training. Deep Mutual Learning (DML) \cite{ZhangXHL18} is the first work to enable students to teach each other based on the predictions throughout the training process, but maybe harm the students due to the conflicts between predictions of students. Knowledge Distillation method via Collaborative Learning (KDCL) \cite{GuoWWYLHL20} further enables students to learn collaboratively to generate reliable soft supervision and boost the performance of all students with different capacities.  

% % Figure environment removed

% However, these methods only perform mutual learning between isomorphic students without transferring knowledge between CNNs \cite{HowardZCKWWAA17,he2016deep} and Vision Transformer (ViT) \cite{DosovitskiyB0WZ21}. ViT achieves a significant performance boost over CNNs for learning visual representations, but the computational cost constrains its application. Moreover, CNNs performs well due to its efficiency in local processing filters via decomposition of depth-wise and point-wise convolution. Recently, this work \cite{RaghuUKZD21} reveals that CNNs are hardcoded to attend only locally and ViT does not learn to attend locally in earlier layers. Furthermore, existing online knowledge distillation only transfers knowledge in the logit space, which lacks effective and reliable guidance in the feature space. Therefore, this in turn naturally gives rise to one intuitive question: 

% \textit{\textbf{`` How to explore an online distillation learning strategy by selecting reliable knowledge of CNNs and ViT to learn collaboratively for semantic segmentation? "}}


% Figure environment removed

\section{Related work}

\noindent \textbf{KD for Segmentation.} 
% Semantic segmentation aims to make pixel-wise predictions that can be utilized in a large number of real-world applications, such as autonomous driving \cite{WangCY20} and metaverse \cite{abs-2110-05352}. 
% To minimize the computational costs of segmentation networks,
% Various works  have been proposed for obtaining compact segmentation models by using KD techniques. 
The mainstream methods~\cite{WangZJBX20, LiuCLQLW19, JiWTH0L22, ShuLGYS21, HeSTGSY19, AnilPPODH18, YaoS20, KimHCK20, BeyerZRMA022} for segmentation mostly focus on learning a compact CNN-based student model by distilling the knowledge from a cumbersome CNN-based teacher model with the same network architecture. 
CD \cite{ShuLGYS21} proposes a channel-wise KD approach by normalizing the activation map of each channel. IFVD \cite{WangZJBX20} characterizes the intra-class feature variation (IFV) and makes the student mimic the IFV of the teacher. SSTKD \cite{ji2022structural} exploits the structural and statistical knowledge to enrich low-level information of the student model. \textit{Differently, we propose a first online KD approach, which collaboratively learns compact yet effective CNN-based and ViT-based models for segmentation.}

\noindent \textbf{Vision Transformer} has demonstrated its effectiveness on several vision tasks but is less applicable in case of limited computational resources. Recently, several attempts have been made to obtain compact ViT models via network pruning~\cite{abs-2204-07154} or KD~\cite{JiaoYSJCL0L20}. Moreover, some works \cite{abs-2108-05895} combine the advantages of CNNs and ViT, and design hybrid models for classification.
\textit{By contrast, we explore simultaneously learning compact yet effective CNNs and ViT models by bidirectionally learning the feature and prediction information from both models for semantic segmentation.
}
% To demonstrate the effectiveness of our method, we utilize the CNNs (MobileNet \cite{HowardZCKWWAA17} and ResNet \cite{he2016deep}) and ViT (SegFormer \cite{XieWYAAL21}) to assign labels of pixels in images.  


% model compression \cite{abs-2204-07154} or knowledge distillation \cite{JiaoYSJCL0L20} techniques to obtain compact ViT models. Moreover, some works \cite{abs-2108-05895} design hybrid light-weight models by leveraging the advantages of CNNs and ViT. In this work, we propose a collaborative learning strategy CTCL to make CNNs and ViT learn from each other. To demonstrate the effectiveness of our method, we utilize the CNNs (MobileNet \cite{HowardZCKWWAA17} and ResNet \cite{he2016deep}) and ViT (SegFormer \cite{XieWYAAL21}) to assign labels of pixels in images.  

\noindent \textbf{Online KD.} Some works \cite{ZhangXHL18, AnilPPODH18, ChenMWF020, KimHCK20} focus on the online KD without a pre-trained teacher model. DML~\cite{ZhangXHL18} proposes a mutual learning strategy, where an ensemble of students' logits is deployed, for classification task. Co-distillation \cite{AnilPPODH18} further extends this idea and explores the potential in distributed learning. ONE \cite{LanZG18} constructs a multi-branch network and assembles the on-the-fly logit information from the branches to enhance the performance on the target network. CLNN \cite{SongC18} proposes multiple generated classifier heads to obtain supplementary information for improving the generalization ability of the target network. KDCL \cite{GuoWWYLHL20} aggregates the outputs of numerous students with different learning capacities to generate high-quality labels for supervision. PCL \cite{WuG21b} integrates online ensemble and collaborative learning into a unified framework. \textit{Unlike these works generating a soft target in the logit space and transferring knowledge between the isomorphic CNN-based models, we introduce a collaborative learning strategy between the heterogeneous CNN-based and ViT-based models. We propose to bidirectionally exchange the reliable knowledge in feature and logit spaces for semantic segmentation}.

\section{The Proposed Approach}
% In this section, we first briefly describe the fundamental elements of online knowledge distillation for segmentation and then introduce the proposed CTCL framework (as illustrated in Figure \ref{fig:famework} (b)). Subsequently, we present details of the proposed HFCL, RBSS, and SBSS, respectively.
\subsection{Overview}
An overview of the proposed framework is depicted in Fig.~\ref{fig:famework}\textcolor{red}{(b)}, which consists of three components: a CNN-based student $f\left(\theta^{\mathrm{C}}\right)$, a ViT-based student $f\left(\theta^{\mathrm{V}}\right)$, and the proposed KD modules. Given an input image set $X$, our objective is to enable $f\left(x; \theta^{\mathrm{V}}\right)$ and $f\left(x; \theta^{\mathrm{C}}\right)$ to learn collaboratively that can assign a pixel-wise label $l \in 1, \ldots, K$ to each pixel $p_{i, j}$ in image $x \in X (x \in \mathbb{R}^{H \times W \times 3})$ more accurately than the student itself, where $H$ and $W$ are the height and width of $x, K$ is the number of categories. To achieve this goal, given specific input $x$, we attain the segmentation prediction maps $\left(P^{\mathrm{C}}\right.$ and $\left.P^{\mathrm{V}}\right)$ and feature representations ( $F^{\mathrm{C}}$ and $\left.F^{\mathrm{V}}\right)$ from the two students $f\left(x; \theta^{\mathrm{C}}\right)$ and $f\left(x; \theta^{\mathrm{V}}\right)$, respectively, which can be formulated as:
{\setlength\abovedisplayskip{1pt}
\setlength\belowdisplayskip{1pt}
\begin{equation*}
 \left(P^{\mathrm{C}}, F^{\mathrm{C}}\right)=f\left(x ;\theta^{\mathrm{C}}\right), \quad\left(P^{\mathrm{V}}, F^{\mathrm{V}}\right)=f\left(x ; \theta^{\mathrm{V}}\right).   
\end{equation*}}

The pixel-wise segmentation loss $\text{CE}(\cdot)$ is based on the cross-entropy (CE) loss with the ground-truth (GT) label :
{\setlength\abovedisplayskip{2pt}
\setlength\belowdisplayskip{2pt}
\begin{equation}
\small
\begin{aligned}
  &\mathcal{L}_{\text{CE}}^{C}=\frac{1}{H \!\times\! W} \sum_{h=1}^{H} \sum_{w=1}^{W} \text{CE}\left(\sigma\left({P^{C}}_{(h, w)}\right), y_{(h, w)}\right), \\
  &\mathcal{L}_{\text{CE}}^{V}=\frac{1}{H \!\times\! W} \sum_{h=1}^{H} \sum_{w=1}^{W} \text{CE}\left(\sigma\left({P^{V}}_{(h, w)}\right), y_{(h, w)}\right).
\end{aligned}
\end{equation}}

Here, $\sigma$ is the softmax function, and $y_{(h, w)}$ denotes the GT of the $(h, w)$-th pixel of image $x$.
Our key ideas are two folds.  
% Firstly, to enable the heterogeneous students to learn collaboratively, we exploit DeepLabv3+ \cite{ChenZPSA18} as a CNN-based student and SegFormer as a ViT-based student for semantic segmentation. 
To compensate for the limitations of CNNs and ViT, we first propose HFD to align the features in the low-layer feature space. Secondly, we propose a BSD module to selectively enable both two students to mimic the region-wise and pixel-wise information from each other. We now describe the technical details. 
% More details are provided in the following sections. 
\subsection{Heterogeneous Feature Distillation (HFD) }
Inspired by the observations that CNNs are hard-coded to attend only locally while ViT does not learn to attend locally in earlier layers~\cite{RaghuUKZD21}, we propose a novel HFD module \textit{to make the students learn the heterogeneous features from each other for complementary knowledge in the low-layer feature space}. Specifically, it is efficiently achieved by aligning the transformed features between CNNs and ViT (see Fig.\ref{fig:famework} \textcolor{red}{(a)}). At the top of Fig.\ref{fig:famework} \textcolor{red}{(a)}, we transfer knowledge from the ViT-based student to the CNN-based student. To match the shapes and channels between the first-layer features $F_{1}^{C}$ of CNN-based student and the first-stage features $F_{1}^{V}$, we utilize a linear transformation ${\Gamma}^{C}_{1}$ which consists of $1\times 1$ convolution (conv) and pooling layers. And $F_{1}^{C}$ is transformed to be $F_{1}^{\hat{C}} = {\Gamma}^{C}_{1}(F_{1}^{C})$.  Then, the second-stage ViT block `Attn' has two inputs: (a) the feature $F_{1}^{V}$ and (b) the transformed feature $F_{1}^{\hat{C}}$ and outputs second-stage feature $F_{2}^{V}=\text{Attn}(F_{1}^{V})$ and $\text{Attn}(F_{1}^{\hat{C}})$. To enable the low-layer features of CNNs to mimic low-layer features of ViT, we align $F_{2}^{V}$ and $\text{Attn} (F_{1}^{\hat{C}})$ by using cosine distance and use the discrepancy to optimize CNNs. Similarly, as shown in the right of figure of Fig. \ref{fig:famework}\textcolor{red}{(a)}, we exploit the linear transformation ${\Gamma}^{V}_{1}$ to match the spatial size of CNNs and ViT. We also utilize a linear transformation ${\Gamma}^{V}_{1}$ comprising of $1\times 1$ convolution (conv) and pooling layers to transform $F_{1}^{V}$ as $F_{1}^{\hat{V}}={\Gamma}^{V}_{1}(F_{1}^{V})$. The second-layer of CNNs MLP takes the transformed features $F_{1}^{\hat{V}}$ and $F_{1}^{C}$ as inputs, and outputs $\text{MLP}(F_{1}^{\hat{V}})$ and $F_{2}^{C}=\text{MLP}(F_{2}^{C})$, correspondingly. Then, aligning these outputs with the cosine distance facilitates ViT-based student to learn from CNN-based student and thus improves the performance of ViT. Finally, we make ViT-based student learn the local feature representations and CNN-based student learn global feature representations by HFD, which is defined as:

% Since the distinct network structures and computing paradigms of two students, the inner high-level features can not be aligned naively. 

% Consequently, we exploit to match the channels and shapes of the low-layer heterogeneous features by using specific layers, including $1\times 1$ convolution (conv) and pooling layers. 


% According to the different characteristics of features from the two students, we utilize features from different layers (stages) in CNNs (ViT). 

% For CNNs, we leverage the first-layer features $F_{1}^{C}$

% Due to the different architectures of CNNs and ViT, we exploit the $1\times 1$ convolution (conv) and pooling layers to match the channels and shapes. 

% In this work, we input the first-layer features $F_{1}^{C}$ of CNNs into the second stage of ViT.


% Then, we attain the outputs of the second stage with transformed the first-layer features as inputs.

% and align the outputs with the outputs of the stage where the first-stage patches are inputs. 

% After this alignment, we optimize CNNs by computing the discrepancy between these two different outputs, which means this approach enables the CNN features to mimic the ViT features. This can facilitate CNNs to learn the global interaction as ViT does. 

% Similarly, we input the first-stage features into CNNs and align the two outputs of CNNs, which can address ViT limitation of local processing in earlier layers. 


{\setlength\abovedisplayskip{2pt}
\setlength\belowdisplayskip{2pt}
\begin{equation}
\small
\begin{aligned}
  &\mathcal{L}_{\text{HFD}}^{C}=cos({\text{Attn}}((F_{1}^{\hat{C}})),F_{2}^{V}), \\
  &\mathcal{L}_{\text{HFD}}^{V}=cos({\text{MLP}}({F_{1}^{\hat{V}}}),F_{2}^{C}),
\end{aligned}
\end{equation}}
here $cos$ is the cosine distance measuring the consistency between CNNs and ViT. $\text{Attn}((F_{1}^{\hat{C}})$ is defined as 
{\setlength\abovedisplayskip{2pt}
\setlength\belowdisplayskip{2pt}
\begin{equation*}
\small
\begin{aligned}
\text{Attn}(F_{1}^{\hat{C}}) =
\operatorname{softmax}\left(\frac{F_{1}^{\hat{C}}W^Q (F_{1}^{\hat{C}}W^K)^{T}}{\sqrt{d}}\right) (F_{1}^{\hat{C}}W^V),
\end{aligned}
\end{equation*}}

where $W^Q$, $K^Q$, and $V^Q$ are the projections of $\text{Attn}$ and $d$ is the number of multi-head in ViT. Similarly, ${\text{MLP}}$ block is the second layer of CNNs.

\subsection{Bidirectional Selective Distillation (BSD)}
Due to the different performance in different regions between the ViT and CNN students, we intend to dynamically select useful knowledge between the two students in the feature space, so as to benefit each other. However, there is a challenging problem: `\textit{how to decide the directions of transferring knowledge for different regions during training}?'. To this end, we propose to manage the directions of KD via combining the predictions and GT labels, where we regard the directions of KD for different regions as a sequential decision-making problem. Consequently, we propose a directional selective distillation (BSD) for \textit{enabling students to learn collaboratively}, as shown in Fig. ~\ref{fig:module2}. Our BSD module transfers knowledge in two aspects. Firstly, the region-wise distillation determines the distillation direction of each region for supervising students in each region. Secondly, the pixel-wise distillation decides which of the prediction knowledge to be transferred in the logit space.  
% Figure environment removed 

\subsubsection{Region-wise distillation} 
% Different from previous online knowledge distillation works which only learn collaboratively in the logit space, our work utilizes more rich knowledge in the feature space than the logit space to learn cooperatively. Moreover, 
% This strategy dynamically distills region-wise knowledge between CNN-based and ViT-based students.

Given the last-layer feature $F^{C}_{l} \in \mathbb{R}^{\hat{H} \times \hat{W} \times \hat{D}}$ and the last-stage feature $F^{V}_{l}$, we exploit $1\times1$ conv and pooling layers to transform $F^{C}_{l}$ and $F^{V}_{l}$ for matching the channels and shapes of them. The transformation functions are denoted as $\Gamma_{l}^{C}$ and $\Gamma_{l}^{V}$, respectively. Then $F^{\hat{C}}_{l}=\Gamma_{l}^{C}(F^{C}_{l})$ matches the dimensions of $F^{\hat{V}}_{l}=\Gamma_{l}^{V}(F^{V}_{l})$ and $F^{\hat{C}}_{l} \in R^{\hat{H}\times \hat{W}\times \hat{D}}$. To transfer knowledge from regions between students, we calculate the cross-student region-wise similarity matrix $S_{(\hat{h},\hat{w})} = cos({F^{\hat{C}}_{l}}_{(\hat{h},\hat{w})},(F^{\hat{V}}_{l})_{(\hat{h},\hat{w})}).$ Then, we exploit the cross entropy between the predictions and GT labels to quantify the most reliable knowledge for each region between the students. As shown in Fig.~\ref{fig:module2}, the red grids $\{\textcolor{red}{{r_{1}}}, \textcolor{red}{r_{3}}, \textcolor{red}{r_{8}}\}$ indicate that the knowledge of these regions are more reliable than the white cubes $\{r_{1}, r_{3}, r_{8}\}$, which means the CE losses of these red regions are smaller than blue regions. Then, the direction of KD for these three regions is from red regions to white regions. Specifically, we utilize the matrix $\hat{m}_{(\hat{h},\hat{w})}\in R^{\hat{H} \!\times\! \hat{W}}$ which is 0 or 1 to decides the direction of KD for each region. The value of $\hat{m}_{(\hat{h},\hat{w})}$ is 1 when the CE of this region in CNN-based prediction map is smaller than that of ViT-based prediction map, enabling the knowledge to be transferred from ${F^{\hat{C}}_{l}}_{(\hat{h},\hat{w})}$ to ${F^{\hat{V}}_{l}}_{(\hat{h},\hat{w})}$, and vice versa for $\hat{m}_{(\hat{h},\hat{w})}$=0. Note that for matching the size of ${F^{\hat{C}}_{l}}$ and $P^{C}$, we divide the prediction map into $\hat{H} \times \hat{W}$ size. A $\frac{H}{\hat{H}} \!\times\! \frac{W}{\hat{W}}$ sized prediction map $P^{C}$ at the same location corresponds to one region in ${F^{\hat{C}}_{l}}$. After determining the direction of KD for each region, the two students can exchange reliable region-wise knowledge. Therefore, we weight the similarity matrix $S_{(\hat{h},\hat{w})}$ based on the matrix $\hat{m}_{(\hat{h},\hat{w})}$, which denotes the process of KD from the more reliable regions to these less reliable regions. To achieve this region-wise KD, we propose to minimize the loss function as follows:
{\setlength\abovedisplayskip{2pt}
\setlength\belowdisplayskip{2pt}
\begin{equation}
\small
\begin{aligned}
&\mathcal{L}_{R}^{C} = \frac{1}{\hat{H} \!\times\! \hat{W}\!-\!\hat{M}} \sum_{\hat{h}=1}^{\hat{H}} \sum_{\hat{w}=1}^{\hat{W}} (1-\hat{m}_{(\hat{h},\hat{w})})S_{(\hat{h},\hat{w})},\\ 
&\mathcal{L}_{R}^{V} =  \frac{1}{\hat{M}}\sum_{\hat{h}=1}^{\hat{H}}\sum_{\hat{w}=1}^{\hat{W}}   m_{(\hat{h},\hat{w})}S_{(\hat{h},\hat{w})},
\end{aligned}
\end{equation}}

where $\hat{M} = \sum_{\hat{h}=1}^{\hat{H}} \sum_{\hat{w}=1}^{\hat{W}} \hat{m}_{(\hat{h},\hat{w})}$.


% \subsection{Structural Bidirectional Selective Supervision}
% As we know, semantic segmentation is a structured prediction problem, therefore, we present structured knowledge distillation, named structural bidirectional selective supervision (SBSS). It is worth noting that. we just transfer reliable of structure, which is comprised of reliable regions. This is because we think selective regions contain more useful knowledge. We take Figure \ref{fig:famework} (b) as example to illustrate it. As we introduce these reliable regions, the light solid red parts $\{{r_{1}}, r_{1}, r_{3}\}$, in RBSS, we conduct reliable structure which is made up of these light solid red parts $\{{r_{1}}, r_{1}, r_{3}\}$. Then we transfer the reliable knowledge of the structure to the structure which is comprised of light blue dotted part $\{r_{1}, r_{2}, r_{3}\}$. Specifically, based on the RBSS and the weighting matrix $c$, we define the similarity between regions for CNNs and ViT as follows:
% $$
% M^{C} = cos(F^{C}_{l}, {F^{C}_{l}}), M^{V} = cos(\Gamma_{l}^{V}(F^{V}_{l}),{\Gamma_{l}^{V}(F^{V}_{l})}),
% $$
% where $F^{C}_{l}$ ($F^{V}_{l}$) is the last-layer (last-stage) of CNN-based (ViT-based) student and $cos$ is the cosine distance function. To match the shape and channel of these features, we design a linear transformation $\Gamma_{l}^{V}$ to achieve it. To transfer reliable structure-level knowledge, we propose a new loss function to be minimized as:

% \begin{equation}
% \small
% \begin{aligned}
% &\mathcal{L}_{SBSS}^{C} = \sum_{i=1}^{\hat{h}}\sum_{i=j}^{\hat{w}} (1-c_{i})(1-c_{j}){L}_{2}(M^{C}_{(i,j)},M^{V}_{(i,j)}), \\
% &\mathcal{L}_{SBSS}^{V} = \sum_{i=1}^{\hat{h}}\sum_{i=j}^{\hat{w}} c_{i}c_{j}{L}_{2}(M^{C}_{(i,j)},M^{V}_{(i,j)}),
% \end{aligned}
% \end{equation}
% where $\hat{h},\hat{w}$ are height and width of the last-layer feature and $L_2$ represents the Euclidean distance.


% \subsection{Mutual Decoupled Distillation}

% In prior methods \cite{ZhangXHL18, AnilPPODH18, ChenMWF020}, the students learn from each other by obtaining reliable supervision with ensembling or weighting their predictions. However, naively using these methods degrades the performance efficiency if the students make mistakes when students perform not well. Therefore, inspired by \cite{abs-2203-08679} for classification tasks, we further propose a novel Bidirectional Decoupled Distillation (BDD) module to tackle this problem for segmentation tasks. Specifically, for a pixel $p_{i, j}$ in the segmentation prediction maps $S$, the predicted logits can be denoted as $p_{i, j}=\left[l_{1}, l_{2}, \ldots, \ldots l_{i}, \ldots, l_{K}\right]$, where $l_{k}$ is the probability of the $k$-th class. We separate the pixel-wise predictions relevant and irrelevant to the target class (cls-th) and non-target classes into binary probabilities as follows:
% \begin{equation*}
% \small
%  l_{t}=l_{K}, \quad l_{n t}=\sum_{k=1}^{c l s-1} l_{k}+\sum_{k=c l s+1}^{N} l_{k}, \quad\left[l_{t}, l_{n t}\right] \in \mathbb{R}^{2}.
% \end{equation*}
% Then we use image-level $P^{t}$ and $P^{n t}$ to show target and non-target class binary probability maps. As such, we use the KL-Divergence to achieve BDD, which can be formulated as:

% \begin{equation}
% \small
% \begin{aligned}
% &\mathcal{L}_{mdd}^{V}=\frac{1}{|P|}\left( \cdot S_{t}^{\mathrm{V}} \log \left(\frac{S_{t}^{\mathrm{V}}}{S_{t}^{\mathrm{C}}+Y_{n}^{t}}\right)+\cdot S_{n t}^{\mathrm{V}} \log \left(\frac{S_{n t}^{\mathrm{V}}}{S_{n t}^{\mathrm{C}}+Y_{n}^{n t}}\right)\right), \\
% &\mathcal{L}_{mdd}^{C}=\frac{1}{|P|}\left(\cdot S_{t}^{\mathrm{C}} \log \left(\frac{S_{t}^{\mathrm{C}}}{S_{t}^{\mathrm{V}}+Y_{n}^{t}}\right)+\cdot S_{n t}^{\mathrm{C}} \log \left(\frac{S_{n t}^{\mathrm{C}}}{S_{n t}^{\mathrm{V}}+Y_{n}^{n t}}\right)\right), 
% \end{aligned}
% \end{equation}

% where $Y_{n}$ denotes the corresponding label for the input image $x, \alpha$ and $\beta$ are trade-off weights.


\subsubsection{Pixel-wise distillation}
Previous KD approaches~\cite{WangZJBX20, ShuLGYS21} for semantic segmentation apply the fundamental response-based distillation loss $\mathcal{L}_{\text{KL}}$ for the stable gradient descent optimization: $\mathcal{L}_{\text{KL}}=\frac{1}{H \times W} \sum_{h=1}^{H} \sum_{w=1}^{W} \text{KL} \left({P^{C}}_{(h, w)}\|{P^{V}}_{(h, w)}\right)$, where $\text{KL}(\cdot)$ is the Kullback-Leibler divergence (KL divergence) between two probabilities. However, due to the performance gap, the heterogeneous students have their own strengths in predicting different segmentation categories. Therefore, the pixel-wise distillation \textit{aims to transfer the knowledge of more reliable pixel-wise predictions to less reliable pixel-wise predictions in the logit space}. As shown in Fig.~\ref{fig:module2}, the black squares $\{p_{1},p_{5},p_{7}\}$ of $P^{V}$ means that the CE losses of these pixels are smaller than the gray squares $\{p_{1},p_{5},p_{7}\}$ of $P^{C}$. Therefore, we transfer the reliable knowledge from these more black squares to light black squares. Specifically, we utilize the matrix ${m}_{(h,w)} \in R^{H\!\times\! W}$ which is 0 or 1 to decide the direction of KD for each pixel. The value of ${m}_{(h,w)}$ is 1 when the CE of pixel from CNN-based student is smaller than that of ViT, enabling the knowledge to be transferred from ${p^{C}_{(h,w)}}$ to ${p^{V}_{(h,w)}}$, and vice versa for ${m}_{(h,w)}$=0. Moreover, we use the KL divergence to the effectiveness of transferring knowledge from ViT-based student to CNN-based student: $\text{KL} \left({P^{C}}_{(h, w)}\|{P^{V}}_{(h, w)}\right)$.  After determining the direction of KD for each pixel, the two students can exchange useful pixel-wise knowledge. Therefore, we weight the KL divergence $\text{KL}\left({P^{C}}_{(h, w)}\|{P^{V}}_{(h, w)}\right)$ based on the matrix ${m}_{(h,w)}$, which denotes the process of KD from the more reliable pixels to these less reliable pixels. To achieve pixel-wise distillation, we propose to minimize the loss function as follows:
{\setlength\abovedisplayskip{2pt}
\setlength\belowdisplayskip{2pt}
\begin{equation}
\small
\begin{aligned}
&\mathcal{L}_{P}^{C} = \frac{1}{H \!\times\! W \!-\!M} \sum_{i=1}^{H}\sum_{i=j}^{W} (1\!-\!m_{(h,w)}) \text{KL} \left({P^{C}}_{(h, w)}\|{P^{V}}_{(h, w)}\right), \\
&\mathcal{L}_{P}^{V} =\frac{1}{M} \sum_{i=1}^{H}\sum_{i=j}^{W} m_{(h,w)}\text{KL}\left({P^{V}}_{(h, w)}\|{P^{C}}_{(h, w)}\right),
\end{aligned}
\end{equation}}

where $M = \sum_{h=1}^{H} \sum_{w=1}^{W} m_{(h,w)}$. 
Finally, combining the region-wise and pixel-wise KD losses, the BSD loss is defined as:
{\setlength\abovedisplayskip{2pt}
\setlength\belowdisplayskip{2pt}
\begin{equation}
\small
\begin{aligned}
&\mathcal{L}_{\text{BSD}}^{C} = \mathcal{L}_{R}^{C} + \alpha \mathcal{L}_{P}^{C} , \\
&\mathcal{L}_{\text{BSD}}^{V} =\mathcal{L}_{R}^{V} + \alpha \mathcal{L}_{P}^{V},
\end{aligned}
\end{equation}}
where $\alpha$ is the trade-parameter to balance the region-wise and pixel-wise losses, and $\alpha$ is set to 1.
% $c_{(h,w)}$ is 1 when the cross entropy loss of pixel of CNN-based student is smaller than ViT-based student, enabling the distribution or prediction of ViT-based student to match the CNN-based student, and vice versa for $c_{(h,w)}$. 


% \subsection{Optimization}
% Following the common practical and previous knowledge distillation approaches for semantic segmentation, we also add the fundamental response-based distillation loss $\mathcal{L}^{C}$ and $\mathcal{L}^{V}$ for the stable gradient descent optimization:

% \begin{equation}
% \small
% \begin{aligned}
% &\mathcal{L}^{C}_{KL}=\frac{1}{H \times W} \sum_{h=1}^{H} \sum_{w=1}^{W} KL \left({P^{C}}_{h, w}\|{P^{V}}_{h, w}\right), \\
% &\mathcal{L}^{V}_{KL}=\frac{1}{H \times W} \sum_{h=1}^{H} \sum_{w=1}^{W} KL \left({P^{V}}_{h, w}\|{P^{C}}_{h, w}\right),
% \end{aligned}
% \end{equation}
% where $KL(\cdot)$ is the Kullback-Leibler divergence between two probabilities. 



\subsection{Optimization}

Overall, the objectives of the proposed method for CNN-based and ViT-based students are given as 

\begin{equation}
\small
\begin{aligned}
&\mathcal{L}^{C}=\mathcal{L}^{C}_{\text{CE}}+\beta \mathcal{L}^{C}_{\text{HFD}}+\gamma\mathcal{L}^{C}_{\text{BSD}}, \\
&\mathcal{L}^{V}=\mathcal{L}^{V}_{\text{CE}}+\beta \mathcal{L}^{V}_{\text{HFD}}+\gamma\mathcal{L}^{V}_{\text{BSD}},
\end{aligned}
\end{equation}

where $\beta$ and $\gamma$ are hyperparameters and set to 0.1 and 1, respectively.
% % Figure environment removed

% % Figure environment removed

\section{Experiments and Evaluation}
\subsection{Setup}
\textbf{Datasets.} In this work, we conduct extensive experiments to demonstrate the effectiveness of the proposed method on three public datasets: \textbf{PASCAL VOC 2012} \cite{EveringhamEGWWZ15}, \textbf{Cityscapes} \cite{CordtsORREBFRS16}, and \textbf{CamVid} \cite{BrostowSFC08}. Following previous works \cite{WangZJBX20, ShuLGYS21}, we adopt the augmented \textbf{PASCAL VOC 2012} set \cite{hariharan2011semantic} consisting of 10,582 training and 1,449 validation images with 21 pixel-wise annotated classes. \textbf{Cityscapes} is a dataset for urban scene understanding and consists of 5,000 fine-annotated 1024 $\times$ 2048 images with 19 categories for segmentation. \textbf{CamVid} is another widely used urban scene dataset with 11 classes, such as building, tree, sky, car, road, etc., and the 12th class indicates unlabeled data. It contains 367 training, 101 validation, and 233 testing images of 720 $\times$ 960, where we resize them to 360 $\times$ 480 following previous work.

\begin{table*}[t]
    \centering
    \small 
    \captionsetup{font=small}
    \renewcommand{\tabcolsep}{10.5pt}
    \begin{tabular}{|l|c|c|c|c|c|c|c|}
    \hline
    Dataset&Method& MobileNet&MiT-B1&$\Delta$&ResNet-50&MiT-B2&$\Delta$\\
    \hline\hline
    \multirow{6}{*}{PASCAL VOC 2012} &Vanilla & 67.54 &78.48&0.00& 76.05 &82.03&0.00 \\
    &Offline KD &$67.40_{\textcolor{blue}{-0.14}}$&$78.87_{\textcolor{red}{+0.39}}$&\textcolor{red}{+0.25}&$76.77_{\textcolor{red}{+0.68}}$&$82.19_{\textcolor{red}{+0.16}}$&\textcolor{red}{+0.88}\\
    &DML&$67.43_{\textcolor{blue}{-0.11}}$&$78.76_{\textcolor{red}{+0.28}}$&\textcolor{red}{+0.17}&$76.51_{\textcolor{red}{+0.46}}$&$82.10_{\textcolor{red}{+0.07}}$&\textcolor{red}{+0.53}\\
    &KDCL&$67.41_{\textcolor{blue}{-0.13}}$&$78.76_{\textcolor{red}{+0.28}}$&\textcolor{red}{+0.15}&$76.46_{\textcolor{red}{+0.41}}$&$82.01_{\textcolor{blue}{-0.02}}$&$\textcolor{red}{+0.39}$\\
    &IFVD & $67.70_{\textcolor{red}{+0.16}}$ & $77.61_{\textcolor{blue}{-0.87}}$& \textcolor{blue}{-0.71}&$76.52_{\textcolor{red}{+0.47}}$ & $81.52_{\textcolor{blue}{-0.51}}$ &\textcolor{blue}{-0.03}\\
%    DKD \cite{abs-2203-08679}&69.43&77.46&0.87\\
   & \textbf{Ours}&$\textbf{69.57}_{\textcolor{red}{+2.03}}$&$\textbf{79.17}_{\textcolor{red}{+0.69}}$&\textcolor{red}{\textbf{+2.72}}& $\textbf{76.99}_{\textcolor{red}{+0.94}}$ & $\textbf{82.67}_{\textcolor{red}{+0.64}}$ &$\textcolor{red}{\textbf{+1.58}}$\\
    \hline
    \end{tabular}
    \caption{Comparison with the SoTA KD methods on the \textbf{PASCAL VOC 2012} dataset for our CNN-based (MobileNetV2 and ResNet-50) and ViT-based (MiT-B1 and MiT-B2) students.
% Quantitative results our method with the state-of-the-art KD methods on PASCAL VOC 2012.
}
    \label{tab:Results_VOC}
\end{table*}
% Figure environment removed

\noindent \textbf{Implementation and Evaluation}
We implement our method on PyTorch framework. For CNN-based students, we adopt the widely applied segmentation architecture DeepLabV3+ with encoders of MobileNetV2 and ResNet-50;  for ViT-based students, we utilize the efficient SegFormer with encoders of MiT-B1 and MiT-B2, which have comparable or smaller parameters with their CNN counterparts, respectively. Due to the page limit, we put the details in the supplementary.

In each dataset, CNN-based students are trained by mini-batch stochastic gradient descent (SGD) where the momentum is 0.9, and weight decay is 0.0005; and ViT-based students are trained by AdamW optimizer with a learning rate 0.00006 and weight decay of 0.01. We train Pascal VOC 2012 for 60 epochs with image size 512 $\times$ 512, where the learning rate for CNN-based models is set to 0.0025 and ViT-based to 0.00006. We evaluate the performance by the mean Intersection over Union (mIoU) score and report our results on the validation sets. We use center-crop evaluation for Pascal VOC 2012 and sliding windows evaluation for Cityscapes. We randomly crop images as 512 $\times$ 512 inputs trained with a batch size of 4. It is worth mentioning that for Offline KD and IFVD approaches, the CNN-based and ViT-based students guide each other as teachers. 
\begin{table*}[t]
    \centering
    \small
    \captionsetup{font=small}
    \renewcommand{\tabcolsep}{10.5pt}
    \begin{tabular}{|l|c|c|c|c|c|c|c|}
    \hline
    Dataset&Method& MobileNet&MiT-B1&$\Delta$&ResNet-50&MiT-B2&$\Delta$\\
    \hline\hline
     \multirow{6}{*}{Cityscapes}&Vanilla &73.23&74.95&0.00&76.83&78.77&0.00 \\
    &Offline KD  &$74.11_{\textcolor{red}{+0.84}}$  & $75.50_{\textcolor{red}{+0.55}}$ &\textcolor{red}{+1.43} &$77.68_{\textcolor{red}{+0.85}}$ &$78.84_{\textcolor{red}{+0.07}}$ &\textcolor{red}{+0.92}\\
    &DML &$73.68_{\textcolor{red}{+0.45}}$&$75.13_{\textcolor{red}{+0.18}}$&\textcolor{red}{+0.63}&$77.22_{\textcolor{red}{+0.39}}$&$78.91_{\textcolor{red}{+0.14}}$&\textcolor{red}{+0.53}\\
    &KDCL &$73.41_{\textcolor{red}{+0.28}}$&$75.51_{\textcolor{red}{+0.56}}$&\textcolor{red}{+0.74}&$77.94_{\textcolor{red}{+1.11}}$&$78.81_{\textcolor{red}{+0.04}}$&\textcolor{red}{+1.15}\\
    &IFVD&$73.13_{\textcolor{blue}{-0.10}}$&$75.25_{\textcolor{red}{+0.30}}$ & \textcolor{red}{+0.20} &$77.57_{\textcolor{red}{+0.74}}$ &$78.90_{\textcolor{red}{+0.13}}$&\textcolor{red}{+0.83}\\
    %DKD \cite{abs-2203-08679}&&74.72&&&&\\
    &\textbf{Ours} &  $\textbf{74.42}_{\textcolor{red}{+1.19}}$& $\textbf{75.62}_{\textcolor{red}{+0.37}}$ &\textcolor{red}{\textbf{+1.86}}&$\textbf{78.03}_{\textcolor{red}{+1.20}}$&$\textbf{79.71}_{\textcolor{red}{+0.94}}$&\textcolor{red}{\textbf{+2.14}}\\
    \hline
    \end{tabular}
    \caption{Comparison with the SoTA KD methods on the \textbf{Cityscapes} dataset for our CNN-based (MobileNetV2 and ResNet-50) and ViT-based (MiT-B1 and MiT-B2) students.
% Quantitative results our method with the state-of-the-art KD methods on PASCAL VOC 2012.
}
    \label{tab:Results_Cityscapes}
    \vspace{-5pt}
\end{table*}
\begin{table*}[t]
\captionsetup{font=small}
    \centering
    \small
    \renewcommand{\tabcolsep}{10.5pt}
    \begin{tabular}{|l|c|c|c|c|c|c|c|}
    \hline
    Dataset&Method& MobileNet&MiT-B1&$\Delta$&ResNet-50&MiT-B2&$\Delta$\\
    \hline\hline
      \multirow{6}{*}{CamVid}&  Vanilla &  71.28 & 76.10  &0.00& 73.97 & 77.04 &  0.00 \\
   & Offline KD  &$69.48_{\textcolor{blue}{-1.80}}$  &  $75.76_{\textcolor{blue}{-0.34}}$ &  \textcolor{blue}{-2.14} & $71.90_{\textcolor{blue}{-2.07}}$& $77.29_{\textcolor{red}{+0.21}}$ &\textcolor{blue}{-1.28}  \\
   & DML&$70.73_{\textcolor{blue}{-0.55}}$&$75.85_{\textcolor{blue}{-0.25}}$&\textcolor{blue}{-0.80}&$73.75_{\textcolor{blue}{-0.22}}$&$77.15_{\textcolor{red}{+0.11}}$&\textcolor{blue}{-0.41}\\
    &KDCL &$71.96_{\textcolor{red}{+0.68}}$&$76.40_{\textcolor{red}{+0.30}}$&\textcolor{red}{+0.98}&$73.19_{\textcolor{blue}{-0.78}}$&$77.56_{\textcolor{red}{+0.51}}$&\textcolor{blue}{-0.26}\\
   & IFVD  & $71.08_{\textcolor{blue}{-0.20}}$ & $75.38_{\textcolor{blue}{-0.72}}$ & \textcolor{blue}{-0.92} & $74.22_{\textcolor{red}{+0.25}}$ & $77.25_{\textcolor{red}{+0.21}}$ &\textcolor{red}{+0.46}  \\
   % DKD \cite{abs-2203-08679}&&&&&&\\
    &\textbf{Ours}  & $\textbf{73.09}_{\textcolor{red}{+1.81}}$ & $\textbf{77.04}_{\textcolor{red}{+0.94}}$ & \textcolor{red}{\textbf{+2.75}} &  $\textbf{75.26}_{\textcolor{red}{+1.29}}$   &     $\textbf{78.52}_{\textcolor{red}{+1.48}}$ &  \textcolor{red}{\textbf{+2.77}} \\
    \hline
    \end{tabular}
    \caption{Comparison with the SoTA KD methods on the \textbf{CamVid} dataset for our CNN-based (MobileNetV2 and ResNet-50) and ViT-based (MiT-B1 and MiT-B2) students.
% Quantitative results our method with the state-of-the-art KD methods on PASCAL VOC 2012.
}
\vspace{-5pt}
    \label{tab:Results_Camvid}
\end{table*}

\subsection{Experiments results}
We conduct experiments with two pairs: MobileNetV2 and MiT-B1; ResNet-50 and MiT-B2. It is worth noting that we build two tasks on online KD, which means that students can transfer knowledge from each other. And we compare our proposed method with some SoTA KD methods Offline KD \cite{HintonVD15}, DML \cite{ZhangXHL18}, KDCL \cite{GuoWWYLHL20}, and IFVD \cite{WangZJBX20}. Furthermore, the $\Delta$ is the sum of the performance improvement of each pair compared with Vanilla. 

\noindent\textbf{Results on PASCAL VOC 2012.} We first evaluate the proposed method on the PASCAL VOC 2012 dataset and report the quantitative results in Tab. \ref{tab:Results_VOC}. Our findings show that existing SoTA KD methods designed for isomorphic models have inferior generalization abilities in online KD between CNN and ViT, compared to the vanilla KD method. In contrast, our proposed approach exhibits significantly better performance. Specifically, our method improves the mIoU of MobileNetV2 and MiT-B1 by \textbf{+2.72\%}. In contrast, the prior SoTA KD methods DML and KDCL only achieved a mIoU increment of \textbf{+0.25\%} and \textbf{+0.15\%}, respectively.

Note that the offline KD method IFVD, designed to transfer knowledge between isomorphic models, impedes the performance of online KD between CNN and ViT. This leads to a drop in performance of -0.71\% in mIoU. The first reason mentioned in the introduction causes this outcome. Our proposed method consistently outperforms SoTA KD methods with larger backbone models, ResNet50 and MiT-B2, by achieving a \textbf{+1.58\%} mIoU increment. This result indicates the superiority of our method, which enables students to learn heterogeneous features from each other to acquire complementary knowledge in the feature space.

Fig.~\ref{cityresults} shows the qualitative results and a comparison of the SoTA KD methods on the PASCAL VOC 2012 dataset. Intuitively, the vanilla KD method (3rd column) and DML (4th column) produce unsatisfactory predictions and even erroneous segmentation. In contrast, the results of our method are closer to the ground truth with better segmentation. \textit{\textbf{These outcomes demonstrate the effectiveness and superiority of the proposed BSD module, which selectively distills the reliable region-wise and pixel-wise knowledge.}}

\noindent\textbf{Results on \textbf{Cityscapes}.} 
Tab. \ref{tab:Results_Cityscapes} presents the quantitative results on Cityscapes validation set. Our proposed method consistently outperforms the SoTA KD methods. In comparison with other online KD methods, our method demonstrates a remarkable increase of mIoU by \textbf{+1.86\%}, which is much higher than the online DML's improvement of +0.63\%, while KDCL and IFVD show only +0.74\% and +0.20\% improvements in mIoU, respectively. Our method also outperforms the offline KD methods, Offline KD and IFVD, in terms of segmentation results. Specifically, our method achieves an improvement of \textbf{+2.14\%} in mIoU with the larger backbone ResNet-50 and MiT-B2, while IFVD and KDCL only show \textbf{+0.83\%} and \textbf{+1.15\%} improvements, respectively.
\begin{table}[t!]
\captionsetup{font=small}
  \renewcommand{\tabcolsep}{5pt}
    \centering
    \small
    \begin{tabular}{|l|c|c|c|c|c|}
\hline
$\mathcal{L}_{s}$&$\mathcal{L}_{\text{HFD}}$&$\mathcal{L}_{\text{BSD}}$ & MobileNetV2&MiT-B1&$\Delta$\\
    \hline
    \hline
    \makecell[c]{\cmark}&\makecell[c]{\xmark}&\makecell[c]{\xmark}&67.54 &78.48&0 \\
    \makecell[c]{\cmark}&\makecell[c]{\cmark}&\makecell[c]{\xmark}&$67.89_{\textcolor{red}{+0.35}}$&$78.78_{\textcolor{red}{+0.30}}$&\textcolor{red}{+0.65} \\
    \makecell[c]{\cmark}&\makecell[c]{\xmark}&\makecell[c]{\cmark}&$69.19_{\textcolor{red}{+1.65}}$&$78.91_{\textcolor{red}{+0.43}}$&\textcolor{red}{+2.08} \\
    \makecell[c]{\cmark}&\makecell[c]{\cmark}&\makecell[c]{\cmark} &$69.57_{\textcolor{red}{+2.03}}$&$79.17_{\textcolor{red}{+0.0.69}}$&\textcolor{red}{+2.72}\\
   \hline
    \end{tabular}
    \caption{Ablation of two components of the proposed method evaluated on \textbf{PASCAL VOC 2012}.}
    \vspace{-10pt}
    \label{tab:ablation1}
    \vspace{-5pt}
\end{table}
\begin{table}[t!]
\captionsetup{font=small}
  \renewcommand{\tabcolsep}{10pt}
    \centering
    \small
    \begin{tabular}{|l|c|c|c|c|}
    \hline
    $\mathcal{L}_{R}$&$\mathcal{L}_{P}$& MobileNetV2&MiT-B1&$\Delta$\\
    \hline\hline
    \makecell[c]{\xmark}&\makecell[c]{\xmark}&67.54 &78.48&0 \\
    \makecell[c]{\cmark}&\makecell[c]{\xmark}&$67.92_{\textcolor{red}{+0.38}}$& $78.82_{\textcolor{red}{+0.34}}$&\textcolor{red}{+0.72}\\
    \makecell[c]{\xmark}&\makecell[c]{\cmark}&$68.87_{\textcolor{red}{+1.33}}$&$78.81_{\textcolor{red}{+0.33}}$&\textcolor{red}{+1.66}\\
    \makecell[c]{\cmark}&\makecell[c]{\cmark}&$69.19_{\textcolor{red}{+1.65}}$&$78.91_{\textcolor{red}{+0.43}}$&\textcolor{red}{+2.08}\\
    \hline
    \end{tabular}
    \vspace{-5pt}
    \caption{Ablation of two components of BSD evaluated on \textbf{PASCAL VOC 2012}.}
    \label{tab:ablation2}
    \vspace{-10pt}
\end{table}

\noindent\textbf{Results on CamVid.} Tab. \ref{tab:Results_Camvid} presents a comparison of our proposed method with SoTA KD methods on the CamVid dataset. The results demonstrate the significant performance improvement of ViT-based and CNN-based student models achieved by our method. In contrast to the students without distillation, our method produces a remarkable improvement of \textbf{1.81\%}, \textbf{0.94\%}, \textbf{1.29\%}, and \textbf{1.48\%} in MobileNetV2, MiT-B1, ResNet-50, and MiT-B2, respectively. While most of the previous KD methods show decreased performance on this dataset, our method exhibits better generalization and enables the students to learn collaboratively. Additionally, our method outperforms the compared KD methods, regardless of the choice of different architectures and backbones for the student networks.

\begin{table*}[t]
  \renewcommand{\tabcolsep}{5pt}
  \captionsetup{font=small}
    \centering
    \begin{tabular}{|l|c|c|c|c|c|c|}
    \hline
    Method&MobileNetV2&MiT-B2&$\Delta$&ResNet-50&MiT-B1&$\Delta$\\
    \hline\hline
    Vanilla&67.54&82.03&0.00&76.05&78.48&0.00\\
    \textbf{Ours}&$\textbf{69.21}_{\textcolor{red}{+1.67}}$&$\textbf{82.27}_{\textcolor{red}{+0.24}}$&\textcolor{red}{\textbf{+1.91}}&$\textbf{77.59}_{\textcolor{red}{+1.54}}$&$\textbf{79.56}_{\textcolor{red}{+1.08}}$&\textcolor{red}{\textbf{+2.62}}\\
    \hline
    \end{tabular}
    \vspace{-5pt}
    \caption{Comparison with the Vanilla  methods on the \textbf{PASCAL VOC 2012} dataset for our CNN-based (MobileNetV2 and ResNet-50) and ViT-based (MiT-B1 and MiT-B2) students.}
    \label{tab:coma}
    \vspace{-10pt}
\end{table*}

\begin{table}[t!]
  \renewcommand{\tabcolsep}{6pt}
  \captionsetup{font=small}
  \small
    \centering
    \scalebox{0.78}{
    \begin{tabular}{|c|c|c|c|c|c|}
    \hline
    $\alpha$&0.1&0.5&1.0&2.0&5.0\\
    \hline\hline
     MobileNetV2&$68.14_{\textcolor{red}{+1.41}}$&$68.76_{\textcolor{red}{+1.22}}$&$69.19_{\textcolor{red}{+1.65}}$&$69.47_{\textcolor{red}{+1.93}}$&$70.69_{\textcolor{red}{+3.15}}$\\
    MiT-B1  &$78.94_{\textcolor{red}{+0.46}}$&$78.83_{\textcolor{red}{+0.35}}$&$78.91_{\textcolor{red}{+0.48}}$&$78.55_{\textcolor{red}{+0.07}}$&$78.25_{\textcolor{blue}{-0.23}}$\\
    $\Delta$&\textcolor{red}{1.06} &\textcolor{red}{+1.57}&\textcolor{red}{+2.08}&\textcolor{red}{+2.00}&\textcolor{red}{+2.92}\\
    \hline
    \end{tabular}}
    \caption{Sensitivity of $\alpha$ evaluated on \textbf{PASCAL VOC 2012}.}
    \label{tab:Sensitivity to loss parameter2}
    \vspace{-10pt}
\end{table}

\begin{table}[t!]
  \renewcommand{\tabcolsep}{3.5pt}
    \centering
    \small
    \captionsetup{font=small}
    \begin{tabular}{|c|c|c|c|c|c|c|}
    \hline
    $\beta$&0.05&0.1&0.5&1.0\\
    \hline\hline
    MobileNetV2&$67.60_{\textcolor{red}{+0.06}}$&$67.89_{\textcolor{red}{+0.35}}$&$67.76_{\textcolor{red}{+0.22}}$&$67.44_{\textcolor{blue}{-0.10}}$\\
    MiT-B1 &$78.70_{\textcolor{red}{+0.22}}$&$78.78_{\textcolor{red}{+0.30}}$&$78.69_{\textcolor{red}{+0.21}}$&$78.43_{\textcolor{blue}{-0.05}}$\\
    $\Delta$&\textcolor{red}{+0.28}&\textcolor{red}{+0.65}&\textcolor{red}{+0.43}&\textcolor{blue}{-0.15}\\
    \hline
    \end{tabular}
    \caption{Sensitivity of $\beta$ evaluated on \textbf{PASCAL VOC 2012}.}
    \label{tab:Sensitivity to loss parameter1_beta}
    \vspace{-13pt}
\end{table}
\subsection{Ablation study and analysis}
\noindent\textbf{Effectiveness of the two proposed modules.} We investigate the impact of enabling and disabling the two components of our proposed method on the PASCAL VOC 2012 dataset using MobileNetV2 and MiT-B1. Tab. \ref{tab:ablation1} reports the results of the different student settings. The table shows that both proposed components can enhance the performance of both students, and the selection of reliable knowledge aids better collaborative learning. Specifically, the BSD module improves performance by \textbf{2.08\%}, demonstrating the effectiveness of selecting reliable knowledge to transfer between heterogeneous models.

\noindent\textbf{Effectiveness of the Two Components of BSD.} Tab. \ref{tab:ablation2} demonstrates the effectiveness of the different components in the BSD module. The CNN-based and ViT-based students with the region-wise distillation module achieve results of \textbf{67.92\%} and \textbf{78.42\%}, respectively. The pixel-wise distillation module boosts the improvement to \textbf{2.08\%}, a substantial enhancement to the students' performance. We observe that pixel-wise distillation can enhance CNNs with a small capacity guided by ViT with a larger capacity. However, distilling knowledge from CNNs improves ViT slightly, indicating a further potential for exploration.

\noindent\textbf{Sensitivity of $\alpha$, $\beta$, and $\gamma$.} Tab. \ref{tab:Sensitivity to loss parameter2}, \ref{tab:Sensitivity to loss parameter1_beta}, and \ref{tab:Sensitivity to loss parameter1_gama} report the mIoU(\%) of the students with different ratios of $\alpha$, $\beta$, and $\gamma$ on the PASCAL VOC 2012 dataset. The students' encoders are MobileNetV2 and MiT-B1.

As shown in Tab. \ref{tab:Sensitivity to loss parameter2}, increasing the importance of $\alpha$ significantly improves the performance of CNNs due to the proposed BSD module's transfer of reliable knowledge from ViT to CNN. However, the performance of ViT degrades slightly when $\alpha$=5.0. As our goal is to facilitate the two students' learning from each other, we choose $\alpha$=1.0 as it presents the best trade-off for both students.

From Tab. \ref{tab:Sensitivity to loss parameter1_beta}, it can be seen that aligning the heterogeneous features in the low-layer feature space directly may degrade the students' performance due to the considerable learning capacity gap between CNNs and ViT. Therefore, we set $\beta$ as 0.1 to facilitate the students to learn from each other, which leads to an absolute improvement of \textbf{+0.65\%}, demonstrating that the HFD module enables the students to learn the heterogeneous features from each other for complementary knowledge in the low-layer feature space.

In Tab. \ref{tab:Sensitivity to loss parameter1_gama}, it shows that as $\gamma$ increases, the performance of CNNs is continually improved, but the performance of ViT degrades slightly. Therefore, we set $\gamma$ to 1.0 as it shows the best trade-off for improving the performance of both CNNs and ViT simultaneously. The results prove that our approach is suitable for situations when there is a significant performance gap between homogeneous students.

\noindent\textbf{Students with different performance ability.} To demonstrate the effectiveness of our proposed method, we conduct experiments with two pairs: MobileNetV2 and MiT-B2, and ResNet-50 and MiT-B1. Tab. \ref{tab:coma} shows the quantitative results on the PASCAL VOC 2012 dataset. Compared to the vanilla methods, our proposed method achieves a dramatic increase in mIoU by \textbf{+1.91\% }and \textbf{+2.62\%}, respectively. The results demonstrate the effectiveness of our proposed method in improving the performance of different heterogeneous students with varying performance abilities. \textit{More details of the method can be found in
suppl. material.}

\begin{table}[t!]
  \renewcommand{\tabcolsep}{6pt}
  \captionsetup{font=small}
    \centering
    \small
    \scalebox{0.78}{
    \begin{tabular}{|c|c|c|c|c|c|}
    \hline
    $\gamma$& 0.1&0.5&1.0&2.0&5.0\\
    \hline\hline
     MobileNetV2&$68.57_{\textcolor{red}{+1.03}}$&$68.92_{\textcolor{red}{+1.38}}$&$69.03_{\textcolor{red}{+1.49}}$&$69.08_{\textcolor{red}{+1.54}}$&$70.12_{\textcolor{red}{+2.58}}$\\
    MiT-B1 &$79.10_{\textcolor{red}{+0.62}}$&$78.95_{\textcolor{red}{+0.47}}$&$78.94_{\textcolor{red}{+0.46}}$&$78.60_{\textcolor{red}{+0.12}}$&$78.30_{\textcolor{blue}{-0.18}}$\\
    $\Delta$&\textcolor{red}{+1.65}&\textcolor{red}{+1.85}&\textcolor{red}{+1.95}&\textcolor{red}{+1.66}&\textcolor{red}{+2.40}\\
    \hline
    \end{tabular}}
    \vspace{-5pt}
    \caption{Sensitivity of $\gamma$ evaluated on \textbf{PASCAL VOC 2012}.}
    \label{tab:Sensitivity to loss parameter1_gama}
    \vspace{-10pt}
\end{table}



% \begin{table*}
%     \centering
    
%     \begin{tabular}{l|c|c|c|c|c|c}
%     \toprule
%     Method& MobileNetV2&MiT-B1&Gain&ResNet-50&MiT-B2&Gain\\
%     \midrule
%     Vanilla &72.31&79.15&0&78.15&82.24&0 \\
%     0.0025/0.00006&67.42&78.53&&76.00&82.15\\
%     0.00025/0.0006&60.62& 71.53&&67.99&73.74\\
%     0.00025/0.00006& 61.73&79.04&&68.45&82.17\\
%     \bottomrule
%     \end{tabular}
%     \caption{Ablation study about learning rate on PASCAL VOC 2012}
%     \label{tab:VOC}
% \end{table*}

% \begin{table*}
%     \centering
    
%     \begin{tabular}{l|c|c|c|c|c|c}
%     \toprule
%     Method& MobileNetV2&MiT-B1&Gain&ResNet-50&MiT-B2&Gain\\
%     \midrule
%     Vanilla &72.31&79.15&0&78.15&82.24&0 \\
%     0.1&67.69&79.23\\
%     0.5&\\
    

%     \bottomrule
%     \end{tabular}
%     \caption{Ablation study about trade off parameter on PASCAL VOC 2012}
%     \label{tab:VOC}
% \end{table*}

% \begin{table*}
%     \centering
    
%     \begin{tabular}{l|c|c|c|c|c|c}
%     \toprule
%     Method& MobileNetV2&MiT-B1&Gain&ResNet-50&MiT-B2&Gain\\
%     \midrule
%     Vanilla &72.31&79.15&0&78.15&82.24&0 \\
%     ResNet50-ResNet-50&\\
%     MiT1-MiT1&&78.10\\
%     \bottomrule
%     \end{tabular}
%     \caption{Ablation study about same models on PASCAL VOC 2012}
%     \label{tab:VOC}
% \end{table*}

% \begin{table*}
%     \centering
    
%     \begin{tabular}{l|c|c|c|c|c|c}
%     \toprule
%     Method& MobileNetV2&MiT-B1&Gain&ResNet-50&MiT-B2&Gain\\
%     \midrule
%     Vanilla (new)&66.87&78.82 \\
%     Mutual-0.1 (new2)& 66.79&78.47&\\
%     Mutual-0.1-warm up (new3)&64.20&77.37\\
%     Mutual-1-warm up(test-62)&66.6&78.40\\
%     Mutual-1 (test2-63)&64.21&77.81\\
%     Mutual-1-war (Test)(Mob-Mob)&66.51&\\
%     Mutual-0.1 (test2)(SegB1-SegB1)&&78.31/78.01\\
%     Mutual-1 (New-KL Test) (SegB1-SegB1)\\
%     Mutual-1 (test2) (Vanilla)&67.71&77.92\\
%     Mutual-1 (Test) (SegB1-1e-4)& &78.48/79.11\\
%     Mutual-10 (Test-10)(SegB1)&&78.35/77.93\\
%     Mutual-10 (test2-10-e-4)(SegB1)&&79.12/78.18\\
%     Mutual-10 (Test warm-10-1e-4)(SegB1)&&78.43/79.00\\
%     Mutual-10 (Test 100-e-4)(Mob)&64.70/64.93&\\
%     Mutual-10 (s-64 10)(Mob)&67.22/67.34&\\
    
    
%     \bottomrule
%     \end{tabular}
%     \caption{New Ablation study}
%     \label{tab:VOC}
% \end{table*}

% \begin{table*}
%     \centering
    
%     \begin{tabular}{l|c|c|c|c|c|c}
%     \toprule
%     Method& MobileNetV2&MiT-B1&Gain&ResNet-50&MiT-B2&Gain\\
%     \midrule
%     Vanilla (new)&68.73&78.48 \\
%     Mutual(1)&66.42&78.10\\
%     mutual(1-warm)&67.43&78.76\\
%     \bottomrule
%     \end{tabular}
%     \caption{Newest Ablation study}
%     \label{tab:VOC}
% \end{table*}
\section{Conclusion}
This paper presented the \textbf{first} online KD framework to collaboratively learn compact yet effective CNN-based and ViT-based models for semantic segmentation. Specifically, we proposed the heterogeneous feature distillation (HFD) module to improve students' consistency in the low-layer feature space by mimicking heterogeneous features between CNNs and ViT. Then, we also proposed bidirectional selective distillation (BSD) to select reliable region-wise and pixel-wise knowledge to transfer and enable students to learn from each other better. Comparison with the SoTA KD methods for semantic segmentation shows that our proposed method significantly outperforms these SoTA methods by a large margin and demonstrates our proposed method's effectiveness for semantic segmentation. 

\noindent \textbf{Limitation and future work:} 
% Due that students learn reliable knowledge with the supervision from each other,
% %from each other with the guidance of cross entropy, 
% the performance improvement of the high-performance model is slight, and the high-performance model lacks reliable guidance from low-performance models. 
% supervision
Our method has one limitation.
% As we focus on learning compact ViT-based and CNN-based models. 
The cross-model distillation may lead to an unbalanced performance gain in the online KD training process. That is, if one student's knowledge is less instructive, the performance of the other student may be marginally improved. 
% the distillation from low-performance student  degrade the high-performance model's performance.
% Therefore, we will further explore improving the model with a larger by using a smaller-performance heterogeneous model in an online KD framework. 
Therefore, we will explore the online KD between the heterogeneous ViT-based and CNN-based models with more distinct learning capacities.
Moreover, it is promising to extend the proposed collaborative learning paradigm to learn other tasks than semantic segmentation or the cross tasks between depth estimation and semantic segmentation.

\noindent \textbf{Acknowledgement}: This joint paper is supported by Alibaba Cloud, Alibaba Group through Alibaba Innovative Research Program, and the National Natural Science Foundation of China (NSF) under Grant No. NSFC22FYT45. 
{\small
\bibliographystyle{ieee_fullname}
\bibliography{egbib}
}

\end{document}
