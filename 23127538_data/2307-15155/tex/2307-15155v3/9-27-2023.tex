\documentclass{article}
\usepackage[utf8]{inputenc}

 \parindent0mm
\textwidth155mm
\textheight200mm
\oddsidemargin0mm
\evensidemargin0mm

\usepackage{latexsym,amsfonts,amssymb,amsmath,amsthm}

\usepackage{cite}

\usepackage{booktabs} % For pretty tables
\usepackage{caption} % For caption spacing   
\usepackage{subcaption} % For sub-figures
\usepackage{graphicx}
\usepackage{graphics}
\usepackage{pgfplots}
\usepackage[all]{nowidow}
\usepackage[utf8]{inputenc}
%\usepackage{tikz}
%\usetikzlibrary{er,positioning,bayesnet}
\usepackage{tikz}
\usetikzlibrary{automata}
\usepackage{multicol,comment}
\usepackage{algpseudocode,algorithm,algorithmicx}
  \usepackage[frozencache=true,cachedir=minted-cache]{minted} 
% \usepackage{hyperref}
\usepackage[inline]{enumitem} % Horizontal lists
% Used for displaying a sample figure. If possible, figure files should
% be included in EPS format.
%
% If you use the hyperref package, please uncomment the following line
% to display URLs in blue roman font according to Springer's eBook style:
% \renewcommand\UrlFont{\color{blue}\rmfamily}



\newcommand{\card}[1]{\left\vert{#1}\right\vert}
\newcommand*\Let[2]{\State #1 $\gets$ #2}
\definecolor{blue}{HTML}{1F77B4}
\definecolor{orange}{HTML}{FF7F0E}
\definecolor{green}{HTML}{2CA02C}

\pgfplotsset{compat=1.14}

\renewcommand{\topfraction}{0.85}
\renewcommand{\bottomfraction}{0.85}
\renewcommand{\textfraction}{0.15}
\renewcommand{\floatpagefraction}{0.8}
\renewcommand{\textfraction}{0.1}
\setlength{\floatsep}{3pt plus 1pt minus 1pt}
\setlength{\textfloatsep}{3pt plus 1pt minus 1pt}
\setlength{\intextsep}{3pt plus 1pt minus 1pt}
\setlength{\abovecaptionskip}{2pt plus 1pt minus 1pt}

%%%%%%%%%%%%%%%%%%%%%%%%%%%%%%
%%%%%%%%%



\newtheorem{tm}{Theorem}[section]
\newtheorem{prop}[tm]{Proposition}
\newtheorem{defin}[tm]{Definition}
\newtheorem{coro}[tm]{Corollary}
\newtheorem{lem}[tm]{Lemma}
\newtheorem{assumption}[tm]{Assumption}
\newtheorem{rk}[tm]{Remark}
\newtheorem{nota}[tm]{Notation}
\numberwithin{equation}{section}
\numberwithin{tm}{section}

%%%%%%%%%%%






\newcommand{\eqd}{\sim}
\def\p{\partial}
\def\R{{\mathbb R}}
\def\N{{\mathbb N}}
\def\Q{{\mathbb Q}}
\def\C{{\mathbb C}}
\def\l{{\langle}}
\def\r{\rangle}
\def\t{\tau}
\def\k{\kappa}
\def\a{\alpha}
\def\la{\lambda}
\def\De{\Delta}
\def\de{\delta}
\def\ga{\gamma}
\def\Ga{\Gamma}
\def\ep{\varepsilon}
\def\si{\sigma}
\def\Re {{\rm Re}\,}
\def\Im {{\rm Im}\,}
\def\E{{\mathbb E}}
\def\P{{\mathbb P}}
\def\Z{{\mathbb Z}}
\def\D{{\mathbb D}}
\newcommand{\ceil}[1]{\lceil{#1}\rceil}



\title{ Multiplicity of endemic equilibria for a diffusive SIS epidemic model with mass-action transmission mechanism }
\author{Keoni Castellano\footnote{castek1@unlv.nevada.edu} \quad  and \quad  Rachidi B. Salako \footnote{rachidi.salako@unlv.edu}  
\\
\\
{\small Department of  Mathematical Sciences,  University of Nevada Las Vegas,}\\
{\small Las Vegas, NV 89154, USA}}
\date{}

\begin{document}

\maketitle

\begin{abstract} 

We study a diffusive   SIS epidemic model with the mass-action transmission mechanism and show, under appropriate assumptions on the parameters, the existence of multiple endemic equilibria (EE) when the basic reproduction number, $\mathcal{R}_0$, is either less or greater than one. Our results answer some   open questions  on previous studies related to  extinction of disease or persistence when $\mathcal{R}_0<1$ and the multiplicity of EE solutions when $\mathcal{R}_0>1$.  Interestingly, even with such a  simple nonlinearity induced by the mass-action transmission mechanism,  we show that the diffusive epidemic model may have an S-shape or backward bifurcation curve of EE solutions. We, in addition,  present results on the nonexistence/existence and uniqueness of endemic equilibrium. 

    
\end{abstract}

\noindent{\bf Keywords}: Infectious Disease Models; Reaction-Diffusion Systems;
Asymptotic Behavior.
\smallskip

{
\noindent{\bf 2010 Mathematics Subject Classification}: 92D25, 35B40, 35K57}

\section{Introduction}

 \quad Infectious diseases remain a leading cause of deaths around the word.  As of May 2023, according to the World Health Organization (WHO)  coronavirus (COVID-19) dashboard, SARS-CoV-2 has infected more than seven hundred million  people  and   claimed more than six million deaths worldwide. This fast spread of the SARS-CoV-2 virus is partly due to globalization that has  made the world more connected. To better predict the dynamics of infectious diseases and  develop effective and adequate control strategies, researchers have incorporated population movement and spatial heterogeneity into  epidemic models.

  \quad Consider   the ODE  Susceptible-Infected-Susceptible (ODE-SIS) mathematical epidemic system 

 \begin{equation}\label{model-001}
     \begin{cases}\frac{dS}{dt}= \gamma I -f(S,I)S & t>0,\cr 
     \frac{dI}{dt}= f(S,I)S-\gamma I & t>0,
     \end{cases}
 \end{equation}
 where $f$ is a locally Lipschitz  function on $\mathbb{R}_+^2$ satisfying $f(S,0)=0$ for $S\ge 0$. The epidemic model \eqref{model-001}  describes the dynamics of a population living on a single patch and affected by  an  infectious disease. In  system \eqref{model-001}, $S$ and $I$ denote the size of the susceptible and infected populations, respectively; $\gamma$ is the disease recovery rate;  the function $f(S,I)$ accounts for the force of infection.  Two force of infection functions are common: $f(S,I)=\beta\frac{I}{I+S}$ referred to as the frequency-dependent transmission mechanism, and $f(S,I)=\beta I$ referred to as the mass-action transmission mechanism. Here, $\beta$ is the disease transmission rate. A basic fact about system \eqref{model-001} is that the total population size $N=S+I$ is preserved. As a result, the system \eqref{model-001} can be reduced to one equation, mainly by replacing $S=N-I$ in its second equation. Moreover, for simple force of infections  (e.g.,   frequency-dependent or  mass-action incidence mechanism) the {\it basic reproduction number } $\mathcal{R}_0^{\rm ODE}$, defined as  the expected number of cases directly generated by one case in a population where all individuals are susceptible to infection, is enough to completely understand the dynamics of the disease:  if $\mathcal{R}_0^{\rm ODE}\le 1$, then the disease can be eventually eradicated, while if $\mathcal{R}_0^{\rm ODE}>1$ then the disease persists and the population will eventually stabilize at a unique endemic equilibrium (EE) solution (i.e., a time independent solution).  If $f(S,I)=\beta\frac{I}{S+I}$ then $\mathcal{R}_0^{\rm ODE}=\frac{\beta}{\gamma}$, while if $f(S,I)=\beta I$ then $\mathcal{R}_0^{\rm ODE}=N\frac{\beta}{\gamma}$. In reality,  all countries  are made of several cities (or  states). People move between these cities  for several reasons---academic, economic, social,  politic, and among others. However, these movements of the population further contribute in   circulating  infectious diseases. Hence, to have a more realistic epidemic model in order to obtain  accurate predictions on the dynamics of the disease, the ODE-SIS  model \eqref{model-001} must be adjusted to incorporate  population movements and spatial heterogeneity of the environment.
 
 

\quad In  2008, to study the effect of population movement and environmental heterogeneity on disease persistence, Allen et al. \cite{Allen2008}  included diffusion of populations into the system \eqref{model-001} with the frequency-dependent transmission mechanism and studied   the diffusive epidemic model 
\begin{equation}\label{e1-prime}
    \begin{cases}
    S_t=d_S\Delta S+\gamma I -\beta \frac{SI}{S+I} & x\in\Omega,\ t>0,\cr
    I_{t}=d_I\Delta I+\beta \frac{SI}{S+I}-\gamma I& x\in\Omega,\ t>0,\cr
     0=\partial_{\vec{n}}S=\partial_{\vec{n}}I & x\in\partial\Omega,\ t>0,\cr
    N=\int_{\Omega}(S+I),
    \end{cases}
\end{equation}
where $\Omega$ is a bounded domain in $\mathbb{R}^n$ ($n\ge 1$) with a smooth boundary $\partial\Omega$ and  $\vec{n}$ denotes the outward unit normal vector at $\partial\Omega$. $S(x,t)$ and $I(x,t)$ are the local densities of the susceptible and infected populations, respectively, hence are location dependent. The positive constants $d_S$ and $d_I$  represent the diffusion rates of the susceptible and infected populations, respectively; $\beta$ and $\gamma$ are positive and H\"older continuous functions on $\overline{\Omega}$.  The authors of \cite{Allen2008} gave a variational formula for the basic reproduction number, denoted as $\mathcal{R}(d_I)$, of \eqref{e1-prime}  (see formula \ref{R-star-eq} below). They then established that the disease will be eventually eradicated if $\mathcal{R}(d_I)<1$, while system  \eqref{e1-prime} has a unique EE solution  if $\mathcal{R}(d_I)>1$ (see Proposition \ref{Proposition1}).  Hence, as in  the corresponding ODE-SIS model \eqref{model-001}, the diffusive epidemic model \eqref{e1-prime} has a (unique) EE if and only if $\mathcal{R}(d_I)>1$. Interestingly, they showed that $\mathcal{R}(d_I)$ is nonincreasing with respect to $d_I$, indicating that lowering the diffusion rate of the infected population may lead to disease persistence.  They further investigated the asymptotic limits of the EE as   $d_S$  approaches zero  and established that the $I$ component of the EE solution approaches zero uniformly in space  if the environment has a nonempty low-risk area (i.e, the set $\{x\in\Omega : \beta(x)<\gamma(x)\}$ is not empty). This suggests that the impact of the disease can be significantly controlled by reducing the susceptible population movement rate. The asymptotic profiles of the EE  as the diffusion rates of the infected population becomes very small or either of the diffusion rates get arbitrarily large are studied in \cite{Peng2009a,Peng_Yi2013}. Partial results on the stability of EE of \eqref{e1-prime} are obtained in \cite{Peng2009b}. 

 \quad Inspired by the above mentioned works, several studies have been devoted to the investigations of diffusive epidemic models (\cite{Cui_Lou2016, Cui2017, MCH2001, LP2022, LSS2023, LouSalako2021, Peng_Shi2008, DeJong1995, GKLZ2015, Peng_Zhao, Salako2023_1, TW2023} and the references therein). In particular, Deng and Wu \cite{DengWu2016}  considered the diffusive counterpart of the ODE-SIS epidemic model \eqref{model-001} with the mass-action transmission mechanism: 

\begin{equation}\label{e1}
    \begin{cases}
    S_t=d_S\Delta S+\gamma I -\beta SI & x\in\Omega,\ t>0,\cr
    I_{t}=d_I\Delta I+\beta SI-\gamma I& x\in\Omega,\ t>0,\cr
     0=\partial_{\vec{n}}S=\partial_{\vec{n}}I & x\in\partial\Omega,\ t>0,\cr
    N=\int_{\Omega}(S+I).
    \end{cases}
\end{equation}
The variables in \eqref{e1} have the same meanings as those in \eqref{e1-prime}.   They also defined the basic reproduction number, denoted as $\mathcal{R}_0(N,d_I)$, of \eqref{e1} (see \eqref{R-0-def} below) and established the existence of EE when $\mathcal{R}_0(N,d_I)>1$. Furthermore, they obtained some partial results on the nonexistence of EE of \eqref{e1} when $\mathcal{R}_0(N,d_I)<1$ (see Proposition \ref{Proposition2}). The works \cite{CastellanoSalako2021, Wu_Zou2016, Wen2018} further studied the asymptotic profiles of EE of \eqref{e1} as the diffusion rates of the population get small or large. However, the question of existence of the EE solutions of \eqref{e1} when its basic reproduction is less than one was left open by all these works. In particular, it is unclear whether \eqref{e1} may have multiple EE solutions. 

 \quad At first glance, given the fact that the structure of the EE solutions of the diffusive epidemic model \eqref{e1-prime} is similar to that of the corresponding ODE-SIS model \eqref{model-001} (since the existence and uniqueness of such solutions are completely determined by whether the basic reproduction number is bigger than one), one may erroneously expect that the structure of the EE solutions of the diffusive epidemic model \eqref{e1} would be as simple as that of its corresponding ODE model. Surprisingly, as we shall see later, our results show that the diffusive epidemic model \eqref{e1} with the mass-action transmission mechanics may lead to complicated dynamics of the disease. This strongly highlights how the prediction of  disease dynamics may immensely depend on the force of infection used in the mathematical model.

\quad Our aim, in this study, is to investigate the global structure (i.e, multiplicity and uniqueness) of the EE solutions  of the diffusive epidemic model \eqref{e1}. In particular, we show that
% unlike the ODE model \eqref{model-001} (either with the mass-action or frequency-dependent transmission mechanisms) and the diffusive epidemic model \eqref{e1-prime}, we will obtain that the structure of the EE solutions of the diffusive epidemic model \eqref{e1} may be very rich. In fact,
the diffusive epidemic model \eqref{e1} may have an S-shape (see Theorem \ref{T2}) or backward  (see Theorem \ref{T2-2} and Remark \ref{RK3}) bifurcation curve of EE solutions for certain range of  parameters when the diffusion rate $d_S$ of the susceptible population is very small and the total population size $N$ falls within some range. However, when $d_S$ is slightly smaller than $d_I$ or $N$ is relatively large, then the structure of the EE solutions of the diffusive epidemic model \eqref{e1} is similar to  its corresponding ODE model (see Theorem \ref{T1}).   In general, when the total population size is sufficiently small, irrespective of the diffusion rate of the susceptible population, system \eqref{e1} has no EE solution (see Theorem \ref{T0}).





\quad In Section \ref{main-results}, we state our main results. Section \ref{Preliminaries} contains some preliminaries  essential for the proofs of our main results, which are presented in Section. \ref{proofs}










   
   
   
   






\section{Main Results}\label{main-results}


\quad  We say that   $(S,I)$ is an equilibrium solution of \eqref{e1} if it is a classical solution of 
\begin{equation}\label{e2}
    \begin{cases}
  0=d_S\Delta S+\gamma I -\beta SI & x\in\Omega,\cr
    0=d_I\Delta I+\beta SI-\gamma I& x\in\Omega,\cr
     0=\partial_{\vec{n}}S=\partial_{\vec{n}}I & x\in\partial\Omega,\cr
    N=\int_{\Omega}(S+I).
    \end{cases}
\end{equation}
 A solution of system \eqref{e2} of the form $(S,0)$ is called a disease free equilibrium (DFE). Observe that $(\frac{N}{|\Omega|},0)$  is the unique DFE of \eqref{e1}.  
 An equilibrium solution for which $I>0$  is called an endemic equilibrium (EE).


\quad Given $q\in[1,\infty)$ and a positive integer $k\ge 1$, let $L^q(\Omega)$ denote the Banach space of $L^q$-integrable functions on $\Omega$, and  $W^{k,q}(\Omega)$ the usual Sobolev space. For every $d_I>0$, define 
\begin{equation}\label{R-star-eq}
    \mathcal{R}(d_I)=\sup_{\varphi\in W^{1,2}(\Omega)\setminus\{0\}}\frac{\int_{\Omega}\beta\varphi^2}{\int_{\Omega}[d_I|\nabla \varphi|^2+\gamma\varphi^2]}.
\end{equation}
It is well known (see \cite{Allen2008}) that the supremum in \eqref{R-star-eq} is achieved and there is a unique positive function $\varphi_{d_I}\in C^2(\overline{\Omega})$ with $\|\varphi_{d_I}\|_{L^2(\Omega)}=1$ satisfying 
\begin{equation}\label{R-star-pde}
    \begin{cases}
    0=d_I\Delta \varphi-\gamma\varphi+\frac{1}{\mathcal{R}(d_I)}\beta\varphi & x\in\Omega,\cr 
    0=\partial_{\vec{n}}\varphi & x\in\partial\Omega.
    \end{cases}
\end{equation}
Furthermore, any solution of \eqref{R-star-pde} is spanned by $\varphi_{d_I}$.  %We recall some important properties of $\mathcal{R}(d_I)$ with respect to $d_I$ in  Lemma \ref{lem1}. 
Thanks to \cite{Allen2008}, the quantity $\mathcal{R}(d_I)$ is the basic reproduction number for \eqref{e1-prime}. Moreover, when the  assumption {\bf (A)},

\medskip

\noindent{\bf (A)} The function $\frac{\beta}{\gamma}$ is not constant,

\medskip

holds, it follows from Lemma \ref{lem1} that $\mathcal{R}(d_I)$ is strictly decreasing in $d_I$, and has an inverse function, which we denote by $\mathcal{R}^{-1}$. Throughout the remainder of this work, we shall always suppose that assumption {\bf (A)} holds. It follows from \cite{DengWu2016} that  the quantity  $\mathcal{R}_0(N,d_I)$, defined by 
\begin{equation}\label{R-0-def}
    \mathcal{R}_0(N,d_I)=\frac{N}{|\Omega|}\mathcal{R}(d_I),
\end{equation}
is the basic reproduction number of \eqref{e1}. It is clear from \eqref{R-0-def} that $\mathcal{R}_0(N,d_I)$ is strictly increasing with respect to $N$.   Hence, at a glance, the dependence of $\mathcal{R}_0(N,d_I)$ on the total population size $N$ yields an important difference between the prediction on the persistence of the infectious disease based on the mathematical models  \eqref{e1} and \eqref{e1-prime}.  For instance, if $\frac{N}{|\Omega|}<1$ (resp. $\frac{N}{|\Omega|}>1$), then the epidemic model \eqref{e1} predicts a lower (resp. higher) basic reproduction number compared to that predicted by the epidemic model \eqref{e1-prime}. In fact,  the dependence of $\mathcal{R}_0(N,d_I)$ on $N$  also induces  fundamental facts on the dynamics of classical solutions of \eqref{e1} which do not hold for classical solutions of \eqref{e1-prime}.  Indeed, regarding the existence and uniqueness of EE solution of \eqref{e1-prime}, Allen et al. \cite{Allen2008} proved  that:% following result.

\begin{prop}\cite{Allen2008, Cui2017} \label{Proposition1}\begin{itemize} \item[\rm (i)]For every $d_S>0$ and $d_I>0$, the diffusive epidemic model \eqref{e1-prime} has a (unique) EE if and only if $\mathcal{R}(d_I)>1$.

\item[\rm (ii)] If $\mathcal{R}(d_I)\le 1$, then for every $d_S>0$, the DFE is globally stable for classical solutions of \eqref{e1-prime}.

    \end{itemize}
\end{prop}

\quad This result shows that the existence of EE solutions of \eqref{e1-prime} is completely determined by the sign of $\mathcal{R}(d_I)-1$ and is independent of the diffusion rate of the susceptible population and the total size of the population. Moreover, when $\mathcal{R}(d_I)\le 1$, the diffusive epidemic model \eqref{e1-prime} predicts that the disease will be eradicated in the long run.  It becomes a natural question to know whether the results of Proposition   \ref{Proposition1} on \eqref{e1-prime} extend to system \eqref{e1}. Mainly: (i) Does it also hold that the epidemic model \eqref{e1}  has a (unique) EE solution if and only if $\mathcal{R}_0(N,d_I)>1$? (ii) Does the epidemic model \eqref{e1} predict the eventual extinction of the disease when $\mathcal{R}_0(N,d_I)<1$? Surprisingly, as seen from Theorem \ref{T1} below,  the answers to these two question could be negative. 

\quad For convenience, let us recall the following result on the existence/nonnexistence and uniqueness of EE solution of \eqref{e1} from previous studies.

\begin{prop}\cite{DengWu2016}\label{Proposition2}
\begin{itemize}
     \item[\rm (i)] If $d_S\ge d_I$, then \eqref{e1} has a (unique) EE solution if and only if $\mathcal{R}_0(N,d_I)>1$.
     \item[\rm (ii)] If $d_S<d_I$, then  \eqref{e1} has no EE if $\mathcal{R}_0(N,d_I)\le \frac{d_S}{d_I}$ and has at least one EE if $\mathcal{R}_0(N,d_I)>1$.
\end{itemize}
    
\end{prop}

\quad Thanks to Proposition \ref{Proposition2}, when $d_S\ge d_I$, \eqref{e1} has a (unique) EE if and only if $\mathcal{R}_0(N,d_I)>1$. However, when $d_S<d_I$, it is not clear whether \eqref{e1} has an EE solution if $\frac{d_S}{d_I}<\mathcal{R}_0(N,d_I)\le 1$. To address these questions, we first establish  the following result.



\begin{tm}\label{T0}
 Fix $d_I>0$. There is $0<N_{\rm low}(d_I)\le \min\Big\{ \frac{|\Omega|}{\mathcal{R}(d_I)},\int_{\Omega}\frac{\gamma}{\beta}\Big\}$ such that the following  hold.
\begin{itemize}
    \item[\rm (i)] For every $N>0$ satisfying $N\le N_{\rm low}(d_I)$,  \eqref{e1}  has no EE solution for every $d_S>0$. 
    
    \item[\rm (ii)] For every $N>0$ satisfying $N>N_{\rm low}(d_I)$,  there is $d(N,d_I)>0$  such that \eqref{e1} has an EE solution $(S_{\rm high}(\cdot;d_S),I_{\rm high}(\cdot;d_S))$ for every $0<d_S<d(N,{d_I})$. Furthermore,  any other EE solution $(S(\cdot;d_S),I(\cdot;d_S))$ of \eqref{e1}, if exists, must satisfy
    \begin{equation}\label{T0-eq1}
       I(x;d_S)<I_{\rm high}(x;d_S) \quad \forall\ x\in\Omega. 
    \end{equation}
    
    
    \item[\rm (iii)]If $N_{\rm low}(d_I)<\frac{|\Omega|}{\mathcal{R}(d_I)}$, then for every $N_{\rm low}(d_I)<N<\frac{|\Omega|}{\mathcal{R}(d_I)}$ and $0<d_S<d(N,{d_I})$, \eqref{e1} has a   EE solution $(S_{\rm low}(\cdot;d_S),I_{\rm low}(\cdot;d_S))$ satisfying 
    \begin{equation}\label{T0-eq2}
        I_{\rm low}(x;d_S)<I_{\rm high}(x;d_S) \quad \forall\ x\in\overline{\Omega},
    \end{equation}
    such that any other EE solution $(S(\cdot;d_S),I(\cdot;d_S))$ of \eqref{e1}, if exists, must satisfy
    \begin{equation}\label{T0-eq3}
        I_{\rm low}(x;d_S)<I(x;d_S)\quad \forall\ x\in\Omega. 
    \end{equation}
    Furthermore, the following hold.
    \begin{itemize}
        \item[\rm (iii-1)] If $N<\int_{\Omega}\frac{\gamma}{\beta}$, then there is a positive constant $C>0$ such that \begin{equation}\label{T0-eq4}
        \frac{d_{S}}{C}\le I_{\rm low}(\cdot)< I_{\rm high}(\cdot)\le Cd_{S}\quad  \forall\ 0<d_S<\frac{d(N,d_I)}{2}
    \end{equation}
     Moreover, as $d_S\to 0^+$,
    \begin{equation}\label{T0-eq5}
        S_{\rm high}\to S_{\rm high}^*(\cdot,d_I):=\frac{N(1-d_Iu^*_{\rm high})}{\int_{\Omega}(1-d_Iu^*_{\rm high})} \quad \text{and}\quad  S_{\rm low}\to S_{\rm low}^*(\cdot,d_I):=\frac{N(1-d_Iu^*_{\rm low})}{\int_{\Omega}(1-d_Iu^*_{\rm low})}
    \end{equation}
    in $C^{1}(\overline{\Omega})$ where $ 0<u^*_{\rm low}<u^*_{\rm high}<\frac{1}{d_I}$ are  classical solutions of the nonlocal elliptic equation
    \begin{equation}\label{T0-eq6}
    \begin{cases}
    0=d_I\Delta u^*+\big( \frac{N\beta}{\int_{\Omega}(1-d_Iu^*)}(1-d_Iu^*)-\gamma\big)u^* & x\in\Omega,\cr
    0=\partial_{n}u^* & x\in\partial\Omega.
    \end{cases}
    \end{equation}

        \item[\rm (iii-2)] If $N>\int_{\Omega}\frac{\gamma}{\beta}$, then 
        \begin{equation}\label{T0-eq7}
        \lim_{d_S\to0^+}\left[\Big\|S_{\rm high}(\cdot;d_S)-\frac{\gamma}{\beta}\Big\|_{\infty}+\Big\|I_{\rm high}(\cdot;d_S)-\frac{1}{|\Omega|}\Big(N-\int_{\Omega}\frac{\gamma}{\beta}\Big)\Big\|_{\infty}\right] = 0
    \end{equation}
    and $(S_{\rm low}(\cdot;d_S),I_{\rm low}(\cdot;d_S))$ satisfies \eqref{T0-eq4} and \eqref{T0-eq5}.
    \end{itemize}
  \end{itemize}  
    
\end{tm}
Let $d_I>0$ and $ N_{\rm low}(d_I) $ be given by Theorem \ref{T1}. It follows from Theorem \ref{T0}-{\rm (i)} and {\rm (ii)} that the quantity  $\mathcal{R}_0(N_{\rm low}(d_I),d_I)$ is a sharp critical number that the basic reproduction number must exceed for the existence of  EE of system \eqref{e1} for some range of the diffusion rate of susceptible population. Note that since $\mathcal{R}_0(N,d_I)$ is strictly increasing in $N$ and $\mathcal{R}_0(\frac{|\Omega|}{\mathcal{R}(d_I)},d_I)=1$, then $\mathcal{R}_0(N_{\rm low}(d_I),d_I)\le 1$. As a result, if $N_{\rm low}(d_I)<\frac{|\Omega|}{\mathcal{R}(d_I)}$, then for every $N\in (N_{\rm low}(d_I),\frac{|\Omega|}{\mathcal{R}(d_I)})$, $\mathcal{R}_0(N,d_I)<1$ and Theorem \ref{T0}-{\rm (iii)}  shows that there is $d(N,d_I)>0$ such that \eqref{e1} has at least two EE solutions for every $0<d_S<d(N,d_I)$. Clearly, we note from Proposition \ref{Proposition2}-{\rm (i)} that $d(N,S)\le d_I$. Hence, it is important to know whether $d(N,{d_I})=d_I$ for some range of the parameters $N$. This question is also related to whether the lower bound $d_I$ for $d_S$ in Proposition \ref{Proposition2}-{\rm (i)} is sharp. In this regards, we have the the following result. 



\begin{tm}\label{T1}For every $d_I>0$, there is a positive number $m^*_{d_I}$ satisfying 
     \begin{equation}\label{T1-main-eq}
         0<m^*_{d_I}\le \min\Big\{ 1,\Big(|\Omega|\int_{\Omega}\beta\varphi_{d_I}^3\Big)/{\Big(\int_{\Omega}\varphi_{d_I}\Big)\Big(\int_{\Omega}\beta\varphi_{d_I}^2\Big)} \Big\},
     \end{equation}
     where $\varphi_{d_I}$ is a positive solution of \eqref{R-star-pde}, such that for every $d_S>d_I(1-m^*_{d_I})$, \eqref{e1} has a (unique) EE solution if and only if $\mathcal{R}_0(N,d_I)>1$. However, if $0<m_{d_I}^*<1$ and $0<d_S<d_I(1-m_{d_I}^*)$, then there is $N= N_{d_S,d_I}>0$ such that \eqref{e1} has at least two EE solutions.
    
    
\end{tm}

Fix $d_I>0$. It is clear from \eqref{T1-main-eq} that $m_{d_I}^*>0$, hence $d_I>d_I(1-m^*_{d_I})$. Therefore, Theorem \ref{T2} significantly improves Proposition \ref{Proposition2}-{\rm (i)} by providing a sharp lower bound for the diffusion rate of the susceptible population after which system \eqref{e1} has a (unique) EE solution if and only if its basic reproduction number is greater than one. Moreover, the quantity $d_{I}(1-m_{d_I}^*)$ serves as a sharp upper bound for $d(N,{d_I})$ for every $N>0$.  It is clear from \eqref{T1-main-eq} again that if $|\Omega|\int_{\Omega}\beta\varphi_{d_I}^3<\Big[\int_{\Omega}\varphi_{d_I}\Big]\Big[\int_{\Omega}\beta\varphi_{d_I}^2\Big]$, then  $m_{d_I}^*<1$. Our next result discusses the sufficient conditions to have either $m^*_{d_I}<1$ or $N_{\rm low}(d_I)<|\Omega|/\mathcal{R}(d_I)$.

\begin{tm}\label{T2-2}Fix $d_I>0$ and let $N_{\rm low}(d_I)$ and $m_{d_I}^*$ be given by Theorems \ref{T0} and \ref{T1}, respectively. It always holds that $m_{d_I}^*<1$ whenever $N_{\rm low}(d_I)<\frac{|\Omega|}{\mathcal{R}(d_I)}$.  Furthermore, the latter inequality holds if either: {\rm (i)} $\frac{1}{\mathcal{R}(d_I)}>\frac{1}{|\Omega|}\int_{\Omega}\frac{\gamma}{\beta}$, or {\rm(ii)} ${|\Omega|\int_{\Omega}\beta\varphi_{d_I}^3}<{\big(\int_{\Omega}\varphi_{d_I}\big)\big(\int_{\Omega}\beta\varphi_{d_I}^2\big)} $.
    
\end{tm}

\begin{rk}\label{RK3}
\begin{itemize}
    \item[\rm (i)] Theorem \ref{T2-2} identified sufficient conditions that lead to the multiplicity of EE solutions of \eqref{e1} for small diffusion rate of the susceptible population $d_S$. In fact, as shall be discussed later in Remark \ref{RK1} below, the condition of Theorem \ref{T2-2}-{\rm (ii)} gives a backward bifurcation curve of the EE solutions of \eqref{e1}. Proposition \ref{appen-prop3}-{\rm(i)} of the Appendix gives some examples of the parameters satisfying all the requirements of \ref{T2-2}-{\rm (ii)}.

    \item[\rm (ii)] If \begin{equation}\label{lem2-eq3}
       \frac{1}{{|\Omega|}}{\int_{\Omega}\frac{\gamma}{\beta}}<\Big({ \frac{1}{|\Omega|}\int_{\Omega}\gamma}\Big)/\Big({\frac{1}{|\Omega|}\int_{\Omega}\beta}\Big),
    \end{equation} 
    if follows from Lemma \ref{lem1}-{\rm (ii)} that $\frac{1}{\mathcal{R}(d_I)}>\frac{1}{|\Omega|}\int_{\Omega}\frac{\gamma}{\beta}$ for every $d_I>\mathcal{R}^{-1}\Big(\frac{|\Omega|}{\int_{\Omega}\frac{\gamma}{\beta}}\Big)$. In this case, it follows from Theorem \ref{T2-2}-{\rm (i)}  that $N_{\rm low}(d_I)<\frac{|\Omega|}{\mathcal{R}(d_I)}$ and $m_{d_I}^*<1$  for every $d_I>\mathcal{R}^{-1}\Big(\frac{|\Omega|}{\int_{\Omega}\frac{\gamma}{\beta}}\Big)$. Note for instance that \eqref{lem2-eq3}  holds when $\gamma=\beta^2$ and is not constant. Under hypothesis \eqref{lem2-eq3}, we see that there is a range of parameters satisfying $\mathcal{R}_0(N,d_I)<1$ such that \eqref{e1} has at least two EE solutions for small values of the diffusion rates of the susceptible population. This shows that the results of Proposition \ref{Proposition1} on the epidemic model \eqref{e1-prime} can not be extended in general to  the epidemic model \eqref{e1}. In particular, it is possible for the basic reproduction number of \eqref{e1} to be less than one and the disease to still persist.
\end{itemize} 
\end{rk}

Thanks to Theorems \ref{T0} $\&$ \ref{T2-2}, we have explicit hypotheses on the parameters of system \eqref{e1} that give  at least two EE solutions for sufficiently small $d_S$ when $\mathcal{R}_0(N,d_I)<1$. However, it is not clear whether we may have multiplicity of the EE solutions when $\mathcal{R}_0(N,d_I)\ge 1$. Similarly,   when $0<m_{d_I}^*<1$ and $0<d_S<d_I(1-m_{d_I}^*)$, it is not clear whether the positive number $N_{d_I,d_S}$ of Theorem \ref{T1} can be bigger than $|\Omega|/\mathcal{R}(d_I)$.   Our next result is concerned with some answers to these two questions and also provides more information on the structure of the EE solutions of \eqref{e1} if some further assumptions are satisfied. 


% {\color{blue} However, the later condition seems difficult to check due to the dependence of $\varphi_{d_I}$ on $d_I$.} Thanks to Theorems \ref{T0} and \ref{T1}, it is necessary to find out  whether ${N}_{\rm low}(d_I)<\frac{|\Omega|}{\mathcal{R}(d_I)}$ is equivalent to $m_{d_I}^*<1$. While we are not  yet able to prove this equivalence, at least  one implication holds as shown in  the following result.

\begin{tm}\label{T2} Fix $d_I>0$ and let $N_{\rm low}(d_I)$ and $m_{d_I}^*$ be given by Theorems \ref{T0} and \ref{T1}, respectively. Suppose that
\begin{equation}\label{T2-eq1}
\frac{1}{\mathcal{R}(d_I)}>\frac{1}{|\Omega|}\int_{\Omega}\frac{\gamma}{\beta}\quad \text{and}\quad  {|\Omega|\int_{\Omega}\beta\varphi_{d_I}^3}>{\Big(\int_{\Omega}\varphi_{d_I}\Big)\Big(\int_{\Omega}\beta\varphi_{d_I}^2\Big)} .
\end{equation}
\begin{itemize}
\item[\rm (i)]There is $N>{|\Omega|}/{\mathcal{R}(d_I)}$ and $ 0<\tilde{d}(N,d_I)<d_I(1-m_{d_I}^*)$  such that \eqref{e1}  has at least two EE solutions for $0<d_S<\tilde{d}(N,d_I)$. \item[\rm (ii)]
There is $0<d_S^*<d_I(1-m_{d_I}^*)$ such that for every $0<d_S<d_S^*$, there exist $N_{\rm low}(d_I)<N^{\rm inf }_{d_S,d_I}<\frac{|\Omega|}{\mathcal{R}(d_I)}<\tilde{N}^{\rm sup}_{d_I,d_S}\le \hat{N}^{\rm sup}_{d_I,d_S}$ such that  \eqref{e1} has: {\rm(ii-1)} no EE solution for $N<N^{\rm inf}_{d_I,d_S}$, {\rm(ii-2)}  at  least one EE solution for $N=N^{\rm inf}_{d_I,d_S}$, {\rm(ii-3)} at least two EE solutions for $N\in (N^{\rm inf}_{d_I,d_S}, \frac{|\Omega|}{\mathcal{R}(d_I)}\big]\cup\{\tilde{N}^{\rm sup}_{d_I,d_S}\}$, {\rm(ii-4)}  at least three EE solutions for $N\in (\frac{|\Omega|}{\mathcal{R}(d_I)},\tilde{N}^{\rm \sup}_{d_I,d_S})$, and exactly one EE solution for $N>\hat{N}^{\rm sup}_{d_I,d_S}$.   
\end{itemize}

    
\end{tm}


% Figure environment removed


\begin{rk}\label{RK2} 
    Suppose that    \eqref{T2-eq1} holds. Theorem \ref{T2}-{\rm (ii)} shows   that  for $d_S$ sufficiently small and fixed, taking $N$ as the bifurcation parameter, the bifurcation diagram has an S-shape and the total population size $N$  may be selected such that  $\mathcal{R}_0(N,d_I)>1$ and \eqref{e1} has at least three EE solutions.  On the other hand, by Theorem \ref{T2}-{\rm (i)}, the total population  size $N$ satisfying $\mathcal{R}_0(N,d_I)>1$ can be  fixed such that for sufficiently small $d_{S}$, \eqref{e1} has at least two EE solutions. Moreover, in this case as in Theorem \ref{T0}, system \eqref{e1} has a maximal and minimal EE solutions in the sense of \eqref{T0-eq2} and \eqref{T0-eq3} for sufficiently small values of $d_S$. Furthermore, as $d_S$ tends to zero, the minimal EE solutions satisfy the asymptotic profiles as in \eqref{T0-eq4}, while the maximal EE solutions satisfy the asymptotic profiles as in \eqref{T0-eq7}. This shows that the profiles of the EE solutions as $d_S$ tends to zero depends on  the subsequence chosen and that both  scenarios described in \cite[Theorem 2.3]{Wu_Zou2016} are possible. Proposition \ref{appen-prop3}-{\rm(i)} of the Appendix gives some examples of the parameters satisfying \eqref{T2-eq1}.

    


\end{rk}

 
For each $d_I>0$, set $l^*=1/\mathcal{R}(d_I)$ and recall that $\mathcal{R}_0(N,d_I)={N}/{(|\Omega|l^*)}.$ Hence $\mathcal{R}_0(N,d_I)=1$ if $N=|\Omega|l^*$, $\mathcal{R}_0(N,d_I)>1$ if $N>|\Omega|l^*$, and $\mathcal{R}_0(N,d_I)<1$ if $N<|\Omega|l^*$. So,  $N=|\Omega|l^*$ is a critical number for the total population size. We provide a brief overview of the ideas of the proofs of our main results in the next remark.

\begin{rk}\label{RK1}   Our  approach  to establish the above results is to treat $N$ as a bifurcation parameter and  relate the question of the existence of the EE solution of the diffusive epidemic model \eqref{e1} to that of the level sets of some appropriate function. Indeed,  for each $d_I>0$ and $d_S>0$, there is a smooth function $\mathcal{N}_{d_I,d_S}\ : [l^*,\infty)\to (0,\infty)$ such that the diffusive epidemic model \eqref{e1} has an EE solution if and only if $\mathcal{N}_{d_I,d_S}(l)=N$ for some $l>l^*$ (see Lemma \ref{lem3}).  Furthermore, $\mathcal{N}_{d_I,d_S}$ can be written as $\mathcal{N}_{d_I,d_S}(l)=\mathcal{N}_{d_I}(l)+d_Sl\tilde{\mathcal{N}
}_{d_I}(l)$ for every $l\ge l^*$, where $\tilde{\mathcal{N}}_{d_I}$ is a smooth, strictly increasing, positive function on $(l^*,\infty)$, and $\tilde{\mathcal{N}}_{d_I}(l^*)=0$. Furthermore, when $\mathcal{N}_{d_I,d_I}(l)=N$ for some $l>l^*$, then $d_Sl\tilde{N}_{d_I}(l)$ is the total size of the infected population of the corresponding EE solution.  Therefore, the positive number $N_{\rm low}(d_I)$ in Theorem \ref{T0} is given by $\inf_{l\ge l^*}\mathcal{N}_{d_I}(l)$, and  the level sets $\mathcal{S}_{d_I,d_S}^N:=\{ l\in [l^*,\infty) : N=\mathcal{N}_{d_I,d_S}(l)$\}, $N>0$, provides the bifurcation diagram for the EE  solutions of the diffusive epidemic model \eqref{e1}. Hence, an S-shape in the graph of the function $\mathcal{N}_{d_I,d_S}$ results in an S-shape for the bifurcation curve of the EE solutions of the diffusive epidemic model \eqref{e1}. Due to the fact that $\tilde{\mathcal{N}}_{d_I}$ is strictly increasing, we can show that there is a minimal diffusion rate $d_I(1-m_{d_I}^*)$, where $0<m_{d_I}^*\le 1$ is given as in Theorem \ref{T1}, such that $\mathcal{N}_{d_I,d_S}$ is strictly increasing whenever $d_S>d_I(1-m_{d_I}^*)$. However, when $m_{d_I}^*<1$,  $\mathcal{N}_{d_I,d_S}$ has local extremum values for every $0<d_I<d_I(1-m_{d_I}^*)$, yielding to the multiplicity of the EE solution.

\quad Observing that $0<m_{d_I}^*<1$ if and only if $\mathcal{N}_{d_I}$ has local extremum values, another  important ingredient in our arguments is to obtain precise information on $\mathcal{N}_{d_I}$ near $l^*$. This requires us to find explicit formula for the right-hand derivative of  $\mathcal{N}_{d_I}$ at $l^*$, which turns out to be the quantity $ |\Omega|-(\int_{\Omega}\varphi_{d_I})(\int_{\Omega}\beta\varphi_{d_I}^2)/(\int_{\Omega}\beta\varphi_{d_I}^3)$.  Moreover, it holds that $\mathcal{N}_{d_I}(l^*)=|\Omega|l^*$ and $\mathcal{N}_{d_I,d_S}(l)\to \infty$ as $l\to \infty$, for every $d_I>0$ and $d_S>0$.  Hence, if $ |\Omega|-(\int_{\Omega}\varphi_{d_I})(\int_{\Omega}\beta\varphi_{d_I}^2)/(\int_{\Omega}\beta\varphi_{d_I}^3)>0$ and  $N_{\rm low}(d_I)<\mathcal{N}_{d_I}(l^*)$ for some $d_I>0$, then the bifurcation curve of the EE solutions of the diffusive epidemic model \eqref{e1} has an S-shape for $0<d_S\ll 1$ as given in Theorem \ref{T2}.   Note in this case that we have a forward bifurcation at $N=|\Omega|l^*$. However, if  $ |\Omega|-(\int_{\Omega}\varphi_{d_I})(\int_{\Omega}\beta\varphi_{d_I}^2)/(\int_{\Omega}\beta\varphi_{d_I}^3)<0$ we have a backward bifurcation at  $N=|\Omega|l^*$ for $0<d_S\ll 1$ as given by Theorem \ref{T2-2}.
\end{rk}

% % Figure environment removed



\subsection*{Discussion.} We examined the question of multiplicity and uniqueness of   the EE solutions of a diffusive epidemic model with the mass-action transmission mechanism, system \eqref{e1}, and obtained some interesting  results. In particular, our results revealed some new phenomona, which cannot be observed from  neither the ODE-model \eqref{model-001} nor from the corresponding PDE-model with the frequency-dependent transmission mechanism \eqref{e1-prime}. 

\quad As mentioned in the introduction, it is known that for  ODE-SIS epidemic model \eqref{model-001}, with simple nonlinearity (i.e.,   frequency-dependent/mass-action transmission mechanism), the basic reproduction number $\mathcal{R}_0^{\rm 0DE}$ is enough to completely characterize the dynamics of the disease: if $\mathcal{R}_0^{\rm  ODE}\le 1$ then the disease will be eventually eradicated, while if $\mathcal{R}_0^{\rm  ODE}> 1$, then the disease is endemic and there is a unique EE solution.  These results also hold true  for the diffusive epidemic model \eqref{e1-prime} with  the frequency-dependent transmission mechanism.     Surprisingly, it turns out that by switching from the frequency-dependent to the mass-action transmission mechanism in the diffusive epidemic model, these conclusions don't hold anymore.   

\quad Indeed,  Theorem \ref{T0} indicates that, for the dynamics of solutions of the diffusive epidemic model \eqref{e1}, the disease may still persist even if its basic reproduction number $\mathcal{R}_0(N,d_I)<1$. Precisely, there is some critical number given by $\mathcal{R}_0(N_{\rm low}(d_I),d_I)\le 1$, uniquely determined by $d_I$, such that \eqref{e1} has no EE if $\mathcal{R}_0(N,d_I)\le \mathcal{R}_0(N_{\rm low}(d_I),d_I)$ for any diffusion rate $d_S$ of the susceptible population. However, thanks to Theorem \ref{T0}-{\rm (iii)}, if $\mathcal{R}_0(N_{\rm low}(d_I),d_I)<\mathcal{R}_0(N,d_I)<1$, then there exist at least two EE solutions for sufficiently small values of $d_S$. In Theorem \ref{T2} and Remark \ref{RK3}, we identified two main factors that lead to $\mathcal{R}_0(N_{\rm low}(d_I),d_I)<1 $. First, it follows from \eqref{lem2-eq3} that if the average of the ratio of the recovery rate over the transmission rate is smaller than the ratio of the average of the recovery rate over the average of the transmission rate, then $\mathcal{R}_0(N_{\rm low}(d_I),d_I)<1$ for sufficiently large values of $d_I$. Second, by  Theorem \ref{T2-2}-{\rm (ii)}, if  the  average of $\beta\varphi_{d_I}^3$ is smaller than the product of the  averages of $\varphi_{d_I}$ and $\beta\varphi_{d_I}^2$, again $\mathcal{R}_0(N_{\rm low}(d_I),d_I)<1$. In these two scenarios,  we see that the basic reproduction number is not enough to predict the dynamics of the disease.

\quad Next, when $\mathcal{R}_0(N,d_I)>1$, Theorem \ref{T2} shows that system \eqref{e1} may have up to three EE solutions for small $d_S$, which is something really unexpected.  Nonetheless, when the susceptible population moves slightly slower or faster than the infected population, our Theorem \ref{T1} show that the global structure of the EE solutions of system \ref{e1} is very simple: a unique EE exists if and only if  $\mathcal{R}_0(N,d_I)>1$.

\quad In view of the above, we see that lowering diffusion rate of the susceptible individuals  may be an effective disease control strategy if one also avoid the accumulation of the population. Moreover, our results suggest that the prediction on a disease dynamics may strongly dependent on the type of transmission mechanism used in the mathematical model.


%

\section{Preliminaries}\label{Preliminaries}
\quad We present a few preliminary results here. 



\begin{lem}\label{lem1} 
\begin{itemize}
    \item[\rm (i)] If $\frac{\beta}{\gamma}$ is constant, then $\mathcal{R}(d_I)=\frac{\beta}{\gamma}$ for all $d_I>0$. \item[\rm (ii)] If  $\frac{\beta}{\gamma}$ is not constant, then $\mathcal{R}(d_I)$ is strictly decreasing in $d_I$. Moreover,
    \begin{equation}\label{R-star-limit-eq}
        \lim_{d_I\to 0^+}\mathcal{R}(d_I)=\max_{x\in\overline{\Omega}}\frac{\beta(x)}{\gamma(x)}\quad \text{and}\quad \lim_{d_I\to\infty}\mathcal{R}(d_I)=\frac{\int_{\Omega}\beta}{\int_{\Omega}\gamma}.
    \end{equation} 
    In particular, if \eqref{lem2-eq3} holds, then
      $   \frac{|\Omega|}{\mathcal{R}(d_I)}<\int_{\Omega}\frac{\gamma}{\beta}$  {if} $ 0<d_I<\mathcal{R}^{-1}\Big(\frac{|\Omega|}{\int_{\Omega}\frac{\gamma}{\beta}}\Big)$, $ 
        \frac{|\Omega|}{\mathcal{R}(d_I)}=\int_{\Omega}\frac{\gamma}{\beta}$ if $  d_I=\mathcal{R}^{-1}\Big(\frac{|\Omega|}{\int_{\Omega}\frac{\gamma}{\beta}}\Big)$, and $ 
        \frac{|\Omega|}{\mathcal{R}(d_I)}>\int_{\Omega}\frac{\gamma}{\beta}$  if $d_I>\mathcal{R}^{-1}\Big(\frac{|\Omega|}{\int_{\Omega}\frac{\gamma}{\beta}}\Big)$ .
    
\end{itemize}
\end{lem}


\quad Consider the one-parameter family of elliptic equations
\begin{equation}\label{Eq1}
    \begin{cases}
    0=d_I\Delta u +(l\beta(1-d_Iu)-\gamma)u & x\in\Omega,\cr 
    0=\partial_{\vec{n}}u & x\in\partial\Omega
    \end{cases}
\end{equation}
Note \eqref{Eq1} is a one parameter family of the classical diffusive logistic elliptic equation. Hence, several interesting results have been established as with respect to the existence, uniqueness of stability of its positive solution whenever it exists. 


{\quad Throughout the rest of the paper, we set $l^*:=\frac{1}{\mathcal{R}(d_I)}$. For each $q\in[2,\infty)$, consider the linear  elliptic operator  on $L^{q}(\Omega)$ (resp. on $C(\overline{\Omega})$), defined by
$$
\mathcal{L}^*w=d_I\Delta w+(l^*\beta-\gamma)w \quad w\in {\rm Dom}_{q}, (\text{resp.}\ w\in {\rm Dom}_{\infty} )\quad , 
$$ 
where ${\rm Dom}_{q}=\{u\in W^{2,q}(\Omega) | \partial_{\vec{n}}u=0\ \text{on}\ \partial\Omega\}$ and  $
{\rm Dom}_{\infty}=\left\{u\in\cap_{q\ge 1}{\rm Dom}_{q}| \ \mathcal{L}^* u\in C(\overline\Omega)\right\}.$  By \cite[Theorem 1, page 357]{Evans}, since $\mathcal{L}^*$ is symmetric on $L^2(\Omega)$, then ${L}^2(\Omega)$ has an orthonormal basis formed of eigenfunctions of $\mathcal{L}^*$. Moreover,  since by \eqref{R-star-pde} equation $\varphi_{d_I}$ is an eigenvector of $\mathcal{L}^*$ associated with its principal eigenvalue zero, we can decompose $L^2(\Omega)$ as $L^2(\Omega)={\rm span}(\varphi_{d_I})\oplus\{w\in L^2(\Omega)| \int_{\Omega}w\varphi_{d_I}=0\}$.  Thus,   
$$
C(\overline{\Omega})={\rm span}(\varphi_{d_I})\oplus\mathcal{Z}_{\infty}\quad \text{and}\quad L^q(\Omega)={\rm span}(\varphi_{d_I})\oplus \mathcal{Z}_q,\quad q\ge 2, 
$$
where $\mathcal{Z}_{\infty}=\{g\in C(\overline{\Omega})|\int_{\Omega}g\varphi_{d_I}=0 \}$ and $\mathcal{Z}_q=\{w\in L^q(\Omega)| \int_{\Omega}w\varphi_{d_I}=0\}$ for $q\ge 2$. Furthermore, we have that $\mathcal{L}^*_{|\mathcal{Z}_q\cap{\rm Dom}_{q}}\ :\ \mathcal{Z}_q\cap{\rm Dom}_{q} \to \mathcal{Z}_q$ is invertible for each $q\in[2,\infty]$.

\quad Given $f\in C(\overline{\Omega})$, we set $f_{\min}:=\min_{x\in\overline{\Omega}}f(x)$ and $f_{\max}:=\max_{x\in\overline{\Omega}}f(x)$. }  The next lemma collects some results on the positive solution of \eqref{Eq1}.

\begin{lem}\label{lem2} Fix $d_I>0$ and let $\mathcal{R}(d_I)$ be given by \eqref{R-star-eq}.
\begin{itemize}
    \item[\rm (i)] The elliptic equation \eqref{Eq1} has a (unique) positive solution, $u^l$,  if and only if $l>l^*$. Moreover, 
    \begin{equation}\label{lem2-eq1}
        0<u^l<\frac{1}{d_I},\quad \lim_{l\to l^*}\|u^l\|_{\infty}=0,\quad \lim_{l\to\infty}\|u^l-\frac{1}{d_I}\|_{\infty}=0,\quad  \lim_{l\to\infty}\Big\|l(1-d_Iu^l)-\frac{\gamma}{\beta}\Big\|_{\infty}=0,
    \end{equation}
    {and \begin{equation}\label{rev-0}
      \lim_{l\to l^*}\Big\|\frac{u^l}{l-l^*}-\Big(\frac{\mathcal{R}(d_I)\int_{\Omega}\beta\varphi_{d_I}^2}{d_I\int_{\Omega}\beta\varphi_{d_I}^3}\Big)\varphi_{d_I}\Big\|_{C^1(\overline{\Omega})}=0.
  \end{equation} }

    
    \item[\rm (ii)] The mapping $\big(l^*,\infty\big)\ni l\mapsto u^l\in C^1(\overline{\Omega})$ is smooth and strictly increasing. Setting $v^l=\partial_l u^l$ for every $l>l^*$, then $v^l$ satisfies
    \begin{equation}\label{v-l-eq1}
        \begin{cases}
        0=d_I\Delta v^l+(l\beta(1-2d_Iu^l)-\gamma)v^l+\beta(1-d_Iu^l)u^l & x\in\Omega,\cr 
        0=\partial_{\vec{n}}v^l & x\in\partial\Omega,
        \end{cases}
    \end{equation}
    \begin{equation}\label{v-l-eq2}
       \lim_{l\to l^*}\left\|v^l-\left(\frac{\mathcal{R}(d_I)\int_{\Omega}\beta\varphi_{d_I}^2}{d_I\int_{\Omega}\beta\varphi_{d_I}^3}\right)\varphi_{d_I}\right\|_{C^1(\overline{\Omega})}=0,\quad \text{and}\quad \lim_{l\to\infty}\Big\|l^2v^l-\frac{\gamma}{d_I\beta}\Big\|_{\infty}=0.
    \end{equation}

    
    
    \item[\rm (iii)] The function $\mathcal{N}_{d_I}$ defined by 
    \begin{equation}\label{N-d_I-def}
        \mathcal{N}_{d_I}(l)=l\int_{\Omega}(1-d_Iu^l)\quad \forall\ l>l^*,
    \end{equation}
    is continuously differentiable, with \begin{equation}\label{lem2-eq2}
        \lim_{l\to l^*}\mathcal{N}_{d_I}(l)=l^*|\Omega|\quad \text{and}\quad \lim_{l\to\infty}\mathcal{N}_{d_I}(l)=\int_{\Omega}\frac{\gamma}{\beta}.
    \end{equation} 
\end{itemize}

\end{lem}
\begin{proof}{\rm (i)} It follows from standard results on the diffusive logistic equations. See for example \cite[Lemma 4.2-(i)]{CastellanoSalako2021}. { Note also that \eqref{lem2-eq1} is proved in \cite[Lemma 4.2-(i)]{CastellanoSalako2021}}. So, we shall show that \eqref{rev-0} holds.   To this end, we first write $u^l$  as
\begin{equation}\label{rev-1}
    u^{l}=(l-l^*)\psi^l\quad \forall\ l>l^*,
\end{equation}
and then find the limit of $\psi^l$ as $l$ approaches $l^*$. To achieve this,   we set 
\begin{equation}\label{rev-8}
c(l):=\int_{\Omega}\varphi_{d_I}\psi^{l}\quad \forall\ l>l^*
\end{equation}
and 
$$
\tilde{\psi}^l:=\psi^l-c(l)\varphi_{d_I}\quad \forall\ l> l^*.
$$
Observing that $\int_{\Omega}\tilde{\psi}^l\varphi_{d_I}=0$ since $\int_{\Omega}\varphi_{d_I}^2=1$, we have that 
\begin{equation}\label{rev-2}
  \quad\tilde{\psi}^l\in\mathcal{Z}_{\infty}\quad \text{and}\quad  u^l=(l-l^*)(c(l)\varphi_{d_I}+\tilde{\psi}^l)\quad \forall\ l>l^*.
\end{equation}
Thanks to \eqref{rev-2}, our  aim is to show that $c(l)\to \mathcal{R}(d_I)\int_{\Omega}\beta\varphi_{d_I}\varphi_{d_I}^2/(d_I\int_{\Omega}\beta\varphi_{d_I}^3)$  and  $\hat{\psi}^l :=\frac{\tilde{\psi}^l}{l-l^*}\to \hat{\psi}^*$ as $l\to l^*$  where $\hat{\psi}^*\in C^{2}(\overline{\Omega})$. From this point, we proceed in two steps.\\
{\bf Step 1.} In this step, we show that there is a positive constant $K_{d_I}>0$ such that
\begin{equation}\label{rev-6}
    \frac{|\Omega|\beta_{\min}\varphi_{d_I,\min}}{ld_I\|\beta\|_{\infty}K_{d_I}}\le c(l)\le \frac{|\Omega|\|\beta\|_{\infty}\|\varphi_{d_I}\|_{\infty}}{ld_I\beta_{\min}}
    \quad \forall\ l>l^*.
\end{equation}
Multiply \eqref{R-star-pde} and \eqref{Eq1} by $u^l$ and $\varphi_{d_I}$ respectively. After integrating the resulting equations and then taking their difference, we obtain 
$$
0=\int_{\Omega}(l\beta -ld_I\beta u^l -\gamma)u^l\varphi_{d_I}-\int_{\Omega}(l^*\beta-\gamma)u^{l}\varphi_{d_I}.
$$
This gives 
\begin{equation}\label{rev-10}
    (l-l^*)\int_{\Omega}\beta u^l\varphi_{d_I}=d_Il\int_{\Omega}\beta (u^l)^2\varphi_{d_I}\quad \forall\ l>l^*.
\end{equation}
 Using the H\"older's inequality and \eqref{rev-10} we obtain that 
\begin{align*}
    (l-l^*)\|\beta\|_{\infty}\int_{\Omega}u^l\varphi_{d_I}\ge  (l-l^*)\int_{\Omega}\beta u^l\varphi_{d_I}
    =&d_Il\int_{\Omega}\Big(u^l\varphi_{d_I}\Big)^2\frac{\beta}{\varphi_{d_I}}\cr 
    \ge& ld_{I}\Big(\frac{\beta}{\varphi_{d_I}}\Big)_{\min}\int_{\Omega}\Big(u^{l}\varphi_{d_I}\Big)^2\cr 
    \ge& \frac{ld_I\beta_{\min}}{|\Omega|\|\varphi_{d_I}\|_{\infty}}\Big(\int_{\Omega}u^l\varphi_{d_I}\Big)^2\quad \forall\ l>l^*.
\end{align*}
As a result,
\begin{equation}\label{rev-7}
\frac{|\Omega|\|\beta\|_{\infty}\|\varphi_{d_I}\|_{\infty}}{ld_I\beta_{\min}}\ge\frac{1}{l-l^*}\int_{\Omega}u^l\varphi_{d_I}\quad \forall\ l>l^*.
\end{equation}
On the other hand, it follows from \eqref{rev-1} and \eqref{rev-8} that 
\begin{equation}\label{rev-11}
\int_{\Omega}u^{l}\varphi_{d_I}=(l-l^*)\int_{\Omega}\psi^l\varphi_{d_I}=c(l)(l-l^*) \quad \forall\ l>l^*.
\end{equation}
By \eqref{rev-11} and \eqref{rev-7}, we have that
\begin{equation}\label{rev-9}
    c(l)\le \frac{|\Omega|\|\beta\|_{\infty}\|\varphi_{d_I}\|_{\infty}}{ld_I\beta_{\min}}\quad \forall\ l>l^*.
\end{equation}
Next, since $\|l\beta(1-d_Iu^l)-\gamma\|_{\infty}\to 0$ as $l\to\infty$ and $\|l\beta(1-d_Iu^l)-\gamma\|_{\infty}\to \|l^*\beta-\gamma\|_{\infty}$ as $l\to l^*$, then 
$$
\sup_{l>l^*}\|l\beta(1-d_Iu^l)-\gamma\|_{\infty}<\infty.
$$
It then follows from the Harnack's inequality for elliptic equations and the fact that $u^l$ solves \eqref{Eq1} that there is a positive constant $K_{d_I}$ independent of $l>l^*$ such that 
\begin{equation*}
    \|u^l\|_{\infty}\le K_{d_I}u^l_{\min}\quad \forall\ l>l^*.
\end{equation*}
This together with \eqref{rev-10} yields that 
\begin{align*}
    (l-l^*)\beta_{\min}\int_{\Omega}u^l\varphi_{d_I}\le  (l-l^*)\int_{\Omega}\beta u^l\varphi_{d_I} =&d_Il\int_{\Omega}\beta(u^l)^2\varphi_{d_I}\cr 
    \le&d_IlK_{d_I}\|\beta\|_{\infty}u^l_{\min}\int_{\Omega}u^l\varphi_{d_I}\cr 
    = &\frac{d_IlK_{d_I}\|\beta\|_{\infty}}{\varphi_{d_I,\min}}(u^l_{\min}\varphi_{d_I,\min})\Big(\int_{\Omega}u^l\varphi_{d_I}\Big)\cr 
    \le &\frac{ld_IK_{d_I}\|\beta\|_{\infty}}{|\Omega|\varphi_{d_I,\min}}\Big(\int_{\Omega}u^l\varphi_{d_I}\Big)^2\quad \forall\ l>l^*.
\end{align*}
Hence, by \eqref{rev-11},
\begin{equation}\label{rev-12}
\frac{|\Omega|\beta_{\min}\varphi_{d_I,\min}}{ld_IK_{d_I}\|\beta\|_{\infty}}\le\frac{1}{l-l^*}\int_{\Omega}u^l\varphi_{d_I}=c(l)\quad \forall\ l>l^*.
\end{equation}
It is clear from \eqref{rev-9} and \eqref{rev-11} that \eqref{rev-6} holds.\\
{\bf Step 2.} In this step, we show that 
\begin{equation}
    \lim_{l\to l^*}c(l)=\frac{\int_{\Omega}\beta\varphi_{d_I}^2}{l^*d_I\int_{\Omega}\beta\varphi^3_{d_I}}.
\end{equation}

Note from \eqref{rev-2} and the fact that $u^l$ solves \eqref{Eq1} that 
\begin{equation}\label{rev-3}
    \begin{cases}
    0=d_I\Delta(c(l)\varphi_{d_I}+\tilde{\psi}^l) +(l\beta(1-d_Iu)-\gamma)(c(l)\varphi_{d_I}+\tilde{\psi}^l) & x\in\Omega,\cr 
    0=\partial_{\vec{n}}\tilde{\psi}^l & x\in\partial\Omega.
    \end{cases}
\end{equation}
Moreover, since $\varphi_{d_I}$ satisfies \eqref{R-star-pde}, we get from \eqref{rev-3} that 
\begin{equation}\label{rev-4}
    \begin{cases}
    0=d_I\Delta\tilde{\psi}^l+(l-l^*)c(l)\beta( 1-ld_I(c(l)\varphi_{d_I}+\tilde{\psi}_{d_I})))\varphi_{d_I} +(l\beta(1-d_Iu)-\gamma)\tilde{\psi}^l & x\in\Omega,\cr 
    0=\partial_{\vec{n}}\tilde{\psi}^l & x\in\partial\Omega.
    \end{cases}
\end{equation}
Next, setting $\hat{\psi}^l:=\frac{\tilde{\psi}^l}{l-l^*}$ for every $l>l^*$, we deduce from \eqref{rev-4} that 
\begin{equation}\label{rev-5}
    \begin{cases}    0=\mathcal{L}^*\hat{\psi}^l+c(l)\beta( 1-ld_I(c(l)\varphi_{d_I}+(l-l^*)\hat{\psi}_{d_I})))\varphi_{d_I} +\beta(l(1-d_Iu)-l^*)\hat{\psi}^l & x\in\Omega,\cr 
    0=\partial_{\vec{n}}\hat{\psi}^l & x\in\partial\Omega,
    \end{cases}
\end{equation}
which gives that 
$$
\hat{\psi}^l=\mathcal{L}^{*,-1}_{|\mathcal{Z}_q\cap{\rm Dom}_q}\Big(c(l)\beta( 1-ld_I(c(l)\varphi_{d_I}+(l-l^*)\hat{\psi}_{d_I})))\varphi_{d_I} +\beta(l(1-d_Iu)-l^*)\hat{\psi}^l \Big)\quad \forall\ q\ge 2,\  l>l^*.
$$
 Setting $M_q:=\big\|\mathcal{L}^{*,-1}_{|\mathcal{Z}_q\cap{\rm Dom}_q}\big\|$, this implies that,  
\begin{align}\label{YTYT1}
    \|\hat{\psi}^l\|_{W^{2,q}(\Omega)} 
    \le & M_q\|c(l)\beta( 1-ld_I(c(l)\varphi_{d_I}+(l-l^*)\hat{\psi}_{d_I})))\varphi_{d_I} +\beta(l(1-d_Iu)-l^*)\hat{\psi}^l\|_{L^q(\Omega)}\cr 
    \le & M_q\|\beta\|_{\infty}\Big((l-l^*)(ld_Ic(l)\|\varphi_{d_I}\|_{\infty}+1)+ld_I\| u^l\|_{\infty}\Big)\|\hat{\psi}^l\|_{L^q(\Omega)}\cr
    &+ M_qc(l)\|\beta(1-ld_Ic(l)\varphi_{d_I})\|_{\infty}|\Omega|^{\frac{1}{q}}\cr
    \le & M_q\|\beta\|_{\infty}\Big((l-l^*)(ld_Ic(l)\|\varphi_{d_I}\|_{\infty}+1)+ld_I\| u^l\|_{\infty}\Big)\|\hat{\psi}^l\|_{W^{2,q}(\Omega)}\cr
    &+ M_qc(l)\|\beta(1-ld_Ic(l)\varphi_{d_I})\|_{\infty}|\Omega|^{\frac{1}{q}}.
\end{align}
Observing from \eqref{rev-6}
 and the fact that $u^l\to 0$ as $l\to l^*$ that, for each $q\ge 1$, 
 $$ M_q\|\beta\|_{\infty}\Big((l-l^*)(ld_Ic(l)\|\varphi_{d_I}\|_{\infty}+1)+ld_I\| u^l\|_{\infty}\Big)\|\to 0\quad \text{as}\ l\to l^*,
 $$
 then, for each $q\ge 2$, there is $ K^*_{q}>0$ such that 
 \begin{equation}\label{b-rev-1}
     \|\hat{\psi}^l\|_{W^{2,q}(\Omega)}\le K^*_q\quad \forall\ 0<l^*<l<l^*+1.
 \end{equation}
 Choosing $q\gg n$ such that $W^{2,q}(\Omega)$ is compactly embedded in $C^1(\overline{\Omega})$, it follows from \eqref{rev-6}, \eqref{b-rev-1}, and the fact that $u^l\to 0$ as $l\to l^*$ (possibly after passing to a subsequence)  that $c(l)\to c^*$  and $\hat{\psi}^l\to \hat{\psi}^*$ in $C^{1}(\overline{\Omega})$ as $l\to l^*$, where  $c^*$ is a positive number and $ \hat{\psi}^*\in C^2(\Omega)\cap\mathcal{Z}_{\infty}$  satisfies 
\begin{equation}\label{rev-13}
    \begin{cases}
        -\mathcal{L}^*\hat{\psi}^*=c^*\beta(1-l^*d_Ic^*\varphi_{d_I})\varphi_{d_I} & x\in\Omega,\cr 
        0=\partial_{\vec{n}}\hat{\psi}^* & x\in\partial\Omega.
    \end{cases}
\end{equation}
Hence, since $\mathcal{L}^*({\rm Dom}_{\infty}\cap \mathcal{Z}_{\infty})=\mathcal{Z}_{\infty}\subset\mathcal{Z}_2$,  multiplying \eqref{rev-13} by $\varphi_{d_I}$ and integrating the resulting equation on $\Omega$, we obtain that 
\begin{align*}
    0=c^*\int_{\Omega}\beta(1-l^*v^*d_I\varphi_{d_I})\varphi_{d_I}^2,
\end{align*}
which due to the fact that $c^*>0$ gives
$$
c^*=\frac{\int_{\Omega}\beta\varphi_{d_I}^2}{l^*d_I\int_{\Omega}\beta\varphi_{d_I}^3}.
$$
Since $c^*$ is independent of the chosen subsequence, we then conclude that $c(l)\to c^*$ as $l\to l^*$. Moreover, since $\mathcal{L}^*$ is invertible on $\mathcal{Z}_{\infty}\cap {\rm Dom}_{\infty}$, we have that $\hat{\psi}^*$ is the unique solution of \eqref{rev-13} in $\mathcal{Z}_{\infty}\cap{\rm Dom}_{\infty}$ and  $\hat{\psi}^l\to \hat{\psi}^* $ in $C^1(\overline{\Omega})$ as $l\to l^*$. Therefore, we have 
$$
\frac{u^l}{l-l^*}=\psi^l=c(l)\varphi_{d_I}+(l-l^*)\hat{\psi}^l\to \left(\frac{\int_{\Omega}\beta\varphi_{d_I}^2}{l^*d_I\int_{\Omega}\beta\varphi_{d_I}^3}\right)\varphi_{d_I}\quad \text{as}\ l\to l^* \ \text{ in }\ C^1(\overline{\Omega}).
$$

\quad {\rm (ii)} The regularity of $u^l$ with respect to $l$ and the fact that $v^l>0$ and solves \eqref{v-l-eq1} is already proved in \cite[Lemma 4.2-(i)]{CastellanoSalako2021}.  Next, we prove that \eqref{v-l-eq2} holds.  For each $l>l^*$, let $\psi^l=\frac{u^l}{l-l^*}=c(l)\varphi_{d_I}+\tilde{\psi}^l$ be defined as in \eqref{rev-1}. Thanks to \eqref{rev-0}, for any positive number $A>0$, 
$$
1-d_Iu^l-lA\psi^l\to 1-Ac^*\varphi_{d_I}\quad \text{as}\quad l\to l^*\ \ \text{uniformly in}\ \Omega,
$$
where $c^*=\mathcal{R}(d_I)\int_{\Omega}\beta\varphi_{d_I}^2/(d_I\int_{\Omega}\beta\varphi_{d_I}^3)>0$. Therefore, we can choose $0<A_1<A_2$ such that 
\begin{equation*}
  1-d_Iu^l-lA_2\psi^l<0<1 -d_Iu^l-lA_1\psi^l\quad l^*<l<l^*+\varepsilon_0 
\end{equation*}
for some $\varepsilon_0>0$.  Hence, thanks to \eqref{rev-3}, we have that 
\begin{equation*}
    d_I\Delta (A_1\psi^l)+(l\beta(1-2d_Iu^l)-\gamma)(A_1\psi^l)+\beta(1-d_Iu^l)u^l= \beta(1-d_Iu^l-lA_1\psi^l)u^l>0 \quad x\in\Omega
\end{equation*}
and 
\begin{equation*}
    d_I\Delta (A_2\psi^l)+(l\beta(1-2d_Iu^l)-\gamma)(A_2\psi^l)+\beta(1-d_Iu^l)u^l= \beta(1-d_Iu^l-lA_2\psi^l)u^l<0 \quad x\in\Omega
\end{equation*}
for every $l^*<l<l^*+\varepsilon_0$. It then follows from \eqref{v-l-eq1} and the comparison principle for elliptic equations that 
\begin{equation*}
    A_1\psi^l<v^l<A_2\psi^l\quad 0<l^*<l<l^*+\varepsilon_0.
\end{equation*}
Therefore, by \eqref{v-l-eq1} and the estimates for elliptic equations (possibly after passing to a subsequence), there is a strictly positive function $v^*\in C^2(\Omega)$ and $v^l\to v^*$ as $l\to l^*$ in $C^1(\overline{\Omega})$. Moreover, $v^*$ satisfies
$$
\begin{cases}
    0=d_I\Delta v^*+(l^*\beta-\gamma)v^* & x\in\Omega\cr
    0=\partial_{\vec{n}}v^* & x\in\partial\Omega.
\end{cases}
$$
So, we must have that $v^*=c^{**}\varphi_{d_I}$ for some positive number $c^{**}$. Next,  we multiple \eqref{rev-3} and \eqref{v-l-eq1} by $v^l$ and $\psi^l$ respectively, and then integrate on $\Omega$. After taking the difference the resulting equations, we obtain that 
$$
0=\int_{\Omega}\beta(1-d_Iu^l-ld_Iv^l)u^l\psi^l\quad \forall\ l>l^*,
$$
which is equivalent to  
$$0=\int_{\Omega}\beta(1-d_Iu^l-ld_Iv^l)(\psi^l)^2\quad \forall\ l>l^*,$$
since $u^l=(l-l^*)\psi^l$. Letting $l\to l^*$ in the last equation yields 
$$
0=\int_{\Omega}\beta(1-d_Il^*c^{**}\varphi_{d_I})(c^*\varphi_{d_I})^2.
$$
Solving for $c^{**}$, we get  $ 
c^{**}=\frac{\int_{\Omega}\beta\varphi_{d_I}^2}{l^*d_I\int_{\Omega}\beta\varphi_{d_I}^3}=c^*
$  is independent of the chosen subsequence. Therefore,  $v^l\to c^*\varphi_{d_I}$ as $l\to l^*$ in $C^1(\overline{\Omega})$.
 
\quad Finally, set $p^l=l^2v^{l}$ for each $l>l^*$. By direct computations, it follows from \eqref{v-l-eq1} that $p^l$ satisfies
\begin{equation}\label{D1}
    \begin{cases}
        0=\frac{d_I}{l}\Delta p^l +\frac{\beta}{l}\big( z^l-\frac{\gamma}{\beta}\big)p^l+u^l\beta(z^l-d_Ip^l) & x\in\Omega,\cr 
        0=\partial_{\vec{n}}p^l & x\in\partial\Omega.
    \end{cases}
\end{equation}
where $z^l=l(1-d_Iu^l)$ for all $l>l^*$. Therefore, since $u^{l}\to \frac{1}{d_I}$ and $z^l\to\frac{\gamma}{\beta}$ as $l\to\infty$ in $C(\overline{\Omega})$ by \eqref{lem2-eq1}, we can employ the singular perturbation theory for elliptic equations to deduce from \eqref{D1} that $p^l\to \frac{\gamma}{d_I\beta}$ as $l\to\infty$ uniformly on $\overline{\Omega}$, which completes the proof of  \eqref{v-l-eq2}.

\quad {\rm (iii)} It follows from \cite[Lemma 4.2-(iii)]{CastellanoSalako2021}
\end{proof}

The next result shows the relationship between EE solutions of \eqref{e1} and the solutions of \eqref{Eq1}

\begin{lem}\label{lem3} Let $d_S>0$, $d_I>0$ and $N>0$.
\begin{itemize}
    \item[\rm (i)] If $(S,I)$ is an EE solution of \eqref{e1} then 
    \begin{equation}\label{kappa-def}
        \kappa =d_SS+d_II
    \end{equation} 
    is a constant function. Furthermore, setting 
    \begin{equation}\label{tilde-s-i-ded}
        \tilde{S}=\frac{S}{\kappa}\quad \text{and}\quad \tilde{I}=\frac{I}{\kappa},
    \end{equation}
    then $\tilde{I}$ is the positive solution of \eqref{Eq1} with $l=\frac{\kappa}{d_S}>l^*$.
    \item[\rm (ii)] If $l>l^*$ and  \begin{equation}\label{N-equation}
        N=l\left[\int_{\Omega}(1-d_Iu^l)+d_S\int_{\Omega}u^l\right]=\mathcal{N}_{d_I}(l)+d_Sl\int_{\Omega}u^l,
    \end{equation}
    where $\mathcal{N}_{d_I}(l)$ is defined as in \eqref{N-d_I-def}, then $(S,I):=(l(1-d_Iu^l),d_Slu^l)$ is an EE solution of \eqref{e1}.
\end{itemize}
\end{lem}
\begin{proof} The proof is similar to that of \cite[Lemma 4.4]{CastellanoSalako2021}, hence it is omitted.
    
\end{proof}

\section{Proofs of Main Results}\label{proofs}

We present the proof of our main results in this section.

\begin{proof}[Proof of Theorem \ref{T0}] Let $d_I>0$ and define \begin{equation}\label{C0} 
N_{\rm low}(d_I)=\inf_{l\ge l^*}\mathcal{N}_{d_I}(l),
\end{equation}
where $\mathcal{N}_{d_I}(l)$ is defined by \eqref{N-d_I-def}.  It is clear from \eqref{lem2-eq2} that $N_{\rm low}(d_I)\le \min\{l^*|\Omega|,\int_{\Omega}\frac{\gamma}{\beta}\}=\min\Big\{ \frac{|\Omega|}{\mathcal{R}(d_I)},\int_{\Omega}\frac{\gamma}{\beta}\Big\}$. Furthermore, since $\mathcal{N}_{d_I}(l)>0$ for every $l\ge l^*$ and converges to a positive number as $l$ approaches infinity, then $N_{\rm low}(d_I)>0$.  Next, we prove assertions  {\rm (i)}-{\rm (iii)}.



\quad {\rm (i)}   If $(S,I)$ is an EE solution of \eqref{e1} for some $N>0$ and $d_S>0$, then by Lemma \ref{lem3}, $l:=\frac{\kappa}{d_S}>l^*$, where $\kappa$ is defined by \eqref{kappa-def}. Furthermore, by \eqref{N-equation}, we have that 
$$ 
N=\frac{\kappa}{d_S}\int_{\Omega}(1-d_Iu^{\frac{\kappa}{d_S}})+\kappa\int_{\Omega}u^{\frac{\kappa}{d_S}}=\mathcal{N}_{d_I}(\frac{\kappa}{d_S}) +\kappa\int_{\Omega}u^{\frac{\kappa}{d_S}}>N_{\rm low}(d_I).
$$
This shows that \eqref{e1} has no EE solution for every $N\le N_{\rm low}(d_I) $ and $d_S>0$.

\quad {\rm (ii)} Let $N>N_{\rm low}(d_I)$ be fixed. Then, there is $l(N,d_I)>l^*$ such that 
\begin{equation}\label{C4}
\mathcal{N}_{d_I}(l(N,d_I))<N.
\end{equation} 
Set 
\begin{equation}\label{C4-2}
    d(N,d_I):=
        \frac{N-\mathcal{N}_{d_I}(l(N,d_I))}{l(N,d_I)\int_{\Omega}u^{l(N,d_I)}}>0. 
\end{equation}
For every $d_S>0$, consider the function $\mathcal{N}_{d_I,d_S}$ defined by 
    \begin{equation}\label{N-d_I-d_S-def}
        \mathcal{N}_{d_I,d_S}(l)=\mathcal{N}_{d_I}(l)+d_Sl\int_{\Omega}u^l\quad \forall\ l\ge l^*.
    \end{equation}
     Then $\mathcal{N}_{d_I,d_S}$ is continuously differentiable in $l>l^*$. Furthermore, 
    \begin{equation}\label{EE1}
    \mathcal{N}_{d_I,d_S}(l^*)=l^*|\Omega| \quad \text{and}\quad \lim_{l\to\infty}\mathcal{N}_{d_I,d_S}(l)=\infty.
    \end{equation} 
    Next, fix $0<d_S<d(N,d_I)$. It follows from \eqref{C4-2} that  \begin{align}\label{C2}
    \mathcal{N}_{d_I,d_S}(l(N,d_I))=&\mathcal{N}_{d_I}(l(N,d_I))+d_Sl(N,d_I)\int_{\Omega}u^{l(N,d_I)}\cr <&\mathcal{N}_{d_I}(l(N,d_I))+d(N,d_I)l(N,d_I)\int_{\Omega}u^{l(N,d_I)} =N.
    \end{align}
      Therefore, by the intermediate value theorem, there is $l(N,d_I,d_S)>l(N,d_I)$ such that $\mathcal{N}_{d_I,d_S}(l(N,d_I,d_S))=N$. This together with \eqref{EE1} imply that the quantity  
      \begin{equation}\label{l-high-def}
        l_{\rm high}(N,d_I,d_S):=\max\{l>l(N,d_I)\ :\ \mathcal{N}_{d_I,d_S}(l)=N\}
    \end{equation}
    is a positive real number.  Observe that
    \begin{equation}\label{a-rev-0}
        \mathcal{N}_{d_I,d_S}(l_{\rm high}(N,d_I,d_S))=N\quad \text{and}\quad \mathcal{N}_{d_I,d_S}(l)>N\quad \forall\ l>l_{\rm high}(N,d_I,d_S).
    \end{equation}
     By Lemma \ref{lem3}-(ii),
    \begin{equation}\label{high-ee}(S_{\rm high}(\cdot;d_S),I_{\rm high}(\cdot;d_S)):=(l_{\rm high}(N,d_I,d_S)(1-d_Su^{l_{\rm high}(N,d_I,d_S)}),d_Sl_{\rm high}(N,d_I,d_S)u^{l_{\rm high}(N,d_I,d_S)})
    \end{equation}
    is an EE solution of  \eqref{e1}. Finally, we show that \eqref{T0-eq1} holds. So, suppose that $(S,I)$ is another EE solution of \eqref{e1}. Then, by Lemma \ref{lem3}  we  have that $\frac{\kappa}{d_S}>l^*$ and  $ I=d_{S}(\frac{\kappa}{d_S})u^{\frac{\kappa}{d_S}}$. Hence, since the mapping $(l^*,\infty)\ni l\mapsto d_Slu^{l}$ is strictly increasing, and $\mathcal{N}_{d_I,d_S}(\frac{\kappa}{d_S})=N=\mathcal{N}_{d_I,d_S}(l_{\rm high}(N,d_I,d_S))$, then  $\frac{\kappa}{d_S}<l_{\rm high}(N,d_I,d_S)$, which yields the desired result.

    \quad {\rm (iii)} Suppose that $N_{\rm low}(d_I)<N<\frac{|\Omega|}{\mathcal{R}(d_I)}=l^*|\Omega|$ and $d(N,d_I)$ be given by {\rm (i)} and   $l(N,d_I)$ be as in \eqref{C4}. Fix $0<d_S<d(N,d_I)$. Observe that 
    $$ 
    \mathcal{N}_{d_I,d_S}(l(N,d_I))<N<l^*|\Omega|=\mathcal{N}_{d_{I},d_S}(l^*).
    $$ 
    Therefore, by the intermediate value theorem, there is $\tilde{l}(N,d_I,d_S)\in (l^*,l(N,d_I))$ such that $\mathcal{N}_{d_I,d_S}(\tilde{l}(N,d_I,d_S))=N$. This implies that the quantity

    %l-high-def, l-low-def
    \begin{equation}\label{l-low-def}
        l_{\rm low}(N,d_I,d_S):=\min\{l\in [l^*,l(N,d_I))\ :\ \mathcal{N}_{d_I,d_S}(l)=N\}
    \end{equation}
    is well defined and satisfies $l^*<l_{\rm low}(N,d_I,d_S)<l(N,d_I)$.  Observe that
    \begin{equation}\label{a-rev-1}
        \mathcal{N}_{d_I,d_S}(l_{\rm low}(N,d_I,d_S))=N\quad \text{and}\quad \mathcal{N}_{d_I,d_S}(l)>N\quad \forall\ l^*\le l< l_{\rm low}(N,d_I,d_S)).
    \end{equation}
      Now, by Lemma \ref{lem3}-(ii),
    \begin{equation}\label{low-ee}(S_{\rm low}(\cdot;d_S),I_{\rm low}(\cdot;d_S)):=(l_{\rm low}(N,d_I,d_S)(1-d_Su^{l_{\rm low}(N,d_I,d_S)}),d_Sl_{\rm low}(N,d_I,d_S)u^{l_{\rm low}(N,d_I,d_S)})
    \end{equation}
    is an EE solution of  \eqref{e1}.  Since $l_{\rm low}(N,d_I,d_S)<l(N,d_I)<l_{\rm high}(N,d_I,d_S)$ and the mapping $(l^*,\infty)\ni l\mapsto lu^l$ is strictly increasing, then \eqref{T0-eq2} holds.  Using again the fact that the mapping $(l^*,\infty)\ni l\mapsto lu^l$ is strictly increasing, it can be shown as in the case of \eqref{T0-eq1} that any other EE solution of \eqref{e1}, if exists, must satisfy \eqref{T0-eq3}.   We proceed to establish that the following two claims hold. 
    
    \noindent{\bf Claim 1.} In this step, we show that 
    \begin{equation}\label{a-rev-2}
        l_{\rm low}(N,d_I,d_{S,1})<l_{\rm low}(N,d_I,d_{S,2})\quad \forall\ 0<{d_{S,1}}<d_{S,2}<d(N,d_I).
    \end{equation}
    Indeed, fix $0<d_{S,1}<d_{S,2}<d(N,d_I)$. Then,
    \begin{align*}
        \mathcal{N}_{d_I,d_{S,1}}(l_{\rm low}(N,d_I,d_{S,2}))
        =&\mathcal{N}_{d_I,d_{S,2}}(l_{\rm low}(N,d_I,d_{S,2}))-(d_{S,2}-d_{S,1})l_{\rm low}(N,d_I,d_{S,2})\int_{\Omega}u^{l_{\rm low}(N,d_I,d_{S,2})}\cr 
        =&N-(d_{S,2}-d_{S,1})l_{\rm low}(N,d_I,d_{S,2})\int_{\Omega}u^{l_{\rm low}(N,d_I,d_{S,2})}<N.
        \end{align*}
    Hence, by \eqref{l-low-def} and \eqref{a-rev-1}, we have that \eqref{a-rev-2} holds.

    \noindent{\bf Claim  2.} In the current step, we show that 
    \begin{equation}\label{a-rev-3}
        l_{\rm high}(N,d_I,d_{S,1})>l_{\rm high}(N,d_I,d_{S,2})\quad \forall\ 0<{d_{S,1}}<d_{S,2}<d(N,d_I).
    \end{equation}
    Indeed, fix $0<d_{S,1}<d_{S,2}<d(N,d_I)$. Then,
    \begin{align*}
        \mathcal{N}_{d_I,d_{S,1}}(l_{\rm high}(N,d_I,d_{S,2}))=&\mathcal{N}_{d_I,d_{S,1}}(l_{\rm high}(N,d_I,d_{S,2}))-(d_{S,2}-d_{S-1})l_{\rm high}(N,d_I,d_{S,2})\int_{\Omega}u^{l_{\rm high}(N,d_I,d_{S,2})}\cr
        =&N-(d_{S,2}-d_{S_1})l_{\rm high}(N,d_I,d_{S,2})\int_{\Omega}u^{l_{\rm high}(N,d_I,d_{S,2})}<N.
    \end{align*}
    Hence, by \eqref{l-high-def} and \eqref{a-rev-0}, we have that \eqref{a-rev-3} holds.
    
    \quad Thanks to {\bf Claim 1} and {\bf Claim 2}, the following limits exist
    \begin{equation*}
        l_{\rm low}(N,d_I)=\lim_{d_S\to 0^+}l_{\rm low}(N,d_I,d_S)=\inf_{0<d_S<d(N,d_I)}l_{\rm low}(N,d_I,d_S)<l(l,N)
    \end{equation*}
    and 
    \begin{equation*}
        l_{\rm high}(N,d_I):=\lim_{d_S\to 0^+}l_{\rm high}(N,d_I,d_S)=\sup_{0<d_S<d(N,d_I)}l_{\rm high}(N,d_I,d_S)>l(N,d_I).
    \end{equation*}

    \quad {\rm (iii-1)} Next, suppose that $N<\int_{\Omega}\frac{\gamma}{\beta}$ and we establish that \eqref{T0-eq4} holds.    In the current case, we first proceed by contradiction to show that 
    \begin{equation}\label{G1}
        l_{\rm high}(N,d_I)<\infty.
    \end{equation}
    Indeed, if \eqref{G1} were false, then  $l_{\rm high}(N,d_I,d_{S})\to \infty$. As, a result, it follows from \eqref{lem2-eq2} that 
    \begin{equation}\label{G2}
        \lim_{d_S\to0}\int_{\Omega}l_{\rm high}(N,d_I,d_{S})(1-d_Iu^{l_{\rm high}(N,d_I,d_{S})})=\int_{\Omega}\frac{\gamma}{\beta}.
    \end{equation}
    On the other hand, using the fact that
    \begin{equation*}
        N=\mathcal{N}_{d_I,d_{S}}(l_{\rm high}(N,d_I,d_{S}))>\int_{\Omega}l_{\rm high}(N,d_I,d_{S})(1-d_Iu^{l_{\rm high}(N,d_I,d_{S})})\quad \forall\ 0<d_S<d(N,d_I),
    \end{equation*}
    we obtain that 
    $$ 
    N\ge \lim_{d_S\to 0}\int_{\Omega}l_{\rm high}(N,d_I,d_{S})(1-d_Iu^{l_{\rm high}(N,d_I,d_{S})}),
    $$
    which clearly contradicts with \eqref{G2} since $N<\int_{\Omega}\frac{\gamma}{\beta}$. Thus, \eqref{G1} holds.  This shows that there is a positive constant $C_1(N,d_I)$ such that 
    \begin{equation}\label{G3}
        l_{\rm high}(N,d_I,d_S)\le C_1(N,d_I) \quad \forall\ 0<d_S<\frac{d(N,d_I)}{2}.
    \end{equation}
    In particular,
\begin{equation}\label{G7}
    I_{\rm low}(\cdot;d_S)<I_{\rm high}(\cdot;d_S)=d_Sl_{\rm high}(N,d_I,d_S)u^{l_{\rm high}(N,d_I,d_S)}\le \frac{C_1(N,d_I)}{d_I}d_S\quad \forall\ 0<d_S<\frac{d(N,d_I)}{2}.
\end{equation}
Next, we claim that 
\begin{equation}\label{G4}
    l_{\rm low}(N,d_I)>l^*.
\end{equation}
If \eqref{G4} were false, then  $l_{\rm low}(N,d_I,d_{S})\to l^*$ as $d_S\to 0$. This in turn implies that 
\begin{align*}
    N=&\mathcal{N}_{d_I,d_{S}}(l_{\rm low}(N,d_I,d_{S}))\cr 
    =&\mathcal{N}_{d_{I}}(l_{\rm low}(N,d_I,d_{S}))+d_{S}l_{\rm low}(N,d_I,d_{S})\int_{\Omega}u^{l_{\rm low}(N,d_I,d_{S})} 
     \to  l^*|\Omega|\quad \text{as}\ d_{S}\to 0,
\end{align*}
which contradicts our initial assumption that $N<l^*|\Omega|$. Therefore, \eqref{G4} holds. Thus, there is $C_2(N,d_I)>l^*$ such that 
\begin{equation}\label{G5}
    C_2(N,d_I)\le l_{\rm low}(N,d_I,d_S)\le l(N,d_I)\quad \forall\  0<d_S<\frac{d(N.d_I)}{2}.
\end{equation}
As a result, we obtain that 
\begin{align}\label{G6}
I_{\rm low}(\cdot;d_S)=&d_{S}l_{\rm low}(N,d_I,d_S)u^{l_{\rm low}(N,d_I,s_S)}
\ge C_2(N,d_I)d_Su^{C_2(N,d_I)}(\cdot)\cr 
\ge & C_2u^{C_2(N,d_I)}_{\min}d_S\quad \forall\ 0<d_S<\frac{d(N,d_I)}{2}.
\end{align}
Combining \eqref{G6} and \eqref{G7} we derive that \eqref{T0-eq5} holds. 

\quad Next, since $l_{\rm high}>l_{\rm low}>l^*$, then by Lemma \ref{lem2}-{\rm (ii)}, we have that 
$$ 
S_{\rm low}(\cdot;d_S)=l_{\rm low}(N,d_I,d_S)(1-d_Iu^{l_{\rm low}(N,d_I,d_S)})\to l_{\rm low}(1-d_Iu^{l_{\rm low}(N,d_I)})
$$
and $$ 
S_{\rm high}(\cdot;d_S)=l_{\rm high}(N,d_I,d_S)(1-d_Iu^{l_{\rm high}(N,d_I,d_S)})\to l_{\rm high}(1-d_Iu^{l_{\rm high}(N,d_I)})
$$
as $d_S\to 0$ in $C^{1}(\overline{\Omega})$. On the other hand, since $N=\int_{\Omega}(S+I)$, we have that 
$$ 
N=l_{\rm low}(N,d_I)\int_{\Omega}(1-d_Iu^{l_{\rm low}(N,d_I)}) \quad \text{and}\quad N=l_{\rm high}(N,d_I)\int_{\Omega}(1-d_Iu^{l_{\rm high}(N,d_I)}).
$$
Since  $l_{\rm low}(N,d_I)<l(N,d_I)<l_{\rm high}(N,d_I)$, then $u^*_{\rm low}:=u^{l_{\rm low}(N,d_I)}<u^{l_{\rm high}(N,d_I)}=:u^*_{\rm high}$,  which completes the proof of \eqref{T0-eq5}.


\quad {\rm (iii-2)} Finally suppose that $N>\int_{\Omega}\frac{\gamma}{\beta}$. Note that the proof of \eqref{G5} only relies on the fact that $N\ne l^*|\Omega|$. Hence, $(S_{\rm low}(\cdot;d_S),I_{\rm low}(\cdot;d_S))$ satisfies \eqref{T0-eq4} and \eqref{T0-eq5} as $d_S\to 0$. Next, we claim that 
\begin{equation}\label{G8}
    l_{\rm high}(N,d_I)=\infty.
\end{equation}
Indeed, since $N>\int_{\Omega}\frac{\gamma}{\beta}=\lim_{l\to\infty}\mathcal{N}_{d_I}(l)$, then for every $m>1$, there is $l_m(N,d_I)>m$ such that 
$$ 
\mathcal{N}_{d_I}(l_m(N,d_I))<N.
$$
Therefore, taking this time $d_m(N,d_I):=\frac{N-\mathcal{N}_{d_I}(l_m(N,d_I))}{l_m(N,d_I)\int_{\Omega}u^{l_m(N,d_I)}}$, for every $0<d_S<d_m(N,d_I)$, we can employ similar arguments as in \eqref{C2} and the intermediate value theorem to conclude that there is ${l}_m(N,d_I,d_S)>l_m(N,d_I)$ such that $\mathcal{N}_{d_I,d_S}(l_{m}(N,d_I,d_S))=N$. This shows that 
$$ 
l_{\rm high}(N,d_I,d_S)\ge l_{m}(N,d_I,d_S)>l_{m}(N,d_I)>m\quad \forall\ 0<d_S<d_{m}(N,d_I).
$$
Letting $m\to \infty$ in this inequality leads to \eqref{G8}. Thus, since $l(1-d_Iu^l)\to \frac{\gamma}{\beta}$ as $l\to\infty$ in $C(\overline{\Omega})$ (see \eqref{lem2-eq1}), we conclude that 
$$
S_{\rm high}(\cdot,d_S)=l_{\rm high}(N,d_I,d_S)(1-du^{l_{\rm high}(N,d_I,d_S)})\to \frac{\gamma}{\beta}\quad \text{as}\ d_S\to 0
$$
uniformly in $C(\overline{\Omega})$. Observing that
$$ 
d_Sl_{\rm high}(N,d_I,d_S)=\frac{N-\int_{\Omega}S_{\rm high}}{\int_{\Omega}u^{l_{\rm high}(N,d_I,d_S)}}\to \frac{N-\int_{\Omega}\frac{\gamma}{\beta}}{\frac{|\Omega|}{d_I}}\quad \text{as}\ d_S\to 0,
$$
where we have used the fact that $u^{l_{\rm high}(N,d_I,d_S)}\to \frac{1}{d_I}$ as $d_S\to 0$ uniformly on $\Omega$, then 
$$
I_{\rm high}(\cdot;d_S)=d_Sl_{\rm high}(N,d_I,d_S)u^{l_{\rm high}(N,d_I,d_S)}\to \frac{\Big(N-\int_{\Omega}\frac{\gamma}{\beta}\Big)}{\frac{|\Omega|}{d_I}}\frac{1}{d_I}=\frac{1}{|\Omega|}\left(N-\int_{\Omega}\frac{\gamma}{\beta}\right)\quad \text{as}\ d_S\to 0,
$$
in $C(\overline{\Omega})$. This completes the proof of the theorem.
    
    
    
\end{proof}

Next, we give a proof of Theorem \ref{T1}.

\begin{proof}[Proof of Theorem \ref{T1}] Fix $d_I>0$ and define 
\begin{equation*}
    M_{d_I}^*=\frac{d_I}{|\Omega|}\sup_{l>l^*}\int_{\Omega}(u^l+lv^l),
\end{equation*}
where for every $l>l^*$, $u^l$ and $v^l$ are the unique positive solutions of \eqref{Eq1} and \eqref{v-l-eq1}, respectively. Note from  \eqref{lem2-eq1} and \eqref{v-l-eq2} that $u^l\to \frac{1}{d_I}$ and $lv^l\to 0$ as $l\to\infty$ uniformly on $\Omega$.  Hence, 
$$ 
\lim_{l\to\infty}\int_{\Omega}(u^l+lv^l)=\frac{|\Omega|}{d_I}.
$$
Note also from \eqref{lem2-eq1} and \eqref{v-l-eq2} that 
$  
\lim_{l\to l^*}\int_{\Omega}(u^l+lv^l)=\frac{\Big(\int_{\Omega}\varphi_{d_I}\Big)\Big(\int_{\Omega}\beta\varphi_{d_I}^2\Big)}{d_I\int_{\Omega}\beta\varphi_{d_I}^3}.
$   Hence 
\begin{equation}\label{K1}
\max\left\{1,\frac{\Big(\int_{\Omega}\varphi_{d_I}\Big)\Big(\int_{\Omega}\beta\varphi_{d_I}^2\Big)}{|\Omega|\int_{\Omega}\beta\varphi_{d_I}^3}\right\}\le M^*_{d_I}<\infty.
\end{equation}
Therefore, defining 
\begin{equation}\label{K2}
    m_{d_I}^*=\frac{1}{M^*_{d_I}},
\end{equation}
$m^*_{d_I}$ satisfies \eqref{T1-main-eq}. Next, $d_S>0$ and consider the function $\mathcal{N}_{d_I,d_S}$ be defined as in \eqref{N-d_I-d_S-def}. Taking the derivative of the function $\mathcal{N}_{d_I,d_S}$ if \eqref{N-d_I-d_S-def} with respect to $l$, we get 
\begin{equation}\label{C1-1}
    \frac{d\mathcal{N}_{d_I,d_S}(l)}{dl}= |\Omega|+(d_S-d_I)\int_{\Omega}(lv^l+u^l) \quad \forall\ l>l^*.
\end{equation} 
From this point, we distinguish two cases.

\noindent{\bf Case 1.} Fix $d_S>d_{I}(1-m_{d_I}^*)$. We shall show that 
\begin{equation}\label{C1-3}
\frac{d\mathcal{N}_{d_I,d_S}(l)}{dl}>0 \quad \forall\ l>l^*.
\end{equation}
If $d_S\ge d_I$, it is easy to see from \eqref{C1-1} that \eqref{C1-3} holds. So, we suppose that $d_I(1-m^*_{d_I})<d_S<d_I$. Hence, by \eqref{C1-1}, we have that 
\begin{align*}
  \frac{d\mathcal{N}_{d_I,d_S}(l)}{dl} =& |\Omega|\left(1-\Big(1-\frac{d_S}{d_{I}}\Big)\frac{d_I}{|\Omega|}\int_{\Omega}(u^l+lv^l)\right) \cr 
  \ge & |\Omega|\left(1-\Big(1-\frac{d_S}{d_{I}}\Big)M_{d_I}^*\right) 
  =\frac{M_{d}^*|\Omega|}{d_I}\Big(d_S-d_I(1-m_{d_I}^*)\Big)>0,  
\end{align*}
which shows that \eqref{C1-3} holds in this subcase as well. Therefore, when $d_S>d_I(1-m_{d_I}^*)$, the mapping $\mathcal{N}_{d_I,d_S}$ is strictly increasing on $[l^*,\infty)$. Hence, thanks to Lemma \ref{lem3}, \eqref{e1} has an EE solution if and only if $N>\mathcal{N}_{d_I,d_S}(l^*)$, that is $\mathcal{R}_0(N,d_I)>1$. Moreover, in this case, when an EE exists, it is unique.

\noindent{\bf Case 2.} Fix $0<d_S<d_{I}(1-m_{d_I}^*)$ or equivalently  $m_{d_I}^*<1-\frac{d_S}{d_I}$.  Then  
 $ 
0<\frac{1}{1-\frac{d_S}{d_I}}<\frac{1}{m_{d_I}^*}=M_{d_I}^*. $
Therefore, there is $l_0>l^*$ such that 
$  
0<\frac{1}{1-\frac{d_S}{d_I}}<\frac{d_I}{|\Omega|}\int_{\Omega}(u^{l_0}+l_0v^{l_0}),
$ 
 which implies that 
\begin{equation}\label{GH1}
    \frac{d\mathcal{N}_{d_I,d_S}(l_0)}{dl}=|\Omega|\left(1-\Big(1-\frac{d_S}{d_{I}}\Big)\frac{d_I}{|\Omega|}\int_{\Omega}(u^{l_0}+l_0v^{l_0})\right)<0.
\end{equation}
On the other hand, we know from \eqref{lem2-eq1} and \eqref{v-l-eq2} that 
\begin{equation}\label{GH2}
    \lim_{l\to\infty}\frac{d\mathcal{N}_{d_I,d_S}(l)}{dl}=|\Omega|+(d_S-d_I)\frac{|\Omega|}{d_I}=\frac{d_S}{d_I}|\Omega|>0.
\end{equation}
Thanks to \eqref{GH1} and \eqref{GH2}, we deduce that  there are $l_0<l_1<l_2$ such that $\mathcal{N}_{d_I,d_S}(l_1)=\mathcal{N}_{d_I,d_S}(l_2)$. As a result, for $N=\mathcal{N}_{d_I,d_S}(l_1)=\mathcal{N}_{d_I,d_S}(l_2)$, we have from Lemma \ref{lem3} that 
$$ 
(S_1,I_1)=(l_1(1-d_Iu^{l_1}),d_Sl_1u^{l_1})
\quad 
\text{and} 
\quad  
(S_2,I_2)=(l_2(1-d_Iu^{l_2}),d_Sl_2u^{l_2})
$$
 are two distinct EE solutions of \eqref{e1}. This completes the proof of the theorem.
\end{proof}

Next, we give a proof of Theorem \ref{T2-2}.

\begin{proof}[Proof of Theorem \ref{T2-2}] Let $d_I>0$ and $N_{\rm low}(d_I)$ and $m_{d_I}^*$ be given by Theorems \ref{T0} and \ref{T1}, respectively. Suppose that $N_{\rm low}(d_I)<\frac{|\Omega|}{\mathcal{R}(d_I)}=l^*|\Omega|$. We proceed by contradiction to show that $m_{d_I}^*<1$. Indeed, if it was the case that $m_{d_I}^*=1$, then for every $d_S>0=d_{I}(1-m_{d_I}^*)$, \eqref{e1} has a (unique) EE if and only if $\mathcal{R}_0(N,d_I)>1$. However, by Theorem \ref{T0}, we know that for every $N_{\rm low}(d_I)<N<l^*|\Omega|$, $\mathcal{R}_0(N,d_I)<1$ and \eqref{e1} has at least two EE solution. So, we obtain a contradiction. Hence, we must have that $m_{d_I}^*<1$. It remains to prove {\rm (i)} and {\rm (ii)}.

\quad {\rm (i)} If $\frac{1}{|\Omega|}\int_{\Omega}\frac{\gamma}{\beta}<\frac{1}{\mathcal{R}(d_I)}$, then by Theorem \ref{T0}, $N_{\rm low}(d_I)\le \int_{\Omega}\frac{\gamma}{\beta}<l^*|\Omega|$.

\quad {\rm (ii)} If ${|\Omega|\int_{\Omega}\beta\varphi_{d_I}^3}<{\big(\int_{\Omega}\varphi_{d_I}\big)\big(\int_{\Omega}\beta\varphi_{d_I}^2\big)} $, then by \eqref{v-l-eq2} and \eqref{C1-1},  $\frac{d\mathcal{N}_{d_I}(l)}{dl}\to |\Omega|-\frac{\big(\int_{\Omega}\varphi_{d_I}\big)\big(\int_{\Omega}\beta\varphi_{d_I}^2\big)}{\int_{\Omega}\beta\varphi_{d_I}^3}<0$ as $l\to l^*$. Hence $N_{\rm low}(d_I)=\inf_{l\ge l^*}\mathcal{N}_{d_I}(l)<\mathcal{N}_{d_I}(l^*)=l^*|\Omega|.$  

    
\end{proof}


We complete this section with  a proof of Theorem \ref{T2-2}.

\begin{proof}[Proof of Theorem \ref{T2}] Suppose that \eqref{T2-eq1} holds.
 Then by \eqref{v-l-eq2},  
\begin{equation}\label{ZXZ1}
\lim_{l\to l^*}\frac{d\mathcal{N}_{d_I}(l)}{dl}= |\Omega|-\frac{\Big(\int_{\Omega}\varphi_{d_I}\Big)\Big(\int_{\Omega}\beta\varphi_{d_I}^2\Big)}{\int_{\Omega}\beta\varphi_{d_I}^3}>0.
\end{equation}
Thus,  since 
\begin{equation}\label{ZXZ2}
\lim_{l\to\infty}\mathcal{N}_{d_I}(l)= \int_{\Omega}\frac{\gamma}{\beta}<l^*|\Omega|=\mathcal{N}_{d_I}(l^*),
\end{equation}
then 
$$ 
\max_{l\ge l^*}\mathcal{N}_{d_I}(l)>\max\left\{\lim_{l\to\infty}\mathcal{N}_{d_I}(l), \mathcal{N}_{d_I}(l^*)\right\}.
$$
 We can now employ similar arguments as in the proof of Theorem \ref{T0}-{\rm (iii)} to conclude that for every $N\in\Big(\max_{l\ge l^*}\mathcal{N}_{d_I}(l),\mathcal{N}_{d_I}\Big(l^*\Big) \Big)$, there is $\tilde{d}(N,d_I)$ such that \eqref{e1} has at least two EE solutions for every $0<d_S<\tilde{d}(N,d_I)$. This completes the proof of {\rm (i)}. Next, we prove {\rm (ii)}.
 
 \quad Next, by \eqref{ZXZ1} and \eqref{ZXZ2}, there exit $l_2^*>l_1^*>l^*$ such that $\mathcal{N}_{d_I}$ is strictly increasing on $[l^*,l_1^*] $ and  
 \begin{equation}
    M^*:= \sup_{l\ge l_2^*}\mathcal{N}_{d_I}(l)<\mathcal{N}_{d_I}(l^*).
 \end{equation}
 Take $d_S^*=\frac{\mathcal{N}_{d_I}(l^*)-M^*}{l_2^*\int_{\Omega}u^{l_2^*}}$ and  fix $0<d_S<d_S^*$. Note that $\mathcal{N}_{d_I,d_S}(l^*)=\mathcal{N}_{d_I}(l^*)=|\Omega|l^*$ and
 $$
 \mathcal{N}_{d_I,d_S}(l_{2}^*)=\mathcal{N}_{d_I}(l_2^*)+l_2^*d_S\int_{\Omega}u^{l^*_2}<\mathcal{N}_{d_I}(l_2^*)+l_2^*d_S^*\int_{\Omega}u^{l^*_2}=\mathcal{N}_{d_I}(l_2^*)+\mathcal{N}_{d_I,d_S}(l^*)-M^*\le \mathcal{N}_{d_I,d_S}(l^*).
 $$
 Hence, since $\mathcal{N}_{d_I,d_S}(l)\to \infty$ as $l\to\infty$, the global minimum value of $\mathcal{N}_{d_I,d_S}(l)$ is achieved at some interior point $l_{d_S}^*>l^*$ and satisfies   
 \begin{equation*}
     N^{\rm low}_{d_I,d_S}:=\mathcal{N}_{d_I,d_S}(l^*_{d_S})=\min_{l\ge l^*}\mathcal{N}_{d_I,d_S}(l)<\mathcal{N}_{d_I,d_S}(l^*)\quad \text{and}\quad \mathcal{N}_{d_I,d_S}(l)>N^{\rm low}_{d_I,d_S}\quad \forall\ l>l^*_{d_{S}}.
 \end{equation*}
 So, by Lemma \ref{lem3}, system \eqref{e1} has no EE solution for $N<N^{\rm low}_{d_I,d_S}$ and $$
 (S,I):=(l^*_{d_S}(1-d_Iu^{l^*_{d_S}}),d_Sl^*_{d_S}u^{l^*_{d_S}})
 $$ is an EE solution of \eqref{e1} for $N=N^{\rm low }_{d_I,d_S}$.  Recalling that $\frac{d\mathcal{N}_{d_I,d_S}(l)}{dl}>\frac{d\mathcal{N}_{d_I}(l)}{dl}$ for all $l>l^*$, then we have that $\mathcal{N}_{d_I,d_S}$ is also strictly increasing on $[l^*,l_1^*]$. Hence, the absolute maximum of $\mathcal{N}_{d_I,d_S}$ on $[l^*,l_{d_S}^*]$ is achieved at some $\tilde{l}^*_{d_S}\in[l_1^*,l_{d_S}^*)$. So, we define 
 $ 
 \tilde{N}^{\rm sup }_{d_I,d_S}:=\mathcal{N}_{d_I,d_S}(\tilde{l}^*_{d_S}).
 $  Note that $\tilde{N}^{\rm sup}_{d_I,d_S}>|\Omega|l^*$. Next, by \eqref{GH2}, there is $\hat{l}>l^*_{d_S}$ such that $\mathcal{N}_{d_I,d_S}$ is strictly increasing on $[\hat{l},\infty)$. Hence, we can define $\hat{l}^*_{d_S}:=\inf\{\hat{l}>l_{d_S}^* : \mathcal{N}_{d_I,d_S}\ \text{is strictly increasing on}\ [\hat{l},\infty)\}$  and $\hat{N}^{\rm sup}_{d_I,d_S}:=\max_{l\in [l^*,\hat{l}^{\rm sup}_{d_S}]}\mathcal{N}_{d_I,d_S}(l)$. From this point, we distinguish four cases in order to complete the proof of the theorem.

 \medskip
 
 {\bf Case 1}. $N\in (N^{\rm low}_{d_I,d_S}, ,|\Omega|l^*]$. By the intermediate value theorem,  it follows as in \eqref{l-high-def} and \eqref{l-low-def} that both $ l_{\rm high}(N,d_I,d_S)>l_{d_S}^*$ and $l_{\rm low}(N,d_I,d_S)\in (\tilde{l}^*_{d_S},l_{d_S}^*) $ are well defined. Moreover, $(S_{\rm high},I_{\rm high})$ and $(S_{\rm low}, I_{\rm low})$ defined as in \eqref{high-ee} and \eqref{low-ee}, respectively, are two distinct EE solutions of \eqref{e1}. 

 \medskip
 
 {\bf Case 2.} $N=\tilde{N}^{\rm sup}_{d_I,d_S}$.  By the intermediate value theorem,  it follows as in \eqref{l-high-def}  that $ l_{\rm high}(N,d_I,d_S)>l_{d_S}^*$ is well defined. Moreover, $(S_{\rm high},I_{\rm high})$, defined as in \eqref{high-ee},  and $(\tilde{l}^*_{d_S}(1-d_Iu^{\tilde{l}^*_{d_S}}),d_S\tilde{l}^*_{d_S}u^{\tilde{l}^*_{d_S}})$  are two distinct EE solutions of \eqref{e1}.

 \medskip
 
 {\bf Case 3.} $N\in (|\Omega|l^*,\tilde{N}^{\rm sup}_{d_I,d_S})$, By the intermediate theorem, we have that there exist $l_{N,1}\in (l^*,\tilde{l}^*_{d_S})$, $l_{N,2}\in (\tilde{l}^*_{d_S},l_{d_S}^*)$, and $l_{N,3}>l_{d_S}^*$ such that $(l_{N,i}(1-d_Iu^{l_{N,i}}),d_Sl_{N,i}u^{l_{N,i}})$, $i=1,2,3$, are three different EE solutions of \eqref{e1}.

 \medskip
 
 {\bf Case 4.} $N>\hat{N}^{\rm sup}_{d_I,d_S}$. Then, by the intermediate value theorem, the definition of $\hat{N}^{\rm sup}_{d_I,d_S}$, and the fact that $\mathcal{N}_{d_I,d_S}$ is strictly increasing on $[\hat{l}^*_{d_S},\infty)$ there is a unique $\hat{l}>\hat{l}^*_{d_S}$ such that $\mathcal{N}_{d_I,d_S}(\hat{l})=N$ and $\mathcal{N}_{d_I,d_S}(l)<N$ for all $l\in[l^*,\hat{l}^*_{d_S}]$. Therefore, $(\hat{l}(1-d_Iu^{\hat{l}}),d_S\hat{l}u^{\hat{l}})$ is the unique EE solution of \eqref{e1}. 
 
  
 

 
    
\end{proof}



\section{Appendix} 
Let $\lambda_0=0<\lambda_1\le \lambda_2\le \cdots\le \lambda_m\le\cdots$ satisfying $\lambda_m\to \infty$ as $m\to\infty$ denote the eigenvalues of 
\begin{equation*}
    \begin{cases}
        0=\Delta \phi +\lambda\phi & x\in\Omega,\cr
        0=\partial_{\vec{n}}\phi & x\in\partial\Omega.
    \end{cases}
\end{equation*}
Let $\{\phi_m\}_{m\ge 0}$ be the orthonormal basis  of $L^2(\Omega)$ where $\phi_m$ is an eigenfunction associated with $\lambda_m$ for each $m\ge 0$. Next, consider the Banach space $\tilde{\mathcal{Z}}:=\{w\in L^2(\Omega) : \int_{\Omega}w=0\}={\rm span}(\phi_0)$.  For every $q\ge 2$, the restriction of the Laplace operator on ${\rm Dom}_q\cap\tilde{\mathcal{Z}}$ to $\tilde{\mathcal{Z}}_q:=L^q(\Omega)\cap \tilde{\mathcal{Z}}$ is invertible. Lettting $C_q^*:=\|\Delta^{-1}_{|{\rm Dom}_q\cap\tilde{\mathcal{Z}}}\|$,  for every $w\in \tilde{\mathcal{Z}}_q$, the unique solution $W\in {\rm Dom}_q\cap\tilde{\mathcal{Z}}$  of 
\begin{equation}\label{appen-1}
    \begin{cases}
        0=\Delta W+w & x\in\Omega,\cr 
        0=\partial_{\vec{n}}W & x\in\partial\Omega,\cr
        0=\int_{\Omega}W
    \end{cases}
\end{equation}
satisfies
\begin{equation}\label{appen-2}\|W\|_{W^{2,q}(\Omega)}\le C^*_q\|w\|_{L^q(\Omega)}.\end{equation}
Fix  a H\"older continuous and non-constant function $h$ on $\overline{\Omega}$,  and positive constants $k>0$ and $d_I>0$. Define
\begin{equation}\label{UO1}
    l_0^*=k,\quad \tilde{\varphi}_0=\frac{1}{|\Omega|},\quad \quad \text{and}\quad l_1^*=\frac{1}{|\Omega|}\int_{\Omega}h.
\end{equation}
Let $\tilde{\varphi}_1$ be  the unique solution \eqref{appen-1} with $w_1:=l_0^*(l_1^*-h)\tilde{\varphi}_0/d_I$. Note that $\tilde{\varphi}_1$ is well defined since $\int_{\Omega}w_1=0.$ Next, define 
\begin{equation}\label{l-star-2-eq}
l_2^*=\frac{1}{k}\Big(\frac{1}{|\Omega|}\int_{\Omega}(h-l_1^*)h+k\int_{\Omega}(h-l_1^*)\tilde{\varphi}_1\Big),
\end{equation}
and $\tilde{\varphi}_2$ is the unique solution of \eqref{appen-1} with $w_2:=(kl_2^*\tilde{\varphi}_0 -h(h-l_1^*)\tilde{\varphi}_0-k(h-l_1^*)\tilde{\varphi}_1)/d_I.$ Note also that $\tilde{\varphi}_2$ is well defined since $\int_{\Omega}w_2=0$.  Throughout the rest of this section, whenever $h$, $k$ and $d_I$ are given, we shall suppose that $l^*_0$, $l^*_1$, $l_2^*$, $\tilde{\varphi}_0$, $\tilde{\varphi}_1$, and $\tilde{\varphi}_2$ are defined as above.

\begin{prop}\label{appen-prop2} Fix 
$k>0$ and $d_I>0$ and suppose that $h=c_m\phi_m$ for some $m\ge 1$ where $c_m$ is a nonzero constant. Then 
\begin{equation}
    \tilde{\varphi}_1=-\frac{k}{|\Omega|d_I\lambda_m}h\quad \text{and}\quad l_2^*=\frac{1}{k|\Omega|}\Big(1-\frac{k^2}{d_I\lambda_m}\Big)\int_{\Omega}h^2.
\end{equation}


% H\"older continuous and non-constant function $h$ on $\overline{\Omega}$. Then, there is $d_{k,h}^*>0$ such that $l_2^*>0$  for any $d_I>d_{k,h}^*$, $l^*_2$ is given by \eqref{l-star-2-eq}. 




% fix $d_I>0$ such that  
% \begin{equation}\label{prop-eq1}
%     \frac{1}{|\Omega|}\Big(\int_{\Omega}h^2-\frac{1}{|\Omega|}\Big(\int_{\Omega}h\Big)^2\Big)-\frac{(C_2^{*}k)^2}{{|\Omega|}d_I}\Big\|\frac{1}{|\Omega|}\int_{\Omega}h-h\Big\|_{L^2(\Omega)}^2>0.
% \end{equation}
% where $C_2^*$ is given by \eqref{appen-2}. Then $l_2^*>0$. 
    
\end{prop}

\begin{proof}It can be verified by inspection.


% Since $\tilde{\varphi}_1$ solves \eqref{appen-1} with $w_1=\frac{k}{|\Omega|d_I}\Big( \frac{1}{|\Omega|}\int_{\Omega}h- h\Big)$, then 
%  by \eqref{appen-1}
% \begin{equation}\label{RT5-2-1}
%    \|\tilde{\varphi}_1\|_{W^{2,2}(\Omega)}\le \frac{k C_2^*}{|\Omega|d_I}\Big\| \frac{1}{|\Omega|}\int_{\Omega}h- h\Big\|_{L^2(\Omega)}
% \end{equation}
% It then follows from \eqref{appen-2} that 
% \begin{align*}
% l_2^*\ge & \frac{1}{|\Omega|}\Big(\int_{\Omega}h^2-\frac{1}{|\Omega|}\Big(\int_{\Omega}h\Big)^2\Big)-\frac{(k C_2^*)^2}{|\Omega|d_I}\Big\| \frac{1}{|\Omega|}\int_{\Omega}h- h\Big\|_{L^2(\Omega)}^2>0.
% \end{align*}



\end{proof}
 

%Hence $\beta_{k,h,\varepsilon}$ is nonconstant, positive and H\"older continuous on $\overline{\Omega}$.\vspace{1 in}

% where $l_0^*=k$, $\tilde{\varphi}_0=1/|\Omega|$, $l^*_1=\tilde{\varphi}_0\int_{\Omega}h$, $\tilde{\varphi}_1$ is the unique solution \eqref{appen-1} with $w_1:=l_0^*(l_1^*-h)\tilde{\varphi}_0/d_I$,
% \begin{equation} \label{l-star-2-eq}
%  l_2^*=\frac{1}{|\Omega|}\Big(\int_{\Omega}h^2-\frac{1}{|\Omega|}\Big(\int_{\Omega}h\Big)^2\Big)-d_I{|\Omega|}\int_{\Omega}|\nabla \tilde\varphi_1|^2,
% \end{equation}
% and $\tilde{\varphi}_2$ is the unique solution of \eqref{appen-1} with $w_2:=(kl_2^*\tilde{\varphi}_0 +h(l_1^*-h)\tilde{\varphi}_0+k(l_1^*-h)\tilde{\varphi}_1)/d_I.$


\begin{prop}\label{appen-prop1} Fix $d_I>0$,
$k>0$, and $h$  H\"older continuous and non-constant function $h$ on $\overline{\Omega}$. For every, $0<\varepsilon<\varepsilon_{k,h}:=\frac{k}{\|h\|_{\infty}}$, define \begin{equation}\label{appen-3}
    \beta_{k,h,\varepsilon}=k+\varepsilon h,
\end{equation} 
and  let $l^*(\varepsilon)$ denote the principal eigenvalue of the weighted linear elliptic equation
\begin{equation}\label{prop-eq2}
    \begin{cases}
        0=d_I\Delta \tilde{\varphi}+\beta_{k,h,\varepsilon}(l(\varepsilon)-\beta_{k,h,\varepsilon})\tilde{\varphi} & x\in\Omega,\cr
        0=\partial_{\vec{n}}\tilde{\varphi} & x\in\Omega.
    \end{cases}
\end{equation}
Denote by $\tilde{\varphi}(\cdot;\varepsilon)$ the unique positive solution of \eqref{prop-eq2} satisfying $\int_{\Omega}\tilde{\varphi}(\cdot;\varepsilon)=1$. Next, define 
\begin{equation}
    \tilde{\varphi}_3(\cdot;\varepsilon)=\frac{\tilde{\varphi}(\cdot;\varepsilon)-\tilde{\varphi}_0-\varepsilon\tilde{\varphi}_1-\varepsilon^2\tilde{\varphi}_2}{\varepsilon^3} \quad \text{and}\quad l^*_3(\varepsilon)=\frac{l^*(\varepsilon)-{l}^*_0-\varepsilon l^*_1-\varepsilon^2l^*_2}{\varepsilon^3}.
\end{equation}
 It holds that 
\begin{equation}\label{prop-eq4}
    \limsup_{\varepsilon\to 0}\|\tilde{\varphi}_3(\cdot;\varepsilon)\|_{C^1(\overline{\Omega})}<\infty \quad \text{and}\quad \limsup_{\varepsilon\to 0}|l_3^*(\varepsilon)|<\infty.
\end{equation}
    
\end{prop}
\begin{proof} By computations, we have that  $\tilde{\varphi}_3$ and $l_3^*$ satisfy
\begin{equation}\label{proof-prop-app-eq1}
    \begin{cases}
        0=d_I\Delta\tilde\varphi_3+  \big(kl_2^*+h(l_1^*-h)\big)(\tilde{\varphi}_1+\varepsilon\tilde{\varphi}_2+\varepsilon^2\tilde{\varphi}_3)+(\beta_{k,h,\varepsilon}l_3^*+hl_2^*)\tilde{\varphi}+ k(l_1^*-h)(\tilde\varphi_2+\varepsilon\tilde{\varphi}_3)& x\in\Omega,\cr   0=\partial_{n}\tilde\varphi_3 & x\in\partial\Omega,\cr
        0=\int_{\Omega}\tilde\varphi_3
    \end{cases}
\end{equation}
Now, fix $q\gg 1$ such that $W^{2,q}(\Omega)$ is continuously embbeded in $C^1(\overline{\Omega})$. Then, by the similar arguments leading to \eqref{YTYT1} and the fact that, there exist $0<\varepsilon_{q,k,h}\ll 1$  and $M_{q,k,h}>0$ such that 
\begin{equation}\label{proof-prop-app-eq2}
    \|\tilde{\varphi}_3(\cdot;\varepsilon)\|_{W^{2,q}(\Omega)}\le M_{q,k,h}(1+|l_3^*|)\quad 0<\varepsilon\le  \varepsilon_{q,k,h}.
\end{equation}
Setting $A_1=kl_2^*+h(l_1^*-h)$ and $A_2=k(l_1^*-h)$, integrating \eqref{proof-prop-app-eq1} and rearranging the terms yield:
\begin{equation*}
    l_3^*\int_{\Omega}\beta_{k,h,\varepsilon}\tilde{\varphi}=-\int_{\Omega}A_1(\tilde{\varphi}_1+\varepsilon\tilde{\varphi}_2)-l_2^*\int_{\Omega}h\tilde{\varphi}-\int_{\Omega}A_2\tilde{\varphi}_2-\varepsilon \int_{\Omega}(A_2+\varepsilon A_1)\tilde{\varphi}_3.
\end{equation*}
Hence, setting $B:=|\Omega|\|A_1\|_{\infty}(\|\tilde{\varphi}_1\|_{\infty}+\|\tilde{\varphi}_2\|_{\infty})+\sum_{i=1}^2\|A_i\|_{\infty}+\sum_{i=1}^3\|\tilde{\varphi}_i\|_{\infty} $ and using \eqref{proof-prop-app-eq2}, for every $0<\varepsilon<\varepsilon_{q,k,h}$, we have
\begin{align*}
    |l_3^*(\varepsilon)|\int_{\Omega}\beta_{k,h,\varepsilon}\tilde{\varphi}\le & |\Omega|\|A_1\|_{\infty}(\|\tilde{\varphi}_1\|_{\infty}+\varepsilon\|\tilde{\varphi}_2\|_{\infty})+|\Omega||l_2^*|\|\tilde{\varphi}\|_{\infty}+\varepsilon(\|A_1\|_{\infty}+\varepsilon\|A_2\|_{\infty})\|\tilde\varphi_{3}\|_{\infty}\cr 
    \le & B+|\Omega||l_2^*|\|\tilde{\varphi}\|_{\infty} +\varepsilon B\|\tilde{\varphi}_3\|_{\infty}\cr 
    \le & B+|\Omega||l_2^*|(\|\tilde{\varphi}_0\|_{\infty}+\varepsilon\|\tilde{\varphi}_1\|_{\infty}+\varepsilon^2\|\tilde{\varphi}_2\|_{\infty}+\varepsilon^3\|\tilde{\varphi}_3\|_{\infty}) +\varepsilon B\|\tilde{\varphi}_3\|_{\infty}\cr 
    \le & B+|\Omega||l_2^*|(\|\tilde{\varphi}_0\|_{\infty}+\|\tilde{\varphi}_1\|_{\infty}+\|\tilde{\varphi}_2\|_{\infty})+\varepsilon(B+\varepsilon^2|\Omega||l_2^*|)\|\tilde{\varphi}_3\|_{\infty}\cr
    \le & (1+|\Omega||l_2^*|)B+\varepsilon M_{q,k,k}(B+|\Omega||l_2^*|)(1+|l_3^*(\varepsilon)|).
\end{align*}
Equivalently,
\begin{equation}\label{proof-prop-app-eq3}
    |l_3^*|\Big(\int_{\Omega}\beta_{k,h,\varepsilon}\tilde{\varphi}-\varepsilon M_{q,k,h}(B+|\Omega||l_2^*|)\Big)\le (1+|\Omega||l_2^*|)B+\varepsilon M_{q,k,h}(B+|\Omega||l_2^*|)\quad 0<\varepsilon<\varepsilon_{q,h,k}.
\end{equation}
Next, observing that 
$$
\int_{\Omega}\beta_{k,h,\varepsilon}\tilde{\varphi}\ge (k-\varepsilon\|h\|_{\infty})\int_{\Omega}\tilde{\varphi}=k-\varepsilon\|h\|_{\infty}\quad 0<\varepsilon<\frac{k}{\|h\|_{\infty}},
$$
it follows from \eqref{proof-prop-app-eq3} that 
$$
\limsup_{\varepsilon\to 0}|l_3^*(\varepsilon)|\le \frac{(1+|\Omega||l_2^*|)B}{k},
$$
which in view of \eqref{proof-prop-app-eq2} yields also that 
$$
\limsup_{\varepsilon\to 0}\|\tilde{\varphi}_3(\cdot;\varepsilon)\|_{C^1(\overline{\Omega})}<\infty
$$
since $W^{2,q}(\Omega)$ is continuously embedded in $C^1(\overline{\Omega})$.



    
\end{proof}



\begin{prop}\label{appen-prop3} Fix 
$k>0$ and $d_I>0$ and suppose that $h=c_m\phi_m$ for some $m\ge 1$. For every $0<\varepsilon\ll 1$, let $\gamma_{k,h,\varepsilon}=\beta_{k,h,\varepsilon}^2$ where $\beta_{h,k,\varepsilon}$ is defined by \eqref{appen-3}. Then $\mathcal{R}(d_I)=\frac{1}{l^*(\varepsilon)}$. Moreover, for sufficiently small values of $\varepsilon$, it holds that 
\begin{equation}\label{TTH1}
     \frac{|\Omega|}{\mathcal{R}(d_I)}-\int_{\Omega}\frac{\gamma_{k,h,\varepsilon}}{\beta_{k,h,\varepsilon}}\begin{cases}
         >0 & \text{if}\quad d_I\lambda_m>k^2,\cr
         <0 & \text{if}\quad d_I\lambda_m<k^2.
     \end{cases}
\end{equation}
and 
\begin{equation}\label{TTH2}
   |\Omega|\int_{\Omega}\beta_{h,k,\varepsilon}\varphi_{d_I}^3-\Big(\int_{\Omega}\varphi_{d_I}\Big)\Big(\int_{\Omega}\beta_{k,h,\varepsilon}\varphi_{d_I}^2\Big)\begin{cases}
     >0 & \text{if}\quad d_I\lambda_m<2k^2,\cr 
     <0 & \text{if}\quad d_I\lambda_m>2k^2.
   \end{cases}
\end{equation}
Therefore 
\begin{enumerate}
    \item[\rm (i)] If $k^2<d_I\lambda_m<2k^2$, then $ \frac{|\Omega|}{\mathcal{R}(d_I)}>\int_{\Omega}\frac{\gamma_{k,h,\varepsilon}}{\beta_{k,h,\varepsilon}}$ and $|\Omega|\int_{\Omega}\beta_{h,k,\varepsilon}\varphi_{d_I}^3>\Big(\int_{\Omega}\varphi_{d_I}\Big)\Big(\int_{\Omega}\beta_{k,h,\varepsilon}\varphi_{d_I}^2\Big) $  for $0<\varepsilon\ll 1$.
    \item[\rm (ii)] If $2k^2<d_I\lambda_m$, then $|\Omega|\int_{\Omega}\beta_{h,k,\varepsilon}\varphi_{d_I}^3<\Big(\int_{\Omega}\varphi_{d_I}\Big)\Big(\int_{\Omega}\beta_{k,h,\varepsilon}\varphi_{d_I}^2\Big)$  for $0<\varepsilon\ll 1$.
\end{enumerate}
    
\end{prop}
\begin{proof}
    It is easy to see from \eqref{R-star-pde} and \eqref{prop-eq2} that $l^*(\varepsilon)=\frac{1}{\mathcal{R}(d_I)}$.  Now, by Proposition, $l^*(\varepsilon)$ can be written as 
    \begin{equation}\label{PLP1}
        l^*(\varepsilon)=k+\frac{\varepsilon}{|\Omega|}\int_{\Omega}h+\varepsilon^2l_2^*+\varepsilon^3l_2^*(\varepsilon)\quad 0<\varepsilon\ll 1,
    \end{equation}
    where $l_2^*$ is given by \eqref{l-star-2-eq} and $l_3(\varepsilon)$ satisfies \eqref{prop-eq4}.   Then 
    \begin{align*}
        \frac{|\Omega|}{\mathcal{R}(d_I)}=|\Omega|l^*(\varepsilon)=k|\Omega|+{\varepsilon}\int_{\Omega}h+\varepsilon^2|\Omega|l_2^*+\varepsilon^3|\Omega|l_2^*(\varepsilon).
    \end{align*}
    As a result, since 
    $$
    \int_{\Omega}\frac{\gamma_{k,h,\varepsilon}}{\beta_{k,h,\varepsilon}}=\int_{\Omega}\beta_{k,h,\varepsilon}=\int_{\Omega}(k+\varepsilon h)=k|\Omega|+\varepsilon\int_{\Omega}h,
    $$
    we can then employ  proposition \eqref{appen-prop2} to get
    $$
    \frac{1}{|\Omega|\varepsilon^2}\Big(\frac{|\Omega|}{\mathcal{R}(d_I)}-\int_{\Omega}\frac{\gamma_{k,h,\varepsilon}}{\beta_{k,h,\varepsilon}}\Big)=l_2^*-\varepsilon l_{3}^*(\varepsilon)=\frac{1}{k|\Omega|}\Big(1-\frac{k^2}{d_I\lambda_m}\Big)\int_{\Omega}h^2-\varepsilon l^*_3(\varepsilon)\quad 0<\varepsilon\ll 1.
    $$
    Thus \eqref{TTH1} holds since $\varepsilon l^*_3(\varepsilon)\to 0 $ as $\varepsilon\to 0$ by \eqref{prop-eq4}. Next, we prove that \eqref{TTH2} holds.

\quad    To this end, set $\tilde{\varphi}_{d_I}=\varphi_{d_I}/(\int_{\Omega}\varphi_{d_I})$. Then 

\begin{equation}\label{TTH3}
|\Omega|\int_{\Omega}\beta_{h,k,\varepsilon}\varphi_{d_I}^3-\Big(\int_{\Omega}\varphi_{d_I}\Big)\Big(\int_{\Omega}\beta_{k,h,\varepsilon}\varphi_{d_I}^2\Big)=\left(|\Omega|\int_{\Omega}\beta_{k,h,\varepsilon}\tilde{\varphi}_{d_I}^3-\int_{\Omega}\beta_{k,h,\varepsilon}\tilde{\varphi}_{d_I}^2\right)\Big(\int_{\Omega}\varphi_{d_I}\Big)^3.
\end{equation}
Now, observe from \eqref{R-star-pde}, and the fact $l^*(\varepsilon)=\frac{1}{\mathcal{R}(d_I)}$ and $\int_{\Omega}\tilde{\varphi}_{d_I}=1$, that $\tilde{\varphi}_{d_I}$ is the unique solution of \eqref{prop-eq2}.  Then, by proposition \ref{appen-prop1}, $\tilde{\varphi}_{d_I}$ can be written as
\begin{equation}\label{TTH4}
    \tilde{\varphi}_{d_I}=\frac{1}{|\Omega|}+\varepsilon\tilde{\varphi}_1+\varepsilon^2\tilde{\varphi}_2+\varepsilon^3\tilde{\varphi}_3(\cdot;\varepsilon)\quad 0<\varepsilon\ll 1,
\end{equation}
where $\tilde{\varphi}_{3}$ satisfies \eqref{prop-eq4}. Hence 
\begin{align}\label{TTH7}
|\Omega|\int_{\Omega}\beta_{k,h,\varepsilon}\tilde{\varphi}_{d_I}^3-\int_{\Omega}\beta_{k,h,\varepsilon}\tilde{\varphi}_{d_I}^2=&\int_{\Omega}\beta_{k,h,\varepsilon}\tilde{\varphi}_{d_I}^2(|\Omega|\tilde{\varphi}-1)=\varepsilon|\Omega|\int_{\Omega}\beta_{k,h,\varepsilon}\tilde{\varphi}_{d_I}^2(\tilde{\varphi}_1+\varepsilon\tilde{\varphi}_2+\varepsilon^2\tilde{\varphi}_3)
\end{align}
For convenience, we set $P=\tilde{\varphi}_2+\varepsilon\tilde{\varphi}_3$ and $Q=\tilde{\varphi}_1+\varepsilon P$. So, $\tilde{\varphi}_{d_I}=\frac{1}{|\Omega|}+\varepsilon Q$ and 
\begin{align*}
    \beta_{k,h,\varepsilon}\tilde{\varphi}^2_{d_I}Q=&(k+\varepsilon h)\tilde{\varphi}^2_{d_I}Q=k\tilde{\varphi}^2_{d_I}Q+\varepsilon h \tilde{\varphi}^2_{d_I}Q 
    = k\tilde{\varphi}^2_{d_I}\tilde{\varphi}_1+\varepsilon( k\tilde{\varphi}^2_{d_I} P+ h \tilde{\varphi}^2_{d_I}Q)\cr  
    =& \frac{k}{|\Omega|^2}\tilde{\varphi}_1+\frac{2 k \varepsilon}{|\Omega|}Q\tilde{\varphi}_1+k\varepsilon^2Q^2\tilde{\varphi}_1+\varepsilon\Big(k \tilde{\varphi}^2_{d_I} P +h \tilde{\varphi}^2_{d_I}Q\Big)\cr
    %=& \frac{k}{|\Omega|^2}\tilde{\varphi}_1+\frac{2 k \varepsilon}{|\Omega|}\tilde{\varphi}_1^2+\frac{2k\varepsilon^2}{|\Omega|}P\tilde{\varphi}_1 +k\varepsilon^2Q^2\tilde{\varphi}_1+\varepsilon\Big(k \tilde{\varphi}^2_{d_I} P +h \tilde{\varphi}^2_{d_I}Q\Big)\cr
    =& \frac{k}{|\Omega|^2}\tilde{\varphi}_1+\varepsilon\Big(\frac{2 k }{|\Omega|}\tilde{\varphi}_1^2+k \tilde{\varphi}^2_{d_I} P +h \tilde{\varphi}^2_{d_I}Q\Big)+k\varepsilon^2\Big(\frac{2}{|\Omega|}P +Q^2\Big)\tilde{\varphi}_1 
\end{align*}
Observing that
\begin{align*}
    k \tilde{\varphi}^2_{d_I} P%= k\Big(\frac{1}{|\Omega|^2}+\frac{2\varepsilon}{|\Omega|}Q+\varepsilon^2Q^2 \Big)P
    =\frac{k P}{|\Omega|^2}+k\varepsilon\Big(\frac{2}{|\Omega|}+\varepsilon Q\Big)QP 
    =\frac{k}{|\Omega|^2}\tilde{\varphi}_2+k\varepsilon\Big(\frac{\tilde{\varphi}_3}{|\Omega|^2}+\Big(\frac{2}{|\Omega|}+\varepsilon Q\Big)QP\Big)
    \end{align*}
    and 
    \begin{align*}
    h \tilde{\varphi}^2_{d_I} Q= h\Big(\frac{1}{|\Omega|^2}+\frac{2\varepsilon}{|\Omega|}Q+\varepsilon^2Q^2 \Big)Q
    %=&\frac{1}{|\Omega|^2}hQ+\varepsilon h\Big(\frac{2}{|\Omega|}Q+Q^2\Big)Q\cr 
    =\frac{1}{|\Omega|^2}h\tilde{\varphi}_1+\varepsilon h\Big(\frac{P}{|\Omega|^2}+\Big(\frac{2}{|\Omega|}+\varepsilon Q\Big)Q^2\Big)
    \end{align*}
then
\begin{align*}
    \beta_{k,h,\varepsilon}\tilde{\varphi}^2_{d_I}Q
    %=&\frac{k}{|\Omega|^2}\tilde{\varphi}_1+\frac{2 k \varepsilon}{|\Omega|}\tilde{\varphi}_1^2+k\varepsilon^2\Big(\frac{2}{|\Omega|}P +Q^2\Big)\tilde{\varphi}_1+\varepsilon\Big(k \tilde{\varphi}^2_{d_I} P +h \tilde{\varphi}^2_{d_I}Q\Big)\cr 
    =\frac{k}{|\Omega|^2}\tilde{\varphi}_1+\frac{\varepsilon}{|\Omega|^2}\Big({2k}{|\Omega|}\tilde{\varphi}_1^2+k\tilde{\varphi}_2+h\tilde{\varphi}_1\Big)+\varepsilon^2\tilde{\mathbb{H}}(\cdot;\varepsilon),
\end{align*}
where $$\tilde{\mathbb{H}}(\cdot;\varepsilon):=k\Big(\frac{2}{|\Omega|}P +Q^2\Big)\tilde{\varphi}_1+k\Big(\frac{\tilde{\varphi}_3}{|\Omega|^2}+\Big(\frac{2}{|\Omega|}+\varepsilon Q\Big)QP\Big)+h\Big(\frac{P}{|\Omega|^2}+\Big(\frac{2}{|\Omega|}+\varepsilon Q\Big)Q^2\Big).$$
As a result, since $\int_{\Omega}\tilde{\varphi}_i=0$, $i=1,2$, we get
\begin{equation}\label{TTH6}
    \frac{1}{\varepsilon}\int_{\Omega}\beta_{k,h,\varepsilon}\tilde{\varphi}^2_{d_I}Q=\frac{1}{|\Omega|^2}\Big(2k|\Omega|\int_{\Omega}\tilde{\varphi}_1^2+\int_{\Omega}h\tilde{\varphi}_1\Big)+\varepsilon\int_{\Omega}\tilde{\mathbb{H}}(\cdot;\varepsilon)\quad 0<\varepsilon\ll 1.
\end{equation}
Now, thanks to proposition \ref{appen-prop2}
$$
2k|\Omega|\int_{\Omega}\tilde{\varphi}_1^2+\int_{\Omega}h\tilde{\varphi}_1=\Big(\frac{2k^2}{d_I\lambda_m}-1\Big)\frac{k}{|\Omega|d_I\lambda_m}\int_{\Omega}h^2.
$$
Therefore, since by \eqref{prop-eq4} $\varepsilon\|\tilde{\mathbb{H}}(\cdot;\varepsilon)\|_{\infty}\to 0$ as $\varepsilon\to 0$, we deduce from \eqref{TTH3}, \eqref{TTH7}, and \eqref{TTH6}  that \eqref{TTH2} holds.

    
    
\end{proof}




\begin{thebibliography}{9}



\bibitem{Allen2008}L.J.S. Allen, B.M. Bolker, Y. Lou, A.L. Nevai, Asymptotic profiles of the steady states for an SIS epidemic reaction-diffusion model, {\it Disc. Cont. Dyn. Syst.} {\bf 21} (2008), 1-20.




\bibitem{CastellanoSalako2021} K. Castellano, R. B. Salako, On the effect of lowering population's movement to control the spread of infectious disease  {\it J. Diff. Equat.}, {\bf316} (2022), 1-27. 


\bibitem{Cui_Lou2016}R. Cui, Y. Lou, A spatial SIS model in advective heterogeneous environments, {\it J. Diff. Equat.}, {\bf 261}
(2016), 3305-3343.



\bibitem{Cui2017} R. Cui, K.-Y. Lam, Y. Lou, Dynamics and asymptotic profiles of steady states of an epidemic model in advective environments, {\it J. Diff. Equat.}, {\bf263} (2017), 2343-2373.



\bibitem{DengWu2016}K. Deng, Y. Wu, Dynamics of a susceptible-infected-susceptible epidemic reaction-diffusion model, {\it  Proc. Roy.
Soc. Edinburgh Sect. A}, {\bf 146} (2016), 929-946.

\bibitem{Evans} L. C. Evans, Partial Differential Equations: Second Edition (Graduate Studies in Mathematics) 2nd Edition

\bibitem{GKLZ2015}J. Ge, K.I. Kim, Z. Lin, H. Zhu, A SIS reaction-diffusion-advection model in a low-risk and high-risk domain,
{\it J. Diff. Equat.}, {\bf259} (2015), 5486-5509.



\bibitem{DeJong1995} M. C. M. de Jong, et al., How does transmission of infection depend on population size?, in Epidemic Models: Their Structure and Relation to Data, {\it Cambridge University Press}, 1995, pp.84–89.

\bibitem{LP2022} H. Li, R. Peng, An SIS epidemic model with mass action infection mechanism in a patchy environment, Studies in Applied Mathematics, {\bf 150} (2022), 650 - 704.



\bibitem{LSS2023}  Y. Lou, R. B. Salako, P. Song, Human Mobility and Disease Prevalence, J.  Math. Biol., {\bf 87}, no. 1, (2023), 1-32.

%\bibitem{LS2023_2}  Y. Lou, R. B. Salako, Mathematical analysis on the coexistence of strains  in some reaction-diffusion systems, Journal of Differential Equations, {\bf 370}, (2023), 424-469.

\bibitem{LouSalako2021} Y. Lou, R. B. Salako, Control Strategy for multiple strains epidemic model, {\it Bulletin of Mathematical Biology}, {\bf 84} 10 (2022), p.1-47.

\bibitem{MCH2001} H. McCallum, N. Barlow, J. Hone, How should pathogen transmission be modelled?, {\it Trends Ecol. Evol.} {\bf16} (2001) 295–300.



\bibitem{Peng2009a} R. Peng, Asymptotic profiles of the positive steady state for an SIS epidemic reaction-diffusion model. Part I, {\it J. Diff. Equat.}, {\bf 247} (2009), 1096-1119.

\bibitem{Peng2009b} R. Peng, S. Liu, Global stability of the steady states of an SIS epidemic reaction-diffusion model, {\it Nonlinear Anal.} {\bf 71} (2009), 239-247.

\bibitem{Peng_Shi2008} R. Peng, J. Shi, M. Wang, On stationary patterns of a reaction-diffusion model with autocatalysis and saturation law, {\it Nonlinearity}, 21 (2008), 1471-1488.

\bibitem{Peng_Yi2013} R. Peng, F. Yi, Asymptotic profile of the positive steady state for an SIS epidemic reaction-diffusion model:
Effects of epidemic risk and population movement, {\it Phys. D}, {\bf 259} (2013), 8-25.

\bibitem{Peng_Zhao} R. Peng, X. Zhao, A reaction-diffusion SIS epidemic model in a time-periodic environment, Nonlinearity, 25 (2012), 1451-1471.

\bibitem{Salako2023_1} R. B. Salako, Impact of environmental heterogeneity,  population size and movement on the persistence of a two-strain infectious disease, Journal of Mathematical Biology {\bf 86},  1 (2023), 1-36.

\bibitem{TW2023}Y. Tao, M. Winkler, Analysis of a chemotaxis-SIS epidemic model with unbounded infection force, Nonlinear Analysis: Real World Applications, {\bf  71} (2023), 103820

%
\bibitem{Wu_Zou2016} Y. Wu and Z. Zou, Asymptotic profiles of steady states for a diffusive SIS epidemic model with mass action infection mechanism, {\it J. Diff. Equat.} {\bf 261}(2016) 4424–4447. 

\bibitem{Wen2018}X. Wen, J. Ji, B. Li, Asymptotic profiles of the endemic equilibrium to a diffusive SIS epidemic model with mass action infection mechanism, {\it J. Math. Anal. Appl.} {\bf 458} (2018), 715-729.




\end{thebibliography}

\end{document}



