\documentclass{article}
\usepackage[utf8]{inputenc}

 \parindent0mm
\textwidth155mm
\textheight200mm
\oddsidemargin0mm
\evensidemargin0mm

\usepackage{latexsym,amsfonts,amssymb,amsmath,amsthm}

\usepackage{cite}

\usepackage{booktabs} % For pretty tables
%\usepackage{caption} % For caption spacing   
%\usepackage{subcaption} % For sub-figures
\usepackage{graphicx}
\usepackage{pgfplots}
\usepackage[all]{nowidow}
\usepackage[utf8]{inputenc}
%\usepackage{tikz}
%\usetikzlibrary{er,positioning,bayesnet}
\usepackage{tikz}
\usetikzlibrary{automata}
\usepackage{multicol,comment}
\usepackage{algpseudocode,algorithm,algorithmicx}
  \usepackage[frozencache=true,cachedir=minted-cache]{minted} 
\usepackage{hyperref}
\usepackage[inline]{enumitem} % Horizontal lists
% Used for displaying a sample figure. If possible, figure files should
% be included in EPS format.
%
% If you use the hyperref package, please uncomment the following line
% to display URLs in blue roman font according to Springer's eBook style:
% \renewcommand\UrlFont{\color{blue}\rmfamily}



\newcommand{\card}[1]{\left\vert{#1}\right\vert}
\newcommand*\Let[2]{\State #1 $\gets$ #2}
\definecolor{blue}{HTML}{1F77B4}
\definecolor{orange}{HTML}{FF7F0E}
\definecolor{green}{HTML}{2CA02C}

\pgfplotsset{compat=1.14}

\renewcommand{\topfraction}{0.85}
\renewcommand{\bottomfraction}{0.85}
\renewcommand{\textfraction}{0.15}
\renewcommand{\floatpagefraction}{0.8}
\renewcommand{\textfraction}{0.1}
\setlength{\floatsep}{3pt plus 1pt minus 1pt}
\setlength{\textfloatsep}{3pt plus 1pt minus 1pt}
\setlength{\intextsep}{3pt plus 1pt minus 1pt}
\setlength{\abovecaptionskip}{2pt plus 1pt minus 1pt}

%%%%%%%%%%%%%%%%%%%%%%%%%%%%%%
%%%%%%%%%



\newtheorem{tm}{Theorem}[section]
\newtheorem{prop}[tm]{Proposition}
\newtheorem{defin}[tm]{Definition}
\newtheorem{coro}[tm]{Corollary}
\newtheorem{lem}[tm]{Lemma}
\newtheorem{assumption}[tm]{Assumption}
\newtheorem{rk}[tm]{Remark}
\newtheorem{nota}[tm]{Notation}
\numberwithin{equation}{section}
\numberwithin{tm}{section}

%%%%%%%%%%%






\newcommand{\eqd}{\sim}
\def\p{\partial}
\def\R{{\mathbb R}}
\def\N{{\mathbb N}}
\def\Q{{\mathbb Q}}
\def\C{{\mathbb C}}
\def\l{{\langle}}
\def\r{\rangle}
\def\t{\tau}
\def\k{\kappa}
\def\a{\alpha}
\def\la{\lambda}
\def\De{\Delta}
\def\de{\delta}
\def\ga{\gamma}
\def\Ga{\Gamma}
\def\ep{\varepsilon}
\def\si{\sigma}
\def\Re {{\rm Re}\,}
\def\Im {{\rm Im}\,}
\def\E{{\mathbb E}}
\def\P{{\mathbb P}}
\def\Z{{\mathbb Z}}
\def\D{{\mathbb D}}
\newcommand{\ceil}[1]{\lceil{#1}\rceil}



\title{ Multiplicity of endemic equilibria for a diffusive SIS epidemic model with mass-action transmission mechanism }
\author{Keoni Castellano\footnote{castek1@unlv.nevada.edu} \quad  and \quad  Rachidi B. Salako \footnote{rachidi.salako@unlv.edu}  
\\
\\
{\small Department of  Mathematical Sciences,  University of Nevada Las Vegas,}\\
{\small Las Vegas, NV 89154, USA}}
\date{}

\begin{document}

\maketitle

\begin{abstract} We study a diffusive   SIS epidemic model with mass-action transmission mechanism and show, under appropriate assumptions on the parameters, the existence of multiple endemic equilibria when the basic reproduction number, $\mathcal{R}_0$, is either less than one or greater than one.  Previous studies have left open the question of extinction of disease or persistence when $\mathcal{R}_0<1$.  Our results settle completely this open question. Results on the nonexistence/existence and uniqueness of endemic equilibrium are also presented.

    
\end{abstract}

\noindent{\bf Keywords}: Infectious Disease Models; Reaction-Diffusion Systems;
Asymptotic Behavior.
\smallskip

{
\noindent{\bf 2010 Mathematics Subject Classification}: 92D25, 35B40, 35K57}

\section{Introduction}

 \quad Infectious diseases remain a leading cause of deaths around the word.  Indeed, as of May 2023, according to the World Health Organization (WHO)  coronavirus (COVID-19) dashboard, SARS-CoV-2 has infected more than seven hundred million  people  and   claimed more than six million deaths worldwide. This fast spread of the SARS-CoV-2 virus is partly due to globalization which has  made the world more connected. As a result, to better predict the dynamics of infectious diseases and  develop effective and adequate control strategies, researchers have incorporated population movement and spatial heterogeneity into  epidemic models.

\quad In 2008, to study the effect of population movement and environmental heterogeneity on disease persistence, Allen et al. \cite{Allen2008} proposed and studied the mathematical model 
\begin{equation}\label{e1-prime}
    \begin{cases}
    S_t=d_S\Delta S+\gamma I -\beta \frac{SI}{S+I} & x\in\Omega,\ t>0,\cr
    I_{t}=d_I\Delta I+\beta \frac{SI}{S+I}-\gamma I& x\in\Omega,\ t>0,\cr
     0=\partial_{\vec{n}}S=\partial_{\vec{n}}I & x\in\partial\Omega,\ t>0,\cr
    N=\int_{\Omega}(S+I),
    \end{cases}
\end{equation}
where $\Omega$ is a bounded domain in $\mathbb{R}^n$ ($n\ge 1$) with a smooth boundary $\partial\Omega$; $\vec{n}$ denotes the outward unit normal vector at $\partial\Omega$; $S(x,t)$ and $I(x,t)$ are the local densities of the susceptible and infected populations, respectively; $d_S$ and $d_I$ are positive constants and represent the diffusion rates of the susceptible and infected populations, respectively; $\beta$ and $\gamma$ are positive, H\"older continuous functions on $\overline{\Omega}$, and account for the disease transmission and recovery rates, respectively; and the positive number $N$ is the total size of the population. The authors of \cite{Allen2008} gave a variational formula for the basic reproduction number of \eqref{e1-prime} (see formula \ref{R-star-eq} below). They then established that the disease will be eventually eradicated if the basic reproduction number is less than one while \eqref{e1-prime} has a unique endemic equilibrium (EE)  (that is a time independent solution) if the basic reproduction number is bigger than one. They further investigated the asymptotic limits of the EE as   $d_S$  approaches zero. The asymptotic profiles of the EE  as the diffusion rates of the infected population becomes very small or either of the diffusion rates get arbitrarily large are studied in \cite{Peng2009a,Peng_Yi2013}. Partial results on the stability of EE of \eqref{e1-prime} are obtained in \cite{Peng2009b}. 

 \quad Inspired by the above mentioned works, several studies have been devoted to the investigations of diffusive epidemic models (\cite{Cui_Lou2016, Cui2017, MCH2001, LP2022, Peng_Shi2008, DeJong1995, GKLZ2015, Peng_Zhao, TW2023} and the references therein). In particular, Deng and Wu \cite{DengWu2016} studied the epidemic model 

\begin{equation}\label{e1}
    \begin{cases}
    S_t=d_S\Delta S+\gamma I -\beta SI & x\in\Omega,\ t>0,\cr
    I_{t}=d_I\Delta I+\beta SI-\gamma I& x\in\Omega,\ t>0,\cr
     0=\partial_{\vec{n}}S=\partial_{\vec{n}}I & x\in\partial\Omega,\ t>0,\cr
    N=\int_{\Omega}(S+I).
    \end{cases}
\end{equation}
The variables in \eqref{e1} have the same meanings as those in \eqref{e1-prime}. The authors of \cite{DengWu2016} also defined the basic reproduction number of \eqref{e1} (see \eqref{R-0-def} below) and established the existence of EE when this reproduction number is bigger than one. They also obtained some partial results on the nonexistence of EE of \eqref{e1} when the basic reproduction number is less than one. The works \cite{CastellanoSalako2021, Wu_Zou2016, Wen2018} further studied the asymptotic profiles of EE of \eqref{e1} as the diffusion rates of the population get small or large. However, the question of existence of the EE solutions of \eqref{e1} when its basic reproduction is less than one was left open by all these works. In particular, it is unclear whether \eqref{e1} may have multiple EE solutions. Our aim in this study is to tackle these questions. In particular, our results show that there are possible range of parameters of the model \eqref{e1} for which there exist more than one EE solution. In particular, Theorem \ref{T2} shows that \eqref{e1} may have at least two EE solutions when its basic reproduction number is less than one. Hence, these results highlight important differences on the structure of EE solutions of the epidemic models \eqref{e1-prime} and \eqref{e1}.



\quad In section \ref{main-results}, we state our main results. Section \ref{Preliminaries} contains some preliminaries  essential for the proofs of our main results, which are presented in section \ref{proofs}










   
   
   
   






\section{Main Results}\label{main-results}


\quad  We say that   $(S,I)$ is an equilibrium solution of \eqref{e1} if it is a classical solution of 
\begin{equation}\label{e2}
    \begin{cases}
  0=d_S\Delta S+\gamma I -\beta SI & x\in\Omega,\cr
    0=d_I\Delta I+\beta SI-\gamma I& x\in\Omega,\cr
     0=\partial_{\vec{n}}S=\partial_{\vec{n}}I & x\in\partial\Omega,\cr
    N=\int_{\Omega}(S+I).
    \end{cases}
\end{equation}
 A solution of system \eqref{e2} of the form $(S,0)$ is called a disease free equilibrium (DFE). It is easy to see that the unique DFE of \eqref{e1} is $(\frac{N}{|\Omega|},0)$.  
 An equilibrium solution for which $I>0$  is called an endemic equilibrium (EE).


\quad For every $d_I>0$, define 
\begin{equation}\label{R-star-eq}
    \mathcal{R}(d_I)=\sup_{\varphi\in W^{1,2}(\Omega)\setminus\{0\}}\frac{\int_{\Omega}\beta\varphi^2}{\int_{\Omega}[d|\nabla \varphi|^2+\gamma\varphi^2]}.
\end{equation}
It is well known (see \cite{Allen2008}) that the supremum in \eqref{R-star-eq} is achieved and there is a unique positive function $\varphi_{d_I}\in C^2(\overline{\Omega})$ with $\|\varphi_{d_I}\|_{L^2(\Omega)}=1$ satisfying 
\begin{equation}\label{R-star-pde}
    \begin{cases}
    0=d_I\Delta \varphi-\gamma\varphi+\frac{1}{\mathcal{R}(d_I,)}\beta\varphi & x\in\Omega,\cr 
    0=\partial_{\vec{n}}\varphi & x\in\partial\Omega.
    \end{cases}
\end{equation}
Furthermore, any solution of \eqref{R-star-pde} is spanned by $\varphi_{d_I}$.  We recall some important properties of $\mathcal{R}(d_I)$ with respect to $d_I$ in  Lemma \ref{lem1}. Thanks to \cite{Allen2008}, the quantity $\mathcal{R}(d_I)$ is the basic reproduction number for \eqref{e1-prime}. Moreover, when the  assumption {\bf (A)},

\medskip

\noindent{\bf (A)} The function $\frac{\beta}{\gamma}$ is not constant,

\medskip

holds, it follows from Lemma \ref{lem1} that $\mathcal{R}(d_I)$ is strictly decreasing in $d_I$, and has an inverse function, which we denote by $\mathcal{R}^{-1}$. Throughout the remainder of this work, we shall always suppose that assumption {\bf (A)} holds. It follows from \cite{DengWu2016} that  the quantity  $\mathcal{R}_0(N,d_I)$, defined by 
\begin{equation}\label{R-0-def}
    \mathcal{R}_0(N,d_I)=\frac{N}{|\Omega|}\mathcal{R}(d_I),
\end{equation}
is the basic reproduction number of \eqref{e1}. It is clear from \eqref{R-0-def} that $\mathcal{R}_0(N,d_I)$ is strictly increasing with respect to $N$.   Hence, at a glance, the dependence of $\mathcal{R}_0(N,d_I)$ on the total population size $N$ yields an important difference between the prediction on the persistence of the infectious disease based on the mathematical models  \eqref{e1} and \eqref{e1-prime}. % where the transmission mechanism follows the mass-action mechanics, and the mathematical model \eqref{e1-prime}, where the transmission mechanism is density dependent. 
 For instance, if $\frac{N}{|\Omega|}<1$ (resp. $\frac{N}{|\Omega|}>1$), then the epidemic model \eqref{e1} predicts a lower (resp. higher) basic reproduction number compared to that predicted by the epidemic model \eqref{e1-prime}. In fact,  the dependence of $\mathcal{R}(N,d_I)$ on $N$  also induces  fundamental facts on the dynamics of classical solutions of \eqref{e1} which do not hold for classical solutions of \eqref{e1-prime}.  Indeed, regarding the existence and uniqueness of EE solution of \eqref{e1-prime}, Allen et al. \cite{Allen2008} established the following result.

\begin{prop}\cite{Allen2008, Cui2017} \label{Proposition1}\begin{itemize} \item[\rm (i)]For every $d_S>0$ and $d_I>0$, the diffusive epidemic model \eqref{e1-prime} has a (unique) EE if and only if $\mathcal{R}(d_I)>1$.

\item[\rm (ii)] If $\mathcal{R}(d_I)\le 1$, then for every $d_S>0$, the DFE is globally stable for classical solutions of \eqref{e1-prime}.

    \end{itemize}
\end{prop}

\quad This result shows that the existence of EE solutions of \eqref{e1-prime} is completely determined by the sign of $\mathcal{R}(d_I)-1$ and is independent of the diffusion rate of the susceptible population and the total size of the population. Moreover, when the basic reproduction number is less than or equal to zero, the mathematical model \eqref{e1-prime} predicts that the disease can be controlled and eradicated in the long run.  Since the mass-action transmission mechanism adopted in modeling \eqref{e1} and the density dependent transmission mechanism (also known as standard incidence) are the most commonly used transmission mechanisms in the literature for modeling infectious disease, it becomes a natural question to know whether the results of Proposition   \ref{Proposition1} on \eqref{e1-prime} extend to system \eqref{e1}. Mainly: (i) Does it also hold that the epidemic model \eqref{e1}  has a (unique) EE solution if and only if its basic reproduction number is bigger than one? (ii) Does the epidemic model \eqref{e1} predict the eventual extinction of the disease when its basic reproduction number is less than or equal to one? Surprisingly, as seen from Theorem \ref{T1} below,  the answers to these two question could be negative. 

\quad For convenience, let us recall the following result on the existence/nonnexistence and uniqueness of EE solution of \eqref{e1} from previous studies.

\begin{prop}\cite{DengWu2016}\label{Proposition2}
\begin{itemize}
    %\item[\rm (i)] If $\mathcal{R}_0(N,d_I)>1$, then \eqref{e1} has at least one EE for every $d_S>0$.
     \item[\rm (i)] If $d_S\ge d_I$, then \eqref{e1} has a (unique) EE solution if and only if $\mathcal{R}_0(N,d_I)>1$.
     \item[\rm (ii)] If $d_S<d_I$, then  \eqref{e1} has no EE if $\mathcal{R}(N,d_I)\le \frac{d_S}{d_I}$ and has at least one EE if $\mathcal{R}(N,d_I)>1$.
\end{itemize}
    
\end{prop}

\quad Thanks to Proposition \ref{Proposition2}, when $d_S\ge d_I$, \eqref{e1} has a (unique) EE if and only if $\mathcal{R}(N,d_I)>1$. However, when $d_S<d_I$, it is not clear whether \eqref{e1} has an EE solution if $\frac{d_S}{d_I}<\mathcal{R}(N,d_I)\le 1$. %Moreover, it is unclear whether the lower bound $d_I$ for $d_S$ in Proposition \ref{Proposition2}-{\rm (i)} is sharp.
To address these questions, we first establish  the following result.



\begin{tm}\label{T0}
 Fix $d_I>0$. There is $0<N_{\rm low}(d_I)\le \min\Big\{ \frac{|\Omega|}{\mathcal{R}(d_I)},\int_{\Omega}\frac{\gamma}{\beta}\Big\}$ such that the following  hold.
\begin{itemize}
    \item[\rm (i)] For every $N>0$ satisfying $N\le N_{\rm low}(d_I)$,  \eqref{e1}  has no EE for every $d_S>0$. 
    
    \item[\rm (ii)] For every $N>0$ satisfying $N>N_{\rm low}(d_I)$,  there is $d(N,d_I)>0$ such that \eqref{e1} has an EE solution $(S_{\rm high}(\cdot;d_S),I_{\rm high}(\cdot;d_S))$ for every $0<d_S<d(N,{d_I})$. Furthermore,  any other EE solution $(S(\cdot;d_S),I(\cdot;d_S))$ of \eqref{e1}, if exists, must satisfy
    \begin{equation}\label{T0-eq1}
       I(x;d_S)<I_{\rm high}(x;d_S) \quad \forall\ x\in\Omega. 
    \end{equation}
    
    \item[\rm (iii)]If $N_{\rm low}(d_I)<\frac{|\Omega|}{\mathcal{R}(d_I)}$, then for every $N_{\rm low}(d_I)<N<\frac{|\Omega|}{\mathcal{R}(d_I)}$ and $0<d_S<d(N,{d_I})$, \eqref{e1} has a   EE solution $(S_{\rm low}(\cdot;d_S),I_{\rm low}(\cdot;d_S))$ satisfying 
    \begin{equation}\label{T0-eq2}
        I_{\rm low}(x;d_S)<I_{\rm high}(x;d_S) \quad \forall\ x\in\overline{\Omega},
    \end{equation}
    such that any other EE solution $(S(\cdot;d_S),I(\cdot;d_S))$ of \eqref{e1}, if exists, must satisfy
    \begin{equation}\label{T0-eq3}
        I_{\rm low}(x;d_S)<I(x;d_S)\quad \forall\ x\in\Omega. 
    \end{equation}
    Furthermore, the following hold.
    \begin{itemize}
        \item[\rm (iii-1)] If $N<\int_{\Omega}\frac{\gamma}{\beta}$, then there is a positive constant $C>0$ such that \begin{equation}\label{T0-eq4}
        \frac{d_{S}}{C}\le I_{\rm low}(\cdot)< I_{\rm high}(\cdot)\le Cd_{S}\quad  \forall\ 0<d_S<\frac{d(N,d_I)}{2}
    \end{equation}
     Moreover, as $d_S\to 0^+$,
    \begin{equation}\label{T0-eq5}
        S_{\rm high}\to S_{\rm high}^*(\cdot,d_I):=\frac{N(1-d_Iu^*_{\rm high})}{\int_{\Omega}(1-d_Iu^*_{\rm high})} \quad \text{and}\quad  S_{\rm low}\to S_{\rm low}^*(\cdot,d_I):=\frac{N(1-d_Iu^*_{\rm low})}{\int_{\Omega}(1-d_Iu^*_{\rm low})}
    \end{equation}
    in $C^{1}(\overline{\Omega})$ where $ 0<u^*_{\rm low}<u^*_{\rm high}<\frac{1}{d_I}$ are  classical solutions of the nonlocal elliptic equation
    \begin{equation}\label{T0-eq6}
    \begin{cases}
    0=d_I\Delta u^*+\big( \frac{N\beta}{\int_{\Omega}(1-d_Iu^*)}(1-d_Iu^*)-\gamma\big)u^* & x\in\Omega,\cr
    0=\partial_{n}u^* & x\in\partial\Omega.
    \end{cases}
    \end{equation}

        \item[\rm (iii-2)] If $N>\int_{\Omega}\frac{\gamma}{\beta}$, then 
        \begin{equation}\label{T0-eq7}
        \lim_{d_S\to0^+}\left[\Big\|S_{\rm high}(\cdot;d_S)-\frac{\gamma}{\beta}\Big\|_{\infty}+\Big\|I_{\rm high}(\cdot;d_S)-\frac{1}{|\Omega|}\Big(N-\int_{\Omega}\frac{\gamma}{\beta}\Big)\Big\|_{\infty}\right] = 0
    \end{equation}
    and $(S_{\rm low}(\cdot;d_S),I_{\rm low}(\cdot;d_S))$ satisfies \eqref{T0-eq4} and \eqref{T0-eq5}.
    \end{itemize}
  \end{itemize}  
    
\end{tm}
Let $d_I>0$ and $ N_{\rm low}(d_I) $ be given by Theorem \ref{T1}. It follows from Theorem \ref{T0}-{\rm (i)} and {\rm (ii)} that the quantity  $\mathcal{R}_0(N_{\rm low}(d_I),d_I)$ is a sharp critical number that the basic reproduction number must exceed for the existence of  EE of system \eqref{e1} for some range of the diffusion rate of susceptible population. Note that since $\mathcal{R}_0(N,d_I)$ is strictly increasing in $N$ and $\mathcal{R}_0(\frac{|\Omega|}{\mathcal{R}(d_I)},d_I)=1$, then $\mathcal{R}_0(N_{\rm low}(d_I),d_I)\le 1$. As a result, if $N_{\rm low}(d_I)<\frac{|\Omega|}{\mathcal{R}(d_I)}$, then for every $N\in (N_{\rm low}(d_I),\frac{|\Omega|}{\mathcal{R}(d_I)})$, $\mathcal{R}_0(N,d_I)<1$ and Theorem \ref{T0}-{\rm (iii)}  shows that there is $d(N,d_I)>0$ such that \eqref{e1} has at least two EE solutions for every $0<d_S<d(N,d_I)$. Clearly, we note from Proposition \ref{Proposition2}-{\rm (i)} that $d(N,S)\le d_I$. Hence, it is important to know whether $d(N,{d_I})=d_I$ for some range of the parameters $N$. This question is also related to whether the lower bound $d_I$ for $d_S$ in Proposition \ref{Proposition2}-{\rm (i)} is sharp. In this regards, we have the the following result. 



\begin{tm}\label{T1}For every $d_I>0$, there is a positive number $m^*_{d_I}$ satisfying 
     \begin{equation}\label{T1-main-eq}
         0<m^*_{d_I}\le \min\left\{ 1,\frac{|\Omega|\int_{\Omega}\beta\varphi_{d_I}^3}{\Big[\int_{\Omega}\varphi_{d_I}\Big]\Big[\int_{\Omega}\beta\varphi_{d_I}^2\Big]} \right\},
     \end{equation}
     where $\varphi_{d_I}$ is a positive solution of \eqref{R-star-pde}, such that for every $d_S>d_I(1-m^*_{d_I})$, \eqref{e1} has a (unique) EE solution if and only if $\mathcal{R}_0(N,d_I)>1$. However, if $0<d_S<d_I(1-m_{d_I}^*)_{+}$, then there is $N= N_{d_S,d_I}>0$ such that \eqref{e1} has at least two EE solutions.
    
    
\end{tm}

Fix $d_I>0$. It is clear from \eqref{T1-main-eq} that $m_{d_I}^*>0$, hence $d_I>d_I(1-m^*_{d_I})$. Therefore, Theorem \ref{T2} significantly improves Proposition \ref{Proposition2}-{\rm (i)} by providing a sharp lower bound for the diffusion rate of the susceptible population after which system \eqref{e1} has a (unique) EE solution if and only if its basic reproduction number is greater than one. Moreover, the quantity $d_{I}(1-m_{d_I}^*)$ serves as a sharp upper bound for $d(N,{d_I})$ for every $N>0$.  It is clear from \eqref{T1-main-eq} again that if $|\Omega|\int_{\Omega}\beta\varphi_{d_I}^3<\Big[\int_{\Omega}\varphi_{d_I}\Big]\Big[\int_{\Omega}\beta\varphi_{d_I}^2\Big]$, then  $m_{d_I}^*<1$. However, the later condition seems difficult to check due to the dependence of $\varphi_{d_I}$ on $d_I$. Thanks to Theorems \ref{T0} and \ref{T1}, it is necessary to find out  whether $\mathcal{N}_{\rm low}(d_I)<\frac{|\Omega|}{\mathcal{R}(d_I)}$ is equivalent to $m_{d_I}^*<1$. While we are not  yet able to prove this equivalence, at least  one implication holds as shown in  the following result.

\begin{tm}\label{T2} Fix $d_I>0$ and let $N_{\rm low}(d_I)$ and $m_{d_I}^*$ be given by Theorems \ref{T0} and \ref{T1}, respectively. 
\begin{itemize}
\item[\rm (i)]If $N_{\rm low}(d_I)<\frac{|\Omega|}{\mathcal{R}(d_I)}$, then $m_{d_I}^*<1$.
\item[\rm (ii)] If $\frac{|\Omega|}{\mathcal{R}(d_I)}>\int_{\Omega}\frac{\gamma}{\beta}$, then $N_{\rm low}(d_I)<\frac{|\Omega|}{\mathcal{R}(d_I)}$. If, in addition, it holds that $\frac{|\Omega|\int_{\Omega}\beta\varphi_{d_I}^3}{\Big[\int_{\Omega}\varphi_{d_I}\Big]\Big[\int_{\Omega}\beta\varphi_{d_I}^2\Big]} >1$, then  there is $N>\frac{|\Omega|}{\mathcal{R}(d_I)}$ and $ 0<\tilde{d}(N,d_I)<d_I(1-m_{d_I})$  such that \eqref{e1}  has at least two EE solutions for $0<d_S<\tilde{d}(N,d_I)$.

\item[\rm (iii)] If  $\frac{|\Omega|\int_{\Omega}\beta\varphi_{d_I}^3}{\Big[\int_{\Omega}\varphi_{d_I}\Big]\Big[\int_{\Omega}\beta\varphi_{d_I}^2\Big]} <1$,  then $N_{\rm low}(d_I)<\frac{|\Omega|}{\mathcal{R}(d_I)}$. 
\end{itemize}
    
\end{tm}


\begin{rk}
\begin{itemize}
    \item[\rm (i)] 
Theorem  \ref{T2}-{\rm (i)} shows that  $m_{d_I}^*<1$ whenever $N_{\rm low}(d_I)<\frac{|\Omega|}{\mathcal{R}(d_I)}$.  We don't know yet whether  the two inequalities are equivalent. In some scenarios, it turns out that the later inequality can be often verified for large values of $d_I$. Indeed, if \begin{equation}\label{lem2-eq3}
       \frac{\int_{\Omega}\beta}{ \int_{\Omega}\gamma}<\frac{|\Omega|}{\int_{\Omega}\frac{\gamma}{\beta}},
    \end{equation} 
    if follows from Lemma \ref{lem1}-{\rm (ii)} that $\frac{|\Omega|}{\mathcal{R}(d_I)}>\int_{\Omega}\frac{\gamma}{\beta}$ for every $d_I>\mathcal{R}^{-1}\Big(\frac{|\Omega|}{\int_{\Omega}\frac{\gamma}{\beta}}\Big)$. In this case, it follows from Theorem \ref{T2}-{\rm (i) - (ii)}  that $N_{\rm low}(d_I)<\frac{|\Omega|}{\mathcal{R}(d_I)}$ and $m_{d_I}^*<1$  for every $d_I>\mathcal{R}^{-1}\Big(\frac{|\Omega|}{\int_{\Omega}\frac{\gamma}{\beta}}\Big)$. Note for instance that \eqref{lem2-eq3}  holds when $\gamma=\beta^2$ and is not constant. Under hypothesis \eqref{lem2-eq3}, we see that there is a range of parameters satisfying $\mathcal{R}(N,d_I)<1$ such that \eqref{e1} has at least two EE solutions for small values of the diffusion rates of the susceptible population. This shows that the results of Proposition \ref{Proposition1} on the epidemic model \eqref{e1-prime} can not be extended in general to  the epidemic model \eqref{e1}. In particular, it is possible for the basic reproduction number of \eqref{e1} to be less than one and the disease to still persist, in which case there is a backward bifurcation for the EE as the total size $N$ of the population varies.

    \item[\rm (ii)] If   $\frac{|\Omega|}{\mathcal{R}(d_I)}>\int_{\Omega}\frac{\gamma}{\beta}$ and $\frac{|\Omega|}{\mathcal{R}(d_I)}>\int_{\Omega}\frac{\gamma}{\beta}$, Theorem \ref{T2}-{\rm (ii)} indicates that it is possible to have a multiple EE for some choice of the total population size $N$ satisfying $\mathcal{R}_0(N,d_I)>1$. 
\end{itemize}

\end{rk}


%as shown by our next result.   %Theorem \ref{T0} gives the existence of multiple EE solutions of \eqref{e1} provided either $N_{\rm low}(d_I)<\frac{|\Omega|}{\mathcal{R}(d_I)}$ of $m^*_{d_I}<1$ for some $d_I>0$. Hence, it is necessary to provide sufficient conditions on the parameters of the model which may guarantee that either  $N_{\rm low}(d_I)<\frac{|\Omega|}{\mathcal{R}(d_I)}$ of $m^*_{d_I}<1$ for some $d_I>0$, which is considered in our next result.


%\begin{tm}\label{T2} For every $d_I>0$, let $N_{\rm low}(d_I)$ and $m^*_{d_I}$ be given as in Theorems \ref{T0} and \ref{T1} respectively. The following conclusions hold.\begin{itemize}    \item[\rm (i)] If      then $N_{\rm low}(d_I)<\frac{|\Omega|}{\mathcal{R}(d_I)}$ for every $d_I>\mathcal{R}^{-1}\Big(\frac{|\Omega|}{\int_{\Omega}\frac{\gamma}{\beta}}\Big)$. %Furthermore, if in addition \begin{equation}    \int_{\Omega}\frac{\gamma}{\beta}(1-\frac{1}{\beta})>0, \end{equation} then $N_{\rm low}(d_I)<\int_{\Omega}\frac{\gamma}{\beta}<\frac{|\Omega|}{\mathcal{R}(d_I)}$ for every $d_I>\mathcal{R}^{-1}\Big(\frac{|\Omega|}{\int_{\Omega}\frac{\gamma}{\beta}}\Big)$.      \item[\rm (ii)] Fix  $d_I>0$. If     \begin{equation}       \frac{|\Omega|\int_{\Omega}\beta\varphi_{d_I}^3}{\Big[\int_{\Omega}\varphi_{d_I}\Big]\Big[\int_{\Omega}\beta\varphi_{d_I}^2\Big]} <1,    \end{equation}    then $N_{\rm low}(d_I)<\frac{|\Omega|}{\mathcal{R}(d_I)}$ and $m_{d_I}^*<1$.    \item[\rm (iii)]  Fix  $d_I>0$. If     \begin{equation}        \frac{|\Omega|\int_{\Omega}\beta\varphi_{d_I}^3}{\Big[\int_{\Omega}\varphi_{d_I}\Big]\Big[\int_{\Omega}\beta\varphi_{d_I}^2\Big]} >1\quad \text{and}\quad \frac{|\Omega|}{\mathcal{R}(d_I)}>\int_{\Omega}\frac{\gamma}{\beta},   \end{equation}    then $N_{\rm low}(d_I)<\frac{|\Omega|}{\mathcal{R}(d_I)}$,  $m_{d_I}^*<1$ and there is $N>\frac{|\Omega|}{\mathcal{R}(d_I)}$ and $ 0<d_{S}^N<d_I(1-m_{d_I})$  such that \eqref{e1}  has at least two EE solutions for $d_S=d_{S}^N$. \end{itemize}\end{tm}



%{\color{blue}(Add an illustrative bifurcation diagram with $N$ as the bifurcation parameter )}  

%\quad We complete our results with the study of the limiting profiles of EE solutions of \eqref{e1} given by Theorem \ref{T1}. In this direction, the following result holds.


%\begin{tm}\label{T2} Suppose that \eqref{lem2-eq3} holds. Fix $d_I> \mathcal{R}^{-1}\Big(\frac{|\Omega|}{\int_{\Omega}\frac{\gamma}{\beta}}\Big)$ and let $N_{\rm low}(d_I)$ be given by Theorem \ref{T1}-{\rm (i)}. Then \begin{equation}\label{T2-eq1}    \int_{\Omega}\frac{\gamma}{\beta}<\frac{|\Omega|}{\mathcal{R}(d_I)}.\end{equation}Furthermore, the following conclusions hold.\begin{itemize}    \item[\rm (i)] If $N_{\rm low}(d_I)<N<\int_{\Omega}\frac{\gamma}{\beta}$, then $\mathcal{R}_0(N,d_I)<1$ and any EE solution of \eqref{e1} for small diffusion rate of the susceptible population $d_S$ satisfies    \item[\rm (ii)] If $\int_{\Omega}\frac{\gamma}{\beta}<N<\frac{|\Omega|}{\mathcal{R}(d_I)}$, then $\mathcal{R}_0(N,d_I)<1$ and there exist sequence of positive numbers $\{d^1_{S,n}\}_{n\ge 1}$ and $\{d^2_{S,n}\}_{n\ge 1}$ converging both to zero that for every $n\ge 1$, \eqref{e1} has  EE solutions $(S_{1,n},I_{1,n})$ and $(S_{2,n},I_{2,n})$ for $d_S=d_{S,n}^1$ and $d_{S,n}^2$, respectively. Furthermore,    \begin{equation}\label{T2-eq5}        \lim_{n\to \infty}\left[\Big\|S_{1,n}-\frac{\gamma}{\beta}\Big\|_{\infty}+\Big\|I_{1,n}-\frac{1}{|\Omega|}\Big(N-\int_{\Omega}\frac{\gamma}{\beta}\Big)\Big\|_{\infty}\right]=0,    \end{equation}    and there is a positive constant $C>0$, independent of $n$, such that $I_{2,n}$ satisfies \eqref{T2-eq2} for $d_S=d_{S,n}^2$ for all $n\ge 1$ and $S_{2,n}$ satisfies \eqref{T2-eq3}.\end{itemize}\end{tm}

%\subsection{Discussion} {\color{blue} $\bullet$ Link the result to backward bifurcation.  $\bullet$ Talk about the difference between the ode model $\bullet$ Point out the open problem whether $\mathcal{R}(N,d_I)$ could be greater than 1 in Theorem \ref{T1}. If this is the case, then we may have $N_{\rm low}(d_I)=\frac{|\Omega|}{\mathcal{R}(d_I)}$ and $m_{d_I}^*<1$, which yields  multiple EE for basic reproduction number bigger than one. $\bullet$ Give a conjecture on the simple possible bifurcation diagram for EE when $d_I>0$ and  $0<d_S<d_I(1-m_{d_I}^*)$ are fixed and $N$ varies from zero to infinity.$\bullet$ Comment on the comparison between the two models.}

\section{Preliminaries}\label{Preliminaries}
\quad We present a few preliminary results here. % to use in the subsequent sections in the proofs of our main results. We start with the following important on the limit of $\mathcal{R}(d_I)$ with respect to $d_I$.



\begin{lem}\label{lem1} 
\begin{itemize}
    \item[\rm (i)] If $\frac{\beta}{\gamma}$ is constant, then $\mathcal{R}(d_I)=\frac{\beta}{\gamma}$ for all $d_I>0$. \item[\rm (ii)] If  $\frac{\beta}{\gamma}$ is not constant, then $\mathcal{R}(d_I)$ is strictly decreasing in $d_I$. Moreover,
    \begin{equation}\label{R-star-limit-eq}
        \lim_{d_I\to 0^+}\mathcal{R}(d_I)=\max_{x\in\overline{\Omega}}\frac{\beta(x)}{\gamma(x)}\quad \text{and}\quad \lim_{d_I\to\infty}\mathcal{R}(d_I)=\frac{\int_{\Omega}\beta}{\int_{\Omega}\gamma}.
    \end{equation} 
    In particular, if \eqref{lem2-eq3} holds, then
      $   \frac{|\Omega|}{\mathcal{R}(d_I)}<\int_{\Omega}\frac{\gamma}{\beta}$  {if} $ 0<d_I<\mathcal{R}^{-1}\Big(\frac{|\Omega|}{\int_{\Omega}\frac{\gamma}{\beta}}\Big)$, $ 
        \frac{|\Omega|}{\mathcal{R}(d_I)}=\int_{\Omega}\frac{\gamma}{\beta}$ if $  d_I=\mathcal{R}^{-1}\Big(\frac{|\Omega|}{\int_{\Omega}\frac{\gamma}{\beta}}\Big)$, and $ 
        \frac{|\Omega|}{\mathcal{R}(d_I)}>\int_{\Omega}\frac{\gamma}{\beta}$  if $d_I>\mathcal{R}^{-1}\Big(\frac{|\Omega|}{\int_{\Omega}\frac{\gamma}{\beta}}\Big)$ .
    
\end{itemize}
\end{lem}
%We shall always suppose that $\frac{\beta}{\gamma}$ is not constant. Set \begin{equation}\label{l-star-def}    l^*(d_I)=\frac{1}{\mathcal{R}(d_I)}.\end{equation} Note that $\mathcal{R}_0(N,d_I)-1$ and  $N-|\Omega|l^*(d_I)$ have the same sign. For convenience, we state the results of this section in term of  $l^*(d_I)$. 

\quad Consider the one-parameter family of elliptic equations
\begin{equation}\label{Eq1}
    \begin{cases}
    0=d_I\Delta u +(l\beta(1-d_Iu)-\gamma)u & x\in\Omega,\cr 
    0=\partial_{\vec{n}}u & x\in\partial\Omega
    \end{cases}
\end{equation}
Note \eqref{Eq1} is a one parameter family of the classical diffusive logistic elliptic equation. Hence, several interesting results have been established as with respect to the existence, uniqueness of stability of it positive solution whenever it exists. The next lemma collects some results on the positive solution of \eqref{Eq1}.

\begin{lem}\label{lem2} Fix $d_I>0$ and let $\mathcal{R}(d_I)$ be given by \eqref{R-star-eq}.
\begin{itemize}
    \item[\rm (i)] The elliptic equation \eqref{Eq1} has a (unique) positive solution, $u^l$,  if and only if $l>\frac{1}{\mathcal{R}(d_I)}$. Moreover, 
    \begin{equation}\label{lem2-eq1}
        0<u^l<\frac{1}{d_I},\quad \lim_{l\to \frac{1}{\mathcal{R}(d_I)}}\|u^l\|_{\infty}=0,\quad \lim_{l\to\infty}\|u^l-\frac{1}{d_I}\|_{\infty}=0,\quad \text{and}\quad \lim_{l\to\infty}\Big\|l(1-d_Iu^l)-\frac{\gamma}{\beta}\Big\|_{\infty}=0.
    \end{equation}
    \item[\rm (ii)] The mapping $\big(\frac{1}{\mathcal{R}(d_I)},\infty\big)\ni l\mapsto u^l\in C^1(\overline{\Omega})$ is smooth and strictly increasing. Setting $v^l=\partial_t u^l$ for every $l>\frac{1}{\mathcal{R}(d_I)}$, then $v^l$ satisfies
    \begin{equation}\label{v-l-eq1}
        \begin{cases}
        0=d_I\Delta v^l+(l\beta(1-2d_Iu^l)-\gamma)v^l+\beta(1-d_Iu^l)u^l & x\in\Omega,\cr 
        0=\partial_{\vec{n}}v^l & x\in\partial\Omega,
        \end{cases}
    \end{equation}
    \begin{equation}\label{v-l-eq2}
       \lim_{l\to \frac{1}{\mathcal{R}(d_I)}}\left\|v^l-\left(\frac{\mathcal{R}(d_I)\int_{\Omega}\beta\varphi_{d_I}^2}{d_I\int_{\Omega}\beta\varphi_{d_I}^3}\right)\varphi_{d_I}\right\|_{\infty}=0,\quad \text{and}\quad \lim_{l\to\infty}\Big\|l^2v^l-\frac{\gamma}{d_I\beta}\Big\|_{\infty}=0.
    \end{equation}
   % and %there is $l_1^*(d_I)\ge l^*(d_I) $ such that 
    %\begin{equation}\label{v-l-eq3}      0<lv^l<u^l<\frac{1}{d_I}\quad \forall\ l>l_1^*(d_I).   \end{equation} Furthermore, \begin{equation}\label{v-l-eq5}      \lim_{l\to\infty}\|lv^l\|_{\infty}=0.  \end{equation}   In fact, it holds that
 % \begin{equation}\label{v-l-eq4}  \end{equation}
    
    
    \item[\rm (iii)] The function $\mathcal{N}_{d_I}$ defined by 
    \begin{equation}\label{N-d_I-def}
        \mathcal{N}_{d_I}(l)=l\int_{\Omega}(1-d_Iu^l)\quad \forall\ l\ge \frac{1}{\mathcal{R}(d_I)},
    \end{equation}
    is continuously differentiable, with \begin{equation}\label{lem2-eq2}
        \lim_{l\to \frac{1}{\mathcal{R}(d_I)}}\mathcal{N}_{d_I}(l)=l^*(d_I)|\Omega|\quad \text{and}\quad \lim_{l\to\infty}\mathcal{N}_{d_I}(l)=\int_{\Omega}\frac{\gamma}{\beta}.
    \end{equation} 
%    \item[\rm (iv)] 
\end{itemize}

\end{lem}
\begin{proof}{\rm (i)} It follows from standard results on the diffusive logistic equations. See for example \cite[Lemma 4.2-(i)]{CastellanoSalako2021}.

\quad {\rm (ii)} The regularity of $u^l$ with respect to $l$ and the fact that $v^l>0$ and solves \eqref{v-l-eq1} is already proved in \cite[Lemma 4.2-(i)]{CastellanoSalako2021}.  Next, we prove that \eqref{v-l-eq2} holds.  Since $l^*:=\frac{1}{\mathcal{R}(d_I)}$ is a simple eigenvalue, it follows from standard arguments from the bifurcation theory (see \cite{CR1971}) that there exist $\varepsilon_0>0$ and smooth functions 
$c : (l^*-\varepsilon_0,l^*+\varepsilon_0)\to \mathbb{R}$ and $\psi : (l^*-\varepsilon_0, l^*+\varepsilon_0)\to C^2(\overline{\Omega})\cap\text{span}^T(\varphi_{d_I})$ such that 
\begin{equation}\label{b1}
    u^l=(l-l^*)\big(c(l)\varphi_{d_I}+(l-l^*)\psi(l)\big)\quad \forall\ l^*\le l<l^*+\varepsilon_0,
\end{equation}
where $\text{span}^T(\varphi_{d_I})=\big\{g\in  L^2(\Omega)\ :\ \int_{\Omega}g\varphi_{d_I}=0\big\}$. %we can can express $u^l$ in the neighborhood of $l^*(d_I)$ as \begin{equation}\label{b2}    u^l= (l-l^*)(c_1\varphi_{d_I}+(l-l^*(d_I))\tilde{\varphi}(l)) \quad 0\le l-l^*(d_I)<\varepsilon_0 \end{equation} for some $\varepsilon_0>0$, $c_1\ge 0$ and $\tilde{\varphi}\in C([l^*(d_I), l^*(d_I)+\varepsilon_0]: C^2(\overline{\Omega}))$. 
 For every $l\in (l^*,l^*+\varepsilon_0)$, 
inserting \eqref{b1} in \eqref{Eq1}, and dividing the resulting equation by $l-l^*$ and recalling that $\varphi_{d_I}$ satisfies \eqref{R-star-pde}, we obtain
\begin{align*}
    0=&d_I\Delta(c(l)\varphi_{d_I}+(l-l^*)\psi(l))+(l\beta(1-d_Iu^l)-\gamma)(c(l)\varphi_{d_I}+(l-l^*)\psi(l))\cr 
   % =&(l-l^*)\left(d_I\Delta\psi(l)+(l\beta(1-d_Iu^l)-\gamma)\psi(l)\right)+c(l)\beta\big((l-l^*)-ld_Iu^l\big)\varphi_{d_I}\cr 
%&+c_1\Big(d_I\Delta\varphi_{d_I}+\big(l^*(d_I)\beta-\gamma\big)\varphi_{d_I}\Big)\cr 
    =& (l-l^*)\Big(d_I\Delta\psi(l) +(l\beta(1-d_Iu^l)-\gamma)\psi(l) +c(l)\beta(1-ld_I(c(l)\varphi_{d_I}+(l-l^*)\psi(l)))\varphi_{d_I}\Big).
\end{align*}
It then follows that 
\begin{equation*}
    \begin{cases}
        0=d_I\Delta\psi(l) +(l\beta(1-d_Iu^l)-\gamma)\psi(l) +c(l)\beta(1-ld_I(c(l)\varphi_{d_I}+(l-l^*)\psi(l)))\varphi_{d_I} & x\in\Omega,\cr 
        0=\partial_{\vec{n}}\psi(l) & x\in\partial\Omega.
    \end{cases}
\end{equation*}
Hence, letting $l\to l^*$, we obtain that 
\begin{equation}\label{b3}
    \begin{cases}
        0=d_I\Delta\psi(l^*) +(l^*\beta-\gamma)\psi(l^*) +c(l^*)\beta(1-l^*d_Ic(l^*)\varphi_{d_I})\varphi_{d_I} & x\in\Omega,\cr 
        0=\partial_{\vec{n}}\psi(l^*) & x\in\partial\Omega.
    \end{cases}
\end{equation}
Multiplying \eqref{b3} by $\varphi_{d_I}$ and integrating the resulting equation yields 
\begin{align*}    0=&d_I\int_{\Omega}\varphi_{d_I}\Delta\psi(l^*)+\int_{\Omega}(l^*\beta-\gamma)\psi(l^*)\varphi_{d_I}+c(l^*)\int_{\Omega}(\beta-l^*d_Ic(l^*)\varphi_{d_I})\varphi_{d_I}^2\cr  =&d_I\int_{\Omega}\psi(l^*)\Delta\varphi_{d_I}+\int_{\Omega}(l^*\beta-\gamma)\psi(l^*)\varphi_{d_I}+c(l^*)\int_{\Omega}(\beta-l^*d_Ic(l^*)\varphi_{d_I})\varphi_{d_I}^2\cr 
    %=&\int_{\Omega}(d_I\Delta\varphi_{d_I}+(l^*\beta-\gamma)\varphi_{d_I})\psi(l^*) +c(l^*)\int_{\Omega}(\beta-l^*d_Ic(l^*)\varphi_{d_I})\varphi_{d_I}^2\cr 
    =&c(l^*)\int_{\Omega}(\beta-l^*d_Ic(l^*)\varphi_{d_I})\varphi_{d_I}^2.
\end{align*}
This implies that 
\begin{equation}\label{b4}
    c(l^*)\int_{\Omega}(\beta-l^*d_Ic(l^*)\varphi_{d_I})\varphi_{d_I}^2=0.
\end{equation}
It is clear from \eqref{b1} that $v^l\to c(l^*)\varphi_{d_I}$ as $l\to l^*$ in $C(\overline{\Omega})$. Thus, thanks to \eqref{b4}, to complete the proof of \eqref{v-l-eq2}, it remains to show that $c(l^*)\ne 0$. To see this, choose $0<m_0\ll 1$ such that 
$  
1\ge 4l^*d_Im_0\varphi_{d_I}.
$ 
Hence, since $u^l\to 0$ as $l\to l^*$, there is $0<\varepsilon_1\ll1$ such that 
$$
1>d_Iu^l+2d_Ilm_0\varphi_{d_I}\quad l^*<l<l^*+\varepsilon_1.
$$
As a result, for every $l^*<l<l^*+\varepsilon_1$, 
\begin{align*}
    & d_I\Delta (m\varphi_{d_I})+(l\beta(1-2d_Iu^l)-\gamma)(m\varphi_{d_I})+\beta(1-d_Iu^l)u^l\cr 
    =& m\beta\Big((l-l^*)-2ld_Iu^l\Big)\varphi_{d_I}+\beta(1-d_Iu^l)u^l\cr 
    =& m_0\beta(l-l^*)\varphi_{d_I}+\beta(1-d_Iu_I-2ld_Im\varphi_{d_I})u^l >0.
\end{align*}
Therefore, since $\partial_{\vec{n}}(m_0\varphi_{d_I})=0$ on $\partial\Omega$, it follows from the maximum principle and the fact that $v^l$ is the unique positive solution of \eqref{v-l-eq1} that 
$$
m_0\varphi_{d_I}\le v^l\quad \forall l\in (l^*,l^*+\varepsilon_1),
$$
which implies that $m_0\le c(l^*)$ since $v^l\to c(l^*)\varphi_{d_I}$ as $l\to l^*$ in $C(\overline{\Omega})$. This shows that $c(l^*)>0$, which in view of \eqref{b4} implies that  
$$
c(l^*)=\frac{\int_{\Omega}\beta\varphi^2_{d_I}}{l^*d_I\int_{\Omega}\beta\varphi_{d_I}^3}.
$$
%Thus $v^l\to $.

%\quad Now, we show that \eqref{v-l-eq3} hold. Since $d_Iu^l\to 1$ as $l\to\infty$ uniformly on $\overline{\Omega}$, then there is $l^*_1(d_1)>l^*(d_I)$ such that \begin{equation}   d_Iu^l>\frac{4}{5} \quad \forall\ l\ge l^*_1(d_I).\end{equation}Setting $z^l=\frac{u^l}{l}$ for every $l>l^*_1(d_I)$, then \begin{align}    & d_I\Delta z^l+(l\beta(1-2d_Iu^l)-\gamma)z^l+\beta(1-d_Iu^l)u^l \cr     =& \frac{1}{l}\left(d_I\Delta u^l+(l\beta(1-d_Iu^l)-\gamma)u^l-ld_I\beta(u^l)^2 \right) +\beta(1-d_Iu^l)u^l\cr     =&-\beta d_I(u^l)^2+\beta(1-d_Iu^l)u^l\cr     =&\beta(1-2d_Iu^l)u^l\cr     <&\beta(1-\frac{8}{5})u^l\cr     =&-\frac{3}{5}\beta u^l<0.\end{align}Hence, it follows from the comparison principle for elliptic equation that \eqref{v-l-eq3} holds. Next, for every $l>$, $w^l:=lv^l$ satisfies $0<w^l<u^l<\frac{1}{d_I}$ for all $l>l_1^*(d_I)$ by \eqref{v-l-eq3}, and\begin{equation*}    \begin{cases}    0=d_I\Delta w^l +(l\beta(1-2d_Iu^l)-\gamma)w^l +l\beta(1-d_Iu^l)u^l & x\in\Omega,\cr     0=\partial_{\vec{n}}w^l & x\in\partial\Omega.    \end{cases}\end{equation*}Thus, since $l(1-d_Iu^l)\to \frac{\gamma}{\beta}$ as $l\to\infty$ uniformly on $\overline{\Omega}$ (see \eqref{lem2-eq1}), it follows the maximum principle for elliptic equations that  $w^l\to 0$ as $l\to\infty$ uniformly on $\overline{\Omega}$, which proves \eqref{v-l-eq4}. 

%\quad It remains to prove \eqref{v-l-eq4}. To this end, 
\quad Next, set $p^l=l^2v^{l}$ for every $l>l^*(d_I)$. By direct computations, it follows from \eqref{v-l-eq1} that $p^l$ satisfies
\begin{equation}\label{D1}
    \begin{cases}
        0=\frac{d_I}{l}\Delta p^l +\frac{\beta}{l}\big( z^l-\frac{\gamma}{\beta}\big)p^l+u^l\beta(z^l-d_Ip^l) & x\in\Omega,\cr 
        0=\partial_{\vec{n}}p^l & x\in\partial\Omega.
    \end{cases}
\end{equation}
where $z^l=l(1-d_Iu^l)$ for all $l>l^*(d_I)$. Therefore, since $u^{l}\to \frac{1}{d_I}$ and $z^l\to\frac{\gamma}{\beta}$ as $l\to\infty$ in $C(\overline{\Omega})$ by \eqref{lem2-eq1}, we can employ the singular perturbation theory for elliptic equations to deduce from \eqref{D1} that $p^l\to \frac{\gamma}{d_I\beta}$ as $l\to\infty$ uniformly on $\overline{\Omega}$, which completes the proof of  \eqref{v-l-eq2}.
\end{proof}

The next result shows the relationship between EE solutions of \eqref{e1} and the solutions of \eqref{Eq1}

\begin{lem}\label{lem3} Let $d_S>0$, $d_I>0$ and $N>0$.
\begin{itemize}
    \item[\rm (i)] If $(S,I)$ is an EE solution of \eqref{e1} then 
    \begin{equation}\label{kappa-def}
        \kappa =d_SS+d_II
    \end{equation} 
    is a constant function. Furthermore, setting 
    \begin{equation}\label{tilde-s-i-ded}
        \tilde{S}=\frac{S}{\kappa}\quad \text{and}\quad \tilde{I}=\frac{I}{\kappa},
    \end{equation}
    then $\tilde{I}$ is the positive solution of \eqref{Eq1} with $l=\frac{\kappa}{d_S}$.
    \item[\rm (ii)] If $l>\frac{1}{\mathcal{R}(d_I)}$ and  \begin{equation}\label{N-equation}
        N=l\left[\int_{\Omega}(1-d_Iu^l)+d_S\int_{\Omega}u^l\right],
    \end{equation}
    then $(S,I):=(l(1-d_Iu^l),d_Slu^l)$ is an EE solution of \eqref{e1}.
\end{itemize}
\end{lem}
\begin{proof} The proof is similar to that of \cite[Lemma 4.4]{CastellanoSalako2021}, hence it is omitted.
    
\end{proof}

\section{Proofs of Main Results}\label{proofs}

We present the proof of our main results in this section.

\begin{proof}[Proof of Theorem \ref{T0}] Let $d_I>0$ and define \begin{equation}\label{C0} 
N_{\rm low}(d_I)=\inf_{l\ge \frac{1}{\mathcal{R}(d_I)}}\mathcal{N}_{d_I}(l),
\end{equation}
where $\mathcal{N}_{d_I}(l)$ is defined by \eqref{N-d_I-def}.  It is clear from \eqref{lem2-eq2} that $N_{\rm low}(d_I)\le \min\Big\{ \frac{|\Omega|}{\mathcal{R}(d_I)},\int_{\Omega}\frac{\gamma}{\beta}\Big\}$. Furthermore, since $\mathcal{N}_{d_I}(l)>0$ for every $l\ge \frac{1}{\mathcal{R}(d_I)}$ and converges to a positive number as $l$ approaches infinity, then $N_{\rm low}(d_I)>0$.  Next, we prove assertions  {\rm (i)}-{\rm (iii)}.



\quad {\rm (i)}   If $(S,I)$ is an EE solution of \eqref{e1} for some $N>0$ and $d_S>0$, then by Lemma \ref{lem3}, $l:=\frac{\kappa}{d_S}>\frac{1}{\mathcal{R}(d_I)}$, where $\kappa$ is defined by \eqref{kappa-def}. Furthermore, by \eqref{N-equation}, we have that 
$$ 
N=\frac{\kappa}{d_S}\int_{\Omega}(1-d_Iu^{\frac{\kappa}{d_S}})+\kappa\int_{\Omega}u^{\frac{\kappa}{d_S}}=\mathcal{N}_{d_I}(\frac{\kappa}{d_S}) +\kappa\int_{\Omega}u^{\frac{\kappa}{d_S}}>N_{\rm low}(d_I).
$$
This shows that \eqref{e1} has no EE solution for every $N\le N_{\rm low}(d_I) $ and $d_S>0$.

\quad {\rm (ii)} Let $N>N_{\rm low}(d_I)$ be fixed. Then, there is $l(N,d_I)>\frac{1}{\mathcal{R}(d_I)}$ such that 
\begin{equation}\label{C4}
\mathcal{N}_{d_I}(l(N,d_I))<N.
\end{equation}
%Set \begin{equation}    l^*(N,d_I):=\sup\{l>\frac{1}{\mathcal{R}(d_I)}\ :\ \mathcal{N}_{d_I}(l)<N\}.\end{equation}It follows from \eqref{C4} that $l_{*}(N,d_I)\in(l(N,d_I),\infty]$. Note $l^*(N,d_I)$ is finite if and only if $N$ is the range of $\mathcal{N}_{d_I}$. 
Set 
\begin{equation}\label{C4-2}
    d(N,d_I):=
        \frac{N-\mathcal{N}_{d_I}(l(N,d_I))}{l(N,d_I)\int_{\Omega}u^{l(N,d_I)}}>0. 
\end{equation}
For every $d_S>0$, consider the function $\mathcal{N}_{d_I,d_S}$ defined by 
    \begin{equation}\label{N-d_I-d_S-def}
        \mathcal{N}_{d_I,d_S}(l)=\mathcal{N}_{d_I}(l)+d_Sl\int_{\Omega}u^l\quad \forall\ l\ge l^*(d_I).
    \end{equation}
     Then $\mathcal{N}_{d_I,d_S}$ is continuously differentiable in $l>\frac{1}{\mathcal{R}(d_I)}$. Furthermore, 
    \begin{equation}\label{EE1}
    \mathcal{N}_{d_I,d_S}(\frac{1}{\mathcal{R}(d_I)})=\frac{|\Omega|}{\mathcal{R}(d_I)} \quad \text{and}\quad \lim_{l\to\infty}\mathcal{N}_{d_I,d_S}(l)=\infty.
    \end{equation} 
    Next, fix $0<d_S<d(N,d_I)$. It follows from \eqref{C4-2} that  \begin{align}\label{C2}
    \mathcal{N}_{d_I,d_S}(l(N,d_I))=&\mathcal{N}_{d_I}(l(N,d_I))+d_Sl(N,d_I)\int_{\Omega}u^{l(N,d_I)}\cr <&\mathcal{N}_{d_I}(l(N,d_I))+d(N,d_I)l(N,d_I)\int_{\Omega}u^{l(N,d_I)} =N.
    \end{align}
      Therefore, by the intermediate value theorem, there is $l(N,d_I,d_S)>l(N,d_I)$ such that $\mathcal{N}_{d_I,d_S}(l(N,d_I,d_S))=N$. This together with \eqref{EE1} imply that the quantity  
      \begin{equation}\label{l-high-def}
        l_{\rm high}(N,d_I,d_S):=\max\{l>l(N,d_I)\ :\ \mathcal{N}_{d_I,d_S}(l)=N\}
    \end{equation}
    is a positive real number.  Hence, by Lemma \ref{lem3}-(ii),
    $$(S_{\rm high}(\cdot;d_S),I_{\rm high}(\cdot;d_S)):=(l_{\rm high}(N,d_I,d_S)(1-d_Su^{l_{\rm high}(N,d_I,d_S)}),d_Sl_{\rm high}(N,d_I,d_S)u^{l_{\rm high}(N,d_I,d_S)})$$  is an EE solution of  \eqref{e1}. Finally, we show that \eqref{T0-eq1} holds. So, suppose that $(S,I)$ is another EE solution of \eqref{e1}. Then, by Lemma \ref{lem3}  we  have that $\frac{\kappa}{d_S}>\frac{1}{\mathcal{R}(d_I)}$ and  $ I=d_{S}(\frac{\kappa}{d_S})u^{\frac{\kappa}{d_S}}$. Hence, since the mapping $(\frac{1}{\mathcal{R}(d_I)},\infty)\ni l\mapsto d_Slu^{l}$ is strictly increasing, and $\mathcal{N}_{d_I,d_S}(\frac{\kappa}{d_S})=N=\mathcal{N}_{d_I,d_S}(l_{\rm high}(N,d_I,d_S))$, then  $\frac{\kappa}{d_S}<l_{\rm high}(N,d_I,d_S)$, which yields the desired result.

    \quad {\rm (iii)} Suppose that $N_{\rm low}(d_I)<N<\frac{|\Omega|}{\mathcal{R}(d_I)}$ and $d(N,d_I)$ be given by {\rm (i)} and   $l(N,d_I)$ be as in \eqref{C4}. Fix $0<d_S<d(N,d_I)$. Observe that 
    $$ 
    \mathcal{N}_{d_I,d_S}(l(N,d_I))<N<\frac{|\Omega|}{\mathcal{R}(d_I)}=\mathcal{N}_{d_{I},d_S}(\frac{1}{\mathcal{R}(d_I)}).
    $$ 
    Therefore, by the intermediate value theorem, there is $\tilde{l}(N,d_I,d_S)\in (\frac{1}{\mathcal{R}(d_I)},l(N,d_I))$ such that $\mathcal{N}_{d_I,d_S}(\tilde{l}(N,d_I,d_S))=N$. This implies that the quantity
    \begin{equation}\label{l-low-def}
        l_{\rm low}(N,d_I,d_S):=\min\{l\in (\frac{1}{\mathcal{R}(d_I)},l(N,d_I))\ :\ \mathcal{N}_{d_I,d_S}(l)=N\}
    \end{equation}
    is well defined and satisfies $\frac{1}{\mathcal{R}(d_I)}<l_{\rm low}(N,d_I,d_S)<l(N,d_I)$.  Now, by Lemma \ref{lem3}-(ii),
    $$(S_{\rm low}(\cdot;d_S),I_{\rm low}(\cdot;d_S)):=(l_{\rm low}(N,d_I,d_S)(1-d_Su^{l_{\rm low}(N,d_I,d_S)}),d_Sl_{\rm high}(N,d_I,d_S)u^{l_{\rm low}(N,d_I,d_S)})$$  is an EE solution of  \eqref{e1}.  Since $l_{\rm low}(N,d_I,d_S)<l(N,d_I)<l_{\rm high}(N,d_I,d_S)$ and the mapping $(\frac{1}{\mathcal{R}(d_I)},\infty)l\mapsto lu^l$ is strictly increasing, then \eqref{T0-eq2} holds.  Using again the fact that the mapping $(\frac{1}{\mathcal{R}(d_I)},\infty)l\mapsto lu^l$ is strictly increasing, it can be shown as in the case of \eqref{T0-eq1} that any other EE solution of \eqref{e1}, if exists, must satisfy \eqref{T0-eq3}.
    
    \quad {\rm (iii-1)} Next, suppose that $N<\int_{\Omega}\frac{\gamma}{\beta}$ and we establish that \eqref{T0-eq4} holds. First, since $\mathcal{N}_{d_I,d_{S,1}}<\mathcal{N}_{d_I,d_{S,2}}$ on $(\frac{1}{\mathcal{R}(d_I)},\infty)$ for every $0<d_{S,1}<d_{S,2}$, it follows from \eqref{l-low-def} and \eqref{l-high-def} that  $l_{\rm low}(N,d_I,d_S)$  is strictly decreasing in $d_S$ and  $l_{\rm high}(N,d_I,d_S)$  is strictly increasing in $d_S$. Thus, the following limits exist
    \begin{equation}
        l_{\rm low}(N,d_I)=\lim_{d\to 0}l_{\rm low}(N,d_I,d_S)\quad \text{and} \quad l_{\rm high}(N,d_I):=\lim_{d_S\to 0}l_{\rm high}(N,d_I,d_S).
    \end{equation}
   In the current case, we first proceed by contradiction to show that 
    \begin{equation}\label{G1}
        l_{\rm high}(N,d_I)<\infty.
    \end{equation}
    Indeed, if \eqref{G1} were false, then  $l_{\rm high}(N,d_I,d_{S})\to \infty$. As, a result, it follows from \eqref{lem2-eq2} that 
    \begin{equation}\label{G2}
        \lim_{d_S\to0}\int_{\Omega}l_{\rm high}(N,d_I,d_{S})(1-d_Iu^{l_{\rm high}(N,d_I,d_{S})})=\int_{\Omega}\frac{\gamma}{\beta}.
    \end{equation}
    On the other hand, using the fact that
    \begin{equation*}
        N=\mathcal{N}_{d_I,d_{S}}(l_{\rm high}(N,d_I,d_{S}))>\int_{\Omega}l_{\rm high}(N,d_I,d_{S})(1-d_Iu^{l_{\rm high}(N,d_I,d_{S})})\quad \forall\ 0<d_S<d(N,d_I),
    \end{equation*}
    we obtain that 
    $$ 
    N\ge \lim_{d_S\to 0}\int_{\Omega}l_{\rm high}(N,d_I,d_{S})(1-d_Iu^{l_{\rm high}(N,d_I,d_{S})}),
    $$
    which clearly contradicts with \eqref{G2} since $N<\int_{\Omega}\frac{\gamma}{\beta}$. Thus, \eqref{G1} holds.  This shows that there is a positive constant $C_1(N,d_I)$ such that 
    \begin{equation}\label{G3}
        l_{\rm high}(N,d_I,d_S)\le C_1(N,d_I) \quad \forall\ 0<d_S<\frac{d(N,d_I)}{2}.
    \end{equation}
    In particular,
\begin{equation}\label{G7}
    I_{\rm low}(\cdot;d_S)<I_{\rm high}(\cdot;d_S)=d_Sl_{\rm high}(N,d_I,d_S)u^{l_{\rm high}(N,d_I,d_S)}\le \frac{C_1(N,d_I)}{d_I}d_S\quad \forall\ 0<d_S<\frac{d(N,d_I)}{2}.
\end{equation}
Next, we claim that 
\begin{equation}\label{G4}
    l_{\rm low}(N,d_I)>\frac{1}{\mathcal{R}(d_I)}.
\end{equation}
If \eqref{G4} were false, then  $l_{\rm low}(N,d_I,d_{S})\to\frac{1}{\mathcal{R}(d_I)}$ as $d_S\to 0$. This in turn implies that 
\begin{align*}
    N=&\mathcal{N}_{d_I,d_{S}}(l_{\rm low}(N,d_I,d_{S}))\cr 
    =&\mathcal{N}_{d_{I}}(l_{\rm low}(N,d_I,d_{S}))+d_{S}l_{\rm low}(N,d_I,d_{S})\int_{\Omega}u^{l_{\rm low}(N,d_I,d_{S})} 
     \to  \frac{|\Omega|}{\mathcal{R}(d_I)}\quad \text{as}\ d_{S}\to 0,
\end{align*}
which contradicts our initial assumption that $N<\frac{|\Omega|}{\mathcal{R}(d_I)}$. Therefore, \eqref{G4} holds. Thus, there is $C_2(N,d_I)>\frac{1}{\mathcal{R}(d_I)}$ such that 
\begin{equation}\label{G5}
    C_2(N,d_I)\le l_{\rm low}(N,d_I,d_S)\le l(N,d_I)\quad \forall\  0<d_S<\frac{d(N.d_I)}{2}.
\end{equation}
As a result, we obtain that 
\begin{align}\label{G6}
I_{\rm low}(\cdot;d_S)=&d_{S}l_{\rm low}(N,d_I,d_S)u^{l_{\rm low}(N,d_I,s_S)}
\ge C_2(N,d_I)d_Su^{C_2(N,d_I)}(\cdot)\cr 
\ge & C_2u^{C_2(N,d_I)}_{\min}d_S\quad \forall\ 0<d_S<\frac{d(N,d_I)}{2}.
\end{align}
Combining \eqref{G6} and \eqref{G7} we derive that \eqref{T0-eq5} holds. 

\quad Next, since $l_{\rm high}>l_{\rm low}>\frac{1}{\mathcal{R}(d_I)}$, then by Lemma \ref{lem2}-{\rm (ii)}, we have that 
$$ 
S_{\rm low}(\cdot;d_S)=l_{\rm low}(N,d_I,d_S)(1-d_Iu^{l_{\rm low}(N,d_I,d_S)})\to l_{\rm low}(1-d_Iu^{l_{\rm low}(N,d_I)})
$$
and $$ 
S_{\rm high}(\cdot;d_S)=l_{\rm low}(N,d_I,d_S)(1-d_Iu^{l_{\rm high}(N,d_I,d_S)})\to l_{\rm low}(1-d_Iu^{l_{\rm high}(N,d_I)})
$$
as $d_S\to 0$ in $C^{1}(\overline{\Omega})$. On the other hand, since $N=\int_{\Omega}(S+I)$, we have that 
$$ 
N=l_{\rm low}(N,d_I)\int_{\Omega}(1-d_Iu^{l_{\rm low}(N,d_I)}) \quad \text{and}\quad N=l_{\rm high}(N,d_I)\int_{\Omega}(1-d_Iu^{l_{\rm high}(N,d_I)}).
$$
Since  $l_{\rm low}(N,d_I)<l(N,d_I)<l_{\rm high}(N,d_I)$, then $u^*_{\rm low}:=u^{l_{\rm low}(N,d_I)}<u^{l_{\rm high}(N,d_I)}=:u^*_{\rm high}$,  which completes the proof of \eqref{T0-eq5}.


\quad {\rm (iii-2)} Finally suppose that $N>\int_{\Omega}\frac{\gamma}{\beta}$. Note that the proof of \eqref{G5} only relies on the fact that $N\ne \frac{1}{\mathcal{R}(d_I)}$. Hence, $(S_{\rm low}(\cdot;d_S),I_{\rm low}(\cdot;d_S))$ satisfies \eqref{T0-eq4} and \eqref{T0-eq5} as $d_S\to 0$. Next, we claim that 
\begin{equation}\label{G8}
    l_{\rm high}(N,d_I)=\infty.
\end{equation}
Indeed, since $N>\int_{\Omega}\frac{\gamma}{\beta}=\lim_{l\to\infty}\mathcal{N}_{d_I}(l)$, then for every $m>1$, there is $l_m(N,d_I)>m$ such that 
$$ 
\mathcal{N}_{d_I}(l_m(N,d_I))<N.
$$
Therefore, taking this time $d_m(N,d_I):=\frac{N-\mathcal{N}_{d_I}(l_m(N,d_I))}{l_m(N,d_I)\int_{\Omega}u^{l_m(N,d_I)}}$, for every $0<d_S<d_m(N,d_I)$, we can employ similar arguments as in \eqref{C2} and the intermediate value theorem to conclude that there is ${l}_m(N,d_I,d_S)>l_m(N,d_I)$ such that $\mathcal{N}_{d_I,d_S}(l_{m}(N,d_I,d_S))=N$. This shows that 
$$ 
l_{\rm high}(N,d_I,d_S)\ge l_{m}(N,d_I,d_S)>l_{m}(N,d_I)>m\quad \forall\ 0<d_S<d_{m}(N,d_I).
$$
Letting $m\to \infty$ in this inequality leads to \eqref{G8}. Thus, since $l(1-d_Iu^l)\to \frac{\gamma}{\beta}$ as $l\to\infty$ in $C(\overline{\Omega})$ (see \eqref{lem2-eq1}), we conclude that 
$$
S_{\rm high}(\cdot,d_S)=l_{\rm high}(N,d_I,d_S)(1-du^{l_{\rm high}(N,d_I,d_S)})\to \frac{\gamma}{\beta}\quad \text{as}\ d_S\to 0
$$
uniformly in $C(\overline{\Omega})$. Observing that
$$ 
d_Sl_{\rm high}(N,d_I,d_S)=\frac{N-\int_{\Omega}S_{\rm high}}{\int_{\Omega}u^{l_{\rm high}(N,d_I,d_S)}}\to \frac{N-\int_{\Omega}\frac{\gamma}{\beta}}{\frac{|\Omega|}{d_I}}\quad \text{as}\ d_S\to 0,
$$
where we have used the fact that $u^{l_{\rm high}(N,d_I,d_S)}\to \frac{1}{d_I}$ as $d_S\to 0$ uniformly on $\Omega$, then 
$$
I_{\rm high}(\cdot;d_S)=d_Sl_{\rm high}(N,d_I,d_S)u^{l_{\rm low}(N,d_I,d_S)}\to \frac{\Big(N-\int_{\Omega}\frac{\gamma}{\beta}\Big)}{\frac{|\Omega|}{d_I}}\frac{1}{d_I}=\frac{1}{|\Omega|}\left(N-\int_{\Omega}\frac{\gamma}{\beta}\right)\quad \text{as}\ d_S\to 0,
$$
in $C(\overline{\Omega})$. This completes the proof of the theorem.
    
    
    
\end{proof}

Next, we give a proof of Theorem \ref{T1}.

\begin{proof}[Proof of Theorem \ref{T1}] Fix $d_I>0$ and define 
\begin{equation*}
    M_{d_I}^*=\frac{d_I}{|\Omega|}\sup_{l>\frac{1}{\mathcal{R}(d_I)}}\int_{\Omega}(u^l+lv^l),
\end{equation*}
where for every $l>\frac{1}{\mathcal{R}(d_I)}$, $u^l$ and $v^l$ are the unique positive solutions of \eqref{Eq1} and \eqref{v-l-eq1}, respectively. Note from  \eqref{lem2-eq1} and \eqref{v-l-eq2} that $u^l\to \frac{1}{d_I}$ and $lv^l\to 0$ as $l\to\infty$ uniformly on $\Omega$.  Hence, 
$$ 
\lim_{l\to\infty}\int_{\Omega}(u^l+lv^l)=\frac{|\Omega|}{d_I}.
$$
Note also from \eqref{lem2-eq1} and \eqref{v-l-eq2} that 
$ 
\lim_{l\to\frac{1}{\mathcal{R}(d_I)}}\int_{\Omega}(u^l+lv^l)=\frac{\Big(\int_{\Omega}\varphi_{d_I}\Big)\Big(\int_{\Omega}\beta\varphi_{d_I}^2\Big)}{d_I\int_{\Omega}\beta\varphi_{d_I}^3}.
$  Hence 
\begin{equation}\label{K1}
\max\left\{1,\frac{\Big(\int_{\Omega}\varphi_{d_I}\Big)\Big(\int_{\Omega}\beta\varphi_{d_I}^2\Big)}{|\Omega|\int_{\Omega}\beta\varphi_{d_I}^3}\right\}\le M^*_{d_I}<\infty.
\end{equation}
Therefore, defining 
\begin{equation}\label{K2}
    m_{d_I}^*=\frac{1}{M^*_{d_I}},
\end{equation}
$m^*_{d_I}$ satisfies \eqref{T1-main-eq}. Next, $d_S>0$ and consider the function $\mathcal{N}_{d_I,d_S}$ be defined as in \eqref{N-d_I-d_S-def}. Taking the derivative of the function $\mathcal{N}_{d_I,d_S}$ if \eqref{N-d_I-d_S-def} with respect to $l$, we get 
\begin{equation}\label{C1-1}
    \frac{d\mathcal{N}_{d_I,d_S}(l)}{dl}= |\Omega|+(d_S-d_I)\int_{\Omega}(lv_l+u^l) \quad \forall\ l>\frac{1}{\mathcal{R}(d_I)}.
\end{equation} 
From this point, we distinguish two cases.

\noindent{\bf Case 1.} Fix $d_S>d_{I}(1-m_{d_I}^*)$. We shall show that 
\begin{equation}\label{C1-3}
\frac{d\mathcal{N}_{d_I,d_S}(l)}{dl}>0 \quad \forall\ l>\frac{1}{\mathcal{R}(d_I)}.
\end{equation}
If $d_S\ge d_I$, it is easy to see from \eqref{C1-1} that \eqref{C1-3} holds. So, we suppose that $d_I(1-m^*_{d_I})<d_S<d_I$. Hence, by \eqref{C1-1}, we have that 
\begin{align*}
  \frac{d\mathcal{N}_{d_I,d_S}(l)}{dl} =& |\Omega|\left(1-\Big(1-\frac{d_S}{d_{I}}\Big)\frac{d_I}{|\Omega|}\int_{\Omega}(u^l+lv^l)\right) \cr 
  \ge & |\Omega|\left(1-\Big(1-\frac{d_S}{d_{I}}\Big)M_{d_I}^*\right) 
  =\frac{M_{d}^*|\Omega|}{d_I}\Big(d_S-d_I(1-m_{d_I}^*)\Big)>0,  
\end{align*}
which shows that \eqref{C1-3} holds in this subcase as well. Therefore, when $d_S>d_I(1-m_{d_I}^*)$, the mapping $\mathcal{N}_{d_I,d_S}$ is strictly increasing on $[\frac{1}{\mathcal{R}(d_I)},\infty)$. Hence, thanks to Lemma \ref{lem3}, \eqref{e1} has an EE solution if and only if $N>\mathcal{N}_{d_I,d_S}(\frac{1}{\mathcal{R}(d_I)})$, that is $\mathcal{R}_0(N,d_I)>1$. Moreover, in this case, when an EE exists, it is unique.

\noindent{\bf Case 2.} Fix $0<d_S<d_{I}(1-m_{d_I}^*)$ or equivalently  $m_{d_I}^*<1-\frac{d_S}{d_I}$.  Then  
 $ 
0<\frac{1}{1-\frac{d_S}{d_I}}<\frac{1}{m_{d_I}^*}=M_{d_I}^*. $
Therefore, there is $l_0>\frac{1}{\mathcal{R}(d_I)}$ such that 
$  
0<\frac{1}{1-\frac{d_S}{d_I}}<\frac{d_I}{|\Omega|}\int_{\Omega}(u^{l_0}+l_0v^{l_0}),
$ 
 which implies that 
\begin{equation}\label{GH1}
    \frac{d\mathcal{N}_{d_I,d_S}(l_0)}{dl}=|\Omega|\left(1-\Big(1-\frac{d_S}{d_{I}}\Big)\frac{d_I}{|\Omega|}\int_{\Omega}(u^{l_0}+l_0v^{l_0})\right)<0.
\end{equation}
On the other hand, we know from \eqref{lem2-eq1} and \eqref{v-l-eq2} that 
\begin{equation}\label{GH2}
    \lim_{l\to\infty}\frac{d\mathcal{N}_{d_I,d_S}(l)}{dl}=|\Omega|+(d_S-d_I)\frac{|\Omega|}{d_I}=\frac{d_S}{d_I}|\Omega|>0.
\end{equation}
Thanks to \eqref{GH1} and \eqref{GH2}, we deduce that  there are $l_0<l_1<l_2$ such that $\mathcal{N}_{d_I,d_S}(l_1)=\mathcal{N}_{d_I,d_S}(l_2)$. As a result, for $N=\mathcal{N}_{d_I,d_S}(l_1)=\mathcal{N}_{d_I,d_S}(l_2)$, we have from Lemma \ref{lem3} that 
$$ 
(S_1,I_1)=(l_1(1-d_Iu^{l_1}),d_Sl_1u^{l_2})
\quad 
\text{and} 
\quad  
(S_2,I_2)=(l_2(1-d_Iu^{l_2}),d_Sl_2u^{l_2})
$$
 are two distinct EE solutions of \eqref{e1}. This completes the proof of the theorem.
\end{proof}

Next, we give a proof of Theorem \ref{T2}.

\begin{proof}[Proof of Theorem \ref{T2}] Let $d_I>0$ and $N_{\rm low}(d_I)$ and $m_{d_I}^*$ be given by Theorems \ref{T0} and \ref{T1}, respectively.

\quad {\rm (i)} Suppose that $N_{\rm low}(d_I)<\frac{|\Omega|}{\mathcal{R}(d_I)}$. We proceed by contradiction to show that $m_{d_I}^*<1$. Indeed, if it was the case that $m_{d_I}^*=1$, then for every $d_S>0=d_{I}(1-m_{d_I}^*)$, \eqref{e1} has a (unique) EE if and only if $\mathcal{R}_0(N,d_I)>1$. However, by Theorem \ref{T0}, we know that for every $N_{\rm low}(d_I)<N<\frac{|\Omega|}{\mathcal{R}(d_I)}$, $\mathcal{R}_0(N,d_I)<1$ and \eqref{e1} has at least two EE solution. So, we obtain a contradiction. Hence, we must have that $m_{d_I}^*<1$.

\quad {\rm (ii)} Suppose that $\frac{|\Omega|}{\mathcal{R}(d_I)}>\int_{\Omega}\frac{\gamma}{\beta}$. Then by Theorem \ref{T0}, $N_{\rm low}(d_I)\le \int_{\Omega}\frac{\gamma}{\beta}<\frac{|\Omega|}{\mathcal{R}(d_I)}$. Now, suppose in addition that $\frac{|\Omega|\int_{\Omega}\beta\varphi_{d_I}^3}{\Big[\int_{\Omega}\varphi_{d_I}\Big]\Big[\int_{\Omega}\beta\varphi_{d_I}^2\Big]} >1 $. Then by \eqref{v-l-eq2},  $\frac{d\mathcal{N}_{d_I,d_S}(l)}{dl}\to |\Omega|-\frac{\Big(\int_{\Omega}\varphi_{d_I}\Big)\Big(\int_{\Omega}\beta\varphi_{d_I}^2\Big)}{\int_{\Omega}\beta\varphi_{d_I}^3}>0$ as $l\to\frac{1}{\mathcal{R}(d_I)}$. Thus,  since $$
\lim_{l\to\infty}\mathcal{N}_{d_I}(l)= \int_{\Omega}\frac{\gamma}{\beta}<\frac{|\Omega|}{\mathcal{R}(d_I)}=\mathcal{N}_{d_I}(\frac{1}{\mathcal{R}(d_I)}),$$
then 
$$ 
\max_{l\ge\frac{1}{\mathcal{R}(d_I)}}\mathcal{N}_{d_I}(l)>\max\left\{\lim_{l\to\infty}\mathcal{N}_{d_I}(l), \mathcal{N}_{d_I}\Big(\frac{1}{\mathcal{R}(d_I)}\Big)\right\}.
$$
We can now employ similar arguments as in the proof of Theorem \ref{T0}-{\rm (iii)} to conclude that for every $N\in\Big(\max_{l\ge\frac{1}{\mathcal{R}(d_I)}}\mathcal{N}_{d_I}(l),\mathcal{N}_{d_I}\Big(\frac{1}{\mathcal{R}(d_I)}\Big) \Big)$, there is $\tilde{d}(N,d_I)$ such that \eqref{e1} has at least two EE solutions for every $0<d_S<\tilde{d}(N,d_I)$. 

\quad {\rm (iii)} Suppose that $\frac{|\Omega|\int_{\Omega}\beta\varphi_{d_I}^3}{\Big[\int_{\Omega}\varphi_{d_I}\Big]\Big[\int_{\Omega}\beta\varphi_{d_I}^2\Big]} <1$ and fix $d_S>0$. Then by \eqref{v-l-eq2},  $\frac{d\mathcal{N}_{d_I,d_S}(l)}{dl}\to |\Omega|-\frac{\Big(\int_{\Omega}\varphi_{d_I}\Big)\Big(\int_{\Omega}\beta\varphi_{d_I}^2\Big)}{\int_{\Omega}\beta\varphi_{d_I}^3}<0$ as $l\to\frac{1}{\mathcal{R}(d_I)}$. Hence $N_{\rm low}(d_I)=\inf_{l\ge\frac{1}{\mathcal{R}(d_I)}}\mathcal{N}_{d_I}(l)<\mathcal{N}_{d_I}(\frac{1}{\mathcal{R}(d_I)})=\frac{|\Omega|}{\mathcal{R}(d_I)}.$   
    
\end{proof}




\begin{thebibliography}{9}

%\bibitem{ADW2016} A. S. Ackleh, K. Deng, Y. Wu, Competitive exclusion and coexistence in a two-strain pathogen model with diffusion, {\it Mathematical Biosciences and engineering}, {\bf 13} (2016), 1-18.

\bibitem{Allen2008}L.J.S. Allen, B.M. Bolker, Y. Lou, A.L. Nevai, Asymptotic profiles of the steady states for an SIS epidemic reaction-diffusion model, {\it Discrete Contin. Dyn. Syst.} {\bf 21} (2008), 1-20.

%\bibitem{BT1989}  H. J. Bremermann and H. R.  Thieme, A competitive exclusion principle for pathogen virulence. {\it J. Math. Biol.} {\bf 27} (1989), 179-190.

%\bibitem{CC2003} R. S. Cantrell, C. Cosner, Spatial Ecology via Reaction-Diffusion Equations, {\it Series in Mathematical and Computational Biology}, John Wiley and Sons, Chichester, UK, 2003.


\bibitem{CastellanoSalako2021} K. Castellano, R. B. Salako, On the effect of lowering population's movement to control the spread of infectious disease  {\it Journal of Differential Equations}, {\bf316} (2022), 1-27. 

\bibitem{CR1971}  M. G. Crandall, P. H. Rabinowitz, {Bifurcation from simple eigenvalues},  {J. Funct. Anal.}, {\bf 8} (1971), {321–340}.

\bibitem{Cui_Lou2016}R. Cui, Y. Lou, A spatial SIS model in advective heterogeneous environments, {\it J. Differential Equations}, {\bf 261}
(2016), 3305-3343.



\bibitem{Cui2017} R. Cui, K.-Y. Lam, Y. Lou, Dynamics and asymptotic profiles of steady states of an epidemic model in advective environments, {\it J. Differential Equations}, {\bf263} (2017), 2343-2373.

%

\bibitem{DengWu2016}K. Deng, Y. Wu, Dynamics of a susceptible-infected-susceptible epidemic reaction-diffusion model, {\it  Proc. Roy.
Soc. Edinburgh Sect. A}, {\bf 146} (2016), 929-946.



\bibitem{GKLZ2015}J. Ge, K.I. Kim, Z. Lin, H. Zhu, A SIS reaction-diffusion-advection model in a low-risk and high-risk domain,
{\it J. Differential Equations}, {\bf259} (2015), 5486-5509.

%\bibitem{Henry} D. Henry, Geometric Theory of Semilinear Parabolic Equations, {\it Springer-Verlag Berlin Heidelberg}, 1981.

%\bibitem{Hess} P. Hess,  Periodic-parabolic Boundary Value Problems and Positivity,  Pitman Res. Notes in Mathematics  247, Longman Sci. Tech., Harlow, 1991.

%\bibitem{HLMV1995} V. Hutson et al., Limit behavior for a competing species problem with diffusion, Dy. Sys. and Appl., World Scientific Publishing Company, {\bf 4} (1995), 343-358.

\bibitem{DeJong1995} M. C. M. de Jong, et al., How does transmission of infection depend on population size?, in Epidemic Models: Their Structure and Relation to Data, {\it Cambridge University Press}, 1995, pp.84–89.

\bibitem{LP2022} H. Li, R. Peng, An SIS epidemic model with mass action infection mechanism in a patchy environment, Studies in Applied Mathematics, {\bf 150} (2022), 650 - 704.

%\bibitem{LouSalako2021} Y. Lou, R. B. Salako, Control Strategy for multiple strains epidemic model, {\it Bulletin of Mathematical Biology}, {\bf 84} 10 (2022), p.1-47.

\bibitem{MCH2001} H. McCallum, N. Barlow, J. Hone, How should pathogen transmission be modelled?, {\it Trends Ecol. Evol.} {\bf16} (2001) 295–300.



\bibitem{Peng2009a} R. Peng, Asymptotic profiles of the positive steady state for an SIS epidemic reaction-diffusion model. Part I, {\it J. Differential Equations}, {\bf 247} (2009), 1096-1119.

\bibitem{Peng2009b} R. Peng, S. Liu, Global stability of the steady states of an SIS epidemic reaction-diffusion model, {\it Nonlinear Anal.} {\bf 71} (2009), 239-247.

\bibitem{Peng_Shi2008} R. Peng, J. Shi, M. Wang, On stationary patterns of a reaction-diffusion model with autocatalysis and saturation law, {\it Nonlinearity}, 21 (2008), 1471-1488.

\bibitem{Peng_Yi2013} R. Peng, F. Yi, Asymptotic profile of the positive steady state for an SIS epidemic reaction-diffusion model:
Effects of epidemic risk and population movement, {\it Phys. D}, {\bf 259} (2013), 8-25.

\bibitem{Peng_Zhao} R. Peng, X. Zhao, A reaction-diffusion SIS epidemic model in a time-periodic environment, Nonlinearity, 25 (2012), 1451-1471.

\bibitem{TW2023}Y. Tao, M. Winkler, Analysis of a chemotaxis-SIS epidemic model with unbounded infection force, Nonlinear Analysis: Real World Applications, {\bf  71} (2023), 103820

%
\bibitem{Wu_Zou2016} Y. Wu and Z. Zou, Asymptotic profiles of steady states for a diffusive SIS epidemic model with mass action infection mechanism, {\it J. Diff. Equat.} {\bf 261}(2016) 4424–4447. 

\bibitem{Wen2018}X. Wen, J. Ji, B. Li, Asymptotic profiles of the endemic equilibrium to a diffusive SIS epidemic model with mass action infection mechanism, {\it J. Math. Anal. Appl.} {\bf 458} (2018), 715-729.




\end{thebibliography}

\end{document}



\begin{proof}[Proof of Theorem \ref{T0}]
    It follows from Theorem \ref{T3}-{\rm (i)} and {\rm (ii)}. 
\end{proof}

\begin{lem}\label{lem4}
For every $d_I>0$, let $\varphi_{d_I}$ denote the unique positive solution of \eqref{R-star-pde} satisfying $\|\varphi_{d_I}\|_{L^2(\Omega)}=1$. Then \begin{equation}\label{lem4-eq1}
    \lim_{d_{I}\to 0^+}\|\varphi_{d_I}\|_{C(K)}=0 \quad \text{for all compact subsets}\ K\subset \Omega^*:=\Big\{x\in\Omega\ :\ \frac{\gamma(x)}{\beta(x)}>\Big(\frac{\gamma}{\beta}\Big)_{\min}\Big\}.
\end{equation}
 In particular, if  \eqref{lem4-eq2} holds, then there is $d_I^{up}>0$ such that 
\begin{equation}\label{lem4-eq3}
    |\Omega|>\frac{\|\varphi_{d_I}\|_{L^1(\Omega)}\|\beta\varphi_{d_I}^2\|_{L^1(\Omega)}}{\|\beta\varphi_{d_I}^3\|_{L^1(\Omega)}}\quad \forall\ 0<d_I<d_{I}^{**}.
\end{equation}
    
\end{lem}
\begin{proof} Let $K\subset\subset \tilde{K}\subset\subset \Omega^*$. Since $\varphi_{d_I}$ satisfies \eqref{R-star-pde} and \eqref{R-star-limit-eq} holds, then there exist $\delta_0>0$ and $d_0>0$ such that 
\begin{equation}\label{lem4-proof-eq1}
\frac{\beta(x)}{\mathcal{R}(d_I)}-\gamma(x)<-\delta_0 \quad \forall\ x\in \tilde{K}, \ 0<d_{I}<d_0
\end{equation}
which implies that $\varphi_{d_I}$ is nonnegative and subharmonic on $\tilde{K}$ for every $0<d_I<d_0$. Hence, by the maximum principle for subharmonic equations, there is positive constant $m_0$ such that 
\begin{equation}\label{lem4-proof-eq2}
    \|\varphi_{d_I}\|_{C(K)}\le m_0\int_{K}\varphi_{d_I}\quad \forall\ 0<d_I<d_0.
\end{equation}
Next, chose a smooth function $\psi\in C^{\infty}_{c}(\Omega)$ such that,  $\psi= 0$ on $\Omega\setminus\tilde{K}$, $\psi= 1$ on $K$ and $0\le \psi\le 1$.  In view of \eqref{lem4-proof-eq1}, after multiplying \eqref{R-star-pde} by $\psi$ and integrate by parts on $\Omega$, we obtain that 
\begin{equation*}
   \delta_0\int_{K}\varphi_{d_I}\le \delta_0\int_{\Omega}\psi\varphi_{d_I}\le d_I\||\Delta \psi|\|_{L^2(\Omega)}\|\varphi_{d_I}\|_{L^2(\Omega)}= d_I\||\Delta \psi|\|_{L^2(\Omega)}\to 0 \text{as}\ d_I\to 0.
\end{equation*}
This together with \eqref{lem4-proof-eq2} yields \eqref{lem4-eq1}.  

\quad Finally suppose that \eqref{lem4-eq2} holds and we proceed by contradiction to establish that \eqref{lem4-eq3} holds. So, suppose that that there is sequence of posiyive numbers $\{d_{I,n}\}_{n\ge 1}$ converging to zero such that 
\begin{equation}\label{lem4-proof-eq3}
    |\Omega|\le \liminf_{n\to\infty}\frac{\|\varphi_{d_{I,n}}\|_{L^1(\Omega)}\|\beta\varphi_{d_{I,n}}\|_{L^1(\Omega)}}{\|\beta\varphi_{d_{I,n}}^2\|_{L^1(\Omega)}}.
\end{equation}
 Next, since $L^2(\Omega)$ is a reflexive space, and $\|\varphi_{d_I}\|_{L^2(\Omega)}=1$ for every $d_I>0$, then after passing to a  further subsequence, we may suppose without loss of generality that there is $\varphi_0\in L^2(\Omega)$, greater of equal to zero almost every, such that $\varphi_{d_{I,n}}\to \varphi_0$ as $d_I\to 0$ weakly in $L^2(\Omega)$. In view of \eqref{lem4-eq1}, we have that $\varphi_0=0$ almost everywhere on $\Omega^*$. Hence, using the fact that 
 $$
 \|\varphi_0\|_{L^2(\Omega)}\le \lim_{n\to\infty}\|\varphi_{d_{I,n}}\|_{L^2(\Omega)}\le 1
 $$
 and Holder's inequality, we obtain from \eqref{lem4-proof-eq3} that
 \begin{align*}
|\Omega|\le \liminf_{n\to\infty}\frac{\|\varphi_{d_{I,n}}\|_{L^1(\Omega)}\|\beta\varphi_{d_{I,n}}\|_{L^1(\Omega)}}{\|\beta\varphi_{d_{I,n}}^2\|_{L^1(\Omega)}}\le &  \liminf_{n\to\infty}\frac{\|\varphi_{d_{I,n}}\|_{L^1(\Omega)}\|\beta\varphi_{d_{I,n}}\|_{L^1(\Omega)}}{\beta_{\min}\|\varphi_{d_{I,n}}\|^2_{L^2(\Omega)}}\cr
 =&\lim_{n\to\infty}\frac{\|\varphi_{d_{I,n}}\|_{L^1(\Omega)}\|\beta\varphi_{d_{I,n}}\|_{L^1(\Omega)}}{\beta_{\min}}\cr
=&\frac{\|\varphi_{0}\|_{L^1(\Omega\setminus\Omega^*)}\|\beta\varphi_{0}\|_{L^1(\Omega\setminus\Omega^*)}}{\beta_{\min}}\cr
\le & \frac{|\Omega\setminus\Omega^*|^{\frac{1}{2}}\|\beta\|_{L^2(\Omega\setminus\Omega^*)}\|\varphi_0\|_{L^2(\Omega)}^2}{\beta_{\min}}\cr 
\le &\frac{|\Omega\setminus\Omega^*|^{\frac{1}{2}}\|\beta\|_{L^2(\Omega\setminus\Omega^*)}}{\beta_{\min}},
 \end{align*}
 contradict with our standing assumption \eqref{lem4-eq2}. Therefore, \eqref{lem4-eq3} must hold.


%Furthermore, for every up to a subsequence, there is $\varphi_{0}\in L^{2}(\Omega )$ such that $\varphi_{d_I}$ converges to $\varphi_{0}$ weakly in $L^2(\Omega)$ as $d_{I}\to 0$.

    
\end{proof}



 %However, it is not easy to check the validity of \eqref{T1-eq2} due to the fact that $\varphi_{d_I}$ depends implicitly in on $d_{I}$. We give a sufficient condition on the parameters $\beta$ and $\gamma$ which guarantee that \eqref{T1-eq2} holds for sufficiently small values of $d_I$.

%\begin{prop} Define $\Omega^*:=\Big\{x\in\Omega\ :\ \frac{\gamma(x)}{\beta(x)}>\Big(\frac{\gamma}{\beta}\Big)_{\min}\Big\}$. If the inequality \begin{equation}\label{lem4-eq2}|\Omega\setminus\Omega^*|\int_{\Omega\setminus\Omega^*}\beta^2<(|\Omega|\beta_{ \min})^2\end{equation}holds, then there is $d_I^{\rm up}>0$ such that \eqref{T1-eq2} for all $0<d_I<d_I^{\rm up}$.\end{prop}Note also that \eqref{T1-eq2} holds {\color{blue}comment of the result}. The next result gives the asymptotic profiles of EE solutions of \eqref{e1} when the basic reproduction number is less than one.

       
%This result has incited several work [cite] to investigate the extent to which such result hold for \eqref{e1}. Surprisingly, it turns out that \eqref{e1} may have multiple EE solutions. In fact, basing the prediction of disease persistence on the mathematical model \eqref{e1}, we shall show below that it is possible for the disease to become endemic even when $\mathcal{R}_0(N,d_I)<1$. To facilitate our discussion, it seems convenient to start with the general following result  on the existence/nonexistence of EE solution of \eqref{e1}. 
    
    %Furthermore, if $\mathcal{R}_0(N,d_I)>1$, then \eqref{e1} has at least one EE solution for every $d_S>0$.

     %\item[\rm (ii)] For every $d_S>0$ and $d_I>0$, there is $N^*_{d_S}(d_I)\ge \frac{|\Omega|}{\mathcal{R}(d_I)}$ such that \eqref{e1} has no EE solution if $\mathcal{R}_0(N,d_I)\le \frac{1}{1+\Big(\frac{d_I}{d_S}-1\Big)_+}$,  while it has a unique EE solution if $\mathcal{R}_0(N,d_I)>\mathcal{R}_0(N^*_{d_S}(d_I),d_I)$. Furthermore, if $d_S\ge d_I$, then  $N^*_{d_S}(d_I)=\frac{|\Omega|}{\mathcal{R}(d_I)}$.
     
%First, we note that the existence of at least one EE solution of \eqref{e1} when $\mathcal{R}_0(N,d_I)>1$ was already established in the previous work \cite{CastellanoSalako2021,DengWu2016}. Moreover, when $d_S\ge d_S$ the fact that $N_{\rm low}(d_I)=\frac{|\Omega|}{\mathcal{R}(d_I)}$ follows from \cite{CastellanoSalako2021,DengWu2016}. However, the existence and positive of $N_{\rm low}(d_I)$ and the fact it is independent of the diffusion rate of the susceptible population as stated in Theorem \ref{T0}-(i) seems to be new. Moreover, it helps to recast the question of existence EE of \eqref{e1} when $\mathcal{R}_0(N,d_I)<1$, in the simple form : Is it possible to have $N_{\rm low}(d_I)<\frac{|\Omega|}{\mathcal{R}(d_I)}$? Note that a positive answer to this question will immediately imply that, on the one hand, the existence of multiple EE solutions of \eqref{e1}, and the other hand, the DFE may fail to be globally stable even when $\mathcal{R}_0(N,d_I)<1$. The later result, if establish, highlights another significant disparity between the prediction on disease control using the mathematical models \eqref{e1} and \eqref{e1-prime}. Indeed, thanks to \cite{Allen2008,Cui2017}, it known that the DFE is globally stable for classical solution of \eqref{e1-prime}, when $\mathcal{R}(d_I)\le 1$.

    %\item[\rm (ii)] If \begin{equation}\label{T1-eq2}    \frac{\int_{\Omega}\varphi_{d_I}}{\int_{\Omega}\beta\varphi_{d_I}^2}<\frac{|\Omega|}{\int_{\Omega}\varphi_{d_I}\beta},    \end{equation}  then $N_{\rm low}(d_I)<\frac{|\Omega|}{\mathcal{R}(d_I)}$ where $N_{\rm low}(d_I)$ is given in Theorem \ref{T0}-{\rm (i)}. In particular for every $d_I>\mathcal{R}^{-1}\Big(\frac{|\Omega|}{\int_{\Omega}\frac{\gamma}{\beta}}\Big)$ and $\mathcal{R}_0(N_{\rm low}(d_I),d_I) <\mathcal{R}_0(N,d_I)<1$, there is $d_S^N>0$ such that \eqref{e1} has  at least two EE solutions for every $0<d_S<d_S^N$.\end{itemize}