\documentclass[issupp,10pt]{wlscirep}
\usepackage[utf8]{inputenc}
\usepackage[T1]{fontenc}
\usepackage{amsmath}
\usepackage{bm}
\usepackage{multirow}
\usepackage[draft]{pgf}
\usepackage{hyperref}
\usepackage{makecell}
\usepackage{float}
% \usepackage{lineno}
% \linenumbers
\newcommand\setcurrentname[1]{\def\@currentlabelname{#1}}
\DeclareMathAlphabet{\mathcal}{OMS}{cmsy}{m}{n}

\makeatletter
\def\thickhline{%
  \noalign{\ifnum0=`}\fi\hrule \@height \thickarrayrulewidth \futurelet
   \reserved@a\@xthickhline}
\def\@xthickhline{\ifx\reserved@a\thickhline
    \vskip\doublerulesep
    \vskip-\thickarrayrulewidth
    \fi
    \ifnum0=`{\fi}}
\makeatother
\newlength{\thickarrayrulewidth}
\setlength{\thickarrayrulewidth}{4\arrayrulewidth}

% \newcommand{\makesupptitle}{%
% {%
% \thispagestyle{empty}%
% \vskip-36pt%
% {\raggedright\sffamily\bfseries\fontsize{20}{25}\selectfont \@title\par}%
% \vskip10pt
% {\raggedright\sffamily\fontsize{12}{16}\selectfont  \@author\par}
% \vskip25pt%
% }%
% }%


\title{An Empirical Study of Large-Scale Data-Driven Full Waveform Inversion}

\author[1,2, *]{Peng Jin}
\author[1]{Yinan Feng}
\author[1]{Shihang Feng}
\author[1]{Hanchen Wang}
\author[3]{Yinpeng Chen}
\author[4]{\\Benjamin Consolvo}
\author[5]{Zicheng Liu}
\author[6,*]{Youzuo Lin}
\affil[1]{Earth and Environmental Sciences Division, Los Alamos National Laboratory}
\affil[2]{College of Information Sciences and Technology, The Pennsylvania State University}
\affil[3]{Google Research}
\affil[4]{Intel}
\affil[5]{Microsoft}
\affil[6]{School of Data Science and Society, The University of North Carolina at Chapel Hill}
\affil[*]{Corresponding Authors: pqj5125@psu.edu, yzlin@unc.edu}

\begin{document}
\maketitle

\section*{Supplementary}
\renewcommand{\thetable}{S\arabic{table}}
\renewcommand{\thefigure}{S\arabic{figure}}

\color{black}
\subsection*{Wasserstein Distance}
We follow the previous literature~\cite{yang2018application} to calculate the squared 2-Wasserstein Distance. The definition of the distance between two distributions $f:X\rightarrow \mathbb{R}^+$ and $g:Y\rightarrow \mathbb{R}^+$ can be formulated as: 
\begin{equation}
    W_2^2(f, g)=\inf_{T\in \mathcal{M}}\int\limits_{X} |x-T(x)|^2 f(x)dx,
\end{equation}
where $f$ and $g$ denotes the probability density functions of the distributions, $T:X\rightarrow Y$ is a transport plan, and $\mathcal{M}$ is the set of all possible transport plans.

For seismic data, we compute the average trace-by-trace distance. Each trace is considered as a density function. Additionally, we extract the envelope of data by applying the Hilbert transform to generate non-negative density functions. We also normalize the data of both ground truth and predictions so that the functions have equal mass, as required by Wasserstein Distance. We use $\tilde{f}$ and $\tilde{g}$ to denote the modified traces of the ground truth and the predictions. The average trace-by-trace distance is given as:
\begin{equation}
    \mathrm{WD}_{Seis}= \frac{1}{S\cdot R}  \sum_{s=1}^{S} \sum_{r=1}^{R} W_2^2(\tilde{f}(r,t;s), \tilde{g}(r,t;s)),
\end{equation}
where $S$ is the number of sources, and $R$ is the number of receivers.

Similarly, for velocity maps, we consider a map as a joint probability density function and estimate the distances by computing the Sliced Wasserstein Distances with 50 projections~\cite{bonneel2015sliced}.

\subsection*{Enlarged Single Dataset}
We further conduct an experiment by simply enlarging the split datasets in OpenFWI. Specifically, we follow the velocity map generation process in OpenFWI and generate additional pairs of velocity maps and seismic data for the \textit{Vel Family} and the \textit{Fault Family} so that each split dataset contains 204K training samples. The test sets remain unchanged. For brevity, we name the datasets as OpenFWI-204K-Extended, and the models trained on the enlarged datasets are still referred as BigFWI for consistency. For the \textit{Style Family}, there exist some factors that may cause distribution shift. For instance, the specific Marmousi velocity map used as the style image is unknown. Therefore, we choose to exclude the \textit{Style Family} in this experiment. The hyperparameter setting is the same as the one in the experiment of OpenFWI-204K. 

The quantitative results are listed in Table~\ref{tab:204k-se} and \ref{tab:204k-se-wd}. We observe that BigFWI outperforms InversionNet on all the datasets by a large extent, which still demonstrates the impact of data scaling. It is also worth noting that the performance improvement from enlarging single datasets is larger than the one of combining all the datasets. However, training on the single dataset may also limit the generalization of the model as the velocity maps with other subsurface structures are unseen during training.

% whether compare to the results OpenFWI-408k or not

% Figure environment removed



\begin{table}[t]
\centering
\setlength{\tabcolsep}{6pt}
\renewcommand{\arraystretch}{1.25}
\begin{tabular}{c c c c c }
\thickhline
% Model & \makecell{Encode\\Layers} & \makecell{Decode\\Layers} & \makecell{Latent\\Length} & Parameters 
Model & Encode Layers & Decode Layers & Latent Length & Parameters \\ \thickhline
BigFWI-B & 14 & 11 & 512 & 24M \\ \hline
BigFWI-M & 17 & 16 & 512 & 28M \\ \hline
BigFWI-L & 16 & 14 & 1024 & 87M \\ \thickhline
\end{tabular}
\caption{Configurations of BigFWI model variants.}
\label{tab:model}
\end{table}

\begin{table}[t]
\centering
\setlength{\tabcolsep}{10pt}
\renewcommand{\arraystretch}{1.5}
\begin{tabular}{c|c c c|c c c}
\thickhline
\multirow{2}{*}{Dataset} & \multicolumn{3}{c|}{InversionNet} & \multicolumn{3}{c}{BigFWI} \\ \cline{2-7} 
& MAE$\downarrow$ & RMSE$\downarrow$ & SSIM$\uparrow$  & MAE$\downarrow$ & RMSE$\downarrow$ & SSIM$\uparrow$ \\ \thickhline
FlatVel-A  & 0.0092 & 0.0163 & \textbf{0.9930} & \textbf{0.0066} & \textbf{0.0137} & 0.9922 \\ \hline
FlatVel-B  & \textbf{0.0288} & \textbf{0.0797} & \textbf{0.9582} & 0.0298 & 0.0888 & 0.9498 \\ \hline
CurveVel-A  & 0.0560 & 0.1131 & 0.8405 & \textbf{0.0415} & \textbf{0.0936} & \textbf{0.8751}  \\ \hline
CurveVel-B  & 0.1344 & 0.2697 & 0.6995 & \textbf{0.1155} & \textbf{0.2471} & \textbf{0.7309}  \\ \hline
FlatFault-A  & 0.0138 & 0.0382 & 0.9813 & \textbf{0.0111} & \textbf{0.0332} & \textbf{0.9841}  \\ \hline
FlatFault-B  & 0.1012 & 0.1672 & 0.7267 & \textbf{0.0855} & \textbf{0.1514} & \textbf{0.7573}  \\ \hline
CurveFault-A  & 0.0220 & 0.0607 & 0.9605 & \textbf{0.0193} & \textbf{0.0556} & \textbf{0.9636}  \\ \hline
CurveFault-B  & 0.1610 & 0.2419 & 0.5981 & \textbf{0.1425} & \textbf{0.2251} & \textbf{0.6292}  \\ \hline
Style-A  & 0.0625 & 0.1024 & 0.8839 & \textbf{0.0592} & \textbf{0.0994} & \textbf{0.8908}  \\ \hline
Style-B & 0.0609 & 0.0971 & 0.7303 & \textbf{0.0600} & \textbf{0.0962} & \textbf{0.7324}  \\ \thickhline
\end{tabular}
\caption{Quantitative comparison between the results of InversionNet and BigFWI on OpenFWI-204k in terms of MAE, RMSE and SSIM.}
\label{tab:204k}
\end{table}

\begin{table}[t]
\centering
\setlength{\tabcolsep}{10pt}
\renewcommand{\arraystretch}{1.5}
\begin{tabular}{c|c c|c c}
\thickhline
\multirow{2}{*}{Dataset} & \multicolumn{2}{c|}{InversionNet} & \multicolumn{2}{c}{BigFWI} \\ \cline{2-5} 
 & WD$_{Seis}\downarrow$ & WD$_{Vmap}\downarrow$  & WD$_{Seis}\downarrow$ & WD$_{Vmap}\downarrow$  \\ \thickhline
FlatVel-A  & 15.8109 & 0.0417 & \textbf{10.4228} & \textbf{0.0372} \\ \hline
FlatVel-B  & 79.0238 & 0.1339 & \textbf{76.6044} & \textbf{0.1328} \\ \hline
CurveVel-A & 478.6276 & 0.1898 & \textbf{291.5563} & \textbf{0.1192} \\ \hline
CurveVel-B & 997.6450 & 0.4120 & \textbf{819.0640} & \textbf{0.3354} \\ \hline
FlatFault-A & \textbf{150.4610} & 0.0481 & 170.1272 & \textbf{0.0436} \\ \hline
FlatFault-B & 722.4929 & 0.3139 & \textbf{544.0522} & \textbf{0.2260} \\ \hline
CurveFault-A & 264.9666 & 0.0692 & \textbf{260.3809} & \textbf{0.0577} \\ \hline
CurveFault-B & 1342.0856 & 0.4040 & \textbf{992.1356} & \textbf{0.3091} \\ \hline
Style-A & 156.1971 & 0.1967 & \textbf{145.9914} & \textbf{0.1691} \\ \hline
Style-B & \textbf{131.8703} & 0.0994 & 182.7130 & \textbf{0.0934} \\ \thickhline
\end{tabular}
\caption{\color{black}Quantitative comparison between the results of InversionNet and BigFWI on OpenFWI-204k in terms of Wasserstein Distance.}
\label{tab:204k-wd}
\end{table}


\begin{table}[t]
\centering
\setlength{\tabcolsep}{10pt}
\renewcommand{\arraystretch}{1.5}
\begin{tabular}{c|c c c|c c c}
\thickhline
\multirow{2}{*}{Dataset} & \multicolumn{3}{c|}{InversionNet} & \multicolumn{3}{c}{BigFWI} \\ \cline{2-7} 
& MAE$\downarrow$ & RMSE$\downarrow$ & SSIM$\uparrow$  & MAE$\downarrow$ & RMSE$\downarrow$ & SSIM$\uparrow$ \\ \thickhline
FlatVel-A  & 0.0092 & 0.0163 & 0.9930 & \textbf{0.0035} & \textbf{0.0056} & \textbf{0.9985} \\ \hline
FlatVel-B  & 0.0288 & 0.0797 & 0.9582 & \textbf{0.0116} & \textbf{0.0421} & \textbf{0.9867} \\ \hline
CurveVel-A  & 0.0560 & 0.1131 & 0.8405 & \textbf{0.0236} & \textbf{0.0703} & \textbf{0.9198}  \\ \hline
CurveVel-B  & 0.1344 & 0.2697 & 0.6995 & \textbf{0.0704} & \textbf{0.1953} & \textbf{0.8209}  \\ \hline
FlatFault-A  & 0.0138 & 0.0382 & 0.9813 & \textbf{0.0048} & \textbf{0.0196} & \textbf{0.9952}  \\ \hline
FlatFault-B  & 0.1012 & 0.1672 & 0.7267 & \textbf{0.0598} & \textbf{0.1288} & \textbf{0.8366}  \\ \hline
CurveFault-A  & 0.0220 & 0.0607 & 0.9605 & \textbf{0.0080} & \textbf{0.0324} & \textbf{0.9910}  \\ \hline
CurveFault-B  & 0.1610 & 0.2419 & 0.5981 & \textbf{0.1132} & \textbf{0.1964} & \textbf{0.6937}  \\ \hline
\end{tabular}
\caption{\color{black}Quantitative comparison between the results of InversionNet and BigFWI on OpenFWI-204k-Extended in terms of MAE, RMSE and SSIM.}
\label{tab:204k-se}
\end{table}

\begin{table}[t]
\centering
\setlength{\tabcolsep}{10pt}
\renewcommand{\arraystretch}{1.5}
\begin{tabular}{c|c c|c c}
\thickhline
\multirow{2}{*}{Dataset} & \multicolumn{2}{c|}{InversionNet} & \multicolumn{2}{c}{BigFWI} \\ \cline{2-5} 
 & WD$_{Seis}\downarrow$ & WD$_{Vmap}\downarrow$  & WD$_{Seis}\downarrow$ & WD$_{Vmap}\downarrow$  \\ \thickhline
FlatVel-A  & 15.8109 & 0.0417 & \textbf{2.6449} & \textbf{0.0355} \\ \hline
FlatVel-B  & 79.0238 & 0.1339 & \textbf{16.8437} & \textbf{0.0634} \\ \hline
CurveVel-A & 478.6276 & 0.1898 & \textbf{122.2878} & \textbf{0.0859} \\ \hline
CurveVel-B & 997.6450 & 0.4120 & \textbf{393.6323} & \textbf{0.2119} \\ \hline
FlatFault-A & 150.4610 & 0.0481 & \textbf{66.2190} & \textbf{0.0284} \\ \hline
FlatFault-B & 722.4929 & 0.3139 & \textbf{276.2333} & \textbf{0.1632} \\ \hline
CurveFault-A & 264.9666 & 0.0692 & \textbf{119.6634} & \textbf{0.0381} \\ \hline
CurveFault-B & 1342.0856 & 0.4040 & \textbf{696.8380} & \textbf{0.2314} \\ \hline
\end{tabular}
\caption{\color{black}Quantitative comparison between the results of InversionNet and BigFWI on OpenFWI-204k-Extended in terms of Wasserstein Distance.}
\label{tab:204k-se-wd}
\end{table}


\begin{table}[t]
\small
\centering
\setlength{\tabcolsep}{5pt}
\renewcommand{\arraystretch}{1.5}
\begin{tabular}{c|c c c|c c c|c c c|c c c}
\thickhline
\multirow{2}{*}{Dataset} & \multicolumn{3}{c|}{InversionNet} & \multicolumn{3}{c|}{BigFWI-B} & \multicolumn{3}{c|}{BigFWI-M} & \multicolumn{3}{c}{BigFWI-L} \\ \cline{2-13} 

& MAE$\downarrow$ & RMSE$\downarrow$ & SSIM$\uparrow$  & MAE$\downarrow$ & RMSE$\downarrow$ & SSIM$\uparrow$ & MAE$\downarrow$ & RMSE$\downarrow$ & SSIM$\uparrow$ & MAE$\downarrow$ & RMSE$\downarrow$ & SSIM$\uparrow$ \\ \thickhline

FlatVel-A & 0.0055 & 0.0104 & 0.9964  & 0.0076 & 0.0130 & 0.9943  & 0.0045 & 0.0085 & 0.9965 & \textbf{0.0041} & \textbf{0.0079} & \textbf{0.9965} \\ \hline
FlatVel-B & 0.0210 & 0.0632 & 0.9718  & 0.0233 & 0.0696 & 0.9658  & 0.0193 & 0.0621 & 0.9729 & \textbf{0.0173} & \textbf{0.0584} & \textbf{0.9756} \\ \hline
CurveVel-A & 0.0409 & 0.0944 & 0.8796  & 0.0343 & 0.0798 & 0.9027  & 0.0272 & 0.0725 & 0.9180 & \textbf{0.0260} & \textbf{0.0705} & \textbf{0.9199} \\ \hline
CurveVel-B & 0.1073 & 0.2349 & 0.7527  & 0.0933 & 0.2154 & 0.7808  & 0.0816 & 0.2006 & 0.8053 & \textbf{0.0772} & \textbf{0.1947} & \textbf{0.8134} \\ \hline
FlatFault-A & 0.0096 & 0.0278 & 0.9880  & 0.0106 & 0.0286 & 0.9871  & 0.0075 & 0.0229 & 0.9904 & \textbf{0.0066} & \textbf{0.0208} & \textbf{0.9918} \\ \hline
FlatFault-B & 0.0843 & 0.1497 & 0.7635  & 0.0710 & 0.1321 & 0.8027  & \textbf{0.0636} & \textbf{0.1259} & \textbf{0.8137} & 0.0644 & 0.1269 & 0.8033 \\ \hline
CurveFault-A & 0.0164 & 0.0485 & 0.9712  & 0.0167 & 0.0474 & 0.9712  & 0.0130 & 0.0404 & 0.9771 & \textbf{0.0117} & \textbf{0.0369} & \textbf{0.9801} \\ \hline
CurveFault-B & 0.1444 & 0.2248 & 0.6274  & 0.1245 & 0.2027 & 0.6781  & \textbf{0.1161} & \textbf{0.1954} & \textbf{0.6896} & 0.1169 & 0.1960 & 0.6790 \\ \hline
Style-A & 0.0567 & 0.0947 & 0.8972  & 0.0514 & 0.0868 & 0.9125  & \textbf{0.0480} & \textbf{0.0829} & \textbf{0.9187} & 0.0483 & 0.0831 & 0.9136 \\ \hline
Style-B & 0.0542 & 0.0890 & 0.7646  & 0.0553 & 0.0876 & 0.7567  & \textbf{0.0538} & \textbf{0.0867} & \textbf{0.7600} & 0.0563 & 0.0908 & 0.7429 \\ \thickhline
\end{tabular}
\caption{Quantitative comparison between the results of InversionNet and BigFWIs on OpenFWI-408k in terms of MAE, RMSE and SSIM.}
\label{tab:408k}
\end{table}

\begin{table}[t]
\small
\centering
\setlength{\tabcolsep}{5pt}
\renewcommand{\arraystretch}{1.5}
\begin{tabular}{c|c c|c c|c c|c c}
\thickhline
\multirow{2}{*}{Dataset} & \multicolumn{2}{c|}{InversionNet} & \multicolumn{2}{c|}{BigFWI-B} & \multicolumn{2}{c|}{BigFWI-M} & \multicolumn{2}{c}{BigFWI-L} \\ \cline{2-9} 

& WD$_{Seis}\downarrow$ & WD$_{Vmap}\downarrow$  & WD$_{Seis}\downarrow$ & WD$_{Vmap}\downarrow$ & WD$_{Seis}\downarrow$ & WD$_{Vmap}\downarrow$  & WD$_{Seis}\downarrow$ & WD$_{Vmap}\downarrow$ \\ \thickhline

FlatVel-A    & 7.1659    & 0.0363  & 6.5469     & 0.0319 & \textbf{3.9796}     & 0.0299 & 4.2737     & \textbf{0.0273} \\ \hline
FlatVel-B    & 44.9913   & 0.1059  & 56.9304    & 0.1017 & 37.7425    & 0.0901 & \textbf{34.0136}    & \textbf{0.0875} \\ \hline
CurveVel-A   & 256.9994  & 0.1335  & 215.5561   & 0.0964 & 160.5040   & 0.0851 & \textbf{154.5497}   & \textbf{0.0780} \\ \hline
CurveVel-B   & 732.5136  & 0.3340  & 607.3340   & 0.2661 & 537.0858   & 0.2378 & \textbf{498.6442}   & \textbf{0.2216} \\ \hline
FlatFault-A  & 96.9786   & 0.0358  & 126.8728   & 0.0364 & 90.7467    & 0.0312 & \textbf{80.5870}    & \textbf{0.0293} \\ \hline
FlatFault-B  & 521.3496  & 0.2458  & 440.6822   & 0.1838 & \textbf{345.5444}   & 0.1662 & 362.3163   & \textbf{0.1468} \\ \hline
CurveFault-A & 186.2397  & 0.0507  & 218.2358   & 0.0449 & 166.9954   & 0.0421 & \textbf{152.6674}   & \textbf{0.0373} \\ \hline
CurveFault-B & 1044.2372 & 0.3357  & 891.3471   & 0.2604 & \textbf{773.0494}   & 0.2367 & 773.9282   & \textbf{0.2179} \\ \hline
Style-A      & 125.4812  & 0.1617  & 126.1750   & 0.1426 & \textbf{113.1486}   & 0.1301 & 121.2802   & \textbf{0.1235} \\ \hline
Style-B      & \textbf{96.8782}   & 0.0866  & 169.6936   & 0.0828 & 169.4446   & 0.0787 & 199.0436   & \textbf{0.0784} \\ \thickhline
\end{tabular}
\caption{\color{black}Quantitative comparison between the results of InversionNet and BigFWIs on OpenFWI-408k in terms of Wasserstein Distance.}
\label{tab:408k-wd}
\end{table}


\begin{table}[t]
\centering
\setlength{\tabcolsep}{10pt}
\renewcommand{\arraystretch}{1.5}
\begin{tabular}{c|c c c c|c c c}
\thickhline
\multirow{2}{*}{\makecell{Target\\Dataset}} & \multicolumn{4}{c|}{InversionNet} & \multicolumn{3}{c}{BigFWI} \\ \cline{2-8} 
& Source & MAE$\downarrow$ & RMSE$\downarrow$ & SSIM$\uparrow$  & MAE$\downarrow$ & RMSE$\downarrow$ & SSIM$\uparrow$ \\ \thickhline
FlatVel-A & FlatVel-B & 0.0207 & 0.0381 & 0.9723 & \textbf{0.0137} & \textbf{0.0285} & \textbf{0.9795} \\ \hline
FlatVel-B & CurveVel-B & 0.1076 & 0.2331 & 0.7797 & \textbf{0.0820} & \textbf{0.1970} & \textbf{0.8345} \\ \hline
CurveVel-A & CurveVel-B & 0.0833 & 0.1458 & 0.7828 & \textbf{0.0578} & \textbf{0.1114} & \textbf{0.8404} \\ \hline
CurveVel-B & FlatFault-B & 0.4267 & 0.5649 & 0.4234 & \textbf{0.2543} & \textbf{0.4042} & \textbf{0.5373} \\ \hline
FlatFault-A & CurveFault-A & 0.0394 & 0.0979 & 0.9224 & \textbf{0.0211} & \textbf{0.0626} & \textbf{0.9618} \\ \hline
FlatFault-B & CurveFault-B & 0.1213 & 0.1895 & 0.6677 & \textbf{0.0998} & \textbf{0.1630} & \textbf{0.7343} \\ \hline
CurveFault-A & FlatFault-B & 0.0834 & 0.1537 & 0.8364 & \textbf{0.0398} & \textbf{0.0955} & \textbf{0.9198} \\ \hline
CurveFault-B & FlatFault-B & 0.1898 & 0.2840 & 0.5369 & \textbf{0.1638} & \textbf{0.2528} & \textbf{0.5905} \\ \hline
Style-A & Style-B & 0.1195 & 0.1655 & 0.7653 & \textbf{0.0948} & \textbf{0.1372} & \textbf{0.8049} \\ \hline
Style-B & Style-A & 0.0858 & 0.1226 & 0.6817 & \textbf{0.0801} & \textbf{0.1142} & \textbf{0.6871} \\ \thickhline
\end{tabular}
\caption{Quantitative comparison between the generalization results of InversionNet and BigFWIs (leave-one-out) in terms of MAE, RMSE and SSIM.}
\label{tab:gen}
\end{table}

\begin{table}[t]
\centering
\setlength{\tabcolsep}{10pt}
\renewcommand{\arraystretch}{1.5}
\begin{tabular}{c|c c c|c c}
\thickhline
\multirow{2}{*}{\makecell{Target\\Dataset}} & \multicolumn{3}{c|}{InversionNet} & \multicolumn{2}{c}{BigFWI} \\ \cline{2-6} 
& Source & WD$_{Seis}\downarrow$ & WD$_{Vmap}\downarrow$ & WD$_{Seis}\downarrow$ & WD$_{Vmap}\downarrow$ \\ \thickhline
FlatVel-A & FlatVel-B & \textbf{31.5622} & 0.0912 & 32.2925 & \textbf{0.0616}\\ \hline
FlatVel-B & CurveVel-B & 471.8385 & 0.3892 & \textbf{364.7918} & \textbf{0.3283} \\ \hline
CurveVel-A & CurveVel-B & 631.7191 & 0.2903 & \textbf{472.7896} & \textbf{0.1658} \\ \hline
CurveVel-B & FlatFault-B & 5445.9127 & 1.8209 & \textbf{2597.8299} & \textbf{0.7624} \\ \hline
FlatFault-A & CurveFault-A & 631.0078 & 0.1113 & \textbf{454.4401} & \textbf{0.0801} \\ \hline
FlatFault-B & CurveFault-B & 863.8857 & 0.3363 & \textbf{795.1951} & \textbf{0.2664} \\ \hline
CurveFault-A & FlatFault-B & 1419.7535 & 0.3397 & \textbf{671.2540} & \textbf{0.1189} \\ \hline
CurveFault-B & FlatFault-B & 1703.5499 & 0.4681 & \textbf{1307.8235} & \textbf{0.3566} \\ \hline
Style-A & Style-B & 317.7699 & 0.5207 & \textbf{285.1514} & \textbf{0.3141} \\ \hline
Style-B & Style-A & \textbf{284.4557} & 0.2024 & 406.7438 & \textbf{0.1798} \\ \thickhline
\end{tabular}
\caption{\color{black}Quantitative comparison between the generalization results of InversionNet and BigFWIs (leave-one-out) in terms of Wasserstein Distance.}
\label{tab:gen-wd}
\end{table}


\begin{table}[t]
\small
\centering
\setlength{\tabcolsep}{5pt}
\renewcommand{\arraystretch}{1.5}
\begin{tabular}{c|c c c|c c c|c c c|c c c}
\thickhline
\multirow{2}{*}{Dataset} & \multicolumn{3}{c|}{InversionNet-SA} & \multicolumn{3}{c|}{InversionNet-SB} & \multicolumn{3}{c|}{BigFWI-M} & \multicolumn{3}{c}{BigFWI-L} \\ \cline{2-13} 
& MAE$\downarrow$ & RMSE$\downarrow$ & SSIM$\uparrow$  & MAE$\downarrow$ & RMSE$\downarrow$ & SSIM$\uparrow$ & MAE$\downarrow$ & RMSE$\downarrow$ & SSIM$\uparrow$ & MAE$\downarrow$ & RMSE$\downarrow$ & SSIM$\uparrow$ \\ \thickhline

\makecell{Marmousi\\(smooth)}  & 0.1410 & 0.1996 & 0.7408 & 0.1456 & 0.1988 & 0.5886 & \textbf{0.0792} & \textbf{0.1164} & \textbf{0.8356} & 0.0823 & 0.1244 & 0.7808 \\ \hline
\makecell{Marmousi\\(original)} & 0.1783 & 0.2505 & 0.4749 & 0.2161 & 0.2942 & 0.3798 & \textbf{0.1492} & \textbf{0.2323} &\textbf{0.4936} & 0.1549 & 0.2419 & 0.4806 \\ \hline
\makecell{Overthrust\\(smooth)}  & 0.1213 & 0.1719 & 0.7511 & 0.1062 & 0.1398 & 0.6898 & \textbf{0.0722} & \textbf{0.1000} & \textbf{0.7599} & 0.0760 & 0.1001 & 0.7491 \\ \hline
\makecell{Overthrust\\(original)}  & 0.2052 & 0.2742 & 0.4177 & 0.1819 & 0.2414 & 0.4252 & \textbf{0.1549} & \textbf{0.2040} & \textbf{0.4608} & 0.1775 & 0.2297 & 0.3938 \\ \hline

\end{tabular}
\caption{Quantitative comparison betwen the generalization results of InversionNet and BigFWIs on Marmousi and Overthrust in terms of MAE, RMSE and SSIM.}
\label{tab:gen_marm_over}
\end{table}

\begin{table}[t]
\small
\centering
\setlength{\tabcolsep}{5pt}
\renewcommand{\arraystretch}{1.5}
\begin{tabular}{c|c c|c c|c c|c c}
\thickhline
\multirow{2}{*}{Dataset} & \multicolumn{2}{c|}{InversionNet-SA} & \multicolumn{2}{c|}{InversionNet-SB} & \multicolumn{2}{c|}{BigFWI-M} & \multicolumn{2}{c}{BigFWI-L} \\ \cline{2-9} 
& WD$_{Seis}\downarrow$ & WD$_{Vmap}\downarrow$ & WD$_{Seis}\downarrow$ & WD$_{Vmap}\downarrow$ & WD$_{Seis}\downarrow$ & WD$_{Vmap}\downarrow$ & WD$_{Seis}\downarrow$ & WD$_{Vmap}\downarrow$ \\ \thickhline

\makecell{Marmousi\\(smooth)}  & 1140.5796 & 0.6277 & \textbf{1092.4845} & 0.4691 & 1104.5630 & \textbf{0.2706} & 1125.5029 & 0.2882 \\ \hline
\makecell{Marmousi\\(original)} & 2160.1560 & 0.6764 & 3215.7143 & 0.5241 & 1460.0678 & 0.2840 & \textbf{1293.1754} & \textbf{0.2558} \\ \hline
\makecell{Overthrust\\(smooth)} & 2203.9591 & 0.6695 & 1059.8792 & 0.4295 & 1045.3390 & 0.1136 & \textbf{976.3700} & \textbf{0.0752} \\ \hline
\makecell{Overthrust\\(original)} & 2794.4750 & 0.7481 & \textbf{1376.2766} & 0.4220 & 1805.2714 & \textbf{0.0981} & 1697.1583 & 0.1567 \\ \hline

\end{tabular}
\caption{\color{black}Quantitative comparison between the generalization results of InversionNet and BigFWIs on Marmousi and Overthrust in terms of Wasserstein Distance.}
\label{tab:gen_marm_over-wd}
\end{table}

\begin{table}[h]
\centering
\setlength{\tabcolsep}{3pt}
\renewcommand{\arraystretch}{2.0}
\small
\begin{tabular}{c |c c|c c|c c|c c}
\thickhline
\multirow{2}{*}{Dataset} & \multicolumn{2}{c|}{InversionNet-SA} & \multicolumn{2}{c|}{InversionNet-SB} & \multicolumn{2}{c|}{BigFWI-M} & \multicolumn{2}{c}{BigFWI-L} \\ \cline{2-9} 
& \makecell{RMS$_{RTM}\downarrow$\\($\times10^{-3})$} & \makecell{2-norm$_{RTM}\downarrow$\\($\times10^{-1})$} & \makecell{RMS$_{RTM}\downarrow$\\($\times10^{-3})$} & \makecell{2-norm$_{RTM}\downarrow$\\($\times10^{-1})$} & \makecell{RMS$_{RTM}\downarrow$\\($\times10^{-3})$} & \makecell{2-norm$_{RTM}\downarrow$\\($\times10^{-1})$} & \makecell{RMS$_{RTM}\downarrow$\\($\times10^{-3})$} & \makecell{2-norm$_{RTM}\downarrow$\\($\times10^{-1})$} \\ \thickhline

\makecell{Marmousi\\(smooth)}  & 0.8396 & 0.5877 & 1.3680 & 0.9576 & \textbf{0.5962} & \textbf{0.4174} & 0.7789 & 0.5452 \\ \hline
\makecell{Marmousi\\(original)} & 2.1476 & 1.5033 & 2.3882 & 1.6717 & \textbf{2.0772} & \textbf{1.4540} & 2.1852 & 1.5296 \\ \hline
\makecell{Overthrust\\(smooth)} & \textbf{0.9596} & \textbf{0.6717} & 1.1037 & 0.7726 & 1.0319 & 0.7223 & 1.0512 & 0.7359 \\ \hline
\makecell{Overthrust\\(original)} & 2.7633 & 1.9343 & 2.7022 & 1.8915 & \textbf{2.5804} & \textbf{1.8063} & 2.7460 & 1.9222 \\ \thickhline

\end{tabular}
\caption{\color{black}Quantitative comparison between the generalization results of InversionNet and BigFWIs on Marmousi and Overthrust in terms of Wasserstein Distance.}
\label{tab:gen_marm_over-rtm}
\end{table}

% Figure environment removed

% Figure environment removed
\clearpage
\bibliography{supplement}

\end{document}