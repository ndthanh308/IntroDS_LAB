\documentclass{amsart}
%\usepackage{epsfig}
\usepackage{graphicx}
\usepackage{amssymb,amsthm, amsmath}
\usepackage[monochrome]{color}
\usepackage[foot]{amsaddr}

\def\N{\mathbb N}
\def\Z{\mathbb Z}
\def\Q{\mathbb Q}
\def\R{\mathbb R}
\def\T{\mathbb T}
\def\C{\mathbb C}
\def\Ast{\mathcal{A}^{*}}
\def\A{\mathcal{A}}
\def\D{\mathcal{D}}
\def\L{\mathcal{L}}
\def\Ainf{\mathcal{A}^{\mathbb{N}}}
\def\AZ{\mathcal{A}^{\mathbb{Z}}}
\def\Af{\mathcal{A}_{\mathcal{F}}}
\def\eps{\varepsilon}
\def\GSE{\mathrm{GSP}}
\def\T{\mathrm{T}}
\def\PD{\mathrm{PD}}
\def\ZD{\mathrm{ZD}}
\def\ZT{\mathrm{ZT}}

\newtheorem{thm}{Theorem}
\newtheorem{prop}{Proposition}
\newtheorem{cor}{Corollary}
\newtheorem{conj}{Conjecture}
\newtheorem{lem}{Lemma}
\newtheorem{rem}{Remark}
\newtheorem{ex}{Example}

\begin{document}
\title{An alternative proof for an aperiodic monotile}
\author{Shigeki Akiyama}
\address{Institute of Mathematics, University of Tsukuba}
\email{akiyama@math.tsukuba.ac.jp}
\author{Yoshiaki Araki}
\address{Japan Tessellation Design Association}
\email{yoshiaki.araki@tessellation.jp}
\date{}
\maketitle

\begin{abstract}
We give an alternative simple proof that 
the monotile introduced by \cite{SMKGS:23_1} is aperiodic.
\end{abstract}

\section{Introduction}
Smith Hat tiles 
found in \cite{SMKGS:23_1} form a one-parameter family Tile $(b:1)$ of aperiodic monotiles: each member tiles a plane but only non-periodically 
except when $b=0,1,\infty$.  
In this paper, we pick out one special case Tile $(\sqrt{3}:1)$ and 
give an alternative simple proof of this fact.

The proof in \cite{SMKGS:23_1} that it tiles a plane depends on a ``combinatorial'' substitution rule, which differs each time of its application and converges to a geometric substitution rule whose attractors give a limit tiling having fractal boundaries.
By this combinatorial nature, the proof becomes a little involved. 

Firstly, we give a concrete ``Golden Hex substitution",
whose tiles are essentially a regular triangle, a parallelogram
and two isosceles trapezoids.
The consistency of this new rule is shown 
by giving the explicit
transition diagram of all configurations around vertices 
under the application 
of the substitution. 
Note that the proof is finished without computer programs, 
except for the help of computer drawings and readers can easily check it
by hands.
This proves that tilings by Tile $(\sqrt{3}:1)$ do exist.

Secondly, we show that all tilings generated by Tile $(\sqrt{3}:1)$ are
non-periodic, using a special linear marking; ``Golden Ammann bar". We call it GAB in short. 
Such markings were 
originally introduced by R.~Ammann to construct 
a finite set of tiles having aperiodicity \cite{AGS:92,Gruenbaum-Shephard:87,Akiyama:12}. 
The choice $b=\sqrt{3}$ is essential to introduce
these Ammann bars to Tile $(b:1)$. 
Our proof requires only several elementary properties of this GAB which 
are easily checked, 
and it does not require the meta-tiles or 
substitution structures found in \cite{SMKGS:23_1}.
%Since a tiling by Tile $(b:1)$ are uniquely transformed to the one by
%Tile $(\sqrt{3},1)$ and vice versa when $b\neq 0,1,\infty$, 

Tilings produced by Tile $(b:1)$ and the ones by  Tile $(c:1)$ are combinatorially equivalent if $b,c\not \in\{0,1,\infty\}$. 
Combining the above discussion,
we obtain a simple independent proof of the
aperiodicity of Smith Hat tile.
%The markings corresponds to the tiling generated by Golden hex substitution is non periodic.
%Indeed, the bars consists 
%a lacunary hexagonal grid whose gaps of adjacent lines
%give $1-2$ sequences, which forms sturmian sequences 
%of golden mean slope.
%We also discuss cut and project scheme associated with the fractal tilings using the algebraic dual scheme.
%We see that the fractal tilings generated by Tile $(b:1)$ does not depend on 
%$b$ up to similitudes and 
%the division of internal space gives a tiling consists of fractal curves and segments. 
%Indeed, we rediscover that the segments forms a Golden Hex tiling, which gives a self-duality of this aperiodic monotile. 

\section{Golden Hex substitution}

We assume that a tile is a set homeomorphic to a closed ball in this paper. 
A tiling is the covering of $\R^2$ by a finite set of tiles and their 
images under rigid motion.
A patch is a collection of tiles whose union is homeomorphic to a ball. 
If a tiling is reclassified as a tiling of a finite set of patches, 
a patch in this set is called a {\bf patch-tile}.
We shall define four growing sequences $(\T_n,\PD_n,\ZD_n,\ZT_n)_{n=0,1,\dots}$
of patch-tiles generated by 
Tile $(\sqrt{3}:1)$ depicted in Figure \ref{GH}.
Here after we use lighter gray scale in the order $\T<\PD<\ZD<\ZT$
in the figures.
% Figure environment removed
Their limit shapes are 
a regular triangle, a parallelogram and two isosceles trapezoids, 
whose schematic view of the substitution is found in Figure \ref{Scheme}.


Readers might wonder why we do not find small regular triangles
in Figure \ref{GH} which appeared in Figure \ref{Scheme}.
Indeed, we do not yet see them because $\T_1$ is a single point.
See Figure \ref{Tn}, how the small triangles emerge from $n\ge 4$
in $\T_n$.

% Figure environment removed

% Figure environment removed

Let $\tau=(\sqrt{5}+1)/2$ be the golden ratio. 
% four base edges 
%have the same length, say $1$, and others are $1/\tau,1$ or $\tau$. 
Figure \ref{Scheme}
gives a well-defined self-similar substitution of ratio $\tau^2$.
On the other hand, our sequences of patches 
are defined on the hexagonal lattice, but no polygons whose vertices are on the hexagonal lattice are self-similar by the ratio $\tau^2$.
Moreover the boundaries of the corresponding 
patches consist of broken lines. 
Therefore we have to appropriately define the process of substitution
and check its consistency, i.e., whether all the boundaries fit together as we iterate our substitution infinitely many times. 

A vertex atlas is a patch in a given tiling 
that shares a common point (a vertex) 
on the boundary and the common point is an inner point 
of the union of the patch, having minimal cardinality with this property.
Intuitively, we are interested in collecting all possible
vertex atlases of our possible tiling by an experimental approach.
We wish to find a finite set of vertex atlases 
that are stable under our expected substitution
according to the schematic substitution in Figure \ref{Scheme}.
We shall first prove that this is possible.

\begin{thm}\label{Well}
Golden Hex substitution in Figure \ref{GH} is well-defined.
Therefore a tiling by Tile $(\sqrt{3}:1)$ exists.
\end{thm}

\begin{proof}
Because at least one small triangle shows in all
$4$-level patch-tiles, 
we take the union of 
all vertex atlases by $4$-level patch-tiles
which appeared in the $5$-level patch 
as an initial set. 
Figures of each $5$-level patch-tile as a composition of $4$-th level 
patch-tiles are produced by continuing the construction in Figure \ref{GH}.
Our purpose is to show that this generation 
process can be repeated infinitely many times.
We mimic the substitution in Figure \ref{Scheme}
to obtain a larger patch and collect 
all vertex atlases that
appeared after (one time) substitution.
Taking the union of vertex atlases and we repeat this collection 
until we find no new vertex atlas.
In this collection process, we always check two consistency properties.

\begin{enumerate}
\item
A vertex atlas of the $4$-level patch-tiles
is substituted to a larger vertex atlas 
consisting of $5$-level patch-tiles
in a unique way keeping the
outline of the original vertex atlas.
\item
Overlapping adjacent vertex atlases are consistently substituted in a way that
the two patch-tiles share the same edge configuration (possibly empty).
In other words, the
union of two adjacent vertex atlases gives an ``edge" atlas and this atlas is substituted with consistency.
\end{enumerate}

Figure \ref{Con} gives the result of this collection:
the transition diagram of this process. 
% Figure environment removed
% Figure environment removed
Indeed, we found a stable set of 
27 vertex atlases in Figure \ref{Atlas}, all of which consist of three 
$4$-level patch-tiles.
Note that we are dealing with the tiling, not by the 
schematic substitution but by substitution on patch-tiles, 
we have to look into the detailed structure around the vertices.
Readers may imagine the shapes of vertex atlases from Figure \ref{Tn}, 
where we find 20 vertex atlases by level 3 patch-tiles in $T_5$.
Input vertex atlases are listed in the top row in Figure \ref{Con}, 
and output atlases are in the leftmost column. 
Vertex atlases that are produced by the substitution are marked
in the corresponding column. 
We see that all boundaries of these vertex atlases fit in this process 
of the collection and do not produce any contradictory configuration.

Now the transition diagram in Figure \ref{Con} is understood as the definition
of substitution rule of all local configurations, when we admit overlaps
of patch-tiles. 
By the two consistency properties that we checked while our collection, 
iterating this transition, we obtain larger and larger 
patches by Tile $(\sqrt{3}:1)$ containing balls of arbitrary radius with
consistency. 
Indeed, the location of a $4$-level patch is 
determined by following some path of overlapping vertex atlases. 
If there are two different ways to reach 
the same patch, there is a trivial ``homotopy" to connect two paths of vertex atlases  and we see that
its location is well-defined
by the above two consistency properties.
Therefore the tiling by Tile $(\sqrt{3}:1)$'s do exist (c.f. \cite[Section 3.8]{Gruenbaum-Shephard:87}).
If the existence of the tiling is the only purpose, then 
we may stop the proof here. 

Let us discuss the patch-tile sequences in Figure \ref{GH}.
%For consistency, we have to show that this remains 
%valid for $n$-th level patch-tiles for $n\ge 5$. 
It remains to show that if we adopt the transition diagram in Figure \ref{Con}
as our definition, then we can define 
the $n$-level patch-tiles in Figure \ref{GH}
with the desired outlines, i.e., 
a triangle, a parallelogram and two isosceles trapezoids.
%and the normalized shape converges to the tile in Figure \ref{Scheme}.
To this matter, a detailed study of the boundary is required. 

Let us define special patch sequences $\GSE_1(n), \GSE_2(n)$ by recursive patch concatenation in Figure \ref{GSErec}. Here $\GSE_1(1)=\GSE_2(0)=a$ is 
the Tile $(\sqrt{3}:1)$ set in this direction and  $\mathrm{D}$ designates the flipped Tile $(\sqrt{3}:1)$ and the recurrence is given by tile
concatenation
\begin{align*}
\GSE_1(n)&=\GSE_2(n-1)\GSE_1(n-1)\\ 
\GSE_2(n)&=\GSE_2(n-1)\GSE_1(n-1)\; \mathrm{a D}\; \GSE_2(n-1).
\end{align*}
Hereafter we use capital characters for flipped tiles. 
% Figure environment removed
The elements are called ``Golden Sturmian" patches. 
They consist only of two translationally inequivalent tiles and all tiles in 
one $\GSE$ share two parallel supporting lines. 
In a $\GSE$, two tiles consist of one non-flipped tile and one flipped tile. 
In Golden Hex Substitution of Figure \ref{GH},
readers immediately find that some of the edges of four patch-tiles is made of 
$\GSE$'s as depicted in Figure \ref{Four}.

This substitutive structure is explained by the edge substitution with beards
in Figure \ref{BeardSubst}. 
By checking the transition diagram in Figure \ref{Con},
we notice that edges have several possibilities.
It is already occupied by 
$\GSE_1$ then we designate it by the single notch, and by
$\GSE_2$ then by the double notch as in Figure \ref{Four} and \ref{Init}.
In fact, we find another $\GSE$ on $\ZT_n$, which turned out to be 
decomposed into $\GSE_1,\GSE_2$ in the next level
but in the opposite order than $\GSE_1$, which is designated as a triple notch.
Otherwise, it is not a $\GSE$ edge and is
classified into three types of dents.
As we apply the transition, 
we find that the notches/dents should be substituted to 
the concatenation of several notches/dents 
as described in Figure \ref{Beard} and \ref{BeardSubst} with some lacuna where
we can not decide which tile(s) will fit at this moment.
The function of the beard is to indicate this lacuna to insert some tiles. 
By thorough checking of the transition diagram in Figure \ref{Con},
the lacuna slot that the beard made in the middle of substituted double dents
must be filled by two tiles $a$ and $D$ (or their rotated images) so that 
the two collinear $\GSE$'s form a longer $\GSE$, i.e., it justifies the
the Figure \ref{Beard}, \ref{BeardSubst}, \ref{BoundarySubst}, 
\ref{ES} and the definition of $\GSE$ in Figure \ref{GSErec}.
Therefore the beard has several important functions. 
If \textcolor{red}{a dent is} filled by \textcolor{red}{a} notch of the same shape, the beard does nothing in the middle but gives information on which tile may fit at the end.
If it is filled by two collinear $\GSE$'s, then the beard helps to show
the place to put additional tiles $a$ and $D$ (or their rotated images)
which stick together two $\GSE$'s.
The beard that remains at the end of $\GSE$ restricts 
possible shapes to be filled in this end slot. 
Through this laborious inspection, finally, we see that 
all dented edges are covered by some $\GSE$.
The quadruple dent receives the $\GSE_2$ but the left beard has a
different shape and receive an end of $\GSE$ having another orientation.
We confirm that this rule is consistent with 
the transition diagram of Figure \ref{Con}. 

% Figure environment removed

% Figure environment removed

% Figure environment removed

% Figure environment removed

% Figure environment removed

% Figure environment removed

%% Figure environment removed

Note that by the substitution in Figure \ref{ES}, the vertex atlases at the top of $\T_n, \ZD_n, \ZT_n$
remain invariant, and the one at the \textcolor{red}{right} top corner of $\PD_n$ is 
also invariant. We may treat these points as fixed points of the substitution and there is no ambiguity to place the tiles around this vertex.
Starting from this fixed vertex, the 
concrete location of $n$-level patches within the $(n+1)$-level patch
is completely determined by the shape of $\GSE's$.
From the consistency of our transition diagram, 
our substitution is well-defined and 
works consistently along the boundary and produces the new 
boundary of the $(n+1)$-level by beard substitution as in Figure~\ref{ES}.
This gives a way to define 
the $(n+1)$-level patch-tiles from the ones of $n$-level 
with the help of $\GSE$.
By the induction hypothesis, the boundaries of 
patch-tiles in $(n+1)$-level consist of either $\GSE$'s or 
we may attach a $\GSE$ as well.
As a result, all $n$-level patch-tiles are surrounded by $\GSE$'s, 
and approximates a triangle, parallelogram or trapezoid.
%By the beard substitution we easily see that
%the limit shapes converges to the tiles in Figure \ref{Scheme}.
Summarizing this construction all the $(n+1)$-level
patch-tiles are surrounded by $\GSE$'s, and we can compute the exact 
location of the vertices of the $(n+1)$-level vertex atlases 
by beard substitution from the one of the $n$-level vertex atlas.
This definition is consistent with the transition 
diagram of Figure \ref{Con}.
The proof is finished.
\end{proof}

At this moment, some reader may think we should have defined $\T_n, \PD_n, \ZD_n, \ZT_n$ inductively by assuming the existence of $\GSE$ around the $n$-level shapes according to the schematic substitution in Figure \ref{ES}. Indeed this is also possible. However, to show its consistency in the substituted patches around the vertices, 
we have to compute the transition diagram in Figure \ref{Con}. 
Therefore this does not drastically reduce our effort.

Our $\GSE$'s are well described by the combinatorial
theory of Sturmian words. For the terminology below within this section, see  
\cite[Section 2]{Lothaire:02} for details. 
Standard words $s_n\ (n=0,1,\dots)$ of slope 
$$
\frac{5+\sqrt{5}}{10}=[0;3,\overset{\cdot}{1}]
$$ 
form a sequence of words generated by
$s_0=0,s_1=001$ and $s_{n}=s_{n-1}s_{n-2}$.
We have
$$
0,001,0010,0010001,
00100010010,
001000100100010001,\dots
$$
There are unique palindromes 
$p_n,q_n,r_n$ for $n\ge 1$, such that
$$
s_n= p_n w= q_n r_n 
$$
with $w\in \{01,10\}$ and $p_n$ is called a central word.
By the geometric realization 
$0\rightarrow a$ and $1\rightarrow D$, we see 
$p_n \leftrightarrow \GSE_1(n)$ and $s_n \leftrightarrow
\GSE_2(n)$.
It is also remarkable 
that for any palindrome, the upper boundary of
the corresponding geometric realization is congruent to 
the lower boundary of it by $\pi$ rotation, 
except for broken lines around two ends (see Figure \ref{Palindrome})!

% Figure environment removed

As a result, there are two ways to match $\GSE_1(n)$ to the
corresponding dent.
This seems to lead us to an ambiguous situation, but it is not in reality as we described in the above proof.
The double dent of $\ZD_n$ always receives a $\GSE_1$ in the direction that
the dent is filled by the corresponding notch. On the other hand, 
$\ZT_n$ is a patch generated by attaching 
$\pi$-rotated $\GSE_1$ at the double dents of $\ZD_n$
by the above palindrome property.
Indeed, we can check that all the dents receive $\GSE_1$ in the correct direction, i.e., the dents are filled by the corresponding notches.
The distinction between $\ZD_n$ and $\ZT_n$ is legitimated 
in this manner.
Only ambiguities remain at the corners where the beard 
indicates two possibilities to fill, see Figure \ref{Beard}.
Indeed, more precise study shows 
that two choices happen at the west corner of $\ZD_n$, 
 but at the \textcolor{red}{left} 
corner of $\ZT_n$, the \textcolor{red}{top-left/bottom-right}
 corners 
of $\PD_n$ and the three corners of $\T_n$ the way to fill it is unique.

One can define $\GSE$'s by Sturmian sequence of slope $(5-\sqrt{5})/10$ as 
well since we obtain standard words where the letters $0,1$ are flipped.
The $\GSE$ naturally extends to the infinite word over $\{a,D\}$ 
by boundary substitution and forms a geometric realization
of Sturmian word of slope $w=(5\pm\sqrt{5})/10$. 
%Here we used capital characters to represent the flipped tile.
The frequency of $a$ is $(5+\sqrt{5})/10$ and $D$ is $(5-\sqrt{5})/10$. 
The ratio of these frequencies is $\tau^2$, which justifies the name ``Golden Sturmian Edge". It is remarkable that this linear structure is found everywhere 
in Golden Hex substitution tilings. 



From the substitution rule, we can derive a linear recursion

\begin{align*}
t_{n}&=3 t_{n-1}+3pd_{n-1}+t_{n-2}\\
pd_{n}&=3 pd_{n-1}+2zd_{n-1}+2t_{n-2}\\
zd_{n}&=t_{n-1}+4 pd_{n-1}+3zd_{n-1}+zt_{n-1}+3t_{n-2}\\
zt_{n}&=3 pd_{n-1}+3zd_{n-1}+2zt_{n-1}+3t_{n-2}
\end{align*}
with initial values
$(t_{0},pd_{0},zd_{0},zt_{0})=(0,0,0,0), (t_1,pd_1,zd_1,zt_1)=(0,2,6,9)$. 
This gives the number of Tiles $(\sqrt{3}:1)$ used in $\T_n, \PD_n, \ZD_n, \ZT_n$
for $n\ge 1$ like
$$
(0,2,6,9), (6, 18, 35, 42), (72, 124, 225, 243), \dots
$$
Corresponding $8\times 8$ matrix has characteristic polynomial
$$x^3(x^2-7x-1)(x^2-3x-1)(x-1)$$ 
whose Perron Frobenius root is $\tau^4$
which matches the ones of substitution rules found in \cite{SMKGS:23_1}.
%Here $\tau=(1+\sqrt{5})/2$ is the golden mean. 

\section{Golden Ammann bar and Aperiodicity}

If a tiling is invariant by a translation of a vector $v\in \R^2$, then $v$ is a period of the tiling. If any period of the tiling must be zero, 
we say that the tiling is non-periodic\footnote{See \cite[Section 1.3]{SMKGS:23_1} for different definitions of non-periodicity.}.
%In \cite{Gruenbaum-Shephard:87}, it is shown that 
%if there exists a non-zero period of a tiling by a given set of tiles
%in $\R^2$, then we can find a
%possibly different tiling by the same set of tiles 
%having two linearly independent periods, i.e., being lattice periodic. Therefore in our setting, we do not have to distinguish two concepts: the existense of a period and lattice periodicity.

In this section, we will prove Theorem \ref{AP}, which implies 
that any tiling generated by Tile $(\sqrt{3}:1)$ is non-periodic.
We introduce a special marking in Figure \ref{Amm0}.
We call them Golden Amman Bar, in short GAB. We draw one dashed segment in the 
fore side, and three in the rear side.

% Figure environment removed

Given a polygon that forms the boundary of a patch, 
the inner angle $\theta$ of a vertex is defined as usual. 
The ``complementary angle" of a vertex is $2\pi-\theta$, 
which is not the external angle $\pi-\theta$.
The set of inner angles of 
Tile $(\sqrt{3}:1)$ is $\{\pi/2,2\pi/3,4\pi/3, 3\pi/2,\pi\}$. Here the last
$\pi$ means it is on the edge.
If a patch contains an 
acute complementary angle 
then it is impossible to 
extend to a tiling.
Note that all endpoints of GAB in Tile $(\sqrt{3}:1)$
are located in the middle of an edge or a vertex whose angle or complementary angle is the right angle, and the outward 
extension of this GAB must be covered by an edge or a right-angle vertex.
These simple observations are enough to see that any tiling by
Tile $(\sqrt{3}:1)$, each GAB must continue across the boundary as in Figure \ref{AmmPatch0}.
Indeed, at the endpoint of GAB, the possible angle configurations are $\pi/2 +\pi/2+\pi/2+ \pi/2$, $\pi/2+3\pi/2$, $\pi/2+\pi/2+\pi$, $\pi+\pi$ 
and the proof is plain
in the latter three cases. For $\pi/2 +\pi/2+\pi/2+ \pi/2$, we easily
see that
GAB can not bend at the endpoint.
% Figure environment removed
There are $16$ configurations as in Figure \ref{Disconnected}
that $\pi$ angle vertices meet 
but GAB does not continue and one can immediately confirm 
that none of the configurations extends to a tiling.
%For safety, we also 
%checked this property of GAB
%by listing all patches of minimum cardinality
%which contains an endpoint of GAB as an inner point.

%Such markings are originally introduced by R.~Ammann to construct a set of aperiodic tiles. Note that the situation around aperiodicity is very different. 
%The tiles by Ammann admits a periodic tiling without the marking bar, but this GAB is expected to be not necessary in the tiling of Tile $(\sqrt{3}:1)$. 
Our GAB serves supplementary information 
for the proof of aperiodicity of Tile $(\sqrt{3}:1)$. Hereafter we assume 
the edge length of the
small regular triangle formed by GAB's of 
the flipped tile is $1/2$.

% Figure environment removed

We claim an important fact shown by the similar angle consideration: 
at the intersection of GAB's, we have to use a flipped tile (the right of 
Figure \ref{Amm0}). Four segments
of length $1/2$ are emanating from
the crossing of two GAB's.
It is impossible to cover this local 
configuration with the GAB's of four non-flipped tiles.
Thus there exists at least one flipped tile. 
However, once we use a
flipped tile to cover this crossing, three or four segments are covered
among the four segments.
Therefore this crossing point must be covered in two ways.
It is either 
covered by a single flipped tile, or by exactly one flipped tile and 
one non-flipped tile. Thus each crossing requires exactly one flipped tile.
The situation is 
easily understood in Figure \ref{AmmPatch0}
 when all crossings are found 
in small regular triangles consisting of GAB's.\footnote{
Some crossings may lie outside this small GAB triangle when the GAB appears more often. See the complementary GAB's in Figure \ref{AmmPatch}.}. 

One can also define complementary markings as in Figure \ref{Amm}; three red segments on the fore-side and one segment in the rear.

% Figure environment removed

A generalized GAB is either a GAB or a complementary GAB. 
We can easily check that the set of generalized GAB's  
must form a `Kagome' tiling (also called trihexagonal tiling) 
as in Figure \ref{AmmPatch}, one of the 2-uniform tilings whose signature is
$3636$ and Wythoff symbol $2|63$.\footnote{This generalized GAB and 
Figure \ref{AmmPatch}
were
observed in \cite{Reitebuch:23}.}.
To see this, we just 
check that every parallel generalized GAB's separated by distance
$\sqrt{3}/2$, and there exist three generalized 
GAB's that
produce a small regular triangle of edge length $1/2$.

% Figure environment removed

We consider the Kagome GAB parallelogram $K(n)$ in Figure \ref{Kagome}
consisting of $n$ segments in the horizontal direction, $n$ segments of
slope $\sqrt{3}$ and $2n-2$ segments of slope $-\sqrt{3}$.
By assumption, 
the small regular triangles in the Kagome pattern $K(n)$ has edge length $1/2$.

% Figure environment removed

We associate a
hexagonal coordinate $\langle x,y \rangle\in \R^2$ to the 
Kagome parallelogram by 
$\langle x,y\rangle:=x(1,0)+y(1/2,\sqrt{3}/2)$. 
The horizontal segment
$H_i\ (i=0,1,\dots, n-1)$ connect $\langle 0,i\rangle$ to $\langle n-1,i\rangle$ and 
segments $L_j\ (j=0,1,\dots, n-1)$ of $\sqrt{3}$ slope
connect $\langle j,0\rangle$ to $\langle
j,n-1\rangle$. Finally
$M_k\ (k=0,\dots,2n-1)$ are segments of $-\sqrt{3}$ slope 
 which connect
$\langle 1/2+k,0\rangle$ and $\langle 0,k+1/2\rangle$ for $k\in \{0,1,\dots,n-2\}$, 
and $\langle n-1,k-n+3/2\rangle$ and $\langle k-n+3/2,n-1\rangle$ for $k\in \{n-1,\dots, 2n-1\}$.
Let us fix a tiling by Tile $(\sqrt{3}:1)$. 
Assume that $(a_i)_{i=0}^{h(n)}$, $(b_j)_{j=0}^{\ell(n)}$, $(c_k)_{k=0}^{m(n)}$
are three increasing sequences of non negative integers
such that $H_{a_i}\ (i=0,\dots, h(n))$, 
 $L_{b_j}\ (j=0,\dots, \ell(n))$ and $M_{c_k}\ (k=0,\dots, m(n))$ form 
the set of GAB's in $K(n)$. We may assume that $h(n)$ and $\ell(n)$ are 
positive. Indeed since one may take the mirror image
of the tiling, we may assume that there are infinitely many GAB's in at least two directions, say in $H_i$ and $L_j$ directions.
Since $H_{a_i}$ and $L_{b_j}$ intersect at $\langle a_i,b_j
\rangle$, there exists a unique 
flipped tile whose GAB's cover three or four segments of length $1/2$
emanating from 
$\langle a_j,b_j\rangle$. 
Therefore by the definition of GAB, there exists $k$ such that 
$c_k-a_i-b_j$ is equal to $\pm 1/2$. There is a natural ordering that
if  $i\le i'$, $j\le j'$ and $c_{k'}-a_{i'}-b_{j'}=\pm 1/2$ then $k\le k'$.
Symmetric discussion holds 
for the intersection of $H_{a_i}$ and $M_{c_k}$, as well as for 
the intersection of $L_{b_j}$ and $M_{c_k}$. 
By using this observation, starting from the case 
$i=j=k=0$, we obtain a relation $k=i+j$ and a formula 
\begin{equation}
\label{Hex}
c_{i+j}-a_i-b_j = \pm 1/2
\end{equation}
by induction and we have $m(n)=h(n)+\ell(n)$. Note that the sign $\pm$ in (\ref{Hex}) depends on $i$ and $j$.

Indeed let $c_k-a_0-b_0=\pm 1/2$. The above
cyclic logic gives $c_0-a_{\ell}-b_0=\pm 1/2$ with a unique $\ell$. By the natural ordering, we have $k=\ell=0$. 
If $c_k-a_i-b_j=\pm 1/2$ and $k\neq i+j>0$, then 
take the minimum $i+j$ with this property. 
Since $k<i+j$ gives a contradiction to the uniqueness of $k$, we have $k>i+j$. Using the above cyclic logic, there exists a unique $\ell$ with 
$c_{i+j}-a_{\ell}-b_j=\pm 1/2$. By the induction hypothesis, we see $\ell\ge i$.
By the natural ordering, we infer $\ell\le i$ and thus $\ell=i$. 
This gives a contradiction to the uniqueness of $k$. Thus (\ref{Hex}) is proved.

%Therefore
%\begin{align*}
%c_{i+j}-a_i-b_j&=\pm 1\\
%c_{i+j}-a_{i+1}-b_{j-1}&=\pm 1
%\end{align*}
%gives
%$$
%a_{i+1}-a_i -(b_j-b_{j-1})\in \{-2,0,2\}.
%$$
%Similar discussion yields
%\begin{align*}
%&c_{i+1}-c_{i}-(a_{j+1}-a_j) \in \{-2,0,2\}\\
%&c_{i+1}-c_i-(b_{j+1}-b_j) \in \{-2,0,2\}.
%\end{align*}
%This shows that dashed GAB's must have a structure close to a hexagonal
%lattice.
%The gap between adjacent GAB's of the same orientation is close to a constant 
%function, see Figure \ref{AmmPatch0}.\footnote{These results suggest that the set of all gaps $a_{i+1}-a_i$,$b_{i+1}-b_i$ and
%$c_{i+1}-c_i$ may be small in cardinarity, 
%but this kind of fact is not necessary to show non-periodicity of GAB.}
%The situation is the same for the complemetary GAB as in Figure \ref{AmmPatch}.
We proceed to the proof of aperiodicity of Tile $(\sqrt{3}:1)$. Our goal is to show that if GAB's in the horizontal direction has the frequency $q$
among the horizontal lines of Kagome tiling, 
i.e., there exists $q\in [0,1]$ such that
\begin{equation}
\label{Freq0}
\lim_{n\to \infty} \frac {h(n)}{n} = q
\end{equation}
then $q$ must be irrational. 
If there exists a period $v\neq 0$ of a tiling by Tile $(\sqrt{3}:1)$, then 
every tile is sent to the tile of the same orientation. 
Thus the set of GAB's is invariant by this translation. Rotating the tiling 
if necessary, we may assume that the horizontal frequency $q$ 
must exist and it is rational, we obtain a contradiction. 

Assume (\ref{Freq0}) and let $h=h(n)$. From (\ref{Hex}) we have
\begin{align*}
&c_{j}-a_j-b_0\in [-1/2,1/2],\\
&c_{j}-a_0-b_j\in [-1/2,1/2],\\
&a_j-b_j -(a_0-b_0)\in [-1,1].
\end{align*}
Taking $n\to \infty$, we may extend 
$(a_i),(b_j),(c_k)$ to 
infinite sequences having non-negative integers
indexes $i,j,k$.
Since $a_h\le n < a_{h+1}$, we see 
\begin{align*}
&b_h\le n-a_0+b_0+1\\
&b_{h+1}\ge n+1+a_0-b_0-1
\end{align*}
and thus
\begin{align*}
&b_{h-t+1}\le n-t+1-a_0+b_0+1\\
&b_{h+t}\ge n+t+a_0-b_0-1
\end{align*}
holds for $t=1,2,\dots$. Thus we have
$$
h(n)+a_0-b_0-1 \le \ell(n)\le h(n)-a_0+b_0+1
$$\
which implies
\begin{equation}
\label{Freq1}
\lim_{n\to \infty} \frac {\ell(n)}{n} = q.
\end{equation}
Since
\begin{align*} 
|c_k-(a_k+b_0)| &\le 1/2 & k\le h(n)\\
|c_k-(a_{h(n)}+b_{k-h(n)})|&\le 1/2 & h(n)<k\le m(n),
\end{align*}
$c_k$ and $a_k$ are in one to one correspondence in $k\le h(n)$ 
and $c_k$ and $b_{k-h(n)}$ are one to one in $k>h(n)$, because of
the uniqueness of (\ref{Hex}) and the natural ordering. 
Thus the sequence 
$(c_k)$ inherits the frequency of $(a_i)$ and $(b_j)$ and we see
\begin{equation}
\label{Freq2}
\lim_{n\to \infty} \frac {m(n)}{n} = q.
\end{equation}

Let $\N$ be the set of non-negative integers and 
$U$ be a subset of $\N$. If
$$
\delta(U):=
\lim_{n\to \infty}
\frac 1{n+1} \mathrm{Card}(U \cap [0,n])
$$
exists then $\delta(U)$ is called the natural density of $U$. We prepare 
an easy
\begin{lem}
\label{Abel}
The natural density $\delta(U)$ exists if and only if 
$$
\lim_{n\to \infty} \frac 1{n^2} \sum_{j\in U\cap [0,n]} j
$$
exists. In this case, the last limit is equal to $\delta(U)/2$.
\end{lem}

\begin{proof}
We show it for the convenience of readers.
Let $\chi_U$ be the indicator function of $U$. 
Then we see
$$
\delta(U)=\lim_{n\to \infty} \frac 1{n+1}
\sum_{j=0}^n \chi_U(j).
$$
Assume that $S_n=\sum_{j=0}^n \chi_U(j)j=n^2 q/2 +o(n^2)$. Then
\begin{align*}
\sum_{j=0}^n \chi_U(j)
%&=\chi_U(0)+ \sum_{j=1}^n \chi_U(j)\\
&=\chi_U(0)+ \sum_{j=1}^n \frac{S_j-S_{j-1}}j\\
&=\frac {S_n}n+ \sum_{j=1}^{n-1} \frac {S_j}{j(j+1)}+\chi_U(0)\\
&=\frac {nq}2 + o(n)+ \sum_{j=1}^{n-1} \left(\frac q2\left(1-\frac 1{j+1}\right)+o(1)\right)+\chi_U(0)\\
&=\frac {nq}2 + \frac {(n-1)q}2 - \frac {q\log n}2 + o(n)\\
&=nq + o(n).
\end{align*}
For the converse, assume that $T_n=\sum_{j=0}^n \chi_U(j)=n q +o(n)$.
Then we have
\begin{align*}
\sum_{j=0}^n \chi_U(j)j &= \sum_{j=1}^n (T_j-T_{j-1})j\\
&=n T_n -\sum_{j=1}^{n-1} T_j\\
&=n^2 q + o(n^2) -\sum_{j=1}^{n-1} (jq + o(j))\\
&=\frac {n^2q}2+o(n^2).
\end{align*}
\end{proof}

The area of Tile $(\sqrt{3}:1)$ is $13+1/3=40/3$ times the area of the small regular triangle in Kagome tiling and $K(n)$ consists of $8n^2$ small regular triangles. 
This shows that
the minimum number of Tile $(\sqrt{3}:1)$'s which covers $K(n)$ is
\begin{equation}
\label{Count0}
\frac{3}{5} n^2 + O(n).
\end{equation}
Since there are $O(n)$ tiles which intersect the outermost parallelogram of $K(n)$, the number of Tile $(\sqrt{3}:1)$'s lie strictly within $K(n)$ is also 
(\ref{Count0}).
Later we shall implicitly use this fact that the number of Tile $(\sqrt{3}:1)$'s is insensitive to the way we count them, up to this error term.

We will
compute the sum of all lengths of GAB's in two ways.
By (\ref{Freq0}), the sum of length of $H_{a_i}$ are
$$
\textcolor{red}{(n-1)}(nq +o(n))= n^2 q +o(n^2).
$$
%Note that the distance between two parallel lines in $K(n)$ is $2$ by definition. 
The same is valid for $L_{b_j}$ by (\ref{Freq1}).
The sum of length of $M_{c_k}$ is divided into two parts:
$$
\sum_{c_k<n} \left(c_k-\frac 12\right) + \sum_{c_k\ge n} 
\left(2n-1-c_k+\frac 12\right).
$$
By using (\ref{Freq2}) and Lemma \ref{Abel} with $U=\{c_k\ |\ c_k<n\}$, 
we have
$$
\sum_{c_k<n} \left(c_k-\frac 12\right) = \frac{n^2 q}2 + o(n^2).
$$
Similar computation shows
$$
\sum_{c_k\ge n} \left(2n-1-c_k+\frac 12\right)= 2n^2 q- \left(\textcolor{red}{\frac{(2n-1)^2 q}2}
-\frac{n^2 q}2\right) + o(n^2)=\frac{n^2 q}2 + o(n^2).
$$
Therefore we see that the total length of GAB's is
\begin{equation}
\label{Count1}
3 n^2 q + o(n^2).
\end{equation}
Recall that flipped tiles appear at each 
crossing of $H_{a_i}$ and $L_{b_j}$ exactly once. Then we find 
$n^2 q^2+O(n)$ flipped tiles in $K(n)$. 
Consider the parallelogram tiling $P(n)$
generated by $H_{a_i}$ and $L_{b_j}$, a sub-configuration of $K(n)$. 
Every crossing of $P(n)$ is already covered 
by the flipped tiles and the remainder has no crossing of GAB's in $P(n)$. 
This means in $K(n)$ we can not find a small triangle after the
removal of
the above flipped tiles. Therefore the above 
$n^2 q^2+O(n)$ flipped tiles are the totality of flipped tiles in $K(n)$.
The length of GAB on the fore-sided
tile is $1$ while the length is $4$ in the rear side. Thus
the total length of GAB in $K(n)$ is computed in a different way:
\begin{equation}
\label{Count2}
\left(\frac{3}{5} n^2 -n^2 q^2 +o(n^2)\right) + 4 \left(n^2 q^2 +o(n^2)\right)
= \left(\frac {3}5+3 q^2 \right)n^2 + o(n^2).
\end{equation}
Comparing (\ref{Count1}) and (\ref{Count2}) as $n\to \infty$, 
we obtain
$$
q^2-q+\frac 15=0.
$$
Thus
$$
q_1=\frac{5-\sqrt{5}}{10}=\frac 1{1+\tau^2}\approx 0.276393, \quad 
q_2=\frac{5+\sqrt{5}}{10}=\frac {\tau^2}{1+\tau^2}\approx 0.723607
$$
are the possible two values of frequency $q$.
Since they are both irrational, the proof is finished.

\begin{thm}
\label{AP}
For a tiling by Tile $(\sqrt{3}:1)$, Golden Ammann Bars in Figure \ref{Amm0} gives a sub-configuration of Kagome tiling in Figure \ref{Kagome}. 
If Kagome lines become GAB with frequency $q$ in one direction, then it has 
the same frequency $q$ in the
other two directions. In this case, the value $q$
must be equal to one of $\frac {5\pm \sqrt{5}}{10}$. 
Consequently, each tiling by Tile $(\sqrt{3}:1)$ 
is non-periodic.
\end{thm}

Note that two values correspond to the frequency of GAB and that of 
complementary GAB, and $q_2/q_1=\tau^2$. The Golden Hex tiling in the previous section has the GAB frequency $(5\pm \sqrt{5})/10$, which is easily shown by the existence of arbitrary long $\GSE$'s.


\begin{rem} 
\textcolor{red}{After the submission to ArXiv, 
we are informed that a different 
proof of aperiodicity using GAB was released 
several days prior to our post, see \cite{MathBlock}. Here is our comparison.
The proof in \cite{MathBlock} relies on an assumption that the tiling by Tile $(\sqrt{3}:1)$ must have an underlying Laves tiling.
A proof of this assumption is found in Appendix A of \cite{SMKGS:23_1} 
for the Smith Hat Tile $(1:\sqrt{3})$, but 
the one for Tile $(\sqrt{3}:1)$ is postponed in \cite{MathBlock}. In contrast, our proof is self-contained.
Two proofs look similar
at a glance but pretty different: we computed the total
length of GAB, while \cite{MathBlock} computed the number of essential 
crossings of GAB's and that of complementary GAB's using angle consideration 
on the corresponding Laves tiling. 
Another small difference exists in the definitions of non-periodicity: we showed that any period must be zero, while \cite{MathBlock} denied 
the existence of two linearly independent periods.
}
\end{rem}

\section{Future works}

Two substitutions are proposed in \cite{SMKGS:23_1} based on the extensive search of the corona shapes. One of them is based on a patch-tiles H7 and H8, 
and it is of interest to study the existence of the
related cut and project scheme of the limit H7H8 tiling.
We checked the pure discreteness of H7H8 tiling dynamics 
by the algorithm in \cite{Akiyama-Lee:10}. 
This shows that 
there exists a $2\times 2$ cut and project scheme (c.f. \cite{BaakeGaehlerSadun:23}).
We originally found the Golden Hat substitution structure
in the subdivision of the associated
cut and project window (c.f. \cite{Baake-Grimm:13,Akiyama-Barge-Berthe-Lee-Siegel}). We shall discuss this relationship elsewhere. 

From our Golden Hat substitution whose consistency is legitimated in the first section, we expect that the combinatorial substitution rule of H7 and H8 in \cite{SMKGS:23_1} is realized as a concrete geometric substitution. 
It is plausible that our Golden Hex tiling is MLD with H7H8 tiling.
Our preliminary experiments show that it seems to be the case 
and therefore this Golden Hex substitutive structure may be 
automatically enforced for all tilings by Tile $(\sqrt{3}:1)$.

Apart from Smith Hat in \cite{SMKGS:23_1}, 
several other aperiodic monotiles are studied
in \cite{SocolarTaylor:10, MampustiWhittaker:20, SMKGS:23_2} under 
different conditions. 
The idea of the proofs of aperiodicity is 
to enforce some large structure in the resulting tilings.
Our proof using the
statistical property of GAB seems 
to be new and we expect some further applications.

\begin{thebibliography}{10}

\bibitem{Akiyama:12}
S.~Akiyama, \emph{A note on aperiodic {A}mmann tiles}, Discrete Comput. Geom.
  \textbf{48} (2012), no.~3, 702--710.

\bibitem{Akiyama-Barge-Berthe-Lee-Siegel}
S.~Akiyama, M.~Barge, V.~Berth\'e, J.-Y. Lee, and A.~Siegel, \emph{On the
  {P}isot {S}ubstitution {C}onjecture}, Mathematics of Aperiodic Order,
  Progress in Mathematics, vol. 309, Birkh\"auser, Basel, 2015, pp.~33--72.

\bibitem{Akiyama-Lee:10}
S.~Akiyama and J.-Y. Lee, \emph{Algorithm for determining pure pointedness of
  self-affine tilings}, Adv. Math. \textbf{226} (2011), no.~4, 2855--2883.

\bibitem{AGS:92}
R.~Ammann, B.~Gr{\"u}nbaum, and G.~C. Shephard, \emph{Aperiodic tiles},
  Discrete Comput. Geom. \textbf{8} (1992), no.~1, 1--25.

\bibitem{BaakeGaehlerSadun:23}
M.~Baake, F.~G\"ahler, and L.~Sadun, \emph{Dynamics and topology of the hat
  family of tilings}, Ar{X}iv:2305.05639.

\bibitem{Baake-Grimm:13}
M.~Baake and U.~Grimm, \emph{Aperiodic {O}rder. {V}ol. 1}, Encyclopedia of
  Mathematics and its Applications, vol. 149, Cambridge University Press,
  Cambridge, 2013.

\bibitem{Gruenbaum-Shephard:87}
B.~Gr{\"u}nbaum and G.~C. Shephard, \emph{Tilings and patterns}, W. H. Freeman
  and Company, New York, 1987.

\bibitem{Lothaire:02}
M.~Lothaire, \emph{Algebraic combinatorics on words}, Encyclopedia of
  Mathematics and its Applications, vol.~90, Cambridge University Press,
  Cambridge, 2002.

\bibitem{MathBlock}
\emph{
The Turtle prototile is not periodic, a simple proof},
https://archive.li/rKD0U

\bibitem{MampustiWhittaker:20}
M.~Mampusti and M.~F. Whittaker, \emph{An aperiodic monotile that forces
  nonperiodicity through dendrites}, Bull. Lond. Math. Soc. \textbf{52} (2020),
  no.~5, 942--959.

\bibitem{Reitebuch:23}
U.~Reitebuch, \emph{Direct construction of aperiodic tilings with the hat
  monotile}, Ar{X}iv:2306.06512.

\bibitem{SMKGS:23_1}
D.~Smith, J.~S. Myers, C.~S. Kaplan, and C.~Goodman-Strauss, \emph{An aperiodic
  monotile}, arXiv:2303.10798.

\bibitem{SMKGS:23_2}
\bysame, \emph{A chiral aperiodic monotile}, arXiv:2305.17743.

\bibitem{SocolarTaylor:10}
J.~E.~S. Socolar and J.~M. Taylor, \emph{An aperiodic hexagonal tile}, Journal
  of Combinatorial Theory \textbf{18} (2011), 2207--2231.

\end{thebibliography}

%\bibliographystyle{amsplain}
%\bibliography{../reflist}
\end{document}
