% ****** Start of file apssamp.tex ******
%
%   This file is part of the APS files in the REVTeX 4.2 distribution.
%   Version 4.2a of REVTeX, December 2014
%
%   Copyright (c) 2014 The American Physical Society.
%
%   See the REVTeX 4 README file for restrictions and more information.
%
% TeX'ing this file requires that you have AMS-LaTeX 2.0 installed
% as well as the rest of the prerequisites for REVTeX 4.2
%
% See the REVTeX 4 README file
% It also requires running BibTeX. The commands are as follows:
%
%  1)  latex apssamp.tex
%  2)  bibtex apssamp
%  3)  latex apssamp.tex
%  4)  latex apssamp.tex
%
\documentclass[%
reprint,
%superscriptaddress,
%groupedaddress,
%unsortedaddress,
%runinaddress,
%frontmatterverbose, 
%preprint,
%preprintnumbers,
%nofootinbib,
%nobibnotes,
%bibnotes,
 amsmath,amssymb,
 aps,
%pra,
%prb,
%rmp,
%prstab,
%prstper,
%floatfix,
]{revtex4-2}

\usepackage{graphicx}% Include figure files
\usepackage{dcolumn}% Align table columns on decimal point


\usepackage{appendix}
\usepackage{color}
\usepackage{xcolor}
\usepackage{bm}% bold math
%\usepackage{hyperref}% add hypertext capabilities
%\usepackage[mathlines]{lineno}% Enable numbering of text and display math
%\linenumbers\relax % Commence numbering lines

%\usepackage[showframe,%Uncomment any one of the following lines to test 
%%scale=0.7, marginratio={1:1, 2:3}, ignoreall,% default settings
%%text={7in,10in},centering,
%%margin=1.5in,
%%total={6.5in,8.75in}, top=1.2in, left=0.9in, includefoot,
%%height=10in,a5paper,hmargin={3cm,0.8in},
%]{geometry}

\begin{document}

%\preprint{APS/123-QED}

\title{ Unraveling the role of interactions in non-equilibrium transformations }% Force line breaks with \\

\author{Maria Rose}
 \affiliation{School of Pure and Applied Physics, Mahatma Gandhi University, Kottayam, India }%Lines break automatically or can be forced with \\
\author{Sreekanth K Manikandan}%
 \email{sreekm@stanford.edu}
 \thanks{In this study, part of the work was conducted by the author while employed at NORDITA, and the remaining part was carried out while the author was affiliated with Stanford University.}
\affiliation{NORDITA, KTH Royal Institute of Technology and
Stockholm University, Roslagstullsbacken 23, 10691 Stockholm, Sweden}
\affiliation{Department of Chemistry, Stanford University, Stanford, CA, USA 94305
}%


\date{\today}% It is always \today, today,
             %  but any date may be explicitly specified

\begin{abstract}
For arbitrary non-equilibrium transformations in complex systems, we show that the distance between the current state and a target state can be decomposed into two terms: one corresponding to an \textit{independent} estimate of the distance, and another corresponding to interactions, quantified using the relative mutual information between the variables. This decomposition is a special case of a more general decomposition involving successive orders of correlation or interactions among the system's degrees of freedom. To illustrate its practical significance, we study the thermalization of two optically trapped colloidal particles with hydrodynamic coupling. Our results show that, 
increasing pairwise interaction strength enhances the separation between the time-dependent non-equilibrium state and the final state, prolonging the non-equilibrium state's longevity. In more general setups, where it is possible to control the strength of different orders of interactions, our findings offer a way to disentangle their effects on the transformation process.
\end{abstract}

%\keywords{Suggested keywords}%Use showkeys class option if keyword
                              %display desired
\maketitle

%\tableofcontents
%Paper for introduction \cite{fewlevelQ}
A broad range of microscopic non-equilibrium processes are time-dependent, where the state of the system, described in terms of probability distributions, changes as a function of time. Examples include the thermalization of systems prepared in an arbitrary initial state \cite{dattagupta2012relaxation}, self-assembly of biological molecules \cite{whitesides2002self,pollard_cellular_2003,mauro2014self}, protein folding \cite{dobson2003protein,creighton1990protein}, several single-molecule experiments \cite{ritort2006single,ciliberto2017experiments}, and microscopic devices that are time-dependently controlled \cite{pop2010energy,bergfield2013forty,martinez2016brownian}. In all these cases, the trajectory of the system progresses through a series of states, influenced by interactions among the different degrees of freedom of the system, with the environment, and external controls/feedbacks \cite{sagawa2012fluctuation,barato2014unifying}.

Several recent studies have tried to identify governing principles for such processes in terms of the distance between the initial and final states of the system, the time taken for the transformation, and the associated thermodynamic costs. These include the refinements of the Second Law \cite{aurell2012refined,Lutz,kim2021information,nakazato2021geometrical}, optimal connections \cite{ito2023geometric,chennakesavalu2023unified,rotskoff2015optimal}, speed limits \cite{shiraishi2018speed,van2023topological,yoshimura2021thermodynamic,funo2019speed} as well as their trade-offs with the entropic costs \cite{lee2022speed,falasco2020dissipation,van2023thermodynamic,yoshimura2021thermodynamic,kuznets2021dissipation}. However, the fundamental effects of interactions among the different degrees of freedom of the system, on the distance or time taken for non-equilibrium transformations are relatively less understood.


In a recent development, Refs. \cite{arrowoftime,lynn2022emergence} made significant progress in this direction. They demonstrated that in systems with multiple degrees of freedom and having multi-partite dynamics, the estimate of \textit{irreversibility} in a non-equilibrium steady state can be decomposed into contributions from individual variables, and a series of non-negative contributions from correlations among
variable pairs, triplets, and higher-order combinations. Their proof is based on representing irreversibility as a Kullback-Leibler divergence, which measures the relative likelihood of trajectories over their time-reversed counterparts. 

In general, the Kullback-Leibler divergence quantifies the distance between any two probability distributions, and it has recently gained renewed interest in studying non-equilibrium transformations and control of microscopic systems \cite{ito2018stochastic,aurell2011optimal,nakazato2021geometrical}. In certain cases, it also provides estimates of the thermodynamic cost of the process \cite{shiraishi2019information,chetrite2021metastable,sagawa2012fluctuation}. Hence, understanding how this distance function depends on interactions is crucial, as it enables the optimization of processes based on interactions, and the design of more efficient and reliable non-equilibrium controls.

Here we address this problem by implementing a non-trivial decomposition of the Kullback-Leibler divergence. This decomposition primarily consists of two terms: one corresponding to an \textit{independent} estimate of the distance, representing hypothetical marginal processes which are non-interacting, and another corresponding to interactions, quantified using the relative mutual information between the variables. This decomposition is further shown to arise from a previously known decomposition of the joint distribution involving successive orders of correlation or interactions among the system's degrees of freedom \cite{mcclendon2012comparing,galas2017expansion,tritchler2011information}. Crucially this decomposition is not limited to multi-partite systems. Applying the decomposition to an interacting pair of colloids that undergoes thermalization, we find that increasing the strength of pairwise interactions generically increases the distance between the current state and the target state, prolonging the longevity of the non-equilibrium initial state. In more general setups, where it is possible to control the strength of different orders of interactions, our results provide means to separate out their effects on the transformation process.

We begin by considering a system whose state is described using the variable ${\bm x}_t \in \mathbb{R}^N$, and probability distribution $P({\bm x}_t)$. We have dropped the explicit dependence on $t$ for simplicity of notation. Note that one of the elements of vector ${\bm x}_t$ can also be an external control or a feedback protocol. Let us now consider a scenario where the probability distribution $P({\bm x}_t)$ dynamically evolves from an initial distribution $P_i({\bm x}_{t_i})$ to a final / target distribution $P_f({\bm x}_{t_f})$ in a time-dependent manner. At any given time $t$, the distance of the instantaneous distribution $P({\bm x}_t)$ to the target distribution can be computed in terms of the Kullback-Leibler (KL) divergence between the two distributions as \cite{lu2017nonequilibrium}, 
\begin{align}
\begin{split}
   D_{\rm KL}(P({\bm x}_t) \vert \vert P_f({\bm x}_{t})) &= \int_{{\bm x}_t} P( {\bm x}_t) \log \frac{P( {\bm x}_t)}{P_f({\bm x}_{t})}.
\end{split}
\end{align}
Next, assume we know the marginal distributions, $P_m^i(x_t^i) = \int_{{\bm x}_{-i}}P( {\bm x}_t)$,
where ${\bm x}_{-i}$ corresponds to all variables except $x_t^i$.
One can obtain an \textit{independent} distance in terms of these marginals as, 
\begin{align}
\begin{split}
\label{eq:Dind}
        D^i_{\rm{Ind}}  = \int_{x_t^i}P_m^i(x_t^i)\log \frac{P_m^i(x_t^i)}{P_{f,m}^i(x_t^i)}.
\end{split}
\end{align}
The sum of the independent distances over all variables, $D_{\rm Ind.} = \sum_i D^i_{\rm{Ind}}$, provides an estimate of the distance that one would have got if the variables were independently measured. By examining the difference $D - D_{\rm Ind.}$, we find,
\begin{align}
\label{eq:KLinteraction}
\begin{split}
    D &- \sum_i D^i_{\rm{Ind}} \\
    &=\int_{{\bm x}_t} P( {\bm x}_t) \left[\log \frac{P( {\bm x}_t)}{\prod_i P_m^i(x_t^i)} - \log \frac{P_f( {\bm x}_t)}{\prod_i P_{m,f}^i(x_t^i)}\right]\\
    &= I({\bm x}_t) - I_f^\prime ({\bm x}_t),
    \end{split}
\end{align}
where $I({\bm x}_t)$ is the mutual information of the current state, generalized to $N$ variables (also referred to as the total correlation \cite{watanabe1960information}), and $I_f^\prime({\bm x}_t)$ is the cross mutual information of the target state, where the average is computed with respect to the current state. 

Eq.\ \eqref{eq:KLinteraction} is our first key observation: the distance between any two distributions can be decomposed into two terms: a term coming from the marginal probabilities and another coming from interactions between the local variables, {\it i.e.}, 
\begin{align}
\label{eq:decomposition0}
    D =  D_{\rm Ind.} + D_{\rm Int.},
\end{align}
where $D_{\rm Int.} \equiv I({\bm x}_t) - I_f^\prime ({\bm x}_t)$, appears as the relative mutual information between the current state and the target state. Note that the sign of this interaction term could be positive or negative, depending on the choice of the final distribution and the nature of interactions. Eq.\ \eqref{eq:KLinteraction} also has a simple information theoretic interpretation: Interactions contribute to the distance only if the mutual information of the current state differs from the cross mutual information of the target state. This means, there could be instances where accurate distance measurements can be solely obtained from the marginal statistics, even when the local variables are correlated.


Interestingly, using a known decomposition of the joint probability distribution, one can show that the total distance further breaks down into contributions from interactions among subsets of $k<N$ variables. This decomposition is due to the generalized Kirkwood superposition approximation \cite{mcclendon2012comparing,tritchler2011information,somani2009sampling, killian2007extraction,galas2017expansion}, which is traditionally used in the context of sampling of equilibrium distributions.  We briefly describe it in the following:
\noindent
Assume that we know all the $(N-1)^{\rm th}$ order marginal distributions,
\begin{align}
    P_{N-1}(x_1,\dots x_{N-1}) = \int _{{\bm x_t}^{-[N-1]}}P({\bm x}_t),
\end{align}
where the integration is done over the variable that is not in the subset $\lbrace x_1,\dots x_{N-1} \rbrace$. The Kirkwood superposition approximation provides an estimate to the joint probability distribution $\hat{P}_{N-1}({\bm x}_t) \simeq P({\bm x}_t)$ in terms of these marginals, as \cite{mcclendon2012comparing,tritchler2011information},
\begin{align}
\label{eq:final}
    \log \hat{P}_{N-1}({\bm x}_t) = \sum_{\alpha=1}^{N-1}(-1)^{N-\alpha+1} \log \prod_{j=1}^{C^N_\alpha} P_{\alpha}^j,
\end{align}
where the product is over all marginal densities $P_{\alpha}^j$ obtained for a subset of variables of size $\alpha \leq N-1$. By successively applying the Kirkwood approximation to the RHS of Eq.\ \eqref{eq:final}, we can get an estimate of the joint distribution $P({\bm x}_t)$ in terms of marginals of any order $k < N$. We refer to the resulting $k-$th order approximation as $\hat{P}_{k}({\bm x}_t)$. In particular, for $k=1$, we will arrive at the product of single-variable marginals \cite{killian2007extraction,galas2017expansion}. 

Using these distributions, we can obtain an estimate of the distance that is accurate to $k$-th order interactions, as,
\begin{align}
    D^{(k)} = \int_{{\bm x}_t} P({\bm x}_t)\log \frac{ \hat{P}_{k}({\bm x}_t)}{\hat{P}_{f,k}({\bm x}_t)}.
\end{align}
Due to the expansion in Eq.\ \eqref{eq:final}, $D^{(k)}$ is fully determined in terms of marginal probabilities upto order $k$. For $k=1$, we recover $D^{(1)} = D_{\rm Ind.}$. We can also safely define $D^{(N)} \equiv D$.  It is then natural to compare $D^{(k)}$ with $D^{(k-1)}$. If $D^{(k)} = D^{(k-1)}$, it implies that the $k$th-order dynamics is redundant, as it does not contribute to the total distance. However, if that is not the case, then the $k$th-order dynamics contribute, and we can separate the contribution as,  
\begin{align}
    D^{(k)}_{\rm Int.}=D^{(k)} -D ^{(k-1)}.
\end{align}
This yields the full decomposition of the total distance into interactions of different orders as,
\begin{align}
\label{eq:decomposition_new}
   D =  D^{(1)}_{\rm Int.}+D^{(2)}_{\rm Int.}+D^{(3)}_{\rm Int.} \cdots + D^{(N)}_{\rm Int.}.
\end{align}
Note that the decomposition above is similar in spirit to the decomposition of irreversibility in Refs. \cite{arrowoftime,lynn2022emergence}, breaking down the distance between two distributions into contributions from individual elements in the system, interactions between pairs of elements, interactions among triplets, and so on. However, the derivation of Eq. \eqref{eq:decomposition_new} does not assume multi-partite dynamics. Additionally, individual terms in the expansion, $D^{(k)}$, can be negative. In practice, $D^{(k)}$ can be computed from the knowledge of the full joint distribution or empirically obtained distributions, where only a collection of $k$ variables are measured simultaneously. 

%In a general case, this decomposition is given as, 
%\begin{align}
 %   D = \sum_{k=0}^N \sum_{S \subset X } (-1)^{k +1} \left[I_k^\prime(S) - I_k(S)\right],
%\end{align}
%where $I_k \; (I_k^\prime)$ is the $k-$way (cross) interaction information defined as, 
%\begin{align}
 %   I_k(S) = \sum_{Z \subset S} (-1)^{\vert Z-S \vert} H(Z),
%\end{align}
%where $H(Z)$ is the entropy of the variables in the subset $Z$. Terminating this expansion at different values of $k$ leads to approximations of the total distance which only considers interactions to order $k$. For example, terminating the expansion at $k=1$ can be shown to recover the independent distance in terms of the individual marginal distributions (Eq.\ \eqref{eq:Dind}). Terminating at $k=2$ is equivalent to saying that we know all the bi-variate marginals, 
%\begin{align}
 
    %P_m^{i,j}(x_t^i,x_t^j) = \int _{{\bm x_t}^{-\lbrace i,j \rbrace}}P({\bm x}_t),
%\end{align}
%which leads to the probabilistic version of the Kirkwood superposition approximation for the joint distribution $P({\bm x}_t)$. In general, for arbitrary $k<N$, the generalized Kirkwood superposition approximation is given by, 


%Now following similar steps as in Refs.\ \cite{arrowoftime,lynn2022emergence}, it is possible to show that the contributions to the distance from interactions can be further decomposed into interactions among pairs, triplets and so on. For simplicity let’s say we only know upto the interactions among pairs. This is equivalent to saying that we know all the bi-variate marginals, 
%\begin{align}
 %   P_m^{i,j}(x_t^i,x_t^j) = \int _{{\bm x_t}^{-\lbrace i,j \rbrace}}P({\bm x}_t).
%\end{align}


%We can then find out the minimum distance we get by including these pairwise interactions. We can formally do this by evaluating the distance function for all hypothetical systems that are consistent with the marginals above. Among these hypothetical systems, one will achieve a minimum, which we call the minimum distance consistent with the observed pairwise dynamics ($\equiv D^{(2)}$). By construction, this hypothetical system captures the least distance that can be inferred from the knowledge of the system up to pairwise interactions, and nothing more. If we follow the same procedure with higher order interactions, we can get a hierarchy of bounds,
%\begin{align}
%\label{eq:decomposition1}
 %   0\leq D^{(1)}\leq D^{(2)}\leq D^{(3)} ... \leq D^{(N)} = D,
%\end{align}
%where it can be verified that $ D^{(1)} = D_{\rm Ind.}$ (see Appendix B). Note that the inequalities naturally appear, because the higher order hypothetical systems are all contained in the lower order ones.

%Finally, it is natural to compare $D^{(k)}$ with $D^{(k-1)}$. If $D^{(k)} = D^{(k-1)}$, it implies that the $k$th-order dynamics are redundant, as it does not contribute to the total distance. On the other hand, if $D^{(k)} > D^{(k-1)}$, then the $k$th-order dynamics positively contributes, and we can separate out the contribution as,  
%\begin{align}
%    D^{(k)}_{\rm Int.}=D^{(k)} -D ^{(k-1)}.
%\end{align}
%This yields,
%\begin{align}
%\label{eq:decomposition2}
 %  D =  D^{(1)}_{\rm Int.}+D^{(2)}_{\rm Int.}+D^{(3)}_{\rm Int.} \cdots + D^{(N)}_{\rm Int.}.
%\end{align}
%The above equation proves our main claim: the distance between two distributions can be decomposed into non-negative contributions from individual elements in the system, interactions between pairs of elements, interactions among triplets, and so on. Note that by construction, $D^{(1)}_{\rm Int.} =D_{\rm Ind.} $.

%A few remarks are made in order: \textit{i)} Eqns.\ \eqref{eq:decomposition0} - \eqref{eq:decomposition2} demonstrate the potential decomposition of the total distance based on interactions.  In practice, $D^{(k)}$ can be computed in terms of empirically obtained distributions where only a collection of $k$ variables are measured simultaneously, similar to techniques used in Quantum-state tomography \cite{lvovsky2009continuous} or Copula modeling \cite{trivedi2007copula}. 
%Another approach involves solving a constrained convex minimization problem [as described above Eq.\ \eqref{eq:decomposition1}] over a set of hypothetical processes. Further details on this approach can be found in Ref.\ \cite{lynn2022emergence}. \textit{ii)} It is reasonable to assume that the magnitude of $D_{\rm Int.}$ in Eq.\ \eqref{eq:decomposition0} monotonically increases with the parameters $\lbrace g_{ij} \rbrace$ that determine the strength of physical interactions in the system. Therefore, by changing these parameters, one can potentially control the instantaneous distance between $P({\bm x}_t)$ and $P_f({\bm x}_t)$ in a monotonic manner. \textit{iii)} The marginal distributions that define the independent distance $D_{\rm Ind.}$ is in general different from the distributions obtained by setting the interaction parameters $\lbrace g_{ij} \rbrace \rightarrow 0$, a limit which can even be ill-defined. However, when such a limit can be taken, $D_{\rm Int.}$ vanishes and we will get $\lim_{\lbrace g_{ij} 
%\rightarrow 0\rbrace} D = \lim_{\lbrace g_{ij} 
%\rightarrow 0\rbrace} D_{\rm Ind.} \equiv D_{\rm non-interacting}$ for all $t$.  %\textit{iv)} In situations where hidden degrees of freedom are present or when not all degrees of freedom can be simultaneously measured, the actual distance between the system and a target state will exceed the \textit{partial} distance measured solely based on observable degrees of freedom. This is particularly relevant in scenarios such as the thermalization of non-equilibrium systems, where a direct relationship exists between the distance and the time required for transformations to occur \cite{bao2023universal}. Consequently, the apparent time measured based on the observable degrees will be a lower bound for true time needed for thermalization.
% Figure environment removed

% Figure environment removed

% Figure environment removed
To demonstrate the usefulness of the decomposition, we consider the problem of thermalization of two identical, hydrodynamically coupled colloidal particles in two spatially separated quadratic potential wells, as shown in Fig.\ \ref{fig:schematics}. These colloidal particles are prepared in an equilibrium state at temperature $T_0$ and then let to thermalize in an aqueous solution at temperature $T$. This model has been extensively studied both theoretically \cite{hough2002correlated,kotar2010hydrodynamic,reichert2004hydrodynamic} and experimentally \cite{paul2018two,paul2017direct}. The dynamics is governed by the Langevin equations:\\
\begin{align}
\begin{split}
\label{ eq:l}
 \dot{x}(t)&=H_{11}(-k_1x + f_1(t)) + H_{12}(-k_2y + f_2(t) )\\
 \dot{y}(t)&=H_{21}(-k_1x + f_1(t)) + H_{22}(-k_2y + f_2(t) ),
 \end{split}
\end{align} 
where  $x(t)$ and $y(t)$ represents the positions of these particles at different times. The parameters $k_1$ and $k_2$ denote the optical stiffness of the two traps. The constants $H_{11} = H_{22} = 1/(6\pi \eta a) = 1/\gamma$ \; and \; $H_{12} = H_{21} = 1/(4\pi \eta R)$, where $R$ is the center - center distance between the two traps and $a$ is the radius of the particle, are the lowest order components, in $1/R$, of the Oseen Tensor \cite{doi1988theory} for motions in the longitudinal directions.  Here $\gamma$ is the viscous drag coefficient. The value of $R$ determines the interaction between the colloidal particles. As $R\to \infty$, the interaction between the colloidal particles vanishes and our system turns to a non-interacting system. The terms $f_1(t)$ and $f_2(t)$ are the random Brownian forces which are delta correlated in time. 

Given that the system is initially prepared in a state different from its thermal equilibrium state in the new environment, it exists in a non-equilibrium state characterized by a certain distance from its eventual thermal state. Quantifying this distance in terms of Kullback-Leibler divergence has gained significant interest in recent times, primarily in the context of non-trivial thermalization behaviours such as Mpemba effects \cite{kumar2020exponentially,bechhoefer2021fresh,biswas2023mpemba,chetrite2021metastable,degunther2022anomalous} or the study of asymmetries of thermal relaxation \cite{Uphill,fewlevelQ,van2021toward,dieball2023asymmetric,meibohm2021relaxation}. In these cases,  $D_{\rm KL}\left( P({\bm x}_t) \vert\vert P_{\rm Eq} ({\bm x}_t) \right)$ is also the same as the excess free energy of the state $P({\bm x}_t)$ \textcolor{black}{which vanishes as the system equilibrates (see Refs.\ \cite{Uphill,chetrite2021metastable} for a simple derivation)}. 

For the model we consider, leveraging the fact that it is a linear system of stochastic differential equations, it is possible to analytically compute the instantaneous probability distribution $P({\bm x}_t)$ in terms of all the parameters in the system, for any value of time $t$ (See Appendix A). 
Using these solutions, it can be verified that the variables $x$ and $y$ are anti-correlated for any $t>0$. The strength of correlations increases when $R$ decreases. Further, at equilibrium (in the $t\rightarrow 0$ and $t\rightarrow \infty$ limit), the correlations vanish.



Using the exact solutions for the distributions, we can further compute  the distance function $D_{\rm KL}\left( P({\bm x}_t) \vert\vert P_{\rm Eq} ({\bm x}_t) \right)$.
In particular, when $t=0$, we get the distance between the initial equilibrium system at temperature $T_0$ and the final equilibrium system at temperature $T$, which can be used to compare initial states and pick the equivalent ones that are equidistant \cite{Uphill,fewlevelQ,van2021toward,meibohm2021relaxation} from the final thermal state. For our model, this initial distance function is found to only depend on the ratio $T_0/T$ and is given by, 
\begin{align}
    D_{\rm KL}\left( P^{T_0}_{\rm Eq}({\bm x}) \vert\vert P_{\rm Eq}^T ({\bm x})\right) = -1+\frac{T_0}{T} +\log \frac{T}{T_0}.
\end{align}
Thus, if we consider an ensemble of systems with different values of $R$, a fixed initial temperature, and an ambient temperature, all of them will have the same distance to the final thermal state at $t=0$. 
For a particular choice of parameters, we show this initial distance function in Fig.\ \ref{fig:schematics}b. The rest of the plots in this paper correspond to the point $T_0/T = 2.5$ in this curve, which has the initial distance $D_{\rm KL}\left( P^{T_0}_{\rm Eq}({\bm x}) \vert\vert P_{\rm Eq}^T ({\bm x})\right) = 0.5837$.

For arbitrary times, the distance functions  $D_{\rm KL}\left( P({\bm x}_t) \vert\vert P_{\rm Eq}^T ({\bm x}_t)\right)$ will in general depend on the the parameter $R$. Furthermore, using explicit analytical solutions of $P({\bm x}_t)$ and its marginals, we can separately compute the independent distance, interaction distance as well as the distance function in the non-interacting limit of $R\rightarrow \infty$. Since our system consists of only two interacting particles,  the decomposition in Eq.\ \eqref{eq:decomposition_new} only has two terms, namely $D^{(1)}_{\rm Int.}=D_{\rm Ind.}$ and $D^{(2)}_{\rm Int.}=D_{\rm Int.} = D - D_{\rm Ind.}$, given by,
\begin{align}
    D_{\rm Int.} = \scalebox{1}{$\int_{x_t,y_t}P(x_t,y_t)\log \frac{P(x_t,y_t)P_{{\rm Eq}, m}(x_t)P_{{\rm Eq},m}(y_t)}{P_{\rm Eq}(x_t,y_t)P_{m}(x_t)P_{m}(y_t)}$}
\end{align}
In Figure \ref{fig:Rt_dependence}, we present our central findings. Figure \ref{fig:Rt_dependence}a illustrates the plots of $D$ and $D_{\rm Ind.}$ for various values of $R$ and $t$, while keeping other model parameters fixed. At $t=0$, all states are equidistant from the final thermal state, as expected. We also find that $D_{\rm Int.} =0$ for any fixed value of $R$, even though the variables are correlated in the initial and final distributions.  This means the initial distance function can entirely be determined by the marginal statistics of $x$ and $y$.
However, for $t>0$, and any value of $R$ we observe that $D > D_{\rm Ind.}$, which means interactions positively contribute to the total distance. Specifically, when the two traps are brought closer, the value of $D$ increases for all $t$. Refer to Figure \ref{fig:Rt_dependence}b for a demonstration of this behavior with two different values of $R$.

In Figure \ref{fig:R_dependence}a, we plot the interaction distance $D_{\rm Int.}$ for varying time $t$ and different values of $R$. As $R$ decreases, the interaction distance contribution $D_{\rm Int.}$ increases. Finally, in Figure \ref{fig:R_dependence}b, we compare the total distance $D$ with the distance computed for the non-interacting case, denoted as $D_{\rm non-interacting} \equiv \lim_{R \rightarrow \infty } D $. We observe that $D_{\rm non-interacting} \leq D$ for all values of $R$ and $t$. Moreover, this bound saturates in the limit $R \rightarrow \infty$.

In summary, we have shown that, in arbitrary non-equilibrium transformations, the distance between the current state and a target state can be decomposed into two terms: one corresponding to an \textit{independent} estimate of the distance, representing hypothetical marginal processes which are non-interacting, and another corresponding to interactions, quantified using the relative mutual information between the variables. The interaction term can further be decomposed into contributions from interactions between pairs of elements, interactions among triplets, and so on. The results are demonstrated by considering the example of the thermalization of two optically trapped colloidal particles that are hydrodynamically coupled. In this case,
increasing the pairwise interaction strength is found to enhance the separation between the time-dependent non-equilibrium state and the final state, thereby increasing the longevity of the non-equilibrium initial state. 

Our results suggest that harnessing local interactions could have applications in controlling and manipulating systems towards desired states.  In setups where it is possible to control the strength of different orders of interactions, our findings offer a way to disentangle their effects on the transformation process, and to identify the ones that can assist the transformation. Further research could delve into specific applications in non-equilibrium control problems \cite{aurell2011optimal,yan2022learning,rotskoff2017geometric,abreu2012thermodynamics,chennakesavalu2023unified} where understanding these effects could be valuable, or resource theories \cite{gour2015resource}, where maintaining non-equilibrium states for extended periods could be beneficial.

\textit{Acknowledgements.-}
Nordita is partially supported by Nordforsk. MR and SKM thank the Kerala Theoretical Physics Initiative - Active Research Training (KTPI - ART) program for facilitating the research collaboration. 
SKM acknowledges the Knut and Alice Wallenberg Foundation for financial support through Grant No. KAW 2021.0328. SKM thanks the members of the Soft-Matter Group, NORDITA, Stockholm, Sweden, for helpful discussions on Refs. \cite{lynn2022emergence,arrowoftime}. SKM thanks Biswajit Das and Shuvojit Paul, Light Matter Lab, IISER Kolkata, India for helpful discussions on the model studied.

\appendix

\section{Exact calculation for the system of interacting colloids}
Here, we describe the calculation of the distance functions for the model of interacting colloids. We follow the notations in Ref.\ \cite{aykut}. To begin with, we rewrite Eq.\ \eqref{ eq:l} as a matrix equation,
\begin{equation}
    \dot {{\bm r}}(t) = - {\bm A} {\bm r}(t) + {\bm \epsilon}(t)
\end{equation}\\
with $\langle {\bm \epsilon}(t){\bm \epsilon}(s) \rangle = 2 {\bm D} \delta(t-s)$ where 
\begin{align}
\begin{split}
{\bm r}(t)&= \begin{bmatrix}
x(t) \\
y(t)
\end{bmatrix},\\ {\bm A} &= \begin{bmatrix}
\frac{k_1}{\gamma} & \frac{k_2}{4\pi \eta R} \\
\frac{k_1}{4\pi \eta R}  & \frac{k_2}{\gamma}
\end{bmatrix},\\ {\bm D} &= \left[
\begin{array}{cc}
 \frac{k_B T}{\gamma } & \frac{k_B T}{4 \pi  \eta  R} \\
 \frac{k_B T}{4 \pi  \eta  R} & \frac{k_B T}{\gamma } \\
\end{array}
\right], \\
{\bm \eta}(t)&= \begin{bmatrix}
\epsilon_1(t) \\
\epsilon_2(t)
\end{bmatrix}.
\end{split}
\end{align}
Now to find the probability distribution of the system at any time, first, we write a Fokker - Planck equation equivalent to our Langevin equation as \\
\begin{equation}\label{A.4}
    \frac{\partial P(t,{\bm r}\vert t_0,{\bm r}_0)}{\partial t} = \sum_{i,j}(\frac{\partial}{\partial r_i}[A_{ij} r_j + D_{ij} \frac{\partial}{\partial r_j}]P(t,{\bm r}\vert t_0,{\bm r}_0)) \;,\\ 
\end{equation}
where $P(t,{\bm r}|t_0,{\bm r}_0)$ is  the conditional probability that the system is in a position ${\bm r}$ at time $t$, given that it was at ${\bm r}_0$ at time $t_0$. 

The Fokker - Planck equation (\ref{A.4}) is exactly solvable, and the solution is found to be \\
\begin{equation}\label{eq:transitionProb}
 P(t,{\bm r}|t_0,{\bm r}_0)  = \scalebox{0.9}{$\frac{e^{-\frac{1}{2} [ {\bm r} - e^{-(t-t_0){\bm A}}{\bm r}_0]^T\; {\bm \Sigma}^{-1}(t-t_0) \;[{\bm r} - e^{-(t-t_0){\bm A}}{\bm r}_0]}}{\sqrt{(2 \pi )^2\; \det \;{\bm \Sigma}(t-t_0)}}$} \; ,
\end{equation}\\
where the covariance matrix is \\
\begin{equation}
    {\bm \Sigma}(t)= {\bm \Sigma} (\infty) - e^{-t {\bm A}}\; {\bm \Sigma}(\infty)\; e^{-t {\bm A}^T} \;, 
\end{equation}
and ${\bm \Sigma}(\infty)$ is found by solving the below matrix equation
\begin{equation}
    {\bm A} {\bm \Sigma}(\infty) + {\bm \Sigma}(\infty) {\bm A}^T = 2{\bm D}\\
\end{equation}
If the matrix ${\bm A}$ is positive definite, it is guaranteed that the system will reach a stationary Gaussian distribution at $t \rightarrow \infty$, which will have the covariance matrix ${\bm \Sigma}^{-1}(\infty)$.
For our model, we obtain,
\begin{align}
    {\bm \Sigma}(\infty) =\left(
\begin{array}{cc}
 \frac{k_B T }{k_1} & 0 \\
 0 & \frac{k_B T }{k_2} 
\end{array}
\right).
\end{align}
In terms of this matrix, we can obtain the equilibrium distribution of the system as, 
\begin{align}
    P_{\rm Eq}({\bm x})= \frac{1}{\sqrt{(2\pi)^2 \det {\bm \Sigma}(\infty)}} e^{-\frac{1 }{2} {\bm x} {\bm \Sigma}^{-1}(\infty) {\bm x}}.
\end{align}
Note that this distribution explicitly depends on the temperature $T$. When we set $T=T_0$, we get the equilibrium distribution at temperature $T_0$. Furthermore, the time-dependent distribution corresponding to the thermal relaxation from a distribution at an initial temperature $T_0$ to an ambient temperature $T$ can be obtained by performing the integration, 
\begin{align}
    P({\bm x}_t) = \int_{{\bm x}_0} P_{\rm Eq}^{T_0}({\bm x}_0) P(t,{\bm x}_t\vert t_0,{\bm x}_0)
\end{align}
where $P(t,{\bm x}_t\vert t_0,{\bm x}_0)$ is given by Eq.\ \eqref{eq:transitionProb}. The results in this manuscript are obtained by first explicitly evaluating this integral to get $P({\bm x}_t)$ and computing the relevant distance functions in terms of that. %For completeness, we provide a \textit{mathematica} notebook with the details of the evaluation.
%\bibliography{apssamp}% Produces the bibliography via BibTeX.
%apsrev4-2.bst 2019-01-14 (MD) hand-edited version of apsrev4-1.bst
%Control: key (0)
%Control: author (8) initials jnrlst
%Control: editor formatted (1) identically to author
%Control: production of article title (0) allowed
%Control: page (0) single
%Control: year (1) truncated
%Control: production of eprint (0) enabled
\begin{thebibliography}{61}%
\makeatletter
\providecommand \@ifxundefined [1]{%
 \@ifx{#1\undefined}
}%
\providecommand \@ifnum [1]{%
 \ifnum #1\expandafter \@firstoftwo
 \else \expandafter \@secondoftwo
 \fi
}%
\providecommand \@ifx [1]{%
 \ifx #1\expandafter \@firstoftwo
 \else \expandafter \@secondoftwo
 \fi
}%
\providecommand \natexlab [1]{#1}%
\providecommand \enquote  [1]{``#1''}%
\providecommand \bibnamefont  [1]{#1}%
\providecommand \bibfnamefont [1]{#1}%
\providecommand \citenamefont [1]{#1}%
\providecommand \href@noop [0]{\@secondoftwo}%
\providecommand \href [0]{\begingroup \@sanitize@url \@href}%
\providecommand \@href[1]{\@@startlink{#1}\@@href}%
\providecommand \@@href[1]{\endgroup#1\@@endlink}%
\providecommand \@sanitize@url [0]{\catcode `\\12\catcode `\$12\catcode
  `\&12\catcode `\#12\catcode `\^12\catcode `\_12\catcode `\%12\relax}%
\providecommand \@@startlink[1]{}%
\providecommand \@@endlink[0]{}%
\providecommand \url  [0]{\begingroup\@sanitize@url \@url }%
\providecommand \@url [1]{\endgroup\@href {#1}{\urlprefix }}%
\providecommand \urlprefix  [0]{URL }%
\providecommand \Eprint [0]{\href }%
\providecommand \doibase [0]{https://doi.org/}%
\providecommand \selectlanguage [0]{\@gobble}%
\providecommand \bibinfo  [0]{\@secondoftwo}%
\providecommand \bibfield  [0]{\@secondoftwo}%
\providecommand \translation [1]{[#1]}%
\providecommand \BibitemOpen [0]{}%
\providecommand \bibitemStop [0]{}%
\providecommand \bibitemNoStop [0]{.\EOS\space}%
\providecommand \EOS [0]{\spacefactor3000\relax}%
\providecommand \BibitemShut  [1]{\csname bibitem#1\endcsname}%
\let\auto@bib@innerbib\@empty
%</preamble>
\bibitem [{\citenamefont {Dattagupta}(2012)}]{dattagupta2012relaxation}%
  \BibitemOpen
  \bibfield  {author} {\bibinfo {author} {\bibfnamefont {S.}~\bibnamefont
  {Dattagupta}},\ }\href@noop {} {\emph {\bibinfo {title} {Relaxation phenomena
  in condensed matter physics}}}\ (\bibinfo  {publisher} {Elsevier},\ \bibinfo
  {year} {2012})\BibitemShut {NoStop}%
\bibitem [{\citenamefont {Whitesides}\ and\ \citenamefont
  {Grzybowski}(2002)}]{whitesides2002self}%
  \BibitemOpen
  \bibfield  {author} {\bibinfo {author} {\bibfnamefont {G.~M.}\ \bibnamefont
  {Whitesides}}\ and\ \bibinfo {author} {\bibfnamefont {B.}~\bibnamefont
  {Grzybowski}},\ }\bibfield  {title} {\bibinfo {title} {Self-assembly at all
  scales},\ }\href@noop {} {\bibfield  {journal} {\bibinfo  {journal}
  {Science}\ }\textbf {\bibinfo {volume} {295}},\ \bibinfo {pages} {2418}
  (\bibinfo {year} {2002})}\BibitemShut {NoStop}%
\bibitem [{\citenamefont {Pollard}\ and\ \citenamefont
  {Borisy}(2003)}]{pollard_cellular_2003}%
  \BibitemOpen
  \bibfield  {author} {\bibinfo {author} {\bibfnamefont {T.~D.}\ \bibnamefont
  {Pollard}}\ and\ \bibinfo {author} {\bibfnamefont {G.~G.}\ \bibnamefont
  {Borisy}},\ }\bibfield  {title} {\bibinfo {title} {Cellular {{Motility
  Driven}} by {{Assembly}} and {{Disassembly}} of {{Actin Filaments}}},\ }\href
  {https://doi.org/10.1016/S0092-8674(03)00120-X} {\bibfield  {journal}
  {\bibinfo  {journal} {Cell}\ }\textbf {\bibinfo {volume} {112}},\ \bibinfo
  {pages} {453} (\bibinfo {year} {2003})}\BibitemShut {NoStop}%
\bibitem [{\citenamefont {Mauro}\ \emph {et~al.}(2014)\citenamefont {Mauro},
  \citenamefont {Aliprandi}, \citenamefont {Septiadi}, \citenamefont {Kehr},\
  and\ \citenamefont {De~Cola}}]{mauro2014self}%
  \BibitemOpen
  \bibfield  {author} {\bibinfo {author} {\bibfnamefont {M.}~\bibnamefont
  {Mauro}}, \bibinfo {author} {\bibfnamefont {A.}~\bibnamefont {Aliprandi}},
  \bibinfo {author} {\bibfnamefont {D.}~\bibnamefont {Septiadi}}, \bibinfo
  {author} {\bibfnamefont {N.~S.}\ \bibnamefont {Kehr}},\ and\ \bibinfo
  {author} {\bibfnamefont {L.}~\bibnamefont {De~Cola}},\ }\bibfield  {title}
  {\bibinfo {title} {When self-assembly meets biology: luminescent platinum
  complexes for imaging applications},\ }\href@noop {} {\bibfield  {journal}
  {\bibinfo  {journal} {Chemical Society Reviews}\ }\textbf {\bibinfo {volume}
  {43}},\ \bibinfo {pages} {4144} (\bibinfo {year} {2014})}\BibitemShut
  {NoStop}%
\bibitem [{\citenamefont {Dobson}(2003)}]{dobson2003protein}%
  \BibitemOpen
  \bibfield  {author} {\bibinfo {author} {\bibfnamefont {C.~M.}\ \bibnamefont
  {Dobson}},\ }\bibfield  {title} {\bibinfo {title} {Protein folding and
  misfolding},\ }\href@noop {} {\bibfield  {journal} {\bibinfo  {journal}
  {Nature}\ }\textbf {\bibinfo {volume} {426}},\ \bibinfo {pages} {884}
  (\bibinfo {year} {2003})}\BibitemShut {NoStop}%
\bibitem [{\citenamefont {Creighton}(1990)}]{creighton1990protein}%
  \BibitemOpen
  \bibfield  {author} {\bibinfo {author} {\bibfnamefont {T.~E.}\ \bibnamefont
  {Creighton}},\ }\bibfield  {title} {\bibinfo {title} {Protein folding.},\
  }\href@noop {} {\bibfield  {journal} {\bibinfo  {journal} {Biochemical
  journal}\ }\textbf {\bibinfo {volume} {270}},\ \bibinfo {pages} {1} (\bibinfo
  {year} {1990})}\BibitemShut {NoStop}%
\bibitem [{\citenamefont {Ritort}(2006)}]{ritort2006single}%
  \BibitemOpen
  \bibfield  {author} {\bibinfo {author} {\bibfnamefont {F.}~\bibnamefont
  {Ritort}},\ }\bibfield  {title} {\bibinfo {title} {Single-molecule
  experiments in biological physics: methods and applications},\ }\href@noop {}
  {\bibfield  {journal} {\bibinfo  {journal} {Journal of Physics: Condensed
  Matter}\ }\textbf {\bibinfo {volume} {18}},\ \bibinfo {pages} {R531}
  (\bibinfo {year} {2006})}\BibitemShut {NoStop}%
\bibitem [{\citenamefont {Ciliberto}(2017)}]{ciliberto2017experiments}%
  \BibitemOpen
  \bibfield  {author} {\bibinfo {author} {\bibfnamefont {S.}~\bibnamefont
  {Ciliberto}},\ }\bibfield  {title} {\bibinfo {title} {Experiments in
  stochastic thermodynamics: Short history and perspectives},\ }\href@noop {}
  {\bibfield  {journal} {\bibinfo  {journal} {Physical Review X}\ }\textbf
  {\bibinfo {volume} {7}},\ \bibinfo {pages} {021051} (\bibinfo {year}
  {2017})}\BibitemShut {NoStop}%
\bibitem [{\citenamefont {Pop}(2010)}]{pop2010energy}%
  \BibitemOpen
  \bibfield  {author} {\bibinfo {author} {\bibfnamefont {E.}~\bibnamefont
  {Pop}},\ }\bibfield  {title} {\bibinfo {title} {Energy dissipation and
  transport in nanoscale devices},\ }\href@noop {} {\bibfield  {journal}
  {\bibinfo  {journal} {Nano Research}\ }\textbf {\bibinfo {volume} {3}},\
  \bibinfo {pages} {147} (\bibinfo {year} {2010})}\BibitemShut {NoStop}%
\bibitem [{\citenamefont {Bergfield}\ and\ \citenamefont
  {Ratner}(2013)}]{bergfield2013forty}%
  \BibitemOpen
  \bibfield  {author} {\bibinfo {author} {\bibfnamefont {J.~P.}\ \bibnamefont
  {Bergfield}}\ and\ \bibinfo {author} {\bibfnamefont {M.~A.}\ \bibnamefont
  {Ratner}},\ }\bibfield  {title} {\bibinfo {title} {Forty years of molecular
  electronics: Non-equilibrium heat and charge transport at the nanoscale},\
  }\href@noop {} {\bibfield  {journal} {\bibinfo  {journal} {physica status
  solidi (b)}\ }\textbf {\bibinfo {volume} {250}},\ \bibinfo {pages} {2249}
  (\bibinfo {year} {2013})}\BibitemShut {NoStop}%
\bibitem [{\citenamefont {Mart{\'\i}nez}\ \emph {et~al.}(2016)\citenamefont
  {Mart{\'\i}nez}, \citenamefont {Rold{\'a}n}, \citenamefont {Dinis},
  \citenamefont {Petrov}, \citenamefont {Parrondo},\ and\ \citenamefont
  {Rica}}]{martinez2016brownian}%
  \BibitemOpen
  \bibfield  {author} {\bibinfo {author} {\bibfnamefont {I.~A.}\ \bibnamefont
  {Mart{\'\i}nez}}, \bibinfo {author} {\bibfnamefont {{\'E}.}~\bibnamefont
  {Rold{\'a}n}}, \bibinfo {author} {\bibfnamefont {L.}~\bibnamefont {Dinis}},
  \bibinfo {author} {\bibfnamefont {D.}~\bibnamefont {Petrov}}, \bibinfo
  {author} {\bibfnamefont {J.~M.}\ \bibnamefont {Parrondo}},\ and\ \bibinfo
  {author} {\bibfnamefont {R.~A.}\ \bibnamefont {Rica}},\ }\bibfield  {title}
  {\bibinfo {title} {Brownian carnot engine},\ }\href@noop {} {\bibfield
  {journal} {\bibinfo  {journal} {Nature physics}\ }\textbf {\bibinfo {volume}
  {12}},\ \bibinfo {pages} {67} (\bibinfo {year} {2016})}\BibitemShut {NoStop}%
\bibitem [{\citenamefont {Sagawa}\ and\ \citenamefont
  {Ueda}(2012)}]{sagawa2012fluctuation}%
  \BibitemOpen
  \bibfield  {author} {\bibinfo {author} {\bibfnamefont {T.}~\bibnamefont
  {Sagawa}}\ and\ \bibinfo {author} {\bibfnamefont {M.}~\bibnamefont {Ueda}},\
  }\bibfield  {title} {\bibinfo {title} {Fluctuation theorem with information
  exchange: Role of correlations in stochastic thermodynamics},\ }\href@noop {}
  {\bibfield  {journal} {\bibinfo  {journal} {Physical review letters}\
  }\textbf {\bibinfo {volume} {109}},\ \bibinfo {pages} {180602} (\bibinfo
  {year} {2012})}\BibitemShut {NoStop}%
\bibitem [{\citenamefont {Barato}\ and\ \citenamefont
  {Seifert}(2014)}]{barato2014unifying}%
  \BibitemOpen
  \bibfield  {author} {\bibinfo {author} {\bibfnamefont {A.}~\bibnamefont
  {Barato}}\ and\ \bibinfo {author} {\bibfnamefont {U.}~\bibnamefont
  {Seifert}},\ }\bibfield  {title} {\bibinfo {title} {Unifying three
  perspectives on information processing in stochastic thermodynamics},\
  }\href@noop {} {\bibfield  {journal} {\bibinfo  {journal} {Physical review
  letters}\ }\textbf {\bibinfo {volume} {112}},\ \bibinfo {pages} {090601}
  (\bibinfo {year} {2014})}\BibitemShut {NoStop}%
\bibitem [{\citenamefont {Aurell}\ \emph {et~al.}(2012)\citenamefont {Aurell},
  \citenamefont {Gawedzki}, \citenamefont {Mejia-Monasterio}, \citenamefont
  {Mohayaee},\ and\ \citenamefont {Muratore-Ginanneschi}}]{aurell2012refined}%
  \BibitemOpen
  \bibfield  {author} {\bibinfo {author} {\bibfnamefont {E.}~\bibnamefont
  {Aurell}}, \bibinfo {author} {\bibfnamefont {K.}~\bibnamefont {Gawedzki}},
  \bibinfo {author} {\bibfnamefont {C.}~\bibnamefont {Mejia-Monasterio}},
  \bibinfo {author} {\bibfnamefont {R.}~\bibnamefont {Mohayaee}},\ and\
  \bibinfo {author} {\bibfnamefont {P.}~\bibnamefont {Muratore-Ginanneschi}},\
  }\bibfield  {title} {\bibinfo {title} {Refined second law of thermodynamics
  for fast random processes},\ }\href@noop {} {\bibfield  {journal} {\bibinfo
  {journal} {Journal of statistical physics}\ }\textbf {\bibinfo {volume}
  {147}},\ \bibinfo {pages} {487} (\bibinfo {year} {2012})}\BibitemShut
  {NoStop}%
\bibitem [{\citenamefont {Deffner}\ and\ \citenamefont {Lutz}(2010)}]{Lutz}%
  \BibitemOpen
  \bibfield  {author} {\bibinfo {author} {\bibfnamefont {S.}~\bibnamefont
  {Deffner}}\ and\ \bibinfo {author} {\bibfnamefont {E.}~\bibnamefont {Lutz}},\
  }\bibfield  {title} {\bibinfo {title} {Generalized clausius inequality for
  nonequilibrium quantum processes},\ }\href
  {https://doi.org/10.1103/PhysRevLett.105.170402} {\bibfield  {journal}
  {\bibinfo  {journal} {Phys. Rev. Lett.}\ }\textbf {\bibinfo {volume} {105}},\
  \bibinfo {pages} {170402} (\bibinfo {year} {2010})}\BibitemShut {NoStop}%
\bibitem [{\citenamefont {Kim}(2021)}]{kim2021information}%
  \BibitemOpen
  \bibfield  {author} {\bibinfo {author} {\bibfnamefont {E.-j.}\ \bibnamefont
  {Kim}},\ }\bibfield  {title} {\bibinfo {title} {Information geometry,
  fluctuations, non-equilibrium thermodynamics, and geodesics in complex
  systems},\ }\href@noop {} {\bibfield  {journal} {\bibinfo  {journal}
  {Entropy}\ }\textbf {\bibinfo {volume} {23}},\ \bibinfo {pages} {1393}
  (\bibinfo {year} {2021})}\BibitemShut {NoStop}%
\bibitem [{\citenamefont {Nakazato}\ and\ \citenamefont
  {Ito}(2021)}]{nakazato2021geometrical}%
  \BibitemOpen
  \bibfield  {author} {\bibinfo {author} {\bibfnamefont {M.}~\bibnamefont
  {Nakazato}}\ and\ \bibinfo {author} {\bibfnamefont {S.}~\bibnamefont {Ito}},\
  }\bibfield  {title} {\bibinfo {title} {Geometrical aspects of entropy
  production in stochastic thermodynamics based on wasserstein distance},\
  }\href@noop {} {\bibfield  {journal} {\bibinfo  {journal} {Physical Review
  Research}\ }\textbf {\bibinfo {volume} {3}},\ \bibinfo {pages} {043093}
  (\bibinfo {year} {2021})}\BibitemShut {NoStop}%
\bibitem [{\citenamefont {Ito}(2023)}]{ito2023geometric}%
  \BibitemOpen
  \bibfield  {author} {\bibinfo {author} {\bibfnamefont {S.}~\bibnamefont
  {Ito}},\ }\bibfield  {title} {\bibinfo {title} {Geometric thermodynamics for
  the fokker--planck equation: stochastic thermodynamic links between
  information geometry and optimal transport},\ }\href@noop {} {\bibfield
  {journal} {\bibinfo  {journal} {Information Geometry}\ ,\ \bibinfo {pages}
  {1}} (\bibinfo {year} {2023})}\BibitemShut {NoStop}%
\bibitem [{\citenamefont {Chennakesavalu}\ and\ \citenamefont
  {Rotskoff}(2023)}]{chennakesavalu2023unified}%
  \BibitemOpen
  \bibfield  {author} {\bibinfo {author} {\bibfnamefont {S.}~\bibnamefont
  {Chennakesavalu}}\ and\ \bibinfo {author} {\bibfnamefont {G.~M.}\
  \bibnamefont {Rotskoff}},\ }\bibfield  {title} {\bibinfo {title} {Unified,
  geometric framework for nonequilibrium protocol optimization},\ }\href@noop
  {} {\bibfield  {journal} {\bibinfo  {journal} {Physical Review Letters}\
  }\textbf {\bibinfo {volume} {130}},\ \bibinfo {pages} {107101} (\bibinfo
  {year} {2023})}\BibitemShut {NoStop}%
\bibitem [{\citenamefont {Rotskoff}\ and\ \citenamefont
  {Crooks}(2015)}]{rotskoff2015optimal}%
  \BibitemOpen
  \bibfield  {author} {\bibinfo {author} {\bibfnamefont {G.~M.}\ \bibnamefont
  {Rotskoff}}\ and\ \bibinfo {author} {\bibfnamefont {G.~E.}\ \bibnamefont
  {Crooks}},\ }\bibfield  {title} {\bibinfo {title} {Optimal control in
  nonequilibrium systems: Dynamic riemannian geometry of the ising model},\
  }\href@noop {} {\bibfield  {journal} {\bibinfo  {journal} {Physical Review
  E}\ }\textbf {\bibinfo {volume} {92}},\ \bibinfo {pages} {060102} (\bibinfo
  {year} {2015})}\BibitemShut {NoStop}%
\bibitem [{\citenamefont {Shiraishi}\ \emph {et~al.}(2018)\citenamefont
  {Shiraishi}, \citenamefont {Funo},\ and\ \citenamefont
  {Saito}}]{shiraishi2018speed}%
  \BibitemOpen
  \bibfield  {author} {\bibinfo {author} {\bibfnamefont {N.}~\bibnamefont
  {Shiraishi}}, \bibinfo {author} {\bibfnamefont {K.}~\bibnamefont {Funo}},\
  and\ \bibinfo {author} {\bibfnamefont {K.}~\bibnamefont {Saito}},\ }\bibfield
   {title} {\bibinfo {title} {Speed limit for classical stochastic processes},\
  }\href@noop {} {\bibfield  {journal} {\bibinfo  {journal} {Physical review
  letters}\ }\textbf {\bibinfo {volume} {121}},\ \bibinfo {pages} {070601}
  (\bibinfo {year} {2018})}\BibitemShut {NoStop}%
\bibitem [{\citenamefont {Van~Vu}\ and\ \citenamefont
  {Saito}(2023{\natexlab{a}})}]{van2023topological}%
  \BibitemOpen
  \bibfield  {author} {\bibinfo {author} {\bibfnamefont {T.}~\bibnamefont
  {Van~Vu}}\ and\ \bibinfo {author} {\bibfnamefont {K.}~\bibnamefont {Saito}},\
  }\bibfield  {title} {\bibinfo {title} {Topological speed limit},\ }\href@noop
  {} {\bibfield  {journal} {\bibinfo  {journal} {Physical review letters}\
  }\textbf {\bibinfo {volume} {130}},\ \bibinfo {pages} {010402} (\bibinfo
  {year} {2023}{\natexlab{a}})}\BibitemShut {NoStop}%
\bibitem [{\citenamefont {Yoshimura}\ and\ \citenamefont
  {Ito}(2021)}]{yoshimura2021thermodynamic}%
  \BibitemOpen
  \bibfield  {author} {\bibinfo {author} {\bibfnamefont {K.}~\bibnamefont
  {Yoshimura}}\ and\ \bibinfo {author} {\bibfnamefont {S.}~\bibnamefont
  {Ito}},\ }\bibfield  {title} {\bibinfo {title} {Thermodynamic uncertainty
  relation and thermodynamic speed limit in deterministic chemical reaction
  networks},\ }\href@noop {} {\bibfield  {journal} {\bibinfo  {journal}
  {Physical review letters}\ }\textbf {\bibinfo {volume} {127}},\ \bibinfo
  {pages} {160601} (\bibinfo {year} {2021})}\BibitemShut {NoStop}%
\bibitem [{\citenamefont {Funo}\ \emph {et~al.}(2019)\citenamefont {Funo},
  \citenamefont {Shiraishi},\ and\ \citenamefont {Saito}}]{funo2019speed}%
  \BibitemOpen
  \bibfield  {author} {\bibinfo {author} {\bibfnamefont {K.}~\bibnamefont
  {Funo}}, \bibinfo {author} {\bibfnamefont {N.}~\bibnamefont {Shiraishi}},\
  and\ \bibinfo {author} {\bibfnamefont {K.}~\bibnamefont {Saito}},\ }\bibfield
   {title} {\bibinfo {title} {Speed limit for open quantum systems},\
  }\href@noop {} {\bibfield  {journal} {\bibinfo  {journal} {New Journal of
  Physics}\ }\textbf {\bibinfo {volume} {21}},\ \bibinfo {pages} {013006}
  (\bibinfo {year} {2019})}\BibitemShut {NoStop}%
\bibitem [{\citenamefont {Lee}\ \emph {et~al.}(2022)\citenamefont {Lee},
  \citenamefont {Lee}, \citenamefont {Kwon},\ and\ \citenamefont
  {Park}}]{lee2022speed}%
  \BibitemOpen
  \bibfield  {author} {\bibinfo {author} {\bibfnamefont {J.~S.}\ \bibnamefont
  {Lee}}, \bibinfo {author} {\bibfnamefont {S.}~\bibnamefont {Lee}}, \bibinfo
  {author} {\bibfnamefont {H.}~\bibnamefont {Kwon}},\ and\ \bibinfo {author}
  {\bibfnamefont {H.}~\bibnamefont {Park}},\ }\bibfield  {title} {\bibinfo
  {title} {Speed limit for a highly irreversible process and tight finite-time
  landauer’s bound},\ }\href@noop {} {\bibfield  {journal} {\bibinfo
  {journal} {Physical review letters}\ }\textbf {\bibinfo {volume} {129}},\
  \bibinfo {pages} {120603} (\bibinfo {year} {2022})}\BibitemShut {NoStop}%
\bibitem [{\citenamefont {Falasco}\ and\ \citenamefont
  {Esposito}(2020)}]{falasco2020dissipation}%
  \BibitemOpen
  \bibfield  {author} {\bibinfo {author} {\bibfnamefont {G.}~\bibnamefont
  {Falasco}}\ and\ \bibinfo {author} {\bibfnamefont {M.}~\bibnamefont
  {Esposito}},\ }\bibfield  {title} {\bibinfo {title} {Dissipation-time
  uncertainty relation},\ }\href@noop {} {\bibfield  {journal} {\bibinfo
  {journal} {Physical Review Letters}\ }\textbf {\bibinfo {volume} {125}},\
  \bibinfo {pages} {120604} (\bibinfo {year} {2020})}\BibitemShut {NoStop}%
\bibitem [{\citenamefont {Van~Vu}\ and\ \citenamefont
  {Saito}(2023{\natexlab{b}})}]{van2023thermodynamic}%
  \BibitemOpen
  \bibfield  {author} {\bibinfo {author} {\bibfnamefont {T.}~\bibnamefont
  {Van~Vu}}\ and\ \bibinfo {author} {\bibfnamefont {K.}~\bibnamefont {Saito}},\
  }\bibfield  {title} {\bibinfo {title} {Thermodynamic unification of optimal
  transport: Thermodynamic uncertainty relation, minimum dissipation, and
  thermodynamic speed limits},\ }\href@noop {} {\bibfield  {journal} {\bibinfo
  {journal} {Physical Review X}\ }\textbf {\bibinfo {volume} {13}},\ \bibinfo
  {pages} {011013} (\bibinfo {year} {2023}{\natexlab{b}})}\BibitemShut
  {NoStop}%
\bibitem [{\citenamefont {Kuznets-Speck}\ and\ \citenamefont
  {Limmer}(2021)}]{kuznets2021dissipation}%
  \BibitemOpen
  \bibfield  {author} {\bibinfo {author} {\bibfnamefont {B.}~\bibnamefont
  {Kuznets-Speck}}\ and\ \bibinfo {author} {\bibfnamefont {D.~T.}\ \bibnamefont
  {Limmer}},\ }\bibfield  {title} {\bibinfo {title} {Dissipation bounds the
  amplification of transition rates far from equilibrium},\ }\href@noop {}
  {\bibfield  {journal} {\bibinfo  {journal} {Proceedings of the National
  Academy of Sciences}\ }\textbf {\bibinfo {volume} {118}},\ \bibinfo {pages}
  {e2020863118} (\bibinfo {year} {2021})}\BibitemShut {NoStop}%
\bibitem [{\citenamefont {Lynn}\ \emph
  {et~al.}(2022{\natexlab{a}})\citenamefont {Lynn}, \citenamefont {Holmes},
  \citenamefont {Bialek},\ and\ \citenamefont {Schwab}}]{arrowoftime}%
  \BibitemOpen
  \bibfield  {author} {\bibinfo {author} {\bibfnamefont {C.~W.}\ \bibnamefont
  {Lynn}}, \bibinfo {author} {\bibfnamefont {C.~M.}\ \bibnamefont {Holmes}},
  \bibinfo {author} {\bibfnamefont {W.}~\bibnamefont {Bialek}},\ and\ \bibinfo
  {author} {\bibfnamefont {D.~J.}\ \bibnamefont {Schwab}},\ }\bibfield  {title}
  {\bibinfo {title} {Decomposing the local arrow of time in interacting
  systems},\ }\href {https://doi.org/10.1103/PhysRevLett.129.118101} {\bibfield
   {journal} {\bibinfo  {journal} {Phys. Rev. Lett.}\ }\textbf {\bibinfo
  {volume} {129}},\ \bibinfo {pages} {118101} (\bibinfo {year}
  {2022}{\natexlab{a}})}\BibitemShut {NoStop}%
\bibitem [{\citenamefont {Lynn}\ \emph
  {et~al.}(2022{\natexlab{b}})\citenamefont {Lynn}, \citenamefont {Holmes},
  \citenamefont {Bialek},\ and\ \citenamefont {Schwab}}]{lynn2022emergence}%
  \BibitemOpen
  \bibfield  {author} {\bibinfo {author} {\bibfnamefont {C.~W.}\ \bibnamefont
  {Lynn}}, \bibinfo {author} {\bibfnamefont {C.~M.}\ \bibnamefont {Holmes}},
  \bibinfo {author} {\bibfnamefont {W.}~\bibnamefont {Bialek}},\ and\ \bibinfo
  {author} {\bibfnamefont {D.~J.}\ \bibnamefont {Schwab}},\ }\bibfield  {title}
  {\bibinfo {title} {Emergence of local irreversibility in complex interacting
  systems},\ }\href@noop {} {\bibfield  {journal} {\bibinfo  {journal}
  {Physical Review E}\ }\textbf {\bibinfo {volume} {106}},\ \bibinfo {pages}
  {034102} (\bibinfo {year} {2022}{\natexlab{b}})}\BibitemShut {NoStop}%
\bibitem [{\citenamefont {Ito}(2018)}]{ito2018stochastic}%
  \BibitemOpen
  \bibfield  {author} {\bibinfo {author} {\bibfnamefont {S.}~\bibnamefont
  {Ito}},\ }\bibfield  {title} {\bibinfo {title} {Stochastic thermodynamic
  interpretation of information geometry},\ }\href@noop {} {\bibfield
  {journal} {\bibinfo  {journal} {Physical review letters}\ }\textbf {\bibinfo
  {volume} {121}},\ \bibinfo {pages} {030605} (\bibinfo {year}
  {2018})}\BibitemShut {NoStop}%
\bibitem [{\citenamefont {Aurell}\ \emph {et~al.}(2011)\citenamefont {Aurell},
  \citenamefont {Mej{\'\i}a-Monasterio},\ and\ \citenamefont
  {Muratore-Ginanneschi}}]{aurell2011optimal}%
  \BibitemOpen
  \bibfield  {author} {\bibinfo {author} {\bibfnamefont {E.}~\bibnamefont
  {Aurell}}, \bibinfo {author} {\bibfnamefont {C.}~\bibnamefont
  {Mej{\'\i}a-Monasterio}},\ and\ \bibinfo {author} {\bibfnamefont
  {P.}~\bibnamefont {Muratore-Ginanneschi}},\ }\bibfield  {title} {\bibinfo
  {title} {Optimal protocols and optimal transport in stochastic
  thermodynamics},\ }\href@noop {} {\bibfield  {journal} {\bibinfo  {journal}
  {Physical review letters}\ }\textbf {\bibinfo {volume} {106}},\ \bibinfo
  {pages} {250601} (\bibinfo {year} {2011})}\BibitemShut {NoStop}%
\bibitem [{\citenamefont {Shiraishi}\ and\ \citenamefont
  {Saito}(2019)}]{shiraishi2019information}%
  \BibitemOpen
  \bibfield  {author} {\bibinfo {author} {\bibfnamefont {N.}~\bibnamefont
  {Shiraishi}}\ and\ \bibinfo {author} {\bibfnamefont {K.}~\bibnamefont
  {Saito}},\ }\bibfield  {title} {\bibinfo {title} {Information-theoretical
  bound of the irreversibility in thermal relaxation processes},\ }\href@noop
  {} {\bibfield  {journal} {\bibinfo  {journal} {Physical review letters}\
  }\textbf {\bibinfo {volume} {123}},\ \bibinfo {pages} {110603} (\bibinfo
  {year} {2019})}\BibitemShut {NoStop}%
\bibitem [{\citenamefont {Ch{\'e}trite}\ \emph {et~al.}(2021)\citenamefont
  {Ch{\'e}trite}, \citenamefont {Kumar},\ and\ \citenamefont
  {Bechhoefer}}]{chetrite2021metastable}%
  \BibitemOpen
  \bibfield  {author} {\bibinfo {author} {\bibfnamefont {R.}~\bibnamefont
  {Ch{\'e}trite}}, \bibinfo {author} {\bibfnamefont {A.}~\bibnamefont
  {Kumar}},\ and\ \bibinfo {author} {\bibfnamefont {J.}~\bibnamefont
  {Bechhoefer}},\ }\bibfield  {title} {\bibinfo {title} {The metastable mpemba
  effect corresponds to a non-monotonic temperature dependence of extractable
  work},\ }\href@noop {} {\bibfield  {journal} {\bibinfo  {journal} {arXiv
  preprint arXiv:2101.06394}\ } (\bibinfo {year} {2021})}\BibitemShut {NoStop}%
\bibitem [{\citenamefont {McClendon}\ \emph {et~al.}(2012)\citenamefont
  {McClendon}, \citenamefont {Hua}, \citenamefont {Barreiro},\ and\
  \citenamefont {Jacobson}}]{mcclendon2012comparing}%
  \BibitemOpen
  \bibfield  {author} {\bibinfo {author} {\bibfnamefont {C.~L.}\ \bibnamefont
  {McClendon}}, \bibinfo {author} {\bibfnamefont {L.}~\bibnamefont {Hua}},
  \bibinfo {author} {\bibfnamefont {G.}~\bibnamefont {Barreiro}},\ and\
  \bibinfo {author} {\bibfnamefont {M.~P.}\ \bibnamefont {Jacobson}},\
  }\bibfield  {title} {\bibinfo {title} {Comparing conformational ensembles
  using the kullback--leibler divergence expansion},\ }\href@noop {} {\bibfield
   {journal} {\bibinfo  {journal} {Journal of chemical theory and computation}\
  }\textbf {\bibinfo {volume} {8}},\ \bibinfo {pages} {2115} (\bibinfo {year}
  {2012})}\BibitemShut {NoStop}%
\bibitem [{\citenamefont {Galas}\ \emph {et~al.}(2017)\citenamefont {Galas},
  \citenamefont {Dewey}, \citenamefont {Kunert-Graf},\ and\ \citenamefont
  {Sakhanenko}}]{galas2017expansion}%
  \BibitemOpen
  \bibfield  {author} {\bibinfo {author} {\bibfnamefont {D.~J.}\ \bibnamefont
  {Galas}}, \bibinfo {author} {\bibfnamefont {G.}~\bibnamefont {Dewey}},
  \bibinfo {author} {\bibfnamefont {J.}~\bibnamefont {Kunert-Graf}},\ and\
  \bibinfo {author} {\bibfnamefont {N.~A.}\ \bibnamefont {Sakhanenko}},\
  }\bibfield  {title} {\bibinfo {title} {Expansion of the kullback-leibler
  divergence, and a new class of information metrics},\ }\href@noop {}
  {\bibfield  {journal} {\bibinfo  {journal} {Axioms}\ }\textbf {\bibinfo
  {volume} {6}},\ \bibinfo {pages} {8} (\bibinfo {year} {2017})}\BibitemShut
  {NoStop}%
\bibitem [{\citenamefont {Tritchler}\ \emph {et~al.}(2011)\citenamefont
  {Tritchler}, \citenamefont {Sucheston}, \citenamefont {Chanda},\ and\
  \citenamefont {Ramanathan}}]{tritchler2011information}%
  \BibitemOpen
  \bibfield  {author} {\bibinfo {author} {\bibfnamefont {D.~L.}\ \bibnamefont
  {Tritchler}}, \bibinfo {author} {\bibfnamefont {L.}~\bibnamefont
  {Sucheston}}, \bibinfo {author} {\bibfnamefont {P.}~\bibnamefont {Chanda}},\
  and\ \bibinfo {author} {\bibfnamefont {M.}~\bibnamefont {Ramanathan}},\
  }\bibfield  {title} {\bibinfo {title} {Information metrics in genetic
  epidemiology},\ }\href@noop {} {\bibfield  {journal} {\bibinfo  {journal}
  {Statistical applications in genetics and molecular biology}\ }\textbf
  {\bibinfo {volume} {10}} (\bibinfo {year} {2011})}\BibitemShut {NoStop}%
\bibitem [{\citenamefont {Lu}\ and\ \citenamefont
  {Raz}(2017)}]{lu2017nonequilibrium}%
  \BibitemOpen
  \bibfield  {author} {\bibinfo {author} {\bibfnamefont {Z.}~\bibnamefont
  {Lu}}\ and\ \bibinfo {author} {\bibfnamefont {O.}~\bibnamefont {Raz}},\
  }\bibfield  {title} {\bibinfo {title} {Nonequilibrium thermodynamics of the
  markovian mpemba effect and its inverse},\ }\href@noop {} {\bibfield
  {journal} {\bibinfo  {journal} {Proceedings of the National Academy of
  Sciences}\ }\textbf {\bibinfo {volume} {114}},\ \bibinfo {pages} {5083}
  (\bibinfo {year} {2017})}\BibitemShut {NoStop}%
\bibitem [{\citenamefont {Watanabe}(1960)}]{watanabe1960information}%
  \BibitemOpen
  \bibfield  {author} {\bibinfo {author} {\bibfnamefont {S.}~\bibnamefont
  {Watanabe}},\ }\bibfield  {title} {\bibinfo {title} {Information theoretical
  analysis of multivariate correlation},\ }\href@noop {} {\bibfield  {journal}
  {\bibinfo  {journal} {IBM Journal of research and development}\ }\textbf
  {\bibinfo {volume} {4}},\ \bibinfo {pages} {66} (\bibinfo {year}
  {1960})}\BibitemShut {NoStop}%
\bibitem [{\citenamefont {Somani}\ \emph {et~al.}(2009)\citenamefont {Somani},
  \citenamefont {Killian},\ and\ \citenamefont {Gilson}}]{somani2009sampling}%
  \BibitemOpen
  \bibfield  {author} {\bibinfo {author} {\bibfnamefont {S.}~\bibnamefont
  {Somani}}, \bibinfo {author} {\bibfnamefont {B.~J.}\ \bibnamefont
  {Killian}},\ and\ \bibinfo {author} {\bibfnamefont {M.~K.}\ \bibnamefont
  {Gilson}},\ }\bibfield  {title} {\bibinfo {title} {Sampling conformations in
  high dimensions using low-dimensional distribution functions},\ }\href@noop
  {} {\bibfield  {journal} {\bibinfo  {journal} {The Journal of chemical
  physics}\ }\textbf {\bibinfo {volume} {130}} (\bibinfo {year}
  {2009})}\BibitemShut {NoStop}%
\bibitem [{\citenamefont {Killian}\ \emph {et~al.}(2007)\citenamefont
  {Killian}, \citenamefont {Yundenfreund~Kravitz},\ and\ \citenamefont
  {Gilson}}]{killian2007extraction}%
  \BibitemOpen
  \bibfield  {author} {\bibinfo {author} {\bibfnamefont {B.~J.}\ \bibnamefont
  {Killian}}, \bibinfo {author} {\bibfnamefont {J.}~\bibnamefont
  {Yundenfreund~Kravitz}},\ and\ \bibinfo {author} {\bibfnamefont {M.~K.}\
  \bibnamefont {Gilson}},\ }\bibfield  {title} {\bibinfo {title} {Extraction of
  configurational entropy from molecular simulations via an expansion
  approximation},\ }\href@noop {} {\bibfield  {journal} {\bibinfo  {journal}
  {The Journal of chemical physics}\ }\textbf {\bibinfo {volume} {127}}
  (\bibinfo {year} {2007})}\BibitemShut {NoStop}%
\bibitem [{\citenamefont {Hough}\ and\ \citenamefont
  {Ou-Yang}(2002)}]{hough2002correlated}%
  \BibitemOpen
  \bibfield  {author} {\bibinfo {author} {\bibfnamefont {L.}~\bibnamefont
  {Hough}}\ and\ \bibinfo {author} {\bibfnamefont {H.}~\bibnamefont
  {Ou-Yang}},\ }\bibfield  {title} {\bibinfo {title} {Correlated motions of two
  hydrodynamically coupled particles confined in separate quadratic potential
  wells},\ }\href@noop {} {\bibfield  {journal} {\bibinfo  {journal} {Physical
  Review E}\ }\textbf {\bibinfo {volume} {65}},\ \bibinfo {pages} {021906}
  (\bibinfo {year} {2002})}\BibitemShut {NoStop}%
\bibitem [{\citenamefont {Kotar}\ \emph {et~al.}(2010)\citenamefont {Kotar},
  \citenamefont {Leoni}, \citenamefont {Bassetti}, \citenamefont
  {Lagomarsino},\ and\ \citenamefont {Cicuta}}]{kotar2010hydrodynamic}%
  \BibitemOpen
  \bibfield  {author} {\bibinfo {author} {\bibfnamefont {J.}~\bibnamefont
  {Kotar}}, \bibinfo {author} {\bibfnamefont {M.}~\bibnamefont {Leoni}},
  \bibinfo {author} {\bibfnamefont {B.}~\bibnamefont {Bassetti}}, \bibinfo
  {author} {\bibfnamefont {M.~C.}\ \bibnamefont {Lagomarsino}},\ and\ \bibinfo
  {author} {\bibfnamefont {P.}~\bibnamefont {Cicuta}},\ }\bibfield  {title}
  {\bibinfo {title} {Hydrodynamic synchronization of colloidal oscillators},\
  }\href@noop {} {\bibfield  {journal} {\bibinfo  {journal} {Proceedings of the
  National Academy of Sciences}\ }\textbf {\bibinfo {volume} {107}},\ \bibinfo
  {pages} {7669} (\bibinfo {year} {2010})}\BibitemShut {NoStop}%
\bibitem [{\citenamefont {Reichert}\ and\ \citenamefont
  {Stark}(2004)}]{reichert2004hydrodynamic}%
  \BibitemOpen
  \bibfield  {author} {\bibinfo {author} {\bibfnamefont {M.}~\bibnamefont
  {Reichert}}\ and\ \bibinfo {author} {\bibfnamefont {H.}~\bibnamefont
  {Stark}},\ }\bibfield  {title} {\bibinfo {title} {Hydrodynamic coupling of
  two rotating spheres trapped in harmonic potentials},\ }\href@noop {}
  {\bibfield  {journal} {\bibinfo  {journal} {Physical Review E}\ }\textbf
  {\bibinfo {volume} {69}},\ \bibinfo {pages} {031407} (\bibinfo {year}
  {2004})}\BibitemShut {NoStop}%
\bibitem [{\citenamefont {Paul}\ \emph {et~al.}(2018)\citenamefont {Paul},
  \citenamefont {Kumar},\ and\ \citenamefont {Banerjee}}]{paul2018two}%
  \BibitemOpen
  \bibfield  {author} {\bibinfo {author} {\bibfnamefont {S.}~\bibnamefont
  {Paul}}, \bibinfo {author} {\bibfnamefont {R.}~\bibnamefont {Kumar}},\ and\
  \bibinfo {author} {\bibfnamefont {A.}~\bibnamefont {Banerjee}},\ }\bibfield
  {title} {\bibinfo {title} {Two-point active microrheology in a viscous medium
  exploiting a motional resonance excited in dual-trap optical tweezers},\
  }\href@noop {} {\bibfield  {journal} {\bibinfo  {journal} {Physical Review
  E}\ }\textbf {\bibinfo {volume} {97}},\ \bibinfo {pages} {042606} (\bibinfo
  {year} {2018})}\BibitemShut {NoStop}%
\bibitem [{\citenamefont {Paul}\ \emph {et~al.}(2017)\citenamefont {Paul},
  \citenamefont {Laskar}, \citenamefont {Singh}, \citenamefont {Roy},
  \citenamefont {Adhikari},\ and\ \citenamefont {Banerjee}}]{paul2017direct}%
  \BibitemOpen
  \bibfield  {author} {\bibinfo {author} {\bibfnamefont {S.}~\bibnamefont
  {Paul}}, \bibinfo {author} {\bibfnamefont {A.}~\bibnamefont {Laskar}},
  \bibinfo {author} {\bibfnamefont {R.}~\bibnamefont {Singh}}, \bibinfo
  {author} {\bibfnamefont {B.}~\bibnamefont {Roy}}, \bibinfo {author}
  {\bibfnamefont {R.}~\bibnamefont {Adhikari}},\ and\ \bibinfo {author}
  {\bibfnamefont {A.}~\bibnamefont {Banerjee}},\ }\bibfield  {title} {\bibinfo
  {title} {Direct verification of the fluctuation-dissipation relation in
  viscously coupled oscillators},\ }\href@noop {} {\bibfield  {journal}
  {\bibinfo  {journal} {Physical Review E}\ }\textbf {\bibinfo {volume} {96}},\
  \bibinfo {pages} {050102} (\bibinfo {year} {2017})}\BibitemShut {NoStop}%
\bibitem [{\citenamefont {Doi}\ and\ \citenamefont
  {Edwards}(1988)}]{doi1988theory}%
  \BibitemOpen
  \bibfield  {author} {\bibinfo {author} {\bibfnamefont {M.}~\bibnamefont
  {Doi}}\ and\ \bibinfo {author} {\bibfnamefont {S.~F.}\ \bibnamefont
  {Edwards}},\ }\href@noop {} {\emph {\bibinfo {title} {The theory of polymer
  dynamics}}},\ Vol.~\bibinfo {volume} {73}\ (\bibinfo  {publisher} {oxford
  university press},\ \bibinfo {year} {1988})\BibitemShut {NoStop}%
\bibitem [{\citenamefont {Kumar}\ and\ \citenamefont
  {Bechhoefer}(2020)}]{kumar2020exponentially}%
  \BibitemOpen
  \bibfield  {author} {\bibinfo {author} {\bibfnamefont {A.}~\bibnamefont
  {Kumar}}\ and\ \bibinfo {author} {\bibfnamefont {J.}~\bibnamefont
  {Bechhoefer}},\ }\bibfield  {title} {\bibinfo {title} {Exponentially faster
  cooling in a colloidal system},\ }\href@noop {} {\bibfield  {journal}
  {\bibinfo  {journal} {Nature}\ }\textbf {\bibinfo {volume} {584}},\ \bibinfo
  {pages} {64} (\bibinfo {year} {2020})}\BibitemShut {NoStop}%
\bibitem [{\citenamefont {Bechhoefer}\ \emph {et~al.}(2021)\citenamefont
  {Bechhoefer}, \citenamefont {Kumar},\ and\ \citenamefont
  {Ch{\'e}trite}}]{bechhoefer2021fresh}%
  \BibitemOpen
  \bibfield  {author} {\bibinfo {author} {\bibfnamefont {J.}~\bibnamefont
  {Bechhoefer}}, \bibinfo {author} {\bibfnamefont {A.}~\bibnamefont {Kumar}},\
  and\ \bibinfo {author} {\bibfnamefont {R.}~\bibnamefont {Ch{\'e}trite}},\
  }\bibfield  {title} {\bibinfo {title} {A fresh understanding of the mpemba
  effect},\ }\href@noop {} {\bibfield  {journal} {\bibinfo  {journal} {Nature
  Reviews Physics}\ ,\ \bibinfo {pages} {1}} (\bibinfo {year}
  {2021})}\BibitemShut {NoStop}%
\bibitem [{\citenamefont {Biswas}\ and\ \citenamefont
  {Rajesh}(2023)}]{biswas2023mpemba}%
  \BibitemOpen
  \bibfield  {author} {\bibinfo {author} {\bibfnamefont {A.}~\bibnamefont
  {Biswas}}\ and\ \bibinfo {author} {\bibfnamefont {R.}~\bibnamefont
  {Rajesh}},\ }\bibfield  {title} {\bibinfo {title} {Mpemba effect for a
  brownian particle trapped in a single well potential},\ }\href@noop {}
  {\bibfield  {journal} {\bibinfo  {journal} {arXiv preprint arXiv:2305.06613}\
  } (\bibinfo {year} {2023})}\BibitemShut {NoStop}%
\bibitem [{\citenamefont {Deg{\"u}nther}\ and\ \citenamefont
  {Seifert}(2022)}]{degunther2022anomalous}%
  \BibitemOpen
  \bibfield  {author} {\bibinfo {author} {\bibfnamefont {J.}~\bibnamefont
  {Deg{\"u}nther}}\ and\ \bibinfo {author} {\bibfnamefont {U.}~\bibnamefont
  {Seifert}},\ }\bibfield  {title} {\bibinfo {title} {Anomalous relaxation from
  a non-equilibrium steady state: An isothermal analog of the mpemba effect},\
  }\href@noop {} {\bibfield  {journal} {\bibinfo  {journal} {Europhysics
  Letters}\ }\textbf {\bibinfo {volume} {139}},\ \bibinfo {pages} {41002}
  (\bibinfo {year} {2022})}\BibitemShut {NoStop}%
\bibitem [{\citenamefont {Lapolla}\ and\ \citenamefont {Godec}(2020)}]{Uphill}%
  \BibitemOpen
  \bibfield  {author} {\bibinfo {author} {\bibfnamefont {A.}~\bibnamefont
  {Lapolla}}\ and\ \bibinfo {author} {\bibfnamefont {A.~c.~v.}\ \bibnamefont
  {Godec}},\ }\bibfield  {title} {\bibinfo {title} {Faster uphill relaxation in
  thermodynamically equidistant temperature quenches},\ }\href
  {https://doi.org/10.1103/PhysRevLett.125.110602} {\bibfield  {journal}
  {\bibinfo  {journal} {Phys. Rev. Lett.}\ }\textbf {\bibinfo {volume} {125}},\
  \bibinfo {pages} {110602} (\bibinfo {year} {2020})}\BibitemShut {NoStop}%
\bibitem [{\citenamefont {Manikandan}(2021)}]{fewlevelQ}%
  \BibitemOpen
  \bibfield  {author} {\bibinfo {author} {\bibfnamefont {S.~K.}\ \bibnamefont
  {Manikandan}},\ }\bibfield  {title} {\bibinfo {title} {Equidistant quenches
  in few-level quantum systems},\ }\href
  {https://doi.org/10.1103/PhysRevResearch.3.043108} {\bibfield  {journal}
  {\bibinfo  {journal} {Phys. Rev. Res.}\ }\textbf {\bibinfo {volume} {3}},\
  \bibinfo {pages} {043108} (\bibinfo {year} {2021})}\BibitemShut {NoStop}%
\bibitem [{\citenamefont {Van~Vu}\ and\ \citenamefont
  {Hasegawa}(2021)}]{van2021toward}%
  \BibitemOpen
  \bibfield  {author} {\bibinfo {author} {\bibfnamefont {T.}~\bibnamefont
  {Van~Vu}}\ and\ \bibinfo {author} {\bibfnamefont {Y.}~\bibnamefont
  {Hasegawa}},\ }\bibfield  {title} {\bibinfo {title} {Toward relaxation
  asymmetry: Heating is faster than cooling},\ }\href@noop {} {\bibfield
  {journal} {\bibinfo  {journal} {Physical Review Research}\ }\textbf {\bibinfo
  {volume} {3}},\ \bibinfo {pages} {043160} (\bibinfo {year}
  {2021})}\BibitemShut {NoStop}%
\bibitem [{\citenamefont {Dieball}\ \emph {et~al.}(2023)\citenamefont
  {Dieball}, \citenamefont {Wellecke},\ and\ \citenamefont
  {Godec}}]{dieball2023asymmetric}%
  \BibitemOpen
  \bibfield  {author} {\bibinfo {author} {\bibfnamefont {C.}~\bibnamefont
  {Dieball}}, \bibinfo {author} {\bibfnamefont {G.}~\bibnamefont {Wellecke}},\
  and\ \bibinfo {author} {\bibfnamefont {A.}~\bibnamefont {Godec}},\ }\bibfield
   {title} {\bibinfo {title} {Asymmetric thermal relaxation in driven systems:
  Rotations go opposite ways},\ }\href@noop {} {\bibfield  {journal} {\bibinfo
  {journal} {arXiv preprint arXiv:2304.06702}\ } (\bibinfo {year}
  {2023})}\BibitemShut {NoStop}%
\bibitem [{\citenamefont {Meibohm}\ \emph {et~al.}(2021)\citenamefont
  {Meibohm}, \citenamefont {Forastiere}, \citenamefont {Adeleke-Larodo},\ and\
  \citenamefont {Proesmans}}]{meibohm2021relaxation}%
  \BibitemOpen
  \bibfield  {author} {\bibinfo {author} {\bibfnamefont {J.}~\bibnamefont
  {Meibohm}}, \bibinfo {author} {\bibfnamefont {D.}~\bibnamefont {Forastiere}},
  \bibinfo {author} {\bibfnamefont {T.}~\bibnamefont {Adeleke-Larodo}},\ and\
  \bibinfo {author} {\bibfnamefont {K.}~\bibnamefont {Proesmans}},\ }\bibfield
  {title} {\bibinfo {title} {Relaxation-speed crossover in anharmonic
  potentials},\ }\href@noop {} {\bibfield  {journal} {\bibinfo  {journal}
  {Physical Review E}\ }\textbf {\bibinfo {volume} {104}},\ \bibinfo {pages}
  {L032105} (\bibinfo {year} {2021})}\BibitemShut {NoStop}%
\bibitem [{\citenamefont {Yan}\ \emph {et~al.}(2022)\citenamefont {Yan},
  \citenamefont {Touchette}, \citenamefont {Rotskoff} \emph
  {et~al.}}]{yan2022learning}%
  \BibitemOpen
  \bibfield  {author} {\bibinfo {author} {\bibfnamefont {J.}~\bibnamefont
  {Yan}}, \bibinfo {author} {\bibfnamefont {H.}~\bibnamefont {Touchette}},
  \bibinfo {author} {\bibfnamefont {G.~M.}\ \bibnamefont {Rotskoff}}, \emph
  {et~al.},\ }\bibfield  {title} {\bibinfo {title} {Learning nonequilibrium
  control forces to characterize dynamical phase transitions},\ }\href@noop {}
  {\bibfield  {journal} {\bibinfo  {journal} {Physical Review E}\ }\textbf
  {\bibinfo {volume} {105}},\ \bibinfo {pages} {024115} (\bibinfo {year}
  {2022})}\BibitemShut {NoStop}%
\bibitem [{\citenamefont {Rotskoff}\ \emph {et~al.}(2017)\citenamefont
  {Rotskoff}, \citenamefont {Crooks},\ and\ \citenamefont
  {Vanden-Eijnden}}]{rotskoff2017geometric}%
  \BibitemOpen
  \bibfield  {author} {\bibinfo {author} {\bibfnamefont {G.~M.}\ \bibnamefont
  {Rotskoff}}, \bibinfo {author} {\bibfnamefont {G.~E.}\ \bibnamefont
  {Crooks}},\ and\ \bibinfo {author} {\bibfnamefont {E.}~\bibnamefont
  {Vanden-Eijnden}},\ }\bibfield  {title} {\bibinfo {title} {Geometric approach
  to optimal nonequilibrium control: Minimizing dissipation in nanomagnetic
  spin systems},\ }\href@noop {} {\bibfield  {journal} {\bibinfo  {journal}
  {Physical Review E}\ }\textbf {\bibinfo {volume} {95}},\ \bibinfo {pages}
  {012148} (\bibinfo {year} {2017})}\BibitemShut {NoStop}%
\bibitem [{\citenamefont {Abreu}\ and\ \citenamefont
  {Seifert}(2012)}]{abreu2012thermodynamics}%
  \BibitemOpen
  \bibfield  {author} {\bibinfo {author} {\bibfnamefont {D.}~\bibnamefont
  {Abreu}}\ and\ \bibinfo {author} {\bibfnamefont {U.}~\bibnamefont
  {Seifert}},\ }\bibfield  {title} {\bibinfo {title} {Thermodynamics of genuine
  nonequilibrium states under feedback control},\ }\href@noop {} {\bibfield
  {journal} {\bibinfo  {journal} {Physical review letters}\ }\textbf {\bibinfo
  {volume} {108}},\ \bibinfo {pages} {030601} (\bibinfo {year}
  {2012})}\BibitemShut {NoStop}%
\bibitem [{\citenamefont {Gour}\ \emph {et~al.}(2015)\citenamefont {Gour},
  \citenamefont {M{\"u}ller}, \citenamefont {Narasimhachar}, \citenamefont
  {Spekkens},\ and\ \citenamefont {Halpern}}]{gour2015resource}%
  \BibitemOpen
  \bibfield  {author} {\bibinfo {author} {\bibfnamefont {G.}~\bibnamefont
  {Gour}}, \bibinfo {author} {\bibfnamefont {M.~P.}\ \bibnamefont
  {M{\"u}ller}}, \bibinfo {author} {\bibfnamefont {V.}~\bibnamefont
  {Narasimhachar}}, \bibinfo {author} {\bibfnamefont {R.~W.}\ \bibnamefont
  {Spekkens}},\ and\ \bibinfo {author} {\bibfnamefont {N.~Y.}\ \bibnamefont
  {Halpern}},\ }\bibfield  {title} {\bibinfo {title} {The resource theory of
  informational nonequilibrium in thermodynamics},\ }\href@noop {} {\bibfield
  {journal} {\bibinfo  {journal} {Physics Reports}\ }\textbf {\bibinfo {volume}
  {583}},\ \bibinfo {pages} {1} (\bibinfo {year} {2015})}\BibitemShut {NoStop}%
\bibitem [{\citenamefont {Argun}\ \emph {et~al.}(2017)\citenamefont {Argun},
  \citenamefont {Soni}, \citenamefont {Dabelow}, \citenamefont {Bo},
  \citenamefont {Pesce}, \citenamefont {Eichhorn},\ and\ \citenamefont
  {Volpe}}]{aykut}%
  \BibitemOpen
  \bibfield  {author} {\bibinfo {author} {\bibfnamefont {A.}~\bibnamefont
  {Argun}}, \bibinfo {author} {\bibfnamefont {J.}~\bibnamefont {Soni}},
  \bibinfo {author} {\bibfnamefont {L.}~\bibnamefont {Dabelow}}, \bibinfo
  {author} {\bibfnamefont {S.}~\bibnamefont {Bo}}, \bibinfo {author}
  {\bibfnamefont {G.}~\bibnamefont {Pesce}}, \bibinfo {author} {\bibfnamefont
  {R.}~\bibnamefont {Eichhorn}},\ and\ \bibinfo {author} {\bibfnamefont
  {G.}~\bibnamefont {Volpe}},\ }\bibfield  {title} {\bibinfo {title}
  {Experimental realization of a minimal microscopic heat engine},\ }\href
  {https://doi.org/10.1103/PhysRevE.96.052106} {\bibfield  {journal} {\bibinfo
  {journal} {Phys. Rev. E}\ }\textbf {\bibinfo {volume} {96}},\ \bibinfo
  {pages} {052106} (\bibinfo {year} {2017})}\BibitemShut {NoStop}%
\end{thebibliography}%
\end{document}
%
% ****** End of file apssamp.tex ******
