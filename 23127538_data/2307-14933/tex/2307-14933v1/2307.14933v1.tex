\documentclass[a4paper]{amsart} 

 
\usepackage{verbatim}
\usepackage{lmodern}
\usepackage{textcomp}
\usepackage{color}
\usepackage{mathrsfs}
\usepackage{enumerate}
\usepackage{amsthm,amssymb} 
\usepackage{fancyhdr}
\DeclareFontFamily{OT1}{pzc}{}
\DeclareFontShape{OT1}{pzc}{m}{it}{<-> s * [1.10] pzcmi7t}{}
\DeclareMathAlphabet{\mathpzc}{OT1}{pzc}{m}{it}

%%%%%%%%%%%%%%%%%%%%%%%%%%%%%%%%%%%%%%%%%%%%
%
% Hyperref
% 
\usepackage{color}
\definecolor{citation}{rgb}{0.2,0.58,0.2} 
\definecolor{formula}{rgb}{0.1,0.2,0.6}
\definecolor{url}{rgb}{0.3,0,0.5} 
\usepackage[colorlinks=true,linkcolor=formula,urlcolor=url,citecolor=citation]{hyperref} 
%\usepackage{refcheck}

%layout
\textwidth = 13.4 cm
\textheight = 21.4 cm 
\oddsidemargin = 1.33cm
\evensidemargin = 1.33cm 
\topmargin = 0mm
\headheight = 2mm
\headsep = 10mm




\allowdisplaybreaks

% <--- Theorem 
\newtheorem{theorem}{Theorem}[section]
\newtheorem{prop}[theorem]{Proposition}
\newtheorem{corol}[theorem]{Corollary}
\newtheorem{lemma}[theorem]{Lemma}
\newtheorem{defn}[theorem]{Definition}
\newtheorem{rem}[theorem]{Remark}
\newtheorem{example}[theorem]{\bf Example}
\numberwithin{equation}{section}
%
%
%
% <--- Comandi

\def\dys{\displaystyle}
\def\vs{\vspace{1mm}}
\def\dd{\snr{\,\cdot\,}_{\h}}
%
\allowdisplaybreaks
\makeatletter
\DeclareRobustCommand*{\bfseries}{%
		\not@math@alphabet\bfseries\mathbf
		\fontseries\bfdefault\selectfont
		\boldmath
}
%\DeclareMathOperator*{\osc}{osc}
%
\makeatother
\newlength{\defbaselineskip}
\setlength{\defbaselineskip}{\baselineskip}
\newcommand{\setlinespacing}[1]
{\setlength{\baselineskip}{#1 \defbaselineskip}}
\definecolor{verde}{rgb}{0.13, 0.55, 0.13}
\definecolor{verde}{rgb}{0.13, 0.55, 0.13}


%
% <--- Nuove \def
\def \eps{{\varepsilon}}
\def \r{{\mathbb{R}}}
\def \h{{\mathbb{H}^n}}
\def \N{{\mathbb{N}}}
\def\gamp{\Gamma^{+}}
\def\ue{u_{\eps}}
\def\mue{\mu_{\eps}}
\def\mea{\mathcal{M}(\overline{\Omega})}
\newcommand{\twe}{T{w}_{\eps}}
\newcommand{\uer}{u_{\eps,r}}
\newcommand{\tuer}{T{u}_{\eps,r}}
\newcommand{\bu}{\bar{u}_{\eps}}
\newcommand{\mbu}{\bar{\mu}_{\eps}}
\newcommand{\El}{\mathcal{E}_\lambda}
\newcommand{\Es}{\mathcal{E}^*}
\newcommand{\Fep}{\mathcal{F}_{\eps}}
\newcommand{\Fc}{\mathcal{F}}
\newcommand{\Sc}{{S}^1_0}
\newcommand{\Xc}{{X}}
\newcommand{\Car}{\Sigma(\Omega)}
\newcommand{\Ssub}{\Sob_\eps}
\newcommand{\Som}{\Sob_\Om}
\newcommand{\Sob}{S^\ast}
\newcommand{\uz}{u^{(0)}}
\newcommand{\lambdakj}{{\lambda_k^{(j)}}}
\newcommand{\xikj}{{\xi_k^{(j)}}}
\newcommand{\ukj}{{u_k^{(j)}}}
\newcommand{\rkj}{{r_k^{(j)}}}
\newcommand{\uj}{{u^{(j)}}}
\newcommand{\rr}{\rho}
\newcommand{\snr}[1]{\lvert #1\rvert}


%
% <--- Convergenze
\newcommand{\tows}{\stackrel{\ast}{\rightharpoonup}}
\newcommand{\towsloc}{{\stackrel{\ast}{\rightharpoonup}}_{loc}}
\newcommand{\tow}{\rightharpoonup}
\newcommand{\towt}{\stackrel{\mathpzc{t}}{\rightarrow}}
%


% <--- Insiemi
\newcommand{\Om}{\Omega}
\newcommand{\Omb}{\overline{\Omega}}
\newcommand{\Irm}{\text{I}}

\def\Xint#1{\mathchoice
	{\XXint\displaystyle\textstyle{#1}}%
	{\XXint\textstyle\scriptstyle{#1}}%
	{\XXint\scriptstyle\scriptscriptstyle{#1}}%
	{\XXint\scriptscriptstyle\scriptscriptstyle{#1}} %
	\!\int\limits}
\def\XXint#1#2#3{{\setbox0=\hbox{$#1{#2#3}{\int\limits}$}
		\vcenter{\hbox{$#2#3$}}\kern-.5\wd0}}
\def\ddashint{\Xint=}
\def\dashint{\Xint-}
%
\def\ap{``}


% <--- Titolo, autori
\title[Critical Sobolev embedding in the Heisenberg group]
{Asymptotic approach to singular solutions \\ for the CR~Yamabe equation, \\ and  a conjecture by H.~Brezis and L.~\!A. Peletier\\ in the Heisenberg group}
\author[G. Palatucci]{Giampiero Palatucci}  \address{Giampiero Palatucci\\Dipartimento di Scienze Matematiche, Fisiche e Informatiche, Universit\`a di Parma\\ Parco Area delle Scienze 53/a, Campus, 43124 Parma, Italy} \email{\url{giampiero.palatucci@unipr.it}}
\author[M. Piccinini]{Mirco Piccinini}  \address{Mirco Piccinini\\Dipartimento di Scienze Matematiche, Fisiche e Informatiche, Universit\`a di Parma\\ Parco Area delle Scienze 53/a, Campus, 43124 Parma, Italy}
\email{\url{mirco.piccinini@unipr.it}}



%%%%%%%%%%%%%%%%
\usepackage{etoolbox}
\let\svlabel\label
\let\svref\ref
\def\unusedlabels{}
\renewcommand\label[1]{\svlabel{#1}\global\edef\unusedlabels{\unusedlabels$<$#1$>$ }}
\renewcommand\ref[1]{\svref{#1}%
  \edef\teststring{$<$#1$>$}%
  \edef\tmp{\unusedlabels}%
  \def\unusedlabels{}%
  \expandafter\refhelper\tmp\relax%
}
\def\refhelper#1 #2\relax{%
  \edef\expandcase{#1}%
  \ifdefstrequal{\teststring}{\expandcase}{}{\edef\unusedlabels{\unusedlabels#1 }}%
  \if\relax#2\relax\else\refhelper#2\relax\fi
}
%%%%%%%%%%%%%%%%



\begin{document}
	
	\subjclass[2010]{35R03, 46E35, 35B33, 35J08, 35A15\vspace{1mm}}
	
	\keywords{Sobolev embeddings,  Heisenberg group, CR Yamabe, Green's Function\vspace{1mm}}
	

	\thanks{{\it Aknowledgements.}
%	The authors are supported by INdAM Project \!\char`_
%	E55F22000270001.
     The authors are also supported by INdAM project  ``Problemi non locali: teoria cinetica e non uniforme ellitticit\`a'', CUP\_E53C220019320001. The second author is also supported by the Project ``Local vs Nonlocal: mixed type operators and nonuniform ellipticity", CUP\_D91B21005370003. 
     \\ The results in this paper have been announced in the preliminary research report~\cite{PP23}}

	
	\maketitle

	\begin{abstract}	
			\vspace{-3mm} %ALERT SPAZIO PER PAGINA INTERA
		 We investigate some effects of the lack of compactness in the critical Sobolev embedding
       by proving that a famous conjecture of Brezis\,\&\,Peletier ({\it Essays in honor of Ennio De Giorgi -- Progr.~Differ. Equ. Appl.}~1989) does still hold in the Heisenberg framework: optimal functions for a natural subcritical approximations of the Sobolev quotient concentrate energy at one point which can be localized via the Green function associated to the involved domain and in clear accordance with the underlying sub-Riemannian geometry -- and consequently
       a new suitable definition of domains {\it geometrical regular near their characteristic set} is given.
       In order to achieve the aforementioned result, we need to combine  proper  estimates and  tools to attack the related CR~Yamabe equation (Jerison\,\&\,Lee, {\it J.~Diff.~Geom.}~1987) with novel feasible ingredients in PDEs and Calculus of Variations which also aim to constitute general independent results in the Heisenberg framework, as e.~g. 
%       a De Giorgi's $\Gamma$-convergence approach to provide fine energy approximations in very general domains; integral and boundary regularity estimates for   solutions to subcritical equations; 
        a fine asymptotic control of the optimal functions via the Jerison\,\&\,Lee extremals realizing the equality in the critical Sobolev inequality ({\it J.~Amer.~Math.~Soc.}~1988).
\end{abstract}

	
	
	
	%%%%%%%%%%%%%%%%%%%%%%%%%%%%%%%
	%									
	%                                INTRODUCTION
	%
	%%%%%%%%%%%%%%%%%%%%%%%%%%%%%%%
 
\setcounter{equation}{0}\setcounter{theorem}{0}
		
	
	\setcounter{tocdepth}{2} 
{\footnotesize 	\setlinespacing{0.79}
	\vspace{2mm}
	\centering
	
	\tableofcontents 
}


\setlinespacing{1.07}
	 

    \section{Introduction}
  Let  $\h := (\mathbb{C}^n\times\mathbb{R},\circ, \{\delta_\lambda\}_{\lambda>0})$ be the usual Heisenberg-Weyl group and
define the standard Folland-Stein-Sobolev space~$\Sc(\h)$ as the completion of~$C^\infty_0(\h)$ with respect to the horizontal gradient norm~$\|D_H\cdot\|_{L^2(\h)}$.  In~\cite{FS74} the authors prove the that the following Sobolev-type inequality does hold, 
\begin{equation}\label{folland}
\|u\|^{2^\ast}_{L^{2^\ast}(\h)} \leq \Sob \|D_Hu\|^{2^\ast}_{L^2(\h)} \qquad \forall u \in S^1_0(\h)\,,
\end{equation}
where~$\Sob$ is positive constant,~$2^\ast =2^\ast(Q):=2Q/(Q-2)$ stands for the Folland-Stein-Sobolev critical exponent, which depends on the {\it homogeneous dimension}~$Q:=2n+2$ associated to the group of  dilations~$\{\delta_\lambda\}_{\lambda>0}$.
\vspace{2mm}


The critical Sobolev inequality~\eqref{folland} has been an attractive object of study for the last decades, since it is inextricably linked to the lack of compactness of the related critical (Folland-Stein-)Sobolev embedding, and to the correspondent Euler-Lagrange equation in turn describing the important CR Yamabe problem. Several results in accordance with the classical critical inequality in the Euclidean framework have been proven, despite the difficulties given by the sub-Riemannian geometry of the Heisenberg framework. On the contrary, several (somewhat expected) results are still open for the same reasons; that is, the substantial difference with respect to the Euclidean framework in view of the complex metric structure of the Heisenberg framework as well as the presence of characteristic points in the involved domains.
The literature is too wide to attempt any comprehensive treatment in a single paper. We refer the interested readers to the very important papers~\cite{JL88,GL92,LU98,GV00,CU01}, the recent book~\cite{IV11}, and the references therein.

\vspace{2mm}

In the present paper, we are interesting into investigating some of the effects of the lack of compactness in the critical Sobolev embedding~\eqref{folland}, by analyzing the asymptotic behavior of a natural subcritical approximation.


Consider the following maximization problem,
\begin{equation}\label{critica0}
\Sob:=\sup\left\{\,\int\limits_{\h}|u(\xi)|^{2^\ast}\,{\rm d}\xi \, : \, u\in \Sc(\h), \int\limits_{\h}|D_H u(\xi)|^2{\rm d}\xi \leq 1\right\}.
\end{equation}
The validity of~\eqref{folland} is equivalent to show that the constant~$\Sob$ defined in the display above is finite. The existence of the maximizers in~\eqref{critica0} is a difficult problem because of the intrinsic dilations and translations invariance of such inequality, as it analogously happens for the classical critical Sobolev inequality. The situation here is even more delicate because of the underlying non-Euclidean geometry of the Heisenberg group, and the  obstacles due to the related non-commutativity. The explicit form of the maximizers has been presented, amongst other results, in the breakthrough paper by Jerison and Lee~\cite{JL88}, together with the computation of the optimal constant in~\eqref{critica0}.
%;  for similar related investigation for the Moser-Trudinger inequality in the sub-Riemannian setting, see also~\cite{LL12,BFM13}.

\vspace{2mm}

For any bounded domain~$\Omega\subset\h$, consider now the following Sobolev embedding in the same variational form as the one in~\eqref{critica0},
\begin{equation}\label{critica}
\Som:=\sup\left\{\,\int\limits_{\Omega}|u(\xi)|^{2^\ast}\,{\rm d}\xi \, : \, u\in \Sc(\Omega), \int\limits_{\Omega}|D_H u(\xi)|^2{\rm d}\xi \leq 1\right\},
\end{equation}
where the Folland-Stein-Sobolev space $\Sc(\Omega)$ is given as the closure of $C^\infty_0(\Omega)$ with respect to the homogeneous $L^2$-subgradient norm in $\Omega$. One can check that $\Som\equiv \Sob$ via a standard
scaling argument on compactly supported smooth functions. For this, in view of the explicit form of the optimal functions in~\eqref{critica0} -- see forthcoming~Theorem~\ref{thm_optimal} -- the variational problem~\eqref{critica} has no maximizers. The situation changes considerably for the subcritical embeddings. Indeed, since $\Omega$ is bounded,  the embedding $\Sc(\Omega) \hookrightarrow L^{2^\ast-\eps}(\Omega)$ is compact (for any $\eps<2^\ast-2$), and this does guarantee the existence of a maximizer~$u_\eps\in \Sc(\Omega)$ for the related variational problem
   \begin{equation}\label{sobolev}
\Sob_\eps:=\sup\left\{\,\int\limits_{\Omega}|u(\xi)|^{2^\ast-\eps}\,{\rm d}\xi \, : \, u\in \Sc(\Omega), \int\limits_{\Omega}|D_H u(\xi)|^2{\rm d}\xi \leq 1\right\}.
\end{equation}

Such a dichotomy is evident in the Euler-Lagrange equation for the energy functionals in~\eqref{sobolev}; that is,
\begin{equation}\label{equazione}
-\Delta_H u_\eps = \lambda |u_\eps|^{2^\ast-\eps-2} u_\eps \, \ \text{in} \ (\Sc(\Omega))',
\end{equation}
where $\lambda$ is a Lagrange multiplier. Whereas when $\eps>0$ the problem above has a solution~$u_\eps$, it becomes very delicate when $\eps=0$: one falls in the aforementioned CR Yamabe equation, and even the existence of solutions is not granted. In particular, the existence and various properties of the solutions do strongly depend on the geometry and the topology of the domain~$\Omega$. We refer for instance to:~\cite{LU98}
for nonexistence of nonnegative solutions when~$\Om$ is a certain half-space; \cite{CU01}~where the authors show the existence of a solution in the case when the domain~$\Omega$ has at least a nontrivial suitable homology group; 
 \cite{GL92} for existence and nonexistence results for even more general nonlinearity.
\vspace{2mm}

In view of such a qualitative change when $\eps=0$ in both~\eqref{sobolev} and \eqref{equazione}, it seems natural to analyze the asymptotic behaviour as $\eps$ goes to $0$ of  the corresponding optimal functions~$u_\eps$ of the embedding $\Sc(\Omega) \hookrightarrow L^{2^\ast\!-\eps}(\Omega)$. This is the aim of the present paper.
\vspace{2mm}

For what concerns {\it the~Euclidean~counterpart} of such an investigation, several results have been obtained, mostly via fine estimates and a standard regularity elliptic approach of the special class of solutions of the equation~\eqref{equazione} being maximizers for the related Sobolev embedding. As for instance, the energy concentration results of those sequences and the subsequent localization of such concentration on special points have been settled in~\cite{BP89,Rey89,Han91} and many others. 
\\ On the contrary, for what concerns {\it the~Heisenberg~panorama} the scene is basically empty  in view of the many difficulties  naturally arising in such a framework. Indeed, the sub-Riemannian geometry precludes the free generalization of several tools and symmetrization techniques as well as regularity approximations. 
%{\color{blue} Nevertheless, very fundamental results available in the Euclidean framework where successfully extended in the Heisenberg setting. Indeed, starting from the pioneering work of Jerison and Lee~\cite{JL88}, where the authors found the best Sobolev constant~$\Sob$ in~\eqref{folland} and the explicit expression of its related optimal functions, several results and properties well established in the Euclidean framework where successfully extended in the Heisenberg setting: the Bahri\,\&\,Coron conjecture~(\cite{BC88}) in the aforementioned paper~\cite{CU01}, as well as the  non-existence criteria on several relvant class of proper subset of $\h$~(\cite{LU98,GL92}).
%\vspace{2mm}}
Very recently, it has been proven in~\cite{PPT23} that, up to subsequences, optimal functions~$u_\eps$ for the subcritical Sobolev embedding~\eqref{sobolev} do concentrate energy at one point~$\xi_{\rm o} \in \Omb$, in clear accordance with the Euclidean counterpart~(\cite{FM99, AG03}); see Theorem~\ref{cor_concentration} in Section~\ref{sec_cca} for a precise statement.
\vspace{2mm}

Now, a natural question arises: can we  localize the blowing up; i.~\!e., is the concentration point~$\xi_{\rm o}$ related to some extent to the geometry of the domain~$\Omega$\,?
 \\ In the Euclidean framework, under standard regularity assumptions,  
  Han~(\cite{Han91}) and Rey~(\cite{Rey89}) were able to prove the connection with the Green function associated with the domain~$\Om$ by answering to a famous conjecture by Brezis\,\&\,Peletier~(\cite{BP89}), who had previously investigated the spherical domains setting. The involved proofs strongly rely on the regularity of Euclidean domains, which is in clear contrast with the complexity of the underlying sub-Riemannian geometry here. As well known, even if the domain~$\Omega$ is $C^\infty$, the situation is dramatically different because of the possible presence of characteristic points on the boundary~$\partial\Omega$. 
  At such points the vector fields forming the  principal part of the relevant operator~$\Delta_H$ become tangent to the boundary. Hence, near those characteristic points -- as firstly discovered by Jerison~\cite{Jer81,Jer81-2} -- even harmonic functions on the Heisenberg group can encounter a sudden loss of regularity. Also, one did not want to work in the restricted class of domains not having characteristic points; that is, by still including interesting sets as e.~\!g. the torus obtained by revolting the sphere~$\mathbb{S}^{2n}$ around the $t$-axis, but unfortunately excluding an extremely  wide class of regular sets  which play a pervasive role in several fundamental problems in the Heisenberg group, as e.~\!g. the level sets of the Jerison\,\&\,Lee extremal functions~\eqref{talentiane_2} and those of the Folland fundamental solutions; i.~\!e., the Kor\'anyi balls. 
  For this reason, and in order to deal with the aforementioned difficulties, it is then quite natural to work under the assumption that the domain~$\Om$ is {\it geometrical regular near its characteristic set} 
   in the sense of  the definitions firstly provided in the seminal paper~\cite{GV00} by Garofalo and Vassilev; see conditions~{\rm(}$\Om1${\rm)}\textup{--}{\rm(}$\Om4${\rm)} in forthcoming {\rm Section~\ref{sec_garofalo}}.
   
   We are eventually able to deal with the aforementioned sub-Riemannian framework obstacles by proving the desired localization result for the concentration point~$\xi_{\rm o}$ of the maximizing sequence~$u_\eps$ in terms of the Green function associated with the domain~$\Om$, in turn establishing the validity of the aforementioned Brezis\,\&\,Peletier conjecture in the Heisenberg group. We have the following
    \begin{theorem}\label{thm_green}
    Consider a bounded domain~$\Om\subset\h$ geometrical regular near its characteristic set,  and  let~$u_\eps \in \Sc(\Om)$  be a maximizer for~$\Sob_\eps$. Then, up to subsequences, $u_\eps$~concentrates at some point~$\xi_{\rm o} \in  \Om$ such that
       \begin{equation}\label{robin_condition}
      	\int\limits_{\partial \Om} \snr{D_H G_{\Om}(\cdot,\xi_{\rm o})}^2 \langle\mathcal{D},\mathfrak{n}\rangle \, {\rm d}{\it H}^{Q-2}=0,
      \end{equation}
      with~$G_{\Om}(\cdot;\xi_{\rm o})$ being the Green function associated to~$\Om$ with pole in~$\xi_{\rm o}$, and $\mathcal{D}$ being the infinitesimal generator of the one-parameter group of non-isotropic dilations in the Heisenberg group. 
    \end{theorem}
    
    The proof of our main result stated above will be postponed in the final section of the present manuscript, because it involves several new results -- see in particular the theorem stated below and those in Section~\ref{sec_cca} -- together with various general tools in the sub-Riemaniann framework, as e.\!~g., maximum principles, Caccioppoli-type estimates, $H$-Kelvin transform, boundary Schauder-type regularity estimates, as well as with a fine boundary analysis for the subcritical CR Yamabe equation.
    \vspace{2mm}
    
     Among other results, in order to prove Theorem~\ref{thm_green} above, an asymptotic control of the maximizing sequence~$u_\eps$ for~$\Ssub$ in~\eqref{sobolev} via the Jerison\,\&\,Lee extremals is needed. This is shown in Theorem~\ref{han} below, and it reveals to be an independent result which could be also useful to investigate further properties related to the subcritical Folland-Stein embedding. 
    \begin{theorem}\label{han}
   	Let~$\Omega \subset \h$ be a  
	smooth bounded  domain  such that
    \begin{equation}\label{non_deg_Om}
     \liminf_{\rr\to0^+}\frac{\snr{(\h \smallsetminus \Om) \cap B_\rr(\xi)}}{\snr{B_\rr(\xi)}}  >0 \qquad \forall \xi \in \partial \Om.
    \end{equation}
    Then, for    each~$0<\eps< 2^*-2$ letting~$u_\eps \in \Sc(\Om)$  being a maximizer for~$\Sob_\eps$, there exist~$\{\eta_\eps\} \subset \Om$,~$\{\lambda_\eps\} \subset \r^+$ such that, up to choosing~$\eps$ sufficiently small, we have that
   	\begin{equation}\label{bound_max_seq}
   		u_\eps \lesssim  \,U_{\lambda_\eps,\eta_\eps} \qquad ~\textrm{on}~\Om,
   	\end{equation}
   	where~$U_{\lambda_\eps,\eta_\eps}$ are the {\rm Jerison{\,\rm\&\,}Lee} extremal functions given in~\eqref{talentiane} and the sequences~$\{\eta_\eps\}$ and~$\{\lambda_\eps\}$ satisfy
   	\begin{equation}\label{conv_xi_eps}
   	\eta_\eps \sim \,  \xi_{\rm o} \quad \textrm{and} \quad
   		\lambda_\eps^\eps \sim 1 \quad \textrm{as} \ \eps \searrow 0,
   	\end{equation}  
    with~$\xi_{\rm o}$ being the concentration point.
    % given in {\rm Theorem~\ref{cor_concentration}}.
   \end{theorem}
The result above reminds somehow to the literature following the pioneering work in the Euclidean setting due to Aubin and Talenti, and this is one of the key-points in the subtle proof of the related conjecture by~Han in~\cite{Han91}. Here, we have also to deal with the fact that, in strong contrast with the Euclidean setting, the Jerison\,\&\,Lee extremals cannot be reduced to functions depending only on the standard Kor\'anyi gauge.
For this, we need to pursuit a  delicate strategy which makes use and refines the concentration result obtained in~\cite{PPT23} via the $\Gamma$-convergence approach in order to detect the right scalings~$\lambda_\eps$ and~$\eta_\eps$ above. Also, we would need a Global Compactness-type result in~$\h$, for which we immediately refer to Section~\ref{sec_cca}.


\subsection{Related open problems and further developments} Starting from the results proven in the present paper, several questions naturally arise.

\vspace{1mm}
	
		$\bullet$~~One could be interested in extending the aforementioned results in the very general $H$-type groups setting. In such a framework, a lot of expected result, especially for what concerns important properties of the related extremal functions, are still unknown. A starting point could be the investigation of their subclass of groups of Iwasawa type. For this, one can take advantage of the involved group structure, as well as of important results present in the literature; that is, the investigation 
 in~\cite{GV01}, where positive solutions to the CR~Yamabe equations being invariant with respect to the action of the orthogonal group in the first layer of the Lie algebra have been precisely characterized.


\vspace{1mm}
	
	$\bullet$~~For what concerns the natural hypotheses on the domain~$\Omega$ in order to achieve the localization Theorem~\ref{thm_green}, it could be interesting to ask if one can obtain such a delicate result under somewhat different assumptions, still in the spirit of treating a very wide class of domains also  possibly involving the presence of characteristic points. In this respect, it would be interesting to pursuit such an investigation by taking into account the different assumptions of nontangentially accessible domains satisfying an intrinsic outer ball condition, as firstly introduced in the relevant paper~\cite{CGN02} to deal with the solvability of the related Dirichlet problem with summable boundary data.
	
	
	\vspace{1.3mm}
	
	$\bullet$~~Still for what concerns the possible localization of the concentration point in non-smooth domains, it is worth mentioning the paper~\cite{FGM02} in the Euclidean framework, where Flucher, Garroni and M\"uller were able to construct an example of a peculiar {\it non-smooth} domain~$\tilde{\Omega}$ (see Example~9 there), whose related Robin function~$\mathcal{R}_{\tilde{\Omega}}$, the diagonal of the regular part of the associated Green function, achieves its infimum on the boundary; and subsequently Pistoia and Rey in~\cite{PR03} showed that concentration can occur on the boundary in such a domain~$\tilde{\Omega}$. It could be interesting to understand whether or not one can construct similar examples in the sub-Riemannian setting.
	
\vspace{1mm}
	
	$\bullet$~~The link with the Robin function mentioned above could be expected even in the case of a wider class of general bounded smooth domains as for instance those with no characteristic points. For this, it could be interesting demonstrate whether or not the localization given in the form of Formula~\eqref{robin_condition} in Theorem~\ref{thm_green} is related to the horizontal gradient of the diagonal of the regular part of the Green function, as seen in the Euclidean framework in~\cite[Theorem~4.4]{BP89}; see also~\cite{MMP13}. This is not linked to the problem we are dealing with, but it is a general fact on Green functions to be proven in the Heisenberg group.
 
	
	\vspace{1mm}
	
		$\bullet$~~The localization result in Theorem~\ref{thm_green} can be generalized in the case when one considers more general nonconvex and discontinuous energies with critical growth. This seems a very challenging task; we refer to the delicate approach in the Euclidean framework in~\cite{FGM02}.
		
 
	
	\vspace{1mm}
	
	$\bullet$ Finally, in the same flavour of subcritical approximations, still in clear accordance with the Euclidean framework studied in~\cite{BP89,Rey89,Han91}, one can consider to investigate the asymptotic behaviour of the  sequences approaching the critical Sobolev inequality which solve the auxiliary family of equations $-\Delta_H u_\eps = \lambda u_\eps^{2^\ast-1} +\eps u_\eps$, \,  
	so that the lack of compactness does similarly come into play when~$\eps$ goes to $0$.
 
	\vspace{2mm}	
	 We hope that our estimates and techniques will be important in further developments for a better comprehension of the effects of the lack of compactness in the critical Sobolev embedding in the Heisenberg group.

    \vspace{2mm}
    	\subsection{The paper is organized as follows} In Section~\ref{sec_preliminaries} below we briefly fix the notation and recall some important results on the effects of the lack of compactness of the critical Sobolev embedding in~$\h$ which will be necessary in the rest of the paper. We will also introduce the relevant class of ``geometrical regular'' sets near their characteristic points appearing in the statement of our main result in Theorem~\ref{thm_green}. The fine asymptotic control of the optimal functions via the Jerison \& Lee extremals is achieved in Section~\ref{sec_asymptotic}. In Section~\ref{sec_localization} we prove the localization result in~Theorem~\ref{thm_green} after pursuing a fine boundary analysis for solutions to the subcritical CR Yamabe equation. A remark about the validity of such localization result in the simpler case with no characteristic points will close the paper.




  \vspace{2mm}
   \section{Preliminaries}\label{sec_preliminaries}
   In this section, we briefly fix the notation by recalling a very few properties of the Heisenberg group; we also present some well-known results regarding the lack of compactness in the critical Sobolev embedding in the Folland-Stein spaces in the Heisenberg group.



   \vspace{2mm}
   \subsection{The Heisenberg-Weyl group}
    We start by summarily recalling a few well-known facts about the Heisenberg group. 
\vspace{1mm}

We denote points~$\xi$ in~$\mathbb{C}^n\times\r \simeq \r^{2n+1}$ by
   \[
   \xi := (z,t) = (x+iy, t) \simeq (x_1,\dots,x_n, y_1,\dots,y_n,t) 
   \in \r^n\times \r^n \times \r.
   \]
   For any~$\xi,\xi'\in \r^{2n+1}$,  {\it the group multiplication law}~$\circ$ is defined by
   \[
   	\xi \circ \xi' := \Big(x+x',\, y+y',\, t+t'+2\langle y,x'\rangle-2\langle x,y'\rangle \Big).
  \]
 Given~$\xi'\in\h$, {\it the left translation}~$\tau_{\xi'}$ is defined by
\begin{equation}\label{def_tau}
 \tau_{\xi'}(\xi):=\xi'\circ\xi \qquad \forall \xi \in \h.
 \end{equation}
The group of non-isotropic {\it dilations}~$\{\delta_\lambda\}_{\lambda>0}$ on~$\r^{2n+1}$ is defined by
 \begin{equation}\label{def_philambda}
   	                 \xi   \mapsto \delta_{\lambda}(\xi):=(\lambda x,\, \lambda y,\, \lambda^2 t),
	               \end{equation}
	                  and, 	
	                 as customary,
   $Q\equiv 2n+2$ is the {\it homogeneous dimension\,} with respect to~$\{\delta_\lambda\}_{\lambda>0}$,
  so  that the Heisenberg-Weyl group~$\h := (\r^{2n+1},\circ, \{\delta_\lambda\}_{\lambda>0})$ is a homogeneous Lie group.
\vs 

   The Jacobian base of the Heisenberg Lie algebra~$\mathscr{H}^n$ is given by
   \[
   Z_j := \partial_{x_j} +2y_j\partial_t, \quad Z_{n+j}:= \partial_{y_j}-2x_j\partial_t, \quad 1 \leq j\leq n, \quad T:=\partial_t.
   \]
   Since
   $[Z_j,Z_{n+j}]=-4\partial_t$ for every  $1 \leq j \leq n$, it plainly follows that
   \begin{eqnarray*}
   	&& \textup{rank}\Big(\textup{Lie}\{Z_1,\dots,Z_{2n},T\}(0,0)\Big)
   	\ = \ 2n+1,
   \end{eqnarray*}
   so that~$\h$ is a Carnot group with the following stratification of the algebra
   \[
   \mathscr{H}^n  = \textup{span}\{Z_1,\dots,Z_{2n}\} \,\oplus\, \textup{span}\{T\}.
   \]
   The horizontal (or intrinsic) gradient~$D_H$ of the group is given by
   \[
   D_H u(\xi) := \left( Z_1 u(\xi),\dots, Z_{2n}u(\xi)\right).
   \]
   The Kohn Laplacian (or sub-Laplacian)~$\Delta_H$ on~$\h$ is the second order operator invariant with respect to the left-translations~$\tau_{\xi'}$ defined in~\eqref{def_tau} and homogeneous of degree~$2$ with respect to the dilations~$\{\delta_\lambda\}_{\lambda>0}$ defined in~\eqref{def_philambda},
   \[
   \Delta_H :=  \sum_{j=1}^{2n} Z^2_j.
   \]

 

    \vs
	A {\it homogeneous norm} on~$\h$ is a continuous function (with respect to the Euclidean topology)~$\dd : \h \rightarrow [0,+\infty)$ such that:
	\begin{enumerate}[\qquad\qquad\ \,   \rm(i)]
		\item{
			$ \snr{\delta_\lambda(\xi)}_{\h}=\lambda \snr{\xi}_{\h}$, for every $\lambda>0$ and every $\xi \in \h$;
		}\vspace{1mm}
		\item{
			$\snr{\xi}_{\h}=0$ if and only if $\xi=0$.}
	\end{enumerate}
We say that the homogeneous norm~$\dd$ is {\it symmetric} if
	$
    \snr{\xi^{-1}}_{\h}= \snr{\xi}_{\h}$ for all $\xi \in \h$.
     If~$\dd$ is a homogeneous norm on~$\h$, then the function
	$ 
	(\xi,\eta)\mapsto \snr{\eta^{-1}\circ \xi}_{\h}$
	is a pseudometric on~$\h$. In particular, we will work with the standard homogeneous norm on~$\h$, also known as {\it Kor\'anyi gauge},
    \begin{equation*} 
	|\xi|_{\h} := \left(|z|^4 +t^2\right)^\frac{1}{4} \qquad \forall \xi=(z,t) \in \h.
    \end{equation*}
    As customary, we will denote by~$B_\rr\equiv B_\rr(\eta_{\rm o})$ the ball with center~$\eta_{\rm o} \in \h$ and radius~$\rr>0$ given by
    \[
    B_\rr(\eta_{\rm o}):=\Big\{\xi \in \h : |\eta_{\rm o}^{-1}\circ \xi|_{\h} < \rr\Big\}.
    \]
    
 
\vspace{2mm}
\subsection{Geometrical regularity near the characteristic set}\label{sec_garofalo}
Some further notation is needed in order to introduce the natural assumptions on the domains in accordance with the by-now classical paper~\cite{GV00}.  We denote by~$\mathcal{D}$ the infinitesimal generator of the one-parameter group of non-isotropic dilations~$\{\delta_\lambda\}_{\lambda>0}$ in~\eqref{def_philambda}
\begin{equation}\label{dilation_vector_field}
	\mathcal{D} := \sum_{j=1}^n \big(x_j\partial_{x_j}+y_j \partial_{y_j}\big) +2t\partial_t.
\end{equation}
\begin{defn}[\bf $\delta_\lambda$-starlike sets]
	Let~$\Om$ be a~$C^{1}$~connected open set of\,~$\h$ containing the group identity~$\mathfrak{e}$. We say that $\Om$ is \textup{$\delta_\lambda$-starlike} {\rm (}with respect to the identity~$\mathfrak{e}${\rm)} along a subset~$K \subseteq \partial \Om$ if
	\[
	\langle \mathcal{D},\mathfrak{n}\rangle (\eta) \leq 0, 
	\]
	at every~$\eta \in K$;  in the display above~$\mathfrak{n}$ indicates the exterior unit normal to~$\partial \Om$. 
	
	We say that~$\Om$ is \textup{uniformly $\delta_\lambda$-starlike} {\rm (}with respect to the identity~$\mathfrak{e}${\rm)} along~$K$ if there exists~$\kappa_\Om >0$ such that, at every~$\eta \in K$,
	\[
	\langle \mathcal{D},\mathfrak{n}\rangle (\eta) \geq \kappa_\Om.
	\]
	A domain as above~$\Om$ is {$\delta_\lambda$-starlike} (uniformly {$\delta_\lambda$-starlike}, respectively) with respect to one of its point~$\zeta \in  \Om$ along~$K$ if~$\tau_{\zeta^{-1}}(\Om)$ is {$\delta_\lambda$-starlike} (uniformly {$\delta_\lambda$-starlike}, respectively) with respect to the group identity~$\mathfrak{e}$ along~$\tau_{\zeta^{-1}}(K)$.
\end{defn}
\vs

Given a domain~$\Om\subset\h$, we recall that its {\it characteristic set}\,~$\Car$ is given by
\[
\Car :=\Big\{\xi \in \partial \Om \, | \, Z_j (\xi) \in T_\xi (\partial\Om), \, \textrm{for}~j=1,\dots,2n\Big\}.
\]


We finally are in the position to introduce a natural class of regular sets that we take the liberty to christening for shorten; we refer to~\cite{GV00} for further details.
\begin{defn}[\bf Geometrical regular domains near their characteristic set]
	A smooth domain~$\Om\subset\h$ such that~$\partial \Omega$ is an orientable hypersurface is {\rm  
		``geometrical regular near its characteristic set''} if
	the following conditions hold true,
	\begin{itemize}
		\item[($\Om1$)]
		There exist~$\varPhi \in C^\infty(\h)$,~$c_\Om >0 $ and~$\rr_\Om \in \r$ such that
		\[
		\Om := \big\{\varPhi < \rr_\Om\big\}, \quad \textrm{and} \quad \snr{D \varPhi} \geq c_\Om.
		\]
		\vs
		\item[($\Om2$)]
		For any~$\xi \in \partial \Om$ it holds
		\[
		\liminf_{\rr\to0^+}\frac{\snr{(\h \smallsetminus \Om) \cap B_\rr(\xi)}}{ \snr{B_\rr(\xi)}}>0.
		\]
		\vs
		\item[($\Om3$)] 
		There exist~$M_\Om$ such that
		\[
		\Delta_H \varPhi \geq \frac{4\snr{z}}{M_\Om } \langle D_H \varPhi, D_H \snr{z} \rangle \quad \textrm{in}~\omega,
		\]
		where~$\omega$ is an interior neighborhood of~$\Car$.
		\vs
		\item[($\Om4$)]
		$\Om$~ is {$\delta_\lambda$-starlike} with respect to one of its point~$\zeta_{\rm o} \in \Om$ and uniformly {$\delta_\lambda$-starlike} with respect to~$\zeta_{\rm o}$ along~$\Car$.
	\end{itemize}
\end{defn}






\vspace{2mm}
\subsection{Lack of compactness in the critical Sobolev embedding}\label{sec_cca}
In this section, we recall some important results in the Heisenberg framework regarding the analysis of the effect of the lack of compactness in the critical Sobolev embedding.
\vspace{1mm}

Firstly, we state (in the form adapted to our framework) the aforementioned pioneering result by Jerison and Lee~\cite{JL88} which gives the explicit expression of the functions giving the equality in~\eqref{folland}.  
\begin{theorem}[Corollary~C in~\cite{JL88}]\label{thm_optimal}
Let $2^\ast=2Q/(Q-2)$. Then for any $\lambda>0$ and any $\xi_{\rm o}\in \h$, the function~$U_{ {\lambda}, \xi_{\rm o}}$ defined by
\begin{equation}\label{talentiane}
U_{{\lambda}, \xi_{\rm o}} := U \left(\delta_{\frac{1}{\lambda}}\big(\tau_{\xi^{-1}_0}(\xi)\big)\right)\,,
\end{equation}
where
\begin{equation}\label{talentiane_2}
U(\xi)=c_0\left(\big(1+|z|^2\big)^2+t^2\right)^{-\frac{Q-2}{4}}\quad \forall \xi\in\h,
\end{equation}
is solution to the variational problem~\eqref{critica0}; that is,
\[
\|U_{{\lambda}, \xi_{\rm o}}\|^{2^\ast}_{L^{2^\ast}(\h)} = \Sob \|D_H U_{{\lambda}, \xi_{\rm o}}\|_{L^2(\h)}^{2^\ast}.
\]
where $S^*$ is the best Sobolev constant.\end{theorem}
\vspace{2mm}

As in the classical Euclidean setting, several natural properties of the subcritical extremal functions can be proven. We start by recalling recalling the aforementioned concentration result in~\cite{PPT23}.


%As main consequence of the theorem above, we can deduce that the sequences of maximizers~$\{u_\eps\}$ for the subcritical variational problem~\eqref{sobolev} do concentrate energy at one point. Moreover, as consequence of its own definition, we can prove an expected convergence result for the Sobolev quotient~$\Sob_\eps$.

 \begin{theorem}[Theorem~1.2 in~\cite{PPT23}]\label{cor_concentration}
		Let $\Om \subset \h$ be a bounded domain and  
	let~$\ue\in \Sc(\Omega)$ be a maximizer for~$S^{\ast}_{\eps}$.  Then, as~$\varepsilon=\varepsilon_k \to 0$, up to subsequences, we have that
  	there exists $\xi_{\rm o} \in \Omb$ such that
  	\[
  	u_k=u_{\eps_k}  \rightharpoonup 0  \ \text{in} \ L^{2^\ast}\!(\Omega),
  	\]
  	and 
  	\[
  	\dys  |D_H u_k|^{2}{\rm d}\xi \tows \delta_{\xi_{\rm o}} \ \text{in} \ \mea,
  	\]
  	with~$\delta_{\xi_{\rm o}}$ being the Dirac mass at~$\xi_{\rm o}$
\end{theorem}

We conclude this section by recalling the Global Compactness-result for Palais-Smale sequences in~$\h$.  For any fixed~$\lambda\in\r$ consider the problem
\begin{equation}\label{plambda}
	-\Delta_Hu-\lambda u-|u|^{2^*-2}u=0\qquad\text{in } (\Sc(\Omega))', \tag{$P_\lambda$}
\end{equation}
together with its corresponding Euler--Lagrange functional~$\El:\Sc(\Omega)\to\r$ given by
\begin{equation*} 
	\El(u) =\frac12 \int\limits_{\Om}|D_H u|^2 	\,{\rm d}\xi -\frac{\lambda}{2}\int\limits_{\Om}|u|^2	\,{\rm d}\xi-\frac{1}{2^*}\int\limits_{\Om}|u|^{2^*}	\,{\rm d}\xi.
\end{equation*}
Consider  also the following limiting problem,
\begin{equation}\label{pzero}
	-\Delta_Hu-|u|^{2^*-2}u=0\qquad\text{in } (\Sc(\Om_{\rm o}))',\tag{$P_0$}
\end{equation}
where $\Om_{\rm o}$ is either a half-space or the whole~$\h$;
i.~\!e., the Euler-Lagrange equation corresponding to the energy functional~$\Es: \Sc(\Om_{\rm o})\to\r$,
\begin{equation*} 
	\Es(u)=\frac12 \int\limits_{\Om_{\rm o}}|D_H u|^2 	\,{\rm d}\xi -\frac{1}{2^*}\int\limits_{\Om_{\rm o}}|u|^{2^*}	\,{\rm d}\xi.
\end{equation*}

We have the following
\begin{theorem}[Theorem~1.3 in~\cite{PPT23}]\label{thm_glob_comp}
Let~$\{u_k\}\subset \Sc(\Omega)$ be a Palais-Smale sequence for~$\El$; i.~\!e., such that
	\begin{eqnarray}
		&&\El(u_k)\leq c\quad \text{for all }k,\label{PS1}\\*
		&&d\El(u_k) \rightarrow 0\quad \text{as } k\to\infty \quad \text{in }(\Sc(\Omega))'\label{PS2}.
	\end{eqnarray}
	Then, there exists a (possibly trivial) solution~$\uz\in \Sc(\Omega)$ to~\eqref{plambda} such that, up to a subsequence, we have
	\[
	u_k\rightharpoonup\uz\quad \text{as } k\to\infty \quad\text{in }\Sc(\Omega).
	\]
	Moreover, either the convergence is strong or there is a finite set of indexes~$\Irm=\{1,\dots,J\}$ such that for all~$j\in\Irm$ there exist a
	nontrivial solution~$\uj\in \Sc(\Om_{\rm o}^{(j)})$ to~\eqref{pzero} with $\Om_{\rm o}^{(j)}$ being either a half-space or the whole~$\h$,
	a sequence of nonnegative numbers~$\{\lambdakj\}$ converging to zero and a sequences of points~$\{\xikj\}\subset\Om$ such that, for a renumbered subsequence, we have for any~$j\in\Irm$
	\[
	\ukj(\cdot):=\lambdakj^{\frac{Q-2}{2}}u_k\big(\tau_{\xikj}\big(\delta_{\lambdakj}(\cdot)\big)\big) \rightharpoonup \uj(\cdot)\quad\text{in }\Sc(\h) \quad\text{ as }k \to\infty.
	\]
	In addition, as~$k\to\infty$ we have
	\begin{eqnarray*}
		&&u_k(\cdot)=\uz(\cdot)+\sum_{j=1}^{J}\lambdakj^{\frac{2-Q}{2}}u_k\big(\delta_{1/\lambdakj}\big(\tau_{\xikj}^{-1}(\cdot)\big)\big)+o(1) \quad\text{ in }\Sc(\h);\\*
		&&\left|\log{\frac{\lambda_k^{(i)}}{\lambdakj}}\right|+\left|\delta_{1/\lambdakj}\big(\xikj^{-1}\circ \xi_k^{(i)}\big)\right|_{\h}\to\infty\quad\text{for }i\neq j,\ \,i,j\in\Irm;\\*
		&&\|u_k\|_{\Sc}^2=\sum_{j=1}^{J}\|\uj\|_{\Sc}^2+o(1);\\*
		&&\El(u_k)=\El(\uz)+\sum_{j=1}^{J}\Es(\uj)+o(1).
	\end{eqnarray*}
\end{theorem}
As firstly shown in a very general setting in~\cite{PP15}, the proof of the result above can be deduced via the
so-called {\it Profile Decomposition}, originally proven by  G\'erard  for bounded sequences in the fractional Euclidean space~${H}^s$, and extended to the Heisenberg framework in~\cite{Ben08}.




\vspace{2mm}
\section{Asymptotic control via the Jerison\,\& \,Lee extremals}\label{sec_asymptotic}


Before going into the proof of Theorem~\ref{han}, we need to 
recall some integrability and boundedness estimates for weak solutions to subelliptic equations in the Heisenberg group as well as the notion of $H$-Kelvin transform and a maximum principle for the sub-Laplacian~$\Delta_H$.

\subsection{Regularity properties for subelliptic equations}
Let us consider the following problem
\begin{equation}\label{pbm_vass}
	\begin{cases}
		-\Delta_H u = f(\xi,u)u,\\*[0.5ex]
		u \in \Sc(\Om), \ u \geq 0.
	\end{cases}
\end{equation}
We recall that~$u \in \Sc(\Om)$ is a weak solution to~\eqref{pbm_vass} if and only if
\begin{equation}\label{weak_form}
	\int\limits_\Om D_H u \cdot D_H \phi \, {\rm d}\xi = \int\limits_\Om f(\xi,u)u\phi \, {\rm d}\xi,
\end{equation}
for any~$\phi \in \Sc(\Om)$. %With this bit of notation we have that the following result holds true.
In Lemma~\ref{appendix_prop1} below, we state and prove some estimates in the same spirit of classical Caccioppoli-type inequalities and consequently boundedness results. We refer also to the Euclidean counterpart in~\cite{BP89,Han91} and to related results on Carnot groups in~\cite{Vas06}. 
\begin{lemma}\label{appendix_prop1}   
	Let~$u\in\Sc(\Om)$ be a weak solution to~\eqref{pbm_vass}. Then, the following statements hold true.
	\begin{enumerate}[\rm(i)]
		\item If~$f(\xi,\cdot)\in L^{Q/2}(\Om)$, for a.~\!e.~$\xi \in \Om$, then for any~$q \in [2^*,+\infty)$, it holds
		\begin{equation}\label{gain_int}%ALERT label ancora da usare
			\|u\|_{L^\frac{q2^*}{2}(\Om)} \leq c \|u\|_{L^q(\Om)},
		\end{equation}
		where~$c \equiv c(n,q,\|f\|_{L^{Q/2}(\Om)})>0$.
		\item If~$f(\xi,\cdot)\in L^{q/2}(\Om)$, for a.~\!e.~$\xi \in \Om$ and for some~$q>Q$, then it holds
		\begin{equation}\label{sup_est}%ALERT label ancora da usare
			\sup_{B_\rr}u \leq c \left(\frac{1}{\rr^Q} \int\limits_{\Om \cap B_{2\rr}} u^{2^*}\right)^\frac{1}{2^*} \qquad \forall B_{2\rr}\subset \Om,
		\end{equation}
		where~$c\equiv c(n,q,\|f\|_{L^{q/2}(\Om)})>0$.
	\end{enumerate}
\end{lemma}
\begin{proof} %{\color{black}
		%	We separately prove~\eqref{gain_int} and~\eqref{sup_est}.
		%
		%	\vspace{2mm}
		%\mbox{} 
		Let us prove~\eqref{gain_int}. Consider~$\varphi \in C^\infty_0(\h)$  and, in the weak formulation~\eqref{weak_form}, choose~$\phi:=\varphi^2u^q$ as test functions, with~$q \geq 2^*$; it yields 
		\begin{equation}\label{app_eq_3}
			\int\limits_\Om D_H(\varphi^2u^q)\cdot D_H u \, {\rm d}\xi \,=\, \int\limits_\Om f(\xi,u)u^{q+1}\varphi^2 \, {\rm d}\xi.
		\end{equation}
		The integral on the left-hand side in~\eqref{app_eq_3} can be treated as follows via integration by parts and Young's Inequality
		\begin{eqnarray}\label{app_eq_4}
			&&\int\limits_\Om D_H(\varphi^2u^q)\cdot D_H u \, {\rm d}\xi \nonumber\\*[0.5ex]
			&&\qquad \qquad\quad = \int\limits_\Om 2\varphi u^q D_H\varphi\cdot D_H u  + q\varphi^2u^{q-1}\snr{D_H u}^2 \, {\rm d}\xi\nonumber\\*
			&&\qquad \qquad\quad \geq -\,2\int\limits_\Om \varphi u^{\frac{q+1}{2}+\frac{q-1}{2}} \snr{D_H\varphi} \snr{D_H u}\, {\rm d}\xi  + q\int\limits_{\Om}\varphi^2u^{q-1}\snr{D_H u}^2 \, {\rm d}\xi\nonumber\\*[0.5ex]
			%		&&\qquad \qquad\quad \geq -\,\frac{1}{\epsilon}\int\limits_\Om \varphi^2 u^{q-1} \snr{D_H u}^2\, {\rm d}\xi -\epsilon\int\limits_\Om  u^{q+1} \snr{D_H u}^2\, {\rm d}\xi\nonumber\\*
			%		&&\qquad \qquad\qquad + \,q\int\limits_{\Om}\varphi^2u^{q-1}\snr{D_H u}^2 \, {\rm d}\xi\nonumber\\*[0.5ex]
			&&\qquad \qquad\quad\geq \frac{q}{2}\int\limits_\Om\varphi^2u^{q-1}\snr{D_H u}^2 \, {\rm d}\xi  -\frac{2}{q}\int\limits_\Om u^{q+1}\snr{D_H \varphi}^2 \, {\rm d}\xi.
		\end{eqnarray}
		Using Sobolev's and H\"older's Inequality,~\eqref{app_eq_3} and~\eqref{app_eq_4} we obtain
		\begin{eqnarray}\label{app_eq_5}
			&& (S^*)^{-\frac{2}{2^*}}\left(\int\limits_\Om \snr{\varphi u^{(q+1)/2}}^{2^*} \, {\rm d}\xi\right)^\frac{2}{2^*}\nonumber\\*[0.5ex]
			&&\qquad \qquad\qquad \qquad\quad \leq  \int\limits_\Om \snr{D_H(\varphi u^{(q+1)/2})}^2 \, {\rm d}\xi\notag\\*[0.5ex]
			&&\qquad \qquad\qquad \qquad\quad \leq \frac{(q+1)^2}{q}\frac{q}{2}\int\limits_\Om \varphi^2 u^{q-1}\snr{D_H u}^2 \, {\rm d}\xi +2\int\limits_\Om u^{q+1}\snr{D_H\varphi}^2 \, {\rm d}\xi\nonumber\\*[0.5ex]
			&&\qquad \qquad\qquad \qquad\quad \leq \frac{(q+1)^2}{q}\left[\int\limits_\Om \snr{f(\xi,u)}u^{q+1}\varphi^2 \, {\rm d}\xi + \frac{2}{q}\int\limits_\Om u^{q+1}\snr{D_H \varphi}^2 \, {\rm d}\xi\right]\\*
			&&\qquad \qquad\qquad \qquad\qquad +
			\,2\int\limits_\Om u^{q+1}\snr{D_H\varphi}^2 \, {\rm d}\xi\,.\notag
		\end{eqnarray}
		Note now that, by H\"older's Inequality, for any~$M>0$ we have
		\begin{eqnarray}\label{app_eq_6}
			\int\limits_\Om \snr{f(\xi,u)}u^{q+1}\varphi^2 \, {\rm d}\xi  &\leq & \int\limits_{\Om \, \cap \,\{\snr{f(\xi,u)} \geq M\}} \snr{f(\xi,u)}u^{q+1}\varphi^2 \, {\rm d}\xi\notag\\*
			&&  +\, M \int\limits_{\Om \, \cap \,\{\snr{f(\xi,u)} \leq  M\}} u^{q+1}\varphi^2 \, {\rm d}\xi \notag\\*[0.5ex]
			&\leq &  \|f\|_{L^{Q/2}(\Om \, \cap \, \{\snr{f(\xi,u)} \geq M\})}\left(\int\limits_\Om \snr{\varphi u^{(q+1)/2}}^{2^*}\, {\rm d}\xi\right)^\frac{2}{2^*}\\*
			&&  +\, M \int\limits_{\Om \, \cap \,\{\snr{f(\xi,u)} \leq  M\}} u^{q+1}\varphi^2 \, {\rm d}\xi\notag\,.
		\end{eqnarray}
		Thus, choosing~$M>0$ sufficiently large such that
		\[
		\frac{(q+1)^2}{q}\|f\|_{L^{Q/2}(\Om \, \cap \, \{\snr{f(\xi,u)}
			\, \geq \,M\})} \leq \frac{(S^*)^{-\frac{2}{2^*}}}{2}\,,
		\]
		and absorbing the first integral on~\eqref{app_eq_6} on the left-hand side of~\eqref{app_eq_5} yields
		\[
		\frac{(S^*)^{-\frac{2}{2^*}}}{2}\left(\int\limits_\Om \snr{\varphi u^{(q+1)/2}}^{2^*} \, {\rm d}\xi\right)^\frac{2}{2^*} \,\leq \,, M \int\limits_\Om u^{q+1}\varphi^2 \, {\rm d}\xi +c(n,q)\int\limits_\Om u^{q+1}\snr{D_H\varphi}^2 \, {\rm d}\xi \,
		\]
		which implies the desired estimate in~\eqref{gain_int}.
		
		\vspace{2mm}
		\mbox{}	Now we prove~\eqref{sup_est}. 
		With no loss of generality, we assume that~$\rr=1$. Consider for~$p > 1$, $B_4(0)\equiv B_4$ and a nonnegative cut-off function~$\varphi \in C^\infty_0(B_{4})$,  the test function $\phi:= \varphi^2u^p$. So that, the weak formulation in~\eqref{weak_form} yields
		\begin{equation}\label{app_eq_9}
			\int\limits_{\Om}  D_H u \cdot D_H (\varphi^2u^p) \, {\rm d}\xi= \int\limits_{\Om} f(\xi,u)  \varphi^2u^{p+1}\, {\rm d}\xi.
		\end{equation}
		Integrating by parts the first integral on the left-hand side in~\eqref{app_eq_9} and using Young's Inequality yield
		\begin{eqnarray*}
			&& \int\limits_{\Om}  D_H u \cdot D_H (\varphi^2u^p) \, {\rm d}\xi\nonumber\\*
			&&\qquad \qquad \geq -\,2\int\limits_{\Om} \varphi u^p \snr{D_Hu} \snr{ D_H \varphi} \, {\rm d}\xi + p\int\limits_{\Om}  \varphi^2u^{p-1}\snr{D_H u}^2 \, {\rm d}\xi\nonumber\\*[0.5ex]
			%	   && \quad\quad \geq -\,\frac{1}{\epsilon}\int\limits_{\Om} \varphi^2 u^{p-1} \snr{D_H u}^2 \, {\rm d}\xi -\epsilon\int\limits_{\Om} u^{p+1} \snr{D_H \varphi}^2 \, {\rm d}\xi+ p\int\limits_{\Om}  \varphi^2u^{p-1}\snr{D_H u}^2 \, {\rm d}\xi\nonumber\\*[0.5ex]
			&&\qquad \qquad \geq \frac{p}{2}\int\limits_{\Om} \varphi^2 u^{p-1} \snr{D_H u}^2 \, {\rm d}\xi -\frac{2}{p}\int\limits_{\Om} u^{p+1} \snr{D_H \varphi}^2 \, {\rm d}\xi.
		\end{eqnarray*}
		Putting all together we have that
		$$
		\int\limits_{\Om} \varphi^2 u^{p-1}\snr{D_H u}^2 \, {\rm d}\xi \,\leq ,  \frac{2}{p}\int\limits_{\Om} f(\xi,u)\varphi^2u^{p+1} \, {\rm d}\xi + \frac{4}{p^2}\int\limits_{\Om} u^{p+1} \snr{D_H \varphi}^2 \, {\rm d}\xi.
		$$
		Applying now Sobolev's Inequality yields that
		\begin{eqnarray}\label{app_eq_10}
			\|\varphi u^\frac{p+1}{2}\|_{L^{2^*}(\Om)}^2 &\leq & 2(S^*)^\frac{2}{2^*}\int\limits_{\Om} \varphi^2 u^{p-1}\snr{D_H u}^2 \, {\rm d}\xi + 2(S^*)^\frac{2}{2^*}\int\limits_{\Om}  u^{p+1}\snr{D_H \varphi}^2 \, {\rm d}\xi\nonumber\\*[0.5ex]
			&\leq & \underbrace{2(S^*)^\frac{2}{2^*}\frac{(1+p)^2+3}{p^2}}_{=: \bar{c}}\int\limits_{\Om} \Big(f(\xi,u)\varphi^2+\snr{D_H \varphi}^2\Big)u^{p+1} \, {\rm d}\xi,
		\end{eqnarray}
		where the constant~$\bar{c}$ is bounded for any~$p$ greater than~$1$. 
		
		Using H\"older's Inequality with $q/2$ and $q/(q-2)$ and an interpolative inequality, we obtain
		\begin{eqnarray*}
			\int\limits_{\Om} f(\xi,u)\big(\varphi u^\frac{p+1}{2}\big)^2 \, {\rm d}\xi 
			&\leq &   \|f\|_{L^{q/2}(\Om)}  \|\varphi u^\frac{p+1}{2}\|_{L^{2q/(q-2)}(\Om)}^2\\*[0.5ex]
			&\leq &  \|f\|_{L^{q/2}(\Om)}\left( \epsilon\|\varphi u^\frac{p+1}{2}\|_{L^{2^*}(\Om)} +\epsilon^{-\frac{Q}{q-Q}}\| \varphi u^\frac{p+1}{2}\|_{L^{2}(\Om)}\right)^2, 
			% uso la (7.10) del Gilbarg-Trudinger
		\end{eqnarray*}
		Thus, by choosing~$\epsilon$ sufficiently small and absorbing the terms in the left-hand side in~\eqref{app_eq_10}, we have 
		\begin{equation}\label{app_eq_11}
			\|\varphi u^\frac{{\tt p}}{2}\|_{L^{2(2^*/2)}(\Om)} \leq  c\,{\tt p}^\frac{q}{q-Q}\|(\varphi+\snr{D_H\varphi})u^\frac{{\tt p}}{2}\|_{L^2(\Om)},
		\end{equation}
		where we denoted by~${\tt p}:=p+1$.
		We specify now the cut-off function~$\varphi$.  Let $ 1 \leq \sigma < \rho \leq 3$ and choose~$\varphi$ such that
		\[
		0 \leq \varphi \leq 1, \quad \varphi \equiv 1 \, \text{on}~B_\sigma, \quad \varphi \equiv 0 \, \text{on}~\h \setminus B_\rho, \quad \snr{D_H \varphi} \leq \frac{2}{\rho-\sigma}.
		\]
		With such a choice of~$\varphi$, the estimate in~\eqref{app_eq_11} becomes
		\begin{equation}\label{app_eq_12}
			\| u^{{\tt p}/2}\|_{L^{2(2^*/2)}(B_\sigma)} \, \leq \, \frac{c\,{\tt p}^\frac{q}{q-Q}}{\rho-\sigma}\|u^{{\tt p}/2}\|_{L^2(B_\rho)}.
		\end{equation}
		Thus, once defined~$A_{{\tt{q}},s}:= \left(\int\limits_{B_s} u^{\tt{q}} \, {\rm d}\xi\right)^\frac{1}{{\tt{q}}}$, we have that the inequality in~\eqref{app_eq_12} becomes
		\begin{equation}\label{app_eq_13}
			A_{\frac{2^*{\tt p}}{2},\sigma}\, \leq \, \left(\frac{c\,{\tt p}^\frac{q}{q-Q}}{\rho-\sigma}\right)^\frac{2}{{\tt p}}	A_{{\tt p},\rho}.
		\end{equation}
		We are finally in the position to start a classical iteration method in order to get the desired supremum estimate. Taking~${\tt p}_j := (2^*/2)^j {\tt p}$ and~$\rho_j:=1+2^{-j}$, for~$j =0,1,\dots$, we prove that 
		\begin{equation}\label{app_eq_14}
			A_{{\tt p}_N,\rho_N} \, \leq \, \prod_{j=0}^{N}\left(\frac{c\, {\tt p}_j^\frac{q}{q-Q}}{\rho_j-\rho_{j+1}}\right)^\frac{2}{{\tt p}_j}A_{{\tt p},2} \qquad\text{for any}~N\geq1.
		\end{equation}
		Clearly the case~$N \equiv 1$ follows from~\eqref{app_eq_13}. We now assume that the estimate above holds for~$N$ and prove it for~$N+1$. Indeed, recalling~\eqref{app_eq_13} we have
		\begin{eqnarray*}
			A_{{\tt p}_{N+1}{\tt p},\rho_{N+1}} &\leq & \left(\frac{c\,{\tt p}_N^\frac{q}{q-Q}}{\rho_{N}-\rho_{N+1}}\right)^\frac{2}{{\tt p}_N}	A_{{\tt p}_N,\rho_N} \\*[0.5ex]
			&\leq & \left(\frac{c\,{\tt p}_N^\frac{q}{q-Q}}{\rho_{N}-\rho_{N+1}}\right)^\frac{2}{{\tt p}_N} \prod_{j=0}^{N}\left(\frac{c\, {\tt p}_j^\frac{q}{q-Q}}{\rho_j-\rho_{j+1}}\right)^\frac{2}{{\tt p}_j}	A_{{\tt p},2}\\*[0.5ex]
			&=&\prod_{j=0}^{N+1}\left(\frac{c\, {\tt p}_j^\frac{q}{q-Q}}{\rho_j-\rho_{j+1}}\right)^\frac{2}{{\tt p}_j}	A_{{\tt p},2}\,,
		\end{eqnarray*}	
		and the induction step does follow.
		
		Moreover, note that
		\[
		\prod_{j=0}^{\infty}\left(\frac{c\, {\tt p}_j^\frac{q}{q-Q}}{\rho_j-\rho_{j+1}}\right)^\frac{2}{{\tt p}_j} = \left(\frac{c\, {\tt p}^\frac{q}{q-Q}}{\rho_0-\rho_{1}}\right)^\frac{2}{{\tt p}}{\rm e}^{\, \sum_{j} \frac{2\log \Big(c\, 2^{j+1} {\tt p}_j^\frac{q}{q-Q}\Big)}{{\tt p}_j}},
		\]
		where we also used that $
		\displaystyle	\sum_{j=1}^\infty \frac{2\log \Big(c\, 2^{j+1} {\tt p}_j^\frac{q}{q-Q}\Big)}{{\tt p}_j} 
		%= 	2\log c\sum_{j=1}^\infty \frac{1}{{\tt p}_j} +	2\log2\sum_{j=1}^\infty \frac{j+1}{{\tt p}_j}
		%  + 	\frac{2q}{q-Q}\sum_{j=1}^\infty \frac{\log  {\tt p}_j}{{\tt p}_j}
		<\infty.
		$
		%	\]
		\vspace{2mm}
		
		Thus, letting~$N$ going to infinity in~\eqref{app_eq_14}, we eventually arrive at
		$$
		\sup_{B_1} u \leq  c \left( \int\limits_{B_2}u^{\tt p} \, {\rm d}\xi\right)^\frac{1}{{\tt p}},
		$$
		which gives the desired estimate~\eqref{sup_est} choosing~${\tt p}=2^*$.
	\end{proof}

 \subsection{{\it H}-Kelvin transform}
 	We briefly recall  some notion about the {\it $H$-Kelvin transform} which will play an important role in the proof of~Theorem~\ref{han}. 
 	
 	
 	
 	\begin{defn}\label{def_kelvin}
 		For any~$\xi =(z,t) \in \h \setminus \{0\}$ we call \textup{$H$-inversion}  the map
 		\begin{eqnarray*}
 			&& \mathpzc{k}: \h \setminus \{0\} \longmapsto  \h \setminus \{0\},\\
 			&& \mathpzc{k} (\xi) := \left(\frac{z}{\snr{z}^2+i t}, \frac{t}{\snr{z}^4+t^2}\right).
 		\end{eqnarray*}
 		With~$\mathpzc{k}$ defined above, given a function~$u : \h \rightarrow \mathbb{R}$ we denote by~$u^\sharp$ its {\rm $H$-Kelvin transform} defined by
 		\begin{eqnarray*}
 			&& u^\sharp : \h \setminus \{0\} \rightarrow \mathbb{R},\\
 			&& u^\sharp (\xi) := \snr{\xi}_{\h}^{-(Q-2)}\, u \big(\mathpzc{k}(\xi)\big).
 		\end{eqnarray*}
 	\end{defn}
 	It can be easily verified that
 	\begin{equation}\label{H-inve-prop}
 		\mathpzc{k}(\mathpzc{k}(\xi))=g, \quad \textup{and} \quad \snr{\mathpzc{k}(\xi)}_{\h} = \snr{\xi}_{\h}^{-1}.
 	\end{equation}
 	
 	We would now need to present a few properties of the $H$-Kelvin transform adapted to our framework.
 	\begin{prop}[See  Theorem~2.3.5 in \cite{IV11}]\label{H-Kelv-iso}
 		Let~$\Omega$ be a domain and denote by~${\Omega^{\sharp}}$ the image of~$\Omega$ under the $H$-inversion~$\mathpzc{k}$. Then, we have that the $H$-Kelvin transform is an isometry between $\Sc(\Om)$ and~$\Sc({\Omega^{\sharp}})$.
 	\end{prop} 
 	
 	\begin{prop}[See Lemma~2.3.6 in \cite{IV11}]\label{CR-lap}
 		Let~$u$ be a solution to
 		\begin{equation*}
 			\begin{cases}
 				-\Delta_H u = u^p,\\[1ex]
 				u \in \Sc(\Om), \quad u\ge0,
 			\end{cases}
 		\end{equation*}
 		for some positive exponent~$p>0$. Then, its $H$-Kelvin transform~$u^\sharp$ satisfies 
 		\begin{equation*}
 			\begin{cases}
 				-\Delta_H u^\sharp(\xi) = \snr{\xi}_{\h}^{p(Q-2)-(Q+2)}u^\sharp(\xi)^p,\\[1ex]
 				u^\sharp \in \Sc(\Om^\sharp), \quad u^\sharp\ge0.
 			\end{cases}
 		\end{equation*} 
 	\end{prop}
 	
 	Lastly we will take advantage of the maximum principle stated below.
 	\begin{prop}[See Proposition 1.3 in \cite{BP02}]\label{biridnelli_max_prin}
 		Let $E$ be a smooth bounded domain on $\h$ and let $f \in L^\infty(E)$. Then, there exists $\delta>0$ depending only on $n$ and  $\|f\|_{L^\infty(E)}$ such that the maximum principle holds for $\Delta_H +f$ provided that
 		$
 		\snr{E} < \delta.
 		$ 
 \end{prop}



\subsection{Proof of Theorem~\ref{han}}
Consider a maximizing sequence~$\{u_\eps\}$ of~\eqref{sobolev}. Then it holds that
\begin{equation}\label{eq_han1}
	\int\limits_\Om \snr{u_\eps}^{2^*-\eps}\, {\rm d}\xi = S^* + {o}(1), \qquad \text{as}~\eps \to 0,
\end{equation}
where we also used Proposition~2.5 in~\cite{PPT23}.

\vspace{2mm}
   {\bf Step 1. The sequence of the supremum norms~$\|u_\eps\|_{L^\infty}$ diverges; i.~\!e., 
	\begin{equation}\label{han_supremum_limit}
		\|u_\eps \|_{L^\infty(\Om)} \to \infty, \qquad \text{as}~\eps \to 0.
	\end{equation} 
}
	By contradiction assume that there exists a sequence~$\{\eps_k\}_k$ for which~$u_{\eps_k}$ remains bounded in~$\Om$ for~$\eps_k \to 0^+$ as~$k \to +\infty$. This allows
	to apply the local regularity theory firstly developed by Folland and Stein in~\cite{Fol75,FS74} to conclude that~$u_\eps \in C^\infty(\Om)$. Moreover, by taking into account the regularity assumptions on the set~$\Omega$ as in particular the measure density condition~\eqref{non_deg_Om},  
	the classical argument via Moser's Iteration leads to~$u_\eps \in \Gamma^{0,\alpha_{\eps_k}}(\partial \Om)$, for some exponent~$1>\alpha_{\eps_k} > \alpha_{\bar{\eps}} >0$ for any~$\eps_k < \bar{\eps}$. Hence, up to subsequences, we can assume that $u_{\eps_k} \to v \neq \infty$ uniformly on~$\Omega$.
	
	If the limit function~$v \equiv 0$, then by~\eqref{eq_han1} we have reached a contradiction since~$S^* \neq 0$. On the other hand, if~$v \ne 0$ we would have obtained a maximizer of~\eqref{critica}	which is a contradiction as well. Thus,~\eqref{han_supremum_limit} holds true. 

    \vspace{2mm}
    Choose now a sequence of points~$\{\eta_\eps\} \subseteq \Om$ and a sequence of numbers~$\{\lambda_\eps\} \subseteq \mathbb{R}^+$ such that
    \begin{equation}\label{eq_han2}
	u_\eps(\eta_\eps) =\lambda_\eps^{-\frac{Q-2}{2}}\equiv\|u_\eps\|_{L^\infty(\Om)}.
    \end{equation}
    Consider the function
      \begin{equation}\label{han_v_eps}
	  v_\eps (\xi):= \lambda_\eps^\frac{Q-2}{2}u_\eps \bigg(\tau_{\eta_\eps}\Big( \delta_{\lambda_\eps^{1-\frac{Q-2}{4}\eps}}(\xi)\Big)\bigg),
      \end{equation}
       which is a weak solution to
	\begin{equation}\label{eq_han3}
		\begin{cases}
			-\Delta_H v_\eps = v_\eps^{2^*-1-\eps} & \text{in}~\Om_\eps := \delta_{\lambda_\eps^{\frac{Q-2}{4}\eps-1}}\Big(\tau_{\eta_\eps}^{-1}(\Om)\Big)\\*[1ex]
			v_\eps (0) =1,\\[1ex]
			0\leq  v_\eps \leq 1.
		\end{cases}
	\end{equation}
	Indeed, by the homogeneity of the sub-Laplacian we get
	\begin{eqnarray*}
		-\Delta_H v_\eps(\xi) &=& \lambda_\eps^\frac{Q-2}{2} \lambda_\eps^{2-\frac{Q-2}{2}\eps}(-\Delta_H u_\eps) \bigg(\tau_{\eta_\eps}\Big( \delta_{\lambda_\eps^{1-\frac{Q-2}{4}\eps}}(\xi)\Big)\bigg)\\*
		&=& \lambda_\eps^{\frac{Q+2}{2} -\frac{Q-2}{2}\eps}u_\eps^{2^*-1-\eps} \bigg(\tau_{\eta_\eps}\Big( \delta_{\lambda_\eps^{1-\frac{Q-2}{4}\eps}}(\xi)\Big)\bigg)\\*
		&=& \lambda_\eps^{\frac{Q-2}{2}(\frac{Q+2}{Q-2} -\eps)}u_\eps^{2^*-1-\eps} \bigg(\tau_{\eta_\eps}\Big( \delta_{\lambda_\eps^{1-\frac{Q-2}{4}\eps}}(\xi)\Big)\bigg)\, =\, v_\eps^{2^*-1-\eps}\,.
	\end{eqnarray*}
	Also, recalling the choice of~$\{\eta_\eps\}$ and~$\{\lambda_\eps\}$ in~\eqref{eq_han2}, we have that
	\[
	v_\eps(0) = \lambda_\eps^\frac{Q-2}{2}u_\eps(\eta_\eps) = 1,
	\]
     and
      that~$0 \leq v_\eps 
      \leq 1$ in~$\Om_\eps$. 
      
      Now, since the sequence~$\{v_\eps\}$ is bounded it is equicontinuous on compact subset of~$\h$, and by Ascoli-Arzel\`a's Theorem, up to subsequences, there exists a function~$v_\infty \not\equiv 0$ such that~$v_\eps \to v_\infty$ uniformly on compact set.

      \vspace{2mm}
     {\bf Step 2. The asymptotics in~\eqref{conv_xi_eps} for the sequence $\{\eta_\eps\}$ chosen in Step~1 holds true.} Recall that $\snr{D_H u_\eps}^2 \, {\rm d}\xi \tows \delta_{\xi_{\rm o}}$ in~$\mea$ for a given point~$\xi_{\rm o} \in \overline{\Om}$. Then, considering the function~$v_\eps$ in~\eqref{han_v_eps}, which converges to~$v_\infty$ uniformly on compact set, it yields
    \begin{eqnarray*}
    	0 < \int\limits_{B_\rr(0)} \snr{D_H v_\infty}^2 \, {\rm d}\xi &=& \lim_{\eps\to 0^+}\int\limits_{B_\rr(0)}\snr{D_Hv_\eps}^2 \, {\rm d}\xi \\*[1ex]
    	&=& \lim_{\eps \to 0^+} \big(\lambda_\eps^\eps\big)^\frac{(Q-2)^2}{4}\int\limits_{B_{\lambda_\eps^{1-\frac{Q-2}{4}\eps}R}(\eta_\eps)}\, \snr{D_H u_\eps}^2 \, {\rm d}\xi,
    \end{eqnarray*}
    which, in view of the fact that $\lambda_\eps^\eps \to 1$, gives a contradiction if~$\eta_\eps$ does not converge to~$\xi_{\rm o}$.

    \vspace{2mm}
   {\bf Step 3. The concentration point~$\xi_{\rm o}$ is away from the boundary of~$\partial \Om$.} In order to prove this result, we will employ the maximum principle stated in Proposition~\ref{biridnelli_max_prin}. 

    We show that there exists an exterior direction~$\varsigma$, such that~$\partial_\varsigma u_\eps<0$. By contradiction assume that there is no such direction. Then, by choosing~$\varsigma$ with~$\snr{\varsigma}_{\h}=1$, and denoting $u_{\lambda,\eps} := u_\eps(\tau_{\delta_\lambda(\varsigma)}(\cdot))$, we have that $u_{\lambda,\eps}$ is such that~$u_{\lambda,\eps} \geq u_\eps$ 
    and it solves the following problem,
    \[
    \begin{cases}
    -\Delta_H u_{\lambda,\eps} = u_{\lambda,\eps}^{2^*-1-\eps}  & \textrm{in} \  \tau_{(\delta_{\lambda}(\varsigma))^{-1}}(\Omega)\\[1ex]
     u_{\lambda,\eps} =0 \, &\textrm{on} \ \partial \tau_{(\delta_{\lambda}(\varsigma))^{-1}}(\Omega).
    \end{cases}
     \]
Clearly, because of the boundedness of~$\Omega$ one has that there exists~$\lambda_{\rm o}\geq0$ such that~$\tau_{(\delta_{\lambda_{\rm o}}(\varsigma))^{-1}}(\Omega) \cap \Omega = \emptyset$ and~$\Omega\cap\tau_{(\delta_{\lambda}(\varsigma))^{-1}}(\Omega) \neq \emptyset$ for any~$\lambda \in [0,\lambda_{\rm o})$. We immediately notice that in the case when~$\lambda=0$, one has~$\tau_{(\delta_{\lambda}(\varsigma))^{-1}}(\Omega) = \Omega$. For any~$\lambda <\lambda_{\rm o}$ let us consider the function~$w_{\lambda,\eps}$ defined as follows,
\[
w_{\lambda,\eps} := u_{\lambda,\eps}-u_\eps.
\]
Such a function is the solution of the following problem,
\[
\begin{cases}
	\Delta_H w_{\lambda,\eps} - f(\xi) w_{\lambda,\eps}\leq 0 \, & \textrm{in} \ \tau_{(\delta_{\lambda}(\varsigma))^{-1}}(\Omega) \cap \Omega,\\[1ex]
	w_{\lambda,\eps} = 0 \, &\textrm{on} \ \partial \big(\tau_{(\delta_{\lambda}(\varsigma))^{-1}}(\Omega) \cap \Omega\big),
\end{cases}
\]
where~$f  \in L^\infty(\tau_{(\delta_{\lambda}(\varsigma))^{-1}}(\Omega) \cap \Omega)$. Letting~$\delta >0$ be the one given by Proposition \ref{biridnelli_max_prin} and choosing~$\lambda_\delta $ such that 
\begin{equation}\label{delta}
\big|{\tau_{(\delta_{\lambda}(\varsigma))^{-1}}(\Omega) \cap \Omega\big|}<\delta \quad \textrm{for any} \ \lambda>\lambda_\delta,
\end{equation}
 we can apply the maximum principle to get that
\[
w_{\lambda,\eps} \geq 0 \quad \textrm{in}~\tau_{(\delta_{\lambda}(\varsigma))^{-1}}(\Omega) \cap \Omega.
\]
Moreover, by the strong maximum principle, we get in particular that
\[
w_{\lambda,\eps} >0 \quad \textrm{in}~\tau_{(\delta_{\lambda}(\varsigma))^{-1}}(\Omega) \cap \Omega.
\]
Now, define 
\[
\lambda_1 := \inf\Big\{\lambda>0: w_{s,\eps} >0 \ \, \textrm{for any} \ s > \lambda \Big\} \geq 0.
\]
We show that $\lambda_1 =0$. By contradiction assume that $\lambda_1 >0$. Note that $w_{\lambda_1}\geq 0$ on~$\tau_{(\delta_{\lambda_1}(\varsigma))^{-1}}(\Omega) \cap \Omega$ and, by the strong maximum principle,~$w_{\lambda_1,\eps}>0$. 
\vspace{1mm}

Choose a compact set~$K \subset \tau_{(\delta_{\lambda_1}(\varsigma))^{-1}}(\Omega) \cap \Omega$ such that $\snr{\big(\tau_{(\delta_{\lambda_1}(\varsigma))^{-1}}(\Omega) \cap \Omega\big)\smallsetminus K} < \delta/3$ and
\begin{equation}\label{pos_Ds}
w_{s,\eps} \geq 0 \ \, \textrm{on}~K, \, \textrm{for}~s <\lambda_1.%, \,  \textrm{sufficiently near to}~\lambda_1. ALERT
\end{equation}
Fix now~$0<\lambda < \lambda_1$ such that
\begin{equation}\label{pos_sigma}
K \subset \tau_{(\delta_{\lambda}(\varsigma))^{-1}}(\Omega) \cap \Omega, \quad \textrm{and} \quad \snr{\big(\tau_{(\delta_{\lambda}(\varsigma))^{-1}}(\Omega) \cap \Omega\big) \smallsetminus K}<\delta, 
\end{equation}
where~$\delta$ is the one appearing in~\eqref{delta}.

Choose~$\lambda < \lambda_1$ sufficiently near to~$\lambda_1$  such  that~\eqref{pos_Ds} and~\eqref{pos_sigma} are satisfied for any~$s \in (\lambda,\lambda_1)$. This plainly leads to a contradiction. Indeed,~$w_{s,\eps} \geq 0$ on~$\big(\tau_{(\delta_{s}(\varsigma))^{-1}}(\Omega) \cap \Omega\big) \smallsetminus K$ by Proposition~\ref{biridnelli_max_prin} which together with \eqref{pos_Ds} does imply that~$w_{s,\eps} \geq 0$ on~$\tau_{(\delta_{s}(\varsigma))^{-1}}(\Omega) \cap \Omega$. Moreover, recalling that~$w_{s,\eps} >0$ since~$s>\lambda>0$, from the strong maximum principle it follows that~$w_{s,\eps}>0$ for some~$s <\lambda_1$, which is a contradiction.
Therefore, we have that~$u_\eps\big(\tau_{\delta_{\lambda}(\varsigma)}(\cdot)\big)$ is decreasing for any$\lambda$~in $(0,\lambda_{\rm o})$. Hence, there exists at least one exterior direction such that~$u_\eps$ decreases along~$\varsigma$.
\vspace{2mm}

Now, for any~$\xi \in \{\xi' \in \Omb: {\rm dist}(\xi',\partial \Om) <\vartheta\}$, up to choosing a constant~$\vartheta>0$  small enough, we can find~$K_\xi $ such that
\[
\snr{K_\xi}>0, \quad K_\xi \subset \left\{\xi' \in \Omega: {\rm dist}(\xi',\partial \Om) > \frac{\vartheta}{2}\right \}, \quad \textup{and}~u_\eps(\xi') >  u_\eps(\xi), \,  \forall \xi ' \in K_\xi.
\]
Thus, being~$\{u_\eps\}$ a maximizing sequence for~$\Ssub$, we finally arrive at
\[
u_\eps(\xi)\, <\, \dashint_{K_\xi}u_\eps \, {\rm d}\xi 
\,\leq , \snr{K_\xi}^{-\frac{1}{2^*-\eps}}\|u_\eps\|_{L^{2^*-\eps}(\Omega)}
\, =\,  \snr{K_\xi}^{-\frac{1}{2^*-\eps}}(S^* \big)^\frac{1}{2^*-\eps} +{\rm o}(1),
 \]
 Then, since~$u_\eps (\eta_\eps) \to \infty$, it must follow that~$\eta_\eps$ is away from the boundary (up to choosing~$\eps$ small enough).
 
  \vspace{2mm}
   {\bf Step 4. The function $v_\eps$ in \eqref{han_v_eps} converges to~$U$ in~\eqref{talentiane_2} locally uniformly on compact sets.}
    As proven above we have that the sequence $\{v_\eps\}$ converges uniformly on compact set to $v_\infty$. 
      
	 Moreover, if we denote with~$\Om_{\rm o}$ the limiting set for~$\eps \to 0$ of~$\Om_\eps$, we have that~$\Om_{\rm o} \equiv \h$.
     Indeed, thanks to Step~2 and Step~3 we have that the sequence 
     \[
     \Big\{\lambda_\eps^{\frac{Q-2}{4}\eps-1} {\rm dist}(\eta_\eps, \partial \Om)\Big\}_{\eps>0}
     \] 
     is unbounded. Then, recalling~\cite[Lemma~3.4]{CU01} we have that the limiting space $\Om_{\rm o}$ does coincide with $\h$.
   
	
	This yields that~$v_\infty$ is a solution to 
	\[
	\begin{cases}
		-\Delta_H v_\infty = v_\infty^{2^*-1}, & \text{in}~\h,\\
		v_\infty(0) =1,\\
		0\leq  v_\infty\leq 1,
	\end{cases}
	\]
	which implies that~$v_\infty$ coincides with the function~$U$, defined in~\eqref{talentiane_2}.   
	
	  \vspace{2mm}
	{\bf Step 5. The sequence~$\{\lambda_\eps\}$ satisfies
 \begin{equation}\label{bound_lambda_eps}
  0 <c \leq \lambda_\eps^\eps \leq 1, \qquad \textrm{with}~c\equiv c(n).
 \end{equation}
 } First of all, since~$\lambda_\eps \to 0$ when~$\eps \to 0$, by~\eqref{eq_han2}, it trivially follows that~$\lambda_\eps^\eps < 1$, for~$\eps$ sufficiently small. Moreover, since the function~$v_\eps$ defined in~\eqref{han_v_eps} tends to~$U$ in~\eqref{talentiane_2} uniformly in~$B_1(0)$, we have that there exists a constant~$c$ such that
	\[
	\int\limits_{B_1(0)}v_\eps^{2^*-\eps} \, {\rm d}\xi \geq c.
	\]
	Then, we get
	\begin{eqnarray*}
		\int\limits_{B_1(0)}v_\eps^{2^*-\eps} \, {\rm d}\xi & = & \int\limits_{B_1(0)}\lambda_\eps^{\frac{Q-2}{2}\big(\frac{2Q}{Q-2}-\eps\big)} u_\eps^{\frac{2Q}{Q-2}-\eps} \Big(\tau_{\eta_\eps}\Big( \delta_{\lambda_\eps^{1-\frac{Q-2}{4}\eps}}(\xi)\Big)\Big)\, {\rm d}\xi\\
		&=&\lambda_\eps^{\frac{(Q-2)^2}{4}\eps}\int\limits_{\snr{\xi\circ\eta_\eps^{-1}}_{\h} \leq \lambda_\eps^{1-\frac{Q-2}{4}}} u_\eps^{\frac{2Q}{Q-2}-\eps} (\xi)\, {\rm d}\xi\\
		& \leq &  \lambda_\eps^{\frac{(Q-2)^2}{4}\eps}\int\limits_{\Om} u_\eps^{2^*-\eps} (\xi)\, {\rm d}\xi,
	\end{eqnarray*}
	which, together with~\eqref{eq_han1}, gives the desired estimate for a proper constant~$c\equiv c(n) >0$.
	
	\vspace{2mm}
	{\bf Step 6. The sequence~$\{\lambda_\eps\}$ satisfies~\eqref{conv_xi_eps}.} By means of the Mean Value Theorem we have that there exists $\vartheta \in (\lambda_\eps^\eps,1)$ such that
	\[
	\frac{1}{\vartheta}\,=\, \int\limits_{\lambda_\eps^\eps}^1 \frac{1}{s} \, {\rm d}s
	\,=\, \frac{\ln \lambda_\eps^\eps}{\lambda_\eps^\eps-1}.
	\]
	Hence, considering $\textup{\texttt{t}}:= \ln\vartheta/\ln \lambda_\eps^\eps \in (0,1)$ we have that
	\[
	\snr{\lambda_\eps^\eps -1} = \lambda_\eps^{\textup{\texttt{t}}\eps}\eps \snr{\ln \lambda_\eps} ,
	\]
	by \eqref{bound_lambda_eps}, which gives the desired result.
	
	
	\vspace{2mm}
	{\bf Step 7. The sequence~$\{v_\eps\}$ in~\eqref{han_v_eps} converges to~$U$ in~$\Sc(\Om)$.}
	We show now that the sequence~$\{v_\eps\}$ is a Palais-Smale sequence for~$\mathcal{E}_0$; i.~\!e., \eqref{PS1} and~\eqref{PS2} hold true. We start by showing the bound in~\eqref{PS1}. For this, we recall the homogeneity of the horizontal gradient~$D_H$ and the result in~\eqref{conv_xi_eps}; we have
		\begin{eqnarray}\label{PS1_1}
	\int\limits_{\Om_\eps}\snr{D_Hv_\eps}^2 \, {\rm d}\xi &=&\int\limits_{\Om_\eps}\lambda_\eps^{Q-\frac{Q-2}{2}\eps}\Big|(D_Hu_\eps) \Big(\tau_{\eta_\eps}\Big( \delta_{\lambda_\eps^{1-\frac{Q-2}{4}\eps}}(\xi)\Big)\Big)\Big|^2 \, {\rm d}\xi \notag\\*[1ex]
	&=& \lambda_\eps^{\frac{(Q-2)^2}{4}\eps}\int\limits_\Om \snr{D_Hu_\eps}^2 \, {\rm d}\xi \notag\\*[1ex]
	&=& \lambda_\eps^{\frac{(Q-2)^2}{4}\eps} < c.
	\end{eqnarray}
    In a similar fashion, by means of~\eqref{folland} and~\eqref{conv_xi_eps}, one has
    	\begin{eqnarray}\label{PS1_2}
    	\int\limits_{\Om_\eps}\snr{v_\eps}^{2^*} \, {\rm d}\xi
    	&=& \lambda_\eps^{\frac{Q(Q-2)}{4}\eps}\int\limits_\Om \snr{u_\eps}^{2^*} \, {\rm d}\xi \notag\\*[1ex]
    	&\leq & \lambda_\eps^{\frac{Q(Q-2)}{4}\eps}S^*\|D_Hu_\eps\|_{L^2(\Om)}^{2^*} \, <\,c.
    \end{eqnarray}
    In view of the estimate~\eqref{PS1_1} and~\eqref{PS1_2} above, condition~\eqref{PS1} plainly follows. Condition~\eqref{PS2} can be deduced by Ascoli-Arzel\`a's Theorem recalling the definition of~$U$. Thus, thanks to the Global Compactness~Theorem~\ref{thm_glob_comp} we have that the~$v_\eps \to U$ in~$\Sc(\Om)$.
    
    
    \vspace{2mm}
    {\bf Step 8. The asymptotic estimate~\eqref{bound_max_seq} holds true.}
    We start by observing that proving~\eqref{bound_max_seq} is equivalent to show that the following estimate holds,
	\begin{equation}\label{boun_max_seq2}
		v_\eps (\xi) \lesssim U(\xi) \ \, \text{in}~\Om_\eps \, \ ~\text{as}~\eps \to 0^+.
		\end{equation}   
	Consider the $H$-Kelvin transform~$v^\sharp_\eps$ of~$v_\eps$, according to Definition~\ref{def_kelvin}; we now have that the proof of~\eqref{boun_max_seq2} is equivalent to show that
	\begin{equation}\label{boun_max_seq3}
		v^\sharp_\eps(\xi) \leq  c \ \, \text{in}~\Om^\sharp_\eps \ \,~\text{as}~\eps \to 0^+,
	\end{equation}
    for a suitable constant~$c\equiv c(n)$ and where~$\Om^\sharp_\eps$ is the $H$-Kelvin transformed of~$\Om_\eps$.
    Also,~$v_\eps \leq 1$ yields that~$v^\sharp_\eps \leq \snr{\xi}_{\h}^{-(Q-2)}$, and this will reduce the estimate for~$v^\sharp_\eps$ to be proven just near the origin.

 
\vspace{2mm}

Let us now focus on the local estimate in~\eqref{boun_max_seq3}; it suffices to prove it in~$B_\rr \equiv B_\rr(0) \subset \Om_\eps$, for $\rr>0$ sufficiently small. Moreover, since we want to apply the $H$-Kelvin transform, we consider {the punctured ball}~$B_\rr^0 = B_\rr \setminus \{0\}.$ By~\eqref{H-inve-prop} it immediately follows that the image of $B_\rr^0$ under the $H$-inversion is~$\h \setminus \overline{B_\rr}$.
   
   Since~$v_\eps$ satisfies~\eqref{eq_han3}, by Proposition~\ref{CR-lap}, its $H$-Kelvin transform is a solution to
	\[
		\begin{cases}
			-\Delta_H v^\sharp_\eps(\xi) = \snr{\xi}_{\h}^{-(Q-2)\eps}\big(v^\sharp_\eps(\xi)\big)^{2^*-1-\eps} & \text{in}~\h \setminus \overline{B_\rr},\\*[1ex] 
       v^\sharp_\eps \geq 0.
		\end{cases}
	\]
	 We apply Proposition~\ref{appendix_prop1}~(i), with~$f(\xi,v^\sharp_\eps) \equiv \snr{\xi}_{\h}^{-(Q-2)\eps}(v_\eps)^{2^*-2-\eps}$ there. Notice that such a function~$f$ does belong to~$L^\frac{Q}{2}(\h \setminus \overline{B}_\rr)$, for any~$\rr>0$. Indeed, we apply H\"older's Inequality with exponents
	 \[
	 \frac{22^*}{Q(2^*-2-\eps)} = \frac{4}{4-(Q-2)\eps} \quad \text{and} \quad \frac{4}{(Q-2)\eps},
	 \]
     yielding
    \begin{eqnarray}\label{diag_arg1}
        \int\limits_{\h \setminus \overline{B_\rr}}\,\snr{f(\xi,v^\sharp_\eps)}^\frac{Q}{2} \, {\rm d}\xi &=& \int\limits_{\h \setminus \overline{B_\rr}}\snr{\xi}_{\h}^{-\frac{Q(Q-2)\eps}{2}}(v^\sharp_\eps)^\frac{Q(2^*-2-\eps)}{2}   \, {\rm d}\xi\notag\\*[1ex]
        &\leq & \left(\,\int\limits_{\h \setminus \overline{B_\rr}}\snr{\xi}_{\h}^{-2Q} \, {\rm d}\xi \right)^{\frac{(Q-2)\eps}{4}}\left(\,\int\limits_{\h \setminus \overline{B_\rr}}(v^\sharp_\eps)^{2^*}  \, {\rm d}\xi\right)^{1-\frac{(Q-2)\eps}{4}}\,.
    \end{eqnarray}
     We now apply Proposition~\ref{H-Kelv-iso} to get
    \[
	\int\limits_{\h \setminus \overline{B_\rr}}(v^\sharp_\eps)^{2^*} \, {\rm d}\xi
	\,=\, \int\limits_{B_\rr^0}v_\eps^{2^*} \, {\rm d}\xi
	\,\leq \,,	\int\limits_{\Om_\eps}v_\eps^{2^*} \, {\rm d}\xi
	\,\leq \,, \lambda_\eps^{\frac{Q(Q-2)}{4}\eps}S^*\|D_Hu_\eps\|_{L^2(\Om)}^{2^*},
	\]
	where we also used the estimate~\eqref{PS1_2} in the previous step. An application of Proposition~\ref{appendix_prop1}-(i) (with $q=2^*$ there) yields %ALERT adj ref prop
	\begin{equation}\label{eq_han4}
		\int\limits_{\h \setminus \overline{B_\rr}}(v^\sharp_\eps)^\frac{(2^*)^2}{2} \, {\rm d}\xi
		\,\leq \,c \quad \textrm{for}\ c \equiv c(n)>0.
	\end{equation}	
	Moreover,
	% note that 
	by~\eqref{eq_han4}
	 we get~$f(\xi,v^\sharp_\eps)\in L^{q/2}\big(\h \setminus \overline{B_\rr} \big)$ once chosen
	\[
	\frac{q}{2}:=\frac{(2^*)^2 }{2(2^*-2)} \,  =\,  \frac{Q}{Q-2}\cdot\frac{Q}{2}
	\, >\, \frac{Q}{2}.
	\]
	Indeed, by H\"older's Inequality,
	\begin{eqnarray}\label{diag_arg2}
		&& \int\limits_{\h \setminus \overline{B_\rr}}\snr{f(\xi,v^\sharp_\eps)}^\frac{Q^2}{2(Q-2)} \, {\rm d}\xi\notag\\*[0.5ex] 
		&&\qquad\quad \leq \int\limits_{\h \setminus \overline{B_\rr}}\snr{\xi}_{\h}^{-\frac{Q^2\eps}{2}} (v^\sharp_\eps)^\frac{(4-(Q-2)\eps)Q^2}{2(Q-2)^2}  \, {\rm d}\xi\notag\\*[0.5ex]
		&&\qquad\quad \leq \left(\,\int\limits_{\h \setminus \overline{B_\rr}}\snr{\xi}_{\h}^{-\frac{2Q^2}{Q-2}} \, {\rm d}\xi \right)^{\frac{(Q-2)\eps}{4}}\left(\,\int\limits_{\h \setminus \overline{B_\rr}}(v^\sharp_\eps)^\frac{(2^*)^2}{2}  \, {\rm d}\xi\right)^{1-\frac{(Q-2)\eps}{4}}.
	\end{eqnarray}
    Thus, Proposition~\ref{appendix_prop1}~(ii) yields that
	\begin{equation}\label{sup_est_v_diesis}
	\sup_{(\h \setminus \overline{B_\rr}) \cap B_\sigma} v^\sharp_\eps \leq c,
	\end{equation}
    where, by Proposition~\ref{H-Kelv-iso} and estimate~\eqref{PS1_2}, the right-hand side in~\eqref{sup_est_v_diesis} is bounded as~$\eps\to0^+$. Hence, passing to the limit for~$\rr \to 0^+$ and~$\eps \to 0^+$, since by a diagonal argument both integrals in~\eqref{diag_arg1} and~\eqref{diag_arg2} stays bounded, we eventually arrive at
    \[
    \sup_{B_\sigma \setminus \{0\}} v^\sharp_\eps
    \, \leq \, c \ \, \text{as}~\eps \to 0^+\,,
    \]
    which finally gives the desired estimate~\eqref{boun_max_seq3}.


 
\vspace{2mm}



\section{Proof of the localization result in Theorem \ref{thm_green}}\label{sec_localization}
 
This section is devoted to the proof of the localization result in Theorem~\ref{thm_green}. As mentioned in the introduction, for such a proof we will literarily need all the results proven in the previous sections and those stated in Section~\ref{sec_cca}. We would also need a few further independent results, as integral estimates for the horizontal derivatives and boundedness up to the characteristic set for the $\mathcal{D}$-derivatives of~$u_\eps$. This is done in Section~\ref{sec_decay} below by adapting to our subcritical Heisenberg framework the approach firstly proposed by Garofalo and Vassilev in the work~\cite{GV00},  and by Vassilev in~\cite{Vas06} for the CR Yamabe equation. We also recall in the first part of the section some known results about intrinsic H\"older classes, Schauder-type estimates on the boundary, and Pohozaev-type identity.


\vspace{2mm}
\subsection{Taylor polynomials, boundary Schauder estimates and Pohozaev identity}
We start by recalling the definition of intrinsic H\"older classes~$\Gamma^{{m},\alpha}$; for further details we refer to the work by Folland and Stein in~\cite{Fol75,FS74}.
\begin{defn}
	Let~$\alpha \in (0,1)$ and~$\Om \subset \h$. A function~$u: \Om \rightarrow \mathbb{R}$ belongs to\,~$\Gamma^{0,\alpha}(\Om)$ if there exists a positive constant~$c$ such that
	\[
	[u]_{\Gamma^{0,\alpha}(\Om)}:=\sup_{\xi \neq \eta\,\, \xi, \eta  \in \Om}\frac{\snr{u(\xi)-u(\eta)}}{\snr{\eta^{-1}\circ \xi}_{\h}^\alpha} \leq c.
	\]
	For any~${m} \in \mathbb{N}$, we say that $u \in \Gamma^{{m},\alpha}(\Om)$ if~$Z_i u \in \Gamma^{{m}-1,\alpha}(\Om)$, for any~$i=1,\dots,2n$.
\end{defn}
The space~$\Gamma^{0,\alpha}(\Om)$ is a Banach space endowed with the following norm,
\[
\|u\|_{\Gamma^{0,\alpha}(\Om)}:= \|u\|_{L^\infty(\Om)} + [u]_{\Gamma^{0,\alpha}(\Om)}.
\] 
With this notation we recall some boundary Schauder-type estimates.



\begin{theorem}[Theorem 7.1 in \cite{Jer81}]\label{jerison_schauder_boundary}
	Let~$\Om$ be a smooth bounded domain of $\h$ and let $\phi \in C^\infty_0(\h)$ be supported in a small neighborhood of a non-characteristic point $\xi \in \partial \Om$. Given $f \in \Gamma^{{m},\alpha}(\overline{\Om})$, ${m} \in \mathbb{N}\cup \{0\}$, $0<\alpha<1$, then for the unique solution $u$ to 
	\[
	-\Delta_H u =f \ \, \textrm{in} \ \, \Om, \qquad u =0 \, \ \textrm{in} \, \ \partial \Om,
	\]
	one has $\phi u \in \Gamma^{{m}+2,\alpha}(\overline{\Om})$. 
\end{theorem}
\vs

\begin{theorem}[Theorem 1.1 in \cite{BGM19}]\label{boundary_hld}
	Let $\Om$ be a bounded subset of $\h$ of class $C^{1,\beta}$ for some $\beta \in (0,1)$ and assume that the set $B_\rr \cap \partial \Om$ is non-characteristic. Let $u \in \Sc (B_\rr \cap\Om)\cap C(\overline{B_\rr \cap  \Om})$ be a weak solution to
	\[
	-\Delta_H u = f \ \, \text{in}~B_\rr\cap\Om, \qquad 
	u_{|B_\rr \cap\partial\Om}=0,
	\]
	with~$f \in L^\infty (\overline{B_\rr \cap  \Om})$. Then, $u\in\Gamma^{1,\beta}(\overline{B_{\rr/2} \cap  \Om})$, and the following estimate does hold,
	\[
		[D_H u]_{\Gamma^{0,\beta}({B_\rr \cap \Om}) }
		\, \leq \, c\rr^{-1-\beta}\big(\|u\|_{L^\infty(B_\rr \cap \Om )} +\rr^2\|f\|_{L^\infty(B_\rr\cap \Om)}\big).
	\]
\end{theorem}
\vs 

Moreover, when sufficiently regularity is assumed the following Pohozaev identity holds true.
\begin{lemma}[{\bf Pohozaev-type identity}; see Theorem 3.4 in \cite{GV00} and Theorem 2.1 in \cite{GL92}]\label{pohozaev}\text{}
	Let $\Om$ be a $C^1$ domain and let $u\in \Gamma^2(\overline{\Om})$ be a solution to
	\[
	-\Delta_H u =f(u) \qquad \textrm{in} \quad \Om,
	\]
	for some function $f \in C(\r)$ such that $f(0)=0$. Setting $F(s) := \int\limits_0^s f(t) \, {\rm d}t$\,, the following identity holds
	\begin{eqnarray*}
		&& \int\limits_\Om \big(2QF(u)-(Q-2)uf(u)\big) \, {\rm d}\xi \\*
		&& \qquad = 2 \sum_{j=1}^{2n}\int\limits_{\partial \Om}\mathcal{D}u Z_j u \langle Z_j,\mathfrak{n}\rangle \,{\rm d}{\it H}^{Q-2} -\int\limits_{\partial \Om}\snr{D_H u}^2 \langle \mathcal{D},\mathfrak{n}\rangle \,{\rm d}{\it H}^{Q-2}\\*
		&&\qquad \quad + 2\int\limits_{\partial \Om}F(u)\langle \mathcal{D},\mathfrak{n}\rangle \,{\rm d}{\it H}^{Q-2} +(Q-2)\sum_{j=1}^{2n}\int\limits_{\partial \Om}u Z_j u \langle Z_j,\mathfrak{n}\rangle \,{\rm d}{\it H}^{Q-2}\,,
	\end{eqnarray*}
	where $\nu$ is the exterior unit normal and $\mathcal{D}$ is the vector field generating the anisotropic dilations $\{\delta_\lambda\}_{\lambda >0}$.
\end{lemma}


Lastly, in order to obtain the desired Theorem~\ref{thm_green}, overcoming some difficulties coming from the non-Euclidean structure considered here, we make use of a suitable Taylor-type expansion.
\begin{defn}
	Let~$u \in C^\infty(\h$. Then, for any~$m \in \mathbb{N} \cup \{0\}$ there exists a unique polynomial~$P$ being $\delta_\lambda$-homogeneous of degree at most~$m$ such that
	\[
	(Z_1,\dots,Z_{2n},T)^I P(\mathfrak{e})= (Z_1,\dots,Z_{2n},T)^I u(\mathfrak{e})
	\]
	for any multi-index~$I= (i_1,\dots,i_{2n+1})$ with ${\rm deg}_H(I):=i+\cdots+i_{2n}+2i_{2n+1} \leq m$. We say that~$P:=P_m(u,0)(\xi)$ is {\rm``}the {\rm MacLaurin polynomial} of $\delta_\lambda$-degree $m$ associated to $u${\rm ''}.
\end{defn}
Let~$u \in C^\infty(\h)$,~$\xi \in \h$ and~$m \in \mathbb{N} \cup \{0\}$. Let us consider the MacLaurin polynomial~$P_m (u(\xi \circ \cdot),\mathfrak{e})$ of the function
$\eta \longmapsto u(\xi \circ \eta)$ 
The polynomial 
\[
P_m(u,\xi)(\eta):=P_m(u(\xi \circ \cdot),\mathfrak{e})(\xi^{-1} \circ \eta),
\]
is the {\it Taylor polynomial} of $\h$-degree $m$ centered at~$\xi$ associated to~$u$.

\vspace{2mm}
\subsection{Boundary behaviour of the solutions to the subcritical CR~Yamabe problems}\label{sec_decay}  
    In view of the assumptions~($\Om1$)--($\Om4$), one can build fine subelliptic barriers as firstly seen in~\cite{GV00}.
    \begin{lemma}[See Theorem~4.3 in~\cite{GV00}]\label{barrier}
    	Let~$\Om$ be a smooth bounded domain of~$\h$ satisfying~{\rm(}$\Om1${\rm)}\textup{--}{\rm(}$\Om4${\rm)}. For any~$\alpha \in (0,1]$ define
    	\[
    	\varPsi_\alpha := (\rr_\Om-\varPhi)^\alpha e^{-\snr{z}^2/M_\Om}.
    	\]
    	Given a neighborhood~$K$ of the characteristic set~$\Car$ such that
    	\begin{equation}\label{cond_neighbor_char_set}
    		\overline{K} \subset \Big\{\snr{z}^2 \leq \frac{nM_\Om}{2}\Big\}, 
    	\end{equation}
    	we have that
    	\[
    	\Delta_H \varPsi_\alpha \, \leq \, -\frac{2n}{M_\Om} \varPsi_\alpha \, \  \textrm{on}~\omega:=\Om \cap K.
    	\]
    	Furthermore, there exist~$c_1,c_2 >0$ such that for any~$\eta_\lambda \in \partial \Om \cap K$ and any $\lambda \in [\lambda_{\rm o},1]$ it holds
    	\[
    	c_1 (1-\lambda)^\alpha \, \leq \, \varPsi_\alpha(\eta_\lambda) \, \leq \, c_2(1-\lambda)^\alpha.
    	\]
    \end{lemma}
    
    We now prove the main result of this section. We remark that we denote by~$K$ an open neighborhood  of the characteristic set~$\Car$ not containing the concentration point~$\xi_{\rm o}$ and with~$\omega := \Om \cap K$.
    \begin{theorem}\label{vassilev_type}
       Let~$\Om$ be a smooth bounded domain of~$\h$ satisfying~{\rm(}$\Om1${\rm)}\textup{--}{\rm(}$\Om4${\rm)}, and %for each~$0<\eps<2^*-2$ 
       let~$u_\eps \in \Sc(\Om)$ be a positive maximizer for~$\Ssub$. Then, there exists~$\bar{\eps} \equiv \bar{\eps}(n)>0$ such that, for any~$\eps < \bar{\eps}$,
        \begin{equation}\label{decaying_D_H_u_eps}
        	\int\limits_\gamma \snr{D_H u_\eps}^2 \langle \mathcal{D},\mathfrak{n}\rangle \, {\rm d}{\it H}^{Q-2} \leq c\sup_{\eta \in \partial \Om} \langle \mathcal{D},\mathfrak{n}\rangle (\eta) {\it H}^{Q-2}(\gamma),
        \end{equation}
       where~$\gamma$ is any hypersurface contained in~$\omega$ and~$c\equiv c(n)>0$.
    \end{theorem}

   \vspace{2mm}
   Let us remark that by the uniformly~$\delta_\lambda$-starlikeness of~$\Om$  along its characteristic set~$\Car$ in~($\Om4$), up to taking a smaller neighborhood~$K$, we get 
   \begin{equation}\label{trasversality}
	\mathcal{D}\varPhi (\eta) \geq c >0, \qquad \forall \eta \in \partial \Om \cap K,
    \end{equation}
   where~$\varPhi$ is the defining function of~$\Om$ in~($\Om1$) and~$\mathcal{D}$ is defined in~\eqref{dilation_vector_field}.

   Condition~\eqref{trasversality} implies that the trajectories of~$\mathcal{D}$ starting from~$\partial \Om \cap K$ fill a full open set interior of~$\Om$. Indeed, considering~$\eta \in \partial \Om \cap K$ and taking the Taylor expansion of~$f(\lambda):=\varPhi(\delta_\lambda(\eta))$ around~$\lambda =1$ we obtain that
   \[
   \rr_\Om-\varPhi(\delta_\lambda(\eta)) =\mathcal{D}\varPhi(\eta)(1-\lambda) + {\rm o}(1-\lambda) \geq c(1-\lambda).
   \]   
   Hence, with no loss of generality, up to further shrinking~$K$, we assume there exists~$\lambda_{\rm o}$ such that
    \begin{equation}\label{trasversality_2}
    \delta_{\lambda}(\eta) \in \Om \cap K, \qquad \textrm{for}~\lambda_{\rm o}<\lambda <1.
   \end{equation}
 
   \vspace{2mm} 
   \begin{proof}[\bf Proof of Theorem~{\rm\ref{vassilev_type}}]
   	Consider an open neighborhood~$K$ of the characteristic set~$\Car$ not containing~$\xi_{\rm o}$ and such that~\eqref{trasversality_2} holds true for any~$\eta \in \partial\Omega \cap K$.
   	
   	\vspace{1mm}
   	With the notation above, we prove that
   	 \begin{equation}\label{u_eps_lips_omega}
   		u_\eps(\delta_\lambda(\eta)) \leq c (1-\lambda) \ \, \textrm{for any}~\eta \in \partial \Om \cap K,
   \end{equation}
   for any~$\eps < \bar{\eps}(n)$; with~$\lambda$ as above.
    

   	Indeed, fix~$\eps \in (0,2^*-2)$. By Theorem~\ref{han} we get that there exists~$\bar{\eps} \equiv \bar{\eps}(n) >0$ such that
   \[
    	u_\eps \lesssim U_{\lambda_\eps,\eta_\eps}\ \, \textrm{on}~\omega :=\Om \cap   K,~\textrm{for any}~\eps < \bar{\eps}.
   \]
    Hence,~$u_\eps \in L^\infty(\omega)$,  since~$U_{\lambda_\eps,\eta_\eps}$ stays bounded in~$\omega$, giving that~$\xi_{\rm o} \not \in \omega$. As explained in the proof of Theorem~\ref{han}, form this we can deduce that~$u_\eps \in C^\infty(\omega)$ for any~$\eps<\bar{\eps}$.
    Moreover, using condition~($\Om2$) one can adapt to the present setting the classical Moser iteration argument to get~$u_\eps \in \Gamma^{0,\alpha_\eps}(\partial \Om \cap K)$, for some exponent~$1>\alpha_\eps >  0$ for any~$\eps < \bar{\eps}$, up to scaling down~$\bar{\eps}$. Thus,
    \[
    u_\eps \in \Gamma^{0,\alpha_\eps}(\overline{\omega})\cap C^\infty(\omega).
   \]
    Now, since~$u_\eps \equiv 0$ on~$\partial \Om$, we have that for any~$\eta \in \partial \Om$ 
    \[
    u_\eps(\eta_\lambda) \lesssim \snr{\eta_\lambda\circ \eta}_{\h}^{\alpha_\eps}\, \
   \text{for any}\ \eps < \bar{\eps}.
    \]
 %   up to scaling down~$\bar{eps}$.
    Since homogeneous norms can be controlled up to a constant~$c\equiv c(\Om)$ in the following way as shown in Lemma~4.1 in~\cite{GV00},
    \[
    \snr{\eta_\lambda\circ \eta}_{\h} \leq c(1-\lambda)^{1/2},
    \]
    where we denoted by~$\eta_\lambda:=\delta_{\lambda}(\eta)$,
    we obtain that
    \begin{equation}\label{hold_bound_u_eps}
    u_\eps(\eta_\lambda) \leq c (1-\lambda)^\frac{\alpha_\eps}{2} , \qquad\forall \eta \in \partial \Om \cap K, \, \forall \lambda \in [\lambda_{\rm o},1].
    \end{equation}
    Noting that, for any~$\eps < \min\big\{\bar{\eps},\, 4/(Q-2)\big\}$, we have that~$2^*-1-\eps >1$ we choose~$m \in \mathbb{N}$ such that~$(2^*-1-\eps )^{-m} \leq \alpha_{{\eps}}/2$. Since the estimate in~\eqref{hold_bound_u_eps} does hold for any~$\eta \in \partial \Om \cap K$ and any~$\lambda \in [\lambda_{\rm o},1]$, we can assume that the points~$\eta_\lambda$ cover~$\overline{\omega}$. Thus,
    \begin{equation}\label{refine_hld_u_eps}
     u_\eps \leq c(1-\lambda)^{(2^*-1-\eps )^{-m}} \ \,  \textrm{on}~\,\overline{\omega}\,,
    \end{equation}
    up to relabelling~$\bar{\eps}$ as above.
    Hence, up to taking a smaller neighborhood~$K$ such that~\eqref{cond_neighbor_char_set} is satisfied, we have that for~$\eta_\lambda$, by~\eqref{refine_hld_u_eps} and Lemma~\ref{barrier}, it holds
    \begin{eqnarray*}
    -\Delta_H u_\eps (\eta_\lambda) &=& u(\eta_\lambda)^{2^*-1-\eps}\\*[0.5ex]
                                    &=& c^{2^*-1-\eps}(1-\lambda)^{(2^*-1-\eps )^{1-m}}\\*[0.5ex]
                                    &\leq & c^{2^*-1-\eps}c_1^{-1} \varPsi_{(2^*-1-\eps )^{1-m}}(\eta_\lambda) \\*[0.5ex]
                                 &   \leq & -c^{2^*-1-\eps}c_1^{-1} \frac{M_\Om}{2n}\Delta_H \varPsi_{(2^*-1-\eps )^{1-m}}(\eta_\lambda)\\*[0.5ex]
                                    &\leq & - \Delta_H(c_{\rm o}\varPsi_{(2^*-1-\eps )^{1-m}})(\eta_\lambda).
    \end{eqnarray*} 
   Thus, 
   \begin{equation}\label{before_max_princ}
   	\Delta_H(c_{\rm o}\varPsi_{(2^*-1-\eps )^{1-m}}-u_\eps) \leq 0 \,\  \textrm{in}\,~\omega.
   \end{equation}
   Moreover, proceeding as in ~\cite[Theorem~5.14]{Vas06}, we get
   \begin{equation}\label{boundary_max_prin}
   	c_{\rm o}\varPsi_{(2^*-1-\eps )^{1-m}} \geq u_\eps \, \ \textrm{on}~\partial \omega.
   \end{equation}
   Combining together~\eqref{before_max_princ} with~\eqref{boundary_max_prin}, by Bony's Maximum Principle we obtain
   \[
   u_\eps(\eta_\lambda) \, \leq \,  c_{\rm o}\varPsi_{(2^*-1-\eps )^{1-m}}  \, \leq \,  c_{\rm o}c_2 (1-\lambda)^{(2^*-1-\eps )^{1-m}} \, \ \textrm{on}~\overline{\omega},
   \]
   which is a refinement of~\eqref{refine_hld_u_eps}. Iterating this procedure $m$-times yields
   \[
   		u_\eps(\delta_\lambda(\eta)) \leq c (1-\lambda) \ \, \textrm{for any}~\eta \in \partial \Om \cap K.
   \]
%   Notice that the constant appearing in~\eqref{u_eps_lips_omega} stays bounded when~$\eps \to 0^+$ as well as the exponent~$\alpha_\eps$. 
%   Finally, the estimate in~\eqref{u_eps_lips_omega} does hold for any~$0<\eps < \min\{\bar{\eps}_1,\bar{\eps}_2,\bar{\eps}_3\}=:\bar{\eps}$.
%  
 % \vspace{2mm}
  
  Fix now an hypersurface~$\gamma \subset \omega$ and cover it with a family~$\{B_\rr^{(i)}\}_{i=1}^\vartheta$ such that~$B_\rr(\xi_i)\equiv B_\rr^{(i)} \subset \omega$, for any~$i=1,\dots,\vartheta$.
  By the interior estimate of~\cite[Theorem~3.1]{BGM19}, we have that on every ball~$B_\rr^{(i)}$ it holds
   \begin{eqnarray}\label{interior_est}
   	\|D_H u_\eps\|_{L^\infty(B_\rr^{(i)})} &\leq & \frac{2c}{\rr}\Big(\|u_\eps\|_{L^\infty(\omega)}+ \rr^2\|u_\eps\|_{L^\infty(\omega)}^{2^*-1-\eps}\Big)\notag\\*[0.5ex]
   	&\leq & \frac{c}{\rr}\Big((1-\lambda)+\rr^2(1-\lambda)^{2^*-1-\eps}\Big),
   \end{eqnarray}
   by~\eqref{u_eps_lips_omega}, for any~$\lambda \in [\lambda_{\rm o},1]$. Note that the  estimate above is bounded for any ~$\lambda \in [\lambda_{\rm o},1]$ and for any value of~$\eps$.
   	Moreover,  by the $\delta_\lambda$-starlikeness we have that 
  \begin{equation}\label{fourth}
  0  \, < \, \int\limits_\gamma \langle \mathcal{D},\mathfrak{n}\rangle \, {\rm d}{\it H}^{Q-2} \,  \leq  \,   \sup_{\zeta \in \partial \Om} \langle \mathcal{D},\mathfrak{n}\rangle (\zeta) {\it H}^{Q-2}(\gamma).
  \end{equation}
  Thus, combining~\eqref{interior_est} and~\eqref{fourth} with a standard covering argument yields
  \begin{equation*}
	\int\limits_\gamma \snr{D_H u_\eps}^2 \langle \mathcal{D},\mathfrak{n}\rangle {\rm d}{\it H}^{Q-2}  \, \leq \,   c\sup_{\zeta \in \partial \Om} \langle \mathcal{D},\mathfrak{n}\rangle (\zeta) {\it H}^{Q-2}(\gamma) \, \ \text{for any} \ \eps<\bar\eps,
 \end{equation*}
  where~$c$ does not depend on~$\gamma$.
\end{proof}
\vs

We conclude this section by noticing that, in view of the hypotheses as in~($\Om1$)--($\Om4$), it readily follows the boundedness of the $\mathcal{D}$-derivatives using the same argument developed in~\cite{GV00}.

\begin{theorem}\label{vassilev_type_2}
 Let~$\Om$ be a smooth bounded domain of~$\h$ satisfying~{\rm(}$\Om1${\rm)}\textup{--}{\rm(}$\Om4${\rm)}, and %for each~$0<\eps<2^*-2$
  let~$u_\eps \in \Sc(\Om)$ be a positive maximizer for~$\Ssub$. Then, there exists~$\bar{\eps} \equiv \bar{\eps}(n)$ such that, for any~$\eps < \bar{\eps}$,
       \begin{equation}\label{bdd_Z_derivatives}
       	\mathcal{D}u_\eps \in L^\infty(\overline{\omega}),
       \end{equation}
with~$\omega:=\Om \cap K$, where~$K$ is an open neighborhood of the characteristic set~$\Car$ not containing~$\xi_{\rm o}$.
\end{theorem}
  
 

\subsection{Proof of the localization result}
We are finally in the position to present the proof of Theorem~\ref{thm_green}, whose argument involves different techniques and results such as the asymptotic control via the Jerison\,\&\,Lee optimal functions established in Theorem~\ref{han}, the Pohozaev identity, the regularity theory for the subcritical CR Yamabe equation in Theorem~\ref{vassilev_type_2} as well as the integral estimate in Theorem~\ref{vassilev_type}. Moreover, we will use the negligibility of the characteristic set, established in the result below.

\begin{theorem}[See Theorem~1.2 in \cite{Der71}]  \label{characteristic_set_negligible}
	Let~$\Om \subset \h$ be a~$C^\infty$ domain and let~$\Car$ be its characteristic set. Then, being~${\it H}^{Q-2}$ the $(Q-2)$-dimensional Hausdorff measure, it holds that
	\[
	{\it H}^{Q-2}(\Car)=0.
	\]
\end{theorem}
\vs\vs


\begin{proof}[\bf Proof of Theorem~{\rm\ref{thm_green}}]
 For the sake of readability, we divide the proof into several steps.

\vspace{2mm}
{\bf Step 1. The following limit holds true
\begin{equation}\label{han_lim_green}
	\|u_\eps\|_{L^\infty(\Omega)}u_\eps  \xrightarrow{\eps\to 0} \bar{c}\, G_{\Om}(\cdot,\xi_{\rm o}) \, \ \text{in}~\Gamma^{1,\beta}(\partial \Om \setminus \Car),
\end{equation}
for some~$\beta\in(0,1)$ and~$\bar{c}$ is given by
\[
\bar{c}:=\int\limits_{\h}U^\frac{Q+2}{Q-2} \, {\rm d}\xi.
\]
with~$U$ defined in~\eqref{talentiane_2}.
}

Define the function
\[
w_\eps := \|u_\eps\|_{L^\infty(\Om)}u_\eps^{2^*-1-\eps} \equiv \lambda_\eps^{-\frac{Q-2}{2}}u_\eps^{2^*-1-\eps}.
\]
We show that~$w_\eps \to \bar{c} \delta_0$, in the sense of distribution.
Clearly we have that
\begin{eqnarray}\label{han_lim_green2}
\int\limits_{\Om} w_\eps \, {\rm d}\xi &=& \lambda_\eps^{-\frac{Q-2}{2}}\int\limits_{\Om}u_\eps^{2^*-1-\eps} \, {\rm d}\xi \nonumber\\
&=& \lambda_\eps^{-\big(\frac{Q-2}{2}\big)^2\eps}\int\limits_{\Om_\eps}v_\eps^{2^*-1-\eps} \, {\rm d}\xi \to \int\limits_{\h}U^\frac{Q+2}{Q-2} \, {\rm d}\xi \equiv \bar{c},
\end{eqnarray}
where we have used the definition in~\ref{han_v_eps}, the fact that~$v_\eps \to U$ uniformly on compact set (by Ascoli-Arzel\`a's Theorem), % as also seen in the beginning of the proof of~Lemma~1 in~\cite{Han91} and ??? \cite{BP89},
 $v_\eps \to 0$ when $\snr{\xi}_{\h} \to \infty$ uniformly in $\eps$, $\Om_\eps \to \h$ and~ the asymptotic \eqref{conv_xi_eps} for $\lambda_\eps^\eps$. 
 

Moreover, note that when~$\xi \neq \xi_{\rm o}$, by~\eqref{bound_max_seq} of Theorem~\ref{han}, we get that
\begin{equation}\label{han_lim_green3}
w_\eps \lesssim \lambda_\eps^{-\frac{Q-2}{2}}U_{\lambda_\eps,\eta_\eps}^{2^*-1-\eps} \to 0.
\end{equation}
Thus, combining~\eqref{han_lim_green2} and~\eqref{han_lim_green3} we obtain that~$w_\eps \to \bar{c}\delta_{0}$ in the sense of distribution.

Furthermore, note that the function~$\|u_\eps\|_{L^\infty(\Om)}u_\eps$ is a solution to
\[
\begin{cases}
-\Delta_H \big(\|u_\eps\|_{L^\infty(\Om)}u_\eps\big) = \|u_\eps\|_{L^\infty(\Om)}u_\eps^{2^*-1-\eps} &\text{in}~\Om\cap \omega,\\*[1ex]
\|u_\eps\|_{L^\infty(\Om)}u_\eps =0 &\text{in}~\partial \Om \cap \omega,
\end{cases}
\]
where~$\omega$ is an interior neighborhood of~$\partial \Om \smallsetminus \Car$. 

%As already recalled before, thanks to hypothesis~($\Om2$), one can apply the classical iteration argument which ensures % that the existence of an exponent~$\bar{\eps}\equiv \bar{\eps}(n)>0$ such 
%that~$u_\eps$ belongs to $\Gamma^{0,\alpha_\eps}(\overline{\omega})$, 
%%for some $\alpha_\eps$ 
% for some~$1>\alpha_\eps \geq \alpha_{\bar{\eps}}
% >0$.%, for any~$\eps <\bar{\eps}$.

Consider a ball~$B_\rr$ such that~$B_\rr\cap (\partial \Om \setminus \Car)$ is not characteristic and choose~$r>0$ sufficiently small such that $B_\rr \cap \Om$ does not contain~$\xi_{\rm o}$. By~\eqref{han_lim_green3} since~$\|u_\eps\|_{L^\infty(\Om)}u_\eps =0$ on~$\partial \Om$, we have that the hypotheses in Theorem~\ref{boundary_hld} are satisfied. Thus, the sequence~$\{\|u_\eps\|_{L^\infty(\Om)}u_\eps \}$ is compact in~$\Gamma^{1,\beta}(\overline{B_\rr\cap \Om})$, for some~$\beta\in (0,1)$. Moreover, from~\eqref{han_lim_green2} and~\eqref{han_lim_green3} it has to converge to~$\bar{c} G_{\Om}(\cdot,\xi_{\rm o})$. Indeed, for any test function~$\phi \in C^\infty_0(B_\rr \cap \Om)$ it holds
\begin{eqnarray*}
-\int\limits_{B_\rr\cap \Om} \Delta_H\big(\|u_\eps\|_{L^\infty(\Om)}u_\eps\big) \phi \, {\rm d}\xi &=& \int\limits_{B_\rr\cap \Om} w_\eps \phi \, {\rm d}\xi\\*
&\to& \int\limits_{B_\rr\cap \Om} \bar{c}\delta_{0}\phi \, {\rm d}\xi\\*
&=& -\int\limits_{B_\rr\cap \Om} \bar{c}\Delta_H\big(G_{\Om}(\cdot,\xi_{\rm o})\big)\phi \, {\rm d}\xi.
\end{eqnarray*}
Hence,~\eqref{han_lim_green} follows taking an open cover of~$\partial \Om \setminus \Car$.


\vspace{2mm}
{\bf Step. 2 For any $\eps \in (0,\bar{\eps})$ it holds
\begin{equation}\label{pohozaev_sub_approx}
\frac{\eps(Q-2)}{2^*-\eps}\int\limits_{\Om} u_\eps^{2^*-\eps} \, {\rm d}\xi = \int\limits_{\partial \Om}\snr{D_H u_\eps}^2 \langle \mathcal{D},\mathfrak{n}\rangle \,{\rm d}{\it H}^{Q-2}.
\end{equation}
}
We start recalling that since $\Om$ is a smooth domain the characteristic set $\Car$ is compact. Moreover, thanks to Theorem \ref{characteristic_set_negligible} and Theorem \ref{boundary_hld}, we consider an exhaustion of $\Om$ of $C^\infty$ connected open sets $\{\Om_i\}$ such that $\Om_i \uparrow \Om$, $u_\eps \in C^2(\overline{\Om_i})$ and $\partial \Om_i = \gamma^{(1)}_i \cup \gamma^{(2)}_i$ with $\gamma^{(1)}_i \subset\partial \Om \smallsetminus \Car$, $\gamma^{(1)}_i \uparrow \partial \Om \smallsetminus \Car$ and ${\it H}^{Q-2}(\gamma^{(2)}_i) \to 0$.

We apply the Pohozaev identity on $u_\eps$ in $\Om_i$, getting
\begin{eqnarray}\label{loc_pohozaev_1}
	&& \int\limits_{\Om_i} \big(\frac{2Q}{2^*-\eps}u_\eps^{2^*-\eps}-(Q-2)u_\eps^{2^*-\eps}\big) \, {\rm d}\xi \notag\\*[0.5ex]
	&& \quad = 2 \sum_{j=1}^{2n}\int\limits_{\gamma^{(1)}_i \cup \gamma^{(2)}_i}\mathcal{D}u_\eps \, Z_j u_\eps \langle Z_j,\mathfrak{n}\rangle \,{\rm d}{\it H}^{Q-2} -\int\limits_{\gamma^{(1)}_i \cup \gamma^{(2)}_i}\snr{D_H u_\eps}^2 \langle \mathcal{D},\mathfrak{n}\rangle \,{\rm d}{\it H}^{Q-2}\notag\\*[0.5ex]
	&& \quad = 2 \sum_{j=1}^{2n}\int\limits_{\gamma^{(1)}_i }\mathcal{D}u_\eps \, Z_j u_\eps \langle Z_j,\mathfrak{n}\rangle \,{\rm d}{\it H}^{Q-2} -\int\limits_{\gamma^{(1)}_i}\snr{D_H u_\eps}^2 \langle \mathcal{D},\mathfrak{n}\rangle \,{\rm d}{\it H}^{Q-2}\notag\\*[0.5ex]\\*[-1.5ex]
	&&\qquad +\,2 \sum_{j=1}^{2n}\int\limits_{\gamma^{(2)}_i}\mathcal{D}u_\eps \, Z_j u_\eps \langle Z_j,\mathfrak{n}\rangle \,{\rm d}{\it H}^{Q-2} -\int\limits_{\gamma^{(2)}_i}\snr{D_H u_\eps}^2 \langle \mathcal{D},\mathfrak{n}\rangle \,{\rm d}{\it H}^{Q-2}.\notag
\end{eqnarray}
Note that since $u_\eps >0$ in $\Om_i$ and $u_\eps =0$ on $\gamma^{(1)}_i$, then there exists a function $w\leq 0$ such that $Du_\eps = w \mathfrak{n}$ on $\gamma^{(1)}_i$. Hence,
\begin{eqnarray*}
	\sum_{j=1}^{2n}\mathcal{D}u_\eps \, Z_j u_\eps \langle Z_j,\mathfrak{n}\rangle &=& w \langle \mathcal{D}, \mathfrak{n}\rangle \sum_{j=1}^{2n} Z_j u_\eps \langle Z_j,\mathfrak{n}\rangle \\*
	&=& \langle \mathcal{D}, \mathfrak{n}\rangle \sum_{j=1}^{2n} Z_j u_\eps \langle Z_j,w\mathfrak{n}\rangle\\*
	&=& \langle \mathcal{D}, \mathfrak{n}\rangle \sum_{j=1}^{2n} Z_j u_\eps \underbrace{\langle Z_j,Du_\eps\rangle}_{=:Z_j u_\eps}= \snr{D_H u_\eps}^2\langle \mathcal{D}, \mathfrak{n}\rangle.
\end{eqnarray*}
Then, putting the computation above inside \eqref{loc_pohozaev_1} yields
\begin{eqnarray*}
	\frac{\eps(Q-2)}{2^*-\eps}\int\limits_{\Om_i} u_\eps^{2^*-\eps} \, {\rm d}\xi &=& \int\limits_{\gamma^{(1)}_i}\snr{D_H u_\eps}^2 \langle \mathcal{D},\mathfrak{n}\rangle \,{\rm d}{\it H}^{Q-2}\notag\\
	&& +\,2 \sum_{j=1}^{2n}\int\limits_{\gamma^{(2)}_i}\mathcal{D}u_\eps \, Z_j u_\eps \langle Z_j,\mathfrak{n}\rangle \,{\rm d}{\it H}^{Q-2}\notag\\
	&&-\,\int\limits_{\gamma^{(2)}_i}\snr{D_H u_\eps}^2 \langle \mathcal{D},\mathfrak{n}\rangle \,{\rm d}{\it H}^{Q-2}.
\end{eqnarray*}
Now recalling Theorem \ref{vassilev_type} we have that
\[
\int\limits_{\gamma^{(2)}_i}\snr{D_H u_\eps}^2 \langle \mathcal{D},\mathfrak{n}\rangle \,{\rm d}{\it H}^{Q-2} \approx  {\it H}^{Q-2}(\gamma^{(2)}_i),
\]
 
and by Theorem \ref{vassilev_type_2} the $\mathcal{D}$-derivative of $u_\eps$ stays bounded, uniformly in $\eps$ near the characteristic set. Thus, passing to the limit as~$i\to \infty$, recalling the $\delta_\lambda$-starlikeness of $\Om$ and estimate \eqref{interior_est}, by the Dominate Convergence and the Monotone Convergence theorem we eventually arrive at the desired estimate~\eqref{pohozaev_sub_approx}.


\vspace{2mm}
{\bf Step 3. There exists a positive constant $c\equiv c(n,\Om)$  such that
	\begin{equation}\label{bounded_sup_norm}
		\eps \|u_\eps\|_{L^\infty(\Om)}^2 \leq c\lambda_\eps^{Q+6-2\eps(Q-2)}.
	\end{equation}}

We start proving that there exists a constant~$c\equiv c(n,\Om)>0$ such that	
\begin{equation}\label{decay_Du_eps}
    \int\limits_{\partial \Om} \snr{D_H u_\eps}^2 \langle \mathcal{D}, \mathfrak{n}\rangle \, {\rm d}{\it H}^{Q-2} \leq c\lambda_\eps^{2(Q+2-\eps(Q-2))}.
\end{equation}
Consider the Taylor polynomial~$P_1(u_\eps,\xi)$ of~$u_\eps$ with center in~$\xi \in \partial \Om \smallsetminus \Car$. As proven in~\cite[Formula~(4.55)]{BGM19} for any point~$\varsigma \in \Om$ such that
\[
\snr{\varsigma^{-1}\circ \xi}_{\h} \sim {\bar{\sigma}}, \quad {\rm dist}(\varsigma,\partial \Om) \sim {\bar{\sigma}},
\]
for~${\bar{\sigma}}>0$,  % since~$D_H P_1(u_\eps,\xi) \equiv D_H u_\eps (\xi)$,
it holds
\begin{eqnarray*}
\snr{D_H u_\eps(\xi)} &\leq & \snr{D_H u_\eps(\varsigma)-D_HP_1(u_\eps,\xi)} +\snr{D_H u_\eps(\varsigma)}\notag\\*[0.5ex]
&\leq & \frac{c}{{\bar{\sigma}}}\big( \|u_\eps-P_1(u_\eps,\xi)\|_{L^\infty(B_{a{\bar{\sigma}}}(\varsigma))} + {\bar{\sigma}}^2\|u_\eps\|_{L^\infty(B_{a{\bar{\sigma}}}(\varsigma))}^{2^*-1-\eps} \big) + \snr{D_H u_\eps(\varsigma)},
\end{eqnarray*}
where,  for~$a>0$,~$B_{a{\bar{\sigma}}}$ is a non-tangential ball from inside of~$\Om$. 

 Such a ball can be constructed since we are considering non-characteristic points. Indeed, up to left translations we assume that~$\xi \equiv \mathfrak{e}$. Moreover, since~$\mathfrak{e}$ is not characteristic there exists~$j \in \{1,\dots,2n\}$ such that
\[
\langle Z_i, \mathfrak{n}\rangle (\mathfrak{e}) \neq 0.
\]
Now, by an orthogonal transformation and the implicit function theorem we can assume the existence of~$\rr_0>0$ such that~$\Om \cap B_{\rr_0}$ can be represented as
\[
\Big\{x_n>\widetilde{\varPhi}(x',y,t)\Big\} \, \ \text{where}~x':=(x_1,\dots,x_{n-1}) \in \mathbb{R}^{n-1},
\]
for a Lipschitz function~$\widetilde{\varPhi}$ such that~$\widetilde{\varPhi}(\mathfrak{e})=0$,~$\nabla_{x'}\widetilde{\varPhi}(\mathfrak{e})=0$. In view of the Lipschitz continuity of the function~$\widetilde{\varPhi}$ we can assert that, up to taking~$\lambda$ sufficiently small, the ball~$B_{\lambda s} (\delta_\lambda(e_n,0))$ is strictly contained in~$\Om$, for~$s>0$  small enough. We also refer the reader to the proof of Proposition~3.3 in~\cite{BGM19} where a non-tangential ball from outside was determined, and to the proof of Theorem~7.6 in~\cite{DGP07}.
 \vspace{1mm}
 
Now, we take~${\bar{\sigma}}$ sufficiently small such that~$ \xi_{\rm o} \not \in B_{a{\bar{\sigma}}}$ and such that
\begin{equation}\label{726}
\frac{c}{{\bar{\sigma}}} \|u_\eps-P_1(u_\eps,\xi)\|_{L^\infty(B_{a{\bar{\sigma}}}(\varsigma))}\, \leq \, c {\bar{\sigma}}^\beta \, \leq \, \frac{1}{4} \snr{D_H u_\eps(\xi)}.
\end{equation}
Note that the first estimate in the display above comes from the proof of Theorem~1.1 in~\cite{BGM19}; see in particular Page~26  there.

Recalling now that, choosing~$\phi \in C^\infty_0(B_{\rr})$ with~$ \xi_{\rm o} \not \in B_\rr$ and~$B_\rr \subset \Omb$,  the concentration result of Theorem~\ref{cor_concentration} yields
\[
\lim_{\eps \to 0^+}\int\limits_{B_\rr}\phi \snr{D_H u_\eps}^2 \, {\rm d}\xi =0.
\]
Thus, up to taking a smaller ball~$B_\sigma \subset B_\rr$ and~$\phi$ such that~$\phi \equiv 1$ on~$B_\sigma$, we have that~$\|D_Hu_\eps\|_{L^2(B_\sigma)} \to 0^+$ as~$\eps \to 0^+$. Hence, up to subsequences, we have that~$\snr{D_H u_\eps(\varsigma)}\to 0^+$ as~$\eps \to 0^+$, for a.~\!e.~$\varsigma \neq \xi_{\rm o}$. Then, take~$\eps$ sufficiently small such that, still up to subsequences,
\begin{equation}\label{777}
\snr{D_H u_\eps(\varsigma)} \leq \frac{1}{4}\snr{D_H u_\eps(\xi)} \, \ \text{for a.~\!e.}~\varsigma \neq \xi_{\rm o}\,.
\end{equation}

All in all, by inserting~\eqref{726} and \eqref{777} in~\eqref{decay_Du_eps} and recalling the interior asymptotic estimate~\eqref{bound_max_seq}, we get that for~$\eps$ sufficiently small
\[
	\snr{D_H u_\eps(\xi)} \, \leq \, c \|u_\eps\|_{L^\infty(B_{a{\bar{\sigma}}}(\varsigma))}^{2^*-1-\eps}\, \leq \, c \lambda_\eps^{Q+2-\eps(Q-2)} \, \ \text{for a.~\!e.}~\xi \in \partial \Om \smallsetminus \Car,
\]
for some~$c>0$ bounded uniformly as~$\eps \to 0^+$, since~$\xi_{\rm o} \not \in B_{a{\bar{\sigma}}}(\varsigma)$, which implies the desired estimate in~\eqref{decay_Du_eps}, recalling that
\[
\int\limits_{\partial \Om} \langle \mathcal{D},\mathfrak{n}\rangle \, {\rm d}{\it H}^{Q-2} \leq \sup_{\zeta \in \partial \Om} \langle \mathcal{D},\mathfrak{n}\rangle(\zeta) {\it H}^{Q-2}(\partial \Om),
\]
with~$\sup_{\zeta \in \partial \Om} \langle \mathcal{D},\mathfrak{n}\rangle(\zeta)>0$ by  the $\delta_\lambda$-starlikeness of~$(\Om4)$.

Thus, in view of~\eqref{eq_han2}, multiplying the Pohozaev identity~\eqref{pohozaev_sub_approx} by~$\|u_\eps\|_{L^\infty(\Om)}^2 \equiv \lambda_\eps^{2-Q}$ yields
\begin{eqnarray*}
&& \frac{\eps \|u_\eps\|_{L^\infty(\Om)}^2(Q-2)}{2^*-\eps}\int\limits_\Om u_\eps^{2^*-\eps}\, {\rm d}\xi \\*[0.5ex]
&& \qquad \qquad \qquad \qquad = \|u_\eps\|_{L^\infty(\Om)}^2\int\limits_{\partial \Om}	\snr{D_H u_\eps}^2 \langle \mathcal{D},\mathfrak{n}\rangle \, {\rm d}{\it H}^{Q-2}\\*[0.7ex]
&&  \qquad \qquad \qquad\qquad \leq c \lambda_\eps^{2(Q+2-\eps(Q-2)) +2-Q},
\end{eqnarray*}
from which~\eqref{bounded_sup_norm} follows. 


\vspace{2mm}
{\bf Step 4. We prove that \eqref{robin_condition} holds true.
     } 

    Multiplying~\eqref{pohozaev_sub_approx} on both sides by~$\|u_\eps\|_{L^\infty(\Om)}^2$ gives
    \begin{eqnarray*}   
        &&\frac{\eps \|u\|_{L^\infty(\Om)}^2(Q-2)}{2^*-\eps} \int\limits_{\Om}u_\eps^{2^*-\eps}\,{\rm d}\xi\notag\\*[0.5ex]
        && \qquad \qquad\qquad = \int\limits_{\partial \Om} \big|{D_H \|u_\eps\|_{L^\infty(\Om)}u_\eps}\big|^2 \langle \mathcal{D},\mathfrak{n}\rangle \, {\rm d}{\it H}^{Q-2}.
    \end{eqnarray*}
    Passing to the limit as~$\eps \to 0^+$ gives the desired result~\eqref{robin_condition} recalling~\eqref{han_lim_green} and~\eqref{bounded_sup_norm}.
\end{proof}


%%%
%
% CASO NON CARATTERISTICO
%
%%%
\subsection{The non-characteristic case}
In this section we  
state the analogous of Theorem~\ref{thm_green} in the case when the set~$\Om$ has an empty characteristic set~$\Car$. 

\vspace{2mm}
In this framework the hypothesis~{\rm(}$\Om1${\rm)}\textup{--}{\rm(}$\Om4${\rm)} are not required, except for condition~($\Om2$) in order to apply the Folland-Stein boundary regularity yielding the~$\Gamma^{0,\alpha}$-regularity up to~$\partial \Om$.   The remaining part of the proof of Theorem~\ref{thm_green} is relatively simpler. Indeed, the approximation argument used in Step.~2 in order to establish the relation~\eqref{pohozaev_sub_approx} as well as the integral estimate in~\eqref{decaying_D_H_u_eps} and the boundedness of the $\mathcal{D}$-derivatives in~\eqref{bdd_Z_derivatives} are not needed. The boundary Schauder estimate due to Jerison of Theorem~\ref{jerison_schauder_boundary} ensures, via iteration, the smoothness of the maximizer~$u_\eps$ up to the boundary of~$\Om$ for any choice of the parameter~$\eps$. So that the Pohozaev identity of  Lemma~\ref{pohozaev} can be applied in turn yielding the desired~\eqref{pohozaev_sub_approx}. The rest of the proof will follow in a similar fashion as in the characteristic case without considering the presence of the characteristic set~$\Car$; especially in Step.~1 for the convergence~\eqref{han_lim_green} of the (rescaled) maximizers~$\|u_\eps\|_{L^\infty(\Om)}u_\eps$ to the Green function~${G}_\Om(\cdot,\xi_{\rm o})$.
\vs

Thus, Theorem~\ref{thm_green} can be re-stated in the following simpler way,
\begin{theorem}
	Let~$\Om$ be a smooth bounded domain of~$\h$ such that~$\Car \equiv \emptyset$ and for  each~$0<\eps< 2^*-2$ let~$u_\eps \in \Sc(\Om)$  be a maximizer for~$\Sob_\eps$. Then, up to subsequences, $u_\eps$~concentrates at some point~$\xi_{\rm o}\in\Omega$ such that
	\[
		\int\limits_{\partial \Om} \snr{D_H G_{\Om}(\cdot,\xi_{\rm o})}^2 \langle\mathcal{D},\mathfrak{n}\rangle \, {\rm d}{\it H}^{Q-2}=0,
	\]
	with~$G_{\Om}(\cdot;\xi_{\rm o})$ being the Green function related to~$\Om$ with pole in~$\xi_{\rm o}$ and $\mathcal{D}$ being the infinitesimal generator of the one-parameter group of non-isotropic dilations in the Heisenberg group.
\end{theorem}

\vspace{2mm}
Note that a first class of non-characteristic sets where the result presented above finds its application is torus obtained by revolting the sphere~$\mathbb{S}^{2n}$ around the~$t$-axis in~$\h$.
Moreover, some relevant results have been obtained for maximizing sequence of~$\Ssub$ in non-characteristic domain. Indeed, in~\cite{MMP13}, the authors are able to construct a concentrating sequence of solutions for certain non-degenerate critical point of  the regular part of the Green function of~$\Om$. We also refer  to the references in the aforementioned paper.

\vspace{2mm}
%\section{Appendix}


	
\vspace{2mm}

\begin{thebibliography}{99}

\setlinespacing{0.83}
		
\bibitem{AG03} {\sc M. Amar, A. Garroni}: {$\Gamma$-convergence of concentration problems}. {\it Ann. Scuola Norm. Sup. Pisa Cl. Sci.} {\bf 2}~(2003), 151--179.
\vs

\bibitem{BC88} {\sc A.~Bahri, J.~\!-M. Coron}: On a nonlinear elliptic equation involving the critical Sobolev exponent: the effect of the topology of the domain. {\it Comm. Pure Appl. Math.} {\bf 41} (1988), no. 3, 253--294.
\vs

      \bibitem{BGM19} {\sc A. Banerjee, N. Garofalo, I.~\!H.~Munive}: Compactness methods for $\Gamma^{1,\alpha}$ boundary Schauder estimates in Carnot groups. {\it Calc. Var. Partial Differential Equations} {\bf 58} (2019), no. 3, 29--97.
     \vs
  
    
\bibitem{Ben08}{\sc J.~Benameur}: {Description du d\'efaut de compacit\'e de l'injection de Sobolev sur le groupe de Heisenberg}. {\it Bull. Belg. Math. Soc. Simon Stevin}~{\bf 15} (2008), no.~4, 599--624.
\vs 

		
		\bibitem{BP02} {\sc I. Birindelli, J. Prajapat}: Monotonicity and symmetry results for degenerate elliptic equations on nilpotent Lie groups. {\it Pacific J. Math.} {\bf 204} (2002), no. 1, 1--17. 
		\vs
		

\bibitem{BP89}{\sc H. Brezis, L.~\!A. Peletier}: Asymptotic for Elliptic Equations involving critical growth. In {\it Partial Differential Equations and the Calculus of Variations. Essays in Honor of Ennio De Giorgi, Vol. 1}, Progr. Differ. Equ. Appl., Birkh\"auser, Boston (1989), 149--192.
\vs

 

\bibitem{CGN02} {\sc L. Capogna, N. Garofalo, D.-M.~Nhieu}: Properties of Harmonic Measures in the Dirichlet Problem for Nilpotent Lie Groups of Heisenberg Type. {\it Amer. J. Math.}~{\bf 124} (2002), no.~2, 273--306.
\vspace{-2.5mm}



 

\bibitem{CU01}{\sc G. Citti, F. Uguzzoni}: Critical semilinear equations on the Heisenberg group: the effect of the topology of the domain. {\it Nonlinear Anal.}~{\bf 46}~(2001), 399--417.
\vs



 
 
     \bibitem{DGP07} {\sc D. Danielli, N. Garofalo, A. Petrosyan}: The sub-elliptic obstacle problem: $C^{1,\alpha}$ regularity of the free boundary in Carnot groups of step two. {\it Adv. Math.} {\bf 211} (2007), no. 2, 485--516.
    \vs
    

   	\bibitem{Der71} {\sc M. Derridj}: Un probl\'eme aux limites pour une classe d'op\'erateurs du second ordre hypoelliptiques. (French) {\it Ann. Inst. Fourier (Grenoble)} {\bf21} (1971), no. 4, 99--148.
   	\vs
  
 

   \bibitem{Fol75}{\sc G.~\!B.~Folland}: Subelliptic estimates and function spaces on nilpotent Lie groups. {\it Ark. Math.} {\bf 13} (1975), 161--207.
   \vs 
   
 

 \bibitem{FS74}{\sc   G.~\!B.~Folland, E.~\!M. Stein}: Estimates for the $\bar\partial_b$ complex and analysis on the Heisenberg group. {\it Comm. Pure Appl. Math.}~{\bf 27} (1974), 429--522.
 \vs
	
 

\bibitem{FGM02}{\sc M. Flucher, A. Garroni, S. M\"uller}: Concentration of low energy extremals: Identification of concentration points. {\it Calc. Var. Partial Differential Equations}~{\bf 14} (2002), 483--516.
\vs

\bibitem{FM99}{\sc M. Flucher, S. M\"uller}: Concentration of low energy extremals. {\it Ann. Inst. H. Poincar\'e - Anal. Non Lin\'eeaire}~{\bf 16} (1999), no.~3, 269--298.
\vs

 
 \bibitem{GL92}{\sc N. Garofalo, E. Lanconelli}: Existence and Nonexistence Results for Semilinear Equations on the Heisenberg Group. {\it Indiana Univ. Math. J.} {\bf 41}~(1992), no.~1, 71--98.
 
\vs


 


  \bibitem{GV00}{\sc N. Garofalo, D. Vassilev}: Regularity near the characteristic set
in the non-linear Dirichlet problem
and conformal geometry of sub-Laplacians on Carnot groups. {\it Math. Ann.} {\bf 318}~(2000), 453--516.
\vs

  \bibitem{GV01}{\sc N. Garofalo, D. Vassilev}: Symmetry properties of positive entire solutions of Yamabe-type equations on groups of Heisenberg type. {\it Duke Math. J.} {\bf 106}~(2001), 411--448.
\vs



 

  \bibitem{Han91} {\sc Z.-C. Han}: {Asymptotic approach to singular solutions for nonlinear elliptic equations involving critical Sobolev exponent}. {\it Ann. Inst. Henri Poincar\'e Anal. Non Lin\'eaire} {\bf 8}~(1991),  159--174.
  	\vs	
 
\bibitem{IV11}{\sc S.~\!P. Ivanov, D.~\!B. Vassilev}: {\it Extremals for the Sobolev Inequality and the Quaternionic Contact Yamabe Problem}. World Scientific Publishing, Singapore, 2011.

\vs

\bibitem{Jer81} {\sc D. Jerison}: The Dirichlet problem for the Kohn Laplacian on the Heisenberg group. I. {\it J. Functional Analysis} {\bf43} (1981), no. 1, 97--142.
\vs

	\bibitem{Jer81-2} {\sc D. Jerison}: The Dirichlet problem for the Kohn Laplacian on the Heisenberg group. II. {\it J. Functional Analysis} {\bf43}  (1981), no. 2, 224--257.
\vs




\bibitem{JL88} {\sc D. Jerison, J.~\!M. Lee}: Extremals for the Sobolev inequality on the Heisenberg group and the~CR~Yamabe problem. {\it J. Amer. Math. Soc.}~{\bf 1}~(1988), no.~1, 1--13.
\vs


\bibitem{LU98} {\sc E. Lanconelli, F. Uguzzoni}:
Asymptotic behavior and non-existence theorems for semilinear Dirichlet problems involving critical exponent on unbounded domains of the Heisenberg group.
{\it Boll. UMI (Serie 8)} {\bf 1-B}~(1998), no.~1, 139--168.
\vs


 

\bibitem{MMP13} {\sc A. Maalaoui, V. Martino, A. Pistoia}: Concentrating solutions for a sub-critical sub-elliptic problem. {\it Diff. Int. Eq.}~{\bf 26}~(2013), no. 11-12, 1263--1274.
\vs

 

		\bibitem{PPT23} {\sc G. Palatucci, M. Piccinini, L. Temperini}:
		Struwe's Global Compactness and energy approximation of the critical Sobolev embedding in the Heisenberg group.	{\it Preprint} (2023). 
		\vs

		\bibitem{PP23} {\sc G. Palatucci, M. Piccinini, L. Temperini}:
		{Global Compactness, subcritical approximation of the Sobolev quotient, and a related concentration result in the Heisenberg group}.
In ``Research Perspectives Ghent Analysis and PDE Center''. 
		{\it Trends in Mathematics},  Birkh\"auser,  
		2024.
		\vs


\bibitem{PP15} {\sc G. Palatucci, A. Pisante}: A Global Compactness type result for Palais-Smale sequences in fractional Sobolev spaces.
{\it Nonlinear Anal.} {\bf 117}~(2015), 1--7.
\vs


\bibitem{PR03} {\sc   A. Pistoia,  O. Rey}: {Boundary blow-up
  for a Brezis-Peletier problem on a singular domain}.  {\it Calc. Var. Partial Differential Equations}~{\bf 18} (2003), no.~3, 243--251.
\vs
 


		\bibitem{Rey89} {\sc O. Rey}: {Proof of the conjecture of H.~Brezis and L.~\!A.~Peletier}. {\it Manuscripta math.} {\bf 65} (1989),  19--37. 
					\vs

      
      \bibitem{Vas06} {\sc D. Vassilev}: Existence of solutions and regularity near the characteristic boundary for sub-Laplacian equations on Carnot groups. {\it Pacific J. Math.} {\bf 227} (2006), no. 2, 361--397. 
      \vs


\end{thebibliography}

 \vspace{3mm}

\end{document} 