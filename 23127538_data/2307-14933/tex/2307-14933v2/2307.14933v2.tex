%%-----------------------------------------------------%%
 %%%%% DOCUMENT CLASS %%%%%
\documentclass[10pt,a4paper]{amsart} 


%%%%% PACKAGES %%%%%
\usepackage{lmodern}
\usepackage{textcomp}
\usepackage{amssymb}
\usepackage{amsmath} 
\usepackage{amsfonts}
\usepackage{amsthm}
\usepackage{dsfont}
\usepackage{amstext}
\usepackage{amscd}
\usepackage{microtype}
\usepackage[utf8]{inputenc}
\usepackage{enumerate}
\usepackage{esint} 
\usepackage{epsfig,graphics}
\usepackage{cite}
\usepackage{bbm,bm}
\usepackage{comment} 
\usepackage{url} 
\usepackage{tikz}
\usepackage{mathrsfs}
\usepackage[T1]{fontenc}
\usepackage{cancel}
\usepackage{latexsym} 
\usepackage{longtable,booktabs,setspace} 
\usepackage{url}
\usepackage{bbm,bm}
\usepackage{appendix}

%% Define theorem-like environments as usual: 
\newtheorem{theorem}{Theorem}[section]
\newtheorem{prop}[]{Proposition}[section]
\newtheorem{proposition}{Proposition}[section]
\newtheorem{lemma}{Lemma}[section]
\newtheorem{cor}{Corollary}[section]
\newtheorem{corollary}{Corollary}[section]
\newtheorem{defn}{Definition}[section]
\newtheorem{rem}{Remark}[section]
\newtheorem{example}{Example}
\newtheorem{problem}{Problem}
\numberwithin{equation}{section}


% Load the dutchcal font
\DeclareMathAlphabet{\mathdutchcal}{U}{dutchcal}{m}{n}

\usepackage{color}
\definecolor{citation}{rgb}{0.16, 0.5, 0.0}
\definecolor{formula}{rgb}{0.8, 0.25, 0.33}
\definecolor{emph}{rgb}{0.24,0.35,0.63}
\definecolor{url}{rgb}{0.5, 0, 0.8}
\usepackage[colorlinks=true,linkcolor=formula,urlcolor=url,citecolor=citation]{hyperref} 

%\usepackage{refcheck}

%%%%%%%%%%%%%%
%%%%% Custom short %%%%%

\def\loc{\operatorname{loc}}
\setlength{\delimitershortfall}{-0.1pt}
\allowdisplaybreaks[4]

% <--- Caratteri

% Save the default \mathcal
\let\mathcaldefault\mathcal



% <--- Comandi

%%% DEF
\def\dys{\displaystyle}
\def\vs{\vspace{1.2mm}}
\def\dd{\snr{\,\cdot\,}_{\mathbb{H}}}
\def \eps{{\varepsilon}}
\def \r{{\mathbb{R}}}
\def \h{{\mathbb{H}^n}}
\def \N{{\mathbb{N}}}
\def\gamp{\Gamma^{+}}
\def\ue{u_{\eps}}
\def\mue{\mu_{\eps}}
\def\mea{\mathcal{M}(\overline{\Omega})}
\def\Xint#1{\mathchoice
	{\XXint\displaystyle\textstyle{#1}}%
	{\XXint\textstyle\scriptstyle{#1}}%
	{\XXint\scriptstyle\scriptscriptstyle{#1}}%
	{\XXint\scriptscriptstyle\scriptscriptstyle{#1}} %
	\!\int}
\def\XXint#1#2#3{{\setbox0=\hbox{$#1{#2#3}{\int}$}
		\vcenter{\hbox{$#2#3$}}\kern-.5\wd0}}
\def\ddashint{\Xint=}
\def\dashint{\Xint-}
\def\ap{``}



%%%% renew
\renewcommand{\Delta}{\triangle}
\renewcommand{\Sigma}{\varSigma}
\renewcommand{\Lambda}{\varLambda}
\renewcommand{\Psi}{\varPsi}
\renewcommand{\Gamma}{\varGamma}
\renewcommand{\epsilon}{\varepsilon}
\renewcommand{\geq}{\geqslant}
\renewcommand{\leq}{\leqslant}


%%%% new command
\newcommand{\twe}{T{w}_{\eps}}
\newcommand{\uer}{u_{\eps,r}}
\newcommand{\tuer}{T{u}_{\eps,r}}
\newcommand{\bu}{\bar{u}_{\eps}}
\newcommand{\mbu}{\bar{\mu}_{\eps}}
\newcommand{\El}{\mathcal{E}_\lambda}
\newcommand{\Es}{\mathcal{E}^*}
\newcommand{\Sc}{{S}^1_0}
\newcommand{\Xc}{{X}}
\newcommand{\Car}{\Sigma(\Omega)}
\newcommand{\Ssub}{\Sob_\eps}
\newcommand{\Som}{\Sob_\Om}
\newcommand{\Sob}{S^\ast}
\newcommand{\uz}{u^{(0)}}
\newcommand{\lambdakj}{{\lambda_k^{(j)}}}
\newcommand{\xikj}{{\xi_k^{(j)}}}
\newcommand{\ukj}{{u_k^{(j)}}}
\newcommand{\rkj}{{r_k^{(j)}}}
\newcommand{\uj}{{u^{(j)}}}
\newcommand{\rr}{\rho}
\newcommand{\snr}[1]{\lvert #1\rvert}
\newcommand{\Dh}{D_{\mathbb{H}}}


% <--- Convergenze
\newcommand{\tows}{\stackrel{\ast}{\rightharpoonup}}
\newcommand{\towsloc}{{\stackrel{\ast}{\rightharpoonup}}_{loc}}
\newcommand{\tow}{\rightharpoonup}
\newcommand{\towt}{\stackrel{\mathpzc{t}}{\rightarrow}}
%


% <--- Insiemi
\newcommand{\Om}{\Omega}
\newcommand{\Omb}{\overline{\Omega}}
\newcommand{\Irm}{\text{I}}



\newlength{\defbaselineskip}
\setlength{\defbaselineskip}{\baselineskip}
\newcommand{\setlinespacing}[1]
{\setlength{\baselineskip}{#1 \defbaselineskip}}


\hyphenation{par-ma hei-sen-berg  pa-la-tuc-ci pic-ci-ni-ni} 






\begin{document}
		%%%% TITOLO & AUTORI
	
	\title[Critical Sobolev embedding in the Heisenberg group]{Asymptotic approach to singular solutions\\ for the CR Yamabe equation}
	
	\author[G. Palatucci]{Giampiero Palatucci}  
	\address{Giampiero Palatucci\\
		Dipartimento di Scienze Matematiche, Fisiche e Informatiche, Universit\`a di Parma\\ 
		Parco Area delle Scienze 53/a, Campus, 43124 Parma, Italy} 
	\email[Giampiero Palatucci]{\url{giampiero.palatucci@unipr.it}}
	
	
	\author[M. Piccinini]{Mirco Piccinini}  
	\address{Mirco Piccinini\\
		Dipartimento di Scienze Matematiche, Fisiche e Informatiche, Universit\`a di Parma\\
		Parco Area delle Scienze 53/a, Campus, 43124 Parma, Italy, \ and \ Dipartimento di Matematica, Universit\`a di Pisa, ~L.go~B.~Pontecorvo~5, 56127, Pisa, Italy}
	\email[Mirco Piccinini]{\url{mirco.piccinini@dm.unipi.it}}
	
	
	%%% MSC 2020
	\makeatletter
	\@namedef{subjclassname@2020}{\textup{2020} Mathematics Subject Classification}
	\makeatother
	\subjclass[2020]{35R03, 46E35, 35B33, 35J08, 35A15\vspace{1mm}}
	
	%%% KEYWORDS
	\keywords{Sobolev embeddings,  Heisenberg group, CR Yamabe, Green's Function\vspace{1mm}}
	
	%%%%% SPECIAL THANKS
	\thanks{{\it Aknowledgements.}
		%	The authors are supported by INdAM Project \!\char`_
		%	E55F22000270001.
		The authors are supported by INdAM projects  \ap Problemi non locali: teoria cinetica e non uniforme ellitticit\`a'', CUP\_E53C220019320001, and \ap Problemi ellittici e sub-ellittici: singolarit\`a e crescita critica'', CUP\_E53C23001670001.~The second author is also supported by the project \ap Local vs Nonlocal: mixed type operators and nonuniform ellipticity'', CUP\_D91B21005370003. 
		The results in this paper have been announced in the preliminary research report~\cite{PP23}}
	
	
	


\begin{abstract}		\vspace{1mm}
		 We investigate some effects of the lack of compactness in the critical Sobolev embedding
       by proving that a famous conjecture of Brezis and Peletier~\cite{BP89} does still hold in the Heisenberg framework: optimal functions for a natural subcritical approximations of the Sobolev quotient concentrate energy at exactly one point which is a critical point of the Robin function (i.~\!e., the diagonal of the regular part of the Green function  associated to the involved domain), in clear accordance with the underlying sub-Riemannian geometry. Consequently,
       a new suitable definition of domains {\it geometrical regular near their characteristic set} is introduced.
   In order to achieve the aforementioned result, we need to combine  proper  estimates and  tools to attack the related CR~Yamabe equation with novel feasible ingredients in PDEs and Calculus of Variations which also aim to constitute general independent results in the Heisenberg framework, as for instance a fine asymptotic control of the optimal functions via the Jerison and Lee extremals realizing the equality in the critical Sobolev inequality~\cite{JL88}.
\end{abstract}
\maketitle


\setlinespacing{1.061} 

   \section{Introduction}
  Let  $\h := (\mathbb{C}^n\times\mathbb{R},\circ, \{\delta_\lambda\}_{\lambda>0})$ be the usual Heisenberg-Weyl group and
define the standard Folland-Stein-Sobolev space~$\Sc(\h)$ as the completion of~$C^\infty_0(\h)$ with respect to the horizontal gradient norm~$\|\Dh \cdot\|_{L^2(\h)}$.  In~\cite{FS74} the authors prove  that the following Sobolev-type inequality does hold, 
\begin{equation}\label{folland}
\|u\|^{2^\ast}_{L^{2^\ast}(\h)} \leq \Sob \|\Dh u\|^{2^\ast}_{L^2(\h)} \qquad \forall u \in S^1_0(\h)\,,
\end{equation}
where~$\Sob$ is a positive constant, and~$2^\ast :=2Q/(Q-2)$ stands for the Folland-Stein-Sobolev critical exponent, which depends on the {\it homogeneous dimension}~$Q:=2n+2$ associated to the group of  dilations~$\{\delta_\lambda\}_{\lambda>0}$.
\vspace{2mm}


The critical Sobolev inequality~\eqref{folland} has been an attractive object of study for the last decades, since it is inextricably linked to the lack of compactness of the related critical (Folland-Stein-)Sobolev embedding, and to the correspondent Euler-Lagrange equation in turn describing the important CR Yamabe problem, namely, the problem of finding, for  a compact CR manifold $\mathbb{M}$ of real dimension $2n+1$  a choice of contact form in the conformal class of  its  equipped contact form $\Theta$ for which the Webster curvature $\mathfrak{R}$ is constant.
{If we denote with $\Theta^*=u^\frac{4}{Q-2}\Theta$  a conformal change of $\Theta$ one gets for the corresponding scalar curvature that
\[
\mathfrak{R}^* = u^{-\frac{Q+2}{Q-2}}\big(2^* \Delta_\mathbb{H}u +\mathfrak{R}u\big).
\]
It is then clear that in the flat case when~$\mathfrak{R}=0$ the PDE associated with the CR Yamabe problem is actually the one obtained in considering the variational problem related to finding the best Sobolev constant in \eqref{folland}; see forthcoming equation \eqref{equazione}.} A breakthrough results on the CR Yamabe problem is due to Jerison and Lee in~\cite{JL87}. Also, very recently, the case when $n=1$ has been solved by Gamara in~\cite{Gam01}.
\vspace{2mm}

	For what concern the strictly related study of the optimality of \eqref{folland}, several results in accordance with the classical critical inequality in the Euclidean framework have been proven, despite the difficulties given by the sub-Riemannian geometry of the Heisenberg {group $\h$}. On the contrary, several (somewhat expected) results are still open for the same reason; that is, the substantial difference with respect to the Euclidean framework in view of the complex {non-commutative} structure. 
The literature is too wide to attempt any comprehensive treatment in a single paper. We refer the interested readers to the very important papers~\cite{JL88,GL92,LU98,GV00,CU01,Loi05}, the recent book~\cite{IV11}, and the references therein.

\vspace{2mm}

In the present paper, we are {interested} into investigating some of the effects of the lack of compactness in the critical Sobolev embedding~\eqref{folland}, by analyzing the asymptotic behavior of the natural subcritical approximation of the Sobolev quotient.

\vspace{2mm}
Consider the following maximization problem,
\begin{equation}\label{critica0}
\Sob:=\sup\left\{\,\int_{\h}|u(\xi)|^{2^\ast}\,{\rm d}\xi \, : \, u\in \Sc(\h), \int_{\h}|\Dh u(\xi)|^2{\rm d}\xi \leq 1\right\}.
\end{equation}
The validity of~\eqref{folland} is equivalent to show that the constant~$\Sob$ defined in the display above is finite. The existence of the maximizers in~\eqref{critica0} is a difficult problem because of the intrinsic dilations and translations invariance of such inequality, as it analogously happens for the classical critical Sobolev inequality. The situation here is even more delicate because of the underlying non-Euclidean geometry of the Heisenberg group, and the  obstacles due to the related non-commutativity. The explicit form of the maximizers has been presented, amongst other results, in the breakthrough paper by Jerison and Lee~\cite{JL88}, together with the computation of the optimal constant in~\eqref{critica0}.
We also refer to the fundamental paper~\cite{FL12} where sharp constants for inequalities on~$\h$ have been derived for even more general cases, in turn obtaining sharp constants for the corresponding duals, which are the Sobolev inequalities for the sub-Laplacian and the conformal fractional Laplacians. 
\vspace{2mm}

For any bounded domain~$\Omega\subset\h$, consider now the following Sobolev embedding in the same variational form as the one in~\eqref{critica0},
\begin{equation}\label{critica}
\Som:=\sup\left\{\,\int_{\Omega}|u(\xi)|^{2^\ast}\,{\rm d}\xi \, : \, u\in \Sc(\Omega), \int_{\Omega}|\Dh u(\xi)|^2{\rm d}\xi \leq 1\right\},
\end{equation}
where the Folland-Stein-Sobolev space $\Sc(\Omega)$ is given as the closure of $C^\infty_0(\Omega)$ with respect to the $L^2$-{norm of the horizontal gradient} in $\Omega$.

\vspace{2mm}
One can check that $\Som\equiv \Sob$ via a standard
scaling argument on compactly supported smooth functions. For this, in view of the explicit form of the optimal functions in~\eqref{critica0} -- see forthcoming~Theorem~\ref{thm_optimal} -- the variational problem~\eqref{critica} has no maximizers. The situation changes considerably for the subcritical embeddings. Indeed, since $\Omega$ is bounded,  the embedding $\Sc(\Omega) \hookrightarrow L^{2^\ast-\eps}(\Omega)$ is compact (for any $0 <\eps<2^\ast-2$), and this does guarantee the existence of a maximizer~$u_\eps\in \Sc(\Omega)$ for the related variational problem
   \begin{equation}\label{sobolev}
\Sob_\eps:=\sup\left\{\,\int_{\Omega}|u(\xi)|^{2^\ast-\eps}\,{\rm d}\xi \, : \, u\in \Sc(\Omega), \int_{\Omega}|\Dh u(\xi)|^2{\rm d}\xi \leq 1\right\}.
\end{equation}

Such a dichotomy is evident in the Euler-Lagrange equation for the energy functionals in~\eqref{sobolev}; that is,
\begin{equation}\label{equazione}
-\Delta_{\mathbb{H}}   u_\eps = \lambda |u_\eps|^{2^\ast-\eps-2} u_\eps \, \ \text{in} \ (\Sc(\Omega))',
\end{equation}
where $\lambda$ is a Lagrange multiplier. Whereas when $\eps>0$ the problem above has a solution~$u_\eps$, it becomes very delicate when $\eps=0$: one falls in the aforementioned CR Yamabe equation, and even the existence of solutions is not granted. In particular, the existence and various properties of the solutions do strongly depend on the geometry and the topology of the domain~$\Omega$. We refer for instance to:~\cite{LU98,Ugu99}
for nonexistence of nonnegative solutions when~$\Om$ is a certain half-space; \cite{CU01}~where the authors show the existence of a solution in the case when the domain~$\Omega$ has at least a nontrivial suitable homology group; 
 \cite{GL92} for existence and nonexistence results for even more general nonlinearity.
\vspace{2mm}

In view of such a qualitative change when $\eps=0$ in both~\eqref{sobolev} and \eqref{equazione}, it seems natural to analyze the asymptotic behaviour as $\eps$ goes to $0$ of  the corresponding optimal functions~$u_\eps$ of the embedding $\Sc(\Omega) \hookrightarrow L^{2^\ast\!-\eps}(\Omega)$. This is the aim of the present paper.
\vspace{2mm}

For what concerns {\it the~Euclidean~counterpart} of such an investigation, several results have been obtained, mostly via fine estimates and a standard regularity elliptic approach of the special class of solutions {of} the equation~\eqref{equazione} being maximizers for the related Sobolev embedding.   


 \vspace{2mm}
 On the contrary, for what concerns {\it the~Heisenberg~panorama} the scene is basically empty  in view of the many difficulties  naturally arising in such a framework. Indeed, the non-{commutative group structure} precludes the free generalization of several tools {such as
 symmetric decreasing rearrangements, ODEs techniques} as well as regularity approximations. 
{Nevertheless, some fundamental results available in the Euclidean framework where successfully extended in $\h$: the Bahri and Coron conjecture in~\cite{BC88} in the aforementioned paper~\cite{CU01}, as well as the  non-existence criteria on several relevant class of proper subset of $\h$ in~\cite{LU98,GL92}.
\vspace{2mm}}


Very recently, it has been proven in~\cite{PPT23} that, up to subsequences, optimal functions~$u_\eps$ for the subcritical Sobolev embedding~\eqref{sobolev} do concentrate {horizontal} energy at one point~$\xi_{\rm o} \in \Omb$, even without {requiring} any regularity assumptions {nor special geometric features} on the domain~$\Omega$, in clear accordance with their Euclidean counterpart in~\cite{FM99,AG03,Pal11,Pal11b}; {see also the recent results in the nonlocal settings in~\cite{PT23}.}
 \begin{theorem}[Theorem~1.2 in~\cite{PPT23}]\label{cor_concentration}
			Let $\Om \subset \h$ be a bounded domain,~{$\mea$ being the family of positive Radon measure in~$\Omb$}  and  
		let~$\ue\in \Sc(\Omega)$ be a maximizer for~$S^{\ast}_{\eps}$.  Then, as~$\varepsilon=\varepsilon_k \to 0$, up to subsequences, we have that
	  	there exists $\xi_{\rm o} \in \Omb$ such that
	  	\[
	  	u_k=u_{\eps_k}  \rightharpoonup 0  \ \text{in} \ L^{2^\ast}\!(\Omega),
	  	\]
	  	and 
	  	\[
	  	\dys  |\Dh u_k|^{2}{\rm d}\xi \tows \boldsymbol{\delta}_{\xi_{\rm o}} \ \text{in} \ \mea,
	  	\]
	  	with~$\boldsymbol{\delta}_{\xi_{\rm o}}$ being the Dirac mass at~$\xi_{\rm o}$.
	\end{theorem}
\vspace{2mm}

Now, a natural question arises: can we  localize the {blow up}; i.~\!e., is the concentration point~$\xi_{\rm o}$ related to some extent to the geometry of the domain~$\Omega$\,?
 \\{For classical elliptic equations with critical growth nonlinearities}, 
 {Atkinson and Peletier, via ODEs methods, proved in~\cite{AP87} that the blow up (as~$\eps \searrow 0$) of solutions to the critical equation
 	\begin{equation}\label{eq:el-sob}
 		-\Delta u = u^{\frac{n+2}{n-2}-\eps} \qquad \text{in}~(H^1_0(\Om))'\,,
 	\end{equation}
 	in the case when~$\Om \subset \r^3$ does coincide with a ball, satisfy
 	\[
 	\lim_{\eps \to 0^+} \eps u_\eps^2(0) = \frac{32}{\pi} \quad \text{and} \quad  \lim_{\eps \to 0^+} \frac{ u_\eps(x)}{\sqrt{\eps}} =\frac{1}{4}\sqrt{\frac{\pi}{2}}\left(\frac{1}{\snr{x}}-1\right).
 	\]
 	Subsequently,
 }
{by relying on purely variational techniques, Brezis and Peletier   in~\cite{BP89}   extended such results to the case of spherical domains. In particular, in~\cite{BP89} it has been showed that  subcritical solutions concentrate at one special point of the domain. Moreover, the authors also conjectured that an analogous result should hold for non spherical	domains {and for higher dimensions as well}. This conjecture was later proved {to be true, independently,} in the case of smooth domains by Han~\cite{Han91} and Rey~\cite{Rey89} by showing that the solutions of~\eqref{eq:el-sob}, with maximal Sobolev energy, concentrate {energy} at one point which can be localized via the Green's function associated with the underlying domain.
 }

\vspace{2mm}
  The involved proofs strongly rely {as well as on various available techniques in the Euclidean framework such as,~e.~\!g., moving planes method, Kelvin transform, etc..., also} on {the availability  of various boundary} regularity {results for standard elliptic equations}. {This last feature} is in clear contrast with the complexity {faced in the present paper}. As well known, even if the domain~$\Omega$ is smooth, the situation is dramatically different because of the possible presence of characteristic points on the boundary~$\partial\Omega$. 
  At such points the vector fields forming the  principal part of the relevant operator~$\Delta_{\mathbb{H}}  $ become tangent to the boundary. Hence, near those characteristic points -- as firstly discovered by Jerison~\cite{Jer81,Jer81-2} -- even harmonic functions on the Heisenberg group can encounter a sudden loss of regularity. {Indeed, Jerison built an explicit solution in the domain $\big\{\xi=(z,t) \in \h: \, t>-M\snr{z}^2\big\}$, for a suitably choice fo $M>0$, vanishing on the boundary and {having at most H\"older regularity} near its isolated characteristic point~${0}$.}
   Also, one did not want to work in the restricted class of domains not having characteristic points; that is, by still including interesting sets as e.~\!g. the torus obtained by revolting the sphere~$\mathbb{S}^{2n}$ around the $t$-axis~\cite{AG21}, but unfortunately excluding an extremely  wide class of regular sets  which play a pervasive role in several relevant problems in the Heisenberg group, as e.~\!g. the level sets of the Jerison and Lee extremal functions~\eqref{talentiane_2} and those of the Folland fundamental solution; i.~\!e., the Kor\'anyi balls. {Nevertheless, some important results have been obtained for maximizing sequence of~$\Ssub$ in non-characteristic domain. Indeed, in~\cite{MMP13}, the authors are able to construct a concentrating sequence of solutions for certain non-degenerate critical point of  the regular part of the Green function of~$\Om$. We also refer  to the references in the aforementioned paper.}
  
  For this reason, and in order to deal with the aforementioned difficulties, it is then quite natural to work under the assumption that the domain~$\Om$ is {\it geometrical regular near its characteristic set} 
   in accordance with the hypotheses firstly assumed in~\cite{GV00} by Garofalo and Vassilev; see conditions~{\rm(}$\Om1${\rm)}\textup{--}{\rm(}$\Om3${\rm)} in forthcoming {\rm Section~\ref{sec_garofalo}}.
   
   We are eventually able to deal with the aforementioned sub-Riemannian framework obstacles by proving the desired localization result for the concentration point~$\xi_{\rm o}$ of the maximizing sequence~$u_\eps$ in terms of the Green function associated with the domain~$\Om$, in turn establishing the validity of the aforementioned Brezis and Peletier conjecture in the Heisenberg group. 
   
   {
   	\vspace{2mm}
   In order to state our main result let us recall for any open domain $\Om \subset \mathbb{H}^n$ and any $\eta \in \Om$ the definition of the Green function of $\Om$ as
   \[
   \Om \times \Om \ni (\xi;\eta) \mapsto G_\Om(\xi;\eta) := K(\eta^{-1}\circ \xi)+H(\xi;\eta)\,,
   \]
   where~$K(\cdot)$ is the fundamental solution of the sub-Laplacian --see Section~\ref{sec_preliminaries} below -- and $H(\cdot)$ is the regular part of the Green function and it can be define in the sense of Perron-Wiener-Brelot as
   \[
   \begin{cases}
   \Delta_{\mathbb{H}}H(\cdot;\eta) =0 \quad & \text{in}~\Om,\\
   H(\xi;\eta) =K(\eta^{-1} \circ \xi) \quad & \text{in}~\partial\Om,
   \end{cases}
   \]
   see, for instance, Theorem~9.2.4 in~\cite{BLU07}. We call {\it Robin function} of $\Om$ the diagonal of $H(\cdot)$, i.~\!e.
   $    \mathcal{R}_\Om(\xi):= H(\xi;\xi) $.
   
   
    With this bit of notation we have the following theorem.
   }
   
   
   
  
    \begin{theorem}\label{thm_green}
  Let~$\Om\subset\h$ be geometrical regular near its characteristic set  and let $u_\eps \in \Sc(\Om)$ be maximizer for $\Sob_\eps$. Then, up to subsequences, 
    \begin{enumerate}[\rm(1)]
   \item $u_\eps$~concentrates at some point~$\xi_{\rm o} \in  \Om$ such that
  \[
   Z_{k} \mathcal{R}_\Om(\xi_{\rm o}) = 0  \qquad \text{for any} \quad k=1,\dots,2n.
  \]
   \item 
   \begin{equation}\label{eq:blow-up}
   	\lim_{\eps \to 0+}   \eps\|u_\eps\|_{L^\infty(\Omega)}^2 =  \frac{2n+2}{nS^*}		\left(\frac{\omega_{2n}}{2n}\frac{\sqrt{\pi}{\bm \Gamma}(\frac{n+1}{2})}{{\bm \Gamma}(\frac{n+2}{2})}\right)^2\snr{\mathcal{R}_\Om(\xi_0)}\,,
 \end{equation}
 with~$S^*$ being the best Sobolev constant, ${\bm \Gamma}(x)$ being Euler's Gamma function and $\omega_{2n}$ being the $(2n-1)$-dimensional measure of the Euclidean $2n$-sphere.\vspace{1.5mm}
    \item  for any~$\xi \in \Om \setminus \{\xi_0\}$  it holds
   \begin{equation}\label{eq:blow-up-2}
   	\lim_{\eps \to 0+}  \frac{u(\xi)}{\sqrt{\eps}} = \sqrt{ \frac{nS^*}{2n+2}} \!\frac{G_\Om(\xi;\xi_{\rm o})}{\sqrt{\snr{\mathcal{R}_\Om(\xi_0)}}}\,,
  \end{equation}
   {where}, as above,~$S^*$ denotes the best Sobolev constant.
   \end{enumerate}
    \end{theorem}
 
    The proof of our main result
      stated in Theorem~\ref{thm_green}
      will be postponed to  {Section~\ref{sec_localization}} of the present manuscript, because it involves several new results -- see in particular forthcoming Theorem~\ref{han} -- together with various general tools in the sub-Riemaniann framework, as e.\!~g., maximum principles, Caccioppoli-type estimates, $H$-Kelvin transform, boundary Schauder-type regularity estimates, as well as with a fine boundary analysis for the subcritical CR Yamabe equation.
  \vspace{2mm}
  

    
\begin{rem}We remark that, as in~{\rm\cite{Han91}}, one could deduce the results in {\rm Theorem~\ref{thm_green}} under the slightly weaker assumption that~$u_\eps$~solves~\eqref{equazione} and approximates the Sobolev quotient. Moreover,  in the same flavour of subcritical approximations, still in clear accordance with the Euclidean framework studied in~{\rm\cite{BP89,Rey89,Han91}}, one can consider to investigate the asymptotic behaviour of the  sequences approaching the critical Sobolev inequality which solve the auxiliary family of equations $-\Delta_{\mathbb{H}}   u_\eps = \lambda u_\eps^{2^\ast-1} +\eps u_\eps$, 
	so that the lack of compactness does similarly come into play when~$\eps$ goes to $0$. 
\end{rem}





    
    \vspace{2mm}
    
     Among other results, in order to prove our main Theorem~\ref{thm_green} above, an asymptotic control of the maximizing sequence~$u_\eps$ for~$\Ssub$ in~\eqref{sobolev} via the Jerison and Lee extremals is needed. This is shown in Theorem~\ref{han} below, and it reveals to be an independent result which could be also useful to investigate further properties related to the subcritical Folland-Stein embedding. 
    \begin{theorem}\label{han}
   	Let~$\Omega \subset \h$ be a  
	smooth bounded  domain  and let~$u_\eps \in \Sc(\Om)$  be a maximizer for~$\Sob_\eps$. Then, there exist~$\{\eta_\eps\} \subset \Om$,~$\{\lambda_\eps\} \subset \r^+$ such that, up to choosing~$\eps$ sufficiently small, we have that, on $\Om$, there exists a universal constant $c>0$ such that
   	\begin{equation}\label{bound_max_seq}
   		 {u_\eps(\xi) \leq c\, \lambda_\eps^{-\frac{Q-2}{2}} U_{\eps}(\xi)}\,,
   	\end{equation}
   	where~$U_{\eps}(\cdot):=U\big(\delta_{\lambda_\eps^{\frac{Q-2}{4}\eps -1}}(\tau_{\eta_\eps^{-1}}(\xi))\big)$ with $U(\cdot)$ being the {Jerison and Lee} extremal functions given in~forthcoming~\eqref{talentiane_2} and the sequence~$\{\eta_\eps\}$ {satisfies}
   	\begin{equation}\label{conv_xi_eps}
   	\lim_{\eps \to 0^+}\eta_\eps =\,  \xi_{\rm o}\,,
   	\end{equation}  
    {where} the concentration point~$\xi_{\rm o}$ is the one given in~{\rm Theorem~\ref{thm_green}}.
   \end{theorem}
The result above reminds somehow to the literature following the pioneering work in the Euclidean setting due to Aubin and Talenti, and this is one of the key-points in the subtle proof of the related conjecture by~Han in~\cite{Han91}. Here, we have also to deal with the fact that, in strong contrast with the Euclidean setting, the Jerison and Lee extremals cannot be reduced to functions depending only on the standard Kor\'anyi gauge.
For this, we need to pursuit a  delicate strategy which makes use and refines the concentration result obtained in~\cite{PPT23} via the $\Gamma$-convergence approach in order to detect the right scalings~$\lambda_\eps$ and~$\eta_\eps$ above.




\subsection{Related open problems and further developments} Starting from the results proven in the present paper, several questions naturally arise.

\vspace{1.3mm}
	
		$\bullet$~~	The {localization} results as well as the $\Gamma$-convergence approach carried out in \cite{PPT23} can be still pursued in the more involved setting of $H$-type groups, giving thus a precise study of the asymptotic behaviour for critical Yamabe-type equations.
		{Such groups of Heisenberg type were introduced by Kaplan in his seminal paper~\cite{Kap80} on hypoelliptic PDEs generated by composition of quadratic forms. They consist into  a natural generalization of the Heisenberg group, including also the nilpotent component in the Iwasawa decomposition of simple groups of rank one. In} extending the aforementioned results {in such framework, one could} take advantage of the involved group structure, as well as of important results present in the literature; that is, the investigation 
		in~\cite{Yan24}, where the explicit expression of the optimal function realizing the inequality in the Sobolev embedding have been provided, positively answering to a conjecture formulated by Garofalo and Vassilev in~\cite{GV01}. 
		

		
		\vspace{1.3mm}
		
		
	{$\bullet$ A strictly related problem could be the asymptotic analysis of singular solutions to critical equations for the Grushin operator.   Such an operator, introduced by Grushin~\cite{Gru71,Gru71-2} and Baouendi~\cite{Bao67}, is strictly related to the study of PDEs on manifolds~\cite{KL12} and to PDEs on the hyperbolic space~\cite{BS12}. Moreover, there is a deep connection between the sub-Laplacian on a group of Heisenberg-type and its Baouendi-Grushin operator on its Lie algebra, via the natural action of the $k$-dimensional torus $\mathbb{T}^k$; see~\cite{GV01}, as well as the role of hyperbolic symmetry in building Grushin-type operators~\cite{MM06}. The sharp form of the Sobolev embedding for the Grushin operator is not yet completely solved.  Nevertheless,  an increasing interest has been focused in deriving properties and existence results of related critical semilinear equations, as, e.~\!g., in the recent paper~\cite{AGLT24},  where the authors prove a Brezis-Nirenberg-type results as well as a version of Lions's Concentration-Compactness principle for the Grushin operator.}
		
				\vspace{1.3mm}
				
		$\bullet$~~Also, one can consider to investigate the fractional counterpart of the results proven here; that is, by replacing the $S^1_0$-norm in~\eqref{folland} by the~$S^s_0$-norm of differentiability order~$s\in(0,1)$.  In this respect, some recent results about decaying properties of subsolutions to  fractional Yamabe-type equations have been obtained in~\cite{GLV23}, via a precise boundedness estimate~\cite[Theorem~1.1]{GLV23}, which is the linear version of the one already obtained in~\cite[Theorem~1.1]{MPPP23} where a precise quantity, the so-called ``nonlocal tail'', has been firstly introduced to attack very general equations led by fractional sub-Laplacian-type operators; {see also~\cite{PP22} for related results on fractional nonlinear equations in $\h$.}
		
		Otherwise, quite a different energy approach in the nonlocal framework could be carried out via an auxiliary harmonic extension problem to the Siegel upper half-space, by taking into account that conformally invariant fractional powers of the sub-Laplacian on the Heisenberg group can be given in terms of the scattering operator, as seen in the relevant paper~\cite{FGMT15}.
		

\vspace{1.3mm}
	
	$\bullet$~~For what concerns the natural hypotheses on the domain~$\Omega$ in order to achieve the localization Theorem~\ref{thm_green}, it could be interesting to ask if one can obtain such a delicate result under somewhat different assumptions, still in the spirit of treating a very wide class of domains also  possibly involving the presence of characteristic points. In this respect, it would be interesting to pursuit such an investigation by taking into account the different assumptions of nontangentially accessible domains satisfying an intrinsic outer ball condition, as firstly introduced in the relevant paper~\cite{CGN02} to deal with the solvability of the related Dirichlet problem with summable boundary data.
	
	
	\vspace{1.3mm}
	
	$\bullet$~~Still for what concerns the possible localization of the concentration point in non-smooth domains, it is worth mentioning the paper~\cite{FGM02} in the Euclidean framework, where Flucher, Garroni and M\"uller were able to construct an example of a peculiar {\it non-smooth} domain~$\tilde{\Omega}$ (see Example~9 there), whose related Robin function~$\mathcal{R}_{\tilde{\Omega}}$ achieves its infimum on the boundary; and subsequently Pistoia and Rey in~\cite{PR03} showed that concentration can occur on the boundary in such a domain~$\tilde{\Omega}$. It could be interesting to understand whether or not one can construct similar intricate examples in the sub-Riemannian setting; {see also the related relevant estimates in~\cite{Wei98}.}
	
\vspace{1.3mm}


	
		$\bullet$~~The localization result in Theorem~\ref{thm_green} can be generalized in the case when one considers more general nonconvex and discontinuous energies with critical growth. This seems a very challenging task; we refer to the delicate approach in the Euclidean framework in~\cite{FGM02}.
		
 
	
	\vspace{1.3mm}
	
	{
			$\bullet$~~Related concentration phenomena could be investigated in the sub-Riemannian setting by considering the second critical exponent, see~\cite{Pas93}, in the same spirit of~\cite{DPMP10}, where the authors are able to prove the existence of solutions whose energy concentrates to a Dirac measure of  given geodesics of the boundary of domains with negative inner normal curvature.
	}


	
	\vspace{1.3mm}
	
	$\bullet$ Finally, another very important conjecture, {\it the third Brezis-Peletier conjecture} in~\cite{BP89}, is addressed to the asymptotics of the solutions to~$-\Delta u_\eps + a u_\eps =3 u_\eps^{5-\eps}$ with the function~$a$ assumed to be critical in the sense of Hebey and Vaugon.
This has been lastly solved in the Euclidean setting in the very relevant paper~\cite{FKK23}, where, under a natural nondegeneracy assumption, the authors are able to derive the exact rate of the blow up and the location of the concentration point. Despite the flexible tricky energy-based approach by the authors, it is unclear whether or not similar results could be achieved in the Heisenberg group.

	\vspace{2mm}	
	 We hope that our estimates and techniques will be important in further developments for a better comprehension of the effects of the lack of compactness in the critical Sobolev embedding in the Heisenberg group.

        
    	\subsection{The paper is organized as follows} In Section~\ref{sec_preliminaries} below we briefly fix the notation and recall some important results on the effects of the lack of compactness of the critical Sobolev embedding in~$\h$ which will be necessary in the rest of the paper. We will also introduce the relevant class of ``geometrical regular'' sets near their characteristic points appearing in the statement of our main result in Theorem~\ref{thm_green}. The fine asymptotic control of the optimal functions via the Jerison\,and\,Lee extremals is achieved in Section~\ref{sec_asymptotic}. In Section~\ref{sec_localization} we prove the localization result in~Theorem~\ref{thm_green} after pursuing a fine boundary analysis for solutions to the subcritical CR Yamabe equation.




\vspace{2mm}
   \section{Preliminaries}\label{sec_preliminaries}
   In this section, we briefly fix the notation by recalling a very few properties of the Heisenberg group; we also present some well-known results regarding the lack of compactness in the critical Sobolev embedding in the Folland-Stein spaces in the Heisenberg group.




   \subsection{The Heisenberg-Weyl group}
    We start by summarily recalling a few well-known facts about the Heisenberg group. 
    We denote points~$\xi$ in~$\mathbb{C}^n\times\r \simeq \r^{2n+1}$ by
    \[
    \xi := (z,t) = (x+iy, t) \simeq (x_1,\dots,x_n, y_1,\dots,y_n,t) 
    \in \r^n\times \r^n \times \r.
    \]
The Heisenberg group $\h$ is an analytic, simply
connected  $(2n+1)$-dimensional Lie group  such that its Lie algebra $\mathfrak{g}$ admits a stratification
\[
\mathfrak{g}=V^1\oplus V^2, \quad [V^1,V^1]=V^2 \quad \text{and}  \quad   [V^1,V^{2}]=\{0\}.                                                                                                                                                                                                                                   \]
A basis of left invariant vector fields of the subspace $V_1$ is given by 
\[
Z_j := \partial_{x_j} +2y_j\partial_t, \quad Z_{n+j}:= \partial_{y_j}-2x_j\partial_t, \quad 1 \leq j\leq n.
\]
Moreover, note that since
$[Z_j,Z_{n+j}]=-4\partial_t$ for every  $1 \leq j \leq n$, {while all the others are zero,} a basis for the subspace $V_2$ is given by 
\[
T:= \partial_t.
\]
Hence, the stratification of the algebra $\mathfrak{g}$ is given by
\[
\mathfrak{g}  = \textup{span}\{Z_1,\dots,Z_{2n}\} \,\oplus\, \textup{span}\{T\}.
\]
The Heisenberg group can be identified with the triple $(\mathbb{R}^{2n+1}, \circ, \{\delta_{\lambda}\}_{\lambda>0})$, where~$\circ$ is  {\it the polynomial group multiplication law} given by 
\[
\xi \circ \xi' := (x+x',\, y+y',\, t+t'+2\langle y,x'\rangle-2\langle x,y'\rangle)\,,
\]
 for any $\xi,\xi' \in \h$, and $\{\delta_\lambda\}_{\lambda>0}$ is the automorphism of $(\r^{2n+1}, \circ)$ of non-isotropic {\it dilations} 
 \begin{equation}\label{def_philambda}
 	\xi   \mapsto \delta_{\lambda}(\xi):=(\lambda x,\, \lambda y,\, \lambda^2 t).
 \end{equation}
Given~$\xi'\in\h$, {\it the left translation}~$\tau_{\xi'}$ is defined by
\begin{equation}\label{def_tau}
	\tau_{\xi'}(\xi):=\xi'\circ\xi \qquad \forall \xi \in \h.
\end{equation}
As customary,
$Q$ is the {\it homogeneous dimension\,} with respect to~$\{\delta_\lambda\}_{\lambda>0}$ given by 
\[
Q:= \dim(V^1)+ 2 \dim(V^2)= 2n+2.
\]
We denote by $\Dh$   the horizontal (or intrinsic) gradient of the group
\[
\Dh  := ( Z_1,\dots, Z_{2n}).
\]
and we indicate by $D=(\partial_{x_1},\dots,\partial_t)$ the standard Euclidean gradient.   
The Kohn Laplacian (or sub-Laplacian)~$\Delta_{\mathbb{H}}  $ on~$\h$ is the second order operator invariant with respect to the left-translations~$\tau_{\xi'}$ defined in~\eqref{def_tau} and homogeneous of degree~$2$ with respect to the dilations~$\{\delta_\lambda\}_{\lambda>0}$ defined in~\eqref{def_philambda},
\[
\Delta_{\mathbb{H}}   :=  \sum_{j=1}^{2n} Z^2_j.
\]
It is well known (see \cite{Fol75}) that~$\Delta_{\mathbb{H}}$ admits a unique fundamental solution ${K}(\cdot) \in C^{\infty}(\r^{2n+1} \setminus \{{0}\})$, ${K}(\cdot) \in L_{\text{loc}}^1(\r^{2n+1})$, ${K}(\xi) \to 0$ when  $\xi$  tends to infinity and such that 
\[
\int_{\h} {K}(\xi) \,  \Delta_{\mathbb{H}}   \phi(\xi) \,  {\rm d}\xi = -\phi({0}) \quad \forall \phi \in C^{\infty}(\h).
\]
\begin{defn}
	We call Gauge norm on $\h$ a homogeneous symmetric norm $d$ smooth out of the origin and satisfying
	\[
	\Delta_{\mathbb{H}} (d(\xi)^{2-Q})=0 \quad \forall \xi \ne {0}.
	\]
\end{defn}
In particular, we will work with the standard gauge on~$\h$, also known as {\it Kor\'anyi gauge},
\begin{equation*} 
	|\xi|_{\mathbb{H}} := (|z|^4 +t^2)^\frac{1}{4} \qquad \forall \xi=(z,t) \in \h.
\end{equation*}
In this way we have that  
\[
{K}(\xi)=C_Q^{-1}|\xi|_{\mathbb{H}}^{2-Q}= \tfrac{1}{C_Q\big( ( \snr{x}^2+\snr{y}^2)^2 + 
		t^2 \big)^{(Q-2)/4}}\,,
\]
where the constant~$C_Q$ can be computed explicitly.
We define the fundamental solution with pole in $\eta$ as  $K(\eta^{-1}\circ \xi) =  C_Q^{-1}\snr{\eta^{-1}\circ \xi}_{\mathbb{H}}^{2-Q}$.
 As customary, we will denote by~$ B_\rr(\eta)$ the gauge ball with center~$\eta \in \h$ and radius~$\rr>0$ given by
$
B_\rr(\eta):=\{\xi \in \h : |\eta^{-1}\circ \xi|_{\mathbb{H}} < \rr\}.
$

\vspace{2mm}
\begin{defn}\label{def:def-func}
	A domain $\Om$ is said to be of class $C^\infty$, if for any $\xi \in \partial \Om$ there exists a neighborhood $U_\xi$ of $\xi$ and a function $\varPhi_\xi \in C^\infty(U_\xi)$ such that
	\[
	\begin{split}
		U_\xi \cap \Om & = \{\eta \in U_\xi\,: \, \varPhi_\xi(\eta)<0\}\\
		U_\xi \cap \partial \Om & = \{\eta \in U_\xi\,: \, \varPhi_\xi(\eta)=0\}
	\end{split}
	\]
\end{defn}
We will say that $\xi \in \partial \Om$ is a {\it characteristic point} of $\Om$ whenever $\Dh \varPhi_\xi =0$ and we will denote with $\Sigma(\Om)$ the set of all characteristic point of $\Om$.


We recall the definition of intrinsic H\"older spaces~$C^{{k},\beta}$.
\begin{defn}
	Let~$\beta \in (0,1)$ and~$\Om \subset \h$. A function~$u: \Om \rightarrow \mathbb{R}$ belongs to\,~$C^{0,\beta}(\Om)$ if there exists a positive constant~$c$ such that
	\[
	[u]_{C^{0,\beta}(\Om)}:=\sup_{\xi \neq \eta \atop \xi, \eta  \in \Om}\frac{\snr{u(\xi)-u(\eta)}}{\snr{\eta^{-1}\circ \xi}_{\mathbb{H}}^\beta} \leq c.
	\]
	For any~$k \in \mathbb{N}$, we say that $u \in C^{{k},\beta}(\Om)$ if~$Z_j  u \in C^{{k}-1,\beta}(\Om)$, for any~$j=1,\dots,2n$.
\end{defn}
The space~$C^{0,\beta}(\Om)$ is a Banach space endowed with the following norm,
\[
\|u\|_{C^{0,\beta}(\Om)}:= \|u\|_{L^\infty(\Om)} + [u]_{C^{0,\beta}(\Om)}.
\] 


Let $\xi, \eta$ in $\partial \Omega$, we define the induced distance $\hat{d}$ on $\partial \Omega$ by
\[
\hat{d}(\xi,\eta):=\snr{\eta^{-1}\circ\xi}_{\mathbb{H}},
\]
where $\snr{\cdot}_{\mathbb{H}}$ is the Kor\'anyi norm in $\h$ and for $r>0$ we call $\hat B_r( \xi)$ the induced ball given by 
\[
\hat B_r( \xi)=B_r(\xi) \cap \partial \Omega,
\]

\begin{defn}
	Let $0<\beta<1$. We say that a continuous function $u$ belongs to $C^{0,\beta}(\partial \Omega)$ if  there exists a constant $c$ such that 
	\[
	[u]_{\alpha}= \sup_{\substack{\xi ,\eta  \in \partial \Omega \\ \xi \neq \eta} } 	\dfrac{|u(\xi)-u(\eta)|}{\hat{d}(\xi,\eta)^\beta} < c
	\]
	and  the H\"older norm is defined by
	\[
	\|u\| _{\alpha}= [u]_{\alpha}+ \sup_{\xi \in \partial \Omega} |u(\xi)|
	\]
\end{defn}

Consider now a non-characteristic point $\xi \in \partial \Om$. Hence, the outward horizontal unit normal
\[
{\bm n}_\mathbb{H} := \frac{\Dh \varPhi_\xi}{\snr{\Dh \varPhi_\xi}}\,,
\]
is well define, and so we can express the boundary $\partial \Om \cap U_\xi$ in local coordinates~$\hat{\zeta}:= (\zeta_1,\dots,\zeta_{2n}) \in \partial \Om \cap U_\xi$ by means of the inverse $\Xi_\xi$ of the exponential map; see~\cite[Section~3]{Jer81}. 
Set now, $J := (j_1,\dots,j_{2n})$, for any $\zeta$ define
\[
\text{deg}(J) := j_1+\dots + j_{2n-1} + 2j_{2n}
\]
and
\[
\zeta^J= \zeta_1^{j_1}\cdots\zeta_{2n}^{j_{2n}}
\]
A polynomial of order $k$ in local coordinates on the boundary is  given by
\[
P(\hat{\zeta}):= \sum_{\text{deg}(J)\leq k} a_J \hat{\zeta}^J\,,
\]
for some constant~$a_J$. If $\eta = \Xi_\xi^{-1}(\hat{\zeta})$, let $P_\xi(\eta):= P(\hat{\zeta})$.

We can give the following definition.

\begin{defn}\label{def:intr-hld}
Let $\beta \in (0,1)$ and $k \in \mathbb{N} \cup \{0\}$. We say that a bounded function $u$ belongs to $\Gamma^{k,\beta}(\partial \Om)$ if for each $\xi \in \partial \Om$ and any $\sigma>0$, there exists a polynomial $P_\xi$  of degree $k$ in local coordinates on $\partial \Om$ and a uniform constant $c>0$ such that
\[
\snr{u(\eta)-P_\xi(\eta)}\leq \sigma^{k+\beta} \qquad |\eta^{-1}\circ\xi|_{\mathbb{H}} \leq \sigma.
\]
Then, the H\"older seminorm $[u]_{k,\beta}$ is the least of possible constant $c>0$ above plus the supremum of the coefficient of $P_\xi$ and the  H\"older norm is given by
\[
\|u\|_{\Gamma^{k,\beta}(\partial \Om)} := [u]_{k,\beta} + \sup_{\xi \in \partial \Om} \snr{u(\xi)}.
\]
\end{defn}
With this notation we recall the boundary Schauder-type estimate below; for other results regarding Schauder estimates at the boundary in Lie groups we refer the reader to~\cite{CGS21,BGM22}.

\begin{theorem}[Theorem 1.1 in \cite{BGM19}]\label{boundary_hld}
	Let $\Om\subset \h$ be a bounded subset of class $C^{1,\beta}$ for some $\beta \in (0,1)$, and assume that the set $B_\rr \cap \partial \Om$ is non-characteristic. Let $u \in \Sc (B_\rr \cap\Om)\cap C(\overline{B_\rr \cap  \Om})$ be a weak solution to
	\begin{equation}\label{eq:bgm}
	\begin{cases}
     -\Delta_{\mathbb{H}}   u = f \quad & \text{in}~B_\rr\cap\Om,\\ 
     u=0 \quad  & \text{in}~B_\rr \cap\partial\Om,
	\end{cases}
	\end{equation}
	with~$f \in L^\infty (\overline{B_\rr \cap  \Om})$. Then, $u\in C^{1,\beta}(\overline{B_{\rr/2}\cap \Om})$ and we have the apriori estimate
	\[
		[\Dh u]_{\Gamma^{0,\beta}({B_\rr \cap \Om}) }
		\, \leq \, c\rr^{-1-\beta}\big(\|u\|_{L^\infty(B_\rr \cap \Om )} +\rr^2\|f\|_{L^\infty(B_\rr\cap \Om)} +\rr^{1+\beta}[\phi]_{\Gamma^{1,\beta}(\overline{B_{\rr} \cap  \Om})}\big).
	\]
\end{theorem}
{
\begin{rem}
    We remark that in particular, in {\rm\cite{BGM19}} the authors prove that on the boundary $\partial \Om \cap B_{\rr/2}$ the solution $u$ to \eqref{eq:bgm} belongs to the intrinsic H\"older class $\Gamma^{1,\beta}$ according to Definition {\rm\ref{def:intr-hld}}.
\end{rem}}

\vs 



Moreover, when sufficiently regularity is assumed, the  Pohozaev identity stated in forthcoming Lemma~\ref{pohozaev} holds true. We denote by~$\mathdutchcal{D}$ the infinitesimal generator of the one-parameter group of non-isotropic dilations~$\{\delta_\lambda\}_{\lambda>0}$ in~\eqref{def_philambda}; that is,
\begin{equation}\label{dilation_vector_field}
	\mathdutchcal{D} := \sum_{j=1}^n \big(x_j\partial_{x_j}+y_j \partial_{y_j}\big) +2t\partial_t.
\end{equation}
\begin{lemma}[see Theorem 3.4 in \cite{GV00} and Theorem 2.1 in \cite{GL92}]\label{pohozaev}
	Let $\Om$ be a $C^1$~domain and let $u\in C^2(\overline{\Om})$ be a solution to
	\[
	-\Delta_{\mathbb{H}}   u =f(u) \qquad \textrm{in} \quad \Om,
	\]
	for some function $f \in C(\r)$ such that $f(0)=0$. Setting $F(s) := \int_0^s f(t) \, {\rm d}t$\,, the following identity holds
	\begin{eqnarray*}
		&& \int_\Om \big(2QF(u)-(Q-2)uf(u)\big) \, {\rm d}\xi \\
		&& \qquad = 2 \sum_{j=1}^{2n}\int_{\partial \Om}\mathdutchcal{D}u Z_j u \langle Z_j,\bm{n}\rangle \,{\rm d}\mathcal{H}^{Q-2} -\int_{\partial \Om}\snr{\Dh u}^2 \langle \mathdutchcal{D},\bm{n}\rangle \,{\rm d}\mathcal{H}^{Q-2}\\
		&&\qquad \quad + 2\int_{\partial \Om}F(u)\langle \mathdutchcal{D},\bm{n}\rangle \,{\rm d}\mathcal{H}^{Q-2} +(Q-2)\sum_{j=1}^{2n}\int_{\partial \Om}u Z_j u \langle Z_j,\bm{n}\rangle \,{\rm d}\mathcal{H}^{Q-2}\,,
	\end{eqnarray*}
	where $\nu$ is the exterior unit normal and $\mathdutchcal{D}$ is the vector field generating the anisotropic dilations $\{\delta_\lambda\}_{\lambda >0}$ {and 	where we denote by~$\mathcal{H}^{Q-2}$ the $(Q-2)$-dimensional Hausdorff measure.}
\end{lemma}

\vspace{2mm}
\subsection{Geometrical regularity near the characteristic set}\label{sec_garofalo}
Some further notation is needed in order to introduce the natural assumptions on the domains in accordance with the by-now classical paper~\cite{GV00}.  
\begin{defn}
	Let~$\Om$ be a~$C^{1}$~connected open set of\,~$\h$ containing the group identity~${0}$. We say that $\Om$ is \textup{$\delta_\lambda$-starlike} {\rm (}with respect to the identity~${0}${\rm)} along a subset~$K \subseteq \partial \Om$ if
	\[
	\langle \mathdutchcal{D},\bm{n}\rangle (\eta) \geq 0, 
	\]
	at every~$\eta \in K$;  in the display above~$\bm{n}$ indicates the exterior unit normal to~$\partial \Om$. 
	
	We say that~$\Om$ is \textup{uniformly $\delta_\lambda$-starlike} {\rm (}with respect to the identity~${0}${\rm)} along~$K$ if there exists~$C_\Om >0$ such that, at every~$\eta \in K$,
	\[
	\langle \mathdutchcal{D},\bm{n}\rangle (\eta) \geq C_\Om.
	\]
	A domain $\Om$ as above is {$\delta_\lambda$-starlike} (uniformly {$\delta_\lambda$-starlike}, respectively) with respect to one of its point~$\zeta \in  \Om$ along~$K$ if~$\tau_{\zeta^{-1}}(\Om)$ is {$\delta_\lambda$-starlike} (uniformly {$\delta_\lambda$-starlike}, respectively) with respect to the group identity~${0}$ along~$\tau_{\zeta^{-1}}(K)$.
\end{defn}
\vs

For recent interesting results on starshaped domains in sub-Riemannian frameworks, we refer to~\cite{DF19,FP23}.

\vspace{2mm}

We finally are in the position to introduce a natural class of regular sets that we take the liberty to christening for shorten.
\begin{defn}
	A smooth domain~$\Om\subset\h$ such that~$\partial \Omega$ is an orientable hypersurface is {\rm  
		``geometrical regular near its characteristic set''} if here exist~$\varPhi \in C^\infty(\h)$,~$c_\Om >0 $ and~$\rr_\Om \in \r$ such that
	the following conditions hold true,
	\begin{itemize}
		\item[($\Om1$)]
		There exist~$\varPhi \in C^\infty(\h)$,~$c_\Om >0 $ and~$\rr_\Om \in \r$ such that
		\[
		\Om := \{\varPhi < \rr_\Om\} \quad \textrm{and} \quad \snr{D \varPhi} \geq c_\Om.
		\]
		\item[($\Om2$)] 
		There exist~$M_\Om$ such that {for an interior neighborhood~$\omega$ of $\Car$}
		\[
		\Delta_{\mathbb{H}}   \varPhi \geq \frac{4}{M_\Om } \langle \Dh \varPhi, x \rangle.
		\]
		\item[($\Om3$)]
		$\Om$~ is {$\delta_\lambda$-starlike} with respect to one of its point~$\zeta_{\rm o} \in \Om$ and uniformly {$\delta_\lambda$-starlike} with respect to~$\zeta_{\rm o}$ along~$\Car$.
	\end{itemize}
\end{defn}


As mentioned in the Introduction, recall that notable sets having characteristic points do naturally belong to such a wide class of domains, as for instance the Kor\'any balls {as well as level sets for the Jerison and Lee extremals; i.~\!e., domains of the type 
\[
\Om_{\sigma,R} :=\big\{\xi= (z,t) \in \h: (\sigma^2 + |z|^2)^2 + t^2 < R^4\big\} \quad \forall R>\sigma >0.
\]
Also}, the class is safe from problematic sets in the sense that the definition above does permit to exclude those sets like for instance the  characteristic cones which the famous boundary Schauder counter-example by Jerison has been based on.



\vspace{2mm}
\subsection{Lack of compactness in the critical Sobolev embedding}\label{sec_cca}
In this section, we recall some important results in the Heisenberg framework regarding the analysis of the effect of the lack of compactness in the critical Sobolev embedding.
\vspace{1mm}

Firstly, we state (in the form adapted to our framework) the aforementioned pioneering result by Jerison and Lee~\cite{JL88} which gives the explicit expression of the functions giving the equality in~\eqref{folland}.  
\begin{theorem}[Corollary~C in~\cite{JL88}]\label{thm_optimal}
Let $2^\ast=2Q/(Q-2)$. Then for any $\lambda>0$ and any $\xi_{\rm o}\in \h$, the function~$U_{ {\lambda}, \xi_{\rm o}}$ defined by
\[
U_{{\lambda}, \xi_{\rm o}} := U \left(\delta_{\frac{1}{\lambda}}\big(\tau_{\xi^{-1}_0}(\xi)\big)\right)\,,
\]
where
\begin{equation}\label{talentiane_2}
U(\xi)=c_0\left(\big(1+|z|^2\big)^2+t^2\right)^{-\frac{Q-2}{4}}\quad \forall \xi\in\h,
\end{equation}
is solution to the variational problem~\eqref{critica0}; that is,
\[
\|U_{{\lambda}, \xi_{\rm o}}\|^{2^\ast}_{L^{2^\ast}(\h)} = \Sob \|\Dh U_{{\lambda}, \xi_{\rm o}}\|_{L^2(\h)}^{2^\ast}.
\]
where $S^*$ is the best Sobolev constant.


\end{theorem}
\vspace{2mm}
We conclude this section by recalling the Global Compactness-result for Palais-Smale sequences in~$\Om\subseteq \h$.  For any fixed~$\lambda\in\r$ consider the problem
\begin{equation}\label{plambda}
	-\Delta_{\mathbb{H}}   u-\lambda u-|u|^{2^*-2}u=0\qquad\text{in } (\Sc(\Omega))', \tag{$P_\lambda$}
\end{equation}
together with its corresponding Euler--Lagrange functional~$\El:\Sc(\Omega)\to\r$ given by
\begin{equation*} 
	\El(u) =\frac12 \int_{\Om}|\Dh u|^2 	\,{\rm d}\xi -\frac{\lambda}{2}\int_{\Om}|u|^2	\,{\rm d}\xi-\frac{1}{2^*}\int_{\Om}|u|^{2^*}	\,{\rm d}\xi.
\end{equation*}
Consider  also the following limiting problem,
\begin{equation}\label{pzero}
	-\Delta_{\mathbb{H}}  u-|u|^{2^*-2}u=0\qquad\text{in } (\Sc(\Om_{\rm o}))',\tag{$P_0$}
\end{equation}
where $\Om_{\rm o}$ is either a half-space or the whole~$\h$;
i.~\!e., the Euler-Lagrange equation corresponding to the energy functional~$\Es: \Sc(\Om_{\rm o})\to\r$,
\begin{equation*} 
	\Es(u)=\frac12 \int_{\Om_{\rm o}}|\Dh u|^2 	\,{\rm d}\xi -\frac{1}{2^*}\int_{\Om_{\rm o}}|u|^{2^*}	\,{\rm d}\xi.
\end{equation*}

We have the following
\begin{theorem}[Theorem~1.3 in~\cite{PPT23}]\label{thm_glob_comp}
Let~$\{u_k\}\subset \Sc(\Omega)$ be a Palais-Smale sequence for~$\El$; i.~\!e., such that
	\begin{eqnarray*}
		&&\El(u_k)\leq c\quad \text{for all }k,\\
		&& {\rm d}\El(u_k) \rightarrow 0\quad \text{as } k\to\infty \quad \text{in }(\Sc(\Omega))'.
	\end{eqnarray*}
	Then, there exists a (possibly trivial) solution~$\uz\in \Sc(\Omega)$ to~\eqref{plambda} such that, up to a subsequence, we have
	\[
	u_k\rightharpoonup\uz\quad \text{as } k\to\infty \quad\text{in }\Sc(\Omega).
	\]
	Moreover, either the convergence is strong or there is a finite set of indexes~$\Irm=\{1,\dots,J\}$ such that for all~$j\in\Irm$ there exist a
	nontrivial solution~$\uj\in \Sc(\Om_{\rm o}^{(j)})$ to~\eqref{pzero} with $\Om_{\rm o}^{(j)}$ being either a half-space or the whole~$\h$,
	a sequence of nonnegative numbers~$\{\lambdakj\}$ converging to zero and a sequences of points~$\{\xikj\}\subset\Om$ such that, for a renumbered subsequence, we have for any~$j\in\Irm$
	\[
	\ukj(\cdot):=\lambdakj^{\frac{Q-2}{2}}u_k\big(\tau_{\xikj}\big(\delta_{\lambdakj}(\cdot)\big)\big) \rightharpoonup \uj(\cdot)\quad\text{in }\Sc(\h) \quad\text{ as }k \to\infty.
	\]
	In addition, as~$k\to\infty$ we have
	\begin{eqnarray*}
		&&u_k(\cdot)=\uz(\cdot)+\sum_{j=1}^{J}\lambdakj^{\frac{2-Q}{2}}u_k\big(\delta_{1/\lambdakj}\big(\tau_{\xikj}^{-1}(\cdot)\big)\big)+{\rm o}(1) \quad\text{ in }\Sc(\h);\\
		&&\left|\log{\frac{\lambda_k^{(i)}}{\lambdakj}}\right|+\left|\delta_{1/\lambdakj}\big(\xikj^{-1}\circ \xi_k^{(i)}\big)\right|_{\h}\to\infty\quad\text{for }i\neq j,\ \,i,j\in\Irm;\\
		&&\|u_k\|_{\Sc}^2=\sum_{j=1}^{J}\|\uj\|_{\Sc}^2+{\rm o}(1);\\
		&&\El(u_k)=\El(\uz)+\sum_{j=1}^{J}\Es(\uj)+{\rm o}(1).
	\end{eqnarray*}
\end{theorem}

 Several remarkable results regarding the behaviour of Palais-Smale sequences for the critical energy~$\Es$ can be  found in~\cite{Cit95} where the author proves an analogous representation result for {nonnegative Palais-Smale sequences}.
	Moreover, it is worth mentioning the relevant paper~\cite{GMM18}, where the authors deduce the desired Global Compactness in the important case of critical energies on the $(2n+1)$-dimensional sphere (equipped with the CR structure) associated to the sub-elliptic intertwining operator~$\mathcal{L}_{2k}$ of order~$2k$, with~$k \in \mathbb{R}$ being such that~$0 < 2k < Q$; also covering fractional CR Yamabe energies.
	
	
	As firstly shown in a very general setting in~\cite{PP15}, the proof of the result in the theorem above  is deduced in~\cite{PPT23} via a subtle application of the
	so-called {\it Profile Decomposition}, originally proven by  G\'erard  for bounded sequences in the fractional Euclidean space~${H}^s$, and extended to the Heisenberg framework in~\cite{Ben08}.


\vspace{2mm}
\section{Asymptotic control via the Jerison and Lee extremals}\label{sec_asymptotic}


Before going into the proof of Theorem~\ref{han}, we need some integrability and boundedness estimates for weak solutions to subelliptic equations in the Heisenberg group as well as the notion of $H$-Kelvin transform and a maximum principle for the sub-Laplacian~$\Delta_{\mathbb{H}}  $.

\subsection{Regularity properties for subelliptic equations}
Below, we state and prove some estimates in the same spirit of classical Caccioppoli-type inequalities and consequently boundedness results. We refer also to the Euclidean counterpart in~\cite{BP89,Han91} and to related results on Carnot groups in~\cite{Vas06}. 

\begin{lemma}\label{lemma:gain}{\bf [Caccioppoli-type estimate].}
	Let us consider the following problem,
\begin{equation}\label{pbm:han}
	\begin{cases}
		-\Delta_{\mathbb{H}}   u = a(\xi)u^{p-1},\\
		u \in \Sc(\Om), \ u \geq 0.
	\end{cases}
\end{equation}
where~$a \in L^\infty(\Om)$,~$2< p_0 < p \leq 2^*$. Then, for any~$1<q \leq 2^*-1$, there exists~$\upsilon_0$ and~$\rr_0$ depending only on~$\|a\|_{L^\infty(\Om)}$,~$n$,~$p$ and~$q$ such that, for any~$\bar{\xi}$ with
\begin{equation}\label{eq:first-gain}
\int_{\Om \cap B_{2\rr}(\bar{\xi})} u^p \,{\rm d}\xi \leq \upsilon_0 \quad \text{for any}~\rr \leq \rr_0,
\end{equation}
we have
	\begin{equation}\label{gain_int}%ALERT label ancora da usare
				\|u\|_{L^\frac{(q+1)2^*}{2}(\Om \cap B_\rr(\bar{\xi}))} \leq \frac{c}{\rr^{2/(q+1)}} \|u\|_{L^{q+1}(\Om \cap B_\rr(\bar{\xi}))}\,,
	\end{equation}
for a constant~$c >0$ depending  on the dimension~$n$ only.
\end{lemma}

\begin{proof}[\bf Proof]
{
    Test \eqref{pbm:han} with~$\phi:=\varphi^2u^q$, with~$q >1$  and $\varphi \in C^\infty_0(\Om)$ being a cut-off function, so that
   \[
    \varphi \equiv 1 \, \text{on} \ \Om \cap B_\rr(\bar{\xi}), \quad \varphi \equiv 0 \ \text{outside} \ \Om \cap B_{2\rr}(\bar{\xi}) \quad \text{and} \ \snr{\Dh \varphi} \leq c/r
   \]     
    Then,	via integration by parts and Young's Inequality
	\begin{eqnarray}\label{app_eq_4}
		&&\int_\Om \Dh (\varphi^2u^q)\cdot \Dh  u \, {\rm d}\xi \nonumber\\
		&&\qquad \qquad\quad = \int_\Om 2\varphi u^q \Dh \varphi\cdot \Dh  u  + q\varphi^2u^{q-1}\snr{\Dh  u}^2 \, {\rm d}\xi\nonumber\\
		&&\qquad \qquad\quad \geq -\,2\int_\Om \varphi u^{\frac{q+1}{2}+\frac{q-1}{2}} \snr{\Dh \varphi} \snr{\Dh  u}\, {\rm d}\xi  + q\int_{\Om}\varphi^2u^{q-1}\snr{\Dh  u}^2 \, {\rm d}\xi\nonumber\\
		&&\qquad \qquad\quad \geq -\,\frac{1}{\epsilon}\int_\Om \varphi^2 u^{q-1} \snr{\Dh  u}^2\, {\rm d}\xi -\epsilon\int_\Om  u^{q+1} \snr{\Dh  \varphi}^2\, {\rm d}\xi\nonumber\\
		&&\qquad \qquad\qquad + \,q\int_{\Om}\varphi^2u^{q-1}\snr{\Dh  u}^2 \, {\rm d}\xi\nonumber\\
		&&\qquad \qquad\quad\ge \frac{q}{2}\int_\Om\varphi^2u^{q-1}\snr{\Dh  u}^2 \, {\rm d}\xi  -\frac{2}{q}\int_\Om u^{q+1}\snr{\Dh  \varphi}^2 \, {\rm d}\xi.
	\end{eqnarray}
	Using Sobolev's and H\"older's Inequality and~\eqref{app_eq_4} we obtain
	\begin{eqnarray}\label{app_eq_5}
		&&( S^*)^{-\frac{2}{2^*}}\left(\int_{\Om \cap B_{2\rr}(\bar{\xi})} \snr{\varphi u^{(q+1)/2}}^{2^*} \, {\rm d}\xi\right)^\frac{2}{2^*}\nonumber\\
		&&\quad \le  \int_{\Om \cap B_{2\rr}(\bar{\xi})} \snr{\Dh (\varphi u^{(q+1)/2})}^2 \, {\rm d}\xi\notag\\
		&&\quad \le \frac{(q+1)^2}{q}\frac{q}{2}\int_{\Om \cap B_{2\rr}(\bar{\xi})} \varphi^2 u^{q-1}\snr{\Dh  u}^2 \, {\rm d}\xi +2\int_{\Om \cap B_{2\rr}(\bar{\xi})} u^{q+1}\snr{\Dh \varphi}^2 \, {\rm d}\xi\nonumber\\
		&&\quad \le \frac{2(q+1)^2}{q}\left[\int_{\Om \cap B_{2\rr}(\bar{\xi})} a(\xi)u^{q+p-1}\varphi^2 \, {\rm d}\xi + \frac{2}{q}\int_{\Om \cap B_{2\rr}(\bar{\xi})} u^{q+1}\snr{\Dh  \varphi}^2 \, {\rm d}\xi\right]\notag\\
		&&\qquad +
		\,2\int_{\Om \cap B_{2\rr}(\bar{\xi})} u^{q+1}\snr{\Dh \varphi}^2 \, {\rm d}\xi\notag\\
      &&\quad \stackrel{\eqref{app_eq_4}}{\leq} c\|a\|_{L^\infty(\Om)}\left(\int_{\Om \cap B_{2\rr}(\bar{\xi})} \snr{\varphi u^{(q+1)/2}}^{2^*}\, {\rm d}\xi\right)^\frac{2}{2^*}\left(\int_{\Om \cap B_{2\rr}(\bar{\xi})} \snr{u}^{(p-2)\frac{Q}{2}}\, {\rm d}\xi\right)^\frac{2}{Q}\\
      &&\qquad +c\int_{\Om \cap B_{2\rr}(\bar{\xi})} u^{q+1}\snr{\Dh  \varphi}^2 \, {\rm d}\xi. \notag
	\end{eqnarray}
    Note now that, since $p \leq  2^*$, $Q(p-2)/(2p) \leq 1$, one has
  \begin{eqnarray*}
     && c\|a\|_{L^\infty(\Om)}\left(\int_{\Om \cap B_{2\rr}(\bar{\xi})}\snr{u}^{(p-2)\frac{Q}{2}}\, {\rm d}\xi\right)^\frac{2}{Q}\\
      &&\qquad \quad \leq c\|a\|_{L^\infty(\Om)}\left(\int_{\Om \cap B_{2\rr}(\bar{\xi})} \snr{u}^p\, {\rm d}\xi\right)^\frac{p-2}{p}\snr{\Om \cap B_{2\rr}(\bar{\xi})}^{\frac{2}{Q}-\frac{p-2}{p}}\\
      &&\qquad \quad \stackrel{\eqref{eq:first-gain}}{\leq} c\|a\|_{L^\infty(\Om)}\upsilon_0^\frac{p-2}{p}\snr{\Om \cap B_{2\rr}(\bar{\xi})}^{\frac{2}{Q}-\frac{p-2}{p}}\, \leq\ \frac{( S^*)^{-\frac{2}{2^*}}}{2}\,,
  \end{eqnarray*}
up to choosing $\upsilon_0$ and $\rr_0$ sufficiently small.

Then, re-absorbing on the left-hand side in \eqref{app_eq_5}, it yields
\[
\frac{( S^*)^{-\frac{2}{2^*}}}{2}\left(\int_{\Om \cap B_{2\rr}(\bar{\xi})} \snr{\varphi u^{(q+1)/2}}^{2^*} \, {\rm d}\xi\right)^\frac{2}{2^*} \leq c\int_{\Om \cap B_{2\rr}(\bar{\xi})} u^{q+1}\snr{\Dh  \varphi}^2 \, {\rm d}\xi.
\]
 which gives the desired estimate in~\eqref{gain_int}. 
 }
\end{proof}

Moreover, the following $L^\infty$-estimate holds true.
\begin{lemma}\label{lemma:sup}{\bf [Local boundedness estimate].}
	Let us consider the following problem,
\[
	\begin{cases}
		-\Delta_{\mathbb{H}}   u = f(\xi,u)u,\\
		u \in \Sc(\Om), \ u \geq 0.
	\end{cases}
\]
  If~{$\xi \mapsto f(\xi,\cdot)\in L^{q/2}(\Om)$, for a.~\!e.~$\xi \in \Om$} and for some~$q>Q$, then it holds
  \begin{equation}\label{sup_est} 
  	\sup_{B_\rr(\bar{\xi})}u \leq c \left(\frac{1}{\rr^Q} \int_{ B_{2\rr}(\bar{\xi})} u^{2^*}\right)^\frac{1}{2^*} \qquad \forall B_{2\rr}(\bar{\xi})\subset \Om,
  \end{equation}
  where~$c\equiv c(n,q,\|f\|_{L^{q/2}(\Om)})>0$.
\end{lemma}
{\bf Proof.}
		With no loss of generality, we assume that~$\rr=1$ {and $\bar{\xi}={0}$}. Consider for~$p > 1$, $B_4(0)\equiv B_4$ and a nonnegative cut-off function~$\varphi \in C^\infty_0(B_{4})$,  the test function $\phi:= \varphi^2u^p$. So that
		\begin{equation}\label{app_eq_9}
			\int_{\Om}  \Dh u \cdot \Dh (\varphi^2u^p) \, {\rm d}\xi= \int_{\Om} f(\xi,u)  \varphi^2u^{p+1}\, {\rm d}\xi.
		\end{equation}
		Integrating by parts the first integral on the left-hand side in~\eqref{app_eq_9} and using Young's Inequality yield
		\begin{eqnarray*}
			&& \int_{\Om}  \Dh u \cdot \Dh (\varphi^2u^p) \, {\rm d}\xi\nonumber\\
			&&\qquad \qquad \geq -\,2\int_{\Om} \varphi u^p \snr{\Dh u} \snr{ \Dh \varphi} \, {\rm d}\xi + p\int_{\Om}  \varphi^2u^{p-1}\snr{\Dh u}^2 \, {\rm d}\xi\nonumber\\
			&&\qquad \qquad \geq \frac{p}{2}\int_{\Om} \varphi^2 u^{p-1} \snr{\Dh u}^2 \, {\rm d}\xi -\frac{2}{p}\int_{\Om} u^{p+1} \snr{\Dh \varphi}^2 \, {\rm d}\xi.
		\end{eqnarray*}
		Putting all together we have that
		$$
		\int_{\Om} \varphi^2 u^{p-1}\snr{\Dh u}^2 \, {\rm d}\xi \,\leq  \frac{2}{p}\int_{\Om} f(\xi,u)\varphi^2u^{p+1} \, {\rm d}\xi + \frac{4}{p^2}\int_{\Om} u^{p+1} \snr{\Dh \varphi}^2 \, {\rm d}\xi.
		$$
		Applying now Sobolev's Inequality yields that
		\begin{eqnarray}\label{app_eq_10}
			\|\varphi u^\frac{p+1}{2}\|_{L^{2^*}(\Om)}^2 &\leq & 2(S^*)^\frac{2}{2^*}\frac{(p+1)^2}{2}\int_{\Om} \varphi^2 u^{p-1}\snr{\Dh u}^2 \, {\rm d}\xi\notag\\
			&& + 2(S^*)^\frac{2}{2^*}\int_{\Om}  u^{p+1}\snr{\Dh \varphi}^2 \, {\rm d}\xi\nonumber\\
			&\leq & \underbrace{c(p+1)^2}\int_{\Om} \Big(f(\xi,u)\varphi^2+\snr{\Dh \varphi}^2\Big)u^{p+1} \, {\rm d}\xi
		\end{eqnarray}
		
		Using H\"older's Inequality with $q/2$ and $q/(q-2)$ and {the interpolative inequality
		\[
		\|g\|_{L^t} \leq \varepsilon\|g\|_{L^{\bar{r}}} + \eps^{-\sigma}\|g\|_{L^r}\,,
		\]
		for $r \leq t \leq \bar{r}$ and $\mu:= (\frac 1r -\frac 1{\bar{r}})/(\frac 1{t} - \frac 1{\bar{r}})$}, we obtain
		\begin{eqnarray*}
			\int_{\Om} f(\xi,u)\big(\varphi u^\frac{p+1}{2}\big)^2 \, {\rm d}\xi 
			&\leq &   \|f\|_{L^{q/2}(\Om)}  \|\varphi u^\frac{p+1}{2}\|_{L^{2q/(q-2)}(\Om)}^2\\
			&\leq &  \|f\|_{L^{q/2}(\Om)}\left( \epsilon\|\varphi u^\frac{p+1}{2}\|_{L^{2^*}(\Om)} +\epsilon^{-\frac{Q}{q-Q}}\| \varphi u^\frac{p+1}{2}\|_{L^{2}(\Om)}\right)^2, 
		\end{eqnarray*}
		Thus, by choosing~$\epsilon$ sufficiently small and absorbing the terms in the left-hand side in~\eqref{app_eq_10}, we have 
		\begin{equation}\label{app_eq_11}
			\|\varphi u^\frac{\vartheta}{2}\|_{L^{2(2^*/2)}(\Om)} \leq  c\,\vartheta^\frac{q}{q-Q}\|(\varphi+\snr{\Dh \varphi})u^\frac{\vartheta}{2}\|_{L^2(\Om)},
		\end{equation}
		where we denoted by~$\vartheta:=p+1$.
		We specify now the cut-off function~$\varphi$.  Let $ 1 \leq \sigma < \rho \leq 3$ and choose~$\varphi$ such that
		\[
		0 \leq \varphi \leq 1, \quad \varphi \equiv 1 \, \text{on}~B_\sigma, \quad \varphi \equiv 0 \, \text{on}~\h \setminus B_\rho, \quad \snr{\Dh \varphi} \leq \frac{2}{\rho-\sigma}.
		\]
		With such a choice of~$\varphi$, the estimate in~\eqref{app_eq_11} becomes
		\begin{equation}\label{app_eq_12}
			\| u^{\vartheta/2}\|_{L^{2(2^*/2)}(B_\sigma)} \, \leq \, \frac{c\,\vartheta^\frac{q}{q-Q}}{\rho-\sigma}\|u^{\vartheta/2}\|_{L^2(B_\rho)}.
		\end{equation}
		Thus, once defined~${\bm A}_{{\tt{q}},s}:= \left(\int_{B_s} u^{\tt{q}} \, {\rm d}\xi\right)^\frac{1}{{\tt{q}}}$, we have that the inequality in~\eqref{app_eq_12} becomes
		\begin{equation}\label{app_eq_13}
			{\bm A}_{\frac{2^*\vartheta}{2},\sigma}\, \leq \, \left(\frac{c\,\vartheta^\frac{q}{q-Q}}{\rho-\sigma}\right)^\frac{2}{\vartheta}{\bm A}_{\vartheta,\rho}.
		\end{equation}
		We are finally in the position to start a classical iteration method in order to get the desired supremum estimate. Taking~$\vartheta_j := (2^*/2)^j \vartheta$ and~$\rho_j:=1+2^{-j}$, for~$j =0,1,\dots$, we prove that 
		\begin{equation}\label{app_eq_14}
			{\bm A}_{\vartheta_N,\rho_N} \, \leq \, \prod_{j=0}^{N}\left(\frac{c\, \vartheta_j^\frac{q}{q-Q}}{\rho_j-\rho_{j+1}}\right)^\frac{2}{\vartheta_j}{\bm A}_{\vartheta,2} \qquad\text{for any}~N\geq1.
		\end{equation}
		Clearly the case~$N \equiv 1$ follows from~\eqref{app_eq_13}. We now assume that the estimate above holds for~$N$ and prove it for~$N+1$. Indeed, recalling~\eqref{app_eq_13} we have
		\begin{eqnarray*}
			{\bm A}_{\vartheta_{N+1},\rho_{N+1}} &\leq & \left(\frac{c\,\vartheta_N^\frac{q}{q-Q}}{\rho_{N}-\rho_{N+1}}\right)^\frac{2}{\vartheta_N}	{\bm A}_{\vartheta_N,\rho_N} \\
			&\leq & \left(\frac{c\,\vartheta_N^\frac{q}{q-Q}}{\rho_{N}-\rho_{N+1}}\right)^\frac{2}{\vartheta_N} \prod_{j=0}^{N}\left(\frac{c\, \vartheta_j^\frac{q}{q-Q}}{\rho_j-\rho_{j+1}}\right)^\frac{2}{\vartheta_j}	{\bm A}_{\vartheta,2}\\
			&=&\prod_{j=0}^{N+1}\left(\frac{c\, \vartheta_j^\frac{q}{q-Q}}{\rho_j-\rho_{j+1}}\right)^\frac{2}{\vartheta_j}	{\bm A}_{\vartheta,2}\,,
		\end{eqnarray*}	
		and the induction step does follow.
		
		Moreover, note that
		\[
		\prod_{j=0}^{\infty}\left(\frac{c\, \vartheta_j^\frac{q}{q-Q}}{\rho_j-\rho_{j+1}}\right)^\frac{2}{\vartheta_j} = \left(\frac{c\, \vartheta^\frac{q}{q-Q}}{\rho_0-\rho_{1}}\right)^\frac{2}{\vartheta}{\rm e}^{\, \sum_{j} \frac{2\log \Big(c\, 2^{j+1} \vartheta_j^\frac{q}{q-Q}\Big)}{\vartheta_j}},
		\]
		where we also used that 
		\[
	{	
		\begin{split}
\sum_{j=1}^\infty \frac{2\log \Big(c\, 2^{j+1} \vartheta_j^\frac{q}{q-Q}\Big)}{\vartheta_j} 
& = 	2\log c\sum_{j=1}^\infty \frac{1}{\vartheta_j} +	2\log2\sum_{j=1}^\infty \frac{j+1}{\vartheta_j}
\\
& \quad \quad + 	\frac{2q}{q-Q}\sum_{j=1}^\infty \frac{\log  \vartheta_j}{\vartheta_j}
<\infty.
		\end{split}}
		\]
		%	\]
		\vspace{2mm}
		
		Thus, letting~$N$ going to infinity in~\eqref{app_eq_14}, we eventually arrive at
		$$
		\sup_{B_1} u \leq  c \left( \int_{B_2}u^\vartheta \, {\rm d}\xi\right)^\frac{1}{\vartheta},
		$$
		which gives the desired estimate~\eqref{sup_est} choosing~$\vartheta=2^*$.
	\hfill $\square$

\vspace{2mm}
 \subsection{{H}-Kelvin transform}
 	We briefly recall  some notion about the {\it $H$-Kelvin transform} which will play an important role in the proof of~Theorem~\ref{han}. 
 	
 	
 	
 	\begin{defn}\label{def_kelvin}
 		For any~$\xi =(z,t) \in \h \setminus \{0\}$ we call \textup{$H$-inversion}  the map
 		\begin{eqnarray*}
 			&&\kappa_{\mathbb{H}}: \h \setminus \{0\} \longmapsto  \h \setminus \{0\},\\
 			&&\kappa_{\mathbb{H}} (\xi) := \left(\frac{z}{\snr{z}^2+i t}, \frac{t}{\snr{z}^4+t^2}\right).
 		\end{eqnarray*}
 		With~$\kappa_{\mathbb{H}}$ defined above, given a function~$u : \h \rightarrow \mathbb{R}$ we denote by~$u^\sharp$ its {\rm $H$-Kelvin transform} defined by
 		\begin{eqnarray*}
 			&& u^\sharp : \h \setminus \{0\} \rightarrow \mathbb{R},\\
 			&& u^\sharp (\xi) := \snr{\xi}_{\mathbb{H}}^{-(Q-2)}\, u \big(\kappa_{\mathbb{H}}(\xi)\big).
 		\end{eqnarray*}
 	\end{defn}
 	It can be easily verified that
 	\begin{equation}\label{H-inve-prop}
 		\kappa_{\mathbb{H}}(\kappa_{\mathbb{H}}(\xi))=\xi, \quad \textup{and} \quad \snr{\kappa_{\mathbb{H}}(\xi)}_{\mathbb{H}} = \snr{\xi}_{\mathbb{H}}^{-1}.
 	\end{equation}
 	
 	We would now need to present few properties of the $H$-Kelvin transform adapted to our framework.
 	\begin{prop}[See  Theorem~2.3.5 in \cite{IV11}]\label{H-Kelv-iso}
 		Let~$\Omega$ be a domain and denote by~${\Omega^{\sharp}}$ the image of~$\Omega$ under the $H$-inversion~$\kappa_{\mathbb{H}}$. Then, we have that the $H$-Kelvin transform is an isometry between $\Sc(\Om)$ and~$\Sc({\Omega^{\sharp}})$.
 	\end{prop} 
 	
 	\begin{prop}[See Lemma~2.3.6 in \cite{IV11}]\label{CR-lap}
 		Let~$u$ be a solution to
 		\begin{equation*}
 			\begin{cases}
 				-\Delta_{\mathbb{H}}   u = u^p,\\[1ex]
 				u \in \Sc(\Om), \quad u\ge0,
 			\end{cases}
 		\end{equation*}
 		for some positive exponent~$p>0$. Then, its $H$-Kelvin transform~$u^\sharp$ satisfies 
 		\begin{equation*}
 			\begin{cases}
 				-\Delta_{\mathbb{H}}   u^\sharp(\xi) = \snr{\xi}_{\mathbb{H}}^{p(Q-2)-(Q+2)}u^\sharp(\xi)^p,\\[1ex]
 				u^\sharp \in \Sc(\Om^\sharp), \quad u^\sharp\ge0.
 			\end{cases}
 		\end{equation*} 
 	\end{prop}
 	
 	Lastly we will take advantage of the maximum principle stated below.
 	
 	
 	\begin{prop}[See Proposition 1.3 in \cite{BP02}]\label{birindelli_max_prin}
 		Let $E$ be a {smooth} bounded domain on $\h$ and let $f \in L^\infty(E)$. Then, there exists $\bar{\delta}>0$ depending only on $n$ and  $\|f\|_{L^\infty(E)}$ such that the maximum principle holds for $\Delta_{\mathbb{H}}   +f$ provided that
 		$
 		\snr{E} < \bar{\delta}.
 		$ 
 \end{prop}



\vspace{1mm}
\subsection{Proof of Theorem~\ref{han}}
Consider a maximizing sequence~$\{u_\eps\}$ of~\eqref{sobolev}. Then it holds that
\begin{equation}\label{eq_han1}
	\int_\Om \snr{u_\eps}^{2^*-\eps}\, {\rm d}\xi = S^* + {o}(1), \qquad \text{as}~\eps \to 0,
\end{equation}
where we also used Proposition~2.5 in~\cite{PPT23}.

\vspace{2mm}
   {\bf Step 1. The sequence of the supremum norms~$\|u_\eps\|_{L^\infty}$ diverges; i.~\!e., 
	\begin{equation}\label{han_supremum_limit}
		\|u_\eps \|_{L^\infty(\Om)} \to \infty, \qquad \text{as}~\eps \to 0.
	\end{equation} 
}
	By contradiction assume that there exists a sequence~$\{\eps_k\}_k$ for which~$u_{\eps_k}$ remains bounded in~$\Om$ for~$\eps_k \to 0^+$ as~$k \to +\infty$.
	Hence, up to subsequences, we can assume that $u_{\eps_k} \to v \neq \infty$ uniformly on~$\Omega$. If the limit function~$v \equiv 0$, then by~\eqref{eq_han1} we have reached a contradiction since~$S^* \neq 0$. On the other hand, if~$v \ne 0$ we would have obtained a maximizer of~\eqref{critica}	which is a contradiction as well. Thus,~\eqref{han_supremum_limit} holds true. 

    \vspace{2mm}
    Choose now a sequence of points~$\{\eta_\eps\} \subseteq {\Omb}$ and a sequence of numbers~$\{\lambda_\eps\} \subseteq \mathbb{R}^+$ such that
    \begin{equation}\label{eq_han2}
	u_\eps(\eta_\eps) =\lambda_\eps^{-\frac{Q-2}{2}}\equiv\|u_\eps\|_{L^\infty(\Om)}.
    \end{equation}
    Consider the function
      \begin{equation}\label{han_v_eps}
	  v_\eps (\xi):= \lambda_\eps^\frac{Q-2}{2}u_\eps \bigg(\tau_{\eta_\eps}\Big( \delta_{\lambda_\eps^{1-\frac{Q-2}{4}\eps}}(\xi)\Big)\bigg),
      \end{equation}
       which is a weak solution to
	\begin{equation}\label{eq_han3}
		\begin{cases}
			-\Delta_{\mathbb{H}}   v_\eps = v_\eps^{2^*-1-\eps} & \text{in}~\Om_\eps := \delta_{\lambda_\eps^{\frac{Q-2}{4}\eps-1}}\Big(\tau_{\eta_\eps}^{-1}(\Om)\Big)\\[1ex]
			v_\eps ({0}) =1,\\[1ex]
			0\leq  v_\eps \leq 1.
		\end{cases}
	\end{equation}
	Indeed, by the homogeneity of the sub-Laplacian we get
	\begin{eqnarray*}
		-\Delta_{\mathbb{H}}   v_\eps(\xi) &=& \lambda_\eps^\frac{Q-2}{2} \lambda_\eps^{2-\frac{Q-2}{2}\eps}(-\Delta_{\mathbb{H}}   u_\eps) \bigg(\tau_{\eta_\eps}\Big( \delta_{\lambda_\eps^{1-\frac{Q-2}{4}\eps}}(\xi)\Big)\bigg)\\
		&=& \lambda_\eps^{\frac{Q+2}{2} -\frac{Q-2}{2}\eps}u_\eps^{2^*-1-\eps} \bigg(\tau_{\eta_\eps}\Big( \delta_{\lambda_\eps^{1-\frac{Q-2}{4}\eps}}(\xi)\Big)\bigg)\\
		&=& \lambda_\eps^{\frac{Q-2}{2}(\frac{Q+2}{Q-2} -\eps)}u_\eps^{2^*-1-\eps} \bigg(\tau_{\eta_\eps}\Big( \delta_{\lambda_\eps^{1-\frac{Q-2}{4}\eps}}(\xi)\Big)\bigg)\, =\, v_\eps^{2^*-1-\eps}\,.
	\end{eqnarray*}
	Also, recalling the choice of~$\{\eta_\eps\}$ and~$\{\lambda_\eps\}$ in~\eqref{eq_han2}, we have that
	\[
	v_\eps({0}) = \lambda_\eps^\frac{Q-2}{2}u_\eps(\eta_\eps) = 1,
	\]
     and
      that~$0 \leq v_\eps 
      \leq 1$ in~$\Om_\eps$. 
      
      Now, since the sequence~$\{v_\eps\}$ is bounded it is equicontinuous on compact subset of~$\h$, and by Ascoli-Arzel\`a's Theorem, up to subsequences, there exists a function~$v_\infty \not\equiv 0$ such that~$v_\eps \to v_\infty$ uniformly on compact set.

        \vspace{2mm}
	{\bf Step 2. As $\eps \to 0^+$, the sequence~$\{\lambda_\eps\}$ defined in~\eqref{eq_han2} satisfies
 \begin{equation}\label{bound_lambda_eps}
  0 <c \leq \lambda_\eps^\eps \leq 1, \qquad \textrm{with}~c\equiv c(n).
 \end{equation}
 }
First of all, since~$\lambda_\eps \to 0$ when~$\eps \to 0$, by~\eqref{eq_han2}, it trivially follows that~$\lambda_\eps^\eps < 1$, for~$\eps$ sufficiently small. Moreover, since the function~$v_\eps$ defined in~\eqref{han_v_eps} tends to~$v_\infty {\not \equiv}  0$  uniformly {on compact set and $v_\infty({0})=1$, it exists $\sigma>0$, sufficiently small, such that in~$B_\sigma({0}) \subset \Om_\eps$ and $v_\infty >0$}. Hence,  { by Fatou's Lemma}, we get that 
   { 
	\[
 \begin{split}
  \liminf_{\eps \to 0^+}\int_{B_\sigma({0})}v_\eps^{2^*-\eps} \, {\rm d}\xi & \geq \int_{B_\sigma({0})}\liminf_{\eps\to 0^+}v_\eps^{2^*-\eps} \, {\rm d}\xi\\
     & = \int_{B_\sigma({0})}v_\infty^{2^*} \, {\rm d}\xi=:c>0.
 \end{split}
	\]
 }
	Then, we have
	\begin{eqnarray*}
		\int_{B_\sigma({0})}v_\eps^{2^*-\eps} \, {\rm d}\xi & \stackrel{\eqref{han_v_eps}}{=}& \int_{B_\sigma({0})}\lambda_\eps^{\frac{Q-2}{2}\big(\frac{2Q}{Q-2}-\eps\big)} u_\eps^{\frac{2Q}{Q-2}-\eps} \Big(\tau_{\eta_\eps}\Big( \delta_{\lambda_\eps^{1-\frac{Q-2}{4}\eps}}(\xi)\Big)\Big)\, {\rm d}\xi\\
		&\leq &\lambda_\eps^{Q-\frac{Q-2}{2}\eps} \int_{\Om_\eps} u_\eps^{\frac{2Q}{Q-2}-\eps} \Big(\tau_{\eta_\eps}\Big( \delta_{\lambda_\eps^{1-\frac{Q-2}{4}\eps}}(\xi)\Big)\Big)\, {\rm d}\xi\\
		& \leq &  \lambda_\eps^{\frac{(Q-2)^2}{4}\eps}\int_{\Om} u_\eps^{2^*-\eps} (\xi)\, {\rm d}\xi,
	\end{eqnarray*}
	which, by { the previous computation gives 
 \[
  \liminf_{\eps \to 0^+} \lambda_\eps^\eps \stackrel{\eqref{eq_han1}}{\geq} \big(\frac{c}{S^*+{\rm o}(1)}\big)^\frac{4}{(Q-2)^2}>0,\,
 \]
 which is the desired estimate}.
 
      \vspace{2mm}
     {\bf Step 3. The asymptotics in~\eqref{conv_xi_eps} for the sequence $\{\eta_\eps\}$ holds true.} Recall that $\snr{\Dh u_\eps}^2 \, {\rm d}\xi \tows \boldsymbol{\delta}_{\xi_{\rm o}}$ in~$\mea$ for a given point~$\xi_{\rm o} \in \overline{\Om}$. Then, considering the function~$v_\eps$ in~\eqref{han_v_eps}, which converges to~$v_\infty \not \equiv 0$ uniformly on compact set, it yields { that, up to choosing $\rho>0$  large enough,}
    \begin{eqnarray*}
    	0 < \int_{B_\rho({0})} \snr{\Dh v_\infty}^2 \, {\rm d}\xi &\leq& \liminf_{\eps\to 0^+}\int_{B_\rho({0})}\snr{\Dh v_\eps}^2 \, {\rm d}\xi \\
    	&=& \liminf_{\eps \to 0^+} \big(\lambda_\eps^\eps\big)^\frac{(Q-2)^2}{4}\int_{B_{\rho\lambda_\eps^{1-\frac{Q-2}{4}\eps}}(\eta_\eps)}\, \snr{\Dh u_\eps}^2 \, {\rm d}\xi,
    \end{eqnarray*}
    which, in view of the  {asymptotic of $\{\lambda_\eps^\eps\}$ %, as $\eps \to 0^+$}, 
    in~\eqref{bound_lambda_eps}}
    gives a contradiction if~$\eta_\eps$ does not converge to~$\xi_{\rm o} \in \Omb$.


    \vspace{2mm}  
   {\bf Step 4. The concentration point~$\xi_{\rm o}$ is away from the boundary of~$\partial \Om$.} In order to prove this result, we will employ the maximum principle stated in Proposition~\ref{birindelli_max_prin}. 
   
    First, let us  show that there exists a direction~$\varsigma$, with $\snr{\varsigma}_{\mathbb{H}} =1$, along which $u_\eps$  decreases;  {i.~\!e., for any $\xi \in \Om$ it holds
    \[
     u_\eps (\delta_{\lambda_1}(\varsigma) \circ \xi ) -  u_\eps (\delta_{\lambda_2}(\varsigma) \circ \xi ) < 0 \quad \text{for any} \, 0 < \lambda_2 < \lambda_1 < \lambda_{\rm o}\,,
    \]
    for some $\lambda_{\rm o} >0$.} By contradiction assume that there is no such direction. Then, {for any~$\varsigma$} with~$\snr{\varsigma}_{\mathbb{H}}=1$,  denoting $u_{\lambda,\eps} := u_\eps(\tau_{\delta_\lambda(\varsigma)}(\cdot))$, we have that the function $u_{\lambda,\eps}$ is such that~$u_{\lambda,\eps} \geq u_\eps$ {for any $\lambda >0 $} ,   and it solves the following problem,
    \[
    \begin{cases}
    -\Delta_{\mathbb{H}}   u_{\lambda,\eps} = u_{\lambda,\eps}^{2^*-1-\eps}  & \text{in} \  \Om_\lambda := \tau_{(\delta_{\lambda}(\varsigma))^{-1}}(\Omega)\,,\\[1ex]
     u_{\lambda,\eps} =0 \, &\textrm{on} \ \partial \Om_\lambda.
    \end{cases}
     \]
Because of the boundedness of~$\Omega$ one has that there exists~$\lambda_{\rm o}\geq0$ such that~$\Om_{\lambda_{\rm o}} \cap \Omega = \emptyset$ and~$\Omega\cap\Om_\lambda \neq \emptyset$ for any~$\lambda \in [0,\lambda_{\rm o})$. We immediately notice that in the case when~$\lambda=0$, one has~$\Om_\lambda \equiv  \Omega$. For any~$\lambda <\lambda_{\rm o}$ let us consider the function~$w_{\lambda,\eps}$ defined as follows,
\[
w_{\lambda,\eps} := u_{\lambda,\eps}-u_\eps.
\]
Such a function is the solution of the following problem,
\[
\begin{cases}
	\Delta_{\mathbb{H}}   w_{\lambda,\eps} - f(\xi) w_{\lambda,\eps}\leq 0 \, & \textrm{in} \ \Om_\lambda \cap \Omega,\\[1ex]
	w_{\lambda,\eps} = 0 \, &\textrm{on} \ \partial (\Om_\lambda \cap \Omega),
\end{cases}
\]
where~$f  \in L^\infty(\Om_\lambda \cap \Omega)$. Letting~$\bar{\delta} >0$ be the one given by Proposition \ref{birindelli_max_prin} and choosing~$\lambda_{\bar{\delta}} $ such that 
\begin{equation}\label{delta}
|{\Om_\lambda \cap \Omega|}<\bar{\delta} \qquad \forall \lambda>\lambda_{\bar{\delta}},
\end{equation}
 we can apply the maximum principle to get that
\[
w_{\lambda,\eps} \geq 0 \qquad \textrm{in}~\Om_\lambda \cap \Omega.
\]
Moreover, by the strong maximum principle, we get in particular that
\[
w_{\lambda,\eps} >0 \qquad \textrm{in}~\Om_\lambda \cap \Omega.
\]
Now, define 
\[
\lambda_1 := \inf\big\{\lambda>0: w_{s,\eps} >0 \ \, \textrm{for any} \ s > \lambda \big\} \geq 0.
\]
We show that $\lambda_1 =0$. By contradiction assume that $\lambda_1 >0$. Note that $w_{\lambda_1,\eps}\geq 0$ on~$\Om_{\lambda_1}\cap \Omega$ and, by the strong maximum principle,~$w_{\lambda_1,\eps}>0$. 
Choose a compact set~$K \subset \Om_{\lambda_1} \cap \Omega$ such that 
\[
\snr{(\Om_{\lambda_1} \cap \Omega)\setminus K} < \bar{\delta}/3\,,
\]
and
\begin{equation}\label{pos_Ds}
w_{s,\eps} \geq 0 \qquad \text{on}~K \quad \, \text{for}~s <\lambda_1
\end{equation}
Fix now~$0<\lambda < \lambda_1$ such that
\begin{equation}\label{pos_sigma}
K \subset \Om_\lambda \cap \Omega \quad \textrm{and} \quad \snr{(\Om_\lambda \cap \Omega) \setminus K}<\bar{\delta}, 
\end{equation}
where~$\bar{\delta}$ is the one appearing in~\eqref{delta}.

Choose~$\lambda < \lambda_1$ sufficiently near to~$\lambda_1$  such  that~\eqref{pos_Ds} and~\eqref{pos_sigma} are satisfied for any~$s \in (\lambda,\lambda_1)$. This plainly leads to a contradiction. Indeed,~$w_{s,\eps} \geq 0$ on~$(\Om_s \cap \Omega) \setminus K$ by Proposition~\ref{birindelli_max_prin} which together with \eqref{pos_Ds} does imply that~$w_{s,\eps} \geq 0$ on~$\Om_s \cap \Omega$. Moreover, recalling that~$w_{s,\eps} {\not \equiv} 0$ and ~$s>\lambda>0$, from the strong maximum principle it follows that~$w_{s,\eps}>0$ for some~$s <\lambda_1$, which is a contradiction.
Therefore, we have that~$u_\eps(\tau_{\delta_{\lambda}(\varsigma)}(\cdot))$ is decreasing for any~$\lambda$~in $(0,\lambda_{\rm o})$. 
\vspace{2mm}

Now, {we can find $\sigma,\vartheta>0$ such that }
for any~$\xi \in \big \{\xi' \in \Omb: {\rm dist}(\xi',\partial \Om) <\vartheta\big\}$ there exists {a measurable set}~$K_\xi $ such that
\[
\snr{K_\xi}>{\sigma}, \quad K_\xi \subset \left\{\xi' \in \Omega: {\rm dist}(\xi',\partial \Om) > \frac{\vartheta}{2}\right \}, \quad \textup{and}~u_\eps(\xi') >  u_\eps(\xi), \,  \forall \xi ' \in K_\xi.
\]
Thus, being~$\{u_\eps\}$ a maximizing sequence for~$\Ssub$, we finally arrive at
\begin{eqnarray*}
u_\eps(\xi)\, & < &\, \dashint_{K_\xi}u_\eps \, {\rm d}\xi'\\
&\leq &\left(\dashint_{K_\xi}u_\eps^{2^*-\eps}\,{\rm d} \xi'\right)^\frac{1}{2^*-\eps} \ < \ \sigma^{-\frac{1}{2^*-\eps}}\big(S^* + {\rm o}(1)\big)^\frac{1}{2^*-\eps} \,,
 \end{eqnarray*}
 Then, since~$u_\eps (\eta_\eps) \to \infty$, it must follow that~$\eta_\eps$ is away from the boundary (up to choosing~$\eps$ small enough).
 
  \vspace{2mm}
   {\bf Step 5. The function $v_\eps$ in \eqref{han_v_eps} converges to~$U$ in~\eqref{talentiane_2} locally uniformly on compact sets.}
    As proven above we have that the sequence $\{v_\eps\}$ converges uniformly on compact set to $v_\infty$.  Moreover, if we denote with~$\Om_{\rm o}$ the limiting set for~$\eps \to 0$ of~$\Om_\eps$, we have that~$\Om_{\rm o} \equiv \h$.
     Indeed, thanks to Step~2 and Step~4 we have that the sequence 
     $
     \{\lambda_\eps^{(Q-2)\eps/4-1} {\rm dist}(\eta_\eps, \partial \Om)\}_{\eps}
     $
     is unbounded. Then, recalling~\cite[Lemma~3.4]{CU01} we have that the limiting space $\Om_{\rm o}$ does coincide with $\h$.
   
	
	This yields that~$v_\infty$ is a solution to 
	\[
	\begin{cases}
		-\Delta_{\mathbb{H}}   v_\infty = v_\infty^{2^*-1}, & \text{in}~\h,\\
		v_\infty({0}) =1,\\
		0\leq  v_\infty\leq 1,
	\end{cases}
	\]
	which implies that~$v_\infty$ coincides with the function~$U$, defined in~\eqref{talentiane_2}.   
	
	
	\vspace{2mm}
	{\bf Step 6. The sequence~$\{\lambda_\eps\}$ satisfies
	\begin{equation}\label{eq:lambda-log}
	\snr{\lambda_\eps^\eps -1} = \lambda_\eps^{\bar{t}\eps}\eps \snr{\ln \lambda_\eps}\,,
	\end{equation}		
    for some~$\bar{t} \in (0,1)$.
	}
    By means of the Mean Value Theorem we have that there exists $\vartheta \in (\lambda_\eps^\eps,1)$ such that
	\[
	\frac{1}{\vartheta}\,=\, \frac{1}{1-\lambda_\eps^\eps}\int_{\lambda_\eps^\eps}^1 \frac{1}{s} \, {\rm d}s
	\,=\, \frac{\ln \lambda_\eps^\eps}{\lambda_\eps^\eps-1}.
	\]
	Hence, considering $\bar{t}:= \ln\vartheta/\ln \lambda_\eps^\eps \in (0,1)$ we have that
	\[
	\snr{\lambda_\eps^\eps -1} = \lambda_\eps^{\bar{t}\eps}\eps \snr{\ln \lambda_\eps} ,
	\]
	by \eqref{bound_lambda_eps}, which gives the desired result.
	
	
	\vspace{2mm}
	{\bf Step 7. Useful estimates.}
	 Recalling the homogeneity of the horizontal gradient~$\Dh $ and the result in~\eqref{conv_xi_eps}; we have
		\begin{eqnarray}\label{PS1_1}
	\int_{\Om_\eps}\snr{\Dh v_\eps}^2 \, {\rm d}\xi &=&\int_{\Om_\eps}\lambda_\eps^{Q-\frac{Q-2}{2}\eps}\Big|(\Dh u_\eps) \Big(\tau_{\eta_\eps}\Big( \delta_{\lambda_\eps^{1-\frac{Q-2}{4}\eps}}(\xi)\Big)\Big)\Big|^2 \, {\rm d}\xi \notag\\*[0.5ex]
	&=& \lambda_\eps^{\frac{(Q-2)^2}{4}\eps}\int_\Om \snr{\Dh u_\eps}^2 \, {\rm d}\xi  \ \leq\  \lambda_\eps^{\frac{(Q-2)^2}{4}\eps},
	\end{eqnarray}
	as similarly done in Step~2.
    Also, by means of~\eqref{conv_xi_eps}, one has
    	\begin{eqnarray}\label{PS1_2}
    	\int_{\Om_\eps}\snr{v_\eps}^{2^*} \, {\rm d}\xi
    	\ =\ \lambda_\eps^{\frac{(Q-2)^2}{4}\eps}\int_\Om \snr{u_\eps}^{2^*} \, {\rm d}\xi\ =\ \lambda_\eps^{\frac{(Q-2)^2}{4}\eps} (S^* +{\rm o}(1)).
    \end{eqnarray}

    
    \vspace{2mm}
  {\bf Step 8. The asymptotic estimate in~\eqref{bound_max_seq} holds true.}
    We start by observing that proving~\eqref{bound_max_seq} is equivalent to show that the following estimate holds, for some universal constant $c>0$,
	\begin{equation}\label{boun_max_seq2}
		v_\eps (\xi) \leq C\, U(\xi) \ \qquad \text{in}~\Om_\eps.
		\end{equation}  		 
	Consider the $H$-Kelvin transform~$v^\sharp_\eps$ of~$v_\eps$, according to Definition~\ref{def_kelvin}; we now have that~\eqref{boun_max_seq2} follows from
	\begin{equation}\label{boun_max_seq3}
		v^\sharp_\eps(\xi) \leq  C \ \qquad \text{in}~\Om^\sharp_\eps\,,
	\end{equation}
    for a suitable (relabed) constant~$C\equiv C(n)$, which for shortness we relabel as~$C$, and where~$\Om^\sharp_\eps$ is the $H$-Kelvin transformed of~$\Om_\eps$. {Indeed, recalling Definition \ref{talentiane_2} of $U$ we have that \eqref{boun_max_seq2} is equivalent to say that
    \[
    \big((1+\snr{z}^2)^2+t^2\big)^\frac{Q-2}{4}v_\eps(\xi) \leq C.
    \]
    Hence, since $ \big((1+\snr{z}^2)^2+t^2\big)^\frac{Q-2}{4} \leq c +c\,\snr{\xi}_{\mathbb{H}}^{Q-2}$, for some dimensional constant $c \equiv c(n)>0$ and for any $\xi =(z,t)$, the  inequality above will follow from
    \[
    cv_\eps(\xi) + c\,\snr{\xi}_{\mathbb{H}}^{Q-2}v_\eps(\xi) \, \leq \, c + c\,\snr{\xi}_{\mathbb{H}}^{Q-2}v_\eps(\xi) \, \leq \, C\,,
    \]
    once recalled that $ 0 \leq v_\eps \leq 1$. Hence, applying the $H$-inversion map, by \eqref{H-inve-prop}, we have that the above estimate can be rewritten in the following way
    \[
    c+c\snr{\xi}_{\mathbb{H}}^{2-Q}v_\eps(\kappa_{\mathbb{H}}(\xi)) \, \leq\, C.
    \]
    Thus, recalling the definition of $H$-Kelvin transform we are left to prove \eqref{boun_max_seq3}.
    }
    Begin noticing that~$v_\eps \leq 1$ yields that~$v^\sharp_\eps \leq \snr{\xi}_{\mathbb{H}}^{-(Q-2)}$, and this will reduce the estimate for~$v^\sharp_\eps$ to be proven just near the origin. {Also, we observe that we can choose $\rr >0$ sufficiently small such that} 
    $
    {B}_{\rr\lambda_\eps^{(Q-2)\eps/4-1}} \equiv {B}_{\rr\lambda_\eps^{(Q-2)\eps/4-1}}({0}) \subset \Om_\eps\,,
    $
    Hence, applying the $H$-inversion map, we have that ${B}_{\rr\lambda_\eps^{(Q-2)\eps/4-1}}$ is mapped into $\mathcal{C}{B}_{\rr^{-1}\lambda_\eps^{1-(Q-2)\eps/4}}:=\h \setminus {B}_{\rr^{-1}\lambda_\eps^{1-(Q-2)\eps/4}} \subset \Om_\eps^\sharp$.
Since~$v_\eps$ satisfies~\eqref{eq_han3}, by Proposition~\ref{CR-lap}, its $H$-Kelvin transform is a solution to
	\[
			-\Delta_{\mathbb{H}}   v^\sharp_\eps(\xi) = \snr{\xi}_{\mathbb{H}}^{-(Q-2)\eps}\big(v^\sharp_\eps(\xi)\big)^{2^*-1-\eps} \qquad \text{in}~\mathcal{C}{B}_{\rr^{-1}\lambda_\eps^{1-(Q-2)\eps/4}}.
	\]
   Now, calling,~$a(\xi) := |\xi|_{\mathbb{H}}^{-(Q-2)\eps}$, we have that, by \eqref{bound_lambda_eps}, $a(\cdot)$ is uniformly bounded in $\mathcal{C}{B}_{\rr^{-1}\lambda_\eps^{1-(Q-2)\eps/4}}$ independently of~{$\eps$}.  {Indeed, for any $\xi \in \mathcal{C}{B}_{\rr^{-1}\lambda_\eps^{1-(Q-2)\eps/4}}$, it holds
   \[
   a(\xi) \leq \rr^{\eps(Q-2)}\lambda_\eps^{\frac{[(Q-2)\eps]^2}{4}-\eps(Q-2)}\,,
   \]
   which by \eqref{bound_lambda_eps} is bounded  as $\eps \searrow 0$.
   }  For  any $\sigma \leq \sigma_0$ and $B_{2\sigma} \equiv B_{2\sigma}({0})$, we have that
	\begin{eqnarray*}
      \int_{ \mathcal{C}{B}_{\rr^{-1}\lambda_\eps^{1-(Q-2)\eps/4}}  \cap B_{2\sigma}} (v^\sharp_\eps)^{2^*-\eps}\,{\rm d}\xi
       & \leq & \|v^\sharp_\eps\|_{L^{2^*}(\mathcal{C}{B}_{\rr^{-1}\lambda_\eps^{1-(Q-2)\eps/4}}  \cap B_{2\sigma}) }^{2^*-\eps}| \Om_\eps^\sharp \cap B_{2\sigma}|^\frac{\eps}{2^*}\notag\\
      & \leq &  \|v^\sharp_\eps\|_{L^{2^*}(\mathcal{C}{B}_{\rr^{-1}\lambda_\eps^{1-(Q-2)\eps/4}}  \cap B_{2\sigma}) }^{2^*-\eps}| B_{2\sigma}|^\frac{\eps}{2^*}\notag\\
     & = &  \|v_\eps\|_{L^{2^*}({B}_{\rr\lambda_\eps^{(Q-2)\eps/4 -1}})}^{2^*-\eps}|  B_{2\sigma}|^\frac{\eps}{2^*}\notag\\
     & \leq & {\|v_\eps\|_{L^{2^*}(\Om_\eps)}^{2^*-\eps}|  B_{2\sigma}|^\frac{\eps}{2^*}}\notag\\
     &\stackrel{\eqref{PS1_2}}{\leq} &
       \left(	\lambda_\eps^{\frac{(Q-2)^2}{4}\eps}S^* + {\rm o}(1)\right)^\frac{2^*-\eps}{2^*}|  B_{2\sigma_0}|^\frac{\eps}{2^*}
	\end{eqnarray*}
	where we have {also} used Proposition~\ref{H-Kelv-iso}.
	By taking~$\sigma_0$ sufficiently small we have that~\eqref{eq:first-gain} is satisfied for some~$\upsilon_0$, who can be chosen independently of~$\eps$.


     An application of Lemma~\ref{lemma:gain} (with $q=2^*-1$ there) yields 
	\begin{eqnarray}\label{eq_han4}
		\int_{\mathcal{C}{B}_{\rr^{-1}\lambda_\eps^{1-(Q-2)\eps/4}} \cap B_{\sigma}}(v^\sharp_\eps)^\frac{(2^*)^2}{2} \, {\rm d}\xi
		& \leq & {\frac{c}{\sigma^{2/2^*}}\|v_\eps^\sharp\|_{L^{2^*}(\Om_\eps^\sharp)}} \notag\\
		& = & {\frac{c}{\sigma^{2/2^*}}\|v_\eps\|_{L^{2^*}(\Om_\eps)}} \notag\\
			& \stackrel{\eqref{PS1_2}}{=} & {\frac{c}{\sigma^{2/2^*}}\big(\lambda_\eps^{\frac{[(Q-2)\eps]^2}{4}} S^* +{\rm o}(1)\big)^\frac{1}{2^*}. }
	\end{eqnarray}
      {In particular, the  constant appearing in the preceding  display  stays bounded as $\eps$ goes to 0.}
	Moreover,
	by~\eqref{eq_han4}
	 we get~$\xi \mapsto f(\xi,v^\sharp_\eps) \equiv \snr{\xi}_{\mathbb{H}}^{-(Q-2)\eps}(v^\sharp_\eps)^{2^*-2-\eps}\in L^{q/2}(\mathcal{C}{B}_{\rr^{-1}\lambda_\eps^{1-(Q-2)\eps/4}} \cap B_\sigma )$, a.~\!e., once chosen
	\[
	\frac{q}{2}:=\frac{(2^*)^2 }{2(2^*-2-\eps)} >\frac{(2^*)^2 }{2(2^*-2)} \,  =\,  \frac{Q}{Q-2}\frac{Q}{2}
	\, >\, \frac{Q}{2}.
	\]
	Indeed, by  \eqref{eq_han4}, we have
	\begin{eqnarray*}
	 &&\hspace{-9mm} \int_{\mathcal{C}{B}_{\rr^{-1}\lambda_\eps^{1-(Q-2)\eps/4}} \cap B_\sigma}\snr{f(\xi,v^\sharp_\eps)}^\frac{q}{2} \, {\rm d}\xi \\*[0.7ex]
	 && \quad \leq  {\int_{\mathcal{C}{B}_{\rr^{-1}\lambda_\eps^{1-(Q-2)\eps/4}} \cap B_\sigma}\snr{\xi}_{\mathbb{H}}^{-\frac{(Q-2)q\eps}{2}}(v^\sharp_\eps)^\frac{(2^*)^2}{2} \, {\rm d}\xi}\\*[0.7ex]
	 && \quad \leq {\rr^\frac{(Q-2)q\eps}{2} \lambda_\eps^{\frac{[(Q-2)\eps]^2q}{8}-\frac{(Q-2)q\eps}{2}}\int_{\mathcal{C}{B}_{\rr^{-1}\lambda_\eps^{1-(Q-2)\eps/4}} \cap B_\sigma}(v^\sharp_\eps)^\frac{(2^*)^2}{2} \, {\rm d}\xi}\\*[0.7ex]
	 && \quad \stackrel{\eqref{eq_han4}}{\leq} {\frac{c\,  \rr^\frac{(Q-2)q\eps}{2}}{\sigma^{2/2^*}}\lambda_\eps^{\frac{[(Q-2)\eps]^2q}{8}-\frac{(Q-2)q\eps}{2}}\big(\lambda_\eps^{\frac{[(Q-2)\eps]^2}{4}} S^* +{\rm o}(1)\big)^\frac{1}{2^*}}.
	\end{eqnarray*}
 {with the term on the right-hand side in the last display above being bounded as $\eps$ goes to 0, thanks to \eqref{bound_lambda_eps}}. 
    Thus, Lemma~\ref{lemma:sup} yields that
	\begin{eqnarray}\label{sup_est_v_diesis}
	\sup_{\mathcal{C}{B}_{\rr^{-1}\lambda_\eps^{1-(Q-2)\eps/4}} \cap B_\sigma} v^\sharp_\eps & \leq &  {\frac{c}{\sigma^{Q/2^*}}\|v_\eps^\sharp\|_{L^{2^*}(\Om_\eps^\sharp)}}\notag\\
	& = & {\frac{c}{\sigma^{Q/2^*}}\|v_\eps\|_{L^{2^*}(\Om_\eps)}}\notag\\
	& \stackrel{\eqref{PS1_2}}{=} & {\frac{c}{\sigma^{Q/2^*}}\big(\lambda_\eps^{\frac{[(Q-2)\eps]^2}{4}} S^* +{\rm o}(1)\big)^\frac{1}{2^*}   }
	\end{eqnarray}
    where the right-hand side in~\eqref{sup_est_v_diesis} is bounded as $\eps$ goes to 0.
    
   \vspace{2mm}
   {\bf Step 9. Conclusion.} Recalling the definition of $v_\eps$ in \eqref{han_v_eps} and the estimate in~\eqref{boun_max_seq2}, we obtain that
   \[
    u_\eps(\xi) \leq c\, \lambda_\eps^{-\frac{Q-2}{2}}U\big(\delta_{\lambda_\eps^{\frac{Q-2}{4}\eps -1}}(\tau_{\eta_\eps^{-1}}(\xi))\big)\,,
   \]
   which is the desired control stated in~\eqref{bound_max_seq}.
\hfill$\square$

\vspace{2mm}
\section{Proof of the localization result in Theorem \ref{thm_green}}\label{sec_localization}
 
This section is devoted to the proof of the localization result in Theorem~\ref{thm_green}. As mentioned in the introduction, for such a proof we will  need all the results proven in the previous sections and also  a few further independent results, as integral estimates for the horizontal derivatives and boundedness up to the characteristic set for the $\mathdutchcal{D}$-derivatives of~$u_\eps$.


\subsection{Boundary behaviour of subcritical extremals}\label{sec_decay}  

{In this section, we recall some results about the boundary behaviour of solutions to the subcritical CR Yamabe equation. {The results in the two forthcoming theorems  can be essentially obtained by their critical counterparts in \cite{GV00,Vas06}. For the sake of the reader, we present {them} as well with the needed modifications in the corresponding proofs.}
\vspace{2mm}
}


  Firstly, in view of the assumptions~($\Om1$)--($\Om3$), one can build fine subelliptic barriers as firstly seen in~\cite{GV00}. We have the following
    \begin{lemma}[See Theorem~4.3 in~\cite{GV00}]\label{barrier}
    	Let~$\Om$ be a smooth bounded domain of~$\h$ satisfying~{\rm(}$\Om1${\rm)}\textup{--}{\rm(}$\Om3${\rm)}. For any~$\alpha \in (0,1]$ define
    	\[
    	\varPsi_\alpha := (\rr_\Om-\varPhi)^\alpha e^{-\frac{\snr{z}^2}{M_\Om}}.
    	\]
    	Given a neighborhood~$K$ of the characteristic set~$\Car$ such that
    	\begin{equation}\label{cond_neighbor_char_set}
    		\overline{K} \subset \Big\{\snr{z}^2 \leq \frac{nM_\Om}{2}\Big\}, 
    	\end{equation}
    	we have that
    	\[
    	\Delta_{\mathbb{H}}   \varPsi_\alpha \, \leq \, -\frac{2n}{M_\Om} \varPsi_\alpha \, \  \textrm{on}~\omega:=\Om \cap K.
    	\]
    	Furthermore, there exist~$c_1,c_2 >0$ such that for any~$\eta \in \partial \Om \cap K$ and any $\lambda \in [\lambda_{\rm o},1]$ it holds
    	\[
    	c_1 (1-\lambda)^\alpha \, \leq \, \varPsi_\alpha(\delta_\lambda(\eta)) \, \leq \, c_2(1-\lambda)^\alpha.
    	\]
    \end{lemma}
    
    {Moreover, the result below will be needed in the rest of the present section.
    \begin{lemma}[Lemma~4.1 in~\cite{GV00}]\label{lemma:GV-4.1}
    For any $D \subset \h$, there exists a constant $c \equiv c (D)>0$ such that, for any $\eta \in D$ and $0 \leq \lambda \leq 1$, we have
    \[
    \snr{\eta^{-1}\circ \delta_{\lambda}(\eta)}_{\mathbb{H}} \leq c\, (1-\lambda)^\frac{1}{2}
    \]
    
    \end{lemma}
    }
    
    We now prove the main result of this section. We remark that we denote by~$K$ an open neighborhood  of the characteristic set~$\Car$ not containing the concentration point~$\xi_{\rm o}$ and with~$\omega := \Om \cap K$.
    \begin{theorem}\label{vassilev_type}
      Let~$\Om\subset\h$ be geometrical regular near its characteristic set  and let $u_\eps \in \Sc(\Om)$ be maximizer for $\Sob_\eps$. Then, {for~$0 < \eps$ sufficiently small,}  it holds that
        \[
        	\int_\gamma \snr{\Dh u_\eps}^2 \langle \mathdutchcal{D},\bm{n}\rangle \, {\rm d}\mathcal{H}^{Q-2} \,\leq\, c\, \mathcal{H}^{Q-2}(\gamma) \sup_{\Omb} \langle \mathdutchcal{D},\bm{n}\rangle,
      \]
       where~$\gamma$ is any {compact} hypersurface contained in~$\overline{\omega}$.
    \end{theorem}

   \vspace{2mm}
   Let us remark that by the uniformly~$\delta_\lambda$-starlikeness of~$\Om$  along its characteristic set~$\Car$ in~($\Om3$), up to taking a smaller neighborhood~$K$, we get 
   \begin{equation}\label{trasversality}
	\mathdutchcal{D}\varPhi (\eta) \geq c >0, \qquad \forall \eta \in \partial \Om \cap K,
    \end{equation}
   where~$\varPhi$ is the defining function of~$\Om$ in~($\Om1$) and~$\mathdutchcal{D}$ is defined in~\eqref{dilation_vector_field}.
 
   Condition~\eqref{trasversality} implies that the trajectories of~$\mathdutchcal{D}$ starting from~$\partial \Om \cap K$ fill a full open set interior of~$\Om$. Indeed, considering~$\eta \in \partial \Om \cap K$ and taking the Taylor expansion of~$f(\lambda):=\varPhi(\delta_\lambda(\eta))$ around~$\lambda =1$ we obtain that
   \[
   \rr_\Om-\varPhi(\delta_\lambda(\eta)) =\mathdutchcal{D}\varPhi(\eta)(1-\lambda) + {\rm o}(1-\lambda) \geq c(1-\lambda).
   \]   
   Hence, with no loss of generality, up to further shrinking~$K$, we assume there exists~$\lambda_{\rm o}$ such that
    \begin{equation}\label{trasversality_2}
    \delta_{\lambda}(\eta) \in \Om \cap K, \qquad \textrm{for}~\lambda_{\rm o}<\lambda <1.
   \end{equation}
 
   \vspace{2mm} 
   
   \begin{proof}[\bf Proof of Theorem~\ref{vassilev_type}.]
   	Consider an open neighborhood~$K$ of the characteristic set~$\Car$ not containing~$\xi_{\rm o}$ and such that~\eqref{trasversality_2} holds true for any~$\eta \in \partial\Omega \cap K$. With the notation above, we prove that {there exists $\bar{\eps}\equiv \bar{\eps}(n)>0$, such that, for any $\eps < \bar{\eps}$, it holds}
   	 \begin{equation}\label{u_eps_lips_omega}
   		u_\eps(\delta_\lambda(\eta)) \leq c (1-\lambda) \ \, \textrm{for any}~\eta \in \partial \Om \cap K,
   \end{equation}
    where~$\lambda$ is as in~\eqref{trasversality_2}; above the positive constant~{$c$ also depends on the dimension~$n$.} 
    

   	Indeed, {let us} fix~$\eps \in (0,2^*-2)$. By~{\eqref{bound_max_seq} in} Theorem~\ref{han} we get that there exists a dimensional costant~$c>0$ such that
   \[
    	u_\eps \leq c\, \lambda_\eps^{-\frac{Q-2}{2}} U_{\eps} + {{\rm o}(1)}\ \, \qquad \text{on}~\omega :=\Om \cap   K,~\text{for any}~\eps < \bar{\eps}.
   \]
    Hence,~$u_\eps \in L^\infty(\omega)$,  since~$U_\eps$ stays bounded in~$\omega$, giving that~$\xi_{\rm o} \not \in \omega$. 
    For this, we can deduce that~$u_\eps \in C^\infty(\omega)$.
    
    Moreover, using condition~($\Om2$) one can adapt to the present setting the classical Moser iteration argument to get~$u_\eps \in C^{0,\alpha_\eps}(\partial \Om \cap K)$, for some exponent~$1>\alpha_\eps >  0$. Thus,
    $
    u_\eps \in C^{0,\alpha_\eps}(\overline{\omega})\cap C^\infty(\omega).
   $
   
    Now, since~$u_\eps \equiv 0$ on~$\partial \Om$, we have that for any~$\eta \in \partial \Om \, {\cap \, K}$ {it holds}
    \[
    u_\eps(\delta_\lambda(\eta)) \leq {c} \,  \snr{ \eta^{-1}\circ \delta_\lambda(\eta)}_{\mathbb{H}}^{\alpha_\eps}.
    \]
    {Moreover, by Lemma~\ref{lemma:GV-4.1}, we have that there exist a dimensional constant~$c>0$ such that}
    $
    \snr{\eta^{-1}\circ \delta_\lambda(\eta)}_{\mathbb{H}} \leq c(1-\lambda)^{1/2}.
    $
   Thus, {for any $\eta \in \partial \Om \cap K$ and any $\lambda \in [\lambda_{\rm o},1]$, we get that}
    \begin{equation}\label{hold_bound_u_eps}
    u_\eps(\delta_\lambda(\eta)) \leq c (1-\lambda)^\frac{\alpha_\eps}{2}.
    \end{equation}
    Note that, for any~$\eps < 2^*-2$, we have that~$2^*-1-\eps >1$. Hence, we can choose~$m \in \mathbb{N}$ such that~$(2^*-1-\eps )^{-m} \leq \alpha_{{\eps}}/2$. Since the estimate in~\eqref{hold_bound_u_eps} does hold for any~$\eta \in \partial \Om \cap K$ and any~$\lambda \in [\lambda_{\rm o},1]$, we can assume that the points~$\delta_\lambda(\eta)$ cover~$\overline{\omega}$. Thus, {on $\overline{\omega}$ we have}
    \begin{equation}\label{refine_hld_u_eps}
     u_\eps \leq c(1-\lambda)^{(2^*-1-\eps )^{-m}}.
    \end{equation}
    Hence, up to taking a smaller neighborhood~$K$ such that~\eqref{cond_neighbor_char_set} is satisfied, we have that for~$\delta_\lambda(\eta)$, by~\eqref{refine_hld_u_eps} and Lemma~\ref{barrier}, it holds
    \begin{eqnarray*}
    -\Delta_{\mathbb{H}}   u_\eps (\delta_\lambda(\eta)) &=& u(\delta_\lambda(\eta))^{2^*-1-\eps}\\
                                    &=& c^{2^*-1-\eps}(1-\lambda)^{(2^*-1-\eps )^{1-m}}\\
                                    &\leq & c^{2^*-1-\eps}c_1^{-1} \varPsi_{(2^*-1-\eps )^{1-m}}(\delta_\lambda(\eta)) \\
                                 &   \leq & -c^{2^*-1-\eps}c_1^{-1} \frac{M_\Om}{2n}\Delta_{\mathbb{H}}   \varPsi_{(2^*-1-\eps )^{1-m}}(\delta_\lambda(\eta))\\
                                    &=: & - \Delta_{\mathbb{H}}  (c_{\rm o}\varPsi_{(2^*-1-\eps )^{1-m}})(\delta_\lambda(\eta)).
    \end{eqnarray*} 
   Thus, 
   \begin{equation}\label{before_max_princ}
   	\Delta_{\mathbb{H}}  (c_{\rm o}\varPsi_{(2^*-1-\eps )^{1-m}}-u_\eps) \leq 0 \quad  \text{in} \quad \omega.
   \end{equation}
   Moreover, proceeding as in~\cite[Theorem~5.14]{Vas06}, we get
   \begin{equation}\label{boundary_max_prin}
   	c_{\rm o}\varPsi_{(2^*-1-\eps )^{1-m}} \geq u_\eps \, \ \textrm{on}~\partial \omega.
   \end{equation}
   Combining together~\eqref{before_max_princ} with~\eqref{boundary_max_prin}, by the maximum principle, {on $\overline{\omega}$} we obtain
   \[
   u_\eps(\delta_\lambda(\eta)) \, \leq \,  c_{\rm o}\varPsi_{(2^*-1-\eps )^{1-m}}  \, \leq \,  c_{\rm o}c_2 (1-\lambda)^{(2^*-1-\eps )^{1-m}} \,,
   \]
   which is a refinement of~\eqref{refine_hld_u_eps}. Iterating this procedure $m$-times yields
   \[
   		u_\eps(\delta_\lambda(\eta)) \leq c (1-\lambda) \ \, \textrm{for any}~\eta \in \partial \Om \cap K.
   \]  
  Fix now a {compact} hypersurface~$\gamma \subset \overline{\omega}$, and cover it with a family~$\{B_{2^{-i}}^{(i)}\}_{i=1}^\vartheta$ such that $B_{2^{-i}}(\xi_i)\equiv B_{2^{-i}}^{(i)} \subset \overline{\omega}$, for any~$i=1,\dots,\vartheta$.
  By the interior estimate in~\cite[Corollary~3.2]{BGM19}, we have that on every ball~$B_{2^{-i}}^{(i)}$ it holds, {for any~$\lambda \in [\lambda_{\rm o},1]$,}
   \begin{eqnarray}\label{interior_est}
   	\|\Dh u_\eps\|_{L^\infty(B_{2^{-i}}^{(i)})} &\leq & {2^{i}c}\Big(\|u_\eps\|_{L^\infty(\overline{\omega})}+2^{-2i}\|u_\eps\|_{L^\infty(\overline{\omega})}^{2^*-1-\eps}\Big)\notag\\
   	&\stackrel{\eqref{u_eps_lips_omega}}{\leq} & c\,\sum_{i=1}^{\vartheta}\Big((1-\lambda)+2^{-2i}(1-\lambda)^{2^*-1-\eps}\Big) =: C \,< \,\infty.
   \end{eqnarray}
   	Moreover,  by the $\delta_\lambda$-starlikeness we have that 
  \begin{equation}\label{fourth}
  0  \, < \, \int_\gamma \langle \mathdutchcal{D},\bm{n}\rangle \, {\rm d}\mathcal{H}^{Q-2} \,  \leq  \,   \,\mathcal{H}^{Q-2}(\gamma) \sup_{\Omb} \langle \mathdutchcal{D},\bm{n}\rangle\,;
  \end{equation}
  {note that, being ${\bm n}$ the outward normal, by Cauchy-Schwartz's Inequality we also have that the last term on the right-hand side in the preceding estimate is bounded from above, since $\mathdutchcal{D}$ is a vector field with smooth coefficients and $\Omb$ is bounded.}
  
  Thus, combining~\eqref{interior_est} and~\eqref{fourth} with a standard covering argument yields
  \begin{equation*}
	\int_\gamma \snr{\Dh u_\eps}^2 \langle \mathdutchcal{D},\bm{n}\rangle {\rm d}\mathcal{H}^{Q-2}  \, \leq \,  c\, \mathcal{H}^{Q-2}(\gamma) \sup_{\Omb} \langle \mathdutchcal{D},\bm{n}\rangle\,,
 \end{equation*}
  where~$c$ does not depend on~$\gamma$.
\end{proof}


We conclude this section by noticing that, in view of the hypotheses as in~($\Om1$)--($\Om3$), it readily follows the boundedness of the $\mathdutchcal{D}$-derivatives {basically generalizing the argument developed in~\cite{GV00}. For the sake of ease, we will present the proof in our specific subcritical setting.
}

\begin{theorem}\label{vassilev_type_2}
 Let~$\Om\subset\h$ be geometrical regular near its characteristic set  and let $u_\eps \in \Sc(\Om)$ be maximizer for $\Sob_\eps$.  Then, {or~$0 < \eps$ sufficiently small,}  it holds that
      \[
       	\mathdutchcal{D}u_\eps \in L^\infty(\overline{\omega}),
      \]
with~$\omega$ being an interior neighborhood of the characteristic set~$\Car$.
\end{theorem}
  
  
\begin{proof}[\bf Proof]
By extending $u_\eps$ to be equal $0$ outside $\Om$, we have that, by standard subelliptic regularity properties, that
\[
u_\eps \in C^{0,\alpha_\eps}(\h) \cap C^\infty (\Om).
\]

Moreover, let us consider the fundamental solution $K (\cdot)$ of the sub-Laplacian $\Delta_{\mathbb{H}} $. Hence, if we define $v:= u_\eps^{2^*-1-\eps} \star K(\cdot)$ {-- see the Appendix on Page~\pageref{sec_app} for the related definition of convolution--} we have that $v$ is a solution to $\Delta_{\mathbb{H}}  v = u_\eps^{2^*-1-\eps}$ in $\h$. Also, since $u_\eps \in C^{0,\alpha_{\eps}}(\h)$ and has compact support from \cite[Theorem~6.1]{Fol75} we obtain that
\[
v \in C^{2,\alpha_\eps}_{\loc}(\h).
\]
Define now $w:= u_\eps-v$. Recall that we denote with $K$ an open neighborhood of the characteristic set $\Sigma(\Om)$ and with $\omega := \Omega \cap K$. Moreover, let $D:= \partial \Om \, \cap \, K$. Hence, for a given $\lambda \approx 1$, let us define

\vs
$\bullet$ $\omega_\lambda: = \delta_\lambda(\omega) \, \cap \, \omega$;

\vs
$\bullet$ $D_\lambda := \delta_\lambda (D)$;

\vs 
$\bullet$ the different quotient
\[
\phi_\lambda(\xi) := \frac{w(\xi)-w(\delta_\frac{1}{\lambda}(\xi))}{1-\frac{1}{\lambda}} \qquad \xi \in \omega_\lambda\,,
\]
and where $w$ is defined above.

Let us assume that there exists a constant $c>0$ such that, for $\lambda \approx 1$ and for $\xi \in \omega_\lambda$ it holds
\begin{equation}\label{eq:claim-phi-lambda}
\snr{\phi_\lambda(\xi)} \leq c.
\end{equation}
If condition~\eqref{eq:claim-phi-lambda} holds true than, passing to the limit in $\lambda\to 1$ yields that $\snr{\mathdutchcal{D}w}\leq c$, which proves the theorem.

Thus, we are left with the proof of \eqref{eq:claim-phi-lambda}. Let us start noticing that $\phi_\lambda$ is $H$-harmonic. Indeed,
\begin{eqnarray*}
\Delta_{\mathbb{H}}  \phi_\lambda & = & \frac{\Delta_{\mathbb{H}}  w(\xi)-\Delta_{\mathbb{H}} \big( w(\delta_\frac{1}{\lambda}(\xi))\big)}{1-\frac{1}{\lambda}}\\
& = & \frac{\Delta_{\mathbb{H}}  w(\xi)- \lambda^{-2}\Delta_{\mathbb{H}}  w(\delta_\frac{1}{\lambda}(\xi))}{1-\frac{1}{\lambda}} =0.
\end{eqnarray*}
Hence, from maximum principle, it is enough to prove \eqref{eq:claim-phi-lambda} for $\xi \in \partial \omega_\lambda$, for some $\lambda \in [\lambda_1,1]$ sufficiently near $1$. 
Start noticing that $\partial \omega_\lambda:= D_\lambda \cup (\partial\omega_\lambda \setminus D_\lambda)$ and separately consider the two parts of the boundary. 

Let us start with $D_\lambda$. Since on $D_\lambda$ any point $\xi$ can be written as $\delta_\lambda(\eta)$ for some $\eta \in D$, we have that (since $u_\eps =0$ on $\partial \Om$)
\begin{eqnarray*}
\snr{\phi_\lambda(\xi)} & = & \left|\lambda \frac{w(\delta_{\lambda}(\eta))-w(\eta)}{\lambda-1}\right|\\
& =  & \left|\lambda \frac{u_\eps(\delta_{\lambda}(\eta))-u_\eps(\eta) + v(\eta) -v(\delta_{\lambda}(\eta))}{\lambda-1}\right|\\
& \leq & \frac{u_\eps(\delta_{\lambda}(\eta))}{1-\lambda} + \left|\frac{v(\delta_{\lambda}(\eta))-v(\eta)}{1-\lambda}\right|\\
& \stackrel{\eqref{u_eps_lips_omega}}{\leq}& c + \left|\frac{v(\delta_{\lambda}(\eta))-v(\eta)}{1-\lambda}\right|.
\end{eqnarray*}

Moreover, by \cite[Theorem~5.26]{Fol75} we have that $C^{2,\alpha_{{\eps}}}_{\loc}(\h) \subset C^{1,\frac{\alpha_{{\eps}}}{2}}_{\loc}(\h)$, hence we can deduce that $v$ is locally Lipschitz continuous. This yields that
\[
\snr{v(\delta_{\lambda}(\eta))-v(\eta)} \leq c\, (1-\lambda).
\]
Then,~\eqref{eq:claim-phi-lambda} holds on $D_\lambda$. For the remaining part of the boundary let us note that $\partial \omega_\lambda \setminus D_\lambda$ is uniformly away from the characteristic sets. Hence \eqref{eq:claim-phi-lambda} follows from Lipschitz continuity estimate in a uniform neighborhood of such set.
\end{proof}  
 
 
\vspace{1mm}
\subsection{Proof of the localization result}

We are finally in the position to present the proof of Theorem~\ref{thm_green}, whose argument involves different techniques and results such as the asymptotic control via the Jerison and Lee optimal functions established in Theorem~\ref{han}, the Pohozaev identity, the regularity theory for the subcritical CR Yamabe equation in Theorem~\ref{vassilev_type_2} as well as the integral estimate in Theorem~\ref{vassilev_type}. Moreover, we will use the negligibility of the characteristic set 

\begin{theorem}[See Theorem~1.2 in \cite{Der71}]  \label{characteristic_set_negligible}
	Let~$\Om \subset \h$ be a~$C^\infty$ domain and let~$\Car$ be its characteristic set. Then,
	\[
	\mathcal{H}^{Q-2}(\Car)=0,
	\]
\end{theorem}
\vspace{2mm}

\mbox{}
\\ {\bf Proof of Theorem~\ref{thm_green}.}
 For the sake of readability, we divide the proof into several steps.



\vspace{2mm}

{\bf Step. 1 For any $\eps \in (0,\bar{\eps})$ it holds
\begin{equation}\label{pohozaev_sub_approx}
\frac{\eps(Q-2)}{2^*-\eps}\int_{\Om} u_\eps^{2^*-\eps} \, {\rm d}\xi = \int_{\partial \Om}\snr{\Dh u_\eps}^2 \langle \mathdutchcal{D},\bm{n}\rangle \,{\rm d}\mathcal{H}^{Q-2}.
\end{equation}
}
We start recalling that, since $\Om$ is a smooth domain, the characteristic set $\Car$ is compact. Moreover, thanks to Theorem \ref{characteristic_set_negligible} and Theorem~1.1 in \cite{Xu99}, we can consider an exhaustion of~$\Om$ of~$C^\infty$~connected open sets $\{\Om_i\}$ such that $\Om_i \uparrow \Om$, $u_\eps \in C^2(\Omb_i)$ and $\partial \Om_i = \gamma^{(1)}_i \cup \gamma^{(2)}_i$ with $\gamma^{(1)}_i \subset\partial \Om \setminus \Car$, $\gamma^{(1)}_i \uparrow \partial \Om \setminus \Car$ and $\mathcal{H}^{Q-2}(\gamma^{(2)}_i) \to 0$.

We apply the Pohozaev identity {of Lemma~\ref{pohozaev}} on $u_\eps$ in $\Om_i$, getting
\begin{eqnarray}\label{loc_pohozaev_1}
	&& \int_{\Om_i} \big(\frac{2Q}{2^*-\eps}u_\eps^{2^*-\eps}-(Q-2)u_\eps^{2^*-\eps}\big) \, {\rm d}\xi \notag\\
	&& \qquad = 2 \sum_{j=1}^{2n}\int_{\gamma^{(1)}_i \cup \gamma^{(2)}_i}\mathdutchcal{D}u_\eps \, Z_j u_\eps \langle Z_j,\bm{n}\rangle \,{\rm d}\mathcal{H}^{Q-2} -\int_{\gamma^{(1)}_i \cup \gamma^{(2)}_i}\snr{\Dh u_\eps}^2 \langle \mathdutchcal{D},\bm{n}\rangle \,{\rm d}\mathcal{H}^{Q-2}\notag\\
	&& \qquad = 2 \sum_{j=1}^{2n}\int_{\gamma^{(1)}_i }\mathdutchcal{D}u_\eps \, Z_j u_\eps \langle Z_j,\bm{n}\rangle \,{\rm d}\mathcal{H}^{Q-2} -\int_{\gamma^{(1)}_i}\snr{\Dh u_\eps}^2 \langle \mathdutchcal{D},\bm{n}\rangle \,{\rm d}\mathcal{H}^{Q-2}\\
	&&\qquad \quad +\,2 \sum_{j=1}^{2n}\int_{\gamma^{(2)}_i}\mathdutchcal{D}u_\eps \, Z_j u_\eps \langle Z_j,\bm{n}\rangle \,{\rm d}\mathcal{H}^{Q-2} -\int_{\gamma^{(2)}_i}\snr{\Dh u_\eps}^2 \langle \mathdutchcal{D},\bm{n}\rangle \,{\rm d}\mathcal{H}^{Q-2}.\notag
\end{eqnarray}
Note that since $u_\eps >0$ in $\Om_i$ and $u_\eps =0$ on $\gamma^{(1)}_i$, then there exists a function $w\leq 0$ such that $Du_\eps = w \bm{n}$ on $\gamma^{(1)}_i$. Hence, on~$\gamma^{(1)}_i$ we can write
\begin{eqnarray*}
	\sum_{j=1}^{2n}\mathdutchcal{D}u_\eps \, Z_j u_\eps \langle Z_j,\bm{n}\rangle &=& w \langle \mathdutchcal{D}, \bm{n}\rangle \sum_{j=1}^{2n} Z_j u_\eps \langle Z_j,\bm{n}\rangle \\
	&=& \langle \mathdutchcal{D}, \bm{n}\rangle \sum_{j=1}^{2n} Z_j u_\eps \langle Z_j,w\bm{n}\rangle\\
	&=& \langle \mathdutchcal{D}, \bm{n}\rangle \sum_{j=1}^{2n} Z_j u_\eps \underbrace{\langle Z_j,Du_\eps\rangle}_{=:Z_j u_\eps}= \snr{\Dh u_\eps}^2\langle \mathdutchcal{D}, \bm{n}\rangle.
\end{eqnarray*}
Then, putting the computation above inside the equality in~\eqref{loc_pohozaev_1} yields
\begin{eqnarray*}
	\frac{\eps(Q-2)}{2^*-\eps}\int_{\Om_i} u_\eps^{2^*-\eps} \, {\rm d}\xi &=& \int_{\gamma^{(1)}_i}\snr{\Dh u_\eps}^2 \langle \mathdutchcal{D},\bm{n}\rangle \,{\rm d}\mathcal{H}^{Q-2}\notag\\
	&& +\,2 \sum_{j=1}^{2n}\int_{\gamma^{(2)}_i}\mathdutchcal{D}u_\eps \, Z_j u_\eps \langle Z_j,\bm{n}\rangle \,{\rm d}\mathcal{H}^{Q-2}\notag\\
	&&-\,\int_{\gamma^{(2)}_i}\snr{\Dh u_\eps}^2 \langle \mathdutchcal{D},\bm{n}\rangle \,{\rm d}\mathcal{H}^{Q-2}.
\end{eqnarray*}
Now recalling Theorem \ref{vassilev_type} we have that
\[
\int_{\gamma^{(2)}_i}\snr{\Dh u_\eps}^2 \langle \mathdutchcal{D},\bm{n}\rangle \,{\rm d}\mathcal{H}^{Q-2} \approx  \mathcal{H}^{Q-2}(\gamma^{(2)}_i),
\]
 by Theorem \ref{vassilev_type_2} the $\mathdutchcal{D}$-derivative of $u_\eps$ are bounded near the characteristic set, {and by recalling also that $Z_j u_\eps$ stays bounded up to the characteristic set, for any $j \in \{1,\dots,2n\}$, which can be deduced by following Theorem~4.6 in~\cite{GV00}}. Thus, passing to the limit as~$i\to \infty$, recalling the $\delta_\lambda$-starlikeness of $\Om$ and the estimate in~\eqref{interior_est}, by the Dominate Convergence and the Monotone Convergence theorem we eventually arrive at the desired estimate~\eqref{pohozaev_sub_approx}. 

\vspace{2mm}

{\bf Step 2. The following limit holds,
\begin{equation}\label{conv_xi_eps-final}
\lim_{\eps \to 0^+} \lambda_\eps^\eps =1.
\end{equation}
}
We start proving that there exists a constant~$c\equiv c(n,\Om)>0$ such that	
\begin{equation}\label{decay_Du_eps}
\int_{\partial \Om} \snr{\Dh  u_\eps}^2 \langle \mathdutchcal{D}, {\bm n}\rangle \, {\rm d}\mathcal{H}^{Q-2} \leq c\,\lambda_\eps^{Q-2-\frac{(Q-2)^2}{2}\eps}.
\end{equation}
Consider the Taylor polynomial~$P_\xi$ of~$u_\eps$ with center in~$\xi \in \partial \Om \setminus \Car$. As proven in~\cite[Formula~(4.55)]{BGM19} for any point~$\varsigma$ such that
\[
\snr{\varsigma^{-1}\circ \xi}_{\mathbb{H}} ={\rm dist}(\varsigma,\partial \Om) = {\bar{\sigma}} \quad {\text{and} \quad \varsigma \in B_{\rr}(\eta)\subset \Om}
\]
for~${\bar{\sigma}}, {\rr}>0$, it holds 
\begin{eqnarray*}
\snr{\Dh  u_\eps(\xi)} &\leq & \snr{\Dh  u_\eps(\varsigma)-\Dh P_\xi} +\snr{\Dh  u_\eps(\varsigma)}\notag\\
&\leq & \frac{c}{{\bar{\sigma}}}\big( \|u_\eps-P_\xi\|_{L^\infty(B_{a{\bar{\sigma}}}(\varsigma))} + \,{\bar{\sigma}}^2\|u_\eps\|_{L^\infty(B_{a{\bar{\sigma}}}(\varsigma))}^{2^*-1-\eps} \big) +\snr{\Dh  u_\eps(\varsigma)}.
\end{eqnarray*}
where,  for~$a>0$,~$B_{a{\bar{\sigma}}}(\varsigma)$ is a  non-tangential balls from inside of~$\Om$. 

Such a ball can be constructed since we are considering non-characteristic points. 
Indeed, up to left translations we assume that~$\xi \equiv 0$. Moreover, since~$0$ is not characteristic there exists~$j \in \{1,\dots,2n\}$ such that
\[
\langle Z_j, {\bm n}\rangle (0) \neq 0.
\]
Now, by an orthogonal transformation and the implicit function theorem, we can assume the existence of~$\rr_0>0$ such that~$\Om \cap B_{\rr_0}$ can be represented as
\[
\big\{x_{2n}>\widetilde{\varPhi}(x',t)\big\} \, \ \text{where}~x':=(x_1,\dots,x_{2n-1}) \in \r^{2n-1},
\]
for a Lipschitz function~$\widetilde{\varPhi}$ such that~$\widetilde{\varPhi}(0)=0$,~$\nabla_{x'}\widetilde{\varPhi}(0)=0$. In view of the Lipschitz continuity of the function~$\widetilde{\varPhi}$, we can assert that, up to taking~$\lambda$ sufficiently small, the ball~$B_{\lambda s} (\delta_\lambda(e_{2n}))$ is strictly contained in~$\Om$, for~$s>0$  small enough. 
We also refer the reader to the proof of Proposition~3.3 in~\cite{BGM19} where a non-tangential ball from outside was determined, and to the proof of Theorem~7.6 in~\cite{DGP07}.
\vs

{For any interior ball $ B_\rr(\eta) \subset \Om$, not containing the concentration point $\xi_{\rm o}$, let us choose $\bar{\sigma}$ sufficiently small so that $B_{a\bar{\sigma}}(\varsigma) \subset B_\rr(\eta)$. Then, by the interior estimate of Corollary~3.2 in~\cite{BGM19}, we get that}
\begin{eqnarray}\label{eq:sup-grad}
\snr{\Dh  u_\eps(\xi)}
& \leq &\frac{c}{{\bar{\sigma}}}\big( \|u_\eps-P_\xi\|_{L^\infty(B_{a{\bar{\sigma}}}(\varsigma))} + {\bar{\sigma}}^2\|u_\eps\|_{L^\infty(B_{a{\bar{\sigma}}}(\varsigma))}^{2^*-1-\eps} \big) \\
	&& {+ \, \frac{c}{\rr}\big(  \|u_\eps\|_{L^\infty(B_{\rr}(\eta))} + {\rr}^2\|u_\eps\|_{L^\infty(B_{\rr}(\eta))}^{2^*-1-\eps} \big) } \,,         \notag
\end{eqnarray} 

 
Now, {note that for any}~${\bar{\sigma}}$ we have that
\begin{equation}\label{726}
\frac{c}{{\bar{\sigma}}} \|u_\eps-P_\xi\|_{L^\infty(B_{a{\bar{\sigma}}}(\varsigma))}\, \leq \, c {\bar{\sigma}}^\beta \,,
\end{equation}
Note that the first estimate in the display above comes from the proof of Theorem~1.1 in~\cite{BGM19}; see in particular Page~26  there, and note also that any~$\beta$ in $(0,1)$ can be chosen in view of the smoothness of the set~$\Omega$.


\vspace{1mm}
{Since $\xi_{\rm o} \not \in B_\rr(\eta)$, we have that   the interior asymptotic estimate~\eqref{bound_max_seq} implies that $u_\eps = {\rm o}(\lambda_\eps^{(Q-2)/2 -(Q-2)^2\eps/4})$ when $\xi \neq \xi_{\rm o} $. Then \eqref{eq:sup-grad} and \eqref{726} yields that, for~$\eps$ sufficiently small, it holds
\[
\snr{D_{\mathbb{H}} u_\eps (\xi)}  \leq  {\rm o}(\bar{\sigma}) + {\rm o}_\eps(\bar{\sigma}) + {\rm o}(\lambda_\eps^{(Q-2)/2 -(Q-2)^2\eps/4})\,,
\]	
where with ${\rm o}_\eps(\bar{\sigma})$ we indicate a quantity that for any fixed $\eps>0$, satisfies ${\rm o}_\eps(\bar{\sigma}) \to 0^+$ when $\sigma \to 0^+$.
Hence, passing to the limit $\bar{\sigma}\to 0^+$ yields that}
$
    \snr{\Dh  u_\eps(\xi)} =  {\rm o}(\lambda_\eps^{(Q-2)/2 -(Q-2)^2\eps/4})$ 
for a.~\!e.~$\xi \in \partial \Om \setminus \Car$, which implies the desired estimate in~\eqref{decay_Du_eps}, recalling that, by the divergence theorem since ${\rm div}~\mathdutchcal{D} =Q$, it holds
\[
\int_{\partial \Om} \langle \mathdutchcal{D},{\bm n}\rangle \, {\rm d}\mathcal{H}^{Q-2} = \int_{\Om} \text{div}~\mathdutchcal{D} \,{\rm d}\xi = Q \snr{\Om}.
\]
Now, by \eqref{pohozaev_sub_approx} {and \eqref{eq_han1}}, we obtain that
\[
\eps \leq c\, \lambda_\eps^{{Q-2} -\frac{(Q-2)^2}{2}\eps}\,,
\]
which gives 
\[
\snr{\lambda_\eps^\eps-1} = {\rm O}(\lambda_\eps^{Q-2}\snr{\ln \lambda_\eps})\,,
\]
once recalled \eqref{eq:lambda-log} and {the asymptotic of $\lambda_\eps^\eps$ in } \eqref{bound_lambda_eps}. Hence, the desired \eqref{conv_xi_eps-final} follows.



\vspace{2mm}
{\bf Step 3. The following limit holds true, as $\eps \to 0^+$
\begin{equation}\label{han_lim_green}
	\|u_\eps\|_{L^\infty(\Omega)}u_\eps \to  \frac{\omega_{2n}}{2n}\frac{\sqrt{\pi}{\bm \Gamma}(\frac{n+1}{2})}{{\bm \Gamma}(\frac{n+2}{2})}\, G_{\Om}(\cdot,\xi_{\rm o}) \, \ \text{in}~\Gamma^{1,\beta}(\partial \Om \setminus \Car)\,,
\end{equation}
for~$\beta\in(0,1)$, where ${\bm \Gamma}(x)$ is Euler's Gamma function and $\omega_{2n}$ is the $(2n-1)$-dimensional measure of the Euclidean $2n$-sphere..
}\vspace{1mm}

Define the function
\[
w_\eps := \|u_\eps\|_{L^\infty(\Om)}u_\eps^{2^*-1-\eps} \equiv \lambda_\eps^{-\frac{Q-2}{2}}u_\eps^{2^*-1-\eps}.
\]
We show that~$w_\eps \to \bar{c} \boldsymbol{\delta}_{\xi_{\rm o}}$ in the sense of distributions. Indeed, for any $\phi \in C^\infty_0(\Om)$, we have that
\begin{eqnarray}\label{han_lim_green2}
\int_{\Om} \phi w_\eps \, {\rm d}\xi &=& \lambda_\eps^{-\frac{Q-2}{2}}\int_{\Om}\phi u_\eps^{2^*-1-\eps} \, {\rm d}\xi \nonumber\\*[0.7ex]
&=& \lambda_\eps^{-\frac{({Q-2})^2}{4}\eps}\int_{\Om_\eps}\phi(\tau_{\eta_\eps}(\delta_{\lambda_\eps^{1-\frac{Q-2}{4}\eps}}(\eta))v_\eps^{2^*-1-\eps}(\eta)\, {\rm d}\eta \notag\\*[0.7ex]
&\to & \phi(\xi_{\rm o})\int_{\h}U^\frac{Q+2}{Q-2} \, {\rm d}\xi =: \phi(\xi_{\rm o})\bar{c},
\end{eqnarray}
where we have used the definition in~\eqref{han_v_eps}, the fact that since~$v_\eps \to U$ uniformly on compact set (by Ascoli-Arzel\`a's Theorem), 
and $v_\eps \to 0$ when $\snr{\xi}_{\mathbb{H}} \to \infty$ uniformly in $\eps$, given that $\Om_\eps \to \h$, {we have that {it also holds on} $\h$}, and~ the asymptotics in~\eqref{conv_xi_eps-final} for $\lambda_\eps^\eps$. Moreover, note that when~$\xi \neq \xi_{\rm o}$, by~\eqref{conv_xi_eps-final}, we get that

\begin{eqnarray*}
     w_\eps  & = & \|u_\eps\|_{L^\infty(\Om)}u_\eps^{2^*-1-\eps}\\
     & = & \lambda_\eps^{-\frac{Q-2}{2}}u_\eps^{2^*-1-\eps}\\
     & \leq & c\, \lambda_\eps^{-\frac{Q-2}{2} -\frac{Q-2}{2}(2^*-1-\eps)}U\big(\delta_{\lambda_\eps^{\frac{Q-2}{4}\eps -1}}(\tau_{\eta_\eps^{-1}}(\xi))\big)^{2^*-1-\eps}
\,   =\,  {\rm o}(\lambda_\eps)\,,  
\end{eqnarray*}
which yields that
\begin{equation}\label{han_lim_green3}
      w_\eps \to 0 \qquad \text{as} \, \eps \to 0^+.
\end{equation}
Thus, combining~\eqref{han_lim_green2} and~\eqref{han_lim_green3} we obtain that~$w_\eps \to \bar{c}\boldsymbol{\delta}_{\xi_{\rm o}}$ in the sense of distributions.

Furthermore, note that the function~$\|u_\eps\|_{L^\infty(\Om)}u_\eps$ is a solution to
\[
\begin{cases}
-\Delta_{\mathbb{H}}   \big(\|u_\eps\|_{L^\infty(\Om)}u_\eps\big) = \|u_\eps\|_{L^\infty(\Om)}u_\eps^{2^*-1-\eps} &\text{in}~\Om\cap \omega,\\[1ex]
\|u_\eps\|_{L^\infty(\Om)}u_\eps =0 &\text{in}~\partial \Om \cap \omega,
\end{cases}
\]
where~$\omega$ is an interior neighborhood of~$\partial \Om \setminus \Car$. 

Consider a ball~$B_\rr$ such that
$\rr>0$ sufficiently small so that $B_\rr \cap \Om$ does not contain~$\xi_{\rm o}$. By~\eqref{han_lim_green3} since $\|u_\eps\|_{L^\infty(\Om)}u_\eps =0$ on~$\partial \Om$, we have that the hypotheses in Theorem~\ref{boundary_hld} are satisfied. Thus, the sequence~$\{\|u_\eps\|_{L^\infty(\Om)}u_\eps \}$ is compact in~$\Gamma^{1,\beta}({B_\rr\cap \partial\Om})$, for some~$\beta\in (0,1)$, which in fact is arbitrary because of the smoothness of~$\Om$. Moreover, from~\eqref{han_lim_green2} and~\eqref{han_lim_green3} it has to converge to~$\bar{c} G_{\Om}(\cdot,\xi_{\rm o})$. Indeed, for any test function~$\phi \in C^\infty_0(\Om)$ it holds
\begin{eqnarray*}
-\int_{\Om}\|u_\eps\|_{L^\infty(\Om)}u_\eps\Delta_{\mathbb{H}}  \phi \, {\rm d}\xi &=& 	-\int_{\Om}\big(\Delta_{\mathbb{H}}  \|u_\eps\|_{L^\infty(\Om)}u_\eps\big) \phi \, {\rm d}\xi\\	
&=& \int_{\Om} w_\eps \phi \, {\rm d}\xi\\
&\to& \int_{\Om}	\bar{c}\boldsymbol{\delta}_{\xi_{\rm o}}\phi \, {\rm d}\xi = -\int_{\Om}\bar{c}G_{\Om}(\cdot,{\xi_{\rm o}})\big(\Delta_{\mathbb{H}}  \phi\big) \, {\rm d}\xi.
\end{eqnarray*}
Hence,~\eqref{han_lim_green} follows taking an open covering of~$\partial \Om \setminus \Car$ and computing the explit value of the constant $\bar{c}$ as follows
\begin{eqnarray*}
    \bar{c} & = & \int_{\r^{2n+1}}\frac{{\rm d}z {\rm d}t}{\big((1+\snr{z}^2)^2+t^2\big)^\frac{Q+2}{4}}\notag\\*[0.7ex]
    & =& \int_{\r^{2n}}\frac{{\rm d}z}{(1+\snr{z}^2)^{n+1}}\int_{\r}\frac{{\rm d}t}{(1+t^2)^{\frac{n}{2}+1}}
    \\*[1ex]
    & = & \frac{\omega_{2n}}{2n}\frac{\sqrt{\pi}{\bm \Gamma}(\frac{n+1}{2})}{{\bm \Gamma}(\frac{n+2}{2})}\,,
\end{eqnarray*}
where ${\bm \Gamma}(x)$ is Euler's Gamma function, $\omega_{2n}$ is the {$(2n-1)$-dimensional measure of the} Euclidean $2n$-sphere, and where we have also used that
\[
\int_{\r^k}\frac{{\rm d}\zeta}{(1+\snr{\zeta}^2)^\alpha}\, = \, \frac{\snr{\omega_k}}{2}\frac{{\bm \Gamma}(\frac k2){\bm \Gamma}( \alpha - \frac k2)}{{\bm \Gamma}(\alpha)}\,.
\]




\vspace{3mm}
{\bf Step~4. For any~$k \in \{1,\dots,2n\}$ it holds
\begin{equation}\label{eq:han-pohozaev}
\int_{\partial\Om} |\Dh u_\eps|^2 \langle Z_k , \bm{n}\rangle \,{\rm d}\mathcal{H}^{Q-2} =0.
\end{equation}}
Consider the equation  
\begin{equation}\label{eq:EL-critica}
\begin{cases}
-\Delta_{\mathbb{H}} u_\eps = u_\eps^{2^*-1-\eps} & \text{in} \, \Om,\\
u_\eps =0 & \text{on} \, \partial \Om\,,
\end{cases}
\end{equation}
and lets $\{\Om_i\}$ be the exhaustion of $\Om$ of $C^\infty$ connected open set of the previous step; i.~\!e., such that $\Om_i \uparrow \Om$, $u_\eps \in C^2(\overline{\Om_i})$ and $\partial \Om_i = \gamma^{(1)}_i \cup \gamma^{(2)}_i$ with $\gamma^{(1)}_i \subset\partial \Om \setminus \Car$, $\gamma^{(1)}_i \uparrow \partial \Om \setminus \Car$ and $\mathcal{H}^{Q-2}(\gamma^{(2)}_i) \to 0$ as $i\to\infty$.

{Now, we note that on $\Om_i$ the function $u_\eps$ is $C^2(\Omb_i)$. Also since $u_\eps$ is a solution to~\eqref{eq:EL-critica}
we get that
\[
\int_{\Om} u_\eps^{2^*-1-\eps}\phi \,{\rm d}\xi = \int_\Om \Dh u_\eps \Dh \phi \,{\rm d}\xi \qquad \forall \phi \in \Sc(\Om).
\]
Moreover, integrating the right-hand side by parts yields
\[
\int_{\Om} u_\eps^{2^*-1-\eps}\phi \,{\rm d}\xi = \int_\Om (-\Delta_{\mathbb{H}} u_\eps) \phi \,{\rm d}\xi \qquad \forall \phi \in \Sc(\Om)
\]
and thus
\[
\int_{\Om} \left(\Delta_{\mathbb{H}}u_\eps+u_\eps^{2*-1-\eps}\right)\phi\,{\rm d}\xi =0 \qquad \forall \phi \in \Sc(\Om),
\]
which by Lemma~\ref{lemma:app} yields that $-\Delta_{\mathbb{H}}u_\eps= u_\eps^{2^*-1-\eps}$ holds a.~\!e.~in~$\Om$. Then, by the regularity of $u_\eps$ in $\Om_i$ yields that $-\Delta_{\mathbb{H}}u_\eps= u_\eps^{2^*-1-\eps}$ holds pointwise in $\Om_i$. Multiplying it with $Z_k u_\eps$ on both sides and integrating over $\Om_i$ yields
 }
\begin{eqnarray}\label{eq:integration}
&&-\int_{\Om_i} \Delta_{\mathbb{H}}   u_\eps Z_k  u_\eps \, {\rm d}\xi\notag\\
&&\qquad =   -\sum_{j=1}^{2n}\int_{\Om_i} Z_j^2 u_\eps Z_k  u_\eps \, {\rm d}\xi\notag\\
&&\qquad = \sum_{j=1}^{2n}\int_{\Om_i} Z_j u_\eps Z_j Z_k u_\eps \, {\rm d}\xi - \sum_{j=1}^{2n}\int_{\partial\Om_i} Z_j u_\eps Z_k  u_\eps \langle Z_j,\bm{n}\rangle \, {\rm d}\mathcal{H}^{Q-2}\notag\\
&&\qquad =  \sum_{j=1}^{2n}\int_{\Om_i} Z_j u_\eps Z_k  Z_j u_\eps \, {\rm d}\xi - \sum_{j=1}^{2n}\int_{\partial\Om_i} Z_j u_\eps Z_k  u_\eps \langle Z_j,\bm{n}\rangle \, {\rm d}\mathcal{H}^{Q-2}\notag\\
&&\quad \qquad -\, 4\int_{\Om_i} Z_{k'} u_\eps  T u_\eps \, {\rm d}\xi\notag\\*[0.6ex]
&&\qquad =  \frac{1}{2}\sum_{j=1}^{2n}\int_{\Om_i} Z_k  \big(Z_j u_\eps\big)^2 \, {\rm d}\xi - \sum_{j=1}^{2n}\int_{\partial\Om_i} Z_j u_\eps Z_k  u_\eps \langle Z_j,\bm{n}\rangle \, {\rm d}\mathcal{H}^{Q-2}\\
&&\quad \qquad + \, 4\int_{\Om_i} TZ_j u_\eps   u_\eps \, {\rm d}\xi - 4\int_{\partial\Om_i} Z_{k'} u_\eps   u_\eps \langle T,\bm{n} \rangle  \, {\rm d}\mathcal{H}^{Q-2}\notag\,,
\end{eqnarray}
where we have integrated by parts and used the fact that~$[Z_k ,Z_j] = - 4 T$ whenever~$|k-j| =n$, and $[Z_k ,Z_j] = 0$ otherwise; thus $k'$ being such that $|k'-k|=n$.
Again, since~$u_\eps >0$ in~$\Om$ and $u_\eps =0$ on~$\partial \Om_i$, then there exists a function $w\leq 0$ such that $Du_\eps = w \bm{n}$ on $\gamma^{(1)}_i $. Hence,
\[
\begin{split}
\sum_{j=1}^{2n}Z_j u_\eps \, Z_k  u_\eps \langle Z_j,\bm{n}\rangle &= w \langle Z_k , \bm{n}\rangle \sum_{j=1}^{2n} Z_j u_\eps \langle Z_j,\bm{n}\rangle \\
&= \langle Z_k , \bm{n}\rangle \sum_{j=1}^{2n} Z_j u_\eps \langle Z_j,w\bm{n}\rangle\\
&= \langle Z_k , \bm{n}\rangle \sum_{j=1}^{2n} Z_j u_\eps \underbrace{\langle Z_j,Du_\eps\rangle}_{=:Z_j u_\eps}= \snr{\Dh u_\eps}^2\langle Z_k , \bm{n}\rangle.
\end{split}
\]
Moreover, since~$[T,Z_j] = 0 $ for any~$j \in \{1,\dots,2n\}$ and~$u=0$ on the boundary~$\gamma^{(1)}_i $, from~\eqref{eq:integration} we get
\begin{eqnarray*}
&& \hspace{-5mm}-\int_{\Om_i} \Delta_{\mathbb{H}}   u_\eps Z_k  u_\eps \, {\rm d}\xi \notag\\
&& \quad =  \frac{1}{2}\sum_{j=1}^{2n}\int_{\Om_i} Z_k  \big(Z_j u_\eps\big)^2 \, {\rm d}\xi - \int_{\gamma^{(1)}_i }|\Dh u_\eps |^2\langle Z_k ,\bm{n}\rangle \, {\rm d}\mathcal{H}^{Q-2} \notag\\
&&\qquad- 4\int_{\gamma^{(2)}_i} Z_{k'} u_\eps   u_\eps \langle T,\bm{n} \rangle  \, {\rm d}\mathcal{H}^{Q-2} - \sum_{j=1}^{2n}\int_{\gamma^{(2)}_i} Z_j u_\eps Z_k  u_\eps \langle Z_j,\bm{n}\rangle \, {\rm d}\mathcal{H}^{Q-2} \notag\\
&& \quad =  \frac{1}{2}\int_{\gamma^{(1)}_i} |\Dh u_\eps |^2\langle Z_k ,\bm{n}\rangle \, {\rm d}\mathcal{H}^{Q-2} - \int_{\gamma^{(1)}_i}|\Dh u_\eps |^2\langle Z_k ,\bm{n}\rangle \, {\rm d}\mathcal{H}^{Q-2}\notag\\
 &&\qquad-4 \int_{\gamma^{(2)}_i} Z_{k'} u_\eps   u_\eps \langle T,\bm{n} \rangle  \, {\rm d}\mathcal{H}^{Q-2} - \sum_{j=1}^{2n}\int_{\gamma^{(2)}_i} Z_j u_\eps Z_k  u_\eps \langle Z_j,\bm{n}\rangle \, {\rm d}\mathcal{H}^{Q-2} \notag\\
 && \qquad+ \frac{1}{2}\int_{\gamma^{(2)}_i} |\Dh u_\eps|^2  \langle Z_k ,\bm{n}\rangle \, {\rm d}\mathcal{H}^{Q-2}\notag\\*[0.7ex]
 && \quad =-\frac{1}{2} \int_{\gamma^{(1)}_i}|\Dh u_\eps |^2\langle Z_k ,\bm{n}\rangle \, {\rm d}\mathcal{H}^{Q-2}-4 \int_{\gamma^{(2)}_i} Z_{k'} u_\eps   u_\eps \langle T,\bm{n} \rangle  \, {\rm d}\mathcal{H}^{Q-2}\\
 &&\qquad - \sum_{j=1}^{2n}\int_{\gamma^{(2)}_i} Z_j u_\eps Z_k  u_\eps \langle Z_j,\bm{n}\rangle \, {\rm d}\mathcal{H}^{Q-2}  + \frac{1}{2}\int_{\gamma^{(2)}_i} |\Dh u_\eps|^2  \langle Z_k ,\bm{n}\rangle \, {\rm d}\mathcal{H}^{Q-2}\notag
\end{eqnarray*}
after integrating by parts; $k'$ being such that $|k'-k|=n$.  Hence, passing to the limit as $i \to \infty$, by an analogous argument as that in Step~2, {by recalling also that $Z_j u_\eps$ stays bounded up to the characteristic set, for any $j \in \{1,\dots,2n\}$, which can be deduced by following Theorem~4.6 in~\cite{GV00},} we obtain
\begin{equation}\label{eq:integration-2}
-\int_{\Om} \Delta_{\mathbb{H}}   u_\eps Z_k  u_\eps \, {\rm d}\xi 	=  -\frac{1}{2} \int_{\partial\Om}|\Dh u_\eps |^2\langle Z_k ,\bm{n}\rangle \, {\rm d}\mathcal{H}^{Q-2}.
\end{equation}
Moreover, denoting with~$f(u):= u_\eps^{2^*-1-\eps}$ and with~$F(u):= \displaystyle\int_0^u f(\tau) \, {\rm d}\tau$, we have
\begin{eqnarray}\label{eq:integration-3}
\int_\Om f(u)Z_k  u \,{\rm d}\xi &=& \int_\Om Z_k  \big( F(u)\big) \,{\rm d}\xi\notag\\
&=& \int_{\partial \Om}F(u) \langle Z_k , \bm{n}\rangle \,{\rm d}\mathcal{H}^{Q-2} =0\,,
\end{eqnarray}
recalling that~$u_\eps =0$ on the boundary~$\partial \Om$. 

Then,~\eqref{eq:han-pohozaev} follows combining together~\eqref{eq:integration-2} with~\eqref{eq:integration-3}.


\vspace{2mm}
{\bf Step 5. Let us prove that, for any~$k \in \{1,\dots,2n\}$,
\begin{equation}\label{eq:robin-green-1}
\int_{\partial \Om} \snr{\Dh G_{\Om}(\cdot,\xi_{\rm o})}^2 \langle Z_k  ,\bm{n}\rangle \, {\rm d}\mathcal{H}^{Q-2} = - Z_k \mathcal{R}_\Om(\xi_{\rm o})\,,
\end{equation}
where~$\mathcal{R}_\Om(\cdot)$ is the Robin function.
}{The proof of identity \eqref{eq:robin-green-1} firstly appear in the Euclidean framework in~\cite[Theorem~4.3]{BP89}. Here further care is needed in order to carefully deal with the underlying geometry of $\h$ as well as with the possible lacking of regularity at the characteristic part of the boundary of $\Om$.}

\vs
With no loss of generality let~$\xi_{\rm o}= {0}$. 
For any~$\rr>0$ consider~$\boldsymbol{\delta}^{(\rr)}:= \chi_{B_\rr}/|B_\rr|$; we have that {the function $\boldsymbol{\delta}^{(\rr)}$}  converges weakly$^*$ to~$\boldsymbol{\delta}_{0}$ as $\rr\to 0$. Indeed, for any~$\phi \in C^0_0(\Om)$ we have that
\[
\begin{split}
\lim_{\rr \to 0^+}\int_\Om \boldsymbol{\delta}^{(\rr)}\phi \,{\rm d}\xi = \lim_{\rr \to 0+} \dashint_{B_\rr({0})}\phi\,{\rm d}\xi = \phi({0})\,,
\end{split}
\]
for the Lebsegue-Besicovitch theorem.
 Now let~$v_\rr$ be the  solution to
{\[
-\Delta_{\mathbb{H}} v = {\bm \delta}^{(\rr)} \qquad \text{in}~\h\,,
\]
such that $v_\rr \to 0$ as $\snr{\xi}_{\mathbb{H}}\to \infty$. Note that by \cite[Corollary~2.8]{Fol75} such function can be built by $v_\rr = {\bm\delta}^{(\rr)}\star K(\cdot)$}
where~$K(\cdot)$ is Folland's fundamental solution with pole in ${0}$. { Then we have
\[
v_\rr(\xi) := \int_{\h} K(\eta^{-1}\circ \xi){\bm \delta}^{(\rr)}(\eta)\,{\rm d}\eta = \dashint_{B_\rr(0)}K(\eta^{-1}\circ \xi)\,{\rm d}\eta = \dashint_{B_\rr(\xi)} K(\eta)\,{\rm d}\eta.
\]

{Moreover, note that 
\begin{eqnarray}\label{eq:sym-v}
v_\rr(\xi^{-1}) & = & \dashint_{B_\rr(0)}K(\eta^{-1}\circ \xi^{-1})\,{\rm d}\eta \notag\\
&= & -\dashint_{B_\rr(0)}K(\eta\circ \xi^{-1})\,{\rm d}\eta \notag\\
&= & -\dashint_{B_\rr(0)}K(\eta^{-1}\circ \xi)\,{\rm d}\eta = - v_\rr(\xi)\,,
\end{eqnarray}
by symmetry of the fundamental solution~$K(\cdot)$.
}


Then, by Lebesgue-Besicovitch Theorem, for a.~\!e. $\xi \in \h$, one get
\[
\lim_{\rr\to 0^+}v_\rr = K(\xi).
\]
Moreover, for any $\xi \in\partial \Om$ such that -- up to choosing $\rr>0$ sufficiently small -- it holds that $0 \not \in B_\rr(\xi)'$, we have that
\begin{equation}\label{eq:est-v_rr}
	\snr{v_\rr(\xi)} \leq \dashint_{B_\rr(\xi)}\snr{K(\xi)}\,{\rm d}\xi \leq \sup_{\xi \in\partial\Om}\snr{K(\xi)} <\infty
\end{equation}
given that $0 \not \in \partial \Om$.
}

\vspace{1mm}
Now, consider the solution~$u_\rr$ to
\begin{equation}\label{eq:green-aprox}
	\begin{cases}
		-\Delta_{\mathbb{H}}   u_\rr = \boldsymbol{\delta}^{(\rr)} & \quad \text{in}~\Om\,,\\
		u_\rr=0 & \quad \text{on}~\partial\Om\,.
	\end{cases}
\end{equation} 
{We start noticing that the function $u_\rr - v_\rr$ satisfies the following problem
\begin{equation}\label{eq:robin-approx}
\begin{cases}
	-\Delta_{\mathbb{H}}   (u_\rr-v_\rr) = 0 & \quad \text{in}~\Om\,,\\
	u_\rr-v_\rr=-v_\rr & \quad \text{on}~\partial\Om.
\end{cases}
\end{equation}
Hence, by Bony's Maximum Principle we have that
\begin{equation}\label{eq:bound-robin-approx}
\sup_{\Om} \snr{u_\rr-v_\rr} \, \leq  \, \sup_{\partial \Om} \snr{u_\rr-v_\rr} \, \leq  \, \sup_{\partial \Om}v_\rr  \, \stackrel{\eqref{eq:est-v_rr}}{\leq} \,  \sup_{\partial\Om}K
\end{equation}
which is finite since ${0}$ is an interior point of $\Om$. Then, up to choosing $\rr>0$ sufficiently small, so that given an interior neighborhood $\omega$ of $ \partial \Om \setminus\Sigma(\Om)$ it holds $B_\rr(\xi) \subset \omega$ and $0 \not \in \omega$, we have that
\begin{eqnarray}\label{eq:bound-green-approx}
\sup_{\omega} \snr{u_\rr} & = & \sup_{\omega} \snr{u_\rr-v_\rr +v_\rr}\notag\\*
& \leq & \sup_{\omega}\snr{u_\rr-v_\rr} + \sup_{\omega}\snr{v_\rr}\notag\\*
&\leq & \sup_{\Omega}\snr{u_\rr-v_\rr} + \sup_{\omega}\snr{v_\rr}\notag\\*
& \leq & \sup_{\partial \Om} \snr{u_\rr-v_\rr} + \sup_{\omega}\snr{v_\rr}\notag\\*
& \stackrel{\eqref{eq:est-v_rr}, \eqref{eq:bound-robin-approx}}{\leq} & \sup_{\partial\Om}K + \sup_{\omega}K.
\end{eqnarray}
}
{Moreover, in a similar fashion as in Step~3,} we can deduce that an analogous estimate such as~\eqref{eq:integration-2} holds for~$u_\rr$ as well
\begin{equation}\label{eq:right-robin}
-\int_{\Om} \Delta_{\mathbb{H}}   u_\rr Z_k  u_\rr \, {\rm d}\xi 	=  -\frac{1}{2} \int_{\partial\Om}|\Dh u_\rr |^2\langle Z_k ,\bm{n}\rangle \, {\rm d}\mathcal{H}^{Q-2}.
\end{equation}
Indeed, considering the same exhaustion of $\Om$ of $C^\infty$ connected open sets $\{\Om_i\}$ of the previous step, we have that
	\begin{eqnarray*}
&&	-\int_{\Om_i} \Delta_{\mathbb{H}}   u_\rr Z_k  u_\rr \, {\rm d}\xi \\*
&&\quad = -\frac{1}{2} \int_{\gamma^{(1)}_i}|\Dh u_\rr |^2\langle Z_k ,\bm{n}\rangle \, {\rm d}\mathcal{H}^{Q-2}- 4 \int_{\gamma^{(2)}_i} Z_{k'} u_\rr   u_\rr \langle T,\bm{n} \rangle  \, {\rm d}\mathcal{H}^{Q-2}\\*
&&\qquad - \sum_{j=1}^{2n}\int_{\gamma^{(2)}_i} Z_j u_\rr Z_k  u_\rr \langle Z_j,\bm{n}\rangle \, {\rm d}\mathcal{H}^{Q-2}  + \frac{1}{2}\int_{\gamma^{(2)}_i} |\Dh u_\rr|^2  \langle Z_k ,\bm{n}\rangle \, {\rm d}\mathcal{H}^{Q-2},
\end{eqnarray*}
where $k'$ is such that $|k'-k|=n$.
Now, recalling that -- up to taking $\rr$ sufficiently small such that in an interior neighborhood $\omega$ of the characteristic set $\boldsymbol{\delta}^{(\rr)}=0$ --  by~\cite[Theorem~5.7]{Vas06}  $u_\rr$ has bounded horizontal gradient,  we can pass to the limit in the  inequality  above getting~\eqref{eq:right-robin}, by Theorem~\ref{characteristic_set_negligible}.
{Moreover, the sequence~$\{u_\rr\}_\rr$ converge weakly
 to~$G_\Om(\cdot;{0})$. Indeed, for any test function~$\phi \in C^\infty_0(\Om)$ it holds (since $u_\rr =0$ on $\partial \Om$)}
\begin{eqnarray}\label{eq4stella}
	-\int_{\Om}u_\rr(\Delta_{\mathbb{H}}  \phi) \, {\rm d}\xi &=& 	-\int_{\Om}(\Delta_{\mathbb{H}}  u_\rr) \phi \, {\rm d}\xi\nonumber\\	
	&=& \int_{\Om} \boldsymbol{\delta}^{(\rr)} \phi \, {\rm d}\xi\\
	&\to& \int_{\Om}\boldsymbol{\delta}_{0}\phi \, {\rm d}\xi = -\int_{\Om}G_{\Om}(\cdot,{0})(\Delta_{\mathbb{H}}  \phi) \, {\rm d}\xi.\nonumber
\end{eqnarray}
Also, {thanks to the bound on its $L^\infty$-norm up to the non-characteristic boundary in~\eqref{eq:bound-green-approx}, by Theorem~\ref{boundary_hld} we eventually have that} such convergence is actually in $\Gamma^{1,\beta}(\partial \Om \setminus \Sigma (\Om))$.
{Note that this is possible by taking} $\rr$ sufficiently small so that~$\boldsymbol{\delta}^{(\rr)}\equiv 0$ near $\partial \Om$, since the $\xi_{\rm o} \equiv {0}$ is away from the boundary.


As for the left-hand side in~\eqref{eq4stella}, we have
\begin{eqnarray}\label{eq:left-robin}
\int_\Omega \boldsymbol{\delta}^{(\rr)} Z_k  u_\rr \,{\rm d}\xi & = & \int_\Omega \boldsymbol{\delta}^{(\rr)} Z_k  (u_\rr-v_\rr) \,{\rm d}\xi + \int_\Omega \boldsymbol{\delta}^{(\rr)} Z_k  v_\rr \,{\rm d}\xi \notag\\
& = &  \int_\Omega \boldsymbol{\delta}^{(\rr)} Z_k  (u_\rr-v_\rr) \,{\rm d}\xi
\end{eqnarray}
where the last integral is $0$ {by the symmetry of $v_\rr$ in~\eqref{eq:sym-v}}, since it is an integral over $B_\rr$.
\vspace{2mm}

{Let us recall the Dirichlet problem \eqref{eq:robin-approx}}. By the boundedness of its supremum norm in \eqref{eq:bound-robin-approx}, the  sequence~$\{u_\rr-v_\rr\}_\rr$ is compact in~$C^{k}(\Om)$, for any $k \in \mathbb{N}$; {see for example the interior estimates in \cite[Theorem~3.3]{BGM22}}. Also, the sequence~$\{u_\rr-v_\rr\}_\rr$ is compact in~$\Gamma^{k,\beta}(\partial \Om \setminus \Sigma(\Om))$, for any $k \in \mathbb{N}$. { Indeed,} this can be shown considering an open covering of the boundary made of $B_\sigma \cap \overline{\Om}$ so that $B_\sigma \cap \partial \Om$ is non-characteristic and then applying Theorem~1.1 in~\cite{BGM22}. Also, {the sequence $\{u_\rr-v_\rr\}$} converges to~$H(\xi,{0})$, with~$H(\xi,{0})$ being the regular part of the Green function. Hence, passing to the limit in~\eqref{eq:left-robin} as~$\rr \to 0$, by Lebesgue-Besicovitch Differentiation Lemma, we have that
\[
\lim_{\rr \to 0^+}\int_\Omega \boldsymbol{\delta}^{(\rr)} Z_k  (u_\rr-v_\rr) \,{\rm d}\xi  = Z_k  H({0},{0}) = \frac{1}{2}Z_k  \mathcal{R}_\Om({0})\,,
\]
since, by symmetry of the function $H(\xi,\eta)$, we have
\[
Z_k ^{(\xi)} H({0},{0})= Z_k ^{(\xi)}  H({0},{0}) +  Z_k ^{(\eta)}H({0},{0}) = 2Z_k ^{(\xi)}H({0},{0}) \,,
\]
with~$Z_k ^{(\xi)}$, being the $Z_k $-derivative with respect the $\xi$ variables. Thus, passing also to the limit on~\eqref{eq:right-robin} yields the desired~\eqref{eq:robin-green-1}.


\vspace{2mm}
{\bf Step 6. Let us prove that
	\begin{equation}\label{eq:robin-green-2}
		\int_{\partial \Om} \snr{\Dh G_{\Om}(\cdot,\xi_{\rm o})}^2 \langle \mathdutchcal{D} ,\bm{n}\rangle \, {\rm d}\mathcal{H}^{Q-2} = -(Q-2)\mathcal{R}_\Om(\xi_{\rm o}).
	\end{equation}
}

As in the previous steps, consider an exhaustion of $\Om$ of $C^\infty$ connected open sets $\{\Om_i\}$ such that $\Om_i \uparrow \Om$, and $\partial \Om_i = \gamma^{(1)}_i \cup \gamma^{(2)}_i$ with $\gamma^{(1)}_i \subset\partial \Om \setminus \Car$, $\gamma^{(1)}_i \uparrow \partial \Om \setminus \Car$ and $\mathcal{H}^{Q-2}(\gamma^{(2)}_i) \to 0$ as $\rr\to0$. We apply the Pohozaev identity in~\cite[Corollary~3.3]{GV00} on~$\Omega_i \setminus B_\rr({0})$, to the Green function $G_\Om(;{0})$. We get
	\begin{eqnarray}\label{eq:pohozaev-1}
	&& \hspace{-1cm}2 \sum_{j=1}^{2n}\int_{\partial (\Omega_i \setminus B_\rr({0}))}\mathdutchcal{D}G_\Om(;{0}) Z_j G_\Om(;{0}) \langle Z_j,\bm{n}\rangle \,{\rm d}\mathcal{H}^{Q-2}\notag\\*
	&&\hspace{5cm}+(Q-2)\int_{\Omega_i\setminus B_\rr({0})} \snr{\Dh G_\Om(\cdot;{0})}^2 \,{\rm d}\xi\notag\\*
	&& =  \int_{\partial (\Omega_i \setminus B_\rr({0}))}\snr{\Dh G_\Om(;{0})}^2 \langle \mathdutchcal{D},\bm{n}\rangle \,{\rm d}\mathcal{H}^{Q-2}\,.
\end{eqnarray}
Now, note that
\begin{eqnarray}\label{eq:pohozaev-2}
\int_{\Omega_i\setminus B_\rr({0})} \snr{\Dh G_\Om(\cdot;{0})}^2 \,{\rm d}\xi &=&\sum_{j=1}^{2n} \int_{\Omega_i\setminus B_\rr({0})} \big(Z_j G_\Om(\cdot;{0})\big)^2 \,{\rm d}\xi\notag\\*[0.7ex]
& = & - \sum_{j=1}^{2n} \int_{\Omega_i\setminus B_\rr({0})} \big(Z_j^2G_\Om(\cdot;{0}) \big) G_\Om(\cdot;{0}) \,{\rm d}\xi\notag\\*
&& + \sum_{j=1}^{2n} \int_{\partial (\Omega_i\setminus B_\rr({0}))} \big(Z_j G_\Om(\cdot;{0}) \big) G_\Om(\cdot;{0}) \langle Z_j,{\bm n}\rangle \,{\rm d}\mathcal{H}^{Q-2}\notag\\*[0.7ex]
& = & \sum_{j=1}^{2n} \int_{\partial (\Omega_i\setminus B_\rr({0}))} \big(Z_j G_\Om(\cdot;{0}) \big) G_\Om(\cdot;{0}) \langle Z_j,{\bm n}\rangle \,{\rm d}\mathcal{H}^{Q-2}\,,
\end{eqnarray}
since~$\Delta_{\mathbb{H}}   G_\Om (\cdot;{0}) =0$ in~$\Om_i \setminus B_\rr({0})$. Hence, combining~\eqref{eq:pohozaev-2} with~\eqref{eq:pohozaev-1}, it yields
\begin{eqnarray*}
	&&\hspace{-1cm} 2 \sum_{j=1}^{2n}\int_{\partial (\Omega_i \setminus B_\rr({0}))}\mathdutchcal{D}G_\Om(;{0}) Z_j G_\Om(;{0}) \langle Z_j,\bm{n}\rangle \,{\rm d}\mathcal{H}^{Q-2}\notag\\
	&&\hspace{2cm} +(Q-2) \sum_{j=1}^{2n} \int_{\partial (\Omega_i\setminus B_\rr({0}))} \big(Z_j G_\Om(\cdot;{0}) \big) G_\Om(\cdot;{0}) \langle Z_j,{\bm n}\rangle \,{\rm d}\mathcal{H}^{Q-2}\notag\\
	&& \quad =  \int_{\partial (\Omega_i \setminus B_\rr({0}))}\snr{\Dh G_\Om(;{0})}^2 \langle \mathdutchcal{D},\bm{n}\rangle \,{\rm d}\mathcal{H}^{Q-2}\,.
\end{eqnarray*}
Passing now to the limit as~$i \to \infty$, recalling that the Green function has bounded horizontal gradient near the characteristic set by~\cite[Theorem~5.7]{Vas06}, {as well as bounded $\mathdutchcal{D}$-derivatives by~\cite[Theorem~5.8]{Vas06} and the subsequent Remark~5.9} there, and that, in view of Theorem~\ref{characteristic_set_negligible}, the characteristic set has negligible surface measure, we finally get
\begin{eqnarray*}
	&& \hspace{-1cm}2 \sum_{j=1}^{2n}\int_{\partial (\Omega \setminus B_\rr({0}))}\mathdutchcal{D}G_\Om(;{0}) Z_j G_\Om(;{0}) \langle Z_j,\bm{n}\rangle \,{\rm d}\mathcal{H}^{Q-2}\notag\\*
	&& \qquad -(Q-2) \sum_{j=1}^{2n} \int_{\partial  B_\rr({0})} \big(Z_j G_\Om(\cdot;{0}) \big) G_\Om(\cdot;{0}) \langle Z_j,{\bm n}\rangle \,{\rm d}\mathcal{H}^{Q-2}\notag\\*
	&& \quad =  \int_{\partial (\Omega \setminus B_\rr({0}))}\snr{\Dh G_\Om(;{0})}^2 \langle \mathdutchcal{D},\bm{n}\rangle \,{\rm d}\mathcal{H}^{Q-2}\,.
\end{eqnarray*}
since~$G_\Om (\cdot;{0})=0$ on $\partial \Om$. Thus, we can rewrite the  identity above as follows,
\begin{eqnarray}\label{eq:phozaev-final}
	&& 2 \sum_{j=1}^{2n}\int_{\partial\Omega }\mathdutchcal{D}G_\Om(;{0}) Z_j G_\Om(;{0}) \langle Z_j,\bm{n}\rangle \,{\rm d}\mathcal{H}^{Q-2} -\int_{\partial \Omega }\snr{\Dh G_\Om(;{0})}^2 \langle \mathdutchcal{D},\bm{n}\rangle \,{\rm d}\mathcal{H}^{Q-2}\notag\\
	&& \quad =   (Q-2) \sum_{j=1}^{2n} \int_{\partial  B_\rr({0})} \big(Z_j G_\Om(\cdot;{0}) \big) G_\Om(\cdot;{0}) \langle Z_j,{\bm n}\rangle \,{\rm d}\mathcal{H}^{Q-2}\\
	&& \qquad- \int_{\partial B_\rr({0})}\snr{\Dh G_\Om(;{0})}^2 \langle \mathdutchcal{D},\bm{n}\rangle \,{\rm d}\mathcal{H}^{Q-2}\notag\\
	&& \qquad  +\, 2 \sum_{j=1}^{2n}\int_{\partial B_\rr({0}) }\mathdutchcal{D}G_\Om(;{0}) Z_j G_\Om(;{0}) \langle Z_j,\bm{n}\rangle \,{\rm d}\mathcal{H}^{Q-2}.\notag
\end{eqnarray}
Starting with the left-hand side term in \eqref{eq:phozaev-final}, we note that~$D G_\Om (\xi;{0}) = g {\bm n}$, for some function $g$, being $G_\Om(\cdot;{0})=0$ on $\partial \Om$, so that
\[
\mathdutchcal{D} G_\Om (\xi;{0}) \langle Z_j, {\bm n}\rangle 
 = Z_j G_\Om (\xi;{0}) \langle \mathdutchcal{D}, {\bm n}\rangle\,,
\]
which in turn yields
\begin{eqnarray}\label{eq:left-final}
{\eqref{eq:phozaev-final}}_\text{(l.h.s.)} & = & 
	2 \sum_{j=1}^{2n}\int_{\partial\Omega }\mathdutchcal{D}G_\Om(;{0}) Z_j G_\Om(;{0}) \langle Z_j,\bm{n}\rangle \,{\rm d}\mathcal{H}^{Q-2}\notag\\
	&& -\int_{\partial \Omega }\snr{\Dh G_\Om(;{0})}^2 \langle \mathdutchcal{D},\bm{n}\rangle \,{\rm d}\mathcal{H}^{Q-2}\notag\\
 & = &  \int_{\partial \Omega }\snr{\Dh G_\Om(;{0})}^2 \langle \mathdutchcal{D},\bm{n}\rangle \,{\rm d}\mathcal{H}^{Q-2}.
\end{eqnarray}	

As for the right-hand side term in~\eqref{eq:phozaev-final}, we first  note that on~$\partial B_\rr ({0})$ the fundamental solutions $K(\cdot) = (c_Q)^{-1} \rr^{2-Q}$, so $\mathdutchcal{D}K(\cdot) = 0$. Moreover, except for the term~$H(\cdot;{0})Z_j K(\cdot)$, we can consider any other quantity as~${\rm o}(\rr)$, since both the $\mathdutchcal{D}$-derivative and the $Z_j$-derivative of $G_\Om (\cdot;{0})$ do coincide with the analogous one of $H$, given that $K(\cdot) = (c_Q)^{-1} \rr^{2-Q}$ on~$\partial B_\rr$. 
 Thus, {the right-hand side in~\eqref{eq:phozaev-final} can be treated as follows 
 \begin{eqnarray}\label{eq:pohozaev-rhs}
{\eqref{eq:phozaev-final}}_\text{(r.h.s.)} & = &  - \int_{\partial B_\rr({0})}\snr{\Dh H(;{0})}^2 \langle \mathdutchcal{D},\bm{n}\rangle \,{\rm d}\mathcal{H}^{Q-2}\notag\\*
&&   +\, 2 \sum_{j=1}^{2n}\int_{\partial B_\rr({0}) }\mathdutchcal{D}H(;{0}) Z_j H(;{0}) \langle Z_j,\bm{n}\rangle \,{\rm d}\mathcal{H}^{Q-2}\notag\\* 
&&  +\, (Q-2) \sum_{j=1}^{2n} \int_{\partial  B_\rr({0})} \big(Z_j K(\xi)\big)  H(\cdot;{0}) \langle Z_j,{\bm n}\rangle \,{\rm d}\mathcal{H}^{Q-2}\notag\\* 
&&  +\, (Q-2) \sum_{j=1}^{2n} \int_{\partial  B_\rr({0})} \big(Z_j H(\cdot;{0})\big)  H(\cdot;{0}) \langle Z_j,{\bm n}\rangle \,{\rm d}\mathcal{H}^{Q-2}\notag\\*
&&  +\, (Q-2) \sum_{j=1}^{2n} \int_{\partial  B_\rr({0})} \big(Z_j H(\cdot;{0})\big)  K(\xi) \langle Z_j,{\bm n}\rangle \,{\rm d}\mathcal{H}^{Q-2}\notag\\*
& =: & I_1 +I_2 +I_3 + I_4 +I_5.
 \end{eqnarray}
Let us start noticing that, since the regular part of the Green function $H(\cdot;{0})$ is smooth in $\Omega$, {we have that~$
I_j \leq c \mathcal{H}^{Q-2}(\partial B_\rr({0}))$, for~$j=1,2,4$}. As for~$I_5$ we can reason in a similar way, recalling that $K(\xi) = c_Q\rr^{2-Q}$ on $\partial B_\rr({0})$, getting $
I_5  \leq c\mathcal{H}^{Q-2}(\partial B_\rr({0}))/\rr^{Q-2}= c\rr$.
Thus, \eqref{eq:pohozaev-rhs}
} 
can be rewritten  when~$\rr \to 0^+$ as  
\begin{eqnarray*} 
	{\eqref{eq:phozaev-final}}_\text{(r.h.s.)} & = & (Q-2)\sum_{j=1}^{2n}\int_{\partial B_\rr({0})}\big(Z_jK(\xi)\big)H(;{0}) \langle Z_j,\bm{n}\rangle \,{\rm d}\mathcal{H}^{Q-2} + {\rm o}(\rr).
\end{eqnarray*}
 We rewrite the  integral above in a more convenient way now. Let us star noting that
 \[
 \begin{split}
 Z_jK(\xi)\langle Z_j,\bm{n}\rangle & = - \frac{Q-2}{c_Q \rr^{Q-1}} \langle Z_j, D |\xi|_{\mathbb{H}}\rangle \left\langle Z_j, \frac{D |\xi|_{\mathbb{H}}}{\snr{D |\xi|_{\mathbb{H}}}}\right\rangle \\
 &  = -\frac{Q-2}{c_Q \rr^{Q-1}}  \frac{\big(Z_j(|\xi|_{\mathbb{H}})\big)^2}{\snr{D |\xi|_{\mathbb{H}}}}.
 \end{split}
 \]
 Thus, 
\begin{eqnarray}\label{eq:pohozaev-rhs-final}
		{\eqref{eq:phozaev-final}}_\text{(r.h.s.)} & = &
  (Q-2)\sum_{j=1}^{2n}\int_{\partial B_\rr({0})}Z_jK(\xi))H(;{0}) \langle Z_j,\bm{n}\rangle \,{\rm d}\mathcal{H}^{Q-2} + {\rm o}(\rr)\notag\\
 & = &  - \frac{(Q-2)^2}{c_Q\rr^{Q-1}}\int_{\partial B_\rr({0})}H(\cdot;{0})\frac{\snr{\Dh (|\xi|_{\mathbb{H}})}^2}{\snr{D |\xi|_{\mathbb{H}}}} \,{\rm d}\mathcal{H}^{Q-2} +{\rm o}(\rr)\notag\\
 & = & - (Q-2)H({0},{0}) + {\rm o}(\rr)\,, 
 \end{eqnarray}
 by the representation of harmonic functions with their surface measure; see for example \cite[Theorem~5.5.4]{BLU07}.
 Finally,~\eqref{eq:robin-green-2} plainly follows { by collecting estimates~\eqref{eq:left-final} and~\eqref{eq:pohozaev-rhs-final} and} passing to the limit as $\rr \to 0^+$.


\vspace{2mm}
{\bf Step~7. Conclusion.}
The thesis follows by the Pohozaev identity obtained in~\eqref{pohozaev_sub_approx} and~\eqref{eq:han-pohozaev}, the identity~\eqref{eq:robin-green-1} and~\eqref{eq:robin-green-2} and  the limiting behaviour in~\eqref{han_lim_green}.\\

\hfill$\square$




\appendix

	\vspace{2mm}
	\section{Convolution}\label{sec_app}
We conclude the paper by recalling  some basic properties of subelliptic mollifiers.

\begin{defn}
Let $B_1(0)\subset \h$ and let $J \in C^\infty_0(\r^{2n+1})$ be such $0 \leq J \leq 1$, $\text{supp}~J \subset B_1(0)$ and
\[
\int_{\h} J(\xi)\,{\rm d}\xi =1.
\]
Define $J_h(\xi):= h^{-Q}J(\delta_\frac{1}{h}(\xi))$ and for any $u \in L^1_{\loc}(\h)$, define
\[
u_h (\xi) := (u \star J_h)(\xi)  := \int_{\h} J_h(\eta^{-1}\circ \xi)u(\eta) \,{\rm d}\eta
\]
\end{defn}


\begin{prop}\label{prop:reg-conv}
The following results hold true.
\begin{enumerate}[\rm(1)]
    \item If $ g \in L^p(\h)$, $1 \leq p< \infty$, then $g \star J_h \in C^\infty_0(\h)$ and $g \star J_h \to g$ in $L^p(\h)$ as $h \to 0$
    \item Let $g \in L^1_{\loc}(\h)$. Then $(\phi \star g) \in C^1_0(\h)$ and
\[
Z_j (g \star J_h) = g \star (Z_j J_h)
\]
for any $j \in \{1,\dots,2n\}$.
\end{enumerate}
\end{prop}
\begin{proof}[\bf Proof]
{The statements of Proposition \ref{prop:reg-conv} can be proven by a quite standard argument. For the sake of completeness, we just give a proof of (2).} By left invariance of the vector field $Z_j$ we have that
\begin{eqnarray*}
Z_j \left(g \star J_h\right) & = & Z_j \left(\int_{\h} J_h(\eta^{-1}\circ \xi)g(\eta)\,{\rm d}\eta\right)\\
& = & \int_{\h} Z_j \left( J_h(\eta^{-1}\circ \xi)g(\eta)\right)\,{\rm d}\eta\\
& = & \int_{\h} (Z_jJ_h)(\eta^{-1}\circ \xi)(\eta)\,{\rm d}\eta
\end{eqnarray*}
\end{proof}
\vspace{1mm}

\begin{lemma}\label{lemma:app}
Let $\Om \subset \h$ be an open and bounded set and let $u \in L^1_{\loc}(\Om)$ be such that
\[
\int_\Om u \phi \,{\rm d}\xi =0 \qquad \forall \phi \in C^\infty_0(\Om).
\]
Then, $u =0$ a.~\!e. in $\Om$.
\end{lemma}

\begin{proof}[\bf Proof]
 Consider a sequence of compact set $\{K_h\} \subset \Om$ such that
\[
\bigcup_{h=1}^\infty K_h =\Om \quad \text{and} \quad \text{dist}(K_h, \partial \Om) \leq 2/h.
\]
Choose now $\phi \in L^\infty(\h)$ be such that $\text{supp}~\phi \subset K_h$ and consider $J_h \in C^\infty_0(\h)$. Setting $\phi_h := \phi \star J_h$ we have that
\[
\text{supp}~\phi_h \subset \overline{B_{\frac{1}{h}}(0)} + K_h \subset \Om.
\]
By Proposition \ref{prop:reg-conv}, we have that $\phi_h \in C^\infty_0(\Om)$.
Thus by hypothesis, we have that
\begin{equation}\label{eq:lemma-app-1}
\int_\Om u\phi_h \,{\rm d}\xi =0.
\end{equation}
Moreover, since by \cite[Remark~5.3.8]{BLU07} the sequence $\phi_h$ converges to $\phi$ in $L^1_{\loc}(\h)$, up to subsequences, we have that $\phi_h \to \phi$ a.~\!e. in $\h$ and $\|\phi_h\|_{L^\infty(\h)} \leq c \|\phi\|_{L^\infty(\h)}$; see \cite[Theorem~4.9]{Bre10}. Hence, passing to the limit in \eqref{eq:lemma-app-1} yields that
\[
\int_\Om u\phi \,{\rm d}\xi =0.
\]
We now choose
\[
\phi \equiv \phi_h :=
\begin{cases}
\text{sgn}~u \quad & \text{in}~K_h\,,\\
0 \quad & \text{in}~\Om \setminus K_h.
\end{cases}
\]
This yields that $\displaystyle\int_{K_h} \snr{u} \,{\rm d}\xi=0$ for any $h \in \mathbb{N}$. Hence, 
 $u=0$ a.~\!e. on $\Om$.
\end{proof}

\vspace{2mm}

\begin{thebibliography}{99}
\bibliographystyle{amsplain}


\bibitem{AG21} {\sc  N. Alamri, N. Gamara}: Non-characteristic Heisenberg group domains. {\it Period. Math. Hung.} {\bf 82} (2021), 16--28.
\vs
		
		\bibitem{AGLT24} {\sc C.~\!O.~Alves, S. Gandal, A. Loiudice, J. Tyagi}: A Brezis-Nirenberg Type Problem for a Class of Degenerate Elliptic Problems Involving the Grushin Operator. {\it J. Geom. Anal.} {\bf34}~(2024), Art.~52.
		\vs
		
\bibitem{AG03} {\sc M. Amar, A. Garroni}: {$\Gamma$-convergence of concentration problems}. {\it Ann. Scuola Norm. Sup. Pisa Cl. Sci.} {\bf 2}~(2003), 151--179.
\vs

\bibitem{AP87} {\sc  F. V. Atkinson, L. A. Peletier}: Elliptic equations with nearly critical growth. {\it J. Differential Equations} {\bf 70} (1987), no. 3, 349--365.
\vs

\bibitem{BC88} {\sc A.~Bahri, J.~\!-M. Coron}: On a nonlinear elliptic equation involving the critical Sobolev exponent: the effect of the topology of the domain. {\it Comm. Pure Appl. Math.} {\bf 41} (1988), no. 3, 253--294.
\vs




      \bibitem{BGM19} {\sc A. Banerjee, N. Garofalo, I.~\!H.~Munive}: Compactness methods for $\Gamma^{1,\alpha}$ boundary Schauder estimates in Carnot groups. {\it Calc. Var. Partial Differential Equations} {\bf 58} (2019), no. 3, 29--97.
     \vs
  
  \bibitem{BGM22} {\sc A. Banerjee, N. Garofalo, I.~\!H.~Munive}:
  Higher order Boundary Schauder Estimates in Carnot Groups. {\it Math. Ann.}, to appear.
    \vs
    
    \bibitem{Bao67} {\sc M. S. Baouendi}: Sur une Classe d'Operateurs Elliptiques Degener\'es. {\it Bull. Soc. Math. Fr.} {\bf95} (1967) , 45--87.
    \vs

    
\bibitem{Ben08}{\sc J.~Benameur}: {Description du d\'efaut de compacit\'e de l'injection de Sobolev sur le groupe de Heisenberg}. {\it Bull. Belg. Math. Soc. Simon Stevin}~{\bf 15} (2008), no.~4, 599--624.
\vs 

\bibitem{BS12} {\sc M. Bhakta, K. Sandeep}: Poincar\'e-Sobolev equations in the hyperbolic space. {\it Calc. Var. Partial Differential Equations} {\bf 44} (2012), 247--269.
\vs

		
		\bibitem{BP02} {\sc I. Birindelli, J. Prajapat}: Monotonicity and symmetry results for degenerate elliptic equations on nilpotent Lie groups. {\it Pacific J. Math.} {\bf 204} (2002), no. 1, 1--17. 
		\vs
				
		\bibitem{BLU07} {\sc A. Bonfiglioli, E. Lanconelli, F. Uguzzoni}:
		{\it Stratified Lie Groups and their sub-Laplacians}.
		Springer Monographs in Mathematics, Springer, Berlin, 2007. 
		\vs
		
		
	\bibitem{Bre10} {\sc H. Brezis}: {\it Functional Analysis, Sobolev Spaces and Partial Differential Equations}, Springer, 2010.
	\vs
		

\bibitem{BP89}{\sc H. Brezis, L.~\!A. Peletier}: Asymptotic for Elliptic Equations involving critical growth. In {\it Partial Differential Equations and the Calculus of Variations. Essays in Honor of Ennio De Giorgi, Vol. 1}, Progr. Differ. Equ. Appl., Birkh\"auser, Boston (1989), 149--192.
\vs

 

\bibitem{CGN02} {\sc L. Capogna, N. Garofalo, D.-M.~Nhieu}: Properties of Harmonic Measures in the Dirichlet Problem for Nilpotent Lie Groups of Heisenberg Type. {\it Amer. J. Math.}~{\bf 124} (2002), no.~2, 273--306.
\vs




\bibitem{Cit95} {\sc G. Citti}:  Semilinear Dirichlet problem involving critical exponent for the Kohn Laplacian. {\it Ann. Mat. Pura Appl.} (4) {\bf 169}~(1995), 375--392.
\vs

 
 \bibitem{CGS21}{\sc G. Citti, G. Giovannardi, Y. Sire}: Schauder estimates up to the boundary on H-type groups: an approach via the double layer potential. {\it Preprint} (2021). Available at~\href{https://arxiv.org/abs/2104.08115}{\tt arXiv:2104.08115}
 \vs
 

\bibitem{CU01}{\sc G. Citti, F. Uguzzoni}: Critical semilinear equations on the Heisenberg group: the effect of the topology of the domain. {\it Nonlinear Anal.}~{\bf 46}~(2001), 399--417.
\vs



 
 
     \bibitem{DGP07} {\sc D. Danielli, N. Garofalo, A. Petrosyan}: The sub-elliptic obstacle problem: $C^{1,\alpha}$ regularity of the free boundary in Carnot groups of step two. {\it Adv. Math.} {\bf 211} (2007), no. 2, 485--516.
    \vs
    
    \bibitem{DPMP10} {\sc M.~del Pino, M.~Musso, F.~Pacard}:
Bubbling along boundary geodesics near the second critical exponent. {\it J. Eur. Math. Soc. (JEMS)} {\bf 12}~(2010), No.~6, 1553--1605.
\vs
    

   	\bibitem{Der71} {\sc M. Derridj}: Un probl\'eme aux limites pour une classe d'op\'erateurs du second ordre hypoelliptiques. (French) {\it Ann. Inst. Fourier (Grenoble)} {\bf21} (1971), no. 4, 99--148.
   	\vs
  
 
  \bibitem{DF19}{\sc F. Dragoni, D. Filali}:
Starshaped and convex sets in Carnot groups and in the geometries of vector fields. {\it J. Convex Anal.}~{\bf 26} (2019), no.~4, 1349--1372.
\vs
 
	
 

\bibitem{FGM02}{\sc M. Flucher, A. Garroni, S. M\"uller}: Concentration of low energy extremals: Identification of concentration points. {\it Calc. Var. Partial Differential Equations}~{\bf 14} (2002), 483--516.
\vs

\bibitem{FM99}{\sc M. Flucher, S. M\"uller}: Concentration of low energy extremals. {\it Ann. Inst. H. Poincar\'e - Anal. Non Lin\'eeaire}~{\bf 16} (1999), no.~3, 269--298.
\vs

\bibitem{FP23}{\sc M. Fogagnolo, A. Pinamonti}: {Strict starshapedness of solutions to the horizontal $p$-Laplacian in the Heisenberg group}. {\it Math. Eng.} {\bf 3} (2021), no.~6., 1--15.

\vs

\bibitem{Fol75}{\sc G.~\!B.~Folland}: Subelliptic estimates and function spaces on nilpotent Lie groups. {\it Ark. Math.} {\bf 13} (1975), 161--207.
\vs 




\bibitem{FS74}{\sc   G.~\!B.~Folland, E.~\!M. Stein}: Estimates for the $\bar\partial_b$ complex and analysis on the Heisenberg group. {\it Comm. Pure Appl. Math.}~{\bf 27} (1974), 429--522.
\vs
 
\bibitem{FL12}{\sc R. Frank, E.~\!H. Lieb}: Sharp constants in several inequalities on the Heisenberg group. {\it Ann. Math.}~{\bf 176}~(2012), 349--381. 
\vs


\bibitem{FGMT15} {\sc R.~\!L. Frank, M.~\!d.~\!M. Gonz\'alez, D.~\!D. Monticelli, J. Tan}: An extension problem for the CR fractional Laplacian. {\it Adv. Math.}~{\bf 270} (2015), 97--137.
\vs


\bibitem{FKK23} {\sc R. L. Frank, T. K\"onig and H. Kovarik}: Blow-up of solutions of critical elliptic equation in three dimensions. {\it Anal. PDE}, to appear. \href{https://arxiv.org/abs/2102.10525}{\tt arXiv:2102.10525} 
\vs

 
 
 \bibitem{Gam01}{\sc N. Gamara}: The CR Yamabe conjecture -- the case $n=1$. {\it J. Eur. Math. Soc.} {\bf 3} (2001), no.~2, 105--137.
 \vs

 \bibitem{GL92}{\sc N. Garofalo, E. Lanconelli}: Existence and Nonexistence Results for Semilinear Equations on the Heisenberg Group. {\it Indiana Univ. Math. J.} {\bf 41}~(1992), no.~1, 71--98.
 
\vs


\bibitem{GLV23}{\sc N. Garofalo, A. Loiudice, D. Vassylev}:  Optimal decay for solutions of nonlocal semilinear equations with critical exponent in homogeneous group. {\it Proc. Royal Soc. Edinburgh Section~A}, to appear.
\vs

 


  \bibitem{GV00}{\sc N. Garofalo, D. Vassilev}: Regularity near the characteristic set
in the non-linear Dirichlet problem
and conformal geometry of sub-Laplacians on Carnot groups. {\it Math. Ann.} {\bf 318}~(2000), 453--516.
\vs

  \bibitem{GV01}{\sc N. Garofalo, D. Vassilev}: Symmetry properties of positive entire solutions of Yamabe-type equations on groups of Heisenberg type. {\it Duke Math. J.} {\bf 106}~(2001), 411--448.
\vs

                                                                             
\bibitem{Gru71} {\sc V.~\!V. Grushin}: On a class of hypoelliptic operators. {\it Math. USSR Sb.} {\bf 12}, 458 (1970)
\vs            
                                                                         
\bibitem{Gru71-2} {\sc V.~\!V. Grushin}:: On a class of hypoelliptic pseudodifferential operators degenerate on a submanifold.
{\it Math. USSR Sb.} {\bf 13}, 155 (1971)
\vs 


\bibitem{GMM18}{\sc C. Guidi, A. Maalaoui, V. Martino}: Palais-Smale sequences for the fractional CR Yamabe functional and multiplicity results. {\it Calc. Var. Partial Differential Equations}~{\bf 57} (2018), Art.~152.
\vs


  \bibitem{Han91} {\sc Z.-C. Han}: {Asymptotic approach to singular solutions for nonlinear elliptic equations involving critical Sobolev exponent}. {\it Ann. Inst. Henri Poincar\'e Anal. Non Lin\'eaire} {\bf 8}~(1991),  159--174.
  	\vs	
 
\bibitem{IV11}{\sc S.~\!P. Ivanov, D.~\!B. Vassilev}: {\it Extremals for the Sobolev Inequality and the Quaternionic Contact Yamabe Problem}. World Scientific Publishing, Singapore, 2011.

\vs

\bibitem{Jer81} {\sc D. Jerison}: The Dirichlet problem for the Kohn Laplacian on the Heisenberg group. I. {\it J. Functional Analysis} {\bf43} (1981), no. 1, 97--142.
\vs

	\bibitem{Jer81-2} {\sc D. Jerison}: The Dirichlet problem for the Kohn Laplacian on the Heisenberg group. II. {\it J. Functional Analysis} {\bf43}  (1981), no. 2, 224--257.
\vs


\bibitem{JL87} {\sc D. Jerison, J.~\!M. Lee}: The Yamabe problem on CR manifolds. {\it J. Differential Geom.} {\bf 25} (1987), no. 2, 167--197.
\vs

\bibitem{JL88} {\sc D. Jerison, J.~\!M. Lee}: Extremals for the Sobolev inequality on the Heisenberg group and the~CR~Yamabe problem. {\it J. Amer. Math. Soc.}~{\bf 1}~(1988), no.~1, 1--13.
\vs

\bibitem{Kap80} {\sc A. Kaplan}:  Fundamental solutions for a class of hypoelliptic PDE generated by composition of quadratic forms. {\it Trans. Amer. Math. Soc.} {\bf 258} (1980), no. 1, 147--153.
\vs

\bibitem{KL12} {\sc A.~\!E. Kogoj, E Lanconelli}: On semilinear $\Delta_\lambda$-Laplace equation. {\it Nonlinear Anal.} {\bf 75} (2012), 4637--4649.
\vs

\bibitem{LU98} {\sc E. Lanconelli, F. Uguzzoni}:
Asymptotic behavior and non-existence theorems for semilinear Dirichlet problems involving critical exponent on unbounded domains of the Heisenberg group.
{\it Boll. UMI (Serie 8)} {\bf 1-B}~(1998), no.~1, 139--168.
\vs


 
 \bibitem{Loi05} {\sc A. Loiudice}: Improved Sobolev inequalities on the Heisenberg group. {\it Nonlinear Analysis: Theory, Methods \& Applications}, {\bf 62}~(2005), no.~5, 953--962.
 \vs

\bibitem{MMP13} {\sc A. Maalaoui, V. Martino, A. Pistoia}: Concentrating solutions for a sub-critical sub-elliptic problem. {\it Diff. Int. Eq.}~{\bf 26}~(2013), no. 11-12, 1263--1274.
\vs

 

	\bibitem{MPPP23} {\sc M. Manfredini, G. Palatucci, M. Piccinini, S. Polidoro}:
{H\"older continuity and boundedness estimates for nonlinear fractional equations in the    Heisenberg group}. {\it J. Geom. Anal.}~{\bf 33}~(2023), no.~3, Art.~77. 
\vs


    		\bibitem{MM06} {\sc R. Monti, D. Morbidelli}: Kelvin transform for Grushin operators and critical semilinear equations. {\it Duke Math. J.} {\bf 131} (2006) 167--202
    		\vs 
    		
		    \bibitem{Pal11}{\sc G. Palatucci}: Subcritical approximation of the Sobolev quotient and a related concentration result. {\it Rend. Sem. Mat. Univ. Padova} {\bf 125}~(2011), 1--14.
		    \vs

    \bibitem{Pal11b}{\sc G. Palatucci}: $p$-Laplacian problems with critical Sobolev exponent.
{\it Asymptot. Anal.} {\bf 73}~(2011), no.~1-2, 37--52.
\vs 

    \bibitem{PP22} {\sc G. Palatucci, M. Piccinini}:
    {Nonlocal Harnack inequalities in the Heisenberg group}.
    {\it Calc. Var. Partial Differential Equations}~{\bf 61} (2022), Art.~185.
    \vs

		\bibitem{PPT23} {\sc G. Palatucci, M. Piccinini, L. Temperini}:
		Struwe's Global Compactness and energy approximation of the critical Sobolev embedding in the Heisenberg group.	{\it Preprint} (2023). 
\vs

		\bibitem{PP23} {\sc G. Palatucci, M. Piccinini, L. Temperini}:
		{Global Compactness, subcritical approximation of the Sobolev quotient, and a related concentration result in the Heisenberg group}.
		{\it Trends in Mathematics},  Birkh\"auser,  
		2024.
\vs

\bibitem{PP14} {\sc G. Palatucci, A. Pisante}: Improved Sobolev embeddings, profile decomposition, and concentration-compactness for fractional Sobolev spaces. {\it Calc. Var. Partial Differential Equations} {\bf 50}~(2014), no.~3-4, 799--829.
\vs

\bibitem{Pas93}{\sc D.  Passaseo}: Nonexistence results for elliptic problems with supercritical nonlinearity in nontrivial domains. {\it J. Funct. Anal.}  {\bf 114} (1993), 97--105.
\vs


\bibitem{PP15} {\sc G. Palatucci, A. Pisante}: A Global Compactness type result for Palais-Smale sequences in fractional Sobolev spaces.
{\it Nonlinear Anal.} {\bf 117}~(2015), 1--7.
\vs


\bibitem{PR03} {\sc   A. Pistoia,  O. Rey}: {Boundary blow-up
  for a Brezis-Peletier problem on a singular domain}.  {\it Calc. Var. Partial Differential Equations}~{\bf 18} (2003), no.~3, 243--251.
\vs
 
\bibitem{PT23} {\sc P. Pucci, L. Temperini}: {On the Concentration-compactness principle for Folland-Stein spaces and for fractional horizontal Sobolev spaces}. {\it Math. Eng.}~{\bf 5} (2023), no.~1, 1--21. 
\vs

		\bibitem{Rey89} {\sc O. Rey}: {Proof of the conjecture of H.~Brezis and L.~\!A.~Peletier}. {\it Manuscripta math.} {\bf 65} (1989),  19--37. 
					\vs

      
      
     \bibitem{Ugu99} {\sc F. Uguzzoni}:  A non-existence theorem for a semilinear Dirichlet problem involving critical exponent on halfspaces of the Heisenberg group. {\it NoDEA Nonlinear Differential Equations Appl.} {\bf 6} (1999), no. 2, 191--206.
     \vs
      
      \bibitem{Vas06} {\sc D. Vassilev}: Existence of solutions and regularity near the characteristic boundary for sub-Laplacian equations on Carnot groups. {\it Pacific J. Math.} {\bf 227} (2006), no. 2, 361--397. 
      \vs
      
      \bibitem{Wei98}{\sc J. Wei}: Asymptotic behavior of least-energy solutions of a semilinear Dirichlet problem involving critical Sobolev exponent. {\it J. Math. Soc. Japan}~{\bf 1}~(1998), 139--153.
\vs

     \bibitem{Xu99} {\sc C. J. Xu}:
The Dirichlet problems for a class of semilinear sub-elliptic equations. {\it Nonlinear Anal.} {\bf 37} (1999), no.8, 1039--1049.
\vs

      
      \bibitem{Yan24} {\sc Q. Yang}: The optimal constant in the $L^2$ Folland-Stein inequality on the $H$-type group. {\it J.~Funct. Anal.} {\bf 286}~(2024), no.~2, Art.~110209. 
	    \vs


\end{thebibliography}

\vspace{2mm}

\end{document}
