%In this section we model the 802.1Qbv opening gates using Network Calculus~\cite{netcal8021qbv,leboudec}, and the wireless errors using stochastic scaling~\cite{stochastic-scaling}.

\section{Modelling 802.1Qbv as TDMA curves}
\label{sec:Qbv_tdma}
The \gls{tas} in 802.1Qbv decides when to open and close each queue using a \gls{gcl}. The latter defines both the periodicity $T_q$ of each queue and how long, $L_q$, it is open for transmission. Using \gls{tdma}, a user queue $q$ has a transmit service curve (cummulative bits processed over time), $\beta_{L,T}(t)$, given by \eqref{eq:tdma-service-curve}
\begin{equation}
    \beta_{L,T}(t)=
    C\cdot\max\left\{ \left\lfloor \frac{t}{T}\right\rfloor L,
    t-\left\lceil\frac{t}{T} \right\rceil (T-L) \right\}
    \label{eq:tdma-service-curve}
\end{equation}
with $C$ being the transmission rate and $T,L$ the periodicity and length of the \gls{tdma} transmission opportunities. The service curve in 802.1Qbv for a queue $q$ is given by $\beta_{L_q,T_q}([t-\delta_q]^+)$ where $\delta_q>0$ is the time at which queue $q$ gate is first opened.
%% Figure environment removed

Note that in 802.11Qbv, two or more queues may have their gates opened concurrently\footnote{\ca{In this paper}, we consider 2 queues for brevity. Our analysis can be easily extended to $N$ concurrent queues.}. We use an affine lower envelope of the \gls{tdma} service curve~\eqref{eq:tdma-service-curve} to handle gate concurrency as shown in the  (top) dashed line of Figure~\ref{fig:concurrent}. The \gls{tdma} affine lower envelope for queue $q$ in 802.1Qbv is
the rate-latency function
\begin{equation}
    \beta_q(t)=\frac{C\cdot L_q}{T_q+L_q}[t-\delta_q]^+.
\end{equation}


\section{Concurrent 802.1Qbv opened gates}
\label{subsec:concurrent}




When high priority queue $q$ and low priority queue $q'$ are opened concurrently, packets of $q$ are transmitted ahead of those from $q'$.

%We can tell $q$ incoming traffic is governed by a strict arrival curve $\gamma_{C_q,b_q}(t)=C_q\cdot t + b_q$, and queue $q'$ by a strict arrival curve $\gamma_{C_{q'},b_{q'}}(t)$. Therefore, it is possible to know how the service curve of the low priority queue $q'$ is impacted by the high priority traffic.


\begin{corollary}[802.1Qbv High priority service curve]
    \label{cor:HPservice}
    Given two queues $q,q'$ whose gates
    are opened concurrently; the high
    priority queue $q$ has a strict service
    curve
    \begin{equation}
        \label{eq:HPservice}
        \beta_q^*(t) = R_q\left[t-\left(\delta_q+\frac{l_{\max}^{q'}}{R_q}\right) \right]^+
    \end{equation}
    with $R_q=\tfrac{C L_q}{T_q+L_q}$ and
    $l_{\max}^{q'}$ the maximum packet size
    of the low priority queue $q'$.
\end{corollary}
\begin{proof}
    We mimic the proof of
    \cite{leboudec}[Proposition 1.3.4].
    Take the affine lower envelope of queue
    $q$ defined
    in~\eqref{eq:tdma-service-curve}. Consider
    $s$ as the start of the backlog period of
    queue $q$, hence, the arrival and departing
    curve at queue $q$ is the same at that time
    $D_q(s)=A_q(s)$.

    For 802.1Qbv is non-preemptive, if a
    low priority packet from $q'$ arrives before
    a high priority packet from $q$ in the
    interval $(s,t]$; the high priority packet
    will wait, i.e.
    \begin{equation}
        D_q(t)-D_q(s)\geq \beta_q(t-s)-l_{\max}^{q'}
    \end{equation}
    given $D_q(s)=A_q(s)$ and
    $A,D\in\mathcal{F}=\{f: \mathbb{R}^+\to \mathbb{R}, \forall t\geq s:\ f(t)\geq f(s), f(0)=0\}$;
    we know
    \begin{equation}
        D_q(t)\geq A_q(s)+
        \left[ R_q\left[ (t-s)-\delta_q \right]^+  - l_{\max}^{q'} \right]^+
        \label{eq:depart-high}
    \end{equation}
    If $(t-s)>\delta_q$
    \eqref{eq:depart-high} becomes
    \begin{equation}
        D_q(t)\geq
            A_q(s)+
            R_q\left[ (t-s)- \left(\delta_q+ \frac{l_{\max}^{q'}}{R_q} \right) \right]^+
    \end{equation}
    If $(t-s)\leq\delta_q$ 
    \eqref{eq:depart-high} becomes
    $D_q(t)\geq A_q(s)$.
    That is, the 
    departing flow of the high priority
    queue $q$ satisfies
    \begin{equation}
        D_q(t)\geq A_q\otimes \beta_q^*(t)
    \end{equation}
    with $\beta_q^*(t)$ being the
    rate-latency service curve with
    rate $R_q$ and latency 
    $\delta_q+\frac{l_{\max}^{q'}}{R_q}$.
\end{proof}

Similarly, it is also possible to obtain the
service curve for the traffic in the
low priority queue $q'$ when such queue is
opened concurrently with the higher priority
queue $q$.


\begin{corollary}[802.1Qbv Low priority service curve]
    \label{cor:LPservice}
    Given two queues $q,q'$ whose gates
    are opened concurrently;
    if the flow at $q$ has
    a strict affine arrival curve
    $\gamma_{C_q,b_q}$, then the low
    priority queue $q'$ has a strict service
    curve
    \begin{equation}
        \label{eq:LPservice}
        \beta_{q'}^*(t)=
        \left(R_{q'}-C_q \right)
        \left[
            t
            - \frac{ R_{q'}\delta_{q'}+ b_q}{R_{q'}-C_q}
        \right]^+
    \end{equation}
    with $R_{q'}=\tfrac{C L_{q'}}{T_{q'}+L_{q'}}$.
\end{corollary}
\begin{proof}
    We again mimic the proof presented in \cite{leboudec} [Proposition 1.3.4].
    Take the affine lower envelope of
    queue $q'$ as defined
    in~\eqref{eq:tdma-service-curve}. Consider
    $s'$ the start of the busy period, that is,
    $s'<s$ with $s$ the start of the backlog
    period. In the interval $(s',t]$ the
    low priority queue departing traffic
    satisfies
    \begin{equation}
        D_{q'}(t)-D_{q'}(s')=
        \beta_{q'}(t-s') -
        \left[ D_q(t) - D_q(s') \right]
        \label{eq:departing-low}
    \end{equation}
    Given that at $s'$ there is no backlog,
    and the strict arrival curve for the
    high priority queue $q$, we know
    \begin{multline}
        D_q(t)-D_q(s')=D_q(t)-A_q(s')\\
        \leq A_q(t)-A_q(s')
        \leq \gamma_{C_q,b_q}(t-s')
        \label{eq:low-depart-difference_prior}
    \end{multline}
    Then, we substitute ~\eqref{eq:low-depart-difference_prior}
    in~\eqref{eq:departing-low} to obtain
    \begin{multline}
        D_{q'}(t)-D_{q'}(s')
        = D_{q'}(t)-A_{q'}(s')\\
        \geq \left[ R_{q'}\left[
            (t-s)-\delta_{q'}
        \right]^+
        - \left( C_q(t-s')+b_q \right)
        \right]^+
        \label{eq:low-depart-difference}
    \end{multline}
    Again, depending on $(t-s)$ the above
    expression changes. For
    $(t-s)>\delta_{q'}$,
    \eqref{eq:low-depart-difference}~becomes
    \begin{equation}
        D_{q'}(t)\geq A_{q'}(s')+\left[
            \left( R_{q'}-C_q \right)
            (t-s') - \left( R_{q'}\delta_{q'}
            + b_q\right)
        \right]^+
    \end{equation}
    and with $(t-s')\leq\delta_{q'}$,
    inequality~\eqref{eq:low-depart-difference}
    becomes $D_{q'}(t)\geq A_{q'}(s')$.
    Overall, we can tell that the departing
    flow at the low priority queue $q'$
    satisfies
    \begin{equation}
        D_{q'}(t)\geq A_{q'}\otimes\beta_{q'}^*(t)
    \end{equation}
    with $\beta_{q'}^*(t)$ being the
    rate-latency service curve with rate
    $R_{q'}-C_q$ and latency
    $\tfrac{R_{q'}\delta_{q'}+b_q}{R_{q'}-C_q}$.
\end{proof}


