\pdfoutput=1
\documentclass[sigconf,x11names]{acmart}

\newcommand{\ca}[1]{{\color{black}{#1}}}

\usepackage{xcolor}

%%%%%%%%%%
%% TIKZ %%
%%%%%%%%%%
\usepackage{tikz}
\usepackage{pgfplots}
\usepackage{animate}
\usetikzlibrary{calc}
\usetikzlibrary{positioning}
\usetikzlibrary{shapes,arrows, positioning, calc}
\usetikzlibrary{overlay-beamer-styles}
\usetikzlibrary{chains,shapes.multipart}
\usetikzlibrary{scopes}
\usetikzlibrary{automata}
\usetikzlibrary{positioning}  %                 ...positioning nodes
\usetikzlibrary{arrows}       %                 ...customizing arrows
\usetikzlibrary{intersections}

% Glossary
\usepackage{glossaries}
\chapter*{Glossary of Sets and Constructions}
\label{chap:Glossary}
% next resets the equation numbers to start at 1 at the start of the chapter
\setcounter{equation}{0}
\renewcommand{\theequation}{\thechapter.\arabic{equation}}

We give a table that details all of the major sets and operators that are used in this thesis, for convenience and for reference.

%\begin{table}[h]
%\begin{tabular}{p{2.5cm}|p{8cm}|p{2cm}}
\begin{longtable}{p{2.5cm}|p{8.25cm}|p{1.75cm}}
\textbf{Name} & \textbf{Description} & \textbf{Thesis Ref.} \\
 \midrule
$m$-reducibility & Given two sets $A$ and $B$, $A$ is $m$-reducible to $B$, written $A \leq_m B$, if there exists some computable function $f: \omega \rightarrow \omega$ such that for all $x \in \omega$, $x \in A \iff f(x) \in B$ & \ref{def:mred}\\
\hline
Weihrauch Reducibility & Given two operators $f$ and $g$ on represented spaces, we say $f \leq_W g$, if there exist computable $H,K :\subseteq \omega^\omega \rightarrow \omega^\omega$ such that for any realizer $G \vdash g$, $F = K\langle id_{\omega^\omega}, GH \rangle$ is a realizer for $f$. & \ref{def:Weihrauch} \\
\hline
$WELL$ & The set of all indices $e$ such that $\varphi_e$ is the characteristic function of a well-founded tree $T \subseteq \omega^{<\omega}$. & \ref{def:WELL}\\
\hline
$ILL$ & The set of all indices $e$ such that $\varphi_e$ is the characteristic function of an ill-founded tree $T \subseteq \omega^{<\omega}$. & \ref{def:ILL}\\
\hline
$TILE$ & The set of all indices $e$ such that $\varphi_e$ is the characteristic function of an infinite Wang prototile set whose tilings are total in the plane. & \ref{def:TILE} \\
\hline
$WTILE$ & The set of all indices $e$ such that $\varphi_e$ is the characteristic function of an infinite Wang prototile set whose tilings are infinite, connected, but not necessarily total in the plane. & \ref{def:WTILE} \\
\hline
$SNT$ & The set of all indices $e$ such that $\varphi_e$ is the characteristic function of an infinite Wang prototile set whose connected tilings are all finite. & \ref{def:SNT}\\
\hline
$ATile$ & Set of all $e$ such that $\varphi_e$ is the characteristic function for a set of prototiles who planar tilings are all total and aperiodic. & \ref{def:ATile} \\
\hline
$PTile$ & Set of all $e$ such that $\varphi_e$ is the characteristic function for a set of prototiles who planar tilings are all total and periodic. & \ref{def:PTile} \\
\hline
$ATile_{FIN}$ & Set of all $e$ such that $\varphi_e$ is the characteristic function for a \emph{finite} set of prototiles who planar tilings are all total and aperiodic. & \ref{def:ATileFIN}\\
\hline
$PTile_{FIN}$ & Set of all $e$ such that $\varphi_e$ is the characteristic function for a \emph{finite} set of prototiles who planar tilings are all total and periodic. & \ref{def:PTileFIN}\\
\hline
AIT & The construction found in the proof of theorem \ref{thm:TILE-ILL} that creates an aperiodic prototile set given an ill-founded tree. & \ref{def:AITPIT}\\
\hline
PIT & The construction found in the proof of theorem \ref{thm:ILL-PTile} that creates an aperiodic prototile set given an ill-founded tree. & \ref{def:AITPIT}\\
\hline
$CT$ & The operator that takes some set of Wang prototiles as input and returns a total tiling of the plane. & \ref{def:ChooseTiling} \\
\hline
$CWPT$ & An operator that takes a set of Wang prototiles and returns a connected planar, but not necessarily total tiling. & \ref{ref:CWPT} \\
\hline
$CIPT$ & An operator that takes a prototile set $S$ that has total planar tilings, and returns an infinite `slice' of this tiling as a tiling of an infintie region of $\mathbb{Z}^2$. & \ref{def:CIPT} \\
\hline
$WIPT$ & An operator that takes a set of prototiles and return a tiling that has an infinite patch tiled within it, but we do not know where it is. & \ref{def:WIPT} \\
\hline
$DPW$ & The $DPW$ operator takes some set of prototiles and return a tiling that has an infinite connected patch within it. & \ref{def:DPW} \\
\hline
$C_{\omega^\omega}$ & Closed choice on Baire space - equivalent to finding a path through an ill-founded Baire space tree. & \ref{def:ClosedChoice} \\
\hline
$C_{2^\omega}$ & Closed choice on Cantor space - equivalent to Weak K\"onig's Lemma. & Sec \ref{sec:FurtherWeakTilingProblems} \\
\hline
$C_{\omega}$ & closed choice on the natural numbers - this takes a function $f: \omega \rightarrow \omega$ such that $range(f) \neq \omega$, and returns some point $n \notin range(f)$. & Sec. \ref{sec:FurtherWeakTilingProblems} \\
\end{longtable}
%\end{tabular}
%\end{table}



% MATH OPERATORS
\DeclareMathOperator*{\argmax}{arg\,max}
\DeclareMathOperator*{\argmin}{arg\,min}


% THEOREMS
\newtheorem{theorem}{Theorem}
\newtheorem{corollary}{Corollary}[theorem]
%%
%% \BibTeX command to typeset BibTeX logo in the docs
\AtBeginDocument{%
  \providecommand\BibTeX{{%
    Bib\TeX}}}

%% Rights management information.  This information is sent to you
%% when you complete the rights form.  These commands have SAMPLE
%% values in them; it is your responsibility as an author to replace
%% the commands and values with those provided to you when you
%% complete the rights form.
\setcopyright{none}
\copyrightyear{2023}
\acmYear{2023}
%\acmDOI{XXXXXXX.XXXXXXX}
\acmDOI{}

%% These commands are for a PROCEEDINGS abstract or paper.
%\acmConference[6G-PDN Workshop]{ACM MobiHoc 2023}{October 23-26, 2023}{Washington DC, USA}
\acmConference{}{}{}
%%
%%  Uncomment \acmBooktitle if the title of the proceedings is different
%%  from ``Proceedings of ...''!
%%
%%\acmBooktitle{Woodstock '18: ACM Symposium on Neural Gaze Detection,
%%  June 03--05, 2018, Woodstock, NY}
\acmPrice{}
\acmISBN{}


%%
%% Submission ID.
%% Use this when submitting an article to a sponsored event. You'll
%% receive a unique submission ID from the organizers
%% of the event, and this ID should be used as the parameter to this command.
%%\acmSubmissionID{123-A56-BU3}

%%
%% For managing citations, it is recommended to use bibliography
%% files in BibTeX format.
%%
%% You can then either use BibTeX with the ACM-Reference-Format style,
%% or BibLaTeX with the acmnumeric or acmauthoryear sytles, that include
%% support for advanced citation of software artefact from the
%% biblatex-software package, also separately available on CTAN.
%%
%% Look at the sample-*-biblatex.tex files for templates showcasing
%% the biblatex styles.
%%

%%
%% The majority of ACM publications use numbered citations and
%% references.  The command \citestyle{authoryear} switches to the
%% "author year" style.
%%
%% If you are preparing content for an event
%% sponsored by ACM SIGGRAPH, you must use the "author year" style of
%% citations and references.
%% Uncommenting
%% the next command will enable that style.
%%\citestyle{acmauthoryear}

\pgfplotsset{compat=1.18}

%%
%% end of the preamble, start of the body of the document source.
\begin{document}

%%
%% The "title" command has an optional parameter,
%% allowing the author to define a "short title" to be used in page headers.
%\title{To TWT, or not to TWT,
%that is the question}
\title{Aligning rTWT with 802.1Qbv: a Network Calculus Approach}

%%
%% The "author" command and its associated commands are used to define
%% the authors and their affiliations.
%% Of note is the shared affiliation of the first two authors, and the
%% "authornote" and "authornotemark" commands
%% used to denote shared contribution to the research.
\author{Carlos Barroso-Fernández}
%\authornote{Both authors contributed equally to this research.}
\email{cbarroso@pa.uc3m.es}
\affiliation{%
  \institution{Universidad Carlos III de Madrid}
  \streetaddress{Avda. Universidad, 30}
  \city{Leganés}
  \state{Madrid}
  \country{Spain}
  \postcode{28911}
}
\author{Jorge Martín-Pérez}
\orcid{0000-0001-9295-1601}
%\authornotemark[1]
\email{jorge.martin.perez@upm.es}
\affiliation{%
  \institution{Universidad Politécnica de Madrid}
  \streetaddress{Avda. Complutense, 30}
  \city{Madrid}
  \state{Madrid}
  \country{Spain}
  \postcode{28040}
}
\author{Constantine Ayimba}
\email{ayconsta@it.uc3m.es}
\affiliation{%
  \institution{Universidad Carlos III de Madrid}
  \streetaddress{Avda. Universidad, 30}
  \city{Leganés}
  \state{Madrid}
  \country{Spain}
  \postcode{28911}
}
\author{Antonio de la Oliva}
\email{aoliva@it.uc3m.es}
\affiliation{%
  \institution{Universidad Carlos III de Madrid}
  \streetaddress{Avda. Universidad, 30}
  \city{Leganés}
  \state{Madrid}
  \country{Spain}
  \postcode{28911}
}
%%
%% By default, the full list of authors will be used in the page
%% headers. Often, this list is too long, and will overlap
%% other information printed in the page headers. This command allows
%% the author to define a more concise list
%% of authors' names for this purpose.
\renewcommand{\shortauthors}{Barroso et al.}

%%
%% The abstract is a short summary of the work to be presented in the
%% article.
\begin{abstract}
    \ca{Industry 4.0 applications impose the challenging demand of delivering packets with bounded latencies via a wireless network. This is further complicated if the network is not dedicated to the time critical application. In this paper we use network calculus analysis to derive closed form expressions of latency bounds for time critical traffic when 802.11 \gls{twt} and 802.1Qbv work together in a shared 802.11 network.}
\end{abstract}

%%
%% The code below is generated by the tool at http://dl.acm.org/ccs.cfm.
%% Please copy and paste the code instead of the example below.
%%
\begin{CCSXML}
<ccs2012>
   <concept>
       <concept_id>10003033.10003079.10011672</concept_id>
       <concept_desc>Networks~Network performance analysis</concept_desc>
       <concept_significance>500</concept_significance>
       </concept>
   <concept>
       <concept_id>10003033.10003079.10003080</concept_id>
       <concept_desc>Networks~Network performance modeling</concept_desc>
       <concept_significance>300</concept_significance>
       </concept>
 </ccs2012>
\end{CCSXML}

\ccsdesc[500]{Networks~Network performance analysis}
\ccsdesc[300]{Networks~Network performance modeling}

%%
%% Keywords. The author(s) should pick words that accurately describe
%% the work being presented. Separate the keywords with commas.
\keywords{Network Calculus, Scheduling, 802.11Qbv, restricted Target Wake Time (rTWT)}


%\received{20 February 2007}
%\received[revised]{12 March 2009}
%\received[accepted]{5 June 2009}

%%
%% This command processes the author and affiliation and title
%% information and builds the first part of the formatted document.
\maketitle

\section{Introduction}

\ca{Unscheduled channel access in 802.11 networks negatively impacts their reliability particularly in dense deployments. This is a particularly challenging phenomenon given the imminent use of wireless networks for industry 4.0 applications~\cite{industry4_0}. %Recent improvements proposed in 802.11ax regarding 
\gls{twt} reduces contention by allowing the AP to instruct \gls{sta}s to turn their transceivers on or off at agreed intervals. The discipline thus transforms a dense deployment to one with a manageable number of \gls{sta}s during a \gls{twt} session therefore reducing contention and improving the radio channel access delay. 
Moreover, 802.11be will extend it to \gls{rtwt} to completely eliminate contention\footnote{In 802.11ax contention can be ameliorated with techniques such as trigger-based \gls{ofdma}.}. %Maybe we need to change rTWT to TWT.
%The standard allows \gls{ofdma} in both uplink~(UL) and downlink~(DL) therefore \gls{sta}s can be assigned different \gls{ru}s for transmission in both directions. 
This method has been proposed as a scheduling mechanism in \cite{Chen_TWTscheduler} that achieves improvement in overall network throughput but does not consider the priority of packets. This approach is therefore inadequate for \gls{tsn} flows requiring bounded latency. We propose an enhancement, incorporating \gls{tsn}, by aligning channel access via \gls{rtwt} with 802.1Qbv scheduling. The authors of~\cite{TWT_PoC} provide a proof of concept implementation of \gls{twt} for time aware scheduling that isolates priority traffic from best effort traffic achieving promising results on bounded latency and jitter. We leverage network calculus analysis~\cite{leboudec} to prove that bounded latencies, required by \gls{tsn}, can be achieved using this approach when both high and low priority packets are present in a \gls{sta} with some violation probability that depends on channel quality. In Section~\ref{sec:Qbv_tdma} we model the 802.1Qbv opening gates using Network Calculus~\cite{netcal8021qbv,leboudec}, and in Section~\ref{subsec:concurrent} the wireless errors using stochastic scaling~\cite{stochastic-scaling}. We discuss results in Section~\ref{sec:results} and conclusions in Section~\ref{sec:conclusions}.}

% Figure environment removed

%\section{A Network Calculus approach}
%In this section we model the 802.1Qbv opening gates using Network Calculus~\cite{netcal8021qbv,leboudec}, and the wireless errors using stochastic scaling~\cite{stochastic-scaling}.

\section{Modelling 802.1Qbv as TDMA curves}
\label{sec:Qbv_tdma}
The \gls{tas} in 802.1Qbv decides when to open and close each queue using a \gls{gcl}. The latter defines both the periodicity $T_q$ of each queue and how long, $L_q$, it is open for transmission. Using \gls{tdma}, a user queue $q$ has a transmit service curve (cummulative bits processed over time), $\beta_{L,T}(t)$, given by \eqref{eq:tdma-service-curve}
\begin{equation}
    \beta_{L,T}(t)=
    C\cdot\max\left\{ \left\lfloor \frac{t}{T}\right\rfloor L,
    t-\left\lceil\frac{t}{T} \right\rceil (T-L) \right\}
    \label{eq:tdma-service-curve}
\end{equation}
with $C$ being the transmission rate and $T,L$ the periodicity and length of the \gls{tdma} transmission opportunities. The service curve in 802.1Qbv for a queue $q$ is given by $\beta_{L_q,T_q}([t-\delta_q]^+)$ where $\delta_q>0$ is the time at which queue $q$ gate is first opened.
%% Figure environment removed

Note that in 802.11Qbv, two or more queues may have their gates opened concurrently\footnote{\ca{In this paper}, we consider 2 queues for brevity. Our analysis can be easily extended to $N$ concurrent queues.}. We use an affine lower envelope of the \gls{tdma} service curve~\eqref{eq:tdma-service-curve} to handle gate concurrency as shown in the  (top) dashed line of Figure~\ref{fig:concurrent}. The \gls{tdma} affine lower envelope for queue $q$ in 802.1Qbv is
the rate-latency function
\begin{equation}
    \beta_q(t)=\frac{C\cdot L_q}{T_q+L_q}[t-\delta_q]^+.
\end{equation}


\section{Concurrent 802.1Qbv opened gates}
\label{subsec:concurrent}




When high priority queue $q$ and low priority queue $q'$ are opened concurrently, packets of $q$ are transmitted ahead of those from $q'$.

%We can tell $q$ incoming traffic is governed by a strict arrival curve $\gamma_{C_q,b_q}(t)=C_q\cdot t + b_q$, and queue $q'$ by a strict arrival curve $\gamma_{C_{q'},b_{q'}}(t)$. Therefore, it is possible to know how the service curve of the low priority queue $q'$ is impacted by the high priority traffic.


\begin{corollary}[802.1Qbv High priority service curve]
    \label{cor:HPservice}
    Given two queues $q,q'$ whose gates
    are opened concurrently; the high
    priority queue $q$ has a strict service
    curve
    \begin{equation}
        \label{eq:HPservice}
        \beta_q^*(t) = R_q\left[t-\left(\delta_q+\frac{l_{\max}^{q'}}{R_q}\right) \right]^+
    \end{equation}
    with $R_q=\tfrac{C L_q}{T_q+L_q}$ and
    $l_{\max}^{q'}$ the maximum packet size
    of the low priority queue $q'$.
\end{corollary}
\begin{proof}
    We mimic the proof of
    \cite{leboudec}[Proposition 1.3.4].
    Take the affine lower envelope of queue
    $q$ defined
    in~\eqref{eq:tdma-service-curve}. Consider
    $s$ as the start of the backlog period of
    queue $q$, hence, the arrival and departing
    curve at queue $q$ is the same at that time
    $D_q(s)=A_q(s)$.

    For 802.1Qbv is non-preemptive, if a
    low priority packet from $q'$ arrives before
    a high priority packet from $q$ in the
    interval $(s,t]$; the high priority packet
    will wait, i.e.
    \begin{equation}
        D_q(t)-D_q(s)\geq \beta_q(t-s)-l_{\max}^{q'}
    \end{equation}
    given $D_q(s)=A_q(s)$ and
    $A,D\in\mathcal{F}=\{f: \mathbb{R}^+\to \mathbb{R}, \forall t\geq s:\ f(t)\geq f(s), f(0)=0\}$;
    we know
    \begin{equation}
        D_q(t)\geq A_q(s)+
        \left[ R_q\left[ (t-s)-\delta_q \right]^+  - l_{\max}^{q'} \right]^+
        \label{eq:depart-high}
    \end{equation}
    If $(t-s)>\delta_q$
    \eqref{eq:depart-high} becomes
    \begin{equation}
        D_q(t)\geq
            A_q(s)+
            R_q\left[ (t-s)- \left(\delta_q+ \frac{l_{\max}^{q'}}{R_q} \right) \right]^+
    \end{equation}
    If $(t-s)\leq\delta_q$ 
    \eqref{eq:depart-high} becomes
    $D_q(t)\geq A_q(s)$.
    That is, the 
    departing flow of the high priority
    queue $q$ satisfies
    \begin{equation}
        D_q(t)\geq A_q\otimes \beta_q^*(t)
    \end{equation}
    with $\beta_q^*(t)$ being the
    rate-latency service curve with
    rate $R_q$ and latency 
    $\delta_q+\frac{l_{\max}^{q'}}{R_q}$.
\end{proof}

Similarly, it is also possible to obtain the
service curve for the traffic in the
low priority queue $q'$ when such queue is
opened concurrently with the higher priority
queue $q$.


\begin{corollary}[802.1Qbv Low priority service curve]
    \label{cor:LPservice}
    Given two queues $q,q'$ whose gates
    are opened concurrently;
    if the flow at $q$ has
    a strict affine arrival curve
    $\gamma_{C_q,b_q}$, then the low
    priority queue $q'$ has a strict service
    curve
    \begin{equation}
        \label{eq:LPservice}
        \beta_{q'}^*(t)=
        \left(R_{q'}-C_q \right)
        \left[
            t
            - \frac{ R_{q'}\delta_{q'}+ b_q}{R_{q'}-C_q}
        \right]^+
    \end{equation}
    with $R_{q'}=\tfrac{C L_{q'}}{T_{q'}+L_{q'}}$.
\end{corollary}
\begin{proof}
    We again mimic the proof presented in \cite{leboudec} [Proposition 1.3.4].
    Take the affine lower envelope of
    queue $q'$ as defined
    in~\eqref{eq:tdma-service-curve}. Consider
    $s'$ the start of the busy period, that is,
    $s'<s$ with $s$ the start of the backlog
    period. In the interval $(s',t]$ the
    low priority queue departing traffic
    satisfies
    \begin{equation}
        D_{q'}(t)-D_{q'}(s')=
        \beta_{q'}(t-s') -
        \left[ D_q(t) - D_q(s') \right]
        \label{eq:departing-low}
    \end{equation}
    Given that at $s'$ there is no backlog,
    and the strict arrival curve for the
    high priority queue $q$, we know
    \begin{multline}
        D_q(t)-D_q(s')=D_q(t)-A_q(s')\\
        \leq A_q(t)-A_q(s')
        \leq \gamma_{C_q,b_q}(t-s')
        \label{eq:low-depart-difference_prior}
    \end{multline}
    Then, we substitute ~\eqref{eq:low-depart-difference_prior}
    in~\eqref{eq:departing-low} to obtain
    \begin{multline}
        D_{q'}(t)-D_{q'}(s')
        = D_{q'}(t)-A_{q'}(s')\\
        \geq \left[ R_{q'}\left[
            (t-s)-\delta_{q'}
        \right]^+
        - \left( C_q(t-s')+b_q \right)
        \right]^+
        \label{eq:low-depart-difference}
    \end{multline}
    Again, depending on $(t-s)$ the above
    expression changes. For
    $(t-s)>\delta_{q'}$,
    \eqref{eq:low-depart-difference}~becomes
    \begin{equation}
        D_{q'}(t)\geq A_{q'}(s')+\left[
            \left( R_{q'}-C_q \right)
            (t-s') - \left( R_{q'}\delta_{q'}
            + b_q\right)
        \right]^+
    \end{equation}
    and with $(t-s')\leq\delta_{q'}$,
    inequality~\eqref{eq:low-depart-difference}
    becomes $D_{q'}(t)\geq A_{q'}(s')$.
    Overall, we can tell that the departing
    flow at the low priority queue $q'$
    satisfies
    \begin{equation}
        D_{q'}(t)\geq A_{q'}\otimes\beta_{q'}^*(t)
    \end{equation}
    with $\beta_{q'}^*(t)$ being the
    rate-latency service curve with rate
    $R_{q'}-C_q$ and latency
    $\tfrac{R_{q'}\delta_{q'}+b_q}{R_{q'}-C_q}$.
\end{proof}




\section{Retransmissions in 802.11}
\label{sec:losses}

So far, we have obtained the service curves $\beta_q^*(t),\beta_{q'}^*(t)$ of high/low priority traffic upon the opened/closed 802.1Qbv gates. By aligning gate opening with the \gls{rtwt} sessions as depicted in Figure~\ref{fig:concurrent}, channel contention is avoided since the \gls{sta} has a dedicated \gls{ru} in 802.11ax \gls{mu} Tx, and the \gls{rtwt} session access is restricted.

However, packet errors may still occur due to channel impairments resulting in retransmissions. To deal with the latter, we use the stochastic scaling approach proposed in~\cite{stochastic-scaling}. The idea is to send the departing traffic of 802.1Qbv $D_q(t),D_{q'}(t)$ to a wireless channel modeled with a scaling process $S$ as shown in Figure~\ref{fig:scaling}. If the Tx does not succeed, the \gls{sta} waits time $W$ (as maximum) before retransmission of the i\textsuperscript{th} packet. This process is modelled with the scaling curve $\delta_W$.
% Figure environment removed

Following~\cite{stochastic-scaling}, we consider a binary symmetric channel modelled with stochastic scaling process $S(b)=\sum_i^b X_i,\ X_i\overset{\mathrm{iid}}{\sim} \text{Bernoulli}(p)$, with $p$ the
loss probability.
We now define a stochastic scaling
curve $S^\varepsilon(b)=pb+1-\varepsilon$
that, according to~\cite[Theorem~1]{stochastic-scaling},
satisfies
\begin{equation}
    \mathbb{P}\left( \sup_{0\leq a\leq b} \left\{ S(b)-S(a)-S^\varepsilon(b-a) \right\}\leq0\right)\geq 1-\varepsilon,\quad \forall b\geq0
\end{equation}
that is, the queue experiencing Tx errors $S(D(t))$ is upper bounded by the outgoing traffic (which is bounded by $S^\varepsilon(D(t))$) with probability $1-\varepsilon$.

The scaled flow $S(D(t))$ will result in
retransmissions, namely, we have
$\alpha^{(i)}(t),\ i=0,1,\ldots,N$ being the
arrival curves for the 
i\textsuperscript{th} retransmissions{\color{black}, where $\alpha^{(0)}=A(t)=\gamma_{r,b}$ represents the original input flow}. The higher the Rtx index, the higher the priority, thus each retransmission flow has arrival and service curves
\begin{equation}
    \beta_q^{(i)}=\left[
        \beta-\sum_{k=i+1}^N \alpha^{k}
    \right]^+,
    \qquad
    \alpha_q^{(i)}=
    S^\varepsilon\left(
        \alpha^{(i-1)}\oslash\beta^{(i-1)}
    \right)\oslash \delta_W.
\end{equation}

%{\color{red}TODO update from here and
%say that the above results in a system
%of equations, and refer to \cite{stochastic-scaling}. Then, mention that the theorem in
%the paper ensures (1-epsion)N reliability,
%and show that the delay is obtained using
%h(...).}

%We can then define a stochastic scaling curve for $S(b)$ to know how likely the packets are retransmitted. 
{\color{black}
%As \cite[Section~IV.C]{stochastic-scaling} demonstrates, above definitions result in a system of equations for obtaining the arrival curves of all retransmissions. 

Once the system is solved, we define the aggregated arrival curve for the queue $q$ as 
%\begin{equation*}
    $\alpha^{(Tot)}_q=\sum_{j=0}^N \alpha^{(j)}_q.$
%\end{equation*}
At this point, given the aggregated arrival curve and the service curve, Network Calculus provides bounds to obtain the maximum delay experienced by queue $q$ using the following expression:
\begin{equation*}
    h(\alpha,\beta)\, \raisebox{.5pt}{:}\!= \sup_{s\geq0} \left\{ \inf\{u\geq0: \alpha(s)\leq\beta(s+u)\} \rule{0pt}{1em} \right\}.
\end{equation*}

Note that previous curves are derived from the stochastic scaling process that model the channel, hence, they are not deterministic. In fact, as \cite{stochastic-scaling} highlights, the reliability of the delay bound (i.e., the probability that it holds) will be 
\begin{equation}
    \mathbb P\left( d(t) \leq h(\alpha^{(Tot)},\beta) \right)
    \geq (1-\varepsilon)^N \qquad \forall t\in\mathbb R^+,
\end{equation}
with $d(t)$ representing the real delay experienced by packets at time $t$.
}

% Notice that the wireless medium is randomly faulty by nature. Hence, it is impossible to obtain strict bounds that are never violated. However, Network Calculus provides solutions for this kind of links, in particular, also considering the retransmissions of the missing packets (feature that IEEE 802.11 \gls{phy} layer supports) -- see \cite{stochastic-scaling}.
% %Following the same reasoning as \cite{stochastic-scaling}, we represent 
% The failure and retransmission processes are represented with a stochastic scaling which is fulfilled with a certain probability, and a fixed waiting time $W$ representing the feedback delay for detecting a faulty packet.

%Consider the scaling process $S$. A function $S^\varepsilon$ (real-valued, non-negative and wide-sense increasing) is said to be a maximum stochastic scaling curve of $S$ if for all $b\geq0$:
%\begin{equation*}
%    p\left( \sup_{0\leq a\leq b} \left\{ s(b)-s(a)-s^\varepsilon(b-a) \right\}\leq0\right)\geq 1-\varepsilon
%\end{equation*}

%Let $\alpha^{(i)}\ i=0,1,...,N$ be the arrival curves for the $i^{\text{th}}$ retransmission flow, (e.g., $\alpha^{(0)}$ is the flow with non-retransmitted packets), being $N$ the maximum number of allowed retransmissions for a single packet.
%In the same form, $\beta^{(i)}\ i=0,1,...,N$ are the service curves for the $i^{\text{th}}$ retransmission flow. 
%To obtain the aggregated arrival curve, the following assumptions are required: $i$) $\alpha^{(0)}=\gamma_{r,b}$, i.e., the original input flow is an affine curve; $ii$) the service curve is of the form $\beta_{R,T}$ (notice that both equations \eqref{eq:HPservice} and \eqref{eq:LPservice} can be expressed in this form); and $iii$) the stochastic scaling of all flows are 
%$S(b)=X_1+X_2+...+X_b$, where $X_i\in\{0,1\}\ b=1,2,...,b$ are $i.i.d.$ Bernoulli random variables with parameter $C$, and bounded by the same stochastic scaling curve $S^\varepsilon=Cb+B$, where $B\geq 0$ and $0\leq C<1$ --- see \cite{stochastic-scaling} for demonstration.

%\begin{corollary}[802.1Qbv High priority aggregated arrival curve] \label{cor:HParrival}
%    Given the scenario described in Corollary \ref{cor:HPservice}, the aggregated arrival curve for the high priority queue for the case of $N$ retransmissions is
%    \begin{equation}
%        \label{eq:HParrival}
%        \alpha^{(Tot)}=\sum_{j=0}^N \alpha^{(j)}, \qquad \alpha^{(j)}=\gamma_{C^jr,b_{j,\infty}},
%    \end{equation}
%    where $b_{j,\infty} = C^jr(T_{j,\infty}+T_{j-1,\infty}+...+T_{1,\infty})+C^jb+(C^{j-1}+...+C+1)B+jC^jrW$; $T_{j,\infty}={det(A_j)}/{det(A)}$; and $A_j$ is the matrix $A$ with the $j^{\text{th}}$ column substituted by $\phi$, being
%    \begin{align*}
%        A=\begin{pmatrix} R_q-2r\sum_{i=1}^N C^i & -r\sum_{i=2}^N C^i & \cdots & -r\sum_{i=N}^N C^i \\ -r\sum_{i=2}^N C^i & R_q-2r\sum_{i=2}^N C^i & \cdots & -r\sum_{i=N}^N C^i \\ \vdots & \vdots & \ddots & \vdots \\ -r\sum_{i=N}^N C^i & -r\sum_{i=N}^N C^i & \cdots & R_q-2r\sum_{i=N}^N C^i
%        \end{pmatrix}\!, \\
%        \phi = \begin{pmatrix}
%            R_q(\delta_q+\frac{l_\text{max}^{q'}}{R_q})+b\sum_{i=1}^N C^i + B\sum_{p=0}^{N-1}\sum_{i=0}^p C^i + rW\sum_{i=1}^N iC^i \\
%           R_q(\delta_q+\frac{l_\text{max}^{q'}}{R_q})+b\sum_{i=2}^N C^i + B\sum_{p=1}^{N-1}\sum_{i=0}^p C^i + rW\sum_{i=2}^N iC^i \\ \vdots \\
%          %R_q(\delta_q+\frac{l_\text{max}^{q'}}{R_q})+b\sum_{i=N}^N C^i + B\sum_{p=N-1}^{N-1}\sum_{i=0}^p C^i + rW\sum_{i=N}^N iC^i
%         R_q(\delta_q+\frac{l_\text{max}^{q'}}{R_q})+bC^N + B\sum_{i=0}^{N-1} C^i + rWNC^N
%        \end{pmatrix}\!.
%    \end{align*}
%\end{corollary}
%\begin{proof}
%    From the results obtained in \cite{stochastic-scaling}, it is straightforward to demonstrate the corollary, by substituting $\beta_{R,T}$ by Eq. \eqref{eq:HPservice}.
%\end{proof}

%Notice that to guarantee stability, the rate of the service curve must be greater than the rate of the aggregated arrival curve, i.e., $R_q>r\sum_{i=0}^N C^i$.
%
%\begin{corollary}[802.1Qbv Low priority aggregated arrival curve]
%    \label{cor:LParrival}
%        Given the scenario described in Corollary \ref{cor:LPservice}, and an arrival curve $\gamma_{r,b}$ for the high priority queue, the aggregated arrival curve for the low priority queue for the case of $N$ retransmissions is the same as Eq. \eqref{eq:HParrival}, changing $A$ and $\phi$ as follows:
%    \begin{align*}
%        A=\begin{pmatrix} \omega_1 %R_{q'}-r-2r\sum_{i=1}^N C^i 
%        & -r\sum_{i=2}^N C^i & \cdots & -r\sum_{i=N}^N C^i \\ -r\sum_{i=2}^N C^i & \omega_2 %R_{q'}-r-2r\sum_{i=2}^N C^i 
%        & \cdots & -r\sum_{i=N}^N C^i \\ \vdots & \vdots & \ddots & \vdots \\ -r\sum_{i=N}^N C^i & -r\sum_{i=N}^N C^i & \cdots & \omega_N %R_{q'}-r-2r\sum_{i=N}^N C^i
%        \end{pmatrix}, \\
%        \phi = \begin{pmatrix}
%            %(R_{q'}-r)(\frac
%            {R_{q'}\delta_{q'}+b} %{R_{q'}-r})
%            +b\sum_{i=1}^N C^i + B\sum_{p=0}^{N-1}\sum_{i=0}^p C^i + rW\sum_{i=1}^N iC^i \\
%            %(R_{q'}-r)(\frac
%            {R_{q'}\delta_{q'}+b} %{R_{q'}-r})
%            +b\sum_{i=2}^N C^i + B\sum_{p=1}^{N-1}\sum_{i=0}^p C^i + rW\sum_{i=2}^N iC^i \\ \vdots \\
%            %(R_{q'}-r)(\frac
%            {R_{q'}\delta_{q'}+b} %{R_{q'}-r})
%            %+b\sum_{i=N}^N C^i + B\sum_{p=N-1}^{N-1}\sum_{i=0}^p C^i + rW\sum_{i=N}^N iC^i
%            +bC^N + B\sum_{i=0}^{N-1} C^i + rWNC^N
%        \end{pmatrix},
%    \end{align*}
%    being $\omega_j=R_{q'}-r-2r\sum_{i=j}^N C^i$.
%\end{corollary}
%\begin{proof}
%    Again, using the results from \cite{stochastic-scaling}, it is straightforward to demonstrate it, by substituting $\beta_{R,T}$ by Eq. \eqref{eq:LPservice}.
%\end{proof}

%In this case, the condition for the stability is $R_{q'}-r>r\sum_{i=0}^N C^i$. Notice also that one can substitute $r$ and $b$ by $r\sum_{i=0}^N C^i$ and $\sum_{i=0}^N b_{i,\infty}$ from Corollary \ref{cor:HParrival}, respectively.


%However, as it is mentioned earlier, these bounds may be violated. The probability that the arrival curves hold is 
%\begin{equation}
%    (1-\varepsilon)^N,
%\end{equation}
%obtained from \cite{stochastic-scaling}.
%%@CB This is not what I meant, the figure needs to be in the % Figure environment removed 
%\input{tikz/plot}
%% Figure environment removed


\section{Results}
\label{sec:results}
\section{Experimental Results}\label{sec:results}
    \subsection{General Results}
        The basic ResSAN model is used to determine reference results which our expanded model can be compared to as it is structurally similar to ResLAN but does not possess the Lidar adaptive components of it. Further, we compare with the full-size PackNet-SAN and the unmodified NLSPN architecture. 
        As it can be seen from Tab.\,\ref{tab:sota-results}, our LiDAR-adaptive ResLAN achieves competitive performance compared to state-of-the-art standard depth completion methods, which are specialized to the unfiltered 64-beam-LiDAR. The performance differences are in the range of a few centimetres in terms of MAE, which is acceptable given the practical advantage that ResLAN can generalize to different beam patterns as will be shown below.

        Furthermore, we compared the architectures for a set of three different input types that contained 64, 32 or 16 LiDAR channels using both filter types on the metrics from the KITTI benchmark. The NLSPN model was trained for the standard depth completion task and then evaluated with different input data. As for the ResSAN models, we trained one model for each input type and tested it for the corresponding one which serve serve as the \emph{Baseline} in Tab.\,\ref{tab:overall-results}. Our ResLAN model was jointly trained for all three settings. As listed in Tab.\,\ref{tab:overall-results}, the ResLAN models outperform the challenging baseline in all metrics for FOV filtering and all but one for sparse filtering. This implies that our LiDAR adaptive model is able to outperform dedicated models in case of very sparse input depth. Fig.\,\ref{fig:comp-plot} shows this is indeed the case for 32 and even more for 16 channels. For FOV-filtered inputs with 16 channels, the ResLAN exhibits approx. $10\%$ smaller MAE than the baseline. As for the NLSPN, it becomes apparent that it is not capable of generalizing to other input types since it shows clearly worse results. The difference is especially pronounced for the FOV filtering where on average more than every fourth predicted pixel is more than $25 \%$ deviating from the ground truth\,($\delta_{1.25}$). Therefore, using a weight-adapting network in combination with differently filtered input depths allows us to train models that outperform their non-adaptive counterparts.

        \begin{table}[]
            \centering
    	    \small
            \vspace{0.4cm}
            \caption{\textbf{Depth estimation result for standard depth completion} when the ResSAN model was only trained for 64 channels and the ResLAN model for multiple tasks. The PackNet-SAN and NLSPN models were trained with the setup that was also used for our model architecture.}
            \footnotesize
            \setlength{\tabcolsep}{5pt}
            \begin{tabular}{@{}lrrrrl@{}}
            \toprule
            \multicolumn{6}{c}{\textbf{Standard LiDAR Depth Completion}}                                                                                                                         \\ \midrule
            \multicolumn{1}{l|}{Method}          & RMSE $\downarrow$            & MAE  $\downarrow$            & iRMSE $\downarrow$             & iMAE $\downarrow$ & $\delta_{1.25}$ $\uparrow$ \\
            \multicolumn{1}{l|}{}                & \multicolumn{1}{l}{{[}mm{]}} & \multicolumn{1}{l}{{[}mm{]}} & \multicolumn{1}{l}{{[}1/km{]}} & {[}1/km{]}        &                            \\ \midrule
            \multicolumn{1}{l|}{PackNet-SAN}     &  914                            &  298                            &  2.78                              &  1.4                 &  99.65 \%                          \\
            \multicolumn{1}{l|}{NLSPN}           &  \textbf{889}                            &   \textbf{263}                           &  \textbf{2.62}                              &   \textbf{1.3}                &   \textbf{99.61} \%                         \\ \midrule
            \multicolumn{1}{l|}{ResSAN (Ours)}   & 948                             &  275                            &  2.75                              &    1.4               &   99.58 \%                         \\
            \multicolumn{1}{l|}{ResLAN (Ours)} &   969                           &  283                            &   2.83                             &   1.4                &  99.56 \%                          \\ \bottomrule
            \end{tabular}
            \vspace{0.2cm}
            \label{tab:sota-results}
        \end{table}

        \begin{table}[]
    	    \centering
    	    \small
    	    \caption{\textbf{Depth estimation results of the two baseline setups and the explicit and implicit ResSAN} when evaluated on a combination of 16, 32 and 64 channel depth inputs. Please note that Specialist Methods need to train three specialized networks, one for each of the three types of inputs while our method only uses one network.}
            \footnotesize
            \setlength{\tabcolsep}{4.8pt}
            \begin{tabular}{@{}lrrrrl@{}}
                \toprule
                \multicolumn{6}{c}{\textbf{Sparse Channel Filter}}                                                                                                                                  \\ \midrule
                \multicolumn{1}{l|}{Method}        & RMSE $\downarrow$            & MAE  $\downarrow$            & iRMSE $\downarrow$             & iMAE $\downarrow$ & $\delta_{1.25}$ $\uparrow$  \\
                \multicolumn{1}{l|}{}              & \multicolumn{1}{l}{{[}mm{]}} & \multicolumn{1}{l}{{[}mm{]}} & \multicolumn{1}{l}{{[}1/km{]}} & {[}1/km{]}        &                             \\ \midrule
                \multicolumn{1}{l|}{NLSPN}         &  1396                            &  437                            & 5.54                               &  2.2                 &  98.82 \%                           \\
                \multicolumn{1}{l|}{Baseline}      & \textbf{1207}                             &  381                            & 4.41                               &  1.8                 &  \textbf{99.37} \%                           \\
                \multicolumn{1}{l|}{ResLAN (Ours)} &  1215                            &  \textbf{378}                            &  \textbf{4.27}                              &  \textbf{1.7}                 &  99.31 \%                           \\ \toprule
                \multicolumn{6}{c}{\textbf{Field-of-View Filter}}                                                                                                                                   \\ \midrule
                \multicolumn{1}{l|}{Method}        & RMSE $\downarrow$            & MAE  $\downarrow$            & iRMSE $\downarrow$             & iMAE $\downarrow$ & $\delta_{1.25}$ $\uparrow$ \\
                \multicolumn{1}{l|}{}              & \multicolumn{1}{l}{{[}mm{]}} & \multicolumn{1}{l}{{[}mm{]}} & \multicolumn{1}{l}{{[}1/km{]}} & {[}1/km{]}        &                             \\ \midrule
                \multicolumn{1}{l|}{NLSPN}         &  2738                            &  1702                            & 12.3                              &  4.3                 &  74.69 \%                           \\
                \multicolumn{1}{l|}{Baseline}      &  1556                            &  525                            &  6.8                              &  3.0                 & 98.14 \%                            \\
                \multicolumn{1}{l|}{ResLAN (Ours)} &  \textbf{1548}                            &  \textbf{519}                            &  \textbf{6.44}                              &  \textbf{2.8}                 & \textbf{98.52 \%}                            \\ \bottomrule
            \end{tabular}
            \label{tab:overall-results}
        \end{table}

        
        
        % Figure environment removed
        
        % Figure environment removed

    \subsection{Filter Effects}
        Comparing the effect of the two different types of depth input filters on the model performance, it becomes apparent that FOV filtering is the more challenging task. In that setting, reducing LiDAR channels is more detrimental to the performance than sparse filtering as it creates regions where no depth information is available. Effectively, the model is forced to perform depth prediction in these regions. These effects are highlighted in the depth images in Fig.\,\ref{fig:dense-maps} where the effect of a 16-channel sparse depth filter and a 16-channel FOV can be compared.

    \subsection{Generalization Capabilities}
        We trained three models for both filter types eaach, so the combinations and number of filtered depth inputs they receive are different. This serves the purpose of testing the generalization capabilities of the ResLAN architecture as well as the robustness to different filter settings. After training, the models were evaluated for the depth input settings they were trained for, as well as for ones they weren't exposed to. Overall, ResLAN shows good generalization capabilities. As one can gather from Fig.\,\ref{fig:explicit-comp} and Fig.\,\ref{fig:implicit-comp}, the consequences of slightly varying sets of input depth settings are limited. The most considerable deviations can be seen when the model is tasked to extrapolate. For instance, the model $\{64, 32, 16\}$ shows a noticeably higher MAE for eight-channel depth inputs than the model that was trained for it. Similar behaviour can be seen for the FOV filtering case as well for the model $\{64, 48, 32\}$ when tasked to generalize for a 16-channel input. There is no such pronounced effect for generalization tasks that lie between two filter settings the model was trained for. At most, it can be observed that models that were trained for a smaller range of filter values perform slightly better than ones that have to cover a wider range. The number of filter settings used in a fixed range does not relevantly influence the model performance, as can be seen, when comparing the two models in Fig.\,\ref{fig:implicit-comp}, which are both trained for a range of 64 to 32 channels but one with three filter settings and the other one with five.
    
    % Figure environment removed
    
    
    % Figure environment removed

\section{Conclusions and future work}
\label{sec:conclusions}
%% -*- mode: LaTeX; fill-column: 78; -*-

\section{Concluding Remarks}
\label{sec:conclusions}

In this paper, we presented a novel SMC algorithm, \EventDPOR, tailored to the
characteristics of event-driven multi-threaded programs running under the SC
semantics. The algorithm was proven correct and optimal for event-driven
programs in which the variable accesses of events do not depend on how their
execution is interleaved with other threads.

We have implemented \EventDPOR in the \Nidhugg tool, and we will open-source
our implementation.
%
With a wide range of event-driven programs, we have shown that \EventDPOR
incurs only a moderate constant overhead over its baseline implementation
(\OptimalDPOR), it is exponentially faster than existing state-of-the-art SMC
algorithms in time and number of traces examined on programs where events'
actions do not conflict, and does not suffer from performance degradation
caused by having to examine
% a significant number of
non-serializable executions.
%
%% \bjcom{Should we include:
%% Moreover, in our benchmarks, also those that are not non-branching,
%% \EventDPOR explores only the optimal number of executions, and never
%% had to resort to a potentially expensive decision procedure.}

\EventDPOR assumes that handlers can process their events in arbitrary order.
Directions for future work include to retarget \EventDPOR for event-driven
programs with other policies (e.g., FIFO), and for specific event-driven
execution models.



\section*{Acknowledgment}
This work has been partially funded by the European Commission Horizon Europe SNS JU PREDICT-6G (GA 101095890) Project and the Spanish Ministry of Economic Affairs and Digital Transformation and the European Union-NextGenerationEU through the UNICO 5G I+D 6G-EDGEDT and 6G-DATADRIVEN.
\bibliographystyle{acm}%{ACM-Reference-Format}
\bibliography{bibliography.bib}


%%
%% If your work has an appendix, this is the place to put it.

\end{document}
\endinput
