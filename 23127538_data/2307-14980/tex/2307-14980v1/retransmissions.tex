
So far, we have obtained the service curves $\beta_q^*(t),\beta_{q'}^*(t)$ of high/low priority traffic upon the opened/closed 802.1Qbv gates. By aligning gate opening with the \gls{rtwt} sessions as depicted in Figure~\ref{fig:concurrent}, channel contention is avoided since the \gls{sta} has a dedicated \gls{ru} in 802.11ax \gls{mu} Tx, and the \gls{rtwt} session access is restricted.

However, packet errors may still occur due to channel impairments resulting in retransmissions. To deal with the latter, we use the stochastic scaling approach proposed in~\cite{stochastic-scaling}. The idea is to send the departing traffic of 802.1Qbv $D_q(t),D_{q'}(t)$ to a wireless channel modeled with a scaling process $S$ as shown in Figure~\ref{fig:scaling}. If the Tx does not succeed, the \gls{sta} waits time $W$ (as maximum) before retransmission of the i\textsuperscript{th} packet. This process is modelled with the scaling curve $\delta_W$.
% Figure environment removed

Following~\cite{stochastic-scaling}, we consider a binary symmetric channel modelled with stochastic scaling process $S(b)=\sum_i^b X_i,\ X_i\overset{\mathrm{iid}}{\sim} \text{Bernoulli}(p)$, with $p$ the
loss probability.
We now define a stochastic scaling
curve $S^\varepsilon(b)=pb+1-\varepsilon$
that, according to~\cite[Theorem~1]{stochastic-scaling},
satisfies
\begin{equation}
    \mathbb{P}\left( \sup_{0\leq a\leq b} \left\{ S(b)-S(a)-S^\varepsilon(b-a) \right\}\leq0\right)\geq 1-\varepsilon,\quad \forall b\geq0
\end{equation}
that is, the queue experiencing Tx errors $S(D(t))$ is upper bounded by the outgoing traffic (which is bounded by $S^\varepsilon(D(t))$) with probability $1-\varepsilon$.

The scaled flow $S(D(t))$ will result in
retransmissions, namely, we have
$\alpha^{(i)}(t),\ i=0,1,\ldots,N$ being the
arrival curves for the 
i\textsuperscript{th} retransmissions{\color{black}, where $\alpha^{(0)}=A(t)=\gamma_{r,b}$ represents the original input flow}. The higher the Rtx index, the higher the priority, thus each retransmission flow has arrival and service curves
\begin{equation}
    \beta_q^{(i)}=\left[
        \beta-\sum_{k=i+1}^N \alpha^{k}
    \right]^+,
    \qquad
    \alpha_q^{(i)}=
    S^\varepsilon\left(
        \alpha^{(i-1)}\oslash\beta^{(i-1)}
    \right)\oslash \delta_W.
\end{equation}

%{\color{red}TODO update from here and
%say that the above results in a system
%of equations, and refer to \cite{stochastic-scaling}. Then, mention that the theorem in
%the paper ensures (1-epsion)N reliability,
%and show that the delay is obtained using
%h(...).}

%We can then define a stochastic scaling curve for $S(b)$ to know how likely the packets are retransmitted. 
{\color{black}
%As \cite[Section~IV.C]{stochastic-scaling} demonstrates, above definitions result in a system of equations for obtaining the arrival curves of all retransmissions. 

Once the system is solved, we define the aggregated arrival curve for the queue $q$ as 
%\begin{equation*}
    $\alpha^{(Tot)}_q=\sum_{j=0}^N \alpha^{(j)}_q.$
%\end{equation*}
At this point, given the aggregated arrival curve and the service curve, Network Calculus provides bounds to obtain the maximum delay experienced by queue $q$ using the following expression:
\begin{equation*}
    h(\alpha,\beta)\, \raisebox{.5pt}{:}\!= \sup_{s\geq0} \left\{ \inf\{u\geq0: \alpha(s)\leq\beta(s+u)\} \rule{0pt}{1em} \right\}.
\end{equation*}

Note that previous curves are derived from the stochastic scaling process that model the channel, hence, they are not deterministic. In fact, as \cite{stochastic-scaling} highlights, the reliability of the delay bound (i.e., the probability that it holds) will be 
\begin{equation}
    \mathbb P\left( d(t) \leq h(\alpha^{(Tot)},\beta) \right)
    \geq (1-\varepsilon)^N \qquad \forall t\in\mathbb R^+,
\end{equation}
with $d(t)$ representing the real delay experienced by packets at time $t$.
}

% Notice that the wireless medium is randomly faulty by nature. Hence, it is impossible to obtain strict bounds that are never violated. However, Network Calculus provides solutions for this kind of links, in particular, also considering the retransmissions of the missing packets (feature that IEEE 802.11 \gls{phy} layer supports) -- see \cite{stochastic-scaling}.
% %Following the same reasoning as \cite{stochastic-scaling}, we represent 
% The failure and retransmission processes are represented with a stochastic scaling which is fulfilled with a certain probability, and a fixed waiting time $W$ representing the feedback delay for detecting a faulty packet.

%Consider the scaling process $S$. A function $S^\varepsilon$ (real-valued, non-negative and wide-sense increasing) is said to be a maximum stochastic scaling curve of $S$ if for all $b\geq0$:
%\begin{equation*}
%    p\left( \sup_{0\leq a\leq b} \left\{ s(b)-s(a)-s^\varepsilon(b-a) \right\}\leq0\right)\geq 1-\varepsilon
%\end{equation*}

%Let $\alpha^{(i)}\ i=0,1,...,N$ be the arrival curves for the $i^{\text{th}}$ retransmission flow, (e.g., $\alpha^{(0)}$ is the flow with non-retransmitted packets), being $N$ the maximum number of allowed retransmissions for a single packet.
%In the same form, $\beta^{(i)}\ i=0,1,...,N$ are the service curves for the $i^{\text{th}}$ retransmission flow. 
%To obtain the aggregated arrival curve, the following assumptions are required: $i$) $\alpha^{(0)}=\gamma_{r,b}$, i.e., the original input flow is an affine curve; $ii$) the service curve is of the form $\beta_{R,T}$ (notice that both equations \eqref{eq:HPservice} and \eqref{eq:LPservice} can be expressed in this form); and $iii$) the stochastic scaling of all flows are 
%$S(b)=X_1+X_2+...+X_b$, where $X_i\in\{0,1\}\ b=1,2,...,b$ are $i.i.d.$ Bernoulli random variables with parameter $C$, and bounded by the same stochastic scaling curve $S^\varepsilon=Cb+B$, where $B\geq 0$ and $0\leq C<1$ --- see \cite{stochastic-scaling} for demonstration.

%\begin{corollary}[802.1Qbv High priority aggregated arrival curve] \label{cor:HParrival}
%    Given the scenario described in Corollary \ref{cor:HPservice}, the aggregated arrival curve for the high priority queue for the case of $N$ retransmissions is
%    \begin{equation}
%        \label{eq:HParrival}
%        \alpha^{(Tot)}=\sum_{j=0}^N \alpha^{(j)}, \qquad \alpha^{(j)}=\gamma_{C^jr,b_{j,\infty}},
%    \end{equation}
%    where $b_{j,\infty} = C^jr(T_{j,\infty}+T_{j-1,\infty}+...+T_{1,\infty})+C^jb+(C^{j-1}+...+C+1)B+jC^jrW$; $T_{j,\infty}={det(A_j)}/{det(A)}$; and $A_j$ is the matrix $A$ with the $j^{\text{th}}$ column substituted by $\phi$, being
%    \begin{align*}
%        A=\begin{pmatrix} R_q-2r\sum_{i=1}^N C^i & -r\sum_{i=2}^N C^i & \cdots & -r\sum_{i=N}^N C^i \\ -r\sum_{i=2}^N C^i & R_q-2r\sum_{i=2}^N C^i & \cdots & -r\sum_{i=N}^N C^i \\ \vdots & \vdots & \ddots & \vdots \\ -r\sum_{i=N}^N C^i & -r\sum_{i=N}^N C^i & \cdots & R_q-2r\sum_{i=N}^N C^i
%        \end{pmatrix}\!, \\
%        \phi = \begin{pmatrix}
%            R_q(\delta_q+\frac{l_\text{max}^{q'}}{R_q})+b\sum_{i=1}^N C^i + B\sum_{p=0}^{N-1}\sum_{i=0}^p C^i + rW\sum_{i=1}^N iC^i \\
%           R_q(\delta_q+\frac{l_\text{max}^{q'}}{R_q})+b\sum_{i=2}^N C^i + B\sum_{p=1}^{N-1}\sum_{i=0}^p C^i + rW\sum_{i=2}^N iC^i \\ \vdots \\
%          %R_q(\delta_q+\frac{l_\text{max}^{q'}}{R_q})+b\sum_{i=N}^N C^i + B\sum_{p=N-1}^{N-1}\sum_{i=0}^p C^i + rW\sum_{i=N}^N iC^i
%         R_q(\delta_q+\frac{l_\text{max}^{q'}}{R_q})+bC^N + B\sum_{i=0}^{N-1} C^i + rWNC^N
%        \end{pmatrix}\!.
%    \end{align*}
%\end{corollary}
%\begin{proof}
%    From the results obtained in \cite{stochastic-scaling}, it is straightforward to demonstrate the corollary, by substituting $\beta_{R,T}$ by Eq. \eqref{eq:HPservice}.
%\end{proof}

%Notice that to guarantee stability, the rate of the service curve must be greater than the rate of the aggregated arrival curve, i.e., $R_q>r\sum_{i=0}^N C^i$.
%
%\begin{corollary}[802.1Qbv Low priority aggregated arrival curve]
%    \label{cor:LParrival}
%        Given the scenario described in Corollary \ref{cor:LPservice}, and an arrival curve $\gamma_{r,b}$ for the high priority queue, the aggregated arrival curve for the low priority queue for the case of $N$ retransmissions is the same as Eq. \eqref{eq:HParrival}, changing $A$ and $\phi$ as follows:
%    \begin{align*}
%        A=\begin{pmatrix} \omega_1 %R_{q'}-r-2r\sum_{i=1}^N C^i 
%        & -r\sum_{i=2}^N C^i & \cdots & -r\sum_{i=N}^N C^i \\ -r\sum_{i=2}^N C^i & \omega_2 %R_{q'}-r-2r\sum_{i=2}^N C^i 
%        & \cdots & -r\sum_{i=N}^N C^i \\ \vdots & \vdots & \ddots & \vdots \\ -r\sum_{i=N}^N C^i & -r\sum_{i=N}^N C^i & \cdots & \omega_N %R_{q'}-r-2r\sum_{i=N}^N C^i
%        \end{pmatrix}, \\
%        \phi = \begin{pmatrix}
%            %(R_{q'}-r)(\frac
%            {R_{q'}\delta_{q'}+b} %{R_{q'}-r})
%            +b\sum_{i=1}^N C^i + B\sum_{p=0}^{N-1}\sum_{i=0}^p C^i + rW\sum_{i=1}^N iC^i \\
%            %(R_{q'}-r)(\frac
%            {R_{q'}\delta_{q'}+b} %{R_{q'}-r})
%            +b\sum_{i=2}^N C^i + B\sum_{p=1}^{N-1}\sum_{i=0}^p C^i + rW\sum_{i=2}^N iC^i \\ \vdots \\
%            %(R_{q'}-r)(\frac
%            {R_{q'}\delta_{q'}+b} %{R_{q'}-r})
%            %+b\sum_{i=N}^N C^i + B\sum_{p=N-1}^{N-1}\sum_{i=0}^p C^i + rW\sum_{i=N}^N iC^i
%            +bC^N + B\sum_{i=0}^{N-1} C^i + rWNC^N
%        \end{pmatrix},
%    \end{align*}
%    being $\omega_j=R_{q'}-r-2r\sum_{i=j}^N C^i$.
%\end{corollary}
%\begin{proof}
%    Again, using the results from \cite{stochastic-scaling}, it is straightforward to demonstrate it, by substituting $\beta_{R,T}$ by Eq. \eqref{eq:LPservice}.
%\end{proof}

%In this case, the condition for the stability is $R_{q'}-r>r\sum_{i=0}^N C^i$. Notice also that one can substitute $r$ and $b$ by $r\sum_{i=0}^N C^i$ and $\sum_{i=0}^N b_{i,\infty}$ from Corollary \ref{cor:HParrival}, respectively.


%However, as it is mentioned earlier, these bounds may be violated. The probability that the arrival curves hold is 
%\begin{equation}
%    (1-\varepsilon)^N,
%\end{equation}
%obtained from \cite{stochastic-scaling}.
%%@CB This is not what I meant, the figure needs to be in the % Figure environment removed 
%\input{tikz/plot}
%% Figure environment removed
