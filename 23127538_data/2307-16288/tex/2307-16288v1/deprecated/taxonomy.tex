\documentclass{article}
\begin{document}	
    Having presented the definitions of proactive prediction, we now present a brief overview of the prediction methods proposed in recent top conferences or journals, and taxonomize them according to their applicable domains and the techniques adopted.
    
    \subsection{Performance Prediction}
    \label{sec:3.1}
    Performance prediction techniques can be taxonomized along two axes. The first axis is the applicable domain, i.e., whether this technique is developed for end-to-end applications (especially data analytics) or only storage systems. The second axis is how they view the underlying device or system, which is discussed below.
    
    \begin{itemize}
        \item \textbf{Whitebox:} Whitebox approaches assume full knowledge of the underlying system internals. For instance, MittOS\cite{hao2017mittos} built its prediction model based on its understanding of contention and queueing discipline of the underlying resource (e.g., disk spindles vs. SSD channels, FIFO vs. priority). Some whitebox approaches may even re-architect the system in order to build a better prediction model\cite{ousterhout2017monotasks}.
        \item \textbf{Blackbox:} Blackbox approaches have no information about the internal components or algorithms of the underlying system. Thus, most blackbox approaches utilize machine learning to learn the systems behavior from examples. 

        Blackbox approaches are more generalizable and scalable than whitebox ones. Often, the device internals are proprietary and not exposed to researchers. Further, it's almost infeasible to deploy whitebox approaches in production deployment, since numerous vendors --- each with various device types --- are involved, where each type of device need expertise's inspection in order to build the whitebox model. 
    \end{itemize}
    
    \begin{table}[H]
    \centering
    $$
    \begin{array}{|c|c|c|}
    \hline \text {} & \textbf {Whitebox} & \textbf {Blackbox} \\
    \hline \textbf {Storage Systems} & \text {\cite{hao2017mittos},\cite{varki2004issues}} & \text{\textbf{\cite{hao2020linnos}},\cite{wang2004storage},\cite{yin2006empirical}}  \\
    \hline \textbf {End-to-End Applications} & \text {\cite{ousterhout2017monotasks},\cite{verma2011aria}} & \text{\textbf{\cite{fu2021use}},\cite{venkataraman2016ernest},\cite{hsu2016inside},\cite{yadwadkar2014wrangler}}  \\
    \hline
    \end{array}
    $$ 
    \caption{Classification of Performance Prediction}
    \label{tab:performancepreiction}
    \end{table}
    
    \subsection{Failure Prediction}
    \label{sec:3.2}
    Most failure prediction techniques are blackbox approaches, and they can be further taxonomized along two axes. The first axis is the applicable device, i.e., whether this technique is developed for HDDs or SSDs. The second axis is how the model is trained or built, which is discussed below.
    
    \begin{itemize}
        \item \textbf{Offline:} Offline scheme means that all training data must be available before training the prediction model. In other words, the model won't upgrade once it is deployed in production.
        \item \textbf{Online:} Online scheme means that the prediction model is updated incrementally in real time (i.e., after being deployed), upon receiving each newly generated healthy or failure sample. Some work utilizes online random forest algorithm\cite{xiao2018disk}, while other work formulates a stream mining problem so as to combat concept drift\cite{han2020toward}.

        Online approaches are generally more preferred than offline approaches, since disk logs are continuously generated, in which the statistic patterns may vary over time. That being said, offline schemes may suffer in long-term use, because the testing data is going to come from a different distribution from training data.
    \end{itemize}
    
    \begin{table}[H]
    \centering
    $$
    \begin{array}{|c|c|c|}
    \hline \text {} & \textbf {Offline} & \textbf {Online} \\
    \hline \textbf {SSD} & \text {\cite{xugeneral},\cite{mahdisoltani2017proactive},\cite{alter2019ssd}} & \text{-}  \\
    \hline \textbf {HDD} & \text {\textbf{\cite{lu2020making}},\cite{mahdisoltani2017proactive},\cite{xu2018improving}} & \text{\cite{han2020toward},\cite{xiao2018disk}}  \\
    \hline
    \end{array}
    $$ 
    \caption{Classification of Failure Prediction}
    \label{tab:failureprediction}
    \end{table}
    
    \subsection{Summary}
    \label{sec:3.3}
    In Table~\ref{tab:performancepreiction} and \ref{tab:failureprediction}, we conclude the classification of recent research efforts on performance prediction and failure prediction. It is not feasible to cover all the related literatures in this field. Thus, we are going to selectively focus on some representative ones (shown in bold in Table~\ref{tab:performancepreiction} and \ref{tab:failureprediction}), which are all ML based blackbox approaches due to their ability to scale and generalize.
    
    
    \ifcsdef{mainfile}{}{\bibliography{../references/primary}}
\end{document}