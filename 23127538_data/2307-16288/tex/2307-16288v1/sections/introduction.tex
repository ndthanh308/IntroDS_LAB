\documentclass{article}
\begin{document}	

    With the rapid development of cloud computing and big data technologies, storage systems have become a fundamental building block of datacenters, incorporating hardware innovations such as flash Solid State Drives (SSDs) and Non-Volatile Memories (NVMs), as well as software infrastructures such as RAID and distributed file systems. Despite of the growing popularity and interests in storage, designing and implementing \textit{reliable} storage systems are still challenging. One major reason is performance instability: RAIDs suffer 1.5\% (for HDDs) and 2.2\% (for SSDs) of RAID hours with at least one slow drive, which is 2x slower that its peer drives\cite{hao2016tail}. This indicates stable latencies at 99th percentile is hard to achieve in current RAID deployments, let alone the stricter application level SLOs. The other reason is prevailing hardware failures: the annualized failure rate is reported to be 1.36\% for HDDs\cite{lu2020making}, and 3.92\% for SSDs\cite{han2021depth,xugeneral}. Such high level of failure rates along with the massive scale of hardware fleets require careful device level health monitoring.
    
    Proactive prediction provides a new angle to achieve reliable storage systems in two dimensions: performance and failure. By detecting in advance the I/O requests whose performance expectations cannot be fulfilled, we can eventually cut the tail latencies and thereby guarantee SLOs; by predicting HDDs/SSDs failures before they really happen, we can achieve high availability even at the device level. It is well recognized that proactive prediction is integral for developing reliable storage systems in commercial datacenters, and it can act as a plug-in complement for existing approaches that aim to achieve reliable storage in datacenters(e.g., hedged requests, replication and RAIDs). While its importance is recognized, predictability of storage systems is particularly difficult, mainly due to the heterogeneity of drives in deployment, and the complex internal idiosyncrasies of modern storage devices.
    
    There are some popular approaches to mask such unpredictability, i.e. live with and embrace this unpredictability. A common practice for masking performance unpredictability is "hedged requests"\cite{dean2013tail}, which sends a duplicated I/O request to another node if it has been outstanding for more than the 95th percentile of expected latency. However, this approach increases the I/O tensity and results in poor resource utilization. A common practice for masking failure unpredictability is replication and erase coding, which reconstructs corrupted chunks using healthy data and parity chunks. However, faced with prevalent correlated failures\cite{han2021depth}, such redundancy schemes are shown to be insufficient. All these facts suggest that researchers should move \textit{towards the predictability of storage systems} instead of escaping from them.
    
    Over the past few years, researchers have devoted intensive effort in exploring proactive and accurate prediction of performance and failure in storage systems. Various techniques have been proposed for storage performance prediction, including "white-box" approaches that require detailed understanding of the device\cite{hao2017mittos}, as well as ML based "black-box" approaches\cite{hao2020linnos}. For storage failure prediction, researchers have proposed several ML based techniques for both SSDs\cite{xugeneral,mahdisoltani2017proactive} and HDDs\cite{lu2020making,han2020toward,mahdisoltani2017proactive}. In this paper, we are going to provide a survey of \textbf{machine learning based "black-box" approaches} for prediction tasks in storage systems.
    
    The rest of this paper is organized as follows. We start with problem statement in Section~\ref{sec:problemstatement}, by introducing the storage hierarchy in datacenters and defining the specific prediction tasks. Section~\ref{sec:taxonomy} provides a taxonomy of the mechanisms and field studies for prediction tasks in storage systems, according to the techniques adopted. In Section~\ref{sec:linnos}-\ref{sec:smarter}, we then present three representative systems/field studies that aim to achieve predictability in storage systems from different aspects. Finally, Section~\ref{sec:discussion} reviews all the mechanisms and field studies and discusses challenges.
    
    
    \ifcsdef{mainfile}{}{\bibliography{../references/main}}
\end{document}