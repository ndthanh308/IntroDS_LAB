%%%%%%%% ICML 2023 EXAMPLE LATEX SUBMISSION FILE %%%%%%%%%%%%%%%%%

\documentclass{article}

% % Recommended, but optional, packages for figures and better typesetting:
% \usepackage{microtype}
% \usepackage{graphicx}
% \usepackage{subfigure}
% \usepackage{booktabs} % for professional tables

% hyperref makes hyperlinks in the resulting PDF.
% If your build breaks (sometimes temporarily if a hyperlink spans a page)
% please comment out the following usepackage line and replace
% \usepackage{icml2023} with \usepackage[nohyperref]{icml2023} above.
% \usepackage{hyperref}


% Attempt to make hyperref and algorithmic work together better:
% \newcommand{\theHalgorithm}{\arabic{algorithm}}

% Use the following line for the initial blind version submitted for review:
% \usepackage{icml2023}

% If accepted, instead use the following line for the camera-ready submission:
\usepackage[accepted]{icml2023}

% if you use cleveref..
% \usepackage[capitalize,noabbrev]{cleveref}
\usepackage[fleqn]{nccmath}

\usepackage[utf8]{inputenc} % allow utf-8 input
\usepackage[T1]{fontenc}    % use 8-bit T1 fonts
\usepackage{hyperref}       % hyperlinks
\usepackage{url}            % simple URL typesetting
\usepackage{booktabs}       % professional-quality tables
\usepackage{amsfonts}       % blackboard math symbols
\usepackage{nicefrac}       % compact symbols for 1/2, etc.
\usepackage{microtype}      % microtypography
\usepackage{xcolor}         % colors
\usepackage{subcaption}
\usepackage{tabularray}
\usepackage{adjustbox}
\usepackage{hyperref}
\usepackage{url}
\usepackage{xspace}
\usepackage{algorithm2e}
\RestyleAlgo{ruled}
\usepackage{amsmath}
\usepackage{amsthm}
\usepackage{wrapfig}
\usepackage{authblk}
\usepackage{float}

% \usepackage{minipage}

% Self-defined macros
\usepackage{colortbl}
\newcommand{\method}{\textsc{Counterpol}\xspace}
% \newcommand{\method}{\textsc{Crepe}\xspace}
\newcommand{\regret}{\textsc{Regret}}
\newcommand{\rtarget}{R_{\text{target}}}
\newcommand{\pithetacf}{\pi_{\theta_\text{cf}}}
\newcommand{\Jpitheta}{J_{\pi_{\theta}}}
\newcommand{\Dkl}{D_{\text{kl}}}
\newcommand{\mc}[2]{\multicolumn{#1}{c}{#2}}
\newcommand{\swap}[3][-]{#3#1#2} 
\newcommand{\xhdr}[1]{\vspace{0em}\noindent{{\bf #1.}}}
\newcommand{\na}{~~~~~~~N/A~~~~~~~~}
\newcommand{\ie}{\textit{i.e., }}
\newcommand{\eg}{\textit{e.g., }}
\newcommand{\std}[1]{\scriptsize{$\pm$#1}}
\newcommand{\deriv}[2]{\frac{\partial{#1}}{\partial{#2}}}
\newcommand{\chirag}[1]{{\color{red}[[chirag: #1]]}}
\newcommand{\shripad}[1]{{\color{blue}[[shripad: #1]]}}
\newcommand{\indep}{\perp \!\!\! \perp}
\newcommand{\enc}{\textsc{Enc}\xspace}
\newenvironment{graytext}{\color{gray}}{\ignorespacesafterend}
\newcolumntype{a}{>{\columncolor{Gray}}c}
\newcolumntype{b}{>{\columncolor{white}}c}
% \newcommand{\std}[1]{\scriptsize{$\pm$#1}}
\newcommand\ccg[1]{\cellcolor{gray!35}{#1}} % for cells in second column % and gray colored rows
\newcommand\ccr[1]{\cellcolor{red!25}{#1}} % for cells in second column % and gray colored rows
\newcommand\ccb[1]{\cellcolor{blue!25}{#1}} % for cells in second column % and gray colored rows
\newcommand\ccy[1]{\cellcolor{yellow!35}{#1}} % for cells in second column % and gray colored rows
% \newcommand\ccw[1]{\cellcolor{red!15}{#1}}         % for cells in  second column % white colored rows
\usepackage{booktabs} % To thicken table lines
\usepackage{color, colortbl}
\definecolor{LightCyan}{rgb}{0.88,1,1}

% Conditions allows to add descriptions in the equation
\newenvironment{conditions}
  {\par\vspace{\abovedisplayskip}\noindent\begin{tabular}{>{$}l<{$} @{${}={}$} l}}
  {\end{tabular}\par\vspace{\belowdisplayskip}}
\newcommand{\norm}[1]{\left\lVert#1\right\rVert}

% Adding theorems
\newtheorem{theorem}{Theorem}[section]
\newtheorem{lemma}[theorem]{Lemma}
\newtheorem{proposition}[theorem]{Proposition}
\newtheorem{corollary}[theorem]{Corollary}
\DeclareMathOperator*{\argmax}{arg\,max}
\DeclareMathOperator*{\argmin}{arg\,min}

% \newenvironment{proof}[1][Proof]{\begin{trivlist}\item[\hskip \labelsep {\bfseries #1}]}{\end{trivlist}}
\newenvironment{definition}[1][Definition]{\begin{trivlist}\item[\hskip \labelsep {\bfseries #1}]}{\end{trivlist}}
\newenvironment{example}[1][Example]{\begin{trivlist}\item[\hskip \labelsep {\bfseries #1}]}{\end{trivlist}}
\newenvironment{remark}[1][Remark]{\begin{trivlist}\item[\hskip \labelsep {\bfseries #1}]}{\end{trivlist}}

% Todonotes is useful during development; simply uncomment the next line
%    and comment out the line below the next line to turn off comments
%\usepackage[disable,textsize=tiny]{todonotes}
\usepackage[textsize=tiny]{todonotes}


% The \icmltitle you define below is probably too long as a header.
% Therefore, a short form for the running title is supplied here:
\icmltitlerunning{Counterfactual Explanation Policies in RL}

\begin{document}

\twocolumn[
\icmltitle{Counterfactual Explanation Policies in RL}

% It is OKAY to include author information, even for blind
% submissions: the style file will automatically remove it for you
% unless you've provided the [accepted] option to the icml2023
% package.

% List of affiliations: The first argument should be a (short)
% identifier you will use later to specify author affiliations
% Academic affiliations should list Department, University, City, Region, Country
% Industry affiliations should list Company, City, Region, Country

% You can specify symbols, otherwise they are numbered in order.
% Ideally, you should not use this facility. Affiliations will be numbered
% in order of appearance and this is the preferred way.
\icmlsetsymbol{equal}{*}

% \affil[1]{Media and Data Science Research, Adobe, Noida, UP, India - 201304}
% \affil[2]{Indian Institute of Technology, Madras, Chennai, TN, India - 600036}
% \affil[3]{Manipal Institute Of Technology, Manipal, KA, India - 576104}
% \affil[4]{Harvard University, Cambridge, MA, United States - 02138}

\begin{icmlauthorlist}
\icmlauthor{Shripad V. Deshmukh}{aa}
\icmlauthor{Srivatsan R.}{aa,bb}
\icmlauthor{Supriti Vijay}{aa,cc}
\icmlauthor{Jayakumar Subramanian}{aa}
\icmlauthor{Chirag Agarwal}{dd}
% \icmlauthor{Firstname6 Lastname6}{sch,yyy,comp}
% \icmlauthor{Firstname7 Lastname7}{comp}
% %\icmlauthor{}{sch}
% \icmlauthor{Firstname8 Lastname8}{sch}
% \icmlauthor{Firstname8 Lastname8}{yyy,comp}
%\icmlauthor{}{sch}
%\icmlauthor{}{sch}
\end{icmlauthorlist}

\icmlaffiliation{aa}{Media and Data Science Research, Adobe}
\icmlaffiliation{bb}{Indian Institute of Technology, Madras}
\icmlaffiliation{cc}{Manipal Institute Of Technology}
\icmlaffiliation{dd}{Harvard University. Work done while at Adobe}

\icmlcorrespondingauthor{Chirag Agarwal}{chiragagarwall12@gmail.com}
% \icmlcorrespondingauthor{Firstname2 Lastname2}{first2.last2@www.uk}

% You may provide any keywords that you
% find helpful for describing your paper; these are used to populate
% the "keywords" metadata in the PDF but will not be shown in the document
\icmlkeywords{Machine Learning, ICML}

\vskip 0.3in
]

% this must go after the closing bracket ] following \twocolumn[ ...

% This command actually creates the footnote in the first column
% listing the affiliations and the copyright notice.
% The command takes one argument, which is text to display at the start of the footnote.
% The \icmlEqualContribution command is standard text for equal contribution.
% Remove it (just {}) if you do not need this facility.

\printAffiliationsAndNotice{}  % leave blank if no need to mention equal contribution
% \printAffiliationsAndNotice{\icmlEqualContribution} % otherwise use the standard text.

\begin{abstract}
    %v3
    As Reinforcement Learning (RL) agents are increasingly employed in diverse decision-making problems using reward preferences, it becomes important to ensure that policies learned by these frameworks in mapping observations to a probability distribution of the possible actions are explainable. However, there is little to no work in the systematic understanding of these complex policies in a contrastive manner, \ie what minimal changes to the policy would improve/worsen its performance to a desired level. In this work, we present \method, the first framework to analyze RL policies using counterfactual explanations in the form of minimal changes to the policy that lead to the desired outcome. We do so by incorporating counterfactuals in supervised learning in RL with the target outcome regulated using desired return. We establish a theoretical connection between \method and widely used trust region-based policy optimization methods in RL. Extensive empirical analysis shows the efficacy of \method in generating explanations for (un)learning skills while keeping close to the original policy. Our results on five different RL environments with diverse state and action spaces demonstrate the utility of counterfactual explanations, paving the way for new frontiers in designing and developing counterfactual policies.   
\end{abstract}
\section{Introduction}
\label{sec:intro}
Reinforcement learning (RL) has been used successfully to train autonomous agents capable of achieving better than human-level performance in simulated environments like Go~\citep{go_rl} and a suite of Atari games~\citep{atari_rl}. They are increasingly finding new applications across computational analysis~\citep{matrix},  marketing~\citep{marketing},  education~\citep{tutor}, and biomedical research~\citep{protein} and. Recent breakthroughs in large-language models (LLMs)~\citep{gpt4} are primarily attributed to key RL components that have improved the generative capability of state-of-the-art LLMs. With RL frameworks being deployed at scale as well as performing autonomously, it becomes imperative to incorporate explainability in them, resulting in increased user trust in autonomous decision-making. Explaining the decisions of black-box RL agents for a given environment state is non-trivial as it not only involves explaining the final agent action but also includes complex decision-making and planning behind the output action.

A myriad of RL explainability methods with various attribution techniques has recently been proposed~\citep{greydanus2018visualizing,deshmukhexplaining,iyer2018transparency,puri2019explain}. In particular, they focus on identifying input states and past experiences (trajectories) that led the RL agent to learn complex behaviors. While these methods output important input state features (agent's observation) and trajectories, they fail to explain the minimal change in the trained policy leading to a desired outcome or (un)learning of a specific skill. Intuitively, this requires generating \textit{counterfactuals} for a given desired outcome (\ie identifying \textit{what} and \textit{how much} to change a given RL policy to obtain a target return for its current state). While some previous works have explored causal reinforcement learning~\citep{causal_rl}, there is little to no research on systematically explaining the mechanism of the complex policies learned by a given RL agent using counterfactual explanations.

\xhdr{Present Work} We propose \method, a framework for counterfactual analysis of RL policies. In our framework, we generate explanations by asking the question: ``What least change to the current policy would improve or worsen it to a new policy with a specified target return?'' To estimate such counterfactual policies, we present an objective that aims to obtain a new policy with an average performance equal to that of a specified return while limiting its modifications with respect to the given policy. The generated policies provide direct insights into how a policy can be modified to achieve better results as well as what to avoid in order not to deteriorate the performance. Further, we theoretically prove the connection between popular trust region-based optimization methods in RL with \method, bringing a new perspective of looking at RL optimization using a prominent explainability tool. Formally, the \method learns minimal changes in the current policy without changing its general behavior. To optimize counterfactual explanation policies, we specify a novel objective function that can be solved using basic on-policy Monte Carlo policy gradients. In our experiments across diverse RL environments, we show how our algorithm reliably achieves counterfactual for any the set target return for a given policy.

\xhdr{Our Contributions} We present our contributions as follows: 1) We formalize the problem of counterfactual explanation policy for explaining RL policies. 2) We propose \method, an explanatory framework for generating contrastive explanations for RL policies that identify \textit{minimal} changes in the current policy, which would lead to improving/worsening the performance of a given policy. 3) We derive a theoretical equivalence between the \method objective with the widely used trust region-based policy gradient methods. 4) We demonstrate the flexibility of \method through empirical evaluations of explanations generated for five OpenAI gym environments. Qualitative and quantitative results show that \method successfully generates a counterfactual policy for (un)learning skills while keeping close to the original policy.
\section{Related Work}
\label{sec:related_work}
This work lies at the intersection of counterfactual explanations, explainability in RL, and proximal gradients. Next, we discuss the related work for each of these topics.

% \looseness=-1
\xhdr{Counterfactual Explanations} Several techniques have been recently proposed to generate counterfactual explanations for providing recourses to individuals negatively impacted by complex black-box model decisions~\citep{wachter2017counterfactual,Ustun2019ActionableRI,van2019interpretable,mahajan2019preserving,karimi2019model}. In particular, these techniques aim to provide model explanations in the form of minimal changes to an input instance that changes the original model. Broadly, these methods are categorized based on access to the predictive model (white- vs. black-box), enforce sparsity (more interpretable explanations), and whether the counterfactuals belong to the original data manifold~\cite{verma2020counterfactual}. To this end, ~\citet{wachter2017counterfactual} is one of the most widely used methods that use a gradient-based method to obtain counterfactual explanations for models using a distance-based penalty. In this work, we extend the counterfactual explanations to RL policies. 

\xhdr{Explainability in RL} Explainable RL (XRL)~\citep{puiutta2020explainable} methods are a sub-field of explainable artificial intelligence (XAI) that aims to interpret and explain the complex behaviour of RL agents. Recent works in XRL are broadly categorized into i) gradient-based methods~\citep{greydanus2018visualizing}, which analyzed Atari RL agents that use raw visual input to make their decision and identify salient information in the image that the policy uses to select an action, ii) attribution-based methods~\citep{deshmukhexplaining, puri2019explain, iyer2018transparency}, which explain an agent’s behaviour by attributing its actions to global or local observation data, and iii) surrogate-based methods, which distill complex RL policy into simpler models such as decision trees~\citep{coppens2019distilling} or to human understandable decision language~\citep{verma2018programmatically}. While all existing XRL methods generate explanations using state features or trajectories, none explore counterfactual explanations. To the best of our knowledge, for the first time, we explore counterfactual explanation policies for contrastive analysis of policies in terms of what minimal change to the policy would get us to desired output returns.

% \looseness=-1
\xhdr{Proximal Gradient Methods in Optimization} General idea behind proximal gradient methods is to solve an optimization problem $\min_x f(x)$ in the neighborhood of certain $x_{k-1}$ at $k^{th}$ optimization iteration by modifying the objective as $\min_x f(x) + \lambda \cdot \norm{x - x_{k-1}}_p$~\citep{parikh2014proximal,NIPS2014_d554f7bb}. This objective modification is termed as proximal operator. In machine learning, these operators find great significance in regularization, constraint-based optimization, convex structuring of objectives, etc. They have several applications in a multitude of fields ranging from risk-aware forecasting systems, and mathematical reasoning to reinforcement learning~\citep{9341559,asadi2023faster, pmlr-v162-ding22b, hirano2022policy, pmlr-v100-khan20a, lu2023dynamic}. One major advantage of proximal gradient methods is in their ability to handle noisy and incomplete data, which is common in real-world applications. Subsequently, in RL, they have been referred by state-of-the-art algorithms like PPO~\citep{ppo} to develop ways to achieve monotonic gains. In this light and in contrast, we incorporate a proximal operator to ensure counterfactual RL policies are closer to the original input policy. % Related work 
\section{Preliminaries}
\label{sec:prelims}

\xhdr{Supervised Learning} Consider a standard classification setup with a given set of training examples, 
\[\mathbb{X} = \{(\mathbf{x}^{(i)}, y^{(i)})~|~\mathbf{x}^{(i)} \in \mathbb{R}^d, y^{(i)} \in C\},\] % i \in \{1, 2, \dots, N\}, C = \{1, 2, \dots, K\}\},\]
where $i \in \{1, 2, \dots, N\}$, $C = \{1, 2, \dots, K\}\}$, $\mathbf{x}^{(i)}$ is a $d$-dimension vector, $N$ is the number of training examples, and $C$ denotes set of $K$ classes. Let a classifier $f$ be trained on $\mathbb{X}$ to predict the correct label for any unseen input $\textbf{x}$.

\xhdr{Counterfactual Explanations}
XAI literature~\citep{cfxai_survey,cfml_survey} defines counterfactual explanations as a technique to analyze  ``what if'' scenarios for model predictions, \eg a counterfactual explanation for a loan application model rejecting an individual's loan application could be ``if you increased your income by \$500, your application would have been accepted''.  Note that in the present work we work in a non-causal setup, where following the previous works~\citep{wachter2017counterfactual}, we define a counterfactual explanation for a given model prediction of input $\mathbf{x}$ as the \textit{minimal variation in the features of $\mathbf{x}$} that changes the prediction of the underlying model from $y$ to $y_{\text{target}}$. 

Formally, in for the aforementioned supervised learning setup,  \textit{counterfactual explanation} of model prediction $\hat{y}_0 = f(\mathbf{x}_0)$ conditioned on a target class $y_{\text{target}} \in C$ is given by,
\begin{equation}
    \mathbf{x}_{\text{cf}} = \argmin_{\mathbf{x}}[ \text{CE}(f(\mathbf{x}),~y_{\text{target}}) + k\cdot \norm{\mathbf{x}_{\text{0}} - \mathbf{x}}_2^2 ],
    \label{eqn:cf_sl}
\end{equation}
% \looseness=-1
where cross-entropy (CE) loss guides explanation for achieving $y_{\text{target}}$ by modifiying $\mathbf{x}_0$ to $\mathbf{x}_{\text{cf}}$ and the mean-squared distance  between $\mathbf{x}_{\text{cf}}$ and $\mathbf{x}_0$ ensures the proximity between the two, regulated via coefficient $k$.

\xhdr{Reinforcement Learning} Consider a finite horizon Markov Decision Process (MDP)~\citep{mdp} defined as $\mathcal{M} = (\mathcal{S}, \mathcal{A}, P, R, d_0, \gamma)$, where $\mathcal{S}$ denotes the state-space,  $\mathcal{A}$ denotes action-space, $P: (\mathcal{S} \times \mathcal{A} \times \mathcal{S}) \xrightarrow{} [0, 1]$ denotes state transition function, $R: (\mathcal{S} \times \mathcal{A} \times \mathcal{S}) \xrightarrow{} \mathbb{R}$ denotes the reward function, $d_0: \mathcal{S} \xrightarrow[]{} [0,1]$ represents distribution over starting states, and $\gamma \in (0, 1]$ denotes the discount factor. Let $\pi: \mathcal{S} \times \mathcal{A} \xrightarrow[]{} [0, 1]$ denote the learnt agent policy. Then, we measure the performance $J_{\pi}$ of the policy in terms of expected return as follows:
\begin{equation}
    J_{\pi} = \mathbb{E}_{(s_0, a_0, s_1, a_1, \dots, s_T)}\Big[\sum_{t=0}^{T-1}{\gamma^{t}r(s_t, a_t, s_{t+1})}\Big],
    \label{eqn:j_pi}
\end{equation}
where $s_0 \sim d_0$, $a_t \sim \pi(a_t | s_t)$, and $T$ denotes the episode terminating time-step.
% !TeX root = main. tex
\begin{algorithm}[bt]
    \caption{\method: Locally optimal weight distribution for densest subgraph detection}
    \label{alg:lowd}
    \KwIn{Undirected graph $\graph$, iteration count $T$.}
    \KwOut{An approximately densest subgraph of $\graph$.}
        % $\optset \leftarrow \nodes$
        \For{$e = (u, v) \in \edges$}{
            $f_e(u) = f_e(v) = \frac{w_e}{2}$;
            \hfill \linecomment{initialize edge weight}
        }
        \For{$u \in \nodes$}{
            $\ell_u = \sum_{e\in \edges:u\in e}{f_e(u)}$
            \hfill \linecomment{initialize node load}
        }
        \For{$k : 1 \to T$}{
            \For{$e = (u, v) \in \edges$}{ 
                % \linecomment{balance weight of $u$ and $v$ with $w_e$}
                \If{$ \ell_{u} > \ell_{v}$}{
                    \hfill \linecomment{balance nodes loads as much as possible}
                    
                    $d \leftarrow \min \{(\ell_{u} - \ell_{v})/2, \, f_{e}(u) \} $;

                    $\ell_{u} \leftarrow \ell_{u} - d$;
                    
                    $ f_e(u) \leftarrow f_e(u) - d$;  
                    
                    $\ell_{v} \leftarrow \ell_{v} + d$;
                    
                    $f_e(v) \leftarrow f_e(v) + d$;
                } 
                \Else {
                    $d \leftarrow \min \{(\ell_{v} - \ell_{u})/2, \, f_{e}(v) \} $;
                    
                    $\ell_{u} \leftarrow \ell_{u} + d$;
                    
                    $f_e(u) \leftarrow f_e(u)+d$;
                     
                    $\ell_{v} \leftarrow \ell_{v} - d$;
                    
                    $f_e(v) \leftarrow f_e(v)-d$;
                }
            }
        }
        $\optset, \subnode \leftarrow \nodes, \nodes $; 
        \hfill \linecomment{sort $l_u$ in a non-decreasing order}
        
        $V_{s} \leftarrow \argsort_{u \in \nodes} l_u$;
        
        \For{$i :  1 \to N$}{
            $\subnode \leftarrow \subnode \setminus \{\nodes_{s}(i)\}$; \hfill \linecomment{$\nodes_{s}(i):$ the $i$-th element of $\nodes_{s}$}
            
            \If{$\rho(\optset) < \rho(\subnode)$}{
                $\optset \leftarrow \subnode$;
            }   
        }
    \Return{$ \graph(\optset) $.}
\end{algorithm}

In this section, through the lens of the primal-dual formulation of linear programming, 
we illustrate our idea for solving DSP
and propose the \textbf{l}ocally \textbf{o}ptimal \textbf{w}eight \textbf{d}istribution algorithm, \emph{\method}, 
which is a fast iterative approach to searching the densest subgraph according to the LP dual of DSP.
We also theoretically prove that \emph{\method} makes the graph converge into locally-dense decomposition.
To search the densest subgraph more efficiently, we propose a pruning algorithm, rendering 
the necessary condition for DSP and corresponding to a subprocess of Greedy. 

\subsection{Locally Optimal Weight Distribution Algorithm}

Firstly we present the pseudo code of \emph{\method} in algorithm \ref{alg:lowd} and explain it. In algorithm \ref{alg:lowd}, given an undirected graph $\graph$ and iteration count T, we distribute each edge $e$'s weight to its two endpoints $u$ and $v$ and use vector $\bm{f}$ to describe it. Specifically, $f_e(u)$ is the weight distributed to node $u$ from edge $e$. Besides, we use vector $\bm{\ell}$ to remark the load of each node received from all corresponding edges, specifically, $\ell_u=\sum_{e\in \edges:u\in e}{f_e(u)}$. In lines 1-4, we distribute each edge equally to two endpoints as the initial state and
accordingly calculate node loads. Its time complexity is $O(M)$. Lines 5-14 mean that it redistributes the weight of edge $e=(u,v)$ to minimize the difference of $\ell_u$ and $\ell_v$ for every $e\in \edges$ in iterations. Its time complexity is $O(MT)$. In lines 7-10 we distribute more edge weight from $u$ to $v$, and update step $d$ is $(\ell_{u}-\ell_{v})/2$ if it will not make any $f_{e}(u)<0$, otherwise $d$ is $f_{e}(u)$. In lines 11-14 the circumstance is the opposite. In line 15 we set the whole nodeset as the initial nodeset $S$. In lines 16-20 we sort S according to node loads and delete the node with the lowest weight one by one to get a subgraph with high density, its time complexity is $O(M+NlogN)$.

\begin{complexity}
    The time complexity of \emph{\method} is $O(MT+M+NlogN)$.
\end{complexity}

\textbf{Remark.} Although \emph{\method} can detect the densest subgraph with enough iterations, but we cannot determine iteration count required. Therefore, our \emph{\method} belongs to the approximation algorithm in strict terms because actually we just set the iteration count to get an approximate solution for DSP. The same circumstance happens on other approximation algorithms including Frank-Wolfe in \cite{danisch2017large}, Greedy++ in \cite{boob2020flowless} and FISTA in \cite{harb2022faster}.

% After defining the above problems, in section $IV.A$ we introduce its primal-dual form of linear programming and explain it, then we develop a fast iterative algorithm called LOWD to search the densest subgraph according to the LP dual of DSP. In section $IV.B$ we analyse the reason why it also deals with the locally-dense decomposition problem. At last, we introduce a pruning technique that can prune the graph and lock the densest subgraph into a much smaller subgraph, we use modified Counting Sort to speed it up and prove it is a subprocess of Greedy in \cite{charikar2000greedy}.

\subsection{\emph{\method} converges to DSP solution}

In order to figure out how \emph{\method} searches the densest subgraph iteratively, we first introduce the LP primal-dual of DSP from \cite{charikar2000greedy,boob2020flowless,danisch2017large} as follows. The notation of LP primal-dual is the same as \cite{boob2020flowless}.
%\cite{boob2020flowless} gives an unweighted version but LP \eqref{eq:primal} is a weighted version, the weighted version of DSP can also be attained in \cite{danisch2017large}.



\begin{equation}
    \centering
    \label{eq:primal}
    \begin{aligned}
        \textrm{maximize}   \qquad & \sum_{e\in \edges}{w_{e}y_{e}} \\
        \textrm{subject to} \qquad & y_e \le x_u, \qquad \forall e = uv \in \edges \\
                            & y_e \le x_v, \qquad \forall e = uv \in \edges \\
                            & \sum_{v \in \nodes}{x_v} \le 1 \\
                            & y_e \ge 0 \qquad \forall e \in \edges \\
                            & x_v \ge 0 \qquad \forall v \in \nodes \\
    \end{aligned}
\end{equation}

In LP \eqref{eq:primal}, the binary $x_u$ and $y_e$ indicate the contribution to density of the densest subgraph from node $u$ and edge $e$. The maximum of LP \eqref{eq:primal} is $\rho^*$, i.e., the maximum density in DSP. You can set $y_e=\frac{1}{\optset}$ if $e \in \edges(\optset)$ otherwise $y_e=0$ and set $x_u=\frac{1}{\optset}$ if $u \in \optset$ otherwise $x_u=0$, then you will get $\frac{\weights(\optset)}{\optset}$ as the optimal value of $\sum_{e\in \edges}{w_{e}y_{e}}$. Instead of using the primal problem, we resort to the LP dual for DSP to illustrate our motivation:

\begin{equation}
    \centering
    \begin{aligned}
        \textrm{minimize}  \qquad &  D \\
        \textrm{subject to} \qquad & f_e(u)+f_e(v)\ge w_{e}\qquad\forall e=uv\in \edges \\
                        & \ell_v \overset{\text{def}}{=} \sum_{e \ni v}{f_e(v)} \le D \qquad \forall v \in \nodes\\
                        & f_e(u) \ge0\qquad\forall e = uv \in E \\
                        & f_e(v) \ge0\qquad\forall e = uv \in E \\
    \end{aligned}
    \label{eq:dual}
\end{equation}

From strong duality, we know its optimal value is also $\rho^*$. The symbols in LP \eqref{eq:dual} can be interpreted in accordance with the description in \emph{\method}. $f_e(u)$ and $f_e(v)$ should be both positive and the sum is not less than $w_e$ to ensure the edge is distributed thoroughly. Actually, in the LP dual of DSP, we can keep $f_e(u)+f_e(v)=w_{e}$ instead of $f_e(u)+f_e(v)\ge w_{e}$, because the former can keep constraint condition and doesn't increase optimization objective $D$. Consequently, we can get the optimization objective $D$ with tighter constraint in LP \eqref{eq:tighter_dual} as below.

\textbf{note:} During the whole process we set $D =\max_{v\in \nodes} \ell_v$ to minimize it as much as possible.

\begin{equation}
    \centering
    \label{eq:tighter_dual}
    \begin{aligned}
        \textrm{minimize}   \qquad & D \\
        \textrm{subject to} \qquad & f_e(u) + f_e(v) = w_e \qquad \forall e = uv \in \edges \\
                        & \ell_v = \sum_{e\ni v}{f_e(v)} \le D \qquad \forall v \in \nodes\\
                        & f_e(u) \ge 0 \qquad \forall e = uv \in \edges \\
                        & f_e(v) \ge 0 \qquad \forall e = uv \in \edges \\
    \end{aligned}
\end{equation}

Intuitively speaking, what \emph{\method} does is to redistribute each edge weight to minimize the difference between loads of two endpoints for each edge in iterations, i.e., propagate edge weights from nodes with higher loads to nodes with lower loads. And the impact of \emph{\method} on LP \eqref{eq:tighter_dual} is that the node $v$ with the highest node load $l_{v}=D$ will decrease its weight because $v$ transits its edge weights to its neighbors. After experiencing some iterations, the highest node load will decrease gradually to some value, in fact, the value is the minimum of $D$, i.e., $\rho^*$. Formally we propose the following theorem:

\begin{theorem}
    The iterative operation in \emph{\method} can make the sequence $\left \{ D_t \right \}$ converge to $\rho^*$, where $D_t (t>0)$  means the optimization objective $D$ after $t$ iterations of \emph{\method} and $D_{0}$ is the value in the initial state. 
    \label{th:dsp}
\end{theorem}

%We give an intuitive explanation for LOWD and claim that it can minimize D and detect the densest subgraph. The impact of LOWD on LP \eqref{eq:tighter_dual} is that the node $v$ with the highest node load $l_{v}=D$ will decrease its weight because $v$ transits its edge weights to its neighbors. After experiencing some iterations, the highest node load will decrease gradually to the minimum of $D$, i.e., $\rho^{*}$. And then we can search the densest subgraph like Greedy, i.e., delete the node with the lowest degree from the left nodeset $\subnode$ one by one.

\begin{comment}
Next, we will give theoretical proofs for our claim, and we raise the following questions:

\begin{enumerate}[label={\arabic*.}]
    \item Why does the optimization objective $D$ converge through LOWD?
    \item Why will the optimization objective $D$ converge to the minimum, which is $\rho^*$, through LOWD?
    \item How can we get the densest subgraph through LOWD?
\end{enumerate}

To answer the first question, we have the following classical theorem.
\end{comment}

Next, we will give theoretical proofs for Theorem \ref{th:dsp}, first we need the following classical theorem for convergence.


\begin{theorem}[Monotone Convergence Theorem~\cite{bibby1974axiomatisations}]
If the sequence $\left \{ a_n \right \} $ has an upper bound and it is monotonically non-decreasing (or has a lower bound and it is monotonically non-increasing), then the sequence $\left \{ a_n \right \} $ converges, i.e., a monotonically bounded sequence must have a limit.
\label{th:tmct}
\end{theorem}

Now that $\left\{D_t \right \}$ has a lower bound $\rho^{*}$ according to strong duality, and $\left \{ D_t \right \} $ is monotonically non-increasing under \emph{\method}'s iterative operation (because in the whole process \emph{\method} just balances loads of two endpoints and it will not produce any new node with load higher than the optimization objective $D$). Then $D_t$ is non-increasing and it must converge to some value.

It is helpless for solving DSP if the optimization objective converges to some value which is not the minimum, i.e., $\rho^{*}$. However, we can use the following lemma to help to prove Theorem \ref{th:dsp}.

\begin{lemma}
    for $\forall t \in \mathbb{N}$, there must be $D_{t+N}<D_{t}$ if $D_t \ne \rho^{*}$, where N is the node number of the whole graph.
    \label{lem:decrease}
\end{lemma}

\begin{proof}
    For $\forall t \in \mathbb{N}$, if $D_t>\rho^{*}$, we make an assertion that there must be some nodes $u$ and $v$ and an edge $e=(u,v)$ with $\ell_u=D_t$, $\ell_v<D_t$ and $f_e(u)>0$. Suppose this doesn't stand up, so if we set $A=\left \{u|l_u=D_t \right \}$ and $B=\left \{u|l_u<D_t \right \}$, for $\, \forall u \in A,v \in B$ and $e=(u,v)$, then $f_e(u)=0$. Therefore we conclude that:
    \begin{equation*}
        \footnotesize
        \centering
        \begin{aligned}
            \rho \left ( A \right ) 
            &=\frac{\sum_{e \in \edges(A)} w_e}{|A|}=\frac{\sum_{e=(u,v),\,u,v\in A} f_e(u)+f_e(v)}{|A|}\\
            &=\frac{\sum_{e=(u,v),\,u,v\in A} {(f_e(u)+f_e(v))}+\sum_{e=(u,v),\,u\in A,v\in B}{f_e(u)}}{|A|}\\
            &=\frac{\sum_{u\in A}{\sum_{e\ni u}{f_e(u)}}}{|A|}\\
            &=\frac{\sum_{u\in A}{l_u}}{|A|}=\frac{\sum_{u\in A}{D_t}}{|A|}=D_t>\rho^*\\
        \end{aligned}
    \end{equation*}

    It will produce a subgraph whose density is larger than the densest subgraph. That will lead to a contradiction. 

    Therefore in each iteration, node $u$ will transit edge weight to node $v$ and then there will be $l_u<D_t$ according to our assertion. Notice that in the whole process, \emph{\method} will not produce any new node $v$ with $l_v\ge D_t$ after t iterations. The number of nodes with load $D_t$ will decrease in each iteration until 0. Given that $N$ is the node number of the whole graph, after $N$ iterations, there isn't any node $v$ with $l_v=D_t$, i.e., $D_{t+N}<D_{t}$.
\end{proof}

Combining Lemma \ref{lem:decrease} and Theorem \ref{th:tmct} is not enough to prove our claim because $\left\{D_t \right \}$ may decrease infinitesimally and converge to another value instead of the minimum. However, when dealing with this difficulty, it is useful to combine the proof of Lemma \ref{lem:decrease} in the limit sense.

\begin{proof}[Proof of Theorem \ref{th:dsp}]
    We adopt the proof by contradiction which is similar to the proof of Lemma \ref{lem:decrease}. Suppose that \emph{\method} makes $\left \{ D_t \right \}$ converge to any other value $D$ which $D > \rho^{*}$. We set $A=\left \{u|l_u\to D \right \}$ and $B=\left \{u|l_u\not\to D \right \}$. When $T \to \infty$, $|A|$ will decrease and converge to a fixed number according to Theorem \ref{th:tmct}, and all edges connecting $A$ and $B$ are distributed to $B$ in the limit sense. Then $\rho(A)$ will be tending to $D$, which makes a contradiction because $D > \rho^{*}$. Only when $D = \rho^{*}$, there will be no contradiction.
\end{proof}

The explanation for lines 19-24 in Algorithm \ref{alg:lowd} is closely related to the locally-dense decomposition, which will be proved in the next subsection. In fact, $B_1$ in LDD is the maximal densest subgraph and nodes in $B_1$ will have the max node load $\rho^*$. Therefore in LDD, as long as we delete all the nodes whose weights are not the maximum, the remaining subgraph is the maximal densest subgraph. If \emph{\method} makes the graph converge into LDD, given that the node load in the densest subgraph will be not completely the same after several iterations, it is safe to delete the node with the lowest weight one by one.

A drawback of the iterative algorithm is that we don't know whether we have found the densest subgraph so as to stop iterations. However, on unweighted graphs, if the difference between optimization objective $D$ and the maximum density found by \emph{\method} is less than $\frac{1}{n(n-1)}$, we can confirm that \emph{\method} has found the densest subgraph and it can stop iterations, which is similar with the maximum flow algorithms in \cite{goldberg1984finding}. Given that the edge weights satisfy $\weights \in \numR_{+}$ on weighted graphs, it doesn't work to use the difference $\frac{1}{n(n-1)}$ to determine whether we have found the densest subgraph if it is weighted.

%Take Figure \ref{fig:example} as an example, in the locally-dense decomposition, the nodes in the maximal densest subgraph have the same loads $1.6$, and the subgraph consisting of these nodes has the maximal density $1.6$, while the other nodes have lower loads $1.5$. In the locally-dense decomposition, they form nodesets respectively and edge loads between different nodesets are distributed from nodes in the maximal densest subgraph to nodes with loads $1.5$ in one-way. Although node loads in the densest subgraph may be not completely the same after several iterations, it is safe to delete the node with the lowest load one by one, and then get the maximal densest subgraph, i.e, the subgraph consisting of nodes $1$,$2$,$3$,$4$,$5$. 

Next, we will explan the relationship between \emph{\method} and LDD.

\subsection{\emph{\method} converges to LDD's solution}
\label{subsecion:ldd}
There are many iterative methods dealing with the LP dual of DSP including Frank-Wolfe in \cite{danisch2017large}, Greedy++ in \cite{boob2020flowless} and FISTA in \cite{harb2022faster}. Among them, \cite{harb2022faster,danisch2017large} claim that their methods can converge into the locally-dense decomposition. And \cite{harb2022faster} also claims that Greedy++ will converge into it. As an iterative method, \emph{\method} also does it. Firstly, let's introduce the quadratic program(QP) formula of locally-dense decomposition in \cite{danisch2017large,harb2022faster}.

\begin{equation}
    \label{eq:decomposition}
    \centering
    \begin{aligned}
        \textrm{minimize} \qquad  & \sum_{v \in \nodes}{\ell_v^2} \\
        \textrm{subject to} \qquad & f_e(u) + f_e(v) = 1 \qquad \forall e = uv \in \edges \\
                    & \ell_v = \sum_{e \ni v}{f_e(v)} \le D \qquad \forall v \in \nodes\\
                    & f_e(u) \ge 0 \qquad \forall e = uv \in \edges \\
                    & f_e(v) \ge 0 \qquad \forall e = uv \in \edges \\
    \end{aligned}
\end{equation}

The relationship between QP \eqref{eq:decomposition} and LDD is: When \emph{\method} makes $\sum_{v \in \nodes}{{\ell_v}^2}$ converge to the minimum, then the solution $(\bm{f},\bm{\ell})$ converges to the solution of locally-dense decomposition. 

In algorithm \ref{alg:lowd}, \emph{\method} will decide an update step d and redistribute the weight of edge e to decrease the optimization objective of QP \eqref{eq:decomposition} as much as possible. For example, for $e=(v_1,v_2)$ where $\ell_{v_1}>\ell_{v_2}$ and $f_e(v_1)>0$, we use $\bm{\ell}^{'}$ to represent the node loads after updating, i.e., ${\ell_{v_1}}^{'}\gets \ell_{v_1}-d$, ${\ell_{v_1}}^{'}\gets \ell_{v_1}+d$ and ${\ell_{v}}^{'} \gets \ell_v$ for $v\neq v_1,v_2$. Then:

\begin{center}
    \vspace{-0.15in}
    \begin{equation*}
        \begin{aligned}
            \sum_{v \in \nodes}{{{\ell_{v}}^{'}}^{2}}&=\sum_{v \in \nodes,v\ne v_1,v_2}{{{\ell_{v}}^{'}}^{2}}+{{\ell_{v_1}}^{'}}^{2}+{{\ell_{v_2}}^{'}}^{2}\\
            &=\sum_{v \in \nodes,v\ne v_1,v_2}{{\ell_{v}}^{2}}+{(\ell_{v_1}-d)}^2+{(\ell_{v_2}+d)}^2\\
            &=\sum_{v \in \nodes}{{\ell_{v}}^2}+2d\cdot(d+\ell_{v_2}-\ell_{v_1})<\sum_{v \in \nodes}{{\ell_{v}}^2}\\
        \end{aligned}
        \label{eq:decrease}
    \end{equation*}
\end{center}

In algorithm \ref{alg:lowd}, update step $d=\min \{(\ell_{v_1}-\ell_{v_2})/2, \, f_{e}(v_1) \}$ then $d+\ell_{v_2}-\ell_{v_1}<0$, and in this case $d>0$. The less-than sign in the last line holds true.

The local optimality of \emph{\method} is because it decreases the optimization objective of QP \eqref{eq:decomposition} and LP \eqref{eq:tighter_dual} as much as possible. In the above equation of QP \eqref{eq:decomposition}, $d=(\ell_{v_1}-\ell_{v_2})/2$ can minimize $2d\cdot(d+\ell_{v_2}-\ell_{v_1})$ according to the mean inequality so that $\sum_{v \in \nodes}{{{\ell_{v}}^{'}}^2}$ is the minimum, and there should be $d\ge f_e(v_1)$ to satisfy constraint $f_e(v_1)\ge 0$, then $d=\min \{(\ell_{v_1}-\ell_{v_2})/2, \, f_{e}(v_1) \}$ is a locally optimal operation which satisfies the constraint and decrease $\sum_{v \in \nodes}{{\ell_{v}}^2}$ as much as possible. As for LP \eqref{eq:tighter_dual}, \emph{\method} is also locally optimal to minimize $D$.

\begin{theorem}
    \emph{\method} will optimize QP \eqref{eq:decomposition} until $\sum_{v \in \nodes}{{l_v}^2}$ converge to the minimum.
    \label{th:decomposition}
\end{theorem}

\begin{proof}
    First, any edge redistribution in \emph{\method} will decrease the optimization objective of QP \eqref{eq:decomposition} as we claimed before. According to Theorem \ref{th:tmct}, $\sum_{v \in \nodes}{{\ell_{v}}^2}$ will converge to some value. Suppose it is not the minimum, if \emph{\method} stops changing any edge redistribution, the node loads in $\graph$ must be satisfied with property \ref{prop:one-way}, i.e., if two endpoints have different loads, the edge connecting them should be only distributed in one-way to the node with a lower load. Otherwise, \emph{\method} can continue its iterative operation to decrease $\sum_{v \in \nodes}{{{\ell_{v}}^{'}}^{2}}$. Then, the nodes with the same loads will consist of new nodesets $B_i$, from a global perspective, it will result in a new sequence $\emptyset=B'_0 \subsetneqq B' _1 \subsetneqq B' _2 \subsetneqq ... \subsetneqq B' _k=\nodes$ with $\lambda' _1>\lambda' _2>...>\lambda' _k$. According to property \ref{prop:one-way}, $B' _i=\mathop{\arg\max}\limits_{W \supsetneqq B' _i-1}{\frac{\weights(\edges(W))-\weights(\edges(B' _{i-1}))}{|W\setminus B' _{i-1}|}}$ so it is satisfied with the definition of locally-dense decomposition. Now that we suppose this decomposition doesn't converges to the minimum of QP \eqref{eq:decomposition}, it must be another LDD with a different {$\bm{\ell}^*$}, which contradicts with the property \ref{prop:uniquity1} and \ref{prop:uniquity2}.
    
    If \emph{\method} doesn't stop, it must decrease the optimization objective of QP \eqref{eq:decomposition} infinitesimally. If so, edge weights must be changed infinitesimally when $T \to \infty $, otherwise the optimization objective will not decrease infinitesimally, which makes a contradiction to convergence. Given that \emph{\method} manages to balance loads of two endpoints connected by an edge, node loads in the locally-dense decomposition must obey property \ref{prop:one-way} in the sense of limit because edge weights only can be changed infinitesimally. At last, these node loads will produce a new sequence $\emptyset=B_0 \subsetneqq B _1 \subsetneqq B_2 \subsetneqq ... \subsetneqq B_k=\nodes$ in the limit sense, which makes a contradiction similarly as above.

    Therefore, the optimization objective in QP \eqref{eq:decomposition} will converge to the minimum, and the minimum represents the graph converges into the locally-dense decomposition.
\end{proof}

Therefore, \emph{\method} can make $(\bm{f},\bm{\ell)}$ converge into locally-dense decomposition. LDD is an important basic problem for many variants of DSP. \cite{ma2022finding} use it to detect a variant of DSP called locally densest subgraph. Besides, we provide a perspective on the relationship between LDD and another variant of DSP concerning the densest subgraph with size constraint, called \textit{densest k-subgraph} (DkS) and \textit{at-least-k subgraph} (DalkS) problems. Its proof is provided in the appendix.  

\begin{corollary}
     \label{coroll:dks}
    The following results hold up on DkS and DalkS problems:
    \begin{enumerate}[label={\arabic*.}]
        \item For $k=|B_j|\,,\forall j \in \left \{ 1,2,...,k \right \} $, the DkS (or DalkS) is just the subgraph composed of nodes in $B_j$.
        \item For $|B_{j-1}|<k<|B_j|\,, \forall j \in \left \{ 1,2,...,k \right \}$, the upper bound of density in DkS (or DalkS) is $\frac{\sum_{i=0}^{j-1}{\lambda_{i}*|B_i|}+(k-|B_{j-1}|)*\lambda_{j}}{k}$.
    \end{enumerate}
\end{corollary}

\begin{comment}

The second paragraph in proof of the Theorem  \ref{th:decomposition} also shows an important conclusion, we give it as the following lemma:

\begin{lemma}
    As $\sum_{v \in \nodes}{{l_v}^2}$ converge to the minimum, it has $l_v \to {l_v}^*$ for any node $v$. 
    \label{lem:convergence}
\end{lemma}

\begin{proof}
    As $\sum_{v \in \nodes}{{l_v}^2}$ converges to the minimum, then any edge redistribution will only decrease it infinitesimally. If there are nodes $u$ and $v$, an edge e with $f_e(u)>0$ and $f_e(u)>0$, then there must be $|l_u-l_v| \to 0$ or $|l_u-l_v| = 0$, otherwise $\sum_{v \in \nodes}{{l_v}^2}$ will decrease non-infinitesimally which leads to a contradiction. Then node loads must converge to the optimal value of the locally-dense decomposition, or it will produce another decomposition in the sense of limit and make a contradiction with its uniqueness.
\end{proof}

\end{comment}

\subsection{Pruning pre-process}

In our experiments, \emph{\method} can detect the densest subgraph much faster than other baselines without pruning, but it is more efficient to use a pruning technique to locate the densest subgraph before using any iterative algorithm. And the pruning is also a subprocess of Greedy, which is also useful to speed up other approximation and exact algorithms.

First, we have the following necessity condition about the optimal solution for DSP, i.e., the optimal set $\optset$.

\begin{theorem}[Lower Bound\cite{khuller2009finding}]
    For each node $v \in \optset$ of the densest subgraph, $\setndeg{\optset}{v} \ge \rho(\optset)$.
    \label{th:lowerbound}
\end{theorem}

Therefore, we can conclude that $\setndeg{\mathcal{H}}{v} \ge \setndeg{\optset}{v} \ge \rho(\optset) \ge \rho(\mathcal{H})$ if $\optset \subset \mathcal{H}$, which means $v\not\in \optset$ if $v\in \mathcal{H}$ and $\setndeg{\mathcal{H}}{v}<\rho(\mathcal{H})$. Then we can use the density of a subgraph $\mathcal{H}$ as the lower bound to filter out the candidates of $\optset$ as long as the density of $\mathcal{H}$ can be easily obtained.  The pruning technique first estimates the lower bound based on the density of the remaining subgraph $\mathcal{H}$ at the current time and deletes nodes whose degrees are lower than $\rho(\mathcal{H})$ iteratively and their adjacent edges, then it updates the bound using updated $\rho(\mathcal{H})$. Details are in Algorithm \ref{alg:pruning}.

\begin{algorithm}[t]
    \SetKwFunction{DSPSolver}{DSPSolver}
    \caption{\textsc{Pruning}}
    \label{alg:pruning}
    \KwIn{Undirected graph $\graph$; plug-in \DSPSolver: \{ \textsc{Maxflow}, \textsc{greedy}, \textsc{greedy++}, \method, \textsc{Frank-Wolfe}, \textsc{FISTA}, etc. \}.}
    \KwOut{Solution of the densest subgraph of $\graph$.}

    $\mathcal{H} \leftarrow \graph$;

    $\delta \leftarrow \rho(\mathcal{H})$;

    % \If{$ \exists \mathcal{H^{'}} \subseteq  \mathcal{H}$ and $\setndeg{\mathcal{H}}{u} < \delta$}
    % {
    %     \For
    % }
    
    \While{$\exists \, \mathcal{H^{'}} \subseteq  \mathcal{H}$ with $\setndeg{\mathcal{H}}{u} < \delta \quad \forall u \in \mathcal{H^{'}}$}{
            
            Remove all nodes in $\mathcal{H^{'}}$ and all its associated edges from $\mathcal{H}$;
            
            $\delta \leftarrow \rho(\mathcal{H})$;
    }
    
    $\subnode \leftarrow$ \DSPSolver{$\mathcal{H}$};    
    \hfill \linecomment{plug-in DSP solver}
    
    \Return{$\graph(\subnode)$.}
\end{algorithm}

This pruning technique is similar to pruning 1 in \cite{fang2019efficient}. However, we don't consider k-core explicitly and we modify the data structure Counting Sort to speed it up on unweighted graphs, which assigns it a specific time complexity, i.e., $O(M+N)$. On an unweighted graph, first, we calculate the degree of all nodes and sort them by Counting Sort, i.e., we record nodesets of each degree 0-$d_{max}$, and update the degrees of remaining nodes and corresponding sets after each iteration. In each round of scanning, we only need to scan all the sets in the range of $\left [ \left  \lfloor LowBound_{1}  \right \rfloor, \left \lceil LowBound_{2}-1  \right \rceil  \right ] $ , which are exactly the nodes to be deleted in the next iteration. $LowBound_{1}$ is the bound before updating $\rho(\mathcal{H})$ while $LowBound_{2}$ is the bound after updating $\rho(\mathcal{H})$. Although nodes with degrees less than $LowBound_{1}$ will appear after deleting nodes in each iteration, we can just put them into the set $\left \lfloor LowBound_{1} \right \rfloor $. Throughout the process, we check all the nodesets in the range of $\left [ 0,d_{max}  \right ] $ and only repeat searching $\lfloor LowBound_{1} \rfloor$ at most $T$ times. It is obvious to know $d_{max}$ and $T$ are both lower than $N$. So the time complexity of checking the nodeset of each degree is $O(N)$ on an unweighted graph. Therefore, the time complexity of the pruning on an unweighted graph is $O(M+N)$. 

On a weighted graph, after each iteration, we have to traverse the remaining nodes to find the nodes that can be deleted in the next iteration. Then the time complexity of the pruning on weighted graphs is $O(M+TN)$, where $T$ is the number of iterations until it stops.

Their difference in time complexity is because we use modified Counting Sort to make sure each node is only checked one time on unweighted graphs. 

\begin{complexity}
    For the pruning technique, its time complexity is $O(M+N)$ on unweighted graphs and $O(M+TN)$ on positive weighted graphs.
\end{complexity}

It seems that on weighted graphs it is not efficient because we don't know iteration count $T$ in advance, in fact in our experiments the time consumption of these two versions on unweighted and weighted graphs doesn't differ a lot. It is also an efficient solution that we can set an upper bound for $T$ on weighted graphs.

Next, we claim that pruning is a specific subprocess of Greedy as below. Lemma \ref{lem:k-core} is a crucial property of Greedy and we use it to prove Theorem \ref{th:pruning}. The definition of k-core and their proofs are in the appendix.

\begin{lemma}
     For any k, k-core can be achieved by the Greedy algorithm in \cite{charikar2000greedy}.
    \label{lem:k-core}
\end{lemma}
\begin{theorem}
    In Greedy\cite{charikar2000greedy}, when the density decreases for the first time, the remaining subgraph is the subgraph $\mathcal{H}$ when the pruning in Algorithm \ref{alg:pruning} stops iterations.
    \label{th:pruning}
\end{theorem}

In the above theorem, we can see that although the deletion rule is different in the pruning and Greedy. Pruning will get the same subgraph as the end of the monotonic increase of density in Greedy. Given that the pruning doesn't need to know which node has the lowest degree, it can achieve a faster deletion speed than Greedy.

In fact, the pruning is so efficient that it can delete most nodes and ensure that the densest graph is in the remaining subgraph with a much smaller size. 
\subsection{Connection between Counterfactual Explanation Policies and Trust Region Methods}
\label{sec:trpo_connection}
Trust region-based RL policy optimization methods like TRPO~\citep{trpo}, ACKTR~\citep{acktr} and PPO~\citep{ppo} have been foundational in the unprecedented success of deep RL~\citep{dota2, gpt4}. The primary aspect behind their optimization involves iteratively updating the policy within a \textit{trust} region, leading to a monotonic improvement in the policy's performance. Formally, the objective is defined as:
\begin{equation}
    \label{eqn:tr_1}
    \pi_{\theta_{i+1}} =  \argmax_{\theta}~~ J_{\pi_\theta}\text{~~such that~~} D_{\text{KL}}(\pi_{\theta_i} || \pi_{\theta}) \leq \delta
\end{equation}
The above objective can be written in its Lagrangian form using penalty instead of the constraint as: 
\begin{equation}
    \label{eqn:tr_2}
    \begin{aligned}
    \pi_{\theta_{i+1}} =  {} & \argmax_{\theta}~~ J_{\pi_\theta} - \lambda\cdot D_{\text{KL}}(\pi_{\theta_i} || \pi_{\theta}), \\
    {} & \text{where $\lambda$ is treated as a hyper-parameter.}
    \end{aligned}
\end{equation}

Next, we show the equivalence between counterfactual explanation policy and the policy obtained using a trust region-based policy gradient method.
\begin{theorem}(Equivalence)
\label{thm:equivalence}
Let $\rtarget$ be the maximum possible return in the MDP under consideration. Then, for any $L_{\text{ret}}$ estimated using $\ell_1$-norm, the generated counterfactual explanation policy through iterative KL-pivoting is equivalent to optimizing the policy for best return using a trust region-based policy gradient method.
\end{theorem}
\begin{proof}
For a given $\rtarget$, let us assume that the desired return from a policy is equal to the maximum possible return, \ie $\rtarget = R_{\text{max}}$, and the return penalty calculated using $\ell_1$-norm, we rewrite the counterfactual generation objective using KL-pivoting~ Eqn.~\ref{eqn:cf_iterative_defn} as:
\begin{equation}
\label{eqn:proof_1}
    \pi_{\theta_{\text{cf}}} = \argmin_{\theta} (|J_{\pi_\theta} - R_{\text{max}}| + k \cdot D_{\text{kl}}(\pi_{\theta_i} || \pi_\theta)) 
\end{equation}
We have $J_{\pi_\theta}=\mathbb{E}_{\pi_\theta}[\sum_{t=0}^{T-1}\gamma^t r(s_t, a_t, s_{t+1})| s_0 {\sim} d_0, a_t \sim \pi_\theta(\cdot| s_t))]$ as defined in Eq.~\ref{eqn:j_pi}. Now, let $R{=}\sum_{t=0}^{T-1}\gamma^t r(s_t, \\a_t, s_{t+1})$ denote the discounted return for the episode $(s_0, a_0, s_1, a_1, \dots, s_T)$, then $J_{\pi_\theta}=\mathbb{E}_{\pi_\theta}[R]$. Further,
\[ 
    R \leq R_\text{max} \implies J_{\pi_\theta} = \mathbb{E}_{\pi_\theta}[R] \leq \mathbb{E}_{\pi_\theta}[R_\text{max}] = R_\text{max}
\]
Hence, $J_{\pi_\theta} - R_\text{max} \leq 0$, or $R_\text{max} - J_{\pi_\theta} \geq 0$, which allows us to write $|J_{\pi_\theta} - R_\text{max}|$ simply as $(R_\text{max} - J_{\pi_\theta})$. Rewriting Eqn.~\ref{eqn:proof_1}, we get:
\begin{equation}
\label{eqn:proof_2}
\begin{aligned}
\pi_{\theta_{\text{cf}}}
= {} & \argmin_{\theta}~~ R_\text{max} - J_{\pi_\theta} + k \cdot D_{\text{KL}}(\pi_{\theta_i} || \pi_\theta) \\
= {} & \argmin_{\theta}~~ - J_{\pi_\theta} + k \cdot D_{\text{KL}}(\pi_{\theta_i} || \pi_\theta) \\
= {} & \argmax_{\theta}~~ J_{\pi_\theta} - k \cdot D_{\text{KL}}(\pi_{\theta_i} || \pi_\theta)
\end{aligned}
\end{equation}
% \looseness=-1
Comparing the above equations with Eqn.~\ref{eqn:tr_2} and treating the hyper-parameters $k$ and $\lambda$ interchangeably, we find that both these objectives of policy updation are the same, which completes our proof.
\end{proof}
Consequently, the trust region-based policy gradient methods can be interpreted as finding a \textit{counterfactual explanation policy} of the original policy, which achieves the maximum possible return (\ie the best performance). This also opens up possibility for using sophisticated ways to choose distance regulation parameter $k$ similar to $\lambda$ following ~\cite{cpi}.
\section{Experiments and Results}
\label{sec:expts}
Here, we study the counterfactual explanations of policies trained using policy gradient methods, particularly the actor-critic methods~\citep{sutton_book}, as they converge more faithfully compared to vanilla policy gradients and also maintain the simplicity required to analyze them in a contrastive fashion rigorously.

\xhdr{Setup} We employ the widely used actor-critic algorithm of Advantage Actor-Critic (A2C)~\citep{a2c_1}, and only in the case of complex environments, we shift to Proximal Policy Optimization (PPO)~\citep{ppo} for training the original policies in our empirical analysis. We use the standard implementations provided in stable-baselines library~\footnote{Documentation for the library: \url{https://stable-baselines3.readthedocs.io/en/master/}}~\citep{stable-baselines3} for RL training using the above-mentioned algorithms. We conduct our analysis in five OpenAI gym environments~\citep{openaigym}, \textit{viz.} i) \textit{CartPole-v1}, ii) \textit{Acrobot-v1}, iii) \textit{Pendulum-v1}, iv) \textit{LunarLander-v2} and v) \textit{BipedalWalker-v3}~\footnote{Additional environment details can be found at \url{https://gymnasium.farama.org/environments/}}. 

\xhdr{Implementation details}
We set the threshold return values of $\delta$ for the five environments to $\{10, 2.5, 37.5, 5, 10\}$ according to the order of magnitude of returns in their respective environment,  the number of KL-pivoting policy iterations $m$ as 10 for all environments (except BipedalWalker, where $m=5$), and the number of trajectory rollouts $N$ to 10 for all four environments except BipedalWalker, where $N=2$. Further, we chose the distance regularizer parameter $k$ values as $\{10, 1, 10^5, 10, 1\}$ for the five environments in the same order as discussed above. For computation purposes, we use a single NVIDIA A100 (40GB) GPU. We share the code for generating counterfactual explanation policies using \method in the supplementary material.
\begin{table*}[ht]
\fontsize{7.5pt}{7.5pt}\selectfont
\centering
\setlength{\tabcolsep}{0.5pt}
\renewcommand{\arraystretch}{1.3}
\caption{\textbf{\method Optimization on Control Environments.} Counterfactual explanations for a given policy $\pi_0$ with performance $J_{\pi_0}$ and three target returns. Shown are the number of outer policy updates ($n_{\pi}$) for generating counterfactuals and the resulting performance of the counterfactual policy $\pi_{\theta_{cf}}$. \method faithfully estimates the counterfactuals across all the $\pi_0$-$\rtarget$ pairs and diverse environments.
}
\label{tab:control_env_results}
\begin{tabular}{lccccccccc}
\toprule
\multicolumn{10}{c}{CartPole-v1} \\
\cmidrule{1-10}
$J_{\pi_{\theta_o}}$ & \multicolumn{3}{c}{235.6} & \multicolumn{3}{c}{368.2} & \multicolumn{3}{c}{500.0} \\ 
$R_\text{target}$ & 50 & 250 & 450 & 50 & 250 & 450 & 50 & 250 & 450 \\
$n_{\pi}$ & 278.0\std{22.9} & 10.0\std{6.5} & 303.3\std{67.1} & 108.0\std{16.6} & 1340.0\std{278.8} & 68.7\std{5.0} & 833.3\std{56.3} & 721.3\std{14.7} & 557.3\std{10.0} \\
\rowcolor{gray!30}
$J_{\pi_{\theta_\text{cf}}}$ & 48.6\std{4.9} & 245.0\std{4.4} & 450.6\std{7.5} & 48.2\std{3.4} & 245.5\std{1.9} & 453.5\std{0.5} & 56.4\std{2.4} & 246.0\std{4.7} & 452.1\std{1.0} \\
\midrule
\multicolumn{10}{c}{Acrobot-v1} \\
\cmidrule{1-10}
$J_{\pi_{\theta_o}}$ & \multicolumn{3}{c}{-146.7} & \multicolumn{3}{c}{-89.0} & \multicolumn{3}{c}{-84.3} \\
$R_\text{target}$ & -120 & -100 & -80 & -120 & -100 & -80 & -120 & -100 & -80 \\
$n_{\pi}$ & 44.0\std{18.8} & 31.3\std{1.9} & 18.0\std{12.8} & 62.0\std{34.8} & 38.0\std{37.3} & 6.0\std{2.8} & 117.3\std{95.3} & 26.0\std{36.8} & 8.7\std{5.7} \\
\rowcolor{gray!30}
$J_{\pi_{\theta_\text{cf}}}$ & -119.5\std{1.3} & -100.5\std{2.1} & -80.6\std{1.6} & -119.1\std{0.5} & -99.9\std{0.2} & -80.6\std{0.5} & -120.1\std{1.5} & -99.8\std{1.8} & -81.6\std{0.3} \\
\midrule
\multicolumn{10}{c}{Pendulum-v1} \\
\cmidrule{1-10}
$J_{\pi_{\theta_o}}$ & \multicolumn{3}{c}{-853.5} & \multicolumn{3}{c}{-792.6} & \multicolumn{3}{c}{-568.0} \\
$R_\text{target}$ & -1000 & -750 & -500 & -1000 & -750 & -500 & -1000 & -750 & -500 \\
$n_{\pi}$ & 1.3\std{0.9} & 15.3\std{14.8} & 174.7\std{38.0} & 6.0\std{3.3} & 10.0\std{11.4} & 198.7\std{157.4} & 113.3\std{139.1} & 22.0\std{25.7} & 6.0\std{3.1} \\
\rowcolor{gray!30}
$J_{\pi_{\theta_\text{cf}}}$ & -996.9\std{19.6} & ~-773.5\std{6.4} & ~-507.9\std{19.7} & ~-992.1\std{25.9} & -777.1\std{6.2} & -507.3\std{7.1} & -997.6\std{15.1} & -764.9\std{19.8} & ~-501.0\std{8.8} \\
\bottomrule
\end{tabular}
\end{table*}
% Figure environment removed
\subsection{Results on \method Optimization}
\looseness=-1
We conduct experiments to understand and verify the optimization of our proposed \method framework. In doing so, we present the results of \method optimization on A2C-trained agents of \textit{CartPole-v1}, \textit{Acrobot-v1}, and \textit{Pendulum-v1} environments. We sample three distinct policy checkpoints from the A2C training of each RL environment and generate counterfactual policies using \method for these checkpoints and three different target returns $\rtarget$ (chosen with respect to the RL environment). We train each tuple of the RL environment, A2C checkpoint, and target return for three different seeds to reduce variance arising from stochasticity. Our results in Table~\ref{tab:control_env_results} show that \method faithfully achieves target returns for diverse starting checkpoints across all environments, \ie \method converges to the target return value generating $\pi_{\text{cf}}$ that obtains return very close to the given $\rtarget$. Further, we find an intuitive trend in the number of outer policy (KL-pivoting) updates, where we 
observe a lesser number of outer policy updates when the $\rtarget$ value is closer to the performance of the original policy $J_{\pi_{\theta_0}}$. Our results on these standard control environments form the basis for further investigation of more complex environments having a discrete or a continuous action space, which we explore in the next section.

\subsection{Contrastive Insights into RL policies}
% using \method}
Next, we present our analysis on generating counterfactual explanations using more complex environments.

\looseness=-1
\xhdr{1) Lunar Lander} We train an RL agent on \textit{LunarLander-v2 }using A2C, and intervene the training to retrieve the policy $\pi_0$ for our contrastive analysis. The original policy, on average, achieves a return of 50 (refer Figure~\ref{fig:lunarlander_quantitative}). We then generate its counterfactuals for target returns of $\{100, 150\}$ to understand how the policy can be improved further and also for target returns of $\{0, -50\}$ to understand how the policy can become worse. We present landing scenarios of the policy $\pi_0$ on three different surfaces in Figure~\ref{fig:lunarlander_qualitative} and their respective contrastive explanations for improvement and deterioration. Our results for $\pi_0$ rollouts show interesting agent characteristics like ``slow start'', ``a quick fall after covering half the distance,'' and ``landing near the right flag''. We find an improved version of the original policy in the generated counterfactuals with $\rtarget=100$ (Fig.~\ref{fig:lunarlander_qualitative}; column 2), where we observe that a uniform descent and shifting the landing slightly to the left between the two flags can improve the original policy $\pi_0$. Further, the counterfactual with  $\rtarget=150$ (Fig.~\ref{fig:lunarlander_qualitative}; column 3) shows uniform descent with decelerated landing can lead to even higher gains. In contrast, $\rtarget=0$ (Fig.~\ref{fig:lunarlander_qualitative}; column 4) shows how by starting very fast and making \textit{free fall} before landing outside the space between flags, $\pi_0$ can go worse. Similarly, $\rtarget=-50$ (Fig.~\ref{fig:lunarlander_qualitative}; column 5) show how the policy could worsen/collapse by making the agent land further right. Notably, the generated counterfactuals for targeted deterioration of performance using \method can be interpreted as a robust way to unlearn~\citep{unlearning, unlearning_survey} RL skills as it might be required to forget certain aspects of learning(\eg in Fig.~\ref{fig:lunarlander_qualitative}, the slow start of $\pi_0$).

\xhdr{2) Bipedal Walker} For the BipedalWalker-v3 environment, we train a PPO agent that demonstrates early success in achieving walking behavior. We conduct experiments to analyze the trained Bipedal agent contrastively by estimating counterfactuals at target returns of 50 and 150 (refer Fig.~\ref{fig:bipedal_convergence}). We demonstrate the qualitative results in Figure~\ref{fig:bipedal_qualitative}, where we find that the given policy $\pi_0$ has a peculiar walk, kneeling on the right leg and taking stride with the left one. In contrast, our generated counterfactual policy using $\rtarget=150$ shows improved (upright and faster) walk of the Bipedal agent. Further, when we reduce the target return to $\rtarget=50$, we observe that the kneeling gets intensified and the agent starts to drag itself to the finishing line, making the agent to walk slow and also fall (as shown in Scenario 1).

Across both environments, we observe that \method generates counterfactual policies that look similar to the original policy, keeping the essential characteristics of the

% Figure environment removed
policy the same. This enables us to easily contrast between different versions of the same policies.
% Figure environment removed
\section{Conclusion and Discussion}
\label{sec:conclusion}
In this work, we present a systematic framework for contrastive analysis of reinforcement learning policies. In doing so, we propose \method for generating counterfactual explanations for a given RL policy. We  demonstrate the flexibility of \method across multiple RL benchmarks and find insights into RL policy learning. We carry out a detailed theoretical analysis to connect the counterfactual estimation with eminent trust region optimization methods in RL. Further, results across five OpenAI gym environments show that \method generates faithful counterfactuals for (un)learning skills while keeping close to the original policy. To the best of our knowledge, our work presents the first technique to generate counterfactual explanation policies. Overall, we believe our work paves the way toward a new direction in the methodical contrastive analysis of RL policies and brings more transparency to RL decision-making. In our present work, we mainly used on-policy policy gradient-based RL optimization techniques, which are not very sample efficient, and we assume the stochastic nature of the policies which might not be the case always. It would be interesting to explore the utility of \method in unlearning options (skills) in hierarchical RL~\citep{hrl,hrl_survey}. Also, \method gives a glimpse into return contours in RL which could be further explored for enhanced optimization.

% \clearpage
\bibliography{references}
\bibliographystyle{icml2023}


%%%%%%%%%%%%%%%%%%%%%%%%%%%%%%%%%%%%%%%%%%%%%%%%%%%%%%%%%%%%%%%%%%%%%%%%%%%%%%%
%%%%%%%%%%%%%%%%%%%%%%%%%%%%%%%%%%%%%%%%%%%%%%%%%%%%%%%%%%%%%%%%%%%%%%%%%%%%%%%
% APPENDIX
%%%%%%%%%%%%%%%%%%%%%%%%%%%%%%%%%%%%%%%%%%%%%%%%%%%%%%%%%%%%%%%%%%%%%%%%%%%%%%%
%%%%%%%%%%%%%%%%%%%%%%%%%%%%%%%%%%%%%%%%%%%%%%%%%%%%%%%%%%%%%%%%%%%%%%%%%%%%%%%
% \newpage
% \appendix
% \onecolumn
% \section{You \emph{can} have an appendix here.}

% You can have as much text here as you want. The main body must be at most $8$ pages long.
% For the final version, one more page can be added.
% If you want, you can use an appendix like this one, even using the one-column format.
%%%%%%%%%%%%%%%%%%%%%%%%%%%%%%%%%%%%%%%%%%%%%%%%%%%%%%%%%%%%%%%%%%%%%%%%%%%%%%%
%%%%%%%%%%%%%%%%%%%%%%%%%%%%%%%%%%%%%%%%%%%%%%%%%%%%%%%%%%%%%%%%%%%%%%%%%%%%%%%


\end{document}


% This document was modified from the file originally made available by
% Pat Langley and Andrea Danyluk for ICML-2K. This version was created
% by Iain Murray in 2018, and modified by Alexandre Bouchard in
% 2019 and 2021 and by Csaba Szepesvari, Gang Niu and Sivan Sabato in 2022.
% Modified again in 2023 by Sivan Sabato and Jonathan Scarlett.
% Previous contributors include Dan Roy, Lise Getoor and Tobias
% Scheffer, which was slightly modified from the 2010 version by
% Thorsten Joachims & Johannes Fuernkranz, slightly modified from the
% 2009 version by Kiri Wagstaff and Sam Roweis's 2008 version, which is
% slightly modified from Prasad Tadepalli's 2007 version which is a
% lightly changed version of the previous year's version by Andrew
% Moore, which was in turn edited from those of Kristian Kersting and
% Codrina Lauth. Alex Smola contributed to the algorithmic style files.
