
\subsection{Truncation over WPTS's} \label{sec:truncation}

%Below we further introduce a truncation operation over WPTS's to tighten the lower and upper bounds derived by the two theoretical approaches above. 
In the following, we propose a truncation operation for a WPTS that restricts the value of every program variable in the WPTS to a prescribed bounded range. We consider that a bounded range for a program variable could be either $[-R,R]$ ($R> 0$), or $[0,R], [-R,0]$ if the value of the program variable is guaranteed to be non-negative or non-positive. 

To present our truncation operation, we define the technical notions of truncation and approximation functions.  
%first give the definition for a truncating function that %the formal definition of 
A \emph{truncation function} 
%$\trunc:\pvars\to \mathbb{I}_\Rset$ 
$\trunc$ is a function that maps every program variable $x\in\pvars$ to a bounded interval $\trunc(x)$ in $\Rset$ that specifies the bounded range of the variable $x$. We denote by $\Phi_\trunc$ the 
%logical 
formula $\bigwedge_{x\in\pvars} x\in \trunc(x)$
for a truncation function $\trunc$. 
An \emph{approximation function} is a function $\calM:\mathbb{R}^{|\pvars|}\to [0,\infty)$ such that each $\calM(\pv)$ ($\pv\in \mathbb{R}^{|\pvars|}$) is intended to be an over- or under-approximation of the expected weight 
%acts as an upper/lower bound for the expected weight 
$\llbracket \Pi\rrbracket_{\pv} (\Rset^{|\pvars|})$ 
outside the bounded range specified by $\Phi_\trunc$. 
The truncation operation is given by the following definition. 



\begin{definition}[Truncation Operation]\label{def:truncation}
Let 
%$\Pi = (\pvars, \rvars, \mathcal{D}, L,\transset, \lin,\valin, \lout,\win)$ 
$\Pi$ be a WPTS in the form of \eqref{eq:wpts}. Given a truncation function $\trunc$ and an approximation function $\calM$,
%that maps every program variable $x\in\pvars$ to a bounded interval $B(x)$ in 
%$\mathbb{R}$ (that specifies the bounded range of the variable $x$), 
the \emph{truncated} WPTS $\Pi_{\trunc,\calM}$ w.r.t. $\trunc$ and $\calM$ is defined as
	$\Pi_{\trunc,\calM}:=( \pvars, \rvars, L\cup\{\#\}, \lin, \lout,\mu_{\mathrm{init}},\rdvarjdis, \transset_{\trunc,\calM})$
	where $\#$ is a fresh deadlock location and the transition relation $\transset_{\trunc,\calM}$ is given by
	\begin{align*}
	&\transset_{\trunc,\calM}:=\{\langle \loc, \phi\wedge \Phi_\trunc, F_1, \dots, F_k \rangle\mid \langle \loc, \phi, F_1,\dots, F_k \rangle\in\transset\mbox{ and } \loc\ne \lout\}\\
 &\quad\cup \{\langle \loc, \phi\wedge (\neg\Phi_\trunc), F^{\calM,\sharp}_1, \dots, F^{\calM,\sharp}_k \rangle\mid \langle \loc, \phi, F_1,\dots, F_k \rangle\in\transset\mbox{ and } \loc\ne \lout\}\tag{\ddag}\\
 &\quad\cup\{\langle \lout, \mathbf{true}, F_{\lout}\rangle, \langle \sharp, \mathbf{true}, F_\sharp\rangle \}
\end{align*}
for which (a) we have $F_\loc:=\langle \loc,1,\mbox{\sl id},\overline{1} \rangle$ ($\loc\in\{\lout,\sharp\}$) where
 %$F'_\sharp:=\langle \sharp,1,\mbox{\sl id},\mathbf{1} \rangle$  
 $\mbox{\sl id}$ is the identity function and $\overline{1}$ is the constant function that always takes the value $1$, and (b) for a fork $F=\langle \loc', p, \mbox{\sl upd}, \wet\rangle$ in the original WPTS $\Pi$ we have 
$F^{\calM,\sharp}:=F$ if $\loc'=\lout$ and $F^{\calM,\sharp}:=\langle \sharp, p, \mbox{\sl upd}, \calM\rangle$ otherwise.
\end{definition}

Thus, the truncated WPTS is obtained from the original one by first restraining each transition to the bounded range $\Phi_\trunc$ 
%for all $x\in\pvars$ 
and then redirecting to the fresh deadlock location $\sharp$ all the situations jumping out of the bounded range and not going to the termination location. To make the truncated WPTS deterministic and total, we add the self-loop $\langle \sharp, \mathbf{true}, F_\sharp\rangle$. %to the location 
Our main theorem shows that by choosing an appropriate approximation function $\calM$ in the truncation, 
%on the transitions to the deadlock location $\sharp$ in the truncation, 
one can obtain upper/lower approximation of the original WPTS. 


\begin{theorem}\label{thm:upperlower}
Let $\Pi$ be a WPTS in the form of \eqref{eq:wpts}, $\trunc$ a truncation function and $\calM$ an approximation function.
Suppose that the following condition ($\ast$) holds:
\begin{itemize}
\item[($\ast$)] for each fork $F^{M,\sharp}=\langle \sharp, p, \mbox{\sl upd}, \calM\rangle$ in the truncated WPTS $\Pi_{\trunc,\calM}$ that is derived from
some fork $F=\langle \loc', p, \mbox{\sl upd}, \wet\rangle$ with the source location $\loc$ in the original WPTS (see sentence (b) in Definition~\ref{def:truncation}),   
we have that $\llbracket \Pi\rrbracket_{\pv}(\Rset^{|\pvars|})\le \calM(\pv)$ for all $\pv$ such that the state $(\loc,\pv)$ is reachable and $\pv\not\models\Phi_\trunc$. 
\end{itemize}
Then $\llbracket \Pi\rrbracket_{\valin} (\Rset^{|\pvars|})\le \llbracket \Pi_{\trunc,\calM}\rrbracket_{\valin}(\Rset^{|\pvars|})$ for all initial program valuations $\valin$. 
Analogously, if it holds the condition ($\star$) which is almost the same as ($\ast$) except for that ``$\llbracket \Pi\rrbracket_{\pv}(\Rset^{|\pvars|})\le \calM(\pv)$'' is replaced with ``$\llbracket \Pi\rrbracket_{\pv}(\Rset^{|\pvars|})\ge \calM(\pv)$'', then we have $\llbracket \Pi\rrbracket_{\valin} (\Rset^{|\pvars|})\ge \llbracket \Pi_{\trunc,\calM}\rrbracket_{\valin}(\Rset^{|\pvars|})$ for all initial program valuations $\valin$. 
\end{theorem}

The theorem above states that if the approximation function gives correct bounds for the expected weights of the original WPTS outside the bounded range, then the bounds for the expected weights of the truncated WPTS are also correct bounds for the expected weights of the original WPTS. The detailed proof is relegated to \cref{app:sec5}.




\begin{example}\label{ex:pedestrian-trunc}
Recall the Pedestrian example in \cref{fig:pedestrian-program}, here we make truncation to this example and generate its truncated WPTS. The truncation function $\trunc$ is defined such that $\trunc(pos)=[0,5]$, $\trunc(dis)=[0,5]$, so $\Phi_\trunc=0\le pos\le 5\wedge 0\le dis\le 5$. 
%Let the approximation function $\calM=0$. 
Following the truncation operation in ($\ddag$) from Definition~\ref{def:truncation}, we can obtain $\Pi_{\trunc,\calM}$ as shown in \cref{fig:truncated-wpts}.\qed



% Figure environment removed

\end{example}




