\subsection{The Fixed-Point Approach}

We review some basic concepts of lattice theory. Given a partial order $\sle$ on a set $K$ and a subset $K' \subseteq K,$ an \emph{upper bound} of $K'$ is an element $u \in K$ that is no smaller than every element of $K'$, i.e.,~$\forall k' \in K'.~k' \sle u.$ Similarly, a \emph{lower bound} for $K'$ is an element $l$ that is no greater than every element of $K',$ i.e.~$\forall k' \in K'.~l \sle k'.$ The \emph{supremum} of $K',$ denoted by $\bigsqcup K'$, is an element $u^* \in K$ such that $u^*$ is an upper-bound of $K'$ and for every upper bound $u$ of $K',$ we have $u^* \sle u.$ Similarly, the \emph{infimum} $\bigsqcap K'$ is a lower bound $l^*$ of $K'$ such that for every lower-bound $l$ of $K',$ we have $l \sle l^*.$ We define $\bot\!:=\!\bigsqcap K$ and $\top\!:=\!\bigsqcup K.$ In general, suprema and infima may not exist.

A partially-ordered set $(K, \sle)$ is called a \emph{complete lattice} if every subset $K'\subseteq K$ has a supremum and an infimum.
Given a partial order $(K, \sle)$, a function $f: K \to K$ is called \textit{monotone} if for every $k_1 \sle k_2$ in $K$, we have $f(k_1) \sle f(k_2).$

Given a complete lattice $(K, \sle),$ a function $f: K \to K$ is called \emph{continuous} if for every increasing chain $k_0 \sle k_1 \sle \ldots$ in $K,$ we have $f(\bigsqcup \{k_n\}_{n=0}^\infty) = \bigsqcup \{f(k_n)\}_{n=0}^\infty,$ and \emph{cocontinuous} if for every decreasing chain $k_0 \sge k_1 \sge \ldots$ of elements of $K,$ we have $f(\bigsqcap \{k_n\}_{n=0}^\infty) = \bigsqcap \{f(k_n)\}_{n=0}^\infty.$ An element $k \in K$ is called a \emph{fixed-point} if $f(k) = k.$ Moreover, $k$ is a \emph{pre fixed-point} if $f(k) \sle k$ and a \emph{post fixed-point} if $k\sle f(k)$. The \emph{least fixed-point} of $f$, denoted by $\lfp f,$ is the fixed-point that is no greater than every fixed-point under $\sle.$ Analogously, the \emph{greatest fixed-point} of $f$, denoted by $\gfp f$, is the fixed-point that is no smaller than all fixed-points.




\begin{theorem}[\textit{Kleene}~\cite{Sangiorgibook}]
\label{thm:kleene}
Let $(K, \sle)$ be a complete lattice and $f: K \to K$ be an continuous function. Then, we have
	$$\textstyle \lfp\ f = {\textstyle \mathop{\bigsqcup}_{i \ge 0}} \left\{f^{(i)}(\bot)\right\}.$$
Analogously, if $f$ is cocontinuous, then we have
	$$\textstyle \gfp\ f = {\textstyle \mathop{\bigsqcap}_{i \ge 0}} \left\{f^{(i)}(\top)\right\}.$$
\end{theorem}


In this work, we apply the well-known Tarski's fixed-point theorem. 

\begin{theorem}[\textit{Tarski}~\cite{KnasterTarski}]\label{thm:tarski}
Let $(K, \sle)$ be a complete lattice and $f:K \to K$ a monotone function. Then, both $\lfp\ f$ and $\gfp\ f$ exist. Moreover, we have
%$\lfp\ f$ is the infimum of all pre fixed-points, and $\gfp\ f$ is the supremum of all post fixed-points, i.e.,
\begin{align*}
	\textstyle \lfp\ f  = \mathop{\bigsqcap} \left\{x\ |\ f(x)\sle x\right\}\mbox{ and }\gfp\ f = \mathop{\bigsqcup} \left\{x\ |\ x\sle f(x)\right\}. %\label{formula:upper}
\end{align*}
\end{theorem}



Based on \cref{thm:tarski}, %and \cref{thm:kleene}, 
we present our fixed-point approach
%. Our fixed-point approach works 
for non-score-recursive WPTS's. 
Below we fix a non-score-recursive WPTS $\Pi$ in the form of \eqref{eq:wpts}. 
Given a maximum finite value $M\in [0,\infty)$, we define a \emph{state function} as a %measurable 
function $h:\Lambda\to [-M,M]$ such that for all $\pv\in\Rset^{|\pvars|}$, we have that $h(\lout,\pv)\in [0,M]$.
%both $h(\lout,\pv)$ and $h(\sharp,\pv)$ equals one.
We denote the set of all state functions with maximum value $M$ by $\mathcal{K}_M$. We also use the usual partial order $\le$
%$\sqsubseteq$ 
on $\mathcal{K}_M$ 
%(for which we follow the convention in mathematics that still uses $\le$ to denote the partial order) 
that is defined in the pointwise fashion, i.e., for any $h_1,h_2\in \mathcal{K}_M$, $h_1\le h_2$ iff $h_1(\Xi)\le h_2(\Xi)$ for all $\Xi\in\Lambda$. It is straightforward to verify that  $(\mathcal{K}_M,\le)$ is a complete lattice. For example, the top element $\top$ (resp. the bottom element $\bot$) in the set $\mathcal{K}_M$ is the constant function that maps every state $\Xi$ to $M$ (resp. $-M$), and the least upper bounds (resp. greatest lower bound) can be given via the pointwise infimum (resp. supremum), respectively. 



To connect the complete lattice $(\mathcal{K}_M,\le)$ of state functions with expected weights (\cref{def:exp-wt}), we define a special state function, called \emph{expected-weight function} $\mbox{\sl ew}_\Pi$, by $\mbox{\sl ew}_\Pi(\lin,\pv):=\llbracket \Pi\rrbracket_{\pv}(\Rset^{|\pvars|})$, and omit the subscript $\Pi$ if it is clear from the context. Informally, $\mbox{\sl ew}_\Pi(\lin,\pv)$ is the expected weight of all program runs that start from $\pv$ without the restriction of the subset $\calU$. In this work, we consider the following monotone function over the complete lattice $(\mathcal{K}_M,\le)$. 



\begin{definition}[Expected-Weight Transformer]\label{def:ewt}
Given a finite maximum value $M\in [1,\infty)$, the \emph{expected-weight transformer} $\ewt_\Pi:\mathcal{K}_M\to \mathcal{K}_M$ is the higher-order function such that for each state function $h\in \mathcal{K}_M$ and state $(\loc,\pv)$, if  $\tau = \langle \loc, \phi, F_1,\dots,F_k \rangle$ is the unique transition that satisfies $\pv\models\phi$ and $F_j=\langle \loc'_j,p_j,\upd_j,\wet_j\rangle$ for each $1\le j\le k$, then we have that
\begin{equation}\label{eq:ewt}
\ewt_\Pi(h)(\loc,\pv)\,:=\, \begin{cases}  \sum_{j=1}^k p_j\cdot \expectdist{\rv}{\wet_j(\pv,\rv)\cdot h(\loc'_j,\upd_j(\pv,\rv))} & \mbox{if } \loc\neq \lout \\
1 & \mbox{otherwise} 
\end{cases}\enskip. 
\end{equation}
Here the expectation $\expectdist{\rv}{-}$ is taken over a sampling valuation $\rv$ that observes the independent joint probability distributions of all the sampling variables $r\in \rvars$. 	
\end{definition}
Informally, given a state function $h$, the expected-weight transformer $\ewt_\Pi$ computes the expected weight $\ewt_\Pi(h)$  after one step of the WPTS execution. From the monotonicity of the Lebesgue integral (from the definition of expectation), we have that $\ewt_\Pi$ is monotone. 
Note that although the Lebesgue integral of a function $f$ usually requires measurability of the function, its preliminary definition is given via a supremum over simple functions not exceeding the function, and hence can be applied to all functions (including non-measurable functions) for which the monotonicity remains to be true. Therefore, we do not need to impose measurability here.
In the following, we will omit the subscript $\Pi$ in $\ewt_\Pi$ if it is clear from the context.

\begin{definition}[Potential Weight Functions]\label{def:puwf}
	A \emph{potential upper weight function} (PUWF) 
	%w.r.t. an invariant $I$ 
	is a function $h:\locs{}\times\val{V_p} \rightarrow\Rset$
	that has the properties below:
	\begin{itemize}
		\item[\emph{(C1)}] for all reachable states $(\loc,\pv)$ with $\loc\neq\lout$, 
		we have $\ewt({h})(\loc,\pv) \le h(\loc,\pv)$;
		\item[\emph{(C2)}] for all reachable states $(\loc,\pv)$ such that $\loc=\lout$, we have $h(\loc,\pv)=1$.
	\end{itemize}
	Analogously, 
	a \emph{potential lower weight function} (PLWF) is a function $h:\locs{}\times\val{V_p} \rightarrow\Rset$
	that satisfies the conditions (C1') and (C2), for which the condition (C1') is almost the same as (C1) except for that ``$\ewt({h})(\loc,\pv) \le h(\loc,\pv)$'' is replaced with ``$\ewt({h})(\loc,\pv) \ge h(\loc,\pv)$''. 
\end{definition}

Informally, a PUWF is a state function that satisfies the pre fixed-point condition of $\ewt$ at non-terminating locations, and equals one at the termination location. A PLWF is defined similarly, with the difference that we use the post fixed-point condition instead. 




The following result lays the backbone of our fixed-point approach. 



\begin{theorem}\label{thm:lfp}
Let $\Pi$ be a non-score-recursive WPTS that is score-bounded by a positive real $M>0$.
%whose weights at termination are bounded in $[-M,M]$ for a finite $M\ge 1$. 
If the WPTS $\Pi$ has the AST property, then the expected-weight function $\mbox{\sl ew}$ is the unique fixed-point of the higher-order function $\ewt$ on the complete lattice $(\mathcal{K}_{\max\{M,1\}},\le)$. 
\end{theorem}


\begin{proof}[Proof sketch.]
Let $M':=\max\{M,1\}$. We first prove that the expected-weight function $\mbox{\sl ew}$ is the least fixed-point of the higher-order function $\ewt$. Then we prove that 
%if $M>0$, then 
the expected-weight transformer $\ewt:\mathcal{K}_{M'}\to \mathcal{K}_{M'}$ is both continuous and cocontinuous. Finally, according to \cref{thm:kleene}, we prove the uniqueness that $\lfp\ \ewt(\loc, \pv) = \gfp\ \ewt(\loc, \pv)$, i.e., the fixed-point $\mbox{\sl ew}$ is unique. 
For the space limitation, we relegate the detailed proof for Theorem~\ref{thm:lfp} 
%and Theorem~\ref{thm:unique} 
to Appendix~\ref{app:fixedpoint}. 
\end{proof}


By a combination of~\cref{thm:lfp} and~\cref{thm:tarski}, one has that it suffices to derive a pre fixed-point of $\ewt$ to obtain an upper bound for $\mbox{\sl ew}$, and a post fixed-point to obtain a lower bound in the case of almost-sure termination of the program . 

\begin{theorem}[Fixed-Point Approach]\label{thm:fix-point-bounds}
Let $\Pi$ be a non-score-recursive WPTS that is score-bounded by a finite value $M>0$. 
%whose weights at termination are bounded in $[-M,M]$ for a finite $M\ge 1$. 
If the WPTS $\Pi$ has the AST property, then $\llbracket \Pi\rrbracket_{\valin} (\Rset^{|\pvars|})\le h(\lin,\valin)$ (resp. $\llbracket \Pi\rrbracket_{\valin} (\Rset^{|\pvars|})\ge h(\lin,\valin)$) for any PUWF (resp. PLWF) $h$ over $\Pi$ and initial state $(\lin,\valin)$, respectively.
\end{theorem}


\begin{remark}\label{rmk:maxvalue}
Notice that we always require a finite maximal value $M$ in our fixed-point approach. In general, our fixed-point approach cannot be applied to score-recursive Bayesian probabilistic programs, as in such a program a finite maximum weight $M$ may not exist. 
\end{remark}





