
Below we compare our results with the most related work in the literature.
 

\paragraph{Static analysis in Bayesian probabilistic programming.}  There are a lot of works on NPD inference for probabilistic programs, such as $(\lambda)$PSI~\cite{DBLP:conf/cav/GehrMV16,DBLP:conf/pldi/GehrSV20}, AQUA~\cite{DBLP:conf/atva/HuangDM21}, Hakaru~\cite{DBLP:conf/flops/NarayananCRSZ16} and SPPL~\cite{DBLP:conf/pldi/SaadRM21}. However, these methods are restricted to specific kinds of programs, e.g., programs with closed-form solutions to NPD or without continuous distributions, and none of them can handle probabilistic programs with unbounded while-loops/recursion. As far as we know, the most revelant
 work on static analysis of posterior distribution over unbounded loops/recursion is the approach~\cite{Beutner2022b} that 
infers the bounds for posterior distributions by recursion unrolling and bounding the non-termination case via the widening operator of abstract interpretation. By unrolling recursion to arbitrary depth, this approach can achieve high precision on the derive bounds. However, a major drawback 
of this approach is that the recursion unrolling may cause path explosion. 
Our approach circumvents the path explosion problem by constraint solving. 
%when there is a non-negligible number of conditional branches inside the loop body. 
Another major drawback is that this approach cannot handle score-recursive programs as simply applying the approach to score-recursive programs leads to the trivial bound $[0,\infty]$, and we address this issue by a novel OST variant. 

\paragraph{MCMC and variational inference.} As mentioned previously, statistical approaches such as MCMC~\cite{rubinstein2016simulation,gamerman2006markov} and variational inference~\cite{blei2017variational} cannot provide formal guarantee on the bounds for posterior distributions in a finite time limit. In contrast, our approach has formal guarantee on the derived %upper and lower 
bounds.

\paragraph{Static analysis of probabilistic programs.} In recent years, there have been an abundance of works on static analysis of probabilistic programs. Most of them address fundamental aspects such as 
termination~\cite{DBLP:conf/cav/ChakarovS13,DBLP:conf/popl/ChatterjeeFNH16,DBLP:conf/vmcai/FuC19}, sensitivity~\cite{DBLP:journals/pacmpl/BartheEGHS18,DBLP:journals/pacmpl/WangFCDX20}, expectation~\cite{DBLP:conf/pldi/NgoC018,cost2019wang}, tail bounds~\cite{kura2019tail,DBLP:conf/pldi/Wang0R21,wang2022tail}, assertion probability~\cite{DBLP:conf/pldi/SankaranarayananCG13,DBLP:conf/pldi/WangS0CG21}, etc. Compared with these results, we have:
\begin{itemize}
\item Our work focuses on normalized posterior distribution in Bayesian probabilistic programming, and hence is an orthogonal objective. 
\item Our algorithm follows the previous works on the synthesis of polynomial templates~\cite{DBLP:conf/cav/ChakarovS13,cost2019wang,DBLP:journals/toplas/ChatterjeeFNH18,ChatterjeeFG16}, but we have a truncation operation to increase the accuracy which to our best knowledge is novel.
\item Our approach extends the classical OST as the previous works~\cite{cost2019wang,DBLP:conf/pldi/Wang0R21} do, but we consider a multiplicative variant, while the work ~\cite{cost2019wang} considers only an additive variant, and the work~\cite{DBLP:conf/pldi/Wang0R21} considers a general extension through the uniform integrability condition and an implementation via polynomial functions, but does not have a detailed treatment for a multiplicative variant. 
\end{itemize}
A very recent work~\cite{DBLP:conf/tacas/BatzCJKKM23} considers the synthesis of piecewise bounds for probabilistic programs. We focus on non-piecewise polynomial bounds and hence is orthognal. A promising future direction would be also to consider piecewise bounds in the NPD problem.

