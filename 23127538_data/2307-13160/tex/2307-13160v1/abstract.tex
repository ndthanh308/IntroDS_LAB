%% Abstract
%% Note: \begin{abstract}...\end{abstract} environment must come
%% before \maketitle command

\begin{abstract}
In Bayesian probabilistic programming, a central problem is to estimate the normalised posterior distribution (NPD) of a probabilistic program with {\tt score} (a.k.a. {\tt observe}) statements. 
Prominent approximate approaches 
to address this problem 
include Markov chain Monte Carlo (MCMC) and variational inference (VI), 
but neither of them is capable of generating guaranteed outcomes within a finite time limit.
Moreover, most existing formal approaches that perform exact inference for NPD are restricted to programs with closed-form solutions to NPD or bounded loops/recursion.
A recent work (Beutner \emph{et al.}, PLDI 2022) 
proposed an automated approach that derives guaranteed bounds for NPD over programs with unbounded recursion. 
However, as this approach requires recursion unrolling, 
it suffers from the path explosion problem. 
Furthermore, previous approaches do not consider \emph{score-recursive} probabilistic programs that allow  %probabilistic programs with 
score statements inside loops, which is non-trivial and requires careful treatment to ensure the integrability of the normalising constant in NPD.

In this work, we propose a novel automated approach to derive 
bounds for NPD via polynomial templates, fixed-point theorems and Optional Stopping Theorem (OST). Our approach can handle probabilistic programs with unbounded while loops and continuous distributions with infinite supports.
The novelties in our approach
are three-fold: First, we use polynomial templates to 
circumvent the path explosion problem from recursion unrolling; 
Second, we derive a novel multiplicative variant of OST that addresses the integrability issue
in score-recursive programs; Third, to increase the accuracy of the derived bounds via polynomial templates, we propose a novel technique of truncation that truncates a program into a bounded range of program values. 
%follow the heuristics to 
%synthesizes polynomial templates over a truncated bounded range of program values, 
%for which the heuristics is justified by the facts that (i) polynomials can approximate any continuous function over a bounded range (Weierstrass Approximation Theorem) and (ii) continuous functions can almost approximate any measurable function in any compact set (Lusin Theorem). 
Experiments over a wide range of benchmarks demonstrate that our approach is time-efficient and can derive bounds for NPD that are comparable with (or even tighter than) the recursion-unrolling approach (Beutner \emph{et al.}, PLDI 2022).
\end{abstract}


%%% 2012 ACM Computing Classification System (CSS) concepts
%%% Generate at 'http://dl.acm.org/ccs/ccs.cfm'.
\begin{CCSXML}
	<ccs2012>
	<concept>
	<concept_id>10003752.10003790.10002990</concept_id>
	<concept_desc>Theory of computation~Logic and verification</concept_desc>
	<concept_significance>500</concept_significance>
	</concept>
	<concept>
	<concept_id>10003752.10003790.10003794</concept_id>
	<concept_desc>Theory of computation~Automated reasoning</concept_desc> 
	<concept_significance>500</concept_significance>
	</concept>
	<concept>
	<concept_id>10003752.10010124.10010138.10010142</concept_id>
	<concept_desc>Theory of computation~Program verification</concept_desc>
	<concept_significance>500</concept_significance>
	</concept>
	</ccs2012>
\end{CCSXML}

\ccsdesc[500]{Theory of computation~Logic and verification}
\ccsdesc[500]{Theory of computation~Automated reasoning}
\ccsdesc[500]{Theory of computation~Program verification}
%
\keywords{Probabilistic programs, Bayesian inference, Quantitative verification, Martingales, Fixed-point Theory, Posterior distributions} 