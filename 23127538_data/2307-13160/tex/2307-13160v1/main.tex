% This is samplepaper.tex, a sample chapter demonstrating the
% LLNCS macro package for Springer Computer Science proceedings;
% Version 2.21 of 2022/01/12
%
%\documentclass[review,runningheads]{llncs}
\documentclass[acmsmall,screen]{acmart}
\settopmatter{printfolios=true,printccs=true,printacmref=false}

%%
%% \BibTeX command to typeset BibTeX logo in the docs
\AtBeginDocument{%
	\providecommand\BibTeX{{%
			Bib\TeX}}}
%% Rights management information.  This information is sent to you
%% when you complete the rights form.  These commands have SAMPLE
%% values in them; it is your responsibility as an author to replace
%% the commands and values with those provided to you when you
%% complete the rights form.
%\setcopyright{acmcopyright}
\setcopyright{none}
\copyrightyear{2018}
\acmYear{2018}
%\acmDOI{}

\acmPrice{}
%%
%% These commands are for a JOURNAL article.
%\acmJournal{Proc. ACM Program. Lang.}
%\acmVolume{1}
%\acmNumber{1}
%\acmArticle{1}
%\acmMonth{1}
\acmJournal{PACMPL}
\acmVolume{1}
\acmNumber{CONF}
\acmArticle{1}
\acmYear{2018}
\acmMonth{1}
\acmDOI{} % \acmDOI{10.1145/nnnnnnn.nnnnnnn}
\startPage{1}

\usepackage[T1]{fontenc}
% T1 fonts will be used to generate the final print and online PDFs,
% so please use T1 fonts in your manuscript whenever possible.
% Other font encondings may result in incorrect characters.
%
\usepackage{hyperref}
\usepackage{fancyhdr}
\usepackage{graphicx}
\usepackage{amsmath}
%\usepackage{amssymb}
\usepackage{stmaryrd,wasysym}
\usepackage{booktabs,comment} 

\usepackage{multirow}

\usepackage[lined,boxed,ruled]{algorithm2e}

%\usepackage{amsthm}

\usepackage[normalem]{ulem}
\usepackage{tikz,pgffor}
\usetikzlibrary{arrows}
\usetikzlibrary{shapes}
\usetikzlibrary{calc}
\usetikzlibrary{automata}
\usetikzlibrary{positioning}

%\sloppy
%%\usepackage{times}
%\def\baselinestretch{0.96}


\usepackage{bbm}

\tikzstyle{proglabel}=[shape=circle,draw,inner sep=0pt,minimum size=5mm]
\tikzstyle{tran}=[draw,->,>=stealth, rounded corners]

\usepackage{listings}
\lstdefinelanguage{prog}
{
	morekeywords={if, then, else, fi, while, do, od, true, false, and, or, skip, sample, observe,return,score,normal,uniform, prob},
	sensitive = false
}

\usepackage{threeparttable}
\usepackage{subfigure}
% \usepackage{subcaption}

\usepackage{tabularx}
\usepackage{dsfont}

\usepackage{bussproofs}

\usepackage[capitalise]{cleveref}

\makeatletter
\providecommand{\bigsqcap}{
	\mathop{
		\mathpalette\@updown\bigsqcup
	}
}
\newcommand*{\@updown}[2]{
	\rotatebox[origin=c]{180}{$\m@th#1#2$}
}
\makeatother

\newenvironment{scprooftree}[1]%
{\gdef\scalefactor{#1}\begin{center}\proofSkipAmount \leavevmode}%
	{\scalebox{\scalefactor}{\DisplayProof}\proofSkipAmount \end{center} }


\let\Cross\relax
\let\Square\relax

\newif\ifdraft
\draftfalse

\ifdraft
\usepackage[draft,nompar]{commenting}
\else
\usepackage[nompar]{commenting}
\fi

\declareauthor{lo}{lo}{olive}
\authorcommand{lo}{comment}


\declareauthor{fu}{fu}{blue}
\authorcommand{fu}{comment}

 
\declareauthor{pw}{pw}{orange}
\authorcommand{pw}{comment}

\declareauthor{ts}{ts}{purple}
\authorcommand{ts}{comment}

 
\newcommand\calF{\mathcal{F}}
\newcommand\calG{\mathcal{G}}
\newcommand\calM{\mathcal{M}}
\newcommand\calV{\mathcal{V}}
\newcommand\calU{\mathcal{U}}
\newcommand\calW{\mathcal{W}}
\newcommand\calP{\mathcal{P}}
\newcommand\calD{\mathbb{D}}
%%%%%%%%%%%%%%%%%
%% macros introduced by Luke 
\newcommand\mydef[1]{{\bf\em #1}}
%%%%%%%%%%%%%%%%%

\newcommand{\numviparams}{{| \lambda |}}
\newcommand{\scoreaccvars}[1]{s_1^{#1}, \ldots, s_{\numviparams}^{#1}}
\newcommand{\scoreaccvar}[2]{s_{#1}^{#2}}
\newcommand{\isdeterm}[1]{\text{Deterministic}({#1})}


\newcommand{\expect}[1]{\mathbb{E}\left[{#1}\right]}
\newcommand{\var}[1]{\mathbb{V}\left[ {#1} \right]}
\newcommand{\expectdist}[2]{\mathbb{E}_{#1}\left[ {#2} \right]}
\newcommand{\vardist}[2]{\mathbb{V}_{#1}\left[ {#2} \right]}
\newcommand{\cov}[2]{\mathbb{C}\text{ov}[{#1}][{#2}]}
\newcommand{\covv}[1]{\mathbb{C}\text{ov}[{#1}]}
\newcommand{\corr}[1]{\mathbb{C}\text{orr}[{#1}]}

\newcommand{\fix}[1]{\mathit{fix}\left({#1}\right)}
\newcommand{\sbr}[1]{\left\llbracket {#1} \right\rrbracket}
\newcommand{\ctxtype}[3]{{#1} \cong_\text{ctx} {#2} : {#3}}
\newcommand{\bigstep}[3]{{#1} \Downarrow_{#2} {#3}}


% PCF types
\newcommand{\bool}{\mathit{bool}}
\newcommand{\nat}{\mathit{nat}}

\newcommand{\ctx}[1]{\mathcal{C}\left[ {#1}\right] }
\newcommand{\pcft}[1]{\text{PCF}_{#1}}

\newcommand{\nfl}{\mathbb{N}_\bot}
\newcommand{\bfl}{\mathbb{B}_\bot}

% PCF constructs
\newcommand{\succc}[1]{\mathbf{succ}({#1})}
\newcommand{\succcn}[2]{\mathbf{succ}^{#1}({#2})}
\newcommand{\zero}{\mathbf{0}}
\newcommand{\zerotest}[1]{\mathbf{zero}\left({#1}\right)}
\newcommand{\pred}[1]{\mathbf{pred}\left( {#1} \right)}
\newcommand{\predn}[2]{\mathbf{pred}^{#1}\left( {#2} \right)}
\def\solvable{\#}

\newcommand{\true}{\mathbf{true}}
\newcommand{\false}{\mathbf{false}}
\newcommand{\pcffix}[1]{\mathbf{fix}\left({#1}\right)}
\newcommand{\pcffn}[3]{\mathbf{fn}~{#1}:{#2}\mathpunct{.}{#3}}
\newcommand{\pairtype}[2]{{#1} * {#2}}
\newcommand{\pairexp}[2]{\mathbf{pair}({#1}, {#2})}
\newcommand{\leftexp}[1]{\mathbf{left}({#1})}
\newcommand{\rightexp}[1]{\mathbf{right}({#1})}

\newcommand{\RationalPos}{\mathbb{Q}^{+}}

\newcommand{\meas}[1]{\mathbb{M}\left( {#1} \right) }
\newcommand{\integ}[1]{\sbr{#1}_I}

\newcommand{\notbigstep}[2]{{#1}~\cancel{\Downarrow}_{#2}}
\newcommand{\subtrace}[3]{{#1}^{{#2} \ldots {#3}}}
\newcommand{\supp}[1]{\textsf{supp}\left({#1}\right)}
\newcommand{\dom}[1]{\textsf{Dom}\left({#1}\right)}
\newcommand{\suppk}[2]{\textsf{Supp}^{#1}\left({#2}\right)}
\newcommand{\tracespace}{\bigcup_{n \in \mathbb{N}}[0, 1]^n}
\newcommand{\generictracespace}{\mathbb{T}}
\newcommand{\nnreals}{\mathbb{R}_{\geq 0}}
\newcommand{\posreals}{\mathbb{R}_{> 0}}
\newcommand{\reals}{\mathbb{R}}

\newcommand{\unrollkM}[2]{\textsf{unroll}_{#1}\left({#2}\right)}
\newcommand{\nphmcint}[5]{\Psi_\textsf{NP}\left({#1}, {#2}, {#3}, {#4}, {#5}\right)}

%SPCF constructs
\newcommand{\spcfvalues}{\Lambda^0_v}

\newcommand{\prevalueM}[1]{\textsf{value}^{-1}_{#1}(\spcfvalues{})}
\newcommand{\num}[1]{\underline{#1}}

% \theoremstyle{definition}
% \newtheorem{thm}{Theorem}
% \newtheorem{lem}{Lemma}
% \newtheorem{defn}{Definition}
% \newtheorem{conj}{Conjecture}
% \newtheorem{prop}{Proposition}

%\theoremstyle{definition}
%\newtheorem{defn}{Definition}[section]
%\newtheorem{example}[defn]{Example}
%
%
%\theoremstyle{plain}
%\newtheorem{thm}{Theorem}[section]
%\newtheorem{lem}[thm]{Lemma}
%\newtheorem{cor}[thm]{Corollary}
%\newtheorem{conj}[thm]{Conjecture}
%\newtheorem{prop}[thm]{Proposition}
%\newtheorem{remark}[thm]{Remark}

%% Proofs
%\let\oldproof\proof
%\renewcommand{\proof}{\color{blue}\oldproof}


\definecolor{codegreen}{rgb}{0,0.6,0}
\definecolor{codegray}{rgb}{0.5,0.5,0.5}
\definecolor{codepurple}{rgb}{0.58,0,0.82}
\definecolor{backcolour}{rgb}{0.95,0.95,0.92}

\lstdefinestyle{myStyle}{
    belowcaptionskip=1\baselineskip,
    breaklines=true,
    frame=none,
    basicstyle=\footnotesize\ttfamily,
    keywordstyle=\bfseries\color{green!40!black},
    commentstyle=\itshape\color{purple!40!black},
    identifierstyle=\color{blue},
    backgroundcolor=\color{gray!10!white},
    %backgroundcolor=\color{backcolour}, 
    numberstyle=\tiny\color{codegray},
    stringstyle=\color{codepurple},
    breakatwhitespace=false,                          
    keepspaces=true,                 
    numbers=left,       
    numbersep=5pt,                  
    showspaces=false,                
    showstringspaces=false,
    showtabs=false,                  
    tabsize=2,
}

% argmin/argmax
\DeclareMathOperator*{\argmax}{arg\,max}
\DeclareMathOperator*{\argmin}{arg\,min}

% Concatenation of lists
\newcommand\doubleplus{+\kern-1.3ex+\kern0.8ex}

% Program configurations
\newcommand{\tuple}[1]{\ensuremath{\langle #1 \rangle}}
% Rule based definitions
\newcommand{\Rule}[4][]{\ensuremath{\inferrule*[lab={\hypertarget{#2}{(\TirName{#2})}},#1]{#3}{#4}}}

% Calligraphic symbols
\newcommand{\calI}{{\mathcal I}} 
\newcommand{\calT}{{\mathcal T}}

%  Macro for new Y operator.
\newcommand{\yBounded}[3]{\mu^{#1}_{#2}\rvert_{#3}}

%%%%%%%%%%%%%%%%%
 
%%%%%%%%%%%%%%%%%

\newcommand{\expv}{\mathbb{E}}

\newcommand{\combTr}[2]{\left[\begin{matrix}
		#1\\
		#2
	\end{matrix} \right]}

\newcommand{\exType}[2]{\left\{\begin{matrix}
		#1\\
		#2
	\end{matrix} \right\}}
\newcommand{\myint}[1]{ [#1]}
\newcommand{\Uniform}{\ensuremath{\mathrm{Uniform}}}
\newcommand{\Normal}{\ensuremath{\mathrm{normal}}}
\DeclareMathOperator{\abs}{abs}
\DeclareMathOperator{\pdf}{pdf}

\newcommand{\intConf}[1]{\lceil#1\rceil}
\newcommand{\tr}{\boldsymbol{t}}

\newcommand{\sample}{\tt{sample}}
%\newcommand{\fix}{\texttt{fix}}
%\newcommand{\num}[1]{\underline{#1}}
\newcommand{\myif}{\texttt{if}}
\newcommand{\mylet}{\texttt{let} \, }
\newcommand{\myin}{\, \texttt{in} \,}
\newcommand{\mythen}{\, \texttt{then} \,}
\newcommand{\myelse}{\, \texttt{else} \,}
\newcommand{\score}{\tt{score}}
\newcommand{\tick}{\tt{tick}}

\newcommand{\term}{\tt{term}}
\newcommand{\pv}{\mathbf{v}}
\newcommand{\rv}{\mathbf{r}}

\newcommand{\interval}{\mathfrak{I}}

\newcommand{\typeReal}{\textbf{\textsf{R}}}

\newcommand{\symbolInt}{\myint{\cdot}}

\newcommand{\LambdaInterval}{\Lambda_{\interval}}
\newcommand{\LambdaSymbolic}{\Lambda_{\text{sym}}}

\newcommand{\toIntervalTerm}[1]{#1^{2\interval}}

%Others
\newcommand{\Sset}{\mathbb{S}}
\newcommand{\Iset}{\mathbb{I}}
\newcommand{\Rset}{\mathbb{R}}
\newcommand{\Nset}{\mathbb{N}}
\newcommand{\Zset}{\mathbb{Z}}

\newcommand{\Term}{\mathbb{T}}
\newcommand{\prob}{\mathbb{P}}
\newcommand{\expt}{\mathbb{E}}


\newcommand{\Leb}{\tt{Leb}}
\newcommand{\Red}{\tt{Red}}
\newcommand{\cost}{\text{cost}}

%\newcommand{\intervalab}[2]{\underline{[#1,#2]}}
\newcommand{\intervalab}{\underline{[a,b]}}
\newcommand{\interI}{\mathcal{I}}
\newcommand{\trans}{\mathcal{T}}

\newcommand{\iv}{\mathbb{I}}

% Programming language constructs
\newcommand{\lit}[1]{\underline{#1}}
\newcommand{\letIn}[1]{\mathsf{let}\,{#1}\,\mathsf{in}\,}
\newcommand{\fixLam}[2]{\mu {#1} {#2}.}
\newcommand{\ifElse}[3]{\mathsf{if} (#1 \le \num{0}) \, {#2} \,\mathsf{else}\, {#3}}

%%Basic notions
\newcommand{\pspace}{(\Omega,\mathcal{F},\probm)}
\newcommand{\probm}{\mathbb{P}}
\newcommand{\condexpv}[2]{{\expt}{\left[{#1} \mid {#2}\right]}}

\newcommand{\stdConf}[1]{(#1)}
%\newcommand{\intConf}[1]{\lceil#1\rceil}
%\newcommand{\intConf}[1]{(#1)}
%\newcommand{\symConf}[1]{\langle\!\langle  #1 \rangle\!\rangle}
%\newcommand\symPath[1]{(#1)}
\newcommand{\symPath}[1]{\langle\!\langle  #1 \rangle\!\rangle}
\newcommand\symConf[1]{(#1)}

\newcommand{\ifSimple}[3]{\mathsf{if}(#1, #2, #3)}
%\newcommand{\ifElse}[3]{\mathsf{if} (#1 \le 0) \, \allowbreak {#2} \, \allowbreak \mathsf{else}\, {#3}}
%\newcommand{\ifElse}[3]{\ifSimple{#1}{#2}{#3}}

%\newcommand{\trace}{\mathsf{s}}
%
%\newcommand\defn[1]{{\bf \em #1}}
\newcommand{\traces}{\mathbb{T}}
%
%\newcommand{\stdConf}[1]{(#1)}
%%\newcommand{\intConf}[1]{\lceil#1\rceil}
%\newcommand{\intConf}[1]{(#1)}
%%\newcommand{\symConf}[1]{\langle\!\langle  #1 \rangle\!\rangle}
%%\newcommand\symPath[1]{(#1)}
%\newcommand{\symPath}[1]{\langle\!\langle  #1 \rangle\!\rangle}
%\newcommand\symConf[1]{(#1)}

\newcommand{\valueSem}[1]{\mathsf{val}_{#1}} % value (semantics)
\newcommand{\weightSem}[1]{\mathsf{wt}_{#1}} % weight (semantics)
\newcommand{\measureSem}[1]{\llbracket #1 \rrbracket}
\newcommand{\posterior}{\mathsf{posterior}}


%%%%%%%%%
% 
%%%%%%%%
\newcommand{\loc}{\ell}
\newcommand{\locs}{\mathit{L}}
\newcommand{\blocs}{\mathit{L}_{\mathrm{b}}}

\newcommand{\iflocs}{\mathit{L}_{\mathrm{if}}}
\newcommand{\looplocs}{\mathit{L}_{\mathrm{while}}}

\newcommand{\alocs}{\mathit{L}_{\mathrm{a}}}
\newcommand{\wlocs}{\mathit{L}_{\mathrm{w}}}
\newcommand{\rlocs}{\mathit{L}_{\mathrm{r}}}
\newcommand{\Alocs}[1]{\mathit{L}_{\mathrm{A}}^{\mathsf{#1}}}
\newcommand{\Dlocs}{\mathit{L}_{\mathrm{nd}}}
\newcommand{\transitions}{{\rightarrow}}

%%% 
\newcommand{\plocs}{\mathit{L}_{\mathrm{p}}}
\newcommand{\tlocs}{\mathit{L}_{\mathrm{t}}}

\newcommand{\lin}{\loc_\mathrm{init}}
\newcommand{\lout}{\loc_\mathrm{out}}
\newcommand{\val}[1]{\mbox{\sl Val}_{#1}}

\newcommand{\pvars}{V_\mathrm{p}}
\newcommand{\rvars}{V_{\mathrm{r}}}
\newcommand{\pre}{\mathrm{pre}}

\newcommand{\sle}{\sqsubseteq}
\newcommand{\sge}{\sqsupseteq}

\newcommand{\lfp}{\mathrm{lfp}}
\newcommand{\gfp}{\mathrm{gfp}}

\newcommand{\rdvarjdis}{\mathcal D}
\newcommand{\sampset}{\textit{supp}}

\newcommand{\upd}{\mbox{\sl upd}}
\newcommand{\wet}{\mbox{\sl wt}}
\newcommand{\transset}{\mathfrak T}
\newcommand{\valin}{\pv_{\mathrm{init}}}
\newcommand{\ret}{\mbox{\sl ret}}

\newcommand{\win}{w_{\mathrm{init}}}

\newcommand{\sampdpd}{\overline{\Upsilon}}

\newcommand{\outmap}{\text{O}}
\newcommand{\sat}[1]{\langle #1 \rangle}
\newcommand{\monoid}{\mbox{\sl Monoid}}
\newcommand{\handelmanformat}{(\dagger)}

\newcommand{\trunc}{\mathcal{B}}

\newcommand{\ewt}{\mbox{\sl ewt}}
\newcommand{\statemap}{\text{St}}

\newcommand{\valrd}{{\mathbf{r}}}
\newcommand{\frmloc}{\ell^{\mathrm{src}}}
\newcommand{\toloc}{\ell^{\mathrm{dst}}}

\newcommand{\monomials}{\mathbf{M}}

\renewcommand{\paragraph}[1]{\smallskip\noindent {\em #1}}  

\newtheorem{remark}{Remark}

% Used for displaying a sample figure. If possible, figure files should
% be included in EPS format.
%
% If you use the hyperref package, please uncomment the following two lines
% to display URLs in blue roman font according to Springer's eBook style:
%\usepackage{color}
%\renewcommand\UrlFont{\color{blue}\rmfamily}
%
\begin{document}
%
\title{Template-Based Static Posterior Inference for Bayesian Probabilistic Programming}
\titlenote{This work was originally posted to HAL. \url{https://hal.science/hal-04022797}}

\author{Peixin Wang}
\affiliation{%
  \institution{University of Oxford}
  \country{United Kingdom}
  }
\email{peixin.wang@cs.ox.ac.uk}

\author{Hongfei Fu}
\affiliation{%
  \institution{Shanghai Jiao Tong University}
  \country{China}
  }
\authornote{Corresponding author}
\email{fuhf@cs.sjtu.edu.cn}

\author{Tengshun Yang}
\affiliation{%
  \institution{SKLCS, Institute of Software, Chinese Academy of
Sciences \& University of Chinese Academy of Sciences}
  \country{China}
  }
\email{yangts@ios.ac.cn}

\author{Guanyan Li}
\affiliation{%
  \institution{University of Oxford}
  \country{United Kingdom}
  }
\email{guanyan.li@cs.ox.ac.uk}

\author{Luke Ong}
\affiliation{%
  \institution{University of Oxford, United Kingdom \& Nanyang Technological University}
  \country{Singapore}
  }
\email{luke.ong@ntu.edu.sg}


%
%\titlerunning{Abbreviated paper title}
% If the paper title is too long for the running head, you can set
% an abbreviated paper title here
%
%\author{First Author\inst{1}\orcidID{0000-1111-2222-3333} \and
%Second Author\inst{2,3}\orcidID{1111-2222-3333-4444} \and
%Third Author\inst{3}\orcidID{2222--3333-4444-5555}}
%%
%\authorrunning{F. Author et al.}
% First names are abbreviated in the running head.
% If there are more than two authors, 'et al.' is used.
%
%\institute{Princeton University, Princeton NJ 08544, USA \and
%Springer Heidelberg, Tiergartenstr. 17, 69121 Heidelberg, Germany
%\email{lncs@springer.com}\\
%\url{http://www.springer.com/gp/computer-science/lncs} \and
%ABC Institute, Rupert-Karls-University Heidelberg, Heidelberg, Germany\\
%\email{\{abc,lncs\}@uni-heidelberg.de}}
%
           % typeset the header of the contribution
%
%\vspace{-8ex}
\begin{abstract}

The Fast Reciprocal Square Root Algorithm is a well-established approximation technique consisting of two stages: first, a coarse approximation is obtained by manipulating the bit pattern of the floating point argument using integer instructions, and second, the coarse result is refined through one or more steps, traditionally using Newtonian iteration but alternatively using improved expressions with carefully chosen numerical constants found by other authors. The algorithm was widely used before microprocessors carried built-in hardware support for computing reciprocal square roots. At the time of writing, however, there is in general no hardware acceleration for computing other fixed fractional powers. This paper generalises the algorithm to cater to all rational powers, and to support any polynomial degree(s) in the refinement step(s), and under the assumption of unlimited floating point precision provides a procedure which automatically constructs provably optimal constants in all of these cases. It is also shown that, under certain assumptions, the use of monic refinement polynomials yields results which are much better placed with respect to the cost/accuracy tradeoff than those obtained using general polynomials. Further extensions are also analysed, and several new best approximations are given.

\end{abstract}

%\begin{abstract}
%The abstract should briefly summarize the contents of the paper in
%150--250 words.
%
%\keywords{probabilistic programs  \and Bayesian inference \and integrability.}
%\end{abstract}
%
%
%
\maketitle   


\section{Introduction}
\section{Introduction}
Current quantum hardware is unable to carry out universal quantum computations due to the buildup of errors that occur during the computation. 
The magnitude of the individual error is currently above the value that the Threshold Theorem requires in order to kick-start quantum error correction and fault-tolerant quantum computation~\cite[Section 10.6]{nielsen_chuang_2010}. 
Although the experimentally achieved fidelity rates are promising and the error bounds are inching closer to the required threshold, we will have to work for the foreseeable future with quantum hardware with errors that build-up during the computation.  This implies that we can only do a limited number of steps before the output of the computation has become completely uncorrelated with the intended one.

For fault-tolerant quantum computing, we repeat four steps: 
1) We apply a number of single and two-qubit quantum gates, in parallel whenever possible; 
2) We perform a syndrome measurement on a subset of the qubits; 
3) We perform fast classical computations to determine which errors have occurred and how to correct them; 
and, 4) We apply correction terms based on the classical computations.
We then repeat these four steps with a next sequence of gates. 
These four steps are essential to fault-tolerant quantum computing. 


The starting point of this work is to use the four steps outlined above, not to carry out error correction and fault-tolerant computation, but to enhance short, constant-depth, {\em uncorrected} quantum circuits that perform single qubit gates and {\em nearest-neighbor} two qubit gates. 
Since in the long run we will have to implement error-correction and fault-tolerant computation anyhow, and this is done by such a four-step process, why not make other use of this architecture? Moreover, on some of the quantum hardware platforms, these operations are already in place.
Embracing this idea we naturally arrive at the question: what is the computational power of \textit{low-depth} quantum-classical circuits organized as in the four steps outlined above? 
We thus investigate circuits that execute a small, ideally constant, number of stages, where at each stage we may apply, in parallel, single qubit gates and {\em nearest-neighbor} two qubit gates, followed by measurements, followed by low-depth classical computations of which the outcome can control quantum gates in later stages. 
It is not clear, at first, whether such circuits, especially with constant depth, can do anything remotely useful. 
But we will see that this is indeed the case: many quantum computations can be done by such circuits in constant depth. 
By parallelizing quantum computations in this way, we improve the overall computational capabilities of these circuits, as we do not incur errors on qubits that are idle, simply because qubits are not idle for a very long time. 
Furthermore, reducing the depth of quantum circuits, at the cost of increasing width, allows the circuit to be run faster even if errors occur.

The first usage of such a four-step layout, not to do error correction, but to perform computations, can be found in the paradigm of measurement-based quantum computing~\cite{gottesman1999demonstrating,raussendorf2001one,jozsa2006introduction,clark2007generalised}: 
A universal form of quantum computing where a quantum state is prepared and operations are performed by measuring qubits in different bases, depending on previous measurements and intermediate measurements.

\citeauthor{PhamSvore2013} were the first to formalize the four-step protocol for performing computations~\cite{PhamSvore2013}. They included specific hardware topologies by considering two-dimensional graphs for imposing constraints on qubit interactions. In their model, they develop circuits for particularly useful multi-qubit gates, including specifying costs in the width, number of qubits, depth, number of concurrent time steps, size, and total number of non-Identity operations.
As a result, they find an algorithm that factors integers in polylogarithmic depth.
\citeauthor{Browne:2011} showed that the main tool in the work by \citeauthor{PhamSvore2013}, the fan-out gate, can also be replaced by additional log-depth classical computations in the measurement-based quantum computing setting~\cite{Browne:2011}.

More recently, \citeauthor{Cirac:2021} introduced a scheme to implement unitary operations involving quantum circuits combined with Local Operations and Classical Communication ($\mathsf{LOCC}$) channels: $\mathsf{LOCC}$-assisted quantum circuits~\cite{Cirac:2021}. Similarly to the four-step scheme we just described, they allow for a short depth circuit to be run on the qubits, followed by one round of $\mathsf{LOCC}$, in which ancilla qubits are measured and local unitaries are applied based on the measurement outcomes. They show that in this model any 1D transitionally invariant matrix-product state (MPS) with fixed bond dimension is in the same phase of matter as the trivial state. Similar ideas can be found in~\cite{TVV_NonAbelianTopologicalOrder_2022, tantivasadakarn2021long}.

In this work, we introduce a new model, called \textit{Local Alternating Quantum-Classical Computations} ($\LAQCC$). In this model we alternate between running quantum circuits (constrained by locality), ending in the measurement of a subset of qubits, and fast classical computations based on the measurement results. The outcome of the classical computations are then used to control future quantum circuits. We allow for flexibility in this model, by giving different constraints to the power of both the quantum circuits and the classical circuits as well as the number of alternations between them. 
Most attention will be given to $\LAQCC$ containing quantum circuits of constant depth, classical circuits of logarithmic depth and at most a constant number of alternations between them. 
Any circuit constructed in this model is considered to be of constant depth. 
We restrict ourselves to logarithmic depth classical computations, as this is the first natural and non-trivial extension beyond constant-depth classical computations. 
Constant-depth classical computations do however also have an equivalent constant-depth quantum implementation.

The definition of $\LAQCC$ sharpens the original definition of \citeauthor{PhamSvore2013} by adding constraints to the intermediate classical computations. This allows us to bound the power of $\LAQCC$ from above. 

The main result of \citeauthor{Cirac:2021}, that 1D translational invariant MPS with fixed bond dimension can be prepared by $\mathsf{LOCC}$-assisted circuits, relies on local symmetries of the MPS. These symmetries allow them to prepare local states (on a constant number of qubits) and glue them together by doing one round of the appropriate entangling measurement and corrections, after which they run a round of local unitaries to get the desired result. This general scheme for preparing states that exhibit an MPS description with the appropriate local symmetries requires only geometrically local unitaries and one round of measurement and corrections an therefore is accessible in $\LAQCC$. Studying different local symmetries, known as Symmetry Protected Topological (SPT) phases of matter, to find measurement-based constant depth circuits for states is a broad ongoing field of research~\cite{TVV_NonAbelianTopologicalOrder_2022, tantivasadakarn2021long, smith2023deterministic}. 
All these schemes have a $\LAQCC$ implementation.

%$\LAQCC$-circuits also exist for general schemes of preparing local states, based on the local tensors, and gluing them together using one round of entangled measurement and corrections, based on the local symmetry. 
%The main result of \citeauthor{Cirac:2021}, that 1D translational invariant MPS with fixed bond dimension can be prepared by $\mathsf{LOCC}$-assisted circuits, relies heavily on local symmetries of the MPS and as a result also has an equivalent $\LAQCC$ implementation. 
%The corrections applied after the measurement round are local unitaries depending on the local symmetries of the MPS. 

 

%This general scheme of preparing local states, based on the local tensors, and gluing it together by doing one round of entangled measurement and corrections, based on the local symmetry, is accessible in $\LAQCC$.
Note however that \citeauthor{Cirac:2021} also suggest a circuit for the $W$-state.
This circuit uses sequentially and dependent measurement-based corrections of the ancilla qubits. 
These dependent measurements translate to sequential alternations between the quantum and classical circuits and therefore increase the total depth to linear depth, exceeding the constant-depth constraints imposed by $\LAQCC$-circuits. 

We study the power of the $\LAQCC$ model with respect to state preparation, showing that even with only constant quantum-depth and logarithmic classical depth it remains possible to prepare states with long-range entanglement.
Another surprising result is that it is unlikely that $\LAQCC$ circuits are classically simulatable. We show that any instantaneous quantum polynomial-time (IQP) circuit~\cite{Bremner2010,Shepherd2009} has an $\LAQCC$ implementation.
Classical simulation of IQP circuits implies the collapse of the polynomial hierarchy to the third level, which is not believed to be true~\cite{Bremner2017}. Therefore, we expect that $\LAQCC$ circuits are unlikely to be classically simulatable. We bound the power of $\LAQCC$ by showing that it is contained in $\QNC^1$, the class of polynomial-size, log-depth circuits.

Next, we also study the power that intermediate classical calculations can add to quantum computations, by considering a new model that alternates between polynomially many polynomial-depth quantum circuits and unbounded classical computations
We study this model by doing a complexity theoretical analysis, where we draw inspiration from the notions of complexity given by \citeauthor{RosenthalYuen:2022}, \citeauthor{MetgerYuen:2023}, and \citeauthor{Aaronson:2004}.
All three complexity notions are based on the notion of state preparation, instead of more traditional definition of complexity such as the decidability of a computational problem. 
The first two consider classes based on sequences of quantum states preparable by a polynomial-sized quantum circuit, where the circuits are uniformly generated by a computational class, for instance, the class $\mathsf{PSPACE}$, which results in the complexity class $\mathsf{StatePSPACE}$~\cite{RosenthalYuen:2022,MetgerYuen:2023}.
The third notion considers a relative complexity, where the complexity is measured between two given states, and is measured by the number of gates, from a given gate-set, required to transform one state in another state~\cite{Aaronson:2004}. 
For our definition of state preparation complexity, we drop the uniformity constraint from~\cite{RosenthalYuen:2022,MetgerYuen:2023} and define a class as $\mathsf{StateX}$, which refers to states preparable by circuits of type $\mathsf{X}$. 
As an example, if $\mathsf{X} = \QNC^0$, this results in the class $\mathsf{StateQNC^0}$, which is the set of states preparable from the $\ket{0}^n$ state by poly-size constant-depth circuits. 
This notion is similar to the relative complexity from~\cite{Aaronson:2004}, where one state is the  $\ket{0}^n$ state and instead of counting the number of gates we consider the set of states preparable by a fixed number of gates. Using this notion of complexity we show that any state preparable by an $\LAQCC^*$ circuit is also preparable by a $\mathsf{PostQPoly}$ circuit, the class of circuits of polynomial depth with an additional post-selection gate. 

All Clifford circuits have a constant-depth $\LAQCC$ implementation, implying that any stabilizer state can be implemented by a constant-depth $\LAQCC$ circuit, see Section~\ref{sec:clifford_circuits} for a proof of this statement. 
Efficient circuits for stabilizer states have been known already through measurement-based quantum computing. Therefore this paper focuses on the preparation of non-stabilizer states, and as a surprising result we find novel constant-depth protocols for four very natural classes of non-stabilizer states.
Despite the extensive research into these four classes of non-stabilizer states and the many applications of them, no efficient constant- or low-depth state preparation protocols are known yet. We specifically consider these four classes as they are all often used as initial states in other algorithms.

The first state is a uniform superposition over an arbitrary number of states. 
This state finds applications in many quantum algorithms, as they often start with a uniform superposition over multiple states. 
This superposition is often achieved by applying Hadamard gates to every qubit due to its simplicity to prepare. 
Yet, the analysis of many algorithms, such as Shor's algorithm~\cite{Shor:1997}, would benefit from a different initial superposition. 
The circuit to prepare the uniform superposition over an arbitrary number of states uses an exact version of Grover search as a subroutine, that turns a probabilistic circuit, with a known constant probability of success, into a deterministic circuit. 
We use the circuit for preparing a uniform superposition over an arbitrary number of states as a subroutine in the next two quantum state preparation protocols. 

The second state is the $W$-state, the uniform superposition over all computational basis states of Hamming-weight~$1$, a natural long-ranged entangled state that displays a fundamentally nonequivalent type of entanglement from the Greenberger–Horne–Zeilinger state~\cite{WState:2000}, for which $\LAQCC$-type constant-depth circuits were previously known~\cite{PhamSvore2013, Cirac:2021}. 
The $W$-state is often used as benchmark for new quantum hardware~\cite{Haffner2005,Neeley2010,GarciaPerez:2021}. 
A novel way to prepare the $W$-state therefore gives a new way to benchmark different quantum devices with each other. 
A circuit for preparing the $W$-state was given in~\cite{Cirac:2021}, but this implementation requires sequentially alternating measurements followed by local unitaries, which in the $\LAQCC$ model is not considered to be of constant depth. 
We improve this protocol by giving an $\LAQCC$ implementation of the $W$-state, based on a compress-uncompress method that links the one-hot and binary encoding of integers.

The third state considered is the Dicke state, a generalization of the $W$-state, a superposition over all computational basis states with Hamming-weight $k$~\cite{Dicke:1954}. 
Dicke states have relevance in various practical settings.
For instance, for quantum game theory~\cite{zdemir2007}, quantum storage~\cite{Bacon_Compress:2006,Plesch:2010}, quantum error correction~\cite{ouyang2014permutation}, quantum metrology~\cite{toth2012multipartite}, and quantum networking~\cite{prevedel2009experimental}. 
Dicke states have been used as a starting state for variational optimization algorithms, most notably Quantum Alternating Operator Ansatz (QAOA)~\cite{Hadfield2019}, to find solutions to problems such as Maximum k-vertex Cover~\cite{Brandhofer2022,cook2020quantum}.
The ground states of physical Hamiltonians describing one-dimensional chains tend to show a resemblance to Dicke states such as states resulting from the Bethe ansatz, making them an ideal starting state when investigating the ground state behavior of these Hamiltonians~\cite{TDL_BetheAnsatzDerivation:2010,B_ExcitedStateQuantumPhaseTransitions:2013,DickeTransitions:2021}. 
For instance, the algorithm by \citeauthor{van2021preparing}, who give an algorithm to prepare the Bethe ansatz eigenstates of the spin-1/2 XXZ spin chain, starts by first preparing a Dicke state~\cite{van2021preparing}. 
A Dicke-state preparation protocol based on the compress-uncompress methodology used in the $W$-state furthermore finds applications in entanglement distillation, where the entanglement of a large state is concentrated on only a few qubits. 
Efficient deterministic circuits for preparing Dicke states have been proposed by \citeauthor{bartschi2019deterministic}~\cite{bartschi2019deterministic, bartschi2022deterministic_short_depth}. 
They provide a quantum circuit of depth $\mathO(k \log(\frac{n}{k}))$, allowing arbitrary connectivity, to prepare a Dicke state, which they conjecture to be optimal when $k$ is constant. 
In this work, we provide a constant-depth $\LAQCC$ circuit below their conjectured bound already for constant $k$. 
However, this does not directly disprove their conjecture, as we allow for intermediate measurements and classical computations. 
More significantly, we even construct constant-depth $\LAQCC$ circuits for $k = \mathO(\sqrt{n})$ greatly improving their bound.
This construction extends the compress-uncompress method for the $W$-state combined with additional subroutines. 

We continue with a log-depth state preparation protocol for the Dicke-state for arbitrary $k$. 
This protocol implements an efficient transformation between the factoradic number representation and the combinatorial number representation of a positive integer. 
The combinatorial number representation relates directly to the Dicke state. 
The provided efficient transformation between number representation systems might be of independent interest. 

We conclude by modifying our protocol for preparing a Dicke-state to a protocol that prepares quantum many-body scar states in constant-depth. 
These states have low entanglement and longer coherence times than states with similar energy density.
These characteristics make many-body scar states interesting to analyze and relevant within physics.
Many-body scar states appear for instance in the AKLT model~\cite{AKLT:1987,MRBAR:2018,MRB:2018} and different spin models~\cite{SI:2019,MOBFR:2020}.
Known methods for preparing these states have polynomial-depth~\cite{Gustafson:2023}, whereas our circuit has constant depth. 

% We conclude by studying the power that intermediate classical calculations can add to quantum computations. 
% In this study, we define a new model that relaxes constant-depth quantum circuits to polynomial depth quantum circuits, log-depth classical calculations to unbounded classical computations and a constant number of alternations to a polynomial number of alternations. 
% We call this model $\LAQCC^*$. 
% We study this model by doing a complexity theoretical analysis, where we draw inspiration from the notions of complexity given by \citeauthor{RosenthalYuen:2022}, \citeauthor{MetgerYuen:2023}, and \citeauthor{Aaronson:2004}.
% All three complexity notions are based on the notion of state preparation, instead of more traditional definition of complexity such as the decidability of a computational problem. 
% The first two consider classes based on sequences of quantum states preparable by a polynomial-sized quantum circuit, where the circuits are uniformly generated by a computational class, for instance, the class $\mathsf{PSPACE}$, which results in the complexity class $\mathsf{StatePSPACE}$~\cite{RosenthalYuen:2022,MetgerYuen:2023}.
% The third notion considers a relative complexity, where the complexity is measured between two given states, and is measured by the number of gates, from a given gate-set, required to transform one state in another state~\cite{Aaronson:2004}. 
% For our definition of state preparation complexity, we drop the uniformity constraint from~\cite{RosenthalYuen:2022,MetgerYuen:2023} and define a class as $\mathsf{StateX}$, which refers to states preparable by circuits of type $\mathsf{X}$. 
% As an example, if $\mathsf{X} = \QNC^0$, this results in the class $\mathsf{StateQNC^0}$, which is the set of states preparable from the $\ket{0}^n$ state by poly-size constant-depth circuits. 
% This notion is similar to the relative complexity from~\cite{Aaronson:2004}, where one state is the  $\ket{0}^n$ state and instead of counting the number of gates we consider the set of states preparable by a fixed number of gates. Using this notion of complexity we show that any state preparable by an $\LAQCC^*$ circuit is also preparable by a $\mathsf{PostQPoly}$ circuit, the class of circuits of polynomial depth with an additional post-selection gate. 

\paragraph{Summary of results}
\begin{itemize}
    \item We give a new definition of a computational model that captures the power of the four step process: applying a constant number of layers of one- and two-qubit gates; performing a syndrome measurement; perform a fast classical computation determining corrections; apply corrections. We call this model \emph{Local Alternating Quantum Classical Computations}, or $\LAQCC$ for short. In this model we bound the allowed quantum operations, intermediate classical calculations, and number of rounds separately. In Section~\ref{sec:LAQCC_model} we define this model and give a list of operations based on results from literature contained in this computational model. In some of these operations we explicitly use that we allow for multiple, but at most constant, rounds  of corrections.
    \item  We show show that there exist $\LAQCC$ circuits that can not be weakly simulated in Section~\ref{sec:IQP_in_LAQCC}. We further show that for every $\LAQCC$ circuit there exists a $\QNC^1$ circuit simulating it perfectly, in Section~\ref{sec:LAQCC_in_QNC1}.
    \item We introduce a new type computational complexity for preparing states and show that the extension of $\LAQCC$ where we allow a polynomial number of rounds and unbounded classical computation, is contained in $\mathsf{PostQPoly}$, the class of polynomial circuits with post-selection, in Section~\ref{sec:Complexity results}.
    \item We show a protocol to prepare the uniform superposition state of size $q$ in $\LAQCC$ using $\mathO(\ceil{\log_2(q)}^2)$ qubits in Section~\ref{sec:superposition_modulo_q}. 
    \item We show a protocol to prepare the $W_n$ state in $\LAQCC$ using $\mathO(n\log(n))$ qubits in Section~\ref{sec:W_state_in_LAQCC}.
    \item We show two ways of preparing the Dicke-$(n,k)$ state. The first method is in $\LAQCC$, works up to $k = \mathO(\sqrt{n})$, uses $\mathO(n^2\log(n))$ qubits, and is found in Section~\ref{sec:dicke:small_k}. The second method is in $\LAQCC\text{-}\mathsf{LOG}$ (an extension of $\LAQCC$ allowing for logarithmic number of alterations instead of constant), works for any $k$, uses $\mathO(\text{poly}(n))$ qubits, and is found in Section~\ref{sec:Dicke_in_LAQCC_LOG}. 
    \item We extend on our $\LAQCC$ method of generating Dicke-$(n,k)$ states for $k = \mathO(\sqrt{n})$ and show a protocol to generate many-body scar states for a particular Hamiltonian in $\LAQCC$ (Section~\ref{sec:many_body_scar}). 
\end{itemize}
Summarized in a table, we provide the following state generation protocols:
\begin{table}[htb]
\centering
\begin{tabular}{l|l|l|l}
\textbf{State description} & \textbf{Width} & \textbf{Depth} & \textbf{Implementation}\\
\hline 
Uniform superposition mod $q$: $\frac{1}{\sqrt{q}} \sum_{i = 0}^{q-1}\ket{i}$ & $\mathO(\ceil{\log^2 q})$ & $\mathO(1)$ & Section~\ref{sec:superposition_modulo_q}\\

$W$-state: $\frac{1}{\sqrt{n}}\sum_{i = 0}^{n-1}\ket{e_i}$ & $\mathO(n \log n)$ & $\mathO(1)$ & Section~\ref{sec:W_state_in_LAQCC}\\

Dicke-$(n,k)$, $k = \mathO(\sqrt{n})$: $\binom{n}{k}^{-1/2}\sum_{x \in \{0,1\}^n: |x| = k} \ket{x}$ &  $\mathO(n^2\log n)$ & $\mathO(1)$ 
&Section~\ref{sec:dicke:small_k}\\

Dicke-$(n,k)$: $\binom{n}{k}^{-1/2}\sum_{x \in \{0,1\}^n: |x| = k} \ket{x}$ & $\mathO(\text{poly}(n))$ & $\mathO(\log n)$ &Section~\ref{sec:Dicke_in_LAQCC_LOG}\\

QMBS: $\ket{S_k} = \frac{1}{k! \sqrt{\mathcal N(n,k)}}(Q^\dagger)^k \ket{\Omega}$ &  $\mathO(n^2\log n)$ & $\mathO(1)$  &  Section~\ref{sec:many_body_scar}
\end{tabular}
\caption{Summary of state preparation protocols given in this paper.}
\label{tab:sate_prep}
\end{table}
In the entry for the quantum many-body scar state $Q$ denotes the raising operator and $\mathcal N(n,k)=\binom{n-k-1}{k}$. 
Section~\ref{sec:many_body_scar} will provide more details on the variables and the implementation. 

\paragraph{Organization of the paper}
\noindent We first introduce relevant preliminaries in Section~\ref{sec:preliminaries}. 
In Section~\ref{sec:LAQCC_model} we formally define the class of Local Alternating Quantum-Classical Computations ($\LAQCC$). We also show that any Clifford circuit can be implemented in constant depth $\LAQCC$ (a result based on a result from measurement-based quantum computing~\cite{jozsa2006introduction}). 
This result allows us to give many useful multi-qubit gates and routines in Section~\ref{sec:gates_created_in_LAQCC}. 
Beyond that we show that constant depth $\LAQCC$ circuits are contained in $\QNC^1$ and that any $\mathsf{IQP}$ circuit has an $\LAQCC$ implementation.
We conclude this section with an analysis of a more powerful instantiation of $\LAQCC$ and show an inclusion with respect to the class $\mathsf{PostQPoly}$, which is the class of circuits of polynomial depth with one additional post-selection gate. 
In Section~\ref{sec:state_prep_in_LAQCC} we give $\LAQCC$ circuit implementations for preparing the uniform superposition over an arbitrary number of states, the $W$-state and the Dicke state up to $k = \mathO(\sqrt{n})$. We furthermore give a log-depth circuit implementation for preparing the Dicke state for any $k$. We conclude by showing a $\LAQCC$ circuit for generating many body scar states of a particular type of Hamiltonian.




\section{Preliminaries}\label{sec:prelim}
We first review some basic concepts from probability theory (see standard textbooks such as \cite{pollard2002user,williams1991probability} for a detailed treatment), 
%the background of Bayesian inference, and finally 
%We first review some basic concepts from probability theory, 
and then present the Bayesian probabilistic programming language and the normalised posterior distribution (NPD) problem.
%we consider in this work. 
Throughout the paper,
we denote by $\Nset$, $\Zset$ and $\Rset$ the sets of all natural numbers (including zero), integers, and real numbers, respectively.

\vspace{-1.5ex}
\subsection{Basics of Probability Theory}
%We assume familiarity with basic probability theory (see \cref{app:prelim} for details). 

A \emph{measurable space} is a pair $(U,\Sigma_U)$, where $U$ is a nonempty set and $\Sigma_U$ is a $\sigma$-algebra on $U$, i.e., a family of subsets of $U$ such that $\Sigma_U\subseteq \mathcal{P}(U)$ contains $\emptyset$ and is closed under complementation and countable union. Elements of $\Sigma_U$ are called \emph{measurable} sets. A function $f$ from a measurable space $(U_1,\Sigma_{U_1})$ to another measurable space $(U_2,\Sigma_{U_2})$ is \emph{measurable} if $f^{-1}(A)\in\Sigma_{U_1}$ for all $A\in\Sigma_{U_2}$.

A \emph{measure} $\mu$ on a measurable space $(U,\Sigma_U)$ is a mapping from $\Sigma_U$ to $[0,\infty]$ such that (i) $\mu(\emptyset)=0$ and (ii) $\mu$ 
%satisfies the
is countably additive:
%condition: 
for every pairwise-disjoint set sequence $\{A_n\}_{n\in\Nset}$ in $\Sigma_U$, it holds that $\mu(\bigcup_{n\in\Nset}A_n)=\sum_{n\in\Nset}\mu(A_n)$. We call the triple $(U,\Sigma_U,\mu)$ a \emph{measure space}. 
%If $\mu(U)\le 1$, we call $\mu$ a \emph{subprobability measure}. 
If $\mu(U)=1$, we call $\mu$ a \emph{probability measure}, and $(U,\Sigma_U,\mu)$ a \emph{probability space}.
The Lebesgue measure $\lambda$ is the unique measure on $(\Rset,\Sigma_{\Rset})$ satisfying $\lambda([a,b))=b-a$ for all valid intervals $[a,b)$ in $\Sigma_{\Rset}$. For each $n\in\Nset$, we have a measurable space $(\Rset^n,\Sigma_{\Rset^n})$ 
%such that there exists 
and
a unique product measure $\lambda_n$ on $\Rset^n$ satisfying $\lambda_n(\prod_{i=1}^n A_i)=\prod_{i=1}^n \lambda(A_i)$ for all $A_i\in\Sigma_{\Rset}$.


The \emph{Lebesgue} integral operator $\int$ is a partial operator that maps a measure $\mu$ on $(U,\Sigma_U)$ and a real-valued function $f$ on the same space $(U,\Sigma_U)$ to a real number or infinity, which is denoted by $\int f \mathrm{d}\mu$ or $\int f(x)\mu(\mathrm{d}x)$. 
The detailed definition of Lebesgue integral is somewhat technical, see \cite{rankin1968real,rudin1976principles} for more details. 
Given a measurable set $A\in\Sigma_U$, the integral of $f$ over $A$ is defined by $\int_A f(x)\mu(\mathrm{d} x):=\int f(x) \cdot [x\in A] \mu(\mathrm{d}x)$
%\begin{align*}
%\textstyle\int_A f(x)\mu(\mathrm{d} x):=\int f(x) \cdot [x\in A] \mu(\mathrm{d}x)
%\end{align*} 
where $[-]$ is the Iverson bracket such that $[\phi]=1$ if 
%the predicate 
$\phi$ is true, and $0$ otherwise. If $\mu$ is a probability measure, then we call the integral as the \emph{expectation} of $f$, denoted by $\expectdist{x\sim\mu;A}{f}$, or $\expv[f]$ when the scope is clear from the context.

For a measure $v$ on $(U,\Sigma_U)$, a measurable function $f:U\to \Rset_{\ge 0}$ is the \emph{density} of $v$ with respect to $\mu$ if $v(A)=\int f(x)\cdot [x\in A] \mu(\mathrm{d} x)$ for all measurable $A\in\Sigma_U$, and $\mu$ is called the \emph{reference measure} (most often $\mu$ is the Lebesgue measure). Common families of probability distributions on the reals, e.g., uniform, normal distributions, are measures on $(\Rset,\Sigma_{\Rset})$. Most often these are defined in terms of probability density functions with respect to the Lebesgue measure. That is, for each $\mu_D$ there is a measurable function $\text{pdf}_D:\Rset\to\Rset_{\ge 0}$ that determines it: $\mu_D(A):=\int_A \text{pdf}_D (\mathrm{d}\lambda) $. As we will see, density functions such as $\text{pdf}_D$ play an important role in Bayesian inference.

Given a probability space $\pspace$, a \emph{random variable} is an $\mathcal{F}$-measurable function $X: \Omega \rightarrow \Rset \cup \{+\infty,-\infty\}$. The expectation of a random variable $X$, denoted by $\expv(X)$, is the Lebesgue integral of $X$ w.r.t. $\probm$, i.e., $\int X\,\mathrm{d}\probm$. A \emph{filtration} of $\pspace$ is an infinite sequence $\{ \mathcal{F}_n \}_{n=0}^{\infty}$ such that for every $n\ge 0$, the triple $(\Omega, \mathcal{F}_n, \probm)$ is a probability space and $\mathcal{F}_n \subseteq \mathcal{F}_{n+1} \subseteq \mathcal{F}$. A \emph{stopping time} w.r.t. $\{ \mathcal{F}_n \}_{n=0}^{\infty}$ is a random variable $T: \Omega \rightarrow \Nset \cup \{0, \infty\}$ such that for every $n \geq 0$, the event \{$T \leq n$\} is in $\mathcal{F}_n$. 

A \emph{discrete-time stochastic process} is a sequence $\Gamma = \{X_n\}_{n=0}^\infty$ of random variables in $\pspace$. The process $\Gamma$ is \emph{adapted} to a filtration $\{ \mathcal{F}_n \}_{n=0}^{\infty}$, if for all $n \geq 0$, $X_n$ is a random variable in $(\Omega, \mathcal{F}_n, \probm)$. A discrete-time stochastic process $\Gamma=\{X_n\}_{n=0}^\infty$ adapted to a filtration $\{\mathcal{F}_n\}_{n=0}^\infty$ is a \emph{martingale} (resp. \emph{supermartingale}, \emph{submartingale})
if for all $n \geq 0$, $\expv(|X_n|)<\infty$ and it holds almost surely (i.e.,~with probability $1$) that
$\condexpv{X_{n+1}}{\mathcal{F}_n}=X_n$ (\mbox{resp. } $\condexpv{X_{n+1}}{\mathcal{F}_n}\le X_n$, $\condexpv{X_{n+1}}{\mathcal{F}_n}\ge X_n$).
See~\cite{williams1991probability} for details.
%Intuitively, a martingale is a discrete-time stochastic process, in which at any time $n$, the expected value $\condexpv{X_{n+1}}{\mathcal{F}_n}$ in the next step, given all previous values, is equal to the current value $X_n$. In a supermartingale, this expected value is less than or equal to the current value and a submartingale is defined conversely.
Applying martingales to qualitative and quantitative analysis of probabilistic programs is a well-studied technique~\cite{SriramCAV,ChatterjeeFG16,ChatterjeeNZ2017}.


\subsection{Bayesian Probabilistic Programming Language}

%We consider an imperative arithmetic probabilistic programming language. 
The syntax of our probabilistic programming language (PPL) is given in \cref{fig:syntax}, where the metavariables $S$, $B$ and $E$ stand for statements, boolean expressions and arithmetic expressions, respectively.   
Our PPL is imperative with the usual conditional and loop structures (i.e.,~\textbf{if} and \textbf{while}), as well as the following new structures: (a)~sample constructs of the form ``$\textbf{sample}\  D$'' that sample a value from a prescribed distribution $D$ over $\mathbb{R}$ and then assign this value to a sampling variable $r$; (b)~score statements of the form ``\textbf{score}($EW$)'' that weight the current execution with a value expressed by $EW$ (note that $\textit{pdf}(D,x)$ means the value of a probability density function w.r.t. $D$ at $x$);
%\footnote{Instead of the hard conditioning that refutes the execution when the observation mismatches the value of the sampling variable, we use the more general soft conditioning and assume the existence of a global weight variable initialized  to $1$.}
%for each program
(c)~probabilistic branching statements of the form
``$\textbf{if}\ \textbf{prob}(p)\dots$'' that lead to the then part with probability
$p\in (0,1]$ and to the else part with probability $1-p$. We also have sequential compositions (i.e., ";") and support return statements (i.e., \textbf{return}) that 
return the value of the program variable of interest. %The set of all statements is denoted by $Stmt$.
Note that $c,c_1,c_2\in\Rset$ are constants, and our language supports any distributions with continuous density functions and infinite supports, 
including but not limited to uniform and normal distributions. 



% Figure environment removed





Given a probabilistic program in our language, we distinguish two disjoint sets of variables in the program: (i) the set $\pvars$ of \emph{program variables} whose values are determined by assignments in the program (i.e., the expressions at the LHS of ``:="); (ii)~the set $\rvars$ of \emph{sampling variables} whose values are independently sampled from prescribed probability distributions each time they are accessed (i.e., each ``$\textbf{sample}\ D$" can be regarded as a sampling variable $r$). 




\begin{example}\label{ex:pedestrian-program}

%Consider the pedestrian random walk example~\cite{DBLP:conf/esop/MakOPW21}, a pedestrian is lost on a road, and she only knows that she is away from her house at most $3$ km. Thus, she starts to repeatedly walk a uniformly random distance of at most $1$ km in either direction, until reaching her house. Upon she arrives, an  odometer tells that she has walked $1.1$ km totally. However, this odometer was once broken and the measured distance is normally distributed around the true distance with standard deviation $0.1$ km. 
\cref{fig:pedestrian-program} shows a Bayesian probabilistic program written in our PPL language. In this program, the set of program variables is $\pvars=\{start,pos,dis,step\}$, and the set of sampling variables is $\rvars=\{ \textbf{sample uniform}(0,1)\}$. Each time $\textbf{sample uniform}(0,1)$ is executed, it samples a value uniformly from $[0,1]$ and then assigns the value to the variable $step$. 
%Thus, $step$ is associated with the probability distribution $\textbf{uniform}(0,1)$.
\qed


	
% Figure environment removed
\end{example}

\subsection{The Semantics of Our Programming Language}

%To relate variables with their values, we introduce the notion of valuations. 
Let $V$ be a finite set of variables with an implicit linear order over its elements. A \emph{valuation} on $V$ is a function $\pv: V \rightarrow \Rset$ that assigns a real value to each variable in $V$. We denote the set of all valuations on $V$ by $\val{V}$. For each $1\le i\le |V|$, we denote the value of the $i$-th variable (in the implicit linear order) in $\pv$ by $\pv[i]$, so that we can view each valuation as a real vector on $V$. A \emph{program} (resp. \emph{sampling}) valuation is a valuation on $\pvars$ (resp. $\rvars$), respectively. 
For the sake of convenience, we fix the notations in the following way, i.e., we always use $\pv\in\val{\pvars}$ to denote a program valuation, and $\rv\in\val{\rvars}$ to denote a sampling valuation; we also write $\pv[\mathit{ret}]$ for the value of the return variable in $\pv$. 



Below we present the semantics for our programming language. Existing semantics in the literature are either measure-\cite{DBLP:conf/lics/StatonYWHK16,LeeYRY20} or sampling-based  \cite{DBLP:conf/esop/MakOPW21,Beutner2022b}. To facilitate the development of our algorithm, we consider the \emph{transition-based} semantics~\cite{DBLP:conf/cav/ChakarovS13,DBLP:conf/popl/ChatterjeeFNH16} to our language and 
%To apply template-based algorithmic approaches to NPD problems, we consider  that 
treat each probabilistic program as a \emph{weighted probabilistic transition system} (WPTS). A WPTS extends a PTS  ~\cite{DBLP:conf/cav/ChakarovS13,DBLP:conf/popl/ChatterjeeFNH16} with weights and an initial probability distribution. 





%Below we present a variant of probabilistic transition systems \cite{DBLP:conf/cav/ChakarovS13}.
\begin{definition}
%[Weighted Probabilistic Transition Systems]
[WPTS]\label{def:wpts}
	A \emph{weighted probabilistic transition system} (WPTS) $\Pi$
	is a tuple
\begin{equation}\label{eq:wpts} 
\tag{\dag}
\Pi = (\pvars, \rvars,  L,\lin,\lout,\mu_{\mathrm{init}}, \rdvarjdis,\transset)%\win)
\end{equation}
for which:
	\begin{itemize}
		\item
		$\pvars$ and $\rvars$ are finite disjoint sets of \emph{program} and resp. \emph{sampling} variables.
%  (variables}) 
%  such that $\pvars\cap \rvars=\emptyset$.
    \item $\locs$ is a finite set of \emph{locations} 
  %or \emph{program counters} 
  with special locations $\lin,\lout\in \locs$. Informally, $\lin$ is the initial location and $\lout$ represents program termination. 
		\item
		$\mu_{\mathrm{init}}$ is the \emph{initial probability distribution} over $\mathbb{R}^{\pvars}$ with a finite support (denoted by $\supp{\mu_{\mathrm{init}}}$), 
  %from which the initial program valuation %$\valin$ is sampled, 
  while $\rdvarjdis$ is a function that assigns a probability distribution $\rdvarjdis(r)$ to each 
  %sampling variable 
  $r \in \rvars$. We call each $\pv\in\supp{\mu_{\mathrm{init}}}$ an \emph{initial program valuation}, and abuse the notation so that $\rdvarjdis$ also denotes the independent joint distribution of all $\rdvarjdis(r)$'s ($r\in \rvars$).
		\item 
		$\transset$ is a finite set of \emph{transitions} where
		each transition $\tau \in \transset$ is a tuple $\langle \loc, \phi, F_1,\dots,F_k \rangle$ such that 
(i) $\loc\in L$ is the \emph{source location} of the transition, 
%\item 
(ii) $\phi$ is the \emph{guard condition} which is a predicate over variables $\pvars$, %which serves as the \emph{guard condition}, 
and (iii) each $F_j:=\langle \loc'_j, p_j, \upd_j,\wet_j \rangle$ is called a \emph{weighted fork} for which (a) $\loc'_j\in L$ is the \emph{destination location} of the fork, (b) $p_j\in (0,1]$ is the probability of this fork, (c) $\upd_j:\Rset^{|\pvars|} \times \Rset^{|\rvars|} \rightarrow \Rset^{|\pvars|}$ is an {\em update function} that takes as inputs the current program and sampling valuations  and returns an updated program valuation in the next step, and (d) $\wet_j:\Rset^{|\pvars|} \times \Rset^{|\rvars|}\to [0,\infty)$ is a \emph{score function} that gives the likelihood weight of this fork depending on the current program and sampling valuations.	
\end{itemize}
\end{definition}


In a WPTS, we use update and score functions to model the update on the program variables and resp. the likelihood weight when running a basic block of statements in a program, respectively.  
%and use score functions to model  caused by the execution of the score statements (if exists) in this block. 
If there is no score statement in the block, then the score function is constantly $1$. 
We always assume that a WPTS $\Pi$ is \emph{deterministic} and \emph{total}, i.e., (i) there is no program valuation that simultaneously satisfies the guard conditions of two distinct transitions from the same source location, and (ii) the disjunction of the guard conditions of all the transitions from any source location is a tautology. 
The transformation from a probabilistic program into its WPTS can be done in a straightforward way (see e.g.~\cite{DBLP:journals/toplas/ChatterjeeFNH18,DBLP:conf/cav/ChakarovS13}). 

\begin{example}\label{ex:pedestrian-semantics} 
\cref{fig:pedestrian-wpts} shows the WPTS of the program in \cref{fig:pedestrian-program} which has two locations $\lin,\lout$. 
 %In the WPTS, 
The circle nodes represent locations and square nodes model the forking behavior of transitions. An edge entering a square node is labeled with the condition of its respective transition, while an edge entering a circle node stands for a fork, which is associated with its probability, update functions and score functions that marked by $w$.\footnote{Here we omit the update functions if the values of program variables are unchanged.} The value of $step$ is initialised to $0$. An the initial probability distribution $\mu_{\mathrm{init}}$ is determined by the joint distribution of $(start,pos,dis,step)$ where $start\sim uniform(0,3)$ and $pos,dis,step$ observe the Dirac measures $Dirac(\{start\})$, $Dirac(\{0\})$ and $Dirac(\{0\})$, respectively, e.g., the probability of the event ``$dis\in\{0\}$'' equals $1$. As $step$ simply receives values from a sampling variable, we neglect it in the WPTS.\qed
\end{example}

%\paragraph{Score-recursive WPTS.} 

We say that a WPTS is \emph{non-score-recursive} if for all transitions $\tau=\langle \loc, \phi, F_1,  F_2,\dots,F_k \rangle$ in the WPTS with each fork $F_j=\langle \loc'_j, p_j, \upd_j,\wet_j \rangle$ ($1\le j\le k$), we have that each score function $\wet_j$ is constantly $1$ (i.e., the multiplicative weight does not change) for every $\loc'_j\ne \lout$. Otherwise, the WPTS is \emph{score-recursive}.
Informally, a non-score-recursive WPTS has non-trivial score functions only on the transitions to the termination of a program, while a score-recursive WPTS has {\tt score} statements in the execution of the program. 
For example, the WPTS of the program in~\cref{sec3:pedestrian} is non-score-recursive as the nontrivial (i.e., score values not equal to $1$) {\tt score} statement only appears to the termination, while the WPTS of the program in \cref{sec3:phylogenetic} is score recursive since it has {\tt score} statements inside the loop body.
In the case of a non-score-recursive WPTS, we say that the WPTS is \emph{score-bounded} by a positive real $M>0$ if for every $\tau=\langle \loc, \phi, F_1, F_2,\dots,F_k \rangle$ in the WPTS with $F_j=\langle \loc'_j, p_j, \upd_j,\wet_j \rangle$ ($1\le j\le k$), we have that 
$|\wet_j|\le M$ whenever $\loc'_j=\lout$.


Given a program valuation $\mathbf{v}$ and a predicate $\phi$ over variables $\pvars$, we say that $\mathbf{v}$ \emph{satisfies} $\phi$ (written as $\mathbf{v}\models\phi$) if $\phi$ holds when the variables in $\phi$ are substituted by their values in $\mathbf{v}$. 
A \emph{state} 
%of the WPTS $\Pi$ 
is a pair $\Xi=(\loc, \pv)$ where $\loc \in L$ (resp. $\pv \in \Rset^{|\pvars|}$) represents the current location (resp. program valuation), respectively, while a \emph{weighted state} is a triple 
%$\Xi^w:=(\loc, \pv,w)$ 
$\Theta=(\loc, \pv, w)$ 
where $(\loc, \pv)$ is a state and $w\in [0,\infty)$ represents the multiplicative likelihood weight accumulated so far. 


 
%\paragraph{Semantics.} 
Below we specify the semantics of a WPTS. Consider a WPTS $\Pi$ in the form of \eqref{eq:wpts}. The semantics of $\Pi$ is formalized by the infinite sequence $\Gamma=\{\widehat{\Theta}_n=(\widehat{\loc}_n,\widehat{\pv}_n,\widehat{w}_n)\}_{n\ge 0}$ 
%of \emph{random weighted states} 
where each $(\widehat{\loc}_n,\widehat{\pv}_n,\widehat{w}_n)$ is the random weighted state at the $n$th execution step of the WPTS such that $\widehat{\loc}_n$ (resp. $\widehat{\pv}_n$, $\widehat{w}_n$) is the random variable for the location (resp. the random vector 
%of random variables 
for the program valuation, the random variable for the multiplicative likelihood weight) at the $n$th step, respectively. %The initial random state $\widehat{\Theta}_0$ is constant and equals $(\lin,\valin,\win)$. 
%its corresponding stochastic process $\Gamma:=\{\hat{\Xi}_n\}_{n\ge 0}$ on states.
The sequence $\Gamma$ starts with the initial random weighted state 
$\widehat{\Theta}_0=(\widehat{\loc}_0,\widehat{\pv}_0,\widehat{w}_0)$ such that $\widehat{\loc}_0$ is constantly $\lin$, $\widehat{\pv}_0\in \supp{\mu_\mathrm{init}}$ is sampled from the initial distribution $\mu_\mathrm{init}$ and the initial weight $\widehat{w}_0$ is constantly set to $1$\footnote{This follows the traditional setting in e.g.~\cite{Beutner2022b}.}. 
Then, given the current random weighted state $\widehat{\Theta}_n=(\widehat{\loc}_n,\widehat{\pv}_n,\widehat{w}_n)$ at the $n$th step, the next random weighted state $\widehat{\Theta}_{n+1}=(\widehat{\loc}_{n+1},\widehat{\pv}_{n+1},\widehat{w}_{n+1})$ is determined by:
(a) If $\widehat{\loc}_n=\lout$, then $(\widehat{\loc}_{n+1}, \widehat{\pv}_{n+1},\widehat{w}_{n+1})$ takes the same weighted state as $(\widehat{\loc}_n,\widehat{\pv}_n,\widehat{w}_n)$ (i.e., the next weighted state stays at the termination location $\lout$);
(b) Otherwise, $\widehat{\Theta}_{n+1}$ is determined by the following procedure:
\begin{itemize}
\item First, since the WPTS $\Pi$ is deterministic and total, we take the unique transition $\tau=\langle \hat{\loc}_n,\phi,F_1,\dots, F_k \rangle$ such that $\hat{\pv}_n\models\phi$. 
\item Second, we choose a fork $F_j=\langle \loc_j, p_j,\upd_j,\wet_j\rangle$ with probability $p_j$.
\item 
Third, we obtain a sampling valuation $\rv\in \supp{\rdvarjdis}$ 
%over the sampling variables $\rvars$ 
by sampling each $r \in \rvars$ independently w.r.t the probability distribution $\rdvarjdis(r).$
\item Finally, the value of the next random weighted state $(\widehat{\loc}_{n+1}, \widehat{\pv}_{n+1},\widehat{w}_{n+1})$ is determined as that of 
$(\loc'_j, \upd_j(\hat{\pv}_n,\rv),\widehat{w}_n\cdot \wet_j(\widehat{\pv}_n,\rv))$. 
\end{itemize}


Given the semantics, a \emph{program run} of the WPTS $\Pi$ is a concrete instance of $\Gamma$, i.e., an infinite sequence $\omega=\{\Theta_n\}_{n\ge 0}$ of weighted states where each $\Theta_n=(\loc_n,\pv_n,w_n)$ is the concrete weighted state at the $n$th step in this program run with location $\loc_n$, program valuation $\pv_n$ and multiplicative likelihood weight $w_n$. A state $(\loc,\pv)$ is called \emph{reachable} if there exists a program run $\omega=\{\Theta_n\}_{n\ge 0}$ such that $\Theta_n=(\loc,\pv,w_n)$ for some $n$. 


 
\begin{example}\label{ex:pedestrian-run}
Consider the WPTS in \cref{ex:pedestrian-semantics}. Consider an initial program valuation $(1,1,0)$ which means that the initial values of $start,pos,dis$ are $1,1,0$, respectively. Then starting from the initial weighted state $(\lin,(1,1,0),1)$, a program run w.r.t the WPTS semantics above could be 
\[
(\lin,(1,1,0),1)\to (\lin,(1,0.5,0.5),1)\to (\lin,(1,-0.1,1.1),1)\to (\lout,(1,-0.1,1.1),3.9894).\qed
\]
\end{example}

Given an initial program valuation $\valin$ of a WPTS, one could construct a probability space over the program runs by their probabilistic evolution described above and standard constructions such as general state space Markov chains~\cite{meyn2012markov}. We denote the probability measure in the probability space by $\probm_{\valin}(-)$ and the expectation operator by $\expectdist{\valin}{-}$.  



\subsection{Normalised Posterior Distribution}\label{sec2:NPD}


Before presenting the central problem of Bayesian probabilistic programming, i.e., analyzing normalised posterior distribution with our WPTS models, we introduce some technical concepts.

%\paragraph{Termination.}
\begin{definition}[Termination]
The \emph{termination time} of a WPTS
%The \emph{termination time} of the WPTS 
$\Pi$ 
%is a random variable $T$ defined on programs runs given 
is the random variable $T$ given by
%a program run  $\omega=\{\Xi_n=(\loc_n,\pv_n,w_n)\}_{n\in\Nset}$,
%\begin{align*}	
$T(\omega):=\text{min}\{n\in\Nset\mid \loc_n=\lout\}$ for every program run  $\omega=\{(\loc_n,\pv_n,w_n)\}_{n\ge 0}$
%\end{align*}
where $\text{min}\,\emptyset:=\infty$. That is, $T(\omega)$ is the number of steps a program run $\omega$ takes to reach the termination location $\lout$. A WPTS $\Pi$ is \emph{almost-surely terminating} (AST) if $\probm_{\valin}(T<\infty)=1$ for all initial program valuations $\valin\in \supp{\mu_{\mathrm{init}}}$.  
%in the case that the program run never terminates. 
\end{definition}




\begin{definition}[Expected Weights]\label{def:exp-wt}
 Given a WPTS $\Pi$ in the form of \eqref{eq:wpts}, a designated initial program valuation $\valin$ and a measurable subset $\calU\in\Sigma_{\Rset^{|\pvars|}}$, the \emph{expected weight} $\measureSem{\Pi}_{\valin}(\calU)$ 
%$\measureSem{\Pi}(\valin)$ 
%of $\Pi$ w.r.t $\pv$ 
is defined as
%$\measureSem{\Pi}_\calU(\valin):=\expectdist{\valin}{\widehat{w}_T}$. 
$\measureSem{\Pi}_{\valin}(\calU):=\expectdist{\valin}{[\widehat{\pv}_T\in \calU]\cdot\widehat{w}_T}$. 
\end{definition}

By definition, we have that $\widehat{\pv}_T$ (resp. $\widehat{w}_T$) is the random vector (resp. variable) of the program valuation (resp. the multiplicative likelihood weight) at termination, respectively. Thus, $\measureSem{\Pi}_{\valin}(\calU)$ is the expectation of $\widehat{w}_T$ 
%over all program runs 
that start from the state $(\lin,\valin,1)$ and end with $\widehat{\pv}_T\in\calU$. If $\calU=\Rset^{|\pvars|}$, the restriction of $\widehat{\pv}_T\in\calU$ can be removed.

Below we define the normalised posterior distribution (NPD) problem. %under our WPTS semantics. 

 
\begin{definition}[Normalised Posterior Distribution]\label{def:npd}
Given a WPTS $\Pi$ in the form of \eqref{eq:wpts},
%We write $\measureSem{\Pi}(\valin)$ iff $\calU=\Rset^{|\pvars|}$.)
%Then given a probability distribution $\mu$ over initial program valuations, 
the \emph{normalised posterior distribution} (NPD) $\posterior_\Pi$ of $\Pi$ 
%over $U$ 
is defined by:
\begin{align*}
\posterior_{\Pi}(\calU):=\measureSem{\Pi}(\calU)/Z_\Pi\mbox{ for all measurable subsets } \calU\in \Sigma_{\Rset^{|\pvars|}},   
\end{align*}	
where 
$\measureSem{\Pi}(\calU):=\int_{\calV} \measureSem{\Pi}_{\pv}(\calU)\cdot \mu_{\mathrm{init}}(\mathrm{d} \pv)$ is the \emph{unnormalised posterior distribution} w.r.t. $\calU$, $\calV:=\supp{\mu_{\mathrm{init}}}$, %is the support of $\mu_{\mathrm{init}}$
%is the integral of all expected weights with an initial program valuation $\pv\in U$, 
and $Z_\Pi:=\measureSem{\Pi}(\Rset^{|\pvars|})$ is the \emph{normalising constant}.  
The WPTS $\Pi$ is called \emph{integrable} 
%w.r.t a probability distribution (for initial program valuations) 
if we have $0<Z_{\Pi}<\infty$. 
%\pw{Shall we mention that $\measureSem{\Pi}_{\pv}(\calU)$ is an integrable function here?}
\end{definition}

%We call a WPTS $\Pi$ \emph{integrable} 
%w.r.t a probability distribution (for initial program valuations) 
%if the normalising constant is finite, i.e., ~$0<Z_{\Pi}<\infty$. %for any $\pv\in\val{\pvars}$. 
%Given an integrable program, we are interested in deriving lower and upper bounds on the normalised posterior distribution over some measurable set $U\in \Sigma_\Rset$.
\paragraph{Interval Bounds for NPD.} In this work, we consider the automated interval-bound analysis for NPD of a WPTS. Formally, we aim to derive an interval $[l,u]\subseteq [0,\infty)$ 
for an integrable WPTS $\Pi$ and any measurable sets $\calU\in\Sigma_{\Rset^{|\pvars|}}$ as tight as possible such that $l\le \posterior_{\Pi}(\calU) \le u$. 
%$l,u$ are called \emph{interval bounds} for the NPD $\posterior_{\Pi}(\calU)$. 
%To achieve this, in the following (\cref{sec:math}) we develop approaches to obtain interval bounds for expected weights as $\measureSem{\Pi}(\calU)$ and $Z_\Pi$ are integrations of expected weights over $\calV$. 
 



To achieve interval bounds for NPD, below we introduce the construction of a new WPTS $\Pi_\calU$ based on the original WPTS $\Pi$ and a measurable set $\calU\in \Sigma_{\Rset^{|\pvars|}}$.  

\paragraph{Construction of $\Pi_\calU$.} Consider a probabilistic program $P$ and its WPTS $\Pi$, given a measurable set $\calU\in\Sigma_{\Rset^{|\pvars|}}$, we construct a new program $P_\calU$ by adding a conditional branch of the form ``\textbf{if} $\pv_T\notin\calU$ \textbf{then} \textbf{score}($0$) \textbf{fi}'' immediately after the termination of $P$ and obtain the WPTS $\Pi_\calU$ of $P_\calU$. Therefore, $\Pi$ and $\Pi_\calU$ have the same initial probability distribution $\mu_{\mathrm{init}}$ and the same finite support $\calV=\supp{\mu_{\mathrm{init}}}$. The following proposition shows that interval-bound analysis for NPD can be reduced to interval-bound analysis for expected weights in the form $\llbracket \Pi\rrbracket_{\pv}(\Rset^{|\pvars|})$. 

\begin{proposition}\label{prop:unnorm-norm}
   Given a WPTS $\Pi$ in the form of \eqref{eq:wpts}, a measurable set $\calU\in\Sigma_{\Rset^{|\pvars|}}$ and the WPTS $\Pi_\calU$ constructed as above, we have that $\llbracket \Pi \rrbracket_{\pv}(\calU)=\llbracket \Pi_\calU\rrbracket_{\pv}(\Rset^{|\pvars|})$ for any $\pv\in\calV=\supp{\mu_{\mathrm{init}}}$. Furthermore,
   if there exist intervals $[l_1,u_1],[l_2,u_2]\subseteq [0,\infty)$ such that $\llbracket \Pi_\calU\rrbracket_{\pv}(\Rset^{|\pvars|})\in [l_1,u_1]$ and $\llbracket \Pi\rrbracket_{\pv}(\Rset^{|\pvars|})\in [l_2,u_2 ]$ for any $\pv\in\calV$, then we have two intervals $[l_\calU,u_\calU],[l_Z,u_Z]\subseteq [0,\infty)$ such that the unnormalised posterior distribution $\llbracket \Pi\rrbracket (\calU)\in [l_\calU,u_\calU]$ and the normalising constant $Z_\Pi\in [l_Z,u_Z]$. Moreover, if $\Pi$ is integrable, i.e., $[l_Z,u_Z]\subseteq (0,\infty)$, then we can obtain the NPD $\posterior_{\Pi}(\calU)\in [\frac{l_\calU}{u_Z},\frac{u_\calU}{l_Z}]$.\footnote{The interval bounds derived in this manner may be loose, but they are definitely correct.}  Note that by \cref{def:npd}, $l_\calU=\int_\calV l_1 \cdot\mu_{\mathrm{init}}(\mathrm{d} \pv)$, $u_\calU=\int_\calV u_1 \cdot\mu_{\mathrm{init}}(\mathrm{d} \pv)$, $l_Z=\int_\calV l_2 \cdot\mu_{\mathrm{init}}(\mathrm{d} \pv)$ and $u_Z=\int_\calV u_1 \cdot\mu_{\mathrm{init}}(\mathrm{d} \pv)$.

\end{proposition}

The proof of \cref{prop:unnorm-norm} is relegated to \cref{app:sec2-prop}. In the following, we will develop approaches to obtain interval bounds for expected weights.
%in the form $\llbracket \Pi \rrbracket_{\pv}(\Rset^{|\pvars|})$ where $\pv$ is an initial program valuation.











 

\section{Motivating Examples}\label{sec:overview}
\section{Secure Design of \puma}\label{sec:design}
In this section, we first present an overview of \puma, and present the protocols for secure $\gelu$ , $\softmax$, embedding, and $\layernorm$ used by \puma. Note that the linear layers such as matrix multiplication are straightforward in replicated secret sharing, so we mainly describe our protocols for non-linear layers in this manuscript.

\subsection{Overview of \puma}\label{sec:overview}
To achieve secure inference of Transformer models, \puma\ defines three kinds of roles: one model owner, one client, and three computing parties. The model owner and the client  provide their models or inputs to the computing parties (i.e., $P_0$, $P_1$, and $P_2$) in a secret-shared form, then the computing parties execute the MPC protocols and send the results back to the client. Note that the model owner and client can also act as one of the computing party, we describe them separately for generality. \eg, when the model owner acts as $P_0$, the client acts as  $P_1$, a third-party dealer acts as $P_2$, the system model becomes the same with \mpcformer~\citep{li2023mpcformer}.

During the secure inference process, a key invariant is maintained: For any layer, the computing parties always start with 2-out-of-3 replicated secret shares of the previous layer's output and the model weights, and end with 2-out-of-3 replicated secret shares of this layer's output. As the shares do not leak any information to each party, this ensures that the layers can be sequentially combined for arbitrary depths to obtain a secure computation scheme for any Transformer-based model.
%The main focus of \puma\ is to reduce the computation and communication costs between the computing parties while maintaining the desired level of security. 



\iffalse
\textbf{Threat Model.}
Following previous works~\citep{aby3,li2023mpcformer},
\puma\ resists a semi-honest (a.k.a., honest-but-curious) adversary in honest-majority~\citep{lindell2009proof}, where the adversary passively corrupts no more than one computing party. Such an adversary follows the protocol specification exactly, but may try to learn more information than permitted. Please note that \puma\ cannot protect against the extraction of information from the inference results, and the examination of mitigating solutions (\eg, differential privacy~\citep{abadi2016deep}) falls outside the scope of this study.
\fi 

\subsection{Protocol for Secure GeLU}\label{sec:gelu}
Most of the current approaches view the $\gelu$ function as a composition of smaller functions and try to optimize each piece of them, making them to miss the
chance of optimizing the private $\gelu$ as a whole. Given the $\gelu$ function:
\begin{equation}\label{eq:gelu}
\begin{split}
    \gelu(x) &= \frac{x}{2} \cdot \left(1 + \tanh \left( \sqrt{\frac{2}{\pi}} \cdot \left(x + 0.044715 \cdot x^3 \right) \right) \right)\\
    &\approx x\cdot \mathsf{sigmoid}(0.071355\cdot x^3 + 1.595769\cdot x) 
\end{split},
\end{equation}
these approaches~\citep{hao2022iron,characmpctranformer} focus either on designing efficient protocols for function $\tanh$
or using the existing MPC protocols of exponentiation and reciprocal for $\mathsf{sigmoid}$. 

However, none of current approaches have utilized the fact that $\gelu$ function is almost linear on the two sides (\ie, $\gelu(x)\approx 0$ for $x<-4$ and $\gelu(x)\approx x$ for $x>3$). 
Within the short interval $[-4,3]$ of $\gelu$,
we suggest a piece-wise approximation of low-degree polynomials is a more efficient and easy-to-implement choice for its secure protocol. Concretely, our piece-wise low-degree polynomials are shown as equation~(\ref{eq:geluapprox}):
\begin{equation}\label{eq:geluapprox}
\gelu(x)=
\begin{cases}
0, & x<-4 \\
F_0(x), & -4 \le x < -1.95 \\
F_1(x), & -1.95 \le x \le 3 \\
x, & x >3
\end{cases},
\end{equation}
where polynomials $F_0()$ and $F_1()$ are computed by library $\mathsf{numpy.ployfit}$\footnote{\url{https://numpy.org/doc/stable/reference/generated/numpy.polyfit.html}} as equation~(\ref{eq:f0f1}). Surprsingly, the above simple poly fit works very well and our $\mathsf{max\ error}< 0.01403$, $\mathsf{median\ error}< 4.41e-05$, and $\mathsf{mean\ error}< 0.00168$.
\begin{equation}\label{eq:f0f1}
\begin{cases}
F_0(x) &= -0.011034134030615728 x^3 -0.11807612951181953 x^2 \\
&- 0.42226581151983866 x -0.5054031199708174\\
F_1(x) &= 0.0018067462606141187x^6 -0.037688200365904236 x^4 \\
&+ 0.3603292692789629x^2 + 0.5x + 0.008526321541038084
\end{cases}
\end{equation}

Formally, given secret input $\share{x}$, our secure $\gelu$ protocol $\Pi_{\gelu}$ is constructed as algorithm~\ref{protocol:gelu}. 
\iffalse
\begin{itemize}
    \item The parties jointly compute
$\share{b_0}^2 = \Pi_{\mathsf{LT}}(\share{x}, 4)$,
$\share{b_1}^2 = \Pi_{\mathsf{LT}}(\share{x}, -1.95)$, and
$\share{b_2}^2 = \Pi_{\mathsf{LT}}(3, \share{x})$.

\item  Then, each $P_i$ locally compute
$\share{b_3}^2 = \share{b_1}^2 \oplus \share{b_2}^ \oplus 1$ and
$\share{b_4}^2 = \share{b_0}^2 \oplus \share{b_1}^2$

\item Finally, the parties compute and return 
$\share{b_2}^2 \cdot \share{x} + \share{b_4}^2 \cdot F_0(\share{x}) + \share{b_3}^2 \cdot F_1(\share{x})$, where polynomials $(F_0, F_1)$ can be computed easily using secure addition and multiplication (and its variants, \eg, secure square)~\citep{spu}. 
\end{itemize}
\fi 

\begin{algorithm}[tp]
\caption{Secure $\gelu$ Protocol $\Pi_{\mathsf{GeLU}}$}\label{protocol:gelu}
\begin{algorithmic}[1]
\REQUIRE
$P_i$ holds the 2-out-of-3 replicate secret share $\share{x}_i$ for $i\in \{0,1,2\}$ 
\ENSURE
$P_i$ gets the 2-out-of-3 replicate secret share $\share{y}_i$ for $i\in \{0,1,2\}$, where $y=\gelu(x)$.

\STATE $P_0$, $P_1$, and $P_2$ jointly compute
\begin{equation*}
\begin{split}
&\shareb{b_0} = \Pi_{\mathsf{LT}}(\share{x}, -4),~~~\vartriangleright b_0 = 1\{x<-4\}\\
&\shareb{b_1} = \Pi_{\mathsf{LT}}(\share{x}, -1.95),~~~\vartriangleright b_1 = 1\{x<-1.95\} \\
&\shareb{b_2} = \Pi_{\mathsf{LT}}(3, \share{x}),~~~~~~\vartriangleright b_2 = 1\{3<x\}
\end{split}
\end{equation*}
and compute 
$\shareb{z_0} = \shareb{b_0} \oplus \shareb{b_1}$,
$\shareb{z_1} = \shareb{b_1} \oplus \shareb{b_2} \oplus 1$, and $\shareb{z_2}=\shareb{b_2}$. Note that $z_0 = 1\{-4\le x < -1.95\}$, $z_1 = 1\{-1.95\le x\le 3\}$, and $z_2 = 1\{x>3\}$.

\STATE Jointly compute $\share{x^2} = \Pi_{\mathsf{Square}}(\share{x})$, $\share{x^3} = \Pi_{\mathsf{Mul}}(\share{x}, \share{x^2})$, $\share{x^4} = \Pi_{\mathsf{Square}}(\share{x^2})$, and $\share{x^6} = \Pi_{\mathsf{Square}}(\share{x^3})$.

\STATE Computing polynomials $\share{F_0(x)}$ and $\share{F_1(x)}$ based on $\{\share{x}, \share{x^2}, \share{x^3}, \share{x^4}, \share{x^6}\}$ as equation~(\ref{eq:geluapprox}) securely.


\RETURN$\share{y} = \Pi_{\mathsf{Mul_{BA}}}(\shareb{z_0}, \share{F_0(x)}) + \Pi_{\mathsf{Mul_{BA}}}(\shareb{z_1}, \share{F_1(x)})+\Pi_{\mathsf{Mul_{BA}}}(\shareb{z_2}, \share{x})$.

\end{algorithmic}
\end{algorithm}



\subsection{Protocol for Secure Softmax}\label{sec:secureatten}

In the function $\attention(\Q,\K,\V)=
\softmax(\Q \cdot \K^\mathsf{T} + \M) \cdot \V$, where $\M$ can be viewed as a bias matrix, the key challenge is computing function $\softmax$. For the sake of numerical stability, the $\softmax$ function is computed as
\begin{equation}\label{eq:softmax}
    \softmax(\x)[i]=\frac{\exp(\x[i] - \bar{x} - \epsilon)}{\sum_i \exp(\x[i] - \bar{x} - \epsilon)},
\end{equation}
where $\bar{x}$ is the maximum element of the input vector $\x$. 
For the normal plaintext softmax, $\epsilon=0$. For a two-dimension matrix, we apply equation~(\ref{eq:softmax}) to each of its row vector.

Formally, our detailed secure protocol  $\Pi_{\softmax}$ is illustrated in algorithm~\ref{protocol:softmax}, where we propose two optimizations:
\begin{itemize}
\item 
For the first optimization, we set $\epsilon$ in equation~\ref{eq:softmax} to a tiny and positive
value, e.g., $\epsilon =
10^{-6}$, so that the inputs to exponentiation
in equation~\ref{eq:softmax} are all negative. We exploit the negative operands
for acceleration. Particularly, we compute the exponentiation using the Taylor series~\citep{tan2021cryptgpu} with a simple clipping
\begin{equation}\label{eq:negexp}
\mathsf{negExp}(x) = \begin{cases}
    0, &x < T_{\exp} \\
    (1+\frac{x}{2^t})^{2^t}, &x\in [T_{\exp},0].
\end{cases}
\end{equation}
Indeed, we apply the less-than for the branch $x < T_{\exp}$
The division by $2^t$ can be achieved using
$\Pi_{\mathsf{Trunc}}^t$ since the input is already negative. Also, we can
compute the power-of-$2^t$ using $t$-step sequences of square function $\Pi_{\mathsf{square}}$ and $\Pi_{\mathsf{Trunc}}^f$. Suppose our MPC program uses
$18$-bit fixed-point precision. Then we set $T_{\exp}=-14$ given $\exp(-14) < 2^{-18}$, and empirically set $t = 5$.


\item 
Our second optimization is to reduce the number of divisions, which ultimately saves computation and communication costs.
To achieve this, for a vector $\x$ of size $n$, we have replaced the operation $\mathsf{Div}(\x, \mathsf{Broadcast}(y))$ with $\x \cdot  \mathsf{Broadcast}(\frac{1}{y})$, where $y=\sum_{i=1}^n\x[i]$. By making this replacement, we effectively reduce $n$ divisions to just one reciprocal operation and $n$ multiplications.
This optimization is particularly beneficial in the case of the $\softmax$ operation. The $\frac{1}{y}$ in the $\softmax$ operation is still large enough to maintain sufficient accuracy under fixed-point values. As a result, this optimization can significantly reduce the computational and communication costs while still providing accurate results.
\end{itemize}

\begin{algorithm}[tp]
\caption{Secure $\softmax$ Protocol $\Pi_{\softmax}$}\label{protocol:softmax}
\begin{algorithmic}[1]
\REQUIRE
$P_i$ holds the 2-out-of-3 replicate secret share $\share{\x}_i$ for $i\in \{0,1,2\}$, and $\x$ is a vector of size $n$. 
\ENSURE
$P_i$ gets the 2-out-of-3 replicate secret share $\share{\y}_i$ for $i\in \{0,1,2\}$, where $\y=\softmax(\x)$.

\STATE $P_0$, $P_1$, and $P_2$ jointly compute
$\shareb{\mathbf{b}} = \Pi_{\mathsf{LT}}(T_{\exp}, \share{\x})$ and the maximum $\share{\bar{x}} = \Pi_{\mathsf{Max}}(\share{\x})$.

\STATE Parties locally computes $\share{\hat{\x}} = \share{\x} - \share{\bar{x}} - \epsilon$, and jointly compute $\share{\z_0} = 1+  \Pi_{\mathsf{Trunc}}^t(\share{\hat{\x}})$.

\FOR{$j=1,2,\dots, t$}
\STATE $\share{\z_j} = \Pi_{\mathsf{Square}}(\share{\z_{j-1}})$.
\ENDFOR

\STATE Parties locally compute $\share{z} = \sum_{i=1}^n \share{\z[i]}$ and jointly compute $\share{1/z} = \Pi_{\mathsf{Recip}}(\share{z})$.

\STATE Parties jointly compute $\share{\z / z} = \Pi_{\mathsf{Mul}}(\share{\z}, \share{1/z})$

\RETURN $\share{\y} = \Pi_{\mathsf{Mul}_{\mathsf{BA}}}( \shareb{\mathbf{b}}, \share{\z / z})$.

\end{algorithmic}
\end{algorithm}

\subsection{Protocol for Secure Embedding}\label{sec:embed}


The current secure embedding procedure described in~\citep{li2023mpcformer} necessitates the client to  generate a one-hot vector using the token $\tokenid$ locally. This deviates from a plaintext Transformer workflow where the one-hot vector is generated inside the model. As a result, they have to carefully strip off the one-hot step from the pre-trained models, and add the step to the client side, which could be an obstacle for deployment. 



To address this issue, we propose a secure embedding design as follows. Assuming that the token $\tokenid\in [n]$ and all embedding vectors are denoted by $\E= (\e_1^T, \e_2^T, \dots, \e_n^T)$, the embedding can be formulated as $\e_{\tokenid} = \mathbf{E}[\tokenid]$. Given $(\tokenid, \E)$ are in secret-shared fashion, our secure embedding protocol $\Pi_{\mathsf{Embed}}$ works as follows:
\begin{itemize}
    \item The computing parties securely compute the one-hot vector $\shareb{\mathbf{o}}$ after receiving $\share{\tokenid}$ from the client. Specifically, $\shareb{\mathbf{o}[i]}=\Pi_{\mathsf{Eq}}(i,\share{\tokenid})$ for $i\in [n]$.
    \item The parties can compute the embedded vector via $\share{\e_{\tokenid}} = \Pi_{\mathsf{Mul_{BA}}}(\share{\E}, \shareb{\mathbf{o}})$, where  does not require secure truncation.
\end{itemize}
In this way, our $\Pi_{\mathsf{Embed}}$ does not require explicit modification of the workflow of plaintext Transformer models, at the cost of more $\Pi_{\mathsf{Eq}}$ and $\Pi_{\mathsf{Mul_{BA}}}$ operations. 



\subsection{Protocol for Secure LayerNorm}\label{sec:seclayernorm}
Recall that given a vector $\x$ of size $n$, $\layernorm(\x)[i] =  \gamma \cdot \frac{\x[i]-\mu}{\sqrt{\sigma}} + \beta$, where $(\gamma, \beta)$ are trained parameters, $\mu = \frac{\sum_{i=1}^n \x[i]}{n}$, and $\sigma = \sum_{i=1}^n (\x[i] - \mu)^2$. In MPC, the key challenge is the evaluation of the divide-square-root $\frac{\x[i]-\mu}{\sqrt{\sigma}}$ formula. To securely evaluate this formula, CrypTen sequentially executes the MPC protocols of square-root, reciprocal, and multiplication. However, we observe that $\frac{\x[i]-\mu}{\sqrt{\sigma}}$ is equal to $(\x[i]-\mu)\cdot \sigma^{-1/2}$. And in the MPC side, the costs of computing the inverse-square-root $\sigma^{-1/2}$ is similar to that of the square-root operation~\citep{rSqrt}. Besides, inspired by the second optimization of \S~\ref{sec:secureatten}, we can first compute $\sigma^{-1/2}$ and then $\mathsf{Broadcast}(\sigma^{-1/2})$ to support fast and secure $\layernorm(\x)$. And our formal protocol $\Pi_{\layernorm}$ is shown in algorithm~\ref{protocol:layernorm}.

\begin{algorithm}[tp]
\caption{Secure $\mathsf{LayerNorm}$ Protocol $\Pi_{\mathsf{LayerNorm}}$}\label{protocol:layernorm}
\begin{algorithmic}[1]
\REQUIRE
$P_i$ holds the 2-out-of-3 replicate secret share $\share{\x}_i$ for $i\in \{0,1,2\}$, and $\x$ is a vector of size $n$. 
\ENSURE
$P_i$ gets the 2-out-of-3 replicate secret share $\share{\y}_i$ for $i\in \{0,1,2\}$, where $\y=\mathsf{LayerNorm}(\x)$.

\STATE $P_0$, $P_1$, and $P_2$ compute $\share{\mu} = \frac{1}{n}\cdot \sum_{i=1}^n\share{\x[i]}$ and $\share{\sigma} = \sum_{i=1}^n \Pi_{\mathsf{Square}}(\share{\x} - \share{\mu})[i]$.

\STATE Parties jointly compute $\share{\sigma^{-1/2}} = \Pi_{\mathsf{rSqrt}}(\share{\sigma})$.

\STATE Parties jointly compute $\share{\mathbf{c}} = \Pi_{\mathsf{Mul}}((\share{\x} - \share{\mu}), \share{\sigma^{-1/2}})$

\RETURN $\share{\y} = \Pi_{\mathsf{Mul}}(\share{\gamma}, \share{\mathbf{c}}) + \share{\beta}$.

\end{algorithmic}
\end{algorithm}


\section{Theoretical Approaches}\label{sec:math}

In this section, we present our theoretical approaches for interval-bound analysis of expected weights in the form $\llbracket \Pi \rrbracket_{\pv}(\Rset^{|\pvars|})$, namely the fixed-point approach, the OST approach. To obtain tight results, we also introduce the truncation operation.  

\subsection{The Fixed-Point Approach}\label{sec:fixed-point}

Our fixed-point approach targets \emph{score-at-end} Bayesian programs that terminate almost-surely and have a single score statement with a bounded weight at the termination. 
We formally define the notion of score-at-end programs directly over WPTS's. A WPTS $\Pi$ is \emph{score-at-end} if: (i) $\Pi$ has AST; (ii) there is exactly one transition $\tau$ in the WPTS that involves the destination location $\lout$, and this transition $\tau$ takes the form $\tau=\langle \loc, \mathbf{-}, F\rangle$ with the single weighted fork $F=\langle \lout, 1, \mbox{\sl id}, wt\rangle$ where $\mbox{\sl id}$ is the identity function and $wt$ is \emph{score-bounded} by a constant $M>0$, i.e., $wt\in [0, M]$; and (iii) all transitions other than the transition $\tau$ have the score function constantly equal to $1$. Such programs widely exist in the literature~\cite{DBLP:conf/cav/GehrMV16,DBLP:conf/pldi/GehrSV20,Beutner2022b}.

To demonstrate the approach, we recall several basic concepts in lattice theory (see Appendix~\ref{app:fixed-point-materials} for details). 
Given a complete lattice $(K, \sle)$ and a function $f: K \to K$, the \emph{supremum} of $K$ is denoted by $\bigsqcup K$, while the \emph{infimum} of $K$ is denoted by $\bigsqcap K$. An element $k \in K$ is called a \emph{fixed-point} if $f(k) = k.$ The \emph{least} (resp. \emph{greatest}) \emph{fixed-point}  of $f$, denoted by $\lfp f$ (resp. $\gfp f$),  is the fixed-point that is no greater (resp. smaller) than every fixed-point under $\sle$. 
Moreover, $k$ is a \emph{prefixed-point} if $f(k) \sle k$ and a \emph{postfixed-point} if $k\sle f(k)$. 
We apply Tarski's fixed-point theorem as follows. 

\begin{theorem}[\textit{Tarski}~\cite{KnasterTarski}]\label{thm:tarski}
Let $(K, \sle)$ be a complete lattice and $f:K \to K$ be a monotone function. Then, both $\lfp\ f$ and $\gfp\ f$ exist. Moreover,
$\textstyle \lfp\ f  = \mathop{\bigsqcap} \left\{x\ |\ f(x)\sle x\right\}\mbox{ and }\gfp\ f = \mathop{\bigsqcup} \left\{x\ |\ x\sle f(x)\right\}$. 
\end{theorem}

Based on \cref{thm:tarski}, we present our fixed-point approach for score-at-end WPTS's. Below we fix a score-at-end WPTS $\Pi$. 
We define $\Lambda$ as the set of states $(\loc, \pv)$ where $\loc$ is a location and $\pv$ is a program valuation. Given a maximum finite value $M\in [0,\infty)$, we define a \emph{state function} as a function $h:\Lambda\to [-M,M]$ such that for all $\pv\in\Rset^{|\pvars|}$, $h(\lout,\pv)\in [0,M]$.
The intuition of a state function $h$ is that $h(\loc,\pv)$ gives an estimation of the expected weight when the WPTS $\Pi$ starts with the initial location $\loc$ and the initial program valuation $\pv$.
We denote the set of all state functions with the maximum value $M$ by $\mathcal{K}_M$. We also use the usual partial order $\le$ on $\mathcal{K}_M$ that is defined in the pointwise fashion, i.e., for any $h_1,h_2\in \mathcal{K}_M$, $h_1\le h_2$ iff $h_1(\loc,\pv)\le h_2(\loc,\pv)$ for all $(\loc,\pv)\in\Lambda$. It is straightforward to verify that  $(\mathcal{K}_M,\le)$ is a complete lattice. 

To connect the complete lattice $(\mathcal{K}_M,\le)$ with expected weights, we define the \emph{expected-weight function} $\mbox{\sl ew}_\Pi$ by $\mbox{\sl ew}_\Pi(\lin,\pv):=\llbracket \Pi\rrbracket_{\pv}(\Rset^{|\pvars|})$, and omit the subscript $\Pi$ if it is clear from the context. Informally, $\mbox{\sl ew}_\Pi(\lin,\pv)$ is the expected weight of the program starting from an initial program valuation $\pv$. 
Note that if the weight at termination is bounded by a maximum value $M$, then we have $\mbox{\sl ew}_\Pi\in \mathcal{K}_M$. 
We consider the following higher-order function over the complete lattice $(\mathcal{K}_M,\le)$. 

\begin{definition}[Expected-Weight Transformer]\label{def:ewt}
Given a finite maximum value $M\in [1,\infty)$, the \emph{expected-weight transformer} $\ewt_\Pi:\mathcal{K}_M\to \mathcal{K}_M$ is the higher-order function such that for each state function $h\in \mathcal{K}_M$ and state $(\loc,\pv)$, if  $\tau = \langle \loc, \phi, F_1,\dots,F_k \rangle$ is the unique transition that satisfies $\pv\models\phi$ and $F_j=\langle \loc'_j,p_j,\upd_j,\wet_j\rangle$ for each $1\le j\le k$, then we have that
\begin{equation}\label{eq:ewt}
\ewt_\Pi(h)(\loc,\pv)\,:=\, \begin{cases}  \sum_{j=1}^k p_j\cdot \expectdist{\rv}{\wet_j(\pv,\rv)\cdot h(\loc'_j,\upd_j(\pv,\rv))} & \mbox{if } \loc\neq \lout \\
1 & \mbox{otherwise} 
\end{cases}\enskip. 
\end{equation}
In \eqref{eq:ewt}, the expectation $\expectdist{\rv}{-}$ is taken over a sampling valuation $\rv$ that observes the distribution $\rdvarjdis$.
\end{definition}
Informally, given a state function $h$, the expected-weight transformer $\ewt_\Pi$ computes the expected weight $\ewt_\Pi(h)$ after one step of WPTS transition. Note that in \eqref{eq:ewt}, the weight $\wet_j(\pv,\rv)$ equals $1$
when the location $\loc$ does not refer to the score statement at the end of the program, as we consider that the WPTS $\Pi$ is score-at-end. This implies that $\ewt_\Pi$ is indeed a higher-order operator for the complete lattice $(\mathcal{K}_M,\le)$ when the maximum value $M$ is a bound for the weights in the score statement. By the monotonicity of expectation, we have that $\ewt_\Pi$ is monotone. 

We will omit the subscript $\Pi$ in $\ewt_\Pi(h)$ if it is clear from the context. In the next definition, we define potential weight functions that are prefixed/postfixed points of the operator $\ewt_\Pi(h)$.

\begin{definition}[Potential Weight Functions]\label{def:puwf}
    A \emph{potential upper weight function} (PUWF) is a function $h:\locs{}\times\val{V_p} \rightarrow\Rset$ that has the following properties:
	\begin{itemize}
		\item[\emph{(C1)}] for all reachable states $(\loc,\pv)$ with $\loc\neq\lout$, we have $\ewt({h})(\loc,\pv) \le h(\loc,\pv)$;
		\item[\emph{(C2)}] for all reachable states $(\loc,\pv)$ such that $\loc=\lout$, we have $h(\loc,\pv)=1$.
	\end{itemize}
	Analogously, a \emph{potential lower weight function} (PLWF) is a function $h:\locs{}\times\val{V_p} \rightarrow\Rset$
	that satisfies the conditions (C1') and (C2), for which the condition (C1') is almost the same as (C1) except for that ``$\ewt({h})(\loc,\pv) \le h(\loc,\pv)$'' is replaced with ``$\ewt({h})(\loc,\pv) \ge h(\loc,\pv)$''. 
\end{definition}

Informally, a PUWF is a state function that satisfies the prefixed-point condition of $\ewt$ at non-terminating locations, and equals one at termination. A PLWF is defined similarly by using the postfixed-point condition. 
The main theorem below states that potential weight functions serve as bounds for expected weights. The proof establishes that the fixed point of $\ewt_\Pi$ is unique and equals $\mbox{\sl ew}_\Pi$ 
when the WPTS $\Pi$ has AST, and applies Tarski's Fixed Point Theorem to use prefixed and post-fixed points to derive upper and lower bounds. See Appendix~\ref{app:fixedpoint} for the detailed proof.

\begin{theorem}[Fixed-Point Approach]\label{thm:fix-point-bounds}

$\llbracket \Pi\rrbracket_{\valin} (\Rset^{|\pvars|})\le h(\lin,\valin)$ (resp. $\llbracket \Pi\rrbracket_{\valin} (\Rset^{|\pvars|})\ge h(\lin,\valin)$) for any bounded PUWF (resp. PLWF) $h$ over $\Pi$ and initial state $(\lin,\valin)$.
\end{theorem}

\begin{remark}%[Generality of our fixed-point approach]
Although our fixed-point approach targets score-at-end programs, it can be extended to Bayesian programs such that in any execution of the program, only a bounded number of {\tt score}  statements are executed and they are all bounded. 
This is because one can find a bound for the overall weight of the boundedly many score functions to apply our fixed point theorem.  The intuition here is that our fixed point approach can handle Bayesian programs whose multiplicative cumulative score values are always bounded by some constant. Especially, this approach can handle programs where all execution cycles are score-bounded by one. \qed
\end{remark}

 

\subsection{The OST Approach}\label{sec:ostapproach}


As stated in Remark~\ref{rmk:maxvalue}, our fixed-point approach cannot handle score-recursive Bayesian probabilistic programs. To tackle score-recursive programs, we consider the adaption of Optional Stopping Theorem (OST) to our case. OST is a classical theorem in martingale theory that characterizes the relationship between the expected values initially and at a stopping time in a %discrete-time 
supermartingale. Below we first present the classical form of OST.


\begin{theorem}[Optional Stopping Theorem (OST) \cite{williams1991probability}]
Let $\{X_n\}_{n=0}^\infty$ be a supermartingale adapted to a filtration $\mathcal{F}=\{\mathcal{F}_n\}_{n=0}^\infty$, and $\kappa$ be a stopping time w.r.t. the filtration $\mathcal{F}$. 
%$\{\mathcal{F}_n\}_{n=0}^\infty$. 
Then the following condition is sufficient to ensure that $\expv\left(|X_\kappa|\right)<\infty$ and %$\expv\left(X_\kappa\right) = \expv(X_0)$ 
%(resp. 
$\expv\left(X_\kappa\right)\le\expv(X_0)$:
\begin{itemize}
\item  $\expv(\kappa)<\infty$, and
\item (\emph{bounded difference}) there exists a constant $C>0$ such that for all $n\ge 0$, $|X_{n+1}-X_n|\le C$ holds almost surely.
\end{itemize}	
	
\end{theorem}


In our NPD problem, as the score statements accumulate weights in a multiplicative fashion, the classical OST cannot be applied since the bounded difference condition may be violated. To address this difficulty, we propose a novel variant of OST that tackles the multiplicative feature from score statements.



\begin{theorem}[OST Variant]\label{thm:ost-variant}
Let $\{X_n\}_{n=0}^\infty$ be a %martingale 
supermartingale 
adapted to a filtration 
$\mathcal{F}=\{\mathcal{F}_n\}_{n=0}^\infty$, and $\kappa$ be a stopping time w.r.t. the filtration $\mathcal{F}$. 
%$\{\mathcal{F}_n\}_{n=0}^\infty$. 
Then the following condition $(\mho)$ is sufficient to ensure that $\expv\left(|X_\kappa|\right)<\infty$ and %$\expv\left(X_\kappa\right) = \expv(X_0)$ 
$\expv\left(X_\kappa\right)\le\expv(X_0)$:
\begin{itemize}
\item[$(\mho)$] There exist integers $b_1,b_2>0$ and real numbers $c_1>0,c_2>c_3> 0$ such that (i) $\probm(\kappa>n) \leq c_1 \cdot e^{-c_2 \cdot n}$ for sufficiently large $n \in \Nset$, and (ii) for all $n \in \Nset$, $\left\vert X_{n+1}-X_n \right\vert \le b_1\cdot n^{b_2}\cdot e^{c_3\cdot n}$ holds almost surely.
\end{itemize}
\end{theorem}



Our OST variant extends the classical OST with the relaxation that we allow the magnitude of the next random variable $X_{n+1}$ to be bounded by that of $X_n$ with a multiplicative factor $e^{c_3}$, which corresponds to the multiplicative feature from score statements. The intuition of the theorem is that 
to cancel the effect of the multiplicative factor, we require in extra the exponential decrease in $\probm(\kappa>n) \leq c_1 \cdot e^{-c_2 \cdot n}$. The proof resembles \cite[Theorem 5.2]{cost2019wang} and is relegated to \cref{app:ost-variant-proof}. 




Below we show how our OST variant can be applied to handle score-recursive WPTS's. Fix a WPTS $\Pi$ in the form of \eqref{eq:wpts}. 
%The key concepts in the application of the OST variant are \emph{potential weight functions} over the WPTS $\Pi$ as follows. 
In the rest of this subsection, we reuse the expected-weight transformer $\ewt$ defined in \cref{def:ewt} and  potential weight functions given in \cref{def:puwf}. The slight difference is that for the expected-weight transformer in the context of a score-recursive WPTS, we consider that the weight function $\wet_j$ before termination may not be constantly $1$. 



To apply our multiplicative OST variant, we impose a bounded update requirement as in \cite{cost2019wang}. A WPTS $\Pi$ has the \emph{bounded update} property 
%over its program variables, 
if there exists a constant $\varkappa>0$ such that for every reachable state $(\loc,\pv)$, if $\tau=\langle \loc,\phi, F_1,\dots,F_k \rangle$ is the unique transition with each fork $F_j=\langle \loc'_j,p_j,\upd_j,\wet_j \rangle$ such that $\pv\models \phi$, then we have that 
	%\[
	$\forall \rv\in\supp{\rdvarjdis}\,~\forall x\in \pvars,~~ |\upd_j(\pv,\rv)(x)-\pv(x)|\le \varkappa$.

We apply our OST variant (Theorem~\ref{thm:ost-variant}) in a way similar to \cite[Theorem 6.10 and Theorem 6.12]{cost2019wang} to obtain the main theorem for our OST-based approach. To be more precise, we construct a stochastic process from a PUWF (or the negative of a PLWF) and show that the stochastic process is a supermartingale and the condition $(\mho)$ is fulfilled. The statement of the theorem is as follows.  

\begin{theorem}[OST Approach]\label{thm:puwf-normalizing}
Let $\Pi$ be a score-recursive WPTS that has the bounded update property. Suppose that there exist real numbers $c_1>0$ and $c_2>c_3>0$ such that 
\begin{itemize}
\item[(E1)] $\probm(T>n) \leq c_1 \cdot e^{-c_2 \cdot n}$ for sufficiently large $n\in\Nset$, and 
\item[(E2)] for each score function $\wet$ in $\Pi$ we have $|\wet|\le e^{c_3}$. 
\end{itemize}
 Then for any PUWF (resp. PLWF) $h$ over $\Pi$, we have that $\llbracket \Pi\rrbracket_{\valin} (\Rset^{|\pvars|})\le h(\lin,\valin)$ (resp. $\llbracket \Pi\rrbracket_{\valin} (\Rset^{|\pvars|})\ge h(\lin,\valin)$) for any initial state $(\lin,\valin)$, respectively.  
\end{theorem}

\begin{proof}[Proof Sketch.]
For the upper bounds, we define the stochastic process $\{X_n\}_{n=0}^\infty$ as $X_n:=h(\loc_n,\pv_n)$ where $(\loc_n,\pv_n)$ is the program state at the $n$th step of a program run. Then we construct a stochastic process $\{Y_n\}_{n=0}^\infty$ such that $Y_n:=X_n\cdot \prod_{i=0}^{n-1} W_i$ where $W_i$ is the weight at the $i$th step of the program run. We consider the termination time $T$ of $\Pi$ and prove that $\{Y_n\}_{n=0}^\infty$ satisfies the prerequisites of our OST variant (\cref{thm:ost-variant}). This proof depends on the assumption that $\Pi$ has concentration and bounded update properties, and the score functions in $\Pi$ are also bounded (Item 2, \cref{thm:puwf-normalizing})). Then by applying \cref{thm:ost-variant}, we obtain that $\expect{Y_T}\le \expect{Y_0}$. By (C2) in \cref{def:puwf}, we have that $Y_T=h(\loc_T,\pv_T)\cdot \prod_{i=0}^{T-1} W_i=\widehat{w}_T$. Thus, we have that $\llbracket \Pi\rrbracket_{\valin} (\Rset^{|\pvars|})=\expectdist{\valin}{\widehat{w}_T}=\expect{\prod_{i=0}^{T-1} W_i}\le \expect{Y_0}=h(\lin,\valin)$. For the lower bounds, the proof is similar. The more detailed proof is relegated to~\cref{app:ost}.
\end{proof}






\subsection{Truncation over WPTS's} \label{sec:truncation}

%Below we further introduce a truncation operation over WPTS's to tighten the lower and upper bounds derived by the two theoretical approaches above. 
In the following, we propose a truncation operation for a WPTS that restricts the value of every program variable in the WPTS to a prescribed bounded range. We consider that a bounded range for a program variable could be either $[-R,R]$ ($R> 0$), or $[0,R], [-R,0]$ if the value of the program variable is guaranteed to be non-negative or non-positive. 

To present our truncation operation, we define the technical notions of truncation and approximation functions.  
%first give the definition for a truncating function that %the formal definition of 
A \emph{truncation function} 
%$\trunc:\pvars\to \mathbb{I}_\Rset$ 
$\trunc$ is a function that maps every program variable $x\in\pvars$ to a bounded interval $\trunc(x)$ in $\Rset$ that specifies the bounded range of the variable $x$. We denote by $\Phi_\trunc$ the 
%logical 
formula $\bigwedge_{x\in\pvars} x\in \trunc(x)$
for a truncation function $\trunc$. 
An \emph{approximation function} is a function $\calM:\mathbb{R}^{|\pvars|}\to [0,\infty)$ such that each $\calM(\pv)$ ($\pv\in \mathbb{R}^{|\pvars|}$) is intended to be an over- or under-approximation of the expected weight 
%acts as an upper/lower bound for the expected weight 
$\llbracket \Pi\rrbracket_{\pv} (\Rset^{|\pvars|})$ 
outside the bounded range specified by $\Phi_\trunc$. 
The truncation operation is given by the following definition. 



\begin{definition}[Truncation Operation]\label{def:truncation}
Let 
%$\Pi = (\pvars, \rvars, \mathcal{D}, L,\transset, \lin,\valin, \lout,\win)$ 
$\Pi$ be a WPTS in the form of \eqref{eq:wpts}. Given a truncation function $\trunc$ and an approximation function $\calM$,
%that maps every program variable $x\in\pvars$ to a bounded interval $B(x)$ in 
%$\mathbb{R}$ (that specifies the bounded range of the variable $x$), 
the \emph{truncated} WPTS $\Pi_{\trunc,\calM}$ w.r.t. $\trunc$ and $\calM$ is defined as
	$\Pi_{\trunc,\calM}:=( \pvars, \rvars, L\cup\{\#\}, \lin, \lout,\mu_{\mathrm{init}},\rdvarjdis, \transset_{\trunc,\calM})$
	where $\#$ is a fresh deadlock location and the transition relation $\transset_{\trunc,\calM}$ is given by
	\begin{align*}
	&\transset_{\trunc,\calM}:=\{\langle \loc, \phi\wedge \Phi_\trunc, F_1, \dots, F_k \rangle\mid \langle \loc, \phi, F_1,\dots, F_k \rangle\in\transset\mbox{ and } \loc\ne \lout\}\\
 &\quad\cup \{\langle \loc, \phi\wedge (\neg\Phi_\trunc), F^{\calM,\sharp}_1, \dots, F^{\calM,\sharp}_k \rangle\mid \langle \loc, \phi, F_1,\dots, F_k \rangle\in\transset\mbox{ and } \loc\ne \lout\}\tag{\ddag}\\
 &\quad\cup\{\langle \lout, \mathbf{true}, F_{\lout}\rangle, \langle \sharp, \mathbf{true}, F_\sharp\rangle \}
\end{align*}
for which (a) we have $F_\loc:=\langle \loc,1,\mbox{\sl id},\overline{1} \rangle$ ($\loc\in\{\lout,\sharp\}$) where
 %$F'_\sharp:=\langle \sharp,1,\mbox{\sl id},\mathbf{1} \rangle$  
 $\mbox{\sl id}$ is the identity function and $\overline{1}$ is the constant function that always takes the value $1$, and (b) for a fork $F=\langle \loc', p, \mbox{\sl upd}, \wet\rangle$ in the original WPTS $\Pi$ we have 
$F^{\calM,\sharp}:=F$ if $\loc'=\lout$ and $F^{\calM,\sharp}:=\langle \sharp, p, \mbox{\sl upd}, \calM\rangle$ otherwise.
\end{definition}

Thus, the truncated WPTS is obtained from the original one by first restraining each transition to the bounded range $\Phi_\trunc$ 
%for all $x\in\pvars$ 
and then redirecting to the fresh deadlock location $\sharp$ all the situations jumping out of the bounded range and not going to the termination location. To make the truncated WPTS deterministic and total, we add the self-loop $\langle \sharp, \mathbf{true}, F_\sharp\rangle$. %to the location 
Our main theorem shows that by choosing an appropriate approximation function $\calM$ in the truncation, 
%on the transitions to the deadlock location $\sharp$ in the truncation, 
one can obtain upper/lower approximation of the original WPTS. 


\begin{theorem}\label{thm:upperlower}
Let $\Pi$ be a WPTS in the form of \eqref{eq:wpts}, $\trunc$ a truncation function and $\calM$ an approximation function.
Suppose that the following condition ($\ast$) holds:
\begin{itemize}
\item[($\ast$)] for each fork $F^{M,\sharp}=\langle \sharp, p, \mbox{\sl upd}, \calM\rangle$ in the truncated WPTS $\Pi_{\trunc,\calM}$ that is derived from
some fork $F=\langle \loc', p, \mbox{\sl upd}, \wet\rangle$ with the source location $\loc$ in the original WPTS (see sentence (b) in Definition~\ref{def:truncation}),   
we have that $\llbracket \Pi\rrbracket_{\pv}(\Rset^{|\pvars|})\le \calM(\pv)$ for all $\pv$ such that the state $(\loc,\pv)$ is reachable and $\pv\not\models\Phi_\trunc$. 
\end{itemize}
Then $\llbracket \Pi\rrbracket_{\valin} (\Rset^{|\pvars|})\le \llbracket \Pi_{\trunc,\calM}\rrbracket_{\valin}(\Rset^{|\pvars|})$ for all initial program valuations $\valin$. 
Analogously, if it holds the condition ($\star$) which is almost the same as ($\ast$) except for that ``$\llbracket \Pi\rrbracket_{\pv}(\Rset^{|\pvars|})\le \calM(\pv)$'' is replaced with ``$\llbracket \Pi\rrbracket_{\pv}(\Rset^{|\pvars|})\ge \calM(\pv)$'', then we have $\llbracket \Pi\rrbracket_{\valin} (\Rset^{|\pvars|})\ge \llbracket \Pi_{\trunc,\calM}\rrbracket_{\valin}(\Rset^{|\pvars|})$ for all initial program valuations $\valin$. 
\end{theorem}

The theorem above states that if the approximation function gives correct bounds for the expected weights of the original WPTS outside the bounded range, then the bounds for the expected weights of the truncated WPTS are also correct bounds for the expected weights of the original WPTS. The detailed proof is relegated to \cref{app:sec5}.




\begin{example}\label{ex:pedestrian-trunc}
Recall the Pedestrian example in \cref{fig:pedestrian-program}, here we make truncation to this example and generate its truncated WPTS. The truncation function $\trunc$ is defined such that $\trunc(pos)=[0,5]$, $\trunc(dis)=[0,5]$, so $\Phi_\trunc=0\le pos\le 5\wedge 0\le dis\le 5$. 
%Let the approximation function $\calM=0$. 
Following the truncation operation in ($\ddag$) from Definition~\ref{def:truncation}, we can obtain $\Pi_{\trunc,\calM}$ as shown in \cref{fig:truncated-wpts}.\qed



% Figure environment removed

\end{example}







\section{Algorithmic Approaches}\label{sec:algorithm}
\begin{algorithm}[!t]
    \caption{\method{}}\label{alg: iterative training}
    \begin{algorithmic}
        \State Input: dataset $\gD$, oracle $\oracle$, balanced synthetic dataset size $N$
        \State $i \leftarrow 0$
        \State $\theta_i \leftarrow \argmin_\theta \gL^{\text{DM}}_\theta$ \Comment{train baseline DM, \cref{eq:score-matching}}
        \State ${s_i} \leftarrow s_{\theta_i}(\rvx_t; t)$
        \While{not done}
            \State $\synth^+_i,\synth^-_i \leftarrow$ generate samples from DM with score function ${s_i}$ and label with $\oracle$
            \While{$\min(|\synth^+_i|,|\synth^-_i|)<N$}
             \State $\synth^+,\synth^- \leftarrow$ generate more samples  from DM with score function ${s_i}$ and label with $\oracle$
             \State $\synth^+_i \leftarrow \synth^+_i \cup \synth^+$, $\synth^-_i \leftarrow \synth^-_i \cup \synth^-$
            \EndWhile
            \State $\alpha_i \leftarrow |\synth^+_i| / (|\synth^+_i| + |\synth^-_i|)$ \Comment{Estimate class prior probabilities for Bayes optimal classifier} 
            \State $\synth^+_i \leftarrow \subsample(N,\synth^+_i), \synth^-_i \leftarrow \subsample(N,\synth^-_i)$ \Comment{balance dataset for IS classifier training}
            \State $\phi_i \leftarrow \argmin_\phi \hat{\gL}^{\text{cls}}_\phi(\alpha_i, \synth^+_i, \synth^-_i)$ \Comment{train guidance classifier, \cref{eq: classifier loss}}
            \State $i \leftarrow i+1$

            \If{distill}
                \State $\psi \leftarrow \argmin_\psi \gL^{\text{dtl}}_\psi$\Comment{distill into single DM, \cref{eq: distillation}}
                \State ${s_i} \leftarrow  s_{\psi}(\rvx_t; t)$ 
            \Else  \Comment{``stack'' guidance classifiers}
                \State ${s_i} \leftarrow {s_{i-1}} + \nabla_{\rvx_t}\log C_{\phi_i}(\rvx_t; t)$ \Comment{See \cref{eq:gen-neg-score}}
            \EndIf
        \EndWhile
        \State \Return DM score function $s_i$
    \end{algorithmic}
\end{algorithm}


\section{Experimental Results}\label{sec:experiment}
\section{Experiments}
% \haizhou{Follow the same way of introduction as we did in Section2.}
% \noindent In this section, we will introduce datasets and experimental setups that we used. Then we evaluate our method, other self-supervised methods, and supervised methods under different distribution shifts (\ie, concept shifts and covariate shifts) under common settings (\ie, transductive, inductive settings). It has to note that we focus on node-level tasks (\eg, node classification) in this work. As for graph-level tasks, we leave it as our future work and some simple experiments can be found in Appendix~\ref{app:graph_classification}. 
In this section, we first introduce the experimental setup including datasets, training, and evaluation protocol in Section~\ref{sec:dataset}~and~\ref{sec:unsupervised}. 
% Next, we present our experimental setup and conduct extensive experiments to evaluate our method in Section~\ref{sec:unsupervised}. 
We then perform an ablation study to demonstrate the effectiveness of each proposed component in Section~\ref{sec:ablation}. 
Additionally, we analyze the impact of important hyper-parameters in Section~\ref{sec:sensitivity}. 
Subsequently, we integrate our method with various encoding models, showcasing the model-agnostic nature of our recipe in Section~\ref{sec:other_models}. 
Finally, we provide some qualitative results such as feature visualization in Section~\ref{sec:vis}.
It is important to note that we focus on node-level tasks (\eg, node classification) in this work. As for graph-level tasks, we leave it as our future work, while some simple experiments are also provided in Appendix~\ref{app:graph_classification}.

\subsection{Datasets}\label{sec:dataset}
There exist some benchmarks for evaluating graph out-of-distribution generalization~\cite{good,ji2022drugood,gds}. 
Among them, GOOD~\cite{good} is the most representative and comprehensive benchmark that curates more diverse graph datasets with diverse tasks, including single/multi-task graph classification, graph regression, and node classification involving more distribution shifts (\ie, concept shifts and covariate shifts). Hence in this work, we follow the evaluation protocol proposed in \cite{good}. Furthermore, we validate the effectiveness of our method in the datasets (\ie, Amazon-Photo, Elliptic) that are used in EERM~\cite{eerm}. The statistics and detailed introduction to these datasets can be found in Table~\ref{tab:dataset} and Appendix~\ref{app:datasets}.

\begin{table*}[htp]
\caption{The descriptions of datasets. ``Domain-Level'' means splitting by graphs, ``Time-Aware'' denotes splitting according to chronological order.``Word'' and ``Degree'' represent splitting according to word diversity and node degree respectively. ``Language'' means splitting by user language, suggesting the prediction should not be impacted by the language the user use. ``University'' denotes splitting according to the domain university, implying that the prediction of webpages should be based on word contents and link connections rather than university features. ``Color'' means that nodes are split according to node differences in covariate shift and color-label correlations in concept shift.}
\label{tab:dataset}
\centering
\begin{tabular}{cccccccc}
\toprule
Datasets     & Network Type        & \#Nodes & \#Edges & \#Attributes &\#Classes& Train/Val/Test Split     & Metric   \\
% Cora         & Artificial Transformation & 2,703   &         &              &         &                      & Accuracy \\
Amazon-Photo\footnotemark
             & Co-purchasing network      & 7,650   & 119,081   & 755          & 10      & Domain-Level         & Accuracy \\
Elliptic\footnotemark  
             & Bitcoin transactions       & 203,769 & 234,355   & 165          & 2       & Time-Aware           & F1-Score \\
GOOD-Cora    & Scientific publications    & 19,793  & 126,842   & 8,710         & 70      & Word/Degree          & Accuracy \\
% GOOD-Arxiv   & arXiv papers               & 169,343 & 2,315,598 & 128          & 40      & Time/Degree          & Accuracy \\
GOOD-Twitch  & Gamer network              & 34,120  & 892,346   & 128          & 2       & Language             & ROC-AUC  \\
GOOD-CBAS    & A BA-house graph           & 700     & 3,962     & 4             & 4       & Color                & Accuracy \\
GOOD-WebKB   & Webpage network            & 617     & 1,138     & 1,703         & 5       & University           & Accuracy \\
\bottomrule
\end{tabular}
\end{table*}
\footnotetext[5]{This dataset is adopted from~\cite{yang2016revisiting}. \cite{eerm} constructs ten graphs with different environment id’s for each graph.} 
\footnotetext[6]{The original is available on \hyperlink{https://www.kaggle.com/ellipticco/elliptic-data-set}{https://www.kaggle.com/ellipticco/elliptic-data-set}}

\subsection{Unsupervised Representation Learning}\label{sec:unsupervised}
\subsubsection{Transductive Setting}~\label{sec:trans}
% \noindent\textbf{Baselines.}\quad We conduct experiments with 12 baselines which consist of three categories: supervised methods and self-supervised generative methods, self-supervised contrastive methods. Specifically, we compare with three supervised baselines: empirical risk minimization~(ERM)~\cite{erm}, invariant risk minimization (IRM)~\cite{irm}, and a recent proposed graph OOD method dubbed EERM~\cite{eerm}. We also compare various unsupervised node-level representation learning methods: three self-supervised generative methods including GAE~\cite{gae}, VGAE~\cite{gae}, GraphMAE~\cite{gmae} and seven self-supervised contrastive methods: DGI~\cite{dgi}, MVGRL~\cite{mvgrl}, GRACE~\cite{grace}, RoSA~\cite{rosa}, BGRL~\cite{bgrl}, COSTA~\cite{costa}, SwAV~\cite{swav}. The descriptions of these methods can be found in Appendix~\ref{app:baselines}.
In this subsection, we focus on validating our proposed algorithm under the transductive setting, where the test nodes will participate in message passing~\cite{gilmer2017neural} during training following~\cite{good}. 

\noindent\textbf{Baselines.} We conduct experiments with 12 baselines from three categories: (i)~supervised methods, including empirical risk minimization~(\textbf{ERM})~\cite{erm}, invariant risk minimization (\textbf{IRM})~\cite{irm}, and a recent proposed graph OOD method \textbf{EERM}~\cite{eerm}; (ii)~self-supervised generative methods including Graph Autoencoder (\textbf{GAE})~\cite{gae}, Variational Graph Autoencoder (\textbf{VGAE})~\cite{gae}, Self-Supervised Masked Graph Autoencoders (\textbf{GraphMAE})~\cite{gmae}; (iii)~self-supervised contrastive methods including Deep Graph Infomax (\textbf{DGI})~\cite{dgi}, Contrastive Multi-View Representation Learning on Graphs (\textbf{MVGRL})~\cite{mvgrl}, Deep Graph Contrastive Representation Learning (\textbf{GRACE})~\cite{grace}, A Robust Self-Aligned Framework for Node-Node Graph Contrastive Learning (\textbf{RoSA})~\cite{rosa}, Bootstrapped Representation Learning on Graphs (\textbf{BGRL})~\cite{bgrl}, Covariance-Preserving Feature Augmentation for Graph Contrastive Learning (\textbf{COSTA})~\cite{costa}, Unsupervised Learning of Visual Features by Contrasting Cluster Assignments (\textbf{SwAV})~\cite{swav}. The detailed descriptions of these baselines can be found in Appendix~\ref{app:baselines}.

\noindent\textbf{Experimental setup.} We use the same graph encoder across different datasets for a fair comparison following~\cite{good}. We use grid search to find other hyper-parameters (\eg, learning rate, epochs) for different methods. For all experiments, we select the best checkpoints for ID and OOD tests according to results on ID and OOD validation sets following~\cite{good}, respectively. Experimental details and hyper-parameter selections are provided in Appendix~\ref{app:hyper}. For evaluating unsupervised methods, a linear classifier will be built on the frozen trained encoder after finishing pre-training. The reported results are the mean performance with standard deviation after 10 runs following~\cite{good}.

\noindent\textbf{Analysis.}\quad Based on the experimental results listed in Table~\ref{tab:trans_concept} and \ref{tab:trans_covariate}, we can draw the following conclusions: firstly, we find strong self-supervised methods (\eg, GRACE, BGRL, COSTA) are more robust to distribution shifts (concept shift in Table~\ref{tab:trans_concept} and covariate shift in Table~\ref{tab:trans_covariate}) compared to supervised methods. For instance, on GOOD-CBAS and GOOD-WebKB datasets, GRACE surpasses the best supervised method by large margins (over 6\% absolute improvement). Interestingly, we find the methods designed for OOD generalization (\ie, IRM) and graph OOD generalization (\ie, EERM) do not attain superior performance than the standard ERM on most of the datasets. For example, EERM shows superior OOD performance compared to ERM in only one experiment, and IRM outperforms ERM in four out of ten experiments across the conducted evaluations. This phenomenon is also observed in \cite{good,ahuja2020empirical,rosenfeld2021risks}, showcasing the challenge of achieving invariant prediction in non-Euclidean graph settings. 

Furthermore, our method surpasses other SOTA self-supervised methods on the OOD test set of all datasets by a considerable margin while achieving comparable performance in the in-distribution test set. For instance, on small datasets such as GOOD-CBAS and GOOD-WebKB, our method outperforms GRACE\footnote{MARIO is built up on GRACE according to our recipe. So, we make a comparison with GRACE here.} by over 2\% absolute accuracy on the OOD test set. On larger datasets such as GOOD-Cora and GOOD-Twitch, our method still outperforms other methods which shows its superiority. For instance, under covariate shift, MARIO surpasses other methods by over 7\% absolute accuracy on the GOOD-Twitch OOD test set. These statistics prove the effectiveness of our design.


\begin{table*}[htp]
\caption{Experimental results of all methods under concept shift. The bold font means the top-1 performance and the underline represents the second performance across the unsupervised methods. 'ID' represents in-distribution test performance and 'OOD' means out-of-distribution test performance. (OOM: out-of-memory on a GPU with 24GB memory)}
\label{tab:trans_concept}
\centering
\scalebox{0.95}{
\begin{tabular}{l|cc|cc|cc|cc|cc}
\toprule
\toprule
\multirow{3}{*}{concept shift} & \multicolumn{4}{c|}{GOOD-Cora}                   & \multicolumn{2}{c|}{GOOD-CBAS} & \multicolumn{2}{c|}{GOOD-Twitch} & \multicolumn{2}{c}{GOOD-WebKB} \\
                           & \multicolumn{2}{c}{word} & \multicolumn{2}{c|}{degree}& \multicolumn{2}{c|}{color}    & \multicolumn{2}{c|}{language}   & \multicolumn{2}{c}{university} \\
                           & ID         & OOD         & ID          & OOD          & ID            & OOD           & ID             & OOD            & ID            & OOD            \\
\midrule
ERM                        & 66.38±0.45 & 64.44±0.18  & 68.60±0.40  & 60.76±0.34   & 89.79±1.39    & 83.43±1.19    & 80.80±1.00     & 56.92±0.92     & 62.67±1.53    & 26.33±1.09     \\
IRM                        & 66.42±0.41 & 64.29±0.31  & 68.57±0.35  & 61.45±0.24   & 89.64±1.21    & 82.29±1.14    & 78.87±1.04     & 59.30±1.79     & 62.67±1.10    & 26.88±1.42     \\
EERM                       & 65.10±0.44 & 62.45±0.19  & 66.95±0.44  & 56.58±0.25   & 79.07±2.12    & 64.50±1.01    & OOM            & OOM            & 62.50±2.01    & 28.07±3.23      \\
\midrule
% Random-Init                & 37.53±1.74 & 32.12±1.24  & 37.82±1.71  & 27.74±1.14   &               &               &                &                & 60.33±2.21    & 27.07±1.70     \\
GAE                        & 60.65±0.89 & 58.00±0.55  & 62.59±1.11  & 53.44±0.80   & 75.28±1.36    & 68.07±2.05    & 81.25±0.81     & 51.51±1.05     & 62.17±3.34    & 25.78±1.85     \\
VGAE                       & 63.19±0.53 & 60.35±0.47  & 61.65±0.66  & 54.28±0.28   & 76.50±0.50    & 59.07±0.56    & 80.46±0.53     & 55.56±4.53     & 62.50±2.38    & 24.40±2.57     \\
GraphMAE                   & \underline{66.44±0.46} & \underline{64.87±0.30}  & 67.95±0.46  & 59.41±0.39   & 89.14±0.89    & 82.93±0.93    & 80.05±0.64     & 59.38±1.49     & 61.83±3.37    & 29.27±2.15     \\
DGI                        & 63.33±0.56 & 60.71±0.49  & 65.93±1.02  & 55.83±0.53   & 91.22±1.47    & 85.00±1.66    & 80.05±0.87     & 59.16±1.88     & 61.83±2.83    & 28.63±1.92      \\
MVGRL                      & OOM        & OOM         & OOM         & OOM          & 88.57±1.15    & 76.50±1.17    & OOM            & OOM            & 62.00±3.79    & 28.26±4.20     \\
GRACE                      & 65.61±0.61 & 63.92±0.44  & \textbf{68.59±0.35}  & 60.15±0.45   & 92.00±1.39    & 88.64±0.67    & \textbf{83.43±0.63}     & \underline{60.45±1.46}     & 64.00±3.43    & \underline{34.86±3.43}  \\
RoSA                       & 64.06±0.67 & 62.44±0.39  & 67.07±0.65  & 57.68±0.44   & 90.78±2.27    & 85.93±2.14    & 82.39±0.42     & 57.45±2.16     & 64.17±4.10    & 32.20±2.15     \\
BGRL                       & 65.18±0.43 & 63.43±0.45  & 66.83±0.80  & 59.63±0.38   & 92.36±1.16    & 87.14±1.60    & 82.52±0.60     & 55.48±1.48     & 63.67±2.33    & 31.47±3.43     \\
COSTA                      & 65.05±0.80 & 62.37±0.45  & 66.76±0.87  & 55.73±0.36   & \underline{93.50±2.62}    & \underline{89.29±3.11}    & 83.15±0.30 & 55.03±3.22     & 61.66±2.58    & 32.39±2.13 \\
% ArCL                       &            &             & 67.64±0.57  & 59.71±0.44   &               &               &                &                & 65.00±3.94    & 35.41±1.97 \\      
SwAV                       & 62.22±0.53 & 59.79±0.53  & 64.65±0.94  & 55.06±0.39   & 89.00±0.79    & 81.72±0.66    & \underline{83.32±0.15}     & 59.69±1.97     & \underline{65.17±3.76}    & 29.36±2.01    \\
\midrule
MARIO                       & \textbf{67.11±0.46} & \textbf{65.28±0.34}  & \underline{68.46±0.40}  & \textbf{61.30±0.28}   & \textbf{94.36±1.21}    & \textbf{91.28±1.10}    & 82.31±0.54     & \textbf{63.33±1.72}     & \textbf{65.67±2.81}    & \textbf{37.15±2.37}     \\
\bottomrule
\end{tabular}}
\end{table*}

\begin{table*}[htp]
\caption{Experimental results of all methods under covariate shift. The bold font means the top-1 performance and the underline represents the second performance across the unsupervised methods. 'ID' represents in-distribution test performance and 'OOD' means out-of-distribution test performance. (OOM: out-of-memory on a GPU with 24GB memory)}
\label{tab:trans_covariate}
\centering
\scalebox{0.95}{
\begin{tabular}{l|cc|cc|cc|cc|cc}
\toprule
\toprule
\multirow{3}{*}{covariate shift} & \multicolumn{4}{c|}{GOOD-Cora}                                   & \multicolumn{2}{c|}{GOOD-CBAS} & \multicolumn{2}{c|}{GOOD-Twitch} & \multicolumn{2}{c}{GOOD-WebKB} \\
                           & \multicolumn{2}{c}{word} & \multicolumn{2}{c|}{degree}& \multicolumn{2}{c|}{color}    & \multicolumn{2}{c|}{language}   & \multicolumn{2}{c}{university} \\
                           & ID         & OOD         & ID          & OOD          & ID            & OOD           & ID             & OOD            & ID            & OOD            \\
\midrule
ERM                        & 70.50±0.41 & 64.69±0.33  & 72.46±0.49  & 55.53±0.50   & 92.00±3.08    & 77.57±1.29    & 70.98±0.41     & 49.35±5.09     & 39.34±1.79    & 14.52±3.14   \\
IRM                        & 70.48±0.26 & 64.53±0.57  & 71.98±0.34  & 53.72±0.46   & 90.86±2.41    & 78.86±1.67    & 69.81±0.95     & 49.11±2.82     & 38.52±3.30    & 13.97±2.80     \\
EERM                       & OOM        & OOM         & OOM         & OOM          & 65.00±2.57    & 57.43±3.60    & OOM            & OOM            & 46.07±4.55    & 27.40±7.65     \\
\midrule
GAE                        & 56.63±0.79 & 48.93±0.93  & 66.30±0.88  & 34.01±0.87   & 73.00±2.16    & 60.86±3.01    & 67.24±1.23     & 47.65±2.49     & 45.08±6.32    & 28.02±6.29    \\
VGAE                       & 62.02±0.66 & 54.12±0.86  & 69.41±0.57  & 44.20±1.29   & 62.29±2.04    & 63.29±1.11    & 66.99±1.43     & \underline{50.48±4.58}     & 48.85±4.68    & 20.87±6.69     \\
GraphMAE                   & 68.14±0.43 & 64.00±0.33  & \textbf{73.36±0.56}  & 53.75±0.55   & 67.28±3.03    & 67.28±1.49    & 68.84±1.20     & 48.02±2.79     & 48.03±4.34    & 30.00±8.09     \\
DGI                        & 60.85±0.75 & 57.03±0.67  & 68.97±0.41  & 41.75±0.88   & 69.57±4.09    & 59.71±3.43    & 68.43±1.05     & 44.83±1.61     & 48.52±5.04    & 21.11±7.50     \\
MVGRL                      & OOM        & OOM         & OOM         & OOM          & 65.00±1.94    & 64.15±0.77    & OOM            & OOM           & \textbf{54.10±5.39}    & 16.59±6.51     \\
GRACE                      & \underline{68.77±0.33} & \underline{64.21±0.41}  & 72.69±0.34  & \underline{56.10±0.63}   & \underline{93.57±1.83}    & \underline{89.29±3.40}    & \underline{71.12±0.87} & 46.21±1.54 & 49.67±5.82    & 28.10±4.68    \\
RoSA                       & 68.19±0.56 & 62.48±0.61  & 71.04±0.62  & 52.72±0.79   & 84.71±4.14    &79.14±3.51     & 70.58±0.36     & 45.83±1.72     & 52.30±4.24    & \underline{34.24±7.92}     \\
BGRL                       & 67.23±0.43 & 61.33±0.36  & 72.11±0.39  & 49.15±0.73   & 89.00±2.56    & 79.86±3.29    & \textbf{71.43±0.53}     & 43.86±0.94     & 51.80±5.55    & 30.32±7.61    \\
COSTA                      & 65.28±0.60 & 60.33±0.53  & 70.65±0.62  & 54.03±0.28   & 92.29±1.59    & 82.71±2.74    & 69.29±1.37     & 49.07±2.13     & 50.49±3.01    & 29.84±4.75   \\
SwAV                       & 63.29±1.01 & 56.98±0.94  & 70.27±0.73  & 43.00±0.52   & 89.57±1.12    & 81.43±1.69    & 69.19±0.93     & 49.37±2.96     & 49.84±4.82    & 30.55±6.72   \\
\midrule
MARIO                       & \textbf{69.99±0.54} & \textbf{65.06±0.34}  & \underline{72.73±0.43}  & \textbf{57.73±0.45}  & \textbf{94.57±2.46}    & \textbf{91.00±2.48}     & 68.31±0.78 & \textbf{57.37±1.37}     & \underline{53.94±3.23}    & \textbf{35.24±4.98}   \\
\bottomrule
\end{tabular}}

\end{table*}

\subsubsection{Inductive Setting}
In this subsection, we conduct experiments under the inductive settings, where the test nodes are kept unseen during training. This setting is more suitable for domain generalization.
% But we think it is more convincing that conduct experiments under inductive settings which means test nodes are unseen during training. This setting is more appropriate for domain generalization.

\noindent\textbf{Baselines:} For GOOD-WebKB and GOOD-CBAS datasets, we adopt ERM, IRM, GraphMAE, and GRACE as our baselines. And for Amazon-Photo and Elliptic datasets, we select ERM, EERM, and GRACE as our baselines.

\noindent\textbf{Experimental setup:} For GOOD-WebKB and GOOD-CBAS datasets, we use the same model configuration in Section~\ref{sec:trans}.
% Besides, we add experiments on Amazon-Photo dataset~\cite{yang2016revisiting} and Elliptic~\cite{elliptic} dataset in this subsection. 
For Amazon-Photo dataset~\cite{yang2016revisiting} and Elliptic~\cite{elliptic} dataset, they consist of many snapshots (training data and testing data use different snapshots) which are naturally inductive. For Amazon-Photo dataset, we use 2-layer GCN~\cite{gcn} as the encoder and for elliptic dataset, we use 5-layer GraphSAGE~\cite{sage} as encoder following~\cite{eerm}.

% Figure environment removed

\noindent\textbf{Analysis:}
According to Figure~\ref{fig:amazon},\ref{fig:elliptic},\ref{fig:ind_con},\ref{fig:ind_cov}, we can draw following conclusions:
firstly, based on Figure~\ref{fig:amazon}, it is evident that our method outperforms other representative supervised and self-supervised methods on all test graphs (T1$\sim$T8). This superiority is reflected in the larger median value of our method compared to others. For instance, MARIO achieves over a 3\% absolute improvement compared to ERM in terms of the mean value of eight median values. Additionally, our method demonstrates higher stability across different random initializations, as indicated by the closer proximity of the first and third quartile values to the median value~(\eg, the difference of first and third quartile values of ERM, EERM, GRACE and MARIO are 4.2, 3.3, 6.7 and 1.0 on T8 respectively which indicates MARIO is much more stable than other methods). Furthermore, our method exhibits consistent performance across different graphs (\eg, The standard deviation of median values on T1$\sim$T8 for ERM, EERM, GRACE, and MARIO are 0.4, 1.1, 1.2, and 0.3, respectively.), indicating its robustness to environmental variations and its ability to extract invariant features: $g(G^e) \approx g(G^{e'})$ for all $e, e' \in \mathcal{E}^\text{train}$. In summary, our method showcases enhanced OOD generalization capabilities.
% $g(G^e)g(G^e^\prime)$ where $any e, e^\prime in \mathcal{E}^{train}$

Secondly, from the results presented in Figure~\ref{fig:elliptic}, we can observe that our method averagely harvests 10.9\% absolute improvement over GRACE and 12.5\% absolute improvement over EERM in terms of F1 scores on Elliptic dataset. This demonstrates the effectiveness of our method in handling distribution shifts and improving performance compared to existing approaches. It is worth noting that GRACE's performance worsens over time, indicating its inability to handle distribution shifts effectively. In contrast, our method consistently achieves better F1 scores, except for T9, which is caused by the dark market shutdown occurred after T7~\cite{elliptic}. The emergence of such an event introduces significant variations in data distributions, which subsequently results in performance degradation for all methods. Indeed, this event serves as an unpredictable external factor that introduces significant challenges for models trained on limited training data. The results indicate that the performance heavily depends on available training data. Nonetheless, our approach outperforms other methods even in such an extreme case. This highlights the effectiveness of our method in addressing distribution shifts and improving generalization performance.

Finally, based on the observations from Figure~\ref{fig:ind_con} and Figure~\ref{fig:ind_cov} MARIO demonstrates the best performances on both ID and OOD test sets for GOOD-WebKB and GOOD-CBAS datasets, under both concept shift and covariate shift. Notably, MARIO outperforms other methods by more than 3\% and 10\% absolute improvement on GOOD-WebKB and GOOD-CBAS, respectively, under covariate shift. We can draw similar conclusions as discussed in Section~\ref{sec:trans}. Even under the inductive setting, our method continues to demonstrate excellent OOD generalization capabilities and achieves comparable or even improved in-distribution test performance. These statistical results further validate the effectiveness of our method in handling distribution shifts and enhancing generalization performance.

Overall, the observations we have made provide strong evidence of the great capacity of our method for handling distribution shifts, validating its effectiveness and potential for real-world applications.



% Figure environment removed

% Figure environment removed


% Figure environment removed


\subsection{Ablation Studies}\label{sec:ablation}
\noindent Table~\ref{tab:aba} provides a detailed analysis of the effect of each component according to our proposed recipe for improving OOD generalization in graph contrastive learning. Let's examine the different variants of our method and their impact on performance.
Specifically, MARIO~(w/o ad) represents MARIO without  adversarial augmentation. MARIO~(w/o cmi) denotes we only maximize the mutual information between positive pairs without considering conditional mutual information. MARIO~(w/o cmi, ad) means a vanilla graph contrastive method that is similar to GRACE. 

From Table~\ref{tab:aba}, we can find MARIO~(w/o cmi) lags far behind MARIO on OOD test set which demonstrates appropriately minimizing the redundant information (\ie, conditional mutual information) is essential to improve OOD generalization of GCL methods. And adversarial augmentation can also boost OOD generalization because it can approximately serve as a supermum operator to learn more invariant features  discussed in Section~\ref{sec:aug}. Based on the analysis of these variants, it is evident that the proposed improvements on data augmentation and contrastive loss in the recipe are both effective in enhancing graph OOD generalization. Each component contributes to the overall performance improvement, and their combination leads to a stronger self-supervised graph learner in terms of graph OOD generalization. 

In short, the findings from Table~\ref{tab:aba} support the rationale behind your proposed recipe and provide empirical evidence of the effectiveness of each proposed component. By incorporating these enhancements, our method achieves superior performance in handling distribution shifts and improving graph OOD generalization in graph contrastive learning.
\begin{table*}[htp]
\caption{Ablation studies for MARIO by masking each component.}
\label{tab:aba}
\centering
\scalebox{0.9}{
\begin{tabular}{l|cc|cc|cc|cc|cc}
\toprule
\toprule
\multirow{3}{*}{concept shift} & \multicolumn{4}{c|}{GOOD-Cora}                       & \multicolumn{2}{c|}{GOOD-CBAS} & \multicolumn{2}{c|}{GOOD-Twitch} & \multicolumn{2}{c}{GOOD-WebKB} \\
                           & \multicolumn{2}{c}{word} & \multicolumn{2}{c|}{degree}& \multicolumn{2}{c|}{color}    & \multicolumn{2}{c|}{language}   & \multicolumn{2}{c}{university} \\
                           & ID         & OOD         & ID          & OOD          & ID            & OOD           & ID             & OOD            & ID            & OOD            \\
\midrule
MARIO                      & \textbf{67.11±0.46} & \textbf{65.28±0.34}  & \textbf{68.46±0.40}  & \textbf{61.30±0.28}      & \textbf{94.36±1.21}  & \textbf{91.28±1.10}    & 82.31±0.54     & \textbf{63.33±1.72}     & \textbf{65.67±2.81}    & \textbf{37.15±2.37}     \\
MARIO(w/o ad)              & 66.23±0.53 & 64.02±0.18  & 67.88±0.38  & 60.46±0.29   & 93.21±1.25    & 90.29±0.91    & 82.42±0.73     & 60.50±1.02     & 64.83±2.83    & 36.51±3.25    \\
MARIO(w/o cmi)             & 65.32±0.60 & 63.51±0.32  & 68.14±0.32  & 61.19±0.34   & 94.15±1.23    & 90.57±1.96    & \textbf{82.51±0.56}     & 61.41±2.63     & 64.50±4.35    & 35.78±2.53     \\
MARIO(w/o cmi, ad)         & 64.67±0.55 & 63.11±0.32  & 67.95±0.65  & 60.01±0.57   & 93.36±1.66    & 89.64±1.73    & 81.90±0.75     & 60.12±1.60     & 64.17±3.67    & 34.13±2.38     \\
\bottomrule
\end{tabular}}
\end{table*}
% & 65.32±0.60 & 63.51±0.32 exchange 64.67±0.55 & 63.11±0.32
% 68.14±0.32       id ood test: 60.95±0.43       ood ood test: 61.19±0.34


\subsection{Sensitivity Analysis}\label{sec:sensitivity}
\noindent In this subsection, we will analyze some important hyper-parameters of our method. We conduct sensitivity analysis on GOOD-WebKB dataset with concept shift, we chose two sensitive hyper-parameters (\ie, the coefficient $\gamma$ of condition mutual information in Equation~\ref{equ:cmi} and the number of prototypes $|C|$ in Equation~\ref{equ:pq}). The coefficient of CMI range in $[0.001, 0.01, 0.1, 0.5, 1]$ and the number of prototypes $|C|$ ranges in $[10, 50, 100, 200, 300]$. From Figure~\ref{fig:sensitivity}, we can observe that $\gamma$ reaches 0.1 and $|C|$ reaches 100 or 200 can achieve the best OOD test accuracy. Both higher and lower values of $\gamma$ result in suboptimal performance. This finding aligns with previous research such as DIB~\cite{dib}, indicating that an appropriate compression level is crucial for achieving optimal performance. Extremely high or low compression values are not ideal. 

Regarding the number of prototypes $|C|$, based on the results shown in Figure~\ref{fig:sensitivity}, it is found that setting $|C|=100$ leads to the best performance in terms of OOD test accuracy. This choice provides a moderate number of pseudo labels, which is beneficial for the learning process. 

Based on the sensitivity analysis, we determined that setting $\gamma=0.1$ and $|C|=100$ on most datasets. These hyperparameter values strike a balance between compression level and the number of prototypes, resulting in improved graph OOD generalization.
% Figure environment removed


\subsection{Integrated with Other Models}\label{sec:other_models}
% Figure environment removed

\begin{table}[htp]
\caption{Results of different learning approaches with different encoding models (\ie, GCN, GraphSAGE, GAT).}
\label{tab:others}
\centering
\scalebox{0.9}{
\begin{tabular}{cc|cc|cc}
\toprule
\toprule
\multirow{3}{*}{Model}& \multirow{3}{*}{Method} & \multicolumn{2}{c|}{GOOD-CBAS} & \multicolumn{2}{c}{GOOD-WebKB} \\
                & & \multicolumn{2}{c|}{color}    & \multicolumn{2}{c}{university} \\
                &   & ID          & OOD         & ID          & OOD            \\
\midrule
\multirow{3}{*}{GCN} 
&ERM               & 89.79±1.39 & 83.43±1.19  &  62.67±1.53 & 26.33±1.09         \\
&GRACE             & 92.00±1.39 & 88.64±0.67  &  64.00±3.43 & 34.86±3.43        \\
&MARIO             & 94.36±1.21 & 91.28±1.10  &  65.67±2.81 & 37.15±2.37        \\ \bottomrule
\multirow{3}{*}{SAGE} 
&ERM               & 95.07±1.51 & 75.14±1.19  & 73.67±2.08  & 46.33±3.42       \\
&GRACE             & 95.29±1.11 & 74.43±2.36  & 70.50±5.06  & 49.54±3.83        \\
&MARIO             & 96.00±1.07 & 76.29±3.01  & 71.00±3.82  & 51.74±4.63        \\ \bottomrule
\multirow{3}{*}{GAT} 
&ERM               & 78.64±3.63 & 72.93±2.64  & 61.33±3.71  & 28.99±2.63        \\
&GRACE             & 84.57±1.79 & 78.36±1.60  & 59.50±2.36  & 35.78±3.26        \\
&MARIO             & 84.93±1.95 & 80.43±1.89  & 62.17±4.78  & 38.17±3.10        \\
\bottomrule
\end{tabular}}
\end{table}



\noindent In the subsection, we demonstrate the model-agnostic nature of the recipe by integrating it with various graph neural network (GNN) models, including GCN, GraphSAGE, and GAT.

From Table~\ref{tab:others}, it can be observed that regardless of the specific GNN model used as the encoder, our method consistently achieves the best performance on the OOD test set. This indicates the effectiveness and robustness of our method across different GNN models.
By achieving superior performance across different GNN models, MARIO demonstrates its versatility and ability to improve the OOD generalization of various graph neural models. This highlights the broad applicability and effectiveness of our recipe in enhancing the performance of different GNN encoders.

Furthermore, we integrate our recipe with other GCL methods in Appendix~\ref{app:other_methods}. The results demonstrate our recipe can boost the OOD generalization ability of various GCL methods which means our recipe can serve as a plug-in for many current classical GCL methods.

% Figure environment removed

\subsection{Visualization}\label{sec:vis}
\subsubsection{Metric Score Curves}
We present metric score curves for ERM and MARIO, including training, ID validation, ID testing, OOD validation, and OOD testing accuracy, in Figure~\ref{fig:curve2}. Notably, MARIO demonstrates superior convergence with approximately 10\% absolute improvement on the OOD test set compared to ERM. Furthermore, MARIO effectively narrows the performance gap between in-distribution and out-of-distribution performance, showcasing its efficacy in enhancing OOD generalization for graph data. More metric score curves can be found in Appendix~\ref{app:curves}.


\subsubsection{Feature Visualization}
In order to assess the quality of learned embeddings, we adopt t-SNE~\cite{tsne} to visualize the node embedding on GOOD-Cora dataset (concept shift in word domain) using random-init of GCN, EERM, GRACE, and MARIO, where different classes have different colors in Figure~\ref{fig:vis}. For clarity, we select eight classes with the largest number of nodes to enhance the informativeness and interpretability of the visualization. We can observe that the 2D projection of node embeddings learned by MARIO has a better separation of clusters, which indicates the model can help learn representative features for downstream tasks. It has to note that we depict both ID nodes and OOD nodes in the same figure. 

Besides, we also separately visualize ID nodes and OOD nodes in the different figures in the Appendix~\ref{app:feature}. And we can find MARIO performs a clearer separation of clusters whether on ID nodes or OOD nodes compared to other methods.





\section{Related Works}
\section{Related Work}
\label{appsec: related work}
Bayesian causal discovery literature has primarily focused on inference in linear models with closed-form posteriors or marginalized parameters. Early works considered sampling directed acyclic graphs (DAGs) for discrete~\cite{cooper1992bayesian, madigan1995bayesian, heckerman2006bayesian} and Gaussian random variables~\cite{friedman2003being, tong2001active} using Markov chain Monte Carlo (MCMC) in the DAG space. However, these approaches exhibit slow mixing and convergence~\cite{eaton2012bayesian,grzegorczyk2008improving}, often requiring restrictions on number of parents~\cite{kuipers2017partition}. %Alternative exact dynamic programming methods are limited to small settings~\cite{koivisto2012advances}. 

Recent advances in variational inference~\cite{zhang2018advances} have facilitated graph inference in DAG space, with gradient-based methods employing the NOTEARS DAG penalty \cite{zheng2018dags}.\cite{annadani2021variational} samples DAGs from autoregressive adjacency matrix distributions, while \cite{lorch2021dibs} utilizes Stein variational approach \cite{liu2016stein} for DAGs and causal model parameters. \cite{cundy2021bcd} proposed a variational inference framework on node orderings using the gumbel-sinkhorn gradient estimator \cite{mena2018learning}. \cite{deleu2022bayesian,nishikawa2022bayesian} employ the GFlowNet framework \cite{bengio2021gflownet} for inferring the DAG posterior. Most methods, except\cite{lorch2021dibs} are restricted to linear models, while \cite{lorch2021dibs} has high computational costs and lacks DAG generation guarantees compared to our method.
% at least quadratic scaling complexity, both with respect to the number of nodes (due to the DAG penalty) as well as number of posterior samples. Our proposed approach instead has linear complexity with respect to number of posterior samples and does not require any additional DAG penalty.     

In contrast, \emph{quasi-Bayesian} methods, such as DAG bootstrap \cite{friedman2013data}, demonstrate competitive performance. DAG bootstrap resamples data and estimates a single DAG using PC \cite{spirtes2000causation}, GES \cite{chickering2002optimal}, or similar algorithms, weighting the obtained DAGs by their unnormalized posterior probabilities. Recent neural network-based works employ variational inference to learn DAG distributions and point estimates for nonlinear model parameters \cite{charpentier2022differentiable,geffner2022deep}.

\section{Conclusion}
\section{Conclusion and Future Work}
In this work, I design corruption-robust algorithms for the Lipschitz contextual search problem. I present the \emph{agnostic checking} technique and demonstrate its effectiveness in designing corruption-robust algorithms. There are several open problems for future research. First, in the algorithm I propose for pricing loss, the schedule for agnostic checks is fixed upfront. Can the learner design an adaptive checking schedule for the pricing loss? Second, this work assumes the learner has knowledge of the Lipschitz constant $L$. Can the learner design efficient no-regret algorithms without knowledge of $L$? 
%
% ---- Bibliography ----
%
% BibTeX users should specify bibliography style 'splncs04'.
% References will then be sorted and formatted in the correct style.
%
\clearpage
%% the bibliography file.
\bibliographystyle{ACM-Reference-Format}
 \bibliography{references}
% \bibliography{mybibliography}
%

\clearpage
\appendix
\begin{comment}
\section{System Architecture}
\label{appendix:architecture}
\system has a novel modularized system architecture with three key components: 
\emph{StreamManager}, 
\emph{TxnManager} and \emph{TxnScheduler}. 
These components are instantiated in each thread locally.
The execution outline of \system is presented in Algorithm~\ref{alg:algo}.
Transactional stream processing is continuous and potentially never ends (Line 1$\sim$8).
The dependency resolution and execution of state transactions are separated into two non-overlapping phases by punctuations~\cite{Tucker:2003:EPS:776752.776780} (Line 2 and 5), which guarantees that no subsequent input event will have a smaller timestamp. 
Effectively, a batch of state transactions is collected during the first phase, and processed during the second phase.

In the first phase (i.e., stream processing phase), 
the \emph{StreamManager} conducts preprocessing for every input event ($e$). Similar to some prior works~\cite{tstream}, state transactions may be issued but not immediately processed during preprocessing (Line 3).
The \emph{pre\_processing} and \emph{post\_processing} functions are exposed as APIs to users.
The \emph{TxnManager} handles dependency resolution (Line 4) among state transactions and insert decomposed operations to construct a \tpg. We discuss the detailed two-phase \tpg construction process in Section~\ref{subsec:construction}.

In the second phase  (i.e., transaction processing phase), 
the \emph{TxnManager} is first involved again to refine (Line 6) the constructed \tpg with further dependency resolution.
The \emph{TxnScheduler} 
schedules operations for concurrent execution based on the constructed \tpg according to the three dimensions of scheduling decisions (Line 7). 
In particular, a scheduling decision model $M$ is instantiated based on the constructed \tpg (Line 14).
\textbf{\circled{1}} Guided by $M$, execution threads adopt an exploration strategy (Section~\ref{subsec:explore}) to explore the constructed \tpg for operations available to be scheduled constrained by dependencies. 
\textbf{\circled{2}} 
During exploration, one or multiple operations may be treated as the 
% basic 
unit of scheduling (Section~\ref{subsec:granularity}). 
Subsequently, \textbf{\circled{3}} every thread executes operation(s) in the unit of scheduling with various abort handling mechanisms (Section~\ref{subsec:abort_handling}).
Only when state transactions are processed (i.e., committed or aborted) can the associated input events be postprocessed (Line 8) by the \emph{StreamManager} based on transaction processing results.
\end{comment}

\begin{comment}
\begin{algorithm}
\footnotesize
    \KwData{$e$ \tcp{Input event}}
    \KwData{$txn_{ts}$ \tcp{State transaction}}
    \KwData{$G$ \tcp{The currently constructed TPG}}
    \While{!finish processing of input streams}{
        \eIf(\tcp*[h]{Phase 1}){\text{$e$ is not a $punctuation$}}{
                $txn_{ts}$ $\gets$ PRE\_Processing($e$)\;
                \textbf{TPG\_Construction}($G$, $txn_{ts}$)\; 
          }(\tcp*[h]{Phase 2}){
                \textbf{TPG\_Refinement}($G$)\; 
                \textbf{TXN\_Scheduling}($G$)\; 
                POST\_Processing()\;
          }
    }
    
    \SetKwFunction{FMain}{TPG\_Construction}
    \SetKwProg{Fn}{Function}{:}{}
    \Fn{\FMain{$G$, $txn_{ts}$}}{
        $O_{1..k}$ $\gets$ \textbf{Partition} $txn_{ts}$\;
        \ForEach{\text{operation $O_{i}$ $\in$ $O_{1..k}$}}{
            \textbf{Identify} its \ld\;
            $G$ $\gets$ $G$ + $O_{i}$ \;
        }
    }
    \SetKwFunction{FMain}{TPG\_Refinement}
    \SetKwProg{Fn}{Function}{:}{}
    \Fn{\FMain{$G$}}{
        \ForEach{\text{vertex $e_{i}$ $\in$ $G$}}{
            \textbf{Identify} its \td, \pd\;
        }
    }
    
    \SetKwFunction{FMain}{TXN\_Scheduling}
    \SetKwProg{Fn}{Function}{:}{}
    \Fn{\FMain{$G$}}{
        $M$ $\gets$ Instantiated with $G$;\tcp{A decision model}
        \While{!finish scheduling of $G$
        }{
          \textbf{\circled{2}} $Scheduling Unit$ $\gets$ \textbf{\circled{1}} \emph{Explore}($G$, $M$)\; 
            \textbf{\circled{3}} \emph{Execute with Abort Handling} ($Scheduling Unit$)\; 
        }
    }
  \caption{Execution Outline of \system}
  \label{alg:algo}
\end{algorithm}
\end{comment}


\end{document}
