In this section, we present an overview of our novelties via two motivating examples, namely the Pedestrian example and the Phylogenetic model. The Pedestrian example is non-score-recursive and handled by our fixed-point approach, while the Phylogenetic model is score-recursive and addressed by our OST approach.


% Figure environment removed

 

%\paragraph{Workflow.} 
Before we go into the details of the examples, we first present the workflow of our approaches. Given a probabilistic program $P$ and a measurable set $\calU\in\Sigma_{\Rset^{|\pvars|}}$, the workflow of our approaches is shown in \cref{fig:overview}: First, our parser transforms the input program $P$ into its WPTS $\Pi$ in the form of \eqref{eq:wpts}; Second, our approaches perform a truncation operation (to be introduced in \cref{sec:truncation}) that restricts the WPTS $\Pi$ into the bounded range of program values so that whenever the program runs out of the range, our approaches directly have upper and lower bounds to over- and under-approximate the behaviour of the program outside the bounded range, the purpose of which is that a bounded range allows our approach to derive tight bounds via templates; Third,  
%is truncated into two truncated WPTSs $\Pi_\top,\Pi_\bot$ (the aim of the truncation operation is to obtain tighter interval bounds which will be introduced in \cref{sec:truncation}); 
the truncated WPTS is further handled by 
%(3) $\Pi_\top,\Pi_\bot$ are handled by 
our fixed-point or OST  approach, depending on whether or not $\Pi$ is score-recursive, to generate polynomial bounds for expected weights at the termination of the program; Fourth, the interval bounds for the NPD $\posterior_{\Pi}(\calU)$ are calculated by the interval-bound analysis for expected weights. 
%the formula \eqref{eq:interval-analysis} 
%in \cref{sec2:NPD}. 
An illustration of our workflow is given in Fig.~\ref{fig:overview}. 


\subsection{Pedestrian Random Walk}\label{sec3:pedestrian}
%\begin{example}[Pedestrian Random Walk]
%\begin{example}
%	\label{ex:pedestrian}
Consider the pedestrian random walk example~ \cite{DBLP:conf/esop/MakOPW21} in \cref{fig:pedestrian-program}. 
In this example, a pedestrian is lost on the way home, and she only knows that she is away from her house at most $3$ km. Thus, she starts to repeatedly walk a uniformly random distance of at most $1$ km in either direction of the road, until reaching her house. Upon she arrives, an odometer tells that she has walked $1.1$ km totally. However, this odometer was once broken and the measured distance is normally distributed around the true distance with a standard deviation of $0.1$ km. The question of this example is: what is the posterior distribution of the starting point? This example is modeled as a non-parametric probabilistic program whose number of loop iterations is unbounded. In the program, the variable $start$ represents the starting point, $pos$ records the current position of the pedestrian, $step$ records the distance she walks in the next step, and $dis$ records the total distance the pedestrian travelled so far. The probabilistic branch in the loop body specifies that the pedestrian walks either forward or backward, both with probability $0.5$. 
	
This program is non-score-recursive and was previously handled in~\cite{Beutner2022b} by exhaustive recursion unrolling that has the path-explosion problem. To circumvent the path-explosion problem, in this work we propose a fixed-point theorem to establish constraints for this example, and solve the constraints by polynomial templates. To utilize the fact that polynomial approximation is usually accurate over a bounded range, during the solving of the polynomial templates, we restrict the behaviour of the program within a bounded range (e.g., $\{(pos,dis)\mid pos,dis\in [0,5]\}$) and over-approximate the expected weights outside the bounded range by an interval (e.g. $[0,2.1\times 10^{-330}]$). Note that here we omit the program variable $start$ as its value is determined once and has nothing to do with loop iterations. 
%This omission does not affect the correctness of our approaches and can be done by manual check or automated methods(?).
The solving of the polynomial template follows the previous approaches~\cite{DBLP:conf/cav/ChakarovS13,DBLP:journals/toplas/ChatterjeeFNH18,ChatterjeeFG16} that consider linear/semidefinite programming. 
%and template-based approach that synthesizes polynomial bounds for the NPD via a novel fixed-point theorem, a truncation operation that allows the polynomial synthesis to be over a bounded range. 
Our approach derives comparable bounds to the approach in~\citet{Beutner2022b} while our runtime is two-thirds of that of~\citet{Beutner2022b}. 



\subsection{Phylogenetic Birth Model}\label{sec3:phylogenetic}
%\begin{example}[Phylogenetic Birth Model] 
	
Consider a simplified version of the phylogenetic birth model \cite{ronquist2021universal} in \cref{fig:phylogenetic}, where a species arises with a birth-rate $lambda$, and it propagates with a constant likelihood of $1.1$ at some time interval.\footnote{For simplicity, we assume constant weights that can be viewed as over-approximation for a continuous density function.} This example is modelled as a probabilistic loop, where the variables $lambda, time, amount, wait$ stand for the birth rate of the species, the remaining propagation time, the current amount of the species and the propagation time to be spent, respectively. The variable $lambda$ is associated with a prior distribution, and the NPD problem is to infer its posterior distribution given the species evolution described by the loop.
The WPTS of the program is given in Fig.~\ref{fig:phylogenetic-wpts}. 
%Note that the {\tt score} statement is included in the loop body in the example.
%\end{example}

This program cannot be handled by previous approaches (such as~\cite{Beutner2022b,DBLP:conf/pldi/GehrSV20}) since it is a score-recursive program with an unbounded while loop and its score weight is greater than $1$.
%an unbounded while-loop and includes a score statement whose weight is greater than $1$ inside the loop body. 
To see why such a score-recursive program is nontrivial to tackle, consider a simple loop   %\cref{fig:running}. 
where in each loop iteration, the loop terminates directly with probability $\frac{1}{2}$, and continues to the next loop iteration with the same probability. At the end of each loop iteration, a score command ``{\tt score}$(3)$'' is executed. It follows that the normalising constant in NPD is equal to $\sum_{n=1}^{\infty} \probm(T=n)\cdot 3^n=\sum_{n=1}^{\infty} (\frac{3}{2})^n=\infty$,
%where $T$ is the random variable that measures the number of loop iteration before termination, 
so that the infinity makes the posterior distribution invalid. 
One can observe that in this example the main problem lies at the fact that the scaling speed of the likelihood weight (i.e.,~$3$) is higher than that for program termination (i.e.,~$\frac{1}{2}$). 

To tackle the difficulty of having scores greater than $1$ in a loop iteration, we propose a novel multiplicative variant of Optional Stopping Theorem (OST) that addresses score statements. Based on our OST variant, we apply truncation and polynomial template solving as in the non-score-recursive case.
%and derive bounds for outside the truncated range via the same OST variant and polynomial-template method but without truncation. 
For example, we can restrict the behaviour of the program within a bounded range such as $\{(lambda,time)\mid lambda\in [0,3]$ and $time\in [0,10]\}$, and over-approximate the expected weights outside the bounded range by an interval bound of polynomial functions 
%are over-approximated by two polynomial functions (e.g., degree $4$ polynomials over program variables) that 
derived from the same OST variant and polynomial-template method but without truncation. 
For the same reason as in \cref{sec3:pedestrian}, here we can safely omit the program variable $amount$. 
%phylogenetic birth model, our approach synthesizes polynomial bounds via a novel variant of Optional Stopping Theorem (OST) and again the truncation operation. 
Our experimental result on this example 
%phylogenetic birth model 
shows that the derived bounds match the simulation result with $10^6$ samples.  




% Figure environment removed

