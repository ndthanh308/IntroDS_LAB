\subsection{The OST Approach}\label{sec:ostapproach}

Below we propose a novel approach to address the NPD problem of Bayesian programs that have score statements with weights greater than $1$ inside loop bodies. We refer to such programs \emph{score-recursive}, and they have received  significant attention recently (such as the phylogenetic birth model in Section~\ref{sec3:phylogenetic}). 
We say that a WPTS is \emph{score-recursive} if it has a cycle of transitions on which there is a score statement with score function taking a value greater than $1$. Our fixed-point approach requires bounded score, and therefore cannot handle all score-recursive programs. 

We first derive a novel multiplicative variant of the classical Optional Stopping Theorem (OST) that tackles the multiplicative feature of score statements, then applies our OST variant to potential weight functions (\cref{def:puwf}) to address the NPD problem. The classical OST requires bounded changes of the weight value, and thus cannot handle score-recursive programs.  

Our OST variant applies to bounded-update score-recursive WPTS's. Informally, a WPTS has the bounded-update property if the change of values of the program variables is bounded by a global constant for every transition in the WPTS. This property is fulfilled in many realistic probabilistic models as the change of value of a variable in a single step is often bounded.
Formally, a WPTS $\Pi$ has the \emph{bounded-update} property if there exists a real constant $\varkappa>0$ such that for every reachable state $(\loc,\pv)$ and fork $F_j=\langle \loc'_j,p_j,\upd_j,\wet_j \rangle$ from a transition with the source location $\loc$, we have that $\forall \rv\in\supp{\rdvarjdis}\,~\forall x\in \pvars,~~ |\upd_j(\pv,\rv)(x)-\pv(x)|\le \varkappa$. The OST variant is given as follows.


\begin{theorem}[OST Variant]\label{thm:ost-variant}
Let $\{X_n\}_{n=0}^\infty$ be a supermartingale adapted to a filtration 
$\mathcal{F}=\{\mathcal{F}_n\}_{n=0}^\infty$, and $\kappa$ be a stopping time w.r.t. the filtration $\mathcal{F}$. 
Suppose that there exist positive real numbers $b_1,b_2,c_1,c_2,c_3$ such that $c_2>c_3$ and
\begin{itemize}
\item[(A1)] $\probm(\kappa>n) \leq c_1 \cdot e^{-c_2 \cdot n}$ for sufficiently large $n \in \Nset$, and
\item[(A2)] for all $n \in \Nset$, $\left\vert X_{n+1}-X_n \right\vert \le b_1\cdot n^{b_2}\cdot e^{c_3\cdot n}$ holds almost surely. 
\end{itemize}
Then we have that 
$\expv\left(|X_\kappa|\right)<\infty$ and %$\expv\left(X_\kappa\right) = \expv(X_0)$ 
$\expv\left(X_\kappa\right)\le\expv(X_0)$.
\end{theorem}


Our OST variant relaxes the classical OST that we allow the next random variable $X_{n+1}$ to be bounded by that of $X_n$ with a multiplicative factor $e^{c_3}$. The intuition is to  
cancel the multiplicative factor with the exponential decrease in $\probm(\kappa>n) \leq c_1 \cdot e^{-c_2 \cdot n}$.
We note that the exponential decrease is essential to cancel multiplicative scaling in score statements, as shown in \emph{challenges and gaps} in~\cref{sec:intro}.  
The proof resembles \cite[Theorem 5.2]{cost2019wang} and is relegated to \cref{app:ost-variant-proof}.

Below we show how our OST variant can be applied to handle bounded-update score-recursive programs. Fix a bounded-update score-recursive WPTS $\Pi$ in the form of \eqref{eq:wpts}. We reuse the expected-weight transformer defined in  \cref{def:ewt} and  potential weight functions given in \cref{def:puwf}. The difference with our fixed-point approach is that we no longer require that the score function $\wet_j$ in \eqref{eq:ewt} equals one for locations that do not lead to termination. 


\begin{theorem}[OST Approach]\label{thm:puwf-normalizing}
Let $\Pi$ be a bounded-update score-recursive WPTS. 
Suppose that there exist real numbers $c_1>0$ and $c_2>c_3>0$ such that 
\begin{itemize}
\item[(E1)] $\probm(T>n) \leq c_1 \cdot e^{-c_2 \cdot n}$ for sufficiently large $n\in\Nset$, and 
\item[(E2)] for each score function $\wet$ in $\Pi$, we have $|\wet|\le e^{c_3}$. 
\end{itemize}
 Then for any polynomial PUWF (resp. PLWF) $h$ over $\Pi$, we have that \  $\llbracket \Pi\rrbracket_{\valin} (\Rset^{|\pvars|})\le h(\lin,\valin)$ (resp. $\llbracket \Pi\rrbracket_{\valin} (\Rset^{|\pvars|})\ge h(\lin,\valin)$) for any initial state $(\lin,\valin)$, respectively.  
\end{theorem}

\begin{proof}[Proof Sketch.]
For upper bounds, we define the stochastic process $\{X_n\}_{n=0}^\infty$ as $X_n:=h(\loc_n,\pv_n)$ where $(\loc_n,\pv_n)$ is the program state at the $n$-th step of a program run. Then we construct another stochastic process $\{Y_n\}_{n=0}^\infty$ such that $Y_n:=X_n\cdot \prod_{i=0}^{n-1} W_i$ where $W_i$ is the weight at the $i$-th step of the program run. We consider the termination time $T$ of $\Pi$ and prove that $\{Y_n\}_{n=0}^\infty$ satisfies the prerequisites of our OST variant (\cref{thm:ost-variant})
by matching (A1) with (E1) and (A2) with (E2). 
Then by \cref{thm:ost-variant}, we obtain that $\expect{Y_T}\le \expect{Y_0}$. By (C2) in \cref{def:puwf}, we have that $Y_T=h(\loc_T,\pv_T)\cdot \prod_{i=0}^{T-1} W_i=\widehat{w}_T$. Thus, we have that $\llbracket \Pi\rrbracket_{\valin} (\Rset^{|\pvars|})=\expectdist{\valin}{\widehat{w}_T}=\expect{\prod_{i=0}^{T-1} W_i}\le \expect{Y_0}=h(\lin,\valin)$. Lower bounds are derived similarly. The detailed proof is relegated to~\cref{app:ost}.
\end{proof}


\begin{remark}%[Non-score-recursive WPTS's]
Our OST approach can handle programs with unbounded score values as caused by weights greater than $1$ that appear inside loop bodies, but with prerequisites in Theorem~\ref{thm:puwf-normalizing} (i.e., bounded update, (E1) and (E2)) to ensure the integrability to apply our OST variant. Note that our OST approach can directly handle non-score-recursive WPTS's 
as in such WPTS the score value  inside loops is bounded by one. \qed
\end{remark}
