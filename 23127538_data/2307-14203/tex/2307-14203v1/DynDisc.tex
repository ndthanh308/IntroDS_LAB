%
\documentclass[12pt,notitlepage]{article}
\usepackage{amssymb}
\usepackage{amsmath}
\usepackage{graphicx}
\usepackage{graphics}
\usepackage{epstopdf}
\usepackage{pdflscape}
%\usepackage{subfig}
\usepackage{tabularx}
\usepackage{longtable}
\usepackage{array}
\usepackage{dsfont}
\usepackage{float}
\usepackage{booktabs}
\usepackage{marvosym}
\usepackage{multirow}
\usepackage{pdflscape}
\usepackage[hyphenbreaks]{breakurl}
\usepackage[hyphens]{url}
\usepackage{setspace}
\usepackage{epigraph}
\usepackage{bm}
\usepackage{textcomp}
\usepackage{caption}
\usepackage{subcaption}
\usepackage{bbm}
\usepackage{tikz}
\usepackage{siunitx}
\usetikzlibrary{positioning}
\usepackage{afterpage}
\usepackage{lipsum}
\usetikzlibrary{fit}
\tikzset{mynode/.style={draw,text width=1in,align=center}}
\usepackage{verbatim}
\usepackage{amsmath}
\usepackage{setspace}
\usepackage{multibib}
\newcites{online}{References}
\usepackage[round]{natbib}
\usepackage[shortlabels]{enumitem}
\setlength{\epigraphrule}{0pt}

\setcounter{MaxMatrixCols}{10}

\usepackage{natbib,hyperref}
\bibliographystyle{chicago}  

\newcolumntype{C}[1]{>{\centering\let\newline\\\arraybackslash\hspace{0pt}}m{#1}}

%\newcounter{subfigure}

\topmargin=-1.5cm \textheight=23cm \oddsidemargin=-0.0cm
\evensidemargin=-0.0cm \textwidth=16.5cm
\usepackage[bottom]{footmisc}
\newtheorem{ass}{Assumption}
\newtheorem{definit}{Definition}
\newtheorem{prop}{Proposition}
\newtheorem{thm}{Theorem}
\newtheorem{lem}{Lemma}
\newtheorem{conj}{Conjecture}
\newtheorem{cor}{Corollary}
\newtheorem{rem}{Remark}
\renewcommand{\thesection}{\arabic{section}}
\renewcommand{\thesubsection}{\arabic{section}.\arabic{subsection}}
\renewcommand{\thesubsubsection}{\arabic{section}.\arabic{subsection}.\arabic{subsubsection}}

\newcommand\independent{\protect\mathpalette{\protect\independenT}{\perp}}
\def\independenT#1#2{\mathrel{\rlap{$#1#2$}\mkern2mu{#1#2}}}

\newcolumntype{L}[1]{>{\raggedright\let\newline\\\arraybackslash\hspace{0pt}}m{#1}}
\newcolumntype{C}[1]{>{\centering\let\newline\\\arraybackslash\hspace{0pt}}m{#1}}

\newcommand{\I}{\mathbb{I}}
\newcommand{\E}{\mathbb{E}}
\newcommand{\Ll}{\mathrm{L}}
\renewcommand{\L}{\mathbb{L}}
\newcommand{\Var}{\mathrm{Var}}
\newcommand{\Cov}{\mathrm{Cov}}
\newcommand{\Corr}{\mathrm{Corr}}
\newcommand{\Prob}{\mathbb{P}}
\newcommand{\supp}{\mathrm{supp}}
\newcommand{\notimplies}{\mathrel{{\ooalign{\hidewidth$\not\phantom{=}$\hidewidth\cr$\implies$}}}}

%Figure path
\def \oldfigroot{figs/old/}
\def \tabroot{tables/}
\def \figroot{figs/}


\usepackage{epsfig,hyperref}

\hypersetup{
    pdftitle={DynDisc},    % title
    pdfauthor={Francesco Ruggieri},     % author
    pdfnewwindow=true,      % links in new window
    colorlinks=true,       % false: boxed links; true: colored links
    linkcolor=black,          % color of internal links
    citecolor=blue,        % color of links to bibliography
    filecolor=black,      % color of file links
    urlcolor=blue           % color of external links
}

\begin{document}

\title{Dynamic Regression Discontinuity: \\A Within-Design Approach\thanks{I am grateful to Alex Torgovitsky for his invaluable guidance and support. I also thank Scott Behmer, St\'ephane Bonhomme, Matias Cattaneo, Hazen Eckert, Nadav Kunievsky, Jonathan Roth, and participants to the Econometrics Advising group at the University of Chicago for helpful comments.}}

\author{Francesco Ruggieri\thanks{University of Chicago, Kenneth C. Griffin Department of Economics. E-mail: \href{mailto:ruggieri@uchicago.edu}{ruggieri@uchicago.edu}}}

\date{\today}
\maketitle

\begin{center}
	Preliminary Draft: Comments Welcome
\end{center}

\medskip

\begin{abstract}
	I propose a novel argument to point identify economically interpretable intertemporal treatment effects in dynamic regression discontinuity designs (RDDs). Specifically, I develop a dynamic potential outcomes model and specialize two assumptions of the difference-in-differences literature, the no anticipation and common trends restrictions, to point identify cutoff-specific impulse responses. The estimand associated with each target parameter can be expressed as the sum of two static RDD outcome contrasts, thereby allowing for estimation via standard local polynomial tools. I leverage a limited path independence assumption to reduce the dimensionality of the problem.
\end{abstract}

\setcounter{page}{0}\thispagestyle{empty}
\baselineskip1.47\baselineskip%
%\doublespacing


\newpage

\section{Introduction}

Regression discontinuity designs (RDDs) are commonly used to identify causal parameters when the probability of exposure to a treatment changes discontinuously at a known deterministic threshold. In a seminal contribution, \cite{cfr2010} extended the canonical design to multi-period settings in which treatment assignment is sequential and exhibits path dependence. Dynamic RDDs have proven popular in the empirical public finance literature that exploits local referenda to estimate the effect of increased government expenditure on various outcomes (\citealt{darolia2013}, \citealt{hongzimmer2016}, \citealt{martorelletal2016}, \citealt{abottetal2020}, \citealt{baron2022}, \citealt{rohlinetal2022}, \citealt{baronetal2022}). 

Despite their growing relevance in empirical research, dynamic RDDs are methodologically understudied. This is especially true in relation to the challenge posed by the identification of economically interpretable long-term causal parameters. Because units repeatedly participate to treatment assignment and the outcome at any point in time may reflect the contribution of past and future treatment states, it is generally hard to disentangle effects associated with sequential RD experiments. In recent work, \cite{hsushen2022} addresses this challenge in a potential outcomes framework that allows for flexible patterns of path dependence and heterogeneity in treatment effects. To identify dynamic effects, they leverage predetermined covariates and an assumption of selection on observables that is inspired by the literature on treatment effect extrapolation in static RDDs (\citealt{angristrokkanen2015}, \citealt{rokkanen2015}).

In this paper, I consider a dynamic model in which potential outcomes at any point in time depend on past, contemporaneous, and future treatment states (\citealt{robins1986}). I subsequently propose an identification argument that does not hinge on external observables. Instead, I specialize two assumptions of the difference-in-differences literature, the no anticipation and common trends restrictions, and show that they are sufficient to point identify interpretable causal parameters. The resulting, time-specific estimands can be expressed as sums of two standard RDD outcome contrasts and can subsequently be aggregated with a set of nonnegative weights that add up to one. Finally, recognizing that the number of parameters grows exponentially with the number of time periods in the model, I reduce the dimensionality of the problem with a limited path independence assumption. This restriction takes advantage of the sparse and cyclical nature of treatment assignment that is typically true of empirical applications of dynamic RDDs.

The within-design approach presented in this paper is similar in spirit to recently proposed solutions to extrapolate effects away from the threshold in static designs only leveraging features of the joint distribution of the outcome, the running variable, and one or more cutoffs (\citealt{donglewbel2015}, \citealt{bertanhaimbens2020}, \citealt{cattaneoetal2021}). I view this approach as offering a number of advantages. First, by construction, it is applicable to settings in which external information may not be available or fail to satisfy prerequisites for validity, such as observed characteristics co-determined by the treatment of interest. This concern may be especially salient in dynamic settings in which units repeatedly participate to treatment assignment. Second, my proposed approach is based on well-understood identification assumptions with testable suggestive implications. Third, the method is easy to implement because the resulting estimators can be expressed as linear combinations of standard regression discontinuity contrasts.

This paper contributes to three related strands of the econometric literature. First, it proposes a novel approach to point identify intertemporal treatment effects in dynamic RDDs. Second, it adds to the broad literature on the identification of economically interpretable dynamic effects under treatment effect heterogeneity (\citealt{callawaysantanna2021}, \citealt{sunabraham2021}, \citealt{dcdh_intertemp2022}) and path dependence in treatment assignment (\citealt{heckmanhumphriesveramendi2016}). Third, it contributes to the growing literature on the causal interpretation of standard estimands in the presence of multiple treatments (\citealt{pgphullkolesar2022}, \citealt{dcdh_multi2022}) or multiple instrumental variables (\citealt{mogstadtorgowalters2021}) when treatment effects are allowed to be stochastic.

The remainder of the paper is organized as follows. In Section 2, I develop a simple two-period regression discontinuity design and illustrate the core of my identification strategy. In Section 3, I extend the two-period model to fuzzy designs. In Section 4, I generalize the model to a setting with multiple time periods. In Section 5, I consider alternative approaches to aggregate time-specific parameters. 

%In Section 6, I discuss estimation and inference. In Section 7, I revisit the application of dynamic regression discontinuity to the effect of California school bonds on education expenditures.

\section{A Two-Period Regression Discontinuity Design}

\subsection{Model Setup}

Consider a dynamic setting in which time periods are indexed by $t \in \left \{ 1, 2 \right \}$. Let $D_t \in \left \{ 0,1 \right \}$ denote a binary treatment that is assigned at the beginning of period $t$. The treatment is determined by whether the realization of an observed running variable $R_t \in \mathbb{R}$ exceeds a known deterministic constant $c \in \mathbb{R}$:
\begin{align}
	D_t \equiv \mathbb{I} \left [ R_t \geq c \right ]
\end{align}
Let $Y_t \in \mathbb{R}$ be an outcome that is observed at the end of period $t$. In the canonical static setting, potential outcomes depend on a single treatment state, and the observed outcome satisfies the switching equation $Y_t = \sum_{d \in \left \{ 0,1 \right \}} \mathbb{I} \left [ D_t = d \right ] Y_t \left ( d \right )$. In a two-period model with repeated treatment assignment, the potential outcome in period 1 may reflect the treatment state in period 2 due to anticipation and, symmetrically, the potential outcome in period 2 may be affected by the treatment state in period 1 due to path dependence. Thus, following \cite{robins1986}, it is natural to model potential outcomes as depending on the path of past, contemporaneous, and future treatment states at any point in time. Specifically, let the period-$t$ potential outcome be $Y_t \left ( d_1, d_2 \right )$, where $d_1, d_2 \in \left \{ 0,1 \right \}$. Without further assumptions, the evolution of treatment states is unrestricted. In fact, a unit may be repeatedly assigned the treatment, never receive it, or take it only in one period. As in \cite{hsushen2022}, I allow for path dependence in treatment assignment by modeling potential treatments as functions of the history of treatment states. Specifically, let $D_1$ and $D_2 \left ( d_1 \right )$ denote the first- and second-period potential treatments, respectively\footnote{A more flexible setting could allow for anticipation in treatment assignment by letting the potential treatment in period 1 depend on the treatment state in period 2, i.e., $D_1 \left ( d_2 \right )$. For simplicity, I abstract from this scenario and allow for anticipation only in potential outcomes.}. In the canonical application of dynamic RDDs to local referenda, it is useful to interpret potential treatments as characterizing jurisdiction ``types''. For example, a school district in which voters repeatedly reject a bond for school construction ($D_1 = 0$ and $D_2 \left ( 0 \right ) = D_2 \left ( 1 \right ) = 0$) is likely more fiscally conservative than one in which increased public expenditure is approved at the first round of elections ($D_1 = 1$) or at the second round ($D_1 = 0$ and $D_2 \left ( 0 \right ) = 1$). In this two-period model, the observed outcome in period $t$ is related to potential outcomes as
\begin{align}\label{potout}
	Y_t = \sum_{\left ( d_1, d_2 \right ) \in \left \{ 0,1 \right \}^2} \mathbb{I} \left [ D_1 = d_1, D_2 \left ( d_1 \right ) = d_2 \right ] Y_t \left ( d_1, d_2 \right )
\end{align}

\subsection{Target Parameters}

Having outlined features of the dynamic potential outcomes model, it is natural to define target parameters. As in static RDDs, all causal parameters of interest are local to the nonstochastic threshold above which the treatment is assigned (\citealt{hahnetal2001}). Without loss, the analysis will focus on target parameters implied by the first-period discontinuity. Letting $\tau \in \left \{ 0,1 \right \}$ denote the number of leads, consider a class of threshold-specific Average Treatment Effects (ATEs),
\begin{align}
	\mathrm{ATE}_{1,\tau} \left ( c \right ) \equiv \E \left [ Y_{1+\tau} \left ( 1,d_2 \right ) - Y_{1+\tau} \left ( 0,d_2' \right ) | R_1 = c \right ]
\end{align}
where $d_2, d_2' \in \left \{ 0,1 \right \}$. Each of these average effects is constructed as the average contrast between a potential outcome that is treated in the first period and a potential outcome that is not. The choice of a second-period treatment state naturally gives rise to causal parameters with a different interpretation. Since \cite{cfr2010}, the empirical public finance literature that studies the effects of increased public expenditure authorized by local referenda has focused on a class of Average Primary Treatment Effects (\citealt{hsushen2022}) that sets $d_2 = d_2' = 0$,
\begin{align}
	\mathrm{APTE}_{1,\tau} \left ( c \right ) \equiv \E \left [ Y_{1+\tau} \left ( 1,0 \right ) - Y_{1+\tau} \left ( 0,0 \right ) | R_1 = c \right ]
\end{align}
Hereafter, I will refer to $\mathrm{APTE}_{1,0} \left ( c \right )$ and $\mathrm{APTE}_{1,1} \left ( c \right )$ as the immediate and cumulative Average Primary Treatment Effects, respectively. These target parameters can be interpreted as tracing out the impulse response of the outcome to the first-period discontinuity. Economically, they correspond to the thought experiment of measuring the cutoff-specific average effect of first-period treatment assignment in the counterfactual scenario in which exposure to the treatment is set to zero thereafter.

\subsection{What Do Standard RD Estimands Identify?}

In what follows, I will show that standard RDD estimands do not, in general, identify APTEs and will subsequently specify assumptions to achieve this goal. First, the (sharp) RDD estimand implied by the first-period discontinuity is defined as
\begin{align}
	\mathrm{RD}_{t} \left ( c \right ) \equiv \lim_{r \downarrow c} \E \left [ Y_{t} | R_1 = r \right ] - \lim_{r \uparrow c} \E \left [ Y_{t} | R_1 = r \right ]
\end{align}

\cite{hahnetal2001} shows that continuity in average potential outcomes at the cutoff is sufficient to point identify the Average Treatment Effect at the threshold. It is natural to generalize this assumption to a dynamic setting.

\begin{ass}[Continuity at the Cutoff]\label{ass_cont0} For any $t \in \left \{ 1,2 \right \}$ and $d_1, d_2 \in \left \{ 0,1 \right \}$,
	\begin{align*}
		\E \left [ D_{2} \left ( d_1 \right ) | R_1 = r \right ] \quad \text{ and } \quad \E \left [ Y_{t} \left ( d_1, d_2 \right ) | R_1 = r, D_{2} \left ( d_1 \right ) \right ]
	\end{align*}
	are continuous at $r=c$.
\end{ass}

Assumption \ref{ass_cont0} generalizes the restriction that agents do not sort nonrandomly around the threshold. First, the probability of potential treatment assignment in the second period evolves smoothly at the first-period cutoff. Importantly, this statement does \textit{not} imply that the probability of observing $D_2 = 1$ is unaffected by the first-period treatment state. Instead, it restricts the probability of any jurisdiction ``type'' -- as pinned down by the potential treatment $D_2 \left ( d_1 \right )$ -- not to vary abruptly at the first-period cutoff. Analogously, Assumption \ref{ass_cont0} rules out a scenario in which agents systematically choose either side of the first-period cutoff based on observed and unobserved determinants of the outcome, within each jurisdiction ``type'' and for each possible path of treatment states. Under this assumption, it is immediate to derive the following identification argument.

\begin{prop}[Standard RD Estimand]\label{prop_rd}
	Suppose that Assumption \eqref{ass_cont0} holds. Then
	\begin{align*}
		\mathrm{RD}_t \left ( c \right ) & = \sum_{d_2 \in \left \{ 0,1 \right \}} \E \left [ Y_t \left ( 1,d_2 \right )  | R_1 = c, D_2 \left ( 1 \right ) = d_2 \right ] \times \Prob \left ( D_2 \left ( 1 \right ) = d_2 | R_1 = c \right ) \\
		& - \sum_{d_2 \in \left \{ 0,1 \right \}} \E \left [ Y_t \left ( 0,d_2 \right )  | R_1 = c, D_2 \left ( 0 \right ) = d_2 \right ] \times \Prob \left ( D_2 \left ( 0 \right ) = d_2 | R_1 = c \right )
	\end{align*}
\end{prop}

Without further assumptions, the sharp regression discontinuity estimand identifies the contrast between two weighted sums of potential outcomes over second-period treatment states, where weights are nonnegative and add up to one. As discussed in \cite{cfr2010} and similarly in \cite{heckmanhumphriesveramendi2016}, this target parameter may be interpreted as the ``intent-to-treat'' implied by the first-period discontinuity. Because treatment assignment exhibits path dependence and outcomes reflect past and future treatment states, the ITT is generally hard to interpret. By way of exposition, consider the extreme scenario in which $\Prob \left ( D_2 \left ( 1 \right ) = 1 | R_1 = c \right ) = 1$ and $\Prob \left ( D_2 \left ( 0 \right ) = 0 | R_1 = c \right ) = 1$. In this case, $\mathrm{RD}_{2} \left ( c \right )$ identifies $\E \left [ Y_2 \left ( 1,1 \right ) - Y_2 \left ( 0,0 \right )  | R_1 = c \right ]$. However, the treatment state in period 1 perfectly predicts the treatment state in period 2, making it infeasible to disentangle the cumulative effect of the first-period discontinuity, namely $\mathrm{APTE}_{1,1} \left ( c \right )$, from the immediate effect of the second-period discontinuity. Formally,
\begin{align}
	\mathrm{RD}_2 \left ( c \right ) = \underbrace{\E \left [ Y_2 \left ( 1,1 \right ) - Y_2 \left ( 1,0 \right )  | R_1 = c \right ]}_{\text{immediate effect of $D_2$}} + \underbrace{\E \left [ Y_2 \left ( 1,0 \right ) - Y_2 \left ( 0,0 \right )  | R_1 = c \right ]}_{\text{cumulative effect of $D_1$}}
\end{align}
To conclude, under Assumption \ref{ass_cont0}, standard RD estimands may not identify interpretable causal parameters. Motivated by this negative result, in the next two sections I will specify additional sufficient restrictions to identify the immediate and cumulative Average Primary Treatment Effects implied by the first-period discontinuity.

\subsection{Identification of the Immediate APTE}

The goal of this section is to provide an identification argument for the immediate Average Primary Treatment Effect, $\E \left [ Y_1 \left ( 1,0 \right ) | R_1 = c \right ] - \E \left [ Y_1 \left ( 0,0 \right ) | R_1 = c \right ]$.
By an application of the Law of Iterated Expectations, the first term can be expressed as
\begin{align}\label{apte10}
	\E \left [ Y_1 \left ( 1,0 \right ) | R_1 = c \right ] & = \overbrace{\E \left [ Y_1 \left ( 1,0 \right ) | R_1 = c, D_2 \left ( 1 \right ) = 0 \right ]}^{\text{identified}} \times \overbrace{\Prob \left ( D_2 \left ( 1 \right ) = 0 | R_1 = c \right )}^{\text{identified}} \nonumber \\
	& + \underbrace{\E \left [ Y_1 \left ( 1,0 \right ) | R_1 = c, D_2 \left ( 1 \right ) = 1 \right ]}_{\text{counterfactual}} \times \underbrace{\Prob \left ( D_2 \left ( 1 \right ) = 1 | R_1 = c \right )}_{\text{identified}}
\end{align}
Notice that both probabilities are identified under Assumption \ref{ass_cont0}. In fact, for $d_2 \in \left \{ 0,1 \right \}$,
\begin{align}\label{probs}
	\Prob \left ( D_2 \left ( 1 \right ) = d_2 | R_1 = c \right ) = \lim_{r \downarrow c} \Prob \left ( D_{2} = d_2 | R_1 = r \right )
\end{align}
Similarly, the first conditional expectation in equation \eqref{apte10} is identified as
\begin{align}
	\E \left [ Y_1 \left ( 1,0 \right ) | R_1 = c, D_2 \left ( 1 \right ) = 0 \right ] = \lim_{r \downarrow c} \E \left [ Y_1 | R_1 = r, D_2 = 0 \right ]
\end{align}
On the other hand, $\E \left [ Y_1 \left ( 1,0 \right ) | R_1 = c, D_2 \left ( 1 \right ) = 1 \right ]$ is counterfactual. Intuitively, this term captures the average treated-untreated potential outcome among units who would receive the treatment in $t=2$ after receiving it in $t=1$. To identify this counterfactual expectation, it is sufficient to assume that, on average and within bins implied by potential treatment states, the first-period outcome is independent of the second-period treatment state.

\begin{ass}[No Anticipation] \label{ass_noant} For any $d_1, d_2 \in \left \{ 0,1 \right \}$,
	\begin{align*}
		\E \left [ Y_{1} \left ( d_1, 0 \right ) | R_1 = c, D_{2} \left ( d_1 \right ) = d_2 \right ] = \E \left [ Y_{1} \left ( d_1, 1 \right ) | R_1 = c, D_{2} \left ( d_1 \right ) = d_2 \right ]
	\end{align*}
\end{ass}

Assumption \ref{ass_noant} is a cutoff-specific version of the canonical no anticipation assumption required to point identify the Average Treatment Effect on the Treated in difference-in-differences designs (\citealt{abbringvanderberg2003}). It restricts current potential outcomes not to reflect (expectations over) future treatment states. While this statement may appear to rule out the applicability of this argument to forward-looking outcome variables, it should be noted that the lack of anticipatory behavior must hold only at the first-period cutoff. In the application of dynamic RDDs to the effect of increased school spending on student outcomes, Assumption \ref{ass_noant} entails that average test scores after a narrowly rejected bond do not reflect the possibility that a bond proposition may be approved or rejected in a subsequent round of elections. Under Assumption \ref{ass_cont0} and Assumption \ref{ass_noant}, the counterfactual conditional mean is identified as
\begin{align}
	\E \left [ Y_1 \left ( 1,0 \right ) | R_1 = c, D_2 \left ( 1 \right ) = 1 \right ] = \lim_{r \downarrow c} \E \left [ Y_1 | R_1 = r, D_2 = 1 \right ]
\end{align}
Thus, the left-hand side of $\textsc{APTE}_{1,0} (c)$ can be expressed as a function of the following observables:
\begin{align}
	\E \left [ Y_1 \left ( 1,0 \right ) | R_1 = c \right ] & = \lim_{r \downarrow c} \E \left [ Y_1 | R_1 = r, D_2 = 0 \right ] \times \lim_{r \downarrow c} \Prob \left ( D_{2}=0 | R_1 = r \right ) \nonumber \\
	& + \lim_{r \downarrow c} \E \left [ Y_1 | R_1 = r, D_2 = 1 \right ] \times \lim_{r \downarrow c} \Prob \left ( D_{2}=1 | R_1 = r \right ) \nonumber \\
	& = \lim_{r \downarrow c} \E \left [ Y_1 | R_1 = r \right ]
\end{align}
where the last equality follows from an application of the Law of Iterated Expectations. By a symmetric argument, the right-hand side of $\textsc{APTE}_{1,0} (c)$ is point identified, too. Combining results yields the following proposition.

\begin{prop}[Identification of Immediate APTE]\label{prop_apte0}
	Suppose that Assumption \eqref{ass_cont0} and Assumption \eqref{ass_noant} hold. Then
	\begin{align*}
		\textsc{APTE}_{1,0} (c) = \lim_{r \downarrow c} \E \left [ Y_1 | R_1 = r \right ] - \lim_{r \uparrow c} \E \left [ Y_1 | R_1 = r \right ] \equiv \mathrm{RD}_1 \left ( c \right )
	\end{align*}
\end{prop}

This result follows from specializing Proposition \ref{prop_rd} to the case in which Assumption \ref{ass_noant} holds. Intuitively, the lack of anticipation implies that the second-period potential treatment state is irrelevant for the first-period outcome. Setting $d_2 = 0$ yields the identification result in Proposition \ref{prop_apte0}. To summarize, continuity and no anticipation are sufficient for the immediate regression discontinuity estimand to identify an interpretable causal parameter in this model.

\subsection{Identification of the Cumulative APTE}

Consider, instead, the more interesting goal of identifying the cumulative Average Primary Treatment Effect, $\E \left [ Y_2 \left ( 1,0 \right ) | R_1 = c \right ] - \E \left [ Y_2 \left ( 0,0 \right ) | R_1 = c \right ]$. Mirroring the argument in the previous section, an application of the Law of Iterated Expectations implies that the first term can be expressed as
\begin{align}\label{apte11}
	\E \left [ Y_2 \left ( 1,0 \right ) | R_1 = c \right ] & = \overbrace{\E \left [ Y_2 \left ( 1,0 \right ) | R_1 = c, D_2 \left ( 1 \right ) = 0 \right ]}^{\text{identified}} \times \overbrace{\Prob \left ( D_2 \left ( 1 \right ) = 0 | R_1 = c \right )}^{\text{identified}} \nonumber \\
	& + \underbrace{\E \left [ Y_2 \left ( 1,0 \right ) | R_1 = c, D_2 \left ( 1 \right ) = 1 \right ]}_{\text{counterfactual}} \times \underbrace{\Prob \left ( D_2 \left ( 1 \right ) = 1 | R_1 = c \right )}_{\text{identified}}
\end{align}
As shown in equation \eqref{probs}, both conditional probabilities are identified under Assumption \ref{ass_cont0}. In addition, the first conditional expectation in equation \eqref{apte11} is identified as
\begin{align}
	\E \left [ Y_2 \left ( 1,0 \right ) | R_1 = c, D_2 \left ( 1 \right ) = 0 \right ] = \lim_{r \downarrow c} \E \left [ Y_2 | R_1 = r, D_2 = 0 \right ]
\end{align}
On the other hand, $\E \left [ Y_2 \left ( 1,0 \right ) | R_1 = c, D_2 \left ( 1 \right ) = 1 \right ]$ is counterfactual because $Y_2 \left ( 1,0 \right )$ is unobserved among units who would sort into the untreated arm in $t=2$ after being treated in $t=1$. Differently from the previous section, Assumption \ref{ass_noant} alone is not sufficient to identify the counterfactual mean because the outcome is measured in $t=2$. However, point identification of the target parameter may be achieved by also assuming that the expected growth in untreated-at-2 potential outcomes is independent of the potential second-period treatment state at the cutoff.

\begin{ass}[Common Trends] \label{ass_ct0} For $d_1 \in \left \{ 0,1 \right \}$,
	\begin{align*}
		\E \left [ \Delta_{1,2} \left ( d_1, 0 \right ) | R_1 = c, D_{2} \left ( d_1 \right ) = 0 \right ] = \E \left [ \Delta_{1,2} \left ( d_1, 0 \right ) | R_1 = c, D_{2} \left ( d_1 \right ) = 1 \right ]
	\end{align*}
	where $\Delta_{1,2} \left ( d_1, 0 \right ) \equiv Y_{2} \left ( d_1,0 \right ) - Y_1 \left ( d_1, 0 \right )$.
\end{ass}

Assumption \ref{ass_ct0} is a cutoff-specific version of the common trends assumption required to point identify the Average Treatment Effect on the Treated in difference-in-differences designs (\citealt{abadie2005}). As in that setting, groups (``treatment'' and ``control'') are implied by paths of potential treatment states. Differently from that setting, the treatment is available also in the first period, implying that there are four, not two, groups or ``types''. As a consequence, the parallel trends restriction also applies to units who are potentially treated in the first period. In the classic application of dynamic RDDs to local public finance, Assumption \ref{ass_ct0} entails that the expected growth in potential test scores absent a second referendum approval is independent on a jurisdiction's ``type'', i.e., it is independent of whether a jurisdiction would approve or reject a school bond referendum in the second period. Under Assumption \ref{ass_cont0}, Assumption \ref{ass_noant}, and Assumption \ref{ass_ct0}, the counterfactual conditional mean is identified as
\begin{align*}
	\E \left [ Y_{2} \left ( 1,0 \right ) | R_1 = c, D_{2} \left ( 1 \right )=1 \right ] & = \lim_{r \downarrow c} \E \left [ Y_1 | R_1 = r, D_{2}=1 \right ] + \lim_{r \downarrow c} \E \left [ Y_{2} - Y_1 | R_1 = r, D_{2}=0 \right ]
\end{align*}
As in the canonical difference-in-differences design, the counterfactual term is identified as the sum of a ``level'' and a ``growth'' term. In this case, the former measures the average observed first-period outcome among units who later sort into the treated arm. The latter captures the average observed outcome growth among units who sort into the untreated arm in the second period. Thus, the left-hand side of $\mathrm{APTE}_{1,1} \left ( c \right )$ can be expressed as a function of the following observables:
\begin{align}
	\E \left [ Y_2 \left ( 1,0 \right ) | R_1 = c \right ] & = \lim_{r \downarrow c} \E \left [ Y_2 | R_1 = r, D_2 = 0 \right ] \times \lim_{r \downarrow c} \Prob \left ( D_{2}=0 | R_1 = r \right ) \nonumber \\
	& + \left ( \lim_{r \downarrow c} \E \left [ Y_1 | R_1 = r, D_{2}=1 \right ] + \lim_{r \downarrow c} \E \left [ Y_{2} - Y_1 | R_1 = r, D_{2}=0 \right ] \right ) \nonumber \\
	& \times \lim_{r \downarrow c} \Prob \left ( D_{2}=1 | R_1 = r \right )
\end{align}
By a symmetric argument, the right-hand side of $\textsc{APTE}_{1,1} (c)$ is point identified, too. Combining results yields the following proposition.

\begin{prop}[Identification of Cumulative APTE]\label{prop_apte1}
	Suppose that Assumption \ref{ass_cont0}, Assumption \ref{ass_noant}, and Assumption \ref{ass_ct0} hold. Then
	\begin{align*}
		\textsc{APTE}_{1,1} (c) & = \lim_{r \downarrow c} \E \left [ Y_1 | R_1 = r \right ] - \lim_{r \uparrow c} \E \left [ Y_1 | R_1 = r \right ] \\
		& + \lim_{r \downarrow c} \E \left [ Y_2 - Y_1 | R_1 = r, D_2 =0 \right ] - \lim_{r \uparrow c} \E \left [ Y_2 - Y_1 | R_1 = r, D_2 =0 \right ]
	\end{align*}
\end{prop}

This proposition entails that an interpretable long-term causal parameter determined by the first-period discontinuity is identified by the sum of two standard sharp Regression Discontinuity estimands, namely the RD estimand that uses $Y_1$ as an outcome and the RD estimand that uses $Y_2 - Y_1$ as an outcome in the subpopulation implied by $D_2 = 0$.




\section{Generalization to Multiple Leads}

It is natural to generalize previous arguments to identify long-term causal parameters in a dynamic potential outcomes model with more than two periods. In this section, I will extend the setup and identifying assumptions of the two-period model to a setting with an arbitrary number of time periods indexed by $t \in \left \{ 1, 2, \dots, \overline{t} \right \}$. First, potential outcomes at any point in time reflect the path of past, contemporaneous, and future treatment states. Specifically, for $\left \{ d_j \right \}_{j=1}^{\overline{t}}$ with $d_j \in \left \{ 0,1 \right \}$ for all $j$, the period-$t$ potential outcome is denoted as $Y_t \left ( d_1, \dots, d_{\overline{t}} \right )$. Second, to allow for path dependence in treatment assignment, the period-$t$ potential treatment is modeled as $D_t \left ( d_1, \dots, d_{t-1} \right )$. As a generalization of equation \eqref{potout}, the relationship between potential and observed outcomes can thus be characterized as
\begin{align}\label{potout_gen}
	Y_t = \sum_{\left ( d_1, \dots, d_{\overline{t}} \right ) \in \left \{ 0,1 \right \}^{\overline{t}}} \mathbb{I} \left [ D_1 = d_1, D_2 \left ( d_1 \right ) = d_2, \dots, D_{\overline{t}} \left ( d_1, \dots, d_{\overline{t}-1} \right ) = d_{\overline{t}} \right ] Y_t \left (  d_1, \dots, d_{\overline{t}} \right )
\end{align}
To keep notation concise, let $P: \left \{ 0,1 \right \}^{\overline{t}} \to \left \{ 0,1 \right \}$ be a function mapping a path of potential treatment states to an indicator that takes the value one if a unit's treatment-taking behavior for a given $d_1 \in \left \{ 0,1 \right \}$ is described by that path, i.e.,
\begin{align}
	P \left ( d_1, \dots, d_{\overline{t}} \right ) = \mathbb{I} \left [ D_2 \left ( d_1 \right ) = d_2, D_3 \left ( d_1, d_2 \right ) = d_3, \dots, D_{\overline{t}} \left ( d_1, \dots, d_{\overline{t}-1} \right ) = d_{\overline{t}} \right ]
\end{align} 
Therefore, a compact formulation of equation \eqref{potout_gen} is
\begin{align}\label{potout_gen2}
	Y_t = \sum_{\left ( d_1, \dots, d_{\overline{t}} \right ) \in \left \{ 0,1 \right \}^{\overline{t}}} \mathbb{I} \left [ D_1 = d_1 \right ] P \left ( d_1, \dots, d_{\overline{t}} \right ) Y_t \left (  d_1, \dots, d_{\overline{t}} \right )
\end{align}
As in \cite{hsushen2022}, the target parameter of interest is a class of Average Primary Treatment Effects implied by the first-period discontinuity. Letting $0_{\overline{t}-1}$ denote a vector of $\overline{t}-1$ zeros, the $\tau$-period-ahead APTE at the first-period cutoff is defined as
\begin{align}\label{class_apte_mult}
	\mathrm{APTE}_{1,\tau} \left ( c \right ) \equiv \E \left [ Y_{1+\tau} \left ( 1, 0_{\overline{t}-1} \right ) - Y_{1+\tau} \left ( 0, 0_{\overline{t}-1} \right ) | R_1 = c \right ]
\end{align}
It is convenient to interpret this class of target parameters as tracing out the impulse response function of the outcome $Y$ in a dynamic model in which the initial ``shock'' is determined by the threshold-crossing rule $D_1 \equiv \mathbb{I} \left [ R_1 \geq c \right ]$. Because treatment assignment exhibits path dependence, the sequence of states $0_{\overline{t}-1}$ may not be observed. As a matter of fact, it is counterfactual in most empirical applications. For instance, local jurisdictions in which voters reject a referendum for increased public spending are likely to submit the same proposition again and have it approved in a subsequent round of elections. In this scenario, a standard regression discontinuity estimand will fail to identify an APTE or a causal parameter with a clear economic interpretation. To formalize this argument, the following assumption generalizes the continuity restriction naturally embedded in the standard RDD model.

\begin{ass}[Continuity at the Cutoff]\label{ass_cont1} For any $t \in \left \{ 1, \dots, \overline{t} \right \}$, any $\left \{ d_j \right \}_{j=1}^{\overline{t}}$ with $d_j \in \left \{ 0,1 \right \}$ for all $j$, and any $\left \{ d'_j \right \}_{j=2}^{\overline{t}}$ with $d'_j \in \left \{ 0,1 \right \}$ for all $j$,
	\begin{align*}
		\E \left [ P \left ( d_1, \dots, d_{\overline{t}} \right ) | R_1 = r \right ] \quad \text{ and } \quad \E \left [ Y_{t} \left ( d_1, d_2, \dots, d_{\overline{t}} \right ) | R_1 = r, P \left ( d_1, d'_2, \dots, d'_{\overline{t}} \right ) \right ]
	\end{align*}
	are continuous at $r=c$.
\end{ass}
Assumption \ref{ass_cont1} generalizes Assumption \ref{ass_cont0} to a model with an arbitrary number of time periods. Intuitively, restricting $\E \left [ P \left ( d_1, \dots, d_{\overline{t}} \right ) | R_1 = r \right ]$ to be continuous at the first-period cutoff entails that agents must not systematically sort around the threshold to make one treatment path more likely than another. Similarly, the continuity of $\E \big [ Y_{t} \left ( d_1, d_2, \dots, d_{\overline{t}} \right ) | R_1 = r, P \left ( d_1, d'_2, \dots, d'_{\overline{t}} \right ) \big ]$ implies that observed and unobserved determinants of the outcome evolve smoothly at the cutoff, within each possible ``type'' defined by a path of potential treatment states. Under Assumption \ref{ass_cont1} and without further restrictions, it is immediate to show that a naive regression discontinuity estimand identifies a causal parameter with an undesirable interpretation.

\begin{prop}[Standard RD Estimand]\label{prop_rd1}
	Suppose that Assumption \ref{ass_cont1} holds. Then the estimand $\mathrm{RD}_t \left ( c \right ) \equiv \lim_{r \downarrow c} \E \left [ Y_t | R_1 = r \right ] - \lim_{r \uparrow c} \E \left [ Y_t | R_1 = r \right ]$ identifies
	
	\begin{small}
	\begin{align*}
		& \sum_{\left ( d_2, \dots, d_{\overline{t}} \right ) \in \left \{ 0,1 \right \}^{\overline{t}-1}} \E \left [ Y_t \left ( 1,d_2, \dots, d_{\overline{t}} \right )  | R_1 = c, P \left ( 1, d_2, \dots, d_{\overline{t}} \right ) = 1 \right ] \times \E \left [ P \left ( 1, d_2, \dots, d_{\overline{t}} \right ) | R_1 = c \right ] \\
		- \ & \sum_{\left ( d_2, \dots, d_{\overline{t}} \right ) \in \left \{ 0,1 \right \}^{\overline{t}-1}} \E \left [ Y_t \left ( 0, d_2, \dots, d_{\overline{t}} \right )  | R_1 = c, P \left ( 0, d_2, \dots, d_{\overline{t}} \right ) = 1 \right ] \times \E \left [ P \left ( 0, d_2, \dots, d_{\overline{t}} \right ) | R_1 = c \right ]
	\end{align*}
	\end{small}
\end{prop}

Since $P \left ( d_1, \dots, d_{\overline{t}} \right )$ is a Bernoulli random variable, each $\E \left [ P \left ( d_1, \dots, d_{\overline{t}} \right ) | R_1 = c \right ]$ is a conditional probability and $\sum_{\left ( d_2, \dots, d_{\overline{t}} \right ) \in \left \{ 0,1 \right \}^{\overline{t}-1}} \E \left [ P \left ( d_1, d_2, \dots, d_{\overline{t}} \right ) | R_1 = c \right ]$ is equal to one for $d_1 \in \left \{ 0,1 \right \}$. Thus, the causal parameter identified by a standard RDD estimand under Assumption \ref{ass_cont1} can be viewed as the contrast between two weighted sums of potential outcome means, where only the first-period treatment state is held fixed and each subsequent treatment path is weighted according to its probability at the focal threshold. This ``intent-to-treat'' may be sufficient should a researcher be interested in measuring the reduced-form (or ``total'') effect of a policy initiated by the first-period discontinuity, regardless of the subsequent path of treatment states. However, the composition of a long-term causal parameter is intrinsically of value in several empirical settings. For instance, does increased funding for local schools lead to sustained and permanent gains in student achievement or the effect of extra expenditure decays over time? To answer this and similar policy-relevant questions, a researcher may prefer to impose additional restrictions and estimate Average Primary Treatment Effects that trace out the impulse response of an outcome to the first-period round of treatment assignment. In the remainder of this section, I will first provide an identification argument for the immediate APTE and then consider long-term parameters. To identify the cutoff-specific effect of the first-period discontinuity on the outcome at the end of the same period, it is sufficient to assume that the outcome of interest does not reflect agents' knowledge of or expectations over future treatment states.

\begin{ass}[No Anticipation] \label{ass_noant1} For any $\left \{ d_j \right \}_{j=1}^{\overline{t}}$ with $d_j \in \left \{ 0,1 \right \}$ for all $j$, any $\left \{ d'_j \right \}_{j=2}^{\overline{t}}$ with $d'_j \in \left \{ 0,1 \right \}$ for all $j$, and any $\left \{ d''_j \right \}_{j=2}^{\overline{t}}$ with $d''_j \in \left \{ 0,1 \right \}$ for all $j$,
	\begin{equation*}
		\E \left [ Y_{1} \left ( d_1, d_2, \dots, d_{\overline{t}} \right ) - Y_{1} \left ( d_1, d'_2, \dots, d'_{\overline{t}} \right ) \big | R_1 = c, P \left ( d_1, d''_{2}, \dots,  d''_{\overline{t}} \right ) \right ] = 0
	\end{equation*}
\end{ass}

Starting from Proposition \ref{prop_rd1}, it is easy to show that that the immediate APTE is point identified under Assumption \ref{ass_noant1}. In fact, letting $t=1$, the no anticipation restriction entails that the treatment path subsequent to the first period can be equivalently replaced with a sequence of $t-1$ zeros. Then, repeated applications of the Law of Iterated Expectations and Assumption \ref{ass_cont1} yield the following proposition.

\begin{prop}[Identification of Immediate APTE]\label{prop_apte0_gen}
	Suppose that Assumption \eqref{ass_cont1} and Assumption \eqref{ass_noant1} hold. Then
	\begin{align*}
		\textsc{APTE}_{1,0} (c) = \lim_{r \downarrow c} \E \left [ Y_1 | R_1 = r \right ] - \lim_{r \uparrow c} \E \left [ Y_1 | R_1 = r \right ] \equiv \mathrm{RD}_1 \left ( c \right )
	\end{align*}
\end{prop}

This result generalizes Proposition \ref{prop_apte0} to a dynamic potential outcomes model with an arbitrary number of periods. In words, the immediate regression discontinuity estimand identifies an impulse response parameter provided that the outcome is not forward-looking at the first-period cutoff. However, by placing restrictions only on the conditional expectation of the first-period outcome, the no anticipation assumption is not sufficient to trace out the full impulse response function. To identify long-term Average Primary Treatment Effects, it is sufficient to generalize the common trends assumption introduced in the previous section.

\begin{ass}[Common Trends] \label{ass_ct1} For all $\left \{ d_j \right \}_{j=1}^{\overline{t}}$ with $d_j \in \left \{ 0,1 \right \}$,
	\begin{align*}
		& \E \left [ \Delta_{1,t} \left ( d_1, 0_{\overline{t}-1} \right ) | R_1 = c, P \left ( d_1, 0_{2:\overline{t}} \right ) \right ] = \E \left [ \Delta_{1,t} \left ( d_1, 0_{\overline{t}-1} \right ) | R_1 = c, P \left ( d_1, d_2, \dots, d_{\overline{t}} \right ) \right ]
	\end{align*}
with $\Delta_{1,t} \left ( d_1, 0_{\overline{t}-1} \right ) \equiv Y_{t} \left ( d_1, 0_{\overline{t}-1} \right ) - Y_1 \left ( d_1, 0_{\overline{t}-1} \right )$.
\end{ass}

For any initial potential treatment state $d_1$, $\Delta_{1,t} \left ( d_1, 0_{2:\overline{t}} \right )$ is a random variable that measures the level shift experienced by the never-treated potential outcome between the first and a subsequent period. Assumption \ref{ass_ct1} restricts the expectation of this random variable, conditional on $R_1 = c$, not to depend on the potential treatment path after $d_1$. In the canonical application of dynamic RDDs to the effect of local school bonds on student test scores, the path indicator $P \left ( d_1, d_2, \dots, d_{\overline{t}} \right )$ collapses the heterogeneity in jurisdiction ``types'' into a sequence of potential treatment states. Thus, Assumption \ref{ass_ct1} restricts the expected change in test scores after the first period, absent any subsequent referendum approval, not to depend on whether a jurisdiction is more or less likely to approve said referenda. This local ``parallel trends'' assumption, jointly with continuity and no anticipation, are sufficient to point identify all Average Primary Treatment Effects.

\begin{prop}[Identification of Cumulative APTE]\label{prop_apte1_gen}
	Suppose that Assumption \ref{ass_cont1}, Assumption \ref{ass_noant1}, and Assumption \ref{ass_ct1} hold. Then, for $\tau \in \left \{ 1, \dots, \overline{t}-1 \right \}$,
	
	\begin{small}
	\begin{align*}
		\mathrm{APTE}_{1,\tau} \left ( c \right ) & = \lim_{r \downarrow c} \E \left [ Y_{1} | R_1 = r \right ] - \lim_{r \uparrow c} \E \left [ Y_{1} | R_1 = r \right ] \\
		& + \lim_{r \downarrow c} \E \left [ Y_{1+\tau} - Y_{1} \bigg | R_1 = r, \bigcap_{s=2}^{1+\tau} \left \{ D_{s} = 0 \right \} \right ] - \lim_{r \uparrow c} \E \left [ Y_{1+\tau} - Y_{1} \bigg | R_1 = r, \bigcap_{s=2}^{1+\tau} \left \{ D_{s} = 0 \right \} \right ]
	\end{align*}
	\end{small}
\end{prop}

Proposition \ref{prop_apte1_gen} states that continuity, no anticipation, and common trends are sufficient for any long-term Average Primary Treatment Effect to be point identified as the sum of two regression discontinuity estimands implied by the first-period cutoff. The first estimand measures the average outcome contrast at the cutoff immediately after the first round of treatment assignment. On the other hand, the second term compares the average outcome growth on the two sides of the cutoff among agents who are not exposed to the treatment forever after.

\section{Estimation}

\subsection{Two-Period Model}

In this section, I will illustrate that standard linear regression tools are applicable to the estimation of Average Primary Treatment Effects in a two-period RDD. I will first focus on the nonparametric estimation achieved with a local constant regression and subsequently consider parametric, i.e., higher-order polynomial, specifications. For simplicity, I will abstract from sampling issues and keep relying on population operators until Section 6. Given a bandwidth $h > 0$, let $\mathcal{R}_h = \left [ c-h, c+h \right ]$ be a discontinuity window implied by realizations of the running variable around the first-period cutoff. Moreover, let $\mathcal{R}^{-}_h = \left [ c-h, c \right )$ and $\mathcal{R}^{+}_h = \left [ c, c+h \right ]$ indicate the left and right discontinuity windows, respectively. In the canonical static setting, the Average Treatment Effect at the cutoff is nonparametrically estimated as $\alpha_{+} - \alpha_{-}$, where $\alpha^{+}$ and $\alpha^{-}$ are linear regression coefficients in the local constant specifications
\begin{align}
	\E \left [ Y | R \in \mathcal{R}^{-}_{h} \right ] = \alpha^{-} \quad \text{ and } \quad \E \left [ Y | R \in \mathcal{R}^{+}_{h} \right ] = \alpha^{+}
\end{align}
In a dynamic model with two periods, the conditional outcome mean at the cutoff depends not only on the first-period treatment state (as in the static setting), but also on calendar time and the second-period treatment state. Thus, each local regression in a side of the discontinuity window will replicate the underlying conditional expectation as long as it contains four regressors. By way of illustration, consider the following specifications:
\begin{align}
	\E \left [ Y | R_1, R_1 \in \mathcal{R}^{-}_{h}, D_2, T \right ] & = \alpha^{-} + \beta^{-} D_2 + \gamma^{-} \mathbb{I} \left [ T=2 \right ] + \delta^{-} D_2 \mathbb{I} \left [ T=2 \right ] \\
	\E \left [ Y | R_1, R_1 \in \mathcal{R}^{+}_{h}, D_2, T \right ] & = \alpha^{+} + \beta^{+} D_2 + \gamma^{+} \mathbb{I} \left [ T=2 \right ] + \delta^{+} D_2 \mathbb{I} \left [ T=2 \right ]
\end{align}
For the purpose of estimating the one-period-ahead APTE, the coefficients of interest are $\gamma^{+}$ and $\gamma^{-}$, which measure $\E \left [ Y_2 - Y_1 | R \in \mathcal{R}^{+}_{h}, D_2 = 0 \right ]$ and $\E \left [ Y_2 - Y_1 | R \in \mathcal{R}^{-}_{h}, D_2 = 0 \right ]$, respectively. Combining these two conditional expectations yields
\begin{align}
	\E \left [ Y | R_1, R_1 \in \mathcal{R}_h, D_2, T \right ] & = \E \left [ Y | R_1, R_1 \in \mathcal{R}^{-}_{h}, D_2, T \right ] \times \left ( 1 - D_1 \right ) \nonumber \\
	& + \E \left [ Y | R_1, R_1 \in \mathcal{R}^{+}_{h}, D_2, T \right ] \times D_1 \nonumber \\
	& = \alpha^{-} + \beta^{-} D_2 + \gamma^{-} \mathbb{I} \left [ T=2 \right ] + \underbrace{\left ( \alpha^{+}-\alpha^{-} \right )}_{\equiv \alpha} D_1 \nonumber \\
	& + \delta^{-} D_2 \mathbb{I} \left [ T=2 \right ] + \underbrace{\left ( \beta^{+} - \beta^{-} \right )}_{\equiv \beta} D_1 D_2 + \underbrace{\left ( \gamma^{+}-\gamma^{-} \right )}_{\equiv \gamma} D_1 \mathbb{I} \left [ T=2 \right ] \nonumber \\
	& + \underbrace{\left ( \delta^{+} - \delta^{-} \right )}_{\equiv \delta} D_1 D_2 \mathbb{I} \left [ T=2 \right ]
\end{align}
In words, the coefficient of interest can be backed out by regressing the outcome on a constant, an indicator for the second period, two period-specific treatment indicators, and all of their interactions. In this saturated specification, $\gamma$ captures the second outcome mean contrast in Proposition \ref{prop_apte1}. On the other hand, the first outcome contrast can be nonparametrically estimated as in static regression discontinuity designs, i.e., as the slope coefficient in a saturated regression of the first-period outcome on a constant and an indicator for the first-period treatment. Departing from nonparametric estimation, the conditional outcome mean in the discontinuity window can also be approximated with a polynomial function of the first-period running variable. Let $\overline{p}$ denote the order of the polynomial basis, where $\overline{p} = 1$ corresponds to a local linear specification, and consider the following mean-squared error minimization problems:
\begin{align}
	\min_{\left \{ \alpha^{-}_p, \beta^{-}_p, \gamma^{-}_p, \delta^{-}_p \right \}_{p=0}^{\overline{p}}} \E \left [ \left ( \sum_{p=0}^{\overline{p}} \left ( \alpha^{-}_p + \beta^{-}_p D_2 + \gamma^{-}_p \mathbb{I} \left [ T=2 \right ] + \delta^{-}_p D_2 \mathbb{I} \left [ T=2 \right ] \right ) \left ( R_1 - c \right )^p \right )^2 \Bigg | R_1 \in \mathcal{R}^{-}_{1} \right ] \\
	\min_{\left \{ \alpha^{+}_p, \beta^{+}_p, \gamma^{+}_p, \delta^{+}_p \right \}_{p=0}^{\overline{p}}} \E \left [ \left ( \sum_{p=0}^{\overline{p}} \left ( \alpha^{+}_p + \beta^{+}_p D_2 + \gamma^{+}_p \mathbb{I} \left [ T=2 \right ] + \delta^{+}_p D_2 \mathbb{I} \left [ T=2 \right ] \right ) \left ( R_1 - c \right )^p \right )^2 \Bigg | R_1 \in \mathcal{R}^{+}_{1} \right ]
\end{align}
For the purpose of estimating the second outcome contrast in Proposition \ref{prop_apte1}, the parameter of interest is $\gamma^{+}_{0} - \gamma^{-}_{0}$. Instead, the immediate APTE can be estimated with a standard local polynomial specification.


\subsection{General Model}

To estimate immediate and cumulative APTEs in a model with $\overline{t}$ periods, it is sufficient to generalize the approach described in the two-period model. Given a bandwidth $h > 0$, let $\mathcal{R}^{-}_h = \left [ c-h, c \right )$ and $\mathcal{R}^{+}_h = \left [ c, c+h \right ]$ denote the left and right discontinuity windows implied by the first-period cutoff. To achieve nonparametric estimation of any cumulative Average Primary Treatment Effect based on Proposition \ref{prop_apte1_gen}, it is sufficient to replicate the conditional outcome mean with a saturated specification on either side of the threshold. In a dynamic model with $\overline{t}$ periods, this condition is met by including a set of $\overline{t}^2$ non perfectly multicollinear regressors. By way of illustration, consider the following two specifications:
\begin{align}
	\E \left [ Y \big | R_1, R_1 \in \mathcal{R}^{-}_{h}, \left \{ D_t \right \}_{t=2}^{\overline{t}}, T \right ] & = \alpha^{-} + \sum_{s=2}^{\overline{t}} \left ( \beta_{s}^{-} D_s + \gamma^{-}_{s} \mathbb{I} \left [ T=s \right ] \right ) + \sum_{s,s'} \delta^{-}_{s,s'} D_s \mathbb{I} \left [ T=s' \right ] \\
	\E \left [ Y \big | R_1, R_1 \in \mathcal{R}^{+}_{h}, \left \{ D_t \right \}_{t=2}^{\overline{t}}, T \right ] & = \alpha^{+} + \sum_{s=2}^{\overline{t}} \left ( \beta_{s}^{+} D_s + \gamma^{+}_{s} \mathbb{I} \left [ T=s \right ] \right ) + \sum_{s,s'} \delta^{+}_{s,s'} D_s \mathbb{I} \left [ T=s' \right ]
\end{align}
where $s,s' \in \left \{ 2, \dots, \overline{t} \right \}$. With this specification choice, the target estimands belong to the set $\left \{ \gamma^{+}_{s} - \gamma^{-}_{s} \right \}_{s=2}^{\overline{t}}$. Each element in this set approximates the second outcome contrast in Proposition \ref{prop_apte1_gen}, namely the difference between $\lim_{r \downarrow c} \E \left [ Y_{s} - Y_{1} \big | R_1 = r, \bigcap_{\tau=2}^{s} \left \{ D_{\tau} = 0 \right \} \right ]$ and $\lim_{r \uparrow c} \E \left [ Y_{s} - Y_{1} \big | R_1 = r, \bigcap_{\tau=2}^{s} \left \{ D_{\tau} = 0 \right \} \right ]$. On the other hand, the first outcome contrast at the threshold can be estimated as in static designs. For a higher-order polynomial approximation to the conditional mean of the outcome in the discontinuity window, it is immediate to extend the approach presented in Section 2. Letting $\overline{p}$ denote the order of the polynomial basis of the first-period running variable, the target parameters belong to the set $\left \{ \gamma^{+}_{0,s} - \gamma^{-}_{0,s} \right \}_{s=2}^{\overline{t}}$ which solve the following Mean Squared Error minimization problems:
\begin{align}
	\min_{\alpha^{-} \in \mathbb{R}^{p}, \beta^{-} \in \mathbb{R}^{p \times \overline{t}}, \gamma^{-} \in \mathbb{R}^{p \times \overline{t}}, \delta^{-} \in \mathbb{R}^{p \times \overline{t}^2}} \E \left [ \left ( \sum_{p=0}^{\overline{p}} G^{-}_{p} \left ( R_1 - c \right )^p \right )^2 \Bigg | R_1 \in \mathcal{R}^{-}_{1} \right ] \\
	\min_{\alpha^{+} \in \mathbb{R}^{p}, \beta^{+} \in \mathbb{R}^{p \times \overline{t}}, \gamma^{+} \in \mathbb{R}^{p \times \overline{t}}, \delta^{+} \in \mathbb{R}^{p \times \overline{t}^2}} \E \left [ \left ( \sum_{p=0}^{\overline{p}} G^{+}_{p} \left ( R_1 - c \right )^p \right )^2 \Bigg | R_1 \in \mathcal{R}^{+}_{1} \right ]
\end{align}
with
\begin{align}
	G^{-}_{p} \equiv \alpha^{-}_{p} + \sum_{s=2}^{\overline{t}} \left ( \beta_{s,p}^{-} D_s + \gamma^{-}_{s,p} \mathbb{I} \left [ T=s \right ] \right ) + \sum_{s,s'} \delta^{-}_{s,s',p} D_s \mathbb{I} \left [ T=s' \right ] \\
	G^{+}_{p} \equiv \alpha^{+}_{p} + \sum_{s=2}^{\overline{t}} \left ( \beta_{s,p}^{+} D_s + \gamma^{+}_{s,p} \mathbb{I} \left [ T=s \right ] \right ) + \sum_{s,s'} \delta^{+}_{s,s',p} D_s \mathbb{I} \left [ T=s' \right ]
\end{align}
As in the two-period model, the immediate APTE can be estimated with standard local polynomial specifications.


\section{Aggregation of Time-Specific Parameters}

The identification arguments presented so far have focused on causal parameters implied by the first-period discontinuity. In practice, the focal round of treatment assignment may not occur at $t=1$. Moreover, agents may be exposed to multiple rounds of treatment assignment at different points in time. To allow for the focal discontinuity to occur in an arbitrary period $t$, it is convenient to introduce a random variable $H_{t-1} \in \left \{ 0,1 \right \}^{t-1}$ that summarizes the history of treatment states up to and excluding period $t$. Without further restrictions, past treatment states may affect the probability distribution of potential outcomes when the treatment is assigned. Thus, for a causal parameter to be interpretable, it must contrast agents who have been exposed to an identical path of treatment states. Letting $t$ denote the focal discontinuity period, the Average Primary Treatment Effect at the period-$t$ cutoff subsequent to the treatment history $h_{t-1} \in \supp \left ( H_{t-1} \right )$ is defined as
\begin{align}
	\mathrm{APTE}_{t,\tau} \left ( h_{t-1}, c \right ) \equiv \E \left [ Y_{t+\tau} \left ( h_{t-1}, 1, 0_{\overline{t}-t} \right ) - Y_{t+\tau} \left ( h_{t-1}, 0, 0_{\overline{t}-t} \right ) | H_{t-1} = h_{t-1}, R_t = c \right ]
\end{align}
for any $\tau \in \left \{ 0, 1, \dots, \overline{t} - t \right \}$\footnote{It is not necessary for the conditioning set to include the event $H_{t-1} = h_{t-1}$. However, an empirical researcher is unlikely to be interested in the average effect of a treatment subsequent to a treatment history $h_{t-1}$ in the subpopulation implied by a \textit{different} treatment history $h'_{t-1}$. For practical purposes, it is natural for \textit{one} treatment history to both be an argument of potential outcomes and belong to the conditioning set.}. Importantly, potential outcomes differ solely by the period-$t$ treatment state, implying that this class of impulse response parameters generalizes equation \eqref{class_apte_mult}. Because each $\mathrm{APTE}_{t,\tau} \left ( h_{t-1}, c \right )$ is specific to a history of treatment states, focal period of treatment assignment, and number of leads, it is natural to think of $\mathrm{APTE}_{t,\tau} \left ( h_{t-1}, c \right )$ as a building block to construct aggregate policy-relevant parameters. This interpretation is similar in spirit to the one put forward by \cite{callawaysantanna2021}, who proposes several approaches to aggregate cohort- and time-specific causal parameters in the context of difference-in-differences designs with staggered adoption of an absorbing treatment. Analogously, \cite{bojinovetal2021} aggregates unit-time-specific causal parameters in sequentially randomized experiments. In dynamic RDDs, a natural first step is to aggregate target parameters over histories of treatment states, for given focal period of treatment assignment and number of leads. The resulting parameter is defined as
\begin{align}
	\mathrm{APTE}_{t,\tau} \left ( c \right ) \equiv \sum_{h_{t-1} \in \supp \left ( H_{t-1} \right )} \mathrm{APTE}_{t,\tau} \left ( h_{t-1}, c \right ) \times \Prob \left ( H_{t-1} = h_{t-1} \right )
\end{align}
$\mathrm{APTE}_{t,\tau} \left ( c \right )$ integrates Average Primary Treatment Effects over the probability distribution of histories up to the focal round of treatment assignment. It can thus be interpreted as the average impulse response of the outcome to the period-$t$ discontinuity, $\tau$ periods after the initial shock. In addition, an empirical researcher may be interested in aggregating time-specific parameters into a single summary measure of the $\tau$-period-ahead effect of the treatment. In this case, assuming a balanced panel, each term of the aggregate impulse response function is defined as
\begin{align}
	\mathrm{APTE}_{\tau} \left ( c \right ) \equiv \frac{1}{\overline{t}-\tau} \sum_{t=1}^{\overline{t}-\tau} \mathrm{APTE}_{t,\tau} \left ( c \right )
\end{align}

\subsection{Dimension Reduction via Limited Path Independence}

Assumptions \ref{ass_cont1}, \ref{ass_noant1}, and \ref{ass_ct1} are easily generalized to achieve the point identification of each possible $\mathrm{APTE}_{t,\tau} \left ( h_{t-1}, c \right )$. In observational settings, however, it is generally infeasible to estimate one target parameter for each possible treatment history, focal calendar time, and number of leads. For instance, in the canonical application of dynamic RDDs to the effects of local referenda, units infrequently participate to treatment assignment because jurisdictions typically hold new elections only once funds approved by a previously approved proposition are close to expiration. As a consequence, only a limited number of local governments participates to treatment assignment every year and any given jurisdiction does so in a cyclical fashion. In addition, a narrowly rejected referendum is generally submitted to voters again only once or twice briefly after the initial attempt, implying that the identification challenge posed by repeated treatment assignment is practically salient only in a narrow time window. To summarize, in the local public finance application of dynamic RDDs, treatment assignment is sparse and cyclical. Motivated by this fact, it is natural to impose additional restrictions to limit the number of target parameters at stake and reduce the dimensionality of the problem. In a similar fashion to \cite{hsushen2022}, I will assume that all causal parameters of interest share a weak form of limited path independence. Specifically, for any given focal round of treatment assignment, I restrict the Average Primary Treatment Effect not to depend on the history of treatment states provided that units are not exposed to the treatment for a specific number of periods prior to the focal round of treatment assignment.

\begin{ass}[Limited Path Independence] \label{ass_mark} Let $k \in \mathbb{N}_{+}$ be a deterministic constant. For any $h_{t-k-1}, h'_{t-k-1} \in \supp \left ( H_{t-k-1} \right )$ such that $h_{t-1} = \left ( h_{t-k-1},0_k \right )$ and $h'_{t-1} = \left ( h'_{t-k-1},0_k \right )$,
	\begin{align*}
		& \E \left [ Y_{t} \left ( h_{t-1},1,0_{\overline{t}-t} \right ) - Y_{t} \left ( h_{t-1},0,0_{\overline{t}-t} \right ) | H_{t-1} = h_{t-1}, R_t = c \right ] \\
		= \ & \E \left [ Y_{t} \left ( h'_{t-1},1,0_{\overline{t}-t} \right ) - Y_{t} \left ( h'_{t-1},0,0_{\overline{t}-t} \right ) | H_{t-1} = h'_{t-1}, R_t = c \right ]
	\end{align*}
for all $t > k$. Thus, $\mathrm{APTE}_{t,\tau} \left ( h_{t-1}, c \right ) = \mathrm{APTE}_{t,\tau} \left ( h'_{t-1}, c \right )$ for all $\tau \in \left \{ 0,1, \dots, \overline{t} - t \right \}$.
\end{ass}

The constant $k$ is to be determined by the researcher depending on the empirical application\footnote{In a similar spirit, \cite{callawaysantanna2021} considers a limited treatment anticipation assumption, based on which the researcher must specify the number of time periods the outcome variable may reflect agents' knowledge of (or expectations over) future exposure to the treatment.}. Clearly, the choice of a larger $k$ entails a more stringent requirement for the limited path independence assumption to be satisfied. On the other hand, a smaller $k$ restricts APTEs not to be determined in time periods further antecedent to the focal round of treatment assignment. The practical implication of this Markov-type assumption is that, for the purpose of estimating a target parameter, units that share a common ``recent'' past can be pooled, thereby improving statistical inference. Another consequence of Assumption \ref{ass_mark} is that the plausibility of the local common trends restriction, namely Assumption \ref{ass_ct1}, may be tested as in difference-in-differences and event-study designs. Because all units are untreated for $k$ periods before the focal calendar time period, a researcher may test whether the conditional mean of the outcome exhibits pre-trends at the focal cutoff. For ease of exposition, consider a four-period model ($\overline{t}=4$) in which all units are untreated until $t=3$, when a round of treatment assignment occurs. For $\left ( t,\tau \right ) = \left ( 3,1 \right )$, let the target parameter be
\begin{align}
	\mathrm{APTE}_{t,\tau} \left ( 0_2, c \right ) \equiv \E \left [ Y_{t+\tau} \left ( 0_2,1,0 \right ) - Y_{t+\tau} \left ( 0_2,0,0 \right ) | H_2 = 0_2, R_3 = c \right ]
\end{align}
This is the cutoff-specific, one-period-ahead average effect of the treatment assigned in $t=3$, conditional on a never-treated path in the first two periods. Because, by construction, $D_1 = D_2 = 0$, the history of treatment states $H_2 = 0_2$ is observed. To identify $\mathrm{APTE}_{3,1} \left ( 0_2, c \right )$, Assumption \ref{ass_ct1} restricts $\E \left [ Y_4 \left ( 0_2,d_1,0 \right ) - Y_3 \left ( 0_2,d_1,0 \right ) | H_2 = 0_2, R_3 = c, P \left ( 0_2, d_1, 0 \right ) = 1 \right ]$ to be equal to $\E \left [ Y_4 \left ( 0_2,d_1,0 \right ) - Y_3 \left ( 0_2,d_1,0 \right ) | H_2 = 0_2, R_3 = c, P \left ( 0_2, d_1, 1 \right ) = 1 \right ]$ for $d_1 \in \left \{ 0,1 \right \}$. Leveraging the intuition that the path indicator random variable $P$ describes a unit's treatment-taking behavior and thus their ``type'', it is natural to expect that a similar common trends assumption hold in time periods preceding the focal round of treatment assignment. In other words, if a unit's ``type'' is time-invariant, one may plausibly conjecture that
\begin{align}
	& \E \left [ Y_2 \left ( 0_2,d_1,0 \right ) - Y_1 \left ( 0_2,d_1,0 \right ) | H_2 = 0_2, R_3 = c, P \left ( 0_2, d_1, 1 \right ) = 0 \right ] \nonumber \\
	= \ & \E \left [ Y_2 \left ( 0_2,d_1,0 \right ) - Y_1 \left ( 0_2,d_1,0 \right ) | H_2 = 0_2, R_3 = c, P \left ( 0_2, d_1, 1 \right ) = 1 \right ]
\end{align}
for $d_1 \in \left \{ 0,1 \right \}$. If the outcome variable is non-anticipatory and Assumption \ref{ass_cont1} holds, these conditional expectations are identified for both the upper and lower limit of the running variable in $t=3$. Specifically,
\begin{align}
	\lim_{r \downarrow c} \E \left [ Y_2 - Y_1 | H_2 = 0_2, R_3 = r, D_4 = 0 \right ] = \lim_{r \downarrow c} \E \left [ Y_2 - Y_1 | H_2 = 0_2, R_3 = r, D_4 = 1 \right ] \\
	\lim_{r \uparrow c} \E \left [ Y_2 - Y_1 | H_2 = 0_2, R_3 = r, D_4 = 0 \right ] = \lim_{r \uparrow c} \E \left [ Y_2 - Y_1 | H_2 = 0_2, R_3 = r, D_4 = 1 \right ]
\end{align}
These equalities naturally give rise to testable restrictions aimed at assessing the plausibility of Assumption \ref{ass_ct1}. Specifically, an empirical researcher may provide a more compelling argument in favor of the local common trends restriction by failing to reject the equality of average differences in outcome trends at the cutoff in all available ``pre-periods''. In the canonical difference-in-differences design, it is common practice to test the equality of pre-period outcome trends across groups implied by whether units are assigned to the ``treatment'' or ``control'' arm. In a dynamic RDD, groups are implicitly determined by the treatment path subsequent to the focal time period. Unless more restrictions are placed on the evolution of treatment states, the number of groups or ``types'' will in general exceed two, thereby increasing the number of testable equalities.

%\section{Simulation}

\section{Conclusion}

This paper proposed a novel argument to point identify economically interpretable intertemporal treatment effects in dynamic regression discontinuity designs. I developed a dynamic potential outcomes model and showed that two assumptions of the difference-in-differences literature, the no anticipation and common trends restrictions, can be specialized to point identify local impulse response-like treatment effects. The estimand associated with each target parameter can be expressed as the sum of two static RDD outcome contrasts, thereby allowing for estimation via standard local polynomial tools. I leveraged a limited path independence assumption to reduce the dimensionality of the problem.










%%%%%%%%%%%%%%%%%%%%%%%%%%%%%%%%%%%%%%
%%%%%%%%%%%%% REFERENCES %%%%%%%%%%%%%
%%%%%%%%%%%%%%%%%%%%%%%%%%%%%%%%%%%%%%

\newpage

\begin{spacing}{1} 
	%\def\bibfont{\small}
	\renewcommand\refname{References}
	\bibliography{references}
\end{spacing}



%%%%%%%%%%%%%%%%%%%%%%%%%%%%%%%%%%%%%
%%%%%%%%%%%%%% FIGURES %%%%%%%%%%%%%%
%%%%%%%%%%%%%%%%%%%%%%%%%%%%%%%%%%%%%

\newpage

\section*{Figures}



\newpage

%%%%%%%%%%%%%%%%%%%%%%%%%%%%%%%%%%%%%%
%%%%%%%%%%%%%%% TABLES %%%%%%%%%%%%%%%
%%%%%%%%%%%%%%%%%%%%%%%%%%%%%%%%%%%%%%

\newpage

\section*{Tables}



\newpage



\appendix
\section*{Appendix}

%\pagenumbering{arabic}
\renewcommand{\thesubsection}{\Alph{section}.\arabic{subsection}}
\renewcommand{\thesubsubsection}{\Alph{section}.\arabic{subsection}.\arabic{subsubsection}}

\setcounter{figure}{0} \renewcommand{\thefigure}{A\arabic{figure}}
\setcounter{table}{0} \renewcommand{\thetable}{A\arabic{table}}

\section{Proofs}

\subsection{Proof of Proposition \ref{prop_rd}}

The sharp RDD estimand implied by the first-period discontinuity is defined as
\begin{align*}
	\mathrm{RD}_{t} \left ( c \right ) \equiv \lim_{r \downarrow c} \E \left [ Y_t | R_1 = r \right ] - \lim_{r \uparrow c} \E \left [ Y_t | R_1 = r \right ] 
\end{align*}
Consider the first term:
\begin{align*}
	\lim_{r \downarrow c} \E \left [ Y_t | R_1 = r \right ] & = \lim_{r \downarrow c} \E \left [ Y_t | R_1 = r, D_1 = 1 \right ] \\
	& = \lim_{r \downarrow c} \E \left [ \mathbb{I} \left [ D_1 = 1, D_2 \left ( 1 \right ) = 0 \right ] Y_t \left ( 1,0 \right )  | R_1 = r, D_1 = 1 \right ] \\
	& + \lim_{r \downarrow c} \E \left [ \mathbb{I} \left [ D_1 = 1, D_2 \left ( 1 \right ) = 1 \right ] Y_t \left ( 1,1 \right )  | R_1 = r, D_1 = 1 \right ] \\
	& + \lim_{r \downarrow c} \E \left [ \mathbb{I} \left [ D_1 = 0, D_2 \left ( 1 \right ) = 0 \right ] Y_t \left ( 0,0 \right )  | R_1 = r, D_1 = 1 \right ] \\
	& + \lim_{r \downarrow c} \E \left [ \mathbb{I} \left [ D_1 = 0, D_2 \left ( 1 \right ) = 1 \right ] Y_t \left ( 0,1 \right )  | R_1 = r, D_1 = 1 \right ]
\end{align*}
The first equality exploits the assumption that $D_1 \equiv \mathbb{I} \left [ R_1 \geq c \right ]$, which implies that the conditioning set can be expanded to include the event $D_1=1$. The second equality replaces the observed outcome $Y_t$ with its stochastic linear combination of potential outcomes, according to equation \eqref{potout}. Furthermore, because each of the four expectations is conditional on the event $D_1 = 1$, any indicator containing the event $D_1 = 0$ will be false. For the same reason, the event $D_1 = 1$ can be subsumed from any indicator that contains it. Thus,
\begin{align*}
	\lim_{r \downarrow c} \E \left [ Y_t | R_1 = r \right ] & = \lim_{r \downarrow c} \E \left [ \mathbb{I} \left [ D_2 \left ( 1 \right ) = 0 \right ] Y_t \left ( 1,0 \right )  | R_1 = r, D_1 = 1 \right ] \\
	& + \lim_{r \downarrow c} \E \left [ \mathbb{I} \left [ D_2 \left ( 1 \right ) = 1 \right ] Y_t \left ( 1,1 \right )  | R_1 = r, D_1 = 1 \right ] \\
	& = \lim_{r \downarrow c} \E \left [ \mathbb{I} \left [ D_2 \left ( 1 \right ) = 0 \right ] Y_t \left ( 1,0 \right )  | R_1 = r \right ] \\
	& + \lim_{r \downarrow c} \E \left [ \mathbb{I} \left [ D_2 \left ( 1 \right ) = 1 \right ] Y_t \left ( 1,1 \right )  | R_1 = r \right ] \\
	& =  \lim_{r \downarrow c} \E \left [ Y_t \left ( 1,0 \right )  | R_1 = r, D_2 \left ( 1 \right ) = 0 \right ] \times \Prob \left ( D_2 \left ( 1 \right ) = 0 | R_1 = r \right ) \\
	& + \lim_{r \downarrow c} \E \left [ Y_t \left ( 1,1 \right )  | R_1 = r, D_2 \left ( 1 \right ) = 1 \right ] \times \Prob \left ( D_2 \left ( 1 \right ) = 1 | R_1 = r \right ) \\
	& = \E \left [ Y_t \left ( 1,0 \right )  | R_1 = c, D_2 \left ( 1 \right ) = 0 \right ] \times \Prob \left ( D_2 \left ( 1 \right ) = 0 | R_1 = c \right ) \\
	& + \E \left [ Y_t \left ( 1,1 \right )  | R_1 = c, D_2 \left ( 1 \right ) = 1 \right ] \times \Prob \left ( D_2 \left ( 1 \right ) = 1 | R_1 = c \right )
\end{align*}
As above, the second equality exploits the assumption that $D_1 \equiv \mathbb{I} \left [ R_1 \geq c \right ]$. The third equality follows from an application of the Law of Iterated Expectations. The fourth equality leverages Assumption \ref{ass_cont0} and the algebraic properties of limits. By a symmetric argument, the right-hand side of $\mathrm{RD}_t \left ( c \right )$ identifies
\begin{align*}
	\lim_{r \uparrow c} \E \left [ Y_t | R_1 = r \right ] & = \E \left [ Y_t \left ( 0,0 \right )  | R_1 = c, D_2 \left ( 0 \right ) = 0 \right ] \times \Prob \left ( D_2 \left ( 0 \right ) = 0 | R_1 = c \right ) \\
	& + \E \left [ Y_t \left ( 0,1 \right )  | R_1 = c, D_2 \left ( 0 \right ) = 1 \right ] \times \Prob \left ( D_2 \left ( 0 \right ) = 1 | R_1 = c \right )
\end{align*}
which completes the proof.

\subsection{Proof of Proposition \ref{prop_apte0}}

The immediate Average Primary Treatment Effect (APTE) implied by the first-period discontinuity is defined as
\begin{align*}
	\textsc{APTE}_{1,0} \left ( c \right ) \equiv \E \left [ Y_1 \left ( 1,0 \right ) | R_1 = c \right ] - \E \left [ Y_1 \left ( 0,0 \right ) | R_1 = c \right ]
\end{align*}
By an application of the Law of Iterated Expectations, the first term can be expressed as
\begin{align}\label{apte0_decomp}
	\E \left [ Y_1 \left ( 1,0 \right ) | R_1 = c \right ] & = \E \left [ Y_1 \left ( 1,0 \right ) | R_1 = c, D_2 \left ( 1 \right ) = 0 \right ] \times \Prob \left ( D_2 \left ( 1 \right ) = 0 | R_1 = c \right ) \\
	& + \E \left [ Y_1 \left ( 1,0 \right ) | R_1 = c, D_2 \left ( 1 \right ) = 1 \right ] \times \Prob \left ( D_2 \left ( 1 \right ) = 1 | R_1 = c \right )
\end{align}
Both conditional probabilities are identified under Assumption \ref{ass_cont0}. In fact, for $d_2 \in \left \{ 0,1 \right \}$,
\begin{align*}
	\Prob \left ( D_2 \left ( 1 \right ) = d_2 | R_1 = c \right ) & = \lim_{r \downarrow c} \Prob \left ( D_{2} \left ( 1 \right ) = d_2 | R_1 = r \right ) \\
	& = \lim_{r \downarrow c} \Prob \left ( D_{2} \left ( 1 \right ) = d_2 | R_1 = r, D_1 = 1 \right ) \\
	& = \lim_{r \downarrow c} \Prob \left ( D_{2} = d_2 | R_1 = r, D_1 = 1 \right ) \\
	& = \lim_{r \downarrow c} \Prob \left ( D_{2} = d_2 | R_1 = r \right )
\end{align*}
The first equality leverages Assumption \ref{ass_cont0}, which states that $\E \left [ D_2 \left ( 1 \right ) | R_1 = r \right ]$ is a continuous function of $r$ at $r=c$. The second and fourth equalities follow from the threshold-crossing definition of the treatment, $D_1 \equiv \mathbb{I} \left [ R_1 \geq c \right ]$. The third equality stems from the fact that, conditional on $D_1 = 1$, the second-period potential treatment $D_2 \left ( 1 \right )$ is observed. By a similar argument, the first conditional expectation in equation \eqref{apte0_decomp} is identified as
\begin{align*}
	\E \left [ Y_1 \left ( 1,0 \right ) | R_1 = c, D_2 \left ( 1 \right ) = 0 \right ] & = \lim_{r \downarrow c} \E \left [ Y_1 \left ( 1,0 \right ) | R_1 = r, D_2 \left ( 1 \right ) = 0 \right ] \\
	& = \lim_{r \downarrow c} \E \left [ Y_1 \left ( 1,0 \right ) | R_1 = r, D_1 = 1, D_2 \left ( 1 \right ) = 0 \right ] \\
	& = \lim_{r \downarrow c} \E \left [ Y_1 \left ( 1,0 \right ) | R_1 = r, D_1 = 1, D_2 = 0 \right ] \\
	& = \lim_{r \downarrow c} \E \left [ Y_1 | R_1 = r, D_1 = 1, D_2 = 0 \right ] \\
	& = \lim_{r \downarrow c} \E \left [ Y_1 | R_1 = r, D_2 = 0 \right ]
\end{align*}
The first equality leverages Assumption \ref{ass_cont0}, which states that $\E \left [ Y_1 \left ( 1,0 \right ) | R_1 = r, D_2 \left ( 1 \right ) = 0 \right ]$ is a continuous function of $r$ at $r=c$. The second and fifth equalities follow from the threshold-crossing definition of the treatment, $D_1 \equiv \mathbb{I} \left [ R_1 \geq c \right ]$. The third equality stems from the fact that, conditional on $D_1 = 1$, the second-period potential treatment $D_2 \left ( 1 \right )$ is observed. Analogously, the potential outcome $Y_1 \left ( 1,0 \right )$ is observed conditional on $D_1 = 1$ and $D_2 = 0$, thus justifying the fourth equality. Finally, under Assumption \ref{ass_cont0} and Assumption \ref{ass_noant}, the counterfactual conditional mean in equation \eqref{apte0_decomp} is identified as
\begin{align*}
	\E \left [ Y_1 \left ( 1,0 \right ) | R_1 = c, D_2 \left ( 1 \right ) = 1 \right ] & = \E \left [ Y_1 \left ( 1,1 \right ) | R_1 = c, D_2 \left ( 1 \right ) = 1 \right ]\\
	& = \lim_{r \downarrow c} \E \left [ Y_1 \left ( 1,1 \right ) | R_1 = r, D_2 \left ( 1 \right ) = 1 \right ] \\
	& = \lim_{r \downarrow c} \E \left [ Y_1 \left ( 1,1 \right ) | R_1 = r, D_1 = 1, D_2 \left ( 1 \right ) = 1 \right ] \\
	& = \lim_{r \downarrow c} \E \left [ Y_1 \left ( 1,1 \right ) | R_1 = r, D_1 = 1, D_2 = 1 \right ] \\
	& = \lim_{r \downarrow c} \E \left [ Y_1 | R_1 = r, D_1 = 1, D_2 = 1 \right ] \\
	& = \lim_{r \downarrow c} \E \left [ Y_1 | R_1 = r, D_2 = 1 \right ]
\end{align*}
The first equality applies the no anticipation restriction embedded in Assumption \ref{ass_noant}. The second equality leverages Assumption \ref{ass_cont0}, which states that $\E \left [ Y_1 \left ( 1,1 \right ) | R_1 = r, D_2 \left ( 1 \right ) = 1 \right ]$ is a continuous function of $r$ at $r=c$. The third and sixth equalities follow from the threshold-crossing definition of the treatment, $D_1 \equiv \mathbb{I} \left [ R_1 \geq c \right ]$. The fourth equality stems from the fact that, conditional on $D_1 = 1$, the second-period potential treatment $D_2 \left ( 1 \right )$ is observed. Analogously, the potential outcome $Y_1 \left ( 1,1 \right )$ is observed conditional on $D_1 = 1$ and $D_2 = 1$, thus justifying the fifth equality. Combining these identification results, the first term in $\textsc{APTE}_{1,0} (c)$ can be expressed as
\begin{align*}
	\E \left [ Y_1 \left ( 1,0 \right ) | R_1 = c \right ] & = \lim_{r \downarrow c} \E \left [ Y_1 | R_1 = r, D_2 = 0 \right ] \times \lim_{r \downarrow c} \Prob \left ( D_{2}=0 | R_1 = r \right ) \\
	& + \lim_{r \downarrow c} \E \left [ Y_1 | R_1 = r, D_2 = 1 \right ] \times \lim_{r \downarrow c} \Prob \left ( D_{2}=1 | R_1 = r \right ) \\
	& = \lim_{r \downarrow c} \E \left [ Y_1 | R_1 = r \right ]
\end{align*}
where the second equality follows from another application of the Law of Iterated Expectations. By a symmetric argument, $\lim_{r \uparrow c} \E \left [ Y_1 | R_1 = r \right ]$ identifies $\E \left [ Y_1 \left ( 0,0 \right ) | R_1 = c \right ]$. Thus,
\begin{align*}
	\textsc{APTE}_{1,0} \left ( c \right ) = \lim_{r \downarrow c} \E \left [ Y_1 | R_1 = r \right ] - \lim_{r \uparrow c} \E \left [ Y_1 | R_1 = r \right ]
\end{align*}


\subsection{Proof of Proposition \ref{prop_apte1}}

The cumulative Average Primary Treatment Effect (APTE) implied by the first-period discontinuity is defined as
\begin{align*}
	\textsc{APTE}_{1,1} \left ( c \right ) \equiv \E \left [ Y_2 \left ( 1,0 \right ) | R_1 = c \right ] - \E \left [ Y_2 \left ( 0,0 \right ) | R_1 = c \right ]
\end{align*}
By an application of the Law of Iterated Expectations, the first term can be expressed as
\begin{align}\label{apte1_decomp}
	\E \left [ Y_2 \left ( 1,0 \right ) | R_1 = c \right ] & = \E \left [ Y_2 \left ( 1,0 \right ) | R_1 = c, D_2 \left ( 1 \right ) = 0 \right ] \times \Prob \left ( D_2 \left ( 1 \right ) = 0 | R_1 = c \right ) \\
	& + \E \left [ Y_2 \left ( 1,0 \right ) | R_1 = c, D_2 \left ( 1 \right ) = 1 \right ] \times \Prob \left ( D_2 \left ( 1 \right ) = 1 | R_1 = c \right )
\end{align}
As shown in the previous section, both conditional probabilities are identified under Assumption \ref{ass_cont0}. Furthermore, the first conditional expectation is identified as
\begin{align*}
	\E \left [ Y_2 \left ( 1,0 \right ) | R_1 = c, D_2 \left ( 1 \right ) = 0 \right ] & = \lim_{r \downarrow c} \E \left [ Y_2 \left ( 1,0 \right ) | R_1 = r, D_2 \left ( 1 \right ) = 0 \right ] \\
	& = \lim_{r \downarrow c} \E \left [ Y_2 \left ( 1,0 \right ) | R_1 = r, D_1 = 1, D_2 \left ( 1 \right ) = 0 \right ] \\
	& = \lim_{r \downarrow c} \E \left [ Y_2 \left ( 1,0 \right ) | R_1 = r, D_1 = 1, D_2 = 0 \right ] \\
	& = \lim_{r \downarrow c} \E \left [ Y_2 | R_1 = r, D_1 = 1, D_2 = 0 \right ] \\
	& = \lim_{r \downarrow c} \E \left [ Y_2 | R_1 = r, D_2 = 0 \right ]
\end{align*}
The first equality leverages Assumption \ref{ass_cont0}, which states that $\E \left [ Y_2 \left ( 1,0 \right ) | R_1 = r, D_2 \left ( 1 \right ) = 0 \right ]$ is a continuous function of $r$ at $r=c$. The second and fifth equalities follow from the threshold-crossing definition of the treatment, $D_1 \equiv \mathbb{I} \left [ R_1 \geq c \right ]$. The third equality stems from the fact that, conditional on $D_1 = 1$, the second-period potential treatment $D_2 \left ( 1 \right )$ is observed. Analogously, the potential outcome $Y_2 \left ( 1,0 \right )$ is observed conditional on $D_1 = 1$ and $D_2 = 0$, thus justifying the fourth equality. Finally, under Assumption \ref{ass_cont0}, Assumption \ref{ass_noant}, and Assumption \ref{ass_ct0}, the counterfactual conditional mean in equation \eqref{apte1_decomp} is identified as
\begin{align*}
	\E \left [ Y_{2} \left ( 1,0 \right ) | R_1 = c, D_{2} \left ( 1 \right )=1 \right ] & = \E \left [ Y_1 \left ( 1,0 \right ) | R_1 = c, D_{2} \left ( 1 \right ) = 1 \right ] \\
	& + \E \left [ Y_{2} \left ( 1,0 \right ) - Y_1 \left ( 1,0 \right ) | R_1 = c, D_{2} \left ( 1 \right ) = 0 \right ] \\
	& = \E \left [ Y_1 \left ( 1,1 \right ) | R_1 = c, D_{2} \left ( 1 \right ) = 1 \right ] \\
	& + \E \left [ Y_{2} \left ( 1,0 \right ) - Y_1 \left ( 1,0 \right ) | R_1 = c, D_{2} \left ( 1 \right ) = 0 \right ] \\
	& = \lim_{r \downarrow c} \E \left [ Y_1 \left ( 1,1 \right ) | R_1 = r, D_{2} \left ( 1 \right ) = 1 \right ] \\
	& + \lim_{r \downarrow c} \E \left [ Y_{2} \left ( 1,0 \right ) - Y_1 \left ( 1,0 \right ) | R_1 = r, D_{2} \left ( 1 \right ) = 0 \right ] \\
	& = \lim_{r \downarrow c} \E \left [ Y_1 \left ( 1,1 \right ) | R_1 = r, D_1 = 1, D_{2} \left ( 1 \right ) = 1 \right ] \\
	& + \lim_{r \downarrow c} \E \left [ Y_{2} \left ( 1,0 \right ) - Y_1 \left ( 1,0 \right ) | R_1 = r, D_1 = 1, D_{2} \left ( 1 \right ) = 0 \right ] \\
	& = \lim_{r \downarrow c} \E \left [ Y_1 \left ( 1,1 \right ) | R_1 = r, D_1 = 1, D_{2} = 1 \right ] \\
	& + \lim_{r \downarrow c} \E \left [ Y_{2} \left ( 1,0 \right ) - Y_1 \left ( 1,0 \right ) | R_1 = r, D_1 = 1, D_{2} = 0 \right ] \\
	& = \lim_{r \downarrow c} \E \left [ Y_1 | R_1 = r, D_1 = 1, D_{2}=1 \right ] \\
	& + \lim_{r \downarrow c} \E \left [ Y_{2} - Y_1 | R_1 = r, D_1 = 1, D_{2}=0 \right ] \\
	& = \lim_{r \downarrow c} \E \left [ Y_1 | R_1 = r, D_{2}=1 \right ] \\
	& + \lim_{r \downarrow c} \E \left [ Y_{2} - Y_1 | R_1 = r, D_{2}=0 \right ]
\end{align*}
The first equality uses Assumption \ref{ass_ct0}, which states that $Y_{2} \left ( 1,0 \right ) - Y_1 \left ( 1,0 \right )$ is mean independent of $D_{2} \left ( 1 \right )$ at the first-period cutoff. The second equality applies the no anticipation restriction embedded in Assumption \ref{ass_noant}. The third equality leverages Assumption \ref{ass_cont0}, which states that $\E \left [ Y_1 \left ( 1,1 \right ) | R_1 = r, D_2 \left ( 1 \right ) = 1 \right ]$, $\E \left [ Y_2 \left ( 1,0 \right ) | R_1 = r, D_2 \left ( 1 \right ) = 0 \right ]$, and $\E \left [ Y_1 \left ( 1,0 \right ) | R_1 = r, D_2 \left ( 1 \right ) = 0 \right ]$ are continuous functions of $r$ at $r=c$. The fourth and seventh equalities follow from the threshold-crossing definition of the treatment, $D_1 \equiv \mathbb{I} \left [ R_1 \geq c \right ]$. The fifth equality stems from the fact that, conditional on $D_1 = 1$, the second-period potential treatment $D_2 \left ( 1 \right )$ is observed. Analogously, the potential outcome $Y_1 \left ( 1,1 \right )$ is observed conditional on $D_1 = D_2 = 1$ and the potential outcomes $Y_2 \left ( 1,0 \right )$ and $Y_1 \left ( 1,0 \right )$ are observed conditional on $D_1 = 1$ and $D_2 = 0$, thus justifying the sixth equality. Combining these results, the first term in $\textsc{APTE}_{1,1} (c)$ can be expressed as
\begin{align*}
	\E \left [ Y_2 \left ( 1,0 \right ) | R_1 = c \right ] & = \lim_{r \downarrow c} \E \left [ Y_2 | R_1 = r, D_2 = 0 \right ] \times \lim_{r \downarrow c} \Prob \left ( D_{2}=0 | R_1 = r \right ) \\
	& + \left ( \lim_{r \downarrow c} \E \left [ Y_1 | R_1 = r, D_{2}=1 \right ] + \lim_{r \downarrow c} \E \left [ Y_{2} - Y_1 | R_1 = r, D_{2}=0 \right ] \right ) \\
	& \times \lim_{r \downarrow c} \Prob \left ( D_{2}=1 | R_1 = r \right )
\end{align*}
$\E \left [ Y_2 \left ( 0,0 \right ) | R_1 = c \right ]$ is point identified with a symmetric argument. Thus,
\begin{align*}
	\textsc{APTE}_{1,1} \left ( c \right ) & = \lim_{r \downarrow c} \E \left [ Y_2 | R_1 = r, D_2 = 0 \right ] \times \lim_{r \downarrow c} \Prob \left ( D_{2}=0 | R_1 = r \right ) \\
		& + \left ( \lim_{r \downarrow c} \E \left [ Y_1 | R_1 = r, D_{2}=1 \right ] + \lim_{r \downarrow c} \E \left [ Y_{2} - Y_1 | R_1 = r, D_{2}=0 \right ] \right ) \\
		& \times \lim_{r \downarrow c} \Prob \left ( D_{2}=1 | R_1 = r \right ) \\
		& - \lim_{r \uparrow c} \E \left [ Y_2 | R_1 = r, D_2 = 0 \right ] \times \lim_{r \uparrow c} \Prob \left ( D_{2}=0 | R_1 = r \right ) \\
		& - \left ( \lim_{r \uparrow c} \E \left [ Y_1 | R_1 = r, D_{2}=1 \right ] + \lim_{r \uparrow c} \E \left [ Y_{2} - Y_1 | R_1 = r, D_{2}=0 \right ] \right ) \\
		& \times \lim_{r \uparrow c} \Prob \left ( D_{2}=1 | R_1 = r \right ) \\
		& = \lim_{r \downarrow c} \E \left [ \left ( 1 - D_2 \right ) Y_2 | R_1 = r \right ] - \lim_{r \uparrow c} \E \left [ \left ( 1 - D_2 \right ) Y_2 | R_1 = r \right ] \\
		& + \lim_{r \downarrow c} \E \left [ D_2 Y_1 | R_1 = r \right ] - \lim_{r \uparrow c} \E \left [ D_2 Y_1 | R_1 = r \right ] \\
		& - \lim_{r \downarrow c} \E \left [ \left ( 1 - D_2 \right ) \left ( Y_2 - Y_1 \right ) | R_1 = r \right ] - \lim_{r \uparrow c} \E \left [ \left ( 1 - D_2 \right ) \left ( Y_2 - Y_1 \right ) | R_1 = r \right ] \\
		& + \lim_{r \downarrow c} \E \left [ Y_2 - Y_1 | R_1 = r, D_2 =0 \right ] - \lim_{r \uparrow c} \E \left [ Y_2 - Y_1 | R_1 = r, D_2 =0 \right ] \\
		& = \lim_{r \downarrow c} \E \left [ Y_1 | R_1 = r \right ] - \lim_{r \uparrow c} \E \left [ Y_1 | R_1 = r \right ] \\
		& + \lim_{r \downarrow c} \E \left [ Y_2 - Y_1 | R_1 = r, D_2 =0 \right ] - \lim_{r \uparrow c} \E \left [ Y_2 - Y_1 | R_1 = r, D_2 =0 \right ]
\end{align*}
where the second equality follows from an application of the Law of Iterated Expectations.

\subsection{Proof of Proposition \ref{prop_rd1}}

The sharp RDD estimand implied by the first-period discontinuity is defined as
\begin{align*}
	\mathrm{RD}_{t} \left ( c \right ) \equiv \lim_{r \downarrow c} \E \left [ Y_t | R_1 = r \right ] - \lim_{r \uparrow c} \E \left [ Y_t | R_1 = r \right ] 
\end{align*}
Consider the first term:
\begin{align*}
	\lim_{r \downarrow c} \E \left [ Y_t | R_1 = r \right ] & = \lim_{r \downarrow c} \E \left [ Y_t | R_1 = r, D_1 = 1 \right ] \\
	& = \lim_{r \downarrow c} \sum_{\left ( d_2, \dots, d_{\overline{t}} \right )} \E \left [ \mathbb{I} \left [ D_1 = 1 \right ] P \left ( 1, d_2, \dots, d_{\overline{t}} \right ) Y_t \left ( 1,d_2, \dots, d_{\overline{t}} \right )  | R_1 = r, D_1 = 1 \right ] \\
	& + \lim_{r \downarrow c} \sum_{\left ( d_2, \dots, d_{\overline{t}} \right )} \E \left [ \mathbb{I} \left [ D_1 = 0 \right ] P \left ( 0, d_2, \dots, d_{\overline{t}} \right ) Y_t \left ( 0,d_2, \dots, d_{\overline{t}} \right )  | R_1 = r, D_1 = 1 \right ]
\end{align*}
The first equality exploits the assumption that $D_1 \equiv \mathbb{I} \left [ R_1 \geq c \right ]$, which implies that the conditioning set can be expanded to include the event $D_1=1$. The second equality replaces the observed outcome $Y_t$ with its stochastic linear combination of potential outcomes, according to equation \eqref{potout_gen2}. Furthermore, because each of the four expectations is conditional on the event $D_1 = 1$, any indicator containing the event $D_1 = 0$ will be false. For the same reason, the event $D_1 = 1$ can be subsumed from any indicator that contains it. Thus,
\begin{align*}
	\lim_{r \downarrow c} \E \left [ Y_t | R_1 = r \right ] & = \lim_{r \downarrow c} \sum_{\left ( d_2, \dots, d_{\overline{t}} \right )} \E \left [ P \left ( 1, d_2, \dots, d_{\overline{t}} \right ) Y_t \left ( 1,d_2, \dots, d_{\overline{t}} \right )  | R_1 = r, D_1 = 1 \right ] \\
	& = \lim_{r \downarrow c} \sum_{\left ( d_2, \dots, d_{\overline{t}} \right )} \E \left [ P \left ( 1, d_2, \dots, d_{\overline{t}} \right ) Y_t \left ( 1,d_2, \dots, d_{\overline{t}} \right )  | R_1 = r \right ] \\
	& = \lim_{r \downarrow c} \sum_{\left ( d_2, \dots, d_{\overline{t}} \right )} \E \left [ Y_t \left ( 1,d_2, \dots, d_{\overline{t}} \right )  | R_1 = r, P \left ( 1, d_2, \dots, d_{\overline{t}} \right ) = 1 \right ] \\
	& \times \E \left [ P \left ( 1, d_2, \dots, d_{\overline{t}} \right ) | R_1 = r \right ] \\
	& = \sum_{\left ( d_2, \dots, d_{\overline{t}} \right )} \E \left [ Y_t \left ( 1,d_2, \dots, d_{\overline{t}} \right )  | R_1 = c, P \left ( 1, d_2, \dots, d_{\overline{t}} \right ) = 1 \right ] \\
	& \times \E \left [ P \left ( 1, d_2, \dots, d_{\overline{t}} \right ) | R_1 = c \right ]
\end{align*}
As above, the second equality exploits the assumption that $D_1 \equiv \mathbb{I} \left [ R_1 \geq c \right ]$. The third equality follows from an application of the Law of Iterated Expectations. The fourth equality leverages Assumption \ref{ass_cont1} and the algebraic properties of limits. By a symmetric argument, the right-hand side of $\mathrm{RD}_t \left ( c \right )$ identifies
\begin{align*}
	\lim_{r \uparrow c} \E \left [ Y_t | R_1 = r \right ] & = \sum_{\left ( d_2, \dots, d_{\overline{t}} \right )} \E \left [ Y_t \left ( 0,d_2, \dots, d_{\overline{t}} \right )  | R_1 = c, P \left ( 0, d_2, \dots, d_{\overline{t}} \right ) = 1 \right ] \\
	& \times \E \left [ P \left ( 0, d_2, \dots, d_{\overline{t}} \right ) | R_1 = c \right ]
\end{align*}
which completes the proof.


\subsection{Proof of Proposition \ref{prop_apte0_gen}}

The immediate Average Primary Treatment Effect (APTE) implied by the first-period discontinuity is defined as
\begin{align*}
	\textsc{APTE}_{1,0} \left ( c \right ) \equiv \E \left [ Y_1 \left ( 1,0_{2:\overline{t}} \right ) | R_1 = c \right ] - \E \left [ Y_1 \left ( 0,0_{2:\overline{t}} \right ) | R_1 = c \right ]
\end{align*}
By an application of the Law of Iterated Expectations, the first term can be expressed as
\begin{align}\label{apte0_decomp}
	\E \left [ Y_1 \left ( 1,0_{2:\overline{t}} \right ) | R_1 = c \right ] & = \sum_{\left ( d_2, \dots, d_{\overline{t}} \right ) \in \left \{ 0,1 \right \}^{\overline{t}-1}} \E \left [ Y_t \left ( 1,0_{2:\overline{t}} \right )  | R_1 = c, P \left ( 1, d_2, \dots, d_{\overline{t}} \right ) = 1 \right ] \\
	& \times \E \left [ P \left ( 1, d_2, \dots, d_{\overline{t}} \right ) | R_1 = c \right ]
\end{align}
Each conditional expectation of $P$ is identified under Assumption \ref{ass_cont1}. In fact, for any sequence of treatment states $\left ( d_2, \dots, d_{\overline{t}} \right )$,
\begin{align*}
	\E \left [ P \left ( 1, d_2, \dots, d_{\overline{t}} \right ) | R_1 = c \right ] & = \lim_{r \downarrow c} \E \left [ P \left ( 1, d_2, \dots, d_{\overline{t}} \right ) | R_1 = r \right ] \\
	& = \lim_{r \downarrow c} \E \left [ P \left ( 1, d_2, \dots, d_{\overline{t}} \right ) | R_1 = r, D_1 = 1 \right ] \\
	& = \lim_{r \downarrow c} \Prob \left ( D_{2} = d_2, \dots, D_{\overline{t}} = d_{\overline{t}} | R_1 = r, D_1 = 1 \right ) \\
	& = \lim_{r \downarrow c} \Prob \left ( D_{2} = d_2, \dots, D_{\overline{t}} = d_{\overline{t}} | R_1 = r \right )
\end{align*}
The first equality leverages Assumption \ref{ass_cont1}, which states that $\E \left [ P \left ( 1, d_2, \dots, d_{\overline{t}} \right ) | R_1 = r \right ]$ is a continuous function of $r$ at $r=c$. The second and fourth equalities follow from the threshold-crossing definition of the treatment, $D_1 \equiv \mathbb{I} \left [ R_1 \geq c \right ]$. The third equality stems from the fact that, conditional on $D_1 = 1$, the path of potential treatment states $P \left ( 1, d_2, \dots, d_{\overline{t}} \right )$ is observed. By a similar argument, one conditional expectation in equation \eqref{apte0_decomp} is identified without further restrictions. Specifically,
\begin{align*}
	\E \left [ Y_1 \left ( 1,0_{2:\overline{t}} \right ) | R_1 = c, P \left ( 1, 0_{2:\overline{t}} \right ) = 1 \right ] & = \lim_{r \downarrow c} \E \left [ Y_1 \left ( 1,0_{2:\overline{t}} \right ) | R_1 = r, P \left ( 1, 0_{2:\overline{t}} \right ) = 1 \right ] \\
	& = \lim_{r \downarrow c} \E \left [ Y_1 \left ( 1,0_{2:\overline{t}} \right ) | R_1 = r, D_1 = 1, P \left ( 1, 0_{2:\overline{t}} \right ) = 1 \right ] \\
	& = \lim_{r \downarrow c} \E \left [ Y_1 \left ( 1,0_{2:\overline{t}} \right ) | R_1 = r, D_1 = 1, D_2 = 0, \dots, D_{\overline{t}} = 0 \right ] \\
	& = \lim_{r \downarrow c} \E \left [ Y_1 | R_1 = r, D_1 = 1, D_2 = 0, \dots, D_{\overline{t}} = 0 \right ] \\
	& = \lim_{r \downarrow c} \E \left [ Y_1 | R_1 = r, D_2 = 0, \dots, D_{\overline{t}} = 0 \right ]
\end{align*}
The first equality leverages Assumption \ref{ass_cont1}, which states that $\E \big [ Y_1 \left ( 1,0_{2:\overline{t}} \right ) | R_1 = r, P \left ( 1, 0_{2:\overline{t}} \right ) = 1 \big ]$ is a continuous function of $r$ at $r=c$. The second and fifth equalities follow from the threshold-crossing definition of the treatment, $D_1 \equiv \mathbb{I} \left [ R_1 \geq c \right ]$. The third equality stems from the fact that, conditional on $D_1 = 1$, the path of potential treatment states $P \left ( 1, d_2, \dots, d_{\overline{t}} \right )$ is observed. Analogously, the potential outcome $Y_1 \left ( 1,0_{2:\overline{t}} \right )$ is observed conditional on $D_1 = 1$ and $D_2 = \dots = D_{\overline{t}} = 0$, thus justifying the fourth equality. Finally, under Assumption \ref{ass_cont1} and Assumption \ref{ass_noant1}, any counterfactual conditional mean in equation \eqref{apte0_decomp} is identified as
\begin{align*}
	& \E \left [ Y_1 \left ( 1,0_{2:\overline{t}} \right ) | R_1 = c, P \left ( 1, d_2, \dots, d_{\overline{t}} \right ) = 1 \right ] \\
	= \ & \E \left [ Y_1 \left ( 1,d_2, \dots, d_{\overline{t}} \right ) | R_1 = c, P \left ( 1, d_2, \dots, d_{\overline{t}} \right ) = 1 \right ]\\
	= \ & \lim_{r \downarrow c} \E \left [ Y_1 \left ( 1,d_2, \dots, d_{\overline{t}} \right ) | R_1 = r, P \left ( 1, d_2, \dots, d_{\overline{t}} \right ) = 1 \right ] \\
	= \ & \lim_{r \downarrow c} \E \left [ Y_1 \left ( 1,d_2, \dots, d_{\overline{t}} \right ) | R_1 = r, D_1 = 1, P \left ( 1, d_2, \dots, d_{\overline{t}} \right ) = 1 \right ] \\
	= \ & \lim_{r \downarrow c} \E \left [ Y_1 \left ( 1,d_2, \dots, d_{\overline{t}} \right ) | R_1 = r, D_1 = 1, D_2 = d_2, \dots, D_{\overline{t}} = d_{\overline{t}} \right ] \\
	= \ & \lim_{r \downarrow c} \E \left [ Y_1 | R_1 = r, D_1 = 1, D_2 = d_2, \dots, D_{\overline{t}} = d_{\overline{t}} \right ] \\
	= \ & \lim_{r \downarrow c} \E \left [ Y_1 | R_1 = r, D_2 = d_2, \dots, D_{\overline{t}} = d_{\overline{t}} \right ]
\end{align*}
The first equality applies the no anticipation restriction embedded in Assumption \ref{ass_noant1}. The second equality leverages Assumption \ref{ass_cont1}, which states that $\E \big [ Y_1 \left ( 1,,0_{2:\overline{t}} \right ) | R_1 = r, P ( 1, d_2, \dots, d_{\overline{t}} ) = 1 \big ]$ is a continuous function of $r$ at $r=c$. The third and sixth equalities follow from the threshold-crossing definition of the treatment, $D_1 \equiv \mathbb{I} \left [ R_1 \geq c \right ]$. The fourth equality stems from the fact that, conditional on $D_1 = 1$, the path of potential treatment states $P \left ( 1, d_2, \dots, d_{\overline{t}} \right )$ is observed. Analogously, the potential outcome $Y_1 \left ( 1,d_2, \dots, d_{\overline{t}} \right )$ is observed conditional on $D_1 = 1$ and $D_2 = d_2, \dots, D_{\overline{t}} = d_{\overline{t}}$, thus justifying the fifth equality. Combining these identification results, the first term in $\textsc{APTE}_{1,0} (c)$ can be expressed as
\begin{align*}
	\E \left [ Y_1 \left ( 1,0_{2:\overline{t}} \right ) | R_1 = c \right ] & = \sum_{\left ( d_2, \dots, d_{\overline{t}} \right ) \in \left \{ 0,1 \right \}^{\overline{t}-1}} \lim_{r \downarrow c} \E \left [ Y_1 | R_1 = r, D_2 = 0, \dots, D_{\overline{t}} = d_{\overline{t}} \right ] \\
	& \times \lim_{r \downarrow c} \Prob \left ( D_{2}=0, \dots, D_{\overline{t}} = d_{\overline{t}} | R_1 = r \right ) \\
	& = \lim_{r \downarrow c} \E \left [ Y_1 | R_1 = r \right ]
\end{align*}
where the second equality follows from another application of the Law of Iterated Expectations and the algebraic properties of limits. By a symmetric argument, $\lim_{r \uparrow c} \E \left [ Y_1 | R_1 = r \right ]$ identifies $\E \left [ Y_1 \left ( 0,0_{2:\overline{t}} \right ) | R_1 = c \right ]$. Thus,
\begin{align*}
	\textsc{APTE}_{1,0} \left ( c \right ) = \lim_{r \downarrow c} \E \left [ Y_1 | R_1 = r \right ] - \lim_{r \uparrow c} \E \left [ Y_1 | R_1 = r \right ]
\end{align*}


\subsection{Proof of Proposition \ref{prop_apte1_gen}}

For $\tau \in \left \{ 1, \dots, \overline{t} - 1 \right \}$, the cumulative Average Primary Treatment Effect (APTE) implied by the first-period discontinuity is defined as
\begin{align*}
	\textsc{APTE}_{1,\tau} \left ( c \right ) \equiv \E \left [ Y_{1+\tau} \left ( 1,0_{2:\overline{t}} \right ) | R_1 = c \right ] - \E \left [ Y_{1+\tau} \left ( 0,0_{2:\overline{t}} \right ) | R_1 = c \right ]
\end{align*}
To keep notation concise, let $Y_t$ denote $Y_{1+\tau}$ for $t \in \left \{ 2, \dots, \overline{t} \right \}$. By an application of the Law of Iterated Expectations, the first term can be expressed as
\begin{align}\label{apte1_decomp}
	\E \left [ Y_t \left ( 1,0_{2:\overline{t}} \right ) | R_1 = c \right ] & = \sum_{\left ( d_2, \dots, d_{\overline{t}} \right ) \in \left \{ 0,1 \right \}^{\overline{t}-1}} \E \left [ Y_t \left ( 1,0_{2:\overline{t}} \right )  | R_1 = c, P \left ( 1, d_2, \dots, d_{\overline{t}} \right ) = 1 \right ] \\
	& \times \E \left [ P \left ( 1, d_2, \dots, d_{\overline{t}} \right ) | R_1 = c \right ]
\end{align}
As shown in the previous section, each $\E \left [ P \left ( 1, d_2, \dots, d_{\overline{t}} \right ) | R_1 = c \right ]$ is identified under Assumption \ref{ass_cont1}. Furthermore, the expected potential outcome conditional on a never-untreated path indicator is identified as
\begin{align*}
	\E \left [ Y_t \left ( 1,0_{2:\overline{t}} \right )  | R_1 = c, P \left ( 1, 0_{2:\overline{t}} \right ) = 1 \right ] & = \lim_{r \downarrow c} \E \left [ Y_t \left ( 1,0_{2:\overline{t}} \right )  | R_1 = r, P \left ( 1, 0_{2:\overline{t}} \right ) = 1 \right ] \\
	& = \lim_{r \downarrow c} \E \left [ Y_t \left ( 1,0_{2:\overline{t}} \right ) | R_1 = r, D_1 = 1, P \left ( 1, 0_{2:\overline{t}} \right ) = 1 \right ] \\
	& = \lim_{r \downarrow c} \E \left [ Y_t \left ( 1,0_{2:\overline{t}} \right ) | R_1 = r, D_1 = 1, D_2 = 0, \dots, D_{\overline{t}} = 0 \right ] \\
	& = \lim_{r \downarrow c} \E \left [ Y_t | R_1 = r, D_1 = 1, D_2 = 0, \dots, D_{\overline{t}} = 0 \right ] \\
	& = \lim_{r \downarrow c} \E \left [ Y_t | R_1 = r, D_2 = 0, \dots, D_{\overline{t}} = 0 \right ]
\end{align*}
The first equality leverages Assumption \ref{ass_cont1}, which states that $\E \big [ Y_t \left ( 1,0_{2:\overline{t}} \right ) | R_1 = r, P \left ( 1, 0_{2:\overline{t}} \right ) = 1 \big ]$ is a continuous function of $r$ at $r=c$. The second and fifth equalities follow from the threshold-crossing definition of the treatment, $D_1 \equiv \mathbb{I} \left [ R_1 \geq c \right ]$. The third equality stems from the fact that, conditional on $D_1 = 1$, the path of potential treatment states $P \left ( 1, d_2, \dots, d_{\overline{t}} \right )$ is observed. Analogously, the potential outcome $Y_t \left ( 1,d_2, \dots, d_{\overline{t}} \right )$ is observed conditional on $D_1 = 1$ and $D_2 = d_2, \dots, D_{\overline{t}} = d_{\overline{t}}$, thus justifying the fourth equality. Finally, under Assumption \ref{ass_cont1}, Assumption \ref{ass_noant1}, and Assumption \ref{ass_ct1}, any counterfactual conditional mean in equation \eqref{apte1_decomp} is identified as
\begin{align*}
	& \E \left [ Y_{t} \left ( 1,0_{2:\overline{t}} \right ) | R_1 = c, P \left ( 1, d_2, \dots, d_{\overline{t}} \right ) = 1 \right ] \\
	= \ & \E \left [ Y_1 \left ( 1,0_{2:\overline{t}} \right ) | R_1 = c, P \left ( 1, d_2, \dots, d_{\overline{t}} \right ) = 1 \right ] \\
	+ \ & \E \left [ Y_{t} \left ( 1,0_{2:\overline{t}} \right ) - Y_1 \left ( 1,0_{2:\overline{t}} \right ) | R_1 = c, P \left ( 1, 0_{2:\overline{t}} \right ) = 1 \right ] \\
	= \ & \E \left [ Y_1 \left ( 1,d_2, \dots, d_{\overline{t}} \right ) | R_1 = c, P \left ( 1, d_2, \dots, d_{\overline{t}} \right ) = 1 \right ] \\
	+ \ & \E \left [ Y_{t} \left ( 1,0_{2:\overline{t}} \right ) - Y_1 \left ( 1,0_{2:\overline{t}} \right ) | R_1 = c, P \left ( 1, 0_{2:\overline{t}} \right ) = 1 \right ] \\
	= \ & \lim_{r \downarrow c} \E \left [ Y_1 \left ( 1,d_2, \dots, d_{\overline{t}} \right ) | R_1 = r, P \left ( 1, d_2, \dots, d_{\overline{t}} \right ) = 1 \right ] \\
	+ \ & \lim_{r \downarrow c} \E \left [ Y_{t} \left ( 1,0_{2:\overline{t}} \right ) - Y_1 \left ( 1,0_{2:\overline{t}} \right ) | R_1 = r, P \left ( 1, 0_{2:\overline{t}} \right ) = 1 \right ] \\
	= \ & \lim_{r \downarrow c} \E \left [ Y_1 \left ( 1,d_2, \dots, d_{\overline{t}} \right ) | R_1 = r, D_1 = 1, P \left ( 1, d_2, \dots, d_{\overline{t}} \right ) = 1 \right ] \\
	+ \ & \lim_{r \downarrow c} \E \left [ Y_{t} \left ( 1,0_{2:\overline{t}} \right ) - Y_1 \left ( 1,0_{2:\overline{t}} \right ) | R_1 = r, D_1 = 1, P \left ( 1, 0_{2:\overline{t}} \right ) = 1 \right ] \\
	= \ & \lim_{r \downarrow c} \E \left [ Y_1 \left ( 1,d_2, \dots, d_{\overline{t}} \right ) | R_1 = r, D_1 = 1, D_{2} = d_2, \dots, D_{\overline{t}} = d_{\overline{t}} \right ] \\
	+ \ & \lim_{r \downarrow c} \E \left [ Y_{t} \left ( 1,0_{2:\overline{t}} \right ) - Y_1 \left ( 1,0_{2:\overline{t}} \right ) | R_1 = r, D_1 = 1, D_{2} = 0, \dots, D_{\overline{t}} = 0 \right ] \\
	= \ & \lim_{r \downarrow c} \E \left [ Y_1 | R_1 = r, D_1 = 1, D_{2}=d_2, D_{\overline{t}} = d_{\overline{t}} \right ] \\
	+ \ & \lim_{r \downarrow c} \E \left [ Y_{t} - Y_1 | R_1 = r, D_1 = 1, D_{2}=0, \dots, D_{\overline{t}} = 0 \right ] \\
	= \ & \lim_{r \downarrow c} \E \left [ Y_1 | R_1 = r, D_{2}=d_2, \dots, D_{\overline{t}} = d_{\overline{t}} \right ] \\
	+ \ & \lim_{r \downarrow c} \E \left [ Y_{t} - Y_1 | R_1 = r, D_{2}=0, \dots, D_{\overline{t}} = 0 \right ]
\end{align*}
The first equality uses Assumption \ref{ass_ct1}, which states that $Y_{t} \left ( 1,0_{2:\overline{t}} \right ) - Y_1 \left ( 1,0_{2:\overline{t}} \right )$ is mean independent of $P \left ( 1,d_2, \dots, d_{\overline{t}} \right )$ at the first-period cutoff. The second equality applies the no anticipation restriction embedded in Assumption \ref{ass_noant1}. The third equality leverages Assumption \ref{ass_cont1}, which states that $\E \big [ Y_1 \left ( 1,d_2, \dots, d_{\overline{t}} \right ) | R_1 = c, P \left ( 1, d_2, \dots, d_{\overline{t}} \right ) = 1 \big ]$, $\E \big [ Y_{t} \left ( 1,0_{2:\overline{t}} \right ) | R_1 = c, P \left ( 1, 0_{2:\overline{t}} \right ) = 1 \big ]$, and $\E \big [ Y_{1} \left ( 1,0_{2:\overline{t}} \right ) | R_1 = c, P \left ( 1, 0_{2:\overline{t}} \right ) = 1 \big ]$ are continuous functions of $r$ at $r=c$. The fourth and seventh equalities follow from the threshold-crossing definition of the treatment, $D_1 \equiv \mathbb{I} \left [ R_1 \geq c \right ]$. The fifth equality stems from the fact that, conditional on $D_1 = 1$, the paths of potential treatment states $P \left ( 1, d_2, \dots, d_{\overline{t}} \right )$ and $P \left ( 1, 0_{2:\overline{t}} \right )$ are observed. Analogously, the potential outcome $Y_1 \left ( 1,d_2, \dots, d_{\overline{t}} \right )$ is observed conditional on $D_1 = d_1, \dots, D_{\overline{t}} = d_{\overline{t}}$ and the potential outcomes $Y_2 \left ( 1,0_{2:\overline{t}} \right )$ and $Y_1 \left ( 1,0_{2:\overline{t}} \right )$ are observed conditional on $D_1 = 0, D_2 = 0, \dots, D_{\overline{t}} = 0$, thus justifying the sixth equality. Combining these results, the first term in $\textsc{APTE}_{1,\tau} (c)$ can be expressed as
\begin{align*}
	\E \left [ Y_{t} \left ( 1,0_{2:\overline{t}} \right ) | R_1 = c \right ] & = \lim_{r \downarrow c} \E \left [ Y_t | R_1 = r, D_2 = 0, \dots, D_{\overline{t}} = 0 \right ] \\
	& \times \lim_{r \downarrow c} \Prob \left ( D_{2} = 0, \dots, D_{\overline{t}} = 0 | R_1 = r \right ) \\
	& + \sum_{\left ( d_2, \dots, d_{\overline{t}} \right ) \neq 0^{\overline{t}-1}} \bigg ( \lim_{r \downarrow c} \E \left [ Y_1 | R_1 = r, D_{2}=d_2, \dots, D_{\overline{t}} = d_{\overline{t}} \right ] \\
	& + \lim_{r \downarrow c} \E \left [ Y_{t} - Y_1 | R_1 = r, D_{2}=0, \dots, D_{\overline{t}} = 0 \right ] \bigg ) \\
	& \times \lim_{r \downarrow c} \Prob \left ( D_{2} = d_2, \dots, D_{\overline{t}} = d_{\overline{t}} | R_1 = r \right )
\end{align*}
$\E \left [ Y_t \left ( 0,0_{2:\overline{t}} \right ) | R_1 = c \right ]$ is point identified with a symmetric argument. Thus,
\begin{align*}
	\textsc{APTE}_{1,1} \left ( c \right ) & = \lim_{r \downarrow c} \E \left [ Y_t | R_1 = r, D_2 = 0, \dots, D_{\overline{t}} = 0 \right ] \times \lim_{r \downarrow c} \Prob \left ( D_{2} = 0, \dots, D_{\overline{t}} = 0 | R_1 = r \right ) \\
	& + \sum_{\left ( d_2, \dots, d_{\overline{t}} \right ) \neq 0_{\overline{t}-1}} \bigg ( \lim_{r \downarrow c} \E \left [ Y_1 | R_1 = r, D_{2}=d_2, \dots, D_{\overline{t}} = d_{\overline{t}} \right ] \\
	& + \lim_{r \downarrow c} \E \left [ Y_{t} - Y_1 | R_1 = r, D_{2}=0, \dots, D_{\overline{t}} = 0 \right ] \bigg ) \\
	& \times \lim_{r \downarrow c} \Prob \left ( D_{2} = d_2, \dots, D_{\overline{t}} = d_{\overline{t}} | R_1 = r \right ) \\
	& - \lim_{r \uparrow c} \E \left [ Y_t | R_1 = r, D_2 = 0, \dots, D_{\overline{t}} = 0 \right ] \times \lim_{r \uparrow c} \Prob \left ( D_{2} = 0, \dots, D_{\overline{t}} = 0 | R_1 = r \right ) \\
	& - \sum_{\left ( d_2, \dots, d_{\overline{t}} \right ) \neq 0_{\overline{t}-1}} \bigg ( \lim_{r \uparrow c} \E \left [ Y_1 | R_1 = r, D_{2}=d_2, \dots, D_{\overline{t}} = d_{\overline{t}} \right ] \\
	& + \lim_{r \uparrow c} \E \left [ Y_{t} - Y_1 | R_1 = r, D_{2}=0, \dots, D_{\overline{t}} = 0 \right ] \bigg ) \\
	& \times \lim_{r \uparrow c} \Prob \left ( D_{2} = d_2, \dots, D_{\overline{t}} = d_{\overline{t}} | R_1 = r \right )
\end{align*}
By an application of the Law of Iterated Expectations, the target parameter can be compactly expressed as
\begin{align*}
	\textsc{APTE}_{1,1} \left ( c \right ) & = \lim_{r \downarrow c} \E \left [ \prod_{s=2}^{\overline{t}} \mathbb{I} \left [ D_s = 0 \right ] Y_t \big | R_1 = r \right ] - \lim_{r \uparrow c} \E \left [ \prod_{s=2}^{\overline{t}} \mathbb{I} \left [ D_s = 0 \right ] Y_t \big | R_1 = r \right ] \\
	& + \sum_{\left ( d_2, \dots, d_{\overline{t}} \right ) \neq 0_{\overline{t}-1}} \lim_{r \downarrow c} \E \left [ \prod_{s=2}^{\overline{t}} \mathbb{I} \left [ D_s = d_s \right ] Y_1 | R_1 = r \right ] \\
	& - \lim_{r \uparrow c} \E \left [ \prod_{s=2}^{\overline{t}} \mathbb{I} \left [ D_s = d_s \right ] Y_1 | R_1 = r \right ] \\
	& - \lim_{r \downarrow c} \E \left [ \prod_{s=2}^{\overline{t}} \mathbb{I} \left [ D_s = 0 \right ] \left ( Y_t - Y_1 \right ) \big | R_1 = r \right ] \\
	& - \lim_{r \uparrow c} \E \left [ \prod_{s=2}^{\overline{t}} \mathbb{I} \left [ D_s = 0 \right ] \left ( Y_t - Y_1 \right ) \big | R_1 = r \right ] \\
	& + \lim_{r \downarrow c} \E \left [ Y_t - Y_1 | R_1 = r, D_2 = 0, \dots, D_{\overline{t}} = 0 \right ] \\
	& - \lim_{r \uparrow c} \E \left [ Y_t - Y_1 | R_1 = r, D_2 =0, \dots, D_{\overline{t}} = 0 \right ]
\end{align*}
To conclude,
\begin{align*}
	\textsc{APTE}_{1,1} \left ( c \right ) & = \lim_{r \downarrow c} \E \left [ Y_1 | R_1 = r \right ] - \lim_{r \uparrow c} \E \left [ Y_1 | R_1 = r \right ] \\
	& + \lim_{r \downarrow c} \E \left [ Y_t - Y_1 | R_1 = r, D_2 = 0, \dots, D_{\overline{t}} = 0 \right ] \\
	& - \lim_{r \uparrow c} \E \left [ Y_t - Y_1 | R_1 = r, D_2 = 0, \dots, D_{\overline{t}} = 0 \right ]
\end{align*}
which completes the proof.






\end{document}