% This must be in the first 5 lines to tell arXiv to use pdfLaTeX, which is strongly recommended.
\pdfoutput=1
% In particular, the hyperref package requires pdfLaTeX in order to break URLs across lines.

\documentclass[11pt]{article}

% Remove the "review" option to generate the final version.
% \usepackage[review]{EMNLP2023}
\usepackage[]{EMNLP2023}
% Standard package includes
\usepackage{times}
\usepackage{latexsym}


% For proper rendering and hyphenation of words containing Latin characters (including in bib files)
\usepackage[T1]{fontenc}
% For Vietnamese characters
% \usepackage[T5]{fontenc}
% See https://www.latex-project.org/help/documentation/encguide.pdf for other character sets

% This assumes your files are encoded as UTF8
\usepackage[utf8]{inputenc}

% This is not strictly necessary and may be commented out.
% However, it will improve the layout of the manuscript,
% and will typically save some space.
\usepackage{microtype}

% This is also not strictly necessary and may be commented out.
% However, it will improve the aesthetics of text in
% the typewriter font.
\usepackage{inconsolata}

% 添加的package
\usepackage{bm}
\usepackage{soul, color}
\usepackage{amssymb}
\usepackage{graphicx}
\usepackage{amsmath}
\usepackage{booktabs} 
\usepackage{algpseudocode}
\usepackage{multirow}
\usepackage{diagbox}
\usepackage{xcolor}
\definecolor{mygreen}{RGB}{209,255,200}
\definecolor{myred}{RGB}{255,205,196}
\usepackage[ruled,linesnumbered]{algorithm2e}
\newcommand{\yd}[1]{\textcolor{orange}{\small{\bf [ #1 -- yd ]}}}


\usepackage{xcolor}
\usepackage{makecell}
\usepackage[most]{tcolorbox}
\usepackage{url}
\usepackage{tabularx}

\definecolor{Lavender}{RGB}{230, 230, 250}
\definecolor{YellowOrange}{RGB}{255, 204, 0}

\colorlet{LightLavender}{Lavender!30!}
\colorlet{LightRed}{YellowOrange!20!}
\colorlet{LightOrange}{myred!20!}
\tcbset{on line, 
    boxsep=1.0pt, left=0pt, right=0pt,top=0pt,bottom=0pt,
    colframe=white, colback=LightRed,  
    highlight math style={enhanced}
}
\pagestyle{plain} %added for the page number
\newcommand{\yue}[1]{\textcolor{blue}{Yue: #1}} % Yue's comment

\newcommand{\yu}[1]{\textcolor{red}{Yu: #1}} %


% If the title and author information does not fit in the area allocated, uncomment the following
%
%\setlength\titlebox{<dim>}
%
% and set <dim> to something 5cm or larger.

% \title{Watermarking increases AI detection but hurts Conditional Text Generation, Semantic-aware mitigation}
\title{Watermarking Conditional Text Generation for AI Detection: \\
Unveiling Challenges and a Semantic-Aware Watermark Remedy}
% \title{Watermarking Conditional Text Generation for AI Detection: \\
% Unraveling Challenges and a Semantic-Aware Watermark Solution}
% Author information can be set in various styles:
% For several authors from the same institution:
% \author{Author 1 \and ... \and Author n \\
%         Address line \\ ... \\ Address line}
% if the names do not fit well on one line use
%         Author 1 \\ {\bf Author 2} \\ ... \\ {\bf Author n} \\
% For authors from different institutions:
% \author{Author 1 \\ Address line \\  ... \\ Address line
%         \And  ... \And
%         Author n \\ Address line \\ ... \\ Address line}
% To start a separate ``row'' of authors use \AND, as in
% \author{Author 1 \\ Address line \\  ... \\ Address line
%         \AND
%         Author 2 \\ Address line \\ ... \\ Address line \And
%         Author 3 \\ Address line \\ ... \\ Address line}

\author{
    Yu Fu \\
  University of California, Riverside \\
  \texttt{yfu093@ucr.edu} \\ \And
  Deyi Xiong \\
  Tianjin University \\
  \texttt{dyxiong@tju.edu.cn}\\\And
  Yue Dong \Thanks{~Corresponding author.}\\ 
  University of California, Riverside \\
  \texttt{yue.dong@ucr.edu} \\}

\begin{document}
\maketitle
\begin{abstract}

To mitigate potential risks associated with language models, recent AI detection research proposes incorporating watermarks into machine-generated text through random vocabulary restrictions and utilizing this information for detection. While these watermarks only induce a slight deterioration in perplexity, our empirical investigation reveals a significant detriment to the performance of conditional text generation. To address this issue, we introduce a simple yet effective semantic-aware watermarking algorithm that considers the characteristics of conditional text generation and the input context. Experimental results demonstrate that our proposed method yields substantial improvements across various text generation models, including BART and Flan-T5, in tasks such as summarization and data-to-text generation while maintaining detection ability.


\end{abstract}

% flan t5
% 30\% performance drop on web nlg \\
% (53-44)/40 our method improve 23\% on BLEU\\
% AUG detection drop (0.75-0.9)/0.9= 16.7\%

\section{Introduction}
% Figure environment removed

Language Models (LMs) have demonstrated remarkable effectiveness in generating content that closely resembles human performances across diverse tasks \cite{tan2023evaluation, dong2023self, liu2023your}. As large-scale models such as ChatGPT \cite{openai2021chatgpt} evolve and produce increasingly human-like content, concerns have surged around potential limitations 
 and risks tied to their use \cite{bender2021dangers}. These include hallucination \citep{alkaissi2023artificial}, failure in commonsense reasoning \cite{bian2023chatgpt}, and  misinformation and malicious use \citep{openai2023gpt4}. 


To mitigate potential risks associated with LMs, it's crucial to develop methods that differentiate between AI and human-generated content.  Current AI-detection tools primarily  rely on perplexity-based classifiers, assuming lower perplexity in AI-generated text \cite{solaiman2019release,jawahar-etal-2020-automatic,mitchell2023detectgpt,mitrovic2023chatgpt}. Conversely, an alternative approach is to inject watermarks during generation for subsequent detection. For instance, \citet{kirchenbauer2023watermark} proposed using hash functions to randomly bifurcate the vocabulary into `green' and `red' lists at each decoding step, serving as watermarks. This watermark provides reliable detection signals without the need to train a classifier, and produce high-quality generated texts with a minor perplexity drop in language modeling \citep{bengio2000neural}.

Different from existing research, our focus is on watermarks for conditional text generation (CTG), and we unveil the challenges associated with the use of watermarks \citep{kirchenbauer2023watermark}. Our research findings suggest that \textbf{watermarking algorithms cannot be seamlessly applied to CTG tasks without a notable decline in performance}: the omission of task-specific considerations leads to significant decreases observed -- up to 96.99\% drop with hard watermarks and 27.54\% drop with soft watermarks  -- in conditional generation tasks including summarization \citep{see-etal-2017-get, narayan-etal-2018-dont} and data-to-text generation \citep{nan-etal-2021-dart,gardent-etal-2017-webnlg}. Additionally, our detection results reveal a paradox, which indicates another challenge in applying watermarks for CTG: \textbf{the prevalent human habit of using tokens similar to the input for text generation complicates the detection of watermarks}.
%Figure \ref{fig:enter-label} illustrates an example where the randomly bifurcated red list contains key entities from the source and compels models to generate from the red list, which not only impairs detection but also introduces 12 hallucinated words in a 30-token generation.


To enhance the effectiveness of watermarks for CTG, we propose a simple yet effective semantic-aware watermarking algorithm that leverages hash functions to embed watermarks, while also taking into account the input context and the distinctive characteristics of conditional generation tasks. In particular, we strategically bifurcate the vocabulary to balance randomness and semantic relatedness to the input source using word vector similarity, based on hash functions for detection. These semantically-related tokens can efficiently cover a substantial portion of the information that needs to be generated in conditional text generation tasks. Consequently, their inclusion in the `green list' acts as a buffer, reducing the adverse impact of adding watermarks on the generated results, while maintaining the detection ability.


 % To overcome this limitation, we present a novel semantic-aware watermarking algorithm tailored for CTG. This algorithm aims to mitigate the performance drop while minimizing the trade-off in detection sensitivity, which arises due to the unique human-generation behaviors in CTG. The observations and findings from this study underscore the need to address these specific challenges to develop effective watermarks for CTG.

Our contributions can be summarized as follows:

\begin{itemize}
    \item We show that directly applying \citet{kirchenbauer2023watermark}'s watermark method to conditional text generation tasks, without task-specific considerations, can lead to a significant performance drop (up to 96.99\%). This significant decline is observed across multiple tasks like summarization and data-to-text generation, and various text generation models such as BART and Flan-T5.
    \item We propose a semantic-aware watermarking algorithm that utilizes hash functions while considering the input context  of CTG tasks. Automatic and human evaluations on multiple datasets and models indicate that our method effectively mitigates quality degradation associated with the use of watermarks, while minimizing the trade-off in detection. 
\end{itemize}


\section{Related Work}
\paragraph{Automatic Detection}
The detection of AI-generated text, particularly in the context of large language models (LLMs), has recently attracted significant research interest  \cite{bakhtin2019real, schuster-etal-2020-limitations, frohling2021feature, sadasivan2023can, mitchell2023detectgpt}. Previous approaches have primarily focused on leveraging the perplexities of generated texts for detection. For example, \citet{solaiman2019release} utilize a classifier to evaluate the total log probability of the text, using it as a means to determine whether the content originated from a machine. Building on this premise, \citet{mitchell2023detectgpt} further hypothesize and validate that the log probability of machine-generated text diminishes upon perturbation, while the log probability of human-written text remains unpredictable when perturbed.

Besides the aforementioned detection classifiers, there has been a recent emergence of approaches that involve watermarking specific patterns into generated text. For instance, \citet{kirchenbauer2023watermark} proposed a method that randomly bifurcates the vocabulary and modifies the probability distribution during each decoding step, thereby ensuring the inclusion of detectable patterns (watermarks) in the generated text. On the other hand, \citet{yang2023watermarking} focused on revising and recognizing generated text without having access to the decoding process, enabling watermarking even in the case of black-box LLMs. Their approach identifies words in the generated text that deviate from a predefined pattern and then watermark them with synonymous words that adhere to the pattern.
% By leveraging deep learning techniques and utilizing appropriate training data, such a classifier can be trained to effectively differentiate between the two types of text. However, models trained using deep learning methods often lack generalization ability and perform poorly when faced with out-of-distribution data\cite{jawahar-etal-2020-automatic, uchendu2020authorship}. Additionally, employing deep learning unavoidably incurs time and parameter costs due to the need for additional training. 

\paragraph{Conditional Text Generation}

Conditional text generation aims to produce texts based on given inputs while considering task-specific  characteristics and requirements. The core objective is to generate text that is conditioned on specific information or conditions to fulfill the intended purpose of the task. Classical conditional text generation tasks include machine translation \citep{stahlberg2020neural}, dialogue \citep{huang2020challenges}, summarization \cite{xu2022sequence, goyal2022news}, question answering \cite{karpukhin-etal-2020-dense, lazaridou2022internet}, data-to-text generation \cite{goyal2022news, keymanesh-etal-2022-makes}, and code generation \cite{vaithilingam2022expectation, zhang2023self}.
% Each task within conditional text generation has its own specific requirements and objectives, but they all involve generating text based on given conditions or input.


\section{Method}

This section provides an overview of the basic principles of watermarks, elaborates on our proposed semantic-aware method, and discusses how it's integrated into the watermarking procedure for CTG.

\paragraph{Original Watermark}  Considering a language model with parameters denoted by $\theta$, the probability distribution for the $t$-th token in  sequence $\mathbf{S}=\{s_1, s_2, \dots, s_{|\mathbf{S}|}\}$ can be formulated as :

\begin{equation}
    p(s_{t}) = p_{\theta}(s_t | s_{<t})
    \label{lms}
\end{equation}
By considering all preceding tokens, language models (LMs) generate a probability distribution across the vocabulary and sample tokens accordingly.

Watermarking is a technique designed to incorporate robust detection signals into machine-generated text. \citet{kirchenbauer2023watermark} propose two methods, namely hard and soft watermarks, for adding watermarks to text by imposing vocabulary restrictions during each decoding step. Specifically, the "Hard Red List" watermark algorithm randomly divides the vocabulary into "green" and "red" lists using a hash function and previously generated tokens. During the generation process, only tokens from the green list can be selected for the $t$-th position. To detect the presence of the watermark in the generated text, a statistical analysis such as the \emph{one proportion z-test} can be employed.


However, randomly partitioning the vocabulary and solely selecting words from the green list can hinder the generation of crucial tokens that are not included in the green list. As an alternative, the "Soft Red List" watermark approach introduces a constant $\delta$ to the logit $l^{(t)}_k$ of tokens in the green list during prediction:
\begin{equation}
    p^{(t)}_k = \exp (l^{(t)}_k + \delta) / \sum_{i} \exp (l^{(t)}_i)
\end{equation}
This adjustment ensures that even if there are deterministic tokens not included in the green list, they can still be generated. We observe that hard watermarking can be seen as a special case of soft watermarking, achieved by adding a large $\delta$ to the tokens in the green list. Therefore, we choose soft watermarking algorithm as the unified formulation in our paper.

% \paragraph{Conditional Text Generation}


\subsection{semantic-aware Watermark}
In contrast to text generation tasks involving language models, conditional text generation (CTG) tasks often exhibit significant textual overlap, either at the token level or the semantic level.  For instance, \citet{chen-etal-2020-cdevalsumm} demonstrate that in the CNN/DailyMail dataset \citep{see-etal-2017-get}, over 80\% of the tokens found in the summary can be located within the original document. Even in the case of the XSUM dataset \citep{narayan-etal-2018-dont}, known for its "abstractive" nature, this percentage remains above 60\%. Consequently, random watermarking algorithms, which bifurcate the vocabulary arbitrarily at each decoding step, can drastically impair the performance of generation models.
% For LMs, there can be many possible candidates so the impact on LMs performance is not obvious, but simply using these watermark methods to conduct CTG will hurt the performance as many of the target tokens cannot be replaced by other tokens, especially under the estimation of n-gram metrics.
% Seq2seq models are commonly used as the backbone model, taking into account the characteristics of CTG tasks. 
% For example, in text summarization, the input is an original document represented as $\mathbf{x}={ x_1, x_2, \dots, x_{|\mathbf{x}|}}$, and the output is a summary represented as $\mathbf{y}={y_1, y_2, \dots, y_{|\mathbf{y}|}}$. The core idea of summarization is to convert the meaning of the original document into a short summary, thus tokens in the generated summary are mostly from the original document.

Considering this characteristic of CTG tasks, we propose a simple yet effective semantic-aware watermarking method to enhance performance. Our approach uses the input context to extract semantically related tokens, measured by word vector similarity to the source.  By incorporating semantically related tokens as a constraint, we ensure the quality of the generated output. We then apply the original watermark and randomly bifurcate the remaining vocabulary.

\SetKwInOut{KwOut}{Output}
\SetKwInOut{KwIn}{Input}
\begin{algorithm}[t]
\begin{small}
\caption{semantic-aware Watermark}
\KwIn{Input sequence $\mathbf{x} = \{x_1, x_2, \dots, x_{|\mathbf{x}|} \}$, Conditional model $p_{\theta}$, green list size: $\gamma \in (0,1)$, hardness parameter: $\delta > 0$ cluster parameter: $k \in [1, 2, 5, 10]$}
\KwOut{Watermarked text}
\emph{Get word vector from model embedding and Compute word similarity matrix $\mathbf{M} \in \big[|V|, |V| \big]$} \;
 \emph{Using input sequence $\mathbf{x}$ and parameter $k$ to get semantically related tokens $S$ and Insert them to "green list" $G$} \;
\For{$t \leftarrow 1 \KwTo \dots$}{
 Apply the conditional model to input sequence $\mathbf{x}$ and get a logit vector $l^{(t)}$ over the vocabulary $V$ \;

 Compute a hash of token $y_{t-1}$ and use it to seed a random number generator \;
 
 \emph{Using the random number generator and partition the remaining vocabulary into $G$ of size $\gamma V - len(S)$ and a "red list" $R$ of size $(1-\gamma) |V|$} \;

 Add $\delta$ to each green list logit. Apply these modified logits to get a probability distribution over V \;

 $$
 \hspace{-3mm}\begin{small}
    \hat p^{(t)}_k =
  \begin{cases}  \frac{ \exp(l^{(t)}_k+\delta)}{\sum_{i\in R} \exp(l^{(t)}_i)+\sum_{i\in G} \exp(l^{(t)}_i+\delta)}, \quad k\in G\\
  \frac{ \exp(l^{(t)}_k)}{\sum_{i\in R} \exp(l^{(t)}_i)+\sum_{i\in G} \exp(l^{(t)}_i+\delta)}, \quad k\in R
  \end{cases}
 \end{small}
$$\\
Sample the next token $y_{t}$ according to watermark distribution $\hat{p}^{(t)}$.
}


\end{small}
\label{main-algorithm}
\end{algorithm}

To implement this approach, we tokenize the input sequence $\mathbf{x}$ to $\hat{\mathbf{x}} = \{\hat{x}_1, \hat{x}_2, \dots, \hat{x}_{|\mathbf{\hat{x}}|} \}$. Next, the tokenized sequence $\hat{\mathbf{x}}$ is transformed into contextualized vector representations using the model's embedding layer. Integrating input information into the watermark's green list is a direct and crucial step (step 2\&6 in Algorithm 1), consistent with the requirements of CTG tasks where the output is dependent on the input. However, it's crucial to note that output information isn't solely determined by the input. Thus, relying exclusively on input as a constraint may not yield optimal results. To overcome this limitation, we broaden the constraints by incorporating token embeddings to measure token similarities.

We extend the constraints to prioritize the inclusion of content closely related to the input within the partitioned green list, as detailed in Algorithm 1. This strategy effectively minimizes the impact of random vocabulary partitioning on the quality of generated results. The decision to utilize model embeddings to acquire semantically related tokens -- steps 1\&2 in Algorithm 1 --  is motivated by the following reasons:

\begin{itemize}
    \item Semantic Relevance: By exploiting model embeddings, we capture semantic token relationships. This ensures coherent and semantically consistent text generation by identifying tokens closely linked to the input.
    \item Enhanced Output Quality: Including semantically related tokens in the green list elevates the relevance and quality of the generated text, aligning it more effectively with the CTG task objectives.
    % \item Performance Optimization: The use of model embeddings provides an efficient approach to select tokens that enhance the output generation, thereby optimizing overall task performance.
\end{itemize}

% 1). The basic objects manipulated by the watermark are the tokens in the vocabulary. Therefore, it is necessary to maintain consistency with them during the enhancement process. 2). Model embedding are directly aligned with the size of the vocabulary, making it a suitable representation for corresponding tokens to calculate and get similarity between different tokens. 
% % 3). The watermark performs partitioning at each decoding step. Obtaining semantically related tokens dynamically before each partitioning would incur significant time costs. However, by using embedding layer to measure the similarity between tokens, it can be performed as a preprocessing step. 




For a specific model, the embedding size is represented as $\big[|V|, d_{\rm{emb}} \big]$, where $V$ denotes the vocabulary size, and $d_{\rm{emb}}$ represents the dimension of the model's embedding. Each row of the embedding matrix serves as the representation for the corresponding token. With the token representations in hand, we can calculate vector similarity using methods like cosine similarity to assess the similarity between different tokens. By sorting the tokens based on their similarity values, we construct a similarity matrix $\mathbf{M}$ of size $\big[ |V| \times |V| \big]$.

In the similarity matrix $\mathbf{M}$, each row contains the IDs of all tokens in the vocabulary, sorted according to their similarity to the token represented by that row. Since each row includes the token itself as the first entry, the matrix exhibits diagonal symmetry, forming a symmetric matrix along the diagonal.

In the semantic watermarking method, prior to partitioning the green list, we leverage the input as a foundation and utilize the similarity matrix $\mathbf{M}$. By combining this similarity matrix with a hyperparameter $k$, we identify semantically related tokens. These semantically related tokens are then included in the green list, while the remaining portion of the vocabulary is randomly partitioned after the incorporation of semantically related tokens into the green list.

\begin{table*}[ht]
\centering
% \resizebox{\textwidth}{!}{
% \begin{tabular}{lllccc|llll}
% \toprule
% Dataset  & Model   & Method   & R-1 & R-2 & R-L & Dataset    & Model  & Method  & BLEU  \\ \midrule
% \multirow{12}{*}{\textbf{CNN}}  & \multirow{3}{*}{BART-base}  & NW     & 42.02 & 19.46 & 39.04 & \multirow{12}{*}{\textbf{DART}}   & \multirow{3}{*}{BART-base} & NW  & 45.90 \\
%                         &                           & OW & 38.13  & 16.56 & 35.33 &   &                             & OW & 35.99 \tcbhighmath[colback=LightOrange]{\downarrow 21.6\%} \\
%                         &                           & SW (ours) & \textbf{41.65} &\textbf{19.30} & \textbf{38.54} &   &   & SW (ours) & \textbf{42.89} \tcbhighmath[colback=LightOrange]{\downarrow 6.6\%}  \\ \cline{2-6} \cline{8-10} 
%                        & \multirow{3}{*}{BART-large} & NW       & 43.80 & 20.88 & 40.73 &     & \multirow{3}{*}{BART-large}    & NW       & 47.78 \\
%                        &                             & OW & 42.46 & 18.33 & 39.52 &     &                                & OW & 37.06 \tcbhighmath[colback=LightOrange]{\downarrow 22.4\%}\\
%                        &                             & SW (ours) & \textbf{43.50} & \textbf{20.83} & \textbf{40.62} &     &     & SW (ours) & \textbf{44.04} \tcbhighmath[colback=LightOrange]{\downarrow 7.8\%} \\ \cline{2-6} \cline{8-10} 
%                        & \multirow{3}{*}{Flan-T5-small} & NW    & 38.96 & 17.35 & 35.84 &     & \multirow{3}{*}{Flan-T5-small} & NW       & 47.99 \\
%                        &                                & OW & 35.51 & 14.56 & 32.90 &  &                                & OW & 36.32 \tcbhighmath[colback=LightOrange]{\downarrow 24.3\%} \\
%                        &                                & SW (ours) & \textbf{39.76} & \textbf{17.88} & \textbf{36.58} &  &     & SW (ours) & \textbf{42.61} \tcbhighmath[colback=LightOrange]{\downarrow 11.2\%} \\ \cline{2-6} \cline{8-10} 
%                        & \multirow{3}{*}{Flan-T5-base}  & NW       & 41.78 & 19.57 & 38.66 &  & \multirow{3}{*}{Flan-T5-base}  & NW       & 49.55 \\
%                        &                                & OW & 38.60 & 16.29 & 35.90 &  &                                & OW & 39.19 \tcbhighmath[colback=LightOrange]{\downarrow 20.9\%}\\
%                        &                                & SW (ours) & \textbf{41.90} & \textbf{19.86} & \textbf{38.80} &  &     & SW (ours) & \textbf{44.18} \tcbhighmath[colback=LightOrange]{\downarrow 10.8\%}\\ \midrule
% \multirow{12}{*}{\textbf{XSUM}} & \multirow{3}{*}{BART-base}     & NW       & 42.36 & 19.42 & 34.40 & \multirow{12}{*}{\textbf{WebNLG}} & \multirow{3}{*}{BART-base} & NW  & 54.45 \\
%                        &                                & OW & 37.99 & 14.60 & 29.83 &  &                                & OW & 43.14 \tcbhighmath[colback=LightOrange]{\downarrow 20.8\%} \\
%                        &                                & SW (ours) & \textbf{41.36} & \textbf{17.97} & \textbf{33.04} &  &     & SW (ours) & \textbf{50.45} \tcbhighmath[colback=LightOrange]{\downarrow 7.0\%} \\ \cline{2-6} \cline{8-10} 
%                        & \multirow{3}{*}{BART-large}    & NW       & 45.25 & 22.15 & 37.03 &  & \multirow{3}{*}{BART-large}    & NW       & 57.18 \\
%                        &                                & OW & 40.07 & 16.51 & 31.50 &  &                                & OW & 44.58 \tcbhighmath[colback=LightOrange]{\downarrow 22.1\%}\\
%                        &                                & SW (ours) & \textbf{43.83} & \textbf{20.39} & \textbf{35.42} &  &     & SW (ours) & \textbf{52.50} \tcbhighmath[colback=LightOrange]{\downarrow 8.2\%} \\ \cline{2-6} \cline{8-10} 
%                        & \multirow{3}{*}{Flan-T5-small} & NW      & 33.57 & 12.00 & 26.50 &  & \multirow{3}{*}{Flan-T5-small} & NW       & 56.41 \\
%                        &                                & OW & 30.51 & 9.13  & 23.53 &  &                                & OW & 40.88 \tcbhighmath[colback=LightOrange]{\downarrow 27.5\%} \\
%                        &                                & SW (ours) & \textbf{33.15} & \textbf{11.40} & \textbf{26.01} &  &     & SW (ours) & \textbf{49.74} \tcbhighmath[colback=LightOrange]{\downarrow 11.8\%}\\ \cline{2-6} \cline{8-10} 
%                        & \multirow{3}{*}{Flan-T5-base}  & NW      & 39.51 & 16.92 & 31.90 &  & \multirow{3}{*}{Flan-T5-base}  & NW       & 59.77 \\
%                        &                                & OW & 35.23 & 12.58 & 27.52 &  &                                & OW & 45.42 \tcbhighmath[colback=LightOrange]{\downarrow 24.0\%}\\
%                        &                                & SW (ours) & \textbf{38.79}  & \textbf{15.91} & \textbf{31.03} &  &    & SW (ours) & \textbf{52.93} \tcbhighmath[colback=LightOrange]{\downarrow 11.4\%} \\ \bottomrule
% \end{tabular}}
\resizebox{\textwidth}{!}{
\begin{tabular}{lllccc|llll}
\toprule
Dataset  & Model   & Method   & R-1 & R-2 & R-L & Dataset    & Model  & Method  & BLEU  \\ \midrule
\multirow{12}{*}{\textbf{CNN}}  & \multirow{6}{*}{BART-large}  & NW & 43.80 & 20.88 & 40.73  & \multirow{12}{*}{\textbf{DART}} & \multirow{6}{*}{BART-large} & NW  & 47.78 \\
&  & OW (Hard) & 33.38  & 8.73 & 30.61 &  &  & OW (Hard) & 6.65 \tcbhighmath[colback=LightOrange]{\downarrow 86.1\%} \\
&  & SW (Hard) & \textbf{43.46} &\textbf{20.75} & \textbf{40.45} &   &   & SW (Hard) & \textbf{41.04} \tcbhighmath[colback=LightOrange]{\downarrow 14.1\%}  \\
&  & OW (Soft) & 42.46 & 18.33 & 39.52 &   &  & OW (Soft)& 37.06 \tcbhighmath[colback=LightOrange]{\downarrow 22.4\%}\\
&  & SW (Soft) & \textbf{43.50} & \textbf{20.83} & \textbf{40.62} &     &     & SW (Soft) & \textbf{44.04} \tcbhighmath[colback=LightOrange]{\downarrow 7.8\%} \\ \cline{2-6} \cline{8-10} 
& \multirow{6}{*}{Flan-T5-base} & NW   & 41.78 & 19.57 & 38.66 &  & \multirow{6}{*}{Flan-T5-base} & NW       & 49.55 \\
&  & OW (Hard) & 24.47 & 5.60 & 22.48  &  &  & OW (Hard) & 5.35 \tcbhighmath[colback=LightOrange]{\downarrow 89.2\%} \\
&  & SW (Hard) & \textbf{41.80} & \textbf{19.80} & \textbf{38.72} &  &  & SW (Hard) & \textbf{35.36} \tcbhighmath[colback=LightOrange]{\downarrow 28.6\%} \\
&  & OW (Soft) & 38.60 & 16.29 & 35.90 &  &   & OW (Soft)& 39.19 \tcbhighmath[colback=LightOrange]{\downarrow 20.9\%}\\
&  & SW (Soft) & \textbf{41.90} & \textbf{19.86} & \textbf{38.80} &  &   & SW (Soft) & \textbf{44.18} \tcbhighmath[colback=LightOrange]{\downarrow 10.8\%}\\ \midrule
\multirow{12}{*}{\textbf{XSUM}} & \multirow{6}{*}{BART-large}     & NW  & 45.25 & 22.15 & 37.03 & \multirow{12}{*}{\textbf{WebNLG}} & \multirow{6}{*}{BART-large} & NW  & 57.18  \\
&   & OW (Hard)& 29.60 & 7.15 & 20.83 &  &  & OW (Hard) & 9.25 \tcbhighmath[colback=LightOrange]{\downarrow 83.8\%} \\
&   & SW (Hard) & \textbf{42.44} & \textbf{18.64} & \textbf{33.91} &  &    & SW (Hard) & \textbf{48.02} \tcbhighmath[colback=LightOrange]{\downarrow 16.0\%} \\
&   & OW (Soft)& 40.07 & 16.51 & 31.50 &  &  & OW (Soft) & 44.58 \tcbhighmath[colback=LightOrange]{\downarrow 22.1\%}\\
&   & SW (Soft) & \textbf{43.83} & \textbf{20.39} & \textbf{35.42} &  &     & SW (Soft) & \textbf{52.50} \tcbhighmath[colback=LightOrange]{\downarrow 8.2\%} \\ \cline{2-6} \cline{8-10} 
& \multirow{6}{*}{Flan-T5-base} & NW      & 39.51 & 16.92 & 31.90 &  & \multirow{6}{*}{Flan-T5-base} & NW       & 59.77 \\
& & OW (Hard) & 22.98 & 4.80  & 16.66 &  &     & OW (Hard) & 1.80 \tcbhighmath[colback=LightOrange]{\downarrow 97.0\%} \\
& & SW (Hard) & \textbf{37.67} & \textbf{14.69} & \textbf{29.94} &  &     & SW (Hard) & \textbf{40.89} \tcbhighmath[colback=LightOrange]{\downarrow 31.6\%}\\
& & OW (Soft) & 35.23 & 12.58 & 27.52 &  &    & OW (Soft)& 45.42 \tcbhighmath[colback=LightOrange]{\downarrow 24.0\%}\\
& & SW (Soft) & \textbf{38.79}  & \textbf{15.91} & \textbf{31.03} &  &    & SW (Soft) & \textbf{53.27} \tcbhighmath[colback=LightOrange]{\downarrow 10.9\%} \\ \bottomrule
\end{tabular}}
\caption{Main results of comparing different watermarking strategies across various datasets and models. NW (no watermark) serves as the baseline, and adding a watermark is expected to decrease performance to trade-off detection. OW (original watermark) denotes the use of the Soft or Hard watermark \citep{kirchenbauer2023watermark} with hyperparameters $\gamma=0.5$ and $\delta \in \{2,10\}$. Our proposed SW (semantic-aware watermark) approach employs semantically related tokens to partition the green and red lists, with hyperparameters $k=1/2/5/10$, while keeping the same values of $\gamma$ and $\delta$ to ensure a fair comparison.}

\label{main_result}
\end{table*}

\section{Experiments and Results}

This section provides  an overview of the datasets and models utilized in the experiments. We also present the main experimental results, including both automatic and human evaluations.

\subsection{Datasets and Models}

We conducted experiments to assess the generalization ability of our proposed method by utilizing models with different parameter sizes and architectures, including BART-base, BART-large \citep{lewis-etal-2020-bart}, Flan-T5-small, and Flan-T5-base \citep{chung2022scaling}. Our focus was on two distinct conditional text generation tasks: summarization - CNN/DailyMail \citep{see-etal-2017-get} and XSUM \citep{narayan-etal-2018-dont},  and data-to-text generation - DART \cite{nan-etal-2021-dart} and WebNLG \citep{gardent-etal-2017-webnlg}. These datasets are widely recognized for evaluating text summarization and data-to-text generation models, respectively. By conducting comprehensive evaluations across multiple datasets, tasks, and models, our objective was to thoroughly compare the differences between the original watermarking algorithm \citep{kirchenbauer2023watermark} and our proposed semantic-aware watermarking approach. 



\subsection{Main Results}
Our main experimental results are presented in Table \ref{main_result}. The summarization task was evaluated using the ROUGE metric \citep{lin-2004-rouge}, while the data-to-text generation task was evaluated using BLEU \citep{papineni-etal-2002-bleu}. The table illustrates the performance of the models under various watermarking methods, highlighting the enhancements achieved by incorporating semantic constraints in watermarking for both the summarization and data-to-text generation tasks. Our proposed semantic-aware watermark method exhibits significant improvements in comparison to the original watermark method across all datasets and models. 

Additionally, we observe that hard watermarks invariably cause a greater decline in CTG performance compared to soft watermarks. The hard watermarks designed for language models \cite{kirchenbauer2023watermark} essentially completely forbid generation from the red list that might contain key input context, potentially leading to near-ineffective generations with almost no overlap with the reference generations.  For example, in the data-to-text generation task, the original hard watermark method adversely affects Flan-T5-small's performance on WebNLG, resulting in a decrease of over 57.97 BLEU points with 97.0\% of performance drop. In contrast, our semantic-aware watermark effectively mitigates the impact of adding the watermark, demonstrating an 39.09 BLEU point increase over the original watermark with a performance improvement of 21.67 times.

More notably, on the CNN/DailyMail dataset, our semantic-aware watermarking method applied to the Flan-T5-small and Flan-T5-base models not only mitigates the drawbacks of watermark injection but also surpasses the performance of the original generation without watermark. This can be credited to the nature of the summarization task, where a considerable amount of the target information is already present in the input. The semantic-aware watermark method enhances the generation process by effectively harnessing this input, enabling it to capture the essential details for creating high-quality summaries. This synergy between input and target data contributes to the superior performance of the Flan-T5-small and Flan-T5-base models when utilizing the semantic-aware watermark method in summarization tasks.

\paragraph{Human Evaluation}
\label{sub-sec:human_eval_result}

In addition, we conducted a human evaluation comparing BART-base with the original and our proposed watermarks on the XSUM dataset. The human judges\footnote{All judges are native English speakers with a minimum of a bachelor's degree and were compensated at a rate of \$19.5/h.} were presented with reference summaries and generations from different watermarking algorithms in a random and anonymized order. The judges were asked to evaluate which system's summary was better and more similar to the reference. They were instructed to read the source article only when they were unable to decide or needed additional information\footnote{We made the decision to make reading the source article optional for the judges in order to prevent creating a significant cognitive burden and to encourage them to take shortcuts.}. 

 % TABLE [HUMAN EVAL]
    \begin{table}[t]
    \centering
    \resizebox{\columnwidth}{!}{%
    \begin{tabular}{l c c c|c}
    \toprule
     SW (ours) vs. OW & Judge 1 &   Judge 2 & Judge 3 &Avg.\\ 
    \midrule
    % RL preferred&  30.52 & 48.70 \\ 
    SW (ours) preferred & 58\% & 54\% & 54\% & 55.33\% \\ 
    \bottomrule
    \end{tabular}
    }
\caption{\label{table:pubmed-human_eval_results} Human evaluation results on 100 randomly sampled examples, accompanied by generations from BART-base with original or semantic-aware watermarks, presented in a random and anonymized order.   Each example was independently annotated by three annotators, resulting in an average pairwise inter-annotator agreement of 63.33\%.}\label{tab:human_eval}
\end{table}
% percentages of 55\%, 60\%, and 57\%, respectively
% The overall ratings were then averaged across 300 annotations.

Table~\ref{tab:human_eval} presents the results of the human evaluation. With a confidence level of 95\% and one-sided A/B tests, the semantic-aware watermark exhibits a significantly higher preference according to human judges ($p=0.0358$). Specifically, the preference for the semantic-aware watermark (55.33\%) surpasses that of the original watermark (48.00\%) by a substantial margin of 15.28\%. Moreover, pairwise inter-annotator agreement was assessed, resulting in agreement percentages of 70\%, 66\%, and 54\% for the respective evaluations. These findings strongly support the effectiveness of the semantic-aware watermark method, highlighting its ability to enhance the quality of summarization outputs.




\subsection{Watermark Strength and Detection}
% Figure environment removed

% Figure environment removed
To evaluate the quality of watermarking for detection, we followed established research \cite{kirchenbauer2023watermark, yang2023watermarking} and assessed the strength using the average $z$-score and the area under the curve (AUC) score. Figure \ref{zscore-figure} and Figure \ref{aucscore-figure} present the $z$-score and AUC results, respecively. 


A higher $z$-score generally indicates a greater presence of tokens from the "green list" in the generated results, increasing the likelihood of successful detection. However, in the context of conditional text generation tasks, maintaining consistency in the length of the generated results with the original model is crucial. It has been observed that the $z$-score tends to increase with the length of the generated text \cite{kirchenbauer2023watermark}. To address this, we introduce an additional penalty term to the $z$-score, incorporating the ratio of the average length of the generated results to the average length of the original model's output without the watermark.

 % Figure environment removed


As seen in Figure \ref{zscore-figure}, the semantic-aware watermark method significantly outperforms its counterpart in terms of $z$-score, reflecting a higher inclusion of "green list" tokens in the generated output. Under normal circumstances, an elevated average $z$-score should boost detectability \cite{kirchenbauer2023watermark}. Yet, as Figure \ref{aucscore-figure} illustrates, the AUC curve for the original watermark method surpasses ours. This paradox suggests another challenge in applying watermarks for CTG: \textbf{the prevalent human habit of using input-similar tokens for CTG adds complexity to the detection of watermarks}. Our method, despite showing remarkable improvements in ROUGE metrics and hence bearing closer resemblance to the reference, contributes to a slight dip in the final AUC scores. This scenario indicates a trade-off between enhancing the ROUGE score, indicative of increased similarity to the reference, and preserving detectability.  Notwithstanding this, our empirical results compellingly argue that the significant rise in performance (up to $\sim 2167\%$) outweighs the detection decreases (Avg. $\sim 12.6\%$); further increasing this advantage margin remains an area for future exploration.





\section{Analysis}
This section analyzes the hyperparameters, focusing on: $k$, introduced by our semantic watermark; $\gamma$ and $\delta$, inherited from \citet{kirchenbauer2023watermark}.

\subsection{Semantic $k$ Analysis}
    % 随着K的增大,semantic 找到的token能够更好的覆盖coverage,也是我们的目的并佐证motivation。

    % k的增大对最终的结果会产生影响(从appendix结果看到)。为了进一步理解k对最终ROUGE的影响,我们使用Coverage作为桥梁来具体分析(figure4包括coverage的绝对值。coverge增加的相对percent)从表中可以看到:
    % (为什么用coverage作为桥梁? 因为增加k最终导致增加了相关的token,而coverage代表了当前获得所有token对target token的覆盖。所以coverage的增加,必然会导致ROUGE指标的增加。)
    % 1. k的增加会导致coverage在不同模型和不同数据集上的提升。这也对应了不同的数据集上随着k增加伴随的ROUGE指标的增加。
    % 2. data2text数据集对k的增大更加敏感(增加的BLEU更大)。COverage增加percent能够对应并解释这个现象。(data2text数据集随着k的增加,coverage percent更大)

\begin{table}[h]
\centering
\resizebox{0.4\textwidth}{!}{\begin{tabular}{lcccc}
\toprule
Method & \multicolumn{3}{c}{BLEU} \\ \cline{2-4}
$\gamma$ & $\gamma=0.25$ & $\gamma=0.5$ & $\gamma=0.75$\\ \midrule

NW & 45.90 & - & - \\ 
\midrule
OW & 37.32 & 35.99 & 39.01 \\ 
SW (k=1) & 37.23 & 38.46 & 41.36 \\
SW (k=2) & 38.10 & 39.29 & 42.01 \\
SW (k=5) & 38.87 & 38.63 & 42.24 \\
SW (k=10) & \textbf{41.37} & \textbf{42.89} & \textbf{44.59} \\ \bottomrule
\end{tabular}}
\caption{The effect of the hyperparameter $k$ on the results of the DART dataset using the BART-base with $\gamma \in \{0.25, 0.5, 0.75\}$ and $\delta=2$.}
\label{k-table} 
\end{table}

The semantic-aware watermark uses a hyperparameter, $k$, to determine the extent of semantically related tokens, derived from word embedding similarities during decoding, that are integrated into the green list. Table \ref{k-table} shows that \textbf{increasing  $k$ in semantic-aware watermarks improve the CTG performance}.  We hypothesize that this improvement stems from that increasing $k$ includes more reference tokens in the green list, leading to a broader coverage of tokens that humans typically use for CTG generation.


To validate our hypothesis and study the relationship between $k$ and target token coverage, we carried out experiments by measuring the overlaps between semantically related tokens and the reference target tokens under different $k$ values. Figure \ref{coverage-figure} (left) presents curves, which, with increasing $k$, demonstrate a correlation with an increased proportion of target unigram text tokens covered by semantically related tokens.

Interestingly, when we adjust the setup to measure the relative percentage of coverage increase with higher $k$ values, we observe different trends for various CTG tasks. Figure \ref{coverage-figure} (right) indicates that watermarks with larger $k$ values have a more significant performance improvement impact on data-to-text generation tasks compared to summarization tasks. This observation is also reflected in the findings that an increased $k$ leads to substantial improvements in BLEU scores for data-to-text generation, compared to the ROUGE score improvements for summarization (Appendix \ref{sec:appendix}). Specifically, DART and WEBNLG show greater sensitivity to $k$, where its increase yields better results. 

% In addition, we have found this can also explain why data-to-text tasks in more sensitive to $k$,   we can get the conclusion that DART and WEBNLG is more sensitive to k, increasing k can get a higher results.
% Table \ref{k-table} further explores this phenomenon by examining the impact of different $k$ values under varying $\gamma$ on the performance of the semantic-aware watermark method using BART-base models on the DART dataset. The results show an increasing trend in performance similar to the coverage result when $k$ is increased under different $\gamma$ settings. This suggests that the employed semantic-aware watermark method effectively covers information and mitigates the performance degradation caused by adding the watermark.


\subsection{$\gamma$ and $\delta$ Analysis}

% 分析 在org-watermark 的两个重要的超参数设置下,our semantic watermark方法相较于org-watermark方法的ROUGE指标对比(从上到下 soft-hard.)。得到的结论;
% our semantic-watermark方法在绝大多数不同设置的情况下,能够取得比 org-watermark更好的效果。 

% 并且从这个图中我们可以观测到一个现象:在 delta=2并且gamma<0.2的设置下,org-watermark有一个不同寻常的上升,这是不应该的。这个观测引出了下面的假设以及实验。

%the orginal watermark has two parametrs, we study the effects of having them in our algorithm.
% figure 5, fix delta, using different gamma, and see the difference with gamma. our model outperform. 

The soft watermark method \citep{kirchenbauer2023watermark} depends on two hyperparameters: $\gamma$ and $\delta$. $\gamma$ regulates the size of the green list during partitioning, whereas $\delta$ dictates the intensity of watermarks applied to the logits of green list tokens. Essentially, a very large $\delta$ (e.g., 10) is equivalent to the hard watermark that entirely prohibits tokens from the red list from being generated. This section compares original and semantic-aware watermarks under varying $\gamma$ and $\delta$ values, demonstrating that our proposed watermark consistently outperforms the original across different hyperparameter settings.


Increasing $\gamma$ incorporates more words into the green list, typically lessening the watermark's impact on model performance. Surprisingly, Table \ref{k-table} shows that the original watermark method performs poorly when $\gamma = 0.5$. To further explore possible reasons for this and to test our methods under different setups, we conducted a comparative analysis with varying $\gamma$ and $\delta$ set to 2, 5, and 10. Figure \ref{gamma-figure} indicates that the semantic-aware watermark \textbf{consistently} outperforms the original watermark, except when $\delta$ is set to 2 with relatively small $\gamma$ values. Decreasing $\gamma$ reduces the number of selected and enhanced tokens due to the smaller green list size. As a result, the model's performance is expected to gradually decrease with a smaller watermark. However, the change curve of the original method in the $\gamma < 0.2$ range deviates from these expectations.


% Figure environment removed

We hypothesize that this irregularity arises from the negligible impact of soft watermark when $\gamma$ is small. This happens when soft watermarks with an extremely small green list scarcely affect logits predictions. To confirm this, we examined the impact of varying $\delta$ on the BART-base model's performance using the DART dataset under extrem small $\gamma$, as shown in Figure \ref{delta-figure}. We observe that when $\gamma$ is set extremely low ($\gamma=0.05$) in the soft watermark settings (i.e., $\delta$ < 4), there is hardly any performance trade-off upon adding watermarks, suggesting ineffective watermarks for detection.

% In addition, the semantic-aware watermark method exhibits a diminishing trend that gradually flattens as $\delta$ increases, while the original watermark method continues to decline steadily without convergence. When $\delta$ is sufficiently large, the semantic method surpasses the effectiveness of the original method. Increasing $\delta$ enhances the magnitude of tokens in the green list, resulting in more categorized tokens. The original method, relying on random partitioning, cannot guarantee the inclusion of target tokens in the green list, leading to continuous performance decline. In contrast, the semantic method adds semantically related tokens as a pre-conditioning constraint, ensuring the presence of target tokens in the green list and establishing a performance baseline.  
In addition, to ensure that semantically related tokens included in the green list for the semantic-aware watermark do not negatively affect the performance, especially the ones obtained with a large $k$, we calculate the percentage of these semantically related tokens relative to the overall vocabulary size. Table \ref{target-percent-table} reveals that it is significantly lower than the green list size dictated by $\gamma$.

\begin{table}[h]
\resizebox{0.47\textwidth}{!}{
\begin{tabular}{c|cccc}
% \toprule
\diagbox[width=6em]{Dataset}{\\$k$}     & 1      & 2      & 5      & 10     \\ \hline
DART   & 0.0004 & 0.0009 & 0.0020 & 0.0037 \\ 
WebNLG & 0.0005 & 0.0009 & 0.0022 & 0.0039 \\ 
% \bottomrule
\end{tabular}}
\caption{The percentage of semantically related tokens to the size of the vocabulary $V$.}
\label{target-percent-table}
\end{table}

% Figure environment removed







\section{Conclusion}
% We conducted explorations on adding watermark to conditional text generation tasks and demonstrated the impact of the original watermark method on the quality of generated text across multiple datasets and models. Moreover, taking into account the characteristics of conditional text generation tasks, we proposed the semantic-watermark method to mitigate the quality degradation caused by adding watermark. The experimental results have shown significant improvements in evaluation metrics for the semantic-watermark method and it has consistently demonstrated good performance across different settings.
Our study reveals a significant performance drop when random watermarks are directly applied to conditional text generation tasks without considering the task-specific context. To tackle this challenge, we propose a semantic-aware watermarking algorithm that incorporates hash functions and carefully takes into account the input context of conditional generation tasks. We extensively evaluated our method on diverse datasets and models, including summarization, data-to-text generation, and various text generation models like BART and Flan-T5. The results demonstrate that our proposed method effectively mitigates the quality degradation associated with watermark techniques, as confirmed by both automatic and human evaluations. These findings emphasize the importance of task-specific approaches when applying watermarking methods to ensure optimal performance in conditional text generation tasks.

\section*{Limitations}
One limitation that we did not address in our study, which we leave for future work, is how our approach handles different types of attacks for AI detection, specifically paraphrasing attacks. In addition, while our approach has shown significant improvements in downstream performance, we have also observed a slight compromise in the detection sensitivity. This trade-off can be attributed to the fact that humans often use tokens similar to the input for generation, making it more challenging to detect our semantic-aware watermark. While our research clearly shows that performance improvements outweigh the detection decreases, the challenge of further expanding this margin of advantage remains a topic for future exploration.

% We employ the $L_2$ distance for determining word vector similarity and identifying semantically-related tokens. However, there are alternate techniques, such as cosine similarity, available for measuring vector similarity. The selection of distinct similarity computation methods could influence the final outcomes, particularly in the identification of semantically-related tokens. Furthermore, in the preprocessing phase, we generate representations for each token using the associated model's embeddings. If other methods are used for the vector representations  of tokens, additional investigations are needed to assess their effectiveness.


% Entries for the entire Anthology, followed by custom entries
\bibliography{anthology,custom}
\bibliographystyle{acl_natbib}

\clearpage

\appendix
\section{Appendix}
\label{sec:appendix}

\subsection{Comprehensive Results}
In this section, we furnish a detailed analysis of our experimental outcomes to augment the findings presented in Table \ref{main_result}. These encompass comparative evaluations conducted across diverse models and datasets, under a variety of experimental conditions.

 % Figure environment removed

% Figure environment removed

% Figure environment removed

% Figure environment removed


\begin{table*}[t]
\resizebox{\textwidth}{!}{
% Please add the following required packages to your document preamble:
% \usepackage{multirow}
\begin{tabular}{lllllll|lllll}
\hline
Dataset               & model                           & method   & $\gamma$ and $k$     & ROUGE-1 & ROUGE-2 & ROUGE-L & Dataset                & model                           & method   & $\gamma$ and $k$     & BLEU  \\ \hline
\multirow{64}{*}{CNN} & \multirow{16}{*}{BART-base}     & NW       & -                    & 42.02   & 19.46   & 39.04   & \multirow{64}{*}{DART} & \multirow{16}{*}{BART-base}     & NW       & -                    & 45.90 \\ \cline{3-7} \cline{10-12} 
                      &                                 & OW       & $\gamma=0.25$       & 39.65   & 17.28   & 36.84   &                        &                                 & OW       & $\gamma=0.25$       & 37.32 \\
                      &                                 & SW (ours) & $\gamma=0.25 + k=1$  & 40.93   & 18.81   & 37.68   &                        &                                 & SW (ours) & $\gamma=0.25+k=1$  & 37.23 \\
                      &                                 & SW (ours) & $\gamma=0.25+k=2$  & 41.34   & 19.12   & 38.20   &                        &                                 & SW (ours) & $\gamma=0.25+k=2$  & 38.10 \\
                      &                                 & SW (ours) & $\gamma=0.25+k=5$  & 41.36   & 19.14   & 38.21   &                        &                                 & SW (ours) & $\gamma=0.25+k=5$  & 38.87 \\
                      &                                 & SW (ours) & $\gamma=0.25+k=10$ & 41.45   & 19.22   & 38.28   &                        &                                 & SW (ours) & $\gamma=0.25+k=10$ & 41.37 \\ \cline{3-7} \cline{10-12} 
                      &                                 & OW       & $\gamma=0.5$        & 38.13   & 16.56   & 35.33   &                        &                                 & OW       & $\gamma=0.5$        & 35.99 \\
                      &                                 & SW (ours) & $\gamma=0.5+k=1$   & 40.93   & 18.81   & 37.68   &                        &                                 & SW (ours) & $\gamma=0.5+k=1$   & 38.46 \\
                      &                                 & SW (ours) & $\gamma=0.5+k=2$   & 41.46   & 19.17   & 38.37   &                        &                                 & SW (ours) & $\gamma=0.5+k=2$   & 39.29 \\
                      &                                 & SW (ours) & $\gamma=0.5+k=5$   & 41.65   & 19.30   & 38.54   &                        &                                 & SW (ours) & $\gamma=0.5+k=5$   & 38.36 \\
                      &                                 & SW (ours) & $\gamma=0.5+k=10$  & 41.59   & 19.28   & 38.49   &                        &                                 & SW (ours) & $\gamma=0.5+k=10$  & 42.89 \\ \cline{3-7} \cline{10-12} 
                      &                                 & OW       & $\gamma=0.75$       & 40.47   & 18.08   & 37.71   &                        &                                 & OW       & $\gamma=0.75$       & 39.01 \\
                      &                                 & SW (ours) & $\gamma=0.75+k=1$  & 41.82   & 19.33   & 38.73   &                        &                                 & SW (ours) & $\gamma=0.75+k=1$  & 41.36 \\
                      &                                 & SW (ours) & $\gamma=0.75+k=2$  & 41.73   & 19.29   & 29.01   &                        &                                 & SW (ours) & $\gamma=0.75+k=2$  & 42.01 \\
                      &                                 & SW (ours) & $\gamma=0.75+k=5$  & 41.78   & 19.34   & 38.75   &                        &                                 & SW (ours) & $\gamma=0.75+k=5$  & 42.24 \\
                      &                                 & SW (ours) & $\gamma=0.75+k=10$ & 41.84   & 19.38   & 38.80   &                        &                                 & SW (ours) & $\gamma=0.75+k=10$ & 44.59 \\ \cline{2-7} \cline{9-12} 
                      & \multirow{16}{*}{BART-large}    & NW       & -                    & 43.80   & 20.88   & 40.73   &                        & \multirow{16}{*}{BART-large}    & NW       & -                    & 47.78 \\ \cline{3-7} \cline{10-12} 
                      &                                 & OW       & $\gamma=0.25$       & 42.17   & 18.40   & 39.26   &                        &                                 & OW       & $\gamma=0.25$       & 38.66 \\
                      &                                 & SW (ours) & $\gamma=0.25+k=1$  & 43.31   & 20.61   & 40.25   &                        &                                 & SW (ours) & $\gamma=0.25+k=1$  & 38.08 \\
                      &                                 & SW (ours) & $\gamma=0.25+k=2$  & 43.35   & 20.68   & 40.31   &                        &                                 & SW (ours) & $\gamma=0.25+k=2$  & 39.24 \\
                      &                                 & SW (ours) & $\gamma=0.25+k=5$  & 43.41   & 20.73   & 40.38   &                        &                                 & SW (ours) & $\gamma=0.25+k=5$  & 39.71 \\
                      &                                 & SW (ours) & $\gamma=0.25+k=10$ & 43.50   & 20.82   & 40.50   &                        &                                 & SW (ours) & $\gamma=0.25+k=10$ & 41.81 \\ \cline{3-7} \cline{10-12} 
                      &                                 & OW       & $\gamma=0.5$        & 42.46   & 18.33   & 39.52   &                        &                                 & OW       & $\gamma=0.5$        & 37.07 \\
                      &                                 & SW (ours) & $\gamma=0.5+k=1$   & 43.38   & 20.71   & 40.34   &                        &                                 & SW (ours) & $\gamma=0.5+k=1$   & 39.63 \\
                      &                                 & SW (ours) & $\gamma=0.5+k=2$   & 43.50   & 20.83   & 40.62   &                        &                                 & SW (ours) & $\gamma=0.5+k=2$   & 40.26 \\
                      &                                 & SW (ours) & $\gamma=0.5+k=5$   & 43.49   & 20.81   & 40.47   &                        &                                 & SW (ours) & $\gamma=0.5+k=5$   & 42.03 \\
                      &                                 & SW (ours) & $\gamma=0.5+k=10$  & 43.50   & 20.82   & 40.50   &                        &                                 & SW (ours) & $\gamma=0.5+k=10$  & 44.04 \\ \cline{3-7} \cline{10-12} 
                      &                                 & OW       & $\gamma=0.75$       & 43.13   & 19.55   & 40.16   &                        &                                 & OW       & $\gamma=0.75$       & 40.69 \\
                      &                                 & SW (ours) & $\gamma=0.75+k=1$  & 43.46   & 20.76   & 40.43   &                        &                                 & SW (ours) & $\gamma=0.75+k=1$  & 42.71 \\
                      &                                 & SW (ours) & $\gamma=0.75+k=2$  & 43.46   & 20.76   & 40.51   &                        &                                 & SW (ours) & $\gamma=0.75+k=2$  & 44.27 \\
                      &                                 & SW (ours) & $\gamma=0.75+k=5$  & 43.57   & 20.88   & 40.56   &                        &                                 & SW (ours) & $\gamma=0.75+k=5$  & 44.22 \\
                      &                                 & SW (ours) & $\gamma=0.75+k=10$ & 43.53   & 20.87   & 40.54   &                        &                                 & SW (ours) & $\gamma=0.75+k=10$ & 45.75 \\ \cline{2-7} \cline{9-12} 
                      & \multirow{16}{*}{Flan-T5-small} & NW       & -                    & 38.96   & 17.35   & 35.84   &                        & \multirow{16}{*}{Flan-T5-small} & NW       & -                    & 47.99 \\ \cline{3-7} \cline{10-12} 
                      &                                 & OW       & $\gamma=0.25$       & 35.44   & 14.46   & 32.84   &                        &                                 & OW       & $\gamma=0.25$       & 37.47 \\
                      &                                 & SW (ours) & $\gamma=0.25+k=1$  & 39.75   & 17.86   & 36.56   &                        &                                 & SW (ours) & $\gamma=0.25+k=1$  & 40.70 \\
                      &                                 & SW (ours) & $\gamma=0.25+k=2$  & 39.86   & 17.96   & 36.68   &                        &                                 & SW (ours) & $\gamma=0.25+k=2$  & 40.77 \\
                      &                                 & SW (ours) & $\gamma=0.25+k=5$  & 39.82   & 17.91   & 36.64   &                        &                                 & SW (ours) & $\gamma=0.25+k=5$  & 40.63 \\
                      &                                 & SW (ours) & $\gamma=0.25+k=10$ & 39.82   & 17.91   & 36.64   &                        &                                 & SW (ours) & $\gamma=0.25+k=10$ & 40.58 \\ \cline{3-7} \cline{10-12} 
                      &                                 & OW       & $\gamma=0.5$        & 35.51   & 14.56   & 32.90   &                        &                                 & OW       & $\gamma=0.5$        & 36.32 \\
                      &                                 & SW (ours) & $\gamma=0.5+k=1$   & 39.76   & 17.88   & 36.58   &                        &                                 & SW (ours) & $\gamma=0.5+k=1$   & 41.66 \\
                      &                                 & SW (ours) & $\gamma=0.5+k=2$   & 39.86   & 17.95   & 36.68   &                        &                                 & SW (ours) & $\gamma=0.5+k=2$   & 42.52 \\
                      &                                 & SW (ours) & $\gamma=0.5+k=5$   & 39.83   & 17.91   & 36.65   &                        &                                 & SW (ours) & $\gamma=0.5+k=5$   & 42.65 \\
                      &                                 & SW (ours) & $\gamma=0.5+k=10$  & 39.80   & 17.90   & 36.63   &                        &                                 & SW (ours) & $\gamma=0.5+k=10$  & 42.61 \\ \cline{3-7} \cline{10-12} 
                      &                                 & OW       & $\gamma=0.75$       & 37.15   & 15.83   & 34.30   &                        &                                 & OW       & $\gamma=0.75$       & 39.93 \\
                      &                                 & SW (ours) & $\gamma=0.75+k=1$  & 39.81   & 17.91   & 36.61   &                        &                                 & SW (ours) & $\gamma=0.75+k=1$  & 45.16 \\
                      &                                 & SW (ours) & $\gamma=0.75+k=2$  & 39.85   & 17.94   & 36.68   &                        &                                 & SW (ours) & $\gamma=0.75+k=2$  & 45.04 \\
                      &                                 & SW (ours) & $\gamma=0.75+k=5$  & 39.83   & 17.91   & 36.66   &                        &                                 & SW (ours) & $\gamma=0.75+k=5$  & 45.01 \\
                      &                                 & SW (ours) & $\gamma=0.75+k=10$ & 39.83   & 17.92   & 36.66   &                        &                                 & SW (ours) & $\gamma=0.75+k=10$ & 45.00 \\ \cline{2-7} \cline{9-12} 
                      & \multirow{16}{*}{Flan-T5-base}  & NW       & -                    & 41.78   & 19.57   & 38.66   &                        & \multirow{16}{*}{Flan-T5-base}  & NW       & -                    & 49.55 \\ \cline{3-7} \cline{10-12} 
                      &                                 & OW       & $\gamma=0.25$       & 38.24   & 16.12   & 35.59   &                        &                                 & OW       & $\gamma=0.25$       & 38.92 \\
                      &                                 & SW (ours) & $\gamma=0.25+k=1$  & 41.70   & 19.70   & 38.60   &                        &                                 & SW (ours) & $\gamma=0.25+k=1$  & 42.06 \\
                      &                                 & SW (ours) & $\gamma=0.25+k=2$  & 41.78   & 19.77   & 38.68   &                        &                                 & SW (ours) & $\gamma=0.25+k=2$  & 42.74 \\
                      &                                 & SW (ours) & $\gamma=0.25+k=5$  & 41.88   & 19.82   & 38.78   &                        &                                 & SW (ours) & $\gamma=0.25+k=5$  & 42.87 \\
                      &                                 & SW (ours) & $\gamma=0.25+k=10$ & 41.88   & 19.84   & 38.78   &                        &                                 & SW (ours) & $\gamma=0.25+k=10$ & 42.83 \\ \cline{3-7} \cline{10-12} 
                      &                                 & OW       & $\gamma=0.5$        & 38.60   & 16.29   & 35.90   &                        &                                 & OW       & $\gamma=0.5$        & 39.13 \\
                      &                                 & SW (ours) & $\gamma=0.5+k=1$   & 41.81   & 19.80   & 38.70   &                        &                                 & SW (ours) & $\gamma=0.5+k=1$   & 43.12 \\
                      &                                 & SW (ours) & $\gamma=0.5+k=2$   & 41.87   & 19.87   & 38.79   &                        &                                 & SW (ours) & $\gamma=0.5+k=2$   & 43.64 \\
                      &                                 & SW (ours) & $\gamma=0.5+k=5$   & 41.90   & 19.86   & 38.80   &                        &                                 & SW (ours) & $\gamma=0.5+k=5$   & 44.18 \\
                      &                                 & SW (ours) & $\gamma=0.5+k=10$  & 41.88   & 19.85   & 38.80   &                        &                                 & SW (ours) & $\gamma=0.5+k=10$  & 44.06 \\ \cline{3-7} \cline{10-12} 
                      &                                 & OW       & $\gamma=0.75$       & 39.21   & 17.37   & 36.42   &                        &                                 & OW       & $\gamma=0.75$       & 41.36 \\
                      &                                 & SW (ours) & $\gamma=0.75+k=1$  & 41.88   & 19.84   & 38.78   &                        &                                 & SW (ours) & $\gamma=0.75+k=1$  & 46.43 \\
                      &                                 & SW (ours) & $\gamma=0.75+k=2$  & 41.90   & 19.89   & 38.82   &                        &                                 & SW (ours) & $\gamma=0.75+k=2$  & 46.78 \\
                      &                                 & SW (ours) & $\gamma=0.75+k=5$  & 41.91   & 19.87   & 38.83   &                        &                                 & SW (ours) & $\gamma=0.75+k=5$  & 46.67 \\
                      &                                 & SW (ours) & $\gamma=0.75+k=10$ & 41.89   & 19.86   & 38.81   &                        &                                 & SW (ours) & $\gamma=0.75+k=10$ & 46.46 \\ \hline
\end{tabular}
}
\caption{Complete results on the CNN and DART dataset.}
\end{table*}


% Please add the following required packages to your document preamble:
% \usepackage{multirow}
\begin{table*}[t]
\resizebox{\textwidth}{!}{\begin{tabular}{lllllll|lllll}
\hline
Dataset                & model                           & method   & $\gamma$ and $k$     & ROUGE-1 & ROUGE-2 & ROUGE-L & Dataset                  & model                           & method   & $\gamma$ and $k$     & BLEU  \\ \hline
\multirow{64}{*}{XSUM} & \multirow{16}{*}{BART-base}     & NW       & -                    & 42.36   & 19.42   & 34.40   & \multirow{64}{*}{WebNLG} & \multirow{16}{*}{BART-base}     & NW       & -                    & 54.45 \\ \cline{3-7} \cline{10-12} 
                   &                                 & OW       & $\gamma==0.25$       & 37.84   & 14.76   & 29.89   &                          &                                 & OW       & $\gamma==0.25$       & 44.46 \\
                   &                                 & SW (ours) & $\gamma=0.25$+$k=1$  & 40.09   & 16.67   & 31.65   &                          &                                 & SW (ours) & $\gamma=0.25$+$k=1$  & 47.03 \\
                   &                                 & SW (ours) & $\gamma=0.25$+$k=2$  & 40.28   & 16.95   & 31.93   &                          &                                 & SW (ours) & $\gamma=0.25$+$k=2$  & 48.17 \\ 
                   &                                 & SW (ours) & $\gamma=0.25$+$k=5$  & 40.78   & 17.27   & 32.36   &                          &                                 & SW (ours) & $\gamma=0.25$+$k=5$  & 48.53 \\
                   &                                 & SW (ours) & $\gamma=0.25$+$k=10$ & 40.81   & 17.41   & 32.38   &                          &                                 & SW (ours) & $\gamma=0.25$+$k=10$ & 49.99 \\ \cline{3-7} \cline{10-12} 
                   &                                 & OW       & $\gamma==0.5$        & 37.99   & 14.66   & 29.82   &                          &                                 & OW       & $\gamma==0.5$        & 43.14 \\
                   &                                 & SW (ours) & $\gamma=0.5$+$k=1$   & 40.95   & 17.53   & 32.56   &                          &                                 & SW (ours) & $\gamma=0.5$+$k=1$   & 47.48 \\
                   &                                 & SW (ours) & $\gamma=0.5$+$k=2$   & 41.00   & 17.65   & 32.70   &                          &                                 & SW (ours) & $\gamma=0.5$+$k=2$   & 48.49 \\
                   &                                 & SW (ours) & $\gamma=0.5$+$k=5$   & 41.27   & 17.82   & 32.92   &                          &                                 & SW (ours) & $\gamma=0.5$+$k=5$   & 49.98 \\
                   &                                 & SW (ours) & $\gamma=0.5$+$k=10$  & 41.36   & 17.97   & 33.03   &                          &                                 & SW (ours) & $\gamma=0.5$+$k=10$  & 51.25 \\ \cline{3-7} \cline{10-12} 
                   &                                 & OW       & $\gamma==0.75$       & 40.07   & 16.67   & 31.80   &                          &                                 & OW       & $\gamma==0.75$       & 45.88 \\
                   &                                 & SW (ours) & $\gamma=0.75$+$k=1$  & 41.60   & 18.26   & 33.29   &                          &                                 & SW (ours) & $\gamma=0.75$+$k=1$  & 50.95 \\
                   &                                 & SW (ours) & $\gamma=0.75$+$k=2$  & 41.65   & 18.43   & 33.42   &                          &                                 & SW (ours) & $\gamma=0.75$+$k=2$  & 50.50 \\
                   &                                 & SW (ours) & $\gamma=0.75$+$k=5$  & 41.82   & 18.59   & 33.61   &                          &                                 & SW (ours) & $\gamma=0.75$+$k=5$  & 51.60 \\
                   &                                 & SW (ours) & $\gamma=0.75$+$k=10$ & 41.85   & 18.61   & 33.60   &                          &                                 & SW (ours) & $\gamma=0.75$+$k=10$ & 53.01 \\ \cline{2-7} \cline{9-12} 
                   & \multirow{16}{*}{BART-large}    & NW       & -                    & 45.25   & 22.15   & 37.03   &                          & \multirow{16}{*}{BART-large}    & NW       & -                    & 57.18 \\ \cline{3-7} \cline{10-12} 
                   &                                 & OW       & $\gamma==0.25$       & 40.13   & 16.75   & 31.83   &                          &                                 & OW       & $\gamma==0.25$       & 47.87 \\
                   &                                 & SW (ours) & $\gamma=0.25$+$k=1$  & 42.26   & 18.63   & 33.54   &                          &                                 & SW (ours) & $\gamma=0.25$+$k=1$  & 49.03 \\
                   &                                 & SW (ours) & $\gamma=0.25$+$k=2$  & 42.51   & 18.92   & 33.90   &                          &                                 & SW (ours) & $\gamma=0.25$+$k=2$  & 49.14 \\
                   &                                 & SW (ours) & $\gamma=0.25$+$k=5$  & 43.01   & 19.48   & 34.39   &                          &                                 & SW (ours) & $\gamma=0.25$+$k=5$  & 49.60 \\
                   &                                 & SW (ours) & $\gamma=0.25$+$k=10$ & 43.26   & 19.68   & 34.69   &                          &                                 & SW (ours) & $\gamma=0.25$+$k=10$ & 51.29 \\ \cline{3-7} \cline{10-12} 
                   &                                 & OW       & $\gamma==0.5$        & 40.07   & 16.51   & 31.49   &                          &                                 & OW       & $\gamma==0.5$        & 44.58 \\
                   &                                 & SW (ours) & $\gamma=0.5$+$k=1$   & 43.25   & 19.63   & 34.70   &                          &                                 & SW (ours) & $\gamma=0.5$+$k=1$   & 49.66 \\
                   &                                 & SW (ours) & $\gamma=0.5$+$k=2$   & 43.41   & 19.82   & 34.94   &                          &                                 & SW (ours) & $\gamma=0.5$+$k=2$   & 50.77 \\
                   &                                 & SW (ours) & $\gamma=0.5$+$k=5$   & 43.65   & 20.14   & 35.18   &                          &                                 & SW (ours) & $\gamma=0.5$+$k=5$   & 51.50 \\
                   &                                 & SW (ours) & $\gamma=0.5$+$k=10$  & 43.83   & 20.39   & 35.41   &                          &                                 & SW (ours) & $\gamma=0.5$+$k=10$  & 52.50 \\ \cline{3-7} \cline{10-12} 
                   &                                 & OW       & $\gamma==0.75$       & 42.02   & 18.43   & 33.27   &                          &                                 & OW       & $\gamma==0.75$       & 47.31 \\
                   &                                 & SW (ours) & $\gamma=0.75$+$k=1$  & 44.24   & 20.80   & 35.82   &                          &                                 & SW (ours) & $\gamma=0.75$+$k=1$  & 52.59 \\
                   &                                 & SW (ours) & $\gamma=0.75$+$k=2$  & 44.27   & 20.84   & 35.86   &                          &                                 & SW (ours) & $\gamma=0.75$+$k=2$  & 53.27 \\
                   &                                 & SW (ours) & $\gamma=0.75$+$k=5$  & 44.49   & 21.05   & 36.04   &                          &                                 & SW (ours) & $\gamma=0.75$+$k=5$  & 54.04 \\
                   &                                 & SW (ours) & $\gamma=0.75$+$k=10$ & 44.45   & 21.11   & 36.10   &                          &                                 & SW (ours) & $\gamma=0.75$+$k=10$ & 54.45 \\ \cline{2-7} \cline{9-12} 
                   & \multirow{16}{*}{Flan-T5-small} & NW       & -                    & 33.57   & 12.00   & 26.50   &                          & \multirow{16}{*}{Flan-T5-small} & NW       & -                    & 56.41 \\ \cline{3-7} \cline{10-12} 
                   &                                 & OW       & $\gamma==0.25$       & 29.53   & 8.55    & 22.69   &                          &                                 & OW       & $\gamma==0.25$       & 37.93 \\
                   &                                 & SW (ours) & $\gamma=0.25$+$k=1$  & 32.35   & 10.68   & 25.26   &                          &                                 & SW (ours) & $\gamma=0.25$+$k=1$  & 48.45 \\
                   &                                 & SW (ours) & $\gamma=0.25$+$k=2$  & 32.63   & 10.91   & 25.4    &                          &                                 & SW (ours) & $\gamma=0.25$+$k=2$  & 49.47 \\
                   &                                 & SW (ours) & $\gamma=0.25$+$k=5$  & 32.61   & 10.99   & 25.55   &                          &                                 & SW (ours) & $\gamma=0.25$+$k=5$  & 49.73 \\
                   &                                 & SW (ours) & $\gamma=0.25$+$k=10$ & 32.73   & 11.08   & 25.67   &                          &                                 & SW (ours) & $\gamma=0.25$+$k=10$ & 50.03 \\ \cline{3-7} \cline{10-12} 
                   &                                 & OW       & $\gamma==0.5$        & 30.51   & 9.13    & 23.53   &                          &                                 & OW       & $\gamma==0.5$        & 40.88 \\
                   &                                 & SW (ours) & $\gamma=0.5$+$k=1$   & 32.80   & 11.08   & 25.74   &                          &                                 & SW (ours) & $\gamma=0.5$+$k=1$   & 50.05 \\
                   &                                 & SW (ours) & $\gamma=0.5$+$k=2$   & 33.00   & 11.26   & 25.92   &                          &                                 & SW (ours) & $\gamma=0.5$+$k=2$   & 51.30 \\
                   &                                 & SW (ours) & $\gamma=0.5$+$k=5$   & 33.12   & 11.40   & 25.98   &                          &                                 & SW (ours) & $\gamma=0.5$+$k=5$   & 51.65 \\
                   &                                 & SW (ours) & $\gamma=0.5$+$k=10$  & 33.15   & 11.39   & 26.01   &                          &                                 & SW (ours) & $\gamma=0.5$+$k=10$  & 52.16 \\ \cline{3-7} \cline{10-12} 
                   &                                 & OW       & $\gamma==0.75$       & 31.86   & 10.24   & 24.91   &                          &                                 & OW       & $\gamma==0.75$       & 47.19 \\
                   &                                 & SW (ours) & $\gamma=0.75$+$k=1$  & 33.30   & 11.53   & 26.27   &                          &                                 & SW (ours) & $\gamma=0.75$+$k=1$  & 52.84 \\
                   &                                 & SW (ours) & $\gamma=0.75$+$k=2$  & 33.36   & 11.60   & 26.27   &                          &                                 & SW (ours) & $\gamma=0.75$+$k=2$  & 53.85 \\
                   &                                 & SW (ours) & $\gamma=0.75$+$k=5$  & 33.46   & 11.68   & 26.32   &                          &                                 & SW (ours) & $\gamma=0.75$+$k=5$  & 54.23 \\
                   &                                 & SW (ours) & $\gamma=0.75$+$k=10$ & 33.30   & 11.60   & 26.21   &                          &                                 & SW (ours) & $\gamma=0.75$+$k=10$ & 54.29 \\ \cline{2-7} \cline{9-12} 
                   & \multirow{16}{*}{Flan-T5-base}  & NW       & -                    & 39.51   & 16.92   & 31.89   &                          & \multirow{16}{*}{Flan-T5-base}  & NW       & -                    & 59.77 \\ \cline{3-7} \cline{10-12} 
                   &                                 & OW       & $\gamma==0.25$       & 34.47   & 12.00   & 26.94   &                          &                                 & OW       & $\gamma==0.25$       & 41.83 \\
                   &                                 & SW (ours) & $\gamma=0.25$+$k=1$  & 37.66   & 14.92   & 29.91   &                          &                                 & SW (ours) & $\gamma=0.25$+$k=1$  & 51.56 \\
                   &                                 & SW (ours) & $\gamma=0.25$+$k=2$  & 37.85   & 15.06   & 30.12   &                          &                                 & SW (ours) & $\gamma=0.25$+$k=2$  & 51.53 \\
                   &                                 & SW (ours) & $\gamma=0.25$+$k=5$  & 38.23   & 15.40   & 30.55   &                          &                                 & SW (ours) & $\gamma=0.25$+$k=5$  & 52.15 \\
                   &                                 & SW (ours) & $\gamma=0.25$+$k=10$ & 38.24   & 15.45   & 30.53   &                          &                                 & SW (ours) & $\gamma=0.25$+$k=10$ & 51.99 \\ \cline{3-7} \cline{10-12} 
                   &                                 & OW       & $\gamma==0.5$        & 35.23   & 12.57   & 27.52   &                          &                                 & OW       & $\gamma==0.5$        & 45.42 \\
                   &                                 & SW (ours) & $\gamma=0.5$+$k=1$   & 38.34   & 15.45   & 30.66   &                          &                                 & SW (ours) & $\gamma=0.5$+$k=1$   & 51.98 \\
                   &                                 & SW (ours) & $\gamma=0.5$+$k=2$   & 38.49   & 15.72   & 30.82   &                          &                                 & SW (ours) & $\gamma=0.5$+$k=2$   & 52.38 \\
                   &                                 & SW (ours) & $\gamma=0.5$+$k=5$   & 38.79   & 15.91   & 31.02   &                          &                                 & SW (ours) & $\gamma=0.5$+$k=5$   & 52.89 \\
                   &                                 & SW (ours) & $\gamma=0.5$+$k=10$  & 38.67   & 15.89   & 31.01   &                          &                                 & SW (ours) & $\gamma=0.5$+$k=10$  & 53.27 \\ \cline{3-7} \cline{10-12} 
                   &                                 & OW       & $\gamma==0.75$       & 36.98   & 14.13   & 29.34   &                          &                                 & OW       & $\gamma==0.75$       & 50.39 \\
                   &                                 & SW (ours) & $\gamma=0.75$+$k=1$  & 38.92   & 16.15   & 31.26   &                          &                                 & SW (ours) & $\gamma=0.75$+$k=1$  & 55.19 \\
                   &                                 & SW (ours) & $\gamma=0.75$+$k=2$  & 39.00   & 16.31   & 31.38   &                          &                                 & SW (ours) & $\gamma=0.75$+$k=2$  & 55.28 \\
                   &                                 & SW (ours) & $\gamma=0.75$+$k=5$  & 39.13   & 16.35   & 31.47   &                          &                                 & SW (ours) & $\gamma=0.75$+$k=5$  & 55.82 \\
                   &                                 & SW (ours) & $\gamma=0.75$+$k=10$ & 39.03   & 16.34   & 31.44   &                          &                                 & SW (ours) & $\gamma=0.75$+$k=10$ & 55.47 \\ \hline
\end{tabular}}

\caption{Complete results on the XSUM and WebNLG datasets.}
\end{table*}







\end{document}
