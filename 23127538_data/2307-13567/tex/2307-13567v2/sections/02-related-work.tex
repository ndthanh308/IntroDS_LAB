\begin{table}[t]
\caption{Types and percentages of charts composed of lines, circles, pies, arcs, rectangles, and other marks in the Beagle dataset~\cite{battle2018beagle}.}
    \centering
    {\small
    \setlength\tabcolsep{1.5pt}
    \begin{tabular}{lp{0.725\linewidth}r}
    \toprule
     Mark   & Chart  & Percentage\\
     \midrule
      Rectangle & bar chart~(histogram), grouped bar chart, stacked bar chart, diverging bar chart (pyramid chart), Marimekko chart, heatmap, bullet chart, treemap, waffle chart, waterfall chart, range chart, gantt chart, matrix chart, cartogram, calendar chart & 32.85\%\\
      Line  &  line graph, parallel coordinates, Kagi chart &  30.51\%\\
      Pie &  pie chart, donut chart & 16.50\%\\
      Circle &  scatter plot, bubble plot, dot plot, circle packing & 14.96\%\\
      Others & geographic map, area chart, stream graph, chord chart, hexbin plot, Sankey diagram, Voronoi diagram, word cloud, sunburst chart, boxplot, network diagram, contour plot, radial plot & 5.18\%\\
     \bottomrule
    \end{tabular}
    }
    \label{tbl:lineCircleRect}
\end{table}

\section{Related Work}   
\subsection{\revise{Chart Reuse} Approaches}\label{sec:2.1}
\revise{To create new charts, previous studies on visualization designers' practices~\cite{bigelow_reflections_2014,walny_data_2019} suggest that it is more natural to change existing graphics than to start from scratch.}
Templates are generally recognized as a user-friendly way to create charts, especially for beginners. In traditional template-based systems, templates are created by system developers and they usually suffer from limited expressivity and quantity. 
Previous research thus has investigated how to turn existing visualizations into reusable templates without involving developers. For example, D3 Deconstructor \cite{harper_deconstructing_2014,harper_converting_2017} works on basic charts created using D3.js; iVoLVER \cite{mendez_ivolver_2016} extracts data from charts and updates them with new data; %However, they both support only basic chart types. 
Ivy \cite{mcnutt_integrated_2021} supports turning JSON-based declarative specifications into parameterized templates; 
% Mystique focuses on SVG charts, and does not require the examples to be created using a specification language. 
Chen~\etal~\cite{chen_towards_2020} use deep learning to extract timelines from infographics as templates; and Chartreuse \cite{cui_mixed-initiative_2022} supports reusing infographics bar chart templates. 

Overall, D3 Deconstructor and Chartreuse are the closest work to Mystique. D3 Deconstructor only takes charts created using D3 \cite{bostock_d3_2011}, which have the source data embedded, and Chartreuse primarily works on Microsoft PowerPoint graphics assets. In contrast, Mystique works on visualizations in the general SVG format, does not require access to underlying data, \revise{and supports more advanced layouts}. 
% We elaborate on the differences between Mystique and these tools in~\cref{sec:comparison}.

% Similar to Chartreuse, Mystique adopts a mixed initiative approach to visualization reuse, the primary difference lies in the types of visualizations. Chartreuse primarily works on Microsoft PowerPoint graphics assets and designs with highly customized glyphs. While the variation of glyph design is rich, the layout and structure of the visualizations are simple: Chartreuse only works on infographics bar charts. Mystique handles real-world SVGs with diverse and complex structures such as nested hierarchy and layouts resulting from multi-variate datasets even though it focuses on a rectangle mark.

% Mystique focuses on charts composed of rectangle marks, with an emphasis on layouts and nested structures. 

\subsection{Chart Understanding and Deconstruction}
Making a visualization example reusable requires understanding and deconstructing visualizations.  
Various automated or semi-automated methods have been proposed to detect marks \cite{ying_glyphcreator_2022,chen_towards_2020} as well as axes and legends \cite{shukla_recognition_2008, choudhury_scalable_2016}, classify chart types \cite{savva_revision_2011,shukla_recognition_2008}, and extract data \cite{jung_chartsense_2017,harper_converting_2017,harper_deconstructing_2014,masson2023chartdetective} and visual encodings \cite{poco_reverse-engineering_2017,poco_extracting_2018,harper_deconstructing_2014,harper_converting_2017,cui_mixed-initiative_2022}. Due to the vast space of visualization examples, these methods typically narrow the scope by focusing on specific glyph or chart types. 

\bpstart{Mark Detection}~Many approaches assume that input visualizations are in a raster image format, where object detection is essential. For example, GlyphCreator \cite{ying_glyphcreator_2022} focuses on circular glyphs, and uses deep learning to perform object and bounding box detection. Similarly, visual elements in timeline infographics can be identified using deep learning \cite{chen_towards_2020}. OCR is typically used to recognize text elements \cite{poco_reverse-engineering_2017}. Since our input format is SVG, mark detection is not necessary.

\bpstart{Axis and Legend Detection}~Simple heuristics \cite{shukla_recognition_2008,poco_extracting_2018} or supervised learning \cite{poco_reverse-engineering_2017} can be used to extract
axes and legends.
%Poco and Heer  detected axis and legend byclassifying text role into categories such as legend title, legend label, and axis label using SVM. 
However, these methods can still be error-prone. Since it is relatively easy to indicate where the axes and legends are, some tools expect users to provide such information \cite{poco_extracting_2018}. Mystique uses heuristics to find axes and legends, and provides a user interface for authors to correct potential mistakes.

\bpstart{Data Extraction} ~Previous work also addressed extracting data values from visualization images \cite{savva_revision_2011,jung_chartsense_2017} or vector graphics \cite{harper_converting_2017,harper_deconstructing_2014, masson2023chartdetective}. In Mystique, we demonstrate that a chart can be effectively reused without recovering the original data. Thus, data extraction is not necessary. 

\bpstart{Extraction of Visual Encoding and Spatial Arrangements} Inferring a visual encoding concerns the identification of relevant visual channel, data type, and potentially scale type. For glyphs with regular shapes (e.g., rectangles), visual encodings can be inferred using heuristics by combining information from mark type and axis \cite{poco_reverse-engineering_2017}. For custom glyphs (e.g., those used in infographics), \revise{sometimes the positions are not strictly encoded by data, but instead determined by specific spatial relationships or constraints. In these cases, current approaches usually classify charts into a predefined set of spatial arrangements~\cite{cui_mixed-initiative_2022,chen_towards_2020}. In Mystique, we break down the spatial arrangement of a chart into semantic components to handle more complex layouts.}

% \subsection{Mark and Chart Type Classification}

\bpstart{Chart Type Classification} Previous work also tackled the chart type classification problem. Most approaches are based on a simple chart taxonomy that roughly corresponds to mark types. For example, Revision \cite{savva_revision_2011} classifies chart images into 10 categories using SVM: area, bar, line, map, Pareto, pie, radar, scatter plot, table, and Venn diagram. This taxonomy is used in subsequent neural network-based methods \cite{poco_reverse-engineering_2017,jung_chartsense_2017}. In this work, we decided not to classify mark or chart types because such taxonomies are inadequate to capture the richness and variations of visualization design. Instead, we deconstruct charts into finer-grained semantic components.