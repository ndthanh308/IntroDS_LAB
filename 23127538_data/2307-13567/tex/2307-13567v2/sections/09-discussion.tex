\section{Discussion, Limitations, and Future Work}
\label{sec:discussion}

% \chen{end chart produced by atlas}


\noindent \textbf{Composite Designs.} The current deconstruction algorithm in Mystique cannot fully handle composite visualizations \cite{javed2012exploring} involving 
superimposition (e.g., \cref{fig:errors}), 
% , one type of composite visualizations \cite{javed2012exploring}. Other types of composite visualizations are 
juxtaposition, overloading, and nesting.
% , \markup{which currently cannot be fully handled either}. 
% For rectangle-based charts, we have demonstrated Mystique's ability to handle nested groups and layouts, and small multiples (which is a form of juxtaposition). However, the deconstruction algorithm cannot handle scatterplot matrix yet when extending to circle-based designs, as discussed in \cref{sec:circle}. 
Future work needs to extend the deconstruction algorithm or devise new methods to handle composite designs. For instance, given a design consisting of multiple views, we first need to dissect it into multiple visualizations. It remains to be seen how existing techniques on decomposing complex figures \cite{jiang_two-stage_2021,lee_detecting_2015,shi_layout-aware_2019} perform on real-world SVG visualizations.

\bpstart{Handling More Deconstruction Errors}
Mystique handles the potential errors and uncertainties in encoding detection by letting users choose the correct visual channel from a drop-down menu. As future work expands the scope to handle more complex charts like composite designs, it is expected that more errors will arise in the deconstruction process. How to support the user's understanding and provide ways to correct these errors remains an open problem. The challenge here is to minimize the requirement of the users' knowledge of abstract concepts and operations related to the GREC-component model. To do so, we need to have a thorough understanding of the error space, and then devise representation and interaction mechanisms to let users provide input in ways they can understand and perform. 

% \bpstart{Data Format and Transformation} 
% Users' data may not be readily consumable by an authoring tool and needs to be transformed \cite{satyanarayan_critical_2019}. In particular, users face two challenges: conceptualizing the
% expected data layout and implementing the transformation \cite{wang_falx_2021}. Previous work (e.g., Falx \cite{wang_falx_2021}) has demonstrated the possibility of using program synthesis to automatically infer the required layout and transform the input data. 
% Mystique adopts a different approach: it addresses the first challenge 
% through auto-generated sample data and compatibility checking. 
% Users can get a clear idea of what the input dataset should look like, and prepare the data either from scratch or by transforming an existing dataset. Mystique does not directly address the second challenge (implementing the transformation). 

% Each of the two approaches has its pros and cons. The synthesis-based approach removes the need to perform data transformation, but still requires the data to be in a tidy format \cite{wickham_tidy_2014}. It is thus likely only going to work on input data that can be automatically morphed using the predefined data transformation operations in the system. If the system cannot find a feasible data transformation process~(\eg when the input data is not in a tidy format), users would have no clue what went wrong and how to intervene. Mystique, on the other hand, treats data preparation as a separate stage. Interactive data transformation tools like Data Wrangler \cite{kandel_wrangler_2011}, Tableau Data Prep \cite{tableau_prep}, and Trifacta \cite{trifacta} can be used in conjunction with Mystique to reuse examples effectively without programming, and they are more likely to support a wider range of idiosyncratic input data. 

\bpstart{\markup{Algorithmic Layouts}} \markup{Currently Mystique is unable to detect variants of the packing relationship. For instance, Mystique cannot tell apart a squarified treemap layout~\cite{bruls2000squarified} from a spatially ordered treemap layout~\cite{shneiderman2001ordered,wood2008spatially}. Even if such layout differences are known, the underlying library Mascot.js cannot reproduce the required layout yet. More investigations are necessary to accommodate those cases.}

% \markup{Mystique uses Atlas.js~\cite{liu_atlas_2021} as the underlying library, which implements its own tiling algorithm for the packing relationship. However, the tiling algorithm has many variants, some of which may not require a universal constant gap parameter and introduce SVG examples that cannot be fully reused by Mystique. How to accommodate those cases in Mystique needs more investigation.
% }

\bpstart{\markup{Corpus Generalizability}} \markup{
We manually collected the corpus for evaluation to ensure diversity~\cite{chen2023state}, but the corpus size is small and may not be sufficient for further investigations of alternative deconstruction and reuse methods (e.g., neural network models). 
% the size of which might be insufficient for developing neural network-based models. 
The corpus can be augmented in the following two aspects to support future research: (1) incorporating charts composed of other types of marks to enhance diversity and expressiveness and (2) for each chart type, increasing the number of charts evenly across different tools/sources.
}

\bpstart{Additional Features} Beside the research challenges outlined above, Mystique can benefit from a few feature enhancements. 
The user study revealed that people wanted to customize the design while performing the reuse steps~(e.g., categorical label ordering, the color set in the legend). Such functionality can be integrated into the reuse UI. 
Handling bespoke axis design is another potential future improvement. Mystique currently generates simple axes automatically based on detected axis information, and needs to support customizations such as label formatting, flipped axis, and dual axes (e.g., \cref{fig:teaser}(h)).
Finally, the visual style information of an online SVG chart sometimes is not embedded inside the SVG file, but stored in the web page or even a separate style sheet.
Capturing such visual styles is currently a manual process.
We expect that a future version of Mystique can support the automatic capturing of SVG charts and associated visual styles with simple interactions directly in the browser. 

% \bpstart{Reusing Interaction}
% In this paper, we focus the reuse of visual representations in examples. While beyond the scope of this paper, it would be valuable to support the reuse of an example's interactive behavior as well. Future work along this direction needs to investigate the following issues: (1) model and represent the interaction behavior of an example, (2) deconstruct interactive behavior into composable components, (3) design interfaces for the user to verify the components, and (4) establish relationships between the interactive components and the user's dataset. 