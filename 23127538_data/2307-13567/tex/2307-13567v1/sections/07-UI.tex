\section{\revise{Data Compatibility and Step Generation}}\label{sec:reuseUI}

\subsection{\revise{Data Schema Inference}}\label{sec:checkingData}
To mitigate the risk of example misuse, Mystique infers data schema from the deconstructed semantic components. First, Mystique calculates $C_{group}$, the number of categorical data fields required to generate the grouping structure: each level in the deconstructed grouping structures corresponds to a unique categorical field. For instance, the diverging stacked bar chart in \cref{fig:teaser}d has a two-level nested structure 
% : one collection of collections and eight collections of rectangles, 
which requires at least two categorical fields. 

Mystique also infers the number of categorical fields ($C_{encode}$) and the number of quantitative fields ($Q_{encode}$) required for encodings. The number of encoded channels is used as the number of data fields, and the data field types are obtained through the extracted and user-corrected axis and legend information. Since it is possible that the same field may be used to generate grouping structures and encode visual channels (e.g., \textit{response type} in \cref{fig:teaser}d), Mystique uses $\max(C_{group},\ C_{encode})$ as the minimum number of categorical fields.

Based on this analysis, Mystique generates a sample dataset for a given example (\cref{fig:sample_data}). It tries to find data values for each field based on axis and legend labels. If such data is not available, the field values are generated as random strings or numbers. The sample dataset is then created as the permutation of all the field values. Mystique also displays a guideline on the minimum numbers of categorical fields and quantitative fields in the Dataset Compatibility Panel in the reuse UI~(\cref{fig:reuseUI}), and checks if the dataset satisfies this requirement whenever a new dataset is imported. Mystique issues a popup-dialog warning whenever the requirement is not met. Since users may use the same field to encode multiple channels, Mystique does not enforce the data schema as a strict rule and users can dismiss the dialog.
% Users can dismiss the dialog and continue working with Mystique. Mystique does not enforce the data schema as a strict rule because users may choose to use the same field to encode multiple channels.

\subsection{Generating Reuse Steps}\label{sec:generateReuseSteps}
Based on the deconstruction results, Mystique generates a sequence of data mapping steps that guide users toward the creation of a new chart. The steps are arranged in the following order: (1) mapping groups to categorical or date fields, from the highest level to the lowest level, (2) mapping rectangle marks to categorical or date fields, (3) choosing visual channels and data fields for size, position, area and fill encodings. 

As mentioned in \cref{sec:encInfer}, by default Mystique chooses x/y as the channels for position encodings, and width/height as the channels for size encodings. In some visualization designs, however, position and size encodings can be interchangeable and the distinction between the two may not be clear-cut. Consider the range chart in \cref{fig:teaser}b, it applies position encodings to the top segment and the bottom segment of each rectangle, which represent the daily maximum and minimum temperature respectively. However, it is also reasonable to infer the presence of a size encoding, where the height encodes the temperature range. In such ambiguous cases, Mystique is unable to clearly distinguish between position and size encodings. It thus provides multiple possible visual channels (e.g., \textit{top side}, \textit{bottom side}, and \textit{height}) through a drop-down menu for users to specify which channel should be used.

To update the visualization result at every step, we use \markup{Mascot.js (previously known as Atlas.js}~\cite{liu_atlas_2021}) as the underlying library. Its graphics-centric and procedural design enables displaying intermediate visualizations incrementally, so that users can evaluate whether they are on the right track.
The demo website showcases the results of reusing a variety of example charts, as a demonstration of Mystique's expressiveness. 


