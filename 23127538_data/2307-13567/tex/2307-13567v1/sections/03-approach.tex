\section{Overall Approach and Usage Scenario}
\label{sec:overview}

% \subsection{Challenges in Reusing Real-World SVG Visualizations}\label{svg}
\subsection{Challenges and Processing Pipeline}

% % Figure environment removed

% Figure environment removed

% \subsubsection{Unreliable Semantic Information}\label{sec:SVGchallenges}
\markup{We identify two main challenges in understanding and reusing SVG chart layouts. First, the semantic information such as mark attributes and hierarchical grouping in SVG specifications are not reliable. As we examined online SVG charts from different sources, we found that a majority of them were not readily usable. The following observations characterize the uncertainties in semantic structures: \textbf{ (1) Inconsistent SVG Element Types}: the same mark type may be represented using different types of SVG elements. For instance, we have observed in multiple examples that a rectangle mark is represented using a $<$path$>$ element, and an axis is drawn as a thin rectangle. Thus, we cannot determine the mark type based on the SVG element type; \textbf{(2) Missing Absolute Positions}: the absolute positions of elements in a chart are crucial for determining their graphical roles and spatial relationships. But in many cases, an element's position is not expressed in absolute coordinates. Instead, the positions are often described using transformations such as ``translate'' or the matrix function; \textbf{(3) Noise in Scene Graph}: it is non-trivial to distinguish visualization marks from graphical objects that are not part of the main visualization, which include off-screen tooltips, transparent or white rectangles serving as backgrounds, and random watermarks drawn as $<$path$>$ elements; and \textbf{ (4) Arbitrary Grouping of Elements}: 
% the semantic information in SVG visualizations in the wild is often unreliable. In particular, 
the grouping of elements can be unpredictable in SVG charts from the wild. For instance, grouped bar charts created using different tools exhibit vastly different grouping structures, and axis labels can be either within one group or in their own individual groups~\cite{chen2023state}.} 
% Given the structured format and clear element tags in the SVG specification, we initially expected that it would be straightforward to obtain semantic information such as mark attributes and hierarchical grouping. To our surprise, as we examined more online SVG charts from different sources, a majority of them are not readily usable. The following observations constitute challenges in understanding semantic structures: (1) Inconsistent SVG Element Types,~(2) Missing Absolute Position,~(3) Noise in the Scene Graph,~and (4) Arbitrary Grouping of Elements.~\Cref{sec:svg} includes details about these challenges and required pre-processing, and~\Cref{sec:chartDecomp} covers our core algorithm to determine semantic components (including grouping) deciding chart layouts.

% \subsubsection{Data Format and Transformation}
\markup{Secondly, the \textbf{schematic congruency} \cite{satyanarayan_critical_2019}  between user's data and an example's layout structure is not guaranteed: the data may be formatted or structured in a way that cannot be readily applied to an extracted layout. To reuse the layout, users may face two obstacles: conceptualizing the
expected data layout and implementing the transformation \cite{wang_falx_2021}. Previous work (e.g., Falx \cite{wang_falx_2021}) has demonstrated the possibility of using program synthesis to automatically infer the required layout and transform the input data. Such an approach removes the need to perform data transformation, but
still requires the data to be in a tidy format \cite{wickham_tidy_2014}. It is thus likely only
going to work on input data that can be automatically morphed using
the predefined data transformation operations in the system. If the
system cannot find a feasible data transformation process (e.g., when
the input data is not in a tidy format), users would have no clue what
went wrong and how to intervene.}

% Users' data may not be readily consumable by an authoring tool and needs to be transformed \cite{satyanarayan_critical_2019}. In particular, users face two challenges: conceptualizing the
% expected data layout and implementing the transformation \cite{wang_falx_2021}. Previous work (e.g., Falx \cite{wang_falx_2021}) has demonstrated the possibility of using program synthesis to automatically infer the required layout and transform the input data. 
% Mystique adopts a different approach: it addresses the first challenge 
% through auto-generated sample data and compatibility checking. 
% Users can get a clear idea of what the input dataset should look like, and prepare the data either from scratch or by transforming an existing dataset. Mystique does not directly address the second challenge (implementing the transformation). \Cref{sec:reuseUI} covers how Mystique infers the required data format from a given chart, and \Cref{sec:discussion} includes more and discussions.

% Each of the two approaches has its pros and cons. The synthesis-based approach removes the need to perform data transformation, but still requires the data to be in a tidy format \cite{wickham_tidy_2014}. It is thus likely only going to work on input data that can be automatically morphed using the predefined data transformation operations in the system. If the system cannot find a feasible data transformation process~(\eg when the input data is not in a tidy format), users would have no clue what went wrong and how to intervene. Mystique, on the other hand, treats data preparation as a separate stage. Interactive data transformation tools like Data Wrangler \cite{kandel_wrangler_2011}, Tableau Data Prep \cite{tableau_prep}, and Trifacta \cite{trifacta} can be used in conjunction with Mystique to reuse examples effectively without programming, and they are more likely to support a wider range of idiosyncratic input data. 

\markup{To address these two challenges, Mystique adopts a pipeline (\Cref{fig:overview}) for extracting and reusing layouts. 
% Receiving raw SVG-chart inputs, we introduce a pipeline (\Cref{fig:overview}) for reusing layouts and realize it in Mystique. 
The pipeline consists of the following stages:
\textit{pre-processing} raw SVGs~(\cref{sec:svg})}, \textit{detecting} axis and legend information with a user \textit{correcting} axis and legend detection mistakes~(\cref{sec:axisLegend}), \textit{deconstructing} the chart content into semantic components \revise{that jointly decide the chart layout~(\cref{sec:chartDecomp}), %\textit{checking} data compatibility~(\cref{sec:checkingData}), 
\textit{generating} chart reuse steps after checking data compatibility}~(\cref{sec:reuseUI}), and finally the user \textit{specifying} how these components map to data to create a visualization with new data. 
While users collaborate with the system throughout the process, Mystique strives to minimize effort and skills from them. 
\markup{The pre-processing stage handles inconsistent element types, missing absolute positions, and noise in SVG specifications; the mixed-initiative stages of axis \& legend detection and chart deconstruction handle arbitrary grouping of elements; finally, the pipeline includes a step to help users understand expected data layout through auto-generated sample data and compatibility checking. 
Users can get a clear idea of what the input dataset should look like, and prepare the data either from scratch or by transforming an existing dataset. Mystique does not directly address the issue of implementing the transformation. By treating data preparation as a separate stage, Mystique can be used in conjunction with interactive data transformation tools like Data Wrangler \cite{kandel_wrangler_2011}, Tableau Data Prep \cite{tableau_prep}, and Trifacta \cite{trifacta}, which are more likely to support a wider range of idiosyncratic input data.  Sections 4 - 7 provide details on these stages in the pipeline.}

\subsection{Usage Scenario}

% With SVG charts preprocessed into tidy JSON objects, we introduce a pipeline (\Cref{fig:overview}) for reusing layouts and realize it in Mystique. The pipeline consists of the following stages:
% \markup{Receiving raw SVG-chart inputs, we introduce a pipeline (\Cref{fig:overview}) for reusing layouts and realize it in Mystique. The pipeline consists of the following stages:
% \textit{pre-processing} raw SVGs to extract low-level semantic information}, \textit{detecting} axis and legend information with a user \textit{correcting} axis and legend detection mistakes~(\cref{sec:axisLegend}), \textit{deconstructing} the chart content into semantic components \revise{that jointly decide the chart layout~(\cref{sec:chartDecomp}), %\textit{checking} data compatibility~(\cref{sec:checkingData}), 
% \textit{generating} chart reuse steps after checking data compatibility}~(\cref{sec:reuseUI}), and finally the user \textit{specifying} how these components map to data to create a visualization with new data. While users collaborate with the system throughout the process, Mystique strives to minimize effort and skills from them. 

In this section, we illustrate the pipeline and how a user interacts with Mystique using a treemap grouped bar chart~\cite{treemapBar} as an example. The complete reuse process is presented on our demo website, \url{https://mystique-vis.github.io}. The chart uses a hybrid layout design, where the overall grouped bar chart representation shows trade values in different years, with the bar height encoding the total value. Within each bar, a two-level treemap shows the proportion contributed by each country, colored and grouped by continent. 
Currently, the only way to create \revise{a chart of such a bespoke layout} is to program using libraries like D3, and it requires deep D3 expertise and significant time to modify the code for a new dataset. In contrast, Mystique enables the reuse of such a chart on new datasets with simple interactions. Mystique does not require users to learn a new language or framework. It only expects the ability to understand the chart.

% Figure environment removed% <<<

With the treemap bar charts loaded, the user sees the automatically detected axis and legend information including labels and data types in the result panel~(\cref{fig:legend_axis_UI}). Since Mystique does not guarantee 100\% accuracy, it allows a user to easily fix potential detection errors through simple interactions. Once the user has verified the detection results, Mystique analyzes the semantic structure of the main chart content, synthesizes data requirements as well as chart reuse steps, and then prepares a wizard interface that guides the users to reuse the example.



% The Mystique interface consists of six components~(\cref{fig:reuseUI}). The Canvas (1) initially shows the chart, highlights different parts of the visualization at each step to solicit user specification, and updates the visualization based on the user's operation during the reuse process. The Reference Panel (2) on the bottom-left contains the original input SVG, allowing the user to inspect it. The Dataset Compatibility Panel~(3) shows a guideline on the minimum number of categorical and quantitative data columns to reuse the example, and allows the user to see a sample dataset~(\cref{fig:sample_data}) and upload new datasets. The Step Indicator (4) right under the Dataset Compatibility Panel presents the ordered steps needed to reuse the example, and highlights the steps that have been completed so far; it also allows the user to go back to previous steps to change any mapping they have made.
% In the Instruction Panel (5), the user chooses the visual channel and data field for mapping through drop-down menus at each step. 
% At the bottom, the Data Table Panel (6) shows the new dataset table.

% Figure environment removed% <<<

\definecolor{blue11}{RGB}{46, 117, 182}
\newcommand*\bluecircle[1]{\tikz[baseline=(char.base)]{
\node[shape=circle,draw,inner sep=0pt, blue11,fill=blue11,text=white] (char) { {#1}}
}}

The user first sees the original chart displayed at the bottom left as a reference (\cref{fig:reuseUI} \bluecircle{2}).  The Canvas (\cref{fig:reuseUI} \bluecircle{1}) initially shows the same chart without any axis or legend information, so that the visual representation can be updated with new data. The interface also shows a guideline on the minimum number of categorical and quantitative data columns required to reuse the example (\cref{fig:reuseUI} \bluecircle{3}). The user can download a sample dataset~(\cref{fig:sample_data}) which helps them understand the expected data format. After the user uploads their own dataset about product sales, which is shown in the Data Table Panel (\cref{fig:reuseUI} \bluecircle{6}), Mystique generates an ordered set of steps to reuse the chart in the Step Indicator (\cref{fig:reuseUI} \bluecircle{4}), and highlights the steps that have been completed so far. At each step, the Canvas highlights different parts of the visualization to solicit user specification (\cref{fig:reuseUI} \bluecircle{1}), and the user chooses the visual channel and data field for mapping through drop-down menus (\cref{fig:reuseUI} \bluecircle{5}). The visualization updates based on the user's operation at every step. The user can always go back to previous steps by clicking the Back button in the Step Indicator.

Mystique starts by asking what the highlighted highest-level group in the canvas should represent in the new dataset~(\cref{fig:reuseUI} \bluecircle{5}). Everything but this group fades into the background. The user selects a value from the ``Category'' attribute, resulting in three highest-level groups with labels from the ``Category'' attribute shown in~\cref{fig:reuseUI} \bluecircle{1}. The user clicks on the ``Next'' button in the Step Indicator to proceed. \revise{Mystique then highlights the first treemap bar and asks what this subgroup should represent~(\cref{fig:step2}); similar to the previous step, the user selects a value from the ``Subcategory'' attribute, which leads to two subgroups updated with labels~(\textit{Bookcases} and \textit{Chairs}) from the ``Subcategory'' attribute.
Mystique next highlights the yellow-colored group of rectangles in the first treemap bar and asks what it should represent~(\cref{fig:step3}); the user selects a value from the ``Region'' attribute, resulting in four packed groups of rectangles~(previously there are six) of distinct colors that correspond to four regions within each product subcategory.
In the following step, Mystique highlights the first rectangle and asks what it should represent~(\cref{fig:step4}); the user selects a value from the ``Order ID'' attribute, resulting in the system populating rectangles, one for each distinct order ID.}
% In the following two steps, Mystique highlights a yellow-colored group of rectangles (\cref{fig:step3}) and a rectangle (\cref{fig:step4}) respectively, and asks what each of these objects should represent in the new dataset. The user selects ``Region'' and ``Order ID'', respectively. 
After that, Mystique starts handling encodings. Mystique first guesses that the height of each treemap bar encodes ``Sales'', which is accepted by the user. Mystique continues to infer that the ``fill'' channel of each rectangle is mapped to ``Sales'', which is undesired, thus the user changes `Sales'' to ``Regions''. Mystique further suggests that the ``area'' channel of each rectangle is mapped to `Sales'', 
% and the user accepts this last step, 
resulting in the final visualization shown in full opacity~(\cref{fig:step8}). The user can export the chart into Data Illustrator~\cite{liu_data_2018} for further customization such as axis range and color scheme. 

% \bpstart{Identifying Axis and Legend Information} 
% Mystique first detects the axes and legend in a chart. With the detected elements, it infers corresponding data encoding information such as axis field type and color mapping, and records the visual styles of the axis and legend (e.g., stroke color and stroke width) for reuse in the final visualization. 

% % Figure environment removed

% Mystique performs this step first so that rectangles in the axes and legend can be separated from the main chart content, and the inferred field types and color mappings can be used for deconstructing the chart content. Mystique utilizes a heuristics-based detection method and achieves high accuracy on our dataset~(\cref{result4axisLegendHeuristics}).
% Because our detection method does not guarantee 100\% accuracy, Mystique provides an interface for the user to verify detection results and correct errors through simple interactions. \Cref{fig:legend_axis_UI} shows that Mystique correctly extracts the y-axis labels and the first level of the x-axis, but misses the higher level axis labels (``1985'', ``1995'', ``2005'', and ``2015'') and misclassifies the y-axis label type~(should be ``numbers'' instead of ``categories''). Users can drag and drop the missing labels directly from the chart to the results region to fix this error, and select the correct label type from the drop-down list. \Cref{sec:correctUI} presents more details on the detection errors and how to fix them in the Mystique UI.

% % Figure environment removed


% \bpstart{Deconstructing Main Chart Content} After the axis and legend are identified and corrected, Mystique analyzes the semantic structure of the main chart content.
% One potential alternative was to classify the examples 
% based on a fine-grained taxonomy with categories like waffle chart or bullet chart. We decided not to take this approach for three reasons. First, we observed that multiple examples do not clearly fit into a certain chart type. For example, the chart in \cref{fig:legend_axis_UI} integrates elements from a treemap and a grouped bar chart into a single design. Second, many examples have nested structures~(e.g., small multiples) and require deconstruction into multiple instances of the same chart type.
% Finally, knowing the fine-grained chart type is still not enough to reuse the example charts; for instance, given a stacked bar chart, we still need to obtain information such as the location of each stacked group, the orientation of the stack layout, and the distance between the groups. 
% Therefore, we decided to deconstruct a chart into the following four types of semantic components: \textit{groups} (G), \textit{spatial relationships} (R), \textit{encodings} (E) and \textit{graphical constraints} (C):

% % \vspace{-1mm}
% \begin{itemize}
%     \setlength{\itemsep}{2pt}
%     \setlength{\parskip}{0pt}
%     \setlength{\parsep}{0pt}
%     \item \textbf{Groups (G)} refers to the hierarchical clustering of rectangles that reflects the semantic structure of the visualization. We detect two kinds of groups: \textit{collections} and \textit{glyphs}. 
%     \item \revise{\textbf{Spatial Relationships (R)} estimate the relative placement and organization of same-level rectangles or groups.
%     Mystique currently supports three types of relationships: grid, stack, and packing. \Cref{sec:GLEC} provides more details.}
%     \item \textbf{Encodings (E)} specify the mapping between data attributes and visual properties of rectangles or groups.
%     \item \textbf{Graphical Constraints (C)} enforce requirements (e.g., data-related alignment) on the spatial arrangements of rectangles or groups. For example, in \cref{fig:teaser}(d), all the gray rectangles representing ``Neither agree nor disagree'' are aligned in the center.
% \end{itemize}
% % \vspace{-2mm}

% Consider the treemap grouped bar chart in 
% \cref{fig:legend_axis_UI}, showing total trade values grouped by continents~(indicated by color), Imports/Exports, and four different years. Excluding the axes and grid lines, the main chart area can be described as 4 high-level \textit{groups} (G), each representing a year, arranged in a \textit{grid \revise{relationship} with 1 row and 4 columns} (R). Within each group, two \textit{sub-groups} (G) of rectangles, representing Imports and Exports, are arranged in a \textit{grid \revise{relationship} with 1 row and 2 columns} (R). Within each subgroup, six sub-subgroups (G) in distinct colors are arranged in a \textit{packing \revise{relationship}} (R), each of which is composed of rectangles arranged in a \textit{packing \revise{relationship}} (R). The \textit{height} of the bars representing Imports or Exports \textit{encodes} the \textit{total trade value} (E), the \textit{area} of the rectangles \textit{encodes} the \textit{trade value} (E) for a certain country, and the \textit{color} of the rectangles \textit{encodes} the \textit{continent} (E). In this example, there are no special graphical constraints.

% These four semantic components are based on the Atlas visualization framework \cite{liu_atlas_2021} and correspond to the outputs of four grammatical rules:  glyph-generation, graphics-data join, visual encoding, and spatial arrangement. They can describe all the examples in our collection (See \cref{sec:chartDecomp} for details about this four-component model).

% \bpstart{Sample Data and Compatibility Checking} Mystique provides two functionalities to ensure that the user's dataset is compatible with the example. First, it automatically generates a sample dataset based on the deconstructed chart content. The sample dataset shows exemplary columns and values to convey to the user the desired number and types of data fields (\cref{fig:sample_data}). 
% This information is also summarized in the Dataset Compatibility Panel (component 3 in \cref{fig:reuseUI}). Second, when the user uploads a dataset by clicking on the ``Import Data'' button, Mystique performs a compatibility check and issues a warning if the imported data does not contain the required columns.

% \bpstart{Reusing Example with a Wizard Assistant} Finally, based on the deconstruction results obtained from the previous step, Mystique provides a wizard interface that guides the user to reuse the example through a series of steps.
% The interface comprises six components~(\cref{fig:reuseUI}). The Canvas (1) initially shows the chart, highlights different parts of the visualization at each step to solicit user specification, and updates the visualization based on the user's operation at every step during the reuse process. The Reference Panel (2) on the bottom-left contains the original input SVG, allowing the user to inspect it at any time. Directly to its right, the Dataset Compatibility Panel~(3) shows a guideline on the minimum number of categorical and quantitative data columns to reuse the example, and provides two buttons for the user to see a sample dataset and upload their dataset. The Step Indicator (4) right under the Dataset Compatibility Panel presents the ordered steps needed to reuse the example and create a new chart, and highlights the steps that have been completed so far;
% it also allows the user to go back to previous steps to change any mapping they have made.
% In the Instruction Panel (5), the user chooses the visual channel and data field for mapping through drop-down menus at each step. 
% At the bottom, the Data Table Panel (6) shows all the rows and columns of the new dataset.

% % Figure environment removed


% Suppose the user wants to reuse the treemap grouped bar chart on a dataset about orders of a superstore. The data record information about each order: order ID, product category and subcategory, geographic region of the store, and sales from the order. The new visualization would show the sales values for each order colored by region in a packing layout grouped by product category and subcategory.



% \revise{Mystique starts by asking what the highlighted highest-level group in the canvas should represent in the new dataset~(\cref{fig:reuseUI}). Everything but this group fades into the background. The user selects a value from the ``Category'' attribute in the new dataset, resulting in three highest-level groups with labels from the ``Category'' attribute. The user then clicks on the ``Next'' button in the Step Indicator to proceed. Mystique subsequently highlights a subgroup (\cref{fig:reuseExample}(a)), a yellow-colored group of rectangles, as well as an individual rectangle (\cref{fig:reuseExample}(b)), and asks what each of these objects should represent in the new dataset. The user selects a ``subcategory'', a ``Region'', and an ``Order ID'' respectively.}
% \revise{After that, Mystique starts handling encodings. First, Mystique guesses that the height of each subgroup bar encodes ``Sales'' since it is the only numerical attribute in the dataset, and generates a y-axis and grid lines. The user reviews the result and accepts this step. Mystique continues to infer encodings for rectangle marks. It guesses that the ``fill'' channel of each rectangle is mapped to the `Sales'' field; The user changes `Sales'' into ``Regions'', which gives the results in~\cref{fig:reuseExample}(c)). Mystique further suggests that the ``area'' channel of each rectangle is mapped to the `Sales'' attribute and renders the corresponding output; the user accepts this last step, leading to the final chart shown in full opacity~(\cref{fig:reuseExample}(d)). Mystique finishes all steps and tells the user the reuse process is done. The user can export the chart and open it in Data Illustrator \cite{liu_data_2018} to further make customization in terms of layout and visual style.}