\section{Axis and Legend Detection}
\label{sec:axisLegend}
% \subsection{Heuristics-based Detection}\label{sec:heuristic4axis_legend}

We develop heuristics to detect the axes and legend in a chart based on the relative positioning of marks, texts, and lines. For example, an axis area typically consists of a set of text labels, a set of small ticks close to corresponding labels, and an axis line spanning the vertical or horizontal range of the ticks; a legend area can be either a set of horizontal or vertical [mark, text] pairs~(the discrete case) or a gradient-colored bar associated with ticks and numbers~(the continuous case). \markup{The accuracy of the heuristics is reported in~\cref{quantatitive}, and the supplementary materials\footnote{All the supplementary materials are available at \url{https://osf.io/pt3yq/?view_only=76280bd6d25f47b199280d79446ac34e}} contains more details}.

% \subsection{User Interface to Fix Detection Errors}\label{sec:correctUI}
The heuristics don't guarantee 100\% accuracy, thus we further build a user interface in Mystique to allow fixing potential errors through simple interactions.
The user interface consists of two areas: the chart area displaying the original SVG example, and the result panel under the chart area (\cref{fig:legend_axis_UI}). The result panel includes three sub-panels: x-axis, y-axis, and legend, showing the extracted labels respectively.
The legend labels' background colors indicate the color mappings extracted from the legend. Mystique also infers the data types for the x- and y-axis, which are displayed in the form of a drop-down menu. 

Five kinds of detection mistakes have been observed: missing some labels~(M1), false-positive axis labels~(M2), missing higher-level labels~(M3), missing axis or legend~(M4), false-positive axis or legend~(M5). To this end, we enable the following features to let users fix them: 
(1) drag-and-drop over chart texts into or out of the result sub-panels~(M1,2,3); (2) the % Figure removed buttons right to the label display boxes that add display boxes for higher-level labels~(M3); (3) the % Figure removed buttons left to the label display boxes to activate a region select tool for users to select an area that covers the missing axis or legend~(M4); (4) the drop-down selections left to the label display boxes which allow users to remove false-positive axis or legend~(set to ``none'') and modify axis label data type~(M5).

% \bpstart{Axis Detection} 
% An axis typically consists of labels, ticks, and an axis path (the latter two are optional). Our approach first detects labels, and then finds ticks and path based on  element positions and SVG hierarchy.
% For instance, to detect the x-axis, we first search for a set of same-size \textit{text} elements that share the same SVG hierarchy level and the same $y$ coordinate to be the x-axis labels; after that, within the hierarchy the labels reside in, we recursively search for a horizontal element~(\textit{line}, \textit{path} or \textit{rect}) close to the labels, whose width is no shorter than the distance spanned by the labels to be the axis path. A set of vertical short \textit{line} or \textit{path} elements close to the labels that share the same $y$ coordinate is identified as the axis ticks.

% \bpstart{Legend Detection} 
% In general, there are two types of color legend: discrete mapping and continuous mapping. 
% For a discrete mapping, the legend contains a set of (\textit{mark}, \textit{text}) pairs where \textit{mark} can be \textit{rect} or \textit{circle}, and the legend orientation can be horizontal or vertical. To find a horizontal legend, Mystique
% first searches for marks based on the following criteria: same size, different fill colors, and same $y$ coordinate. It then looks for \textit{text} elements in between each pair of neighboring marks and after the rightmost mark. 
% \revise{The detection of a vertical legend follows the same approach with the opposite orientation.}

% When a continuous color mapping is present, the legend typically contains a rectangle whose fill color is a gradient function, with associated ticks and labels. For such a legend area, Mystique first searches for a rectangle with a gradient fill color to be the legend bar. Based on the orientation of the rectangle~(width$>$height or height$>$width), it further looks for a set of same-size \textit{text} elements and an optional set of short \textit{line} or \textit{path} elements to be the labels and ticks, respectively. The label and tick sets should satisfy the following conditions: (1) close to the legend bar and (2) elements within each set share the same $y$ or $x$ coordinate and span the width or height of the legend bar. 