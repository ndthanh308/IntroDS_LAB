%Version 2.1 April 2023
% See section 11 of the User Manual for version history
%
%%%%%%%%%%%%%%%%%%%%%%%%%%%%%%%%%%%%%%%%%%%%%%%%%%%%%%%%%%%%%%%%%%%%%%
%%                                                                 %%
%% Please do not use \input{...} to include other tex files.       %%
%% Submit your LaTeX manuscript as one .tex document.              %%
%%                                                                 %%
%% All additional figures and files should be attached             %%
%% separately and not embedded in the \TeX\ document itself.       %%
%%                                                                 %%
%%%%%%%%%%%%%%%%%%%%%%%%%%%%%%%%%%%%%%%%%%%%%%%%%%%%%%%%%%%%%%%%%%%%%

%%\documentclass[referee,sn-basic]{sn-jnl}% referee option is meant for double line spacing

%%=======================================================%%
%% to print line numbers in the margin use lineno option %%
%%=======================================================%%

%%\documentclass[lineno,sn-basic]{sn-jnl}% Basic Springer Nature Reference Style/Chemistry Reference Style

%%======================================================%%
%% to compile with pdflatex/xelatex use pdflatex option %%
%%======================================================%%

%%\documentclass[pdflatex,sn-basic]{sn-jnl}% Basic Springer Nature Reference Style/Chemistry Reference Style


%%Note: the following reference styles support Namedate and Numbered referencing. By default the style follows the most common style. To switch between the options you can add or remove Numbered in the optional parenthesis. 
%%The option is available for: sn-basic.bst, sn-vancouver.bst, sn-chicago.bst, sn-mathphys.bst. %  
%\documentclass[pdflatex,iicol]{sn-jnl}
\documentclass[sn-nature]{sn-jnl}% Style for submissions to Nature Portfolio journals
%%\documentclass[sn-basic]{sn-jnl}% Basic Springer Nature Reference Style/Chemistry Reference Style
%%\documentclass[sn-mathphys,Numbered]{sn-jnl}% Math and Physical Sciences Reference Style
%%\documentclass[sn-aps]{sn-jnl}% American Physical Society (APS) Reference Style
%%\documentclass[sn-vancouver,Numbered]{sn-jnl}% Vancouver Reference Style
%%\documentclass[sn-apa]{sn-jnl}% APA Reference Style 
%%\documentclass[sn-chicago]{sn-jnl}% Chicago-based Humanities Reference Style
%%\documentclass[default]{sn-jnl}% Default
%%\documentclass[default,iicol]{sn-jnl}% Default with double column layout

%%%% Standard Packages
%%<additional latex packages if required can be included here>

\usepackage[vietnamese, english]{babel}
\usepackage{braket}%
\usepackage{graphicx}%
\usepackage{multirow}%
\usepackage{amsmath,amssymb,amsfonts}%
\usepackage{amsthm}%
\usepackage{mathrsfs}%
\usepackage[title]{appendix}%
\usepackage{xcolor}%
\usepackage{textcomp}%
\usepackage{manyfoot}%
\usepackage{booktabs}%
\usepackage{algorithm}%
\usepackage{algorithmicx}%
\usepackage{algpseudocode}%
\usepackage{listings}%
%%%%

%%%%%=============================================================================%%%%
%%%%  Remarks: This template is provided to aid authors with the preparation
%%%%  of original research articles intended for submission to journals published 
%%%%  by Springer Nature. The guidance has been prepared in partnership with 
%%%%  production teams to conform to Springer Nature technical requirements. 
%%%%  Editorial and presentation requirements differ among journal portfolios and 
%%%%  research disciplines. You may find sections in this template are irrelevant 
%%%%  to your work and are empowered to omit any such section if allowed by the 
%%%%  journal you intend to submit to. The submission guidelines and policies 
%%%%  of the journal take precedence. A detailed User Manual is available in the 
%%%%  template package for technical guidance.
%%%%%=============================================================================%%%%

%\jyear{2021}%

%% My commands
\addto\captionsenglish{\renewcommand{\figurename}{Supplementary Figure}}
\addto\captionsenglish{\renewcommand{\tablename}{Supplementary Table}}
\addto\captionsenglish{\renewcommand{\refname}{Supplementary References}}

%%


%% as per the requirement new theorem styles can be included as shown below
\theoremstyle{thmstyleone}%
\newtheorem{theorem}{Theorem}%  meant for continuous numbers
%%\newtheorem{theorem}{Theorem}[section]% meant for sectionwise numbers
%% optional argument [theorem] produces theorem numbering sequence instead of independent numbers for Proposition
\newtheorem{proposition}[theorem]{Proposition}% 
%%\newtheorem{proposition}{Proposition}% to get separate numbers for theorem and proposition etc.

\theoremstyle{thmstyletwo}%
\newtheorem{example}{Example}%
\newtheorem{remark}{Remark}%

\theoremstyle{thmstylethree}%
\newtheorem{definition}{Definition}%

\raggedbottom
%%\unnumbered% uncomment this for unnumbered level heads

\begin{document}

\title[Supplementary]{Supplementary Information}

%%=============================================================%%
%% Prefix	-> \pfx{Dr}
%% GivenName	-> \fnm{Joergen W.}
%% Particle	-> \spfx{van der} -> surname prefix
%% FamilyName	-> \sur{Ploeg}
%% Suffix	-> \sfx{IV}
%% NatureName	-> \tanm{Poet Laureate} -> Title after name
%% Degrees	-> \dgr{MSc, PhD}
%% \author*[1,2]{\pfx{Dr} \fnm{Joergen W.} \spfx{van der} \sur{Ploeg} \sfx{IV} \tanm{Poet Laureate} 
%%                 \dgr{MSc, PhD}}\email{iauthor@gmail.com}
%%=============================================================%%

\author*[1]{\fnm{Simone} \sur{Traverso}}\email{simone.traverso@edu.unige.it}

\author[1,2]{\fnm{Maura} \sur{Sassetti}}\email{sassetti@fisica.unige.it}

\author[1,2]{\fnm{Niccol\`o Traverso} \sur{Ziani}}\email{traversoziani@fisica.unige.it}

\affil*[1]{\orgdiv{Physics Department}, \orgname{University of Genoa}, \orgaddress{\street{Via Dodecaneso 33}, \city{Genova}, \postcode{16146}, \country{Italia}}}

\affil[2]{\orgname{CNR-SPIN}, \orgaddress{\street{Via Dodecaneso 33}, \city{Genova}, \postcode{16146}, \country{Italia}}}

\keywords{topological insulators, Haldane model, bound states, Zak phase}

%%\pacs[JEL Classification]{D8, H51}

%%\pacs[MSC Classification]{35A01, 65L10, 65L12, 65L20, 65L70}

\maketitle

\section*{Supplementary Figures}
% Figure environment removed

% Figure environment removed

% Figure environment removed

% Figure environment removed

% Figure environment removed

% Figure environment removed


% Figure environment removed

% Figure environment removed

% Figure environment removed
\clearpage

\section*{Supplementary Tables}
\begin{table}[h]
    \centering
    \begin{tabular}{c|c|c|}
        $N_y$ & $\varphi^{m<0}$& $\varphi^{m>0}$ \\
        \hline
        $8\ell$ & $0$ & $0$ \\
        $8\ell+2$ & $0$ & $\pi$ \\
        $8\ell+4$ & $\pi $& $\pi$ \\
        $8\ell+6 $& $\pi$ & $0$ \\
        \hline
    \end{tabular}
    \caption{Zak Phase in the large $m$ limit.}
    \label{tab:tab1}
\end{table}
\clearpage

\section*{Supplementary Notes}
\subsection*{Supplementary Note 1}
Given the Hamiltonian with PBC along the $\mathbf{a}_1$ direction in Eq.~(4) of the main text, the corresponding Schr\"odinger equation
\begin{equation}
    \mathcal{H}(k) \Psi = \epsilon \Psi,
\end{equation}
can be recast in a more explicit way by denoting
\begin{equation}
    \Psi_{2n-1}(k)= \psi_n^A(k), \ \Psi_{2n}(k)= \psi_n^B(k), \qquad n \in \{1,\ldots,N_y/2\}.
\end{equation}
In this notation we can can synthetically write 
\begin{align*}
    \varepsilon\psi_n^A(k) &= \bigg\{g_0\psi_n^B(k)+t_1\psi_{n-1}^B(k)+(g(k,\phi)+m)\psi_n^A(k)+g(k/2,-\phi)\big[\psi_{n+1}^A(k)+\psi_{n-1}^A(k)\big]\bigg\},\\
    \varepsilon\psi_n^B(k) &= \bigg\{g_0\psi_n^A(k)+t_1\psi_{n+1}^A(k)+(g(k,-\phi)-m)\psi_n^B(k)+g(k/2,\phi)\big[\psi_{n+1}^B(k)+\psi_{n-1}^B(k)\big]\bigg\},
\end{align*}
where $g(k,\phi) = 2t_2\cos(k+\phi)$ and $g_0=2t_1\cos(k/2)$. The above equations are to be read assuming that $\psi_{0}^{A/B}(k)=0=\psi_{\Tilde{N}+1}^{A/B}(k), \ \Tilde{N}=N_y/2$, because of open boundary conditions in the $y$ direction. With $\phi = \pi/2$ we get: $g(k,\pm \pi/2) = \mp 2t_2\sin(k)$. Thus, after renormalizing every constant in units of $t_1$, we have
{\small
\begin{align*}
    \varepsilon\psi_n^A(k) &= \bigg\{2\cos(k/2)\psi_n^B(k)+\psi_{n-1}^B(k)+(-2t_2\sin(k)+m)\psi_n^A(k)+2t_2\sin(k/2)\big[\psi_{n+1}^A(k)+\psi_{n-1}^A(k)\big]\bigg\},\\
    \varepsilon\psi_n^B(k) &= \bigg\{2\cos(k/2)\psi_n^A(k)+\psi_{n+1}^A(k)+(2t_2\sin(k)-m)\psi_n^B(k)-2t_2\sin(k/2)\big[\psi_{n+1}^B(k)+\psi_{n-1}^B(k)\big]\bigg\}.
\end{align*}
}
What we want to prove here is that for $\Tilde{N}=2M+1$ (\textit{i.e.} $N_y=4M+2$), $M\in \mathbb N$, and $m=0$, the dispersion relation is always metallic. By setting $k=\pi$ we get:
\begin{align*}
    \varepsilon\psi_n^A(\pi) &= \bigg\{\psi_{n-1}^B(\pi)+2t_2\big[\psi_{n+1}^A(\pi)+\psi_{n-1}^A(\pi)\big]\bigg\},\\
    \varepsilon\psi_n^B(\pi) &= \bigg\{\psi_{n+1}^A(\pi)-2t_2\big[\psi_{n+1}^B(\pi)+\psi_{n-1}^B(\pi)\big]\bigg\}.
\end{align*}
We now start looking for a zero energy solution. Enforcing $\varepsilon=0$, the above equations reduce to
\begin{align*}
    0&= \psi_{n-1}^B+2t_2\big[\psi_{n+1}^A+\psi_{n-1}^A\big],\\
    0 &= \psi_{n+1}^A-2t_2\big[\psi_{n+1}^B+\psi_{n-1}^B\big],
\end{align*}
where, to simplify the notation, we have omitted the fact that the $\psi_n^{A/B}(k)$ are calculated for $k=\pi$.

Imposing the boundary condition $\psi_{0}^{A/B}(k)=0$ we get
\begin{align*}
    0 &= 2t_2\psi_{2}^A,\\
    0 &= \psi_{2}^A-2t_2\psi_{2}^B, \\
    0 &= \psi_{1}^B+2t_2\big[\psi_{3}^A+\psi_{1}^A\big],\\
    0 &= \psi_{3}^A-2t_2\big[\psi_{3}^B+\psi_{1}^B\big],
    \\
    &\vdots
\end{align*}
so that in the end $\psi_{2\ell}^{A/B}=0$ for any $\ell \in \mathbb N$. On the other hand, the boundary condition $\psi_{\Tilde{N}+1}^{A/B}(k)=0$ imposes no more conditions if $\Tilde{N}=2M+1$, since then $\Tilde{N}+1$ is even, while for $\Tilde{N}=2M$ it forces $\psi_{2\ell+1}^{A/B}=0$ for any $\ell \in \mathbb N$ as well. Thus we conclude that if $N_y=4M$ no zero energy solution exists for $m=0$. Note that the dispersion relation can only be gapless at $k=\pi$, since that is the position of the Dirac point for the edge states of the Haldane model on a strip with zigzag edges (see Fig.~1\textbf{b} of the main text).

For the $\Tilde{N}=2M+1$ case we look for a solution of the following kind:
\begin{align*}
    \psi_{2\ell}^{A/B} &=0, \\
    \psi_{2\ell+1}^{A/B} &= \xi^\ell \psi_{1}^{A/B}.
\end{align*}
The first condition is coherent with the boundary conditions, as already discussed. About the second one, by substituting $n=2\ell$ into the equations one gets
\begin{align*}
    0&= \psi_{2\ell-1}^B+2t_2\big[\psi_{2\ell+1}^A+\psi_{2\ell-1}^A\big]=\xi^{\ell-1}\psi_{1}^B+2t_2\big[\xi^{\ell}\psi_{1}^A+\xi^{\ell-1}\psi_{1}^A\big],\\
    0 &= \psi_{2\ell+1}^A-2t_2\big[\psi_{2\ell+1}^B+\psi_{2\ell-1}^B\big]=\xi^{\ell}\psi_{1}^A-2t_2\big[\xi^{\ell}\psi_{1}^B+\xi^{\ell-1}\psi_{1}^B\big],
\end{align*}
which upon dividing by $\xi^{\ell-1}$ yield
\begin{align*}
    0&=\psi_{1}^B+2t_2\big[\xi\psi_{1}^A+\psi_{1}^A\big],\\
    0 &= \xi\psi_{1}^A-2t_2\big[\xi\psi_{1}^B+\psi_{1}^B\big].
\end{align*}
Solving the system we get
\begin{equation}
    \psi_1^B=\dfrac{\xi}{2t_2(1+\xi)}\psi_1^A \implies \dfrac{\xi}{2t_2(1+\xi)}+2t_2(1+\xi)=0 \implies \xi + 4t_2^2(1+\xi)^2=0.
\end{equation}
Solving the equation for $\xi$ we get:
\begin{equation}
    \xi^2 + \xi (1/(4t_2^2) + 2) + 1 = 0 \implies \xi_{\pm} = - (1/(8t_2^2) + 1) \pm \sqrt{(1/(8t_2^2) + 1)^2-1}.
\end{equation}
Thus, we find two solutions with zero energy at $k=\pi$ for $m=0$ in the $N_y=4M+2$ case, as claimed in the main text. Finally, we observe that the product of the solutions is $1$, meaning that they have the same sign, and their sum equals $-(1/(4t_2^2) + 2)$,  meaning that they are both negative, one bigger and one smaller then unity in modulus respectively. These two degenerate solutions correspond to a state localized on the upper edge and to a state localized on the lower one, as expected.

\subsection*{Supplementary Note 2}
Let $N_y$ be the number of sites in the vertical direction (or, equivalently in the $\vec{a}_2$ direction).
In the large $m$ limit ($m\gg t_1,t_2$), the Hamiltonian can be brutally approximated as 
\begin{equation}
    \mathcal{H}(k)= \mathrm{diag}(m,-m,m,-m,\ldots).
\end{equation}
If the system is at half filling, the occupied states will be those localized on the sublattice with negative value of the on-site potential, \textit{i.e.} the $B$ sublattice for $m>0$ and the $A$ sublattice for $m<0$.

Let us start by considering $m>0$. For each $k \in \{0,\frac{2\pi}{Na}, \ldots,\frac{2\pi}{Na}(N-1)\}$ a basis of eigenstates for the $N_y/2\ (\equiv \Bar{n})$  occupied bands is given by
\begin{equation}
    \Psi^{m>0}_{1_\ell}(k) = \delta_{2,\ell},\ \ldots, \ \Psi^{m>0}_{\Bar{n}_\ell}(k) = \delta_{2\Bar{n},\ell}, \qquad \ell \in 1,\ldots,N_y,
\end{equation}
where the $\Psi_n(k)$ are $N_y$-dimensional vectors and the index $\ell$ refers to their components.
Thus, for $j=0,\ldots, N-2$, the $S$ matrix is trivially given by
\begin{equation}
    S^{m>0}(k_j,k_{j+1})= \mathbb I_{\Bar{n}},
\end{equation}
so that $\det[S^{m>0}(k_j,k_{j+1})]=+1 \ \forall j \in \{0,\ldots, N-2\}$.
However, from the periodic gauge enforcing,
\begin{equation}
    \Psi^{m>0}_{n_\ell}(k_N) = \mathrm{e}^{-\mathrm{i}\frac{2\pi}{a}t_\ell}  \Psi^{m>0}_{n_\ell}(k_0),
\end{equation}
where $t_\ell$ is the $x$-position of the $\ell$-th site in the strip unit cell. With our conventions (see main text) we get
\begin{align*}
    \mathrm{e}^{-\mathrm{i}\frac{2\pi}{a}t_{4r}} &= 1, \\
    \mathrm{e}^{-\mathrm{i}\frac{2\pi}{a}t_{4r+1}} &= 1, \\
    \mathrm{e}^{-\mathrm{i}\frac{2\pi}{a}t_{4r+2}} &= -1, \\
    \mathrm{e}^{-\mathrm{i}\frac{2\pi}{a}t_{4r+3}} &= -1.
\end{align*}
Thus, for $m>0$ we get
\begin{align*}
    \Psi^{m>0}_{2n_\ell}(k_N) &= \delta_{4n,\ell}, \\
    \Psi^{m>0}_{{2n-1}_\ell}(k_N) &= -\delta_{4n-2,\ell},
\end{align*}
or, equivalently
\begin{equation}
    \Psi^{m>0}_{n_\ell}(k_N) = (-1)^n \delta_{2n,\ell}.
\end{equation}
Recalling that at half filling $N_y=2\Bar{n}$, we have
\begin{equation}
    [S^{m>0}(k_{N-1},k_{N})]_{rs}= \Psi^{m>0\dagger}_r(k_{N-1})\Psi^{m>0}_s(k_N) = (-1)^s \sum_{\ell=1}^{N_y}\delta_{2r,\ell}\delta_{\ell, 2s}= (-1)^{r}\delta_{rs},
\end{equation}
and therefore the determinant turns out to depend on the value of $\Bar{n}$
\begin{equation}
    \det[S^{m>0}(k_{N-1},k_{N})]= 
    \begin{cases}
        +1 &\Bar{n}= 4\ell \\
        -1 &\Bar{n} = 4\ell +1 \\
        -1 &\Bar{n} = 4\ell +2 \\
        +1 &\Bar{n} = 4\ell + 3 \\
    \end{cases}
\end{equation}

We now turn to the $m<0$ case.
For each $k \in \{0,\frac{2\pi}{Na}, \ldots,\frac{2\pi}{Na}(N-1)\}$ a basis of eigenstates of the the $N_y/2\ (\equiv \Bar{n})$ occupied bands is now given by
\begin{equation}
    \Psi^{m<0}_{1_\ell}(k) = \delta_{1,\ell},\ \ldots, \ \Psi^{m<0}_{\Bar{n}_\ell}(k) = \delta_{2\Bar{n}-1,\ell}, \qquad \ell \in 1,\ldots,N_y.
\end{equation}
Again for $j=0,\ldots, N-2$, the $S$ matrix is trivially given by
\begin{equation}
    S^{m<0}(k_j,k_{j+1})= \mathbb I_{\Bar{n}},
\end{equation}
so that $\det[S^{m<0}(k_j,k_{j+1})]=+1 \ \forall j \in \{0,\ldots, N-2\}$. However, enforcing the periodic gauge for $m<0$ yields ($\Psi^{m<0}_{n_\ell}(k)=\delta_{2n-1,\ell}$)
\begin{align*}
    \Psi^{m<0}_{2n_\ell}(k_N) &= -\delta_{4n-1,\ell}, \\
    \Psi^{m<0}_{{2n-1}_\ell}(k_N) &= +\delta_{4n-3,\ell},
\end{align*}
or, equivalently
\begin{equation}
    \Psi^{m<0}_{n_\ell}(k_N) = (-1)^{n-1} \delta_{2n-1,\ell}.
\end{equation}
Then, proceeding as in the $m>0$ case
\begin{equation}
    [S^{m<0}(k_{N-1},k_{N})]_{rs} = (-1)^{r-1}\delta_{rs} \implies \det[S(k_{N-1},k_{N})]= 
    \begin{cases}
        +1 &\Bar{n}= 4\ell \\
        +1 &\Bar{n} = 4\ell +1 \\
        -1 &\Bar{n} = 4\ell +2 \\
        -1 &\Bar{n} = 4\ell + 3 \\
    \end{cases}
\end{equation}

We have found that in the large $m$ limit the sign of  $\prod_{j=0}^{N-1} \det[S(k_{j},k_{j+1})] $ is entirely determined by the sign of $\det[S(k_{N-1},k_{N})]$. Knowing that, we can easily determine the Zak phase, which is given by
\[
    \varphi = -\Im \log \prod_{j=0}^{N-1} \det[S(k_{j},k_{j+1})], \qquad \varphi \in (-\pi,\pi].
\]
The resulting values of the Zak phase in the large (positive and negative) $m$ limit are reported as a function of $N_y$ ($N_{y}=2\Bar{n}$) in Supplementary Table~\ref{tab:tab1}. In particular, by inspecting Supplementary Table~\ref{tab:tab1} we conclude that for $N_y=4M+2$ the Zak phase differs of $\pi$ between the $m\to \pm \infty$ limits, while this is not the case for $N_y=4M$.

\subsection*{Supplementary Note 3}
In order to test the robustness of the bound states, we assess their stability against the introduction of random on-site disorder of the following form
\begin{equation}
    \mathcal{H}_\text{RN} = \sum_{i} \omega_i c_i^\dagger c_i,
\end{equation}
where the sum runs over all lattice sites, the operator $c_i$ destroys a fermion at site $i$ and $\omega_i$ is a real number randomly extracted from the interval $[-V,V]$, $V>0$. In Supplementary Figure~\ref{fig:noise_N_2} we compare two Haldane zigzag nanoribbons (OBC geometry) with $N_y=4$ and $L=20a$ without (Panels \textbf{a}, \textbf{c}, \textbf{e}, \textbf{g}) and with (Panels \textbf{b}, \textbf{d}, \textbf{f}, \textbf{h}) disorder respectively; an analogous comparison for a strip with $N_y=6$ and $L=40a$ is reported in Supplementary Figure~\ref{fig:noise_N_3}. The parameters of the Haldane model are set as in the main text to $t_1=1, \ t_2=0.3, \ \phi=\frac{\pi}{2}$, while the maximum intensity of the on-site disorder is chosen as $V=0.2$. The mass term is set to $m=0$ for the $N_y=4$ strip and to $m=0.5$ for the $N_y=6$ one, so that both systems are in their topological phase and host end states.

For both configurations, we find that the effect of the random on-site disorder is to break the degeneracy between the end states, which, however, survive to the perturbation.
In Panels \textbf{a} and \textbf{b} of Supplementary Figure~\ref{fig:noise_N_2} (Supplementary Figure~\ref{fig:noise_N_3}) we report a graphical representation of the on-site potential in absence and presence of the random noise term for the strip with $N_y=4$ ($N_y=6$). In Panels \textbf{c} and \textbf{d} of Supplementary Figure~\ref{fig:noise_N_2} (Supplementary Figure~\ref{fig:noise_N_3}) we compare the corresponding low energy spectra: even in the presence of noise, two eigenmodes (marked in red and orange) are present inside the energy gap. However, in contrast to the unperturbed case, the in-gap eigenvalues in Supplementary Figure~\ref{fig:noise_N_2}\textbf{d} (Supplementary Figure~\ref{fig:noise_N_3}\textbf{d}) are slightly split in energy. In Panels \textbf{e} and \textbf{g} of Supplementary Figure~\ref{fig:noise_N_2} (Supplementary Figure~\ref{fig:noise_N_3}) are reported the probability density plots of the eigenstates corresponding to the eigenvalues marked in red and orange in Panel \textbf{c}, while in Panels \textbf{f} and \textbf{h} of Supplementary Figure~\ref{fig:noise_N_2} (Supplementary Figure~\ref{fig:noise_N_3}) are reported the ones corresponding to the eigenvalues marked in red and orange in Panel \textbf{d}. By comparing them one can see that, as a consequence of the energy splitting, the two bound states, which in absence of noise are degenerate in energy and equally localized on both ends, localize themselves on one end of the strip each.

Similar results to those discussed here hold for strips of greater width or different random noise configurations. However, being that the splitting between the end states is due to the value of the on-site disorder at the two ends of the strip (\textit{i.e.} where the bound states are localized), the splitting of the eigenmodes (and so their localization pattern) may differ for different noise configurations.

\subsection*{Supplementary Note 4}
To derive an effective analytical expression for the coupling term $\mathcal{M}$ in Eq.~9 of the main text in the case $m=0$, we start by considering the coupling between two one dimensional chains, formed by the edge sites of the zigzag nanoribbon. We introduce an exponentially suppressed coupling between the two chains, of the form
\[
    t_{ij} = \mathrm{e}^{-d_{ij}/\xi},
\]
with $d_{ij}$ the distance between the site $i$ of one edge and the site $j$ of the other and $\xi$ some length scale.

Also, we set a sharp ``first-neighbor'' cutoff, effectively limiting the interaction to the edge sites that in the full strip are connected by the minimum amount of first neighbor hoppings (see Supplementary Figure~\ref{fig:eff_coupl_scheme}). According to this criterion, the number of coordination is easily found to be $n_c = \frac{N_y}{2}+1$. Thus, the interaction Hamiltonian is written as
\begin{align*}
    &N_y=4M: && H_{\text{int}} = \Tilde{\Delta}\sum_{\ell=1}^{N}\sum_{j=-(n_c-1)/2}^{(n_c-1)/2} \mathrm{e}^{-(\sqrt{w^2+(ja)^2}-w)/\xi} a^\dagger_\ell b_{\ell+j} + \text{h.c.}, \\
    &N_y=4M+2: && H_{\text{int}} = \Tilde{\Delta}\sum_{\ell =1}^{N}\sum_{j=-n_c/2}^{n_c/2-1} \mathrm{e}^{-(\sqrt{w^2+(ja+a/2)^2}-w)/\xi} a^\dagger_\ell b_{\ell+j} +\text{h.c.},
\end{align*}
where $\Tilde{\Delta}$ is some arbitrary energy scale and $w$ the strip width. The operator $a_\ell$ ($b_\ell$) destroys a fermion on the site $\ell $ of the lower (upper) edge, corresponding to a $A$ ($B$) site.
The strip width $w$ can be expressed in terms of $N_y$ as (first neighbor distance set to unity)
\[
    w = \dfrac{\sqrt{3}}{2}\dfrac{N_y}{2}-\dfrac{1}{\sqrt{3}}.
\]
About the length scale $\xi$, it is reasonable to assume that it should be of the same order of magnitude of the localization length of the edge states. This is given analytically for $k=\pi$ (assuming $m=0$ and $\phi=\frac{\pi}{2}$) in~\cite{Doh_2013, Cano_2013}
\[
    \xi_{\text{loc}}(\pi) = \dfrac{\sqrt{3}}{2}\left[\log\left\{\sqrt{1+\left(\frac{t_1}{4t_2}\right)^2} +\frac{t_1}{4t_2}\right\}\right]^{-1},
\]
and is shown to be more or less constant around the Dirac point and for small values of $m$. More specifically, since the coupling results from the interaction between the two opposite chiral modes, each of which has a decay length $\sim\xi_{\text{loc}}(\pi)$, we pick $\xi=2\xi_{\text{loc}}(\pi)$ for the plots in Figs.~5\textbf{a} to 5\textbf{d} of the main text.

By Fourier transforming we find the interacting Hamiltonian in $k$-space. First we rewrite expressing $n_c$ in terms of $M$:
\begin{align*}
    &N_y=4M: && H_{\text{int}} = \Tilde{\Delta}\sum_{\ell =1}^{N}\sum_{j=-M}^{M} \mathrm{e}^{-(\sqrt{w^2+(ja)^2}-w)/\xi} a^\dagger_\ell b_{\ell +j} + \text{h.c.}, \\
    &N_y=4M+2: && H_{\text{int}} = \Tilde{\Delta}\sum_{\ell =1}^{N}\sum_{j=-(M+1)}^{M} \mathrm{e}^{-(\sqrt{w^2+(ja+a/2)^2}-w)/\xi} a^\dagger_\ell b_{\ell+j} +\text{h.c.}
\end{align*}
The Fourier transformation is defined as
\begin{align}
    a_{\ell}&= \dfrac{1}{\sqrt{N}}\sum_{k} \mathrm{e}^{-\mathrm{i}k\ell a}a(k), \\
    b_{\ell}&= \dfrac{1}{\sqrt{N}}\sum_{k} \mathrm{e}^{-\mathrm{i}k(\ell a+\delta)}b(k) ,
\end{align}
with $\delta = 0$ if $N_y=4M$ and $\delta = 1/2$ if $N_y=4M+2$. Thus
\begin{align*}
    &N_y=4M: && H_{\text{int}} = \frac{\Tilde{\Delta}}{N}\sum_{\ell=1}^{N} \sum_{k,q}\sum_{j=-M}^{M} \mathrm{e}^{-(\sqrt{w^2+j^2}-w)/\xi} \mathrm{e}^{\mathrm{i}k\ell}\mathrm{e}^{-\mathrm{i}q(\ell+j)}a^\dagger(k)b(q) + \text{h.c.} \\
    &N_y=4M+2: && H_{\text{int}} = \frac{\Tilde{\Delta}}{N}\sum_{\ell=1}^{N}\sum_{k,q}\sum_{j=-(M+1)}^{M} \mathrm{e}^{-(\sqrt{w^2+(j+1/2)^2}-w)/\xi} \mathrm{e}^{\mathrm{i}k\ell}\mathrm{e}^{-\mathrm{i}q(\ell+j+1/2)} a^\dagger(k)b(q)+\text{h.c.}
\end{align*}
Summing over $\ell$ and $q$:
\begin{align*}
    &N_y=4M: && H_{\text{int}} = \Tilde{\Delta}\sum_{k}\sum_{j=-M}^{M} \mathrm{e}^{-(\sqrt{w^2+j^2}-w)/\xi} \mathrm{e}^{-\mathrm{i}kj}a^\dagger(k)b(k) + \text{h.c.} \\
    &N_y=4M+2: && H_{\text{int}} = \Tilde{\Delta}\sum_{k}\sum_{j=-(M+1)}^{M} \mathrm{e}^{-(\sqrt{w^2+(j+1/2)^2}-w)/\xi} \mathrm{e}^{-\mathrm{i}k(j+1/2)} a^\dagger(k)b(k) +\text{h.c.}
\end{align*}
Now we consider the $N_y=4M$ case first. We have
\[
    H_{\text{int}} = \Tilde{\Delta}\sum_{k} [1 + \sum_{j=1}^{M} 2\cos(kj)\mathrm{e}^{-(\sqrt{w^2+j^2}-w)/\xi}]a^\dagger(k)b(k) + \text{h.c.},
\]
yielding
\begin{equation}
    \mathcal{H}^{\text{edge}}_{\text{int}}(k)= \Tilde{\Delta} [1 + \sum_{j=1}^{M} 2\cos(kj)\mathrm{e}^{-(\sqrt{w^2+j^2}-w)/\xi}] \tau_x.
\end{equation}
Proceeding analogously for the $N_y=4M+2$ case
\[
    \begin{split}
        H_{\text{int}} &=\Tilde{\Delta}\sum_{k}\left\{\sum_{j=0}^{M} \mathrm{e}^{-(\sqrt{w^2+(j+1/2)^2}-w)/\xi} \mathrm{e}^{-\mathrm{i}k(j+1/2)} a^\dagger(k)b(k) +\text{h.c.} \right.\\
        &\left.+ \sum_{j=1}^{M+1} \mathrm{e}^{-(\sqrt{w^2+(-j+1/2)^2}-w)/\xi} \mathrm{e}^{-\mathrm{i}k(-j+1/2)} a^\dagger(k)b(k) +\text{h.c.}\right\}\\
        &= \Tilde{\Delta}\sum_{k}\left\{\sum_{j=0}^{M} \mathrm{e}^{-(\sqrt{w^2+(j+1/2)^2}-w)/\xi} \mathrm{e}^{-\mathrm{i}k(j+1/2)} a^\dagger(k)b(k) +\text{h.c.} \right.\\
        &\left.+ \sum_{j=0}^{M} \mathrm{e}^{-(\sqrt{w^2+(j+1/2)^2}-w)/\xi} \mathrm{e}^{\mathrm{i}k(j+1/2)} a^\dagger(k)b(k) +\text{h.c.}\right\}\\
        &= \Tilde{\Delta}\sum_{k}\sum_{j=0}^{M} \mathrm{e}^{-(\sqrt{w^2+(j+1/2)^2}-w)/\xi} 2\cos(k(j+1/2)) a^\dagger(k)b(k) +\text{h.c.} \\
    \end{split}
\]
This yields
\begin{equation}
    \mathcal{H}^{\text{edge}}_{\text{int}}(k)= \Tilde{\Delta} \sum_{j=0}^{M} 2\cos(k(j+1/2)) \mathrm{e}^{-(\sqrt{w^2+(j+1/2)^2}-w)/\xi}  \tau_x.
\end{equation}
Thus, if we identify $\mathcal{H}^{\text{edge}}_{\text{int}}(k)= \mathcal{M}^{\text{teo}}(k)\tau_x$ in the end we have
\begin{align}
    &N_y=4M: && \mathcal{M}^{\text{teo}}(k) = \Tilde{\Delta} [1 + \sum_{j=1}^{M} 2\cos(kj)\mathrm{e}^{-(\sqrt{w^2+j^2}-w)/\xi}],
    \label{eq:delta_even}
    \\
    &N_y=4M+2: && \mathcal{M}^{\text{teo}}(k) = \Tilde{\Delta} \sum_{j=0}^{M} 2\cos(k(j+1/2)) \mathrm{e}^{-(\sqrt{w^2+(j+1/2)^2}-w)/\xi}.
    \label{eq:delta_odd}
\end{align}

Moreover, as explained in the main text, we can numerically retrieve the mass term $\mathcal{M}$ as a function of the model parameters for a given $N_y$ via
\begin{equation}
    |\mathcal{M}(m;k)|= \sqrt{E_{N_y}(m; k)^2-E_{\infty}(m; k)^2},
    \label{eq:mass_num}
\end{equation}
where $E_{N_y}(m; k)$ is the-- exact, numerically computed --lowest positive energy band for a strip of width $N_y$ (all the other parameters of the model fixed), and $E_{\infty}(m; k)$ is the lowest positive energy band for a model with the same identical parameters but, on a much wider strip ($N_y\to \infty)$.

In Supplementary Figures~\ref{fig:Mk_t2_015-num}, \ref{fig:Mk_t2_02-num} and \ref{fig:Mk_t2_03-num} are shown the plots of $\mathcal{M}$ as a function of $k$ and for different values of the staggered mass $m$-- each column corresponding to one of the widths considered in the main text ($N_y=4,6,8,10$) --for $t_2=0.15$, $t_2=0.2$ and $t_2=0.3$ respectively. The reference spectrum ($E_{\infty}(m; k)$ in Eq.~\ref{eq:mass_num}) was obtained considering a strip with $N_y =202$ sites in the vertical direction.

The general features that we can retrieve from these plots are the following: The number of nodes in the effective mass increases of exactly one from one zigzag strip to the next larger one; for $m=0$ the effective mass has a node at $\pi$ when $N_y=4M+2, \ M \in \mathbb{Z}$; as $m$ is increased (to positive values) the profile of $\mathcal{M}$ moves to the right in the Brillouin zone; the higher the value of $t_2$ the more the profile of $\mathcal{M}$ spreads in $k$-space. The points in $k$ space where $\mathcal{M}$ stops oscillating and steeply goes up correspond to the points where the edge states of the wide (reference) strip merge with the bulk states. That is, where the low energy theory ceases to work.

\subsection*{Supplementary Note 5}
Here we report some more phenomenology regarding the Jackiw-Rebbi like bound states occurring at the interface between sides of the strip belonging to topologically distinct regions of the phase diagram. We set the parameters as in the main text to $t_1=1, \ t_2=0.3, \ \phi=\frac{\pi}{2}$.
In the main text we considered a scenario with $N_y=6$ and with $m$ interpolating between two regions of the phase diagram both trivial with respect to the presence of end states, but with the Zak phase differing of $\pi$. Here we consider a more detailed, though necessarily not exhaustive, phenomenology.

In Supplementary Figure~\ref{fig:VarPot_N_4-M_-0.3_0.3} we consider a zigzag Haldane nanoribbon with $N_y=4$ and $L=40a$, with the staggered on-site potential $m$ interpolating between $-0.3$ and $0.3$ (Panel \textbf{a}). With reference to the phase diagram in Fig.~2\textbf{a} of the main text, the two sides of the strip belong to the same region of the phase diagram: Therefore, no bound states at the domain wall would be expected. Indeed, from numerical diagonalization we find that two isolated degenerate eigenvalues occur inside the gap (Panel \textbf{b}). These correspond to two bound states which, according to the probability density plots in Panels \textbf{c}, \textbf{d}, are equally localized on the two ends of the strip. These are actually expected since both sides of the strip are in the topologically non trivial region (cf. Fig.~2\textbf{a} of the main text).



In Supplementary Figure~\ref{fig:VarPot_N_6-M_0.5_1.3} we consider a zigzag Haldane nanoribbon with $N_y=6$ and $L=80a$, with the staggered on-site potential $m$ interpolating between $0.5$ and $1.3$ (Panel \textbf{a}). With reference to the phase diagram in Fig.~2\textbf{b} of the main text, the two sides of the strip belong to distinct contiguous regions of the phase diagram: in terms of end states $m=0.5$ belongs to the topologically non-trivial region, while $m=1.3$ to the trivial (and connected to the atomic limit) one. Since the Zak phase differs of $\pi$ between the two regions, a bound state is expected at the domain wall. The low energy spectrum obtained via numerical diagonalization is reported in Panel \textbf{b}. We find that two isolated eigenvalues occur inside the gap: The lower one in energy (coloured in red) corresponds to an eigenstate localized at the domain wall in the on-site potential, as shown by the probability density plot in Panel \textbf{c}. The one at higher energy instead (coloured in orange), is localized at the left end of the strip (Panel \textbf{d}): this is expected since the left side, having $m=0.5$, pertain to a topological region (cf. Fig.~2\textbf{b} of the main text).

In Supplementary Figure~\ref{fig:VarPot_N_6-M_-0.5_0.5} we consider a zigzag Haldane nanoribbon with $N_y=6$ and $L=80a$, with the staggered on-site potential $m$ interpolating between $-0.5$ and $0.5$ (Panel \textbf{a}). With reference to the phase diagram in Fig.~2\textbf{b} of the main text, the two sides of the strip belong to distinct contiguous regions of the phase diagram: in terms of end states, both sides belong to topologically non trivial regions of the phase diagrams. Despite this, since the Zak phase differs of $\pi$ between the two regions, a bound state is expected at the domain wall (similarly to the case considered in the main text). The low energy spectrum obtained via numerical diagonalization is reported in Panel \textbf{b}. We find that three isolated eigenvalues occur inside the gap: The mid one in energy (coloured in red) corresponds to an eigenstate localized at the domain wall in the on-site potential, as shown by the probability density plot in Panel \textbf{c}. The one at higher energy instead (coloured in orange), is localized at the right end of the strip (Panel \textbf{d}). Finally, the lower one in energy correspond to an eigenstate localized at the left end of the strip (the corresponding probability density plot is not reported for brevity). Again, these latter two bound states are expected, since the right (left) side, having $m=0.5$ ($m=-0.5$), pertain to a topological region (cf. Fig.~2\textbf{b} of the main text).

\beginsupplement
\begin{refsection}

\section{Pseudocode Example of Cumulative Disruption Algorithm} \label{sec:psuedocode}

For readers seeking a succinct code-like description of our cumulative disruption curve algorithm, we have included \cref{lst:psuedocode}.

\begin{lstlisting}[label=lst:psuedocode, language=Python, caption=Pseudocode for disruption algorithm]
disruption = []
for c in communities:
    remaining = 0
    original = 0
    removeCommunity(c)
    for user in users:
        if degree(user) > 0:
            remaining += degree(user)
            original += originalDegree(user)
    disruption += [1 - (remaining / original)]
\end{lstlisting}

Note that when calculating disruption on large networks, it is much more efficient to cache the size of the smallest community that each user participates in. We can then sort all users by the order in which they will be removed, and avoid computationally expensive references to a graph or adjacency matrix for each removal-step in the algorithm.

\section{Applications to Unipartite Networks} \label{sec:unipartite}

Our influence metric is intended for settings with clearly defined communities. For example, participation in subreddits, membership on a Mastodon server, or committing to a software code repository, all discretely identify users as members of those explicitly-bounded groups. However, network data is often presented in a unipartite configuration such as users following other users. If it is still desirable to delineate communities and measure their influence in these settings, then they can be converted into compatible bipartite networks using the following procedure:

\begin{enumerate}
    \item Apply a context-appropriate community detection algorithm to label each user as belonging to one community

    \item Create a vertex for each community

    \item Replace all user-user edges with user-to-community edges, where the edge weight is equal to the number of unipartite edges each user had to other nodes in that community

    \item Apply our influence metric to the resulting bipartite graph
\end{enumerate}

An example of this procedure is illustrated in \cref{fig:unipartite}, using a unipartite Watts-Strogatz small-world network (100 nodes, 5 neighbors, rewiring probability of 5\%), and label-propagation for community detection. The unipartite graph is shown in the top-left with community labels visualized with color. It is converted to a bipartite representation shown in the upper-right, and the effect of removing each community is illustrated in the bottom frame.

% Figure environment removed






\section{Calculating the Area Under the Disruption Curve} \label{sec:auc_explanation}

For \cref{fig:real_networks_auc,fig:toy_networks_auc,fig:assortativity_auc} we use the area under the disruption curve as a single-variable summary of how centralized a network is around its largest communities. To calculate the AUC, we use a trapezoidal approximation in logarithmic space.

We chose a trapezoidal approximation to calculate the area even with limited sample points from real-world networks. Integration is possible for purely analytic disruption curve simulations as in \cref{sec:analytic_simulations}, but this is not feasible for our non-Erd\H{o}s-R\'{e}nyi networks, so we use a trapezoidal approximation for all synthetic networks for consistency.

We measure the AUC in logarithmic space, because measuring in linear space would heavily weight the influence of the smallest communities that are removed last, and our primary interest is in examining the influence of the largest communities on the broader population. 

\section{Synthetic Network Topology Details} \label{sec:toy_examples}

We measure centralization on a variety of synthetic networks introduced in \cref{sec:disruption_toy}. In this section, we include further description and visualization of the synthetic networks used.

Bipartite Near-Star networks are analogous to a unipartite star network with duplicate edges, but in a bipartite setting. Starting with a unipartite star, replace each edge from the hub to a leaf with a two-path from the hub community to a new ``user" vertex, to the leaf community. Duplicate edges from the unipartite hub to leaves are converted into multiple users that share a community, and serve to break ties when pruning communities for disruption curves. This is illustrated in \cref{fig:star}.

% Figure environment removed

For our ``Powerlaw" networks we follow a bipartite configuration model. We first create vertices representing the desired number of communities and users. We then draw from a powerlaw distribution with an assigned $\gamma$ exponent, and assign the drawn degree to each community. Then, we create a corresponding number of edges, wiring each community to users drawn uniformly at random without replacement. This yields networks where communities follow a powerlaw degree distribution, while users follow a normal degree distribution.

Bipartite community-user networks can be visualized in a flat plane, as in \cref{fig:centralization-pl}, or as a multi-layer graph, as in \cref{fig:pl-toy}. A multi-layer representation may be beneficial for representing inter-community relationships that are not explained by shared users, such as Mastodon federation agreements, or shared moderator staff in two subverses. However, these multiplex relationships were deemed out-of-scope for our current work.

% Figure environment removed




\begin{comment}
  #data structure for the dispersion metric
  D = np.zeros(nm)

  #calculate dispersion
  cumu_sum=0
  for n in np.arange(0,nm):
    cumu_sum += n*pn[n]
    #calculate U_n
    if pn[n]==0:
      continue
    Pnpm = Pnm[n,:]/np.sum(Pnm[n,:])
    U=0
    for m in np.arange(0,mm):
      if(np.sum(Pnm[:,m])>0):
        Pnmp = Pnm[:,m]/np.sum(Pnm[:,m])
        prob = np.sum(Pnmp[n:-1])
        U+=Pnpm[m]*prob**(m-1)
    D[n] = n*pn[n]*(1-U)/(cumu_sum-n*pn[n]*U)
\end{comment}

\section{Mathematical Analysis of Disruption in Random Networks} \label{sec:analytic_simulations}

We here calculate the disruption curves for random bipartite networks parameterized by their joint-degree distribution. This approach therefore fixes the distribution $\lbrace g_m \rbrace$ of communities $m$ per user, the distribution $\lbrace p_n \rbrace$ of community size $n$, and the joint-distribution $P_{n,m}$ for the degree of the node and community involved in a random bipartite link. Beyond these constraints, the networks are fully random but allow us to explore the role of heterogeneous connectivity at the user and community level as well as the impact of correlations between both levels.

We wish to calculate the disruption $D(n)$ involved when removing communities of size $n'<n$ in these random networks. By definition of the bipartite network, we know that $np_n$ edges are removed when removing communities of size $n$. Once again, we define disruption as the fraction of \textit{remaining} edges disrupted by communities of size $n$ during the pruning process. It is thus given by the number of edges that belong to communities of size $n$ minus the fraction $u_n$ of those that are the sole edge of the corresponding users (since these users are removed in the pruning) divided by the number of edges belonging to communities of size equal or smaller than $n$ minus the $u_nnp_n$ users removed. We write:

\vspace{2em}
\begin{equation}
    D(n) = \frac{
            \eqnmarkbox[NavyBlue]{bigedges}{np_n}
            -
            \eqnmarkbox[OliveGreen]{prunededges}{u_nnp_n}
        }{
            \eqnmarkbox[WildStrawberry]{remainingedges}{\sum_{n'\leq n}n'p_{n'}}
            -
            \eqnmarkbox[OliveGreen]{prunededges2}{u_nnp_n}
        } \; .
    % Here's Laurent's original expression
    %D(n) = \frac{np_n-u_nnp_n}{-u_nnp_n + \sum_{n'\leq n}n'p_{n'}} \; .
\end{equation}
\annotate[yshift=1em]{above,left}{bigedges}{Edges to comms. of size n}
\annotate[yshift=1em]{above,right}{prunededges}{Edges to removed users}
%\annotate[yshift=-1em]{below,right}{prunededges2}{Edges for removed users}
\annotate[yshift=-0.5em]{below}{remainingedges}{Edges to comms. n or smaller}
\vspace{2em}

The quantity $u_n$ can also be defined as the probability that a random user of a community of size $n$ has no community smaller than $n$. It can therefore be calculated like so:

\vspace{1em}
\begin{equation}
    u_n = \mathlarger{\sum}_m 
        \eqnmarkbox[NavyBlue]{users_in_n_with_m}{\frac{P_{n,m}}{\sum_{m'}P_{n,m'}}}
        \left(
            \eqnmarkbox[OliveGreen]{users_with_m_larger_than_n}{\frac{\sum_{n'\geq n} P_{n',m}}{\sum_{n'}P_{n',m}}}
        \right)^{m-1} \; .
    %u_n = \mathlarger{\sum}_m \frac{P_{n,m}}{\sum_{m'}P_{n,m}} \left(\frac{\sum_{n'\geq n} P_{n',m}}{\sum_{n'}P_{n',m}}\right)^{m-1} \; .
    \label{eq:un}
\end{equation}
\annotate[yshift=1em]{above,right}{users_in_n_with_m}{Fraction of users in comm. \\ \sffamily \footnotesize size n that have m edges}
\annotate[yshift=-0.5em]{below,left}{users_with_m_larger_than_n}{Fraction of users with m edges\\ \sffamily \footnotesize in comms. larger than size n}
\vspace{2.5em}

In the previous equation, we sum over every possible type of node in a community of size $n$, which will have a number of \textit{other} communities $m-1$ proportional to $P_{n,m}$, and ask for all of these communities to be larger or equal to $n$, which will be proportional to the sum of $P_{n',m}$ over all $n'$ larger or equal to $n$. Normalizing the probabilities appropriately yields Eq. (\ref{eq:un}) as written.

Note that these equations assume that edges are unweighted, and that there are no duplicate edges, which is what we expect from an infinite random simple graph. In our real-world data sets there are often duplicate edges (for example, one user following several different users on a Mastodon instance), which we compress to weighted edges for convenience.

Despite this difference between the analytical expression and real socio-technical networks, the analysis of random infinite graphs can be useful to test how disruption is impacted by simple network statistics such as degree distributions or correlations in the joint community-user degree matrix $P_{n,m}$. 

In a simple experiment, we create a random Erd\H{o}s-R\'{e}nyi-like bipartite network and correlated equivalent networks with the same degree distributions and variable community-user degree matrices $P_{n,m}$. The random network has a simple $P^{\textrm{rand}}_{n,m} \propto np_n mg_m$ (normalized) which we can modify manually. To do so, we calculate the maximally correlated $P^{\textrm{max}}_{n,m}$ by assigning users with highest degrees $m_{\textrm{max}}$ to the largest communities available before doing the same to users with the next higher degree and so on all the way down. We can do the same to calculate $P^{\textrm{min}}_{n,m}$ by assigning users with the lowest degree to the largest communities and working our way up in the user degree distribution. We can then create arbitrary community-user degree matrix $P_{n,m}$ by interpolating between linearly with $(1-\rho) P^{\textrm{rand}}_{n,m} + \rho P^{\textrm{max}}_{n,m}$ or $(1-\rho) P^{\textrm{rand}}_{n,m} + \rho P^{\textrm{min}}_{n,m}$.

Our results are shown in \cref{fig:assortivity_random_networks}. We find that positive user-community degree correlations increase disruption and therefore \textit{centralizes} the resulting socio-technical network. Conversely, negative correlations decreases correlations and \textit{decentralizes} the network. That being said, the relative effect of correlations is relatively small as the networks are still otherwise completely random.

% Figure environment removed

\section{Further Analysis of Assortativity} \label{sec:supplemental_assortativity}

There are multiple interpretations of degree assortativity in a bipartite setting. The linear correlation between user degrees and community degrees measures whether high-degree users are likely to be connected to high-degree communities. In our network definitions edges represent activity, like follow relationships or participation in conversations, so this measures whether active users are likely to be connected to communities with lots of activity. However, a second metric of interest is whether large communities are likely to be connected to other large communities, or in other words, the  assortativity of a unipartite-projected community-community graph. This can also be broken into two sub-cases: assortativity of community size (do communities with many users share users with other high-population communities), and assortativity of degree (do communities with lots of activity share users with other high-activity communities).

These three notions of assortativity are not independent; we might expect that users with lots of activity are active in communities with high populations, and may act as bridges between multiple communities with high activity and high population. However, the three metrics are not guaranteed to correlate and should be measured separately.

While rewiring to promote user-community degree assortativity, we also plotted the changes in community-community degree assortativity, shown in \cref{fig:assortivity_user_vs_community}. Strikingly, the community assortativity \textit{decreases} as we rewire to promote user assortativity. This is because as we rewire edges to focus user connections on the largest communities we implicitly decrease the number of edges between communities. This also matches the changes in disruption in \cref{fig:assortativity_auc}: increasing assortativity may reconnect large and insular communities with the rest of the network, briefly increasing their influence, but continued assortativity rewiring also cuts bridges to and between smaller communities, yielding a sparse network that is far less centralized.

% Figure environment removed

To further explore the relationship between these types of assortativity, we also rewired networks in the reverse direction: for randomly selected pairs of edges, we rewired those edges to \textit{decrease} user to community activity assortativity. We have plotted the change in disruption curves (\cref{fig:disassortative_auc}) and correlation between assortativity metrics (\cref{fig:disassortivity_user_vs_community}). In most networks, decreasing activity assortativity lowers centralization, although the effect diminishes as the network topology more closely approximates a random network. The one exception is the Penumbra; this network has such sparse inter-community connections that any perturbation of edges increases the cross-community links and therefore \textit{increases} centralization.

% Figure environment removed

% Figure environment removed

\section{Cumulative Impact on Giant Component Size} \label{sec:giant_components}

Some readers may be interested in how removing large communities influences the giant component size on each network. This is closely related to the cumulative population size in the top sub-plots of \cref{fig:real_networks_size_comparison} and \cref{fig:toy_networks_size_comparison}. Intuition suggests that the size of the giant component will be inversely proportional to the number of cumulative communities removed; as more large communities are pruned, the giant component should shrink. This relationship holds so long as the remaining communities are interlinked, but falters once a ``bridge" community is removed and the giant component splinters. Therefore, sparsely connected networks where bridges are more prominent will have a chaotic giant component size, while more densely connected networks will present a smooth curve until most communities are pruned. This relationship is illustrated in \cref{fig:real_giant_component}. Most curves are smooth until the tail of the distribution, with two notable exceptions: Voat's giant component changes once the largest insular communities are removed (see \cref{fig:voat_render}), and the Penumbra's curve is much ``spikier" as a result of its highly sparse structure.

% Figure environment removed

Measuring the change in giant component size captures some of the same features as our disruption metric. In particular, removing large insular communities may not change the giant component size if the community is completely isolated from the giant component, so this captures some aspect of both the size and topological role of a community. However, the impact of a community is boolean: if it touches the giant component, then removing the community will shrink the giant component by the size of that community. There is no distinction between a minimally integrated and tightly integrated community. Measuring the impact of a community in terms of fraction of edges severed, rather than component vertex size, offers finer insight into the interplay between size distribution and network structure.



\section{Comparison to Network Bottlenecking} \label{sec:cheeger}

The Cheeger number \cite{cheeger} is a single-valued metric representing how large of a ``bottleneck" inhibits conductance across a graph. It is typically written as:

\vspace{2em}
\begin{equation}
    h(G) = \min \left\{
        \frac{
                \eqnmarkbox[NavyBlue]{cheeger_crossedges}{|\partial A|}
            }{
                \eqnmarkbox[OliveGreen]{cheeger_alledges}{|A|}
            }
        : \eqnmarkbox[WildStrawberry]{cheeger_subset}{A \subseteq V(G)}, 
        \eqnmarkbox[Plum]{cheeger_bounds}{0 < |A| \leq \frac{1}{2} |V(G)|}
    \right\} 
\end{equation}
\annotate[yshift=1.2em]{above}{cheeger_crossedges}{Edges crossing the boundary of A}
\annotate[yshift=-0.2em]{below}{cheeger_alledges}{All edges in+across A}
\annotate[yshift=0.8em]{above}{cheeger_subset}{A is a subset of vertices of G}
\annotate[yshift=-2em]{below,left}{cheeger_bounds}{A contains at most half of all vertices}
\vspace{2em}

Our measurement of how much a community influences a larger population, and the Cheeger measurement of whether a community is a ``bottleneck" bear some conceptual similarities. Therefore, we compare our metric to the Cheeger number in two ways. First, we create a ``local Cheeger number," following an identical equation $\frac{|\partial A|}{|A|}$, but where $A$ is defined as the set of communities we are pruning, rather than via a global search. Second, we estimate bounds on the global Cheeger value of the graph. Since evaluating the graph conductance of all possible subsets of vertices is an NP-hard problem \cite{kaibel2004expansion}, it is impractical to directly measure the Cheeger constant on most large graphs. Fortunately, the Cheeger inequality offers upper and lower bounds on the Cheeger number based on the second eigenvalue of the normalized Laplacian of the adjacency matrix of G as follows:

$$\lambda_2/2 \leq h(G) \leq \sqrt{2\lambda_2}$$

Since they are sparse, these bounds can be calculated even on large real-world datasets. 
Unfortunately, in our tests the bounds are quite wide (see \cref{fig:cheeger}), limiting the utility of this approximation. We have plotted a comparison of the ``local" Cheeger number, bounds of the global Cheeger number, and our disruption metric, for a variety of simulated networks.

% Figure environment removed

\printbibliography[heading=subbibliography]
\end{refsection}


\end{document}