%Version 2.1 April 2023
% See section 11 of the User Manual for version history
%
%%%%%%%%%%%%%%%%%%%%%%%%%%%%%%%%%%%%%%%%%%%%%%%%%%%%%%%%%%%%%%%%%%%%%%
%%                                                                 %%
%% Please do not use \input{...} to include other tex files.       %%
%% Submit your LaTeX manuscript as one .tex document.              %%
%%                                                                 %%
%% All additional figures and files should be attached             %%
%% separately and not embedded in the \TeX\ document itself.       %%
%%                                                                 %%
%%%%%%%%%%%%%%%%%%%%%%%%%%%%%%%%%%%%%%%%%%%%%%%%%%%%%%%%%%%%%%%%%%%%%

%%\documentclass[referee,sn-basic]{sn-jnl}% referee option is meant for double line spacing

%%=======================================================%%
%% to print line numbers in the margin use lineno option %%
%%=======================================================%%

%%\documentclass[lineno,sn-basic]{sn-jnl}% Basic Springer Nature Reference Style/Chemistry Reference Style

%%======================================================%%
%% to compile with pdflatex/xelatex use pdflatex option %%
%%======================================================%%

%%\documentclass[pdflatex,sn-basic]{sn-jnl}% Basic Springer Nature Reference Style/Chemistry Reference Style


%%Note: the following reference styles support Namedate and Numbered referencing. By default the style follows the most common style. To switch between the options you can add or remove Numbered in the optional parenthesis. 
%%The option is available for: sn-basic.bst, sn-vancouver.bst, sn-chicago.bst, sn-mathphys.bst. %  
%\documentclass[pdflatex,iicol]{sn-jnl}
\documentclass[sn-nature]{sn-jnl}% Style for submissions to Nature Portfolio journals
%%\documentclass[sn-basic]{sn-jnl}% Basic Springer Nature Reference Style/Chemistry Reference Style
%%\documentclass[sn-mathphys,Numbered]{sn-jnl}% Math and Physical Sciences Reference Style
%%\documentclass[sn-aps]{sn-jnl}% American Physical Society (APS) Reference Style
%%\documentclass[sn-vancouver,Numbered]{sn-jnl}% Vancouver Reference Style
%%\documentclass[sn-apa]{sn-jnl}% APA Reference Style 
%%\documentclass[sn-chicago]{sn-jnl}% Chicago-based Humanities Reference Style
%%\documentclass[default]{sn-jnl}% Default
%%\documentclass[default,iicol]{sn-jnl}% Default with double column layout

%%%% Standard Packages
%%<additional latex packages if required can be included here>

\usepackage[vietnamese, english]{babel}
\usepackage{braket}

\usepackage{graphicx}%
\usepackage{multirow}%
\usepackage{amsmath,amssymb,amsfonts}%
\usepackage{amsthm}%
\usepackage{mathrsfs}%
\usepackage[title]{appendix}%
\usepackage{xcolor}%
\usepackage{textcomp}%
\usepackage{manyfoot}%
\usepackage{booktabs}%
\usepackage{algorithm}%
\usepackage{algorithmicx}%
\usepackage{algpseudocode}%
\usepackage{listings}%

\usepackage{bm}
%%%%

%%%%%=============================================================================%%%%
%%%%  Remarks: This template is provided to aid authors with the preparation
%%%%  of original research articles intended for submission to journals published 
%%%%  by Springer Nature. The guidance has been prepared in partnership with 
%%%%  production teams to conform to Springer Nature technical requirements. 
%%%%  Editorial and presentation requirements differ among journal portfolios and 
%%%%  research disciplines. You may find sections in this template are irrelevant 
%%%%  to your work and are empowered to omit any such section if allowed by the 
%%%%  journal you intend to submit to. The submission guidelines and policies 
%%%%  of the journal take precedence. A detailed User Manual is available in the 
%%%%  template package for technical guidance.
%%%%%=============================================================================%%%%

%\jyear{2021}%

%% as per the requirement new theorem styles can be included as shown below
\theoremstyle{thmstyleone}%
\newtheorem{theorem}{Theorem}%  meant for continuous numbers
%%\newtheorem{theorem}{Theorem}[section]% meant for sectionwise numbers
%% optional argument [theorem] produces theorem numbering sequence instead of independent numbers for Proposition
\newtheorem{proposition}[theorem]{Proposition}% 
%%\newtheorem{proposition}{Proposition}% to get separate numbers for theorem and proposition etc.

\theoremstyle{thmstyletwo}%
\newtheorem{example}{Example}%
\newtheorem{remark}{Remark}%

\theoremstyle{thmstylethree}%
\newtheorem{definition}{Definition}%

\raggedbottom
%%\unnumbered% uncomment this for unnumbered level heads

\begin{document}

\title[Topological bound states in the Haldane model]{Emerging topological bound states in Haldane model zigzag nanoribbons}

%%=============================================================%%
%% Prefix	-> \pfx{Dr}
%% GivenName	-> \fnm{Joergen W.}
%% Particle	-> \spfx{van der} -> surname prefix
%% FamilyName	-> \sur{Ploeg}
%% Suffix	-> \sfx{IV}
%% NatureName	-> \tanm{Poet Laureate} -> Title after name
%% Degrees	-> \dgr{MSc, PhD}
%% \author*[1,2]{\pfx{Dr} \fnm{Joergen W.} \spfx{van der} \sur{Ploeg} \sfx{IV} \tanm{Poet Laureate} 
%%                 \dgr{MSc, PhD}}\email{iauthor@gmail.com}
%%=============================================================%%

\author*[1]{\fnm{Simone} \sur{Traverso}}\email{simone.traverso@edu.unige.it}

\author[1,2]{\fnm{Maura} \sur{Sassetti}}\email{sassetti@fisica.unige.it}

\author[1,2]{\fnm{Niccol\`o Traverso} \sur{Ziani}}\email{traversoziani@fisica.unige.it}

\affil*[1]{\orgdiv{Physics Department}, \orgname{University of Genoa}, \orgaddress{\street{Via Dodecaneso 33}, \city{Genova}, \postcode{16146}, \country{Italia}}}

\affil[2]{\orgname{CNR-SPIN}, \orgaddress{\street{Via Dodecaneso 33}, \city{Genova}, \postcode{16146}, \country{Italia}}}

%%==================================%%
%% sample for unstructured abstract %%
%%==================================%%

\abstract{Zigzag nanoribbons hosting the Haldane Chern insulator model are considered. In this context, a reentrant topological phase, characterized by the emergence of quasi zero dimensional in-gap states, is discussed. The bound states, which reside in the gap opened by the hybridization of the counter-propagating edge modes of the Haldane phase, are localized at the ends of the strip and are found to be robust against on-site disorder. These findings are supported by the behavior of the Zak phase over the parameter space, which exhibits jumps of $\pi$ in correspondence to the phase transitions between the trivial and the non-trivial phases. The effective mass inversion leading to the jumps in the Zak phase is interpreted in a low energy framework. Setups with non-uniform parameters also show topological bound states via the Jackiw-Rebbi mechanism. All the properties reported are shown to be extremely sensitive to the strip width.}

\keywords{topological insulators, Haldane model, bound states, Zak phase}

%%\pacs[JEL Classification]{D8, H51}

%%\pacs[MSC Classification]{35A01, 65L10, 65L12, 65L20, 65L70}

\maketitle

\section{Introduction}\label{intro}
The discovery of topological insulators and topological superconductors completely revolutionized the usual classification of the phases of matter, shedding light on the fact that the Ginzburg-Landau classification was but a partial description~\cite{Zhang_rev_2011}. Starting from the first feasible proposal for a topological insulator \cite{bernevig_2007}, in less than two decades the field has undergone several major breakthroughs, like the classification of topological phases by symmetry classes \cite{ryu_2010}, the discovery of higher order topology~\cite{Neupert_2018} and, very recently, of non-Hermitian topology~\cite{Bergholtz_2021}. {\color{black} In this context, the dimensionality of the systems plays a crucial role for the definition of the topological invariants. Indeed, these are typically defined relative to systems compactified in all directions. The bulk boundary correspondence, then, suggests that when the system is non-trivial, and is made semi-infinite in one direction, metallic states appear at the boundary. These metallic states persist as long as the uncompactified dimension of the system does not become comparable with their decay length.

The improvement in the nanostructuration of topological phases of matter has made it possible to realize samples in which this condition is not met~\cite{Molenkamp_2020, Claessen_2022}. In this regard,} a very active branch of research is nowadays related to the study of finite size effects on topological phases~\cite{niu_2008, PhysRevB.92.235407, PhysRevB.98.205129, Claessen_2022, Teemu_2011, Molenkamp_2020, Rodriguez2020, Fleckenstein2021, Traverso_2022, Vigliotti_2023, PhysRevB.107.245409, Zhu_2023, Saha_2023}. Indeed, dimensional crossovers between topological phases is a promising way to engineer novel topological systems~\cite{Potter_2010, Zhang2022}. In this context, the way has been paved by the extensive studies performed on graphene nanoribbons (GNRs), that have revealed an extremely rich phenomenology. For instance, it was found that these systems, depending on the nanoribbon width and on the nature of its terminations, can host robust topological bound states~\cite{PhysRevLett.119.076401, Li_2021} amenable to detection with local probes~\cite{local1,local2,local3}.

Taking a step back, even before the first theoretical proposal for a symmetry protected topological insulator was conceived, many topologically non-trivial systems had been object of study, first of all the quantum Hall system~\cite{Klitzing_1980}.
In that context, a milestone for the comprehension of topological phases had been conceived: the Haldane model~\cite{haldane_1988}. Dated back to 1988, it represented the first theoretical proposal for a system realising a quantum anomalous Hall phase and is now regarded as the most famous model for a Chern insulator. Even more importantly, the time reversal doubling of the Haldane model results in the Kane-Mele model which, describing spinful fermions on a honeycomb lattice with strong spin-orbit interaction, is the prototypical model of a time reversal protected topological phase~\cite{Kane-Mele-1_2005, Kane-Mele-2_2005}. Although the Kane-Mele model was originally proposed to describe the electrons in graphene, where the predicted spin Hall phase cannot be observed since the spin-orbit coupling is too small to yield a sizable topological gap, it has recently found a direct experimental realization in Bismuthene~\cite{Reis_2017} {\color{black} and Germanene~\cite{Bampoulis_2023}}, and is hence receiving renewed attention.

In light of this, Haldane model nanoribbons represent a significant model for studying the physics of these newly discovered honeycomb-based topological materials. Even more importantly, they configure themselves as the optimal theoretical playground for merging the physics of nanostructured topological insulators and the one inherited from graphene nanoribbons. In this paper, we focus on zigzag Haldane nanoribbons and assess the effects of dimensional reduction on the topological phase of the Haldane model. We find that, for thin enough strips, there are multiple regions of the parameter space in which the chiral edge states gap out and, correspondingly, degenerate pairs of quasi zero-dimensional (0D) end-states appear whose energy lies inside the gap. Such regions however are intercalated, through topological quantum phase transitions, to phases without bound states. We hence unveil a complex, width-dependent, reentrant quantum phase diagram, which we characterize by numerically computing a well established indicator for the classification of topological phases in (effectively) one-dimensional systems, that is, the Zak phase~\cite{Zak_1989, Jens_2017,Jeong2022}. Moreover, {\color{black} we explain the mechanism leading to the mass inversion via a phenomenological low energy theory, effectively modelling the chiral edge states coupling. Finally,} we show that domain walls in the on-site staggered potential distribution can localize quasi 0D bound states. These Jackiw-Rebbi like states~\cite{Jackiw_1976}, ubiquitous in topological phases of matter~\cite{Qi_2008, Teemu_2011, Ziani_2017, Fu2022, Cheng2023}, retain a fractional charge of $\pm \frac{e}{2}$ ($e$ the electron charge) and are found to be robust against the introduction of random on-site disorder. More generally, our results demonstrate that the physics of coupled topological edges can be way richer than what is expected from naive low energy theories.

\section{Results}
\label{sec:Res_disc}
\subsection{Haldane model on zigzag nanoribbons}
The Haldane model~\cite{haldane_1988} describes spinless fermions on a honeycomb lattice, pierced by an orthogonal periodic magnetic field and having the full symmetry of the lattice and zero net flux over the unit cell. Time reversal symmetry is broken for the model, so that the transverse Hall conductance can be non-trivially quantized~\cite{bernevig2013topological}. We denote the two sublattices of the honeycomb lattice as $A$ and $B$ and we choose as primitive vectors $\mathbf{a}_1 = a (1,0) $ and $\mathbf{a}_2 = a(\frac{1}{2},\frac{\sqrt{3}}{2})$. Moreover, we place the unit cell origin on the $A$ sites, so that the basis vectors are given by ${\bm \delta}_a = (0,0)$ and ${\bm \delta}_b = a(\frac{1}{2},\frac{1}{2\sqrt{3}})$. The Hamiltonian is thus given by
\begin{equation}
    \begin{split}
        \mathcal{H} &= t_1 (\sum_{l,n} a^\dagger_{ln} b_{ln} + a^\dagger_{ln} b_{l-1n}+a^\dagger_{ln}b_{ln-1})+\text{h.c.}\\
        &+t_2 \mathrm{e}^{\mathrm{i}\phi} (\sum_{l,n} a_{ln}^\dagger a_{l+1n} + a^\dagger_{ln}a_{l-1n+1}+a^\dagger_{ln}a_{ln-1})+\text{h.c.}\\
        &+t_2 \mathrm{e}^{-\mathrm{i}\phi} (\sum_{l,n} b_{ln}^\dagger b_{l+1n} + b^\dagger_{ln}b_{l-1n+1}+b^\dagger_{ln}b_{ln-1})+\text{h.c.}\\
        &+m (\sum_{l,n} a^\dagger_{ln}a_{ln}-b^\dagger_{ln}b_{ln}),
    \end{split}
\end{equation}
where $a_{ln}$ ($b_{ln}$) destroys a fermion at $l\mathbf{a}_1+n\mathbf{a}_2+\bm{\delta}_{a(b)}$; $t_1$ and $t_2$ parameterize the intensity of the nearest and next to nearest neighbor hoppings respectively; $\phi$ is a phase accounting for the staggered magnetic flux inside the unit cell and $m$ tunes a term of staggered on-site potential breaking inversion symmetry. In the topological phase, that occurs for $|m/t_2|<3\sqrt{3}\sin\phi$, the bulk bands are gapped and the Chern number~\cite{bernevig2013topological} has a non-trivial value ($c=\pm 1)$~\cite{haldane_1988}.

Correspondingly, in a strip geometry chiral modes localized on opposite edges and with gapless dispersion relation occur, in accordance with the bulk boundary correspondence. This can be explicitly seen by imposing periodic boundary conditions (PBC) in the $\mathbf{a}_1$ direction and going in $k$-space. For a strip of length $L=Na$, the Fourier transformation is defined as
\begin{align}
    a_{ln}&= \dfrac{1}{\sqrt{N}}\sum_{k} \mathrm{e}^{-\mathrm{i}kx_{ln}^a}a_n(k) & x_{ln}^a&=la+\frac{n}{2}a, \\
    b_{ln}&= \dfrac{1}{\sqrt{N}}\sum_{k} \mathrm{e}^{-\mathrm{i}kx_{ln}^b}b_n(k) & x_{ln}^b&=la+\frac{n+1}{2}a,
\end{align}
with the Bloch momenta discretized as $k_j=\dfrac{2\pi}{Na}j, \ j=0,\ldots,N-1$. The Bloch Hamiltonian is found to be of the form (see also \cite{Wang_2008})
\begin{equation}
    \mathcal{H}(k) =
    \begin{bmatrix}
        g(\tilde{k},\phi)+m & 2t_1\cos(\frac{\tilde{k}}{2}) & g(\frac{\tilde{k}}{2},-\phi) & 0 & 0 & 0 &\cdots & 0 \\
        2t_1\cos(\frac{\tilde{k}}{2}) & g(\tilde{k},-\phi)-m & t_1 & g(\frac{\tilde{k}}{2},\phi) & 0 & 0 & \cdots & 0 \\
        g(\frac{\tilde{k}}{2},-\phi) & t_1 & g(\tilde{k},\phi)+m & 2t_1\cos(\frac{\tilde{k}}{2}) & g(\frac{\tilde{k}}{2},-\phi) & 0 &\cdots & 0 \\
        0& g(\frac{\tilde{k}}{2},\phi) & 2t_1\cos(\frac{\tilde{k}}{2}) & g(\tilde{k},-\phi)-m & t_1 & g(\frac{\tilde{k}}{2},\phi) & \ddots & 0 \\
        \vdots & \ddots & \ddots &  \ddots & \ddots & \ddots & \cdots &\vdots
    \end{bmatrix},
    \label{eq:Bloch_matrix}
\end{equation}
with $g(\tilde{k},\phi)=2t_2\cos(\tilde{k}+\phi)$ and $\tilde{k}=ka$.

In Fig.~\ref{fig:fig1}\textbf{a} is shown a schematic representation of a strip with zigzag edges, $N_y=60$ sites in the vertical direction, and PBC along the $\mathbf{a}_1$ direction. In Fig.~\ref{fig:fig1}\textbf{b} are reported the corresponding energy bands, obtained via numerical tight-binding diagonalization. Inside the bulk gap are clearly visible the two modes corresponding to the chiral edge states characterizing the topological phase.

% Figure environment removed

It can be expected that by reducing the strip width, the chiral edge states may hybridize because of spatial overlap, giving rise to a gap opening in the edge spectrum. {\color{black} The length scale at which this phenomenon becomes relevant is defined by the decay length of the edge states, that, for the Haldane model on zigzag nanoribbons, has been found to be~\cite{Doh_2013, Cano_2013}
\begin{equation}
    \xi_{\text{loc}}\approx \dfrac{\sqrt{3}}{2}\left[\log\left\{\sqrt{1+\left(\frac{t_1}{4t_2}\right)^2} +\frac{t_1}{4t_2}\right\}\right]^{-1}.
    \label{eq:loc_len}
\end{equation}
This kind of behavior has been explicitly proven} in several contexts~\cite{niu_2008,Ohyama_2011}. In spite of this, an analysis of the topological character of the gapped phases originated by finite size effects is still missing in the context of Chern insulators. In the next section, the results of such a study are presented.

\subsection{Emerging quasi 1D topological phase diagram}
We start by numerically computing the amplitude of the gap $\Delta$ as a function of the staggered on-site potential $m$. We set the energy scale to $t_1 =1$, and we impose $t_2=0.3$ and $\phi=\pi/2$, so that the topological bulk gap of the Haldane model is maximized. We perform our analysis for the Haldane strip with PBC for the different widths $N_y= 4,6,8,10$. The results are reported in Fig.~\ref{fig:Gap-Zak}.

% Figure environment removed

By comparing the top panels (Fig.~\ref{fig:Gap-Zak}\textbf{a}-\textbf{d}) it appears that the number of gap closings and reopenings grows monotonically with the strip width. This non trivial pattern hints to the fact that a topological phase transition may be associated with the gap closings. Interestingly, for the strips whose number of sites in the vertical direction is given by $N_y=4M+2$, the edge spectrum is gapless for $m=0$: in these cases, despite the wave functions of the chiral modes on opposite edges are brought in close proximity, they do not hybridize with each other. This counterintuitive behavior is proven analytically in Supplementary Note 1.

In order to characterize the topology of the zigzag Haldane strips, we use a well established tool for assessing the topological properties of 1D solids: the Zak phase~\cite{Zak_1989}, \textit{i.e.} the natural open-loop extension of the Berry phase~\cite{Berry_1984} when the parameter space is the Brillouin zone. The Zak phase associated with an isolated band was originally defined in terms of the cell-periodic Bloch functions $\ket{u_k}$ as~\cite{Zak_1989}:
\begin{equation}
    \varphi = \mathrm{i} \int_{0}^{2\pi/a}\mathrm{d}k \braket{u_k|\partial_k |u_k}.
\end{equation}
However, the above definition cannot be applied in the present case. Indeed, we are dealing with a multiband system in which the valence bands may cross each other and whose Hamiltonian can only be diagonalized numerically. Thus, we follow the prescription given in~\cite{Resta_1994} for the multi-band case. Given a discretization of the Brillouin zone $k_j=\dfrac{2\pi}{a}\dfrac{j}{N}, \ j \in \{0,\ldots,N-1\}$, for each momentum $k_j$ we compute a basis of eigenstates of the Hamiltonian in Eq.~\ref{eq:Bloch_matrix} for the occupied bands. The resulting $N_y$-dimensional vectors will be denoted as $\ket{u_{nk_0}},\ldots,\ket{u_{nk_{N-1}}}$, $n$ being the band index. Then we enforce the periodic gauge, by defining~\cite{Resta_1994}
\begin{equation}
    \ket{u_{nk_N}}_\ell=\mathrm{e}^{-\mathrm{i}\frac{2\pi}{a}t_\ell}\ket{u_{nk_0}}_\ell,
\end{equation}
where $\ket{u_{nk_j}}_\ell$ is the $\ell$-th component of the eigenvector $\ket{u_{nk_j}}$ and $t_\ell$ the $x$ position of the $\ell$-th site inside the strip unit cell~\cite{yusufaly2013tight}, whose origin we place on the $A$ site at the bottom edge. The Zak phase for the occupied bands is thus defined as~\cite{Resta_1994, vanderbilt_2018, yusufaly2013tight}
\begin{equation}
    \varphi=-\Im \log \det \prod_{j=0}^{N-1}S(k_{j},k_{j+1}),
\end{equation}
with the overlap matrix $S$ given by~\cite{Resta_1994, vanderbilt_2018, yusufaly2013tight}
\begin{equation}
    S(k_{j},k_{j+1})_{mn}=\braket{u_{mk_j}|u_{nk_{j+1}}}.
\end{equation}
This formula is unaffected by any erratic behavior of the phase randomly appended to the eigenvectors by the numerical diagonalization routine~\cite{vanderbilt_2018}. It must be noted that since the Hamiltonian in Eq.~\ref{eq:Bloch_matrix} is real, the Zak phase can only be $0$ or $\pi$. The values obtained for the Zak phase at different values of $m$ are reported in the bottom panels of Fig.~\ref{fig:Gap-Zak}, below the corresponding plots of the energy gap. We find that the Zak phases jump of $\pi$ at each gap closing, confirming that some kind of topological phase transition actually occurs.

In order to gain any insight about which phase is trivial and which is topological we need to make some more physical considerations. As a matter of fact, the Zak phase itself does not have an absolute meaning~\cite{Jens_2017, Atala2013, cayssol_2021}, since its value depends on the choice of the unit cell origin in real space. However, the difference between the values of the Zak phases in two regions of the parameter space is uniquely defined (modulo $2\pi$).

In the large $m$ limit, the system at half filling is expected to be a trivial insulator: in fact, in this scenario the electrons localize on the sublattice which is lower in energy (depending on the sign of $m$) and the hoppings between sites are suppressed. In view of this, we expect the regions of parameter space characterized by a value of the Zak phase differing by $\pi$ ($0$) from that of the large $m$ limit to be topologically non-trivial (trivial). It is worth pointing out that for $N_y=6$ and $N_y=10$ the limits $m \to \pm \infty$ correspond to different values of the Zak phase. We prove analytically in Supplementary Note 2 that this is a general fact for $N_y=4M+2$. In these cases, the comparison of the Zak phases should be made with the $m\to -\infty$ limit for negative values of $m$ and with the $m\to \infty$ limit for positive ones. The phase diagram emerging from this argument is depicted in Fig.~\ref{fig:phase-diag}, where for each of the widths considered a star is drawn in every region of the parameter space which is expected to be topologically non-trivial. By virtue of bulk boundary correspondence, quasi 0D bound states should occur when the quasi 1D strips are considered under open boundary conditions (OBC) along the $\mathbf{a}_1$ direction and the parameters fall inside one of the topological regions depicted in Fig.~\ref{fig:phase-diag}.

% Figure environment removed

To check our predictions we perform numerical diagonalization of the model in an uncompactified geometry along the $\mathbf{a}_1$ direction and we inspect the low energy spectra in the different regions of the parameter space. The OBC strips are cropped with a rectangular geometry, with the long edges parallel to $\mathbf{a}_1$. In Fig.~\ref{fig:bound_states}, for each of the widths taken into account, the bands in PBC geometry (Panels \textbf{a}-\textbf{d}) and the corresponding low energy spectra in the OBC case (Panels \textbf{e}-\textbf{h}) are reported. The values of $m$ at which the diagonalization was performed for each strip, were chosen close to the point of the topological regions where the gap was maximum (cf. Fig.~\ref{fig:Gap-Zak}\textbf{a}-\textbf{d}). The spectra of the finite size systems present two degenerate eigenvalues located inside the gap. In Panels \textbf{i}-\textbf{l} we report the 1D profile of the probability density distributions corresponding to the eigenvalue n${}^\circ 10$ (marked in orange) of the spectra in Panels \textbf{e}-\textbf{h} respectively as a function of the position, together with insets representing the corresponding unprojected probability densities directly on the strips. From the localization pattern of these states, we conclude that the in-gap doublets characterizing the topological phases depicted in Fig.~\ref{fig:phase-diag} correspond to quasi zero-dimensional bound states exponentially localized at the two ends of the strips. We numerically checked the robustness of these 0D bound states against the introduction of random on-site disorder finding that, though their energy is inevitably slightly shifted, they survive as long as the disorder strength does not become comparable with the gap width (see Supplementary Note 3).

% Figure environment removed

The results just discussed prove that the dimensional reduction of the Haldane model, when operated on a strip with zigzag edges, gives rise to a reentrant topological phase diagram, characterized by the emergence of degenerate doublets of in-gap 0D end states. Quite curiously, such phenomenology has no counterpart in the case of armchair nanoribbons, at least in the parameter range we numerically inspected. A qualitative motivation of this peculiar asymmetry is given by the end of the next subsection, where we interpret our results through an effective model.

\subsection{Low energy theory}
In order to understand our findings, we frame them into an effective low energy theory. The minimal low energy model describing the coupling between the two counter-propagating edge states in a zigzag Haldane nanoribbon would be
\begin{equation}
    H_{\text{edge}} = [v_F^{\text{ZZ}}(k-\pi)+\Tilde{m}]\tau_z+\mathcal{M} \tau_x,
    \label{eq:low_en_ham}
\end{equation}
where~\cite{Doh_2013}
\[
    v_{F}^{\text{ZZ}} \approx \dfrac{6 t_2 t_1}{\sqrt{t_1^2+16 t_2^2}}, \qquad \Tilde{m}\approx\dfrac{ m t_1}{\sqrt{t_1^2+16 t_2^2}}.
\]
Unfortunately, here $\mathcal{M}$ is an effective coupling induced by the spatial proximity between the two zigzag edges in the thin strip limit. As such, in principle it depends in a non trivial way on all of the model parameters, as well as on the Bloch momentum $k$.
This severely hinders the derivation of a reliable low energy theory for the present model. Nevertheless, we can still achieve some qualitative predictions-- at least for $m=0$ --by the following hand waving argument. To effectively describe the coupling of the chiral edge states, we consider two 1D chains, representing the two edges of the zigzag nanoribbon. We assume that the sites of the two chains are connected via a coupling which decreases exponentially with the distance, as it is usually done when modelling proximity effects. However, we add a crucial physical input: We introduce a sharp cutoff in the hopping range, effectively coupling only sites that in the actual nanoribbon are connected by the minimum amount of first neighbor hoppings. The motivation behind this choice, is that the coupling between any other pair of edge sites would represent a higher order (negligible) correction.

Omitting the details of the derivation, for which we refer the interested reader to Supplementary Note 4, we find for the two classes considered ($N_y=4M$ or $N_y=4M+2$) and assuming $m=0$
\begin{align}
    &N_y=4M: && \mathcal{M}^{\text{teo}}(k) = \Tilde{\Delta} [1 + \sum_{j=1}^{M} 2\cos(kj)\mathrm{e}^{-(\sqrt{w^2+j^2}-w)/\xi}],
    \label{eq:delta_even}
    \\
    &N_y=4M+2: && \mathcal{M}^{\text{teo}}(k) = \Tilde{\Delta} \sum_{j=0}^{M} 2\cos(k(j+1/2)) \mathrm{e}^{-(\sqrt{w^2+(j+1/2)^2}-w)/\xi},
    \label{eq:delta_odd}
\end{align}
where $w=\frac{\sqrt{3}}{2}\frac{N_y}{2}-\frac{1}{\sqrt{3}}$ is the strip width and $\xi$ is of the order of the chiral edge states localization length (Eq.~\ref{eq:loc_len}).

To benchmark our results, we extract the low energy bands directly from the exact numerical diagonalization. Let us denote by $E_{N_y}(m; k)$ the lowest (positive) energy band for a strip of width $N_y$ (all the other parameters of the model fixed). We can assume that, close to the Dirac point, $E_{N_y}(m; k)$ is well described by the spectrum of the low energy model in Eq.~\ref{eq:low_en_ham}, with a certain unknown function $\mathcal{M}$. Thus, noting that the hybridization between the chiral modes is exponentially suppressed with the strip width-- \textit{i.e.} $\mathcal{M} \xrightarrow[]{N_y\to \infty} 0$ --we can recover (the modulus of) $\mathcal{M}$ for a given $N_y$ as
\begin{equation}
    |\mathcal{M}(m;k)| = \sqrt{E_{N_y}(m; k)^2-E_{\infty}(m; k)^2}.
    \label{eq:mass_numeric}
\end{equation}

In Fig.~\ref{fig:mass_teo_vs_num}\textbf{a}-\textbf{d} are shown the plots of $\mathcal{M}^{\text{teo}}(k)$ as a function of $k$ for the different widths considered, with $\xi$ set to twice the localization length of the edge states ($\xi_{\text{loc}}$) for $t_2=0.3$ (assuming $t_1=1$). In Panels \textbf{e}-\textbf{h} instead, are reported the corresponding plots of $|\mathcal{M}(0;k)|$, obtained numerically as described in Eq.~\ref{eq:mass_numeric}.
Note that the points in $k$ space where $\mathcal{M}$ stops oscillating and steeply goes up correspond to the points where the edge states of the reference strip ($N_y\to \infty$) merge with the bulk states. Beyond this limit, the low energy Hamiltonian in Eq.~\ref{eq:low_en_ham} is no longer valid, since it does not account for the bulk degrees of freedom.

% Figure environment removed

By comparison between the two rows of plots, one can see that the effective mass term in Eqs.~\ref{eq:delta_even} and \ref{eq:delta_odd} correctly reproduce some qualitative features of the one obtained numerically for $m=0$. More specifically, setting the parameter $t_2$ in the range $[0.1 \sim 0.3]$, we correctly recover the number of nodes of $\mathcal{M}(0;k)$ for the different widths, the fact that the mass term spreads in $k$-space as $t_2$ is increased and, crucially, the fact that $\mathcal{M}(0;\pi)=0$ for $N_y=4M+2$. Further support to these statements can be found in Supplementary Figs.~4, 5 and 6, where, for each of the widths considered, we report the plots of $|\mathcal{M}|$ as a function of $k$ for different values of $m$ and $t_2$. It is worth mentioning that the shape of $\mathcal{M}(0;k)$ more closely resembles that of a sinusoidal-like function for smaller values of $t_2$ ($t_2\sim 0.1-0.2$). The deviations from the sinusoidal pattern predicted by the effective low energy theory observed in Fig.~\ref{fig:mass_teo_vs_num}\textbf{e}-\textbf{l} ($t_2=0.3$), are probably due to the fact that the first has been derived taking into account first neighbor hoppings only. That being the case, it is expected to be more reliable for smaller values of $t_2$.

Furthermore, the numerical results for $\mathcal{M}$ show that, at least at a qualitative level, for $m>0$ the mass term is shifted to the \emph{right} in the Brillouin zone (see Fig.~\ref{fig:mass_teo_vs_num}\textbf{i}-\textbf{l} and Supplementary Figs.~4, 5 and 6). On the other hand, we know that for $m>0$ the Dirac point moves to the \emph{left} (dashed vertical lines in Fig.~\ref{fig:mass_teo_vs_num}\textbf{i}-\textbf{l}). Therefore, there must be a (set of) value(s) of $m$ for which the nodes of $\mathcal{M}(m;k)$ coincide with the shifted Dirac point. This mechanism qualitatively explains the mass inversion in our model and, consequently, the reentrant topological phase diagram retrieved numerically.

Finally, we briefly address the reason why armchair nanoribbons do not share this kind of phenomenology. In the Haldane model on armchair nanoribbons the Dirac point is at $k=0$, and is insensitive to variation of the staggered mass. This can be understood by noticing that, in contrast with the zigzag case, each armchair edge has the same number of $A$ and $B$ sites. Thus, varying $m$ does not shift the energy of the edge states. This hints to the fact that in the case of armchair geometry, the properties of the edge states are less likely to be tuned by varying the staggered mass $m$. Moreover, the edge states in zigzag Haldane nanoribbons present a peculiar property: as shown by Eq.~\ref{eq:loc_len}, $\xi_{\text{loc}}$ grows with $t_2$ in the zigzag case. On the other hand, the Haldane edge states on armchair nanoribbons have their localization length inversely proportional to $t_2$~\cite{Cano_2013}. Thus, in the zigzag case, bigger values of $t_2$ lead both to a wider topological bulk gap in the 2D limit, and to a larger localization length of the edge states. This feature, which is not shared by armchair nanoribbons, elects thin zigzag nanoribbons as the optimal playground to inspect the finite size effects on the Haldane edge states.

\subsection{Jackiw-Rebbi like bound states}
Interestingly, if we consider a setup of our model in which the staggered mass term $m$ interpolates between two values in adjacent regions of the phase diagram with different Zak phases, we observe the occurrence of a bound state, localized at the transition point and with energy lying inside the gap. Thus, leaning on the results from Jackiw and Rebbi~\cite{Jackiw_1976}, we can conclude that such bound state retains a fractional charge of $\pm \frac{e}{2}$~\cite{Qi_2008, Ziani_2020}. Strikingly, the bound state is present even when the mass $m$ interpolates between two regions that, despite having different Zak phases, do not host bound states against the vacuum (more details in Supplementary Note 5).

In Fig.~\ref{fig:jackiw-bs} we report an explicit example for a strip with $N_y=10$ and $L=200a$, in which the on-site staggered potential is set to $\mp 0.3$ for the $A$ and $B$ sites respectively on the left part and to $\pm 0.3$ on the right part (with $m=0$ at the interface to smoothen out the jump). These two regions of the phase diagram are marked with a yellow and red dot respectively in Fig.~\ref{fig:phase-diag}\textbf{d}. All the other parameters of the model are left unchanged. In Panel \textbf{a} is reported a density plot of the on-site potential close to the transition point, where the sign of $m$ switches for the two sublattices. By performing real space numerical diagonalization we obtain the low energy spectrum and the corresponding eigenvectors. The first is shown in Panel \textbf{b}, where an isolated eigenvalue is clearly visible inside the gap: this corresponds to a bound state localized at the point where $m$ switches its sign, as demonstrated by the plot of its probability density in Panel \textbf{c}. From Fig.~\ref{fig:Gap-Zak}\textbf{d}, we see that from $m=-0.3$ to $m=+0.3$ the Zak phase jumps of $\pi$, so that the occurrence of a bound state at the domain wall is actually expected according to our analysis.

% Figure environment removed


\section{Discussion}\label{sec:Disc}

In this paper we have studied the role of finite size effects on the Haldane model. We have shown that, in the case of zigzag strip geometry, the chiral edge modes can, as expected, hybridize and develop a gap. Surprisingly, however, such gap can be both trivial or topological in the sense that, in the uncompactified geometry, bound states can be present or absent depending on the value of the trivial mass, even in the topological phase of the Haldane model. In other words, we unveiled a phase diagram which presents a width-dependent reentrant behavior with respect to the tuning of the on-site staggered potential. Moreover, we have reinterpreted such reentrant structure within an effective minimal model describing the coupling between the chiral edge states of the Haldane model in zigzag nanoribbons.

We have then proven that, when present, the bound states are robust against on-site random disorder. Besides, we have shown that they also occur in correspondence of domain walls in the on-site staggered potential and, consequently, that they bear a fractional charge of $\pm\frac{e}{2}$. 
The topological nature of the bound states is witnessed by the behavior of the Zak phase. Indeed, we can hence conclude that the mass associated to the tunneling between the edges competes in a definitely non-trivial fashion with the other masses of the model, generating a rich phenomenology.

The implications of our results are diverse. In the context of two-dimensional topological insulators and Chern insulators they imply, for instance, that the transport properties of setups where constrictions are present might be affected by the presence of zero modes, and hence show resonances. Moreover, given that a similar gap structure also characterizes the 2D Kitaev model~\cite{Potter_2010,Shen_2011} for topological superconductivity, the impact of our results bears also consequences on the field of Majorana zero modes~\cite{Kitaev2001, Albrecht_2016, Mourik_2012} and parafermions~\cite{Fendley_2012, Alicea_2016, calzona_2018}, paving the way to new possibilities for implementing such non-abelian excitations.

\section{Methods}\label{sec:Methods}
\subsection{Numerical diagonalization tools}
The finite size model construction and the numerical diagonalization have been performed using the package Pybinding~\cite{dean_moldovan_2020_4010216}.
\subsection{Numerical computation of the Zak phase}
The computation of the gaps and of the Zak phase as a function of the staggered on-site potential have been performed with an original code and the results for the Zak phase have been benchmarked with existing packages.
\backmatter

\section*{Declarations}

\bmhead{Data availability}
All data relevant to the paper are reported in the main text and in the Supplementary Information. All the numerically generated points reported in the plots of this paper and in the accompanying supplementary are obtained as described in section \ref{sec:Methods}. The actual codes used to produce the results reported in this paper and in the accompanying supplementary are available from the corresponding author upon request.

\bmhead{Acknowledgments}
N.T.Z. acknowledges the funding through the NextGenerationEu Curiosity Driven Project ``Understanding even-odd criticality''. N.T.Z. and M.S. acknowledge the funding through the ``Non-reciprocal supercurrent and topological transitions in hybrid Nb-InSb nanoflags'' project (Prot. 2022PH852L) in the framework of PRIN 2022 initiative of the Italian Ministry of University (MUR) for the National Research Program (PNR).

\bmhead{Author contributions}
Conceptualization, M.S. and N.T.Z. Development and preparation of the first draft S.T. All authors contributed to the interpretation of the results and reviewed the manuscript.

\bmhead{Competing interests}
The authors declare no competing interests.

\bmhead{Supplementary information}
Details of the calculations and further numerical results can be found in the accompanying Supplementary Information.


%%===========================================================================================%%
%% If you are submitting to one of the Nature Portfolio journals, using the eJP submission   %%
%% system, please include the references within the manuscript file itself. You may do this  %%
%% by copying the reference list from your .bbl file, paste it into the main manuscript .tex %%
%% file, and delete the associated \verb+\bibliography+ commands.                            %%
%%===========================================================================================%%

\bibliography{biblio}% common bib file
%% if required, the content of .bbl file can be included here once bbl is generated

% Version 1.2 of SN LaTeX, November 2022
%
% See section 11 of the User Manual for version history 
%
%%%%%%%%%%%%%%%%%%%%%%%%%%%%%%%%%%%%%%%%%%%%%%%%%%%%%%%%%%%%%%%%%%%%%%
%%                                                                 %%
%% Please do not use \input{...} to include other tex files.       %%
%% Submit your LaTeX manuscript as one .tex document.              %%
%%                                                                 %%
%% All additional figures and files should be attached             %%
%% separately and not embedded in the \TeX\ document itself.       %%
%%                                                                 %%
%%%%%%%%%%%%%%%%%%%%%%%%%%%%%%%%%%%%%%%%%%%%%%%%%%%%%%%%%%%%%%%%%%%%%

%%\documentclass[referee,sn-basic]{sn-jnl}% referee option is meant for double line spacing

%%=======================================================%%
%% to print line numbers in the margin use lineno option %%
%%=======================================================%%

%%\documentclass[lineno,sn-basic]{sn-jnl}% Basic Springer Nature Reference Style/Chemistry Reference Style

%%======================================================%%
%% to compile with pdflatex/xelatex use pdflatex option %%
%%======================================================%%

%%\documentclass[pdflatex,sn-basic]{sn-jnl}% Basic Springer Nature Reference Style/Chemistry Reference Style


%%Note: the following reference styles support Namedate and Numbered referencing. By default the style follows the most common style. To switch between the options you can add or remove “Numbered” in the optional parenthesis. 
%%The option is available for: sn-basic.bst, sn-vancouver.bst, sn-chicago.bst, sn-mathphys.bst. %  
 
%%\documentclass[sn-nature]{sn-jnl}% Style for submissions to Nature Portfolio journals
\documentclass[sn-basic,Numbered]{sn-jnl}% Basic Springer Nature Reference Style/Chemistry Reference Style
%%\documentclass[sn-mathphys,Numbered]{sn-jnl}% Math and Physical Sciences Reference Style
%%\documentclass[sn-aps]{sn-jnl}% American Physical Society (APS) Reference Style
%%\documentclass[sn-vancouver,Numbered]{sn-jnl}% Vancouver Reference Style
%%\documentclass[sn-apa]{sn-jnl}% APA Reference Style 
%%\documentclass[sn-chicago]{sn-jnl}% Chicago-based Humanities Reference Style
%%\documentclass[default]{sn-jnl}% Default
%%\documentclass[default,iicol]{sn-jnl}% Default with double column layout

%%%% Standard Packages
%%<additional latex packages if required can be included here>

\usepackage{graphicx}%
\usepackage{multirow}%
\usepackage{amsmath,amssymb,amsfonts}%
\usepackage{amsthm}%
\usepackage{mathrsfs}%
\usepackage[title]{appendix}%
\usepackage{xcolor}%
\usepackage{textcomp}%
\usepackage{manyfoot}%
\usepackage{booktabs}%
\usepackage{algorithm}%
\usepackage{algorithmicx}%
\usepackage{algpseudocode}%
\usepackage{listings}%
%%%%

%%%%%=============================================================================%%%%
%%%%  Remarks: This template is provided to aid authors with the preparation
%%%%  of original research articles intended for submission to journals published 
%%%%  by Springer Nature. The guidance has been prepared in partnership with 
%%%%  production teams to conform to Springer Nature technical requirements. 
%%%%  Editorial and presentation requirements differ among journal portfolios and 
%%%%  research disciplines. You may find sections in this template are irrelevant 
%%%%  to your work and are empowered to omit any such section if allowed by the 
%%%%  journal you intend to submit to. The submission guidelines and policies 
%%%%  of the journal take precedence. A detailed User Manual is available in the 
%%%%  template package for technical guidance.
%%%%%=============================================================================%%%%

%\jyear{2021}%

%% as per the requirement new theorem styles can be included as shown below
\theoremstyle{thmstyleone}%
\newtheorem{theorem}{Theorem}%  meant for continuous numbers
%%\newtheorem{theorem}{Theorem}[section]% meant for sectionwise numbers
%% optional argument [theorem] produces theorem numbering sequence instead of independent numbers for Proposition
\newtheorem{proposition}[theorem]{Proposition}% 
%%\newtheorem{proposition}{Proposition}% to get separate numbers for theorem and proposition etc.

\theoremstyle{thmstyletwo}%
\newtheorem{example}{Example}%
\newtheorem{remark}{Remark}%

\theoremstyle{thmstylethree}%
\newtheorem{definition}{Definition}%

\raggedbottom
%%\unnumbered% uncomment this for unnumbered level heads

%------------------------- Text Count -----------------------------
\usepackage{verbatim}
%TC:group table 0 1
%TC:group tabular 1 1
\newcommand{\detailtexcount}[1]{%
  \immediate\write18{texcount -merge -sum -q sn-article.tex sn-bibliography.bib > sn-article.wcdetail }%
  \verbatiminput{sn-article.wcdetail}%
}

\newcommand{\quickwordcount}[1]{%
  \immediate\write18{texcount -1 -sum -merge -q sn-article.tex sn-bibliography.bib > sn-article-words.sum }%
  \input{sn-article-words.sum} words%
}

\newcommand{\quickcharcount}[1]{%
  \immediate\write18{texcount -1 -sum -merge -char -q sn-article.tex sn-bibliography.bib > sn-article-chars.sum }%
  \input{sn-article-chars.sum} characters (not including spaces)%
}

%---------------------------------------------------------------- -

\begin{document}

\title[Bayesian Active Learning in MSK Segmentation]{Hybrid Representation-Enhanced Sampling for Bayesian Active Learning in Musculoskeletal Segmentation of Lower Extremities}

%%=============================================================%%
%% Prefix	-> \pfx{Dr}
%% GivenName	-> \fnm{Joergen W.}
%% Particle	-> \spfx{van der} -> surname prefix
%% FamilyName	-> \sur{Ploeg}
%% Suffix	-> \sfx{IV}
%% NatureName	-> \tanm{Poet Laureate} -> Title after name
%% Degrees	-> \dgr{MSc, PhD}
%% \author*[1,2]{\pfx{Dr} \fnm{Joergen W.} \spfx{van der} \sur{Ploeg} \sfx{IV} \tanm{Poet Laureate} 
%%                 \dgr{MSc, PhD}}\email{iauthor@gmail.com}
%%=============================================================%%

\author*[1]{\fnm{Ganping} \sur{Li}}\email{li.ganping.lc2@is.naist.jp}

\author[1]{\fnm{Yoshito} \sur{Otake}}\email{otake@is.naist.jp}

\author[1]{\fnm{Mazen} \sur{Soufi}} \email{msoufi@is.naist.jp}

\author[2]{\fnm{Masashi} \sur{Taniguchi}}\email{taniguchi.masashi.7a@kyoto-u.ac.jp}

\author[2]{\fnm{Masahide} \sur{Yagi}}\email{yagi.masahide.5s@kyoto-u.ac.jp}

\author[2]{\fnm{Noriaki} \sur{Ichihashi}}\email{ichihashi.noriaki.5z@kyoto-u.ac.jp}

\author[3]{\fnm{Keisuke} \sur{Uemura}}\email{surmountjp@gmail.com}

\author[4]{\fnm{Masaki} \sur{Takao}}\email{takao.masaki.ti@ehime-u.ac.jp}

\author[3]{\fnm{Nobuhiko} \sur{Sugano}}\email{n-sugano@umin.net}

\author[1]{\fnm{Yoshinobu} \sur{Sato}}\email{yoshi@is.naist.jp}


\affil*[1]{\orgdiv{Division of Information Science, Graduate School of Science and Technology}, \orgname{Nara Institute of Science and Technology}, \orgaddress{\street{8916-5 Takayama}, \city{Ikoma}, \postcode{630-0192}, \state{Nara}, \country{Japan}}}

\affil[2]{\orgdiv{Human Health Sciences, Graduate School of Medicine}, \orgname{Kyoto University}, \orgaddress{\street{53-Kawahara-cho, Shogoin}, \city{Sakyo-ku}, \postcode{606-8507}, \state{Kyoto}, \country{Japan}}}

\affil[3]{\orgdiv{Department of Orthopedic Surgery, Osaka University Graduate School of Medicine}, \orgname{Osaka University}, \orgaddress{\street{2-2 Yamadaoka}, \city{Suita}, \postcode{565-0871}, \state{Osaka}, \country{Japan}}}

\affil[4]{\orgdiv{Department of Bone and Joint Surgery, School of Medicine}, \orgname{Ehime University}, \orgaddress{\street{454 Shitsugawa}, \city{Toon}, \postcode{791-0295}, \state{Ehime}, \country{Japan}}}

%------------------------- Text Count -------------------------------
% Don't count these!
% TC:ignore
% \quickwordcount{sn-article}
% \quickcharcount{sn-article}
% \detailtexcount{sn-article}
% TC:endignore

%--------------------------------------------------------------------

%%==================================%%
%% sample for unstructured abstract %%
%%==================================%%

\abstract{Purpose: Obtaining manual annotations to train deep learning (DL) models for auto-segmentation is often time-consuming. Uncertainty-based Bayesian active learning (BAL) is a widely-adopted method to reduce annotation efforts. Based on BAL, this study introduces a hybrid representation-enhanced sampling strategy that integrates density and diversity criteria to save manual annotation costs by efficiently selecting the most informative samples.

Methods: The experiments are performed on two lower extremity (LE) datasets of MRI and CT images by a BAL framework based on Bayesian U-net. Our method selects uncertain samples with high density and diversity for manual revision, optimizing for maximal similarity to unlabeled instances and minimal similarity to existing training data. We assess the accuracy and efficiency using Dice and a proposed metric called reduced annotation cost (RAC), respectively. We further evaluate the impact of various acquisition rules on BAL performance and design an ablation study for effectiveness estimation.

Results: The proposed method showed superiority or non-inferiority to other methods on both datasets across two acquisition rules, and quantitative results reveal the pros and cons of the acquisition rules. Our ablation study in volume-wise acquisition shows that the combination of density and diversity criteria outperforms solely using either of them in musculoskeletal segmentation.

Conclusion: Our sampling method is proven efficient in reducing annotation costs in image segmentation tasks. The combination of the proposed method and our BAL framework provides a semi-automatic way for efficient annotation of medical image datasets.}

%%================================%%
%% Sample for structured abstract %%
%%================================%%

% \abstract{\textbf{Purpose:} The abstract serves both as a general introduction to the topic and as a brief, non-technical summary of the main results and their implications. The abstract must not include subheadings (unless expressly permitted in the journal's Instructions to Authors), equations or citations. As a guide the abstract should not exceed 200 words. Most journals do not set a hard limit however authors are advised to check the author instructions for the journal they are submitting to.
% 
% \textbf{Methods:} The abstract serves both as a general introduction to the topic and as a brief, non-technical summary of the main results and their implications. The abstract must not include subheadings (unless expressly permitted in the journal's Instructions to Authors), equations or citations. As a guide the abstract should not exceed 200 words. Most journals do not set a hard limit however authors are advised to check the author instructions for the journal they are submitting to.
% 
% \textbf{Results:} The abstract serves both as a general introduction to the topic and as a brief, non-technical summary of the main results and their implications. The abstract must not include subheadings (unless expressly permitted in the journal's Instructions to Authors), equations or citations. As a guide the abstract should not exceed 200 words. Most journals do not set a hard limit however authors are advised to check the author instructions for the journal they are submitting to.
% 
% \textbf{Conclusion:} The abstract serves both as a general introduction to the topic and as a brief, non-technical summary of the main results and their implications. The abstract must not include subheadings (unless expressly permitted in the journal's Instructions to Authors), equations or citations. As a guide the abstract should not exceed 200 words. Most journals do not set a hard limit however authors are advised to check the author instructions for the journal they are submitting to.}

\keywords{Active learning, Bayesian deep learning, Image segmentation, Bayesian Uncertainty}

%%\pacs[JEL Classification]{D8, H51}

%%\pacs[MSC Classification]{35A01, 65L10, 65L12, 65L20, 65L70}

\maketitle

\section{Introduction}\label{sec1}

Medical image segmentation is crucial in extracting quantitative imaging markers for better observations of anatomical or pathological structure changes to improve medical diagnosis and treatment \cite{ogawa2020validation, yagi2022age}. However, obtaining manual annotations for training deep learning (DL) models is often time-consuming, resulting in insufficient model performance \cite{sourati2018active}. Active learning (AL) is a widely-adopted approach to address the above-mentioned issue \cite{settles2008analysis}. The method is regarded as a training schema that reduces annotation effort by sequentially annotating the most informative instances. It involves iterative steps where an AL framework selects a batch of samples from an unlabeled pool to be manually annotated by annotators and subsequently added to the training pool. Afterward, a new model is trained on the updated training pool, superseding the previous model in the framework. Though many recent studies have proposed competitive strategies to address the issue, the best sampling policy is still a matter of debate \cite{budd2021survey}. 

Prior AL approaches focus on selecting samples with high model uncertainty \cite{gal2016dropout,gal2017deep,lakshminarayanan2017simple}, which is called uncertainty-based sampling. Among the uncertain samples, two criteria have been used for further selection \cite{yang2017suggestive,hiasa2019automated, ozdemir2021active, smailagic2018medal, nath2020diminishing}. One criterion \cite{yang2017suggestive,hiasa2019automated} seeks to select representative samples of high-density dominant classes in data distribution by maximizing their similarity to the unlabeled data. Alternatively, the others \cite{smailagic2018medal,nath2020diminishing} select samples minimizing the similarity between the chosen samples and the existing labeled data, ensuring diversity and less redundancy. Although the two criteria of density and diversity have been integrated for classification tasks in both medical and non-medical fields \cite{liu2022survey}, to the best of our knowledge, no existing work has applied this integration to medical image segmentation.

In this study, we introduce an AL scheme incorporating the two criteria above to ensure the chosen samples' representativeness and the labeled data's diversity. In order to further identify the informative instances, we implement a Bayesian active learning (BAL) framework based on Bayesian U-net \cite{hiasa2019automated} for uncertainty estimation and sampling. Since CT and MR data typically consist of volumetric (3D) images, volume-wise sample acquisition is preferable. However, the investigation of AL characteristics in volume-wise acquisition remains inadequate as prior research has mainly addressed slice (2D image)-wise acquisition, except for \cite{nath2020diminishing, ozdemir2021active}. Therefore, our proposed approach will be assessed in the volume-wise acquisition in addition to the slice-wise one. In brief, our contributions can be summarized as follows:

\begin{quote}
    \begin{itemize}
        \item We proposed a hybrid representation-enhanced sampling strategy that integrates similarity measures to detect high-density samples while ensuring diversity and less redundancy. The method is adopted to a BAL framework to prioritize both uncertainty and representativeness of the queried samples for medical image segmentation.
        \item We validate our method on two lower extremity (LE) image datasets of MRI and CT, and further estimated the impact of the volume-wise acquisition in addition to the slice-wise on BAL performance and annotation efficiency, addressing the insufficiency in previous works.
    \end{itemize}
\end{quote}

\section{Related work}\label{sec2}

\subsection{Uncertainty-based sampling}\label{subsec1}
Uncertainty-based sampling assesses a sample's informativeness by measuring the uncertainty of a trained DL model's prediction, where a higher uncertainty indicates greater informativeness. Gal et al. \cite{gal2016dropout,gal2017deep} introduced a prevalent implementation to approximate Bayesian inference using Monte Carlo (MC) dropouts. The method efficiently estimates model uncertainty by measuring each test sample's degree of difference and identifying the training data's deficiency at the inference step \cite{hiasa2019automated}. Nevertheless, given that a model in an early stage of AL tends to be uncertain for similar types of instances, relying solely on uncertainty approaches may skew the model to focus on a particular area of the data distribution within the target domain.

\subsection{Representativeness-based sampling}\label{subsec2}
Representativeness-based sampling is widely employed with uncertainty approaches \cite{budd2021survey} in medical analysis, mainly grouped by density-based and diversity-based approaches. As a typical density method, Yang et al. \cite{yang2017suggestive} utilized similarity measures to select dense samples that offer the most comprehensive representation of the unlabeled pool with a step-by-step optimization. In \cite{ozdemir2021active}, a variational autoencoder (VAE)-based density sampling was proposed for the same purpose, although it requires an auxiliary model. These methods, however, might skew the training pool by selecting only the majority when handling an imbalanced dataset, especially in the early iterations. In order to tackle this challenge, diversity-based approaches have been proposed. Smailagic et al. \cite{smailagic2018medal} quantified the dissimilarity between feature maps of the chosen samples and the training pool, intending to maximize this dissimilarity. Then, Nath et al. \cite{nath2020diminishing} adopted mutual information as a regularizer to ensure diversity in training data with a similar aim. However, solely maximizing diversity may result in querying outliers. Thus, developing a hybrid sampling strategy that maximizes the density and diversity of the training data would be necessary.

\section{Materials and methods}\label{sec3}

\subsection{Dataset}\label{subsec3}

% Figure environment removed

We gathered and annotated two LE datasets (Fig. \ref{fig1}) used in our studies; 1) a T-1 weighted MRI dataset with four quadriceps muscles annotated \cite{fukumoto2022influence} and 2) a CT dataset with 22 musculoskeletal structures (19 muscles and 3 bones) annotated. Notice that the manual labels of the unlabeled pool were used for evaluation only.

1) MRI dataset: The dataset contained 119 volumes (21490 axial slices) of 82 elders ($>60$ years) and 37 youths (20-39 years) with manual annotation of four quadriceps muscles (Rectus femoris, Vastus lateralis, Vastus intermedius, and Vastus medialis) by annotators. Each MR volume contains 163 to 201 (182 on average) slices. The images were resized to $256\times256\times n$ with voxel spacing from $[0.5, 0.5, 4]$ mm to $[1, 1, 4]$ mm, and normalized from $[0, 1000]$ to $[0, 1]$. The dataset was then divided into 1/89/9/20 for the training/unlabeled pool/validation/testing for initialization of all AL experiments.

2) CT dataset: This dataset consisted of 30 CT volumes (17909 axial slices) labeled with 22 musculoskeletal structures (19 muscles and 3 bones). Each CT volume contained 526 to 700 (599 on average) slices with a matrix size of $512\times512$. We resized the images to $256\times256\times n$ with voxel spacing of $[1.4, 1.4, 1]$ mm, normalized them from $[-150, 350]$ to $[0, 1]$, and divided the dataset into 1/24/1/4 for the training/unlabeled pool/validation/testing.

\subsection{Sampling techniques}\label{subsec4}
In this section, we present sampling techniques employed within our BAL framework. We start by introducing uncertainty sampling to select the most uncertain samples. Next, we incorporate a hybrid scoring approach designed to select high-density and diverse samples from the uncertain subset, thereby ensuring representativeness among the unlabeled data.

\textbf{Bayesian uncertainty sampling.}
Our uncertainty estimation step follows the method described in \cite{hiasa2019automated}, which investigates the model uncertainty in a scalable manner by the approximate Bayesian inference of predictive distributions (details shown in Online Resource 1.1). We implemented a dropout-based Bayesian U-net for multi-class segmentation and uncertainty estimation, where an average uncertainty for each class is defined by
\begin{equation}\label{eq1}
    m_{unc}(y=l) = \frac{1}{N}\sum\limits_{n=1}^{N}\text{var}[p(y=l\mid x, \Theta_{t})^{(n)}]
\end{equation}
where $N$ is the number of pixels of input $x$ and var$[p(y=l\mid x, \Theta_{t})^{(n)}]$ indicates the prediction variance under $T$ times Bayesian inference at pixel $n$. The equation above is cited and summarized from \cite{ozdemir2021active}. 

\textbf{Hybrid representation-enhanced sampling.}
We introduce a hybrid scoring approach to select high-density samples following the method described in \cite{yang2017suggestive,hiasa2019automated} with a constraint to maintain diversity \cite{nath2020diminishing}. 

Given an unlabeled pool $\mathcal{D}_u$, a training pool $\mathcal{D}_t$, and a subset of uncertain images $\mathcal{D}_{c}\subseteq\mathcal{D}_{u}$, the algorithm first measures the norm of cosine similarity $norm(sim(I_{i}^{c}, I_{j}^{u}))$ between $\mathcal{D}_{c}$ and $\mathcal{D}_{u}$ for representative samples, where $I_{i}^{c}$ and $I_{j}^{u}$ are the $i^{th}$ and $j^{th}$ images from $\mathcal{D}_{c}$ and $\mathcal{D}_{u}$, respectively. Next, a regularization term of the mutual information's norm $norm(mi(I_{i}^{c}, I_{k}^{t}))$ between $\mathcal{D}_{c}$ and $\mathcal{D}_{t}$ minimizes the redundancy in the training pool $\mathcal{D}_{t}$ while encouraging minority samples. Overall, we select samples maximizing
\begin{equation}\label{eq2}
    m_{repr} = \underbrace{norm(sim(\mathcal{D}_{c}, \mathcal{D}_{u}))}_\text{density module} - \lambda\cdot\underbrace{norm(mi(\mathcal{D}_{c}, \mathcal{D}_{t}))}_\text{diversity module}
\end{equation}
where hyper-parameter $\lambda$ determines the balance between the density and diversity of the chosen samples. The top-$k$ samples will then be annotated and added to the training pool $\mathcal{D}_{t}$. Details of the step-by-step optimization algorithm extended from \cite{hiasa2019automated, nath2020diminishing} are demonstrated in Online Resource 1.2.

\subsection{Bayesian active learning}\label{subsec6}
We present a BAL framework for medical image segmentation to validate the proposed method, as illustrated in Fig. \ref{fig2} (a). Starting with a Bayesian U-net trained on a limited number of labeled data, our schema iteratively selects uncertain and representative samples, revises them by annotators, and incorporates them into the training set.

% Figure environment removed

\textbf{Segmentation model.}
Our segmentation tasks are conducted on a 4-layer Bayesian U-net \cite{hiasa2019automated} of 2.73 million trainable parameters whose architecture is depicted in Fig. \ref{fig2} (b). To tackle class imbalance, we employ a multi-class focal loss \cite{lin2017focal} with class weighter $\alpha=0.67$ and regularizer $\gamma=2$. Our experiments use no augmentation since the study focuses on sampling strategy performance. During the training phase, we use the AdamW optimizer with decay weights of $1 \times 10^{-5}$, a learning rate of $4 \times 10^{-4}$, and a batch size of $8$. After $40000$ iterations training, the model checkpoint with the highest Dice similarity score (DSC) in the validation set will be selected for inference. The dropout rate is $0.5$ with $T=10$ times MC dropouts during the training and inference phases.

\textbf{Acquisition rules.}
Selecting all sequential images from one volume (i.e., \textit{volume-wise} acquisition) may introduce redundant information to the training pool \cite{chen2023survey}, as neighboring images usually exhibit similar features. On the other hand, from an annotator's point of view, annotating an entire volume is more efficient than annotating an equal number of slices among different volumes (i.e., \textit{slice-wise} acquisition), as it requires less time to operate the software to locate the target slice, and the annotation of consecutive slices can leverage semi-automatic tools for the slice interpolation. In order to analyze this trade-off, all experiments are conducted under slice-wise and volume-wise acquisition.

\textbf{Sampling strategies.}
Our method was compared with several recent AL algorithms in medical image segmentation tasks, including random selection, uncertainty-only selection \cite{gal2017deep}, two state-of-the-art (SOTA) methods \cite{hiasa2019automated, nath2020diminishing}, and the proposed method, with details as follows:

\noindent$\rightarrow$ BAL$_{rand}$: a simple baseline of randomly selecting samples from $\mathcal{D}_{u}$.

\noindent$\rightarrow$ BAL$_{unc}$: select the most uncertain samples from $\mathcal{D}_{u}$ based on Section \ref{subsec4}

\noindent$\rightarrow$ BAL$_{unc+sim}$: an implementation of \cite{hiasa2019automated}, consisting of uncertainty and density-based representative sampling. 

\noindent$\rightarrow$ BAL$_{unc+mi}$: a resemble method of \cite{nath2020diminishing} combining uncertainty sampling and mutual information-based diversity constraint, where our Bayesian estimation replaces the uncertainty calculation with ``Delete Flag'' set to $1$.

\noindent$\rightarrow$ BAL$_{proposed}$: Our proposed method that selects samples combining uncertainty sampling and hybrid representation-enhanced sampling (Section \ref{subsec4}), with the hyperparameter $\lambda$ empirically set to 0.5 and 0.25 for volume-wise and slice-wise acquisition, respectively. 

\textbf{Evaluation metrics}
The segmentation accuracy was assessed at each acquisition step using DSC. To quantify the manual labor saved by our BAL framework, we proposed a metric called $reduced$ $annotation$ $cost$ (RAC) as
\begin{equation}\label{eq3}
    RAC(I)=1-\frac{|I^{revised}|}{|I^{ROI}|}
\end{equation}
with the queried label image $I$. $| I^{revised}|$ denotes the number of pixels to be revised, whereas $| I^{ROI} |$ is the number of non-background pixels in the corresponding ground truth. Unlike the manual annotation cost (MAC) used in \cite{hiasa2019automated} that considers all image pixels, RAC considers non-background pixels, as annotation tools initially assign a zero value to all pixels and annotators modify only non-background ones.

\section{Results}\label{sec4}
Comparative results on both MRI and CT datasets. t-SNE maps and raw data of DSC and RAC for all strategies at each active iteration are available in Online Resource 1.4 and Online Resource 2. 
\subsection{MRI dataset}\label{subsec7}
% Figure environment removed
\begin{table}[t]
\begin{center}
\resizebox{1.0\textwidth}{!}{%
\begin{minipage}{1.05\textwidth}
\caption{Comparison of reduced annotation cost (RAC) metric on MRI dataset presented as mean $\pm$ std \%, where a larger value indicates fewer pixels to be revised. Notice that the mean RAC of the upper bound is $88.2$\% as the model trained on the fully labeled dataset.}\label{tab2}
\begin{tabular*}{\textwidth}{@{\extracolsep{\fill}}lcccccc@{\extracolsep{\fill}}}
\toprule%
& \multicolumn{6}{@{}c@{}}{Annotation\footnotemark[1]\% (number of active iteration)} \\\cmidrule{2-7}%
RAC \% & 1.25 (0) & 2.50 (1) & 5.00 (3) & 7.50 (5) & 10.0 (7) & 12.5 (9)\\
\midrule
BAL$_{rand}$  & $\mathbf{48.3}\pm1.0$ & $59.1\pm0.6$ & $73.7\pm0.1$ & $79.5\pm0.0$ & $82.3\pm0.0$ & $83.2\pm0.0$\\
BAL$_{unc}$  & $46.3\pm1.4$ & $64.9\pm0.1$ & $76.9\pm0.0$ & $81.4\pm0.0$ & $84.0\pm0.0$ & $85.0\pm0.0$\\
BAL$_{unc+mi}$  & $43.5\pm1.5$ & $66.9\pm0.4$ & $75.8\pm0.1$ & $81.0\pm0.0$ & $83.8\pm0.0$ & $84.9\pm0.0$\\
BAL$_{unc+sim}$  & $45.2\pm1.1$ & $67.3\pm0.2$ & $77.3\pm0.0$ & $\mathbf{82.5}\pm0.0$ & $83.9\pm0.0$ & $84.8\pm0.0$\\
BAL$_{proposed}$  & $48.0\pm0.6$ & $\mathbf{69.2}\pm0.3$ & $\mathbf{79.4}\pm0.0$ & $82.1\pm0.0$ & $\mathbf{84.2}\pm0.0$ & $\mathbf{85.3}\pm0.0$\\
\botrule
\end{tabular*}
\footnotetext{Exp. settings: volume-wise acquisition; 4-layer network of 2.73 million trainable parameters.}
\footnotetext[1]{Percentile of the annotated data used for model training.}
\end{minipage}
}
\end{center}
\end{table}

Initialized  with one randomly selected volume of 180 slices, we selected and revise one volume (for volume-wise selection) or 180 slices (average slice count per volume in the unlabeled pool, for slice-wise selection) from $\mathcal{D}_{u}$ and added them to $\mathcal{D}_{t}$ at each iteration. Data partitioning of volume-wise experiments was conducted three times with different random seeds to ensure reliability. The DSCs of all methods on the MRI dataset are shown in Fig. \ref{fig3}, where BAL$_{proposed}$ is depicted as the purple line. Comparing Fig. \ref{fig3} (a) with (b), we can infer that slice-wise acquisition systematically surpasses volume-wise by around 0.1 DSC, reaching close to the upper bound within five active iterations.

Focusing on the method comparisons, the DSC of BAL$_{proposed}$ was consistently superior to the SOTA BAL$_{unc+sim}$ and BAL$_{unc+mi}$. To estimate framework efficiency, we quantitatively estimated the RAC of BAL methods in the volume-wise acquisition, as shown in Table \ref{tab2}. The table illustrates that our proposed method contributed the most to reducing the annotation cost. 

\subsection{CT dataset}\label{subsec8}
Experiments on the CT dataset employed similar settings in Section. \ref{subsec7}, except that the slice-wise experiment chose 540 slices per iteration corresponding to the size of one volume in $\mathcal{D}_{u}$. The DSC results in Fig. \ref{fig4} show a similar trend to those obtained from the MRI dataset. When focusing on volume-wise acquisition (Fig. \ref{fig4} (b)),  BAL$_{proposed}$ was superior or non-inferior to other methods,  especially at the early stages of the 1st and 3rd active iterations. Table \ref{tab3} lists the RAC results in line with those shown in Section \ref{subsec7}. Our method achieved the upper bound RAC using only 12\% and 40\% of the full training data with slice-wise and volume-wise acquisition, respectively. 
% Figure environment removed

\begin{table}[t]
\begin{center}
\resizebox{1.0\textwidth}{!}{%
\begin{minipage}{1.05\textwidth}
\caption{Comparison of RAC metric on the CT dataset presented as mean $\pm$ std \%, where a larger value denotes fewer pixels from the prediction to be revised. Notice that the mean RAC of the upper bound is $88.9$\% as the model trained on the fully labeled dataset.}\label{tab3}
\begin{tabular*}{\textwidth}{@{\extracolsep{\fill}}lcccccc@{\extracolsep{\fill}}}
\toprule%
& \multicolumn{6}{@{}c@{}}{Annotation\footnotemark[1] \% (number of active iteration)} \\\cmidrule{2-7}%
RAC \% & 4.0 (0) & 8.0 (1) & 16 (3) & 24 (5) & 32 (7) & 40 (9)\\
\midrule
BAL$_{rand}$  & $60.4\pm0.7$ & $69.9\pm0.1$ & $78.1\pm0.2$ & $81.2\pm0.3$ & $84.9\pm0.0$ & $86.8\pm0.1$\\
BAL$_{unc}$  & $60.2\pm0.6$ & $75.0\pm0.3$ & $83.6\pm0.1$ & $\mathbf{87.6}\pm0.0$ & $\mathbf{88.0}\pm0.0$ & $88.7\pm0.0$\\
BAL$_{unc+mi}$  & $60.0\pm0.5$ & $75.1\pm0.1$ & $83.3\pm0.0$ & $86.7\pm0.0$ & $87.9\pm0.0$ & $88.4\pm0.0$\\
BAL$_{unc+sim}$  & $\mathbf{60.5}\pm0.4$ & $75.7\pm0.0$ & $83.6\pm0.0$ & $86.0\pm0.0$ & $87.9\pm0.0$ & $88.4\pm0.0$\\
BAL$_{proposed}$  & $60.4\pm0.8$ & $\mathbf{76.5}\pm0.3$ & $\mathbf{85.0}\pm0.0$ & $86.6\pm0.0$ & $87.9\pm0.0$ & $\mathbf{88.9}\pm0.0$\\
\botrule
\end{tabular*}
\footnotetext{Exp. settings: volume-wise acquisition; 4-layer network of 2.73m trainable parameters.}
\footnotetext[1]{Percentile of the annotated data used for model training.}
% \footnotetext[2]{Initialization results trained on randomly selected data}
\end{minipage}
}
\end{center}
\end{table}

\begin{table}[t]
\begin{center}
\resizebox{0.75\textwidth}{!}{%
\begin{minipage}{0.85\textwidth}
\caption{Ablation study of different components in BAL$_{proposed}$ on the two datasets. The results are collected at the 3$^{th}$ active iteration.}\label{tab4}
\begin{tabular}{@{}l|c|cc|cc|cc@{}}
\toprule
 \multirow{3}{*}{\textbf{Method}} & \multicolumn{3}{@{}c@{}}{AL Selection} & \multicolumn{2}{|@{}c@{}}{DSC} & \multicolumn{2}{|@{}c@{}}{RAC}\\\cmidrule{2-4}\cmidrule{5-6}\cmidrule{7-8}
 & \textbf{UNC} & \textbf{MI} & \textbf{SIM} & \textbf{MRI}\footnotemark[1] & \textbf{CT}\footnotemark[2] & \textbf{MRI}\footnotemark[1] & \textbf{CT}\footnotemark[2]\\
\midrule
BAL$_{rand}$         &              &                &           &  81.8 & 79.7 & 73.7 & 78.1\\
\midrule
BAL$_{unc}$ \cite{gal2017deep}        & \checkmark   &                &           &  84.3 & \textbf{84.8} & 76.9 & 83.6 \\
BAL$_{unc+mi}$ \cite{nath2020diminishing}      & \checkmark   & \checkmark     &           &  83.3 & 84.3 & 75.8 & 83.3 \\
BAL$_{unc+sim}$ \cite{hiasa2019automated}     & \checkmark   &                & \checkmark  & \textbf{84.6} & 84.6 & \textbf{77.3} & \textbf{83.6} \\
\midrule
BAL$_{proposed}$     & \checkmark   & \checkmark     & \checkmark  & \textbf{86.6} & \textbf{85.5} & \textbf{79.4} & \textbf{85.0} \\
\botrule
\end{tabular}
\footnotetext{\textbf{UNC}, \textbf{MI}, and \textbf{SIM} denote uncertainty module, diversity-enhanced module by mutual information, and density-enhanced module by cosine similarity, respectively.}
\footnotetext[1]{The quadriceps dataset with 5\% data annotated.}
\footnotetext[2]{The musculoskeletal dataset with 16\% data annotated.}
\end{minipage}
}
\end{center}
\end{table}

\subsection{Ablation study}\label{subsec9}
To illustrate the contribution of each component in BAL$_{proposed}$, we performed an ablation study shown in Table \ref{tab4}. Upon the comparison of BAL$_{rand}$ and BAL$_{unc}$, the uncertainty-based sampling significantly contributes 2.5\% and 5.1\% of DSC, and 3.2\% and 5.5\% of RAC on the MRI and the CT dataset, respectively. The results of BAL$_{mi}$ and BAL$_{sim}$ indicate the impact of enhanced diversity and density. Compared to BAL$_{unc}$, solely incorporating an MI-based diversity regularizer can deteriorate the BAL performance. However, BAL$_{proposed}$ suggests that integrating the regularizer with a density-enhanced module effectively counteracts its negative impact, as this combination tends to select fewer redundant samples or outliers.

\section{Discussion}\label{sec5}
 % Since our sampling algorithm (Online Resource 1. b) demonstrates sensitivity to increment size, the computational expense of slice-wise acquisition is exponentially higher.
We proposed a hybrid representation-enhanced sampling strategy in BAL by integrating density-based and diversity-based criteria and evaluated its performance on MRI and CT datasets. Slice-wise acquisition outperformed volume-wise using fewer annotations as it addresses redundant information and involves greater diversity by selecting slice images from multiple volumes. This approach prefers smaller regularization weighter $\lambda$ for milder diversity enhancement but necessitates more manual operations than volume-wise acquisition, which handles one volume per iteration. Hence, it is vital to consider the trade-off between the annotation time-cost and iteration times for two acquisition rules. Our observation reveals that, in the context of sequential image datasets, the volume-wise acquisition shows preferable adaptability and efficacy.

One can infer from Fig. \ref{fig3} and \ref{fig4}, and Table \ref{tab2} and \ref{tab3} that BAL$_{proposed}$ outperforms two SOTA samplings in early iterations, though this superiority consistently decreases over time. One possible explanation is that BAL$_{proposed}$ identified the key samples on both datasets ahead of other methods. Another point is that although BAL$_{proposed}$ showed minor DSC improvements for the CT dataset, it excelled in RAC estimation during the 1st and 3rd active iterations. This is because mean DSC was used for multi-class segmentation, while RAC focused solely on misclassified pixels unaffected by the number of classes.

Despite proven effectiveness, our study shows several limitations. Firstly, a greedy algorithm implemented our hybrid scoring approach, whose computational time grows exponentially with increased dataset size and acquisition granularity. Secondly, we have limited our analysis of the impact of two acquisition rules on LE datasets, while alternative conclusions may be reached on other datasets. Finally, our AL sampling strategy concentrates solely on leveraging the most valuable samples, neglecting the potential information within the remaining unlabeled data. Thus, future works shall include 1) algorithm optimization (e.g., VAE-based measures) for efficiency improvement, 2) extensive experiments on various datasets for quantitative estimation of acquisition rules' impact, and 3) incorporating the semi-supervised learning (SSL) \cite{nath2022warm} to unleash the potential of unlabeled data. 

\section{Conclusion}\label{sec6}
This paper has described a BAL framework based on Bayesian U-net that leverages the advantage of AL to reduce annotation efforts. At the algorithmic level, we introduced a novel hybrid representation-enhanced sampling that ensures high density and diversity of the training data to boost the BAL framework's performance. Moreover, we conducted a comprehensive study to reveal the impact of acquisition rules on BAL, as well as parameter sweeping for a real-world clinical setting. The experiment results indicated that our proposed sampling strategy outperforms SOTA representativeness-based sampling approaches on musculoskeletal segmentation. We also summarized previous works for a better comprehension of our experiments (Online Resource 1.3), and our code will be available on GitHub.\footnotemark[1]
\backmatter

\bmhead{Supplementary information}
Online Resource 1:  1) Details of Estimation of model uncertainty, 2) the hybrid representation-enhanced sampling algorithm, 3) a table summarizing previous works, and 4) t-SNE maps. 
Online Resource 2: Raw data of DSC and RAC for all methods and active iterations, available on GitHub.\footnotemark[1]
\footnotetext[1]{\url{https://github.com/RIO98/Hybrid-Representation-Enhanced-Bayesian-Active-Learning}.}
\bmhead{Acknowledgments}

This work was funded by MEXT/JSPS KAKENHI (19H01176, 20H04550, 20K19376, 21H03303, 21K16655, 21K18080).

\section*{Declarations}
\textbf{Conflict of interest} The authors declare that Yoshinobu Sato is a member of the IJCARS Editorial Board. \\


\noindent\textbf{Ethics approval} Ethical approval was obtained from the Institutional Review Boards (IRBs) of the institutions participating in this study (IRB approval numbers: 21115 for Osaka University Hospital, R1746-2 for Kyoto University Hospital, and 2020-M-7 for Nara Institute of Science and Technology.)

%%===========================================================================================%%
%% If you are submitting to one of the Nature Portfolio journals, using the eJP submission   %%
%% system, please include the references within the manuscript file itself. You may do this  %%
%% by copying the reference list from your .bbl file, paste it into the main manuscript .tex %%
%% file, and delete the associated \verb+\bibliography+ commands.                            %%
%%===========================================================================================%%

\bibliography{sn-bibliography}% common bib file
%% if required, the content of .bbl file can be included here once bbl is generated
%%\input sn-article.bbl


\end{document}

\end{document}
