\documentclass[submitting]{nst}
\usepackage{subfigure,dcolumn}
\usepackage{epstopdf}
\usepackage{mhchem}
\usepackage{upgreek}
\usepackage{graphicx}
\usepackage{xcolor}
\usepackage{amsmath}
\usepackage{ulem}
\usepackage{multirow}
\begin{document}

\title{Dark Count of 20-inch PMTs Generated by Natural Radioactivity}

\thanks{This work was supported by the National Natural Science Foundation of China No. 11875282, the Strategic Priority Research Program of the Chinese Academy of Sciences (Grant No. XDA100102).
}

\author{Yu Zhang}
\affiliation{Institute of High Energy Physics, Beijing 100049, China}
\affiliation{University of Chinese Academy of Sciences, Beijing 100049, China}

\author{Zhimin Wang}
\email[Email: ]{wangzhm@ihep.ac.cn}
\affiliation{Institute of High Energy Physics, Beijing 100049, China}
\affiliation{University of Chinese Academy of Sciences, Beijing 100049, China}
\affiliation{State Key Laboratory of Particle Detection and Electronics, Beijing 100049, China}

\author{Min Li}
\author{Caimei Liu}
\author{Narongkiat Rodphai}
\affiliation{Institute of High Energy Physics, Beijing 100049, China}
\affiliation{University of Chinese Academy of Sciences, Beijing 100049, China}

\author{Yongpeng Zhang}
\author{Jilei Xu}
\affiliation{Institute of High Energy Physics, Beijing 100049, China}
\affiliation{University of Chinese Academy of Sciences, Beijing 100049, China}
\affiliation{State Key Laboratory of Particle Detection and Electronics, Beijing 100049, China}

\author{Changgen Yang}
\affiliation{Institute of High Energy Physics, Beijing 100049, China}
\affiliation{University of Chinese Academy of Sciences, Beijing 100049, China}

\author{Yuekun Heng}
\affiliation{Institute of High Energy Physics, Beijing 100049, China}
\affiliation{University of Chinese Academy of Sciences, Beijing 100049, China}
\affiliation{State Key Laboratory of Particle Detection and Electronics, Beijing 100049, China}

\begin{abstract}
The primary objective of the JUNO experiment is to determine the ordering of neutrino masses using a 20-kton liquid-scintillator detector. The 20-inch photomultiplier tube (PMT) plays a crucial role in achieving excellent energy resolution of at least 3\,\% at 1\,MeV. Understanding the characteristics and features of the PMT is vital for comprehending the detector's performance, particularly regarding the occurrence of large pulses in PMT dark counts. This research paper aims to further investigate the origin of these large pulses in the 20-inch PMT dark count rate through measurements and simulations. The findings confirm that the main sources of the large pulses are natural radioactivity and muons striking the PMT glass. By analyzing the PMT dark count rate spectrum, it becomes possible to roughly estimate the radioactivity levels in the surrounding environment.
\end{abstract}

\keywords{photon detectors for UV, visible and IR photons (vacuum) (photomultipliers, HPDs, others), PMT, MCP-PMT, natural radioactivity, cosmic ray, glass, Cherenkov light}

\maketitle

\section{Introduction}
\label{1:intro}
Photomultiplier tubes (PMTs) are extensively utilized in particle physics experiments that require single-photon-sensitive light detection, such as Super-K\,\cite{Super-Kamiokande:1998uiq,PhysRevD.83.052010}, KamLAND\,\cite{PhysRevLett.90.021802}, SNO\,\cite{SNO}, IceCube\,\cite{PhysRevLett.110.131302}, Double Chooz\,\cite{chooz}, Daya Bay\,\cite{dayabay}, and RENO\,\cite{KIM201324}. The performance and characterization of PMTs have been extensively studied and well-understood\,\cite{AugerPMT,GE2016175,BorexinoPMT,DayabayPMT,ChoozPMT,HKPMT,JUNO3inchPMT,JUNOPMTinstr,KM3NeTPMT,JUNOPMTflasher,MCPPMT2018,YWang_newMCP,wavesamplingPMT,waveAnalysisHaiqiong}.

\par However, recent rare-event neutrino experiments worldwide, such as Double Chooz\,\cite{Abe_2016}, Daya Bay\,\cite{DWYER201330}, IceCube\,\cite{IceCube-inproceedings}, and RENO\,\cite{JANG2014145}, have encountered issues caused by large pulses in PMTs. It is known that the passage of a muon through the PMT glass (e.g., Hamamatsu PMT R5912) generates a large pulse due to Cherenkov radiation, as studied in\,\cite{PMTmuon2007,BAYAT20141,Zhang_2022}. Additionally, ongoing research is being conducted on scintillation glass\,\cite{glass-2015,TANG2022112585-2022,AMELINA2022121393-2022}. Studies have also focused on flasher and large pulses, as seen in\,\cite{Yang_2020,Qian_2020,JUNOPMTflasher}. To expand on previous investigations regarding muon-generated large pulses, we further explore the influence of natural radioactivity for a comprehensive understanding of related signals.

\par The Jiangmen Underground Neutrino Observatory (JUNO)\,\cite{JUNOCDR,JUNOphysics} is currently under construction in Jiangmen, Guangdong, China. JUNO aims to study neutrino mass ordering with 3\% energy resolution at 1 MeV, accurately determine neutrino oscillation parameters, and explore other aspects of neutrino physics using a 20-kton liquid scintillator monitored by up to 20,000 high quantum efficiency (QE) 20-inch PMTs. JUNO has selected two types of 20-inch PMTs\,\cite{JUNOdetector,JUNOPMTinstr}: 5,000 Hamamatsu Photonics K.K. (HPK, R12860) dynode PMTs\,\cite{HPK-R12860} and 15,000 newly developed MCP PMTs from North Night Vision Technology Co., LTD (MCP, GDB6201)\,\cite{NNVT-GDB6201-note}.

\par In this article, we present a detailed investigation of the dark count rate of 20-inch PMTs associated with natural radioactivity. Section\,\ref{1:largepulse} provides a brief description of the testing system and configurations. A dedicated simulation is performed for better understanding and is compared with the experimental measurements, as presented in section\,\ref{1:sim}. The measurement results are shown in section\,\ref{1:comp}.

\section{Large pulses from PMT dark count}
\label{1:largepulse}
The dark count in a photomultiplier tube (PMT) primarily arises from the thermal electron emission of the PMT photocathode when the PMT is in the dark. In general, the amplitude of dark counts is expected to be at the level of single photoelectron (SPE), mostly below 3 p.e. (photoelectrons), and the rate is much lower than 1 Hz for signals larger than 3 p.e. (assuming a DCR in SPE of approximately 10 kHz and a 10 ns coincidence window) \cite{HamManual,POLYAKOV201369}. To gain a better understanding of the sources contributing to the dark count, including thermal emission, flashers of the 20-inch PMT, muons, and natural radioactivity, a 20-inch PMT test system is employed to measure the dark count rate (DCR) for large pulses \cite{zhang2022study}. This measurement aims to investigate the DCR versus threshold, as well as the amplitude and charge spectra of the dark count.

\par The dark count of the 20-inch PMT is extensively examined through detailed measurements. Waveforms are captured during the threshold survey for both HPK and NNVT PMTs using the test system. The charge distributions of the DCR are depicted in Figure \ref{fig:dn:wave}, where all plots are normalized relative to the result obtained at the lowest threshold, considering events with amplitudes higher than 200 mV. Both types of PMTs exhibit similar overall trends in their charge spectra, showing a distinctive structure. The large signals ranging from approximately 20 p.e. to at least 150 p.e. in charge are primarily attributed to cosmic muons. On the other hand, signals ranging from approximately 3 p.e. to 20 p.e. in charge require further investigation and are the focal point of this study. Charge levels from zero to around 3 p.e. are primarily contributed by thermal electron emission. For a more comprehensive analysis, please refer to \cite{zhang2022study}.

% Figure environment removed

\par In a simulation, the charge output of PMT is represented in individual photoelectrons (p.e.) before undergoing further processing. As illustrated in Figure \ref{fig:rad-sim}, it is necessary to apply charge smearing to the simulated charge spectrum before comparing it with experimental measurements, particularly for the NNVT PMT, which exhibits a long tail in its charge spectrum \cite{JUNOPMTgain}. To achieve this, a measured charge spectrum of a single p.e.~(Figure \ref{fig:spe}) of NNVT PMT obtained using a pulsed light source\,\cite{PMTgainmodel1994,Luo_2019} is utilized. Each simulated event's charge is convoluted through a random sampling process using the single p.e.~spectrum. During the sampling, only events with a magnitude higher than 0.4 p.e. in the measured spectrum are considered. The charge spectrum of the HPK PMT simulation is not subjected to further processing since its single photoelectron spectrum closely follows a Gaussian distribution. The observed disparity in the charge spectra of simulation between the NNVT and HPK PMTs predominantly arises from the charge smearing applied to the NNVT PMT.

% Figure environment removed
\section{Simulation}
\label{1:sim}
The Geant4 simulation framework \cite{geant4} is employed to simulate the behavior of 20-inch PMTs in the presence of natural radioactivity and muons traversing the PMT glass bulb. These simulations play a crucial role in enhancing our understanding of the experimental measurements by providing insights into the underlying physics processes and interactions. Geant4 allows for the accurate modeling of particle interactions and the propagation of radiation through matter, enabling us to simulate the response of the PMTs to various sources of radiation. By comparing the simulation results with the experimental data, we can validate the simulation models and gain valuable insights into the performance of the 20-inch PMTs under different conditions.

\subsection{Thermal Emission}
\label{1:sim:thermal}
The primary source of dark count in PMTs is the thermal electron emission from the PMT photocathode in a dark environment. This emission process, known as thermal emission, involves the spontaneous transfer of internal energy from the thermal reservoir to electrons. The control and understanding of thermal emission are crucial due to its significance and its presence in various applications, including 20-inch PMTs where it serves as the main source of dark count.

\par Thermal electron emissions from the PMT photocathode events are typically independent of each other. In cases where multiple thermal electron emissions occur, their occurrences are generally consistent with a random distribution. To describe the interval between neighboring thermal electron emissions, an exponential distribution model is often adopted. In this study, we utilized the exponential distribution model as a toy Monte Carlo (MC) simulation to numerically characterize the contribution of thermal electrons to the dark count rate in PMTs.

\par By employing this model, we aim to gain further insight into the behavior and characteristics of thermal electron emissions in PMTs, enhancing our understanding of the dark count phenomenon and its underlying mechanisms.

\subsection{Natural Radioactivity}
\label{1:sim:radioactivity}
Radionuclides emit alpha, beta particles, and gamma radiations. Naturally occurring radioactive materials are widespread in nature, found in soil, underground rock layers, water bodies, the atmosphere, as well as plants and animals. The primary nuclides of interest include isotopes of uranium (U), thorium (Th), radium-226 (Ra-226), potassium-40 (K-40), polonium-210 (Po-210), lead-210 (Pb-210), tritium (H-3), and others. These radioactive nuclides maintain a dynamic equilibrium, resulting in a consistent level of natural radioactivity over an extended period.

\par In this study, we consider two sources of natural radioactivity: the PMT glass and the surrounding environment. To simulate this, we employed a simplified geometry where a generator containing $^{238}U/^{232}Th/^{40}K$ isotopes was placed in a uniform thickness (3 mm) of PMT glass, with an additional uniform thickness of 1 m of rock surrounding the PMT (the rock is selected for an equal model only), as depicted in Figure \ref{fig:geometry-model}. The chosen geometry represents an approximation of a realistic configuration, incorporating the concrete material of the experimental room, and enables the generation of uniform natural radioactivity in a 4$\pi$ distribution. The simulation process was validated by considering the PMT glass properties, including Cherenkov radiation and the quantum efficiency (QE) curve of the photocathode. The $^{238}U/^{232}Th/^{40}K$ isotopes were randomly generated, following a 4$\pi$ solid angle distribution on the shell of the rock and PMT glass. The simulation results, shown in Figure \ref{fig:rad-sim}, illustrate the response of the 20-inch PMT to these radioactive sources with the activity and simulated time length as detailed in Table \ref{tab:rate}. The charge spectra originating from the PMT glass are nearly identical to those from the rocks, with only variations in rates.

% Figure environment removed

% Figure environment removed

\begin{table*}[!htb]
\centering
\caption{The activity of natural radioactivity originating from the surrounding rock and PMT glass was assessed in this study. The simulation time difference between HPK and NNVT PMT glass was determined based on the PMT response model, ensuring that the statistical simulation amount for both types of PMT glass was equivalent.}
\label{tab:rate}
\resizebox{0.95\textwidth}{!}{
\begin{tabular}{|cc|ccc|ccc|c|c|}
\hline
\multicolumn{2}{|c|}{\multirow{2}{*}{Natural Radioactivity}}                & \multicolumn{3}{c|}{Rock} & \multicolumn{3}{c|}{Glass} & \multirow{2}{*}{Noise} & \multirow{2}{*}{Muon} \\ \cline{3-8}
\multicolumn{2}{|c|}{}  & \multicolumn{1}{c|}{U238}     & \multicolumn{1}{c|}{Th232}  & K40   & \multicolumn{1}{c|}{U238}      & \multicolumn{1}{c|}{Th232}     & K40   &    &    \\ \hline
\multicolumn{1}{|c|}{\multirow{2}{*}{Content (Sample measurement)}} & HPK  & \multicolumn{1}{c|}{\multirow{2}{*}{124 Bq/kg}} & \multicolumn{1}{c|}{\multirow{2}{*}{121.5 Bq/kg}} & \multirow{2}{*}{1355 Bq/kg} & \multicolumn{1}{c|}{5 Bq/kg} & \multicolumn{1}{c|}{1.6 Bq/kg} & 11 Bq/kg & /  & /  \\ \cline{2-2} \cline{6-10} 
\multicolumn{1}{|c|}{}  & NNVT & \multicolumn{1}{c|}{}  & \multicolumn{1}{c|}{}  &  & \multicolumn{1}{c|}{0.93 Bq/kg} & \multicolumn{1}{c|}{0.3 Bq/kg}  & 0.96 Bq/kg & /  & / \\ \hline
\multicolumn{1}{|c|}{\multirow{2}{*}{Rate (Sample measurement)}}    & HPK  & \multicolumn{1}{c|}{600 Hz}  & \multicolumn{1}{c|}{786 Hz}      & 817 Hz   & \multicolumn{1}{c|}{41 Hz}   & \multicolumn{1}{c|}{8.2 Hz}    & 26 Hz   & 5.6 kHz                & 44 Hz               \\ \cline{2-10} 
\multicolumn{1}{|c|}{}                                               & NNVT & \multicolumn{1}{c|}{592 Hz}                  & \multicolumn{1}{c|}{775 Hz}                    & 785 Hz                   & \multicolumn{1}{c|}{7.8 Hz}    & \multicolumn{1}{c|}{1.5 Hz}    & 2.3 Hz    & 8.3 kHz                & 39 Hz               \\ \hline
\multicolumn{1}{|c|}{\multirow{2}{*}{Statistical   Magnitude (s)}}   & HPK  & \multicolumn{1}{c|}{\multirow{2}{*}{21}}    & \multicolumn{1}{c|}{\multirow{2}{*}{30}}      & \multirow{2}{*}{27}     & \multicolumn{1}{c|}{3.2 $\times$ 10$^5$}    & \multicolumn{1}{c|}{1.4 $\times$ 10$^6$}   & 2.1 $\times$ 10$^6$   & /  & /  \\ \cline{2-2} \cline{6-10} 
\multicolumn{1}{|c|}{}   & NNVT & \multicolumn{1}{c|}{}  & \multicolumn{1}{c|}{}   &   & \multicolumn{1}{c|}{1.8 $\times$10$^6$}   & \multicolumn{1}{c|}{7.6 $\times$10$^6$}   & 2.4 $\times$10$^7$  & /  & /   \\ \hline
\end{tabular}
}
\end{table*}

It is worth to note that the charge spectra of NNVT PMT without charge smearing are almost the same as that from HPK PMT from simulation. The charge smearing of NNVT PMT is further discussed in section\,\ref{1:largepulse} for the related process. The $^{238}U/^{232}Th/^{40}K$ outside of the 1\,m rock shell has significantly low contribution to the results and can be ignored.
\subsection{Cosmic Muon}
\label{1:sim:muon}
It is well-known that when a muon penetrates through the PMT glass (such as the Hamamatsu PMT R5912), it generates a pulse through the phenomenon of Cherenkov radiation. This phenomenon has been extensively studied in previous works \cite{PMTmuon2007,BAYAT20141}. The large pulses observed in the 20-inch PMT dark count can be attributed to the photons generated by muons passing through the PMT glass via Cherenkov radiation. This occurs when the muon's speed exceeds the phase velocity of light in the glass. Comparing the experimental data with theoretical predictions, it becomes evident that there is a noticeable increase in the pulse rate, exceeding 10 Hz, when the threshold is set above approximately 3 p.e. This rate can even exceed 10 p.e. for both HPK and NNVT PMTs \cite{zhang2022study}.

\par To further investigate this phenomenon, we employed geometry models of the 20-inch NNVT PMT and HPK PMT. Muons were generated randomly according to a grounded muon distribution from a plane measuring 10$\times$10 m$^{2}$ located above the PMT. We performed simulations to analyze the PMT response to muons passing through the PMT glass \cite{zhang2022study}. The resulting charge curves were then used for subsequent analysis. Our findings confirm that the large pulses observed in the 20-inch PMT dark count primarily originate from the photons generated by muons that penetrate the PMT glass through Cherenkov radiation, regardless of the PMT type (HPK or NNVT). For a more comprehensive understanding, please refer to \cite{zhang2022study}.



\section{Comparison with measurement}
\label{1:comp}

By employing Geant4 and toy MC simulations, we investigated the behavior of a PMT in a dark environment, considering the thermal electron emission, natural radioactivity, and cosmic muons. These simulations provided valuable insights into the structure of the PMT dark count rate (DCR) charge spectrum. However, to gain a comprehensive understanding and accurately determine the contribution of each component, further comparisons with experimental measurements are required.

\par The preliminary findings from the simulations have shed light on the overall characteristics of the PMT DCR charge spectrum. However, a more detailed analysis is necessary to explore the individual contributions of thermal electron emission, natural radioactivity, and cosmic muons when comparing the simulation results with experimental data. This comparative analysis will allow us to disentangle the distinct signatures and quantify the impact of each component on the PMT DCR.

\par Through this comprehensive comparison between simulation and measurement, we aim to refine our understanding of the relative importance of thermal electron emission, natural radioactivity, and cosmic muons in shaping the PMT DCR charge spectrum. This knowledge will enhance our ability to accurately model and predict the behavior of PMTs in dark conditions.
\subsection{Fitting Model}
\label{1:comp:fit}

In Section \ref{1:sim:thermal}, we discussed the fitting of the thermal electron emission in the charge range of 1 p.e. to 3 p.e. for the HPK PMT (and extended to around 10 p.e. for the NNVT PMTs considering their charge smearing, as shown in Figure \ref{fig:Thermal emission-nnvt}) for the measured data. The fitting process aims to accurately capture the characteristics of the thermal electron emission mechanism.

\par To ensure accurate fitting results and avoid the confounding effects of natural radioactivity and thermal noise, a hypothetical exponential distribution of natural radioactivity is introduced in the transition section after the thermal noise. This approach allows for a smoother and more precise fitting of the thermal noise index. The fitted value of the thermal noise index can be expressed as follows:
\begin{equation}\label{con:Thermal_emission}
    f\left( x\right) =A_{1} \times e^{b_{1}x}+A_{2} \times e^{b_{2}x}
     \end{equation}
\par By adopting this fitting technique, we obtain a realistic and reliable estimation of the thermal noise index, which is essential for understanding and modeling the thermal electron emission in 20-inch PMTs. The fitting results provide valuable insights into the behavior of the PMT in the low charge range and contribute to our overall understanding of the PMT characteristics.

\par The fitting results for the thermal emission of the 20-inch HPK PMT indicate that the distribution can be described by the function $1.365 \times 10^{6} \times e^{-4.08x}$, which dominates in the low part of the fitting range. This distribution corresponds to a dark count rate of approximately 43.5 kHz when integrated over the distribution and considering a threshold of 0.5 p.e., aligning well with the measurement data. On the other hand, the green lines in Figure \ref{fig:sim-model} represent the estimated radioactivity contribution, modeled by $1.81 \times 10^{1} \times e^{-0.165x}$ for the 20-inch HPK PMT, resulting in a rate of approximately 101.0 Hz. The red and blue lines in the figure represent the results of separate fits for the thermal noise and estimated radioactivity components.

\par For the 20-inch NNVT PMT, the fitted constants are $5.20 \times 10^{3} \times e^{-0.416x}$ and $2.25 \times 10^{1} \times e^{-0.0584x}$, corresponding to a dark count rate of 10.2 kHz for thermal noise and 374.2 Hz for estimated radioactivity, respectively, when considering a threshold of 0.5 p.e. It is worth noting that the dark count rate decreases more rapidly with increasing charge for the HPK PMT compared to the NNVT PMT.

\par These fitting results provide valuable insights into the behavior of the PMTs in terms of their thermal noise and estimated radioactivity contributions. The different trends observed in the HPK and NNVT PMTs suggest distinct characteristics in their dark count behavior with respect to charge levels.

% Figure environment removed

The initial values for the fitting process were obtained from the simulated natural radioactivity of $^{238}U/^{232}Th/^{40}K$ in the rock and PMT glass, as shown in Figure \ref{fig:rad-sim}. The TMinuit software package was utilized to perform the minimization procedure, resulting in the final fitting result. The complete contribution of natural radioactivity consists of six components originating from the glass and surrounding rock, as expressed in Eq.\,\ref{con:NaturalRad}. In this equation, $x_j$ represents the contribution of $N_i^{NaturalRad}$ within the light intensity range of [a,b] p.e. The fitting of $x_j$ was carried out using Eq.\,\ref{con:chi}, which takes into account the contributions from thermal electron emission and muons. The parameters $x_{Muon}$ and $x_{Noise}$ correspond to the contributions of $N_i^{Muon}$ and $N_i^{Noise}$ within the light intensity range of [a,b] p.e., respectively.

% Figure environment removed

Based on the fitting results and the covariance matrix of parameters, it was observed that the natural radioactivity of the glass and rock, represented by $x_j$, exhibited a strong correlation with each other.

To further refine the analysis, the natural radioactivity contribution from the glass was fixed, considering its much smaller contribution compared to the surrounding rock. This was performed using two strategies: (1) fixing only the three parameters related to natural radioactivity ($^{238}U/^{232}Th/^{40}K$) in the rock, while fixing the thermal electron emission, glass, and muon parameters (mode \#1), and (2) incorporating thermal electron emission and muon parameters to have a total of five parameters (mode \#2).


In order to obtain accurate results, it is necessary to consider all the components that contribute to the full spectrum, including noise, natural radioactivity, and muon. However, it is also important to take into account the distortion around the amplitude threshold in the data.

\par For the HPK PMT, various fitting ranges were explored within the ranges of [3,80]\,p.e.~and [5,80]\,p.e.(see table~\ref{tab:HPK-table}). In the case of the NNVT PMT, wider fitting ranges were tested (see table~\ref{tab:MCP-table}).

For the HPK PMT, both fitting modes \#1 and \#2 were examined, and they yielded similar fitting quality. However, there were slight differences in the values and uncertainties of each decay chain of $^{238}U/^{232}Th/^{40}K$, as indicated in table~\ref{tab:HPK-table}. The $^{238}U/^{232}Th/^{40}K$ in the PMT glass was kept fixed during the fitting within the range of [5,80]\,p.e., as shown in Figure\,\ref{fig:fit5-80hpk}. Mode \#2 included five parameters: $^{238}U/^{232}Th/^{40}K$ in the rock, noise, and cosmic muon, which were compared with the experimental data.
% Figure environment removed

\par For the NNVT PMT, the fitting results of mode \#1 showed insignificant changes for different fitting ranges, while mode \#2 was further adjusted using several fitting ranges. Although slight differences were observed in the fitting parameters for different ranges, the overall fitting quality remained relatively unchanged, as demonstrated in table~\ref{tab:MCP-table}. In mode \#2, the $^{238}U/^{232}Th/^{40}K$ in the PMT glass was fixed during the fitting within the range of [3,140]\,p.e., as depicted in Figure\,\ref{fig:fit3-140-nnvt}. Five parameters were considered, including $^{238}U/^{232}Th/^{40}K$ in the rock, noise, and cosmic muon, which were compared with the available data.

\begin{table*}[!htb]
\centering
\caption{The fitting results of the HPK PMT are obtained by applying multiple fitting ranges to cover different contributions and verify the fitting quality. By utilizing various fitting ranges, we aim to account for the diverse sources of signal and assess the reliability of the fitting process.}
\label{tab:HPK-table}
\resizebox{0.95\textwidth}{!}{
\begin{tabular}{|cc|c|c|c|c|c|c|c|c|}
\hline
\multicolumn{2}{|c|}{Rate   (Hz)}                                         & Rock-U238     & Rock-Th232    & Rock-K40      & Glass-238 & Glass-Th232 & Glass-K40 & Noise          & Muon       \\ \hline
\multicolumn{1}{|c|}{ Only free rock }   & [3,80]p.e. & 97±38   & 199±36  & 361±34  & /         & /           & /         & /              & /          \\ \cline{2-10} 
\multicolumn{1}{|c|}{}                                       & [5,80]p.e. & 176 ± 45 & 164 ± 42 & 261 ± 40  & /         & /           & /         & /              & /          \\ \hline
\multicolumn{1}{|c|}{Rock/Noise/Muon} & [3,80]p.e. & 138 ± 430 & 167 ± 197 & 317 ± 242 & /         & /           & /         & 150k ± 60k & 27 ± 8.7 \\ \cline{2-10} 
\multicolumn{1}{|c|}{}                                       & [5,80]p.e. & 220 ± 45 & 134 ± 42 & 247 ± 40  & /         & /           & /         & 5.0k ± 0.8k & 28 ± 7.1 \\ \hline
\end{tabular}
}
\end{table*}

\begin{table*}[!htb]
\centering
\caption{The fitting results of the NNVT PMT in mode \#2 are obtained by applying multiple fitting ranges to cover different contributions and verify the fitting quality. By testing various fitting ranges, we aim to encompass the different sources of signal and ensure the accuracy of the fitting process.} 
\label{tab:MCP-table}
\resizebox{0.95\textwidth}{!}{

\begin{tabular}{|cc|c|c|c|c|c|c|c|c|}
\hline
\multicolumn{2}{|c|}{Rate   (Hz)}    & Rock-U238     & Rock-Th232   & Rock-K40      & Glass-238 & Glass-Th232 & Glass-K40 & Noise          & Muon        \\ \hline
\multicolumn{1}{|l|}{} & [3,120]p.e. & (1.6 ± 2.3)$\times 10^2$ & (0.5 ± 1.8)$\times 10^2$ & (2.3 ± 1.8)$\times 10^2$ & / & / & / & (7.7 ± 0.1)$\times 10^3$ & (0.15 ± 0.11)$\times 10^2$ \\ \cline{2-10} 
\multicolumn{1}{|l|}{} & [5,120]p.e. & (1.1 ± 2.3)$\times 10^2$ & (0.4 ± 2.0)$\times 10^2$ & (2.6 ± 1.8)$\times 10^2$ & / & / & / & (8.4 ± 0.2)$\times 10^3$ & (0.19 ± 0.14)$\times 10^2$ \\ \cline{2-10} 
\multicolumn{1}{|c|}{Fit Rock} & [3,140]p.e. & (1.4 ± 2.2)$\times$10$^2$ & (0.63 ± 1.6)$\times 10^2$ & (2.4 ± 1.7)$\times 10^2$ & / & / & / &  (7.7 ± 0.1)$\times 10^3$ & (0.15 ± 0.10)$\times 10^2$ \\ \cline{2-10}


\multicolumn{1}{|c|}{/Noise} & [5,140]p.e. & (0.0 ± 3.7)$\times 10^4$ & (0.51 ± 1.8)$\times 10^2$ & (3.2 ± 2.5)$\times$10$^2$ & / & / & / & (8.6 ± 0.3)$\times$10$^3$ & (0.23 ± 0.16)$\times 10^2$ \\ \cline{2-10} 
\multicolumn{1}{|c|}{/Muon} & [3,100]p.e. & (0.9 ± 2.5)$\times 10^2$ & (0.3 ± 3.0)$\times 10^2$ & (2.9 ± 1.9)$\times$ 10$^2$ & /   & /    & /         &  (8.0 ± 0.1)$\times$ 10$^3$ & (0.17 ± 0.13)$\times$ 10$^2$ \\ \cline{2-10} 
\multicolumn{1}{|l|}{} & [3,80]p.e.  & (1.7 ± 2.9)$\times 10^2$ & (0.0 ± 5.8)$\times 10^3$ & (2.7 ± 2.1)$\times 10^2$ & /  & /  & /   & (7.8 ± 0.1)$\times 10^3$ & (0.13 ± 0.16)$\times 10^2$ \\ \cline{2-10} 
\multicolumn{1}{|l|}{} & [2,140]p.e. & (0.6 ± 2.2)$\times 10^2$ & (0.3 ± 2.2)$\times 10^2$ & (3.2 ± 1.8)$\times 10^2$ & / & /   & /  & (8.2 ± 0.1)$\times 10^3$ & (0.21 ± 0.15)$\times 10^2$ \\ \hline
\end{tabular}

}
\end{table*}




\iffalse
% Figure environment removed

% Figure environment removed
\fi

\subsection{Comparing with Measured Radioactivity}
\label{1:comp:expectation}

Considering the fitted errors obtained from all the fitting results, it is worth noting that the NNVT PMT exhibits larger errors compared to the HPK PMT. Due to this discrepancy, it is not appropriate to express the overall fitting result simply as an arithmetic average. Instead, a weighted average is employed to account for the variations in the fitting results. Each fitting result is denoted as $L_i$, with an associated error of $m_i$, and the weight is calculated as $p_i=\frac{1}{m_i}$. Consequently, the weighted average $\overline{x}_i$ and the corresponding error $m_x$ can be defined as follows:
\begin{equation}
\begin{split}
  %  \begin{align}
    \label{con:mean}
    \overline{x}_i&=\dfrac{\sum_{i=1}^{n}p_{i}l_{i}}{\sum^{n}_{i=1}p_{i}} \\%\label{con:error}
    m_{x}&=\sqrt{\dfrac{1}{\sum ^{n}_{i=1}p_{i}}}
 %   \end{align}
\end{split}
\end{equation}


The compositions of $^{238}U/^{232}Th/^{40}K$ in the rock, as well as the rates of each component obtained from the weighted average strategy, are presented in table~\ref{tab:sum1} in radioactivity and rates, and comparison of sample measurements is shown in table~\ref{tab:rate}. A comprehensive comparison between the measurement and the expected values, incorporating the averaged compositions, can be observed in Figure\,\ref{fig:sim-sum}. The plots also take into account the errors associated with the parameters.
 

% Figure environment removed

\par Considering the relatively larger uncertainties in the results, the comparison between the NNVT and HPK PMTs demonstrates a consistent trend. However, when comparing the fitted values with the initial values, a notable reduction is observed in the $^{238}U/^{232}Th/^{40}K$ composition of the rock.

\par It should be noted that the difference between the laboratory testing environment and the simulation setting might contribute to these variations. The natural radioactivity of the surrounding material, as obtained from the fitting process, aligns more closely with the concrete model\,\cite{armengaud2017performance} rather than the rock itself. Additionally, the simulation configuration assumes an ideal scenario with 4$\pi$ coverage, which could further contribute to the observed differences.

\begin{table*}[!htb]
\centering
\caption{Summary in radioactivity of measurement fitting: HPK and NNVT PMT Comparison. The next section needs to predict the 20-inch PMT spectrum of natural radioactivity in JUNO. The simulated rate of muon is 100 Hz/m$^2$.}
\label{tab:sum1}
\resizebox{0.95\textwidth}{!}{
\begin{tabular}{|cc|ccc|ccc|c|c|}
\hline
\multicolumn{2}{|c|}{\multirow{2}{*}{Fit Rock}}                 & \multicolumn{3}{c|}{Rock}     & \multicolumn{3}{c|}{Glass}     & \multirow{2}{*}{Noise} & \multirow{2}{*}{Muon} \\ \cline{3-8}
\multicolumn{2}{|c|}{/Noise/Muon}      & \multicolumn{1}{c|}{U238}         & \multicolumn{1}{c|}{Th232}  & K40     & \multicolumn{1}{c|}{U238}   & \multicolumn{1}{c|}{Th232}  & K40    &   &    \\ \hline

\multicolumn{1}{|c|}{\multirow{2}{*}{Content }} & HPK     & \multicolumn{1}{c|}{(0.43 ± 0.09)$\times 10^2$}   & \multicolumn{1}{c|}{(0.21±0.06)$\times 10^2$}   & (4.1±0.7)$\times 10^2$  & \multicolumn{1}{c|}{/}      & \multicolumn{1}{c|}{/}      & /      & /                      & /                     \\ \cline{2-10} 
\multicolumn{1}{|c|}{(Bq/kg)}                                             & NNVT       & \multicolumn{1}{c|}{(0.25±0.20)$\times 10^2$}  & \multicolumn{1}{c|}{7±13}   & (4.7 ± 1.2)$\times 10^2$ & \multicolumn{1}{c|}{/}      & \multicolumn{1}{c|}{/}      & /      & /                      & /                     \\ \hline
\multicolumn{1}{|c|}{\multirow{4}{*}{Rate (Hz)}}    & \multirow{2}{*}{HPK}  & \multicolumn{1}{c|}{(2.2±0.5)$\times 10^2$}                & \multicolumn{1}{c|}{(1.4±0.4)$\times 10^2$}                & (2.5±0.4)$\times 10^2$                  & \multicolumn{1}{c|}{/}      & \multicolumn{1}{c|}{/}      & /      & (5.1±0.8)$\times 10^3$            & (0.27±0.06)$\times 10^2$            \\   
\multicolumn{1}{|c|}{}     &       & \multicolumn{1}{c|}{(3.8\%)}           & \multicolumn{1}{c|}{(2.4\%)}    & (4.4\%)        & \multicolumn{1}{c|}{} & \multicolumn{1}{c|}{} &  & (89\%)    & (0.48\%)    \\ \cline{2-10} 
\multicolumn{1}{|c|}{}    & \multirow{2}{*}{NNVT} & \multicolumn{1}{c|}{(1.2±1.0)$\times 10^2$}        & \multicolumn{1}{c|}{(0.48±0.80)$\times 10^2$}     & (2.7±0.7)$\times 10^2$     & \multicolumn{1}{c|}{/}      & \multicolumn{1}{c|}{/}      & /      & (8.0±0.1)$\times 10^3$            & (0.17±0.05)$\times 10^2$            \\  
\multicolumn{1}{|c|}{}      &      & \multicolumn{1}{c|}{(1.40\%)}          & \multicolumn{1}{c|}{(0.57\%)}       & (3.2\%)                        & \multicolumn{1}{c|}{} & \multicolumn{1}{c|}{} &  & (94\%)                & (0.20\%)       \\ \hline
\end{tabular}
}
\end{table*}


\section{Underground prediction}
\label{1:exp} 
The JUNO experiment is expected to encounter several primary background sources, including cosmic muons, fast neutrons, cosmogenic isotopes generated by muon spallation in the liquid scintillator, and natural radioactivity. The strict control of natural radioactivity aims to reduce the accidental count rate in the JUNO detector and minimize low-energy events. Natural radioactivity arises from various materials in the environment. The main source of natural radioactivity detected by PMTs in JUNO is the $^{238}U/^{232}Th/^{40}K$ content present in both JUNO's surrounding rocks (as shown in table~\ref{tab:juno-like}) and the glass components of the PMTs\,\cite{abusleme2021juno}.

\par By utilizing mode \#2, it is possible to provide preliminary predictions of the 20-inch HPK and NNVT PMT dark noise spectra in the underground environment of JUNO, as depicted in Figure\,\ref{fig:juno-like}. These predictions are based on the assumption of pre-water filling, and it is anticipated that the contribution from radioactivity in the surrounding rocks will be further reduced after filling the detector with pure water. The presence of a thick rock layer underground significantly diminishes the contribution of muons to the charge spectrum compared to muons at the ground level.

At a depth of 700 m, the underground environment effectively suppresses the muon rate, ensuring that no muon contributions are observed in the actual PMT spectrum. The distinctions between HPK and NNVT PMTs have been taken into account in section\,\ref{1:comp:fit}. Consequently, it is possible to provide preliminary predictions of the 20-inch PMT spectra influenced by natural radioactivity within a specific experimental environment.

\begin{table}[!ht]
\centering
\caption{Natural radioactivity of underground rocks in the JUNO site \cite{junocollaboration2023juno}.}
\label{tab:juno-like}
\begin{tabular}{|c|c|c|c|}
\hline
Isotope & Rock-U238 & Rock-Th232 & Rock-K40 \\ \hline
Activity (Bq/kg)  & 110±10 & 105±10 & 1340±50 \\ \hline
\end{tabular}
\end{table}

% Figure environment removed

\iffalse
% Figure environment removed
\fi

\section{Summary}
\label{1:summary}

This study focuses on investigating the occurrence of large pulses in the dark count of 20-inch PMTs attributed to natural radioactivity, excluding contributions from muons. These large pulses primarily originate from photons generated by natural radioactivity in the surrounding environment, which penetrate through the PMT glass and produce Cherenkov radiation. Both HPK and NNVT PMTs were analyzed in terms of their dark count rate (DCR), and the results were compared and fitted using a combination of measurements and a Geant4-based simulation. This approach enabled the determination of the expected charge spectra resulting from the passage of natural radioactivity through both types of PMTs.
The measurement of the PMT DCR spectrum plays a crucial role in providing a rough estimation of the radioactivity present in the environment. Additionally, it allows for the preliminary prediction of the 20-inch PMT spectra influenced by natural radioactivity in specific experimental environments, such as the JUNO experiment.


\begin{acknowledgements}
This work was supported partially by the National Natural Science Foundation of China (Grant No. 11875282 and 12022505), the Strategic Priority Research Program of the Chinese Academy of Sciences (Grant No. XDA10011200), and the Youth Innovation Promotion Association of CAS.
\end{acknowledgements}

\bibliographystyle{unsrt}
\bibliography{allcites}  

\end{document}

