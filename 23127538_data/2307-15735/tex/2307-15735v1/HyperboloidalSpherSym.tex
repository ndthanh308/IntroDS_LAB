\documentclass[12pt]{iopart}
\usepackage{iopams}

%\pdfoutput=1
\usepackage{amssymb}
\usepackage{graphicx}
\usepackage{epsfig}
\usepackage{color}
\usepackage{url}
\usepackage{times}
\usepackage{bm}
\usepackage{mathrsfs}
\usepackage[utf8]{inputenc}
\usepackage{hyperref}
\usepackage{enumerate}
\usepackage{amsthm}
\usepackage{verbatim}
\usepackage{cite}
\usepackage{bbm}
\usepackage{stmaryrd}
\usepackage{slashed}
\usepackage{upgreek}
\usepackage{bbold}



\usepackage{caption}

\newcommand{\beq}{\begin{equation}}
\newcommand{\eeq}{\end{equation}}
\newcommand{\bea}{\begin{eqnarray}}
\newcommand{\eea}{\end{eqnarray}}
\newcommand{\bit}{\begin{itemize}}
\newcommand{\eit}{\end{itemize}}
\newcommand{\ben}{\begin{enumerate}}
\newcommand{\een}{\end{enumerate}}
\newcommand{\nn}{\nonumber}
\newcommand{\dfrac}[2]{{\displaystyle\frac{#1}{#2}}}
\newcommand{\eqref}[1]{(\ref{#1})}

\newcommand{\coord}{\tau, \sigma, \theta,\varphi}
\newcommand{\gridcoord}{\tau_k, \sigma_i, \theta_j}
\newcommand{\spacecoord}{\sigma, \theta}

% Scri
\def\scri{\mathscr{I}}

\renewcommand{\d}{\,{\rm d}}

\renewcommand{\r}{\mathbbm r}
\newcommand{\rh}{r_{\rm h}}
\newcommand{\rhi}{r_{{\rm h}_i}}
\newcommand{\rhj}{r_{{\rm h}_j}}
\newcommand{\sigmah}{\sigma_{\rm h}}
\newcommand{\sigmahi}{\sigma_{{\rm h}_i}}
\newcommand{\sigmai}{\sigma_{\rm i}}
\newcommand{\sigmaf}{\sigma_{\rm f}}
\newcommand{\io}{\rm in\mbox{-}out}
\newcommand{\oi}{\rm out\mbox{-}in}

\newcommand{\rpm}[1]{\textcolor{red} {\texttt{R: #1}} }

          
\newcommand{\QMUL}{\address{School of Mathematical Sciences, Queen Mary, University of
  London, \\ Mile End Road, London E1 4NS, United Kingdom}}
  
  \newcommand{\NBI}{\address{$^{1}$Niels Bohr International Academy, Niels Bohr Institute, Blegdamsvej 17, 2100 Copenhagen, Denmark}}
          
\theoremstyle{plain}
\newtheorem{proposition}{Proposition}
\newtheorem{lemma}{Lemma}
\newtheorem{theorem}{Theorem}
\newtheorem{assumption}{Assumption}
\newtheorem*{conjecture}{Conjecture}
\newtheorem*{subconjecture}{Subconjecture}
\newtheorem{corollary}{Corollary}
\newtheorem{main}{Main Result}
\newtheorem{definition}{Definition}
\newtheorem{remark}{Remark}      

%Counter variable for the margin notes
\newcounter{mnotecount}%[section]

% This code generates the margin notes
\newcommand{\mnotex}[1]%{}
{\protect{\stepcounter{mnotecount}}$^{\mbox{\footnotesize $\bullet$\themnotecount}}$ 
\marginpar{%\color{red}%
\raggedright\tiny\em
$\!\!\!\!\!\!\,\bullet$\themnotecount: #1} }
     

%%document
\begin{document}
\title{Hyperboloidal approach for static spherically symmetric spacetimes: a didactical introduction and applications in black-hole physics}
\author{Rodrigo Panosso Macedo}
\NBI
\eads{\mailto{rodrigo.macedo@nbi.ku.dk}}
\date{\today}

\begin{abstract}
This work offers a didactical introduction to the calculations and geometrical properties of a static, spherically symmetric spacetime foliated by hyperboloidal time surfaces. We discuss the various degrees of freedom involved, namely the height function, responsible for introducing the hyperboloidal time coordinate, and a radial compactification function. A central outcome is the expression of the Trautman-Bondi mass in terms of the hyperboloidal metric functions. Moreover, we apply this formalism to a class of wave equations commonly used in black-hole perturbation theory. Additionally, we provide a comprehensive derivation of the hyperboloidal minimal gauge, introducing two alternative approaches within this conceptual framework: the in-out and out-in strategies. Specifically, we demonstrate that the height function in the in-out strategy follows from the well-known tortoise coordinate by changing the sign of the terms that become singular at future null infinity. Similarly, for the out-in strategy, a sign change also occurs in the tortoise coordinate's regular terms. We apply the methodology to the following spacetimes: Trumpet slices in Schwarzschild, Reisnner-Nordstr\"om-deSitter, Higher-dimensional black holes, and black hole with matter halo. From this heuristic study, we conjecture that the out-in strategy is best adapted for black hole geometries that account for environmental or effective quantum effects.
 \end{abstract}

\maketitle
\pagestyle{plain} 
%%%%%%%%%%%%%%%%%%%%%%%%%%%%%%%%%%%%%%%%%%%%%%%%%%%%%%%%%%%%%%%%
\section{Introduction}
In 1963, Penrose's seminal work on the ``Conformal Treatment of Infinity"\cite{Penrose:1964ge,Penrose:1964Republication} introduced a powerful tool to studies on the asymptotic behaviour of gravitational fields. The use of conformal methods in general relativity also enhanced our comprehension of black hole horizons and their geometrical properties. Apart from offering new perspectives on the nature of spacetime singularities, the formalism also introduced the concept of conformal (Carter-Penrose) diagrams, which provides an intuitive visualisation for the causal structure of black hole spacetimes.

Here, we consider the contribution of the conformal approach in general relativity to theoretical studies underpinning gravitational wave astronomy. We focus on scenarios modelled by black-hole perturbation theory. In particular, we are interested in two stages during the dynamical evolution of binary black holes: (i) the inspiral of extreme mass ratio (EMRI) binaries and (ii) the after merger evolution of a deformed black hole. The former constitutes one of the main sources of gravitational wave for the future space antenna LISA~\cite{LISA}, with the dynamics of EMRIs rigorously modelled by the gravitational self-force programme \cite{Barack:2018yvs,Pound:2021qin}. The latter underlies the black-hole spectroscopy programme \cite{Dreyer:2003bv,Berti:2005ys,Giesler:2019uxc,Cabero:2019zyt,Baibhav:2023clw}, for which the notion of quasi-normal modes \cite{Chandrasekhar:579245,Kokkotas99a,Nollert99,Berti:2009kk,Konoplya:2011qq} plays a crucial role.

For astrophysics and gravitational wave physics, two particular null hyper-surfaces are of fundamental importance: the black-hole horizon ${\cal H}^+$ and future null infinity $\scri^+$ (or a cosmological horizon $r_{\Lambda}$). Despite the elegant abstract geometrical description of spacetimes given by Einstein's theory and the conformal approach to general relativity, the theoretical and numerical calculations underlying the study of astrophysical black holes and gravitational wave astronomy heavily rely on the use of coordinate systems. In the context, a natural foliation of the conformal spacetime arises in terms of the so-called hyperboloidal surfaces\cite{Frauendiener:2000mk,Zenginoglu:2007it}. Hyperboloidal foliations are spacelike hypersurfaces with asymptotically hyperbolic geometry. Intuitively, they can be visualised as any horizontal line in a Carter-Penrose diagram that crosses ${\cal H}^+$ and $\scri^+$.

In perturbation theory, the idea of adjusting the coordinate system to exploit the causal structure of asymptotic regions goes back to ref.~\cite{Schmi93}. However, it was only in the past decade that the hyperboloidal approach became a central framework to black hole perturbation theory. Indeed, ref.~\cite{Zenginoglu:2011jz} points out that the framework resolves problems regarding the representation of quasinormal mode eigenfunctions, whereas ref.~\cite{jaramillo2021pseudospectrum} imports the notion of pseudospectra into gravity by identifying that the hyperboloidal approach casts black-hole perturbation theory in terms of the spectral problem of a non-self adjoint operator\cite{Trefethen:2005,Sjostrand2019,dyatlov2019mathematical,Ashida:2020dkc}.

One of the main contribution from the hyperboloidal framework is the treatment of boundary conditions to the wave equations underlying the physical problems. Traditionally, black-hole perturbation theory is formulated in terms of coordinates $(t,r_*,\theta,\varphi)$ which closely resemble familiar coordinates in flat spacetime. Well-Posedness follows after boundary conditions are specified at the asymptotic region $r_*\rightarrow \infty$ and at the horizon $r_*\rightarrow -\infty$. However, along $t=$constant the limits $r_*\rightarrow \infty$ and $r_*\rightarrow -\infty$ correspond to spatial infinity $i^0$, and the bifurcation sphere ${\cal B}$. Therefore, one must impose external boundary conditions ensuring that the energy is absorbed by the black hole and propagates out to the wave zone to model the relevant physical scenario. Accessing these infinitely far regions is not straightforward from a numerical perspective. Besides, quasi-normal modes diverges at $i^0$ and ${\cal B}$, which prevents their formal interpretation as the eigenvalues and eigenfunctions in an appropriated Hilbert space. In practical terms, the numerical grid is cut at finite radius $r=\pm r_o$, where the relevant boundary conditions are approximated. Then, the systematic erros are controlled by the location of this artificial cut and the information about the solution for $r>\pm r_o$ is obtained by a numerical extrapolation. 

This strategy, however, has reached its limits. High precision gravitational astronomy requires information form perturbation theory at second order\cite{Brizuela:2009qd,Loutrel:2020wbw,Miller:2020bft}. The dynamics of second order perturbation fields are sourced by the solution at first order, and thus they are dictated by the global behaviour of first order solutions. Any type of systematic error arising from the numerical solutions at first order will accumulate quadratically and jeopardise the accuracy for the second order studies. Besides, rigorous studies on the field equations shows that the second order sources have a better regularity behaviour when the system is parametrised by coordinates adapted to causal structure of the wave zone and the black hole \cite{Miller:2020bft}.


The hyperboloidal approach resolves the boundary issues altogether. The application of the framework in black-hole perturbation theory has reached a mature stage, and it has established itself as an essential method to the study of wave propagation on a fixed background. Initially, the works focused on the development of numerical codes for time evolutions, benchmarked by the study of the late time decay of several fields propagating in black-hole spacetimes~\cite{Zenginoglu:2008wc,Zenginoglu:2008uc,Zenginoglu:2009hd,Zenginoglu:2009ey,Bizon:2010mp,Zenginoglu:2010zm,Zenginoglu:2010cq,Racz:2011qu,Jasiulek:2011ce,Harms:2013ib,Yang:2013uba,Spilhaus:2013zqa,Macedo2014,Hilditch:2016xzh,Csukas:2019kcb,OBoyle:2022yhp}.  In this context, the hyperboloidal foliations also offers the correct tool for rigours mathematical statements about the perturbations' late time decay and black-hole dynamical stability \cite{Andersson:2019dwi,Angelopoulos:2021hlw,Angelopoulos:2021cpg,Gajic:2022czq,Gajic:2022pst}. 

Complementary to works in the time domain, refs.~\cite{Ansorg:2016ztf,PanossoMacedo:2018hab,PanossoMacedo:2019npm} established the use of hyperboloidal approach in the frequency domain. Of particular relevance was the identification of the so-called minimal gauge, which provides a geometrical understanding for the methods introduced by the seminal works of Leaver~\cite{Leaver85,Leaver90}. The hyperboloidal framework also allows to formally describe quasi-normal modes as honest eigenfunctions defined on a given Hilbert space\cite{Gajic:2019oem,Gajic:2019qdd}. Furthermore, the notion of quasinormal mode pseudospectra \cite{jaramillo2021pseudospectrum,jaramillo2021gravitational,Destounis:2021lum,Boyanov:2022ark,Sarkar:2023rhp,Arean:2023ejh} sheds light on the phenomenon of quasi-normal modes spectral instability~\cite{Aguirregabiria:1996zy,Vishveshwara:1996jgz,Nollert:1996rf}, which might impact predictions from the black-hole spectroscopy programme\cite{jaramillo2021gravitational,Cheung:2021bol,Berti:2022xfj,Courty:2023rxk}. In the context of wave form modelling for EMRI's, the hyperboloidal approach enhanced the predictions arising from the effective-one-body approach~\cite{Bernuzzi:2010xj,Zenginoglu:2011zz,Bernuzzi:2011aj,Bernuzzi:2012ku,Harms:2014dqa,Nagar:2014kha,Harms:2015ixa,Harms:2016ctx,Lukes-Gerakopoulos:2017vkj}, and the framework has also excelled first tests in the gravitational self-force programme~\cite{Zenginoglu:2012xe,Wardell:2014kea,Thornburg:2016msc,PanossoMacedo:2022fdi,DaSilva:2023xif,Vishal:2023fye}.

By dwelling into the hyperboloidal framework for static, spherically symmetric spacetimes, the goal of this paper is to provide a didactical introduction to the topic so that students and interested researchers may familiarise themselves with the techniques currently employed in black-hole perturbation theory. In particular, eq.~\eqref{eq:BondiMass} presents an expression for the Trautman-Bondi mass in terms of generic hyperboloidal metric functions. Besides, eqs.~\eqref{eq:H_inout_flat} and \eqref{eq:Houtin} shows that hyperboloidal coordinates in the minimal gauge follows from a very simple algorithmic procedure: first one construct the well-known tortoise coordinate for the underlying spacetime; then the height function follows by a simple change in the sign of the terms with a singular behaviour at future null infinity.

The paper is organised as following. We introduce in the next section the hyperboloidal framework of a generic static, spherically symmetric spacetime. In particular, we provide a comprehensive discussion of the role played by all degrees of freedom and their geometrical significance. Sec.~\ref{sec:sphericalsymmetric} ends with a concrete example of a hyperboloidal foliation for the Schwarzschild solution. Then, sec.~\ref{sec:pert_theory} applies the framework to a class of wave equations commonly used in black-hole perturbation theory. Sec.~\ref{sec:min_gauge} offers a didactical exposition to the hyperboloidal minimal gauge. Finally, sec.~\ref{sec:examples_spacetime} applies the framework to several spacetimes. We employ geometrical units with the speed of light and Newton's gravitational constant $c=G=1$. Moreover, physical and conformal spacetimes are denoted by $({\cal M}, g_{ab})$ and $({\bar {\cal M}},\bar g_{ab} )$, respectively.

\section{Spherical Symmetric Spacetime}\label{sec:sphericalsymmetric}
\subsection{Schwarzschild coordinates $(t,r, \theta,\varphi)$}
We first introduce the generic parametrisation for a spherically symmetric  line element in Schwarzschild-like coordinates $(t,r,\theta,\varphi)$ as
\beq
\label{eq:metric_tr}
ds^2 = - a(r) dt^2 + \dfrac{dr^2}{b(r)} + r^2 d\omega^2,
\eeq
with  $d\omega^2=\left( d\theta^2 + \sin^2\theta d\varphi^2\right)$ the line element of the $2$-sphere. 
%The roots of the functions $a(r)$ and $b(r)$ give rises to spacetime horizons \rpm{Is this generic? Can one have horizons where only $a(r)$ or $b(r)$ vanish?}
%Assuming the existence of a horizon at the coordinate $\rh$, it is convenient to parametrise the metric functions as
%\bea
%\label{eq:a_b}
%a(r) = \left( 1- \dfrac{r_{\rm h}}{r} \right) A(r), \quad b(r) = \left( 1- \dfrac{r_{\rm h}}{r} \right) B(r), 
%\eea
%with $A(r)>0$ and $B(r)>0$ in the black-hole exterior region $r\in[\rh,\infty)$

%
The study of null geodesic leads to the definition of the tortoise coordinate $r_*(r)$ via
\bea
\label{eq:def_tortoise}
\dfrac{dr_*}{dr} = \dfrac{1}{f(r)}, \quad f(r) = \sqrt{a(r) b(r)}. 
\eea
With eq.~\eqref{eq:def_tortoise}, an alternatively parametrisation to the line element eq.~\eqref{eq:metric_tr} in terms of the tortoise coordinate $r_*$ follows via
\beq
\label{eq:metric_tr*}
ds^2 =  a(r)\bigg(-  dt^2 + dr_*^2\bigg) + r^2 d\omega^2, \quad r=r(r_*).
\eeq
In the definition of the tortoise coordinate, the asymptotic behaviour of $f(r)$ as $r\rightarrow \infty$, as well in the vicinity of spacetime horizons is of particular relevance. We assume the function $f(r)$ has $n_{\rm h}=N_{\rm h}+1$ real roots $\rhi$ ($i=0\cdots N_{\rm h}$), which allows for a decomposition
\beq
\label{eq:func_f_horizons}
f(r) =  {K}(r) \prod_{i=0}^{N_{\rm h}} \left(1- \dfrac{\rhi}{r} \right).
\eeq
The positive roots represent horizons in the spacetime. Around a value $\rhj$, we may also introduce the representation
\beq
\label{eq:func_f_horizons_j}
f(r) =  {K_j}(r) \left(1- \dfrac{\rhj}{r} \right). 
\eeq
%
Moreover, asymptotically flat $4$ dimensional spacetimes have the asymptotic expansion
\beq
\label{eq:f_asymp_Mink}
f(r) = 1 - \dfrac{2M}{r} + {\cal O}\left( r^{-2} \right),
\eeq 
with $M$ the spacetime ADM mass. For asymptotic (Anti-)de Sitter spacetime, on the other hand, one encounters
\beq
\label{eq:f_asymp_dS}
f(r) = -\dfrac{\Lambda}{3}r^2 + {\cal O}\left( r^{-2} \right),
\eeq 
with the cosmological constant $\Lambda$ assuming positive values for de Sitter, and negative for Anti-de Sitter spacetimes.






\subsection{Hyperboloidal Coordinates $(\tau,\sigma,\theta,\varphi)$}\label{sec:Hyp_coord}
We introduce the hyperboloidal coordinates $(\tau, \sigma, \theta, \varphi)$ via the scri-fixing technique~\cite{Zenginoglu:2007jw}, together with the compactification strategy from ref.~\cite{PanossoMacedo:2018hab}
\beq
\label{eq:HypCoord}
t = \lambda \bigg(\tau-H(\sigma)\bigg), \quad r= \lambda \dfrac{\rho(\sigma)}{\sigma}.
\eeq
Here, $\lambda$ is a given length scale of the spacetime, and the compact radial coordinate is defined in the domain $\sigma\in[\sigmai,\sigmaf]$. For instance, in an asymptotically flat black hole spacetime, the exterior region has $\sigmai=0$ locating $\scri^+$ and a final coordinate value $\sigmaf=1$ locating the event horizon. Here, we leave the notation generic to be adapted according to the specific scenario under consideration.

As we will show in sec.~\ref{sec:conformal}, $\rho(\sigma)$ represents the areal radius in the conformal space. Therefore we assume $\rho(\sigma)>0$ \cite{PanossoMacedo:2018hab}. Motivated by the differential form $dr = -\lambda(\rho - \sigma \rho')/\sigma^2 \, d\sigma$ that follows from the radial transformation eq.~\eqref{eq:HypCoord}, ref.~\cite{PanossoMacedo:2018hab} defines 
\beq
\label{eq:beta}
\beta(\sigma) := \rho(\sigma) - \sigma \rho'(\sigma).
\eeq
%
Besides, the coordinate change \eqref{eq:HypCoord} implies the transformation in the tortoise coordinate via
\beq
\label{eq:def_x}
x(\sigma) = \dfrac{r_*(r(\sigma))}{\lambda}.
\eeq
With the help of eq.~\eqref{eq:def_tortoise}, the dimensionless tortoise coordinate $x(\sigma)$ follows from integrating the differential equation
\beq
\label{eq:dx_dsigma}
x'(\sigma) = - \dfrac{\beta(\sigma)}{\sigma^2 {\mathcal F(\sigma)}}, \quad {\mathcal F(\sigma)}= f(r(\sigma)).
\eeq
%
As expected from eq.~\eqref{eq:func_f_horizons}, ${\mathcal F(\sigma)}$ vanishes at the horizons $\sigmahi$ defined by eq.~\eqref{eq:HypCoord} as $\rhi=r(\sigmahi)$. Indeed, the representations \eqref{eq:func_f_horizons} or \eqref{eq:func_f_horizons_j} yield respectvely
\bea
\label{eq:F_of_sigma}
{\mathcal F(\sigma)} &=& {\cal K}(\sigma) \prod_{i=0}^{N_{\rm h}} \left(1- \dfrac{\rhi}{r(\sigma)} \right), \quad {\cal K}(\sigma) = K(r(\sigma)),\\
\label{eq:F_of_sigma_j}
{\mathcal F(\sigma)} 	&=& {\cal K}_j(\sigma) \left(1- \dfrac{\rhj}{r(\sigma)} \right), \quad  {\cal K}_j(\sigma) = K_j(r(\sigma)).
\eea
%
For asymptotically flat spacetimes one reads from eq.~\eqref{eq:f_asymp_Mink} the behaviour
\beq
\label{eq:F_of_sigma_asymp_Mink}
{\mathcal F(\sigma)} = 1 - \dfrac{2M}{\lambda} \dfrac{\sigma}{\rho(0)} + {\cal O}(\sigma^2).
\eeq
For asymptotically (Anti-)deSitter spacetimes one reads from eq.~\eqref{eq:f_asymp_dS} the behaviour
\beq
\label{eq:F_of_sigma_asymp_dS}
{\mathcal F(\sigma)} = -\dfrac{\lambda^2 \Lambda}{3}  \dfrac{\rho(0)^2}{\sigma^2}\bigg( 1 + {\cal O}\left(\sigma^{-1}\right)\bigg).
\eeq

The height function $H(\sigma)$ must ensure that the hypersurfaces $\tau=$ constant penetrate the black-hole horizon at $\sigmah$ and that it intersects future null infinity at $\sigma=0$. These conditions are discussed in ref.~\cite{Zenginoglu:2011jz}, and we will review them in the next sections according to the notation employed here. In particular, sec.~\ref{sec:min_gauge} will introduce choices for eqs.~\eqref{eq:HypCoord} within the so-called minimal gauge class.

The coordinate transformation \eqref{eq:HypCoord} yields the following spherical symmetric line element in hyperboloidal coordinates $(\tau,\sigma,\theta, \varphi)$ 
\beq
\label{eq:eq_hyp}
d s^2 =  \dfrac{\lambda^2}{\sigma^2}\, \Bigg\{  \zeta(\sigma) \beta(\sigma) \left[-p(\sigma) d\tau^2 + 2\gamma(\sigma)d\tau d\sigma + w(\sigma) d\sigma^2 \right] +\rho(\sigma)^2d\omega^2 \Bigg\}.
\eeq
Despite the rather cumbersome appearance, each one of the functions carries information either from the physical character of the spacetime, or the hyperboloidal gauge degrees of freedom.

\subsubsection{Conformal Transformation}\label{sec:conformal}
The first important property to observe is that in the hyperboloidal coordinates $(\tau, \sigma, \theta, \varphi)$, the line element \eqref{eq:eq_hyp} assumes explicitly a conformal representation $ds^2 = \Omega^{-2} d\bar s^2$ with
\beq
\label{eq:conf_metric}
d\bar s^2 =\Xi(\sigma)  \left[-p(\sigma) d\tau^2 + 2\gamma(\sigma)d\tau d\sigma + w(\sigma) d\sigma^2 \right] + \rho(\sigma)^2d\omega^2, \quad \Omega = \dfrac{\sigma}{\lambda}.
%\zeta(\sigma) \beta(\sigma)
\eeq
This choice employs the conformal factor itself as the radial coordinate $\sigma = \Omega \lambda$. All functions $\Xi(\sigma)$, $p(\sigma)$, $\gamma(\sigma)$, $w(\sigma)$ and $\rho(\sigma)$ are regular in the domain $\sigma\in[\sigmai,\sigmaf]$. 

The next sections explain the interpretation of these functions in terms of the original physical degree of freedom in line element \eqref{eq:metric_tr} --- or alternatively eq.~\eqref{eq:metric_tr*} --- and the hyperboloidal gauge freedom when specifying the coordinate transformation \eqref{eq:HypCoord}. 

\subsubsection{The $(\tau,\sigma)$-sector of the conformal metric}
The spherically symmetric line element \eqref{eq:eq_hyp} decomposes the $4$-dimensional conformal manifold $\overline{\cal M}$ with coordinates $(\tau, \sigma, \theta, \varphi)$ into two sub manifolds $\overline{\cal M}= \overline{{\cal M}}^2 \times {\cal S}^2$. The $2$-dimensional Lorentzian manifold $\overline {\cal M}^2$ has coordinates $x^a=\{\tau, \sigma\}$, whereas ${\cal S}^2$ represents a $2$-dimensional sphere with coordinates $x^{A}=\{\theta, \varphi\}$. The metric for $\overline{{\cal M}}^2$ reads
\bea
\left.d\bar s^2\right|_{\overline {\cal M}^2} &=& \bar\eta_{ab}dx^a dx^b \\
&=& \Xi(\sigma)  \left[-p(\sigma) d\tau^2 + 2\gamma(\sigma)d\tau d\sigma + w(\sigma) d\sigma^2 \right].
\eea
From the coordinate change leading to eq.~\eqref{eq:eq_hyp}, one identifies 
\beq
\label{eq:def_Xi_0}
\Xi(\sigma) = \zeta(\sigma) \beta(\sigma).
\eeq  
Anticipating a result from the relation between $p(\sigma)$, $\gamma(\sigma)$ and $w(\sigma)$ to be presented in the the next sections --- c.f.~\eqref{eq:def_w} --- we can establish an alternative interpretation to $\Xi(\sigma)$ and define it as
\beq
\label{eq:def_Xi}
\Xi(\sigma) = \sqrt{ -\boldsymbol {\bar\eta}}, \quad \boldsymbol {\bar\eta} = \det{{\bar\eta_{ab}}}.
\eeq

\subsubsection{The $(\theta,\varphi)$-sector of the conformal metric}
The function $\rho(\sigma)$ introduced in eq.~\eqref{eq:HypCoord} relate to the gauge choice when compactifying the radial direction. As anticipated in sec.~\ref{sec:Hyp_coord}, $\rho(\sigma)$ is the areal radial function in the conformal metric. Indeed, the metric for ${\cal S}^2$ reads
\bea
\left.d\bar s^2\right|_{\overline {\cal S}^2} &=& \bar q_{AB}dX^A dX^B \\
&=& \rho(\sigma)^2 \left( d\theta^2 + \sin^2\theta d\varphi^2\right).
\eea
Thus, we restrict ourselves to radial coordinate transformations \eqref{eq:HypCoord} with $\rho(\sigma)>0$.

\subsubsection{Interpretation of the metric components}




%The most straightforward choice to the compactification of the black-hole exterior region is 
%\beq
%r=\dfrac{\rh}{\sigma}, \Longrightarrow \rho(\sigma) = \beta(\sigma) = \rh/\lambda.
%\eeq
%which implies  $\rho(\sigma) = \beta(\sigma) = \rh/\lambda =$ constant. This choice is one particular option within the so-called minimal gauge which we will discuss in sec.~\ref{sec:min_gauge}.

The physical properties of the spacetime are encoded in two degrees of freedom expressed either by $a(r)$ and $b(r)$ in eq.~\eqref{eq:metric_tr}, or $a(r)$ and $r_*(r)$ in eq.~\eqref{eq:metric_tr*}. In the hyperboloidal coordinates, these degrees of freedom are capture first by the function\footnote{Unless necessary, from now on we omit the explicit $\sigma$-dependence on the radial coordinate mapping $r(\sigma)$. Thus, an expression of the type $F(\sigma) = f(r)$ is understood as $F(\sigma) = f(r(\sigma))$ for generic functions $f$ and $F$. }
\beq
\label{eq:zeta}
\zeta(\sigma) = \sqrt{\dfrac{a(r)}{b(r)}},
\eeq
which measures the deviation between $g_{tt}$ and  $1/g_{rr}$ in eq.~\eqref{eq:metric_tr}. In particular, $\zeta(\sigma) \neq 1$ only for more intricate geometries. For asymptotically flat spacetimes, we can derive the asymptotic behaviour of $\zeta(\sigma)$ with the help from the Newtonian limit
\beq
\label{eq:zeta_asympot}
a(r)\sim 1 - \dfrac{\phi_{N}}{r},\quad b(r)^{-1}\sim 1 + \dfrac{\phi_{N}}{r} \Longrightarrow \zeta(\sigma) = 1 + {\cal O}(\sigma^2).
\eeq
In the hyperboloidal representation of the conformal metric \eqref{eq:conf_metric}, $\zeta(\sigma)$ follows from the definition \eqref{eq:def_Xi} via $\zeta(\sigma) = \Xi(\sigma)/\beta(\sigma)$.

The physical information carried originally by the tortoise coordinate $r_*(r)$ is now encoded in 
\beq
\label{eq:p_from_x}
p(\sigma) := -\lambda \dfrac{d\sigma}{dr_*}= -\dfrac{1}{x'(\sigma)}.
\eeq 
With the help of eq.~\eqref{eq:dx_dsigma}, it reads
 \beq
 \label{eq:def_p}
 p(\sigma)= \dfrac{\sigma^2 {\cal F}(\sigma)}{\beta(\sigma)}.
 \eeq
From eq.~\eqref{eq:F_of_sigma}, one notice that the roots of $p(\sigma)$ coincides with those of ${\cal F}(\sigma)$, i.e., the horizons of the spacetime. 
Besides, in asymptotically flat spacetimes, $p(\sigma)$ also vanishes at future null infinity $\sigma=0$. More specifically, considering eqs.~\eqref{eq:F_of_sigma} and \eqref{eq:F_of_sigma_asymp_Mink} it follows
\beq
\label{eq:prop_p}
p(\sigma) \sim \sigma^2 \prod_{i=0}^{N_{\rm h}}(\sigmahi - \sigma), 
\eeq 
i.e., $p(\sigma)$ vanishes quadratically at future null infinity ($\sigma=0$). Typically, $p(\sigma)$ vanishes linearly at the horizons, unless the spacetime contains an extremal black-hole, where the values of two horizons coincide.

The gauge freedom for the time transformation $t=t(\tau,\sigma)$ is encoded in the so-called boost-function~\cite{Zenginoglu:2011jz}
\bea
\label{eq:def_gamma}
\gamma(\sigma) := -\lambda \dfrac{dH}{dr_*} = H'(\sigma)p(\sigma).
\eea
The functions $\rho(\sigma), \zeta(\eta), p(\sigma)$ and $\gamma(\sigma)$ capture all degrees of freedom in the framework, i.e., two in the generic spherical symmetric line element \eqref{eq:metric_tr} and two from the the mapping $(t,r)\leftrightarrow(\tau,\sigma)$ in eq.~\eqref{eq:HypCoord}. Thus, the remaining function $w(\sigma)$ in the line element is not independent, but rather related to  $p(\sigma)$ and $\gamma(\sigma)$ via
\beq
\label{eq:def_w}
w(\sigma) := \dfrac{1-\gamma(\sigma)^2}{p(\sigma)}.
\eeq
Eq.~\eqref{eq:def_w} imposes constraints on the boost function $\gamma(\sigma)$, see ref.~\cite{Zenginoglu:2011jz}. The first arises from demanding $ g_{\sigma \sigma}>0$, so that the metric retains the correct signature. This conditions yields
\beq
\label{eq:spacelike}
\left| \gamma \right| < 1. 
\eeq
From the geometrical perspective, this condition ensures that the hypersurfaces $\tau=$ constant remain spacelike in the domains $\sigma\in[\sigmai,\sigmaf]$

The second set of conditions follows from imposing regularity of $w(\sigma)$ at the roots of $p(\sigma)$, i.e., the horizons $\sigma=\sigmahi$, and $\sigma=0$ for asymptotically flat spacetimes. At the horizons, these conditions read
\beq
\label{eq:align_nullvectors_hrz}
\gamma(\sigma)^2 = 1 +{\cal O}(\sigmahi-\sigma) \quad \Longrightarrow \quad \gamma(\sigma) = \pm1 +{\cal O}(\sigmahi-\sigma).
\eeq
The choice for the positive or negative signal is discussed in the sec.~\ref{sec:nullvectors} as it implies having either the ingoing or an outgoing null vector as generator of the horizon surface.

For asymptotically flat spacetimes, $\gamma(\sigma)$ must behave as 
\bea
\label{eq:align_nullvectors_scri}
\gamma(\sigma)^2 = 1 +{\cal O}(\sigma^2) \quad &\Longrightarrow& \quad \gamma(\sigma) = 1 +{\cal O}(\sigma^2).
\eea
In this case, the particular choice for the plus sign, ensure that the vector $\partial_\tau$ is the generator of future null infinity. These geometrical interpretations will be elucidated in the next sections.


\subsubsection{3+1 representation}
With the conformal line element in the form \eqref{eq:conf_metric}, we identify the relevant quantities for a $3+1$ representation of the spacetime. We begin by considering the form $\bar n_a = -\bar \nabla_a \tau$, which gives the (not normalised) normal vector to the hypersurfaces $\tau =$ constant 
\beq
\bar n_a = -\delta_{a}^\tau, \quad \bar n^a = \dfrac{w}{\Xi} \delta^a_\tau -\dfrac{\gamma}{\Xi}\delta^a_\sigma, \quad ||n^a||^2 = - \dfrac{w}{\Xi},
\eeq
with $\Xi = \zeta \beta$. 

The hypersurfaces $\tau =$ constant remain spacelike as long as $||n^a||^2<0 \leftrightarrow w(\sigma)>0$, c.f.~eq.~\eqref{eq:spacelike}.
%
Besides, we can read the $3+1$ quantities from eq.~\eqref{eq:conf_metric}, 
\bea
\label{eq:3+1}
\alpha^2 = \dfrac{\Xi}{w},  \quad \upbeta_\sigma = \gamma \Xi,  \quad \upbeta^\sigma = \dfrac{\gamma}{w}, \quad \upgamma_{\sigma \sigma} = \Xi w,  \quad \upgamma_{\theta \theta} = \dfrac{\upgamma_{\theta \theta}}{\sin^\theta} = \rho^2,
\eea
with $\alpha$ the lapse function, $\upbeta^i$ the shift vector and $\upgamma_{ij}$ the spatial metric.

\subsubsection{The causal structure: ingoing and outgoing null vectors}\label{sec:nullvectors}
Outgoing and ingoing null vectors are given, respectively, by
\beq
\bar l^a = \nu \left( \delta^a_\tau - \dfrac{1+\gamma}{w}\delta^a_\sigma \right), \quad \bar k_a = \dfrac{w}{2\Xi \nu} \left( \delta^a_\tau + \dfrac{1 - \gamma}{w}\delta^a_\sigma \right),
\eeq
with $\nu(\sigma)$ a freely specifiable boost parameter. 

To align either $\bar l^a$ or $\bar k^a$ with the vector $\left(\partial_\tau\right)^a$, one must choose the signs $-$ or $+$ in the conditions \eqref{eq:align_nullvectors_hrz}, respectvely. Indeed, conditions \eqref{eq:align_nullvectors_scri} imply the behaviour at future null infinity
\beq
\label{eq:nullvector_scri}
\left. \bar l^a\right|_{\sigma=0} = \nu \left( \delta^a_\tau- \dfrac{2}{w}\delta^a_\sigma \right), \quad \left. \bar k^a\right|_{\sigma=0}  =\dfrac{w}{2\Xi\nu} \delta^a_\tau,
\eeq
whereas at the black-hole horizon, the positive sign in eq.~\eqref{eq:align_nullvectors_hrz} provides
\beq
\left. \bar l^a\right|_{\sigma=\sigmah} = \nu \delta^a_\tau, \quad \left. \bar k^a\right|_{\sigma=\sigmah}  =\dfrac{w}{2\Xi\nu} \left( \delta^a_\tau  +\dfrac{2}{w}\delta^a_\sigma \right).
\eeq

\subsubsection{Trautman-Bondi Mass}
For asymptotically flat spacetimes, the Trautman-Bondi mass gives the energy content at future null infinity. We calculate this quantity via the asymptotic limit of the Hawking mass~\cite{Szabados:2009eka,Jaramillo:2010ay}. In terms of the hyperboloidal metric functions from eq.~\eqref{eq:conf_metric}, the Trautman-Bondi mass reads
\bea
\label{eq:BondiMass}
M_{\rm TB} =\dfrac{\lambda}{2} \lim_{\sigma\rightarrow 0} \Bigg[  \dfrac{\rho(\sigma) \Xi(\sigma)}{\sigma \beta(\sigma) } \bigg(1 - \dfrac{p(\sigma) \beta(\sigma)^2}{\Xi(\sigma) \sigma^2}  \bigg)\Bigg].
\eea
With the definitions \eqref{eq:def_p} and \eqref{eq:def_Xi_0}, together with asymptotic expansions \eqref{eq:F_of_sigma_asymp_Mink} and \eqref{eq:zeta_asympot}, one verifies that the Trautman-Bondi mass coincides with the ADM mass defined via eq.~\eqref{eq:f_asymp_Mink}. While this result is expected due to the static character of the spacetime, eq.~\eqref{eq:BondiMass} provides a calculation of the mass direct from the hyperboloidal metric functions.



\subsubsection{The Schwarzschild spacetime: a first example}\label{sec:mingauge_schwarzschild}
To exemplify the above discussion we construct a hyperboloidal coordinate for the Schwarzschild spacetime. The line element \eqref{eq:metric_tr} has the functions $a (r)= b(r) = 1-\dfrac{\rh}{r}$ which already implies $\zeta(\sigma)=1$ in eq.~\eqref{eq:zeta}. Besides, eq.~\eqref{eq:def_tortoise} for the tortoise coordinate integrates to
\bea
\label{eq:r*_Schwarzschild}
r_*(r) &=& r + \rh \ln\left( \dfrac{r}{\rh} -1 \right) \nn \\
         &=& r + \rh \ln\left( \dfrac{r}{\rh}\right)  + \rh \ln\left( 1- \dfrac{\rh}{r} \right).
         \label{eq:r*_Schwarzschild}
\eea
The first line in the above expression corresponds to the well-known results in textbooks, whereas the second line makes explicit the behaviour of $r_*$ as $r\rightarrow \infty$ and $r\rightarrow \rh$.

To construct the height function $H(\sigma)$, we follow the strategy from refs.~\cite{Ansorg:2016ztf}. First, we introduce the ingoing Eddington-Finkelstein coordinate $v=t + r_*$. This step ensures the horizon penetrating character of the coordinate system. Along $v=$ constant, however, the limit $r\rightarrow\infty$ corresponds to past null infinity. The goal is to deform these coordinates, such that $r\rightarrow\infty$ reaches future null infinity. For this purpose we consider outgoing null rays within the coordinate system $(v,r,\theta,\varphi)$, which are characterised by
\bea
\label{eq:outgoing_rays_inv}
\dfrac{dv}{dr} &=& \dfrac{2}{1-\rh/r}  \nn \\
&= &2\left( 1 + \dfrac{\rh}{r} \right) + {\cal O}\left(\left(\dfrac{\rh}{r}\right)^2\right).
\eea
Considering just the leading order asymptotic behaviour, eq.~\eqref{eq:outgoing_rays_inv} integrates to
\beq
\label{eq:v_to_tau}
v = 2 r + 2 \rh \ln\left( \dfrac{r}{\rh} \right) + C.
\eeq
The hyperboloidal time coordinate $\tau$ is then chosen to be proportional to the integration constant $C=\lambda \tau$. Mapping eq.~\eqref{eq:v_to_tau} back into the original Schwarzschild time coordinate $t$ yields
\bea
t &=& v - r_* \nn \\
  &=& \lambda \tau + r +\rh\ln\left( \dfrac{r}{\rh}\right) - \rh\ln\left( 1-\dfrac{\rh}{r} \right).
  \label{eq:t_to_tau_in_r}
\eea
Before discussing the compactifcation function $r=r(\sigma),$ we are already able at this stage to identify the structure for a height function $h(r)$ via $t = \lambda \tau - h(r)$. From eq.~\eqref{eq:t_to_tau_in_r}, it follows
\bea
\label{eq:h_r}
h(r) = -r  - \rh\ln\left( \dfrac{r}{\rh}\right) + \rh\ln\left(1- \dfrac{r}{\rh} \right).
\eea
The structure of eq.~\eqref{eq:h_r} is very similar to the tortoise coordinate $r_*(r)$ in eq.~\eqref{eq:r*_Schwarzschild}, with the only difference in the sign for the singular terms in the limit $r\rightarrow \infty$.

Finally, one must specify a compactification function $r(\sigma)$. The most simple strategy to parametrise the black-hole external region is
\beq
\label{eq:r_of_sigma}
r = \dfrac{\rh}{\sigma}.
\eeq
With this choice, the definitions \eqref{eq:HypCoord} and \eqref{eq:beta} yield the quantities
\beq
\label{eq:rho_beta_cte}
\rho(\sigma) = \beta(\sigma) = \dfrac{\rh}{\lambda}.
\eeq
 Besides, using eq.~\eqref{eq:r_of_sigma} in \eqref{eq:r*_Schwarzschild}, we read the dimensionless tortoise coordinate \eqref{eq:def_x}
 \beq
 \label{eq:tortoise_x_schwarzschild}
x(\sigma) = \dfrac{\rh}{\lambda}\left( \dfrac{1}{\sigma} -\ln\sigma + \ln(1-\sigma) \right).
 \eeq
 
Moreover, eqs.~\eqref{eq:t_to_tau_in_r} and \eqref{eq:r_of_sigma} provide the mapping $t=t(\tau,\sigma)$ from which one reads the height function 
\beq
\label{eq:Height_H}
H(\sigma) = \dfrac{\rh}{\lambda}\left( -\dfrac{1}{\sigma} + \ln \sigma + \ln\left( 1-\sigma \right) \right).
\eeq
It becomes evident once again, that the height function $H(\sigma)$ in the minimal gauge \eqref{eq:Height_H} follows from changing the sign of the singular components around $\sigma =0$ in the tortoise coordinate $x(\sigma)$ \eqref{eq:tortoise_x_schwarzschild}. This property is generic, as it will be showed in sec.~\ref{sec:min_gauge}

Even though this strategy was the one used in refs.~\cite{Schinkel:2013tka,Schinkel:2013zm,Ansorg:2016ztf,PanossoMacedo:2018hab,PanossoMacedo:2019npm}, we present here an alternative derivation of eq.~\eqref{eq:Height_H} based on a similar intuitive process. Now, instead of choosing the ingoing Eddington-Finkelstein coordinate $v=t + r_*$ as the first step, we opt to first consider the outgoing null coordinate $u=t - r_*$, together with the compactification \eqref{eq:r_of_sigma}. This choice ensures that future null infinity is reached for $\sigma=0$. Along $u =$ constant, however, $\sigma = 1$ corresponds to the white-hole horizon. Ingoing null geodesic in the $(u,\sigma, \theta,\varphi)$ coordinate system are given by
\bea
\label{eq:du_dsigma}
\dfrac{du}{dr} =- \dfrac{2}{1-\rh/r} \quad \stackrel{{\rm eq.~}\eqref{eq:r_of_sigma}}{\longrightarrow} \quad \dfrac{du}{d\sigma} = \dfrac{2 \rh}{\sigma^2 \left(1-\sigma\right)} = \dfrac{2\rh}{1-\sigma} +{\cal O}(1-\sigma). 
\eea
Integrating the leading order behaviour of eq.~\eqref{eq:du_dsigma} around $\sigma=1$ yields
\beq
\label{eq:u_of_tau}
u = - 2\rh \ln\left(1-\sigma \right) + C.
\eeq
Once again, we can choose the the integration constant to be proportional to the new time coordinate $\tau$. Mapping eq.~\eqref{eq:u_of_tau} back to $t = u + r_*$ yields the height function $H(\sigma)$ as in eq.~\eqref{eq:Height_H}.


% 
From eqs.~\eqref{eq:tortoise_x_schwarzschild} and \eqref{eq:Height_H} the metric components follow from the definitions \eqref{eq:def_p}, \eqref{eq:def_gamma} and \eqref{eq:def_w}
\beq
\label{eq:metric_func_mingauge_Schwarzschild}
p(\sigma) = \dfrac{\lambda}{\rh}\sigma^2(1-\sigma), \quad \gamma(\sigma) = 1-2\sigma^2, \quad w(\sigma) = \dfrac{4\rh}{\lambda}(1+\sigma).
\eeq
The properties discussed in eq.~\eqref{eq:prop_p}, \eqref{eq:spacelike}-\eqref{eq:align_nullvectors_hrz} are directly verified.
 
The behaviour outlined here provides the basics steps to construct hyperboloidal slices in the so-called minimal gauge. Sec.~\ref{sec:min_gauge} brings further generic consideration about this gauge.




\section{Black Hole Perturbation Theory}\label{sec:pert_theory}
To exemplify the effects of the hyperboloidal coordinates on the perturbation equations for black-hole perturbation theory, we consider scenarios in which the dynamics of the fields are reducible to a single master equation of the form
\beq
\label{eq:waveeq_Schwarzschild}
-\Psi_{,tt} + \Psi_{,r_* r_*} - {\cal V}(r)\,\Psi = R(r).
\eeq
We recall that this form is not necessarily the most generic expression for spherically symmetric spacetimes, with line element of the form \eqref{eq:metric_tr} as discussed, for instance, in refs.~\cite{Liu:2022csl,Cardoso:2022whc}. However, eq.~\eqref{eq:waveeq_Schwarzschild} applies for a wide range of scenarios with spherical symmetry, and it provides us with an useful proxy for understanding how to apply the hyperboloidal methods in black-hole perturbation theory. The discussion in this section will focus on two independent approaches to tackle eq.~\eqref{eq:waveeq_Schwarzschild}: time and frequency domain calculations.

\subsection{Time domain}
To express the wave equation \eqref{eq:waveeq_Schwarzschild} in the hyperboloidal coordinates $(\tau,\sigma,\theta,\varphi)$, we first derive the following relation from the coordinate transformations \eqref{eq:HypCoord}
 and \eqref{eq:def_x}\footnote{We also express here the relation $\partial_r = -\dfrac{1}{\lambda {\cal F}(\sigma)} \bigg( \gamma(\sigma)\partial_\tau  + p(\sigma) \partial_\sigma  \bigg)$ for consistency and completeness. }

\bea
\partial_t = \lambda^{-1} \partial_\tau, \quad 
\partial_{r_*} = - \lambda^{-1}  \bigg( \gamma(\sigma) \partial_\tau  + p(\sigma) \partial_\sigma \bigg) \\
\eea
Eq.~\eqref{eq:waveeq_Schwarzschild} then transforms to
\beq
\label{eq:AxialWaveEq_hyperboloidal}
-w\, \overline \Psi_{,\tau \tau} + \boldsymbol{L_1}  \overline \Psi + \boldsymbol{L_2} \overline \Psi_{,\tau}=  \overline R
\eeq
with 
\beq 
\label{eq:masterfunc_coordmapp}
\overline \Psi(\tau,\sigma) = \Psi(t,r)
\eeq 
and
\beq
\overline R(\tau, \sigma) = \dfrac{\lambda^2}{p(\sigma)} R(t,r).
\eeq
Besides, the operators $\boldsymbol{L_1}$ and $\boldsymbol{L_2}$ read
\bea
\boldsymbol{L_1} &=&   \partial_\sigma \bigg(  p(\sigma) \partial_\sigma  \bigg) - {\overline {\cal V}}(\sigma) ,\\
\boldsymbol{L_2} &=&   2\gamma(\sigma)\partial_\sigma + \partial_\sigma \gamma(\sigma),
\eea 
with $w(\sigma)$, $p(\sigma)$ and $\gamma(\sigma)$ given by eqs.~\eqref{eq:def_p},\eqref{eq:def_gamma} and \eqref{eq:def_w}. The operator $\boldsymbol{L_1}$ assumes a Sturm–Liouville form, whereas $\boldsymbol{L_2}$ is responsible for the energy dissipation at the boundary~\cite{jaramillo2021pseudospectrum,Gasperin:2021kfv}.
%
The re-scaled potential reads 
\bea
\label{eq:HypFunc_Pot}
{\overline {\cal V}}(\sigma) &= \dfrac{\lambda^2}{p(\sigma)} {{\cal V}}(r).
\eea

With a first order reduction in time $\overline \Phi = \overline \Psi_{,\tau}$, one re-casts eq.~\eqref{eq:AxialWaveEq_hyperboloidal} in terms of an evolution operator $\boldsymbol{L}$ as
\begin{equation}
\label{matrix_eq_time_domain}
- \vec{U}_{,\tau} + \boldsymbol{L} \vec{U} = \vec{S},
\end{equation}
with
\begin{equation}\label{matrix evolution}
 	\boldsymbol {L} =
\left(
\begin{array}{c  c}
	{0} & {1} \\
	%\hline 
	w^{-1}\boldsymbol{L}_1 & w^{-1}\boldsymbol{L}_2
\end{array}
\right), \quad \vec U=\left(
\begin{array}{c}\overline\Psi \\ \overline{\Phi} \end{array}\right), \quad
\vec S=\left(
\begin{array}{c}0 \\ w^{-1} \overline{R} \end{array}\right).
\end{equation}
Of particular relevance for the study of QNMs is the homogenous version of eq.~\eqref{matrix_eq_time_domain}
\begin{equation}
\label{eigenvalue_eq_time_domain}
 \vec{U}_{,\tau} = \boldsymbol{L} \vec{U}.
\end{equation}

\subsection{Frequency domain}
In coordinates $(t,r)$, the frequency domain equation follows from a Fourier transform 
\beq
\label{eq:fourierTransf_tr_coord}
\Psi(t,r) = e^{-i \omega t} \psi(r)
\eeq
and also 
\beq
R(t,r) = e^{-i \omega t} {\cal R}(r).
\eeq 
Applied to \eqref{eq:waveeq_Schwarzschild}, these transformations yield
\beq
\label{eq:fourierEq_tr_coord}
\psi_{,r_* r_*} - \left( {\cal V} - \omega^2 \right)\,\psi = {\cal R}.
\eeq
%
The corresponding frequency domain equation within the hyperboloidal approach is obtained in two equivalent ways: (i) via direct Fourier transformation of eq.~\eqref{matrix_eq_time_domain}; or (ii) via a coordinate change and conformal re-scaling of the frequency domain eq.~\eqref{eq:fourierEq_tr_coord}.

The first approach follows from directly transforming eq.~\eqref{eq:AxialWaveEq_hyperboloidal} into the frequency domain via 
\beq
\label{eq:fourierTransf_tausigma_coord}
\overline \Psi(\tau,\sigma) = e^{s \tau} \overline \psi(\sigma),
\eeq
together with 
\beq
\overline  R(\tau, \sigma) = e^{s\tau} \overline {\cal R}(\sigma).
\eeq
The dimensionless frequency $s$ relates to the Fourier frequency $\omega$ via $s=-i \lambda \omega$. The conformal field $\overline \psi(\sigma)$ then satisfies 
\beq
\label{axial_eq}
\boldsymbol{A} \overline \psi = \overline {\cal R} ,
\eeq
with the operator $\boldsymbol{A}$
\beq
\label{eq:OperatorA_firstOrderRed}
\boldsymbol A = ( \boldsymbol{L_1} - w(\sigma) s^2) + s \boldsymbol{L_2}.
\eeq
We will present the expression for the source term $\overline{R}(\sigma)$ in the end of this section. For now, we concentrate on the homogenous equation $\boldsymbol{A} \overline \psi = 0$, for which eq.~\eqref{eigenvalue_eq_time_domain} yields the eigenvalue problem 
\begin{equation}\label{eigenvalue_eq}
	\boldsymbol{L} \vec{u} = s \vec{u}, \quad \vec u=\left(
\begin{array}{c}\overline\psi \\ \overline{\phi} \end{array}\right).
\end{equation}

The second option derives the hyperboloidal frequency domain expression directly from the equivalent frequency domain eq.~\eqref{eq:fourierEq_tr_coord} via a re-scaling 
\beq
\label{eq:FuncTransfHyp_FreqDomain}
\psi(r) = Z(\sigma) \, \overline \psi(\sigma).
\eeq
The function $Z(\sigma)$ follows from the identity \eqref{eq:masterfunc_coordmapp}, after one relates the Fourier transformation \eqref{eq:fourierTransf_tr_coord} and \eqref{eq:fourierTransf_tausigma_coord} via the coordinate mapping \eqref{eq:HypCoord}. Indeed, by noticing that
\bea
e^{-i\omega t} &=& e^{-i\omega \lambda\left(\tau -H(\sigma) \right)}  \nn \\
&=& e^{s \tau} e^{-s H(\sigma)}.
\eea
one reads
\beq
\label{eq:funcZ}
 Z(\sigma) = e^{s H(\sigma)}.
\eeq
Recall that $\overline \psi(\sigma)$ is defined in the conformal manifold with metric $\overline g_{ab}$. This function is, therefore, regular in the domain $\sigma\in[\sigma_i, \sigma_f]$. The pre-factor $Z(\sigma)$ is responsible for introducing the boundary behaviours into the physical field $\psi(r)$ via eq.~\eqref{eq:FuncTransfHyp_FreqDomain}. Eq.~\eqref{eq:funcZ} shows that the boundary behaviour is geometrically capture by the specific choice for the height function $H(\sigma)$.

Finally, to map the operator from eq.~\eqref{eq:fourierEq_tr_coord} into \eqref{axial_eq} via the frequency domain re-scaling \eqref{eq:FuncTransfHyp_FreqDomain}, one needs the relations arising from the radial coordinate change in eq.~\eqref{eq:HypCoord} 
\bea
%\dfrac{d}{dr} = -r_{\rm h}^{-1} \sigma^2  \dfrac{d}{d\sigma} \leftrightarrow
\dfrac{d}{dr_*} = - \dfrac{p(\sigma)}{\lambda} \dfrac{d}{d\sigma} \quad \Longleftrightarrow \quad \dfrac{d}{dr} = - \dfrac{\sigma^2}{\lambda \beta(\sigma)} \dfrac{d}{d\sigma}.
\eea
In particular, one may express the operator $\boldsymbol{A}$ in the alternative form
\bea
\boldsymbol A &=& \alpha_2 \dfrac{d^2}{d\sigma^2} +  \alpha_1 \dfrac{d}{d\sigma} +  \alpha_0, \label{eq:OperatorA_SecOrde}
\eea
with the coefficients
\bea
\alpha_2(\sigma) &=& p(\sigma), \label{eq:alpha_2_axial}\\
\alpha_1(\sigma) &=& p'(\sigma) + 2s\, \gamma(\sigma), \\
\alpha_0(\sigma) &=& - s^2\, w(\sigma) + s\, \gamma'(\sigma) - {\overline{\cal V}}(\sigma).
\eea
Beside, the source term reads
\beq
\label{eq:Source_Hyp}
\overline{\cal R}(\sigma) =  \dfrac{\lambda^2}{p(\sigma) Z(\sigma)} {\cal R}(r).
\eeq
% 
While the form \eqref{eq:OperatorA_firstOrderRed} is more suitable for calculating QNMs~\cite{jaramillo2021pseudospectrum} through the eigenvalue system \eqref{eigenvalue_eq}, the representation \eqref{eq:OperatorA_SecOrde} is better adapted to the techniques introduced in ref.~\cite{PanossoMacedo:2022fdi} for solving the inhomogeneous equation with a source supported on dirac delta distributions.

%%%%%%%%%%%%%%%%%
\section{The hyperboloidal Minimal Gauge}\label{sec:min_gauge}
In this section we lay out the procedure to construct hyperboloidal slices within a class referred to as the minimal gauge \cite{Schinkel:2013tka,Schinkel:2013zm,Ansorg:2016ztf,PanossoMacedo:2018hab,PanossoMacedo:2019npm}. It expands on the intuitive scheme presented in sec.~\ref{sec:mingauge_schwarzschild} and it consist in retaining the minimal structure for radial function $\rho(\sigma)$ and the height function $H(\sigma)$ needed to achieve a hyperboloidal foliation of the spacetime.

\subsection{The radial compactification degree of freedom}\label{sec:min_gauge_radial}
Motivated by eq.~\eqref{eq:rho_beta_cte}, we define the radial compactification in the minimal gauge by imposing $\beta(\sigma)$ to be constant~\cite{PanossoMacedo:2018hab}. From eq.~\eqref{eq:beta}, this choice yields
\beq
\label{eq:rho_mingauge}
\rho(\sigma) = \rho_0 + \rho_1 \sigma \Longrightarrow \beta(\sigma) = \rho_0.
\eeq
%
Without loss of generality, one can fix the black-hole horizon $\rh$ at the coordinate vaule $\sigmah=1$. Eqs.~\eqref{eq:HypCoord} and \eqref{eq:rho_mingauge} then yield
\beq
\label{eq:eventhorz_fix}
\rho(\sigmah)=\dfrac{\rh}{\lambda} \Longrightarrow \rho_0 = \dfrac{\rh}{\lambda} - \rho_1.
\eeq

The radial compactification simplifies even further if one opts to set $\rho_1=0$, i.e., with the mapping $r(\sigma)$ given by eq.~\eqref{eq:r_of_sigma}. 
%
However, one can exploit the freedom in the choice $\rho_1$ to fix relevant spacetime surfaces in the compact coordinate $\sigma$. For instance, refs.~\cite{PanossoMacedo:2018hab,PanossoMacedo:2019npm} show that one can choose $\rho_1$ such that the Cauchy horizon in Reisner-Nordstr\"on (or Kerr) spacetime is always fixed at a given coordinate value $\sigma_{c} \rightarrow \infty$, regardless of the charge (or spin) parameters. Here, we exploit this property to introduce trumpet-minimal gauge hyperboloidal hypersurface in sec.~\ref{sec:trumpet}.

The compactification along the radial direction is also expressed in terms of the dimensionless tortoise coordinate $x(\sigma)$ via eqs.~\eqref{eq:def_x} and \eqref{eq:dx_dsigma}. The specific form for $x(\sigma)$ depends on the spacetime under consideration, a behaviour determined by the particular form of ${\cal F}(\sigma)$. 

We recall that ${\cal F}(\sigma)=0$ at the horizons $\sigmahi$ and, for asymptotically flat spacetimes, at $\sigma=0$. Since eq.~\eqref{eq:dx_dsigma} behaves as $x' \sim 1/{\cal F}$, the roots of the function ${\cal F}$ will introduce singular terms to $x(\sigma)$. Let us express the dimensionless tortoise coordinate as 
\beq
\label{eq:x_reg+sing}
x(\sigma) = x_{\rm sing}(\sigma) + x_{\rm reg}(\sigma).
\eeq
In the next paragraphs, we discuss how to construct an explicit representation for the terms contributing to $x_{\rm sing}(\sigma)$. The regular piece will then follow from integrating
\bea
\label{eq:dxreg_dsigma}
x_{\rm reg}'(\sigma) &=& x'(\sigma) - x_{\rm sing}'(\sigma) \nn \\
			       &=& - \dfrac{\rho_0}{\sigma^2 {\mathcal F(\sigma)}} - x_{\rm sing}'(\sigma).
\eea

\paragraph{Asymptotically flat spacetimes:}
In asymptotically flat spacetimes, one term in $x_{\rm sing}(\sigma)$ comes from the behaviour of ${\cal F}$ around $\sigma=0$. Eq.~\eqref{eq:dx_dsigma} with the asymptotic behaviour \eqref{eq:F_of_sigma_asymp_Mink} lead to
\beq
x'(\sigma) = -\dfrac{\rho_0}{\sigma^2}\left( 1+ \dfrac{2M}{\lambda} \dfrac{\sigma}{\rho_0} + {\cal O}(\sigma^2)\right).
\eeq
Thus, the leading terms contribute with the singular quantities
\beq
\label{eq:x0}
x_0(\sigma) = \dfrac{\rho_0}{\sigma} - \dfrac{2M}{\lambda}\ln \sigma.
\eeq

\paragraph{Horizons:} Each horizon $\sigmahi$ contributes to $x_{\rm sing}(\sigma)$ with a respective $x_{{\rm h}_i}(\sigma)$ term. From the definition \eqref{eq:dx_dsigma} and the representation \eqref{eq:F_of_sigma_j} around a given horizon $\sigmahi$, one obtains
\beq
x'(\sigma) = -\dfrac{\rho_0}{\sigma^2 {\cal K}_i(\sigma) \left(1- \dfrac{\rhi}{r(\sigma)} \right) }.
\eeq
The integration around the horizon $\sigmahi$ yields
\beq
\label{eq:x_horizons}
x_{{\rm h}_i}(\sigma) =\dfrac{\rhi}{\lambda\, {\cal K}_{{\rm h}_i}} \ln\left| \sigma - \sigmahi \right|, \quad {\cal K}_{{\rm h}_i}={\cal K}_i(\sigmahi).
%\dfrac{ r_{{\rm h}_i}  }{\lambda}\, \dfrac{1}{{\cal K}_{{\rm h}_i}}\, \ln\left( 1- \dfrac{r_{{\rm h}_i}}{r(\sigma)}\right), \quad {\cal K}_{{\rm h}_i}={\cal K}_i(\sigmahi).
\eeq

%Finally, $x_{\rm reg}(\sigma)$ is a regular part defined by the differential equation
%\bea
%\label{eq:dxreg_dsigma}
%x'_{\rm reg}(\sigma) &=& - \dfrac{\rho_0}{\sigma^2 {\mathcal F(\sigma)}} - \Bigg(\sum_{i=0}^{N_{\rm h}} x'_{h_i} + x'_0 \Bigg). %\nn \\
%%&=& \dfrac{\rho_0}{\sigma^2} \left(1 + \dfrac{2M}{\lambda\, \rho_0}\sigma + \sum_{i=0}^{N_{\rm h}}\left[ \left(\dfrac{\rhi}{r(\sigma)}\right)^2 \dfrac{1}{{\cal K}_{{\rm h}_i}} \left(1-\dfrac{r_{{\rm h}_i}}{r(\sigma)} \right)^{-1}\right] - \dfrac{1}{{\cal F}(\sigma)} \right).
%\eea

\paragraph{Asymptotically deSitter spacetimes:}
For asymptotically deSitter spacetimes, the contribution around $\sigma=0$ to the dimensionless tortoise coordinate is actually regular. Indeed, around $\sigma=0$, eq.~\eqref{eq:dx_dsigma} with the asymptotic expansion \eqref{eq:F_of_sigma_asymp_dS} yields
\beq
\label{eq:x0_dS}
x'(\sigma) = \dfrac{3}{\Lambda \lambda^2 \rho_0}\left( 1 + {\cal O}(\sigma^2)\right) \Longrightarrow x_0(\sigma) = \dfrac{3}{\Lambda \lambda^2 \rho_0} \sigma,
\eeq
which is a term contributing only to the regular part $x_{\rm reg}(\sigma)$ in eq.~\eqref{eq:x_reg+sing}. All terms contributing $x_{\rm sing}(\sigma)$ come actually from the behaviours around the horizon as in eq.~\eqref{eq:x_horizons}. 

Given the importance of the cosmological horizon in asymptotically deSitter spacetimes, we explicitly single out its contribution from eq.~\eqref{eq:x_horizons}. For that purpose, we label this horizon $\sigma=\sigma_{\Lambda}$ and we associate the index value $i=N_h$ in the product expansion showed in eqs.~\eqref{eq:func_f_horizons} or \eqref{eq:F_of_sigma}. The contribution to $x_{\rm sing}(\sigma)$ from $\sigma_{\Lambda}$ reads 
\beq
x_{\Lambda}(\sigma) =\dfrac{r_{\Lambda}}{\lambda\, {\cal K}_{\Lambda}} \ln\left| \sigma - \sigma_\Lambda \right|.
\eeq

All in all, the dimensionless tortoise coordinate reads
\bea
\label{eq:x_flat}
x(\sigma) &=&  \sum_{i=0}^{N_{\rm h}} x_{h_i}(\sigma) + x_0(\sigma) + x_{\rm reg}(\sigma) \quad \textnormal{(Asymptotically flat)}, \\
\label{eq:x_dS}
&=&\sum_{i=0}^{N_{\rm h}-1} x_{h_i}(\sigma) + x_{\Lambda}(\sigma) + x_{\rm reg}(\sigma) \quad \textnormal{(Asymptotically deSitter)}.
\eea



%An explicit representation for the function $x(\sigma)$ in asymptotically flat spacetimes is found in the form
%\beq
%x(\sigma) = \sum_{i=0}^{N_{\rm h}} x_{h_i}(\sigma) + x_0(\sigma) + x_{\rm reg}(\sigma)
%\eeq

%The functions $x_i(\sigma)$ are the singular contributions arising from the $n_{\rm h}$ spacetime horizons. From the definition \eqref{eq:dx_dsigma} and the behaviour \eqref{eq:F_of_sigma_j} around a given horizon $\sigmahi$, one obtains
%\beq
%x'(\sigma) = -\dfrac{\rho_0}{\sigma^2 {\cal K}_i(\sigma) \left(1- \dfrac{\rhi}{r(\sigma)} \right) },
%\eeq
%whose integration around the horizon $\sigmahi$ yields
%\beq
%\label{eq:x_horizons}
%x_{{\rm h}_i}(\sigma) =\dfrac{\rhi}{\lambda\, {\cal K}_{{\rm h}_i}} \ln\left| \sigma - \sigmahi \right|, \quad {\cal K}_{{\rm h}_i}={\cal K}_i(\sigmahi).
%%\dfrac{ r_{{\rm h}_i}  }{\lambda}\, \dfrac{1}{{\cal K}_{{\rm h}_i}}\, \ln\left( 1- \dfrac{r_{{\rm h}_i}}{r(\sigma)}\right), \quad {\cal K}_{{\rm h}_i}={\cal K}_i(\sigmahi).
%\eeq



\subsection{The height function}\label{sec:MinGauge_Height}
Following the receipt from sec.~\eqref{sec:mingauge_schwarzschild}, the height function in the minimal gauge $H(\sigma)$ arises from studying the behaviour of ingoing and outgoing null geodesics around the surfaces one wishes the hyperboloidal slice to intersect. 

For the Schwarzschild spacetime exemplified in sec.~\ref{sec:mingauge_schwarzschild}, this goal was achieved by either one of the procedures: (i) consider ingoing null geodesics $v = t + r_*$, and then integrate outgoing null geodesics asymptotically around future null infinity $\sigma=0$; or (ii) consider outgoing null geodesics $u = t - r_*$, and then integrate ingoing null geodesics around the black hole horizon $\sigma=1$. 

While in Schwarzschild, both approaches are equivalent, more generic spacetimes may display a richer structure, for instance with Cauchy and/or cosmological horizons. Thus, the strategy may need adaption to account for each one of the horizons in the spacetime. Here, we will discuss the approaches (i) and (ii) in a generic formalism, and then apply them to specific examples in sec.~\ref{sec:examples_spacetime}.

\subsubsection{The in-out strategy}
The in-out strategy is the approach originally described in refs.~\cite{Schinkel:2013tka,Schinkel:2013zm,Ansorg:2016ztf,PanossoMacedo:2018hab,PanossoMacedo:2019npm}. It consists of first considering {\em ingoing} null geodescis $v=t+r_*$, which ensures that the time hypersurface penetrates the black-hole horizon. If a Cauchy horizon also exists in the spacetime, such as in the Reisnner-Nordstr\"om case, this choice also ensures that the hypersurface intersects the Cauchy horizon. In the coordinate $(v,\sigma, \theta, \varphi)$, outgoing null geodesics follow from
\bea
\label{eq:out_null}
\dfrac{dv}{d\sigma}  =  -\dfrac{2 \lambda \rho_0}{\sigma^2{\cal F}(\sigma)}.
\eea
\paragraph{Asymptotically flat spacetimes\\}
We first consider asymptotically flat spacetimes, for which ${\cal F}(\sigma)$ behaves as eq.~\eqref{eq:F_of_sigma_asymp_Mink}. Eq.~\eqref{eq:out_null} expands around $\sigma=0$ as
\bea
\label{eq:out_null_asymptopic}
\dfrac{dv}{d\sigma}  =  -\dfrac{2 \lambda \rho_0}{\sigma^2}\Bigg( 1 + \dfrac{2M}{\lambda} \dfrac{\sigma}{\rho_0} + {\cal O}(\sigma^2)  \Bigg).
\eea
Considering only the leading terms, eq.~\eqref{eq:out_null_asymptopic} integrates to
\beq
v = \lambda \bigg( \tau  - {\cal H}_0(\sigma) \bigg), \quad {\cal H}_0(\sigma) = - \dfrac{2  \rho_0}{\sigma} + \dfrac{4M}{\lambda}\ln \sigma,
\eeq
where the integration constant is chosen to be the proportional to the new time coordinate $\tau$.
Mapping ingoing coordinate $v$ back to the Schwarzschild time $t=v-r_*$ yields
\beq
t = \lambda \bigg( \tau - {\cal H}_0(\sigma) - x(\sigma) \bigg) \Longrightarrow H(\sigma) = x(\sigma) + {\cal H}_0(\sigma).
\eeq
From eq.~\eqref{eq:x0}, one observes that ${\cal H}_0(\sigma) = - 2\,x_0(\sigma)$. Thus, as in the Schwarzschild case, the height function in the minimal gauge follows from the tortoise coordinate $x(\sigma)$ in eq.~\eqref{eq:x_flat} by just changing the sign from the singular contribution around $\sigma=0$ 
\beq
\label{eq:H_inout_flat}
H(\sigma) = \sum_{i=0}^{N_{\rm h}} x_{h_i}(\sigma) - x_0(\sigma) + x_{\rm reg}(\sigma).
\eeq


\paragraph{Asymptotically deSitter spacetimes\\}
In asymptotically deSitter spacetimes however, the contribution around $\sigma=0$ is regular, as discussed in sec.~\ref{sec:min_gauge_radial}. Thus, the outgoing null geodesics \eqref{eq:out_null} must be integrated in the vicinity of the Cosmological horizon. For that purpose, eq.~\eqref{eq:out_null} is reformulated with the help of eq.~\eqref{eq:F_of_sigma_j} as
\beq
\label{eq:out_null_asymptopic_dS}
\dfrac{dv}{d\sigma}  =  -\dfrac{2 \lambda \rho_0}{\sigma^2 {\cal K}_\Lambda(\sigma)}\left(1- \dfrac{r_\Lambda}{r(\sigma)} \right)^{-1}.
\eeq
Integration around $r_\Lambda=r(\sigma_\Lambda)$ provides
\beq
v = \lambda \bigg( \tau - {\cal H}_{\Lambda}(\sigma)\bigg), \quad {\cal H}_{\Lambda}(\sigma) = -2 \dfrac{r_{\Lambda}}{\lambda} \dfrac{1}{ {\cal K}_{\Lambda}}\ln\left|\sigma- \sigma_{\Lambda}\right|.
\eeq
As in the asymptotically flat case, we observe that ${\cal H}_{\Lambda}(\sigma) = -2 x_\Lambda(\sigma)$. Thus, the contribution of ${\cal H}_{\Lambda}(\sigma)$ when mapping the coordinate $v$ back into $t = v - r_*$ is just a sign change between the height function $H(\sigma)$ and the dimensionless tortoise coordinate $x(\sigma)$ in the singular part corresponding to the Cosmological horizon
\beq
\label{eq:H_inout_dS}
H(\sigma) = \sum_{i=0}^{N_{\rm h}-1} x_{h_i}(\sigma) - x_{\Lambda}(\sigma)  + x_{\rm reg}(\sigma).
\eeq

\subsubsection{The out-in strategy}
The out-in strategy is analogous to the procedure describe in the previous section, but the first step in the construction of height function is taken by considered {\em outgoing} null geodesics $u = t- r_*$. This choice ensure that that along $u=$ constant, the limit $r\rightarrow \infty$ leads to future infinity, regardless whether the spacetime is asymptotically flat, or deSitter. For asymptotically deSitter spacetimes, this option also ensures that the Cosmological horizon is intersected at $r=r_\Lambda$. On the other end of the surface $u=$ constant, event and Cauchy horizons are those associated with the white hole region. Thus, a hyperboloidal foliation will require modifications on the time coordinate around the horizons. 

Ingoing null geodesics in the $(u,\sigma,\theta, \varphi)$ coordinates is described by 
\bea
\dfrac{du}{d\sigma}  &=&  \dfrac{2 \lambda \rho_0}{\sigma^2{\cal F}(\sigma)} \nn \\
\label{eq:in_null}
&=& \dfrac{2 \rhi}{{\cal K}_{{\rm h}_i}}\left( \sigmahi - \sigma  \right)^{-1},
\eea
where we have employed representation eq.~\eqref{eq:F_of_sigma_j} in the second line to facilitate the study around a given horizon $\rhi=r(\sigmahi)$. Integrating \eqref{eq:in_null} around $\sigmahi$ and choosing the integration constant as proportional to the hyperboloidal time provide
\beq
\label{eq:height_u_hrz}
u = \lambda\bigg(\tau - {\cal H}_{{\rm h}_i}(\sigma)\bigg), \quad {\cal H}_{{\rm h}_i}(\sigma) =  2 \dfrac{\rhi}{\lambda \, {\cal K}_{{\rm h}_i}} \ln\left|\sigmahi -\sigma \right|.
\eeq
Mapping the outgoing coordinate $u$ back to the Schwarzschild time $t = u +r_*$ yields
\beq
t = \lambda \Bigg( \tau - \bigg({\cal H}_{{\rm h}_i}(\sigma) - x(\sigma)\bigg) \Bigg).
\eeq
The contribution to the height function from each horizon is ${\cal H}_{\rm i}(\sigma) = 2 x_{{\rm h}_i}(\sigma)$, c.f. eq.~\eqref{eq:x_horizons}. 

If procedure \eqref{eq:height_u_hrz} is applied to {\em all} horizons $\rhi$ 
 ($i=0\cdots N_{\rm h}$) in asymptotically flat spacetimes, then the height functions becomes
 \bea
 \label{eq:Houtin}
 H(\sigma) &=& \sum_{i=0}^{N_{\rm h}} {\cal H}_{{\rm h}_i}(\sigma) - x(\sigma) \nn \\
 		 &=& \sum_{i=0}^{N_{\rm h}} x_{h_i}(\sigma) - x_0(\sigma) - x_{\rm reg}(\sigma).
    \label{eq:H_outin_flat}
 \eea
As already discussed, in asymptotically deSitter spacetimes the cosmological horizon $r_\Lambda$ is treated differently from black-hole horizons. Thus, applying procedure \eqref{eq:height_u_hrz} to the horizons $\rhi$ 
 ($i=0\cdots N_{\rm h}-1$) gives
  \beq
  \label{eq:H_outin_dS}
 H(\sigma) = \sum_{i=0}^{N_{\rm h}-1} x_{h_i}(\sigma) - x_\Lambda(\sigma) - x_{\rm reg}(\sigma).
 \eeq
 
 As for the in-out strategy, the height function in the minimal gauge displays just a change in sign from the singular contribution around $\sigma=0$ (asymptotically flat) or $\sigma=\sigma_{\Lambda}$ (asymptotically deSitter) when compared against the dimensionless tortoise coordinate \eqref{eq:x_flat}-\eqref{eq:x_dS}. Eqs.~\eqref{eq:Houtin}-\eqref{eq:H_outin_dS}, however, also show the opposite sign in the regular term with respect to $x(\sigma)$.
 
 \subsubsection{In-out versus out-in}
 A direct comparison between the expressions for the height function in the in-out \eqref{eq:H_inout_flat} and the out-in \eqref{eq:H_outin_flat} strategies --- or equivalently \eqref{eq:H_inout_dS} and \eqref{eq:H_outin_dS} --- shows that they differ by a factor $2 x_{\rm reg}(\sigma)$. In the example of the Schwarzschild case, discussed in sec.~\ref{sec:mingauge_schwarzschild}, $x_{\rm reg}(\sigma) = 0$ and both strategies provide the same results. All in all, these strategies will be equivalent whenever  $x_{\rm reg}(\sigma)$ is an overall constant. Such a constant only shifts the origin of the time coordinate $\tau$, and therefore, it does not contribute to the geometrical properties of the hyperboloidal slices. Indeed, eq.~\eqref{eq:def_gamma} shows that the boost function $\gamma(\sigma)$ depends only on derivatives of $H(\sigma)$. 
 
 For more intricate solutions, the contribution of a non constant regular part $x_{\rm reg}(\sigma)$ may change some geometrical properties of the hyperboloidal hypersurfaces.  Indeed, from the identity 
 \beq
 \label{eq:Hinout-Houtin}
 H^{\io} -H^{\oi} = 2 \, x_{\rm reg},
 \eeq
 one derives
 \bea
 \label{eq:gamma_inout_outin}
 \gamma^{\io} - \gamma^{\oi} &=& 2 \, x_{\rm reg}' \, p, \\ 
  \label{eq:w_inout_outin}
 w^{\io}- w^{\oi} &=& - 2 x_{\rm reg}' ( \gamma^{\io} +  \gamma^{\oi}).
 \eea
 Thus, in some cases one approach will be more favourable towards the other, specifically when leading to the violation of conditions \eqref{eq:spacelike}, \eqref{eq:align_nullvectors_hrz} and \eqref{eq:align_nullvectors_hrz}.
 

The next section applies this formalism to several spacetimes, where we discuss different features, properties and possible issues arising during the calculation.
 
 
\section{Example of spacetimes}\label{sec:examples_spacetime}
We now expand on the example from section \ref{sec:mingauge_schwarzschild}, where the hyperboloidal minimal gauge for Schwarzschild was introduced. We apply the generic formalism developed for for the minimal gauge to several spacetimes to exemplify the several features discussed. First, we revisit the minimal gauge on Schwarzschild and discuss the role of the parameter $\rho_1$ in compactification function eq.~\eqref{eq:rho_mingauge}. The second case discusses the Reisnner-Nordstrom-deSitter spacetime. This solution has several horizons and it provides a good application to exemplify the in-out and out-in strategies. Then, we discuss the case of high-dimensional Schwarzschild black-hole, where the in-out and out-in strategies lead to two alternative hyperboloidal foliations in the minimal gauge. Finally, the solution modelling a central black hole with a dark matter halo from ref.~\cite{Cardoso:2021wlq} brings an example in which the minimal gauge strategy only works with out-in strategy.



\subsection{Hyperboloidal trumpet slices in Schwarzschild}\label{sec:trumpet}
Though section \ref{sec:mingauge_schwarzschild} constructed the minimal gauge hyperboloidal slice for Schwarzschild, eq.~\eqref{eq:rho_mingauge} shows that there are two free parameters, namely $\rho_0$ and $\rho_1$, when choosing the radial compactfication $r(\sigma)$. The former is fixed by imposing that the black-hole horizon $\rh$ is at the coordinate $\sigma=1$, c.f.~eq~.\eqref{eq:eventhorz_fix}. Sec.~\ref{sec:mingauge_schwarzschild} fixed the later to $\rho_1=0$ and the result hyperboloidal foliation is adequate to describe the exterior black hole region in the domain $\sigma\in[0,1]$. With this choice the physical singularity $r=0$ is mapped into $\sigma_o \rightarrow \infty$. 

Here we follow the strategy devised in refs.~\cite{PanossoMacedo:2018hab,PanossoMacedo:2019npm} and exploit the extra degree of freedom to extend the hyperboloidal foliation up to the singularity, by mapping the surface $r=0$ into the coordinate value $\sigma=\sigma_o>1$. This property follows for
\beq
\rho_1 = \dfrac{\rh}{\lambda} \dfrac{1}{1-\sigma_o}.
\eeq
Thus, the complete radial transformation \eqref{eq:HypCoord} with $\rho(\sigma)$ in the minimal gauge \eqref{eq:rho_mingauge} is given by
\beq
 r = \dfrac{\rh}{\sigma}\dfrac{1-\sigma/\sigma_o}{1 - 1/\sigma_o}.
\eeq
%
With this choice, the dimensionless tortoise coordinate $x(\sigma)$ mapped directly from $r_*(r)$ in eq.~\eqref{eq:r*_Schwarzschild} reads
\bea
&&x(\sigma) = \dfrac{\rh}{\lambda}\Bigg[ \dfrac{1-\sigma/\sigma_o}{\sigma(1 - 1/\sigma_o)} + \ln\left( \dfrac{1-\sigma}{\sigma(1-1/\sigma_o)}  \right) \Bigg]  \\
&&= \dfrac{\rh}{\lambda}\Bigg[ \dfrac{1}{\sigma(1 - 1/\sigma_o)}  - \ln  \sigma   + \ln\left| 1-\sigma \right|   - \left(  \dfrac{1}{\sigma_o(1 - 1/\sigma_o)} + \ln\left( 1-1/\sigma_o  \right)    \right)  \Bigg]. \nn
	       \label{eq:x_trumpet}
\eea
From the second line in the expression above we identify the contributions coming from future null infinity as in eq.~\eqref{eq:x0}
\beq
x_0(\sigma) =\dfrac{\rh}{\lambda}\Bigg( \dfrac{1}{\sigma(1 - 1/\sigma_o)}  - \ln  \sigma\Bigg),
\eeq
and, as expressed in eq.~\eqref{eq:x_horizons}, the contribution from horizon
\beq
x_{\rm h}(\sigma)= \dfrac{\rh}{\lambda} \ln\left| 1-\sigma \right|.
\eeq
%
With these two individual terms, one observes that $x'_{\rm reg} = 0$ in eq.~\eqref{eq:dxreg_dsigma}. Indeed, the regular piece in eq.~\eqref{eq:x_trumpet} is just the constant 
\beq
x_{\rm reg}(\sigma)=- \dfrac{\rh}{\lambda}\ \left(  \dfrac{1}{\sigma_o(1 - 1/\sigma_o)} + \ln\left( 1-1/\sigma_o  \right)    \right),
\eeq 
which ensures that $x\rightarrow 0$ as $\sigma\rightarrow \sigma_o$.

The hyperboloidal foliation is now valid in the domain is $\sigma\in[0,\sigma_o]$, with $\sigma=0$ corresponding to $\scri^+$, $\sigma=1$ the black-hole horizon, and $\sigma=\sigma_o>1$ the singularity. With this choice the functions in eqs.~\eqref{eq:def_p}, \eqref{eq:def_gamma} and \eqref{eq:def_w} become
\bea
\label{eq:metric_funcs_trumpet_Schwarzschild}
p(\sigma)&=&\dfrac{\lambda}{\rh}\dfrac{\sigma^2 (1-\sigma) (1-1/\sigma_o)}{1-\sigma/\sigma_o}, \nn \\
\upgamma(\sigma) &=& \dfrac{(1-2\sigma^2) - (1 - 2\sigma)\sigma/\sigma_o}{1 - \sigma/\sigma_o}, \\
w(\sigma) &=& \dfrac{4 \rh}{\lambda} \dfrac{1+\sigma(1 - 1/\sigma_o)}{1-\sigma/\sigma_o}. \nn
\eea
In the limit $\sigma_o\rightarrow \infty$, the above expressions reduce to the ones in refs.~\cite{Ansorg:2016ztf,PanossoMacedo:2018hab,PanossoMacedo:2019npm}, c.f. eq.~\eqref{eq:metric_func_mingauge_Schwarzschild}.

Of particular interest is the behaviour of the lapse function $\alpha \sim \sqrt{\sigma_o - \sigma}$, c.f. \eqref{eq:3+1}, which is typical for a so-called trumpet slice \cite{Dennison:2014sma}. Even though such slices may play an important role when performing full non-linear evolutions of the Einstein's equations in the hyperboloidal setup, a complete study of this foliation goes beyond the scope of this work. Figure \ref{fig:trumpet} displays the Penrose diagram for the Schwarzschild spacetime with the compact hyperboloidal coordinates extending between the singularity at $\sigma=\sigma_o$ and future null infinity at $\sigma=0$.

% Figure environment removed

\subsection{Reisnner-Nordstr\"om-de Sitter spacetimes}
The Reisnner-Nordstr\"om-de Sitter spacetime is given by the line element \eqref{eq:metric_tr} with metric functions
\beq
a(r) = b(r) = 1 - \dfrac{2M}{r} + \dfrac{Q^2}{r^2} - \dfrac{\Lambda}{3}r^2.
\eeq
Instead of parametrising the spacetime by its mass $M$, charge $Q$ and cosmological constant $\Lambda$, we opt to express the metric functions in terms of the event horizon $\rh$, Cauchy horizon $r_{\rm C}$ and cosmological horizon $r_\Lambda$ via
\beq
\label{eq:RNdSHorizons}
a(r) = b(r) =- r^2\dfrac{\Lambda}{3}\left( 1- \dfrac{\rh}{r}\right)\left( 1- \dfrac{r_{\rm C}}{r}\right)\left( 1-\dfrac{r_\Lambda}{r} \right)\left( 1  - \dfrac{r_o}{r} \right).
\eeq
Even though the root $r_o = - \left( \rh + r_{\rm C} + r_\Lambda\right)$ is non-physical because it assumes negative values, eq.~\eqref{eq:RNdSHorizons} has precisely the form presented in eq.~\eqref{eq:func_f_horizons}. 

With the radial compactification \eqref{eq:r_of_sigma}, the horizons and the negative root are mapped into
\beq
\sigmah = 1, \quad \sigma_{\rm C}= \rh/r_{\rm C}, \quad \sigma_{\Lambda}= \rh/r_{\Lambda}, \quad \sigma_{o}= \rh/r_{o}.
\eeq
Then, eq.~\eqref{eq:x_horizons} applied to each one of the $n_{\rm h}= 4$ roots of \eqref{eq:RNdSHorizons} yields the dimensionless tortoise coordinate
\beq
\label{eq:x_RNdS}
x(\sigma) = x_{\rm h}(\sigma) + x_{\rm C}(\sigma) + x_{\Lambda}(\sigma) + x_{o}(\sigma),
\eeq
with
\bea
x_{\rm h}(\sigma) &=&- \dfrac{3\, \sigmah}{\lambda r_h \Lambda}\dfrac{\ln(\sigmah-\sigma)}{(1 - \sigmah/\sigma_{\rm C})(1 - \sigmah/\sigma_\Lambda)(1 - \sigmah/\sigma_o)}, \\
x_{\rm C}(\sigma) &=&- \dfrac{3 \, \sigma_{\rm C}}{\lambda r_h \Lambda}\dfrac{\ln(\sigma_{\rm C}-\sigma)}{(1 - \sigma_{\rm C}/\sigmah)(1 - \sigma_{\rm C}/\sigma_\Lambda)(1 - \sigma_{\rm C}/\sigma_o)}, \\
x_{\Lambda}(\sigma) &=& - \dfrac{3 \, \sigma_\Lambda}{\lambda r_h \Lambda}\dfrac{\ln(\sigma-\sigma_{\Lambda})}{(1 - \sigma_{\Lambda}/\sigmah)(1 -  \sigma_{\Lambda}/\sigma_{\rm C})(1 -  \sigma_{\Lambda}/\sigma_o)}, \\
x_{o}(\sigma)&=& - \dfrac{3 \sigma_o}{\lambda r_h \Lambda}\dfrac{\ln(\sigma-\sigma_{o})}{(1 - \sigma_{o}/\sigmah)(1 -  \sigma_{o}/\sigma_{\rm C})(1 -  \sigma_o/\sigma_{\Lambda})}.
\eea
One can verify that eq.~\eqref{eq:x_RNdS} reproduces the behaviour \eqref{eq:x0_dS} as $\sigma\rightarrow 0$. Besides, when applied to eq.~\eqref{eq:dxreg_dsigma}, eq.~\eqref{eq:x_RNdS} yields $x'_{\rm reg} =0$, which allows one to chose $x_{\rm reg}(\sigma)=0$. 

The results from sec.~\ref{sec:MinGauge_Height} shows that the height function in the minimal gauge follows from just changing the sign of the cosmological term $x_{\Lambda}(\sigma)$ within the expression for the dimensionless tortoise coordinate, i.e.
\beq
\label{eq:H_RNdS}
H(\sigma) = x_{\rm h}(\sigma) + x_{\rm C}(\sigma) - x_{\Lambda}(\sigma) + x_{o}(\sigma).
\eeq
Moreover, thanks to the vanishing of the regular term $x_{\rm reg}(\sigma)$, the in-out and out-in strategies lead to the same final result. %The expressions for the metric components \eqref{eq:def_p}, \eqref{eq:def_gamma} and \eqref{eq:def_w} become rat


\subsection{Higher dimension spacetimes: the Schwarzschild-Tangherlini spacetime}{\label{sec:HigDimSch}}
The Schwarzschild solution in spacetimes dimensions $d$ higher than $4$ provides an useful exercise to expand on the minimal gauge calculation laid out in sec.~\ref{sec:min_gauge}. In particular, it is the first example in which the in-out and out-in strategies give different results.

The line element for $d>4$ reads
\beq
\label{eq:metric_tr_highdim}
ds^2 = - f(r) dt^2 + \dfrac{dr^2}{f(r)} + r^2 d\omega^2_{d-2}, \quad f(r) = 1 -\left(\dfrac{\rh}{r}\right)^{d-3}
\eeq
with $d\omega^2_{d-2}$ the metric for the unit sphere in $d-2$ dimensions \cite{Tangherlini:1963bw}. We work in the minimal gauge with the compactficaiton function \eqref{eq:rho_mingauge}. Then, eq.~\eqref{eq:F_of_sigma} assumes the form
\bea
{\cal F}(\sigma) &=& (1-\sigma^{d-3}) \nn \\
\label{eq:F_highdim}
		 	&=&	  (1-\sigma)\, {\cal K}(\sigma), \quad {\cal K}(\sigma) = \sum_{k=0}^{d-4} \sigma^k.
\eea
To construct the dimensionless tortoise coordinate $x(\sigma)$, one needs to modify slightly the contribution from $\sigma\rightarrow 0$ originally given by eq.~\eqref{eq:x0}. Indeed, expanding eq.~\eqref{eq:dx_dsigma} around $\sigma=0$ leads to
\beq
x'(\sigma) = -\dfrac{\rh}{\lambda} \Bigg( \sigma^{-2} + (d-3) \sigma^{d-5} + {\cal O}(\sigma^{d-4}) \Bigg).
\eeq
When $d>4$, the singular contribution comes only from the leading term $\sigma^{-2}$, and therefore
\beq
x_0(\sigma) = \dfrac{\rh}{\lambda\, \sigma},
\eeq
i.e. the logarithmic term from eq.~\eqref{eq:x0} is present only for $d=4$. The singular contribution to $x(\sigma)$ arising from the horizon, though, follows directly from eq.~\eqref{eq:x_horizons}
\beq
x_{\rm h}(\sigma) = \dfrac{\rh}{\lambda}\dfrac{\ln|1-\sigma|}{d-3}.
\eeq
After removing the singular contributions as in eq.~\eqref{eq:dxreg_dsigma}, one observes that the regular term $x_{\rm reg}(\sigma)$ is non-trivial and defined by the differential equation
\beq
\label{eq:dx_reg_dsigma_highdim}
x'_{\rm reg}(\sigma) = \dfrac{\rh}{\lambda} \Bigg(  \dfrac{1}{d-3} - \dfrac{\sigma^{d-5}}{\cal K(\sigma)}\Bigg)  \left(1-\sigma\right)^{-1}.
\eeq
From eq.~\eqref{eq:F_highdim} it follows ${\cal K}(0)=1$ and ${\cal K}(1)=d-3$. These properties ensure that eq.~\eqref{eq:dx_reg_dsigma_highdim} is indeed regular in the interval $\sigma\in[0,1]$. 

Due to the non-trivial character of the regular term $x_{\rm reg}(\sigma)$, the height function in the minimal gauge differs in the in-out and out-in strategies. With the singular contributions $x_0(\sigma)$ and $x_{\rm h}(\sigma)$ the height functions read from eqs.~\eqref{eq:H_inout_flat} and \eqref{eq:H_outin_flat}, respectively
\beq
\label{eq:H_MinGague_IO_OI}
H^{\io}(\sigma) = - x_0(\sigma) + x_{\rm h}(\sigma) + x_{\rm reg}(\sigma), \quad H^{\oi}(\sigma) = - x_0(\sigma) + x_{\rm h}(\sigma) - x_{\rm reg}(\sigma)
\eeq
%
The explicit form of $x_{\rm reg}(\sigma)$ is not needed to construct the line element functions \eqref{eq:def_p}, \eqref{eq:def_gamma} and \eqref{eq:def_w}. The function $p(\sigma)$ is independent of the height function, and it reads
\beq
p(\sigma)= \dfrac{\lambda}{\rh}\sigma^2\left( 1-\sigma^{d-3} \right).
\eeq
For the in-out strategy, the functions $\gamma(\sigma)$ and $w(\sigma)$ assume the rather simple form
\beq
\gamma^{\io}(\sigma) = 1 - 2 \sigma^{d-3}, \quad w^{\io}(\sigma) = \dfrac{4\rh}{\lambda}\sigma^{d-5}.
\eeq
Conditions \eqref{eq:spacelike}, \eqref{eq:align_nullvectors_hrz} and \eqref{eq:align_nullvectors_scri} are all satisfied since $\gamma^{\io}(\sigma)$ has a similar structure to its $d=4$ counterpart in eq.~\eqref{eq:metric_func_mingauge_Schwarzschild}. However, the expression in the limit $d\rightarrow 4$ is not directly recovered because the minimal gauge includes only the leading order term in $x_0(\sigma)$. As explained, $x_0(\sigma)$ contains a logarithmic divergence in $d=4$, which is absent when $d>4$. 

% Figure environment removed

It is also intersting to observe that the function $w^{\io}(\sigma)$ vanishes at $\sigma=0$ for $d>5$. This property indicates that the hypersurfaces with $\tau =$ constant becomes null at future null infinity when $d>5$. Recall that eqs.~\eqref{eq:nullvector_scri} provide the expression for the outgoing and ingoing conformal null vectors at $\scri^+$. To ensure a regular expression for their components, one must choose the parameter $\nu(\sigma) = w(\sigma)/2$, which leads to 
\beq
\label{eq:nullvector_scri_highdim}
\left. \bar l^a\right|_{\sigma=0} = - \delta^a_\sigma, \quad \left. \bar k^a\right|_{\sigma=0}  = \delta^a_\tau \quad (d>5).
\eeq
Expressions \eqref{eq:nullvector_scri_highdim} confirms that, at future null infinity, not only does the ingoing null vector $\bar k^a$ aligns with the coordinate basis vector $\partial_\tau$, but also the outgoing null vector $\bar l^a$ aligns with the $\partial_\sigma$.

These properties change if one follows the out-in strategy, from which one derives
\beq
\gamma^{\oi}(\sigma) = 1  - 2   \dfrac{\sigma^2{\cal K}(\sigma)}{d-3}, \quad
 w^{\oi}(\sigma) = \dfrac{4\rh}{\lambda (d-3)} \Bigg(1 - \sigma^2 \dfrac{{\cal K}(\sigma)}{d-3} \Bigg)\left( 1-\sigma\right)^{-1}.
\eeq
Though less straightforward, one can again verify that conditions \eqref{eq:spacelike}, \eqref{eq:align_nullvectors_hrz} and \eqref{eq:align_nullvectors_scri} are all satisfied. With this strategy, however,  $w^{\oi}(0)$ never vanishes and the hyperboloidal surfaces are spacelike in the entire domain $\sigma\in[0,1]$ for all dimensions $d$. Fig.~\ref{fig:gamma_w_highdim} shows the functions $\gamma(\sigma)$ and $w(\sigma)$ for the in-out (solid lines) and out-in (dashed lines) strategies for $d=5\cdots 10$.

\subsection{Black-hole + matter halo}\label{sec:BHHalo}
Ref.~\cite{Cardoso:2021wlq} introduced a family of solutions to Einstein's equations with gravity minimally coupled to an anisotropic fluid. These asymptotically flat spacetimes describe a regular horizon with "hair", and they model the geometry of galaxies with supermassive black holes. The functions in the line element ~\eqref{eq:metric_tr} read
\beq
a(r) = \left( 1- \dfrac{M_{\rm BH}}{r} \right) e^{\Upsilon(r)}, \quad b(r) = 1 - \dfrac{2 m(r)}{r},
\eeq
with the red-shift $\Upsilon(r)$ and mass functions $m(r)$ given by
\bea
\Upsilon(r)&=&\sqrt{\frac{M}{\xi}}\Bigg(2\arctan\left(\frac{r+a_0-M_0}{\sqrt{M_0\xi}}\right)-\pi\Bigg), \\ 
m(r) &=& M_{\rm BH} + \dfrac{M_0 r^2}{(a_0 +r)^2}\left(1 - \dfrac{2M_{\rm BH}}{r} \right)^2.
\eea
The parameters of the solution are the black-hole mass $M_{\rm BH}$, the mass $M_0$ of a "halo" surrounding the black hole, and a typical lengthscale $a_0$ for the matter distribution. From these quantities one derives the the black-hole radius $\rh=2M_{\rm BH}$ and the parameter $\xi=2 a_0 - M_0 + 4 M_{\rm BH}$.

This is the first example in which the functions $a(r)$ and $b(r)$ in eq.~\eqref{eq:metric_tr} do not coincide. To employ the formalism developed in the previous sections, it is more convenient to re-express
\beq
b(r) = \left( 1- \dfrac{r_{\rm h}}{r} \right) B(r), \quad B(r)= 1 - 2\left( 1- \dfrac{r_{\rm h}}{r}\right)\dfrac{M r}{(a_0+r)^2},
\eeq
from which the function $f(r)$ defined in eq.~\eqref{eq:def_tortoise} assumes directly the form \eqref{eq:func_f_horizons}
\beq
\label{eq:f_BHHalo}
f(r) = \left( 1- \dfrac{r_{\rm h}}{r} \right) K(r), \quad K(r) = \sqrt{e^{\Upsilon(r)} B(r)}.
\eeq
Comparing the asymptotic expansion of eq.~\eqref{eq:f_BHHalo} against eq.~\eqref{eq:f_asymp_Mink}, one reads the total ADM mass $M = M_{\rm BH} + M_0$. As expected, the same result follows if one uses eq.~\eqref{eq:BondiMass} to calculate the Trautman-Bondi mass.
Following the parametrisation employed in refs.~\cite{Cardoso:2021wlq,Cardoso:2022whc}, we re-scale all dimensional quantities in terms of the black-hole mass $M_{\rm BH}$, so that $M_0 = \mu\, M_{\rm BH},$ $a_0 = \upalpha_0 \, M_{\rm BH}$ and $\xi = \upxi \, M_{\rm BH}$.

To construct a hyperboloidal foliation in the minimal gauge, we first compactify the radial coordinate according to eq.~\eqref{eq:r_of_sigma}. In terms of the compact radial coordinate, the relevant metric functions read
\bea
\upupsilon(\sigma) &=& \Upsilon(r(\sigma)) \nn\\
			     &=& \sqrt{\frac{\mu}{\upxi}}\Bigg(2\arctan\left(\frac{2+(\upalpha_0-\mu)\sigma}{\sigma\sqrt{\mu\upxi}}\right)-\pi\Bigg), \\
\mathfrak{B}(\sigma)	    &=& B(r(\sigma)) \nn \\
				&=& 1 - \dfrac{4\sigma(1-\sigma)\mu}{(2+\upalpha_0 \sigma)^2}.	
\eea
With these expressions, one derives from eqs.~\eqref{eq:F_of_sigma} and \eqref{eq:zeta}
\beq
{\cal K}(\sigma) = \sqrt{e^{\upupsilon(\sigma)} \mathfrak{B}(\sigma) }, \quad \zeta(\sigma) = \sqrt{\dfrac{e^{\upupsilon(\sigma)}}{\mathfrak{B}(\sigma) }}.
\eeq
At future null infinity $\sigma=0$ and the horizon $\sigma=1$, the function ${\cal K}(\sigma)$ assumes the values
\beq
{\cal K}(0)=1, \quad {\cal K}(1)={\cal K}_{\rm h} =  e^{ \sqrt{\frac{\mu}{\upxi}}\Bigg(\arctan\left(\frac{2+(\upalpha_0-\mu)}{\sqrt{\mu\upxi}}\right)-\dfrac{\pi}{2}\Bigg)}.
\eeq
Further metric components in the hyperboloidal line element \eqref{eq:conf_metric} requires the calculation of the dimensionless tortoise coordinate $x(\sigma)$. The singular terms follows from eqs.~\eqref{eq:x0} and \eqref{eq:x_horizons} and they read
\beq
x_0(\sigma) = \dfrac{\rh}{\lambda}\left( \dfrac{1}{\sigma} - (1+\mu) \ln \sigma \right), \quad x_{\rm h}(\sigma) = \dfrac{\rh}{\lambda} \dfrac{\ln(1-\sigma)}{{\cal K}_{\rm h}}.
\eeq

Similar to the previous section, the regular contribution to $x(\sigma)$ is non-trivial, and it is defined by the differential equation
\beq
\label{eq:xreg_BHHalo}
\dfrac{dx_{\rm reg}}{d\sigma} = \dfrac{r_{\rm h}}{\lambda}\dfrac{1}{\sigma^2 (1-\sigma)} \Bigg( -\dfrac{1}{{\cal K}(\sigma)} + \dfrac{\sigma^2}{{\cal K}_{\rm h}} + (1-\sigma) \bigg[1+(1+\mu)\sigma \bigg]\Bigg). 
\eeq
The limits $\sigma\rightarrow 0$ and $\sigma \rightarrow 1$ are regular in the above expression. The tortoise coordinate defines the metric function $p(\sigma)$, c.f.~eqs.~\eqref{eq:p_from_x} and \eqref{eq:def_p}
\beq
p(\sigma) = \dfrac{\rh}{\lambda} \sigma^2 (1-\sigma){\cal K}(\sigma).
\eeq

% Figure environment removed

With the expression for $x_0(\sigma)$, $x_{\rm h}(\sigma)$ and $x_{\rm reg}(\sigma)$, the height function in the minimal gauge assumes the structure given by eq.~\eqref{eq:H_MinGague_IO_OI}. Here, we also observe that the in-out and out-in strategies provide different solutions, with the remaining metric functions $\gamma(\sigma)$ and $w(\sigma)$ given by
\bea
\gamma^{\io}(\sigma) &=& -1 + 2(1-\sigma)\bigg(1 + \sigma(1+\mu) \bigg) {\cal K}(\sigma), \\
w^{\io}(\sigma) &=& \dfrac{4\rh}{\lambda \sigma^2}\bigg( 1+ \sigma (1+\mu) \bigg) \Bigg( 1 - (1-\sigma)\bigg[1+\sigma(1+\mu) \bigg] {\cal K}(\sigma)\Bigg),
\eea
or
\beq
\gamma^{\oi}(\sigma) = 1- \dfrac{2\sigma^2 {\cal K}(\sigma)}{{\cal K}_{\rm h}}, \quad w^{\oi}(\sigma) = \dfrac{4\rh}{\lambda (1-\sigma){\cal K}_{\rm h}} \bigg( 1 - \sigma^2 \dfrac{{\cal K}(\sigma)}{{\cal K}_{\rm h}}\bigg).
\eeq
Contrary to the results in sec.~\ref{sec:HigDimSch}, not only are the expressions for the in-out strategy lengthier than then ones derived with the out-in strategy, but they may also violate conditions \eqref{eq:spacelike}, \eqref{eq:align_nullvectors_hrz} and \eqref{eq:align_nullvectors_scri}. The violation is observed in fig.~\ref{fig:gamma_w_BHHalo}, where the functions assumes values $|\gamma(\sigma)|>1$ and $w(\sigma)<1$ within the exterior black-hole region $\sigma\in[0,1]$ (in-out strategy --- solid lines). On the other hand, the conditions \eqref{eq:spacelike}, \eqref{eq:align_nullvectors_hrz} and \eqref{eq:align_nullvectors_scri} are always satisfied when using the out-in strategy (dashed lines). Fig.~\ref{fig:gamma_w_BHHalo} shows results for the parameters $(\mu, \upalpha_0)=(1,10)$ (purple) and $(\mu, \upalpha_0)=(10,10)$ (green). We have empirically observed that conditions \eqref{eq:spacelike}, \eqref{eq:align_nullvectors_hrz} and \eqref{eq:align_nullvectors_scri} holds in the out-in strategy for all studies and experiments developed in refs.~\cite{Cardoso:2021wlq,Cardoso:2022whc}.

\section{Conclusion}
This work presents a comprehensive discussion on the conformal compactification of spherically symmetric spacetimes via the hyperboloidal approach. The hyperboloidal foliations consists of spacelike hypersurfaces with asymptotically hyperbolic geometry. On a fixed background,  these surfaces emerges as the level set of a time coordinate $\tau=$ constant, which provides a notion for accessing spacetime null surfaces (such as the black-hole horizon and future null infinity) "at the same time".  

The choice for a particular hyperboloidal foliation is not unique, and we present the formalism for a generic spherically symmetric spacetime without fixing ourselves to a particular gauge, at first. In particular, starting from the most generic line element in the usual Scharzschild-like coordinates $(t,r,\theta,\varphi)$, we introduce the transformation to hyperboloidal coordinates $(\tau, \sigma,\theta,\varphi)$ and discuss the geometrical and physical interpretation to all functions in the hyperboloidal line element \eqref{eq:conf_metric}. For instance, eq.~\eqref{eq:BondiMass} allows us to calculate the Trautman-Bondi mass directly from the hyperboloidal quantities.

The hyperboloidal approach is a powerful tool for studying wave propagation problems and it has proved its value in the field of black-hole perturbation theory. Thus, this work reviews and applies the generic hyperboloidal framework to a wide class of wave equations relevant to the linear regime of general relativity. The framework provides the starting point for studies of binary black holes dynamics via the the hyperboloidal approach where the linear regime is applicable: the inspiral of objects with extreme mass ratio and the ring-down. 

Even though the discussion presented is generic enough to comprise a large class of scenarios relevant to astrophysical studies, it is restricted to problems in which the dynamics are reducible to one wave equation for a single master function. The presented framework does not include, for instance, the use of hyperboloidal tools to studies of EMRI's via the self-force programme in the Lorentz gauge\cite{Akcay:2010dx,Akcay:2013wfa,Osburn:2022bby}. Further development is also required for scenarios with matter distributions, where the wave equations for density fluctuations couple with the gravitational waves degrees of freedom\cite{Allen:1997xj,Cardoso:2022whc}.

Among the possible choices for hyperboloidal foliations, this work re-visits the so-called minimal gauge. This class of gauges provides the minimal structure for the coordinate degrees of freedom, so that the surfaces of $\tau=$ constant are hyperboloidal. The procedure for its construction was initially identified by Marcus Ansorg in refs.~\cite{Schinkel:2013tka,Schinkel:2013zm,PanossoMacedo:2014dnr} and formalised in refs.~\cite{Ansorg:2016ztf,PanossoMacedo:2018hab,PanossoMacedo:2019npm}. This work develops further this technique to make it applicable to any spherically symmetric spacetime. 

The most important result is that the height function in the minimal gauge follows directly from the tortoise coordinate by a simple change of sign in the terms singular at future null infinity (or cosmological horizon for asymptotically de-Sitter spacetimes). Even though this results provides a simple algorithm for constructing the height function in the minimal gauge, one must ensure that the resulting $\tau=$ constant surfaces are indeed hyperboloidal and satisfy conditions \eqref{eq:spacelike}-\eqref{eq:align_nullvectors_scri}. 

More specifically, this works shows that the conceptual strategy for constructing the minimal gauge height function can be cast into two alternative approaches: (i) the in-out strategy and (ii) the out-in strategy. The former considers first ingoing null geodesics, to ensure that black-hole horizons are intersected by the time surfaces. Then, it integrates outgoing null geodesics asymptotically around future null infinity (or cosmological horizons). The latter considers first outgoing null geodesics, to ensure that future null infinity (or cosmological horizons) are intersected. Then, it integrates ingoing null geodesics around the horizons. 

These two strategies are completely equivalent for a large class of spacetimes, such as Schwarzschild, Reisnner-N\"ordstrom(-deSitter), extending up to Kerr. However, they may differ for more generic solutions. In this work, we identify the reason why the in-out and out-in strategies may not lead to the same results. The answer lies in the structure of the tortoise coordinate. We express the tortoise coordinates as the sum of terms singular at future null infinity and horizons, plus a regular term. For solutions in which the regular term is a constant (usually zero), the in-out and out-in strategies provide the same hyperboloidal foliations in the minimal gauge. For solutions with a non-trivial regular piece, these strategies yield different results, as expressed in eq.~\eqref{eq:Hinout-Houtin}. The preference for one strategy over the other is essentially based first on mathematical grounds, i.e., if conditions \eqref{eq:spacelike}-\eqref{eq:align_nullvectors_scri} are violated and the resulting $\tau=$ constant surfaces are not hyperboloidal, and second on practical terms, i.e., if one approach gives simpler expressions than the other.   

Finally, we apply the formalism to the following spacetimes: Schwarzschild, Reisnner-N\"ostroim-deSitter, Schwarschild-Tangherlini and black-hole + halo. Each one of these examples discusses one particular aspect of the hyperboloidal minimal gauge, such as the choice for compactification function \eqref{eq:rho_mingauge} and the outcomes for the in-out and out-in strategy. Most importantly, the example in sec.~\ref{sec:BHHalo} shows that the out-in strategy is preferable. This heuristic results motivates us to conjecture that hyperboloidal slices in the minimal gauge are always available via the out-in strategy, but a formal proof goes beyond the scope of this work.


\section*{Acknowledgments}
The author would like to thank Anil Zenginoglu, David Hilditch and Alex Va\~no-Vi\~nuales for valuable discussions. Also, the author thanks Juan Valiente Kroon, Grigalus Tayjanskas and the Royal Society for the invitation and financial support to join the scientific meeting "At the interface of asymptotic, conformal methods and analysis in general relativity". This work was financed by the VILLUM Foundation (grant no. VIL37766), the DNRF Chair program (grant no. DNRF162) by the Danish National Research Foundation, and the European Union’s H2020 ERC Advanced Grant "Black holes: gravitational engines of discovery" grant agreement no. Gravitas–101052587. Views and opinions expressed are however those of the author only and do not necessarily reflect those of the European Union or the European Research Council. Neither the European Union nor the granting authority can be held responsible for them.


\section*{Bibliography}
\bibliographystyle{unsrt.bst}
\bibliography{bibitems}


\end{document}

