%%%%%%%%%%%%%%%%%%%%%%% file template.tex %%%%%%%%%%%%%%%%%%%%%%%%%
%
% This is a template file for COCV 
%
% Copy it to a new file with a new name and use it as the basis
% for your article
%
%%%%%%%%%%%%%%%%%%%%%%%%   EDP Sciences  %%%%%%%%%%%%%%%%%%%%%%%%%%
%
\documentclass{article}
\usepackage{amsfonts,amsmath,amsthm,amssymb}

\date{}

\usepackage{algorithm}
\usepackage{algorithmic}
%\usepackage{amsthm}
\usepackage{subfigure}
\usepackage{epsfig}
\usepackage{graphicx}
\usepackage{url}
\usepackage{multirow}
\usepackage{amssymb}
\usepackage{mathrsfs}
\usepackage{epstopdf}
\usepackage{multicol}
\usepackage{amsmath}
\usepackage{mathtools}

\usepackage{multicol}
\usepackage{wrapfig,lipsum,booktabs}
\usepackage{bm}
\usepackage{boxedminipage}
\usepackage{xcolor}

\usepackage{multirow}

\usepackage{pifont}% http://ctan.org/pkg/pifont
\newcommand{\cmark}{\ding{51}}%
\newcommand{\xmark}{\ding{55}}%


\allowdisplaybreaks[4]

\newtheorem{theorem}{\textbf{Theorem}}
\newtheorem{assumption}{\textbf{Assumption}}
\newtheorem{lemma}{\textbf{Lemma}}
\newtheorem{corollary}{\textbf{Corollary}}
\newtheorem{remark}{\textbf{Remark}}

\renewcommand{\algorithmicrequire}{\textbf{Input:}}
\renewcommand{\algorithmicensure}{\textbf{Output:}}

\DeclarePairedDelimiter\ceil{\lceil}{\rceil}
\DeclarePairedDelimiter\floor{\lfloor}{\rfloor}


%
%%%%%%%%%%%%%--PREAMBLE--%%%%%%%%%%%%%%%%%%
\numberwithin{equation}{section}
\newtheorem{ssmptn}{Assumption}
\DeclareMathOperator*{\dive}{div}
\DeclareMathOperator*{\esssup}{ess\,sup}
\DeclareMathOperator*{\essinf}{ess\,inf}
\newcommand{\e}{\varepsilon}
\newcommand{\al}{\alpha}
\newcommand{\bino}{\bigskip\noindent}
\newcommand{\lmi}{\mathop{\mbox{\rm liminf}}\limits}

\theoremstyle{plain}
\newtheorem{thrm}{Theorem}[section]
\newtheorem{lmm}[thrm]{Lemma}
\newtheorem{crllr}[thrm]{Corollary}
\newtheorem{prpstn}[thrm]{Proposition}
\newtheorem{crtrn}[thrm]{Criterion}
\newtheorem{lgrthm}[thrm]{Algorithm}
\theoremstyle{definition}
\newtheorem{dfntn}[thrm]{Definition}
\newtheorem{cnjctr}[thrm]{Conjecture}
\newtheorem{xmpl}[thrm]{Example}
\newtheorem{prblm}[thrm]{Problem}
\newtheorem{rmrk}[thrm]{Remark}
\newtheorem{nt}[thrm]{Note}
\newtheorem{clm}[thrm]{Claim}
\newtheorem{smmr}[thrm]{Summary}
\newtheorem{cs}[thrm]{Case}
\newtheorem{bsrvtn}[thrm]{Observation}

%%%%%%%%%%%%%%%%%%%%%%%%%%%%%%%%%%%%%%%%%%%

\setlength{\oddsidemargin}{0in}
\setlength{\evensidemargin}{0in}
\setlength{\textwidth}{6.5in}
\setlength{\textheight}{9.3in}
\setlength{\topmargin}{-.75in}
\setlength{\abovedisplayskip}{16pt}
\setlength{\belowdisplayskip}{16pt}
\setlength{\abovedisplayshortskip}{16pt}
\setlength{\belowdisplayshortskip}{16pt}

\title{Achieving Linear Speedup in Decentralized Stochastic Compositional Minimax Optimization}


%%%%%%%%%%%%%%%%%%%%%%%%%%%%%%%%%%%%%%%%%%%
\author{
	Hongchang Gao \thanks{Temple University, {\tt hongchang.gao@temple.edu}} 
%	\and  
	}
%%%%%%%%%%%%%%%--BODY--%%%%%%%%%%%%%%%%%%
\begin{document}
\maketitle
%%-----------------------------
%%      the top matter
%%-----------------------------


%
\begin{abstract}
	The stochastic compositional minimax problem has attracted a surge of attention in recent years since it covers many emerging machine learning models. Meanwhile, due to the emergence of distributed data, optimizing this kind of problem under the decentralized setting becomes badly needed. However, the compositional structure in the loss function brings unique challenges to designing efficient decentralized optimization algorithms. In particular, our study shows that the standard gossip communication strategy cannot achieve linear speedup for decentralized compositional minimax problems due to the large consensus error about the inner-level function. To address this issue, we developed a novel decentralized stochastic compositional gradient descent ascent with momentum algorithm to reduce the consensus error in the inner-level function. As such, our theoretical results demonstrates that it is able to achieve linear speedup with respect to the number of workers. We believe this novel algorithmic design could benefit the development of decentralized compositional optimization.   Finally, we applied our methods to the imbalanced classification problem. The extensive experimental results provide evidence for the effectiveness of our algorithm. 
\end{abstract}



\section{Introduction}
In this paper, we consider the decentralized stochastic compositional minimax optimization problem:
\begin{equation} \label{loss}
	\begin{aligned}
		& \min_{\mathbf{x}\in\mathbb{R}^{d_1}} \max_{\mathbf{y}\in \mathbb{R}^{d_2}} \frac{1}{K}\sum_{k=1}^{K} f_k(g_k(\mathbf{x}), \mathbf{y})  \ .
	\end{aligned}
\end{equation}
%where $\mathcal{X}=\mathbb{R}^d$ and $\mathcal{Y}\subset\mathbb{R}^{d'}$ are two convex sets.
Specifically, there are $K$ workers and each worker  communicates with its neighbors in a decentralized manner. Moreover,  each worker $k$ ($k\in \{1,2,\cdots, K\}$) has its own loss function  $f_k(g_k(\mathbf{x}), \mathbf{y})$, which is a compositional function with respect to $\mathbf{x}$. More specifically, the inner-level function is defined as $g_k(\mathbf{x})=\mathbb{E}_{\xi_k}[g_k(\mathbf{x}; \xi_k)]: \mathbb{R}^{d_1} \rightarrow \mathbb{R}^{d'}$, where $\xi_k$ represents the data distribution on the $k$-th worker for the inner-level function. The outer-level function is defined as $f_k(g_k(\mathbf{x}), \mathbf{y})=\mathbb{E}_{\zeta_k}[f_k(g_k(\mathbf{x}), \mathbf{y}; \zeta_k) ]: \mathbb{R}^{d'}\times \mathbb{R}^{d_2}\rightarrow \mathbb{R}$, where $\zeta_k$ denotes the data distribution on the $k$-th worker for the outer-level function.  


The stochastic compositional minimax optimization problem in Eq.~(\ref{loss}) covers many machine learning problems, such as  distributionally robust stochastic compositional optimization problems \cite{gao2021convergence},  stochastic compositional AUROC maximization problems \cite{yuan2021compositional}.  Thus, it has been actively studied and applied to numerous machine learning models in the past few years. However, the compositional structure in the loss function incurs significant challenges to optimize it. Specifically, the stochastic gradient $\nabla_{\mathbf{x}} f_k(g_k(\mathbf{x}; \xi_k), \mathbf{y}; \zeta_k)$  regarding $\mathbf{x}$ is not an unbiased estimation for the full gradient  due to the  compositional structure in the objective function.  This biased estimator causes new challenges for both \textit{the update on individual workers} and \textit{the communication across workers}. 


To address the challenge caused by the compositional structure in the loss function, a lot of efforts have been made under the single-machine setting (i.e., the \textit{update on individual workers}) in the past few years. For instance, \cite{wang2017stochastic} developed the stochastic compositional gradient descent method for  stochastic compositional \textit{minimization} problems. After that, numerous  methods have been developed to improve the convergence rate \cite{zhang2019composite,yuan2019stochastic,zhang2019stochastic,chen2020solving,yuan2020stochastic}. However, all of them only concentrate on the compositional \textit{minimization} problem. It is unclear whether they can converge for  compositional \textit{minimax} problems.  Recently, \cite{gao2021convergence}  developed a stochastic compositional gradient descent ascent method for optimizing Eq.~(\ref{loss}) under the single-machine setting. More recently, \cite{yuan2021compositional} proposed a primal-dual stochastic compositional adaptive method, which also concentrates on the update on individual workers. Thus, these optimization methods cannot be used to optimize  Eq.~(\ref{loss}) since they are not able to address the challenges regarding the \textit{communication across workers} caused by the compositional structure. 

As for the decentralized setting, numerous decentralized optimization methods have been developed to optimize large-scale machine learning models in recent years.  For instance, \cite{lian2017can} developed stochastic gradient descent for nonconvex optimization problems for the \textit{minimization} problem. Afterwards, a wide variety of methods \cite{lian2017can,lu2019gnsd,pu2020distributed,koloskova2019decentralized,tang2019deepsqueeze,sun2020improving,lin2021quasi,yuan2021decentlam,koloskova2021improved}  have been proposed to improve the computation complexity and communication complexity, and some works \cite{xian2021faster} developed algorithms for the minimax optimization problem.  However, all these methods only concentrate on the problem without  the compositional structure. Thus, they fail to  address the unique challenges regarding  \textit{the update on individual workers} and \textit{the communication across workers} for  compositional minimax optimization problems.  Recently, a couple of parallel compositional optimization algorithms  have been proposed under either the decentralized \cite{gao2021fast,zhao2022distributed,yang2022decentralized} or centralized setting \cite{tarzanagh2022fednest,gao2022convergence,wang2021memory}. However, all of them only concentrate on the compositional minimization problem. Additionally, they fail to achieve the linear speedup except \cite{wang2021memory}, which requires an unrealistic operation, i.e., communicating the high dimensional $\nabla g_k(\mathbf{x}; \xi_k)\in \mathbb{R}^{d_2\times d_1}$.   Therefore, \textit{how to address the challenges about the update on individual workers and the communication across workers to achieve the \textbf{linear speedup} for decentralized \textbf{compositional minimax} problems} is still an open problem.  


\vspace{-5pt}
\subsection{Our Contributions}
\vspace{-5pt}
The contributions of this paper lie in the development of novel 
decentralized  stochastic compositional gradient descent ascent methods and the establishment of the linear speedup convergence rate, which are summarized below. 
\begin{itemize}
	\item We first developed a  decentralized stochastic compositional gradient descent ascent with momentum  algorithm based on the gossip communication strategy (D-SCGDAM-GP).  We found D-SCGDAM-GP can only achieve the $O(\frac{1}{\epsilon^2})$ iteration/communication complexity to find the $\epsilon$-accuracy solution, failing to achieve the linear speedup regarding the number of workers, i.e.,  $O(\frac{1}{K\epsilon^2})$. The reason is that the large consensus error of the inner-level function prohibits D-SCGDAM-GP  from achieving that. 
	
	
	
	
	

	\item Based on our findings in the first algorithm, we developed a novel decentralized stochastic compositional gradient descent ascent with momentum  algorithm based on the gradient-tracking communication strategy (D-SCGDAM-GT). In particular, D-SCGDAM-GT communicates the inner-level function value to control the consensus error so that it is able to theoretically achieve linear speedup regarding the number of workers.  To the best of our knowledge, this is the first algorithm  communicating the inner-level function. We believe this novel algorithmic design can benefit the development of  other decentralized compositional optimization problems. 
	
	\item The communication of the inner-level function causes new challenges for convergence analysis. We successfully addressed these challenges and established the convergence rate of our D-SCGDAM-GT algorithm. In particular, it  can achieve the $O(\frac{1}{K\epsilon^2})$ iteration/communication complexity and  $O(\frac{1}{K\epsilon^2})$ sample complexity for  under the nonconvex-strongly-concave problem.  We believe our theoretical analysis strategy regarding the design of communicating the inner-level function can also benefit the  development of  decentralized compositional optimization. 
	
	
	\item Finally, we applied our proposed methods to the decentralized compositional AUROC maximization problem for imbalanced data classification. The extensive experimental results on multiple benchmark datasets confirm the superior performance of our proposed methods. 
\end{itemize}




%\vspace{-15pt}
\section{Related Work}
%\vspace{-5pt}
%\subsection{Compositional Optimization}
\subsection{Stochastic Compositional Minimization and Minimax Problems}
\vspace{-5pt}
%\subsection{Stochastic compositional minimization problem}
The stochastic compositional minimization problem has been extensively studied in the past few years due to its widespread application in machine learning models, e.g., sparse additive models \cite{wang2017stochastic}, model-agnostic meta-learning \cite{finn2017model}. To alleviate the bias of  stochastic gradients caused by the compositional structure in the loss function, \cite{wang2017stochastic} developed a stochastic compositional gradient descent (SCGD) method, which utilizes a moving average strategy to reduce the estimation variance of the inner-level function. As such, SCGD can achieve the  sample complexity $O(\frac{1}{\epsilon^4})$ for nonconvex problems. Afterwards, a series of methods have been proposed to improve its sample complexity. For instance, \cite{yuan2019stochastic} utilized the SPIDER gradient estimator \cite{fang2018spider} so that the sample complexity is improved to $O(\frac{1}{\epsilon^{3/2}})$ . However, this method requires to periodically use a large batch size of samples to compute the gradient, which is not practical for large-scale data.  Recently, \cite{ghadimi2020single} developed a momentum stochastic compositional   gradient descent (SCGDM) method, whose sample complexity is $O(\frac{1}{\epsilon^2})$ with a small batch size.  Thus, this method is efficient for large-scale data and it has been applied to various applications, such as AUPRC maximization problem \cite{qi2021stochastic}. However, all these methods can only optimize the stochastic compositional \textit{minimization} problem, rather than the \textit{minimax compositional}   problem in Eq.~(\ref{loss}).   

%\textbf{AUROC maximization.} 
As for the \textit{compositional minimax} problem, a typical application is Area Under the Curve (AUROC) maximization.  It is an effective method for the imbalanced classification problem. Its goal is to directly optimize the AUROC score, instead of the cross-entropy loss function. Recently, under the\textit{ single-machine setting}, to enable the stochastic training of AUROC maximization  problems,  \cite{ying2016stochastic} reformulates AUROC maximization as a minimax optimization problem under the \textit{single-machine setting} as follows:
\vspace{-10pt}
\begin{equation} \label{auc}
	\begin{aligned}
		& \min_{\bm{\theta}, \hat{\theta}_1, \hat{\theta}_2}\max_{\tilde{\theta}} \frac{1}{n}\sum_{i=1}^{n} \mathcal{L}_i(\bm{\theta}, \hat{\theta}_1, \hat{\theta}_2, \tilde{\theta}) \ , 
	\end{aligned}
	\vspace{-5pt}
\end{equation}
where  
\begin{equation}
	\begin{aligned}
		& \mathcal{L}_i(\bm{\theta}, \hat{\theta}_1, \hat{\theta}_2, \tilde{\theta})=(1-p)(f(\bm{\theta} ; \mathbf{a}_{i})-\hat{\theta}_1)^{2} \mathbb{I}_{[b_{i}=1]} +p(f(\bm{\theta} ; \mathbf{a}_{i})-\hat{\theta}_2)^{2} \mathbb{I}_{[b_{i}=-1]}  -p(1-p) \tilde{\theta}^{2}\\
		& \quad \quad \quad \quad \quad \quad \quad  +2(1+\tilde{\theta} )(p f(\bm{\theta}; \mathbf{a}_{i}) \mathbb{I}_{[b_{i}=-1]}-(1-p) f(\bm{\theta}; \mathbf{a}_{i}) \mathbb{I}_{[b_{i}=1]}) \ .
	\end{aligned}
\end{equation}
Here, $\bm{\theta}\in \mathbb{R}^d$ denotes the model parameter of the classifier $f(\bm{\theta} ; \mathbf{a}_{i})$, $\hat{\theta}_1\in \mathbb{R}$ and $\hat{\theta}_2 \in \mathbb{R}$ are two additional parameters in the minimization subproblem, $\tilde{\theta} \in \mathbb{R}$ is the parameter for the maximization subproblem, $(\mathbf{a}_{i}, b_{i})$ is the input sample, and $p$ is the ratio of positive samples. When the classifier is a nonconvex function, such as deep neural network, Eq.~(\ref{auc}) is a nonconvex-strongly-concave minimax problem, which can be efficiently optimized by stochastic gradient descent ascent. 
However, its empirical performance is not satisfied. Recently, to address this problem, \cite{yuan2021compositional} developed a compositional AUROC maximization method. In particular, it is to optimize a compositional loss function, where the outer-level function is an AUROC loss function and the inner-level function is induced by a cross-entropy loss function. As such, it becomes a nonconvex-strongly-concave compositional minmax  optimization problem:
\begin{equation} \label{compositional_auc}
	\begin{aligned}
		& \min_{\bm{\theta}, \hat{\theta}_1, \hat{\theta}_2}\max_{\tilde{\theta}} \frac{1}{n}\sum_{i=1}^{n} \mathcal{L}_i(\bm{\theta} -\frac{\rho}{n}\sum_{j=1}^{n}\nabla \mathcal{L}_{j}^{cr}(\bm{\theta}), \hat{\theta}_1, \hat{\theta}_2, \tilde{\theta} )\ , 
	\end{aligned}
\end{equation}
where $\mathcal{L}_j^{cr}(\bm{\theta})$ denotes the cross-entropy loss function for the $j$-th sample, $\rho>0$ is the step size. Then, the inner-level function is $g(\bar{\bm{\theta}}) = \bar{\bm{\theta}} - \rho\Delta(\bar{\bm{\theta}})$ where $\bar{\bm{\theta}} = [\bm{\theta}^T, \hat{\theta}_1, \hat{\theta}_2]^T$ and $\Delta(\bar{\bm{\theta}}) = [\frac{1}{n}\sum_{j=1}^{n}(\nabla \mathcal{L}_{j}^{cr}(\bm{\theta}))^T, 0, 0]^T$, and the outer-level function is $f(g(\bar{\bm{\theta}} ), \tilde{\theta})=\frac{1}{n}\sum_{i=1}^{n}\mathcal{L}_i(g(\bar{\bm{\theta}} ), \tilde{\theta})$.  




%\cite{gao2021fast} 
Recently,  \cite{gao2021convergence} studied the stochastic \textit{compositional} minimax problem under the single-machine setting, i.e.,  $K=1$ in Eq.~(\ref{loss}),  and developed the stochastic compositional gradient descent ascent (SCGDA) method. Even though SCGDA can achieve the $O(\frac{1}{\epsilon^2})$ sample complexity when the loss function is  nonconvex-strongly-concave, its batch size should be as large as $O(\frac{1}{\epsilon})$, which is not practical in real-world applications. More recently, \cite{yuan2021compositional} developed the primal-dual stochastic compositional adaptive (PDSCA) method to solve Eq.~(\ref{compositional_auc}). It can achieve the same sample complexity as SCGDA with a constant batch size. However, both SCGDA and PDSCA only concentrate on the single-machine setting. Thus, it is unclear whether they will converge under the decentralized setting. 






\vspace{-5pt}
\subsection{Decentralized Stochastic Compositional Optimization Problems} 
%\vspace{-5pt}
The decentralized optimization method has been extensively studied in recent years. Typically, there are two communication strategies: the gossip strategy and gradient-tracking strategy. The gossip method only communicates the model parameter, while the gradient tracking method communicates both model parameters and gradients. Based on these two communication strategies, numerous decentralized optimization methods have been explored. For instance, \cite{lian2017can} established the convergence rate of the gossip-based decentralized stochastic gradient descent (DSGD) method for nonconvex problems, while \cite{lu2019gnsd} studied the convergence rate of  gradient-tracking-based DSGD. In addition, some efforts have been made to improve the sample complexity and communication complexity by compressing the communicated variables \cite{tang2019deepsqueeze,koloskova2019decentralized} or reducing the gradient variance \cite{sun2020improving,zhang2021gt}. However, all of these methods only concentrate on the \textit{minimization} problem. Even though some efforts \cite{xian2021faster,tsaknakis2020decentralized} have been made for the decentralized minimax optimization methods, they only concentrate on the \textit{non-compositional} problem.  
Thus, they cannot be used to optimize Eq.~(\ref{loss}) and their theoretical analyses are not applicable to our methods. 
As for the compositional optimization problem, \cite{tarzanagh2022fednest,gao2022convergence,wang2021memory} developed stochastic compositional gradient descent methods for minimization problems under the centralized federated learning setting. All of them fail to achieve linear speedup. Recently, \cite{zhang2023federated} developed a centralized federated compositional minimax optimization algorithm. However, due to the periodic global communication, the consensus error is reset to zero periodically so that it is much easier to achieve linear speedup than the decentralized setting, where the consensus error is always non-zero. Under the decentralized setting, \cite{gao2021fast,zhao2022distributed,yang2022decentralized} developed  a couple of algorithms that only work for the compositional minimization problem. Here, only \cite{yang2022decentralized} can achieve linear speedup. But it requires to communicate a very high dimensional matrix $\nabla g_k(\mathbf{x}; \xi_k)\in \mathbb{R}^{d_2\times d_1}$. Taking Eq.~(\ref{compositional_auc}) as an example, if we use ResNet50 as the classifier where  $d_1=d_2$ is as large as  25 million, it is impossible to communicate $d_1^2$ parameters in each iteration in a real-world system. 
In summary, all existing methods are not capable of optimizing Eq.~(\ref{loss}), and their theoretical analyses are not applicable to our settings. Thus, it is necessary to develop new methods with rigorous theoretical guarantees to optimize Eq.~(\ref{loss}).




%\vspace{-5pt}
\section{Problem Setup}
We make the following commonly used assumptions for the stochastic compositional optimization problem \cite{wang2017stochastic,ghadimi2020single,gao2021convergence}. 
\begin{assumption} \label{assumption_smooth}
	For any outer-level function $f_k(\cdot, \cdot)$,  it has $L_f$-Lipschitz continuous gradient, i.e.,  for $\forall (\mathbf{x}_1, \mathbf{y}_1), (\mathbf{x}_2, \mathbf{y}_2)\in \mathbb{R}^{d_1}\times \mathbb{R}^{d_2}$,   $ \|\nabla_{g} f_k(g_k(\mathbf{x}_1), \mathbf{y}_1)-\nabla_{g}  f_k(g_k(\mathbf{x}_2), \mathbf{y}_2)\|^2 \leq L_f^2(\|g_k(\mathbf{x}_1)-g_k(\mathbf{x}_2)\|^2  + \|\mathbf{y}_1- \mathbf{y}_2\|^2 )$ and $\|\nabla_{\mathbf{y}}  f_k(g_k(\mathbf{x}_1), \mathbf{y}_1)-\nabla_{\mathbf{y}}  f_k(g_k(\mathbf{x}_2), \mathbf{y}_2)\|^2 \leq  L_f^2(\|g_k(\mathbf{x}_1)-g_k(\mathbf{x}_2)\|^2  + \|\mathbf{y}_1- \mathbf{y}_2\|^2 ) $,
	%\begin{equation}
	%	\begin{aligned}
	%		& \|\nabla_{g} f_k(g_k(\mathbf{x}_1), \mathbf{y}_1)-\nabla_{g}  f_k(g_k(\mathbf{x}_2), \mathbf{y}_2)\|^2 \leq L_f^2(\|g_k(\mathbf{x}_1)-g_k(\mathbf{x}_2)\|^2  + \|\mathbf{y}_1- \mathbf{y}_2\|^2 )\ , \\
	%		& \|\nabla_{\mathbf{y}}  f_k(g_k(\mathbf{x}_1), \mathbf{y}_1)-\nabla_{\mathbf{y}}  f_k(g_k(\mathbf{x}_2), \mathbf{y}_2)\|^2 \leq  L_f^2(\|g_k(\mathbf{x}_1)-g_k(\mathbf{x}_2)\|^2  + \|\mathbf{y}_1- \mathbf{y}_2\|^2 ) \  , \\
	%	\end{aligned}
	%\end{equation}	
	where $L_f>0$.  For any inner-level function $g_k(\cdot)$,  it has $L_g$-Lipschitz continuous gradient, i.e.,  for $\forall \mathbf{x}_1, \mathbf{x}_2 \in \mathbb{R}^{d_1}$,  $ \|\nabla g_k(\mathbf{x}_1) - \nabla g_k(\mathbf{x}_2)\|^2 \leq L_g^2 \|\mathbf{x}_1-\mathbf{x}_2\|^2$,
	%\begin{equation}
	%	\begin{aligned}
	%		& \|\nabla g_k(\mathbf{x}_1) - \nabla g_k(\mathbf{x}_2)\|^2 \leq L_g^2 \|\mathbf{x}_1-\mathbf{x}_2\|^2  \ ,
	%	\end{aligned}
	%\end{equation}
	where $L_g>0$.
	
	
\end{assumption}


\begin{assumption} \label{assumption_bound_gradient}
	For any outer-level function $f_k(\cdot, \cdot)$ and inner-level function $g_k(\cdot)$, $\forall (\mathbf{x}, \mathbf{y}) \in \mathbb{R}^{d_1}\times\mathbb{R}^{d_2}$,   their stochastic gradients satisfy: $ \mathbb{E}[\|\nabla_{g} f_k(g_k(\mathbf{x}), \mathbf{y}; \zeta)\|^2] \leq C_f^2$, $\mathbb{E}[\|\nabla_{\mathbf{y}} f_k(g_k(\mathbf{x}), \mathbf{y}; \zeta)\|^2] \leq C_f^2$, and $\mathbb{E}[\|\nabla g_k(\mathbf{x}; \xi)\|^2]\leq C_g^2$, 
	%	\begin{equation}
	%		\begin{aligned}
	%			& \mathbb{E}[\|\nabla_{g} f_k(g_k(\mathbf{x}), \mathbf{y}; \zeta)\|^2] \leq C_f^2  \ , \quad  \mathbb{E}[\|\nabla_{\mathbf{y}} f_k(g_k(\mathbf{x}), \mathbf{y}; \zeta)\|^2] \leq C_f^2, 
	%			& \mathbb{E}[\|\nabla g_k(\mathbf{x}; \xi)\|^2]\leq C_g^2 \ , \\
	%		\end{aligned}
	%	\end{equation}
	where $C_f>0$ and $C_g>0$ are two constants.  
\end{assumption}


\begin{assumption} \label{assumption_bound_variance}
	For any outer-level function $f_k(\cdot, \cdot)$ and  $\forall (\mathbf{x}, \mathbf{y}) \in \mathbb{R}^{d_1}\times\mathbb{R}^{d_2}$, the variance of its stochastic gradient satisfies: $ \mathbb{E}[\|\nabla_{g} f_k(g_k(\mathbf{x}), \mathbf{y}; \zeta) - \nabla_{g} f_k(g_k(\mathbf{x}), \mathbf{y})\|^2] \leq \sigma_f^2$ and $\mathbb{E}[\|\nabla_{\mathbf{y}} f_k(g_k(\mathbf{x}), \mathbf{y}; \zeta) - \nabla_{\mathbf{y}} f_k(g_k(\mathbf{x}), \mathbf{y})\|^2] \leq \sigma_f^2$, 
	%	\begin{equation}
	%		\begin{aligned}
	%			& \mathbb{E}[\|\nabla_{g} f_k(g_k(\mathbf{x}), \mathbf{y}; \zeta) - \nabla_{g} f_k(g_k(\mathbf{x}), \mathbf{y})\|^2] \leq \sigma_f^2,  \\
	%			& \mathbb{E}[\|\nabla_{\mathbf{y}} f_k(g_k(\mathbf{x}), \mathbf{y}; \zeta) - \nabla_{\mathbf{y}} f_k(g_k(\mathbf{x}), \mathbf{y})\|^2] \leq \sigma_f^2, \\
	%		\end{aligned}
	%	\end{equation}
	where $\sigma_f>0$.
	For any inner-level function $g_k(\cdot)$ and $\forall \mathbf{x} \in \mathbb{R}^{d_1}$,  the variance of its stochastic gradient and function value satisfy: $\mathbb{E}[\|\nabla g_k(\mathbf{x}; \xi) - \nabla g_k(\mathbf{x})\|^2] \leq \sigma_{g'}^2 $ and $\mathbb{E}[\| g_k(\mathbf{x}; \xi) -  g_k(\mathbf{x})\|^2] \leq \sigma_g^2$, 
	%	\begin{equation}
	%		\begin{aligned}
	%			& \mathbb{E}[\|\nabla g_k(\mathbf{x}; \xi) - \nabla g_k(\mathbf{x})\|^2] \leq \sigma_{g'}^2 \ ,\quad  \mathbb{E}[\| g_k(\mathbf{x}; \xi) -  g_k(\mathbf{x})\|^2] \leq \sigma_g^2 \ , \\
	%		\end{aligned}
	%	\end{equation}
	where $\sigma_g>0$ and $\sigma'_g>0$. 
\end{assumption}

\begin{assumption} \label{assumption_strong}
	For any function $f_k(g_k(\mathbf{x}), \mathbf{y})$, it is $\mu$-strongly-concave regarding $\mathbf{y}$
	where $\mu>0$. 
\end{assumption}

\begin{assumption} \label{assumption_graph}
	The adjacency matrix $W=[w_{ij}]\in \mathbb{R}^{K\times K}$ of the graph that composed by all workers is a symmetric and doubly stochastic matrix. Its second largest absolute eigenvalue $\lambda$ satisfies $\lambda<1$.  
\end{assumption}




In terms of the aforementioned assumptions, we denote the condition number as $\kappa=L_f/\mu$ and the spectral gap as $1-\lambda$. Moreover, throughout this paper,  we use $\mathbf{a}_{k, t}$ to denote the variable $\mathbf{a}$ on the $k$-th worker at the $t$-th iteration, and  denote    $\bar{\mathbf{a}}_t=\frac{1}{K}\sum_{k=1}^{K}\mathbf{a}_{k,t}$. We further
introduce the auxiliary function $\Phi_k(\mathbf{x}) = \max_{\mathbf{y}\in \mathbb{R}^{d_2}} f_k(g_k(\mathbf{x}), \mathbf{y})$  and $\mathbf{y}_k^*(\mathbf{x}) = \arg \max_{\mathbf{y}\in \mathbb{R}^{d_2}} f_k(g_k(\mathbf{x}), \mathbf{y})$, where  $\Phi_k(\mathbf{x})$ is $L_{\Phi}$-smooth with $L_{\Phi}=2C_g^2L_f^2/\mu+ C_fL_g$ \cite{gao2021convergence}.
Additionally, we represent the loss function as $F(\mathbf{x}, \mathbf{y})=\frac{1}{K}\sum_{k=1}^{K}F_k(\mathbf{x}, \mathbf{y})= \frac{1}{K}\sum_{k=1}^{K} f_k(g_k(\mathbf{x}), \mathbf{y})$. Then, $\min_{\mathbf{x}\in\mathbb{R}^{d_1}} \max_{\mathbf{y}\in \mathbb{R}^{d_2}} F(\mathbf{x}, \mathbf{y}) = \min_{\mathbf{x}\in\mathbb{R}^{d_1}} \Phi(\mathbf{x})$.  where $\Phi(\mathbf{x}) = \frac{1}{K}\sum_{k=1}^{K}\Phi_k(\mathbf{x})$. 





\begin{algorithm}[h]
	\caption{D-SCGDAM-GP}
	\label{alg_dscgdamgp}
	\begin{algorithmic}[1]
		\REQUIRE $\mathbf{x}_0$, $\mathbf{y}_0$, $\eta\in (0, 1)$, $\gamma_x>0$, $\gamma_y>0$, $\beta_x>0$, $\beta_y>0$, $\alpha>0$,  $\alpha\eta \in (0, 1)$, $\beta_x\eta \in (0, 1)$, $\beta_y\eta \in (0, 1)$. \\
		
		
		\FOR{$t=0,\cdots, T-1$, each worker $k$} 
		\IF {$t==0$}
		\STATE 
		$\mathbf{x}_{k,0}=\mathbf{x}_0$,\   $\mathbf{y}_{k,0}=\mathbf{y}_0$, \ 
		$\mathbf{h}_{k, 0}= g_k(\mathbf{x}_{k, 0};  \xi_{k, 0})$, \\
		$\mathbf{u}_{k, 0}=\nabla g_k(\mathbf{x}_{k, 0};  \xi_{k, 0})^T\nabla_{g}  f_k(\mathbf{h}_{k,0}, \mathbf{y}_{k,0}; \zeta_{k,0})$, \ 
		$\mathbf{v}_{k,0}=\nabla_{y}  f_k(\mathbf{h}_{k,0}, \mathbf{y}_{k,0}; \zeta_{k,0})$,
		\ENDIF
		\STATE 
		$\tilde{\mathbf{x}}_{k, t+1} = \sum_{j: w_{kj}> 0}w_{kj}\mathbf{x}_{j,t} - \gamma_x \mathbf{u}_{k, t}$,  \ 
		$\mathbf{x}_{k, t+1}= \mathbf{x}_{ k,t} + \eta(\tilde{\mathbf{x}}_{k, t+1}-\mathbf{x}_{ k,t} )$,
		\STATE 
		%				${y}_{k, t+\frac{1}{2}} = \sum_{j\in \mathcal{N}_{k}}w_{kj}y_{j,t}+ \gamma_y v_{k, t}$, \\
		%				$\tilde{y}_{k, t+1} = \mathcal{P}_{\mathcal{Y}}({y}_{k, t+\frac{1}{2}})$, \\
		%				 ${y}_{k, t+1}= {y}_{ k,t} + \eta(\tilde{y}_{k, t+1}-y_{ k,t} )$ 
		$\tilde{\mathbf{y}}_{k, t+1}  = \sum_{j: w_{kj}> 0}w_{kj}\mathbf{y}_{j,t}+ \gamma_y \mathbf{v}_{k, t}$,  \ 
		%				$\tilde{y}_{k, t+1} = \mathcal{P}_{\mathcal{Y}}({y}_{k, t+\frac{1}{2}})$, \\
		$\mathbf{y}_{k, t+1}= \mathbf{y}_{ k,t} + \eta(\tilde{\mathbf{y}}_{k, t+1} -\mathbf{y}_{ k,t} )$,
		
		%		\STATE Update momentum: \\
		%		$\mathbf{h}_{k, t} = (1- \alpha\eta) \mathbf{h}_{k, t-1} +  \alpha\eta g_k(\mathbf{x}_{k,t};  \xi_{k, t})$, \\
		%		$\mathbf{u}_{k, t} = (1-\beta_x\eta)\mathbf{u}_{k, t-1} + \beta_x\eta \nabla g_k(\mathbf{x}_{k, t};  \xi_{k, t})^T\nabla_{g}  f_k(\mathbf{h}_{k,t}, \mathbf{y}_{k,t}; \zeta_{k,t})$,\\
		%		$\mathbf{v}_{k, t} = (1-\beta_y\eta)\mathbf{v}_{k, t-1} + \beta_y\eta\nabla_{y}  f_k(\mathbf{h}_{k,t}, \mathbf{y}_{k,t}; \zeta_{k,t})$,
		%		
		
		
		\STATE 
		$\mathbf{h}_{k, t+1} = (1- \alpha\eta) \mathbf{h}_{k, t} +  \alpha\eta g_k(\mathbf{x}_{k,t+1};  \xi_{k, t+1})$, \\
		$\mathbf{u}_{k, t+1} = (1-\beta_x\eta)\mathbf{u}_{k, t} + \beta_x\eta \nabla g_k(\mathbf{x}_{k, t+1};  \xi_{k, t+1})^T\nabla_{g}  f_k(\mathbf{h}_{k,t+1}, \mathbf{y}_{k,t+1}; \zeta_{k,t+1})$,\\
		$\mathbf{v}_{k, t+1} = (1-\beta_y\eta)\mathbf{v}_{k, t} + \beta_y\eta\nabla_{y}  f_k(\mathbf{h}_{k,t+1}, \mathbf{y}_{k,t+1}; \zeta_{k,t+1})$,
		\ENDFOR
	\end{algorithmic}
\end{algorithm}

%\vspace{-5pt}
\section{Gossip-based Algorithm}
%\paragraph{D-SCGDAM-GP.}  

In this section, we developed a decentralized algorithm based on the gossip communication strategy, which employs the  stochastic compositional gradient descent with momentum algorithm to address \textit{the challenge regarding the update on individual workers caused by the compositional structure}. 

\subsection{Algorithmic Design}
In Algorithm~\ref{alg_dscgdamgp}, we developed the gossip-based decentralized  stochastic compositional gradient descent ascent with momentum (D-SCGDAM-GP) algorithm.  Generally speaking, there are $K$ workers, where each worker conducts local updates and then communicates the local model parameter with its neighboring workers. In detail, the model parameters $\mathbf{x}_{k, 0}$ and $\mathbf{y}_{k, 0}$ on all workers are intilized with the same value $\mathbf{x}_0$ and  $\mathbf{y}_{0}$, respectively.  Then, for the minimization subproblem, at the $t$-th iteration, the $k$-th worker computes the stochastic compositional gradient based on the estimation for the inner-level function $g_k(\mathbf{x})$ as below:
\begin{equation}
	\begin{aligned}
		& \mathbf{h}_{k, t} = (1- \alpha\eta) \mathbf{h}_{k, t-1} +  \alpha\eta g_k(\mathbf{x}_{k,t};  \xi_{k, t}) \ ,
	\end{aligned}
\end{equation}
where $\alpha$ and $\eta$ are two positive hyperparameters, $\xi_{k,t}$ denotes the selected samples on the $k$-th worker at the $t$-th iteration. Here, $\mathbf{h}_{k, t}$ can be viewed as the moving average estimation for the inner-level function $g_k(\mathbf{x})$ when $\alpha\eta<1$. This strategy is commonly used in stochastic compositional gradient descent method \cite{wang2017stochastic}. Based on $\mathbf{h}_{k, t}$, our method computes the momentum as below:
\begin{equation}
	\begin{aligned}
		& \mathbf{u}_{k, t} = (1-\beta_x\eta)\mathbf{u}_{k, t-1}+ \beta_x\eta \nabla g_k(\mathbf{x}_{k, t};  \xi_{k, t})^T\nabla_{g}  f_k(\mathbf{h}_{k,t}, \mathbf{y}_{k,t}; \zeta_{k,t}) \ ,
	\end{aligned}
\end{equation}
where $\beta_x$ and $\eta$ are two positive hyperparameters and $\beta_x\eta<1$,  $\mathbf{u}_{k, t}$ denotes the momentum for the stochastic compositional gradient $\nabla g_k(\mathbf{x}_{k, t};  \xi_{k, t})^T\nabla_{g}  f_k(\mathbf{h}_{k,t}, \mathbf{y}_{k,t}; \zeta_{k,t})$, and $\zeta_{k,t}$ denotes the selected samples on the $k$-th worker at the $t$-th iteration for the outer-level function. After obtaining the momentum $\mathbf{u}_{k, t}$, the $k$-th worker uses it to update the model parameter $\mathbf{x}_{k, t}$, which is shown below:
\begin{equation} \label{update_x}
	\begin{aligned}
		& \tilde{\mathbf{x}}_{k, t+1} = \sum_{j: w_{kj}> 0}w_{kj}\mathbf{x}_{j,t} - \gamma_x \mathbf{u}_{k, t} \ ,   \mathbf{x}_{k, t+1}= \mathbf{x}_{ k,t} + \eta(\tilde{\mathbf{x}}_{k, t+1}-\mathbf{x}_{ k,t} ) \ ,
	\end{aligned}
	%\vspace{-5pt}
\end{equation}
where $\gamma_x>0$, $\eta \in (0, 1)$, and $w_{kj}$ denotes the edge weight between the $k$-th worker and the $j$-th worker. Here, $\sum_{j: w_{kj}> 0}w_{kj}\mathbf{x}_{j,t}$ denotes the communication with the neighoring workers  to get their model parameters. Then,  the $k$-th worker  updates the communicated model parameter with the momentum $\mathbf{u}_{k, t}$. The second step in Eq.~(\ref{update_x}) is the linear combination between $\mathbf{x}_{k,t}$ and the intermediate model parameter $\tilde{\mathbf{x}}_{k, t+1}$. 

Similar to the update of $\mathbf{x}_{k,t}$, each worker uses a similar strategy to update the model parameter  $\mathbf{y}_{k,t}$ in the maximization subproblem. It is worth noting that the maximization subproblem is not a compositional problem. Thus, we use the standard  stochastic gradient ascent with momentum algorithm to update $\mathbf{y}_{k,t}$, which can be found in Steps 4-5 of Algorithm~\ref{alg_dscgdamgp}.  





\subsection{Theoretical Analysis}
We provide the convergence rate of our D-SCGDAM-GP algorithm below and the proof can be found in Appendix. 
\begin{theorem} \label{theorem_1}
	Under Assumptions~\ref{assumption_smooth}-\ref{assumption_graph}, by setting  $\beta_x>0$, $\beta_y>0$, $\alpha>0$,  $\eta< \min\{\frac{1}{\alpha}, \frac{1}{\beta_x}, \frac{1}{\beta_y}, \frac{1}{2\gamma_x L_{\Phi}}, 1\}$,  $\gamma_x\leq \min\{\frac{\gamma_y\mu^2} {20 C_g^2L_f^2}, \frac{\mu}{8L_f\sqrt{\gamma_{x, 1}}}, \frac{1-\lambda}{4\sqrt{\gamma_{x, 2}}} \}$, $\gamma_y\leq \min\{\frac{1}{6L_f}, \frac{1-\lambda}{3L_f\sqrt{\hat{C}_5 + 1 + 32/\beta_x^2 + 400/\beta_y^2}}, \frac{ 9\mu}{8L_f^2(8/\beta_x^2+100/\beta_y^2)} \}$, where $\gamma_{x, 1}=(8/\beta_x^2 )C_f^2L_g^2+  (104/\alpha + 315/\alpha^2+8/\beta_x^2+ 100/\beta_y^2) C_g^4L_f^2$, $\gamma_{x, 2}=\hat{C}_4+ C_f^2L_g^2 +    1264 C_g^4L_f^2 + (32/\beta_x^2 )C_f^2L_g^2+  4(104/\alpha + 315/\alpha^2+8/\beta_x^2+ 100/\beta_y^2) C_g^4L_f^2$, $\hat{C}_4=(2566 + 64/\beta_x^2 + 800/\beta_y^2+832/\alpha + 2520/\alpha^2) L_f^2C_g^4+ (5+64/\beta_x^2)  C_f^2L_g^2$, $\hat{C}_5= 55+ 64/\beta_x^2 + 800/\beta_y^2 $, 
	D-SCGDAM-GP in Algorithm~\ref{alg_dscgdamgp} has the  convergence rate:
	\vspace{-5pt}
	\begin{equation} \label{eq_convergence_1}
		\small
		\begin{aligned}
			&    \frac{1}{T}\sum_{t=0}^{T-1}(\mathbf{E}[ \|\nabla \Phi(\bar{\mathbf{x}}_{t})\|^2] + C_g^2L_f^2 \mathbf{E}[\|\mathbf{y}^*(\bar{\mathbf{x}}_{t}) -\bar{\mathbf{y}}_{t}\|^2])  \leq  \frac{2( \Phi({\mathbf{x}}_{0})- \Phi({\mathbf{x}}_{*}))}{\eta\gamma_xT} \\
			& +\frac{12 C_g^2L_f^2}{\eta T\gamma_y\mu}\mathbf{E}[\|\bar{\mathbf{y}}_{0}   - \mathbf{y}^{*}({\mathbf{x}}_0)\| ^2]   + O\Big(\frac{  L_f^2}{\eta T\beta_x\mu^2  }\Big) +  O\Big( \frac{L_f^2}{\eta T\beta_y\mu^2 }\Big)
			+ O\Big(\frac{L_f^2  }{\eta T\alpha\mu^2} \Big) +  O\Big(\frac{\eta\beta_x }{K}  \Big)
			\\
			&    + O\Big( \frac{\eta \beta_y L_f^2}{\mu^2K }\Big)   +   O\Big(\frac{\alpha\eta L_f^2 }{\mu^2K}\Big )  + O\Big(\frac{ \beta_x\eta  L_f^2}{\mu^2  } \Big) + O\Big(\frac{ \beta_y\eta L_f^2}{  \mu^2}\Big)  +  O\Big(\frac{   \alpha\eta L_f^2 }{ \mu^2}  \Big) +   O\Big( \frac{ \alpha^2\eta^2L_f^2}{\mu^2  }\Big) \  . \\
		\end{aligned}
	\end{equation}
	
	
	
	
\end{theorem}

\begin{remark}
	From Theorem~\ref{theorem_1}, it can be seen that  the dependence of $\gamma_x$ and $\gamma_y$ over the spectral gap and condition number is  in the order of  $O(\frac{1-\lambda}{\kappa^3})$ and $O(\frac{1-\lambda}{\kappa})$, respectively. Additionally,  it is easy to know $\beta_x=O(1)$, $\beta_y=O(1)$, and $\alpha=O(1)$. 
\end{remark}


\begin{corollary} \label{corollary_1}
	Under Assumptions~\ref{assumption_smooth}-\ref{assumption_graph},  by setting $\beta_x=O(1)$, $\beta_y=O(1)$,  $\alpha=O(1)$,  $\gamma_x=O(\frac{1-\lambda}{\kappa^3})$,  $\gamma_y = O(\frac{1-\lambda}{\kappa})$,   $\eta=O(\frac{\epsilon^2}{\kappa^2})$,  and $T=\frac{\kappa^5}{(1-\lambda)\epsilon^4}$,  D-SCGDAM-GP in Algorithm~\ref{alg_dscgdamgp} can achieve the $\epsilon$-accuracy solution: $\frac{1}{T}\sum_{t=0}^{T-1}(\mathbf{E}[\|\nabla \Phi(\bar{\mathbf{x}}_{t})\|^2] + C_g^2L_f^2\mathbf{E}[\|\mathbf{y}^*(\bar{\mathbf{x}}_{t}) -\bar{\mathbf{y}}_{t}\|^2] ) \leq \epsilon^2$.
	%	\begin{equation}
	%		\begin{aligned}
	%			& \frac{1}{T}\sum_{t=0}^{T-1}(\mathbf{E}[\|\nabla \Phi(\bar{\mathbf{x}}_{t})\|^2] + C_g^2L_f^2\mathbf{E}[\|\mathbf{y}^*(\bar{\mathbf{x}}_{t}) -\bar{\mathbf{y}}_{t}\|^2] ) \leq \epsilon^2 \ .
	%		\end{aligned}
	%	\end{equation}
\end{corollary}



\begin{remark}
	The dependence of $\eta$ on the condition number is caused by the term with the factor $L_f^2/\mu^2$, e.g., $O({ \beta_x\eta  L_f^2}/{\mu^2  } )$. 
\end{remark}

\begin{remark}
	To achieve the $\epsilon$-accuracy solution, the iteration  (communication) complexity  of D-SCGDAM-GP  is $O(\frac{\kappa^5}{(1-\lambda)\epsilon^4})$, and the   sample complexity on each worker is $O(\frac{\kappa^5}{(1-\lambda)\epsilon^4})$, which indicates D-SCGDAM-GP cannot achieve the linear speedup regarding the number of workers. 
	

\end{remark}





\subsection{Discussion} \label{sec_communication_challenge}
From the convergence upper bound in Eq.~(\ref{eq_convergence_1}), it is easy to find that there are three terms that are in the order of  $O(\eta)$ and are not scaled by the number of workers $K$. As a result, the learning rate $\eta$ cannot be set to $O(\frac{K\epsilon^2}{\kappa^2})$ to achieve the linear speedup. In fact, those three terms are introduced by the consensus error about the momentum $\mathbf{E}[\| \mathbf{u}_{k,t} -  \bar{\mathbf{u}}_{t}\|^2]$, $\mathbf{E}[\| \mathbf{v}_{k,t} -  \bar{\mathbf{v}}_{t}\|^2]$, and the inner-level function $\mathbf{E}[\| \mathbf{h}_{k,t} -  \bar{\mathbf{h}}_{t}\|^2]$ (See Lemmas~5, 6, 7). Taking the inner-level function as an example, via the recursive expansion in terms of Lemma~5, we have $\mathbf{E}[\| \mathbf{h}_{k,t} -  \bar{\mathbf{h}}_{t}\|^2]\propto O(\eta)$. Such a large consensus error makes D-SCGDAM-GP fail to achieve the linear speedup.  
In other words, D-SCGDAM-GP is not sufficient to address \textit{the  challenge about the communication across workers caused by the compositional structure in the loss function}. 
Therefore, to achieve this issue, \textit{a feasible way is to reduce the consensus error for those three terms, e.g., having a higher-order dependence on the learning rate. }




%\subsection{Gradient-tracking-based Decentralized Momentum Stochastic Compositional Gradient Descent Ascent Method }

\begin{algorithm}[h]
	\caption{D-SCGDAM-GT}
	\label{alg_dscgdamgt}
	\begin{algorithmic}[1]
		\REQUIRE $\mathbf{x}_0$, $\mathbf{y}_0$, $\eta\in (0, 1)$, $\gamma_x>0$, $\gamma_y>0$, $\beta_x>0$, $\beta_y>0$, $\alpha>0$,  $\alpha\eta \in (0, 1)$, $\beta_x\eta \in (0, 1)$, $\beta_y\eta \in (0, 1)$. 
		$\mathbf{p}_{k,-1}=\mathbf{0}$ ,  $\mathbf{q}_{k,-1}=\mathbf{0}$ ,$\mathbf{u}_{k,-1}=\mathbf{0}$ ,  $\mathbf{v}_{k,-1}=\mathbf{0}$
		
		\FOR{$t=0,\cdots, T-1$, each worker $k$} 
		
		\IF {$t==0$}
		\STATE  
		$\mathbf{x}_{k,0}=\mathbf{x}_0$ ,  $\mathbf{y}_{k,0}=\mathbf{y}_0$ ,
		$\mathbf{r}_{k, 0}=\mathbf{h}_{k, 0}= g_k(\mathbf{x}_{k, 0};  \xi_{k, 0})$ , \\
		$\mathbf{u}_{k, 0}=\nabla g_k(\mathbf{x}_{k, 0};  \xi_{k, 0})^T\nabla_{g}  f_k(\mathbf{h}_{k,0}, \mathbf{y}_{k,0}; \zeta_{k,0})$, 
		$\mathbf{v}_{k,0}=\nabla_{y}  f_k(\mathbf{h}_{k,0}, \mathbf{y}_{k,0}; \zeta_{k,0})$,
		\ENDIF
		%		$\tikz[baseline]{\node[fill=blue!20,anchor=base] (t1) {$\displaystyle \mathbf{p}_{k, t} = \sum_{j: w_{ij}> 0}w_{kj}\mathbf{p}_{j,t-1} + \mathbf{u}_{k, t}- \mathbf{u}_{k, t-1} $}}$ , \\
		\STATE 
		$\mathbf{p}_{k, t} = \sum_{j: w_{kj}> 0}w_{kj}\mathbf{p}_{j,t-1} + \mathbf{u}_{k, t}- \mathbf{u}_{k, t-1}$ , \\
		$\tilde{\mathbf{x}}_{k, t+1} = \sum_{j: w_{kj}> 0}w_{kj}\mathbf{x}_{j,t} - \gamma_x \mathbf{p}_{k, t}$ , \ 
		$\mathbf{x}_{k, t+1}= \mathbf{x}_{ k,t} + \eta(\tilde{\mathbf{x}}_{k, t+1}-\mathbf{x}_{ k,t} )$ , 
		\STATE 
		%		$\tikz[baseline]{\node[fill=blue!20,anchor=base] (t1) {$\displaystyle \mathbf{q}_{k, t} = \sum_{j: w_{ij}> 0}w_{kj}\mathbf{q}_{j,t-1} + \mathbf{v}_{k, t}- \mathbf{v}_{k, t-1}$}}$ , \\
		$\mathbf{q}_{k, t} = \sum_{j: w_{kj}> 0}w_{kj}\mathbf{q}_{j,t-1} + \mathbf{v}_{k, t}- \mathbf{v}_{k, t-1}$ ,\\
		$\tilde{\mathbf{y}}_{k, t+1}  = \sum_{j: w_{kj}> 0}w_{kj}\mathbf{y}_{j,t}+ \gamma_y \mathbf{q}_{k, t}$ , \ 
		%				$\tilde{y}_{k, t+1} = \mathcal{P}_{\mathcal{Y}}({y}_{k, t+\frac{1}{2}})$, \\
		$\mathbf{y}_{k, t+1}= \mathbf{y}_{ k,t} + \eta(\tilde{\mathbf{y}}_{k, t+1} -\mathbf{y}_{ k,t} ) $  ,
		\STATE 
		$\mathbf{h}_{k, t+1} = (1- \alpha\eta) \mathbf{h}_{k, t} +  \alpha\eta g_k(\mathbf{x}_{k,t+1};  \xi_{k, t+1})$ , \\
		\colorbox{pink}{$\mathbf{r}_{k, t+1} = \sum_{j: w_{ij}> 0}w_{kj}\mathbf{r}_{j,t} + \mathbf{h}_{k, t+1}- \mathbf{h}_{k, t}$}, \\
		%				$\mathbf{r}_{k, t+1} = \sum_{j: w_{ij}> 0}w_{kj}\mathbf{r}_{j,t} + \mathbf{h}_{k, t+1}- \mathbf{h}_{k, t}$ , \\
		$\mathbf{u}_{k, t+1} = (1-\beta_x\eta)\mathbf{u}_{k, t} + \beta_x\eta \nabla g_k(\mathbf{x}_{k, t+1};  \xi_{k, t+1})^T\nabla_{g}  f_k(\colorbox{pink}{$\mathbf{r}_{k,t+1}$}, \mathbf{y}_{k,t+1}; \zeta_{k,t+1})$,\\
		$\mathbf{v}_{k, t+1} = (1-\beta_y\eta)\mathbf{v}_{k, t} + \beta_y\eta\nabla_{y}  f_k(\colorbox{pink}{$\mathbf{r}_{k,t+1}$}, \mathbf{y}_{k,t+1}; \zeta_{k,t+1})$,
		
		\ENDFOR
	\end{algorithmic}
\end{algorithm}

\vspace{-10pt}
\section{Gradient-Tracking-based  Algorithm}
\vspace{-10pt}
In this section, we develop a novel decentralized algorithms for Eq.~(\ref{loss}) to address the communication challenge as discussed in Section~\ref{sec_communication_challenge}.

%which enjoys a small consensus error so that it can achieve linear speedup.
%\paragraph{D-SCGDAM-GT.} 

\subsection{Algorithmic Design: Reducing Consensus Errors}
As discussed in Section~\ref{sec_communication_challenge}, the large consensus error regarding the momentum and inner-level function makes Algorithm~\ref{alg_dscgdamgp} fail to achieve  linear speedup. To address this issue, 
in Algorithm~\ref{alg_dscgdamgt}, we develop a new algorithm, i.e.,  the gradient-tracking-based decentralized  stochastic compositional gradient descent ascent with momentum (D-SCGDAM-GT) algorithm. Generally speaking, D-SCGDAM-GT employs the gradient-tracking communication strategy to communicate the momentum and inner-level function to reduce the consensus error. It is worth noting that this is \textit{the first work communicating the inner-level function} to address the  challenge caused by the compositional structure. We believe this novel algorithmic design can be applied to other decentralized compositional optimization problems, such as  the compositional minimization problem. 

In detail, Algorithm~\ref{alg_dscgdamgt} uses the same way as Algorithm~\ref{alg_dscgdamgp} to compute the momentum $\mathbf{u}_{k, t}$ for the stochastic compositional gradient $\nabla g_k(\mathbf{x}_{k, t};  \xi_{k, t})^T\nabla_{g}  f_k(\mathbf{r}_{k,t}, \mathbf{y}_{k,t}; \zeta_{k,t})$ and $\mathbf{v}_{k, t}$ for the stochastic gradient $\nabla_{y}  f_k(\mathbf{r}_{k,t}, \mathbf{y}_{k,t}; \zeta_{k,t})$, which is demonstrated in Step 5 of Algorithm~\ref{alg_dscgdamgt}.  However, the stochastic gradient $\nabla_{g}  f_k(\mathbf{r}_{k,t}, \mathbf{y}_{k,t}; \zeta_{k,t})$ and $\nabla_{y}  f_k(\mathbf{r}_{k,t}, \mathbf{y}_{k,t}; \zeta_{k,t})$ are computed based on the tracked estimator $\mathbf{r}_{k,t}$ regarding the inner-level function, rather than the original inner-level function estimator $\mathbf{h}_{k,t}$.  Specifically, $\mathbf{r}_{k,t}$ is updated with the gradient-tracking communication strategy:
\begin{equation}
	\mathbf{r}_{k, t+1} = \sum_{j: w_{kj}> 0}w_{kj}\mathbf{r}_{j,t} + \mathbf{h}_{k, t+1}- \mathbf{h}_{k, t} \ . 
\end{equation} 
Due to this additional communication step, the consensus error regarding $\mathbf{E}[\| \mathbf{r}_{k,t} -  \bar{\mathbf{r}}_{t}\|^2]$ is supposed to be smaller than the consensus error $\mathbf{E}[\| \mathbf{h}_{k,t} -  \bar{\mathbf{h}}_{t}\|^2]$ in Algorithm~\ref{alg_dscgdamgp}, which can facilitate the linear speedup \footnote{Here, we use $\mathbf{r}_{k,t}$ to compute stochastic gradients rather than $\mathbf{h}_{k,t}$ so that we should consider the consensus error about $\mathbf{r}_{k,t}$ in Algorithm~\ref{alg_dscgdamgt}, rather than $\mathbf{h}_{k,t}$. }. In fact, this is confirmed by Lemma~19,
%\ref{lemma_h_consensus_gt}, 
where we can have $\mathbf{E}[\| \mathbf{r}_{k,t} - \bar{\mathbf{r}}_{t}\|^2] \propto O(\eta^2)$ via the recursive expansion. When $\eta=O(\epsilon^2)$, this consensus error is much smaller than   $\mathbf{E}[\| \mathbf{h}_{k,t} -  \bar{\mathbf{h}}_{t}\|^2]\propto O(\eta)$ obtained from Lemma~5.
%\ref{lemma_h_consensus_gp}. 



After obtaining the momentum, each worker $k$ applies the gradient tracking communication strategy to the momentum $\mathbf{u}_{k, t}$ and $\mathbf{v}_{k, t}$ to reduce the consensus error regarding the momentum, which is shown in Step 5 of Algorithm~\ref{alg_dscgdamgt}. Then, the tracked momentum $\mathbf{p}_{k, t}$ and $\mathbf{q}_{k, t}$ are used to  update the model parameter $\mathbf{x}_{k, t}$ and $\mathbf{y}_{k, t}$, respectively. 








\vspace{-5pt}
\subsection{Theoretical Analysis}
\vspace{-5pt}





%In what follows, we provide the convergence rate of   Algorithm~\ref{alg_dscgdamgt}. Its proof can be found in Appendix. 
\begin{theorem} \label{theorem_2}
	Under Assumptions~\ref{assumption_smooth}-\ref{assumption_graph}, by setting  $\beta_x>0$, $\beta_y>0$, $\alpha>0$,  $\eta< \min\{\frac{1}{\alpha}, \frac{1}{\beta_x}, \frac{1}{\beta_y}, \frac{1}{2\gamma_x L_{\Phi}}, 1\}$, $\gamma_x\leq \min\{ \frac{\mu^2\gamma_y}{20 C_g^2L_f^2} , \frac{\mu (1-\lambda)}{8L_f\sqrt{\gamma_{x, 1}}}\}$, $\gamma_y \leq \min\{\frac{1}{6L_f},  \frac{9\mu}{8L_f^2(8/\beta_x^2 + 100/\beta_y^2)}  \}$, where $\gamma_{x, 1}=8(C_g^4L_f^2+ C_f^2L_g^2 )/\beta_x^2 +  100C_g^4L_f^2/\beta_y^2  +  312C_g^4L_f^2/\alpha + 4056C_g^4L_f^2$, 
	Algorithm~\ref{alg_dscgdamgt} has the  convergence rate:
	\begin{equation}
		\small
		\begin{aligned}
			&  \frac{1}{T}\sum_{t=0}^{T-1}(\mathbf{E}[ \|\nabla \Phi(\bar{\mathbf{x}}_{t})\|^2] +  C_g^2L_f^2\mathbf{E}[\|\mathbf{y}^*(\bar{\mathbf{x}}_{t}) -\bar{\mathbf{y}}_{t}\|^2] ) \leq \frac{2(\Phi({\mathbf{x}}_{0})- \Phi({\mathbf{x}}_{*}))}{\eta\gamma_x T} \\
			&  + \frac{12C_g^2L_f^2}{\eta T\gamma_y\mu}\mathbf{E}[\|\bar{\mathbf{y}}_{0}   - \mathbf{y}^{*}(\bar{\mathbf{x}}_0)\| ^2]     + O\Big( \frac{1}{\eta T\beta_x} \Big)  +  O\Big(\frac{L_f^2}{\eta T\beta_y\mu^2} \Big) +  O\Big( \frac{L_f^2}{\eta T\mu^2} \Big)+  O\Big(\frac{\beta_x\eta}{K} \Big) \\
			&  +  O\Big(\frac{ \alpha\eta L_f^2}{ T\mu^2(1-\lambda)^2}\Big)   +O\Big(\frac{ L_f^2}{T(1-\lambda) \mu^2} \Big) +   O\Big(\frac{\beta_y\eta L_f^2}{K\mu^2}\Big) +O\Big(\frac{ \gamma_x^2   \eta^2 L_f^2 }{(1-\lambda)^4\mu^2}\Big) + O\Big(\frac{ \gamma_x^2   \beta_x^2\eta^2  L_f^2}{\beta_y^2\mu^2(1-\lambda)^4}   \Big)  \\
			&    +  O\Big( \frac{\eta\alpha^2 L_f^2}{K\mu^2} \Big) + O\Big(\frac{ \gamma_x^2   \beta_x^2\eta^4L_f^2}{\mu^2(1-\lambda)^6} \Big)+ O\Big(\frac{\gamma_y^2\beta_y^2\eta^2 L_f^2}{\mu^2(1-\lambda)^4} \Big)+   O\Big(\frac{\gamma_y^2\beta_y^2\eta^2 L_f^2}{\beta_x^2\mu^2(1-\lambda)^4}\Big)  + O\Big(\frac{\gamma_y^2\eta^2 L_f^2}{\mu^2(1-\lambda)^4}\Big) \\
			&   +  O\Big(\frac{\alpha^2\eta^2 L_f^2}{(1-\lambda)^2 \mu^2}\Big) +   O\Big(\frac{ \alpha^3\eta^3  L_f^2}{\mu^2(1-\lambda)^2}\Big)   +  O\Big(\frac{ \gamma_x^2   \beta_x^2\eta^2L_f^2}{(1-\lambda)^5\mu^2}\Big ) + O\Big( \frac{ \gamma_x^2   \beta_x^2\eta^2 L_f^2 }{\alpha\mu^2(1-\lambda)^4}  \Big)\ .  \\
		\end{aligned}
	\end{equation}
	
	
	
\end{theorem}

\begin{remark}
	From Theorem~\ref{theorem_2}, it can be seen that  the dependence of  $\gamma_x=O(\frac{1-\lambda}{\kappa^3})$,  $\gamma_y=O(\frac{1}{\kappa})$, $\beta_x=O(1)$, $\beta_y=O(1)$,  and $\alpha=O(1)$. Compared with Theorem~\ref{theorem_1}, $\gamma_y$ does not depend on the spectral gap $1-\lambda$. 
\end{remark}

\begin{corollary} \label{corollary_2}
	Under Assumptions~\ref{assumption_smooth}-\ref{assumption_graph},  by setting $\beta_x=O(1)$, $\beta_y=O(1)$,   $\alpha=O(1)$,  $\gamma_x=O(\frac{1-\lambda}{\kappa^3})$,  $\gamma_y=O(\frac{1}{\kappa})$, $\eta=O(\frac{K\epsilon^2}{\kappa^2})$, $T=O(\frac{\kappa^5}{(1-\lambda)\epsilon^4K})$,  D-SCGDAM-GT in Algorithm~\ref{alg_dscgdamgt} can achieve the $\epsilon$-accuracy solution.

\end{corollary}



\begin{remark}
	To achieve the $\epsilon$-accuracy solution, the iteration (communication) complexity is $O(\frac{\kappa^5}{(1-\lambda)\epsilon^4K})$, which  indicates the linear speedup regarding the number of workers $K$ and is better than Algorithm~\ref{alg_dscgdamgp}. Additionally,  the sample complexity on each worker is $O(\frac{\kappa^5}{(1-\lambda)\epsilon^4K})$, which  also indicates  the linear speedup.
\end{remark}

%To the best of our knowledge, this is the first algorithm achieving the linear speedup for decentralized compositional minimax optimization problem. We believe our novel algorithmic design and theoretical analysis strategies can be applied to other decentralized compositional optimization problems. Especially, \textit{the communication regarding the inner-level function causes new challenges for establishing the convergence rate.}   To address this challenge, we develop a novel Lyapunov function (See Eq.~(\ref{Lyapunov2})) and study how the Lyapunov function  evolves across iterations. As such, we are able to establish the convergence rate. 






%\newpage
%\vspace{-5pt}
\section{Experiment}
%\vspace{-5pt}
%In this section, we apply our proposed two algorithms to the AUROC maximization problem to demonstrate their empirical performance.

\subsection{Experimental Setup}
%\paragraph{AUROC maximization.} 




%\paragraph{Datasets.} 
\textbf{Datasets.} In our experiments, we will use our methods to optimize Eq.~(\ref{compositional_auc}).   We use three benchmark datasets: catvsdog \footnote{\url{https://www.kaggle.com/c/dogs-vs-cats}}, stl10 \cite{coates2011analysis}, fashionmnist \cite{xiao2017/online}. Specifically, we  split the dataset into two groups where the first half of classes compose the first group and the other half of classes compose the second group. Then, for one of these two groups, we randomly eliminate some examples such that these two groups form an imbalanced binary classification dataset. In our experiments, the ratio between the number of positive samples and all samples is set to $0.1$. In addition, the ratio between the training and testing set is $9:1$.  
Then, the dataset is randomly distributed to all engaged workers. Each worker uses the obtained dataset to compute stochastic (compositional) gradients to update local model parameters. 

% Figure environment removed

%% Figure environment removed

%\paragraph{Settings.} 
\textbf{Settings.} The classifier for imaging datasets is ResNet20 where the last layer is revised to perform binary classification, while the classifier for tabular datasets is an MLP, which has one hidden layer with 16 neurons.  Since  our method is the first parallel method for the compositional minimax problem, there are no counterpart methods. Thus, in our experiment, we compare our methods with the single-machine method: SCGDA \cite{gao2021convergence} and PDSCA \cite{yuan2021compositional}, as well as a non-compositional decentralized stochastic gradient descent ascent (DSGDA) method \cite{tsaknakis2020decentralized}.  Note that, to make a fair comparison with SCGDA and PDSCA, we parallelize these two baseline methods  based on a fully connected communication graph, i.e., $1-\lambda=1$. The reason is that batch normalization used in  ResNet20 for imaging datasets is sensitive to the batch size. Thus, to make a fair comparison, we should enforce all methods are trained with the same batch size on each worker.  
The hyperparameters of our methods are set as: $\gamma_x=\gamma_y=0.99$, $\beta_x=\beta_y=9.9$, $\alpha=9.0$, $\eta=0.1$, $\rho=0.1$. As for baseline methods, we use the same learning rate with our methods. Finally, all methods are implemented with PyTorch and MPI. 






%\vspace{-5pt}
\subsection{Result and Analysis}
%\vspace{-5pt}
In Figure~\ref{testauc}, we plot the test AUROC score of our methods and baseline methods. In this experiment, we use four workers and the communication graph is a ring graph. The batch size on each worker is 32 for catvsdog, fashionmnist, 8 for stl10,  512 for credit fraud, 128 for w8a and ijcnn1. Here, the test AUROC score is computed upon the averaged model $\bar{\mathbf{x}}$. 
From Figure~\ref{testauc}, it can be seen that our two algorithms can converge to almost the same test AUROC score with the single-machine momentum-based algorithm PDSCA, which confirms the correctness of our algorithms. Additionally, our algorithms can achieve a  better AUROC score than DSGDA,  confirming the effectiveness of  compositional gradient. 
%In addition, our methods and PDSCA converges a little faster than SCGDA because the latter one does not use momentum. 

\begin{wrapfigure}[12]{R}{2.4in}
	\vspace{-15pt}
	\centering
	% Figure removed
	\vspace{-0.05in}
	\caption{The test AUROC score  for catvsdog when using the mini-batch size 16.}
	\label{fig_batchsize}
\end{wrapfigure} 

To demonstrate the capability of D-SCGDAM-GT in controlling the consensus error, we use a small mini-batch size, i.e., 16, so that the stochastic gradient and stochastic inner-level function value on each worker are more noisy than the case with a large batch size, i.e., 32. The other settings do not change.  In Figure~\ref{fig_batchsize}, we plot the test AUROC score for CATvsDOG dataset, where D-SCGDAM-GT-M means we apply gradient tracking to the momentum but not the inner-level function. It can be seen that D-SCGDAM-GT-M performs slightly better than D-SCGDAM-GP, but much worse than D-SCGDAM-GT. This confirms the necessity of communicating the inner-level function for improving the convergence performance. 





%\vspace{-10pt}
\section{Conclusion}
%\vspace{-5pt}
In this paper, we have developed two novel decentralized stochastic compositional gradient descent ascent methods, demonstrating how to achieve linear speedup for stochastic compositional minimax optimization problems. The superior experimental performance confirms the correctness and effectiveness of our methods.  All these theoretical and empirical results corroborate the novelty and contribution of our work. 
%%%%%%%%%%%%%%%%%%%%%%%%%%%%%%%%%%%%%%%%%%%%%%%%%%%%%%%%%%%%

%\newpage
%%-----------------------------
\bibliographystyle{abbrv}
\bibliography{egbib}
%%-----------------------------

%\newpage
%\appendix
%%\input{supplement_exp}
%%\newpage
%\newpage
\appendix

\section{Proof of Lemma \ref{lemma, equivalence of two def of MDDO}}
\begin{proof}
For any ${\bs{\beta}}\in\mc H$, according to the definition of $G_{\bs s}$ (see Definition $\ref{def: MDDO}$), one has
\begin{align*}
\langle G_{\bs s},{\bs{\beta}}\rangle&=\int_{[0,1]} G_{\bs s}(t){\bs{\beta}}(t)~\mathrm{d}t=\int_{[0,1]}\mathrm{cov}\hspace{-0.9mm}\left(\bs{X}(t),\mathrm{e}^{\mi\langle \bs s,\Y\rangle}\right){\bs{\beta}}(t)~\mathrm{d}t\\
&=\int_{[0,1]}\mathrm{cov}\hspace{-0.9mm}\left(\bs{X}(t){\bs{\beta}}(t),\mathrm{e}^{\mi \langle \bs s,\Y\rangle}\right)~\mathrm{d}t.
\end{align*}
By Fubini theorem, under Assumption $\ref{as:joint distribution assumption}$, one can exchange the order of integration and covariance above and get that
\begin{align*}
 \langle G_{\bs s},{\bs{\beta}}\rangle&=\int_{[0,1]}\mathrm{cov}\hspace{-0.9mm}\left(\bs{X}(t){\bs{\beta}}(t),\mathrm{e}^{\mi \langle \bs s,\Y\rangle}\right)~\mathrm{d}t\\ &=\mathrm{cov}\hspace{-0.9mm}\left(\int_{[0,1]}\bs{X}(t){\bs{\beta}}(t)~\mathrm{d}t,\mathrm{e}^{\mi \langle \bs s,\Y\rangle}\right)=\mathrm{cov}\hspace{-0.9mm}\left(\langle \bs{X},{\bs{\beta}}\rangle,\mathrm{e}^{\mi \langle \bs s ,\Y\rangle}\right).
\end{align*}
Thus for any $\bs\alpha(t),{\bs{\beta}}(t)\in\mc H$, one can get
\begin{align*}
\big\langle \big(G_{\bs s}\otimes \overline{G}_{\bs s}\big)\bs\alpha,{\bs{\beta}}\big\rangle=\langle G_{\bs s},\bs\alpha\rangle\langle \overline{G}_{\bs s},{\bs{\beta}}\rangle=\mathrm{cov}\hspace{-0.9mm}\left(\langle \bs{X},\bs\alpha\rangle,\mathrm{e}^{\mi \langle \bs s,\Y\rangle}\right)\hspace{-0.9mm}\mathrm{cov}\hspace{-0.9mm}\left(\langle \bs{X},{\bs{\beta}}\rangle,\mathrm{e}^{-\mi\langle \bs s,\Y\rangle}\right)\\
=\mb{E}\hspace{-0.9mm}\left(\langle \bs{X},\bs\alpha\rangle\mathrm{e}^{\mi \langle \bs s,\Y\rangle}\right)\mb{E}\hspace{-0.8mm}\left(\langle \bs{X},{\bs{\beta}}\rangle\mathrm{e}^{-\mi \langle \bs s,\Y\rangle}\right)=\mb{E}\Big(\langle \bs{X},\bs\alpha\rangle\langle \bs{X}',{\bs{\beta}}\rangle\mathrm{e}^{\mi \langle \bs s,\Y-\Y'\rangle}\Big).
\end{align*}
Considering that $\mb{E}\big(\langle \bs{X},\alpha\rangle\langle \bs{X}',{\bs{\beta}}\rangle\big)=0$, one has
\begin{align*}
\big\langle \big(G_{\bs s}\otimes \overline{G}_{\bs s}\big)\bs\alpha,{\bs{\beta}}\big\rangle
=- \mb{E}\Big(\langle \bs{X},\bs\alpha\rangle\langle \bs{X}',{\bs{\beta}}\rangle\big(1-\mr{e}^{\mi \langle \bs s,\Y-\Y'\rangle}\big)\Big)&\\
=- \mb{E}\Big(\langle \bs{X},\bs\alpha\rangle\langle \bs{X}',{\bs{\beta}}\rangle\big[1-\cos\big(\langle \bs s,\Y-\Y'\rangle\big)\big]\Big)&\\
+\mi\mb{E}\Big(\langle \bs{X},\bs\alpha\rangle\langle \bs{X}',{\bs{\beta}}\rangle\big[\sin\big(\langle\bs s,\Y-\Y'\rangle\big)\big]\Big)&.
\end{align*}
It is easy to check that
\[\int_{\mb R^q}\frac{\sin \big(\langle\bs s,\Y-\Y'\rangle)\big)}{\|\bs s\|^{1+q}}~\mr{d}\bs s=\lim_{\varepsilon\to0^+}\int_{\bs s\in\mb{R}^q:\varepsilon\leqslant\|\bs s\|\leqslant \varepsilon^{-1}}\frac{\sin \big(\langle \bs s,\Y-\Y'\rangle\big)}{\|\bs s\|^{1+q}}~\mr{d}\bs s=0,\]
because the integrand is an odd function. By Lemma 1 in \cite{szekely2007measuring},  one can also get
\[\int_{\R^q}\frac{1-\cos\big(\langle \bs s,\Y-\Y'\rangle\big)}{\|\bs s\|^{1+q}}~\mr{d}\bs s=c_q\|\Y-\Y'\|.
\]
Combining above results with Definition $\ref{def: MDDO}$, one can obtain that 
\begin{align}\label{proof: lemma MDDO}
\langle\mathrm{MDDO}(\bs{X}|Y)\bs\alpha,{\bs{\beta}}\rangle=- \mb{E}\Big(\langle \bs{X},\bs\alpha\rangle\langle \bs{X}',{\bs{\beta}}\rangle\|\Y-\Y'\|\Big) .
\end{align}
Then by the arbitrariness of $\bs\alpha,{\bs{\beta}}\in\mc H$, the proof is completed. 
\end{proof}

\section{Proof of Theorem \ref{theorem, MDDO and conditional mean independence}}



According to \eqref{proof: lemma MDDO}, one can get the following useful lemma.
\begin{lemma}\label{lemma, MDDO and FMDD}
Under Assumption $\ref{as:joint distribution assumption}$, for all ${\bs{\beta}}\in\mathcal H$, $\|{\bs{\beta}}\|=1$, we have
\begin{align*}
\langle \mathrm{MDDO}(\boldsymbol{X}|\Y)({\bs{\beta}}),{\bs{\beta}}\rangle &=- \mathbb E\Big[ \langle \boldsymbol{X},{\bs{\beta}}\rangle \langle \boldsymbol{X}',{\bs{\beta}}\rangle \|\Y-\Y'\|\Big]\\
&=- \mathbb E\Big[\big\langle\langle \boldsymbol{X},{\bs{\beta}}\rangle{\bs{\beta}},\langle \boldsymbol{X}',{\bs{\beta}}\rangle{\bs{\beta}}\big\rangle\|\Y-\Y'\|\Big].
\end{align*}
\end{lemma}
This conclusion links MDDO with functional martingale
difference divergence  (FMDD, \citealt{lee2020testing}). 
Next we give the following two lemmas to finish the proof of Theorem $\ref{theorem, MDDO and conditional mean independence}$.
\begin{lemma}\label{lem: Txx=0tuiTx=0}If $T$ is a positive semi-definite operator on a Hilbert space $\wt{\mathcal{H}}$, then for all $x\in\wt{\mathcal{H}}$, one has $\langle Tx,x\rangle=0\Longleftrightarrow Tx=0$.
\end{lemma}
\begin{proof}
`$\Longleftarrow$': It is obvious.

`$\Longrightarrow$': It is easy to check that $f(a,b)=\langle Ta,b\rangle$ $(a,b\in\wt{\mc H})$ is a 
positive semi-definite Hermitian form. Thus, for any $y\in\wt{\mathcal{H}}$, one can use Cauchy inequality to get
\[|\langle Tx,y\rangle|^2\leqslant\langle Tx,x\rangle\langle Ty,y\rangle=0\Longrightarrow \langle Tx,y\rangle=0.\]
By the arbitrariness of $y\in\wt{\mc H}$, one has $Tx=0$.
\end{proof}

Our proof of Theorem $\ref{theorem, MDDO and conditional mean independence}$ is mainly inspired by the following property of
FMDD in \cite{lee2020testing}.
\begin{lemma}[Proposition 1 of \cite{lee2020testing}]\label{lem:prop1inlee}
If $\E[\|\X\|+\|\Y\|]<\infty$ and $\E[\|\bs X\|\|\Y\|]<\infty$, then we have
\[\E[\langle \X,\X'\rangle\|\Y-\Y'\|]=0\Longleftrightarrow \E[\X|\Y]=0\quad\text{almost surely},\]
where $(\X',\Y')$ is an i.i.d. copy of $(\X,\Y)$.
\end{lemma}
\paragraph{Proof of Theorem $\ref{theorem, MDDO and conditional mean independence}$}
\begin{proof}
Clearly, (ii) is a direct consequence of Lemma $\ref{lemma, equivalence of two def of MDDO}$ and the following lemma.

\begin{lemma}[Lemma 15 in \citealt{chen2023optimality}]\label{lem:cov TX}
If $T$ is an operator defined on $\mc H_1\to\mc H_2$ where $\mc H_i,i=1,2$ is a Hilbert space. $\bs X\in\mc H_1$ is a random element satisfying $\mb E[\bs X]=0$ . Then we have $\mr{var}(T\bs X)=T\mr{var}(\bs X)T^*$.
\end{lemma}

Now we start  to prove (i).
 First, one has
\begin{align*}\mathrm{MDDO}(\boldsymbol{X}|\Y)=0 &\Longleftrightarrow \mathrm{MDDO}(\boldsymbol{X}|\Y)({\bs{\beta}})=0,\quad\forall{\bs{\beta}}\in\mb{S}_{\mathcal H};\\
\mathbb E[\boldsymbol{X}|\Y]=0~~\text{a.s.}&\Longleftrightarrow\langle\mb E[\boldsymbol{X}|\Y],{\bs{\beta}}\rangle{\bs{\beta}}=0~~\text{a.s.} \quad\forall{\bs{\beta}}\in\mb{S}_{\mathcal H},
\end{align*}
where $\mb{S}_\mc{H}=\{{\bs{\beta}}\in\mc H:\|{\bs{\beta}}\|=1\}$. Second, from Lemma $\ref{lem: Txx=0tuiTx=0}$, one knows that
\begin{align*}
\mathrm{MDDO}(\boldsymbol{X}|\Y)({\bs{\beta}})=0&\Longleftrightarrow\langle\mathrm{MDDO}(\boldsymbol{X}|\Y)({\bs{\beta}}),{\bs{\beta}}\rangle=0.
\end{align*}
 Then under Assumption $\ref{as:joint distribution assumption}$, by Lemma $\ref{lemma, MDDO and FMDD}$ and $\ref{lem:prop1inlee}$, one has
\begin{align*}
&\langle\mathrm{MDDO}(\boldsymbol{X}|\Y)({\bs{\beta}}),{\bs{\beta}}\rangle=0\Longleftrightarrow\mathbb E[\big\langle\langle \boldsymbol{X},{\bs{\beta}}\rangle{\bs{\beta}},\langle \boldsymbol{X}',{\bs{\beta}}\rangle{\bs{\beta}}\rangle\|\Y-\Y'\|]=0\\
&\qquad\qquad\qquad\qquad\qquad\Longleftrightarrow\mathbb E[\langle \bs X,{\bs{\beta}}\rangle{\bs{\beta}}|\Y]=\langle \mb E[\boldsymbol{X}|\Y],{\bs{\beta}}\rangle{\bs{\beta}}=0~~\text{a.s.}
\end{align*}
This finishes the proof of Theorem $\ref{theorem, MDDO and conditional mean independence}$.
\end{proof}

% {\color{blue}\paragraph{Proof of Lemma \ref{lem:cov TX} (Repeated)}
% \begin{proof}
% For any $\u_1,\u_2\in\mc H_2$, we have
% \begin{align*}
% &\left\langle  T\mr{var}(\vX)T^*\u_1,\u_2  \right\rangle=\left\langle  T\mb E[\vX\otimes\vX]T^*\u_1,\u_2  \right\rangle
% =\left\langle  \mb E[\vX\otimes\vX]T^*\u_1,T^*\u_2  \right\rangle    
% \end{align*}
% since $\mb E[\vX]=0$. By the definition of convariance operator and expectation, we have 
% \begin{align*}
% \left\langle  \mb E[\vX\otimes\vX]T^*\u_1,T^*\u_2  \right\rangle=&\left\langle  \mb E[\left\langle\vX,  T^*\u_1 \right\rangle       \vX            ],T^*\u_2  \right\rangle
% =\mb E[  \left\langle\vX,  T^*\u_1 \right\rangle      \left\langle \vX            ,T^*\u_2  \right\rangle].
% \end{align*}
% Similarly, we have
% \begin{align*}
%  \left\langle  \mr{var}(T\vX)\u_1,\u_2  \right\rangle=\left\langle  \mb E[T\vX\otimes T\vX]\u_1,\u_2  \right\rangle=\mb E[  \left\langle T\vX,  \u_1 \right\rangle      \left\langle T\vX            ,\u_2  \right\rangle].\\    
% \end{align*}
% Then the proof is completed by noticing the following
% \begin{align*}
% \mb E[  \left\langle T\vX,  \u_1 \right\rangle      \left\langle T\vX            ,\u_2  \right\rangle]=\mb E[  \left\langle\vX,  T^*\u_1 \right\rangle      \left\langle \vX            ,T^*\u_2  \right\rangle].
% \end{align*}
% \end{proof}}



\section{Proof of Lemma \ref{lemma: SE=GammaS}}
Recall the following fact in FSIR.
\begin{lemma}\label{lemma, direct result of linearity condition}~\\
Under Assumption $\ref{as:Linearity condition and Coverage condition}~ \boldsymbol{\mathrm{{i)}}}$, we have $\mathcal S_{\mathbb E(\boldsymbol{X}|\Y)}\subseteq \Gamma \mc S_{\Y|\bs X}\subseteq \mc H$.
\end{lemma}
It is a trivial generalization of    \cite[Theorem 2.1]{ferre2003functional} from univariate response to multivariate response.
\paragraph{Proof of Lemma $\ref{lemma: SE=GammaS}$}
\begin{proof}
First, we prove that $\mathcal{S}_{\mathbb{E}(\bs X|\Y)}^\perp\subseteq \mathrm{Im}\{\mathrm{var(\mb{E}(\bs X|\Y))}\}^\perp$. For any ${\bs{\beta}}\in\mathcal{S}_{\mathbb{E}(\bs X|\Y)}^\perp$, one has $\langle{\bs{\beta}},\mb{E}(\bs X|\Y)\rangle=0$ a.s. Then for any $\bs\alpha\in\mathcal{H}$, one can get
\begin{align*}
\langle{\bs{\beta}},\mathrm{var}(\mb{E}(\bs X|\Y))\bs\alpha\rangle&=\langle{\bs{\beta}},\mb E\lmi\mb{E}(\bs X|\Y)\otimes \mb{E}(\bs X|\Y)\rmi\bs\alpha\rangle\\
&=\mb E\big(\langle\mb{E}(\bs X|\Y),\bs\alpha\rangle\langle{\bs{\beta}},\mb{E}(\bs X|\Y)\rangle\big)=0,
\end{align*}
which means that ${\bs{\beta}}\in\mathrm{Im}\{\mathrm{var}(\mb{E}(\bs X|\Y))\}^\perp$. Moreover, one has
\begin{align*}\mathcal{S}_{\mathbb{E}(\bs X|\Y)}^\perp\subseteq \mathrm{Im}\{\mathrm{var}(\mb{E}(\bs X|\Y))\}^\perp
%&\Rightarrow\left(\mathcal{S}_{\mathbb{E}(\bs X|Y)}^\perp\right)^\perp\supseteq \left(\mathrm{Im}\{\mathrm{var(\mb{E}(\bs X|Y))}\}^\perp\right)^\perp\\&
\Longrightarrow\overline{\mathcal{S}_{\mathbb{E}(\bs X|\Y)}}\supseteq\overline{\mathrm{Im}}\{\mathrm{var}(\mb{E}(\bs X|\Y))\}.
\end{align*}
Thus, $\overline{\mathrm{Im}}\{\mathrm{var}(\mb{E}(\bs X|\Y))\}\subseteq\overline{\mathcal{S}_{\mathbb{E}(\bs X|\Y)}}\subseteq\overline{\Gamma\mathcal{S}_{\Y|\bs X}}$ by Lemma $\ref{lemma, direct result of linearity condition}$. According to Assumption $\ref{as:Linearity condition and Coverage condition}$ \textbf{ii)}, one can get
\[\mathrm{dim}\left(\overline{\mathrm{Im}}\{\mathrm{var}(\mb{E}(\bs X|\Y))\}\right)=\mathrm{dim}\left(\overline{\mathcal{S}_{\mathbb{E}(\bs X|\Y)}}\right)=\mathrm{dim}(\overline{\Gamma\mathcal{S}_{\Y|\bs X}})=d.\]
One can complete the proof since finite dimension subspaces are closed.
\end{proof}

\section{Proof of Theorem \ref{theorem, MDDO and IRS}}
\begin{proof}
For convenience, we abbreviate $\mathrm{MDDO}(\boldsymbol{X}|\Y)$ to ${M}$. According to Theorem $\ref{theorem, MDDO and conditional mean independence}$ and Lemma $\ref{lem: Txx=0tuiTx=0}$, one can get
\begin{align*}{\bs{\beta}}\in\mathcal S_{\mb E(\boldsymbol{X}|\Y)}^\perp&\Longleftrightarrow\langle {\bs{\beta}},\mathbb E(\boldsymbol{X}|\Y)\rangle=0~~\text{a.s.}\Longleftrightarrow\mathbb E(\langle {\bs{\beta}},\boldsymbol{X}\rangle|\Y)=0~~\text{a.s.}\\
&\Longleftrightarrow\mathrm{MDDO}(\langle {\bs{\beta}},\boldsymbol{X}\rangle|\Y)=0\Longleftrightarrow\langle {M}{\bs{\beta}},{\bs{\beta}}\rangle=0\\
&\Longleftrightarrow{M}{\bs{\beta}}=0\Longleftrightarrow {\bs{\beta}}\in\mathrm{null}(M)=\overline{\mathrm{Im}}(M)^\perp,
\end{align*}
which means that $\mathcal S_{\mb E(\boldsymbol{X}|\Y)}^\perp=\overline{\mathrm{Im}}(M)^\perp$ and $\overline{\mathcal S_{\mb E(\boldsymbol{X}|\Y)}}=\overline{\mathrm{Im}}(M)$.
One can complete the proof since finite dimension subspaces are closed.
\end{proof}
\section{Proof of Lemma \ref{lemma, way of estimate truncate central subspace}}
Before proving Lemma $\ref{lemma, way of estimate truncate central subspace}$, we give the following lemma.
\begin{lemma}\label{lem: colPBP equal colPB operator}
Assume that $P$ is a bounded linear operator from a Hilbert space $\wt{\mc H}$ to itself and $B$ is a positive semi-definite operator from $\wt{\mc H}$ to itself. 
Then we have $\overline{\mathrm{Im}}(PBP^*)=\overline{\mathrm{Im}}(PB)$.
\end{lemma}
\begin{proof}
It suffices to show that $\mnull(BP^*)=\mnull(PBP^*)$. First, since $B$ is positive semi-definite, one has $\langle x,PBP^*x\rangle = \langle P^*x,BP^*x \rangle\geqslant 0~(\forall x\in\wt{\mc H})$. Thus $PBP^*$ is a positive semi-definite operator on $\wt{\H}$.
For any $y\in\wt{\H}$, we have 
\begin{align*}
PBP^*y=0\overset{(a)}{\Longleftrightarrow}\langle y,PBP^*y\rangle = \langle P^*y,BP^*y \rangle=0\overset{(b)}{\Longleftrightarrow} BP^*y=0. 
\end{align*}
where $(a)$ and $(b)$ come from Lemma $\ref{lem: Txx=0tuiTx=0}$.
Thus $\mnull(PBP^*)=\mnull(BP^*)$.
\end{proof}

\paragraph{Proof of Lemma $\ref{lemma, way of estimate truncate central subspace}$}
\begin{proof}
For convenience, we abbreviate $\mathrm{MDDO}(\boldsymbol{X}|\Y)$ and $\mathrm{MDDO}(\boldsymbol{X}^{(m)}|\Y)$ to ${M}$ and $M_m$ respectively. 

By Corollary $\ref{corollary, MDDO and central subspace}$, one can get $\Gamma\mathcal{S}_{\Y|\boldsymbol{X}}=\mathrm{Im}(M)$. Thus,
\begin{align}\label{eq: corollary, MDDO and central subspace}
\Pi_m\Gamma\mathcal{S}_{\Y|\boldsymbol{X}}=\Pi_m\mathrm{Im}(M)=\mathrm{Im}(\Pi_m M).
\end{align}
It is easy to check that
\begin{align}
\Gamma_m&:=\mathrm{var}(\bs X^{(m)})=\Pi_m\Gamma\Pi_m=\Pi_m\Gamma=\Gamma\Pi_m=\sum\limits_{i=1}^m\lambda_i\phi_i\otimes\phi_i.\label{eq: Gamma m def}
\end{align}
On the one hand, by the definition of $\mathcal{S}^{(m)}_{{\Y|\boldsymbol{X}}}$ and $\Gamma_m$ (see \eqref{def: truncated central subspace} and \eqref{eq: Gamma m def}), one can get
\begin{align}\label{eq:Pim Gamma S}
\Pi_m\Gamma\mathcal{S}_{\Y|\boldsymbol{X}}&=\Pi_m\Gamma\Pi_m\mathcal{S}_{\Y|\boldsymbol{X}}=(\Pi_m\Gamma)(\Pi_m\mathcal{S}_{\Y|\boldsymbol{X}})=\Gamma_m\mathcal{S}^{(m)}_{{\Y|\boldsymbol{X}}}.
\end{align}
On the other hand, one has $\overline{\mathrm{Im}}(\Pi_m M)=\overline{\mathrm{Im}}(\Pi_m M\Pi_m)$ by Lemma $\ref{lem: colPBP equal colPB operator}$. Since $\Pi_m M$ and $\Pi_m M\Pi_m$ are both of finite rank, one can further get
\begin{align*}
\mathrm{Im}(\Pi_m M)&=\overline{\mathrm{Im}}(\Pi_m M)=\overline{\mathrm{Im}}(\Pi_m M\Pi_m)=\mathrm{Im}(\Pi_mM\Pi_m).
\end{align*}
Then according to Theorem $\ref{theorem, MDDO and conditional mean independence}$(ii), one has
\begin{align}
\mathrm{Im}(\Pi_m M)=\mathrm{Im}(\Pi_mM\Pi_m)=\mathrm{Im}(M_m).\label{eq:Pim span M}
\end{align}
Combining \eqref{eq:Pim Gamma S}, \eqref{eq:Pim span M} with \eqref{eq: corollary, MDDO and central subspace}, one has $\Gamma_m\mathcal{S}^{(m)}_{{\Y|\boldsymbol{X}}}=\mathrm{Im}\{M_m\}$.
Finally, one can get $ \Gamma_m^\dagger\mathrm{Im}\{M_m\}=\Gamma_m^\dagger\Gamma_m\mathcal{S}^{(m)}_{{\Y|\boldsymbol{X}}}=\Pi_m\mathcal{S}^{(m)}_{{\Y|\boldsymbol{X}}}=\mathcal{S}^{(m)}_{{\Y|\boldsymbol{X}}}$.
\end{proof}

\section{Wely Inequality for a Self-adjoint and Compact Operator}\label{ap:Wely inequality for self-adjoint and compact operators}
First, we show the following three results in standard functional analysis textbook.
\begin{lemma}[Spectral theorem]\label{thm: Spectral theorem}Let $\wt{\mathcal{H}}$ be a Hilbert space and $A:\wt{\mc{H}}\to\wt{\mc{H}}$ be a compact, self-adjoint operator. There is an at most countable orthonormal basis $\{\wt e_j\}_{j\in J}$ ($J=\{1,\cdots,n\}$ or $\mathbb{Z}_{\geqslant1}$) of $\wt{\mathcal{H}}$ and eigenvalues $\{\wt\lambda_j\}_{j\in J}$ with $|\wt\lambda_1|\geqslant|\wt\lambda_2|\geqslant\cdots\geqslant0$ converging to zero, such that
\begin{align*}
x=\sum_{j\in J}\langle x,\wt e_j\rangle \wt e_j;\qquad Ax=\sum_{j\in J}\wt\lambda_j\langle x,\wt e_j\rangle \wt e_j,\qquad x \in\wt{\mathcal{H}}.
\end{align*}
\end{lemma}

\begin{lemma}[Rayleigh's principle]\label{lem:Rayleigh operator}Let $A$ be a compact, self-adjoint operator. If $\{\wt e_j\}_{j\in J}$ and $\{\wt\lambda_j\}_{j\in J}$ are eigenvectors and eigenvalues define in Lemma $\ref{thm: Spectral theorem}$ respectively. Then
\[|\wt\lambda_1|=\mathop{\sup\limits_{\|u\|=1}}|\langle Au,u\rangle|;\qquad|\wt\lambda_n|=\mathop{\sup\limits_{\|u\|=1}}_{u\in\{\wt e_1,\cdots,\wt e_{n-1}\}^\perp}|\langle Au,u\rangle|~(n\geqslant 2).\]
\end{lemma}
\begin{lemma}[Minimax theorem]\label{lem:minimax operator}
Assume that $A$ is a positive semi-definite and compact operator with its eigenvalues $\{\wt\lambda_i\}$ ordered as $\wt\lambda_1\geqslant\dots\geqslant \wt\lambda_n\geqslant\dots\geqslant 0$, then
$$
\wt\lambda_n=\inf_{E_{n-1}}\sup_{x\in E_{n-1}^\perp,\|x\|=1}\langle Ax,x\rangle
$$
where $E_{n-1}$ with dimension $n-1$ is a closed linear subspace of $\wt{\mc H}$.
\end{lemma}
Then we give the Wely inequality for a self-adjoint and compact operator.
\begin{proposition}\label{prop: wely operator}
Let $M=N+R$ where $M$, $N$ and $R$ are three self-adjoint and compact operators defined on a Hilbert space $\wt{\mc H}$. Also, $M$ and $N$ are positive semi-definite with their respective eigenvalues $\{\mu_i\},\{\nu_i\}$ ordered as follows
\begin{align*}
M:\mu_1\geqslant\dots\geqslant \mu_n\geqslant\dots\geqslant 0;\qquad
N:\nu_1\geqslant\dots\geqslant \nu_n\geqslant\dots\geqslant 0,
\end{align*}
while $R$'s eigenvalues are $\{\rho_i\}$ ordered as follows:
\[R:|\rho_1|\geqslant\dots\geqslant |\rho_n|\geqslant\dots\geqslant 0.\]
Then the following inequalities hold: $|\mu_k-\nu_k|\leqslant|\rho_1|=\|R\| $, $k\geqslant1$.
\end{proposition}
\begin{proof}
From Lemma $\ref{lem:minimax operator}$, we have:
\[\mu_n=\inf_{E_{n-1}}\sup_{x\in E_{n-1}^\perp,\|x\|=1}\langle Mx,x\rangle;\qquad\nu_n=\inf_{E_{n-1}}\sup_{x\in E_{n-1}^\perp,\|x\|=1}\langle Nx,x\rangle,\]
where $E_{n-1}$ with dimension $n-1$ is a closed linear subspace of $\wt{\mc H}$.
By Lemma $\ref{lem:Rayleigh operator}$, we have:
$$
\sup_{\|u\|=1}|\langle Ru,u\rangle|=|\rho_1|.
$$
Since $\langle Mu,u\rangle=\langle Nu,u\rangle+\langle Ru,u\rangle$, for any $\|u\|=1$, we have:
$$
\langle Nu,u\rangle-|\rho_1|\leqslant\langle Mu,u\rangle \leqslant \langle Nu,u\rangle+|\rho_1|.
$$
Then for any given $n-1$ dimensional closed linear subspace of $\wt{\mc H}$, we conclude
\begin{equation}\label{eq:max ineq}
\sup_{u\in E_{n-1}^\perp,\|u\|=1}\langle Nu,u\rangle-|\rho_1|\leqslant\sup_{u\in E_{n-1}^\perp,\|u\|=1}\langle Mu,u\rangle\leqslant \sup_{u\in E_{n-1}^\perp,\|u\|=1}\langle Nu,u\rangle+|\rho_1|.
\end{equation}
Take the infimum with respective to $E_{n-1}$ in \eqref{eq:max ineq}, we have
\[\nu_n-|\rho_1|\leqslant\mu_n\leqslant \nu_n+|\rho_1|\]
by Lemma $\ref{lem:minimax operator}$.
\end{proof}
The next result is a direct corollary of Proposition $\ref{prop: wely operator}$.
\begin{corollary}\label{coro:wely ineq operator}
Let $M$ and $N$ be two self-adjoint, positive semi-definite and compact operators defined on a Hilbert space $\wt{\mc H}$ with their respective eigenvalues $\{\mu_i\},\{\nu_i\}$ ordered as follows
\begin{align*}
M:\mu_1\geqslant\dots\geqslant \mu_n\geqslant\dots\geqslant 0\quad\text{and}\quad
N:\nu_1\geqslant\dots\geqslant \nu_n\geqslant\dots\geqslant 0.
\end{align*}
Then the following inequalities hold: $|\mu_k-\nu_k|\leqslant\|M-N\| $, $ k\geqslant1$.
\end{corollary}




\section{Proof of Proposition \ref{prop:bound hatMmd Mm}}
Before proving Proposition $\ref{prop:bound hatMmd Mm}$, we give the following conclusion, whose proof is deferred to the end of this section.
\begin{proposition}\label{proposition, concentration of MDDO}
Under Assumptions $\ref{as:joint distribution assumption}$ and $\ref{assumption: sub-Gaussian}$, for all $\gamma\in(0,1/2)$, there exist positive constants $D_0=D_0(\gamma,\sigma_0,\sigma_1)$, $D_1=D_1(\sigma_1)$, $D_2=D_2(\sigma_0,\sigma_1)$ and $n_0=n_0(\gamma,\sigma_0,\sigma_1)$ such that for all $n\geqslant n_0$ and
\[C\in \l D_0n^{\frac{2\gamma}{5}}-\ln\l D_1m^2n \r,D_2 n^{\frac{1}{5}}-\ln\l D_1m^2n \r \rmi,\]
we have
\begin{equation*}
\mathbb{P}\l\left\|\wh M_m- M_m\right\| <\l \frac{C+\ln( D_1m^2n)}{D_2}\r^{\frac52}\frac{12m}{\sqrt n}\r\geqslant 1-\exp(- C).
\end{equation*}
\end{proposition}
\paragraph{Proof of Proposition $\ref{prop:bound hatMmd Mm}$}
\begin{proof}




Using Corollary $\ref{coro:wely ineq operator}$, one can get
$
\lambda_i\l\wh M_m\r\leqslant \lno\wh M_m-M_m\rno +\lambda_i\l M_m\r
$. 
Since $\rank(M_m)=d$, one can get $\lambda_i(M_m)=0,~i\geqslant d+1$. Thus by Proposition $\ref{proposition, concentration of MDDO}$, one has
\begin{align}\label{eq:lambdai hat Mm upper bound}
\mathbb{P}\l\lambda_{d+1}(\wh M_m)<\l \frac{C+\ln\l D_1m^2n\r}{D_2}\r^{\frac52}\frac{12m}{\sqrt n}\r\geqslant 1-\exp(- C)\qquad(i\geq d+1). 
\end{align}
Notice that 
\begin{align*}\lno\wh M_m^d- M_m\rno &\leqslant\lno M_m-\wh M_m\rno +\lno\wh M_m-\wh M_m^d\rno ;\\
\lno\wh M_m-\wh M_m^d\rno &=\left\|\sum_{i=d+1}^\infty\wh\mu_i\wh\gamma_i\otimes \wh\gamma_i\right\| =\widehat{\lambda}_{d+1}=\lambda_{d+1}(\widehat{M}_m)
\end{align*}
by \eqref{wh M_m spectral decomposition}.
Then combing Proposition $\ref{proposition, concentration of MDDO}$ with \eqref{eq:lambdai hat Mm upper bound} can complete the proof.
\end{proof}


\paragraph{Proof of Proposition \ref{proposition, concentration of MDDO}}
\begin{proof}
Note that $\boldsymbol{X}^{(m)}=\sum\limits_{j=1}^m\langle \boldsymbol{X},\phi_j\rangle\phi_j$, then a simple calculation leads to
\begin{align*}
M_m&=-\sum_{i,j=1}^m\mathbb E\big[\langle \boldsymbol{X},\phi_i\rangle\langle \boldsymbol{X}',\phi_{j}\rangle\|\Y-\Y'\|\big]\phi_i\otimes\phi_j;\\
\widehat{M}_m&=-\sum_{i,j=1}^m\frac1{n^2}\sum_{k,\ell=1}^n\langle \boldsymbol{X}_k,\phi_i\rangle\langle \boldsymbol{X}_\ell,\phi_j\rangle\|\Y_k-\Y_\ell\|\phi_i\otimes\phi_j.
\end{align*}

For a operator $\Gamma'$ that can be expanded as $\Gamma':=\sum\limits_{i,j=1}^\infty a_{ij}\phi_i\otimes\phi_{j}$, let us define its maximal norm as $\|\Gamma'\|_{\mathrm{max}}=\sup\limits_{i,j}|a_{ij}|$.



\begin{lemma}\cite[Theorem 1]{mai2021slicing}\label{lemma, concentration of MDDOnm}
Under Assumptions $\ref{as:joint distribution assumption}$ and $\ref{assumption: sub-Gaussian}$, for all
$\gamma\in(0,1/2)$, there exist positive
constants $C_0=C_0(\gamma,\sigma_0,\sigma_1)$, $C_1=C_1(\sigma_1)$, $C_2 = C_2(\sigma_0;\sigma_1)$ and $n_0 = n_0(\gamma,\sigma_0,\sigma_1)$
such that for all $n\geqslant n_0$ and $\varepsilon\in(C_0 n^{-(1/2-\gamma)},1]$, we have
\begin{equation*}
\mathbb{P}\l\lno \widehat{M}_m-M_m\rno_{\max}>12\varepsilon\r\leqslant C_1 m^2n\exp\l- C_2\l\varepsilon^2 n\r^{1/5}\r.
\end{equation*}
\end{lemma}
\noindent Since $\lno\widehat{M}_m-M_m\rno \leqslant m\lno\widehat{M}_m-M_m\rno_{\mathrm{max}}$, one has
\begin{equation*}
\mathbb{P}\l\lno\widehat{M}_m-M_m\rno >12m\varepsilon\r\leqslant C_1 m^2n\exp\l-C_2\l\varepsilon^2 n\r^{1/5}\r.
\end{equation*}
Let $C=C_2\l\ve^2n\r^{1/5}-\ln\l C_1m^2n\r$ satisfying 
\begin{align*}
C\in\l C_2C_0^{2/5}n^{2\gamma/5}-\ln\l C_1m^2n\r,C_2n^{1/5}-\ln\l C_1m^2n\r\rmi,
\end{align*}
then one has
\begin{equation*}
\mathbb{P}\l\lno\widehat{M}_m-M_m\rno \leqslant\l \frac{C+\ln\l C_1m^2n\r}{C_2}\r^{\frac52}\frac{12m}{\sqrt{n}}\r>1- \exp(- C).
\end{equation*}
Then in order to complete the proof, one only need to choose $D_0$, $D_1$ and $D_2$ to be $C_2C_0^{2/5}$, $C_1$ and $C_2$ respectively. 
\end{proof}





\section{Properties of Sub-Gaussian Random Vectors}
We first review the definition of sub-Gaussian random variables.
\begin{definition}[Sub-Gaussian random variable and its upper-exponentially bounded constant]\label{def:sub gaussian variable}
A random variable $X$ is called a sub-Gaussian random variable if $X$ satisfies one of the following equivalent properties:
\begin{itemize}
 \item[1).] Tails. $\P(|X|>t)\leqslant \exp(1-t^{2}/K^{2}_{1})$ for any $t>0$;
 \item[2).] Moments. $\E[|X|^{p}]^{1/p}\leqslant K_{2}\sqrt{p}$ for any $p\geqslant 1$;
 \item[3).]Super-exponential moment: $\E[\exp(X^{2}/K^{2}_{3})]\leqslant \mr{e}$.

\noindent Moreover, if $\E[X]=0$, then the properties $1)-3)$ are also equivalent to the following one:
\item[4).] Moment generating function: $\E[\exp(tX)]\leqslant \exp(t^{2}K^{2}_{4})$ for all $t\in\R$.
\end{itemize}
Here $K_1$, $K_2$, $K_3$ and $K_4$ are four constants.
$K$ is called an upper-exponentially bounded constant of $X$ if 
$K\geqslant \max\{K_{1},K_{2},K_{3},K_{4}\}$.
\end{definition}
\begin{definition}[Sub-Gaussian random vector and its upper-exponentially bounded constant]\label{def,sub-Gaussian random vector,upper-exponentially bounded constant}
 ${X}\in\R^m$ is called a sub-Gaussian random vector if for all $x\in\R^m$, one-dimensional marginal $\langle{X},x\rangle$ is sub-Gaussian random variable. $K$ is called an upper-exponentially bounded constant of $X$ if $K$ satisfies:
 \begin{align*}
K\geqslant \sup_{x\in\mb{S}^{m-1}}K(\langle X,x\rangle) 
 \end{align*}
 where $K(\langle X,x\rangle)$ denotes an upper-exponentially bounded constant of $\langle X,x\rangle$.
Moreover, $K$ is called a uniform (about $m$) upper-exponentially bounded constant of $X$ if $K$ satisfies:
 \begin{align*}
K\geqslant \sup_m\sup_{x\in\mb{S}^{m-1}}K\l \langle X,x\rangle\r.
 \end{align*}
Furthermore, $X$ is called a uniform (about $m$) sun-Gaussian random vector.
 \end{definition}
The following is an application of sub-Gaussian random vectors.
\begin{lemma}[\citealt{vershynin2010introduction}]\label{lem:esgrm}
 Let $\M=[\bs m_1~\cdots~\bs m_n]$ be an $m\times n$ matrix ($n>m$) whose columns $\m_{i}$ are 
 independent centered sub-Gaussian random vectors with 
 covariance matrix $\mathbf{I}_{m}$. Let $\sigma^{+}_{\min}(\M)$ and $\sigma_{\max}(\M)$ be the infimum and supremum of positive singular values of $\M$ respectively. Then, for any $t>0$, with probability at least $1-2\exp(- C^{\prime}t^{2})$, we have
 \begin{equation*}
 \sqrt{n}-C_0\sqrt{m}-t\leqslant \sigma^{+}_{\min}(\M)\leqslant \sigma_{\max}(\M)\leqslant \sqrt{n}+C_0\sqrt{m}+t
 \end{equation*}
 where $C'$ and $C_0$ are two positive constants depending only on $K(\bs m_1)$:
 the upper-exponentially bounded constant of $\bs m_1$.
\end{lemma}
\noindent Let $t=\sqrt m$, then one can easily get
\begin{align}\label{equation, min max eval}
\begin{split}
\lambda_{\max}\left(\frac1n \M\M^\top\right)\leqslant \left(1+\frac{(C_0+1)\sqrt m}{\sqrt n}\right)^2;\\
\lambda_{\min}^+\left(\frac1n \M\M^\top\right)\geqslant \left(1-\frac{(C_0+1)\sqrt m}{\sqrt n}\right)^2, 
\end{split}
\end{align}
with probability at least $1-2\exp(- C'm)$ where $\lambda^{+}_{\min}(\cdot)$ and $\lambda_{\max}(\cdot)$ stands for the infimum and supremum of the positive spectrum respectively.



\begin{lemma}\label{lemma, estiamtion error of inverse sample cov}
Assume that $\x_1,\x_2,...,\x_n$ are $n$ i.i.d. samples from an $m$-dimensional centered sub-Gaussian vector with an invertible covariance matrix $\Sigma$. Let $\wh\Sigma:=\frac1n\sum_i \x_i\x_i^\top$.
Then there exists a positive constant $n_1'=n_1'(K(\bs m_1),c_1)$ ($c_1$ is defined in \eqref{eq: m n relationship}), such that when $n\geqslant n_1'$, we have
\begin{align*}
\lno\wh{\Sigma}-\Sigma\rno\hspace{-1.5mm}&\leqslant (C_0+2)^2\lambda_{\max}(\Sigma)\sqrt{\frac mn}~~\text{and}~~ \lno\wh{\Sigma}^{-1}-\Sigma^{-1}\rno\hspace{-1.5mm}\leqslant \frac{4(C_0+2)^2}{\lambda_{\min}(\Sigma)}\sqrt{\frac mn},
 \end{align*}
 with probability at least $1-2\exp(- C'm)$, where $C_0$ is defined in Lemma $\ref{lem:esgrm}$.
\end{lemma}
\begin{proof}
Let $\x_i=\Sigma^{\frac12}\m_i$ and $\bs{M}=[\bs m_1~\cdots~\bs m_n]$ where $\m_i$ is a centered sub-Gaussian random vector with covariance $\mathbf I_m$. Then one has 
\begin{align*}
\lno\wh\Sigma-\Sigma\rno&\leqslant\lno\Sigma^{\frac12}\rno\cdot\left\|\frac1n \M\M^\top-\mathbf I\right\|\cdot\lno\Sigma^{\frac12}\rno\\
&= \lambda_{\max}(\Sigma)\cdot\left[\lambda_{\max}\left(\frac1n \M\M^\top\right)-1\right]
\end{align*}
and 
\begin{align*}
\lno\wh{\Sigma}^{- 1}-\Sigma^{- 1}\rno
&\leqslant \lno\Sigma^{-\frac12}\rno\cdot\left\|\frac1n \M\M^\top-\mathbf I\right\|\cdot\lno\l\frac1n \M\M^\top\r^{-1}\rno\cdot\lno\Sigma^{-\frac12}\rno\\
&=\frac{1}{\lambda_{\min}(\Sigma)}\left[\lambda_{\max}\left(\frac1n \M\M^\top\right)-1\right]\cdot\lambda_{\min}\left(\frac1n \M\M^\top\right)^{-1}.
\end{align*}
By \eqref{equation, min max eval}, it is easy to check that
\begin{align*}&\lambda_{\max}\left(\frac1n \M\M^\top\right)-1\leqslant\left(1+\frac{(C_0+1)\sqrt m}{\sqrt n}\right)^2-1\leqslant\frac{(C_0+2)^2\sqrt m}{\sqrt n};\\
&\lambda_{\min}\left(\frac1n \M\M^\top\right)\geqslant \left(1-\frac{(C_0+1)\sqrt m}{\sqrt n}\right)^2\geqslant \frac14~\text{for}~n\geqslant [2(C_0+1)]^{\frac2{1-c_1}},
\end{align*}
with probability at least $1-2\exp(- C'm)$. Thus the proof is completed by choosing $n_1'(C_0,c_1):=[2(C_0+1)]^{\frac{2}{1-c_1}}$. 
\end{proof}

\section{Proof of Proposition \ref{prop:concentration Gammam dag Mmd}}\label{ap:concentration inequality}
We first give the following lemma whose proof is deferred to the end of this section.
\begin{lemma}\label{lem:PimTPimtoT}If $T$ is of finite rank, then we have $\lim\limits_{m\to \infty}\|\Pi_m T\Pi_m-T\| =0$.
\end{lemma}
A direct corollary of this lemma is as follows.
\begin{corollary}\label{lemma, M go to Mm}
%For any $\varepsilon>0$, one has $\|M-M_m\| <\varepsilon$ when $m$ is sufficiently large.
Under Assumptions $\ref{as:joint distribution assumption}$ and $\ref{as:Linearity condition and Coverage condition}$, we have $\lim\limits_{m\to\infty}\|M-M_m\| =0$.
\end{corollary}
\noindent We denote by $m_M(\varepsilon)$ the minimal integer $m_M$ satisfying $\|M-M_m\| \leqslant \varepsilon$ for all $m\geqslant m_M$.

Proposition $\ref{prop:concentration Gammam dag Mmd}$ is a direct corollary of the following Proposition.
\begin{proposition}
\label{prop:bound of finite estimate}
 Suppose that Assumptions $\ref{as:joint distribution assumption}$ to $\ref{assumption: rate-type condition}$ hold, then $\forall \gamma\in(0,1/2)$, there exist positive constants
 \begin{align*}
 n_1=n_1(\gamma,\sigma_0,\sigma_1,\bs K,m_M(1),c_1),\quad D_3=D_3(\|M\| ,\wt C,\bs K) 
 \end{align*}
and $C'=C'(\bs K)$
, such that when $n\geqslant n_1$, we have
\begin{equation*}
\begin{aligned}
\mb P\l \lno\widehat\Gamma_m^\dagger \widehat M_m^d-\Gamma_m^\dagger M_m\rno  \leqslant \left[\frac{C+\ln(D_1m^2n)}{D_2}\right]^{\frac52}\frac{24m^{\alpha_1+1}}{\wt C\sqrt n}+D_3\frac{m^{(2\alpha_1+1)/2}}{n^{1/2}} \r&\\
\geqslant 1-\exp(- C)-2\exp(- C'm).&
\end{aligned}
\end{equation*}
Here $D_1,D_2$ and $C$ are defined in Proposition $\ref{prop:bound hatMmd Mm}$ and $\bs K$ is the uniform upper-exponentially bounded constant of $(\sqrt{\lambda_1}w_1,\dots,\sqrt{\lambda_m}w_m)$. 
\end{proposition}
\begin{proof}
By triangle inequality, one has
\begin{align*}
&\lno\widehat{\Gamma}_m^\dagger \widehat M_m^d-\Gamma_m^\dagger M_m\rno 
=\lno\widehat\Gamma_m^\dagger \widehat M_m^d-\wh\Gamma_m^\dagger M_m+\wh\Gamma_m^\dagger M_m-\Gamma_m^\dagger M_m\rno 
\\&\qquad\leqslant\lno\Gamma_m^\dagger\rno \cdot \lno\widehat M_m^d-M_m\rno +\lno\widehat\Gamma_m^\dagger-\Gamma_m^\dagger\rno \cdot \lno M_m\rno .
\end{align*}
Thus one can bound $\lno\Gamma_m^{\dag}M_m-\widehat\Gamma_m^{\dag}\widehat M_m^d\rno $ by bound $\lno\Gamma_m^\dagger\rno $, $\lno\widehat\Gamma_m^\dagger-\Gamma_m^\dagger\rno $, $\lno\widehat M_m^d-M_m\rno $ and $\lno M_m\rno $ respectively.
\begin{itemize}
 \item\textbf{Bound of $\lno\Gamma_m^\dagger\rno $}: By Assumption $\ref{assumption: rate-type condition}$, one has 
\begin{align}\label{eq:bound Gammam dagger}
\lambda_j\geqslant \wt C j^{-\alpha_1}\Rightarrow\lno\Gamma_m^\dagger\rno =\lambda_m^{-1}\leqslant \wt{C}^{-1} m^{\alpha_1}. 
\end{align} 
 \item\textbf{Bound of $\lno\widehat\Gamma_m^\dagger-\Gamma_m^\dagger\rno $}:
 Let us define $\mc H_m:=\mathrm{span}\{\phi_1,\dots,\phi_m\}$ where $\{\phi_i\}$ is introduced in Equation $\eqref{eq:X expansion}$. It is easy to check that
$\lno\widehat\Gamma_m^\dagger-\Gamma_m^\dagger\rno =\lno(\widehat\Gamma_m^\dagger-\Gamma_m^\dagger)|_{\mc H_m}\rno $ since $\l\widehat\Gamma_m^\dagger-\Gamma_m^\dagger\r{\bs{\beta}}=0$ for any ${\bs{\beta}}\in\mc{H}_m^\perp$. 
Because $\l\widehat\Gamma_m^\dagger-\Gamma_m^\dagger\r|_{\mc H_m}$ can be represented by matrix $\widehat{\Sigma}^{-1}-\Sigma^{-1}$ defined in Lemma $\ref{lemma, estiamtion error of inverse sample cov}$ under orthonormal basis $\{\phi_i\}_{i=1}^m$, one can get $\lno\widehat\Gamma_m^\dagger-\Gamma_m^\dagger\rno =\|\widehat{\Sigma}^{-1}-\Sigma^{-1}\|$.
Similarly, one can also get $\lno\Gamma_m^\dagger\rno =\lno\Sigma^{-1}\rno=\lambda_{\min}^{-1}(\Sigma)$. Thus, by Lemma $\ref{lemma, estiamtion error of inverse sample cov}$ one has
\[\mb P\l\lno\widehat\Gamma_m^\dagger-\Gamma_m^\dagger\rno \leqslant {4(C_0+2)^2}\lno\Gamma^{\dag}_m\rno \sqrt{\frac mn}\r\geqslant 1-2\exp(- C'm)\]
for sufficiently large $n\geqslant n_1'(\bs K,c_1)$
. Combing with $\lno\Gamma_m^\dagger\rno \hspace{-1mm}\leqslant \wt{C}^{-1} m^{\alpha_1}$, one can get
\begin{equation}\label{eq: distance hat gamma m dagger hat gamma m dagger}
\mb P\l\lno\widehat\Gamma_m^\dagger-\Gamma_m^\dagger\rno \leqslant \frac{4(C_0+2)^2m^{(2\alpha_1+1)/2}}{\wt Cn^{1/2}}\r\geqslant 1-2\exp(- C'm)
\end{equation}
for sufficiently large $n\geqslant n_1'(\bs K,c_1)$.
 \item\textbf{Bound of $\lno\widehat M_m^d-M_m\rno $}:
 See Proposition $\ref{prop:bound hatMmd Mm}$.
 \item \textbf{Bound of $\lno M_m\rno $}: By Corollary $\ref{lemma, M go to Mm}$, $\|M-M_m\| \leqslant 1$ for sufficiently large $m\geqslant m_M(1)$. Then by triangle inequality, one can get
\[\|M_m\| -\|M\| \leqslant \|M-M_m\| \leqslant 1.\]
Hence,
\begin{align}\label{eq:Mm leq M C}
\|M_m\| \leqslant \|M\| +1.
\end{align}
\end{itemize}
Combing \eqref{eq:bound Gammam dagger}, \eqref{eq: distance hat gamma m dagger hat gamma m dagger}, Proposition $\ref{prop:bound hatMmd Mm}$ with \eqref{eq:Mm leq M C}, one can choose $D_3$ and $n_1$ to be $\frac{4(C_0+2)^2(\|M\| +1)}{\wt C}$ and $\max\{n_0,n_1'(\bs K,c_1),m_M(1)^{1/c_1}\}$ respectively to
complete the proof where $n_0$ is defined in Proposition $\ref{prop:bound hatMmd Mm}$.
\end{proof}

\paragraph{Proof of Lemma \ref{lem:PimTPimtoT}}
\begin{proof}By the triangle inequality and compatibility of operator norm, one has
\begin{align*}
\|\Pi_m T\Pi_m-T\| &\leqslant\|\Pi_mT\Pi_m-\Pi_mT\| +\|\Pi_mT-T\| \\
&\leqslant\|(\Pi_m-I)T^*\| +\|(\Pi_m-I)T\| 
\end{align*}
where $I=\sum\limits_{i=1}^\infty\phi_i\otimes\phi_i$ for $\{\phi_i\}_{i\in\mb{Z}_{\geqslant 1}}$ defined in \eqref{eq:X expansion} being an orthonormal basis of $\mc H$. 
% Since the adjoint of $M(\Pi_m-I)$ is $(\Pi_m-I)M$, we have
% \begin{align*}&\|M(\Pi_m-I)\| +\|(\Pi_m-I)M\| \\
% =&
% \end{align*}

Since $T$ is of finite rank, let us assume that $\{e_i\}_{i=1}^k$ is an orthonormal basis of $\mathrm{Im}(T)$ where $k=\mr{rank}(T)$. For any ${\bs{\beta}}\in\mathcal{H}$ such that $\|{\bs{\beta}}\|=1$, one has $\|T{\bs{\beta}}\|\leqslant\|T\| \|{\bs{\beta}}\|=\|T\| $, so one can assume that $T{\bs{\beta}}\in\mathrm{Im}(T)$ admits the following expansion under basis $\{e_i\}_{i=1}^k$:
\[T{\bs{\beta}}=\sum_{i=1}^k b_ie_i,\quad \sum_{i=1}^k b^2_i\leqslant\|T\| ^2<\infty.\]
Thus
\[\|(I-\Pi_m)T{\bs{\beta}}\|=\left\|\sum_{i=1}^k(I-\Pi_m) b_ie_i\right\|\leqslant\sum_{i=1}^k |b_i|\cdot\|(I-\Pi_m) e_i\|.\]
Clearly, $\|(\Pi_m-I)\alpha\|~(\forall\alpha\in\H)$ tends to $0$ as $m\to\infty$ since 
\[(I-\Pi_m)\alpha=\left(\sum_{i={m+1}}^\infty\phi_i\otimes\phi_i\right)\left(\sum\limits_{i=1}^\infty c_i\phi_i\right)=\sum_{i=m+1}^\infty c_i\phi_i\xrightarrow{m\to\infty} 0\]
where we have assumed that $\alpha=\sum\limits_{i=1}^\infty c_i\phi_i$ .

Thus $\forall\varepsilon>0$, there exists some $N_i>0$ such that $\forall m> N_i$ one has $\|(\Pi_m-I)e_i\|<\varepsilon$, $(\forall i=1,...,k)$. Let $N=\max\{N_1,\cdots,N_k\}$, then $\forall m>N$ one has
\[\|(I-\Pi_m)T{\bs{\beta}}\|\leqslant\sum_{i=1}^k |b_i|\cdot\|(I-\Pi_m) e_i\|\leqslant\sum_{i=1}^k |b_i|\varepsilon\leqslant k\varepsilon\|T\| ,\]
which means that $\forall m>N$, one has
\begin{align*}
\|(\Pi_m-I)T\| &=\sup_{\|{\bs{\beta}}\|=1}\|(\Pi_m-I)T{\bs{\beta}}\|\leqslant k\varepsilon\|T\| . 
\end{align*}
Thus $\lim\limits_{m\to\infty}\|(\Pi_m-I)T\| =0$. 

Similarly, one can also get $\lim\limits_{m\to\infty}\|(\Pi_m-I)T^*\| =0$. Then the proof of Lemma $\ref{lem:PimTPimtoT}$ is completed.
\end{proof}
% \section{Sin Theta Theorem}\label{ap:Sin Theta theorem}
% \subsection{Sin Theta Theorem for Self-adjoint Operators}
% \begin{lemma}[Proposition 2.3 in \cite{seelmann2014notes}]\label{lemma, sin theta of infinite dimension operator}
% Let $B$ be a self-adjoint operator on a separable Hilbert space $\widetilde{\mathcal{H}}$, and let ${V}\in\mathcal{L}(\widetilde{\mathcal{H}})$ be another self-adjoint operator where $\mathcal{L}\l\widetilde{\mc H}\r$ stands for the space of bounded linear operators from a Hilbert space $\widetilde{\mc H}$ to $\widetilde{\mc H}$.
% Write \[\mathrm{spec}( B)=\sigma\cup\Sigma\quad\text{and}\quad \mathrm{spec}( B+ V)=\omega\cup\Omega
% \]
% with $\sigma\cap\Sigma=\varnothing=\omega\cap\Omega$, and suppose that there is $\widehat d>0$ such that
% \[\mathrm{dist}(\sigma,\Omega)\geqslant \widehat d\quad\text{and}\quad\mathrm{dist}(\Sigma,\omega)\geqslant \wh d\]
% where $\mathrm dist(\sigma,\Sigma):=\min\{|a-b|:a\in\sigma,b\in\Omega\}$.
% Then, the operator angle $\Theta=\Theta(P_{ B}(\sigma),P_{ B+ V}(\omega))$ satisfies the bound
% \[\|\sin\Theta\|:=\|P_{{B}}(\sigma)-P_{{B}+{V}}(\omega)\| \leqslant\frac\pi2\frac{\| V\| }{\wh d}\]
% where $P_{ B}(\sigma)$ denotes the spectral projection for $ B$ associated with $\sigma$, i.e., 
% \[P_{B}(\sigma):=\frac{1}{2\pi\mathrm{i}}\oint_{\gamma}\frac{\mathrm{d}z}{z-B},\]
% where $\gamma$ is a contour on $\mathbb{C}$ that encloses $\sigma$ but no other elements of $\mathrm{spec}( B)$.
% \end{lemma}
% \begin{remark}
% We note that, 
% if further $ B$ is compact, 
% the spectral projection coincide with projection operator onto the closure of the space spanned by the eigenfunctions associated with the eigenvalues in $\sigma$.

% If $B$ is compact, by the spectral decomposition theorem one has
% \[B=\sum_{i=1}^\infty\mu_ie_i\otimes e_i\quad\text{and}\quad(z- B)^{-1}=\sum_{i=1}^\infty(z-\mu_i)^{-1}e_i\otimes e_i,\]
% where $\mr{spec}(B):=\{\mu_i\}_{i=1}^\infty$ satisfies $|\mu_i|\xrightarrow{i\to\infty} 0$.
% Then $\forall v\in \mathcal{H}$,
% \begin{align*}P_{B}(\sigma)v&=\frac{1}{2\pi\mathrm{i}}\oint_{\gamma}({z-B})^{-1}v~{\mathrm{d}z}=\frac{1}{2\pi\mathrm{i}}\oint_{\gamma}\sum_{i=1}^\infty(z-\mu_i)^{- 1}\langle e_i,v\rangle e_i~{\mathrm{d}z}\\
% &=\sum_{i=1}^\infty\left[\left(\frac{1}{2\pi\mathrm{i}}\oint_{\gamma}(z-\mu_i)^{-1}~{\mathrm{d}z}\right)\langle e_i,v\rangle e_i\right]=\sum_{i\in\{i:\mu_i\in\sigma\}}\langle e_i,v\rangle e_i.
% \end{align*}
% Especially, if $\sigma=\mr{spec}(B)\backslash\{0\}$, then $P_{B}(\sigma)$ is the projection operator onto the $\overline{\mathrm{Im}}(B)$.
% \end{remark}
% Splitting eigenvalues into nonzero part and zero part yields the following useful corollary.
% \begin{corollary}\label{cor: sin theta self adjoint}
% Let $B$ and $B'$ be two positive semi-definite {and compact} operators with finite rank on a separable Hilbert space $\widetilde{\mathcal{H}}$. Let $\lambda_{\min}^+( B)$ and $\lambda_{\min}^+(B')$ be the infimum of the positive eigenvalues of ${B}$ and ${B}'$ respectively. Then we have
% \[\left\|P_{ B}-P_{ B'}\right\| \leqslant\frac\pi2\frac{\| B- B'\| }{\min\{\lambda_{\min}^+( B),\lambda_{\min}^+( B')\}}.\]
% \end{corollary}
% \subsection{Sin Theta Theorem for General Operators}
% When ${B}$ and ${V}$ in Lemma $\ref{lemma, sin theta of infinite dimension operator}$ are not self-adjoint, we use the symmetrization trick, which mainly depends on the following Lemma.
% \begin{lemma}\label{lem:projection equality}
% $P_A=P_{AA^*}$ for any bounded linear operator $A$ from a Hilbert space $\wt\H$ to $\wt\H$. Especially, $P_A=P_{AA^{\top}}$ for any matrix $A$.
% \end{lemma}
% \begin{proof}This lemma is a direct corollary of Lemma $\ref{lem: colPBP equal colPB operator}$.
% \end{proof}

% Then we have the following Sin Theta theorem for general operator.
% \begin{lemma}\label{lemma, sin theta of nonadjoint operator}
% Let $ B,B'\in\mathcal{L}(\widetilde{\mathcal{H}})$ be two compact operators (not necessarily self-adjoint) with finite rank.
% Then we have
% \begin{align*}
% \left\|P_{ B}-P_{ B'}\right\| &\leqslant\frac\pi2\frac{\| B B^*- B'B'^*\| }{\min\lb\sigma_{\min}^+( B)^2,\sigma_{\min}^+(B')^2\rb}\\
% &\leqslant \frac\pi2\frac{\| B- B'\| ^2+2\| B- B'\| \| B'\| }{\min\lb\sigma_{\min}^+( B)^2,\sigma_{\min}^+( B')^2\rb}.
% \end{align*}
% \end{lemma}
% \begin{proof}By Lemma $\ref{lem:projection equality}$, one can get $\left\|P_{ B}-P_{ B'}\right\| =\left\|P_{ B B^*}-P_{ B' B'^*}\right\| $.
% Since $ BB^*, B'B'^*$ are both self-adjoint and compact, by Lemma $\ref{cor: sin theta self adjoint}$, one has
% \begin{align*}
% \left\|P_{ B B^*}-P_{ B' B'^*}\right\| \leqslant \frac{\pi}{2}\frac{\| B B^*- B' B'^*\| }{\min\lb\lambda_{\min}^+\l B B^*\r,\lambda_{\min}^+\l B' B'^*\r\rb}.
% \end{align*}
% Then the proof is completed in view of the following inequality:
% % of $\| B B^*- B' B'^*\| $:
% \begin{align}
% \lno B B^*- B' B'^*\rno &= \|( B- B')( B- B')^*\hspace{-0.5mm}+\hspace{-0.5mm}( B-B')(B')^*\hspace{-0.5mm}+\hspace{-0.5mm} B'( B- B')^*\| \nonumber\\
% &\leqslant \| B- B'\| ^2+2\| B- B'\| \| B'\| . \label{eq:sy ineq}
% \end{align}
% \end{proof}


\section{Sin Theta Theorem}\label{ap:Sin Theta theorem}
\subsection{Sin Theta Theorem for Self-adjoint Operators}
\begin{lemma}[Proposition 2.3 in \cite{seelmann2014notes}]\label{lemma, sin theta of infinite dimension operator}
Let $B$ be a self-adjoint operator on a separable Hilbert space $\widetilde{\mathcal{H}}$, and let ${V}\in\mathcal{L}(\widetilde{\mathcal{H}})$ be another self-adjoint operator where $\mathcal{L}\left(\widetilde{\mc H}\right)$ stands for the space of bounded linear operators from a Hilbert space $\widetilde{\mc H}$ to $\widetilde{\mc H}$.
Write the spectra of $B$ and $B+V$ as \[\mathrm{spec}( B)=\sigma\cup\Sigma\quad\text{and}\quad \mathrm{spec}( B+ V)=\omega\cup\Omega
\]
with $\sigma\cap\Sigma=\varnothing=\omega\cap\Omega$, and suppose that there is $\widehat d>0$ such that
\[\mathrm{dist}(\sigma,\Omega)\geqslant \widehat d\quad\text{and}\quad\mathrm{dist}(\Sigma,\omega)\geqslant \wh d\]
where $\mathrm dist(\sigma,\Sigma):=\min\{|a-b|:a\in\sigma,b\in\Omega\}$.
Then it holds that
\[\|P_{{B}}(\sigma)-P_{{B}+{V}}(\omega)\| \leqslant\frac\pi2\frac{\| V\| }{\wh d}\]
where $P_{ B}(\sigma)$ denotes the spectral projection for $ B$ associated with $\sigma$, i.e., 
\[P_{B}(\sigma):=\frac{1}{2\pi\mathrm{i}}\oint_{\gamma}\frac{\mathrm{d}z}{z-B},\]
where $\gamma$ is a contour on $\mathbb{C}$ that encloses $\sigma$ but no other elements of $\mathrm{spec}( B)$.
\end{lemma}
\begin{remark}
We note that, 
if further $ B$ is compact, 
the spectral projection coincide with projection operator onto the closure of the space spanned by the eigenfunctions associated with the eigenvalues in $\sigma$. 
% For more details, see, e.g., Remark 1 in \cite{chen2023optimality}.

Specifically, if $B$ is compact, by the spectral decomposition theorem one has
\[B=\sum_{i=1}^\infty\mu_ie_i\otimes e_i\quad\text{and}\quad(z- B)^{-1}=\sum_{i=1}^\infty(z-\mu_i)^{-1}e_i\otimes e_i,\]
where $\mr{spec}(B):=\{\mu_i\}_{i=1}^\infty$ satisfies $|\mu_i|\xrightarrow{i\to\infty} 0$.
Then $\forall v\in \mathcal{H}$, it holds that
\begin{align*}P_{B}(\sigma)v&=\frac{1}{2\pi\mathrm{i}}\oint_{\gamma}({z-B})^{-1}v~{\mathrm{d}z}=\frac{1}{2\pi\mathrm{i}}\oint_{\gamma}\sum_{i=1}^\infty(z-\mu_i)^{- 1}\langle e_i,v\rangle e_i~{\mathrm{d}z}\\
&=\sum_{i=1}^\infty\left[\left(\frac{1}{2\pi\mathrm{i}}\oint_{\gamma}(z-\mu_i)^{-1}~{\mathrm{d}z}\right)\langle e_i,v\rangle e_i\right]=\sum_{i\in\{i:\mu_i\in\sigma\}}\langle e_i,v\rangle e_i.
\end{align*}
In particular, if $\sigma=\mr{spec}(B)\backslash\{0\}$, then $P_{B}(\sigma)$ is the projection operator onto the $\overline{\mathrm{Im}}(B)$.
\end{remark}

Splitting eigenvalues into nonzero part and zero part yields the following useful corollary.
\begin{corollary}\label{cor: sin theta self adjoint}
Let $B$ and $B'$ be two positive semi-definite {and compact} operators with finite rank on a separable Hilbert space $\widetilde{\mathcal{H}}$. Let $\lambda_{\min}^+( B)$ and $\lambda_{\min}^+(B')$ be the infimum of the positive eigenvalues of ${B}$ and ${B}'$ respectively. Then we have
\[\left\|P_{ B}-P_{ B'}\right\| \leqslant\frac\pi2\frac{\| B- B'\| }{\min\{\lambda_{\min}^+( B),\lambda_{\min}^+( B')\}}.\]
\end{corollary}
\subsection{Sin Theta Theorem for General Operators}
When ${B}$ and ${V}$ in Lemma $\ref{lemma, sin theta of infinite dimension operator}$ are not self-adjoint, we use the symmetrization trick, which mainly depends on the following Lemma.
\begin{lemma}\label{lem:projection equality}
$P_A=P_{AA^*}$ for any bounded linear operator $A$ from a Hilbert space $\wt\H$ to $\wt\H$. Especially, $P_A=P_{AA^{\top}}$ for any matrix $A$.
\end{lemma}
\begin{proof}First we show that the null space of  $A^*$ is the same as the null space of $AA^*$.
On the one hand, 
\[x\in\mathrm{null}(A^*)\Longrightarrow
A^*x=0\Longrightarrow AA^*x=0\Longrightarrow x\in\mathrm{null}(AA^*); 
\]
One the other hand,
\begin{align*}x\in\mathrm{null}(AA^*)&\Longrightarrow
AA^*x=0\Longrightarrow \langle x,AA^*x\rangle=\langle A^*x,A^*x\rangle=\|A^*x\|^2=0\\
&\Longrightarrow A^*x=0\Longrightarrow x\in\mathrm{null}(A^*).
\end{align*}
Hence, we have $\mathrm{null}(A^*)=\mathrm{null}(AA^*)$. Take the orthogonal complement of the both sides of this equality, we can get
\[\mathrm{null}(A^*)^{\perp}=\mathrm{null}(AA^*)^{\perp}\Longrightarrow {\mathrm{Im}(A)}={\mathrm{Im}(AA^*)}.\]
\end{proof}
Then we have the following Sin Theta theorem for general operator.
\begin{lemma}\label{lemma, sin theta of nonadjoint operator}
Let $ B,B'\in\mathcal{L}(\widetilde{\mathcal{H}})$ be two compact operators (not necessarily self-adjoint) with finite rank.
Then we have
\begin{align*}
\left\|P_{ B}-P_{ B'}\right\| &\leqslant\frac\pi2\frac{\| B B^*- B'B'^*\| }{\min\left\{\sigma_{\min}^+( B)^2,\sigma_{\min}^+(B')^2\right\}}\\
&\leqslant \frac\pi2\frac{\| B- B'\| ^2+2\| B- B'\| \| B'\| }{\min\left\{\sigma_{\min}^+( B)^2,\sigma_{\min}^+( B')^2\right\}}.
\end{align*}
\end{lemma}
\begin{proof}By Lemma $\ref{lem:projection equality}$, one can get $\left\|P_{ B}-P_{ B'}\right\| =\left\|P_{ B B^*}-P_{ B' B'^*}\right\| $.
Since $ BB^*, B'B'^*$ are both self-adjoint and compact, by Lemma $\ref{cor: sin theta self adjoint}$, one has
\begin{align*}
\left\|P_{ B B^*}-P_{ B' B'^*}\right\| \leqslant \frac{\pi}{2}\frac{\| B B^*- B' B'^*\| }{\min\left\{\lambda_{\min}^+\left( B B^*\right),\lambda_{\min}^+\left( B' B'^*\right)\right\}}.
\end{align*}
Then the proof is completed in view of the following inequality:
% of $\| B B^*- B' B'^*\| $:
\begin{align}
\left\| B B^*- B' B'^*\right\| &= \|( B- B')( B- B')^*\hspace{-0.5mm}+\hspace{-0.5mm}( B-B')(B')^*\hspace{-0.5mm}+\hspace{-0.5mm} B'( B- B')^*\| \nonumber\\
&\leqslant \| B- B'\| ^2+2\| B- B'\| \| B'\| . \label{eq:sy ineq}
\end{align}
\end{proof}


\section{Proof of Theorem \ref{theorem, total convergence rate}}
Thanks to the triangle inequality, one can bound the subspace estimation error by bounding the error term (i): $\mathbf{ Loss}_1:=\left\|P_{\mc S_{\Y|\X}^{(m)}}-P_{ \widehat {\mc S}_{\Y|\X}^{(m)}}\right\| $ and error term (ii): $\mathbf{ Loss}_2:= \left\|P_{\mathcal S_{\Y|\boldsymbol{X}}}-P_{\mathcal S_{\Y|\boldsymbol{X}}^{(m)}}\right\| $ respectively.
\subsection{Upper bound of error term (i)}
We first give the following lemmas, whose proofs are all deferred to the end of this section.
\begin{lemma}\label{lem:Gammam dagger Mm uniformly bounded}
% Under Assumptions $\ref{as:joint distribution assumption}$ and $\ref{as:Linearity condition and Coverage condition}$,
% $\{\|\Gamma_m^\dagger M_m\| \}_{m=1}^\infty$ is uniformly (about $m$) bounded by $\|\Gamma^{-1}M\| $.
Under Assumptions $\ref{as:joint distribution assumption}$ and $\ref{as:Linearity condition and Coverage condition}$, it holds that $\|\Gamma_m^\dagger M_m\| \leq \|\Gamma^{-1}M\| (\forall m).$
% \begin{align*}
% \|\Gamma_m^\dagger M_m\| \leq \|\Gamma^{-1}M\| \quad\forall m.
% \end{align*}
% $\{\|\Gamma_m^\dagger M_m\| \}_{m=1}^\infty$ is uniformly (about $m$) bounded by $\|\Gamma^{-1}M\| $.
\end{lemma}
\begin{lemma}\label{lem: Gamma inverse M to Gammam dagger Mm}Under Assumptions $\ref{as:joint distribution assumption}$ and $\ref{as:Linearity condition and Coverage condition}$, we have \[\lim\limits_{m\to\infty}\lno\Gamma^{-1}M-\Gamma_m^\dagger M_m\rno =0.\]
\end{lemma}
\noindent We denote by $m_T(\varepsilon)$ the minimal integer $m_T$ satisfying $\lno\Gamma^{- 1}M-\Gamma_m^\dagger M_m\rno \hspace{-1mm}\leqslant \varepsilon$ for all $m\geqslant m_T$ and define an event 
$$\ttE:=\lb \left\|\widehat\Gamma_m^\dagger \widehat M_m^d-\Gamma_m^\dagger M_m\right\|  \leqslant\hspace{-0.5mm}\left(\tfrac{D_0+1}{D_2}\right)^{\frac52}\tfrac{24}{\wt C}n^{c_1(\alpha_1+1)+\gamma-\frac{1}{2}}+D_3n^{\frac{c_1(2\alpha_1+1)-1}{2}}\rb.$$
Then by taking $C$ to be $(D_0+1)n^{\frac{2\gamma}{5}}-\ln\l D_1m^2n \r$ in  Proposition \ref{prop:bound of finite estimate}, one has: for $n\geqslant \l\frac{D_0+1}{D_2}\r^{\frac{5}{1-2\gamma}}$,
$$\P(\ttE)\geq 1-D_1m^2n\exp\left[-(D_0+1)n^{\frac{2\gamma}{5}}\right] -2\exp(- C'm).$$
\begin{lemma}\label{lem:lower bound sigma min total}
Introducing $
\bigtriangleup :=\max\lb \frac{\sigma_d(\Gamma^{-1} M)}{2},\frac{\sigma_d(\Gamma^{-1} M)^2}{4\|\Gamma^{-1}M\| } \rb$.
Suppose that Assumptions $\ref{as:joint distribution assumption}$ to $\ref{assumption: rate-type condition}$ hold, $c_1(2\alpha_1+1)-1<0$ and $2(c_1(\alpha_1+1)+\gamma)-1<0$. Then there exists a positive constant
\begin{align*}
n_2'=n_2'\l\sigma_d(\Gamma^{-1}M),\|\Gamma^{-1}M\| , \gamma,\sigma_0,\sigma_1,\bs K,m_M(1),c_1,m_T\l \tfrac{\bigtriangleup}{2}\r,\wt C,\alpha_1\r
\end{align*}
such that when $n\geqslant n_2'$, we have
\begin{align}
\sigma_{\min}^+(\Gamma_m^{\dagger} M_m)^2\geqslant \tfrac{\sigma_d(\Gamma^{-1}M)^2}{2} \label{eq: lower bound of sigma min}. 
\end{align}
Furthermore, Conditioning on $\ttE$, we have
\begin{align}\label{eq: lower bound of sigma min hat}
&\sigma_{\min}^+(\wh\Gamma_m^{\dagger}\wh M_m^d)^2\geqslant \tfrac{\sigma_d(\Gamma^{-1}M)^2}{2}.
\end{align}
\end{lemma}
The following proposition is an upper bound of error term (i):
\begin{proposition}\label{proposition, estimation error}
Positive constants $D_1$, $D_2$ and $C'$  as in Proposition $\ref{prop:bound of finite estimate}$,
suppose that Assumptions $\ref{as:joint distribution assumption}$ to $\ref{assumption: rate-type condition}$ hold, then $\forall \gamma\in(0,1/2)$, if $c_1$ satisfies $2c_1(\alpha_1+1)+2\gamma-1<0$ and $c_1(2\alpha_1+1)-1<0$, there exists a positive constant $C_1:=C_1\l \|\Gamma^{-1}M\| ,\sigma_d(\Gamma^{-1}M) ,\wt C,\gamma,\sigma_0,\sigma_1\r$ such that
\begin{align*}
\P\l
\lno P_{\mc{S}_{\Y|\X}^{(m)}}-P_{ \widehat{\mc{S}}_{\Y|\X}^{(m)}}\rno \leqslant C_1\frac{m^{\alpha_1+1}}{n^{1/2-\gamma}}\r\geqslant1-2\exp(- C'm)&\\
- D_1m^2n\exp\l -(D_0+1)n^{\frac{2\gamma}{5}} \r&,
\end{align*}
when 
\begin{align*}
n\geqslant\max\Bigg\{ n_1,\l\tfrac{D_0+1}{D_2}\r^{\frac{5}{1-2\gamma}},\left[\tfrac{\|\Gamma^{-1}M\|  \wt C}{48}\l\tfrac{D_2}{D_0+1}\r^{\frac52}\right]^{\frac{2}{2(c_1(\alpha_1+1)+\gamma)-1}}&,\\
\l \tfrac{\|\Gamma^{-1}M\| }{2D_3}\r^{\frac{2}{c_1(2\alpha_1+1)-1}},n_2',\left[ \tfrac{D_3\wt C}{24}\l \tfrac{D_2}{D_0+1} \r^{\frac52} \right]^{\frac2{2\gamma+c_1}}&\Bigg\}
\end{align*}
where $n_2'$ is defined in Lemma $\ref{lem:lower bound sigma min total}$.
\end{proposition}
\begin{proof}
By Lemma $\ref{lemma, way of estimate truncate central subspace}$, $\eqref{def: estimator central subspace}$ and Lemma $\ref{lemma, sin theta of nonadjoint operator}$, one has
\begin{align}
&\left\|P_{\mc S_{\Y|\vX}^{(m)}}-P_{\wh{\mc{S}}_{\Y|\vX}^{(m)}}\right\| =\left\|P_{\Gamma_m^{\dagger}M_m}-P_{\wh\Gamma_m^{\dagger}\wh M_m^d}\right\| \nonumber\\
&\qquad\leqslant\frac{\pi}{2}\frac{\lno\widehat\Gamma_m^\dagger \widehat M_m^d-\Gamma_m^\dagger M_m\rno ^2+\lno\widehat\Gamma_m^\dagger \widehat M_m^d-\Gamma_m^\dagger M_m\rno \lno\Gamma_m^\dagger M_m\rno }{\min\lb\sigma_{\min}^+\l\wh\Gamma_m^\dagger \wh M_m^d\r^2,\sigma_{\min}^+\l\Gamma_m^\dagger M_m\r^2\rb}\label{eq: PS minus P hat S norm}.
% &\leqslant C_5\|\widehat\Gamma_m^\dagger \widehat M_m^d-\Gamma_m^\dagger M_m\|\\
% &=\widetilde O_{\mathbb{P}}\l\frac{m^{\alpha_1+1}}{n^{1/2}}\r,
\end{align}
% with probability at least $1-\exp(- C)-2\exp(- C'm)$.
Because of $c_1(2\alpha_1+1)-1<0$ and $2(c_1(\alpha_1+1)+\gamma)-1<0$, it is easy to check that when
\[n\geqslant\max\lb\left[\tfrac{\|\Gamma^{-1}M\|  \wt C}{48}\l\tfrac{D_2}{D_0+1}\r^{\frac52}\right]^{\frac{2}{2(c_1(\alpha_1+1)+\gamma)-1}},\l \tfrac{\|\Gamma^{-1}M\| }{2D_3}\r^{\frac{2}{c_1(2\alpha_1+1)-1}}\rb,\]
both $\l\tfrac{D_0+1}{D_2}\r^{\frac52}\tfrac{24}{\wt C}n^{c_1(\alpha_1+1)+\gamma-\frac{1}{2}}$ and $D_3n^{\frac{c_1(2\alpha_1+1)-1}{2}}$ are less than or equal to $\frac{\|\Gamma^{-1}M\| }{2}$. Thus, on the event $\ttE$,
\begin{align}\label{eq: high prob upper bound is Gamma minus 1 M}
\lno\widehat\Gamma_m^\dagger \widehat M_m^d-\Gamma_m^\dagger M_m\rno \leqslant \lno\Gamma^{-1}M\rno .
\end{align}
By Lemma $\ref{lem:Gammam dagger Mm uniformly bounded}$, inserting \eqref{eq: high prob upper bound is Gamma minus 1 M} into \eqref{eq: PS minus P hat S norm} leads to
$$
\lno P_{\mc{S}_{\Y|\X}^{(m)}}-P_{ \widehat{\mc{S}}_{\Y|\X}^{(m)}}\rno
\leqslant \frac{\pi\lno\widehat\Gamma_m^\dagger \widehat M_m^d-\Gamma_m^\dagger M_m\rno \lno\Gamma^{-1}M\rno }{\min\lb\sigma_{\min}^+\l\wh\Gamma_m^\dagger \wh M_m^d\r^2,\sigma_{\min}^+\l\Gamma_m^\dagger M_m\r^2\rb},
$$
on the event $\ttE$.
Furthermore, when $n\geqslant \left[ \frac{D_3\wt C}{24}\l \frac{D_2}{D_0+1} \r^{\frac52} \right]^{\frac2{2\gamma+c_1}}$ and $n\geq n_2'$, one can get
$\l \tfrac{D_0+1}{D_2}\r^{\frac52}\tfrac{24m^{\alpha_1+1}}{\wt C n^{1/2-\gamma}}$ is greater than or equal to $D_3\tfrac{m^{(2\alpha_1+1)/2}}{n^{1/2}}$
and then on the event $\ttE$,
\begin{align*}
\lno P_{\mc{S}_{\Y|\X}^{(m)}}-P_{ \widehat{\mc{S}}_{\Y|\X}^{(m)}}\rno \leqslant \tfrac{96\pi\|\Gamma^{-1}M\| }{\sigma_d(\Gamma^{-1}M)^2}\l \tfrac{D_0+1}{D_2}\r^{\frac52}\tfrac{m^{\alpha_1+1}}{\wt C n^{1/2-\gamma}}.
\end{align*}
 by
Lemma $\ref{lem:lower bound sigma min total}$.
Then choosing $C_1=\tfrac{96\pi\|\Gamma^{-1}M\| }{\wt C\sigma_d(\Gamma^{-1}M)^2}\l \tfrac{D_0+1}{D_2}\r^{\frac52}$ can complete the proof.
\end{proof}



\paragraph{Proof of Lemma \ref{lem:Gammam dagger Mm uniformly bounded}}
\begin{proof}
First, it is easy to check that:
\begin{align}
\Gamma^\dag_m=\Pi_m\Gamma^{-1}\Pi_m=\Pi_m\Gamma^{-1}=\Gamma^{-1}\Pi_m=\sum\limits_{i=1}^m\lambda_i^{-1}\phi_i\otimes\phi_i.\label{eq: Gamma m dag def}
\end{align}
According to \eqref{eq: Gamma m dag def} and $M_m=\Pi_mM\Pi_m$, it is easy to check that $\Gamma_m^\dagger M_m=\Pi_m \Gamma^{- 1}M\Pi_m$. Then by the compatibility of operator norm, one can get
\begin{align*}
\lno\Gamma_m^\dagger M_m\rno =\lno\Pi_m \Gamma^{-1}M\Pi_m\rno \leqslant \lno\Pi_m\rno  \lno\Gamma^{-1}M\rno \lno\Pi_m\rno =\lno\Gamma^{-1}M\rno .
\end{align*}
Note that $\Gamma^{-1}M$ is bounded since $\Gamma^{-1}M$ is of finite rank by Corollary $\ref{corollary, MDDO and central subspace}$. Thus the proof is completed. 
\end{proof}


\paragraph{Proof of Lemma \ref{lem: Gamma inverse M to Gammam dagger Mm}}
\begin{proof}
It is easy to check that
$\Gamma_m^\dagger M_m=\Pi_m\Gamma^{-1}M\Pi_m$ and $\Gamma^{-1}M$ is of finite rank by Corollary $\ref{corollary, MDDO and central subspace}$.
Thus the proof is completed by Lemma $\ref{lem:PimTPimtoT}$.
\end{proof}
\paragraph{Proof of Lemma \ref{lem:lower bound sigma min total}}
\begin{proof}
We first prove \eqref{eq: lower bound of sigma min}.
By Corollary $\ref{corollary, MDDO and central subspace}$ and Lemma $\ref{lem:projection equality}$, one has $\rank(\Gamma^{- 1}M)=\rank\l\Gamma^{- 1}M(\Gamma^{- 1}M)^*\r=d$. Thus
\begin{align*}
\sigma_{\min}^+(\Gamma^{-1}M)^2=\lambda_{\min}^+\l\Gamma^{-1}M(\Gamma^{-1}M)^*\r=\lambda_d\l \Gamma^{-1}M(\Gamma^{-1}M)^*\r. 
\end{align*}
 It is easy to see $\rank(\Gamma_m^\dagger M_m)=\rank\l \Gamma_m^\dagger M_m(\Gamma_m^\dagger M_m)^*\r\leqslant d$ by $\Gamma_m^\dagger M_m=\Pi_m \Gamma^{-1} M \Pi_m$ and Lemma $\ref{lem:projection equality}$, thus one can assume that 
 \begin{align*}
\sigma_{\min}^+(\Gamma^\dagger_m M_m)^2=\lambda_{\min}^+\l\Gamma_m^\dagger M_m(\Gamma_m^\dagger M_m)^*\r=\lambda_j\l \Gamma_m^\dagger M_m(\Gamma_m^\dagger M_m)^*\r
\end{align*}
for some $j\leqslant d$.
By Corollary $\ref{coro:wely ineq operator}$, $\eqref{eq:sy ineq}$ and
% (Notice that $M_m$ and $M$ are both compact and self-adjoint)
Lemma $\ref{lem: Gamma inverse M to Gammam dagger Mm}$
%and Lemma \ref{lem:Gammam dagger Mm uniformly bounded}
, one has
\begin{align*}
&\left|\sigma_{\min}^+(\Gamma^\dagger_m M_m)^2\hspace{-0.5mm}-\hspace{-0.5mm}\sigma_j(\Gamma^{-1} M)^2\right|\hspace{-0.5mm}=\hspace{-0.5mm}\left|\lambda_{j}\hspace{-1mm}\l\Gamma^\dagger_m M_m(\Gamma^\dagger_m M_m)^{*}\hspace{-0.5mm}\r\hspace{-0.5mm}-\hspace{-0.5mm}\lambda_j\hspace{-1mm}\l \Gamma^{-1} M(\Gamma^{-1} M)^*\hspace{-0.5mm}\r\right|\\
&\qquad\leqslant
\|\Gamma^{-1} M(\Gamma^{-1} M)^*- \Gamma_m^\dagger M_m(\Gamma_m^\dagger M_m)^*\| \\
&\qquad\leqslant \|\Gamma^{-1} M- \Gamma_m^\dagger M_m\| ^2+
\|\Gamma^{-1} M- \Gamma_m^\dagger M_m\| \cdot\|\Gamma^{-1} M\| \xrightarrow{m\to\infty} 0. 
% &\leqslant\|\Gamma^{-1} M- \Gamma_m^\dagger M_m\|\cdot3\|\Gamma^{-1} M\|
\end{align*}
Thus for 
$
n\geqslant m_T(\bigtriangleup)^{\frac1{c_1}}=m_T\l\max\lb\frac{\sigma_d(\Gamma^{-1} M)}{2},\frac{\sigma_d(\Gamma^{-1} M)^2}{4\|\Gamma^{-1}M\| }\rb\r^{\frac1{c_1}}, 
$
one has $\|\Gamma^{-1} M- \Gamma_m^\dagger M_m\| ^2$ and $\|\Gamma^{-1} M- \Gamma_m^\dagger M_m\| \cdot\|\Gamma^{-1} M\| $ are both less than or equal to $\frac{1}{4}\sigma_d(\Gamma^{-1} M)^2$. Hence one can get
$\left|\sigma_{\min}^+(\Gamma^\dagger_m M_m)^2-\sigma_j(\Gamma^{-1} M)^2\right|\leqslant\frac{1}{2}\sigma_d(\Gamma^{-1} M)^2$
% \begin{align*}\label{eq:sigma min Mm}
% \left|\sigma_{\min}^+(\Gamma^\dagger_m M_m)^2-\sigma_j(\Gamma^{-1} M)^2\right|\leqslant\frac{1}{2}\sigma_d(\Gamma^{-1} M)^2
% \|
% \lambda_j\l \Gamma_m^\dagger M_m\l\Gamma_m^\dagger M_m\r^*\r\geqslant \lambda_j\l \Gamma^{-1} M\l\Gamma^{-1} M\r^*\r-\frac{\lambda_d\l \Gamma^{-1} M\l\Gamma^{-1} M\r^*\r}{2}
% \geqslant\frac{\lambda_d\l \Gamma^{-1} M\l\Gamma^{-1} M\r^*\r}{2}. 
% \end{align*}
and
\begin{equation}
\sigma_{\min}^+(\Gamma^\dagger M_m)^2\geqslant \sigma_j(\Gamma^{-1} M)^2-\frac{1}{2}\sigma_d(\Gamma^{-1} M)^2\geqslant\frac{1}{2}\sigma_d(\Gamma^{-1} M)^2
\end{equation}
for sufficiently large $n$. This completes the proof of \eqref{eq: lower bound of sigma min}.





Next we prove $\eqref{eq: lower bound of sigma min hat}$. Combining Proposition $\ref{prop:bound of finite estimate}$ with Lemma $\ref{lem: Gamma inverse M to Gammam dagger Mm}$ leads to that on the event $\ttE$, 
$$
\lno\wh \Gamma_m^\dag\wh M^d_m- \Gamma^{-1}M\rno \leqslant\ve+\l\tfrac{D_0+1}{D_2}\r^{\frac52}\tfrac{24}{\wt C}n^{c_1(\alpha_1+1)+\gamma-\frac{1}{2}}+D_3n^{\frac{c_1(2\alpha_1+1)-1}{2}}
$$
for  $n\geqslant \max\{n_1,m_T( \ve)^{1/c_1}\}$.
Assuming that $c_1(2\alpha_1+1)-1<0$ and $2(c_1(\alpha_1+1)+\gamma)-1<0$, it is easy to check that when $$n\geqslant\max\lb\left[\frac{\bigtriangleup \wt C}{96}\l\frac{D_2}{D_0+1}\r^{\frac52}\right]^{\frac{2}{2(c_1(\alpha_1+1)+\gamma)-1}},\l \frac{\bigtriangleup}{4D_3}\r^{\frac{2}{c_1(2\alpha_1+1)-1}}\rb$$, both $\l\tfrac{D_0+1}{D_2}\r^{\frac52}\tfrac{24}{\wt C}n^{c_1(\alpha_1+1)+\gamma-\frac{1}{2}}$ and $D_3n^{\frac{c_1(2\alpha_1+1)-1}{2}}$ are less than or equal to $\frac{\bigtriangleup}{4}$. Letting $\varepsilon=\frac12\bigtriangleup$, one can get on the event $\ttE$,
when
\begin{align*}
n&\geqslant n_2'=n_2'\hspace{-0.5mm}\l\hspace{-0.5mm}\sigma_d(\Gamma^{-1}M),\|\Gamma^{-1}M\| , \gamma,\sigma_0,\sigma_1,\bs K,m_M(1),c_1,m_T\l \tfrac{\bigtriangleup}{2}\r,\wt C,\alpha_1\hspace{-0.5mm}\r\\
&:=\max\bigg\{ n_1,m_T\l \tfrac{\bigtriangleup}{2}\r^{1/c_1}, \left[\tfrac{\bigtriangleup \wt C}{96}\l\tfrac{D_2}{D_0+1}\r^{\frac52}\right]^{\frac{2}{2(c_1(\alpha_1+1)+\gamma)-1}},\l \tfrac{\bigtriangleup}{4D_3}\r^{\frac{2}{c_1(2\alpha_1+1)-1}}\bigg\},
\end{align*}
one has $\lno\wh \Gamma_m^\dag\wh M^d_m- \Gamma^{-1}M\rno \leqslant\bigtriangleup$ and further
$\sigma_{\min}^+(\wh\Gamma^\dagger \wh M^d_m)^2\hspace{-1mm}\geqslant\hspace{-1mm} \tfrac{\sigma_d(\Gamma^{-1} M)^2}{2}$ by the same argument as the proof of \eqref{eq: lower bound of sigma min}.
 This completes the proof of \eqref{eq: lower bound of sigma min hat}.
Considering that $m_T(\bigtriangleup)\leqslant m_{T}\l\frac\bigtriangleup2\r$, one can also get $\eqref{eq: lower bound of sigma min}$ when $n\geqslant n_2'$. Thus the proof is completed.
\end{proof}
\subsection{Upper bound of error term (ii)}\label{ap, subs, truncation error}
\begin{proposition}\label{proposition, truncation error}
Under Assumption $\ref{assumption: rate-type condition}$, there exists a positive constant $C_2:=C_2\l d,\wt C,\lambda_d(\mc{B}),\alpha_2\r$ where $\mc{B}:=\sum\limits_{i=1}^d {\bs{\beta}}_i\otimes{\bs{\beta}}_i$ for ${\bs{\beta}}_i$ defined in \eqref{def: central subspace}, such that when $n\geqslant \l \frac{\lambda_d({\mc{B}})}{4d\wt C^2}\sqrt{\frac{2\alpha_2-1}{\zeta(2\alpha_2)}}\r^{\frac{2}{c_1(1-2\alpha_2)}}$, we have
\begin{equation}\label{equation, truncation error}
 \left\|P_{\mathcal S_{\Y|\boldsymbol{X}}}-P_{\mathcal S_{\Y|\boldsymbol{X}}^{(m)}}\right\| \leqslant C_2m^{-\frac{2\alpha_2-1}{2}},
\end{equation}
where $\zeta(\cdot)$ is Riemann $\zeta$ function.
% \begin{equation}
% \|P_{\mathcal S_{Y|\boldsymbol{X}}}-P_{\mathcal S_{Y|\boldsymbol{X}}^{(m)}}\|\leqslant O_{\mathbb{P}}(dn^{-(\alpha_2-1)/(2\alpha_1+\alpha_2)}) 
% \end{equation}
\end{proposition}
\begin{proof}
Let ${\mc{B}^{(m)}}:=\sum\limits_{i=1}^d {\bs{\beta}}_i^{(m)}\otimes{\bs{\beta}}_i^{(m)}$ for ${\bs{\beta}}_i^{(m)}$ defined in \eqref{def: truncated central subspace}.
Combing with Equation $\eqref{def: central subspace}$, it is easy to check that $\left\|P_{\mathcal S_{\Y|\boldsymbol{X}}}-P_{\mathcal S_{\Y|\boldsymbol{X}}^{(m)}}\right\| =\|P_{\mc{B}}-P_{\mc{B}^{(m)}}\| $. By Corollary $\ref{cor: sin theta self adjoint}$, we have
\begin{align}\label{eq:sin theta for B Bm}
\|P_{\mc{B}}-P_{\mc{B}^{(m)}}\| \leqslant \frac{\pi}{2}\frac{\|{\mc{B}}-{\mc{B}^{(m)}}\| }{\min\{\lambda_{\min}^+({\mc{B}}),\lambda_{\min}^+({\mc{B}^{(m)}})\}}.
\end{align}

Note that ${\mc{B}}-{\mc{B}^{(m)}}$ is self-adjoint, then
\begin{align*}
&\lno{\mc{B}}-{\mc{B}^{(m)}}\rno =\sup_{{\bs{\beta}}\in\mathbb{S}_{ \mathcal H}}|\langle ({\mc{B}}-{\mc{B}^{(m)}})({\bs{\beta}}),{\bs{\beta}}\rangle|=\sup_{{\bs{\beta}}\in\mathbb{S}_{\mathcal H}}|\langle {\mc{B}}{\bs{\beta}},{\bs{\beta}}\rangle-\langle {\mc{B}^{(m)}}{\bs{\beta}},{\bs{\beta}}\rangle|\\
&~~=\sup_{{\bs{\beta}}\in\mathbb{S}_{\mathcal H}}\hspace{-0.9mm}\left|\sum_{i=1}^d\hspace{-0.9mm}\left[\langle{\bs{\beta}}_i,{\bs{\beta}}\rangle^2-\langle{\bs{\beta}}_i^{(m)},{\bs{\beta}}\rangle^2\right]\right|=\sup_{{\bs{\beta}}\in\mathbb{S}_{\mathcal H}}\hspace{-0.9mm}\left| \sum_{i=1}^d\langle{\bs{\beta}}_i-{\bs{\beta}}_i^{(m)},{\bs{\beta}}\rangle\langle{\bs{\beta}}_i+{\bs{\beta}}_i^{(m)},{\bs{\beta}}\rangle\right|\\
&~~\leqslant\sup_{{\bs{\beta}}\in\mathbb{S}_{\mathcal H}}\sum_{i=1}^d\left| \langle{\bs{\beta}}_i-{\bs{\beta}}_i^{(m)},{\bs{\beta}}\rangle\langle{\bs{\beta}}_i+{\bs{\beta}}_i^{(m)},{\bs{\beta}}\rangle\right|
\leqslant\sum_{i=1}^d\left\|{\bs{\beta}}_i-{\bs{\beta}}_i^{(m)}\right\|\left\|{\bs{\beta}}_i+{\bs{\beta}}_i^{(m)}\right\|,
\end{align*}
where the first inequality comes from the triangle inequality, and the 
second inequality comes from the Cauchy-Schwarz inequality and $\|{\bs{\beta}}\|=1$. 
 Then one has ${\bs{\beta}}_i=\sum\limits_{j=1}^\infty b_{ij}\phi_j$ and 
\[{\bs{\beta}}^{(m)}_i=\Pi_m{\bs{\beta}}_i=\sum_{j'=1}^m\phi_{j'}\otimes\phi_{j'}\sum_{j=1}^\infty b_{ij}\phi_j=\sum_{j'=1}^m\sum_{j=1}^\infty\langle\phi_{j'},\phi_j\rangle b_{ij}\phi_{j'}=\sum_{j=1}^mb_{ij}\phi_j.\]
According to Assumption $\ref{assumption: rate-type condition}$, one can get
\begin{align*}
\left\|{\bs{\beta}}_i-{\bs{\beta}}_i^{(m)}\right\|&=\left\|\sum_{j=m+1}^\infty b_{ij}\phi_j\right\|=\sqrt{\sum_{j=m+1}^\infty b_{ij}^2}\leqslant \wt C\sqrt{\sum_{j=m+1}^\infty j^{-2\alpha_2}};\\
\left\|{\bs{\beta}}_i+{\bs{\beta}}_i^{(m)}\right\|&\leqslant\|{\bs{\beta}}_i\|+\lno{\bs{\beta}}_i^{(m)}\rno\leqslant2\|{\bs{\beta}}_i\|=2\sqrt{\sum_{j=1}^\infty b_{ij}^2}\leqslant 2\wt C\sqrt{\sum_{j=1}^\infty j^{- 2\alpha_2}}.
\end{align*}
Because $\alpha_2>1/2$, one has
\[\sum\limits_{j=m+1}^\infty \frac{1}{j^{2\alpha_2}}\leqslant \frac{1}{2\alpha_2-1}\frac{1}{m^{2\alpha_2-1}};\qquad \sum_{j=1}^\infty \frac 1{j^{2\alpha_2}}=\zeta(2\alpha_2)\text{ is convergent},\]
where $\zeta(\cdot)$ is Riemann $\zeta$ function. Thus, one can get
\begin{equation}\label{eq: upper bound of operator norm of A minus B}
\lno{\mc{B}}-{\mc{B}^{(m)}}\rno \leqslant 2d\wt C^2\sqrt{\frac{\zeta(2\alpha_2)}{2\alpha_2-1}}m^{-\frac{2\alpha_2-1}{2}}.
\end{equation}

Furthermore, 
{since $\mr{rank}(\mc{B})=d$, one can get that $\lambda_{\min}^+(\mc{B})=\lambda_{d}(\mc{B})$. It is easy to see $\rank(\mc{B}^{(m)})\leqslant d$ by $\mc{B}^{(m)}=\Pi_m \mc{B} \Pi_m$, thus one can assume that $\lambda_{\min}^+(\mc{B}^{(m)})=\lambda_j( \mc{B}^{(m)})$ for some $j\leqslant d$.
By Corollary $\ref{coro:wely ineq operator}$
% (Notice that $M_m$ and $M$ are both compact and self-adjoint)
and \eqref{eq: upper bound of operator norm of A minus B}, one has:
$$
|\lambda_j( \mc{B}^{(m)})-\lambda_j\l \mc{B}\r|\leqslant\lno \mc{B}-\mc{B}^{(m)}\rno \leqslant 2d\wt C^2\sqrt{\frac{\zeta(2\alpha_2)}{2\alpha_2-1}}m^{-\frac{2\alpha_2-1}{2}}.
$$
Thus for sufficiently large {$n\geqslant \l \frac{\lambda_d({\mc{B}})}{4d\wt C^2}\sqrt{\frac{2\alpha_2-1}{\zeta(2\alpha_2)}} \r^{\frac{2}{c_1(1-2\alpha_2)}}$}, one has
\begin{align}
&\lambda_j\l \mc{B}^{(m)}\r\geqslant \lambda_j\l \mc{B}\r-\frac{\lambda_d\l \mc{B}\r}{2}
\geqslant\frac{\lambda_d\l \mc{B}\r}{2}\nonumber\\
&\qquad\Longrightarrow \min\{\lambda_{\min}^+({\mc{B}}),\lambda_{\min}^+({\mc{B}^{(m)}})\}\geqslant \frac{\lambda_d({\mc{B}})}{2}. \label{eq:lower bound lambda min plus B Bm}
\end{align}}
Inserting \eqref{eq: upper bound of operator norm of A minus B} and \eqref{eq:lower bound lambda min plus B Bm} into \eqref{eq:sin theta for B Bm} leads to
\begin{align*}
\left\|P_{\mathcal S_{\Y|\boldsymbol{X}}}-P_{\mathcal S_{\Y|\boldsymbol{X}}^{(m)}}\right\| \leqslant \frac{2\pi d\wt C^2}{\lambda_{d}(\mc{B})}\sqrt{\frac{\zeta(2\alpha_2)}{2\alpha_2-1}}m^{-\frac{2\alpha_2-1}{2}}.
\end{align*}
Then choosing $C_2:=\frac{2\pi d\wt C^2}{\lambda_d({\mc{B}})}\sqrt{\frac{\zeta(2\alpha_2)}{2\alpha_2-1}}$ can complete the proof.
\end{proof}



\subsection{Proof of Theorem \ref{theorem, total convergence rate}}
\begin{proof}
Note that
\begin{equation}
\begin{aligned}
\left\|P_{\mc{S}_{\Y|\X}}-P_{\widehat{\mc{S}}_{\Y|\X}^{(m)}}\right\| 
&\leqslant \left\|P_{\mc{S}_{\Y|\X}}-P_{\mc{S}_{\Y|\X}^{(m)}}\right\| +\left\|P_{\mc{S}_{\Y|\X}^{(m)}}-P_{ \widehat{\mc{S}}_{\Y|\X}^{(m)}}\right\| .\\
\end{aligned}
\end{equation}
Next we select $m$ to be $n^{\frac{1-2\gamma}{2\alpha_1+2\alpha_2+1}}$, i.e.,  $c_1:=\frac{1-2\gamma}{2\alpha_1+2\alpha_2+1}$. And it is easy to check that $c_1$ satisfies $2c_1(\alpha_1+1)+2\gamma-1=-\frac{(1-2\gamma)(2\alpha_2-1)}{2\alpha_1+2\alpha_2+1}<0$ and $c_1(2\alpha_1+1)-1=-\frac{2[\gamma(2\alpha_1+1)+\alpha_2]}{2\alpha_1+2\alpha_2+1}<0$.
Then combining Proposition $\ref{proposition, estimation error}$ with Proposition $\ref{proposition, truncation error}$ leads to
\begin{align*}
\P\left[\left\|P_{\mc S_{\Y|\X}}-P_{\widehat{\mc{S}}_{\Y|\X}^{(m)}}\right\| \leqslant\hspace{-0.5mm} (C_1+C_2)n^{-\frac{(2\alpha_2-1)(1-2\gamma)}{2(2\alpha_1+2\alpha_2+1)}}\right]\hspace{-1mm}\geqslant\hspace{-1mm} 1-2\exp\hspace{-0.5mm}\l\hspace{-1mm}- C'n^{\frac{1-2\gamma}{2\alpha_1+2\alpha_2+1}}\r&\\
-\exp\left[\ln\l D_1n^{\frac{2\alpha_1+2\alpha_2+3-4\gamma}{2\alpha_1+2\alpha_2+1}} \r-(D_0+1)n^{\frac{2\gamma}{5}}\right]&
\end{align*}
when $n\geqslant n_3'$, where
\begin{align*}
n_3'=\max\Bigg\{n_1,n_2',\left[\tfrac{\|\Gamma^{-1}M\|  \wt C}{48}\l\tfrac{D_2}{D_0+1}\r^{\frac52}\right]^{\frac{2}{2(c_1(\alpha_1+1)+\gamma)-1}}\hspace{-0.9mm},\l \tfrac{\|\Gamma^{-1}M\| }{2D_3}\r^{\frac{2}{c_1(2\alpha_1+1)-1}}\hspace{-0.9mm},\\
\l\tfrac{D_0+1}{D_2}\r^{\frac{5}{1-2\gamma}},\left[ \tfrac{D_3\wt C}{24}\l \tfrac{D_2}{D_0+1} \r^{\frac52} \right]^{\frac2{2\gamma+c_1}},\l\tfrac{\lambda_d(\mc{B})}{4d\wt C^2}\sqrt{\tfrac{{2\alpha_2-1}}{\zeta(2\alpha_2)}} \r^{\frac{2}{c_1(1-2\alpha_2)}}\Bigg\}
\end{align*}

It is easy to check that as long as $\frac{2\gamma}{5}<\frac{1-2\gamma}{2\alpha_1+2\alpha_2+1}\Longrightarrow\gamma<\frac{5}{4(\alpha_1+\alpha_2+3)}$, 
there exists a constant $n_3''=n_3''\l \gamma,\alpha_1,\alpha_2,D_0,D_1,C'\r$ such that when $n\geqslant n_3'$ further, we have 
\begin{align*}
\P\l\left\|P_{\mc S_{\Y|\X}}-P_{\widehat{\mc{S}}_{\Y|\X}^{(m)}}\right\| \leqslant (C_1+C_2)n^{-\frac{(2\alpha_2-1)(1-2\gamma)}{2(2\alpha_1+2\alpha_2+1)}} \r
\geqslant1-2\exp\l-\tfrac{D_0+1}{2}n^{\frac{2\gamma}{5}} \r.
\end{align*}
Thus one can choose $n_3=\max\{n_3',n_3''\}$ to get the following conclusion.
\begin{proposition}
Under Assumptions $\ref{as:joint distribution assumption}$ to $\ref{assumption: rate-type condition}$, for any $\gamma\in\l0,\tfrac{5}{4(\alpha_1+\alpha_2+3)}\r$, choosing 
$m=n^{\frac{1-2\gamma}{2\alpha_1+2\alpha_2+1}}$ (i.e.,  $c_1=\frac{1-2\gamma}{2\alpha_1+2\alpha_2+1}$) yields a positive constant
\begin{align*}
D_4:=D_4\l \|\Gamma^{-1}M\| ,\sigma_d(\Gamma^{-1}M) ,\gamma,\sigma_0,\sigma_1,d,\wt C,\lambda_d\l\sum\limits_{i=1}^d {\bs{\beta}}_i\otimes{\bs{\beta}}_i\r,\alpha_2\r 
\end{align*}
such that when $n$ is sufficiently large, we have:
\begin{align*}
\P\l\left\|P_{\mc{S}_{\Y|\X}}-P_{\widehat{\mc{S}}_{\Y|\X}^{(m)}}\right\| \leqslant D_4n^{-\frac{(2\alpha_2-1)(1-2\gamma)}{2(2\alpha_1+2\alpha_2+1)}} \r
\geqslant1-2\exp\l -\tfrac{D_0+1}{2}n^{\frac{2\gamma}{5}} \r,
\end{align*}
where $D_0$ and $D_1$ are defined in Proposition $\ref{prop:bound hatMmd Mm}$.
\end{proposition}
\noindent
% Theorem $\ref{theorem, total convergence rate}$ is a direct corollary of above proposition.
Define 
$$\mathtt F:=\left\{\left\|P_{\mc{S}_{\Y|\X}}-P_{\widehat{\mc{S}}_{\Y|\X}^{(m)}}\right\| \leqslant D_4n^{-\frac{(2\alpha_2-1)(1-2\gamma)}{2(2\alpha_1+2\alpha_2+1)}}\right\}.$$
Then 
\begin{align*}
 \mb E\left[\left\|P_{\mc{S}_{\Y|\X}}-P_{\widehat{ \mc{S}}_{\Y|\X}^{(m)}}\right\|^2\right] =&
  \mb E\left[\left\|P_{\mc{S}_{\Y|\X}}-P_{\widehat{ \mc{S}}_{\Y|\X}^{(m)}}\right\|^21_{\mathtt{F}}\right] +
   \mb E\left[\left\|P_{\mc{S}_{\Y|\X}}-P_{\widehat{ \mc{S}}_{\Y|\X}^{(m)}}\right\|^21_{\mathtt{F}^c}\right]\\ 
 \leqslant &
 D_4^2n^{-\frac{(2\alpha_2-1)(1-2\gamma)}{2\alpha_1+2\alpha_2+1}}+4\mb P\left( \mathtt F^c\right)\\
 \lesssim&n^{-\frac{(2\alpha_2-1)(1-2\gamma)}{2\alpha_1+2\alpha_2+1}}+\exp\l -\tfrac{D_0+1}{2}n^{\frac{2\gamma}{5}} \r\\
\lesssim&n^{-\frac{(2\alpha_2-1)(1-2\gamma)}{2\alpha_1+2\alpha_2+1}}.
\end{align*}
This completes the proof of  Theorem \ref{theorem, total convergence rate}.
\end{proof}







\section{Additional Simulation Results of Section \ref{sec:Synthetic}}
This section contains the additional  simulation results  of Sections \ref{sec:Synthetic}  when $\varepsilon\sim N(0,1)$.



We show the average $\mc D(\bs B;\bs{\wh B})$ with different $m$ or $\rho$ for three methods under $\mc M_1$ to $\mc M_3$ in Figure \ref{fig:error 3models,noise1},
where we mark minimal error in each model with red `$\times$'. The shaded areas represent the standard error associated with these estimates and all of them are less than  $0.01$. For FSFIR, the  minimal errors for $\mc M_1-\mc M_3$ are  $0.08,0.02,0.01$ respectively.
For TFSIR, the  minimal errors are  $0.08,0.02,0.01$ and for regularized FSIR,  the  minimal errors are $0.13,0.06,0.01$.  

% Figure environment removed


Figure \ref{fig:error 3models,noise1} shows that FSFIR attains the best performance among  all models. 
Moreover, FSFIR is easier to practice as it does not need a slice number $H$ in advance. 








\end{document}

