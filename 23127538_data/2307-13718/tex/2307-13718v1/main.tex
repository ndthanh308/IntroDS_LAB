\documentclass[aps,pra,twocolumn,noshowpacs,superscriptaddress,groupedaddress]{revtex4}  % for review and submission
%\documentclass[aps,preprint,showpacs,superscriptaddress,groupedaddress]{revtex4}  % for double-spaced preprint
\usepackage{graphicx}  % needed for figures
\usepackage{dcolumn}   % needed for some tables
\usepackage{bm}        % for math
\usepackage{amssymb}   % for math
\usepackage{amsmath}
\usepackage{color}
\usepackage[free-standing-units=true]{siunitx} % for consistent handling of SI units
\usepackage[colorlinks=true, pdfstartview=FitV, linkcolor=blue, citecolor=blue, urlcolor=blue]{hyperref} % enable links
\usepackage{epstopdf}

\usepackage{appendix} % for appendix


% avoids incorrect hyphenation, added Nov/08 by SSR
\hyphenation{ALPGEN}
\hyphenation{EVTGEN}
\hyphenation{PYTHIA}

\newcommand{\ee}[1]{\times 10^{#1}}
\newcommand{\mr}[1]{\mathrm{#1}}
\renewcommand{\Re}{\text{Re}}
\renewcommand{\Im}{\text{Im}}

\newcommand{\affilITP}{Institute for Theoretical Physics, ETH Z\"{u}rich, CH-8093 Z\"urich, Switzerland.}
\newcommand{\affilIFP}{Laboratory for Solid State Physics, ETH Z\"{u}rich, CH-8093 Z\"urich, Switzerland.}
\newcommand{\affilQC}{Quantum Center, ETH Zurich, CH-8093 Zurich, Switzerland}
\newcommand{\affilUKON}{Department of Physics, University of Konstanz, D-78457 Konstanz, Germany}

\addtolength{\topmargin}{+10mm}

%\bibliographystyle{apsrev}

\begin{document}

\preprint{}

\title{Proliferation of unstable states \\and their impact on stochastic out-of-equilibrium dynamics}

\author{Toni L. Heugel}\affiliation{\affilITP}
%\author{Christian Marty}\affiliation{\affilIFP}
\author{R. Chitra}\affiliation{\affilITP}
\author{Alexander Eichler}\affiliation{\affilIFP}\affiliation{\affilQC}
\author{Oded Zilberberg}\affiliation{\affilUKON}
%\altaffiliation[Present address: ]{IBM Research-Zurich, Säumerstrasse 4, CH-8803 Rüschlikon, Switzerland}
%
%
\date{\today}% It is always \today, today, but any date may be explicitly specified

% possible Referees: Grimm, Hiebert, Weig, Villanueva, Yamaguchi, Mahboob, Shaw, Dykman

\begin{abstract}
Networks of nonlinear parametric resonators are promising candidates as Ising machines for annealing and optimization. 
These many-body out-of-equilibrium systems host complex phase diagrams of  coexisting stationary states. The plethora of states manifest via a series of bifurcations, including bifurcations that proliferate purely unstable solutions, which  we term ``ghost bifurcations''. Here, we demonstrate that the latter take a fundamental role in the stocahstic dynamics of the system in the presence of noise. Specifically, they determine the switching paths and the switching rates between stable solutions.  We demonstrate experimentally the impact of ghost bifurcations on the noise-activated switching dynamics in a network of two coupled parametric resonators. %Our work emphasizes the importance of the bifurcation topology on the noise-induced switching and hence on the annealing performance.
%An important question concerns the validity of this Ising analogy for this intrinsically nonlinear many-body systems. Here we address this question by considering a network of two parametric resonators. Even in the weak coupling regime, 
%intrinsically nonlinear many-body systems
%driven out-of-equilibrium system. Nonlinear -> complex bifurcation series
%The presented results demonstrate that the 
%dynamics of these networks are not fully captured by the Ising analogy. 
%Our work emphasizes the need for further exploration of 
%many body effects beyond the Ising analogy for applications as optimization devices.
\end{abstract}
\maketitle

% literature to read and cite:
% Kramers rule (only underdamped branch) in an optical double potential well: https://pa.msu.edu/sites/_pa/assets/File/research/cmp/dykman/publications/nature99.pdf

% switching in a period-tripling system with a shallo metastable state at the origin: https://www.ncbi.nlm.nih.gov/pmc/articles/PMC7319998/

% https://academic.oup.com/book/7274/chapter/151994029?login=true

% https://journals.aps.org/pre/pdf/10.1103/PhysRevE.57.5202


\section{Introduction}
Statistical physics has provided valuable insights into the phenomenon of noise-induced switching between local energy minima, as exemplified by the well-known Kramers double-well problem~\cite{Kramers1940, Hanggi_1990, rondin2017direct}.
Such stochastic dynamics is highly relevant for a wide variety of phenomena spanning protein folding~\cite{Best2006,Chung2015}, chemical reactions~\cite{GarciaMuller2008}, as well as stability in mechanical~\cite{Badzey2005,Rondin2017} and electrical systems~\cite{Fulton1974,Silvestrini1988}. While the noisy dynamics of equilibrium  systems have been extensively studied, that of systems driven far out of equilibrium remains largely unexplored~\cite{Dykman_1998,Dykman_2007,tadokoro2020noise}. 

An important class of out-of-equilibrium systems are driven systems. Such systems are characterized by stationary oscillation states that manifest when the conserving and nonconserving forces in the system are in balance. For nonlinear systems, there may be several such stationary states that act as attractors, just like potential wells do in equilibrium systems. In a rotating frame, the resulting dynamics can resemble that of an equilibrium potential landscape~\cite{Hanggi_1990}. Extending the analogy, noise can induce stochastic switching between the attractors, and the switching rate can be treated with the abstract notion of a potential activation barrier~\cite{Dykman_1998,Luchinsky1999,lapidus,kkim2005,Aldridge2005,Chan_2007,Chan_2008,Venstra2013,Mahboob_2014_2,margiani2022extracting}. However, it is important to emphasize that this activation barrier is not related to a gap in free energy, but instead to a ``phase gap'' that separates the attractors~\cite{Frimmer_2019}. Understanding how often the system switches between attractors, and which path it selects during the switch, requires different methods and can be cumbersome~\cite{Hanggi_1990,Dykman_1998}.

A paradigmatic example of a bistable out-of-equilibrium system is the Kerr parametric oscillator (KPO)~\cite{Ryvkine_2006, Mahboob_2008, Wilson_2010, Eichler_2011_NL, Leuch_2016, Gieseler_2012, Lin_2014, Puri_2017, Eichler_2018, Nosan_2019, Frimmer_2019, Grimm_2019, Puri_2019_PRX, Miller_2019_phase, DykmanBook,Eichler_Zilberberg_book}.
%The driving is achieved by the modulation (pumping) of the potential energy at an angular frequency $\omega_p$~\cite{Faraday_1831, Mathieu, Landau_Lifshitz}. Depending on the pumping strength $\lambda$ and the detuning of $\omega_p$ from twice the angular resonance frequency $\omega_0$, there is a characteristic region termed `Arnold tongue'. Inside the Arnold togue, the KPO has exactly two stable solutions that we refer to as `phase states', which have identical amplitudes but differ in phase by $\pi$~\cite{mclachlan1951theory,Lifshitz_Cross}. Noise-induced switching in a single KPO is well-studied ~\cite{Dykman_1998,lapidus,Ryvkine_2006,Stambaugh_2006,Stambaugh_2007,Chan_2008,margiani2022extracting}, with inherent non-equilibrium physics leading to interesting transition paths that lack time-reversal symmetry~\cite{Chan_2008}. These switching trajectories pass through an unstable state at zero amplitude and the transition rates  can be described through an effective activation barrier with universal power-law scaling around the bifurcation points~\cite{Stambaugh_2006,Stambaugh_2007}.
Recently, networks of coupled, driven KPOs have been proposed as a simulation platform to solve complex problems optimally.~\cite{ Mahboob_2016, Inagaki_2016, Goto_2016, Puri_2017_NC,Nigg_2017,Dykman_2018,Okawachi_2020}. Such networks typically possess a large number of stationary states, analogous to an multi-well potential with rich phase transitions~\cite{Strinanti_2021,Bello_2019,Heugel_2022,heugel2022role}. Studying activated switching between states in the presence of fluctuations is crucial for understanding their stability and lifetimes. It will influence how these networks are operated, and it can also provide a characterization method that is unaffected by the danger of local trapping~\cite{margiani2022deterministic}.
%In practice, the search for an optimal network configurations can follow various protocols, such as simulated annealing~\cite{Puri_2017_NC, Nigg_2017, Heim_2015} or stochastic sampling~\cite{margiani2022deterministic}. In all of these protocols, the different coupling strengths between the individual KPOs break the symmetry of the phase states, resulting in favored configurations similar to the situation in an Ising system~\cite{Ryvkine_2006,Dykman_2018, Rota_2019}. In contrast to optimization algorithms on conventional computers, KPO networks encode problems with many degrees of freedom, such as the Ising Hamiltonian, in an analogue, parallel fashion, with the phase states of each KPO taking the role of a single bistable node (e.g. a classical spin). It is expected that these systems are in some cases equivalent to a Hopfield network, and that they can solve many tasks much more efficiently than software-based Hopfield networks in current digital, sequential computer algorithms.


%Operating the network as an annealing device, the system switches between phase state configurations approaching the solution of the corresponding optimization problem.
%In a frame rotating at $\omega_p/2$, the system can be mapped to a potential landscape with minima, maxima and saddle points that define stable, unstable and quasi-stable solutions, respectively [we need to formulate this more pedantically]. Stable network solutions are separated by effective barriers that the system can overcome under the influence of stochastic activation.


%In a previous works, we explored the phase diagram of two-KPO networks with strong coupling~\cite{Heugel_2022}. We found that the coupling leads to the formation of split normal modes, and to partially overlapping Arnold tongues for parallel and antiparallel KPO oscillation states. The rich phase diagram of the system included doubly unstable solutions stemming from a pitchfork bifurcation only involving unstable solutions. Investigating the steady-state physics with adiabatic sweeps, this kind of states remained unobserved and hence were dubbed `ghosts'.

In this work, we address the physics of stochastic activation in a  tractable system of two strongly coupled, classical KPOs. The system possesses various stable and unstable stationary states~\cite{Heugel_2019_TC,Heugel_2022}, and we observe stochastic switching between two such states in the presence of fluctuations. Surprisingly, the switching rate $\Gamma$ deviates significantly from the exponential model expected for a single KPO~\cite{Dykman_1998}. Seeking to explain this deviation, we calculate the dominant transition paths between the states with the Onsager-Machlup function~\cite{Lehmann_2003, Wio_2013}. Our analysis shows that new kinds of unstable states, termed ghost states  emerge in the two KPO system. These states do not manifest in the stationary deterministic dynamics of the system. Interestingly, however, they offer new transition paths and contribute significantly to the transition paths and the corresponding transition rates. We thus identify a striking example of out-of-equilibrium statistical physics in a nonlinear multistable system. Our work paves the way for the exploration of larger systems, especially in view of KPO networks as solvers for complex optimization tasks.
 

\section{System}  
In the following, we analyze noise-induced switching dynamics using an experimental setup composed of two electrical KPOs with capacitive coupling~\cite{Nosan_2019,Heugel_2022}. Each KPO consists of a coil with inductance $L$ and a diode that provides a nonlinear capacitance $C$, cf. Appendix~\ref{sec_single_KPO}. The resonance frequency of each KPO can be tuned by applying a DC voltage across the diode. %Each resonator (marked by index $i$) is composed of  an inductance $L = \SI{87}{\micro\henry}$, a tuning voltage $U_i = \SI{2}{\volt}$, and a nonlinear capacitance $C\approx\SI{20}{\pico\farad}$ realized by a varicap diode, . 
We drive and measure the resonators inductively through auxiliary coils. Our electrical circuits are well described by the following coupled equations of motion
\begin{multline}
	\ddot{x}_i + \omega_i^2\left[1-\lambda\cos\left(2\omega_d t\right)\right]x_i + \alpha_i x_i^3 \\+ \gamma_i \dot{x}_i -\sum_{j\neq i}J_{ij} x_j = \xi_i(t)\,, \label{eq:coupled_EOM}
\end{multline}
where dots indicate time derivatives, $x_i = u_i \cos(\omega t) - v_i \sin(\omega t)$ is the measured voltage with quadrature amplitudes $u_i$ and $v_i$, $\omega_i = 2\pi f_i$ is the angular eigenfrequency, and $J_{ij}$ ($i\neq j$) denotes the inter-circuit linear coupling strength. Each resonator has an effective Duffing (Kerr) nonlinearity with coefficient $\alpha_i$ and a damping rate $\gamma_i = \omega_i/Q_i$, with  $Q_i$ the quality factor. Our resonators are constructed and tuned to be (nearly) identical in their bare characteristics, $\omega_i \approx \omega_0 = 2\pi f_0$.  The same parametric pumping with angular modulation frequency $2\omega_d = 4\pi f_d \approx 2\omega_0$ and modulation depth $\lambda \propto U_d$ is applied to all resonators. 

Crucially, beyond a frequency-dependent driving threshold $U_\mathrm{th}$, the KPO has exactly two stable solutions that we refer to as `phase states', which have identical amplitudes but differ in phase by $\pi$~\cite{mclachlan1951theory,Lifshitz_Cross,Eichler_Zilberberg_book}. These are the two attractors of a single KPO in a frame rotating at $\omega_d$. In order to induce switching events between the attractors, we add an artificial noise $\xi_i$ generated by a fluctuating voltage  with white power spectral density $S_n$. The noise $\xi_i$ simulates a thermal force noise with $\langle \xi_i(t_1) \xi_j(t_2)\rangle = \varsigma^2 \delta_{ij} \delta(t_1-t_2)$ and  power spectral density calibrated to be $\varsigma^2 = \SI{4.93e-20}{\hertz^4} S_{n}$, see Appendix~\ref{sec:fluctuating_coherent}.

In our experiments, we use a lock-in amplifier to measure the quadratures $(u_i, v_i)$ which vary on timescales much longer than $1/\omega_0$. The evolution of these quadratures is well
captured in the rotating-frame picture obtained by applying the averaging method~\cite{guckenheimer_1990, Papariello_2016, Heugel_2019_TC}. to our model. Equation~\eqref{eq:coupled_EOM} then leads to the following slow-flow equations: 
\begin{align}
    \label{eq:slowflow}
    \dot{u}_i &= -\frac{\gamma  u_i}{2}- \left(\frac{3 \alpha}{8 \omega_d }X_i^2 +\frac{\omega_0^2 - \omega_d^2}{2\omega_d}+\frac{\lambda \omega_0^2}{4 \omega_d }\right) v_i +\frac{J  v_j}{2 \omega_d }+\Xi_{u_i}\,,\nonumber\\
    \dot{v}_i &= -\frac{\gamma  v_i}{2} + \left(\frac{3 \alpha}{8 \omega_d }X_i^2 + \frac{\omega_0^2 - \omega_d^2 }{2\omega_d} -\frac{\lambda \omega_0^2}{4 \omega_d }\right)u_i-\frac{J  u_j}{2 \omega_d }+\Xi_{v_i}\,,
\end{align}
where $X_i^2 = u_i^2 + v_i^2$, and we have additive uncorrelated noise terms $\Xi_{u_i}$, $\Xi_{v_i}$, whose power spectral densities are given by $\sigma^2=~ \varsigma^2/2\omega_d^2$~\cite{Khasminskii_66, Roberts_86}. As shown in previous works, the averaging method fully captures the physics of our
networks in the regime where $\lambda$, $\gamma/\omega_0$, $J/\omega_0^2$, $(\alpha/\omega_0^2)x_i^2$, and $\alpha \varsigma^2 / \omega_0^5$ are all of order $\epsilon$ with $0<\epsilon\ll 1$~\cite{nayfeh2008,Khasminskii_66,Papariello_2016,Eichler_2018}. Here and in the following, we assume identical dissipation rates $\gamma_i = \gamma$, nonlinearities $\alpha_i = \alpha$, and coupling $J_{ij}=J$.

% Figure environment removed

%- introduce phase states


%A single parametric resonator can be driven into parametric resonance when $U_d$ exceeds  a certain driving threshold $U_{th}$~\cite{Landau_Lifshitz, Lifshitz_Cross}. 

\section{Transition Rates}
A switching experiment with KPO 1 (while KPO 2 is detuned) is shown in Fig.~\ref{fig:Fig_1}(a). There, we observe that the system resides in each phase state for a certain dwell time before switching to the opposite state. The average dwell time $\tau$ can be expressed as a rate of activated switching $\Gamma = \tau^{-1}$. In the rotating phase space, the same measurement data can be represented as a density of count rates, see Fig.~\ref{fig:Fig_1}(b). As discussed in Ref.~\cite{Dykman_1998}, the logarithm of the switching rate is inversely proportional to the distance between the two phase states. In Fig.~\ref{fig:Fig_1}(c), this leads to an exponential decrease in $\Gamma$ with decreasing $f_d$ (the direction in $f_f$ in which $\Gamma$ increases depends on the sign of the nonlinearity).

% Figure environment removed

We now consider the case that the two KPO are tuned to have the same frequency. The experimental phase diagram of the system as a function of $\lambda$ and $\omega_p$ was characterized in Ref.~\cite{Heugel_2022}. 
In Fig.~\ref{fig:Fig_2}(a), we show the results of a measured frequency sweep passing through various regions containing up to four different types of stable two-oscillator states: a state with both resonators having amplitude zero; (S) symmetric states, i.e., the two oscillators have the same phase; (A) antisymmetric states where the two oscillators have $\pi$-shifted phases; and (M) mixed-symmetry states that are neither symmetric nor antisymmetric. The symmetric/antisymmetric solutions can be interpreted as the parametrically driven symmetric/antisymmetric normal modes of the two resonators. See Fig.~\ref{fig:FigS1} for more information regarding the stability diagram.

Typically, when initialized in one such state, the resonator explores the vicinity of the attractor under the influence of the noise terms $\Xi_{u_i}$ and $\Xi_{v_i}$ ~\cite{heugel2022role}. Occasionally, the noise activates the system over the quasi-potential barrier to reach a different attractor, as seen for instance in Fig.~\ref{fig:Fig_2}(b)~\cite{margiani2022extracting}. The system can switch back and forth in time and is characterized by a switching rate $\Gamma$. In the following, we analyze $\Gamma$ along the sweep in Fig.~\ref{fig:Fig_2}(a) for $f_d>\SI{2.36}{\mega\hertz}$. In this frequency range, only the symmetric solution is stable~\cite{Heugel_2022}. We therefore expect $\Gamma(f_d)$ to manifest an exponential behaviour analogous to that seen in a single KPO in Fig.~\ref{fig:Fig_1}. Motivated by this idea, we set out to confirm the hypothesis that each normal mode has the same activation rate scaling as a single KPO.



%There, it is well known that the transition rate decreases monotonically with increasing separation between the phase states, which can be controlled through $f_d$~\cite{Dykman_1998, Ryvkine_2006, Stambaugh_2006, Chan_2008,margiani2022extracting}.
%decreases monotonically as the attractors are moved appart by detuning $f_d$~\cite{Dykman_1998, Ryvkine_2006, Chan_2008, Dykman_2018}. 
%and our results in Fig.~\ref{fig:Fig_Single_Noisy}(c).
%First we inspect the noise-induced switching of a single parametron whose properties are well-known ~\cite{Dykman_1998, Ryvkine_2006, Chan_2008, Dykman_2018}.
%We find that $\Gamma$ decreases monotonically with increasing separation between the phase states, which we control here through $f_d$~\cite{Dykman_1998}. 
%Similar results have previously been measured in other parametron implementations~\cite{Chan_2008}.
%The monotonic decrease of $\Gamma$ in Fig.~\ref{fig:Fig_Single_Noisy}(c) is readily derived using the Onsager-Machlup approach~\cite{Dykman_1998}.

In Fig.~\ref{fig:Fig_2}(c), we show the measured transition rate $\Gamma$ for switches between the two symmetric states as a function of $f_d$ while the parametric driving strength $U_d$ is fixed. With decreasing $f_d$, the transition rate $\Gamma$ decreases exponentially as expected from previous single KPO studies~\cite{Dykman_1998, Ryvkine_2006, Stambaugh_2006, Chan_2008,margiani2022extracting}. This holds for a distinct range marked as I and II until $f_d \approx \SI{2.365}{\mega \hertz}$ is reached. Surprisingly, below this frequency, we observe substantial deviations from the simple exponential model even though the symmetric state remains the only stable configuration. First, a kink  manifests at $f_d= \SI{2.365}{\mega\hertz}$, implying a relative enhancement of the switching in region III. At even lower frequencies, the slope of $\Gamma$ changes sign, signaling a significant increase of the switching with decreasing frequency in region IV. %, signalling the formation of additional stable  configurations.
This behaviour is fundamentally at odds with the
standard expectation of  decreasing rates.

To obtain deeper insights into the curious switching behaviour of the two-KPO system, we look at several measured transition events and systematically collect the 4-quadrature state vector $\mathbf{Y}=(u_1,v_1,u_2,v_2)$, cf. Appendix~\ref{sec:antisymm_switching}.
To visualize the transitions in a 2-dimensional space, we use symmetric and antisymmetric coordinates, $v_{\rm S} = (v_1 + v_2)/\sqrt{2}$,  $v_{\rm A} = (v_1 - v_2)/\sqrt{2}$ (and analogous for $u$).
By plotting several time traces containing multiple transition events in the phase space spanned by $v_S$ and $v_A$, we obtain the corresponding probability distribution. 
%The resonator mostly dwells in the vicinity of a given phase state and explores the local fluctuations of the specific attractor.

Comparing the distributions at four representative frequencies, we find striking differences, see Fig.~\ref{fig:Fig_2}(d). Regions I and II are characterized by two attractors with high probabilities, corresponding to the two phase states of the symmetric mode. By contrast, for $f_d < \SI{2.365}{\mega\hertz}$ in the regions III and IV, we find the emergence of a substantial probability centered around the attractors of the antisymmetric modes. This indicates that the appearance of the kink at $f_d=\SI{2.365}{\mega\hertz}$ follows a drastic change of the transition path chosen by the system, and of the underlying quasi-potential landscape. Such a deviation in region IV is to be expected as the activation dynamics now involves four different stable attractors. In other words, the naive model of activation between two states is insufficient in region IV. This is confirmed by a numerical simulation of the noisy time evolution of the EOMs, given in Eq.~\eqref{eq:slowflow}, which is in accord with the experimental results, see bright blue data in Fig.~\ref{fig:Fig_2}(c). Crucially, however, we note that the antisymmetric state is not stable in region III, as indicated by a red square (instead of a circle). This observation raises the question: why should the unstable solutions of the syste influence the transition paths?
%In a non-driven noisy system, this would correspond to the emergence of new stable attractors or saddle points. 
In the following, we precisely address this question through an in-depth theoretical study of the transition dynamics.



%The investigated frequency range is divided into  four different regions separated by three bifurcations, cf. dashed line in Fig.~\ref{fig:Fig_Coupled_Steady}(d).%, out of which two are ghost bifurcations
%In region I, $\Gamma$ decreases exponentially with decreaseing $f_d$ as expected from previous single  parameteron studies~\cite{Dykman_1998, Ryvkine_2006, Stambaugh_2006, Chan_2008}. The same trend continues in region II. Surprisingly, we observe a different behavior in region III and IV. distinct kink at the border between regions II and III, coincident with a ghost bifurcation. Finally, in region IV the slope changes sign and  $\Gamma$ increases as $f_d$ is decreased, signalling the formation of additional stable  configurations.








%: (i) a state with both oscillators at amplitude 0, (ii) both oscillators occupy the same phase state (`symmetric' = S), (iii) both oscillators occupy the opposite phase state (`antisymmetric' = A), and (iv) both states have different amplitude but approximately the same or $\pi$-shifted phase (`mixed-symmetry' = M). As a function of $f_d$ these states appear, disappear, or change stability at the bifurcation points, see Fig.~\ref{fig:Fig_Coupled_Steady}(c). 
%The resulting phase diagram is shown in Fig.~\ref{fig:Fig_Coupled_Steady}(d). In some phases several states coexist. For low driving strengths there are two separated instability lobes corresponding to the symmetric and antisymmetric normal mode of the linearly coupled system.
%When increasing the drive to $U_d = \SI{3.7}{\volt}$, the two lobes overlap and the order (in frequency) of the two central bifurcation points is inverted, see red and blue squares in Fig.~\ref{fig:Fig_Coupled_Steady}(d). 
%This inversion is accompanied by a change in the characteristics of the involved solutions. Both bifurcation points now connect only to unstable solutions, some of which are even \textit{doubly unstable}, meaning that the system is unstable with respect to two fluctuation directions in phase space, cf. Fig.~\ref{fig:Fig_Coupled_Steady}(c). Since no stable solutions are involved, these `ghost bifurcations' are not directly accessible in standard frequency sweeps. Nevertheless, they have important consequences for the coupled system. The first of these consequences is that the antisymmetric solution connecting the right-hand side red bifurcation point to the brown one has become unstable, making the symmetric solution the only stable state across a large frequency range as discussed in~\cite{Heugel2021}. The second important consequence will only be visible in the presence of noise. In the following, we will investigate the impact of the two ghost bifurcations on noise-induced switching between the stable steady states.
%The antisymmetric state only becomes stable after a pitchfork bifurcation caused by the interplay of the inherent nonlinearity and the linear coupling between the resonators. A stability analysis against small fluctuations yields the expression ~\cite{Heugel_2021}
% \begin{align}
%\label{eq:brownline}
%\lambda_A = \frac{2\sqrt{ \gamma ^2 \omega ^2+\left(2J- \left(\omega ^2-\omega_0^2\right)\right)^2}}{\omega_0^2}\,.
%\end{align}
%In the following, we will investigate the impact of the two ghost bifurcations on noise-induced switching between stable steady states, after reviewing the situation for a single parametric resonator.

%The most prominent feature in this diagram is the pitchfork bifurcation shown as a solid brown line. It is caused by the combination of the inherent nonlinearity and the linear coupling between the resonators in Eq.~\eqref{eq:coupled_EOM}. Together, these terms produce an effective parametric coupling between the symmetric and antisymmetric modes (cf. Appendix~\ref{sec:nonlinear_coupling}). This bifurcation represents the border for the stability of the antisymmetric solution (A) with the mixed solution (M) or the purely symmetric region (S). The position of this bifurcation is obtained by a stability analysis against small fluctuations (cf. Appendix~\ref{sec:weak_noise}), which yields the expression


%As the coupling coefficient $J$ is decreased (increased), the stability boundary of the antisymmetric solution ($\lambda_A$) approaches (departs) the right boundary of the antisymmetric instability lobe. This observation bears important implications for using parametron networks for Ising machines. Commonly, the weak coupling limit ($J\rightarrow 0$) is explored, where the symmetric and antisymmetric solutions are stable in the entire overlap region ~\cite{Heugel_2019_TC}, and an Ising spin-based description correctly captures the symmetry (and behavior) of the stable solutions~\cite{Dykman_2018, Rota_2019}. However, as we find here, an increase in the coupling ($J$) shifts the brown line to lower frequencies, and the region where the symmetric solution is the unique stable solution grows.
%The overlap region of the two instability lobes that is marked as a grey area in Fig.~\ref{fig:Fig1}(d) is therefore no longer 
%Both the symmetric and antisymmetric solutions remain stable in the triangular overlap region delineated by the brown and blue dashed lines in Fig.~\ref{fig:Fig3}(g), but not in the full overlap of the two instability lobes that is marked as a grey area in Fig.~\ref{fig:Fig1}(d).





\section{Transition Paths}
%In the following, we study the transition dynamics between the two symmetric states theoretically. 
The weak noise-induced switching between stable oscillation states is analogous to noise-activated jumping over a barrier $W$  studied in an equilibrium system~\cite{Stambaugh_2006,Hanggi_1990}. The principal difference is that in our driven system,  the barrier $W$ between two stable attractors resides in a quasipotential structure in a rotating frame. We use the  Onsager-Machlup formalism to identify the optimal transition paths in phase space between two stable attractors, whose corresponding action then provides an estimation for the barrier $W$. We first define the Onsager-Machlup function~\cite{Lehmann_2003, Wio_2013} 
\begin{equation}\label{eq:Onsager_Machlup}
    S_{\rm OM}[\mathbf{Y}] = \int_{t_i}^{t_f} \frac{1}{4}\left(\dot{\mathbf{Y}}-\mathbf{f}(\mathbf{Y}) \right)^2 dt\,,
\end{equation}
where $t_i$ ($t_f$) is the initial (final) time of the trajectory of a system composed of $N$ resonators, $\mathbf{Y}=(u_1,v_1,...,u_N,v_N)^T$, and $\mathbf{f}(\mathbf{Y})$ is the right hand side of Eq.~\eqref{eq:slowflow} without noise terms written as a column vector.
The switching probability density between two stable states  connected by a  path $\mathbf{Y}(t)$ is given by $e^{-2 S_{\rm OM}[\mathbf{Y}]/\sigma^2}$. The total switching probability $P_{if}$ from  a stable attractor at $\mathbf{Y}_i$ to one at $\mathbf{Y}_f$ is obtained by integration over all  allowed trajectories connecting them. From this probability one can derive the switching rate $\Gamma$~\cite{Lehmann_2003}, which in the weak-noise limit scales as $\Gamma \propto \exp(-2 W/\sigma^2)$ with barrier $W$~\cite{Stambaugh_2006}.

%two main effects: (1) small fluctuations around the stable state of the system, and (2) noise-induced switching between stable states. The analysis of (1) is drawing much interest in recent years as a probe for dynamical responses in out-of-equilibrium systems~\cite{Huber_2020, Soriente_2020, Soriente_2021}. Our experimental and theoretical study of the rich fluctuation spectra in the coupled parametron network is discussed in Ref.~\cite{PumpNoisyProbe}. Here, we focus on (2) 

%After charting the stable attractors of the system, we set out to explore its basic functionality as an Ising simulator. 
%

%The switching rate is obtained by taking the time-derivative of the total probability $\Gamma_{if}=\dot{P}_{if}$.%=\Gamma_0 \exp(-2 W/\sigma^2)$, with a prefactor $\Gamma_0$ and barrier $W$~\cite{Stambaugh_2006}. 
%The probability density of a given path $\mathbf{Y}(t)$ can be calculated from the action $S_{\rm OM}$ and is given by  $e^{-2 S_{\rm OM}/\sigma^2}$. The switching probability from $\mathbf{Y}_0$ to $\mathbf{Y}_f$ is then obtained by integration over all trajectories with this property. In the low noise limit the main contribution comes/originates from paths around $\mathbf{Y}_{\rm min}(t)$ minimizing $S_{\rm OM}$.
%More formally in the saddle point approximation, the switching rate is a linear combination of $e^{-2 S_{\rm OM}[\mathbf{Y}_{\rm min}_{,i}]/\sigma^2}$~\cite{}, with $\mathbf{Y}_{\rm min}_{,i}$ the $i$th local minimum of $S_{\rm OM}$.


At low noise, the switching rate $\Gamma$ is dominated by the path $\mathbf{Y}_{\rm min}$ that minimizes $S_{\rm OM}$~\cite{Lehmann_2003}. Hence, we can neglect the integration over all possible paths and the switching rate is approximately given by
\begin{equation}
\label{eq:Gamma}
\Gamma \approx \Gamma_{\rm min}\equiv \Gamma_0 e^{-2 S_{\rm OM}[\mathbf{Y}_{\rm min}]/\sigma^2}\,, 
\end{equation}
where $\Gamma_0$ is an overall prefactor and we identify the effective activation barrier $W_{\rm eff}=S_{\rm OM}[\mathbf{Y}_{\rm min}]$~\cite{Lehmann_2003}. If $S_{\rm OM}$ has multiple local minima, one needs to find the contribution from all relevant minimizing paths and weigh their relative contributions.







%In the vicinity of the ghost bifurcations, where the unstable solutions bifurcate, we expect new minimal switching paths to manifest and hence impact $\Gamma$.


% We obtain this minimal path $\mathbf{Y}_{\rm min}$ using the sgMAM method~\cite{Grafke_2017}, see Fig.~\ref{fig:Fig4}(b). It is an improved path optimization scheme using scaled time, leading to consistent converged results. We start with a guessed initial path that connects the phase states via the unstable (0-amplitude) attractor. Then, we perform numerical minimization of $S_{\rm OM}$ by varying the path in phase space between the chosen end points. 




%In the single parametron we observed, that the switching paths go via the unstable 0-amplitude state. 

The landscape of $S_{OM}$ is characterized by a bifurcation diagram of the system equations, which categorize both stable and unstable solutions as effective minima and saddles/maxima, respectively. In Fig.~\ref{fig:Fig_2}(d), we schematically show this bifurcation diagram along the frequency sweep in Fig.~\ref{fig:Fig_2}(a). Various pitchfork bifurcations lead to the emergence of stable and unstable states. Interestingly, some of the bifurcations involve purely unstable states and are not visible in a frequency sweep. We therefore dub them ``ghost bifurcations'' as they leave the deterministic steady state physics unaffected. Such ghost bifurcations separate regions I and II, as well as II and III. At each ghost bifurcation, additional unstable states emerge. To emphasize their origin, we refer to these unstable states as ``ghost states''. From comparing Fig.~\ref{fig:Fig_2}(c) and (e), it becomes clear that the emergence of the ghost states is accompanied by new switching paths, and that these changes impact $\Gamma$. Specifically, rather than acting as additional obstacles in the switching paths, the ghost states appear to favor an \textit{increase} in $\Gamma$.

To elucidate the surprising role of the ghost states, we study the transition paths $\mathbf{Y}_{\rm min}$ and the corresponding barriers $W$ for the different representative frequencies along the sweep shown in Fig.~\ref{fig:Fig_2}(a). To this end, we need to minimize  Eq.~\eqref{eq:Onsager_Machlup} which is a complex task. A simple variational scheme with equal timesteps gives inconsistent results and more advanced methods as the sgMAM-method~\cite{Grafke_2017} are necessary  to obtain the correct physical paths (see details in Appendix~\ref{sec:gamma_det}). %The numerical minimization of $S_{\rm OM}$ is explained in Appendix~\ref{sec:gamma_det}.

In Fig.~\ref{fig:Fig_2}(d), we show the calculated switching paths corresponding to the experimental parameters, which are representative for the regions (I-IV). In region I where the symmetric states are the only stable attractors, we find only one switching path, which passes through the intermediate 0-amplitude state in agreement with our experimentally observed distributions. As expected, this is 
equivalent to the single KPO case in Fig.~\ref{fig:Fig_1}, confirming that we should expect a monotonic decrease transition rate $\Gamma(f_d)$ with decreasing frequency as the stable attractors move apart~\cite{Dykman_1998,Chan_2008}.  
%we expect  to find switching rates $\Gamma$ between the two symmetric states that obey the same functional form. 

In region II, we find two additional switching paths that avoid the 0-amplitude state and instead pass through emergent unstable antisymmetric states. This is in line with the experimental data that exhibits a broader distribution around the 0-amplitude state, extending to the two unstable antisymmetric states. Similar paths also arise in regions III and IV, where the unstable states provide transient ledges where the system can hover during switching events, cf. Appendix~\ref{sec:antisymm_switching}. The additional switching paths are visible in the experiment and seem to be the dominant paths in regions III and IV. These alternative switching paths  bring forth a complexity to the system dynamics, a feature intrinsically related to the existence of multiple normal modes in KPO networks. In our system, we observe that the minimal path, $\mathbf{Y}_{\rm min}$, connects the stable states always via an unstable one in see Fig.~\ref{fig:Fig_2}(d)~\cite{Maier_1992, Tang_2017}, although exceptions have been observed~\cite{Luchinsky_1999, Feng_2014}. This underscores the importance of investigating stable as well as unstable states of the system in order to understand the stochastic dynamics of Ising networks. 

Based on our theoretical analysis of optimal switching paths, we can now obtain $W$ and compare it with that extracted from the experimental data, cf. Appendix~\ref{sec:gamma_det}. From this, we identify the dominant transition path, see Fig.~\ref{fig:Fig_2}(d).
%The bifurcation topography of the coupled system is more complex and new unstable states emerge at the ghost bifurcations.
%there are three unstable states: two antisymmetric solutions that are unstable in one direction (singly unstable), and the solution at 0-amplitude that is unstable in two directions (doubly unstable). Correspondingly, we find two additional switching paths that avoid the 0-amplitude state and rather pass through the antisymmetric ones. This is in line with the experimental data that exhibits a broader distribution around the 0-amplitude state, extending to the two unstable antisymmetric states. Similar paths appear also in region III and IV, where the additional states that emerged from the ghost bifurcations provide unstable `ledges' during switching events. The additional switching paths are visible in the experiment and seem to be the dominant paths in region III and IV. This alternative switching paths add a complexity to the system, which is not present in the single parametron case. We expect that this additional paths affect the transition rate.
%In region III, after the second ghost bifurcation, there is a distinct aggregation around the antisymmetric solutions, meaning that switches through these points are more likely than those through the 0-amplitude state.  This trend becomes even more pronounced in region IV, where the antisymmetric states bifurcate to become stable solutions. At this point, all states corresponding to an Ising system are nominally stable. Due to the large noise, however, the system can only dwell in the antisymmetric state for a short time. With decreasing $f_d$, the attracting forces towards the antisymmetric state become stronger and the dwell-time increases (not shown). 
% apply the sgMAM method~\cite{Grafke_2017} to  Eq.~\eqref{eq:Onsager_Machlup} of the coupled system with $\mathbf{Y}=(u_1,v_1,u_2,v_2)^T$, where we choose one of the symmetric states as the initial point and the other one as the final point, and try different unstable states as intermediate points. We thus obtain the corresponding locally-minimizing switching paths $\mathbf{Y}_{\rm min}$, see Fig.~\ref{fig:Fig5}(a).
%Doing so, we first verify that our microscopic stochastic model indeed captures the observed physics. We then simulate the coupled parametron system at different noise strengths, and verify that the activation rate satisfies the weak noise scaling form of Eq.~\eqref{eq:Gamma}~\cite{Stambaugh_2006}, see Appendix~\ref{sec:gamma_det}. In this limit, $\Gamma_0$ and $W$ are independent of the noise and purely depend on the properties of the non-stochastic system, and on the switching paths in phase space. This procedure allows us to extract $\Gamma_0$ and $W$ for the experimental and the numerical data [Fig.~\ref{fig:Fig5}(c)] as well as calculate $\Gamma$ at different noise strengths.
Within regions I-III, we find good qualitative and approximate quantitative agreement (save for a small overall shift). Interestingly, in region II, both the paths contribute equally to the activation: one via the antisymmetric ghost states and the other via the origin. However, this effect is not sufficiently strong to be observed in Fig.~\ref{fig:Fig_2}(c). In region III, the former overtakes the latter, which manifests as the observed kink in Fig.~\ref{fig:Fig_2}(c). This indicates that the ghost states support the antisymmetric switching, and markedly participate and modify the expected stochastic dynamics of the system. In region IV, both symmetric and antisymmetric phase states are stable. Here, the analytical calculation deviates from the experimental and numerical results. This deviation is likely due to the fact that the Onsager-Machlup method only considers switches that connect the two symmetrical states, while the counting algorithm that we used for the experimental and numerical data includes all possible switches between states. The latter includes repeated switches between a symmetric and an antisymmetric state as individual events.

%To summarize, we have shown the response of the two-KPO network to stochastic force noise, which is a basic building block for simulated annealing~\cite{Puri_2017_NC,Heim_2015,Goto_2019,Kirkpatrick_1983} in so-called `Ising machines'~\cite{Mahboob_2016,Inagaki_2016,Goto_2019}. The switching between a single pair of phase states is readily configurable and obeys a simple exponential scaling. Surprisingly, we see that the switching dynamics between Ising-type states of the system are severely impacted by the ghost bifurcations, as these dictate the path of least action that the system can take.



\section{Discussion}
%Our experimental and theoretical study bridges the gap from weak to strong coupling in parametron networks. We demonstrate the feasibility of realizing Ising simulators in this regime. 
At a fundamental level, our results demonstrate unambiguously the existence of an unusual type of bifurcation arising from the coupling between the individual resonators. Although these ghost bifurcations remain undetected in a deterministic system characterization~\cite{Heugel_2022}, they impact the inter-state switching path, switching rate $\Gamma$, and the relative dwell times in the symmetric and antisymmetric states. This inevitably affects the stochastic switching processes in KPO networks, and their characterization via stochastic sampling~\cite{margiani2022deterministic}.

Generalizing our results to $N$ identical resonators with identical all-to-all coupling, we theoretically predict that the ghost bifurcations, as well as the altered stability of the  symmetric/antisymmetric regions, persist. Such a network has a single symmetric lobe and a $N-1$-fold degenerate antisymmetric lobe~\cite{Heugel_2019_TC}. The boundaries delineating the overlapping region once more involve ghost bifurcations. However, one of the ghost bifurcations is now $N-1$ fold degenerate. As demonstrated above for two resonators, the antisymmetric states only become stable when they undergo another bifurcation, described by the general equation
\begin{equation}
\lambda_A = \frac{\sqrt{4 \gamma ^2 \omega ^2+\left(J (N+2)-2 \left(\omega ^2-\omega_0^2\right)\right)^2}}{\omega_0^2}\, ,
\end{equation}
resulting in an extended symmetric region. We thus expect a KPO network with identical all-to-all coupling to manifest the four different regimes of activation seen in the two-KPO case. When the coupling coefficient $J_{ij}$ is different for each resonator pair, the degeneracy is lifted. However, the instability lobes still overlap and generally form ghost bifurcations which will also impact the switching behaviour of the system.  In analogy to our observations in Fig.~\ref{fig:Fig_2}, many competing switching paths open up. Finding the dominant switching paths is a complex and demanding task. Our work provides the motivation for such future investigations.

All of our observations have important consequences for logic networks built from KPOs and nonlinear resonators in general, because they impact the solution that a network will choose after a finite transition time. 
With proper modelling and calibration of the ghost bifurcations, a network can be operated at a position in $f_d-\lambda$ space where the many-body character of the network is preserved and the annealing speed is optimized.
It becomes clear from our work that large networks bear very complex switching dynamics and a careful analysis of the bifurcation topology is very important. Future work might find ways to use this complexity in an advantageous manner to perform faster calculations.
%In the most extreme case, a computation at a particular point in $f_d-\lambda$ space might always produce the same state because all other states have unexpectedly become unstable. 
%Of course, such a computation would be of no value. Understanding the complexity of the solution space will help to avoid parameter regimes that host such `frozen' dynamics.

%Our work marks the realization of parametron Ising machines in the strong-coupling limit. With proper modelling and calibration of the ghost bifurcations and the mixed states, a network can be operated at a position in $f_d-\lambda$ space where the many-body character of the network is preserved; in our example, this would be the case in the small triangle between the solid brown and the dashed blue lines in Fig.~\ref{fig:Fig3}(c). This area is expected to become larger for weaker coupling $J$, or when accessing larger values of the parametric modulation depth $\lambda \propto U_d$. It becomes clear from our work that a careful analysis of the bifurcation topology of a network is indispensable for performing any meaningful calculation.

%Accordingly, we find smaller $S_{\rm OM}$ for the path via the anti-symmetric state. Because of the 'ghost bifurcation' new doubly unstable states emerge and block the switchings via the center. Thus the system switches  via the anti-symmetric states. 
%This means that switchings via the anti-symmetric state are more probable than switchings via the 0-amplitude state in agreement with the experiment. 

%In  region IV $S_{\rm OM}$  increases with decreasing $f$ opposed to the experiment and the simulation, since the semi-analytic calculation avoids the problem of counting false events. 
%In region IV the anti-symmetric states have bifurcated and become stable solutions. The probability to switch via the anti-symmetric states is still larger but the system additionally remains there for increasing time.

\section{Summary and Outlook} 
In summary, we experimentally and theoretically investigated the noise-induced dynamics of a system of two coupled nonlinear Kerr parametric oscillators (KPOs). Our study implements the smallest form of a KPO network, and tests the switching behavior in so-called Ising machines. 
We found that ghost bifurcations play an important role, with consequences for the switching dynamics of the system as it progresses towards its most stable configuration. A better understanding of such effects can be very helpful for the calibration to and for stochastic logic protocols, such as simulated annealing. As coupled networks of parametric resonators are one of the main candidates for future parallel computation architectures, our study provides crucial input for a growing subcommunity working towards classical and quantum analog computation~\cite{Gottesman_2001, Devoret_2013, Mahboob_2016, Inagaki_2016, Goto_2016, Puri_2019_PRX,Nigg_2017,Dykman_2018}. Furthermore, it provides additional incentives for the fundamental exploration of complex driven-dissipative nonlinear networks in a multitude of fields~\cite{DykmanBook}.



%and puts to the test whether Ising machines can operate in the strong-coupling regime. We reveal that such operation is possible, paving the way for potentially faster convergence of simulated annealing processes. At the same time, we find that the coupling leads to an unusual type of `ghost bifurcation' that reduces the number of available solutions in some regions of the stability diagram. 




\acknowledgments
This work received financial support from the Swiss National Science Foundation through grants (CRSII5\_206008/1) and (PP00P2\_163818), and the Deutsche Forschungsgemeinschaft (DFG) through project number 449653034 and SFB1432. We thank Peter M\"{a}rki, \v{Z}iga Nosan and Christian Marty for technical help.

%Additional points
% - n resonators, identical coupling: very similar instability diagram, two lobes. ghosts in overlapping region. anti-sym is degenerate -> many ghost-bifurcations on top of each other. "brown line" persists.
% - n resonators, non-identical coupling: degeneracy is lifted, there are still ghost bifurcations in overlapping region, series of many ghosts (degeneracy lifted). What happens to brown line?
% most stable solution?
% switching rate?
% connect to annealing?





 







% stuff for Appendix: counting algorithm


\appendix

\section{Single KPO}\label{sec_single_KPO} 

For $J=0$, each resonator can be driven into parametric resonance when $U_d \geq U_{th}$~\cite{Landau_Lifshitz, Lifshitz_Cross}. We characterize each resonator using frequency sweeps as described in~\cite{Leuch_2016,Heugel_2019_TC,Nosan_2019} and obtain the values $Q_1 = 295$, $f_0 = \SI{2.6784}{\mega\hertz}$, $\alpha_1 = \SI{-9e17}{\per\square\volt\per\square\second}$, and $U_{th} = \SI{1.21}{\volt}$.
Using Eqs.~\eqref{eq:slowflow}, we can describe the steady-state of the single KPO by applying the condition ($\dot{u}_1=\dot{v}_1=0$)~\cite{Papariello_2016, Leuch_2016,Eichler_Zilberberg_book,kovsata2022harmonicbalance}. This yields a quintic characteristic polynomial with up to three different stable solutions (attractors) in phase space, cf. Fig.~\ref{fig:Fig_Single_Steady}(b). As a function of $f_d$, the number of stable solutions changes at specific bifurcation points. In the single KPO, we only observe pitchfork bifurcations, which involve at least one stable solution, cf. Fig.~\ref{fig:Fig_Single_Steady}(c). 
In Fig.~\ref{fig:Fig_Single_Steady}(d) we show the characteristic parametric instability lobe. Region (i) accommodates only one stable state with amplitude 0.
Inside the region marked as (ii), the linear resonator becomes unstable, bifurcates, and settles into one of the two steady states that are stabilized by $\alpha$~\cite{Lifshitz_Cross}. These phase states have the same amplitude but are $\pi$-shifted in phase, cf. Fig.~\ref{fig:Fig_Single_Steady}(b). In region (iii), the phase states coexist with the amplitude 0 solution.

%In Fig.~\ref{fig:Fig2}(b), we show experimental sweeps with increasing and decreasing $f_d$ for constant $U_d$, exhibiting the standard nonlinear parametric response and hysteresis of the parametron labelled as 1 (similar results were obtained for parametron 2). 



\textit{Single KPO -- } In Fig.~\ref{fig:Fig_1} we inspect the noise-induced switching of a single KPO, whose properties are well known ~\cite{Dykman_1998, Ryvkine_2006, Chan_2008, Dykman_2018}.
%In Fig.~\ref{fig:Fig_1}(a), we show an example of a measured time trace of $u_1$ and $v_1$ in the presence of applied noise. The resonator mostly dwells in the vicinity of a given phase state and explores the local fluctuations of the specific attractor. Occasionally, the noise activates the system to switch to the other phase state. Collecting and plotting such a time trace in phase space yields a probability distribution which highlights both of these aspects, cf. Fig.~\ref{fig:Fig_Single_Noisy}(b). The weak fluctuations around the attractors activate the parametron fluctuation spectrum~\cite{PumpNoisyProbe}. We concentrate here on the switching rate $\Gamma$ between phase states, which we extract from the experimental data, cf.~Appendix~\ref{sec:gamma_det} and Fig.~\ref{fig:Fig_Single_Noisy}(c).
As expected, we find that $\Gamma$ decreases monotonically with increasing separation between the phase states, which we control here through $f_d$~\cite{Dykman_1998}. Similar results have been previously measured in other KPO implementations~\cite{Chan_2008}.

The monotonic decrease of $\Gamma$ in Fig.~\ref{fig:Fig_1}(c) is derived using the Onsager-Machlup approach~\cite{Dykman_1998}. Specifically, at low noise, the switching rate $\Gamma$ is dominated by the path $\mathbf{Y}_{\rm min}$ that minimizes $S_{\rm OM}$~\cite{Lehmann_2003}. 
Repeating this estimation as a function of $f_d$ and calculating $\Gamma_{\rm min}$ yields good agreement with the experimentally observed $\Gamma$, cf. Fig.~\ref{fig:Fig_1}(c). Note that the prefactor $\Gamma_0$ is not obtained by this method but reused from Ref.~\cite{Dykman_1998}, leading to a slight overall shift towards larger $\Gamma$. The analytical formula derived in Ref.~\cite{Dykman_1998} produces a similarly good agreement, cf. Eq.~\eqref{eq:analyticRate} in Appendix~\ref{sec:gamma_det}.

% Figure environment removed

\section{Fluctuating versus Coherent Signal Amplitude}\label{sec:fluctuating_coherent}
To obtain an optimal agreement between the measured switching rates and the theoretical predictions, we consistently found that the noise power spectral density in the model had to be a factor $\approx4.2$ smaller than the value applied in the experiment. This discrepancy is likely due to an additional attenuation of a factor 2 in the path of the fluctuating voltage, for instance a voltage division at a $\SI{50}{\ohm}$ matched input port. The fluctuating signal with power spectral density $S_{n}$ was provided by two dedicated voltage sources with the same output intensity and added to the coherent signal via the ADD channel of the Zurich Instruments HF2LI lock-in amplifier. The resulting noise process $\xi_i$ acting on our system has a power spectral density $\varsigma^2 = C_{in} S_{n}$, where the coefficient for the signal in-coupling efficiency is $C_{in} = \SI{4.93e-20}{\hertz^4}$ for the single-KPO experiment. For the two-KPO experiment, we find best agreement for a slightly lower value for $C_{in}$, which is probably due to differences in the coil geometry between the devices or between the experimental runs.

% Figure environment removed

% Figure environment removed

\section{Details on the calculations for noise-induced switching}\label{sec:gamma_det}
\textit{Determination of the switching rate -- } The experimental determination of the switching rate $\Gamma$ in Fig.~\ref{fig:Fig_2}(c) was performed with a lock-in amplifier (Zurich Instruments HF2LI). We used a sampling rate of \SI{450}{\hertz} and a total measurement time of \SI{300}{\second} for $f_d \leq \SI{2.3696}{\hertz}$ and \SI{60}{\second} for $f_d > \SI{2.3696}{\hertz}$. Counting of the switching events was done with a numerical algorithm that compared the amplitudes and phases of successive measurement points for an entire time trace measurement. Concretely, the program increased the switching counter by 1 if the phase difference of two successive points was above a `phase threshold' (\SI{130}{\degree}) while at least one of the points was above an `amplitude threshold' (\SI{0.5}{\milli\volt}), or if exactly one out of two successive points was above the amplitude threshold. The same algorithm was used to evaluate the switching rate in numerical simulations that emulated the measurements (including the effective sampling rate). Similar results were obtained by finding the maximal turning point of Allan deviations of the phase~\cite{margiani2022extracting}.



% Figure environment removed

\textit{Determination of the activation barrier $W$ -- }
We  simulate the coupled KPO system at different noise strengths, see Fig~\ref{fig:FigA5}, and we verify that our analysis is in the low-noise limit by showing that Eq.~\eqref{eq:Gamma} is obeyed.
The optimal fit (gray) yields $\Gamma_0 = \SI{2e4}{\hertz}$ and $W = \SI{0.024}{\hertz \volt^{2}}$.
In this limit, $\Gamma_0$ and $W$ are independent of the noise and purely depend on the properties of the non-stochastic system, and on the switching paths in phase space. This procedure allows us to extract $\Gamma_0$ and $W$ for the experimental and the numerical data  in Fig.~\ref{fig:Fig_2}(d), as well as calculate $\Gamma$ at different noise strengths.
For the numerical switching rate in Fig.~\ref{fig:Fig_2}(c), we used this scaling law to convert the numerical data, simulated at $1.3$ times stronger noise, to the experimentally applied noise.

\textit{Analytic expression for a single KPO -- }
For a single parametric Kerr oscillator, the switching rate was calculated in~\cite{Dykman_1998} and is given by
\begin{widetext}
\begin{equation}
\label{eq:analyticRate}
    \Gamma = \frac{ \left(\gamma  \sqrt{\frac{\lambda ^2 \omega_0^4}{\gamma ^2 \omega ^2}-4}-4 \omega +4 \omega_0\right) \sqrt{\left| 1-\frac{\lambda ^2 \omega_0^4}{4 \gamma ^2 \omega ^2}\right| } \exp \left(-\frac{ \gamma ^2 \omega ^3 \left(\gamma  \sqrt{\frac{\lambda ^2 \omega_0^4}{\gamma ^2 \omega ^2}-4}-4 \omega +4 \omega_0\right)^2 \sqrt{\left| 1-\frac{\lambda ^2 \omega_0^4}{4 \gamma ^2 \omega ^2}\right| }}{3 \alpha  \lambda ^2 \sigma ^2 \omega_0^4}\right)}{2 \sqrt{2} \pi   }\,.
\end{equation}
\end{widetext}

\textit{Details on the path optimization -- }
The switching rate $\Gamma$ can be described by $S_{\rm OM}[\mathbf{Y}_{\rm min}]$.  Obtaining the minimal path $\mathbf{Y}_{\rm min}$ is a complex task for coupled parametric resonators. As a simple variation of a discretized path with equal timesteps fails to obtain correct results, we use the sgMAM method~\cite{Grafke_2017}. It is an improved path optimization scheme based on  scaled time, leading to consistent converged results. We start with a guessed initial path that connects two stable states via an unstable attractor. Then, we perform numerical minimization of $S_{\rm OM}$ by varying the path in phase space between the chosen end points.  
For the single KPO, we choose the two phase states as initial and final state, see Fig~\ref{fig:Fig_1}(b). 
In the coupled system with $\mathbf{Y}=(u_1,v_1,u_2,v_2)^T$, we choose one of the symmetric states as the initial point and the other one as the final point, and try different unstable states as intermediate points. We thus obtain the corresponding locally-minimizing switching paths $\mathbf{Y}_{\rm min}$, see Fig.~\ref{fig:Fig_2}(d).



\section{Switching via the antisymmetric state}\label{sec:antisymm_switching}

In Fig.~\ref{fig:FigA4}, we show examples of timetraces during noise-induced switching between symmetric states. Figure~\ref{fig:FigA4}(a) corresponds to $f_d=\SI{2.37}{\mega\hertz}$ in region I of Fig.~\ref{fig:Fig_2}(c), where switches occur via the unstable 0-amplitude state because both resonators switch synchronously. In Fig.~\ref{fig:FigA4}(b), we show an example for $f_d=\SI{2.36}{\mega\hertz}$ in region IV, where the two resonators switch with a finite delay. In the short time interval between the two switches, the system dwells in the antisymmetric state.






%\begin{itemize}
%	\item We consider two coupled parametric resonators which are expected to form symmetric and anti-symmetric phase-states in their corresponding instability lobes. (Fig. 1a, upper part)
%	\begin{itemize}
%		\item show EOM
%		\item single resonator
%		\item low coupling strength arguments
%		\item refer to our DTC paper
%	\end{itemize}
%\paragraph{Model.} Two coupled parametrically driven resonators with Duffing nonlinearity $\alpha$ and damping $\gamma$ are described by the equations of motion:
%\begin{align}
%    \ddot{x}_i + \omega_0^2\left( 1-\lambda \cos(2 \omega t) \right) x_i + \gamma x_i + \alpha x_i^3 - J x_j = 0 \,
%\end{align}
%for $i=1,2$ and $j=2,1$. $x_i$ is the position of resonator $i = 1,2$, the dot indicates differentiation w.r.t. time, $\omega_0$ is the eigenfrequency, $\lambda$ the driving strength, $2\omega$ the driving frequency and $J$ the coupling strength.	
%Without nonlinearity the system of equations separates for the symmetric and anti-symmetric normal mode. For driving strengths above a certain threshold $\lambda_{\rm S,A} = ...$ the corresponding mode becomes exponentially unstable and is only stabilized by $\alpha$~\cite{us}. Within this instability regions the corresponding modes form phase states of same amplitude but $\pi$-shifted phase. When the two regions overlap there are both symmetric and anti-symmetric steady states. 








%% Figure environment removed

%For this purpose we calculate the power spectral density (PSD) of the symmetric and the anti-symmetric coordinates. In both cases we find a Lorentzian shaped peak but at different frequencies. The experimental PSD of the symmetric fluctuations matches the numerical and the analytical PSD very well while there is a small difference for the anti-symmetric one. The peak frequency (width/lifetime) is determined by the imaginary (real) part of the eigenvalues of the linearized Eq.~\eqref{eq:slowflow}. The corresponding eigenvectors determine the symmetry of the fluctuations, i.e. symmetric and anti-symmetric. In Fig~\ref{fig:FigA2}(b) the theoretically calculated eigenvalues of the symmetric states are shown as a function of frequency. For the symmetric fluctuations we observe that the imaginary part goes to 0 while the real part splits. Here, the fluctuations are critically damped and go from an underdamped to an overdamped motion. The anti-symmetric fluctuations always show underdamped characteristics. When the real part of the symmetric fluctuations changes sign the symmetric solution disappears as the right boundary of the instability lobe is crossed.


\bibliographystyle{apsrev4-1}
%\bibliography{references}

%merlin.mbs apsrev4-1.bst 2010-07-25 4.21a (PWD, AO, DPC) hacked
%Control: key (0)
%Control: author (72) initials jnrlst
%Control: editor formatted (1) identically to author
%Control: production of article title (-1) disabled
%Control: page (0) single
%Control: year (1) truncated
%Control: production of eprint (0) enabled
\begin{thebibliography}{70}%
\makeatletter
\providecommand \@ifxundefined [1]{%
 \@ifx{#1\undefined}
}%
\providecommand \@ifnum [1]{%
 \ifnum #1\expandafter \@firstoftwo
 \else \expandafter \@secondoftwo
 \fi
}%
\providecommand \@ifx [1]{%
 \ifx #1\expandafter \@firstoftwo
 \else \expandafter \@secondoftwo
 \fi
}%
\providecommand \natexlab [1]{#1}%
\providecommand \enquote  [1]{``#1''}%
\providecommand \bibnamefont  [1]{#1}%
\providecommand \bibfnamefont [1]{#1}%
\providecommand \citenamefont [1]{#1}%
\providecommand \href@noop [0]{\@secondoftwo}%
\providecommand \href [0]{\begingroup \@sanitize@url \@href}%
\providecommand \@href[1]{\@@startlink{#1}\@@href}%
\providecommand \@@href[1]{\endgroup#1\@@endlink}%
\providecommand \@sanitize@url [0]{\catcode `\\12\catcode `\$12\catcode
  `\&12\catcode `\#12\catcode `\^12\catcode `\_12\catcode `\%12\relax}%
\providecommand \@@startlink[1]{}%
\providecommand \@@endlink[0]{}%
\providecommand \url  [0]{\begingroup\@sanitize@url \@url }%
\providecommand \@url [1]{\endgroup\@href {#1}{\urlprefix }}%
\providecommand \urlprefix  [0]{URL }%
\providecommand \Eprint [0]{\href }%
\providecommand \doibase [0]{http://dx.doi.org/}%
\providecommand \selectlanguage [0]{\@gobble}%
\providecommand \bibinfo  [0]{\@secondoftwo}%
\providecommand \bibfield  [0]{\@secondoftwo}%
\providecommand \translation [1]{[#1]}%
\providecommand \BibitemOpen [0]{}%
\providecommand \bibitemStop [0]{}%
\providecommand \bibitemNoStop [0]{.\EOS\space}%
\providecommand \EOS [0]{\spacefactor3000\relax}%
\providecommand \BibitemShut  [1]{\csname bibitem#1\endcsname}%
\let\auto@bib@innerbib\@empty
%</preamble>
\bibitem [{\citenamefont {Kramers}(1940)}]{Kramers1940}%
  \BibitemOpen
  \bibfield  {author} {\bibinfo {author} {\bibfnamefont {H.}~\bibnamefont
  {Kramers}},\ }\href {\doibase https://doi.org/10.1016/S0031-8914(40)90098-2}
  {\bibfield  {journal} {\bibinfo  {journal} {Physica}\ }\textbf {\bibinfo
  {volume} {7}},\ \bibinfo {pages} {284} (\bibinfo {year} {1940})}\BibitemShut
  {NoStop}%
\bibitem [{\citenamefont {H\"anggi}\ \emph {et~al.}(1990)\citenamefont
  {H\"anggi}, \citenamefont {Talkner},\ and\ \citenamefont
  {Borkovec}}]{Hanggi_1990}%
  \BibitemOpen
  \bibfield  {author} {\bibinfo {author} {\bibfnamefont {P.}~\bibnamefont
  {H\"anggi}}, \bibinfo {author} {\bibfnamefont {P.}~\bibnamefont {Talkner}}, \
  and\ \bibinfo {author} {\bibfnamefont {M.}~\bibnamefont {Borkovec}},\ }\href
  {\doibase 10.1103/RevModPhys.62.251} {\bibfield  {journal} {\bibinfo
  {journal} {Rev. Mod. Phys.}\ }\textbf {\bibinfo {volume} {62}},\ \bibinfo
  {pages} {251} (\bibinfo {year} {1990})}\BibitemShut {NoStop}%
\bibitem [{\citenamefont {Rondin}\ \emph
  {et~al.}(2017{\natexlab{a}})\citenamefont {Rondin}, \citenamefont {Gieseler},
  \citenamefont {Ricci}, \citenamefont {Quidant}, \citenamefont {Dellago},\
  and\ \citenamefont {Novotny}}]{rondin2017direct}%
  \BibitemOpen
  \bibfield  {author} {\bibinfo {author} {\bibfnamefont {L.}~\bibnamefont
  {Rondin}}, \bibinfo {author} {\bibfnamefont {J.}~\bibnamefont {Gieseler}},
  \bibinfo {author} {\bibfnamefont {F.}~\bibnamefont {Ricci}}, \bibinfo
  {author} {\bibfnamefont {R.}~\bibnamefont {Quidant}}, \bibinfo {author}
  {\bibfnamefont {C.}~\bibnamefont {Dellago}}, \ and\ \bibinfo {author}
  {\bibfnamefont {L.}~\bibnamefont {Novotny}},\ }\href@noop {} {\bibfield
  {journal} {\bibinfo  {journal} {Nature nanotechnology}\ }\textbf {\bibinfo
  {volume} {12}},\ \bibinfo {pages} {1130} (\bibinfo {year}
  {2017}{\natexlab{a}})}\BibitemShut {NoStop}%
\bibitem [{\citenamefont {Best}\ and\ \citenamefont {Hummer}(2006)}]{Best2006}%
  \BibitemOpen
  \bibfield  {author} {\bibinfo {author} {\bibfnamefont {R.~B.}\ \bibnamefont
  {Best}}\ and\ \bibinfo {author} {\bibfnamefont {G.}~\bibnamefont {Hummer}},\
  }\href {\doibase 10.1103/PhysRevLett.96.228104} {\bibfield  {journal}
  {\bibinfo  {journal} {Phys. Rev. Lett.}\ }\textbf {\bibinfo {volume} {96}},\
  \bibinfo {pages} {228104} (\bibinfo {year} {2006})}\BibitemShut {NoStop}%
\bibitem [{\citenamefont {Chung}\ \emph {et~al.}(2015)\citenamefont {Chung},
  \citenamefont {Piana-Agostinetti}, \citenamefont {Shaw},\ and\ \citenamefont
  {Eaton}}]{Chung2015}%
  \BibitemOpen
  \bibfield  {author} {\bibinfo {author} {\bibfnamefont {H.~S.}\ \bibnamefont
  {Chung}}, \bibinfo {author} {\bibfnamefont {S.}~\bibnamefont
  {Piana-Agostinetti}}, \bibinfo {author} {\bibfnamefont {D.~E.}\ \bibnamefont
  {Shaw}}, \ and\ \bibinfo {author} {\bibfnamefont {W.~A.}\ \bibnamefont
  {Eaton}},\ }\href {\doibase 10.1126/science.aab1369} {\bibfield  {journal}
  {\bibinfo  {journal} {Science}\ }\textbf {\bibinfo {volume} {349}},\ \bibinfo
  {pages} {1504} (\bibinfo {year} {2015})},\ \Eprint
  {http://arxiv.org/abs/https://www.science.org/doi/pdf/10.1126/science.aab1369}
  {https://www.science.org/doi/pdf/10.1126/science.aab1369} \BibitemShut
  {NoStop}%
\bibitem [{\citenamefont {Garc\'{\i}a-M\"uller}\ \emph
  {et~al.}(2008)\citenamefont {Garc\'{\i}a-M\"uller}, \citenamefont {Borondo},
  \citenamefont {Hernandez},\ and\ \citenamefont {Benito}}]{GarciaMuller2008}%
  \BibitemOpen
  \bibfield  {author} {\bibinfo {author} {\bibfnamefont {P.~L.}\ \bibnamefont
  {Garc\'{\i}a-M\"uller}}, \bibinfo {author} {\bibfnamefont {F.}~\bibnamefont
  {Borondo}}, \bibinfo {author} {\bibfnamefont {R.}~\bibnamefont {Hernandez}},
  \ and\ \bibinfo {author} {\bibfnamefont {R.~M.}\ \bibnamefont {Benito}},\
  }\href {\doibase 10.1103/PhysRevLett.101.178302} {\bibfield  {journal}
  {\bibinfo  {journal} {Phys. Rev. Lett.}\ }\textbf {\bibinfo {volume} {101}},\
  \bibinfo {pages} {178302} (\bibinfo {year} {2008})}\BibitemShut {NoStop}%
\bibitem [{\citenamefont {Badzey}\ and\ \citenamefont
  {Mohanty}(2005)}]{Badzey2005}%
  \BibitemOpen
  \bibfield  {author} {\bibinfo {author} {\bibfnamefont {R.~L.}\ \bibnamefont
  {Badzey}}\ and\ \bibinfo {author} {\bibfnamefont {P.}~\bibnamefont
  {Mohanty}},\ }\href {\doibase 10.1038/nature04124} {\bibfield  {journal}
  {\bibinfo  {journal} {Nature}\ }\textbf {\bibinfo {volume} {437}},\ \bibinfo
  {pages} {995} (\bibinfo {year} {2005})}\BibitemShut {NoStop}%
\bibitem [{\citenamefont {Rondin}\ \emph
  {et~al.}(2017{\natexlab{b}})\citenamefont {Rondin}, \citenamefont {Gieseler},
  \citenamefont {Ricci}, \citenamefont {Quidant}, \citenamefont {Dellago},\
  and\ \citenamefont {Novotny}}]{Rondin2017}%
  \BibitemOpen
  \bibfield  {author} {\bibinfo {author} {\bibfnamefont {L.}~\bibnamefont
  {Rondin}}, \bibinfo {author} {\bibfnamefont {J.}~\bibnamefont {Gieseler}},
  \bibinfo {author} {\bibfnamefont {F.}~\bibnamefont {Ricci}}, \bibinfo
  {author} {\bibfnamefont {R.}~\bibnamefont {Quidant}}, \bibinfo {author}
  {\bibfnamefont {C.}~\bibnamefont {Dellago}}, \ and\ \bibinfo {author}
  {\bibfnamefont {L.}~\bibnamefont {Novotny}},\ }\href {\doibase
  10.1038/nnano.2017.198} {\bibfield  {journal} {\bibinfo  {journal} {Nature
  Nanotechnology}\ }\textbf {\bibinfo {volume} {12}},\ \bibinfo {pages} {1130}
  (\bibinfo {year} {2017}{\natexlab{b}})}\BibitemShut {NoStop}%
\bibitem [{\citenamefont {Fulton}\ and\ \citenamefont
  {Dunkleberger}(1974)}]{Fulton1974}%
  \BibitemOpen
  \bibfield  {author} {\bibinfo {author} {\bibfnamefont {T.~A.}\ \bibnamefont
  {Fulton}}\ and\ \bibinfo {author} {\bibfnamefont {L.~N.}\ \bibnamefont
  {Dunkleberger}},\ }\href {\doibase 10.1103/PhysRevB.9.4760} {\bibfield
  {journal} {\bibinfo  {journal} {Phys. Rev. B}\ }\textbf {\bibinfo {volume}
  {9}},\ \bibinfo {pages} {4760} (\bibinfo {year} {1974})}\BibitemShut
  {NoStop}%
\bibitem [{\citenamefont {Silvestrini}\ \emph {et~al.}(1988)\citenamefont
  {Silvestrini}, \citenamefont {Pagano}, \citenamefont {Cristiano},
  \citenamefont {Liengme},\ and\ \citenamefont {Gray}}]{Silvestrini1988}%
  \BibitemOpen
  \bibfield  {author} {\bibinfo {author} {\bibfnamefont {P.}~\bibnamefont
  {Silvestrini}}, \bibinfo {author} {\bibfnamefont {S.}~\bibnamefont {Pagano}},
  \bibinfo {author} {\bibfnamefont {R.}~\bibnamefont {Cristiano}}, \bibinfo
  {author} {\bibfnamefont {O.}~\bibnamefont {Liengme}}, \ and\ \bibinfo
  {author} {\bibfnamefont {K.~E.}\ \bibnamefont {Gray}},\ }\href {\doibase
  10.1103/PhysRevLett.60.844} {\bibfield  {journal} {\bibinfo  {journal} {Phys.
  Rev. Lett.}\ }\textbf {\bibinfo {volume} {60}},\ \bibinfo {pages} {844}
  (\bibinfo {year} {1988})}\BibitemShut {NoStop}%
\bibitem [{\citenamefont {Dykman}\ \emph {et~al.}(1998)\citenamefont {Dykman},
  \citenamefont {Maloney}, \citenamefont {Smelyanskiy},\ and\ \citenamefont
  {Silverstein}}]{Dykman_1998}%
  \BibitemOpen
  \bibfield  {author} {\bibinfo {author} {\bibfnamefont {M.~I.}\ \bibnamefont
  {Dykman}}, \bibinfo {author} {\bibfnamefont {C.~M.}\ \bibnamefont {Maloney}},
  \bibinfo {author} {\bibfnamefont {V.~N.}\ \bibnamefont {Smelyanskiy}}, \ and\
  \bibinfo {author} {\bibfnamefont {M.}~\bibnamefont {Silverstein}},\ }\href
  {\doibase 10.1103/PhysRevE.57.5202} {\bibfield  {journal} {\bibinfo
  {journal} {Phys. Rev. E}\ }\textbf {\bibinfo {volume} {57}},\ \bibinfo
  {pages} {5202} (\bibinfo {year} {1998})}\BibitemShut {NoStop}%
\bibitem [{\citenamefont {Dykman}(2007)}]{Dykman_2007}%
  \BibitemOpen
  \bibfield  {author} {\bibinfo {author} {\bibfnamefont {M.~I.}\ \bibnamefont
  {Dykman}},\ }\href {\doibase 10.1103/PhysRevE.75.011101} {\bibfield
  {journal} {\bibinfo  {journal} {Phys. Rev. E}\ }\textbf {\bibinfo {volume}
  {75}},\ \bibinfo {pages} {011101} (\bibinfo {year} {2007})}\BibitemShut
  {NoStop}%
\bibitem [{\citenamefont {Tadokoro}\ \emph {et~al.}(2020)\citenamefont
  {Tadokoro}, \citenamefont {Tanaka},\ and\ \citenamefont
  {Dykman}}]{tadokoro2020noise}%
  \BibitemOpen
  \bibfield  {author} {\bibinfo {author} {\bibfnamefont {Y.}~\bibnamefont
  {Tadokoro}}, \bibinfo {author} {\bibfnamefont {H.}~\bibnamefont {Tanaka}}, \
  and\ \bibinfo {author} {\bibfnamefont {M.}~\bibnamefont {Dykman}},\
  }\href@noop {} {\bibfield  {journal} {\bibinfo  {journal} {Scientific
  Reports}\ }\textbf {\bibinfo {volume} {10}},\ \bibinfo {pages} {10413}
  (\bibinfo {year} {2020})}\BibitemShut {NoStop}%
\bibitem [{\citenamefont {Luchinsky}\ \emph
  {et~al.}(1999{\natexlab{a}})\citenamefont {Luchinsky}, \citenamefont {Maier},
  \citenamefont {Mannella}, \citenamefont {McClintock},\ and\ \citenamefont
  {Stein}}]{Luchinsky1999}%
  \BibitemOpen
  \bibfield  {author} {\bibinfo {author} {\bibfnamefont {D.~G.}\ \bibnamefont
  {Luchinsky}}, \bibinfo {author} {\bibfnamefont {R.~S.}\ \bibnamefont
  {Maier}}, \bibinfo {author} {\bibfnamefont {R.}~\bibnamefont {Mannella}},
  \bibinfo {author} {\bibfnamefont {P.~V.~E.}\ \bibnamefont {McClintock}}, \
  and\ \bibinfo {author} {\bibfnamefont {D.~L.}\ \bibnamefont {Stein}},\ }\href
  {\doibase 10.1103/PhysRevLett.82.1806} {\bibfield  {journal} {\bibinfo
  {journal} {Phys. Rev. Lett.}\ }\textbf {\bibinfo {volume} {82}},\ \bibinfo
  {pages} {1806} (\bibinfo {year} {1999}{\natexlab{a}})}\BibitemShut {NoStop}%
\bibitem [{\citenamefont {Lapidus}\ \emph {et~al.}(1999)\citenamefont
  {Lapidus}, \citenamefont {Enzer},\ and\ \citenamefont {Gabrielse}}]{lapidus}%
  \BibitemOpen
  \bibfield  {author} {\bibinfo {author} {\bibfnamefont {L.~J.}\ \bibnamefont
  {Lapidus}}, \bibinfo {author} {\bibfnamefont {D.}~\bibnamefont {Enzer}}, \
  and\ \bibinfo {author} {\bibfnamefont {G.}~\bibnamefont {Gabrielse}},\ }\href
  {\doibase 10.1103/PhysRevLett.83.899} {\bibfield  {journal} {\bibinfo
  {journal} {Phys. Rev. Lett.}\ }\textbf {\bibinfo {volume} {83}},\ \bibinfo
  {pages} {899} (\bibinfo {year} {1999})}\BibitemShut {NoStop}%
\bibitem [{\citenamefont {Kim}\ \emph {et~al.}(2005)\citenamefont {Kim},
  \citenamefont {Heo}, \citenamefont {Lee}, \citenamefont {Ha}, \citenamefont
  {Jang}, \citenamefont {Noh},\ and\ \citenamefont {Jhe}}]{kkim2005}%
  \BibitemOpen
  \bibfield  {author} {\bibinfo {author} {\bibfnamefont {K.}~\bibnamefont
  {Kim}}, \bibinfo {author} {\bibfnamefont {M.-S.}\ \bibnamefont {Heo}},
  \bibinfo {author} {\bibfnamefont {K.-H.}\ \bibnamefont {Lee}}, \bibinfo
  {author} {\bibfnamefont {H.-J.}\ \bibnamefont {Ha}}, \bibinfo {author}
  {\bibfnamefont {K.}~\bibnamefont {Jang}}, \bibinfo {author} {\bibfnamefont
  {H.-R.}\ \bibnamefont {Noh}}, \ and\ \bibinfo {author} {\bibfnamefont
  {W.}~\bibnamefont {Jhe}},\ }\href {\doibase 10.1103/PhysRevA.72.053402}
  {\bibfield  {journal} {\bibinfo  {journal} {Phys. Rev. A}\ }\textbf {\bibinfo
  {volume} {72}},\ \bibinfo {pages} {053402} (\bibinfo {year}
  {2005})}\BibitemShut {NoStop}%
\bibitem [{\citenamefont {Aldridge}\ and\ \citenamefont
  {Cleland}(2005)}]{Aldridge2005}%
  \BibitemOpen
  \bibfield  {author} {\bibinfo {author} {\bibfnamefont {J.~S.}\ \bibnamefont
  {Aldridge}}\ and\ \bibinfo {author} {\bibfnamefont {A.~N.}\ \bibnamefont
  {Cleland}},\ }\href {\doibase 10.1103/PhysRevLett.94.156403} {\bibfield
  {journal} {\bibinfo  {journal} {Phys. Rev. Lett.}\ }\textbf {\bibinfo
  {volume} {94}},\ \bibinfo {pages} {156403} (\bibinfo {year}
  {2005})}\BibitemShut {NoStop}%
\bibitem [{\citenamefont {Chan}\ and\ \citenamefont
  {Stambaugh}(2007)}]{Chan_2007}%
  \BibitemOpen
  \bibfield  {author} {\bibinfo {author} {\bibfnamefont {H.~B.}\ \bibnamefont
  {Chan}}\ and\ \bibinfo {author} {\bibfnamefont {C.}~\bibnamefont
  {Stambaugh}},\ }\href {\doibase 10.1103/PhysRevLett.99.060601} {\bibfield
  {journal} {\bibinfo  {journal} {Phys. Rev. Lett.}\ }\textbf {\bibinfo
  {volume} {99}},\ \bibinfo {pages} {060601} (\bibinfo {year}
  {2007})}\BibitemShut {NoStop}%
\bibitem [{\citenamefont {Chan}\ \emph {et~al.}(2008)\citenamefont {Chan},
  \citenamefont {Dykman},\ and\ \citenamefont {Stambaugh}}]{Chan_2008}%
  \BibitemOpen
  \bibfield  {author} {\bibinfo {author} {\bibfnamefont {H.~B.}\ \bibnamefont
  {Chan}}, \bibinfo {author} {\bibfnamefont {M.~I.}\ \bibnamefont {Dykman}}, \
  and\ \bibinfo {author} {\bibfnamefont {C.}~\bibnamefont {Stambaugh}},\ }\href
  {\doibase 10.1103/PhysRevLett.100.130602} {\bibfield  {journal} {\bibinfo
  {journal} {Phys. Rev. Lett.}\ }\textbf {\bibinfo {volume} {100}},\ \bibinfo
  {pages} {130602} (\bibinfo {year} {2008})}\BibitemShut {NoStop}%
\bibitem [{\citenamefont {Venstra}\ \emph {et~al.}(2013)\citenamefont
  {Venstra}, \citenamefont {Westra},\ and\ \citenamefont {Van
  Der~Zant}}]{Venstra2013}%
  \BibitemOpen
  \bibfield  {author} {\bibinfo {author} {\bibfnamefont {W.~J.}\ \bibnamefont
  {Venstra}}, \bibinfo {author} {\bibfnamefont {H.~J.}\ \bibnamefont {Westra}},
  \ and\ \bibinfo {author} {\bibfnamefont {H.~S.}\ \bibnamefont {Van
  Der~Zant}},\ }\href@noop {} {\bibfield  {journal} {\bibinfo  {journal}
  {Nature communications}\ }\textbf {\bibinfo {volume} {4}},\ \bibinfo {pages}
  {1} (\bibinfo {year} {2013})}\BibitemShut {NoStop}%
\bibitem [{\citenamefont {Mahboob}\ \emph {et~al.}(2014)\citenamefont
  {Mahboob}, \citenamefont {Mounaix}, \citenamefont {Nishiguchi}, \citenamefont
  {Fujiwara},\ and\ \citenamefont {Yamaguchi}}]{Mahboob_2014_2}%
  \BibitemOpen
  \bibfield  {author} {\bibinfo {author} {\bibfnamefont {I.}~\bibnamefont
  {Mahboob}}, \bibinfo {author} {\bibfnamefont {M.}~\bibnamefont {Mounaix}},
  \bibinfo {author} {\bibfnamefont {K.}~\bibnamefont {Nishiguchi}}, \bibinfo
  {author} {\bibfnamefont {A.}~\bibnamefont {Fujiwara}}, \ and\ \bibinfo
  {author} {\bibfnamefont {H.}~\bibnamefont {Yamaguchi}},\ }\href@noop {}
  {\bibfield  {journal} {\bibinfo  {journal} {Scientific Reports}\ }\textbf
  {\bibinfo {volume} {4}},\ \bibinfo {pages} {4448} (\bibinfo {year}
  {2014})}\BibitemShut {NoStop}%
\bibitem [{\citenamefont {Margiani}\ \emph {et~al.}(2022)\citenamefont
  {Margiani}, \citenamefont {Guerrero}, \citenamefont {Heugel}, \citenamefont
  {Marty}, \citenamefont {Pachlatko}, \citenamefont {Gisler}, \citenamefont
  {Vukasin}, \citenamefont {Kwon}, \citenamefont {Miller}, \citenamefont
  {Bousse} \emph {et~al.}}]{margiani2022extracting}%
  \BibitemOpen
  \bibfield  {author} {\bibinfo {author} {\bibfnamefont {G.}~\bibnamefont
  {Margiani}}, \bibinfo {author} {\bibfnamefont {S.}~\bibnamefont {Guerrero}},
  \bibinfo {author} {\bibfnamefont {T.~L.}\ \bibnamefont {Heugel}}, \bibinfo
  {author} {\bibfnamefont {C.}~\bibnamefont {Marty}}, \bibinfo {author}
  {\bibfnamefont {R.}~\bibnamefont {Pachlatko}}, \bibinfo {author}
  {\bibfnamefont {T.}~\bibnamefont {Gisler}}, \bibinfo {author} {\bibfnamefont
  {G.~D.}\ \bibnamefont {Vukasin}}, \bibinfo {author} {\bibfnamefont {H.-K.}\
  \bibnamefont {Kwon}}, \bibinfo {author} {\bibfnamefont {J.~M.}\ \bibnamefont
  {Miller}}, \bibinfo {author} {\bibfnamefont {N.~E.}\ \bibnamefont {Bousse}},
  \emph {et~al.},\ }\href@noop {} {\bibfield  {journal} {\bibinfo  {journal}
  {Applied Physics Letters}\ }\textbf {\bibinfo {volume} {121}},\ \bibinfo
  {pages} {164101} (\bibinfo {year} {2022})}\BibitemShut {NoStop}%
\bibitem [{\citenamefont {Frimmer}\ \emph {et~al.}(2019)\citenamefont
  {Frimmer}, \citenamefont {Heugel}, \citenamefont {Nosan}, \citenamefont
  {Tebbenjohanns}, \citenamefont {H\"alg}, \citenamefont {Akin}, \citenamefont
  {Degen}, \citenamefont {Novotny}, \citenamefont {Chitra}, \citenamefont
  {Zilberberg},\ and\ \citenamefont {Eichler}}]{Frimmer_2019}%
  \BibitemOpen
  \bibfield  {author} {\bibinfo {author} {\bibfnamefont {M.}~\bibnamefont
  {Frimmer}}, \bibinfo {author} {\bibfnamefont {T.~L.}\ \bibnamefont {Heugel}},
  \bibinfo {author} {\bibfnamefont {i.~c.~v.}\ \bibnamefont {Nosan}}, \bibinfo
  {author} {\bibfnamefont {F.}~\bibnamefont {Tebbenjohanns}}, \bibinfo {author}
  {\bibfnamefont {D.}~\bibnamefont {H\"alg}}, \bibinfo {author} {\bibfnamefont
  {A.}~\bibnamefont {Akin}}, \bibinfo {author} {\bibfnamefont {C.~L.}\
  \bibnamefont {Degen}}, \bibinfo {author} {\bibfnamefont {L.}~\bibnamefont
  {Novotny}}, \bibinfo {author} {\bibfnamefont {R.}~\bibnamefont {Chitra}},
  \bibinfo {author} {\bibfnamefont {O.}~\bibnamefont {Zilberberg}}, \ and\
  \bibinfo {author} {\bibfnamefont {A.}~\bibnamefont {Eichler}},\ }\href
  {\doibase 10.1103/PhysRevLett.123.254102} {\bibfield  {journal} {\bibinfo
  {journal} {Phys. Rev. Lett.}\ }\textbf {\bibinfo {volume} {123}},\ \bibinfo
  {pages} {254102} (\bibinfo {year} {2019})}\BibitemShut {NoStop}%
\bibitem [{\citenamefont {Ryvkine}\ and\ \citenamefont
  {Dykman}(2006)}]{Ryvkine_2006}%
  \BibitemOpen
  \bibfield  {author} {\bibinfo {author} {\bibfnamefont {D.}~\bibnamefont
  {Ryvkine}}\ and\ \bibinfo {author} {\bibfnamefont {M.~I.}\ \bibnamefont
  {Dykman}},\ }\href@noop {} {\bibfield  {journal} {\bibinfo  {journal}
  {Physical Review E}\ }\textbf {\bibinfo {volume} {74}},\ \bibinfo {pages}
  {061118} (\bibinfo {year} {2006})}\BibitemShut {NoStop}%
\bibitem [{\citenamefont {Mahboob}\ and\ \citenamefont
  {Yamaguchi}(2008)}]{Mahboob_2008}%
  \BibitemOpen
  \bibfield  {author} {\bibinfo {author} {\bibfnamefont {I.}~\bibnamefont
  {Mahboob}}\ and\ \bibinfo {author} {\bibfnamefont {H.}~\bibnamefont
  {Yamaguchi}},\ }\href {https://doi.org/10.1038/nnano.2008.84} {\bibfield
  {journal} {\bibinfo  {journal} {Nature Nanotechnology}\ }\textbf {\bibinfo
  {volume} {3}},\ \bibinfo {pages} {275} (\bibinfo {year} {2008})}\BibitemShut
  {NoStop}%
\bibitem [{\citenamefont {Wilson}\ \emph {et~al.}(2010)\citenamefont {Wilson},
  \citenamefont {Duty}, \citenamefont {Sandberg}, \citenamefont {Persson},
  \citenamefont {Shumeiko},\ and\ \citenamefont {Delsing}}]{Wilson_2010}%
  \BibitemOpen
  \bibfield  {author} {\bibinfo {author} {\bibfnamefont {C.~M.}\ \bibnamefont
  {Wilson}}, \bibinfo {author} {\bibfnamefont {T.}~\bibnamefont {Duty}},
  \bibinfo {author} {\bibfnamefont {M.}~\bibnamefont {Sandberg}}, \bibinfo
  {author} {\bibfnamefont {F.}~\bibnamefont {Persson}}, \bibinfo {author}
  {\bibfnamefont {V.}~\bibnamefont {Shumeiko}}, \ and\ \bibinfo {author}
  {\bibfnamefont {P.}~\bibnamefont {Delsing}},\ }\href {\doibase
  10.1103/PhysRevLett.105.233907} {\bibfield  {journal} {\bibinfo  {journal}
  {Phys. Rev. Lett.}\ }\textbf {\bibinfo {volume} {105}},\ \bibinfo {pages}
  {233907} (\bibinfo {year} {2010})}\BibitemShut {NoStop}%
\bibitem [{\citenamefont {Eichler}\ \emph {et~al.}(2011)\citenamefont
  {Eichler}, \citenamefont {Chaste}, \citenamefont {Moser},\ and\ \citenamefont
  {Bachtold}}]{Eichler_2011_NL}%
  \BibitemOpen
  \bibfield  {author} {\bibinfo {author} {\bibfnamefont {A.}~\bibnamefont
  {Eichler}}, \bibinfo {author} {\bibfnamefont {J.}~\bibnamefont {Chaste}},
  \bibinfo {author} {\bibfnamefont {J.}~\bibnamefont {Moser}}, \ and\ \bibinfo
  {author} {\bibfnamefont {A.}~\bibnamefont {Bachtold}},\ }\href {\doibase
  10.1021/nl200950d} {\bibfield  {journal} {\bibinfo  {journal} {Nano Letters}\
  }\textbf {\bibinfo {volume} {11}},\ \bibinfo {pages} {2699} (\bibinfo {year}
  {2011})},\ \bibinfo {note} {pMID: 21615135}\BibitemShut {NoStop}%
\bibitem [{\citenamefont {Leuch}\ \emph {et~al.}(2016)\citenamefont {Leuch},
  \citenamefont {Papariello}, \citenamefont {Zilberberg}, \citenamefont
  {Degen}, \citenamefont {Chitra},\ and\ \citenamefont {Eichler}}]{Leuch_2016}%
  \BibitemOpen
  \bibfield  {author} {\bibinfo {author} {\bibfnamefont {A.}~\bibnamefont
  {Leuch}}, \bibinfo {author} {\bibfnamefont {L.}~\bibnamefont {Papariello}},
  \bibinfo {author} {\bibfnamefont {O.}~\bibnamefont {Zilberberg}}, \bibinfo
  {author} {\bibfnamefont {C.~L.}\ \bibnamefont {Degen}}, \bibinfo {author}
  {\bibfnamefont {R.}~\bibnamefont {Chitra}}, \ and\ \bibinfo {author}
  {\bibfnamefont {A.}~\bibnamefont {Eichler}},\ }\href {\doibase
  10.1103/PhysRevLett.117.214101} {\bibfield  {journal} {\bibinfo  {journal}
  {Phys. Rev. Lett.}\ }\textbf {\bibinfo {volume} {117}},\ \bibinfo {pages}
  {214101} (\bibinfo {year} {2016})}\BibitemShut {NoStop}%
\bibitem [{\citenamefont {Gieseler}\ \emph {et~al.}(2012)\citenamefont
  {Gieseler}, \citenamefont {Deutsch}, \citenamefont {Quidant},\ and\
  \citenamefont {Novotny}}]{Gieseler_2012}%
  \BibitemOpen
  \bibfield  {author} {\bibinfo {author} {\bibfnamefont {J.}~\bibnamefont
  {Gieseler}}, \bibinfo {author} {\bibfnamefont {B.}~\bibnamefont {Deutsch}},
  \bibinfo {author} {\bibfnamefont {R.}~\bibnamefont {Quidant}}, \ and\
  \bibinfo {author} {\bibfnamefont {L.}~\bibnamefont {Novotny}},\ }\href
  {\doibase 10.1103/PhysRevLett.109.103603} {\bibfield  {journal} {\bibinfo
  {journal} {Phys. Rev. Lett.}\ }\textbf {\bibinfo {volume} {109}},\ \bibinfo
  {pages} {103603} (\bibinfo {year} {2012})}\BibitemShut {NoStop}%
\bibitem [{\citenamefont {Lin}\ \emph {et~al.}(2014)\citenamefont {Lin},
  \citenamefont {Inomata}, \citenamefont {Koshino}, \citenamefont {Oliver},
  \citenamefont {Nakamura}, \citenamefont {Tsai},\ and\ \citenamefont
  {Yamamoto}}]{Lin_2014}%
  \BibitemOpen
  \bibfield  {author} {\bibinfo {author} {\bibfnamefont {Z.}~\bibnamefont
  {Lin}}, \bibinfo {author} {\bibfnamefont {K.}~\bibnamefont {Inomata}},
  \bibinfo {author} {\bibfnamefont {K.}~\bibnamefont {Koshino}}, \bibinfo
  {author} {\bibfnamefont {W.~D.}\ \bibnamefont {Oliver}}, \bibinfo {author}
  {\bibfnamefont {Y.}~\bibnamefont {Nakamura}}, \bibinfo {author}
  {\bibfnamefont {J.~S.}\ \bibnamefont {Tsai}}, \ and\ \bibinfo {author}
  {\bibfnamefont {T.}~\bibnamefont {Yamamoto}},\ }\href
  {https://doi.org/10.1038/ncomms5480} {\bibfield  {journal} {\bibinfo
  {journal} {Nature Communications}\ }\textbf {\bibinfo {volume} {5}},\
  \bibinfo {pages} {4480} (\bibinfo {year} {2014})}\BibitemShut {NoStop}%
\bibitem [{\citenamefont {Puri}\ \emph
  {et~al.}(2017{\natexlab{a}})\citenamefont {Puri}, \citenamefont {Boutin},\
  and\ \citenamefont {Blais}}]{Puri_2017}%
  \BibitemOpen
  \bibfield  {author} {\bibinfo {author} {\bibfnamefont {S.}~\bibnamefont
  {Puri}}, \bibinfo {author} {\bibfnamefont {S.}~\bibnamefont {Boutin}}, \ and\
  \bibinfo {author} {\bibfnamefont {A.}~\bibnamefont {Blais}},\ }\href
  {https://doi.org/10.1038/s41534-017-0019-1} {\bibfield  {journal} {\bibinfo
  {journal} {npj Quantum Information}\ }\textbf {\bibinfo {volume} {3}},\
  \bibinfo {pages} {18} (\bibinfo {year} {2017}{\natexlab{a}})}\BibitemShut
  {NoStop}%
\bibitem [{\citenamefont {Eichler}\ \emph {et~al.}(2018)\citenamefont
  {Eichler}, \citenamefont {Heugel}, \citenamefont {Leuch}, \citenamefont
  {Degen}, \citenamefont {Chitra},\ and\ \citenamefont
  {Zilberberg}}]{Eichler_2018}%
  \BibitemOpen
  \bibfield  {author} {\bibinfo {author} {\bibfnamefont {A.}~\bibnamefont
  {Eichler}}, \bibinfo {author} {\bibfnamefont {T.~L.}\ \bibnamefont {Heugel}},
  \bibinfo {author} {\bibfnamefont {A.}~\bibnamefont {Leuch}}, \bibinfo
  {author} {\bibfnamefont {C.~L.}\ \bibnamefont {Degen}}, \bibinfo {author}
  {\bibfnamefont {R.}~\bibnamefont {Chitra}}, \ and\ \bibinfo {author}
  {\bibfnamefont {O.}~\bibnamefont {Zilberberg}},\ }\href {\doibase
  10.1063/1.5031058} {\bibfield  {journal} {\bibinfo  {journal} {Applied
  Physics Letters}\ }\textbf {\bibinfo {volume} {112}},\ \bibinfo {pages}
  {233105} (\bibinfo {year} {2018})},\ \Eprint
  {http://arxiv.org/abs/https://doi.org/10.1063/1.5031058}
  {https://doi.org/10.1063/1.5031058} \BibitemShut {NoStop}%
\bibitem [{\citenamefont {Nosan}\ \emph {et~al.}(2019)\citenamefont {Nosan},
  \citenamefont {M\"arki}, \citenamefont {Hauff}, \citenamefont {Knaut},\ and\
  \citenamefont {Eichler}}]{Nosan_2019}%
  \BibitemOpen
  \bibfield  {author} {\bibinfo {author} {\bibfnamefont {Z.}~\bibnamefont
  {Nosan}}, \bibinfo {author} {\bibfnamefont {P.}~\bibnamefont {M\"arki}},
  \bibinfo {author} {\bibfnamefont {N.}~\bibnamefont {Hauff}}, \bibinfo
  {author} {\bibfnamefont {C.}~\bibnamefont {Knaut}}, \ and\ \bibinfo {author}
  {\bibfnamefont {A.}~\bibnamefont {Eichler}},\ }\href {\doibase
  10.1103/PhysRevE.99.062205} {\bibfield  {journal} {\bibinfo  {journal} {Phys.
  Rev. E}\ }\textbf {\bibinfo {volume} {99}},\ \bibinfo {pages} {062205}
  (\bibinfo {year} {2019})}\BibitemShut {NoStop}%
\bibitem [{\citenamefont {Grimm}\ \emph {et~al.}(2019)\citenamefont {Grimm},
  \citenamefont {Frattini}, \citenamefont {Puri}, \citenamefont {Mundhada},
  \citenamefont {Touzard}, \citenamefont {Mirrahimi}, \citenamefont {Girvin},
  \citenamefont {Shankar},\ and\ \citenamefont {Devoret}}]{Grimm_2019}%
  \BibitemOpen
  \bibfield  {author} {\bibinfo {author} {\bibfnamefont {A.}~\bibnamefont
  {Grimm}}, \bibinfo {author} {\bibfnamefont {N.~E.}\ \bibnamefont {Frattini}},
  \bibinfo {author} {\bibfnamefont {S.}~\bibnamefont {Puri}}, \bibinfo {author}
  {\bibfnamefont {S.~O.}\ \bibnamefont {Mundhada}}, \bibinfo {author}
  {\bibfnamefont {S.}~\bibnamefont {Touzard}}, \bibinfo {author} {\bibfnamefont
  {M.}~\bibnamefont {Mirrahimi}}, \bibinfo {author} {\bibfnamefont {S.~M.}\
  \bibnamefont {Girvin}}, \bibinfo {author} {\bibfnamefont {S.}~\bibnamefont
  {Shankar}}, \ and\ \bibinfo {author} {\bibfnamefont {M.~H.}\ \bibnamefont
  {Devoret}},\ }\href {\doibase 10.1038/s41586-020-2587-z} {\bibfield
  {journal} {\bibinfo  {journal} {Nature}\ }\textbf {\bibinfo {volume} {584}},\
  \bibinfo {pages} {205} (\bibinfo {year} {2019})}\BibitemShut {NoStop}%
\bibitem [{\citenamefont {Puri}\ \emph {et~al.}(2019)\citenamefont {Puri},
  \citenamefont {Grimm}, \citenamefont {Campagne-Ibarcq}, \citenamefont
  {Eickbusch}, \citenamefont {Noh}, \citenamefont {Roberts}, \citenamefont
  {Jiang}, \citenamefont {Mirrahimi}, \citenamefont {Devoret},\ and\
  \citenamefont {Girvin}}]{Puri_2019_PRX}%
  \BibitemOpen
  \bibfield  {author} {\bibinfo {author} {\bibfnamefont {S.}~\bibnamefont
  {Puri}}, \bibinfo {author} {\bibfnamefont {A.}~\bibnamefont {Grimm}},
  \bibinfo {author} {\bibfnamefont {P.}~\bibnamefont {Campagne-Ibarcq}},
  \bibinfo {author} {\bibfnamefont {A.}~\bibnamefont {Eickbusch}}, \bibinfo
  {author} {\bibfnamefont {K.}~\bibnamefont {Noh}}, \bibinfo {author}
  {\bibfnamefont {G.}~\bibnamefont {Roberts}}, \bibinfo {author} {\bibfnamefont
  {L.}~\bibnamefont {Jiang}}, \bibinfo {author} {\bibfnamefont
  {M.}~\bibnamefont {Mirrahimi}}, \bibinfo {author} {\bibfnamefont {M.~H.}\
  \bibnamefont {Devoret}}, \ and\ \bibinfo {author} {\bibfnamefont {S.~M.}\
  \bibnamefont {Girvin}},\ }\href {\doibase 10.1103/PhysRevX.9.041009}
  {\bibfield  {journal} {\bibinfo  {journal} {Phys. Rev. X}\ }\textbf {\bibinfo
  {volume} {9}},\ \bibinfo {pages} {041009} (\bibinfo {year}
  {2019})}\BibitemShut {NoStop}%
\bibitem [{\citenamefont {Miller}\ \emph {et~al.}(2019)\citenamefont {Miller},
  \citenamefont {Shin}, \citenamefont {Kwon}, \citenamefont {Shaw},\ and\
  \citenamefont {Kenny}}]{Miller_2019_phase}%
  \BibitemOpen
  \bibfield  {author} {\bibinfo {author} {\bibfnamefont {J.~M.}\ \bibnamefont
  {Miller}}, \bibinfo {author} {\bibfnamefont {D.~D.}\ \bibnamefont {Shin}},
  \bibinfo {author} {\bibfnamefont {H.-K.}\ \bibnamefont {Kwon}}, \bibinfo
  {author} {\bibfnamefont {S.~W.}\ \bibnamefont {Shaw}}, \ and\ \bibinfo
  {author} {\bibfnamefont {T.~W.}\ \bibnamefont {Kenny}},\ }\href {\doibase
  10.1103/PhysRevApplied.12.044053} {\bibfield  {journal} {\bibinfo  {journal}
  {Phys. Rev. Applied}\ }\textbf {\bibinfo {volume} {12}},\ \bibinfo {pages}
  {044053} (\bibinfo {year} {2019})}\BibitemShut {NoStop}%
\bibitem [{\citenamefont {Dykman}(2012)}]{DykmanBook}%
  \BibitemOpen
  \bibfield  {author} {\bibinfo {author} {\bibfnamefont {M.}~\bibnamefont
  {Dykman}},\ }\href@noop {} {\emph {\bibinfo {title} {Fluctuating Nonlinear
  Oscillators}}}\ (\bibinfo  {publisher} {Oxford University Press},\ \bibinfo
  {year} {2012})\BibitemShut {NoStop}%
\bibitem [{\citenamefont {Eichler}\ and\ \citenamefont
  {Zilberberg}(2023)}]{Eichler_Zilberberg_book}%
  \BibitemOpen
  \bibfield  {author} {\bibinfo {author} {\bibfnamefont {A.}~\bibnamefont
  {Eichler}}\ and\ \bibinfo {author} {\bibfnamefont {O.}~\bibnamefont
  {Zilberberg}},\ }\href@noop {} {\emph {\bibinfo {title} {Classical and
  Quantum Parametric Phenomena}}}\ (\bibinfo  {publisher} {Oxford University
  Press},\ \bibinfo {year} {2023})\BibitemShut {NoStop}%
\bibitem [{\citenamefont {Mahboob}\ \emph {et~al.}(2016)\citenamefont
  {Mahboob}, \citenamefont {Okamoto},\ and\ \citenamefont
  {Yamaguchi}}]{Mahboob_2016}%
  \BibitemOpen
  \bibfield  {author} {\bibinfo {author} {\bibfnamefont {I.}~\bibnamefont
  {Mahboob}}, \bibinfo {author} {\bibfnamefont {H.}~\bibnamefont {Okamoto}}, \
  and\ \bibinfo {author} {\bibfnamefont {H.}~\bibnamefont {Yamaguchi}},\ }\href
  {https://advances.sciencemag.org/content/2/6/e1600236} {\bibfield  {journal}
  {\bibinfo  {journal} {Science Advances}\ }\textbf {\bibinfo {volume} {2}},\
  \bibinfo {pages} {e1600236} (\bibinfo {year} {2016})}\BibitemShut {NoStop}%
\bibitem [{\citenamefont {Inagaki}\ \emph {et~al.}(2016)\citenamefont
  {Inagaki}, \citenamefont {Inaba}, \citenamefont {Hamerly}, \citenamefont
  {Inoue}, \citenamefont {Yamamoto},\ and\ \citenamefont
  {Takesue}}]{Inagaki_2016}%
  \BibitemOpen
  \bibfield  {author} {\bibinfo {author} {\bibfnamefont {T.}~\bibnamefont
  {Inagaki}}, \bibinfo {author} {\bibfnamefont {K.}~\bibnamefont {Inaba}},
  \bibinfo {author} {\bibfnamefont {R.}~\bibnamefont {Hamerly}}, \bibinfo
  {author} {\bibfnamefont {K.}~\bibnamefont {Inoue}}, \bibinfo {author}
  {\bibfnamefont {Y.}~\bibnamefont {Yamamoto}}, \ and\ \bibinfo {author}
  {\bibfnamefont {H.}~\bibnamefont {Takesue}},\ }\href
  {https://doi.org/10.1038/nphoton.2016.68} {\bibfield  {journal} {\bibinfo
  {journal} {Nature Photonics}\ }\textbf {\bibinfo {volume} {10}},\ \bibinfo
  {pages} {415} (\bibinfo {year} {2016})}\BibitemShut {NoStop}%
\bibitem [{\citenamefont {Goto}(2016)}]{Goto_2016}%
  \BibitemOpen
  \bibfield  {author} {\bibinfo {author} {\bibfnamefont {H.}~\bibnamefont
  {Goto}},\ }\href {https://doi.org/10.1038/srep21686} {\bibfield  {journal}
  {\bibinfo  {journal} {Scientific Reports}\ }\textbf {\bibinfo {volume} {6}},\
  \bibinfo {pages} {21686} (\bibinfo {year} {2016})}\BibitemShut {NoStop}%
\bibitem [{\citenamefont {Puri}\ \emph
  {et~al.}(2017{\natexlab{b}})\citenamefont {Puri}, \citenamefont {Andersen},
  \citenamefont {Grimsmo},\ and\ \citenamefont {Blais}}]{Puri_2017_NC}%
  \BibitemOpen
  \bibfield  {author} {\bibinfo {author} {\bibfnamefont {S.}~\bibnamefont
  {Puri}}, \bibinfo {author} {\bibfnamefont {C.~K.}\ \bibnamefont {Andersen}},
  \bibinfo {author} {\bibfnamefont {A.~L.}\ \bibnamefont {Grimsmo}}, \ and\
  \bibinfo {author} {\bibfnamefont {A.}~\bibnamefont {Blais}},\ }\href
  {\doibase 10.1038/ncomms15785} {\bibfield  {journal} {\bibinfo  {journal}
  {Nature Communications}\ }\textbf {\bibinfo {volume} {8}},\ \bibinfo {pages}
  {15785} (\bibinfo {year} {2017}{\natexlab{b}})}\BibitemShut {NoStop}%
\bibitem [{\citenamefont {Nigg}\ \emph {et~al.}(2017)\citenamefont {Nigg},
  \citenamefont {L{\"o}rch},\ and\ \citenamefont {Tiwari}}]{Nigg_2017}%
  \BibitemOpen
  \bibfield  {author} {\bibinfo {author} {\bibfnamefont {S.~E.}\ \bibnamefont
  {Nigg}}, \bibinfo {author} {\bibfnamefont {N.}~\bibnamefont {L{\"o}rch}}, \
  and\ \bibinfo {author} {\bibfnamefont {R.~P.}\ \bibnamefont {Tiwari}},\
  }\href {\doibase 10.1126/sciadv.1602273} {\bibfield  {journal} {\bibinfo
  {journal} {Science Advances}\ }\textbf {\bibinfo {volume} {3}} (\bibinfo
  {year} {2017}),\ 10.1126/sciadv.1602273}\BibitemShut {NoStop}%
\bibitem [{\citenamefont {Dykman}\ \emph {et~al.}(2018)\citenamefont {Dykman},
  \citenamefont {Bruder}, \citenamefont {L\"orch},\ and\ \citenamefont
  {Zhang}}]{Dykman_2018}%
  \BibitemOpen
  \bibfield  {author} {\bibinfo {author} {\bibfnamefont {M.~I.}\ \bibnamefont
  {Dykman}}, \bibinfo {author} {\bibfnamefont {C.}~\bibnamefont {Bruder}},
  \bibinfo {author} {\bibfnamefont {N.}~\bibnamefont {L\"orch}}, \ and\
  \bibinfo {author} {\bibfnamefont {Y.}~\bibnamefont {Zhang}},\ }\href
  {\doibase 10.1103/PhysRevB.98.195444} {\bibfield  {journal} {\bibinfo
  {journal} {Phys. Rev. B}\ }\textbf {\bibinfo {volume} {98}},\ \bibinfo
  {pages} {195444} (\bibinfo {year} {2018})}\BibitemShut {NoStop}%
\bibitem [{\citenamefont {Okawachi}\ \emph {et~al.}(2020)\citenamefont
  {Okawachi}, \citenamefont {Yu}, \citenamefont {Jang}, \citenamefont {Ji},
  \citenamefont {Zhao}, \citenamefont {Kim}, \citenamefont {Lipson},\ and\
  \citenamefont {Gaeta}}]{Okawachi_2020}%
  \BibitemOpen
  \bibfield  {author} {\bibinfo {author} {\bibfnamefont {Y.}~\bibnamefont
  {Okawachi}}, \bibinfo {author} {\bibfnamefont {M.}~\bibnamefont {Yu}},
  \bibinfo {author} {\bibfnamefont {J.~K.}\ \bibnamefont {Jang}}, \bibinfo
  {author} {\bibfnamefont {X.}~\bibnamefont {Ji}}, \bibinfo {author}
  {\bibfnamefont {Y.}~\bibnamefont {Zhao}}, \bibinfo {author} {\bibfnamefont
  {B.~Y.}\ \bibnamefont {Kim}}, \bibinfo {author} {\bibfnamefont
  {M.}~\bibnamefont {Lipson}}, \ and\ \bibinfo {author} {\bibfnamefont {A.~L.}\
  \bibnamefont {Gaeta}},\ }\href {\doibase 10.1038/s41467-020-17919-6}
  {\bibfield  {journal} {\bibinfo  {journal} {Nature Communications}\ }\textbf
  {\bibinfo {volume} {11}},\ \bibinfo {pages} {4119} (\bibinfo {year}
  {2020})}\BibitemShut {NoStop}%
\bibitem [{\citenamefont {Calvanese~Strinati}\ \emph
  {et~al.}(2021)\citenamefont {Calvanese~Strinati}, \citenamefont {Bello},
  \citenamefont {Dalla~Torre},\ and\ \citenamefont {Pe'er}}]{Strinanti_2021}%
  \BibitemOpen
  \bibfield  {author} {\bibinfo {author} {\bibfnamefont {M.}~\bibnamefont
  {Calvanese~Strinati}}, \bibinfo {author} {\bibfnamefont {L.}~\bibnamefont
  {Bello}}, \bibinfo {author} {\bibfnamefont {E.~G.}\ \bibnamefont
  {Dalla~Torre}}, \ and\ \bibinfo {author} {\bibfnamefont {A.}~\bibnamefont
  {Pe'er}},\ }\href {\doibase 10.1103/PhysRevLett.126.143901} {\bibfield
  {journal} {\bibinfo  {journal} {Phys. Rev. Lett.}\ }\textbf {\bibinfo
  {volume} {126}},\ \bibinfo {pages} {143901} (\bibinfo {year}
  {2021})}\BibitemShut {NoStop}%
\bibitem [{\citenamefont {Bello}\ \emph {et~al.}(2019)\citenamefont {Bello},
  \citenamefont {Calvanese~Strinati}, \citenamefont {Dalla~Torre},\ and\
  \citenamefont {Pe'er}}]{Bello_2019}%
  \BibitemOpen
  \bibfield  {author} {\bibinfo {author} {\bibfnamefont {L.}~\bibnamefont
  {Bello}}, \bibinfo {author} {\bibfnamefont {M.}~\bibnamefont
  {Calvanese~Strinati}}, \bibinfo {author} {\bibfnamefont {E.~G.}\ \bibnamefont
  {Dalla~Torre}}, \ and\ \bibinfo {author} {\bibfnamefont {A.}~\bibnamefont
  {Pe'er}},\ }\href {\doibase 10.1103/PhysRevLett.123.083901} {\bibfield
  {journal} {\bibinfo  {journal} {Phys. Rev. Lett.}\ }\textbf {\bibinfo
  {volume} {123}},\ \bibinfo {pages} {083901} (\bibinfo {year}
  {2019})}\BibitemShut {NoStop}%
\bibitem [{\citenamefont {Heugel}\ \emph
  {et~al.}(2022{\natexlab{a}})\citenamefont {Heugel}, \citenamefont
  {Zilberberg}, \citenamefont {Marty}, \citenamefont {Chitra},\ and\
  \citenamefont {Eichler}}]{Heugel_2022}%
  \BibitemOpen
  \bibfield  {author} {\bibinfo {author} {\bibfnamefont {T.~L.}\ \bibnamefont
  {Heugel}}, \bibinfo {author} {\bibfnamefont {O.}~\bibnamefont {Zilberberg}},
  \bibinfo {author} {\bibfnamefont {C.}~\bibnamefont {Marty}}, \bibinfo
  {author} {\bibfnamefont {R.}~\bibnamefont {Chitra}}, \ and\ \bibinfo {author}
  {\bibfnamefont {A.}~\bibnamefont {Eichler}},\ }\href@noop {} {\bibfield
  {journal} {\bibinfo  {journal} {Physical Review Research}\ }\textbf {\bibinfo
  {volume} {4}},\ \bibinfo {pages} {013149} (\bibinfo {year}
  {2022}{\natexlab{a}})}\BibitemShut {NoStop}%
\bibitem [{\citenamefont {Heugel}\ \emph
  {et~al.}(2022{\natexlab{b}})\citenamefont {Heugel}, \citenamefont {Eichler},
  \citenamefont {Chitra},\ and\ \citenamefont {Zilberberg}}]{heugel2022role}%
  \BibitemOpen
  \bibfield  {author} {\bibinfo {author} {\bibfnamefont {T.~L.}\ \bibnamefont
  {Heugel}}, \bibinfo {author} {\bibfnamefont {A.}~\bibnamefont {Eichler}},
  \bibinfo {author} {\bibfnamefont {R.}~\bibnamefont {Chitra}}, \ and\ \bibinfo
  {author} {\bibfnamefont {O.}~\bibnamefont {Zilberberg}},\ }\href@noop {}
  {\bibfield  {journal} {\bibinfo  {journal} {arXiv preprint arXiv:2203.05577}\
  } (\bibinfo {year} {2022}{\natexlab{b}})}\BibitemShut {NoStop}%
\bibitem [{\citenamefont {Margiani}\ \emph {et~al.}(2023)\citenamefont
  {Margiani}, \citenamefont {del Pino}, \citenamefont {Heugel}, \citenamefont
  {Bousse}, \citenamefont {Guerrero}, \citenamefont {Kenny}, \citenamefont
  {Zilberberg}, \citenamefont {Sabonis},\ and\ \citenamefont
  {Eichler}}]{margiani2022deterministic}%
  \BibitemOpen
  \bibfield  {author} {\bibinfo {author} {\bibfnamefont {G.}~\bibnamefont
  {Margiani}}, \bibinfo {author} {\bibfnamefont {J.}~\bibnamefont {del Pino}},
  \bibinfo {author} {\bibfnamefont {T.~L.}\ \bibnamefont {Heugel}}, \bibinfo
  {author} {\bibfnamefont {N.~E.}\ \bibnamefont {Bousse}}, \bibinfo {author}
  {\bibfnamefont {S.}~\bibnamefont {Guerrero}}, \bibinfo {author}
  {\bibfnamefont {T.~W.}\ \bibnamefont {Kenny}}, \bibinfo {author}
  {\bibfnamefont {O.}~\bibnamefont {Zilberberg}}, \bibinfo {author}
  {\bibfnamefont {D.}~\bibnamefont {Sabonis}}, \ and\ \bibinfo {author}
  {\bibfnamefont {A.}~\bibnamefont {Eichler}},\ }\href {\doibase
  10.1103/PhysRevResearch.5.L012029} {\bibfield  {journal} {\bibinfo  {journal}
  {Phys. Rev. Res.}\ }\textbf {\bibinfo {volume} {5}},\ \bibinfo {pages}
  {L012029} (\bibinfo {year} {2023})}\BibitemShut {NoStop}%
\bibitem [{\citenamefont {Heugel}\ \emph {et~al.}(2019)\citenamefont {Heugel},
  \citenamefont {Oscity}, \citenamefont {Eichler}, \citenamefont {Zilberberg},\
  and\ \citenamefont {Chitra}}]{Heugel_2019_TC}%
  \BibitemOpen
  \bibfield  {author} {\bibinfo {author} {\bibfnamefont {T.~L.}\ \bibnamefont
  {Heugel}}, \bibinfo {author} {\bibfnamefont {M.}~\bibnamefont {Oscity}},
  \bibinfo {author} {\bibfnamefont {A.}~\bibnamefont {Eichler}}, \bibinfo
  {author} {\bibfnamefont {O.}~\bibnamefont {Zilberberg}}, \ and\ \bibinfo
  {author} {\bibfnamefont {R.}~\bibnamefont {Chitra}},\ }\href {\doibase
  10.1103/PhysRevLett.123.124301} {\bibfield  {journal} {\bibinfo  {journal}
  {Phys. Rev. Lett.}\ }\textbf {\bibinfo {volume} {123}},\ \bibinfo {pages}
  {124301} (\bibinfo {year} {2019})}\BibitemShut {NoStop}%
\bibitem [{\citenamefont {Lehmann}\ \emph {et~al.}(2003)\citenamefont
  {Lehmann}, \citenamefont {Reimann},\ and\ \citenamefont
  {Hänggi}}]{Lehmann_2003}%
  \BibitemOpen
  \bibfield  {author} {\bibinfo {author} {\bibfnamefont {J.}~\bibnamefont
  {Lehmann}}, \bibinfo {author} {\bibfnamefont {P.}~\bibnamefont {Reimann}}, \
  and\ \bibinfo {author} {\bibfnamefont {P.}~\bibnamefont {Hänggi}},\ }\href
  {\doibase https://doi.org/10.1002/pssb.200301774} {\bibfield  {journal}
  {\bibinfo  {journal} {physica status solidi (b)}\ }\textbf {\bibinfo {volume}
  {237}},\ \bibinfo {pages} {53} (\bibinfo {year} {2003})},\ \Eprint
  {http://arxiv.org/abs/https://onlinelibrary.wiley.com/doi/pdf/10.1002/pssb.200301774}
  {https://onlinelibrary.wiley.com/doi/pdf/10.1002/pssb.200301774} \BibitemShut
  {NoStop}%
\bibitem [{\citenamefont {Wio}(2013)}]{Wio_2013}%
  \BibitemOpen
  \bibfield  {author} {\bibinfo {author} {\bibfnamefont {H.~S.}\ \bibnamefont
  {Wio}},\ }\href {\doibase 10.1142/8695} {\emph {\bibinfo {title} {Path
  Integrals for Stochastic Processes}}}\ (\bibinfo  {publisher} {WORLD
  SCIENTIFIC},\ \bibinfo {year} {2013})\ \Eprint
  {http://arxiv.org/abs/https://www.worldscientific.com/doi/pdf/10.1142/8695}
  {https://www.worldscientific.com/doi/pdf/10.1142/8695} \BibitemShut {NoStop}%
\bibitem [{\citenamefont {McLachlan}(1951)}]{mclachlan1951theory}%
  \BibitemOpen
  \bibfield  {author} {\bibinfo {author} {\bibfnamefont {N.}~\bibnamefont
  {McLachlan}},\ }\href@noop {} {\emph {\bibinfo {title} {Theory and
  application of Mathieu functions}}}\ (\bibinfo  {publisher} {Clarendon},\
  \bibinfo {year} {1951})\BibitemShut {NoStop}%
\bibitem [{\citenamefont {Lifshitz}(2009)}]{Lifshitz_Cross}%
  \BibitemOpen
  \bibfield  {author} {\bibinfo {author} {\bibfnamefont {M.~C.}\ \bibnamefont
  {Lifshitz}, \bibfnamefont {R.~Cross}},\ }\enquote {\bibinfo {title}
  {Nonlinear dynamics of nanomechanical and micromechanical resonators},}\ in\
  \href {\doibase 10.1002/9783527626359.ch1} {\emph {\bibinfo {booktitle}
  {Reviews of Nonlinear Dynamics and Complexity}}}\ (\bibinfo  {publisher}
  {Wiley-VCH},\ \bibinfo {year} {2009})\ pp.\ \bibinfo {pages}
  {1--52}\BibitemShut {NoStop}%
\bibitem [{\citenamefont {Guckenheimer}\ and\ \citenamefont
  {Holmes}(1990)}]{guckenheimer_1990}%
  \BibitemOpen
  \bibfield  {author} {\bibinfo {author} {\bibfnamefont {J.}~\bibnamefont
  {Guckenheimer}}\ and\ \bibinfo {author} {\bibfnamefont {P.}~\bibnamefont
  {Holmes}},\ }\href@noop {} {\emph {\bibinfo {title} {Nonlinear oscillations,
  dynamical systems, and bifurcations of vector fields}}},\ Applied
  mathematical sciences\ (\bibinfo  {publisher} {Springer-Verlag},\ \bibinfo
  {year} {1990})\BibitemShut {NoStop}%
\bibitem [{\citenamefont {Papariello}\ \emph {et~al.}(2016)\citenamefont
  {Papariello}, \citenamefont {Zilberberg}, \citenamefont {Eichler},\ and\
  \citenamefont {Chitra}}]{Papariello_2016}%
  \BibitemOpen
  \bibfield  {author} {\bibinfo {author} {\bibfnamefont {L.}~\bibnamefont
  {Papariello}}, \bibinfo {author} {\bibfnamefont {O.}~\bibnamefont
  {Zilberberg}}, \bibinfo {author} {\bibfnamefont {A.}~\bibnamefont {Eichler}},
  \ and\ \bibinfo {author} {\bibfnamefont {R.}~\bibnamefont {Chitra}},\
  }\href@noop {} {\bibfield  {journal} {\bibinfo  {journal} {Phys. Rev. E}\
  }\textbf {\bibinfo {volume} {94}},\ \bibinfo {pages} {022201} (\bibinfo
  {year} {2016})}\BibitemShut {NoStop}%
\bibitem [{\citenamefont {Khas’minskii}(1966)}]{Khasminskii_66}%
  \BibitemOpen
  \bibfield  {author} {\bibinfo {author} {\bibfnamefont {R.~Z.}\ \bibnamefont
  {Khas’minskii}},\ }\href {\doibase 10.1137/1111038} {\bibfield  {journal}
  {\bibinfo  {journal} {Theory of Probability \& Its Applications}\ }\textbf
  {\bibinfo {volume} {11}},\ \bibinfo {pages} {390} (\bibinfo {year} {1966})},\
  \Eprint {http://arxiv.org/abs/https://doi.org/10.1137/1111038}
  {https://doi.org/10.1137/1111038} \BibitemShut {NoStop}%
\bibitem [{\citenamefont {Roberts}\ and\ \citenamefont
  {Spanos}(1986)}]{Roberts_86}%
  \BibitemOpen
  \bibfield  {author} {\bibinfo {author} {\bibfnamefont {J.}~\bibnamefont
  {Roberts}}\ and\ \bibinfo {author} {\bibfnamefont {P.}~\bibnamefont
  {Spanos}},\ }\href {\doibase https://doi.org/10.1016/0020-7462(86)90025-9}
  {\bibfield  {journal} {\bibinfo  {journal} {International Journal of
  Non-Linear Mechanics}\ }\textbf {\bibinfo {volume} {21}},\ \bibinfo {pages}
  {111 } (\bibinfo {year} {1986})}\BibitemShut {NoStop}%
\bibitem [{\citenamefont {Nayfeh}\ and\ \citenamefont
  {Mook}(2008)}]{nayfeh2008}%
  \BibitemOpen
  \bibfield  {author} {\bibinfo {author} {\bibfnamefont {A.~H.}\ \bibnamefont
  {Nayfeh}}\ and\ \bibinfo {author} {\bibfnamefont {D.~T.}\ \bibnamefont
  {Mook}},\ }\href@noop {} {\emph {\bibinfo {title} {Nonlinear
  Oscillations}}},\ Physics textbook\ (\bibinfo  {publisher} {Wiley},\ \bibinfo
  {year} {2008})\BibitemShut {NoStop}%
\bibitem [{\citenamefont {Stambaugh}\ and\ \citenamefont
  {Chan}(2006)}]{Stambaugh_2006}%
  \BibitemOpen
  \bibfield  {author} {\bibinfo {author} {\bibfnamefont {C.}~\bibnamefont
  {Stambaugh}}\ and\ \bibinfo {author} {\bibfnamefont {H.~B.}\ \bibnamefont
  {Chan}},\ }\href {\doibase 10.1103/PhysRevB.73.172302} {\bibfield  {journal}
  {\bibinfo  {journal} {Phys. Rev. B}\ }\textbf {\bibinfo {volume} {73}},\
  \bibinfo {pages} {172302} (\bibinfo {year} {2006})}\BibitemShut {NoStop}%
\bibitem [{\citenamefont {Grafke}\ \emph {et~al.}(2017)\citenamefont {Grafke},
  \citenamefont {Sch{\"a}fer},\ and\ \citenamefont
  {Vanden-Eijnden}}]{Grafke_2017}%
  \BibitemOpen
  \bibfield  {author} {\bibinfo {author} {\bibfnamefont {T.}~\bibnamefont
  {Grafke}}, \bibinfo {author} {\bibfnamefont {T.}~\bibnamefont {Sch{\"a}fer}},
  \ and\ \bibinfo {author} {\bibfnamefont {E.}~\bibnamefont {Vanden-Eijnden}},\
  }\enquote {\bibinfo {title} {Long term effects of small random perturbations
  on dynamical systems: Theoretical and computational tools},}\ in\ \href
  {\doibase 10.1007/978-1-4939-6969-2_2} {\emph {\bibinfo {booktitle} {Recent
  Progress and Modern Challenges in Applied Mathematics, Modeling and
  Computational Science}}},\ \bibinfo {editor} {edited by\ \bibinfo {editor}
  {\bibfnamefont {R.}~\bibnamefont {Melnik}}, \bibinfo {editor} {\bibfnamefont
  {R.}~\bibnamefont {Makarov}}, \ and\ \bibinfo {editor} {\bibfnamefont
  {J.}~\bibnamefont {Belair}}}\ (\bibinfo  {publisher} {Springer New York},\
  \bibinfo {address} {New York, NY},\ \bibinfo {year} {2017})\ pp.\ \bibinfo
  {pages} {17--55}\BibitemShut {NoStop}%
\bibitem [{\citenamefont {Maier}\ and\ \citenamefont
  {Stein}(1992)}]{Maier_1992}%
  \BibitemOpen
  \bibfield  {author} {\bibinfo {author} {\bibfnamefont {R.~S.}\ \bibnamefont
  {Maier}}\ and\ \bibinfo {author} {\bibfnamefont {D.~L.}\ \bibnamefont
  {Stein}},\ }\href {\doibase 10.1103/PhysRevLett.69.3691} {\bibfield
  {journal} {\bibinfo  {journal} {Phys. Rev. Lett.}\ }\textbf {\bibinfo
  {volume} {69}},\ \bibinfo {pages} {3691} (\bibinfo {year}
  {1992})}\BibitemShut {NoStop}%
\bibitem [{\citenamefont {Tang}\ \emph {et~al.}(2017)\citenamefont {Tang},
  \citenamefont {Yuan}, \citenamefont {Wang}, \citenamefont {Zhu},\ and\
  \citenamefont {Ao}}]{Tang_2017}%
  \BibitemOpen
  \bibfield  {author} {\bibinfo {author} {\bibfnamefont {Y.}~\bibnamefont
  {Tang}}, \bibinfo {author} {\bibfnamefont {R.}~\bibnamefont {Yuan}}, \bibinfo
  {author} {\bibfnamefont {G.}~\bibnamefont {Wang}}, \bibinfo {author}
  {\bibfnamefont {X.}~\bibnamefont {Zhu}}, \ and\ \bibinfo {author}
  {\bibfnamefont {P.}~\bibnamefont {Ao}},\ }\href {\doibase
  10.1038/s41598-017-15889-2} {\bibfield  {journal} {\bibinfo  {journal}
  {Scientific Reports}\ }\textbf {\bibinfo {volume} {7}},\ \bibinfo {pages}
  {15762} (\bibinfo {year} {2017})}\BibitemShut {NoStop}%
\bibitem [{\citenamefont {Luchinsky}\ \emph
  {et~al.}(1999{\natexlab{b}})\citenamefont {Luchinsky}, \citenamefont {Maier},
  \citenamefont {Mannella}, \citenamefont {McClintock},\ and\ \citenamefont
  {Stein}}]{Luchinsky_1999}%
  \BibitemOpen
  \bibfield  {author} {\bibinfo {author} {\bibfnamefont {D.~G.}\ \bibnamefont
  {Luchinsky}}, \bibinfo {author} {\bibfnamefont {R.~S.}\ \bibnamefont
  {Maier}}, \bibinfo {author} {\bibfnamefont {R.}~\bibnamefont {Mannella}},
  \bibinfo {author} {\bibfnamefont {P.~V.~E.}\ \bibnamefont {McClintock}}, \
  and\ \bibinfo {author} {\bibfnamefont {D.~L.}\ \bibnamefont {Stein}},\ }\href
  {\doibase 10.1103/PhysRevLett.82.1806} {\bibfield  {journal} {\bibinfo
  {journal} {Phys. Rev. Lett.}\ }\textbf {\bibinfo {volume} {82}},\ \bibinfo
  {pages} {1806} (\bibinfo {year} {1999}{\natexlab{b}})}\BibitemShut {NoStop}%
\bibitem [{\citenamefont {Feng}\ \emph {et~al.}(2014)\citenamefont {Feng},
  \citenamefont {Zhang},\ and\ \citenamefont {Wang}}]{Feng_2014}%
  \BibitemOpen
  \bibfield  {author} {\bibinfo {author} {\bibfnamefont {H.}~\bibnamefont
  {Feng}}, \bibinfo {author} {\bibfnamefont {K.}~\bibnamefont {Zhang}}, \ and\
  \bibinfo {author} {\bibfnamefont {J.}~\bibnamefont {Wang}},\ }\href {\doibase
  10.1039/C4SC00831F} {\bibfield  {journal} {\bibinfo  {journal} {Chem. Sci.}\
  }\textbf {\bibinfo {volume} {5}},\ \bibinfo {pages} {3761} (\bibinfo {year}
  {2014})}\BibitemShut {NoStop}%
\bibitem [{\citenamefont {Gottesman}\ \emph {et~al.}(2001)\citenamefont
  {Gottesman}, \citenamefont {Kitaev},\ and\ \citenamefont
  {Preskill}}]{Gottesman_2001}%
  \BibitemOpen
  \bibfield  {author} {\bibinfo {author} {\bibfnamefont {D.}~\bibnamefont
  {Gottesman}}, \bibinfo {author} {\bibfnamefont {A.}~\bibnamefont {Kitaev}}, \
  and\ \bibinfo {author} {\bibfnamefont {J.}~\bibnamefont {Preskill}},\ }\href
  {\doibase 10.1103/PhysRevA.64.012310} {\bibfield  {journal} {\bibinfo
  {journal} {Phys. Rev. A}\ }\textbf {\bibinfo {volume} {64}},\ \bibinfo
  {pages} {012310} (\bibinfo {year} {2001})}\BibitemShut {NoStop}%
\bibitem [{\citenamefont {Devoret}\ and\ \citenamefont
  {Schoelkopf}(2013)}]{Devoret_2013}%
  \BibitemOpen
  \bibfield  {author} {\bibinfo {author} {\bibfnamefont {M.~H.}\ \bibnamefont
  {Devoret}}\ and\ \bibinfo {author} {\bibfnamefont {R.~J.}\ \bibnamefont
  {Schoelkopf}},\ }\href {\doibase 10.1126/science.1231930} {\bibfield
  {journal} {\bibinfo  {journal} {Science}\ }\textbf {\bibinfo {volume}
  {339}},\ \bibinfo {pages} {1169} (\bibinfo {year} {2013})}\BibitemShut
  {NoStop}%
\bibitem [{\citenamefont {Landau}\ and\ \citenamefont
  {Lifshitz}(1976)}]{Landau_Lifshitz}%
  \BibitemOpen
  \bibfield  {author} {\bibinfo {author} {\bibfnamefont {L.}~\bibnamefont
  {Landau}}\ and\ \bibinfo {author} {\bibfnamefont {E.}~\bibnamefont
  {Lifshitz}},\ }\href@noop {} {\emph {\bibinfo {title} {Mechanics}}},\
  Butterworth-Heinemann\ (\bibinfo {year} {1976})\BibitemShut {NoStop}%
\bibitem [{\citenamefont {Košata}\ \emph {et~al.}(2022)\citenamefont
  {Košata}, \citenamefont {del Pino}, \citenamefont {Heugel},\ and\
  \citenamefont {Zilberberg}}]{kovsata2022harmonicbalance}%
  \BibitemOpen
  \bibfield  {author} {\bibinfo {author} {\bibfnamefont {J.}~\bibnamefont
  {Košata}}, \bibinfo {author} {\bibfnamefont {J.}~\bibnamefont {del Pino}},
  \bibinfo {author} {\bibfnamefont {T.~L.}\ \bibnamefont {Heugel}}, \ and\
  \bibinfo {author} {\bibfnamefont {O.}~\bibnamefont {Zilberberg}},\ }\href
  {\doibase 10.21468/SciPostPhysCodeb.6} {\bibfield  {journal} {\bibinfo
  {journal} {SciPost Phys. Codebases}\ ,\ \bibinfo {pages} {6}} (\bibinfo
  {year} {2022})}\BibitemShut {NoStop}%
\end{thebibliography}%


\end{document}
