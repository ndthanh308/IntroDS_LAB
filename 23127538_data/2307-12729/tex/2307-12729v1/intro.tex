% Figure environment removed
In a planned activity that involves interaction with objects, the
patterns of human motion vary greatly, contextualized to the situation
and stages of the plan. The variety of patterns comes from the fact
that the subject needs to switch between multiple modes of operations:
navigating the global scene according to the plan, and occasionally
concentrate on a small set of objects to act on them. This mode-switching
nature requires a machine model to adapt quickly in structure, representation,
and inference mechanisms to follow the patterns of the behavior. This
observation is confirmed by neuroscience findings that human brain
activity contains transient networks that deals with particular situation
\cite{baker2014fast}. In cognitive science, human activities are
also proved to follow parent-child planning patterns \cite{ajzen1985intentions}.

Recent advancements in graph neural networks \cite{velivckovic2017graph}
allow motion models to dynamically adjust their relational structure
to adapt to changing situations \cite{corona2020context}. However,
with a singular inference mechanism, they can only have \emph{gradual
adaptation}: smoothly adjusting parameters of the same model. They
cannot account for quick and abrupt changes\emph{ }regarding the discrete
switching between distinctive mechanisms and as a result fail to keep
up with the movement patterns. This limitation is inherent in human-object
interaction motion (HOI-M) prediction. \ref{fig:intro_demonstration}
visualizes an example activity of cooking a dinner meal. Here the
subject navigates around the kitchen following a recipe and consider
the whole kitchen floor plan (red trajectory in \ref{fig:intro_demonstration}).
Occasionally, they deviate to perform a particular action by interacting
with an object (green sections in \ref{fig:intro_demonstration})
such as moving a bowl, where only the bowl needs their attention.
When this happens, these models will continue to consider the interacted
object as an equal member of the scene and miss its importance as
the direct recipient of the action.

To address this limitation, we explore a new modeling paradigm that
factorizes the human-object interaction into two internal types of
processes: a \emph{Persistent process} (red box in \ref{fig:intro_demonstration})
which maintains a continuous large-scale activity progress; and\emph{
Transient sub-processes} (green boxes in \ref{fig:intro_demonstration})
which has an adaptive life cycle and a personalized structure to reflects
the small-scale local interaction with objects. The Transient act
as Persistent\textquoteright s sub-processes, and they can be turned
on or off by a switch (on/off arrows in \ref{fig:intro_demonstration})
operating on signal from the main Persistent process. The\emph{ parent-child
relationship} connects the two types of processes and constitute the
\emph{Persistent-Transient Duality}. 

This modeling is related to similar transient concepts in other fields
such as control theory \cite{henzinger2000theory}, electrical \cite{greenwood1991electrical},
and chemical engineering \cite{van2001transients,duncan2019chemical}.
In computer engineering, the parent-child relationship also resembles
the operating system and the children task-specific processes.

We model this concept into a multi-channel neural network called \emph{Persistent-Transient
Duality }(PTD) networks for Human-object interaction motion (HOI-M)
prediction. The \emph{Persistent channel} contains a recurrent relational
network operating on the global scene spatially and throughout the
session temporally. The \emph{Transient channels} instead have contextualized
structures constructed on the spot whenever the human subjects shift
the priority toward interacting with a subset of objects. The life
cycles of these spontaneous channels are managed by a neural \emph{Transient
Switch}, which anticipates the initialization and termination of many
Transient processes along a single Persistent process. 

The benefit of PTD is demonstrated via motion forecasting experiments
on WBHM and Bimanual Actions datasets where it sets new SOTA performances
and generalizability tests. Being a generic framework for human behavior
modeling, PTD is readily applicable to other problems such as pedestrian
trajectory prediction of which preliminary adaptation and experiment
are shown in the supplementary. 

The key contributions of this work are:

1. The exploration of \emph{the new Persistent-transient duality concept}
to model the multi-mechanism nature of human behavior reflected in
the large-small temporal scale and global-local spatial scope of HOI
motions.

2. A \emph{parent-children neural framework} with egocentric design
that applies the PTD concept in HOI-M prediction.

3. The extensive analysis demonstrating that PTD \emph{sets new SoTA
in HOI-M prediction} across multiple datasets and settings, and generalizes
better to new scenarios.
