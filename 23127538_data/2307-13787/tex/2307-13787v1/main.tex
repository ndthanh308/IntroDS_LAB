\documentclass[sigconf, screen=false, review=false, anonymous=false, balance=false]{acmart}
\usepackage{graphicx}
\usepackage{float}
\usepackage{caption}
\usepackage{subcaption}
\usepackage{svg}
\usepackage{enumitem}

% Optional math commands from https://github.com/goodfeli/dlbook_notation.
%%%%% NEW MATH DEFINITIONS %%%%%
\newtheorem{property}{Property}
\newtheorem{definition}{Definition}
\newtheorem{theorem}{Theorem}
\newtheorem{lemma}{Lemma}
\newtheorem{corollary}{Corollary}
\DeclarePairedDelimiter\abs{\lvert}{\rvert}
\DeclarePairedDelimiter\norm{\lVert}{\rVert}
\makeatletter
\let\oldabs\abs
\def\abs{\@ifstar{\oldabs}{\oldabs*}}
\let\oldnorm\norm
\def\norm{\@ifstar{\oldnorm}{\oldnorm*}}
\makeatother

% Mark sections of captions for referring to divisions of figures
\newcommand{\figleft}{{\em (Left) }}
\newcommand{\figcenter}{{\em (Center) }}
\newcommand{\figright}{{\em (Right)}}
\newcommand{\figtop}{{\em (Top) }}
\newcommand{\figbottom}{{\em (Bottom) }}
\newcommand{\captiona}{{\em (a) }}
\newcommand{\captionb}{{\em (b) }}
\newcommand{\captionc}{{\em (c) }}
\newcommand{\captiond}{{\em (d) }}

% Highlight a newly defined term
\newcommand{\newterm}[1]{{\bf #1}}


\def\figref#1{figure~\ref{#1}}
\def\Figref#1{Figure~\ref{#1}}
\def\twofigref#1#2{figures \ref{#1} and \ref{#2}}
\def\quadfigref#1#2#3#4{figures \ref{#1}, \ref{#2}, \ref{#3} and \ref{#4}}
\def\secref#1{section~\ref{#1}}
\def\Secref#1{Section~\ref{#1}}
\def\twosecrefs#1#2{sections \ref{#1} and \ref{#2}}
\def\secrefs#1#2#3{sections \ref{#1}, \ref{#2} and \ref{#3}}
\def\eqref#1{equation~\ref{#1}}
\def\Eqref#1{Equation~\ref{#1}}
% A raw reference to an equation---avoid using if possible
\def\plaineqref#1{\ref{#1}}
% Reference to a chapter, lower-case.
\def\chapref#1{chapter~\ref{#1}}
% Reference to an equation, upper case.
\def\Chapref#1{Chapter~\ref{#1}}
% Reference to a range of chapters
\def\rangechapref#1#2{chapters\ref{#1}--\ref{#2}}
% Reference to an algorithm, lower-case.
\def\algref#1{algorithm~\ref{#1}}
% Reference to an algorithm, upper case.
\def\Algref#1{Algorithm~\ref{#1}}
\def\twoalgref#1#2{algorithms \ref{#1} and \ref{#2}}
\def\Twoalgref#1#2{Algorithms \ref{#1} and \ref{#2}}
% Reference to a part, lower case
\def\partref#1{part~\ref{#1}}
% Reference to a part, upper case
\def\Partref#1{Part~\ref{#1}}
\def\twopartref#1#2{parts \ref{#1} and \ref{#2}}

% Random variables
\def\reta{{\textnormal{$\eta$}}}
\def\ra{{\textnormal{a}}}

% Random vectors
\def\rvepsilon{{\mathbf{\epsilon}}}
\def\rvtheta{{\mathbf{\theta}}}
\def\rva{{\mathbf{a}}}

% Elements of random vectors
\def\erva{{\textnormal{a}}}
\def\ervb{{\textnormal{b}}}

% Random matrices
\def\rmA{{\mathbf{A}}}
\def\rmB{{\mathbf{B}}}

% Elements of random matrices
\def\ermA{{\textnormal{A}}}
\def\ermB{{\textnormal{B}}}

\def\fvec{{\mathbf{f}}}
\def\bff{{\mathbf{f}}}
\def\bfg{{\mathbf{g}}}
% Vectors
\def\vzero{{\bm{0}}}
\def\vone{{\bm{1}}}
\def\vmu{{\bm{\mu}}}
\def\vtheta{{\bm{\theta}}}
\def\va{{\bm{a}}}
\def\vb{{\bm{b}}}
\def\vc{{\bm{c}}}
\def\vd{{\bm{d}}}
\def\ve{{\bm{e}}}
\def\vf{{\bm{f}}}
\def\vg{{\bm{g}}}
\def\vh{{\bm{h}}}
\def\vi{{\bm{i}}}
\def\vj{{\bm{j}}}
\def\vk{{\bm{k}}}
\def\vl{{\bm{l}}}
\def\vm{{\bm{m}}}
\def\vn{{\bm{n}}}
\def\vo{{\bm{o}}}
\def\vp{{\bm{p}}}
\def\vq{{\bm{q}}}
\def\vr{{\bm{r}}}
\def\vs{{\bm{s}}}
\def\vt{{\bm{t}}}
\def\vu{{\bm{u}}}
\def\vv{{\bm{v}}}
\def\vw{{\bm{w}}}
\def\vx{{\bm{x}}}
\def\vy{{\bm{y}}}
\def\vz{{\bm{z}}}

% Matrix
\def\mA{{\bm{A}}}

% Tensor
\DeclareMathAlphabet{\mathsfit}{\encodingdefault}{\sfdefault}{m}{sl}
\SetMathAlphabet{\mathsfit}{bold}{\encodingdefault}{\sfdefault}{bx}{n}
\newcommand{\tens}[1]{\bm{\mathsfit{#1}}}
\def\tA{{\tens{A}}}
\def\tB{{\tens{B}}}
\def\tC{{\tens{C}}}
\def\tD{{\tens{D}}}
\def\tE{{\tens{E}}}
\def\tF{{\tens{F}}}
\def\tG{{\tens{G}}}
\def\tH{{\tens{H}}}
\def\tI{{\tens{I}}}
\def\tJ{{\tens{J}}}
\def\tK{{\tens{K}}}
\def\tL{{\tens{L}}}
\def\tM{{\tens{M}}}
\def\tN{{\tens{N}}}
\def\tO{{\tens{O}}}
\def\tP{{\tens{P}}}
\def\tQ{{\tens{Q}}}
\def\tR{{\tens{R}}}
\def\tS{{\tens{S}}}
\def\tT{{\tens{T}}}
\def\tU{{\tens{U}}}
\def\tV{{\tens{V}}}
\def\tW{{\tens{W}}}
\def\tX{{\tens{X}}}
\def\tY{{\tens{Y}}}
\def\tZ{{\tens{Z}}}


% Graph
\def\gA{{\mathcal{A}}}
\def\gB{{\mathcal{B}}}
\def\gC{{\mathcal{C}}}
\def\dataset{{\mathcal{D}}}
\def\gE{{\mathcal{E}}}
\def\gF{{\mathcal{F}}}
\def\fourier{{\mathcal{F}}}
\def\gG{{\mathcal{G}}}
\def\gH{{\mathcal{H}}}
\def\gI{{\mathcal{I}}}
\def\gJ{{\mathcal{J}}}
\def\gK{{\mathcal{K}}}
\def\gL{{\mathcal{L}}}
\def\loss{{\mathcal{L}}}
\def\gM{{\mathcal{M}}}
\def\gN{{\mathcal{N}}}
\def\normal{{\mathcal{N}}}
\def\gaussian{{\mathcal{N}}}
\def\gO{{\mathcal{O}}}
\def\gP{{\mathcal{P}}}
\def\gQ{{\mathcal{Q}}}
\def\gR{{\mathcal{R}}}
\def\gS{{\mathcal{S}}}
\def\gT{{\mathcal{T}}}
\def\gU{{\mathcal{U}}}
\def\uniform{{\mathcal{U}}}
\def\gV{{\mathcal{V}}}
\def\gW{{\mathcal{W}}}
\def\gX{{\mathcal{X}}}
\def\gY{{\mathcal{Y}}}
\def\gZ{{\mathcal{Z}}}

\def\algebra{{\mathscr{A}}}
\def\borel{{\mathscr{B}}}
\def\manifold{{\mathscr{M}}}

% Sets
\def\sA{{\mathbb{A}}}
\def\sB{{\mathbb{B}}}
\def\complex{{\mathbb{C}}}
\def\sD{{\mathbb{D}}}
\def\expectation{{\mathbb{E}}}
\newcommand{\E}{\mathbb{E}}
\def\sF{{\mathbb{F}}}
\def\sG{{\mathbb{G}}}
\def\sH{{\mathbb{H}}}
\def\sI{{\mathbb{I}}}
\def\sJ{{\mathbb{J}}}
\def\sK{{\mathbb{K}}}
\def\sL{{\mathbb{L}}}
\def\sM{{\mathbb{M}}}
\def\natural{{\mathbb{N}}}
\def\sO{{\mathbb{O}}}
\def\sP{{\mathbb{P}}}
\def\rational{{\mathbb{Q}}}
\def\real{{\mathbb{R}}}
\newcommand{\R}{\mathbb{R}}
\def\sS{{\mathbb{S}}}
\def\sphere{{\mathbb{S}}}
\def\sT{{\mathbb{T}}}
\def\sU{{\mathbb{U}}}
\def\sV{{\mathbb{V}}}
\def\sW{{\mathbb{W}}}
\def\sX{{\mathbb{X}}}
\def\sY{{\mathbb{Y}}}
\def\integer{{\mathbb{Z}}}
\def\indicator{{\mathbbm{1}}}

% Entries of a matrix
\def\emLambda{{\Lambda}}
\def\emA{{A}}
\def\emB{{B}}
\def\emC{{C}}
\def\emD{{D}}
\def\emE{{E}}
\def\emF{{F}}
\def\emG{{G}}
\def\emH{{H}}
\def\emI{{I}}
\def\emJ{{J}}
\def\emK{{K}}
\def\emL{{L}}
\def\emM{{M}}
\def\emN{{N}}
\def\emO{{O}}
\def\emP{{P}}
\def\emQ{{Q}}
\def\emR{{R}}
\def\emS{{S}}
\def\emT{{T}}
\def\emU{{U}}
\def\emV{{V}}
\def\emW{{W}}
\def\emX{{X}}
\def\emY{{Y}}
\def\emZ{{Z}}
\def\emSigma{{\Sigma}}

% entries of a tensor
% Same font as tensor, without \bm wrapper
\newcommand{\etens}[1]{\mathsfit{#1}}
\def\etLambda{{\etens{\Lambda}}}
\def\etA{{\etens{A}}}
\def\etB{{\etens{B}}}
\def\etC{{\etens{C}}}
\def\etD{{\etens{D}}}
\def\etE{{\etens{E}}}
\def\etF{{\etens{F}}}
\def\etG{{\etens{G}}}
\def\etH{{\etens{H}}}
\def\etI{{\etens{I}}}
\def\etJ{{\etens{J}}}
\def\etK{{\etens{K}}}
\def\etL{{\etens{L}}}
\def\etM{{\etens{M}}}
\def\etN{{\etens{N}}}
\def\etO{{\etens{O}}}
\def\etP{{\etens{P}}}
\def\etQ{{\etens{Q}}}
\def\etR{{\etens{R}}}
\def\etS{{\etens{S}}}
\def\etT{{\etens{T}}}
\def\etU{{\etens{U}}}
\def\etV{{\etens{V}}}
\def\etW{{\etens{W}}}
\def\etX{{\etens{X}}}
\def\etY{{\etens{Y}}}
\def\etZ{{\etens{Z}}}

\def\ceil#1{\lceil #1 \rceil}
\def\floor#1{\lfloor #1 \rfloor}
\def\eps{{\epsilon}}

\newcommand{\pder}[1]{\frac{\partial}{\partial #1}}

\newcommand{\half}{\frac{1}{2}}
\newcommand{\limNinf}{\lim_{N \to \infty}}
\newcommand{\limTzero}{\lim_{\tau \to 0}}


\newcommand{\cmark}{\ding{51}}
\newcommand{\xmark}{\ding{55}}

\newcommand{\layer}{\mathcal{H}}
\newcommand{\defeq}{\triangleq}
%\newcommand{\defeq}{vcentcolon=}
\newcommand{\domain}{\Omega}
\newcommand{\grad}{\nabla}

\newcommand{\cin}{c_{\rm{in}}}
\newcommand{\cout}{c_{\rm{out}}}
\newcommand{\intdomain}{\int_{\domain}}
\newcommand{\network}{\gT}
\newcommand{\subnet}{\gK}
\newcommand{\map}{\gR} %\gR

\newcommand{\innerproduct}[2]{\langle #1, #2 \rangle}
\newcommand{\mcsum}[1][j]{\frac{1}{N}\sum_{#1=1}^N}

\newcommand{\inrspace}[1][c]{\gF_{#1}}

\DeclareMathOperator*{\argmax}{arg\,max}
\DeclareMathOperator*{\argmin}{arg\,min}

\let\ab\allowbreak


\usepackage{hyperref}
\usepackage{url}
\usepackage{chngpage}

\newif\ifNota
\Notatrue
\newcommand{\nota}[1]{\ifNota \textcolor{red}{#1}\fi}


\AtBeginDocument{%
  \providecommand\BibTeX{{%
    \normalfont B\kern-0.5em{\scshape i\kern-0.25em b}\kern-0.8em\TeX}}}

\setlength{\headheight}{16pt}

\begin{document}



%% Rights management information.  This information is sent to you
%% when you complete the rights form.  These commands have SAMPLE
%% values in them; it is your responsibility as an author to replace
%% the commands and values with those provided to you when you
%% complete the rights form.
\setcopyright{acmcopyright}
\copyrightyear{2023}
\acmYear{2023}
\acmDOI{XXXXXXX.XXXXXXX}

%% These commands are for a PROCEEDINGS abstract or paper.
\acmConference[arXiv]{pre-print arXiv}{July 2023}{}
%
%  Uncomment \acmBooktitle if th title of the proceedings is different
%  from ``Proceedings of ...''!
%
\acmBooktitle{pre-print ArXiv, July, 2023} 
%\acmPrice{15.00}
%\acmISBN{978-1-4503-XXXX-X/18/06}



\title{The GANfather: Controllable generation of malicious activity to improve defence systems}

\author{Ricardo Ribeiro Pereira}
\authornote{Corresponding author, ricardo.ribeiro@feedzai.com}

\iffalse
\authornote{Feedzai, Portugal}
\authornote{University of Porto, Portugal}
\author{Jacopo Bono\footnotemark[1]}
\author{João Tiago Ascensão\footnotemark[1]}
\author{David Aparício\footnotemark[2]}
\author{Pedro Ribeiro\footnotemark[2]}
\author{Pedro Bizarro\footnotemark[1]}
\fi

\iftrue
\affiliation{%
  \institution{Feedzai / Porto University}
  \country{Portugal}
}
\author{Jacopo Bono}
\affiliation{%
  \institution{Feedzai}
  \country{Portugal}
}
\author{João Tiago Ascensão}
\affiliation{%
  \institution{Feedzai}
  \country{Portugal}
}
\author{David Aparício}
\affiliation{%
  \institution{Porto University}
  \country{Portugal}
}
\author{Pedro Ribeiro}
\affiliation{%
  \institution{Porto University}
  \country{Portugal}
}
\author{Pedro Bizarro}
\affiliation{%
  \institution{Feedzai}
  \country{Portugal}
}
\authornote{Corresponding author, pedro.bizarro@feedzai.com}
\fi




\begin{abstract}
Machine learning methods to aid defence systems in detecting malicious activity typically rely on labelled data.
In some domains, such labelled data is unavailable or incomplete.
In practice this can lead to low detection rates and high false positive rates, which characterise for example anti-money laundering systems.
In fact, it is estimated that 1.7--4 trillion euros are laundered annually and go undetected.
We propose \textit{The GANfather}, a method to generate samples with properties of malicious activity, without label requirements.
We propose to reward the generation of malicious samples by introducing an extra objective to the typical Generative Adversarial Networks (GANs) loss.
Ultimately, our goal is to enhance the detection of illicit activity using the discriminator network as a novel and robust defence system.
Optionally, we may encourage the generator to bypass pre-existing detection systems.
This setup then reveals defensive weaknesses for the discriminator to correct.
We evaluate our method in two real-world use cases, money laundering and recommendation systems.
In the former, our method moves cumulative amounts close to 350 thousand dollars through a network of accounts without being detected by an existing system.
In the latter, we recommend the target item to a broad user base with as few as 30 synthetic attackers.
In both cases, we train a new defence system to capture the synthetic attacks.
\end{abstract}



\maketitle


%!TEX root = ../main.tex

\section{Introduction}
\label{sec:intro}

Climate change and the decline of species richness are severe challenges that influence the living conditions of humans around the world.
Especially the dramatic loss of insects~\cite{hallmann2017more,wagner2021insect} plays a crucial role in many ecological processes that affect agriculture and others.
Hence, monitoring insect species populations becomes more important nowadays to better understand insect decline and long-term trends in species distributions.
Furthermore, there are about one million named species on our planet~\cite{stork2018many}, making manual counting of individuals unrealistic.
Consequently, automated monitoring of insects is inevitably required to infer abundance estimations across larger regions.
One possible way is to use camera traps to collect images of insects that computer vision algorithms can then process to recognize the depicted species automatically.

In this paper, we focus on nocturnal insects, mainly nocturnal moths (Lepidoptera).
Even for this subset, there exist hundred thousands of different species worldwide and depending on the habitat, species lists can be narrowed down based on the study region.
For example, image datasets containing hundreds of moth species from Ecuador and Costa Rica are publicly available and can directly be used for evaluating fine-grained recognition algorithms~\cite{Rodner15:FRD}.
Here, we are interested in monitoring moth species in Central Europe.
We present datasets of moth images we have collected so far and our analysis of algorithms for insect localization and species classification.

% Figure environment removed

Our work is part of a larger project called AMMOD\footnote{\scriptsize{AMMOD = \textbf{A}utomated \textbf{M}ultisensor Station for \textbf{M}onitoring \textbf{o}f Bio\textbf{d}iversity (\url{https://ammod.de/})}}, which aims at developing self-sustaining multi-sensor stations for monitoring species diversity~\cite{Waegele22:TAM}.
One component of these stations is a light-based camera trap for nocturnal insects, called the \emph{moth~scanner}~\cite{Radig2021:AVL,Korsch21_DLP}.
It is a non-invasive monitoring system for automatically gathering images at nighttime.
A UV-LED lamp illuminates a white planar surface to attract the insects that land on this surface.
A high-resolution camera takes an image of the whole surface every two minutes.
Our prototype is shown in Figure~\ref{fig:prototype}.

With this setup, we can collect large-scale datasets of nocturnal insects over a long period that can then be used to develop and evaluate appropriate fine-grained species recognition algorithms.
The moth scanner takes several hundred images during one night, and within five months, we collected more than \num{27000} images with our prototype.
In this paper, we refer to the resulting dataset as the \emph{nocturnal insects dataset~(NID)}, and more details are given in Section~\ref{sec:dataset}.
Note that this dataset is supposed to be extended over time as our system will be in operation within the following years.
We plan to maintain multiple sensor stations in parallel at different locations.
Hence, it has the potential to become a valuable source for large-scale learning and continuous learning within a fine-grained domain.

Besides its impact on research in fine-grained recognition, our developments for automated visual monitoring of nocturnal insects are beneficial for ecologists.
Until now, insect monitoring is mainly done by hand and supported by citizen scientists who manually take images of individual insects in their gardens. 
Previously, we published an image dataset of nocturnal moths captured manually by citizen scientists, called \emph{\mbox{EU-Moths}} dataset at a local workshop~\cite{Korsch21_DLP}.
This paper also includes a dataset description and our baseline results for insect localization and species classification.
There are two reasons for this.
First, we want to announce this dataset to a broader audience interested in fine-grained recognition because it can directly be used for algorithm development and evaluation.
Second, we want to highlight the challenges for recognition algorithms that arise when processing automatically captured camera trap images compared to manually taken images with hand-held cameras.

In general, our paper aims to promote the application of moth species identification as a fine-grained visual recognition problem.
We underpin this with existing datasets, results of baseline algorithms, and a light-based camera trap setup that will be used during the following years to automatically collect further large-scale image data.
We believe that research on automated visual identification of hundreds to thousands of different nocturnal moth species can have a major impact on developing fine-grained recognition algorithms in general, and we, therefore, want to share our insights and datasets with the community.

%The rest of the paper is structured as follows. 
%After a short review of related work in Section~\ref{sec:related_work}, we describe the two abovementioned datasets containing images of nocturnal moths in Section~\ref{sec:dataset}.
%The algorithms we applied to both datasets are described in Section~\ref{sec:methods} and we present the achieved results in Section~\ref{sec:results}. 
%We discuss challenges of processing automatically captured images with light-based camera traps in Section~\ref{sec:challenges} that are also important to consider for similar projects, followed by conclusions in Section~\ref{sec:conclusions}.

%\todo{REWRITE from here}

%Before the classification can be performed, we need to perform a detection of the insects.
%At this stage, the application of the state-of-the-art detection models like SSD~\cite{liu2016ssd} or YOLO~\cite{redmon2016you} is an obvious step.
%On the other hand, these models are computationally expensive and other light-weight methods like the MCC blob detector~\cite{bjerge2021automated} are more suitable for the application in the field.
%Unfortunately, to evaluate and compare different detection methods suitable benchmark datasets are missing.

%In this paper, we present a new dataset collected with the help of our prototype.
%In the period of five months, we captured over \num{27000} images in suburban area in Middle Germany.
%For bootstrapping and first evaluations we annotated a subset of these images with bounding boxes for the captured insects.
%The image data and the annotations will be soon publicly available.

%As a first baseline for insect detection task, we evaluated two different methods on the data and present these results further in our paper.
%First, we used a well-established Deep Learning detection model capable of identifying multiple objects in an image, namely the single-shot MultiBox detector (SSD)~\cite{liu2016ssd}.
%As a light-weight alternative that can be easily deployed directly at the computationally limited hardware of the moth scanner, we developed and evaluated a multi-step blob detection algorithm.
% \todo{Edge Computing as buzzword? EdgeAI may be wrong here?}
%First, it reduces the power consumption due to the reduction of computations.
%Further, applying the detection directly at the moth scanner, we can drastically reduce the amount of data that needs to be transmitted  when the system will gather data autonomously in the field.
%The algorithm is closely related to the blob detection method proposed by Bjerge~\etal\cite{bjerge2021automated} but mitigates some of the method's limits.
%We present the idea and the improvement in more detail in Sect.~\ref{sec:methods}.




% Figure environment removed

\section{Methods}
\label{sec:method}

We provide a general description of our proposed framework in Section~\ref{subsec:overview}.
We proceed to describe two use-cases:  anti-money laundering (AML) (Section~\ref{subsubsec:amlusecase}) and detection of injection attacks in recommendation systems (Section~\ref{subsubsec:rsusecase}).
In Section~\ref{sec:theory}, we show theoretically, in a simplified setting, how our generator's loss function changes the learning dynamics compared to a typical GAN.

\subsection{General description}
\label{subsec:overview}

Figure~\ref{fig:03_fulldiagram} depicts the general structure of our framework.
It comprises a generator, a discriminator, an optimisation objective, and, optionally, an existing alert system. Each component is discussed in more detail below.


\textbf{Generator.} As in the classical GAN architecture, the generator $G$ receives a random noise input vector and outputs an instance of data.
However, unlike classical GANs, the loss of the generator  $\mathcal{L}(G)$ is a linear combination of three components: the optimisation objective for malicious activity $\mathcal{L}_{Obj}(G)$, the GAN loss $\mathcal{L}_{GAN}(G,D)$ that additionally depends on the discriminator $D$, and the loss from an existing detection system $A$, $\mathcal{L}_{Alert}(G,A)$:
\begin{equation}
    \mathcal{L}(G) = \alpha \mathcal{L}_{Obj}(G) + \beta \mathcal{L}_{GAN}(G,D) + \gamma\mathcal{L}_{Alert}(G,A)
    \label{eq:generator_loss}
\end{equation}
where $\alpha$, $\beta$ and $\gamma$ are hyperparameters to tune the strength of each component. The last term is optional, and if no existing detection system is present we simply choose $\gamma = 0$.  Note also that one of the parameters is redundant and we tune only two parameters in our experiments (or one if $\gamma = 0$).

We show in Section~\ref{sec:theory} that the stable point of convergence for the generator in our theoretical example moves away from the data distribution for any $\alpha > 0$. 

\textbf{Discriminator.} The discriminator setup is the same as in a classical GAN. It receives an example and produces a score indicating the likelihood that the example is real or synthetic. Importantly, as explained in Section~\ref{sec:theory}, the generator subject to Equation~\ref{eq:generator_loss} will generate data increasingly out of distribution for larger $\alpha$. Therefore, we do not require the discriminator accuracy to fall to chance level at training convergence, as is usual with GANs. Instead, the discriminator may converge to perfect classification and may be used as a detection system for illicit activity. In our experiments, we use the Wasserstein loss \citep{arjovsky2017wasserstein} as our GAN loss.

\textbf{Malicious optimisation objective.} The optimisation objective quantifies how well the synthetic example is fulfilling the goal of a malicious agent. It can be a mathematical formulation or a differentiable model of the goal. This objective allows the generator to find previously unseen strategies to meet malicious goals.

\textbf{Alert system.} If an existing, differentiable alert system is present, we can add it to our framework to teach the generator to create examples that do not trigger detection (see Equation~\ref{eq:generator_loss}). In that scenario, it is then beneficial for the discriminator to focus on the undetected illicit activity. Whenever the existing system is not differentiable, training a differentiable proxy may be possible.

\textbf{Generator vs. Discriminator views.} If required by the malicious optimisation objective, our generator can be adapted to generate samples which are only partially evaluated by the discriminator. For example, the layering stage of money laundering typically involves moving money through many financial institutions (FIs). However, each detection system operates within single institutions, limiting their view of the entire operation. Our method can be adapted to capture this situation, by generating samples containing various fictitious FIs, but only sending the partial samples corresponding to each FI to the discriminator. In recommender systems, the malicious objective can act on a group of synthetic illicit actors to generate coordinated attacks, while the detection of fraudulent users is typically performed on a single-user level.

\textbf{Architecture optimisations.} In the next sections, we provide more details about the specific architectures used in the two experiments. We note that the architecture details (layer types, widths and number of layers) were first optimised using a vanilla GAN setup (i.e. setting $\alpha=0, \beta=1, \gamma=0$ in Equation~\ref{eq:generator_loss}). With the architecture fixed, the other hyperparameters were tuned as explained in the next sections.

\textbf{Code availability.} The Pytorch code for both models can be found on GitHub
(the link will be provided after double blind review).  
%at https://github.com/feedzai/ganfather .


\subsection{Anti-Money Laundering (AML)}
\label{subsubsec:amlusecase}

We tackle the layering stage of money laundering, in which criminals attempt to conceal the origin of the money by moving large amounts across financial institutions through what are known as ``mule accounts''. 

\textbf{Representing dynamic graphs as tensors.} To represent the dynamic graph of transactions, we can use a 3D tensor as depicted in Figure~\ref{fig:03_amldatarepresentation}. We assume the nodes of the dynamic graph are accounts, and the edges are transactions.
The first two dimensions correspond to the weighted adjacency matrix of the accounts and the third dimension is time.
We discretise the events into time windows of fixed length and group events that belong to the same entry in the tensor by summing their amounts. In other words, each entry $A_{ijk}$ of the tensor corresponds to the cumulative amount sent between account $i$ and account $j$ on timestep $k$.
Our representation covers any dynamic network with a 3D tensor whose size is fixed and pre-specified, which allows us to avoid using recurrent models.
While this approach may limit the size of generated data, domain experts reported that up to 95\% of the money-laundering investigations involve cases containing up to 5 accounts.

% Figure environment removed

\textbf{Architecture.} We implement the generator using a set of dense layers, followed by a set of transposed convolutions. 
Then, we create two branches: one to generate transaction amounts and the other to generate transaction probabilities.
We use the probabilities to perform categorical sampling and generate sparse representations, similar to real transaction data. 
After the sampling step, the two branches are combined by element-wise multiplication, resulting in a final output tensor with the dimensions described above. 
%More details about the generator's architecture can be found on our GitHub repository.

The discriminator receives two tensors with the same shape as inputs: one containing the total amount of money transferred per entry, and the other with the count information (mapping positive amounts to 1 and empty entries to 0). Each tensor passes through convolutional layers, followed by permutation-invariant operations over the internal and external accounts. Then, we concatenate both tensors. We reduce the dimensionality of the resulting vector to a classification outcome using dense layers.
%More details about the discriminator's architecture can be found on our GitHub repository.

We provide more details about both architectures on our GitHub repository.


\textbf{Money Mule objective.} To characterise the money flow behaviour of layering, where money is moved in and out of accounts while leaving little behind, we define the objective function as the geometric mean of the total amount of incoming ($G(z)_{in}$) and outgoing ($G(z)_{out}$) money per generated account (Equation~\ref{eq:aml_reward}).

\begin{equation}
    \mathcal{L}_{Obj}(G) = -\int \sqrt{ G(z)_{in} \times G(z)_{out} } \cdot p(z) dz
    \label{eq:aml_reward}
\end{equation}
Here $z$ represents random noise input to the generator $G$ and $p(z)$ is its probability distribution. This objective encourages the generator to increase the amount of money sent and received per account and keep these two quantities similar, as observed in mule accounts.

\textbf{Existing Alert System.} In AML, it is common to have rule-based detection systems. In our case, the rules detection system contains five alert scenarios, capturing known suspicious patterns such as a sudden change in behaviour or rapid movements of funds. However, these rules are not differentiable, and our generator requires feedback in the form of a gradient. Hence, we construct a deep learning model as a proxy for the rules system. We hard-code a neural network mimicking the rules' logic operations by choosing the weights, biases and activation functions appropriately. This network gives the same feedback as the rules system would, but in a differentiable way.


\subsection{Recommendation System}
\label{subsubsec:rsusecase}

In this work, we consider collaborative filtering recommender systems. However, our method is compatible with any other differentiable recommender systems.
The system receives a matrix of ratings $R$ with shape $(N_u, N_i)$, where $N_u$ is the number of users and $N_i$ is the number of items.
First, we compute cosine distances between users, resulting in the matrix $D$ of shape $(N_u,N_u)$. Then, we compute the predicted ratings $P$ as a matrix product between $D$ and $R$.
We decided to not represent time since most classical recommender systems do not account for it. However, it is possible to include temporal information using a similar setup to what we described in the AML use case. We also note that, unlike in the AML scenario, we do not have an existing detection system in this setup.
We provide details about the architectures of both the generator and the discriminator on our GitHub repository.


\textbf{Injection Attack Objective.} We define the goal of malicious agents as increasing the frequency of recommendation of a specific item.
The objective function in Equation~\ref{eq:rs_objective} incentivizes the generator to increase the rating of the target item $t$ for every user.

\begin{equation}
    \mathcal{L}_{Obj}(G) = \int \sum_{i}^{N_u}\sum_{j}^{N_i} (P_{ij}(z)-P_{it}(z))_+ \cdot p(z)dz
    \label{eq:rs_objective}
\end{equation}
Here, the matrix of predicted ratings $P$ depends on the random inputs $z$ through the generator $G$ and $(\cdot)_+$ denotes a rectifier setting negative values to zero.


\subsection{Theoretical justification}
\label{sec:theory}
In this section, we provide a theoretical justification to enlighten certain aspects of our setup, in a simplified setting. We will assume no existing detection system is available ($\gamma$ = 0 in Equation~\ref{eq:generator_loss}). In the case such a system would be available, we assume its effect is to limit how far the generated data distribution can be from the real data distribution. Furthermore, we assume that a malicious objective would promote a change in the distribution of at least one feature of the generated data compared to the real data. 

In order to facilitate the analytical calculations, we make the following simplifying assumptions. Firstly, we assume that our data consists of only one feature, for which the regular (legitimate) activity is distributed following a normal distribution $p_{\text{data}}$ with mean $\mu_d$ and standard deviation $\sigma_d$:
\begin{equation}
    p_{\text{data}} = \mathcal{N}\left(\mu_d, \sigma_d\right)
\end{equation}
Secondly, we assume that we do not have any samples of malicious activity but that we know that it is characterised by larger values of this feature compared to the legitimate activity. Thirdly, we assume that the generated data follows a normal distribution $p_{\text{gen}}$ with mean $\mu_g$ and standard deviation $\sigma_g$. Using $\gamma =0$ and $\beta=1-\alpha$ in Equation~\ref{eq:generator_loss}, assuming $0\leq\alpha\leq1$, we can write the training criterion of the generator as:
\begin{equation}
    \mathcal{L}(G) = (1-\alpha) \cdot \left(2 \cdot \text{JSD}\left( p_{\text{data}} | p_{\text{gen}} \right) - \text{log}(4) \right) - \alpha \mu_g \label{eq_loss_gen}
\end{equation}
where the first term denotes the GAN loss \cite{goodfellow2014generative} and the second term denotes our \emph{malicious objective} rewarding the generator to produce samples with properties of the malicious data (i.e. increase the mean $\mu_g$ as much as possible).

We can analytically solve the Jenson-Shannon Divergence (JSD) between the normal distributions, using $\sigma_m^2 = \sigma_d^2 + \sigma_g^2$,
\begin{align}
    \text{JSD}\left( p_{\text{data}} | p_{\text{gen}} \right)
    & = \frac{1}{2} \text{KL}\left(p_{\text{data}} | 0.5*(p_{\text{data}} + p_{\text{gen}}) \right) \nonumber \\ & \qquad + \frac{1}{2} \text{KL}\left(p_{\text{gen}} | 0.5*(p_{\text{data}} + p_{\text{gen}}) \right) \nonumber \\
    & = \frac{1}{2} \left[\log \frac{\sigma_m}{\sigma_d} + \frac{\sigma_d^2 + (\mu_d - 0.5(\mu_d + \mu_g))^2}{2\sigma_m^2} - \frac{1}{2} \right. \nonumber \\  & \left. \qquad + \log \frac{\sigma_m}{\sigma_g} + \frac{\sigma_g^2 + (\mu_g - 0.5(\mu_d + \mu_g))^2}{2\sigma_m^2} - \frac{1}{2} \right]
\end{align}

From this, we can calculate the gradient w.r.t. $\mu_g$:
\begin{align}
    \frac{\partial \text{JSD}(p_{\text{data}} | p_{\text{gen}})}{\partial \mu_g} &= \partial \left( \frac{1}{2} \left[\log \frac{\sigma_m}{\sigma_d} + \frac{\sigma_d^2 + (\mu_d - 0.5(\mu_d + \mu_g))^2}{2\sigma_m^2} - \frac{1}{2} \right. \right. \nonumber \\
    & \left. \left. + \log \frac{\sigma_m}{\sigma_g} + \frac{\sigma_g^2 + (\mu_g - 0.5(\mu_d + \mu_g))^2}{2\sigma_m^2} - \frac{1}{2} \right] \right) / \partial \mu_g \nonumber \\
    %& = \frac{1}{2} \partial \left( \frac{(0.5 \mu_d - 0.5 \mu_g)^2}{2\sigma_m^2} + \frac{(0.5 \mu_g - 0.5 \mu_d)^2}{2\sigma_m^2} \right) / \partial \mu_g \\
    & = \frac{\mu_g - \mu_d}{4\sigma_g^2 + 4\sigma_d^2} \label{eq_grad_jsd}
\end{align}

Combining (\ref{eq_loss_gen}) and (\ref{eq_grad_jsd}), we find that the gradient of the training objective of the generator w.r.t. the mean of the generated distribution $\mu_g$ is
\begin{align}
    \frac{\partial \mathcal{L}(G)}{\partial \mu_g} &= \frac{(1-\alpha)}{2} \frac{\mu_g - \mu_d}{\sigma_g^2 + \sigma_d^2} - \alpha
\end{align}

Without loss of generality, we set $\sigma_g^2 + \sigma_{\text{data}}^2 = k/2$, such that
\begin{align}
    \frac{\partial \mathcal{L}(G)}{\partial \mu_g} &= (1-\alpha) \frac{\mu_g - \mu_d}{k} - \alpha
\end{align}

Denoting $\frac{\partial \mu_g}{\partial t}$ as the changes of $\mu_g$ over time (i.e. a continuous version of the discrete gradient updates) and $\eta$ as the learning rate, this leads to the following linear dynamical system which we can analyse in function of $\mu_g$, $\mu_{\text{d}}$ and the hyperparameter $\alpha$:
\begin{align}
    \frac{\partial \mu_g}{\partial t} &= - \eta \frac{\partial \mathcal{L}(G)}{\partial \mu_g} \nonumber \\
    %\frac{\partial \mu_g}{\partial t} &= - \eta (1-\alpha) \frac{\mu_g - \mu_d}{k} + \eta \alpha \\
    &= - \eta (1-\alpha) \frac{\mu_g - \mu_d}{k} + \eta \alpha \nonumber \\
    %\frac{\partial \mu_g}{\partial t} &= -\eta d\mu_g + \eta d \mu_d+ \eta \alpha
    &= -\eta d\mu_g + \eta d \mu_d+ \eta \alpha
\end{align}
where we defined $d = (1-\alpha)/k$. The stability of this linear system is defined by the sign of $-d$, which is always negative and hence the system has a stable fixed point.
The stable fixed point for this dynamical system is easily found to be 
\begin{align}
    \mu_g^{\star} &= \mu_d + \frac{\alpha}{1-\alpha}k
\end{align}
We plot the phase diagram of the dynamical system in Figure \ref{fig:phase}, showing the fixed point in function of the parameter $\alpha$.

% Figure environment removed

\vspace{2.2cm}

From these simplified setting calculations, we can conclude that:
\begin{itemize}
    \item For $\alpha>0$, our generated data will move away from the real data distribution and increasingly comply with the malicious objective.
    \item Different values of $\alpha$ will result in varying levels of deviation from the real data. In the absence of ground truth to evaluate the system, hyperparameter tuning and empirical testing are necessary.
    \item When generated data deviates from real data, the discriminator will increasingly achieve a perfect performance even at training completion. This is a major difference to standard GAN training.
\end{itemize}


% Figure environment removed

\section{Results}
\label{sec:results}

We evaluate the efficiency of \textit{The GANfather} to generate and detect attacks in two use-cases: money laundering (Section~\ref{sec:aml_exps}) and recommendation system (Section~\ref{sec:recs_exps}).



\subsection{Money Laundering}
\label{sec:aml_exps}

\textbf{Setup.}
We use a real-world dataset of financial transactions, containing approximately 200,000 transactions, between 100,000 unique accounts, over 10 months\footnote{Due to the confidential nature we cannot disclose the actual dataset.}.
Some of these accounts are labelled as suspicious of money laundering.
We build a real test set of $5000$ accounts, $184$ of which are label positive (suspicious).
We implement \textit{The GANfather}'s generator and discriminator following the architectures presented in Section~\ref{subsubsec:amlusecase}. 


\textbf{Results.}
We conduct a hyperparameter random search over the learning rate ($[10^{-4}, 3\times10^{-3}]$) and the weights $\alpha$ (set to $1$), $\beta$ ($[10^2, 10^5]$) and $\gamma$ ($[10^3, 4\times10^3]$) mentioned in Equation~\ref{eq:generator_loss}).

In Figure~\ref{fig:04_aml_comparison_distributions}, we compare the distribution of money flows from such a generator compared to the real data distribution.
We can observe that the generated samples successfully move more money through the accounts than real data (up to 350,000 dollars vs. up to 9,000 dollars respectively).
Interestingly, the distribution of amounts used is similar to real data, and the main difference is the number of transactions used.

Next, we test the detection performance of the trained discriminators on generated data.
To detect potential bias in a discriminator trained solely on samples of the corresponding generator, we first build a \emph{mixed} dataset, where synthetic malicious data is sampled from various generators.% at various epochs during training and with different random noise seeds.
We combine this synthetic dataset with real data, and use it to evaluate the trained discriminators.
Importantly, no retraining on this mixed dataset is performed.
We observe that most discriminators can distinguish between real and generated examples with $100\%$ accuracy, especially those trained with higher values of the $\beta$ hyperparameter (see Equation~\ref{eq:generator_loss}, and note that in this experiment $\alpha$ was fixed to a value of 1).
%This can be understood because the $\beta$ parameter limits the generated data distribution to diverge significantly from the real data distribution.
%Therefore, discriminators trained with larger $\beta$ need to more accurately learn a decision boundary around the real data, in turn becoming more robust when evaluating on the mixed dataset.

Then, we evaluate the detection performance on the real test set.
We train a model $DM$ with the same architecture as the discriminator using the mixed dataset mentioned in the previous paragraph.
This training \emph{does not require real labels}, since we use generated data as positive examples (suspicious) and assume that all real examples are negative (legitimate).
After training, we evaluate three detection scenarios: the set of rules mentioned in Section~\ref{subsubsec:amlusecase}; the model $DM$, with the threshold tuned to match the alert rate of the rules\footnote{We assume that the rules are fixed, so we cannot tune the number of their alerts.}; a combination of both (alert if either of them triggers).
The results are shown in Table~\ref{tab:03_aml_reallabels}.
We see that, even though the model $DM$ was trained using only generated data as positive examples, it achieves better detection performance than the rules.
Furthermore, only $10$ of the $128$ alerts of the Rules+Model scenario were alerted by both detection systems, and the true positives had little overlap as well ($5$ out of $54$).
%Furthermore, when we compare the triggers and the true positives of the rules and $DM$, we see that there is little overlap between them ($10$ out of $128$ alerts, $5$ out of $54$ true positives).
This means that, by including the rules' feedback in the loss of the generator, it learns to create synthetic examples that are not captured by the rules but are similar to real examples of suspicious activity.
As such, a model trained with those synthetic examples can be used to complement the rules, with the advantage of being easy to tune to a desired alert rate.

\begin{table}[H]
    \centering
    \begin{tabular}{l|c|c|c}
        & Alert Rate \% & Recall \% & Precision \% \\
        \hline
        Rules & 1.4 & 13.6 & 36.2\\
        Model & 1.4 & 18.5 & 49.3\\
        Rules + Model & 2.6 & 29.3 & 42.2
    \end{tabular}
    \caption{Detection of real labels.}
    \label{tab:03_aml_reallabels}
\end{table}


\subsection{Recommender System}
\label{sec:recs_exps}

\textbf{Setup.} We use the MovieLens 1M dataset\footnote{https://www.kaggle.com/datasets/odedgolden/movielens-1m-dataset}, comprised of a matrix of $6040$ users and $3706$ movies, with ratings ranging from 1 to 5 \citep{harper2015movielens}.
We implement the generator and discriminator and collaborative filtering recommender system as described in Section~\ref{subsubsec:rsusecase}.
To compute the predicted ratings, during training we take a weighted average of ratings considering all users in the dataset.
We consider all users during training because the initially generated ratings are random, and only providing feedback from the top-N closest users limits the strategies that the generator can learn.
In contrast, we consider the top-400 closest neighbours to compute predicted ratings at inference since we observed empirically that this value produces the lowest recommendation loss.

In this scenario, we do not use an existing detection component, corresponding to $\gamma = 0$ in Equation~\ref{eq:generator_loss}.
We train our networks with 300 synthetic attackers but evaluate the generator's ability to influence the recommender system with injection attacks of various sizes.
We also define four baseline attacks: (1) a rating of 5 for the target movie and 0 otherwise, (2) a rating of 5 for the target movie and $\sim$90 random ratings for randomly chosen movies, (3) a rating of 5 for the target movie and $\sim$90 random ratings for the top 10\% highest rated movies, (4) a rating of 5 for the target movie and $\sim$90 random ratings for the top 10\% most rated movies.

\textbf{Results.} We choose $\beta = 1-\alpha$ in Equation~\ref{eq:generator_loss}, with $0\leq \alpha \leq 1$ and perform a hyperparameter search over $\alpha$.
We observe that increasing $\alpha$ leads to generators whose attacks increasingly recommend the target movie, at the cost of moving further away from the rating distributions of real profiles.

In Table~\ref{tab:03_rs_attack}, we show how many real users have the target movie in their top-10 recommendations, depending on the number of generated users that we inject and how they were generated (through \textit{The GANfather} or the described baselines).
We observe that even with a very limited proportion of generated users (30 among 6040 real users, $~0.5\%$), they are able to greatly influence many real users ($~3.7\%$).
In contrast, the baselines have very small impact on the recommendations of real users.
Lastly, as expected, increasing the number of injected users increases the target movie's recommendation frequency to real users.


\begin{table}
    \centering
    \begin{tabular}{l|c|c|c}
        Generation strategy & 30 users & 60 users & 120 users \\
        \hline
        \textbf{The GANfather} & \textbf{225} & \textbf{290} & \textbf{428} \\
        Baseline 1 & 0 & 0 & 0\\
        Baseline 2 & 0 & 0 & 0\\
        Baseline 3 & 1 & 3 & 7\\
        Baseline 4 & 0 & 0 & 0
    \end{tabular}
    \caption{Number of real users with the target movie in their \mbox{top-10} recommendations, after injecting 30, 60, or 120 \mbox{generated} users.}
    \label{tab:03_rs_attack}
\end{table}

\iffalse
\begin{table}[H]
    \centering
    \begin{tabular}{l|c|c|c}
        Generation strategy & 30 users & 60 users & 120 users \\
        \hline
        \textbf{The GANfather} & \textbf{2218} & \textbf{4497} & \textbf{5495} \\
        Baseline 1 & 3 & 3 & 3\\
        Baseline 2 & 3 & 3 & 3\\
        Baseline 3 & 13 & 35 & 77\\
        Baseline 4 & 14 & 23 & 35
    \end{tabular}
    \caption{Number of real users with the target movie in their \mbox{top-50} recommendations, after injecting 30, 60, or 120 \mbox{generated} users.}
    \label{tab:03_rs_attack}
\end{table}
\fi

Finally, we analyse the detection of synthetic attacks.
As in the AML scenario we build a test set containing real and synthetic data, where the synthetic data contains a mixture of samples from various trained generators to identify the possible bias of a discriminator to attacks by the corresponding generator.
We then quantify the AUC of the trained discriminators.
We observe that most discriminators trained in a GAN setting (taking turns with a generator to update their weights) achieve around $0.75$ AUC.
Unlike the AML scenario, this suggests that the discriminators are tuned to detect synthetic data from their respective generators, but less so from other generators.
If instead we build a \emph{mixed} training set combining real samples with synthetic data from various generators and use it to retrain a discriminator, it achieves near-perfect classification (above $0.99$ AUC).


\section{Related Work}
\label{sec:related}

\textbf{Controllable data generation.} \citet{wang2022controllable} review controllable data generation with deep learning. Among the presented works, we highlight \cite{de2018molgan}. It leverages a GAN trained with reinforcement learning to generate small molecular graphs with desired properties. Their work is similar to ours in that we both (1) extend a GAN with an extra objective and (2) use similar data representations, namely sparse tensors. However, whereas \cite{de2018molgan} uses a labelled dataset of molecules and their chemical properties, our method does not rely on any labelled data. 


\textbf{Adversarial Attacks.} A vast amount of literature exists on the generation of adversarial attacks (see \cite{xu2020adversarial} for a recent review). Such attacks have been studied in various domains and using various setups (e.g. cybersecurity evasion using reinforcement learning \citep{apruzzese2020deep}, intrusion detection evasion using GANs \citep{usama2019generative}, sentence sentiment misclassification using BERT \citep{garg2020bae}). In all cases, a requirement is that labelled examples of malicious attacks exist.

\textbf{Anti-Money Laundering.} Typical anti-money laundering solutions are rule-based~\citep{watkins2003tracking, savage2016detection, weber2018scalable}. However, rules suffer from high false-positive rates, may fail to detect complex schemes, and are costly to maintain. Machine learning-based solutions tackle these problems \citep{chen2018machine}. Given the lack of labelled data, most solutions employ unsupervised methods like clustering \citep{wang2009research, soltani2016new}, and anomaly detection \citep{gao2009application, camino2017finding}. These assume that illicit behaviours are rare and distinguishable, which may not hold whenever money launderers mimic legitimate behaviour. Various supervised methods have been explored \citep{jullum2020detecting, raza2011suspicious, lv2008rbf, tang2005developing, oliveira2021guiltywalker}, but most of these works use synthetic positive examples or incompletely labelled datasets. To avoid this, \citet{lorenz2020machine} propose efficient label collection with active learning. \citet{deng2009active} and ~\citep{charitou2021synthetic} explore data augmentation using conditional GANs. Lastly, \citet{li2020flowscope} and \citet{sun2021cubeflow} propose a metric to detect dense money flows in large transaction graphs, resulting in an anomaly score. Their method does not involve training of a classifier, and instead relies on generating many subsets of nodes and iteratively calculating the anomaly score.

\textbf{Recommender systems (RS) injection attacks.} Most injection attacks on RS are hand-crafted according to simple heuristics. Examples include random and average attacks~\citep{lam2004shilling}, bandwagon attacks~\citep{burke2005limited} and segmented attacks \citep{burke2005segment}. However, these strategies are less effective and easily detectable as most generated rating profiles differ significantly from real data and correlate with each other. \citet{tang2020revisiting} address the optimisation problem of finding the generated profiles that maximise their goals directly through gradient descent and a surrogate RS. Some studies apply GANs to RS to generate attacks and defend the system. \citet{wu2021ready} combines a graph neural network (GNN) with a GAN to generate their attack. The former selects which items to rate, and the latter decides the ratings. \citet{zhang2021attacking} and \citet{lin2022shilling} propose a similar setup to ours in which they train a GAN to generate data and add a loss function to guide the generation of rating profiles. The main differences to our work are the usage of template rating profiles to achieve the desired sparsity, the chosen architecture and loss functions. In our work, sparsity is learned by the generator through the categorical sampling branch (see Section~\ref{sec:method}). Moreover, our method allows the generation of coordinated group attacks by generating multiple attackers from a single noise vector.


\section{Conclusion}

In this paper, we proposed \ourmethod{}, an adaptation of diffusion models for motion synthesis which entangles the motion temporal-axis with the diffusion time-axis. This mechanism enables synthesizing arbitrarily long motion sequences in an autoregressive manner using a U-Net architecture. A unique aspect of our work is the notion of a \textit{stationary} motion buffer. Our framework continues to produce clean frames (i.e., progressing along the diffusion-time axis), without \textit{actually} incrementing the diffusion time.
The ability of our pipeline to continually generate motion along the diffusion axis is what enables our framework to robustly and continuously produce novel frames. Interestingly, the ability to naturally use diffusion in such an autoregressive fashion may have implications for other types of sequential data beyond motion, such as audio and video, or modalities where a sequential order can be defined, such as a patch-by-patch order for images.

Our system enables partially-clean-frame to be immediately (or near immediately) popped-off the motion buffer stack. However, a current limitation of our system is that computing a clean from from pure noise requires going through the chain of denoising diffusion.
In the future we are interested in leveraging ideas from DDIM~\cite{song2020denoising} to skip ahead during the denoising process to achieve even lower latency. In addition, our framework may enable future research in long-term text-conditioned motion generation. We are interested in exploring how high-level control may be coupled with low-level user-guidance for the task of long-term generation.




\bibliography{ref}
\bibliographystyle{ref}


%\appendix
%\begin{comment}
\section{System Architecture}
\label{appendix:architecture}
\system has a novel modularized system architecture with three key components: 
\emph{StreamManager}, 
\emph{TxnManager} and \emph{TxnScheduler}. 
These components are instantiated in each thread locally.
The execution outline of \system is presented in Algorithm~\ref{alg:algo}.
Transactional stream processing is continuous and potentially never ends (Line 1$\sim$8).
The dependency resolution and execution of state transactions are separated into two non-overlapping phases by punctuations~\cite{Tucker:2003:EPS:776752.776780} (Line 2 and 5), which guarantees that no subsequent input event will have a smaller timestamp. 
Effectively, a batch of state transactions is collected during the first phase, and processed during the second phase.

In the first phase (i.e., stream processing phase), 
the \emph{StreamManager} conducts preprocessing for every input event ($e$). Similar to some prior works~\cite{tstream}, state transactions may be issued but not immediately processed during preprocessing (Line 3).
The \emph{pre\_processing} and \emph{post\_processing} functions are exposed as APIs to users.
The \emph{TxnManager} handles dependency resolution (Line 4) among state transactions and insert decomposed operations to construct a \tpg. We discuss the detailed two-phase \tpg construction process in Section~\ref{subsec:construction}.

In the second phase  (i.e., transaction processing phase), 
the \emph{TxnManager} is first involved again to refine (Line 6) the constructed \tpg with further dependency resolution.
The \emph{TxnScheduler} 
schedules operations for concurrent execution based on the constructed \tpg according to the three dimensions of scheduling decisions (Line 7). 
In particular, a scheduling decision model $M$ is instantiated based on the constructed \tpg (Line 14).
\textbf{\circled{1}} Guided by $M$, execution threads adopt an exploration strategy (Section~\ref{subsec:explore}) to explore the constructed \tpg for operations available to be scheduled constrained by dependencies. 
\textbf{\circled{2}} 
During exploration, one or multiple operations may be treated as the 
% basic 
unit of scheduling (Section~\ref{subsec:granularity}). 
Subsequently, \textbf{\circled{3}} every thread executes operation(s) in the unit of scheduling with various abort handling mechanisms (Section~\ref{subsec:abort_handling}).
Only when state transactions are processed (i.e., committed or aborted) can the associated input events be postprocessed (Line 8) by the \emph{StreamManager} based on transaction processing results.
\end{comment}

\begin{comment}
\begin{algorithm}
\footnotesize
    \KwData{$e$ \tcp{Input event}}
    \KwData{$txn_{ts}$ \tcp{State transaction}}
    \KwData{$G$ \tcp{The currently constructed TPG}}
    \While{!finish processing of input streams}{
        \eIf(\tcp*[h]{Phase 1}){\text{$e$ is not a $punctuation$}}{
                $txn_{ts}$ $\gets$ PRE\_Processing($e$)\;
                \textbf{TPG\_Construction}($G$, $txn_{ts}$)\; 
          }(\tcp*[h]{Phase 2}){
                \textbf{TPG\_Refinement}($G$)\; 
                \textbf{TXN\_Scheduling}($G$)\; 
                POST\_Processing()\;
          }
    }
    
    \SetKwFunction{FMain}{TPG\_Construction}
    \SetKwProg{Fn}{Function}{:}{}
    \Fn{\FMain{$G$, $txn_{ts}$}}{
        $O_{1..k}$ $\gets$ \textbf{Partition} $txn_{ts}$\;
        \ForEach{\text{operation $O_{i}$ $\in$ $O_{1..k}$}}{
            \textbf{Identify} its \ld\;
            $G$ $\gets$ $G$ + $O_{i}$ \;
        }
    }
    \SetKwFunction{FMain}{TPG\_Refinement}
    \SetKwProg{Fn}{Function}{:}{}
    \Fn{\FMain{$G$}}{
        \ForEach{\text{vertex $e_{i}$ $\in$ $G$}}{
            \textbf{Identify} its \td, \pd\;
        }
    }
    
    \SetKwFunction{FMain}{TXN\_Scheduling}
    \SetKwProg{Fn}{Function}{:}{}
    \Fn{\FMain{$G$}}{
        $M$ $\gets$ Instantiated with $G$;\tcp{A decision model}
        \While{!finish scheduling of $G$
        }{
          \textbf{\circled{2}} $Scheduling Unit$ $\gets$ \textbf{\circled{1}} \emph{Explore}($G$, $M$)\; 
            \textbf{\circled{3}} \emph{Execute with Abort Handling} ($Scheduling Unit$)\; 
        }
    }
  \caption{Execution Outline of \system}
  \label{alg:algo}
\end{algorithm}
\end{comment}


\end{document}