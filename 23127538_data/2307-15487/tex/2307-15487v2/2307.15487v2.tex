\documentclass[letterpaper, 11pt]{amsart}

\usepackage{amsmath,amsthm,amsfonts,amssymb,amscd}
\usepackage{bbm}
\usepackage{bm}
\usepackage{tikz}
\usepackage{tikz-cd}
\usepackage{appendix}
%\usepackage{footnote}
%\usepackage{euler}
%\usepackage{palatino}
%\usepackage{Bickham}
%\usepackage{BOONDOX-cal}
\usepackage{BOONDOX-calo}
%\usepackage{dutchcal}
%\usepackage[symbol*]{footmisc}
\usepackage{standalone}
\usepackage[hidelinks]{hyperref}
\usetikzlibrary{arrows,chains,matrix,positioning,scopes}
\usepackage[letterpaper, left=2.8cm,right=2.8cm, top=2.8cm, bottom=2.8cm]{geometry}
%\usepackage{extarrows}
%\usetikzlibrary{babel}
\usepackage{adjustbox}
%\usepackage{chemarrow}
%\let\rightarrow\chemarrow

\newcommand{\CC}{{\mathbb C}}
\newcommand{\RR}{{\mathbb R}}
\newcommand{\NN}{{\mathbb N}}
\newcommand{\ZZ}{{\mathbb Z}}
\newcommand{\QQ}{{\mathbb Q}}
\newcommand{\hhom}{{\text{hom}}}
\newcommand{\CH}{{\text{CH}}}
\newcommand{\Jac}{{\text{Jac}}}
\newcommand{\rank}{{\text{rank}}}
\newcommand{\inEnd}{{\underline{End}}}
\newcommand{\fu}{{\mathfrak{u}}}
\newcommand{\fg}{{\mathfrak{g}}}
\newcommand{\inHom}{{\underline{Hom}}}
\newcommand{\cupprod}{\mathbin{\smile}}
\newcommand\dual{\raise0.9ex\hbox{$\scriptscriptstyle\vee$}}
\newcommand{\sN}{{\mathcal{N}}}
\newcommand{\sL}{{\mathcal{L}}}
\newcommand{\sM}{{\mathcal{M}}}
\newcommand{\sE}{{\mathcal{E}}}
\newcommand{\sX}{{\mathcal{X}}}
\newcommand{\sY}{{\mathcal{Y}}}
\newcommand{\sZ}{{\mathcal{Z}}}
\newcommand{\bT}{{\mathbf{T}}}
\newcommand{\bS}{{\mathbf{S}}}
\newcommand{\oD}{{\overline{D}}}
\newcommand{\ffu}{\underline{\fu}}
\newcommand{\dbullet}{{\bullet,\bullet}}
\newcommand{\db}{{\bullet,\bullet}}
\renewcommand{\theequation}{\arabic{equation}}


\theoremstyle{plain}
\newtheorem{thm}{Theorem} 
\newtheorem{prop}[thm]{Proposition}
\newtheorem{lemma}[thm]{Lemma}  
\newtheorem{cor}[thm]{Corollary}
%\numberwithin{prop}{subsection}
%\numberwithin{lemma}{subsection}
\numberwithin{thm}{subsection}
%\numberwithin{cor}{subsection}

\newtheorem{manualtheoreminner}{Theorem}
\newenvironment{thm'}[1]{%
  \renewcommand\themanualtheoreminner{#1}%
  \manualtheoreminner
}{\endmanualtheoreminner}

\theoremstyle{definition}
\newtheorem{defn}[thm]{Definition}
\newtheorem{construction}[thm]{Construction}
\newtheorem{problem}[thm]{Problem}
\newtheorem{notation}[thm]{Notation}


\theoremstyle{remark}
\newtheorem{rem}[thm]{Remark}


%\newtheorem{manualtheoreminner}{Theorem}
%\newenvironment{manualtheorem}[1]{%
%  \renewcommand\themanualtheoreminner{#1}%
%  \manualtheoreminner
%}{\endmanualtheoreminner}


% Use a generous paragraph indent so numbers can be fit inside the
% indentation space.
\setlength{\parindent}{2em}

%\tikzset{node distance=2cm, auto} 

\tikzset{>=stealth}

\makeatletter
\def\@seccntformat#1{%
  \protect\textup{\protect\@secnumfont
    \ifnum\pdfstrcmp{subsection}{#1}=0 \bfseries\fi% subsection # in \bfseries
    \csname the#1\endcsname
    \protect\@secnumpunct
  }%
}  
\makeatother

\makeatletter
\@namedef{subjclassname@2020}{%
  $2020$ Mathematics Subject Classification}
\makeatother

%new command for inner products:
%\newcommand{\lang}{\begin{picture}(5,7)
%\put(1.1,2.5){\rotatebox{45}{\line(1,0){6.0}}}
%\put(1.1,2.5){\rotatebox{315}{\line(1,0){6.0}}}
%\end{picture}}
%\newcommand{\rang}{\begin{picture}(5,7)
%\put(.1,2.5){\rotatebox{135}{\line(1,0){6.0}}}
%\put(.1,2.5){\rotatebox{225}{\line(1,0){6.0}}}
%\end{picture}}

%%%%%%%%%%%%%%%%%%%%%%%%%

\begin{document}

\title[On blended extensions in filtered tannakian categories]{On blended extensions in filtered tannakian categories and mixed motives with maximal unipotent radicals}
\author{Payman Eskandari}
\address{Department of Mathematics and Statistics, University of Winnipeg, Winnipeg MB, Canada }
\email{p.eskandari@uwinnipeg.ca}
\subjclass[2020]{14F42 , 18M25 (primary); 11F67, 11M32, 14G10  (secondary).}
\begin{abstract}
Motivated by Grothendieck's period conjecture, our ultimate objective in this paper is to give a homological classification result for isomorphism classes of mixed motives $X$ such that $Gr^WX$ is isomorphic to a given semisimple motive and in addition, the unipotent radical of the motivic Galois group of $X$ is maximal, i.e., its Lie algebra is equal to $W_{-1}\inHom(X,X)$.

Let $(\bT, W_\bullet)$ be a filtered tannakian category over a field of characteristic zero, e.g., the category of rational mixed Hodge structures or any reasonable tannakian category of mixed motives over a field of characteristic zero. Let $A$ be a direct sum of pure objects of $\bT$. In the first part of the paper we consider two classification problems: (1) the classification of pairs $(X,\phi)$ of objects $X$ of $\bT$ and isomorphisms $\phi: Gr^WX\rightarrow A$ up to the natural equivalence of such pairs; (2) the classification of isomorphism classes of objects $X$ of $\bT$ such that $Gr^WX$ is isomorphic to $A$ (with no specific choice of isomorphism made). Building on the notion of a blended extension ({\it extension panach\'{e}e}) due to Grothendieck, we formally introduce the notion of a generalized extension of a given level, a certain type of diagram that allows one to study the structure of the classifying sets of the two aforementioned problems via a simple novel inductive approach that is especially suitable for the second classification problem in an important case where certain extensions are totally nonsplit.

We then go back to the original problem of interest. We assume further that the pure objects of $\bT$ are semisimple (as it is the case for any reasonable category of motives). Building on some ideas of \cite{EM2}, we define the notion of graded-independence and give a simple necessary and sufficient condition for a graded-independent object to have a maximal unipotent radical (i.e., its tannakian group has a maximal unipotent radical, in the sense described above). Combining this with the earlier results of the paper, we obtain a result on the structure of the set of isomorphism classes of objects with associated graded isomorphic to a given graded-independent semisimple object $A$ and maximal unipotent radicals. The special case of this result when $A$ has 3 weights and its graded components satisfy some extra conditions was proved in \cite{EM2}. Our work here generalizes the picture to an arbitrary number of weights as well as removes those extra conditions in the case of 3 weights.

Finally, as an application of the results of the paper we give a classification of mixed Tate motives over $\QQ$ with maximal unipotent radicals and associated graded isomorphic to $\QQ(a+b+c)\oplus \QQ(a+b)\oplus \QQ(a)\oplus \mathbbm{1}$, where $a,b,c$ are distinct positive integers with $a+b\neq c$. This leads to some interesting questions about periods of mixed Tate motives over $\QQ$.
\end{abstract}
\maketitle
\setcounter{tocdepth}{1}
\hypersetup{bookmarksdepth = 2}
\tableofcontents

\section{Introduction}
\subsection{About this paper}\label{sec: about this paper}
Let $\bT$ be a tannakian category over a field of characteristic zero, equipped with a filtration $W_\bullet$ (called the weight filtration) similar to the weight filtration on the category of rational mixed Hodge structures or any reasonable tannakian category of mixed motives over a field of characteristic zero. That is, $W_\bullet$ is indexed by $\ZZ$, functorial, exact, increasing, finite on every object, and compatible with the tensor structure. Consider a graded object 
\[
A = \bigoplus\limits_{r=1}^k A_r,
\]
where the $A_r$ are pure and in an increasing order of weights. One may consider the following two classification problems:
\begin{itemize}
\item[(1)] Classify the equivalence classes of all pairs $(X,\phi)$ of an object $X$ of $\bT$ whose associated graded
\[
Gr^WX = \bigoplus\limits_{n} W_nX/W_{n-1}X
\]
is isomorphic to $A$, and an isomorphism $\phi: Gr^W(X)\rightarrow A$. Here, the equivalence relation for such pairs is defined as follows: two pairs $(X,\phi)$ and $(X',\phi')$ are considered equivalent if there exists an isomorphism $f: X\rightarrow X'$ such that $\phi'\circ Gr^W(f) = \phi$.
\item[(2)] Classify, up to isomorphism in $\bT$, all objects $X$ whose associated graded $Gr^WX$ is isomorphic to $A$. Note that the data of a choice of isomorphism $Gr^W(X)\rightarrow A$ is not recorded here at all.
\end{itemize}
In general, the group 
\[ Aut(A)= \prod\limits_r Aut(A_r)\] 
acts on the collection of equivalence classes of pairs $(X,\phi)$ as in (1), and the orbits of this action will be in bijection with the collection of isomorphism classes of $X$ as in (2).

When $k=2$ and $A=A_1\oplus A_2$, the homological concept that classifies the pairs $(X,\phi)$ up to equivalence of such pairs is the Ext group $Ext^1(A_2,A_1)$. As for the isomorphism classes of $X$, the answer is given by the quotient of $Ext^1(A_2,A_1)$ by the group 
\[
Aut(A_1) \times Aut(A_2),
\]
where the actions of automorphisms of $A_1$ and $A_2$ on extension classes is by pushforward and pullback. 

When $k=3$, the homological concept related to problems (1) and (2) is the concept of a blended extension\footnote{The original French term for the concept is {\it extension panach\'{e}e}. The English translation to the term {\it blended extension}, which we found in \cite{Ber13}, is attributed by Bertrand to L. Breen.}, introduced by Grothendieck in SGA 7.I \cite[\S 9.3]{Gr68} to study 3-step filtrations. A nice detailed discussion of the relation between problem (1) in the case $k=3$ and blended extensions can be found in the appendix of the work \cite{RSZ} of Ramis, Sauloy and Zhang.

The first main objective of the present paper is to study problems (1) and (2) for arbitrary $k$. We introduce a new concept, which we call a {\it generalized extension}, that provides a natural homological framework for the study of problems (1) and (2). This concept naturally leads to an inductive approach (namely, induction on the {\it level}) towards the problems that seems more interesting and powerful than the more obvious approach of induction on the number of weights.

The other main objective of the paper, which was our original motivation for the work, is to consider the problem of classifying the isomorphism classes of (mixed) motives $X$ with associated graded isomorphic to a given semisimple motive $A$ and with maximal unipotent radicals of motivic Galois groups. By maximality of the unipotent radical we mean that the the Lie algebra of the unipotent radical is equal to $W_{-1}\inHom(X,X)$, where $\inHom$ is the internal Hom (that is, the unipotent radical is as large as it can be under the constraints imposed on it by the weight filtration). The interest in motives with maximal unipotent radicals is inspired by Grothendieck's period conjecture, which predicts that the transcendence degree of the field generated by the periods of a motive over $\overline{\QQ}$ should be equal to the dimension of the motivic Galois group of the motive (see Andr\'{e}'s letter to Bertolin published at the end of \cite{Be20} for more about this deep conjecture, including some very interesting remarks on the history of it). It would follow from this conjecture that among motives over $\overline{\QQ}$ with the same associated graded, the field generated by the periods of a motive with a maximal unipotent radical should have the largest transcendence degree. 

When $A$ satisfies a certain property (what we call {\it graded-independence}), we give a particularly nice homological answer to the classification problem of motives with associated graded isomorphic to $A$ and maximal unipotent radicals. The special case of this result for when $A=A_1\oplus A_2\oplus A_3$ with the $A_r$ pure and in an increasing order of weights, $A_3=\mathbbm{1}$ and $Ext^1(\mathbbm{1},A_1)=0$ was proved with K. Murty in \cite{EM2}.

As an example, in the final section of the paper we give a classification of graded-independent 4-dimensional mixed Tate motives over $\QQ$ with 4 weights and maximal unipotent radicals. This builds on the classification of the 3-dimensional case in \cite{EM2} and raises some interesting questions about periods.

\subsection{A more detailed overview of the paper and summary of the main results}\label{sec: intro detailed overview}
\subsubsection{Contents of \S \ref{sec: blended extensions II} and \S \ref{sec: objects in filtered tan cats with given gr}: classification of objects in a filtered tannakian category with a given associated graded}
Let $\bT$ be a filtered tannakian category over a field of characteristic zero, i.e. a neutral tannakian\footnote{We will freely use the language of of tannakian categories. The reader can refer to \cite{DM82} for the basic theory of tannakian categories.} category $\bT$ over a field of characteristic zero, equipped with a filtration $W_\bullet$ satisfying similar properties to the weight filtration on mixed Hodge structures, i.e. $W_\bullet$ is indexed by $\ZZ$, functorial, exact, increasing, finite on every object, and it respects the tensor structure. This means that for every integer $n$ we have an exact linear functor $W_n:\bT\rightarrow \bT$ such that for every object $X$ of $\bT$ we have
\begin{align*}
W_{n-1}X \ &\subset \ W_nX \hspace{.3in}(\forall n)\\
W_nX \ &= \ 0 \hspace{.3in}(\forall n\ll0)\\
W_nX \ &= \ X \hspace{.3in}(\forall n\gg 0),
\end{align*}
and such that the inclusions $W_nX\subset X$ for various $X$ give a morphism of functors from $W_n$ to the identity. Compatibility with the tensor structure means that for every objects $X$ and $Y$ and every $n$,
\[
W_n(X\otimes Y) = \sum\limits_{p+q=n}W_p(X)\otimes W_q(Y).
\]
We will refer to $W_\bullet$ as the weight filtration. Adopting the terminology of mixed Hodge structures (or mixed motives), an object $X$ will be called pure if there is an integer $n$ such that $W_{n-1}X=0$ and $W_nX=X$. If $X$ is nonzero and pure, there is a unique such an integer, which is called the weight of $X$.

Fix objects $A_1,\ldots, A_k$ that are pure and of increasing weights $p_1<\cdots<p_k$, respectively. Set $A$ to be the direct sum of the $A_r$ over $1\leq r\leq k$. We are interested in objects $X$ of $\bT$ such that $Gr^WX$ is isomorphic to $A$. We denote the set of isomorphism classes of such $X$ by $S(A)$, and the set of equivalence classes of pairs $(X,\phi)$ as described in problem (1) earlier by $S'(A)$. As a minor detail, we note that since $\bT$ is tannakian, $S(A)$ and $S'(A)$ are indeed sets (rather than proper classes).

We would like to study $S(A)$ and $S'(A)$, with the study of the former being our main goal. One sometimes prefers to work with the latter because it seems more suitable for moduli problems, as the work \cite{RSZ} of Ramis, Salouy and Zhang in the setting of difference modules illustrates. However, our results in this paper suggest that an important subset of $S(A)$ may also be suitable for moduli problems. Regardless, since our ultimate interest lies in periods of mixed motives, we find the classification of $S(A)$ more interesting. 

We should point out that the classification problem for $S'(A)$ has also been studied in the setting of real mixed Hodge structures by Ferrario in his PhD thesis \cite{Fe20}, leading to some interesting results about a complex analogue of Grothendieck's section conjecture.\footnote{I thank Peter Jossen for bringing this to my attention. Note that Ferrario calls a pair $(X,Gr^WX \xrightarrow{\phi, \simeq} A)$ an amalgam of $A$, see Definition 3.2.4 of \cite{Fe20}. Our $S'(A)$ is the same as $\text{Am}(A)$ in his notation (= the set of isomorphism classes of amalgams of $A$).}

Let us first discuss the case $k=3$ in more details. For any $X$ with 
\[Gr^WX\simeq A= A_1\oplus A_2\oplus A_3\]
where $A_r$ is pure of weight $p_r$ and $p_1<p_2<p_3$, let us set
\[
X_r:=W_{p_r}X
\]
so that the $X_r$ form a 3-step filtration. Given a pair $(X,Gr(X)\xrightarrow{\phi,\, \simeq}A )$, we obtain a blended extension 
\begin{equation}\label{eq58}
\begin{tikzcd}
   & & 0 \arrow{d} & 0 \arrow{d} &\\
   0 \arrow[r] & A_1 \ar[equal]{d} \arrow[r] & X_2  \arrow[d] \arrow[r] &  A_2 \arrow{d} \arrow[r] & 0 \\
   0 \arrow[r] & A_1 \arrow[r] & X \arrow[d] \arrow[r] &  X/X_1 \arrow{d}  \arrow[r] & 0 \\
   & & A_3 \arrow{d} \ar[equal]{r} & A_3 \arrow{d} & \\
   & & 0 & 0 &   
\end{tikzcd}
\end{equation}
with obvious arrows. It is easy to see that by sending the equivalence class of $(X,\phi)$ to the classes of the extensions on the top row and the right column we obtain a (well-defined) map
\begin{equation}\label{eq52}
S'(A) \rightarrow Ext^1(A_2,A_1)\times Ext^1(A_3,A_2).
\end{equation}
Given $\sN\in Ext^1(A_3,A_2)$ and $\sL\in Ext^1(A_2,A_1)$, after choosing representative extensions for $\sN$ and $\sL$ denote the set of equivalence classes of blended extensions of $\sN$ by $\sL$ with respect to the standard equivalence (morphisms between the middle objects that are identity on $\sN$ and $\sL$) by $Extpan(\sN,\sL)$. In view of the fact that
\begin{equation}\label{eq54}
Hom(A_2,A_1)\cong Hom(A_3,A_2) \cong Hom(A_3,A_1) \cong 0,
\end{equation}
one can see that the fiber of the map \eqref{eq52} above $(\sL, \sN)$ is in a canonical bijection with $Extpan(\sN,\sL)$, and hence, thanks to the general theory of blended extensions (see \S \ref{sec: background on blended extensions} for a brief review), each fiber is either empty or a torsor over $Ext^1(A_3,A_1)$. The bijection from the fiber above $(\sL,\sN)$ to $Extpan(\sN,\sL)$ sends the equivalence class of a pair $(X,\phi)$ above $(\sL,\sN)$ to the class of the blended extension \eqref{eq58}, with the top row and right column replaced by the chosen representatives of $\sL$ and $\sN$ via the unique isomorphisms. (Note that since $Hom(A_2,A_1)$ and $Hom(A_3,A_2)$ vanish, there is a unique isomorphism between any two extensions representing $\sL$ or $\sN$.)

Turning our attention to $S(A)$, it follows from \cite[Lemma 6.5.1]{EM2} that the map
\eqref{eq52} descends to a map
\begin{equation}\label{eq51}
S(A) \rightarrow \bigm(Ext^1(A_2,A_1)\times Ext^1(A_3,A_2)\bigm)/Aut(A),
\end{equation}
where an element 
\[(\sigma_1,\sigma_2, \sigma_3)\in Aut(A_1)\times Aut(A_2)\times Aut(A_3) = Aut(A)\]
acts on a pair of extension classes $(\sL, \sN)$ by pushing forward and pulling back $\sL$ along $\sigma_1$ and $\sigma_2^{-1}$ respectively, and pushing forward and pulling back $\sN$ by $\sigma_2$ and $\sigma_3^{-1}$ respectively. In view of the fact that each nonempty fiber of \eqref{eq52} is a torsor over $Ext^1(A_3,A_1)$, one can see that if the aforementioned Ext group is zero, then \eqref{eq51} is injective. We used this in \cite[\S 6.7 and \S 6.8]{EM2} to obtain a classification result for 3-dimensional mixed Tate motives over $\QQ$ with three weights and maximal unipotent radicals.

The first task of the present paper, carried out in \S \ref{sec: blended extensions II}, is to study fibers of \eqref{eq51}. We will see that the fiber above the $Aut(A)$-orbit of $(\sL,\sN)$ is in a bijection with the set
\[
Extpan(\sN,\sL)/\Gamma
\]
of orbits of an action of a certain group $\Gamma$ on $Extpan(\sN,\sL)$. 

With applications to motives with maximal unipotent radicals in mind, of particular interest is the case in which the extensions $\sN$ and $\sL$ are totally nonsplit (see Definition \ref{def: tot nonsplit exts} to recall what this means). This is discussed in \S \ref{sec: tot nonsplit case with 3 weights}. We see that in this case the action of the group $\Gamma$ will be trivial and so in passing from $S'(A)$ to $S(A)$ and from \eqref{eq52} to \eqref{eq51} the fibers do not change.

Motivated by the picture for objects with 3-step filtrations, after this we embark on the route towards our first main objective, which is to have an analogous picture for objects with longer filtrations. This is the content of a large portion of the paper, covered in \S \ref{sec: objects in filtered tan cats with given gr}. We shall prove the following results about $S'(A)$ and $S(A)$ for arbitrary $k$. The missing details of the statements are filled as we go through the constructions of \S \ref{sec: objects in filtered tan cats with given gr}.
\begin{thm}\label{thm: classification of S'(A) in the general case}
There exist sets $S'_\ell(A)$ for $1\leq \ell\leq k-1$ and maps
\[
S'_{k-1}(A)\rightarrow S'_{k-2}(A) \rightarrow S'_{k-3}(A) \rightarrow \cdots \rightarrow S'_2(A) \rightarrow S'_{1}(A)
\]
such that 
\[
S'_{k-1}(A)\cong S'(A) \hspace{.2in}\text{and}\hspace{.2in} S'_1(A)\cong \prod\limits_{r} Ext^1(A_{r+1},A_{r})
\]
and for each $\ell$, every nonempty fiber of the map $S'_{\ell}(A)\rightarrow S'_{\ell-1}(A)$ is canonically a torsor for
\[
\prod\limits_{r}Ext^1(A_{r+\ell},A_r).
\]
Moreover, for each $\ell$, if the $Ext^2$ groups 
\begin{equation}\label{eq66}
Ext^2(A_{r+\ell},A_r) \hspace{.2in} (1\leq r\leq k-\ell)
\end{equation}
vanish, then the map $S'_{\ell}(A)\rightarrow S'_{\ell-1}(A)$ is surjective.
\end{thm}

\begin{thm}\label{thm: classification of objects with prescribed associated graded, general case}
(a) There exist sets $S_\ell(A)$ for $1\leq \ell\leq k-1$ and maps
\begin{equation}\label{eq55}
S_{k-1}(A)\rightarrow S_{k-2}(A) \rightarrow S_{k-3}(A) \rightarrow \cdots \rightarrow S_2(A)\rightarrow S_{1}(A)
\end{equation}
such that 
\[
S_{k-1}(A)\cong S(A) \hspace{.2in}\text{and}\hspace{.2in} S_1(A)\cong \bigm( \prod\limits_{r} Ext^1(A_{r+1},A_{r})\bigm)\bigm/Aut(A)
\]
and for each $\ell$, every nonempty fiber of the map $S_{\ell}(A)\rightarrow S_{\ell-1}(A)$ is in a bijection with the quotient of a certain 
\[
\prod\limits_{r}Ext^1(A_{r+\ell},A_r) 
\]
torsor by an action of a certain group $\Gamma$ that depends on the fiber (and another choice made in the process; see the comments after the statement). Moreover, if the $Ext^2$ groups \eqref{eq66} vanish, then the map $S_{\ell}(A)\rightarrow S_{\ell-1}(A)$ is surjective.

\noindent (b) On set of weakly totally nonsplit elements of $S_{\ell-1}(A)$ the action of the corresponding group $\Gamma$ is trivial, so that a nonempty fiber of the map $S_{\ell}\rightarrow S_{\ell-1}$ above a weakly totally nonsplit element of $S_{\ell-1}(A)$ is a torsor for $\prod_{r}Ext^1(A_{r+\ell},A_r)$. 
\end{thm}

The precise definitions of the sets $S_\ell(A)$ and $S'_\ell(A)$ as well as the maps from ``level" (referring to the index) $\ell$ to $\ell-1$ are given in \S \ref{sec: sets S_ell S'_ell and restatements of Thms A and B}. They involve the notion of {\it generalized extentions}, the machinery of which is developed in \S \ref{sec: gen exts, defn} - \S \ref{sec: equiv rels on gen exts}. We will however give indications on these shortly, after we make a few further clarifying comments on the statements and provide a map of the organization of their proofs in the paper. 

The action of $Aut(A)$ on $\prod\limits_{r} Ext^1(A_{r+1},A_{r})$ in Theorem \ref{thm: classification of objects with prescribed associated graded, general case} is (similar to the case $k=3$) by pushforwards and pullbacks, with automorphisms of $A_r$ for each $1\leq r\leq k$ acting on the (one or two) extension(s) that involves $A_r$. See \ref{sec: equiv rels on gen exts} for more details. The structure of the fibers (more specifically, the torsor structures) for Theorem \ref{thm: classification of S'(A) in the general case} is discussed in \S \ref{sec: fibers 1} and \S \ref{sec: fibers of truncations II}, and the structure of the fibers in the general case of Theorem \ref{thm: classification of objects with prescribed associated graded, general case}, i.e. the torsor structures as well as the $\Gamma$-action, is discussed in \S \ref{sec: fibers 3}. The concept of {\it weakly\footnote{In the later parts of the paper we will also introduce a stronger notion of total nonsplitting.} total nonsplitting} is introduced in \S \ref{sec: thm B, tot nonsplit case}. The weakly totally nonsplit elements of
\[
S_1(A)= \bigm( \prod\limits_{r} Ext^1(A_{r+1},A_{r})\bigm)\bigm/Aut(A)
\]
are simply orbits of tuples of extension classes $(\sE_r)$ in which all the $\sE_r$ are totally nonsplit.

We should say a few more words about the torsor structures on the fibers in Theorem \ref{thm: classification of objects with prescribed associated graded, general case}. The torsor of part (a) of the theorem for each fiber involves making a choice; this same choice also determines the group $\Gamma$ and then the bijection between the fiber and the quotient of the torsor. The different torsors obtained from different choices are however closely related to one another. For the fibers above weakly totally nonsplit elements, the torsor structure on each fiber is ``essentially" canonical, in that if we make two different choices in the process, the torsor structure for one choice is a canonical twist of the torsor structure for the other choice (see Proposition \ref{prop: general thm on S(A) part b}(b)). All of this will become more clear as we go through \S \ref{sec: objects in filtered tan cats with given gr}. 

Let us give a map of how the proofs of Theorems \ref{thm: classification of S'(A) in the general case} and \ref{thm: classification of objects with prescribed associated graded, general case} are organized in the paper. As mentioned earlier, the definition of the sets $S_\ell(A)$ and $S'_\ell(A)$ and the maps from level $\ell$ to $\ell-1$ is given in \S \ref{sec: sets S_ell S'_ell and restatements of Thms A and B}. The identifications of $S_\ell(A)$ and $S'_\ell(A)$ with the desired sets when $\ell\in \{1,k-1\}$ is established in Lemmas \ref{lem: D_2} and \ref{lem: D_k}. The remaining assertions are proved in \S \ref{sec: fibers 1}-\ref{sec: thm B, tot nonsplit case}. In particular, the proof of Theorem \ref{thm: classification of S'(A) in the general case} is completed in \S \ref{sec: fibers of truncations II}. A more precise and explicit version of the assertion in Theorem \ref{thm: classification of objects with prescribed associated graded, general case}(a) about the structure of the fibers is given and deduced in \S \ref{sec: fibers 3} (see Proposition \ref{prop: general thm on S(A) part a}); this version also includes a precise statement for how the torsor structures corresponding to different choices are related. The proof of Theorem \ref{thm: classification of objects with prescribed associated graded, general case}(b) is given in \S \ref{sec: thm B, tot nonsplit case}, where the notion of a weakly totally nonsplit element of $S_\ell(A)$ is also defined and a more precise version of Theorem \ref{thm: classification of objects with prescribed associated graded, general case}(b) is stated as Proposition \ref{prop: general thm on S(A) part b}. This statement also includes a naturalness property for the torsor structure in the weakly totally nonsplit case.

We now give some indications on the proofs of these results, and in particular on the construction of the sets $S'_\ell(A)$ and $S_\ell(A)$. Let $X$ be an object of $\bT$ whose associated graded is isomorphic to $A$. Fix an isomorphism $Gr^WX\rightarrow A$ to identify the two. For any integers $m,n$ with $0\leq m<n\leq k$, set
\[
X_{m,n}:= W_{p_n}X/W_{p_m}X,
\] 
where we have set $p_0=p_1-1$ (so that $W_{p_0}X=0$). It is convenient to introduce the following notation: for $m,n$ with $0\leq m< n\leq k$ and $n-m\geq 2$, let $\sX^h_{m,n}$ and $\sX^v_{m,n}$ be the following two extensions with $X_{m,n}$ in the middle:
\[
\sX^h_{m,n}: \hspace{.3in}
\begin{tikzcd}
   0 \arrow[r] &  A_{m+1}\arrow[r, ] & X_{m,n} \arrow[r, ] &  X_{m+1,n}  \arrow[r] & 0
\end{tikzcd}
\]
and
\[
\sX^v_{m,n}: \hspace{.3in}
\begin{tikzcd}
  0 \arrow[r] &  X_{m, n-1}\arrow[r, ] & X_{m,n} \arrow[r, ] &  A_n  \arrow[r] & 0
\end{tikzcd}
\]
where we have used our fixed isomorphism $Gr^WX\rightarrow A$ to identify each $X_{r-1,r}$ with $A_r$. Here, the superscripts $h$ and $v$ stand for horizontal and vertical, respectively; the reason for the choice of notation is that these will be considered respectively as horizontal and vertical extensions in diagrams of blended extensions.  

The $X_{m,n}$ fit into the commutative diagram
\begin{equation}\label{eq gen ext for M}
\begin{tikzcd}[column sep=small, row sep=small]
A_1         \arrow[d, hookrightarrow] & & & &&& \\
X_{0,2} \arrow[r, twoheadrightarrow] \arrow[d, hookrightarrow] & A_2 \arrow[d, hookrightarrow]     & & & &&\\
X_{0,3} \arrow[r, twoheadrightarrow]  \arrow[d, hookrightarrow] & X_{1,3} \arrow[r, twoheadrightarrow] \arrow[d, hookrightarrow]   & A_3 \arrow[d, hookrightarrow] & & &&\\
X_{0,4} \arrow[r, twoheadrightarrow] \arrow[d, hookrightarrow]  & X_{1,4} \arrow[r, twoheadrightarrow]  \arrow[d, hookrightarrow]    & X_{2,4} \arrow[r, twoheadrightarrow]  \arrow[d, hookrightarrow] & A_4 \arrow[d, hookrightarrow] & &&\\
\vdots& \vdots & \vdots & \vdots &\ddots &&\\
&&&&&&\\
X_{0,k-1} \arrow[d, hookrightarrow] \arrow[r, twoheadrightarrow] & X_{1,k-1} \arrow[d, hookrightarrow] \arrow[r, twoheadrightarrow] &\cdots & ~ & X_{k-3,k-1} \arrow[l, twoheadleftarrow] \arrow[d, hookrightarrow] \arrow[r, twoheadrightarrow] &A_{k-1} \arrow[d,hookrightarrow] &\\
X_{0,k}  \arrow[r, twoheadrightarrow] & X_{1,k} \arrow[r, twoheadrightarrow] &\cdots & ~ & X_{k-3,k} \arrow[l, twoheadleftarrow] \arrow[r, twoheadrightarrow] & X_{k-2,k}  \arrow[r, twoheadrightarrow] &A_k. 
\end{tikzcd}
\end{equation}
Every $X_{m,n}$ appears in the diagram exactly once. Each horizontal arrow is surjective and is given by modding out by the first step in the weight filtration on the domain. The vertical arrows are all injective and are the inclusions $X_{m,n-1}\hookrightarrow X_{m,n}$ given by the weight filtration. 

Roughly speaking, our goal is to obtain all $X$ or $(X, Gr^WX\rightarrow A)$ up to the appropriate equivalence relation. Our approach is to do this step by step as follows: First consider possibilities for $(X_{r-1,r+1})_r$ (i.e. the first diagonal below the $A_r$); the object $X_{r-1,r+1}$ is an extension of $A_{r+1}$ by $A_{r}$. So we must look at 
\[ \prod\limits_r EXT(A_{r+1},A_{r})\] 
up to some equivalence. Here, $EXT(A_{r+1},A_{r})$ means the collection of extensions of $A_{r+1}$ by $A_{r}$ before modding out by any equivalence relation. Fixing $(X_{r-1,r+1})_r$, we now consider the possibilities for $(X_{r-1,r+2})_r$ (i.e. the second diagonal below the $A_r$. The object $X_{r-1,r+2}$ will be the middle object of the blended extension
\[
\begin{tikzcd}
   & & 0 \arrow{d} & 0 \arrow{d} &\\
   0 \arrow[r] & A_{r} \ar[equal]{d} \arrow[r, ] & X_{r-1,r+1}  \arrow[d] \arrow[r, ] &  \displaystyle{\frac{X_{r-1,r+1}}{A_{r}}}=X_{r,r+1} \arrow{d} \arrow[r] & 0 \\
   0 \arrow[r] & A_{r} \arrow[r] & X_{r-1,r+2} \arrow[d] \arrow[r] &  \displaystyle{\frac{X_{r-1,r+2}}{A_{r}}}= X_{r,r+2}\arrow{d}  \arrow[r] & 0 \\
   & & A_{r+2} \arrow{d} \ar[equal]{r} & A_{r+2} \arrow{d} & \\
   & & 0 & 0 &   
\end{tikzcd}
\]
of $\sX^v_{r,r+2}$ by $\sX^h_{r-1,r+1}$. We must thus look at
\[
\prod_r EXTPAN(\sX^v_{r,r+2}, \sX^h_{r-1,r+1})
\]
up to some equivalence, where following \cite{Gr68} the notation $EXTPAN(\sX^v_{r,r+2}, \sX^h_{r-1,r+1})$ means the collection of all blended extensions of $\sX^v_{r,r+2}$ by $\sX^h_{r-1,r+1}$ (before taking any equivalence relations into account). We continue in the same fashion until we get to the possibilities for $X_{0,k}=X$. Of course, one also has to keep track of the appropriate equivalence relations in each step.

To make this approach precise, we introduce the notion of a {\it generalized extension of level $\ell$} in \S \ref{sec: gen exts, defn} (see Definitions \ref{defn: generalized extensions} and \ref{def: gen ext level l}) and study its basic properties in \S \ref{sec: basic properties of gen exts}. A generalized extension of level $\ell$ of $A$ (with $k$ weights) is the abstract data of a diagram as in \eqref{eq gen ext for M}, but only with the first $\ell$ diagonals below the $A_r$ included. Thus a generalized extension of level $k-1$ is an abstract version of the full diagram \eqref{eq gen ext for M}. When $k=2$, a generalized extension of level 1 of $A$ is simply an extension of $A_2$ by $A_1$. When $k=3$, a generalized extension of level 2 of $A$ is simply the data of a blended extension as in \eqref{eq53}, with varying $L$ and $N$, but $A_1, A_2, A_3$ fixed. For any $k$, the data of a generalized extension of level 1 of $A$ consists of an extension of $A_{\ell+1}$ by $A_\ell$ for each $1\leq \ell\leq k-1$. We highlight that our notion of a generalized extension really becomes interesting when the level is less than $k-1$ (as in level $k-1$, everything is determined by the bottom right object).

The sets $S'_\ell(A)$ and $S_\ell(A)$  in Theorems \ref{thm: classification of S'(A) in the general case} and \ref{thm: classification of objects with prescribed associated graded, general case} are the quotients of the collection of all generalized extensions of level $\ell$ of $A$ by suitable equivalence relations. A pair $(X,\phi)$ gives rise to a generalized extension of level $k-1$, inducing the identification $S'(A)\cong S_{k-1}(A)$. The maps $S'_\ell(A)\rightarrow S'_{\ell-1}(A)$ and $S_\ell(A)\rightarrow S_{\ell-1}(A)$ are simply induced by truncation (see \S \ref{sec: equiv rels on gen exts} and \S \ref{sec: sets S_ell S'_ell and restatements of Thms A and B}).

Of course, there is a somewhat more straightforward inductive approach towards the study of $S'(A)$ and $S(A)$, namely to induct on $k$ ( = the number of weights or graded components of $A$). Setting
\[
A_{\leq \ell} := \bigoplus\limits_{r\leq \ell} A_r,
\]
the weight filtration gives rise to maps
\[
S'(A)=S'(A_{\leq k}) \rightarrow S'(A_{\leq k-1}) \rightarrow \cdots \rightarrow S'(A_{\leq 3}) \rightarrow S'(A_{\leq 2})
\]
and 
\[
S(A)=S(A_{\leq k}) \rightarrow S(A_{\leq k-1}) \rightarrow \cdots \rightarrow S(A_{\leq 3}) \rightarrow S(A_{\leq 2}).
\]
The fiber of $S'(A_{\leq \ell})\rightarrow S'(A_{\leq \ell-1})$ above the equivalence class of $(X,\phi)$ is in a canonical bijection with $Ext^1(A_\ell, X)$ (this is written in detail in \cite{Fe20}, see Proposition 3.2.9 and Remark 3.2.10 therein). This inductive approach towards studying $S'(A)$ was successfully pursued by Ramis, Sauloy and Zhang in \cite{RSZ} (see in particular, see \S A.1.2 therein) in the setting of difference modules 
and more recently, in the PhD thesis \cite{Fe20} of Ferrario in the setting of real mixed Hodge structures. In the former work, the approach leads to a very nice solution to the classification problem for $S'(A)$ in the particular setting of difference modules over difference rings: the authors show that in that particular setting, $S'(A)$ is actually a scheme (see \cite{RSZ} for more details). In \cite{Fe20}, Ferrario uses the inductive approach on the number of weights to prove a result about injectivity of a Kummer map for the projective line minus at least 3 points in the context of a complex analogue of the section conjecture.

The inductive approach on the level taken in this paper has better naturalness properties. For instance, every nonempty fiber of $S'_\ell(A)\rightarrow S'_{\ell-1}(A)$ is canonically a torsor for the same group $\prod_r Ext^1(A_{r+\ell}, A_r)$ (compare with the structure of the fibers of $S'(A_{\leq \ell})\rightarrow S'(A_{\leq \ell-1})$). 

The inductive approach on the level seems especially useful for studying $S(A)$. Indeed, the fiber of $S(A_{\leq \ell})\rightarrow S(A_{\leq \ell-1})$ above the isomorphism class of $X$ is in bijection with
\[
Ext^1(A_\ell, X) \bigm/ Aut(X)\times Aut(A_\ell),
\]
where the actions of $Aut(X)$ and $Aut(A_\ell)$ are by pushforward and pullback of extensions. Unless $Ext^1(A_\ell, X)$ is trivial, the action of $Aut(X)\times Aut(A_\ell)$ on $Ext^1(A_\ell, X)$ is never trivial. Comparing this situation with part (b) of Theorem \ref{thm: classification of objects with prescribed associated graded, general case}, it seems that the approach of induction on the level is more likely to lead to moduli results for $S(A)$ (at least in some important cases). In particular, the application to motives in \S \ref{sec: mixed motives with maximal unipotent radicals} (reviewed shortly below) hints that the approach of induction on the level might lead to some moduli results for isomorphism classes of motives with maximal unipotent radicals and a fixed associated graded.

We end our review of \S \ref{sec: blended extensions II} and \S \ref{sec: objects in filtered tan cats with given gr} by noting that while we have written the article working in the setting of filtered tannakian categories, under a standard hypothesis, one should be able to adopt (with some adjustment) much of the constructions and results above to the study of $k$-step filtrations in tannakian categories (in fact, aside from the results about the totally nonsplit case, to the study of such filtrations in abelian categories), even if the filtration does not come from a functorial filtration. The said standard hypothesis that one needs is that $Hom(A_j,A_i)=0$ whenever $i<j$. This hypothesis also appeared in \cite{Ber13} and \cite{RSZ} (see Theorem A.6.1 therein).

\subsubsection{Contents of \S \ref{sec: mixed motives with maximal unipotent radicals}: Application to mixed motives with maximal unipotent radicals}\label{sec: intro details of application to motives with maximal u}
We now assume that $\bT$ is a filtered tannakian category over a field $K$ of characteristic 0 such that the graded objects of $\bT$ (i.e. objects of the form $Gr^WX$) are semisimple. The prototype examples are the category of graded-polarizable mixed Hodge structures over $\QQ$ and any reasonable tannakian category of mixed motives over subfields of $\CC$, e.g. those of Ayoub \cite{Ay14} and Nori \cite{HM17}, or those of Deligne \cite{De89} and Jannsen \cite{Ja90} defined earlier using realizations, or Voevodsky's category of mixed Tate motives over a number field. In fact, inspired by these prototype examples, we refer to the objects of $\bT$ as motives, even though aside from \S \ref{sec: examples} the discussion is valid in the generality of $\bT$ described above. Note that in the prototype examples of the category of rational mixed Hodge structure and the category of mixed motives over a number field, it is either known or expected that the $Ext^2$ groups all vanish\footnote{This is known for the categories of mixed Hodge structures \cite{Bei83} and mixed Tate motives over a number field \cite{DG05}. It is expected for the category of mixed motives over a number field. See for instance, \cite{Ne94} or \cite{Ja94}.}, so that the truncation maps in Theorems \ref{thm: classification of S'(A) in the general case} and \ref{thm: classification of objects with prescribed associated graded, general case} are or should be all surjective.

Let $X$ be a motive. One has a canonical subobject $\ffu(X)$ of $W_{-1}\inEnd(X)$ (where $\inEnd(X)$ is the internal Hom $\inHom(X,X)$) associated with the Lie algebra of the unipotent radical of the tannakian group of $X$ (see \S \ref{sec: setting for section on application to motives} for a brief review). The object $\ffu(X)$ has been studied in various contexts by many, including Deligne (\cite{De89} and \cite[Appendix]{Jo14}), Andr\'{e} \cite{An92}, Bertrand \cite{Ber01}, Bertolin (\cite{Be02} and \cite{Be03}), Hardouin (\cite{Har06} and \cite{Har11}), Jossen \cite{Jo14}, and the author and Murty (\cite{EM1} and \cite{EM2}).

Let us say that $\ffu(X)$ is maximal\footnote{In \cite{EM2} instead of ``maximal" we used the word ``large" for this.}, or that $X$ has a maximal unipotent radical, if $\ffu(X)=W_{-1}\inEnd(X)$. It is easy to see that if $X$ has a maximal unipotent radical, then so does each of its subquotients. In \cite[Theorem 6.3.1]{EM2}, with Murty we gave sufficient conditions, which we called {\it independence axioms}, under which (at least in some situations) if $W_pX$ and $X/W_{p-1}X$ have maximal unipotent radicals, then so does $X$.\footnote{The statement becomes false if we remove the independence axioms from the hypotheses; see \S 6.3 of \cite{EM2}.} This allowed us in \S 6.4-6.7 of the same article to give a homological classification of motives $X$ with maximal unipotent radicals and an associated graded isomorphic to 
\[
A_1\oplus A_2 \oplus \mathbbm{1},
\]
where $A_1$ and $A_2$ are pure of negative weights $p_1<p_2$, $Ext^1(\mathbbm{1},A_1)=0$, and $A_1\oplus A_2 \oplus \mathbbm{1}$ satisfies the following independence axiom: $A_2$ and $\inHom(A_2,A_1)$ have no nonzero isomorphic subobjects.

Broadly speaking, an independence axiom in this context is a property that guarantees that $Gr^W\ffu(X)$ decomposes according to a suitable decomposition of 
\[Gr^WW_{-1}\inEnd(X) = \bigoplus\limits_{i<j}\inHom(Gr^W_jX, Gr^W_iX)\] 
given by a partition of the set $\{(i,j)\in \ZZ^2 : i<j\}$. In \cite{EM2}, our independence axioms simplified the relationship between the extensions of the form
\begin{equation}\label{eq56}
\begin{tikzcd}
0 \arrow[r] & W_pX \arrow[r] & X \arrow[r] & X/W_pX \arrow[r] & 0
\end{tikzcd}
\end{equation}
and $\ffu(X)$, thereby refining a result of Deligne \cite[Proposition A.3]{Jo14} regarding this relationship. This refinement, stated as Corollaries 5.3.2 and 5.3.3 of \cite{EM2}, was one of the main ingredients of the maximality criterion of \cite[Theorem 6.3.1]{EM2} and hence the classification result mentioned earlier. 

The aim of \S \ref{sec: mixed motives with maximal unipotent radicals} of the present article is to generalize the classification result of \cite[\S 6]{EM2} regarding motives with maximal unipotent radicals from the case of 3 weights to an arbitrary number of weights. We also no longer need to assume that the graded piece with the highest weight is $\mathbbm{1}$, or assume anything about the Ext groups between the different graded pieces (compare with the hypotheses of the classification result in the case of 3 weights in \cite{EM2}). After reviewing some background and basic observations, in \S \ref{sec: maximality criterion} we define the notion of graded-independence (Definition \ref{defn: graded independence}), which is an independence axiom in the above sense; in fact, in the case where the associated graded is $A_1\oplus A_2\oplus \mathbbm{1}$ it becomes exactly the independence axiom mentioned earlier. We then give a simple criterion for maximality of the unipotent radical of a graded-independent motive (Theorem \ref{thm: maximality criteria}): it turns out that a graded-independent motive $X$ with weights $p_1<\cdots<p_k$ has a maximal unipotent radical if and only if each of the extensions
\[
\begin{tikzcd}
0 \arrow[r] & Gr^W_{p_{j-1}}X \arrow[r] & W_{p_j}X/W_{p_{j-2}}X \arrow[r] & Gr^W_{p_{j}}X \arrow[r] & 0
\end{tikzcd}
\]
is totally nonsplit. For context, note that without the graded-independence hypothesis, even total nonsplitting of all of the extensions 
\[
0 \rightarrow W_mX/W_\ell X \rightarrow W_nX/W_\ell X \rightarrow W_nX/W_m X \rightarrow 0 \hspace{.4in}(\ell<m<n)
\]
is not enough to guarantee maximality of $\ffu(X)$ (see Lemma \ref{lem: maximality and total nonsplitting}(c)). 

Assuming as before that $A=\bigoplus_{1\leq r\leq k} A_r$ with the $A_r$ pure and in an increasing order of weights, let $S^\ast(A)$ be the subset of $S(A)$ consisting of the isomorphism classes of motives with maximal unipotent radicals. Let $S^\ast_1(A)$ be the subset of 
\[ S_1(A) \cong \bigm( \prod\limits_{r} Ext^1(A_{r+1},A_{r})\bigm)\bigm/Aut(A) \]
(see Theorem \ref{thm: classification of objects with prescribed associated graded, general case}) consisting of the $Aut(A)$-orbits of tuples in which every entry is totally nonsplit. Combining Theorem \ref{thm: classification of objects with prescribed associated graded, general case} with the maximality criterion of Theorem \ref{thm: maximality criteria} we obtain that when $A$ is graded-independent, then there exist naturally defined sets $S^\ast_\ell(A)$ for $2\leq \ell\leq k-1$ and maps
\[
S^\ast(A) \cong S^\ast_{k-1}(A) \rightarrow S^\ast_{k-2}(A)  \rightarrow S^\ast_{k-3}(A) \rightarrow  \cdots  \rightarrow  S^\ast_2(A)  \rightarrow S^\ast_{1}(A)
\]
such that every nonempty fiber of $S^\ast_\ell(A)\rightarrow S^\ast_{\ell-1}(A)$ is a torsor over 
\[
\prod\limits_r Ext^1(A_{r+\ell}, A_r).
\]
Again, if the $Ext^2$ groups \eqref{eq66} vanish in $\bT$, then the map $S^\ast_\ell(A)\rightarrow S^\ast_{\ell-1}(A)$ above is surjective. This is recorded as Theorem \ref{thm: classification of motives with give gr in graded independent case} in the text. Its special case when $k=3$, $A_3=\mathbbm{1}$ and $Ext^1(\mathbbm{1}, A_1)=0$ was proved in \S 6 (see in particular, \S 6.7) of \cite{EM2}.

In \cite[\S 6.8]{EM2} with Murty we gave a classification of isomorphism classes of 3-dimensional graded-independent mixed Tate motives over $\QQ$ with 3 weights and maximal unipotent radicals. This classification led to some questions about periods. In \S \ref{sec: examples} of the present paper, as an example for Theorem \ref{thm: classification of motives with give gr in graded independent case} we consider the analogous problem for graded-independent 4-dimensional mixed Tate motives over $\QQ$. More explicitly, we give a classification up to isomorphism of all mixed Tate motives over $\QQ$ with maximal unipotent radicals and associated graded isomorphic to
\[
\QQ(a+b+c)\oplus \QQ(a+b) \oplus \QQ(a) \oplus \mathbbm{1},
\]
where $a,b,c$ are distinct positive integers and $c\neq a+b$ (these being the graded-independence conditions in this situation). This leads to more questions about periods.

The questions about the periods become particularly interesting when $1\in \{a,b,c\}$, in which case our motives will have a Kummer motive as a subquotient. Let $r$ be a squarefree integer $>1$ and $L_r$ the Kummer motive for $\log(r)$, i.e. an extension of $\mathbbm{1}$ by $\QQ(1)$ with $(2\pi i)^{-1}\log(r)$ as a period (geometrically, $L_r$ can be thought of as the relative homology $H_1(\mathbb{G}_m, \{1,r\})$). For each odd integer $n\geq 3$ let $Z_n$ be the motive that sits as the middle object of nonsplit extensions of $\mathbbm{1}$ by $\QQ(n)$ in the category of mixed Tate motives over $\QQ$. Then $Z_n$ is unique up to isomorphism and has $(2\pi i)^{-n}\zeta(n)$ as a period. In view of the known description of Ext groups in the category of mixed Tate motives over $\QQ$ (thanks to Voevodsky and Borel), it follows from the results of this paper that given any two distinct odd integers $b,c\geq 3$, the set of isomorphism classes of mixed Tate motives $X$ over $\QQ$ with 
\begin{equation}\label{eq57}
Gr^WX \simeq \QQ(1+b+c)\oplus \QQ(1+b) \oplus \QQ(1) \oplus \mathbbm{1}
\end{equation}
\[
X/W_{-3}X \simeq L_r, \ \ \ \ \ W_{-1}X/W_{-2-2b-1}X\simeq Z_b(1), \ \ \ \ \ W_{-2-2b}X\simeq Z_c(1+b)
\]
is a torsor over
\[
Ext^1(\mathbbm{1}, \QQ(1+b+c)) \simeq \QQ. 
\]
Moreover, it follows that these motives all have maximal unipotent radicals. Any such $X$ is unramified outside $r$, i.e. belongs to the category of mixed Tate motives over $\ZZ[1/r]$, and with respect to suitable bases of Betti and de Rham realizations has a period matrix of the form
\[
\begin{pmatrix}
(2\pi i)^{-1-b-c} & (2\pi i)^{-1-b-c} \zeta(c) & (2\pi i)^{-1-b-c} z_{c,b} & (2\pi i)^{-1-b-c}\lambda_{c,b,r}(X)  \\ 
& (2\pi i)^{-1-b} & (2\pi i)^{-1-b} \zeta(b) & (2\pi i)^{-1-b} p'_{b,r}\\
&& (2\pi i)^{-1} & (2\pi i)^{-1} \log(r)\\
&&&1
\end{pmatrix},
\]
where the (1,3) and (2,4) entries can be made to be fixed (i.e. independent of $X$). Since the motivic Galois group of $X$ has dimension 7, Grothendieck's period conjecture predicts that the numbers $\pi$, $\zeta(c)$, $\zeta(b)$, $\log(r)$, $z_{c,b}$, $p'_{b,r}$, and $\lambda_{c,b,r}(X)$ (for a fixed $X$) are algebraically independent.

Since $W_{-2}X$ is a mixed Tate motive over $\ZZ$, by Brown's work \cite{Br12} $z_{c,b}$ is a multiple zeta value. Unless $r\in \{2,3,6\}$, we do not know the natures of $p'_{b,r}$ ( = the ``new period" of the motive $M_{1+b,r}$ in the notation of \cite[\S 6.8]{EM2}) and $\lambda_{c,b,r}(X)$. For $r\in \{2,3,6\}$, one has Deligne's work \cite{De10} for the full subcategory of the category of mixed Tate motives over $\QQ(\mu_r)$ consisting of the motives unramified outside $r$. It follows that for these values of $r$, the unknown periods are generated by the periods of the fundamental group of $\mathbb{G}_m-\mu_r$ (i.e. cyclotomic multiple zeta values, see \cite{DG05}). For more general $r$, as far as the author knows, little is understood about the periods of the category of mixed Tate motives over $\ZZ[1/r]$. We refer the reader to \cite[\S 3]{DW16} and the references therein (in particular, \cite{Go95}) for a nice discussion of this and some conjectures.

The period $\lambda_{c,b,r}(X)$ offers an extra layer of mystery, thanks to its dependence on $X$. Since
\[
Ext^1(\mathbbm{1}, \QQ(1+b+c))
\]
is generated by the motive of $\zeta(1+b+c)$, it will be very interesting to understand the relation between $\zeta(1+b+c)$ and $\lambda_{c,b,r}(X)$ as $X$ varies in a set of representatives of the isomorphism classes. 

We end this overview with a mention of another result proved in the paper, which is in somewhat of a different direction than the rest of the paper but adds to the discussion of periods. Assuming $A=\bigoplus_{1\leq r\leq k} A_r$ (the notation as in earlier, with the $A_r$ arbitrary pure motives, not just Tate motives) is graded-independent, we show in Theorem \ref{thm: Ext groups for non-consecutive graded pieces vanish in the graded-independent situation} that for any motive $X$ for which the unipotent radical is maximal and $Gr^WX$ is isomorphic to $A$, the $Ext^1$ groups
\[
Ext^1_{\langle X\rangle}(A_j, A_i) \hspace{.3in} (j-i\geq 2)
\]
for the tannakian subcategory $\langle X\rangle$ generated by $X$ all vanish. 

Going back to the example of the isomorphism classes of mixed Tate motives $X$ over $\QQ$ satisfying \eqref{eq57}, we thus see that in fact, for any such motive $X$ one has
\[
Ext^1_{\langle X\rangle}(\mathbbm{1}, \QQ(1+b+c)) = 0.
\]
From this one easily sees that assuming Grothendieck's period conjecture, $\zeta(1+b+c)$ is algebraically independent from the 7 numbers  $\pi$, $\zeta(c)$, \dots, $\lambda_{c,b,r}(X)$ listed above. (See \S \ref{sec: examples} for more details.)

\subsection{Conventions}
Here we make some comments about the conventions and notations of the paper. A tannakian category will always be assumed to be a neutral one, i.e. one for which a fiber functor with values in the base field exists. By the term ``filtered tannakian category" we mean a tannakian category equipped with a filtration $W_\bullet$ (called the weight filtration) as described in the beginning of \S \ref{sec: intro detailed overview}. By a weight of an object $X$ of a filtered tannakian category we mean an integer $n$ such that $W_{n-1}X\neq W_nX$.

For the purposes of this paper it is important to carefully distinguish between some terms that are sometimes abused in the literature. By an extension we mean a 1-extension, i.e. a short exact sequence. We will have to distinguish between an extension and the object that sits in its middle. We will often use upper case English letters in script font (e.g. $\sN$, $\sX$) for an extension or its class in the $Ext^1$ group (which of the two is the intended interpretation will be clear from the context or explicitly mentioned), and use upper case English letters in ordinary font (e.g. $N$, $X$) for objects of $\bT$. The notation $EXT^i(X,Y)$ (or simply $EXT(X,Y)$ when $i=1$) will be used for the collection of $i$-extensions (or simply, extensions when $i=1$) of $X$ by $Y$. As usual, $Ext^i(X,Y)$ denotes the group (or vector space, since our categories are tannakian) of the equivalence classes of $i$-extensions with respect to the standard equivalence, given by isomorphism of $i$-extensions, i.e. morphisms between the middle objects that induce identity on $X$ and $Y$. Note that throughout, our notations for Ext and Hom groups (as well as for the proper classes $EXT$) do not include a mention of the category under consideration (simply denoting these groups by $Ext^i$, $Hom$, etc). In a few occasions where the category is not clear from the context, it will be included in the notation as a subscript (e.g. as in $Ext^1_{\langle X\rangle}$).

Finally, internal Homs are denoted by $\inHom$ and all our group actions are always designed to be left actions.
\medskip\par 
\noindent {\bf Acknowledgements.} I would like to thank Kumar Murty for many very helpful conversations. This work is a natural continuation of the work \cite{EM2} with him. I would also like to thank Peter Jossen and Daniel Bertrand for some helpful correspondences, and in particular for bringing possible connections to \cite{Fe20} and \cite{BVK16} to my attention.

\section{Recollections on blended extensions}\label{sec: background on blended extensions}
In this section we recall some of the basics of the theory of blended extensions in abelian categories. The original reference is \S 9.3 of Grothendieck's \cite{Gr68}. Another excellent reference is Bertrand's \cite{Ber13}. 

Let $\bT$ be an abelian category. Fix two extensions
\[
\begin{tikzcd}[row sep = small]
\sL: \hspace{0.3in}0 \arrow[r] & A_1 \arrow[r] & L \arrow[r] & A_2 \arrow[r] & 0\\
\sN: \hspace{0.3in}0 \arrow[r] & A_2 \arrow[r] & N \arrow[r] & A_3 \arrow[r] & 0
\end{tikzcd}
\]
%and
%\[
%\begin{tikzcd}
%\sN: \hspace{0.3in}0 \arrow[r] & A_2 \arrow[r] & N \arrow[r] & A_3 \arrow[r] & 0,
%\end{tikzcd}
%\]
in $\bT$.  A blended extension of $\sN$ by $\sL$ by definition is a diagram of the form 
\begin{equation}\label{eq53}
\begin{tikzcd}
   & & 0 \arrow{d} & 0 \arrow{d} &\\
   0 \arrow[r] & A_1 \ar[equal]{d} \arrow[r, ] & L  \arrow[d] \arrow[r, ] &  A_2 \arrow{d} \arrow[r] & 0 \\
   0 \arrow[r] & A_1 \arrow[r] & X \arrow[d] \arrow[r] &  N \arrow{d}  \arrow[r] & 0 \\
   & & A_3 \arrow{d} \ar[equal]{r} & A_3 \arrow{d} & \\
   & & 0 & 0 &   
\end{tikzcd}
\end{equation}
where the rows and columns are exact. The collection of all blended extensions of $\sN$ by $\sL$ is denoted by  $EXTPAN(\sN,\sL)$ (for {\it extension panach\'{e}es}). We will refer to the object $X$ in the diagram as the ``middle object".

The standard notion of a morphism of blended extensions of $\sN$ by $\sL$ is a morphism in $\bT$ between the middle objects which induces identity on $A_1$, $L$, $A_2$, $N$ and $A_3$. Via this notion of morphisms, $EXTPAN(\sN,\sL)$ is a category in which every morphism is an isomorphism (i.e. is a groupoid category). The collection of isomorphism classes of blended extensions of $\sN$ by $\sL$ is denoted by $Extpan(\sN,\sL)$.

We recall three basic results about blended extensions here, which together form the contents of Proposition 9.3.8 of \cite{Gr68}. The first is that when $Extpan(\sN,\sL)$ is nonempty, it has a natural structure of a torsor over $Ext^1(A_3,A_1)$\footnote{In fact, one can do this at the level of the categories and make $EXTPAN(\sN,\sL)$ a torsor over $EXT(A_3,A_1)$. See page 105 of \cite{Gr68}.}. The action of $Ext^1(A_3,A_1)$ on $Extpan(\sN,\sL)$ can be described as follows. Denote the map $N\twoheadrightarrow A_3$ in $\sN$ by $\omega$. Let $\sX\in EXTPAN(\sN,\sL)$ be the blended extension \eqref{eq53}. Let $\sX^h \in EXT(N,A_1)$ be its second row. Given an element $\sE\in EXT(A_3,A_1)$, consider the Baer sum
\[
\sX^h+\omega^\ast\sE \in EXT(N,A_1).
\]
There is a canonical map from $L$ to the middle object of $\sX^h+\omega^\ast\sE$ and a canonical map from this middle object to $A_3$, and these make $\sX^h+\omega^\ast\sE$ the second row of an element of $EXTPAN(\sN,\sL)$. Denote this element by $\sE\ast\sX$ (we may call it the translation of $\sX$ by $\sE$). One can check that the map
\[
EXT(A_3,A_1) \times EXTPAN(\sN,\sL) \rightarrow EXTPAN(\sN,\sL) \hspace{.5in} (\sE,\sX)\mapsto \sE\ast\sX
\]
descends to a map
\begin{equation}\label{eq64}
Ext^1(A_3,A_1) \times Extpan(\sN,\sL) \rightarrow Extpan(\sN,\sL).
\end{equation}
If $Extpan(\sN,\sL)$ is nonempty, the map above makes it a torsor over $Ext^1(A_3,A_1)$. We also use the symbol $\ast$ for the descended action. 

Of course, there is an alternative way to try to define the translation of $\sX$ by $\sE$, namely by pushing an extension of $A_3$ by $A_1$ forward along the injection $A_1\hookrightarrow L$ in $\sL$, and then adding it in $EXT(A_3,L)$ to the first vertical extension in $\sX$ (in fact, this is the original construction given in \cite{Gr68}). Bertrand has checked that the two constructions coincide after passing to the level of equivalence classes. See the appendix of \cite{Ber13}. (He shows that even without going to equivalence classes the two constructions coincide up to a canonical isomorphism.)

For referencing purposes we record the other two basic results as a lemma below. Before stating the lemma, recall the Yoneda product
\[
EXT(A_2,A_1) \times EXT(A_3,A_2) \rightarrow EXT^2(A_3,A_1)
\]
given by splicing. With $\sL$ and $\sN$ as above, the Yoneda product of $\sL$ and $\sN$ is given by
\[
\begin{tikzcd}
0 \arrow[r] & A_1 \arrow[r] & L \arrow[r] & N \arrow[r] & A_3 \arrow[r] & 0
\end{tikzcd}
\]
where the arrows $A_1\rightarrow L$ and $N\rightarrow A_3$ come from $\sL$ and $\sN$, and $L\rightarrow N$ is the composition $L\rightarrow A_2\rightarrow N$. (See \cite[\S 3]{Yo60}.)
\begin{lemma}\label{lem: criteria for compatibility of extension pairs}
(a) Given $\sN\in EXT(A_3,A_2)$ and $\sL\in EXT(A_2,A_1)$, there exists a blended extension of $\sN$ by $\sL$ if and only if the Yoneda product of $\sL$ and $\sN$ vanishes in $Ext^2(A_3,A_1)$. 

\noindent (b) The automorphism group of a blended extension of $\sN$ by $\sL$ is in a canonical bijection with $Hom(A_3,A_1)$. 
\end{lemma}
For proofs, see Proposition 9.3.8(a,c) of \cite{Gr68} (or Lemma 6.4.1 of \cite{EM2} for part (a)). Note that throughout the paper, we shall only deal with blended extensions for which 
\[ Hom(A_3,A_1) \cong 0, \]
so that they always have a trivial automorphism group.

We end this section with a remark about various equivalence relations for blended extensions and an observation. Throughout the paper, by the standard equivalence relation on blended extensions of $\sN$ by $\sL$ we mean one that is given by the notion of morphisms defined above (with morphisms being identity on the first row and second column), and the notation $Extpan(\sN,\sL)$ is always used in reference to this relation. There are however two coarser equivalence relations that one may alternatively consider. In the first alternative relation, two blended extensions of $\sN$ by $\sL$ are considered equivalent if there is a morphism between the middle objects that induces induced morphisms on the analogous objects of the two diagrams such that the induced morphisms on $A_1$, $A_2$ and $A_3$ (but not necessarily on $L$ and $N$) are identity. In the second alternative relation, which is the coarsest of all three equivalence relations, one declares two blended extensions to be equivalent if there exists a morphism between the middle objects that induces isomorphisms between the analogous objects of the two diagrams. In both cases, again morphisms will be automatically isomorphisms. 

We should note that in the paper, we shall only deal with blended extensions for which 
\begin{equation}\label{eq59}
Hom(A_2,A_1)\cong Hom(A_3,A_2) \cong 0.
\end{equation}
In this case, the only automorphism of $L$ (resp. $N$) that induces identity and $A_2$ and $A_1$ (resp. $A_3$) is the identity, so that the standard notion of equivalence on $EXTPAN(\sN,\sL)$ coincides with the one only requiring morphisms to be identity on the $A_i$. (Later in the paper, when we introduce the notion of generalized extensions, the equivalence that requires inducing identity on the $A_i$ is denoted by $\sim'$, whereas the one that allows for arbitrary automorphisms of the $A_i$ is denoted by $\sim$.)

Finally, we make a further observation about blended extensions in the case where \eqref{eq59} holds. In general, without assuming \eqref{eq59}, thanks to the fact that isomorphisms in $EXTPAN(\sN,\sL)$ are identity on $\sN$ and $\sL$, by sending the isomorphism class of a blended extension to the class of the second row of a representative we have a well-defined map
\[
-^h: Extpan(\sN,\sL) \rightarrow Ext^1(N,A_1).
\]
As before, let $\omega$ be the map $N\twoheadrightarrow A_3$ in $\sN$. Let $\iota$ be the map $A_2\hookrightarrow N$. Given $\sX\in Extpan(\sN,\sL)$ and $\sE\in Ext^1(A_3,A_1)$, by definition of the torsor structure on $Extpan(\sN,\sL)$ we have
\[
(\sE\ast \sX)^h = \omega^\ast\sE+\sX^h.
\]
Thus $(\sE\ast \sX)^h=\sX^h$ in $Ext^1(N,A_1)$ if and only if 
\[\sE \in \ker\bigm(Ext^1(A_3,A_1)\xrightarrow{ \ \omega^\ast \ } Ext^1(N,A_1)\bigm) = Im\bigm(Hom(A_2,A_1)\rightarrow Ext^1(A_3,A_1)\bigm),\]
where the latter map is the connecting homomorphism in the long exact sequence obtained by applying $Hom(-,A_1)$ to $\sN$. In view of the fact that the action of $Ext^1(\sN,\sL)$ on $Extpan(\sN,\sL)$ is transitive we obtain the following statement:
\begin{lemma}\label{lem: X mapsto X^h is injective when Hom(A_2,A_1)=0}
Suppose that $Hom(A_2,A_1)\cong 0$. Then the map
\[
Extpan(\sN,\sL) \rightarrow Ext^1(N,A_1)
\]
which sends the class of a blended extension to the class of its second row is injective.
\end{lemma}

\section{Classification of objects attached to the same extension pair}\label{sec: blended extensions II}
\subsection{A classification problem}
Let $\mathbf{T}$ be a filtered tannakian category over a field $K$ of characteristic zero. Throughout this section, we fix nonzero objects $A_1, A_2, A_3$ of $\mathbf{T}$ such that every weight of $A_1$ is less than every weight of $A_2$, and every weight of $A_2$ is less than every weight of $A_3$ (recall that we say an integer $n$ is a weight of an object $X$ if $W_{n-1}X\neq W_nX$). Note that in this section, we do not assume that $A_1$, $A_2$, $A_3$ are pure. The hypothesis on the weights of $A_1, A_2, A_3$ implies in particular that
\begin{equation}\label{eq2}
Hom(A_j, A_i) \cong 0 \hspace{.3in}\text{for all $i,j$ with $j>i$.}  
\end{equation}
We also fix an extension $\mathcal{L}$ of $A_2$ by $A_1$:
\[
\begin{tikzcd}
   0 \arrow[r] & A_1 \arrow[r, ] & L \arrow[r, ] &  A_2  \arrow[r] & 0
\end{tikzcd}
\]
and an extension $\mathcal{N}$ of $A_3$ by $A_2$:
\[
\begin{tikzcd}
   0 \arrow[r] & A_2 \arrow[r, ] & N \arrow[r, ] &  A_3  \arrow[r] & 0.
\end{tikzcd}
\]
We will adopt the following terminology from \cite{EM2}:
\begin{defn}\label{def: X attached to a pair}
We say an object $X$ of $\bT$ is attached to $(\sL,\sN)$ if there extensions a blended extension of $\sN$ by $\sL$ which has $X$ as its middle object.
\end{defn}
Our goal in this section is to give a solution to the following classification problem:
\begin{problem}\label{problem: classify objects attached to an extension pair}
Classify up to isomorphism in $\bT$ all objects of $\bT$ attached to the pair $(\mathcal{L},\mathcal{N})$. (Note that here an isomorphism is an isomorphism of objects of $\bT$, rather than an isomorphism of blended extensions.)
\end{problem}
Before proceeding any further, we note that if $(\sL ', \sN')$ represents the same element of $Ext^1(A_2,A_1)\times Ext^1(A_3,A_2)$ as $(\sL,\sN)$, then any object is attached to $(\sL ', \sN')$ if and only if it is attached to $(\sL,\sN)$, so that the problem above is really about classification of objects attached to the class of $(\sL,\sN)$ in $Ext^1(A_2,A_1)\times Ext^1(A_3,A_2)$. 

By considering its middle object, every blended extension of $\sN$ by $\sL$ gives an object attached to the pair. Moreover, two isomorphic blended extensions give isomorphic objects. Thus we obtain a map
\[
Extpan(\sN,\sL) \rightarrow \{\text{objects attached to $(\sN,\sL)$}\}/\text{isomorphism in $\bT$}
\] 
that is surjective by definition. Thus Problem \ref{problem: classify objects attached to an extension pair} amounts to understanding when two non-isomorphic blended extensions have isomorphic middle objects.

\subsection{The group $\Gamma$}  
Thanks to the filtered tannakian setting and the hypothesis that every weight of $A_1$ (resp. $A_2$) is less than every weight of $A_2$ (resp. $A_3$), every automorphism of $L$ (resp. $N$) restricts to an automorphism of $A_1$ (resp. $A_2$). In view of  \eqref{eq2},
the maps
\[
Aut(L) \rightarrow Aut(A_1)\times Aut(A_2) \hspace{.3in} \sigma \mapsto (\sigma_{A_1},\sigma_{A_2})
\]
and
\[
Aut(N) \rightarrow Aut(A_2)\times Aut(A_3) \hspace{.3in} \sigma\mapsto (\sigma_{A_2},\sigma_{A_3})
\]
are injective. Here, and throughout the paper, given an automorphism $\sigma$ of an object $X$ and a subquotient $Y$ of $X$, the notation $\sigma_Y$ means the automorphism of $Y$ induced by $\sigma$ (when there is such an induced automorphism). We identify $Aut(L)$ and $Aut(N)$ with their images under the embeddings above. 
\begin{defn}
We set $\Gamma(\sL,\sN)$ to be the intersection of $Aut(L)\times Aut(A_3)$ and $Aut(A_1)\times Aut(N)$ in
\[
Aut(A_1)\times Aut(A_2)\times Aut(A_3).
\]
\end{defn}
An element of $\sigma\in \Gamma(\sL,\sN)$ can be thought of in two equivalent ways: 
\begin{itemize}
\item[-] A triple 
\[(\sigma_{A_1},\sigma_{A_2},\sigma_{A_3})\in Aut(A_1)\times Aut(A_2)\times Aut(A_3)\]
such that $\sigma_{A_1}$ and $\sigma_{A_2}$ (resp. $\sigma_{A_2}$ and $\sigma_{A_3}$) patch up via the extension $\sL$ (resp. $\sN$) to form an automorphism $\sigma_L$ of $L$ (resp. $\sigma_N$ of $N$). The automorphisms $\sigma_N$ and $\sigma_L$ are unique. 
\item[-] A pair 
\[
(\sigma_L,\sigma_N)\in Aut(L)\times Aut(N)
\]
such that the automorphisms induced on $A_2$ by $\sigma_L$ and $\sigma_N$ coincide:
\[
(\sigma_L)_{A_2} = (\sigma_N)_{A_2}.
\]
\end{itemize}
The passage from the first characterization to the second is just as the notation suggests (with the desired $\sigma_L$ and $\sigma_N$ of the second characterization being those mentioned in the first). The passage from the second characterization to the first is by setting
\[
\sigma_{A_1}=(\sigma_L)_{A_1},~\hspace{.2in}~\sigma_{A_2}=(\sigma_L)_{A_2}=(\sigma_N)_{A_2},~\text{and} \hspace{.1in} \sigma_{A_3}=(\sigma_N)_{A_3}.
\]

\subsection{The $\Gamma$-action on $Extpan(\mathcal{N},\mathcal{L})$}
The group $\Gamma(\sL,\sN)$ acts on $EXTPAN(\sN,\sL)$ by twisting the arrows. More precisely, the action (designed as a left action) is given as follows: $\sigma\in \Gamma(\sL,\sN)$ maps the blended extension on the left to the one on the right:
\begin{equation}\label{eq1}
\begin{tikzcd}
   & & 0 \arrow{d} & 0 \arrow{d} &\\
   0 \arrow[r] & A_1 \ar[equal]{d} \arrow[r, ] & L  \arrow[d, "\widetilde{\iota}"] \arrow[r, ] &  A_2 \arrow{d} \arrow[r] & 0 \\
   0 \arrow[r] & A_1 \arrow[r, "\widetilde{j}"] & X \arrow[d, "\widetilde{\omega}"] \arrow[r, "\widetilde{\pi}"] &  N \arrow{d}  \arrow[r] & 0 \\
   & & A_3 \arrow{d} \ar[equal]{r} & A_3 \arrow{d} & \\
   & & 0 & 0 &   
\end{tikzcd} 
\hspace{.2in}
\begin{tikzcd}
   & & 0 \arrow{d} & 0 \arrow{d} &\\
   0 \arrow[r] & A_1 \ar[equal]{d} \arrow[r, ] & L  \arrow[d, "\widetilde{\iota}\sigma_L^{-1}"] \arrow[r, ] &  A_2 \arrow{d} \arrow[r] & 0 \\
   0 \arrow[r] & A_1 \arrow[r, "\widetilde{j}\sigma_{A_1}^{-1}"] & X \arrow[d, "\sigma_{A_3}\widetilde{\omega}"] \arrow[r, "\sigma_N\widetilde{\pi}"] &  N \arrow{d}  \arrow[r] & 0. \\
   & & A_3 \arrow{d} \ar[equal]{r} & A_3 \arrow{d} & \\
   & & 0 & 0 &   
\end{tikzcd} 
\end{equation}
Note that the arrows on the top row and the right column remain unchanged. That the diagram on the right is commutative is thanks to the compatibility properties satisfied by the elements of $\Gamma(\sL,\sN)$. 

We denote the twist of a blended extension $\sX$ by $\sigma$ by $\sigma\cdot \sX$. If two blended extensions $\sX$ and $\sX'$ of $\sN$ by $\sL$ with middle objects denoted by $X$ and $X'$ are isomorphic as blended extensions, the same isomorphism $X\rightarrow X'$ which gives an isomorphism between $\sX$ and $\sX'$ is also an isomorphism between $\sigma\cdot \sX$ and $\sigma\cdot \sX'$. Thus the action of $\Gamma(\sL,\sN)$ on $EXTPAN(\sN, \sL)$ descends to an action on the set $Extpan(\sN, \sL)$ of isomorphism classes of blended extensions of $\sN$ by $\sL$.

\subsection{The solution to the problem}
We are ready to give an answer to Problem \ref{problem: classify objects attached to an extension pair}:
\begin{prop}\label{prop: classifification of objects attached to an extension pair}
Assume the previous hypothesis on $A_1$, $A_2$, and $A_3$ (i.e. that every weight of $A_1$ is less than every weight of $A_2$, and every weight of $A_2$ is less than every weight of $A_3$). Let $\sX$ and $\sX'$ be blended extensions of $\sN$ by $\sL$. Then the middle objects of $\sX$ and $\sX'$ are isomorphic in $\bT$ if and only if the classes of $\sX$ and $\sX'$ in $Extpan(\sN,\sL)$ belong to the same orbit of the action of $\Gamma(\sL,\sN)$ . In particular, sending a blended extension to its middle object descends to a bijection
\[
Extpan(\sN,\sL)/\Gamma(\sL,\sN) ~~\cong  ~~\{\text{objects attached to $(\sL,\sN)$}\}/\text{isomorphism in $\bT$}.
\]
\end{prop}
\begin{proof}
For the ``if" implication, suppose that the classes of $\sX$ and $\sX'$ belong to the same orbit of the action of $\Gamma(\sL,\sN)$ on $Extpan(\sN,\sL)$. Then there is $\sigma\in \Gamma(\sL,\sN)$ such that $\sigma\cdot \sX$ is isomorphic to $\sX'$. In particular, the middle objects of $\sigma \cdot \sX$ and $\sX'$ are isomorphic. The former has the same middle object as $\sX$ does. 

Turning our attention to the ``only if" implication, suppose the middle objects of $\sX$ and $\sX'$, which we call $X$ and $X'$ respectively, are isomorphic. Using an isomorphism between $X$ and $X'$ to replace $X'$ with $X$ in $\sX'$ (and adjusting the arrows in the second row and first column of $\sX'$ accordingly), we obtain a blended extension isomorphic to $\sX'$ which has $X$ as its middle object. Thus we may assume that $X'=X$ and show that there is $\sigma\in \Gamma(\sL,\sN)$ such that $\sigma\cdot \sX=\sX'$ in $EXTPAN(\sN,\sL)$.

Denote the maps on the second row and first column of $\sX$ by $\widetilde{\iota}$, $\widetilde{\omega}$, $\widetilde{j}$ and $\widetilde{\pi}$, as shown in the left diagram of \eqref{eq1}. Denote the counterparts in $\sX'$ by the prime-decorated analogues, i.e. $\widetilde{\iota}'$, etc. By the assumption on the weights of $A_1$, $A_2$ and $A_3$, the images of $L$ and $A_1$ in $X$ for the two blended extensions $\sX$ and $\sX'$ are the same. It follows that there are unique automorphisms $\sigma_{A_1}$, $\sigma_L$, $\sigma_N$ and $\sigma_{A_3}$ of $A_1$, $L$, $N$ and $A_3$ respectively such that
\[
\widetilde{\iota}' = \widetilde{\iota}\sigma_L^{-1} \ ,\hspace{.2in}\widetilde{j}' = \widetilde{j}\sigma_{A_1}^{-1} \ , \hspace{.2in} \widetilde{\pi}'=\sigma_N\widetilde{\pi}\ ,\hspace{.2in} \text{and}\hspace{.2in}  \widetilde{\omega}'=\sigma_{A_3}\widetilde{\omega}.
\]
It follows from the commutativity of the bottom right squares (involving $X$, $N$ and $A_3$) and the top left squares (involving $A_1$, $L$ and $X$) in $\sX$ and $\sX'$ that
\[
\sigma_{A_3}=(\sigma_N)_{A_3} \hspace{.2in} \text{and} \hspace{.2in} \sigma_{A_1}=(\sigma_L)_{A_1}.
\]   
From the commutativity of the top right squares in the two diagrams we get that
\[
(\sigma_N)_{A_2}=(\sigma_L)_{A_2}.
\]
It follows that
\[
\sigma=(\sigma_{A_1},\sigma_{A_2},\sigma_{A_3})
\]
is in $\Gamma(\sL,\sN)$. The element $\sigma$ twists $\sX$ to $\sX'$.

The final assertion in the Proposition about the canonical bijection follows immediately from the first part.
\end{proof}

\subsection{The stabilizers of the $\Gamma$-action}
Fix a blended extension $\sX$ of $\sN$ by $\sL$ with the middle object denoted by $X$ and arrows labelled as in the left diagram of \eqref{eq1}. Let $\sigma\in Aut(X)$. Due to our filtered setting and the assumption that every weight of $A_1$ (resp. $A_2$) is less than every weight of $A_2$ (resp. $A_3$), $\sigma$ restricts via $\widetilde{\iota}$ (resp. descends via $\widetilde{\pi}$) to an automorphism of $L$ (resp. $N$), which we denote by $\sigma_L$ (resp. $\sigma_N$). Via $\widetilde{j}$ and $\widetilde{\omega}$, the automorphism $\sigma$ also induces automorphisms $\sigma_{A_1}$ and $\sigma_{A_3}$ of $A_1$ and $A_3$, which are respectively the same automorphisms that are induced by $\sigma_L$ and $\sigma_N$ via the fixed injection $A_1\hookrightarrow L$ and surjection $N\twoheadrightarrow A_3$ in $\sL$ and $\sN$. Moreover, one can see using the commutativity of the top right square in $\sX$ that $\sigma_L$ and $\sigma_N$ agree on $A_2$, that is $(\sigma_L)_{A_2}=(\sigma_N)_{A_2}$. Denoting this automorphism induced on $A_2$ by $\sigma_{A_2}$, we have a homomorphism
\[
\Theta_\sX: Aut(X) \rightarrow \Gamma(\sL,\sN) \hspace{.3in}\sigma\mapsto (\sigma_{A_1},\sigma_{A_2},\sigma_{A_3}),
\]
which is injective because
\begin{equation}\label{eq3}
Hom(A_2,A_1) \cong Hom(A_3,L) \cong 0.
\end{equation}
In fact, the embedding $\Theta_\sX$ is canonical with respect to the isomorphism class of $\sX$ in $Extpan(\sN,\sL)$, in the following sense. Because of \eqref{eq2}, every blended extension of $\sN$ by $\sL$ has a trivial automorphism group (as a blended extension). Thus if $\sX_1$ and $\sX_2$ are isomorphic blended extensions, then there is a unique isomorphism of blended extensions from one to another. The two maps $\Theta_{\sX_1}$ and $\Theta_{\sX_2}$ translate to one another via this isomorphism. In particular, they have the same image in $\Gamma(\sL,\sN)$.   

We now describe the stabilizers of the action of $\Gamma(\sL,\sN)$ on $Extpan(\sN,\sL)$:
\begin{prop}\label{prop: stabilizer of a blended extension class}
Let $\sX$ be a blended extension of $\sN$ by $\sL$, with the middle object denoted by $X$. Then the stabilizer of the class of $\sX$ in $Extpan(\sN,\sL)$ under the action of $\Gamma(\sL,\sN)$ is the image of $\Theta_\sX$. (We call this image the copy of $Aut(X)$ embedded in $\Gamma(\sL,\sN)$ via $\sX$.) 
\end{prop}
\begin{proof}
This is by construction. Indeed, let $\sigma\in \Gamma(\sL,\sN)$. Then $\sigma$ stabilizes the isomorphism class of $\sX$ if and only if $\sigma\cdot \sX$ is isomorphic (as a blended extension) to $\sX$. Let $\sX$ be the diagram on the left of \eqref{eq1}, so that $\sigma\cdot \sX$ is the diagram on its right. An isomorphism $\sigma\cdot\sX\rightarrow \sX$ is an automorphism $\delta$ of $X$ such that
\[
\delta\widetilde{\iota}\sigma_L^{-1}=\widetilde{\iota} \ ,\hspace{.2in} \delta\widetilde{j}\sigma_{A_1}^{-1}=\widetilde{j}  \ , \hspace{.2in} \widetilde{\pi}\delta=\sigma_N\widetilde{\pi}\ ,\hspace{.2in} \text{and}\hspace{.2in}  \widetilde{\omega}\delta=\sigma_{A_3}\widetilde{\omega},
\]
or equivalently,
\[
\Theta_{\sX}(\delta)=\sigma.
\]
\end{proof}

\subsection{The special case of a totally nonsplit extension pair}\label{sec: tot nonsplit case with 3 weights}
We now specialize the situation to the case of a totally nonsplit extension pair. Let us start by recalling the definiton and some properties:
\begin{defn}\label{def: tot nonsplit exts}
Let $X$ and $Y$ be objects of $\bT$.

\noindent (a) An extension or an extension class $\mathcal{E}$ of $\mathbbm{1}$ by $X$ is called totally nonsplit if for every proper subobject $X'$ of $X$ the pushforward of $\mathcal{E}$ along the quotient map $X\rightarrow X/X'$ is nonsplit.

\noindent (b) An extension or an extension class $\mathcal{E}$ of $Y$ by $X$ is called totally nonsplit if the extension class of $\mathbbm{1}$ by $\inHom(Y,X)$ corresponding to $\mathcal{E}$ under the canonical isomorphism
\begin{equation}\label{eq48}
Ext^1(Y,X) \cong Ext^1(\mathbbm{1}, \inHom(Y,X))
\end{equation}
is totally nonsplit.
\end{defn}
We first make a remark about the special case of the definition about extensions of $\mathbbm{1}$ by $X$. In view of the long exact sequence obtained by applying the functor $Hom(\mathbbm{1},-)$ to the sequence
\[
\begin{tikzcd}
   0 \arrow[r] & X' \arrow[r, ] & X \arrow[r, ] &  X/X'  \arrow[r] & 0,
\end{tikzcd}
\]
an extension (or extension class) $\mathcal{E}$ of $\mathbbm{1}$ by $X$ is totally nonsplit if and only if for every proper subobject $X'$ of $X$, the extension class of $\mathcal{E}$ is not in the image of the pushforward map
\[
Ext^1(\mathbbm{1}, X') \rightarrow Ext^1(\mathbbm{1}, X).
\]
We also make a cautionary remark about the general case of the definition, when $\mathcal{E}$ is an extension of $Y$ by $X$. The reader should keep in mind that in this definition, for $\mathcal{E}$ to be totally nonsplit we need to first consider $\mathcal{E}$ as an extension of $\mathbbm{1}$. The notion would remain the same if we considered the extension as an extension by $\mathbbm{1}$ (with the statements being dualized). However, the more naively defined notion for which one keeps $\mathcal{E}$ as an extension of $Y$ by $X$ and only considers pushforwards by subobjects of $X$ and pullbacks by subobjects of $Y$ is not as well behaved. In any case, we do not work with that weaker notion.

What makes totally nonsplit extensions important for us in this section and \S \ref{sec: objects in filtered tan cats with given gr} is the following property\footnote{Totally nonsplit extensions will also play a crucial role in \S \ref{sec: mixed motives with maximal unipotent radicals}, where we deal with motives with maximal unipotent radicals.}:
\begin{lemma}[Corollary 1.3 of \cite{Es23}, case (ii)]\label{lem: automorphisms of totally nonsplit extensions}
Let $X$ and $Y$ be two nonzero objects of $\bT$. Let $\mathcal{E}$ be a totally nonsplit extension of $Y$ by $X$, with its middle object denoted by $E$. Suppose that every weight of $X$ is less than every weight of $Y$. Then the only endomorphisms of $E$ are the scalar maps.\footnote{The statement in \cite{Es23} asserts that if the set of weights of $X$ and $Y$ are disjoint, then the only endomorphisms of $E$ that stabilize $X$ are the scalar maps. To get the version stated here note that if every weight of $X$ is less than every weight of $Y$, then every endormorphism of $E$ has to stabilize $X$.} 
\end{lemma}
We now come back to our discussion of blended extensions. Putting together the previous lemma together with the earlier material of the section we obtain the following result. Recall that $K$ is the base field of $\bT$ (that is, $\bT$ is a filtered tannakian category over the field $K$ of characteristic zero). The automorphism group of every nonzero object of $\bT$ contains a natural copy of $K^\times$.
\begin{prop}\label{prop: summary of ext pan for totally nonsplit pairs}
Assume as before the every weight of $A_1$ (resp. $A_2$) is less than every weight of $A_2$ (resp. $A_3$). Suppose that $\sN$ and $\sL$ are both totally nonsplit. Then $\Gamma(\sL,\sN)$ is the diagonal copy of $K^\times$ in $Aut(A_1)\times Aut(A_2)\times Aut(A_3)$. In particular, for every blended extension $\sX$ of $\sN$ by $\sL$ the map $\Theta_\sX$ is an isomorphism. For every object $X$ attached to $(\sL,\sN)$ (see Definition \ref{def: X attached to a pair}) the only automorphisms of $X$ (as an object of $\bT$) are the scalar ones. The action of $\Gamma(\sL,\sN)$ on $Extpan(\sN,\sL)$ is trivial, and we have a canonical bijection between the isomorphism classes of objects of $\bT$ attached to $(\sL,\sN)$ (isomorphisms being those in $\bT$) and $Extpan(\sN,\sL)$.
\end{prop}
\begin{proof}
Thanks to the total nonsplitting assumption, $Aut(L)$ (resp. $Aut(N)$) is the diagonal copy of $K^\times$ in $Aut(A_1)\times Aut(A_2)$ (resp. $Aut(A_2)\times Aut(A_3)$). Thus 
\[
\Gamma(\sL,\sN) = (Aut(L)\times Aut(A_3)) \cap (Aut(A_1)\times Aut(N)) \subset Aut(A_1)\times Aut(A_2)\times Aut(A_3)
\] 
is the diagonal copy of $K^\times$ in $Aut(A_1)\times Aut(A_2)\times Aut(A_3)$. 

For every blended extension $\sX$, the image of $\Theta_\sX$ includes the diagonal copy of $K^\times$ (as the image of scalar automorphisms of the middle object), so the assertions about $\Theta_\sX$ and $Aut(X)$ are clear. The triviality of the action of $\Gamma(\sL,\sN)$ on $Extpan(\sN,\sL)$ is by Proposition \ref{prop: stabilizer of a blended extension class}, and the canonical bijection comes from Proposition \ref{prop: classifification of objects attached to an extension pair}.
\end{proof}

\subsection{Beyond the totally nonsplit case: some remarks}\label{sec: beyond totally nonsplit} 
We saw that in the case where $\sN$ and $\sL$ are totally nonsplit, one has $\Gamma(\sL,\sN)\cong K^\times$ (as the diagonal copy), so that for every blended extension $\sX$ of $\sN$ by $\sL$ the map $\Theta_{\sX}$ is an isomorphism. Let us give a definition:
\begin{defn}
We say an object $X$ attached to $(\sL,\sN)$ has a maximal automorphism group if for some (or every) blended extension $\sX$ of $\sN$ by $\sL$ that has $X$ as its middle object the map $\Theta_\sX$ is an isomorphism.
\end{defn}
The equivalent of using the quantifiers {\it some} or {\it every} is because the classes of any two blended extensions of $\sN$ by $\sL$ that have $X$ as the middle object are in the same $\Gamma(\sL,\sN)$-orbit, and hence their stabilizers are conjugates of one another in $\Gamma(\sL,\sN)$.

Let $X$ be an object sitting in the middle of $\sX\in EXTPAN(\sN,\sL)$. By definition, to say that $X$ has a maximal automorphism group is to say that every automorphisms $\sigma_N$ and $\sigma_L$ of $N$ and $L$ which coincide on $A_2$ glue together via $\sX$ to give an automorphism of $X$, i.e. there is a (unique) automorphism of $X$ that induces both $\sigma_N$ and $\sigma_L$. In view of Proposition \ref{prop: stabilizer of a blended extension class}, $X$ has a maximal automorphism group if and only if the class of $\sX$ in $Extpan(\sN,\sL)$ is a fixed point of the action of $\Gamma(\sL,\sN)$.

By Proposition \ref{prop: summary of ext pan for totally nonsplit pairs} if $\sL$ and $\sN$ are totally nonsplit, then every object attached to $(\sL,\sN)$ has a maximal automorphism group. One also has the following:
\begin{cor}\label{cor: Aut(M)=Gamma when Ext(C,A)=0}
Assume as before that the every weight of $A_1$ (resp. $A_2$) is less than every weight of $A_2$ (resp. $A_3$), and that $Extpan(\sN,\sL)$ is nonempty. Suppose that $Ext^1(A_3,A_1)\cong 0$. Then the object (which is unique up to isomorphism) attached to $(\sL,\sN)$ has a maximal automorphism group.
\end{cor}
\begin{proof}
Recall from \S \ref{sec: background on blended extensions} that $Extpan(\sN,\sL)$ is a torsor over $Ext^1(A_3,A_1)$. Thus by our assumption, $Extpan(\sN,\sL)$ is a singleton, so that the action of $\Gamma(\sL,\sL)$ on it must be trivial. The claim follows from Proposition \ref{prop: stabilizer of a blended extension class}. 
\end{proof}
The following result gives a nice property of objects attached to $(\sL,\sN)$ that have a maximal automorphism group. We will not include its proof here, as it is similar to the proof of Proposition \ref{prop: ineraction between the Gamma action and the torsor structure} which will be proved later in the paper. (In fact, it is a special case of the said result from later in the paper if $A_1,A_2,A_3$ are pure.)
\begin{prop}\label{prop: which base points make the two actions of Gamma coincide, 3 weight case}
Let $\sX\in Extpan(\sN,\sL)$. Use $\sX$ as the base point for the torsor structure \eqref{eq64} to identify 
\[
Ext^1(A_3,A_1) \cong Extpan(\sN,\sL).
\]
The action of $\Gamma(\sL,\sN)$ on $Ext^1(A_3,A_1)$ obtained by translating the action on $Extpan(\sN,\sL)$ via this identification is linear if and only if $\sX$ is a fixed point of the $\Gamma(\sL,\sN)$-action on $Extpan(\sN,\sL)$ (or equivelently, the middle object of $\sX$ has a maximal automorphism group). Moreover, if this $\Gamma(\sL,\sN)$-action on $Ext^1(A_3,A_1)$ is linear, then it coincides with the restriction to $\Gamma(\sL,\sN)$ of the action of 
\[
Aut(A_1)\times Aut(A_2)\times Aut(A_3)
\]
on $Ext^1(A_3,A_1)$ given by pushforward and pullback (for which $(\sigma_{A_1},\sigma_{A_2}, \sigma_{A_3})$ sends an extension class $\sE$ to $(\sigma_{A_1})_\ast(\sigma_{A_3}^{-1})^\ast \, \sE$).
\end{prop}
We end this discussion with a few side comments. In general, it is certainly not the case that every object attached to an arbitrary pair of extensions $(\sL,\sN)$ has a maximal automorphism group. On the other hand, one can prove that at least when $A_1, A_2, A_3$ are semisimple, the action of $\Gamma(\sL,\sN)$ on $Extpan(\sN,\sL)$ always has a fixed point when $Extpan(\sN,\sL)$ is nonempty. We won't discuss the arguments for these in this paper because they would take us too far away from our main objectives, and instead we will come back to them and further questions in this direction in another paper. Finally, we note that the question of whether or not a given element of $\Gamma(\sL,\sN)$ belongs to the image of $\Theta_\sX$ (for a given $\sX$) has appeared previously in the work \cite{BVK16} of Barbieri-Viale and Kahn\footnote{I thank Bertrand for bringing possible connections to \cite{BVK16} to my attention.}. Proposition D.1.5 of Appendix D therein gives a necessary and sufficient condition for an element $\sigma\in \Gamma(\sL,\sN)$ to be in the image of $\Theta_\sX$. (In the language of Barbieri-Viale and Kahn, the question is about whether a ``partial gluing" extends to a ``gluing".)
%We end this discussion with two side comments which will not be used anywhere in the paper. In general, it is certainly not the case that every object attached to an arbitrary pair of extensions $(\sL,\sN)$ has a maximal automorphism group. On the other hand, one can prove that at least when $A_1, A_2, A_3$ are semisimple, the action of $\Gamma(\sL,\sN)$ on $Extpan(\sN,\sL)$ always has a fixed point when $Extpan(\sN,\sL)$ is nonempty. We won't discuss the arguments for these in this paper because they would take us too far away from our main objectives, and instead we will come back to them and further questions in this direction in another paper.

\section{Objects with prescribed associated graded}\label{sec: objects in filtered tan cats with given gr}
\subsection{Two classification problems}\label{sec: statement of problems for S(A) and S'(A)}
As before, let $\bT$ be a filtered tannakian category over a field $K$ of characteristic zero. Let $A_1,\ldots, A_k$ be nonzero {\it pure} objects, respectively of weights $p_1,\ldots, p_k$ with $p_1<\cdots<p_k$. Set
\[
A:= \bigoplus\limits_{1\leq r\leq k} A_r.
\]
Our main object of study in this section are the sets $S(A)$ and $S'(A)$ defined below:
\begin{defn}\label{def: S(A) and S'(A)}
\noindent (a) We denote by $S(A)$ the set of isomorphism classes of objects of $\bT$ whose associated graded (with respect to the weight filtration) is isomorphic to $A$:
\[
S(A):=~\{X\in \text{Obj}(\bT): \text{$Gr^WX$ is isomorphic to $A$}\}\bigm/ \text{isomorphism in $\bT$}.
\]
(Note that here, we do not keep track of the data of the isomorphisms between the associated gradeds and $A$.)

\noindent (b) We denote by $S'(A)$ the set of equivalence classes of pairs 
\[
(X, \, Gr^WX\xrightarrow{\phi, \, \simeq }A ) 
\]
of an object $X$ of $\bT$ whose associated graded is isomorphic to $A$ together with a choice of an isomorphism $\phi: Gr^WX\rightarrow A$. Here, two pairs $(X, \phi )$ and $(X', \phi')$ are declared to be equivalent if there exists an isomorphism $f: X\rightarrow X'$ for which we have $\phi' \, Gr^Wf=\phi$, where $Gr^Wf:  Gr^WX\rightarrow Gr^WX'$ is the isomorphism induced by $f$. (Recall that the weight filtration is functorial.)
\end{defn}
The group $Aut(A)$ acts on $S'(A)$ by twisting the isomorphisms between the associated gradeds and $A$. More precisely, given a pair $(X, \phi )$ as in (b) and $\sigma\in Aut(A)$, we set 
\[
\sigma\cdot (X, \phi ) = (X, \sigma\phi ).
\]
This defines an action of $Aut(A)$ on the collection of pairs $(X,\phi)$ as in (b) which is easily seen to descend to an action on $S'(A)$.

There is an obvious surjection 
\[
S'(A)\twoheadrightarrow S(A)
\] 
induced by forgetting the data of $\phi$. The reader can easily see that two elements of $S'(A)$ are mapped to the same element of $S(A)$ if and only if they belong to the same orbit of $Aut(A)$. For future referencing, we record the conclusion:
\begin{lemma}\label{lem: relation between S'(A) and S(A)}
The natural surjection $S'(A)\twoheadrightarrow S(A)$ induced by $(X,\phi)\mapsto X$ descends to a bijection
\[
S'(A)/Aut(A) \cong S(A).
\]
\end{lemma}
Our goal in this section is to study the sets $S(A)$ and $S'(A)$, with the former being our main object of interest. We do not content ourselves in merely the description of $S(A)$ as $S'(A)/Aut(A)$. Much of we shall do amounts exactly to try to understand this quotient better.

The homological characterization of the sets $S(A)$ and $S'(A)$ in case $k=2$ is outlined in \S \ref{sec: about this paper}. In the first subsection below, we consider the first nontrivial\footnote{in the sense that one goes beyond simply the $Ext^1$ groups} case of the problem, which is when $k=3$. We sketch the solution of this case separately as it will prepare us for the general case. The generalization of the picture to an arbitrary $k$ takes up the rest of the section. Our goal will be to prove Theorems \ref{thm: classification of S'(A) in the general case} and \ref{thm: classification of objects with prescribed associated graded, general case} of the Introduction. The main new ingredient of the generalization to the case of an arbitrary $k$ is the notion of a generalized extension, which will be introduced and studied in \S \ref{sec: gen exts, defn} - \S \ref{sec: equiv rels on gen exts}. In \S \ref{sec: sets S_ell S'_ell and restatements of Thms A and B} we give the explicit definition of the sets $S'_\ell(A)$ and $S_\ell(A)$
that appeared in Theorem \ref{thm: classification of S'(A) in the general case} and \ref{thm: classification of objects with prescribed associated graded, general case}. The remaining subsections of address the assertions about the fibers.

In the case where $k=3$, we shall see that for the most part, the problem has already been solved: the main ingredients are results on ``equivalent extension pairs" from \cite{EM2} and the classification problem considered in the previous section.

\subsection{The case with three weights}
In this subsection we discuss the structure of $S'(A)$ and $S(A)$ in the case where $k=3$ (i.e. $A=A_1\oplus A_2\oplus A_3$, where each $A_r$ for $1\leq r\leq 3$ is pure of weight $p_r$ and $p_1<p_2<p_3$). Some of the details of the discussion are left out. These details will be filled out as we cover the case of an arbitrary $k$ later. Nonetheless, the outline of the picture for $k=3$ provided here will hopefully put the results for an arbitrary $k$ in a better context.

Given a pair $(X, Gr^WX\xrightarrow{\phi, \simeq} A)$, identifying $Gr^WX$ with $A$ via $\phi$ the weight filtration on $X$ gives rise to two extensions
\begin{equation}\label{eq60}
\begin{tikzcd}[row sep= small]
0 \arrow[r] & A_1 \arrow[r] & W_{p_2}X \arrow[r] & A_2 \arrow[r] & 0\\
0 \arrow[r] & A_2 \arrow[r] & X/W_{p_1}X \arrow[r] & A_3 \arrow[r] & 0.
\end{tikzcd}
\end{equation}
We have a well-defined map
\begin{equation}\label{eq61}
S'(A) \rightarrow Ext^1(A_2,A_1)\times Ext^1(A_3,A_2)
\end{equation}
that sends the equivalence class of $(X,\phi)$ in $S'(A)$ to the pair of extension classes above given by $(W_{p_2}X,X/W_{p_1}X)$.\footnote{That this is well-defined can be checked directly here, but also follows from our later discussion: the map \eqref{eq61} is the composition 
\[S'(A)\cong S'_2(A)\xrightarrow{\Theta_2} S'_1(A)\cong Ext^1(A_2,A_1)\times Ext^1(A_3,A_2)\]
of \S \ref{sec: sets S_ell S'_ell and restatements of Thms A and B} for $k=3$.}

To study the fibers of \eqref{eq61}, note that each pair $(X,\phi)$ gives rise to a blended extension
\[
\begin{tikzcd}
   & & 0 \arrow{d} & 0 \arrow{d} &\\
   0 \arrow[r] & A_1 \ar[equal]{d} \arrow[r, ] & W_{p_2}X  \arrow[d] \arrow[r, ] &  A_2 \arrow{d} \arrow[r] & 0 \\
   0 \arrow[r] & A_1 \arrow[r] & X \arrow[d] \arrow[r] &  X/W_{p_1}X \arrow{d}  \arrow[r] & 0 \\
   & & A_3 \arrow{d} \ar[equal]{r} & A_3 \arrow{d} & \\
   & & 0 & 0. &   
\end{tikzcd} 
\]
Given an extension $\sN$ (resp. $\sL$) of $A_3$ by $A_2$ (resp. $A_2$ by $A_1$), if the equivalence class of $(X,\phi)$ is above the class of $(\sL,\sN)$, then there is a unique isomorphism in $EXT(A_2,A_1)\times$ $EXT(A_3,A_2)$
from the extensions of \eqref{eq60} to $(\sL,\sN)$ (unique because of \eqref{eq2}). Using this isomorphism we replace the top row and right column of the blended extension above with $\sL$ and $\sN$, respectively. Thus $(X,\phi)$ gives rise to an element of $EXTPAN(\sN,\sL)$. One can check that this induces a bijection between the fiber of \eqref{eq61} above the class of $(\sL,\sN)$ and $Extpan(\sN,\sL)$.

The image of \eqref{eq61} consists of the classes of extension pairs $(\sL,\sN)$ for which $Extpan(\sN,\sL)$ is nonempty. This is described using homological algebra as the kernel of the Yoneda pairing
\[
Ext^1(A_2,A_1)\times Ext^1(A_3,A_2) \rightarrow Ext^2(A_3,A_1) 
\]
(see Lemma \ref{lem: criteria for compatibility of extension pairs}).

We now move on to the structure of $S(A)$. The group $Aut(A)=\prod_r Aut(A_r)$ acts on 
\[
Ext^1(A_2,A_1)\times Ext^1(A_3,A_2)
\]
by pushforwards and pullbacks: $\sigma=(\sigma_{A_1},\sigma_{A_2},\sigma_{A_3})$ acts by sending 
\[
(\sL,\sN) \ \mapsto \ ({\sigma_1}_\ast{\sigma_2^{-1}}^\ast\sL, {\sigma_2}_\ast{\sigma_3^{-1}}^\ast \sN).
\]
The map \eqref{eq61} descends to a map
\begin{equation}\label{eq62}
S(A) \rightarrow (Ext^1(A_2,A_1)\times Ext^1(A_3,A_2))/Aut(A).
\end{equation}
This follows from Lemma 6.5.1 of \cite{EM2}, but it will also follow from the later discussion for arbitrary $k$. The map \eqref{eq62} sends the isomorphism class of an object $X$ to the $Aut(A)$-orbit of $Ext^1(A_2,A_1)\times$ $Ext^1(A_3,A_2)$ consisting of all pairs of extension classes in the aforementioned product to which $X$ is attached (again, see Lemma 6.5.1 of \cite{EM2}). 

Let us now consider the fibers of \eqref{eq62}. Let 
\[
(\sL,\sN)\in EXT(A_2,A_1)\times EXT(A_3,A_2).
\] 
Any pair $(\sL',\sN')$ of extensions whose class in 
\[
Ext^1(A_2,A_1)\times Ext^1(A_3,A_2)
\]
is in the $Aut(A)$-orbit of the class of $(\sL,\sN)$ has the exact same attached objects as $(\sL,\sN)$. Hence the fiber of \eqref{eq62} above the orbit of the class of $(\sL,\sN)$ is equal to the set of isomorphism classes of objects attached to $(\sL,\sN)$.

The problem of classifying isomorphism classes of objects attached to a fixed pair $(\sL,\sN)$ was studies in \S \ref{sec: blended extensions II}. We saw that the set of isomorphism classes of such objects is in a natural bijection with
\[
Extpan(\sN,\sL)/\Gamma(\sL,\sN).
\]
(notation as in \S \ref{sec: blended extensions II}). In the case that the classes of $\sL$ and $\sN$ are totally nonsplit, the action of $\Gamma(\sL,\sN)$ is trivial and the set of the isomorphism classes of objects attached to $(\sL,\sN)$ is just $Extpan(\sN,\sL)$, which is an $Ext^1(A_3,A_1)$-torsor whenever it is nonempty.

We summarize all of this in the following statement:
\begin{prop}\label{prop: objects with given gr, case with 3 weights}
Let $A=A_1\oplus A_2\oplus A_3$, with $A_1,A_2,A_3$ pure of strictly increasing weights. Then we have a map
\begin{equation}\label{eq65}
S'(A) \rightarrow Ext^1(A_2,A_1)\times Ext^1(A_3,A_2)
\end{equation}
such that the fiber above the class of any extension pair $(\sL,\sN)$ is in a natural bijection with $Extpan(\sN,\sL)$ and hence is either empty or an $Ext^1(A_3,A_1)$-torsor. The map \eqref{eq65} descends to a map
\begin{equation}\label{eq63}
S(A) \rightarrow (Ext^1(A_2,A_1)\times Ext^1(A_3,A_2))/Aut(A)
\end{equation}
with the property that the fiber above the orbit of the class of an extension pair $(\sL,\sN)$ is in a natural bijection with the quotient of $Extpan(\sN,\sL)$ by the action of $\Gamma(\sL,\sN)$. If we restrict the codomain of the latter map to the orbits of totally nonsplit extension pairs, then the fiber above the orbit of the class of $(\sL,\sN)$ is in a natural bijection with $Extpan(\sN,\sL)$ and hence either empty or an $Ext^1(A_3,A_1)$-torsor. Both maps above are surjective if $Ext^2(A_3,A_1)$ is trivial.
\end{prop}
We point out that with the proposition as stated, to see the torsor structure on the fiber of \eqref{eq65} above a pair of extension classes we first have to choose a representative $(\sL,\sN)$ in $EXT(A_2,A_1)$ $\times EXT(A_3,A_2)$. This dependence can be removed, as we will see in the general case of the result (see the proof of Theorem \ref{thm: classification of S'(A) in the general case} at the end of \S \ref{sec: fibers of truncations II}, also the final assertion in Lemma \ref{lem: compatibility of torsor structures}). To see the described structure on the fiber of \eqref{eq63} above an orbit of the action of $Aut(A)$ on 
\[Ext^1(A_2,A_1)\times Ext^1(A_3,A_2)\]
we first have to choose a representative of this orbit, and lift it to the level of extensions. The torsor structures that arise from different choices are however closely related to one another (in fact, in the totally nonsplit case, they are related in a canonical way). We will discuss this when we consider the general situation for arbitrary number of weights. See Proposition \ref{prop: general thm on S(A) part b}(b).

\subsection{Generalized extensions}\label{sec: gen exts, defn}
We start the proof of Theorems \ref{thm: classification of S'(A) in the general case} and \ref{thm: classification of objects with prescribed associated graded, general case} by introducing some relevant definitions. From this point on in this section, we fix $k\geq 3$ and
\[
A = \bigoplus_{1\leq r \leq k}A_r
\]
as in \S \ref{sec: statement of problems for S(A) and S'(A)}. That is, $A_r$ (for $1\leq r\leq k$) is a pure object of weight $p_r$ and $p_1<\cdots< p_k$. 

The following definition is modelled based on the diagram \eqref{eq gen ext for M} associated with a pair $(X, \phi)$ of an object $X$ of $\bT$ and an isomorphism $\phi: Gr^WX\rightarrow A$.
\begin{defn}[Generalized extensions of $A$] \label{defn: generalized extensions}~\\
\noindent (a) By a generalized extension (of level $k-1$)\footnote{The mention of the level here is because we will shortly give a truncated version of this notion. Until then, a generalized extension means one of level $k-1$, which is what this definition is for.} of $A$ we mean a collection of objects 
\[(X_{m,n})_{0\leq m<n\leq k}\]
of $\bT$ with $X_{r-1,r}=A_r$ for all $1\leq r\leq k$, together with a surjective morphism $X_{m,n}\rightarrow X_{m+1,n}$ and an injective morphism $X_{m,n-1}\rightarrow X_{m,n}$ for every $m,n$ in the eligible\footnote{Note that here and elsewhere throughout, by the adjective ``eligible" in the context of indices
we mean ``the range in which the indices in question make sense". So here, for instance, we have an injective map $X_{m,n-1}\rightarrow X_{m,n}$ for every pair of integers $(m,n)$ with $0\leq m<n\leq k$ and $m<n-1$.} range, such that the following axioms hold:
\begin{itemize}
\item[(i)] Every diagram of the form
\begin{equation}\label{eq9}
\begin{tikzcd}
X_{m,n-1} \arrow[d, hookrightarrow] \arrow[r, twoheadrightarrow] & X_{m+1,n-1} \arrow[d, hookrightarrow] \\
X_{m,n} \arrow[r, twoheadrightarrow] & X_{m+1,n} 
\end{tikzcd}
\end{equation}
(with the maps as in the given data) commutes.
\item[(ii)] The diagram
\begin{equation}\label{eq11}
\begin{tikzcd}
   0 \arrow[r] &  X_{m,n-1}\arrow[r, ] & X_{m,n} \arrow[r, ] &  A_n  \arrow[r] & 0
\end{tikzcd}
\end{equation}
is an exact sequence for every $m,n$ in the eligible range. 
\end{itemize}
In (ii),  the morphism $X_{m,n}\rightarrow A_n$ is the composition
\[
X_{m,n} \twoheadrightarrow X_{m+1,n} \twoheadrightarrow X_{m+2,n} \twoheadrightarrow \cdots \twoheadrightarrow X_{n-1,n}=A_n.  
\]
\noindent (b) The collection of all generalized extensions (of level $k-1$) of $A$ is denoted by $D_{k-1}(A)$.
\end{defn}
The reason for including axiom (ii) in the definition is to make sure that the $X_{m,n}$ cannot be larger than what we like. With the exact sequences (ii) included as a requirement, one is guaranteed to also get exact sequences
\[
\begin{tikzcd}
   0 \arrow[r] &  A_{m+1}\arrow[r, ] & X_{m,n} \arrow[r, ] &  X_{m+1,n}  \arrow[r] & 0.
\end{tikzcd}
\]
(See Lemma \ref{lem: weight filtration of objects in generalized extensions}.)

Given an object $X$ of $\bT$ whose associated graded is isomorphic to $A$, choosing an isomorphism $\phi:Gr^WX\rightarrow A$ to identify the two, the subquotients $(X_n/X_m)_{0\leq m<n\leq k}$ with $X_r=W_{p_r}X$ together with the natural successive inclusion and projection maps between them form a generalized extension of $A$. We call this the generalized extension of $A$ associated with $(X, \phi)$ and denote it by $ext(X,\phi)$.

In general, a generalized extension of $A$ can be visualized by a diagram as in \eqref{eq gen ext for M}. We will simply speak of ``a generalized extension $(X_{m,n})_{0\leq m<n\leq k}$, or often merely $(X_{\bullet, \bullet})$, without including the arrows in the notation. For simplicity and to save space we may sometimes drop the arrows even from our diagrams.

Note that while the definition of a generalized extension $(X_\db)$ only includes maps between objects in adjacent position in the diagram, for every pairs $(m,n)$ and $(m',n')$ with $m'\geq m$ and $n'\geq n$
by composing the morphisms along any path from $X_{m,n}$ to $X_{m',n'}$ we get a map 
\[
X_{m,n} \rightarrow X_{m',n'}.
\] 
Commutativity of the diagram for $(X_\dbullet)$ guarantees that the outcome does not depend on the choice of the path.

We also need to discuss truncated generalizations of the notion, which only include the data of the top several diagonals of a full diagram:

\begin{defn}[Generalized extensions of level $\ell$ of $A$]\label{def: gen ext level l} ~ \\
\noindent (a) Let $0\leq \ell\leq k-1$. By a generalized extension of level\footnote{The enumeration of the level is chosen so that generalized extensions of level 1 are closely related to 1-extensions; see the rest of the subsection.} $\ell$ of $A$ we mean the data of an object $(X_{m,n})$ of $\bT$ for each pair $(m,n)$ of integers with $0\leq m<n\leq k$ and $n-m\leq \ell+1$, with $X_{r-1,r}=A_r$ for all $1\leq r\leq k$, together with the data of a surjective morphism $X_{m,n}\rightarrow X_{m+1,n}$ and an injective morphism $X_{m,n-1}\rightarrow X_{m,n}$ for every $m$ and $n$ in the eligible range such that axioms (i) and (ii) of Definition \ref{defn: generalized extensions}(a) hold.

\noindent (b) The collection of all generalized extensions of level $\ell$ of $A$ is denoted by $D_\ell(A)$. For convenience, we set $D_{k'}(A)=D_{k-1}(A)$ for $k'\geq k$.
\end{defn}
For instance, $D_0(A)$ is a singleton, with the only element being the tuple $(A_r)_{1\leq r\leq k}$. The data does not include any morphisms. A generalized extension of level 1 of $A$ can be visualized as a diagram of the form
\[
\begin{tikzcd}[column sep=0in, row sep=0in]
A_1         & & & &&& \\
X_{0,2} & A_2  & & & &&\\
  & X_{1,3}  & A_3 & & &&\\
  &  & \ddots &\ddots &&\\
&&&&&\\
   && ~ & \hspace{-.2in} X_{k-3,k-1}  &\hspace{-.2in}A_{k-1} &\\
    & & ~ &  & \hspace{-.2in} X_{k-2,k}  &\hspace{-.1in} A_k 
\end{tikzcd}
\]
where the arrows, dropped from the writing for convenience, satisfy axiom (ii) of the definition. This is simply the data of $k-1$ extensions 
\[
\begin{tikzcd}
0 \arrow[r] & A_r \arrow[r] & X_{r-1,r+1} \arrow[r] & A_{r+1} \arrow[r] &  0
\end{tikzcd}
\hspace{.3in}  (1\leq r\leq k-1).
\]
Note that these are simply short exact sequences, rather than elements of $Ext^1$ (since we have not yet introduced any equivalence relations on $D_1(A)$).

We make each $D_\ell(A)$ the collection of objects of a category by defining the notion of morphisms of generalized extensions as follows: Let $(X_\dbullet)$ and $(X'_\dbullet)$ be generalized extensions of $A$ of the same level. A morphism of generalized extensions $(X_\dbullet)\rightarrow (X'_\dbullet)$ is a collection of morphisms $f_{m,n}: X_{m,n}\rightarrow  X'_{m,n}$ (one for each pair $(m,n)$ in the eligible range) that
commute with the structure morphisms of $(X_\dbullet)$ and $(X'_\dbullet)$; that is, such that each diagram below commutes for all eligible $(m,n)$:
\begin{equation}\label{eq diagrams for morphisms of gen exts}
\begin{tikzcd}
X_{m,n} \arrow[d, "f_{m,n}"] \arrow[r, twoheadrightarrow] & X_{m+1,n} \arrow[d, "f_{m+1,n}"]\\
X'_{m,n} \arrow[r, twoheadrightarrow] & X'_{m+1,n}
\end{tikzcd}
\hspace{.3in}
\begin{tikzcd}
X_{m,n-1} \arrow[d, "f_{m,n-1}"] \arrow[r, hookrightarrow] & X_{m,n} \arrow[d, "f_{m,n}"]\\
X'_{m,n-1} \arrow[r, hookrightarrow] & X'_{m,n}
\end{tikzcd}
\end{equation}
Note here that the morphisms 
\[
f_{r-1,r}: A_r\rightarrow A_r 
\]
are arbitrary. With abuse of notation, the category of generalized extensions of level $\ell$ of $A$ is also denoted by $D_\ell(A)$. A morphism $(f_\db)$ in $D_\ell(A)$ is an isomorphism if and only if all the $f_{m,n}$ are isomorphisms. (We will soon see that $(f_\db)$ is an isomorphism if and only if every $f_{r-1,r}:A_r\rightarrow A_r$ is an isomorphism.)

For every object $X$ of $\bT$ and isomorphism $\phi:Gr^WX\rightarrow A$, the association
\[
(X,\phi) \mapsto \text{the generalized extension $ext(X,\phi)$}
\]
(where $ext(X,\phi)$ is the generalized extension \eqref{eq gen ext for M} with $X_{m,n}=W_{p_n}X/W_{p_m}X$) is functorial in the pair $(X,\phi)$. 

\noindent \underline{Truncation and restriction functors}: One can naturally define two types of functors between categories of generalized extensions. The first are the (diagonal) truncation functors: for $1\leq \ell\leq k-1$,
\[
\Theta_\ell: D_\ell(A) \rightarrow D_{\ell-1}(A)
\]
merely erases the lowest diagonal of a generalized extension of level $\ell$ (together with the arrows going to and coming from those diagonals). (Its action on morphisms of generalized extensions is similar.)

The second are restriction functors to certain subobjects of $A$. Given integers $i$ and $j$ with $1\leq i< j\leq k$, we have a functor
\begin{equation}\label{eq8}
D_\ell(A) \rightarrow D_\ell(\bigoplus\limits_{i< r\leq j} A_r)
\end{equation}
which only keeps the part of a diagram (and the morphisms between diagrams) that lies in the intersection of the columns between $A_{r+1}$ and $A_s$ inclusively, and the rows between $A_{r+1}$ and $A_s$ inclusively.

\subsection{Basic properties}\label{sec: basic properties of gen exts}
In this section we gather some basic results about generalized extensions which will be used in the remainder of the paper. Throughout, unless otherwise indicated, a generalized extension means a generalized extension of $A$ (with $A$ as introduced earlier in \S \ref{sec: gen exts, defn}). We visualize a generalized extension by a diagram of the form \eqref{eq gen ext for M} (with general objects $X_\db$ and arrows that form the data of a generalized extension) or truncated versions of it if the level is $<k-1$. The references in the text to ``the lowest diagonal", ``the above and right of an object", ``entry $(m,n)$", etc. all refer to this visualization (with the object at entry $(m,n)$ of $(X_\db)$ being $X_{m,n}$).

For any generalized extension $(X_\db)$ of any level, for convenience we set $X_{r,r}=0$ for $0\leq r\leq k$. To simplify the writing, as before, if there is no ambiguity we will often simply refer to ``indices in the eligible range" or ``eligible indices"; this simply means that the indices are in the range for which the objects in the equation are available.
\begin{lemma}\label{lem: weight filtration of objects in generalized extensions}
Let $0\leq \ell\leq k-1$. Let $(X_\dbullet)$ be a generalized extension of level $\ell$.\\

\noindent (a) For every $m<r\leq n$ in the eligible range (depending on the level), we have
\[
Im(X_{m,r}\hookrightarrow X_{m,n}) = W_{p_r}X_{m,n}.
\]
That is, the weight filtration for each object of the diagram is given by the objects directly above it. (Note that in particular, the statement asserts that $W_{p_n}X_{m,n}=X_{m,n}$.)

\noindent (b) The isomorphisms
\begin{equation}\label{eq14}
Gr^WX_{m,n} \cong \bigoplus\limits_{m<r\leq n} A_r
\end{equation}
given by 
\[
Gr^W(X_{m,n}) \stackrel{\text{(a)}}{\cong} \bigoplus\limits_{m<r\leq n} \frac{X_{m,r}}{X_{m,r-1}} \stackrel{\eqref{eq11}}{\cong} \bigoplus\limits_{m<r\leq n} A_r
\]
for every $(m,n)$ in the eligible range are compatible with the injective and surjective maps between the $X_{m,n}$. That is, we have commutative diagrams
\begin{equation}\label{eq12}
\begin{tikzcd}
Gr^WX_{m,n-1} \ar[equal]{d} 
\arrow[r, hookrightarrow] & Gr^WX_{m,n} \ar[equal]{d}\\
\bigoplus\limits_{m<r\leq n-1} A_r \arrow[r, hookrightarrow] & \bigoplus\limits_{m<r\leq n} A_r
\end{tikzcd}
\hspace{.2in} \text{and} \hspace{.2in} \begin{tikzcd}
Gr^WX_{m,n} \ar[equal]{d} 
\arrow[r, twoheadrightarrow] & Gr^WX_{m+1,n} \ar[equal]{d}\\
\bigoplus\limits_{m<r\leq n} A_r \arrow[r, twoheadrightarrow] & \bigoplus\limits_{m+1<r\leq n} A_r
\end{tikzcd}
\end{equation}
in which the top arrows are $Gr^W$ applied to the structure arrows of $(X_\dbullet)$ and the bottom arrows 
are the natural embedding and projection maps. The identifications shown as equality in the diagrams are given by the isomorphisms \eqref{eq14}.

\noindent (c) For each $m,n$ in the eligible range, we have an exact sequence
\begin{equation}\label{eq10}
\begin{tikzcd}
   0 \arrow[r] &  A_{m+1}\arrow[r, ] & X_{m,n} \arrow[r, ] &  X_{m+1,n}  \arrow[r] & 0,
\end{tikzcd}
\end{equation}
where the morphism $A_{m+1}\rightarrow X_{m,n}$ is the composition
\[
A_{m+1}=X_{m,m+1}\hookrightarrow X_{m,m+2} \hookrightarrow X_{m,m+3} \hookrightarrow \cdots \hookrightarrow X_{m,n}.
\]

\noindent (d) Let $(f_\dbullet) :(X_\dbullet) \rightarrow (X'_\dbullet)$ be a morphism in $D_\ell(A)$. For each eligible $(m,n)$, the canonical isomorphisms \eqref{eq14} for $(X_\dbullet)$ and $(X'_\dbullet)$ fit into a commutative diagram
\[
\begin{tikzcd}
Gr^WX_{m,n} \ar[equal]{r} \arrow[d, "Gr^Wf_{m,n}"]& \bigoplus\limits_{m<r\leq n} A_r \arrow[d, "(f_{r-1,r})"] \\
Gr^WX'_{m,n} \ar[equal]{r} & \bigoplus\limits_{m<r\leq n} A_r.
\end{tikzcd}
\]
\end{lemma}
\begin{proof}
(a) Fixing $m$, this is seen by induction on $n$ in view of the extension \eqref{eq11}. In the induction step, we first apply the exact functor $W_{p_{n-1}}$ to \eqref{eq11}. Since $W_{p_{n-1}}A_n=0$ we get 
the assertion for $r\leq n-1$. As for when $r=n$, this follows from exactness of $W_{p_n}$ and the fact that $W_{p_n}A_n=A_n$.

%\noindent (b) and (c): That the diagrams for the injective arrows (on the left of \eqref{eq12}) commute is by the construction of the canonical isomorphisms. Jumping to part (c) and leaving the commutativity of the diagrams for the surjective arrows to the end, the exactness of \eqref{eq10} is clear at the first and third object. As for at the middle, from part (a) we know that $W_{p_{m+1}}X_{m+1,n}=0$ and $W_{p_{m+1}}X_{m,n}=A_{m+1}$. Applying $W_{p_{m+1}}$ to $X_{m,n}\twoheadrightarrow X_{m+1,n}$ we see that $A_{m+1}$ is in the kernel of $X_{m,n}\twoheadrightarrow X_{m+1,n}$. On the other hand, by \eqref{eq14} the dimension of the kernel of $X_{m,n}\twoheadrightarrow X_{m+1,n}$ is equal to the dimension of $A_{m+1}$. Thus part (c) is established. We leave the commutativity of the right diagram of \eqref{eq12} to the reader.
\noindent (b) This follows from the construction of the canonical isomorphisms and the commutativity of the diagram of a generalized extension. We leave the details to the reader.

\noindent (c) The exactness of \eqref{eq10} is clear at the first and third object. As for at the middle, from part (a) we know that $W_{p_{m+1}}X_{m+1,n}=0$ and $W_{p_{m+1}}X_{m,n}=A_{m+1}$. Applying $W_{p_{m+1}}$ to $X_{m,n}\twoheadrightarrow X_{m+1,n}$ we see that $A_{m+1}$ is in the kernel of $X_{m,n}\twoheadrightarrow X_{m+1,n}$. On the other hand, by \eqref{eq14} the dimension of the kernel of $X_{m,n}\twoheadrightarrow X_{m+1,n}$ is equal to the dimension of $A_{m+1}$.

\noindent (d) Let $m<r\leq n$. By definition of the canonical isomorphism \eqref{eq14} we have a commutative diagram
\[
\begin{tikzcd}[row sep=small]
 & A_r \arrow[dl, twoheadleftarrow] \ar[equal]{dr} &  \\
X_{m,r}  \arrow[rr, twoheadrightarrow] &  & Gr^W_{p_r}X_{m,n} 
\end{tikzcd}
\]
where the horizontal surjective arrow is given by $X_{m,r}\hookrightarrow X_{m,n}$ (mapping $X_{m,r}$ isomorphically to $W_{p_r}X_{m,n}$) and then passing to $Gr^W_{p_r}X_{m,n}$, and the slant surjective arrow is the composition of the surjective arrows $X_{m',r}\twoheadrightarrow X_{m'+1,r}$. The side donated by equality is the identification of \eqref{eq14}. There is an analogous triangle for $(X'_\dbullet)$. The two triangle can be put into a (to be seen to be commutative) diagram:
\begin{equation}\label{eq15}
\begin{tikzcd}[row sep=small]
 & A_r \arrow[dl, twoheadleftarrow] \ar[equal]{dr}  &  \\
X_{m,r}  \arrow[rr, twoheadrightarrow] \arrow[dd, "f_{m,r}"] &  & Gr^W_{p_r}X_{m,n} \arrow[dd, "Gr^W_{p_r}f_{m,n}"] \\
 & A_r \arrow[dl, twoheadleftarrow] \ar[equal]{dr} \arrow[from=uu, crossing over] &  \\
X'_{m,r}  \arrow[rr, twoheadrightarrow] &  & Gr^W_{p_r}X'_{m,n} 
\end{tikzcd}
\end{equation}
where the map $A_r\rightarrow A_r$ on the top is $f_{r-1,r}$. The front and back (triangular) faces are commutative. The rectangular faces on the top left and the bottom are both commutative, the former (resp. latter) by compatibility of morphisms of generalized extensions with the surjective (resp. injective) structure arrows. It follows that the top right face is also commutative.
\end{proof}
Note that in particular, the previous lemma asserts that for every generalized extension $(X_\dbullet)$ of level $k-1$ of $A$ we have 
\[
Gr^WX_{0,n} \cong \bigoplus\limits_{1\leq r\leq n} A_r
\]
for all $0\leq n\leq k$. 

Before we proceed any further, let us introduce a notation: 

\begin{notation}\label{notation: horizontal and vertical extensions}
Given a generalized extension $(X_\dbullet)$ of $A$ of any level, we denote the two extensions \eqref{eq11} and \eqref{eq10} respectively by $\sX_{m,n}^v$ and $\sX_{m,n}^h$.
\end{notation}

The following two lemmas will be useful in constructing morphisms between generalized extensions $(X_\dbullet)$ and $(X'_\dbullet)$. The first lemma asserts that every morphism between the objects of $(X_\dbullet)$ and $(X'_\dbullet)$ at the same entry extends uniquely to the part of the diagrams to the above and right of that entry.
\begin{lemma} \label{lem: morphisms spread to the top right}
Let $(X_\dbullet)$ and $(X'_\dbullet)$ be generalized extensions of level $\ell$ of $A$. 

\noindent (a) Then for any $i,j$ in the eligible range, every morphism $f: X_{i,j}\rightarrow  X'_{i,j}$ extends uniquely to a collection of morphisms 
\[
f_{m,n}: X_{m,n}\rightarrow  X'_{m,n} \hspace{.3in} (i\leq m<n\leq j)
\]
such that $f_{i,j}=f$ and the $f_{m,n}$ commute with the morphisms in $(X_\dbullet)$ and $(X'_\dbullet)$. (That is, such that the $f_{m,n}$ give a morphism between the restrictions of $(X_\dbullet)$ and $(X'_\dbullet)$ to the part between $A_{i+1}$ and $A_j$. See \eqref{eq8}.) 

\noindent (b) The extension of $f$ to $(f_{m,n})$ (as above) behaves well with respect to compositions: if $(X''_\dbullet)$ is also a generalized extension of level $\ell$ of $A$ and $f': X'_{r,s}\rightarrow  X''_{r,s}$ is a morphism, then
\[
(f'\circ f)_{m,n} = f'_{m,n}\circ f_{m,n}
\]
for all eligible $m,n$.
\end{lemma}
\begin{proof}
We may assume that the level is $k-1$ and $(i,j)=(0,k)$, as the seemingly more general statements follow from applying this case to the restrictions of our generalized extensions (restriction in the sense of \eqref{eq8}).

By the previous lemma, for any $n\leq k$ we have $W_{p_n}X_{0,k}\cong X_{0,n}$ and $W_{p_n}X'_{0,k}\cong X'_{0,n}$ (identifications via the injective structure arrows of the generalized extensions). By functoriality of the weight filtration, the morphism $f$ restricts to morphisms $f_{0,n}:X_{0,n}\rightarrow X'_{0,n}$ compatible with each other. So far, we have extended $f$ to the first column of the two generalized extensions.

Assume $f$ has been extended in the desired way to the $f_{m,n}$ for all $m<m'$. For each eligible $n$ we have a commutative diagram
\[
\begin{tikzcd}
  0 \arrow[r] &  A_{m'}\arrow[r, ] \arrow[d, "f_{m'-1,m'}"]& X_{m'-1,n} \arrow[r, ] \arrow[d, "f_{m'-1,n}"] &  X_{m',n}  \arrow[r] & 0\\
  0 \arrow[r] &  A_{m'}\arrow[r, ] & X'_{m'-1,n} \arrow[r, ] &  X'_{m',n}  \arrow[r] & 0
\end{tikzcd}
\]
with rows $\sX_{m'-1,n}^h$ and ${\sX'}_{m'-1,n}^h$. We thus get a morphism $f_{m',n}: X_{m',n} \rightarrow X'_{m',n}$ making this diagram commute.  

We have extended $f$ to the column of entries $(m',n)$ (with $m'$ fixed), and the commutativity with the surjective arrows from the column for $m'-1$ to the one for $m'$ is known (by construction). We have a diagram
\[
\begin{tikzcd}[row sep=small, column sep=small]
& X_{m'-1,n-1}  \arrow[rr, twoheadrightarrow] \arrow[dd] & & X_{m',n-1}  \arrow[dd, "f"] \\ 
X_{m'-1,n} \arrow[ur, hookleftarrow] \arrow[rr, crossing over, twoheadrightarrow] \arrow[dd, "f"] & & X_{m',n} \arrow[ur, hookleftarrow] \\
& X'_{m'-1,n-1}  \arrow[rr, twoheadrightarrow] & & X'_{m',n-1} \\
X'_{m'-1,n} \arrow[ur, hookleftarrow] \arrow[rr, twoheadrightarrow ] & & X'_{m',n} \arrow[from=uu, crossing over] \arrow[ur, hookleftarrow]\\
\end{tikzcd}
\]
where the top (resp. bottom) face is a part of $(X_\dbullet)$ (resp. $(X'_\dbullet)$), and the downward maps are the maps induced by $f$. We know the top and bottom faces as well as the left, back and front faces are commutative. In view of the surjectivity of $X_{m'-1,n-1}\twoheadrightarrow X_{m',n-1}$ we get the commutativity of the right face. This completes the proof of the fact that $f$ extends to a morphism of generalized extensions.

As for uniqueness, the map $f_{0,k}$ determines every map $f_{m,n}$ with $0\leq m<n\leq k$ because of the commutativity requirements. Indeed, $f_{0,k}$ determines each $f_{0,n}$ and in turn, $f_{0,n}$ determines each $f_{m,n}$ by the commutativity of the following two diagrams:
\[
\begin{tikzcd}
X_{0,n} \arrow[r, hookrightarrow] \arrow[d, "f_{0,n}"] &  X_{0,k}  \arrow[d, "f_{0,k}"]  \\
X'_{0,n} \arrow[r, hookrightarrow] &  X'_{0,k}  
\end{tikzcd}
\hspace{.3in}
\begin{tikzcd}
X_{0,n} \arrow[r, twoheadrightarrow] \arrow[d, "f_{0,n}"] &  X_{m,n}  \arrow[d, "f_{m,n}"]  \\
X'_{0,n} \arrow[r, twoheadrightarrow] &  X'_{m,n}  
\end{tikzcd}
\]
(Alternatively, we can first determine each $f_{m,k}$ from $f_{0,k}$ and then each $f_{m,n}$ from $f_{m,k}$.) 

Part (b) is easily seen from the construction of the $f_{m,n}$ given above.
\end{proof}
The next lemma allows us to ``glue" morphisms between the lowest diagonals of two generalized extensions to a morphism between the two generalized extensions.
\begin{lemma}\label{lem: gluing morphisms between objects on the lowest diagonal}
Suppose $(X_\dbullet)$ and $(X'_\dbullet)$ are generalized extensions of level $\ell$ of $A$. Suppose that for each eligible pair $(i,j)$ with $j-i=\ell+1$ (i.e. on the lowest diagonal) we have a morphism $f_{i,j}: X_{i,j}\rightarrow X'_{i,j}$. For each $(i,j)$ with $j-i=\ell+1$, let $f^v_{i,j-1}$ and $f^h_{i+1,j}$ be the unique morphisms\footnote{That these exist is by Lemma \ref{lem: weight filtration of objects in generalized extensions} and functoriality of the weight filtration.} fitting in the commutative diagrams:
\[
\begin{tikzcd}
X_{i,j-1} \arrow[d, "f^v_{i,j-1}"] \arrow[r, hookrightarrow] & X_{i,j} \arrow[d, "f_{i,j}"]\\
X'_{i,j-1} \arrow[r, hookrightarrow] & X'_{i,j}
\end{tikzcd}
\hspace{.3in}
\begin{tikzcd}
X_{i,j} \arrow[d, "f_{i,j}"] \arrow[r, twoheadrightarrow] & X_{i+1,j} \arrow[d, "f^h_{i+1,j}"]\\
X'_{i,j} \arrow[r, twoheadrightarrow] & X'_{i+1,j}
\end{tikzcd}
\]
Then there exists a morphism of generalized extensions $(X_\dbullet)\rightarrow (X'_\dbullet)$ extending the given $f_{i,j}$ from the lowest diagonals to the full diagrams if and only if for every eligible $(m,n)$ with $n-m=\ell$ (i.e. on the diagonal just above the lowest), we have $f^v_{m,n}=f^h_{m,n}$. Moreover, when such an extension exists, it is unique. (In other words, if two morphisms of generalized extensions agree on the lowest diagonal, then the two morphisms are the same.)
\end{lemma} 
\begin{proof}
The uniqueness is immediate from the uniqueness statement in Lemma \ref{lem: morphisms spread to the top right}. That the given condition is necessary follows from the same result and its proof. The new assertion here is that the compatibility condition on the diagonal just above the lowest is sufficient for the morphisms between the lowest diagonals to glue (or extend) to make a morphism of generalized extensions. 

By Lemma \ref{lem: morphisms spread to the top right} the morphism $f_{0,\ell+1}$ extends to the right and top of entry $(0,\ell+1)$. Assume that the morphisms on the lowest diagonal glue all the way up to and including the entry $(i,j)$ on the lowest diagonal, so that for all eligible pairs $(m,n)$ of indices with $n\leq j$ we already have morphisms $f_{m,n}: X_{m,n}\rightarrow X'_{m,n}$ commuting with the structure injections and surjections. We will argue that if 
$j\neq k$, we can also glue $f_{i+1,j+1}$ to the current data.

Consider the map $f_{i+1,j}: X_{i+1,j}\rightarrow X'_{i+1,j}$ (which is already available). Then $f_{i+1,j}$ is induced by $f_{i,j}$ and hence is $f^h_{i+1,j}$. By Lemma \ref{lem: morphisms spread to the top right}, $f_{i+1,j+1}$ extends uniquely to the objects upward and to the right of entry $(i+1,j+1)$, in particular, inducing the map $f^v_{i+1,j}$ on $X_{i+1,j}$. By the compatibility condition, $f^h_{i+1,j}=f^v_{i+1,j}$. Applying the uniqueness statement of Lemma \ref{lem: morphisms spread to the top right} at entry $(i+1,j)$ it follows that the map induced by $f_{i+1,j+1}$ at every entry $(m,n)$ with $m\geq i+1$ and $n\leq j$ coincides with the map $f_{m,n}$ already there. Together with the new maps $f_{m,j+1}$ for $m<j+1$ induced by $f_{i+1,j+1}$ we have extended the maps between the diagrams one row further down.

There is nothing new to check for the commutativity with the structure injections and surjections: all squares that need to commute are already known to commute.
\end{proof}

\subsection{Equivalence relations on generalized extensions}\label{sec: equiv rels on gen exts}
Recall that for each integer $\ell$ with $0\leq \ell\leq k-1$ the collection (as well as the category) of generalized extensions of level $\ell$ of $A$ is denoted by $D_\ell(A)$ (as before, we may drop the phrase ``of $A$" from the writing). There are two natural equivalence relations on each $D_\ell(A)$. The first is simply given by isomorphisms in the category $D_\ell(A)$, and the second is the finer equivalence given by isomorphisms that are identity on $A$:
\begin{notation}
(a) We write $(X_\dbullet) \sim (X'_\dbullet)$ (or $(X_\dbullet)\simeq (X'_\dbullet)$) if there exists an isomorphism of generalized extensions $(X_\dbullet) \rightarrow (X'_\dbullet)$. That is, if there an isomorphism $f_{m,n}: X_{m,n}\rightarrow X'_{m,n}$ for each pair $(m,n)$  in the eligible range such that the diagrams \eqref{eq diagrams for morphisms of gen exts} commute. 

\noindent (b) We write $(X_\dbullet) \sim' (X'_\dbullet)$ if there exists an isomorphism $(f_\db): (X_\dbullet) \rightarrow (X'_\dbullet)$ that is identity on $A$; that is, such that for every $r$ the isomorphism $f_{r-1,r}: A_r\rightarrow A_r$ is the identity map. 

\noindent (c) Denote the set of equivalence classes of objects of $D_\ell(A)$ with respect to $\sim$ (resp. $\sim'$) by $\oD_\ell(A)$ (resp. $\oD'_\ell(A)$). (Both $\sim$ and $\sim'$ are indeed equivalence relations on $D_\ell(A)$.)
\end{notation}
Before proceeding any further, let us make an observation about the equivalence relation $\sim'$:
\begin{lemma}\label{lem: if identity on A identity everywhere}
Let $(X_\dbullet)$ and $(X'_\dbullet)$ be generalized extensions of level $\ell$ of $A$. If $(X_\dbullet) \sim' (X'_\dbullet)$, then there is only one isomorphism $(X_\dbullet) \rightarrow (X'_\dbullet)$ that is identity on $A$.
\end{lemma}
\begin{proof}
The statement is equivalent to the statement that for every generalized extension $(X_\dbullet)$ of $A$ of level $\ell$ the only automorphism of $(X_\dbullet)$ that is identity on $A$ is the identity map. Suppose $(f_\dbullet)$ is such an automorphism. Taking $(X'_\dbullet)=(X_\dbullet)$ in Lemma \ref{lem: weight filtration of objects in generalized extensions}(d) we obtain that each $Gr^Wf_{m,n}$ and hence $f_{m,n}$ is identity.%\footnote{At first glance, one might think that the assumption that the $A_i$ are in a strictly increasing order of weights (which is the real reason for the statement of the lemma to be true) is strangely not playing a part. Of course, that assumption was already crucially used in characterization of the weight filtration on the $X_{m,n}$ in Lemma \ref{lem: weight filtration of objects in generalized extensions}.}
\end{proof} 
The group $Aut(A)$ acts on $D_\ell(A)$ by twisting the arrows to and from the $A_r$: making the action a left action as usual, $\sigma=(\sigma_r)$ in $Aut(A)=\prod_r Aut(A_r)$ sends the diagram below on the left to the one on the right. Note that the rest of the arrows in the diagram remain unchanged.
\begin{equation}\label{eq17}
\begin{tikzcd}[column sep=small, row sep=small]
A_1 \arrow[d, hookrightarrow, "j_1" ']   & & & &&& \\
X_{0,2} \arrow[r, twoheadrightarrow, "\omega_2"] & A_2 \arrow[d, hookrightarrow, "j_2" '] & & & &&\\
\vdots  & X_{1,3} \arrow[r, twoheadrightarrow, "\omega_3"]  & A_3 \arrow[d, hookrightarrow, "j_3" '] & & &&\\
  &   & \ddots &\ddots &&\\
   && ~ & \hspace{-.2in} X_{k-3,k-1} \arrow[r, twoheadrightarrow, "\omega_{k-1}"]  & A_{k-1} \arrow[d, hookrightarrow, "j_{k-1}" ']&\\
    & &  &\cdots  & \hspace{-.2in} X_{k-2,k} \arrow[r, twoheadrightarrow, "\omega_{k}"] & A_k 
\end{tikzcd}
\hspace{-1.5in}
\begin{tikzcd}[column sep=0.3in, row sep=small]
A_1 \arrow[d, hookrightarrow, "j_1\sigma_1^{-1}" ']   & & & &&& \\
X_{0,2} \arrow[r, twoheadrightarrow, "\sigma_2\omega_2" ] & A_2 \arrow[d, hookrightarrow, "j_2\sigma_2^{-1}" '] & & & &&\\
\vdots  & X_{1,3} \arrow[r, twoheadrightarrow, "\sigma_3\omega_3"]  & A_3 \arrow[d, hookrightarrow, "j_3\sigma_3^{-1}" '] & & &&\\
  &   & \ddots &\ddots &&\\
   && ~ & \hspace{-.2in} X_{k-3,k-1} \arrow[r, twoheadrightarrow, "\sigma_{k-1}\omega_{k-1}"]  & A_{k-1} \arrow[d, hookrightarrow, "j_{k-1}\sigma_{k-1}^{-1}" ']&\\
    & &  & \cdots & \hspace{-.2in} X_{k-2,k} \arrow[r, twoheadrightarrow, "\sigma_{k}\omega_{k}"] & A_k 
\end{tikzcd}
\end{equation}
We use the notation $\sigma\cdot (X_\db)$ for the image of $(X_\db)$ under $\sigma\in Aut(A)$. If $(f_\dbullet): (X_\dbullet)\rightarrow (X'_\dbullet)$ is an isomorphism that is identity on $A$, then $(f_\dbullet)$ is also such an isomorphism $\sigma\cdot (X_\dbullet)\rightarrow \sigma\cdot (X'_\dbullet)$. Indeed, if either of the two diagrams
\[
\begin{tikzcd}
A_r \ar[equal]{d} \arrow[r, hookrightarrow, "j_r"] & X_{r-1,r+1} \arrow[d, "f_{r-1,r+1}"]\\
A_r \arrow[r, hookrightarrow, "j'_r"] & X'_{r-1,r+1}
\end{tikzcd}
\hspace{.3in}\text{or}
\hspace{.3in}
\begin{tikzcd}
X_{r-2,r} \arrow[d, "f_{r-2,r}"] \arrow[r, twoheadrightarrow, "\omega_r"] & A_r \ar[equal]{d}\\
X'_{r-2,r} \arrow[r, twoheadrightarrow, "\omega'_r"] & A_r
\end{tikzcd}
\]
commutes, then so it does after twisting the horizontal arrows by $\sigma_r$ (or $\sigma_r^{-1}$) in $Aut(A_r)$. This shows the first assertion of the following lemma:
\begin{lemma}\label{lem: relation betweem oD and oD'}
The action of $Aut(A)$ on $D_\ell(A)$ descends to an action on $\oD'_\ell(A)$. Moreover, the natural surjection
\[
\oD'_\ell(A) \twoheadrightarrow \oD_\ell(A)
\]
descending from the identity map on $D_\ell(A)$ descends further to a bijection
\[
\oD'_\ell(A)/Aut(A) \cong \oD_\ell(A).
\]
\end{lemma}
\begin{proof} We just need to prove the second assertion. We will show that two generalized extensions $(X_\db)$ and $(X'_\db)$ are $\sim$-equivalent (i.e. isomorphic) if and only if there exists $\sigma\in Aut(A)$ such that $\sigma\cdot (X_\db)\sim' (X'_\db)$.

Suppose $(f_\db): (X_\db) \rightarrow (X'_\db)$ is an isomorphism. Then $\sigma_r:=f_{r-1,r}$ is an automorphism of $A_r$. For each eligible pair $(m,n)$, let $g_{m,n}=f_{m,n}$ if $n-m>1$. Let $g_{r-1,r}$ be the idenity on $A_r$. Set $\sigma=(\sigma_r)\in Aut(A)$. Then $(g_\db): \sigma\cdot (X_\db) \rightarrow (X'_\db)$ is an isomorphism of generalized extensions that is identity on $A$. Thus $\sigma\cdot (X_\db)\sim'  (X'_\db)$.

Conversely, assume $\sigma\cdot (X_\db)\sim'  (X'_\db)$. Let $(g_\db): \sigma\cdot (X_\db) \rightarrow (X'_\db)$ be an isomorphism that is identity on $A$. Then $f_\dbullet: (X_\dbullet) \rightarrow (X'_\dbullet)$ defined by $f_{m,n}=g_{m,n}$ if $n-m>1$ and $f_{r-1,r}=\sigma_r$ is an isomorphism.
\end{proof}
The rest of this subsection studies the sets $\oD_\ell(A)$ and $\oD'_\ell(A)$ when $\ell$ is 1 or $k-1$. We start by recalling the action of
\[
Aut(A) = \prod\limits_{1\leq r\leq k}Aut(A_r)
\]
on
\[
\prod\limits_{1\leq r\leq k-1} Ext^1(A_{r+1},A_{r})
\]
that appeared in the statement of Theorem \ref{thm: classification of objects with prescribed associated graded, general case}. Given $\sigma=(\sigma_r)\in Aut(A)$ and 
\[(\mathcal{E}_r) \in \prod\limits_{1\leq r\leq k-1} Ext^1(A_{r+1},A_{r}),\]
the element $\sigma\cdot (\mathcal{E}_r)$ is the element whose $r$-entry is ${(\sigma_{r})}_\ast{(\sigma_{r+1}^{-1})}^\ast \, \sE_r$. More explicitly, $\sigma\cdot (\mathcal{E}_r)$ is obtained as follows: After taking representatives for the $\sE_r$ (i.e. lifting them to extensions), for each $2\leq r\leq k-1$, there is an arrow to $A_r$ in $\sE_{r-1}$ and an arrow from $A_r$ in $\sE_{r}$. We twist both of these arrows by $\sigma_r$ in a way that the action is a left action. That is, $\pi_r: E_{r-1}\twoheadrightarrow A_r$ is twisted to $\sigma_r \pi_r$ and $\iota_{r}: A_r\hookrightarrow E_{r}$ is twisted to $\iota_{r}\sigma_r^{-1}$. For $r=1$ and $r=k$, the object $A_r$ only appears in one of the extensions, and we twist the arrow in that extension by $\sigma_r$ accordingly in a way that we have a left action. The outcome (which is in $\prod_r EXT(A_{r+1},A_{r})$)\footnote{To shorten the writing sometimes we will not include in our writing the range to which the indices of a product belong. In these cases, the index will run through the entire set of indices for which the expressions are available. For example, in $\prod_r EXT(A_{r+1},A_{r})$ the index $r$ runs through $1\leq r\leq k-1$. \label{footnote: explanation for index of sum short notation}} is a representative of  $\sigma\cdot (\mathcal{E}_r)$.

As we already observed, by definition, the data of a generalized extension of level 1 of $A$ is equivalent to the data of a collection of objects $X_{0,2},X_{1,3},\ldots, X_{k-2,k}$ and short exact sequences
\begin{equation}\label{eq16}
\begin{tikzcd}
0\arrow[r] & A_r \arrow[r] & X_{r-1,r+1} \arrow[r] & A_{r+1} \arrow[r] & 0
\end{tikzcd}\hspace{.5in}( 1\leq r\leq k-1).
\end{equation}
Referring to the notation earlier introduced (see Notation \ref{notation: horizontal and vertical extensions}), the extension above is both $\sX^h_{r-1,r+1}$ and $\sX^v_{r-1,r+1}$. A morphism $(f_\dbullet) :(X_\dbullet)\rightarrow (X'_\dbullet)$ of generalized extensions of level 1 is the data of morphisms $f_{r-1,r+1}: X_{r-1,r+1} \rightarrow  X'_{r-1,r+1}$ and $f_r: A_r\rightarrow A_r$ making the diagrams
\[
\begin{tikzcd}
0\arrow[r] & A_r \arrow[d, "f_r"] \arrow[r] & X_{r-1,r+1} \arrow[r] \arrow[d, "f_{r-1,r+1}"] & A_{r+1} \arrow[d, "f_{r+1}"] \arrow[r] & 0\\0\arrow[r] & A_r \arrow[r] & X_{r-1,r+1} \arrow[r] & A_{r+1} \arrow[r] & 0
\end{tikzcd}
\]
commute. By definition, two generalized extensions $(X_\dbullet)$ and $(X'_\dbullet)$ of level 1 are $\sim'$-equivalent if and only if there are morphisms $f_{r-1,r+1}$ which together with identity maps as the $f_r$ make the diagrams commute. In other words, $(X_\dbullet) \sim' (X'_\dbullet)$ if and only if for every $r$ the extensions $\sX^h_{r-1,r+1}$ and ${\sX'}^h_{r-1,r+1}$ represent the same element in the corresponding $Ext^1$ group. This is summarized in part (a) below. Part (b) of the statement follows from the fact that the actions of $Aut(A)$ on both $\oD'(A)$ and $\prod_r Ext^1(A_{r+1},A_r)$ are given by twisting the same arrows in the same way, and the fact (just observed) that the equivalence $\sim'$ translates to the usual equivalence of 1-extensions.
\begin{lemma}\label{lem: D_2}
(a) Two generalized extensions $(X_\dbullet)$ and $(X'_\dbullet)$ of level 1 are $\sim'$-equivalent if and only if for each $r$ the extensions $\sX^v_{r-1,r+1}$ and ${\sX'}^v_{r-1,r+1}$ (i.e. \eqref{eq16} and its counterpart for $(X'_\dbullet)$) coincide in $Ext^1(A_{r+1},A_r)$. We have a bijection
\[
\oD'_1(A) \xrightarrow{ \ \simeq \ } \prod\limits_r Ext^1(A_{r+1},A_r) \hspace{.3in}  (X_\dbullet)\mapsto (\sX^h_{r-1,r+1})_r.
\]
(Here, by abuse of notation an extension and its class in $Ext^1$ are denoted by the same symbol.)

\noindent (b) Considering the previously defined actions of $Aut(A)$ on $\oD'_1(A)$ and $\prod\limits_r Ext^1(A_{r+1},A_r)$, the bijection of part (a) is $Aut(A)$-equivariant and descends to a bijection
\[
\oD_1(A) \xrightarrow{ \ \simeq \ } \left(\prod\limits_r Ext^1(A_{r+1},A_r)\right)\bigm/ Aut(A).
\]
(Here we have identified $\oD'_1(A)/Aut(A)$ with $\oD_1(A)$ via the bijection of Lemma \ref{lem: relation betweem oD and oD'}.)
\end{lemma}
We now turn our attention to the case $\ell=k-1$. Recall from \S \ref{sec: gen exts, defn} that for every pair $(X,\phi)$ of an object $X$ of $\bT$ whose associated graded is isomorphic to $A$ and an isomorphism $\phi:Gr^WX\rightarrow A$, we have an associated generalized extension $ext(X,\phi)$ of level $k-1$. The object at $(m,n)$ entry of $ext(X,\phi)$ is $W_{p_n}X/W_{p_m}X$, with the graded component $W_{p_r}X/W_{p_{r-1}}X$ identified with $A_r$ via $\phi$. The structure morphisms in $ext(X,\phi)$ are the natural injections and projections. The following statement is easily seen from the definitions: 
\begin{lemma}\label{lem: can iso between Gr(X) and A for ext(M,phi)}
For any pair $(X,\phi)$ as above, the canonical isomorphism \eqref{eq14} for $ext(X,\phi)$ with $(m,n)=(0,k)$ is $\phi$.
\end{lemma}
It is easily seen from the constructions that for every $\sigma\in Aut(A)$,
\begin{equation}\label{eq19}
ext(\sigma\cdot(X,\phi)) = ext(X,\sigma\phi) = \sigma\cdot ext(X,\phi).
\end{equation}
Note that here $\sigma\cdot(X,\phi)$ refers to the action of $Aut(A)$ on the collection of pairs $(X,\phi)$. This action was defined by $\sigma\cdot(X,\phi)=(X,\sigma\phi)$ (see \S \ref{sec: statement of problems for S(A) and S'(A)}).

Recall from \S \ref{sec: statement of problems for S(A) and S'(A)} that two pairs $(X,Gr^WX\xrightarrow{\phi,\simeq}A)$ and $(X',Gr^WX'\xrightarrow{\phi',\simeq}A)$ are said to be equivalent if there exists an isomorphism $f: X\rightarrow X'$ such that $\phi'\circ Gr^Wf=\phi$. Finally, also recall that we denoted the set of equivalence classes of such pairs by $S'(A)$, and that the action of $Aut(A)$ on the collection of pairs $(X,\phi)$ descends to an action on $S'(A)$.
\begin{lemma}\label{lem: D_k}
(a) Two pairs $(X,\phi)$ and $(X',\phi')$ as above are equivalent if and only if the generalized extensions $ext(X,\phi)$ and $ext(X',\phi')$ are $\sim'$-equivalent. 

\noindent (b) Let $(X_\db)$ be a generalized extension of level $k-1$. Let $\phi: Gr^WX_{0,k}\rightarrow A$ be
the canonical isomorphism of \eqref{eq14} for $(X_\db)$ and $(m,n)=(0,k)$. Then the identity map on $X_{0,k}$ extends to an isomorphism $(X_\db)\rightarrow ext(X_{0,k}, \phi)$ that is identity on $A$.

\noindent (c) The association $(X,\phi)\mapsto ext(X,\phi)$ descends to a bijection
\[
S'(A) \xrightarrow{\ \simeq \ } \oD'_{k-1}(A).
\]

\noindent (d) Considering the previously defined actions of $Aut(A)$ on $S'(A)$ and $\oD'_{k-1}(A)$, the bijection of part (b) is $Aut(A)$-equivariant and it descends to a bijection   
\[
S'(A)/Aut(A) \xrightarrow{\ \simeq \ } \oD'_{k-1}(A)/Aut(A).
\]

\noindent (e) Using the bijections of Lemmas \ref{lem: relation between S'(A) and S(A)} and \ref{lem: relation betweem oD and oD'} to translate the bijection of part (d) to a map 
\[S(A) \xrightarrow{ \ \simeq \ } \oD_{k-1}(A),\]
this bijection is described as follows: It sends the isomorphism class of $X$ (an object whose associated graded is isomorphic to $A$) to the isomorphism class (i.e. image in $\oD_{k-1}(A)$) of the generalized extension $ext(X,\phi)$ for any choice of isomorphism $\phi:Gr^WX\rightarrow A$.
\end{lemma}
\begin{proof}
(a) Suppose $(X,\phi)$ and $(X',\phi')$ are equivalent, with $f:X\rightarrow X'$ an isomorphism for which $\phi'\circ Gr^Wf=\phi$. By Lemma \ref{lem: morphisms spread to the top right}, $f$ extends uniquely to an isomorphism $(f_\db): ext(X,\phi)\rightarrow ext(X',\phi')$. In view of Lemmas \ref{lem: can iso between Gr(X) and A for ext(M,phi)} and \ref{lem: weight filtration of objects in generalized extensions}(d) (the latter applied with $(m,n)=(0,k)$), the fact that $\phi'\circ Gr^Wf=\phi$ implies that $(f_\db)$ is identity on $A$.

Conversely, suppose $(f_\dbullet):ext(X,\phi)\rightarrow ext(X',\phi')$ is an isomorphism that is identity on $A$. Then $f_{0,k}: X\rightarrow X'$ is an isomorphism that satisfies $\phi' Gr^Wf_{0,k}=\phi$. This again follows from Lemmas \ref{lem: can iso between Gr(X) and A for ext(M,phi)} and \ref{lem: weight filtration of objects in generalized extensions}(d).

\noindent (b) By Lemma \ref{lem: morphisms spread to the top right}, the identity map on $X_{0,k}$ extends uniquely to an isomorphism $(f_\dbullet): (X_\dbullet) \rightarrow ext(X_{0,k},\phi)$. Now apply Lemma \ref{lem: weight filtration of objects in generalized extensions}(d) with $(m,n)=(0,k)$, $(X'_\dbullet)=ext(X_{0,k},\phi)$ and $(f_\dbullet)$ as in here. The left arrow of the diagram is $Gr^Wf_{0,k}=Gr^WId$ which is just the identity. The top and bottom canonical isomorphisms are both $\phi$. Hence the arrow on the right is the identity map.

\noindent (c) By part (a), $(X,\phi)\mapsto ext(X,\phi)$ descends to an injection $S'(A) \rightarrow \oD'_{k-1}(A)$, which is also surjective by part (b).

\noindent (d) We have a commutative diagram
\[
\begin{tikzcd}[column sep=large]
\bigm\{(X,\phi)\bigm\} \arrow[r, "ext(-)"] \arrow[d, twoheadrightarrow] & D_{k-1}(A) \arrow[d, twoheadrightarrow]\\
S'(A) \arrow[r, "\text{part (b)}" ,  "\simeq" '] & \oD'_{k-1}(A),
\end{tikzcd}
\]
where $\{(X,\phi)\}$ means the collection of all pairs $(X,Gr^WX\xrightarrow{\phi,\simeq} A)$. By \eqref{eq19}, the top arrow is $Aut(A)$-equivariant. By definition of the $Aut(A)$-actions on $S'(A)$ and $\oD'_{k-1}(A)$, so are the two side arrows. It follows that the map of part (b) is $Aut(A)$-equivariant and it descends to a map 
\[
S'(A)/Aut(A) \rightarrow \oD'_{k-1}(A)/Aut(A),
\]
which is surjective thanks to part (b). It remains to show that it is also injective.

Consider two pairs $(X,\phi)$ and $(X',\phi')$ such that the classes of $ext(X,\phi)$ and $ext(X',\phi')$ in $\oD'_{k-1}(A)$ are in the same $Aut(A)$-orbit. In view of \eqref{eq19} this means that there exists $\sigma\in Aut(A)$ such that $ext(X,\sigma\phi)$ is $\sim'$-equivalent to $ext(X',\phi')$. That is, there exists an isomorphism $(f_\dbullet)$ of generalized extensions $ext(X,\sigma\phi)\rightarrow  ext(X',\phi')$ such that for each $r$, the morphism $f_{r-1,r}$ is the identity map on $A_r$. 

Consider $f_{0,k}: X\rightarrow X'$. We claim that $\phi'\circ Gr^Wf_{0,k}=\sigma\phi$; this would show that pairs $(X,\sigma\phi)=\sigma\cdot (X,\phi)$ and $(X',\phi')$ are equivalent, so that the classes of $(X,\phi)$ and $(X',\phi')$ in $S'(A)$ are in the same $Aut(A)$-orbit. 

To see the claim, apply Lemma \ref{lem: weight filtration of objects in generalized extensions}(d) to $(f_\dbullet)$ for $(m,n)=(0,k)$. Since the canonical isomorphism \eqref{eq14} for $ext(X,\sigma\phi)$ (resp. $ext(X', \phi')$) is simply $\sigma\phi$ (resp. $\phi'$), we get that the diagram
\[
\begin{tikzcd}
Gr^WX \arrow[d, "Gr^Wf_{0,k}" '] \arrow[r, "\sigma\phi" , "\simeq" ' ] & A \arrow[d, "(f_{r-1,r}) = Id"] \\
Gr^WX' \arrow[r, "\phi'" , "\simeq" ' ] & A
\end{tikzcd}
\]
commutes.

\noindent (e) The given description is clear from the definitions of the other three arrows of the commutative diagram
\[
\begin{tikzcd}
S'(A)/Aut(A) \arrow[d, "\simeq" ' ] \arrow[r, "\text{part (d)}", "\simeq" '] &  \oD'_{k-1}(A)/Aut(A) \arrow[d, "\simeq" ]\\
S(A) \arrow[r, "\simeq" '] & \oD_{k-1}(A).
\end{tikzcd}
\]
\end{proof}

\subsection{The sets $S_\ell(A)$ and $S'_\ell(A)$ and the maps between them}\label{sec: sets S_ell S'_ell and restatements of Thms A and B}
In this subsection we use generalized extensions to define the objects that appeared in the statements of Theorems \ref{thm: classification of S'(A) in the general case} and \ref{thm: classification of objects with prescribed associated graded, general case}. Recall the notation: for each $1\leq \ell\leq k-1$, by $D_\ell(A)$ we denote the collection (and category) of generalized extensions of level $\ell$ of $A$. By $\oD_\ell(A)$ and $\oD'_\ell(A)$ we denote, respectively, the set of equivalence classes of objects of $D_\ell(A)$ with respect to the equivalence relations $\sim$ and $\sim'$; the former equivalence relation is simply given by isomorphisms (maps between corresponding objects of two generalized extensions that commute with the structure arrows), and the latter is given by isomorphisms that are identity on all the $A_r$. 

Recall also that for each $\ell$ we have a truncation functor $\Theta_\ell: D_\ell(A)\rightarrow D_{\ell-1}(A)$, simply erasing the lowest diagonal of a generalized extension. The truncation functors preserve both $\sim$ and $\sim'$, inducing maps $\oD'_\ell(A)\rightarrow \oD'_{\ell-1}(A)$ and $\oD_\ell(A)\rightarrow \oD_{\ell-1}(A)$ both of which we shall (with abuse of notation) also denote by $\Theta_\ell$.

The following commutative diagram summarizes our picture (see Lemmas \ref{lem: D_2} and \ref{lem: D_k}):
\begin{equation}\label{eq20}
\begin{tikzcd}
\{(X,\phi)\} \arrow[d, twoheadrightarrow] \arrow[r, "ext(-)"] & D_{k-1}(A) \arrow[d, twoheadrightarrow] \arrow[r, "\Theta_{k-1}"] & D_{k-2}(A)\arrow[d, twoheadrightarrow]  \arrow[r, "\Theta_{k-2}"] & ~ \cdots \arrow[r, "\Theta_2"] &  D_1(A)\arrow[d, twoheadrightarrow] & \hspace{-.45in} \cong  \prod\limits_r EXT(A_{r+1},A_r)\arrow[d, twoheadrightarrow] \\
S'(A) \arrow[d, twoheadrightarrow] \arrow[r, "\cong"] & \oD'_{k-1}(A) \arrow[d, twoheadrightarrow] \arrow[r, "\Theta_{k-1}"] & \oD'_{k-2}(A) \arrow[d, twoheadrightarrow] \arrow[r, "\Theta_{k-2}"] & ~ \cdots \arrow[r, "\Theta_2"] &  \oD'_1(A) \arrow[d, twoheadrightarrow] & \hspace{-.45in}  \cong  \prod\limits_r Ext^1(A_{r+1},A_r) \arrow[d, twoheadrightarrow] \\
S(A) \arrow[r, "\cong"] & \oD_{k-1}(A) \arrow[r, "\Theta_{k-1}"] & \oD_{k-2}(A) \arrow[r, "\Theta_{k-2}"] & ~ \cdots \arrow[r, "\Theta_2"] &  \oD_1(A) & \hspace{-.4in}  \cong  \displaystyle{\frac{\prod\limits_r Ext(A_{r+1},A_r)}{Aut(A)}}
\end{tikzcd}
\end{equation}
Here, $\{(X,\phi)\}$ means the collection of all pairs $(X, \phi)$ of consisting of an object $X$ of $\bT$ and an isomorphism $\phi: Gr^WX\rightarrow A$. The map $\{(X,\phi)\}\twoheadrightarrow S'(A)$ sends a pair to its $\sim'$-equivalence class. The map $S'(A)\twoheadrightarrow S(A)$ is induced by $(X,\phi)\mapsto X$. The maps $D_\ell(A)\twoheadrightarrow \oD'_\ell(A)$ and $\oD'_\ell(A)\twoheadrightarrow \oD_\ell(A)$, respectively, are given by modding out by $\sim'$ and (further) by $\sim$. All the maps between the middle and bottom rows can also be thought of as modding out by the action of $Aut(A)$.

The sets $S'_\ell(A)$ and $S_\ell(A)$ referred to in Theorems \ref{thm: classification of S'(A) in the general case} and \ref{thm: classification of objects with prescribed associated graded, general case} are respectively simply $\oD'_\ell(A)$ and $\oD_\ell(A)$, with the desired maps between them is given by truncation. The desired identifications of $S'_{k-1}(A)$ and $S_\ell(A)$ with $S'(A)$ and $S(A)$ are given as in the diagram above via the isomorphisms of Lemma \ref{lem: D_k} (induced by $ext$). The desired identifications at level $\ell=1$ are given by Lemma \ref{lem: D_2}. Thus what remains of Theorems \ref{thm: classification of S'(A) in the general case} and \ref{thm: classification of objects with prescribed associated graded, general case} to be established are the assertions about the structure of the fibers. This will be the subject of the rest of the section.

%\begin{rem}\label{rem: different category structures on the D_ell}
%We make a remark about categorical properties of the picture. So far, we have considered $D_\ell(A)$ as a category by defining a morphism $(X_\dbullet)\rightarrow (X'_\dbullet)$ to be a collection of morphisms $f_{m,n}: X_{m,n}\rightarrow X'_{m,n}$ that commute with the structure arrows. There is also another natural way to make $D_\ell(A)$ into a category, by requiring that morphisms are identity on $A$ (then every morphism will be an isomorphism, and there will be at most one morphism from one generalized extension to another). By definition, the set $\oD'_\ell(A)$ is the set of isomorphism classes of objects of $D_\ell(A)$ as a category with this more restrictive notion of morphisms.  The truncation functors continue to be functors in this setting.
%
%Similarly, we can also consider $\{(X,\phi)\}$ and $EXT(A_{r+1},A_r)$ as categories in two ways. With the more relaxed notion of morphisms, a morphism $(X,\phi)\rightarrow (X',\phi')$ is simply a morphism $X\rightarrow X'$. In the more restrictive notion of morphism, a morphism is a morphism $f:X\rightarrow X'$ such that $\phi'\circ Gr^Wf=\phi$. As for $EXT(A_{r+1},A_r)$, the more relaxed notion of morphisms is one that simply gives a commutative diagram between two extensions, and the more restrictive one is the one that in addition requires that the induced maps on $A_r$ and $A_{r+1}$ are identity (the set of isomorphism classes of the latter being $Ext^1(A_{r+1},A_r)$). In both $\{(X,\phi)\}$ and $EXT(A_{r+1},A_r)$, if considered with the more restrictive notion of a morphism, every morphism is an isomorphism and there will be at most one morphism from one object to another.
%
%Back to \eqref{eq20}, the arrows on the top row are all functors whether all categories (including $\{(X,\phi)\}$ and $\prod_r EXT(A_{r+1},A_r)$) are simultaneously considered with the more relaxed notion of morphisms, or they are all simultaneously considered with the more restrictive notion of morphisms. The results of \S \ref{sec: basic properties of gen exts} and \S \ref{sec: equiv rels on gen exts} show that the functor $ext$ is an equivalence of categories in either situation. The last correspondence between $D_1(A)$ and the product of the $EXT$ categories is an isomorphism of categories in either situation. To go from the first row of the diagram to the second row corresponds to considering the induced maps between the isomorphism classes for the more restrictive notion of morphisms, and to go from the first row to the third row corresponds to considering the induced maps between the isomorphism classes for the more relaxed notion of isomorphisms.
%
%Consistent with the definitions given in \S \ref{sec: gen exts, defn}, in what follows the $D_\ell(A)$ are always considered as categories with the more relaxed notion of morphisms.
%\end{rem}

\subsection{Fibers of truncation maps I: torsor structures}\label{sec: fibers 1}
Assume $2\leq \ell\leq k-1$. In this subsection we fix a generalized extension $(X_{m,n})_{n-m\leq \ell}$ of level $\ell-1$ and study the fiber of the truncation functor $\Theta_\ell: D_\ell(A)\rightarrow D_{\ell-1}(A)$ above it, i.e., the collection of all generalized extensions of level $\ell$ that become $(X_{m,n})_{n-m\leq \ell}$ once their lowest diagonal is erased.

Recall that given objects $X,Y,Z$ of $\bT$, $\sN\in EXT(Z,Y)$ and $\sL\in EXT(Y,X)$, the notation $EXTPAN(\sN,\sL)$ means the collection of all blended extensions of $\sN$ by $\sL$ (with no identification made). Recall also the notations $\sX_{m,n}^v$ and $\sX_{m,n}^h$ for extensions respectively coming from the arrows $X_{m,n-1}\hookrightarrow X_{m,n}$ and $X_{m,n}\twoheadrightarrow X_{m+1,n}$ of a generalized extension $(X_\dbullet)$ (i.e. extensions given by \eqref{eq11} and \eqref{eq10}, respectively, see Notation \ref{notation: horizontal and vertical extensions}).
\begin{construction}\label{cons: fibers to prod of EXTPANs}
There is a natural (to be seen to be bijective) map  
\begin{equation}\label{eq21}
\Theta_\ell^{-1}((X_{m,n})_{n-m\leq \ell}) = \left\{\begin{array}{l}
\text{fiber of} \\ D_\ell(A)\xrightarrow{\Theta_\ell} D_{\ell-1}(A) \\ \text{above $(X_{m,n})_{n-m\leq \ell}$} \end{array}\right\} \xrightarrow{ \ \  \ } \prod\limits_r EXTPAN(\sX^v_{r,r+\ell}, \sX^h_{r-1, r+\ell-1})
\end{equation}
(where the index $r$ on the right runs through the integers $1\leq r\leq k-\ell$, see footnote \ref{footnote: explanation for index of sum short notation}) described as follows. Consider an element $(X_{m,n})_{n-m\leq \ell+1}$ of $D_\ell(A)$ in the fiber above $(X_{m,n})_{n-m\leq \ell}$. For each $X_{r-1, r+\ell}$ on its lowest diagonal, the morphisms
\[\begin{tikzcd}
X_{r-1, r+\ell-1} \arrow[d, hookrightarrow] & \\
X_{r-1, r+\ell} \arrow[r, twoheadrightarrow] &  X_{r, r+\ell}
\end{tikzcd}\]
lead to a blended extension
\begin{equation} \label{eq23}
\begin{tikzcd}
   & & 0 \arrow{d} & 0 \arrow{d} &\\
   0 \arrow[r] & A_r \ar[equal]{d} \arrow[r, ] & X_{r-1, r+\ell-1}  \arrow[d, ] \arrow[r] &  X_{r, r+\ell-1} \arrow{d} \arrow[r] & 0 \\
   0 \arrow[r] & A_r \arrow[r] & X_{r-1, r+\ell} \arrow[d] \arrow[r] &  X_{r, r+\ell} \arrow{d}  \arrow[r] & 0, \\
   & & A_{r+\ell} \arrow{d} \ar[equal]{r} & A_{r+\ell} \arrow{d} & \\
   & & 0 & 0 &   
\end{tikzcd} 
\end{equation}
in which every injective map is a composition (uniquely determined by the indices) of the injective structure arrows and every surjective map is a composition of the surjective structure arrows. The extensions on the top and right are respectively $\sX^h_{r-1,r+\ell-1}$ and $\sX_{r,r+\ell}^v$. The map \eqref{eq21} sends $(X_{m,n})_{n-m\leq \ell+1}$ to the tuple with this blended extension in its $r$-entry. 
\end{construction}
\begin{lemma}\label{lem: bijectivity of map from fibers at D level and prod of EXTPANs}
The map \eqref{eq21} is bijective.
\end{lemma}
\begin{proof}
We construct the inverse of \eqref{eq21}. Note that for each $r$ the two extensions $\sX^v_{r,r+\ell}$ and $\sX^h_{r-1, r+\ell-1}$ come from the data of the generalized extension $(X_{m,n})_{n-m\leq \ell}$ of level $\ell-1$. For each $r$, consider a blended extension of $\sX^v_{r,r+\ell}$ by $\sX^h_{r-1, r+\ell-1}$. It is given by a diagram of the form \eqref{eq23}, with the top and right extensions being $\sX^h_{r-1, r+\ell-1}$ and $\sX^v_{r,r+\ell}$, respectively. Now enlarge $(X_{m,n})_{n-m\leq \ell}$ to a generalized extension of level $\ell$ by adding to its data the object $X_{r-1,r+\ell}$ in entry $(r-1,r+\ell)$, and the morphisms $X_{r-1,r+\ell-1}\hookrightarrow X_{r-1,r+\ell}$ and $X_{r-1,r+\ell}\twoheadrightarrow X_{r,r+\ell}$  of the blended extension. The augmented data $(X_{m,n})_{n-m\leq \ell+1}$ is a generalized extension of level $\ell$. Indeed, the only new squares made by the structure arrows are the ones in the top rights of our blended extensions. So axiom (i) of the definition of a generalized extension holds. As for axiom (ii) (the exactness of the sequences \eqref{eq11}), the new sequences we must consider are exactly the sequences on the second rows of our blended extensions. (Note that the arrow $A_r\hookrightarrow X_{r-1,r+\ell}$ in \eqref{eq23} is exactly the composition of all the injective structure arrows of $(X_{m,n})_{n-m\leq \ell+1}$ all the way from $A_r$ to $X_{r-1,r+\ell}$.)

By sending the tuple of blended extensions we started with to the generalized extension $(X_{m,n})_{n-m\leq \ell+1}$ we obtain a map
\begin{equation}\label{eq22}
\prod\limits_r EXTPAN(\sX^v_{r,r+\ell}, \sX^h_{r-1, r+\ell-1}) \xrightarrow{ \ \ \ } 
\Theta_\ell^{-1}((X_{m,n})_{n-m\leq \ell}).
\end{equation}
That this map is the left inverse of \eqref{eq21} is clear. For the assertion that if we start with a tuple of blended extensions \eqref{eq23}, and apply first the map \eqref{eq22} and then the map \eqref{eq21} then we get the original blended extensions back, simply observe that the arrows $A_r\hookrightarrow X_{r-1,r+\ell}$ and $X_{r-1,r+\ell}\twoheadrightarrow A_{r+\ell}$ in \eqref{eq23} are determined by the other two arrows in the appropriate squares.
\end{proof}
It is convenient to have a notation for blended extensions of the form \eqref{eq23}:
\begin{notation} For a generalized extension $(X_\dbullet)$ of level $\ell \geq 2$, for any $r$ we denote the blended extension \eqref{eq23} with middle object $X_{r-1, r+\ell}$ by $\sX_{r-1,r+\ell}$ (without a superscript $h$ or $v$). The arrows in $\sX_{r-1,r+\ell}$ are the appropriate compositions of the structures arrows of $(X_\dbullet)$.
\end{notation}
Recall from \S \ref{sec: background on blended extensions} that two blended extensions $\sX$ and $\sX '$ of an extension $\sN$ by an extension $\sL$ are said to be equivalent if there is a morphism from every object of $\sX$ to the corresponding object of $\sX'$, commuting with the arrows in the two blended extensions, and such that the morphisms on the objects in $\sN$ and $\sL$ are identity. The collection of equivalence classes of blended extensions of $\sN$ by $\sL$ is denoted by $Extpan(\sN,\sL)$.
\begin{lemma}\label{lem: when are two gen ext in a fiber sim' equivalent}
(a) Two elements of the fiber of $\Theta_\ell: D_\ell(A)\rightarrow D_{\ell-1}(A)$ above $(X_\dbullet)$ are $\sim'$-equivalent if and only if their images under the map \eqref{eq21} coincide in 
\[
\prod\limits_r Extpan(\sX^v_{r,r+\ell}, \sX^h_{r-1, r+\ell-1}).
\]
\noindent (b) The map \eqref{eq21} descends to a bijection
\[
\bigm(\Theta_\ell^{-1}((X_\dbullet))\bigm)\bigm/ \sim'  \ \xrightarrow{ \ \simeq \ }  \prod\limits_r Extpan(\sX^v_{r,r+\ell}, \sX^h_{r-1, r+\ell-1}).
\]
In particular,
\[ \bigm(\Theta_\ell^{-1}((X_\dbullet))\bigm)\bigm/ \sim' \]
is either empty or a torsor over
\[
\prod\limits_r Ext^1(A_{r+\ell}, A_r).
\]
Moreover, it is nonempty if and only if for each $r$, the image of the Yoneda product of $\sX^v_{r,r+\ell}$ and $\sX^h_{r-1, r+\ell-1}$ in $Ext^2(A_{r+\ell}, A_r)$ vanishes.
\end{lemma}
\begin{proof}
We first note that part (b) follows immediately from part (a), Lemma \ref{lem: bijectivity of map from fibers at D level and prod of EXTPANs}, and the general theory of blended extensions (see \S \ref{sec: background on blended extensions}; for the very last assertion about non-emptiness of the fiber see Lemma \ref{lem: criteria for compatibility of extension pairs}(a)). So we will focus on part (a).

Suppose $(Y_\dbullet)$ and $(Y'_\dbullet)$ are in the fiber above $(X_\dbullet)$ (so dropping their lowest diagonal, $(Y_\dbullet)$ and $(Y'_\dbullet)$ are just $(X_\dbullet)$). The blended extensions \eqref{eq23} for $(Y_\dbullet)$ and $(Y'_\dbullet)$ are respectively denoted by $\sY _{r-1,r+\ell}$ and $\sY'_{r-1,r+\ell}$. They have the same top rows and right columns. Since every arrow in $\sY _{r-1,r+\ell}$ (resp. $\sY'_{r-1,r+\ell}$) is the appropriate composition of surjective or injective structure arrows of $(Y_\dbullet)$ (resp. $(Y'_\dbullet)$), every morphism $(f_\dbullet):  (Y_\dbullet)\rightarrow (Y'_\dbullet)$ of generalized extensions includes the data of a collection of maps between the corresponding objects of the two blended extension that commute with the arrows in the two blended extensions.

Let $(f_\dbullet):  (Y_\dbullet)\rightarrow (Y'_\dbullet)$ be a morphism that is identity on $A$ (in particular, $(f_\dbullet)$ is an isomorphism). Our weighted situation together with the fact that $(Y_\dbullet)$ and $(Y'_\dbullet)$ both truncate to $(X_\dbullet)$ implies that $(f_\dbullet)$ must be identity on $(X_\dbullet)$ (see Lemma \ref{lem: if identity on A identity everywhere}). It follows that for each $r$, the isomorphism $f_{r-1,r+\ell}$ from $Y_{r-1,r+\ell}$ to $Y'_{r-1,r+\ell}$ gives an isomorphism of blended extensions from $\sY _{r-1,r+\ell}$ to $\sY'_{r-1,r+\ell}$.

Conversely, suppose that for each $r$, the classes of blended extensions $\sY _{r-1,r+\ell}$ and $\sY'_{r-1,r+\ell}$ coincide in $Extpan(\sX^v_{r,r+\ell}, \sX^h_{r-1, r+\ell-1})$. Let $f_{r-1,r+\ell}$ be the morphism $Y_{r-1,r+\ell}\rightarrow Y'_{r-1,r+\ell}$ that gives an isomorphism of blended extensions $\sY _{r-1,r+\ell}\rightarrow \sY'_{r-1,r+\ell}$. Then $f_{r-1,r+\ell}$ induces identity on $X_{r-1,r+\ell-1}$ and $X_{r,r+\ell}$. By Lemma \ref{lem: gluing morphisms between objects on the lowest diagonal} the collection of morphisms $f_{r-1,r+\ell}$ glues together to give a morphism $(f_\dbullet): (Y_\dbullet)\rightarrow (Y'_\dbullet)$. This morphism is identity on the diagonal just above the lowest, and hence is identity on all of $(X_\dbullet)$. %in particular on $A$.
\end{proof}
%\begin{rem}
%The map \eqref{eq21} can be made an isomorphism of categories as follows. Consider each $EXTPAN(\sX_{r,r+\ell}, \sX_{r-1,r+\ell-1})$ as a category in the standard way (with morphism required to be identity on $\sX_{r,r+\ell}$ and $\sX_{r-1,r+\ell-1}$). Give the product over $r$ of $EXTPAN(\sX_{r,r+\ell}, \sX_{r-1,r+\ell-1})$ the structure of the product of the categories in question. Consider the fiber of $\Theta_\ell: D_\ell(A)\rightarrow D_{\ell-1}(A)$ above $(X_\dbullet)=(X_{m,n})_{n-m\leq \ell}$ as a category by taking the morphisms to be isomorphisms of generalized extensions that are identity on $A$ (i.e. the more restrictive notion of morphisms; see Remark \ref{rem: different category structures on the D_ell}). These morphisms are automatically identity on $(X_\db)$. 
%
%Given $(Y_\dbullet)$ and $(Y'_\dbullet)$ in the fiber above $(X_\db)$, by sending a morphism $(f_\dbullet):(Y_\dbullet)\rightarrow (Y'_\dbullet)$ that is identity on $A$ to the induced morphisms $\sY_{r-1,r+\ell}\rightarrow \sY '_{r-1,r+\ell}$ the map \eqref{eq21} becomes a functor, and in fact, an isomorphism of categories.
%
%It is worth mentioning that if we consider both the fiber above $(X_\dbullet)$ and 
%\[EXTPAN(\sX_{r,r+\ell}, \sX_{r-1,r+\ell-1})\] 
%as categories with the more relaxed notion of morphisms, then \eqref{eq21} is still a faithful functor. However, it will not be full if we simply give the product category structure to the product over $r$ of $EXTPAN(\sX_{r,r+\ell}, \sX_{r-1,r+\ell-1})$, as morphisms between the blended extensions for $r$ and $r+1$ may not be compatible\footnote{Here, it is naive to equip $\prod_r EXTPAN(\sX_{r,r+\ell}, \sX_{r-1,r+\ell-1})$ with the structure of the product category. If we modify the definition of morphisms in $\prod_r EXTPAN(\sX_{r,r+\ell}, \sX_{r-1,r+\ell-1})$ to include appropriate compatibility conditions between the different entries, then the said functor will also be full.}. 
%\end{rem}

\subsection{Fibers of truncation maps II: canonicity of the torsor structure and concluding the proof of Theorem \ref{thm: classification of S'(A) in the general case}}\label{sec: fibers of truncations II}
In the previous subsection we studied the fiber of the truncation map above a fixed generalized extension of level $\ell-1$. We now study how the fibers above equivalent generalized extensions are related to one another. This will be important for descending Lemma \ref{lem: when are two gen ext in a fiber sim' equivalent} from the top level of \eqref{eq20} to the second (and then third) level. At the end of this subsection we will deduce Theorem \ref{thm: classification of S'(A) in the general case} from the results we have proved.%\footnote{Note that Theorem \ref{thm: classification of S'(A) in the general case} also asserts that the torsor structures on the fibers of the truncation map $\oD'_\ell(A)\rightarrow \oD'_{\ell-1}(A)$ are canonical (recall that we defined $S'_\ell(A):= \oD'_\ell(A)$). There is a clear dependence on the choice of $(X_\dbullet)$ in Lemma \ref{lem: when are two gen ext in a fiber sim' equivalent} that we must resolve.}

The following simple definition is convenient.
\begin{defn}\label{def: transports}
Let $(X_\dbullet)$ be a generalized extension of any level. Let $(i,j)$ be an eligible pair, and $f:X_{i,j}\rightarrow X'$ an isomorphism.

\noindent (a) Then the transport of $(X_\dbullet)$ along $f$, denoted by $tr((X_\db), f)$, is the generalized extension obtained from $(X_\dbullet)$ by replacing the object $X_{i,j}$ at entry $(i,j)$ by $X'$ via $f$. That is, by replacing $X_{i,j}$ by $X'$, and the arrows to (resp. from) $X_{i,j}$ by their composition with $f$ (resp. $f^{-1}$).

\noindent (b) The collection of morphisms $(f_\db)$ given by $f_{m,n}=Id$ for $(m,n)\neq (i,j)$ and $f_{i,j}=f$ is called the isomorphism given by the transport data. 
\end{defn}
One easily sees that the transport defined above is indeed a generalized extension, and that $(f_\db): (X_\db)\rightarrow  tr((X_\db), f)$ of (b) is indeed an isomorphism of generalized extensions. It is clear that the transport construction behaves well with respect to compositions.

More generally, if $(X_\db)$ be a generalized extension, $I$ is a set of eligible pairs and for each $(m,n)\in I$ we are given an isomorphism $f_{m,n}:X_{m,n}\rightarrow X'_{m,n}$, we will talk about the transport of $(X_\dbullet)$ along $(f_{m,n})_{(m,n)\in I}$. Making the transport all at once is the same as making the transports for one $(m,n)\in I$ at a time (note that the transport operations along morphisms from $X_{m,n}$ for different $(m,n)$ commute with one another). The collection of morphisms $(g_\db)$ where $g_{m,n}=Id$ if $(m,n)\notin I$ and $g_{m,n}=f_{m,n}$ if $(m,n)\in I$ is an isomorphism from $(X_\db)$ to the transport of $(X_\db)$ along $(f_{m,n})_{(m,n)\in I}$; in line with Definition \ref{def: transports}(b), we call it the isomorphism given by the transport data.

Now let $(X_\db)$ and $(X'_\db)$ be generalized extensions of level $\ell-1$, and $(f_\db):(X_\db)\rightarrow (X'_\db)$ an isomorphism (of generalized extensions). Given $(Y_\db)\in \Theta_\ell^{-1}(X_\db)$, it follows from the commutativity of the diagrams of \eqref{eq diagrams for morphisms of gen exts} that the transport of $(Y_\db)$ along $(f_\db)$ is in the fiber of $\Theta_\ell$ above $(X'_\db)$. The map 
\begin{equation}\label{eq24}
\Theta_\ell^{-1}((X_\db)) \rightarrow \Theta_\ell^{-1}((X'_\db)) \hspace{.3in}(Y_\db)\mapsto tr((Y_\db), (f_\db))
\end{equation}
(i.e. transport along $(f_\db)$) is a bijection, with its inverse given by transport along $(f^{-1}_\db)$. Since every $(Y_\db)$ is isomorphic to its transport, this bijection descends to a bijection between the $\sim$-equivalence classes. 

If $(Y_\db^{(1)})$ and $(Y_\db^{(2)})$ in $\Theta_\ell^{-1}((X_\db))$ are $\sim'$-equivalent with $(g_\db): (Y_\db^{(1)})\rightarrow (Y_\db^{(2)})$ an isomorphism that is identity on $A$, then the composition 
\[\begin{tikzcd}[column sep = large]
tr((Y_\db^{(1)}), (f_\db)) \ \ \arrow[r, "\text{iso. given by}", "\text{tr. data $(f_\db^{-1})$}" '] & \ \ (Y_\db^{(1)}) \arrow[r, "(g_\db)"] & (Y_\db^{(2)}) \ \  \arrow[r, "\text{iso. given by}" , "\text{tr. data $(f_\db)$}" '] & \ \ tr((Y_\db^{(2)}), (f_\db))
\end{tikzcd}
\] 
is identity on $A$ as well (even if $(f_\db)$ is not identity on $A$). Hence \eqref{eq24} also descends to a bijection between the $\sim'$-equivalence classes (with its inverse induced by transport along $(f^{-1}_\db)$). 

In particular, if $(X_\db)$ and $(X'_\db)$ in $D_{\ell-1}(A)$ are $\sim$-equivalent, then for every $(Y_\db)\in \Theta^{-1}((X_\db))$ there exists $(Y'_\db)\in \Theta^{-1}((X'_\db))$ which is $\sim$-equivalent (resp. $\sim'$-equivalent) to $(Y_\db)$. 

Finally, we note that if the morphism $(f_\db):(X_\db)\rightarrow (X'_\db)$ is identity on $A$, then every $(Y_\db)$ above $(X_\db)$ is $\sim'$-equivalent to its transport along $(f_\db)$, with the isomorphism given by the transport data giving the $\sim'$-equivalence. 

We obtain the following lemma regarding the fibers of truncation maps $\oD'_\ell(A)\rightarrow \oD'_{\ell-1}(A)$ and $\oD_\ell(A)\rightarrow \oD_{\ell-1}(A)$ (recall that we refer to both of these by $\Theta_\ell$).
\begin{lemma}\label{lem: fibers of truncations between oDs are projections of fibers of truncations between Ds}
Let $(X_\db)\in D_{\ell-1}(A)$. 

%\noindent (a) The fiber of $\Theta_\ell: \oD_\ell(A)\rightarrow \oD_{\ell-1}(A)$ above the class of $(X_\db)$ is equal to the set of $\sim$-equivalence classes of the fiber of $\Theta_\ell: D_\ell(A)\rightarrow D_{\ell-1}(A)$ above $(X_\db)$. That is, denoting the class of $(X_\db)$ in $\oD_\ell(A)$ by $[(X_\db)]_\sim$ , the natural injection 
\noindent (a) Denoting the class of $(X_\db)$ in $\oD_\ell(A)$ by $[(X_\db)]_\sim$ , the natural injection 
\[
\bigm(\Theta_\ell^{-1}((X_\dbullet))\bigm)\bigm/ \sim \ \rightarrow \ \Theta_\ell^{-1}([(X_\db)]_\sim)
\]
is bijective. If $(X'_\db)\in D_{\ell-1}(A)$ and $(f_\db): (X_\db)\rightarrow (X'_\db)$ is an isomorphism, then we have a commutative diagram
\[
\begin{tikzcd}[column sep = 0in]
\bigm(\Theta_\ell^{-1}((X_\dbullet))\bigm)\bigm/ \sim \arrow[rr, "\text{tr. along $(f_\db)$}" , "\simeq" ' ] \arrow[dr, "\simeq"] & & \bigm(\Theta_\ell^{-1}((X'_\dbullet))\bigm)\bigm/ \sim \arrow[dl, "\simeq"] \\
& \Theta_\ell^{-1}([(X_\db)]_\sim) &
\end{tikzcd}
\]
where the horizontal arrow is given by transport along $(f_\db)$ (descending from \eqref{eq24}) and the other two arrows are the natural maps.

%\noindent (b) The fiber of $\Theta_\ell: \oD'_\ell(A)\rightarrow \oD'_{\ell-1}(A)$ above the class of $(X_\db)$ is equal to the set of $\sim'$-equivalence classes of the fiber of $\Theta_\ell: D_\ell(A)\rightarrow D_{\ell-1}(A)$ above $(X_\db)$. That is, denoting the class of $(X_\db)$ in $\oD'_\ell(A)$ by $[(X_\db)]_{\sim'}$, the natural injection 
\noindent (b) Denoting the class of $(X_\db)$ in $\oD'_\ell(A)$ by $[(X_\db)]_{\sim'}$, the natural injection 
\begin{equation}\label{eq25}
\bigm(\Theta_\ell^{-1}((X_\dbullet))\bigm)\bigm/ \sim' \ \rightarrow \ \Theta_\ell^{-1}([(X_\db)]_{\sim'})
\end{equation}
is bijective. If $(X'_\db)\in D_{\ell-1}(A)$ and $(f_\db): (X_\db)\rightarrow (X'_\db)$ is an isomorphism that is identity on $A$, then we have a commutative diagram
\[
\begin{tikzcd}[column sep = 0in]
\bigm(\Theta_\ell^{-1}((X_\dbullet))\bigm)\bigm/ \sim' \arrow[rr, "\text{tr. along $(f_\db)$}" , "\simeq" ' ] \arrow[dr, "\simeq" '] & & \bigm(\Theta_\ell^{-1}((X'_\dbullet))\bigm)\bigm/ \sim' \arrow[dl, "\simeq"] \\
& \Theta_\ell^{-1}([(X_\db)]_{\sim'}) &
\end{tikzcd}
\]
where the horizontal arrow is given by transport along $(f_\db)$ (descending from \eqref{eq24}) and the other two arrows are the natural maps. If $(f_\db): (X_\db)\rightarrow (X'_\db)$ is an isomorphism that is not necessarily identity on $A$, we still have a bijection
\begin{equation}\label{eq26}
\bigm(\Theta_\ell^{-1}((X_\dbullet))\bigm) \bigm/ \sim'  \ \xrightarrow{\text{tr. along $(f_\db)$}} \ \bigm(\Theta_\ell^{-1}((X'_\dbullet))\bigm)\bigm/ \sim'
\end{equation}
which forms a commutative diagram if we pass along on both sides to $\Theta_\ell^{-1}([(X_\db)]_\sim)$.
\end{lemma}
Combining part (b) with Lemma \ref{lem: when are two gen ext in a fiber sim' equivalent}(b) we obtain torsor structures on the nonempty fibers of $\oD'_\ell(A)\rightarrow \oD'_{\ell-1}(A)$. At the moment however, the torsor structure on the fiber of $\oD'_\ell(A)\rightarrow \oD'_{\ell-1}(A)$ above the class of $(X_\db)$ appears to depend on the choice of representative $(X_\db)$. Our next task is to establish a lemma that allows us to capture the extent of this dependence.\footnote{We will use the lemma also when dealing with the torsor structures for Theorem \ref{thm: classification of objects with prescribed associated graded, general case}.}.

Before stating the lemma, we note that the group $Aut(A)$ acts on
\begin{equation}\label{eq27}
\prod\limits_r Ext^1(A_{r+\ell}, A_r),
\end{equation}
in a similar fashion to the action we already considered when $\ell=1$. An element $\sigma=(\sigma_r)$ sends a tuple of extensions $\mathcal{E}=(\mathcal{E}_r)$ to the tuple which has $(\sigma_r)_\ast (\sigma_{r+\ell}^{-1})^\ast \mathcal{E}_r$ in its $r$-entry. Denoting the image of $\mathcal{E}$ under the action by $\sigma$ by $\sigma\cdot \mathcal{E}$, thus $\sigma\cdot \mathcal{E}$ is obtained (at the level of representatives in the product over $r$ of $EXT(A_{r+\ell}, A_r)$) by composing the arrows coming out of each $A_r$ by $\sigma_r^{-1}$ and the arrows going to $A_r$ by $\sigma_r$.
\begin{lemma}\label{lem: compatibility of torsor structures}
Let $(X_\db)$ and $(X'_\db)$ be in $D_{\ell-1}(A)$ and $(f_\db):(X_\db)\rightarrow (X_\db)$ an isomorphism. Suppose that the fiber of $\Theta_\ell: D_\ell(A)\rightarrow D_{\ell-1}(A)$ above $(X_\db)$ (and hence $(X'_\db)$) is nonempty. Consider 
\[
\bigm(\Theta_\ell^{-1}((X_\dbullet))\bigm) \bigm/ \sim' \hspace{.2in}\text{and} \hspace{.2in} \bigm(\Theta_\ell^{-1}((X'_\dbullet))\bigm)\bigm/ \sim' 
\]
as torsors for \eqref{eq27} via the canonical bijection of Lemma \ref{lem: when are two gen ext in a fiber sim' equivalent}(b) (for $(X_\db)$ and $(X'_\db)$, respectively). Then the bijection \eqref{eq26} satisfies the following identity: denoting the action of the group \eqref{eq27} on the torsors by $\ast$, the map \eqref{eq26} (as well as \eqref{eq24}) by $tr(-,(f_\db))$, and the restriction of $(f_\db)$ to $A$ by $f_A$, then for every $(Y_\db) \in \Theta_\ell^{-1}((X_\dbullet))$ and for every tuple of extension classes $\mathcal{E}=(\mathcal{E}_r)$ in \eqref{eq27}, we have
\begin{equation}\label{eq29}
tr(\mathcal{E}\ast [(Y_\db)]_{\sim'}, \, (f_\db))  \ = \  (f_A\cdot \mathcal{E}) \ast \, tr([(Y_\db)]_{\sim'}, (f_\db)).
\end{equation}
(Here, $[(Y_\db)]_{\sim'}$ is the image of $(Y_\db)$ in the fiber above $(X_\db)$ mod $\sim'$.) %Here, $f_A\cdot \mathcal{E}$ means as it was explained before the lemma.)

\noindent In particular, if the restriction of $(f_\db)$ to $A$ is a scalar multiple of the identity, then the bijection \eqref{eq26} is an isomorphism of torsors.
\end{lemma}
\begin{proof}
On recalling from \S \ref{sec: background on blended extensions} the definition of the torsor structure on the collection of isomorphism classes of blended extensions, the action of $\prod_r Ext^1(A_{r+\ell},A_r)$ on 
\[\prod_r Extpan(\sX^v_{r,r+\ell}, \sX^h_{r-1, r+\ell-1})\] 
descends from a map
\[
\prod\limits_r EXT(A_{r+\ell},A_r)\times \prod\limits_r EXTPAN(\sX^v_{r,r+\ell}, \sX^h_{r-1, r+\ell-1})\rightarrow \prod\limits_r EXTPAN(\sX^v_{r,r+\ell}, \sX^h_{r-1, r+\ell-1}),
\]
for which we also use the notation $\ast$. We use the same notation for the map  
\[
\prod\limits_r EXT(A_{r+\ell},A_r)\times \Theta_\ell^{-1}((X_\dbullet)) \rightarrow \Theta_\ell^{-1}((X_\dbullet))
\]
obtained in view of the bijection of Construction \ref{cons: fibers to prod of EXTPANs}. (which descends to the bijection of Lemma \ref{lem: when are two gen ext in a fiber sim' equivalent}(b)).

Let $(Y_\db)$ be in $\Theta_\ell^{-1}((X_\dbullet))$ and $\sE=(\sE_r)$ a tuple in \eqref{eq27}. Lift each $\sE_r$ to an element of $EXT(A_{r+\ell},A_r)$, which with abuse of notation we also denote by $\sE_r$. 

Set
\[(Z_\db) := \sE\ast (Y_\db) \in \Theta_\ell^{-1}((X_\dbullet))\]
and $(Z'_\db)=tr((Z_\db), (f_\db))$. Suppose that the blended extension of $\sX_{r,r+\ell}^v$ by $\sX_{r-1,r+\ell-1}^h$ associated to $(Z_\db)$ by Construction \ref{cons: fibers to prod of EXTPANs} is given by the diagram on the left below. Then the blended extension of ${\sX'}_{r,r+\ell}^v$ by ${\sX'}_{r-1,r+\ell-1}^h$ associated to $(Z'_\db)$  is given by the diagram on the right below. We have dropped the indices from the $f_{m,n}$ to save space (they are determined by the indices of the objects). 
\begin{equation}\label{eq35}
\begin{tikzcd}[row sep=small, column sep=small]
   & & 0 \arrow{d} & 0 \arrow{d} &\\
   0 \arrow[r] & A_r \ar[equal]{d} \arrow[r, "j"] & X_{r-1, r+\ell-1}  \arrow[d, "\overline{\iota}" ] \arrow[r, "\pi"] &  X_{r, r+\ell-1} \arrow[d, "\iota"] \arrow[r] & 0 \\
   0 \arrow[r] & A_r \arrow[r, "\overline j"] & Z_{r-1, r+\ell} \arrow[r, "\overline \pi"] \arrow[d, "\overline \omega"] &  X_{r, r+\ell} \arrow[d, "\omega"]  \arrow[r] & 0 \\
   & & A_{r+\ell} \arrow{d} \ar[equal]{r} & A_{r+\ell} \arrow{d} & \\
   & & 0 & 0 &   
\end{tikzcd} 
\hspace{.1in}
\begin{tikzcd}[row sep=small, column sep=small]
    & & 0 \arrow{d} & 0 \arrow{d} &\\
   0 \arrow[r] & A_r \ar[equal]{d} \arrow[r, "fjf^{-1}"] & X'_{r-1, r+\ell-1}  \arrow[d, "\overline{\iota} f^{-1}" ] \arrow[r, "f\pi f^{-1}"] &  X'_{r, r+\ell-1} \arrow[d, "f \iota f^{-1}"] \arrow[r] & 0 \\
   0 \arrow[r] & A_r \arrow[r, "\overline j f^{-1}"] & Z_{r-1, r+\ell} \arrow[r, "f\overline \pi"] \arrow[d, "f\overline \omega"] &  X'_{r, r+\ell} \arrow[d, "f\omega f^{-1}"]  \arrow[r] & 0 \\
   & & A_{r+\ell} \arrow{d} \ar[equal]{r} & A_{r+\ell} \arrow{d} & \\
   & & 0 & 0 &  
\end{tikzcd} 
\end{equation}
Bringing the indices back to avoid confusion, comparing the second rows of the two diagrams we have
\begin{equation}\label{eq28}
(f_{r-1,r})_\ast  \sZ_{r-1,r+\ell}^h = f_{r,r+\ell}^\ast {\sZ'}_{r-1,r+\ell}^h,
\end{equation}
where $\sZ_{r-1,r+\ell}^h$ and ${\sZ'}_{r-1,r+\ell}^h$ refer to the second horizontal extensions in the two diagrams respectively (see Notation \ref{notation: horizontal and vertical extensions}). The equality here as well as all the other equalities in the rest of this argument take place in the corresponding $Ext^1$ groups.

By definition of the torsor structure on the set of isomorphism classes of blended extensions, we have 
\[
\sZ_{r-1,r+\ell}^h = \sY_{r-1,r+\ell}^h+\omega^\ast\sE_r
\]
in $Ext^1(X_{r,r+\ell}, A_r)$. Combining the last two equations we obtain
\[
f_{r,r+\ell}^\ast {\sZ'}_{r-1,r+\ell}^h = (f_{r-1,r})_\ast\sY_{r-1,r+\ell}^h + (f_{r-1,r})_\ast \, \omega^\ast\sE_r = f_{r,r+\ell}^\ast {\sY'}_{r-1,r+\ell}^h + (f_{r-1,r})_\ast \, \omega^\ast\sE_r,
\]
where $(Y'_\db)=tr((Y_\db),(f_\db))$. (We have used the analogue of \eqref{eq28} for $(Y_\db)$.) Thus
\[
{\sZ'}_{r-1,r+\ell}^h = {\sY'}_{r-1,r+\ell}^h  + (\omega f^{-1}_{r,r+\ell})^\ast (f_{r-1,r})_\ast \sE_r
\]
as pushforward commutes with pullback. 

Let 
\[
(Z''_\db) :=  (f_A\cdot \mathcal{E}) \ast \, tr((Y_\db), (f_\db)) = (f_A\cdot \mathcal{E}) \ast (Y'_\db),
\]
so that the right hand side of \eqref{eq29} is the class of $(Z''_\db)$ in 
\[
\bigm(\Theta_\ell^{-1}((X'_\dbullet))\bigm)\bigm/ \sim'. 
\]
On recalling that the arrow $X'_{r,r+\ell}\twoheadrightarrow A_{r+\ell}$ is $f_{r+\ell-1, r+\ell}\omega f_{r,r+\ell}^{-1}$ we have
\[
{\sZ''}_{r-1,r+\ell}^h = {\sY'}_{r-1,r+\ell}^h + (f_{r+\ell-1, r+\ell} \, \omega f_{r,r+\ell}^{-1})^\ast (f_{r-1,r})_\ast (f^{-1}_{r+\ell-1,r+\ell})^\ast \, \sE_r = {\sZ'}_{r-1,r+\ell}^h. 
\] 
Thus by Lemma \ref{lem: X mapsto X^h is injective when Hom(A_2,A_1)=0}, $(Z'_\db)$ and $(Z''_\db)$ coincide in  
\[
\prod\limits_r Extpan({\sX'}^v_{r,r+\ell}, {\sX'}^h_{r-1, r+\ell-1})
\]
and hence are in the same $\sim'$-equivalence class. This completes the proof of the identity. 

The statement about special case when $(f_\db)$ is identity on $A$ is clear from the identity.
\end{proof}
We are ready to deduce Theorem \ref{thm: classification of S'(A) in the general case}.
\begin{proof}[Proof of Theorem \ref{thm: classification of S'(A) in the general case}]
Consider an element $\epsilon$ of $\oD'_{\ell-1}(A)$, i.e. a $\sim'$-equivalence class of generalized extensions of level $\ell-1$. Let $(X_\db)$ a representative of the class. Use the bijection \eqref{eq25} and the bijection of Lemma \ref{lem: when are two gen ext in a fiber sim' equivalent}(b) to give the fiber of $\oD'_\ell(A)\rightarrow \oD'_{\ell-1}(A)$ above $\epsilon$ the structure of a torsor for 
\[
\prod\limits_r Ext^1(A_{r+\ell}, A_r)
\]
when this fiber is nonempty. This torsor structure is independent of the choice of $(X_\db)$. Indeed, suppose one chooses another representative $(X'_\db)$ for $\epsilon$. Let $(f_\db):(X_\db)\rightarrow (X'_\db)$ be an isomorphism that is identity on $A$ (by Lemma \ref{lem: if identity on A identity everywhere} $(f_\db)$ is unique). By Lemma \ref{lem: compatibility of torsor structures}, transport along $(f_\db)$ gives an isomorphism of torsors \eqref{eq26}. By commutativity of the diagram of Lemma \ref{lem: fibers of truncations between oDs are projections of fibers of truncations between Ds}(b) the induced torsor structures on $\Theta_\ell^{-1}(\epsilon)$ are the same.

The assertion in Theorem \ref{thm: classification of S'(A) in the general case} giving a the sufficient condition for surjectivity of the truncation map $\oD'_\ell(A)\rightarrow \oD'_{\ell-1}(A)$ is immediate from the constructions and Lemma \ref{lem: criteria for compatibility of extension pairs}(a).
\end{proof}

\subsection{Fibers of truncation maps III: actions of automorphism groups on the fibers and the proof of Theorem \ref{thm: classification of objects with prescribed associated graded, general case}(a)}\label{sec: fibers 3}
The proof of Theorem \ref{thm: classification of objects with prescribed associated graded, general case} involves an additional ingredient, namely the group actions that allow us to pass on from the second row of \eqref{eq20} to its third row. The situation is the generalization of the classification problem we considered in \S \ref{sec: blended extensions II}, where we classified the isomorphism classes of objects attached to a given pair of extensions (in the sense of Definition \ref{def: X attached to a pair}).

Fix $2\leq \ell\leq k-1$ and $(X_\db)\in D_{\ell-1}(A)$. The group $Aut((X_\db))$ of the automorphisms of $(X_\db)$ (as a generalized extension) acts on the fiber of $\Theta_\ell: D_\ell(A)\rightarrow  D_{\ell-1}(A)$ above $(X_\db)$ by transport. That is, an element $\sigma=(\sigma_\db) \in Aut((X_\db))$ acts by sending $(Y_\db)\in \Theta_\ell^{-1}((X_\db))$ to the transport of $(Y_\db)$ along $\sigma$. The fact that this transport is also in $\Theta_\ell^{-1}((X_\db))$ is because $\sigma$ is an automorphism of generalized extensions. Denote the image of $(Y_\db)$ under this action by $\sigma\cdot (Y_\db)$. Then $\sigma\cdot (Y_\db) = tr((Y_\db), \sigma)$ is obtained from $(Y_\db)$ by twisting {\it only} the arrows between the two lowest diagonals of $(Y_\db)$. More explicitly, the horizontal arrow $Y_{r-1, r+\ell}\twoheadrightarrow X_{r, r+\ell}$ gets composed with $\sigma_{r, r+\ell}$ and the vertical arrow $X_{r-1, r+\ell-1}\hookrightarrow Y_{r-1, r+\ell}$ gets composed with $\sigma_{r-1, r+\ell-1}^{-1}$. The rest of the diagram (i.e. arrows that are not between the lowest two diagonals) remains unchanged. %(See Remark \ref{rem: alternative description of Gamma} in the end of the subsection.)

In view of Lemma \ref{lem: fibers of truncations between oDs are projections of fibers of truncations between Ds}(b) (applied with $(X'_\db)=(X_\db)$ and $(f_\db)=\sigma$), the action of $Aut((X_\db))$ on the fiber $\Theta_\ell^{-1}((X_\db))$ descends to an action on
\[
\bigm( \Theta_\ell^{-1}((X_\db)) \bigm) \bigm/ \sim'.
\]

The first part of the following lemma is the last ingredient of the proof of Theorem \ref{thm: classification of objects with prescribed associated graded, general case}(a). The second part, which is included for referencing purposes, follows immediately from the first part.

\begin{lemma}\label{lem: orbits of Gamma action}
As above, let $2\leq \ell\leq k-1$ and $(X_\db)\in D_{\ell-1}(A)$. 

\noindent (a) Let $(Y_\db)$ and $(Y'_\db)$ be in $\Theta_\ell^{-1}((X_\db))$. Then $(Y_\db)$ and $(Y'_\db)$ are $\sim$-equivalent (i.e. are isomorphic generalized extensions) if and only if the classes of $(Y_\db)$ and $(Y'_\db)$ mod $\sim'$ are in the same orbit of the action of $Aut((X_\db))$ on  
\begin{equation}\label{eq30}
\bigm( \Theta_\ell^{-1}((X_\db)) \bigm) \bigm/ \sim'.
\end{equation}
\noindent (b) The identity map on $\Theta_\ell^{-1}((X_\db))$ descends to a bijection
\[
\bigm(\Theta_\ell^{-1}((X_\db))\bigm) \bigm/ \sim \ \xrightarrow{ \ \simeq \ } \ \biggm(\bigm(\Theta_\ell^{-1}((X_\db))\bigm) / \sim' \biggm) \biggm/ Aut((X_\db))
\]
\end{lemma}
\begin{proof} (a) Suppose that the classes of $(Y_\db)$ and $(Y'_\db)$ in \eqref{eq30} are in the same orbit of the action of $Aut((X_\db))$. This means that there exist $\sigma\in Aut((X_\db))$ and an isomorphism $(f_\db): \sigma\cdot (Y_\db)\rightarrow (Y'_\db)$ that is identity on $(X_\db)$. Then we obtain an isomorphism $(Y_\db)\rightarrow (Y'_\db)$ by considering the composition
\[
(Y_\db) \xrightarrow{ \ \ \ } \sigma\cdot (Y_\db) \xrightarrow{ \ (f_\db) \ } (Y'_\db),
\]
where the first map is the isomorphism given by the transport data. (This isomorphism is identity on objects on the lowest diagonal and is $\sigma$ elsewhere. See \S \ref{sec: fibers of truncations II}.)
\medskip\par
Conversely, suppose that $(Y_\db)$ and $(Y'_\db)$ are isomorphic. Let $(f_\db): (Y_\db)\rightarrow (Y'_\db)$ be an isomorphism. Then $(f_\db)$ restricts to an automorphism of $(X_\db)$, which we call $\sigma$. Consider the composition of isomorphisms
\[
\begin{tikzcd}[row sep=large]
\sigma\cdot (Y_\db) \arrow[r] & (Y_\db) \arrow[r, "(f_\db)"] &  (Y'_\db),
\end{tikzcd}
\]
where the first arrow is the isomorphism given by the transport data $\sigma^{-1}$. This isomorphism is identity on $A$.
\medskip\par 
\noindent (b) This is clear from (a).
\end{proof}
We now come back to Theorem \ref{thm: classification of objects with prescribed associated graded, general case}. Transfer the action of $Aut((X_\db))$ on \eqref{eq30} to an action on 
\begin{equation}\label{eq32}
\prod\limits_r Extpan(\sX^v_{r,r+\ell}, \sX^h_{r-1, r+\ell-1})
\end{equation}
via the bijection of Lemma \ref{lem: when are two gen ext in a fiber sim' equivalent}(b). On recalling the construction of this bijection and the action of $Aut((X_\db))$ on $(\Theta_\ell^{-1}((X_\db)))/\sim'$ one sees that the action on \eqref{eq32} can be described as follows: An element $\sigma=(\sigma_\db)\in Aut((X_\db))$ acts on a diagram of the form \eqref{eq23} by composing the injective arrow in the first vertical sequence by $\sigma_{r-1, r+\ell-1}^{-1}$, and composing the surjective arrow in the second horizontal sequence by $\sigma_{r, r+\ell}$, and modifying accordingly the other two arrows connected to the middle object in a way that makes the diagram commute. The arrows in $\sX^v_{r,r+\ell}$ and $\sX^h_{r-1, r+\ell-1}$ remain unchanged. The action on \eqref{eq32} then sends the class of \eqref{eq23} to the class of the twisted diagram just described.

In the original statement of Theorem \ref{thm: classification of objects with prescribed associated graded, general case}, there were mentions of a torsor and a group without the details. Let us rewrite the assertion about the structure of the fibers more precisely and then conclude the result. 

\begin{prop}[Theorem \ref{thm: classification of objects with prescribed associated graded, general case}(a), assertion about the fibers]\label{prop: general thm on S(A) part a} 
Let $\epsilon$ be an element of $\oD_{\ell-1}(A)$, i.e. a $\sim$-equivalence class of generalized extensions of level $\ell-1$. 

\noindent (a) For each $(X_\db)$ in the class, there is a bijection\footnote{The group $\Gamma$ in the statement of Theorem \ref{thm: classification of objects with prescribed associated graded, general case} is $Aut((X_\db))$.}
\[
\Theta_\ell^{-1}(\epsilon) \, \cong \ \biggm(\prod\limits_r Extpan(\sX^v_{r,r+\ell}, \sX^h_{r-1, r+\ell-1})\biggm) \biggm/ \hspace{-.05in} Aut((X_\db))
\]
such that the bijections for various representatives are related by transport maps. That is, if $(X_\db)$ and $(X'_\db)$ are representatives of $\epsilon$ in $D_{\ell-1}(A)$ and $(f_\db):(X_\db)\rightarrow (X'_\db)$ is an isomorphism, then we have a commutative diagram
\begin{equation}\label{eq34}
\begin{tikzcd}[column sep = small]
\displaystyle{\frac{\prod\limits_r Extpan(\sX^v_{r,r+\ell}, \sX^h_{r-1, r+\ell-1})}{Aut((X_\db))}} \ \arrow[rr, "\text{tr. along $(f_\db)$}", "\simeq"'] &  & \ \displaystyle{\frac{\prod\limits_r Extpan({\sX'}^v_{r,r+\ell}, {\sX'}^h_{r-1, r+\ell-1})}{Aut((X'_\db))}} \\
& \Theta_\ell^{-1}(\epsilon) \ar[equal]{ul} \ar[equal]{ur} \ , &
\end{tikzcd}
\end{equation}
%with $(f_\db): (X_\db)\rightarrow (X'_\db)$ an isomorphism, then we have a commutative diagram as below, where the horizontal map is given by transport along $(f_\db)$:
where the horizontal map is given by transport along $(f_\db)$ (see the comments after the statement). 

\noindent (b) Choosing a representative $(X_\db)$ for $\epsilon$, the fiber $\Theta_\ell^{-1}(\epsilon)$ is either empty or the quotient of a torsor for 
\begin{equation}\label{eq33}
\prod\limits_r Ext^1(A_{r+\ell}, A_r)
\end{equation}
by the group $Aut((X_\db))$. The fiber $\Theta_\ell^{-1}(\epsilon)$ in nonempty if and only if for every $r$ the image of the Yoneda product of $\sX^v_{r,r+\ell}$ and $\sX^h_{r-1, r+\ell-1}$ in $Ext^2(A_{r+\ell}, A_r)$ vanishes.
\end{prop}
We make a few comments before the argument. The top row of \eqref{eq34} is the map induced by the map
\begin{equation}\label{eq39}
\prod\limits_r Extpan(\sX^v_{r,r+\ell}, \sX^h_{r-1, r+\ell-1}) \rightarrow \prod\limits_r Extpan({\sX'}^v_{r,r+\ell}, {\sX'}^h_{r-1, r+\ell-1})
\end{equation}
that sends the class of the blended extension on the left of \eqref{eq35} to the class of the one on the right of the same diagram. It is clear from the definitions that the map \eqref{eq39} just described agrees with the transport map \eqref{eq26} if we use the bijection of Lemma \ref{lem: when are two gen ext in a fiber sim' equivalent} to identify the two sides of \eqref{eq39} with $\Theta^{-1}_\ell(X_\db)/\sim'$ and $\Theta^{-1}_\ell(X'_\db)/\sim'$. We note that even though the transport map \eqref{eq39} depends on $(f_\db)$, it follows from Proposition \ref{prop: general thm on S(A) part a} that there is no dependence on the choice of $(f_\db)$ once we pass to the quotients by the actions of the automorphism groups. We also note that the actions of \eqref{eq33} on the two sides of \eqref{eq39} are related by formula of Lemma \ref{lem: compatibility of torsor structures}.
\begin{proof}[Proof of Proposition \ref{prop: general thm on S(A) part a}] We have already constructed bijections
\begin{align*}
\Theta_\ell^{-1}(\epsilon)  & \stackrel{\text{Lem. \ref{lem: fibers of truncations between oDs are projections of fibers of truncations between Ds}(a)}}{\cong} \Theta_\ell^{-1}((X_\db))\bigm/ \sim \\
& \stackrel{\text{Lem. \ref{lem: orbits of Gamma action}(b)}}{\cong} \biggm(\bigm(\Theta_\ell^{-1}((X_\db))\bigm) / \sim' \biggm) \biggm/ Aut((X_\db)) \\ 
& \stackrel{\text{Lem. \ref{lem: when are two gen ext in a fiber sim' equivalent}(b)}}{\cong} \biggm(\prod\limits_r Extpan(\sX^v_{r,r+\ell}, \sX^h_{r-1, r+\ell-1})\biggm) \biggm/ Aut((X_\db)) \, .
\end{align*}
The compatibility with the transport map comes from Lemma \ref{lem: orbits of Gamma action} and the last sentence of Lemma \ref{lem: fibers of truncations between oDs are projections of fibers of truncations between Ds}(b).

\noindent The assertions in part (b) are clear from (a).
%{\color{blue} Once and for all identify isomorphism classes mod $\sim'$ and the product of of EXTPANs.}
\end{proof}
We now prove a result about the stabilizers of the action of $Aut((X_\db))$. The result will be useful for our study of the totally nonsplit case in the next subsection. 
\begin{lemma}\label{lem: stablizer of the action by automorphisms} 
Let $(X_\db)\in D_{\ell-1}(A)$ and $(Y_\db)\in \Theta^{-1}((X_\db))$. Then the stabilizer of the $\sim'$-equivalence class of $(Y_\db)$ for the action of $Aut((X_\db))$ on \eqref{eq36} is the image of the restriction map 
\[Aut((Y_\db)) \hookrightarrow Aut((X_\db)).\]
%{\color{blue} Add a comment somewhere that ``all of these are in $Aut(A)$."}
\end{lemma}
\begin{proof}
First note that the restriction map $Aut((Y_\db)) \rightarrow Aut((X_\db))$ is indeed injective by Lemma \ref{lem: if identity on A identity everywhere}. Let $\sigma=(\sigma_\db)\in Aut((X_\db))$. Then $\sigma$ fixes $[(Y_\db)]_{\sim'}$ if and only if there exists an isomorphism $(f_\db): \sigma\cdot(Y_\db)\rightarrow (Y_\db)$ that is identity on $A$, or equivalently, on $(X_\db)$ (again by Lemma \ref{lem: if identity on A identity everywhere}). If $(f_\db)$ is such an isomorphism, then $(\sigma_\db)$ extends to an automorphism $(\tilde{\sigma}_\db)$ of $(Y_\db)$ as follows: if $n-m \leq \ell$ (i.e. away from the lowest diagonal) let $\tilde\sigma_{m,n}=\sigma_{m,n}$, and if $n-m=\ell+1$ set $\tilde\sigma_{m,n}=f_{m,n}$. One can easily check that $(\tilde\sigma_\db)$ is indeed an automorphism of $(Y_\db)$ as a generalized extension. 

Conversely, if $(\sigma_\db)$ extends to an automorphism $\tilde\sigma=(\tilde\sigma_\db)$ of $(Y_\db)$, then define $(f_\db):\sigma\cdot (Y_\db)\rightarrow (Y_\db)$ as follows: if $n-m \leq \ell$ (i.e. away from the lowest diagonal) let $f_{m,n}$ be the identity, and if $n-m=\ell+1$ set $f_{m,n}=\tilde\sigma_{m,n}$. Then $(f_\db)$ is an isomorphism of generalized extensions $\sigma\cdot (Y_\db)\rightarrow (Y_\db)$ that is identity on $(X_\db)$.
\end{proof}
We end this subsection with a discussion of the interaction between the actions of $Aut((X_\db))$ and \eqref{eq33} on 
\begin{equation}\label{eq36}
\bigm(\Theta^{-1}_\ell((X_\db))\bigm)\bigm / \sim' \ \cong \ \prod\limits_r Extpan(\sX^v_{r,r+\ell}, \sX^h_{r-1, r+\ell-1}).
\end{equation}
This discussion will not play any role in the remainder of the paper, and it is related to the contents of \S \ref{sec: beyond totally nonsplit}. After choosing a base point in the above set (assuming it is nonempty), we way identify it with \eqref{eq33}, so that we obtain an action of $Aut((X_\db))$ on \eqref{eq33}. We also have a second (more natural) action of $Aut((X_\db))$ on \eqref{eq33}, obtained as follows. By Lemma \ref{lem: if identity on A identity everywhere}, the restriction map
\[
Aut((X_\db)) \rightarrow Aut(A) \hspace{.3in} \sigma\mapsto \sigma_A
\]
is injective. We restrict the natural linear\footnote{The fact that pushforwards and pullbacks of extensions respect scalar multiplication is clear. For additivity, see \cite[\S 3]{Bu59}.} action of $Aut(A)$ on \eqref{eq33} given by pushforwards and pullbacks (or twisting the arrows, see just before Lemma \ref{lem: compatibility of torsor structures}) to $Aut((X_\db))$ via the embedding above. The following proposition relates this two actions. We recall that our notation for the natural action of $Aut(A)$ on \eqref{eq33} is $\sE \stackrel{\sigma_A}{\mapsto} \sigma_A\cdot \sE$ (see Lemma \ref{lem: compatibility of torsor structures}).
\begin{prop}\label{prop: ineraction between the Gamma action and the torsor structure}
Let $(X_\db)\in D_{\ell-1}(A)$ and $(Y^{0}_\db)\in \Theta^{-1}((X_\db))$. Use the isomorphism
\begin{equation}\label{eq37}
\prod\limits_r Extpan(\sX^v_{r,r+\ell}, \sX^h_{r-1, r+\ell-1}) \ \rightarrow \ \prod\limits_r Ext^1(A_{r+\ell},A_r) \hspace{.2in} [(Y_\db)]_{\sim'}\mapsto [(Y_\db)]_{\sim'}- [(Y^{0}_\db)]_{\sim'}
\end{equation}
(given by the torsor structure and the choice of $[(Y^{0}_\db)]_{\sim'}$ as the base point) to transfer the action of $Aut((X_\db))$ on the left to an action on the right hand side. For every $\sigma\in Aut((X_\db))$, denote the image of a tuple of extension classes $\sE$ under $\sigma$ for this action by ${}^\sigma \sE$. Then for every $\sigma$ and $\sE$,
\[
{}^\sigma \sE \ = \ \sigma_A\cdot \sE + (\sigma\cdot [(Y^{0}_\db)]_{\sim'} - [(Y^{0}_\db)]_{\sim'}).
\]
In particular, the action given by $\sE\stackrel{\sigma}{\mapsto} {}^\sigma \sE$ is linear if and only if $[(Y^{0}_\db)]_{\sim'}$ is fixed by $Aut((X_\db))$. If the action given by $\sE\stackrel{\sigma}{\mapsto} {}^\sigma \sE$ is linear, then it coincides with the restriction of the natural action of $Aut(A)$. 
\end{prop}
\begin{proof}
Recall that the action of $\sigma \in Aut((X_\db))$ on \eqref{eq36} is by transport along $\sigma$. The formula is obtained from the formula of Lemma \ref{lem: compatibility of torsor structures} (applied with $(f_\db)=\sigma$ and $(X'_\db)=(X_\db)$) upon keeping track of the isomorphism \eqref{eq37}). The remaining assertions are immediate from the formula.
\end{proof}
It is an interesting question to ask whether the action of $Aut((X_\db))$ on \eqref{eq36} always has a fixed point (assuming the latter is nonempty). At least in the case where $A$ is semisimple and has three weights, we can prove that this is indeed the case. We will leave this to another paper because it will be too much of a deviation from the theme of the present paper. 
\begin{rem}\label{rem: alternative description of Gamma}
%\noindent (1) {\color{blue} Add remark about whether the action of $Ext^1$'s descends to an action of $\sim$-equivalence classes?}
Let $(X_\db)\in D_{\ell-1}(A)$. We may think of the group $Aut((X_\db))$ and its action on the fiber above $(X_\db)$ in the following way, which is more in line with \S \ref{sec: blended extensions II}. Given any automorphism $\sigma$ of an object $X_{r,r+\ell}$ on the lowest diagonal, $\sigma$ extends uniquely to an automorphism of the part of $(X_\db)$ to the above and right of $X_{r,r+\ell}$ (see Lemma \ref{lem: morphisms spread to the top right}). Denote the induced automorphism of $X_{m,n}$ with $m\geq r$ and $n\leq r+\ell$ by $\sigma_{X_{m,n}}$. We note that by Lemma \ref{lem: if identity on A identity everywhere}, $\sigma$ is determined by the collection of automorphisms $\sigma_{A_m}$ ($r+1\leq m\leq r+\ell$).

Let 
\[\Gamma((X_\db)) \subset \ \prod\limits_r Aut(X_{r, r+\ell})\]
be the subgroup consisting of all the tuples $(\sigma_{r,r+\ell})$ with $\sigma_{r,r+\ell}\in Aut(A_{r, r+\ell})$ that satisfy the following compatibility condition: for every $r$,
\[{(\sigma_{r-1,r+\ell-1})}_{X_{r,r+\ell-1}} \ = \ {(\sigma_{r,r+\ell})}_{X_{r,r+\ell-1}}.\]
That is, the $\sigma_{r,r+\ell}$ agree on the objects on the diagonal just above the lowest one.

By Lemma \ref{lem: gluing morphisms between objects on the lowest diagonal}, we have an isomorphism
\[
Aut((X_\db)) \rightarrow \Gamma((X_\db))
\]
that sends $(\sigma_\db)$ to the tuple $(\sigma_{r,r+\ell})$ (i.e. given by the restriction to the action on the lowest diagonal). Via this isomorphism, the action of $Aut((X_\db))$ on $\Theta_\ell^{-1}((X_\db))$ translates to an action of $\Gamma((X_\db))$ on $\Theta_\ell^{-1}((X_\db))$ given by twisting the arrows between the lowest two diagonals of an element of $\Theta_\ell^{-1}((X_\db))$ (while keeping the rest of the diagram unchanged).
\end{rem}

\subsection{The totally nonsplit case of Theorem \ref{thm: classification of objects with prescribed associated graded, general case}}\label{sec: thm B, tot nonsplit case}
The goal of this section to discuss part (b) of Theorem \ref{thm: classification of objects with prescribed associated graded, general case}. We will define the relevant notions and give the proof of the result. Note that this special case of Theorem \ref{thm: classification of objects with prescribed associated graded, general case} will play a crucial role in \S \ref{sec: mixed motives with maximal unipotent radicals}.
 
We start by giving a definition that generalizes the notion of total nonsplitting to generalized extensions. To recall the definition of total nonsplitting for an extension, see Definition \ref{def: tot nonsplit exts}.
\begin{defn}\label{def: weakly tot nonsplit gen exts}
We say a generalized extension $(X_\db)$ of positive level $\ell-1$ is weakly totally nonsplit if for every $0\leq r\leq k-\ell$, at least one of the extensions $\sX^v_{r,r+\ell}$ and $\sX^h_{r,r+\ell}$ (arising respectively from the injective arrow going to $X_{r,r+\ell}$ and the surjective arrow coming from it, see Notation \ref{notation: horizontal and vertical extensions}) is totally nonsplit. (Note that $X_{r,r+\ell}$ is on the lowest diagonal of $(X_\db)$.)
\end{defn}
This notion is referred to as {\it weak} total nonsplitting because there will be a stronger variant of it that we will introduce later (see \S \ref{sec: classification of mixed motives with maximal unipotent radicals in graded independent case}). In level 1, where a generalized extension is the data of an extension $\sE_r$ of $A_{r+1}$ by $A_r$ for each $r$, the generalized extension $(\sE_r)$ is weakly totally nonsplit if and only all the $\sE_r$ are totally nonsplit.

What makes the notion of weak total nonsplitting interesting for us is the following property:
\begin{lemma}\label{lem: aut group of weakly tot nonsplit gen exts}
Let $(X_\db)$ be a weakly totally nonsplit generalized extension of level $\geq 1$. Then every automorphism of $(X_\db)$ is a scalar map, i.e. $Aut((X_\db))\cong K^\times$. (Recall that $\bT$ is a tannakian category over $K$, where $K$ is a field of characteristic zero.)
\end{lemma}
\begin{proof}
Let $\ell-1$ be the level. By restricting to the lowest diagonal we have an injection
\begin{equation}\label{eq38}
Aut((X_\db)) \hookrightarrow \prod\limits_r Aut (X_{r,r+\ell})
\end{equation}
whose image consists of those elements $(\sigma_{r,r+\ell})$ (with $\sigma_{r,r+\ell}$ an automorphism of $X_{r,r+\ell}$) the entries of which are compatible on the diagonal just above the lowest, i.e. such that for each $r$, the two automorphisms of $X_{r,r+\ell-1}$ induced by $\sigma_{r-1,r+\ell-1}$ and $\sigma_{r,r+\ell}$ respectively via the arrows $X_{r-1,r+\ell-1}\twoheadrightarrow X_{r,r+\ell-1}$ and $X_{r,r+\ell-1}\hookrightarrow X_{r,r+\ell}$ coincide (see Remark \ref{rem: alternative description of Gamma}). For every $r$, either the extension $\sX^v_{r,r+\ell}$ or $\sX^h_{r,r+\ell}$ is totally nonsplit, so that by Lemma \ref{lem: automorphisms of totally nonsplit extensions} every automorphism of $X_{r,r+\ell}$ is a scalar map. Thus each factor $Aut (X_{r,r+\ell})$ of the codomain of \eqref{eq38} is $K^\times$. Compatibility on the diagonal above the lowest implies that the image of \eqref{eq38} is the diagonal copy of $K^\times$.
\end{proof}
The notion of weakly totally nonsplit generalized extensions clearly descends to isomorphism classes of generalized extensions. We are ready to state the more precise version of Theorem \ref{thm: classification of objects with prescribed associated graded, general case}(b).
\begin{prop}[Theorem \ref{thm: classification of objects with prescribed associated graded, general case}(b), precise version]\label{prop: general thm on S(A) part b}
Let $\ell\geq 2$. Let $\epsilon$ be a weakly totally nonsplit element of $\oD_{\ell-1}(A)$. 

\noindent (a) Let $(X_\db)\in D_{\ell-1}(A)$ be a representative of $\epsilon$. Then the action of $Aut((X_\db))$ on $\Theta^{-1}((X_\db))\bigm/ \sim'$ is trivial. The equivalence relations $\sim$ and $\sim'$ coincide on the fiber of $D_\ell(A)\rightarrow D_{\ell-1}(A)$ above $(X_\db)$, and the bijection of Proposition \ref{prop: general thm on S(A) part a} takes the form
\[
\Theta_\ell^{-1}(\epsilon) \, \cong \ \prod\limits_r Extpan(\sX^v_{r,r+\ell}, \sX^h_{r-1, r+\ell-1}).
\]
In particular, for each choice of $(X_\db)$, we get a $\prod_r Ext^1(A_{r+\ell},A_r)$-torsor structure on $\Theta_\ell^{-1}(\epsilon)$. 

\noindent (b) The torsor structures on $\Theta_\ell^{-1}(\epsilon)$ satisfies the following naturalness property: Let $(X_\db)$ and $(X'_\db)$ be two representatives of $\epsilon$. Given any $\tilde{\epsilon}\in \Theta_\ell^{-1}(\epsilon)$ and $\sE\in \prod_r Ext^1(A_{r+\ell},A_r)$, let $\sE\ast \tilde{\epsilon}$ (resp. $\sE\ast' \tilde{\epsilon}$) denote the translation of $\tilde{\epsilon}$ by $\sE$ via the torsor structure on $\Theta_\ell^{-1}(\epsilon)$ obtained by taking $(X_\db)$ (resp. $(X'_\db)$) as a representative. Then there exists a unique automorphism $\phi$ of $\prod_r Ext^1(A_{r+\ell},A_r)$ (depending on $(X_\db)$ and $(X'_\db)$) such that for every $\sE$ and $\tilde{\epsilon}$ as above, we have
\[
\sE \ast \tilde{\epsilon} = \phi (\sE) \, \ast' \, \tilde{\epsilon}.
\]
Moreover, if $(X_\db)$ and $(X'_\db)$ are $\sim'$-equivalent, then $\phi$ is the identity and the torsor structures for the two choices of representatives coincide.
%\noindent (b) The torsor structure on $\Theta_\ell^{-1}(\epsilon)$ satisfies the following naturalness property: Let $(X_\db)$ and $(X'_\db)$ be representatives of $\epsilon$. Let 
%\[
%tr: \Theta_\ell^{-1}(\epsilon) \rightarrow \Theta_\ell^{-1}(\epsilon). 
%\]
%be the bijection fitting into the diagram
%\[
%\begin{tikzcd}[column sep=large]
%Extpan(\sX^v_{r,r+\ell}, \sX^h_{r-1, r+\ell-1}) \ \ar[equal]{d} \arrow[r, "\text{transport}"] & \ Extpan({\sX'}^v_{r,r+\ell}, {\sX'}^h_{r-1, r+\ell-1}) \ar[equal]{d} \\
%\Theta_\ell^{-1}(\epsilon)  \arrow[r, "tr"] & \Theta_\ell^{-1}(\epsilon)
%\end{tikzcd}
%\]
%(where the top transport map is transport along \emph{any} isomorphism $(X_\db)\rightarrow (X'_\db)$; see the proof). Then there is an element $\sigma_A\in Aut(A)$ such that for every $\sE$ in $\prod_r Ext^1(A_{r+\ell},A_r)$ and $\tilde{\epsilon}\in \Theta_\ell^{-1}(\epsilon)$, we have
%\[
%tr(\sE+ \tilde{\epsilon}) = \sigma_A\cdot \sE + tr(\tilde{\epsilon}).
%\]
%Moreover, if $(X_\db)$ and $(X'_\db)$ are $\sim'$-equivalent, then $tr$ is an isomorphism of $\prod_r Ext^1(A_{r+\ell},A_r)$-torsors.
\end{prop}
Note that the real assertion in (b) is about existence of $\phi$, as its uniqueness will then be clear from the formula. We will see in the proof that the automorphism $\phi$ is induced by an automorphism of $A$ (via pushforwards and pullbacks).
%We point out that even though the element $\sigma_A\in Aut(A)$ in the last sentence is not unique (as for instance, can be scaled), but it is clear from the formula that its image in the automorphism group of $\prod_r Ext^1(A_{r+\ell},A_r)$ must be unique.
\begin{proof}
(a) By the previous Lemma, the automorphism group of $(X_\db)$ consists only of the nonzero scalar maps. Thus for every $(Y_\db)$ in $\Theta_\ell^{-1}((X_\db))$, the restriction map $Aut((Y_\db)) \rightarrow Aut((X_\db))$ is a bijection, so that by Lemma \ref{lem: stablizer of the action by automorphisms} the action of $Aut((X_\db))$ on $\Theta^{-1}((X_\db))\bigm/ \sim'$ fixes the class of $(Y_\db)$. The proves the assertion about the triviality of the action of $Aut((X_\db))$. The remaining assertions now follow from Lemma \ref{lem: orbits of Gamma action} and Proposition \ref{prop: general thm on S(A) part a}.

\noindent (b) Recall that for any choice of isomorphism $(f_\db): (X_\db)\rightarrow (X'_\db)$, the transport map
\[
tr(-, (f_\db)): \Theta_\ell^{-1}((X_\db))/\sim' \ \xrightarrow{ \ \simeq \ } \ \Theta_\ell^{-1}((X'_\db))/\sim'
\]
fits in a commutative diagram once we pass on the both sides to $\Theta_\ell^{-1}(\epsilon)$ via the natural maps (see Lemma \ref{lem: fibers of truncations between oDs are projections of fibers of truncations between Ds}). Since $\sim$ and $\sim'$ coincide on $ \Theta_\ell^{-1}((X_\db))$ and $\Theta_\ell^{-1}((X'_\db))$, in view of the same lemma we have a commutative diagram
\[
\begin{tikzcd}[column sep = .2in]
\bigm(\Theta_\ell^{-1}((X_\dbullet))\bigm)\bigm/ \sim' \  \arrow[rr, "\text{$tr(-, (f_\db))$}" , "\simeq" ' ] \arrow[dr, "\simeq" '] & & \  \bigm(\Theta_\ell^{-1}((X'_\dbullet))\bigm)\bigm/ \sim' \arrow[dl, "\simeq"] \\
& \Theta_\ell^{-1}(\epsilon). &
\end{tikzcd}
\]
In follows that the horizontal map is actually independent of the choice of $(f_\db)$. Part (b) is now obtained from the formula of Lemma \ref{lem: compatibility of torsor structures} upon choosing an isomorphism $(f_\db):(X_\db)\rightarrow (X'_\db)$. The automorphism $\phi$ in the statement is induced by the restriction of $(f_\db)$ to $A$ via pullbacks and pushforwards (i.e. with $\phi(\sE)=f_A\cdot \sE$ with notation as in Lemma \ref{lem: compatibility of torsor structures}). If $(X_\db)$ and $(X'_\db)$ are $\sim'$-equivalent, then taking $(f_\db)$ to be the isomorphism that is identity on $A$ we see that $\phi$ is identity in this case.
\end{proof}

\section{Mixed motives with maximal unipotent radicals}\label{sec: mixed motives with maximal unipotent radicals}
\subsection{Setting and background}\label{sec: setting for section on application to motives}
In this section we assume that $\bT$ is a filtered tannakian category over a field $K$ of characteristic zero such that the pure objects of $\bT$ (and hence their direct sums) are semisimple. Prototype examples of this are the category of graded-polarizable mixed Hodge structures over $\QQ$, and any reasonable tannakian category of mixed motives over a field of characteristic zero (e.g. those of Ayoub \cite{Ay14} and Nori \cite{HM17}, Voevodsky's category of mixed Tate motives over a number field, and categories of mixed motives defined via systems of realizations in \cite{De89} and \cite{Ja90}). Inspired by the latter set of examples, we often refer to an object of $\bT$ as a motive or a mixed motive.

Let $X$ be an object of $\bT$. Let $\ffu(X)$ be the object of $\bT$ associated to the Lie algebra of the tannakian group of $X$. We will take the background material on the definition of $\ffu(X)$ for granted, referring the reader to \S 4.2 and \S 2 of \cite{EM2} for a detailed review of this background. We just recall here that $\ffu(X)$ is the canonical subobject of $W_{-1}\inEnd(X)$ (where $\inEnd(X)$ is the internal Hom $\inHom(X,X)$) that has the following property: for every fiber functor $\omega$ from $\bT$ to the category of vector space over $K$, if we consider the tannakian group of $X$ with respect to $\omega$ as a subgroup of $GL(\omega X)$, then
\[
\omega \ffu(X) \subset \omega W_{-1}\inEnd(X) = W_{-1}End(\omega X)
\]
is the Lie algebra of the kernel of the natural surjection from the tannakian group of $X$ with respect to $\omega$ to the tannakian group of $Gr^WX$ with respect to $\omega$. The fact that this kernel is the unipotent radical of the tannakian group of $X$ is because $Gr^WX$ is semisimple.
\begin{defn}\label{def: maximal unipotent radical}
We say $\ffu(X)$ is maximal if
\[
\ffu(X) = W_{-1}\inEnd(X). 
\]
\end{defn}
As we explained in the Introduction, the interest in motives with maximal unipotent radicals comes in part from Grothendieck's period conjecture. If $\bT$ is the category of mixed motives over a number field, the conjecture predicts that the transcendence degree of the field generated by the periods of a motive $X$ should be equal to the dimension of the motivic Galois group of $X$ (i.e., the dimension of the tannakian group of $X$ with respect to the Betti or equivalently, any fiber functor over $\QQ$). Since the dimension of the motivic Galois group of $X$ is equal to the dimension of its unipotent radical plus the dimension of the motivic Galois group of $Gr^WX$, in view of Grothendieck's period conjecture, among all $X$ with a given $Gr^WX$ the field generated by the periods of an $X$ with a maximal unipotent radical should have the largest possible transcendence degree. We refer the reader to Andr\'{e}'s book \cite{An04} or his letter to Bertolin published at the end of \cite{Be20} for more background on Grothendieck's period conjecture.

The aim of this section is to apply the earlier results of the paper to give a homological classification of isomorphism classes of motives with maximal unipotent radicals and associated graded isomorphic to a given graded object $A$, in the case where $A$ is {\it graded-independent}, a property that is defined in \S \ref{defn: graded independence} below. The final result (Theorem \ref{thm: classification of motives with give gr in graded independent case}) significantly generalizes the results of \S 6 of \cite{EM2}, where with Murty we established this result in the special case where $A=A_1\oplus A_2\oplus \mathbbm{1}$ with $A_1,A_2$ pure of negative increasing weights and $Ext^1(\mathbbm{1}, A_1)=0$.

We start by a brief discussion of the relationship between two related notions, namely maximality of the unipotent radical of an object and total nonsplitting of the extensions that naturally arise from the object. In \S \ref{sec: maximality criterion} we define the notion of graded-independence, which is an independence axiom in the spirit of such axioms in \cite{EM2}, and give our maximality criterion for graded-independent motives (Theorem \ref{thm: maximality criteria}). We then combine this with Theorem \ref{thm: classification of objects with prescribed associated graded, general case} in \S \ref{sec: classification of mixed motives with maximal unipotent radicals in graded independent case} to give a classification result for motives with maximal unipotent radicals and a given graded-independent associated graded. Finally, in \S \ref{sec: examples} we specialize as an example to Voevodsky's category of mixed Tate motives over $\QQ$ and study 4-dimensional objects with maximal unipotent radicals.

\subsection{Maximality and total nonsplitting}\label{sec: maximality and total nonsplitting}
The following lemma, only part (a) of which is new, gives a summary of some aspects of the relationship between maximality of unipotent radicals and total nonsplitting of extensions coming from the weight filtration.
\begin{lemma}\label{lem: maximality and total nonsplitting}
Let $X$ be a motive (i.e. an object of $\bT$, a filtered tannakian category over a field of characteristic zero in which pure objects are semisimple).

\noindent (a) If $\ffu(X)$ is maximal, then for every integers $\ell<m<n$ the extension
\begin{equation}\label{eq50}
\begin{tikzcd}
0 \arrow[r] & W_mX/W_\ell X \arrow[r] & W_nX/W_\ell X \arrow[r] & W_nX/W_m X \arrow[r] & 0
\end{tikzcd}
\end{equation}
is totally nonsplit.

\noindent (b) When $X$ has only two weights, then the necessary condition of part (a) for maximality of $\ffu(X)$ is also sufficient. That is, if $Gr^WX=Gr^W_mX \oplus Gr^W_nX$ with $m<n$, then $\ffu(X)$ is maximal if and only if the extension
\begin{equation}\label{eq49}
\begin{tikzcd}
0 \arrow[r] & Gr^W_mX \arrow[r] & X \arrow[r] & Gr^W_nX \arrow[r] & 0
\end{tikzcd}
\end{equation}
is totally nonsplit.
%\begin{equation}\label{eq49}
%\begin{tikzcd}
%0 \arrow[r] & W_mX \arrow[r] & X \arrow[r] & X/W_mX \arrow[r] & 0
%\end{tikzcd}
%\end{equation}

\noindent (c) Taking $\bT$ to be the category of graded-polarizable rational mixed Hodge structures, there exists an object $X$ of $\bT$ for which all of the extensions of part (a) are totally nonsplit, but $\ffu(X)$ is not maximal. (Thus in general, the condition of part (a) for maximality is not sufficient.)
\end{lemma}
We first discuss parts (b) and (c). The former is by a result of Hardouin (\cite[Theorem 2]{Har11}, see also Theorem 2.1 of the unpublished article \cite{Har06}), which was proved earlier by Bertrand for the special case of D-modules \cite[Theorem 1.1]{Ber01}. In the setting of part (b) with two weights $m<n$ (and semisimple $Gr^WX$), the result asserts that $\ffu(X)$ is the smallest subobject of 
\[W_{-1}\inEnd(X)=\inHom(Gr^W_nX, Gr^W_mX)\] 
with the the following property: the extension \eqref{eq49}, considered as an element of 
\[
Ext^1(\mathbbm{1}, \inHom(Gr^W_nX, Gr^W_mX)) 
\]
via the isomorphism \eqref{eq48}, splits after pushing forward along the quotient map 
\[
\inHom(Gr^W_nX, Gr^W_mX) \rightarrow \inHom(Gr^W_nX, Gr^W_mX)/\ffu(X).
\]
On recalling the definition of total nonsplitting (see Definition \ref{def: tot nonsplit exts}), the statement in (b) is now immediate.\footnote{Both Hardouin and Bertrand consider extensions of $\mathbbm{1}$ by a semisimple object $L$. The version of the result that allows for extensions of a semisimple object $N$ by a semisimple object $L$ (which is what is being used here) can be found in \cite[Corollary 3.4.1]{EM1}.}

As for part (c), we refer the reader to \S 5.2 of \cite{Es23} for an example with $X$ being a 1-motive. (We will not use the statement of part (c) in any result of the present paper; it is included to put the results of the next subsection in a better context.)

We now turn our attention to part (a). The proof uses a result from \cite{EM2}, which we recall here:
\begin{thm}[Theorem 4.9.1 of \cite{EM2}]\label{thm: characterization of u_p from the ANT paper}
Let $X$ be an object of $\bT$. Let $\sE_m(X)$ be the class of the extension
\begin{equation}\label{eq67}
\begin{tikzcd}
0 \arrow[r] & W_mX \arrow[r] & X \arrow[r] & X/W_mX \arrow[r] & 0
\end{tikzcd}
\end{equation}
considered as an element of 
\[
Ext^1(\mathbbm{1}, \inHom(X/W_mX, W_mX))
\]
via the canonical isomorphism \eqref{eq48}. Let 
\[\ffu_m(X) : = \ffu(X)\cap \inHom(X/W_mX, W_mX),\]
where $\inHom(X/W_mX, W_mX)$ is considered as a subobject of $W_{-1}\inEnd(X)$ in the natural way. Then for every subobject $L$ of $\inHom(X/W_mX, W_mX)$ one has 
\[
\ffu_m(X) \subset L
\] 
if and only if the pushforward $\sE_m(X)/L$ of $\sE_m(X)$ along the quotient map
\[
\inHom(X/W_mX, W_mX) \rightarrow \inHom(X/W_mX, W_mX)/L
\] 
is in the image of the obvious map
\[
Ext^1_{\langle W_mX\oplus X/W_mX\rangle}(\mathbbm{1}, \inHom(X/W_mX, W_mX)/L) \rightarrow Ext^1_\bT(\mathbbm{1}, \inHom(X/W_mX, W_mX)/L)
\]
(given by the inclusion $\langle W_mX\oplus X/W_mX\rangle\subset \bT$).
\end{thm}
Here, as well as later in the paper, the notation $\langle Y\rangle$ means the tannakian subcategory generated by $Y$ (i.e., the smallest full tannakian subcategory containing $Y$ and closed under subquotients). Referring to the setting of the theorem, note in particular that the result implies that if a subobject $L$ of $\inHom(X/W_mX, W_mX)$ has the property that the pushforward $\sE_m(X)/L$ splits, then $L$ contains $\ffu_m(X)$. Thus one obtains the following corollary:
\begin{cor}\label{cor: if u_p is maximal then E_p is totally nonsplit}
With the notation and setting as in Theorem \ref{thm: characterization of u_p from the ANT paper}, if 
\[\ffu_m(X)=\inHom(X/W_mX, W_mX),\] 
then the extension \eqref{eq67} (or equivalently, $\sE_m(X)$) is totally nonsplit.
\end{cor}
We are ready to deduce part (a) of Lemma \ref{lem: maximality and total nonsplitting}.
\begin{proof}[Proof of Lemma \ref{lem: maximality and total nonsplitting}(a)]
Suppose $\ffu(X)=W_{-1}\inEnd(X)$. The inclusion $\langle W_nX/W_\ell X\rangle \subset \langle X\rangle$ induces a surjection from the tannakian group of $X$ to the tannakian group of $W_nX/W_\ell X$, which in turn induces a surjection $\ffu(X)\rightarrow \ffu(W_nX/W_\ell X)$. This surjection fits into a commutative diagram
\[
\begin{tikzcd}
\ffu(X) \ar[equal]{d} \arrow[r, twoheadrightarrow] & \ffu(W_nX/W_\ell X)  \arrow[d, hookrightarrow ]\\
W_{-1}\inEnd(X) \arrow[r, twoheadrightarrow ] & W_{-1}\inEnd(W_nX/W_\ell X),
\end{tikzcd}
\]
where the vertical arrows are the canonical inclusions and the bottom horizontal arrow is the map that after applying a fiber functor $\omega$, it sends an element of $W_{-1}End(\omega X)$ to its induced endomorphism of $W_n \omega X/W_\ell \omega X$. It follows that $\ffu(W_nX/W_\ell X)$ is also maximal. The assertion now follows from the previous corollary (applied to $W_nX/W_\ell X$ instead of $X$).
\end{proof}
\begin{rem}
We should point out that the hypothesis of semisimplicity of pure objects of $\bT$ is not needed for Theorem \ref{thm: characterization of u_p from the ANT paper} (and is not assumed in Theorem 4.9.1 of \cite{EM2}). Subsequently, Corollary \ref{cor: if u_p is maximal then E_p is totally nonsplit} and Lemma \ref{lem: maximality and total nonsplitting}(a) are true in arbitrary filtered tannakian categories over fields of characteristic zero. However, Lemma \ref{lem: maximality and total nonsplitting}(b) needs the hypothesis that pure objects are semisimple.
\end{rem}

\subsection{Maximality of unipotent radicals for graded-independent motives}\label{sec: maximality criterion}
In this subsection we study maximality of unipotent radicals of {\it graded-independent} motives (defined below), a class of motives that are the focus of the rest of the paper. We will prove a necessary and sufficient condition for maximality of unipotent radicals of such motives. We will also prove a result about the $Ext^1$ groups in subcategories generated by graded-independent motives with maximal unipotent radicals.

We start with a definition:
\begin{defn}\label{defn: graded independence}
Let $X$ be a motive. Denote the weights of $X$ by $p_1<\cdots<p_k$. We say that $X$ is graded-independent if every two of the $k$ objects
\[
\inHom(Gr^W_{p_{i+1}}X, Gr^W_{p_{i}}X) \hspace{.3in}(1\leq i\leq k-1)
\]
and
\[
\bigoplus_{i<j-1} \inHom(Gr^W_{p_j}X, Gr^W_{p_{i}}X)
\]
do not have any nonzero isomorphic subobjects (or equivalently, subquotients since $Gr^WX$ is semisimple).
\end{defn}
The property is an ``independence axiom"  in the spirit of such axioms in \cite{EM2}. In general, the reason such properties are of interest is twofold. Firstly, they force $Gr^W\ffu(X)$ to decompose in a way that makes it easier to study. Secondly, they are not far too restrictive, in the sense that they are satisfied in some very interesting situations. For instance, the property above (as well as all the independence axioms in \cite{EM2}) is satisfied as long as the weights of $X$ are sufficiently spread out so that the numbers $p_i-p_j$ (notation as in the definition) are all distinct as the integers $i,j$ vary in $1\leq i<j\leq k$.

We now state our maximality criterion:
\begin{thm}\label{thm: maximality criteria}
Suppose that $X$ is a graded-independent motive. Let $p_1<\cdots<p_k$ be the weights of $X$. Then the following two statements are equivalent:
\begin{itemize}
\item[(i)] $\underline{\fu}(X)$ is maximal.
\item[(ii)] For every integer $1\leq r \leq k-1$, the extension
\begin{equation}\label{eq7}
\begin{tikzcd}
   0 \arrow[r] & Gr^W_{p_{r}}X \arrow[r, ] & \displaystyle{\frac{W_{p_{r+1}}X}{W_{p_{r-1}}X}} \arrow[r, ] &  Gr^W_{p_{r+1}}X \arrow[r] & 0
\end{tikzcd}
\end{equation}
is totally nonsplit. (Here we set $p_0=p_1-1$ so $W_{p_0}(X)=0$.)
\end{itemize}
\end{thm}
\begin{proof}
(a) The fact that (i) implies (ii) is clear and in fact, does not require the graded-independence condition. See Lemma \ref{lem: maximality and total nonsplitting}(a).

We now turn our attention to proving that (ii) implies (i). It is enough to show that 
\[
Gr^W\ffu(X) = Gr^WW_{-1}\inEnd(X).
\]
The graded-independence condition (on recalling that $\langle Gr^WX\rangle $ is semisimple) guarantees that the subobject $Gr^W\ffu(X)$ of
\[
Gr^WW_{-1}\inEnd(X) = \bigoplus\limits_{i<j}\inHom(Gr^W_jX,Gr^W_iX)
\]
decomposes according to the decomposition of $Gr^WW_{-1}\inEnd(X)$ as the direct sum of the $k$ objects of Definition \ref{defn: graded independence}. For each $1\leq r \leq k-1$, consider the composition 
\[
\ffu(X) \twoheadrightarrow \ffu(W_{p_{r+1}}X/W_{p_{r-1}}X) \hookrightarrow \inHom(Gr^W_{p_{r+1}}X,Gr^W_{p_{r}}X),
\]
where the first map is induced by the inclusion of $\langle W_{p_{r+1}}X/W_{p_{r-1}}X\rangle$ in $\langle X\rangle$  and the second map is the natural inclusion. Since \eqref{eq7} is totally nonsplit, by Lemma \ref{lem: maximality and total nonsplitting}(b) this second arrow is an equality, so that the composition is surjective. Applying $Gr^W$ we get a surjection of $Gr^W\ffu(X)$ onto $\inHom(Gr^W_{p_{r+1}}X,Gr^W_{p_{r}}X)$. In view of the decomposition of $Gr^W\ffu(X)$ as the direct sum of its intersections with the $k$ objects of Definition \ref{defn: graded independence} and the fact that these $k$ objects do not have any nonzero isomorphic subquotients, it follows that $Gr^W\ffu(X)$ contains
\[
\inHom(Gr^W_{p_{r+1}}X,Gr^W_{p_{r}}X).
\]
Since $Gr^WW_{-1}\inEnd(X)$ is generated\footnote{Say, after applying a fiber functor, although this is not necessary if the term Lie algebra is interpreted as a Lie algebra object in $\bT$.} as a Lie algebra by the $k-1$ objects above (with $1\leq r\leq k-1$), the result follows.
\end{proof}
We note here that the graded-independence hypothesis is crucial for condition (ii) of the previous result to guarantee that $\ffu(X)$ is maximal. See Lemma \ref{lem: maximality and total nonsplitting}(c).

We end this subsection with a result in a different direction compared to the rest of the paper. While we find the result interesting in its own sake, its inclusion here is because we will use it in \S \ref{sec: examples} to put the examples there in a better context.

Suppose that $X$ has $k$ weights $p_1<\cdots<p_k$ and $\ffu(X)$ is maximal. Then in particular, the extensions \eqref{eq7} are nonsplit, so that the tannakian subcategory $\langle X\rangle$ generated by $X$ has a nontrivial extension of $Gr^W_{p_{r+1}}(X)$ by $Gr^W_{p_{r}}(X)$ for each $1\leq r\leq k-1$. The next result concerns the $Ext^1$ groups in $\langle X\rangle$ between non-consecutive graded pieces of $X$.
\begin{thm}\label{thm: Ext groups for non-consecutive graded pieces vanish in the graded-independent situation}
Let $X$ be a graded-independent motive with a maximal unipotent radical. Let $p_1<\cdots<p_k$ be the weights of $X$. Then for every $i,j$ with $i\leq j-2$ we have
\[
Ext^1_{\langle X\rangle}(Gr^W_{p_j}X, Gr^W_{p_i}X) \cong 0,
\]
where $Ext^1_{\langle X\rangle}$ means the $Ext^1$ group for the category $\langle X\rangle$.
\end{thm}
\begin{proof}
The technical heart of the proof is Proposition \ref{prop appendix: Ext groups and maps from the abelianization of u} of the appendix, which we take for granted here. Referring to the notation of the result in the appendix, take $G$ to be the tannakian group of $X$ with respect to a fiber functor $\omega$ with values in the category of vector spaces over $K$ (recall that $\bT$ is a tannakian category over $K$, and $K$ has characteristic zero). Take $R$ to be the tannakian group of $Gr^WX$ (which is a semisimple object, by assumption) with respect to $\omega$, and $U$ the unipotent radical of $G$. Then $U$ is the kernel of the natural map $G\twoheadrightarrow R$, and the Lie algebra $\fu$ of $U$ is the image of $\ffu(X)$ under $\omega$. Choosing a Levi factor of $G$ we may identify $G=U\rtimes R$.

Denote $\ffu^{ab}(X)$ be the abelianization of $\ffu(X)$; it is a semisimple motive (see the appendix for semisimplicity), the quotient of $\ffu(X)$ whose image under $\omega$ is the abelianization of $\fu$, with the quotient map $\ffu(X)\rightarrow \ffu^{ab}(X)$ becoming the quotient map $\fu\rightarrow \fu^{ab}$ after applying $\omega$. By the result in the appendix, for every semisimple object $N$ of $\langle X\rangle$ with $Hom(\mathbbm{1}, N)\cong 0$\footnote{The condition $Hom(\mathbbm{1}, N)\cong 0$ is not needed here because $Ext^1_{\langle X\rangle}(\mathbbm{1}, \mathbbm{1})$ and $Hom(\ffu^{ab}, \mathbbm{1})$ are both zero, the former because pure objects are assumed to be semisimple and the latter because of the weight filtration.} there is a canonical isomorphism
\begin{equation}\label{eq43}
Ext^1_{\langle X\rangle}(\mathbbm{1}, N) \cong Hom(\ffu^{ab}, N).
\end{equation}
($Hom$ on the right hand side is both for $\langle X\rangle$ and $\bT$.)

We are ready to deduce the theorem. By our assumptions, $\ffu(X)=W_{-1}\inEnd(X)$. Thus
\[
Gr^W[\ffu(X), \ffu(X)] = Gr^WW_{-2}\inEnd(X) = \bigoplus\limits_{i\leq j-2}\inHom(Gr^W_{p_j}X, Gr^W_{p_i}X).
\]
Applying the functor $Gr^W$ to the sequence
\[
\begin{tikzcd}
0 \arrow[r] & \text{$[\ffu(X), \ffu(X)]$}  \arrow[r] & \ffu(X) \arrow[r] & \ffu^{ab}(X) \arrow[r] & 0,
\end{tikzcd}
\]
on recalling that $\ffu^{ab}(X)$ is semisimple we obtain
\[
\ffu^{ab}(X) \cong Gr^W\ffu^{ab}(X) \cong  \bigoplus\limits_{j}\inHom(Gr^W_{p_j}X, Gr^W_{p_{j-1}}X).
\]
Hence, thanks to the graded-independence hypothesis, for any $i,j$ with $i\leq j-2$ there are no nonzero morphisms from $\ffu^{ab}(X)$ to $\inHom(Gr^W_{p_j}X, Gr^W_{p_{i}}X)$. For any such $i,j$ we thus have
\[
Ext^1_{\langle X\rangle }(Gr^W_{p_j}X, Gr^W_{p_{i}}X) \cong Ext^1_{\langle X\rangle }(\mathbbm{1}, \inHom(Gr^W_{p_j}X, Gr^W_{p_{i}}X)) \cong 0
\]
by \eqref{eq43}.
\end{proof}

\subsection{Mixed motives with maximal unipotent radicals and a prescribed graded-independent associated graded}\label{sec: classification of mixed motives with maximal unipotent radicals in graded independent case}
In this subsection we combine our maximality criterion from \S \ref{sec: maximality criterion} with our work in \S \ref{sec: objects in filtered tan cats with given gr}. 

Let $A_1,\ldots, A_k$ be pure motives, with $A_i$ of weight $p_i$ and $p_1<\cdots<p_k$. Set
\[
A : =\bigoplus\limits_{1\leq i\leq k} A_i.
\]
We will use the notation and language of \S \ref{sec: objects in filtered tan cats with given gr}. Thus $S(A)$ denotes the set of isomorphism classes of motives whose associated graded is isomorphic to $A$, and for each $0\leq \ell\leq k-1$ by $S_\ell(A)$ ( = $\oD_\ell(A)$) we denote the set of $\sim$-equivalence classes (i.e. isomorphism classes) of generalized extensions of level $\ell$ of $A$. The truncation map from level $\ell$ to level $\ell-1$ is denoted by $\Theta_\ell$. 

Earlier we defined the notion of a weakly totally nonsplit generalized extension (Definition \ref{def: weakly tot nonsplit gen exts}). We now define a stronger variant of the notion.
\begin{defn}\label{def: tot nonsplit gen exts}
We say a generalized extension $(X_\db)$ of any positive level is totally nonsplit if the extensions $\sX^v_{m,n}$ and $\sX^h_{m,n}$ (see Notation \ref{notation: horizontal and vertical extensions}) are totally nonsplit for every eligible pair $(m,n)$. 
\end{defn}
Note that in level 1, the two notions of total nonsplitting and weakly total nonsplitting coincide. In any level, a totally nonsplit generalized extension is weakly totally nonsplit. It is not difficult to see that the converse is not true in general. The notion of a totally nonsplit generalized extension descends to the isomorphism classes of such extensions.

It is clear from the definition that if a generalized extension is totally nonsplit, then so are its truncations. In the graded-independent case, the converse is also true:
\begin{lemma}\label{lem: when A is graded-independent, a gen ext is totally nonsplit if and only if its first level is}
Let $A$ ( = the direct sum of the $A_j$, as above) be graded-independent and $(X_\db)$ be a generalized extension of $A$ of level $\ell\geq 2$. If the truncation $\Theta_\ell((X_\db))$ is totally nonsplit, then so is $(X_\db)$.
\end{lemma}

\begin{proof}
We show that if $(X_{i,j})_{j-i\leq 2}$ (i.e the truncation of $(X_\db)$ to level 1) is totally nonsplit, then so is $(X_\db)$; this will establish the result since total nonsplitting of $\Theta_\ell((X_\db))$ implies total nonsplitting of the truncation to level 1.

Suppose that $(X_{i,j})_{j-i\leq 2}$ is totally nonsplit. By definition, this means that each of the extensions 
\begin{equation}\label{eq41}
\begin{tikzcd}
0 \arrow[r] & A_i \arrow[r] & X_{i-1, i+1} \arrow[r] & A_{i+1}\arrow[r] & 0
\end{tikzcd}  
\end{equation}
is totally nonsplit. Consider an object $X_{r-1,r+\ell}$ on the lowest diagonal of $(X_\db)$. By Lemma \ref{lem: weight filtration of objects in generalized extensions} we have a canonical isomorphism from $Gr^W X_{r-1,r+\ell}$ to the direct sum of $A_r$, $A_{r+1}$, $\ldots$, $A_{r+\ell}$. Denoting this canonical isomorphism by $\phi$, from Lemma \ref{lem: D_k}(b) we know that the two generalized extensions (i) the part of $(X_\db)$ to the above and right of $X_{r-1,r+\ell}$ (i.e. consisting of the $X_{i,j}$ with $i\geq r-1$ and $j\leq r+\ell$) and (ii) $ext(X_{r-1,r+\ell}, \phi)$ (i.e. the generalized extension associated with $(X_{r-1,r+\ell}, \phi)$, see \S \ref{sec: equiv rels on gen exts} or \S \ref{sec: gen exts, defn}) are $\sim'$-equivalent. It follows that after identifying
\[Gr^WX_{r-1,r+\ell} \cong \bigoplus\limits_{r\leq i\leq r+\ell} A_i\]
via $\phi$, the class of the extension \eqref{eq41} is equal to the class of the extension
\[
\begin{tikzcd}
0 \arrow[r] & A_i \arrow[r] & W_{p_{i+1}}X_{r-1,r+\ell}/W_{p_{i-1}}X_{r-1,r+\ell} \arrow[r] & A_{i+1}\arrow[r] & 0
\end{tikzcd}  
\]
coming from the weight filtration on $X_{r-1,r+\ell}$. In particular, the latter extension is totally nonsplit for all $r\leq i<r+\ell$. Since $A$ is graded-independent, so is $X_{r-1,r+\ell}$, so that by Theorem \ref{thm: maximality criteria} $X_{r-1,r+\ell}$ has a maximal unipotent radical. It now follows from Lemma \ref{lem: maximality and total nonsplitting}(a) that {\it all} of the extensions coming from the weight filtration on $X_{r-1,r+\ell}$ (i.e. all of the extensions of \eqref{eq50} with $X_{r-1,r+\ell}$ instead of $X$) are totally nonsplit, so that $ext(X_{r-1,r+\ell}, \phi)$ is totally nonsplit. Being isomorphic to $ext(X_{r-1,r+\ell}, \phi)$, the part of $(X_\db)$ to the above and right of $X_{r-1,r+\ell}$ is thus also totally nonsplit. This is true for all $r$, hence $(X_\db)$ is totally nonsplit.
\end{proof}
We are ready to give our classification result for graded-independent motives with maximal unipotent radicals. Denote the set of all totally nonsplit elements of $S_\ell(A)$ by $S^\ast_\ell(A)$. 
%Recall the notation: $A$ is the direct sum of $A_1$,$\ldots$, $A_k$ with the $A_j$ pure and in a strictly increasing order of weights; $S^\ast(A)$ is the set of isomorphism classes of motives with maximal unipotent radicals and associated graded isomorphic to $A$; and for each $1\leq \ell\leq k-1$, the notation $S^\ast_\ell(A)$ denotes the set of isomorphism classes of totally nonsplit generalized extensions of level $\ell$ of $A$. The bijection of Lemma \ref{lem: D_2}(b) restricts to a bijection
%\[
%S^\ast_1(A) \cong \, \biggm\{(\sE_r)\in \prod\limits_{r}Ext^1(A_{r+1},A_r): \text{each $\sE_r$ is totally nonsplit}\biggm\} \biggm/ \hspace{-.05in} Aut(A)
%\]
%(with the action of $Aut(A)$ given by pushforwards and pullbacks).
\begin{thm}\label{thm: classification of motives with give gr in graded independent case}
(a) We have a succession of maps
\[
S^\ast_{k-1}(A) \xrightarrow{\Theta_{k-1}}S^\ast_{k-2}(A) \xrightarrow{\Theta_{k-2}}S^\ast_{k-3}(A)\xrightarrow{\Theta_{k-3}} \cdots \xrightarrow{ \ \Theta_{3} \ } S^\ast_2(A) \xrightarrow{ \ \Theta_{2} \ }S^\ast_{1}(A)
\]
given by truncation. There is a natural bijection 
\begin{equation}\label{eq68}
S^\ast_1(A) \cong \, \biggm\{(\sE_r)\in \prod\limits_{r}Ext^1(A_{r+1},A_r): \text{each $\sE_r$ is totally nonsplit}\biggm\} \biggm/ \hspace{-.05in} Aut(A),
\end{equation}
with the action of $Aut(A)$ given by pushforwards and pullbacks.

\noindent (b) Suppose that $A$ is graded-independent. Then the following statements are true:
\begin{itemize}
\item[(i)] There is a canonical bijection 
\[
S^\ast(A) \cong S^\ast_{k-1}(A).
\]
\item[(ii)] For every $\ell$, every nonempty fiber of $\Theta_\ell: S_\ell^\ast(A)\rightarrow S_{\ell-1}^\ast(A)$ is a torsor for
\[
\prod\limits_{r} Ext^1(A_{r+\ell}, A_r).
\]
\item[(iii)] If the $Ext^2$ groups 
\[
Ext^2(A_{r+\ell}, A_r)
\]
vanish for all $1\leq r\leq k-\ell$, then $\Theta_\ell: S_\ell^\ast(A)\rightarrow S_{\ell-1}^\ast(A)$ is surjective.
\end{itemize}
%Suppose that $A$ is graded-independent. We have a succession of maps
%\[
%S^\ast_{k-1}(A) \xrightarrow{\Theta_{k-1}}S^\ast_{k-2}(A) \xrightarrow{\Theta_{k-2}}S^\ast_{k-3}(A)\xrightarrow{\Theta_{k-3}} \cdots \xrightarrow{ \ \Theta_{3} \ } S^\ast_2(A) \xrightarrow{ \ \Theta_{2} \ }S^\ast_{1}(A)
%\]
%given by truncation. For every $\ell$, every nonempty fiber of $\Theta_\ell: S_\ell^\ast(A)\rightarrow S_{\ell-1}^\ast(A)$ is a torsor for
%\[
%\prod\limits_{r} Ext^1(A_{r+\ell}, A_r).
%\]
%Moreover, there is a canonical bijection 
%\[
%S^\ast(A) \cong S^\ast_{k-1}(A).
%\]
%If the $Ext^2$ groups for $\bT$ all vanish, then every $\Theta_\ell$ in the diagram above is surjective.
\end{thm}
\begin{proof}
(a) The truncation of a totally nonsplit generalized extension is totally nonsplit, so the truncation maps between the $S_\ell(A)$ restrict to maps between the $S_\ell^\ast(A)$. The bijection \eqref{eq68} is the restriction of the bijection of Lemma \ref{lem: D_2}(b).

\noindent (b) The canonical bijection between $S^\ast(A)$ and $S^\ast_{k-1}(A)$ is the restriction of the bijection 
\begin{equation}\label{eq42}
S(A)\rightarrow S_{k-1}(A)
\end{equation}
of Lemma \ref{lem: D_k}(e), which is given by sending the isomorphism class of $X$ to the isomorphism class of the generalized extension $ext(X,\phi)$ associated with a pair $(X,\phi)$, where $\phi$ is any isomorphism $Gr^WX\rightarrow A$. The fact that \eqref{eq42} restricts to a map $S^\ast(A)\rightarrow S^\ast_{k-1}(A)$ is by Lemma \ref{lem: maximality and total nonsplitting}(a). The fact that this restricted map is surjective is by Theorem \ref{thm: maximality criteria}. 

By Lemma \ref{lem: when A is graded-independent, a gen ext is totally nonsplit if and only if its first level is} the fiber of $S^\ast_\ell(A)\rightarrow S^\ast_{\ell-1}(A)$ above an element $\epsilon$ is the same as the fiber of $S_\ell(A)\rightarrow S_{\ell-1}(A)$ above $\epsilon$. The assertion in (ii) now follow from Proposition \ref{prop: general thm on S(A) part b}(a). The assertion in (iii) follows from the last sentence of Proposition \ref{prop: general thm on S(A) part a}(b). (In fact, Proposition \ref{prop: general thm on S(A) part a}(b) gives us a more precise statement about the nonemptiness of each fiber.)
%about the torsor structure on  fibers is by Lemma \ref{lem: when A is graded-independent, a gen ext is totally nonsplit if and only if its first level is} and Proposition \ref{prop: general thm on S(A) part b}. The assertion about the surjectivity of the truncation maps is seen on recalling from Proposition \ref{prop: general thm on S(A) part b} that each fiber of $S_\ell(A)\rightarrow S_{\ell-1}(A)$, and hence by Lemma \ref{lem: when A is graded-independent, a gen ext is totally nonsplit if and only if its first level is}, each fiber of $S^\ast_\ell(A)\rightarrow S^\ast_{\ell-1}(A)$) is in a bijection with a product of $Extpan$ sets, and that such sets are always nonempty when all the $Ext^2$ groups vanish in the category. (In fact, we have a more precise statement about the nonemptiness of each fiber. See the last sentence of Proposition \ref{prop: general thm on S(A) part a}.)
\end{proof}
We note that the special case of Theorem \ref{thm: classification of motives with give gr in graded independent case} when $k=3$, $A_3=\mathbbm{1}$, and $Ext^1(\mathbbm{1}, A_1)=0$ was proved in \cite{EM2} (see \S 6.7 therein).

\subsection{Example: Mixed Tate motives with four weights}\label{sec: examples}
In \cite[\S 6.8-6.9]{EM2} we gave a homological classification of 3-dimensional graded-independent mixed Tate motives over $\QQ$ with three weights and maximal unipotent radicals. We also discussed some examples which were 4-dimensional with four weights, but we could not classify them up to isomorphism. In this final section, as an application of Theorem \ref{thm: classification of motives with give gr in graded independent case} we take $\bT$ to be the category of mixed Tate motives over $\QQ$ (say, of Voevodsky) and give a classification of isomorphism classes of all 4-dimensional graded-independent mixed Tate motives over $\QQ$ with four weights and maximal unipotent radicals. As we shall see, this will lead to some interesting questions about periods, building on those that arose in \cite{EM2}.

We may assume that the highest weight of our motives is 0. Thus we are interested in the isomorphism classes of motives $X$ over $\QQ$ whose associated graded is isomorphic to
\begin{equation}\label{eq45}
A := \QQ(a+b+c) \oplus \QQ(a+b) \oplus \QQ(a) \oplus \mathbbm{1},
\end{equation}
where $a,b$ and $c$ are positive integers. Such a motive is graded-independent if and only if $a,b,c$ are distinct and $a+b\neq c$. In view of Lemma \ref{lem: D_2} and the fact that the automorphism group of each $\QQ(n)$ is canonically isomorphic to $\QQ$ we easily see that
\begin{align*}
S_1(A) & \cong Ext^1(\QQ(a+b), \QQ(a+b+c))/\QQ^\times \, \times \, Ext^1(\QQ(a), \QQ(a+b))/\QQ^\times \, \times \, Ext^1(\mathbbm{1}, \QQ(a))/\QQ^\times \\
& \cong Ext^1(\mathbbm{1}, \QQ(c))/\QQ^\times \, \times \, Ext^1(\mathbbm{1}, \QQ(b))/\QQ^\times \, \times \, Ext^1(\mathbbm{1}, \QQ(a))/\QQ^\times,
\end{align*}
where in all of these the action of $\QQ^\times$ is via its action as the automorphism group of the object in the first entry of the corresponding Ext group via pushforward, which coincides with the action of $\QQ^\ast$ as the group of units of the scalar field of the vector space structure of the Ext group. Taking the orbits of nonsplit ( = totally nonsplit, in this particular situation) extensions in each factor above we obtain the subset $S_1^\ast(A)$ of $S_1(A)$ consisting of the totally nonsplit elements.

Before proceeding any further, let us recall the characterization of Ext groups for the category of mixed Tate motives over $\QQ$ (see \cite{DG05}, for instance). The $Ext^2$ groups all vanish, so that the $Extpan$ sets are always nonempty. We have
\[
\dim_\QQ Ext^1(\mathbbm{1}, \QQ(n)) = \begin{cases} 1 \hspace{.3in} &\text{if $n$ is odd and $> 1$}\\
0 & \text{otherwise} \end{cases}
\]
and 
\begin{equation}\label{eq44}
Ext^1(\mathbbm{1}, \QQ(1)) \cong \QQ^\times\otimes \QQ.
\end{equation}
All of these descriptions remain the same in the category of mixed Tate motives over $\ZZ$, with the exception of the description of $Ext^1(\mathbbm{1}, \QQ(1))$, which is zero for the category of mixed Tate motives over $\ZZ$.

For any odd integer $n>1$, the middle objects of all nonzero elements of $Ext^1(\mathbbm{1}, \QQ(n))$ are isomorphic (as all nonzero extensions are in the same $\QQ^\times$-orbit). We refer to this middle object (which is unique up to isomorphism) as the motive of $\zeta(n)$ and denote it by $Z_n$. Thanks to Deligne \cite{De89} (also see \cite{DG05} in the setting of Voevodsky motives), we know that after choosing suitable bases for Betti and de Rham realizations, the period matrix of $Z_n$ is
\[
\begin{pmatrix}
(2\pi i)^{-n} & (2\pi i)^{-n}\zeta(n)\\
0 & 1
\end{pmatrix}.
\]
The extensions of $\mathbbm{1}$ by $\QQ(1)$ are given by Kummer motives. Under the isomorphism \eqref{eq44}, for any rational $r>0$ the extension corresponding to $r\otimes 1$ arises from the weight filtration of the relative homology $H_1(\mathbb{G}_m, \{1,r\})$ (or equivalently, the 1-motive $[\ZZ\xrightarrow{1\mapsto r} \mathbb{G}_m]$, see \cite{De74}). We denote this motive by $L_r$ and refer to it as the motive of $\log(r)$; it has a period matrix of the form
\[
\begin{pmatrix}
(2\pi i)^{-1} &(2\pi i)^{-1}\log(r) \\
0 & 1
\end{pmatrix}.
\]
A complete set of (inequivalent) representatives of the nonzero orbits of $Ext^1(\mathbbm{1}, \QQ(1))/\QQ^\times$ consists of the extensions corresponding to the elements $r\otimes 1$, as $r$ runs through the set of rationals $>1$ that are not of the form $s^n$ for any $s\in\QQ$ and integer $n>1$. The $L_r$, as $r$ runs through said set, give a complete set of representatives for isomorphism classes of non-semisimple motives with associated graded  isomorphic to $\QQ(1)\oplus \mathbbm{1}$.

To simplify the notation, we shall allow a few instances of abuse of notation and terminology. Given an object $X$ whose associated graded is isomorphic to $\QQ(n)\oplus \mathbbm{1}$, there is a well-defined element of $Ext^1(\mathbbm{1},\QQ(n))/\QQ^\times$ associated with it. We will use the same notation for the object and this associated element. We will also use the same notation for a non-pure motive and its twists by Tate motives (i.e. for $X$ and $X(n)$, as long as $X$ is not pure). The Tate twist in question will always be clear from our context. Finally, we might speak about (say) $Z_n$ as an extension, in which case we will mean an extension of $\mathbbm{1}$ by $\QQ(n)$ with $Z_n$ as the middle object, or a Tate twist of this extension. 

We now return to the problem of classifying the isomorphism classes of motives with maximal unipotent radicals and associated graded isomorphic to \eqref{eq45}, where $a,b,c$ are distinct positive integers and $a+b\neq c$. If we want, we may also restrict ourself to one of the cases $c>a$ or $a>c$, as one case can be transformed to the other via dualization followed by appropriate Tate twists.

By Theorem \ref{thm: classification of motives with give gr in graded independent case} we have truncation maps
\begin{equation}\label{eq46}
S^\ast(A) \cong S^\ast_3(A) \twoheadrightarrow S^\ast_2(A) \twoheadrightarrow S^\ast_1(A),
\end{equation}
which are surjective because the $Ext^2$ groups vanish. Denoting the set of nonsplit extensions of $\mathbbm{1}$ by $\QQ(n)$ by $Ext^1(\mathbbm{1}, \QQ(n))^\ast$, we have
\begin{equation}\label{eq70}
S^\ast_1(A) \cong Ext^1(\mathbbm{1}, \QQ(c))^\ast/\QQ^\times \, \times \, Ext^1(\mathbbm{1}, \QQ(b))^\ast/\QQ^\times \, \times \, Ext^1(\mathbbm{1}, \QQ(a))^\ast/\QQ^\times.
\end{equation}
Since $Ext^1(\mathbbm{1},\QQ(n))$ vanishes when $n$ is even and is nonzero otherwise for $n>0$, the set $S_1^\ast(A)$ and hence $S^\ast(A)$ is nonempty if and only if $a$, $b$, $c$ are all odd.

Fix odd $a,b,c$ (which are distinct, positive and satisfy $a+b\neq c$). 

\underline{Case I}: Suppose $1\notin\{ a,b,c\}$. Then $S^\ast_1(A)$ is a singleton. The set $S^\ast_2(A)$ is a torsor over 
\[
Ext^1(\mathbbm{1},\QQ(a+b)) \times Ext^1(\QQ(a),\QQ(a+b+c)),
\]
which is zero by the description of $Ext^1$ groups and our parity conditions. In other words, $S^\ast_2(A)$ is also a singleton. It consists of the class of a generalized extension of the form
\[
\begin{array}{cccc}
\QQ(a+b+c)&&&\\
Z_c&\QQ(a+b)&&\\
Z_{c,b}&Z_b&\QQ(a)&\\
&Z_{b,a}&Z_a&\mathbbm{1},
\end{array}
\]
where we have dropped the arrows from the writing and as mentioned earlier, are not keeping track of Tate twists in the notation for the non-pure motives. The motives called $Z_{b,a}$ and $Z_{c,b}$ are respectively, the motives (unique up to isomorphism) that fit in a blended extension of $Z_a$ by $Z_b$ and a blended extension of $Z_b$ by $Z_c$. The motive $Z_{b,a}$ has a period matrix of the form
\[
\begin{pmatrix}
(2\pi i)^{-a-b} & (2\pi i)^{-a-b} \zeta(b) & (2\pi i)^{-a-b}z_{b,a}\\
& (2\pi i)^{-a} & (2\pi i)^{-a}\zeta(a) \\
&& 1
\end{pmatrix},
\]
where $z_{b,a}$ (not uniquely defined, as it can change according to the choice of bases) is another period. By Theorem \ref{thm: maximality criteria} (or its special case \cite[Corollary 6.7.1]{EM2}), the objects on the lowest diagonal of the generalized extension above have maximal unipotent radicals, and hence the motivic Galois group of each has dimension 4 ( = ${3\choose 2}+1$). Thus Grothendieck's period conjecture predicts that $z_{b,a}$ , $\zeta(b)$, $\zeta(a)$, and $\pi$ form an algebraically independent set. Similarly, the four numbers $z_{c,b}$ , $\zeta(c)$, $\zeta(b)$, and $\pi$ should be algebraically independent.

We now consider completions of the generalized extension class above to one of level 3. The fiber of $S^\ast(A)\rightarrow S_2^\ast(A)$, which is simply all of $S^\ast(A)$ because $S_2^\ast(A)$ is a singleton, is a torsor over 
\[
Ext^1(\mathbbm{1}, \QQ(a+b+c))\simeq \QQ \, .\footnote{Note that the torsor structure involves making a choice, namely of a totally nonsplit generalized extension of level 2; that is, a choice of the arrows in the earlier diagram of level 2. See Proposition \ref{prop: general thm on S(A) part b}.}
\]
In particular, we have a noncanonical bijection $S^\ast(A)\simeq \QQ$. 

A discussion of the periods is in order. After choosing suitable bases for Betti and de Rham realizations of (a motive in the isomorphism class) $\epsilon\in S^\ast(A)$, the period matrix of $\epsilon$ is of the form
\[
\begin{pmatrix}
(2\pi i)^{-a-b-c} & (2\pi i)^{-a-b-c} \zeta(c) & (2\pi i)^{-a-b-c}z_{c,b} & (2\pi i)^{-a-b-c}z_{c,b,a}(\epsilon) \\
& (2\pi i)^{-a-b} & (2\pi i)^{-a-b}\zeta(b) & (2\pi i)^{-a-b}z_{b,a}\\
&& (2\pi i)^{-a} & (2\pi i)^{-a} \zeta(a)\\
&&&1
\end{pmatrix}.
\]
(Again, $z_{c,b,a}(\epsilon)$ is only well-defined to the extent allowed by its dependence on the choice of bases.)

The motivic Galois group of $\epsilon$ has a maximal unipotent radical, so that it has dimension 7. Thus Grothendieck's period conjecture predicts that $z_{c,b,a}(\epsilon), z_{c,b}, z_{b,a}$, the zeta values in the matrix, and $\pi$ are algebraically independent. Since all of the conversation above can be thought of as being taken place in the category of mixed Tate motives over $\ZZ$ (as we did not use motives of logarithms in the process), by a theorem of Brown \cite{Br12} all of the above unknown periods are in the algebra generated by multiple zeta values and $\pi$. One should be able to find the unknown periods using Brown's theory of motivic multiple zeta values.

As a final comment in this case, we add that by Theorem \ref{thm: Ext groups for non-consecutive graded pieces vanish in the graded-independent situation}, for any $\epsilon\in S^\ast(A)$ we have
%for any motive $X$ in the isomorphism class $\epsilon\in S^\ast(A)$,
\[
Ext^1_{\langle \epsilon \rangle} (\mathbbm{1}, \QQ(a+b+c)) = 0.
\]
Here, $\langle \epsilon \rangle$ means the subcategory generated by any object in $\epsilon$, or equivalently, by all objects in $\epsilon$. It follows that $\zeta(a+b+c)$ should also be algebraically independent from the set of 7 aforementioned periods $z_{c,b,a}(\epsilon)$, $z_{c,b}$, etc. Indeed, this is because $Z_{a+b+c}$ is not in $\langle \epsilon \rangle$, so that the surjection from the motivic Galois group of $\epsilon \oplus Z_{a+b+c}$ (where with abuse of notation, $\epsilon$ refers to a motive in the isomorphism class) to the motivic Galois group of $\epsilon$ is not an isomorphism. The two groups have the same maximal reductive quotients and hence the unipotent radical of the former must be strictly larger than the latter's, so that the dimension of the motivic Galois group of $\epsilon\oplus Z_{a+b+c}$ is strictly larger than that of $\epsilon$. 

It will be very interesting to understand the role that $\zeta(a+b+c)$ plays in the formation of $z_{c,b,a}(\epsilon)$ as $\epsilon$ varies. 

\underline{Case II}: Suppose $1\in \{a,b,c\}$. This is the more interesting case, as it involves the motives $L_r$ of logarithms and hence the periods that arise may not be multiple zeta values or even their cyclotomic analogues.

We will consider the case where $b=1$. The other two cases, which are related to one another by duality and Tate twists, can be considered similarly. (See \S \ref{sec: intro details of application to motives with maximal u} for an example with $a=1$.)

Again we start forming our generalized extensions from the smallest level, working backwards through the maps of \eqref{eq46}. The set $S^\ast_1(A)$ is no longer a singleton; instead, it is in a one-to-one correspondence with the set of isomorphism classes of Kummer motives. More precisely, from \eqref{eq70} we have
\[
S_1^\ast(A) \cong \{Z_c\}\times  \{L_r \}_r \times  \{Z_a\},
\]
where the $L_r$, as described earlier, form a set of representatives for isomorphism classes of Kummer mortives. Once we fix $r$, the lifting to $S_2^\ast(A)$ involves no choices (as in Case I). There is a unique element of $S_2^\ast(A)$ with $Z_c$, $L_r$, and $Z_a$ on its level 1 diagonal:
\begin{equation}\label{eq47}
\begin{array}{cccc}
\QQ(a+1+c)&&&\\
Z_c &\QQ(a+1)&&\\
M'_{c,r} &L_r&\QQ(a)&\\
&M_{a,r}&Z_a&\mathbbm{1}.
\end{array}
\end{equation}
Here, $M_{a,r}$ (resp. $M'_{c,r}$) is the object (unique up to isomorphism) which fits as the middle object of a blended extension of $Z_a$ by $L_r$ (resp. $L_r$ by $Z_c$). The object $M_{a,r}$ is the unique mixed Tate motive over $\QQ$ with associated graded isomorphic to $\QQ(a+1)\oplus\QQ(a)\oplus\mathbbm{1}$ and a maximal unipotent radical such that its corresponding extension of $\QQ(a)$ by $\QQ(a+1)$ is a Tate twist of the motive of $\log(r)$. There is a similar description for $M'_{c,r}$ (now with associated graded a twist of $\QQ(c+1)\oplus \QQ(1)\oplus \mathbbm{1}$). By their uniqueness properties, $M'_{c,r}$ is isomorphic to a Tate twist of the dual to $M_{c,r}$. We refer the reader to \cite[\S 6.8]{EM2} for a more detailed discussion of these motives. (The notation here for $M_{a,r}$ and $M'_{c,r}$ is consistent with that in {\it loc. cit.}, except for a slight difference in the indices.)

The period matrix of $M_{a,r}$ with respect to suitable choices of bases is of the form
\[
\begin{pmatrix}
(2\pi i)^{-a-1} & (2\pi i)^{-a-1} \log(r) & (2\pi i)^{-a-1}p_{a,r} \\
& (2\pi i)^{-a} & (2\pi i)^{-a}\zeta(a) \\
&& 1
\end{pmatrix}.
\]
The entry $p_{a,r}$ is a period which assuming Grothendieck conjecture, together with $\zeta(a)$, $\log(r)$, and $\pi$ form an algebraically independent set. Here, due to the presence of the motive of $\log(r)$ in the construction, the motive $M_{a,r}$ is not a mixed Tate motive over $\ZZ$, but rather one over $\ZZ[1/r]$ (i.e. with good reduction outside $r$). In the cases\footnote{Deligne's work is for $r=2,3,4,6,8$ but here we are restricted to values of $r$ that are not powers of another rational number.} $r=2,3,6$, Deligne has proved in \cite{De10} that the category of mixed Tate motives over $\mathcal{O}_{\QQ(\mu_r)}[1/r]$ (i.e. the full subcategory of the category of mixed Tate motives over $\QQ(\mu_r)$ consisting of objects that are unramified outside $r$) is generated by the motivic fundamental group of $\mathbb{G}_m-\mu_r$. This implies that the unknown period $p_{a,r}$ in these cases is in the algebra generated by $\pi$ and cyclotomic multiple zeta values (see \cite{De10} and \cite{DG05}). However, as Deligne explains in the Introduction of \cite{De10}, these cases of $r$ are in fact exceptional: if $r$ is a positive integer that has a prime divisor $\geq 5$, Goncharov \cite{Go01} has shown that the motivic fundamental group of $\mathbb{G}_m-\mu_r$ does not generate the category of mixed Tate motives over $\mathcal{O}_{\QQ(\mu_r)}[1/r]$. The upshot of these remarks is that when $r\notin\{2,3,6\}$, the nature of $p_{a,r}$ and other unknown periods that appear below seems more mysterious.

The fiber of $S^\ast(A)\rightarrow S^\ast_2(A)$ above the class of \eqref{eq47} is a torsor over 
\[
Ext^1(\mathbbm{1}, \QQ(a+1+c))\simeq \QQ.
\]
Thus, having fixed $r$, we obtain a collection of non-isomorphic objects parametrized (noncanonically) by the Ext group above. Each $\epsilon\in S^\ast(A)$ above \eqref{eq47} (after making the relevant choices) has a period matrix of the form
\[
\begin{pmatrix}
(2\pi i)^{-a-1-c} & (2\pi i)^{-a-1-c} \zeta(c) & (2\pi i)^{-a-1-c}p'_{c,r} & p_{a,r,c}(\epsilon)  \\ 
& (2\pi i)^{-a-1} & (2\pi i)^{-a-1} \log(r) & (2\pi i)^{-a-1}p_{a,r}\\
&& (2\pi i)^{-a} & (2\pi i)^{-a} \zeta(a)\\
&&&1
\end{pmatrix}.
\]
Assuming Grothendieck's period conjecture, the numbers $p_{a,r,c}(\epsilon)$, $p_{a,r}$, $p'_{c,r}$ (closely related to $p_{c,r}$), $\zeta(c)$, $\zeta(a)$, $\log(r)$, and $\pi$ are algebraically independent. 

As pointed out earlier, the nature of the unknown periods is rather mysterious. It would be very interesting to somehow compute them. We should point out that a geometric construction of these motives (and more generally, of non-semisimple mixed Tate motives with prescribed associated graded with few weights) seems out of reach at the moment. See \cite{Br16} for more details.

Theorem \ref{thm: Ext groups for non-consecutive graded pieces vanish in the graded-independent situation} implies that for any $\epsilon\in S^\ast(A)$ above \eqref{eq47}, there are no nontrivial extensions of $\mathbbm{1}$ by $\QQ(a+1+c)$ in the category generated by (any or all objects of) $\epsilon$. Thus as in case I, Grothendieck's period conjecture predicts that $\zeta(a+1+c)$ and the 7 numbers $p_{a,r,c}(\epsilon)$, $p_{a,r}$, etc. are algebraically independent.

\appendix
\section{$Ext^1$ groups and the abelianization of the unipotent radical}
\numberwithin{thm}{section}
Let $G$ be an algebraic group over a field $K$ of characteristic zero. Let $\mathbf{Rep}(G)$ be the category of finite-dimensional (algebraic) representations of $G$ over $K$. We denote the $Ext^1$ groups for $\mathbf{Rep}(G)$ by $Ext^1_G$. Our goal in this appendix is to give an interpretation of the spaces
\[
Ext^1_G(\mathbbm{1}, N)
\]
for semisimple objects $N$ of $\mathbf{Rep}(G)$ (satisfying an extra condition, we will see shortly) in terms of the unipotent radical of $G$. Such an interpretation has been used frequently in the case where the maximal reductive quotient of $G$ is the multiplicative group $\mathbb{G}_m$, which is the relevant case for motivic Galois groups of mixed Tate motives (e.g. see \cite[\S A.13]{DG05}). But an analogous interpretation for arbitrary maximal reductive quotients seems to be missing from the literature.

In what follows we use the same notation for a representation and its underlying vector space. The notation $Hom_K$ refers to a Hom group in the category of vector spaces over $K$.

Suppose $G=U\rtimes R$, where $U$ is unipotent and $R$ is reductive (so that $U$ is the unipotent radical of $G$). The two maps $G\twoheadrightarrow R$ and $R\hookrightarrow G$ respectively induce (restriction) functors $\mathbf{Rep}(R)\rightarrow \mathbf{Rep}(G)$ and $\mathbf{Rep}(G)\rightarrow \mathbf{Rep}(R)$. Via the former we identify $\mathbf{Rep}(R)$ as a full subcategory of $\mathbf{Rep}(G)$; it is the full subcategory whose objects are exactly all the semisimple objects of $\mathbf{Rep}(G)$. An object of $\mathbf{Rep}(G)$ is in $\mathbf{Rep}(R)$ if and only if $U$ acts trivially on its underlying vector space. The composition
\[
\mathbf{Rep}(R)\rightarrow \mathbf{Rep}(G) \rightarrow \mathbf{Rep}(R)
\]
is the identity.

Let $\fu$ be the Lie algebra of $U$. The adjoint representation of $G$ makes $\fu$ an object of $\mathbf{Rep}(G)$. Similarly, the abelianization $\fu^{ab}:=\fu/[\fu, \fu]$ is also an object of $\mathbf{Rep}(G)$, with the quotient map $\fu\rightarrow \fu^{ab}$ a morphism of $G$-representations. In fact, since the action of $U$ on $\fu^{ab}$ is trivial, $\fu^{ab}$ belongs to the subcategory $\mathbf{Rep}(R)$ and is semisimple.

Before we proceed any further, let us make a comment about morphisms from $G$ to other algebraic groups. For any algebraic group $G'$ over $K$, the data of a morphism of algebraic groups $\phi: G\rightarrow G'$ is equivalent to the data of morphisms of algebraic groups $\phi_U: U\rightarrow G'$ and $\phi_R: R\rightarrow G'$ that are compatible with one another, in the sense that
\[
\phi_R(r)\phi_U(u)\phi_R(r)^{-1} = \phi_U(rur^{-1}) \hspace{.3in}(\forall u\in U, \forall r\in R).
\]
The passage from $\phi$ to $(\phi_U,\phi_R)$ is by taking the restrictions of $\phi$ to $U$ and $R$. The passage in the opposite direction is by taking $\phi$ to be given by $\phi(ur)=\phi_U(u)\phi_R(r)$ for any $u\in U$ and $r\in R$.

%Let $\mathbf{Rep}(G)$ (resp. $\mathbf{Rep}(R)$) be the category of finite-dimensional (algebraic) representations of $G$ (resp. $R$) over $K$. We use the same notation for a representation of $G$ or $R$ and its underlying vector space. We denote the $Ext^1$ groups for $\mathbf{Rep}(G)$ by $Ext^1_G$, and use the notation $Hom_K$ for the Hom groups in the category of vector spaces over $K$.

Let $N$ be a semisimple object of $\mathbf{Rep}(G)$ which does not contain a copy of $\mathbbm{1}$. We will give an interpretation of
\[
Ext^1_G(\mathbbm{1}, N).
\]
Consider an extension of $\mathbbm{1}$ by $N$ in $\mathbf{Rep}(G)$:
\begin{equation}\label{eq6}
\begin{tikzcd}
   0 \arrow[r] & N \arrow[r, ] & E \arrow[r, ] & \mathbbm{1}  \arrow[r] & 0.
\end{tikzcd}
\end{equation}
Since $R$ is reductive, this admits a $R$-equivariant section, which is in fact unique because there are no nonzero maps from $\mathbbm{1}$ to $N$ in $\mathbf{Rep}(R)$ (or equivalently, in $\mathbf{Rep}(G)$). Use this section to identify $E=N\oplus \mathbbm{1}$ as an $R$-representation, where $E$ is considered an $R$-representation via the restriction of the $G$-action to the subgroup $R$. Expressing the action of $U$ on $E$ (obtained by restricting the $G$-action) in terms of this decomposition of $E$, in light of the fact that the actions of $U$ on $N$ and $\mathbbm{1}$ are trivial (the former thanks to the semisimplicity assumption), we obtain a morphism
\[
\psi: U \rightarrow Hom_K(\mathbbm{1}, N)\cong N
\]
of algebraic groups, where $Hom_K(\mathbbm{1}, N)$ and $N$ are considered as additive algebraic groups over $K$. One can check that compatibility of the $U$-action on $E$ with the $R$-action on $E$ translates in terms of the morphism $\psi$ to $R$-equivariance of $\psi$, where the action of $R\subset G$ on $U$ is by conjugation and its action on $N$ is the defining action that makes $N$ a semisimple $G$-representation (which is the same as the restriction the natural $G$-action to the subgroup $R$).

In view of the assumption that there are no nonzero morphisms from $\mathbbm{1}$ to $N$, we see easily that replacing \eqref{eq6} with an equivalent extension does not change the morphism $\psi: U\rightarrow N$. Thus we have a map 
\begin{equation}\label{eq5}
Ext^1_G(\mathbbm{1}, N) \rightarrow Hom_{\text{$R$-equiv.}}(U,N) \hspace{.3in} \text{class of \eqref{eq6}}\mapsto \psi.
\end{equation}
Here, $Hom_{\text{$R$-equiv.}}(U,N)$ denotes the set of $R$-equivariant morphisms of algebraic groups from $U$ to $N$. 

In light of the fact that $N$ is an additive algebraic group, the set $Hom_{\text{$R$-equiv.}}(U,N)$ carries a natural structure of a vector space over $K$. On recalling the vector space structure of $Ext^1_G(\mathbbm{1}, N)$ (with addition and scalar multiplication respectively given by Baer summation and pushforwards along scalar maps), it is not difficult to verify that \eqref{eq5} is linear. In fact, it is an isomorphism of vector spaces, with its inverse obtained as follows: Consider the exact sequence of $R$-representations
\begin{equation}\label{eq4}
\begin{tikzcd}
   0 \arrow[r] & N \arrow[r, ] & N\oplus \mathbbm{1}  \arrow[r, ] & \mathbbm{1}  \arrow[r] & 0
\end{tikzcd}
\end{equation}
with the inclusion and projection maps. Given an $R$-equivariant morphism of algebraic groups $\psi: U\rightarrow N$ (with actions of $R$ as described above), we use it in the natural way to define a $U$-action on $E:=N\oplus \mathbbm{1}$: an element $u\in U$ acts on $E$ by left multiplication by the matrix
\[
\begin{pmatrix}
Id_N & \psi(u)\\
0 & Id_\mathbbm{1}\end{pmatrix}\in GL(N\oplus \mathbbm{1}).
\]
This makes \eqref{eq4} a sequence of $U$-representations. The $R$-equivariance of $\psi$ guarantees that the action of $U$ on $E$ is compatible with the direct sum action of $R$. Thus $E$ (resp. the sequence \eqref{eq4}) becomes a $G$-representation (resp. an extension of $G$-representations). Sending $\psi$ to the class of \eqref{eq4} in $Ext^1_G(\mathbbm{1},N)$ we obtain the inverse of \eqref{eq5}. 

In view of the facts that $U$ and $N$ are unipotent groups, $K$ has characteristic zero, and $N$ is abelian, on passing to the Lie algebras of $U$ and $N$ the isomorphism \eqref{eq5} gives an isomorphism between $Ext^1_G(\mathbbm{1}, N)$ and the space of $R$-equivariant linear maps $\fu^{ab}\rightarrow N$, where the actions of $R$ on $\fu^{ab}$ and $N$ are the natural ones. We obtain the following statement:
%Lie algebra maps $\fu\rightarrow N$, where the action of $R$ on $\fu$ and $N$ are the natural ones (on $\fu$, this is the restriction of the adjoint action of $G$), and $N$ is considered as an abelian Lie algebra. Since $N$ is abelian, such Lie algebra maps factor through $\fu^{ab}$. We obtain a bijection between $Ext^1_G(\mathbbm{1}, N)$ and the set of $R$-equivariant linear maps $\fu^{ab}\rightarrow N$. Since both $\fu^{ab}$ and $N$ belong to the subcategory $\mathbf{Rep}(R)$ of $\mathbf{Rep}(G)$, we obtain the following statement:
\begin{prop}\label{prop appendix: Ext groups and maps from the abelianization of u}
As above, let $G$ be an algebraic group over a field of characteristic zero, and $G=U\rtimes R$ where $U$ is unipotent and $R$ is reductive. Let $\fu$ be the Lie algebra of $U$, considered as an object of $\mathbf{Rep}(G)$ through the adjoint action, and let $\fu^{ab}$ be the abelianization of $\fu$, also an object of $\mathbf{Rep}(G)$. Then for every semisimple object $N$ of $\mathbf{Rep}(G)$ with $Hom(\mathbbm{1}, N) \cong 0$, there is a canonical isomorphism
\[
Ext^1_G(\mathbbm{1}, N) \rightarrow Hom(\fu^{ab}, N).
\]
(In both instances, $Hom:=Hom_{\mathbf{Rep}(G)}=Hom_{\mathbf{Rep}(R)}$.)
\end{prop}

\begin{thebibliography}{00}

\bibitem{An92}
Y. Andr\'{e}, 
Mumford-Tate groups of mixed Hodge structures and the theorem of the fixed part, 
Compositio Mathematica, Volume 82 (1992) no. 1, p. 1-24

\bibitem{An04}
Y. Andr\'{e},
Une introduction aux motifs,
Soci\'{e}t\'{e} Math\'{e}matique de France, 2004

%
\bibitem{Ay14}
J. Ayoub, 
L’alg\`{e}bre de Hopf et le groupe de Galois motiviques d’un corps de caract\'{e}ristique nulle, 
J. Reine Angew. Math. 693 (2014), partie I: 1-149 ; partie II: 151-226
%
\bibitem{Bei83}
A. A. Beilinson, 
Notes on absolute Hodge cohomology, 
Applications of Algebraic K-theory to Algebraic Geometry and Number Theory, Part I, Proceedings of a Summer Research Conference held June 12-18, 1983, in Boulder, Colorado, Contemporary Mathematics 55, American Mathematical Society, Providence, Rhode Island, pp. 35–68
%
\bibitem{Be02}
C. Bertolin,
The Mumford-Tate group of 1-motives,
Ann. Inst. Fourier, Grenoble, 52, 4 (2002), 1041-1059

\bibitem{Be03}
C. Bertolin,
Le radical unipotent du groupe de Galois motovique d'un 1-motif,
Math. Ann. 327, 585-607 (2003)
%

\bibitem{Be20}
C. Bertolin (with a letter of Y. Andr\'{e} and an appendix by M. Waldschmidt),
Third kind elliptic integrals and 1-motives, 
J. Pure Appl. Algebra 224 (2020), no. 10, 106396, 28 pp. 

\bibitem{Ber01}
D. Bertrand,
Unipotent radicals of differential Galois group and integrals of solutions of inhomogeneous equations,
Math. Ann. 321 (2001), no. 3, 645–666
%
%\bibitem{Ber98}
%D. Bertrand, 
%Relative splitting of 1-motives,
%Contemporary Mathematics, Vol. 210, 1998
%
\bibitem{Ber13}
D. Bertrand,
Extensions panach\'{e}es autoduales,
J. K-Theory 11 (2013), no. 2, 393–411
%

\bibitem{BVK16}
L. Barbieri-Viale and B. Kahn,
On the derived category of 1-motives,
Ast\'{e}risque (2016), no. 381, 254 pp


%
\bibitem{Br12}
F. Brown,
Mixed Tate motives over $\ZZ$,
Annals of Mathematic (175), 2012, 949-976
%
%
\bibitem{Br16}
F. Brown,
Irrationality proofs for zeta values, moduli spaces and dinner parties,
Mosc. J. Comb. Number Theory 6 (2016), 102-165
%
%
\bibitem{Bu59}
D. A. Buchsbaum,
A note on homology in categories,
Annals of Mathematics, Second Series, Vol. 69, No. 1 (Jan., 1959), pp. 66-74

%
%
%\bibitem{Ca80}
%J. A. Carlson,
%Extensions of mixed Hodge structures,
%Journ\'{e}es de G\'{e}ometrie Alg\'{e}brique d'Angers, Juillet 1979/Algebraic Geometry, Angers, 1979, pp. 107–127, Sijthoff \& Noordhoff, Alphen aan den Rijn, Md., 1980.
%
%
%\bibitem{ChGA}
%U. Choudhury and M. Gallauer Alves de Souza, 
%An isomorphism of motivic Galois groups,
%Adv. Math. 313 (2017), 470-536
%
%
\bibitem{De74}
P. Deligne,
Theorie de Hodge III,
Publications Math\'{e}matiques de l'I.H.\'{E}.S., tome 44 (1974), p. 5-77
%
%
\bibitem{DM82}
P. Deligne and J.S. Milne,
Tannakian Categories,
In Hodge Cycles, Motives, and Shimura Varieties,
Lecture Notes in Mathematics 900, Springer-Verlog, Berlin (1982)
%
\bibitem{De89}
P. Deligne,
Le groupe fondamental de la droite projective moins trois points,
Galois Groups over $\QQ$, MSRI Publ. 16, pp. 79-313, Springer-Verlag (1989)
%
%\bibitem{De90}
%P. Deligne,
%Cat\'{e}gories tannakiennes,
%Grothendieck Festschrift Vol II, Progress in Mathematics, 87 ( Birkhäuser Boston 1990) pp. 111–195
%
%\bibitem{De94}
%P. Deligne,
%Structures de Hodge mixtes r\'{e}elles, 
%Proceedings of Symposia in Pure Mathematics, Volume 55 (1994), Part I
%
%
\bibitem{DG05}
P. Deligne and A. B. Goncharov,
Groupes fondamentaux motiviques de Tate mixte,
{\it Ann. Sci. de l’\'{E}cole Norm. Sup.}
38 (2005) 1-56 
%
%
\bibitem{De10}
P. Deligne,
Le groupe fondamental motivique de $G_m-\mu_N$ pour $N=2,3,4,6$ ou $8$.
Publications Math\'{e}matiques de l'IH\'{E}S. 112 (2010) pp. 101-141
%
%
\bibitem{DW16}
I. Dan-Cohen and S. Wewers, 
Mixed Tate motives and the unit equation,
Int. Math. Res. Not., 2016, no. 17.
%
%\bibitem{Dunham}
%W. Dunham, Euler and the Cubic Basel Problem, {\em The American Mathematical Monthly}, 128(2021), 291-301.
%
%
%
\bibitem{EM1}
P. Eskandari and V. K. Murty,
The fundamental group of an extension in a Tannakian category and the unipotent radical of the Mumford-Tate group of an open curve, to appear in the Pacific Journal of Mathematics


\bibitem{EM2}
P. Eskandari, V. K. Murty,
On unipotent radicals of motivic Galois groups,
Algebra \& Number Theory 17-1 (2023), pp 165-215


\bibitem{Es23}
P. Eskandari,
On endomorphisms of extensions in tannakian categories,
preprint, arXiv:2306.06817

%
%\bibitem{Euler}
%L. Euler, De relatione inter ternas pluresve quantitates instituenda (E591), Opusc. Anal. 2(1785), 91?101. 
%Reprinted in Opera Omnia. Ser. 1, Vol. 4: 136-145.
%

\bibitem{Fe20}
R. Ferrario,
A complex analogue of Grothendieck's section conjecture,
PhD thesis, ETH Z\"{u}rich, 2020


\bibitem{Go95}
A. B. Goncharov,
Polylogarithms in arithmetic and geometry,
In Proceedings of the International Congress of Mathematicians, Vol. 1, 2 (Z\"{u}rich, 1994), pages 374-387, Basel, 1995, Birkh\"{a}user.

\bibitem{Go01}
A. B. Goncharov, 
The dihedral Lie algebra and Galois symmetries of $\pi_1^{(l)}(\mathbf{P}^1-(\{0,\infty\}\cup \mu_N))$,
Duke Math. Journal, 110 (2001), 397–487

%\bibitem{Go05}
%A. B. Goncharov,
%Galois symmetries of fundamental groupoids and noncommutative geometry,
%Duke Math. Journal, 128, no. 2 (2005) p. 209-284
%
%
\bibitem{Gr68}
A. Grothendieck,
Mod\`{e}les de N\'{e}ron et monodromie,
SGA VII.1, no 9, Springer LN 288, 1968

%
\bibitem{Har06}
C. Hardouin,
Hypertranscendance et Groupes de Galois aux diff\'{e}rences, arXiv 0609646v2, 2006

\bibitem{Har11}
C. Hardouin,
Unipotent radicals of Tannakian Galois groups in positive characteristic, 
Arithmetic and Galois theories of differential equations, 223-239, S\'{e}min. Congr., 23, Soc. Math. France, Paris, 2011
%
\bibitem{HM17}
A. Huber and S. M\"{u}ller-Stach,
Periods and Nori Motives,
Springer, 2017

%\bibitem{JR86}
%O. Jacquinot and K. Ribet,
%Deficient points on extensions of abelian varieties by G,
%J. Number Theory, 24, No. 3 (1986).
%
\bibitem{Ja90}
U. Jannsen,
Mixed motives and algebraic K-theory,
with appendices by S. Bloch and C. Schoen,
Lecture Notes in Mathematics, Vol. 1400,
Springer-Verlag, Berlin, 1990.
%
\bibitem{Ja94}
U. Jannsen,
Motivic sheaves and filtrations on Chow groups,
Proceedings of Symposia in Pure Mathematics,
Volume 55 (1994), Part 1
%
\bibitem{Jo14}
P. Jossen,
On the Mumford-Tate conjecture for 1-motives,
Inventiones Math. (2014) 195: 393-439
%
%
%
%\bibitem{Mi17}
%J. S. Milne,
%Algebraic Groups,
%Cambridge University Press, 2017
%
\bibitem{Ne94}
J. Nekovar, 
Beilinson's conjectures, 
in Motives (Seattle, WA, 1991), 537 - 570, Proc. Symp. Pure Math., 55/I, Amer. Math. Soc., Providence, RI, 1994
%
%

\bibitem{RSZ}
J-P. Ramis, J. Sauloy, C. Zhang,
Local analytic classification of $q$-difference equations,
Ast\'{e}risque (2013), no. 355, vi+151 pp

%
%\bibitem{Riv72}
%N. Saavedra Rivano.
%Cat\'{e}gories Tannakiennes,
%Lecture Notes in Mathematics 265
%
%
%
\bibitem{Yo60}
N. Yoneda,
On Ext and exact sequences,
Journal of the Faculty of Science, Imperial University of Tokyo, Vol. 8, 1960,  507-576

\end{thebibliography}

\end{document}
