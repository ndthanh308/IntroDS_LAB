% \section{Introduction}
% \label{s:intro}

\chapteri{S}ustainability is the practice of performing human
activities in ways that do not leave lasting harmful
effects~\cite{un22}.
Unfortunately, the harm to the planet is clearly growing,
whether the effects are direct (\eg emissions
caused by transportation, farming, or manufacturing) or indirect
(\eg
carbon emissions due to electricity consumed by data centers
and even the energy and materials used for
manufacturing servers and other devices). 
Humans as a species have understood
that sustainability is important to both future generations and the
global quality of life. Yet, we have had only sporadic and uneven
adoption of sustainable practices, and up to 98\% of sustainability
initiatives fail to meet their goals~\cite{dst16}. 
The impacts of a
lack of sustainability have led to---among many other
factors---climate change, widespread pollution of the oceans, sea
bottom desertification,
acidification of land and water, ozone loss, desertization, and loss
of biodiversity. Failure to address this lack of sustainability now
will create long-term problems for future generations~\cite{ipcc22}.

Today, achieving the goals of sustainability requires the honest, best
efforts of humans and an apparatus to measure aspects of the system
under regulation.  Yet, those efforts often fail when bad actors
bypass or cheat sustainability systems.  For example, the car company
Volkswagen installed emissions software on roughly 11 million cars
worldwide that misled the Environmental Protection Agency (EPA) about
emissions when under
test~\cite{vwscandal2015}.  Volkswagen was eventually caught, fined
billions of dollars, and required to recall vehicles and pay financial
settlements---but only \emph{after} the vehicles had polluted for
nearly a decade.

One area with unprecedented impact on our world is the use of
computation and in particular data centers.
With the alarming rise of computation and the pervasive use of
artificial intelligence (\eg ChatGPT~\cite{llm-energy}), data centers pose many negative impacts
on the environment caused by energy use, hardware
manufacturing and disposal, building maintenance, and other factors.  Indeed, a
recent study showed that over 2--4\% of all energy used worldwide was by
data centers~\cite{masanet20, iea}.
%
The current practice of reporting
sustainability information in data centers is, however, mired with
``greenwashing,'' where the true carbon footprint of a data center
is artificially reduced via the purchase of energy from green generation
sources~\cite{drec-initiative} or by paying other entities to be
sustainable.
%
This signifies a lack of transparency and accountability
that hinder efforts to address and mitigate the environmental
consequences associated with data centers.  Such issues are
pervasive as they extend beyond data centers and permeate various industries,
including food, manufacturing, and telecommunication systems.



The lack of accountability and transparency to address sustainability
is primarily rooted in the absence of \emph{complete} and \emph{verifiable}
sustainability data and metrics~\cite{data-disclosure-greener-future,
accountable-sustainability, corporate-sustainability-disclosure}. 
Comprehensive and fine-grained sustainability metrics~\cite{nsfgdi} are
critical to identify performance bottlenecks (\eg the impact of an
application's code or library on sustainability), diagnose security
issues, detect anomalous sustainability activities, provide reliable
audit trail of carbon consumption, ensure accurate and precise
accountability and compliance benefits (\eg
accurately identify entities who made changes or performed certain
actions), and optimize system performance~\cite{bashir21-sustainableclouds,
chasing-carbon-udit2022, nsfgdi}.
Therefore, a necessary first step for any sustainable computing approach is the
ability to measure comprehensive sustainability metrics or cost
functions from all possible sources of carbon consumption and energy
spent in the entire lifecycle of the computing equipment: production,
delivery, and disposal; these are referred to as ``embodied energy.''
However, it has been
found that it is difficult to determine
accurate sustainability metrics because the sources are too many,
untrustworthy, disconnected, or incompatible. Further,
there is no way to combine the data in a meaningful way that will not
compromise the privacy of users or service
providers~\cite{overselling-sustainability-reporting}.
For example,
there are dozens of different ways to calculate data on global data
center energy consumption based on public and private data---each resulting
in an assessment that is often contradictory with others~\cite{eip20}.
Hence, we have at best a vague idea of the impact that, for example,
data centers have on our environment.
%
Even when attempts are made to collect
and combine sustainability metrics from disparate sources,
privacy concerns, exposure of sensitive users' data or service
providers' proprietary algorithms are often ignored, resulting in poor
incentives for users or service providers to opt for accountable
sustainability systems.
Researchers and organizations
trying to understand and create sustainable systems often refer to the
\emph{sustainability data gap}.
%
The inability to collect and verify
accurate, complete, and timely data on the environment in a
privacy-preserving fashion is slowing, and in some cases prohibiting,
the adoption of sustainable systems and practices.
To make matters worse, market forces and
human greed, as we observed earlier, often work against
the goals of sustainability.
%

In the context of data centers, which is the primary focus of this paper,
the infrastructures used to measure and maintain \emph{operational
sustainability}
(\ie environmental footprints transpired within a data center) are
inherently \emph{adversarial}:
because users of technology (\eg data center users)
have an incentive to cheat, the apparatus must strive to ensure that systems continue to
function correctly in the face of actors attempting to thwart the
collection of
sensitive sustainability data and the enforcement of corresponding
security and privacy policies.
% TODO: Move it later
Hence, it is imperative that the environmental footprint
caused by data center operations can be verified by
interested third parties (\eg the EPA~\cite{co2e_epa},
citizen scientists, and the public).


This article, therefore, looks at the security issues in the
sustainability data pipeline comprising of data \emph{collection},
\emph{storage}, \emph{aggregation} (or other processing),
\emph{reporting} and \emph{use} \emph{in situ}.
%
More specifically, we examine threat landscapes and a wide
range of security challenges to build verifiable sustainability within
data centers, highlighting the urgent need to address these
threats.
%
Furthermore, we explore a variety of promising research directions
that will yield novel and practical solutions to combat these security
challenges in sustainable data centers and mitigate the risks
associated with such threat landscapes.
%
Some of our proposed security challenges and solutions also apply to
other industry segments: manufacturing, airlines and transportation,
industrial-scale farming, and more.


%%%%%%%%%%%%%%%%%%%%%%%%%%%%%%%%%%%%%%%%%%%%%%%%%%%%%%%%%%%%%%%%%%%%%%%%%%%%%%
%% For Emacs:
% Local variables:
% fill-column: 70
% End:
%%%%%%%%%%%%%%%%%%%%%%%%%%%%%%%%%%%%%%%%%%%%%%%%%%%%%%%%%%%%%%%%%%%%%%%%%%%%%%
%% For vim:
% vim:textwidth=70
%%%%%%%%%%%%%%%%%%%%%%%%%%%%%%%%%%%%%%%%%%%%%%%%%%%%%%%%%%%%%%%%%%%%%%%%%%%%%%
% LocalWords:  PSU acidification desertization IPCC cybersecurity geo
% LocalWords:  incentivized
