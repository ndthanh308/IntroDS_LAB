%\section{Security Challenges of Sustainable Systems}

\section{Why is Sustainability a Security Problem?}
\label{sec:security-challenges}

Ensuring the accuracy and credibility of sustainability
metrics, as well as supporting audits, require guaranteeing the
trustworthiness
and comprehensiveness of not only the carbon footprints of data center
equipment but also the embodied energy throughout the entire lifecycle
of computing equipment.
%
Although some  external information---such as that for
renewable energy, energy credits, or supplied water---can be
authenticated via trusted third parties~\cite{co2e_epa, iea},
sustainability metrics in data centers require the
authenticity, confidentiality, integrity, and availability of data
collected, processed, stored, and used locally within a
data center~\cite{gandhi2022metrics}.
%
However, unlike traditional cloud computing systems
where the focus is primarily on security and privacy of user applications
and data~\cite{carlin2013cloud, zissis2012addressing, chen2010s},
collecting and measuring data center activities that impact humans and
the environment in a verifiable and privacy-preserving manner
presents a diverse set of new security challenges.
%
Most of these challenges are primarily based on sustainability data,
reliability of equipment, and cleanliness of energy sources---across
both the digital and physical worlds.
%
Unfortunately, no prior research has investigated the threat landscape of
sustainable data centers, nor attempted to provide any techniques or tools
that directly allow authentication of operational sustainability
metrics induced within a data
center to preserve the privacy of users' or operators'
sustainability data.
It is thus imperative to ensure the security of (i) data collection
processes, (ii) the process of generating verifiable, easily auditable
sustainability metrics, and (iii) the storage of all pertinent
information.
Hence, while being indispensable for protecting the environment and our
planet, we have found and argue that the current sustainability
practices---through self-reporting, best-effort measurement, and
anything less than complete verifiable control of
sustainability---will fail.

%%%%%%%%%%%%%%%%%%%%%%%%%%%%%%%%%%%%%%%%%%%%%%%%%%%%%%%%%%%%%%%%%%%%%%
\subsection{Threat Models for Sustainability in Data Centers}

Although the trust assumptions and threat models for sustainable
systems may vary widely based on the system architecture and
requirements, the threat models for a sustainable data center
%\ag{system or data center? make it consistent}
can be primarily derived with respect to three entities: (a) the service
provider, (b) the users, and (c) third-party observers (\eg regulatory
agencies).
%
One may assume that the service provider can be considered
to be trusted but the underlying infrastructure (\eg OS and services)
provided by third-party vendors/suppliers
can be untrusted or become compromised, whereas others may assume that both
the service provider and the underlying infrastructure become rogue.
%
For example, benign and unsuspecting data center providers
often use virtual machines (VMs) or containers that are already offered
by third-party infrastructure providers and can be loaded with
backdoors or malware.
%
A malicious infrastructure provider can
deliberately manipulate energy consumption metrics, bypass
sustainability regulations, and overcharge the data center provider
for the total energy consumption.
%
This not only undermines the data center provider's sustainability
efforts but also leads to inflated costs and financial losses.
%
Moreover, when users' jobs run in an environment where
data center and/or infrastructure providers are malicious, attackers can
gain unauthorized access to read or modify the job's code and data.
%
For example, attackers may introduce unaccounted read/write
operations~\cite{graphene_sgx_atc17, glamdring} to users' jobs which
in turn inflate users' carbon footprints, leading to overbilling the
customers and increasing the financial profits of data center and
infrastructure providers.
Such carbon footprint inflation can also be achieved by violating the integrity
of the sustainability metrics (\eg code or
data)~\cite{graphene_sgx_atc17, glamdring} or by manipulating the
system traces and logs---the evidence trail of carbon
consumption~\cite{sgx_log_security_asiaccs17} by the compromised VMs or
malicious processes in data centers.
%
Similarly, compromised data center providers may report false
carbon footprints to the regulators~\cite{sgx_use_based_privacy_wpes18} to evade
high CO2 taxes or regulations.


The other key entities in data centers (\ie users) can also be
assumed to be untrusted as they may try to launch attacks (\eg DoS)
against other users or data center providers, or obtain higher
levels of service than they are allocated, and thus mislead the cloud service
providers about the user's carbon usage.
Last but not least, third-party observers (\eg regulatory agencies)
may be tasked with verifying the footprint reported by the service
provider in the process of executing policy or oversight (\eg by
comparing sustainability costs reported by cloud operators, users, and
utilities); but even these observers may be considered untrusted, as
they could collude with others to mislead reporting, may have rogue
insider elements within the data center, and may even be under political or other
pressure to ``fudge'' or misrepresent the data.


%%%%%%%%%%%%%%%%%%%%%%%%%%%%%%%%%%%%%%%%%%%%%%%%%%%%%%%%%%%%%%%%%%%%%%
\subsection{New Security Challenges for Sustainability}
\label{subsec:new-security-challenges}
Due to the complex design of data centers, which relies on intricate
trust assumptions among numerous stakeholders, it is necessary to
address diverse security threats ranging from malicious
software/service providers (in Software as a Service or SaaS models),
compromised operating systems or hypervisors (in Platform as a Service
or PaaS models), or compromised sensors, devices, and firmware owned by
infrastructure providers (in Infrastructure as a Service or IaaS
models) to malicious or honest-but-curious users.
%
In light of the above discussion, we discuss next some security
challenges for a system aiming for sustainability and
summarise those in Table~\ref{tab:sustainability_threats}.
Note that the nature of threats will be different for different
sustainable systems (\eg transportation, manufacturing) based on trust
assumptions.


\noindent \ding{113} \textbf{Lack of authenticity of carbon emission
  sources (C1).}  Sensors and devices (\eg PDUs) reporting and computing
sustainability data can be malicious and may become compromised due to
unintentional vulnerabilities or intended backdoors in their hardware,
firmware, and software~\cite{pdu-vulnerabilities}.  As a result, by
taking control of those sensors and devices, attackers may violate the
authenticity and forge carbon footprints to launch nefarious attacks.
For instance, attackers may cause over/under-billing to customers by
forging/manipulating carbon consumption records.
%
Attackers may also induce
carbon-exhaustion attacks on other users by misreporting over-consumption of
carbon, or evade compliance checking of regulatory agencies by misreporting
low carbon emissions when operating in test mode (similar to Volkswagen's
scandal~\cite{vwscandal2015}).  Similar kinds of sustainability
data-forgery attacks can also be carried out if there are vulnerabilities in the
communication protocols (\eg lack of authentication and replay protection) between sensors and the
sustainability data aggregators gleaning carbon footprints from
multiple such sensors.


\noindent \ding{113} \textbf{Untrustworthy physical environment (C2).}
Sensors and apparatuses used to collect carbon footprint data can be
subjected to direct and indirect data manipulation attacks.  For
example, an attacker having direct physical access to sensors or
data structure infrastructure can manipulate sensors' readings
to generate false sustainability data or manipulate the cooling system
to disrupt sustainability operations~\cite{physical-security}.
Conversely, in indirect attacks, the attackers do not
have direct physical access to sensors but exploit physical
side-channels~\cite{ding2021iotsafe} between different
components/sensors in data centers
to affect sustainability operations and cause reputation loss
to their competitors.  Due to such malicious actions,
additional water and electricity would be required to
cool the targeted data center, resulting in an increased
carbon footprint, higher operational costs, and disruption of
sustainability efforts.


\noindent \ding{113} \textbf{Lack of access control and information
flow control (C3).}
Sharing physical resources such as hardware and sensors among multiple
users introduces new challenge of isolating each tenant's data and ensuring that
one tenant cannot access another's sustainability footprints.
The lack of granular and dynamic access control configurations, and adequate
resource isolation, can lead to the failure to ensure that each workload
and its associated users have the appropriate access privileges to sustainability
footprints, without compromising data security.  Without proper access control and
information flow-control measures, there is, therefore, a risk of unauthorized access
to sensitive sustainability data, potentially leading to data breaches, privacy
violations, and other security issues.
Furthermore, sustainability data obtained from
various sources can be illegitimately tampered with by malicious users
processes or compromised system processes.  Malicious processes may
obtain unauthorized (read/write) access to sensitive resources (\eg
databases or protected memory regions storing sustainability data and
states) by exploiting vulnerabilities in the access control
policies~\cite{privilege-escalation}.
%
The lack of access control
mechanisms, such as Discretionary Access Control (DAC),
Mandatory Access Control (MAC), or combinations thereof, therefore, may
enable attackers to manipulate (\ie add, modify, or remove) carbon
footprint and
sustainability states.  As a result, the regular sustainability
operations of the system are likely to be disrupted, which may cause
the system to produce unwarranted carbon footprints or eliminate them.
Tampering with sustainability data by adversaries (\eg malicious
service providers or malicious users) may result in overcharging
legitimate users of the system (such as a data center), undercharging
malicious users attempting to evade sustainability costs, or damaging
the reputation of competing service providers.

\begin{table*}[h]
\footnotesize
\centering
\begin{tabular}{|m{0.3cm}|m{6.8cm} | m{4.7cm} | m{2.3cm}|}
\hline
\centering\textbf{ID} &
\centering\textbf{Vulnerabilities, Threats, and New Security Challenges} &
\centering\textbf{Impacts} &
\centering\textbf{Possible Ideas to Solutions}
\tabularnewline \hline
%\ding{182} &
\textbf{C1} &
Lack of authenticity of carbon emission sources allows malicious processes
to forge, tamper, or misreport carbon usage
&
Cause over-/under-billing to customers by tampering with carbon usage,
evade regulatory agencies by misreporting low carbon emissions
&
Verifiable footprint collection (\S\ref{subsec:verifiable_collection})
\tabularnewline \hline

%\ding{183} &
\textbf{C2} &
Untrustworthy physical environment may allow attackers to manipulate
sensors and apparatuses within a data center directly or indirectly
&
Induce higher operational costs, cause
over-/under-billing to customers, and denial-of-service attacks
&
Verifiable footprint collection (\S\ref{subsec:verifiable_collection})
\tabularnewline \hline

%\ding{186} &
\textbf{C3} &
Cryptographic flaws may allow forging the proof of carbon usage
&
Financial loss and disruption the data center operations
&
Verifiable footprint collection (\S\ref{subsec:privacy-preserving-collection})
\tabularnewline \hline

%\ding{184} &
\textbf{C4 \& C6} &
Disclosure of sustainability metrics to malicious
service providers and other users due to inadequate access control,
cryptographic protections, or side-channel vulnerabilities
&
Exposure of users' private data such as location, behavior, and
intellectual properties
&
Privacy-preserving footprint collection and aggregation
(\S\ref{subsec:privacy-preserving-collection},
\S\ref{subsec:privacy-preserving-aggregation}, \&
\S\ref{subsec:public-sustainability-ledger})
\tabularnewline \hline

%\ding{185} &
\textbf{C5} &
Lack of or flaws in the access control or information flow control
mechanisms may allow
malicious processes (controlled by malicious users or service providers) to
access and
tamper with the databases storing carbon footprint trails
&
Exposure of users' private data such as location, behavior, and intellectual
properties
&
Verifiable carbon footprint collection (\S\ref{subsec:privacy-preserving-collection})
\tabularnewline \hline


%\ding{187} &
\textbf{C7} &
Evasive carbon offset techniques allow corporations
to trade a known amount of carbon emissions with an uncertain
amount of carbon reductions
&
Tax evasion, financial loss, and environmental
hazards
&
Verifiable footprint collection (\S\ref{subsec:privacy-preserving-collection})
\tabularnewline \hline

%\ding{188} &
\textbf{C8} &
Multiple parties may collude to misreport carbon usage
&
Tax evasion, financial loss, and environmental
hazards
&
Verifiable footprint collection (\S\ref{subsec:privacy-preserving-collection})
\tabularnewline \hline

\end{tabular}
\caption{Threats and security challenges for the sustainability of
  data centers and potential research directions.
  %\ag{subsection numbers not showing up in table?} %\David{move C3 two rows up?}
  }
\label{tab:sustainability_threats}
\end{table*}

%%%%%%%%%%%%%%%%%%%%%%%%%%%%%%%%%%%%%%%%%%%%%%%%%%%%%%%%%%%%%%%%%%%%%%%%%%%%%%
%% For Emacs:
% Local variables:
% fill-column: 70
% End:
%%%%%%%%%%%%%%%%%%%%%%%%%%%%%%%%%%%%%%%%%%%%%%%%%%%%%%%%%%%%%%%%%%%%%%%%%%%%%%
%% For vim:
% vim:textwidth=70
%%%%%%%%%%%%%%%%%%%%%%%%%%%%%%%%%%%%%%%%%%%%%%%%%%%%%%%%%%%%%%%%%%%%%%%%%%%%%%
% LocalWords:  HotCarbon externalities Pigovian Ent Jevons Pigou TODO
% LocalWords:  unforgeable tesla youtube de facto SaaS PaaS IaaS




\noindent \ding{113} \textbf{Sensitive information disclosure (C4).}
Collecting sustainability data from disparate carbon sources (\eg
sensors and PDUs) in an unregulated manner may disclose the
sustainability metrics to service providers and other users.  Such
unauthorized exposure of footprint data will violate the privacy of user's
data, location, behavior, and intellectual properties such as
proprietary scheduling techniques, factors used for competitive
pricing for service classes.  Unauthorized access to footprint data
can enable an adversary to prevent a co-tenant from realizing an
improved sustainability target or even allow them to initiate DoS attacks
on the co-tenant.



\noindent \ding{113} \textbf{Cryptographic flaws (C5).}
The ability of a sustainable system to provide proof of
carbon footprint to users and regulators is essential for ensuring the
trustworthiness of the system.  Such proof of footprint should be built
with cryptographic constructs.  But flaws in the integration of
cryptographic constructs with complex data center systems (\eg using
weak cipher suites~\cite{ms365-insecure-block-cipher,
  samba-outdated-crypto}) or flaws in the
implementations~\cite{heartbleed} may fail to generate unforgeable and
accurate proof of consumption, enabling an attacker to drop, modify, replay,
and inject fake footprints of carbon.  This can disrupt the
operations of sustainable systems.


\noindent \ding{113} \textbf{Side-channels in sustainability (C6).}
Due to shared hardware resources, co-located servers, and poor
isolation between different processes running on the same hardware in
data centers, side-channel vulnerabilities (\eg page faults~\cite{xu2015controlled},
cache misses~\cite{wang2007new}, power~\cite{randolph2020power} and
timing~\cite{hund2013practical} channels) may allow a malicious
application to observe or tamper with carbon footprint
patterns of other users' jobs/applications running on the same
hardware.  Such side-channels not only allow an attacker to
fingerprint the data traffic of other users but also to extract the
cryptographic keys or other confidential information of a user
application by looking at the use of sustainability
metrics~\cite{cloud-side-channel-ristenpart}.  Attackers can exploit
such sensitive information to blackmail or embarrass other
users/competitors (\eg to force a competitor's stock to drop, or
short-sell such stock).


\noindent \ding{113} \textbf{Evasive carbon offset techniques (C7).}
Corporations often trade a known amount of carbon emissions with an
uncertain amount of emission reductions to claim carbon neutrality
(\eg by investing in forestation
elsewhere)~\cite{tesla_carbon_offset}.  This technique, also called
carbon credit or climate credit, has been in practice for decades.  It
is often exploited by large corporations as it is extremely difficult,
if not impossible, to track and verify if the amount of emissions
balances out the amount of reductions~\cite{junk_carbon_offset,
  myth_carbon_offset, scam_carbon_offset}.  Often, Renewable Energy
Credits (RECs) are used to offset the carbon footprint of a data
center via the purchase of energy credits from a green energy
generator~\cite{drec-initiative}.  Similarly, Power Purchase
Agreements (PPAs)~\cite{ppa} are used to have the data center operator
finance the installation of a green energy producing farm, run, owned
and managed by an independent party, to provide green energy to the
data center over a long-term period covered under the PPA.  For both
REC and PPAs, the authenticity of green energy is, however, often kept
out of sight of the users.  The lack of authentication, therefore,
enables corporations to make false claims about the energy source,
while appearing in public to support sustainability efforts.


\noindent \ding{113} \textbf{Collusion for evasion (C8).}  Infrastructure
providers and Power Distribution Unit (PDU) providers may collude to
misreport carbon footprints to regulators and users and thus may evade
regulatory agencies.
Such collusion attacks can be of different
combinations as infrastructure providers depend on third-party
software and hardware vendors which may also collude with each other
for malicious purposes.


\begin{comment}
%\item
\noindent $\bullet$ \textbf{Adversarial influence on metrics optimization.}
  Machine-learning--based optimization techniques are often used to optimize the usage of carbon
  based on the sustainability data collected from different sources (\ie
  system components).
  \ag{not sure I agree with above statement. do we have cites? usually, optimization techniques are used, which need not be ML-based.}
  Hence, like any other machine learning
  based system, sustainable systems are also vulnerable to different
  ML-based attacks, including poisoning the training data or adversarial
  samples during inference, model stealing, etc.\ezk{add cites}
\end{comment}


%%%%%%%%%%%%%%%%%%%%%%%%%%%%%%%%%%%%%%%%%%%%%%%%%%%%%%%%%%%%%%%%%%%%%%%%%%%%%%
%% For Emacs:
% Local variables:
% fill-column: 70
% End:
%%%%%%%%%%%%%%%%%%%%%%%%%%%%%%%%%%%%%%%%%%%%%%%%%%%%%%%%%%%%%%%%%%%%%%%%%%%%%%
%% For vim:
% vim:textwidth=70
%%%%%%%%%%%%%%%%%%%%%%%%%%%%%%%%%%%%%%%%%%%%%%%%%%%%%%%%%%%%%%%%%%%%%%%%%%%%%%
% LocalWords:  HotCarbon externalities Pigovian Ent Jevons Pigou TODO
% LocalWords:  unforgeable tesla youtube de facto SaaS PaaS IaaS
