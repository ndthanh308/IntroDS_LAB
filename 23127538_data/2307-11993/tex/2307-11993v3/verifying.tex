\section{Research Directions for Securing Sustainable Data Centers}
\label{s:verify}



% Figure environment removed

Although many solutions~\cite{confidential-computing}
%~\cite{berger2008tvdc, confidential-computing}
have been designed for data-center security, most of them are not directly applicable
to counter the security and privacy challenges towards sustainability as
discussed in Section~\ref{sec:security-challenges}.
Therefore, we must develop technologies that will help build
secure and trustworthy sustainable systems.
%
Particularly, we must develop primitives that allow domain experts to
construct and operate sustainable systems and verify the results.
%
Next, we lay out several potential research directions for improving
sustainability in data centers through security.



%%%%%%%%%%%%%%%%%%%%%%%%%%%%%%%%%%%%%%%%%%%%%%%%%%%%%%%%%%%%%%%%%%%%%%
\subsection{Verifiable Footprint Collection Architecture}
\label{subsec:verifiable_collection}


One of the most important elements of a sustainable system is its
ability to promote the responsible use of system resources, such as
complying with carbon emission restrictions/taxes.  However, claims of
carbon usage must be accompanied by infrastructure that demonstrates
\emph{verifiable footprint} to the public and regulatory organizations.
This calls for architectures and systems that can collect publicly
readable and verifiable sensor readings in adversarial settings.  It
is essential that these systems have the ability to scale seamlessly
from small, low-energy devices to larger, enterprise-level data
centers.  The system architecture should have the ability to generate
tamper-resistant proofs of carbon consumption that are unforgeable,
accurate, and securely retrievable by authorized parties (which might
include the public) in adversarial deployments.  Furthermore, to provide higher
security assurance, the design and implementation of these systems
must be formally verified.

\noindent \textbf{Potential Solutions}:
%
\mycolor{blue}
Developing such a framework poses key challenges, including the need
to establish and preserve a root of trust using trusted hardware, such
as Trusted Platform Module (TPM)
to secure the data center's carbon footprint measurement components.
%
A trusted path should be established from the secure hardware up to the
module that collects all the relevant metrics of a job, and further up
to the component that verifies the accuracy of the reported
metrics.
\mycolor{black}
%
This trusted path will be capable of producing tamper-proof
evidence of sustainability cost metrics using cryptographic proof
systems.

One potential solution to ensure the security of
sustainability-related components is to use a hardware-based Trusted
Execution Environment (TEE) such as ARM TrustZone, 
%\footnote{\href{https://developer.arm.com/documentation/100690/0201}{https://developer.arm.com/documentation/100690/0201}},
%~\cite{armtz2016},
Intel SGX, 
%\footnote{\href{https://www.intel.com/content/www/us/en/architecture-and-technology/software-guard-extensions.html}{https://www.intel.com/content/www/us/en/architecture-and-technology/software-guard-extensions.html}},
%~\cite{mckeen2013innovative},
AMD SEV, 
%\footnote{\href{https://www.amd.com/en/developer/sev.html}{https://www.amd.com/en/developer/sev.html}}
%~\cite{kaplan2016sev},
and Keystone. 
%\footnote{\href{https://github.com/keystone-enclave/keystone}{https://github.com/keystone-enclave/keystone}}.
%~\cite{lee2020keystone}.
%
%\David{I doubt we need these footnotes (maybe Keystone)}
TEEs are deployed in nearly every commercial processor sold today and
are the de-facto standard to provide a tamper-proof execution
environment that preserves the integrity and confidentiality of data
and execution~\cite{graphene_sgx_atc17}.
% , scone_osdi16,
%   sgx_fv_usenix21}.
%
These environments provide isolation guarantees needed to certify that
metric data is collected and reported accurately, even in the presence
of malicious applications, OS, or hypervisor.
%
A \emph{sustainability collector} (see Figure~\ref{fig:arch}) running
in a TEE will securely collect the utilization details of a
bare-metal, virtualized, or containerized job.
%
The gathered metrics
will create a comprehensive timeline of user, system, and
process-oriented carbon footprints, culminating in a
\emph{sustainability provenance record} for the cloud.
%
The sustainability collector will securely report the metrics to a
\emph{sustainability certification agent}, which will produce
lightweight cryptographic proofs that empower third-party regulators
and users to independently verify the claimed consumption.

\mycolor{blue}
Note that any flaws in the design or implementation of
sustainability-related components, \eg measurement or collection code
running within TEEs and owned by respective TEE hosting entities
(\ie data center operators or service providers)
may introduce new security challenges.  For instance,
attackers may exploit such flaws and bypass the tamper-proof guarantees
of the code.
Therefore, it is crucial to ensure high-security assurance of these components through formal
analysis before they are deployed.
Also, the physical or virtual machines hosting the measurement code within TEEs
and the regulatory agencies need to verify during runtime the integrity of the trusted
path from the secure hardware to corresponding TEEs periodically
or when there are major changes (\eg write operations) in the system or
their combinations thereof.
\mycolor{black}



Another potential concern is that current TEE platforms
might lack adequate privileges to monitor the carbon or resource consumption
of workloads that execute outside of the TEE.  This might necessitate
new hardware support for TEEs to allow secure monitoring of external
workloads, including the host OS or hypervisor.

One possible alternative to TEEs is to explore the use of add-on
monitoring hardware, akin to SmartNICs, that can collect
sustainability metrics from outside the host.
%
For example, AWS Nitro\footnote{\href{https://aws.amazon.com/ec2/nitro/}{https://aws.amazon.com/ec2/nitro/}}
%~\cite{nitro}
enables SmartNICs to monitor and
manage VM allocation and scheduling, while being technically
``outside'' the host OS.
%
Similarly, sustainability-related components could potentially run on
such add-on custom hardware with the necessary privileges to gather
data from the host without being vulnerable to compromise by the host.
Finally, sustainability data must be isolated from other workloads running on
the same machine, providing protection against unauthorized access and
tampering.

%%%%%%%%%%%%%%%%%%%%%%%%%%%%%%%%%%%%%%%%%%%%%%%%%%%%%%%%%%%%%%%%%%%%%%
\subsection{Privacy-Preserving Footprint Collection}
\label{subsec:privacy-preserving-collection}
Fine-grained sustainability data collected through disparate carbon sources, such
as sensors and PDUs in an unregulated manner, may induce unintended
disclosure of sensitive data.
The exposure of sustainability records would otherwise break the
users' privacy, data, location, behavior, and intellectual properties
such as proprietary scheduling techniques, trained machine learning
models, and factors used for competitive pricing for service classes~\cite{hlavacs2011energy, mckenna2012smart}.
%
Also, attackers may attempt to tamper with sensor data before it
is aggregated, which can lead to incorrect or misleading results. This
can be especially problematic in safety-critical applications, such as
autonomous vehicles or medical devices.

\noindent \textbf{Potential Solutions}: In concert with the verifiable
sustainability data collection architecture, differential privacy (DP) or
local differential privacy (LDP)
can be used as a probabilistic solution for privacy-preserving sustainability
footprint collection.
%
A certain degree of noise can be added to
the collected data to obscure individual data points but still allow
for useful aggregate analysis~\cite{dwork2006differential}.
%, dwork2008differential}.
%
\mycolor{blue}
A classical challenge of such differential privacy-based solutions would
be to keep the utility (\eg the statistical properties) of the sustainability data
high to the system while still protecting the privacy of users and systems.
In other words, the privacy budget---the amount of noise that can be
added to the sustainability data without compromising privacy---needs
to be determined by the sensitivity of the sustainability data being
collected and the desired level of privacy protection.
Another challenge for DP-based solutions is to keep the total noise
added by all parties within an acceptable range and failure to do so
requires a trusted aggregator to correct the noise.
Since DP-based solutions protect the data owner by providing
indistinguishability of the dataset, they can be used as a
privacy-preserving way of releasing data.  However, one has to ensure
correct-by-constructions~\cite{morinov2012ccs, jin2022we} of such while adopting
them.
\mycolor{black}
%

\mycolor{blue}
To provide cryptographic guarantees and to preserve the utility of sustainability data
utility to a higher extent compared to differential privacy,
an alternate solution is to use homomorphic encryption\footnote{Craig Gentry. A fully homomorphic encryption
scheme. Ph.D. Dissertation, Stanford university, 2009.}. %~\cite{gentry2009fully}.
With this solution, the carbon sources can encrypt the sustainability data as well as enable
the decision-making agent to measure/compute any statistical
information on those encrypted data.
\mycolor{black}
%
There are, however, several challenges associated with this
solution. 
%~\cite{gentry2009fully}.
%~\cite{naehrig2011can}.
%
%\David{I doubt we need this footnote}
Homomorphic encryption (HE) requires significant computational resources
and can increase the size of the actual data (because of encryption) being
transmitted,
%~\cite{naehrig2011can},
making it more difficult to store and transmit efficiently.
Furthermore, there are currently limitations
%~\cite{naehrig2011can}
on the types of computations that can be performed on homomorphically
encrypted data.  For example, homomorphic encryption schemes support
only addition and multiplication.  Complex operations, such as
division or trigonometric functions, may not be efficiently supported. 
\mycolor{blue}
While the direct use of homomorphic encryption may not be appropriate
for resource-constrained carbon emission sources,
further research is warranted to check if optimized versions of HE such as
partial HE, leveled HE, and threshold HE can be utilized or a
new, lightweight, secure, and bespoke HE (\eg selective HE)
needs to be designed for sustainability in data centers.
Nevertheless, many major chip/system vendors such as
Intel, AMD and ARM are actively exploring hardware support for
HE and when these are available, they can provide a trusted basis for
implementing challenges to many of the security solutions identified in this article.
\mycolor{black}

Another alternative approach
involving less computational overhead than homomorphic encryption is
zero-knowledge proofs~\cite{zero-knowledge-proof},
%~\cite{fiege1987zero},
in which the carbon sources
can demonstrate to the sustainability certification agent,
that sustainability footprints are valid, without disclosing
the actual values that would otherwise compromise privacy.
%
\mycolor{blue}
However, zero-knowledge proofs can only be used to prove the authenticity of sustainability data
and is not intended for analyzing and making any decisions.
To address the challenges of each solution, further investigation is needed
to determine if homomorphic encryption or differential privacy can be
combined with zero-knowledge proofs.

\mycolor{black}


%%%%%%%%%%%%%%%%%%%%%%%%%%%%%%%%%%%%%%%%%%%%%%%%%%%%%%%%%%%%%%%%%%%%%%
\subsection{Privacy-Preserving Footprint Aggregation}
\label{subsec:privacy-preserving-aggregation}
Collecting and processing sustainability data from multiple sites in
data centers require secure collaboration between multiple untrusted
parties, including cloud operators, regulators, and users, each with
their own confidentiality, privacy, security, and trust requirements.
%
While being aggregated either in centralized or distributed data
centers, sustainability data can still reveal sensitive information
about users and systems as discussed in Section~\ref{subsec:new-security-challenges}.
Therefore, the high-level goals are to (1) perform aggregation,
summary, or other functions on the sustainability data whose results
do not disclose information about the underlying data; and (2) ensure
that aggregations provide (provably) accurate higher-level data
without exposing underlying sensitive information, \eg proof of
sustainability compliance of the manufacturing process without
exposing unit-wise behaviors or specific metrics.



\noindent \textbf{Potential Solutions}: A plausible approach to
privacy-preserving aggregation for sustainability data is
secure multi-party communication (MPC) in which multiple
carbon footprint aggregators located at different locations
collaborate to perform computations on their combined data without
revealing any individual data points~\cite{goldreich1998secure}.
MPC requires minimal trust and aims to ensure each party's
input is kept private while allowing them to compute the
desired aggregation, summary, or other
functions on their combined data whose results do not disclose
information about the underlying data.
%
One such MPC platform is Confidential Space by Google\footnote{\href{https://cloud.google.com/blog/products/identity-security/announcing-confidential-space}{https://cloud.google.com/blog/products/identity-security/announcing-confidential-space}}, 
%~\cite{confidential-space},
which would allow sustainability data to be encrypted and stored in a
TEE that only authorized workloads are allowed to access.
%
Additionally, such data is isolated from other workloads running on
the same machine, protecting unauthorized access and
tampering.
\mycolor{blue}
MPC-based solutions, however, incur higher computational and communication
overheads due to secure computations and sharing of encrypted results.
%To prevent an adversary from linking a particular record
%to a specific party.An alternate solution is to use K-anonymity

%

%\David{The footnote for FL is not needed}


To minimize sustainability data movement, federated learning 
%\footnote{\href{https://blog.research.google/2017/04/federated-learning-collaborative.html}{https://blog.research.google/2017/04/federated-learning-collaborative.html}}
%\cite{li2020federated}
can be used in which training a machine learning model (\eg carbon footprint
optimization) on decentralized sustainability data/metrics can be
performed without having to transfer the data to a centralized
location.
\mycolor{black}
%
Each site of the distributed data center will train a local model on
its sustainability data and send the updated model weights to a
central server, which aggregates them to create a global model.
%
This approach allows data to remain local and private while still
benefiting from a centralized learning process.
%
Note that existing federated learning techniques are susceptible to
model-poisoning and model-stealing attacks; this further imposes
challenges to adopt federated learning-based solutions for aggregating
sustainability data. %~\cite{li2020federated}.


%%%%%%%%%%%%%%%%%%%%%%%%%%%%%%%%%%%%%%%%%%%%%%%%%%%%%%%%%%%%%%%%%%%%%%
\subsection{Public Sustainability Ledgers}
\label{subsec:public-sustainability-ledger}

Public sustainability ledgers can be used for tracking carbon
emissions or energy consumption and thus can provide transparency and
accountability in the management of resources.
%
However, there are also security and privacy issues that need to be
considered when using these public ledgers.
%
For example, if public ledgers contain sensitive data (\eg carbon
credit allocations, sales, and expenditures) about the sustainability
practices of individuals and organizations, attackers may track the
individuals/organizations or infer proprietary algorithms.
%
Also, sustainability data may be stored on multiple public ledgers or
private databases, which may not be interoperable.
%
This can create challenges in ensuring data consistency and accuracy,
and may also lead to data breaches if not properly secured.

\noindent \textbf{Potential Solutions}: In combination with
privacy-preserving measures, such as homomorphic encryption,
zero-knowledge proofs, multi-party computations, and differential
privacy, public ledgers for sustainability reporting can be provided
through smart contracts~\cite{aloqaily2020energy} deployed on the
public blockchain.
%
The smart contract records the sustainability footprints from different sources
and stores the encrypted records in blocks on the blockchain.
The sustainability footprints submitted to the blockchain undergo
verification by the participating entities through a consensus mechanism,
such as Proof-of-Work (PoW) or Proof-of-Stake (PoS).
%
This ensures the accuracy
and integrity of the recorded footprints.
Consumers, stakeholders, and regulators can access the public blockchain
to track and verify the provenance of sustainability footprints.
Although smart contracts---in concert with a verifiable
sustainability footprint collection architecture (Figure~\ref{fig:arch}) and
privacy-preserving measures---can offer secure and public sustainability ledgers,
smart contracts can also be subject to vulnerabilities that can be exploited by
attackers. %~\cite{perez2021smart}.
%
As such, it is important to thoroughly test and audit smart contracts
to ensure their security and reliability. %~\cite{tsankov2018securify}.
%
Furthermore, blockchain technology 
%\footnote{\href{https://www.ibm.com/topics/blockchain}{https://www.ibm.com/topics/blockchain}}
%~\cite{pilkington2016blockchain}
can be used to address the inconsistency and data-breach issues of
distributed public ledgers.
%
However, current blockchain technologies are susceptible to various
types of attacks including 51\% (majority) attacks and
denial-of-service attacks. %~\cite{zhang2019security}.
%
As such, it is important to ensure that the blockchain network is
properly secured and appropriate security measures are in place
to prevent such attacks.
%

\mycolor{blue}
While the potential security solutions outlined in this paper may contribute to carbon footprints, future research is necessary to rigorously evaluate the performance
and security guarantees of the existing and newly designed solutions.
As discussed in Section~\ref{sec:security-challenges}, the importance
of such security solutions in ensuring the trustworthiness of sustainability data and incentivizing the users toward sustainability practices is crucial for addressing global climate change and is believed to outweigh the impact of systems lacking such guarantees.

\mycolor{black}

\mycolor{blue}
%%%%%%%%%%%%%%%%%%%%%%%%%%%%%%%%%%%%%%%%%%%%%%%%%%%%%%%%%%%%%%%%%%%%%%
\section{Enhancing Standardization of Security
Mechanisms}

Security mechanisms are essential to ensure
compliance with regulations and standards,
preventing unauthorized access, and exposure,
tampering, or misuse of sustainability data.
Irrespective of the specific solution used to ensure security
of sustainability, a common need is to ease the adoption of those mechanisms
and reduce their footprint, both in terms of performance and
sustainability.
%
For instance, a TEE-based solution for verifiable data collection or
a homomorphic encryption-based approach for privacy-preserving
footprint collection should be lightweight and have small footprints
so as to minimize overall carbon consumption.
As trustworthiness is foundational in sustainability initiatives,
stakeholders, including governments, businesses, and users,
need high security and privacy assurance of sustainability data, which
is crucial for the success and adoption of sustainability practices.
Since sustainability data is critical for understanding trends
and for long-term planning and monitoring to counter global
issues such as climate change,
it is necessary to rigorously evaluate the effectiveness of security measures toward sustainability to
make informed decisions for a sustainable future.
As sustainability is a global concern that requires collaboration across borders,
standardizing security mechanisms for sustainability data
will accelerate their adoption in other sectors, facilitate international
cooperation, and ensure consistent protection standards and interoperability. 
Incentives and regulations need to be introduced to motivate
organizations to adopt and implement standardized security
mechanisms.
%
These could include tax incentives, certification programs, or
regulatory requirements that prioritize sustainability and security.
Note that all challenges toward sustainability
cannot be solved with technical solutions alone.
Hence, offering both fundamental principles and secure guarantees is more likely to assist
in the development of policies.  This, in turn, can contribute to and accelerate the global effort to combat climate change.
Without robust policies, all optimizations are susceptible to the
Jevons paradox (i.e., increasing efficiency can lead to 
increased consumption), which signifies that both regulation and security
are crucial components. Hence, collaboration and cooperation among industry players, researchers, and policymakers are necessary to establish these common goals and objectives.
\mycolor{black}
%\zhenhua{Jevons paradox perhaps needs some explanation.}

%%%%%%%%%%%%%%%%%%%%%%%%%%%%%%%%%%%%%%%%%%%%%%%%%%%%%%%%%%%%%%%%%%%%%%%%%%%%%%
%% For Emacs:
% Local variables:
% fill-column: 70
% End:
%%%%%%%%%%%%%%%%%%%%%%%%%%%%%%%%%%%%%%%%%%%%%%%%%%%%%%%%%%%%%%%%%%%%%%%%%%%%%%
%% For vim:
% vim:textwidth=70
%%%%%%%%%%%%%%%%%%%%%%%%%%%%%%%%%%%%%%%%%%%%%%%%%%%%%%%%%%%%%%%%%%%%%%%%%%%%%%
% LocalWords:  PSU unforgeable incentivization TLS allocator SGX SEV
% LocalWords:  TrustZone de facto TEEs SASSY's compositional crypto
% LocalWords:  TEE's HSC sss TODO PDUs IPsec RAPL XMLERS Merkle LDP
% LocalWords:  cryptographically adaptively incentivized SmartNICs
% LocalWords:  PoW PoS Jevons
