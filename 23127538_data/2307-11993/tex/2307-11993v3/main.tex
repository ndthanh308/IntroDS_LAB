\documentclass{IEEEcsmag}

\usepackage[colorlinks,urlcolor=blue,linkcolor=blue,citecolor=blue]{hyperref}
\expandafter\def\expandafter\UrlBreaks\expandafter{\UrlBreaks\do\/\do\*\do\-\do\~\do\'\do\"\do\-}
\usepackage{upmath,color}


\usepackage{graphicx}
\usepackage{epstopdf}
\usepackage{epsfig}
\usepackage{alltt}
\usepackage{times}
\usepackage{caption}
\usepackage{tabularx}
\usepackage{pifont}


\usepackage{etoolbox}

\makeatletter
% \pretocmd{<cmd>}{<prefix>}{<success>}{<failure>}
% \pretocmd{\@sect}{\def\@currentlabel{#8}}{}{}% Store title of \section
% \pretocmd{\@ssect}{\def\@currentlabel{#5}}{}{}% Store title of \section*
\makeatother

\usepackage{todonotes}
\usepackage{comment}
\usepackage{xspace}

%\usepackage[multiple]{footmisc}
\let\oldFootnote\footnote
\newcommand\nextToken\relax

\renewcommand\footnote[1]{%
    \oldFootnote{#1}\futurelet\nextToken\isFootnote}

\newcommand\isFootnote{%
    \ifx\footnote\nextToken\textsuperscript{,}\fi}


\newcommand{\Dongyoon}[1]{\todo[color=orange,inline]{Dongyoon: #1}}
\newcommand{\David}[1]{\todo[color=cyan,inline]{David: #1}}
\newcommand{\ag}[1]{\todo[color=orange,inline]{Anshul: #1}}
\newcommand{\YY}[1]{\todo[color=yellow,inline]{YY: #1}}
\newcommand{\syed}[1]{\todo[color=green,inline]{Syed: #1}}
\newcommand{\ezk}[1]{\todo[color=purple,inline]{EZK: #1}}
\newcommand{\pdm}[1]{\todo[color=lime,inline]{PDM: #1}}
\newcommand{\kartik}[1]{\todo[color=gray,inline]{Kartik: #1}}
\newcommand{\kanad}[1]{\todo[color=gray,inline]{Kanad: #1}}
\newcommand{\zhenhua}[1]{\todo[color=gray,inline]{Zhenhua: #1}}
\newcommand{\addcite}[1]       {\todo[color=red,inline]{CITE: #1}}
\newcommand{\addref}[1]        {\todo[color=red,inline]{REF: #1}}

% Latin abbreviation macros
\newcommand{\ie}{\emph{i.e.},\xspace}
\newcommand{\eg}{\emph{e.g.},\xspace}
\newcommand{\etal}{\emph{et al.\xspace}}

% macro to use Section symbol rather than "Section" (saves space)
\newcommand{\sref}[1]           {\,\S\ref{#1}\xspace}



\let\mycolor\color
\renewcommand{\mycolor}[2][]{}



\jvol{XX}
\jnum{XX}
\paper{8}
\jmonth{Month}
\jname{Publication Name}
\jtitle{Publication Title}
\pubyear{2021}

\newtheorem{theorem}{Theorem}
\newtheorem{lemma}{Lemma}


\setcounter{secnumdepth}{3}



\begin{document}

\sptitle{Security \& Privacy Magazine}

\title{Verifiable Sustainability in Data Centers}


\author{Syed Rafiul Hussain}
\affil{Pennsylvania State University, hussain1@psu.edu}

\author{Patrick McDaniel}
\affil{University of Wisconsin--Madison, mcdaniel@cs.wisc.edu}


\author{Anshul Gandhi}
\affil{Stony Brook University, anshul@cs.stonybrook.edu}

\author{Kanad Ghose}
\affil{Binghamton University, ghose@binghamton.edu}

\author{Kartik Gopalan}
\affil{Binghamton University, kartik@binghamton.edu}

\author{Dongyoon Lee}
\affil{Stony Brook University, dongyoon@cs.stonybrook.edu}

\author{Yu David Liu}
\affil{Binghamton University, davidl@binghamton.edu}

\author{Zhenhua Liu}
\affil{Stony Brook University, zhenhua.liu@stonybrook.edu}

\author{Shuai Mu}
\affil{Stony Brook University, shuai@cs.stonybrook.edu}

\author{Erez Zadok}
\affil{Stony Brook University, erez.zadok@stonybrook.edu}

\markboth{DEPARTMENT}{DEPARTMENT}


\begin{abstract}%\looseness-1
Data centers have significant energy needs, both embodied and operational, affecting sustainability adversely.
The current techniques and tools for collecting, aggregating, and reporting verifiable sustainability data are
vulnerable to cyberattacks and misuse, requiring new security and privacy-preserving solutions.
This paper outlines security challenges and research directions for addressing these pressing requirements.
\end{abstract}

\maketitle

% \section{Introduction}
% \label{s:intro}

\chapteri{S}ustainability is the practice of performing human
activities in ways that do not leave lasting harmful
effects~\cite{un22}.
Unfortunately, the harm to the planet is clearly growing,
whether the effects are direct (\eg emissions
caused by transportation, farming, or manufacturing) or indirect
(\eg
carbon emissions due to electricity consumed by data centers
and even the energy and materials used for
manufacturing servers and other devices). 
Humans as a species have understood
that sustainability is important to both future generations and the
global quality of life. Yet, we have had only sporadic and uneven
adoption of sustainable practices, and up to 98\% of sustainability
initiatives fail to meet their goals~\cite{dst16}. 
The impacts of a
lack of sustainability have led to---among many other
factors---climate change, widespread pollution of the oceans, sea
bottom desertification,
acidification of land and water, ozone loss, desertization, and loss
of biodiversity. Failure to address this lack of sustainability now
will create long-term problems for future generations~\cite{ipcc22}.

Today, achieving the goals of sustainability requires the honest, best
efforts of humans and an apparatus to measure aspects of the system
under regulation.  Yet, those efforts often fail when bad actors
bypass or cheat sustainability systems.  For example, the car company
Volkswagen installed emissions software on roughly 11 million cars
worldwide that misled the Environmental Protection Agency (EPA) about
emissions when under
test~\cite{vwscandal2015}.  Volkswagen was eventually caught, fined
billions of dollars, and required to recall vehicles and pay financial
settlements---but only \emph{after} the vehicles had polluted for
nearly a decade.

One area with unprecedented impact on our world is the use of
computation and in particular data centers.
With the alarming rise of computation and the pervasive use of
artificial intelligence (\eg ChatGPT~\cite{llm-energy}), data centers pose many negative impacts
on the environment caused by energy use, hardware
manufacturing and disposal, building maintenance, and other factors.  Indeed, a
recent study showed that over 2--4\% of all energy used worldwide was by
data centers~\cite{masanet20, iea}.
%
The current practice of reporting
sustainability information in data centers is, however, mired with
``greenwashing,'' where the true carbon footprint of a data center
is artificially reduced via the purchase of energy from green generation
sources~\cite{drec-initiative} or by paying other entities to be
sustainable.
%
This signifies a lack of transparency and accountability
that hinder efforts to address and mitigate the environmental
consequences associated with data centers.  Such issues are
pervasive as they extend beyond data centers and permeate various industries,
including food, manufacturing, and telecommunication systems.



The lack of accountability and transparency to address sustainability
is primarily rooted in the absence of \emph{complete} and \emph{verifiable}
sustainability data and metrics~\cite{data-disclosure-greener-future,
accountable-sustainability, corporate-sustainability-disclosure}. 
Comprehensive and fine-grained sustainability metrics~\cite{nsfgdi} are
critical to identify performance bottlenecks (\eg the impact of an
application's code or library on sustainability), diagnose security
issues, detect anomalous sustainability activities, provide reliable
audit trail of carbon consumption, ensure accurate and precise
accountability and compliance benefits (\eg
accurately identify entities who made changes or performed certain
actions), and optimize system performance~\cite{bashir21-sustainableclouds,
chasing-carbon-udit2022, nsfgdi}.
Therefore, a necessary first step for any sustainable computing approach is the
ability to measure comprehensive sustainability metrics or cost
functions from all possible sources of carbon consumption and energy
spent in the entire lifecycle of the computing equipment: production,
delivery, and disposal; these are referred to as ``embodied energy.''
However, it has been
found that it is difficult to determine
accurate sustainability metrics because the sources are too many,
untrustworthy, disconnected, or incompatible. Further,
there is no way to combine the data in a meaningful way that will not
compromise the privacy of users or service
providers~\cite{overselling-sustainability-reporting}.
For example,
there are dozens of different ways to calculate data on global data
center energy consumption based on public and private data---each resulting
in an assessment that is often contradictory with others~\cite{eip20}.
Hence, we have at best a vague idea of the impact that, for example,
data centers have on our environment.
%
Even when attempts are made to collect
and combine sustainability metrics from disparate sources,
privacy concerns, exposure of sensitive users' data or service
providers' proprietary algorithms are often ignored, resulting in poor
incentives for users or service providers to opt for accountable
sustainability systems.
Researchers and organizations
trying to understand and create sustainable systems often refer to the
\emph{sustainability data gap}.
%
The inability to collect and verify
accurate, complete, and timely data on the environment in a
privacy-preserving fashion is slowing, and in some cases prohibiting,
the adoption of sustainable systems and practices.
To make matters worse, market forces and
human greed, as we observed earlier, often work against
the goals of sustainability.
%

In the context of data centers, which is the primary focus of this paper,
the infrastructures used to measure and maintain \emph{operational
sustainability}
(\ie environmental footprints transpired within a data center) are
inherently \emph{adversarial}:
because users of technology (\eg data center users)
have an incentive to cheat, the apparatus must strive to ensure that systems continue to
function correctly in the face of actors attempting to thwart the
collection of
sensitive sustainability data and the enforcement of corresponding
security and privacy policies.
% TODO: Move it later
Hence, it is imperative that the environmental footprint
caused by data center operations can be verified by
interested third parties (\eg the EPA~\cite{co2e_epa},
citizen scientists, and the public).


This article, therefore, looks at the security issues in the
sustainability data pipeline comprising of data \emph{collection},
\emph{storage}, \emph{aggregation} (or other processing),
\emph{reporting} and \emph{use} \emph{in situ}.
%
More specifically, we examine threat landscapes and a wide
range of security challenges to build verifiable sustainability within
data centers, highlighting the urgent need to address these
threats.
%
Furthermore, we explore a variety of promising research directions
that will yield novel and practical solutions to combat these security
challenges in sustainable data centers and mitigate the risks
associated with such threat landscapes.
%
Some of our proposed security challenges and solutions also apply to
other industry segments: manufacturing, airlines and transportation,
industrial-scale farming, and more.


%%%%%%%%%%%%%%%%%%%%%%%%%%%%%%%%%%%%%%%%%%%%%%%%%%%%%%%%%%%%%%%%%%%%%%%%%%%%%%
%% For Emacs:
% Local variables:
% fill-column: 70
% End:
%%%%%%%%%%%%%%%%%%%%%%%%%%%%%%%%%%%%%%%%%%%%%%%%%%%%%%%%%%%%%%%%%%%%%%%%%%%%%%
%% For vim:
% vim:textwidth=70
%%%%%%%%%%%%%%%%%%%%%%%%%%%%%%%%%%%%%%%%%%%%%%%%%%%%%%%%%%%%%%%%%%%%%%%%%%%%%%
% LocalWords:  PSU acidification desertization IPCC cybersecurity geo
% LocalWords:  incentivized


%\subsubsection{Progress of Systems}\label{sssec:configs_sys_progress}

% 
Building upon the notion of well-formed types (\cref{def:types_well_formed}), a well-formed configuration exhibits behaviour without deadlocks, where there is always some action able to be performed in the future, until the termination type is reached.
This behaviour is referred to as \emph{liveness}, and is given in~\cref{def:configs_iso_live}, and depends on \emph{future enabled} actions.
\input{tex/defs/configs/iso/future_en.tex}
\input{tex/defs/configs/iso/live.tex}
The liveness of a configuration indicates whether a deadlock has been reached. There must either be an action immediately viable, some future enabled actions, or the type must have reached termination.
\input{tex/defs/configs/iso/well_formed.tex}
Given a type \TypeS*\ well-formed against \ValClocks*, it holds that a configuration of \IsoCfg*\ will have future enabled actions, and therefore be live.
\begin{note}
    For a \emph{well-formed} \TypeS*, \CfgIso*1{\ValClocks_{0}} is \emph{live}.
\end{note}


\newcommand{\FullChoice}[0]{\left\{\TypComm_i \,\MsgLabel_i\left\langle T_i\right\rangle \left(\delta_i,\lambda_i\right).\TypeS_i\right\}_{i\in I}}

\begin{lemma}\label{lem:cfg_wf_neq_alpha}
   %
   If \CIso*\ is \emph{well-formed} then $\TypeS\neq\alpha$.
   %
\end{lemma}
\begin{proof}
   %
   By the hypothesis \CIso*\ is \emph{well-formed} meaning $\exists\Const$ such that $\ValClocks\models\Const$ and  $\emptyset;\Const~\Entails\TypeS$.

   Consider by contradiction of the hypothesis that $\TypeS=\alpha$.
   If \CIso*;[\alpha]\ is \emph{well-formed} then, by the judgement of rule \LblTypVar*\ $\alpha:\Const;\Const~\Entails\alpha$ and the evaluation shown in~\Cref{eq:cfg_wf_neq_alpha_eval} must hold, where there must be some recursive definition earlier in the derivation tree $\mu\alpha.\TypeS'$ corresponding to $\alpha$.
   %
   \begin{minieq}\label{eq:cfg_wf_neq_alpha_eval}
   \begin{array}[c]{l}
      \infer[\LblTypRec]{%
         \emptyset;\Const~\Entails\mu\alpha.\TypeS'%
      }{%   
         \infer{\alpha:\Const;\Const~\Entails\TypeS'}{%
            \infer[\vdots]{}{%
               \infer[\LblTypVar]{%
                  \alpha:\Const;\Const~\Entails\alpha%
               }{}%
            }%
         }%
      }
    \end{array}
   \end{minieq}
   
   \noindent If $\exists\ValClocks',\mu\alpha.\TypeS'$ such that \CIso*[\ValClocks'];[\mu\alpha.\TypeS']\ is \emph{well-formed} and \Trans*{\CIso[\ValClocks'];[\mu\alpha.\TypeS']}*[\CIso;[\alpha]], then the immediate transition must be via rule \LblCfgIsoUnfold*\ as shown in~\Cref{eq:cfg_wf_neq_alpha_unfold}; where either $\CIso[{\ValClocks}''];[{\TypeS}'']=\CIso;[\alpha]$ or \Trans*{\CIso[{\ValClocks}''];[{\TypeS}'']}*[\CIso;[\alpha]].
   %
   By the premise of rule \LblCfgIsoUnfold*\ any recursive calls following their definition are replaced by the definition of their next unfolding denoted by $\TypeS'\Subst[\mu\alpha.\TypeS'][\alpha]$.
   %
   \begin{minieq}\label{eq:cfg_wf_neq_alpha_unfold}
   \begin{array}[c]{l}
      \infer[\LblCfgIsoUnfold]{%
      \Trans{\CIso[\ValClocks'];[\mu\alpha.\TypeS']}:{\ProgAction}[\CIso[{\ValClocks}''];[{\TypeS}'']]
      }{%
      \Trans{\CIso[\ValClocks'];[{\TypeS}'\Subst[\mu\alpha.\TypeS'][\alpha]]}:{\ProgAction}[\CIso[{\ValClocks}''];[{\TypeS}'']]
      }
    \end{array}
   \end{minieq}
   
   \noindent Therefore, if \CIso*\ is \emph{well-formed} then $\TypeS\neq\alpha$.
   %
\end{proof}

% \begin{definition}[Live Configurations]\label{def:configs_live}
    \CIso*\ is \emph{live} if
    $\TypeS=\TypeEnd$ or if \VIso*\ is \isFE*.
\end{definition}

\begin{lemma}\label{lem:cfg_wf_end}
   %
   \CIso*;[\TypeEnd]\ is always \emph{well-formed}.
   %
\end{lemma}
\begin{proof}
   %
   By the hypothesis $\exists\Const$ such that $\ValClocks\models\Const$ and $\emptyset;\Const~\Entails\TypeS$, and by rule \LblTypEnd*\ $\emptyset;\TypeTrue~\Entails\TypeEnd$.
%   
   Therefore the hypothesis holds as $\ValClocks\models\TypeTrue$ always holds.
   %
\end{proof}

%
\begin{lemma}\label{lem:cfg_wf_then_live}
    %
    If \CIso*\ is \emph{well-formed}, then \CIso*\ is \emph{live}.
    %
\end{lemma}
\begin{proof}
    %
    By the hypothesis, there must $\exists\Const$ such that $\emptyset;\Const~\Entails\TypeS$ and $\ValClocks\models\Const$.
    %
    We proceed by induction on each case of \TypeS*:
    \begin{inductivecase}
        %
        %
        %
        % ~ choice
        \item\NewCase[$\TypeS=\simplechoice$]\label{itm:cfg_wf_then_live_choice} 
        By the hypothesis and the judgement of rule \LblTypChoice*\ $\exists\Const_i$ such that $\ValClocks\models\Past[\Const_i]$ and $\emptyset;\Past[\Const_i]~\Entails\FullChoice$.
        %
        By~\Cref{def:configs_wf,def:configs_fe} \CIso*\ is \isFE*, and therefore, by~\Cref{def:configs_live} \CIso*\ is \emph{live}.
        %
        %
        %
        % ~ rec def
        \item\NewCase[$\TypeS=\mu\alpha.{\TypeS}'$]\label{itm:wf_then_live_recdef} 
        By the hypothesis $\exists\Const$ such that $\ValClocks\models\Const$ and $\emptyset;\Const~\Entails\TypRecDef$, and by the premise of rule \LblTypRec*\ $\alpha:\Const;\Const~\Entails{\TypeS}'$ and \CIso*;[\TypeS']\ is \emph{well-formed}.
        
        We proceed by inner induction on each case of \TypeS*':
        \begin{inductivecase}
            %
            %
            % ~ rec def -> choice
            \item\NewCase[$\TypeS'=\simplechoice$] 
            By the judgement of rule \LblTypChoice*\ and the premise of rule \LblTypRec*, $\exists\Const_i$ such that $\ValClocks\models\Const_i$ and $\emptyset;\Const_i~\Entails\TypRecDef$ and $\alpha:\Const_i;\Past[\Const_i]~\Entails\FullChoice$.
            %
            It follows~\Cref{itm:cfg_wf_then_live_choice} of~\Cref{lem:cfg_wf_then_live} that, by~\Cref{def:configs_wf,def:configs_fe,def:configs_live}, \CIso*;[\simplechoice]\ is \isFE*, \emph{well-formed} and \emph{live}.
            Therefore, it holds that \CIso*\ is \emph{live}.
            % Then by the judgement of \LblTypChoice*\ and the premise of \LblTypRec*\ $\exists\Const_i$ such that $\ValClocks\models\Const_i$ and $\emptyset;\Const_i~\Entails\TypRecDef$ and $\alpha:\Const_i;\Past[\Const_i]~\Entails\simplechoice$ and \CIso*;[\simplechoice]\ is \emph{well-formed} and \emph{future-enabled} by~\cref{def:configs_fe}.
            %
            % Therefore \CIso*;[\simplechoice]\ is \emph{live} by definition.
            %
            %
            % ~ rec def -> rec def
            \item\NewCase[$\TypeS'=\mu\alpha'.{\TypeS}''$] 
            Then $\alpha:\Const;\Const~\Entails\mu\alpha'.{\TypeS}''$ and $\alpha:\Const,\alpha':\Const;\Const~\Entails{\TypeS}''$ and \CIso*;[{\TypeS}'']\ is \emph{well-formed}.
            %
            Therefore, \CIso*\ is \emph{live} if \CIso*;[{\TypeS}'']\ is \emph{live} (see other cases).
            %
            %
            % ~ rec def -> end
            \item $\TypeS'=\TypeEnd$ is \emph{live} by~\Cref{def:configs_live}.
            %
            %
            % ~ rec def -> alpha
            \item ${\TypeS'}$ cannot equal $\alpha$ by~\Cref{lem:cfg_wf_neq_alpha}.
            %
        \end{inductivecase}
        %
        %
        %
        % ~ end
        \item\NewCase[$\TypeS=\TypeEnd$]\label{itm:wf_then_live_end} 
        By~\Cref{def:configs_live} it holds that \CIso*;[\TypeEnd]\ is live.
        %
        %
        %
        % ~ alpha
        \item\NewCase[$\TypeS\neq\alpha$] By~\Cref{lem:cfg_wf_neq_alpha}.
        %
    \end{inductivecase}

    \noindent Therefore, it holds that for any $S$ that \CIso*\ is \emph{well-formed}, \CIso*\ is \emph{live}.
    %
\end{proof}
%
%

%

\begin{lemma}\label{lem:init_wf_then_live}
   %
   Given a \emph{well-formed} \TypeS*, \CIso*[\ValClocks_0]\ is \emph{live}.
   %
\end{lemma}
\begin{proof}
   %
   By~\Cref{def:types_wf}, \Sat*[\ValClocks_0]:[\Past]\ holds for any valid \Const*.
   %
\end{proof}

% ~ configuration transitions
% ! 
%
% ! (lemma 12) : iso cfg trans
\begin{lemma}\label{lem:configs_iso_trans}
   %
   The following holds for transitions of configurations:
   \begin{minieq}*
      \begin{array}[c]{c c l}
         %
         \Trans{\CIso}:{\ValTime}[\CIso']%
         & \implies & %
         \begin{array}[c]{lcl}
            %
            \ValClocks'=\ValClocks+\ValTime & %
            \land & %
            \TypeS'=\TypeS%
            %
         \end{array}
         %
         \\[-1ex]\\%
         \Trans{\CIso}:{\CommAction}[\CIso']%
         & \implies & %
         \begin{array}[t]{lcl c lcl c lcl}
            %
            \ValClocks & \models & \Const & %
            \land & %
            \ValClocks' & = & \ReSet[\ValClocks] & %
            \land & %
            \Const' & = & \ReSet[\Const]%
            %
            \\[-1ex]\\%
            \mathllap{\land\;}\ValClocks' & \models & \Const' & %
            \land & %
            \Const' & \subseteq & \TypEnvCond[\TypeS'] %
            %
         \end{array}%
         %
      \end{array}%
   \end{minieq}
   %
\end{lemma}
\begin{proof}
   %
   By inspection of the formation and transition rules in~\Cref{fig:types_rule,fig:typesemantics_tuple}.
   %
\end{proof}
%
%
% ! (lemma 11) : soc cfg trans
\begin{lemma}\label{lem:configs_soc_trans}
   %
   The following holds for transitions of configurations with queues:
   \begin{minieq}*
      \begin{array}[c]{c c lcl c lcl c l}
         %
         \Trans{\CSoc}:{\ValTime}[\CSoc']%
         & \implies & %
            %
            \TypeS' & = & \TypeS & \land & %
            \Queue' & = & \Queue & \land & %
            \ValClocks'=\ValClocks+\ValTime%
            %
         \\[1ex]%
         \Trans{\CSoc}:{\RecvMsg}[\CSoc']%
         & \implies & %
            %
            \TypeS' & = & \TypeS & \land & %
            \Queue' & = & \Queue;\Msg & \land & %
            \ValClocks'=\ValClocks
            %
         \\[1ex]%
         \Trans{\CSoc}:{\SendMsg}[\CSoc']%
         & \implies & %
            %
            \MsgType & = & \Msg & \land & %
            \Queue' & = & \Queue & \land & %
            \Trans{\CIso}:{\SendAction}[\CIso'] %
            %
         \\[1ex]%
         \Trans{\CSoc}:{\SiltAction}[\CSoc']%
         & \implies & %
            % 
            \MsgType & = & \Msg & \land & %
            \Queue' & = & \Msg;\Queue & \land & %
            \Trans{\CIso}:{\RecvAction}[\CIso']%
            %
      \end{array}
   \end{minieq}
   %
\end{lemma}
\begin{proof}
   %
   By inspection of the transitions in~\Cref{fig:typesemantics_triple}, supported by~\Cref{lem:configs_iso_trans}.
   %
\end{proof}
%

% ~ preservation of wf and compat
% \begin{definition}[Compatible Systems]\label{def:configs_compat}
    %
    Let $\VSoc_1 = \CSoc_1$ and $\VSoc_2 = \CSoc_2$. 
    System \VSys*\ is \emph{compatible} (written $\VSoc_1\bot\, \VSoc_2$) if:
    \begin{enumerate}
    \item\label{itm:configs_compat_non_empty_queues} $\Queue_1=\emptyset%
            ~\lor~%
            \Queue_2=\emptyset$%
        %
        \\
       \item\label{itm:configs_compat_dual_types} $\Queue_1=\Queue_2=\emptyset%
            ~\implies~%
            \ValClocks_1=\ValClocks_2%
            ~\land~%
            \TypeS_1=\Dual_2$
        %
        \\
  \item
    \label{itm:configs_compat_expected_receive} 
    $\Queue_1=\Msg;\Queue'_1
            ~\Implies~%
            \exists\ValClocks'_1,\TypeS'_1:
            \Trans{\CIso_1}:{\RecvMsg}[\CIso'_1]%
            ~\land~%
            \CSoc'_1 \bot\, \VSoc_2$
         %   \Compat[\VSoc_1'][\VSoc_2]$%
            \\
\item
    %\label{itm:configs_compat_expected_receive} 
    $\Queue_2=\Msg;\Queue'_2
            ~\Implies~%
            \exists\ValClocks'_2,\TypeS'_2:
            \Trans{\CIso_2}:{\RecvMsg}[\CIso'_2]%
            ~\land~%
            \VSoc_1 \bot\, \CSoc'_2$
         %   \Compat[\VSoc_1'][\VSoc_2]$%
            \\
        % $\forall i, j\in\mkSet[1,2]:%
        % i\neq j%
        % \quad%
        % \Queue_i=\Msg;\Queue_i'%
        % ~\Implies~%
        % \exists\ValClocks'_i,\TypeS'_i:%
        % \Trans{\CIso_i}:{\RecvMsg}[\CIso'_i]%
        % ~\land~%
        % \Compat[\VSoc'_i][\VSoc_j]$%
        %
    \end{enumerate}
    %
    % \noindent We write \Compat*\ if system \VSys*\ is compatible.
    % \Cref{itm:configs_compat_expected_receive} is symmetric.
\end{definition}
% \begin{definition}[Latest-enabled Action (\isLE*)]\label{def:configs_le}
    %
    A configuration \CIso*, that is future-enabled, has a \emph{latest-enabled send} (resp. \emph{latest-enabled receive}), or \isLE*!\ for short (resp. \isLE*?), 
    if $\forall t$ such that $\CIso+{+t}\isFE~\Implies~\exists t'\geq t: \Trans{\VIso}:{\ValTime',\SendMsg}$.
\end{definition}

% ! 
%
% ! (lemma 17) : compat wf, single transition -> wf
\begin{lemma}\label{lem:cfgs_trans_wf_pres}
    %
    If \VIso*_1\ and \VIso*_2\ are both \emph{well-formed} 
    and \Compat*[\VSoc_1][\VSoc_2]\ 
    and \Trans*{\Parl{\VSoc_1,\VSoc_2}}[\Parl{\VSoc'_1,\VSoc'_2}], 
    then both \VIso*'_1\ and \VIso*'_2\ are \emph{well-formed}.
    %
\end{lemma}
\begin{proof}
    %
    We proceed by induction on the depth of the derivation tree, analysing each case of the last rule applied for the transition \Trans*{\Parl{\VSoc_1,\VSoc_2}}[\Parl{\VSoc'_1,\VSoc'_2}].
    % \begin{inline}+
    %     \item \LblCfgSysWait*
    %     \item \LblCfgSysLComm*
    %     \item \LblCfgSysLPar*
    % \end{inline}
    %
    \begin{inductivecase}
        %
        %
        %
        %
        %
        % ~ wait
        \item\NewCase[\LblCfgSysWait*]\label{case:cfgs_trans_wf_pres_wait} 
        As shown in~\Cref{eq:cfgs_trans_wf_pres_wait_trans}, both \VSoc*_1\ and \VSoc*_2\ make a rule \LblCfgSocTime*\ transition with the same valuation of \ValTime*.
        If $\ValTime=0$ then by~\Cref{lem:configs_iso_trans,lem:configs_soc_trans} $\VIso_1=\VIso'_1$ and $\VIso_2=\VIso'_2$ and both \VIso*'_1\ and \VIso*'_2\ are \emph{well-formed}.
        %
        \begin{minieq}\label{eq:cfgs_trans_wf_pres_wait_trans}
        \begin{array}[c]{l}
            \infer[\LblCfgSysWait]{%
                \Trans{\Parl{\VSoc_1,\VSoc_2}}:{\ValTime}[\Parl{\VSoc'_1,\VSoc'_2}]
            }{%
                \infer[\LblCfgSocTime]{%
                    \Trans{\VSoc_1}:{\ValTime}[\VSoc'_1]
                }{\dots}
                & %
                \infer[\LblCfgSocTime]{%
                    \Trans{\VSoc_2}:{\ValTime}[\VSoc'_2]
                }{\dots}
            }
            \end{array}
        \end{minieq}
            
        \noindent We proceed with only \VSoc*_1\ as the analysis covers \VSoc*_2.
        If $\ValTime>0$ then by~\Cref{lem:configs_iso_trans,lem:configs_soc_trans} $\ValClocks'_1=\ValClocks_1+\ValTime$ and $\TypeS'_1=\TypeS_1$.
        %
        By induction on the depth of the derivation tree, analysing the last rule applied for the transition \Trans*{\VIso_1}:{\ValTime}[\VIso'_1]:
        \begin{inductivecase}
            %
            %
            % ~ wait -> tick
            \item\NewCase[\LblCfgIsoTick*]\label{case:cfgs_trans_wf_pres_wait_tgtz_tick} 
            Then \Trans*{\CIso_1}:{\ValTime}[\CIso+{+\ValTime}_1].
            We proceed by inner induction on the different cases of \TypeS*_1:
            \begin{inductivecase}
                %
                % ~ wait -> tick -> choice
                \item\NewCase[$\TypeS_1=\simplechoice$]\label{case:cfgs_trans_wf_pres_wait_tgtz_tick_choice} 
                By rule \LblCfgIsoInteract*\ $\exists\Const_i$ such that $\ValClocks_1\models\Past[\Const_i]$ and $\emptyset;\Past[\Const_i]~\Entails\FullChoice$, as in~\Cref{itm:cfg_wf_then_live_choice} of~\Cref{lem:cfg_wf_then_live}.
                %
                By induction hypothesis $\exists\Const'_i$ such that $\ValClocks_1+\ValTime\models\Past[\Const'_i]$ and $\emptyset;\Past[\Const'_i]~\Entails\FullChoice$, which is assured by the (persistency) premise of rule \LblCfgSocTime*.
                %
                Therefore, it holds that \CIso*[\ValClocks_1]+{+\ValTime};[\simplechoice]\ is \emph{well-formed}.
                % By~\Cref{def:configs_fe} \CIso*_1\ is \emph{future-enabled} and by rule \LblCfgSocTime*\ $\text{(persistency)}$ it holds that \CIso*+{+\ValTime}_1\ is also \emph{future-enabled}.
                % %
                % Therefore the hypothesis holds, \CIso*[\ValClocks_1]+{+\ValTime};[\simplechoice]\ is \emph{well-formed}; by~\Cref{def:types_wf} and rule \LblTypChoice*\ $\exists\Const'_i$ such that $\ValClocks_1+\ValTime\models\Past[\Const'_i]$ and $\emptyset;\Past[\Const'_i]~\Entails\simplechoice$.
                %
                %
                % ~ wait -> tick -> recursion
                \item\NewCase[$\TypeS_1=\mu\alpha.{\TypeS}''_1$]\label{case:cfgs_trans_wf_pres_wait_tgtz_tick_recursion} 
                By~\Cref{def:types_wf} $\exists\Const$ such that $\ValClocks_1\models\Const$ and $\emptyset;\Const~\Entails\mu\alpha.{\TypeS}''_1$, and (by rule \LblTypRec*) $\alpha:\Const;\Const~\Entails{\TypeS}''_1$ and \CIso*[\ValClocks_1];[{\TypeS}''_1]\ is \emph{well-formed}.
                %
                Therefore, the well-formedness of \CIso*[\ValClocks_1]+{+\ValTime};[\mu\alpha.{\TypeS}''_1]\ is dependant on the well-formedness of \CIso*[\ValClocks_1]+{+\ValTime};[{\TypeS}''_1]. (See other cases, as in~\Cref{itm:wf_then_live_recdef} of~\Cref{lem:cfg_wf_then_live}.)
                %
                %
                % ~ wait -> tick -> end
                \item\NewCase[$\TypeS_1=\TypeEnd$]\label{case:cfgs_trans_wf_pres_wait_tgtz_tick_end} 
                By~\Cref{lem:cfg_wf_end} \CIso*+{+\ValTime};[\TypeEnd]\ is \emph{well-formed}.
                %
                %
                % ~ wait -> tick -> rec call
                \item\label{case:cfgs_trans_wf_pres_wait_tgtz_tick_reccal} ${\TypeS}_1$ cannot equal $\alpha$ by~\Cref{lem:cfg_wf_neq_alpha}.
                %
            \end{inductivecase}
            %
            %
            % ~ wait -> unfold
            \item\NewCase[\LblCfgIsoUnfold*]\label{case:cfgs_trans_wf_pres_wait_tgtz_unfold} 
            Then $\TypeS_1=\mu\alpha.{\TypeS}''_1$ and by the hypothesis $\exists\Const$ such that $\ValClocks_1\models\Const$ and $\emptyset;\Const~\Entails\TypRecDef$, and by rule \LblTypRec*\ $\alpha:\Const;\Const~\Entails{\TypeS}''_1$.
            The transition is as shown below:
            %
            \begin{minieq}*%\label{eq:cfgs_trans_wf_pres_wait_tgtz_unfold_trans}
        \begin{array}[c]{l}
                \infer[\LblCfgIsoUnfold]{%
                    \Trans{\CIso[\ValClocks_1];[\mu\alpha.{\TypeS}''_1]}:{\ProgAction}[\CIso'_1]
                }{%
                    \Trans{\CIso[\ValClocks_1];[{\TypeS}''_1\Subst[\mu\alpha.{\TypeS}''_1][\alpha]]}:{\ValTime}[\CIso'_1]
                }
                \end{array}
            \end{minieq}
            
            \noindent By inner induction on the different cases of ${\TypeS}''_1$:
            \begin{inductivecase}
                %
                % ~ wait -> unfold -> choice
                \item\NewCase[${\TypeS}''_1=\simplechoice$]\label{case:cfgs_trans_wf_pres_wait_tgtz_unfold_choice} 
                Then, by rule \LblCfgIsoTick*:
                \[\Trans{\CIso[\ValClocks_1];[{\simplechoice}\Subst[\mu\alpha.{\simplechoice}][\alpha]]}:{\ValTime}[\CIso[\ValClocks_1]+{+\ValTime};[\simplechoice]]\]
                
                \noindent By the rules \LblTypRec*\ and \LblTypChoice*\ the following holds:
                \[
                \infer[\LblTypRec]{%
                \emptyset;\Const_i~\Entails\mu\alpha.{\FullChoice}
                }{%
                \infer[\LblTypChoice]{%
                    \alpha:\Const_i;\Past_i~\Entails\FullChoice
                }{%
                \dots
                }
                }
                \]
                
                % \noindent It holds that \CIso*[\ValClocks_1];[{\simplechoice}\Subst[\mu\alpha.{\simplechoice}][\alpha]]\ is \emph{well-formed} and \emph{future-enabled}.
                %
                \noindent By induction hypothesis: \[\exists\Const'_i:\ValClocks_1+\ValTime\models\Past[\Const'_i] ~\land~ \emptyset;\Past[\Const'_i]~\Entails\FullChoice\] which is assured by the (persistency) premise of rule \LblCfgSocTime*, as in~\Cref{case:cfgs_trans_wf_pres_wait_tgtz_tick_choice} of~\Cref{lem:cfgs_trans_wf_pres}.
% 
                Therefore, \CIso*[\ValClocks_1]+{+\ValTime};[\mu\alpha.{\simplechoice}]\ is \emph{well-formed} as \CIso*[\ValClocks_1]+{+\ValTime};[{\simplechoice}\Subst[\mu\alpha.{\simplechoice}][\alpha]]\ the following is \emph{well-formed}.
                %
                %
                % ~ wait -> unfold -> recursion
                \item\NewCase[${\TypeS}''_1=\mu\alpha'.{\TypeS}'''_1$]\label{case:cfgs_trans_wf_pres_wait_tgtz_unfold_recursion} 
                % By the hypothesis $\exists\Const$ such that $\ValClocks_1\models\Const$ and $\emptyset;\Const~\Entails\mu\alpha.\mu\alpha'.{\TypeS}'''_1$, by the premise of rule \LblTypRec*\ $\alpha:\Const;\Const~\Entails\mu\alpha'.{\TypeS}'''_1$ and $\alpha:\Const,\alpha':\Const;\Const~\Entails{\TypeS}'''_1$, and \CIso*[\ValClocks_1];[{\TypeS}'''_1]\ is \emph{well-formed}.
                % 
                The well-formedness of \CIso*[\ValClocks_1]+{+\ValTime};[\mu\alpha'.{\TypeS}'''_1]\ depends on the well-formedness of \CIso*[\ValClocks_1]+{+\ValTime};[{\TypeS}'''_1]. (See other cases of $S$.) %, as in~\Cref{itm:wf_then_live_recdef} of~\Cref{lem:cfg_wf_then_live}.)
                %
                %
                % ~ wait -> unfold -> end
                \item\NewCase[${\TypeS}''_1=\TypeEnd$]\label{case:cfgs_trans_wf_pres_wait_tgtz_unfold_end} 
                By~\Cref{lem:cfg_wf_end} \CIso*+{+\ValTime};[\TypeEnd]\ is \emph{well-formed}.
                %
                %
                % ~ wait -> unfold -> rec call
                \item\label{case:cfgs_trans_wf_pres_wait_tgtz_unfold_reccall} ${\TypeS}''_1$ cannot equal $\alpha$ by~\Cref{lem:cfg_wf_neq_alpha}.
                %
            \end{inductivecase}
            %
        \end{inductivecase}

        \noindent Therefore, it holds that well-formedness is preserved by transition made by \emph{well-formed} configurations via rule \LblCfgSysWait*.
        %
        %
        %
        %
        %
        % ~ comm
        \item\NewCase[\LblCfgSysLComm*]\label{case:cfgs_trans_wf_pres_comm} 
        % Then by~\cref{case:configs_trans_compat_pres_comm} of~\cref{lem:configs_trans_compat_pres} $\Queue_1=\Queue'_1=\emptyset$.
        The transition is as shown below:
        %
        \begin{minieq}*%\label{eq:cfgs_trans_wf_pres_comm_trans}
        \begin{array}[c]{l}
            \infer[\LblCfgSysLComm]{%
                \Trans{\Parl{\VSoc_1,\VSoc_2}}:{\SiltAction}[\Parl{\VSoc'_1,\VSoc'_2}]
            }{%
                \infer[\LblCfgSocSend]{%
                    \Trans{\VSoc_1}:{\SendMsg}[\VSoc'_1]
                }{\dots}
                & %
                % \infer[\LblCfgSocEnqu]{%
                    \Trans{\VSoc_2}:{\RecvMsg}[\VSoc'_2]
					\quad \LblCfgSocEnqu
                % }{\dots}
            }
            \end{array}
        \end{minieq}
        
        % ~ comm-l s1
        \noindent Focusing first on \VSoc*_1, we proceed by induction on the depth of the derivation tree, analysing the last rule applied for the transition \Trans*{\VIso_1}:{\SendMsg}[\VIso'_1]:
        \begin{inductivecase}
            %
            %
            % ~ comm-l s1 -> act
            \item\NewCase[\LblCfgIsoInteract*]\label{case:cfgs_trans_wf_pres_comm_act} 
            Then $\TypeS_1=\simplechoice$, and the evaluation is shown below:
            %
            \begin{minieq}*\label{eq:cfgs_trans_wf_pres_comm_act_trans}
        \begin{array}[c]{l}
                \infer[\LblCfgSocSend]{%
                    \Trans{\CSoc[\ValClocks_1];[\simplechoice]:{\Queue_1}}:{\SendMsg}[\CSoc[\ValClocks_1]+{\ReSet[]_j};[\TypeS_j]:{\Queue_1}]
                }{%
                    \infer[\LblCfgIsoInteract]{%
                        \Trans{\CIso[\ValClocks_1];[\TypInteract]}:{\SendMsg}[\CIso[\ValClocks_1]+{\ReSet[]_j};[\TypeS_j]]
                    }{%
                        \ValClocks_1\models\Const_j
                        & %
                        {m}={l_j\left\langle T_j \right\rangle}
                        & % 
                        {\TypSend=\TypComm_j}
                        & % 
                        j\in I
                    }
                }
                \end{array}
            \end{minieq}
            
            \noindent By rule \LblTypChoice*\ $\exists\Const_i:\ValClocks_1\models\Past[\Const_i]$ and $\emptyset;\Past[\Const_i]~\Entails\FullChoice$, and by the premise of rule \LblTypChoice*\ it holds that $\Const_i\ReSet[]_i\subseteq\gamma$ and $\emptyset;\gamma~\Entails\TypeS_i$.
            %
            Combined with~\Cref{lem:configs_iso_trans} it holds that $\ValClocks_1\models\Const_j$ and $\emptyset;\Const_j\ReSet[]_j~\Entails\TypeS_j$ and $\ValClocks_1\ReSet[]_j\models\Const_j\ReSet[]_j$.
            
            Therefore, \CIso*[\ValClocks_1]+{\ReSet[]_j};[\TypeS_j]\ is \emph{well-formed}.
            %
            %
            % ~ comm-l s1 -> unfold
            \item\NewCase[\LblCfgIsoUnfold*]\label{case:cfgs_trans_wf_pres_comm_unfold} 
            Then $\TypeS_1=\mu\alpha.{\TypeS}''_1$.
            The transition is as shown below:
            %
            \begin{minieq}*\label{eq:cfgs_trans_wf_pres_comm_unfold_trans}
        \begin{array}[c]{l}
                \infer[\LblCfgIsoUnfold]{%
                    \Trans{\CIso[\ValClocks_1];[\mu\alpha.{\TypeS}''_1]}:{\ProgAction}[\CIso'_1]
                }{%
                    \Trans{\CIso[\ValClocks_1];[{\TypeS}''_1\Subst[\mu\alpha.{\TypeS}''_1][\alpha]]}:{\SendMsg}[\CIso'_1]
                }
                \end{array}
            \end{minieq}
            
            \noindent By the hypothesis $\exists\Const$ such that $\ValClocks_1\models\Const$ and $\emptyset;\Const~\Entails\TypRecDef$, and by the premise of rule \LblTypRec*\ $\alpha:\Const;\Const~\Entails{\TypeS}''_1$, and \CIso*[\ValClocks_1];[{\TypeS}''_1\Subst[\mu\alpha.{\TypeS}''_1][\alpha]]\ is \emph{well-formed}.
            %
            The well-formendess of \CIso*'_1\ is dependant on the state of ${\TypeS}''_1$, which for the transition \Trans*{\CIso[\ValClocks_1];[{\TypeS}''_1\Subst[\mu\alpha.{\TypeS}''_1][\alpha]]}:{\SendMsg}[\CIso'_1]\ must be either $\simplechoice$ or $\mu\alpha'.{\TypeS}'''_1$ (see other cases, as in~\Cref{lem:cfg_wf_then_live}).
            %
        \end{inductivecase}
        %
        % ~ comm-l s1
        Now, focusing on \VSoc*_2, the transition \Trans*{\CSoc_2}:{\RecvMsg}[\CSoc'_2]\ via rule \LblCfgSocEnqu*\ yields $\ValClocks_2'=\ValClocks_2$ and $\TypeS_2'=\TypeS_2$ and $\Queue_2'=\Queue_2;\Msg$ by~\Cref{lem:configs_soc_trans}.
        %
        Therefore \CIso*'_2\ is \emph{well-formed} as $\VIso_2=\VIso'_2$.
        %
        Transitions via rule \LblCfgSysRComm*\ are symmetric and omitted.
        %
        %
        %
        %
        %
        % ~ par-l
        \item\NewCase[\LblCfgSysLPar*]\label{case:cfgs_trans_wf_pres_par} 
        By~\Cref{lem:configs_soc_trans} $\Queue'_1=\Msg;\Queue_1$.
        The transition is as shown below: %in~\Cref{eq:cfgs_trans_wf_pres_par_trans}.
        %
        \begin{minieq}*\label{eq:cfgs_trans_wf_pres_par_trans}
        \begin{array}[c]{l}
            \infer[\LblCfgSysLPar]{%
                \Trans{\Parl{\VSoc_1,\VSoc_2}}:{\SiltAction}[\Parl{\VSoc'_1,\VSoc'_2}]
            }{%
                \infer[\LblCfgSocRecv]{%
                    \Trans{\VSoc_1}:{\SiltAction}[\VSoc'_1]
                }{\dots}
            }
            \end{array}
        \end{minieq}
        
        \noindent We proceed by induction on the depth of the derivation tree, analysing the last rule applied for the transition \Trans*{\VIso_1}:{\RecvMsg}[\VIso'_1]\ via the premise of rule \LblCfgSocRecv*:
        \begin{inductivecase}
            %
            %
            % ~ par-l -> act
            \item\NewCase[\LblCfgIsoInteract*]\label{case:cfgs_trans_wf_pres_par_act} 
            Then $\TypeS_1=\simplechoice$.
            The transition is as shown below: %in~\Cref{eq:cfgs_trans_wf_pres_par_act_trans}.
            %
            \begin{minieq}*\label{eq:cfgs_trans_wf_pres_par_act_trans}
        \begin{array}[c]{l}
                \infer[\LblCfgSocRecv]{%
                    \Trans{\CSoc[\ValClocks_1];[\simplechoice]:{\Msg;\Queue_1}}:{\SiltAction}[\CSoc[\ValClocks_1]+{\ReSet[]_j};[\TypeS_j]:{\Queue_1}]
                }{%
                    \infer[\LblCfgIsoInteract]{%
                        \Trans{\CIso[\ValClocks_1];[\TypInteract]}:{\RecvMsg}[\CIso[\ValClocks_1]+{\ReSet[]_j};[\TypeS_j]]
                    }{%
                        \ValClocks_1\models\Const_j
                        & %
                        {m}={l_j\left\langle T_j \right\rangle}
                        & % 
                        {\TypRecv=\TypComm_j}
                        & % 
                        j\in I
                    }
                }
                \end{array}
            \end{minieq}
            
            \noindent By the hypothesis and the judgement of rule \LblTypChoice*\ $\exists\Const_i$ such that $\emptyset;\Past[\Const_i]~\Entails\simplechoice$ and $\ValClocks_1\models\Past[\Const_i]$, and by the premise of rule \LblTypChoice*\ $\Const_i\ReSet[]_i\subseteq\gamma$ and $\emptyset;\gamma~\Entails\TypeS_i$.
            
            It follows~\Cref{case:cfgs_trans_wf_pres_comm_act} of~\Cref{lem:cfgs_trans_wf_pres} that \CIso*[\ValClocks_1]+{\ReSet[]_j};[\TypeS_j]\ is \emph{well-formed}.
            % Therefore \CIso*[\ValClocks_1]+{\ReSet[]_j};[\TypeS_j]\ is \emph{well-formed} as $\ValClocks_1\models\Const_j$ and $\emptyset;\Const_j\ReSet[]_j~\Entails\TypeS_j$ and $\ValClocks_1\ReSet[]_j\models\Const_j\ReSet[]_j$ (as in~\Cref{case:cfgs_trans_wf_pres_comm_act} of~\Cref{lem:cfgs_trans_wf_pres}).
            %
            %
            % ~ par-l -> unfold
            \item\NewCase[\LblCfgIsoUnfold*]\label{case:cfgs_trans_wf_pres_par_unfold} 
            Then $\TypeS_1=\mu\alpha.{\TypeS}''_1$.
            The transition shown below, and is analogous to the one in~\Cref{eq:cfgs_trans_wf_pres_comm_unfold_trans} of~\Cref{lem:cfgs_trans_wf_pres}:
            % in~\Cref{eq:cfgs_trans_wf_pres_par_unfold_trans} and is analogous to the one in~\Cref{eq:cfgs_trans_wf_pres_comm_unfold_trans}.
            %
            \begin{minieq}\label{eq:cfgs_trans_wf_pres_par_unfold_trans}
        \begin{array}[c]{l}
                \infer[\LblCfgIsoUnfold]{%
                    \Trans{\CIso[\ValClocks_1];[\mu\alpha.{\TypeS}''_1]}:{\ProgAction}[\CIso'_1]
                }{%
                    \Trans{\CIso[\ValClocks_1];[{\TypeS}''_1\Subst[\mu\alpha.{\TypeS}''_1][\alpha]]}:{\RecvMsg}[\CIso'_1]
                }
                \end{array}
            \end{minieq}
            
            \noindent By the hypothesis $\exists\Const$ such that $\ValClocks_1\models\Const$ and $\emptyset;\Const~\Entails\TypRecDef$, and by the premise of rule \LblTypRec*\ $\alpha:\Const;\Const~\Entails{\TypeS}''_1$, and \CIso*[\ValClocks_1];[{\TypeS}''_1\Subst[\mu\alpha.{\TypeS}''_1][\alpha]]\ is \emph{well-formed}.
            %
            The well-formendess of \CIso*'_1\ is dependant on ${\TypeS}''_1$, which for the transition by the premise of rule \LblCfgIsoUnfold*\ must be either $\simplechoice$ or $\mu\alpha'.{\TypeS}'''_1$ (see other cases, as in~\Cref{lem:cfg_wf_then_live}).
            %
        \end{inductivecase}
        %
    \end{inductivecase}

    \noindent Therefore, it holds that any transition made by a system composed of compatible and \emph{well-formed} configurations will result in configurations that are \emph{well-formed}.
    %
\end{proof}
% 

% ~ time passing
% ! 
% \newpage
%
% ! (lemma 13) : time passing implies empty queues
\begin{lemma}\label{lem:sys_compat_time_trans}
   %
   If \Compat*[\VSoc_1][\VSoc_2]\ and \Trans*{\Parl{\VSoc_1,\VSoc_2}}:{\ValTime}\ and $\ValTime>0$ then $\Queue_1=\emptyset=\Queue_2$.
   %
\end{lemma}
\begin{proof}
   %
   Such a transition is only specified by \LblCfgSysWait*, which by its premise requires a \LblCfgSocTime*\ transition of \ValTime*\ for each \VSoc*_1\ and \VSoc*_2.
   %
   By contradiction, if one queue were \emph{non-empty}, say $\Queue_1=\Msg;\Queue_1$, then by~\Cref{itm:configs_compat_expected_receive} of~\Cref{def:configs_compat} message \Msg*\ must be able to be received immediately.
   %
   The premise of \LblCfgSocTime*\ (urgency) ensures that \ValTime*\ must be valued such that no time passes while a message is able to be received.
   %
   % It holds that $t$ must equal 0 when there is a non-empty queue.
   
   Therefore the hypothesis holds.
   %if a system makes a $t$ transition where $t>0$ then all queues in the system must be empty.
   % Therefore, \ValTime*\ must equal $0$ when there is a message in any queue in a system composed of compatible configurations.
   %
 \end{proof}
 % 

%
% ! (lemma 15) : compat, single transition -> compat
\begin{lemma}\label{lem:configs_trans_compat_pres}
	%
	If \VIso*_1\ and \VIso*_2\ are both \emph{well-formed} 
    and \Compat*[\VSoc_1][\VSoc_2]\ 
    and \Trans*{\Parl{\VSoc_1,\VSoc_2}}[\Parl{\VSoc'_1,\VSoc'_2}], 
	then \Compat*[\VSoc'_1][\VSoc'_2].
	%
\end{lemma}
\begin{proof}
	%
	We proceed by induction on the depth of the derivation tree, analysing each case of the last rule applied for the transition \Trans*{\Parl{\VSoc_1,\VSoc_2}}[\Parl{\VSoc'_1,\VSoc'_2}]:
	% \begin{inline}+
	% 	\item \LblCfgSysWait*
	% 	\item \LblCfgSysLComm*
	% 	\item \LblCfgSysLPar*
	% \end{inline}
	%
	\begin{inductivecase}
		%
		%
		%
		%
		%
		% ~ wait
		\item\NewCase[\LblCfgSysWait*]\label{case:configs_trans_compat_pres_wait}
		Then both \VSoc*_1\ and \VSoc*_2\ make a \ValTime*\ transition via \LblCfgSocTime*\ as shown in~\Cref{eq:configs_trans_compat_pres_wait_trans}.
		%
        If $\ValTime=0$ then by~\Cref{lem:configs_iso_trans,lem:configs_soc_trans} $\VSoc_1=\VSoc'_1$ and $\VSoc_2=\VSoc'_2$ and the hypothesis holds; \Compat*[\VSoc'_1][\VSoc'_2].
        %
        \begin{minieq}\label{eq:configs_trans_compat_pres_wait_trans}
            \infer[\LblCfgSysWait]{%
                \Trans{\Parl{\VSoc_1,\VSoc_2}}:{\ValTime}[\Parl{\VSoc'_1,\VSoc'_2}]
            }{%
                \infer[\LblCfgSocTime]{%
                    \Trans{\VSoc_1}:{\ValTime}[\VSoc'_1]
                }{\dots}
                & %
                \infer[\LblCfgSocTime]{%
                    \Trans{\VSoc_2}:{\ValTime}[\VSoc'_2]
                }{\dots}
            }
        \end{minieq}
        
        \noindent If $\ValTime>0$ then by~\Cref{lem:configs_iso_trans,lem:configs_soc_trans} $\ValClocks'_1=\ValClocks_1+\ValTime$ and $\TypeS'_1=\TypeS_1$ and $\Queue'_1=\Queue_1$ (and the same for \ValClocks*'_2, \TypeS*'_2\ and \Queue*'_2).
		%
        By~\Cref{lem:cfgs_trans_wf_pres} \VIso*'_1\ and \VIso*'_2\ are both \emph{well-formed}.
		%
		By~\Cref{lem:sys_compat_time_trans} $\Queue_1=\emptyset=\Queue_2$ and by~\Cref{itm:configs_compat_dual_types} of~\Cref{def:configs_compat} $\ValClocks_1=\ValClocks_2$ and $\TypeS_1=\Dual[\TypeS_2]$.
		%
		Therefore \Compat*[\CSoc[\ValClocks_1]+{+\ValTime};[\TypeS_1]:{\emptyset}][\CSoc[\ValClocks_2]+{+\ValTime};[\Dual[\TypeS_2]]:{\emptyset}].
		%
		%
		%
		%
		%
		% ~ comm
		\item\NewCase[\LblCfgSysLComm*]\label{case:configs_trans_compat_pres_comm}
		By~\cref{lem:cfgs_trans_wf_pres} both \VIso*'_1\ and \VIso*'_2\ are \emph{well-formed}.
        The transition is as shown below: %in~\cref{eq:configs_trans_compat_pres_comm_trans}.
		%
		\begin{minieq}*\label{eq:configs_trans_compat_pres_comm_trans}
			% \begin{array}{c}%\mathllap{%
			\resizebox{\linewidth}{!}{$%
				\infer[\LblCfgSysLComm]{%
					\Trans{\Parl{\CSoc[\ValClocks_1];[\TypeS_1]:{\emptyset},\CSoc[\ValClocks_2];[\TypeS_2]:{\Queue_2}}}:{\SiltAction}[\Parl{\CSoc[\ValClocks'_1];[\TypeS'_1]:{\emptyset},\CSoc[\ValClocks_2];[\TypeS_2]:{\Queue_2;\Msg}}]
				}{%
					\infer[\LblCfgSocSend]{%
						\Trans{\CSoc[\ValClocks_1];[\TypeS_1]:{\emptyset}}:{\SendMsg}[\CSoc[\ValClocks_1]+{\ReSet[]_j};[\TypeS_j]:{\emptyset}]
					}{%
						\infer[\LblCfgIsoInteract]{%
							\Trans{\CIso[\ValClocks_1];[\TypInteract]}:{\SendMsg}[\CIso[\ValClocks_1]+{\ReSet[]_j};[\TypeS_j]]
						}{%
                        \ValClocks_1\models\Const_j
                        & %
                        {m}={l_j\left\langle T_j \right\rangle}
                        & % 
                        {\TypSend=\TypComm_j}
                        & % 
                        j\in I
						}
					}
					& %
					% \infer[\LblCfgSocEnqu]{%
						\Trans{\VSoc_2}:{\RecvMsg}[\CSoc[\ValClocks_2];[\TypeS_2]:{\Queue_2;\Msg}]
						\quad \LblCfgSocEnqu
					% }{\dots}
				}
			$}%
	%	}\end{array}
		\end{minieq}
		
        \noindent We proceed by inner induction on each combination of the contents of queues:
		\begin{inductivecase}
			%
			%
			% ~ comm -> e e
			\item\NewCase[$\Queue_1=\emptyset$, $\Queue_2=\emptyset$]\label{case:configs_trans_compat_pres_comm_ee}
			By~\Cref{itm:configs_compat_dual_types} of~\Cref{def:configs_compat} $\ValClocks_1=\ValClocks_2$ and $\TypeS_1=\Dual[\TypeS_2]$.
			%
			The resulting system is no longer \emph{dual}.
			%
			By~\cref{lem:cfgs_trans_wf_pres,lem:sys_compat_time_trans} time cannot pass if $\Queue_2\neq\emptyset$.
			%
			By~\Cref{def:types_dual} the message \Msg*\ sent by \VSoc*_1\ must have a corresponding receiving action in \VSoc*_2\ as in~\Cref{itm:configs_compat_expected_receive} of~\Cref{def:configs_compat}.
			%
			Therefore \Compat*[\CSoc[\ValClocks'_1];[{\TypeS}'_1]:{\emptyset}][\CSoc;+{\Msg}_2].
			%
			%
			%
			% ~ comm -> e m
			\item\NewCase[$\Queue_1=\emptyset$, $\Queue_2\neq\emptyset$]\label{case:configs_trans_compat_pres_comm_en}
			By~\Cref{itm:configs_compat_expected_receive} of~\Cref{def:configs_compat} $\exists\Msg',{\ValClocks}'',{\TypeS}''$ such that $\Queue_2=\Msg';\Queue_2$ and \Trans*{\CIso_2}:{\TypRecv,\Msg'}[\CIso[{\ValClocks}''_2];[{\TypeS}''_2]]\ and \Compat*[\CSoc[\ValClocks'_1];[{\TypeS}'_1]:{\emptyset}][\CSoc[{\ValClocks}''_2];[{\TypeS}''_2]:{{\Queue}_2}]\ and by \LblCfgSocTime*\ (urgency) time cannot pass.
			%
			If a system has a configuration with sequence of outgoing sending actions and each has constraints that are satisfiable immediately after the other, then the system can both receive the messages as they arrive, or accumulate the messages and instantly receive each in succession and become \emph{dual} again (by inspection of~\Cref{def:types_dual,def:configs_compat} and~\Cref{fig:types_rule,fig:typesemantics_tuple,fig:typesemantics_triple}).
			%
			Therefore \Compat*[\CSoc[\ValClocks'_1];[{\TypeS}'_1]:{\emptyset}][\CSoc[{\ValClocks}''_2];[{\TypeS}''_2]:{\Msg';{\Queue}_2;\Msg}].
			%
			%
			% ~ comm -> m e
			\item\NewCase[$\Queue_1\neq\emptyset$, $\Queue_2=\emptyset$]\label{case:configs_trans_compat_pres_comm_ne}
			Contradicts the hypothesis by~\Cref{itm:configs_compat_expected_receive} of~\Cref{def:configs_compat} as by \LblCfgSocTime*\ (urgency) messages must be removed from a queue immediately, and by \LblTypChoice*\ of~\Cref{fig:types_rule} sending and receiving actions cannot be performed at the same time.
			%
			%
			% ~ comm -> m m
			\item\NewCase[$\Queue_1\neq\emptyset$, $\Queue_2\neq\emptyset$]\label{case:configs_trans_compat_pres_comm_nn}
			Contradicts the hypothesis by~\Cref{itm:configs_compat_non_empty_queues} of~\Cref{def:configs_compat}.
			%
		\end{inductivecase}

		\noindent Therefore, compatibility is preserved across \LblCfgSysLComm*\ transitions.
		%
		%
		%
		%
		%
		% ~ dequ
		\item\NewCase[\LblCfgSysLPar*]\label{case:configs_trans_compat_pres_dequ}
		By~\Cref{lem:configs_soc_trans} $\Queue_2=\Msg;\Queue_2$ and by~\Cref{itm:configs_compat_expected_receive} of~\Cref{def:configs_compat} \Compat*[\VSoc'_1][\VSoc'_2], the hypothesis holds.
		%
	\end{inductivecase}

	\noindent Therefore, it holds that any transition made by a compatible system composed of well-formed types will result in configurations that are \emph{compatible}.
	%
\end{proof}
% 

% ~ preservation is preserved 
% ! 
%
% ! (lemma 16) : compat wf, end or future enabled
\begin{lemma}\label{lem:configs_compat_wf_fe}
	%
	If both \VIso*_1\ and \VIso*_2\ are \emph{well-formed} 
    and \Compat*[\VSoc_1][\VSoc_2],
	then both \VSoc*_1\ and \VSoc*_2\ are \emph{final} 
	or $\exists\ValTime$ such that \Trans*{\VSys}:{\ValTime,\SiltAction}[\Parl{\VSoc'_1,\VSoc'_2}].%
	%
\end{lemma}
\begin{proof}
	%
	By~\Cref{lem:cfgs_trans_wf_pres} \VIso*'_1\ and \VIso*'_2\ are \emph{well-formed} and by~\Cref{lem:cfg_wf_then_live} are \emph{live}.
	%
	By~\Cref{def:types_progress} if \VSoc*_2\ is \emph{final} then $\VSoc_2=\CSoc[\ValClocks_2];[\TypeEnd]:{\emptyset}$.
    %
    We proceed with the assumption that at least one participant is \emph{not final}, and hereafter only consider \VSoc*'_1.
	%
	The transition is given below:
 %in~\Cref{eq:configs_compat_wf_fe_trans}.
	%
	\begin{minieq}*\label{eq:configs_compat_wf_fe_trans}
		\Trans{\VSys}:{\ValTime}[\Trans{\Parl{\CSoc+{+\ValTime}_1,\CSoc+{+\ValTime}_2}}:{\SiltAction}[\Parl{\CSoc'_1,\CSoc'_2}]]
	\end{minieq}
	%
	We proceed only considering each case of \VSoc*_1\ not being \emph{final}.
	By induction on the cases of \TypeS*_1:
	%
	\begin{inductivecase}
		%
		%
		% ~ choice
		\item\NewCase[$\TypeS=\simplechoice$] 
		% Then by judgement of \LblTypChoice*\ $\exists\Const_i$ such that $\ValClocks\models\Past[\Const_i]$ and $\emptyset;\Past[\Const_i]~\Entails\simplechoice$.
		%
		As described in~\Cref{sec:types} we write $\Past$ if $\exists\ValTime$ such that $\ValClocks+\ValTime\models\Const$.
		Therefore, if $\ValClocks\not\models\Const$ and $\emptyset;\Const~\Entails\TypeS$ for a \emph{well-formed} \CIso*\ then $\ValClocks\models\Past[\Const]$.
		%
		By~\Cref{lem:cfgs_trans_wf_pres} the only possible values of \ValTime*\ ensure that the latest system-wide sending action is never missed and messages are received as soon as they arrive in a queue by rule \LblCfgSocTime*.
		
		Therefore, the hypothesis holds for systems composed where one participant is known to be a \emph{non-final} choice type.
		%
		%
		% ~ rec def
		\item\NewCase[$\TypeS=\mu\alpha.\TypeS'$]
		It follows~\Cref{lem:cfgs_trans_wf_pres} that \TypeS*'\ is \emph{well-formed} against $\ValClocks+\ValTime$.
		%
	\end{inductivecase}
	
	\noindent Therefore, if a \emph{well-formed} and compatible system \Parl*{\VSoc_1,\VSoc_2}\ that is not \emph{final}, then there is some value of time $\ValTime\geq 0$ that will enable a future action, which will result in a \emph{well-formed} and compatible system \Parl*{\VSoc'_1,\VSoc'_2}, which may or may not be \emph{final}, and to which this behaviour still applies.
	%
\end{proof}
% 
%
% ! (lemma 18) : compat wf, any amount of transitions -> compat wf
\begin{lemma}\label{lem:configs_trans_compat_wf_pres}
    %
    If both \VIso*_1 and \VIso*_2 are \emph{well-formed}
    and \Compat*[\VSoc_1][\VSoc_2]\ 
    and $\Trans{\Parl{\VSoc_1\!,\VSoc_2}}*[\Parl{\VSoc'_1\!,\VSoc'_2}]$, 
    then \Compat*[\VSoc'_1][\VSoc'_2]\ 
    and both \VIso*'_1\ and \VIso*'_2\ are \emph{well-formed}.%
    %
\end{lemma}
\begin{proof}
    %
    By~\Cref{lem:cfgs_trans_wf_pres,lem:configs_trans_compat_pres} the hypothesis holds for single transitions and that the resulting configurations are \emph{live}, and either \emph{final} or \emph{satisfies progress} by~\Cref{lem:cfg_wf_then_live,lem:configs_compat_wf_fe}.
    %
    Therefore it holds that \Compat*[\VSoc'_1][\VSoc'_2]\ 
    and both \VIso*'_1\ and \VIso*'_2\ are \emph{well-formed} across an arbitrary number of transitions, as each single transition preserves compatibility and well-formedness.
    %
\end{proof}
% 

% ~ system progress
% ! 
%
% ! (lemma 20) : sys cfg progress
\begin{lemma}\label{lem:configs_sys_progress}
    %
    For all \TypeS*, \ValClocks* such that \ITJudgement*[\emptyset];{\Const}[\TypeS]\ and \Sat*\ :\par\noindent
    \hfill\ \Parl*{\CSoc:{\emptyset},\CSoc;[\Dual]:{\emptyset}}\ satisfies progress.\hfill\ \ %
    %
\end{lemma}
\begin{proof}
    %
    By the hypothesis the system is compatible and composed of well-formed dual types. 
    %
    By~\Cref{lem:configs_trans_compat_wf_pres} any configurations reachable by such a system will be compatible and \emph{well-formed}.
    %
    By~\Cref{lem:configs_compat_wf_fe} such a system adheres to~\Cref{def:types_progress} and \emph{satisfies progress}.
    %
\end{proof}
% 

% ! thm proof
\ThmProgress*
\begin{proof}
   %
   By~\Cref{def:types_wf}, types \TypeS*\ and \Dual*\ are always \emph{well-formed} against \ValClocks*_0.
   %
   By \Cref{lem:init_wf_then_live} and~\Cref{itm:configs_compat_dual_types} of~\Cref{def:configs_compat}, both \CSoc*[\ValClocks_0]:{\emptyset}\ and \CSoc*[\ValClocks_0];[\Dual]:{\emptyset} are \emph{live} and \emph{compatible}.
   %
   It follows~\Cref{lem:configs_compat_wf_fe,lem:configs_trans_compat_wf_pres} that such as system will always perform actions when possible, waiting if necessary (never missing the \emph{latest-enabled} action), until reaching a \emph{final} configuration, and any \emph{non-final} configuration is guaranteed to be \emph{well-formed}, \emph{compatible} and \emph{live}.

   Therefore, it holds that an initial system composed of dual types that are well-formed is compatible, and guaranteed to \emph{satisfy progress}.
   %
\end{proof}


% 
Building upon the notion of well-formed types (\cref{def:types_well_formed}), a well-formed configuration exhibits behaviour without deadlocks, where there is always some action able to be performed in the future, until the termination type is reached.
This behaviour is referred to as \emph{liveness}, and is given in~\cref{def:configs_iso_live}, and depends on \emph{future enabled} actions.
\input{tex/defs/configs/iso/future_en.tex}
\input{tex/defs/configs/iso/live.tex}
The liveness of a configuration indicates whether a deadlock has been reached. There must either be an action immediately viable, some future enabled actions, or the type must have reached termination.
\input{tex/defs/configs/iso/well_formed.tex}
Given a type \TypeS*\ well-formed against \ValClocks*, it holds that a configuration of \IsoCfg*\ will have future enabled actions, and therefore be live.
\begin{note}
    For a \emph{well-formed} \TypeS*, \CfgIso*1{\ValClocks_{0}} is \emph{live}.
\end{note}


\newcommand{\FullChoice}[0]{\left\{\TypComm_i \,\MsgLabel_i\left\langle T_i\right\rangle \left(\delta_i,\lambda_i\right).\TypeS_i\right\}_{i\in I}}

\begin{lemma}\label{lem:cfg_wf_neq_alpha}
   %
   If \CIso*\ is \emph{well-formed} then $\TypeS\neq\alpha$.
   %
\end{lemma}
\begin{proof}
   %
   By the hypothesis \CIso*\ is \emph{well-formed} meaning $\exists\Const$ such that $\ValClocks\models\Const$ and  $\emptyset;\Const~\Entails\TypeS$.

   Consider by contradiction of the hypothesis that $\TypeS=\alpha$.
   If \CIso*;[\alpha]\ is \emph{well-formed} then, by the judgement of rule \LblTypVar*\ $\alpha:\Const;\Const~\Entails\alpha$ and the evaluation shown in~\Cref{eq:cfg_wf_neq_alpha_eval} must hold, where there must be some recursive definition earlier in the derivation tree $\mu\alpha.\TypeS'$ corresponding to $\alpha$.
   %
   \begin{minieq}\label{eq:cfg_wf_neq_alpha_eval}
   \begin{array}[c]{l}
      \infer[\LblTypRec]{%
         \emptyset;\Const~\Entails\mu\alpha.\TypeS'%
      }{%   
         \infer{\alpha:\Const;\Const~\Entails\TypeS'}{%
            \infer[\vdots]{}{%
               \infer[\LblTypVar]{%
                  \alpha:\Const;\Const~\Entails\alpha%
               }{}%
            }%
         }%
      }
    \end{array}
   \end{minieq}
   
   \noindent If $\exists\ValClocks',\mu\alpha.\TypeS'$ such that \CIso*[\ValClocks'];[\mu\alpha.\TypeS']\ is \emph{well-formed} and \Trans*{\CIso[\ValClocks'];[\mu\alpha.\TypeS']}*[\CIso;[\alpha]], then the immediate transition must be via rule \LblCfgIsoUnfold*\ as shown in~\Cref{eq:cfg_wf_neq_alpha_unfold}; where either $\CIso[{\ValClocks}''];[{\TypeS}'']=\CIso;[\alpha]$ or \Trans*{\CIso[{\ValClocks}''];[{\TypeS}'']}*[\CIso;[\alpha]].
   %
   By the premise of rule \LblCfgIsoUnfold*\ any recursive calls following their definition are replaced by the definition of their next unfolding denoted by $\TypeS'\Subst[\mu\alpha.\TypeS'][\alpha]$.
   %
   \begin{minieq}\label{eq:cfg_wf_neq_alpha_unfold}
   \begin{array}[c]{l}
      \infer[\LblCfgIsoUnfold]{%
      \Trans{\CIso[\ValClocks'];[\mu\alpha.\TypeS']}:{\ProgAction}[\CIso[{\ValClocks}''];[{\TypeS}'']]
      }{%
      \Trans{\CIso[\ValClocks'];[{\TypeS}'\Subst[\mu\alpha.\TypeS'][\alpha]]}:{\ProgAction}[\CIso[{\ValClocks}''];[{\TypeS}'']]
      }
    \end{array}
   \end{minieq}
   
   \noindent Therefore, if \CIso*\ is \emph{well-formed} then $\TypeS\neq\alpha$.
   %
\end{proof}

% \begin{definition}[Live Configurations]\label{def:configs_live}
    \CIso*\ is \emph{live} if
    $\TypeS=\TypeEnd$ or if \VIso*\ is \isFE*.
\end{definition}

\begin{lemma}\label{lem:cfg_wf_end}
   %
   \CIso*;[\TypeEnd]\ is always \emph{well-formed}.
   %
\end{lemma}
\begin{proof}
   %
   By the hypothesis $\exists\Const$ such that $\ValClocks\models\Const$ and $\emptyset;\Const~\Entails\TypeS$, and by rule \LblTypEnd*\ $\emptyset;\TypeTrue~\Entails\TypeEnd$.
%   
   Therefore the hypothesis holds as $\ValClocks\models\TypeTrue$ always holds.
   %
\end{proof}

%
\begin{lemma}\label{lem:cfg_wf_then_live}
    %
    If \CIso*\ is \emph{well-formed}, then \CIso*\ is \emph{live}.
    %
\end{lemma}
\begin{proof}
    %
    By the hypothesis, there must $\exists\Const$ such that $\emptyset;\Const~\Entails\TypeS$ and $\ValClocks\models\Const$.
    %
    We proceed by induction on each case of \TypeS*:
    \begin{inductivecase}
        %
        %
        %
        % ~ choice
        \item\NewCase[$\TypeS=\simplechoice$]\label{itm:cfg_wf_then_live_choice} 
        By the hypothesis and the judgement of rule \LblTypChoice*\ $\exists\Const_i$ such that $\ValClocks\models\Past[\Const_i]$ and $\emptyset;\Past[\Const_i]~\Entails\FullChoice$.
        %
        By~\Cref{def:configs_wf,def:configs_fe} \CIso*\ is \isFE*, and therefore, by~\Cref{def:configs_live} \CIso*\ is \emph{live}.
        %
        %
        %
        % ~ rec def
        \item\NewCase[$\TypeS=\mu\alpha.{\TypeS}'$]\label{itm:wf_then_live_recdef} 
        By the hypothesis $\exists\Const$ such that $\ValClocks\models\Const$ and $\emptyset;\Const~\Entails\TypRecDef$, and by the premise of rule \LblTypRec*\ $\alpha:\Const;\Const~\Entails{\TypeS}'$ and \CIso*;[\TypeS']\ is \emph{well-formed}.
        
        We proceed by inner induction on each case of \TypeS*':
        \begin{inductivecase}
            %
            %
            % ~ rec def -> choice
            \item\NewCase[$\TypeS'=\simplechoice$] 
            By the judgement of rule \LblTypChoice*\ and the premise of rule \LblTypRec*, $\exists\Const_i$ such that $\ValClocks\models\Const_i$ and $\emptyset;\Const_i~\Entails\TypRecDef$ and $\alpha:\Const_i;\Past[\Const_i]~\Entails\FullChoice$.
            %
            It follows~\Cref{itm:cfg_wf_then_live_choice} of~\Cref{lem:cfg_wf_then_live} that, by~\Cref{def:configs_wf,def:configs_fe,def:configs_live}, \CIso*;[\simplechoice]\ is \isFE*, \emph{well-formed} and \emph{live}.
            Therefore, it holds that \CIso*\ is \emph{live}.
            % Then by the judgement of \LblTypChoice*\ and the premise of \LblTypRec*\ $\exists\Const_i$ such that $\ValClocks\models\Const_i$ and $\emptyset;\Const_i~\Entails\TypRecDef$ and $\alpha:\Const_i;\Past[\Const_i]~\Entails\simplechoice$ and \CIso*;[\simplechoice]\ is \emph{well-formed} and \emph{future-enabled} by~\cref{def:configs_fe}.
            %
            % Therefore \CIso*;[\simplechoice]\ is \emph{live} by definition.
            %
            %
            % ~ rec def -> rec def
            \item\NewCase[$\TypeS'=\mu\alpha'.{\TypeS}''$] 
            Then $\alpha:\Const;\Const~\Entails\mu\alpha'.{\TypeS}''$ and $\alpha:\Const,\alpha':\Const;\Const~\Entails{\TypeS}''$ and \CIso*;[{\TypeS}'']\ is \emph{well-formed}.
            %
            Therefore, \CIso*\ is \emph{live} if \CIso*;[{\TypeS}'']\ is \emph{live} (see other cases).
            %
            %
            % ~ rec def -> end
            \item $\TypeS'=\TypeEnd$ is \emph{live} by~\Cref{def:configs_live}.
            %
            %
            % ~ rec def -> alpha
            \item ${\TypeS'}$ cannot equal $\alpha$ by~\Cref{lem:cfg_wf_neq_alpha}.
            %
        \end{inductivecase}
        %
        %
        %
        % ~ end
        \item\NewCase[$\TypeS=\TypeEnd$]\label{itm:wf_then_live_end} 
        By~\Cref{def:configs_live} it holds that \CIso*;[\TypeEnd]\ is live.
        %
        %
        %
        % ~ alpha
        \item\NewCase[$\TypeS\neq\alpha$] By~\Cref{lem:cfg_wf_neq_alpha}.
        %
    \end{inductivecase}

    \noindent Therefore, it holds that for any $S$ that \CIso*\ is \emph{well-formed}, \CIso*\ is \emph{live}.
    %
\end{proof}
%
%

%

\begin{lemma}\label{lem:init_wf_then_live}
   %
   Given a \emph{well-formed} \TypeS*, \CIso*[\ValClocks_0]\ is \emph{live}.
   %
\end{lemma}
\begin{proof}
   %
   By~\Cref{def:types_wf}, \Sat*[\ValClocks_0]:[\Past]\ holds for any valid \Const*.
   %
\end{proof}

% ~ configuration transitions
% ! 
%
% ! (lemma 12) : iso cfg trans
\begin{lemma}\label{lem:configs_iso_trans}
   %
   The following holds for transitions of configurations:
   \begin{minieq}*
      \begin{array}[c]{c c l}
         %
         \Trans{\CIso}:{\ValTime}[\CIso']%
         & \implies & %
         \begin{array}[c]{lcl}
            %
            \ValClocks'=\ValClocks+\ValTime & %
            \land & %
            \TypeS'=\TypeS%
            %
         \end{array}
         %
         \\[-1ex]\\%
         \Trans{\CIso}:{\CommAction}[\CIso']%
         & \implies & %
         \begin{array}[t]{lcl c lcl c lcl}
            %
            \ValClocks & \models & \Const & %
            \land & %
            \ValClocks' & = & \ReSet[\ValClocks] & %
            \land & %
            \Const' & = & \ReSet[\Const]%
            %
            \\[-1ex]\\%
            \mathllap{\land\;}\ValClocks' & \models & \Const' & %
            \land & %
            \Const' & \subseteq & \TypEnvCond[\TypeS'] %
            %
         \end{array}%
         %
      \end{array}%
   \end{minieq}
   %
\end{lemma}
\begin{proof}
   %
   By inspection of the formation and transition rules in~\Cref{fig:types_rule,fig:typesemantics_tuple}.
   %
\end{proof}
%
%
% ! (lemma 11) : soc cfg trans
\begin{lemma}\label{lem:configs_soc_trans}
   %
   The following holds for transitions of configurations with queues:
   \begin{minieq}*
      \begin{array}[c]{c c lcl c lcl c l}
         %
         \Trans{\CSoc}:{\ValTime}[\CSoc']%
         & \implies & %
            %
            \TypeS' & = & \TypeS & \land & %
            \Queue' & = & \Queue & \land & %
            \ValClocks'=\ValClocks+\ValTime%
            %
         \\[1ex]%
         \Trans{\CSoc}:{\RecvMsg}[\CSoc']%
         & \implies & %
            %
            \TypeS' & = & \TypeS & \land & %
            \Queue' & = & \Queue;\Msg & \land & %
            \ValClocks'=\ValClocks
            %
         \\[1ex]%
         \Trans{\CSoc}:{\SendMsg}[\CSoc']%
         & \implies & %
            %
            \MsgType & = & \Msg & \land & %
            \Queue' & = & \Queue & \land & %
            \Trans{\CIso}:{\SendAction}[\CIso'] %
            %
         \\[1ex]%
         \Trans{\CSoc}:{\SiltAction}[\CSoc']%
         & \implies & %
            % 
            \MsgType & = & \Msg & \land & %
            \Queue' & = & \Msg;\Queue & \land & %
            \Trans{\CIso}:{\RecvAction}[\CIso']%
            %
      \end{array}
   \end{minieq}
   %
\end{lemma}
\begin{proof}
   %
   By inspection of the transitions in~\Cref{fig:typesemantics_triple}, supported by~\Cref{lem:configs_iso_trans}.
   %
\end{proof}
%

% ~ preservation of wf and compat
% \begin{definition}[Compatible Systems]\label{def:configs_compat}
    %
    Let $\VSoc_1 = \CSoc_1$ and $\VSoc_2 = \CSoc_2$. 
    System \VSys*\ is \emph{compatible} (written $\VSoc_1\bot\, \VSoc_2$) if:
    \begin{enumerate}
    \item\label{itm:configs_compat_non_empty_queues} $\Queue_1=\emptyset%
            ~\lor~%
            \Queue_2=\emptyset$%
        %
        \\
       \item\label{itm:configs_compat_dual_types} $\Queue_1=\Queue_2=\emptyset%
            ~\implies~%
            \ValClocks_1=\ValClocks_2%
            ~\land~%
            \TypeS_1=\Dual_2$
        %
        \\
  \item
    \label{itm:configs_compat_expected_receive} 
    $\Queue_1=\Msg;\Queue'_1
            ~\Implies~%
            \exists\ValClocks'_1,\TypeS'_1:
            \Trans{\CIso_1}:{\RecvMsg}[\CIso'_1]%
            ~\land~%
            \CSoc'_1 \bot\, \VSoc_2$
         %   \Compat[\VSoc_1'][\VSoc_2]$%
            \\
\item
    %\label{itm:configs_compat_expected_receive} 
    $\Queue_2=\Msg;\Queue'_2
            ~\Implies~%
            \exists\ValClocks'_2,\TypeS'_2:
            \Trans{\CIso_2}:{\RecvMsg}[\CIso'_2]%
            ~\land~%
            \VSoc_1 \bot\, \CSoc'_2$
         %   \Compat[\VSoc_1'][\VSoc_2]$%
            \\
        % $\forall i, j\in\mkSet[1,2]:%
        % i\neq j%
        % \quad%
        % \Queue_i=\Msg;\Queue_i'%
        % ~\Implies~%
        % \exists\ValClocks'_i,\TypeS'_i:%
        % \Trans{\CIso_i}:{\RecvMsg}[\CIso'_i]%
        % ~\land~%
        % \Compat[\VSoc'_i][\VSoc_j]$%
        %
    \end{enumerate}
    %
    % \noindent We write \Compat*\ if system \VSys*\ is compatible.
    % \Cref{itm:configs_compat_expected_receive} is symmetric.
\end{definition}
% \begin{definition}[Latest-enabled Action (\isLE*)]\label{def:configs_le}
    %
    A configuration \CIso*, that is future-enabled, has a \emph{latest-enabled send} (resp. \emph{latest-enabled receive}), or \isLE*!\ for short (resp. \isLE*?), 
    if $\forall t$ such that $\CIso+{+t}\isFE~\Implies~\exists t'\geq t: \Trans{\VIso}:{\ValTime',\SendMsg}$.
\end{definition}

% ! 
%
% ! (lemma 17) : compat wf, single transition -> wf
\begin{lemma}\label{lem:cfgs_trans_wf_pres}
    %
    If \VIso*_1\ and \VIso*_2\ are both \emph{well-formed} 
    and \Compat*[\VSoc_1][\VSoc_2]\ 
    and \Trans*{\Parl{\VSoc_1,\VSoc_2}}[\Parl{\VSoc'_1,\VSoc'_2}], 
    then both \VIso*'_1\ and \VIso*'_2\ are \emph{well-formed}.
    %
\end{lemma}
\begin{proof}
    %
    We proceed by induction on the depth of the derivation tree, analysing each case of the last rule applied for the transition \Trans*{\Parl{\VSoc_1,\VSoc_2}}[\Parl{\VSoc'_1,\VSoc'_2}].
    % \begin{inline}+
    %     \item \LblCfgSysWait*
    %     \item \LblCfgSysLComm*
    %     \item \LblCfgSysLPar*
    % \end{inline}
    %
    \begin{inductivecase}
        %
        %
        %
        %
        %
        % ~ wait
        \item\NewCase[\LblCfgSysWait*]\label{case:cfgs_trans_wf_pres_wait} 
        As shown in~\Cref{eq:cfgs_trans_wf_pres_wait_trans}, both \VSoc*_1\ and \VSoc*_2\ make a rule \LblCfgSocTime*\ transition with the same valuation of \ValTime*.
        If $\ValTime=0$ then by~\Cref{lem:configs_iso_trans,lem:configs_soc_trans} $\VIso_1=\VIso'_1$ and $\VIso_2=\VIso'_2$ and both \VIso*'_1\ and \VIso*'_2\ are \emph{well-formed}.
        %
        \begin{minieq}\label{eq:cfgs_trans_wf_pres_wait_trans}
        \begin{array}[c]{l}
            \infer[\LblCfgSysWait]{%
                \Trans{\Parl{\VSoc_1,\VSoc_2}}:{\ValTime}[\Parl{\VSoc'_1,\VSoc'_2}]
            }{%
                \infer[\LblCfgSocTime]{%
                    \Trans{\VSoc_1}:{\ValTime}[\VSoc'_1]
                }{\dots}
                & %
                \infer[\LblCfgSocTime]{%
                    \Trans{\VSoc_2}:{\ValTime}[\VSoc'_2]
                }{\dots}
            }
            \end{array}
        \end{minieq}
            
        \noindent We proceed with only \VSoc*_1\ as the analysis covers \VSoc*_2.
        If $\ValTime>0$ then by~\Cref{lem:configs_iso_trans,lem:configs_soc_trans} $\ValClocks'_1=\ValClocks_1+\ValTime$ and $\TypeS'_1=\TypeS_1$.
        %
        By induction on the depth of the derivation tree, analysing the last rule applied for the transition \Trans*{\VIso_1}:{\ValTime}[\VIso'_1]:
        \begin{inductivecase}
            %
            %
            % ~ wait -> tick
            \item\NewCase[\LblCfgIsoTick*]\label{case:cfgs_trans_wf_pres_wait_tgtz_tick} 
            Then \Trans*{\CIso_1}:{\ValTime}[\CIso+{+\ValTime}_1].
            We proceed by inner induction on the different cases of \TypeS*_1:
            \begin{inductivecase}
                %
                % ~ wait -> tick -> choice
                \item\NewCase[$\TypeS_1=\simplechoice$]\label{case:cfgs_trans_wf_pres_wait_tgtz_tick_choice} 
                By rule \LblCfgIsoInteract*\ $\exists\Const_i$ such that $\ValClocks_1\models\Past[\Const_i]$ and $\emptyset;\Past[\Const_i]~\Entails\FullChoice$, as in~\Cref{itm:cfg_wf_then_live_choice} of~\Cref{lem:cfg_wf_then_live}.
                %
                By induction hypothesis $\exists\Const'_i$ such that $\ValClocks_1+\ValTime\models\Past[\Const'_i]$ and $\emptyset;\Past[\Const'_i]~\Entails\FullChoice$, which is assured by the (persistency) premise of rule \LblCfgSocTime*.
                %
                Therefore, it holds that \CIso*[\ValClocks_1]+{+\ValTime};[\simplechoice]\ is \emph{well-formed}.
                % By~\Cref{def:configs_fe} \CIso*_1\ is \emph{future-enabled} and by rule \LblCfgSocTime*\ $\text{(persistency)}$ it holds that \CIso*+{+\ValTime}_1\ is also \emph{future-enabled}.
                % %
                % Therefore the hypothesis holds, \CIso*[\ValClocks_1]+{+\ValTime};[\simplechoice]\ is \emph{well-formed}; by~\Cref{def:types_wf} and rule \LblTypChoice*\ $\exists\Const'_i$ such that $\ValClocks_1+\ValTime\models\Past[\Const'_i]$ and $\emptyset;\Past[\Const'_i]~\Entails\simplechoice$.
                %
                %
                % ~ wait -> tick -> recursion
                \item\NewCase[$\TypeS_1=\mu\alpha.{\TypeS}''_1$]\label{case:cfgs_trans_wf_pres_wait_tgtz_tick_recursion} 
                By~\Cref{def:types_wf} $\exists\Const$ such that $\ValClocks_1\models\Const$ and $\emptyset;\Const~\Entails\mu\alpha.{\TypeS}''_1$, and (by rule \LblTypRec*) $\alpha:\Const;\Const~\Entails{\TypeS}''_1$ and \CIso*[\ValClocks_1];[{\TypeS}''_1]\ is \emph{well-formed}.
                %
                Therefore, the well-formedness of \CIso*[\ValClocks_1]+{+\ValTime};[\mu\alpha.{\TypeS}''_1]\ is dependant on the well-formedness of \CIso*[\ValClocks_1]+{+\ValTime};[{\TypeS}''_1]. (See other cases, as in~\Cref{itm:wf_then_live_recdef} of~\Cref{lem:cfg_wf_then_live}.)
                %
                %
                % ~ wait -> tick -> end
                \item\NewCase[$\TypeS_1=\TypeEnd$]\label{case:cfgs_trans_wf_pres_wait_tgtz_tick_end} 
                By~\Cref{lem:cfg_wf_end} \CIso*+{+\ValTime};[\TypeEnd]\ is \emph{well-formed}.
                %
                %
                % ~ wait -> tick -> rec call
                \item\label{case:cfgs_trans_wf_pres_wait_tgtz_tick_reccal} ${\TypeS}_1$ cannot equal $\alpha$ by~\Cref{lem:cfg_wf_neq_alpha}.
                %
            \end{inductivecase}
            %
            %
            % ~ wait -> unfold
            \item\NewCase[\LblCfgIsoUnfold*]\label{case:cfgs_trans_wf_pres_wait_tgtz_unfold} 
            Then $\TypeS_1=\mu\alpha.{\TypeS}''_1$ and by the hypothesis $\exists\Const$ such that $\ValClocks_1\models\Const$ and $\emptyset;\Const~\Entails\TypRecDef$, and by rule \LblTypRec*\ $\alpha:\Const;\Const~\Entails{\TypeS}''_1$.
            The transition is as shown below:
            %
            \begin{minieq}*%\label{eq:cfgs_trans_wf_pres_wait_tgtz_unfold_trans}
        \begin{array}[c]{l}
                \infer[\LblCfgIsoUnfold]{%
                    \Trans{\CIso[\ValClocks_1];[\mu\alpha.{\TypeS}''_1]}:{\ProgAction}[\CIso'_1]
                }{%
                    \Trans{\CIso[\ValClocks_1];[{\TypeS}''_1\Subst[\mu\alpha.{\TypeS}''_1][\alpha]]}:{\ValTime}[\CIso'_1]
                }
                \end{array}
            \end{minieq}
            
            \noindent By inner induction on the different cases of ${\TypeS}''_1$:
            \begin{inductivecase}
                %
                % ~ wait -> unfold -> choice
                \item\NewCase[${\TypeS}''_1=\simplechoice$]\label{case:cfgs_trans_wf_pres_wait_tgtz_unfold_choice} 
                Then, by rule \LblCfgIsoTick*:
                \[\Trans{\CIso[\ValClocks_1];[{\simplechoice}\Subst[\mu\alpha.{\simplechoice}][\alpha]]}:{\ValTime}[\CIso[\ValClocks_1]+{+\ValTime};[\simplechoice]]\]
                
                \noindent By the rules \LblTypRec*\ and \LblTypChoice*\ the following holds:
                \[
                \infer[\LblTypRec]{%
                \emptyset;\Const_i~\Entails\mu\alpha.{\FullChoice}
                }{%
                \infer[\LblTypChoice]{%
                    \alpha:\Const_i;\Past_i~\Entails\FullChoice
                }{%
                \dots
                }
                }
                \]
                
                % \noindent It holds that \CIso*[\ValClocks_1];[{\simplechoice}\Subst[\mu\alpha.{\simplechoice}][\alpha]]\ is \emph{well-formed} and \emph{future-enabled}.
                %
                \noindent By induction hypothesis: \[\exists\Const'_i:\ValClocks_1+\ValTime\models\Past[\Const'_i] ~\land~ \emptyset;\Past[\Const'_i]~\Entails\FullChoice\] which is assured by the (persistency) premise of rule \LblCfgSocTime*, as in~\Cref{case:cfgs_trans_wf_pres_wait_tgtz_tick_choice} of~\Cref{lem:cfgs_trans_wf_pres}.
% 
                Therefore, \CIso*[\ValClocks_1]+{+\ValTime};[\mu\alpha.{\simplechoice}]\ is \emph{well-formed} as \CIso*[\ValClocks_1]+{+\ValTime};[{\simplechoice}\Subst[\mu\alpha.{\simplechoice}][\alpha]]\ the following is \emph{well-formed}.
                %
                %
                % ~ wait -> unfold -> recursion
                \item\NewCase[${\TypeS}''_1=\mu\alpha'.{\TypeS}'''_1$]\label{case:cfgs_trans_wf_pres_wait_tgtz_unfold_recursion} 
                % By the hypothesis $\exists\Const$ such that $\ValClocks_1\models\Const$ and $\emptyset;\Const~\Entails\mu\alpha.\mu\alpha'.{\TypeS}'''_1$, by the premise of rule \LblTypRec*\ $\alpha:\Const;\Const~\Entails\mu\alpha'.{\TypeS}'''_1$ and $\alpha:\Const,\alpha':\Const;\Const~\Entails{\TypeS}'''_1$, and \CIso*[\ValClocks_1];[{\TypeS}'''_1]\ is \emph{well-formed}.
                % 
                The well-formedness of \CIso*[\ValClocks_1]+{+\ValTime};[\mu\alpha'.{\TypeS}'''_1]\ depends on the well-formedness of \CIso*[\ValClocks_1]+{+\ValTime};[{\TypeS}'''_1]. (See other cases of $S$.) %, as in~\Cref{itm:wf_then_live_recdef} of~\Cref{lem:cfg_wf_then_live}.)
                %
                %
                % ~ wait -> unfold -> end
                \item\NewCase[${\TypeS}''_1=\TypeEnd$]\label{case:cfgs_trans_wf_pres_wait_tgtz_unfold_end} 
                By~\Cref{lem:cfg_wf_end} \CIso*+{+\ValTime};[\TypeEnd]\ is \emph{well-formed}.
                %
                %
                % ~ wait -> unfold -> rec call
                \item\label{case:cfgs_trans_wf_pres_wait_tgtz_unfold_reccall} ${\TypeS}''_1$ cannot equal $\alpha$ by~\Cref{lem:cfg_wf_neq_alpha}.
                %
            \end{inductivecase}
            %
        \end{inductivecase}

        \noindent Therefore, it holds that well-formedness is preserved by transition made by \emph{well-formed} configurations via rule \LblCfgSysWait*.
        %
        %
        %
        %
        %
        % ~ comm
        \item\NewCase[\LblCfgSysLComm*]\label{case:cfgs_trans_wf_pres_comm} 
        % Then by~\cref{case:configs_trans_compat_pres_comm} of~\cref{lem:configs_trans_compat_pres} $\Queue_1=\Queue'_1=\emptyset$.
        The transition is as shown below:
        %
        \begin{minieq}*%\label{eq:cfgs_trans_wf_pres_comm_trans}
        \begin{array}[c]{l}
            \infer[\LblCfgSysLComm]{%
                \Trans{\Parl{\VSoc_1,\VSoc_2}}:{\SiltAction}[\Parl{\VSoc'_1,\VSoc'_2}]
            }{%
                \infer[\LblCfgSocSend]{%
                    \Trans{\VSoc_1}:{\SendMsg}[\VSoc'_1]
                }{\dots}
                & %
                % \infer[\LblCfgSocEnqu]{%
                    \Trans{\VSoc_2}:{\RecvMsg}[\VSoc'_2]
					\quad \LblCfgSocEnqu
                % }{\dots}
            }
            \end{array}
        \end{minieq}
        
        % ~ comm-l s1
        \noindent Focusing first on \VSoc*_1, we proceed by induction on the depth of the derivation tree, analysing the last rule applied for the transition \Trans*{\VIso_1}:{\SendMsg}[\VIso'_1]:
        \begin{inductivecase}
            %
            %
            % ~ comm-l s1 -> act
            \item\NewCase[\LblCfgIsoInteract*]\label{case:cfgs_trans_wf_pres_comm_act} 
            Then $\TypeS_1=\simplechoice$, and the evaluation is shown below:
            %
            \begin{minieq}*\label{eq:cfgs_trans_wf_pres_comm_act_trans}
        \begin{array}[c]{l}
                \infer[\LblCfgSocSend]{%
                    \Trans{\CSoc[\ValClocks_1];[\simplechoice]:{\Queue_1}}:{\SendMsg}[\CSoc[\ValClocks_1]+{\ReSet[]_j};[\TypeS_j]:{\Queue_1}]
                }{%
                    \infer[\LblCfgIsoInteract]{%
                        \Trans{\CIso[\ValClocks_1];[\TypInteract]}:{\SendMsg}[\CIso[\ValClocks_1]+{\ReSet[]_j};[\TypeS_j]]
                    }{%
                        \ValClocks_1\models\Const_j
                        & %
                        {m}={l_j\left\langle T_j \right\rangle}
                        & % 
                        {\TypSend=\TypComm_j}
                        & % 
                        j\in I
                    }
                }
                \end{array}
            \end{minieq}
            
            \noindent By rule \LblTypChoice*\ $\exists\Const_i:\ValClocks_1\models\Past[\Const_i]$ and $\emptyset;\Past[\Const_i]~\Entails\FullChoice$, and by the premise of rule \LblTypChoice*\ it holds that $\Const_i\ReSet[]_i\subseteq\gamma$ and $\emptyset;\gamma~\Entails\TypeS_i$.
            %
            Combined with~\Cref{lem:configs_iso_trans} it holds that $\ValClocks_1\models\Const_j$ and $\emptyset;\Const_j\ReSet[]_j~\Entails\TypeS_j$ and $\ValClocks_1\ReSet[]_j\models\Const_j\ReSet[]_j$.
            
            Therefore, \CIso*[\ValClocks_1]+{\ReSet[]_j};[\TypeS_j]\ is \emph{well-formed}.
            %
            %
            % ~ comm-l s1 -> unfold
            \item\NewCase[\LblCfgIsoUnfold*]\label{case:cfgs_trans_wf_pres_comm_unfold} 
            Then $\TypeS_1=\mu\alpha.{\TypeS}''_1$.
            The transition is as shown below:
            %
            \begin{minieq}*\label{eq:cfgs_trans_wf_pres_comm_unfold_trans}
        \begin{array}[c]{l}
                \infer[\LblCfgIsoUnfold]{%
                    \Trans{\CIso[\ValClocks_1];[\mu\alpha.{\TypeS}''_1]}:{\ProgAction}[\CIso'_1]
                }{%
                    \Trans{\CIso[\ValClocks_1];[{\TypeS}''_1\Subst[\mu\alpha.{\TypeS}''_1][\alpha]]}:{\SendMsg}[\CIso'_1]
                }
                \end{array}
            \end{minieq}
            
            \noindent By the hypothesis $\exists\Const$ such that $\ValClocks_1\models\Const$ and $\emptyset;\Const~\Entails\TypRecDef$, and by the premise of rule \LblTypRec*\ $\alpha:\Const;\Const~\Entails{\TypeS}''_1$, and \CIso*[\ValClocks_1];[{\TypeS}''_1\Subst[\mu\alpha.{\TypeS}''_1][\alpha]]\ is \emph{well-formed}.
            %
            The well-formendess of \CIso*'_1\ is dependant on the state of ${\TypeS}''_1$, which for the transition \Trans*{\CIso[\ValClocks_1];[{\TypeS}''_1\Subst[\mu\alpha.{\TypeS}''_1][\alpha]]}:{\SendMsg}[\CIso'_1]\ must be either $\simplechoice$ or $\mu\alpha'.{\TypeS}'''_1$ (see other cases, as in~\Cref{lem:cfg_wf_then_live}).
            %
        \end{inductivecase}
        %
        % ~ comm-l s1
        Now, focusing on \VSoc*_2, the transition \Trans*{\CSoc_2}:{\RecvMsg}[\CSoc'_2]\ via rule \LblCfgSocEnqu*\ yields $\ValClocks_2'=\ValClocks_2$ and $\TypeS_2'=\TypeS_2$ and $\Queue_2'=\Queue_2;\Msg$ by~\Cref{lem:configs_soc_trans}.
        %
        Therefore \CIso*'_2\ is \emph{well-formed} as $\VIso_2=\VIso'_2$.
        %
        Transitions via rule \LblCfgSysRComm*\ are symmetric and omitted.
        %
        %
        %
        %
        %
        % ~ par-l
        \item\NewCase[\LblCfgSysLPar*]\label{case:cfgs_trans_wf_pres_par} 
        By~\Cref{lem:configs_soc_trans} $\Queue'_1=\Msg;\Queue_1$.
        The transition is as shown below: %in~\Cref{eq:cfgs_trans_wf_pres_par_trans}.
        %
        \begin{minieq}*\label{eq:cfgs_trans_wf_pres_par_trans}
        \begin{array}[c]{l}
            \infer[\LblCfgSysLPar]{%
                \Trans{\Parl{\VSoc_1,\VSoc_2}}:{\SiltAction}[\Parl{\VSoc'_1,\VSoc'_2}]
            }{%
                \infer[\LblCfgSocRecv]{%
                    \Trans{\VSoc_1}:{\SiltAction}[\VSoc'_1]
                }{\dots}
            }
            \end{array}
        \end{minieq}
        
        \noindent We proceed by induction on the depth of the derivation tree, analysing the last rule applied for the transition \Trans*{\VIso_1}:{\RecvMsg}[\VIso'_1]\ via the premise of rule \LblCfgSocRecv*:
        \begin{inductivecase}
            %
            %
            % ~ par-l -> act
            \item\NewCase[\LblCfgIsoInteract*]\label{case:cfgs_trans_wf_pres_par_act} 
            Then $\TypeS_1=\simplechoice$.
            The transition is as shown below: %in~\Cref{eq:cfgs_trans_wf_pres_par_act_trans}.
            %
            \begin{minieq}*\label{eq:cfgs_trans_wf_pres_par_act_trans}
        \begin{array}[c]{l}
                \infer[\LblCfgSocRecv]{%
                    \Trans{\CSoc[\ValClocks_1];[\simplechoice]:{\Msg;\Queue_1}}:{\SiltAction}[\CSoc[\ValClocks_1]+{\ReSet[]_j};[\TypeS_j]:{\Queue_1}]
                }{%
                    \infer[\LblCfgIsoInteract]{%
                        \Trans{\CIso[\ValClocks_1];[\TypInteract]}:{\RecvMsg}[\CIso[\ValClocks_1]+{\ReSet[]_j};[\TypeS_j]]
                    }{%
                        \ValClocks_1\models\Const_j
                        & %
                        {m}={l_j\left\langle T_j \right\rangle}
                        & % 
                        {\TypRecv=\TypComm_j}
                        & % 
                        j\in I
                    }
                }
                \end{array}
            \end{minieq}
            
            \noindent By the hypothesis and the judgement of rule \LblTypChoice*\ $\exists\Const_i$ such that $\emptyset;\Past[\Const_i]~\Entails\simplechoice$ and $\ValClocks_1\models\Past[\Const_i]$, and by the premise of rule \LblTypChoice*\ $\Const_i\ReSet[]_i\subseteq\gamma$ and $\emptyset;\gamma~\Entails\TypeS_i$.
            
            It follows~\Cref{case:cfgs_trans_wf_pres_comm_act} of~\Cref{lem:cfgs_trans_wf_pres} that \CIso*[\ValClocks_1]+{\ReSet[]_j};[\TypeS_j]\ is \emph{well-formed}.
            % Therefore \CIso*[\ValClocks_1]+{\ReSet[]_j};[\TypeS_j]\ is \emph{well-formed} as $\ValClocks_1\models\Const_j$ and $\emptyset;\Const_j\ReSet[]_j~\Entails\TypeS_j$ and $\ValClocks_1\ReSet[]_j\models\Const_j\ReSet[]_j$ (as in~\Cref{case:cfgs_trans_wf_pres_comm_act} of~\Cref{lem:cfgs_trans_wf_pres}).
            %
            %
            % ~ par-l -> unfold
            \item\NewCase[\LblCfgIsoUnfold*]\label{case:cfgs_trans_wf_pres_par_unfold} 
            Then $\TypeS_1=\mu\alpha.{\TypeS}''_1$.
            The transition shown below, and is analogous to the one in~\Cref{eq:cfgs_trans_wf_pres_comm_unfold_trans} of~\Cref{lem:cfgs_trans_wf_pres}:
            % in~\Cref{eq:cfgs_trans_wf_pres_par_unfold_trans} and is analogous to the one in~\Cref{eq:cfgs_trans_wf_pres_comm_unfold_trans}.
            %
            \begin{minieq}\label{eq:cfgs_trans_wf_pres_par_unfold_trans}
        \begin{array}[c]{l}
                \infer[\LblCfgIsoUnfold]{%
                    \Trans{\CIso[\ValClocks_1];[\mu\alpha.{\TypeS}''_1]}:{\ProgAction}[\CIso'_1]
                }{%
                    \Trans{\CIso[\ValClocks_1];[{\TypeS}''_1\Subst[\mu\alpha.{\TypeS}''_1][\alpha]]}:{\RecvMsg}[\CIso'_1]
                }
                \end{array}
            \end{minieq}
            
            \noindent By the hypothesis $\exists\Const$ such that $\ValClocks_1\models\Const$ and $\emptyset;\Const~\Entails\TypRecDef$, and by the premise of rule \LblTypRec*\ $\alpha:\Const;\Const~\Entails{\TypeS}''_1$, and \CIso*[\ValClocks_1];[{\TypeS}''_1\Subst[\mu\alpha.{\TypeS}''_1][\alpha]]\ is \emph{well-formed}.
            %
            The well-formendess of \CIso*'_1\ is dependant on ${\TypeS}''_1$, which for the transition by the premise of rule \LblCfgIsoUnfold*\ must be either $\simplechoice$ or $\mu\alpha'.{\TypeS}'''_1$ (see other cases, as in~\Cref{lem:cfg_wf_then_live}).
            %
        \end{inductivecase}
        %
    \end{inductivecase}

    \noindent Therefore, it holds that any transition made by a system composed of compatible and \emph{well-formed} configurations will result in configurations that are \emph{well-formed}.
    %
\end{proof}
% 

% ~ time passing
% ! 
% \newpage
%
% ! (lemma 13) : time passing implies empty queues
\begin{lemma}\label{lem:sys_compat_time_trans}
   %
   If \Compat*[\VSoc_1][\VSoc_2]\ and \Trans*{\Parl{\VSoc_1,\VSoc_2}}:{\ValTime}\ and $\ValTime>0$ then $\Queue_1=\emptyset=\Queue_2$.
   %
\end{lemma}
\begin{proof}
   %
   Such a transition is only specified by \LblCfgSysWait*, which by its premise requires a \LblCfgSocTime*\ transition of \ValTime*\ for each \VSoc*_1\ and \VSoc*_2.
   %
   By contradiction, if one queue were \emph{non-empty}, say $\Queue_1=\Msg;\Queue_1$, then by~\Cref{itm:configs_compat_expected_receive} of~\Cref{def:configs_compat} message \Msg*\ must be able to be received immediately.
   %
   The premise of \LblCfgSocTime*\ (urgency) ensures that \ValTime*\ must be valued such that no time passes while a message is able to be received.
   %
   % It holds that $t$ must equal 0 when there is a non-empty queue.
   
   Therefore the hypothesis holds.
   %if a system makes a $t$ transition where $t>0$ then all queues in the system must be empty.
   % Therefore, \ValTime*\ must equal $0$ when there is a message in any queue in a system composed of compatible configurations.
   %
 \end{proof}
 % 

%
% ! (lemma 15) : compat, single transition -> compat
\begin{lemma}\label{lem:configs_trans_compat_pres}
	%
	If \VIso*_1\ and \VIso*_2\ are both \emph{well-formed} 
    and \Compat*[\VSoc_1][\VSoc_2]\ 
    and \Trans*{\Parl{\VSoc_1,\VSoc_2}}[\Parl{\VSoc'_1,\VSoc'_2}], 
	then \Compat*[\VSoc'_1][\VSoc'_2].
	%
\end{lemma}
\begin{proof}
	%
	We proceed by induction on the depth of the derivation tree, analysing each case of the last rule applied for the transition \Trans*{\Parl{\VSoc_1,\VSoc_2}}[\Parl{\VSoc'_1,\VSoc'_2}]:
	% \begin{inline}+
	% 	\item \LblCfgSysWait*
	% 	\item \LblCfgSysLComm*
	% 	\item \LblCfgSysLPar*
	% \end{inline}
	%
	\begin{inductivecase}
		%
		%
		%
		%
		%
		% ~ wait
		\item\NewCase[\LblCfgSysWait*]\label{case:configs_trans_compat_pres_wait}
		Then both \VSoc*_1\ and \VSoc*_2\ make a \ValTime*\ transition via \LblCfgSocTime*\ as shown in~\Cref{eq:configs_trans_compat_pres_wait_trans}.
		%
        If $\ValTime=0$ then by~\Cref{lem:configs_iso_trans,lem:configs_soc_trans} $\VSoc_1=\VSoc'_1$ and $\VSoc_2=\VSoc'_2$ and the hypothesis holds; \Compat*[\VSoc'_1][\VSoc'_2].
        %
        \begin{minieq}\label{eq:configs_trans_compat_pres_wait_trans}
            \infer[\LblCfgSysWait]{%
                \Trans{\Parl{\VSoc_1,\VSoc_2}}:{\ValTime}[\Parl{\VSoc'_1,\VSoc'_2}]
            }{%
                \infer[\LblCfgSocTime]{%
                    \Trans{\VSoc_1}:{\ValTime}[\VSoc'_1]
                }{\dots}
                & %
                \infer[\LblCfgSocTime]{%
                    \Trans{\VSoc_2}:{\ValTime}[\VSoc'_2]
                }{\dots}
            }
        \end{minieq}
        
        \noindent If $\ValTime>0$ then by~\Cref{lem:configs_iso_trans,lem:configs_soc_trans} $\ValClocks'_1=\ValClocks_1+\ValTime$ and $\TypeS'_1=\TypeS_1$ and $\Queue'_1=\Queue_1$ (and the same for \ValClocks*'_2, \TypeS*'_2\ and \Queue*'_2).
		%
        By~\Cref{lem:cfgs_trans_wf_pres} \VIso*'_1\ and \VIso*'_2\ are both \emph{well-formed}.
		%
		By~\Cref{lem:sys_compat_time_trans} $\Queue_1=\emptyset=\Queue_2$ and by~\Cref{itm:configs_compat_dual_types} of~\Cref{def:configs_compat} $\ValClocks_1=\ValClocks_2$ and $\TypeS_1=\Dual[\TypeS_2]$.
		%
		Therefore \Compat*[\CSoc[\ValClocks_1]+{+\ValTime};[\TypeS_1]:{\emptyset}][\CSoc[\ValClocks_2]+{+\ValTime};[\Dual[\TypeS_2]]:{\emptyset}].
		%
		%
		%
		%
		%
		% ~ comm
		\item\NewCase[\LblCfgSysLComm*]\label{case:configs_trans_compat_pres_comm}
		By~\cref{lem:cfgs_trans_wf_pres} both \VIso*'_1\ and \VIso*'_2\ are \emph{well-formed}.
        The transition is as shown below: %in~\cref{eq:configs_trans_compat_pres_comm_trans}.
		%
		\begin{minieq}*\label{eq:configs_trans_compat_pres_comm_trans}
			% \begin{array}{c}%\mathllap{%
			\resizebox{\linewidth}{!}{$%
				\infer[\LblCfgSysLComm]{%
					\Trans{\Parl{\CSoc[\ValClocks_1];[\TypeS_1]:{\emptyset},\CSoc[\ValClocks_2];[\TypeS_2]:{\Queue_2}}}:{\SiltAction}[\Parl{\CSoc[\ValClocks'_1];[\TypeS'_1]:{\emptyset},\CSoc[\ValClocks_2];[\TypeS_2]:{\Queue_2;\Msg}}]
				}{%
					\infer[\LblCfgSocSend]{%
						\Trans{\CSoc[\ValClocks_1];[\TypeS_1]:{\emptyset}}:{\SendMsg}[\CSoc[\ValClocks_1]+{\ReSet[]_j};[\TypeS_j]:{\emptyset}]
					}{%
						\infer[\LblCfgIsoInteract]{%
							\Trans{\CIso[\ValClocks_1];[\TypInteract]}:{\SendMsg}[\CIso[\ValClocks_1]+{\ReSet[]_j};[\TypeS_j]]
						}{%
                        \ValClocks_1\models\Const_j
                        & %
                        {m}={l_j\left\langle T_j \right\rangle}
                        & % 
                        {\TypSend=\TypComm_j}
                        & % 
                        j\in I
						}
					}
					& %
					% \infer[\LblCfgSocEnqu]{%
						\Trans{\VSoc_2}:{\RecvMsg}[\CSoc[\ValClocks_2];[\TypeS_2]:{\Queue_2;\Msg}]
						\quad \LblCfgSocEnqu
					% }{\dots}
				}
			$}%
	%	}\end{array}
		\end{minieq}
		
        \noindent We proceed by inner induction on each combination of the contents of queues:
		\begin{inductivecase}
			%
			%
			% ~ comm -> e e
			\item\NewCase[$\Queue_1=\emptyset$, $\Queue_2=\emptyset$]\label{case:configs_trans_compat_pres_comm_ee}
			By~\Cref{itm:configs_compat_dual_types} of~\Cref{def:configs_compat} $\ValClocks_1=\ValClocks_2$ and $\TypeS_1=\Dual[\TypeS_2]$.
			%
			The resulting system is no longer \emph{dual}.
			%
			By~\cref{lem:cfgs_trans_wf_pres,lem:sys_compat_time_trans} time cannot pass if $\Queue_2\neq\emptyset$.
			%
			By~\Cref{def:types_dual} the message \Msg*\ sent by \VSoc*_1\ must have a corresponding receiving action in \VSoc*_2\ as in~\Cref{itm:configs_compat_expected_receive} of~\Cref{def:configs_compat}.
			%
			Therefore \Compat*[\CSoc[\ValClocks'_1];[{\TypeS}'_1]:{\emptyset}][\CSoc;+{\Msg}_2].
			%
			%
			%
			% ~ comm -> e m
			\item\NewCase[$\Queue_1=\emptyset$, $\Queue_2\neq\emptyset$]\label{case:configs_trans_compat_pres_comm_en}
			By~\Cref{itm:configs_compat_expected_receive} of~\Cref{def:configs_compat} $\exists\Msg',{\ValClocks}'',{\TypeS}''$ such that $\Queue_2=\Msg';\Queue_2$ and \Trans*{\CIso_2}:{\TypRecv,\Msg'}[\CIso[{\ValClocks}''_2];[{\TypeS}''_2]]\ and \Compat*[\CSoc[\ValClocks'_1];[{\TypeS}'_1]:{\emptyset}][\CSoc[{\ValClocks}''_2];[{\TypeS}''_2]:{{\Queue}_2}]\ and by \LblCfgSocTime*\ (urgency) time cannot pass.
			%
			If a system has a configuration with sequence of outgoing sending actions and each has constraints that are satisfiable immediately after the other, then the system can both receive the messages as they arrive, or accumulate the messages and instantly receive each in succession and become \emph{dual} again (by inspection of~\Cref{def:types_dual,def:configs_compat} and~\Cref{fig:types_rule,fig:typesemantics_tuple,fig:typesemantics_triple}).
			%
			Therefore \Compat*[\CSoc[\ValClocks'_1];[{\TypeS}'_1]:{\emptyset}][\CSoc[{\ValClocks}''_2];[{\TypeS}''_2]:{\Msg';{\Queue}_2;\Msg}].
			%
			%
			% ~ comm -> m e
			\item\NewCase[$\Queue_1\neq\emptyset$, $\Queue_2=\emptyset$]\label{case:configs_trans_compat_pres_comm_ne}
			Contradicts the hypothesis by~\Cref{itm:configs_compat_expected_receive} of~\Cref{def:configs_compat} as by \LblCfgSocTime*\ (urgency) messages must be removed from a queue immediately, and by \LblTypChoice*\ of~\Cref{fig:types_rule} sending and receiving actions cannot be performed at the same time.
			%
			%
			% ~ comm -> m m
			\item\NewCase[$\Queue_1\neq\emptyset$, $\Queue_2\neq\emptyset$]\label{case:configs_trans_compat_pres_comm_nn}
			Contradicts the hypothesis by~\Cref{itm:configs_compat_non_empty_queues} of~\Cref{def:configs_compat}.
			%
		\end{inductivecase}

		\noindent Therefore, compatibility is preserved across \LblCfgSysLComm*\ transitions.
		%
		%
		%
		%
		%
		% ~ dequ
		\item\NewCase[\LblCfgSysLPar*]\label{case:configs_trans_compat_pres_dequ}
		By~\Cref{lem:configs_soc_trans} $\Queue_2=\Msg;\Queue_2$ and by~\Cref{itm:configs_compat_expected_receive} of~\Cref{def:configs_compat} \Compat*[\VSoc'_1][\VSoc'_2], the hypothesis holds.
		%
	\end{inductivecase}

	\noindent Therefore, it holds that any transition made by a compatible system composed of well-formed types will result in configurations that are \emph{compatible}.
	%
\end{proof}
% 

% ~ preservation is preserved 
% ! 
%
% ! (lemma 16) : compat wf, end or future enabled
\begin{lemma}\label{lem:configs_compat_wf_fe}
	%
	If both \VIso*_1\ and \VIso*_2\ are \emph{well-formed} 
    and \Compat*[\VSoc_1][\VSoc_2],
	then both \VSoc*_1\ and \VSoc*_2\ are \emph{final} 
	or $\exists\ValTime$ such that \Trans*{\VSys}:{\ValTime,\SiltAction}[\Parl{\VSoc'_1,\VSoc'_2}].%
	%
\end{lemma}
\begin{proof}
	%
	By~\Cref{lem:cfgs_trans_wf_pres} \VIso*'_1\ and \VIso*'_2\ are \emph{well-formed} and by~\Cref{lem:cfg_wf_then_live} are \emph{live}.
	%
	By~\Cref{def:types_progress} if \VSoc*_2\ is \emph{final} then $\VSoc_2=\CSoc[\ValClocks_2];[\TypeEnd]:{\emptyset}$.
    %
    We proceed with the assumption that at least one participant is \emph{not final}, and hereafter only consider \VSoc*'_1.
	%
	The transition is given below:
 %in~\Cref{eq:configs_compat_wf_fe_trans}.
	%
	\begin{minieq}*\label{eq:configs_compat_wf_fe_trans}
		\Trans{\VSys}:{\ValTime}[\Trans{\Parl{\CSoc+{+\ValTime}_1,\CSoc+{+\ValTime}_2}}:{\SiltAction}[\Parl{\CSoc'_1,\CSoc'_2}]]
	\end{minieq}
	%
	We proceed only considering each case of \VSoc*_1\ not being \emph{final}.
	By induction on the cases of \TypeS*_1:
	%
	\begin{inductivecase}
		%
		%
		% ~ choice
		\item\NewCase[$\TypeS=\simplechoice$] 
		% Then by judgement of \LblTypChoice*\ $\exists\Const_i$ such that $\ValClocks\models\Past[\Const_i]$ and $\emptyset;\Past[\Const_i]~\Entails\simplechoice$.
		%
		As described in~\Cref{sec:types} we write $\Past$ if $\exists\ValTime$ such that $\ValClocks+\ValTime\models\Const$.
		Therefore, if $\ValClocks\not\models\Const$ and $\emptyset;\Const~\Entails\TypeS$ for a \emph{well-formed} \CIso*\ then $\ValClocks\models\Past[\Const]$.
		%
		By~\Cref{lem:cfgs_trans_wf_pres} the only possible values of \ValTime*\ ensure that the latest system-wide sending action is never missed and messages are received as soon as they arrive in a queue by rule \LblCfgSocTime*.
		
		Therefore, the hypothesis holds for systems composed where one participant is known to be a \emph{non-final} choice type.
		%
		%
		% ~ rec def
		\item\NewCase[$\TypeS=\mu\alpha.\TypeS'$]
		It follows~\Cref{lem:cfgs_trans_wf_pres} that \TypeS*'\ is \emph{well-formed} against $\ValClocks+\ValTime$.
		%
	\end{inductivecase}
	
	\noindent Therefore, if a \emph{well-formed} and compatible system \Parl*{\VSoc_1,\VSoc_2}\ that is not \emph{final}, then there is some value of time $\ValTime\geq 0$ that will enable a future action, which will result in a \emph{well-formed} and compatible system \Parl*{\VSoc'_1,\VSoc'_2}, which may or may not be \emph{final}, and to which this behaviour still applies.
	%
\end{proof}
% 
%
% ! (lemma 18) : compat wf, any amount of transitions -> compat wf
\begin{lemma}\label{lem:configs_trans_compat_wf_pres}
    %
    If both \VIso*_1 and \VIso*_2 are \emph{well-formed}
    and \Compat*[\VSoc_1][\VSoc_2]\ 
    and $\Trans{\Parl{\VSoc_1\!,\VSoc_2}}*[\Parl{\VSoc'_1\!,\VSoc'_2}]$, 
    then \Compat*[\VSoc'_1][\VSoc'_2]\ 
    and both \VIso*'_1\ and \VIso*'_2\ are \emph{well-formed}.%
    %
\end{lemma}
\begin{proof}
    %
    By~\Cref{lem:cfgs_trans_wf_pres,lem:configs_trans_compat_pres} the hypothesis holds for single transitions and that the resulting configurations are \emph{live}, and either \emph{final} or \emph{satisfies progress} by~\Cref{lem:cfg_wf_then_live,lem:configs_compat_wf_fe}.
    %
    Therefore it holds that \Compat*[\VSoc'_1][\VSoc'_2]\ 
    and both \VIso*'_1\ and \VIso*'_2\ are \emph{well-formed} across an arbitrary number of transitions, as each single transition preserves compatibility and well-formedness.
    %
\end{proof}
% 

% ~ system progress
% ! 
%
% ! (lemma 20) : sys cfg progress
\begin{lemma}\label{lem:configs_sys_progress}
    %
    For all \TypeS*, \ValClocks* such that \ITJudgement*[\emptyset];{\Const}[\TypeS]\ and \Sat*\ :\par\noindent
    \hfill\ \Parl*{\CSoc:{\emptyset},\CSoc;[\Dual]:{\emptyset}}\ satisfies progress.\hfill\ \ %
    %
\end{lemma}
\begin{proof}
    %
    By the hypothesis the system is compatible and composed of well-formed dual types. 
    %
    By~\Cref{lem:configs_trans_compat_wf_pres} any configurations reachable by such a system will be compatible and \emph{well-formed}.
    %
    By~\Cref{lem:configs_compat_wf_fe} such a system adheres to~\Cref{def:types_progress} and \emph{satisfies progress}.
    %
\end{proof}
% 

% ! thm proof
\ThmProgress*
\begin{proof}
   %
   By~\Cref{def:types_wf}, types \TypeS*\ and \Dual*\ are always \emph{well-formed} against \ValClocks*_0.
   %
   By \Cref{lem:init_wf_then_live} and~\Cref{itm:configs_compat_dual_types} of~\Cref{def:configs_compat}, both \CSoc*[\ValClocks_0]:{\emptyset}\ and \CSoc*[\ValClocks_0];[\Dual]:{\emptyset} are \emph{live} and \emph{compatible}.
   %
   It follows~\Cref{lem:configs_compat_wf_fe,lem:configs_trans_compat_wf_pres} that such as system will always perform actions when possible, waiting if necessary (never missing the \emph{latest-enabled} action), until reaching a \emph{final} configuration, and any \emph{non-final} configuration is guaranteed to be \emph{well-formed}, \emph{compatible} and \emph{live}.

   Therefore, it holds that an initial system composed of dual types that are well-formed is compatible, and guaranteed to \emph{satisfy progress}.
   %
\end{proof}


%
% ~
\begin{restatable}[Subject Reduction of Processes]{theorem}{ThmSubjectReduction}\label{thm:subject_reduction}
    %
    Let ${(\RedTimers,~P)}$ be \emph{live}.
    If $\emptyset\;\Entails\,\Prc\;\PrcTyped\,\emptyset$ and ${(\RedTimers,~P)}~\longrightarrow~{(\RedTimers',~P')}$, then $\emptyset\;\Entails\,\Prc'\;\PrcTyped\,\emptyset$.
    %
\end{restatable}
%

\endinput
System configurations are composed of two distinct social configurations, \SocCfg*_1\ and \SocCfg*_2, running in parallel: \CfgSys*. The rules given in~\cref{fig:configs_sys_rule} reveal the following high-level control flow of a system
\begin{inline}>
	\item \dots
	\item \LblCfgSysWait*
	\item \LblCfgSysLComm*
	\item \LblCfgSysLPar*
	\item \dots
\end{inline}\etc\ .
Describing there being some time to wait \ValTime*\ until a communication between the two parties can occur, which results in the message being received by a queue; at which point the message can be retrieved from the queue and received by the configuration. This cycle continues indefinitely until either both types terminate or a deadlock is reached.


% ~ comm
\NewDocumentCommand{\CfgSysRuleComm}{s}{%
\IfBooleanT{#1}{$}% mm wrap
\infer[\LblCfgSysLComm]{%[{\RName[com\text{-}l]}]{% * judgement
\Trans{\Parl{\VSoc_1,\VSoc_2}}:{\SiltAction}[\Parl{\VSoc'_1,\VSoc'_2}]
}{% * premise
\Trans{\VSoc_1}:{\TypSend\Msg}[\VSoc'_1]
%
&
%
\Trans{\VSoc_2}:{\TypRecv\Msg}[\VSoc'_2]
}
\IfBooleanT{#1}{$}% mm wrap
}


% ~ dequ
\NewDocumentCommand{\CfgSysRuleDequ}{s}{%
\IfBooleanT{#1}{$}% mm wrap
\infer[\LblCfgSysLPar]{%[{\RName[par\text{-}l]}]{% * judgement
\Trans{\Parl{\VSoc_1,\VSoc_2}}:{\SiltAction}[\Parl{\VSoc'_1,\VSoc_2}]
}{% * premise
\Trans{\VSoc_1}:{\SiltAction}[\VSoc'_1]
}
\IfBooleanT{#1}{$}% mm wrap
}


% ~ wait
\NewDocumentCommand{\CfgSysRuleWait}{s}{%
\IfBooleanT{#1}{$}% mm wrap
\infer[\LblCfgSysWait]{%[{\RName[wait]}]{% * judgement
\Trans{\Parl{\VSoc_1,\VSoc_2}}:{\ValTime}[\Parl{\VSoc'_1,\VSoc'_2}]
}{% * premise
\Trans{\VSoc_1}:{\ValTime}[\VSoc'_1]
%
&
%
\Trans{\VSoc_2}:{\ValTime}[\VSoc'_2]
}
\IfBooleanT{#1}{$}% mm wrap
}

\endinput

% ~ template
\NewDocumentCommand{\CfgSysRule}{s}{%
    \IfBooleanT{#1}{$}% mm wrap
    \infer{%[{\RName[temp]}]{% * judgement
        \SocCfg_1\Par\SocCfg_2
        \Act[\SiltAction]
        \SocCfg'_1\Par\SocCfg'_2
    }{% * premise
        \CfgSoc'''
    }
    \IfBooleanT{#1}{$}% mm wrap
}

Rule \LblCfgSysLComm*\ specifies that both configurations can make transitions to new configurations when one performs a transition via \LblCfgSocSend*\ and the other by \LblCfgSocEnqu*; for example:
\begin{minieq}
	\Par[\CfgSoc_1][\CfgSoc_2]
	\;\Act[\SiltAction]\,
	\Par[\CfgSoc_{1}''][\CfgSoc1{\ValClocks_2}2{\TypeS_2}3{\Queue_2;\Msg}]
\end{minieq}
While rule \LblCfgSysLPar*\ describes one party successfully receiving a message from their queue by a transition via \LblCfgSocRecv*.

Rule \LblCfgSysWait*\ allows time to pass evenly over the entire system, requiring both participants to make a transition via \LblCfgSocTime*\ of the same value of \ValTime*. The amount of time \ValTime*\ is restricted to the greatest value shared by both parties by the premise of their respective \LblCfgSocTime*\ transitions. This ensures that time passes evenly for separate participants, effectively halting time for the whole system once a participant has no choice but to perform an action.

\subsubsection{Ensuring System Compatibility}\label{sssec:configs_sys_compatibility}
Given a system composed of configurations with dual well-formed types, there is the potential for the system to exhibit \emph{communication safety} and deadlock freedom.
However, this is not guaranteed just by types being well-formed, but must be ensured by their defined behaviour.

Naturally, the rules of~\cref{fig:configs_iso_rule,fig:configs_soc_rule,fig:configs_sys_rule} specify that messages must be sent my one party before being received by another.
Furthermore, sending actions exhibit a \emph{relaxed} notion of urgency, by the premise of \LblCfgSocTime*, which specifies that a value of time may pass, delaying the sending of messages in order to accumulate as many viable actions as possible, so long as no opportunities for sending a message are missed.
Given that a receiving party is composed of dual types, a message that is delayed will still be successfully received.

Conversely, receiving actions \emph{must} exhibit urgency, as a message existing in a queue indicates that the sender has already progressed to some later type. Consider the trace of a type in~\cref{eq:example_configs_sys_compatability}.
\begin{minieq}\label{eq:example_configs_sys_compatability}
	\TypAct1{\TypSend}2{\mathtt{e}}
	\TypCond1{\Clx<3}2{\Clx}
	.
	\TypAct1{\TypRecv}2{\mathtt{f}}
	\TypCond1{\Clx=0}2{\emptyset}
	\dots
\end{minieq}
In the instance where the sending of \texttt{e} is delayed until the last possible moment $\Clx=2.9$, any delay of its dual counterpart would result in \texttt{f} being send too late. \TODO[note cite?]
\begin{note}
	In this work latency over a network is not considered, but may be accounted for in implementations.
\end{note}

For this reason, \LblCfgSocTime*\ does not allow time to pass for a configuration if there is a message waiting to be received. Additionally, receive urgency ensures that the clocks of each participant are kept synchronised as they progress.
This notion of compatibility is formally defined in~\cref{def:configs_compat}; an explanation of each condition is as follows
\begin{inline}*+;
	\item only one participant may receive messages at any time
	\item compatibility is preserved as messages are received
	\item both initial configurations of a system and systems in waiting are compatible, with both having matching clock valuations and dual types
\end{inline}
\begin{definition}[Compatible Systems]\label{def:configs_compat}
    %
    Let $\VSoc_1 = \CSoc_1$ and $\VSoc_2 = \CSoc_2$. 
    System \VSys*\ is \emph{compatible} (written $\VSoc_1\bot\, \VSoc_2$) if:
    \begin{enumerate}
    \item\label{itm:configs_compat_non_empty_queues} $\Queue_1=\emptyset%
            ~\lor~%
            \Queue_2=\emptyset$%
        %
        \\
       \item\label{itm:configs_compat_dual_types} $\Queue_1=\Queue_2=\emptyset%
            ~\implies~%
            \ValClocks_1=\ValClocks_2%
            ~\land~%
            \TypeS_1=\Dual_2$
        %
        \\
  \item
    \label{itm:configs_compat_expected_receive} 
    $\Queue_1=\Msg;\Queue'_1
            ~\Implies~%
            \exists\ValClocks'_1,\TypeS'_1:
            \Trans{\CIso_1}:{\RecvMsg}[\CIso'_1]%
            ~\land~%
            \CSoc'_1 \bot\, \VSoc_2$
         %   \Compat[\VSoc_1'][\VSoc_2]$%
            \\
\item
    %\label{itm:configs_compat_expected_receive} 
    $\Queue_2=\Msg;\Queue'_2
            ~\Implies~%
            \exists\ValClocks'_2,\TypeS'_2:
            \Trans{\CIso_2}:{\RecvMsg}[\CIso'_2]%
            ~\land~%
            \VSoc_1 \bot\, \CSoc'_2$
         %   \Compat[\VSoc_1'][\VSoc_2]$%
            \\
        % $\forall i, j\in\mkSet[1,2]:%
        % i\neq j%
        % \quad%
        % \Queue_i=\Msg;\Queue_i'%
        % ~\Implies~%
        % \exists\ValClocks'_i,\TypeS'_i:%
        % \Trans{\CIso_i}:{\RecvMsg}[\CIso'_i]%
        % ~\land~%
        % \Compat[\VSoc'_i][\VSoc_j]$%
        %
    \end{enumerate}
    %
    % \noindent We write \Compat*\ if system \VSys*\ is compatible.
    % \Cref{itm:configs_compat_expected_receive} is symmetric.
\end{definition}
In practice conditions 2 and 3 are of interest, as it reflects the control flow of a system described at the beginning of~\cref{ssec:configs_sys} where a system has either
\begin{inline}|;
	\item performed a \LblCfgSysLComm*\ leading to a subsequent \LblCfgSysLPar*, following condition 2
	\item no interactions are possible, meaning only time may pass via \LblCfgSysWait*, following condition 3
\end{inline}
Additionally, condition 2 of compatibility supports situations where one participant sends several messages in succession without the other participant receiving them. This is valid behaviour.


%\section{Security Challenges of Sustainable Systems}

\section{Why is Sustainability a Security Problem?}
\label{sec:security-challenges}

Ensuring the accuracy and credibility of sustainability
metrics, as well as supporting audits, require guaranteeing the
trustworthiness
and comprehensiveness of not only the carbon footprints of data center
equipment but also the embodied energy throughout the entire lifecycle
of computing equipment.
%
Although some  external information---such as that for
renewable energy, energy credits, or supplied water---can be
authenticated via trusted third parties~\cite{co2e_epa, iea},
sustainability metrics in data centers require the
authenticity, confidentiality, integrity, and availability of data
collected, processed, stored, and used locally within a
data center~\cite{gandhi2022metrics}.
%
However, unlike traditional cloud computing systems
where the focus is primarily on security and privacy of user applications
and data~\cite{carlin2013cloud, zissis2012addressing, chen2010s},
collecting and measuring data center activities that impact humans and
the environment in a verifiable and privacy-preserving manner
presents a diverse set of new security challenges.
%
Most of these challenges are primarily based on sustainability data,
reliability of equipment, and cleanliness of energy sources---across
both the digital and physical worlds.
%
Unfortunately, no prior research has investigated the threat landscape of
sustainable data centers, nor attempted to provide any techniques or tools
that directly allow authentication of operational sustainability
metrics induced within a data
center to preserve the privacy of users' or operators'
sustainability data.
It is thus imperative to ensure the security of (i) data collection
processes, (ii) the process of generating verifiable, easily auditable
sustainability metrics, and (iii) the storage of all pertinent
information.
Hence, while being indispensable for protecting the environment and our
planet, we have found and argue that the current sustainability
practices---through self-reporting, best-effort measurement, and
anything less than complete verifiable control of
sustainability---will fail.

%%%%%%%%%%%%%%%%%%%%%%%%%%%%%%%%%%%%%%%%%%%%%%%%%%%%%%%%%%%%%%%%%%%%%%
\subsection{Threat Models for Sustainability in Data Centers}

Although the trust assumptions and threat models for sustainable
systems may vary widely based on the system architecture and
requirements, the threat models for a sustainable data center
%\ag{system or data center? make it consistent}
can be primarily derived with respect to three entities: (a) the service
provider, (b) the users, and (c) third-party observers (\eg regulatory
agencies).
%
One may assume that the service provider can be considered
to be trusted but the underlying infrastructure (\eg OS and services)
provided by third-party vendors/suppliers
can be untrusted or become compromised, whereas others may assume that both
the service provider and the underlying infrastructure become rogue.
%
For example, benign and unsuspecting data center providers
often use virtual machines (VMs) or containers that are already offered
by third-party infrastructure providers and can be loaded with
backdoors or malware.
%
A malicious infrastructure provider can
deliberately manipulate energy consumption metrics, bypass
sustainability regulations, and overcharge the data center provider
for the total energy consumption.
%
This not only undermines the data center provider's sustainability
efforts but also leads to inflated costs and financial losses.
%
Moreover, when users' jobs run in an environment where
data center and/or infrastructure providers are malicious, attackers can
gain unauthorized access to read or modify the job's code and data.
%
For example, attackers may introduce unaccounted read/write
operations~\cite{graphene_sgx_atc17, glamdring} to users' jobs which
in turn inflate users' carbon footprints, leading to overbilling the
customers and increasing the financial profits of data center and
infrastructure providers.
Such carbon footprint inflation can also be achieved by violating the integrity
of the sustainability metrics (\eg code or
data)~\cite{graphene_sgx_atc17, glamdring} or by manipulating the
system traces and logs---the evidence trail of carbon
consumption~\cite{sgx_log_security_asiaccs17} by the compromised VMs or
malicious processes in data centers.
%
Similarly, compromised data center providers may report false
carbon footprints to the regulators~\cite{sgx_use_based_privacy_wpes18} to evade
high CO2 taxes or regulations.


The other key entities in data centers (\ie users) can also be
assumed to be untrusted as they may try to launch attacks (\eg DoS)
against other users or data center providers, or obtain higher
levels of service than they are allocated, and thus mislead the cloud service
providers about the user's carbon usage.
Last but not least, third-party observers (\eg regulatory agencies)
may be tasked with verifying the footprint reported by the service
provider in the process of executing policy or oversight (\eg by
comparing sustainability costs reported by cloud operators, users, and
utilities); but even these observers may be considered untrusted, as
they could collude with others to mislead reporting, may have rogue
insider elements within the data center, and may even be under political or other
pressure to ``fudge'' or misrepresent the data.


%%%%%%%%%%%%%%%%%%%%%%%%%%%%%%%%%%%%%%%%%%%%%%%%%%%%%%%%%%%%%%%%%%%%%%
\subsection{New Security Challenges for Sustainability}
\label{subsec:new-security-challenges}
Due to the complex design of data centers, which relies on intricate
trust assumptions among numerous stakeholders, it is necessary to
address diverse security threats ranging from malicious
software/service providers (in Software as a Service or SaaS models),
compromised operating systems or hypervisors (in Platform as a Service
or PaaS models), or compromised sensors, devices, and firmware owned by
infrastructure providers (in Infrastructure as a Service or IaaS
models) to malicious or honest-but-curious users.
%
In light of the above discussion, we discuss next some security
challenges for a system aiming for sustainability and
summarise those in Table~\ref{tab:sustainability_threats}.
Note that the nature of threats will be different for different
sustainable systems (\eg transportation, manufacturing) based on trust
assumptions.


\noindent \ding{113} \textbf{Lack of authenticity of carbon emission
  sources (C1).}  Sensors and devices (\eg PDUs) reporting and computing
sustainability data can be malicious and may become compromised due to
unintentional vulnerabilities or intended backdoors in their hardware,
firmware, and software~\cite{pdu-vulnerabilities}.  As a result, by
taking control of those sensors and devices, attackers may violate the
authenticity and forge carbon footprints to launch nefarious attacks.
For instance, attackers may cause over/under-billing to customers by
forging/manipulating carbon consumption records.
%
Attackers may also induce
carbon-exhaustion attacks on other users by misreporting over-consumption of
carbon, or evade compliance checking of regulatory agencies by misreporting
low carbon emissions when operating in test mode (similar to Volkswagen's
scandal~\cite{vwscandal2015}).  Similar kinds of sustainability
data-forgery attacks can also be carried out if there are vulnerabilities in the
communication protocols (\eg lack of authentication and replay protection) between sensors and the
sustainability data aggregators gleaning carbon footprints from
multiple such sensors.


\noindent \ding{113} \textbf{Untrustworthy physical environment (C2).}
Sensors and apparatuses used to collect carbon footprint data can be
subjected to direct and indirect data manipulation attacks.  For
example, an attacker having direct physical access to sensors or
data structure infrastructure can manipulate sensors' readings
to generate false sustainability data or manipulate the cooling system
to disrupt sustainability operations~\cite{physical-security}.
Conversely, in indirect attacks, the attackers do not
have direct physical access to sensors but exploit physical
side-channels~\cite{ding2021iotsafe} between different
components/sensors in data centers
to affect sustainability operations and cause reputation loss
to their competitors.  Due to such malicious actions,
additional water and electricity would be required to
cool the targeted data center, resulting in an increased
carbon footprint, higher operational costs, and disruption of
sustainability efforts.


\noindent \ding{113} \textbf{Lack of access control and information
flow control (C3).}
Sharing physical resources such as hardware and sensors among multiple
users introduces new challenge of isolating each tenant's data and ensuring that
one tenant cannot access another's sustainability footprints.
The lack of granular and dynamic access control configurations, and adequate
resource isolation, can lead to the failure to ensure that each workload
and its associated users have the appropriate access privileges to sustainability
footprints, without compromising data security.  Without proper access control and
information flow-control measures, there is, therefore, a risk of unauthorized access
to sensitive sustainability data, potentially leading to data breaches, privacy
violations, and other security issues.
Furthermore, sustainability data obtained from
various sources can be illegitimately tampered with by malicious users
processes or compromised system processes.  Malicious processes may
obtain unauthorized (read/write) access to sensitive resources (\eg
databases or protected memory regions storing sustainability data and
states) by exploiting vulnerabilities in the access control
policies~\cite{privilege-escalation}.
%
The lack of access control
mechanisms, such as Discretionary Access Control (DAC),
Mandatory Access Control (MAC), or combinations thereof, therefore, may
enable attackers to manipulate (\ie add, modify, or remove) carbon
footprint and
sustainability states.  As a result, the regular sustainability
operations of the system are likely to be disrupted, which may cause
the system to produce unwarranted carbon footprints or eliminate them.
Tampering with sustainability data by adversaries (\eg malicious
service providers or malicious users) may result in overcharging
legitimate users of the system (such as a data center), undercharging
malicious users attempting to evade sustainability costs, or damaging
the reputation of competing service providers.

\begin{table*}[h]
\footnotesize
\centering
\begin{tabular}{|m{0.3cm}|m{6.8cm} | m{4.7cm} | m{2.3cm}|}
\hline
\centering\textbf{ID} &
\centering\textbf{Vulnerabilities, Threats, and New Security Challenges} &
\centering\textbf{Impacts} &
\centering\textbf{Possible Ideas to Solutions}
\tabularnewline \hline
%\ding{182} &
\textbf{C1} &
Lack of authenticity of carbon emission sources allows malicious processes
to forge, tamper, or misreport carbon usage
&
Cause over-/under-billing to customers by tampering with carbon usage,
evade regulatory agencies by misreporting low carbon emissions
&
Verifiable footprint collection (\S\ref{subsec:verifiable_collection})
\tabularnewline \hline

%\ding{183} &
\textbf{C2} &
Untrustworthy physical environment may allow attackers to manipulate
sensors and apparatuses within a data center directly or indirectly
&
Induce higher operational costs, cause
over-/under-billing to customers, and denial-of-service attacks
&
Verifiable footprint collection (\S\ref{subsec:verifiable_collection})
\tabularnewline \hline

%\ding{186} &
\textbf{C3} &
Cryptographic flaws may allow forging the proof of carbon usage
&
Financial loss and disruption the data center operations
&
Verifiable footprint collection (\S\ref{subsec:privacy-preserving-collection})
\tabularnewline \hline

%\ding{184} &
\textbf{C4 \& C6} &
Disclosure of sustainability metrics to malicious
service providers and other users due to inadequate access control,
cryptographic protections, or side-channel vulnerabilities
&
Exposure of users' private data such as location, behavior, and
intellectual properties
&
Privacy-preserving footprint collection and aggregation
(\S\ref{subsec:privacy-preserving-collection},
\S\ref{subsec:privacy-preserving-aggregation}, \&
\S\ref{subsec:public-sustainability-ledger})
\tabularnewline \hline

%\ding{185} &
\textbf{C5} &
Lack of or flaws in the access control or information flow control
mechanisms may allow
malicious processes (controlled by malicious users or service providers) to
access and
tamper with the databases storing carbon footprint trails
&
Exposure of users' private data such as location, behavior, and intellectual
properties
&
Verifiable carbon footprint collection (\S\ref{subsec:privacy-preserving-collection})
\tabularnewline \hline


%\ding{187} &
\textbf{C7} &
Evasive carbon offset techniques allow corporations
to trade a known amount of carbon emissions with an uncertain
amount of carbon reductions
&
Tax evasion, financial loss, and environmental
hazards
&
Verifiable footprint collection (\S\ref{subsec:privacy-preserving-collection})
\tabularnewline \hline

%\ding{188} &
\textbf{C8} &
Multiple parties may collude to misreport carbon usage
&
Tax evasion, financial loss, and environmental
hazards
&
Verifiable footprint collection (\S\ref{subsec:privacy-preserving-collection})
\tabularnewline \hline

\end{tabular}
\caption{Threats and security challenges for the sustainability of
  data centers and potential research directions.
  %\ag{subsection numbers not showing up in table?} %\David{move C3 two rows up?}
  }
\label{tab:sustainability_threats}
\end{table*}

%%%%%%%%%%%%%%%%%%%%%%%%%%%%%%%%%%%%%%%%%%%%%%%%%%%%%%%%%%%%%%%%%%%%%%%%%%%%%%
%% For Emacs:
% Local variables:
% fill-column: 70
% End:
%%%%%%%%%%%%%%%%%%%%%%%%%%%%%%%%%%%%%%%%%%%%%%%%%%%%%%%%%%%%%%%%%%%%%%%%%%%%%%
%% For vim:
% vim:textwidth=70
%%%%%%%%%%%%%%%%%%%%%%%%%%%%%%%%%%%%%%%%%%%%%%%%%%%%%%%%%%%%%%%%%%%%%%%%%%%%%%
% LocalWords:  HotCarbon externalities Pigovian Ent Jevons Pigou TODO
% LocalWords:  unforgeable tesla youtube de facto SaaS PaaS IaaS




\noindent \ding{113} \textbf{Sensitive information disclosure (C4).}
Collecting sustainability data from disparate carbon sources (\eg
sensors and PDUs) in an unregulated manner may disclose the
sustainability metrics to service providers and other users.  Such
unauthorized exposure of footprint data will violate the privacy of user's
data, location, behavior, and intellectual properties such as
proprietary scheduling techniques, factors used for competitive
pricing for service classes.  Unauthorized access to footprint data
can enable an adversary to prevent a co-tenant from realizing an
improved sustainability target or even allow them to initiate DoS attacks
on the co-tenant.



\noindent \ding{113} \textbf{Cryptographic flaws (C5).}
The ability of a sustainable system to provide proof of
carbon footprint to users and regulators is essential for ensuring the
trustworthiness of the system.  Such proof of footprint should be built
with cryptographic constructs.  But flaws in the integration of
cryptographic constructs with complex data center systems (\eg using
weak cipher suites~\cite{ms365-insecure-block-cipher,
  samba-outdated-crypto}) or flaws in the
implementations~\cite{heartbleed} may fail to generate unforgeable and
accurate proof of consumption, enabling an attacker to drop, modify, replay,
and inject fake footprints of carbon.  This can disrupt the
operations of sustainable systems.


\noindent \ding{113} \textbf{Side-channels in sustainability (C6).}
Due to shared hardware resources, co-located servers, and poor
isolation between different processes running on the same hardware in
data centers, side-channel vulnerabilities (\eg page faults~\cite{xu2015controlled},
cache misses~\cite{wang2007new}, power~\cite{randolph2020power} and
timing~\cite{hund2013practical} channels) may allow a malicious
application to observe or tamper with carbon footprint
patterns of other users' jobs/applications running on the same
hardware.  Such side-channels not only allow an attacker to
fingerprint the data traffic of other users but also to extract the
cryptographic keys or other confidential information of a user
application by looking at the use of sustainability
metrics~\cite{cloud-side-channel-ristenpart}.  Attackers can exploit
such sensitive information to blackmail or embarrass other
users/competitors (\eg to force a competitor's stock to drop, or
short-sell such stock).


\noindent \ding{113} \textbf{Evasive carbon offset techniques (C7).}
Corporations often trade a known amount of carbon emissions with an
uncertain amount of emission reductions to claim carbon neutrality
(\eg by investing in forestation
elsewhere)~\cite{tesla_carbon_offset}.  This technique, also called
carbon credit or climate credit, has been in practice for decades.  It
is often exploited by large corporations as it is extremely difficult,
if not impossible, to track and verify if the amount of emissions
balances out the amount of reductions~\cite{junk_carbon_offset,
  myth_carbon_offset, scam_carbon_offset}.  Often, Renewable Energy
Credits (RECs) are used to offset the carbon footprint of a data
center via the purchase of energy credits from a green energy
generator~\cite{drec-initiative}.  Similarly, Power Purchase
Agreements (PPAs)~\cite{ppa} are used to have the data center operator
finance the installation of a green energy producing farm, run, owned
and managed by an independent party, to provide green energy to the
data center over a long-term period covered under the PPA.  For both
REC and PPAs, the authenticity of green energy is, however, often kept
out of sight of the users.  The lack of authentication, therefore,
enables corporations to make false claims about the energy source,
while appearing in public to support sustainability efforts.


\noindent \ding{113} \textbf{Collusion for evasion (C8).}  Infrastructure
providers and Power Distribution Unit (PDU) providers may collude to
misreport carbon footprints to regulators and users and thus may evade
regulatory agencies.
Such collusion attacks can be of different
combinations as infrastructure providers depend on third-party
software and hardware vendors which may also collude with each other
for malicious purposes.


\begin{comment}
%\item
\noindent $\bullet$ \textbf{Adversarial influence on metrics optimization.}
  Machine-learning--based optimization techniques are often used to optimize the usage of carbon
  based on the sustainability data collected from different sources (\ie
  system components).
  \ag{not sure I agree with above statement. do we have cites? usually, optimization techniques are used, which need not be ML-based.}
  Hence, like any other machine learning
  based system, sustainable systems are also vulnerable to different
  ML-based attacks, including poisoning the training data or adversarial
  samples during inference, model stealing, etc.\ezk{add cites}
\end{comment}


%%%%%%%%%%%%%%%%%%%%%%%%%%%%%%%%%%%%%%%%%%%%%%%%%%%%%%%%%%%%%%%%%%%%%%%%%%%%%%
%% For Emacs:
% Local variables:
% fill-column: 70
% End:
%%%%%%%%%%%%%%%%%%%%%%%%%%%%%%%%%%%%%%%%%%%%%%%%%%%%%%%%%%%%%%%%%%%%%%%%%%%%%%
%% For vim:
% vim:textwidth=70
%%%%%%%%%%%%%%%%%%%%%%%%%%%%%%%%%%%%%%%%%%%%%%%%%%%%%%%%%%%%%%%%%%%%%%%%%%%%%%
% LocalWords:  HotCarbon externalities Pigovian Ent Jevons Pigou TODO
% LocalWords:  unforgeable tesla youtube de facto SaaS PaaS IaaS

\section{Research Directions for Securing Sustainable Data Centers}
\label{s:verify}



% Figure environment removed

Although many solutions~\cite{confidential-computing}
%~\cite{berger2008tvdc, confidential-computing}
have been designed for data-center security, most of them are not directly applicable
to counter the security and privacy challenges towards sustainability as
discussed in Section~\ref{sec:security-challenges}.
Therefore, we must develop technologies that will help build
secure and trustworthy sustainable systems.
%
Particularly, we must develop primitives that allow domain experts to
construct and operate sustainable systems and verify the results.
%
Next, we lay out several potential research directions for improving
sustainability in data centers through security.



%%%%%%%%%%%%%%%%%%%%%%%%%%%%%%%%%%%%%%%%%%%%%%%%%%%%%%%%%%%%%%%%%%%%%%
\subsection{Verifiable Footprint Collection Architecture}
\label{subsec:verifiable_collection}


One of the most important elements of a sustainable system is its
ability to promote the responsible use of system resources, such as
complying with carbon emission restrictions/taxes.  However, claims of
carbon usage must be accompanied by infrastructure that demonstrates
\emph{verifiable footprint} to the public and regulatory organizations.
This calls for architectures and systems that can collect publicly
readable and verifiable sensor readings in adversarial settings.  It
is essential that these systems have the ability to scale seamlessly
from small, low-energy devices to larger, enterprise-level data
centers.  The system architecture should have the ability to generate
tamper-resistant proofs of carbon consumption that are unforgeable,
accurate, and securely retrievable by authorized parties (which might
include the public) in adversarial deployments.  Furthermore, to provide higher
security assurance, the design and implementation of these systems
must be formally verified.

\noindent \textbf{Potential Solutions}:
%
\mycolor{blue}
Developing such a framework poses key challenges, including the need
to establish and preserve a root of trust using trusted hardware, such
as Trusted Platform Module (TPM)
to secure the data center's carbon footprint measurement components.
%
A trusted path should be established from the secure hardware up to the
module that collects all the relevant metrics of a job, and further up
to the component that verifies the accuracy of the reported
metrics.
\mycolor{black}
%
This trusted path will be capable of producing tamper-proof
evidence of sustainability cost metrics using cryptographic proof
systems.

One potential solution to ensure the security of
sustainability-related components is to use a hardware-based Trusted
Execution Environment (TEE) such as ARM TrustZone, 
%\footnote{\href{https://developer.arm.com/documentation/100690/0201}{https://developer.arm.com/documentation/100690/0201}},
%~\cite{armtz2016},
Intel SGX, 
%\footnote{\href{https://www.intel.com/content/www/us/en/architecture-and-technology/software-guard-extensions.html}{https://www.intel.com/content/www/us/en/architecture-and-technology/software-guard-extensions.html}},
%~\cite{mckeen2013innovative},
AMD SEV, 
%\footnote{\href{https://www.amd.com/en/developer/sev.html}{https://www.amd.com/en/developer/sev.html}}
%~\cite{kaplan2016sev},
and Keystone. 
%\footnote{\href{https://github.com/keystone-enclave/keystone}{https://github.com/keystone-enclave/keystone}}.
%~\cite{lee2020keystone}.
%
%\David{I doubt we need these footnotes (maybe Keystone)}
TEEs are deployed in nearly every commercial processor sold today and
are the de-facto standard to provide a tamper-proof execution
environment that preserves the integrity and confidentiality of data
and execution~\cite{graphene_sgx_atc17}.
% , scone_osdi16,
%   sgx_fv_usenix21}.
%
These environments provide isolation guarantees needed to certify that
metric data is collected and reported accurately, even in the presence
of malicious applications, OS, or hypervisor.
%
A \emph{sustainability collector} (see Figure~\ref{fig:arch}) running
in a TEE will securely collect the utilization details of a
bare-metal, virtualized, or containerized job.
%
The gathered metrics
will create a comprehensive timeline of user, system, and
process-oriented carbon footprints, culminating in a
\emph{sustainability provenance record} for the cloud.
%
The sustainability collector will securely report the metrics to a
\emph{sustainability certification agent}, which will produce
lightweight cryptographic proofs that empower third-party regulators
and users to independently verify the claimed consumption.

\mycolor{blue}
Note that any flaws in the design or implementation of
sustainability-related components, \eg measurement or collection code
running within TEEs and owned by respective TEE hosting entities
(\ie data center operators or service providers)
may introduce new security challenges.  For instance,
attackers may exploit such flaws and bypass the tamper-proof guarantees
of the code.
Therefore, it is crucial to ensure high-security assurance of these components through formal
analysis before they are deployed.
Also, the physical or virtual machines hosting the measurement code within TEEs
and the regulatory agencies need to verify during runtime the integrity of the trusted
path from the secure hardware to corresponding TEEs periodically
or when there are major changes (\eg write operations) in the system or
their combinations thereof.
\mycolor{black}



Another potential concern is that current TEE platforms
might lack adequate privileges to monitor the carbon or resource consumption
of workloads that execute outside of the TEE.  This might necessitate
new hardware support for TEEs to allow secure monitoring of external
workloads, including the host OS or hypervisor.

One possible alternative to TEEs is to explore the use of add-on
monitoring hardware, akin to SmartNICs, that can collect
sustainability metrics from outside the host.
%
For example, AWS Nitro\footnote{\href{https://aws.amazon.com/ec2/nitro/}{https://aws.amazon.com/ec2/nitro/}}
%~\cite{nitro}
enables SmartNICs to monitor and
manage VM allocation and scheduling, while being technically
``outside'' the host OS.
%
Similarly, sustainability-related components could potentially run on
such add-on custom hardware with the necessary privileges to gather
data from the host without being vulnerable to compromise by the host.
Finally, sustainability data must be isolated from other workloads running on
the same machine, providing protection against unauthorized access and
tampering.

%%%%%%%%%%%%%%%%%%%%%%%%%%%%%%%%%%%%%%%%%%%%%%%%%%%%%%%%%%%%%%%%%%%%%%
\subsection{Privacy-Preserving Footprint Collection}
\label{subsec:privacy-preserving-collection}
Fine-grained sustainability data collected through disparate carbon sources, such
as sensors and PDUs in an unregulated manner, may induce unintended
disclosure of sensitive data.
The exposure of sustainability records would otherwise break the
users' privacy, data, location, behavior, and intellectual properties
such as proprietary scheduling techniques, trained machine learning
models, and factors used for competitive pricing for service classes~\cite{hlavacs2011energy, mckenna2012smart}.
%
Also, attackers may attempt to tamper with sensor data before it
is aggregated, which can lead to incorrect or misleading results. This
can be especially problematic in safety-critical applications, such as
autonomous vehicles or medical devices.

\noindent \textbf{Potential Solutions}: In concert with the verifiable
sustainability data collection architecture, differential privacy (DP) or
local differential privacy (LDP)
can be used as a probabilistic solution for privacy-preserving sustainability
footprint collection.
%
A certain degree of noise can be added to
the collected data to obscure individual data points but still allow
for useful aggregate analysis~\cite{dwork2006differential}.
%, dwork2008differential}.
%
\mycolor{blue}
A classical challenge of such differential privacy-based solutions would
be to keep the utility (\eg the statistical properties) of the sustainability data
high to the system while still protecting the privacy of users and systems.
In other words, the privacy budget---the amount of noise that can be
added to the sustainability data without compromising privacy---needs
to be determined by the sensitivity of the sustainability data being
collected and the desired level of privacy protection.
Another challenge for DP-based solutions is to keep the total noise
added by all parties within an acceptable range and failure to do so
requires a trusted aggregator to correct the noise.
Since DP-based solutions protect the data owner by providing
indistinguishability of the dataset, they can be used as a
privacy-preserving way of releasing data.  However, one has to ensure
correct-by-constructions~\cite{morinov2012ccs, jin2022we} of such while adopting
them.
\mycolor{black}
%

\mycolor{blue}
To provide cryptographic guarantees and to preserve the utility of sustainability data
utility to a higher extent compared to differential privacy,
an alternate solution is to use homomorphic encryption\footnote{Craig Gentry. A fully homomorphic encryption
scheme. Ph.D. Dissertation, Stanford university, 2009.}. %~\cite{gentry2009fully}.
With this solution, the carbon sources can encrypt the sustainability data as well as enable
the decision-making agent to measure/compute any statistical
information on those encrypted data.
\mycolor{black}
%
There are, however, several challenges associated with this
solution. 
%~\cite{gentry2009fully}.
%~\cite{naehrig2011can}.
%
%\David{I doubt we need this footnote}
Homomorphic encryption (HE) requires significant computational resources
and can increase the size of the actual data (because of encryption) being
transmitted,
%~\cite{naehrig2011can},
making it more difficult to store and transmit efficiently.
Furthermore, there are currently limitations
%~\cite{naehrig2011can}
on the types of computations that can be performed on homomorphically
encrypted data.  For example, homomorphic encryption schemes support
only addition and multiplication.  Complex operations, such as
division or trigonometric functions, may not be efficiently supported. 
\mycolor{blue}
While the direct use of homomorphic encryption may not be appropriate
for resource-constrained carbon emission sources,
further research is warranted to check if optimized versions of HE such as
partial HE, leveled HE, and threshold HE can be utilized or a
new, lightweight, secure, and bespoke HE (\eg selective HE)
needs to be designed for sustainability in data centers.
Nevertheless, many major chip/system vendors such as
Intel, AMD and ARM are actively exploring hardware support for
HE and when these are available, they can provide a trusted basis for
implementing challenges to many of the security solutions identified in this article.
\mycolor{black}

Another alternative approach
involving less computational overhead than homomorphic encryption is
zero-knowledge proofs~\cite{zero-knowledge-proof},
%~\cite{fiege1987zero},
in which the carbon sources
can demonstrate to the sustainability certification agent,
that sustainability footprints are valid, without disclosing
the actual values that would otherwise compromise privacy.
%
\mycolor{blue}
However, zero-knowledge proofs can only be used to prove the authenticity of sustainability data
and is not intended for analyzing and making any decisions.
To address the challenges of each solution, further investigation is needed
to determine if homomorphic encryption or differential privacy can be
combined with zero-knowledge proofs.

\mycolor{black}


%%%%%%%%%%%%%%%%%%%%%%%%%%%%%%%%%%%%%%%%%%%%%%%%%%%%%%%%%%%%%%%%%%%%%%
\subsection{Privacy-Preserving Footprint Aggregation}
\label{subsec:privacy-preserving-aggregation}
Collecting and processing sustainability data from multiple sites in
data centers require secure collaboration between multiple untrusted
parties, including cloud operators, regulators, and users, each with
their own confidentiality, privacy, security, and trust requirements.
%
While being aggregated either in centralized or distributed data
centers, sustainability data can still reveal sensitive information
about users and systems as discussed in Section~\ref{subsec:new-security-challenges}.
Therefore, the high-level goals are to (1) perform aggregation,
summary, or other functions on the sustainability data whose results
do not disclose information about the underlying data; and (2) ensure
that aggregations provide (provably) accurate higher-level data
without exposing underlying sensitive information, \eg proof of
sustainability compliance of the manufacturing process without
exposing unit-wise behaviors or specific metrics.



\noindent \textbf{Potential Solutions}: A plausible approach to
privacy-preserving aggregation for sustainability data is
secure multi-party communication (MPC) in which multiple
carbon footprint aggregators located at different locations
collaborate to perform computations on their combined data without
revealing any individual data points~\cite{goldreich1998secure}.
MPC requires minimal trust and aims to ensure each party's
input is kept private while allowing them to compute the
desired aggregation, summary, or other
functions on their combined data whose results do not disclose
information about the underlying data.
%
One such MPC platform is Confidential Space by Google\footnote{\href{https://cloud.google.com/blog/products/identity-security/announcing-confidential-space}{https://cloud.google.com/blog/products/identity-security/announcing-confidential-space}}, 
%~\cite{confidential-space},
which would allow sustainability data to be encrypted and stored in a
TEE that only authorized workloads are allowed to access.
%
Additionally, such data is isolated from other workloads running on
the same machine, protecting unauthorized access and
tampering.
\mycolor{blue}
MPC-based solutions, however, incur higher computational and communication
overheads due to secure computations and sharing of encrypted results.
%To prevent an adversary from linking a particular record
%to a specific party.An alternate solution is to use K-anonymity

%

%\David{The footnote for FL is not needed}


To minimize sustainability data movement, federated learning 
%\footnote{\href{https://blog.research.google/2017/04/federated-learning-collaborative.html}{https://blog.research.google/2017/04/federated-learning-collaborative.html}}
%\cite{li2020federated}
can be used in which training a machine learning model (\eg carbon footprint
optimization) on decentralized sustainability data/metrics can be
performed without having to transfer the data to a centralized
location.
\mycolor{black}
%
Each site of the distributed data center will train a local model on
its sustainability data and send the updated model weights to a
central server, which aggregates them to create a global model.
%
This approach allows data to remain local and private while still
benefiting from a centralized learning process.
%
Note that existing federated learning techniques are susceptible to
model-poisoning and model-stealing attacks; this further imposes
challenges to adopt federated learning-based solutions for aggregating
sustainability data. %~\cite{li2020federated}.


%%%%%%%%%%%%%%%%%%%%%%%%%%%%%%%%%%%%%%%%%%%%%%%%%%%%%%%%%%%%%%%%%%%%%%
\subsection{Public Sustainability Ledgers}
\label{subsec:public-sustainability-ledger}

Public sustainability ledgers can be used for tracking carbon
emissions or energy consumption and thus can provide transparency and
accountability in the management of resources.
%
However, there are also security and privacy issues that need to be
considered when using these public ledgers.
%
For example, if public ledgers contain sensitive data (\eg carbon
credit allocations, sales, and expenditures) about the sustainability
practices of individuals and organizations, attackers may track the
individuals/organizations or infer proprietary algorithms.
%
Also, sustainability data may be stored on multiple public ledgers or
private databases, which may not be interoperable.
%
This can create challenges in ensuring data consistency and accuracy,
and may also lead to data breaches if not properly secured.

\noindent \textbf{Potential Solutions}: In combination with
privacy-preserving measures, such as homomorphic encryption,
zero-knowledge proofs, multi-party computations, and differential
privacy, public ledgers for sustainability reporting can be provided
through smart contracts~\cite{aloqaily2020energy} deployed on the
public blockchain.
%
The smart contract records the sustainability footprints from different sources
and stores the encrypted records in blocks on the blockchain.
The sustainability footprints submitted to the blockchain undergo
verification by the participating entities through a consensus mechanism,
such as Proof-of-Work (PoW) or Proof-of-Stake (PoS).
%
This ensures the accuracy
and integrity of the recorded footprints.
Consumers, stakeholders, and regulators can access the public blockchain
to track and verify the provenance of sustainability footprints.
Although smart contracts---in concert with a verifiable
sustainability footprint collection architecture (Figure~\ref{fig:arch}) and
privacy-preserving measures---can offer secure and public sustainability ledgers,
smart contracts can also be subject to vulnerabilities that can be exploited by
attackers. %~\cite{perez2021smart}.
%
As such, it is important to thoroughly test and audit smart contracts
to ensure their security and reliability. %~\cite{tsankov2018securify}.
%
Furthermore, blockchain technology 
%\footnote{\href{https://www.ibm.com/topics/blockchain}{https://www.ibm.com/topics/blockchain}}
%~\cite{pilkington2016blockchain}
can be used to address the inconsistency and data-breach issues of
distributed public ledgers.
%
However, current blockchain technologies are susceptible to various
types of attacks including 51\% (majority) attacks and
denial-of-service attacks. %~\cite{zhang2019security}.
%
As such, it is important to ensure that the blockchain network is
properly secured and appropriate security measures are in place
to prevent such attacks.
%

\mycolor{blue}
While the potential security solutions outlined in this paper may contribute to carbon footprints, future research is necessary to rigorously evaluate the performance
and security guarantees of the existing and newly designed solutions.
As discussed in Section~\ref{sec:security-challenges}, the importance
of such security solutions in ensuring the trustworthiness of sustainability data and incentivizing the users toward sustainability practices is crucial for addressing global climate change and is believed to outweigh the impact of systems lacking such guarantees.

\mycolor{black}

\mycolor{blue}
%%%%%%%%%%%%%%%%%%%%%%%%%%%%%%%%%%%%%%%%%%%%%%%%%%%%%%%%%%%%%%%%%%%%%%
\section{Enhancing Standardization of Security
Mechanisms}

Security mechanisms are essential to ensure
compliance with regulations and standards,
preventing unauthorized access, and exposure,
tampering, or misuse of sustainability data.
Irrespective of the specific solution used to ensure security
of sustainability, a common need is to ease the adoption of those mechanisms
and reduce their footprint, both in terms of performance and
sustainability.
%
For instance, a TEE-based solution for verifiable data collection or
a homomorphic encryption-based approach for privacy-preserving
footprint collection should be lightweight and have small footprints
so as to minimize overall carbon consumption.
As trustworthiness is foundational in sustainability initiatives,
stakeholders, including governments, businesses, and users,
need high security and privacy assurance of sustainability data, which
is crucial for the success and adoption of sustainability practices.
Since sustainability data is critical for understanding trends
and for long-term planning and monitoring to counter global
issues such as climate change,
it is necessary to rigorously evaluate the effectiveness of security measures toward sustainability to
make informed decisions for a sustainable future.
As sustainability is a global concern that requires collaboration across borders,
standardizing security mechanisms for sustainability data
will accelerate their adoption in other sectors, facilitate international
cooperation, and ensure consistent protection standards and interoperability. 
Incentives and regulations need to be introduced to motivate
organizations to adopt and implement standardized security
mechanisms.
%
These could include tax incentives, certification programs, or
regulatory requirements that prioritize sustainability and security.
Note that all challenges toward sustainability
cannot be solved with technical solutions alone.
Hence, offering both fundamental principles and secure guarantees is more likely to assist
in the development of policies.  This, in turn, can contribute to and accelerate the global effort to combat climate change.
Without robust policies, all optimizations are susceptible to the
Jevons paradox (i.e., increasing efficiency can lead to 
increased consumption), which signifies that both regulation and security
are crucial components. Hence, collaboration and cooperation among industry players, researchers, and policymakers are necessary to establish these common goals and objectives.
\mycolor{black}
%\zhenhua{Jevons paradox perhaps needs some explanation.}

%%%%%%%%%%%%%%%%%%%%%%%%%%%%%%%%%%%%%%%%%%%%%%%%%%%%%%%%%%%%%%%%%%%%%%%%%%%%%%
%% For Emacs:
% Local variables:
% fill-column: 70
% End:
%%%%%%%%%%%%%%%%%%%%%%%%%%%%%%%%%%%%%%%%%%%%%%%%%%%%%%%%%%%%%%%%%%%%%%%%%%%%%%
%% For vim:
% vim:textwidth=70
%%%%%%%%%%%%%%%%%%%%%%%%%%%%%%%%%%%%%%%%%%%%%%%%%%%%%%%%%%%%%%%%%%%%%%%%%%%%%%
% LocalWords:  PSU unforgeable incentivization TLS allocator SGX SEV
% LocalWords:  TrustZone de facto TEEs SASSY's compositional crypto
% LocalWords:  TEE's HSC sss TODO PDUs IPsec RAPL XMLERS Merkle LDP
% LocalWords:  cryptographically adaptively incentivized SmartNICs
% LocalWords:  PoW PoS Jevons

\section{Conclusion}
\label{sec:conc}
In this paper, we propose a fast leakage and variability-aware thermal simulation method that also captures the temperature dependence of conductivity. We derive a closed-form of the Green's function considering all these effects using novel insights and algebraic techniques. Our approach provides fast and accurate solutions for both the steady-state and the transient thermal profile and has been validated with a wide variety of test cases. As device dimensions continue to shrink, process variation has become a serious problem. The methods proposed in this work can equip designers to tackle this problem and may spawn further research in this area.


\section*{Acknowledgments}
The work reported in this paper has been supported by NSF under grants 2215017, % Penn State CNS Large
2214980, %  SBU - CNS Large
2046444, % Zhenhua CAREER
2215016, and % Kanad and David - CNS Large
2106263. % Erez-CNS Medium

\def\refname{REFERENCES}

\bibliographystyle{plain}
\bibliography{main} 


\begin{IEEEbiography}{S. R. Hussain}{\,}
 is an Assistant Professor in the Computer Science and Engineering Department at
 Pennsylvania State University.
 His research interests include systems and network security, formal methods, and sustainability.
 Hussain received the Ph.D. degree in Computer Science from Purdue University, West Lafayette.
He is a Member of ACM and IEEE.  Contact him at hussain1@psu.edu.
\end{IEEEbiography}

\begin{IEEEbiography}{P. McDaniel}{\,} is
  the Tsun-Ming Shih Professor of Computer Sciences in the School of Computer,
  Data \& Information Sciences at the University of Wisconsin-Madison.
  McDaniel's research focuses on a wide range of topics in computer and network
  security and technical public policy, with interests in mobile device security,
  the security of machine learning, systems, program analysis for security,
  sustainability and election systems. McDaniel is a Fellow of IEEE, ACM and AAAS,
  a recipient of the SIGOPS Hall of Fame Award and SIGSAC Outstanding Innovation Award,
  and the director of the NSF Frontier Center for Trustworthy Machine Learning.
  He also served as the program manager and lead scientist for the Army Research Laboratory's
  Cyber-Security Collaborative Research Alliance from 2013 to 2018.
  Prior to joining Wisconsin in 2022, he was the William L. Weiss Professor
  of Information and Communications Technology and Director of the Institute
  for Networking and Security Research at Pennsylvania State University. Contact him
  at mcdaniel@cs.wisc.edu.
\end{IEEEbiography}

\begin{IEEEbiography}{A. Gandhi} {\,} is an Associate Professor in the Computer Science Department at Stony Brook
  University.
  His research interests include performance analysis, modeling, and evaluation; distributed systems; and sustainability.
  Gandhi received the Ph.D. degree in Computer Science from Carnegie Mellon University.
  He is a Senior Member of ACM and a Senior Member of IEEE.
  Contact him at anshul@cs.stonybrook.edu.
\end{IEEEbiography}

\begin{IEEEbiography}{K. Ghose} {\,} is a SUNY Distinguished Professor of Computer Science at SUNY Binghamton (Binghamton University).
  His research interests include energy-aware systems at all scales, processor microarchitectures and hardware security.
  Ghose received the Ph.D. degree in Computer Science from Iowa State University.  He is a Member of ACM and IEEE.
  Contact him at ghose@binghamton.edu.
\end{IEEEbiography}

\begin{IEEEbiography}{K. Gopalan} {\,} is a Professor in the Computer Science Department at Binghamton University.
His research interests are in computer systems including virtualization, security, operating systems, networks, and sustainability.
He received his Ph.D. degree in Computer Science from Stony Brook University.
He is a senior member of IEEE and a member of ACM.
Contact him at kartik@binghamton.edu.
\end{IEEEbiography}

\begin{IEEEbiography}{D. Lee} {\,} is an Assistant Professor in the Computer Science Department at Stony Brook University.
His research interests include compilers, operating systems, computer architecture, security, and sustainability.
Lee received the Ph.D. degree in Computer Science Engineering from Michigan University, Ann Arbor.
He is a Member of ACM and IEEE.
Contact him at dongyoon@cs.stonybrook.edu.
\end{IEEEbiography}

\begin{IEEEbiography}{Y. D. Liu} {\,} is a Professor in the Computer Science Department at Binghamton
  University.
  His research interests include programming languages; software engineering; formal methods for security; sustainable and energy-aware applications and systems.
  Liu received the Ph.D. degree in Computer Science from the Johns Hopkins University.
  He is a Member of ACM.
  Contact him at davidl@binghamton.edu.
\end{IEEEbiography}

\begin{IEEEbiography}{Z. Liu} {\,} is an Associate Professor in the Department of Applied Mathematics and Statistics and Department of Computer Science at Stony Brook University. His research interests include optimization; machine learning; big data systems;  sustainable and energy-aware applications and systems. Liu received the Ph.D. degree in Computer Science from California Institute of Technology. He is a Member of ACM. Contact him at zhenhua.liu@stonybrook.edu.
\end{IEEEbiography}


\begin{IEEEbiography}{S. Mu} {\,} is an Assistant Professor in the
Department of Computer Science at Stony Brook University.
His research interests include distributed systems, multi-core systems, and sustainable systems.
Mu received the Ph.D. degree in Computer Science from Tsinghua University.
He is a Member of ACM. Contact him at shuai@cs.stonybrook.edu.
\end{IEEEbiography}


\begin{IEEEbiography}{E. Zadok} {\,} is a Professor at Stony Brook
  University.
  His research interests include computer systems, storage systems,
  security, performance optimizations, and sustainability.
  Zadok received the Ph.D. degree  in Computer Science from Columbia
  University.
  He is a Senior Member of the IEEE Computer Society, an ACM Distinguished
  Member, and a member of USENIX.  Contact him at Erez.Zadok@stonybrook.edu.
\end{IEEEbiography}

\end{document}

%%%%%%%%%%%%%%%%%%%%%%%%%%%%%%%%%%%%%%%%%%%%%%%%%%%%%%%%%%%%%%%%%%%%%%%%%%%%%%
%% For Emacs:
% Local variables:
% fill-column: 70
% End:
%%%%%%%%%%%%%%%%%%%%%%%%%%%%%%%%%%%%%%%%%%%%%%%%%%%%%%%%%%%%%%%%%%%%%%%%%%%%%%
%% For vim:
% vim:textwidth=70
%%%%%%%%%%%%%%%%%%%%%%%%%%%%%%%%%%%%%%%%%%%%%%%%%%%%%%%%%%%%%%%%%%%%%%%%%%%%%%
% LocalWords:  HotCarbon externalities Pigovian Ent Jevons Pigou TODO
% LocalWords:  Syed Rafiul Hussain Anshul Kanad Ghose Kartik Dongyoon
% LocalWords:  Yu Zhenhua Shuai cyberattacks Tsun Shih AAAS SIGOPS
% LocalWords:  microarchitectures Tsinghua
