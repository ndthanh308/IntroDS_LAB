%\section{Security Challenges of Sustainable Systems}

\section{Why is Sustainability a Security Problem?}
\label{sec:security-challenges}

Ensuring the accuracy and credibility of sustainability
metrics, as well as empowering audits by regulatory agencies,
require guaranteeing the trustworthiness
and comprehensiveness of not only the carbon footprints of data center
equipment but also the embodied energy throughout the entire life-cycle
of computing equipment.
%
Although some  external information---such as that for
renewable energy, energy credits, or supplied water---can be
authenticated via trusted third parties,
%~\cite{co2e_epa, iea},
sustainability metrics in data centers require the
authenticity, confidentiality, integrity, and availability of data
collected, processed, stored, and used locally within a
data center~\cite{gandhi2022metrics}.
%
However, unlike traditional cloud computing systems
where the focus is primarily on security and privacy of user applications
and data,  %~\cite{carlin2013cloud}, %
%, zissis2012addressing, chen2010s},
collecting and measuring data center activities that impact humans and
the environment in a verifiable and privacy-preserving manner
presents a diverse set of new security challenges.
%
Most of these challenges are primarily based on sustainability data,
reliability of equipment, and cleanliness of energy sources---across
both the digital and physical worlds.
%
Unfortunately, no prior research has investigated the threat landscape of
sustainable data centers, nor attempted to provide any techniques or tools
that directly allow authentication of operational sustainability
metrics induced within a data
center to preserve the privacy of users' or operators'
sustainability data.
%
\mycolor{blue}
Also, we note that a key distinction with sustainable data, as opposed
to regular data in cloud infrastructure, is that it is generated
independently of user intent, resulting in reduced trust guarantees.
The critical factor here is to prevent users from misrepresenting
their emissions.  Therefore, this data must be generated, collected,
and aggregated in a manner that is tamper-resistant, akin to a
physical value, ensuring that it is nearly impossible for anyone to
manipulate and does not expose any sensitive information about
users.  In a nutshell, this threat model is different from most common
data as the trust has to be minimal.
\mycolor{black}
%
It is thus imperative to ensure the security of (i) data collection
processes, (ii) the process of generating verifiable, easily auditable
sustainability metrics, and (iii) the storage of all pertinent
information.
Hence, while being indispensable for protecting the environment and our
planet, we have found and argue that the current sustainability
practices---through self-reporting, best-effort measurement, and
anything less than complete verifiable control of
sustainability---will fail.




\mycolor{blue}
\subsection{Threat Models in Data Centers}
\label{subsec:threat_model}

The trust assumptions and threat models for sustainable
data centers may vary widely based on data center type (\eg
multi-tenant and hyperscale vs.\ enterprise data centers), service models offered by
the data center or the tenants, and any other specific requirements.
In general,
the threat models for a co-located, hyperscale, or cloud data center's
sustainability can be primarily derived with respect to four entities:
(a) data center providers, (b) tenants or service providers,
(c) users, and (d) third-party observers (\eg regulatory
agencies) leading to the following adversarial capabilities.
%
($\mathcal{A}_1$) Here, data centers provide misleading or
false sustainability data to attract end-users or third-party
service providers.
($\mathcal{A}_2$) Data centers or tenants
providing IaaS, PaaS, and SaaS, often
characterized as \emph{honest but curious}, may attempt to
learn the proprietary or sensitive data of their users and
exfiltrate it to third parties.
($\mathcal{A}_3$) Data center or service providers have access to their users or tenant's sustainability data and can be inherently malicious to exploit this information to harm the users or learn proprietary information that would benefit competitors or harm their tenants/consumers.
($\mathcal{A}_4$) Tenants (\ie service providers), on the other hand,
can also subvert the security and privacy of the other co-located tenants' resources and the facilities provided to them. 
($\mathcal{A}_5$) To make matters worse, resources (hardware and software) served
by tenant (\eg IaaS, PaaS, or SaaS) within a data center can also
be compromised and controlled by external attackers who are nation-states or
rival organizations offering similar services.
This is possible
due to system/service misconfigurations, insecure communication protocols
inadequate access controls and isolation of shared and physical resources,
and vulnerabilities in the hardware, software, or other components of the service providers' supply chains.
For example, benign and unsuspecting data center providers
often use virtual machines (VMs) or containers created by IaaS providers
that are loaded with backdoors or malware illegitimately reading/writing sensitive
carbon footprint data.
%

($\mathcal{A}_6$) The other key entities in data centers (\ie users or customers of a tenant)
can also be considered malicious.  This is because a
user's job (\eg a process) running in a data center
may attempt to gain unauthorized access to read or modify
other jobs' code and data and thus affect the
sustainability data produced by other jobs. 
For example, a malicious process of an end user may
add unaccounted read/write operations~\cite{graphene_sgx_atc17}
%, glamdring}
to users' jobs which
can inflate users' carbon footprints, leading to overbilling the
victim customers. 
Such carbon footprint inflation can also be achieved by violating the integrity
of the sustainability metrics (\eg code or
data)~\cite{graphene_sgx_atc17}
%, glamdring}
or by manipulating the system traces and logs---the evidence trail of carbon
consumption
%~\cite{sgx_log_security_asiaccs17}
by the compromised VMs or
malicious processes in data centers.
%
Similarly, compromised data center providers may exploit the same and
use similar malicious processes to report false
carbon footprints to the regulators
%~\cite{sgx_use_based_privacy_wpes18}
to evade high carbon taxes or regulations~\cite{graphene_sgx_atc17}. 
Users may also try to launch attacks (\eg DoS)
against other users or the tenant who owns that service, another tenant or its users
in the same data center.  Users may also strive to obtain higher
levels of service than they are allocated, and thus mislead the service
providers about the user's carbon usage.
Various surfaces can be utilized by users to
attack the tenant, including the hypervisor, VMs, APIs and web services.

Last but not least, third-party observers (\eg regulatory agencies)
are tasked with verifying the footprint reported by the data center and service
providers in the process of executing policy or oversight (\eg by
comparing sustainability costs reported by cloud operators, users, and
utilities).  ($\mathcal{A}_7$) But even these observers
may be honest but curious, government or law enforcement agencies performing
surveillance,
or untrusted as they could collude with others to mislead reporting, may have rogue
insider elements within the data center, and may even be under political or other
pressure to ``fudge'' or misrepresent the data.
\mycolor{black}


\mycolor{blue}
\noindent\textbf{Differences with other systems}. Although there are some similarities between data centers and IoT and enterprise IT systems regarding data collection, verifiability and storage, the key distinction lies in the threat model between these systems.  
For instance, in most IoT settings, 
users, being the owners of their homes and devices, do not tamper with the devices to generate false data. The users also generally trust the 
trigger-action platforms capable of storing sensor data as those
platforms are the key enablers of automation. 
In most cases, the
third-party smart apps (\ie trigger-action rules) or external attackers 
are untrusted as they are the primary attack vectors.  Another
notable distinction with the sustainability data in data centers, as
compared to regular IoT data, is that the sustainability footprints
recorded by physical and virtual infrastructures (\eg power
generators, cooling systems, virtual machines, and hypervisors) are
shared across mutually untrusted stakeholders.  This gives rise to
privacy concerns, which inherently differ from those in IoT or enterprise IT  systems where multiple users share the same physical environment (\eg
smart home and building) are mutually trusted and hence one user's 
IoT activities are not considered sensitive/private to another user in the
same home/building.
\mycolor{black}




\begin{table*}[h]
\footnotesize
\centering
\begin{tabular}{|m{0.3cm}|m{6.8cm} | m{4.7cm} | m{2.3cm}|}
\hline
\centering\textbf{ID} &
\centering\textbf{Vulnerabilities, Threats, and New Security Challenges} &
\centering\textbf{Impacts} &
\centering\textbf{Possible Ideas to Solutions}
\tabularnewline \hline
%\ding{182} &
\textbf{C1} &
Lack of authenticity of carbon emission sources allows malicious processes
to forge, tamper, or misreport carbon usage
&
Cause over-/under-billing to customers by tampering with carbon usage,
evade regulatory agencies by misreporting low carbon emissions
&
Verifiable footprint collection (\S\ref{subsec:verifiable_collection})
\tabularnewline \hline

%\ding{183} &
\textbf{C2} &
Untrustworthy physical environment may allow attackers to manipulate
sensors and apparatuses within a data center directly or indirectly
&
Induce higher operational costs, cause
over-/under-billing to customers, and denial-of-service attacks
&
Verifiable footprint collection (\S\ref{subsec:verifiable_collection})
\tabularnewline \hline

%\ding{186} &
\textbf{C3} &
Cryptographic flaws may allow forging the proof of carbon usage
&
Financial loss and disruption the data center operations
&
Verifiable footprint collection (\S\ref{subsec:privacy-preserving-collection})
\tabularnewline \hline

%\ding{184} &
\textbf{C4 \& C6} &
Disclosure of sustainability metrics to malicious
service providers and other users due to inadequate access control,
cryptographic protections, or side-channel vulnerabilities
&
Exposure of users' private data such as location, behavior, and
intellectual properties
&
Privacy-preserving footprint collection and aggregation
(\S\ref{subsec:privacy-preserving-collection},
\S\ref{subsec:privacy-preserving-aggregation}, \&
\S\ref{subsec:public-sustainability-ledger})
\tabularnewline \hline

%\ding{185} &
\textbf{C5} &
Lack of or flaws in the access control or information flow control
mechanisms may allow
malicious processes (controlled by malicious users or service providers) to
access and
tamper with the databases storing carbon footprint trails
&
Exposure of users' private data such as location, behavior, and intellectual
properties
&
Verifiable carbon footprint collection (\S\ref{subsec:privacy-preserving-collection})
\tabularnewline \hline


%\ding{187} &
\textbf{C7} &
Evasive carbon offset techniques allow corporations
to trade a known amount of carbon emissions with an uncertain
amount of carbon reductions
&
Tax evasion, financial loss, and environmental
hazards
&
Verifiable footprint collection (\S\ref{subsec:privacy-preserving-collection})
\tabularnewline \hline

%\ding{188} &
\textbf{C8} &
Multiple parties may collude to misreport carbon usage
&
Tax evasion, financial loss, and environmental
hazards
&
Verifiable footprint collection (\S\ref{subsec:privacy-preserving-collection})
\tabularnewline \hline

\end{tabular}
\caption{Threats and security challenges for the sustainability of
  data centers and potential research directions.
  %\ag{subsection numbers not showing up in table?} %\David{move C3 two rows up?}
  }
\label{tab:sustainability_threats}
\end{table*}

%%%%%%%%%%%%%%%%%%%%%%%%%%%%%%%%%%%%%%%%%%%%%%%%%%%%%%%%%%%%%%%%%%%%%%%%%%%%%%
%% For Emacs:
% Local variables:
% fill-column: 70
% End:
%%%%%%%%%%%%%%%%%%%%%%%%%%%%%%%%%%%%%%%%%%%%%%%%%%%%%%%%%%%%%%%%%%%%%%%%%%%%%%
%% For vim:
% vim:textwidth=70
%%%%%%%%%%%%%%%%%%%%%%%%%%%%%%%%%%%%%%%%%%%%%%%%%%%%%%%%%%%%%%%%%%%%%%%%%%%%%%
% LocalWords:  HotCarbon externalities Pigovian Ent Jevons Pigou TODO
% LocalWords:  unforgeable tesla youtube de facto SaaS PaaS IaaS

%%%%%%%%%%%%%%%%%%%%%%%%%%%%%%%%%%%%%%%%%%%%%%%%%%%%%%%%%%%%%%%%%%%%%%


\mycolor{blue}
\subsection{Security Challenges for Sustainability}
\label{subsec:new-security-challenges}
Due to the complex design of data centers and intricate interactions
among their stakeholders, it is necessary to characterize and
address diverse security threats on the sustainability pursuit of data centers. 
Next, we discuss some critical security
challenges for a data center aiming for sustainability,
summarized in Table~\ref{tab:sustainability_threats}.
Note that the nature of threats will be different for different
sustainable systems (\eg transportation, manufacturing) based on trust
assumptions.


\noindent \ding{113} \textbf{Evasive carbon offset techniques (C1).}
Data centers and large corporations often trade a known amount of carbon emissions with an
uncertain amount of emission reductions to claim carbon neutrality
(\eg by investing in forestation
elsewhere). %~\cite{tesla_carbon_offset}.
This practice, also called
carbon crediting or climate crediting, has been in place for decades.  It
is often exploited by large corporations as it is extremely difficult,
if not impossible, to track and verify if the amount of emissions
balances out the amount of reductions.
%~\cite{scam_carbon_offset}.
%~\cite{junk_carbon_offset, myth_carbon_offset, scam_carbon_offset}.
Often, Renewable Energy
Credits (RECs) are used to offset the carbon footprint of a data
center via the purchase of energy credits from a green energy
generator. %~\cite{drec-initiative}.
Similarly, Power Purchase
Agreements (PPAs)\footnote{\href{https://sourcingjournal.com/topics/sustainability/apparel-sustainability-unethical-practices-73619/}{https://sourcingjournal.com/topics/sustainability/apparel-sustainability-unethical-practices-73619/}}
%~\cite{ppa}
are used to have the data center operator
finance the installation of a green energy-producing farm, run, owned
and managed by an independent party, to provide green energy to the
data center over a long-term period covered under the PPA.  For both
RECs and PPAs, the authenticity of green energy is, however, often kept
out of sight of the users.  Therefore, the lack of authentication, accountability, and transparency
enables corporations ($\mathcal{A}_1$) to make false claims about the energy source,
while appearing in public to support sustainability efforts.



\noindent \ding{113} \textbf{Lack of integrity of carbon emission
  sources (C2).}
According to the threat model $\mathcal{A}_3$, sensors and devices (\eg PDUs) 
used for tracking
sustainability data can be tampered with by their owners, \ie
untrusted data centers, IaaS providers, or physically co-located $\mathcal{A}_4$ tenants to either
misreport to the regulatory agencies or overcharge the customers.
Such false reporting by $\mathcal{A}_4$ tenants can cause the data center operator to re-adjust
resource allocations/scheduling unnecessarily to adversely affect the data center's
sustainability footprint.
In a similar vein, these sensors and devices
can also become compromised by external attackers ($\mathcal{A}_5$) due to
unintentional vulnerabilities or intended backdoors in their hardware,
firmware, and software\footnote{\href{https://therecord.media/cisa-claroty-highlight-severe-vulnerabilities-in-popular-power-distribution-unit-product}{https://therecord.media/cisa-claroty-highlight-severe-vulnerabilities-in-popular-power-distribution-unit-product}}.
%~\cite{pdu-vulnerabilities}.
As a result, by
taking control of those sensors and devices, attackers may violate the
authenticity and forge carbon footprints to cause over/under-billing to customers by
forging/manipulating carbon consumption records.
Attackers may also generate false sustainability data or manipulate the cooling system
to disrupt sustainability operations\footnote{\href{https://www.datacenterknowledge.com/security/physical-infrastructure-cybersecurity-growing-problem-data-centers}{https://www.datacenterknowledge.com/security/physical-infrastructure-cybersecurity-growing-problem-data-centers}}.
%
Similar kinds of sustainability
data-forgery attacks can also be carried out if there are vulnerabilities in the
communication protocols (\eg lack of authentication and replay protection) between sensors and the
sustainability data aggregators gleaning carbon footprints from
multiple such sensors.
Due to such malicious actions,
additional water and electricity would be required to
cool the targeted data center, resulting in an increased
carbon footprint, higher operational costs, and disruption of
sustainability efforts.




\noindent \ding{113} \textbf{Inadequate access control and information
flow control (C3).}
While resource sharing in data centers offers cost-efficiency,
it requires robust isolation techniques to prevent unauthorized
access to tenant's sensitive data. 
The lack of fine-grained and dynamic access control
(such as Discretionary Access Control (DAC),
Mandatory Access Control (MAC), or combinations thereof), adequate
resource isolation, and information flow-control measures  
may allow attackers ($\mathcal{A}_3$ and $\mathcal{A}_4$) to obtain unauthorized
access to sensitive sustainability data, potentially leading to data breaches, privacy
violations, and other security issues.
Furthermore, sustainability data can also be illegitimately
tampered with by malicious users processes ($\mathcal{A}_6$) or
compromised system processes ($\mathcal{A}_5$).
Malicious processes may
obtain unauthorized (read/write) access to sensitive resources (\eg
databases or protected memory regions storing sustainability data and
states) by exploiting vulnerabilities in the access control
policies\footnote{\href{https://threatpost.com/bug-linux-kernel-privilege-escalation-container-escape/178808/}{https://threatpost.com/bug-linux-kernel-privilege-escalation-container-escape/178808/}}. 
%~\cite{privilege-escalation}.
As a result, regular sustainability
operations are likely to be disrupted, which may cause
the system to produce unwarranted carbon footprints, including a neutral footprint.
Tampering with sustainability data by attackers (\eg malicious
service providers or malicious users) may result in overcharging
legitimate users of the system (such as a data center), undercharging
malicious users attempting to evade sustainability costs, or damaging
the reputation of competing service providers.
Attackers may also induce
carbon-exhaustion attacks on other users by misreporting of carbon consumption
or evade compliance checking of regulatory agencies by misreporting
low carbon emissions when operating in test mode (similar to Volkswagen's
scandal\footnote{\href{https://www.cpajournal.com/2019/07/22/9187/}{https://www.cpajournal.com/2019/07/22/9187/}}).
%~\cite{vwscandal2015}).





\noindent \ding{113} \textbf{Sensitive information disclosure (C4).}
Collecting fine-grained sustainability data from disparate carbon sources
(\eg sensors and PDUs) to monitor and diagnose sustainability activities may also disclose
the sustainability metrics to service providers ($\mathcal{A}_2$) and other users.
Such unauthorized exposure of footprint will violate the privacy of users'
data, location, behavior, and intellectual properties such as
proprietary scheduling techniques, factors used for competitive
pricing for different service categories~\cite{hlavacs2011energy, mckenna2012smart}.
%\ezk{missing ``CITE''?}
Unauthorized access to footprint data
can enable an adversary to initiate DoS attacks ($\mathcal{A}_4$ and $\mathcal{A}_6$)
on the co-tenant and thus prevent co-tenants from realizing an
desired sustainability target.



\noindent \ding{113} \textbf{Cryptographic flaws and software bugs (C5).}
The ability of a sustainable system to provide proof of
carbon footprint to users and regulators is essential for ensuring the
trustworthiness of the system.  Such proof of footprint should be built
with cryptographic constructs.  However, flaws in the integration of
cryptographic constructs with complex data center systems (\eg using
weak cipher suites\footnote{\href{https://threatpost.com/bug-linux-kernel-privilege-escalation-container-escape/178808/}{https://threatpost.com/bug-linux-kernel-privilege-escalation-container-escape/178808/}}\footnote{\href{https://nakedsecurity.sophos.com/2023/01/30/serious-security-the-samba-logon-bug-caused-by-outdated-crypto/}{https://nakedsecurity.sophos.com/2023/01/30/serious-security-the-samba-logon-bug-caused-by-outdated-crypto/}})
%~\cite{ms365-insecure-block-cipher})
%~\cite{ms365-insecure-block-cipher,samba-outdated-crypto})
or flaws in the
software\footnote{\href{https://heartbleed.com/}{https://heartbleed.com/}}
%~\cite{heartbleed}
($\mathcal{A}_5$)
may fail to generate unforgeable and
accurate proof of consumption, enabling an attacker to drop, modify, replay,
and inject fake footprints of carbon.  This can disrupt the
operations of sustainable systems.


\noindent \ding{113} \textbf{Side-channels in sustainability (C6).}
Due to shared hardware resources, co-located tenants' servers, and poor
isolation between different processes running on the same hardware in
data centers, side-channel vulnerabilities~\cite{cloud-side-channel-ristenpart} %~\cite{harnik2010side}
(\eg page faults, %~\cite{xu2015controlled},
cache misses, %~\cite{wang2007new},
power, %~\cite{randolph2020power}
and timing %~\cite{hund2013practical}
channels) may allow a malicious
process ($\mathcal{A}_4$ and $\mathcal{A}_6$) to observe or tamper with carbon footprint
patterns of other users' jobs/applications running on the same
hardware.  Such side channels allow an attacker to not only
fingerprint the data traffic of other users but also extract the
cryptographic keys or other confidential information of a user
application by looking at the use of sustainability
metrics~\cite{cloud-side-channel-ristenpart}.
Attackers ($\mathcal{A}_3$ and $\mathcal{A}_5$) can exploit
such sensitive information to blackmail or embarrass other
users/competitors (\eg to force a competitor's stock to drop, or
short-sell such stock).



\noindent \ding{113} \textbf{Collusion for evasion (C7).}  Infrastructure
providers ($\mathcal{A}_7$) and Power Distribution Unit (PDU)
providers ($\mathcal{A}_3$) may collude to
misreport carbon footprints to regulators and users and thus may evade
regulatory agencies.
Such collusion attacks can be of different
combinations as infrastructure providers depend on third-party
software and hardware vendors which may also collude with each other
for malicious purposes.

\mycolor{black}

%%%%%%%%%%%%%%%%%%%%%%%%%%%%%%%%%%%%%%%%%%%%%%%%%%%%%%%%%%%%%%%%%%%%%%%%%%%%%%
%% For Emacs:
% Local variables:
% fill-column: 70
% End:
%%%%%%%%%%%%%%%%%%%%%%%%%%%%%%%%%%%%%%%%%%%%%%%%%%%%%%%%%%%%%%%%%%%%%%%%%%%%%%
%% For vim:
% vim:textwidth=70
%%%%%%%%%%%%%%%%%%%%%%%%%%%%%%%%%%%%%%%%%%%%%%%%%%%%%%%%%%%%%%%%%%%%%%%%%%%%%%
% LocalWords:  HotCarbon externalities Pigovian Ent Jevons Pigou TODO
% LocalWords:  unforgeable tesla youtube de facto SaaS PaaS IaaS RECs
% LocalWords:  PPAs PPA PDUs PDU
