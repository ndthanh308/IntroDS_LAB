%%%%%%%%%%%%%%%%%%%%%%%%%%%%%%%%%%%%%%%%%%%%%%%%%%%%%%%%%%%%%%%%%%%%%
%% This is a (brief) model paper using the achemso class
%% The document class accepts keyval options, which should include
%% the target journal and optionally the manuscript type. 
%%%%%%%%%%%%%%%%%%%%%%%%%%%%%%%%%%%%%%%%%%%%%%%%%%%%%%%%%%%%%%%%%%%%%
\documentclass[journal=ancac3,manuscript=suppinfo]{achemso}

%%%%%%%%%%%%%%%%%%%%%%%%%%%%%%%%%%%%%%%%%%%%%%%%%%%%%%%%%%%%%%%%%%%%%
%% Place any additional packages needed here.  Only include packages
%% which are essential, to avoid problems later. Do NOT use any
%% packages which require e-TeX (for example etoolbox): the e-TeX
%% extensions are not currently available on the ACS conversion
%% servers.
%%%%%%%%%%%%%%%%%%%%%%%%%%%%%%%%%%%%%%%%%%%%%%%%%%%%%%%%%%%%%%%%%%%%%
\usepackage[version=3]{mhchem} % Formula subscripts using \ce{}
\usepackage{siunitx}
%\usepackage{subfig}
%\usepackage{floatrow}
\usepackage{graphicx}
\usepackage[label font=bf,labelformat=simple]{subfig}
\usepackage{caption}
%\captionsetup[subfigure]{position=bottom}
\usepackage{chemformula}
%\floatsetup[figure]{style=plain,subcapbesideposition=top}

%%%%%%%%%%%%%%%%%%%%%%%%%%%%%%%%%%%%%%%%%%%%%%%%%%%%%%%%%%%%%%%%%%%%%
%% If issues arise when submitting your manuscript, you may want to
%% un-comment the next line.  This provides information on the
%% version of every file you have used.
%%%%%%%%%%%%%%%%%%%%%%%%%%%%%%%%%%%%%%%%%%%%%%%%%%%%%%%%%%%%%%%%%%%%%
%%\listfiles

%%%%%%%%%%%%%%%%%%%%%%%%%%%%%%%%%%%%%%%%%%%%%%%%%%%%%%%%%%%%%%%%%%%%%
%% Place any additional macros here.  Please use \newcommand* where
%% possible, and avoid layout-changing macros (which are not used
%% when typesetting).
%%%%%%%%%%%%%%%%%%%%%%%%%%%%%%%%%%%%%%%%%%%%%%%%%%%%%%%%%%%%%%%%%%%%%
\newcommand*\mycommand[1]{\texttt{\emph{#1}}}
\newcommand*\nc{NC}
\newcommand*\ncs{NCs}
\newcommand*\tonda[1]{\left(#1\right)}
\newcommand*\of[1]{\left(#1\right)}
\newcommand*\quadra[1]{\left[#1\right]}
\newcommand*\graffa[1]{\left\{#1\right\}}
%%%%%%%%%%%%%%%%%%%%%%%%%%%%%%%%%%%%%%%%%%%%%%%%%%%%%%%%%%%%%%%%%%%%%
%% Meta-data block
%% ---------------
%% Each author should be given as a separate \author command.
%%
%% Corresponding authors should have an e-mail given after the author
%% name as an \email command. Phone and fax numbers can be given
%% using \phone and \fax, respectively; this information is optional.
%%
%% The affiliation of authors is given after the authors; each
%% \affiliation command applies to all preceding authors not already
%% assigned an affiliation.
%%
%% The affiliation takes an option argument for the short name.  This
%% will typically be something like "University of Somewhere".
%%
%% The \altaffiliation macro should be used for new address, etc.
%% On the other hand, \alsoaffiliation is used on a per author basis
%% when authors are associated with multiple institutions.
%%%%%%%%%%%%%%%%%%%%%%%%%%%%%%%%%%%%%%%%%%%%%%%%%%%%%%%%%%%%%%%%%%%%%



\author{Marianna D'Amato}
\affiliation[LKB]{Laboratoire Kastler Brossel, Sorbonne Universit\'e, CNRS, ENS-PSL Research University, Coll\`ege de France, 4 place Jussieu, 75252 Paris Cedex 05, France}

\author{Lucien Belzane}
\affiliation[LKB]{Laboratoire Kastler Brossel, Sorbonne Universit\'e, CNRS, ENS-PSL Research University, Coll\`ege de France, 4 place Jussieu, 75252 Paris Cedex 05, France}

\author{Corentin Dabard}
\affiliation[INSP]{Sorbonne Université, CNRS - UMR 7588, Institut des NanoSciences de Paris, INSP, F-75005 Paris, France}

\author{Mathieu Silly}
\affiliation[SOLEIL]{Synchrotron SOLEIL, L'Orme des Merisiers, Départementale 128, 91190 Saint-Aubin, France}


\author{Gilles Patriarche}
\affiliation[C2N]{Centre de Nanosciences et de Nanotechnologies, CNRS, Université Paris-Saclay, C2N, Palaiseau 2110, France}


\author{Quentin Glorieux}
\affiliation{Laboratoire Kastler Brossel, Sorbonne Universit\'e, CNRS, ENS-PSL Research University, Coll\`ege de France, 4 place Jussieu, 75252 Paris Cedex 05, France}

\author{Hanna Le Jeannic}
\affiliation{Laboratoire Kastler Brossel, Sorbonne Universit\'e, CNRS, ENS-PSL Research University, Coll\`ege de France, 4 place Jussieu, 75252 Paris Cedex 05, France}

\author{Emmanuel Lhuillier}
\affiliation[INSP]{Sorbonne Université, CNRS - UMR 7588, Institut des NanoSciences de Paris, INSP, F-75005 Paris, France}

\author{Alberto Bramati}
\affiliation{Laboratoire Kastler Brossel, Sorbonne Universit\'e, CNRS, ENS-PSL Research University, Coll\`ege de France, 4 place Jussieu, 75252 Paris Cedex 05, France}
\email{alberto.bramati@lkb.upmc.fr}



%%%%%%%%%%%%%%%%%%%%%%%%%%%%%%%%%%%%%%%%%%%%%%%%%%%%%%%%%%%%%%%%%%%%%
%% The document title should be given as usual. Some journals require
%% a running title from the author: this should be supplied as an
%% optional argument to \title.
%%%%%%%%%%%%%%%%%%%%%%%%%%%%%%%%%%%%%%%%%%%%%%%%%%%%%%%%%%%%%%%%%%%%%
\title[]
  {Highly photostable Zn-treated halide perovskite nanocrystals for efficient single photon generation
  }

%%%%%%%%%%%%%%%%%%%%%%%%%%%%%%%%%%%%%%%%%%%%%%%%%%%%%%%%%%%%%%%%%%%%%
%% Some journals require a list of abbreviations or keywords to be
%% supplied. These should be set up here, and will be printed after
%% the title and author information, if needed.
%%%%%%%%%%%%%%%%%%%%%%%%%%%%%%%%%%%%%%%%%%%%%%%%%%%%%%%%%%%%%%%%%%%%%
%\abbreviations{IR,NMR,UV}
\keywords{perovskites, single photon sources, quantum dots, nanocrystals}

%%%%%%%%%%%%%%%%%%%%%%%%%%%%%%%%%%%%%%%%%%%%%%%%%%%%%%%%%%%%%%%%%%%%%
%% The manuscript does not need to include \maketitle, which is
%% executed automatically.
%%%%%%%%%%%%%%%%%%%%%%%%%%%%%%%%%%%%%%%%%%%%%%%%%%%%%%%%%%%%%%%%%%%%%
\begin{document}
\newpage

\section{I. Methods}

\subsection{Chemicals}
\ch{PbBr_2} (Alfa Aesar, $98.5\%$), \ch{Cs2CO3} (Alfa aesar, $99,99\%$), oleylamine (OLA, Acros, $80-90\%$), oleic acid (OA, Sigma-Aldrich), octadecene (ODE, Acros Organics, $90\%$), toluene (VWR, rectapur), Bromic acid (HBr, ABCR, $48\%$ aqueous solution), Ethyl acetate (JT baker), zinc diethyldithiocarbamate (ZnDDTC, alfa Aesar $17-19.5 \%$ in weight).

\subsection{Caesium oleate precursor}

We mix in a \SI{50}{\milli \liter} three-neck flask,  \SI{412}{mg} of \ch{Cs_2CO_3}, \SI{20}{\milli \liter} of ODE and \SI{1.25}{\milli \liter} of OA.  The content of the flask is stirred and degased under vacuum at room temperature for \SI{25}{min}. The flask is heated at \SI{110}{\celsius} for \SI{30}{min}. The atmosphere is switched to nitrogen and the temperature raised to \SI{150}{\celsius}. The reaction is carried on for \SI{30}{min}. At this stage, the salt is fully dissolved. The temperature is cooled down below \SI{100}{\celsius} and the flask degased under vacuum. Finally this solution is used as a stock solution.

\subsection{Oleylamonium bromide (\ch{OLABr})}

We mix \SI{10}{\milli \liter}  of OLA with \SI{1}{\milli \liter} of HBr in a \SI{25}{\milli \liter} three-neck flask that is then heated at \SI{80}{\celsius} under vacuum. The atmosphere is then switched to nitrogen and the temperature raised to \SI{120}{\celsius} for \SI{2}{h}. This solution is then transfered to the glove box using air free method and used as stock solution.

\subsection{\ch{CsPbBr_3} \ncs{} synthesis}
In a three neck flask, \SI{280}{mg} of \ch{PbBr2} are mixed with \SI{20}{\milli \liter} of ODE.  The flask is degased under vacuum at room temperature for \SI{15}{min}. Then, the temperature is raised to \SI{110}{\celsius}. Then \SI{2}{\milli \liter} of OLA are injected. Once the vacuum and temperature have recovered, we inject \SI{2}{\milli \liter} of OA. The solution is further degased at \SI{120}{\celsius} for \SI{30}{\min}. The atmosphere is switched to nitrogen and the temperature raised to \SI{180}{\celsius}. Around \SI{2}{\milli \liter} of CsOA solution are injected, and the solution turns yellow. The reaction is conducted for \SI{30}{s} and finally quickly cooled-down by removing the heating mantle and using a water bath. The solution is transferred to plastic tube and centrifuged. The supernatant is discarded. The pellet is dispersed in \SI{5}{\milli \liter} of toluene. The same volume of ethylacetate is added and the mixture is centrifugated again for \SI{5}{min} at \SI{6000}{rpm}. The obtained dried pellet is stored in the air free glove box.

\subsection{Zn shelling}
\SI{30}{\milli g} of \ch{CsPbBr_3} \ncs{} are mixed with \SI{20} {\micro \liter} of \ch{OLABr}, \SI{16}{mg} of ZnDDTC and \SI{3} {\milli \liter} of ODE. The mixture is then sonicated, before being transfered to a \SI{25}{\milli \liter} three-neck flask. The flask is heated at \SI{120}{\celsius}. The duration is tuned from \SI{5} to \SI{60}{min}. The color tends to swicth from green to brown green but PL remains green. Then the mixture is centrifugated. The obtained is redispersed in toluene. The same volume of ethyl acetate is added and the mixture is centrifugated again. The supernatant is discarded and the pellet dispersed in toluene.

 
\subsection{Absorption spectra}
For absorption, we used dilute solution of NC in hexane. The spectra were acquired using a Jasco V730 spectrometer.

\subsection{Transmission electron microscopy}
 A drop of diluted NCs solution was drop-casted onto a copper grid covered with an amorphous carbon film. The grid was degassed overnight under secondary vacuum. Imaging was conducted using a JEOL \SI{2010} transmission electron microscope operated at \SI{200} kV. Complementarily, TEM/STEM observations were made on a Titan Themis \SI{200} microscope (FEI/Thermo Fischer Scientific) equipped with a geometric aberration corrector on the probe. The microscope was also equipped with the "Super-X" systems for EDX analysis with a detection angle of \SI{0.9} steradian. The observations were made at \SI{200} kV with a probe current of about \SI{35} pA and a half-angle of convergence of \SI{17} mrad. HAADF-STEM images were acquired with a camera length of \SI{110} {\milli \meter} (inner/outer collection angles were respectively \SI{69} and \SI{200} mrad).

\subsection{X-ray photoemission measurements (XPS)}
 For photoemission spectroscopy, we used the Tempo beamline at synchrotron Soleil. Films of nanocrystals were spin-casted onto a gold coated Si substrate with an 80 nm tick gold layer. To avoid any charging effect during measurements, the ligands of the nanocrystals were exchanged by dipping the film in ethyl acetate. Samples were introduced in the preparation chamber and degassed until a vacuum below $10^{-9}$ mbar was reached. Then samples were introduced to the analysis chamber. The signal was acquired by a MBS A-1 photoelectron analyzer. Acquisition was done at constant pass energy (\SI{50} eV) within the detector. Photon energy of \SI{150} eV was used for the acquisition of valence band and work function while \SI{600} eV photon energy was used for the analysis of the core levels. A gold substrate was employed to calibrate the Fermi energy. The absolute value of the incoming photon energy was determined by measuring the first and second orders of Au4f core level peaks. Then for a given analyzer pass energy, we measured the Fermi edge and set its binding energy as zero. The same shift was applied to all spectra acquired with the same pass energy. To determine the work function, an \SI{18} V bias was applied, and its exact value was determined by observing the shift of a Fermi edge.
 
\section{II. Material characterization}

\subsection{II.1 The pristine \ch{CsPbBr_3}}

The pristine \ch{CsPbBr_3} NCs grown using Protecescu’s procedure\cite{protesescu2015nanocrystals} depicts an absorption edge at around 520 nm (Figure\ref{fig:S1}a), giving them a yellow aspect (Figure\ref{fig:S1}b) in solution that is associated with a bright green photoluminescence (PL)(Figure\ref{fig:S1}c)). The particles present a cuboid aspect (Figure\ref{fig:S1}c). Some dark spot can be observed that have been associated with the reduction of the \ch{Pb^{2+}} by the electron beam.

In a second step, the grown cubes are exposed to the ZnDDTC precursor. During the reaction, the solution acquires a darker green aspect, but preserves the green PL. The darker aspect is likely resulting from PbS formation which is not emitting light in the spectral range of interest.

% Figure environment removed

\subsection{II.2 Chemical analysis of the Zn-treated \ch{CsPbBr_3} NCs}
X-ray photoemission (XPS) is used to probe the chemical composition of the \ch{CsPbBr_3} \ncs{} after their exposure to the Zn(DDTC) precursor. The Pb 4f state displays two contributions, see Figure\ref{fig:S2}a. The main one, appearing with a 138.7 eV binding energy, is attibuted to the \ch{Pb_2^+} within the perovskite lattice. The smaller contribution appearing at lower binding energy (137 eV) is attributed to metallic \ch{Pb^0}, as already observed on the TEM image (dark spots in Figure\ref{fig:S1}b)).
The Zn presence is confirmed by XPS, see Figure S 2b. The Zn 3d state, through overlapping with Cs 5p states,present a 10.87 eV binding energy.
To further give insights on the effect of Zn exposure, we have performed energy dispersive X-ray spectroscopy in a TEM, see Figure\ref{fig:S3}. Zn content is found to be low ($0.5\%$ atomic ratio) and we see no correlation between its localization and the NC surface, suggesting that Zn mostly come as a dopant. Note that S analysis is complicated by EDX due to the overlap of the S and Pb contributions.

% Figure environment removed
% Figure environment removed
\newpage
\section{III. Experimental set-up}
Figure\ref{fig:S4} illustrates the experimental setup employed for wide-field and confocal microscopy, respectively in ensemble and single particle measurements. To enable effective separation of the Zn-treated \ch{CsPbBr_3} \ncs{} and facilitate confocal microscopy, the solution containing the particles was appropriately diluted in toluene and spin-coated onto a fused-silica substrate. In the wide-field configuration, the substrate placed on the inverted microscope (Nikon Eclipse Ti) was illuminated with a 400 nm LED lamp (CooLED pE-100) and imaged using a CMOS camera (Hamamatsu ORCA-Flash 4.0) to capture videos for stability evaluation and emitter localization. Once the emitters were located, a picosecond pulsed laser (Pico Quant P-C-405B) with a 405 nm output wavelength, pulse width of less than 100 ps, and repetition rate of \SI{2.5}{\mega Hz} was used for excitation. In the confocal setup, the laser beam was focused onto the sample using a 100X oil immersion microscope objective with a numerical aperture of 1.4. The emitted photoluminescence (PL) was collected back through the same objective, spectrally filtered with a dichroic mirror and a long pass filter (cut-off wavelength of 435 nm) to eliminate the excitation laser, and then directed to various detection systems. These detection part included the CMOS camera for saturation measurements,   a spectrometer for steady-state spectroscopy, and a Hanbury Brown-Twiss setup for time-correlated single photon counting (TCSPC) detection. The TCSPC module employed in our setup (Pico Harp 300) enabled us to capture the complete fluorescence dynamics using the Time-Tagged-Time-Recorded (TTTR) method. This involved recording the arrival times of all photons relative to the beginning of the experiment (time tag), along with the picosecond TCSPC timing relative to the excitation pulses. By implementing this technique, we were able to obtain precise temporal information about the fluorescence events, allowing for detailed analysis of the second order correlation function, the photoluminesce decay and the blinking.  
All measurements were conducted at room temperature.
\newpage
% Figure environment removed
\newpage
\section{IV. Noise cleaning on $g^{(2)}(\tau)$}

To accurately perform the $g^{(2)}(\tau)$ measurement, it is necessary to consider how the coincidences are counted. Let $M(\tau)$ represents the number of counts measured at a specific delay $\tau$. Each count can be generated either by a start from the or from the background signal and by a stop signal or from the background. We define $s(\tau)$ as the probability of having a start (or stop) generated by the signal and $b(\tau)$ as the probability of having the start (or stop) generated by the background. When both $b(\tau)$ and $c(\tau)$ are significantly smaller than 1, we can express this relationship as follows: 
\begin{equation}
M(\tau)=C\tonda{b(\tau)+s(\tau)}\tonda{b(\tau)+s(\tau)}
\end{equation}
where C is a constant of proportionality. We define $M'=\frac{M}{C}$, where $M'$ represents the corrected number of counts. Considering a delay $\tau_b$ between two consecutive peaks where there is no signal, we have that $s(\tau_b)=0$ and we can write:  
\begin{equation}
    M'(\tau_b)=b^2(\tau)
\end{equation}
To find $M_c(\tau)=Cs^2(\tau)$, we solve the system of equations and determine that:
\begin{equation}
M_c(\tau)=M(\tau)+M(\tau_b)-2\sqrt{M(\tau)}\sqrt{M(\tau_b)}
\end{equation}
This formula allows us to obtain the $g^{(2)}(\tau)$ histogram, which is cleaned from background counts.

\newpage 
\section{V. Blinking analysis}
In Figure\ref{fig:decayg2} we report the over-all dacay and the proof of 
antibunching relative to intensity time-trace reported in Figure4. 
In Figure\ref{fig:g2far} we show the intensity time-trace, the \ch{g^2} function at large delays and the FLID distribution in the case
of pristine \ch{CsPbBr_3} \nc{}.

% Figure environment removed

% Figure environment removed
\newpage
\section{VI. FLID images}

To generate the Fluorescence Lifetime Intensity Distribution (FLID) images shown in Figure5, we employed kernel density estimation\cite{rosenblatt1956remarks,parzen1962estimation}. Starting with the photoluminescence (PL) time-traces obtained with a binning time of \SI{10} {\milli s}, we calculated the main arrival time for the photons detected within each bin. This resulted in a corresponding lifetime time-trace, as depicted in Figure4c. The upper panel of Figure4c displays the PL intensity time-trace over an enlarged period of a few seconds, while the bottom panel exhibits the corresponding lifetime trajectories. The main arrival time, combined with the mean intensity of the bin, represents a point in the FLID intensity-time space.
% Figure environment removed
In Figure\ref{fig:powertraces} we report the PL time-trace and the FLID images obtained at different excitation power, corresponding respectively to $0.25,0.5,1,1.5,2,3$ times $P_{sat}$.

\newpage

\bibliography{achemso-demo}
\end{document}

