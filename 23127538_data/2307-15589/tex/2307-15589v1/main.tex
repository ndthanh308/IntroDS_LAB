\documentclass[letterpaper, 10pt, twocolumn]{article}
\usepackage{clean}
\shorttitle{Compliant gripper for assembly of electrical connectors}


\usepackage{gensymb}
% Figures
\usepackage{adjustbox}
\usepackage{graphicx}
%\usepackage{parskip} %%% We shouldn't use this package, it adds the gap between paragraphs and removes indents, which is inconsistent with IEEE style
%\usepackage{subfig}
% \captionsetup{font=footnotesize}
\usepackage{subcaption}
\captionsetup{font=footnotesize}
\captionsetup[sub]{font=footnotesize}
\graphicspath{{./figs}}
\usepackage{epsfig}

% Functionality 
\usepackage{amsmath,amssymb,amsfonts}
\usepackage{algorithm2e}
\usepackage{xcolor}
\usepackage[colorlinks=true, linkcolor=black, urlcolor=cyan, filecolor=black, citecolor=black]{hyperref}
\usepackage{url}
\usepackage{amsmath,bm}
\usepackage{multirow, makecell} % For table magic
\usepackage{adjustbox}
\usepackage{comment}

% Revision 
\usepackage{xcolor}   
%\newcommand{\rev}[1]{{\color{red} #1}}     % Uncomment to highlight \rev text
\newcommand{\rev}[1]{{\color{black} #1}}  % Uncomment to make \rev black
% \usepackage[style=ieee]{biblatex} % bibtex can also be used as needed, replace with package `cite`
% \addbibresource{lib2.bib}
\begin{document}

%\title{Compliant finray-effect gripper for high-speed robotic assembly of electrical components}
\title{High-speed electrical connector assembly by structured \\ compliance in a finray-effect gripper}
\author{Richard Matthias Hartisch$^1$ Kevin Haninger$^2$ 
\thanks{\noindent$^1$ Department of Industrial Automation Technology at TU Berlin, Germany. \\ $^2$ Department of Automation at Fraunhofer IPK, Berlin, Germany.  \\ Corresponding author: {\tt r.hartisch@tu-berlin.de}}
\thanks{This project has received funding from the European Union's Horizon 2020 research and innovation programme under grant agreement No 820689 — SHERLOCK and 101058521 — CONVERGING.}}

\maketitle

\begin{abstract}
% Original abstract
%Fine assembly tasks, such as electrical connector insertion, have tight tolerances and sensitive components, limiting the speed and robustness of robot assembly, even when using vision, tactile, or force sensors. Connector insertion is a common industrial task, requiring compensation of horizontal alignment errors while applying sufficient force in the insertion direction. Additional design objectives are the ability to robustly grasp a variety of connectors, compensate pose variation, and allow high-speed in contact. These objectives are partly supported by robotic fingers from soft materials like silicone, which can increase grasp contact area over variation in surface geometry and reduce contact forces, but may not be able to provide the accuracy or assembly forces required. To address these limitations, this paper proposes monolithic fingers with both structured compliance in the finger body and form-closure features at the fingertip. A finray-effect gripper is adapted to realize directionally-dependent stiffness that allows high-speed mechanical search, self-alignment in insertion, and sufficient assembly force. The design of the finray ribs and fingertips are investigated, identifying the stiffness and maximum loading experimentally. The finray structure realizes an effective remote center of compliance (RCC), but this feature is found to be secondary to the resulting tolerance window. The final design achieves plug insertion with a tolerance window of up to $7.5 mm$ and approach speeds of up to $1.3 m/s$. 

Fine assembly tasks such as electrical connector insertion have tight tolerances and sensitive components, \rev{requiring compensation of alignment errors while applying sufficient force in the insertion direction, ideally at high speeds and while grasping a range of components.} Vision, tactile, or force sensors can compensate alignment errors, but have limited bandwidth, limiting the safe assembly speed. \rev{Passive compliance such as silicone-based fingers can reduce collision forces and grasp a range of components, but often cannot provide the accuracy or assembly forces required. To support high-speed mechanical search and self-aligning insertion, this paper proposes monolithic additively manufactured fingers which realize a moderate, structured compliance directly proximal to the gripped object. The geometry of finray-effect fingers are adapted to add form-closure features and realize a directionally-dependent stiffness at the fingertip, with a high stiffness to apply insertion forces and lower transverse stiffness to support alignment. Design parameters and mechanical properties of the fingers are investigated with FEM and empirical studies, analyzing the stiffness, maximum load, and viscoelastic effects. The fingers realize a remote center of compliance, which is shown to depend on the rib angle, and a directional stiffness ratio of $14-36$. The fingers are applied to a plug insertion task, realizing a tolerance window of $7.5$ mm and approach speeds of $1.3$ m/s.}
\end{abstract}

\section{Introduction}
\label{sec:intro}
The installation of cables and wire harnesses has substantial industrial demand, especially as electrification of automobiles and household appliances increases. While the pre-production of cable harnesses (cutting, mounting of wire seals and attachment of cable heads) can be achieved with specialized machinery \cite{trommnau2019overviewneu}, installation is today largely manual work \cite{Yumbla.2020}. 

Cable installation is challenging to automate due to the high variety in connectors \cite{Yumbla.2020} which can lead to small batch sizes \cite{trommnau2019overviewneu}. The manipulation of cables also introduces technical challenges, where cable routing requires perception and planning methods for deformable linear objects \cite{trommnau2019overviewneu, Chen.2016}. However, connector insertion is a critical functionality for all cable installation tasks. It is also challenging: it requires coordinated vision and touch when done by humans \cite{Chen.2016}, and robotic solutions often require a combination of sensors (vision \cite{trommnau2019overviewneu}, tactile \cite{li2014localization,   bhirangi2021reskin, she2021cable}, or force \cite{ Haraguchi.2011}) and application-specific finger design \cite{chapman2021locally, chen2012design, Yumbla.2019}.

% Figure environment removed
% \fi

\rev{The challenges in the gripping, alignment, and insertion of electrical connectors are largely shared by other fine assembly tasks. In both cases, alignment errors can cause failed assembly \cite{Yumbla.2019}, requiring the use of some error compensation methods \cite{li2019c}. Assembly force must be applied to bring together a snap or friction fit without causing jamming \cite{li2019c}. Inserting in obstructed and complex environments often imposes tight space requirements on the gripper, and the need to leave the inserted portion free often requires a fingertip grip \cite{whitney2004mechanicalneu}.}

These challenges can be partly handled by compliance. In insertion, compliance can compensate misalignment between gripped part and socket \cite{whitney2004mechanicalneu, li2019c}.  Compliance can be active or passive \cite{wang1998passive, li2019c}, where passive compliance is the intrinsic mechanical compliance of the physical structure, and active compliance is achieved by feedback controller design. A relevant example for compliance is the remote center of compliance (RCC) \cite{ciblak2003designneu, whitney2004mechanicalneu}, which allows self-alignment in insertion tasks \cite{yun2008a}. Active compliance, such as impedance or admittance control can adapt the relative pose according to contact forces \cite{baksys2017vibratory, li2019c}. 

The major advantage of active compliance is the possibility to digitally change compliance, e.g. adjusting the RCC location to improve performance \cite{wang1998passive} or using general non-diagonal stiffness matrices \cite{oikawa2021}. The disadvantages of active compliance are the relatively high costs and the limited bandwidth \cite{wang1998passive}, which typically leads to higher collision forces. In contrast, passive compliance has \rev{almost} no bandwidth limits, allowing high-speed contact transitions \cite{bicchi2004}. \rev{However, many materials used in soft robotics have viscoelastic effects \cite{shintake2018}, which can affect high-speed performance.} Physical compliance also reduces the positioning accuracy and ability to detect contact \cite{li2019c, haninger2022b}, but supports motion strategies which exploit contact such as mechanical search \cite{hamaya2020} with self-alignment \cite{Yun.2008}. 

\rev{Passive compliance can be integrated at various locations in the robot; the joints \cite{albu-schaffer2008}, flange \cite{whitney2004mechanicalneu}, fingers \cite{shintake2018} or environment \cite{hartisch2022flexure}. By integrating the compliance directly proximal to the contact, the sprung inertia is reduced, reducing inertial forces at high accelerations. The remote center of compliance can be integrated in the robot flange, but then gripper contributes to sprung inertia, typically contributing $>1$ kg of inertia. On the other hand, soft fingers can provide proximal compliance, as well as realizing a larger contact area over variation in the gripped object's surface geometry \cite{shintake2018}.}

\rev{Soft fingers can be realized with silicone \cite{hao2016universal, manti2015bioinspired, park2018hybrid, liu2020two}, often with the objective of universal gripping. However, these silicon-based soft fingers often render a very low stiffness \cite{hernandez2023}, making it difficult to realize the repeatability or assembly forces needed in many fine assembly tasks \cite{li2019c}.  Soft fingers can be constructed by additive manufacturing, which can be used to construct monolithic soft pneumatic fingers \cite{Tawk.2019} or flexure-based grippers \cite{hernandez2023}.}

\rev{Designing passive compliance for assembly is typically done in terms of the spatial stiffness rendered on the gripped object, which affects collision force \cite{bicchi2004} and alignment error compensation \cite{Yun.2008, huang1998}. Modelling of soft fingers is often done with continuum mechanics, Cosserat rod theory or FEM \cite{schegg2022}, which can support kinematic design and control, but often does not consider rendered stiffness. The parallel or series combination of elemental stiffness can be used to analyze rendered stiffness \cite{huang1998, hartisch2022flexure}, but this is difficult to scale to complex soft fingers.} The limited \rev{a priori design methods for} rendered directional compliance leads to a need for iterative testing for a specific task or part \cite{chen2017improved}. However, this iterative testing can be accelerated with fast prototyping methods like fused deposition modelling \cite{hartisch2022flexure, Elgeneidy.2019}, which can also integrate structural passive compliance in monolithic structures with flexures \cite{hernandez2023}. 

This work proposes structured compliant fingers for electrical component assembly, using fused deposition modeling to produce low-cost monolithic devices which can be \rev{easily integrated to parallel grippers}. \rev{Compared with soft fingers which target universal gripping, either silicone-based \cite{hao2016universal, manti2015bioinspired, park2018hybrid, liu2020two} or finray-based \cite{Elgeneidy.2019}, these fingers realize a structured compliance for passive alignment while providing sufficient force in the assembly direction.} Compared with sensorized fingers for plug insertion \cite{wang2021, jiang2022}, the proposed compliance allows a larger tolerance window and higher speed. \rev{In contrast to} plug insertion approaches using active compliance, which can take up to $\approx 6-16$s from first contact to insertion \cite{park2013intuitive}, the passive compliance allows a successful assembly of the connectors in $\approx 1.2$s. This paper also provides a taxonomy and detailed requirements of the connector insertion problem, whereas existing work \cite{chen2012design, Yumbla.2019, trommnau2019overviewneu} focus on wire harness design and production, not connector mating. 

A previous version of this paper was submitted to the 2023 IEEE/ASME International Conference on Advanced Intelligent Mechatronics \cite{hartisch2023compliant}. This paper adds \rev{analysis of the connector assembly problem, FEM and emprical identification of the directional finger stiffness, empirical investigation of viscoelastic effect,} as well as assembly applications with higher speeds \rev{and a wider range of components}. 

The paper is organized as follows. Section \ref{sec:problem_desc} categorizes the parameters of the plugs used in this work, parameters occurring in the assembly task, and describes the steps of the assembly process. Section \ref{sec:design} introduces the final gripper design, derived from the finray-effect, where the design parameters, the design and manufacturing process, and problems are described. A range of applications to verify the gripper's abilities are presented in Section \ref{sec:app}, consisting of repeatability and robustness experiments to determine design parameters which achieve the widest tolerable scale of misalignment. Finally, the conclusion and future work is given in Section \ref{sec:discussion}.

\section{Electrical connector problem description}
\label{sec:problem_desc}

This section analyzes the problem of connector assembly, providing a taxonomy of electrical connectors and the assembly process itself.

\subsection{Taxonomy of connectors \label{sec:taxonomy}}
%\vspace*{2.cm}
\begin{table*}[ht]
\renewcommand{\arraystretch}{1.15} %<- modify value to suit your needs
\begin{center}
 \vspace*{.2cm}
\caption{Identified important properties of connectors and the assembly task, what parts of the robotic solution they influence, and possible values that the property can take \label{tab:conn_parameters}}
\begin{tabular}{r r|l|l} 
& Property & Effects & Possible values \\
\hline
\multirow{5}{*}{\rotatebox[origin=c]{90}{\large Connector}} & Fit and tolerances & Search strategy, req'd assembly force & Press, running, transition \\
& Plug exposed after insert & Grip location in insertion & Flush, $>0$ mm \\
 & Cable gland orientation & Grip location and free space & Straight, right angle \\
& Pin height & Search strategy & Flush, $<0$ mm \\
& Locking feature & Insertion, validation & Clip, lever  \\
\hline
\multirow{5}{*}{\rotatebox[origin=c]{90}{\large Task}} & Plug availability & Grasp strategy, finger design & magazine, on table, cluttered \\
& Socket availability & Tolerances, search strategy & fixed position, in workpiece, free space \\
& Space requirements & Finger dimensions, robot strategy & free space dimensions \\
& Cable handling & Finger design, expected force profile & need insert clips, need to pull cable \\
& Validation & Insert strategy & Is a validation (e.g. push-pull-push) required? \\
\hline

\end{tabular}
\vspace*{-0.5cm}
\end{center}
\end{table*}
While there is large variation in connector design  \cite{Yumbla.2020}, the parameters summarized in Table \ref{tab:conn_parameters} have a significant influence on the finger design and strategies for grasping, searching, or insertion. 

Some parameters are shown visually in Figure \ref{fig:plug_grip}, left, which shows an inserted plug. The amount that the cable head extends from the socket after insertion limits the feasible gripping area. The cable gland can have different orientations, either straight out or in a right angle into the plug, which changes what space must be left free by the finger design. The cable type can be categorized as either a single-, ribbon-, or multi-cable.  Furthermore, the pin height inside the plug and/or socket can result in collision and jamming for some search strategies. Additional locking features, such as levers or clips, may require additional assembly force or post-processing to secure. The tolerances between the plug and socket influence the search strategy and required assembly force.

\subsection{Categorization of assembly task}

There are additional parameters in typical connector assembly tasks which affect the design. How the plug is supplied affects the uncertainty in grip pose, as the plug could either be fixed rigidly, e.g. in a magazine, lying freely on the table or placed in a cluttered environment. Similarly, the socket could be either be in a fixed position, integrated in a workpiece or in a magazine. Further, space limits from the environment can limit both the finger dimensions and robot search strategy. During assembly, additional cable and wire handling has to be considered, e.g. intermediate clips or cable straightening. After successfully mating the components additional testing could be necessary, e.g. push-pull-push of the cable. 

\subsection{Grip, search, and insert strategies}
\label{subsec:strategies}
The complete assembly process is considered in three stages: grip, search, and insert. From an initial position, the plug is gripped. Without a magazine or jig providing a constant and known pose of the plug, a known pose or at least the orientation of the plug inside the grip should be established within the tolerances of the insertion process. 
% Figure environment removed

The grasping strategy includes aspects of the finger design seen in the right of Fig. \ref{fig:plug_grip}: (i) what contact area between the plug and finger can be used. Multiple grasping strategies are possible here, either a pinch contact or parallel grasp of the wires or the cable head, (ii) what space around the connector must be kept free, (iii) are locating features needed to provide either repeatable position or sufficient assembly force? In addition to the fingertip design, the grasping strategy may include (i) a magazine for providing the plug in a semi-repeatable way and (ii) an adjustment strategy to ensure the connector is in a repeatable position in the fingers.  

The search strategy should achieve alignment of the plug and socket. When using mechanical search, the search strategy should be designed considering: (i) the variation in pose that needs to be covered with the strategy, (ii) the initial contact between plug and socket, which can be a point, line or planar contact depending on how the plug is presented, (iii) the height of the pins, which could be bent if contacted by the tip of the plug, (iv) validating that the plug has successfully slipped into the socket after the alignment.  

The mechanical search strategy used here is shown in Fig. \ref{fig:working principle}.  Initial contact is made with a tilted plug, such that the corner of the plug lightly presses on the edge of the connector, Fig. \ref{subfig:key-a}. A motion in the x-direction allows the plug to slip into the socket when aligned. Next, contact is established with the sides of plug and socket, realized by a motion in the y-direction, Fig. \ref{subfig:key-b}. At this point, the leading corner of the plug should be slightly inserted and resting on the edge of the socket. \rev{The search strategy used in this work utilizes an open-loop position control, without the use of force feedback.}

For the insertion phase following aspects should be considered: (i) the finger design should be able to avoid jamming of the connectors. For this, compliance, either active or passive, could be suitable, where the plug is able to rotate inside the socket due to the contact. (ii) The assembly force should not exceed a certain threshold, to avoid damaging the parts, which can also be realized by the finger compliance.

\section{Design of Compliant Finray-effect Grippers}
\label{sec:design}

In this section, we describe the modification and parameterization of a finray-effect gripper \cite{crooks2016fin} to \rev{meet the requirements of Section \ref{sec:taxonomy}}. Where classical finray-effect grippers \rev{use an enveloping grip where} deformation adapts to variation in surface geometry of grasped parts \cite{Elgeneidy.2019}, we propose a design where the \rev{components are grasped at the fingertip and} rendered compliance on a gripped part \rev{supports mechanical search, passive alignment, while providing sufficient force in the assembly direction}. 

Considering the requirements and constraints in Sec. \ref{sec:problem_desc}, the finray design is adapted with the flexibility of 3D printing to optimize the following design parameters, seen in Figure \ref{finger_design_rotated}: (i) fingertip design, (ii) rib angle / infill direction, (iii) rib density / infill density, (iv) finger mounting angle. 

\subsection{Modified Finray Design}

The finray-effect gripper mimics the deformation of fish fins, which are composed by two outer walls forming a V shape. Between the bones, crossbeams are placed which determine the mechanical properties of the finray-effect gripper. 
The side walls of the standard finray-effect gripper bend from contact when grasping convex parts, which results in a deformation of the base and tip towards the applied force \cite{crooks2016fin}. However, the standard V-shaped finray design is not ideal for the requirements here. Here, an object should be grasped at the fingertip, capable of applying high forces in the assembly direction and offering a lower stiffness laterally to compensate misalignment. 

When a flat object is grasped with a V-shaped fingertip, the assembly forces are carried only by friction, which may be insufficient for small components with small allowed grasping contact areas. A form-closure fingertip can therefore support higher assembly forces. Furthermore, to compensate misalignment for parts grasped at the fingertip: a V-shaped fingertip is stiff in the horizontal direction, and when the finger deforms, rotation of the fingertip occurs, which could result in contact loss with the gripped part. Here, a translational deflection at the fingertip is desired to compensate misalignment while maintaining the contact between gripper and grasped part. 

\rev{The V-shape finger profile is changed, instead of the outer walls approaching another towards the tip, the walls are more rectangular. With the fins, this produces a parallelogram structure which does not rotate the tip as it is displaced. Additionally, form-closure features are added at the fingertips, a shoulder which is discussed in the following Section \ref{subsec:design}. Due to the design of the notched fingertip, the generalizability to handle various connectors is limited, but the ability to apply forces over small contact areas is improved.}

\rev{With this fingertip, a directionally-dependent stiffness can be achieved, that is the effective stiffness between gripped object and robot is higher in the assembly direction, and lower in the orthogonal directions to support mechanical search and alignment compensation.}
%\rev{Hence, self-alignment of the grasped part is achieved by the combination of the search strategy achieving an optimal first alignment of the parts and the directionally-dependent stiffness allowing the compensation of position deviations.}

% From the following paragraph, extract the monolithic + formschluss aspects

% Regarding the requirements and constraints in Sec. \ref{sec:reqconst}, the monolithic compliant gripper can be built to successfully work in narrow environments. This is done without any control methods besides the collision detection from the UR. In future design iterations, a notched fingertip is used to achieve the necessary $15 N$ of assembly force, by form-fitting the various forms and geometries of plugs, which where demonstrated in Fig. \ref{Cables}, inside of the grip, achieving an effectively higher stiffness in the assembly direction. With the compliance realized by this finray-effect gripper, the tolerances described before can be compensated with a certain deflection of the fingers resulting from external forces from contact between the connectors. Using search strategies, described later in Sec. \ref{sec:repeatability_magazine}, allows the gripper to successfully assemble the parts in case of misalignment. Control of the object's pose inside the grip can be assumed to be negligible when combined with a magazine, as described in Sec. \ref{sec:testenv}.


% As seen in Fig. \ref{bending_v} and Fig. \ref{0deg_v}, 


% visualized in Fig. \ref{0deg_roundtip}. This has shown the best results in early tests, however, variations in the form of the fingertips are tested over the following iterations. 

\subsection{Design Parameters}  \label{subsec:design}
Two important parameters of the finger design are optimized to improve performance. 

\subsubsection{Infill Options}
The most important design parameters are the infill options to adjust the density and orientation of the ribs in the finger, i.e. the infill direction, given in $\degree$, and the infill density options, given in $ \%$, as proposed by \cite{Elgeneidy.2019} and visualized in Fig. \ref{finger_design_rotated}. This affects the bulk stiffness realized by the finger on a gripped part, \rev{effective remote center of rotation,} as well as the maximum force that the finger can apply\rev{, as shown in Section \ref{sec:single_finger_stiffness}}.

\subsubsection{Fingertip Options}
\label{subsub:fingertip}
An additional parameter is the form of the fingers which can either be with a rounded top, flat top, notched rounded top, flat angled or notched top with a contact plane, visualized in Fig. \ref{finger_design_rotated}. The notch can be rotated by a certain degree, corresponding to the mounting angle of the finger, 
% inclination of the mount 
used to achieve a parallel contact plane with the grasped part. The size of the notch depends on the connector to be handled, %which could limit the target in developing a gripper able to handle a broad variety of cables with a form-fit connection. However, introducing a notch limits the range of cables to be handled. The main reason for this is that 
an oversized notch won't contact the plug or will interfere with insertion,%. This results in an unstable grip and therefore possible slip. Additionally, if the notch is too big, thus nearly enveloping the cable head, there might not be enough space to insert the connector into the plug. However, if 
and an undersized notch has a smaller contact surface and could result in an unstable grip. 

\rev{While adding a notched fingertip results in form-fit contact, this limits the generalizability of connectors to be handled, which is why both characteristics have to be considered during the design phase. A notched fingertip could result in an ideal solution for one connector type, but could result in an unstable grip for other connectors.}

The contact surface also needs consideration: the friction of PLA+ and PETG are low, which is why additionally a thin rubber layer is applied after printing, which can be seen in Fig. \ref{fig:robot_zoom}. \rev{It is important to note, that the rubber layer consists of one adhesive layer to adhere on the contact plane of the finger. The side of the rubber tape in contact with the grasped part doesn't have an additional adhesive applied to it, only increasing the friction coefficient.}
% Multiple grasping modes are possible here, either aspiring a pinch contact of the wires or the cable head, or a parallel grasp of either. The pinch contact could be varied to either achieve a point or line contact with the corresponding part, or to achieve a planar contact.

% To compensate misalignment parallel to the moving direction of the gripper's jaws, structured compliance in the base-y-direction is desired. However, unintentional DoFs, as a rotation about the base-y-axis or the base-x-axis resulting in a change of pose of the grasped part resulting from contact forces, are possible. 
% The coordinate system is visualized in Fig. \ref{subfig:key-a}.

\iffalse % Commented out Kevin 22.07.23, as the RCC was not seen visually and we're preliminarily dropping this argument.
\rev{Additionally, as seen in Fig. \ref{fig:rcc_effect}, the finray structure of the fingers allow for an effective remote center of compliance (RCC), in that a lateral translation at the base of the plug results in a change in orientation with the tip moving towards the perturbation.

% Figure environment removed
}
\fi

\subsection{Manufacturing Process}
To allow for an easy adaptation of infill density and line directions, these parameters are set directly in the slicer program instead of CAD. Here \textit{Ultimaker Cura} is used, applying a method similar to the method used in \cite{Elgeneidy.2019}. The materials used in this work are orange PLA+  
% from \textit{Zhuhai Sunlu Industrial Co.,Ltd.} 
% \cite{Sunlu}
and black PETG. 
% from \textit{Verbatim GmbH} 
% \cite{verbatim_PETG}, 
% referred to as PETG.
\rev{ Other materials, such as TPU and ABS were found to be unsuitable. Tests indicated TPU's inherent compliance interferes with the effects of structured compliance, e.g. the material would be too compliant to provide the necessary stiffness in assembly direction. ABS has a tendency to warp during printing, which proved to be a major issue with the thin walls of this finger.}

% Original, before Kevin edit 22.07
%To achieve an easy adjustment of these parameters, first, the finger is designed as a solid in CAD as proposed by \cite{Elgeneidy.2019}. The part is exported as an .stl - file and loaded into \textit{Cura}. Here the parameters for the gripper can be set. Inspired by \cite{Elgeneidy.2019}, the infill type is set to lines in \textit{Cura} and the option to connect infill lines is turned off. To allow compliance of the skin of the finger, the wall line count has to be set to one, with a line width close to the nozzle diameter of $0.4 mm$, slightly deviating from the recommended $2x$ nozzle diameter from \cite{Elgeneidy.2019}. However, the line width of the infill is also set to $0.4 mm$. The top and bottom layers are removed to fully achieve compliance through the ribs. 
% In the first tests, infill densities of $10\%, 20\%$ and $30\%$ are used, with the direction of the infill lines varying in $10\degree$ steps from $0\degree$ to $40 \degree$. 
%With these settings the connection between finger and mount to the gripper would also be manufactured the same way, with a high level of compliance and flexibility which is suboptimal for a connection withstanding the applied contact forces of the fingers. For the lower connecting section of the finger, different slicing settings have to be used where \textit{Cura's} ``support blocker" feature is applied, as visualized in Fig \ref{support_blocker}.
%The support blocker allows dividing the two sections of the finger to change selected slicing parameters. For example, other than for the section of the fingers with the ribs, top and bottom layers are needed here. With the option "Per Model Settings" and "Modify Settings for overlaps" the wall/top/bottom thickness, wall line count and top/bottom layers can be changed individually for the section within the support blocker. This allows the part to be manufactured with individual settings, as seen in Fig. \ref{sliced}. The .stl/.stp/.ipt files used here are available at \url{https://github.com/richardhartisch/compliantfinray}.  

\rev{
A CAD model of the solid finger is sliced  in \textit{Cura}, the infill type set to unconnected lines of width $0.4$ mm, the wall line count set to one with a line width close to the nozzle diameter of $0.4$ mm (less than the recommended $2\times$ nozzle diameter of \cite{Elgeneidy.2019}). To print the finger mount as a solid, \textit{Cura's} ``support blocker" feature is applied, as visualized in Fig \ref{support_blocker}. The .stl/.stp/.ipt files and further details on the manufacturing are available at \url{https://github.com/richardhartisch/compliantfinray}.}

% % Figure environment removed

% Figure environment removed


% %\iffalse
% \subsection{Failures and Problems}
% \label{subsec:failures}
% Following, four different noted failure cases are described. The structure can buckle, break or plastically deform due to excessive contact or gripping forces, as seen in \ref{plastic_failure}. Additionally, the cable head can move inside the grip due to a lack of adhesion or an insufficient contact force. Plastic deformations or breaking appear due to an excessive load on the fingers while executing the assembly step because of unwanted contact with one of the walls of the socket when plug and socket are not correctly aligned, as seen in \ref{plastic_failure_2}. 
% The buckling phenomenon is assumed to be caused by a too steep infill direction combined with a too low infill density which are also discussed later. 
% Using fused deposition modeling also comes with trade-offs regarding manufacturing errors, as seen in Fig. \ref{manufacturing_failures}. Especially PETG has proven to be difficult to print, as the issues in Fig. \ref{gaps} and Fig. \ref{seperated_layers} demonstrate. Gaps could result from an under-extrusion \cite{simplify3d_underextrusion}, which could be corrected by setting the extrusion multiplier in \textit{Cura}. The separated layers are assumed because of an insufficient hotend temperature, according to \cite{simplify3d_separation}. As seen in Fig. \ref{stringing} and \ref{plastic_failure}, stringing also occurs, especially with PLA+. Stringing usually occurs when the hotend continues extruding material during movement where no extrusion is intended, as explained by \cite{Elgeneidy.2019}. Changing the retraction settings, e.g. by increasing the retraction distance or the retraction speed, could correct the stringing
% %, as stated by \cite{simplify3d_stringing}.  


% \subsection{Optimal Design}

% In conclusion, the optimal finger design has a notched fingertip. Since the PLA+ used for the fingers has very low friction a layer of adhesive tape is added to the contact area. The size of the notch depends on the cable to be handled, which could limit the target in developing a gripper able to handle a broad variety of cables with a form-fit connection. Further discussion on this is done later in Sec. \ref{sec:repeatability_magazine}. The infill density and infill direction are the major design parameters set in \textit{Cura} to further define the finger's attributes next to the used materials. 

\section{Mechanical Attributes}

In the following section, the mechanical properties of the finray-effect gripper are investigated: the stiffness of a single finger, viscoelastic effects, the maximum force and deflection until component failure is reached. The observations made here shall provide a good overview of the influence of the design parameters on the finger behaviour, providing guidance to the choice of finger parameters given the range of motion and forces required for a new application. The setups used can be seen in Fig. \ref{fig:exp_setup} \rev{and Fig. \ref{fig:stiffness_setup}}.

%Determining the mechanical attributes, i.e. the stiffness of the various combinations would seem to be an unnecessary factor in the decision making process on which finger combination would be most suited, as different combinations can be easily and quickly manufactured and tested for suitability depending on the task. However, the stiffness, maximum tolerable force and maximum deflection values could be implemented in process-controlling to verify a successful assembly. In the insertion-phase, the force sensor would determine a force in the y- direction corresponding to the stiffness value of the individual finger.  

%The obtained measurement values are saved in a rosbag file for each individual run and evaluated. 
%The experiments measure displacement and force for varying finger parameters.  The force here is applied by the red \textit{Comau Racer} robot by the edge of the pneumatic \textit{Schunk} gripper since it provides a $90 \degree$ edge which fits the notches of the fingers \rev{as seen in Fig. \ref{fig:stiffness_setup}}. The used fingers and materials are listed in Tab. \ref{tab:fingers_single_stiffness}. 

% and the fingers are mounted on the \textit{Weiss} gripper module of the previously used \textit{UR5}, as visualized in Fig. \ref{single_mechexp}. 

% Force is applied by moving the tool on the \textit{Comau} robot in z-direction. To characterize the stiffness, a movement in the elastic range is realized, usually $\approx 10-15 mm$. 
%For the maximum force the tool is moved until a component failure occurs. 
% The solid finger is a finger that is printed with 6 top and 6 bottom layers, 3 walls and a gyroid infill with a density of 20 \%. % Following, Sec. \ref{sec:single_finger_stiffness} discusses the results from the single finger stiffness, sec. \ref{sec:max_force} discusses the maximum tolerable force and deflection. 

% % Figure environment removed

% % Figure environment removed

% % Figure environment removed
% \label{fig:exp_setup}


% Figure environment removed

% Figure environment removed

\subsection{Single Finger Stiffness}
\label{sec:single_finger_stiffness}
To measure the stiffness of a single finger, the edge of the \textit{Schunk} gripper module is moved downwards onto the contact surface of the finger, resulting in a normal force on the contact surface, as seen in Fig. \ref{fig:stiffness_setup}. The resulting force is measured via the F/T sensor seen in Fig. \ref{fig:robot_left}. To characterize the stiffness, movements in the $y$ and $z$ direction within the elastic range are applied, usually $10-15$ mm. \rev{The stiffness is identified with a least-squares fit for a linear stiffness model.}

\subsubsection{Stiffness in $y$}
To evaluate the stiffness of the experimental setup (finger fixture, robotic systems, robot pedestal), a solid finger is printed with 6 top layers, 6 bottom layers, 3 walls and a gyroid infill with a density of $20\%$. Its stiffness in $y$ is identified as $8.20$ N/mm, a factor of 4 higher than the compliant fingers. 

% Figure environment removed   

% Figure environment removed


% In Fig. \ref{10infsingle} the fingers with varying infill direction over $10\%$ infill density are compared. Due to some joint clearance of the \textit{UR5} when the brakes are engaged, the contact plane is slightly above the other ones, which is why the initial position in the diagram is slightly shifted in the positive z-direction. 
 Fig. \ref{fig:stiffness_40deg} shows the stiffness of $40 \degree$ infill direction over various infill density values ($10 \%$ to $30 \%$). Here, hysteresis is strongly noticeable on lower infill densities due to plastic deformations of the finger, i.e. buckling, which can be seen in Fig. \ref{fig:stiffness_40deg}. An increase of stiffness with infill density is shown, where the maximum value is reached at $40 \degree$ infill direction and $30\%$ infill density, the maximum values tested. 

\rev{Further tests of linear stiffness in $y$ for other materials and infill densities} are listed in Tab. \ref{tab:fingers_single_stiffness}. Both $10\%$ and $15\%$ infill density are insufficient values for an infill direction of $40 \degree$ resulting in buckling, leading to low stiffness values, as seen in Tab. \ref{tab:fingers_single_stiffness}. Additionally, finger stiffness is more sensitive to infill density at lower infill directions, as seen in Tab. \ref{tab:fingers_single_stiffness}. 
% Fig. \ref{Mech_Infill Direction} displays a further compensation of the infill density over a fixed infill direction. 
As seen in Tab. \ref{tab:fingers_single_stiffness}, for the same parameters the PETG fingers achieve a lower stiffness compared to the PLA+ fingers, by roughly a factor of $2$. The same trend in increasing stiffness over infill density is seen. 

\subsubsection{Directional stiffness}
\rev{The remaining translational stiffnesses are measured on the setup of Fig. \ref{fig:robot_left} as
\begin{equation}
    \begin{bmatrix} F_x \\ F_y \\ F_z \end{bmatrix} = \begin{bmatrix} K_{xx}  & - & - \\ - & K_{yy} & K_{zy} \\ - & -K_{zy} & K_{zz} \end{bmatrix} \begin{bmatrix} \delta_x \\ \delta_y \\ \delta_z \end{bmatrix} \label{eq:stiff}
\end{equation}
where $F_\cdot$ and $\delta_\cdot$ are the forces and displacements in the coordinate system shown on Fig. \ref{fig:robot_zoom} and $K_{ab}$ the stiffness coupling a displacement in $a$ to a force in $b$. Stiffnesses $K_{xz}$ and $K_{xy}$ are assumed to be zero, with the only off-diagonal coupling within the grip plane. $K_{xx}$ is measured at $2.9$ N/mm for a $10\%$, $0\degree$ PLA+ finger, measured with the fingertip fixed. The stiffnesses $K_{yy}$, $K_{yz}$ and $K_{zz}$ are affected by the infill direction as can be seen in Fig. \ref{fig:stiffness_10deg}, where $K_{yz}\in[1.4,2.3]$ and $K_{zz}\in[33.7, 52.4]$ N/mm. 

This gives a stiffness ratio between assembly direction and transverse of $>13.7$, realizing a large difference in stiffness between directions. Additionally, the presence of $K_{yz}$ indicates remote center of compliance (RCC) effects \cite{huang1998, li2019c}. Diagonalizing the $K$ of \eqref{eq:stiff} yields the principle axis of the rendered stiffness \cite{huang1998}, which takes a direction of $3.6$, $9.5$ and $14.6\degree$ from $z$ for $0$, $10$, and $20\degree$ infill, respectively. This shows that the center of the RCC varies with the rib angles, where the lower infill directions provide an RCC farther from the gripper location.
}

%As Fig. \ref{fig:stiffness_10deg} indicates, the stiffness shows a slight non-linear behaviour. Here, first signs of hysteresis and therefore fatigue are noticeable for almost every finger configuration. 
% This could be because the movement in negative z-direction of the \textit{Comau} robot is not big enough to induce plastic behaviour.  
% Fig. \ref{40degsingle} clearly demonstrates the threshold of the instable behaviour for 40° at 20 \% infill density compared to higher infill density values.  


\subsubsection{Viscoelastic effects}
\rev{ Velocity-dependent effects were checked by loading a $4$ mm displacement in the $y$ direction with a variety of velocities, ranging from $2$ to $15$ mm/s. A simple viscoelastic model was fit as $F_y = K_{xx}\delta_x + B_{xx}\dot{\delta}_x$, where $B_{xx}$ is the viscous term.  Fitting the terms on a $10\%$, $0\degree$ PLA+ finger found $K_{xx}=1.45$ N/mm and $B_{xx}=0.055$ Ns/mm, indicating at high speeds, e.g. $>100$ mm/s, the viscous terms will exceed $5$ N. }

\begin{table}[h!]
  \centering
  \caption{The results of the single finger stiffness experiments, stiffness in the y-direction is measured in $N/mm$. The stiffness of a solid PLA+ finger is $8.2 N/mm$.}
  \label{tab:fingers_single_stiffness}

  \begin{adjustbox}{width=\textwidth/2}

\begin{tabular}{|l||c|c|c|c|c|}
\hline
Infill    & $10\%$ infill  &  $15\%$ infill  & $20\%$ infill & $25\%$ infill & $30\%$ infill  \\
Dir. &   &    &  &  &   \\
$[\degree]$ & $[N/mm]$ & $[N/mm]$ & $[N/mm]$ & $[N/mm]$ & $[N/mm]$  \\

\hline
\hline

0  & PETG: 0.50   & - &  PETG: 1.133 & PLA+: 2.20 & PETG: 1.60   \\
    &   PLA+ : 1.2 &    &   PLA+: 1.95 &    &  PLA+: 3.00  \\
\hline
10 & PLA+: 1.05 & - & PLA+: 2.10 & - & PLA+: 3.00 \\
\hline
20 & PLA+: 1.533 & - & PLA+: 2.60 & - & PLA+: 3.2 \\
\hline
30 & PLA+: 1.667 & PLA+: 3.00 & PLA+: 3.00 & PLA+: 3.20 & PLA+: 3.70 \\
\hline
40 & PLA+: 0.85 & PLA+: 1.22 & PLA+: 3.4 & PLA+: 3.6 & PLA+: 3.84 \\
\hline

\hline

\end{tabular}

  \end{adjustbox}
\vspace{-0.5cm}
\end{table}

\subsection{Ultimate Strength and Maximum Deflection}
\label{sec:max_force}
In the following, the ultimate strength for each finger is determined and listed in Tab. \ref{tab:fingers_single_force}, under which the component fails, either by breaking of the outer walls or the inner ribs or by buckling, usually of the outer walls. For this, the the tool is moved in the z-direction until a component failure occurs. The maximum deflection until component failure occurs is also listed in Tab \ref{tab:fingers_single_maxdeflection}. 
% In Fig. \ref{0degbreak}, \ref{25infbreak} and \ref{30infbreak} initially two separated slopes are noticeable for 0° infill direction and 25\% and 30 \% infill density. This is attributable to a false initial contact between the Schunk module and the contact plane of the finger.  In an early stage the module made contact with the upper edge of the notch, instead of the contact plane which is why an early slope is noticeable in the diagrams at a higher position value compared to the other variants. After a few $mm$ the module slipped from the edge into the contact plane, explaining the false initial slope.  
Tab. \ref{tab:fingers_single_stiffness} and Tab. \ref{tab:fingers_single_maxdeflection} display a trend that the maximum achievable deflection reduces with an increasing infill direction. The PETG fingers, which have a comparably low stiffness compared to the PLA+ fingers, achieve a comparable maximum deflection but lower maximum force due to their stiffness.
Tab. \ref{tab:fingers_single_maxdeflection} shows that increasing infill direction results in a lower maximum tolerable force. However, increasing the infill density results in a higher maximum tolerable force and as Tab. \ref{tab:fingers_single_maxdeflection} demonstrates, also increases maximum deflection. The maximum tolerable force and the largest deflection occur for 0° infill direction with $30\%$ infill density. 

Additionally, buckling was noticed in the single finger stiffness experiments earlier, as mentioned in Sec.  \ref{sec:single_finger_stiffness}, where the threshold of the unstable behaviour for $40 \degree$ at $20 \%$ infill density is demonstrated compared to higher infill density values.  



\begin{table}[h!]
  \centering
  \caption{The maximum tolerable force for various finger design parameters is demonstrated until mechanical failure occurs}
  \label{tab:fingers_single_force}

  \begin{adjustbox}{width=0.98\textwidth/2}

\begin{tabular}{|l||c|c|c|c|c|}
\hline
Infill  & 10\% infill  &  15\% infill  & 20\% infill & 25\% infill & 30\% infill   \\
Direction &  &    &  &  &    \\
$[\degree]$ & $[N]$ & $[N]$ & $[N]$ & $[N]$ & $[N]$  \\

\hline
\hline

0  & PETG: 12.50  & - &  PETG: 26.50 & PLA+: 56.50 & PETG: 46.00 \\
  &  PLA+ : 26.00  &  &   PLA+: 40.50 &  &  PLA+: 72.50  \\
\hline
10 & PLA+: 26.25 & - & PLA+: 45.00 & - & PLA+: 69.00 \\
\hline
20 & PLA+: 20.50 & - & PLA+: 41.50 & - & PLA+: 55.50\\
\hline
30 & PLA+: 16.50 & PLA+: 23.00 & PLA+: 34.00 & PLA+: 39.00 & PLA+: 51.00\\
\hline
40 & PLA+: 15.75 & PLA+: 17.00 & PLA+: 32.00 & PLA+: 38.50 & PLA+: 43.00 \\
\hline

\hline

\end{tabular}

  \end{adjustbox}
 \vspace{-0.5cm}
\end{table}


\begin{table}[h!]
  \centering
  \caption{The maximum deflection is shown as the deflection when mechanical failure occurs}
  \label{tab:fingers_single_maxdeflection}

  \begin{adjustbox}{width=0.95\textwidth/2}

\begin{tabular}{|l||c|c|c|c|c|}
\hline
Infill  & 10\% infill  &  15\% infill  & 20\% infill & 25\% infill & 30\% infill   \\
Direction &  &    &  &  &    \\
$[\degree]$ & $mm$ & $mm$ & $mm$ & $mm$& $mm$  \\

\hline
\hline

0  & PETG: 24.00  & - &  PETG: 22.50 & PLA+: 28.00 & PETG: 27.00  \\
  &  PLA+ : 26.75  &  &   PLA+: 27.00 & PLA+: 28.00 &  PLA+: 28.5  \\
\hline
10 & PLA+: 21.75 & - & PLA+: 23.00 & - & PLA+: 26.00 \\
\hline
20 & PLA+: 17.00 & - & PLA+: 19.00 & - & PLA+: 21.00\\
\hline
30 & PLA+: 13.75 & PLA+: 13.50 & PLA+: 15.50 & PLA+: 15.50 & PLA+: 16.50\\
\hline
40 & PLA+: 10.50 & PLA+: 10.00 & PLA+: 11.50 & PLA+: 13.50 & PLA+: 14.00 \\
\hline

\hline

\end{tabular}

  \end{adjustbox}
\vspace{-0.5cm}
\end{table}

\subsection{FEA}

\rev{To iterate a design quickly, an efficient evaluation method is needed. Printing the proposed fingers takes roughly an hour, which can support quick iteration. However, especially in an early design phase, an Finite Elements Analysis (FEA) can provide faster evaluation of mechanical properties. This section performs FEA for the fingers, and compares to the values of the single finger stiffness in Tab. \ref{tab:fingers_single_stiffness}.  
For the FEA, a static stress study is done in Autodesk Fusion 360, where it is assumed that the material (Sunlu PLA+ and Verbatim PETG) is isotropic. This is inaccurate due to the working principle of the extruder, where material is dispensed as a line with which the part is built layer-wise, resulting in a non-isotropic, directional material behaviour \cite{Xiao.2021}, but a needed assumption for feasibility. Additionally, approximations in the mesh process may impact the results due to the thin walls.}

\rev{Verbatim provides a technical data sheet with material properties determined according to various norms \cite{verbatim_PETG}, with a tensile strength of: $50$MPa, the Young's modulus with: $2050$MPa, and the density of $1.27$g/cm$^3$. 
% However, there is no information regarding the Poisson's ratio, which is why this value is set to zero. The same applies to the damping coefficient. The thermal values have been set as low as possible which are assumed to not have any influence in the static stress analysis. With these values, the static stress simulation can be done for the fingers made from PETG. 
Since Sunlu does not provide any mechanical properties, an assumption has to be made, where an already existing material (Profile: "Printed PolyTerra PLA Plastic") from an external database \cite{Graves2021} is used to model the variants, as the results of pre-tests have shown a high level of conformity with the actual material behavior. The values given are: density: $1,14$g/cm$^3$, the Young's modulus with: $1900$MPa, the Poisson's ratio with: $0.36$, the yield strength with: $20.04$MPa and the ultimate tensile strength with: $20.9$MPa}

% Before change Richard 240723

% Important material properties of PLA+, as the Young's Module, tensile strength and the Poisson's ratio, are not provided by the manufacturer. Verbatim provides a technical data sheet with material properties determined according to various norms \cite{verbatim_PETG}.  


% In the data sheet, the tensile modulus in $MPa$, the tensile strength at yield in $MPa$, which is assumed to be the yield strength, and the specific gravity or relative density \cite{helmenstine_2019} in $g/cm^3$ are given. However, there is no information regarding the Poisson's ratio, which is why this value is set to zero. The same applies to the damping coefficient and the tensile strength. The thermal values have been set as low as possible which are assumed to not have any influence in the static stress analysis. With these values, the static stress simulation can be done for the fingers made from PETG.
% In contrast, Sunlu does not provide any technical data sheet for PLA+, hence an already existing database is used to analyse the PLA+ fingers, provided by \cite{Graves2021}. For PLA, the author has used a combination of two different publications (\hspace{1sp}\cite{Torres.2015, travieso2019mechanical}) and one material data sheet (\hspace{1sp}\cite{PolyMaxPLA}), combined into the profile "Printed PLA Plastic" \cite{Graves2021}. \cite{Torres.2015} and its mechanical properties of PLA are further investigated as follows: the given density ($1,24 g/cm^3$), Young's modulus ($3500 MPa$), shear modulus ($1287 MPa$), Poisson's ratio ($0.36$) and the yield - and tensile strength ($70 MPa$, $73 MPa$) according to Tab. 1 from \cite{Torres.2015} are implemented into the material library in \textit{Fusion 360}. The behaviour is tested by loading the contact plane of the finger with 0° infill direction and 10\% infill density with $1.2 N$, which should lead to a displacement of $1 mm$, using the inverse of the determined stiffness in Tab. \ref{tab:fingers_single_stiffness}. The resulting max. deflection of the finger at the contact plane is $\approx 0.5149 mm$, with which the material settings appear to be insufficient to represent the actual material behaviour. 

% Using the predefined profile "Printed PolyTerra PLA Plastic" from \cite{Graves2021}, the displacement is $\approx 0.95 mm$ for the same finger. For 20\% infill density and 0° infill direction, the displacement is $\approx 0.75 mm$, for 30\% and 0° infill direction $\approx 0.83 mm$. The mechanical properties for this profile from \cite{Graves2021} are taken from the data sheet of the material "PolyTerra PLA" \cite{PolyTerraPLA} and presumably using the combined values from the same publications (\hspace{1sp}\cite{Torres.2015, travieso2019mechanical}) of the "Printed PLA Plastic" profile. The density is given with: $1,14 g/cm^3$, the Young's modulus with: $1900 MPa$, the Poisson's ratio with: $0.36$, the yield strength with: $20.04 MPa$ and the ultimate tensile strength with: $20,9 MPa$, according to the profile "Printed PolyTerra PLA Plastic" from \cite{Graves2021}.
% Testing the combined values of the "Printed PLA Plastic" profile from \cite{Graves2021} results in a displacement of $\approx 0.86 mm$ for 20\% infill $\approx 0.68 mm$ and 30\% infill results in $\approx 0.75 mm$.
% Thus, the "Printed PolyTerra PLA Plastic" profile from \cite{Graves2021} provides a more accurate approximation of the actual behaviour, which is used for the FEA of the PLA+ fingers. \\

% Ideally, future work could analyze the already sliced part, considering the infill density and direction. However, to the author's best knowledge, there is currently no possibility to load an already sliced geometry, or a geometry extracted from the G-Code into an FEA or CAD program like Fusion 360. Hence, the fingers have to be modeled in CAD and loaded into Autodesk Fusion 360 for the static stress analysis. Since there is currently, to the author's best knowledge, no way to convert the sliced parts back to a solid the fingers have had to be modeled individually in accordance with the manufactured fingers. This contradicts the ideal conception of the finger's design, basically making the simplification of the design by using the slicer's abilities obsolete, which is why this approach is not ideal. 

%Future work could fill this gap by finding solutions to approximate a solid from the pathways and extruder settings from the G-Code.

% For the static stress a fine mesh with a model-based size of of 4\% is generated, as a compromise between computation time and fineness of the mesh, as proposed by Fusion 360. 
 
% For the constraints the contact surfaces of the connection to the mount are set as fixed in every direction, as seen in Fig. \ref{constraints}, the loads as shown in Fig. \ref{loads}. 

% % Figure environment removed

% A point probe is set in the middle of the contact plane for a precise measurement of the deflection.
\rev{
The force is set, according to the stiffness in Tab. \ref{tab:fingers_single_stiffness}, so that the displacement should be $\approx 1$ mm, and a probe measures the displacement, identifying the linear stiffness as $F_{yy} = K_{yy}\delta_y$. The resulting displacement, stiffness and the relative deviation from the stiffness values in Tab. \ref{tab:fingers_single_stiffness} of the analyzed fingers are listed in Tab. \ref{tab:FEA}.  \rev{Overall, the relative deviation, as seen in Tab. \ref{tab:FEA} is comparably low, deviating between 3\% and 26\%, demonstrating a robust performance of the FEA. The stronger deviations are discussed next.

Especially the results for $40\degree$ infill direction and 10\% infill density are noteworthy.
As the first entry for $40\degree$ infill direction and 10\% infill shows, there is a considerable discrepancy between the calculated stiffness value and the measured value from the experiments. This is because of a component failure of the finger at the experiment, where the deformation is no longer elastic but plastic due to a buckling of the structure. This is not considered in the FEA explaining the discrepancy of both values. To verify the FEA of the model, another approach is needed, demonstrated in the second entry.   
The second entry for $40\degree$ infill direction and 10\% infill is calculated with a force value by linearly extrapolating the two prior stiffness values at $20\degree$ infill direction and $3\degree$ with 10\% infill by
%a = 1.667, b = 1.533
$K_{yy} = ax + b$, where $x$ is the infill direction, $a$ the linear effect, and $b$ the offset. With $K_{yy} = 1.667$ N/mm for $x=30\degree$ and $K_{yy}= 1.533$ for $x=20\degree$, $a=0.0134$ N/mm$\degree$. The extrapolated stiffness at $40\degree$ infill direction is then estimated at $1.801$ N/mm. This value is now used to derive the applied force for the FEA, resulting in a displacement of $\approx 0.97$ mm and a stiffness of $1.86$ N/mm resulting in a deviation of $\approx 3\%$ compared to the extrapolated value giving a high level of consistency. However, it becomes clear that the FEA does not correspond well to the actual measured values where instabilities occur in the real-life experiment as these are not taken into account or do not occur in the FEA, hence the deviation of $\approx 54 \%$.

As Tab. \ref{tab:FEA} demonstrates, the results for the PETG fingers deviate by $\approx 52\%$. Since the offset is relatively constant over the variants, it could be assumed that the deviation does not come from the model or the model's properties itself, but rather from the mechanical attributes provided by Verbatim or the missing Poisson's ratio. As discussed before, the term "tensile strength at yield" could be ambiguous and other important mechanical properties are missing. 

% Since Sunlu does not provide any mechanical properties, an assumption has to be made, where an already existing material (Profile: "Printed PolyTerra PLA Plastic") from an external database \cite{Graves2021} is used to model the variants as previously discussed. 
The used material shows a high level of conformity with the real-life experiments, as long as no instabilities have been present, resulting in deviations between $3\%$ and $26\%$. The relatively high deviation of $54\%$ is discussed before, where structural instabilities are accountable for the high deviation. While a good approximation, if the FEM setup requires substantial time (e.g. to re-construct the finger geometry otherwise determined by the slicer), the print-and-test iteration cycle may be more efficient.
}
%Since Fusion 360 is a CAD program at first with the possibility of performing an FEA, there are features missing that professional FEA programs could provide. For instance, a stiffness matrix could be extracted by using Abaqus or Ansys to determine if there is a remote center of compliance (RCC) present, which is assumed for the variants with the angled ribs. 


\begin{table}[h!]
  \centering
  \caption{FEA results, where the displacement over the applied force is used to determine the respective stiffness which is then compared to the stiffness value in Tab. \ref{tab:fingers_single_stiffness}. $**$ denotes extrapolated force value.}
  \label{tab:FEA}

  \begin{adjustbox}{width=\textwidth/2}

\begin{tabular}{|l||c|c|c|c|c|}
\hline
Finger Type   & Displacement  & Applied Force & Stiffness  &  Relative Deviation \\
$ $ & $[mm]$ & $[N]$ & $[N/mm]$  & $[\%]$  \\

\hline
\hline

0° Infill Direction  &  PLA+: 0.94 &  PLA+: 1.2 &  PLA+: 1.28 &  PLA+: 6\\
10\% Infill Density & PETG: 0.41 & PETG: 0.5 & PETG: 1.22 & PETG: 59 \\
\hline

20° Infill Direction & PLA+: 1.07 & PLA+: 2 & PLA+: 1.43 & PLA+: 7 \\
10\% Infill Density &  & & & \\

\hline
30° Infill Direction & PLA+: 1.06 & PLA+: 1.733 & PLA+: 1.64 & PLA+: 6 \\
10\% Infill Density &  & & & \\
\hline
40° Infill Direction & PLA+: 0.46 & PLA+: 0.85 & PLA+: 1.85 & PLA+: 54 \\
10\% Infill Density &  & & & \\
\hline
40° Infill Direction & &  &  & \\ %test with slope
10\% Infill Density $**$ & PLA+: 0.97 & PLA+: 1.801& PLA+: 1.86 & PLA+: 3 \\
\hline


0° Infill Direction &  PLA+: 0.74 &  PLA+: 1.95 &  PLA+: 2.64 &  PLA+: 26 \\
20\% Infill Density & PETG: 0.54 & PETG: 1.133 & PETG: 2.1 & PETG: 46\\

\hline
20° Infill Direction & PLA+: 1.03 & PLA+: 2.6 & PLA+: 2.52 & PLA+: 3 \\ 
20\% Infill Density &  & & & \\
\hline
30° Infill Direction & PLA+: 1.19 & PLA+: 3.0 & PLA+: 2.52 & PLA+: 19 \\ 
20\% Infill Density &  & & & \\
\hline
40° Infill Direction & PLA+: 1.15 & PLA+: 3.4 & PLA+: 2.96 & PLA+: 15 \\
20\% Infill Density &  & & & \\
\hline
0° Infill Direction & PLA+: 0.82 & PLA+: 3.0 & PLA+: 3.66 & PLA+: 18\\
30\% Infill Density & PETG: 0.48 & PETG: 1.6 & PETG: 3.33 & PETG: 52\\
\hline
10° Infill Direction & PLA+: 0.91 & PLA+: 3 & PLA+: 3.3 & PLA+: 9\\
30\% Infill Density &  & & & \\
\hline
20° Infill Direction & PLA+: 0.94 & PLA+: 3.2 & PLA+: 3.4 & PLA+: 6\\
30\% Infill Density &  & & & \\
\hline
30° Infill Direction & PLA+: 1.07 & PLA+: 3.7 & PLA+: 3.46 & PLA+: 7\\
30\% Infill Density &  & & & \\
\hline
40° Infill Direction & PLA+: 0.94 & PLA+: 3.84 & PLA+: 4.09 & PLA+: 6\\
30\% Infill Density &  & & & \\
\hline
\end{tabular}

  \end{adjustbox}

\end{table}
\vspace{-0.5 cm}


}

\section{Validation}
\label{sec:app}

This section validates the finger's performance in assembly applications. \rev{In addition to the assembly process shown in Fig. \ref{fig:working principle}, various objects shown in Fig. \ref{fig:addl_scenarios} can be assembled by the fingers.} The remainder of this section iteratively validates the fingers for the assembly of a plug into socket. The setup used is the same as in the previous experiments, demonstrated in Fig. \ref{fig:robot_zoom}, implementing the working principle presented in Fig. \ref{fig:working principle}.  
% Figure environment removed

%To further optimize the search process and to optimally compare the different parameters of the fingers, failures due to the plug and sockets attributes, which where described before in Sec. \ref{subsec:failures}, have to be minimized as far as possible. 
% Hence, for the following experiments and to allow tests in an industrial application, an "automation-friendly" plug and socket connection is used. The test environment is equipped with the spring and magazine introduced in Sec. \ref{sec:testenv}. 
The goal is to successfully pick the plug from a magazine and assemble into a socket with and without various misalignment values. 
% As described, the movement of the cable head inside the grip can also lead to failure. This is caused by the weight of the cable exceeding the maximum applicable force of friction consisting of the gripping/normal force and the coefficient of friction given by the adhesive layer. The weight is composed of the weight of the cable head and the wires, as seen in Fig. \ref{automat_cable}. 
% The programs used are programmed via the teach panel of the \textit{Universal Robots UR5}. 
The program is intended for high-speed assembly, with tool speed values of $250 mm/s$, as seen in Fig. \ref{fig:5ovr}, up to over $1.3 m/s$\rev{.}
% , as seen in Fig. \ref{fig:100ovr}. 
In Fig. \ref{fig:5ovr}, the force peak is attributable to the forces caused during the line contact while moving in z-direction, visualized in Fig. \ref{subfig:key-c} to fully assemble the parts. 

% and a tool acceleration of 1200, up to 2000 $mm/s^2$. 
% using the MoveL command of the \textit{UR} to approach the waypoints.

% Figure environment removed   


% % Figure environment removed


% % Figure environment removed

% % Figure environment removed   


The assembly and grasping process is summarized as follows and can be seen in Fig. \ref{fig:robot_zoom}. First, the cable is grabbed from a magazine which provides a repeatable starting point. The contact force of the fingers overcome the contact force of the spring when moving the gripper in a linear movement upwards, with which the cable is removed from the magazine. To compensate slippage during the first phase, afterwards the gripper could push the cable head slightly on the table with a linear movement downwards to ensure a contact with the upper contact surface of the finger. Now, in the second phase, assembly takes place, using the search strategy described earlier in Section \ref{subsec:strategies}. A video is available \rev{as supplementary material and} at \url{https://youtu.be/J7EGXtE54oY}. 

%Generally speaking, the movement values depend on the present task. 
% The connector is held angled in the initial position. The first contact consists of a planar contact of one side of the connector and the front part of the housing of the socket. The first contact point is successful when the front ridge of the connector lies flat on the front part of the housing. The next contact point is achieved by sliding the connector in the negative base-x-axis while moving slightly in the negative base-z-direction. With this, the connector is pushed slightly into the socket while a contact with the back wall is aspired. The connector should now slightly be inserted into the plug. However, there still is no contact with the front wall and at least one side wall to fully align the connector according to the constraints. For this, in the next step, the connector is moved in the positive base-x-direction to achieve contact with the inside of the front wall. To fulfill the last condition, the plug is moved in the positive base-y-direction, so that the side of the connector is in contact with the inside of the socket's side wall.
% Now, since all constraints have been complied with, for the final step, the plug is moved linearly in the negative base-z-direction to assemble the connectors, as seen in Fig. \ref{exp_3_fifth}. 

% After a successful assembly, the closed gripper rotates around the x-axis, releases the grip and moves back to the initial position. To reset, the connector can now be removed from the plug and is clamped back into the magazine.

\subsection{Repeatability Experiments}
\label{sec:repeatability_magazine}

To test for repeatability, the assembly process is repeated 84 times with a fixed socket position, using the aforementioned UR program and manually resetting the plug in the magazine, out of which the assembly failed twice. The first failure occurred at attempt 30 and the second failure at attempt 84 which ultimately lead to a component failure of the fingers. This concludes a roughly $97.6\%$ success rate. 
Assumed causes are either slight slippage in the grip coming from the adhesive tape, or the kinematics of the robot. Additionally, the table on which the robot is mounted is not fixed but on wheels, which could add another level of instability, impairing the robustness. 

\subsection{Robustness Experiments}
\label{sec:robustness_exp}

In the next experiments, the robustness over variation in socket position is tested, to clarify the impact of the design parameters on the robustness and tolerable range.  To control the misalignment, instead of a fixed waypoint for the socket position, a variable waypoint is programmed which can be changed in each iteration. For the initial test, the boundaries of compensable misalignment of the plug to the socket are determined in x- and y-direction in $0.5$ mm steps. A finger with $0 \degree$ infill direction, $10\%$ infill and a $10 \degree$ mount are used. A successful assembly is repeated five times to assure repeatability. If five assemblies in a row are successful, another 0.5 mm is added to the misalignment and the sequence of five trials starts again. This is repeated until the maximum compensable misalignment is met and the assembly fails for the first time. This is done to test the limits both in the  x- and y-direction. With this setup, it can be shown that with a 100\% speed value of the program and the used search algorithm this compliant finger design is capable of tolerating a misalignment in a range of $7.5$ mm in y-direction and $7$ mm in the x-direction. 
To further compare the tolerance windows with varying designs, the limits of the first run are tested with varying infill densities and infill directions. The results are listed in Table \ref{tab:erg_robustness}. 
% Here, $m$ denotes mount, meaning which mount configuration is used (either 10° and 20° have been tested), $i$ denotes the infill density in percentage and $id$ is an abbreviation for the infill direction in deg.  

\begin{table}[ht]
  \centering
  \vspace{0.2cm}
  \caption{Results Robustness Experiment, $m$ denotes mount, meaning which mount configuration is used (either 10° and 20°), $i$ denotes the infill density in percentage and $id$ is an abbreviation for the infill direction in deg. \rev{f denotes failure} } 
  \label{tab:erg_robustness}

     \begin{adjustbox}{width=0.95\textwidth/2}

    \begin{tabular}{|c|c|c|c|c|c|c|c|c|c|c|}
 & x & y & x & y & y & y & y & y & y & y\\
 & 10° m & 10° m  & 20° m & 20° m & 10° m & 10° m & 10° m & 10° m & 10° m & 10° m  \\

Variant & 10\% i & 10\% i & 10\% i & 10 \% i & 20\% i & 30\% i & 10\% i & 20\% i & 30\% i  & 10\% i \\

 & 0° id & 0° id  & 0° id & 0° id & 0° id & 0° id & 0° id & 0° id & 0° id & 10° id   \\
 & PLA+ & PLA+ & PLA+ & PLA+ & PLA+ & PLA+ & PETG & PETG & PETG & PLA+\\

\hline
\hline

range [mm] & 7 & 5.5 & 4.5 & 5.5 & 5 & 4.5 & 7.5 & 6 & 5.5 & 5.5  \\

\hline 

    \end{tabular} 


  \end{adjustbox}

  \vspace{5pt}
    \begin{adjustbox}{width=0.95\textwidth/2}
    \begin{tabular}{|c|c|c|c|c|c|c|c|c|c|c|c|c|}
 & y & y & y & y & y & y & y & y & y & y & y \\
 & 10° m & 10° m & 10° m & 10° m & 10° m & 10° m & 10° m & 10° m & 10° m & 10° m & 10° m \\

Variant& 10 \% i & 10 \% i& 15 \% i & 20 \% i & 25 \% i& 30 \% i & 10 \% i & 15 \% i & 20 \% i & 25 \% i& 30 \% i  \\

 & 20° id & 30° id & 30° id & 30° id & 30° id & 30° id & 40° id & 40° id & 40° id & 40° id & 40° id  \\
 & PLA+ & PLA+ & PLA+ & PLA+ & PLA+ & PLA+ & PLA+ & PLA+ & PLA+ & PLA+ & PLA+ \\

\hline
\hline

range [mm] & 5 & 4 & 5.5 & 5.5 & 5.5 & 6 & f & f & f & 2 & 5.5  \\

\hline

    \end{tabular}    
 \end{adjustbox}

\end{table}


It is important to note that regarding the compensable range in x-direction, the compensation is attributable to the free rotation of the cable head inside the grip about the y-axis, corresponding to the coordinate system visualized in Fig. \ref{subfig:key-a}, and should be treated as a positive side-effect of the finger's design, which can be described as an unforeseen DoF. 
% This is also another reason as to why the contact plane in the notched fingertip is rejected.
The main focus of the finger's design is to allow a compliance in the y-direction due to the ribs, which is why the experiment only shows general feasibility of a compensation in the x-direction for the $10 \degree$ and $20 \degree$ mounts but does not compare the tolerance in x-direction for every finger, as seen in Tab. \ref{tab:erg_robustness}.

\subsection{Discussion}
As Tab. \ref{tab:erg_robustness} shows, with a $10 \degree$ mount the tolerable misalignment-range is slightly larger than with a $20 \degree$ mount. During the tests for $20 \degree$ infill direction and 10 \% infill density at a misalignment of $5$ mm early signs of buckling are noticed. At the next increment of the infill direction, this is noticed already at $4$ mm and plastic deformations at the connections of the ribs to the outer wall appear at $4.5$ mm. $30 \degree$ infill direction with 10 \% infill initially stands out due to a comparably large compensable range of $\approx 5.5 - 6.5$ mm. However, beginning with $+4$ mm misalignment, some slight buckling can be observed building up to slight plastic deformation at the connection of the ribs to the outer wall with the next misalignment increments, which is why this variant is not considered ideal.  
The extreme value of $40 \degree$ direction shows to be difficult to test. At a comparably low misalignment value of $2$ mm, component failure already occurs for $10\%$ infill density resulting in a non-feasible combination for any assembly tasks. This is also noticeable for $15\%$ infill density where buckling and component failure occurred at $3$ mm and $3.5$ mm. Because the initial start-value is set too high, the part is already permanently damaged, resulting in a failed assembly at $-1.5$ and $-2$ mm. At $20 \%$ infill there is strong bulging noticeable at $2$ mm and buckling at $2.5$ mm. $-2$ mm proves to be compensable, however, bulging is noticeable here, too. With $25 \%$ infill strong deformation is noticed at $3.5 mm$. $4 mm$ is also successful, however, some plastic deformation occurs, which is why the experiment is stopped here to prevent any further damage. The last increment for the infill density at $40 \degree$ infill direction proves to be the most stable one. Some strong deformation is observed at $3 mm$ but without plastic deformation. At $-3$ mm the cable head strongly clips into the plug, which is why no further tests are done for this variant to prevent any further damage. This is attributable to an excessive vertical stiffness of the finger, where compliance is still present, with a potentially too high contact force profile which could damage the electrical components. Thus, this variant should not be used to assemble delicate parts.  

To compensate misalignment parallel to the moving direction of the gripper's jaws, structured compliance in the base-y-direction is desired. However, unintentional DoFs, as a rotation about the base-y-axis or the base-x-axis resulting in a change of pose of the grasped part resulting from contact forces, are possible. 

% The robot's collision detection by the \textit{UR} proves to be successful as well. During the tests for 0° infill direction and 20 \% infill density and a misalignment of $- 2 mm$, where a failure is anticipated, the protective stop engages before damaging the fingers. 

Regarding the PETG fingers, $0\degree$ infill direction and $10\%$ infill density proves to achieve the biggest tolerance range of all combinations, tolerating $\approx7.5$ mm misalignment. However, this combination is not suitable for any assembly tasks because the cable head slips easily inside the grip. This is traced to a very low gripping force from the fingers due to the infill settings. Increasing the infill density by $10\%$ already results in a better grip, while achieving a tolerance range of $\approx6$ mm. Another $10\%$ show similar results, the compensation of $+3.5$ mm misalignment cannot be repeated robustly. Using PETG comes with the benefit of a higher flexibility compared to PLA+, which results in a lower risk of plastic deformation during handling. 
The tolerable range can be defined as $\approx5.5$ mm while providing a stable grasp on the gripped cable head. 

% Additionally, as seen in Fig. \ref{fig:rcc_effect}, the finray structure of the fingers realize an effective remote center of compliance (RCC), in that a lateral translation at the base of the plug results in a change in orientation with the tip moving towards the perturbation. As the infill direction increases, the effect is more pronounced, but as the higher infill direction have a smaller tolerance window, as seen in Table \ref{tab:erg_robustness}, this effect has proven to be secondary to the tolerance window. 

%\rev{As discussed earlier, and shown in Fig. \ref{fig:rcc_effect}, an effective remote center of compliance (RCC) is noticeable for the finray structure. As the infill direction increases, the effect is more pronounced, but as the higher infill direction have a smaller tolerance window, as seen in Table \ref{tab:erg_robustness}, this effect has proven to be secondary to the tolerance window.}

% % Figure environment removed





% \subsection{Assembly of Clamp Connections}
% \label{sec:clamping}

% To broaden the range of applications and to show the feasibility of the finger for other tasks, an additional assembly step is added. Other than for the plug and socket assembly here the goal is purely to show general feasibility. This step acts independent from the plug and socket assembly step. 
% After successfully assembling the plug into the socket from position 1 to position 2, as seen in Fig. \ref{ovr_robustnes}, a clamp, attached to the cable used in Fig. \ref{automat_cable} is gripped and connected with the contact point shown in position 3 in Fig \ref{ovr_robustnes}. For this, the gripper slides the clamp diagonally onto the contact point. As for the prior experiments, this approach also allows to compensate positional uncertainties. To support this, here a contact between the chamfer on the contact point and the clamp are used to guide the clamp into position. Any further misalignment is compensated by the finger's compliance. This experiment is performed with the PETG finger with 30 \% infill density. \\
% In Fig. \ref{clamp_pos} the grip of the clamp is shown. In the right finger, the rectangular shape of the clamp is form-fitted to the notch. However, on the left side simply friction forces are used to hold the clamp in place. Fig \ref{clamp_assmbl} demonstrates the mounted clamp, the deformations of the ribs are visible.

% % Figure environment removed


\section{Conclusion and Future Work}

\label{sec:discussion}
To the authors' best knowledge, this work has proposed a first use of a finray-effect gripper for structured compliance in assembly. That is, compared with previous works using the finray-principle which focus on a stable grasp with varying object surface geometry, this design realizes directionally-dependent stiffness on the gripped part. This is used to robustly and repeatedly compensate misalignment in the range of up to $7.5$ mm in high-speed assembly tasks. Additionally, the objective, as defined before in Sec. \ref{sec:intro}, of achieving a comparable success time as in \cite{park2013intuitive} is reached and exceeded, as the assembly time from  first contact is $\approx 1.2$ seconds. Hence, feasibility of the passive compliant fingers to compensate misalignment in high-speed tasks without additional sensing is proven. 
With an increasing infill density and increasing infill direction, the stiffness of the finger can be increased, as shown in Tab. \ref{tab:fingers_single_stiffness}. However, an increasing infill direction results in an earlier structural failure of the finger, in both maximum force and deflection. 
For an optimal finger design, the finger stiffness, the ultimate strength, the maximum deflection, the gripping stability and the compensable range have to be taken into consideration. A variant with a too high stiffness, e.g. variants with a $30\%$ infill density, especially with an increasing infill direction could damage the assembly parts. A too low stiffness, e.g. PETG with $10 \%$ infill density would not be able to lift and transport the cable robustly and maintain a stable grip when external forces occur. Choosing a $40 \degree$ infill direction results in component failure due to plastic deformation especially at lower infill density values.
Most of the variants listed in Tab. \ref{tab:erg_robustness} achieve a tolerable range of $\approx 5.5$ mm, $30 \%$ infill density and $0 \degree$ infill direction achieves the lowest, with $\approx 4.5$ mm. PETG shows the best results here, with a maximum range of $\approx 7.5$ mm for the non-applicable $10 \%$ infill variant. Thus, the higher rib angle PETG variants are recommended in this case. 

% The compensation of combined misalignment values has not been studied in this work. This is because only the passive compensation structures as the controllable parameters could be adjusted, the compensation in the x-direction are primarily unintentional and a result of a free rotational movement inside the grip. Hence, an investigation of the combined values would not provide clear conclusions of the density and direction of the fingers' ribs.

Future work will focus on establishing additional attempts to design the fingers by using FEA or by analytically determining the mechanical properties and to achieve a better intuition of how the design parameters influence the final stiffness of the structure. 

Using fused deposition modeling as an additive manufacturing process comes with its own limitations, as the direction in which the part is built up has to be considered. Certain structures need an optimal orientation to the print bed to be successfully manufactured, as overhangs or otherwise unsupported structures could fail without support. Using alternative manufacturing processes could allow one to create ribs in varying directions which could introduce multi-directional structured compliance into the finger. Additionally, other material could be used which could achieve higher contact forces and would be less sensitive to wear and fatigue.  
% As the fingers have proven to remain reliable in the repeatability experiments described in Sec. \ref{sec:repeatability_magazine}, it is assumed that fatigue does not heavily influence the compliance and the gripping behaviour. However, hysteresis has become noticeable at higher contact forces. Hence, this could be further investigated in future experiments, by applying a constant load and measuring the resulting forces over time as in \cite{hartisch2022flexure}. \\

% and successful assemblies of various electrical components is shown, additionally proving generalizability.  


% Choosing a 40° infill direction results in component failure for almost every variant, except for 30\% infill, however, this combination could damage the assembly parts as the experiment has indicated. The fragility of this particular infill direction is demonstrated in Tab. \ref{tab:fingers_single_force}, clearly showing the lack of suitability for these tasks. 
% As a desirable outcome, a high maximum deflection should be considered to allow the finger to adapt to external forces without being damaged, which is comparably the same for 0° infill direction as seen in Tab. \ref{tab:fingers_single_maxdeflection}. As already demonstrated in the robustness experiments in Sec. \ref{sec:robustness_exp}, the maximum tolerable force is especially important regarding the stability of the gripper when adapting to external forces during the assembly tasks. The higher the infill density and lower the infill direction, the better the results, as seen in Tab. \ref{tab:fingers_single_force}. 
% Additionally, the tolerable range should be as high as possible. Most of the variants listed in Tab. \ref{tab:erg_robustness} achieve a tolerable range of $\approx 5.5 mm$, 30 \% infill density and 0° infill direction achieves the lowest, with $\approx 4.5 mm$. PETG shows the best results here, with a maximum range of $\approx 7.5 mm$ for the non-applicable 10 \% infill variant. The two other PETG variants, however, achieve a slightly lower compensable range of $\approx 5.5 mm - 6 mm$. During the experiments, 10 \% infill with 30° infill direction has shown a high tolerable range of $\approx 6.5 mm$ as well. This could not be repeated for the other 30° infill direction variants as these kept a constant but lower tolerable range at around $\approx 5 mm - 5.5 mm$. 
% Thus, either one of the higher PETG variants or the 30° infill direction and 10 \% infill density finger stand out the most in this case. 
% Considering this, the best trade-off is assumed to be made while using PETG with either 20 \% or 30 \% infill density, as these show a comparably high tolerance range. 30\% infill density can easily handle the rigid cables, thus ensuring a high gripping stability. Additionally, both variants have a lower comparable stiffness as the PLA+ fingers, allowing a higher level of compliance. A disadvantage seems to be the maximum tolerable force, for 20 \% infill its almost $20 N$ lower as for 30 \% infill. With 30 \% infill one of the highest values for the maximum deflection is achieved, only exceeded by PLA+ with the same parameters. In addition to that, the PLA+ fingers with 0° infill direction and lower infill density values could also be effective for the electrical assembly, as these provide a sufficient stiffness to stably handle the components. The maximum tolerable force is comparable to the values of PETG with a higher infill density while achieving a very high maximum deflection, again, close the the maximum achievable value. The compensable range varies between $4.5 mm $ to $5.5 mm$. 
% A 30° infill direction could achieve high compensable ranges as seen in Tab. \ref{tab:erg_robustness} of $\approx 6.5 mm$ with 10\% infill density. However, as Tab. \ref{tab:fingers_single_force} demonstrates, the maximum tolerable force is close to the minimum value, which also applies to the maximum deflection in Tab. \ref{tab:fingers_single_maxdeflection}. Additionally, the fingers have shown slight plastic behaviour for the higher misalignment values which is why this variant is not considered ideal.  

% In this work, structured, passively compliant fingers have been investigated to achieve a certain degree of compensation of misalignment between both mating parts in an industrial environment and to achieve sufficient assembly force. Compliant structures are understood as structures achieving significantly different stiffness values in one or more predefined directions. With this, certain DoFs can be locked or enabled. 





% The proposed search strategy has proven to be sufficient for this task, however, this could be upgraded with force sensing, to detect jamming during the assembly process, as used in the binary search methods in \cite{Chen.2016}, as proposed in Sec. \ref{sec:control_methods}. 
% As the FEA has shown to provide useful data to understand the mechanical properties, the design of the finger in CAD contradicts the simple design principle of using \textit{Cura's} settings to create the finger, as every single rib has to be designed individually. As proposed before, future work could focus on this and try to approximate a solid from the created G-code, as G-code viewers, e.g. \textit{PrusaSlicer G-code viewer}, can be used to visualize the toolpaths and extruder settings creating the desired part.   


% Fused deposition modeling is used to manufacture the parts in this thesis, however, future work could look into using other manufacturing processes, for example vat photopolymerzation, described in \cite{Gibson2021vat}, which is known to quickly create high quality parts \cite{VATpp}. Alternatively, powder bed fusion could be investigated, which can work with one of the widest ranges of material, making tests with metal powders as a material for the fingers possible, as long as the metal can be welded, as stated in \cite{Gibson2021powder}. For polymer based powder bed fusion, there is no need for support structures as the powder itself can be used to support the parts, according to \cite{Gibson2021powder}.
% However, both manufacturing processes come with downsides, as post-processing of metallic parts created by powder bed fusion can be very time consuming, for example when removing support structures which are needed to prevent distortion \cite{Gibson2021powder}. Vat photopolymerization is limited to acrylates and epoxies \cite{Gibson2021vat}, and also requires a time-consuming post-process to remove the resin and for curing \cite{VATpp}.

% As only two different materials have been tested in this work, future work could try to further compare various materials and manufacturing processes and their grasping properties. Ideally, a database could give a clearer recommendation of the materials. Additionally, it could be examined if alternative manufacturing processes could tackle the directional limitations of the build-up of FDM described before.
% Additionally, compliance in only one direction (base-y-direction) is tested, however, adding additional compliant structures in other directions could be investigated in future work, for example by combining ribs in multiple directions in one finger or implementing compliance in other directions in the mount or base of the fingers. \\

% Additionally, other tasks could be investigated in combination with the fingers to further determine the generalizability of the fingers. For this, in future works, a demonstrator is built in combination with the aforementioned visual servoing to localize certain keypoints of a workpiece and therefore approximate the pose of the destination points. From the pose, the destination point can be derived to assemble mechanical and electrical components. Since only plugs and other electrical components have been tested in this work, feasibility could then be tested for the assembly of other mechanical components, e.g. gears or bearings could be mounted onto shafts. 




% \printbibliography

\bibliographystyle{IEEEtran}
\bibliography{lib2.bib} % Export for bibtex! Not biblatex. 

\end{document}
