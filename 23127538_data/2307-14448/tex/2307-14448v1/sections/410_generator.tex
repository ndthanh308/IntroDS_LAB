\subsection{\confounderdashboard}\label{sec:systemdesign:confounderdashboard}
The \confounderdashboard interface in Fig.~\ref{fig:system_overview}A enables users to set up cause/outcome variables and to locate the most likely confounders that are distorting the specified cause-outcome relationship ({\bf T1}). The flexibility of the system allows users to tailor their investigations to their specific research interests. For example, if someone is curious about the effectiveness of the job training programs in boosting annual earnings, they can select {\tt take\_training\_program} as the cause and {\tt yr1978\_earning} as the outcome. But if users are interested in the impact of marriage status on income, they can choose {\tt married} as the cause and {\tt yr1978\_earnings} as the outcome. To guide users in reflecting the causal relationships among covariates with the cause and outcome, \confounderdashboard provides a textual explanation of confounders---``{\it a confounder is a third variable that influences both cause and outcome yet does not lie on a causal pathway between cause and outcome}''---when hovering over the question mark next to confounder selection box in Fig.~\ref{fig:system_overview}A. Furthermore, it also provides a table where covariates are ranked by CF scores (ref. Section~\ref{sec:methodology:cf_metrics}) from the highest to the lowest. \tocomments{In the third column, we utilize two side-by-side, vertically aligned histograms \cite{cheng2020dece,ahn2019fairsight} to depict the feature distribution of the untreated group (blue) and the treated group (purple). When $X$ is a continuous variable, it will be discretized into two bins as the ``pseudo'' untreated/treated groups.} 
In \confounderdashboard, users can draw upon multiple information resources, including their domain knowledge regarding the causal structure of cause/outcome and covariates, the statistical CF scores, and the visual information of feature distribution among treatment groups, to select the most likely confounders and put them into the confounder box.

% {After initializing the system and loading data, users can use the \confounderdashboard interface shown in Fig.~\ref{fig:system_overview}A to set up all the required variables for analysis. First, users need to select a pair of cause and outcome variables as their analysis goal. For example, if the user wants to determine if the job training program increases participants' earnings using the aforementioned Lalonde data, they should select {\tt take\_training\_program} as the cause and {\tt yr1978\_earning} as the outcome. If the user is interested in exploring whether marriage status influences participants' earnings, they should set the cause and outcome variables as {\tt married} and {\tt yr1978\_earnings}.}

