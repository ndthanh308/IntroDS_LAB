\subsection{\subgroupviewer}\label{sec:systemdesign:subgroupviewer}

The {\subgroupviewer} interface in Fig.\ref{fig:system_overview}C provides a comprehensive overview of subgroup patterns, enabling users to understand their heterogeneous characteristics ({\bf T3}). Our design considers subgroup differences not only at the attribute level, but also at the causal behavioral patterns. To achieve this, we have created two views in this panel: {\causalityspace} for exploring causal patterns and {\covariatespace} for analyzing attributes. To ensure a seamless user experience, these two spaces are coordinated with consistent color codings over subgroups. Users can select a subgroup in either space and the interactions will be reflected in both spaces, or hover over subgroups to examine more detailed information.

% {Referring to the causal framework in Fig.\ref{fig:teaser}B (ref. Section~\ref{sec:methodology:diagram}), we focus on subgroup-level {\it propensities}---how likely people in a chosen subgroup are to participant in the training program, {\it base effects}---what are the annual earnings individuals in a subgroup typically obtain in 1978, and cause-outcome {\it associations}---to what extent cause and outcome are correlated.}

\input{images/2dspace_caption}

\subsubsection{\causalityspace} This view is shown in Fig.~\ref{fig:system_overview}C, illustrating subgroups' patterns in terms of {\it propensity}, {\it base effect}, and {\it association}. In particular, it represents subgroups in a two-dimensional coordinate space with the horizontal axis as cause and vertical as outcome. Depending on data types, subgroups are represented by either the circle-line design \cite{rucker2008simpson} or the elliptical geometry design \cite{friendly2013elliptical}, as shown in Fig.~\ref{fig:system_overview}C-1, C-2.

{The {\bf circle-line design} (Fig.~\ref{fig:2dspace}A) \tocomments{has been proposed to visualize Simpson's paradox \cite{rucker2008simpson}, particularly depicting cause-outcome} associations where \tocomments{cause is dichotomous and outcome is either dichotomous or continuous.} Each pair of two circles connected by a line represents a subgroup, whose members are divided into control and treated group each represented as a circle with its size indicating the number of subjects. When comparing the sizes of two connected circles, it reveals the treatment propensity (e.g., subgroup 1 has a stronger propensity than subgroup 3). Circles are positioned along two-sided vertical axes indicating the average value of outcome, with the slope between two circles revealing cause-outcome associations. By comparing the distinguished aggregate (grey) against the subgroup-level circle-lines, users can quickly locate a paradoxical phenomenon, such as the aggregate is positive while subgroup 2 is negative.}
{The {\bf elliptical geometry design} \tocomments{is discussed in \cite{friendly2013elliptical} to visualize Simpson's paradox, for cases} where \tocomments{both cause and outcome are continuous} (see Fig.\ref{fig:2dspace}B). The plot consists of colorful ellipses, each representing a subgroup, while a gray ellipse denotes the aggregate to facilitate comparison between the overall data trend and subgroup-level patterns.
The centroid of each ellipse displays the average values of cause and outcome suggesting propensity and base effect, denoted as $(\bar{X},\bar{Y})$ \cite{friendly2013elliptical}. Additionally, the half-widths of the vertical and horizontal projections of each ellipse geometrically reflect  the standard deviations of cause and outcome, represented by $s_X$ and $s_Y$ (Fig.~\ref{fig:2dspace}). The trend of a regression line that passes through the ellipse centroids is indicative of cause-outcome associations.}

% {In terms of interactions, users could mouse over subgroups to examine more detailed information. We note that, in both designs, we utilize slopes to demonstrate cause-outcome associations so that users are able to grasp diverse trends and to locate paradoxical associations. As the true causal effects are always hidden in observational studies, we provide more information in \decisiondiagnosis interface, in Fig.~\ref{fig:system_overview}E, to assist users to conduct spuriousness diagnosis.}

\subsubsection{\covariatespace} 


% In information visualization, a glyph refers to a small and compact graphic representation that represents a data point with multidimensional features. 
% Compared with other multidimensional visualization techniques, such as multidimensional scaling (MDS)17, parallel coordinates19, scatterplot matrices, and various advanced designs for reducing clutter in multidimensional data25 or for representing data from heterogeneous dimensions26–30, glyphs transform mul- tidimensional data features to composite visual prop- erties (such as shape, color, and size), producing various ‘‘visual signatures’’ of data points that reveal more complex data patterns and offer a richer descrip- tion about data points.

This view (Fig.\ref{fig:system_overview}C-3) represents the characteristics of subgroups using multivariate radar glyphs \tocomments{due to its compactness and richness. Unlike other multidimensional visualizations (e.g., scatterplot matrices, parallel coordinates \cite{inselberg2009parallel}), glyph visualization encodes complex data features to compact ``visual signatures'' of distinct subgroups \cite{cao2018z}.} The leftmost glyph represents the entire population and the rest of them represent subgroups. In a radar glyph, the axes represent features with dots along the axes indicating the mean values of those features as a summary. To account for the differences in feature scales, we normalize their values into a uniform range of zero to one using $\bar{F}_j = (F_j - \mathrm{min}(F_j)) / (\mathrm{max}(F_j) - \mathrm{min}(F_j)), \forall j$. For categorical features with $J$ discrete values, we convert them into $J - 1$ binary features. In the radar glyphs, we only show the most discriminative top-$k$ features (e.g., $k=5$). To measure a feature's discriminativeness, we trained a series of one-vs-rest binary classifiers with the chosen feature $F$ as input and subgroup labels as output. We then computed an aggregated AUC score as a measurement of feature discriminativeness, where a higher AUC value suggests a stronger ability to distinguish data points from different subgroups. Users have the freedom to select features to investigate by clicking ``Choose Axes.'' E.g., the five axes in Fig.\ref{fig:system_overview}C-3 include {\tt age}, {\tt hispanic}, {\tt no\_degree}, {\tt education\_years}, and {\tt married}.

% {Users could clearly observe the phenomenon of Simpson's paradox by comparing the aggregate trend with any of the subgroup-level trends. If a subgroup were non-identifiable (ref. \ref{sec:methodology:framework:assumption}), our design would clearly highlight it by showing a single circle--either the treated or the untreated---in Fig.~\ref{fig:2dspace}(a). To support interactive data exploration, \vispur allow users to mouse over a subgroup to see the detailed statistics (e.g., estimated effect size and its bootstrap confidence interval), and to highlight (by clicking) two or more subgroups for between-subgroup comparison.}


