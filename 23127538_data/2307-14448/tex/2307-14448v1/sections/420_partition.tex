\subsection{\partition}\label{sec:systemdesign:partition}
The {\partition} interface in Fig.\ref{fig:system_overview}B \tocomments{allows users to create subgroups using two methods}---{\tt MANUAL} and {\tt AUTO}---based on a set of features\footnote{The partitioning variables could also be thought of as confounding variables, but they are special in the sense that they are a subset of the confounders with respect to which we want to capture subgroup heterogeneity.} ({\bf T2}). 
\tocomments{Previous subgroup analysis systems have utilized attribute-value pairs to construct subgroups \cite{cheng2020dece, kwon2022rmexplorer, kwon2020dpvis, guo2023causalvis}. \vispur employs a similar design, enabling users to add or remove covariates and specify the corresponding cut points. As shown in Fig.~\ref{fig:system_overview}B, two variables, {\tt black} and {\tt yr1975\_earning}, are selected, multi-thumb sliders are utilized to determine the cut points. A histogram distribution is displayed above the sliders, providing users with a visual reference for selecting appropriate cut points.}
Alternatively, users can opt for the {\tt AUTO} option, which uses algorithm-supported automated partitioning (ref. Section\ref{sec:methodology:algorithm}). By specifying a few configurations, such as the expected number of subgroups and the minimum size of subgroups, users can easily obtain an algorithm-generated partition that mitigates within-subgroup confounding bias. When clicking the {\tt Submit} button, subgroups based on the partition are generated in \subgroupviewer.

% {${\tt yr1974\_earning \in [7.5, 10.5)}$ $\&$ ${\tt black = True}$, ${\tt yr1974\_earning \in [7.5, 10.5)}$ $\&$ ${\tt black = False}$, ${\tt yr1974\_earning \in [0, 7.5)}$ $\&$ ${\tt black = True}$, as well as ${\tt yr1974\_earning \in [0, 7.5)}$ $\&$ ${\tt black = False}$.}

% DECE: users can refine the group by changing ranges for each feature and click the update button. The users can copy or delete an unwanted subgroup to maintain the subgroup list.
% CausalVIS, RMExplorer: users can click a particular name to facet the visualization by this variable, up to three variables can be selected this way.
% DPVis: a list of pairs of an attribute name and its value range.