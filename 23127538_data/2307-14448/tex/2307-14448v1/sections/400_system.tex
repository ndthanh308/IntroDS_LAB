\section{\vispur: System Components} \label{sec:systemdesign}

% Figure environment removed

% {An overview of \vispur system. (A) The {\bf \confounderdashboard} allows selecting cause and outcome, ranking remaining covariates and displaying the comparison of distributions across treatment arms to assist users in selecting confounders. (B) The {\bf \partition} supports both manual and automated subgroup generation by either allowing users to specify partition rules (manual) or algorithm parameters (auto). (C) The {\bf \subgroupviewer} contains two separate spaces: {\causalityspace} (C-1 for binary treatment, C-2 for continuous treatment) and {\covariatespace} (C-3), where the former demonstrates subgroups' causal/assocaitional properties such as {\it propensity}, {\it base effect}, as well as cause-outcome {\it association} for generated subgroups, while the latter space displays a set of radar-shaped glyphs for subgroups which encode the median values of selected features. (D) The {\bf \reasoningstoryboard} allows users to select and compare subgroups of interest by investigating story flows from treatment actions to outcome status. (E) The {\bf \decisiondiagnosis} reports the detailed subgroup-level statistics in {\basicstatistics} (E-1) and {\imbalancechart} (E-2) for a selected subgroup. It pops out warning messages to alert users about the presence of confounding bias and association distortion.}

{In light of the design requirements, we describe how we design \vispur to implement the aforementioned ``de-paradox'' workflow for causal analysis of spurious/paradoxical associations. \tocomments{The demonstration is based on the Lalonde dataset \cite{lalonde1986evaluating}, as it has been a benchmark data in causal inference literature and incorporated for use cases in many libraries such as MatchIt \cite{stuart2011matchit} and Cobalt \cite{greifer2020covariate}.} As shown in Fig.~\ref{fig:system_overview}A-E, our visual design leverages human perceptional features to reduce users' cognitive burdens, highlighting two major components---\subgroupviewer and \reasoningstoryboard---to assist inspecting subgroup characteristics, as well as making sense of association paradox, respectively. (1) In \subgroupviewer, we exploit radar-shaped glyphs to encode rich and multidimensional data features into a small and compact graphic representation, allowing users to visually compare the most important ``feature signatures'' of subgroups (Fig.~\ref{fig:system_overview}C-3). To distinguish subgroups' causal properties, we redefine the most popular two-dimensional space (that data practitioners are familiar with) to represent a new causality space (Fig.~\ref{fig:system_overview}C-1, C-2), where several causal concepts (propensity, base effects, etc) are encoded into simple visual signals, facilitating a straightforward and contrastive comparison. Furthermore, to support paradox reasoning, we leverage flow-based visualizations based on a common idea that the cause is followed by the effect. The progression pathways showcase how subjects chose their preferred treatment options, and how alternative chains of treatments lead to different outcomes (Fig.~\ref{fig:system_overview}D).}

% Users will explore the {\bf \generator} (A) to identify confounding factors that might distort a cause-outcome relationship ({\bf R1 -- confounder identification}), then they will make use of {\bf \partition} (B) to produce a set of meaningful subgroups to further examine their differences {\bf \group}, from the aspects of both causality and features ({\bf R2} -- subgroup heterogeneity). In face of possible paradoxical or conflicting associations, users will interact with {\bf \storyboard} (D) for detailed explanations ({\bf R3} -- reasoning). Finally, users will refer to the interface {\bf \diagnosis} (E) when making a decision regarding the effect of a treatment ({\bf R4} -- decision-making). That said, our system assists users answer a series of questions in the workflow of studying empirical associations. The distinguishing contribution of our work lies in its ability -- visually and interactively -- to locate and explain association spuriousness ({\bf \storyboard}), as well as to explicitly reveal the heterogeneous subgroup patterns from both aspects of causality and features ({\bf \partition} and {\bf \group}).
\subsection{\confounderdashboard}\label{sec:systemdesign:confounderdashboard}
The \confounderdashboard interface in Fig.~\ref{fig:system_overview}A enables users to set up cause/outcome variables and to locate the most likely confounders that are distorting the specified cause-outcome relationship ({\bf T1}). The flexibility of the system allows users to tailor their investigations to their specific research interests. For example, if someone is curious about the effectiveness of the job training programs in boosting annual earnings, they can select {\tt take\_training\_program} as the cause and {\tt yr1978\_earning} as the outcome. But if users are interested in the impact of marriage status on income, they can choose {\tt married} as the cause and {\tt yr1978\_earnings} as the outcome. To guide users in reflecting the causal relationships among covariates with the cause and outcome, \confounderdashboard provides a textual explanation of confounders---``{\it a confounder is a third variable that influences both cause and outcome yet does not lie on a causal pathway between cause and outcome}''---when hovering over the question mark next to confounder selection box in Fig.~\ref{fig:system_overview}A. Furthermore, it also provides a table where covariates are ranked by CF scores (ref. Section~\ref{sec:methodology:cf_metrics}) from the highest to the lowest. \tocomments{In the third column, we utilize two side-by-side, vertically aligned histograms \cite{cheng2020dece,ahn2019fairsight} to depict the feature distribution of the untreated group (blue) and the treated group (purple). When $X$ is a continuous variable, it will be discretized into two bins as the ``pseudo'' untreated/treated groups.} 
In \confounderdashboard, users can draw upon multiple information resources, including their domain knowledge regarding the causal structure of cause/outcome and covariates, the statistical CF scores, and the visual information of feature distribution among treatment groups, to select the most likely confounders and put them into the confounder box.

% {After initializing the system and loading data, users can use the \confounderdashboard interface shown in Fig.~\ref{fig:system_overview}A to set up all the required variables for analysis. First, users need to select a pair of cause and outcome variables as their analysis goal. For example, if the user wants to determine if the job training program increases participants' earnings using the aforementioned Lalonde data, they should select {\tt take\_training\_program} as the cause and {\tt yr1978\_earning} as the outcome. If the user is interested in exploring whether marriage status influences participants' earnings, they should set the cause and outcome variables as {\tt married} and {\tt yr1978\_earnings}.}


\subsection{\partition}\label{sec:systemdesign:partition}
The {\partition} interface in Fig.\ref{fig:system_overview}B \tocomments{allows users to create subgroups using two methods}---{\tt MANUAL} and {\tt AUTO}---based on a set of features\footnote{The partitioning variables could also be thought of as confounding variables, but they are special in the sense that they are a subset of the confounders with respect to which we want to capture subgroup heterogeneity.} ({\bf T2}). 
\tocomments{Previous subgroup analysis systems have utilized attribute-value pairs to construct subgroups \cite{cheng2020dece, kwon2022rmexplorer, kwon2020dpvis, guo2023causalvis}. \vispur employs a similar design, enabling users to add or remove covariates and specify the corresponding cut points. As shown in Fig.~\ref{fig:system_overview}B, two variables, {\tt black} and {\tt yr1975\_earning}, are selected, multi-thumb sliders are utilized to determine the cut points. A histogram distribution is displayed above the sliders, providing users with a visual reference for selecting appropriate cut points.}
Alternatively, users can opt for the {\tt AUTO} option, which uses algorithm-supported automated partitioning (ref. Section\ref{sec:methodology:algorithm}). By specifying a few configurations, such as the expected number of subgroups and the minimum size of subgroups, users can easily obtain an algorithm-generated partition that mitigates within-subgroup confounding bias. When clicking the {\tt Submit} button, subgroups based on the partition are generated in \subgroupviewer.

% {${\tt yr1974\_earning \in [7.5, 10.5)}$ $\&$ ${\tt black = True}$, ${\tt yr1974\_earning \in [7.5, 10.5)}$ $\&$ ${\tt black = False}$, ${\tt yr1974\_earning \in [0, 7.5)}$ $\&$ ${\tt black = True}$, as well as ${\tt yr1974\_earning \in [0, 7.5)}$ $\&$ ${\tt black = False}$.}

% DECE: users can refine the group by changing ranges for each feature and click the update button. The users can copy or delete an unwanted subgroup to maintain the subgroup list.
% CausalVIS, RMExplorer: users can click a particular name to facet the visualization by this variable, up to three variables can be selected this way.
% DPVis: a list of pairs of an attribute name and its value range.
\subsection{\subgroupviewer}\label{sec:systemdesign:subgroupviewer}

The {\subgroupviewer} interface in Fig.\ref{fig:system_overview}C provides a comprehensive overview of subgroup patterns, enabling users to understand their heterogeneous characteristics ({\bf T3}). Our design considers subgroup differences not only at the attribute level, but also at the causal behavioral patterns. To achieve this, we have created two views in this panel: {\causalityspace} for exploring causal patterns and {\covariatespace} for analyzing attributes. To ensure a seamless user experience, these two spaces are coordinated with consistent color codings over subgroups. Users can select a subgroup in either space and the interactions will be reflected in both spaces, or hover over subgroups to examine more detailed information.

% {Referring to the causal framework in Fig.\ref{fig:teaser}B (ref. Section~\ref{sec:methodology:diagram}), we focus on subgroup-level {\it propensities}---how likely people in a chosen subgroup are to participant in the training program, {\it base effects}---what are the annual earnings individuals in a subgroup typically obtain in 1978, and cause-outcome {\it associations}---to what extent cause and outcome are correlated.}

\input{images/2dspace_caption}

\subsubsection{\causalityspace} This view is shown in Fig.~\ref{fig:system_overview}C, illustrating subgroups' patterns in terms of {\it propensity}, {\it base effect}, and {\it association}. In particular, it represents subgroups in a two-dimensional coordinate space with the horizontal axis as cause and vertical as outcome. Depending on data types, subgroups are represented by either the circle-line design \cite{rucker2008simpson} or the elliptical geometry design \cite{friendly2013elliptical}, as shown in Fig.~\ref{fig:system_overview}C-1, C-2.

{The {\bf circle-line design} (Fig.~\ref{fig:2dspace}A) \tocomments{has been proposed to visualize Simpson's paradox \cite{rucker2008simpson}, particularly depicting cause-outcome} associations where \tocomments{cause is dichotomous and outcome is either dichotomous or continuous.} Each pair of two circles connected by a line represents a subgroup, whose members are divided into control and treated group each represented as a circle with its size indicating the number of subjects. When comparing the sizes of two connected circles, it reveals the treatment propensity (e.g., subgroup 1 has a stronger propensity than subgroup 3). Circles are positioned along two-sided vertical axes indicating the average value of outcome, with the slope between two circles revealing cause-outcome associations. By comparing the distinguished aggregate (grey) against the subgroup-level circle-lines, users can quickly locate a paradoxical phenomenon, such as the aggregate is positive while subgroup 2 is negative.}
{The {\bf elliptical geometry design} \tocomments{is discussed in \cite{friendly2013elliptical} to visualize Simpson's paradox, for cases} where \tocomments{both cause and outcome are continuous} (see Fig.\ref{fig:2dspace}B). The plot consists of colorful ellipses, each representing a subgroup, while a gray ellipse denotes the aggregate to facilitate comparison between the overall data trend and subgroup-level patterns.
The centroid of each ellipse displays the average values of cause and outcome suggesting propensity and base effect, denoted as $(\bar{X},\bar{Y})$ \cite{friendly2013elliptical}. Additionally, the half-widths of the vertical and horizontal projections of each ellipse geometrically reflect  the standard deviations of cause and outcome, represented by $s_X$ and $s_Y$ (Fig.~\ref{fig:2dspace}). The trend of a regression line that passes through the ellipse centroids is indicative of cause-outcome associations.}

% {In terms of interactions, users could mouse over subgroups to examine more detailed information. We note that, in both designs, we utilize slopes to demonstrate cause-outcome associations so that users are able to grasp diverse trends and to locate paradoxical associations. As the true causal effects are always hidden in observational studies, we provide more information in \decisiondiagnosis interface, in Fig.~\ref{fig:system_overview}E, to assist users to conduct spuriousness diagnosis.}

\subsubsection{\covariatespace} 


% In information visualization, a glyph refers to a small and compact graphic representation that represents a data point with multidimensional features. 
% Compared with other multidimensional visualization techniques, such as multidimensional scaling (MDS)17, parallel coordinates19, scatterplot matrices, and various advanced designs for reducing clutter in multidimensional data25 or for representing data from heterogeneous dimensions26–30, glyphs transform mul- tidimensional data features to composite visual prop- erties (such as shape, color, and size), producing various ‘‘visual signatures’’ of data points that reveal more complex data patterns and offer a richer descrip- tion about data points.

This view (Fig.\ref{fig:system_overview}C-3) represents the characteristics of subgroups using multivariate radar glyphs \tocomments{due to its compactness and richness. Unlike other multidimensional visualizations (e.g., scatterplot matrices, parallel coordinates \cite{inselberg2009parallel}), glyph visualization encodes complex data features to compact ``visual signatures'' of distinct subgroups \cite{cao2018z}.} The leftmost glyph represents the entire population and the rest of them represent subgroups. In a radar glyph, the axes represent features with dots along the axes indicating the mean values of those features as a summary. To account for the differences in feature scales, we normalize their values into a uniform range of zero to one using $\bar{F}_j = (F_j - \mathrm{min}(F_j)) / (\mathrm{max}(F_j) - \mathrm{min}(F_j)), \forall j$. For categorical features with $J$ discrete values, we convert them into $J - 1$ binary features. In the radar glyphs, we only show the most discriminative top-$k$ features (e.g., $k=5$). To measure a feature's discriminativeness, we trained a series of one-vs-rest binary classifiers with the chosen feature $F$ as input and subgroup labels as output. We then computed an aggregated AUC score as a measurement of feature discriminativeness, where a higher AUC value suggests a stronger ability to distinguish data points from different subgroups. Users have the freedom to select features to investigate by clicking ``Choose Axes.'' E.g., the five axes in Fig.\ref{fig:system_overview}C-3 include {\tt age}, {\tt hispanic}, {\tt no\_degree}, {\tt education\_years}, and {\tt married}.

% {Users could clearly observe the phenomenon of Simpson's paradox by comparing the aggregate trend with any of the subgroup-level trends. If a subgroup were non-identifiable (ref. \ref{sec:methodology:framework:assumption}), our design would clearly highlight it by showing a single circle--either the treated or the untreated---in Fig.~\ref{fig:2dspace}(a). To support interactive data exploration, \vispur allow users to mouse over a subgroup to see the detailed statistics (e.g., estimated effect size and its bootstrap confidence interval), and to highlight (by clicking) two or more subgroups for between-subgroup comparison.}



\input{images/case_study_caption}

\subsection{\reasoningstoryboard}\label{sec:systemdesign:reasoningstoryboard}
\input{images/storyboard_caption}
{To aid in understanding association conflicts, we design {\reasoningstoryboard}, depicted in Fig.~\ref{fig:system_overview}D, which complements {\subgroupviewer} by providing a narrative for the appearance of a conflicting/paradoxical phenomenon ({\bf T4}).} {The diagram comprises three layers: {\tt subgroup}, {\tt cause}, and {\tt outcome}. The {\tt subgroup} node is a rectangle scaled to the size of the chosen subgroup (e.g., the number of participants). In the {\tt cause} layer, multiple nodes depict possible treatments. \tocomments{For a dichotomous treatment, nodes are limited to {treated} and {untreated}; For a continuous treatment, the values are discretized into $L$ ($L = 4$ in our demonstration) bins based on percentiles and ranked in a descending order.} The height of a cause node represents the percentage of participants taking that action. The {\tt outcome} layer is a vertical axis, where the top endpoint indicates the maximum value of the outcome and the bottom indicates the minimum. Pathways originate from the {\tt group} node, traverse through relevant {\tt cause} nodes, and terminate at a specific point along the {\tt outcome} axis. The height of each pathway represents the proportion of participants who took a particular treatment action at each {\tt cause} node. The endpoint of each pathway indicates the average outcome achieved (e.g., average earnings in 1978).}

% little work has studied the flow-based visualization's capability of demonstrating and interpreting a paradoxical phenomenon. Our storyboard design holds contributions by recognizing that the shapes of flows carry valuable information for interpreting paradoxical phenomena.

\tocomments{The storyboard visualization extends the well-known Sankey diagrams \cite{riehmann2005interactive} and parallel sets \cite{kosara2006parallel}, which have been found useful in depicting storylines \cite{kwon2020dpvis} and multidimensional features \cite{kosara2006parallel}. Despite their extensive applications in flow-based visualizations, our contribution lies in presenting paradoxical patterns using flow-based narratives to facilitate the interpretability of such complex phenomena.} Fig.~\ref{fig:storyboard} illustrates three common shape patterns: (A) {\bf pass-through}, (B) {\bf hill}, and (C) {\bf valley}, where (A) depicts the population-level pattern, and (B,C) depict two subgroups' patterns. The {\bf pass-through} shape in Fig.\ref{fig:storyboard}A suggests a {\it negative} cause-outcome association because the subset of subjects taking the largest value of treatment (top) end up having the smallest value of outcome. In contrast, two subgroups in Fig.~\ref{fig:storyboard}B,C exhibit {\bf non-pass-through} (thus {\it positive}) associations. To interpret the reversed associations, users might further investigate Fig.~\ref{fig:storyboard}B,C. Users can see that subgroup (B) tends to take a high-valued treatment but its outcome is relatively low ({\bf hill}), whereas subgroup (C) likes to take a low-valued treatment but its outcome is relatively high ({\bf valley}). When the two mirrored {\bf valley} and {\bf hill} pathways are aggregated together, they could generate a {\bf pass-through} pattern as shown in Fig.~\ref{fig:storyboard}A. {The Lalonde dataset in Fig.~\ref{fig:system_overview}D shows that the valley shape displayed by {\tt GroupID 3} might cancel out the hill patterns of other subgroups, leading to a zero-effect pattern overall. Users can mouse over pathways for detailed information, select subgroup diagrams by clicking the ``add'' icon, and perform shape comparison and paradox reasoning ({\bf T4}).}

% {The visual display of storyboard looks similar to Sankey diagrams \cite{riehmann2005interactive} and parallel sets \cite{kosara2006parallel}. However, Sankey diagrams focus on displaying the proportion of the flow that splits in different ways as well as the pathways through a set of discrete stages, without remarking how different pathways arrive at certain outcomes. Parallel sets, though similar in a way of showing flow and proportions, focus on multidimensional data thus do not enforce the sequential order along the pathways or explain how different stages leading to a certain outcome. We rely on the direction of flows to represent the sequential order along the pathways of taking treatment actions and then obtaining outcomes, and use the width of flows to reveal proportional splits of participants.}

% {The storyboard visualization shares some similarities with Sankey diagrams \cite{riehmann2005interactive} and parallel sets \cite{kosara2006parallel}, but it differs in the sense that it emphasizes the sequential order of pathways and explains how different stages lead to certain outcomes. Sankey diagrams focus on displaying proportions of flow while parallel sets focus on multidimensional data and do not enforce the sequential order along pathways. The storyboard visualization uses the direction of flows to represent the sequential order along pathways and the width of flows to reveal proportional splits of participants.}



% {exhibits two converging pathways (treated/control) leading to a same endpoint. The {\bf valley} pattern of {\tt GroupID 3} indicates that some individuals show little interest in the program and still earn relatively higher income than others (probably because they do not need any training). Such a {valley} shape can dilute the positive effects of others, resulting in a zero-effect pattern overall. With our \vispur system, users can mouse over pathways for detailed information, select subgroup diagrams by clicking the ``add'' icon, and perform shape comparison and paradox reasoning ({\bf T4}).}



% The covariate $X$ influences both treatment choices and outcome values: units of $X = 0$ tend to take lower ranked treatment, but their outcome value is relatively high ({\it valley}); On the other hand, units of $X = 1$ tend to take higher ranked treatment, but their outcome value is relatively low ({\it hill}). These two factors result in a negative overall association. To summarize, the {\it mirror-like} pattern between two subgroups suggest a confounding bias.


\subsection{\decisiondiagnosis}\label{sec:systemdesign:decisiondiagnosis}

To aid decision-making ({\bf T5}), \vispur offers {\decisiondiagnosis} interface (Fig.\ref{fig:system_overview}E) to detect nested confounding bias and spurious associations. Users can select subgroups using the dropdown selector and view statistical results (e.g., effect size, bootstrap CI, $p$-value, sample size) in the {\basicstatistics} table. If the subgroup-level coefficient contradicts the overall association, a warning message---``{Simpson's Paradox}''---is displayed to alert users (Fig.\ref{fig:system_overview}E-1). To detect nested confounding biases in subgroups, we offer the {\imbalancechart} in Fig.~\ref{fig:system_overview}E-2, which computes an {imbalance score} (ref. Section~\ref{sec:methodology:imbalance}) to compare treated and untreated participants within each subgroup. \tocomments{It shares a similar design as the visual components built in Cobalt \cite{greifer2020covariate} and Causalvis \cite{guo2023causalvis} for the purpose of balance checking.} Confounding variables are ranked on the vertical axis by their imbalance score, represented by horizontal position and lollipop size. A warning message will appear when the average imbalance score exceeds a threshold of 0.2, with corresponding lollipops enlarged for visual emphasis. The chart includes two switches for the entire population and the selected subgroup. Users can hover over lollipops to view values and compare pre- and post-partition imbalance scores by toggling the state (on/off) of two switches. {The Lalonde example in Fig.~\ref{fig:system_overview}E-2 demonstrates that partition reduces a large amount of confounding bias caused by {\tt black, yr1974\_earning, yr1975\_earning, married}, however, this chosen subgroup still triggers a warning because of the residual confounder {\tt yr1975\_earning}.}

 


\tocomments{More details of our iterative development, design choices and considerations are provided in Supplementary Materials \ref{sec:appendix}.}