\section{Discussion}\label{sec:discussion}

{We discuss the usability of \vispur system, common human behaviors in using it, and the limitations as well as future directions.}

{\bf Usability of \vispur's components.} Study participants and experts provided positive feedback on the usability and effectiveness of \vispur. We especially found that The \confounderdashboard and \subgroupviewer (along with \partition) components were heavily used by human users to examine distortion effect from confounders and the mixed effect from subgroups. As commented by a subject: ``{\it \vispur let me know more information about subgroups than baseline.}'' When asked to explain Simpson's paradox, four participants mentioned that \vispur's flow charts in \reasoningstoryboard were more intuitive than baseline, as it ``clearly shows data splits'' and ``how data flows through cause to outcome.'' Besides, many users reported that it reminded them of overlooked information and prevented them from making imprudent decisions. The warning messages and imbalance chart features in \decisiondiagnosis were praised by participants and experts for triggering further thinking and helping to locate residual confounders. Additionally, \vispur was found to be useful in generating and validating hypotheses, as well as delivering analysis results to decision-makers or colleagues. As Micheal commented, ``{\it I do not need to plot and explain a lots of figures by myself, I can just modify partitions and discuss the visualizations produced by \vispur with my boss.}''

{\bf Human behaviors in paradox reasoning and decision-making.} Although \vispur improved participants' answering accuracy, questions related to paradox reasoning and decision-making still posed a challenge. When asked to agree or disagree with an explanation for paradoxical associations, many participants struggled to connect observed visual evidence to a final statement. Additionally, many participants relied on their subjective tendency when making a decision, rather than utilizing the system. For instance, one participant stated that they recommended a person take a training program because ``{\it taking a training shouldn't be bad},'' regardless of \vispur's visualizations. To aid such challenging cognitive tasks, a possible solution in the future could be to support interactive collaborations on a visual analytic system among users. For instance, users can highlight visualizations, leave their interpretations, and share ideas to reach a final conclusion.
% {E.g., they could find visual evidence that was concordant or discordant to the first part of the description: {\it the old participants are more likely to take the online course while the reverse is true for the young; meanwhile, they also tend to have a higher likelihood of passing the exam than the young}, however, they made mistakes whilst connecting such visual evidence to the final statement: {\it these two factors---``those who likely to take the treatment happen to be those who likely to succeed''---result in a spurious positive association overall}.}


{\bf Limitation and future work.} We summarize the limitation and the possible future work to tackle those issues.

\begin{enumerate}
    \item \tocomments{\bf Mental causal model and violation of causal framework.} {Although \vispur offers qualitative scores and visual signals to identify confounders, it is important to note that human beliefs and assumptions about causal structure are still necessary. Relying solely on statistical criteria can lead to adjusting for undesirable variables if there is insufficient causal knowledge about the sources of confounding \cite{hernan2020causal}. To mitigate this issue, the use of causal DAG has been suggested and implemented in many causal reasoning tools \cite{yen2019exploratory}. These DAGs capture users' mental models of how variables are related and facilitate reflection on common causes of treatment and outcome as a sufficient set of adjustment variables. In future work, \vispur could be expanded to include a DAG panel that allows users to draw DAGs before exploring confounding tendency scores and visual information. This would encourage causal reflection and explicitly incorporate domain knowledge.}
    
    \item {\bf Scalability.} \vispur is capable of generating subgroups using an unlimited list of partition rules or by utilizing machine learning algorithms to search for optimal partitions. However, the system has limited scalability when dealing with a large number of subgroups. In such cases, the \causalityspace may become crowded, making it difficult to visually distinguish between different subgroups. Future work could address this issue by introducing flexible subgroup operations, such as hierarchical subgrouping, filtering, hiding, highlighting, and zooming in/out, to mitigate issues such as overlapping visualizations and information overload. These operations would allow for effective exploration and comparison of subgroups, ultimately improving the usefulness of the system.
    \item {\bf Imbalance representation.} The \imbalancechart in \decisiondiagnosis (ref. Section~\ref{sec:systemdesign:decisiondiagnosis}) effectively identifies variables with high imbalance scores. However, users with limited knowledge of causality may struggle to grasp the concept of imbalance. Future work could focus on designing an intuitive visual representation to enhance users' understanding, such as visualizing the association between pretreatment variables and treatment, or showcasing the distributions of pretreatment variables across treatment arms.
\end{enumerate}

% {{\bf System guidelines.} \ysc{Shall we remove this part?} Participants in our study session had the chance of getting familiar with \vispur by watching a tutorial video, followed by a self-exploration practice, but a desire for embedded guidelines was a consistent theme to make the system more self-contained and independent from external training. We observed that, participants might forget the meanings of certain visual encodings (e.g., size of circles, axes of radar charts etc), or had no idea what actions (e.g., click, select) to perform to retrieve useful information. Without appropriate guidelines, some participants got confused regarding the variables used in different places, such as in confounder table, in radar charts, and in imbalance chart. A participant commented: ``{\it I hope the system could automatically remind me where to click and what next step to take.} A participant expressed the desire for a new functionality that users could generate analysis reports for teamwork communication and result delivery.}