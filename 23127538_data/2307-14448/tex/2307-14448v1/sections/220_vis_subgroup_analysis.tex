\subsection{Visual Analytics for Subgroup Analysis}

Subgroup analysis has been a popular topic in data-driven decisions, because an aggregated pattern is not always generalized to (even differ from) that of subgroups. Visual analytic systems have been designed to support a variety of subgroup-level analysis, such as visualizing clusters and features \cite{kwon2017clustervision,furmanova2020taggle,blumenschein2018smartexplore,ahn2022tribe}, \tocomments{understanding event sequences or disease progressions \cite{jin2020visual,kwon2020dpvis}, as well as examining model behaviors over subsets of data \cite{dingen2018regressionexplorer, kwon2022rmexplorer, wexler2019if, cabrera2019fairvis, gleicher2020boxer, cheng2020dece}.} For example, Taggle \cite{furmanova2020taggle} is a tabular design to visualize high-dimensional data in terms of record clusters, subspaces, correlations, and pattern similarity across different levels of stacked aggregation.
% { In contrast to directly showing records and features in a table, many systems attempt to visualize clusters with various clustering algorithms through different views (e.g., scatter plots, parallel axes, etc).}
% {ClusterVision \cite{kwon2017clustervision} is designed to compare the clustering results obtained from a series of competing clustering techniques, and rank them through five quality metrics encoded in a radar chart.}
\tocomments{DPVis \cite{kwon2020dpvis} focuses on event sequence data and allows users to investigate heterogeneous disease progression pathways of patient subgroups.} \tocomments{In addition to static/sequential data properties, many tools have been developed to understand the performance of machine learning models over subsets of data. Examples include DECE \cite{cheng2020dece},} What-If Tool \cite{wexler2019if}, Boxer \cite{gleicher2020boxer}, and FairVIS \cite{cabrera2019fairvis} among others. Relevant systems include RegressionExplorer \cite{dingen2018regressionexplorer} and RMExplorer \cite{kwon2022rmexplorer}. RegressionExplorer is tailored for logistic regression model analysis, supporting dynamic subgroup generation and visualizations of subgroup-level parameters. RMExplorer enables users to define patient subgroups based on various characteristics and assess the performance and fairness of risk models on these subgroups \cite{kwon2022rmexplorer}. \tocomments{DECE \cite{cheng2020dece} shares design similarities with our system, as it enables users to create subgroups using multi-feature decision rules and provides contrastive feature comparison through side-by-side histograms.} However, there is a limited amount of work on developing visual analytic systems specifically for investigating subgroup patterns in the space of causality.

{Our system \vispur examines heterogeneous subgroup patterns on the basis of a causal framework (Fig.~\ref{fig:teaser}B), revealing how likely different subgroups take a treatment, how likely they obtain a target outcome, along with the subgroup-level cause-outcome relations. Given those information, \vispur explains how subgroup-level behaviors are linked to an overall spurious association or a Simpson's paradox.}


