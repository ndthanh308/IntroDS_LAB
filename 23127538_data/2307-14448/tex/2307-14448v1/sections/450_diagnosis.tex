\subsection{\decisiondiagnosis}\label{sec:systemdesign:decisiondiagnosis}

To aid decision-making ({\bf T5}), \vispur offers {\decisiondiagnosis} interface (Fig.\ref{fig:system_overview}E) to detect nested confounding bias and spurious associations. Users can select subgroups using the dropdown selector and view statistical results (e.g., effect size, bootstrap CI, $p$-value, sample size) in the {\basicstatistics} table. If the subgroup-level coefficient contradicts the overall association, a warning message---``{Simpson's Paradox}''---is displayed to alert users (Fig.\ref{fig:system_overview}E-1). To detect nested confounding biases in subgroups, we offer the {\imbalancechart} in Fig.~\ref{fig:system_overview}E-2, which computes an {imbalance score} (ref. Section~\ref{sec:methodology:imbalance}) to compare treated and untreated participants within each subgroup. \tocomments{It shares a similar design as the visual components built in Cobalt \cite{greifer2020covariate} and Causalvis \cite{guo2023causalvis} for the purpose of balance checking.} Confounding variables are ranked on the vertical axis by their imbalance score, represented by horizontal position and lollipop size. A warning message will appear when the average imbalance score exceeds a threshold of 0.2, with corresponding lollipops enlarged for visual emphasis. The chart includes two switches for the entire population and the selected subgroup. Users can hover over lollipops to view values and compare pre- and post-partition imbalance scores by toggling the state (on/off) of two switches. {The Lalonde example in Fig.~\ref{fig:system_overview}E-2 demonstrates that partition reduces a large amount of confounding bias caused by {\tt black, yr1974\_earning, yr1975\_earning, married}, however, this chosen subgroup still triggers a warning because of the residual confounder {\tt yr1975\_earning}.}

 
