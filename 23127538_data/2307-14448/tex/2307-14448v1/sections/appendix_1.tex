\section{Supplementary Materials}

\tocomments{{\bf System Design Choices and Considerations.} To design our system, the notion of ``subgroup'' is a central concept, as (1) it is through the aggregation of subgroups that a Simpson's paradox emerges, and (2) this concept bears causal implications in terms of heterogeneous causal effects. In our study, the design of \vispur started with defining the core task: {\it visualizing subgroup heterogeneity} to interpret a paradoxical or spurious association, along with the specific design requirements ({\bf R1---R4}) listed in Section~\ref{sec:designguideline:requirement}. Besides, the causal diagram in Fig.~\ref{fig:teaser}B suggests that our visual design should clearly capture a set of six key elements, including three entities, i.e., cause $X$, outcome $Y$, covariates $Z$ (or group membership), and three arrows, i.e., propensity $\mathbf{Z}\rightarrow T$, base effect $\mathbf{Z}\rightarrow Y$, cause-outcome relationship $X\leftrightarrow Y$. Given such considerations, we investigated five designs in existing works of visualizing Simpson's paradox, including the B-K diagram \cite{baker2001good}, platform scale \cite{rum1980magic}, comet \cite{armstrong2014visualizing}, circle-line design \cite{rucker2008simpson}, as well as ellipse design \cite{friendly2013elliptical}. Table~\ref{tab:comp} provides a summary of advantages and disadvantages of the five visual diagrams examined in our design decision. Table~\ref{tab:visulizing_SP} summarizes whether and how the six key elements are represented in those five visual designs.}

\tocomments{Among them, the design of platform scale includes a set of blocks arranged in stacks of varying heights being placed on a platform and balanced on a pivot at the center of gravity \cite{rum1980magic}. Although the gravity-based design is intuitive, it is difficult to visually comprehend how a static equilibrium could be achieved when more than two subgroups (i.e., stacks of blocks) are present. As the blocks of the same subgroup are placed on different platforms (corresponding to different treatment options), it is not easy to examine the properties of a chosen subgroup. All of the other four designs have utilized a two-dimensional coordinate space, where the vertical axis represents outcome, but the horizontal axis holds different meanings in different designs. For example, the B-K diagram considers the horizontal axis as the percentage of a chosen subgroup (two subgroups in total), therefore it does not have a explicitly visual representation of more than two subgroups, failing to satisfy our design requirement \cite{baker2001good}. The design of comet consists of a set of comets (i.e., subgroups) placed in a two-dimensional space, and the motion of a comet from head to tail captures its change from the starting time point A to the ending time point B \cite{armstrong2014visualizing}. Although it well preserves the notion of subgroup and enables a straightforward comparison of subgroup patterns, the strong sense of motion conveyed by comets only works well for time-based data rather than a more general treatment scenario. We compare the pros and cons of all possible visual representation candidates, and decided that circle-line \cite{rucker2008simpson} and ellipse diagram \cite{friendly2013elliptical} are most suitable to meet the required elements and data types. The first reason is representation consistency. They share the same two-dimensional cause-outcome space, with the horizontal axis as cause and vertical as outcome, and the slope of regression lines as cause-outcome relationship. Being represented in the cause-outcome space, circle-line plot is designed for a dichotomous treatment while ellipse diagram for a continuous treatment. Secondly, the diagram, either a circle-line plot or a data ellipse, is a simple and effective summary of raw data points in the two-dimensional space. They are good at simplifying a possibly large-scale data set and capturing the most important information (e.g., the set of six key elements, including three entities and three arrows, are well represented, see Table~\ref{tab:visulizing_SP}). The third reason is that they enable a good representation and comparison of multiple subgroups. Both provide a high-level overview of multiple subgroups and enable inter-subgroup comparison; besides, an aggregate plot for the entire data (circle-line or ellipse) could be shown to alert Simpson's paradox.}

\tocomments{During the system design and development process, we conducted two rounds of open-ended discussion and feedback collection sessions with 5 colleagues in our research lab. We invited them to raise any concerns and suggestions to improve our system. For instance, the feature histograms in {\confounderdashboard} were initially designed to be overlapped, but users might have difficulty in reading distinctions when two distributions look similar. We decided to put them side by side vertically to give a straightforward comparison. The initial design of multi-thumb sliders didn't contain the associated feature distribution. Users reported that they sometimes had no idea where to put the cutting points. We then follow their suggestions to visualize a histogram distribution to guide the selection of proper cut points. In addition, to design the {\reasoningstoryboard}, we have tested two options to visualize data volume, through flow width versus through color darkness. Our users reported that width is a much clearer visual signal for them to comprehend.}

\begin{table*}
 \centering
 \small
 \caption{\tocomments{Comparison of existing visualization designs for Simpson's paradox. It provides a summary of design ideas, advantages and disadvantages of each of the designs.}}
 \label{tab:comp}
 % \begin{adjustbox}
 \begin{tabular}{ p{0.1 \linewidth} p{0.35 \linewidth} p{0.2 \linewidth} p{0.25 \linewidth} }
    \toprule
    {\bf Graph} & {\bf Design summary} & {\bf Advantages} & {\bf Disadvantages} \\
    \midrule
    {\bf B-K diagram} \cite{baker2001good} & 
    A 2D coordinate space where the horizontal line is the proportion of a certain group (e.g., women) and the vertical line is outcome. Lines in the space demonstrate different treatment, vertical positions indicate outcomes. & 
    Well preserve the outcome differences of distinct treatments using vertical positions of dots or lines. & 
    No clear representation of subgroups, thus does not support comparison of subgroups' properties. \\ \\
    {\bf Platform scale} \cite{rum1980magic} & 
    A set of blocks arranged in stacks of varying heights is located on a platform and balanced on a pivot at the center of gravity. Treatments are represented by multiple platforms, outcomes are by positions of blocks, treatment preferences are by heights of blocks. & 
    Well represent weights (i.e., treatment preferences) by heights of blocks. &
    Hard to visually understand how a static equilibrium is achieved when multiple subgroups (stacks of blocks) are present; Hard to examine subgroup-specific properties as blocks of the same subgroup are placed on different platforms. \\ \\
    {\bf Comet} \cite{armstrong2014visualizing} &
    A set of comets placed in a 2D coordinate space, where each comet is a subgroup, with the thin head being one treatment scenario (A) and the thick tail the other (B). The motion of a comet indicates how the changes of properties from scenario A to B. & 
    Well represent multiple subgroups, and easily compare difference of subgroup patterns, an aggregate comet is shown to alert Simpson's paradox. & 
    The strong sense of motion conveyed by comets only works well for time-based data rather than continuous or categorical treatment. \\ \\
    {\bf Circle-line plot} \cite{rucker2008simpson} &
    A set of circle-line plots placed in a 2D coordinate space, where each circle-line plot is a subgroup with its slope representing per-subgroup association, horizontal axis being treatment, vertical axis being outcome, and circle size represents sample size. &
    Well represent multiple subgroups, and easily compare difference of subgroup patterns, an aggregate circle-line is shown to alert Simpson's paradox. & 
    Large circles obscure small ones if many subgroups exist.  \\ \\
    {\bf Ellipse} \cite{friendly2013elliptical} &
    A set of ellipses placed in a 2D coordinate space, where each ellipse is a subgroup with its slope representing per-subgroup association, horizontal axis being treatment, vertical axis being outcome. &
    Well represent multiple subgroups, and easily compare difference of subgroup patterns, an aggregate ellipse is shown to alert Simpson's paradox. &
    Large ellipses obscure small ones if many subgroups exist. \\
    \bottomrule
 \end{tabular}
 % \end{adjustbox}
\end{table*}

\begin{table*}
\small
\centering
\caption{\tocomments{Comparison of existing designs for Simpson's paradox. It summarizes how six key elements (see Fig.~\ref{fig:teaser}B)---three entities (cause, outcome, subgroup) and three arrows (treatment propensity, base effect, cause-outcome relationship)---are represented using visual encodings in each of the designs.}}
\label{tab:visulizing_SP}
% \begin{adjustbox}{width=\textwidth}
\begin{tabular}{ p{0.1\linewidth} p{0.1\linewidth} p{0.1\linewidth} p{0.1\linewidth} p{0.1\linewidth} p{0.1\linewidth} p{0.1\linewidth}} 
 \toprule
 {\bf Diagram} &  {\bf Subgroup} & {\bf Cause} & {\bf Outcome} & {\bf Treatment propensity} & {\bf Base effect} & {\bf Cause-outcome association} \\
 \midrule
 {\bf B-K diagram} \cite{baker2001good} & - & line segments & y-axis & x-coordinate of data points along line segments & endpoints of line segments & slope of line segments \\ \\
 {\bf Platform scale} \cite{rum1980magic} & block labels & platforms & positions of blocks on platforms & height of blocks & a default stack of blocks & relative positions of blocks with the same label \\ \\
 {\bf Comet} \cite{armstrong2014visualizing} & comets & endpoints of a comet & y-axis & x-coordinate of a comet & y-coordinate of a comet's head & a comet's length along y-axis \\ \\
 {\bf Circle-line plot} \cite{rucker2008simpson} & line segments & 0/1 on x-axis & y-axis & circle size & intercept on y-axis of line segments & slope of line segments \\ \\
 {\bf Ellipse} \cite{friendly2013elliptical} & ellipses & x-axis & y-axis & positions of ellipses along x-axis & positions of ellipses along y-aixs & slope of regression lines \\
 \bottomrule
\end{tabular}
% \end{adjustbox}
\end{table*}



\iffalse
\begin{table}[h!]
\small
\centering
\caption{Comparison of SP visual representations.}
\label{tab:visulizing_SP}
\begin{adjustbox}{width=\textwidth}
\begin{tabular}{ P{0.16\linewidth} P{0.16\linewidth} P{0.16\linewidth} P{0.16\linewidth} P{0.16\linewidth} P{0.16\linewidth}} 
 \toprule
 {\bf Element} & {\bf B-K} & {\bf Platform scale} & {\bf Comet} & {\bf Triplet} & {\bf Ellipse} \\
 \midrule
 Subgroups & —— & block labels & comets & lines & ellipses \\ \\
 Treatment & lines & platforms (two platforms are women vs men) & endpoints of a comet & 0/1 on x-axis & x-axis \\ \\
 Outcome & y-axis &  position of blocks on platform & y-axis & y-axis of a comet's endpoints & y-axis \\ \\
 Propensity &  —— & height of blocks & x-axis of a comet & diamond size & positions of ellipses along x-axis \\ \\
 Prognostic tendency & endpoints of lines & one of the platform (e.g., men) & y-axis of a comet's head & line intercepts on y-axis & ellipse center along y-axis \\ \\
 Effect/Association & line slope & relative position of blocks with the same label & comet's length along y-axis & line slope & slope \\ \\
 Data type & discrete treatment, binary/continuous outcome & discrete treatment, binary outcome & time-based data & binary treatment & continuous treatment and outcome \\ \\
 Advantage & 
 well preserve the differences of treatment arms via vertical positions of dots or lines & 
 well represent weights (i.e., treatment preferences) by heights of blocks &
 well represent multiple subgroups, and easily compare difference of subgroup patterns, an aggregate comet is also shown to alert the presence of a Simpson's paradox & 
 well preserve subgroups, and easily compare difference of subgroup patterns, an aggregate plot is also shown to alert the presence of a Simpson's paradox & 
 well preserve subgroups, and easily compare difference of subgroup patterns, an aggregate plot is also shown to alert the presence of a Simpson's paradox \\ \\
 Limitation & no clear representation of subgroups & weighting scale is not suitable for many subgroups & only works well for time-based data & not able to handle continuous treatment & not suitable for nonlinear relation, no statistical assessment and interaction \\ 
\bottomrule
\end{tabular}
\end{adjustbox}
\end{table}
\fi



\tocomments{{\bf How Participants' Backgrounds Affect Performance in Using \vispur.} In the controlled experiments, we asked participants to report their level (1 to 5) of being familiar with a series of advanced causality topics including random experiments, causal inference, confounding, propensity score, and selection bias. We computed the average value of self-reported scores regarding those advanced topics, and computed the Spearman's rank correlation coefficient ($\rho$) between this score and participants' performance. As shown in Fig.~\ref{fig:bck_results}A, the correlation coefficients between expertise score and accuracy for both methods, \vispur and baseline, are not significant ($\rho_{\mathrm{VISPUR}} = 0.078, p = 0.74$, and $\rho_{\mathrm{Baseline}} = -0.099, p = 0.68$). Similarly, we didn't observe significant positive correlations between expertise score and certainty in Fig.~\ref{fig:bck_results}B ($\rho_{\mathrm{VISPUR}} = 0.097, p = 0.69$, and $\rho_{\mathrm{Baseline}} = 0.12, p = 0.63$). To summarize, we didn't find a strong evidence showing that participants of various causality backgrounds have very different performances. We note that our experiments are not free of limitations. For instance, our system is designed for a broader range of data practitioners, so the majority of our recruited participants are not experts of causality. The scatter plots in Fig.~\ref{fig:bck_results} shows that most participants reported to have a familiarity level at 1-3 with advanced causality topics. Recruiting and testing individuals with greater expertise in the field is critical for gaining a more complete understanding of how expert users perform when utilizing VISPUR compared to ordinary data practitioners. Due to the excessive time needed for collecting data from each participant, we did not administer a pre-study knowledge test to participants; therefore, our analyses may contain self-reporting errors.}

\input{images/bck_results_caption}

\begin{table*}[t]
  \centering
  \small
  \caption{Summary of our interview study. Colors represent four major challenges: \conebox{\bf C1}, \ctwobox{\bf C2}, \cthreebox{\bf C3}, \cfourbox{\bf C4}.}
  \label{tab:interviews}
   % \begin{adjustbox}{width=\textwidth}
   \begin{tabular}{ p{0.10\linewidth} p{0.20\linewidth} p{0.24\linewidth} p{0.36\linewidth}} 
    \toprule
    {\bf Interviewee} & {\bf Focal Interest in Analysis} & {\bf Practices/Tools} & {\bf Challenges/Needs} \\
    \midrule
    {Social worker ({\bf P1})} & {To what extent crime severity is linked to sentence, any differences among different populations (e.g., ethnicity, gender)?} & Fit one or multiple regression models by adjusting demographics features and examine coefficients and $p-$values; SPSS and R are used; & 
    \cone{Have difficulty in variable selection in regression and might overlook confounders;} \cfour{Unable to claim causal effects from statistical analysis;} \ctwo{Unable to automatically detect causal effects for subpopulations especially when multiple variables are involved;} Desire for a user-friendly, interactive system for data exploration; \\
    \midrule
    {Trading analyst ({\bf P2})} & {How and under what special conditions does previous return inform future return in the trading market?} & Build regression models and examine the coefficients and $p-$value; Rely on prior experience to manually search for a subset of market conditions and re-fit regressions; R and Python are used; & 
    \cone{Lack guidance in feature selection;} \cthree{Unable to interpret association change in different models;} \cfour{Unable to claim significance or make decisions given inconsistent associations in different models;} \ctwo{Desire for guidance in searching for market conditions where past return predicts future;} Desire for intuitive visualization to deliver/explain results to leadership; \\
    \midrule
    {Educational system designer ({\bf P3})} & {What is the impact of the designed educational system on students' performance, and how can we explain the counterintuitive pattern of ``more engagement with the system $\rightarrow$ worse performance''?} & Build (stepwise) regression models to examine coefficients and $p-$value; Multicollinearity examination and data-driven feature selection; R and Python are used; & 
    \cthree{Unable to interpret association changes in different models of distinct predictors;} \ctwo{How to define subgroups remain a challenge in face of many features, to reveal students' heterogeneous characteristics (e.g., who like to use the system, how the system affects students' performance);} \cthree{Desire for tools/techniques to explain the counterintuitive association found in their work;} \cfour{Hard to claim a strong conclusion regarding the effectiveness of the designed system;} Interpret the analysis results to people without background of causal inference is a challenge; \\
  \bottomrule
\end{tabular}
% \end{adjustbox}
\end{table*}

\input{images/baseline_caption}

\begin{table*}[ht]
  \centering
  \small
  \caption{The task of analyzing whether a college application training program helps a student in getting offers.}
  \label{tab:task1}
   \begin{tabular}{ p{0.95\linewidth} }
    \toprule
    {\bf Task 1. A high school has launched a college application training program, aiming to help students in getting admitted into colleges. You will be using a visual tool to explore a data set and figure out whether the association between ``taking college application training'' and ``being admitted into colleges'' is spurious. Please note, the treated are people who have taken the training program, while untreated those who didn't take it.} \\ 
    {\begin{itemize}
    \setlength\itemsep{1em}
        \item[{\bf Q1}] {\bf [Confounders]} Based on the given data and visualization, how do you agree with the following descriptions about comparing the treated/untreated?
        \begin{itemize}
        \setlength\itemsep{0em}
            \item[(a)] {\it The treated and untreated differ much in their parents' education attribute.}
            \item[(b)] {\it The treated and untreated differ much in their family income attribute.}
            \item[(c)] {\it The treated and untreated differ much in their attribute of ``having sibling.''}
        \end{itemize}
        \item[{\bf Q2}] {\bf [Subgroups]} Suppose the data is divided into two subgroups based on parent education level (i.e., high vs low), how do you agree with the following descriptions?
        \begin{itemize}
        \setlength\itemsep{0em}
            \item[(a)] {\it In the high-education subgroup, more than 50\% individuals have taken the training.}
            \item[(b)] {\it On average students from high education family are more likely to be admitted into colleges than students from low education family.}
            \item[(c)] {\it Focusing on the high education subgroup, the success rate for the treated is significantly lower than that for the untreated.}
            \item[(d)] {\it Combining high-/low-education subgroups together, the success rate for the treated is significantly higher than that of the untreated.}
            \item[(e)] {\it Splitting participants based on parent education, the proportion of students having siblings in the high-education subgroup is significantly higher than that in the low-education subgroup.}
            \item[(f)] {\it The data exhibits Simpson's paradox, as the subgroup-level treatment-outcome association in low-/high-education subgroups are different or even reversed from the overall.}
        \end{itemize}
        \item[{\bf Q3}] {\bf [Reasoning]} Given the observation that the treatment-outcome association is negative in the high-educated subgroup (e.g., a higher success rate among the untreated instead of the treated), which is opposite from the population-level association, how much do you agree with the following explanations for this phenomenon?
        \begin{itemize}
        \setlength\itemsep{0em}
            \item[(a)] {\it The students with high education parents are more likely to take the training while the reverse is true for those with low education parents; meanwhile, they also tend to have a higher likelihood of getting into colleges than the low-educated. These two factors -- ``those who likely to take treatment happen to be those who likely to succeed'' -- result in a spurious positive association overall.}
            \item[(b)] {\it The students with low education parents are more likely to take the training while the reverse is true for those with high education parents; meanwhile, they also tend to have a higher likelihood of getting into colleges than the high-educated. These two factors -- ``those who likely to take treatment happen to be those who likely to succeed'' -- result in a spurious positive association overall.}
            \item[(c)] {\it The students with high education parents are more likely to take the training while the reverse is true for those with low education parents; meanwhile, they also tend to have a higher likelihood of getting into colleges than the low- educated. These two factors -- ``those who likely to take treatment happen to be those who likely to fail'' -- result in a spurious positive association overall.}
        \end{itemize}
        \item[{\bf Q4}] {\bf [Decisions]} Assuming no other hidden confounders involved, or any other mechanisms distorting the cause- outcome relationships, you’re a decision-maker, based on this given data, which decisions you might agree with?
        \begin{itemize}
        \setlength\itemsep{0em}
            \item[(a)] {\it Might recommend a student who has siblings to take the training program. Because the association is positive among the participants having siblings, implying that taking the program can increase the chance of such a student being admitted into colleges.}
            \item[(b)] {\it Might recommend a student who does not have siblings to take the training. Because the association is positive among the participants who do not have siblings, besides, other covariates are balanced between treated/untreated, implying that taking the program would increase the chance of such a student in being admitted into colleges.}
            \item[(c)] {\it Might not recommend a student from high education family to take the training program. Because the association is negative among the participants who are from high education family, besides, other covariates are balanced between treated/untreated, implying that taking the program would decrease the chance of such a student in getting admitted into colleges.}
        \end{itemize}
    \end{itemize}} \\
    \bottomrule
\end{tabular}
\end{table*}

\begin{table*}[ht]
  \centering
  \small
  \caption{The task of analyzing whether an online course helps students in pass the final examinations.}
  \label{tab:task2}
   \begin{tabular}{ p{0.95 \linewidth} }
    \toprule
    {\bf Task 2. An online course is developed, aiming to help students to achieve better grades to pass an exam. You will be using a visual tool to explore a data set and figure out whether the association between ``taking online course'' and ``passing the exam'' is spurious. Please note, the treated are people who have taken the online course, the untreated otherwise.} \\
    {\begin{itemize}
    \setlength\itemsep{1em}
        \item[{\bf Q1}] {\bf [Confounders]} Based on the given data and visualization, how do you agree with the following descriptions about comparing the treated/untreated?
        \begin{itemize}
        \setlength\itemsep{0em}
            \item[(a)] {\it The treated and untreated differ much in their age attribute.}
            \item[(b)] {\it The treated and untreated differ much in their skill attribute.}
            \item[(c)] {\it The treated and untreated differ much in their gender attribute.}
        \end{itemize}
        \item[{\bf Q2}] {\bf [Subgroups]} Suppose the data is divided into two subgroups based on age (i.e., old vs young), how do you agree with the following descriptions?
        \begin{itemize}
        \setlength\itemsep{0em}
            \item[(a)] {\it In the old subgroup, more than 50\% individuals have taken this online course.}
            \item[(b)] {\it On average, the old people are more likely to pass the exam than the young.}
            \item[(c)] {\it Focusing on the old subgroup, the success rate of passing the exam for the treated is significantly lower than that of the untreated.}
            \item[(d)] {\it Combining the old/young subgroups together, the success rate of passing the exam for the treated is significantly higher than that of the untreated.}
            \item[(e)] {\it Splitting the participants based on age, the proportion of female students in the old subgroup is significantly higher than that in the young subgroup.}
            \item[(f)] {\it The data exhibits Simpson's paradox, as the subgroup-level treatment-outcome association in old/young subgroups are different or even reversed from the overall.}
        \end{itemize}
        \item[{\bf Q3}] {\bf [Reasoning]} Given the observation that the treatment-outcome association is negative in the old subgroup (e.g., a higher success rate among the untreated instead of the treated), which is opposite from the population-level association, how much do you agree with the following explanations for this paradoxical phenomenon?
        \begin{itemize}
        \setlength\itemsep{0em}
            \item[(a)] {\it The old participants are more likely to take the online course while the reverse is true for the young; meanwhile, they also tend to have a higher likelihood of passing the exam than the young. These two factors -- ``those who likely to take the treatment happen to be those who likely to succeed'' -- result in a spurious positive association overall.}
            \item[(b)] {\it The young participants are more likely to take the online course while the reverse is true for the old; meanwhile, they also tend to have a higher likelihood of passing the exam than the old. These two factors -- ``those who likely to take the treatment happen to be those who likely to succeed'' -- result in a spurious positive association overall.}
            \item[(c)] {\it The old participants are more likely to take the online course while the reverse is true for the young; meanwhile, they also tend to have a higher likelihood of passing the exam than the young. These two factors -- ``those who likely to take the treatment happen to be those who likely to fail'' -- result in a spurious positive association overall.}
        \end{itemize}
        \item[{\bf Q4}] {\bf [Decisions]} Assuming no other hidden confounders involved, or any other mechanisms distorting the cause- outcome relationships, you’re a decision-maker, based on this given data, which decisions you might agree with?
        \begin{itemize}
        \setlength\itemsep{0em}
            \item[(a)] {\it Might recommend a male student to take the online course. Because the association is positive among male participants, implying that taking the program can increase the chance of a male student getting a job.}
            \item[(b)] {\it Might recommend a female person to take the online course. Because the association is positive among female participants, besides, other covariates are balanced among treated/untreated, implying that taking the program would increase the chance of a female participant in passing the exam.}
            \item[(c)] {\it Might not recommend an old person to take the online course. Because the association is negative, besides, other covariates are balanced among treated/untreated, implying that taking the program would decrease the chance of an old person in passing the exam.}
        \end{itemize}
    \end{itemize}} \\
\bottomrule
\end{tabular}
\end{table*}

% \tocomments{{\bf System Demo.} We provide two demos:
% \begin{enumerate}
%     \item A longer version of VISPUR demo with a brief background (7 min), please click \href{https://drive.google.com/drive/folders/1mBIHysciAV8kvripizKeE2_2sr2deeeG}{https://shorturl.at/fwJT4}.
%     \item A shorter version of VISPUR demo without background information (5 min), please click \href{https://drive.google.com/drive/folders/1mBIHysciAV8kvripizKeE2_2sr2deeeG}{https://shorturl.at/lmqzE}.
% \end{enumerate}}

