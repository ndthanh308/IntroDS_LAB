\section{Conclusion}\label{sec:conclusion}

% {We introduced \vispur, a novel visual analytic system that offers effective methods and informative visual components to aid in the interpretation of spurious associations, with a specific focus on addressing Simpson's paradox. Our system seeks to establish a ``de-paradox'' workflow to combat two primary sources of spuriousness and paradoxical associations: confounding bias and heterogeneous subgroups. By promoting awareness and understanding of spurious associations, \vispur aims to assist data practitioners in making informed data-driven decisions. The evaluation demonstrated the utility and effectiveness of \vispur against existing tools in four key areas: confounder identification, subgroup examination, paradox interpretation, and decision making. Feedback and behavioral patterns indicated that the system is capable of reducing cognitive burdens with clear and intuitive visualizations, allowing users to reason about paradoxical phenomena and practice accountable decision-making. From evaluations, we proved that the novelty of \vispur lies in its ability to provide a comprehensive approach to tackling spurious associations and addressing Simpson's paradox, making it a valuable tool for data practitioners.}

{We present \vispur, a novel visual analytic system that helps interpret spurious associations, focusing on addressing Simpson's paradox. Our system establishes  a ``de-paradox'' workflow, combating confounding bias and heterogeneous subgroups. \vispur enhances awareness and understanding of spurious associations, assisting data practitioners in making informed decisions. The evaluation demonstrates \vispur's utility and effectiveness in confounder identification, subgroup examination, paradox interpretation, and decision making. Feedbacks and user behaviors confirm that \vispur reduces cognitive load with clear visualizations, enables reasoning about paradoxes and accountable decision-making, making it a valuable tool for data practitioners.}

% {In the future, we plan to extend the current design to better address the three limitations discussed above. To better support subgroup investigation when a large number of subgroups are present, we plan to consider interactive operations so that users might freely search, filter, retrieve, highlight and zoom in/out subgroups. We will improve the system's interface to make it more self-contained and user-friendly by inserting self-exploration guidelines. Besides, we plan to develop a more intuitive visual representation for the concept of imbalance to help users identify residual confounding.}