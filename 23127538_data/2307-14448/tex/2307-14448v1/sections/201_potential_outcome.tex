\subsection{Causal Inference Framework}
% \tocomments{Although randomized controlled trials (RCTs) are considered the ``gold standard'' to establish causality, they are often unethical, impractical, or untimely \cite{guo2021vaine,guo2023causalvis}. Causal inference based on observational data has been widely applied in health domain \cite{cookson2021equity}, social, and political sciences \cite{clark2015big,gerring2005causation}. The randomization design of RCTs ensures that subjects from two treatment groups have comparable characteristics, namely, covariates are {\it balanced}. In contrast, in observational studies two treatment groups might have very distinct feature distributions, and the outcome difference might eventually trace back to factors other than treatment, making it questionable to endow association with a causal interpretation.}

\tocomments{Randomized controlled trials (RCTs) are the ``gold standard'' for causality, but they are often unethical, impractical, or untimely \cite{guo2021vaine,guo2023causalvis}. In the absence of RCTs, causal inference based on observational data has been extensively utilized in many domains \cite{cookson2021equity, clark2015big, gerring2005causation}. RCTs ensure comparable characteristics, {\it balanced}, between treatment groups through randomization, while observational studies may have distinct feature distributions ({\it imbalanced}), making it possible that outcome difference might eventually trace back to factors other than treatment.}

\tocomments{The potential outcomes framework, also called the Rubin Causal Model (RCM) \cite{imbens2010rubin}, is a theoretical framework for causal inference in both observational and experimental studies. Consider again the example in Introduction~\ref{sec:introduction}, it involves defining two potential outcomes for each person -- the potential outcome had they participated in the program and the outcome had they not. By comparing outcome differences across subjects, the average treatment effect (ATE) is estimated. Since it is impossible to observe {\it both} potential outcomes (as one of the potential outcomes is always missing in reality), additional assumptions are necessary for inferring the treatment effect. These assumptions in our study include overlap (or positivity) \cite{heckman1997matching}, which assumes that participants could have chosen not to attend the program and vice versa, and the stable unit treatment value assumption (SUTVA) \cite{vanderweele2013causal,cole2009consistency}, which assumes that subjects do not influence each other's participation decisions and employment outcomes, and there are no hidden variations of treatment that might lead to distinct outcomes. Additionally, the unconfoundedness assumption \cite{rosenbaum1983central} requires that all confounders should have been measured for causal analysis.}

\tocomments{Our system is designed in the context of RCM framework on the basis of above assumptions, allowing users to investigate spurious or paradoxical association in observational studies.}