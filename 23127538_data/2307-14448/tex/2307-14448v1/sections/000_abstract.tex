
  % {Big data and machine learning tools have jointly powered humans in making data-driven decisions. However, they are heavily exploited to capture empirical associations rather than causal effects, posing a risk to efficacy, equity and fairness in societal, health, among other high-stake domains. Existing works have found that taking empirical associations as causal relations could be spurious in many ways. They might reflect inequalities, mask hidden subgroup heterogeneity, or display a counterintuitive trend paradox (e.g, Simpson's paradox), causing confusion and difficulty in decision-making. How can analytic tools facilitate a causality-oriented association analysis for realiable decision-making? In this work, we identify a set of design guidelines for analyzing spurious associations to explicate the needs for locating and interpreting spuriousness as well as characterizing subgroup patterns. Following the design guidelines, we develop \vispur, a visual analytic suite that inspects confounders, subgroup characteristics -- from both causal and attribute aspects, interpretations of spuriousness, as well as decision making in face of spuriousness. To demonstrate the usefulness of \vispur, we conducted a user study and a case study focusing on online educational tools. We showcase how our design and interface can bring a richer understanding of spurious associations in observational data, as well as assist data practitioners make reliable decision-making.
  % A free copy of this paper and all supplemental materials are available at \url{https://OSF.IO/2NBSG}}.

  \abstract{{{Big data and machine learning tools have jointly empowered humans in making data-driven decisions. However, many of them capture empirical associations that might be spurious due to confounding factors and subgroup heterogeneity.}
  {The famous Simpson's paradox is such a phenomenon where aggregated and subgroup-level associations contradict with each other, causing cognitive confusions and difficulty in making adequate interpretations and decisions.}
  {Existing tools provide little insights for humans to locate, reason about, and prevent pitfalls of spurious association in practice.}
  {We propose \vispur, a visual analytic system that provides a causal analysis framework and a human-centric workflow for tackling spurious associations.}
  {These include a \confounderdashboard, which can automatically identify possible confounding factors, and a \subgroupviewer, which allows for the visualization and comparison of diverse subgroup patterns that likely or potentially result in a misinterpretation of causality. Additionally, we propose a \reasoningstoryboard, which uses a flow-based approach to illustrate paradoxical phenomena, as well as an interactive \decisiondiagnosis panel that helps ensure accountable decision-making.}
  % {By allowing data practitioners to easily inspect distortion factors and subgroup patterns behind in a cause-outcome relation, the system facilitates users to reason about the emergence of spuriousness and paradox, as well as to make better decisions for any given subset of data.}
  {Through an expert interview and a controlled user experiment, our qualitative and quantitative results demonstrate that the proposed ``de-paradox'' workflow and the designed visual analytic system are effective in helping human users to identify and understand spurious associations, as well as to make accountable causal decisions.}
  % {To demonstrate the utility of \vispur system we conduct a controlled user experiment and an expert interview. The qualitative result shows a significant 15\% increase in the accuracy of answering questions about locating, reasoning about, and avoiding pitfalls of spurious association. Our quantitative findings suggest that our proposed design help reveal subgroup properties, not only offer a richer understanding of spurious associations, but also help data analysts to make more reliable causal decisions.}
  % {To demonstrate the utility of our proposed \vispur system through rich evaluations including a controlled user experiment and an expert interview. We showcase how our design and interface can bring a richer understanding of spurious associations in observational data, as well as assist data practitioners make reliable decisions.}
}}