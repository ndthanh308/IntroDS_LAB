\input{images/case_study_caption}

\subsection{\reasoningstoryboard}\label{sec:systemdesign:reasoningstoryboard}
\input{images/storyboard_caption}
{To aid in understanding association conflicts, we design {\reasoningstoryboard}, depicted in Fig.~\ref{fig:system_overview}D, which complements {\subgroupviewer} by providing a narrative for the appearance of a conflicting/paradoxical phenomenon ({\bf T4}).} {The diagram comprises three layers: {\tt subgroup}, {\tt cause}, and {\tt outcome}. The {\tt subgroup} node is a rectangle scaled to the size of the chosen subgroup (e.g., the number of participants). In the {\tt cause} layer, multiple nodes depict possible treatments. \tocomments{For a dichotomous treatment, nodes are limited to {treated} and {untreated}; For a continuous treatment, the values are discretized into $L$ ($L = 4$ in our demonstration) bins based on percentiles and ranked in a descending order.} The height of a cause node represents the percentage of participants taking that action. The {\tt outcome} layer is a vertical axis, where the top endpoint indicates the maximum value of the outcome and the bottom indicates the minimum. Pathways originate from the {\tt group} node, traverse through relevant {\tt cause} nodes, and terminate at a specific point along the {\tt outcome} axis. The height of each pathway represents the proportion of participants who took a particular treatment action at each {\tt cause} node. The endpoint of each pathway indicates the average outcome achieved (e.g., average earnings in 1978).}

% little work has studied the flow-based visualization's capability of demonstrating and interpreting a paradoxical phenomenon. Our storyboard design holds contributions by recognizing that the shapes of flows carry valuable information for interpreting paradoxical phenomena.

\tocomments{The storyboard visualization extends the well-known Sankey diagrams \cite{riehmann2005interactive} and parallel sets \cite{kosara2006parallel}, which have been found useful in depicting storylines \cite{kwon2020dpvis} and multidimensional features \cite{kosara2006parallel}. Despite their extensive applications in flow-based visualizations, our contribution lies in presenting paradoxical patterns using flow-based narratives to facilitate the interpretability of such complex phenomena.} Fig.~\ref{fig:storyboard} illustrates three common shape patterns: (A) {\bf pass-through}, (B) {\bf hill}, and (C) {\bf valley}, where (A) depicts the population-level pattern, and (B,C) depict two subgroups' patterns. The {\bf pass-through} shape in Fig.\ref{fig:storyboard}A suggests a {\it negative} cause-outcome association because the subset of subjects taking the largest value of treatment (top) end up having the smallest value of outcome. In contrast, two subgroups in Fig.~\ref{fig:storyboard}B,C exhibit {\bf non-pass-through} (thus {\it positive}) associations. To interpret the reversed associations, users might further investigate Fig.~\ref{fig:storyboard}B,C. Users can see that subgroup (B) tends to take a high-valued treatment but its outcome is relatively low ({\bf hill}), whereas subgroup (C) likes to take a low-valued treatment but its outcome is relatively high ({\bf valley}). When the two mirrored {\bf valley} and {\bf hill} pathways are aggregated together, they could generate a {\bf pass-through} pattern as shown in Fig.~\ref{fig:storyboard}A. {The Lalonde dataset in Fig.~\ref{fig:system_overview}D shows that the valley shape displayed by {\tt GroupID 3} might cancel out the hill patterns of other subgroups, leading to a zero-effect pattern overall. Users can mouse over pathways for detailed information, select subgroup diagrams by clicking the ``add'' icon, and perform shape comparison and paradox reasoning ({\bf T4}).}

% {The visual display of storyboard looks similar to Sankey diagrams \cite{riehmann2005interactive} and parallel sets \cite{kosara2006parallel}. However, Sankey diagrams focus on displaying the proportion of the flow that splits in different ways as well as the pathways through a set of discrete stages, without remarking how different pathways arrive at certain outcomes. Parallel sets, though similar in a way of showing flow and proportions, focus on multidimensional data thus do not enforce the sequential order along the pathways or explain how different stages leading to a certain outcome. We rely on the direction of flows to represent the sequential order along the pathways of taking treatment actions and then obtaining outcomes, and use the width of flows to reveal proportional splits of participants.}

% {The storyboard visualization shares some similarities with Sankey diagrams \cite{riehmann2005interactive} and parallel sets \cite{kosara2006parallel}, but it differs in the sense that it emphasizes the sequential order of pathways and explains how different stages lead to certain outcomes. Sankey diagrams focus on displaying proportions of flow while parallel sets focus on multidimensional data and do not enforce the sequential order along pathways. The storyboard visualization uses the direction of flows to represent the sequential order along pathways and the width of flows to reveal proportional splits of participants.}



% {exhibits two converging pathways (treated/control) leading to a same endpoint. The {\bf valley} pattern of {\tt GroupID 3} indicates that some individuals show little interest in the program and still earn relatively higher income than others (probably because they do not need any training). Such a {valley} shape can dilute the positive effects of others, resulting in a zero-effect pattern overall. With our \vispur system, users can mouse over pathways for detailed information, select subgroup diagrams by clicking the ``add'' icon, and perform shape comparison and paradox reasoning ({\bf T4}).}



% The covariate $X$ influences both treatment choices and outcome values: units of $X = 0$ tend to take lower ranked treatment, but their outcome value is relatively high ({\it valley}); On the other hand, units of $X = 1$ tend to take higher ranked treatment, but their outcome value is relatively low ({\it hill}). These two factors result in a negative overall association. To summarize, the {\it mirror-like} pattern between two subgroups suggest a confounding bias.

