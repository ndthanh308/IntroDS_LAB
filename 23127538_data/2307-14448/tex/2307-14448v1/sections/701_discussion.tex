\section{Discussion}\label{sec:discussion}

% \tocomments{TODO. Revise this section focusing on 3 points: (1) causal knowledge and human beliefs, (2) Limitations in real-world applications, (3) Future work to embed VISPUR as an interactive widget into computational programming environments.}
% {\bf Usability of \vispur's components.} Participants and experts provided positive feedback on \vispur's usability and effectiveness. The \confounderdashboard and \subgroupviewer, along with \partition, were heavily used to examine confounders and subgroup effects. Users found \vispur's flow charts intuitive for explaining Simpson's paradox. The warning messages and imbalance charts in \decisiondiagnosis triggered further thinking and helped identify residual confounders. \vispur was also praised for hypothesis generation and result delivery. Overall, users appreciated the ease of modifying partitions and discussing visualizations.

\tocomments{{\bf Human beliefs and prior causal knowledge.} While \vispur provides qualitative scores and visual signals for confounder identification, it is important to consider human causal knowledge. Relying solely on statistical criteria can lead to the adjustment of undesired variables and introduce bias \cite{hernan2020causal,greenland2003quantifying}. \vispur could incorporate a causality reflection panel \cite{yen2019exploratory} allowing users to draw DAGs before exploring visualizations and data, promoting causal reflection and integration of domain knowledge. Meanwhile, we observed that participants sometimes relied on their prior beliefs in decision-making, overlooking critical visualizations. For example, one participant stated that ``{\it taking a training program shouldn't be bad},'' leading them to always recommend it for individuals in the job market. This confirmation bias \cite{yen2019exploratory} can result in errors as humans tend to ignore information that contradicts their beliefs. One possible solution to counteract individual bias is to encourage collaboration among multiple users. \vispur could support collaboration by allowing people to highlight visualizations, provide interpretations, and engage in discussions with others to collectively reach conclusions.}

\tocomments{{\bf Scalability and applicability in real-world scenarios.} \vispur might encounter two potential limitations in real-world applications. The first limitation is scalability. When users generate a large number of subgroups, the {\subgroupviewer} will become crowed and hard to read. Future work can address this by introducing flexible subgroup operations like hierarchical grouping, filtering, hiding, highlighting, and zooming, to mitigate overlap and information overload. The second limitation pertains to the violation of the causal diagram presented in Figure~\ref{fig:teaser}B. Real-world scenarios often involve complex causal structures, including instrument variables, mediators, and even violations of the four major assumptions (ref. Section~\ref{sec:relatedwork}). A causality reflection panel, as mentioned earlier, could serve as an initial step to assess the alignment between real-world scenarios and our problem setup.}

\tocomments{{\bf Potential of embedding \vispur in programming environments.} Embedding interactive widgets in computational environments (e.g., Jupyter notebooks) has gained popularity among data scientists \cite{kery2020future,wang2022nova}. A future \vispur widget could provide an interactive graphical user interface (GUI) that seamlessly integrates programming and interactive operations. The widget can allow codeless operations that can be synchronized with the notebook, enabling a seamless transition between the visual interface and the coding environment. This integration offers benefits in exploratory data analysis, collaboration, and iterative operations, eliminating the need for users to switch between web-based visual tools and programming environments.}
