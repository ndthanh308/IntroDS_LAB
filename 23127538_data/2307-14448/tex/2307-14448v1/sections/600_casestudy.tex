\section{Case Study \tocomments{In A Real-World Application}: Use and Impact of code examples on students' Java programming skills}\label{sec:casestudy}
\tocomments{In our case study, we worked with {\bf P3}, an educational system designer named Micheal (pseudonym), who specializes in adaptive educational systems and educational data mining. Micheal's objective was to determine whether the Java system developed by his team effectively enhances students' Java programming skills. Fig.~\ref{fig:case_study} demonstrates the process of utilizing \vispur to analyze the relationship between a student's performance and their usage of the system.}

% {We present a case study of an educational system for students to learn Java programming to showcase the effectiveness of \vispur in analyzing how a student's performance is related to use of the system (Fig.~\ref{fig:case_study}). Our interview was with the educational system designer, {\bf P3}, who has expertise in developing adaptive educational systems and educational data mining.}
% {In the given task, he wanted to inspect whether and to what extent, code examples -- provided as non-mandatory learning resources in a developed system -- were beneficial and engaging for students in Java programming education.}

{\bf Settings.} \tocomments{We utilized an educational system dataset provided by Michael, ensuring that its data size and feature dimensions match the real data analysis tasks faced by our target users.} It contains the interactions of 482 undergraduate students with the system who take an introductory course in Java programming from 10 different classrooms during the years of 2020 and 2021. Students used the tool in a non-mandatory manner to study code examples and solve quiz tasks at their own paces, needs, and time. The aim of the data analysis was to determine whether system engagement (measured by the total number of code examples studied by a student {\tt EXAMPLE\_total\_number}) affects a student's performance (measured by the success rate of quiz, {\tt QUIZ\_performance}). The remaining variables include a student's success rate during the first three task attempts, time spent in studying examples and doing quiz, as well as learning speed.

\textbf{Find a counterintuitive association and identify confounders (R1).} He initially observed that the number of code examples studied was not significantly associated with the quiz performance in the \subgroupviewer's {\causalityspace}, which was contrary to his expectation of a positive association. By consulting the table in the \confounderdashboard, he was able to hypothesize potential confounding effects from three variables: {\tt EXAMPLE\_speed}, {\tt QUIZ\_speed}, {\tt first\_attempts\_performance} (Fig.~\ref{fig:case_study}A). Michael explained that {\tt first\_attempts\_performance} might represent a student's prior knowledge of Java, {\tt QUIZ\_speed} and {\tt EXAMPLE\_speed} might indicate a student's learning patience, all of them were potential confounders that might influence a student's attitudes towards the tool and success rate.


{\bf Discover and investigate patterns displayed by different student subgroups (R2).} Micheal decided to use {\tt first\_attempts\_performance} (Fig.~\ref{fig:case_study}B) to divide students into two subgroups: those who failed the initial three tasks (low-score students), and those who answered at least one of them (high-score students). He observed two ellipses with opposite trends in \subgroupviewer, and by hovering over the two ellipses, he confirmed that the positive association for low-score students is significant. He explained that the positive trend in low-score group was very promising, possibly suggesting that the education system was helping those students to improve skills. He also noticed that the ellipse of low-score students was positioned to the lower right of the one of high-score students. He commented: ``{\it Low-score students showed more interest in our code examples, probably because they are not confident and want to learn more.}'' and continued: ``{\it But still, on average, their success rate is lower than that of those capable students.}'' Then he moved to the glyphs in {\covariatespace} to examine how two subgroups differ from each other. He particularly selected two speed-related variables and observed that low-score students clicked faster than high-score students in quiz tasks. He found it reasonable as low-score students tend to quickly click to move on probably because they do not know the answers.

{\bf Reasoning paradoxical associations (R3).} When asked how to explain the paradoxical associations observed in the aggregated data against subgroup data, Micheal moved to \reasoningstoryboard interface to compare two subgroups' flow charts. He observed a hill pattern in low-score chart whereas a valley pattern in high-score chart (Fig.~\ref{fig:case_study}B). Micheal explained: ``{\it Low-score students are more interested in studying code examples. Although they benefit from using the system, they cannot outperform high-score peers eventually, so the aggregated data still shows a negative trend.}''

{\bf Locate nested spuriousness and refine data partition (R4).} Given the mixed effects of two subgroups, Micheal wanted to know more details particularly for the subgroup of low-score students. Through {\imbalancechart} in \decisiondiagnosis, he learned that {\tt QUIZ\_speed} was a nested confounder with a large imbalance score. He added {\tt QUIZ\_speed} as an additional variable and found that low-score students were further divided into slow versus fast learners (Fig.~\ref{fig:case_study}C). Micheal also tried the automated data partition function by setting the number of subgroups to be 3, and a similar data partition was displayed in Fig.~\ref{fig:case_study}C. Micheal explained that, ``{\it high-score students are patient in answering questions in quiz; however, within the low-score subgroup, some students might be more motivated than the others.}''


% {In summary, we demonstrate how \vispur is able to identify distortion and explain the counter-intuitive association, as well as reveal a mixed causal effect pictures over multiple student cohorts. We interviewed a domain expert.}



