\begin{table*}[ht]
  \centering
  \small
  \caption{The task of analyzing whether a college application training program helps a student in getting offers.}
  \label{tab:task1}
   \begin{tabular}{ p{0.95\linewidth} }
    \toprule
    {\bf Task 1. A high school has launched a college application training program, aiming to help students in getting admitted into colleges. You will be using a visual tool to explore a data set and figure out whether the association between ``taking college application training'' and ``being admitted into colleges'' is spurious. Please note, the treated are people who have taken the training program, while untreated those who didn't take it.} \\ 
    {\begin{itemize}
    \setlength\itemsep{1em}
        \item[{\bf Q1}] {\bf [Confounders]} Based on the given data and visualization, how do you agree with the following descriptions about comparing the treated/untreated?
        \begin{itemize}
        \setlength\itemsep{0em}
            \item[(a)] {\it The treated and untreated differ much in their parents' education attribute.}
            \item[(b)] {\it The treated and untreated differ much in their family income attribute.}
            \item[(c)] {\it The treated and untreated differ much in their attribute of ``having sibling.''}
        \end{itemize}
        \item[{\bf Q2}] {\bf [Subgroups]} Suppose the data is divided into two subgroups based on parent education level (i.e., high vs low), how do you agree with the following descriptions?
        \begin{itemize}
        \setlength\itemsep{0em}
            \item[(a)] {\it In the high-education subgroup, more than 50\% individuals have taken the training.}
            \item[(b)] {\it On average students from high education family are more likely to be admitted into colleges than students from low education family.}
            \item[(c)] {\it Focusing on the high education subgroup, the success rate for the treated is significantly lower than that for the untreated.}
            \item[(d)] {\it Combining high-/low-education subgroups together, the success rate for the treated is significantly higher than that of the untreated.}
            \item[(e)] {\it Splitting participants based on parent education, the proportion of students having siblings in the high-education subgroup is significantly higher than that in the low-education subgroup.}
            \item[(f)] {\it The data exhibits Simpson's paradox, as the subgroup-level treatment-outcome association in low-/high-education subgroups are different or even reversed from the overall.}
        \end{itemize}
        \item[{\bf Q3}] {\bf [Reasoning]} Given the observation that the treatment-outcome association is negative in the high-educated subgroup (e.g., a higher success rate among the untreated instead of the treated), which is opposite from the population-level association, how much do you agree with the following explanations for this phenomenon?
        \begin{itemize}
        \setlength\itemsep{0em}
            \item[(a)] {\it The students with high education parents are more likely to take the training while the reverse is true for those with low education parents; meanwhile, they also tend to have a higher likelihood of getting into colleges than the low-educated. These two factors -- ``those who likely to take treatment happen to be those who likely to succeed'' -- result in a spurious positive association overall.}
            \item[(b)] {\it The students with low education parents are more likely to take the training while the reverse is true for those with high education parents; meanwhile, they also tend to have a higher likelihood of getting into colleges than the high-educated. These two factors -- ``those who likely to take treatment happen to be those who likely to succeed'' -- result in a spurious positive association overall.}
            \item[(c)] {\it The students with high education parents are more likely to take the training while the reverse is true for those with low education parents; meanwhile, they also tend to have a higher likelihood of getting into colleges than the low- educated. These two factors -- ``those who likely to take treatment happen to be those who likely to fail'' -- result in a spurious positive association overall.}
        \end{itemize}
        \item[{\bf Q4}] {\bf [Decisions]} Assuming no other hidden confounders involved, or any other mechanisms distorting the cause- outcome relationships, you’re a decision-maker, based on this given data, which decisions you might agree with?
        \begin{itemize}
        \setlength\itemsep{0em}
            \item[(a)] {\it Might recommend a student who has siblings to take the training program. Because the association is positive among the participants having siblings, implying that taking the program can increase the chance of such a student being admitted into colleges.}
            \item[(b)] {\it Might recommend a student who does not have siblings to take the training. Because the association is positive among the participants who do not have siblings, besides, other covariates are balanced between treated/untreated, implying that taking the program would increase the chance of such a student in being admitted into colleges.}
            \item[(c)] {\it Might not recommend a student from high education family to take the training program. Because the association is negative among the participants who are from high education family, besides, other covariates are balanced between treated/untreated, implying that taking the program would decrease the chance of such a student in getting admitted into colleges.}
        \end{itemize}
    \end{itemize}} \\
    \bottomrule
\end{tabular}
\end{table*}

\begin{table*}[ht]
  \centering
  \small
  \caption{The task of analyzing whether an online course helps students in pass the final examinations.}
  \label{tab:task2}
   \begin{tabular}{ p{0.95 \linewidth} }
    \toprule
    {\bf Task 2. An online course is developed, aiming to help students to achieve better grades to pass an exam. You will be using a visual tool to explore a data set and figure out whether the association between ``taking online course'' and ``passing the exam'' is spurious. Please note, the treated are people who have taken the online course, the untreated otherwise.} \\
    {\begin{itemize}
    \setlength\itemsep{1em}
        \item[{\bf Q1}] {\bf [Confounders]} Based on the given data and visualization, how do you agree with the following descriptions about comparing the treated/untreated?
        \begin{itemize}
        \setlength\itemsep{0em}
            \item[(a)] {\it The treated and untreated differ much in their age attribute.}
            \item[(b)] {\it The treated and untreated differ much in their skill attribute.}
            \item[(c)] {\it The treated and untreated differ much in their gender attribute.}
        \end{itemize}
        \item[{\bf Q2}] {\bf [Subgroups]} Suppose the data is divided into two subgroups based on age (i.e., old vs young), how do you agree with the following descriptions?
        \begin{itemize}
        \setlength\itemsep{0em}
            \item[(a)] {\it In the old subgroup, more than 50\% individuals have taken this online course.}
            \item[(b)] {\it On average, the old people are more likely to pass the exam than the young.}
            \item[(c)] {\it Focusing on the old subgroup, the success rate of passing the exam for the treated is significantly lower than that of the untreated.}
            \item[(d)] {\it Combining the old/young subgroups together, the success rate of passing the exam for the treated is significantly higher than that of the untreated.}
            \item[(e)] {\it Splitting the participants based on age, the proportion of female students in the old subgroup is significantly higher than that in the young subgroup.}
            \item[(f)] {\it The data exhibits Simpson's paradox, as the subgroup-level treatment-outcome association in old/young subgroups are different or even reversed from the overall.}
        \end{itemize}
        \item[{\bf Q3}] {\bf [Reasoning]} Given the observation that the treatment-outcome association is negative in the old subgroup (e.g., a higher success rate among the untreated instead of the treated), which is opposite from the population-level association, how much do you agree with the following explanations for this paradoxical phenomenon?
        \begin{itemize}
        \setlength\itemsep{0em}
            \item[(a)] {\it The old participants are more likely to take the online course while the reverse is true for the young; meanwhile, they also tend to have a higher likelihood of passing the exam than the young. These two factors -- ``those who likely to take the treatment happen to be those who likely to succeed'' -- result in a spurious positive association overall.}
            \item[(b)] {\it The young participants are more likely to take the online course while the reverse is true for the old; meanwhile, they also tend to have a higher likelihood of passing the exam than the old. These two factors -- ``those who likely to take the treatment happen to be those who likely to succeed'' -- result in a spurious positive association overall.}
            \item[(c)] {\it The old participants are more likely to take the online course while the reverse is true for the young; meanwhile, they also tend to have a higher likelihood of passing the exam than the young. These two factors -- ``those who likely to take the treatment happen to be those who likely to fail'' -- result in a spurious positive association overall.}
        \end{itemize}
        \item[{\bf Q4}] {\bf [Decisions]} Assuming no other hidden confounders involved, or any other mechanisms distorting the cause- outcome relationships, you’re a decision-maker, based on this given data, which decisions you might agree with?
        \begin{itemize}
        \setlength\itemsep{0em}
            \item[(a)] {\it Might recommend a male student to take the online course. Because the association is positive among male participants, implying that taking the program can increase the chance of a male student getting a job.}
            \item[(b)] {\it Might recommend a female person to take the online course. Because the association is positive among female participants, besides, other covariates are balanced among treated/untreated, implying that taking the program would increase the chance of a female participant in passing the exam.}
            \item[(c)] {\it Might not recommend an old person to take the online course. Because the association is negative, besides, other covariates are balanced among treated/untreated, implying that taking the program would decrease the chance of an old person in passing the exam.}
        \end{itemize}
    \end{itemize}} \\
\bottomrule
\end{tabular}
\end{table*}