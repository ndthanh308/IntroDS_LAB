\begin{table*}[t]
  \centering
  \small
  \caption{Summary of our interview study. Colors represent four major challenges: \conebox{\bf C1}, \ctwobox{\bf C2}, \cthreebox{\bf C3}, \cfourbox{\bf C4}.}
  \label{tab:interviews}
   % \begin{adjustbox}{width=\textwidth}
   \begin{tabular}{ p{0.10\linewidth} p{0.20\linewidth} p{0.24\linewidth} p{0.36\linewidth}} 
    \toprule
    {\bf Interviewee} & {\bf Focal Interest in Analysis} & {\bf Practices/Tools} & {\bf Challenges/Needs} \\
    \midrule
    {Social worker ({\bf P1})} & {To what extent crime severity is linked to sentence, any differences among different populations (e.g., ethnicity, gender)?} & Fit one or multiple regression models by adjusting demographics features and examine coefficients and $p-$values; SPSS and R are used; & 
    \cone{Have difficulty in variable selection in regression and might overlook confounders;} \cfour{Unable to claim causal effects from statistical analysis;} \ctwo{Unable to automatically detect causal effects for subpopulations especially when multiple variables are involved;} Desire for a user-friendly, interactive system for data exploration; \\
    \midrule
    {Trading analyst ({\bf P2})} & {How and under what special conditions does previous return inform future return in the trading market?} & Build regression models and examine the coefficients and $p-$value; Rely on prior experience to manually search for a subset of market conditions and re-fit regressions; R and Python are used; & 
    \cone{Lack guidance in feature selection;} \cthree{Unable to interpret association change in different models;} \cfour{Unable to claim significance or make decisions given inconsistent associations in different models;} \ctwo{Desire for guidance in searching for market conditions where past return predicts future;} Desire for intuitive visualization to deliver/explain results to leadership; \\
    \midrule
    {Educational system designer ({\bf P3})} & {What is the impact of the designed educational system on students' performance, and how can we explain the counterintuitive pattern of ``more engagement with the system $\rightarrow$ worse performance''?} & Build (stepwise) regression models to examine coefficients and $p-$value; Multicollinearity examination and data-driven feature selection; R and Python are used; & 
    \cthree{Unable to interpret association changes in different models of distinct predictors;} \ctwo{How to define subgroups remain a challenge in face of many features, to reveal students' heterogeneous characteristics (e.g., who like to use the system, how the system affects students' performance);} \cthree{Desire for tools/techniques to explain the counterintuitive association found in their work;} \cfour{Hard to claim a strong conclusion regarding the effectiveness of the designed system;} Interpret the analysis results to people without background of causal inference is a challenge; \\
  \bottomrule
\end{tabular}
% \end{adjustbox}
\end{table*}