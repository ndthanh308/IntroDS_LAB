\begin{table*}
 \centering
 \small
 \caption{\tocomments{Comparison of existing visualization designs for Simpson's paradox. It provides a summary of design ideas, advantages and disadvantages of each of the designs.}}
 \label{tab:comp}
 % \begin{adjustbox}
 \begin{tabular}{ p{0.1 \linewidth} p{0.35 \linewidth} p{0.2 \linewidth} p{0.25 \linewidth} }
    \toprule
    {\bf Graph} & {\bf Design summary} & {\bf Advantages} & {\bf Disadvantages} \\
    \midrule
    {\bf B-K diagram} \cite{baker2001good} & 
    A 2D coordinate space where the horizontal line is the proportion of a certain group (e.g., women) and the vertical line is outcome. Lines in the space demonstrate different treatment, vertical positions indicate outcomes. & 
    Well preserve the outcome differences of distinct treatments using vertical positions of dots or lines. & 
    No clear representation of subgroups, thus does not support comparison of subgroups' properties. \\ \\
    {\bf Platform scale} \cite{rum1980magic} & 
    A set of blocks arranged in stacks of varying heights is located on a platform and balanced on a pivot at the center of gravity. Treatments are represented by multiple platforms, outcomes are by positions of blocks, treatment preferences are by heights of blocks. & 
    Well represent weights (i.e., treatment preferences) by heights of blocks. &
    Hard to visually understand how a static equilibrium is achieved when multiple subgroups (stacks of blocks) are present; Hard to examine subgroup-specific properties as blocks of the same subgroup are placed on different platforms. \\ \\
    {\bf Comet} \cite{armstrong2014visualizing} &
    A set of comets placed in a 2D coordinate space, where each comet is a subgroup, with the thin head being one treatment scenario (A) and the thick tail the other (B). The motion of a comet indicates how the changes of properties from scenario A to B. & 
    Well represent multiple subgroups, and easily compare difference of subgroup patterns, an aggregate comet is shown to alert Simpson's paradox. & 
    The strong sense of motion conveyed by comets only works well for time-based data rather than continuous or categorical treatment. \\ \\
    {\bf Circle-line plot} \cite{rucker2008simpson} &
    A set of circle-line plots placed in a 2D coordinate space, where each circle-line plot is a subgroup with its slope representing per-subgroup association, horizontal axis being treatment, vertical axis being outcome, and circle size represents sample size. &
    Well represent multiple subgroups, and easily compare difference of subgroup patterns, an aggregate circle-line is shown to alert Simpson's paradox. & 
    Large circles obscure small ones if many subgroups exist.  \\ \\
    {\bf Ellipse} \cite{friendly2013elliptical} &
    A set of ellipses placed in a 2D coordinate space, where each ellipse is a subgroup with its slope representing per-subgroup association, horizontal axis being treatment, vertical axis being outcome. &
    Well represent multiple subgroups, and easily compare difference of subgroup patterns, an aggregate ellipse is shown to alert Simpson's paradox. &
    Large ellipses obscure small ones if many subgroups exist. \\
    \bottomrule
 \end{tabular}
 % \end{adjustbox}
\end{table*}

\begin{table*}
\small
\centering
\caption{\tocomments{Comparison of existing designs for Simpson's paradox. It summarizes how six key elements (see Fig.~\ref{fig:teaser}B)---three entities (cause, outcome, subgroup) and three arrows (treatment propensity, base effect, cause-outcome relationship)---are represented using visual encodings in each of the designs.}}
\label{tab:visulizing_SP}
% \begin{adjustbox}{width=\textwidth}
\begin{tabular}{ p{0.1\linewidth} p{0.1\linewidth} p{0.1\linewidth} p{0.1\linewidth} p{0.1\linewidth} p{0.1\linewidth} p{0.1\linewidth}} 
 \toprule
 {\bf Diagram} &  {\bf Subgroup} & {\bf Cause} & {\bf Outcome} & {\bf Treatment propensity} & {\bf Base effect} & {\bf Cause-outcome association} \\
 \midrule
 {\bf B-K diagram} \cite{baker2001good} & - & line segments & y-axis & x-coordinate of data points along line segments & endpoints of line segments & slope of line segments \\ \\
 {\bf Platform scale} \cite{rum1980magic} & block labels & platforms & positions of blocks on platforms & height of blocks & a default stack of blocks & relative positions of blocks with the same label \\ \\
 {\bf Comet} \cite{armstrong2014visualizing} & comets & endpoints of a comet & y-axis & x-coordinate of a comet & y-coordinate of a comet's head & a comet's length along y-axis \\ \\
 {\bf Circle-line plot} \cite{rucker2008simpson} & line segments & 0/1 on x-axis & y-axis & circle size & intercept on y-axis of line segments & slope of line segments \\ \\
 {\bf Ellipse} \cite{friendly2013elliptical} & ellipses & x-axis & y-axis & positions of ellipses along x-axis & positions of ellipses along y-aixs & slope of regression lines \\
 \bottomrule
\end{tabular}
% \end{adjustbox}
\end{table*}



\iffalse
\begin{table}[h!]
\small
\centering
\caption{Comparison of SP visual representations.}
\label{tab:visulizing_SP}
\begin{adjustbox}{width=\textwidth}
\begin{tabular}{ P{0.16\linewidth} P{0.16\linewidth} P{0.16\linewidth} P{0.16\linewidth} P{0.16\linewidth} P{0.16\linewidth}} 
 \toprule
 {\bf Element} & {\bf B-K} & {\bf Platform scale} & {\bf Comet} & {\bf Triplet} & {\bf Ellipse} \\
 \midrule
 Subgroups & —— & block labels & comets & lines & ellipses \\ \\
 Treatment & lines & platforms (two platforms are women vs men) & endpoints of a comet & 0/1 on x-axis & x-axis \\ \\
 Outcome & y-axis &  position of blocks on platform & y-axis & y-axis of a comet's endpoints & y-axis \\ \\
 Propensity &  —— & height of blocks & x-axis of a comet & diamond size & positions of ellipses along x-axis \\ \\
 Prognostic tendency & endpoints of lines & one of the platform (e.g., men) & y-axis of a comet's head & line intercepts on y-axis & ellipse center along y-axis \\ \\
 Effect/Association & line slope & relative position of blocks with the same label & comet's length along y-axis & line slope & slope \\ \\
 Data type & discrete treatment, binary/continuous outcome & discrete treatment, binary outcome & time-based data & binary treatment & continuous treatment and outcome \\ \\
 Advantage & 
 well preserve the differences of treatment arms via vertical positions of dots or lines & 
 well represent weights (i.e., treatment preferences) by heights of blocks &
 well represent multiple subgroups, and easily compare difference of subgroup patterns, an aggregate comet is also shown to alert the presence of a Simpson's paradox & 
 well preserve subgroups, and easily compare difference of subgroup patterns, an aggregate plot is also shown to alert the presence of a Simpson's paradox & 
 well preserve subgroups, and easily compare difference of subgroup patterns, an aggregate plot is also shown to alert the presence of a Simpson's paradox \\ \\
 Limitation & no clear representation of subgroups & weighting scale is not suitable for many subgroups & only works well for time-based data & not able to handle continuous treatment & not suitable for nonlinear relation, no statistical assessment and interaction \\ 
\bottomrule
\end{tabular}
\end{adjustbox}
\end{table}
\fi