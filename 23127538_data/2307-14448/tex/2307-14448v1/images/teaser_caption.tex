\teaser{
  \centering
% Figure removed
  \caption{{This work provides a ``de-paradox'' workflow to help analyze observational data and overcome spurious and paradoxical associations that can lead to misleading interpretations of causal effects.}
  {(A) Spurious associations, including Simpson's paradox, are prevalent in observational studies. E.g., in a study that investigates the effect of a job training program, the cause (training program) and outcome (earnings) can be distorted by a third variable (ethnicity), leading to a misleading interpretation of the causal effect. (B) We identify two major sources for spuriousness: (1) confounding bias and (2) subgroup heterogeneity, based on causal literature. 
  % {(C) We propose a novel ``de-paradox'' workflow to tackle the two major sources of spuriousness, which include (C.1) \confounderdashboard---automatically suggests potential confounders, (C.2) \subgroupviewer---visualizes and contrasts heterogeneous subgroup patterns that can lead to misleading causal interpretation, (C.3) \reasoningstoryboard---a flow-based storyboard to illustrate paradoxical phenomena, and (C.4) \decisiondiagnosis---an interactive decision diagnosis panel to make accountable decisions.}
  {(C) We propose a novel ``de-paradox'' workflow to tackle the two major sources of spuriousness, which include (C.1) \confounderdashboard, (C.2) \subgroupviewer, (C.3) \reasoningstoryboard, and (C.4) \decisiondiagnosis, to guide users to ``de-paradox'' and reason about the most appropriate causal interpretation of association paradox.}
  % Please revise the teaser figure to be consistent with the text. Fig 1(B), highlight (1)(2), and replace "race". Fig 1(C), revise the component names and descriptions. If you put the component descriptions on the figure, the caption can be shortened as:
  % (C) We propose a novel ``de-paradox'' workflow to tackle the two major sources of spuriousness, which include (C.1) Counfounder Dashboard -- automatically suggests potential confounders, (C.2) Subgroup Viewer -- visualizes and contrasts heterogeneous subgroup patterns that can lead to misleading causal interpretation, (C.3) Paradox Reasoning Storyboard -- a flow-based storyboard to illustrate paradoxical phenomena, and (C.4) Decision Diagnosis -- an interactive decision diagnosis panel to de-paradox and select most appropriate causal interpretation from the observed data.
  }}
  
  % Overview of our work. (a) An example of a spurious cause-outcome association. The goal is to answer whether a new feature (cause) affects user satisfaction of an online product (outcome). Columns show how many users have used the feature, as well as the percentages of users reporting to be satisfied. The overall cause-outcome relation is positive as the satisfaction rate is higher (68\% versus 38\%) among those who used the feature than those who didn't. However, subgroup analysis reveals mixed effects (an increase +10\% versus a decline -10\%). (b) The causal diagram in our study. There are three components: a pair of treatment and outcome, and a set of additional covariates. Covariates might play as confounders that simultaneously affect both treatment and outcome, or/and are used to define data partitions. The dashed arrows from covariates to treatment represents {\it propensity}, the one from covariates to outcome is {\it base effect}, the one linking treatment to outcome is {\it causal effect}. The black double-ended arrow represents observable cause-outcome {\it association}, however, which could be spurious. (c) The framework underlying our system \vispur. It contains four major components -- (1) confounder identification, (2) subgroup heterogeneity examination, (3) paradox or spuriousness reasoning, as well as (4) decision-making. \vispur uses a ranked-list of bar charts for confounder selection, a list of radar charts and a 2D cause-outcome space for subgroup characterization, a flow-based storyboard to explain spuriousness/paradox, and a lollipop visualization augmented with warning signals for a deeper diagnosis.}
  \label{fig:teaser}
}

% The causal graphs in randomized experiments (left) and observational studies (right). Notations $T$, $Y$, $X$ represents cause, outcome, as well as pretreatment covariates, respectively. Arrows $X \rightarrow T$, $X \rightarrow Y$, $T \rightarrow Y$ represent treatment propensity, prognostic tendency, as well as treatment effect, respectively. In randomized experiments (left), $X$ is independent from $T$, yet in observational studies (right), $X$ or subset of $X$ play as confounding variables affect the treatment assignment.