\begin{teaserfigure}
    \centering
    % Figure removed
    \caption{Our mesh density adaptation improves reconstruction quality for complex structures in inverse rendering (first two columns) and non-rigid surface registration (last two columns) 
    while retaining the original template meshing.}
    \label{fig:teaser}
\end{teaserfigure}

\begin{abstract}

In 3D shape reconstruction based on template mesh deformation, a regularization, such as smoothness energy, is employed to guide the reconstruction into a desirable direction. In this paper, we highlight an often overlooked property in the regularization: the vertex density in the mesh. Without careful control on the density, the reconstruction may suffer from under-sampling of vertices near shape details. We propose a novel mesh density adaptation method  to resolve the under-sampling problem. Our mesh density adaptation energy increases the density of vertices near complex structures via deformation to help reconstruction of shape details. We demonstrate the usability and performance of mesh density adaptation with two tasks, inverse rendering and non-rigid surface registration. Our method produces more accurate reconstruction results compared to the cases without mesh density adaptation. Our code is available at \url{https://github.com/ycjungSubhuman/density-adaptation}.


\end{abstract}