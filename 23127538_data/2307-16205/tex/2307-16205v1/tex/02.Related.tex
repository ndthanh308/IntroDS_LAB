\section{Related Work}
















\subsection{Template-based Shape Reconstruction}


Template mesh fitting is a widely studied approach to reduce the ill-posedness of 3D reconstruction.
Recent approaches in 3D shape reconstruction use spherical template meshes for reconstructing genus-0 3D shapes. Using the spherical template, reconstruction of 3D objects is simplified to a sphere deformation problem. In deep-learning based 3D reconstruction, single-view 3D reconstruction \cite{kato2018neural, wang2018pixel2mesh, wen2019pixel2mesh++} or multi-view 3D reconstruction \cite{chen2022multi} via deformation of spherical template mesh has been explored.

Recent template-based iterative reconstruction methods introduce new mechanisms for high-quality reconstruction.
In terms of optimization technique,
modifications to the Adam optimizer \cite{kingma2015adam} have been proposed to ensure smoothness \cite{nicolet2021large} or rotation equivariance \cite{ling2022vectoradam}.

In terms of surface regularization, Hanocka \Etal\ \shortcite{hanocka2020point2mesh} used convolutional neural networks to parameterize template deformation and regularize reconstruction with self-priors. Self-priors do not explicitly enforce smoothness, but periodic remeshing recovers a watertight manifold mesh. Nicolet \Etal\ \shortcite{nicolet2021large} introduced diffusion re-parameterization to propagate sparse gradients from image silhouettes. The smoothness from the diffusion helps maintaining a smooth manifold mesh throughout the iterative optimization. However, both methods do not address vertex density, potentially leading to sparse vertices, especially near extruded regions like a bunny's ear. To mitigate sparse vertices, periodic remeshing, which changes initial meshing in the template, was employed.




Template-based approach has achieved notable success for 3D reconstruction of human face, body, and animal, in addition to their non-rigid registration \cite{paysan20093d, li2017learning,  dai2018non, amberg2007optimal, schneider2009fast,sela2017unrestricted, bogo2014faust, bogo2017dynamic, hirshberg2012coregistration, yang2015sparse, zuffi20173d}. Nonetheless, the control of vertex density during reconstruction has not been explicitly addressed in those approaches. The lack of control on vertex density may lead to missing details due to under-sampling with sparse vertices on complex structures.


Our method provides the control for vertex density on a mesh in a template-based reconstruction.
Our mesh density adaptation method is based on template mesh deformation and preserves the initial vertex connectivity.
So, our method can be easily incorporated into the iterative deformation update step in a template-based 3D reconstruction framework. We demonstrate the performance of our method in inverse rendering and non-rigid shape registration.


\subsection{Remeshing and Mesh Optimization}

Remeshing has been extensively studied in geometry processing. The goal of remeshing is to obtain a 3D mesh that possess better meshing quality while retaining the original shape. For a thorough review on remeshing in general, refer to \cite{khan2020surface, botsch2010polygon-remesh}.

Remeshing frameworks handle adaptive density needed for preserving fine details in the original shape via re-sampling or re-triangulation. Alliez \Etal\ \shortcite{alliez2003anisotropic} use curvature-based adaptive sampling of vertices for anisotropic remeshing. Chen \Etal\ \shortcite{chen2012isotropic} employ curvature-based vertex sampling and mesh optimization for isotropic remeshing. Dey and Ray \shortcite{dey2010polygonal} perform isotropic remeshing to provide improved mesh density. Jakob \Etal\ \shortcite{jakob2015instant} handles adaptive resolution in remeshing by integrating custom metrics based on mean curvature.

While remeshing is useful for processing a fixed shape to obtain a better quality mesh, it is not suited for template-based reconstruction where the template mesh is deformed toward the target shape without vertex connectivity change.
Nealen \Etal\ \shortcite{nealen2006laplacian} proposes a mesh optimization method to improve mesh quality while retaining the initial vertex connectivity.
This method could be used as a post-process of template-based reconstruction for better mesh quality, but the mesh density is not handled in the method.
