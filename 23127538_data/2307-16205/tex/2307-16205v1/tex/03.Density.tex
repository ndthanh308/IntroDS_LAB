\section{Mesh Density Adaptation}
\label{sec:density}




\begingroup
\setlength{\columnsep}{5pt}
    

Our goal is to control local vertex density in template-based 3D shape reconstruction. Previous literature has defined density on a surface as a density function \cite{du1999centroidal, chen2011efficient}, which requires global parameterization. An alternative definition could be based on triangle area: small triangles in a region induce high vertex density and vice versa. Modeling mesh density via triangle area does not require global parameterization, but triangle area cannot uniquely determine vertex distance, e.g., a long-thin skewed triangle can have the same area as a small equilateral triangle.

In template-based shape reconstruction, triangle shape and edge length have been controlled via remeshing \cite{lyu2020differentiable, nicolet2021large,luan2021unified, palfinger2022continuous} to improve reconstruction quality. For example, Palfinger \shortcite{palfinger2022continuous} applies  an adaptive remeshing step after each shape update to keep optimal edge lengths. However, remeshing breaks the initial meshing, which can be problematic in applications like non-rigid surface registration, where shared meshing between the reconstruction results is crucial for the results to be useful, e.g., for building shape keys for animation and transferring textures.

\begin{wrapfigure}[4]{r}{0.16\textwidth}
\vspace{-13pt}
% Figure removed
\end{wrapfigure} 
In contrast, instead of remeshing, we use deformation to control mesh density. The deformation is guided by an adaptation energy based on edge length to enforce the desired mesh density. Edge lengths represent the distances between vertices. Intuitively, shorter edge lengths around a vertex result in denser sampling of vertices.


\endgroup













\subsection{Adaptation Energy}
We use edge lengths for adapting mesh density. 
Given an array of vertex positions $\mathbf{p} \in \mathbb{R}^{N \times 3}$, the average edge lengths $\mathbf{l}(\mathbf{p})$ for the vertices are calculated by
\begin{equation}
\mathbf{l}(p_i) = \dfrac{1}{|\mathcal{N}_i|}\sum_{j \in \mathcal{N}_i} \lVert {p}_i - {p}_j \rVert_2,
\end{equation} 
where $\mathbf{l}(p_i)$ is the average edge length for the i-th vertex $p_i$ and $\mathcal{N}_i$ is its 1-ring neighborhood. 

Using the edge lengths as a metric, we formulate the density-adaptation problem as optimizing the vertex position $\mathbf{p}$ to minimize the following adaptation energy $E_{a}$:
\begin{equation}
\begin{gathered}
\underset{\mathbf{p}}{\text{minimize  }} E_{a}(\mathbf{p}, \mathbf{l}'), \\
E_{a}(\mathbf{p}, \mathbf{l}') = \dfrac{1}{N} ||\mathbf{l}(\mathbf{p}) - \mathbf{l}' ||^2,
\end{gathered}
\label{eq:energy-adaptation}
\end{equation}
where $\mathbf{l}'$ denotes an array of desired edge lengths that encodes the target mesh density.
$N$ is the total number of vertices. 
Given the initial vertex position $\mathbf{p}_\text{init}$, we can iteratively update the vertex positions $\mathbf{p}$ by minimizing the energy $E_{a}$ via a gradient-based optimizer, such as Adam~\cite{kingma2015adam}.
In our template-based shape reconstruction, $\mathbf{p}_\text{init}$ is provided by the template mesh and $\mathbf{l}'$ is controlled to obtain a desirable mesh density of the reconstructed shape. Computation of desirable mesh density $\mathbf{l}'$ is detailed in \Sec{inverse}.




\subsection{Smoothness Regularization}

The solution of \Eq{energy-adaptation} may contain undesirable artifacts, such as self-intersections, depending on the target density $\mathbf{l}'$. We take \textit{diffusion re-parameterization}~\cite{nicolet2021large} as a regularization for \Eq{energy-adaptation} to avoid such problem.




Recently, Nicolet \Etal\ \shortcite{nicolet2021large} presented a modified gradient descent method for 3D mesh reconstruction from multi-view images. The modification effectively diffuses concentrated gradients from image silhouettes to other parts, allowing for stable deformation of a template mesh to target multi-view silhouettes. An insightful interpretation of their modified gradient descent method is that it is equivalent to ordinary gradient descent on the diffusion re-parameterization $\mathbf{u}$ defined as 
\begin{equation}
\mathbf{u}(\mathbf{p}) = (\mathbf{I} + \lambda \mathbf{L})\mathbf{p}, \label{eq:inverse-parameterization}
\end{equation}
\begin{equation}
\mathbf{p}(\mathbf{u}) = (\mathbf{I} + \lambda \mathbf{L})^{-1}\mathbf{u} \label{eq:re-parameterization}.
\end{equation}
$\mathbf{L}$ is the discrete Laplace operator \cite{taubin1995curve} and $\lambda$ is a constant weight.
The function $\mathbf{u}$ is diffused for a fixed time $\lambda$ using a backward Euler integration step \cite{baraff1998large}.

In this paper, we use the diffusion re-parameterization for mesh density adapation. In the gradient descent optimization, instead of directly updating the vertex position $\mathbf{p}$, we use gradients computed from \Eq{energy-adaptation} to update the re-parameterized coordinate $\mathbf{u}$ through \Eq{inverse-parameterization}, and 
the update of $\mathbf{p}$ is determined by the updated $\mathbf{u}$ through \Eq{re-parameterization}. Note that this indirect update of vertex position $\mathbf{p}$ differs from the direct update on $\mathbf{p}$, since \Eq{re-parameterization} defines $\mathbf{p}$ as diffused $\mathbf{u}$ and globally propagates the values of $\mathbf{u}$ onto $\mathbf{p}$.


Using diffusion re-parameterization, our final definition of the mesh density adaptation energy $E_{a}$ is given by
\begin{align}
&\underset{\mathbf{u}}{\text{minimize  }} E_{a}(\mathbf{p}(\mathbf{u}), \mathbf{l}'), 
    \label{eq:adaptation-re-parameterization}
\end{align} 
where the surface is regularized by simply changing the optimized variable from $\mathbf{p}$ to $\mathbf{u}$ in \Eq{energy-adaptation}.
We apply our adaptation energy in \Eq{adaptation-re-parameterization} to the tasks of inverse rendering (\Sec{inverse}) and non-rigid surface registration (\Sec{other}).

\paragraph{Laplacian regularization}
We briefly discuss the other alternative for smoothness regularization: the Laplacian regularization. We find Laplacian regularizer conflicts with the goal of our adaptation energy $E_a$. Common formulations for smoothness energy are \cite{sela2017unrestricted, wang2018pixel2mesh}:
\begin{equation}
\begin{gathered}
    E_{s}(\mathbf{p}) = \dfrac{1}{2}tr(\mathbf{p}^T\mathbf{L}\mathbf{p}) \text{ and} \\
    E_{s}(\mathbf{p}) = \dfrac{1}{2}\sum_i^N \lVert (\mathbf{L}\mathbf{p})_i \rVert^2 = \dfrac{1}{2}tr(\mathbf{p}^T\mathbf{L}^2\mathbf{p}),
    \label{eq:laplacian-regularizer}
\end{gathered}
\end{equation} where $tr(\cdot)$ is the trace of a matrix. 
The former is referred as the Laplacian energy and the latter as the bi-Laplacian energy.

In discrete surface meshes, the Laplace operator $\mathbf{L}$ is a sparse matrix $\mathbf{L} \in \mathbb{R}^{N \times N}$ that calculates the difference between each vertex's position and the weighted average of its 1-ring neighbors:
\begin{equation}
(\mathbf{L}\mathbf{p})_i = {p}_i - \sum_{j \in \mathcal{N}_i} w_{ij}{p}_j.
\label{eq:differential-coordinates-mesh}
\end{equation} 

Laplacian regularization forces a mesh density encoded in the discrete Laplace operator $\mathbf{L}$. For each $i$-th vertex, the Laplacian becomes small when ${p}_i$ coincides with the weighted average of the neighbors. Minimizing the Laplacian would move the vertices towards the weighted average of their neighbors, causing an adjustment in mesh density that may not match with our desired mesh density specified by $\mathbf{l}'$. Therefore, the Laplacian regularizers may conflict with our goal of adapting mesh density.























