\section{Discussion}
\label{sec:discussion-curvature}





Diffusion re-parameterization \cite{nicolet2021large} has been used for smooth propagation of sparse silhouette gradients in inverse rendering.
In this paper, we utilize diffusion re-parameterization for smoothing high-frequencies in dense gradients from density adaptation energy computed at all vertices.
Simply applying the update using the gradients from the density adaptation term without diffusion re-parameterization may lead to a noisy surface.

Our algorithm is effective at assigning appropriately high vertex density in large extruded regions, such as long ears of a monkey or bunny. Large extruded regions are manifested early in the iterative optimization, so the curvature changes in those regions are easily captured. Refer to the supplementary video for the visualization of the progression of the optimization.

\paragraph{Limitations}
Ideally, in addition to extruded structures, fine details would have benefited from high vertex densities in shape reconstruction. However, our current method cannot attract vertices to fine-scale details on wide and flat regions because fine details do not manifest until the late stage in the optimization. Visualization of this limitation is in the supplementary document. Handling fine structures in the density adaptation is left as future work.






