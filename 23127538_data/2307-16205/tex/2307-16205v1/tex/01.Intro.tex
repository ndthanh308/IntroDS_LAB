\section{Introduction}

3D surface reconstruction is a task where a 3D object or scene is obtained from observations. 
Various approaches have been studied for surface reconstruction based on parametric models \cite{blanz1999morphable}, implicit functions \cite{kazhdan2013screened}, and point set triangulation \cite{cazals2006delaunay}. In this paper, we focus on the template-based approach for surface reconstruction.

In template-based surface reconstruction, a 3D surface is represented as a deformed template mesh. This approach has been widely adopted in non-rigid surface registration frameworks that target 3D reconstructions in specific domains, such as face \cite{li2017learning} and human body \cite{bogo2014faust}. In the frameworks, a template mesh is deformed to fit multiple scans, and the template mesh provides a shared parameter domain for the scans. Then, the dense correspondence between the scans is trivially defined, enabling construction of data-driven parametric models \cite{paysan20093d} and texture transfer \cite{allen2003space} between the models.

Template-based surface reconstruction is also used for reconstructing 3D shapes in inverse rendering problems. In \cite{kato2018neural, nicolet2021large}, a sphere is used as the template mesh that provides effective discretization for a general class of genus-0 shapes. Using the template mesh, surface reconstruction from images reduces to updating 3D positions of template mesh vertices to match the rendering results. The deformed template with updated vertex positions is rendered using a differentiable renderer \cite{laine2020modular} and is optimized via gradient-based optimization.

Template-based surface reconstruction is usually formulated as an iterative optimization of an energy of the form $E_g(\mathbf{p}, \mathbf{o})$, where $\mathbf{p} \in \mathbb{R}^{N \times 3}$ is template vertex position and $\mathbf{o}$ is observation, e.g., images or a given 3D surface. $E_g$ defines the primary goal of the reconstruction, e.g., photometric loss in inverse rendering or Chamfer distance in non-rigid registration. However, as the energy $E_g$ only provides partial or noisy information on the update on $\mathbf{p}$, the optimization is usually accompanied with a regularization on $\mathbf{p}$. 

Popular regularizations on $\mathbf{p}$ are based on enforcing the smoothness of the surface. Usually, an additional smoothness term is defined and added to the original energy $E_g$ \cite{bogo2017dynamic, sela2017unrestricted, wang2018pixel2mesh}. Recent work on inverse rendering proposed new regularization that is based on re-parameterization of $\mathbf{p}$ \cite{nicolet2021large}. However, most regularizations in previous surface reconstruction frameworks only focus on the smoothness of $\mathbf{p}$ and do not handle the density of template mesh vertices.

Vertex density is an important parameter to control for accurate reconstruction of a shape using a fixed number of vertices in the template mesh. Regions with high curvatures require more position samples to express shape details accurately. We observe under-sampling in complex structures prohibits accurate reconstructions in both inverse rendering and non-rigid surface registration.

To this end, our key idea is to design a mesh density adaptation energy that is applied along with smoothness regularization. Our adaptation energy is defined with respect to edge lengths around each vertex. We calculate desired edge lengths to enforce desired adaptive density at each step of the iterative process for optimizing $E_g$. The desired adaptive density is calculated using intermediate reconstruction results to push vertices towards high-curvature regions.
Our density adaptation greatly improves the reconstruction of shape details by resolving the under-sampling problem (\Fig{teaser}).


In summary, our main contributions are as follows:
\begin{itemize}
    \item We present a novel mesh density adaptation method 
    that improves the reconstruction of shape details 
    in the template-based surface reconstruction.
    \item We apply mesh density adaptation to inverse rendering and show improvements in the reconstruction accuracy of complex structures.
    \item We present a non-rigid surface registration method based on mesh density adaptation that achieves faithful reconstruction of complex shapes and meaningful correspondences.
\end{itemize}
