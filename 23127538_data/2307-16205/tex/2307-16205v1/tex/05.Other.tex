\section{Non-rigid Surface Registration}
\label{sec:other}

In this section, we demonstrate application of our mesh density adaptation to the non-rigid surface registration task. 
In our setting of non-rigid surface registration, given a set of 3D meshes, a template mesh is deformed to fit the input meshes while retaining the original template vertex connectivity. The resulting deformed template meshes should also share meaningful point-to-point correspondences.
Such shared point-to-point correspondences among multiple meshes with the same vertex connectivity enable useful applications, such as construction of probabilistic shape models \cite{paysan20093d, li2017learning}, texture transfer \cite{allen2003space}, and seamless shape blending \cite{yu2004mesh, sorkine2004laplacian}.

We find that accurate reconstruction of an input 3D mesh using a template mesh can be achieved by simple modification of our density-adapting inverse rendering framework introduced in \Sec{inverse}, where the only major difference would be the data loss to define the fitting. However, naive fitting of the template mesh may not provide meaningful point-to-point correspondences among the deformed template meshes, although the same vertex connectivity is guaranteed. A common approach for obtaining such correspondences is to annotate a set of sparse landmarks on the template \cite{amberg2007optimal} as well as on the input meshes. The known correspondences between the landmarks can guide the registration process to establish meaningful point-to-point correspondences among the template and the input meshes. To this end, we extend our density adaptation framework for template mesh fitting to reflect the correspondences of sparse landmarks.

We use a sphere as the template mesh as in \Sec{inverse} and reconstruct input meshes using adaptive mesh densities. Then, we resample on the sphere the sparse landmarks annotated on input meshes while reflecting the adaptive mesh densities of the reconstructed meshes. Finally, we fit the sphere again to input meshes allowing adaptive mesh densities, but now reflecting the correspondences among the resampled landmarks on the sphere and the landmarks on input meshes. In the following, we explain each step and show improvements in non-rigid template registration results using 3DCaricshop dataset \cite{qiu20213dcaricshop}, which contains artist-sculpted and exaggerated 3D faces.


\subsection{Template Mesh Fitting}
\label{sec:sphere-fitting}

Given a target 3D mesh $\mathcal{M}$ and the closest points $\hat{\mathbf{p}} \in \mathbb{R}^{N \times 3}$ on the surface $\mathcal{M}$ searched with intermediate result $\mathbf{p}$, the data loss for the fitting $E_d$ is defined as
\begin{equation}
E_d(\mathbf{p}, \hat{\mathbf{p}}) = D_c(\mathbf{p}, \hat{\mathbf{p}}) + D_n(\mathbf{p}, \hat{\mathbf{p}}),
\end{equation} where the Chamfer distance $D_c$ is
\begin{equation}
    D_c(\mathbf{p}, \hat{\mathbf{p}}) = \dfrac{1}{N} \sum_i^N \lVert p_i - \hat{p}_i \rVert_2
\end{equation} and the normal loss $D_n$ is
\begin{equation}
    D_n(\mathbf{p}, \hat{\mathbf{p}}) = \dfrac{1}{N} \sum_i^N \left(1 - n_i \cdot \hat{n}_i \right)
\end{equation} 
with vertex normals $n_i$ and $\hat{n}_i$ of $p_i$ and $\hat{p}_i$, respectively.

Replacing the photometric loss $E_p$ with $E_d$ in \Eq{inverse-new}, the optimization for template fitting is
\begin{equation}
    \underset{\mathbf{u}}{\text{minimize }} E_{d}(\mathbf{p}(\mathbf{u}), \hat{\mathbf{p}}) + w_u E_{a}(\mathbf{p}(\mathbf{u}), \mathbf{l}'_u(\mathbf{p}))+ w_k E_{a}(\mathbf{p}(\mathbf{u}), \mathbf{l}'_k(\mathbf{p})).
    \label{eq:nonrigid-base}
\end{equation}

The results of \Eq{nonrigid-base} are presented in \Fig{sphere-fitting}. Artist-sculpted 3D cartoon faces in 3DCaricShop \cite{qiu20213dcaricshop} are registered using our method. We compare our results with the case of not applying our mesh density adaptation (i.e., $w_u = w_k = 0$ in \Eq{nonrigid-base}), where the shape is only regularized with diffusion re-parameterization. We test the two methods in two settings for the template sphere mesh: with 2.5K vertices and 10K vertices. In both cases, our density adaptation enables accurate reconstruction of details by placing more vertices around complex structures.




\subsection{Landmark Resampling}
\label{sec:template-generation}


We aggregate input mesh landmarks to construct template sphere landmarks.
Given a set of input meshes with landmark annotations, we first fit the template sphere to the input meshes using the optimization in \Eq{nonrigid-base}. For each fitting $\mathbf{p}_i$, we determine the landmark vertex indices on the template sphere by finding the closest vertex in $\mathbf{p}_i$ for each landmark in the input mesh $\hat{\mathbf{p}}_i$. 
We use $\mathbf{c}_i$ to denote the landmark positions on the sphere corresponding to the landmark vertex indices from the $i$-th fitting $\mathbf{p}_i$.


We then select $\mathbf{c}_1$ as the reference set of landmarks and align other landmark sets to $\mathbf{c}_1$ by computing a rotation matrix $\mathbf{R}_i$ for each $\mathbf{c}_i$.
The rotation matrix is calculated \cite{sorkine2009least} as 
\begin{equation}
    \mathbf{R}_i = \mathbf{U}_i\mathbf{V}_i^T,
\end{equation} where orthogonal matrices $\mathbf{U}$ and $\mathbf{V}$ are computed using SVD
\begin{equation}
    \mathbf{c}_i^T \mathbf{W} \mathbf{c}_1  = \mathbf{U}_i\mathbf{\Sigma}_i \mathbf{V}_i^T.
\end{equation} 
$\mathbf{\Sigma}$ is a diagonal singular value matrix. $\mathbf{W}$ is a diagonal matrix for weighting the importance of landmarks. We set a high weight for the landmark on the tip of the nose. For more details on the landmark weighting, refer to the supplementary material.


Finally, the landmarks are selected as the vertices on the sphere mesh closest to $\bar{\mathbf{c}}$, which is the average of the aligned landmarks:
\begin{equation}
    \bar{\mathbf{c}} = \dfrac{1}{M}\sum_i^M \mathbf{c}_i \mathbf{R}_i^T,
\end{equation} where $M$ is the number of input meshes.

\subsection{Non-rigid Registration with Landmarks}
\label{sec:registration-landmark}

Using the resampled landmarks on the sphere, our non-rigid registration is formulated similarly to \Eq{nonrigid-base}, but with an additional landmark loss term:
\begin{equation}
    \underset{\mathbf{u}}{\text{minimize }} E + \dfrac{1}{B}\sum_i^B \lVert c_i - k_i \rVert^2,
\end{equation} where $E=E_d+w_uE_a+w_kE_a$ is the original loss in \Eq{nonrigid-base}, $c_i$ is a landmark position on $\mathbf{p}(\mathbf{u})$ during the optimization, $k_i$ is the corresponding landmark on the target 3D shape example, and $B$ is the number of landmarks.

% Figure environment removed






\subsection{Results}
\label{sec:nonrigid-result}

We evaluate our non-rigid registration method on 3DCaricShop \cite{qiu20213dcaricshop} dataset, which contains artist-sculpted cartoon 3D models with highly exaggerated facial features. The dataset provides two sets of meshes: Raw meshes and registered meshes. The raw meshes were sculpted by artists and are highly detailed, but do not share vertex connectivity. The registered meshes provided by the authors are the results of running NICP \cite{amberg2007optimal} on the raw meshes using a 3D face template with 11,551 vertices.

In \Fig{caricshop}, we compare the results of our proposed non-rigid registration and the original NICP results provided with the dataset. NICP is the standard surface registration method in 3D face domain and used for constructing 3D face datasets, such as BFM \cite{paysan20093d} and LSFM \cite{booth2018large}. Our template mesh is a landmark-annotated sphere mesh with 10K vertices. Our method achieves detailed and accurate reconstruction compared to the original registration, using even lower vertex budget. %

The key factor behind our accurate registration is that we do not use a manually authored template mesh; We use a sphere as the template and determine the landmarks on the sphere in a data-driven manner. 
Although the meshing of our sphere template mesh is uniform, the spacing between landmarks on the sphere reflects the shape complexity of the region between the landmarks in 3D shape examples. Note that landmark spacing on the sphere is determined using the fitting results of 3D shape examples with adaptive density control, and the spacing increases for highly detailed regions.


\begin{table}
\caption{
Comparison of errors for non-rigid reconstruction. Left: Chamfer distance. Right: MSE of normal vectors. DR is our method without density adaptation. \textit{lmk} is landmark usage. Numbers in parenthesis show template vertex count.
}
\label{tbl:error-nonrigid}
\resizebox{.995\linewidth}{!}{
\begin{tabular}{cccccc}
    &\cite{qiu20213dcaricshop} (11K)&DR (2.5K)&Ours (2.5K)&DR (10K)&Ours (10K)\\
    \hline
    w/o lmk &     -/-           & 8.6e-5/0.025 & 8.7e-5/0.024 & 7.4e-5/0.016  & 6.9e-5/0.013\\
    top 20\% sal. &     -/-     & 1.6e-4/0.085 & 1.5e-4/0.078 & 1.3e-4/0.056  & 1.0e-4/0.045\\
    top 5\% sal. &     -/-      & 2.8e-4/0.115 & 2.5e-4/0.111 & 2.3e-4/0.084  & 1.7e-4/0.075\\
    \hline
    w/ lmk        &   0.046/0.11     & 8.5e-5/0.025 & 8.6e-5/0.025 & 7.0e-5/0.014 & \textbf{6.8e-5/0.013}\\
    top 20\% sal. &   0.022/0.22     & 1.4e-4/0.082 & 1.4e-4/0.081 & 1.1e-4/0.049 & \textbf{1.0e-4/0.046}\\
    top 5\% sal.  &   0.022/0.27     & 2.2e-4/0.113 & 2.2e-4/0.114 & 1.7e-4/0.079 & \textbf{1.6e-4/0.076}\\
    
    
\end{tabular}
}
\end{table}

% Figure environment removed


    


\Tbl{error-nonrigid} shows quantitative evaluation on our non-rigid registration. We use the same error metric as in \Sec{inverse-rendering-results}. We perform fitting on hand-sculpted meshes of the 3DCaricShop dataset \cite{qiu20213dcaricshop} and measure average errors after aligning landmarks with similarity transformation. Original fittings in 3DCaricShop show the highest error due to sparse vertices around facial components. Surface registration based on diffusion re-parameterization without density adaptation reduces the error significantly, but suffers from sub-optimal vertex distribution to express details in the target shape. Our method with density adaptation achieves the lowest error by distributing vertices effectively to express the details.

Our density adaptation is effective when a sufficient number of vertices is given; While the results with 2.5K template vertices do not show large improvements in term of the error metric, using 10K template vertices results in a significant drop in the error compared to the case of not using the density adaptation.

Using our re-sampled landmarks in the registration lowers the error while additionally granting dense correspondence between the shapes. \Fig{trans_plant} shows that the dense point-to-point correspondence obtained from our non-rigid registration is valid by applying it for shape transplanting \cite{sorkine2004laplacian}. Due to the dense correspondence, the Laplacian computed from a region in the source shape can be copied onto the corresponding region in the base mesh. Dense correspondence also enables texture transfer, as shown in the supplementary document.








