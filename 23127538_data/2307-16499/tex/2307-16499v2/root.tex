%%%%%%%%%%%%%%%%%%%%%%%%%%%%%%%%%%%%%%%%%%%%%%%%%%%%%%%%%%%%%%%%%%%%%%%%%%%%%%%%
%2345678901234567890123456789012345678901234567890123456789012345678901234567890
%        1         2         3         4         5         6         7         8

\documentclass[letterpaper, 10 pt, conference]{ieeeconf}  % Comment this line out if you need a4paper

%\documentclass[a4paper, 10pt, conference]{ieeeconf}      % Use this line for a4 paper

\IEEEoverridecommandlockouts                              % This command is only needed if 
                                                          % you want to use the \thanks command

\overrideIEEEmargins                                      % Needed to meet printer requirements.

%In case you encounter the following error:
%Error 1010 The PDF file may be corrupt (unable to open PDF file) OR
%Error 1000 An error occurred while parsing a contents stream. Unable to analyze the PDF file.
%This is a known problem with pdfLaTeX conversion filter. The file cannot be opened with acrobat reader
%Please use one of the alternatives below to circumvent this error by uncommenting one or the other
%\pdfobjcompresslevel=0
%\pdfminorversion=4

% See the \addtolength command later in the file to balance the column lengths
% on the last page of the document

% The following packages can be found on http:\\www.ctan.org
%\usepackage{graphics} % for pdf, bitmapped graphics files
%\usepackage{epsfig} % for postscript graphics files
%\usepackage{mathptmx} % assumes new font selection scheme installed
%\usepackage{times} % assumes new font selection scheme installed
\usepackage{amsmath} % assumes amsmath package installed
\usepackage{amssymb}  % assumes amsmath package installed
%\usepackage{soul}
%\usepackage{biblatex}
\usepackage{graphicx}
\usepackage{capt-of}
\usepackage{bm}
\usepackage{caption, subcaption}
\usepackage{subcaption}

\makeatletter
\let\NAT@parse\undefined
\makeatother
\usepackage[hidelinks]{hyperref}
\usepackage[dvipsnames]{xcolor}
\usepackage{dsfont}

\usepackage{outlines}
\usepackage{amsfonts}
\usepackage{booktabs}
\usepackage{multicol}
\usepackage{siunitx}
\usepackage{tikz,graphics,float,epsf}
\usepackage{tabularx}
\usepackage{times}
\usepackage{bbm}
\usetikzlibrary{calc}
\usepackage{xspace}
\usepackage{layouts}

\usepackage[normalem]{ulem}

\definecolor{dark_green}{rgb}{0.18, 0.55, 0.35} 
\definecolor{dark_orange}{rgb}{0.8, 0.45, 0.0} 


% Introduce a new counter for counting the nodes needed for circling
\newcounter{nodecount}
% Command for making a new node and naming it according to the nodecount counter
\newcommand\tabnode[1]{\addtocounter{nodecount}{1} \tikz \node (\arabic{nodecount}) {#1};}

% Some options common to all the nodes and paths
\tikzstyle{every picture}+=[remember picture,baseline]
\tikzstyle{every node}+=[inner sep=0pt,anchor=base,
text depth=.25ex,outer sep=1.5pt]
\tikzstyle{every path}+=[thick, dashed, rounded corners]

\providecommand{\gmargo}[1]{{\color{blue}#1 }\textbf{\color{blue}- \textbf{Gabe}}}
\providecommand{\rev}[1]{{\color{red}#1}\textbf{\color{red}}}




\title{\LARGE \bf
Learning Generalizable Tool Use\\ with Non-rigid Grasp-pose Registration
}


\author{Malte Mosbach and Sven Behnke% <-this % stops a space
\thanks{Both authors are with the Autonomous Intelligent Systems group, University of Bonn, Germany;
        {\tt\small mosbach@ais.uni-bonn.de}}%
%\thanks{}%
}

\makeatletter
\let\@oldmaketitle\@maketitle% Store \@maketitle
\renewcommand{\@maketitle}{\@oldmaketitle% Update \@maketitle to insert...
      \begin{center}\vspace*{1ex}
      % Figure removed
      \label{overview}
      \vspace{-0.5cm}
      \captionof{figure}{A single grasping demonstration is transferred to other instances of a class, including instances not in the training set and only partially observed (left). 
      These generalized demonstrations guide the learning of an interactive policy able to operate a variety of tools (right).}
      \end{center}
      \vspace{-0.5cm}
    }
\makeatother

\begin{document}

%\printinunitsof{cm}\prntlen{\columnwidth}


\maketitle
\thispagestyle{empty}
\pagestyle{empty}

% Fix the numbering of all figures
\renewcommand\thefigure{\arabic{figure}}    
\setcounter{figure}{1} 


%%%%%%%%%%%%%%%%%%%%%%%%%%%%%%%%%%%%%%%%%%%%%%%%%%%%%%%%%%%%%%%%%%%%%%%%%%%%%%%%
\begin{abstract}
Tool use, a hallmark feature of human intelligence, remains a challenging problem in robotics due the complex contacts and high-dimensional action space.
In this work, we present a novel method to enable reinforcement learning of tool use behaviors. 
Our approach provides a scalable way to learn the operation of tools in a new category using only a single demonstration. 
To this end, we propose a new method for generalizing grasping configurations of multi-fingered robotic hands to novel objects. 
This is used to guide the policy search via favorable initializations and a shaped reward signal.
The learned policies solve complex tool use tasks and generalize to unseen tools at test time.
Visualizations and videos of the trained policies are available at {\color{Blue} \url{https://maltemosbach.github.io/generalizable_tool_use}}.
\end{abstract}


%%%%%%%%%%%%%%%%%%%%%%%%%%%%%%%%%%%%%%%%%%%%%%%%%%%%%%%%%%%%%%%%%%%%%%%%%%%%%%%%
\section{Introduction}
Current quantum hardware is unable to carry out universal quantum computations due to the buildup of errors that occur during the computation. 
The magnitude of the individual error is currently above the value that the Threshold Theorem requires in order to kick-start quantum error correction and fault-tolerant quantum computation~\cite[Section 10.6]{nielsen_chuang_2010}. 
Although the experimentally achieved fidelity rates are promising and the error bounds are inching closer to the required threshold, we will have to work for the foreseeable future with quantum hardware with errors that build-up during the computation.  This implies that we can only do a limited number of steps before the output of the computation has become completely uncorrelated with the intended one.

For fault-tolerant quantum computing, we repeat four steps: 
1) We apply a number of single and two-qubit quantum gates, in parallel whenever possible; 
2) We perform a syndrome measurement on a subset of the qubits; 
3) We perform fast classical computations to determine which errors have occurred and how to correct them; 
and, 4) We apply correction terms based on the classical computations.
We then repeat these four steps with a next sequence of gates. 
These four steps are essential to fault-tolerant quantum computing. 


The starting point of this work is to use the four steps outlined above, not to carry out error correction and fault-tolerant computation, but to enhance short, constant-depth, {\em uncorrected} quantum circuits that perform single qubit gates and {\em nearest-neighbor} two qubit gates. 
Since in the long run we will have to implement error-correction and fault-tolerant computation anyhow, and this is done by such a four-step process, why not make other use of this architecture? Moreover, on some of the quantum hardware platforms, these operations are already in place.
Embracing this idea we naturally arrive at the question: what is the computational power of \textit{low-depth} quantum-classical circuits organized as in the four steps outlined above? 
We thus investigate circuits that execute a small, ideally constant, number of stages, where at each stage we may apply, in parallel, single qubit gates and {\em nearest-neighbor} two qubit gates, followed by measurements, followed by low-depth classical computations of which the outcome can control quantum gates in later stages. 
It is not clear, at first, whether such circuits, especially with constant depth, can do anything remotely useful. 
But we will see that this is indeed the case: many quantum computations can be done by such circuits in constant depth. 
By parallelizing quantum computations in this way, we improve the overall computational capabilities of these circuits, as we do not incur errors on qubits that are idle, simply because qubits are not idle for a very long time. 
Furthermore, reducing the depth of quantum circuits, at the cost of increasing width, allows the circuit to be run faster even if errors occur.

The first usage of such a four-step layout, not to do error correction, but to perform computations, can be found in the paradigm of measurement-based quantum computing~\cite{gottesman1999demonstrating,raussendorf2001one,jozsa2006introduction,clark2007generalised}: 
A universal form of quantum computing where a quantum state is prepared and operations are performed by measuring qubits in different bases, depending on previous measurements and intermediate measurements.

\citeauthor{PhamSvore2013} were the first to formalize the four-step protocol for performing computations~\cite{PhamSvore2013}. They included specific hardware topologies by considering two-dimensional graphs for imposing constraints on qubit interactions. In their model, they develop circuits for particularly useful multi-qubit gates, including specifying costs in the width, number of qubits, depth, number of concurrent time steps, size, and total number of non-Identity operations.
As a result, they find an algorithm that factors integers in polylogarithmic depth.
\citeauthor{Browne:2011} showed that the main tool in the work by \citeauthor{PhamSvore2013}, the fan-out gate, can also be replaced by additional log-depth classical computations in the measurement-based quantum computing setting~\cite{Browne:2011}.

More recently, \citeauthor{Cirac:2021} introduced a scheme to implement unitary operations involving quantum circuits combined with Local Operations and Classical Communication ($\mathsf{LOCC}$) channels: $\mathsf{LOCC}$-assisted quantum circuits~\cite{Cirac:2021}. Similarly to the four-step scheme we just described, they allow for a short depth circuit to be run on the qubits, followed by one round of $\mathsf{LOCC}$, in which ancilla qubits are measured and local unitaries are applied based on the measurement outcomes. They show that in this model any 1D transitionally invariant matrix-product state (MPS) with fixed bond dimension is in the same phase of matter as the trivial state. Similar ideas can be found in~\cite{TVV_NonAbelianTopologicalOrder_2022, tantivasadakarn2021long}.

In this work, we introduce a new model, called \textit{Local Alternating Quantum-Classical Computations} ($\LAQCC$). In this model we alternate between running quantum circuits (constrained by locality), ending in the measurement of a subset of qubits, and fast classical computations based on the measurement results. The outcome of the classical computations are then used to control future quantum circuits. We allow for flexibility in this model, by giving different constraints to the power of both the quantum circuits and the classical circuits as well as the number of alternations between them. 
Most attention will be given to $\LAQCC$ containing quantum circuits of constant depth, classical circuits of logarithmic depth and at most a constant number of alternations between them. 
Any circuit constructed in this model is considered to be of constant depth. 
We restrict ourselves to logarithmic depth classical computations, as this is the first natural and non-trivial extension beyond constant-depth classical computations. 
Constant-depth classical computations do however also have an equivalent constant-depth quantum implementation.

The definition of $\LAQCC$ sharpens the original definition of \citeauthor{PhamSvore2013} by adding constraints to the intermediate classical computations. This allows us to bound the power of $\LAQCC$ from above. 

The main result of \citeauthor{Cirac:2021}, that 1D translational invariant MPS with fixed bond dimension can be prepared by $\mathsf{LOCC}$-assisted circuits, relies on local symmetries of the MPS. These symmetries allow them to prepare local states (on a constant number of qubits) and glue them together by doing one round of the appropriate entangling measurement and corrections, after which they run a round of local unitaries to get the desired result. This general scheme for preparing states that exhibit an MPS description with the appropriate local symmetries requires only geometrically local unitaries and one round of measurement and corrections an therefore is accessible in $\LAQCC$. Studying different local symmetries, known as Symmetry Protected Topological (SPT) phases of matter, to find measurement-based constant depth circuits for states is a broad ongoing field of research~\cite{TVV_NonAbelianTopologicalOrder_2022, tantivasadakarn2021long, smith2023deterministic}. 
All these schemes have a $\LAQCC$ implementation.

%$\LAQCC$-circuits also exist for general schemes of preparing local states, based on the local tensors, and gluing them together using one round of entangled measurement and corrections, based on the local symmetry. 
%The main result of \citeauthor{Cirac:2021}, that 1D translational invariant MPS with fixed bond dimension can be prepared by $\mathsf{LOCC}$-assisted circuits, relies heavily on local symmetries of the MPS and as a result also has an equivalent $\LAQCC$ implementation. 
%The corrections applied after the measurement round are local unitaries depending on the local symmetries of the MPS. 

 

%This general scheme of preparing local states, based on the local tensors, and gluing it together by doing one round of entangled measurement and corrections, based on the local symmetry, is accessible in $\LAQCC$.
Note however that \citeauthor{Cirac:2021} also suggest a circuit for the $W$-state.
This circuit uses sequentially and dependent measurement-based corrections of the ancilla qubits. 
These dependent measurements translate to sequential alternations between the quantum and classical circuits and therefore increase the total depth to linear depth, exceeding the constant-depth constraints imposed by $\LAQCC$-circuits. 

We study the power of the $\LAQCC$ model with respect to state preparation, showing that even with only constant quantum-depth and logarithmic classical depth it remains possible to prepare states with long-range entanglement.
Another surprising result is that it is unlikely that $\LAQCC$ circuits are classically simulatable. We show that any instantaneous quantum polynomial-time (IQP) circuit~\cite{Bremner2010,Shepherd2009} has an $\LAQCC$ implementation.
Classical simulation of IQP circuits implies the collapse of the polynomial hierarchy to the third level, which is not believed to be true~\cite{Bremner2017}. Therefore, we expect that $\LAQCC$ circuits are unlikely to be classically simulatable. We bound the power of $\LAQCC$ by showing that it is contained in $\QNC^1$, the class of polynomial-size, log-depth circuits.

Next, we also study the power that intermediate classical calculations can add to quantum computations, by considering a new model that alternates between polynomially many polynomial-depth quantum circuits and unbounded classical computations
We study this model by doing a complexity theoretical analysis, where we draw inspiration from the notions of complexity given by \citeauthor{RosenthalYuen:2022}, \citeauthor{MetgerYuen:2023}, and \citeauthor{Aaronson:2004}.
All three complexity notions are based on the notion of state preparation, instead of more traditional definition of complexity such as the decidability of a computational problem. 
The first two consider classes based on sequences of quantum states preparable by a polynomial-sized quantum circuit, where the circuits are uniformly generated by a computational class, for instance, the class $\mathsf{PSPACE}$, which results in the complexity class $\mathsf{StatePSPACE}$~\cite{RosenthalYuen:2022,MetgerYuen:2023}.
The third notion considers a relative complexity, where the complexity is measured between two given states, and is measured by the number of gates, from a given gate-set, required to transform one state in another state~\cite{Aaronson:2004}. 
For our definition of state preparation complexity, we drop the uniformity constraint from~\cite{RosenthalYuen:2022,MetgerYuen:2023} and define a class as $\mathsf{StateX}$, which refers to states preparable by circuits of type $\mathsf{X}$. 
As an example, if $\mathsf{X} = \QNC^0$, this results in the class $\mathsf{StateQNC^0}$, which is the set of states preparable from the $\ket{0}^n$ state by poly-size constant-depth circuits. 
This notion is similar to the relative complexity from~\cite{Aaronson:2004}, where one state is the  $\ket{0}^n$ state and instead of counting the number of gates we consider the set of states preparable by a fixed number of gates. Using this notion of complexity we show that any state preparable by an $\LAQCC^*$ circuit is also preparable by a $\mathsf{PostQPoly}$ circuit, the class of circuits of polynomial depth with an additional post-selection gate. 

All Clifford circuits have a constant-depth $\LAQCC$ implementation, implying that any stabilizer state can be implemented by a constant-depth $\LAQCC$ circuit, see Section~\ref{sec:clifford_circuits} for a proof of this statement. 
Efficient circuits for stabilizer states have been known already through measurement-based quantum computing. Therefore this paper focuses on the preparation of non-stabilizer states, and as a surprising result we find novel constant-depth protocols for four very natural classes of non-stabilizer states.
Despite the extensive research into these four classes of non-stabilizer states and the many applications of them, no efficient constant- or low-depth state preparation protocols are known yet. We specifically consider these four classes as they are all often used as initial states in other algorithms.

The first state is a uniform superposition over an arbitrary number of states. 
This state finds applications in many quantum algorithms, as they often start with a uniform superposition over multiple states. 
This superposition is often achieved by applying Hadamard gates to every qubit due to its simplicity to prepare. 
Yet, the analysis of many algorithms, such as Shor's algorithm~\cite{Shor:1997}, would benefit from a different initial superposition. 
The circuit to prepare the uniform superposition over an arbitrary number of states uses an exact version of Grover search as a subroutine, that turns a probabilistic circuit, with a known constant probability of success, into a deterministic circuit. 
We use the circuit for preparing a uniform superposition over an arbitrary number of states as a subroutine in the next two quantum state preparation protocols. 

The second state is the $W$-state, the uniform superposition over all computational basis states of Hamming-weight~$1$, a natural long-ranged entangled state that displays a fundamentally nonequivalent type of entanglement from the Greenberger–Horne–Zeilinger state~\cite{WState:2000}, for which $\LAQCC$-type constant-depth circuits were previously known~\cite{PhamSvore2013, Cirac:2021}. 
The $W$-state is often used as benchmark for new quantum hardware~\cite{Haffner2005,Neeley2010,GarciaPerez:2021}. 
A novel way to prepare the $W$-state therefore gives a new way to benchmark different quantum devices with each other. 
A circuit for preparing the $W$-state was given in~\cite{Cirac:2021}, but this implementation requires sequentially alternating measurements followed by local unitaries, which in the $\LAQCC$ model is not considered to be of constant depth. 
We improve this protocol by giving an $\LAQCC$ implementation of the $W$-state, based on a compress-uncompress method that links the one-hot and binary encoding of integers.

The third state considered is the Dicke state, a generalization of the $W$-state, a superposition over all computational basis states with Hamming-weight $k$~\cite{Dicke:1954}. 
Dicke states have relevance in various practical settings.
For instance, for quantum game theory~\cite{zdemir2007}, quantum storage~\cite{Bacon_Compress:2006,Plesch:2010}, quantum error correction~\cite{ouyang2014permutation}, quantum metrology~\cite{toth2012multipartite}, and quantum networking~\cite{prevedel2009experimental}. 
Dicke states have been used as a starting state for variational optimization algorithms, most notably Quantum Alternating Operator Ansatz (QAOA)~\cite{Hadfield2019}, to find solutions to problems such as Maximum k-vertex Cover~\cite{Brandhofer2022,cook2020quantum}.
The ground states of physical Hamiltonians describing one-dimensional chains tend to show a resemblance to Dicke states such as states resulting from the Bethe ansatz, making them an ideal starting state when investigating the ground state behavior of these Hamiltonians~\cite{TDL_BetheAnsatzDerivation:2010,B_ExcitedStateQuantumPhaseTransitions:2013,DickeTransitions:2021}. 
For instance, the algorithm by \citeauthor{van2021preparing}, who give an algorithm to prepare the Bethe ansatz eigenstates of the spin-1/2 XXZ spin chain, starts by first preparing a Dicke state~\cite{van2021preparing}. 
A Dicke-state preparation protocol based on the compress-uncompress methodology used in the $W$-state furthermore finds applications in entanglement distillation, where the entanglement of a large state is concentrated on only a few qubits. 
Efficient deterministic circuits for preparing Dicke states have been proposed by \citeauthor{bartschi2019deterministic}~\cite{bartschi2019deterministic, bartschi2022deterministic_short_depth}. 
They provide a quantum circuit of depth $\mathO(k \log(\frac{n}{k}))$, allowing arbitrary connectivity, to prepare a Dicke state, which they conjecture to be optimal when $k$ is constant. 
In this work, we provide a constant-depth $\LAQCC$ circuit below their conjectured bound already for constant $k$. 
However, this does not directly disprove their conjecture, as we allow for intermediate measurements and classical computations. 
More significantly, we even construct constant-depth $\LAQCC$ circuits for $k = \mathO(\sqrt{n})$ greatly improving their bound.
This construction extends the compress-uncompress method for the $W$-state combined with additional subroutines. 

We continue with a log-depth state preparation protocol for the Dicke-state for arbitrary $k$. 
This protocol implements an efficient transformation between the factoradic number representation and the combinatorial number representation of a positive integer. 
The combinatorial number representation relates directly to the Dicke state. 
The provided efficient transformation between number representation systems might be of independent interest. 

We conclude by modifying our protocol for preparing a Dicke-state to a protocol that prepares quantum many-body scar states in constant-depth. 
These states have low entanglement and longer coherence times than states with similar energy density.
These characteristics make many-body scar states interesting to analyze and relevant within physics.
Many-body scar states appear for instance in the AKLT model~\cite{AKLT:1987,MRBAR:2018,MRB:2018} and different spin models~\cite{SI:2019,MOBFR:2020}.
Known methods for preparing these states have polynomial-depth~\cite{Gustafson:2023}, whereas our circuit has constant depth. 

% We conclude by studying the power that intermediate classical calculations can add to quantum computations. 
% In this study, we define a new model that relaxes constant-depth quantum circuits to polynomial depth quantum circuits, log-depth classical calculations to unbounded classical computations and a constant number of alternations to a polynomial number of alternations. 
% We call this model $\LAQCC^*$. 
% We study this model by doing a complexity theoretical analysis, where we draw inspiration from the notions of complexity given by \citeauthor{RosenthalYuen:2022}, \citeauthor{MetgerYuen:2023}, and \citeauthor{Aaronson:2004}.
% All three complexity notions are based on the notion of state preparation, instead of more traditional definition of complexity such as the decidability of a computational problem. 
% The first two consider classes based on sequences of quantum states preparable by a polynomial-sized quantum circuit, where the circuits are uniformly generated by a computational class, for instance, the class $\mathsf{PSPACE}$, which results in the complexity class $\mathsf{StatePSPACE}$~\cite{RosenthalYuen:2022,MetgerYuen:2023}.
% The third notion considers a relative complexity, where the complexity is measured between two given states, and is measured by the number of gates, from a given gate-set, required to transform one state in another state~\cite{Aaronson:2004}. 
% For our definition of state preparation complexity, we drop the uniformity constraint from~\cite{RosenthalYuen:2022,MetgerYuen:2023} and define a class as $\mathsf{StateX}$, which refers to states preparable by circuits of type $\mathsf{X}$. 
% As an example, if $\mathsf{X} = \QNC^0$, this results in the class $\mathsf{StateQNC^0}$, which is the set of states preparable from the $\ket{0}^n$ state by poly-size constant-depth circuits. 
% This notion is similar to the relative complexity from~\cite{Aaronson:2004}, where one state is the  $\ket{0}^n$ state and instead of counting the number of gates we consider the set of states preparable by a fixed number of gates. Using this notion of complexity we show that any state preparable by an $\LAQCC^*$ circuit is also preparable by a $\mathsf{PostQPoly}$ circuit, the class of circuits of polynomial depth with an additional post-selection gate. 

\paragraph{Summary of results}
\begin{itemize}
    \item We give a new definition of a computational model that captures the power of the four step process: applying a constant number of layers of one- and two-qubit gates; performing a syndrome measurement; perform a fast classical computation determining corrections; apply corrections. We call this model \emph{Local Alternating Quantum Classical Computations}, or $\LAQCC$ for short. In this model we bound the allowed quantum operations, intermediate classical calculations, and number of rounds separately. In Section~\ref{sec:LAQCC_model} we define this model and give a list of operations based on results from literature contained in this computational model. In some of these operations we explicitly use that we allow for multiple, but at most constant, rounds  of corrections.
    \item  We show show that there exist $\LAQCC$ circuits that can not be weakly simulated in Section~\ref{sec:IQP_in_LAQCC}. We further show that for every $\LAQCC$ circuit there exists a $\QNC^1$ circuit simulating it perfectly, in Section~\ref{sec:LAQCC_in_QNC1}.
    \item We introduce a new type computational complexity for preparing states and show that the extension of $\LAQCC$ where we allow a polynomial number of rounds and unbounded classical computation, is contained in $\mathsf{PostQPoly}$, the class of polynomial circuits with post-selection, in Section~\ref{sec:Complexity results}.
    \item We show a protocol to prepare the uniform superposition state of size $q$ in $\LAQCC$ using $\mathO(\ceil{\log_2(q)}^2)$ qubits in Section~\ref{sec:superposition_modulo_q}. 
    \item We show a protocol to prepare the $W_n$ state in $\LAQCC$ using $\mathO(n\log(n))$ qubits in Section~\ref{sec:W_state_in_LAQCC}.
    \item We show two ways of preparing the Dicke-$(n,k)$ state. The first method is in $\LAQCC$, works up to $k = \mathO(\sqrt{n})$, uses $\mathO(n^2\log(n))$ qubits, and is found in Section~\ref{sec:dicke:small_k}. The second method is in $\LAQCC\text{-}\mathsf{LOG}$ (an extension of $\LAQCC$ allowing for logarithmic number of alterations instead of constant), works for any $k$, uses $\mathO(\text{poly}(n))$ qubits, and is found in Section~\ref{sec:Dicke_in_LAQCC_LOG}. 
    \item We extend on our $\LAQCC$ method of generating Dicke-$(n,k)$ states for $k = \mathO(\sqrt{n})$ and show a protocol to generate many-body scar states for a particular Hamiltonian in $\LAQCC$ (Section~\ref{sec:many_body_scar}). 
\end{itemize}
Summarized in a table, we provide the following state generation protocols:
\begin{table}[htb]
\centering
\begin{tabular}{l|l|l|l}
\textbf{State description} & \textbf{Width} & \textbf{Depth} & \textbf{Implementation}\\
\hline 
Uniform superposition mod $q$: $\frac{1}{\sqrt{q}} \sum_{i = 0}^{q-1}\ket{i}$ & $\mathO(\ceil{\log^2 q})$ & $\mathO(1)$ & Section~\ref{sec:superposition_modulo_q}\\

$W$-state: $\frac{1}{\sqrt{n}}\sum_{i = 0}^{n-1}\ket{e_i}$ & $\mathO(n \log n)$ & $\mathO(1)$ & Section~\ref{sec:W_state_in_LAQCC}\\

Dicke-$(n,k)$, $k = \mathO(\sqrt{n})$: $\binom{n}{k}^{-1/2}\sum_{x \in \{0,1\}^n: |x| = k} \ket{x}$ &  $\mathO(n^2\log n)$ & $\mathO(1)$ 
&Section~\ref{sec:dicke:small_k}\\

Dicke-$(n,k)$: $\binom{n}{k}^{-1/2}\sum_{x \in \{0,1\}^n: |x| = k} \ket{x}$ & $\mathO(\text{poly}(n))$ & $\mathO(\log n)$ &Section~\ref{sec:Dicke_in_LAQCC_LOG}\\

QMBS: $\ket{S_k} = \frac{1}{k! \sqrt{\mathcal N(n,k)}}(Q^\dagger)^k \ket{\Omega}$ &  $\mathO(n^2\log n)$ & $\mathO(1)$  &  Section~\ref{sec:many_body_scar}
\end{tabular}
\caption{Summary of state preparation protocols given in this paper.}
\label{tab:sate_prep}
\end{table}
In the entry for the quantum many-body scar state $Q$ denotes the raising operator and $\mathcal N(n,k)=\binom{n-k-1}{k}$. 
Section~\ref{sec:many_body_scar} will provide more details on the variables and the implementation. 

\paragraph{Organization of the paper}
\noindent We first introduce relevant preliminaries in Section~\ref{sec:preliminaries}. 
In Section~\ref{sec:LAQCC_model} we formally define the class of Local Alternating Quantum-Classical Computations ($\LAQCC$). We also show that any Clifford circuit can be implemented in constant depth $\LAQCC$ (a result based on a result from measurement-based quantum computing~\cite{jozsa2006introduction}). 
This result allows us to give many useful multi-qubit gates and routines in Section~\ref{sec:gates_created_in_LAQCC}. 
Beyond that we show that constant depth $\LAQCC$ circuits are contained in $\QNC^1$ and that any $\mathsf{IQP}$ circuit has an $\LAQCC$ implementation.
We conclude this section with an analysis of a more powerful instantiation of $\LAQCC$ and show an inclusion with respect to the class $\mathsf{PostQPoly}$, which is the class of circuits of polynomial depth with one additional post-selection gate. 
In Section~\ref{sec:state_prep_in_LAQCC} we give $\LAQCC$ circuit implementations for preparing the uniform superposition over an arbitrary number of states, the $W$-state and the Dicke state up to $k = \mathO(\sqrt{n})$. We furthermore give a log-depth circuit implementation for preparing the Dicke state for any $k$. We conclude by showing a $\LAQCC$ circuit for generating many body scar states of a particular type of Hamiltonian.



\section{Generalizing Demonstrated Grasps}

% Figure environment removed

In grasping and tool use tasks, human-like robotic hands manipulate objects by inducing contact with the inside of the finger phalanges and the palm.
The corresponding keypoints, also referred to as task-space vectors~\cite{Qin2022a}, which are shown in Fig.~\ref{fig:task_space_vectors}, define the features of a grasp to be preserved during generalization to novel instances.
We found that such detailed multi-fingered grasping configurations can be accurately mapped between instances in a \textit{two-step approach}.
Specifically, we uniquely combine non-rigid registration and hand pose retargeting to construct a system for generalization of multi-fingered grasping configurations.
First, in Sec.~\ref{subsec:latent_deformation_field_manifold}, we leverage latent non-rigid registration to continuously deform the canonical object (and its demonstration keypoints) to match the observed object. 
This preserves characteristic category-level features of a grasp and works directly from partial point cloud observations. 
Second, in Sec.~\ref{subsec:optimization_in_task-space}, we optimize the end-effector pose and joint positions of the robot hand to find a kinematically feasible grasp that minimizes the distance in task space.

\subsection{Category-level Grasp Pose Transfer}
\label{subsec:latent_deformation_field_manifold}

\subsubsection{Coherent Point Drift}
To explain the first step in our grasp pose generalization method, we briefly review the coherent point drift (CPD) algorithm~\cite{Myronenko2010}.
Given a target point set $\bm{X} = (\bm{x}_1, \dots, \bm{x}_N)$ and a source point set $\bm{Y} = (\bm{y}_1, \dots, \bm{y}_M)$, our goal is to find a transformation that maps $\bm{Y}$ to $\bm{X}$.
CPD builds a Gaussian mixture model (GMM) from the moving point set, $\bm{Y}$, and treats the points in $\bm{X}$ as observations drawn from it.
An expectation-maximization (EM) algorithm  is used to optimize the GMM while obeying a smoothness constraint based on motion coherence theory~\cite{Yuille1988}.
The non-rigid transformation $\mathcal{T}$ mapping $\bm{Y}$ to $\bm{X}$ is given by: 
\begin{equation}
        \mathcal{T}(\bm{Y}, v) = \bm{Y} + v(\bm{Y}),
\end{equation}
where the displacement function $v$ is defined for any set of points $\bm{Z}$ as:
\begin{equation}
        v(\bm{Z}) = G(\bm{Y}, \bm{Z}) \bm{W}.
\end{equation}
$G(\cdot, \cdot)$ denotes a Gaussian kernel matrix. CPD estimates the matrix of kernel weights, $\bm{W}$, which can be understood as a set of deformation vectors associated with the points in $\bm{Y}$. 
Thus, the transformation from a canonical model $\bm{C}$ to a training instance $\bm{T}_i$ is defined as:
\begin{equation}
        \mathcal{T}_i(\bm{C}, \bm{W}_i) = \bm{C} + G(\bm{C}, \bm{C}) \bm{W}_i.
\end{equation}

% Figure environment removed


% How to generalize keypoints of a grasp?
\subsubsection{Latent Deformation Field Manifold}
Our goal is to extrapolate from a single demonstration to novel objects of the same category, utilizing understanding of common intra-class variability.
Therefore, we use CPD to find the deformation $\mathcal{T}_i(\bm{C}, \bm{W}_i)$ from the canonical instance, $\bm{C}$, to all other training instances $\bm{T}_i$.
The uniqueness of each deformation is captured entirely by $\bm{W}_i \in \mathbb{R}^{M \times 3}$. 
The corresponding row vectors $\bm{x}_i \in \mathbb{R}^{3M}$, which are the feature descriptors of the deformation fields, are assembled into a design matrix $\bm{X} \in \mathbb{R}^{n \times 3M}$, where $n$ is the number of training instances.
Finally, we perform principal component analysis (PCA) on $\bm{X}$, to find a lower-dimensional manifold of characteristic deformations $\bm{L} \in \mathbb{R}^{q \times 3M}$, where $q \ll n$ is the number of eigenvectors to keep, i.e., the dimensionality of our ensuing latent space.
Characteristic deformations of a category can now be modeled in the $q$-dimensional latent space.

When encountering a new instance that has been partially observed through a segmented point cloud $\bm{O}$, we fit the latent parameter vector $\bm{\ell}$ to match its shape. Specifically, we aim to minimize the energy function:
\begin{equation}
        E(\bm{\ell}) = - \sum_{m=1}^{M} \log \sum_{n=1}^{N} e^{- \frac{1}{2\sigma^2} \lVert \bm{O}_n - \mathcal{T}(\bm{C}_m, W_m(\bm{\ell})) \rVert ^2},
				\label{eq:energy}
\end{equation}
illustrated in Fig.~\ref{fig:energy_landscape}, via gradient descent.  
Fig.~\ref{fig:register_observed} shows how this optimization deforms the canonical object to match the observed instance.
Since non-rigid registration in latent space permits only deformations actually observed in a class, we can register even partially observed objects without invoking undesired deformations of the canonical model.
Once a minimum is found, we can use the resulting deformation field to transform the keypoints of the canonical demonstration into generalized keypoint poses.



% Figure environment removed


\subsection{Pose Regression in Task Space}
\label{subsec:optimization_in_task-space}

% How to find attainable grasp for generalized keypoints?
So far, we have only shown how the deformation field between canonical and observed instances can be used to transform feature points of a grasp.
While previous work~\cite{Stuckler2014, Rodriguez2018} uses this property to generalize trajectory control poses, 
this forgoes the inherent advantage of multi-fingered manipulators to grasp in diverse finger configurations. 
Our aim is to maintain the properties of a grasp when it is transferred to new objects, which entails determining a relationship between the deformation of an object and the joint angles of the resultant grasp.
Thus, while the transferred keypoints represent the desired hand pose, they might not define a reachable position under the kinematic constraints of the used end-effector.  
To address this issue, we introduce a second optimization step, inspired by motion retargeting approaches~\cite{Qin2022,Qin2022a}, to find an attainable grasp configuration.
We optimize the wrist pose $\bm{p}$ and joint positions $\bm{q}$ of the robot hand to minimize the distance to the transferred keypoints $\bm{k}_i$:
\begin{equation}
        \min_{\bm{p}, \bm{q}} \sum_{i=0}^{N} \lVert \bm{k}_i - f_i(\bm{p}, \bm{q}) \rVert ^2,
        \label{eq:task_space_objective}
\end{equation}
where $f_i$ represents the forward kinematics of the $i$th keypoint. 
Solving this optimization problem yields the hand pose and joint angles that best preserve task-space characteristics of the demonstrated grasp.
Fig.~\ref{fig:2_step_generalization} shows the process of optimizing for minimal task-space distance (Eq.~\ref{eq:task_space_objective}).
The robot hand converges to a feasible grasping position that maintains characteristic features of the original demonstration.

\section{Interactive Tool Use}
Thus far, we have introduced our method for intra-class generalization of functional grasps. 
However, we only concerned ourselves with static grasp poses.
In the following, we describe how the obtained demonstrations can be used to guide the learning process of interactive RL policies.
Sec.~\ref{subsec:pre-grasp_guided_exploration} presents pre-grasp poses as efficient exploration primitives.
In Sec.~\ref{subsec:rl_with_privileged_information}, we propose a shaped reward function to direct the policy based on the generalized grasp poses.
Lastly, Sec.~\ref{subsec:transfer_to_unseen_tools} discusses how grasp poses and tool-use policies can generalize to instances beyond the training set.

% Figure environment removed


\subsection{Pre-grasp Poses for Efficient Exploration}
\label{subsec:pre-grasp_guided_exploration}
The process of grasping tools can be described by an initial \textit{reaching} and a subsequent \textit{dexterous manipulation} phase~\cite{Dasari2023}.
The first phase, where the robot reaches for the tool but does not yet make contact, can be solved very efficiently by conventional feedback control methods.
Only the subsequent high-contact manipulation requires an interactive RL policy.
In this context, our generalized grasp poses can be utilized to favorably position the robotic hand at the beginning of each episode by moving to a pose offset from our final target.
Specifically, we have the robot approach a pose that is removed from the target grasp in the direction normal to the palm, and interpolate the joint angles between the open and target configurations.
On the right of Fig.~1, we overlay both the pre-grasp configuration, which represents the beginning of the episode, and a later configuration just before the task is completed. 
Pre-grasp poses serve as a critical precursor to efficient exploration and successful learning of the task at hand.

\begin{table*}[t]
        \caption{Grasp-pose guided reward. The reward function combines {\color{dark_green}\bf task-specific rewards} encouraging goal-directed behavior and {\color{dark_orange}\bf tool grasping rewards} that encourage the agent to reach the demonstrated grasping pose. }
        \begin{minipage}[b]{1 \linewidth}%\centering
          \centering 
          {
          \small
          \begin{tabular}{llr}
          \toprule
          Term & Equation & Weight \\
          \midrule
          \tabnode{\phantom{0}}\hspace*{-0.6ex}$\circ$\,\textit{Place mug} & & \vspace*{-1ex} \tabnode{\phantom{$0.0$}}  \\
          \phantom{0}~~~$r_\textrm{pose}$: target pose matching & $e^{-\alpha \lVert x_t^{(p)} - \overline{x}_t^{(p)}\rVert - \beta \angle (x_t^{(o)}, \overline{x}_t^{(o)})}$ & $25.0$ \\
          \phantom{0}~~~$r_\textrm{success}$: target pose reached & $\mathds{1}(\textrm{pose\_reached})$ & \vspace*{0.5ex} $100.0$ \\
          \phantom{0}$\circ$\,\textit{Position drill} & &  \vspace*{-1ex} \\
          \phantom{0}~~~$r_\textrm{pose}$: target pose matching & $e^{-\alpha \lVert x_t^{(p)} - \overline{x}_t^{(p)}\rVert - \beta \angle (x_t^{(o)}, \overline{x}_t^{(o)})}$ & $25.0$ \\
          \phantom{0}~~~$r_\textrm{success}$: target pose reached & $\mathds{1}(\textrm{pose\_reached})$ & \vspace*{0.5ex} $100.0$ \\
          \phantom{0}$\circ$\,\textit{Drive nail} & &  \vspace*{-0.5ex} \\
          \phantom{0}~~~$r_\textrm{dist}$: move hammer to nail & ${(\epsilon + \Delta x)}^{-1}$ & $0.25$ \\
          \tabnode{\phantom{0}}~~$r_\textrm{depth}$: nail depth & $\Delta d_\textrm{nail}$ & \vspace*{0.5ex} \tabnode{$100.0$ } \\
          
          \tabnode{\phantom{0}}~~$r_\textrm{kp}$: keypoint matching & ${(\epsilon + \Delta k)}^{-1}$ & \tabnode{$0.001$ } \\
          \tabnode{\phantom{0}}~~$r_\textrm{lift}$: tool lifting& ${(\epsilon + \Delta h)}^{-1}$ & \tabnode{$0.05$ } \\
           \bottomrule
          \end{tabular}
          }
          
          \label{tbl:shaped_reward}
          \end{minipage}
          
          \begin{tikzpicture}[overlay]
          % Define the circle paths
          \draw [dark_green] (1.north west) -- (2.north east) --
          (4.south east) -- (3.south west) -- cycle;
          \draw [dark_orange] (5.north west) -- (6.north east) --
          (8.south east) -- (7.south west) -- cycle;
          
          % Labels
          \node [right=2cm,below=0.7cm,minimum width=0pt,anchor=west] at (2) (A) {{\color{dark_green}{\small \bf Task-specific}}};
          \draw [<-,out=0,in=180] (2) to (A);
          \node [right=2cm,above=0.6cm,minimum width=0pt,anchor=west] at (6) (B) {{\color{dark_orange}{\small \bf Tool grasping}}};
          \draw [<-,out=0,in=180] (6) to (B);
          \node [right=0.8cm,above=0.1cm,minimum width=0pt,anchor=west] at (8) (C) {{\color{black}{\footnotesize We use $\alpha=10$, $\beta=1$,}}};
          \node [right=0.8cm,below=0.2cm,minimum width=0pt,anchor=west] at (8) (C) {{\color{black}{\footnotesize  and $\epsilon=0.025$.}}};
          \end{tikzpicture}
      \vspace{-0.3cm}
			
\end{table*}


\subsection{Grasp-pose Guided RL}
\label{subsec:rl_with_privileged_information}
The operation of various tools can be mastered efficiently once robust grasps have been learned.
Hence, the generalized grasp poses are used to inject prior knowledge on how a tool ought to be grasped by parametrizing a shaped reward function.
The reward terms encouraging the agent to reach the demonstrated grasp-pose are detailed at the bottom of Tab.~\ref{tbl:shaped_reward}.
Specifically, an incentive is provided to minimize the distance to the demonstrated keypoints. 
A second reward term trains the agent to pick up the tool from the table, which requires learning a stable grasp and simple maneuvering of the tool.
We have found that defining the object height in terms of the object's root coordinate system can lead to undesirable local optima. 
For example, the coordinate root of the drill is at the tip of the tool, which causes the agent to learn unhelpful solutions, such as tilting the drill up without ever actually lifting it.
To avoid these problems, we decided to sample a synthetic point cloud on the tool's mesh. 
The height of the tool is then defined by the height of its lowest surface point, which is an approximation of the actual clearance between the tool and the table.
We use proximal policy optimization~\cite{Schulman2017} to train our policies to maximize this reward.

\subsection{Transfer to Unseen Tools}
\label{subsec:transfer_to_unseen_tools}
Ultimately, we want the learned policies to be able to operate unseen tools.
Thus, our goal is to generalize the grasp pose to a novel instance without access to its object mesh.
We therefore add a depth camera sensor to the environment, as shown in Fig.~\ref{fig:visual_environment_setup}. 
Unlike the synthetic point clouds, the sensor data suffers from occlusions.
Simply applying CPD would now deform the canonical instance in unhelpful ways. 
However, the learned category-level shape space can circumvent this problem. 
Since the low-dimensional deformation field manifold only allows for deformations that are characteristic of the variance in a class, the canonical model can even be fitted to a partially observed instance.



% Figure environment removed


% Figure environment removed

\section{Experimental Setup}
Our experiments aim to answer how effectively the proposed method can solve the challenging task of robotic tool use based on a single demonstration.
Specifically, we evaluate
(1) Whether a canonical demonstration can be generalized to new instances; 
(2) how effectively model-free RL can solve the posed tasks based on the generalized demonstrations; 
(3) whether our policies can generalize to novel, partially observed tools in a zero-shot manner.



\subsection{Problem Statement}
Our goal is to learn a policy $\pi$ to utilize a tool (${T_i | i=1, \dots, N}$) in order to achieve some goal-directed behavior, e.g. using a hammer to drive a nail. 
Moreover, the policy should be able to operate a variety of tool instances and generalize to unseen tools at test time. 
In a repeated interaction, the policy observes the current state of the environment $\bm{s}_t \in \mathcal{S}$, performs an action $\bm{a}_t \in \mathcal{A}$, and receives a reward signal $r_t$.
We define the observations of the policy to include proprioceptive observations of the robot state (wrist pose and keypoints of the hand), as well as a low-dimensional observation of the tool represented by its generalized demonstration and latent shape parameters. 
Additionally, the policy receives information about task-specific objectives, such as the desired pose of the drill.
The action space $\mathcal{A}$ comprises the desired change to the end-effector pose and joint positions of the robot hand. The agent chooses actions at a frequency of 30\,Hz.
The reward function is the sum of the terms detailed in Tab.~\ref{tbl:shaped_reward}.

\subsection{Tool Use Tasks}
% Which tool categories and how were the models obtained?
We evaluated our method on three tool categories: \textit{Drills}, \textit{Hammers}, and \textit{Mugs}, each with one canonical, 10 training, and 3 test instances. 
The models were obtained from the online databases GrabCAD\footnote{\url{https://grabcad.com/library}}, 3DWarehouse\footnote{\url{https://3dwarehouse.sketchup.com/}}, and Sketchfab\footnote{\url{https://sketchfab.com}}.
The simulated robot combines a UR5e arm controlled by its end effector pose with a Schunk SIH hand that has 11 degrees of freedom (DoFs), 5 of which are fully actuated.
We use NVIDIA Isaac Gym~\cite{Makoviychuk2021} to simulate the tool use tasks shown in Fig.~\ref{fig:task_overview}. 
In each run, 16,384 parallel agents are trained for a total of 134 million simulated steps, which corresponds to approximately 52 real-time days. 
This requires just under 3 hours of wall-clock time on a single NVIDIA A6000 GPU. At test time, we required an average of 3 seconds to match the canonical model to an observed instance.


\subsection{Demonstrations}
Our method draws on human grasping knowledge to accelerate the learning process. 
To demonstrate grasping postures in an intuitive way, we introduce a virtual reality (VR) interface to Isaac Gym. 
The operator's movements are tracked by a SenseGlove DK1, which captures finger angles, and an HTC Vive tracker, which records the hand pose.
This device, worn by the operator, can be seen on the left in Figure~1.
An HTC Vive headset is integrated with Isaac Gym's camera sensors to provide a stereoscopic visualization of the scene.
The operator interacts with the tasks in a natural way, indicating at the push of a button that the current pose should serve as the canonical demonstration.


\subsection{Evaluation Procedure}
For the \textit{Place mug} and \textit{Position drill} tasks, the success criterion is based on the distance of the tool pose and target pose. 
We consider an episode as completed successfully if $d < \bar{d}$ and $\theta < \bar{\theta}$, where $d$ and $\theta$ are the positional and angular distance to the target pose.
For both environments we use $\bar{d} = 0.03\mathrm{m}$ and $\bar{\theta} = 0.2\mathrm{rad}$.
The \textit{Drive nail} environment considers runs successful, where the nail has been driven by a depth of greater than $0.075\mathrm{m}$.

\begin{table}
        \caption{Task-space distance.}
        \centering \normalsize
        \begin{tabular}{rrrr} \toprule
                & Mugs & Drills & Hammers  \\ \midrule
                Ours  & $\mathbf{0.68 \pm 0.26}$ & $\mathbf{0.72 \pm 0.30}$ & $\mathbf{0.78 \pm 0.34}$ \\
                WP  & $2.27 \pm 1.12$ & $2.76 \pm 1.74$ & $2.64 \pm 1.12$ \\
                CG  & $2.00 \pm 0.97$ & $2.89 \pm 1.45$ & $3.88 \pm 2.22$ \\ \bottomrule \vspace*{-3mm}
        \end{tabular}

				\footnotesize Mean distance in cm of the grasps proposed by our method and\\ ablations to the keypoints of the generalized demonstration.
        
        \label{tbl:task_space_distance}
\end{table}


\section{Results}

\subsection{Analysis of Generated Grasps}
First, we investigate the kind of grasp poses that the proposed framework generates. 
Here we compare with two ablations: Retention of the canonical grasp (CG) and transformation of the wrist pose while keeping grasping behavior constant (WP).
To assess the quality of a grasp, we measure the distance to the transformed keypoints over all training instances in a class. 
Quantitative results are shown in Tab.~\ref{tbl:task_space_distance}.
The proposed method outperforms both baselines by a large margin, and the results are consistent across tool categories.
Examples of the generalized grasp-poses shown in Fig.~\ref{fig:generated_grasp_poses} confirm that our approach finds feasible grasps for varying object shapes.



% Figure environment removed



\subsection{Grasp-pose Guided RL}
Next, we evaluate the ability of generalized grasp pose demonstrations to guide policy search on challenging tool use tasks. 
The results in Tab.~\ref{tbl:training_performance} show that the proposed method consistently finds the intended grasps across the tasks and tools studied. 
Furthermore, the Place Mug and Position Drill tasks are solved with high reliability. 
Driving a nail proved to be the most challenging task to complete, as the agent must maintain its grasp on the hammer while making forceful contact with the environment.
Now, we compare the performance of our proposed method to multiple ablations. 
First, we examine how performance changes when we disable our grasp generalization (w/o GG) and instead apply canonical demonstration to all objects.
As can be seen, this still leads to viable training performance for objects with lower variance, such as cups, while performance deteriorates more severely for objects that vary greatly in their extent and grasping position, such as drills and hammers.
Not navigating to a pre-grasp pose at the beginning of the episode (w/o PG) causes the training to fail.
The agent is not able to find the correct grasp posture, but frequently gets stuck in local optima.
Lastly, we compare to a baseline where the task is approached without demonstration guidance (w/o demo). 
Here, the agent receives only task-specific rewards and is initialized in a default neutral position above the table.
Again, learning the full manipulation tasks is unsuccessful, as discovering useful behaviors that make progress on the proposed task is extremely difficult in this situation.

The results show that knowledge about how to grasp an object, which can be incorporated via shaped rewards or pre-grasp poses, is a valuable addition to RL training.
Moreover, having a method that generalizes such demonstrations to new objects in a class removes the high overhead of collecting a large number of demonstrations and allows training to scale more easily.

\begin{table}
    \caption{Training performance. Success rates of the proposed method and studied ablations to grasp the tool and solve the full task.}
    \centering
    \begin{tabular}{rrrcrrcrr} \toprule
           & \multicolumn{2}{c}{Place mug} & \phantom{} & \multicolumn{2}{c}{Position drill} & \phantom{} & \multicolumn{2}{c}{Drive nail}  \\
           \cmidrule{2-3} \cmidrule{5-6} \cmidrule{8-9} 
         & grasp & full & \phantom{} & grasp & full & \phantom{} & grasp & full \\ \midrule

        Ours  & $\mathbf{0.97}$ & $\mathbf{0.96}$ & \phantom{} & $\mathbf{0.94}$ & $\mathbf{0.76}$ & \phantom{} & $\mathbf{0.8}$ & $\mathbf{0.65}$ \\
        w/o GG  & $\mathbf{0.97}$ & 0.95 & \phantom{} & 0.81 & 0.66 & \phantom{} & 0.74 & 0.61 \\
        w/o PG  & 0.01 & 0.0 & \phantom{} & 0.41 & 0.0 & \phantom{} & 0.4 & 0.0 \\
        w/o demo  & 0.0 & 0.0 & \phantom{} & 0.0 & 0.0 & \phantom{} & 0.0 & 0.0 \\ \bottomrule
    \end{tabular}
    \label{tbl:training_performance}
\end{table}


\subsection{Zero-shot Transfer to Unseen Tools}
Finally, we investigate whether the policy is able to transfer to unseen tools in a zero-shot manner.
Here, we do not assume access to the object mesh, instead perceiving the scene via a segmented point-cloud, as shown in Fig.~\ref{fig:visual_environment_setup}.
The canonical demonstration is then adjusted to fit the observed instance and given to the policy. 
It can be seen in Tab.~\ref{tbl:test_performance}, that the policies can grasp and operate even some of the unseen tools without finetuning. 
Extending the training set may help to close the performance gap between the training and test instances in the future.

\section{Related Work}
% RL for robotic grasping and manipulation.
\subsubsection{Robotic grasping}
Despite decades of active research efforts, robotic grasping remains an unsolved problem~\cite{Zeng2022}.
Grasping has traditionally been framed as the open-loop procedure of grasp-pose prediction (grasp synthesis). 
Several prior works estimate grasp-poses through analytical~\cite{Sahbani2012,Ponce1993,Ding2000} or learned~\cite{Bohg2013,Kleeberger2020} methods.
In recent years, RL has become popular for robotic grasping and manipulation due to its ability to generate interactive policies in a model-free manner. 
Kalashnikov et al.~\cite{Kalashnikov2018} train a vision-based grasping policy to control a parallel gripper. 
Shahid et al.~\cite{Shahid2020} demonstrate that RL can be used to continuously control a Franka Emika Panda manipulator to lift objects off a table.
However, RL has struggled with the high-dimensional action space of anthropomorphic end-effectors. 
One group of work has aimed to scale up experience collection via parallelized GPU-accelerated physics simulation~\cite{Makoviychuk2021,Chen2022,Mosbach2022a}.
Alternatively, human demonstrations have been used by themselves~\cite{Qin2022,Qin2022a} or in combination with RL~\cite{Mosbach2022,Rajeswaran2018} to solve grasping and manipulation tasks.


% RL for tool use.
\subsubsection{Robotic tool use}
Prior works studying robotic tool use span classical~\cite{Brown2013,Toussaint2018} and learning-based~\cite{Fang2020,Wenke2019, Xie2019} approaches. 
Xie et al.~\cite{Xie2019} learn to predict the visual outcome of actions based on human demonstration and autonomous interaction data.
Planning with the learned model can solve improvised tool use tasks with a parallel gripper.
Wenke et al.~\cite{Wenke2019} study reasoning and generalization in RL through the lens of tool use. 
They train RL agents to solve grid-world versions of the classical trap-tube experiment.
Notably, Dasari et al.~\cite{Dasari2023} demonstrate that pre-grasp poses can be used to improve dexterous manipulation learning. 
While their objective of using grasp-poses to accelerate RL is aligned with the goal of our work, they do not consider generalization of grasp-poses or policies between different tools.
To the best of our knowledge, the amalgamation of transferring demonstrations between instances and interactive RL training is novel to our work.

\begin{table}
        \caption{Test performance. Success rate of the proposed method when operating unseen tools. The generalized demonstration is estimated from a partial point-cloud of the tool.}
        \centering
        \begin{tabular}{rrcrcr} \toprule
               & \multicolumn{1}{c}{Place mug} & \phantom{} & \multicolumn{1}{c}{Position drill} & \phantom{} & \multicolumn{1}{c}{Drive nail}  \\ \midrule
              & $0.67 \pm 0.11$ & \phantom{} & $0.62 \pm 0.15$ & \phantom{} & $0.55 \pm 0.1$ \\ \bottomrule
        \end{tabular}
        \label{tbl:test_performance}
\end{table}

% Grasp-pose transfer.
\subsubsection{Grasp-pose transfer.}
Multiple lines of work aim to generalize demonstrated behaviors to novel instances in a class. 
Object-meshes segmented via shape and volumetric information are used by Vahrenkamp et al.~\cite{Vahrenkamp2016} to transfer grasps from a template set to familiar objects.
Stückler et al.~\cite{Stuckler2014} transfer poses and trajectories defining grasping motion via the dense deformation field from the known object model to an observed instance.
Rodriguez et al.~\cite{Rodriguez2018} extend this work by modelling deformations not only between a known and observed instance, but within a category. 
This makes it possible to register partially observed objects.
In~\cite{Stouraitis2015} and~\cite{Amor2012}, contact points are warped from a known object to an observed object. 
However, both assume that the objects are fully observed.
Simeonovdu et al.~\cite{Simeonovdu2021} present neural descriptor fields which represent an object by a mapping from each 3D point $\bm{x}$ to a latent descriptor $z$ encoding relations to salient object features.
This description is used to establish correspondences of semantically meaningful object features, and thereby generalize demonstrations to new instances.
Our work builds on~\cite{Rodriguez2018}, but generalizes characteristic features of a multi-fingered grasp through optimization in task space. 
Further, we demonstrate how the generalized demonstrations can be used as a basis for learning interactive tool use policies, rather than as parameters for open loop grasping behavior.

\section{Discussion and Conclusion}
We have shown that the challenging domain of robotic tool use becomes approachable for model-free RL with the use of only a single human demonstration.
The proposed generalization scheme can transfer grasp poses even to partially observed instances while retaining characteristic features of the demonstrated functional grasp.
The RL experiments underscore the benefits of extending grasp pose generalization to the domain of interactive control, as the policies are for example able to continuously manipulate drills lying on the table until a desired grasp is achieved.
Although we only present results in simulation, we have shown how the latent shape parameters and grasping configuration of a novel object can be estimated from its partial point-cloud observation.
Still, there are several limitations and opportunities for future work. 
Transferring the obtained results to the real robot system is the most evident task. 
Developing a way to track tools during the grasping process and obtain well separated point-clouds of a scene are key challenges to be overcome.
In addition, developing an approach that can generate class-independent grasping or pre-grasp poses would be valuable.

\section*{Acknowledgement}
\small{This work has been funded by the German Ministry of Education and Research (BMBF), grant no. 01IS21080, project “Learn2Grasp: Learning Human-like Interactive Grasping based on Visual and Haptic Feedback”.}

% Bibliography
\bibliographystyle{IEEEtran}
%\balance
\bibliography{references.bib}

\end{document}
