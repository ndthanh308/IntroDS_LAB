\documentclass[11pt, oneside]{article}


\usepackage{enumerate}
\usepackage{amscd}
\usepackage{amsmath}
\usepackage{epsfig}
\usepackage{amssymb}
\usepackage{amsthm}    %now using this package instead of package "theorem"
\usepackage{latexsym}
\usepackage{color} %new, added to handle input of .pspdftex file
\usepackage{graphicx}
\usepackage{url}
%\usepackage{hyperref} %new, to handle URL's in bibtex

\setlength{\oddsidemargin}{0.25in}      % 1.25in left margin
\setlength{\evensidemargin}{0.25in}     % 1.25in left margin (even pages)
\setlength{\topmargin}{0.0in}           % 1in top margin
\setlength{\textwidth}{6.0in}           % 6.0in text - 1.25in rt margin
\setlength{\textheight}{8.5in}          % Body ht for 1in margins

\pagestyle{plain}
%%%%



\numberwithin{equation}{section}

\theoremstyle{plain}
\newtheorem{theorem}{Theorem}[section]
\newtheorem{proposition}[theorem]{Proposition} %[section]
\newtheorem{lemma}[theorem]{Lemma} %[section]
\newtheorem{corollary}[theorem]{Corollary} %[section]
\newtheorem{conjecture}[theorem]{Conjecture} %[section]
\theoremstyle{definition}
\newtheorem{definition}[theorem]{Definition} %[section]
\newtheorem{question}[theorem]{Question} %[section]
\theoremstyle{remark}
\newtheorem{remark}[theorem]{Remark} %[section]
\newtheorem{example}[theorem]{Example} %[section]


%%%%%%%%%%%%%%%%
%%%%%%%%%%%%%%%%
\newcommand{\email}[1]{{\scriptsize{\it E-mail address}\/: {\rm #1}} }
\providecommand{\norm}[1]{\lVert#1\rVert}
\DeclareMathOperator{\diag}{diag}
\DeclareMathOperator{\rank}{rank}

%\providecommand{\keywords}[1]{\textbf{\textit{Index terms---}} #1} %added new
\providecommand{\keywords}[1]{\textbf{Keywords }#1} %added new

\makeatletter
\def\blfootnote{\gdef\@thefnmark{}\@footnotetext}
\makeatother

\begin{document}
\blfootnote{\textup{2020} \textit{Mathematics Subject Classification}.
52C25, 52B11, 51M25}

\blfootnote{\textit{key words and phrases}.
rigid motion, flexible simplex, pseudo-Euclidean space.}

\blfootnote{\email{lizhaozhang@alum.mit.edu}}

\begin{titlepage}
\title{Flexible simplices with fixed volumes of codimension 2 faces}

\author{Lizhao Zhang
%\footnote{\email{lizhaozhang@alum.mit.edu}}
}
\date{}
\end{titlepage}



\maketitle


\begin{abstract}
For any $n$-dimensional simplex in the Euclidean space $\mathbb{R}^n$ with $n\ge 4$,
it is asked that if the $(n-2)$-dimensional volumes of all the codimension 2 faces 
always determine the simplex up to a rigid motion.
While the question remains open and the general belief is that the answer is affirmative,
for $n\ge 5$ we provide counterexamples to a variant of the question 
where $\mathbb{R}^n$ is replaced by a pseudo-Euclidean space $\mathbb{R}^{p,n-p}$ 
for some unspecified $p\ge 2$.
%\keywords{rigid motion \and flexible simplex \and pseudo-Euclidean space}
%\subclass{52C25 \and 52B11 \and 51M25}
\end{abstract}


\section{Introduction}

For any $n$-dimensional simplex in $\mathbb{R}^n$,
there are $\binom{n+1}{2}$ edges,
and up to congruence the $n$-simplex is uniquely determined by its edge lengths.
As there are also the same number of $(n-2)$-faces,
it is natural to ask the following question.

\begin{question}
\label{question_simplex_rigid}
For any $n$-simplex $Q$ in $\mathbb{R}^n$ with $n\ge 4$, if a continuous motion preserves 
the $(n-2)$-volumes of all the $(n-2)$-faces of $Q$, then is it necessarily a \emph{rigid} motion?
\end{question}

The discrete version of the question, without any continuous motion involved,
asks that if the $(n-2)$-volumes of all the $(n-2)$-faces always determine the $n$-simplex up to \emph{congruence}.
It asks for $n\ge 4$ because the answer for $n=2$ is trivially negative and for $n=3$ is trivially affirmative. 
This is a classical problem and many people may have asked it in the past,
but it is hard to say who asked it first.
For some account of the background of the question, see \cite{MoharRivin}.
The discrete version was answered negatively for all $n\ge 4$
and various counterexamples were constructed, see McMullen~\cite{McMullen:simplices}
and Mohar and Rivin~\cite{MoharRivin}.
In fact, the counterexample in \cite{MoharRivin} was not only simple,
but can also be easily extended to construct counterexamples to show that
for any $r\ge 2$ the $r$-volumes of all $r$-faces of an $n$-simplex do not necessarily 
determine the $n$-simplex up to congruence;
and it can be extended to the spherical and hyperbolic spaces as well, 
see Zhang~\cite[Section~2.12]{Zhang:rigidity}.

However, Question~\ref{question_simplex_rigid} remains open,
and the general belief is that the answer may be affirmative.
While we do not solve Question~\ref{question_simplex_rigid} in this paper, 
for $n\ge 5$ we provide counterexamples to a close variant of this question 
where $\mathbb{R}^n$ is replaced by a pseudo-Euclidean space $\mathbb{R}^{p,n-p}$ for some unspecified $p\ge 2$.
This might also suggest that there may exist counterexamples to Question~\ref{question_simplex_rigid} for some $n$.

\begin{theorem}
\emph{(Main Theorem 1)}
\label{theorem_flexible_pseudo_Euclidean}
For any $n\ge 5$, in $\mathbb{R}^{p,n-p}$ for some unspecified $p\ge 2$,
their exists a continuous family of non-congruent $n$-simplices $Q$,
such that  all the dihedral angles are Euclidean angles,
and the $(n-2)$-volumes of all the $(n-2)$-faces of $Q$ remain constant during the motion.
\end{theorem}

Here a \emph{Euclidean} dihedral angle means that for an $(n-2)$-face of the $n$-simplex, 
its orthogonal plane in $\mathbb{R}^{p,n-p}$ is a 2 dimensional Euclidean plane,
and thus requires $p\ge 2$.

Rigidity and flexibility of geometric frameworks have been extensively investigated in the past.
A flexible polyhedron in $\mathbb{R}^3$ is a closed polyhedral surface that admits continuous
non-rigid deformation such that all faces remain rigid.
The first embedded (non-self-intersecting) flexible polyhedron in $\mathbb{R}^3$ was discovered by 
Connelly~\cite{Connelly:counterexample}.
It was also shown that the volume of any flexible polyhedron in $\mathbb{R}^3$
remains constant during the continuous deformation, proving the bellows conjecture
(see Sabitov~\cite{Sabitov:invariance} and Connelly \emph{et al.}~\cite{ConnellySabitovWalz}).
The bellows conjecture  in the spherical case $\mathbb{S}^3$ was disproved 
by Alexandrov~\cite{Alexandrov:flexible}.

If we allow the faces of a simplex to be non-rigid but applying volume constraints on the faces instead,
then the simplex, with a slight abuse of terminology in this paper, 
can be loosely treated as a variant of \emph{flexible simplex}.
In this sense, Question~\ref{question_simplex_rigid} can now be interpreted as looking for 
flexible $n$-simplices in $\mathbb{R}^n$ with fixed volumes of codimension 2 faces,
and Theorem~\ref{theorem_flexible_pseudo_Euclidean} says that for $n\ge 5$
those flexible $n$-simplices exist in $\mathbb{R}^{p,n-p}$  for some $p\ge 2$.

For any potential flexible $n$-simplex in $\mathbb{R}^n$ or $\mathbb{R}^{p,n-p}$
with fixed volumes of codimension 2 faces,
analogous to the bellows conjecture, one may ask that if the volume of the flexible $n$-simplex 
also necessarily remains constant during the motion. Again, we do not solve this problem in this paper, 
but for $n$ even we observe that, for the \emph{particular} counterexamples we constructed in
the proof of Theorem~\ref{theorem_flexible_pseudo_Euclidean}, 
the volume of the flexible $n$-simplex \emph{does} remain constant, but it is not so for $n$ odd
(Section~\ref{section_final_remarks}).

If \emph{degenerate} simplices are also considered, it was shown that if a flexible $n$-simplex 
(in the Euclidean, spherical or hyperbolic spaces, 
but without any constraints on the codimension 2 faces as above)
starts as a degenerate simplex,
then its degeneracy is almost always preserved by a single constraint on its facet volumes
 (Zhang~\cite{Zhang:lifting}).




\section{Equivalent questions}

We first reformulate Question~\ref{question_simplex_rigid} (about an $n$-simplex $Q$ in $\mathbb{R}^n$)
to an equivalent form about the \emph{dual} of $Q$.
For an $n$-simplex $Q$ in $\mathbb{R}^n$, without loss of generality,
we assume that the origin $O$ is the centroid of the vertices of $Q$ 
(or simply call it the centroid of $Q$).
This assumption is not essential for $Q$ because the volume of any face of $Q$ is invariant under translation, 
but becomes more important when the dual of $Q$ is introduced next.
For $1\le i\le n+1$, denote $F_i$ an $(n-1)$-face of $Q$,
and $F_{ij}$ ($i\ne j$) the $(n-2)$-face on the intersection of $F_i$ and $F_j$.

Let an $n$-simplex $P$ be a \emph{dual} of $Q$ 
\begin{equation}
\label{equation_dual_Euclidean}
P=\{y \in \mathbb{R}^n: x\cdot y \le c \quad\text{for all $x\in Q$}\}
\end{equation}
for some $c>0$. When $c=1$, $P$ is the \emph{polar dual} $Q^{\ast}$ of $Q$.
Here $c$ may vary when $Q$ varies, but its value is not important and can be adjusted later.
Let the vertices of $P$ be $P_i$ and $v_i=\overrightarrow{OP_i}$,
then $v_i$ is an outward normal vector to $Q$ at $F_i$.
As $O$ is the centroid of $Q$, it can be shown that the length $\norm{v_i}$ 
is proportional to the $(n-1)$-volume of $F_i$.
The Minkowski relation for the facet volumes of $Q$ states that $\sum v_i=0$, so $O$ is also the centroid of $P$.

As $O$ is the centroid of $P$,
it is well known that the area of the triangle $OP_iP_j$ is proportional to the $(n-2)$-volume of $F_{ij}$,
e.g., see Lee~\cite[Theorem 19]{Lee:stress}.
So Question~\ref{question_simplex_rigid} is equivalent to the following question about $P$.

\begin{question}
\label{question_simplex_triangle_rigid}
For any $n$-simplex $P$ in $\mathbb{R}^n$ ($n\ge 4$) with the centroid fixed at the origin $O$, 
if a continuous motion preserves the areas of all triangles $OP_iP_j$,
then is it necessarily a rigid motion?
\end{question}

Note that the square of the area of triangle $OP_iP_j$, multiplied by a factor 4, is the determinant of the $2\times 2$ matrix
\begin{equation}
\begin{pmatrix} v_i^2 & v_i\cdot v_j \\ v_i\cdot v_j & v_j^2 \end{pmatrix},
\end{equation}
where $v_i\cdot v_j$ is the inner product of $v_i$ and $v_j$. 
Conversely, this matrix, without knowing more details of $v_i$ and $v_j$, 
determines the triangle $OP_iP_j$ up to congruence.

If we treat $v_i$ as a column $n$-vector and let $V=(v_1,\dots,v_{n+1})$ be an $n\times (n+1)$ matrix,
then $U:=V^TV=(v_i^T v_j)_{1\le i,j\le n+1}$ is an $(n+1)\times (n+1)$ positive semi-definite matrix.
Notationwise $v_i^T v_j$ and $v_i\cdot v_j$ are the same thing in the \emph{Euclidean} space,
so $U=(v_i\cdot v_j)_{1\le i,j\le n+1}$, and all the principal minors of order 2 of $U$
are the square of the areas of triangles $OP_iP_j$ (multiplied by 4).

Denote $\mathbf{1}$ the  $(n+1)$-vector $(1,\dots,1)^T$.
Because the $v_i$'s span the entire $\mathbb{R}^n$ and $\sum v_i=0$,
so $V$ has rank $n$ and $V\cdot\mathbf{1}=0$,
and thus $U=V^TV$ has rank $n$ and contains $\mathbf{1}$ in the null space. 
As $U$ determines $P$ up to congruence,
so again, Question~\ref{question_simplex_triangle_rigid} is equivalent to the following question.

\begin{definition}
\label{definition_matrix_positive_definite}
Let $\mathcal{F}$ (resp. $\mathcal{F}_0$) be the set of $(n+1)\times (n+1)$ positive semi-definite matrices 
(resp. with rank $n$) with the vector $\mathbf{1}$ in the null space.
\end{definition}

\begin{question}
\label{question_simplex_matrix_rigid}
Let $U\in\mathcal{F}_0$ be an $(n+1)\times (n+1)$ positive semi-definite matrix with rank $n$
and with the vector $\mathbf{1}$ in the null space.
If a continuous change of $U$ in $\mathcal{F}_0$ preserves all the principal minors of order 2 of $U$,
then does it necessarily preserve $U$ as well?
\end{question}

As mentioned earlier, we do not solve Question~\ref{question_simplex_rigid}, 
which is equivalent to Question~\ref{question_simplex_triangle_rigid}
and Question~\ref{question_simplex_matrix_rigid}.
But for a variant of Question~\ref{question_simplex_matrix_rigid}
where the restriction of positive semi-definite matrices is replaced by a weaker notion of symmetric matrices, 
we construct counterexamples for $n\ge 5$.

\begin{definition}
\label{definition_matrix_symmetric}
Let $\mathcal{U}$ (resp. $\mathcal{U}_0)$ be the set of $(n+1)\times (n+1)$ symmetric matrices 
(resp. with rank $n$) with the vector $\mathbf{1}$ in the null space
and whose diagonal entries and principal minors of order 2 are all non-negative (resp. positive).
\end{definition}

\begin{theorem}
\label{theorem_symmetric_matrix}
\emph{(Main Theorem 2)}
For $n\ge 5$, there exists a continuous family of non-identical 
$(n+1)\times (n+1)$ symmetric matrices $U\in\mathcal{U}_0$ with rank $n$ 
and with the vector $\mathbf{1}$ in the null space,
such that all the principal minors of order 2 are positive and remain constant.
\end{theorem}

It is easy to check that for any $U\in\mathcal{U}$,
if its principal minors of order 2 are all positive, 
then it automatically implies that the diagonal entries are also all positive.
Though Theorem~\ref{theorem_symmetric_matrix} does not fully resolve 
Question~\ref{question_simplex_matrix_rigid},
it is a somewhat surprising result, because for a $(n+1)\times (n+1)$ 
symmetric matrix in $\mathcal{U}_0$, the degrees of freedom are $\binom{n+1}{2}$, 
the same as the number of the principal minors of order 2.
Our proof of Theorem~\ref{theorem_symmetric_matrix} is constructive, 
and may possibly be used to search for counterexamples to
Question~\ref{question_simplex_matrix_rigid} as well.



\section{Proof of Theorem~\ref{theorem_symmetric_matrix}}

We first have the following simple observation.

\begin{lemma}
\label{lemma_matrix_2_by_2}
For any $2\times 2$ matrices $A=\begin{pmatrix} a_1^2 & a_1a_2 \\ a_1a_2 & a_2^2 \end{pmatrix}$
and $B=\begin{pmatrix} b_1^2 & b_1b_2 \\ b_1b_2 & b_2^2 \end{pmatrix}$, 
the determinant of $tA+\frac{1}{t}B$
is independent of the value of $t$, and is positive unless $a_1b_2=a_2b_1$.
\end{lemma}

\begin{proof}
Because $\det(A)=0$ and $\det(B)=0$, so the determinant of $tA+\frac{1}{t}B$ 
does not have the $t^2$ and $\frac{1}{t^2}$ terms.
This leaves only the constant term (as a product of $t$ and $\frac{1}{t}$)
\[a_1^2b_2^2+a_2^2b_1^2-2a_1a_2b_1b_2=(a_1b_2-a_2b_1)^2,
\]
which is non-negative, and is zero only when $a_1b_2=a_2b_1$.
\end{proof}

\begin{definition}
Let $\mathcal{D}$ be the set of the $(n+1)\times (n+1)$ symmetric matrices in $\mathcal{U}$
with the vector $\mathbf{1}$ in the null space and whose principal minors of order 2 are all 0.
\end{definition}

\begin{example}
\label{example_symmetric_matrix}
Let $\alpha=(a_1,\dots,a_{n+1})^T$ be a column $(n+1)$-vetor that satisfies $\alpha^T\cdot\mathbf{1}=0$,
and $A:=\alpha\alpha^T$ be an $(n+1)\times (n+1)$ symmetric matrix.
Then $A\cdot\mathbf{1}=0$ and $\rank(A)=1$, and $A\in\mathcal{D}$.
\end{example}

Lemma~\ref{lemma_matrix_2_by_2} immediately leads to the following important result.

\begin{lemma}
\label{lemma_matrix_non_negative}
If $A, B\in\mathcal{D}$, then for $tA+\frac{1}{t}B$ with $t>0$,
it is in $\mathcal{U}$ and all the principal minors of order 2 are non-negative constants.
\end{lemma}

Note that unlike $\mathcal{F}$ or $\mathcal{U}$, 
the set $\mathcal{D}$ is not closed under the plus operation of matrices.
To further prove Theorem~\ref{theorem_symmetric_matrix}, 
we only need to improve Lemma~\ref{lemma_matrix_non_negative}
by finding suitable $A, B\in\mathcal{D}$ and $t>0$
such that $tA+\frac{1}{t}B$ is in $\mathcal{U}_0$ with rank $n$,
then $tA+\frac{1}{t}B$ satisfies the conditions of Theorem~\ref{theorem_symmetric_matrix}.
This is what we plan to do next.

\begin{remark}
\label{remark_matrix_rank}
As $\mathcal{D}$ is closed under multiplication by a positive factor,
so to prove Theorem~\ref{theorem_symmetric_matrix}, 
in fact we can just ignore $t$ and only search for $A, B\in\mathcal{D}$ 
such that $A+B$ is in $\mathcal{U}_0$ with rank $n$.
To do so, the common null space of $A$ and $B$ must be only 1-dimensional (that contains $\mathbf{1}$),
so $\rank(A)+\rank(B)$ should be at least $n$.
Then at least one of the ranks of $A$ and $B$, say $\rank(A)$, should be at least $\lceil n/2 \rceil$,
the least integer greater than or equal to $n/2$.
In Example~\ref{example_symmetric_matrix}, we have $\rank(A)=1$, which is too low for our purpose.
For $n=4$, it can be verified that any non-zero $5\times 5$ matrix in $\mathcal{D}$ has rank 1, 
less than $\lceil n/2 \rceil=2$, so this method does not work for $n=4$,
and we leave the details to the interested reader.
\end{remark}


\subsection{An outline of the proof}

If $A\in\mathcal{D}$, then $A$ can be written as $D^TCD$ (not necessarily unique),
where $D=\diag(d_1,\dots,d_{n+1})$ and $C$ is an $(n+1)\times (n+1)$ symmetric matrix 
with only $\pm 1$ entries and all the diagonal entries are 1.
As $A$ also satisfies $A\cdot\mathbf{1}=0$, so $D^TCD\cdot\mathbf{1}=0$. 

\begin{remark}
\label{remark_zero_entry}
If all the $d_i$'s in $D$ above have at least two zero entries, assume $d_1=d_2=0$.
Then no matter how to choose $B\in\mathcal{D}$, by Lemma~\ref{lemma_matrix_2_by_2}, 
in $A+B$ the determinant of the upper-left $2\times 2$ submatrix (a principal minor of order 2) is 0,
so $A+B$ is \emph{not} in $\mathcal{U}_0$.
So in order to find suitable $A,B\in\mathcal{D}$ to prove Theorem~\ref{theorem_symmetric_matrix},
the $d_i$'s can afford to have at most one zero entry.
\end{remark}

\begin{definition}
Let $\mathcal{H}$ be the finite set of all $(n+1)\times (n+1)$ symmetric matrices 
with only $\pm 1$ entries and all the diagonal entries are 1.
\end{definition}

We remark that unlike $\mathcal{F}$, $\mathcal{U}$ or $\mathcal{D}$, a matrix in $\mathcal{H}$
generally does not contain $\mathbf{1}$ in the null space.
Before we provide some concrete examples of $A,B\in\mathcal{D}$ 
to prove Theorem~\ref{theorem_symmetric_matrix}, 
here we outline our general approach to constructing such a matrix $A\in\mathcal{D}$
(it can be verified that \emph{all} $A\in\mathcal{D}$ can be constructed in the following way,
but we won't need to use this property):
First find a matrix $C\in\mathcal{H}$ such that $C$ does not have a full rank of $n+1$.
In the null space of $C$, pick a vector $d=(d_1,\dots,d_{n+1})^T$
and write it as $D\cdot\mathbf{1}$ where $D=\diag(d)$.
Then $CD\cdot\mathbf{1}=0$ and thus $D^TCD\cdot\mathbf{1}=0$.  
Let $A=D^TCD$, then $A\cdot\mathbf{1}=0$ and thus $A\in\mathcal{D}$. 

Two things to keep in mind, 
(1) by Remark~\ref{remark_matrix_rank} we want $\rank(A)$ to be at least $\lceil n/2\rceil$, 
and (2) by Remark~\ref{remark_zero_entry} we want the $d_i$'s to contain at most one zero entry.
Now we follow with some concrete examples to construct $A\in\mathcal{D}$.
The treatment is different for $n$ odd and $n$ even.



\subsection{For $n$ odd}

For an odd $n\ge 5$, let $n=2m-1$. So $n+1=2m$ and $m\ge 3$. 
Let $C_0$ be a $m\times m$ matrix whose diagonal entries are all 1 
and all off-diagonal entries are $-1$. 
As $m\ge 3$, $C_0$ has a full rank of $m$ (but not for $m=2$ where $\rank(C_0)=1$).
Now for a $2m\times 2m$ matrix $C_1$, we cut it into some $2\times 2$ blocks (there are $m\times m$ of them),
fill the $m$ diagonal $2\times 2$ blocks with $\begin{pmatrix} 1 & 1 \\ 1 & 1 \end{pmatrix}$,
and the rest blocks with $\begin{pmatrix} -1 & -1 \\ -1 & -1 \end{pmatrix}$.
As $m\ge 3$, using the fact that $\rank(C_0)=m$, it is easy to check that $\rank(C_1)=m$,
and the $2m$-vector $d$ in the null space of $C_1$ has the form 
\begin{equation}
\label{equation_null_space_C1}
d=(d_1,d_2,\dots,d_{2m-1},d_{2m})^T=(-d_2,d_2,\dots,-d_{2m},d_{2m})^T.
\end{equation}
Let $D_1=\diag(d)$ and $A_{2m}=D_1^TC_1D_1$,
then as $A_{2m}\cdot\mathbf{1}=D_1^T(C_1d)=0$,
we have $A_{2m}\in\mathcal{D}$.
If $d_2,\dots,d_{2m}$ are non-zero, whose values will be chosen later, then $\rank(A_{2m})=\rank(C_1)=m$,
and the $2m$-vector in the null space of $A_{2m}$ has the form 
\begin{equation}
\label{equation_null_space_A}
(a_2,a_2,\dots,a_{2m},a_{2m})^T.
\end{equation}

For a $2m\times 2m$ matrix $C_2$, let it be obtained 
by moving the last row and last column of $C_1$ to the first row and first column.
So we still have $\rank(C_2)=m$, and the $2m$-vector $d'$ in the null space of $C_2$ has the form 
\begin{equation}
\label{equation_null_space_C2}
d'=(d'_1,d'_2,d'_3,\dots,d'_{2m})^T
=(-d'_{2m},d'_2,-d'_2,\dots,d'_{2m})^T.
\end{equation}
Let $D_2=\diag(d')$ and $B_{2m}=D_2^TC_2D_2$, then $B_{2m}\in\mathcal{D}$.
Similarly, if $d'_2,\dots,d'_{2m}$ are non-zero, whose values will also be chosen later,
then $\rank(B_{2m})=\rank(C_2)=m$, and the $2m$-vector in the null space of $B_{2m}$ has the form 
\begin{equation}
\label{equation_null_space_B}
(b_{2m},b_2,b_2,\dots,b_{2m})^T.
\end{equation}

For any vector in the \emph{common} null space of $A_{2m}$ and $B_{2m}$,
by (\ref{equation_null_space_A}) its $(2k-1)$-th and $2k$-th terms are equal,
and by (\ref{equation_null_space_B}) its $2k$-th and $(2k+1)$-th terms are equal,
so all the terms are equal.
So the common null space of $A_{2m}$ and $B_{2m}$ is 1-dimensional that contains $\mathbf{1}$.
The columns $\alpha_i$ $(1\le i\le 2m)$ of $A_{2m}$ satisfy
\begin{equation}
\label{equation_matrix_A_column_even}
\alpha_1=-\alpha_2, \dots, \alpha_{2m-1}=-\alpha_{2m},
\end{equation}
and the columns $\beta_i$  $(1\le i\le 2m)$ of $B_{2m}$ satisfy
\begin{equation}
\label{equation_matrix_B_column_even}
\beta_2=-\beta_3, \dots, \beta_{2m}=-\beta_1.
\end{equation}
As the common null space of $A_{2m}$ and $B_{2m}$ is 1-dimensional,
so the \emph{row} vectors of $A_{2m}$ and $B_{2m}$ 
span a codimension 1 subspace in $\mathbb{R}^{2m}$, namely, $(2m-1)$-dimensional.
As $A_{2m}$ and $B_{2m}$ are symmetric matrices,
so by (\ref{equation_matrix_A_column_even}) and (\ref{equation_matrix_B_column_even}),
those $2m$ \emph{column} vectors 
\begin{equation}
\label{equation_column_vectors}
\alpha_1, \dots, \alpha_{2m-1}, \beta_2, \dots, \beta_{2m}
\end{equation}
also span a $(2m-1)$-dimensional subspace in $\mathbb{R}^{2m}$.

We next show that for appropriate non-zero $d_i$ and $d'_i$,
the null space of $A_{2m}+B_{2m}$ is also 1-dimensional.

\begin{lemma}
\label{lemma_A_plus_B_U0_even}
For $n=2m-1$ with $m\ge 3$, with proper choice of non-zero $d_i$ and $d'_i$,
the matrix $A_{2m}+B_{2m}$ is in $\mathcal{U}_0$ with rank $2m-1$.
\end{lemma}

\begin{proof}
We first show that if $d_i=d'_i=(-1)^i$, then $A_{2m}+B_{2m}$ has rank $2m-1$.
The $2m$ columns of $A_{2m}+B_{2m}$ are $\alpha_i+\beta_i$,
and our goal is to show that for $1\le k\le 2m-1$, the $2m-1$ vectors 
$\sum_{i=1}^k (\alpha_i+\beta_i)$ are linearly independent,
and thus prove that $A_{2m}+B_{2m}$ has rank $2m-1$.

Let $e=(0,1,\dots,0,1)^T$ be a $2m$-vector, we want to show that $A_{2m}e=B_{2m}e$
(but only for this particular case of $d_i=d'_i=(-1)^i$, not in general for other $d_i$ and $d'_i$).
By how $C_1$ and $C_2$ were constructed above, we have $C_1e=C_2e=-(m-2)\mathbf{1}$.
As $D_1=\diag(d)$ and $D_2=\diag(d')$, so $D_1=D_2$ and $e=D_1e=D_2e$.
Then $C_1D_1e=C_2D_2e=-(m-2)\mathbf{1}$, thus  $D_1^TC_1D_1e=D_2^TC_2D_2e$.
Therefore $A_{2m}e=B_{2m}e$, namely $\sum \alpha_{2i}=\sum \beta_{2i}$.
By (\ref{equation_matrix_A_column_even}), then
\begin{equation}
\label{equation_vector_sum_zero}
\sum\alpha_{2i-1}+\sum\beta_{2i}=0.
\end{equation}

In $\mathbb{R}^{2m}$, for $1\le i\le 2m$,
let $\overrightarrow{OE_i}:=\alpha_i$ for $i$ odd and 
$\overrightarrow{OE_i}:=\beta_i$ for $i$ even.
As $\alpha_1$, \dots, $\alpha_{2m-1}$, $\beta_2$, \dots,  $\beta_{2m}$
span a $(2m-1)$-dimensional subspace (see (\ref{equation_column_vectors})),
so by (\ref{equation_vector_sum_zero}),
$E_i$'s are the vertices of a \emph{non-degenerate} $(2m-1)$-simplex $E$ 
in $\mathbb{R}^{2m}$ with the centroid at the origin $O$.
Notice that $\sum_{i=1}^k (\alpha_i+\beta_i)$,
by (\ref{equation_matrix_A_column_even}) and (\ref{equation_matrix_B_column_even}) and induction on $k$, 
is $\alpha_k+\beta_1$ if $k$ is odd, and is $\beta_k+\beta_1$ is $k$ is even.
As $\beta_1=-\beta_{2m}$ (\ref{equation_matrix_B_column_even}),
so for $k$ odd we have $\alpha_k+\beta_1=\alpha_k-\beta_{2m}=\overrightarrow{E_{2m}E_k}$,
and for $k$ even $\beta_k+\beta_1=\beta_k-\beta_{2m}=\overrightarrow{E_{2m}E_k}$.
As $E$ is a non-degenerate $(2m-1)$-simplex,
so $\overrightarrow{E_{2m}E_k}$ $(1\le k\le 2m-1)$ are linearly independent,
and thus $\sum_{i=1}^k (\alpha_i+\beta_i)$ $(1\le k\le 2m-1)$ are linearly independent.
By the argument at the beginning of the proof, then $A_{2m}+B_{2m}$ has rank $2m-1$
if $d_i=d'_i=(-1)^i$.

This means that for almost all non-zero $d_i$ and $d'_i$ satisfying 
(\ref{equation_null_space_C1}) and (\ref{equation_null_space_C2}), 
$A_{2m}+B_{2m}$ has rank $2m-1$,
and $d_i$ and $d'_i$ can be further adjusted such that for any $i\ne j$,
we have $|d_id'_j|\ne |d_jd'_i|$.
The $2\times 2$ submatrix of $A_{2m}$ that $i$ and $j$ correspond to is
$\begin{pmatrix} d_i^2 & \pm d_id_j  \\ \pm d_id_j & d_j^2 \end{pmatrix}$,
and similarly the corresponding $2\times 2$ submatrix of $B_{2m}$ is
$\begin{pmatrix} {d'_i}^2 & \pm d'_id'_j  \\ \pm d'_id'_j & {d'_j}^2 \end{pmatrix}$.
By Lemma~\ref{lemma_matrix_2_by_2}, for $A_{2m}+B_{2m}$,
all the principal minors of order 2 are positive.
So with proper choice of non-zero $d_i$ and $d'_i$,
$A_{2m}+B_{2m}$ is in $\mathcal{U}_0$ with rank $2m-1$.
\end{proof}


\subsection{For $n$ even}

For an even $n\ge 6$, let $n=2m$. Let $A_{2m}$ be as before
with $A_{2m}=D_1^TC_1D_1$ and proper choice of non-zero $d_i$ in (\ref{equation_null_space_C1}),
and $A_{2m+1}$ be a $(2m+1)\times (2m+1)$ symmetric matrix whose upper-left 
$2m\times 2m$ submatrix is $A_{2m}$ and the other entries are 0. 
Namely, $A_{2m+1}=\begin{pmatrix} A_{2m} & 0  \\ 0 & 0 \end{pmatrix}$.
By (\ref{equation_null_space_A}), we have $\rank(A_{2m+1})=m$,
and the $(2m+1)$-vector in the null space of $A_{2m+1}$ has the form
\begin{equation}
\label{equation_null_space_A_2}
(a_2,a_2,\dots,a_{2m},a_{2m},a_{2m+1})^T.
\end{equation}

Similarly, let $B_{2m+1}$ be a $(2m+1)\times (2m+1)$ symmetric matrix whose lower-right 
$2m\times 2m$ submatrix is $A_{2m}$ (\emph{not} $B_{2m}$ as before) and the other entries are 0.
Namely, $B_{2m+1}=\begin{pmatrix} 0 & 0 \\ 0 & A_{2m}  \end{pmatrix}$.
So $\rank(B_{2m+1})=m$, and the $(2m+1)$-vector in the null space of $B_{2m+1}$ has the form
\begin{equation}
\label{equation_null_space_B_2}
(b_1,b_2,b_2,\dots,b_{2m},b_{2m})^T.
\end{equation}

For any vector in the common null space of $A_{2m+1}$ and $B_{2m+1}$,
by (\ref{equation_null_space_A_2}) its $(2k-1)$-th and $2k$-th terms are equal,
and by (\ref{equation_null_space_B_2}) its $2k$-th and $(2k+1)$-th terms are equal,
so all the terms are equal.
So the common null space of $A_{2m+1}$ and $B_{2m+1}$ is 1-dimensional that contains $\mathbf{1}$.
The non-zero columns $\alpha_i$ $(1\le i\le 2m)$ of $A_{2m+1}$ satisfy 
\begin{equation}
\label{equation_matrix_A_column_odd}
\alpha_1=-\alpha_2, \dots, \alpha_{2m-1}=-\alpha_{2m},
\end{equation}
and the non-zero columns $\beta_i$ $(2\le i\le 2m+1)$ of $B_{2m+1}$ satisfy 
\begin{equation}
\label{equation_matrix_B_column_odd}
\beta_2=-\beta_3, \dots, \beta_{2m}=-\beta_{2m+1}.
\end{equation}
As the common null space of $A_{2m+1}$ and $B_{2m+1}$ is 1-dimensional,
so the row vectors of $A_{2m+1}$ and $B_{2m+1}$ 
span a codimension 1 subspace in $\mathbb{R}^{2m+1}$, namely, $2m$-dimensional.
As $A_{2m+1}$ and $B_{2m+1}$ are symmetric matrices,
so by (\ref{equation_matrix_A_column_odd}) and (\ref{equation_matrix_B_column_odd}),
those $2m$ non-zero column vectors $\alpha_1$, \dots, $\alpha_{2m-1}$, $\beta_2$, \dots, $\beta_{2m}$
also span a $2m$-dimensional subspace in $\mathbb{R}^{2m+1}$, thus are linearly independent.

We next show that the null space of $A_{2m+1}+B_{2m+1}$ is also 1-dimensional.

\begin{lemma}
\label{lemma_A_plus_B_U0_odd}
For $n=2m$ with $m\ge 3$, with proper choice of non-zero $d_i$ $(1\le i\le 2m)$,
the matrix $A_{2m+1}+B_{2m+1}$ is in $\mathcal{U}_0$ with rank $2m$.
\end{lemma}

\begin{proof}
The $2m+1$ columns of $A_{2m+1}+B_{2m+1}$ are $\alpha_i+\beta_i$, $1\le i\le 2m+1$,
with $\beta_1=\alpha_{2m+1}=0$.
Notice that $\sum_{i=1}^k (\alpha_i+\beta_i)$,
by (\ref{equation_matrix_A_column_odd}) and (\ref{equation_matrix_B_column_odd}) and induction on $k$, 
is $\alpha_k$ if $k$ is odd, and is $\beta_k$ is $k$ is even.
As $\alpha_1$, \dots, $\alpha_{2m-1}$, $\beta_2$, \dots, $\beta_{2m}$ are linearly independent, 
so $\sum_{i=1}^k (\alpha_i+\beta_i)$ $(1\le k\le 2m)$ are linearly independent,
and thus $A_{2m+1}+B_{2m+1}$ has rank $2m$.

While both $A_{2m+1}$ and $B_{2m+1}$ each has a zero row and a zero column
(and a $2m\times 2m$ submatrix $A_{2m}$),
with proper choice of non-zero $d_i$ $(1\le i\le 2m)$ in (\ref{equation_null_space_C1}),
by Lemma~\ref{lemma_matrix_2_by_2},
in $A_{2m+1}+B_{2m+1}$ all the principal minors of order 2 are positive.
So $A_{2m+1}+B_{2m+1}$ is in $\mathcal{U}_0$ with rank $2m$.
\end{proof}


The following observation will be useful for the topic in Section~\ref{section_final_remarks}.

\begin{remark}
\label{remark_constant_determinant}
For this particular construction of $A_{2m+1}$ and $B_{2m+1}$ above,
observe that $\rank(A_{2m+1})=\rank(B_{2m+1})=m$.
For the matrix $tA_{2m+1}+\frac{1}{t}B_{2m+1}$,
its determinant (and similarly for any of its minors) can be expressed as a sum of signed products 
such that each summand is the product of a minor (of any order, including 0 and $2m+1$) of $tA_{2m+1}$
and the ``complement'' minor of $\frac{1}{t}B_{2m+1}$.
As any minor of $A_{2m+1}$ and $B_{2m+1}$ of order higher than $m$ is 0,
so the determinant of any $2m\times 2m$ submatrix of $tA_{2m+1}+\frac{1}{t}B_{2m+1}$ is a constant
(because only when $k=m$ the $t^k\times 1/t^{2m-k}$ term is non-vanishing).
This property is analogous to Lemma~\ref{lemma_matrix_2_by_2}.
On the other hand, it is not so for $n=2m-1$. Namely for the matrix $tA_{2m}+\frac{1}{t}B_{2m}$
(not to be confused with the matrix $tA_{2m+1}+\frac{1}{t}B_{2m+1}$),
the determinant of any $(2m-1)\times (2m-1)$ submatrix is not a non-zero constant, in fact, 
it is a linear combination of $t$ and $\frac{1}{t}$ without a constant term.
\end{remark}


\subsection{Summary of the proof}

\begin{theorem}
\emph{(Theorem~\ref{theorem_symmetric_matrix})}
For $n\ge 5$, there exists a continuous family of non-identical 
$(n+1)\times (n+1)$ symmetric matrices $U\in\mathcal{U}_0$ with rank $n$ 
and with the vector $\mathbf{1}$ in the null space,
such that all the principal minors of order 2 are positive and remain constant.
\end{theorem}

\begin{proof}
By Remark~\ref{remark_matrix_rank},
for $n$ odd it is proved by Lemma~\ref{lemma_A_plus_B_U0_even},
and for $n$ even it is proved by Lemma~\ref{lemma_A_plus_B_U0_odd}.
\end{proof}



\section{Relations to Question~\ref{question_simplex_matrix_rigid}
and Question~\ref{question_simplex_triangle_rigid}}

Recall that $\mathcal{F}_0$ is the set of $(n+1)\times (n+1)$ positive semi-definite matrices 
with rank $n$ and with the vector $\mathbf{1}$ in the null space 
(Definition~\ref{definition_matrix_positive_definite}).
With Theorem~\ref{theorem_symmetric_matrix} proved, 
it remains to see if the \emph{positive definite} nature of the matrices in $\mathcal{F}_0$ 
(as compared to the matrices in $\mathcal{U}_0$)
is a barrier to solve Question~\ref{question_simplex_matrix_rigid},
or our method can also be applied to provide counterexamples to Question~\ref{question_simplex_matrix_rigid},
namely, if we can find $A,B\in\mathcal{D}$ such that $A+B\in\mathcal{F}_0$ 
(see Remark~\ref{remark_matrix_rank}).

\begin{remark}
If such $A,B\in\mathcal{D}$ indeed exist, 
because $\mathcal{F}_0$ in an \emph{open} subset of $\mathcal{U}$ 
(Definition~\ref{definition_matrix_symmetric})
and for any small changes of $A$ and $B$ in $\mathcal{D}$ their sum is in $\mathcal{U}$
(Lemma~\ref{lemma_matrix_non_negative}),
so their sum is still in $\mathcal{F}_0$. 
This also makes numerical search possible, at least in theory:
while $\mathcal{D}$ is a high (up to $n$) dimensional space,
it is fairly easy to find all the elements of $\mathcal{D}$,
so we can just sample $A, B\in\mathcal{D}$ numerically, 
and then check if $A+B$ is in $\mathcal{F}_0$. 
\end{remark}

If the numerical search turns out successful for some $n$, then it provides counterexamples to 
Question~\ref{question_simplex_matrix_rigid}, as well as to 
Question~\ref{question_simplex_triangle_rigid} and Question~\ref{question_simplex_rigid}
as they are all equivalent.
Just like Theorem~\ref{theorem_symmetric_matrix} addresses a variant of 
Question~\ref{question_simplex_matrix_rigid}, 
we next address the corresponding variant of Question~\ref{question_simplex_triangle_rigid}.

In general, no matter the numerical search is successful or not,
in the following we provide counterexamples to a variant of Question~\ref{question_simplex_triangle_rigid}
where $\mathbb{R}^n$ is replaced by a pseudo-Euclidean space $\mathbb{R}^{p,n-p}$ 
for some unspecified $p\ge 2$.
Here $\mathbb{R}^{p,n-p}$ is an $n$-dimensional vector space endowed with a bilinear product
\[x\cdot y=x_1y_1+\cdots +x_py_p-x_{p+1}y_{p+1}-\cdots -x_ny_n.
\]

For any $k$-simplex $G$ in $\mathbb{R}^{p,n-p}$,
let $u_i$ $(1\le i\le k)$ be the vectors from a fixed vertex to other vertices. 
Then the $k$-volume (or pseudo-volume) of $G$, multiplied by a factor $k!$, is well defined as
\begin{equation}
\label{equation_volume_pseudo_Euclidean}
k!\cdot V_k(G):=|\det(u_i\cdot u_j)_{1\le i,j\le k}|^{1/2}, 
\end{equation}
where $u_i\cdot u_j$ is the bilinear product in $\mathbb{R}^{p,n-p}$.
Unlike in the Euclidean case, due to the non-positive definite nature of $\mathbb{R}^{p,n-p}$,
here $\det(u_i\cdot u_j)_{1\le i,j\le k}$ may be negative or zero.

\begin{theorem}
\emph{(Main Theorem 3)}
\label{theorem_simplex_triangle_non_rigid}
For any $n\ge 5$, in $\mathbb{R}^{p,n-p}$ for some unspecified $p\ge 2$,
there exists a continuous family of non-congruent $n$-simplices $P$
with the centroid fixed at the origin $O$, such that all triangles $OP_iP_j$ are in Euclidean planes
and the areas remain constant during the motion.
\end{theorem}

\begin{proof}
By Lemma~\ref{lemma_A_plus_B_U0_even} for $n$ odd 
and Lemma~\ref{lemma_A_plus_B_U0_odd} for $n$ even,
we can find $(n+1)\times (n+1)$ symmetric matrices $A,B\in\mathcal{D}$ 
such that $A+B$ is in $\mathcal{U}_0$ with rank $n$.
Denote $A+B$ by $U$, and the upper-left $n\times n$ submatrix of $U$ by $U_0$.
As $U_0$ is symmetric, it can be decomposed to $V_0^TDV_0$,
where $V_0$ is an $n\times n$ matrix
and $D$ is an $n\times n$ diagonal matrix with the diagonal entries being
1's followed by $-1$'s and 0's (we will rule out the 0's next).

Let $V$ be an $n\times (n+1)$ matrix such that its $i$-th column $v_i$ $(1\le i\le n)$
is the $i$-th column of $V_0$, and its $(n+1)$-th column $v_{n+1}$ satisfies 
$v_{n+1}=-\sum_{i=1}^n v_i$.
So $\sum_{i=1}^{n+1} v_i=0$, then $V\cdot\mathbf{1}=0$ and thus $V^TDV\cdot\mathbf{1}=0$.
As $U\cdot\mathbf{1}=0$ and both $U$ and $V^TDV$ share the
same upper-left $n\times n$ submatrix $U_0=V_0^TDV_0$, 
so $U$ and $V^TDV$ have the same upper $n\times (n+1)$ matrix.
As both $U$ and $V^TDV$ are symmetric, 
so $U$ and $V^TDV$ also have the same left $(n+1)\times n$ submatrix;
apply $U\cdot\mathbf{1}=V^TDV\cdot\mathbf{1}=0$ again, then $U=V^TDV$.
As $U$ has rank $n$, so both $V$ and $D$ have full rank $n$.
Thus $D$'s diagonal entries contain 1's followed by $-1$'s only, with no 0's.
Assume the number of 1's and $-1$'s are $p$ and $n-p$ respectively.

In $\mathbb{R}^{p,n-p}$, let $P$ be an $n$-simplex whose vertices $P_i$ $(1\le i\le n+1)$ 
satisfy $v_i=\overrightarrow{OP_i}$.
As $\sum_{i=1}^{n+1} v_i=0$, then $O$ is the centroid of $P$.
The entries of $U$ (same as $V^TDV$), $v_i^TDv_j$, can also be written as $v_i\cdot v_j$
where ``$\cdot$'' is the bilinear product in $\mathbb{R}^{p,n-p}$.
As $U\in\mathcal{U}_0$, so any diagonal $2\times 2$ submatrix
$\begin{pmatrix} v_i^2 & v_i\cdot v_j \\ v_i\cdot v_j & v_j^2 \end{pmatrix}$
of $U$ is positive definite.
Therefore any triangle $OP_iP_j$ spans a two-dimensional Euclidean plane.

For any $t$ in a small neighborhood of 1, we still have $tA+\frac{1}{t}B\in\mathcal{U}_0$.
So we can construct a continuous family of non-congruent simplices $P$ 
in $\mathbb{R}^{p,n-p}$ from $tA+\frac{1}{t}B$.
By Lemma~\ref{lemma_matrix_non_negative}, the areas of all triangles $OP_iP_j$ 
remain constant when $t$ varies. This finishes the proof.
\end{proof}

\begin{remark}
In Theorem~\ref{theorem_simplex_triangle_non_rigid}, for a fixed $n$, the $p$ may not be unique.
Even when $A$ and $B$ are fixed, $tA+\frac{1}{t}B$ and $A+B$ 
may potentially correspond to different $p$'s when $t$ moves away from 1.
This is because $tA+\frac{1}{t}B$ may have a rank lower than $n$ for some $t=t_0$,
then for $t<t_0$ and $t>t_0$ in a small neighborhood of $t_0$,
it may correspond to different $p$'s.
\end{remark}

We have the following interesting observation,
which will be further addressed in Section~\ref{section_final_remarks}.

\begin{remark}
\label{remark_constant_volume_P}
For $n=2m$, in Lemma~\ref{lemma_A_plus_B_U0_odd},
for the \emph{particular} construction of $A_{2m+1}$ and $B_{2m+1}$,
we have $\rank(A_{2m+1})=\rank(B_{2m+1})=m$.
Construct a continuous family of non-congruent $2m$-simplices $P$ from $tA_{2m+1}+\frac{1}{t}B_{2m+1}$,
and use the formula (\ref{equation_volume_pseudo_Euclidean}) to compute the volumes
of the $2m$-simplices formed by $O$ and the $2m+1$ facets of $P$,
then by Remark~\ref{remark_constant_determinant}, 
the volumes of those $2m$-simplices remain constant when $t$ varies.
As those simplices add up to $P$, so $V_{2m}(P)$ also remains constant when $t$ varies 
(but it is not so for $n=2m-1$).
\end{remark}

We use Theorem~\ref{theorem_simplex_triangle_non_rigid} 
to prove Theorem~\ref{theorem_flexible_pseudo_Euclidean} next.



\section{Proof of Theorem~\ref{theorem_flexible_pseudo_Euclidean}}

In $\mathbb{R}^{p,n-p}$, let $P$ be an $n$-simplex with the centroid at the origin $O$
(but for now the triangles $OP_iP_j$ need not all be in Euclidean planes 
as required in Theorem~\ref{theorem_simplex_triangle_non_rigid}).
Let $Q$ be a \emph{dual} of $P$ 
\[Q=\{y \in \mathbb{R}^{p,n-p}: x\cdot y \le c \quad\text{for all $x\in P$}\}
\]
for some $c>0$, see also (\ref{equation_dual_Euclidean}).
When $c=1$, $Q$ is the \emph{polar dual} $P^{\ast}$ of $P$.
Let $v_i=\overrightarrow{OP_i}$. Just like in the Euclidean case, 
it can be verified that $Q$ is an $n$-simplex in a finite region with the centroid at the origin $O$,
and $v_i$ is a normal vector at the $(n-1)$-face $F_i$ of $Q$. 
But unlike in $\mathbb{R}^n$, here $v_i$ is an \emph{outward} normal vector to $Q$ at $F_i$ if $v_i^2>0$,
an \emph{inward} normal vector to $Q$ at $F_i$ if $v_i^2<0$, or parallel to $F_i$ if $v_i^2=0$.
However, this does not affect our results.

Now for $P$ also let all triangles $OP_iP_j$ be in Euclidean planes.
This insures that the area of $OP_iP_j$ is non-zero.
In this case of $P$, we have $v_i^2>0$ for all $i$, 
so $v_i$'s are still all outward normal vectors to $Q$ at $F_i$'s.
In $\mathbb{R}^n$, 
the volume of any face (of any dimension) of $Q$ can be computed from the information of $P$
using a simple formula (e.g. see \cite[Theorem 19]{Lee:stress});
the formula only uses the fact that $O$ is the centroid of $P$
and can be applied to $\mathbb{R}^{p,n-p}$ as well.
As a direct result, it shows that similar to the $\mathbb{R}^n$ case,
for the $(n-2)$-face $F_{ij}$ of $Q$ on the intersection of $F_i$ and $F_j$,
the volume $V_{n-2}(F_{ij})$ is proportional to the area of the triangle $OP_iP_j$.


\begin{lemma}
\label{lemma_volume_proportional_pseudo_Euclidean}
In $\mathbb{R}^{p,n-p}$, 
let $F_{ij}$ be the $(n-2)$-face of $Q$ on the intersection of the $(n-1)$-faces $F_i$ and $F_j$,
then $V_{n-2}(F_{ij})$ is proportional to the area of the triangle $OP_iP_j$.
\end{lemma}

Now we are ready to prove Theorem~\ref{theorem_flexible_pseudo_Euclidean}.

\begin{theorem}
\emph{(Theorem~\ref{theorem_flexible_pseudo_Euclidean})}
For any $n\ge 5$, in $\mathbb{R}^{p,n-p}$ for some unspecified $p\ge 2$,
their exists a continuous family of non-congruent $n$-simplices $Q$,
such that  all the dihedral angles are Euclidean angles,
and the $(n-2)$-volumes of all the $(n-2)$-faces of $Q$ remain constant during the motion.
\end{theorem}

\begin{proof}
By Theorem~\ref{theorem_simplex_triangle_non_rigid},
there exists a continuous family of non-congruent $n$-simplices $P$ 
with the centroid fixed at the origin $O$
such that all triangles $OP_iP_j$ are in Euclidean planes and the areas remain constant.
Let $Q$ be a dual of $P$ for some factor $c$ ($c$ may vary when $P$ varies),
then the dihedral angles of $Q$ are all Euclidean angles.
With the proper scale of $c$ for $Q$,
by Lemma~\ref{lemma_volume_proportional_pseudo_Euclidean},
the $(n-2)$-volumes of all the $(n-2)$-faces of $Q$ can be adjusted to remain constant during the motion.
This finishes the proof.
\end{proof}



\section{Some final remarks}
\label{section_final_remarks}

Let $Q$ be any potential flexible $n$-simplex with fixed volumes of codimension 2 faces 
(in $\mathbb{R}^n$ or $\mathbb{R}^{p,n-p}$, and including $n=4$ as well
though we do not know if counterexamples exist or not).
With Theorem~\ref{theorem_flexible_pseudo_Euclidean} proved,
in the spirit of the bellows conjecture \cite{Sabitov:invariance,ConnellySabitovWalz} (or for other heuristics),
one may ask that if the volume of $Q$ is also necessarily preserved during the motion.
We give a very loose discussion on this topic based on the counterexamples
we constructed in the proof of Theorem~\ref{theorem_flexible_pseudo_Euclidean}.

Let $O$ be the centroid of $Q$ and $P$ be a dual of $Q$. 
By Lemma~\ref{lemma_volume_proportional_pseudo_Euclidean},
with the proper scale of $P$, the areas of all triangles $OP_iP_j$ 
remain constant during the motion.
By applying \cite[Theorem 19]{Lee:stress} again, skipping the proofs,
we can show that for the volume of $Q$ to remain constant,
\emph{if and only if} the volume of $P$ remains constant;
and if so, $P$ can be chosen as the \emph{polar} dual $Q^{\ast}$ of $Q$.
Regarding the volume of $P$, it is addressed in Remark~\ref{remark_constant_volume_P}
that for the \emph{particular} counterexamples of $P$ we constructed in 
Theorem~\ref{theorem_simplex_triangle_non_rigid} (from Lemma~\ref{lemma_A_plus_B_U0_odd}),
the volume of $P$ does remain constant for $n$ even but not for $n$ odd.
Thus, for the particular counterexamples of $Q$ we constructed in
Theorem~\ref{theorem_flexible_pseudo_Euclidean},
the volume of $Q$ remains constant for $n$ even but not for $n$ odd.

To sum up, it seems natural for us to ask the following (open) question for $n$ even
for both the Euclidean and the pseudo-Euclidean cases, including $n=4$ as well.

\begin{question}
For $n$ even $(n\ge 4)$ in $\mathbb{R}^n$ or $\mathbb{R}^{p,n-p}$, 
if there exists a continuous family of $n$-simplices $Q$ 
such that the $(n-2)$-volumes of all the $(n-2)$-faces remain constant during the motion,
then does the volume of $Q$ also necessarily remain constant?
\end{question}






%%%%%%%%%%%%%%%%%%%%%%%%%%%%%%%%%%%%%%%%%%%
%%%%%%%%%%%%%%%%%%%%%%%%%%%%%%%%%%%%%%%%%%%

%\noindent
%{\bf Acknowledgements:}
%%\begin{acknowledgements}
%%\end{acknowledgements}

{\footnotesize
% BibTeX users please use one of
\bibliographystyle{abbrv}  %choose between plain, abbrv, acm
%\bibliographystyle{plain}
%\bibliographystyle{acm}
\bibliography{codimension2_arxiv}   % name your BibTeX data base

% Non-BibTeX users please use
%\begin{thebibliography}{}
%
% and use \bibitem to create references. Consult the Instructions
% for authors for reference list style.
%
% etc
%\end{thebibliography}
}

\end{document}
% end of file template.tex







