\documentclass[11pt, oneside]{article}


\usepackage{enumerate}
\usepackage{amscd}
\usepackage{amsmath}
\usepackage{epsfig}
\usepackage{amssymb}
\usepackage{amsthm}    %now using this package instead of package "theorem"
\usepackage{latexsym}
\usepackage{color} %new, added to handle input of .pspdftex file
\usepackage{graphicx}
\usepackage{url}
%\usepackage{hyperref} %new, to handle URL's in bibtex

\setlength{\oddsidemargin}{0.25in}      % 1.25in left margin
\setlength{\evensidemargin}{0.25in}     % 1.25in left margin (even pages)
\setlength{\topmargin}{0.0in}           % 1in top margin
\setlength{\textwidth}{6.0in}           % 6.0in text - 1.25in rt margin
\setlength{\textheight}{8.5in}          % Body ht for 1in margins

\pagestyle{plain}
%%%%



\numberwithin{equation}{section}

\theoremstyle{plain}
\newtheorem{theorem}{Theorem}[section]
\newtheorem{proposition}[theorem]{Proposition} %[section]
\newtheorem{lemma}[theorem]{Lemma} %[section]
\newtheorem{corollary}[theorem]{Corollary} %[section]
\newtheorem{conjecture}[theorem]{Conjecture} %[section]
\theoremstyle{definition}
\newtheorem{definition}[theorem]{Definition} %[section]
\newtheorem{question}[theorem]{Question} %[section]
\theoremstyle{remark}
\newtheorem{remark}[theorem]{Remark} %[section]
\newtheorem{example}[theorem]{Example} %[section]


%%%%%%%%%%%%%%%%
%%%%%%%%%%%%%%%%
\newcommand{\email}[1]{{\scriptsize{\it E-mail address}\/: {\rm #1}} }
\providecommand{\norm}[1]{\lVert#1\rVert}
\DeclareMathOperator{\diag}{diag}
\DeclareMathOperator{\rank}{rank}

%\providecommand{\keywords}[1]{\textbf{\textit{Index terms---}} #1} %added new
\providecommand{\keywords}[1]{\textbf{Keywords }#1} %added new

\makeatletter
\def\blfootnote{\gdef\@thefnmark{}\@footnotetext}
\makeatother

\begin{document}
\blfootnote{\textup{2020} \textit{Mathematics Subject Classification}.
52C25, 52B11, 51M25}

\blfootnote{\textit{key words and phrases}.
rigid motion, flexible simplex, dual, pseudo-Euclidean space.}

\blfootnote{\email{lizhaozhang@alum.mit.edu}}

\begin{titlepage}
\title{Simplices with fixed volumes of codimension 2 faces in a continuous deformation}

\author{Lizhao Zhang
%\footnote{\email{lizhaozhang@alum.mit.edu}}
}
\date{}
\end{titlepage}



\maketitle


\begin{abstract}
For any $n$-dimensional simplex in the Euclidean space $\mathbb{R}^n$ with $n\ge 4$,
it is asked that if a continuous deformation preserves the volumes of all the codimension 2 faces,
then is it necessarily a \emph{rigid} motion.
While the question remains open and the general belief is that the answer is affirmative,
for all $n\ge 4$, we provide counterexamples to a variant of the question 
where $\mathbb{R}^n$ is replaced by a pseudo-Euclidean space $\mathbb{R}^{p,n-p}$ 
for some unspecified $p\ge 2$.
%\keywords{rigid motion \and flexible simplex \and dual \and pseudo-Euclidean space}
%\subclass{52C25 \and 52B11 \and 51M25}
\end{abstract}



\section{Introduction}

For any $n$-dimensional simplex in $\mathbb{R}^n$,
there are $\binom{n+1}{2}$ edges,
and up to congruence the $n$-simplex is uniquely determined by its edge lengths.
As there are also the same number of $(n-2)$-faces,
it is natural to ask the following question.

\begin{question}
\label{question_simplex_rigid}
For any $n$-simplex $Q$ in $\mathbb{R}^n$ with $n\ge 4$, if a continuous deformation preserves 
the $(n-2)$-volumes of all the $(n-2)$-faces of $Q$, then is it necessarily a \emph{rigid} motion?
\end{question}

The discrete version of the question, without any continuous deformation involved,
asks that if the $(n-2)$-volumes of all the $(n-2)$-faces always determine the $n$-simplex up to \emph{congruence}.
It asks for $n\ge 4$ because the answer for $n=2$ is trivially negative and for $n=3$ is trivially affirmative. 
This is a classical problem that comes natural and many people may have asked it in the past,
but it is hard to say who asked it first;
for some account of the background of the question, see \cite{MoharRivin}.
The discrete version was answered negatively for all $n\ge 4$,
and various counterexamples were constructed, see McMullen~\cite{McMullen:simplices}
and Mohar and Rivin~\cite{MoharRivin}.
In fact, the counterexample in \cite{MoharRivin} was not only simple,
but can also be easily extended to construct counterexamples to show that
for any $r\ge 2$ the $r$-volumes of all $r$-faces of an $n$-simplex do not necessarily 
determine the $n$-simplex up to congruence;
and it can be extended to the spherical and hyperbolic spaces as well, 
see Zhang~\cite[Section~2.12]{Zhang:rigidity}.

However, Question~\ref{question_simplex_rigid} remains open,
and the general belief is that the answer may still be affirmative,
e.g., see \cite[Unsolved problem 10]{Sabitov:algebraic}.
For $n=4$, Gaifullin~\cite{Gaifullin:volume} showed that,
given a \emph{generic} set of 2-face areas,
there exist not more than a finite congruence classes of 4-simplices with the given 2-face areas.
The proof used an algebraic approach to show that the square of any edge length 
is a root of a polynomial whose coefficients (including the leading term) are polynomials
in these variables of the squares of the 2-face areas, 
and thus only has finite possibilities in the generic situation. 
Gaifullin also pointed out that it is much more interesting to obtain results
that hold for \emph{all} simplices, as possible \emph{non}-generic situation 
may arise when the polynomial coefficients are all zero,
so for $n=4$ Question~\ref{question_simplex_rigid} is still not resolved yet.

While we do not solve Question~\ref{question_simplex_rigid} in this paper, 
but in some sense contrary to the general belief,
for all $n\ge 4$, we provide counterexamples to a close variant of this question 
where $\mathbb{R}^n$ is replaced by a pseudo-Euclidean space $\mathbb{R}^{p,n-p}$ 
for some unspecified $p\ge 2$.
The treatment is different for $n=4$ and $n\ge 5$.

\begin{theorem}
\label{theorem_flexible_pseudo_Euclidean_n_4}
\emph{(Main Theorem 1)}
For $n=4$ in $\mathbb{R}^{3,1}$,
their exists a continuous family of non-congruent $4$-simplices $Q$,
such that  all the 2-faces of $Q$ are in Euclidean planes,
and all the areas remain constant during the deformation.
\end{theorem}


\begin{theorem}
\label{theorem_flexible_pseudo_Euclidean}
\emph{(Main Theorem 2)}
For any $n\ge 5$, in $\mathbb{R}^{p,n-p}$ for some unspecified $p\ge 2$,
their exists a continuous family of non-congruent $n$-simplices $Q$,
such that  all the dihedral angles are Euclidean angles,
and the $(n-2)$-volumes of all the $(n-2)$-faces of $Q$ remain constant during the deformation.
\end{theorem}

Here a \emph{Euclidean angle} means that for an $(n-2)$-face of the $n$-simplex, 
its orthogonal plane in $\mathbb{R}^{p,n-p}$ is a 2-dimensional Euclidean plane, and thus requires $p\ge 2$;
besides, it also ensures that the $(n-2)$-face has non-degenerate metric and thus has non-zero volume.
We stress that being Euclidean angles is neither a necessary condition 
nor a feature that we are particularly looking for, but it helps to present our result.
By contrast, in Theorem~\ref{theorem_flexible_pseudo_Euclidean_n_4} 
the dihedral angles are not Euclidean angles.
While not yet resolving Question~\ref{question_simplex_rigid},
Theorem~\ref{theorem_flexible_pseudo_Euclidean_n_4} and \ref{theorem_flexible_pseudo_Euclidean}
are still somewhat surprising results, as the non-positive definite nature of the pseudo-Euclidean space 
does not necessarily seem to make the question easier in an obvious way.
They might also suggest, heuristically,
that there may exist counterexamples to Question~\ref{question_simplex_rigid} for some $n$.
But we remark that both Theorem~\ref{theorem_flexible_pseudo_Euclidean_n_4}
and \ref{theorem_flexible_pseudo_Euclidean} are of interest in their own right,
no matter if Question~\ref{question_simplex_rigid} is finally answered positively or negatively in the future.

Rigidity and flexibility of geometric frameworks have been extensively investigated in the past.
A flexible polyhedron in $\mathbb{R}^3$ is a closed polyhedral surface that admits continuous
non-rigid deformation such that all faces remain rigid.
The first embedded (non-self-intersecting) flexible polyhedron in $\mathbb{R}^3$ was discovered by 
Connelly~\cite{Connelly:counterexample}.
It was also shown that the volume of any flexible polyhedron in $\mathbb{R}^3$
remains constant during the continuous deformation, proving the bellows conjecture
(see Sabitov~\cite{Sabitov:invariance} and Connelly \emph{et al.}~\cite{ConnellySabitovWalz}).
The bellows conjecture  in the spherical case $\mathbb{S}^3$ was disproved 
by Alexandrov~\cite{Alexandrov:flexible}.
If the facets need not be rigid as above, it was shown that if an $n$-simplex starts as a 
\emph{degenerate} simplex in a continuous deformation in the Euclidean, spherical or hyperbolic spaces,
then in generic situation its degeneracy is always preserved by a single constraint on its facet volumes,
with the non-generic situation also precisely specified (Zhang~\cite{Zhang:lifting}).

For any $n$-simplex with fixed volumes of codimension 2 faces
in a continuous deformation in $\mathbb{R}^n$ or $\mathbb{R}^{p,n-p}$
(with a slight abuse of terminology, it can be loosely treated as a variant of \emph{flexible simplex}),
analogous to the bellows conjecture, one may ask that if the volume of the $n$-simplex itself
also necessarily remains constant during the deformation.
A discrete version of the question in $\mathbb{R}^n$ due to Connelly was already answered 
negatively for all $n\ge 4$.
If $n$ is even and $n\ge 6$, while we do not fully answer this question, 
for what it is worth, we observe that, 
for the \emph{particular} counterexamples we constructed in $\mathbb{R}^{p,n-p}$
in Theorem~\ref{theorem_flexible_pseudo_Euclidean}, 
the volume of the $n$-simplex \emph{does} remain constant during the deformation.
We answer this question negatively if $n=4$ or $n$ is odd.

\begin{theorem}
\label{theorem_flexible_volume_non_constant}
If $n=4$ or $n\ge 5$ and $n$ is odd, in $\mathbb{R}^{p,n-p}$ for some unspecified $p\ge 2$,
their exists a continuous family of non-congruent $n$-simplices $Q$,
such that the $(n-2)$-volumes of all the $(n-2)$-faces of $Q$ remain constant during the deformation,
but the volume of $Q$ does not remain constant.
\end{theorem}
 
Our strategy to prove Theorem~\ref{theorem_flexible_pseudo_Euclidean_n_4}
and \ref{theorem_flexible_pseudo_Euclidean} is organized as follows.
We first have a very simple but important observation Lemma~\ref{lemma_matrix_2_by_2},
and apply it directly to prove Theorem~\ref{theorem_flexible_pseudo_Euclidean_n_4}.
For Theorem~\ref{theorem_flexible_pseudo_Euclidean},
we transform it into equivalent questions about the \emph{dual} of the $n$-simplex,
and then apply Lemma~\ref{lemma_matrix_2_by_2} again to prove them
(Theorem~\ref{theorem_symmetric_matrix} and \ref{theorem_simplex_triangle_non_rigid}).
The proofs of Theorem~\ref{theorem_flexible_pseudo_Euclidean_n_4}
and \ref{theorem_flexible_pseudo_Euclidean} are constructive for all $n\ge 4$,
and can be read independently of each other.



\section{Preliminaries}

For $0\le p\le n$, $\mathbb{R}^{p,n-p}$ is an $n$-dimensional vector space endowed with a bilinear product
\[x\cdot y=x_1y_1+\cdots +x_py_p-x_{p+1}y_{p+1}-\cdots -x_ny_n.
\]
For a $k$-simplex $G$ in $\mathbb{R}^{p,n-p}$, choose any vertex as a base point, 
and let $u_i$ $(1\le i\le k)$ be the vectors from this vertex to other $k$ vertices,
we obtain a corresponding matrix
\begin{equation}
\label{equation_simplex_matrix}
(u_i\cdot u_j)_{1\le i,j\le k}.
\end{equation}
As
\begin{equation}
\label{equaiton_dot_product_edge_squared}
u_i\cdot u_j=\frac{1}{2}(u_i^2+u_j^2-(u_i-u_j)^2),
\end{equation}
so the matrix, without knowing more details of $u_i$,
contains all the information of the edges $u_i^2$ and $(u_i-u_j)^2$.
As $u_i^2$ (as well as $(u_i-u_j)^2$) may be negative in $\mathbb{R}^{p,n-p}$,
we refer to it as the \emph{pseudo length squared} of the edge,
in contrast to the \emph{length squared} $|u_i^2|$,
which by definition is always non-negative.
By a variant of the Cayley--Menger determinant, the $k$-volume of $G$, 
independent of the base point chosen, is well defined as
\begin{equation}
\label{equation_volume_pseudo_Euclidean}
k!\cdot V_k(G):=|\det(u_i\cdot u_j)_{1\le i,j\le k}|^{1/2}.
\end{equation}
By the argument above, $\det(u_i\cdot u_j)_{1\le i,j\le k}$ is a polynomial in $u_i^2$ and $(u_i-u_j)^2$.

In the Euclidean case $\mathbb{R}^n$, if the simplex is non-degenerate,
then the matrix $(u_i\cdot u_j)_{1\le i,j\le k}$ is positive definite and 
$\det(u_i\cdot u_j)_{1\le i,j\le k}$ is positive.
If $p<n$, unlike in the Euclidean case, due to the non-positive definite nature of $\mathbb{R}^{p,n-p}$,
here $\det(u_i\cdot u_j)_{1\le i,j\le k}$ may be negative or zero.

\begin{remark}
\label{remark_zero_volume_non_degenerate}
If $k<n$, we caution that if $\det(u_i\cdot u_j)_{1\le i,j\le k}$ is zero,
which means that the matrix $(u_i\cdot u_j)_{1\le i,j\le k}$ does not have full rank,
then $G$ may still be \emph{non}-degenerate.%
\footnote{By degenerate we mean that the vertices of the simplex 
are confined to a lower $(k-1)$-dimensional space.
}
This happens when $G$ is contained in a $k$-dimensional plane that has degenerate metric.
But if $\det(u_i\cdot u_j)_{1\le i,j\le k}$ is non-zero, 
then $G$ is always non-degenerate.
\end{remark}

Particularly for $k=2$, if the $2\times 2$ matrix
\begin{equation}
\label{equation_matrix_triangle}
\begin{pmatrix} u_1^2 & u_1\cdot u_2 \\ u_1\cdot u_2 & u_2^2 \end{pmatrix}
\end{equation}
is positive definite, then this matrix determines a Euclidean triangle
with the squares of the edge lengths being 
$u_1^2$, $u_2^2$, and $(u_1-u_2)^2=u_1^2+u_2^2-2u_1\cdot u_2$.

We have the following simple but important observation.

\begin{lemma}
\label{lemma_matrix_2_by_2}
For any $2\times 2$ matrices $A=\begin{pmatrix} a_1^2 & a_1a_2 \\ a_1a_2 & a_2^2 \end{pmatrix}$
and $B=\begin{pmatrix} b_1^2 & b_1b_2 \\ b_1b_2 & b_2^2 \end{pmatrix}$, 
the determinant of $tA+\frac{1}{t}B$
is independent of the value of $t$, and is positive unless $a_1b_2=a_2b_1$.
\end{lemma}

\begin{proof}
Because $\det(A)=0$ and $\det(B)=0$, so the determinant of $tA+\frac{1}{t}B$ 
does not have the $t^2$ and $\frac{1}{t^2}$ terms.
This leaves only the constant term (as a product of $t$ and $\frac{1}{t}$)
\[a_1^2b_2^2+a_2^2b_1^2-2a_1a_2b_1b_2=(a_1b_2-a_2b_1)^2,
\]
which is positive unless $a_1b_2=a_2b_1$.
\end{proof}

\begin{remark}
\label{remark_geometric_interpretation}
Here is a geometric interpretation of Lemma~\ref{lemma_matrix_2_by_2}.
By the discussion above,  unless $a_1b_2=a_2b_1$,
the $2\times 2$ matrix $tA+\frac{1}{t}B$ determines a Euclidean triangle up to congruence.
Then Lemma~\ref{lemma_matrix_2_by_2} means that,
the square of the area of the triangle is independent of the value of $t$.
Conversely, the matrix $tA+\frac{1}{t}B$ can be viewed as decomposed into
two ``degenerate'' components $tA$ and $\frac{1}{t}B$
that each corresponds to a triangle with zero volume respectively.
\end{remark}

Lemma~\ref{lemma_matrix_2_by_2} plays a key role in the proofs of
both Theorem~\ref{theorem_flexible_pseudo_Euclidean_n_4}
and \ref{theorem_flexible_pseudo_Euclidean},
where both proofs are essentially built on how to apply Lemma~\ref{lemma_matrix_2_by_2} properly
to find counterexamples that can be decomposed into two degenerate components.
The meanings of both \emph{decompose} and \emph{degenerate} are problem-specific.




\section{Proof of Theorem~\ref{theorem_flexible_pseudo_Euclidean_n_4}}

This is a standalone section, and can be read independently of the rest of the paper.

Recall that an $n$-simplex in $\mathbb{R}^{p,n-p}$, with a base point chosen, 
corresponds to a symmetric $n\times n$ matrix as in (\ref{equation_simplex_matrix}).
By (\ref{equaiton_dot_product_edge_squared}) the matrix also determines the 
\emph{pseudo length squared} of all the edges of the simplex, and the relationship is linear.
This linear relationship can be extended to be between \emph{any} symmetric simplex
and what we call a \emph{pseudo $n$-simplex}, degenerate or not,
without concerning whether it can be realized in any space $\mathbb{R}^{p,q}$.
We can assign any real values (including negative numbers and zero) 
to the pseudo length squared of the edges of a pseudo simplex.
With respect to the matrix operations, pseudo simplices can take a \emph{summation} by adding 
the pseudo length squared of the corresponding edges together,
or similarly, take a product with a scalar $t$.
By (\ref{equaiton_dot_product_edge_squared}) those edge information 
can recover $u_i\cdot u_j$, though we do not even know how ``$\cdot$'' is defined here,
and by (\ref{equation_volume_pseudo_Euclidean}) 
we can calculate the pseudo simplex's volume as well as all its face volumes.

For $n=4$, we can apply Lemma~\ref{lemma_matrix_2_by_2} directly to prove 
Theorem~\ref{theorem_flexible_pseudo_Euclidean_n_4}.
The idea is to find two pseudo 4-simplices $Q_1$ and $Q_2$ with the properties:
(1) all the 2-faces of $Q_1$ and $Q_2$ have zero volumes, and
(2) the summation of $tQ_1$ and $\frac{1}{t}Q_2$, denote by $Q$, 
is non-degenerate and contains only Euclidean 2-faces,
and can be realized in $\mathbb{R}^{3,1}$.%
\footnote{If both $Q_1$ and $Q_2$ are ``five points on a line'',
then they easily satisfy condition (1), but not (2) as they both correspond to a matrix with rank 1,
and thus $Q$ corresponds to a matrix whose rank is at most 2,
so we need to search for more complicated $Q_1$ and $Q_2$.
Also, if $Q$ can be realized in $\mathbb{R}^4$ instead (replacing $\mathbb{R}^{3,1}$),
then we would have found counterexamples to Question~\ref{question_simplex_rigid} for $n=4$.
}
Once we find such $Q_1$ and $Q_2$ satisfying the conditions above, 
then by applying Lemma~\ref{lemma_matrix_2_by_2} 
(see also Remark~\ref{remark_geometric_interpretation}),
$Q$ is what we are looking for 
to prove Theorem~\ref{theorem_flexible_pseudo_Euclidean_n_4}.

We directly construct $Q_1$ and $Q_2$ below.
Let $Q_1$'s vertices be numbered from 0 to 4, 
and $u_{ij}^2=u_{ji}^2$ be the pseudo length squared of the edge between vertices $i$ and $j$
(see Figure~\ref{figure_simplex_4} (a)).
First ignore vertex 4 for a moment, and for some non-zero $a_1$, $a_2$ and $a_3$
satisfying $a_1+a_2+a_3=0$, let 
\[u_{01}^2=u_{23}^2=a_1^2,
\quad u_{02}^2=u_{13}^2=a_2^2,
\quad u_{03}^2=u_{12}^2=a_3^2.
\]
So we construct a pseudo 3-simplex with equal pseudo length squared on opposite edges.
Now treat vertex 4 the same as vertex 0, such that
\[u_{04}^2=0, 
\quad u_{41}^2=u_{01}^2=a_1^2,
\quad u_{42}^2=u_{02}^2=a_2^2,
\quad u_{43}^2=u_{03}^2=a_3^2.
\]

% Figure environment removed

It can be verified that all 2-faces of $Q_1$ have zero volumes: 
e.g., for the triangle with vertices 0, 1 and 2, we have
\[\det
\begin{pmatrix}
u_{01}^2 & u_{01}\cdot u_{02} \\
u_{01}\cdot u_{02} & u_{02}^2
\end{pmatrix}
=\det
\begin{pmatrix}
a_1^2 & (a_1^2+a_2^2-a_3^2)/2 \\
(a_1^2+a_2^2-a_3^2)/2 & a_2^2 \\
\end{pmatrix}
,
\]
and because $a_1+a_2+a_3=0$, so it is equal to
$\det\begin{pmatrix}a_1^2 & -a_1a_2 \\ -a_1a_2 & a_2^2\end{pmatrix}$,
which is 0.

Let $Q_2$'s vertices be numbered from 0 to 4 as well, 
and $v_{ij}^2=v_{ji}^2$ be the pseudo length squared of the edge between vertices $i$ and $j$
(see Figure~\ref{figure_simplex_4} (b)).
First ignore vertex 1, and for some non-zero $b_2$, $b_3$ and $b_4$
satisfying $b_2+b_3+b_4=0$, let 
\[v_{02}^2=v_{34}^2=b_2^2,
\quad v_{03}^2=v_{24}^2=b_3^2,
\quad v_{04}^2=v_{23}^2=b_4^2.
\]
Now treat vertex 1 the same as vertex 0, such that
\[v_{01}^2=0, 
\quad v_{12}^2=v_{02}^2=b_2^2,
\quad v_{13}^2=v_{03}^2=b_3^2,
\quad v_{14}^2=v_{04}^2=b_4^2.
\]
Similarly, it can be verified that all 2-faces of $Q_2$ have zero volumes.
So condition (1) is verified.

Now to verify condition (2). Fix vertex 0 as the base point for both $Q_1$ and $Q_2$,
then the corresponding matrices of $Q_1$ and $Q_2$ (as in (\ref{equation_simplex_matrix})) are
\begin{equation}
A=
\begin{pmatrix} 
a_1^2 & -a_1a_2 & -a_1a_3 & 0 \\ 
-a_1a_2 & a_2^2 & -a_2a_3 & 0 \\
-a_1a_3 & -a_2a_3 & a_3^2 & 0 \\
0 & 0 & 0 & 0
\end{pmatrix}
\quad\text{and}\quad
B=
\begin{pmatrix} 
0 & 0 & 0 & 0 \\
0 & b_2^2 & -b_2b_3 & -b_2b_4 \\
0 & -b_2b_3 & b_3^2 & -b_3b_4 \\
0 & -b_2b_4 & -b_3b_4 & b_4^2 
\end{pmatrix}
,
\end{equation}
with $a_1+a_2+a_3=0$ and $b_2+b_3+b_4=0$.
Then the corresponding matrix of $Q$ is $tA+\frac{1}{t}B$, which we denote by $C$.
Its determinant can be expressed as a sum of signed products
such that each summand is the product of a minor of $tA$ and the 
``complement'' minor of $\frac{1}{t}B$.
As $\det(A)=\det(B)=0$, so there are no $t^4$ and $\frac{1}{t^4}$ terms.
The $t^2$ term is the product of the upper left principal minor (of order 3) of $tA$ and $\frac{b_4^2}{t}$,
which is $-4a_1^2a_2^2a_3^2b_4^2t^2$.
Similarly the $\frac{1}{t^2}$ term is $-4a_1^2b_2^2b_3^2b_4^2\frac{1}{t^2}$.
The constant term, with only two non-zero summands, is 
\[-2\det\begin{pmatrix} a_1^2 & -a_1a_2 \\ -a_1a_3 & -a_2a_3 \end{pmatrix}
\cdot\det\begin{pmatrix} -b_2b_3 & -b_2b_4 \\ -b_3b_4 & b_4^2 \end{pmatrix}
=-8a_1^2a_2a_3b_2b_3b_4^2.
\]
So
\begin{align*}
\det(C)
&=-4a_1^2a_2^2a_3^2b_4^2t^2-4a_1^2b_2^2b_3^2b_4^2\frac{1}{t^2}
-8a_1^2a_2a_3b_2b_3b_4^2 \\
&=-4a_1^2b_4^2(a_2a_3t+b_2b_3\frac{1}{t})^2,
\end{align*}
which is negative unless $t^2=-\frac{b_2b_3}{a_2a_3}$.
So except for $t^2=-\frac{b_2b_3}{a_2a_3}$,
$Q$ is non-degenerate and can be realized in $\mathbb{R}^{p,4-p}$ for some $p$.

We next check that if $Q$ contains only Euclidean 2-faces.
By applying Lemma~\ref{lemma_matrix_2_by_2} to the 2-faces of $Q_1$ and $Q_2$
(see Figure~\ref{figure_simplex_4}), we need to exclude a few cases.
E.g., for the 2-faces with vertices 0, 2, and 3, 
we need to exclude the case of $a_2b_3=a_3b_2$, and so on.
Once those cases are removed, as $Q$ contains only Euclidean 2-faces, so $p\ge 2$.
For the symmetric matrix $C$, all the eigenvalues are real,
and the numbers of positive and negative eigenvalues are $p$ and $4-p$ respectively.
Since $\det(C)$ is the product of the eigenvalues of $C$, and $\det(C)$ is negative, so $p$ is 3. 
Thus $Q$ can be realized in $\mathbb{R}^{3,1}$, and this verifies condition (2)
and proves Theorem~\ref{theorem_flexible_pseudo_Euclidean_n_4}.

\begin{theorem}
\emph{(Theorem~\ref{theorem_flexible_pseudo_Euclidean_n_4})}
For $n=4$ in $\mathbb{R}^{3,1}$,
their exists a continuous family of non-congruent $4$-simplices $Q$,
such that  all the 2-faces of $Q$ are in Euclidean planes,
and all the areas remain constant during the deformation.
\end{theorem}

Notice that $\det(C)$ is not a constant over $t$, so the volume of $Q$ is not a constant over $t$,
proving Theorem~\ref{theorem_flexible_volume_non_constant} for $n=4$.

\begin{corollary}
\label{corollary_flexible_volume_non_constant_4}
\emph{(Theorem~\ref{theorem_flexible_volume_non_constant}, for $n=4$)}
For $n=4$ in $\mathbb{R}^{3,1}$,
their exists a continuous family of non-congruent $4$-simplices $Q$,
such that the areas of all the 2-faces of $Q$ remain constant during the deformation,
but the volume of $Q$ does not remain constant.
\end{corollary}


\section{Equivalent questions of Question~\ref{question_simplex_rigid}}

To prove Theorem~\ref{theorem_flexible_pseudo_Euclidean},
we first reformulate Question~\ref{question_simplex_rigid} (about an $n$-simplex $Q$ in $\mathbb{R}^n$)
to an equivalent form about the \emph{dual} of $Q$,
and in the process it also explains that how we come to formulate 
Theorem~\ref{theorem_flexible_pseudo_Euclidean} in the first place.

For an $n$-simplex $Q$ in $\mathbb{R}^n$, without loss of generality,
we assume that the origin $O$ is the centroid of the vertices of $Q$ 
(or simply call it the centroid of $Q$).
This assumption is not essential for $Q$ because the volume of any face of $Q$ is invariant under translation, 
but becomes more important when the dual of $Q$ is introduced next.
For $1\le i\le n+1$, denote $F_i$ an $(n-1)$-face of $Q$,
and $F_{ij}$ ($i\ne j$) the $(n-2)$-face on the intersection of $F_i$ and $F_j$.

Let an $n$-simplex $P$ be a \emph{dual} of $Q$ 
\begin{equation}
\label{equation_dual_Euclidean}
P=\{y \in \mathbb{R}^n: x\cdot y \le c \quad\text{for all $x\in Q$}\}
\end{equation}
for some $c>0$. When $c=1$, $P$ is the \emph{polar dual} $Q^{\ast}$ of $Q$.
Here $c$ may vary when $Q$ varies, but its value is not important and can be adjusted later.
Let the vertices of $P$ be $P_i$ and $v_i=\overrightarrow{OP_i}$,
then $v_i$ is an outward normal vector to $Q$ at $F_i$.
As $O$ is the centroid of $Q$, it can be shown that the length $\norm{v_i}$ 
is proportional to the $(n-1)$-volume of $F_i$.
The Minkowski relation for the facet volumes of $Q$ states that $\sum v_i=0$, so $O$ is also the centroid of $P$.

As $O$ is the centroid of $P$,
it is well known that the area of the triangle $OP_iP_j$ is proportional to the $(n-2)$-volume of $F_{ij}$,
e.g., see Lee~\cite[Theorem 19]{Lee:stress}.
So Question~\ref{question_simplex_rigid} is equivalent to the following question about $P$.

\begin{question}
\label{question_simplex_triangle_rigid}
For any $n$-simplex $P$ in $\mathbb{R}^n$ ($n\ge 4$) with the centroid fixed at the origin $O$, 
if a continuous deformation preserves the areas of all triangles $OP_iP_j$,
then is it necessarily a rigid motion?
\end{question}

Note that the square of the area of the triangle $OP_iP_j$, multiplied by a factor 4, 
is the determinant of the $2\times 2$ matrix
\begin{equation}
\begin{pmatrix} v_i^2 & v_i\cdot v_j \\ v_i\cdot v_j & v_j^2 \end{pmatrix},
\end{equation}
where $v_i\cdot v_j$ is the inner product of $v_i$ and $v_j$ in $\mathbb{R}^n$.
Conversely, this matrix, without knowing more details of $v_i$ and $v_j$, 
determines the triangle $OP_iP_j$ up to congruence.

If we treat $v_i$ as a column $n$-vector and let $V=(v_1,\dots, v_{n+1})$ be an $n\times (n+1)$ matrix,
then $U:=V^TV=(v_i^T v_j)_{1\le i,j\le n+1}$ is an $(n+1)\times (n+1)$ positive semi-definite matrix.
Notationwise $v_i^T v_j$ and $v_i\cdot v_j$ are the same thing in the \emph{Euclidean} space,
so $U=(v_i\cdot v_j)_{1\le i,j\le n+1}$, and all the principal minors of order 2 of $U$
are the squares of the areas of triangles $OP_iP_j$, multiplied by 4.

Denote $\mathbf{1}$ the  $(n+1)$-vector $(1,\dots,1)^T$.
Because the $v_i$'s span the entire $\mathbb{R}^n$ and $\sum v_i=0$,
so $V$ has rank $n$ and $V\cdot\mathbf{1}=0$,
and thus $U=V^TV$ has rank $n$ and contains $\mathbf{1}$ in the null space. 
As $U$ determines $P$ up to congruence,
so again, Question~\ref{question_simplex_triangle_rigid} is equivalent to the following question.

\begin{definition}
\label{definition_matrix_positive_definite}
Let $\mathcal{F}$ (resp. $\mathcal{F}_0$) be the set of $(n+1)\times (n+1)$ positive semi-definite matrices 
(resp. with rank $n$) with the vector $\mathbf{1}$ in the null space.
\end{definition}

\begin{question}
\label{question_simplex_matrix_rigid}
For $n\ge 4$, let $U\in\mathcal{F}_0$ be an $(n+1)\times (n+1)$ positive semi-definite matrix 
with rank $n$ and with the vector $\mathbf{1}$ in the null space.
If a continuous change of $U$ in $\mathcal{F}_0$ preserves all the principal minors of order 2 of $U$,
then does it necessarily preserve $U$ as well?
\end{question}

As mentioned earlier, we do not solve Question~\ref{question_simplex_rigid}, 
which is equivalent to Question~\ref{question_simplex_triangle_rigid}
and Question~\ref{question_simplex_matrix_rigid}.
But for a variant of Question~\ref{question_simplex_matrix_rigid}
where the restriction of positive semi-definite matrices is replaced by a weaker notion of 
\emph{symmetric matrices}, we construct counterexamples for $n\ge 5$.

\begin{definition}
\label{definition_matrix_symmetric}
Let $\mathcal{U}$ (resp. $\mathcal{U}_0)$ be the set of $(n+1)\times (n+1)$ symmetric matrices 
(resp. with rank $n$) with the vector $\mathbf{1}$ in the null space
and whose diagonal entries and principal minors of order 2 are all non-negative (resp. positive).
\end{definition}

\begin{theorem}
\label{theorem_symmetric_matrix}
\emph{(Main Theorem 3)}
For $n\ge 5$, there exists a continuous family of non-identical 
$(n+1)\times (n+1)$ symmetric matrices $U\in\mathcal{U}_0$ with rank $n$ 
and with the vector $\mathbf{1}$ in the null space,
such that all the principal minors of order 2 are positive and remain constant.
\end{theorem}

It is easy to check that for any $U\in\mathcal{U}$,
if its principal minors of order 2 are all positive, 
then it automatically implies that the diagonal entries are also all positive.
Though Theorem~\ref{theorem_symmetric_matrix} does not fully resolve 
Question~\ref{question_simplex_matrix_rigid},
it is a somewhat surprising result, because for an $(n+1)\times (n+1)$ 
symmetric matrix in $\mathcal{U}_0$, the degrees of freedom are $\binom{n+1}{2}$, 
the same as the number of the principal minors of order 2.
Our proof of Theorem~\ref{theorem_symmetric_matrix} is constructive, 
and may possibly be used to search for counterexamples to
Question~\ref{question_simplex_matrix_rigid} as well.



\section{Proof of Theorem~\ref{theorem_symmetric_matrix}}

\begin{definition}
Let $\mathcal{D}$ be the set of the $(n+1)\times (n+1)$ symmetric matrices in $\mathcal{U}$
with the vector $\mathbf{1}$ in the null space and whose principal minors of order 2 are all 0.
\end{definition}

\begin{example}
\label{example_symmetric_matrix}
Let $\alpha=(a_1,\dots, a_{n+1})^T$ be a column $(n+1)$-vetor that satisfies $\alpha^T\cdot\mathbf{1}=0$,
and $A:=\alpha\alpha^T$ be an $(n+1)\times (n+1)$ symmetric matrix.
Then $A\cdot\mathbf{1}=0$ and $\rank(A)=1$, and $A\in\mathcal{D}$.
\end{example}

Lemma~\ref{lemma_matrix_2_by_2} immediately leads to the following result.

\begin{lemma}
\label{lemma_matrix_non_negative}
If $A, B\in\mathcal{D}$, then for $tA+\frac{1}{t}B$ with $t>0$,
it is in $\mathcal{U}$ and all the principal minors of order 2 are non-negative constants.
\end{lemma}

Note that unlike $\mathcal{F}$ or $\mathcal{U}$, 
the set $\mathcal{D}$ is not closed under the plus operation of matrices.
To further prove Theorem~\ref{theorem_symmetric_matrix}, 
we only need to improve Lemma~\ref{lemma_matrix_non_negative}
by finding suitable $A, B\in\mathcal{D}$ and $t>0$
such that $tA+\frac{1}{t}B$ is in $\mathcal{U}_0$ with rank $n$,
then $tA+\frac{1}{t}B$ satisfies the conditions of Theorem~\ref{theorem_symmetric_matrix}.
This is what we plan to do next.

\begin{remark}
\label{remark_matrix_rank}
As $\mathcal{D}$ is closed under multiplication by a positive factor,
so to prove Theorem~\ref{theorem_symmetric_matrix}, 
in fact we can just ignore $t$ and only search for $A, B\in\mathcal{D}$ 
such that $A+B$ is in $\mathcal{U}_0$ with rank $n$.
As $\rank(A)+\rank(B)\ge \rank(A+B)$,
so at least one of the ranks of $A$ and $B$, say $\rank(A)$, should be at least $\lceil n/2 \rceil$,
the least integer greater than or equal to $n/2$.
In Example~\ref{example_symmetric_matrix}, we have $\rank(A)=1$, whose rank is too low for our purpose.
For $n=4$, it can be verified (with the proof skipped) 
that any non-zero $5\times 5$ matrix in $\mathcal{D}$ has rank 1, 
less than $\lceil n/2 \rceil=2$, so this method does not work for $n=4$,
and we leave the details to the interested reader.
\end{remark}


\subsection{An outline of the proof}

If $A\in\mathcal{D}$, then $A$ can be written as $D^TCD$ (not necessarily unique),
where $D=\diag(d_1,\dots, d_{n+1})$ (so $D^T=D$)
and $C$ is an $(n+1)\times (n+1)$ symmetric matrix 
with only $\pm 1$ entries and all the diagonal entries are 1.
As $A$ also satisfies $A\cdot\mathbf{1}=0$, so $D^TCD\cdot\mathbf{1}=0$. 
In Example~\ref{example_symmetric_matrix}, the corresponding $C$ of $A=\alpha\alpha^T$
has all entries as 1.

\begin{remark}
\label{remark_zero_entry}
If all the $d_i$'s in $D$ above have at least two zero entries, assume $d_1=d_2=0$.
Then no matter how to choose $B\in\mathcal{D}$, by Lemma~\ref{lemma_matrix_2_by_2}, 
in $A+B$ the determinant of the upper-left $2\times 2$ submatrix (a principal minor of order 2) is 0,
so $A+B$ is \emph{not} in $\mathcal{U}_0$.
So in order to find suitable $A,B\in\mathcal{D}$ to prove Theorem~\ref{theorem_symmetric_matrix},
the $d_i$'s can afford to have at most one zero entry.
\end{remark}

\begin{definition}
Let $\mathcal{H}$ be the finite set of all $(n+1)\times (n+1)$ symmetric matrices 
with only $\pm 1$ entries and all the diagonal entries are 1.
\end{definition}

We remark that unlike $\mathcal{F}$, $\mathcal{U}$ or $\mathcal{D}$, a matrix in $\mathcal{H}$
generally does not contain $\mathbf{1}$ in the null space.
Before we provide some concrete examples of $A,B\in\mathcal{D}$ 
to prove Theorem~\ref{theorem_symmetric_matrix}, 
here we outline our general approach to constructing such a matrix $A\in\mathcal{D}$
(it can be verified that \emph{all} $A\in\mathcal{D}$ can be constructed in the following way,
but we won't need to use this property):
First find a matrix $C\in\mathcal{H}$ such that $C$ does not have a full rank of $n+1$.
In the null space of $C$, pick a vector $d=(d_1,\dots,d_{n+1})^T$ and denote $D=\diag(d)$.
Then $CD\cdot\mathbf{1}=Cd=0$ and thus $D^TCD\cdot\mathbf{1}=0$.  
Let $A=D^TCD$, then $A\cdot\mathbf{1}=0$ and thus $A\in\mathcal{D}$. 

Two things to keep in mind, 
(1) by Remark~\ref{remark_matrix_rank} we want $\rank(A)$ to be at least $\lceil n/2\rceil$, 
and (2) by Remark~\ref{remark_zero_entry} we want the $d_i$'s to contain at most one zero entry.
Now we follow with some concrete examples to construct $A\in\mathcal{D}$.
The treatment is different for $n$ odd and $n$ even.



\subsection{For $n$ odd}

For an odd $n\ge 5$, let $n=2m-1$. So $n+1=2m$ and $m\ge 3$. 
Let $C_0$ be a $m\times m$ matrix whose diagonal entries are all 1 
and all off-diagonal entries are $-1$. 
As $m\ge 3$, $C_0$ has a full rank of $m$ (but not for $m=2$ where $\rank(C_0)=1$).
Now for a $2m\times 2m$ matrix $C_{2m}$, we cut it into some $2\times 2$ blocks (there are $m\times m$ of them),
fill the $m$ diagonal $2\times 2$ blocks with $\begin{pmatrix} 1 & 1 \\ 1 & 1 \end{pmatrix}$,
and the rest blocks with $\begin{pmatrix} -1 & -1 \\ -1 & -1 \end{pmatrix}$.
As $m\ge 3$ and $\rank(C_0)=m$, so $\rank(C_{2m})=m$.
Thus the null space of $C_{2m}$ has dimension $m$ and the form 
\begin{equation}
\label{equation_null_space_C1}
d=(d_1,d_2,\dots,d_{2m-1},d_{2m})^T=(-d_2,d_2,\dots,-d_{2m},d_{2m})^T.
\end{equation}
Let $D_{2m}=\diag(d)$ and $A_{2m}=D_{2m}^TC_{2m}D_{2m}$,
then as $A_{2m}\cdot\mathbf{1}=D_{2m}^T(C_{2m}d)=0$,
we have $A_{2m}\in\mathcal{D}$.

If $d_2,\dots,d_{2m}$ are non-zero, whose values will be chosen later, then $\rank(A_{2m})=\rank(C_{2m})=m$.
So the null space of $A_{2m}$ has dimension $m$ and the form 
\begin{equation}
\label{equation_null_space_A}
(a_2,a_2,\dots,a_{2m},a_{2m})^T.
\end{equation}

For a $2m\times 2m$ matrix $C'_{2m}$, let it be obtained 
by moving the last row and last column of $C_{2m}$ to the first row and first column.
So we still have $\rank(C'_{2m})=m$, and the null space of $C'_{2m}$ has the form 
\begin{equation}
\label{equation_null_space_C2}
d'=(d'_1,d'_2,d'_3,\dots,d'_{2m})^T
=(-d'_{2m},d'_2,-d'_2,\dots,d'_{2m})^T.
\end{equation}
Let $D'_{2m}=\diag(d')$ and $B_{2m}={D'_{2m}}^TC'_{2m}D'_{2m}$, then $B_{2m}\in\mathcal{D}$.
Similarly, if $d'_2,\dots, d'_{2m}$ are non-zero, whose values will also be chosen later,
then $\rank(B_{2m})=\rank(C'_{2m})=m$, and the null space of $B_{2m}$ has the form 
\begin{equation}
\label{equation_null_space_B}
(b_{2m},b_2,b_2,\dots, b_{2m})^T.
\end{equation}

For any vector in the \emph{common} null space of $A_{2m}$ and $B_{2m}$,
by (\ref{equation_null_space_A}) its $(2i-1)$-th and $2i$-th terms are equal,
and by (\ref{equation_null_space_B}) its $2i$-th and $(2i+1)$-th terms are equal,
so all the terms are equal.
So the common null space of $A_{2m}$ and $B_{2m}$ is 1-dimensional that contains $\mathbf{1}$.
The columns $\alpha_i$ $(1\le i\le 2m)$ of $A_{2m}$ satisfy
\begin{equation}
\label{equation_matrix_A_column_even}
\alpha_1=-\alpha_2, \dots, \alpha_{2m-1}=-\alpha_{2m},
\end{equation}
and the columns $\beta_i$  $(1\le i\le 2m)$ of $B_{2m}$ satisfy
\begin{equation}
\label{equation_matrix_B_column_even}
\beta_2=-\beta_3, \dots, \beta_{2m}=-\beta_1.
\end{equation}
As the common null space of $A_{2m}$ and $B_{2m}$ is 1-dimensional,
so the \emph{row} vectors of $A_{2m}$ and $B_{2m}$ 
span a codimension 1 subspace in $\mathbb{R}^{2m}$, namely, $(2m-1)$-dimensional.
As $A_{2m}$ and $B_{2m}$ are symmetric matrices,
so by (\ref{equation_matrix_A_column_even}) and (\ref{equation_matrix_B_column_even}),
those $2m$ \emph{column} vectors 
\begin{equation}
\label{equation_column_vectors}
\alpha_1, \dots, \alpha_{2m-1}, \beta_2, \dots, \beta_{2m}
\end{equation}
also span a $(2m-1)$-dimensional subspace in $\mathbb{R}^{2m}$.

We next show that for appropriate non-zero $d_i$ and $d'_i$,
the null space of $A_{2m}+B_{2m}$ is also 1-dimensional.

\begin{lemma}
\label{lemma_A_plus_B_U0_even}
For $n=2m-1$ with $m\ge 3$, with proper choice of non-zero $d_i$ and $d'_i$,
the matrix $A_{2m}+B_{2m}$ is in $\mathcal{U}_0$ with rank $2m-1$.
\end{lemma}

\begin{proof}
The $2m$ columns of $A_{2m}+B_{2m}$ are $\alpha_i+\beta_i$.
To prove that $A_{2m}+B_{2m}$ has rank $2m-1$ for proper choice of non-zero $d_i$ and $d'_i$,
our goal is to show that the $2m-1$ vectors $\alpha_i+\beta_i$ $(1\le i\le 2m-1)$ are linearly independent,
which is equivalent to show that for $1\le k\le 2m-1$, 
$\sum_{i=1}^k (\alpha_i+\beta_i)$ are linearly independent.
Notice that $\sum_{i=1}^k (\alpha_i+\beta_i)$,
by (\ref{equation_matrix_A_column_even}) and (\ref{equation_matrix_B_column_even}) and induction on $k$, 
is $\alpha_k+\beta_1$ if $k$ is odd, and is $\beta_k+\beta_1$ if $k$ is even.
As $\beta_1=-\beta_{2m}$ (\ref{equation_matrix_B_column_even}),
so $\sum_{i=1}^k (\alpha_i+\beta_i)$ is $\alpha_k-\beta_{2m}$ if $k$ is odd,
and is $\beta_k-\beta_{2m}$ if $k$ is even.
Thus it is also equivalent to prove that 
$\alpha_1, \dots, \alpha_{2m-1}, \beta_2, \dots, \beta_{2m}$ are \emph{affinely independent}.

As by (\ref{equation_column_vectors}) the $2m$ vectors $\alpha_{2i-1}$ and $\beta_{2i}$
span a $(2m-1)$-dimensional subspace in $\mathbb{R}^{2m}$,
so up to a constant factor, there is a unique linear dependence among the $2m$ vectors.
For the particular case of $d_i=d'_i=(-1)^i$, by symmetry it is not hard to check that
\[\sum \alpha_{2i-1}+\sum \beta_{2i}=0.
\]
As the sum of the coefficients is not zero, so the $2m$ vectors are affinely independent.
By the arguments above, 
this means that for almost all non-zero $d_i$ and $d'_i$,
$A_{2m}+B_{2m}$ has rank $2m-1$.

The $2\times 2$ diagonal submatrices of $A_{2m}$ and $B_{2m}$ 
are $\begin{pmatrix} d_i^2 & \pm d_id_j  \\ \pm d_id_j & d_j^2 \end{pmatrix}$
and $\begin{pmatrix} {d'_i}^2 & \pm d'_id'_j  \\ \pm d'_id'_j & {d'_j}^2 \end{pmatrix}$.
We can further adjust $d_i$ and $d'_i$ such that for any $i\ne j$,
we have $|d_id'_j|\ne |d_jd'_i|$.
So for $A_{2m}+B_{2m}$, by Lemma~\ref{lemma_matrix_2_by_2}, 
all the principal minors of order 2 are positive,
and thus $A_{2m}+B_{2m}$ is in $\mathcal{U}_0$ with rank $2m-1$.
\end{proof}


\subsection{For $n$ even}

For an even $n\ge 6$, let $n=2m$. Let $A_{2m}=D_{2m}^TC_{2m}D_{2m}$ be as before
with proper choice of non-zero $d_i$ in (\ref{equation_null_space_C1}),
and $A_{2m+1}$ be a $(2m+1)\times (2m+1)$ symmetric matrix whose upper-left 
$2m\times 2m$ submatrix is $A_{2m}$ and the other entries are 0. 
Namely, $A_{2m+1}=\begin{pmatrix} A_{2m} & 0  \\ 0 & 0 \end{pmatrix}$.
By (\ref{equation_null_space_A}), we have $\rank(A_{2m+1})=m$.
So the null space of $A_{2m+1}$ has dimension $m+1$ and the form
\begin{equation}
\label{equation_null_space_A_2}
(a_2,a_2,\dots,a_{2m},a_{2m},a_{2m+1})^T.
\end{equation}

Similarly, let $B_{2m+1}$ be a $(2m+1)\times (2m+1)$ symmetric matrix whose lower-right 
$2m\times 2m$ submatrix is $A_{2m}$ (\emph{not} $B_{2m}$ as before) and the other entries are 0.
Namely, $B_{2m+1}=\begin{pmatrix} 0 & 0 \\ 0 & A_{2m}  \end{pmatrix}$.
So $\rank(B_{2m+1})=m$, and the null space of $B_{2m+1}$ has dimension $m+1$ and the form
\begin{equation}
\label{equation_null_space_B_2}
(b_1,b_2,b_2,\dots,b_{2m},b_{2m})^T.
\end{equation}

For any vector in the common null space of $A_{2m+1}$ and $B_{2m+1}$,
by (\ref{equation_null_space_A_2}) its $(2i-1)$-th and $2i$-th terms are equal,
and by (\ref{equation_null_space_B_2}) its $2i$-th and $(2i+1)$-th terms are equal,
so all the terms are equal.
So the common null space of $A_{2m+1}$ and $B_{2m+1}$ is 1-dimensional that contains $\mathbf{1}$.
The non-zero columns $\alpha_i$ $(1\le i\le 2m)$ of $A_{2m+1}$ satisfy 
\begin{equation}
\label{equation_matrix_A_column_odd}
\alpha_1=-\alpha_2, \dots, \alpha_{2m-1}=-\alpha_{2m},
\end{equation}
and the non-zero columns $\beta_i$ $(2\le i\le 2m+1)$ of $B_{2m+1}$ satisfy 
\begin{equation}
\label{equation_matrix_B_column_odd}
\beta_2=-\beta_3, \dots, \beta_{2m}=-\beta_{2m+1}.
\end{equation}
As the common null space of $A_{2m+1}$ and $B_{2m+1}$ is 1-dimensional,
so the row vectors of $A_{2m+1}$ and $B_{2m+1}$ 
span a codimension 1 subspace in $\mathbb{R}^{2m+1}$, namely, $2m$-dimensional.
As $A_{2m+1}$ and $B_{2m+1}$ are symmetric matrices,
so by (\ref{equation_matrix_A_column_odd}) and (\ref{equation_matrix_B_column_odd}),
those $2m$ non-zero column vectors $\alpha_1$, \dots, $\alpha_{2m-1}$, $\beta_2$, \dots, $\beta_{2m}$
also span a $2m$-dimensional subspace in $\mathbb{R}^{2m+1}$, thus are linearly independent.

We next show that the null space of $A_{2m+1}+B_{2m+1}$ is also 1-dimensional.

\begin{lemma}
\label{lemma_A_plus_B_U0_odd}
For $n=2m$ with $m\ge 3$, with proper choice of non-zero $d_i$ $(1\le i\le 2m)$,
the matrix $A_{2m+1}+B_{2m+1}$ is in $\mathcal{U}_0$ with rank $2m$.
\end{lemma}

\begin{proof}
The $2m+1$ columns of $A_{2m+1}+B_{2m+1}$ are $\alpha_i+\beta_i$, $1\le i\le 2m+1$,
with $\beta_1=\alpha_{2m+1}=0$.
Notice that $\sum_{i=1}^k (\alpha_i+\beta_i)$,
by (\ref{equation_matrix_A_column_odd}) and (\ref{equation_matrix_B_column_odd}) and induction on $k$, 
is $\alpha_k$ if $k$ is odd, and is $\beta_k$ is $k$ is even.
As $\alpha_1$, \dots, $\alpha_{2m-1}$, $\beta_2$, \dots, $\beta_{2m}$ are linearly independent, 
so $\sum_{i=1}^k (\alpha_i+\beta_i)$ $(1\le k\le 2m)$ are linearly independent,
then $\alpha_i+\beta_i$ $(1\le i\le 2m)$ are linearly independent,
and thus $A_{2m+1}+B_{2m+1}$ has rank $2m$.

With proper choice of non-zero $d_i$ $(1\le i\le 2m)$ in (\ref{equation_null_space_C1}),
by Lemma~\ref{lemma_matrix_2_by_2},
in $A_{2m+1}+B_{2m+1}$ all the principal minors of order 2 are positive.
So $A_{2m+1}+B_{2m+1}$ is in $\mathcal{U}_0$ with rank $2m$.
\end{proof}


The following observation will be useful for the proof of
Theorem~\ref{theorem_flexible_volume_non_constant}.

\begin{remark}
\label{remark_constant_determinant}
For this particular construction of $A_{2m+1}$ and $B_{2m+1}$ above,
observe that $\rank(A_{2m+1})=\rank(B_{2m+1})=m$.
For the matrix $tA_{2m+1}+\frac{1}{t}B_{2m+1}$,
its determinant (and similarly for any of its minors) can be expressed as a sum of signed products 
such that each summand is the product of a minor of $tA_{2m+1}$ (of any order, including 0 and $2m+1$)
and the ``complement'' minor of $\frac{1}{t}B_{2m+1}$.
As any minor of $A_{2m+1}$ and $B_{2m+1}$ of order higher than $m$ is 0,
so the determinant of any $2m\times 2m$ submatrix of $tA_{2m+1}+\frac{1}{t}B_{2m+1}$ is a constant
(because only when $k=m$ the $t^k\times 1/t^{2m-k}$ term is non-vanishing).
This property is analogous to Lemma~\ref{lemma_matrix_2_by_2}.
On the other hand, it is not so for $n=2m-1$. Namely for the matrix $tA_{2m}+\frac{1}{t}B_{2m}$
(not to be confused with the matrix $tA_{2m+1}+\frac{1}{t}B_{2m+1}$),
the determinant of any $(2m-1)\times (2m-1)$ submatrix is not a non-zero constant, in fact, 
it is a linear combination of $t$ and $\frac{1}{t}$ without a constant term.
\end{remark}


\subsection{Summary of the proof}

\begin{theorem}
\emph{(Theorem~\ref{theorem_symmetric_matrix})}
For $n\ge 5$, there exists a continuous family of non-identical 
$(n+1)\times (n+1)$ symmetric matrices $U\in\mathcal{U}_0$ with rank $n$ 
and with the vector $\mathbf{1}$ in the null space,
such that all the principal minors of order 2 are positive and remain constant.
\end{theorem}

\begin{proof}
By Remark~\ref{remark_matrix_rank},
for $n$ odd it is proved by Lemma~\ref{lemma_A_plus_B_U0_even},
and for $n$ even it is proved by Lemma~\ref{lemma_A_plus_B_U0_odd}.
\end{proof}



\section{Relations to Question~\ref{question_simplex_matrix_rigid}
and Question~\ref{question_simplex_triangle_rigid}}

Recall that $\mathcal{F}_0$ is the set of $(n+1)\times (n+1)$ positive semi-definite matrices 
with rank $n$ and with the vector $\mathbf{1}$ in the null space 
(Definition~\ref{definition_matrix_positive_definite}).
With Theorem~\ref{theorem_symmetric_matrix} proved, 
it remains to see if the \emph{positive semi-definite} nature of the matrices in $\mathcal{F}_0$ 
(as compared to the symmetric matrices in $\mathcal{U}_0$)
is a barrier to solve Question~\ref{question_simplex_matrix_rigid},
or our method can also be applied to provide counterexamples to Question~\ref{question_simplex_matrix_rigid},
namely, if we can find $A,B\in\mathcal{D}$ such that $A+B\in\mathcal{F}_0$ 
(see Remark~\ref{remark_matrix_rank}).

\begin{remark}
If such $A,B\in\mathcal{D}$ indeed exist such that $A+B\in\mathcal{F}_0$,
because $\mathcal{F}_0$ in an \emph{open} subset of $\mathcal{U}$ 
(Definition~\ref{definition_matrix_symmetric}),
so for any small changes of $A$ and $B$ in $\mathcal{D}$ their sum $A+B$ is still in $\mathcal{F}_0$.
This also makes numerical search possible, at least in theory:
while $\mathcal{D}$ is a high (up to $n$) dimensional space,
it is fairly easy to find all the elements of $\mathcal{D}$,
so we can just sample $A, B\in\mathcal{D}$ numerically, 
and then check if $A+B$ is in $\mathcal{F}_0$. 
\end{remark}

If the numerical search turns out successful for some $n$, then it provides counterexamples to 
Question~\ref{question_simplex_matrix_rigid}, as well as to 
Question~\ref{question_simplex_triangle_rigid} and Question~\ref{question_simplex_rigid}
as they are all equivalent.
In general, no matter the numerical search is successful or not,
in the following we provide counterexamples to a variant of Question~\ref{question_simplex_triangle_rigid}
where $\mathbb{R}^n$ is replaced by a pseudo-Euclidean space $\mathbb{R}^{p,n-p}$ 
for some unspecified $p\ge 2$.

\begin{theorem}
\label{theorem_simplex_triangle_non_rigid}
For any $n\ge 5$, in $\mathbb{R}^{p,n-p}$ for some unspecified $p\ge 2$,
there exists a continuous family of non-congruent $n$-simplices $P$
with the centroid fixed at the origin $O$, such that all triangles $OP_iP_j$ are in Euclidean planes
and the areas remain constant during the deformation.
\end{theorem}

\begin{proof}
By Lemma~\ref{lemma_A_plus_B_U0_even} for $n$ odd 
and Lemma~\ref{lemma_A_plus_B_U0_odd} for $n$ even,
we can find $(n+1)\times (n+1)$ symmetric matrices $A,B\in\mathcal{D}$ 
such that $A+B$ is in $\mathcal{U}_0$ with rank $n$.
Denote $A+B$ by $U$, and the upper-left $n\times n$ submatrix of $U$ by $U_0$.
As $U_0$ is symmetric, it can be decomposed to $V_0^TDV_0$,
where $V_0$ is an $n\times n$ matrix
and $D$ is an $n\times n$ diagonal matrix with the diagonal entries being
1's followed by $-1$'s and 0's (we will rule out the 0's next).

Let $V$ be an $n\times (n+1)$ matrix such that its $i$-th column $v_i$ $(1\le i\le n)$
is the $i$-th column of $V_0$, and its $(n+1)$-th column $v_{n+1}$ satisfies 
$v_{n+1}=-\sum_{i=1}^n v_i$.
So $\sum_{i=1}^{n+1} v_i=0$, then $V\cdot\mathbf{1}=0$ and thus $V^TDV\cdot\mathbf{1}=0$. 
As both $U$ and $V^TDV$ share the same upper-left $n\times n$ submatrix $U_0=V_0^TDV_0$, 
and $U\cdot\mathbf{1}=V^TDV\cdot\mathbf{1}=0$,
so $U$ and $V^TDV$ have the same upper $n\times (n+1)$ matrix.
As both $U$ and $V^TDV$ are symmetric, 
so $U$ and $V^TDV$ also have the same left $(n+1)\times n$ submatrix;
apply $U\cdot\mathbf{1}=V^TDV\cdot\mathbf{1}=0$ to the last columns again, then $U=V^TDV$.
As $U$ has rank $n$, so both $V$ and $D$ have full rank $n$.
Thus $D$'s diagonal entries contain 1's followed by $-1$'s only, with no 0's.
Assume the number of 1's and $-1$'s are $p$ and $n-p$ respectively.

In $\mathbb{R}^{p,n-p}$, let $P$ be an $n$-simplex whose vertices $P_i$ $(1\le i\le n+1)$ 
satisfy $v_i=\overrightarrow{OP_i}$.
As $\sum_{i=1}^{n+1} v_i=0$, then $O$ is the centroid of $P$.
The entries of $U$ (same as $V^TDV$), $v_i^TDv_j$, can also be written as $v_i\cdot v_j$
where ``$\cdot$'' is the bilinear product in $\mathbb{R}^{p,n-p}$.
As $U\in\mathcal{U}_0$, so any $2\times 2$ diagonal submatrix
$\begin{pmatrix} v_i^2 & v_i\cdot v_j \\ v_i\cdot v_j & v_j^2 \end{pmatrix}$
of $U$ is positive definite.
Therefore any triangle $OP_iP_j$ spans a two-dimensional Euclidean plane.

For any $t$ in a small neighborhood of 1, we still have $tA+\frac{1}{t}B\in\mathcal{U}_0$.
So we can construct a continuous family of non-congruent simplices $P$ 
in $\mathbb{R}^{p,n-p}$ from $tA+\frac{1}{t}B$.
By Lemma~\ref{lemma_matrix_non_negative}, the areas of all triangles $OP_iP_j$ 
remain constant when $t$ varies. This finishes the proof.
\end{proof}

\begin{remark}
In Theorem~\ref{theorem_simplex_triangle_non_rigid}, for a fixed $n$, the $p$ may not be unique.
Even when $A$ and $B$ are fixed, $tA+\frac{1}{t}B$ and $A+B$ 
may potentially correspond to different $p$'s when $t$ moves away from 1.
This is because $tA+\frac{1}{t}B$ may have a rank lower than $n$ for some $t=t_0$,
then for $t<t_0$ and $t>t_0$ in a small neighborhood of $t_0$,
they may correspond to different $p$'s.
\end{remark}

We have the following interesting observation,
which will be further addressed in the next section.

\begin{remark}
\label{remark_constant_volume_P}
For $n=2m$, in Lemma~\ref{lemma_A_plus_B_U0_odd},
for the \emph{particular} construction of $A_{2m+1}$ and $B_{2m+1}$,
we have $\rank(A_{2m+1})=\rank(B_{2m+1})=m$.
Construct a continuous family of non-congruent $2m$-simplices $P$ from $tA_{2m+1}+\frac{1}{t}B_{2m+1}$,
and use the formula (\ref{equation_volume_pseudo_Euclidean}) to compute the volumes
of the $2m$-simplices formed by $O$ and the $2m+1$ facets of $P$,
then by Remark~\ref{remark_constant_determinant}, 
the volumes of those $2m$-simplices remain constant when $t$ varies.
As those simplices add up to $P$, so $V_{2m}(P)$ also remains constant when $t$ varies 
(but it is not so for $n=2m-1$).
\end{remark}

We use Theorem~\ref{theorem_simplex_triangle_non_rigid} 
to prove Theorem~\ref{theorem_flexible_pseudo_Euclidean} next.



\section{Proofs of Theorem~\ref{theorem_flexible_pseudo_Euclidean}
and \ref{theorem_flexible_volume_non_constant}}

In $\mathbb{R}^{p,n-p}$, let $P$ be an $n$-simplex with the centroid at the origin $O$
(but for now the triangles $OP_iP_j$ need not all be in Euclidean planes 
as required in Theorem~\ref{theorem_simplex_triangle_non_rigid}).
Let $Q$ be a \emph{dual} of $P$ 
\[Q=\{y \in \mathbb{R}^{p,n-p}: x\cdot y \le c \quad\text{for all $x\in P$}\}
\]
for some $c>0$, see also (\ref{equation_dual_Euclidean}).
When $c=1$, $Q$ is the \emph{polar dual} $P^{\ast}$ of $P$.
Let $v_i=\overrightarrow{OP_i}$. Just like in the Euclidean case, 
it can be verified that $Q$ is an $n$-simplex in a finite region with the centroid at the origin $O$,
and $v_i$ is a normal vector at the $(n-1)$-face $F_i$ of $Q$. 
But unlike in $\mathbb{R}^n$, here $v_i$ is an \emph{outward} normal vector to $Q$ at $F_i$ if $v_i^2>0$,
an \emph{inward} normal vector to $Q$ at $F_i$ if $v_i^2<0$, or \emph{parallel} to $F_i$ if $v_i^2=0$.
However, this does not affect our results.

Now for $P$ also let all triangles $OP_iP_j$ be in Euclidean planes.
This insures that the area of $OP_iP_j$ is non-zero.
In this case of $P$, we have $v_i^2>0$ for all $i$, 
so $v_i$'s are still all outward normal vectors to $Q$ at $F_i$'s.
In $\mathbb{R}^n$, 
the volume of any face of $Q$ (of any dimension) can be computed from the information of $P$
using a simple formula (e.g. see \cite[Theorem 19]{Lee:stress});
the formula only uses the fact that $O$ is the centroid of $P$
and can be applied to $\mathbb{R}^{p,n-p}$ as well.
As a direct result, it shows that 
for the $(n-2)$-face $F_{ij}$ of $Q$ on the intersection of $F_i$ and $F_j$,
the volume $V_{n-2}(F_{ij})$ is proportional to the area of the triangle $OP_iP_j$
\begin{equation}
\label{simplex_dual_volume_n_minus_2}
V_{n-2}(F_{ij})=\frac{c_2V_2(OP_iP_j)}{V_n(P)}\cdot c^{n-2},
\end{equation}
where $c_2$ is a constant that only depends on $n$.
Similarly we also have
\begin{equation}
\label{simplex_dual_volume_n}
V_n(Q)=\frac{c_0}{V_n(P)}\cdot c^n,
\end{equation}
where $c_0$ is also a constant that only depends on $n$.

%\begin{lemma}
%\label{lemma_volume_proportional_pseudo_Euclidean}
%In $\mathbb{R}^{p,n-p}$, let $Q$ be a dual of $P$, 
%and $F_{ij}$ be the $(n-2)$-face of $Q$ on the intersection of the $(n-1)$-faces $F_i$ and $F_j$,
%then $V_{n-2}(F_{ij})$ is proportional to the area of the triangle $OP_iP_j$.
%\end{lemma}

Now we are ready to prove Theorem~\ref{theorem_flexible_pseudo_Euclidean}.

\begin{theorem}
\emph{(Theorem~\ref{theorem_flexible_pseudo_Euclidean})}
For any $n\ge 5$, in $\mathbb{R}^{p,n-p}$ for some unspecified $p\ge 2$,
their exists a continuous family of non-congruent $n$-simplices $Q$,
such that  all the dihedral angles are Euclidean angles,
and the $(n-2)$-volumes of all the $(n-2)$-faces of $Q$ remain constant during the deformation.
\end{theorem}

\begin{proof}
By Theorem~\ref{theorem_simplex_triangle_non_rigid},
there exists a continuous family of non-congruent $n$-simplices $P$ 
with the centroid fixed at the origin $O$
such that all triangles $OP_iP_j$ are in Euclidean planes and the areas remain constant.
Let $Q$ be a dual of $P$ for some factor $c$ ($c$ may vary when $P$ varies),
then the dihedral angles of $Q$ are all Euclidean angles.
With the proper scale of $c$ for $Q$, by (\ref{simplex_dual_volume_n_minus_2}),
the $(n-2)$-volumes of all the $(n-2)$-faces of $Q$ can be adjusted to remain constant during the deformation.
This finishes the proof.
\end{proof}

For any $n$-simplex $Q$ with fixed volumes of codimension 2 faces in a continuous defermation
(in $\mathbb{R}^n$ or $\mathbb{R}^{p,n-p}$, and including $n=4$ as well),
in the spirit of the bellows conjecture \cite{Sabitov:invariance,ConnellySabitovWalz}, or for other heuristics,
one may ask that if the volume of $Q$ is also necessarily preserved during the deformation.
For $n=4$ in $\mathbb{R}^{3,1}$, we answer this question negatively in
Corollary~\ref{corollary_flexible_volume_non_constant_4}.
For $n\ge 5$, we give a loose discussion on this topic based on the counterexamples
we constructed in Theorem~\ref{theorem_flexible_pseudo_Euclidean}.

By (\ref{simplex_dual_volume_n_minus_2}) and (\ref{simplex_dual_volume_n}),
because both $V_{n-2}(F_{ij})$ and $V_2(OP_iP_j)$ remain constant during the deformation,
we can show that for the volume of $Q$ to remain constant,
\emph{if and only if} both $V_n(P)$ and $c$ remains constant during the deformation;
and if so, we can set $c=1$ and let $Q$ be the \emph{polar} dual $P^{\ast}$ of $P$.
Regarding the volume of $P$, it is addressed in Remark~\ref{remark_constant_volume_P}
that for the \emph{particular} counterexamples of $P$ we constructed in $\mathbb{R}^{p,n-p}$
in Theorem~\ref{theorem_simplex_triangle_non_rigid} (from Lemma~\ref{lemma_A_plus_B_U0_odd}),
the volume of $P$ \emph{does} remain constant for $n$ even but not for $n$ odd.
Thus, for the particular counterexamples of $Q$ we constructed in $\mathbb{R}^{p,n-p}$
in Theorem~\ref{theorem_flexible_pseudo_Euclidean},
the volume of $Q$ remains constant for $n$ even but not for $n$ odd.
So combining with Corollary~\ref{corollary_flexible_volume_non_constant_4} for $n=4$,
we have the following.

\begin{corollary}
\emph{(Theorem~\ref{theorem_flexible_volume_non_constant})}
If $n=4$ or $n\ge 5$ and $n$ is odd, in $\mathbb{R}^{p,n-p}$ for some unspecified $p\ge 2$,
their exists a continuous family of non-congruent $n$-simplices $Q$,
such that the $(n-2)$-volumes of all the $(n-2)$-faces of $Q$ remain constant during the deformation,
but the volume of $Q$ does not remain constant.
\end{corollary}

To sum up, it seems natural for us to narrow the following (open) question to $n$ even and $n\ge 6$.

\begin{question}
For $n$ even $(n\ge 6)$ in $\mathbb{R}^n$ or $\mathbb{R}^{p,n-p}$, 
if there exists a continuous family of $n$-simplices $Q$ 
such that the $(n-2)$-volumes of all the $(n-2)$-faces remain constant during the deformation,
then does the volume of $Q$ also necessarily remain constant?
\end{question}




%%%%%%%%%%%%%%%%%%%%%%%%%%%%%%%%%%%%%%%%%%%
%%%%%%%%%%%%%%%%%%%%%%%%%%%%%%%%%%%%%%%%%%%

%\noindent
%{\bf Acknowledgements:}
%%\begin{acknowledgements}
%%\end{acknowledgements}

{\footnotesize
% BibTeX users please use one of
\bibliographystyle{abbrv}  %choose between plain, abbrv, acm
%\bibliographystyle{plain}
%\bibliographystyle{acm}
\bibliography{codimension2_arxiv}   % name your BibTeX data base

% Non-BibTeX users please use
%\begin{thebibliography}{}
%
% and use \bibitem to create references. Consult the Instructions
% for authors for reference list style.
%
% etc
%\end{thebibliography}
}

\end{document}
% end of file template.tex







