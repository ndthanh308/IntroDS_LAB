%%
%% Copyright 2007-2020 Elsevier Ltd
%%
%% This is a modified file from the 'Elsarticle Bundle', which is available under the conditions of the LaTeX Project Public
%% License, either version 1.2 of this license or any later version.
%% ---------------------------------------------

\documentclass[11pt]{elsarticle}
%\documentclass[draft, 11pt]{elsarticle}
%% For including figures, graphicx.sty has been loaded in
%% elsarticle.cls. If you prefer to use the old commands
%% please give \usepackage{epsfig}

%% The amssymb package provides various useful mathematical symbols
\usepackage{amssymb}
%% The amsthm package provides extended theorem environments
\usepackage{amsthm}

\usepackage{mathtools}
\mathtoolsset{showonlyrefs}

%margin settings
\usepackage[left=3.0cm, right=3.0cm, top=3.4cm, bottom=3cm, footskip=0.5cm]{geometry}
\usepackage{amsmath, amsfonts, amsthm, amssymb}
\usepackage{verbatim}
\usepackage{hyperref}
\usepackage{color}
\usepackage{graphicx}
\usepackage{enumerate}
\usepackage{eurosym}
\usepackage{hyperref}
\usepackage{subcaption}
\usepackage{ulem}
\usepackage{mathtools} \mathtoolsset{showonlyrefs}

\makeatletter
\def\ps@pprintTitle{%
  \let\@oddhead\@empty
  \let\@evenhead\@empty
  \let\@oddfoot\@empty
  \let\@evenfoot\@oddfoot
}
\makeatother


\usepackage{mathrsfs} 

\usepackage[textsize=tiny]{todonotes}
\newcommand{\jack}[2][]
{\todo[color=green, #1]{#2}}
\newcommand{\lionel}[2][]
{\todo[color=yellow, #1]{#2}}
\newcommand{\jf}[2][]
{\todo[color=magenta, #1]{#2}}
\newcommand{\geraldine}[2][]
{\todo[color=cyan, #1]{#2}}
\newcommand{\smail}[2][]
{\todo[color=orange, #1]{#2}}

\numberwithin{equation}{section}
\linespread{1.1}

\newtheorem{theorem}{Theorem}[section]
\newtheorem{corollary}[theorem]{Corollary}
\newtheorem{lemma}[theorem]{Lemma}
\newtheorem{proposition}[theorem]{Proposition}
\newtheorem{Proof}{Proof}

\theoremstyle{definition}
\newtheorem{definition}[theorem]{Definition}
\newtheorem{remark}[theorem]{Remark}
\newtheorem{assumption}[theorem]{Assumption}
\newtheorem{sassumption}[theorem]{Standing Assumption}
\newtheorem{example}[theorem]{Example}
\numberwithin{equation}{section}


\DeclareMathOperator*{\argmin}{arg\,min}
\DeclareMathOperator*{\argmax}{arg\,max}
\newcommand{\ind}{1\hspace{-2.1mm}{1}}
\newcommand{\RR}{\mathbb{R}}
\newcommand{\PP}{\mathbb{P}}
\newcommand{\FF}{\mathbb{F}}
\newcommand{\GG}{\mathbb{G}}
\newcommand{\BS}{\mathrm{BS}}
\newcommand{\D}{\mathrm{d}}
\newcommand{\Ir}{\mathbf{I}}
\newcommand{\NN}{\mathbb{N}}
\newcommand{\Ff}{\mathcal{F}}
\newcommand{\Kkk}{\mathcal{K}}
\newcommand{\Ttt}{\mathcal{T}}
\newcommand{\Jj}{\mathcal{I}}
\newcommand{\cN}{\mathcal{N}}
\newcommand{\cW}{\mathcal{W}}
\newcommand{\cM}{\mathcal{M}}
\newcommand{\cE}{\mathcal{E}}
\newcommand{\cF}{\mathcal{F}}
\newcommand{\cZ}{\mathcal{Z}}
\newcommand{\cV}{\mathcal{V}}
\newcommand{\cG}{\mathcal{G}}
\newcommand{\cH}{\mathcal{H}}
\newcommand{\cA}{\mathcal{A}}
\newcommand{\cB}{\mathcal{B}}
\newcommand{\cL}{\mathcal{L}}
\newcommand{\Oo}{\mathcal{O}}
\newcommand{\EE}{\mathbb{E}}
\newcommand{\OneN}{\{1,\hdots,N\}}
\newcommand{\tOnetU}{\{0 \hdots t_\circ-1\}}
\newcommand{\VV}{\mathbb{V}}
\newcommand{\bOne}{\mathbf{1}}
\newcommand{\eps}{\varepsilon}
\newcommand{\teps}{\widetilde{\eps}}
\newcommand{\Wf}{\mathfrak{W}}
\newcommand{\vecc}{\mathrm{vec}}
\newcommand{\Diag}{\mathrm{Diag}}
\newcommand{\Id}{\mathrm{I}}
\newcommand{\ee}{\mathfrak{e}}
\newcommand{\bb}{\mathfrak{b}}
\newcommand{\dd}{\mathfrak{d}}
\newcommand{\mfB}{\mathfrak{B}}
\newcommand{\Zz}{{Z}}
\newcommand{\Vv}{{V}}
\newcommand{\Pp}{{P}}
\newcommand{\Ww}{{W}}
\newcommand{\Cc}{{C}}
\newcommand{\Yy}{{Y}}
\newcommand{\Hh}{{H}}
\newcommand{\Bb}{{B}}
\newcommand{\Zbf}{\mathbf{Z}}
\newcommand{\xX}{\mathcal{X}}
\newcommand{\Aa}{{A}}
\newcommand{\Ll}{{\Lambda}}
\newcommand{\llambda}{\boldsymbol{\lambda}}
\newcommand{\psib}{{\psi}}
\newcommand{\varphib}{{\varphi}}
\newcommand{\chib}{{\chi}}

\newcommand{\E}{\mathrm{e}}
\def\blue#1{\textcolor{blue}{#1}}
\def\red#1{\textcolor{red}{#1}}
\def\green#1{\textcolor{green}{#1}}
\newcommand{\imagesource}[1]{{\footnotesize Source: #1}}
\newcommand{\indep}{\perp \!\!\! \perp}
\newcommand{\LGD}{\mathrm{LGD}}
\newcommand{\EAD}{\mathrm{EAD}}
\newcommand{\PD}{\mathrm{PD}}
\newcommand{\cPD}{\mathrm{cPD}}
\newcommand{\EL}{\mathrm{EL}}
\newcommand{\UL}{\mathrm{UL}}
\newcommand{\ES}{\mathrm{ES}}
\newcommand{\VaR}{\mathrm{VaR}}

%%% LONG EQUATIONS

\newcommand{\titre}{Propagation of carbon taxes in credit portfolio through macroeconomic factors}

%%%%%%%%%%%%%%%%%%%%%%%%%%%%%%%%%%%%%%%%%%%%%%%
\usepackage{fancyhdr}
\pagestyle{fancy}
\fancyhf{}
\rhead{\titre}
\lhead{}
\rfoot{Page \thepage}



\begin{document}


\begin{frontmatter}

\title{\titre}
\date{\today}

\author[1]{G\'{e}raldine Bouveret}
\author[2]{Jean-Fran\c{c}ois Chassagneux}
\author[3]{Smail Ibbou}
\author[4,5]{Antoine Jacquier}
\author[2,3,4]{Lionel Sopgoui}

\address[1]{Climate Risks Research Department, Rimm Sustainability}
\address[2]{Laboratoire de Probabilités, Statistique et Modélisation (LPSM), Université Paris Cité}
\address[3]{Validation des modèles, Direction des risques, BPCE S.A.}
\address[4]{Department of Mathematics, Imperial College London}
\address[5]{Alan Turing Institute}

\journal{SIAM Journal on Financial Mathematics (SIFIN)}

\begin{abstract}
We study how the introduction of \textcolor{black}{carbon pricing} in a closed economy propagates in a credit portfolio and precisely describe how \textcolor{black}{carbon price dynamics} affects credit risk measures such as probability of default, expected, and unexpected losses.
We adapt a stochastic multisectoral model to take into account \textcolor{black}{the greenhouse gases (GHG) emissions costs of} both sectoral firms' production and consumption, as well as sectoral household's consumption. \textcolor{black}{GHG emissions costs are the product of carbon prices, provided by the NGFS transition scenarios, and of GHG emissions.} For each sector, our model yields the sensitivity of firms' production and households' consumption to carbon price and the relationships between sectors.
It also allows us to analyze the short-term effects of \textcolor{black}{the introduction of a carbon price} as opposed to standard Integrated Assessment Models (such as REMIND), which are not only deterministic
but also only capture long-term trends of climate transition policy.
Finally, we use a Discounted Cash Flows methodology to compute firms' values which we then combine with a structural credit risk model to describe how the introduction \textcolor{black}{of a carbon price} impacts credit risk measures.
We obtain that the introduction of \textcolor{black}{a carbon price} distorts the distribution of the firm’s value, increases banking fees charged to clients
(materialized by the level of provisions computed from the expected loss), and reduces banks' profitability (translated by the value of the economic capital calculated from the unexpected loss). In addition, the randomness introduced in our model provides extra flexibility to take into account uncertainties on productivity by sector and on the different transition scenarios.
We also compute the sensitivities of the credit risk measures with respect to changes in \textcolor{black}{the carbon price},
yielding further criteria for a more accurate assessment of climate transition risk in a credit portfolio.
This work provides a preliminary methodology to calculate
the evolution of credit risk measures of a multisectoral credit portfolio, starting from a given climate transition scenario described by \textcolor{black}{a carbon price.}

\end{abstract}

\begin{keyword}
%% keywords here, in the form: keyword \sep keyword
Structural credit risk \sep Climate risk \sep Macroeconomic modelling \sep Transition risk \sep Carbon price \sep Firm valuation \sep Stochastic modeling
\end{keyword}

\end{frontmatter}

\end{document}

\endinput
%%
%% End of file main.tex'.
