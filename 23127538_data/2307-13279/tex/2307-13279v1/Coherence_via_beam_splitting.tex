\documentclass[aps,pra,eqsecnum,twocolumn]{revtex4}
\usepackage{graphicx}
\usepackage{bm}
\usepackage{amsmath,amssymb,amsthm,dsfont,bm}
\usepackage{color}
\usepackage{wasysym}
\usepackage{ulem}
\usepackage{appendix}
%\usepackage[applemac]{inputenc}
\usepackage[dvipsnames]{xcolor}
\newcommand{\AL}[1]{\textcolor{red}{#1}}
\newcommand{\LA}[1]{\textcolor{blue}{#1}}
\usepackage[english]{babel} %Español 
\usepackage[utf8]{inputenc} %Para poder poner tildes
\usepackage{hyperref}
\usepackage{cancel}

\begin{document}

\title{Coherence via reiterated beam splitting}
\author{Guillermo D\'iez, Laura Ares and Alfredo Luis}
\affiliation{Departamento de \'Optica, Facultad de Ciencias
F\'{\i}sicas, Universidad Complutense, 28040 Madrid, Spain}
\date{\today}

\begin{abstract}
Beam splitters are not-free operations regarding quantum coherence,  which is the most versatile resource for quantum technologies. As a consequence, they can create coherence from both coherent and incoherent states. We investigate the increase in coherence produced by cascades of beam splitters. To this end, we construct two different configurations of beam splitters and analyze different sequences of input states. 
\end{abstract}
\maketitle


\section{Introduction}

The beam splitter is a basic device in optics conceived to create coherence in the classical realm and is therefore a fundamental element in interferometry. Its role is no less relevant in quantum optics \cite{MW95,LS95}. In this real, it allows to easily perform fundamental quantum state transformations, such as entanglement generation, state measurements like tomography, or certifications of nonclassicality. For this reason, it is present in most of the photonic implementations of the quantum applications \cite{logic,BSampling,DNM21}. Thus we investigate beam splitting as a coherence making process in quantum optics.

In recent years, we have witnessed a rapid growth of the interest in quantum coherence as the cornerstone of the quantum theory \cite{BCP14,SP17,CG19}. 
Current formulations of quantum coherence,  mainly focused on the applicability of this cornerstone feature \cite{SP17, AW16}, are far from the standard theory of coherence in quantum optics, based on the works of Glauber and Sudarshan \cite{RG63a,RG63b,ECGS63}.  The novel approach is definitely more versatile given the ongoing technological perspective \cite{JS23}. In this regard, the question of how much coherence can be introduced by beam splitting is non-trivial, timely and interesting. In the context of quantum correlations, this question has been usually focused on the generation of entanglement, which excludes the option of coherent input states \cite{MSK02}. However,  coherent states have proven to be a useful tool for quantum tasks \cite{TF16}, and quantum coherence goes beyond entanglement, so we broaden the point of view for the analysis.


%One expects that quantities should be rooted in physical intuition as much as in mathematical criteria. The question is whether recent measures of coherence have such roots in physical practice.

This work is grounded on the study of an individual beam splitter as coherence maker introduced  in \cite{AL23}. Here, we extend this analysis to two different combinations or cascades of beam splitters. This is specially  moved by the rather unsettling possibility of unlimited coherence growth with an unlimited increase in the number of beam splitters. We specially focus on the mode decomposition of the total coherence, trying to relate the overall coherence to the coherence of each output mode.

\bigskip

Throughout, we consider lossless beam splitters and compute coherence in the photon-number basis, $|n \rangle$. As the suitable coherence measure, we utilize the $l_1$-norm of coherence \cite{BCP14,SP17},
\begin{equation}
\label{Cl1}
 \mathcal{C} = \sum_{n,n^\prime}| \langle n | \rho |n^\prime \rangle | -1,
\end{equation}
where $\rho$ is the density matrix. It simplifies for pure states, $ |\psi \rangle = \sum_{n} c_n | n \rangle $, in the form 
\begin{equation}
   \quad \mathcal{C} = \left ( \sum_{n} |c_n |\right )^2 -1,
\end{equation}
where here $n$ represents any collection of natural numbers needed to label the photon-number basis in a multi-mode scenario. 

\bigskip

The state acting as the input of the series of beam splitters is considered such that only one input mode is populated, while all the other input modes are vacuum states. This is the traditional way in which beam splitting is used to manage coherence. As the field state in the populated mode, we examine pure and mixed states, both coherent and incoherent. These are, Glauber coherent states $|\alpha \rangle$ as pure and partially coherent inputs, number states $| n\rangle$ as pure and incoherent, and finally phase-averaged and thermal states, both mixed and incoherent inputs. With these choices, we examine whether the increase of coherence depends on the initial coherence conveyed by the input field state, its quantumness, or purity. 

\section{Coherent-state input}

First, we consider a Glauber coherent state $|\alpha \rangle$ in the only populated mode, so that the input state becomes 
\begin{equation}
  |\alpha ,0,\ldots,0 \rangle.  
\end{equation}
Let us compute the coherence for the input state. The coefficients in the single-mode photon-number basis have modulus
\begin{equation}
\label{cn}
| c_n | = \sqrt{\frac{\bar{n}^n}{n!}} e^{-\bar{n}/2},
\end{equation}
where $\bar{n}$ is the mean number of photons $\bar{n} = |\alpha |^2$. The coherence of the input state, which we shall call $\mathcal{C} (\bar{n}, N=0 )$, where $N$ will represent later the number of beam splitters, is 
\begin{equation}
\label{Cc}
    \mathcal{C} (\bar{n}, N=0 ) = e^{-\bar{n}}\left ( \sum_{n=0}^\infty \sqrt{\frac{\bar{n}^n}{n!}}  \right )^2 -1 .
\end{equation}
In order to obtain the intuition of analytical solutions we utilize the following approximation: For large enough $\bar{n}$, the sum in Eq. \ref{Cc} can be approximated by an integral, and the Poissonian statistics in Eq. \ref{cn} by a Gaussian, 
\begin{equation}
\label{Posi}
| c_n | \simeq \frac{1}{(2 \pi \bar{n})^{1/4}} e^{-(n-\bar{n})^2/(4 \bar{n})}.
\end{equation}
Under this assumption, the coherence results in
\begin{equation}
\label{Ccoheapp}
    \mathcal{C} (\bar{n}, N=0 ) \simeq 2\sqrt{2 \pi \bar{n}}-1 .
\end{equation}
In Fig. \ref{cohe}, we plot $\mathcal{C} (\bar{n}, N=0 )$ for a single coherent state as a function of the total mean number of photons $\bar{n}$ in blue line, and its approximation in Eq. (\ref{Ccoheapp}) in orange line. 

% Figure environment removed{}

%% Figure environment removed{}

It is worth noting the increase of coherence with the mean number of photons, without upper limit \cite{ZSLF16}. This behaviour is not contemplated in the Glauber-Sudarshan quantum-optical theory of coherence, nor in the classical theory of coherence, where coherence is independent of the field intensity. This independence was, by construction, reflecting the idea that coherence is a measure of the wave-like quality independently of the amount of light. 

\bigskip

Now, we can address the coherence after a cascade of $N$ arbitrary lossless beam splitters.  We take advantage of the known properties of coherent states and beam splitters to note that the output state will be a product of coherent states in all the $N+1$ output modes, this is 
\begin{equation}
     |\beta_0 \rangle | \beta_1 \rangle \ldots |\beta_N \rangle .
\end{equation}
Thanks to this factorization, the total coherence at the output can be expressed in the form
\begin{equation}
\label{CT}
 \mathcal{C} (\bar{n}, N) =  \Pi_{j=0}^N\left [ \mathcal{C} (\bar{n}_j, 0)+1 \right ]-1,
 \end{equation}
 where $\mathcal{C} (\bar{n}_j, 0)$ is the coherence of a single-mode coherent state in Eq. (\ref{Cc}), and $\bar{n}_j = |\beta_j|^2$ is the mean number of photons of the coherent state in the corresponding output mode, with 
 \begin{equation}
 \label{ec}
     \bar{n} = \sum_{j=0}^N \bar{n}_j ,
 \end{equation}
 by energy conservation. 

\bigskip

In order to characterize the set of beam splitters as a coherence maker entity, we look for the optimum choice of parameters for each individual beam splitter, so that $\mathcal{C} (\bar{n}, N)$ in Eq. (\ref{CT}) becomes maximum. Thanks to the factorized form in Eq. (\ref{CT}), we can easily demonstrate the following theorem: The maximum of $\mathcal{C} (\bar{n}, N)$ in Eq. (\ref{CT}) occurs when the mean photon numbers of all output modes are equal, 
 \begin{equation}
 \label{mpne}
     \bar{n}_j = \frac{\bar{n}}{N+1} ,
 \end{equation}
and then, 
\begin{equation}
\label{TC}
    \mathcal{C}_{\rm max} (\bar{n}, N) = \left [   \mathcal{C} \left ( \frac{\bar{n}}{N+1} , 0 \right ) +1 \right ]^{N+1} - 1 .
\end{equation}
After the Gaussian approximation in Eq. (\ref{Ccoheapp}), which we have shown that works even for small mean number of photons, the coherence becomes
\begin{equation}
\label{CTapp}
\mathcal{C}_{\rm max} (\bar{n}, N ) \simeq \left ( 8 \pi \frac{\bar{n}}{N+1} \right )^{\frac{N+1}{2}}-1 .
\end{equation}

\bigskip

Let us address the demonstration of the theorem. For large mean photon numbers, where Eq. (\ref{Ccoheapp}) holds, a simple proof is available since in such a case the state-dependent term in Eq. (\ref{CT}) is proportional to 
\begin{equation}
 \Pi_{j=0}^N \sqrt{\bar{n}_j} .
\end{equation}
Taking into account energy conservation on Eq. (\ref{ec}), the maximum of this factor via Lagrange multipliers leads to a unique extreme point reached when Eq. (\ref{mpne}) holds, being clearly a maximum.

\bigskip

In the general case, the theorem can be demonstrated without approximations by {\it reductio ad absurdum}. Let us assume that the maximum coherence $\mathcal{C}_{\rm max}$ holds when at least two of the mean photons numbers are not equal, say $\bar{n}_0 \neq \bar{n}_1$ without loss of generality. We can focus on the contribution of these modes to the total coherence 
\begin{equation}
\label{fact}
\left [ \mathcal{C} (\bar{n}_0, 0)+1 \right ] \left [\mathcal{C} (\bar{n}_1, 0)+1 \right ] .
\end{equation}
We can show that this contribution can be increased by an equal distribution of the $\bar{n}_{0,1}=\bar{n}_0 + \bar{n}_1$ photons between the two output modes without affecting the rest, contradicting the assumed maximum. This is because the two-mode contribution is essentially the output coherence of a single beam splitter illuminated by coherent light, which has been shown numerically to be maximized when the average number of photons is equally distributed among the outputs \cite{AL23}. 

Intuitively, the theorem can be understood from the fact that there is full symmetry under the exchange between transmission and reflection so their equality must be an extreme. Such extreme must be maximum since the spreading of the photon number distribution is clearly largest at a 50\% splitting, and the largest the spreading the largest the coherence. 

In Fig. \ref{coheN}, we plot the gain in coherence caused by the reiteration of splitting,  $\mathcal{C} = \mathcal{C}_{\rm max} (\bar{n}, N ) /\mathcal{C}_{\rm max} (\bar{n}, 0 )$ as a function of the number of beam splitters $N$ and the mean number of photons $\bar{n}$. A clear increase of coherence can be appreciated both regarding the increase of the number of photons and the number of beam splitters.


% Figure environment removed{}

\bigskip

Until this point, we have shown that the energy has to be equally distributed between the output modes in order to maximize coherence. Thus, we can apply this consideration to any combination of beam splitters in order to optimize the resulting coherence. Here, we examine whether this optimal distribution can be achieved by two different arrays of beam splitters as shown in Figs. \ref{cascada1} and \ref{cascada2}).

% Figure environment removed{}
% Figure environment removed{}

For the cascade 1 in Fig. \ref{cascada1}, the optimum balanced photon splitting can be achieved simply by considering that all beam splitters are balanced, having equal 50 \% reflectance and transmittance. Concerning the cascade 2 in Fig. \ref{cascada2}, the beam splitters can no longer be 50 \%. Instead, expressing the transmission and reflection coefficients of each beam splitter as, assumed real without loss of generality,
\begin{equation}
    t_j = \cos \theta_j, \quad r_j = \sin \theta_j, 
\end{equation}
for $j=1, \ldots, N$, where the beam splitters are numbered from left to right in Fig. \ref{cascada2}. 

For the last beam splitter on the right we must have equal 50 \% reflectance and transmittance $t_N^2 = r_N^2$. For the second to last beam splitter, that is $j=N-1$, the transmittance and reflectance must be in the ratio of 2 to 1, since by transmission we must feed two modes with the same intensity, while by reflection just one, also with the same intensity, that is $t_{N-1}^2 = 2 r_{N-1}^2$. The general term of this series can be easily constructed to give the optimum configuration as 
\begin{equation}
 t_j^2 = \left ( N+1-j \right ) r_j^2  , 
\end{equation}
and then 
\begin{equation}
    \theta_j = \arcsin \left (\frac{1}{\sqrt{N+2-j}} \right ) ,
\end{equation}
where $ N+1-j$ is the number of modes at the right of the $j$-th beam splitter in cascade 2. There is an actual balance between transmission and reflection which has to take into account the number of output modes after the whole system. 

\bigskip

\section{Number-state input}

In this  section we address the case of a highly nonclassical pure input state, the number state $|n \rangle$, in one of the input modes, being the rest of the input modes again in the vacuum state
\begin{equation}
  |n ,0,\ldots,0 \rangle = \frac{1}{\sqrt{n!}}a^{\dagger n}_0 |0 ,0,\ldots,0 \rangle ,
\end{equation}
where $a_0$ is the complex-amplitude operator for the excited input mode. The output modes can be computed by the linear relation established by any cascade of beam splitters between the input and output modes, expressing $a_0$ in terms of the output modes to get that the output state is
\begin{equation}
\label{ss}
  \frac{1}{\sqrt{n!}} \left ( \sum_{j=0}^N \tau_j a_j^{\dagger} \right )^n |0,0,\ldots,0 \rangle,
\end{equation}
where $\tau_j$ are complex coefficients that must satisfy the following equality required to preserve commutation relations
\begin{equation}
\label{coe}
    \sum_{j=0}^N |\tau_j|^2 = 1 .
\end{equation}
Without loss of generality, from now on we assume real coefficients $\tau_j$. After some simple algebraic manipulations via multinomial coefficients, we get that the output state expressed in the number basis is 
\begin{equation}
\label{ntau}
   |n, \tau \rangle =\sqrt{n!}  \sum_{\{m \}} \frac{\tau_0^{m_0}\tau_1^{m_1}\ldots \tau_N^{m_N}}{\sqrt{m_0 !m_1! \ldots m_N!}} |\{m \} \rangle ,
\end{equation}
where $\{m \}$ labels the collection of $N+1$ non negative integers respecting the photon-number conservation law
\begin{equation}
\label{pncl}
    m_0 + m_1 +\ldots +m_N = n .
\end{equation}
It is worth noting that these states are SU($N$+1) coherent states with a nice internal structure in terms of some kind of nesting of SU(2) coherent states \cite{KN00,LP96}. With this the coherence becomes in this case
\begin{equation}
\label{Cnt}
    \mathcal{C} (n,N) = n! \left ( \sum_{\{m \}} \frac{\tau_0^{m_0}\tau_1^{m_1}\ldots \tau_N^{m_N}}{\sqrt{m_0 !m_1! \ldots m_N!}} \right )^2-1 .
\end{equation}
\bigskip


\bigskip

Now, we address the optimization of the coherence (\ref{Cnt}) when the coefficients $\tau_j$ are varied. We proceed by following the same {\it reductio ad absurdum} strategy assuming that the maximum of oEq. (\ref{Cnt}) is reached when at least two of the coefficients $\tau_j$ are nor equal, say $\tau_0 \neq \tau_1$ without loss of generality. We focus on the factor inside the parenthesis to split the contribution of the photons within the  two assumed unequal modes
\begin{equation}
   \sum_{\{m \}} \frac{\tau_0^{m_0}\tau_1^{m_1}\ldots \tau_N^{m_N}}{\sqrt{m_0 !m_1! \ldots m_N!}} = \sum_{k=0}^n \sum_{m=0}^k \frac{\tau_0^{m}\tau_1^{k-m}}{\sqrt{m! (k-m)!}} c_k ,
\end{equation}
where $c_k$ are the contributions for the rest of the modes that will share $n-k$ photons. For each $k$, the factor 
\begin{equation}
  \sum_{m=0}^k \frac{\tau_0^{m}\tau_1^{k-m}}{\sqrt{m! (k-m)!}} 
\end{equation}
corresponds to the coherence of the two-mode output state when a single beam splitter is illuminated by a number state $|k \rangle$ in one of the modes, and vacuum in the other mode. We also examined this case in a previous work \cite{AL23}, showing that the maximum value for such factor holds for balanced splitting $\tau_0 = \tau_1$ for all $k$. 

This result can be deduced again intuitively from the fact that the factor is symmetric under the exchange of $\tau_0$ and $\tau_1$, so that $\tau_0 = \tau_1$ must be an extreme. That such extreme is maximum follows from the behavior of the binomial, so that the spreading of the photon number distribution is clearly largest when $\tau_0 = \tau_1$, and the largest the spreading the largest the coherence. 

Therefore, the maximum coherence that can be obtained under these circumstances holds when all the $\tau_j$ coefficients are equal, this is 
\begin{equation}
\label{acae}
    \tau_j = \frac{1}{\sqrt{N+1}} ,
\end{equation}
so that 
\begin{equation}
\label{nmax}
   |n, \tau \rangle_{\rm max} =\sqrt{\frac{n!}{(N+1)^n}}  \sum_{\{m \}} \frac{1}{\sqrt{m_0 !m_1! \ldots m_N!}} |\{m \} \rangle ,
\end{equation}
and
\begin{equation}
\label{CSn}
    \mathcal{C}_{\rm max} (n,N)= \frac{n!}{(N+1)^n} \left ( \sum_{\{m \}} \frac{1}{\sqrt{m_0 !m_1! \ldots m_N!}} \right )^2-1 .
\end{equation}

\bigskip

In Fig. \ref{numero}, we show the behavior of $\mathcal{C}_{\rm max} (n,N)$ as a function of $n$ for  a single beam splitter $N=1$ (blue line) and a cascade of two beam splitters $N=2$ (orange line), showing the increase with the number of beam splitters and the number of photons.  


% Figure environment removed{}

\bigskip

It might be interesting to compare $\mathcal{C}_{\rm max} (n,N)$ with the maximum coherence that might be reached in the finite-dimensional subspace that embraces the $n$ photons in $N+1$ modes, $C_{\mathrm{sup}}$. This maximum is obtained for the phase-like state 
\begin{equation}
\label{maxs}
\frac{1}{\sqrt{D(n,N)}} \sum_{\{m \}} e^{i \phi (\{m \})} | \{m \} \rangle ,
\end{equation}
where $\phi (\{m \})$ are arbitrary phases, and $D(n,N)$ is the dimension of the space spanned by the photon number states $|\{m \} \rangle$ with $ m_0 + m_1 +\ldots +m_N = n$. This dimension can be easily computed as the number of different ways of distributing $n$ identical candies between $N+1$ distinguishable children, leading to the binomial coefficient
\begin{equation}
    D (n,N)  =  \begin{pmatrix} n+N\\N \end{pmatrix} .
\end{equation}
The maximum coherence is then, maybe better say supremum coherence,
\begin{equation}
\label{CSnmax}
    \mathcal{C}_{\mathrm{sup}} (n,N)= D(n,N) -1.
\end{equation}

\bigskip

In Fig. \ref{maxsup} we plot the quotient $\mathcal{C}_{\mathrm{max}} (n,N)/\mathcal{C}_{\mathrm{sup}} (n,N)$ as a function of $n$ for a single beam splitter  $N=1$ (blue line), three beam splitters $N=3$ (orange line), and five beam splitters $N=5$ (green line), showing a global departure from the supremum for increasing $n$. 

% Figure environment removed{}

We can note that the single photon case $n=1$ reaches the supremum for all $N$ 
\begin{equation}
   C_{\mathrm{max}} (n=1,N)=C_{\mathrm{sup}} (n=1,N) .
\end{equation}

\bigskip

Furthermore, we can show that this optimum configuration is the same one reached in the preceding section where the input was a coherent state. To this end, we note that
\begin{equation}
\label{cce}
    |\alpha, 0, \ldots,0 \rangle = e^{-|\alpha |^2/2} e^{\alpha a^\dagger_0} |0, 0,\ldots, 0 \rangle .
\end{equation}
If we perform the same input-output replacement made in Eq. (\ref{ss}) we get that
\begin{equation}
    |\alpha, 0, \ldots,0 \rangle \rightarrow e^{-|\alpha |^2/2} e^{\alpha \sum_{j=0}^N \tau_j a_j^{\dagger}} |0, 0,\ldots, 0 \rangle ,
\end{equation}
that after Eqs. (\ref{coe}) and (\ref{cce}), leads to a product of coherent states in the output modes 
\begin{equation}
    |\alpha, 0, \ldots,0 \rangle \rightarrow |\tau_0 \alpha \rangle |\tau_1 \alpha \rangle \ldots |\tau_N \alpha \rangle ,
\end{equation}
and we already showed that maximum coherence is obtained exactly for the same optimum configuration in Eq. (\ref{acae}). 

\bigskip

In Fig. \ref{num/cohe} we plot the quotient $\mathcal{C}_{\mathrm{max}} (n,N)/\mathcal{C}_{\mathrm{max}} (\overline{n}=n,N)$ of the optimum coherence for number states versus the optimum coherence for coherent states of the same mean number of photons, for one beam splitter ($N=1$ blue line), three beam splitters ($N=3$ orange line)  and five beam splitters ($N=5$ green line). This shows that coherent states do much better than number states both, for increasing the number of photons as well as for increasing number of beam splitters. We can appreciate also that the diverse cases of different $N$ tend to a common limit for increasing $n$.

% Figure environment removed{}

\section{Mixed states}

After addressing the case for pure states, we examine the possibility of mixed states in the only excited input mode. More specifically, we consider incoherent mixed states, that is, diagonal in the number basis
\begin{equation}
    \rho = \sum_{n=0}^\infty p_n | n \rangle \langle n | .
\end{equation}
In particular, we focus on phase-averaged coherent states and thermal states. Their photon-number distributions are, respectively, 
\begin{equation}
    p_{\mathrm{pa},n} = \frac{\bar{n}^n}{n!}e^{-\bar{n}}, \quad p_{\mathrm{th},n} = \frac{1}{1+\bar{n}}\left ( \frac{\bar{n}}{\bar{n}+1} \right )^n .
\end{equation}

In any case, the output state will be of the form 
\begin{equation}
  \rho_\tau = \sum_{n=0}^\infty p_n | n, \tau \rangle \langle n, \tau |,
\end{equation}
where $| n, \tau \rangle$ are the states in Eq. (\ref{ntau}). Therefore, the coherence can be computed as
\begin{equation}
    \mathcal{C} \left ( p ,N \right )= \sum_{n=0}^\infty p_n \sum_{\{ m \},\{ m^\prime \}} |c^{(n)}_{\{ m \}}||c^{(n)}_{\{ m^\prime \}}|-1 ,
\end{equation}
where $c^{(n)}_{\{ m \}}$ are the photon-number coefficients in Eq. (\ref{ntau}), which can be expressed as
\begin{equation}
    \mathcal{C} \left ( p ,N \right )=\sum_{n=0}^\infty p_n \left ( \sum_{\{ m \}} |c^{(n)}_{\{ m \}}|\right )^2 -1 .
\end{equation}
Finally, resorting to the coherence of the pure photon-number case, $\mathcal{C} (n,N)$, we get
\begin{equation}
\label{aCn}
    \mathcal{C} \left ( p ,N \right ) = \sum_{n=0}^\infty p_n \mathcal{C} (n,N) = \overline{\mathcal{C} (n,N)} ,
\end{equation}
So the coherence of the output state is the average of the coherences obtained from each photon number state, which respects the convexity condition of coherence measures \cite{BCP14}. 

Finally, it is clear that the maximum coherence will be obtained when the beam splitters are arranged to give Eq. (\ref{acae}) since it does not depend on the number of photons $n$, this is 
\begin{equation}
    \mathcal{C}_{\rm max} \left ( p ,N \right ) =\overline{\mathcal{C}_{\rm max} (n,N)} ,
\end{equation}
where the average is taken with respect to the photon-number distribution $p_n$.

\bigskip

There is another approach to the mixed case in terms of the $P$ function of Glauber and Sudarshan \cite{MW95}
\begin{equation}
    \rho = \int d^2 \alpha P(\alpha) | \alpha \rangle \langle \alpha | ,
\end{equation}
where here we will always consider classical-like states with $P(\alpha)$ functions being legitimate probability distributions on the complex plane. For phase-averaged coherent and thermal states we have, respectively 
\begin{equation}
    P_{\rm pa}(\alpha ) = \delta \left ( |\alpha |^2 = \bar{n} \right ), \qquad  P_{\rm th} (\alpha ) = \frac{1}{\pi \bar{n} } e^{-|\alpha |^2/\bar{n}} .
\end{equation}
In the optimum case the output state $\rho_\tau$ will be of the form  
\begin{equation}
    \rho_\tau = \int d^2 \alpha  P(\alpha) \bigotimes_{j=0}^N \left | \frac{\alpha}{\sqrt{N+1}}  \right\rangle_j  \left\langle \frac{\alpha}{\sqrt{N+1}} \right | .
\end{equation}
Following essentially the same steps above 
\begin{equation}
    \mathcal{C}_{\rm max} \left ( P ,N\right ) = \int d^2 \alpha P(\alpha)  \mathcal{C}_{\rm max}\left ( |\alpha |^2, N \right )  ,
\end{equation}
where here $\mathcal{C}_{\rm max}\left ( |\alpha |^2, N \right )$ is the coherence in Eq. (\ref{TC}) for a pure coherent state with mean number of photons $\bar{n} = |\alpha |^2$. The final result  can be again expressed as an average
\begin{equation}
\label{aCa}
    \mathcal{C}_{\rm max} \left ( P ,N \right ) =\overline{\mathcal{C}_{\rm max} (|\alpha|^2,N)},
\end{equation}
where here the average is with respect to $P(\alpha)$.

\bigskip

As a straightforward conclusion of Eqs. (\ref{aCn}) and (\ref{aCa}) we have that the output coherence when the input is a phase-averaged coherent state, which is an incoherent state, is exactly the same as for a pure coherent input state of the same mean photon number. 

\bigskip

Regarding thermal states, we may proceed further considering a large mean number of photons so the approximation (\ref{Ccoheapp}) can be used in Eq. (\ref{TC}). In that case the coherence reads 
\begin{equation}
    \mathcal{C}_{\rm max} \left ( P_{\rm th} ,N \right ) =\frac{1}{\pi \bar{n}} \int d^2 \alpha e^{-|\alpha |^2/\bar{n}} \mathcal{C}_{\rm max} (|\alpha|^2,N) ,
\end{equation}
where recalling Eq. (\ref{CTapp}) 
\begin{equation}
    \mathcal{C}_{\rm max} (|\alpha|^2,N) \simeq  \left ( \frac{8\pi}{N+1} \right ) ^\frac{N+1}{2} |\alpha|^{N+1} -1 ,
\end{equation}
and then, 
\begin{equation}
\label{Ctapp}
    \mathcal{C}_{\rm max} \left ( P_{\rm th} ,N \right ) =\Gamma \left ( \frac{N+3}{2} \right ) \left ( 8\pi \frac{\bar{n}}{N+1} \right )^\frac{N+1}{2}  -1 ,
\end{equation}
where $\Gamma$ is the Gamma function. 

\bigskip

Comparing Eqs. (\ref{CTapp}) and (\ref{Ctapp}) it can be noted that they only differ in the presence of the $\Gamma$ pre-factor in the thermal case, which makes coherence grow much faster with $N$ than is does for coherent states of the same mean photon number. This is an effect of nonlinearity in the dependence of coherence with intensity, which, as we have already commented, is a rather novel feature in coherence theories. 

\section{Conclusions}

We have seen that coherence can grow without limit by increasing the number of beam splitters in all the scenarios considered here. These devices are usually considered rather passive, so it is interesting that they may easily increase such a valuable quantum resource.

In this coherence production, the initial coherence of the input state plays no role as demonstrated by the phase-averaged states that being incoherent give rise to the same coherence as initial coherent states. Furthermore, incoherent input thermal states produce larger coherence than coherent states of the same mean photon number. The fact that incoherent classical states may produce larger quantum coherence than coherent states or quantum states defies standard  intuition. 

Rather paradoxically, quantumness seems to play no role since strongly quantum states, such as the photon-number states, lead to a much lesser amount of coherence than classical-like states of the same mean number of photons. 

\bigskip

\noindent{\bf Acknowledgments.- }
L. A. and A. L. acknowledge financial support from project PR44/21--29926 from Santander Bank and Universidad Complutense of Madrid. 

\bigskip

\begin{thebibliography}{00}

\bibitem{MW95} 
L. Mandel and E. Wolf, {\it Optical Coherence and Quantum Optics} (Cambridge University Press, Cambridge, U.K., 1995) \textsection 4.3.3.

\bibitem{LS95}
A. Luis and L. L. S\' anchez-Soto, A quantum description of the beam splitter, \href{https://doi.org/10.1088/1355-5111/7/2/005}{Quantum Semiclass. Opt. {\bf 7}, 153--160  (1995)}. 

\bibitem{logic}
T. B. Pittman, B. C. Jacobs, and J. D. Franson, Probabilistic quantum logic operations using polarizing beam splitters,
\href{https://doi.org/10.1103/PhysRevA.64.062311}{Phys. Rev. A \textbf{64}, 062311 (2001)}.

\bibitem{BSampling}
Daniel J. Brod, Ernesto F. Galvão, Andrea Crespi, Roberto Osellame, Nicolò Spagnolo, Fabio Sciarrino, Photonic implementation of boson sampling: a review, Advanced Photonics  \textbf{1},  034001 (2019). 

\bibitem{DNM21}
D.N. Makarov,  E.S. Gusarevich,  A.A. Goshev,   K. A. Makarova, S. N. Kapustin, A. A. Kharlamova and Yu. V. Tsykareva,  Quantum entanglement and statistics of photons on a beam splitter in the form of coupled waveguides, \href{https://doi.org/10.1038/s41598-021-89838-5}{Sci. Rep. \textbf{11}, 10274 (2021)}.

\bibitem{BCP14}
T. Baumgratz, M. Cramer, and M. B. Plenio, Quantifying Coherence, \href{https://doi.org/10.1103/PhysRevLett.113.140401}{Phys. Rev. Lett. {\bf 113}, 140401 (2014)}.

\bibitem{SP17}
A. Streltsov, G. Adesso and M. B. Plenio , Colloquium: Quantum coherence as a resource, \href{https://doi.org/10.1103/RevModPhys.89.041003}{Rev. Mod. Phys. {\bf 89} 041003 (2017)}.

\bibitem{CG19}
E. Chitambar and G. Gour, Quantum resource theories,
\href{https://doi.org/10.1103/RevModPhys.91.025001}{Rev. Mod. Phys. \textbf{91}, 025001 (2019)}.

\bibitem{AW16}
A. Winter and D. Yang, Operational Resource Theory of Coherence, \href{https://doi.org/10.1103/PhysRevLett.116.120404}{Phys. Rev. Lett. \textbf{116}, 120404 (2016)}.

\bibitem{RG63a} 
R. J. Glauber, The quantum theory of optical coherence, \href{https://doi.org/10.1103/PhysRev.130.2529}{Phys. Rev. {\bf 130}, 2529--2539 (1963)}.

\bibitem{RG63b} 
R. J. Glauber, Coherent and Incoherent States of the Radiation Field, \href{https://doi.org/10.1103/PhysRev.131.2766}{Phys. Rev. {\bf 131}, 2766--2788 (1963)}.

\bibitem{ECGS63} 
E. C. G. Sudarshan, Equivalence of semiclassical and quantum mechanical descriptions of statistical light beams, \href{https://doi.org/10.1103/PhysRevLett.10.277}{Phys. Rev. Lett. {\bf 10}, 277--279 (1963)}.

\bibitem{JS23}
C. Lüders, F. Barkhausen, M. Pukrop, E. Rozas, J. Sperling, S. Schumacher, and M. A{\ss}mann, Continuous-variable quantum optics and resource theory for ultrafast semiconductor spectroscopy,\href{https://doi.org/10.48550/arXiv.2306.01550}{arXiv:2306.01550 [cond-mat.mes-hall]}. 

\bibitem{MSK02}
M. S. Kim, W. Son, V. Bužek, and P. L. Knight, Entanglement by a beam splitter: Nonclassicality as a prerequisite for entanglement, \href{https://doi.org/10.1103/PhysRevA.65.032323}{Phys. Rev. A \textbf{65}, 032323 (2002)}.

\bibitem{TF16}
T. Ferreira da Silva, G. B. Xavier, G. C. Amaral, G. P. Temporão, and J. P. von der Weid, Quantum random number generation enhanced by weak-coherent states interference, \href{https://doi.org/10.1364/OE.24.019574}{Opt. Express \textbf{24}, 19574--19580 (2016)}.

\bibitem{AL23}
L. Ares and A. Luis, Beam splitter as quantum coherence-maker, \href{https://doi.org/10.1088/1402-4896/aca1e7}{Phys. Scr. {\bf 98}, 015101 (2023)}.

\bibitem{ZSLF16}
Y.-R. Zhang, L.-H. Shao, Y. Li, and H. Fan, Quantifying coherence in infinite-dimensional systems, \href{https://doi.org/10.1103/PhysRevA.93.012334}{Phys. Rev. A {\bf 93}, 012334 (2016)}.

\bibitem{KN00}
K. Nemoto, Generalized coherent states for SU(n) systems, \href{https://doi.org/10.1088/0305-4470/33/17/307}{J. Phys. A {\bf 33}, 3493--3506 (2000)}.

\bibitem{LP96} 
A. Luis and J. Pe\v{r}ina, SU(2) coherent states in parametric down-conversion, \href{https://doi.org/10.1103/PhysRevA.53.1886}{Phys. Rev. A {\bf 53}, 1886--1893 (1996)}.

\bibitem{PRM15}
K. von Prillwitz, Ł. Rudnicki, and F. Mintert, Contrast in multipath interference and quantum coherence, \href{https://doi.org/10.1103/PhysRevA.92.052114}{Phys. Rev. A {\bf 92}, 052114 (2015)}.

\bibitem{BGW17}
T. Biswas, M. García Díaz, and A. Winter, Interferometric visibility and coherence, \href{https://doi.org/10.1098/rspa.2017.0170}{Proc. Roy. Soc. London A {\bf 473}, 20170170 (2017)}.

\bibitem{SL21}
Y. Sun and S. Luo, Quantifying Interference via Coherence, \href{ https://doi.org/10.1002/andp.202100303}{Ann. Phys. (Berlin) {\bf 2021}, 2100303 (2021)}.

\bibitem{AL08}
A. Luis, Quantum-classical correspondence for visibility, coherence, and relative phase for multidimensional systems, \href{https://doi.org/10.1103/PhysRevA.78.025802}{Phys. Rev. A {\bf 78}, 025802 (2008)}.

\bibitem{AL09}
A. Luis, Ensemble approach to coherence between two scalar harmonic light vibrations and the phase difference, \href{https://doi.org/10.1103/PhysRevA.79.053855}{Phys. Rev. A {\bf 79}, 053855 (2009)}.

\end{thebibliography}


\end{document}