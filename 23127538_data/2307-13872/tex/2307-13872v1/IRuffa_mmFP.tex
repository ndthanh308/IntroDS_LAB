% mnras_template.tex 
%
% LaTeX template for creating an MNRAS paper
%
% v3.0 released 14 May 2015
% (version numbers match those of mnras.cls)
%
% Copyright (C) Royal Astronomical Society 2015
% Authors:
% Keith T. Smith (Royal Astronomical Society)

% Change log
%
% v3.0 May 2015
%    Renamed to match the new package name
%    Version number matches mnras.cls
%    A few minor tweaks to wording
% v1.0 September 2013
%    Beta testing only - never publicly released
%    First version: a simple (ish) template for creating an MNRAS paper

%%%%%%%%%%%%%%%%%%%%%%%%%%%%%%%%%%%%%%%%%%%%%%%%%%
% Basic setup. Most papers should leave these options alone.
\documentclass[fleqn,usenatbib]{mnras}

% MNRAS is set in Times font. If you don't have this installed (most LaTeX
% installations will be fine) or prefer the old Computer Modern fonts, comment
% out the following line
%\usepackage{newtxtext,newtxmath}
% Depending on your LaTeX fonts installation, you might get better results with one of these:
%\usepackage{mathptmx}
%\usepackage{txfonts}

% Use vector fonts, so it zooms properly in on-screen viewing software
% Don't change these lines unless you know what you are doing
\usepackage[T1]{fontenc}

% Allow "Thomas van Noord" and "Simon de Laguarde" and alike to be sorted by "N" and "L" etc. in the bibliography.
% Write the name in the bibliography as "\VAN{Noord}{Van}{van} Noord, Thomas"
\DeclareRobustCommand{\VAN}[3]{#2}
\let\VANthebibliography\thebibliography
\def\thebibliography{\DeclareRobustCommand{\VAN}[3]{##3}\VANthebibliography}


%%%%% AUTHORS - PLACE YOUR OWN PACKAGES HERE %%%%%

% Only include extra packages if you really need them. Common packages are:
% \usepackage{amssymb}	% Extra maths symbols
%\usepackage{MnSymbol}
\usepackage{graphicx}	% Including figure files
\usepackage{amsmath}	% Advanced maths commands
\usepackage{amssymb}	% Extra maths symbols
\usepackage{subcaption}
\captionsetup{compatibility=false}



%%%%%%%%%%%%%%%%%%%%%%%%%%%%%%%%%%%%%%%%%%%%%%%%%%

%%%%% AUTHORS - PLACE YOUR OWN COMMANDS HERE %%%%%

% Please keep new commands to a minimum, and use \newcommand not \def to avoid
% overwriting existing commands. Example:
%\newcommand{\pcm}{\,cm$^{-2}$}	% per cm-squared
\newcommand{\gtsimeq}{\raisebox{-0.6ex}{$\,\stackrel{\raisebox{-.2ex}{$\textstyle >$}}{\sim}\,$}}

%%%%%%%%%%%%%%%%%%%%%%%%%%%%%%%%%%%%%%%%%%%%%%%%%%

%%%%%%%%%%%%%%%%%%% TITLE PAGE %%%%%%%%%%%%%%%%%%%

% Title of the paper, and the short title which is used in the headers.
% Keep the title short and informative.
\title[The mm Fundamental Plane]{A fundamental plane of black hole accretion at millimetre wavelengths}

% The list of authors, and the short list which is used in the headers.
% If you need two or more lines of authors, add an extra line using \newauthor
\author[I. Ruffa et al.]{Ilaria Ruffa,$^{1}$\thanks{E-mail: ruffai@cardiff.ac.uk}
Timothy A. Davis,$^{1}$
Jacob S. Elford,$^{1}$
Martin Bureau,$^{2}$
Michele Cappellari,$^{2}$
\newauthor
Jindra Gensior,$^{3}$
Daryl Haggard,$^{4}$
Satoru Iguchi,$^{5,6}$
Federico Lelli,$^{7}$
Fu-Heng Liang,$^{2}$
Lijie Liu,$^{8,9}$
\newauthor
Marc Sarzi,$^{10}$
Thomas G. Williams,$^{2}$
and Hengyue Zhang$^{2}$
\vspace{0.2cm}
\\
% List of institutions
\vspace{0.1cm}
\parbox{\textwidth}{$^{1}$Cardiff Hub for Astrophysics Research \&\ Technology, School of Physics \&\ Astronomy, Cardiff University, Queens Buildings, The Parade, Cardiff, CF24 3AA, UK}\\
$^{2}$Sub-department of Astrophysics, Department of Physics, University of Oxford, Keble Road, Oxford, OX1 3RH, UK\\
$^{3}$Institute for Computational Science, University of Zurich, Winterthurerstrasse 190, Z{\"u}rich, 8057, Switzerland \\
$^{4}$Trottier Space Institute and Department of Physics, McGill University, 3600 rue University, Montreal, QC H3A 2T8, Canada\\
$^{5}$Department of Astronomical Science, SOKENDAI (The Graduate University of Advanced Studies), Mitaka, Tokyo 181-8588, Japan\\
$^{6}$National Astronomical Observatory of Japan, National Institutes of Natural Sciences, Mitaka, Tokyo 181-8588, Japan\\
$^{7}$INAF, Arcetri Astrophysical Observatory, Largo Enrico Fermi 5, I-50125, Florence, Italy\\
$^{8}$Cosmic Dawn Center (DAWN), Technical University of Denmark, DK2800 Kgs.\ Lyngby, Denmark\\
$^{9}$DTU-Space, Technical University of Denmark, Elektrovej 327, DK2800 Kgs.\ Lyngby, Denmark\\
$^{10}$Armagh Observatory and Planetarium, College Hill, Armagh BT61 9DG, UK
}

% These dates will be filled out by the publisher
\date{Accepted XXX. Received YYY; in original form ZZZ}

% Enter the current year, for the copyright statements etc.
\pubyear{2015}

% Don't change these lines
\begin{document}
\label{firstpage}
\pagerange{\pageref{firstpage}--\pageref{lastpage}}
\maketitle

% Abstract of the paper
\begin{abstract}
We report the discovery of the ``mm fundamental plane of black-hole accretion'', which is a tight correlation between the nuclear 1 mm luminosity ($L_{\rm \nu, mm}$), the $2$ -- $10$~keV X-ray luminosity ($L_{\rm X,2-10}$) and the supermassive black hole (SMBH) mass ($M_{\rm BH}$) with an intrinsic scatter ($\sigma_{\rm int}$) of $0.40$~dex. The plane is found for a sample of 48 nearby galaxies, most of which are radiatively-inefficient, low-luminosity active galactic nuclei (LLAGN). Combining these sources with a sample of high-luminosity (quasar-like) nearby AGN, we find that the plane still holds. We also find that $M_{\rm BH}$ correlates with $L_{\rm \nu, mm}$ at a highly significant level, although such correlation is less tight than the mm fundamental plane ($\sigma_{\rm int}=0.51$~dex). Crucially, we show that spectral energy distribution (SED) models for both advection-dominated accretion flows (ADAFs) and compact jets can explain the existence of these relations, which are not reproduced by the standard torus-thin accretion disc models usually associated to quasar-like AGN. The ADAF models reproduces the observed relations somewhat better than those for compact jets, although neither provides a perfect fit. Our findings thus suggest that radiatively-inefficient accretion processes such as those in ADAFs or compact (and thus likely young) jets may play a key role in both low- and high-luminosity AGN. This mm fundamental plane also offers a new, rapid method to (indirectly) estimate SMBH masses.

%We investigate the correlation between the nuclear 1 mm luminosity ($L_{\rm \nu, mm}$), the $2$ -- $10$~keV X-ray luminosity ($L_{\rm X,2-10}$) and the supermassive black hole (SMBH) mass ($M_{\rm BH}$) of a sample of 48 nearby galaxies, most of which are radiatively-inefficient, low-luminosity active galactic nuclei (LLAGN). We find that the SMBH mass correlates with $L_{\rm \nu, mm}$ at a highly significant level (intrinsic scatter $\sigma_{\rm int}=0.51$~dex), and discover an even tighter correlation when adding $L_{\rm X,2-10}$. These sources define a plane in the three-dimensional space of ($\log M_{\rm BH}$, $\log L_{\rm X,2-10}$, $\log L_{\rm \nu, mm}$) that we dub the ``mm fundamental plane of BH accretion" ($\sigma_{\rm int}=0.40$~dex). When restricting our analysis to only those sources with the most accurate (redshift-independent) distances, we obtain tighter correlations, with $\sigma_{\rm int}=0.19$~dex for the $M_{\rm BH}-L_{\nu, \rm mm}$ relation and only $0.11$~dex for the mm fundamental plane. Furthermore, combining our sources with a comparison sample of high-luminosity radiatively-efficient (quasar-like) nearby AGN, we find that the mm fundamental plane still holds. Crucially, we show that spectral energy distribution (SED) models for both advection-dominated accretion flows (ADAFs) and compact jets can explain the existence of these relations, which are not reproduced by the standard torus-thin accretion disc models. The ADAF models reproduces the observed relations more convincingly than those for compact jets, although neither provides a perfect explanation. %Although slightly less likely, an alternative plausible explanation is that both the observed mm and the X-ray emission arise from compact radio jets. 
%Our findings thus suggest that radiatively-inefficient accretion processes such as those in ADAFs or compact (and thus likely young) jets may play a key role in both low- and high-luminosity AGN. This mm fundamental plane also offers a new, rapid method to (indirectly) estimate SMBH masses.
\end{abstract}

% Select between one and six entries from the list of approved keywords.
% Don't make up new ones.
\begin{keywords}
galaxies: active -- galaxies: nuclei -- black hole physics -- X-rays: galaxies -- submillimetre: galaxies
\end{keywords}

%%%%%%%%%%%%%%%%%%%%%%%%%%%%%%%%%%%%%%%%%%%%%%%%%%

%%%%%%%%%%%%%%%%% BODY OF PAPER %%%%%%%%%%%%%%%%%%
%\clearpage
\section{Introduction}
%Understanding the processes driving and regulating the connection between the growth of central super-massive black holes (SMBHs, found to lie at the centre of almost every galaxy with stellar mass $>10^{9}$~M$_{\odot}$) and the evolution of their host galaxies (so-called "co-evolution”; see e.g.\,\citealp{KormendyHo2013}) is an hot topic in modern astrophysics. The last three decades of observations have indeed demonstrated that the mass of such SMBHs correlates with a number of host galaxy properties (such as the stellar velocity dispersion, $\sigma_{\star}$; e.g.\,\citealp[][]{Ferrarese00,Gultekin09}), suggesting that the two may co-evolve in a self-regulating manner. Active galactic nuclei (AGN) and associated energetic output (i.e.\,feedback) are believed to play a crucial role in setting up such co-evolution \citep[e.g.][]{Harrison18}. 
The many details of the processes driving and regulating the connection between the growth of central super-massive black holes (SMBHs, found to lie at the centre of almost every galaxy with a stellar mass $>10^{9}$~M$_{\odot}$) and the evolution of their host galaxies (so-called "co-evolution”; e.g.\,\citealp{KormendyHo2013}) are still poorly understood \citep[e.g.][]{Donofrio21}. Understanding the physics of accretion onto SMBHs, determining if and how it changes in objects with different types of nuclear activity, as well as setting accurate constraints on fundamental SMBH properties such as its mass, are all fundamental steps to take the next leaps forward in our comprehension of the SMBH-host galaxy interplay.

The so-called “fundamental plane of BH accretion” (heareafter FP) is an empirical correlation between the SMBH masses ($M_{\rm BH}$), 5~GHz radio ($L_{\rm 5 GHz}$) and 2 - 10~keV X-ray ($L_{\rm X,2-10}$) luminosities, which was initially reported by \citet{Merloni03} and \citet{Falcke04}. The origin of the FP is still debated, but it is widely believed to carry information on the physics of SMBH accretion (see e.g.\,\citealp[][]{Gultekin19} for a detailed discussion). However, the scatter around this correlation varies significantly depending on the sample and the method used to fit the plane, reaching values up to $0.88$~dex (see e.g.\,\citealp[][]{Merloni03,Gultekin09,Plotkin12,Saikia15,Gultekin19}, and references therein). Furthermore, the nature of the radio emission in the sources following the FP is not yet well understood (potentially arising from compact jets or complex shock dynamics). All the above somehow limit the diagnostic power of the FP.

In this paper, we report the discovery of a fundamental plane at millimetre wavelengths, namely the existence of a tight correlation between the nuclear (i.e.\,$\ll100$~pc) mm luminosities ($L_{\rm \nu, mm}$), $M_{\rm BH}$ and $L_{\rm X,2-10}$, which appears to hold for both high- and low-luminosity AGN (within $z\lesssim0.05$). We also present potential physical interpretations of such correlation, and discuss how this discovery may have profound implications for our understanding of the central engines of galaxies with many different types of nuclear activities.
%Throughout this work we assume a $\Lambda$CDM cosmology with H$_{\rm 0}=70$\,km\,s$^{-1}$\,Mpc$^{\rm -1}$, $\Omega_{\rm \Lambda}=0.7$ and $\Omega_{\rm M}=0.3$.

\section{Primary sample and data}
\label{sec:methods-sample}
%\subsection{Primary sample}
Our sample was primarily drawn from the mm-Wave Interferometric Survey of Dark Object Masses (WISDOM) project, which has the main aim of exploiting high-resolution Atacama Large Millimeter/submillimeter Array (ALMA) CO data to dynamically estimate SMBH masses in a varied sample of galaxies \citep[e.g.][]{Davis17}. In this work, we included $31$ WISDOM galaxies (Table~\ref{tab:datatable}) at $z\lesssim0.03$, spanning a range of AGN bolometric luminosities ($L_{\rm bol}=10^{41}$ -- $10^{46}$~erg~s$^{-1}$) and mostly (but not exclusively) having very low rates of accretion onto their central SMBHs ($\dot{M}\lesssim10^{-3}$~$\dot{M}_{\rm Edd}$; \citealp{Elford2023}). As such, most of these objects are classified as low-luminosity AGN (LLAGN; \citealp{Ho08}). To increase the statistics, we supplemented the 31 WISDOM galaxies with a further 17 objects (Table~\ref{tab:datatable}), selected from the literature to have dynamical SMBH mass estimates, existing high-resolution ALMA $1$~mm observations and high-quality X-ray data. The majority of these are massive nearby ellipticals and span ranges of $L_{\rm bol}$ and $\dot{M}$ similar to those of the WISDOM sources. Hereafter, we refer to the 31 WISDOM plus 17 literature sources as the {\it primary sample}.

%\subsubsection{Millimetre luminosities}
%\label{sec:methods-mmlum}
The 1 mm luminosities of all the primary sample sources were derived from high-angular-resolution ALMA Band 6 continuum observations, taken between 2013 and 2021 as part of a large number of projects. All data were reduced using the Common Astronomy Software Applications ({\sc casa}) pipeline \citep{McMullin07}, adopting a version appropriate for each dataset and a standard calibration strategy. For more details on the data reduction see \cite{Davis22}.

For each dataset, continuum images were produced by combining the continuum spectral windows (SPWs) and the line-free channels of the line SPW (when included)  using the \textsc{CASA} task \textsc{tclean} in multi-frequency synthesis (MFS) mode. The resulting continuum maps have synthesised beams ranging from $0.\!\!^{\prime\prime}042$ to $0.\!\!^{\prime\prime}723$, corresponding to $6$ -- $330$~pc at the distances of our sources (average spatial resolution $\approx25$~pc). For each source, the continuum flux density $f_{\rm mm}$ was measured from the innermost synthesised beam, coincident with the galaxy core. The mm luminosities were then estimated as $L_{\nu,{\rm mm}}=4\pi D_{\rm L}^{2} f_{\rm mm} \nu_{\rm obs}$, where $D_{\rm L}$ is the luminosity distance and $\nu_{\rm obs}$ the observed frequency (between 231 and 239 GHz). As all the data were obtained with long-baseline configurations, large-scale dust emission is resolved out. We thus typically detect only a point-like source at each galaxy centre, arising from unresolved core emission. In a small number of galaxies (3/48), the continuum emission is slightly resolved, increasing the uncertainties of our measurements. Removing these 3 objects, however, does not affect our results in any way. The obtained $L_{\nu,{\rm mm}}$ are listed in Table~\ref{tab:datatable}.

%\subsubsection{X-ray luminosities}
%\label{sec:methods-xraylum}
The intrinsic (absorption-corrected) $2$--$10$~keV luminosities ($L_{\rm X,2-10}$) of the primary sample sources were retrieved from the literature \citep[see][for details]{Elford2023}. Eight of these galaxies have no X-ray data available (Table~\ref{tab:datatable}) and were thus not considered in the parts of the analysis where $L_{\rm X,2-10}$ was required. For the vast majority of the objects with X-ray data (33/40), the adopted $L_{\rm X,2-10}$ was derived from {\it Chandra} observations, thus including only nuclear emission from the unresolved AGN core. For most of the {\it Chandra}-observed objects (26/33), $L_{\rm X,2-10}$ was retrieved from the catalogue of \citet[][see also Table~\ref{tab:datatable}]{Bi20}. 

%\subsubsection{SMBH masses}
Accurate, dynamically-determined SMBH masses (from stellar, ionised gas, molecular gas and/or maser kinematics) are available for a total of 31 sources in the primary sample (Table~\ref{tab:datatable}). For those galaxies without an existing direct measurement, we estimated $M_{\rm BH}$ using the $M_{\rm BH}$ --  $\sigma_{\star}$ relation of \cite{Vandenbosch16}, where $\sigma_{\star}$ is the stellar velocity dispersion within one effective radius. This was retrieved from the compilations of \citet{Vandenbosch16} and \citet{Cappellari2013} when available, from the HyperLeda database otherwise (\url{http://leda.univ-lyon1.fr}). Crucially, although our primary sample is constructed based on data availability only (and is thus not meant to be complete in any statistical sense), the sample galaxies do span four orders of magnitude in SMBH mass.

% Figure environment removed

\section{The mm fundamental plane}
\label{sec:methods-fitting}
As illustrated in Fig.\,~\ref{fig:fundamental_plane} (left panel), the SMBH masses of our primary sample galaxies strongly correlate with their $L_{\rm \nu, mm}$. A power law was fitted to the observed trend, using the \textsc{lts\_linefit} routine \citep{Cappellari2013}. This combines the least-trimmed-squares (LTS) robust regression technique \citep{Rousseeuw84} with a least-squares fitting algorithm, and allows for intrinsic scatter and uncertainties in all coordinates. The resulting best-fitting power law is: 
\begin{equation}
    \log_{10}\left(\frac{M_{\rm BH}}{M_{\odot}}\right)=(0.79\pm0.08)\left[\log_{10}\left(\frac{L_{\nu, \rm mm}}{\mathrm{erg\,s}^{-1}}\right)-39\right]+ (8.2\pm0.1)\,,
\end{equation}
with an observed scatter ($\sigma_{\rm obs}$) of $0.55$~dex and an estimated intrinsic scatter ($\sigma_{\rm int}$) of $0.51\pm0.08$~dex. When including $L_{\rm X,2-10}$, we discover the existence of a tighter correlation (Fig.\,~\ref{fig:fundamental_plane}, right panel). In this case, we used the \textsc{lts\_planefit} routine \citep{Cappellari2013} to find the best-fitting plane in the ($\log M_{\rm BH}$, $\log L_{\rm X,2-10}$, $\log L_{\nu, \rm mm}$) space:

\begin{eqnarray}
    \nonumber \log_{10}\left(\frac{M_{\rm BH}}{M_{\odot}}\right) =  (-0.23\pm0.05)\left[\log_{10}\left(\frac{L_{\rm X,2-10}}{\mathrm{erg\,s}^{-1}}\right)-40\right] \\+   (0.95\pm0.07)\left[\log_{10}\left(\frac{L_{\nu,\rm mm}}{\mathrm{erg\,s}^{-1}}\right)-39\right] + (8.35\pm0.08)\,,
\end{eqnarray}
with $\sigma_{\rm obs}=0.45$~dex and $\sigma_{\rm int}=0.40\pm0.07$~dex. We verified that this multi-variate plane fit provides a significantly better predictor for the SMBH mass than the simple line fit, having a $\Delta_{\rm BIC}$ $>>$10 (where $\Delta_{\rm BIC}$ is the difference in the Bayesian information criterion between the line and plane fits). For both correlations, we also performed Spearman rank analyses to quantify their statistical significance, and show the resulting correlation coefficients in the top-left corner of each panel of Fig.\,~\ref{fig:fundamental_plane}. Since the mm and X-ray emission from AGN is known to be time variable (by a factor of typically $2$ -- $3$ over year timescales; \citealp{Prieto2016,Fernandez-Ontiveros2019}), variability likely dominates the scatters observed in Fig.\,\ref{fig:fundamental_plane} (and thus the underlying correlations may be even tighter). By analogy with the previous FP, we dub the correlation in the right panel of Fig.\,\ref{fig:fundamental_plane} as the ``mm fundamental plane of BH accretion" (hereafter mmFP).

We note that the flux calibration uncertainty of the ALMA data introduces uncertainties in the derived $L_{\nu,{\rm mm}}$. This is currently estimated to be $\approx$10\% in ALMA band 6. We included such uncertainty in the $L_{\nu,{\rm mm}}$ error budget, but note that this is smaller than the estimated intrinsic scatters and thus likely to have a negligible impact on our results. We also note that the $L_{\rm X,2-10}$ of the five sample sources without available {\it Chandra} data could be slightly overestimated, due to contamination from diffuse hot gas in the galactic and circum-galactic medium (CGM; although this mainly emits in the $0.3$ -- $2$~keV range) and/or X-ray binaries. While any such contamination should be minimal (based on the scaling laws of \citealp{Grimm03}, \citealp{Kim04} and \citealp{Boroson11}), we cannot rule it out entirely. In any case, removing these five sources does not make any relevant change in the parameters of the best-fitting mmFP reported above. The same applies when removing the sources in the primary sample without a dynamical $M_{\rm BH}$ estimate (the best-fitting line and planes are identical, within their respective errors, and the observed scatters become only slightly smaller).

\subsection{BASS galaxies}\label{sec:methods-bass}
Although the majority of the primary sample galaxies are LLAGN, a handful are more luminous systems (see \citealp{Elford2023}), which still follow the mmFP. To investigate whether this result holds more generally, we built a comparison sample from the {\it Swift}-BAT AGN Spectroscopic Survey (BASS), comprising AGN with median $z=0.05$, $L_{\rm bol}=10^{44}$~erg~s$^{-1}$ and $\dot{M} = 0.01-0.1$~$\dot{M}_{\rm Edd}$ \citep[][]{Koss17}. We included in our analysis only the BASS sources for which both ALMA $1$~mm observations (with spatial resolutions similar to those of the primary sample sources) and nuclear $L_{\rm X,2-10}$ were available ($88$ sources; \citealp{Kawamuro2022}). The SMBH masses of these objects were taken from the compilation of \cite{Koss2022}. The BASS galaxies are typically more distant than those in the primary sample, so their $M_{\rm BH}$ have been estimated with a variety of methods (see Fig.\,\ref{fig:bass_residuals}). A small number of sources have dynamical $M_{\rm BH}$ measurements ($9/88$), while others have $M_{\rm BH}$ from reverberation mapping ($13/88$) or the widths of broad Balmer emission lines ($16/88$). For the remaining sources ($50/88$), $M_{\rm BH}$ is indirectly estimated using the $M_{\rm BH}$ --  $\sigma_{\star}$ relation of \cite{KormendyHo2013}. We re-calibrated these measurements using the $M_{\rm BH}$ --  $\sigma_{\star}$ relation of \cite{Vandenbosch16}, for consistency with the 17 sources in the primary sample without a dynamical SMBH mass estimate.

As shown by Fig.\,\ref{fig:ADAF_model_projection}, the BASS sources are in agreement with the best-fitting mmFP, albeit with a larger observed scatter. We performed a Spearman rank analysis to quantify the statistical significance of this relation for the BASS points alone, and verified that they do show a significant correlation ($p=0.002$), but with a correlation coefficient ($\rho$=0.32) smaller than that of the primary sample. Fig.\,\ref{fig:bass_residuals} suggests that the larger scatter in this population is at least partly driven by the $M_{\rm BH}$ uncertainties, as the position of a BASS galaxy with respect to the best-fitting mmFP depends on the method used to estimate its $M_{\rm BH}$. For instance, sources with $M_{\rm BH}$ from reverberation mapping or broad-line methods are located systematically below the best-fitting line, likely reflecting the different biases in place when using such techniques \citep[e.g.][]{Farrah2023}. 

\section{Physical Drivers}\label{sec:models}
The fact that the BASS galaxies are consistent with a relation mainly defined by LLAGN is surprising. As discussed below, if (as expected) their nuclear mm continuum arises from accretion disc light reprocessed by dust in the torus, the BASS sources would be expected to fall off the observed relation by at least two orders of magnitude. Our finding suggests that the dominant mechanism giving rise to the nuclear mm continuum (and its correlation with that producing the $2$ -- $10$~keV emission) may be similar to the one in LLAGN. To determine the underlying physics, we compared the observed nuclear mm and X-ray luminosities of both the primary sample and BASS sources to those extracted from mock nuclear SEDs arising from radiatively-inefficient (ADAF-like) and ``classic'' (geometrically-thin and optically-thick) accretion flows, and from compact radio jets.

\subsection{Torus model}
\label{sec:methods-torusmodel}
AGN in the BASS sample (and a few galaxies in the primary sample) have {\it estimated} accretion rates in the range $\dot{M} \sim 0.01-0.1$~$\dot{M}_{\rm Edd}$. According to the standard paradigm, in this type of systems the accretion should occur through the classic geometrically-thin and optically-thick accretion disc surrounded by a dusty torus \citep[e.g.][]{Heckman14}. In this scenario, both the mm and the $2$ -- $10$~keV emission arise from the accretion disc, reprocessed by dust in the torus in the mm and Compton-up scattered by the hot corona in X-rays. To check if this type of model can explain the correlations in Fig.\,\ref{fig:fundamental_plane}, we used SKIRTOR \citep{Stalevski2012,Stalevski2016} - a library of emission models for classic AGN with dusty torii. The SED models were retrieved from the SKIRTOR webpage ({\url{https://sites.google.com/site/skirtorus/}). The wavelength coverage of SKIRTOR (from $300$~GHz to $1.24$~keV) is slightly shorter than that required for this work. We thus expanded the models to the full range of wavelengths probed here, treating the emission mechanisms self-consistently as prescribed in the original version of the code (i.e.\,using the same grey-body curve for millimetre emission, and a power-law in the X-ray regime; \citealp[][]{Yang2020}). We followed the prescriptions of \cite{Stalevski2012,Stalevski2016} to scale the models for different $L_{\rm bol}$, in the range $10^{7.5}$ -- $10^{12.5}$~L$_{\odot}$ (i.e.\,the range covered by the primary and BASS sources; the torus is expected to disappear at low accretion rates, but the resulting model predictions are nevertheless instructive). For each SKIRTOR SED model, we then extracted the predicted $1$~mm (specifically, the luminosity at 237.5~GHz, that is the median ALMA continuum frequency for both the primary and BASS sources) and intrinsic 2-10~keV luminosities, and compared them with the measured ones. The resulting predictions are shown in Fig.\,\ref{fig:mm_xray_with_grid_and_skirtor} as a hexagonally binned histogram (coloured by mean L$_{\rm bol}$). The torus model reasonably reproduces the slope of the $L_{\rm X,2-10}$ -- $L_{\rm \nu, mm}$ relation of the BASS sources, but with an offset of about two orders of magnitude at a given L$_{\rm bol}$. On the other hand, Fig.\,\ref{fig:mm_xray_with_grid_and_skirtor} shows that, to explain the mm luminosities of the lower accretion rate galaxies, the mm flux in the SKIRTOR models would need to be at least four orders of magnitude larger at a given accretion rate (and thus X-ray luminosity).

\subsection{ADAF model}
\label{sec:methods-adafmodel}
To build model SEDs arising from radiatively-inefficient accretion flows around SMBHs, we used the ``LLAGN'' model of \citet[][itself a development of previous models by \citealt{NarayanYi1995} and \citealt{Mahadevan1997}]{Pesce21}. In typical LLAGN and some (low-accretion-rate) Seyferts, the classic accretion disc is either absent or truncated at some inner radius (the transition usually happens beyond a few tens of Schwarzschild radii), and replaced by a geometrically-thick two-temperature structure in which the ion temperature is greater than the electron temperature and the accretion occurs at rates well below the Eddington limit (i.e.\,$\ll0.01$~$\dot{M}_{\rm Edd}$; \citealp{Nara95,Ho08}). The electrons in such radiatively-inefficient flows (such as advection-dominated accretion flows; ADAFs) cool down via a combination of self-absorbed synchrotron, bremsstrahlung and inverse Compton radiation, which together give rise to the nuclear SED from the mm to the X-rays. The LLAGN model adopted here solves for the energy balance between the heating and cooling of the electrons in the flow. We generated a set of model SEDs for a grid of SMBH masses ($10^{6}$ -- $10^{10}$~M$_{\odot}$) and Eddington ratios ($10^{-7}$ -- $10^{-2}$), while all the other free parameters were kept at the defaults discussed in Appendix~A of \cite{Pesce21}. We then extracted the predicted $237.5$~GHz and $2-10$~keV luminosities, as described above. As illustrated by the shape of the model grid in Fig.\,\ref{fig:mm_xray_with_grid_and_skirtor}, the mm and X-ray luminosities of all the sources (and thus the observed correlations) are well explained if they arise from an ADAF-like accretion mechanism. The grid is almost aligned with the axes, thus predicting that the mm luminosity primarily depends on $M_{\rm BH}$, while $L_{\rm X,2-10}$ primarily traces the Eddington ratio. The tighter correlation obtained when including $L_{\rm X,2-10}$ can be explained by the fact that the slight tilt of the grid is then taken into account, especially at higher Eddington ratios. A 3D version of Fig.\,\ref{fig:mm_xray_with_grid_and_skirtor} is provided as supplementary online material. In Fig.\,\ref{fig:ADAF_model_projection}, we show the projection of the ADAF model grid onto the best-fitting mmFP. This latter seems to arise naturally from these models, as an (almost) edge-on view of the 3D ($M_{\rm BH}$, $L_{\rm X,2-10}$, $L_{\rm \nu, mm}$) relation. We note, however, that keeping the default model parameters from \cite{Pesce21}, the model well predicts the gradient of the mmFP, but is offset by a small amount (i.e.\,the model overpredicts $L_{\rm \nu, mm}$ at a given SMBH mass by $\approx0.5$~dex). Tweaking the model parameters to reduce the effective radiative efficiency easily removes this offset (e.g.\,by changing some combination of the effective viscosity, ratio of gas to magnetic pressure, fraction of viscous heating going directly to the electrons, outer radius of the ADAF, and/or power-law index of the mass accretion rate as a function of radius). However, as the correct values of these parameters is not well constrained, here we simply offset the model grid by a constant $0.5$~dex in SMBH mass to align it with the observed correlation. We stress that this scaling factor is significantly smaller than the one required for torus models to reproduce the observed trend (see Section~\ref{sec:methods-torusmodel}), and is well within the uncertainties for the adopted model parameters (see \citealp{Pesce21}). Future work exploring the parameters of ADAF-like models in sources with more extensive (sub-)mm coverage will allow to better understand the observed correlation, the small offset, and the physics of accretion onto these SMBHs. This is discussed further in Section~\ref{sec:discussion}.

% Figure environment removed

\subsection{Compact jet model}
\label{sec:methods-compactjet}
Unlike extended jets, where the synchrotron emission is optically-thin, compact radio jets have self-absorbed synchrotron spectra in the mm (similar to those from ADAFs). Compact jets have been argued to dominate the SEDs of LLAGN, and in some cases are preferred over a pure ADAF solution (as this would be overly luminous at near-infrared and optical wavelengths; \citealp[e.g.][]{Fernandez-Ontiveros2023}). To determine if compact jets can explain the trends observed in this work, we used the \textsc{Bhjet} model of \cite{Lucchini2022}. We fixed most of the model parameters to the values found for M81, a prototypical galaxy with compact jets (Model B in Table 3 of \citealt{Lucchini2022}), and generated a grid of models varying the SMBH mass ($10^{6}$\,--\,$10^{10}$~M$_{\odot}$), jet power ($10^{-5.5}$\,--\,$10^{-0.5}$ L$_{\rm Edd}$) and jet inclination to the line-of-sight (2.5$^{\circ}$\,--\,90$^{\circ}$). We then extracted from the resulting model SEDs the predicted $L_{\rm \nu, mm}$ and $L_{\rm X,2-10}$ (as above), and compare them with the estimated ones. In Fig.\,\ref{fig:jetmodel_results} we show the $L_{\rm X,2-10}$ -- $L_{\rm \nu, mm}$ relation with the resulting model grids for the extremes in jet inclination (2.5$^{\circ}$ and 90$^{\circ}$) overlaid. Jets at intermediate inclinations lie between these two extremes (but evolve quickly towards the i=90$^{\circ}$ solution once the line-of-sight is no longer aligned along the jet cone). 
The model grids encompass the majority of the LLAGN and some BASS sources, but they are more complex than those of ADAFs, having significant curvature in the 3D $M_{\rm BH}$-$L_{\rm X,2-10}$-$L_{\rm \nu, mm}$ space (a 3D version of Fig.\,\ref{fig:jetmodel_results} is provided as supplementary online material). The correlations in Fig.\,\ref{fig:fundamental_plane} do not seem to occur naturally within this model (as projections of the higher-order surface onto the axes). The luminosities of high-accretion-rate AGN from the primary and the BASS samples are harder to explain with these models, and would require additional X-ray emitting components. This is perhaps unsurprising, as compact jet models are substantially more complex than ADAFs (see also Section~\ref{sec:discussion}).

% Figure environment removed

\subsection{Distance uncertainties}\label{sec:distance_check}
Both SMBH mass and luminosity measurements are systematically affected by the assumed galaxy distance $D$, with $M_{\rm BH}\propto D$ and $L\propto D^{2}$. Large distance errors can thus introduce large uncertainties on $M_{\rm BH}$ and $L$, and the difference in how these quantities scale with distance can give rise to spurious correlations. To test that this is not affecting our results, we performed a simple Monte Carlo simulation, drawing $M_{\rm BH}$ and the luminosities from independent Gaussian distributions that are truly uncorrelated, forcing a correlation to arise due to distance errors alone. The magnitude of the distance errors required to reproduce the Spearman rank correlation coefficients of the observed relations turned out to be very high ($\geq$1.5 dex), much higher than that of our primary sample sources and - more in general - expected for real distance measurements. In addition, the slope of relations purely due to distance uncertainties would be substantially flatter than those observed (gradients of 0.5 for the $M_{\rm BH}$- $L_{ \rm \nu, mm}$ correlation and 0.25 for the mmFP, as opposed to the observed $\approx$0.8 and $\approx$1, respectively). 

To further determine if any systematic distance bias could be affecting our results, we carried out a simple quality-checking exercise for our primary sample sources. We restricted our analysis to only those sources with the most accurate (redshift-independent) distances (26/48), i.e.\,derived from surface brightness fluctuations, tip of the red giant branch methods, supernovae, Cepheids, masers, the planetary nebula luminosity function and the globular cluster luminosity function. This led us to obtain much tighter correlations, with an intrinsic scatter of $0.19$~dex for the $M_{\rm BH}-L_{\nu, \rm mm}$ relation and only $0.11$~dex for the mmFP. The corresponding Spearman rank coefficients are $\rho=0.84$ ($p=5.05\times10^{-7}$) and $\rho=0.93$ ($p=2.67\times10^{-9}$), respectively. We thus conclude that our results are not biased due to distance uncertainties. 

\section{Discussion and conclusions}
\label{sec:discussion}
We report here the finding of tight $M_{\rm BH}-L_{\nu, \rm mm}$ and $M_{\rm BH}-L_{\rm X,2-10}-L_{\nu, \rm mm}$ correlations (Fig.\,\ref{fig:fundamental_plane}). We dub the latter the ``mm fundamental plane of BH accretion" and find it to hold for both low- (mostly WISDOM) and high- (mostly BASS) luminosity AGN. To understand the physics underlying the mmFP, we compared the observed trend with models predicting the emission from different nuclear mechanisms. We find that the results for both our sample and the BASS sources are best explained if their emission in the mm and X-rays primarily arises from an ADAF-like process, but cannot be explained by a classic torus model (see Fig.\,\ref{fig:model_results}). This suggests that some kind of radiatively-inefficient accretion process may play a role in both low- and high-luminosity AGN, at least in the range of luminosities and accretion rates probed by the sources included in this work. While torii are known to exist in many of these AGN, some regions around their SMBHs may be radiatively-inefficient. For instance, some accretion disc solutions allow discs to transition from ADAF-like to geometrically-thin, and vice versa at different radii \citep{Mahadevan1997}. ADAFs could also exist above and below classic accretion discs \citep{Mahadevan1997}. Although the exact conditions under which this applies are still to be investigated, it is clear that - if confirmed - our results will have profound implications for our understanding of the central engines of many different types of AGN. 

We also explored the possibility that both the mm and the X-ray emission arise from compact (and thus likely young; \citealp{ODea20}) radio jets (Section~\ref{sec:methods-compactjet}). These have been argued to dominate the whole SEDs of LLAGN \citep[e.g.][]{Fernandez-Ontiveros2023} and have spectral properties similar to those of an ADAF at the wavelengths probed here. This is also consistent with one of the most popular scenarios for the origin of the radio FP of LLAGN, suggesting that the correlation arises from strongly sub-Eddington jet-dominated emission \citep[e.g.][]{Falcke04,Plotkin12}. The contribution of compact radio jets to the nuclear SEDs of radiatively-efficient, quasar-like AGN is instead still hotly debated \citep[e.g.][]{Fawcett20,Girdhar22}. Our results are marginally consistent with these scenarios, as we find that compact jet models can explain the correlations for most of the LLAGN, but additional X-ray emitting components are required in the higher-luminosity systems. 

In short, we demonstrated that ADAF-like models convincingly predict the mmFP, which is not reproduced by a torus model. Compact jets are also an alternative plausible explanation (at least for LLAGN), but the corresponding models do not reproduce the correlation as naturally as the ADAF-like ones. We caution, however, that the plasma physics underlying both the ADAF and compact jet models is not well characterised, and significant uncertainties are present in all the model parameters and how they interact. We thus conclude that, while ``classic'' torus models seem to be ruled out, either ADAFs or compact jets have the potential to explain the observed trend. The presence of one (or more) of these mechanisms could even help explaining the increased far-infrared/sub-mm contribution attributed to AGN in some empirical SED models \citep[e.g.][]{Symeonidis2022}. The tight $L_{\rm X,2-10}$-$L_{\rm \nu, mm}$ correlation observed in Fig.\,\ref{fig:mm_xray_with_grid_and_skirtor} for the BASS sources is also consistent with the one recently reported by \citet{Ricci23} between the 100~GHz mm and 14-150~keV X-ray luminosities, and our results add interesting clues onto its origin. Determining with certainty the relevant mechanism(s) giving rise to the observed correlations is beyond the scope of this work, but is crucial to further our understanding of the SMBH accretion/ejection processes in different types of AGN.

Beyond carrying information on the nuclear physics, the correlations presented here provide new rapid methods to indirectly estimate the mass of SMBHs (or their accretion rates, if one has alternative, robust estimates of $M_{\rm BH}$ and $L_{\rm \nu, mm}$; see e.g.\,\citealp{Ricci23}). Although direct $M_{\rm BH}$ estimates can be obtained using a variety of techniques (e.g.\,stellar or gas kinematics, reverberation mapping), these typically require very time-consuming observational campaigns and currently have limited application beyond the local Universe. The ability to use nuclear mm and (optionally) X-ray luminosities has the huge advantage of allowing $M_{\rm BH}$ estimates when dynamical measurements are not possible and/or the standard scaling relations are unusable (such as in dwarf or disturbed galaxies). It also allows SMBH mass predictions over a wider range of redshifts. At the high-mass end of the reported correlations, ALMA can allow us to constrain $M_{\rm BH}$ up to $z\approx0.3$ (and is limited more by angular resolution and frequency coverage than sensitivity). Proposed new interferometers (such as the next-generation Very Large Array, ngVLA) should be able to push this to $z=1$ and beyond. Large X-ray surveys that can provide complementary X-ray data are also ongoing (e.g.\,eROSITA), and next-generation satellites (such as the Advanced Telescope for High ENergy Astrophysics, {\it Athena}) will extend these to higher-$z$. We also note that the intrinsic scatter of the overall $M_{\rm BH}-L_{\nu, \rm mm}$ relation is comparable to that of the $M_{\rm BH}-\sigma_{\star}$ relation \citep[e.g.][]{Vandenbosch16}, and $\sigma_{\rm int}$ of the mmFP in Fig.\,\ref{fig:fundamental_plane} is comparable or even lower than that of its radio counterpart (depending on the sample used to fit the plane; see e.g.\,\citealp[][]{Merloni03,Falcke04,Gultekin09,Plotkin12,Gultekin2019}). When restricting our analysis to only those primary sample sources with the most accurate (redshift-independent) distances, we obtain much tighter correlations (see Section~\ref{sec:distance_check}), with $\sigma_{\rm int}$ comparable to that of some of the tightest scaling relations in Astronomy (such as the Baryonic Tully-Fisher relation; e.g.\,\citealp{Lelli16,Lelli19}). This technique - if sufficiently verified - is thus well suited to constrain the details of SMBH-host galaxy co-evolution in regimes that have been difficult to access up to now \citep{Williams2023}.

%\vspace{-0.8cm}
\section*{Acknowledgements}
IR and TAD acknowledge support from grant ST/S00033X/1 through the UK Science and Technology Facilities Council (STFC). MB and TGW were supported by STFC consolidated grant `Astrophysics at Oxford' ST/H002456/1 and
ST/K00106X/1. DH acknowledges support from the Canada Research Chairs (CRC) program, the NSERC Discovery Grant program, and the Canadian Tri-Agency New Frontiers in Research -- Explorations fund. This paper makes use of ALMA data. ALMA is a partnership of ESO (representing its member states), NSF (USA) and NINS (Japan), together with NRC (Canada), NSC and ASIAA (Taiwan), and KASI (Republic of Korea), in cooperation with the Republic of Chile. The Joint ALMA Observatory is operated by ESO, AUI/NRAO and NAOJ. The National Radio Astronomy Observatory is a facility of the National Science Foundation operated under cooperative agreement by Associated Universities, Inc. This paper has also made use of the NASA/IPAC Extragalactic Database (NED) which is operated by the Jet Propulsion Laboratory, California Institute of Technology under contract with NASA. We acknowledge also the usage of the HyperLeda database.

%%%%%%%%%%%%%%%%%%%%%%%%%%%%%%%%%%%%%%%%%%%%%%%%%%
%\vspace{-0.8cm}
\section*{Data Availability}
The ALMA data used in this article are all available to download at the ALMA archive (\url{https://almascience.nrao.edu/asax/}). The calibrated data, final products and original plots generated for this research study will be shared upon reasonable request to the first author. The X-ray data have been retrieved from the catalogue of \citet{Bi20} or from the NASA/IPAC Extragalactic Database (NED; \url{https://ned.ipac.caltech.edu/}). The routines, codes and libraries of emission models used to fit the correlations and obtain the model SEDs are all publicly available at the following links: LLAGNSED at \url{https://github.com/dpesce/LLAGNSED}, \textsc{lts\_linefit} and \textsc{lts\_planefit} at \url{https://www-astro.physics.ox.ac.uk/~cappellari/software/#lts}, SKIRTOR at \url{https://sites.google.com/site/skirtorus/home}.
%\clearpage
%%%%%%%%%%%%%%%%%%%% REFERENCES %%%%%%%%%%%%%%%%%%

% The best way to enter references is to use BibTeX:

\bibliographystyle{mnras}
\bibliography{mybibliography} % if your bibtex file is called example.bib


% Alternatively you could enter them by hand, like this:
% This method is tedious and prone to error if you have lots of references
%\begin{thebibliography}{99}
%\bibitem[\protect\citeauthoryear{Author}{2012}]{Author2012}
%Author A.~N., 2013, Journal of Improbable Astronomy, 1, 1
%\bibitem[\protect\citeauthoryear{Others}{2013}]{Others2013}
%Others S., 2012, Journal of Interesting Stuff, 17, 198
%\end{thebibliography}

%%%%%%%%%%%%%%%%%%%%%%%%%%%%%%%%%%%%%%%%%%%%%%%%%%

%%%%%%%%%%%%%%%%% APPENDICES %%%%%%%%%%%%%%%%%%%%%

\appendix

\section{Data Table}

\begin{table*}
%\begin{footnotesize}
\caption{Full list and main parameters of the galaxies in the primary sample.}\label{tab:datatable}
\setlength{\tabcolsep}{3pt}
%\resizebox{\textwidth}{!}{%
\begin{tabular}{l l c c c c c c c c c c}
\hline
Sample & Galaxy & D & log$M_{\rm BH}$ & $\Delta {\rm log}M_{\rm BH}$ & Method & log$L_{\rm \nu, mm}$ & log$L_{\rm X,2-10}$ & $\Delta$log$L_{\rm X,2-10}$ \\ %& $\alpha_{mm}$ & $\Delta \alpha_{mm}$\\
 & & (Mpc) & (M$_{\odot}$) & (dex) & & (erg s$^{-1}$) & (erg s$^{-1}$) & (dex) \\ 
 (1) & (2) & (3) & (4) & (5) & (6) & (7) & (8) & (9) \\ %& (10) & (11) \\
\hline
WISDOM & FRL49 & 85.7 & 8.20 & 0.2 & Dyn & 39.28 & 43.27 & 0.04 \\ %& -1.3 & 1.1\\
 & FRL1146 & 136.7 & 7.85 & 0.30 & $\sigma_{\star}$ & $<38.76$ & 43.41 & 0.04 \\ %& -0.4 & 1.5 \\
 & MRK567 & 140.6 & 7.48 & 0.30 & $\sigma_{\star}$ & 39.39 & -- & -- \\ % & -- & -- \\
 & {\bf NGC0404} & 3.0 & 5.74 & 0.30 & Dyn & 35.99 & 37.20 & 0.04 \\ %& -- & --\\
 & NGC0449 & 66.3 & 8.77 & 0.30 & $\sigma_{\star}$ & 38.87 & 40.58 & 0.04 \\ %& 1.6 & 1.2 \\
 & NGC0524 & 23.3 & 8.60 & 0.32 & Dyn & 38.94 & 38.55 & 0.04 \\ %& -0.3 & 0.1 \\
 & {\bf NGC0708} & 58.3 & 8.30 & 0.30 & Dyn & 39.10 & 39.39 & 0.04 \\ %& -1.7 & 0.7 \\
 & NGC1194 & 53.2 & 7.85 & 0.10 & Dyn & 39.09 & 41.54 & 0.04 \\ %& 2.9 &  8.6\\
 & {\bf NGC1387} & 19.9 & 6.90 & 0.20 & Dyn & 38.07 & 39.33 & 0.04 \\ %& 0.0, & 1.0\\
 & NGC1574 & 19.3 & 8.05 & 0.20 & Dyn & 38.55 & -- & -- \\ %& 0.5 & 0.4\\
 & {\bf NGC2110} & 35.6 & 8.77 & 0.30 & $\sigma_{\star}$ & 39.89 & 42.71 & 0.04 \\ %& -0.5 & 0.4 \\
 & {\bf NGC3169} & 18.7 & 7.85 & 0.30 & $\sigma_{\star}$ & 38.53 & 41.53 & 0.04 \\ %& -0.6 & 1.0\\
 & NGC3351 & 10.0 & 5.85 & 0.30 & $\sigma_{\star}$ & $<37.25$ & 38.74 & 0.04 \\ %& -- & --\\
 & NGC3368 & 18.0 & 6.87 & 0.10 & Dyn & $<37.73$ & 39.30 & 0.05 \\ %& -- & --\\
 & {\bf NGC3607} & 22.2 & 8.14 & 0.15 & Dyn & 38.58 & 39.16 & 0.04 \\ %& 1.25\\
 & NGC4061 & 94.1 & 9.30 & 0.30 & Dyn & 39.78 & -- & -- \\ %& -2.4 & 1.0\\
 & NGC4429 & 16.5 & 8.17 & 0.10 & Dyn & 38.08 & 39.12 & 0.10 \\ %& -0.3 & 1.9\\
 & {\bf NGC4435} & 16.5 & 7.40 & 0.20 & Dyn & 37.76 & 39.47 & 0.04  \\ %& 1.2 & 0.7\\
 & {\bf NGC4438} & 16.5 & 7.70 & 0.30 & $\sigma_{\star}$ & 37.61 & 39.08 & 0.04 \\ %& -2.2 & 5.5 \\
 & {\bf NGC4501} & 14.0 & 6.79 & 0.30 & $\sigma_{\star}$ & 37.90 & 40.0.9 & 0.04 \\ %& 1.3 & 2.8\\
 & {\bf NGC4697} & 11.4 & 7.20 & 0.10 & Dyn & 37.25 & 38.52 & 0.04 \\ %& -2.1 & 3.4\\
 & NGC4826 & 7.4 & 6.20 & 0.11 & Dyn & 36.99 & 37.96 & 0.11 \\ %&  -- & -- \\
 & NGC5064 & 34.0 & 8.39 & 0.30 & $\sigma_{\star}$ & 37.96 & -- & -- \\ %& 1.2 & 3.0\\
 & {\bf NGC5765b} & 114.0 & 7.32 & 0.30 & $\sigma_{\star}$ & 39.06 & 40.73 & 0.05 \\ %& -- & --\\
 & NGC5806 & 21.4 & 6.95 & 0.30 & $\sigma_{\star}$ & $<37.26$ & -- & -- \\ %& -- & --\\
 & {\bf NGC5995} & 107.5 & 8.55 & 0.30 & $\sigma_{\star}$ & 39.51 & 43.39 & 0.04 \\ %& -1.5 & 1.0\\
 & NGC6753 & 42.0 & 8.42 & 0.30 & $\sigma_{\star}$ & $<37.82$ & -- & -- \\ %& -- & -- \\
 & NGC6958 & 30.6 & 7.89 & 0.30 & $\sigma_{\star}$ & 39.47 & -- & -- \\ %& 0.4 & 0.2\\
 & {\bf NGC7052} & 51.6 & 8.91 & 0.30 & $\sigma_{\star}$ & 40.13 & 40.02 & 0.05 \\ %& -0.1 & 0.2 \\
 & {\bf NGC7172} & 33.9 & 8.05 & 0.30 & $\sigma_{\star}$ & 39.44 & 43.00 & 0.05 \\ %& -0.83 & 0.68\\
 & PGC043387 & 95.8 & 8.42 & 0.30 & $\sigma_{\star}$ & $38.90$ & -- & -- \\ %& -- & --\\
\hline
Literature & {\bf Circinus} & 4.2 & 6.23 & 0.08 & Dyn & 37.57 & 42.20 & 0.04 \\ %& --& --\\
 & {\bf IC1459} & 25.9 & 9.45 & 0.30 & Dyn & 40.60 & 40.91 & 0.10 \\ %& --& --\\
 & {\bf M87} & 16.5 & 9.81 & 0.04 & Dyn & 40.75 & 40.56 & 0.05 \\ %& --& --\\
 & {\bf NGC1316} & 19.9 & 8.23 & 0.08 & Dyn & 38.73 & 39.65 & 0.04 \\ %& --& --\\
 & {\bf NGC1332} & 22.3 & 8.82 & 0.04 & Dyn & 38.91 & 39.15 & 0.08 \\ %& --& --\\
 & {\bf NGC1380} & 17.1 & 8.17 & 0.17 & Dyn & 38.53 & 39.07 & 0.04 \\ %& --& --\\
 & NGC3227 & 17.0 & 7.18 & 0.17 & Dyn & 37.70 & 41.95 & 0.09 \\ %& --& --\\
 & NGC3245 & 21.5 & 8.32 & 0.10 & Dyn & 38.42 & 38.88 & 0.09 \\ %& --& --\\
 & NGC3393 & 53.6 & 7.52 & 0.03 & Dyn & 38.34 & 40.92 & 0.09 \\ %& --& --\\
 & NGC3489 & 12.0 & 6.78 & 0.07 & Dyn & 36.77 & 39.07 & 0.09 \\ %& --& --\\
 & NGC3504 & 13.6 & 7.00 & 0.07 & Dyn & 38.38 & 41.05 & 0.09 \\ %& --& --\\
 & {\bf NGC3585} & 20.6 & 8.52 & 0.13 & Dyn & 37.89 & 38.85 & 0.04 \\ %& --& --\\
 & {\bf NGC4374} & 18.5 & 8.96 & 0.05 & Dyn & 39.84 & 39.95 & 0.04 \\ %& --& --\\
 & {\bf NGC4388} & 19.8 & 6.94 & 0.01 & Dyn & 38.88 & 42.33 & 0.05 \\ %& --& --\\
  & {\bf NGC4395} & 4.4 & 5.60 & 0.31 & Dyn & 35.85 & 39.91 & 0.05 \\ %& --& --\\
 & {\bf NGC6861} & 27.3 & 9.30 & 0.22 & Dyn & 39.52 & 39.29 & 0.04 \\ %& -- & --\\
& {\bf UGC2698} & 91.0 & 9.39 & 0.12 & Dyn & 39.53 & 40.46 & 0.09 \\ %& --& --\\ 
\hline
\end{tabular}
%\end{footnotesize}
\parbox[t]{\textwidth}{\textit{Notes:} (1) Sub-sample. (2) Galaxy name. (3) Most accurate galaxy distance. (4), (5) and (6) SMBH mass, uncertainty, and measurement method. ``Dyn'' refers to dynamical measurements, ``$\sigma_{\star}$'' to estimates from the $M_{\rm BH}$ --  $\sigma_{\star}$ relation of \cite{Vandenbosch16}. (7) Millimetre luminosity. Errors are not reported because they are simply dominated by flux calibration uncertainties, which are $\approx$10\% for ALMA Band 6 (see Section~\ref{sec:methods-sample}). (8) and (9) $2-10$~keV X-ray luminosity, and associated error. Galaxies highlighted in bold have their X-ray luminosity taken from the catalogue of \citet{Bi20}.}
\end{table*}

\section{The scatter of the mm FP}
We show here the residuals of our primary sample and BASS sources from the best-fitting mmFP, plotted as a function of the SMBH mass measurement method. This is important to get insights into the scatter of the two populations around the correlation (see Section~\ref{sec:methods-bass} for details).
% Figure environment removed
 



%%%%%%%%%%%%%%%%%%%%%%%%%%%%%%%%%%%%%%%%%%%%%%%%%%


% Don't change these lines
\bsp	% typesetting comment
\label{lastpage}
\end{document}

% End of mnras_template.tex
