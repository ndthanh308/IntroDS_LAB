%%%%%%%%%%%%%%%%%%%%%%%%%%%%%%%%%%%%%%%%%%%%%%%%%%%%%%%%%%%%%%%%%%%%%%%%%%%%
%% Trim Size: 9.75in x 6.5in
%% Text Area: 8in (include Runningheads) x 5in
%% ws-ijmpa.tex   :   06-04-2015
%% Tex file to use with ws-ijmpa.cls written in Latex2E.
%% The content, structure, format and layout of this style file is the
%% property of World Scientific Publishing Co. Pte. Ltd.
%% Copyright 2015 by World Scientific Publishing Co.
%% All rights are reserved.
%%%%%%%%%%%%%%%%%%%%%%%%%%%%%%%%%%%%%%%%%%%%%%%%%%%%%%%%%%%%%%%%%%%%%%%%%%%%
%%

%\documentclass[draft]{ws-ijmpa}
\documentclass{ws-ijmpa}
\usepackage[super,compress]{cite}
\usepackage{graphicx}
\begin{document}
\markboth{Jiajun Wu, Chao-Qiang Geng, Da Huang}{$W$-boson Mass Anomaly from High-Dimensional Scalar Multiplets}

%%%%%%%%%%%%%%%%%%%%% Publisher's Area please ignore %%%%%%%%%%%%%%%
%
\catchline{}{}{}{}{}
%
%%%%%%%%%%%%%%%%%%%%%%%%%%%%%%%%%%%%%%%%%%%%%%%%%%%%%%%%%%%%%%%%%%%%

\title{$W$-boson Mass Anomaly from High-Dimensional Scalar Multiplets}

\author{Jiajun Wu and Chao-Qiang Geng%\footnote{
%No.1 Xiangshan Branch Lane, Zhuantang Street, Xihu District, Hangzhou City, Zhejiang Province, China
%Typeset names in 8 pt roman, uppercase. Use the footnote to indicate the present or permanent address of the author.}
}

\address{School of Fundamental Physics and Mathematical Sciences, Hangzhou Institute for Advanced Study, UCAS, Hangzhou 310024, Zhejiang Province, China%\footnote{State completely without abbreviations, the affiliation andmailing address, including country. Typeset in 8 pt italic.}
\\
University of Chinese Academy of Sciences (UCAS), Beijing 100049, China\\
International Centre for Theoretical Physics Asia-Pacific, Beijing/Hangzhou, China}
%\\
%wujiajun@itp.ac.cn}

\author{Da Huang}

\address{National Astronomical Observatories, Chinese Academy of Sciences, Beijing 100012, China\\
School of Fundamental Physics and Mathematical Sciences, Hangzhou Institute for Advanced Study, UCAS, Hangzhou 310024, Zhejiang Province, China\\
International Centre for Theoretical Physics Asia-Pacific, Beijing/Hangzhou, China}
%dahuang@bao.ac.cn}

%\author{Chao-Qiang Geng}

%\address{School of Fundamental Physics and Mathematical Sciences, Hangzhou Institute for Advanced Study, UCAS\\
%	Hangzhou 310024, Zhejiang Province, China\\
%	cqgeng@ucas.ac.cn}

\maketitle

\begin{history}
\received{Day Month Year}
\revised{Day Month Year}
\end{history}

\begin{abstract}
In light of the recently discovered $W$-boson mass anomaly by the CDF Collaboration, we discuss two distinct mechanisms that could possibly explain this anomaly through the introduction of $SU(2)_L$ scalar multiplets. The first mechanism  is by the tree-level $W$-boson mass correction, induced by the vacuum expectation values of one or more $SU(2)_L$ scalar multiplets with odd dimensions of $n\geq 3$ and zero hypercharge of
$Y=0$ to avoid the strong constraint from  measurements of the $Z$-boson mass. However, it remains ruled out by the electroweak precision data of the $\rho$ parameter. The second mechanism is by the one-loop level $W$-boson mass correction. In particular, we focus on the case with an additional scalar nonet with $Y=0$ or $Y=4$. As a result, we find that the model can interpret the $W$-boson mass anomaly without violating any other theoretical or experimental constraints.

% it is available of introducing an additional scalar nonet with $Y=0$ or $4$ to explain the $W$-boson mass anomaly without violating other theoretical and experimental constraints.


\keywords{W-boson Mass Anomaly; Scalar Multiplet; CDF Collaboration.}
\end{abstract}

\ccode{PACS numbers:}

%\tableofcontents

\section{Introduction}	

Recently, the CDF-\uppercase\expandafter{\romannumeral 2} Collaboration has reported the most precise measurement of the $W$-boson mass, showing that the observed $W$-boson mass of $m^{\rm CDF}_W = 80433.5\pm9.4$~MeV~\cite{CDF:2022hxs} deviates the latest Standard Model (SM) prediction of $m^{\rm SM}_W= 80357\pm 6$~MeV~\cite{ParticleDataGroup:2022pth}. The significance of this anomaly is more than $7\sigma$, which indicates new physics (NP) beyond the SM (BSM). Therefore, it is a natural and pressing concern to introduce NP models to account for this anomalous $W$-boson mass.

Among a multitude of BSM scenarios, extending the SM Higgs sector by incorporating extra $SU(2)_L$ multiplets~\cite{Sakurai:2022hwh,Peli:2022ybi,Dcruz:2022dao,Asai:2022uix,Fan:2022dck,Lu:2022bgw,Song:2022xts,Bahl:2022xzi,Babu:2022pdn,Heo:2022dey,Ahn:2022,Ghorbani:2022vtv,Lee:2022gyf,Abouabid:2022lpg,Benbrik:2022dja,Botella:2022rte,Kim:2022hvh,Kim:2022xuo,Appelquist:2022qgl,Benincasa:2022elt,Arhrib:2022inj,Han:2022juu,Abdallah:2022shy,deGiorgi:2023wjh,Abouabid:2023mbu,Barrie:2022cub,Cheng:2022jyi,Du:2022brr,FileviezPerez:2022lxp,Kanemura:2022ahw,Mondal:2022xdy,Borah:2022obi,Addazi:2022fbj,Heeck:2022fvl,Chen:2022ocr,Evans:2022dgq,Ghosh:2022zqs,Ma:2022emu,Bahl:2022gqg,Penedo:2022gej,Cheng:2022hbo,Butterworth:2022dkt,Wu:2022uwk,Song:2022jns,Crivellin:2023gtf,Ellis:2023zim,Shimizu:2023rvi} is a promising avenue, since the modification of the scalar sector is intricately connected to the underlying mechanism for the electroweak (EW) gauge symmetry breaking and the related hierarchy problem, which can be studied through the measurement of EW oblique parameters~\cite{Peskin:1990zt,Peskin:1991sw,Maksymyk:1993zm,Burgess:1993mg,Lavoura:1993nq,Albergaria:2021dmq}. Furthermore, the added scalar multiplet has the potential to resolve many puzzles in the SM, such as the nature of dark matter (DM)~\cite{Cirelli:2005uq,Cirelli:2009uv,Guo:2010hq,Barbieri:2006dq,LopezHonorez:2010eeh,Gonderinger:2012rd}, the origin of the matter-antimatter asymmetry~\cite{Cline:2012hg,Grzadkowski:2018nbc,Cline:2021iff,Morrissey:2012db}, and the characteristics of the EW phase transition as well as its related stochastic gravitational wave signals~\cite{Chowdhury:2011ga,Hashino:2018zsi,Chiang:2017nmu,Kannike:2019wsn,Chiang:2020yym,Chiang:2019oms,Cai:2017tmh,Chao:2017vrq,Ellis:2018mja,Alves:2018jsw,Zhou:2018zli,Bian:2019kmg,Ghosh:2020ipy,Zhou:2020irf,Lu:2022zpn,Zhou:2022mlz,Cai:2022bcf,Hashino:2018wee}. Therefore, comprehending the structure of the scalar sector could lead to a more profound understanding of the big picture of the SM and the physics beyond it. 
Extensive studies have been carried out in the literature to explain the CDF-\uppercase\expandafter{\romannumeral 2} $W$-boson mass anomaly with low-dimensional scalar multiplets, which include a scalar singlet~\cite{Sakurai:2022hwh,Peli:2022ybi,Dcruz:2022dao,Asai:2022uix}, a second Higgs doublet~\cite{Fan:2022dck,Lu:2022bgw,Song:2022xts,Bahl:2022xzi,Babu:2022pdn,Heo:2022dey,Ahn:2022,Ghorbani:2022vtv,Lee:2022gyf,Abouabid:2022lpg,Benbrik:2022dja,Botella:2022rte,Kim:2022hvh,Kim:2022xuo,Appelquist:2022qgl,Benincasa:2022elt,Arhrib:2022inj,Han:2022juu,Abdallah:2022shy,deGiorgi:2023wjh,Abouabid:2023mbu}, and a scalar triplet~\cite{Barrie:2022cub,Cheng:2022jyi,Du:2022brr,FileviezPerez:2022lxp,Kanemura:2022ahw,Mondal:2022xdy,Borah:2022obi,Addazi:2022fbj,Heeck:2022fvl,Chen:2022ocr,Evans:2022dgq,Ghosh:2022zqs,Ma:2022emu,Bahl:2022gqg,Penedo:2022gej,Cheng:2022hbo,Butterworth:2022dkt,Song:2022jns,Crivellin:2023gtf,Ellis:2023zim,Shimizu:2023rvi}. More recently, we have explored scalar multiplet scenario up to a maximum of a septuplet~\cite{Wu:2022uwk}. 
In this letter, we aim to explain the $W$-boson mass anomaly with higher dimensional multiplets, at both tree and one-loop levels. In particular, for the one-loop solution, we shall focus on the case with a scalar nonet of $Y=0$ or $Y=4$, which complements the previous studies. %on low-dimensional scalar multiplets by exploring 
%to even higher dimensional multiplets, specifically a scalar nonet with $Y=0$ and $Y=4$, at both tree-level and one-loop level, %which is a valuable and non-trivial extension of such research. 


This paper is organized as follows: In Sec.~\ref{s2}, we investigate the possibility of the tree-level explanation of the CDF-\uppercase\expandafter{\romannumeral 2} $W$-boson mass excess with a high-dimensional multiplet. In Sec.~\ref{s3}, we present a detailed phenomenological study of a scalar nonet with $Y=0$ or $Y=4$, which is of physical interest to explain the CDF-\uppercase\expandafter{\romannumeral 2} $W$-boson mass anomaly at the one-loop level. We conclude in Sec.~\ref{s4}.


\section{Tree-Level Explanation of $W$-boson Mass Anomaly}\label{s2}
One may easily explain the $W$-boson mass anomaly by introducing a $SU(2)_L$ scalar multiplet with its vacuum expectation value (VEV) of the neutral component inducing an additional mass correction to the $W$-boson mass. However, such an idea is hampered by the fact that, since the $W$ and $Z$-bosons have the common origin from the EW gauge symmetry breaking, the associated $Z$-boson mass should also be corrected, which was strongly constrained by the current experiments~\cite{ParticleDataGroup:2022pth}. In this section, we show that if the added scalar multiplet is in an odd-dimensional presentation of $SU(2)_L$ with zero hypercharge, the above problem can be resolved automatically.

Let us begin by considering a $SU(2)_L$ multiplet $\xi$ of dimension $n=2k+1$ with $k$ as a positive integer denoting the weak isospin $SU(2)_L$ representation. The hypercharge of $\xi$ is fixed to $Y=0$. When $\xi$ and the SM Higgs doublet obtain their VEVs, $v_\xi$, and $v_H$, from the spontaneous breaking of the EW gauge symmetry, the $W$ and $Z$-boson mass terms can be written as follows
\begin{eqnarray}
	D^\mu H^\dagger D_\mu H + D^\mu \xi^\dagger D_\mu \xi \supset &-&\left(\frac{1}{4} g^2 v_H^2 + \frac{1}{2} k(1+k)v_\xi^2 \right) W_\mu^+ W^{-\,\mu} \nonumber \\ 
	&-& \frac{1}{8} (g^2 + g^{\prime \,2}) v_H^2 Z_\mu Z^\mu\,, 
\end{eqnarray}
where $D$ denotes the covariant derivatives of the scalar fields $H$ and $\xi$ with $g$ and $g^\prime$ the SM $SU(2)_L$ and $U(1)_Y$ gauge couplings, respectively. 
%Note that in the above formula, we have used the normalization of the $SU(2)_L$ generators as that in Ref.~\cite{Georgi:1999wka}. 
Therefore, the SM gauge boson masses are given by
\begin{eqnarray}
	m_W= \frac{1}{2} g\sqrt{v_H^2+2k(1+k)v_\xi^2}\,,\quad m_Z= \frac{v_H}{2}\sqrt{g^2 + g^{\prime\,2}}\,,
\end{eqnarray} 
indicating that the $Z$-boson mass is not corrected in the presence of the extra multiplet scalar VEV of $v_\xi$. It can be understood by the fact that, for $Y=0$, the couplings of various components in the multiplet with the $Z$-boson are proportional to their electric charges, so that the neutral component and the associated VEV cannot interact with the $Z$ boson. As a consequence, this scalar-multiplet-extended model can avoid the strong constraint from the $Z$-boson mass measurement~\cite{ParticleDataGroup:2022pth}. Further, if we take the CDF-\uppercase\expandafter{\romannumeral 2} value of the $W$-boson mass as the one in our model, then the VEV of the multiplet can be estimated as follows
\begin{eqnarray}\label{mW2sig}
	\frac{(\Delta v)^2}{v_H^2} \equiv \frac{2k(1+k)v_\xi^2}{v_H^2}  =  \left(\frac{m_W^{\rm CDF} }{m_W^{\rm SM}}\right)^2-1 \sim ~ [0.00090, 0.00201]\,,  \quad \mbox{at $2\sigma$ C.L.}\,,
\end{eqnarray}
where the SM Higgs doublet VEV is taken to be $v_H = 246.22$~GeV~\cite{ParticleDataGroup:2022pth}. Moreover, note that the model can be further extended by introducing a series of $SU(2)_L$ scalar multiplets with vanishing hypercharges. In this case, we can still keep the salient feature that the $Z$-boson mass is not modified so that the related constraint is weak.

One additional constraint on the added scalar multiplet is provided by the measurement of  the $\rho_0$ parameter with its current value given by~\cite{ParticleDataGroup:2022pth}
\begin{eqnarray}\label{RhoValue}
	\rho_0 \equiv \frac{m_W^2}{m_Z^2 \hat{c}_Z^2 \hat{\rho}} = 1.00038 \pm 0.00020,
\end{eqnarray}
where $\hat{c}_Z$ is the cosine of the Weinberg angle at the $Z$-boson mass pole, while $\hat{\rho}$ is a dimensionless parameter. Theoretically, our model predicts this important quantity as follows~\cite{ParticleDataGroup:2022pth}
\begin{eqnarray}
	\rho_0 = 1+ \frac{2k(1+k)v_\xi^2}{v_H^2} \,.
\end{eqnarray} 
According to Eq.~(\ref{RhoValue}), we can obtain the following $2\sigma$ region for $(\Delta v)^2/v_H^2$
\begin{eqnarray}\label{ConsRho}
	\frac{(\Delta v)^2}{v_H^2} \in [-0.00002, 0.00078]\,,\quad \mbox{at $2\sigma$ C.L.}\,.
\end{eqnarray}
Obviously, the multiplet VEV range in Eq.~(\ref{ConsRho}) allowed by the $\rho_0$ parameter constraints and the CDF $W$-boson mass signal region in Eq.~(\ref{mW2sig}) do not overlap each other, which implies that the CDF $m_W$ anomaly cannot be solved in this multiplet scalar extension of the SM. Note that the violation of the $\rho$ parameter constraint can be traced back to the fact that the custodial symmetry in the SM is badly violated in the presence of $\xi$. As a result, we conclude that even though the introduction of the scalar multiplet $\xi$ with zero hypercharge can explain the recently observed CDF-\uppercase\expandafter{\romannumeral 2} $W$-boson mass anomaly with its nonzero VEV while satisfying the strong $Z$-boson mass constraint, the model is still ruled out by the $\rho$ parameter measurement. Thus, we shall not consider this tree-level explanation to the $W$-boson mass anomaly any more.  %Consequently, the CDF-\uppercase\expandafter{\romannumeral 2} $W$-boson mass anomaly cannot be explained solely at the tree level.


\section{One-Loop Level Explanation of the $W$-boson Mass Anomaly}\label{s3}
In this section, we discuss the explanation of the $W$-boson mass anomaly with the one-loop corrections from the addition of high-dimensional scalar multiplet fields. %at the one-loop level. Furthermore, 
After a general review of the relation between EW oblique parameters and the one-loop $W$-boson mass correction, we shall focus on the scalar nonet case with hypercharge $Y=0$ or $Y=4$ as a concrete example.% a detailed phenomenological study of a scalar nonet explaining the CDF $W$-boson mass anomaly.

\subsection{Oblique Parameters and the $W$-Boson Mass}
The NP effects in the EW sector are usually encoded by the three oblique parameters, namely $S$, $T$, and $U$~\cite{Peskin:1991sw,Peskin:1990zt}, which are defined at the one-loop level as follows: %These oblique parameters at the one-loop level can be defined as follows:
\begin{eqnarray}
	S&\equiv&\frac{4s_{W}^{2}c_{W}^{2}}{\alpha}\left[A_{ZZ}^{\prime}\left(0\right)-\frac{c_{W}^{2}-s_{W}^{2}}{c_{W}s_{W}}A_{Z\gamma}^{\prime}\left(0\right)-A_{\gamma\gamma}^{\prime}\left(0\right)\right]\,, \nonumber\\
	T&\equiv&\frac{1}{\alpha m_{Z}^{2}}\left[\frac{A_{WW}\left(0\right)}{c_{W}^{2}}-A_{ZZ}\left(0\right)\right]\,, \nonumber\\
	U&\equiv&\frac{4s_{W}^{2}}{\alpha}\left[A_{WW}^{\prime}\left(0\right)-\frac{c_{W}}{s_{W}}A_{Z\gamma}^{\prime}\left(0\right)-A_{\gamma\gamma}^{\prime}\left(0\right)\right]-S\,,
\end{eqnarray}
where the functions $A^{(\prime)}_{VV^\prime} (q^2)$ refer to the vacuum polarizations for EW gauge bosons $V^{(\prime)} = \gamma,\, W,\,Z$. 
Moreover, as shown in Refs.~\cite{Fan:2022dck,Peskin:1991sw}, the general one-loop corrections of non-SM scalars to the $W$-boson mass squared can be expressed in terms of these oblique parameters $S$, $T$, and $U$ as follows
\begin{eqnarray}\label{00}
	\Delta m_W^2 = \frac{\alpha c_W^2 m_Z^2}{c_W^2 - s_W^2} \left[ -\frac{S}{2}  + c_W^2 T + \frac{c_W^2 - s_W^2}{4s_W^2} U \right]\,,
\end{eqnarray}
where $s_W$ ($c_W$) are (co)sine of the Weinberg angle. 

However, as demonstrated in Ref.\cite{Asadi:2022xiy}, the parameters $T$ and $S$ are usually generated by dimension-6 operators, while $U$ can only be induced by a dimension-8 operator, resulting in its significant suppression. Therefore, in a NP model augmented by a scalar multiplet, it is expected that the dominant one-loop contribution to the $W$-boson mass arises from $T$ and $S$, and the impact of $U$ can be neglected. Besides, Ref.~\cite{Wu:2022uwk} has explicitly confirmed this expectation by demonstrating that when the masses of extra scalars exceed 300~GeV and their multiplet dimensions are restricted to be smaller than 10, the correction of $U$ to the $W$ mass is at least one order of magnitude smaller than the leading ones from $T$ and $S$. %, which explicitly confirms the previously stated expectation. 
According to Refs.\cite{Wu:2022uwk,Lavoura:1993nq}, the contribution of a scalar multiplet $\xi$ to $T$ can be expressed as:
\begin{equation}\label{T}
	T_{\xi}=\frac{1}{4\pi s_{w}^{2}m_{W}^{2}}\sum_{I=-k}^{k-1}N_{I+1}^{2}F\left(m_{\xi_{I}^{Q}}^{2},m_{\xi_{I+1}^{Q}}^{2}\right)\,,
\end{equation}
where $I=k,k-1,k-2,......,-k+1,-k$, $N_{I}=\sqrt{{(k+I)(k-I+1)}/{2}}$, $Q=I+Y$ is referred to the electric charge, and the function $F(A,B)$ is given by
\begin{equation}\label{Ffunc}
	F\left(A,B\right)\equiv\begin{cases}
		\frac{A+B}{2}-\frac{AB}{A-B}\ln\frac{A}{B}\,, & A\neq B\,,\\
		0\,, & A=B\,.
	\end{cases}
\end{equation}
On the other hand, the correction of $\xi$ to $S$ can be expressed as follows:
\begin{equation}\label{S}
	S_{\xi}=-\frac{Y}{3\pi}\sum_{I=-k}^{k}I\ln m_{\xi_{I}^{Q}}^{2}\,,
\end{equation}
where $\xi_{I}^{Q}$ represents components of the scalar multiplet $\xi$ with its electric charge $Q$.


\subsection{Scalar Nonet Explanation of the $W$-boson Mass Anomaly}
Ref.~\cite{Wu:2022uwk} has shown that a real scalar multiplet is unable to generate  the mass splitting to produce nonzero values of $T$ and $S$, which is required to explain the $W$-mass anomaly. In the following, we will focus on the complex nonet scalar case, which has not been discussed so far in the literature. % in which the mass splitting can occur. 

Note that the potential in this model can be constructed as follows with the Higgs doublet $H$ and the scalar nonet $\xi$
\begin{equation}\label{potential}
	\begin{aligned}
		V\left(H,\xi\right)=&-\mu_{H}^{2}H^{\dagger}H+\lambda_{H}\left(H^{\dagger}H\right)^{2}+\mu_{\xi}^{2}\xi^{\dagger}\xi+\lambda_{1}\left(\xi^{\dagger}\xi\right)^{2} +\lambda_{2}\left(\xi^{\dagger}T_{\Phi}^{a}\xi\right)^{2} \\
		&+\lambda_{3}\left(\xi^{\dagger}\xi\right)\left(H^{\dagger}H\right)
		+\lambda_{4}\left(\xi^{\dagger}T_{\xi}^{a}\xi\right)\left(H^{\dagger}T_{H}^{a}H\right)+\lambda_{5}\left(\xi^{\dagger}T_{\xi}^{a}T_{\xi}^{b}\xi\right)^{2}\,.
	\end{aligned}
\end{equation}
Here, we present the most generic interaction terms for a scalar nonet $\xi$. It is worth noting that the potential given in Eq.(\ref{potential}) exhibits an $Z_2$ symmetry that arises naturally without the need for any additional assumptions. 
Note that the mass splitting can be only generated in Eq.~(\ref{potential}) by the following term:
\begin{eqnarray}\label{O4}
	O_{4}=\lambda_{4}\left(\xi^{\dagger}T_{\xi}^{a}\xi\right)\left(H^{\dagger}T_{H}^{a}H\right)\,.
\end{eqnarray}
%the scalar multiplet will give no contribution to the parameters $T$ and $S$, as well as the $W$-boson mass, if there are not mass splittings among components in $\xi$. 
Therefore, we will concentrate on this term in our subsequent phenomenological studies. Additionally, it is worth noting that $|\lambda_4|$ is also constrained by perturbativity with $|\lambda_4| < 4\pi$~\cite{Nebot:2007bc}. Hence, In our further discussions, we take into account this perturbative limit by requiring $|\lambda_4| \leq 10$.



\subsubsection{Scalar Nonet with $Y=0$}
According to Eq.~(\ref{S}), the scalar nonet $\xi$ does not contribute to $S$ when $Y=0$. Thus, its one-loop correction to the $W$-boson mass can be expressed solely in terms of $T$:
\begin{eqnarray}
	\Delta m_W^2 = \frac{\alpha c_W^2 m_Z^2}{c_W^2 - s_W^2} c_W^2 T \,.
\end{eqnarray}

Fig.~\ref{f1} displays the parameter spaces in the $M_{L}$-$\sqrt{|\Delta M^{2}|}$ plane for a complex scalar nonet model with $Y=0$. The horizontal axis denotes the mass of the lightest particle in the nonet $M_{L}$, ranging from 300 GeV to 3000 GeV, while the vertical axis represents the mass splitting between adjacent components. The pink region in the figure indicates the parameter space allowed by the EW global fit for $T$ at the $2\sigma$  CL when $S=U=0$, as reported in Ref.~\cite{Cheng:2022hbo}. The cyan area  corresponds to the parameter space that can explain the measured $W$-boson mass by  CDF-\uppercase\expandafter{\romannumeral2} within the $2\sigma$ CL.%, and the shaded regions represent the parameter space that satisfies the corresponding requirements. 
The red line shows the mass difference corresponding to the perturbative limit  $|\lambda_{4}|=10$, and the region below this line can satisfy the perturbativity condition.
% Figure environment removed
As a result, it is illustrated in Fig.~\ref{f1} that the model featuring an additional complex scalar nonet of $Y=0$ exhibits an adequate parameter space to account for the excess of the $W$-boson mass by its one-loop contribution, while it is still allowed by relevant experimental and theoretical constraints.


\subsubsection{Scalar Nonet with $Y$=4}
In this subsection, we shall turn to the model with a scalar nonet $\xi$ of $Y$=4. The phenomenology can be classified into the following two distinct types based on the sign of $\lambda_4$:
\begin{itemlist}
	\item Type A: when $\lambda_4 >0$, the lightest particle in the nonet is the most electrically charged one with its mass denoted as $M_C$, so that $M_L = M_C$.
	\item Type B: when $\lambda_4 <0$, the lightest particle in the nonet is the electrically neutral one with its mass denoted as $M_0$, indicating $M_L = M_0$.
\end{itemlist}
In light of the mass splittings among scalars from $O_4$, this $Y=4$ scalar nonet has the potential to explain the $W$-mass anomaly with its following nonzero corrections to the parameters $T$ and $S$: %with $U$ held constant at a fixed value of zero. 
%The one-loop corrections of additional nonet in $Y=4$ case to the $W$-boson squared can be expressed in terms of $T$ and $S$, as shown below:
\begin{eqnarray}
	\Delta m_W^2 = \frac{\alpha c_W^2 m_Z^2}{c_W^2 - s_W^2} \left[ -\frac{S}{2}  + c_W^2 T \right]\,.
\end{eqnarray}
% Figure environment removed
% Figure environment removed

Figs.~\ref{f2} and \ref{f3} display the parameter spaces in the $M_L$-$\sqrt{\Delta m^2}$ plane for the Type-A and Type-B scalar nonet models, respectively. The color coding is identical to that of Fig.~\ref{f1}, but now the pink area representing the $2\sigma$ CL region for the oblique parameters $T$ and $S$ has been derived from the EW global fits depicted as a red ellipse in Fig.~1 of Ref.\cite{Asadi:2022xiy}. The results show that the Type-A model has a substantial amount of parameter regions that can solve the CDF-\uppercase\expandafter{\romannumeral2} $m_W$ anomaly while satisfying the EW global fits and perturbative limits. But the region with $M_L \leq 1800$ GeV is disfavored by the EW global fits. On the other hand, for the Type-B model, it is seen from Fig.~\ref{f3} that the CDF-\uppercase\expandafter{\romannumeral2} preferred region that explains the $W$-boson mass excess is entirely ruled out by the global fits of EW precision observables.








\section{Conclusion and Discussion}\label{s4}
In order to explain the excess of the $W$-boson mass recently observed by the CDF-\uppercase\expandafter{\romannumeral2} Collaboration, we have studied two promising candidate scenarios both involving high-dimensional $SU(2)_L$ scalar multiplets. %  detailed and concrete study to explain this anomaly in terms of high-dimensional $SU(2)_{L}$ scalar multiplets, considering the tree-level and one-loop level effects respectively. 
The first scenario is to introduce an odd-dimensional scalar multiplet with $Y=0$ so that the $W$ mass correction from its nonzero scalar VEV can explain the CDF-\uppercase\expandafter{\romannumeral 2} excess at the tree level. One salient feature of this mechanism is that the scalar VEV 
does not contribute to the mass of $Z$ boson, so that it can easily evade the associated strong constraint from $m_Z$. However, it is found that the scenario is ruled out by the precise measurement of the $\rho$ parameter.  
% we have considered the scalar multiplet solution at the tree level, the excess of the $W$-boson mass can be explained by introducing an odd-dimensional $Y=0$ scalar multiplet with a non-zero vacuum expectation value that leaves the mass of the $Z$-boson unaffected. However, this scenario is constrained by the EW precise measurement of the $\rho$ parameter, rendering the problem unsolvable at the tree level alone. 
Then, we turn to the second scenario in which the CDF-\uppercase\expandafter{\romannumeral 2} anomaly could be possibly solved by one-loop effects generated by the additional scalar multiplet. We take a complex scalar nonet as a concrete example to investigate this possibility. By making use of the nonet scalars' contributions to the oblique parameters $S$ and $T$, it is found that the CDF measured $W$-boson mass can be easily obtained. We also take into account the constraints from the EW global fits and the perturbativity limit in this model. 
% one-loop , after considering the one-loop effects, the problem becomes tractable. The newly introduced scalar multiplet could correct the mass of the $W$-boson by contributing to the oblique parameters $S$ and $T$, resulting in a $W$-boson mass consistent with the measurement of the CDF Collaboration. We investigate this case phenomenologically by introducing a complex scalar nonet and taking into account constraints from EW global fits and perturbation limit. 
As a result, we have found that the scalar nonet with $Y=0$ could provide a considerable parameter space to account for the anomalous $W$-boson mass, and the model remains valid when the lightest scalar mass $M_L$ is in the range from 500 to 4000 GeV.
For the $Y=4$ case, the current data from the EW precision measurements has excluded the Type-B model in which the lightest particle is the electrically neutral one. % is completely excluded within the parameter space of $M_L$ from 500 to 4000 GeV by the EW global fit. Conversely,
In contrast, for the Type-A model in which the lightest scalar is the most charged one, it is found that there is a substantial parameter region explaining the CDF $m_W$ anomaly while still allowed by the experimental and theoretical constraints. In the latter case, the lightest scalar mass is found to lie in the range from 1800 to 4000 GeV.% the Type-A model with $M_L=M_C$ can survive within the range of $M_L$ from 1800 to 4000 GeV as well as fulfill the perturbation requirement. The parameter space below 1800 GeV is excluded by the electroweak global fit.
We have exclusively investigated two representative cases, namely the high-multiplet scenarios with solely tree-level corrections and one-loop effects. In more generic cases, both mechanisms should work to solve the $W$-boson mass anomaly like a model added by a scalar triplet with $Y=0$~\cite{Cheng:2022hbo,FileviezPerez:2022lxp,Song:2022jns}. 
% Besides these two extreme cases, our results indicate that for an odd-dimensional scalar multiplet with $Y=0$, the anomalous $W$-boson mass can be explained by simultaneously considering the effects of both tree-level (i.e., introducing a sufficiently small vacuum expectation value) and one-loop corrections. 
Finally, we would like to mention that, for the available nonet models, such as the Type-A model with $Y=4$, due to the accidental $Z_2$ symmetry inherent in the model, the lightest multi-charged states should be relatively long-lived, which could potentially serve as key signals for future collider searches~\cite{Wu:2022uwk}.






\section*{Acknowledgments}

This work is supported in part by the National Key Research and Development Program of China (Grant No.~2021YFC2203003 and No.~2020YFC2201501 )
and the National Natural Science Foundation of China (NSFC) (Grant No. 12005254 and No. 12147103).



%\begin{thebibliography}{000} %for 3 digits
%\begin{thebibliography}{00}  %for 2 digits
\begin{thebibliography}{0}    %for 1 digit

%%journal paper
\bibitem{CDF:2022hxs}%\label{CDF}
T.~Aaltonen \textit{et al.} [CDF Collaboration],
%``High-precision measurement of the W boson mass with the CDF II detector,''
Science \textbf{376}, no.6589, 170-176 (2022).
%doi:10.1126/science.abk1781
%195 citations counted in INSPIRE as of 12 Sep 2022


%\cite{ParticleDataGroup:2022pth}
\bibitem{ParticleDataGroup:2022pth}
R.~L.~Workman \textit{et al.} [Particle Data Group],
%``Review of Particle Physics,''
PTEP \textbf{2022}, 083C01 (2022).
%doi:10.1093/ptep/ptac097
%1286 citations counted in INSPIRE as of 12 Jul 2023
%\cite{CDF:2022hxs}


%\cite{Sakurai:2022hwh}
\bibitem{Sakurai:2022hwh}%\label{xx3}
K.~Sakurai, F.~Takahashi and W.~Yin,
%``Singlet extensions and W boson mass in light of the CDF II result,''
Phys. Lett. B \textbf{833}, 137324 (2022).
%doi:10.1016/j.physletb.2022.137324
%[arXiv:2204.04770 [hep-ph]].
%61 citations counted in INSPIRE as of 12 Sep 2022

%\cite{Peli:2022ybi}
\bibitem{Peli:2022ybi}
Z.~P\'eli and Z.~Tr\'ocs\'anyi,
%``Vacuum stability and scalar masses in the superweak extension of the standard model,''
Phys. Rev. D \textbf{106}, no.5, 055045 (2022).
%doi:10.1103/PhysRevD.106.055045
%[arXiv:2204.07100 [hep-ph]].
%23 citations counted in INSPIRE as of 19 Jul 2023


%\cite{Dcruz:2022dao}
\bibitem{Dcruz:2022dao}
R.~Dcruz and A.~Thapa,
%``W boson mass shift, dark matter, and (g-2)\ensuremath{\ell} in a scotogenic-Zee model,''
Phys. Rev. D \textbf{107}, no.1, 015002 (2023).
%doi:10.1103/PhysRevD.107.015002
%[arXiv:2205.02217 [hep-ph]].
%25 citations counted in INSPIRE as of 19 Jul 2023

%\cite{Asai:2022uix}
\bibitem{Asai:2022uix}%\label{xx6}
K.~Asai, C.~Miyao, S.~Okawa and K.~Tsumura,
%``Scalar dark matter with a \ensuremath{\mu}\ensuremath{\tau} flavored mediator,''
Phys. Rev. D \textbf{106}, no.3, 035017 (2022).
%doi:10.1103/PhysRevD.106.035017
%[arXiv:2205.08998 [hep-ph]].
%1 citations counted in INSPIRE as of 12 Sep 2022


%\cite{Fan:2022dck}
\bibitem{Fan:2022dck}%\label{xx7}
Y.~Z.~Fan, T.~P.~Tang, Y.~L.~S.~Tsai and L.~Wu,
%``Inert Higgs Dark Matter for CDF II W-Boson Mass and Detection Prospects,''
Phys. Rev. Lett. \textbf{129}, no.9, 9 (2022).
%doi:10.1103/PhysRevLett.129.091802
%[arXiv:2204.03693 [hep-ph]].
%75 citations counted in INSPIRE as of 12 Sep 2022


%\cite{Lu:2022bgw}
\bibitem{Lu:2022bgw}
C.~T.~Lu, L.~Wu, Y.~Wu and B.~Zhu,
%``Electroweak precision fit and new physics in light of the \ensuremath{<}math display=''inline''\ensuremath{>}\ensuremath{<}mi\ensuremath{>}W\ensuremath{<}/mi\ensuremath{>}\ensuremath{<}/math\ensuremath{>} boson mass,''
Phys. Rev. D \textbf{106}, no.3, 035034 (2022).
%doi:10.1103/PhysRevD.106.035034
%[arXiv:2204.03796 [hep-ph]].
%96 citations counted in INSPIRE as of 12 Sep 2022


%\cite{Song:2022xts}
\bibitem{Song:2022xts}
H.~Song, W.~Su and M.~Zhang,
%``Electroweak phase transition in 2HDM under Higgs, Z-pole, and W precision measurements,''
JHEP \textbf{10}, 048 (2022).
%doi:10.1007/JHEP10(2022)048
%[arXiv:2204.05085 [hep-ph]].
%77 citations counted in INSPIRE as of 19 Jul 2023


%\cite{Bahl:2022xzi}
\bibitem{Bahl:2022xzi}
H.~Bahl, J.~Braathen and G.~Weiglein,
%``New physics effects on the W-boson mass from a doublet extension of the SM Higgs sector,''
Phys. Lett. B \textbf{833}, 137295 (2022).
%doi:10.1016/j.physletb.2022.137295
%[arXiv:2204.05269 [hep-ph]].
%71 citations counted in INSPIRE as of 12 Sep 2022


%\cite{Babu:2022pdn}
\bibitem{Babu:2022pdn}
K.~S.~Babu, S.~Jana and V.~P.~K.,
%``Correlating W-Boson Mass Shift with Muon g-2 in the Two Higgs Doublet Model,''
Phys. Rev. Lett. \textbf{129}, no.12, 12 (2022).
%doi:10.1103/PhysRevLett.129.121803
%[arXiv:2204.05303 [hep-ph]].
%96 citations counted in INSPIRE as of 19 Jul 2023


%\cite{Heo:2022dey}
\bibitem{Heo:2022dey}
Y.~Heo, D.~W.~Jung and J.~S.~Lee,
%``Impact of the CDF W-mass anomaly on two Higgs doublet model,''
Phys. Lett. B \textbf{833}, 137274 (2022).
%doi:10.1016/j.physletb.2022.137274
%[arXiv:2204.05728 [hep-ph]].
%62 citations counted in INSPIRE as of 12 Sep 2022


%\cite{Ahn:2022}
\bibitem{Ahn:2022}
Y.~H.~Ahn, S.~K.~Kang and R.~Ramos,
%``Implications of New CDF-II $W$ Boson Mass on Two Higgs Doublet Model,''
Phys. Rev. D \textbf{106}, no.5, 055038 (2022).
%doi:10.1103/PhysRevD.106.055038
%[arXiv:2204.06485 [hep-ph]].
%75 citations counted in INSPIRE as of 19 Jul 2023


%\cite{Han:2022juu}
\bibitem{Han:2022juu}
X.~F.~Han, {\it et al.}, %F.~Wang, L.~Wang, J.~M.~Yang and Y.~Zhang,
%``A joint explanation of W-mass and muon g-2 in 2HDM,''
Chin. Phys. C \textbf{46}, 103105 (2022).
%doi:10.1088/1674-1137/ac7c63
%[arXiv:2204.06505 [hep-ph]].
%54 citations counted in INSPIRE as of 12 Sep 2022


%\cite{Ghorbani:2022vtv}
\bibitem{Ghorbani:2022vtv}
K.~Ghorbani and P.~Ghorbani,
%``W-boson mass anomaly from scale invariant 2HDM,''
Nucl. Phys. B \textbf{984}, 115980 (2022).
%doi:10.1016/j.nuclphysb.2022.115980
%[arXiv:2204.09001 [hep-ph]].
%61 citations counted in INSPIRE as of 19 Jul 2023



%\cite{Lee:2022gyf}
\bibitem{Lee:2022gyf}
S.~Lee, {\it et al.}, %K.~Cheung, J.~Kim, C.~T.~Lu and J.~Song,
%``Status of the two-Higgs-doublet model in light of the CDF mW measurement,''
Phys. Rev. D \textbf{106}, no.7, 075013 (2022).
%doi:10.1103/PhysRevD.106.075013
%[arXiv:2204.10338 [hep-ph]].
%48 citations counted in INSPIRE as of 19 Jul 2023


%\cite{Abouabid:2022lpg}
\bibitem{Abouabid:2022lpg}
H.~Abouabid, {\it et al.}, %A.~Arhrib, R.~Benbrik, M.~Krab and M.~Ouchemhou,
%``Is the new CDF MW measurement consistent with the two-Higgs doublet model?,''
Nucl. Phys. B \textbf{989}, 116143 (2023).
%doi:10.1016/j.nuclphysb.2023.116143
%[arXiv:2204.12018 [hep-ph]].
%35 citations counted in INSPIRE as of 19 Jul 2023


%\cite{Benbrik:2022dja}
\bibitem{Benbrik:2022dja}
R.~Benbrik, M.~Boukidi and B.~Manaut,
%``$W$-mass and 96 GeV excess in type-III 2HDM,''
[arXiv:2204.11755 [hep-ph]].
%21 citations counted in INSPIRE as of 12 Sep 2022


%\cite{Botella:2022rte}
\bibitem{Botella:2022rte}
F.~J.~Botella, {\it et al.}, %F.~Cornet-Gomez, C.~Mir\'o and M.~Nebot,
%``Muon and electron $g-2$ anomalies in a flavor conserving 2HDM with an oblique view on the CDF $M_W$ value,''
Eur. Phys. J. C \textbf{82}, 915 (2022).
%doi:10.1140/epjc/s10052-022-10893-x
%[arXiv:2205.01115 [hep-ph]].
%33 citations counted in INSPIRE as of 19 Jul 2023


%\cite{Kim:2022hvh}
\bibitem{Kim:2022hvh}
J.~Kim, S.~Lee, P.~Sanyal and J.~Song,
%``CDF W-boson mass and muon g-2 in a type-X two-Higgs-doublet model with a Higgs-phobic light pseudoscalar,''
Phys. Rev. D \textbf{106}, no.3, 035002 (2022).
%doi:10.1103/PhysRevD.106.035002
%[arXiv:2205.01701 [hep-ph]].
%16 citations counted in INSPIRE as of 12 Sep 2022


%\cite{Kim:2022xuo}
\bibitem{Kim:2022xuo}
J.~Kim,
%``Compatibility of muon g \ensuremath{-} 2, W mass anomaly in type-X 2HDM,''
Phys. Lett. B \textbf{832}, 137220 (2022).
%doi:10.1016/j.physletb.2022.137220
%[arXiv:2205.01437 [hep-ph]].
%12 citations counted in INSPIRE as of 12 Sep 2022


%\cite{Appelquist:2022qgl}
\bibitem{Appelquist:2022qgl}
T.~Appelquist, J.~Ingoldby and M.~Piai,
%``Composite two-Higgs doublet model from dilaton effective field theory,''
Nucl. Phys. B \textbf{983}, 115930 (2022).
%doi:10.1016/j.nuclphysb.2022.115930
%[arXiv:2205.03320 [hep-ph]].
%1 citations counted in INSPIRE as of 12 Sep 2022


%\cite{Benincasa:2022elt}
\bibitem{Benincasa:2022elt}
N.~Benincasa, L.~Delle Rose, K.~Kannike and L.~Marzola,
%``Multi-step phase transitions and gravitational waves in the inert doublet model,''
JCAP \textbf{12}, 025 (2022).
%doi:10.1088/1475-7516/2022/12/025
%[arXiv:2205.06669 [hep-ph]].
%11 citations counted in INSPIRE as of 19 Jul 2023


%\cite{Arhrib:2022inj}
\bibitem{Arhrib:2022inj}
A.~Arhrib, {\it et al.}, %R.~Benbrik, M.~Krab, B.~Manaut, S.~Moretti, Y.~Wang and Q.~S.~Yan,
%``Light charged Higgs boson in $H^\pm h$ associated production at the LHC,''
[arXiv:2205.14274 [hep-ph]].
%5 citations counted in INSPIRE as of 19 Jul 2023

%\cite{Abdallah:2022shy}
\bibitem{Abdallah:2022shy}
W.~Abdallah, R.~Gandhi and S.~Roy,
%``LSND and MiniBooNE as guideposts to understanding the muon g \ensuremath{-} 2 results and the CDF II W mass measurement,''
Phys. Lett. B \textbf{840}, 137841 (2023).
%doi:10.1016/j.physletb.2023.137841
%[arXiv:2208.02264 [hep-ph]].
%4 citations counted in INSPIRE as of 19 Jul 2023

%\cite{deGiorgi:2023wjh}
\bibitem{deGiorgi:2023wjh}
A.~de Giorgi, F.~Koutroulis, L.~Merlo and S.~Pokorski,
%``Flavour and Higgs physics in $Z_2$-symmetric 2HD models near the decoupling limit,''
[arXiv:2304.10560 [hep-ph]].
%1 citations counted in INSPIRE as of 19 Jul 2023

%\cite{Abouabid:2023mbu}
\bibitem{Abouabid:2023mbu}
H.~Abouabid, {\it et al.}, %A.~Arhrib, R.~Benbrik, M.~Boukidi and J.~E.~Falaki,
%``The oblique parameters in the 2HDM with Vector-Like Quarks: Confronting $M_W$ CDF-II Anomaly,''
[arXiv:2302.07149 [hep-ph]].
%0 citations counted in INSPIRE as of 19 Jul 2023


%%%%%%%%%%%%%%%%%%%%%%%%%%%triplet%%%%%%%%%%%%%%%%%%%%%%%%%%%%%%%%%%%%%%%%%%%%%%%%%%%%%%%%%%%%%%%%%%%%%%%%%%%%
%%%%%%%%%%%%%%%%%%%%%%%%%%%%%%%%%%%%%%%%%%%%%%%%%%%%%%%%%%%%%%%%%%%%%%%%%%%%%%%%%%%%%%%%%%%%%%%%%%%%%%%%%%%%%%

%\cite{Barrie:2022cub}
\bibitem{Barrie:2022cub}
N.~D.~Barrie, C.~Han and H.~Murayama, 
%``Type II Seesaw leptogenesis,''
JHEP \textbf{05}, 160 (2022).
%doi:10.1007/JHEP05(2022)160
%[arXiv:2204.08202 [hep-ph]].
%5 citations counted in INSPIRE as of 12 Sep 2022


%\cite{Cheng:2022jyi}
\bibitem{Cheng:2022jyi}
Y.~Cheng, X.~G.~He, Z.~L.~Huang and M.~W.~Li,
%``Type-II seesaw triplet scalar effects on neutrino trident scattering,''
Phys. Lett. B \textbf{831}, 137218 (2022).
%doi:10.1016/j.physletb.2022.137218
%[arXiv:2204.05031 [hep-ph]].
%52 citations counted in INSPIRE as of 12 Sep 2022


%\cite{Du:2022brr}
\bibitem{Du:2022brr}
X.~K.~Du, Z.~Li, F.~Wang and Y.~K.~Zhang,
%``Explaining the CDF-II W-boson mass anomaly in the Georgi\textendash{}Machacek extension models,''
Eur. Phys. J. C \textbf{83} (2023) no.2, 139.
%doi:10.1140/epjc/s10052-023-11297-1
%[arXiv:2204.05760 [hep-ph]].
%52 citations counted in INSPIRE as of 31 May 2023



%\cite{FileviezPerez:2022lxp}
\bibitem{FileviezPerez:2022lxp}
P.~Fileviez Perez, H.~H.~Patel and A.~D.~Plascencia,
%``On the W mass and new Higgs bosons,''
Phys. Lett. B \textbf{833}, 137371 (2022).
%doi:10.1016/j.physletb.2022.137371
%[arXiv:2204.07144 [hep-ph]].
%46 citations counted in INSPIRE as of 12 Sep 2022


%\cite{Kanemura:2022ahw}
\bibitem{Kanemura:2022ahw}
S.~Kanemura and K.~Yagyu,
%``Implication of the W boson mass anomaly at CDF II in the Higgs triplet model with a mass difference,''
Phys. Lett. B \textbf{831}, 137217 (2022).
%doi:10.1016/j.physletb.2022.137217
%[arXiv:2204.07511 [hep-ph]].
%45 citations counted in INSPIRE as of 12 Sep 2022


%\cite{Mondal:2022xdy}
\bibitem{Mondal:2022xdy}
P.~Mondal,
%``Enhancement of the W boson mass in the Georgi-Machacek model,''
Phys. Lett. B \textbf{833}, 137357 (2022).
%doi:10.1016/j.physletb.2022.137357
%[arXiv:2204.07844 [hep-ph]].
%27 citations counted in INSPIRE as of 12 Sep 2022


%\cite{Borah:2022obi}
\bibitem{Borah:2022obi}
D.~Borah, S.~Mahapatra, D.~Nanda and N.~Sahu,
%``Type II Dirac seesaw with observable \ensuremath{\Delta}Neff in the light of W-mass anomaly,''
Phys. Lett. B \textbf{833}, 137297 (2022).
%doi:10.1016/j.physletb.2022.137297
%[arXiv:2204.08266 [hep-ph]].
%41 citations counted in INSPIRE as of 12 Sep 2022



%\cite{Addazi:2022fbj}
\bibitem{Addazi:2022fbj}
A.~Addazi, {\it et al.}, %A.~Marciano, A.~P.~Morais, R.~Pasechnik and H.~Yang,
%``CDF II W-mass anomaly faces first-order electroweak phase transition,''
Eur. Phys. J. C \textbf{83}, no.3, 207 (2023).
%doi:10.1140/epjc/s10052-023-11315-2
%[arXiv:2204.10315 [hep-ph]].
%26 citations counted in INSPIRE as of 19 Jul 2023


%\cite{Heeck:2022fvl}
\bibitem{Heeck:2022fvl}
J.~Heeck,
%``W-boson mass in the triplet seesaw model,''
Phys. Rev. D \textbf{106}, no.1, 015004 (2022).
%doi:10.1103/PhysRevD.106.015004
%[arXiv:2204.10274 [hep-ph]].
%26 citations counted in INSPIRE as of 12 Sep 2022


%\cite{Chen:2022ocr}
\bibitem{Chen:2022ocr}
T.~K.~Chen, C.~W.~Chiang and K.~Yagyu,
%``Explanation of the W mass shift at CDF II in the extended Georgi-Machacek model,''
Phys. Rev. D \textbf{106}, no.5, 055035 (2022).
%doi:10.1103/PhysRevD.106.055035
%[arXiv:2204.12898 [hep-ph]].
%25 citations counted in INSPIRE as of 19 Jul 2023


%\cite{Evans:2022dgq}
\bibitem{Evans:2022dgq}
J.~L.~Evans, T.~T.~Yanagida and N.~Yokozaki,
%``W boson mass anomaly and grand unification,''
Phys. Lett. B \textbf{833}, 137306 (2022).
%doi:10.1016/j.physletb.2022.137306
%[arXiv:2205.03877 [hep-ph]].
%9 citations counted in INSPIRE as of 12 Sep 2022



%\cite{Ghosh:2022zqs}
\bibitem{Ghosh:2022zqs}
R.~Ghosh, B.~Mukhopadhyaya and U.~Sarkar,
%``The \ensuremath{\rho} parameter and the CDF-IIW-mass anomaly: observations on the role of scalar triplets,''
J. Phys. G \textbf{50}, no.7, 075003 (2023).
%doi:10.1088/1361-6471/acd0c8
%[arXiv:2205.05041 [hep-ph]].
%11 citations counted in INSPIRE as of 19 Jul 2023



%\cite{Ma:2022emu}
\bibitem{Ma:2022emu}
E.~Ma,
%``Type III neutrino seesaw, freeze-in long-lived dark matter, and the W mass shift,''
Phys. Lett. B \textbf{833}, 137327 (2022).
%doi:10.1016/j.physletb.2022.137327
%[arXiv:2205.09794 [hep-ph]].
%3 citations counted in INSPIRE as of 12 Sep 2022



%\cite{Bahl:2022gqg}
\bibitem{Bahl:2022gqg}
H.~Bahl, {\it et al.}, %W.~H.~Chiu, C.~Gao, L.~T.~Wang and Y.~M.~Zhong,
%``Tripling down on the $W$ boson mass,''
Eur. Phys. J. C \textbf{82}, no.10, 944 (2022).
%doi:10.1140/epjc/s10052-022-10934-5
%[arXiv:2207.04059 [hep-ph]].
%16 citations counted in INSPIRE as of 19 Jul 2023



%\cite{Penedo:2022gej}
\bibitem{Penedo:2022gej}
J.~T.~Penedo, Y.~Reyimuaji and X.~Zhang,
%``Axionic Dirac seesaw mechanism and electroweak vacuum stability,''
Phys. Rev. D \textbf{106}, no.11, 115035 (2022).
%doi:10.1103/PhysRevD.106.115035
%[arXiv:2208.03329 [hep-ph]].
%5 citations counted in INSPIRE as of 19 Jul 2023


%\cite{Cheng:2022hbo}
\bibitem{Cheng:2022hbo}
Y.~Cheng, {\it et al.}, %X.~G.~He, F.~Huang, J.~Sun and Z.~P.~Xing,
%``Electroweak precision tests for triplet scalars,''
Nucl. Phys. B \textbf{989}, 116118 (2023).
%doi:10.1016/j.nuclphysb.2023.116118
%[arXiv:2208.06760 [hep-ph]].
%15 citations counted in INSPIRE as of 19 Jul 2023


%\cite{Butterworth:2022dkt}
\bibitem{Butterworth:2022dkt}
J.~Butterworth, J.~Heeck, S.~H.~Jeon, O.~Mattelaer and R.~Ruiz,
%``Testing the scalar triplet solution to CDF\textquoteright{}s heavy W problem at the LHC,''
Phys. Rev. D \textbf{107}, no.7, 075020 (2023).
%doi:10.1103/PhysRevD.107.075020
%[arXiv:2210.13496 [hep-ph]].
%8 citations counted in INSPIRE as of 19 Jul 2023


%\cite{Wu:2022uwk}
\bibitem{Wu:2022uwk}
J.~Wu, D.~Huang and C.~Q.~Geng,
%``W-boson mass anomaly from a general SU(2)$_{L}$ scalar multiplet*,''
Chin. Phys. C \textbf{47}, no.6, 063103 (2023).
%doi:10.1088/1674-1137/acc8bf
%[arXiv:2212.14553 [hep-ph]].
%3 citations counted in INSPIRE as of 18 Jul 2023


%\cite{Song:2022jns}
\bibitem{Song:2022jns}%yujianghao
H.~Song, X.~Wan and J.~H.~Yu,
%``Custodial Symmetry Violation in Scalar Extensions of the Standard Model,''
[arXiv:2211.01543 [hep-ph]].
%0 citations counted in INSPIRE as of 06 Dec 2022



%\cite{Crivellin:2023gtf}
\bibitem{Crivellin:2023gtf}
A.~Crivellin, M.~Kirk and A.~Thapa,
%``$Y=0$ Scalar Triplet Beyond the $W$ Mass: $(g-2)_\mu$, $h\to\mu^+\mu^-$ and CKM Unitarity,''
[arXiv:2305.03081 [hep-ph]].
%3 citations counted in INSPIRE as of 19 Jul 2023


%\cite{Ellis:2023zim}
\bibitem{Ellis:2023zim}
J.~Ellis, K.~Mimasu and F.~Zampedri,
%``Dimension-8 SMEFT Analysis of Minimal Scalar Field Extensions of the Standard Model,''
[arXiv:2304.06663 [hep-ph]].
%4 citations counted in INSPIRE as of 19 Jul 2023


%\cite{Shimizu:2023rvi}
\bibitem{Shimizu:2023rvi}
Y.~Shimizu and S.~Takeshita,
%``$W$ Boson Mass and Grand Unification via the Type-$\rm{I\hspace{-.01em}I}$ Seesaw-like Mechanism,''
[arXiv:2303.11070 [hep-ph]].
%3 citations counted in INSPIRE as of 19 Jul 2023






%deGiorgi:2022xhr,Amoroso:2022rly,Primulando:2022vip,Pfeifer:2022yrs,Chung:2022avf,Lin:2022khg,Diaz:2022vdo,Liu:2022zie,Rodriguez:2022hsj,Domingo:2022pde,Kim:2022axk,Barger:2022wih,Benakli:2022gjn
%%%%%%%%%%%%%%%%%%%%%%%%%%%%%%%%%%%%%%%%%%%%%%%%%%%%%%%%%%%%%%%%%%%%%%%%%%%%%%%%%%%%%%%%%%%%%%%%%%%%%%%%%%%%%%%%%%%%%%%%%%%%%%%%%%%%%%%%%%%%%%%%emailadd%%%%%%%%

%%%%%%%%%%%%%%%%%%%%%%%%%%%%%%%%%%%%%%%%%%%%%%%%%%%%%%%%%%%%%%%%%%%%%%%%%%%%%%%%%%%%%%%%%%%%%%%%%%%%%%%%%%%%%%%%%%%%%%%%%%%%%%%%
%%%%%%% EW precison tests %%%%%%
%\cite{Peskin:1991sw}
\bibitem{Peskin:1991sw}%\label{xx108P1}
M.~E.~Peskin and T.~Takeuchi,
%``Estimation of oblique electroweak corrections,''
Phys. Rev. D \textbf{46}, 381-409 (1992).
%doi:10.1103/PhysRevD.46.381
%2514 citations counted in INSPIRE as of 14 Sep 2022



%\cite{Peskin:1990zt}
\bibitem{Peskin:1990zt}%\label{xx109P2}
M.~E.~Peskin and T.~Takeuchi,
%``A New constraint on a strongly interacting Higgs sector,''
Phys. Rev. Lett. \textbf{65}, 964-967 (1990).
%doi:10.1103/PhysRevLett.65.964
%2096 citations counted in INSPIRE as of 14 Sep 2022


%\cite{Maksymyk:1993zm}
\bibitem{Maksymyk:1993zm}%\label{xx110wmass-o1}
I.~Maksymyk, C.~P.~Burgess and D.~London,
%``Beyond S, T and U,''
Phys. Rev. D \textbf{50}, 529-535 (1994).
%doi:10.1103/PhysRevD.50.529
%[arXiv:hep-ph/9306267 [hep-ph]].
%234 citations counted in INSPIRE as of 14 Sep 2022

%\cite{Burgess:1993mg,Lavoura:1993nq}
\bibitem{Burgess:1993mg}%\label{xx111wmass-o2}
C.~P.~Burgess, S.~Godfrey, H.~Konig, D.~London and I.~Maksymyk,
%``A Global fit to extended oblique parameters,''
Phys. Lett. B \textbf{326}, 276-281 (1994).
%doi:10.1016/0370-2693(94)91322-6
%[arXiv:hep-ph/9307337 [hep-ph]].
%102 citations counted in INSPIRE as of 14 Sep 2022

%\cite{Lavoura:1993nq,Grimus:2007if,Grimus:2008nb}
\bibitem{Lavoura:1993nq}%\label{xx114L}
L.~Lavoura and L.~F.~Li,
%``Making the small oblique parameters large,''
Phys. Rev. D \textbf{49}, 1409-1416 (1994).
%doi:10.1103/PhysRevD.49.1409
%[arXiv:hep-ph/9309262 [hep-ph]].
%58 citations counted in INSPIRE as of 16 Sep 2022


%\cite{Albergaria:2021dmq}
\bibitem{Albergaria:2021dmq}%\label{xx113FA}
F.~Albergaria and L.~Lavoura,
%``Prescription for finite oblique parameters S and U in extensions of the SM with m $_{W}$ \ensuremath{\neq} m $_{Z}$\,cos\,\ensuremath{\theta} $_{W}$,''
J. Phys. G \textbf{49}, no.8, 085005 (2022).
%doi:10.1088/1361-6471/ac7a56
%[arXiv:2111.02339 [hep-ph]].
%3 citations counted in INSPIRE as of 16 Sep 2022


%%%%%%%% Higgs trilinar coupling %%%%%


%%%%%%%%%%% minimal DM %%%%%%%
%\cite{Cirelli:2005uq,Cirelli:2009uv}
\bibitem{Cirelli:2005uq}%\label{xx107mdm}
M.~Cirelli, N.~Fornengo and A.~Strumia,
%``Minimal dark matter,''
Nucl. Phys. B \textbf{753}, 178-194 (2006).
%doi:10.1016/j.nuclphysb.2006.07.012
%[arXiv:hep-ph/0512090 [hep-ph]].
%905 citations counted in INSPIRE as of 25 Sep 2022

%\cite{Cirelli:2009uv}
\bibitem{Cirelli:2009uv}
M.~Cirelli and A.~Strumia,
%``Minimal Dark Matter: Model and results,''
New J. Phys. \textbf{11}, 105005 (2009).
%doi:10.1088/1367-2630/11/10/105005
%[arXiv:0903.3381 [hep-ph]].
%230 citations counted in INSPIRE as of 24 Dec 2022

%\cite{Guo:2010hq}
\bibitem{Guo:2010hq}
W.~L.~Guo and Y.~L.~Wu,
%``The Real singlet scalar dark matter model,''
JHEP \textbf{10}, 083 (2010).
%doi:10.1007/JHEP10(2010)083
%[arXiv:1006.2518 [hep-ph]].
%113 citations counted in INSPIRE as of 27 Dec 2022

%\cite{Barbieri:2006dq}
\bibitem{Barbieri:2006dq}
R.~Barbieri, L.~J.~Hall and V.~S.~Rychkov,
%``Improved naturalness with a heavy Higgs: An Alternative road to LHC physics,''
Phys. Rev. D \textbf{74}, 015007 (2006).
%doi:10.1103/PhysRevD.74.015007
%[arXiv:hep-ph/0603188 [hep-ph]].
%942 citations counted in INSPIRE as of 27 Dec 2022

%\cite{LopezHonorez:2010eeh}
\bibitem{LopezHonorez:2010eeh}
L.~Lopez Honorez and C.~E.~Yaguna,
%``The inert doublet model of dark matter revisited,''
JHEP \textbf{09}, 046 (2010).
%doi:10.1007/JHEP09(2010)046
%[arXiv:1003.3125 [hep-ph]].
%200 citations counted in INSPIRE as of 27 Dec 2022

%\cite{Gonderinger:2012rd}
\bibitem{Gonderinger:2012rd}
M.~Gonderinger, H.~Lim and M.~J.~Ramsey-Musolf,
%``Complex Scalar Singlet Dark Matter: Vacuum Stability and Phenomenology,''
Phys. Rev. D \textbf{86}, 043511 (2012).
%doi:10.1103/PhysRevD.86.043511
%[arXiv:1202.1316 [hep-ph]].
%185 citations counted in INSPIRE as of 27 Dec 2022

%%%%%%%% EW Baryogenesis %%%%%%%%%
%\cite{Cline:2012hg,Grzadkowski:2018nbc}
\bibitem{Cline:2012hg}
J.~M.~Cline and K.~Kainulainen,
%``Electroweak baryogenesis and dark matter from a singlet Higgs,''
JCAP \textbf{01}, 012 (2013).
%doi:10.1088/1475-7516/2013/01/012
%[arXiv:1210.4196 [hep-ph]].
%223 citations counted in INSPIRE as of 27 Dec 2022

%\cite{Grzadkowski:2018nbc}
\bibitem{Grzadkowski:2018nbc}
B.~Grzadkowski and D.~Huang,
%``Spontaneous $CP$-Violating Electroweak Baryogenesis and Dark Matter from a Complex Singlet Scalar,''
JHEP \textbf{08}, 135 (2018).
%doi:10.1007/JHEP08(2018)135
%[arXiv:1807.06987 [hep-ph]].
%31 citations counted in INSPIRE as of 27 Dec 2022

%\cite{Cline:2021iff}
\bibitem{Cline:2021iff}
J.~M.~Cline, A.~Friedlander, D.~M.~He, K.~Kainulainen, B.~Laurent and D.~Tucker-Smith,
%``Baryogenesis and gravity waves from a UV-completed electroweak phase transition,''
Phys. Rev. D \textbf{103}, no.12, 123529 (2021).
%doi:10.1103/PhysRevD.103.123529
%[arXiv:2102.12490 [hep-ph]].
%38 citations counted in INSPIRE as of 27 Dec 2022

%\cite{Morrissey:2012db}
\bibitem{Morrissey:2012db}
D.~E.~Morrissey and M.~J.~Ramsey-Musolf,
%``Electroweak baryogenesis,''
New J. Phys. \textbf{14}, 125003 (2012).
%doi:10.1088/1367-2630/14/12/125003
%[arXiv:1206.2942 [hep-ph]].
%687 citations counted in INSPIRE as of 27 Dec 2022



%%%%%%%%%%%%%%%%%%%%%%%%%%%%%%%%%%%%%%%%%%%%%%%%%%%%%%%%%%%%%%%%%%%%%%%%%%%%%%%%%%%%%%%%%%%%%%%%%%%%%%%%%%%%%%%%%%%%%%%%%%%%%%%%%%%%%%%%%%%%%%%%%%%%%%%%%%%%%%%%%%%%%%%%%%%%%%%%%%%%%%

%%%%% EW phase transition and GW production %%%%%%%
%\cite{Chowdhury:2011ga,Hashino:2018zsi,Chiang:2017nmu,Kannike:2019wsn,Chiang:2020yym,Chiang:2019oms,Cai:2017tmh,Chao:2017vrq,Ellis:2018mja,Alves:2018jsw,Zhou:2018zli,Bian:2019kmg,Ghosh:2020ipy,Zhou:2020irf,Lu:2022zpn,Zhou:2022mlz,Cai:2022bcf}
\bibitem{Chowdhury:2011ga}
T.~A.~Chowdhury, M.~Nemevsek, G.~Senjanovic and Y.~Zhang,
%``Dark Matter as the Trigger of Strong Electroweak Phase Transition,''
JCAP \textbf{02}, 029 (2012).
%doi:10.1088/1475-7516/2012/02/029
%[arXiv:1110.5334 [hep-ph]].
%104 citations counted in INSPIRE as of 27 Dec 2022

%\cite{Hashino:2018zsi}
\bibitem{Hashino:2018zsi}
K.~Hashino, M.~Kakizaki, S.~Kanemura, P.~Ko and T.~Matsui,
%``Gravitational waves from first order electroweak phase transition in models with the U(1)$_{X}$ gauge symmetry,''
JHEP \textbf{06}, 088 (2018).
%doi:10.1007/JHEP06(2018)088
%[arXiv:1802.02947 [hep-ph]].
%46 citations counted in INSPIRE as of 27 Dec 2022

%\cite{Chiang:2017nmu}
\bibitem{Chiang:2017nmu}
C.~W.~Chiang, {\it et al.}, %M.~J.~Ramsey-Musolf and E.~Senaha,
%``Standard Model with a Complex Scalar Singlet: Cosmological Implications and Theoretical Considerations,''
Phys. Rev. D \textbf{97}, no.1, 015005 (2018).
%doi:10.1103/PhysRevD.97.015005
%[arXiv:1707.09960 [hep-ph]].
%85 citations counted in INSPIRE as of 27 Dec 2022

%\cite{Kannike:2019wsn}
\bibitem{Kannike:2019wsn}
K.~Kannike and M.~Raidal,
%``Phase Transitions and Gravitational Wave Tests of Pseudo-Goldstone Dark Matter in the Softly Broken U(1) Scalar Singlet Model,''
Phys. Rev. D \textbf{99}, no.11, 115010 (2019).
%doi:10.1103/PhysRevD.99.115010
%[arXiv:1901.03333 [hep-ph]].
%29 citations counted in INSPIRE as of 27 Dec 2022

%\cite{Chiang:2020yym}
\bibitem{Chiang:2020yym}
C.~W.~Chiang, D.~Huang and B.~Q.~Lu,
%``Electroweak phase transition confronted with dark matter detection constraints,''
JCAP \textbf{01}, 035 (2021).
%doi:10.1088/1475-7516/2021/01/035
%[arXiv:2009.08635 [hep-ph]].
%4 citations counted in INSPIRE as of 27 Dec 2022

%\cite{Chiang:2019oms}
\bibitem{Chiang:2019oms}
C.~W.~Chiang and B.~Q.~Lu,
%``First-order electroweak phase transition in a complex singlet model with $\mathbb{Z}_3$ symmetry,''
JHEP \textbf{07}, 082 (2020).
%doi:10.1007/JHEP07(2020)082
%[arXiv:1912.12634 [hep-ph]].
%29 citations counted in INSPIRE as of 27 Dec 2022

%\cite{Cai:2017tmh}
\bibitem{Cai:2017tmh}
R.~G.~Cai, M.~Sasaki and S.~J.~Wang,
%``The gravitational waves from the first-order phase transition with a dimension-six operator,''
JCAP \textbf{08}, 004 (2017).
%doi:10.1088/1475-7516/2017/08/004
%[arXiv:1707.03001 [astro-ph.CO]].
%52 citations counted in INSPIRE as of 27 Dec 2022

%\cite{Chao:2017vrq}
\bibitem{Chao:2017vrq}
W.~Chao, H.~K.~Guo and J.~Shu,
%``Gravitational Wave Signals of Electroweak Phase Transition Triggered by Dark Matter,''
JCAP \textbf{09}, 009 (2017).
%doi:10.1088/1475-7516/2017/09/009
%[arXiv:1702.02698 [hep-ph]].
%95 citations counted in INSPIRE as of 27 Dec 2022

%\cite{Ellis:2018mja}
\bibitem{Ellis:2018mja}
J.~Ellis, M.~Lewicki and J.~M.~No,
%``On the Maximal Strength of a First-Order Electroweak Phase Transition and its Gravitational Wave Signal,''
JCAP \textbf{04}, 003 (2019).
%doi:10.1088/1475-7516/2019/04/003
%[arXiv:1809.08242 [hep-ph]].
%208 citations counted in INSPIRE as of 27 Dec 2022

%\cite{Alves:2018jsw}
\bibitem{Alves:2018jsw}
A.~Alves, T.~Ghosh, H.~K.~Guo, K.~Sinha and D.~Vagie,
%``Collider and Gravitational Wave Complementarity in Exploring the Singlet Extension of the Standard Model,''
JHEP \textbf{04}, 052 (2019).
%doi:10.1007/JHEP04(2019)052
%[arXiv:1812.09333 [hep-ph]].
%91 citations counted in INSPIRE as of 27 Dec 2022

%\cite{Zhou:2018zli}
\bibitem{Zhou:2018zli}
R.~Zhou, W.~Cheng, X.~Deng, L.~Bian and Y.~Wu,
%``Electroweak phase transition and Higgs phenomenology in the Georgi-Machacek model,''
JHEP \textbf{01}, 216 (2019).
%doi:10.1007/JHEP01(2019)216
%[arXiv:1812.06217 [hep-ph]].
%26 citations counted in INSPIRE as of 27 Dec 2022

%\cite{Hashino:2018wee}
\bibitem{Hashino:2018wee}
K.~Hashino, R.~Jinno, M.~Kakizaki, S.~Kanemura, T.~Takahashi and M.~Takimoto,
%``Selecting models of first-order phase transitions using the synergy between collider and gravitational-wave experiments,''
Phys. Rev. D \textbf{99}, no.7, 075011 (2019).
%doi:10.1103/PhysRevD.99.075011
%[arXiv:1809.04994 [hep-ph]].
%50 citations counted in INSPIRE as of 27 Dec 2022

%\cite{Bian:2019kmg}
\bibitem{Bian:2019kmg}
L.~Bian, Y.~Wu and K.~P.~Xie,
%``Electroweak phase transition with composite Higgs models: calculability, gravitational waves and collider searches,''
JHEP \textbf{12}, 028 (2019).
%doi:10.1007/JHEP12(2019)028
%[arXiv:1909.02014 [hep-ph]].
%45 citations counted in INSPIRE as of 27 Dec 2022

%\cite{Ghosh:2020ipy}
\bibitem{Ghosh:2020ipy}
T.~Ghosh, H.~K.~Guo, T.~Han and H.~Liu,
%``Electroweak phase transition with an SU(2) dark sector,''
JHEP \textbf{07}, 045 (2021).
%doi:10.1007/JHEP07(2021)045
%[arXiv:2012.09758 [hep-ph]].
%20 citations counted in INSPIRE as of 27 Dec 2022

%\cite{Zhou:2020irf}
\bibitem{Zhou:2020irf}
R.~Zhou and L.~Bian,
%``Gravitational wave and electroweak baryogenesis with two Higgs doublet models,''
Phys. Lett. B \textbf{829}, 137105 (2022).
%doi:10.1016/j.physletb.2022.137105
%[arXiv:2001.01237 [hep-ph]].
%21 citations counted in INSPIRE as of 27 Dec 2022



%\cite{Lu:2022zpn}
\bibitem{Lu:2022zpn}
B.~Q.~Lu, C.~W.~Chiang and D.~Huang,
%``Probing WIMPs in space-based gravitational wave experiments,''
Phys. Lett. B \textbf{833}, 137308 (2022).
%doi:10.1016/j.physletb.2022.137308
%[arXiv:2205.08380 [hep-ph]].
%1 citations counted in INSPIRE as of 27 Dec 2022

%\cite{Cai:2022bcf}
\bibitem{Cai:2022bcf}
R.~G.~Cai, K.~Hashino, S.~J.~Wang and J.~H.~Yu,
%``Gravitational waves from patterns of electroweak symmetry breaking: an effective perspective,''
[arXiv:2202.08295 [hep-ph]].
%7 citations counted in INSPIRE as of 27 Dec 2022


%\cite{Zhou:2022mlz,Cai:2022bcf}
\bibitem{Zhou:2022mlz}
R.~Zhou, L.~Bian and Y.~Du,
%``Electroweak phase transition and gravitational waves in the type-II seesaw model,''
JHEP \textbf{08}, 205 (2022).
%doi:10.1007/JHEP08(2022)205
%[arXiv:2203.01561 [hep-ph]].
%11 citations counted in INSPIRE as of 27 Dec 2022


%\bibitem{Georgi:1999wka}
%XXXXXXXXXXXXXX

%%%%%%%%%%%%%%%%%%%%%%%%%%%%%%%%%%%%%%%%%%
%\cite{Asadi:2022xiy}
\bibitem{Asadi:2022xiy}
P.~Asadi, {\it et al.}, %C.~Cesarotti, K.~Fraser, S.~Homiller and A.~Parikh,
%``Oblique Lessons from the $W$ Mass Measurement at CDF II,''
[arXiv:2204.05283 [hep-ph]].
%89 citations counted in INSPIRE as of 29 Apr 2023

%%%%%%%%%%%%%%%%%%%%%%%%%%%%%%%%%%%%%%%%%%

%%%%%%%% Type-II seesaw model %%%%%%%%%


%%%%%%%  Perturbativity %%%%%
%\cite{Nebot:2007bc}
\bibitem{Nebot:2007bc}
M.~Nebot, J.~F.~Oliver, D.~Palao and A.~Santamaria,
%``Prospects for the Zee-Babu Model at the CERN LHC and low energy experiments,''
Phys. Rev. D \textbf{77}, 093013 (2008).
%doi:10.1103/PhysRevD.77.093013
%[arXiv:0711.0483 [hep-ph]].
%142 citations counted in INSPIRE as of 25 Dec 2022

%%%%%%%%%%%%%%%%%%%%%%%%%%%%%%%%%%%%%%%%%%


%%%%%%%%%%%%%%%%%%%%%%%%%%%%%%%%%%%%%%%%%%

%%%%%%%% Type-II seesaw model %%%%%%%%%










%%%%%%%%%%%%%%%%%%%%%%%%%%%%%%%%%%%%%%%%%%%%%%%%%%%%%%%%%%%%%%%%%%%%%%%%%%%%%%%%

\end{thebibliography}
\end{document}