\documentclass[12pt,oneside,reqno]{amsart}
\usepackage{mathrsfs}
\usepackage{graphics}
\usepackage{amssymb}
\usepackage{enumerate}
\pagestyle{plain} \textwidth=160 mm \textheight=230 mm
\oddsidemargin=0mm \topmargin=-3mm

\newcommand{\dif}{\mathrm{d}}
\newcommand{\me}{\mathrm{e}}
\newcommand{\be}{\begin{eqnarray}}
\newcommand{\ee}{\end{eqnarray}}
\newcommand{\ce}{\begin{eqnarray*}}
\newcommand{\de}{\end{eqnarray*}}
\newtheorem{theorem}{Theorem}[section]
\newtheorem{lemma}[theorem]{Lemma}
\newtheorem{remark}[theorem]{Remark}
\newtheorem{definition}[theorem]{Definition}
\newtheorem{proposition}[theorem]{Proposition}
\newtheorem{Example}[theorem]{Example}
\newtheorem{corollary}[theorem]{Corollary}
\newtheorem{condition}[theorem]{Condition}
\def\e{\varepsilon}
\def\s{\sigma}
\def\t{\theta}
\def\a{\alpha}
\def\o{\omega}
\def\b{\beta}
\def\d{\delta}
\def\p{\partial}
\def\g{\gamma}
\def\l{\lambda}
\def\la{\langle}
\def\ra{\rangle}
\def\[{{\Big[}}
\def\]{{\Big]}}
\def\<{{\langle}}
\def\>{{\rangle}}
\def\({{\Big(}}
\def\){{\Big)}}
\def\Hom{{\mathord{{\rm Hom}}}}
\def\dis{{\mathord{{\rm dis}}}}
\def\tr{{\rm tr}}
\def\W{{\mathcal W}}
\def\Ric{{\rm Ricci}}
\def\vol{\mbox{Vol}}
\def\Cap{\mbox{\rm Cap}}
\def\sgn{\mbox{\rm sgn}}
\def\mathcalV{{\mathcal V}}
\def\bbbn{{\rm I\!N}}
\def\no{\nonumber}
\def\bt{\begin{theorem}}
\def\et{\end{theorem}}
\def\bl{\begin{lemma}}
\def\el{\end{lemma}}
\def\br{\begin{remark}}
\def\er{\end{remark}}
\def\bx{\begin{Example}}
\def\ex{\end{Example}}
\def\bd{\begin{definition}}
\def\ed{\end{definition}}
\def\bp{\begin{proposition}}
\def\ep{\end{proposition}}
\def\bc{\begin{corollary}}
\def\ec{\end{corollary}}
\def\bco{\begin{condition}}
\def\eco{\end{condition}}
\def\cA{{\mathcal A}}
\def\cB{{\mathcal B}}
\def\cC{{\mathcal C}}
\def\cD{{\mathcal D}}
\def\cE{{\mathcal E}}
\def\cF{{\mathcal F}}
\def\cG{{\mathcal G}}
\def\cH{{\mathcal H}}
\def\cI{{\mathcal I}}
\def\cJ{{\mathcal J}}
\def\cK{{\mathcal K}}
\def\cL{{\mathcal L}}
\def\cM{{\mathcal M}}
\def\cN{{\mathcal N}}
\def\cO{{\mathcal O}}
\def\cP{{\mathcal P}}
\def\cQ{{\mathcal Q}}
\def\cR{{\mathcal R}}
\def\cS{{\mathcal S}}
\def\cT{{\mathcal T}}
\def\cU{{\mathcal U}}
\def\cV{{\mathcal V}}
\def\cW{{\mathcal W}}
\def\cX{{\mathcal X}}
\def\cY{{\mathcal Y}}
\def\cZ{{\mathcal Z}}

\def\mA{{\mathbb A}}
\def\mB{{\mathbb B}}
\def\mC{{\mathbb C}}
\def\mD{{\mathbb D}}
\def\mE{{\mathbb E}}
\def\mF{{\mathbb F}}
\def\mG{{\mathbb G}}
\def\mH{{\mathbb H}}
\def\mI{{\mathbb I}}
\def\mJ{{\mathbb J}}
\def\mK{{\mathbb K}}
\def\mL{{\mathbb L}}
\def\mM{{\mathbb M}}
\def\mN{{\mathbb N}}
\def\mO{{\mathbb O}}
\def\mP{{\mathbb P}}
\def\mQ{{\mathbb Q}}
\def\mR{{\mathbb R}}
\def\mS{{\mathbb S}}
\def\mT{{\mathbb T}}
\def\mU{{\mathbb U}}
\def\mV{{\mathbb V}}
\def\mW{{\mathbb W}}
\def\mX{{\mathbb X}}
\def\mY{{\mathbb Y}}
\def\mZ{{\mathbb Z}}

\def\sA{{\mathscr A}}
\def\sB{{\mathscr B}}
\def\sC{{\mathscr C}}
\def\sD{{\mathscr D}}
\def\sE{{\mathscr E}}
\def\sF{{\mathscr F}}
\def\sG{{\mathscr G}}
\def\sH{{\mathscr H}}
\def\sI{{\mathscr I}}
\def\sJ{{\mathscr J}}
\def\sK{{\mathscr K}}
\def\sL{{\mathscr L}}
\def\sM{{\mathscr M}}
\def\sN{{\mathscr N}}
\def\sO{{\mathscr O}}
\def\sP{{\mathscr P}}
\def\sQ{{\mathscr Q}}
\def\sR{{\mathscr R}}
\def\sS{{\mathscr S}}
\def\sT{{\mathscr T}}
\def\sU{{\mathscr U}}
\def\sV{{\mathscr V}}
\def\sW{{\mathscr W}}
\def\sX{{\mathscr X}}
\def\sY{{\mathscr Y}}
\def\sZ{{\mathscr Z}}
\def\fT{\frak T}
\def\tg{\tilde{g}}

\def\th{\theta}
\def\geq{\geqslant}
\def\leq{\leqslant}
\def\epsilon{\varepsilon}

\begin{document}

\allowdisplaybreaks

\title{Average principles and large deviation principles of multiscale multivalued McKean-Vlasov stochastic systems}

\author{Huijie Qiao}

\dedicatory{School of Mathematics,
Southeast University,\\
Nanjing, Jiangsu 211189, P.R.China\\
hjqiaogean@seu.edu.cn}

\thanks{{\it AMS Subject Classification(2020):} 60H10; 60F10; 60F15}

\thanks{{\it Keywords:} Multiscale multivalued McKean-Vlasov stochastic systems, average principles, large deviation principles, a weak convergence approach}

\thanks{This work was partly supported by NSF of China (No.12071071).}

\subjclass{}

\date{}

\begin{abstract}
This work concerns about multiscale multivalued McKean-Vlasov stochastic systems. First of all, we establish the well-posedness for multivalued McKean-Vlasov stochastic systems under non-Lipschitz conditions. Then for multiscale multivalued McKean-Vlasov stochastic systems with parameters, in accordance with the values of these parameters we obtain four different average principles. Finally, based on these results, a large deviation principle is presented by a weak convergence approach.
\end{abstract}

\maketitle \rm

\section{Introduction}

The goal of this paper is to study the asymptotic behavior of the system of slow-fast multivalued McKean-Vlasov stochastic differential equations (SDEs for short) on $\mR^{n} \times \mR^{m}$: for any $T>0$
\be\left\{\begin{array}{l}
\dif X_{t}^{\e,\d}\in -A_1(X_{t}^{\e,\d})\dif t+b_{1}(X_{t}^{\e,\d},\sL_{X_{t}^{\e,\d}},Y_{t}^{\e,\d})\dif t+\e^\t\s_{1}(X_{t}^{\e,\d},\sL_{X_{t}^{\e,\d}},Y_{t}^{\e,\d})\dif W^1_{t},\\
X_{0}^{\e,\d}=\xi\in\overline{\cD(A_1)},\quad  0\leq t\leq T,\\
\dif Y_{t}^{\e,\d}\in -A_2(Y_{t}^{\e,\d})\dif t+\frac{1}{\d}b_{2}(X_{t}^{\e,\d},\sL_{X_{t}^{\e,\d}},Y_{t}^{\e,\d})\dif t+\frac{1}{\sqrt{\d}}\s_{2}(X_{t}^{\e,\d},\sL_{X_{t}^{\e,\d}},Y_{t}^{\e,\d})\dif W^2_{t},\\
Y_{0}^{\e,\d}=y_0\in\overline{\cD(A_2)},\quad  0\leq t\leq T.
\end{array}
\right.
\label{Eqin1}
\ee
The system (\ref{Eqin1}) is defined on a filtered probability space $(\Omega,\sF,\{\sF_{t}\}_{t\in[0,T]},\mP)$ and $(W^1_{t}), (W^2_{t})$ are $d_1$- and $d_2$-dimensional standard Brownian motions defined on it, respectively. Moreover, $(W^1_{t})$ and  $(W^2_{t})$ are mutually independent. $A_1, A_2$ are two maximal monotone operators (cf. Subsection \ref{mmo}), these mappings $b_{1}:\mR^{n}\times\cP_2(\mR^n)\times\mR^{m}\rightarrow\mR^{n}$, $\s_{1} :\mR^{n}\times\cP_2(\mR^n)\times\mR^{m}\rightarrow\mR^{n\times d_1}$, $b_{2} :\mR^{n}\times\cP_2(\mR^n)\times\mR^{m}\rightarrow\mR^{m}$, $\s_{2} :\mR^{n}\times\cP_2(\mR^n)\times\mR^{m}\rightarrow\mR^{m\times d_2}$ are all Borel measurable, and $\cP_{2}(\mR^n)$ is the set of probability measures on $\sB(\mR^n)$ with finite second moments (cf. Subsection \ref{nn}). $\sL_{X_{t}}$ denotes the distribution of $X_{t}$ under the probability measure $\mP$. $\xi$ is a $\sF_0$-measurable $\overline{\cD(A_1)}$-valued random variable with $\mE|\xi|^2<\infty$ and independent of $W:=(W^1,W^2)$. Here $0<\e<1$ is a small parameter, $\t\geq 0$ is a constant and $\d=\d(\e)$ represents the other small parameter (depending on $\e$) which characterizes the ratio of timescales between processes $X_{\cdot}^{\e,\d}$ and $Y_{\cdot}^{\e,\d}$. 

If $A_1=A_2=0$ and $\t=0$, the system (\ref{Eqin1}) goes into the following system:
\be\left\{\begin{array}{l}
\dif X_{t}^{\d}=b_{1}(X_{t}^{\d},\sL_{X_{t}^{\d}},Y_{t}^{\d})\dif t+\s_{1}(X_{t}^{\d},\sL_{X_{t}^{\d}},Y_{t}^{\d})\dif W^1_{t},\\
X_{0}^{\d}=\xi,\quad  0\leq t\leq T,\\
\dif Y_{t}^{\d}=\frac{1}{\d}b_{2}(X_{t}^{\d},\sL_{X_{t}^{\d}},Y_{t}^{\d})\dif t+\frac{1}{\sqrt{\d}}\s_{2}(X_{t}^{\d},\sL_{X_{t}^{\d}},Y_{t}^{\d})\dif W^2_{t},\\
Y_{0}^{\d}=y_0,\quad  0\leq t\leq T.
\end{array}
\right.
\label{Eqin2}
\ee
The system (\ref{Eqin2}) is called a multiscale McKean-Vlasov stochastic system. For the system (\ref{Eqin2}) and more general systems, there have been a lot of average principle results (cf. \cite{bs2, lwx, qw2} for weak convergence and \cite{qw1, rsx, XLLM} for strong convergence). 

If $A_1=A_2=0$ and $\t=\frac{1}{2}$, the system (\ref{Eqin1}) becomes the following system:
\be\left\{\begin{array}{l}
\dif X_{t}^{\e,\d}=b_{1}(X_{t}^{\e,\d},\sL_{X_{t}^{\e,\d}},Y_{t}^{\e,\d})\dif t+\sqrt{\e}\s_{1}(X_{t}^{\e,\d},\sL_{X_{t}^{\e,\d}},Y_{t}^{\e,\d})\dif W^1_{t},\\
X_{0}^{\e,\d}=\xi,\quad  0\leq t\leq T,\\
\dif Y_{t}^{\e,\d}=\frac{1}{\d}b_{2}(X_{t}^{\e,\d},\sL_{X_{t}^{\e,\d}},Y_{t}^{\e,\d})\dif t+\frac{1}{\sqrt{\d}}\s_{2}(X_{t}^{\e,\d},\sL_{X_{t}^{\e,\d}},Y_{t}^{\e,\d})\dif W^2_{t},\\
Y_{0}^{\e,\d}=y_0,\quad  0\leq t\leq T.
\end{array}
\right.
\label{Eqin3}
\ee
We call the system (\ref{Eqin3}) a multiscale McKean-Vlasov stochastic system with a small noise. In \cite{hlls}, Hong et al. mentioned that the slow part $X^{\e,\d}$ converges to an average system in the $L^{2}$ sense under suitable assumptions. Later, in infinite dimensional Hilbert spaces Gao et al. \cite{ghl} proved that the slow part $X^{\e,\d}$ converges to an average system in the $L^{2}$ sense.

If $A_1=\p \varphi, A_2=\p \psi$, where $\varphi, \psi$ are lower semicontinuous convex functions on $\mR^n, \mR^m$, respectively, and $\p \varphi, \p \psi$ are corresponding subdifferential operators (cf. Example \ref{exmmo1}), $\t=0$ and $b_1,\s_1,b_2,\s_2$ are independent of the distribution in the system (\ref{Eqin1}), for different $\s_1$ Chen and Wu \cite{cw} established corresponding average principles. Very recently, the author \cite{q1} improved the result in \cite{cw} to the case of general $A_1, A_2$ and $\t=\frac{1}{2}$.

For the system (\ref{Eqin1}), as far as we know, there are not related average principle results. However, these systems have appeared as models for many fields, such as biology and chemistry (cf. \cite{ahlw, lw}), which motivates us to study the system (\ref{Eqin1}). First of all, we establish the well-posedness for general multivalued McKean-Vlasov stochastic systems under non-Lipschitz conditions. Then for the system (\ref{Eqin1}), we obtain four different average principles according to $\t>0$ and $\t=0$.

The asymptotic theory of large deviation principles (LDPs for short) quantifies the rate of convergence for the probability of rare events. And a Freidlin-Wentzell LDP provides an estimate for the probability that the sample path of an It\^o diffusion will stray far from the mean path when the size of the driving Brownian motion is small with respect to a pathspace norm. Freidlin-Wentzell LDPs have been established for many equations, such as multiscale SDEs (\cite{ds, rl, kp, aap, av2, av3}), multiscale multivalued SDEs (\cite{ku, q1}), multivalued McKean-Vlasov SDEs (\cite{arrst, flqz}) and multiscale McKean-Vlasov SDEs (\cite{bs1, bs3, ghl, hlls}). For multiscale multivalued McKean-Vlasov stochastic systems like (\ref{Eqin1}), as far as we know, there are not related LDP results. Therefore, we study the LDP for the system (\ref{Eqin1}) with $\t=\frac{1}{2}$.

The novelty of this paper lies in two folds. The first fold is that the class of systems like (\ref{Eqin1}) is new. Note that the system (\ref{Eqin1}) has multiscale, multivalue and McKean-Vlasov structures. Thus, it contains many equations, such as multiscale SDEs, multiscale multivalued SDEs and multiscale McKean-Vlasov SDEs. Therefore, our results can be applied to many models. The second fold is that in special cases we establish some orders of convergence. Comparing our results with some known results, we find that the appearance of maximal monotone operators $A_1, A_2$ reduces the order of convergence (cf. Remark \ref{conrat1} $(i)$ and \ref{conrat2} $(i)$). Moreover, we try to improve the order of convergence by some techniques like Poisson equations (cf. \cite{rsx}). Unfortunately they don't seem to work for the system (\ref{Eqin1}).  

The rest of this paper is organized as follows. In Section \ref{pre} we introduce some notations and concepts. And the formulation of main results is placed in Section \ref{main}. Section \ref{wellproo}, \ref{proofirs}, and \ref{prooseco} cover the well-posedness, average principles and LDP results, respectively. Finally, we give an example to explain our results in Section \ref{exam}.

The following convention will be used throughout the paper: $C$ with or without indices will denote different positive constants whose values may change from one place to another.

\section{Priliminary}\label{pre}

In this section, we will recall some notations and concepts.

\subsection{Notations}\label{nn}
In this subsection, we introduce some notations used in the sequel.

Let $|\cdot|, \|\cdot\|$ be the norms of a vector and a matrix, respectively. Let $\langle\cdot,\cdot\rangle$ be the inner product of vectors on $\mR^n$. $U^{*}$ denotes the transpose of the matrix $U$.

Let $C(\mR^n)$ be the set of all  functions which are continuous on $\mR^n$. $C^{1}(\mR^n)$ represents the collection of all functions in $C(\mR^n)$ with continuous derivatives of order $1$. 

Let $\sB(\mR^n)$ be the Borel $\sigma$-algebra on $\mR^n$ and $\cP({\mR^n})$ be the space of all probability measures defined on $\sB(\mR^n)$ carrying the usual topology of the weak convergence. Let $\cP_{2}(\mR^n)$ be the set of probability measures on $\sB(\mR^n)$ with finite second order moments, i.e.
$$
\cP_2\left( \mathbb{R}^n \right) :=\left\{ \mu \in \cP\left( \mathbb{R}^n \right): \|\mu\|^{2}:=\int_{\mathbb{R}^n}{\left| x \right|^2\mu \left( \dif x \right) <\infty} \right\}.
$$
It is known that $\cP_2(\mR^n)$ is a Polish space endowed with the $L^2$-Wasserstein distance defined by
$$
\mathbb{W}_2(\mu,\nu):= \inf\limits_{\pi\in\Psi(\mu,\nu)}\left(\int_{\mathbb{R}^n\times\mathbb{R}^n}|x-y|^{2}\pi(\dif x,\dif y)\right)^{\frac{1}{2}}, \quad \mu , \nu\in \cP_2(\mR^n),
$$
where $\Psi(\mu,\nu)$ is the set of all couplings $\pi$ with marginal distributions $\mu$ and $\nu$. Moreover, if $\xi,\zeta$ are two random variables with distributions $\sL_\xi, \sL_\zeta$ under $\mP$, respectively,
$$
\mathbb{W}^2_2(\sL_\xi, \sL_\zeta)\leq\mE|\xi-\zeta|^2,
$$
where $\mE$ stands for the expectation with respect to $\mP$.

\subsection{Maximal monotone operators}\label{mmo}

In this subsection, we introduce maximal monotone operators.

For a multivalued operator $A: \mR^n\mapsto 2^{\mR^n}$, where $2^{\mR^n}$ stands for all the subsets of $\mR^n$, set
\ce
&&\cD(A):= \left\{x\in \mR^n: A(x) \ne \emptyset\right\},\\
&&Gr(A):= \left\{(x,y)\in \mR^{2n}:x \in \cD(A), ~ y\in A(x)\right\}.
\de
We say that $A$ is monotone if $\langle x_1 - x_2, y_1 - y_2 \rangle \geq 0$ for any $(x_1,y_1), (x_2,y_2) \in Gr(A)$, and $A$ is maximal monotone if 
$$
(x_1,y_1) \in Gr(A) \iff \langle x_1-x_2, y_1 -y_2 \rangle \geq 0, \quad \forall (x_2,y_2) \in Gr(A).
$$

For readers to understand maximal monotone operators very well, we give two examples.

\bx\label{exmmo1}
For a lower semicontinuous convex function $\psi:\mR^n\mapsto(-\infty, +\infty]$, we assume ${\rm Int}(Dom(\psi))\neq \emptyset$, where $Dom(\psi)\equiv\{x\in\mR^n; \psi(x)<\infty\}$ and $\operatorname{Int}(Dom(\psi))$ is the interior of $Dom(\psi)$. Define the subdifferential operator of the function $\psi$:
$$
\partial\psi(x):=\{y\in\mR^n: \<y,z-x\>+\psi(x)\leq \psi(z), \forall z\in\mR^n\}.
$$
Then $\partial\psi$ is a maximal monotone operator. 
\ex

\bx\label{exmmo2}
For a closed convex subset $\mathcal{O}$ of $\mathbb{R}^n$, we suppose $\operatorname{Int}(\mathcal{O})\neq\emptyset$. Define the indicator function of $\mathcal{O}$ as follows:
$$
I_{\mathcal{O}}(x):= \begin{cases}0, & \text { if } x \in \mathcal{O}, \\ 
+\infty, & \text { if } x \notin \mathcal{O}.\end{cases}
$$
The subdifferential operator of $I_{\mathcal{O}}$ is given by
$$
\begin{aligned}
\partial I_{\mathcal{O}}(x) & :=\left\{y \in \mathbb{R}^n:\langle y, x-z\rangle \geq 0, \forall z \in \mathcal{O}\right\} \\
& = \begin{cases}\emptyset, & \text { if } x \notin \mathcal{O}, \\
\{0\}, & \text { if } x \in \operatorname{Int}(\mathcal{O}), \\
\Lambda_x, & \text { if } x \in \partial \mathcal{O},\end{cases}
\end{aligned}
$$
where $\Lambda_x$ is the exterior normal cone at $x$. By simple deduction, we know that $\partial I_{\mathcal{O}}$ is a maximal monotone operator.
\ex
In the following, we recall some properties of a maximal monotone operator $A$ (cf.\cite{cepa1}):
\begin{enumerate}[(i)]
\item
${\rm Int}(\cD(A))$ and $\overline{\mathrm{\cD}(A)}$ are convex subsets of $\mR^n$ with ${\rm Int}\left( \overline{\mathrm{\cD}(A)} \right) = {\rm Int}\( \mathrm{\cD}(A) \) 
$, where ${\rm Int}(\cD(A))$ denotes the interior of the set $\cD(A)$. 
\item For every $x\in\mR^n$, $A(x)$ is a closed and convex subset of $\mR^n$.
\end{enumerate}

Take any $T>0$ and fix it. Let $\sV_{0}$ be the set of all continuous functions $K: [0,T]\mapsto\mR^n$ with finite variations and $K_{0} = 0$. For $K\in\sV_0$ and $s\in [0,T]$, we shall use $|K|_{0}^{s}$ to denote the variation of $K$ on [0,s]
and write $|K|_{TV}:=|K|_{0}^{T}$. Set
\ce
&&\sA:=\Big\{(X,K): X\in C([0,T],\overline{\cD(A)}), K \in \sV_0, \\
&&\qquad\qquad\quad~\mbox{and}~\langle X_{t}-x, \dif K_{t}-y\dif t\rangle \geq 0 ~\mbox{for any}~ (x,y)\in Gr(A)\Big\}.
\de
And about $\sA$ we have two following results (cf.\cite{cepa2, ZXCH}).

\bl\label{equi}
For $X\in C([0,T],\overline{\cD(A)})$ and $K\in \sV_{0}$, the following statements are equivalent:
\begin{enumerate}[(i)]
	\item $(X,K)\in \sA$.
	\item For any $(x,y)\in C([0,T],\mR^d)$ with $(x_t, y_t)\in Gr(A)$, it holds that 
	$$
	\left\langle X_t-x_t, \dif K_t-y_t\dif t\right\rangle \geq0.
	$$
	\item For any $(X^{'},K^{'})\in \sA$, it holds that 
	$$
	\left\langle X_t-X_t^{'},\dif K_t-\dif K_t^{'}\right\rangle \geq0.
	$$
\end{enumerate}
\el

\bl\label{inteineq}
Assume that $\text{Int}(\cD(A))\ne\emptyset$. For any $a\in \text{Int}(\cD(A))$, there exist $M_1 >0$, and $M_{2},M_{3}\geq0$ such that  for any $(X,K)\in \sA$ and $0\leq s<t\leq T$,
$$
\int_s^t{\left< X_r-a, \dif K_r \right>}\geq M_1\left| K \right|_{s}^{t}-M_2\int_s^t{\left| X_r-a\right|}\dif r-M_3\left( t-s \right) .
$$
\el
 
\subsection{Multivalued McKean-Vlasov SDEs}

In this subsection, we introduce multivalued McKean-Vlasov SDEs. 

Fix $T>0$ and consider the following multivalued McKean-Vlasov SDE on $\mR^n$:
\be
\dif X_t\in -A(X_t)\dif t+b(X_t,\sL_{X_t})\dif t+\s(X_t,\sL_{X_t})\dif W^1_t, \quad 0\leq t\leq T, \label{eq1}
\ee
where $A$ is a maximal monotone operator with $\text{Int}(\cD(A))\ne\emptyset$, the coefficients $b: \mR^n\times\cP_2(\mR^n)\mapsto{\mR^n}, \,\,\sigma:\mR^n\times\cP_2(\mR^n)\mapsto{\mR^n}\times{\mR^{d_1}}$ are Borel measurable and $W^1_{\cdot}$ is a $d_1$-dimensional Brownian motion on a filtered probability space $(\Omega, \mathscr{F}, \{\mathscr{F}_t\}_{t\in[0,T]}, \mP)$.

\bd\label{strosolu}
We say that Eq.$(\ref{eq1})$ admits a strong solution with the initial value $X_0\in\overline{\cD(A)}$ if there exists a pair of adapted processes $(X,K)$ on $(\Omega, \mathscr{F}, \{\mathscr{F}_t\}_{t\in[0,T]}, \mP)$ such that

(i) $X_t\in{\mathscr{F}_t^{W^1}}$, where $\{\mathscr{F}_t^{W^1}\}_{t\in[0,T]}$ stands for the $\sigma$-field filtration generated by $W^1$,

(ii) $(X_{\cdot}(\omega),K_{\cdot}(\omega))\in \sA$ a.s. $\mP$,

(iii) it holds that
\ce
\mP\left\{\int_0^T(\mid{b(X_s,\sL_{X_s})}\mid+\parallel{\sigma(X_s,\sL_{X_s})}\parallel^2)\dif s<+\infty\right\}=1,
\de
and
\ce
X_t=X_0-K_{t}+\int_0^tb(X_s,\sL_{X_s})\dif s+\int_0^t\sigma(X_s,\sL_{X_s})\dif W^1_s, \quad 0\leq{t}\leq{T}, \quad a.s.~\mP.
\de
\ed

\subsection{A general criterion of large deviation principles}

In this subsection, we present a general criterion to establish the large deviation principle. 

Let $(\mS,\rho)$ be a Polish space. For each $\e>0$, let $X^{\e}$ be a $\mS$-valued random variable given on $(\Omega, \mathscr{F}, \{\mathscr{F}_t\}_{t\in[0,T]}, \mP)$.

\bd\label{compleve}
The function $I: \mathbb{S}\mapsto [0,\infty]$ is called a rate function if $I$ is lower semicontinuous. Moreover, a rate function $I$ is called a good rate function if for each $M<\infty$, $\{\varsigma\in \mathbb{S}:I(\varsigma)\leq M\}$ is a compact subset of $\mathbb{S}$.
\ed

\bd
We say that $\{X^{\epsilon}\}$ satisfies the large deviation principle with the speed $\e^{-1}$ and the good rate function $I$, if for any subset $B\in \sB(\mS)$,
$$
-\inf\limits_{\varsigma\in \text{Int}(B)}I(\varsigma)\leq\liminf_{\e\rightarrow 0}\e\log\mP(X^{\epsilon}\in\text{Int}(B))\leq \limsup\limits_{\e\rightarrow 0}\e\log\mP(X^{\epsilon}\in \bar{B})\leq -\inf\limits_{\varsigma\in \bar{B}}I(\varsigma),
$$
where the closure and the interior are taken in $\mS$.
\ed

\bd
We say that $\{X^{\epsilon}\}$ satisfies the Laplace principle with the speed $\e^{-1}$ and the good rate function $I$,  if for any real bounded continuous function $G$ on $\mathbb{S}$,
\ce
\lim\limits_{\varepsilon\rightarrow 0}\varepsilon \log \mE\left\{\exp\left[-\frac{G(X^{\epsilon})}{\epsilon}\right]\right\}=-\inf\limits_{\varsigma\in \mathbb{S}}\(G(\varsigma)+I(\varsigma)\).
\de
\ed

Note that the large deviation principle is equivalent to the Laplace principle (cf. \cite{BDM2}). Therefore, in order to obtain the large deviation principle for $\{X^{\epsilon}\}$, we prove the Laplace principle for $\{X^{\epsilon}\}$. Then we state the conditions under which the Laplace principle holds. Set $\mathbb{H}:=L^{2}([0,T]; \mR^{d_1+d_2})$ and $\|h\|_{\mathbb{H}}:=(\int_{0}^{T}|h(t)|^{2}\dif t)^{\frac{1}{2}}$ for $h\in\mH$. Let $\mathcal{A}$ be the collection of predictable processes $u(\omega, \cdot)$ belonging to $\mathbb{H}$ a.s. $\omega$. For each $N\in\mN$ we define two following spaces:
\ce
\mathbf{D}_{2}^{N}:=\left\{h\in\mathbb{H}: \|h\|_{\mathbb{H}}^{2}\leq N \right\}, \quad \mathbf{A}_{2}^{N}:=\left\{u\in\mathcal{A}: u(\omega, \cdot)\in\mathbf{D}_{2}^{N}, a.s.~\omega \right\}.
\de
We equip $\mathbf{D}_{2}^{N}$ with the weak convergence topology in $\mH$. So, $\mathbf{D}_{2}^{N}$ is metrizable as a compact Polish space. In the sequel, $\mathbf{D}_{2}^{N}$ will be always endowed with this topology.

\bco\label{cond}
Let $\cG^{\epsilon} : C([0,T];\mathbb{R}^{d_1+d_2})\mapsto\mS$ be  a family of measurable mappings. There exists a
measurable mapping $\cG^{0} : C([0,T];\mathbb{R}^{d_1+d_2})\mapsto\mS$ such that

$(i)$ for $\{h_{\e}, \e>0\}\subset\mathbf{D}_2^{N}$, $h\in \mathbf{D}_2^{N}$, if $h_{\e}\rightarrow h$ as $\e\rightarrow 0$, then
\ce
\cG^{0}\left(\int_{0}^{\cdot}h_{\e}(s)\dif s\right)\longrightarrow \cG^{0}\left(\int_{0}^{\cdot}h(s)\dif s\right).
\de

$(ii)$ for $\{u_{\epsilon},\epsilon>0\}\subset \mathbf{A}_{2}^{N}$, and any $\eta>0$,
$$
\lim\limits_{\e\rightarrow0}\mP\left(\rho\left(\cG^{\e}\left(\sqrt{\e}W_\cdot+\int_{0}^{\cdot}u_{\e}(s)\dif s\right),\cG^{0}\left(\int_{0}^{\cdot}u_\e(s)\dif s\right)\right)>\eta\right)=0,
$$
where $W_\cdot$ is a $d_1+d_2$-dimensional Brownian motion.

\eco

 Given $\varsigma\in\mS$, let ${\bf D}_{\varsigma}=\{h\in\mH: \varsigma=\cG^{0}(\int_{0}^{\cdot}h(s)\dif s)\}$. Let $I:\mathbb{S}\mapsto [0,\infty]$ be defined by
$$
I(\varsigma)=\frac{1}{2}\inf\limits_{h\in{\bf D}_{\varsigma}}\|h\|_{\mathbb{H}}^{2}.
$$
The following result is due to \cite[Theorem 3.2]{msz}.

\bt\label{ldpbase}
Set $X^{\epsilon}:=\cG^{\epsilon}(\sqrt{\e}W)$. Assume that Condition \ref{cond} holds. Then $\{X^{\epsilon}\}$ satisfies the Laplace principle with the good rate function $I$ given above. In particular, $\{X^{\epsilon}\}$ satisfies the large deviation principle with the same rate function $I$.
\et

\section{Main results}\label{main}

In this section, we formulate the main results in this paper.

\subsection{Well-posedness of multivalued McKean-Vlasov stochastic systems}

In this subsection, we present the well-posedness result for multivalued McKean-Vlasov stochastic systems.

Consider the following system on $\mR^{n} \times \mR^{m}$:
\be\left\{\begin{array}{l}
\dif X_{t}\in -A_1(X_{t})\dif t+b_{1}(X_{t},\sL_{X_{t}},Y_{t})\dif t+\s_{1}(X_{t},\sL_{X_{t}},Y_{t})\dif W^1_{t},\\
X_{0}=\xi\in\overline{\cD(A_1)},\quad  0\leq t\leq T,\\
\dif Y_{t}\in -A_2(Y_{t})\dif t+b_{2}(X_{t},\sL_{X_{t}},Y_{t})\dif t+\s_{2}(X_{t},\sL_{X_{t}},Y_{t})\dif W^2_{t},\\
Y_{0}=y_0\in\overline{\cD(A_2)},\quad  0\leq t\leq T.
\end{array}
\right.
\label{Eqeu}
\ee

Assume:
\begin{enumerate}[$(\mathbf{H}_{A_1})$]
\item $\text{Int}(\cD(A_1))\ne\emptyset$.
\end{enumerate}
\begin{enumerate}[$(\mathbf{H}^1_{b_{1}, \s_{1}})$]
\item $(i)$ For $y\in\mR^m$, $b_1(x,\mu,y)$ is continuous in $(x,\mu)$, and there exists a constant $L_{b_{1}, \s_{1}}>0$ such that for $x_1,x_2\in\mR^n$, $\mu_1, \mu_2\in\cP_{2}(\mR^n)$, $y\in\mR^m$, 
\ce
&&2\<x_1-x_2, b_{1}(x_{1},\mu_1,y)-b_{1}(x_{2},\mu_2,y)\>\leq L_{b_{1},\s_{1}}\(|x_{1}-x_{2}|^2+\mW^2_2(\mu_1,\mu_2)\),\\
&&\|\s_{1}(x_{1},\mu_1,y)-\s_{1}(x_{2},\mu_2,y)\|^2\leq L_{b_{1},\s_{1}}\(|x_{1}-x_{2}|^2+\mW^2_2(\mu_1,\mu_2)\),
\de
and for $x\in\mR^n$, $\mu\in\cP_{2}(\mR^n)$, $y_i\in\mR^m, i=1,2$
\ce
|b_{1}(x,\mu,y_1)-b_{1}(x,\mu,y_2)|^2+\|\s_{1}(x,\mu,y_1)-\s_{1}(x,\mu,y_2)\|^2\leq L_{b_{1},\s_{1}}|y_1-y_2|^2.
\de
$(ii)$ There exists a constant $\bar{L}_{b_{1}, \s_{1}}>0$ such that for $x\in\mR^n$, $\mu\in\cP_{2}(\mR^n)$, $y\in\mR^m$,
\be
|b_{1}(x,\mu,y)|^{2}+\|\s_{1}(x,\mu,y)\|^{2}\leq \bar{L}_{b_{1}, \s_{1}}(1+|x|^{2}+\|\mu\|^2+|y|^{2}).
\label{b1line}
\ee
\end{enumerate}
\begin{enumerate}[$(\mathbf{H}_{A_2})$]
\item $0\in \text{Int}(\cD(A_2))$.
\end{enumerate}
\begin{enumerate}[$(\mathbf{H}^1_{b_{2}, \s_{2}})$]
\item There exists a constant $L_{b_{2}, \s_{2}}>0$ such that for $x_{i}\in\mR^n$, $\mu_{i}\in\cP_{2}(\mR^n)$, $i=1, 2$, $y\in\mR^m$,
\ce
&&|b_{2}(x_{1},\mu_1,y)-b_{2}(x_{2},\mu_2,y)|^{2}+\|\s_{2}(x_{1},\mu_1,y)-\s_{2}(x_{2},\mu_2,y)\|^{2}\\
&\leq& L_{b_{2}, \s_{2}}\(|x_{1}-x_{2}|^{2}+\mW^2_2(\mu_1,\mu_2)\).
\de
\end{enumerate}
\begin{enumerate}[$(\mathbf{H}^2_{b_{2}, \s_{2}})$]
\item For $(x,\mu)\in\mR^n\times\cP_{2}(\mR^n)$, $b_2(x,\mu,y)$ is continuous in $y$, and there exists a constant $\bar{L}_{b_{2}, \s_{2}}>0$ such that for $x\in\mR^n$, $\mu\in\cP_{2}(\mR^n)$, $y_i\in\mR^m, i=1,2$
\ce
&&2\<y_1-y_2, b_{2}(x,\mu,y_1)-b_{2}(x,\mu,y_2)\>\leq\bar{L}_{b_{2},\s_{2}}|y_{1}-y_{2}|^2,\\
&&\|\s_{2}(x,\mu,y_1)-\s_{2}(x,\mu,y_2)\|^2\leq\bar{L}_{b_{2},\s_{2}}|y_1-y_2|^2.
\de
\end{enumerate}
\begin{enumerate}[$(\mathbf{H}^3_{b_{2}, \s_{2}})$]
\item There exists a constant $\bar{\bar{L}}_{b_{2}, \s_{2}}>0$ such that for $x\in\mR^n$, $\mu\in\cP_{2}(\mR^n)$, $y\in\mR^m$, 
\be
|b_{2}(x,\mu,y)|^{2}+\|\s_{2}(x,\mu,y)\|^{2}
\leq \bar{\bar{L}}_{b_{2}, \s_{2}}(1+|x|^{2}+\|\mu\|^2+|y|^{2}).
\label{b2nu}
\ee
\end{enumerate}

\br
$(\mathbf{H}_{A_{2}})$ can be replaced by $\text{Int}(\cD(A_2))\ne\emptyset$. Indeed, if we require $\text{Int}(\cD(A_2))\ne\emptyset$, for any $a\in\text{Int}(\cD(A_2))$, by shifting the domain of $A_2$ and defining $\tilde{b}_2(x,\mu,y)= b_2(x,\mu,y-a), \tilde{\s}_2(x,\mu,y)= \s_2(x,\mu,y-a)$, this situation becomes the case of $0\in\text{Int}(\cD(A_2))$.
\er

Now, it is the position to state the main result in this section.

\bt\label{well}
Assume that $(\mathbf{H}_{A_{1}})$, $(\mathbf{H}_{A_{2}})$, $(\mathbf{H}^1_{b_{1}, \s_{1}})$, $(\mathbf{H}^1_{b_{2}, \s_{2}})$-$(\mathbf{H}^3_{b_{2}, \s_{2}})$ hold. Then the system (\ref{Eqeu}) has a unique strong solution.
\et

The proof of the above theorem is placed in Section \ref{wellproo}.

\subsection{Average principles for multiscale multivalued McKean-Vlasov stochastic systems}\label{averprin}

In this subsection, we state the average principle results for multiscale multivalued McKean-Vlasov stochastic systems.

We recall the system (\ref{Eqin1}), i.e.
\be\left\{\begin{array}{l}
\dif X_{t}^{\e,\d}\in -A_1(X_{t}^{\e,\d})\dif t+b_{1}(X_{t}^{\e,\d},\sL_{X_{t}^{\e,\d}},Y_{t}^{\e,\d})\dif t+\e^\t\s_{1}(X_{t}^{\e,\d},\sL_{X_{t}^{\e,\d}},Y_{t}^{\e,\d})\dif W^1_{t},\\
X_{0}^{\e,\d}=\xi\in\overline{\cD(A_1)},\quad  0\leq t\leq T,\\
\dif Y_{t}^{\e,\d}\in -A_2(Y_{t}^{\e,\d})\dif t+\frac{1}{\d}b_{2}(X_{t}^{\e,\d},\sL_{X_{t}^{\e,\d}},Y_{t}^{\e,\d})\dif t+\frac{1}{\sqrt{\d}}\s_{2}(X_{t}^{\e,\d},\sL_{X_{t}^{\e,\d}},Y_{t}^{\e,\d})\dif W^2_{t},\\
Y_{0}^{\e,\d}=y_0\in\overline{\cD(A_2)},\quad  0\leq t\leq T.
\end{array}
\right.
\label{Eqall}
\ee

We also assume:
\begin{enumerate}[$(\mathbf{H}^{1'}_{b_{1}, \s_{1}})$]
\item
There exists a constant $L'_{b_{1}, \s_{1}}>0$ such that for $x_{i}\in\mR^n$, $\mu_{i}\in\cP_{2}(\mR^n)$, $y_{i}\in\mR^m$, $i=1, 2$,
\ce
&&|b_{1}(x_{1},\mu_1,y_{1})-b_{1}(x_{2},\mu_2,y_{2})|^{2}+\|\s_{1}(x_{1},\mu_1,y_1)-\s_{1}(x_{2},\mu_2,y_2)\|^{2}\\
&\leq& L'_{b_{1},\s_{1}}\(|x_{1}-x_{2}|^{2}+\mW^2_2(\mu_1,\mu_2)+|y_{1}-y_{2}|^{2}\).
\de
\end{enumerate}
\begin{enumerate}[$(\mathbf{H}^{2'}_{b_{2}, \s_{2}})$]
\item For $(x,\mu)\in\mR^n\times\cP_{2}(\mR^n)$, $b_2(x,\mu,y)$ is continuous in $y$, and there exist two constants $L'_{b_{2}, \s_{2}}, \b>0$ satisfying $\b>2L'_{b_2,\sigma_2}$ such that for $x\in\mR^n$, $\mu\in\cP_{2}(\mR^n)$, $y_{i}\in\mR^m$, $i=1, 2$,
\ce
\|\s_{2}(x,\mu,y_1)-\s_{2}(x,\mu,y_2)\|^2\leq L'_{b_{2},\s_{2}}|y_{1}-y_{2}|^2,
\de
\ce
2\<y_{1}-y_{2},b_{2}(x,\mu,y_{1})-b_{2}(x,\mu,y_{2})\>
+\|\s_{2}(x,\mu,y_{1})-\s_{2}(x,\mu,y_{2})\|^{2}\leq -\b|y_{1}-y_{2}|^{2}.
\de
\end{enumerate}

\br
$(i)$ $(\mathbf{H}^{1'}_{b_{1}, \s_{1}})$ is stronger than $(\mathbf{H}^{1}_{b_{1}, \s_{1}})$, and implies (\ref{b1line}). And $(\mathbf{H}^{1'}_{b_{1}, \s_{1}})$ is used to assure the well-posedness of average equations. 
 
$(ii)$ $(\mathbf{H}^{2'}_{b_{2}, \s_{2}})$ is stronger than $(\mathbf{H}^2_{b_{2}, \s_{2}})$. Moreover, by $(\mathbf{H}^{2'}_{b_{2}, \s_{2}})$ $(\mathbf{H}^3_{b_{2}, \s_{2}})$, it holds that for $x\in\mR^n$, $\mu\in\cP_{2}(\mR^n)$, $y\in\mR^m$
\be
2\<y,b_{2}(x,\mu,y)\>+\|\s_{2}(x,\mu,y)\|^{2}\leq -\a|y|^{2}+C(1+|x|^{2}+\|\mu\|^2),
\label{bemu}
\ee
where $\a:=\b-2L'_{b_2,\sigma_2}$ and $C>0$ is a constant. And $(\mathbf{H}^{2'}_{b_{2}, \s_{2}})$ guarantees the ergodicity of a frozen equation.
\er

By Theorem \ref{well}, under $(\mathbf{H}_{A_{1}})$, $(\mathbf{H}_{A_{2}})$, $(\mathbf{H}^{1'}_{b_{1}, \s_{1}})$, $(\mathbf{H}^1_{b_{2}, \s_{2}})$, $(\mathbf{H}^{2'}_{b_{2}, \s_{2}})$ and $(\mathbf{H}^3_{b_{2}, \s_{2}})$ we know that the system (\ref{Eqall}) has a unique strong solution $(X_{\cdot}^{\e,\d},K_{\cdot}^{1,\e,\d},Y_{\cdot}^{\e,\d},K_{\cdot}^{2,\e,\d})$.

Take any $x\in \overline{\cD(A_1)}, \mu\in\cP_2(\overline{\cD(A_1)})$ and fix them. Consider the following multivalued SDE:
\be\left\{\begin{array}{l}
\dif Y_{t}^{x,\mu}\in -A_2(Y_{t}^{x,\mu})\dif t+b_{2}(x,\mu,Y_{t}^{x,\mu})\dif t+\s_{2}(x,\mu,Y_{t}^{x,\mu})\dif W^2_{t},\\
Y_{0}^{x}=y_0\in\overline{\cD(A_2)}, \quad 0 \leq t \leq T.
\end{array}
\right.
\label{Eq2}
\ee
Under $(\mathbf{H}_{A_{2}})$, $(\mathbf{H}^{2'}_{b_{2}, \s_{2}})$-$(\mathbf{H}^3_{b_{2}, \s_{2}})$, we know that the above equation has a unique strong solution $(Y_{\cdot}^{x,\mu,y_0},K_{\cdot}^{2,x,\mu,y_0})$ (\cite{rwzx}). Moreover, by the same deduction as that of \cite[Theorem 3.2]{q2}, one could conclude that there exists a unique invariant probability measure $\nu^{x,\mu}$ for Eq.(\ref{Eq2}). 

Next, since for the cases of $\t>0$ and $\t=0$ the corresponding average equations are different, we divide them into two subsections to formulate the corresponding results.

\subsubsection{$\t>0$}

In this subsubsection, we state the average principle result for multiscale multivalued McKean-Vlasov stochastic systems with small noises.

Set $\bar{b}_{1}(x,\mu):=\int_{\mR^{m}}b_{1}(x,\mu,y)\nu^{x,\mu}(\dif y)$, and we construct the corresponding average equation as follows:
\be\left\{\begin{array}{l}
\dif \bar{X}^0_{t}\in -A_1(\bar{X}^0_{t})\dif t+\bar{b}_{1}(\bar{X}^0_{t},\sL_{\bar{X}_t^0})\dif t,\\
\bar{X}^0_{0}=\xi\in\overline{\cD(A_1)}.
\end{array}
\right.
\label{ldppsio0equ}
\ee
So, we have the following result.

\bt \label{xbarxp}
Suppose that $(\mathbf{H}_{A_{1}})$, $(\mathbf{H}_{A_{2}})$, $(\mathbf{H}^{1'}_{b_{1}, \s_{1}})$, $(\mathbf{H}^{1}_{b_{2}, \s_{2}})$, $(\mathbf{H}^{2'}_{b_{2}, \s_{2}})$, $(\mathbf{H}^{3}_{b_{2}, \s_{2}})$ hold. If
\ce
\lim_{\e\rightarrow 0}\frac{\d}{\e}=\left\{\begin{array}{l}0,\\
\iota\in(0,\infty),
\end{array}
\right.
\de
it holds that
\ce
\lim_{\e\rightarrow 0}\mE\(\sup_{0\leq t\leq T}|X_{t}^{\e,\d}-\bar{X}^0_{t}|^{2}\)=0,
\de
where $(\bar{X}^0_{\cdot},\bar{K}^0_{\cdot})$ is a solution of Eq.(\ref{ldppsio0equ}).
\et

\br
If $A_1=0, A_2=0$, $b_1,b_2$ are independent of the distribution $\sL_{X_{t}^{\e,\d}}$, $\s_1, \s_2$ are $n, m$-order unit matrixes, respectively, $n=d_1, m=d_2$, and $\lim\limits_{\e\rightarrow 0}\frac{\d}{\e}=\iota\in(0,\infty)$, 
the system (\ref{Eqall}) is the same to the system $(1)+(2)$ with $s(\e)=1$ in \cite{abks}. There Athreya et al. proved that $\{X_{t}^{\e,\d}, 0\leq t\leq T\}$ converges in law on $C([0,T]; \mR^n)$ to $\{\bar{X}^0_{t}, 0\leq t\leq T\}$ (cf. \cite[Theorem 1.3]{abks}). Here in 
Theorem \ref{xbarxp}, we show that $\{X_{t}^{\e,\d}, 0\leq t\leq T\}$ converges in the mean square sense on $C([0,T]; \mR^n)$ to $\{\bar{X}^0_{t}, 0\leq t\leq T\}$. Therefore, our result is stronger.
\er

In the following, we take special $A_1$ and obtain a rate of $X^{\e,\d}$ converging to $\bar{X}^0$.

\bt\label{xbarxpcr}
Suppose that $A_1=\p I_{\cO}$, where $\cO$ is a closed and convex domain in $\mR^n$ with ${\rm Int}(\cO)\neq\emptyset$, and $(\mathbf{H}_{A_{2}})$, $(\mathbf{H}^{1'}_{b_{1}, \s_{1}})$, $(\mathbf{H}^{1}_{b_{2}, \s_{2}})$, $(\mathbf{H}^{2'}_{b_{2}, \s_{2}})$, $(\mathbf{H}^{3}_{b_{2}, \s_{2}})$ hold. If
\ce
\lim_{\e\rightarrow 0}\frac{\d}{\e}=\left\{\begin{array}{l}0,\\
\iota\in(0,\infty),
\end{array}
\right.
\de
it holds that for $0<\g<1$
\ce
\mE\(\sup_{0\leq t\leq T}|X_{t}^{\e,\d}-\bar{X}^0_{t}|^{2}\)\leq C(\d^{\g/2}+\d^\g+\d^{\frac{1}{2}(1-\g)}+\e^\t).
\de
\et

\br\label{conrat1}
$(i)$ If $\g=\frac{1}{3}, \t=\frac{1}{2}$, by Theorem \ref{xbarxpcr}, it holds that
\ce
\mE\(\sup_{0\leq t\leq T}|X_{t}^{\e,\d}-\bar{X}^0_{t}|^{2}\)\leq C(\d^{\frac{1}{6}}+\e^{\frac{1}{2}}).
\de
Note that in \cite[Theorem 2.1]{ghl}, for the following system 
\ce\left\{\begin{array}{l}
\dif X_{t}^{\e,\d}=b_{1}(X_{t}^{\e,\d},\sL_{X_{t}^{\e,\d}},Y_{t}^{\e,\d},\sL_{Y_{t}^{\e,\d}})\dif t+\sqrt{\e}\s_{1}(X_{t}^{\e,\d},\sL_{X_{t}^{\e,\d}},Y_{t}^{\e,\d},\sL_{Y_{t}^{\e,\d}})\dif W^1_{t},\\
X_{0}^{\e,\d}=\xi,\quad  0\leq t\leq T,\\
\dif Y_{t}^{\e,\d}=\frac{1}{\d}b_{2}(Y_{t}^{\e,\d})\dif t+\frac{1}{\sqrt{\d}}\s_{2}(Y_{t}^{\e,\d})\dif W^2_{t},\\
Y_{0}^{\e,\d}=y_0,\quad  0\leq t\leq T,
\end{array}
\right.
\de
Gao et al. obtained that
\ce
\mE\(\sup_{0\leq t\leq T}|X_{t}^{\e,\d}-\bar{X}^0_{t}|^{2}\)\leq C(\d^{\frac{1}{3}}+\e).
\de
That is, $A_1, A_2$ reduce the order of convergence.

$(ii)$ If $\g=\frac{1}{2}$, $\d=\e^2$ and $\t=\frac{1}{2}$, by Theorem \ref{xbarxpcr}, it holds that
\ce
\mE\(\sup_{0\leq t\leq T}|X_{t}^{\e,\d}-\bar{X}^0_{t}|^{2}\)\leq C\e^{\frac{1}{2}}.
\de
Thus, inspired by \cite[Theorem 3.1]{ks1}, we will study the limiting behavior of the fluctuations process
$$
\frac{X_{t}^{\e,\d}-\bar{X}^0_{t}}{\sqrt{\e}}
$$
as $\e\rightarrow0$ in the forthcoming work.
\er

The proofs of Theorem \ref{xbarxp} and \ref{xbarxpcr} are placed in Section \ref{proofirs}.

\br
In Theorem \ref{xbarxp} and \ref{xbarxpcr} we don't consider the case of $\lim\limits_{\e\rightarrow 0}\frac{\d}{\e}=\infty$, that is, the speed for $\d$ converging to $0$ is slower than that for $\e$ converging to $0$. The reason is that if $\e=0$ and $\d\neq 0$, the system (\ref{Eqall}) reduces to 
a common multiscale multivalued stochastic system where there is no diffusion term in the slow equation. And Chen and Wu \cite{cw} have studied the average principle for this type of systems.
\er

\subsubsection{$\t=0$}

In this subsubsection, we state the average principle result for general multiscale multivalued McKean-Vlasov stochastic systems.

When $\t=0$, the system (\ref{Eqall}) becomes the following slow-fast system:
\be\left\{\begin{array}{l}
\dif X_{t}^{\d}\in -A_1(X_{t}^{\d})\dif t+b_{1}(X_{t}^{\d},\sL_{X_{t}^{\d}},Y_{t}^{\d})\dif t+\s_{1}(X_{t}^{\d},\sL_{X_{t}^{\d}},Y_{t}^{\d})\dif W^1_{t},\\
X_{0}^{\d}=\xi\in\overline{\cD(A_1)},\quad  0\leq t\leq T,\\
\dif Y_{t}^{\d}\in -A_2(Y_{t}^{\d})\dif t+\frac{1}{\d}b_{2}(X_{t}^{\d},\sL_{X_{t}^{\d}},Y_{t}^{\d})\dif t+\frac{1}{\sqrt{\d}}\s_{2}(X_{t}^{\d},\sL_{X_{t}^{\d}},Y_{t}^{\d})\dif W^2_{t},\\
Y_{0}^{\d}=y_0\in\overline{\cD(A_2)},\quad  0\leq t\leq T.
\end{array}
\right.
\label{Eqap}
\ee
And $(X_{\cdot}^{\d},K_{\cdot}^{1,\d},Y_{\cdot}^{\d},K_{\cdot}^{2,\d})$ denotes the unique strong solution to the system (\ref{Eqap}). Moreover, we require $\s_1(x,\mu,y)=\s_1(x,\mu)$ and construct the corresponding average 
equation on $(\Omega,\sF,\{\sF_{t}\}_{t\in[0,T]},\mP)$ as follows:
\be\left\{\begin{array}{l}
\dif \bar{X}_{t}\in -A_1(\bar{X}_{t})\dif t+\bar{b}_{1}(\bar{X}_{t},\sL_{\bar{X}_{t}})\dif t+\s_{1}(\bar{X}_{t},\sL_{\bar{X}_{t}})\dif W^1_{t},\\
\bar{X}_{0}=\xi\in\overline{\cD(A_1)}.
\end{array}
\right.
\label{Eq3}
\ee
The following theorem characterizes the relationship between $X^{\d}$ and $\bar{X}$.

\bt \label{xbarxap}
Suppose that $(\mathbf{H}_{A_{1}})$, $(\mathbf{H}_{A_{2}})$, $(\mathbf{H}^{1'}_{b_{1}, \s_{1}})$, $(\mathbf{H}^{1}_{b_{2}, \s_{2}})$, $(\mathbf{H}^{2'}_{b_{2}, \s_{2}})$, $(\mathbf{H}^{3}_{b_{2}, \s_{2}})$ hold. Then it holds that
\ce
\lim_{\d\rightarrow 0}\mE\(\sup_{0\leq t\leq T}|X_{t}^{\d}-\bar{X}_{t}|^{2}\)=0,
\de
where $(\bar{X}_{\cdot},\bar{K}_{\cdot})$ is a solution of Eq.(\ref{Eq3}).
\et

Next, we take special $A_1$ and also obtain a rate of $X^{\d}$ converging to $\bar{X}$.

\bt\label{xbarxapcr}
Suppose that $A_1=\p I_{\cO}$, where $\cO$ is a closed and convex domain in $\mR^n$ with ${\rm Int}(\cO)\neq\emptyset$, and $(\mathbf{H}_{A_{2}})$, $(\mathbf{H}^{1'}_{b_{1}, \s_{1}})$, $(\mathbf{H}^{1}_{b_{2}, \s_{2}})$, $(\mathbf{H}^{2'}_{b_{2}, \s_{2}})$, $(\mathbf{H}^{3}_{b_{2}, \s_{2}})$ hold. Then it holds that for $0<\g<1$
\ce
\mE\(\sup_{0\leq t\leq T}|X_{t}^{\d}-\bar{X}_{t}|^{2}\)\leq C(\d^{\g/2}+\d^\g+\d^{\frac{1}{2}(1-\g)}).
\de
\et

\br\label{conrat2}
$(i)$ If $\g=\frac{1}{2}$, by Theorem \ref{xbarxapcr}, we know that 
\ce
\mE\(\sup_{0\leq t\leq T}|X_{t}^{\d}-\bar{X}_{t}|^{2}\)\leq C\d^{\frac{1}{4}}.
\de
Thus, the order of convergence  is $\frac{1}{8}$. Note that in \cite{qw1}, the convergence rate is $\frac{1}{4}$. That is, maximal monotone operators $A_1, A_2$ result in the reduction for the order of convergence.

$(ii)$ If $b_1, \s_1, b_2, \s_2$ don't depend on the distribution $\sL_{X_{t}^{\d}}$, the framework in Theorem \ref{xbarxapcr} is similar to that in \cite[Theorem 4.4]{cw}. There Chen and Wu also obtained 
\ce
\mE\(\sup_{0\leq t\leq T}|X_{t}^{\d}-\bar{X}_{t}|^{2}\)\leq C\(f(\d^{-1/2}) + \d^{1/2}\)^{1/2},
\de
where $f: (0,\infty)\mapsto (0,\infty)$ is Borel measurable with $\lim\limits_{T\rightarrow\infty}f(T)=0$. Thus, it is not difficult to find that our model is more general and our convergence rate is more concrete.
\er

The proofs of Theorem \ref{xbarxap} and \ref{xbarxapcr} are also placed in Section \ref{proofirs}.

\br
In Theorem \ref{xbarxap} and \ref{xbarxapcr}, we don't consider the case of $\s_1(x,\mu,y)$ depending on $y$. The reason lies in that if $\s_1(x,\mu,y)$ contains the component $y$, we only study the weak convergence of $X^{\d}$ to $\bar{X}$. And the martingale characterization 
is a key step to prove this weak convergence. However, because the slow equation and the average equation are two multivalued McKean-Vlasov stochastic differential equations, it is difficult to describe the martingale problems with respect to these two equations (cf. \cite{gq}).
\er

\subsection{Large deviation principle for multiscale multivalued McKean-Vlasov stochastic systems}

In this subsection, we describe the large deviation principle result for multiscale multivalued McKean-Vlasov stochastic systems. 

Here we require that $\t=1/2$, $\s_1(x,\mu,y)=\s_1(x,\mu)$ and $\xi=x_0$ for $x_0\in\overline{\cD(A_1)}$. Thus, the system (\ref{Eqall}) goes into the following slow-fast system:
\be\left\{\begin{array}{l}
\dif X_{t}^{\e,\d}\in -A_1(X_{t}^{\e,\d})\dif t+b_{1}(X_{t}^{\e,\d},\sL_{X_{t}^{\e,\d}},Y_{t}^{\e,\d})\dif t+\e^{1/2}\s_{1}(X_{t}^{\e,\d},\sL_{X_{t}^{\e,\d}})\dif W^1_{t},\\
X_{0}^{\e,\d}=x_0\in\overline{\cD(A_1)},\quad  0\leq t\leq T,\\
\dif Y_{t}^{\e,\d}\in -A_2(Y_{t}^{\e,\d})\dif t+\frac{1}{\d}b_{2}(X_{t}^{\e,\d},\sL_{X_{t}^{\e,\d}},Y_{t}^{\e,\d})\dif t+\frac{1}{\sqrt{\d}}\s_{2}(X_{t}^{\e,\d},\sL_{X_{t}^{\e,\d}},Y_{t}^{\e,\d})\dif W^2_{t},\\
Y_{0}^{\e,\d}=y_0\in\overline{\cD(A_2)},\quad  0\leq t\leq T.
\end{array}
\right.
\label{Eq1}
\ee

We assume more:
\begin{enumerate}[$(\mathbf{H}^4_{\s_{2}})$]
\item There exists a constant $L_{\sigma_2}>0$ such that for $x\in\mR^n$, $\mu\in\cP_{2}(\mR^n)$, $y\in\mR^m$,
\ce
\|\s_2(x,\mu,y)\|\leq L_{\sigma_2}.
\de
\end{enumerate}

\br
$(\mathbf{H}^4_{\s_{2}})$ guarantees the moment boundedness of solutions for controlled equations.
\er

Now we present the main result in this subsection.

\bt\label{ldpmmsde}
Assume that $(\mathbf{H}_{A_{1}})$, $(\mathbf{H}_{A_{2}})$, $(\mathbf{H}^{1'}_{b_{1}, \s_{1}})$, $(\mathbf{H}^{1}_{b_{2}, \s_{2}})$, $(\mathbf{H}^{2'}_{b_{2}, \s_{2}})$, $(\mathbf{H}^{3}_{b_{2}, \s_{2}})$, $(\mathbf{H}^{4}_{\s_{2}})$ hold. If
\ce
\lim\limits_{\e\rightarrow 0}\frac{\d}{\e}=0,
\de
the family $\{X^{\epsilon,\d},\epsilon\in(0,1)\}$ satisfies the LDP in $\mS:=C([0,T],\overline{\mathcal{D}(A_1)})$ with the rate function given by
$$
I(\varsigma)=\frac{1}{2} \inf\limits_{h\in {\bf D}_{\varsigma}: \varsigma=\bar{X}^{h}}\|h\|_{\mH}^2,
$$
where $(\bar{X}^{0},\bar{K}^{0})$ solves the following equation
\be\left\{\begin{array}{l}
\dif \bar{X}^0_{t}\in -A_1(\bar{X}^0_{t})\dif t+\bar{b}_{1}(\bar{X}^0_{t},D_{\bar{X}_t^0})\dif t,\\
\bar{X}^0_{0}=x_0\in\overline{\cD(A_1)},
\end{array}
\right.
\label{ldppsio0equde}
\ee
$D_{\bar{X}_t^0}$ is the Dirac measure at $\bar{X}_t^0$, and $(\bar{X}^{h},\bar{K}^{h})$ solves the following equation
\be\left\{\begin{array}{l}
\dif \bar{X}^h_{t}\in -A_1(\bar{X}^h_{t})\dif t+\bar{b}_{1}(\bar{X}^h_{t},D_{\bar{X}_t^0})\dif t+\s_{1}(\bar{X}^h_{t},D_{\bar{X}_t^0})\pi_1h(t)\dif t,\\
\bar{X}^h_{0}=x_0\in\overline{\cD(A_1)},
\end{array}
\right.
\label{ldppsioequ}
\ee
and $\pi_1: \mR^{d_1+d_2}\mapsto \mR^{d_1}$ is a projection operator.
\et

The proof of Theorem \ref{ldpmmsde} is placed in Section \ref{prooseco}.

\br
We mention that, although we use the same notation for Eq.(\ref{ldppsio0equ}) and Eq.(\ref{ldppsio0equde}), the solution of Eq.(\ref{ldppsio0equ}) is random and the solution of Eq.(\ref{ldppsio0equ}) is determine. 
\er

\br
Here we can't deal with the cases of $\lim\limits_{\e\rightarrow 0}\frac{\d}{\e}=\iota\in(0,\infty)$ and $\lim\limits_{\e\rightarrow 0}\frac{\d}{\e}=\infty$. This is because $\lim\limits_{\e\rightarrow 0}\frac{\d}{\e}=0$ is important for verification of Condition \ref{cond} (ii). That is, if $\lim\limits_{\e\rightarrow 0}\frac{\d}{\e}\neq 0$, the LDP for $\{X^{\epsilon,\d},\epsilon\in(0,1)\}$ does not seem to hold.
\er


\section{Proof of Theorem \ref{well}}\label{wellproo}

In this section, we prove Theorem \ref{well}.

{\bf Proof of Theorem \ref{well}.}
Set for any $t\in[0,T]$ $X^{(0)}_t=\xi, \sL_{X^{(0)}_t}=\sL_{\xi}$ and consider the following equations: for $l\in\mN$,
\be
&&\dif Y^{(l)}_{t}\in -A_2(Y^{(l)}_{t})\dif t+b_{2}(X^{(l-1)}_{t},\sL_{X^{(l-1)}_{t}},Y^{(l)}_{t})\dif t+\s_{2}(X^{(l-1)}_{t},\sL_{X^{(l-1)}_{t}},Y^{(l)}_{t})\dif W^2_{t},\no\\
&&\quad Y^{(l)}_{0}=y_0, \label{yapp}\\
&&\dif X^{(l)}_{t}\in -A_1(X^{(l)}_{t})\dif t+b_{1}(X^{(l)}_{t},\sL_{X^{(l)}_{t}},Y^{(l)}_{t})\dif t+\s_{1}(X^{(l)}_{t},\sL_{X^{(l)}_{t}},Y^{(l)}_{t})\dif W^1_{t},\no\\
&&\quad X^{(l)}_{0}=\xi.\label{xapp}
\ee

{\bf Step 1.} We prove that both Eq.(\ref{yapp}) and Eq.(\ref{xapp}) have unique strong solutions.

For $l=1$, by \cite{rwzx}, Eq.(\ref{yapp}) has a unique strong solution $(Y^{(1)},K^{2,(1)})$ with 
$$
\mE\sup\limits_{t\in[0,T]}|Y^{(1)}_t|^2\leq C(1+|y_0|^2), \quad \mE|K^{2,(1)}|_0^T\leq C(1+|y_0|^2).
$$
Then by \cite{gq}, we know that Eq.(\ref{xapp}) has a unique strong solution $(X^{(1)},K^{1,(1)})$ satisfying
$$
\mE\sup\limits_{t\in[0,T]}|X^{(1)}_t|^2\leq C(1+\mE|\xi|^2+|y_0|^2), \quad \mE|K^{1,(1)}|_0^T\leq C(1+\mE|\xi|^2+|y_0|^2).
$$

Assume that $X^{(l-1)}$ is well-defined and $\mE\sup\limits_{t\in[0,T]}|X^{(l-1)}_t|^2\leq C(1+\mE|\xi|^2+|y_0|^2)$. By the similar deduction to the above, there exists a unique strong solution $(Y^{(l)},K^{2,(l)})$ of Eq.(\ref{yapp}). Moreover, the It\^o formula implies that
\ce
|Y^{(l)}_t|^2&=&|y_0|^2-2\int_0^t\<Y^{(l)}_s,\dif K^{2,(l)}_s\>+2\int_0^t\<Y^{(l)}_s,b_{2}(X^{(l-1)}_{s},\sL_{X^{(l-1)}_{s}},Y^{(l)}_{s})\>\dif s\\
&&+2\int_0^t\<Y^{(l)}_s,\s_{2}(X^{(l-1)}_{s},\sL_{X^{(l-1)}_{s}},Y^{(l)}_{s})\dif W^2_{s}\>+\int_0^t\|\s_{2}(X^{(l-1)}_{s},\sL_{X^{(l-1)}_{s}},Y^{(l)}_{s})\|^2\dif s\\
&\leq&|y_0|^2-2M_1|K^{2,(l)}|_0^t+2M_2\int_0^t|Y^{(l)}_s|\dif s+2M_3t+\int_0^t|Y^{(l)}_s|^2\dif s\\
&&+\bar{L}_{b_1,\s_1}\int_0^t(1+|X^{(l-1)}_{s}|^2+\|\sL_{X^{(l-1)}_{s}}\|^2+|Y^{(l)}_{s}|^2)\dif s\\
&&+2\int_0^t\<Y^{(l)}_s,\s_{2}(X^{(l-1)}_{s},\sL_{X^{(l-1)}_{s}},Y^{(l)}_{s})\dif W^2_{s}\>\\
&\leq&|y_0|^2+(2M_2+2M_3+\bar{L}_{b_1,\s_1})T+(2M_2+1+\bar{L}_{b_1,\s_1})\int_0^t|Y^{(l)}_s|^2\dif s\\
&&+\bar{L}_{b_1,\s_1}\int_0^t(|X^{(l-1)}_{s}|^2+\|\sL_{X^{(l-1)}_{s}}\|^2)\dif s+2\int_0^t\<Y^{(l)}_s,\s_{2}(X^{(l-1)}_{s},\sL_{X^{(l-1)}_{s}},Y^{(l)}_{s})\dif W^2_{s}\>.
\de
The BDG inequality yields that
\ce
\mE\sup\limits_{s\in[0,t]}|Y^{(l)}_s|^2&\leq&C(1+|y_0|^2)+C\int_0^t\mE\sup\limits_{r\in[0,s]}|Y^{(l)}_r|^2\dif s+C(1+\mE|\xi|^2+|y_0|^2)\\
&&+C\mE\left(\int_0^t|Y^{(l)}_s|^2\|\s_{2}(X^{(l-1)}_{s},\sL_{X^{(l-1)}_{s}},Y^{(l)}_{s})\|^2\dif s\right)^{1/2}\\
&\leq&C(1+|y_0|^2)+C\int_0^t\mE\sup\limits_{r\in[0,s]}|Y^{(l)}_r|^2\dif s+C(1+\mE|\xi|^2+|y_0|^2)\\
&&+\frac{1}{2}\mE\sup\limits_{s\in[0,t]}|Y^{(l)}_s|^2+C\mE\int_0^t\|\s_{2}(X^{(l-1)}_{s},\sL_{X^{(l-1)}_{s}},Y^{(l)}_{s})\|^2\dif s,
\de
where the fact $\|\sL_{X^{(l-1)}_{s}}\|^2=\mE|X^{(l-1)}_{s}|^2$ is used. By the Gronwall inequality, we conclude that 
\ce
\sup\limits_{l}\mE\sup\limits_{s\in[0,T]}|Y^{(l)}_s|^2\leq C(1+\mE|\xi|^2+|y_0|^2),
\de
which together with Lemma \ref{inteineq} implies that
\ce
\sup\limits_{l}\mE|K^{2,(l)}|_0^T\leq C(1+\mE|\xi|^2+|y_0|^2).
\de
Moreover, the same deduction to the above yields that there exists a unique strong solution $(X^{(l)},K^{1,(l)})$ of Eq.(\ref{xapp}) and
\ce
\sup\limits_{l}\mE\sup\limits_{s\in[0,T]}|X^{(l)}_s|^2\leq C(1+\mE|\xi|^2+|y_0|^2), \quad \sup\limits_{l}\mE|K^{1,(l)}|_0^T\leq C(1+\mE|\xi|^2+|y_0|^2).
\de

Finally, by induction, we obtain that both Eq.(\ref{yapp}) and Eq.(\ref{xapp}) have unique strong solutions.

{\bf Step 2.} We prove the existence of solutions to the system (\ref{Eqeu}).

Applying the It\^o formula to $|Y^{(l)}_t-Y^{(l-1)}_t|^2$, we obtain that
\ce
&&|Y^{(l)}_t-Y^{(l-1)}_t|^2\\
&=&-2\int_0^t\<Y^{(l)}_s-Y^{(l-1)}_s,\dif (K^{2,(l)}_s-K^{2,(l-1)}_s)\>\\
&&+2\int_0^t\<Y^{(l)}_s-Y^{(l-1)}_s,b_{2}(X^{(l-1)}_{s},\sL_{X^{(l-1)}_{s}},Y^{(l)}_{s})-b_{2}(X^{(l-2)}_{s},\sL_{X^{(l-2)}_{s}},Y^{(l-1)}_{s})\>\dif s\\
&&+2\int_0^t\<Y^{(l)}_s-Y^{(l-1)}_s,\(\s_{2}(X^{(l-1)}_{s},\sL_{X^{(l-1)}_{s}},Y^{(l)}_{s})-\s_{2}(X^{(l-2)}_{s},\sL_{X^{(l-2)}_{s}},Y^{(l-1)}_{s})\)\dif W_s^2\>\\
&&+\int_0^t\|\s_{2}(X^{(l-1)}_{s},\sL_{X^{(l-1)}_{s}},Y^{(l)}_{s})-\s_{2}(X^{(l-2)}_{s},\sL_{X^{(l-2)}_{s}},Y^{(l-1)}_{s})\|^2\dif s\\
&\leq&C\int_0^t|Y^{(l)}_s-Y^{(l-1)}_s|^2\dif s+\int_0^t|b_{2}(X^{(l-1)}_{s},\sL_{X^{(l-1)}_{s}},Y^{(l-1)}_{s})-b_{2}(X^{(l-2)}_{s},\sL_{X^{(l-2)}_{s}},Y^{(l-1)}_{s})|^2\dif s\\
&&+2\int_0^t\<Y^{(l)}_s-Y^{(l-1)}_s,\(\s_{2}(X^{(l-1)}_{s},\sL_{X^{(l-1)}_{s}},Y^{(l)}_{s})-\s_{2}(X^{(l-2)}_{s},\sL_{X^{(l-2)}_{s}},Y^{(l-1)}_{s})\)\dif W_s^2\>\\
&&+\int_0^t\|\s_{2}(X^{(l-1)}_{s},\sL_{X^{(l-1)}_{s}},Y^{(l)}_{s})-\s_{2}(X^{(l-2)}_{s},\sL_{X^{(l-2)}_{s}},Y^{(l-1)}_{s})\|^2\dif s.
\de
By taking the expectation on two sides and noticing the fact $\mW_2^2(\sL_{X^{(l-1)}_{s}},\sL_{X^{(l-2)}_{s}})\leq\mE|X^{(l-1)}_s-X^{(l-2)}_s|^2$, it holds that
\ce
\mE|Y^{(l)}_t-Y^{(l-1)}_t|^2&\leq& C\int_0^t\mE|Y^{(l)}_s-Y^{(l-1)}_s|^2\dif s+C\int_0^T\mE|X^{(l-1)}_s-X^{(l-2)}_s|^2\dif s\\
&&+C\int_0^T\mW_2^2(\sL_{X^{(l-1)}_{s}},\sL_{X^{(l-2)}_{s}})\dif s\\
&\leq& C\int_0^t\mE|Y^{(l)}_s-Y^{(l-1)}_s|^2\dif s+C\int_0^T\mE|X^{(l-1)}_s-X^{(l-2)}_s|^2\dif s,
\de
which together with the Gronwall inequality yields that
\be
\mE|Y^{(l)}_t-Y^{(l-1)}_t|^2\leq C\int_0^T\mE|X^{(l-1)}_s-X^{(l-2)}_s|^2\dif s.
\label{ylyl1}
\ee

Besides, by the similar deduction to the above, it holds that
\be
&&\mE\sup\limits_{s\in[0,T]}|X^{(l)}_s-X^{(l-1)}_s|^2\no\\
&\leq& C\int_0^T\mE\sup\limits_{r\in[0,s]}|X^{(l)}_r-X^{(l-1)}_r|^2\dif s+C\int_0^T\mE|Y^{(l)}_s-Y^{(l-1)}_s|^2\dif s.
\label{xlxl1}
\ee
Inserting (\ref{ylyl1}) into the above inequality and taking the superior limit, by the Fatou lemma we obtain that 
\ce
\limsup\limits_{l\rightarrow\infty}\mE\sup\limits_{s\in[0,T]}|X^{(l)}_s-X^{(l-1)}_s|^2\leq C\int_0^T\limsup\limits_{l\rightarrow\infty}\mE\sup\limits_{r\in[0,s]}|X^{(l)}_r-X^{(l-1)}_r|^2\dif s.
\de
From this, it follows that $\{X^{(l)}\}$ is a Cauchy sequence in $L^2(\Omega, C([0,T],\overline{\cD(A_1)}))$ and there exists a process $\check{X}$ satisfying 
\be
\lim\limits_{l\rightarrow\infty}\mE\sup\limits_{s\in[0,T]}|X^{(l)}_s-\check{X}_s|^2=0.
\label{xlchex}
\ee

Next, we construct the following equation:
\be\left\{\begin{array}{l}
\dif \check{Y}_{t}\in -A_2(\check{Y}_{t})\dif t+b_{2}(\check{X}_{t},\sL_{\check{X}_{t}},\check{Y}_{t})\dif t+\s_{2}(\check{X}_{t},\sL_{\check{X}_{t}},\check{Y}_{t})\dif W^2_{t},\\
\check{Y}_{0}=y_0.
\end{array}
\right.
\label{chey}
\ee
Then by \cite{rwzx}, Eq.(\ref{chey}) has a unique strong solution $(\check{Y},\check{K}^{2})$. Using $\check{Y}$, we consider the following equation:
\be\left\{\begin{array}{l}
\dif X_{t}\in -A_1(X_{t})\dif t+b_{1}(X_{t},\sL_{X_{t}},\check{Y}_{t})\dif t+\s_{1}(X_{t},\sL_{X_{t}},\check{Y}_{t})\dif W^2_{t},\\
X_{0}=\xi.
\end{array}
\right.
\label{chex}
\ee
The result in \cite{gq} assures that Eq.(\ref{chex}) has a unique strong solution $(X,K^1)$. If we show $X_{t}=\check{X}_{t}, t\in[0,T]$, a.s., $(X,K^1,\check{Y},\check{K}^{2})$ is a solution of the system (\ref{Eqeu}).

In the following, we prove $X_{t}=\check{X}_{t}, t\in[0,T]$, a.s.. By the similar deduction to that for (\ref{ylyl1}) and (\ref{xlxl1}), it holds that
\ce
\mE|\check{Y}_t-Y^{(l)}_t|^2\leq C\int_0^T\mE|\check{X}_s-X^{(l-1)}_s|^2\dif s,
\de
and
\ce
&&\mE\sup\limits_{s\in[0,T]}|X_s-X^{(l)}_s|^2\no\\
&\leq& C\int_0^T\mE\sup\limits_{r\in[0,s]}|X_r-X^{(l)}_r|^2\dif s+C\int_0^T\mE|\check{Y}_s-Y^{(l)}_s|^2\dif s\\
&\leq& C\int_0^T\mE\sup\limits_{r\in[0,s]}|X_r-X^{(l)}_r|^2\dif s+C\int_0^T\mE\sup\limits_{r\in[0,s]}|\check{X}_r-X^{(l-1)}_r|^2\dif s.
\de
Based on the above deduction and (\ref{xlchex}), we conclude that 
$$
\lim\limits_{l\rightarrow\infty}\mE\sup\limits_{s\in[0,T]}|X_s-X^{(l)}_s|^2=0, 
$$
which together with (\ref{xlchex}) yields that $X_{t}=\check{X}_{t}, t\in[0,T]$, a.s..

{\bf Step 3.} We prove the uniqueness of solutions to the system (\ref{Eqeu}). 

Assume that $(X_{\cdot},K_{\cdot}^{1},Y_{\cdot},K_{\cdot}^{2}), (\hat{X}_{\cdot},\hat{K}_{\cdot}^{1},\hat{Y}_{\cdot},\hat{K}_{\cdot}^{2})$ are two solutions of the system (\ref{Eqeu}). Then the similar deduction to that for (\ref{ylyl1}) and (\ref{xlxl1}) implies that
\ce
\mE|Y_t-\hat{Y}_t|^2\leq C\int_0^T\mE|X_s-\hat{X}_s|^2\dif s,
\de
and
\ce
&&\mE\sup\limits_{s\in[0,T]}|X_s-\hat{X}_s|^2\no\\
&\leq& C\int_0^T\mE\sup\limits_{r\in[0,s]}|X_r-\hat{X}_r|^2\dif s+C\int_0^T\mE|X_s-\hat{X}_s|^2\dif s\\
&\leq& C\int_0^T\mE\sup\limits_{r\in[0,s]}|X_r-\hat{X}_r|^2\dif s+C\int_0^T\mE\sup\limits_{r\in[0,s]}|X_r-\hat{X}_r|^2\dif s.
\de
From this, we conclude that $X_s=\hat{X}_s, Y_s=\hat{Y}_s, s\in[0,T]$, $\mP$ a.s.. 

Next, since both $(X_{\cdot},K_{\cdot}^{1},Y_{\cdot},K_{\cdot}^{2})$ and $(\hat{X}_{\cdot},\hat{K}_{\cdot}^{1},\hat{Y}_{\cdot},\hat{K}_{\cdot}^{2})$ satisfy the system (\ref{Eqeu}), we know that for $t\in[0,T]$
\ce
K_t^1&=&\xi+\int_0^t b_{1}(X_{s},\sL_{X_{s}},Y_{s})\dif s+\int_0^t\s_{1}(X_{s},\sL_{X_{s}},Y_{s})\dif W^1_{s}-X_t\\
&=&\xi+\int_0^t b_{1}(\hat{X}_{s},\sL_{\hat{X}_{s}},\hat{Y}_{s})\dif s+\int_0^t\s_{1}(\hat{X}_{s},\sL_{\hat{X}_{s}},\hat{Y}_{s})\dif W^1_{s}-\hat{X}_t\\
&=&\hat{K}_t^1,
\de
which together with the continuity of $K^1, \hat{K}^1$ in $t$ implies that $K^1_s=\hat{K}^1_s, s\in[0,T]$, $\mP$ a.s.. By the same deduction to the above, it holds that $K^2_s=\hat{K}^2_s, s\in[0,T]$, $\mP$ a.s.. The proof is complete.

\section{Proofs of Theorem \ref{xbarxp}, \ref{xbarxpcr}, \ref{xbarxap} and \ref{xbarxapcr}}\label{proofirs}

In this section, we prove Theorem \ref{xbarxp}, \ref{xbarxpcr}, \ref{xbarxap}  and \ref{xbarxapcr}. To do them, we make preparations.

\subsection{Some estimates for the frozen equation (\ref{Eq2})} 

In this subsection, we collect some estimates for the frozen equation (\ref{Eq2}) which will be used in the sequel.

\bl\label{emb1}
 Suppose that $(\mathbf{H}^{1'}_{b_{1}, \s_{1}})$, $(\mathbf{H}_{A_{2}})$, $(\mathbf{H}^{2'}_{b_{2}, \s_{2}})$ hold. Then there exists a constant $C>0$ such that for any $t\in[0, T], x\in\overline{\cD(A_1)}, \mu\in\cP_2(\overline{\cD(A_1)}), y\in\overline{\cD(A_2)}$ 
\be
&&\mE|Y_{t}^{x,\mu,y}|^{2}\leq|y|^{2}e^{-\frac{\a}{2} t}+C(1+|x|^{2}+\|\mu\|^2), \label{memu2}\\
&&\int_{\overline{\cD(A_2)}}|u|^2\nu^{x,\mu}(\dif u)\leq C(1+|x|^{2}+\|\mu\|^2),\\
&&|\mE b_{1}(x,\mu,Y_{t}^{x,\mu,y})-\bar{b}_{1}(x,\mu)|^{2}\leq Ce^{-\a t}(1+|x|^{2}+\|\mu\|^2+|y|^{2}).
\label{meu2}
\ee
\el

Since the proof of the above lemma is similar to that of \cite[Lemma 4.2 and 4.3]{q1}, we omit it.

\subsection{Proof of Theorem \ref{xbarxp}}

In this subsection, we show Theorem \ref{xbarxp}. The proof divides into two parts. In the first part (Subsubsection \ref{xtzthatxhatz}), we introduce an auxiliary process: 
\be\left\{\begin{array}{l}
\dif \hat{Y}_{t}^{\e,\d}\in -A_2(\hat{Y}_{t}^{\e,\d})\dif t+\frac{1}{\d}b_{2}(X_{k\triangle}^{\e,\d},\sL_{X_{k\triangle}^{\e,\d}},\hat{Y}_{t}^{\e,\d})\dif t
+\frac{1}{\sqrt{\d}}\s_{2}(X_{k\triangle}^{\e,\d},\sL_{X_{k\triangle}^{\e,\d}},\hat{Y}_{t}^{\e,\d})\dif W^2_{t}, \\
\qquad\qquad\qquad t\in[k\triangle, (k+1)\triangle),\\
\hat{Y}_{k\triangle}^{\e,\d}=Y_{k\triangle}^{\e,\d}, \quad \hat{K}_{k\triangle}^{2,\e,\d}=K_{k\triangle}^{2,\e,\d},
\end{array}
\right.
\label{hatz}
\ee
where $\triangle$ is a fixed positive number depending on $\d$, and $k=0,1,\cdots,[\frac{T}{\triangle}]$, and $[\frac{T}{\triangle}]$ denotes the integer part of $\frac{T}{\triangle}$. Moreover, we mention the fact that $[\frac{t}{\triangle}]=k$ for $t\in[k\triangle, (k+1)\triangle)$. $(\hat{Y}^{\e,\d}, \hat{K}^{2,\e,\d})$ denotes the strong solution of Eq.(\ref{hatz}). Then we estimate $X^{\e,\d}, K^{1,\e,\d}, Y^{\e,\d}, \hat{Y}^{\e,\d}$. In the second part (Subsubsection \ref{froequ}), we present some estimates for the average equation (\ref{ldppsio0equ}).

\subsubsection{Some necessary estimates}\label{xtzthatxhatz}

\bl \label{xgygb}
 Under the assumptions of Theorem \ref{xbarxp}, there exists a constant $C>0$ such that 
\be
&&\mE\left(\sup\limits_{t\in[0,T]}|X_{t}^{\e,\d}|^{2}\right)\leq C(1+\mE|\xi|^{2}+|y_0|^{2}), \label{xgb}\\
&&\sup\limits_{t\in[0,T]}\mE|Y_{t}^{\e,\d}|^{2}\leq C(1+\mE|\xi|^{2}+|y_0|^{2}).\label{ygb}
\ee
\el
\begin{proof}
First of all, we estimate $X_{t}^{\e,\d}$. Note that $X_{t}^{\e,\d}$ satisfies the following equation:
\ce
X_{t}^{\e,\d}=\xi-K_t^{1,\e,\d}+\int_0^t b_{1}(X_{s}^{\e,\d},\sL_{X_{s}^{\e,\d}},Y_{s}^{\e,\d})\dif s+\e^\t\int_0^t\s_{1}(X_{s}^{\e,\d},\sL_{X_{s}^{\e,\d}},Y_{s}^{\e,\d})\dif W^1_{s}.
\de
For any $a\in\text{Int}(\cD(A_1))$, by applying the It\^o formula to $|X_{t}^{\e,\d}-a|^2$, it holds that 
\be
|X_{t}^{\e,\d}-a|^2&=&|\xi-a|^2-2\int_0^t\<X_{s}^{\e,\d}-a, \dif K_s^{1,\e,\d}\>+2\int_0^t\<X_{s}^{\e,\d}-a, b_{1}(X_{s}^{\e,\d},\sL_{X_{s}^{\e,\d}},Y_{s}^{\e,\d})\>\dif s\no\\
&&+2\e^\t\int_0^t\<X_{s}^{\e,\d}-a,\s_{1}(X_{s}^{\e,\d},\sL_{X_{s}^{\e,\d}},Y_{s}^{\e,\d})\dif W^1_{s}\>\no\\
&&+\e^{2\t}\int_0^t\|\s_{1}(X_{s}^{\e,\d},\sL_{X_{s}^{\e,\d}},Y_{s}^{\e,\d})\|^2\dif s, 
\label{xgito}
\ee
and furthermore 
\ce 
|X_{t}^{\e,\d}-a|^2&\leq&|\xi-a|^2-2M_1|K^{1,\e,\d}|_0^t+2M_2\int_0^t|X_{s}^{\e,\d}-a|\dif s+2M_3 t+\int_0^t|X_{s}^{\e,\d}-a|^2\dif s\\
&&+\int_0^t|b_{1}(X_{s}^{\e,\d},\sL_{X_{s}^{\e,\d}},Y_{s}^{\e,\d})|^2\dif s+2\e^\t\left|\int_0^t\<X_{s}^{\e,\d}-a,\s_{1}(X_{s}^{\e,\d},\sL_{X_{s}^{\e,\d}},Y_{s}^{\e,\d})\dif W^1_{s}\>\right|\\
&&+\e^{2\t}\int_0^t\|\s_{1}(X_{s}^{\e,\d},\sL_{X_{s}^{\e,\d}},Y_{s}^{\e,\d})\|^2\dif s,
\de
where Lemma \ref{inteineq} is used in the above inequality. Then the BDG inequality and $(\ref{b1line})$ imply that
\ce
\mE\left(\sup\limits_{s\in[0,t]}|X_{s}^{\e,\d}-a|^{2}\right)&\leq&(\mE|\xi-a|^2+2(M_2+M_3)T)+(2M_2+1)\mE\int_0^t|X_{r}^{\e,\d}-a|^2\dif r\\
&&+\bar{L}_{b_{1}, \s_{1}}\mE\int_0^t(1+|X_{r}^{\e,\d}|^2+\|\sL_{X_{r}^{\e,\d}}\|^2+|Y_{r}^{\e,\d}|^2)\dif r\no\\
&&+2\mE\sup\limits_{s\in[0,t]}\left|\int_0^s\<X_{r}^{\e,\d}-a,\s_{1}(X_{r}^{\e,\d},\sL_{X_{r}^{\e,\d}},Y_{r}^{\e,\d})\dif W^1_{r}\>\right|\no\\ 
&\leq&C(\mE|\xi-a|^{2}+1)+C\mE\int_{0}^{t}|X_{r}^{\e,\d}-a|^2\dif r+C\mE\int_{0}^{t}|Y_{r}^{\e,\d}|^{2}\dif r\no\\
&&+C\mE\left(\int_0^t|X_{r}^{\e,\d}-a|^2\|\s_{1}(X_{r}^{\e,\d},\sL_{X_{r}^{\e,\d}},Y_{r}^{\e,\d})\|^2\dif r\right)^{1/2}\no\\
&\leq&C(\mE|\xi-a|^{2}+1)+C\mE\int_{0}^{t}|X_{r}^{\e,\d}-a|^2\dif r+C\mE\int_{0}^{t}|Y_{r}^{\e,\d}|^{2}\dif r\no\\
&&+\frac{1}{2}\mE\left(\sup\limits_{s\in[0,t]}|X_{s}^{\e,\d}-a|^{2}\right)+C\mE\int_0^t\|\s_{1}(X_{r}^{\e,\d},\sL_{X_{r}^{\e,\d}},Y_{r}^{\e,\d})\|^2\dif r,
\de
and furthermore
\be
\mE\left(\sup\limits_{s\in[0,t]}|X_{s}^{\e,\d}-a|^{2}\right)\leq C(\mE|\xi-a|^{2}+1)+C\int_{0}^{t}\mE|X_{r}^{\e,\d}-a|^{2}\dif r+C\int_{0}^{t}\mE|Y_{r}^{\e,\d}|^{2}\dif r.
\label{exqc0}
\ee

For $Y_{t}^{\e,\d}$, applying the It\^{o} formula to $|Y_{t}^{\e,\d}|^{2}e^{\l t}$ for $\l=\frac{\a}{2\d}$ and taking the expectation, one could obtain that for any $v\in A_2(0)$
\ce
\mE|Y_{t}^{\e,\d}|^{2}e^{\l t}&=& |y_0|^{2}+\l\mE\int_0^t|Y_{s}^{\e,\d}|^{2}e^{\l s}\dif s-2\mE\int_0^te^{\l s}\<Y_{s}^{\e,\d},\dif K_s^{2,\e,\d}\>\\
&&+\frac{2}{\d}\mE\int_{0}^{t}e^{\l s}\<Y_{s}^{\e,\d}, b_{2}(X_{s}^{\e,\d},\sL_{X_{s}^{\e,\d}},Y_{s}^{\e,\d})\>\dif s\\
&&+\frac{1}{\d}\mE\int_{0}^{t}e^{\l s}\|\s_{2}(X_{s}^{\e,\d},\sL_{X_{s}^{\e,\d}},Y_{s}^{\e,\d})\|^2\dif s\\
&\overset{(\ref{bemu})}{\leq}& |y_0|^{2}+\l\mE\int_0^t|Y_{s}^{\e,\d}|^{2}e^{\l s}\dif s+2\mE\int_0^te^{\l s}|v||Y_{s}^{\e,\d}|\dif s\\
&&+\frac{1}{\d}\mE\int_{0}^{t}e^{\l s}\(-\a|Y_{s}^{\e,\d}|^{2}+C(1+|X_{s}^{\e,\d}|^{2}+\|\sL_{X_{s}^{\e,\d}}\|^2)\)\dif s\\
&\leq& |y_0|^{2}+\(\l+\frac{\a}{2\d}-\frac{\a}{\d}\)\mE\int_0^t|Y_{s}^{\e,\d}|^{2}e^{\l s}\dif s+\frac{2\d}{\a}|v|^2\int_0^te^{\l s}\dif s\\
&&+\frac{C}{\d}\mE\int_{0}^{t}e^{\l s}(1+|X_{s}^{\e,\d}|^{2}+\mE|X_{s}^{\e,\d}|^{2})\dif s\\
&\leq& |y_0|^{2}+\frac{2|v|^2}{\a}\frac{e^{\l t}-1}{\l}+C\frac{e^{\l t}-1}{\d\l}\left(\mE\left(\sup\limits_{s\in[0,t]}|X_{s}^{\e,\d}|^{2}\right)+1\right),
\de
where Lemma \ref{equi} is used. From this, it follows that
\be
\mE|Y_{t}^{\e,\d}|^{2}\leq C(|y_0|^{2}+1)+C\mE\left(\sup\limits_{s\in[0,t]}|X_{s}^{\e,\d}|^{2}\right).
\label{yges}
\ee

Inserting (\ref{yges}) in (\ref{exqc0}), by the Gronwall inequality one can get that
\ce
\mE\left(\sup\limits_{t\in[0,T]}|X_{t}^{\e,\d}|^{2}\right)&\leq& 2\mE\left(\sup\limits_{t\in[0,T]}|X_{t}^{\e,\d}-a|^{2}\right)+2|a|^2\\
&\leq& C(1+\mE|\xi-a|^{2}+|y_0|^{2})+2|a|^2\\
&\leq& C(1+\mE|\xi|^{2}+|y_0|^{2}),
\de
which together with (\ref{yges}) implies that
$$
\sup\limits_{t\in[0,T]}\mE|Y_{t}^{\e,\d}|^{2}\leq C(1+\mE|\xi|^{2}+|y_0|^{2}).
$$
The proof is complete.
\end{proof}

By the same deduction to that in Lemma \ref{xgygb}, we obtain the following estimate.
\bl
Under the assumptions of Theorem \ref{xbarxp}, it holds that 
\be
\sup\limits_{t\in[0,T]}\mE|\hat{Y}_{t}^{\e,\d}|^{2}\leq C(1+\mE|\xi|^{2}+|y_0|^{2}).
 \label{hatygb}
\ee
\el

\bl\label{xehatxets}
Under the assumptions of Theorem \ref{xbarxp}, we have that
\be
\lim\limits_{l\rightarrow 0}\sup\limits_{s\in[0,T]}\mE\sup _{s \leqslant t \leqslant s+l}|X_{t}^{\e,\d}-X_{s}^{\e,\d}|^{2}=0.
 \label{xegts}
\ee
\el

Since the proof of the above lemma is similar to that for \cite[Lemma 5.3]{q1}, we omit it. 

\bl
 Suppose that the assumptions of Theorem \ref{xbarxp} hold. Then there exists a constant $C>0$ such that 
 \be
\sup\limits_{t\in[0,T]}\mE|Y_{t}^{\e,\d}-\hat{Y}_{t}^{\e,\d}|^{2}\leq \frac{C}{\a}
\left(\sup\limits_{s\in[0,T]}\mE\sup _{s \leqslant r \leqslant s+\triangle}|X_{r}^{\e,\d}-X_{s}^{\e,\d}|^{2}\right).
\label{unzt}
\ee
\el
\begin{proof}
First of all, by (\ref{Eq1}) and (\ref{hatz}), we have that for $t\in[k\triangle,(k+1)\triangle)$
\ce
Y_{t}^{\e,\d}-\hat{Y}_{t}^{\e,\d}
&=&-K_t^{2,\e,\d}+\hat{K}_t^{2,\e,\d}+\frac{1}{\d}\int_{k\triangle}^{t}\(b_{2}(X_{s}^{\e,\d},\sL_{X_{s}^{\e,\d}},Y_{s}^{\e,\d})
-b_{2}(X_{k\triangle}^{\e,\d},\sL_{X_{k\triangle}^{\e,\d}},\hat{Y}_{s}^{\e,\d})\)\dif s\\
&&+\frac{1}{\sqrt{\d}}\int_{k\triangle}^{t}\(\s_{2}(X_{s}^{\e,\d},\sL_{X_{s}^{\e,\d}},Y_{s}^{\e,\d})
-\s_{2}(X_{k\triangle}^{\e,\d},\sL_{X_{k\triangle}^{\e,\d}},\hat{Y}_{s}^{\e,\d})\)\dif W^2_{s}.
\de
Applying the It\^{o} formula to $|Y_{t}^{\e,\d}-\hat{Y}_{t}^{\e,\d}|^{2}e^{\l t}$ for $\l=\frac{\a}{\d}$ and taking the expectation, by $(\mathbf{H}^{1}_{b_{2},\s_{2}})$ and $(\mathbf{H}^{2'}_{b_{2},\s_{2}})$ one could obtain that
\ce
\mE|Y_{t}^{\e,\d}-\hat{Y}_{t}^{\e,\d}|^{2}e^{\l t}
&=&\l\mE\int_{k\triangle}^{t}|Y_{s}^{\e,\d}-\hat{Y}_{s}^{\e,\d}|^{2}e^{\l s}\dif s-\mE\int_{k\triangle}^{t}2e^{\l s}\<Y_{s}^{\e,\d}-\hat{Y}_{s}^{\e,\d},\dif (K_s^{2,\d}-\hat{K}_s^{2,\d})\>\\
&&+\frac{1}{\d}\mE\int_{k\triangle}^{t}2e^{\l s}\<Y_{s}^{\e,\d}-\hat{Y}_{s}^{\e,\d}, b_{2}(X_{s}^{\e,\d},\sL_{X_{s}^{\e,\d}},Y_{s}^{\e,\d})
-b_{2}(X_{k\triangle}^{\e,\d},\sL_{X_{k\triangle}^{\e,\d}},\hat{Y}_{s}^{\e,\d})\>\dif s\\
&&+\frac{1}{\d}\mE\int_{k\triangle}^{t}e^{\l s}\|\s_{2}(X_{s}^{\e,\d},\sL_{X_{s}^{\e,\d}},Y_{s}^{\e,\d})-\s_{2}(X_{k\triangle}^{\e,\d},\sL_{X_{k\triangle}^{\e,\d}},\hat{Y}_{s}^{\e,\d})\|^{2}\dif s\\
&\leq&\l\mE\int_{k\triangle}^{t}|Y_{s}^{\e,\d}-\hat{Y}_{s}^{\e,\d}|^{2}e^{\l s}\dif s\\
&&+\frac{1}{\d}\mE\int_{k\triangle}^{t}2e^{\l s}\<Y_{s}^{\e,\d}-\hat{Y}_{s}^{\e,\d},b_{2}(X_{s}^{\e,\d},\sL_{X_{s}^{\e,\d}},Y_{s}^{\e,\d})-b_{2}(X_{s}^{\e,\d},\sL_{X_{s}^{\e,\d}},\hat{Y}_{s}^{\e,\d})\>\dif s\\
&&+\frac{1}{\d}\mE\int_{k\triangle}^{t}e^{\l s}\|\s_{2}(X_{s}^{\e,\d},\sL_{X_{s}^{\e,\d}},Y_{s}^{\e,\d})-\s_{2}(X_{s}^{\e,\d},\sL_{X_{s}^{\e,\d}},\hat{Y}_{s}^{\e,\d})\|^{2}\dif s\\
&&+\frac{1}{\d}\mE\int_{k\triangle}^{t}2e^{\l s}\<Y_{s}^{\e,\d}-\hat{Y}_{s}^{\e,\d}, b_{2}(X_{s}^{\e,\d},\sL_{X_{s}^{\e,\d}},\hat{Y}_{s}^{\e,\d})-b_{2}(X_{k\triangle}^{\e,\d},\sL_{X_{k\triangle}^{\e,\d}},\hat{Y}_{s}^{\e,\d})\>\dif s\\
&&+\frac{1}{\d}\mE\int_{k\triangle}^{t}e^{\l s}\|\s_{2}(X_{s}^{\e,\d},\sL_{X_{s}^{\e,\d}},Y_{s}^{\e,\d})-\s_{2}(X_{s}^{\e,\d},\sL_{X_{s}^{\e,\d}},\hat{Y}_{s}^{\e,\d})\|^{2}\dif s\\
&&+\frac{1}{\d}\mE\int_{k\triangle}^{t}2e^{\l s}\|\s_{2}(X_{s}^{\e,\d},\sL_{X_{s}^{\e,\d}},\hat{Y}_{s}^{\e,\d})-\s_{2}(X_{k\triangle}^{\e,\d},\sL_{X_{k\triangle}^{\e,\d}},\hat{Y}_{s}^{\e,\d})\|^{2}\dif s\\
&\leq&(\l-\frac{\b}{\d}+\frac{2L'_{b_2,\s_2}}{\d})\mE\int_{k\triangle}^{t}|Y_{s}^{\e,\d}-\hat{Y}_{s}^{\e,\d}|^{2}e^{\l s}\dif s\\
&&+\frac{C}{\d}\mE\int_{k\triangle}^{t}e^{\l s}\(|X_{s}^{\e,\d}-X_{k\triangle}^{\e,\d}|^{2}+\mW_2^2(\sL_{X_{s}^{\e,\d}},\sL_{X_{k\triangle}^{\e,\d}})\)\dif s\\
&\leq&\frac{C}{\d}\left(\sup\limits_{s\in[0,T]}\mE\sup _{s \leqslant r \leqslant s+\triangle}|X_{r}^{\e,\d}-X_{s}^{\e,\d}|^{2}\right)\frac{e^{\l t}-e^{\l k\triangle}}{\l},
\de
where the fact $\mW_2^2(\sL_{X_{s}^{\e,\d}},\sL_{X_{k\triangle}^{\e,\d}})\leq \mE|X_{s}^{\e,\d}-X_{k\triangle}^{\e,\d}|^{2}$ is applied in the last inequality.

Finally, it follows that
\ce
\mE|Y_{t}^{\e,\d}-\hat{Y}_{t}^{\e,\d}|^{2}\leq\frac{C}{\a}\left(\sup\limits_{s\in[0,T]}\mE\sup _{s \leqslant r \leqslant s+\triangle}|X_{r}^{\e,\d}-X_{s}^{\e,\d}|^{2}\right).
\de
The proof is complete.
\end{proof}

\subsubsection{Some estimates for the average equation (\ref{ldppsio0equ})}\label{froequ}

\bl \label{averc}
Under the assumptions of Theorem \ref{xbarxp}, Eq.(\ref{ldppsio0equ}) has a unique solution $(\bar{X}^0_{\cdot},\bar{K}^0_{\cdot})$. Moreover, it holds that
\be
&&\mE\sup\limits_{t\in[0,T]}|\bar{X}^0_{t}|^{2}\leq C(1+\mE|\xi|^{2}), \label{barx0b}\\
&&\lim\limits_{l\rightarrow 0}\sup\limits_{s\in[0,T]}\mE\sup\limits_{s\leq t\leq s+l}|\bar{X}^0_{t}-\bar{X}^0_{s}|^2=0. \label{barx0ts}
\ee
\el
\begin{proof}
First of all, by the similar deduction to that of \cite[Lemma 3.8]{rsx}, we know that for any $x_1, x_2\in\overline{\cD(A_1)}$ and $\mu_1,\mu_2\in\cP_2(\overline{\cD(A_1)})$
\be
|\bar{b}_1(x_1,\mu_1)-\bar{b}_1(x_2,\mu_2)|\leq C(|x_1-x_2|+\mW_2(\mu_1,\mu_2)),
\label{barb1lip}
\ee
which implies that Eq.(\ref{ldppsio0equ}) has a unique solution $(\bar{X}^0_{\cdot},\bar{K}^0_{\cdot})$. Then, since the proofs of the required estimates are similar to that for $X^{\e,\d}$ in Lemma \ref{xgygb} and Lemma \ref{xehatxets}, we omit them. The proof is complete.
\end{proof}

At present, we are ready to prove Theorem \ref{xbarxp}.

{\bf Proof of Theorem \ref{xbarxp}.} Note that
\ce
X_{t}^{\e,\d}-\bar{X}^0_{t}
&=&-K_{t}^{1,\e,\d}+\bar{K}^0_{t}+\int_{0}^{t}\(b_{1}(X_{s}^{\e,\d},\sL_{X_{s}^{\e,\d}},Y_{s}^{\e,\d})-\bar{b}_{1}(\bar{X}^0_{s},D_{\bar{X}^0_{s}})\)\dif s\\
&&+\e^\t\int_{0}^{t}\s_{1}(X_{s}^{\e,\d},\sL_{X_{s}^{\e,\d}},Y_{s}^{\e,\d})\dif W^1_{s}.
\de
Thus, by the It\^o formula, it holds that
\ce
|X_{t}^{\e,\d}-\bar{X}^0_{t}|^2&=&-2\int_0^t\<X_{s}^{\e,\d}-\bar{X}^0_{s},\dif(K_{s}^{1,\e,\d}-\bar{K}^0_{s})\>\\
&&+2\int_0^t\<X_{s}^{\e,\d}-\bar{X}_{s}^0,b_{1}(X_{s}^{\e,\d},\sL_{X_{s}^{\e,\d}},Y_s^{\e,\d})-\bar{b}_{1}(\bar{X}^0_{s},\sL_{\bar{X}^0_{s}})\>\dif s\\
&&+2\e^\t\int_0^t\<X_{s}^{\e,\d}-\bar{X}_{s}^0,\s_{1}(X_{s}^{\e,\d},\sL_{X_{s}^{\e,\d}},Y_{s}^{\e,\d})\dif W^1_{s}\>\\
&&+\e^{2\t}\int_{0}^{t}\|\s_{1}(X_{s}^{\e,\d},\sL_{X_{s}^{\e,\d}},Y_{s}^{\e,\d})\|^2\dif s\\
&\leq&2\int_0^t\<X_{s}^{\e,\d}-\bar{X}_{s}^0,b_{1}(X_{s}^{\e,\d},\sL_{X_{s}^{\e,\d}},Y_s^{\e,\d})-\bar{b}_{1}(\bar{X}_{s}^0,\sL_{\bar{X}_{s}^0})\>\dif s\\
 &&+2\e^\t\int_0^t\<X_{s}^{\e,\d}-\bar{X}_{s}^0,\s_{1}(X_{s}^{\e,\d},\sL_{X_{s}^{\e,\d}},Y_{s}^{\e,\d})\dif W^1_{s}\>\\
 &&+\e^{2\t}\int_{0}^{t}\|\s_{1}(X_{s}^{\e,\d},\sL_{X_{s}^{\e,\d}},Y_{s}^{\e,\d})\|^2\dif s.
\de
Moreover, based on the BDG inequality, we get that
\be
&&\mE\(\sup_{0\leq t\leq T}|X_{t}^{\e,\d}-\bar{X}_{t}^0|^{2}\)\no\\
&\leq&2\mE\sup_{0\leq t\leq T}\left|\int_0^t\<X_{s}^{\e,\d}-\bar{X}_{s}^0,b_{1}(X_{s}^{\e,\d},\sL_{X_{s}^{\e,\d}},Y_s^{\e,\d})-\bar{b}_{1}(\bar{X}_{s}^0,\sL_{\bar{X}_{s}^0})\>\dif s\right|\no\\
&&+2\e^\t\mE\sup_{0\leq t\leq T}\left|\int_0^t\<X_{s}^{\e,\d}-\bar{X}_{s}^0,\s_{1}(X_{s}^{\e,\d},\sL_{X_{s}^{\e,\d}},Y_{s}^{\e,\d})\dif W^1_{s}\>\right|\no\\
&&+\e^{2\t}\mE\sup_{0\leq t\leq T}\int_{0}^{t}\|\s_{1}(X_{s}^{\e,\d},\sL_{X_{s}^{\e,\d}},Y_{s}^{\e,\d})\|^2\dif s\no\\
&\leq&2\mE\sup_{0\leq t\leq T}\left|\int_0^t\<X_{s}^{\e,\d}-\bar{X}_{s}^0,b_{1}(X_{s}^{\e,\d},\sL_{X_{s}^{\e,\d}},Y_s^{\e,\d})-\bar{b}_{1}(\bar{X}_{s}^0,\sL_{\bar{X}_{s}^0})\>\dif s\right|\no\\
&&+2\e^\t C\mE\left(\int_0^T|X_{s}^{\e,\d}-\bar{X}_{s}^0|^2\|\s_{1}(X_{s}^{\e,\d},\sL_{X_{s}^{\e,\d}},Y_{s}^{\e,\d})\|^2\dif s\right)^{1/2}\no\\
&&+\e^{2\t}\mE\sup_{0\leq t\leq T}\int_{0}^{t}\|\s_{1}(X_{s}^{\e,\d},\sL_{X_{s}^{\e,\d}},Y_{s}^{\e,\d})\|^2\dif s\no\\
&\leq&2\mE\sup_{0\leq t\leq T}\left|\int_0^t\<X_{s}^{\e,\d}-\bar{X}_{s}^0,b_{1}(X_{s}^{\e,\d},\sL_{X_{s}^{\e,\d}},Y_s^{\e,\d})-\bar{b}_{1}(\bar{X}_{s}^0,\sL_{\bar{X}_{s}^0})\>\dif s\right|\no\\
&&+\frac{1}{2}\mE\(\sup_{0\leq t\leq T}|X_{t}^{\e,\d}-\bar{X}_{t}^0|^{2}\)+C(\e^\t+\e^{2\t})\int_0^T\mE\(1+|X_{s}^{\e,\d}|^{2}+|Y_{s}^{\e,\d}|^{2}\)\dif s.
\label{hatxbarx}
\ee

Next, set
$$
I:=\mE\sup_{0\leq t\leq T}\left|\int_0^t\<X_{s}^{\e,\d}-\bar{X}_{s}^0,b_{1}(X_{s}^{\e,\d},\sL_{X_{s}^{\e,\d}},Y_s^{\e,\d})-\bar{b}_{1}(\bar{X}_{s}^0,\sL_{\bar{X}_{s}^0})\>\dif s\right|,
$$
and we investigate $I$. Note that
\be
I&\leq&\mE\sup_{0\leq t\leq T}\left|\int_0^t\<X_{s}^{\e,\d}-X_{s(\triangle)}^{\e,\d}-\bar{X}_{s}^0+\bar{X}^0_{s(\triangle)},b_{1}(X_{s}^{\e,\d},\sL_{X_{s}^{\e,\d}},Y_{s}^{\e,\d})-\bar{b}_{1}(\bar{X}_{s}^0,\sL_{\bar{X}_{s}^0})\>\dif s\right|\no\\
&&+\mE\sup_{0\leq t\leq T}\left|\int_0^t\<X_{s(\triangle)}^{\e,\d}-\bar{X}_{s(\triangle)}^0,b_{1}(X_{s}^{\e,\d},\sL_{X_{s}^{\e,\d}},Y_{s}^{\e,\d})-b_{1}(X_{s(\triangle)}^{\e,\d},\sL_{X_{s(\triangle)}^{\e,\d}},\hat{Y}_{s}^{\e,\d})\>\dif s\right|\no\\
&&+\mE\sup_{0\leq t\leq T}\left|\int_0^t\<X_{s(\triangle)}^{\e,\d}-\bar{X}_{s(\triangle)}^0,b_{1}(X_{s(\triangle)}^{\e,\d},\sL_{X_{s(\triangle)}^{\e,\d}},\hat{Y}_{s}^{\e,\d})-\bar{b}_{1}(X_{s(\triangle)}^{\e,\d},\sL_{X_{s(\triangle)}^{\e,\d}})\>\dif s\right|\no\\
&&+\mE\sup_{0\leq t\leq T}\left|\int_0^t\<X_{s(\triangle)}^{\e,\d}-\bar{X}_{s(\triangle)}^0,\bar{b}_{1}(X_{s(\triangle)}^{\e,\d},\sL_{X_{s(\triangle)}^{\e,\d}})-\bar{b}_{1}(\bar{X}_{s(\triangle)}^0,\sL_{\bar{X}_{s(\triangle)}^0})\>\dif s\right|\no\\
&&+\mE\sup_{0\leq t\leq T}\left|\int_0^t\<X_{s(\triangle)}^{\e,\d}-\bar{X}_{s(\triangle)}^0,\bar{b}_{1}(\bar{X}_{s(\triangle)}^0,\sL_{\bar{X}_{s(\triangle)}^0})-\bar{b}_{1}(\bar{X}_{s}^0,\sL_{\bar{X}_{s}^0})\>\dif s\right|\no\\
&=:&I_{1}+I_{2}+I_{3}+I_{4}+I_{5},
\label{i12345}
\ee
where $s(\triangle):=[\frac{s}{\triangle}]\triangle$. Thus, we estimate $I_{1}, I_{2}, I_{3}, I_{4}, I_{5}$, respectively.

For $I_{1}$, by the H\"older inequality and the linear growth for $b_1, \bar{b}_{1}$, it holds that
\be
I_{1}&\leq& C\left(\int_0^T\mE(|X_{s}^{\e,\d}-X_{s(\triangle)}^{\e,\d}|^2+|\bar{X}_{s}^0-\bar{X}_{s(\triangle)}^0|^2)\dif s\right)^{1/2}\no\\
&&\times\left(\int_0^T\mE(1+|X_{s}^{\e,\d}|^2+|Y_{s}^{\e,\d}|^2+|\bar{X}_{s}^0|^2)\dif s\right)^{1/2}\no\\
&\leq& C\left(\sup\limits_{s\in[0,T]}\mE\sup _{s \leqslant r \leqslant s+\triangle}|X_{r}^{\e,\d}-X_{s}^{\e,\d}|^{2}+\sup\limits_{s\in[0,T]}\mE\sup\limits_{s\leq r\leq s+\triangle}|\bar{X}_{r}^0-\bar{X}_{s}^0|^2\right)^{1/2}.
\label{i1}
\ee

For $I_{2}$, the  Lipschitz continuity of $b_{1}$ yields that
\be
I_2&\leq& \int_0^T\mE|X_{s(\triangle)}^{\e,\d}-\bar{X}_{s(\triangle)}^0|^2\dif s+C\int_0^T(\mE|X_{s}^{\e,\d}-X_{s(\triangle)}^{\e,\d}|^2+\mE|Y_{s}^{\e,\d}-\hat{Y}_{s}^{\e,\d}|^2)\dif s\no\\
&\leq&\int_0^T\mE\(\sup_{0\leq r\leq s}|X_{r}^{\e,\d}-\bar{X}_{r}|^{2}\)\dif s+CT\sup\limits_{s\in[0,T]}\mE\sup _{s \leqslant r \leqslant s+\triangle}|X_{r}^{\e,\d}-X_{s}^{\e,\d}|^{2}.
\label{i2}
\ee

For $I_{3}$, by the deduction in {\bf Step 2.}, we know that 
\be
I_{3}\leq C\((\frac{\d}{\triangle})^{1/2}+\triangle^{1/2}\).
\label{i3}
\ee

For $I_{4}$, the  Lipschitz continuity of $\bar{b}_{1}$ implies that
\be
I_{4}&\leq&\mE\int_0^T|X_{s(\triangle)}^{\e,\d}-\bar{X}_{s(\triangle)}^0|^2\dif s+C\mE\int_0^T(|X_{s(\triangle)}^{\e,\d}-\bar{X}_{s(\triangle)}^0|^2+\mW^2_2(\sL_{X_{s(\triangle)}^{\e,\d}},\sL_{\bar{X}_{s(\triangle)}^0}))\dif s\no\\
&\leq&C\int_0^T\mE|X_{s(\triangle)}^{\e,\d}-\bar{X}_{s(\triangle)}^0|^2\dif s\leq C\int_0^T\mE\(\sup_{0\leq r\leq s}|X_{r}^{\e,\d}-\bar{X}_{r}^0|^{2}\)\dif s.
\label{i4}
\ee

For $I_{5}$, by the  Lipschitz continuity of $\bar{b}_{1}$ and (\ref{barx0ts}), it holds that
\be
I_{5}&\leq&\int_0^T\mE|X_{s(\triangle)}^{\e,\d}-\bar{X}_{s(\triangle)}^0|^2\dif s+C\int_0^T|\bar{X}_{s}^0-\bar{X}_{s(\triangle)}^0|^2\dif s\no\\
&\leq&\int_0^T\mE\(\sup_{0\leq r\leq s}|X_{r}^{\e,\d}-\bar{X}_{r}^0|^{2}\)\dif s+CT\sup\limits_{s\in[0,T]}\mE\sup\limits_{s\leq t\leq s+\triangle}|\bar{X}_{t}^0-\bar{X}_{s}^0|^2.
\label{i5}
\ee

Combining (\ref{i1})-(\ref{i5}) with (\ref{i12345}), we have that
\be
I\leq C\int_0^T\mE\(\sup_{0\leq r\leq s}|X_{r}^{\e,\d}-\bar{X}_{r}^0|^{2}\)\dif s+C\Gamma(\d)+C\((\frac{\d}{\triangle})^{1/2}+\triangle^{1/2}\),
\label{i}
\ee
where 
\ce
\Gamma(\d)&:=&\left(\sup\limits_{s\in[0,T]}\mE\sup _{s \leqslant r \leqslant s+\triangle}|X_{r}^{\e,\d}-X_{s}^{\e,\d}|^{2}+\sup\limits_{s\in[0,T]}\mE\sup\limits_{s\leq t\leq s+\triangle}|\bar{X}_{t}^0-\bar{X}_{s}^0|^2\right)^{1/2}\\
&&+\left(\sup\limits_{s\in[0,T]}\mE\sup _{s \leqslant r \leqslant s+\triangle}|X_{r}^{\e,\d}-X_{s}^{\e,\d}|^{2}\right)+\sup\limits_{s\in[0,T]}\mE\sup\limits_{s\leq t\leq s+\triangle}|\bar{X}_{t}^0-\bar{X}_{s}^0|^2.
\de

Finally, by (\ref{hatxbarx}), (\ref{i}) and the Gronwall inequality, we obtain that
\ce
\mE\(\sup_{0\leq t\leq T}|X_{t}^{\e,\d}-\bar{X}_{t}^0|^{2}\)\leq C\(\Gamma(\d)+(\frac{\d}{\triangle})^{1/2}+\triangle^{1/2}+\e^\t+\e^{2\t}\).
\de
Since $\lim\limits_{\e\rightarrow 0}\d/\e=\iota\in[0,\infty)$, $\d\rightarrow 0$ as $\e$ tends to $0$. Then we take $\triangle=\d^{\g}, 0<\g<1$, and as $\d\rightarrow 0$, $\frac{\d}{\triangle}\rightarrow 0, \triangle\rightarrow 0$ and (\ref{xegts}), (\ref{barx0ts}) imply that 
\ce
\lim\limits_{\e\rightarrow 0}\mE\(\sup_{0\leq t\leq T}|X_{t}^{\e,\d}-\bar{X}_{t}^0|^{2}\)=0.
\de

{\bf Step 2.} We prove (\ref{i3}).

For $I_3$, it holds that
\ce
I_{3}&=&\Bigg(\mE\sup_{0\leq t\leq T}
 \Big|\int_{0}^{[\frac{t}{\triangle}]\triangle}\<X_{s(\triangle)}^{\e,\d}-\bar{X}_{s(\triangle)}^0,b_{1}(X_{s(\triangle)}^{\e,\d},\sL_{X_{s(\triangle)}^{\e,\d}},\hat{Y}_{s}^{\e,\d})
 -\bar{b}_{1}(X_{s(\triangle)}^{\e,\d},\sL_{X_{s(\triangle)}^{\e,\d}})\>\dif s\no\\
 &&\quad\quad\quad\quad+\int_{[\frac{t}{\triangle}]\triangle}^{t}\<X_{s(\triangle)}^{\e,\d}-\bar{X}_{s(\triangle)}^0,b_{1}(X_{s(\triangle)}^{\e,\d},\sL_{X_{s(\triangle)}^{\e,\d}},\hat{Y}_{s}^{\e,\d})
 -\bar{b}_{1}(X_{s(\triangle)}^{\e,\d},\sL_{X_{s(\triangle)}^{\e,\d}})\>\dif s\Big|\Bigg)\no\\
&\leq&\Bigg(\mE\sup_{0\leq t\leq T}
\Big|\int_{0}^{[\frac{t}{\triangle}]\triangle}\<X_{s(\triangle)}^{\e,\d}-\bar{X}_{s(\triangle)}^0,b_{1}(X_{s(\triangle)}^{\e,\d},\sL_{X_{s(\triangle)}^{\e,\d}},\hat{Y}_{s}^{\e,\d})
 -\bar{b}_{1}(X_{s(\triangle)}^{\e,\d},\sL_{X_{s(\triangle)}^{\e,\d}})\>\dif s\Big|\Bigg)\no\\
&&+\Bigg(\mE\sup_{0\leq t\leq T}
\Big|\int_{[\frac{t}{\triangle}]\triangle}^{t}\<X_{s(\triangle)}^{\e,\d}-\bar{X}_{s(\triangle)}^0,b_{1}(X_{s(\triangle)}^{\e,\d},\sL_{X_{s(\triangle)}^{\e,\d}},\hat{Y}_{s}^{\e,\d})
 -\bar{b}_{1}(X_{s(\triangle)}^{\e,\d},\sL_{X_{s(\triangle)}^{\e,\d}})\>\dif s\Big|\Bigg)\no\\
 &=:&I_{31}+I_{32}.
\de

Next, we estimate $I_{31}$. Note that
\be
I_{31}
&=&\Bigg(\mE\sup_{0\leq t\leq T}
\Big|\int_{0}^{[\frac{t}{\triangle}]\triangle}\<X_{s(\triangle)}^{\e,\d}-\bar{X}_{s(\triangle)}^0,b_{1}(X_{s(\triangle)}^{\e,\d},\sL_{X_{s(\triangle)}^{\e,\d}},\hat{Y}_{s}^{\e,\d})
 -\bar{b}_{1}(X_{s(\triangle)}^{\e,\d},\sL_{X_{s(\triangle)}^{\e,\d}})\>\dif s\Big|\Bigg)\no\\
&=&\mE\Bigg(\sup_{0\leq t\leq T}
\Big|\sum\limits_{k=0}^{[\frac{t}{\triangle}]-1}\int_{k\triangle}^{(k+1)\triangle}\<X_{s(\triangle)}^{\e,\d}-\bar{X}_{s(\triangle)}^0,b_{1}(X_{s(\triangle)}^{\e,\d},\sL_{X_{s(\triangle)}^{\e,\d}},\hat{Y}_{s}^{\e,\d})
 -\bar{b}_{1}(X_{s(\triangle)}^{\e,\d},\sL_{X_{s(\triangle)}^{\e,\d}})\>\dif s\Big|\Bigg)\no\\
&\leq&\mE\Bigg(\sup_{0\leq t\leq T}\sum\limits_{k=0}^{[\frac{t}{\triangle}]-1}\left|\int_{k\triangle}^{(k+1)\triangle}\<X_{k\triangle}^{\e,\d}-\bar{X}_{k\triangle}^0,b_{1}(X_{k\triangle}^{\e,\d},\sL_{X_{k\triangle}^{\e,\d}},\hat{Y}_{s}^{\e,\d})
 -\bar{b}_{1}(X_{k\triangle}^{\e,\d},\sL_{X_{k\triangle}^{\e,\d}})\>\dif s\right|\no\\
&\leq&\sum\limits_{k=0}^{[\frac{T}{\triangle}]-1}
\mE\Bigg(\Big|\int_{k\triangle}^{(k+1)\triangle}\<X_{k\triangle}^{\e,\d}-\bar{X}_{k\triangle}^0,b_{1}(X_{k\triangle}^{\e,\d},\sL_{X_{k\triangle}^{\e,\d}},\hat{Y}_{s}^{\e,\d})
 -\bar{b}_{1}(X_{k\triangle}^{\e,\d},\sL_{X_{k\triangle}^{\e,\d}})\>\dif s\Big|\Bigg)\no\\
&\leq&[\frac{T}{\triangle}]\sup_{0\leq k\leq [\frac{T}{\triangle}]-1}
\mE\Bigg(\Big|\int_{k\triangle}^{(k+1)\triangle}\<X_{k\triangle}^{\e,\d}-\bar{X}_{k\triangle}^0,b_{1}(X_{k\triangle}^{\e,\d},\sL_{X_{k\triangle}^{\e,\d}},\hat{Y}_{s}^{\e,\d})
 -\bar{b}_{1}(X_{k\triangle}^{\e,\d},\sL_{X_{k\triangle}^{\e,\d}})\>\dif s\Big|\Bigg)\no\\
&\leq&\d(\frac{T}{\triangle})\sup_{0\leq k\leq [\frac{T}{\triangle}]-1}
\mE\Bigg(\Big|\<X_{k\triangle}^{\e,\d}-\bar{X}_{k\triangle}^0,\int_{0}^{\triangle/\d}(b_{1}(X_{k\triangle}^{\e,\d},\sL_{X_{k\triangle}^{\e,\d}},\hat{Y}_{\d s+k\triangle}^{\e,\d})
 -\bar{b}_{1}(X_{k\triangle}^{\e,\d},\sL_{X_{k\triangle}^{\e,\d}}))\dif s\>\Big|\Bigg)\no\\
&\leq&\d(\frac{T}{\triangle})\sup_{0\leq k\leq [\frac{T}{\triangle}]-1}
\mE|X_{k\triangle}^{\e,\d}-\bar{X}_{k\triangle}^0|\left|\int_{0}^{\triangle/\d}(b_{1}(X_{k\triangle}^{\e,\d},\sL_{X_{k\triangle}^{\e,\d}},\hat{Y}_{\d s+k\triangle}^{\e,\d})
 -\bar{b}_{1}(X_{k\triangle}^{\e,\d},\sL_{X_{k\triangle}^{\e,\d}}))\dif s\right|.
\label{b1barb1}
\ee

In the following, in order to estimate the right side of the above inequality, we construct the following equation: for any $s>0$
\ce\left\{\begin{array}{l}
\dif \check{Y}_t^{\d, s, \varpi,\mu,\zeta}\in -A_2(\check{Y}_t^{\d, s, \varpi,\mu,\zeta})\dif t+\frac{1}{\d}b_2(\varpi,\mu,\check{Y}_t^{\d, s, \varpi,\mu,\zeta})\dif t+\frac{1}{\sqrt{\d}}\s_2(\varpi,\mu,\check{Y}_t^{\d, s, \varpi,\mu,\zeta})\dif W^2_t, ~t>s,\\
\check{Y}_s^{\d, s, \varpi,\mu,\zeta}=\zeta,
\end{array}
\right.
\de
where $\varpi\in L^2(\Omega,\mathscr{F}_s,\mP; \overline{\cD(A_1)}), \mu\in\cP_2(\overline{\cD(A_1)}), \zeta\in L^2(\Omega,\mathscr{F}_s,\mP; \overline{\cD(A_2)})$. Then it holds that
$$
\hat{Y}_t^{\e,\d}=\check{Y}_t^{\d, k \triangle, X_{k\triangle}^{\e,\d},\sL_{X_{k\triangle}^{\e,\d}}, \hat{Y}_{k \triangle}^{\e,\d}}, \quad \hat{K}_t^{2,\e,\d}=\check{K}_t^{2,\d, k \triangle, X_{k \triangle}^{\e,\d},\sL_{X_{k\triangle}^{\e,\d}}, \hat{Y}_{k\triangle}^{\e,\d}}, \quad t \in[k\triangle,(k+1)\triangle).
$$
Note that $X_{k\triangle}^{\e,\d}, \hat{Y}_{k\triangle}^{\e,\d}$ are $\sF_{k\triangle}$-measurable, and for any $x\in\overline{\cD(A_1)}, y\in\overline{\cD(A_2)}$, $\check{Y}_t^{\d,k\triangle, x,\sL_{X_{k\triangle}^{\e,\d}},y}$ is independent of $\sF_{k\triangle}$. Thus, we have that for $0\leq k\leq [\frac{T}{\triangle}]-1$
\ce
&&\mE|X_{k\triangle}^{\e,\d}-\bar{X}_{k\triangle}^0|\left|\int_{0}^{\triangle/\d}(b_{1}(X_{k\triangle}^{\e,\d},\sL_{X_{k\triangle}^{\e,\d}},\check{Y}^{\d, k \triangle, X_{k\triangle}^{\e,\d},\sL_{X_{k\triangle}^{\e,\d}}, \hat{Y}_{k \triangle}^{\e,\d}}_{\d s+k\triangle})
 -\bar{b}_{1}(X_{k\triangle}^{\e,\d},\sL_{X_{k\triangle}^{\e,\d}}))\dif s\right|\\
 &=&\mE\Bigg[\mE\Bigg[|X_{k\triangle}^{\e,\d}-\bar{X}_{k\triangle}^0|\left|\int_{0}^{\triangle/\d}(b_{1}(X_{k\triangle}^{\e,\d},\sL_{X_{k\triangle}^{\e,\d}},\check{Y}^{\d, k \triangle, X_{k\triangle}^{\e,\d},\sL_{X_{k\triangle}^{\e,\d}}, \hat{Y}_{k \triangle}^{\e,\d}}_{\d s+k\triangle})
 -\bar{b}_{1}(X_{k\triangle}^{\e,\d},\sL_{X_{k\triangle}^{\e,\d}}))\dif s\right|\Bigg{|}\sF_{k\triangle}\Bigg]\Bigg]\\
 &=&\mE\Bigg[|X_{k\triangle}^{\e,\d}-\bar{X}_{k\triangle}^0|\mE\Bigg[\left|\int_{0}^{\triangle/\d}(b_{1}(x,\sL_{X_{k\triangle}^{\e,\d}},\check{Y}_{\d s+k\triangle}^{\d,k\triangle, x,\sL_{X_{k\triangle}^{\e,\d}},y})-\bar{b}_{1}(x,\sL_{X_{k\triangle}^{\e,\d}}))\dif s\right|\Bigg]\Bigg{|}_{(x,y)=(X_{k\triangle}^{\e,\d},\hat{Y}_{k\triangle}^{\e,\d})}\Bigg].
  \de
 
 Here, we investigate $\check{Y}^{\d, k\triangle, x,\sL_{X_{k\triangle}^{\e,\d}},y}_{\d s+k\triangle}$. On one hand, it holds that
\ce
\check{Y}^{\d, k \triangle, x,\sL_{X_{k\triangle}^{\e,\d}},y}_{\d s+k\triangle}&=&y-\check{K}_{\d s+k\triangle}^{2,\d, k\triangle, x,\sL_{X_{k\triangle}^{\e,\d}},y}+\check{K}_{k\triangle}^{2,\d, k\triangle, x,\sL_{X_{k\triangle}^{\e,\d}},y}\\
&&+\frac{1}{\d} \int_{k\triangle}^{\d s+k\triangle}b_2(x,\sL_{X_{k\triangle}^{\e,\d}},\check{Y}^{\d, k \triangle, x,\sL_{X_{k\triangle}^{\e,\d}},y}_{r})\dif r\\
&&+\frac{1}{\sqrt{\d}} \int_{k\triangle}^{\d s+k\triangle} \s_2(x,\sL_{X_{k\triangle}^{\e,\d}},\check{Y}^{\d, k\triangle, x,\sL_{X_{k\triangle}^{\e,\d}},y}_{r})\dif W^2_r\\
&=&y-\check{K}_{\d s+k\triangle}^{2,\d, k\triangle, x,\sL_{X_{k\triangle}^{\e,\d}},y}+\check{K}_{k\triangle}^{2,\d, k\triangle, x,\sL_{X_{k\triangle}^{\e,\d}},y}\\
&&+\frac{1}{\d} \int_{0}^{\d s}b_2(x,\sL_{X_{k\triangle}^{\e,\d}},\check{Y}^{\d, k\triangle, x,\sL_{X_{k\triangle}^{\e,\d}},y}_{u+k\triangle})\dif u\\
&&+\frac{1}{\sqrt{\d}} \int_{0}^{\d s} \s_2(x,\sL_{X_{k\triangle}^{\e,\d}},\check{Y}^{\d, k\triangle, x,\sL_{X_{k\triangle}^{\e,\d}},y}_{u+k\triangle})\dif \tilde{W}^2_u\\
&=&y-\check{\tilde{\check{K}}}_{s}^{2,\d, k\triangle, x,\sL_{X_{k\triangle}^{\e,\d}},y}+\int_{0}^{s}b_2(x,\sL_{X_{k\triangle}^{\e,\d}},\check{Y}^{\d, k\triangle, x,\sL_{X_{k\triangle}^{\e,\d}},y}_{\d v+k\triangle})\dif v\\
&&+\int_{0}^{s} \s_2(x,\sL_{X_{k\triangle}^{\e,\d}},\check{Y}^{\d, k\triangle, x,\sL_{X_{k\triangle}^{\e,\d}},y}_{\d v+k\triangle})\dif \check{\tilde{W}}^2_v,
\de
where $\tilde{W}^2_u:=W^2_{u+k\triangle}-W^2_{k\triangle}$ and $\check{\tilde{W}}^2_v:=\frac{1}{\sqrt{\d}}\tilde{W}^2_{\d v}$ are two $m$-dimensional standard Brownian motions, and $\check{\tilde{\check{K}}}_{s}^{2,\d, k\triangle, x,\sL_{X_{k\triangle}^{\e,\d}},y}:=\check{K}_{\d s+k\triangle}^{2,\d, k\triangle, x,\sL_{X_{k\triangle}^{\e,\d}},y}-\check{K}_{k\triangle}^{2,\d, k\triangle, x,\sL_{X_{k\triangle}^{\e,\d}},y}$. On the other hand, note that the frozen equation (\ref{Eq2}) is written as
\ce
Y_{s}^{x,\sL_{X_{k\triangle}^{\e,\d}},y}=y-K^{2,x,\sL_{X_{k\triangle}^{\e,\d}},y}_s+\int_0^s b_{2}(x,\sL_{X_{k\triangle}^{\e,\d}},Y_{r}^{x,\sL_{X_{k\triangle}^{\e,\d}},y})\dif r+\int_0^s\s_{2}(x,\sL_{X_{k\triangle}^{\e,\d}},Y_{r}^{x,\sL_{X_{k\triangle}^{\e,\d}},y})\dif W^2_{r}.
\de
Thus, for $s\in[0,\triangle/\d]$, $\check{Y}^{\d, k\triangle,x,\sL_{X_{k\triangle}^{\e,\d}},y}_{\d s+k\triangle}$ and $Y_{s}^{x,\sL_{X_{k\triangle}^{\e,\d}},y}$ have the same distribution, which implies that
\be
&&\mE|X_{k\triangle}^{\e,\d}-\bar{X}_{k\triangle}^0|\left|\int_{0}^{\triangle/\d}(b_{1}(X_{k\triangle}^{\e,\d},\sL_{X_{k\triangle}^{\e,\d}},\check{Y}^{\d, k \triangle, X_{k\triangle}^{\e,\d},\sL_{X_{k\triangle}^{\e,\d}}, \hat{Y}_{k \triangle}^{\e,\d}}_{\d s+k\triangle})
 -\bar{b}_{1}(X_{k\triangle}^{\e,\d},\sL_{X_{k\triangle}^{\e,\d}}))\dif s\right|\no\\
&=&\mE\Bigg[|X_{k\triangle}^{\e,\d}-\bar{X}_{k\triangle}^0|\mE\Bigg[\left|\int_{0}^{\triangle/\d}(b_{1}(x,\sL_{X_{k\triangle}^{\e,\d}},Y_{s}^{x,\sL_{X_{k\triangle}^{\e,\d}},y})-\bar{b}_{1}(x,\sL_{X_{k\triangle}^{\e,\d}}))\dif s\right|\Bigg]\Bigg{|}_{(x,y)=(X_{k\triangle}^{\e,\d},\hat{Y}_{k\triangle}^{\e,\d})}\Bigg]\no\\
&\leq&\left(\mE\Bigg[\mE\Bigg[\left|\int_{0}^{\triangle/\d}(b_{1}(x,\sL_{X_{k\triangle}^{\e,\d}},Y_{s}^{x,\sL_{X_{k\triangle}^{\e,\d}},y})-\bar{b}_{1}(x,\sL_{X_{k\triangle}^{\e,\d}}))\dif t\right|^2\Bigg]\Bigg{|}_{(x,y)=(X_{k\triangle}^{\e,\d},\hat{Y}_{k\triangle}^{\e,\d})}\Bigg]\right)^{1/2}\no\\
 &&\times \(\mE|X_{k\triangle}^{\e,\d}-\bar{X}_{k\triangle}^0|^2\)^{1/2}.
 \label{xkvbarx}
\ee

Note that 
\ce
&&\mE\left|\int_{0}^{\triangle/\d}(b_{1}(x,\sL_{X_{k\triangle}^{\e,\d}},Y_{s}^{x,\sL_{X_{k\triangle}^{\e,\d}},y})-\bar{b}_{1}(x,\sL_{X_{k\triangle}^{\e,\d}}))\dif t\right|^2\\
&=&2\mE\int_{0}^{\triangle/\d}\int_{r}^{\triangle/\d}\<b_{1}(x,\sL_{X_{k\triangle}^{\e,\d}},Y_{s}^{x,\sL_{X_{k\triangle}^{\e,\d}},y})-\bar{b}_{1}(x,\sL_{X_{k\triangle}^{\e,\d}}),\\
&&b_{1}(x,\sL_{X_{k\triangle}^{\e,\d}},Y_{r}^{x,\sL_{X_{k\triangle}^{\e,\d}},y})-\bar{b}_{1}(x,\sL_{X_{k\triangle}^{\e,\d}})\>\dif s\dif r\\
&=&2\int_{0}^{\triangle/\d}\int_{r}^{\triangle/\d}\mE\<\mE[b_{1}(x,\sL_{X_{k\triangle}^{\e,\d}},Y_{s}^{x,\sL_{X_{k\triangle}^{\e,\d}},y})|\sF_r^{W^2}]-\bar{b}_{1}(x,\sL_{X_{k\triangle}^{\e,\d}}),\\
&&b_{1}(x,\sL_{X_{k\triangle}^{\e,\d}},Y_{r}^{x,\sL_{X_{k\triangle}^{\e,\d}},y})-\bar{b}_{1}(x,\sL_{X_{k\triangle}^{\e,\d}})\>\dif s\dif r\\
&\leq&2\int_{0}^{\triangle/\d}\int_{r}^{\triangle/\d}\left(\mE\left|\mE[b_{1}(x,\sL_{X_{k\triangle}^{\e,\d}},Y_{s}^{x,\sL_{X_{k\triangle}^{\e,\d}},y})|\sF_r^{W^2}]-\bar{b}_{1}(x,\sL_{X_{k\triangle}^{\e,\d}})\right|^2\right)^{1/2}\\
&&\left(\mE|b_{1}(x,\sL_{X_{k\triangle}^{\e,\d}},Y_{r}^{x,\sL_{X_{k\triangle}^{\e,\d}},y})-\bar{b}_{1}(x,\sL_{X_{k\triangle}^{\e,\d}})|^2\right)^{1/2}\dif s\dif r.
\de
where $\mathscr{F}_{r}^{W^2}=\sigma\{W^2_{u},0\leq u \leq r\}\vee \cN$, and $\cN$ denotes  the collection of all $\mP$-zero sets. Moreover, based on (\ref{meu2}), we obtain that
\ce
&&\left(\mE\left|\mE[b_{1}(x,\sL_{X_{k\triangle}^{\e,\d}},Y_{s}^{x,\sL_{X_{k\triangle}^{\e,\d}},y})|\sF_r^{W^2}]-\bar{b}_{1}(x,\sL_{X_{k\triangle}^{\e,\d}})\right|^2\right)^{1/2}\\
&=&\left(\mE\left|\mE\left[b_{1}(x,\sL_{X_{k\triangle}^{\e,\d}},Y_{s-r}^{x,\sL_{X_{k\triangle}^{\e,\d}},\hat{y}})\right]\bigg{|}_{\hat{y}=Y_{r}^{x,\sL_{X_{k\triangle}^{\e,\d}},y}}-\bar{b}_{1}(x,\sL_{X_{k\triangle}^{\e,\d}})\right|^2\right)^{1/2}\\
&\leq&\(Ce^{-\a(s-r)}(1+|x|^{2}+\|\sL_{X_{k\triangle}^{\e,\d}}\|^2+\mE|Y_{r}^{x,\sL_{X_{k\triangle}^{\e,\d}},y}|^{2})\)^{\frac{1}{2}}\\
&\leq&\(Ce^{-\a(s-r)}(1+|x|^{2}+\|\sL_{X_{k\triangle}^{\e,\d}}\|^2+|y|^{2})\)^{\frac{1}{2}}\\
&\leq& Ce^{-\a(s-r)/2}(1+|x|+\|\sL_{X_{k\triangle}^{\e,\d}}\|+|y|),
\de
and
\ce
&&\left(\mE\left|b_{1}(x,\sL_{X_{k\triangle}^{\e,\d}},Y_{r}^{x,\sL_{X_{k\triangle}^{\e,\d}},y})-\bar{b}_{1}(x,\sL_{X_{k\triangle}^{\e,\d}})\right|^2\right)^{1/2}\\
&=&\left(\mE\left|b_{1}(x,\sL_{X_{k\triangle}^{\e,\d}},Y_{r}^{x,\sL_{X_{k\triangle}^{\e,\d}},y})-\int_{\overline{\cD(A_2)}}b_{1}(x,\sL_{X_{k\triangle}^{\e,\d}},u)\nu^{x,\sL_{X_{k\triangle}^{\e,\d}}}(\dif u)\right|^2\right)^{1/2}\\
&\leq&\left(\mE\int_{\overline{\cD(A_2)}}\left|b_{1}(x,\sL_{X_{k\triangle}^{\e,\d}},Y_{r}^{x,\sL_{X_{k\triangle}^{\e,\d}},y})-b_{1}(x,\sL_{X_{k\triangle}^{\e,\d}},u)\right|^2\nu^{x,\sL_{X_{k\triangle}^{\e,\d}}}(\dif u)\right)^{1/2}\\
&\leq&C\left(\int_{\overline{\cD(A_2)}}\mE\left|Y_{r}^{x,\sL_{X_{k\triangle}^{\e,\d}},y}-u\right|^2\nu^{x,\sL_{X_{k\triangle}^{\e,\d}}}(\dif u)\right)^{1/2}\\
&\leq&C\left(|y|^{2}e^{-\frac{\a}{2} t}+C(1+|x|^{2}+\|\sL_{X_{k\triangle}^{\e,\d}}\|^2)\right)^{1/2}\\
&\leq&C(1+|x|+\|\sL_{X_{k\triangle}^{\e,\d}}\|+|y|).
\de

By combining the above deduction, it holds that
\ce
&&\mE\left|\int_{0}^{\triangle/\d}(b_{1}(x,\sL_{X_{k\triangle}^{\e,\d}},Y_{s}^{x,\sL_{X_{k\triangle}^{\e,\d}},y})-\bar{b}_{1}(x,\sL_{X_{k\triangle}^{\e,\d}}))\dif s\right|^2\\
&\leq&C\int_{0}^{\triangle/\d}\int_{r}^{\triangle/\d}e^{-\a(s-r)/2}(1+|x|+\|\sL_{X_{k\triangle}^{\e,\d}}\|+|y|)^2\dif s\dif r\\
&\leq&C(1+|x|+\|\sL_{X_{k\triangle}^{\e,\d}}\|+|y|)^2\frac{\triangle}{\d},
\de
which together with (\ref{b1barb1}) and (\ref{xkvbarx}) yields that
\be
I_{31}\leq C(\frac{\d}{\triangle})^{1/2}.
\label{b4de1}
\ee

Finally, we estimate $I_{32}$. By (\ref{b1line}) (\ref{barx0b})  and the H\"older inequality, one could get that
\ce
I_{32}
&\leq&\Bigg(\mE\sup_{0\leq t\leq T}
\int_{[\frac{t}{\triangle}]\triangle}^{t}|X_{s(\triangle)}^{\e,\d}-\bar{X}_{s(\triangle)}^0||b_{1}(X_{s(\triangle)}^{\e,\d},\sL_{X_{s(\triangle)}^{\e,\d}},\hat{Y}_{s}^{\e,\d})
 -\bar{b}_{1}(X_{s(\triangle)}^{\e,\d},\sL_{X_{s(\triangle)}^{\e,\d}})|\dif s\Bigg)\\
 &\leq& \triangle^{1/2}\Bigg(\mE\sup_{0\leq t\leq T}\int_{[\frac{t}{\triangle}]\triangle}^{t}|b_{1}(X_{s(\triangle)}^{\e,\d},\sL_{X_{s(\triangle)}^{\e,\d}},\hat{Y}_{s}^{\e,\d})
 -\bar{b}_{1}(X_{s(\triangle)}^{\e,\d},\sL_{X_{s(\triangle)}^{\e,\d}})|^2\dif s\Bigg)^{1/2}\\
 &&\times\Bigg(\mE\sup_{0\leq s\leq T}|X_{s}^{\e,\d}-\bar{X}_{s}^0|^2\Bigg)^{1/2}\\
 &\leq& \triangle^{1/2}\Bigg(\mE\int_{0}^{T}|b_{1}(X_{s(\triangle)}^{\e,\d},\sL_{X_{s(\triangle)}^{\e,\d}},\hat{Y}_{s}^{\e,\d})
 -\bar{b}_{1}(X_{s(\triangle)}^{\e,\d},\sL_{X_{s(\triangle)}^{\e,\d}})|^2\dif s\Bigg)^{1/2}\\
 &&\times\Bigg(\mE\sup_{0\leq s\leq T}|X_{s}^{\e,\d}-\bar{X}_{s}^0|^2\Bigg)^{1/2}\\
 &\leq& C\triangle^{1/2}\Bigg(\int_{0}^{T}(1+\mE|X_{s(\triangle)}^{\e,\d}|^2+\mE|\hat{Y}_{s}^{\e,\d}|^2\dif s\Bigg)^{1/2}\Bigg(\mE\sup_{0\leq s\leq T}|X_{s}^{\e,\d}|^2+\sup_{0\leq s\leq T}|\bar{X}_{s}^0|^2\Bigg)^{1/2}\\
&\leq& C\triangle^{1/2},
\de
which together with (\ref{b4de1}) implies (\ref{i3}). The proof is complete.

\subsection{Proof of Theorem \ref{xbarxpcr}}

In this subsection, we prove Theorem \ref{xbarxpcr}. We begin with a key lemma.

\bl\label{xehatxetscr}
Under the assumptions of Theorem \ref{xbarxpcr}, we have that for $l>0$ small enough 
\be
&&\sup\limits_{s\in[0,T]}\mE\sup _{s \leqslant t \leqslant s+l}|X_{t}^{\e,\d}-X_{s}^{\e,\d}|^{2}\leq Cl, \label{xegtscr}\\
&&\sup\limits_{s\in[0,T]}\mE\sup\limits_{s\leq t\leq s+l}|\bar{X}^0_{t}-\bar{X}^0_{s}|^2\leq Cl, \label{barx0tscr}
\ee
where the constant $C>0$ is independent of $\e, \d, l$.
\el
\begin{proof}
Since the proofs of (\ref{xegtscr}) and (\ref{barx0tscr}) are similar, we only prove (\ref{xegtscr}).

First of all, note that for $0\leq s\leq t\leq s+l\leq T$, 
\ce
X_{t}^{\e,\d}-X_{s}^{\e,\d}=-K_t^{1,\e,\d}+K_s^{1,\e,\d}+\int_{s}^{t}b_{1}(X_{r}^{\e,\d},\sL_{X_r^{\epsilon,\d}},Y_{r}^{\e,\d})\dif r+\e^\t\int_{s}^{t}\s_{1}(X_{r}^{\e,\d},\sL_{X_r^{\epsilon,\d}},Y_{r}^{\e,\d})\dif W^1_{r}.
\de
Thus, the It\^o formula, (\ref{b1line}) and \cite[Lemma 2.4]{arrst} imply that
\ce
|X_{t}^{\e,\d}-X_{s}^{\e,\d}|^2&=&-2\int_s^t\<X_{r}^{\e,\d}-X_{s}^{\e,\d},\dif K_r^{1,\e,\d}\>\no\\
&&+2\int_s^t\<X_{r}^{\e,\d}-X_{s}^{\e,\d},b_{1}(X_{r}^{\e,\d},\sL_{X_r^{\epsilon,\d}},Y_{r}^{\e,\d})\>\dif r\no\\
&&+2\e^\t\int_s^t\<X_{r}^{\e,\d}-X_{s}^{\e,\d},\s_{1}(X_{r}^{\e,\d},\sL_{X_r^{\epsilon,\d}},Y_{r}^{\e,\d})\dif W^1_{r}\>\no\\
&&+\e^{2\t}\int_s^t\|\s_{1}(X_{r}^{\e,\d},\sL_{X_r^{\epsilon,\d}},Y_{r}^{\e,\d})\|^2\dif r\no\\
&\leq&\int_s^t|X_{r}^{\e,\d}-X_{s}^{\e,\d}|^2\dif r+C\int_s^t(1+|X_{r}^{\e,\d}|^2+\|\sL_{X_r^{\epsilon,\d}}\|^2+|Y_{r}^{\e,\d}|^2)\dif r\\
&&+2\left|\int_s^t\<X_{r}^{\e,\d}-X_{s}^{\e,\d},\s_{1}(X_{r}^{\e,\d},\sL_{X_r^{\epsilon,\d}},Y_{r}^{\e,\d})\dif W^1_{r}\>\right|.
\de
By the BDG inequality, it holds that
\ce
\mE\sup\limits_{t\in[s,s+l]}|X_{t}^{\e,\d}-X_{s}^{\e,\d}|^2&\leq& \int_s^{s+l}\mE|X_{r}^{\e,\d}-X_{s}^{\e,\d}|^2\dif r+C\int_s^{s+l}(1+\mE|X_{r}^{\e,\d}|^2+\mE|Y_{r}^{\e,\d}|^2)\dif r\\
&&+C\mE\(\int_s^{s+l}|X_{r}^{\e,\d}-X_{s}^{\e,\d}|^2\|\s_{1}(X_{r}^{\e,\d},\sL_{X_r^{\epsilon,\d}},Y_{r}^{\e,\d})\|^2\dif r\)^{1/2}\\
&\leq& \int_s^{s+l}\mE\sup\limits_{u\in[s,r]}|X_{u}^{\e,\d}-X_{s}^{\e,\d}|^2\dif r+Cl+\frac{1}{2}\mE\sup\limits_{t\in[s,s+l]}|X_{t}^{\e,\d}-X_{s}^{\e,\d}|^2\\
&&+C\mE\int_s^{s+l}\|\s_{1}(X_{r}^{\e,\d},\sL_{X_r^{\epsilon,\d}},Y_{r}^{\e,\d})\|^2\dif r\\
&\leq& \int_s^{s+l}\mE\sup\limits_{u\in[s,r]}|X_{u}^{\e,\d}-X_{s}^{\e,\d}|^2\dif r+Cl+\frac{1}{2}\mE\sup\limits_{t\in[s,s+l]}|X_{t}^{\e,\d}-X_{s}^{\e,\d}|^2,
\de
which together with the Gronwall inequality yields (\ref{xegtscr}). The proof is complete.
\end{proof} 

{\bf Proof of Theorem \ref{xbarxpcr}.} We replace (\ref{xegts}), (\ref{barx0ts}) by (\ref{xegtscr}), (\ref{barx0tscr}), respectively, and show Theorem \ref{xbarxpcr} by the same deduction to that of Theorem \ref{xbarxp}.

\subsection{Proof of Theorem \ref{xbarxap}}

In this subsection, we prove Theorem \ref{xbarxap}. 

First of all, we construct an auxiliary process: 
\ce\left\{\begin{array}{l}
\dif \hat{Y}_{t}^{\d}\in -A_2(\hat{Y}_{t}^{\d})\dif t+\frac{1}{\d}b_{2}(X_{k\triangle}^{\d},\sL_{X_{k\triangle}^{\d}},\hat{Y}_{t}^{\d})\dif t
+\frac{1}{\sqrt{\d}}\s_{2}(X_{k\triangle}^{\d},\sL_{X_{k\triangle}^{\d}},\hat{Y}_{t}^{\d})\dif W^2_{t}, \\
\qquad\qquad\qquad t\in[k\triangle, (k+1)\triangle),\\
\hat{Y}_{k\triangle}^{\d}=Y_{k\triangle}^{\d}, \quad \hat{K}_{k\triangle}^{2,\d}=K_{k\triangle}^{2,\d}.
\end{array}
\right.
\de
Then by some similar deductions to that for $X^{\e,\d}, Y^{\e,\d}, \hat{Y}^{\e,\d},\bar{X}^0$, we obtain the following estimates.

\bl \label{averc0}
Under the assumptions of Theorem \ref{xbarxap}, it holds that
\ce
&&\mE\left(\sup\limits_{t\in[0,T]}|X_{t}^{\d}|^{2}\right)\leq C(1+\mE|\xi|^{2}+|y_0|^{2}), \quad \lim\limits_{l\rightarrow 0}\sup\limits_{s\in[0,T]}\mE\sup_{s\leqslant t \leqslant s+l}|X_{t}^{\d}-X_{s}^{\d}|^{2}=0,\\
&&\sup\limits_{t\in[0,T]}\mE|Y_{t}^{\d}|^{2}\leq C(1+\mE|\xi|^{2}+|y_0|^{2}),\quad \sup\limits_{t\in[0,T]}\mE|\hat{Y}_{t}^{\d}|^{2}\leq C(1+\mE|\xi|^{2}+|y_0|^{2}),\\
&&\sup\limits_{t\in[0,T]}\mE|Y_{t}^{\d}-\hat{Y}_{t}^{\d}|^{2}\leq \frac{C}{\a}
\left(\sup\limits_{s\in[0,T]}\mE\sup _{s \leqslant r \leqslant s+\triangle}|X_{r}^{\d}-X_{s}^{\d}|^{2}\right),\\
&&\mE\sup\limits_{t\in[0,T]}|\bar{X}_{t}|^{2}\leq C(1+\mE|\xi|^{2}), \quad \lim\limits_{l\rightarrow 0}\sup\limits_{s\in[0,T]}\mE\sup\limits_{s\leq t\leq s+l}|\bar{X}_{t}-\bar{X}_{s}|^2=0. 
\de
\el

{\bf Proof of Theorem \ref{xbarxap}.} 

We only need to replace $\e^\t\int_{0}^{t}\s_{1}(X_{s}^{\e,\d},\sL_{X_{s}^{\e,\d}},Y_{s}^{\e,\d})\dif W^1_{s}$ by $\int_{0}^{t}(\s_{1}(X_{s}^{\d},\sL_{X_{s}^{\d}})-\s_1(\bar{X}_{s},\sL_{\bar{X}_{s}}))\dif W^1_{s}$
and follow the proof of Theorem \ref{xbarxp} to complete the proof of Theorem \ref{xbarxap}.

\subsection{Proof of Theorem \ref{xbarxapcr}}

In this subsection, we prove Theorem \ref{xbarxapcr}. First of all, by the similar deduction, we have the following estimates.

\bl\label{xdhatxtscr}
Under the assumptions of Theorem \ref{xbarxapcr}, we have that for $l>0$ small enough 
\be
&&\sup\limits_{s\in[0,T]}\mE\sup _{s \leqslant t \leqslant s+l}|X_{t}^{\d}-X_{s}^{\d}|^{2}\leq Cl, \label{xdtscr}\\
&&\sup\limits_{s\in[0,T]}\mE\sup\limits_{s\leq t\leq s+l}|\bar{X}_{t}-\bar{X}_{s}|^2\leq Cl, \label{barxtscr}
\ee
where the constant $C>0$ is independent of $\d, l$.
\el

{\bf Proof of Theorem \ref{xbarxapcr}.} By (\ref{xdtscr}) (\ref{barxtscr}) and the estimates in Lemma \ref{averc0}, we follow the line of Theorem \ref{xbarxap} to prove Theorem \ref{xbarxapcr}.

\section{Proof of Theorem \ref{ldpmmsde}}\label{prooseco}

In this section, we prove Theorem \ref{ldpmmsde}. 

By Theorem \ref{well}, under the assumptions of Theorem \ref{ldpmmsde}, we know that the system (\ref{Eq1}) has a unique strong solution $(X_{\cdot}^{\e,\d},K_{\cdot}^{1,\e,\d},Y_{\cdot}^{\e,\d},K_{\cdot}^{2,\e,\d})$. Thus, there exists a functional $\cG^\e=\cG_{\sL_{X^{\epsilon,\d}}}^{\epsilon}: C([0,T];\mathbb{R}^{d_1+d_2})\mapsto C([0,T],\overline{\cD(A_1)})$ such that 
\ce
X^{\epsilon,\d}=\cG^{\epsilon}(\sqrt{\e}W), \quad W:=(W^1,W^2).
\de
In order to prove the Laplace principle for $X^{\epsilon,\d}$, we will verify Condition \ref{cond} with $\mathbb{S}=C([0,T],\overline{\cD(A_1)})$.

First of all, we consider the following controlled processes:
\be\left\{\begin{array}{l}
\dif X_{t}^{\e,\d,u}\in -A_1(X_{t}^{\e,\d,u})\dif t+b_{1}(X_{t}^{\e,\d,u},\sL_{X_t^{\epsilon,\d}},Y_{t}^{\e,\d,u})\dif t+\sigma_1(X^{\epsilon,\d,u}_{t},\sL_{X_t^{\epsilon,\d}})\pi_1u(t)\dif t\\
\qquad\qquad+\sqrt{\e}\s_{1}(X_{t}^{\e,\d,u},\sL_{X_t^{\epsilon,\d}})\dif W^1_{t},\\
X_{0}^{\e,\d,u}=x_0\in\overline{\cD(A_1)},\quad  0\leq t\leq T,\\
\dif Y_{t}^{\e,\d,u}\in -A_2(Y_{t}^{\e,\d,u})\dif t+\frac{1}{\d}b_{2}(X_{t}^{\e,\d,u},\sL_{X_t^{\epsilon,\d}},Y_{t}^{\e,\d,u})\dif t\\
\qquad\qquad +\frac{1}{\sqrt{\d \e}}\sigma_2(X^{\epsilon,\d,u}_{t},\sL_{X_t^{\epsilon,\d}},Y^{\epsilon,\d,u}_{t})\pi_2u(t)\dif t+\frac{\sqrt{\e}}{\sqrt{\d \e}}\s_{2}(X_{t}^{\e,\d,u},\sL_{X_t^{\epsilon,\d}},Y_{t}^{\e,\d,u})\dif W^2_{t},  \\
Y_{0}^{\e,\d,u}=y_0\in\overline{\cD(A_2)},\quad  0\leq t\leq T,\quad u\in\mathbf{A}_2^{N},
\end{array}
\right.
\label{contproc}
\ee
where $\pi_1: \mR^{d_1+d_2}\mapsto \mR^{d_1}, \pi_2: \mR^{d_1+d_2}\mapsto \mR^{d_2}$ are two projection operators. Thus, by the Girsanov theorem, the system $(\ref{contproc})$ have a unique strong solution denoted by $(X^{\epsilon,\d,u}, K^{1,\epsilon,\d,u},Y^{\epsilon,\d,u}, K^{2,\epsilon,\d,u})$. Moreover, $X^{\epsilon,\d,u}=\cG^{\epsilon}(\sqrt{\epsilon}W+\int_{0}^{\cdot}u(s)\dif s)$. 

\br
Here we emphasize that the controlled system $(\ref{contproc})$ contains the distribution of $X_t^{\epsilon,\d}$ but not the distribution of $X_{t}^{\e,\d,u}$, which is a key characteristic of McKean-Vlasov stochastic systems.
\er

Besides, we recall Eq.(\ref{ldppsio0equde}), i.e.
\ce\left\{\begin{array}{l}
\dif \bar{X}^0_{t}\in -A_1(\bar{X}^0_{t})\dif t+\bar{b}_{1}(\bar{X}^0_{t},D_{\bar{X}_t^0})\dif t,\\
\bar{X}^0_{0}=x_0\in\overline{\cD(A_1)}.
\end{array}
\right.
\de
By Lemma \ref{averc}, the above equation has a unique solution $(\bar{X}^0,\bar{K}^0)$. Then we consider the following multivalued differential equation:
\be\left\{\begin{array}{l}
\dif\bar{X}^{u}_{t}\in A_1(\bar{X}^{u}_{t})\dif t+\bar{b}_{1}(\bar{X}^u_{t},D_{\bar{X}^0_{t}})\dif t+\s_{1}(\bar{X}^u_{t},D_{\bar{X}^0_{t}})\pi_1u(t)\dif t, \quad u\in\mathbf{A}_2^{N},\\
\bar{X}^{u}_{0}=x_0\in\overline{\cD(A_1)}.
\end{array}
\right.
\label{deteequa}
\ee
By (\ref{barb1lip}) and $(\mathbf{H}^{1'}_{b_{1}, \s_{1}})$, it holds that Eq.(\ref{deteequa}) has a unique solution $(\bar{X}^{u}, \bar{K}^{u})$. Define the measurable map $\cG^{0}: C([0,T]; \mR^{d_1+d_2})\mapsto \mS$ by $\cG^{0}(\int_{0}^{\cdot}u(s)\dif s)=\bar{X}^{u}$, and we verify Condition \ref{cond} through $\cG^{\epsilon}, \cG^{0}$. We start with some key estimates.

\subsection{Some key estimates}\label{some}

\bl \label{xutzutc}
Under the assumptions of Theorem \ref{ldpmmsde}, for $\{u_{\epsilon}, \e\in(0,1)\}\subset\mathbf{A}_{2}^{N}$, there exists a constant $C>0$ such that 
\be
&&\mE\left(\sup\limits_{t\in[0,T]}|X_{t}^{\e,\d,u_\e}|^{2}\right)\leq C(1+|x_0|^{2}+|y_0|^{2}), \label{xeub}\\
&&\int_0^T\mE|Y_{r}^{\e,\d,u_\e}|^{2}\dif r\leq C(1+|x_0|^{2}+|y_0|^{2}),\label{yeub}\\
&&\mE|K^{1,\e,\d,u_\e}|_0^T\leq C(1+|x_0|^{2}+|y_0|^{2}). \label{keub}
\ee
\el
\begin{proof}
First of all, we estimate $X_{t}^{\e,\d,u_\e}$. Note that $X_{t}^{\e,\d,u_\e}$ satisfies the following equation:
\ce
X_{t}^{\e,\d,u_\e}&=&x_0-K_t^{1,\e,\d,u_\e}+\int_0^t b_{1}(X_{s}^{\e,\d,u_\e},\sL_{X_s^{\epsilon,\d}},Y_{s}^{\e,\d,u_\e})\dif s+\int_{0}^{t}\sigma_1(X^{\epsilon,\d,u_\e}_{s},\sL_{X_s^{\epsilon,\d}})\pi_1u_\e(s)\dif s\\
&&+\sqrt{\e}\int_0^t\s_{1}(X_{s}^{\e,\d,u_\e},\sL_{X_s^{\epsilon,\d}})\dif W^1_{s}.
\de
The It\^o formula yields that for any $a\in\text{Int}(\cD(A_1))$
\be
&&|X_{t}^{\e,\d,u_\e}-a|^2\no\\
&=&|x_0-a|^2-2\int_0^t\<X_{s}^{\e,\d,u_\e}-a,\dif K_s^{1,\e,\d,u_\e}\>+2\int_0^t\<X_{s}^{\e,\d,u_\e}-a, b_{1}(X_{s}^{\e,\d,u_\e},\sL_{X_s^{\epsilon,\d}},Y_{s}^{\e,\d,u_\e})\>\dif s\no\\
&&+2\int_0^t\<X_{s}^{\e,\d,u_\e}-a,\sigma_1(X^{\epsilon,\d,u_\e}_{s},\sL_{X_s^{\epsilon,\d}})\pi_1u_\e(s)\>\dif s+\e\int_0^t\|\s_{1}(X_{s}^{\e,\d,u_\e},\sL_{X_s^{\epsilon,\d}})\|^2\dif s\no\\
&&+2\sqrt{\e}\int_0^t\<X_{s}^{\e,\d,u_\e}-a,\s_{1}(X_{s}^{\e,\d,u_\e},\sL_{X_s^{\epsilon,\d}})\dif W^1_{s}\>.
\label{itoxegu}
\ee
Then the BDG inequality, Lemma \ref{inteineq} and $(\ref{b1line})$ imply that for any $u\in A_1(0)$
\ce
&&\mE\left(\sup\limits_{s\in[0,t]}|X_{s}^{\e,\d,u_\e}-a|^{2}\right)\\
&\leq&\(|x_0-a|^2+2(M_2+M_3)T\)+(2M_2+1)\mE\int_0^t|X_{r}^{\e,\d,u_\e}-a|^2\dif r\\
&&+\bar{L}_{b_{1}, \s_{1}}\mE\int_0^t(1+|X_{r}^{\e,\d,u_\e}|^2+\|\sL_{X_r^{\epsilon,\d}}\|^2+|Y_{r}^{\e,\d,u_\e}|^2)\dif r\no\\
&&+2\mE\sup\limits_{s\in[0,t]}\left|\int_0^s\<X_{r}^{\e,\d,u_\e}-a,\sigma_1(X^{\epsilon,\d,u_\e}_{r},\sL_{X_r^{\epsilon,\d}})\pi_1u_\e(r)\>\dif r\right|\\
&&+2\mE\sup\limits_{s\in[0,t]}\left|\int_0^s\<X_{r}^{\e,\d,u_\e}-a,\s_{1}(X_{r}^{\e,\d,u_\e},\sL_{X_r^{\epsilon,\d}})\dif W^1_{r}\>\right|\no\\ 
&\leq&C(|x_0-a|^{2}+1)+C\mE\int_{0}^{t}|X_{r}^{\e,\d,u_\e}-a|^2\dif r+C\int_{0}^{t}\|\sL_{X_r^{\epsilon,\d}}\|^2\dif r+C\mE\int_{0}^{t}|Y_{r}^{\e,\d,u_\e}|^{2}\dif r\\
&&+\frac{1}{4}\mE\left(\sup\limits_{s\in[0,t]}|X_{s}^{\e,\d,u_\e}-a|^{2}\right)+C\mE\left(\int_0^t\|\sigma_1(X^{\epsilon,\d,u_\e}_{r},\sL_{X_r^{\epsilon,\d}})\|^2\dif r\right)\left(\int_0^t|u_\e(r)|^2\dif r\right)\no\\
&&+C\mE\left(\int_0^t|X_{r}^{\e,\d,u_\e}-a|^2\|\s_{1}(X_{r}^{\e,\d,u_\e},\sL_{X_r^{\epsilon,\d}})\|^2\dif r\right)^{1/2}\no\\
&\leq&C(|x_0-a|^{2}+1)+C\mE\int_{0}^{t}|X_{r}^{\e,\d,u_\e}-a|^2\dif r+C\int_{0}^{t}\mE|X_r^{\epsilon,\d}|^2\dif r+C\mE\int_{0}^{t}|Y_{r}^{\e,\d,u_\e}|^{2}\dif r\no\\
&&+\frac{1}{2}\mE\left(\sup\limits_{s\in[0,t]}|X_{s}^{\e,\d,u_\e}-a|^{2}\right)+C\mE\int_0^t\|\s_{1}(X_{r}^{\e,\d,u_\e},\sL_{X_r^{\epsilon,\d}})\|^2\dif r\\
&\leq&C(|x_0-a|^{2}+1)+C\mE\int_{0}^{t}|X_{r}^{\e,\d,u_\e}-a|^2\dif r+C\int_{0}^{t}\mE|X_r^{\epsilon,\d}|^2\dif r+C\mE\int_{0}^{t}|Y_{r}^{\e,\d,u_\e}|^{2}\dif r\no\\
&&+\frac{1}{2}\mE\left(\sup\limits_{s\in[0,t]}|X_{s}^{\e,\d,u_\e}-a|^{2}\right),
\de
and furthermore
\be
\mE\left(\sup\limits_{s\in[0,t]}|X_{s}^{\e,\d,u_\e}-a|^{2}\right)&\overset{(\ref{xgb})}{\leq}& C(1+|x_0|^{2}+|y_0|^{2})+C\int_{0}^{t}\mE|X_{r}^{\e,\d,u_\e}-a|^{2}\dif r\no\\
&&+C\int_{0}^{t}\mE|Y_{r}^{\e,\d,u_\e}|^{2}\dif r.
\label{exqcu}
\ee

For $Y_{t}^{\e,\d,u_\e}$, fix $v\in A_2(0)$. Applying the It\^{o} formula to $|Y_{t}^{\e,\d,u_\e}|^{2}e^{\l t}$ for $\l=\frac{\a}{3\d}$ and taking the expectation, one could obtain that
\ce
\mE|Y_{t}^{\e,\d,u_\e}|^{2}e^{\l t}
&=& |y_0|^{2}+\l\mE\int_0^t|Y_{s}^{\e,\d,u_\e}|^{2}e^{\l s}\dif s-2\mE\int_0^te^{\l s}\<Y_{s}^{\e,\d,u_\e},\dif K_s^{2,\e,\d,u_\e}\>\\
&&+\frac{2}{\d}\mE\int_{0}^{t}e^{\l s}\<Y_{s}^{\e,\d,u_\e}, b_{2}(X_{s}^{\e,\d,u_\e},\sL_{X_s^{\epsilon,\d}},Y_{s}^{\e,\d,u_\e})\>\dif s\\
&&+\frac{1}{\d}\mE\int_{0}^{t}e^{\l s}\|\s_{2}(X_{s}^{\e,\d,u_\e},\sL_{X_s^{\epsilon,\d}},Y_{s}^{\e,\d,u_\e})\|^2\dif s\\
&&+\frac{2}{\sqrt{\d \e}}\mE\int_{0}^{t}e^{\l s}\<Y_{s}^{\e,\d,u_\e},\sigma_2(X^{\epsilon,\d,u_\e}_{s},\sL_{X_s^{\epsilon,\d}},Y^{\epsilon,\d,u_\e}_{s})\pi_2u_\e(s)\>\dif s\\
&\overset{(\ref{bemu})}{\leq}& |y_0|^{2}+\l\mE\int_0^t|Y_{s}^{\e,\d,u_\e}|^{2}e^{\l s}\dif s+2\mE\int_0^te^{\l s}|v||Y_{s}^{\e,\d,u_\e}|\dif s\\
&&+\frac{1}{\d}\mE\int_{0}^{t}e^{\l s}\(-\a|Y_{s}^{\e,\d,u_\e}|^{2}+C(1+|X_{s}^{\e,\d,u_\e}|^{2}+\|\sL_{X_s^{\epsilon,\d}}\|^2)\)\dif s\\
&&+\frac{2}{\sqrt{\d \e}}\mE\int_{0}^{t}e^{\l s}|Y_{s}^{\e,\d,u_\e}|\|\sigma_2(X^{\epsilon,\d,u_\e}_{s},\sL_{X_s^{\epsilon,\d}},Y^{\epsilon,\d,u_\e}_{s})\||u_\e(s)|\dif s.
\de
Note that for the third and last terms of the right side for the above inequality
\ce
2|v||Y_{s}^{\e,\d,u_\e}|\leq \frac{\a}{3\d}|Y_{s}^{\e,\d,u_\e}|^2+\frac{3\d}{\a}|v|^2,
\de
and
\ce
&&\frac{2}{\sqrt{\d \e}}|Y_{s}^{\e,\d,u_\e}|\|\sigma_2(X^{\epsilon,\d,u_\e}_{s},\sL_{X_s^{\epsilon,\d}},Y^{\epsilon,\d,u_\e}_{s})\||u_\e(s)|\\
&\leq&\frac{\a}{3\d}|Y_{s}^{\e,\d,u_\e}|^{2}+\frac{C}{\e}\|\sigma_2(X^{\epsilon,\d,u_\e}_{s},\sL_{X_s^{\epsilon,\d}},Y^{\epsilon,\d,u_\e}_{s})\|^2|u_\e(s)|^2\\
&\leq&\frac{\a}{3\d}|Y_{s}^{\e,\d,u_\e}|^{2}+\frac{C}{\e}|u_\e(s)|^2,
\de
where $(\mathbf{H}^4_{\s_{2}})$ is used. Thus, we have that
\ce
\mE|Y_{t}^{\e,\d,u_\e}|^{2}e^{\l t}&\leq& |y_0|^{2}+C\int_0^te^{\l s}\dif s+\(\l+\frac{2\a}{3\d}-\frac{\a}{\d}\)\mE\int_0^t|Y_{s}^{\e,\d,u_\e}|^{2}e^{\l s}\dif s\\
&&+\frac{C}{\d}\mE\int_{0}^{t}e^{\l s}(1+|X_{s}^{\e,\d,u_\e}|^{2}+\|\sL_{X_s^{\epsilon,\d}}\|^2)\dif s+\frac{C}{\e}\mE\int_{0}^{t}e^{\l s}|u_\e(s)|^2\dif s\\
&\leq& |y_0|^{2}+C\frac{e^{\l t}-1}{\l}+C\frac{e^{\l t}-1}{\d\l}\left(1+\mE\sup\limits_{s\in[0,t]}|X_{s}^{\e,\d,u_\e}|^{2}+\mE\sup\limits_{s\in[0,t]}|X_{s}^{\e,\d}|^{2}\right)\\
&&+\frac{C}{\e}\mE\int_{0}^{t}e^{\l s}|u_\e(s)|^2\dif s.
\de
From this, it follows that
\ce
\mE|Y_{t}^{\e,\d,u_\e}|^{2}&\overset{(\ref{xgb})}{\leq}& C(1+|x_0|^{2}+|y_0|^{2})+C\mE\left(\sup\limits_{s\in[0,t]}|X_{s}^{\e,\d,u_\e}|^{2}\right)\\
&&+\frac{C}{\e}\mE\int_{0}^{t}e^{-\l (t-s)}|u_\e(s)|^2\dif s,
\de
and furthermore
\be
\int_0^t\mE|Y_{r}^{\e,\d,u_\e}|^{2}\dif r&\leq& CT(1+|x_0|^{2}+|y_0|^{2})+C\int_0^t\mE\left(\sup\limits_{s\in[0,r]}|X_{s}^{\e,\d,u_\e}|^{2}\right)\dif r\no\\
&&+\frac{C}{\e}\mE\int_0^t\int_{0}^{r}e^{-\l (r-s)}|u_\e(s)|^2\dif s\dif r\no\\
&\leq& CT(1+|x_0|^{2}+|y_0|^{2})+C\int_0^t\mE\left(\sup\limits_{s\in[0,r]}|X_{s}^{\e,\d,u_\e}|^{2}\right)\dif r\no\\
&&+C\left(\frac{\d}{\e}\right)\mE\int_0^t|u_\e(s)|^2\dif s\no\\
&\leq& C(1+|x_0|^{2}+|y_0|^{2})+C\int_0^t\mE\left(\sup\limits_{s\in[0,r]}|X_{s}^{\e,\d,u_\e}|^{2}\right)\dif r.
\label{zeues}
\ee
where we use $u_\e\in {\bf A}_2^N$ and $\lim\limits_{\e\rightarrow0}\frac{\d}{\e}=0$ in the last inequality. 

Inserting (\ref{zeues}) in (\ref{exqcu}), by the Gronwall inequality one can get (\ref{xeub}) and (\ref{yeub}).

Finally, for $K^{1,\e,\d,u_\e}$, by (\ref{itoxegu}) and Lemma \ref{inteineq}, it holds that
\ce
|X_{T}^{\e,\d,u_\e}-a|^2&=&|x_0-a|^2-2\int_0^T\<X_{s}^{\e,\d,u_\e}-a, \dif K_s^{1,\e,\d,u_\e}\>\\
&&+2\int_0^T\<X_{s}^{\e,\d,u_\e}-a, b_{1}(X_{s}^{\e,\d,u_\e},\sL_{X_s^{\epsilon,\d}},Y_{s}^{\e,\d,u_\e})\>\dif s\no\\
&&+2\int_0^T\<X_{s}^{\e,\d,u_\e}-a,\sigma_1(X^{\epsilon,\d,u_\e}_{s},\sL_{X_s^{\epsilon,\d}})\pi_1u_\e(s)\>\dif s\\
&&+2\sqrt{\e}\int_0^T\<X_{s}^{\e,\d,u_\e}-a,\s_{1}(X_{s}^{\e,\d,u_\e},\sL_{X_s^{\epsilon,\d}})\dif W^1_{s}\>\\
&&+\e\int_0^T\|\s_{1}(X_{s}^{\e,\d,u_\e},\sL_{X_s^{\epsilon,\d}})\|^2\dif s\no\\
&\leq&|x_0-a|^2-2M_1\left| K^{1,\e,\d,u_\e} \right|_{0}^{T}+2M _2\int_0^T{\left| X^{\e,\d,u_\e}_s-a\right|}\dif s+2M_3T\\
&&+\int_0^T|X_{s}^{\e,\d,u_\e}|^2\dif s+\int_0^T|b_{1}(X_{s}^{\e,\d,u_\e},\sL_{X_s^{\epsilon,\d}},Y_{s}^{\e,\d,u_\e})|^2\dif s\\
&&+\left(\sup\limits_{s\in[0,T]}|X_{s}^{\e,\d,u_\e}-a|^2\right)+\int_0^T\|\sigma_1(X^{\epsilon,\d,u_\e}_{s},\sL_{X_s^{\epsilon,\d}})\|^2\dif s\int_0^T|u_\e(s)|^2\dif s\\
&&+2\sqrt{\e}\int_0^T\<X_{s}^{\e,\d,u_\e}-a,\s_{1}(X_{s}^{\e,\d,u_\e},\sL_{X_s^{\epsilon,\d}})\dif W^1_{s}\>\\
&&+\e\int_0^T\|\s_{1}(X_{s}^{\e,\d,u_\e},\sL_{X_s^{\epsilon,\d}})\|^2\dif s,
\de
and 
\ce
&&2M_1\mE\left| K^{1,\e,\d,u_\e} \right|_{0}^{T}\\
&\leq& |x_0-a|^2+2(M_2+M_3)T+(2M _2+1)\int_0^T\mE{\left| X^{\e,\d,u_\e}_s-a\right|^2}\dif s+\int_0^T\mE|X_{s}^{\e,\d,u_\e}|^2\dif s\\
&&+\mE\left(\sup\limits_{s\in[0,T]}|X_{s}^{\e,\d,u_\e}-a|^2\right)+C\int_0^T(1+\mE{\left| X^{\e,\d,u_\e}_s\right|^2}+\|\sL_{X_s^{\epsilon,\d}}\|^2+\mE{\left| Y^{\e,\d,u_\e}_s\right|^2})\dif s,
\de
which together with (\ref{xgb}), (\ref{xeub}) and (\ref{yeub}) yields that
$$
\mE\left| K^{1,\e,\d,u_\e} \right|_{0}^{T}\leq  C(1+|x_0|^{2}+|y_0|^{2}).
$$
The proof is complete.
\end{proof}

By the same deduction to that in Lemma \ref{xehatxets}, we obtain the following result.

\bl
Under the assumptions of Theorem \ref{ldpmmsde}, for $\{u_{\epsilon}, \e\in(0,1)\}\subset\mathbf{A}_{2}^{N}$, we have that
\be
\lim\limits_{l\rightarrow 0}\sup\limits_{s\in[0,T]}\mE\sup _{s \leqslant t \leqslant s+l}|X_{t}^{\e,\d,u_\e}-X_{s}^{\e,\d,u_\e}|^{2}=0.
\label{xegutse}
\ee
\el

\bl
Under the assumptions of Theorem \ref{ldpmmsde}, for any $h\in\mathbf{D}_{2}^{N}$, Eq.(\ref{ldppsioequ}) has a unique solution $(\bar{X}^h,\bar{K}^h)$. Moreover, it holds that
\be
&&\sup\limits_{t\in[0,T]}|\bar{X}^h_{t}|^{2}\leq C(1+|x_0|^{2}), \label{barxub}\\
&&|\bar{K}^{h}|_0^T\leq C(1+|x_0|^{2}),\label{barkub}\\
&&\lim\limits_{l\rightarrow 0}\sup\limits_{s\in[0,T]}\sup\limits_{s\leq t\leq s+l}|\bar{X}^h_{t}-\bar{X}^h_{s}|^2=0. \label{barxuts}
\ee
\el

Since the proofs of (\ref{barxub}), (\ref{barkub}) and (\ref{barxuts}) are similar to that of (\ref{xeub}), (\ref{keub}) and (\ref{xegutse}), respectively, we omit them.

Finally we introduce the following auxiliary process: 
\be\left\{\begin{array}{l}
\hat{Y}_{t}^{\e,\d,u_\e}\in -A_2(\hat{Y}_{t}^{\e,\d,u_\e})\dif t+\frac{1}{\d}b_{2}(X_{k\triangle}^{\e,\d,u_\e},\sL_{X_{k\triangle}^{\epsilon,\d}},\hat{Y}_{t}^{\e,\d,u_\e})\dif t\\
\qquad\qquad +\frac{1}{\sqrt{\d}}\s_{2}(X_{k\triangle}^{\e,\d,u_\e},\sL_{X_{k\triangle}^{\epsilon,\d}},\hat{Y}_{t}^{\e,\d,u_\e})\dif W^2_{t}, \quad t\in[k\triangle,(k+1)\triangle),\\
\hat{Y}_{k\triangle}^{\e,\d,u_\e}=Y_{k\triangle}^{\e,\d,u_\e}, \quad \hat{K}_t^{2,\e,\d,u_\e}=K_t^{2,\e,\d,u_\e}.
\end{array}
\right.
\label{hatzu}
\ee

\bl
Under the assumptions of Theorem \ref{ldpmmsde}, for $\{u_{\epsilon}, \e\in(0,1)\}\subset\mathbf{A}_{2}^{N}$, there exists a constant $C>0$ such that 
 \be
 \sup\limits_{t\in[0,T]}\mE|\hat{Y}_{t}^{\e,\d,u_\e}|^{2}&\leq& C(1+|x_0|^{2}+|y_0|^{2}), \label{hatzub}\\
\int_0^T\mE|Y_{t}^{\e,\d,u_\e}-\hat{Y}_{t}^{\e,\d,u_\e}|^{2}\dif t&\leq& CN\left(\frac{\d}{\e}\right)+\frac{CT}{\a}\left(\sup\limits_{s\in[0,T]}\mE\sup _{s \leqslant r \leqslant s+\triangle}|X_{r}^{\e,\d,u_\e}-X_{s}^{\e,\d,u_\e}|^{2}\right)\no\\
&&+\frac{CT}{\a}\left(\sup\limits_{s\in[0,T]}\mE\sup _{s \leqslant r \leqslant s+\triangle}|X_{r}^{\e,\d}-X_{s}^{\e,\d}|^{2}\right).
\label{unztu}
\ee
\el
\begin{proof}
Since the proof of (\ref{hatzub}) is similar to that for (\ref{ygb}), we only prove (\ref{unztu}).

First of all, by (\ref{Eq1}) and (\ref{hatzu}), we have that for $t\in[k\triangle,(k+1)\triangle)$
\ce
&&Y_{t}^{\e,\d,u_\e}-\hat{Y}_{t}^{\e,\d,u_\e}\\
&=&-K_t^{2,\e,\d,u_\e}+\hat{K}_t^{2,\e,\d,u_\e}+\frac{1}{\d}\int_{k\triangle}^{t}\(b_{2}(X_{s}^{\e,\d,u_\e},\sL_{X_{s}^{\e,\d}},Y_{s}^{\e,\d,u_\e})
-b_{2}(X_{k\triangle}^{\e,\d,u_\e},\sL_{X_{k\triangle}^{\e,\d}},\hat{Y}_{s}^{\e,\d,u_\e})\)\dif s\\
&&+\frac{1}{\sqrt{\d}}\int_{k\triangle}^{t}\(\s_{2}(X_{s}^{\e,\d,u_\e},\sL_{X_{s}^{\e,\d}},Y_{s}^{\e,\d,u_\e})
-\s_{2}(X_{k\triangle}^{\e,\d,u_\e},\sL_{X_{k\triangle}^{\e,\d}},\hat{Y}_{s}^{\e,\d,u_\e})\)\dif W^2_{s}\\
&&+\frac{1}{\sqrt{\d \e}}\int_{k\triangle}^{t}\s_{2}(X_{s}^{\e,\d,u_\e},\sL_{X_{s}^{\e,\d}},Y_{s}^{\e,\d,u_\e})\pi_2u_\e(s)\dif s.
\de
Applying the It\^{o} formula to $|Y_{t}^{\e,\d,u_\e}-\hat{Y}_{t}^{\e,\d,u_\e}|^{2}e^{\l t}$ for $\l=\frac{\a}{2\d}$ and taking the expectation, by $(\mathbf{H}^{1}_{b_{2},\s_{2}})$ and $(\mathbf{H}^{2'}_{b_{2},\s_{2}})$ one could obtain that
\ce
&&\mE|Y_{t}^{\e,\d,u_\e}-\hat{Y}_{t}^{\e,\d,u_\e}|^{2}e^{\l t}\\
&=&\l\mE\int_{k\triangle}^{t}|Y_{s}^{\e,\d,u_\e}-\hat{Y}_{s}^{\e,\d,u_\e}|^{2}e^{\l s}\dif s-\mE\int_{k\triangle}^{t}2e^{\l s}\<Y_{s}^{\e,\d,u_\e}-\hat{Y}_{s}^{\e,\d,u_\e},\dif (K_s^{2,\e,\d,u_\e}-\hat{K}_s^{2,\e,\d,u_\e})\>\\
&&+\frac{1}{\d}\mE\int_{k\triangle}^{t}2e^{\l s}\<Y_{s}^{\e,\d,u_\e}-\hat{Y}_{s}^{\e,\d,u_\e}, b_{2}(X_{s}^{\e,\d,u_\e},\sL_{X_{s}^{\e,\d}},Y_{s}^{\e,\d,u_\e})
-b_{2}(X_{k\triangle}^{\e,\d,u_\e},\sL_{X_{k \triangle}^{\e,\d}},\hat{Y}_{s}^{\e,\d,u_\e})\>\dif s\\
&&+\frac{1}{\sqrt{\d \e}}\mE\int_{k\triangle}^{t}2e^{\l s}\<Y_{s}^{\e,\d,u_\e}-\hat{Y}_{s}^{\e,\d,u_\e}, \s_{2}(X_{s}^{\e,\d,u_\e},\sL_{X_{s}^{\e,\d}},Y_{s}^{\e,\d,u_\e})\pi_2u_\e(s)\>\dif s\\
&&+\frac{1}{\d}\mE\int_{k \triangle}^{t}e^{\l s}\|\s_{2}(X_{s}^{\e,\d,u_\e},\sL_{X_{s}^{\e,\d}},Y_{s}^{\e,\d,u_\e})
-\s_{2}(X_{k\triangle}^{\e,\d,u_\e},\sL_{X_{k \triangle}^{\e,\d}},\hat{Y}_{s}^{\e,\d,u_\e})\|^{2}\dif s\\
&\leq&\l\mE\int_{k\triangle}^{t}|Y_{s}^{\e,\d,u_\e}-\hat{Y}_{s}^{\e,\d,u_\e}|^{2}e^{\l s}\dif s\\
&&+\frac{1}{\d}\mE\int_{k \triangle}^{t}2e^{\l s}\<Y_{s}^{\e,\d,u_\e}-\hat{Y}_{s}^{\e,\d,u_\e}, b_{2}(X_{s}^{\e,\d,u_\e},\sL_{X_{s}^{\e,\d}},Y_{s}^{\e,\d,u_\e})-b_{2}(X_{s}^{\e,\d,u_\e},\sL_{X_{s}^{\e,\d}},\hat{Y}_{s}^{\e,\d,u_\e})\>\dif s\\
&&+\frac{1}{\d}\mE\int_{k \triangle}^{t}e^{\l s}\|\s_{2}(X_{s}^{\e,\d,u_\e},\sL_{X_{s}^{\e,\d}},Y_{s}^{\e,\d,u_\e})-\s_{2}(X_{s}^{\e,\d,u_\e},\sL_{X_{s}^{\e,\d}},\hat{Y}_{s}^{\e,\d,u_\e})\|^{2}\dif s\\
&&+\frac{\a}{2\d}\mE\int_{k \triangle}^{t}|Y_{s}^{\e,\d,u_\e}-\hat{Y}_{s}^{\e,\d,u_\e}|^{2}e^{\l s}\dif s+\frac{C}{\e}\mE\int_{k\triangle}^{t}e^{\l s}|u_\e(s)|^2\dif s\\
&&+\frac{1}{\d}\mE\int_{k \triangle}^{t}2e^{\l s}\<Y_{s}^{\e,\d,u_\e}-\hat{Y}_{s}^{\e,\d,u_\e}, b_{2}(X_{s}^{\e,\d,u_\e},\sL_{X_{s}^{\e,\d}},\hat{Y}_{s}^{\e,\d,u_\e})-b_{2}(X_{k \triangle}^{\e,\d,u_\e},\sL_{X_{k \triangle}^{\e,\d}},\hat{Y}_{s}^{\e,\d,u_\e})\>\dif s\\
&&+\frac{1}{\d}\mE\int_{k\triangle}^{t}e^{\l s}\|\s_{2}(X_{s}^{\e,\d,u_\e},\sL_{X_{s}^{\e,\d}},Y_{s}^{\e,\d,u_\e})-\s_{2}(X_{s}^{\e,\d,u_\e},\sL_{X_{s}^{\e,\d}},\hat{Y}_{s}^{\e,\d,u_\e})\|^{2}\dif s\\
&&+\frac{1}{\d}\mE\int_{k\triangle}^{t}2e^{\l s}\|\s_{2}(X_{s}^{\e,\d,u_\e},\sL_{X_{s}^{\e,\d}},\hat{Y}_{s}^{\e,\d,u_\e})-\s_{2}(X_{k \triangle}^{\e,\d,u_\e},\sL_{X_{k \triangle}^{\e,\d}},\hat{Y}_{s}^{\e,\d,u_\e})\|^{2}\dif s\\
&\leq&(\l-\frac{\b}{\d}+\frac{2L'_{b_2,\s_2}}{\d}+\frac{\a}{2\d})\mE\int_{k\triangle}^{t}|Y_{s}^{\e,\d,u_\e}-\hat{Y}_{s}^{\e,\d,u_\e}|^{2}e^{\l s}\dif s+\frac{C}{\e}\mE\int_{k\triangle}^{t}e^{\l s}|u_\e(s)|^2\dif s\\
&&+\frac{C}{\d}\mE\int_{k\triangle}^{t}e^{\l s}|X_{s}^{\e,\d,u_\e}-X_{k \triangle}^{\e,\d,u_\e}|^{2}\dif s+\frac{C}{\d}\int_{k\triangle}^{t}e^{\l s}\mW_2^2(\sL_{X_{s}^{\e,\d}},\sL_{X_{k \triangle}^{\e,\d}})\dif s\\
&\leq&\frac{C}{\e}\mE\int_{k\triangle}^{t}e^{\l s}|u_\e(s)|^2\dif s+\frac{C}{\d}\left(\sup\limits_{s\in[0,T]}\mE\sup _{s \leqslant r \leqslant s+\triangle}|X_{r}^{\e,\d,u_\e}-X_{s}^{\e,\d,u_\e}|^{2}\right)\frac{e^{\l t}-e^{\l k\triangle}}{\l}\\
 &&+\frac{C}{\d}\left(\sup\limits_{s\in[0,T]}\mE\sup _{s \leqslant r \leqslant s+\triangle}|X_{r}^{\e,\d}-X_{s}^{\e,\d}|^{2}\right)\frac{e^{\l t}-e^{\l k \triangle}}{\l}.
\de

Finally, it follows that
\ce
\int_{k\triangle}^{t}\mE|Y_{r}^{\e,\d,u_\e}-\hat{Y}_{r}^{\e,\d,u_\e}|^{2}\dif r&\leq& C\left(\frac{\d}{\e}\right)\mE\int_{k\triangle}^{t}|u_\e(s)|^2\dif s\\
&&+\frac{C}{\a}\left(\sup\limits_{s\in[0,T]}\mE\sup _{s \leqslant r \leqslant s+\triangle}|X_{r}^{\e,\d,u_\e}-X_{s}^{\e,\d,u_\e}|^{2}\right)\triangle\\
&&+\frac{C}{\a}\left(\sup\limits_{s\in[0,T]}\mE\sup _{s \leqslant r \leqslant s+\triangle}|X_{r}^{\e,\d}-X_{s}^{\e,\d}|^{2}\right)\triangle,
\de
and furthermore
\ce
\int_0^T\mE|Y_{t}^{\e,\d,u_\e}-\hat{Y}_{t}^{\e,\d,u_\e}|^{2}\dif t&\leq& CN\left(\frac{\d}{\e}\right)+\frac{CT}{\a}\left(\sup\limits_{s\in[0,T]}\mE\sup _{s \leqslant r \leqslant s+\triangle}|X_{r}^{\e,\d,u_\e}-X_{s}^{\e,\d,u_\e}|^{2}\right)\no\\
&&+\frac{CT}{\a}\left(\sup\limits_{s\in[0,T]}\mE\sup _{s \leqslant r \leqslant s+\triangle}|X_{r}^{\e,\d}-X_{s}^{\e,\d}|^{2}\right).
\de
The proof is complete.
\end{proof}

\subsection{Verification for Condition \ref{cond}}

\bl\label{auxilemm2}
Suppose that the assumptions of Theorem \ref{ldpmmsde} hold, and $h_{\e}\rightarrow h$ in $\mathbf{D}_2^{N}$ as ${\e}\rightarrow0$. Then $\cG^{0}(\int_{0}^{\cdot}h_{\e}(s)\dif s)$ converges to $\cG^{0}(\int_{0}^{\cdot}h(s)\dif s)$.
\el
\begin{proof}
The proof consists of four steps. First of all, we estimate $\rho(\cG^{0}(\int_{0}^{\cdot}h_{\e}(s)\dif s),\cG^{0}(\int_{0}^{\cdot}h(s)\dif s))$. Then in order to prove $\lim\limits_{\e\rightarrow 0}\rho(\cG^{0}(\int_{0}^{\cdot}h_{\e}(s)\dif s),\cG^{0}(\int_{0}^{\cdot}h(s)\dif s))=0$, we establish a related conclusion. Finally, we show that $\lim\limits_{\e\rightarrow 0}\rho(\cG^{0}(\int_{0}^{\cdot}h_{\e}(s)\dif s),\cG^{0}(\int_{0}^{\cdot}h(s)\dif s))=0$.

{\bf Step 1.} We estimate $\rho(\cG^{0}(\int_{0}^{\cdot}h_{\e}(s)\dif s),\cG^{0}(\int_{0}^{\cdot}h(s)\dif s))$.

By the definition of $\cG^{0}$, $\cG^{0}(\int_{0}^{\cdot}h_{\e}(s)\dif s)$ and $\cG^{0}(\int_{0}^{\cdot}h(s)\dif s)$ satisfy the following equations respectively:
\ce
&&\bar{X}^{h_\e}_{t}=x_0-\bar{K}^{h_\e}_{t}+\int_{0}^{t}\bar{b}_1(\bar{X}^{h_\e}_{s}, D_{\bar{X}^{0}_{s}})\dif s+\int_{0}^{t}\sigma_1(\bar{X}^{h_\e}_{s},D_{\bar{X}^{0}_{s}})\pi_1h_{{\e}}(s)\dif s,\\
&&\bar{X}^{h}_{t}=x_0-\bar{K}^{h}_{t}+\int_{0}^{t}\bar{b}_1(\bar{X}^{h}_{s}, D_{\bar{X}^{0}_{s}})\dif s+\int_{0}^{t}\sigma_1(\bar{X}^{h}_{s},D_{\bar{X}^{0}_{s}})\pi_1h(s)\dif s.
\de
Set $Z^{0}(t)=\bar{X}^{h_{\e}}_{t}-\bar{X}^{h}_{t}$, and by Lemma \ref{equi} and (\ref{barb1lip}) we have
\be
|Z^{0}(t)|^{2}&=&-2\int_{0}^{t}\langle Z^{0}(s),\dif(\bar{K}^{h_{\e}}_{s}-\bar{K}^{h}_{s})\rangle+2\int_{0}^{t}\langle Z^{0}(s),\bar{b}_1(\bar{X}^{h_{\e}}_{s},D_{\bar{X}^{0}_{s}})-\bar{b}_1(\bar{X}^{h}_{s},D_{\bar{X}^{0}_{s}})\rangle\dif s  \no\\
&&+2\int_{0}^{t}\langle Z^{0}(s),\sigma_1(\bar{X}^{h_{\e}}_{s},D_{\bar{X}^0_{s}})\pi_1h_{{\e}}(s)-\sigma_1(\bar{X}^{h}_{s},D_{\bar{X}^0_{s}})\pi_1h(s)\rangle\dif s  \no\\
&\leq&2\int_{0}^{t}\langle Z^{0}(s),\bar{b}_1(\bar{X}^{h_{\e}}_{s},D_{\bar{X}^0_{s}})-\bar{b}_1(\bar{X}^{h}_{s},D_{\bar{X}^0_{s}})\rangle\dif s  \no\\
&&+2\int_{0}^{t}\langle Z^{0}(s),\sigma_1(\bar{X}^{h_{\e}}_{s},D_{\bar{X}^0_{s}})\pi_1h_{{\e}}(s)-\sigma_1(\bar{X}^{h}_{s},D_{\bar{X}^0_{s}})\pi_1h(s)\rangle\dif s  \no\\
&\leq&C\int_{0}^{t}|Z^{0}(s)|^2\dif s+2\int_{0}^{t}\langle Z^{0}(s),\(\sigma_1(\bar{X}^{h_{\e}}_{s},D_{\bar{X}^0_{s}})- \sigma_1(\bar{X}^{h}_{s},D_{\bar{X}^0_{s}})\)\pi_1h_{{\e}}(s)\rangle \dif s\no\\
&&+2\int_{0}^{t}\langle Z^{0}(s),\sigma_1(\bar{X}^{h}_{s},D_{\bar{X}^0_{s}})(\pi_1h_{{\e}}(s)-\pi_1h(s))\rangle \dif s\no\\
&=:&C\int_{0}^{t}|Z^{0}(s)|^2\dif s+J_1(t)+J_2(t).
\label{eq10}
\ee

Next, for $J_1(t)$, it follows from $(\mathbf{H}^{1'}_{b_{1}, \s_{1}})$  and $h_\e\in\mathbf{D}_2^{N}$ that
\be
\sup\limits_{s\in[0,t]}|J_1(s)|&\leq&2\sup\limits_{s\in[0,t]}\left|\int_{0}^{s}\langle Z^{0}(r),\(\sigma_1(\bar{X}^{h_\e}_{r},D_{\bar{X}^0_{r}})- \sigma_1(\bar{X}^{h}_{r},D_{\bar{X}^0_{r}})\)\pi_1h_{{\e}}(r)\rangle \dif r\right|\no\\
&\leq& 2\sqrt{L_{b_1,\s_1}}\int_{0}^{t}|Z^{0}(r)|^{2}|h_{{\e}}(r)|\dif r\no\\
&\leq& 2\sqrt{L_{b_1,\s_1}}\left(\int_{0}^{t}|Z^{0}(r)|^{4}\dif r\right)^{\frac{1}{2}}\left(\int_{0}^{t}|h_{{\e}}(r)|^{2}\dif r\right)^{\frac{1}{2}}\no\\
&\leq& 2\sqrt{L_{b_1,\s_1}N}\left(\int_{0}^{t}|Z^{0}(r)|^{4}\dif r\right)^{\frac{1}{2}}\no\\
&\leq& 2\sqrt{L_{b_1,\s_1}N}\left(\sup\limits_{r\in[0,t]}|Z^{0}(r)|\right)\left(\int_{0}^{t}|Z^{0}(r)|^{2}\dif r\right)^{\frac{1}{2}}\no\\
&\leq& \frac{1}{2}\sup\limits_{r\in[0,t]}|Z^{0}(r)|^{2}+C\int_{0}^{t}|Z^{0}(r)|^{2}\dif r.
\label{eq11}
\ee

Inserting (\ref{eq11}) into (\ref{eq10}), we obtain that
\ce
\sup\limits_{s\in[0,t]}|Z^{0}(t)|^{2}\leq \frac{1}{2}\sup\limits_{s\in[0,t]}|Z^{0}(s)|^{2}+C\int_{0}^{t}|Z^{0}(s)|^{2}\dif s+\sup\limits_{s\in[0,T]}|J_2(s)|,
\de
which together with Gronwall's inequality implies that
\be
\sup\limits_{t\in[0,T]}|\bar{X}^{h_{{\e}}}_{t}-\bar{X}^{h}_{t}|^2\leq \sup\limits_{s\in[0,T]}|J_2(s)|e^{CT}.
\label{xhexh}
\ee

{\bf Step 2.} For $h_\e, h\in\mathbf{D}_{2}^{N}$, set 
\ce
g_\e(t):=\int_0^t\sigma_1(\bar{X}^{h}_{r},D_{\bar{X}^0_{r}})(\pi_1h_{{\e}}(r)-\pi_1h(r))\dif r,
\de
and we prove that $g_\e(\cdot)$ tends to $0$ in $C([0,T],\mR^n)$. 

First of all, we justify that 

$(i)$ $\sup\limits_{\e\in (0,1)}\sup\limits_{t\in[0,T]}\left|g_\e(t)\right|<\infty$,

$(ii)$ $\{[0,T]\ni t\mapsto g_\e(t); \e\in (0,1)\}$ is equi-continuous.

For $0\leq s<t\leq T$, it holds that
\ce
|g_\e(s)-g_\e(t)|&\leq&\left|\int_{s}^{t}\sigma_1(\bar{X}^{h}_{r},D_{\bar{X}^0_{r}})(\pi_1h_{\e}(r)-\pi_1h(r))\dif r\right|\\
&\leq&\int_{s}^{t}\|\sigma_1(\bar{X}^{h}_{r},D_{\bar{X}^0_{r}})\||h_{\e}(r)-h(r)|\dif r\\
&\leq&\left(\int_{s}^{t}\|\sigma_1(\bar{X}^{h}_{r},D_{\bar{X}^0_{r}})\|^2\dif r\right)^{1/2}\left(\int_{s}^{t}|h_{\e}(r)-h(r)|^2\dif r\right)^{1/2}\\
&\leq&\left(\int_{s}^{t}\|\sigma_1(\bar{X}^{h}_{r},D_{\bar{X}^0_{r}})\|^2\dif r\right)^{1/2}\times 2N^{1/2}\\
&\leq&C\left(\int_{s}^{t}(1+|\bar{X}^{h}_{r}|^2+|\bar{X}^0_{r}|^2)\dif r\right)^{1/2}\times 2N^{1/2},
\de
where $({\bf H}_{b_1,\s_1}^1)$ is used. Letting $s=0$, by (\ref{barx0b}) and (\ref{barxub})  we have that
\ce
|g_\e(t)|\leq 2N^{1/2}C(1+|x_0|),
\de
where $C$ is independent of $\e$. So, $(i)$ holds.

For $(ii)$, noticing that 
\ce
|g_\e(s)-g_\e(t)|\leq 2N^{1/2}C(1+|x_0|)(t-s)^{1/2},
\de
we know that $(ii)$ holds.

Combining $(i)$ and $(ii)$, by the Ascoli-Arzel\'a lemma we obtain that $\{g_\e; \e\in (0,1)\}$ is relatively compact in $C([0,T],\mR^n)$.

Besides, note that 
\ce
\int_{0}^{t}\|\sigma_1(\bar{X}^{h}_{r},D_{\bar{X}^0_{r}})\|^2\dif r\leq C\int_{0}^{t}(1+|\bar{X}^{h}_{r}|^2+|\bar{X}^0_{r}|^2)\dif r<\infty. 
\de
Since $h_{\e}\rightarrow h$ in $\mathbf{D}_2^{N}$ as ${\e}\rightarrow0$, one get that for any $t\in[0,T]$
$$
\lim\limits_{{\e}\rightarrow0}g_\e(t)=0,
$$
which implies that
\ce
\lim\limits_{{\e}\rightarrow0}\sup\limits_{t\in[0,T]}\left|g_\e(t)\right|=0.
\de

{\bf Step 3.} We prove that $\lim\limits_{{\e}\rightarrow0}\sup\limits_{s\in[0,T]}|J_2(s)|=0$.

For $J_{2}(t)$, applying the Taylor formula to $\<Z^0(s),g_\e(s)\>$, we have that
\ce
\frac{1}{2}J_{2}(t)&=&\<Z^0(t),g_\e(t)\>+\int_0^t\<g_\e(s),\dif (\bar{K}^{h_{\epsilon}}_{s}-\bar{K}^{h}_{s})\>\\
&&-\int_0^t\<g_\e(s),\bar{b}_1(\bar{X}^{h_{\e}}_{s},D_{\bar{X}^{0}_{s}})-\bar{b}_1(\bar{X}^{h}_{s},D_{\bar{X}^{0}_{s}})\>\dif s\\
&&-\int_0^t\<g_\e(s),\sigma_1(\bar{X}^{h_{\e}}_{s},D_{\bar{X}^0_{s}})\pi_1h_{{\e}}(s)-\sigma_1(\bar{X}^{h}_{s},D_{\bar{X}^0_{s}})\pi_1h(s)\>\dif s\\
&=:&J_{21}(t)+J_{22}(t)+J_{23}(t)+J_{24}(t).
\de

For $J_{21}(t)$, note that 
\ce
\sup\limits_{t\in[0,T]}|J_{21}(t)|\leq\sup\limits_{t\in[0,T]}|Z^0(t)|\sup\limits_{t\in[0,T]}|g_\e(t)|.
\de
Thus, by (\ref{barx0b}), (\ref{barxub}) and the result in {\bf Step 2}, it holds that
\ce
\sup\limits_{t\in[0,T]}|J_{21}(t)|\rightarrow 0, \quad \e\rightarrow 0.
\de

For $J_{22}(t)$, noticing that
\ce
\sup\limits_{t\in[0,T]}|J_{22}(t)|\leq\sup\limits_{t\in[0,T]}|g_\e(t)|(|\bar{K}^{h_\e}|_0^T+|\bar{K}^{h}|_0^T),
\de
by (\ref{barkub}) and the result in {\bf Step 2} we obtain that $\sup\limits_{t\in[0,T]}|J_{22}(t)|\rightarrow 0$. 

By the same deduction to the above, one can get that
\ce
\lim\limits_{\e\rightarrow 0}\sup\limits_{t\in[0,T]}|J_{23}(t)|=0, \quad \lim\limits_{\e\rightarrow 0}\sup\limits_{t\in[0,T]}|J_{24}(t)|=0.
\de

Combining the above deduction, we get that 
\be
\lim\limits_{\e\rightarrow 0}\sup\limits_{t\in[0,T]}\left|J_{2}(t)\right|=0.
\label{j22}
\ee

{\bf Step 4.} We prove that $\lim\limits_{\e\rightarrow 0}\rho(\cG^{0}(\int_{0}^{\cdot}h_{\e}(s)\dif s),\cG^{0}(\int_{0}^{\cdot}h(s)\dif s))=0$.

By (\ref{xhexh}) and (\ref{j22}), it holds that 
\ce
\lim\limits_{\e\rightarrow 0}\rho(\cG^{0}(\int_{0}^{\cdot}h_{\e}(s)\dif s),\cG^{0}(\int_{0}^{\cdot}h(s)\dif s))=\lim\limits_{\e\rightarrow 0}\sup\limits_{t\in[0,T]}|\bar{X}^{h_{\e}}_{t}-\bar{X}^{h}_{t}|^2=0,
\de
which completes the proof. 
\end{proof}

\bl\label{auxilemm3}
Suppose that the assumptions of Theorem \ref{ldpmmsde} hold. Assume that $\{u_{\epsilon},\epsilon>0\}\subset \mathbf{A}_{2}^{N}$. Then for any $\eta>0$,
$$
\lim\limits_{\e\rightarrow0}\mP\left(\rho\left(\cG^{\e}\left(\sqrt{\e}W_\cdot+\int_{0}^{\cdot}u_{\e}(s)\dif s\right),\cG^{0}\left(\int_{0}^{\cdot}u_\e(s)\dif s\right)\right)>\eta\right)=0.
$$
\el
\begin{proof}
We divide the proof into two steps. In the first step, we estimate $\rho\(\cG^{\e}\(\sqrt{\e}W_\cdot\\+\int_{0}^{\cdot}u_{\e}(s)\dif s\),\cG^{0}\left(\int_{0}^{\cdot}u_\e(s)\dif s\right)\)$ in $\mS=C([0,T],\overline{\cD(A_1)})$. In the second step, we show the required result.

{\bf Step 1.} We estimate $\rho\left(\cG^{\e}\left(\sqrt{\e}W_\cdot+\int_{0}^{\cdot}u_{\e}(s)\dif s\right),\cG^{0}\left(\int_{0}^{\cdot}u_\e(s)\dif s\right)\right)$ in $\mS=C([0,T],\overline{\cD(A_1)})$.

Note that
$$
X^{\epsilon,\d,u_{\epsilon}}=\cG^{\epsilon}(\sqrt{\epsilon}W+\int_{0}^{\cdot}u_{\epsilon}(s)\dif s), \quad \bar{X}^{u_\e}=\cG^{0}(\int_{0}^{\cdot}u_\e(s)\dif s).
$$
Thus, set $Z^{\epsilon,u_{\epsilon}}(t)=X^{\epsilon,\d,u_{\epsilon}}_{t}-\bar{X}^{u_\e}_{t}$, and it holds that
\ce
Z^{\epsilon,u_{\epsilon}}(t)
&=&-(K^{1,\epsilon,\d,u_{\epsilon}}_{t}-\bar{K}^{u_\e}_{t})+\int_{0}^{t}\left[b_1(X^{\epsilon,\d,u_{\epsilon}}_{s},\sL_{X_s^{\epsilon,\d}},Y^{\epsilon,\d,u_{\epsilon}}_{s})-\bar{b}_1(\bar{X}^{u_\e}_{s},D_{\bar{X}^{0}_{s}})\right]\dif s\no\\
&&+\int_{0}^{t}\left[\s_1(X^{\epsilon,\d,u_{\epsilon}}_{s},\sL_{X_s^{\epsilon,\d}})\pi_1u_{\epsilon}(s)-\sigma_1(\bar{X}^{u_\e}_{s},D_{\bar{X}^{0}_{s}})\pi_1u_\e(s)\right]\dif s\\
&&+\sqrt{\epsilon}\int_{0}^{t}\s_1(X^{\epsilon,\d,u_{\epsilon}}_{s},\sL_{X_s^{\epsilon,\d}})\dif W^1_s.
\de
By It\^o's formula and Lemma \ref{equi}, we get that
\be
|Z^{\epsilon,u_{\epsilon}}(t)|^{2}
&=&-2\int_{0}^{t}\<Z^{\epsilon,u_{\epsilon}}(s),\dif (K^{1,\epsilon,\d,u_{\epsilon}}_{s}-\bar{K}^{u_\e}_{s})\>\no\\
&&+2\int_{0}^{t}\langle   Z^{\epsilon,u_{\epsilon}}(s), b_1(X^{\epsilon,\d,u_{\epsilon}}_{s},\sL_{X_s^{\epsilon,\d}},Y^{\epsilon,\d,u_{\epsilon}}_{s})-\bar{b}_1(\bar{X}^{u_\e}_{s},D_{\bar{X}^{0}_{s}})\rangle  \dif s   \no\\
&&+2\int_{0}^{t}\langle   Z^{\epsilon,u_{\epsilon}}(s), \s_1(X^{\epsilon,\d,u_{\epsilon}}_{s},\sL_{X_s^{\epsilon,\d}})\pi_1u_{\epsilon}(s)-\sigma_1(\bar{X}^{u_\e}_{s},D_{\bar{X}^{0}_{s}})\pi_1u_\e(s) \rangle  \dif s  \no\\
&&+2\sqrt{\epsilon} \int_{0}^{t}\langle  Z^{\epsilon,u_{\epsilon}}(s),\s_1(X^{\epsilon,\d,u_{\epsilon}}_{s},\sL_{X_s^{\epsilon,\d}})\dif W^1_s\rangle +\epsilon\int_{0}^{t}\|\s_1(X^{\epsilon,\d,u_{\epsilon}}_{s},\sL_{X_s^{\epsilon,\d}})\|^{2} \dif s\no\\
&\leq&2\int_{0}^{t}\langle   Z^{\epsilon,u_{\epsilon}}(s), b_1(X^{\epsilon,\d,u_{\epsilon}}_{s},\sL_{X_s^{\epsilon,\d}},Y^{\epsilon,\d,u_{\epsilon}}_{s})-\bar{b}_1(\bar{X}^{u_\e}_{s},D_{\bar{X}^{0}_{s}}) \rangle  \dif s   \no\\
&&+2\int_{0}^{t}\langle   Z^{\epsilon,u_{\epsilon}}(s), \s_1(X^{\epsilon,\d,u_{\epsilon}}_{s},\sL_{X_s^{\epsilon,\d}})\pi_1u_{\epsilon}(s)-\sigma_1(\bar{X}^{u_\e}_{s},D_{\bar{X}^{0}_{s}})\pi_1u_\e(s) \rangle  \dif s  \no\\
&&+2\sqrt{\epsilon} \int_{0}^{t}\langle  Z^{\epsilon,u_{\epsilon}}(s),  \s_1(X^{\epsilon,\d,u_{\epsilon}}_{s},\sL_{X_s^{\epsilon,\d}}) \dif W^1_s\rangle +\epsilon\int_{0}^{t}\|\s_1(X^{\epsilon,\d,u_{\epsilon}}_{s},\sL_{X_s^{\epsilon,\d}})\|^{2} \dif s\no\\
&=:&J_{1}(t)+J_{2}(t)+J_{3}(t)+J_{4}(t).
\label{j1j2j3j4}
\ee

For $J_{1}(t)$, note that
\ce
J_{1}(t)&=&2\int_{0}^{t}\langle   Z^{\epsilon,u_{\epsilon}}(s), b_1(X^{\epsilon,\d,u_{\epsilon}}_{s},\sL_{X_s^{\epsilon,\d}},Y^{\epsilon,\d,u_{\epsilon}}_{s})-b_1(X^{\epsilon,\d,u_{\epsilon}}_{s(\triangle)},\sL_{X_{s(\triangle)}^{\epsilon,\d}},\hat{Y}^{\epsilon,\d,u_{\epsilon}}_{s}) \rangle  \dif s\\
&&+2\int_{0}^{t}\langle   Z^{\epsilon,u_{\epsilon}}(s),-\bar{b}_1(X^{\epsilon,\d,u_{\epsilon}}_{s},\sL_{X_s^{\epsilon,\d}})+\bar{b}_1(X^{\epsilon,\d,u_{\epsilon}}_{s(\triangle)},\sL_{X_{s(\triangle)}^{\epsilon,\d}})\rangle \dif s\\
&&+2\int_{0}^{t}\langle   Z^{\epsilon,u_{\epsilon}}(s),\bar{b}_1(X^{\epsilon,\d,u_{\epsilon}}_{s},\sL_{X_{s}^{\epsilon,\d}})-\bar{b}_1(\bar{X}^{u_\e}_{s},D_{\bar{X}^{0}_{s}}) \rangle  \dif s\\
&&+2\int_{0}^{t}\langle   Z^{\epsilon,u_{\epsilon}}(s)-Z^{\epsilon,u_{\epsilon}}(s(\triangle)),b_1(X^{\epsilon,\d,u_{\epsilon}}_{s(\triangle)},\sL_{X_{s(\triangle)}^{\epsilon,\d}},\hat{Y}^{\epsilon,\d,u_{\epsilon}}_{s})-\bar{b}_1(X^{\epsilon,\d,u_{\epsilon}}_{s(\triangle)},\sL_{X_{s(\triangle)}^{\epsilon,\d}})\rangle \dif s\\
&&+2\int_{0}^{t}\langle   Z^{\epsilon,u_{\epsilon}}(s(\triangle)),b_1(X^{\epsilon,\d,u_{\epsilon}}_{s(\triangle)},\sL_{X_{s(\triangle)}^{\epsilon,\d}},\hat{Y}^{\epsilon,\d,u_\e}_{s})-\bar{b}_1(X^{\epsilon,\d,u_{\epsilon}}_{s(\triangle)},\sL_{X_{s(\triangle)}^{\epsilon,\d}})\rangle \dif s\\
&=:&J_{11}(t)+J_{12}(t)+J_{13}(t)+J_{14}(t)+J_{15}(t).
\de
So, by the H\"older inequality and the Lipschitz continuity of $b_1, \bar{b}_1$, we get that
\be
&&\mE\left(\sup\limits_{t\in[0,T]}|J_{11}(t)|\right)+\mE\left(\sup\limits_{t\in[0,T]}|J_{12}(t)|\right)+\mE\left(\sup\limits_{t\in[0,T]}|J_{13}(t)|\right)\no\\
&\leq& C\int_{0}^{T}\mE|Z^{\epsilon,u_{\epsilon}}(s)|^2\dif s+C\int_{0}^{T}\mE|X^{\epsilon,\d,u_{\epsilon}}_{s}-X^{\epsilon,\d,u_{\epsilon}}_{s(\triangle)}|^2\dif s\no\\
&&+C\int_{0}^{T}\mW_2^2(\sL_{X_s^{\epsilon,\d}},\sL_{X_{s(\triangle)}^{\epsilon,\d}})\dif s+C\int_{0}^{T}\mW_2^2(\sL_{X_s^{\epsilon,\d}},D_{\bar{X}^{0}_{s}})\dif s\no\\
&&+C\int_{0}^{T}\mE|Y^{\epsilon,\d,u_{\epsilon}}_{s}-\hat{Y}^{\epsilon,\d,u_{\epsilon}}_{s}|^2\dif s\no\\
&\leq& C\int_{0}^{T}\mE|Z^{\epsilon,u_{\epsilon}}(s)|^2\dif s+C\left(\sup\limits_{s\in[0,T]}\mE\sup _{s \leqslant r \leqslant s+\triangle}|X_{r}^{\e,\d,u_\e}-X_{s}^{\e,\d,u_\e}|^{2}\right)\no\\
&&+C\left(\sup\limits_{s\in[0,T]}\mE\sup _{s \leqslant r \leqslant s+\triangle}|X_{r}^{\e,\d}-X_{s}^{\e,\d}|^{2}\right)+CT\mE\left(\sup\limits_{s\in[0,T]}|X_s^{\epsilon,\d}-\bar{X}^{0}_{s}|^2\right)\no\\
&&+ CT\left(\frac{\d}{\e}\right).
\label{j111213}
\ee
And the H\"older inequality and the linear growth of $b_1, \bar{b}_1$ imply that
\be
\mE\left(\sup\limits_{t\in[0,T]}|J_{14}(t)|\right)
&\leq& C\left(\int_0^T(\mE|X^{\epsilon,\d,u_{\epsilon}}_{s}-X^{\epsilon,\d,u_{\epsilon}}_{s(\triangle)}|^2+\mE|\bar{X}^{u_\e}_{s}-\bar{X}^{u_\e}_{s(\triangle)}|^2)\dif s\right)^{1/2}\no\\
&&\times\left(\int_0^T(1+\mE|X^{\epsilon,\d,u_{\epsilon}}_{s(\triangle)}|^2+\|\sL_{X_{s(\triangle)}^{\epsilon,\d}}\|^2+\mE|\hat{Y}^{\epsilon,\d,u_{\epsilon}}_{s}|^2)\dif s\right)^{1/2}\no\\
&\leq& C\Bigg(\left(\sup\limits_{s\in[0,T]}\mE\sup _{s \leqslant r \leqslant s+\triangle}|X_{r}^{\e,\d,u_\e}-X_{s}^{\e,\d,u_\e}|^{2}\right)\no\\
&&+\left(\sup\limits_{s\in[0,T]}\mE\sup _{s \leqslant r \leqslant s+\triangle}|\bar{X}^{u_\e}_r-\bar{X}^{u_\e}_s|^{2}\right)\Bigg)^{1/2}.
\label{j14}
\ee
Finally, by the similar deduction to that for (\ref{i3}), we obtain that
\be
\mE\left(\sup\limits_{t\in[0,T]}|J_{15}(t)|\right)\leq C\((\frac{\d}{\triangle})^{1/2}+\triangle^{1/2}\).
\label{j15}
\ee


For $J_{2}$, by $(\mathbf{H}^1_{b_{1}, \s_{1}})$ and the H\"older inequality, it holds that
\be
\mE\sup\limits_{t\in[0,T]}|J_{2}(t)|&\leq& 2\sqrt{L_{b_1,\s_1}}\mE\int_{0}^{T}|Z^{\epsilon,u_{\epsilon}}(s)|\(|Z^{\epsilon,u_{\epsilon}}(s)|+\mW_2(\sL_{X_s^{\epsilon,\d}},D_{\bar{X}^{0}_{s}})\)|u_{\epsilon}(s)|\dif s\no\\
&\leq&C\mE\left(\int_{0}^{T}\(|Z^{\epsilon,u_{\epsilon}}(s)|+\mW_2(\sL_{X_s^{\epsilon,\d}},D_{\bar{X}^{0}_{s}})\)|u_{\epsilon}(s)|\dif s\right)^2\no\\
&&+\frac{1}{4}\mE\left[\sup\limits_{t\in[0,T]}|Z^{\epsilon,u_{\epsilon}}(t)|^{2}\right]\no\\
&\leq&C\mE\left(\int_{0}^{T}\(|Z^{\epsilon,u_{\epsilon}}(s)|+\mW_2(\sL_{X_s^{\epsilon,\d}},D_{\bar{X}^{0}_{s}})\)^2\dif s\right)\left(\int_{0}^{T}|u_{\epsilon}(s)|^{2}\dif s\right)\no\\
&&+\frac{1}{4}\mE\left[\sup\limits_{t\in[0,T]}|Z^{\epsilon,u_{\epsilon}}(t)|^{2}\right]\no\\
&\leq&C\mE\left[\int_{0}^{T}|Z^{\epsilon,u_{\epsilon}}(s)|^{2}\dif s\right]+C\mE\left(\sup\limits_{t\in[0,T]}|X_t^{\epsilon,\d}-\bar{X}^{0}_{t}|^2\right)\no\\
&&+\frac{1}{4}\mE\left[\sup\limits_{t\in[0,T]}|Z^{\epsilon,u_{\epsilon}}(t)|^{2}\right].
\label{j21}
\ee

For $J_{3}(t)$, from the Burkholder-Davis-Gundy inequality and the linear growth of $\s_1$, it follows that
\be
\mE\left(\sup\limits_{t\in[0,T]}|J_{3}(t)|\right)&\leq& 2\sqrt{\epsilon}C\mE\left(\int_{0}^{T}|Z^{\epsilon,u_{\epsilon}}(s)|^2\|\s_1(X^{\epsilon,\d,u_{\epsilon}}_{s},\sL_{X_s^{\epsilon,\d}}) \|^2 \dif s\right)^{1/2}\no\\
&\leq&\frac{1}{4}\mE\left[\sup\limits_{t\in[0,T]}|Z^{\epsilon,u_{\epsilon}}(t)|^{2}\right]+\sqrt{\epsilon}C\mE\int_{0}^{T} \|\s_1(X^{\epsilon,\d,u_{\epsilon}}_{s},\sL_{X_s^{\epsilon,\d}}) \|^2 \dif s\no\\
&\leq&\frac{1}{4}\mE\left[\sup\limits_{t\in[0,T]}|Z^{\epsilon,u_{\epsilon}}(t)|^{2}\right]+\sqrt{\epsilon}C\int_{0}^{T}(1+\mE|X^{\epsilon,\d,u_{\epsilon}}_{s}|^2+\mE|X^{\epsilon,\d}_{s}|^2)\dif s\no\\
&\leq&\frac{1}{4}\mE\left[\sup\limits_{t\in[0,T]}|Z^{\epsilon,u_{\epsilon}}(t)|^{2}\right]+\sqrt{\epsilon}C.
\label{j3}
\ee
For $J_{4}(t)$, by the linear growth of $\s_1$, we know
\be
\mE\left(\sup\limits_{t\in[0,T]}|J_{4}(t)|\right)\leq \e C\int_{0}^{T}(1+\mE|X^{\epsilon,\d,u_{\epsilon}}_{s}|^2+\mE|X^{\epsilon,\d}_{s}|^2)\dif s \leq\epsilon C.
\label{j4}
\ee

Combining (\ref{j111213})-(\ref{j4}) with (\ref{j1j2j3j4}), we can get
\ce
&&\mE\left(\sup\limits_{t\in[0,T]}|Z^{\epsilon,u_{\epsilon}}(t)|^{2}\right)\\
&\leq& C\int_{0}^{T}\mE\sup\limits_{r\in[0,s]}|Z^{\epsilon,u_{\epsilon}}(r)|^2\dif s+C\left(\sup\limits_{s\in[0,T]}\mE\sup _{s \leqslant r \leqslant s+\triangle}|X_{r}^{\e,\d,u_\e}-X_{s}^{\e,\d,u_\e}|^{2}\right)\no\\
&&+C\left(\sup\limits_{s\in[0,T]}\mE\sup _{s \leqslant r \leqslant s+\triangle}|X_{r}^{\e,\d}-X_{s}^{\e,\d}|^{2}\right)+CT\mE\left(\sup\limits_{s\in[0,T]}|X_s^{\epsilon,\d}-\bar{X}^{0}_{s}|^2\right)\no\\
&&+ CT\left(\frac{\d}{\e}\right)+C\Bigg(\left(\sup\limits_{s\in[0,T]}\mE\sup _{s \leqslant r \leqslant s+\triangle}|X_{r}^{\e,\d,u_\e}-X_{s}^{\e,\d,u_\e}|^{2}\right)\no\\
&&+\left(\sup\limits_{s\in[0,T]}\mE\sup _{s \leqslant r \leqslant s+\triangle}|\bar{X}^{u_\e}_r-\bar{X}^{u_\e}_s|^{2}\right)\Bigg)^{1/2}+C\((\frac{\d}{\triangle})^{1/2}+\triangle^{1/2}\)+C(\sqrt{\epsilon}+\epsilon),
\de
which together with the Gronwall inequality implies that
\be
\mE\left(\sup\limits_{t\in[0,T]}|Z^{\epsilon,u_{\epsilon}}(t)|^{2}\right)\leq C\bigg[\Sigma(\e)+\frac{\d}{\e}+(\frac{\d}{\triangle})^{1/2}+\triangle^{1/2}+\sqrt{\epsilon}+\epsilon\bigg],
\label{zeue}
\ee
where
\ce
\Sigma(\e)&:=&\left(\sup\limits_{s\in[0,T]}\mE\sup _{s \leqslant r \leqslant s+\triangle}|X_{r}^{\e,\d,u_\e}-X_{s}^{\e,\d,u_\e}|^{2}\right)+\left(\sup\limits_{s\in[0,T]}\mE\sup _{s \leqslant r \leqslant s+\triangle}|X_{r}^{\e,\d}-X_{s}^{\e,\d}|^{2}\right)\\
&&+\Bigg(\left(\sup\limits_{s\in[0,T]}\mE\sup _{s \leqslant r \leqslant s+\triangle}|X_{r}^{\e,\d,u_\e}-X_{s}^{\e,\d,u_\e}|^{2}\right)+\left(\sup\limits_{s\in[0,T]}\mE\sup _{s \leqslant r \leqslant s+\triangle}|\bar{X}^{u_\e}_r-\bar{X}^{u_\e}_s|^{2}\right)\Bigg)^{1/2}\\
&&+\mE\left(\sup\limits_{s\in[0,T]}|X_s^{\epsilon,\d}-\bar{X}^{0}_{s}|^2\right).
\de

{\bf Step 2.} We prove that for any $\eta>0$,
$$
\lim\limits_{\e\rightarrow0}\mP\left(\rho\left(\cG^{\e}\left(\sqrt{\e}W_\cdot+\int_{0}^{\cdot}u_{\e}(s)\dif s\right),\cG^{0}\left(\int_{0}^{\cdot}u_\e(s)\dif s\right)\right)>\eta\right)=0.
$$

By the Chebyshev inequality, it holds that
\ce
&&\mP\left(\rho\left(\cG^{\e}\left(\sqrt{\e}W_\cdot+\int_{0}^{\cdot}u_{\e}(s)\dif s\right),\cG^{0}\left(\int_{0}^{\cdot}u_\e(s)\dif s\right)\right)>\eta\right)\\
&=&\mP\left(\sup\limits_{t\in[0,T]}|Z^{\epsilon,u_{\epsilon}}(t)|>\eta\right)\leq\frac{1}{\eta^2}\mE\left(\sup\limits_{t\in[0,T]}|Z^{\epsilon,u_{\epsilon}}(t)|^{2}\right)\\
&\overset{(\ref{zeue})}{\leq}&C\frac{1}{\eta^2}\bigg[\Sigma(\e)+\frac{\d}{\e}+(\frac{\d}{\triangle})^{1/2}+\triangle^{1/2}+\sqrt{\e}+\e\bigg].
\de
Since $\lim\limits_{\e\rightarrow0}\d/\e=0$, $\d\rightarrow 0$ as $\e$ tends to $0$. Then we take $\triangle=\d^\g, 0<\g<1$ and have that $\triangle\rightarrow 0, \d/\triangle\rightarrow 0$, when $\e$ approximates to $0$. Hence, as $\e\rightarrow 0$, by (\ref{xegts}) (\ref{xegutse}) (\ref{barxuts}) and Theorem \ref{xbarxp}, it holds that 
$$
\lim\limits_{\e\rightarrow0}\mP\left(\rho\left(\cG^{\e}\left(\sqrt{\e}W_\cdot+\int_{0}^{\cdot}u_{\e}(s)\dif s\right),\cG^{0}\left(\int_{0}^{\cdot}u_\e(s)\dif s\right)\right)>\eta\right)=0.
$$
The proof is complete.
\end{proof}

Now, it is the position to prove Theorem \ref{ldpmmsde}.

{\bf Proof of Theorem \ref{ldpmmsde}.}

By Theorem \ref{ldpbase}, to establish LDP, it is sufficient to verify the two conditions in Condition \ref{cond}.  
In Lemma \ref{auxilemm2} and  \ref{auxilemm3}, we have already proved Condition \ref{cond} $(i)$ and $(ii)$, respectively. Then the proof is complete.

\section{An example}\label{exam}

In this section, we explain our results by an example.

\bx
Consider the following slow-fast system of Aggregation-Diffusions equations on $\mR^n\times\mR^n$:
\be\left\{\begin{array}{l}
\dif X_{t}^{\d}\in -\partial I_{\mathcal{O}}(X_{t}^{\d})\dif t-\left[\nabla V_1(Y_{t}^{\d})+\nabla V_2\ast\sL_{X_{t}^{\d}}(X_{t}^{\d})\right]\dif t+\s_{1}\dif W^1_{t},\\
X_{0}^{\d}=\xi\in\overline{\cD(\partial I_{\mathcal{O}})},\quad  0\leq t\leq T,\\
\dif Y_{t}^{\d}\in -A_2(Y_{t}^{\d})\dif t-\frac{1}{\d}\left[\nabla V_3(Y_{t}^{\d})+\nabla V_4\ast\sL_{X_{t}^{\d}}(X_{t}^{\d})\right]\dif t+\frac{1}{\sqrt{\d}}\s_{2}\dif W^2_{t},\\
Y_{0}^{\d}=y_0\in\overline{\cD(A_2)},\quad  0\leq t\leq T,
\end{array}
\right.
\label{Eqex}
\ee
where $\cO$ is a closed and convex domain in $\mR^n$ with ${\rm Int}(\cO)\neq\emptyset$, $V_i\in C^1(\mR^n)$ for $i=1,2,3,4$, $\ast$ denotes the convolution, $\s_1, \s_2$ are $n\times d_1, n\times d_2$ constant matrixes, respectively, and the rest of the setup is as in the system (\ref{Eqap}) with $n=m$.

Suppose that the derivative $\nabla V_1$ is Lipschitz continuous, the derivatives $\nabla V_2, \nabla V_4$ are bounded and Lipschitz continuous, and there exists a constant $\b>0$ such that for $y_1, y_2\in\mR^n$,
$$
\<y_1-y_2, \nabla V_3(y_1)-\nabla V_3(y_2)\leq -\b|y_1-y_2|^2.
$$
Then by Theorem \ref{xbarxapcr}, we know that for $0<\g<1$
\ce
\mE\(\sup_{0\leq t\leq T}|X_{t}^{\d}-\bar{X}_{t}|^{2}\)\leq C(\d^{\g/2}+\d^\g+\d^{\frac{1}{2}(1-\g)}),
\de
where $(\bar{X},\bar{K})$ solves the following equation:
\ce\left\{\begin{array}{l}
\dif \bar{X}_{t}\in -\partial I_{\mathcal{O}}(\bar{X}_{t})\dif t-\left[\int_{\cD(A_2)}\nabla V_1(y)\nu^{x,\mu}(\dif y)+\nabla V_2\ast\sL_{\bar{X}_{t}}(\bar{X}_{t})\right]\dif t+\s_{1}\dif W^1_{t},\\
\bar{X}_{0}=\xi\in\overline{\cD(\partial I_{\mathcal{O}})},
\end{array}
\right.
\de
and $\nu^{x,\mu}$ is the unique invariant probability measure of the following equation
\ce\left\{\begin{array}{l}
\dif Y_{t}^{x,\mu,y_0}\in -A_2(Y_{t}^{x,\mu,y_0})\dif t-\left[\nabla V_3(Y_{t}^{x,\mu,y_0})+\nabla V_4\ast\mu(x)\right]\dif t+\s_{2}\dif W^2_{t},\\
Y_{0}^{x,\mu,y_0}=y_0\in\overline{\cD(A_2)}.
\end{array}
\right.
\de
Here we mention that if the system (\ref{Eqex}) doesn't have $\partial I_{\mathcal{O}}, A_2$, Bezemek and Spiliopoulos \cite{bs2} obtained the order $1$ of weak convergence under stronger assumptions. 

Next, we consider the LDP for the system (\ref{Eqex}). That is, for the following system
\be\left\{\begin{array}{l}
\dif X_{t}^{\e,\d}\in -\partial I_{\mathcal{O}}(X_{t}^{\e,\d})\dif t-\left[\nabla V_1(Y_{t}^{\e,\d})+\nabla V_2\ast\sL_{X_{t}^{\e,\d}}(X_{t}^{\e,\d})\right]\dif t+\sqrt{\e}\s_{1}\dif W^1_{t},\\
X_{0}^{\e,\d}=x_0\in\overline{\cD(\partial I_{\mathcal{O}})},\quad  0\leq t\leq T,\\
\dif Y_{t}^{\e,\d}\in -A_2(Y_{t}^{\e,\d})\dif t-\frac{1}{\d}\left[\nabla V_3(Y_{t}^{\e,\d})+\nabla V_4\ast\sL_{X_{t}^{\e,\d}}(X_{t}^{\e,\d})\right]\dif t+\frac{1}{\sqrt{\d}}\s_{2}\dif W^2_{t},\\
Y_{0}^{\e,\d}=y_0\in\overline{\cD(A_2)},\quad  0\leq t\leq T,
\end{array}
\right.
\label{Eqexldp}
\ee
under the above assumptions, Theorem \ref{ldpmmsde} implies that when $\lim\limits_{\e\rightarrow 0}\frac{\d}{\e}=0$, the family $\{X^{\epsilon,\d},\epsilon\in(0,1)\}$ satisfies the LDP in $\mS:=C([0,T],\overline{\mathcal{D}(\partial I_{\mathcal{O}})})$ with the rate function given by
$$
I(\varsigma)=\frac{1}{2} \inf\limits_{h\in {\bf D}_{\varsigma}: \varsigma=\bar{X}^{h}}\|h\|_{\mH}^2,
$$
where $(\bar{X}^{0},\bar{K}^{0})$ solves the following equation
\ce\left\{\begin{array}{l}
\dif \bar{X}^0_{t}\in -\partial I_{\mathcal{O}}(\bar{X}^0_{t})\dif t-\left[\int_{\cD(A_2)}\nabla V_1(y)\nu^{x,\mu}(\dif y)+\nabla V_2(0)\right]\dif t,\\
\bar{X}^0_{0}=x_0\in\overline{\cD(\partial I_{\mathcal{O}})},
\end{array}
\right.
\de
and $(\bar{X}^{h},\bar{K}^{h})$ solves the following equation
\ce\left\{\begin{array}{l}
\dif \bar{X}^h_{t}\in -A_1(\bar{X}^h_{t})\dif t-\left[\int_{\cD(A_2)}\nabla V_1(y)\nu^{x,\mu}(\dif y)+\nabla V_2(\bar{X}^h_{t}-\bar{X}^0_{t})\right]\dif t+\s_{1}\pi_1h(t)\dif t,\\
\bar{X}^h_{0}=x_0\in\overline{\cD(\partial I_{\mathcal{O}})}.
\end{array}
\right.
\de
Note that if the slow part of the system (\ref{Eqexldp}) doesn't depend on the fast part, i.e.
\ce
\left\{\begin{array}{l}
\dif X_{t}^{\e}\in -\partial I_{\mathcal{O}}(X_{t}^{\e})\dif t-\nabla V_2\ast\sL_{X_{t}^{\e}}(X_{t}^{\e})\dif t+\sqrt{\e}\s_{1}\dif W^1_{t},\\
X_{0}^{\e}=x_0\in\overline{\cD(\partial I_{\mathcal{O}})},\quad  0\leq t\leq T,
\end{array}
\right.
\de
the above equation falls into the class of equations in \cite{arrst}. There Adams et al. also studied the LDP under some similar assumptions.
\ex


\begin{thebibliography}{999}
\bibitem{abks} S. R. Athreya, V. S. Borkar, K. S. Kumar and R. Sundaresan: Simultaneous Small Noise Limit for Singularly Perturbed Slow-Fast Coupled Diffusions, {\it Applied Mathematics and Optimization}, 83(2021)2327-2374.

\bibitem{arrst} D. Adams, G. Dos Reis, R. Ravaille, W. Salkekd and J. Tugaut: Large deviations and exit-times for reflected McKean-Vlasov equations with self-stabilising terms and superlinear drifts, {\it Stoch. Proc. Appl.}, 146(2022)264-310.

\bibitem{ahlw} D. F. Anderson, D. J. Higham, S. C. Leite, R. J. Williams: On constrained Langevin equations and (bio)chemical reaction networks, {\it Multiscale Model. Simul.}, 17 (1) (2019)1-30.

\bibitem{bs1} Z. W. Bezemek and K. Spiliopoulos: Large deviations for interacting multiscale particle systems, {\it Stochastic Processes and their Applications,} 155(2023)27-108.

\bibitem{bs2} Z. W. Bezemek and K. Spiliopoulos: Rate of homogenization for fully-coupled McKean-Vlasov SDEs, {\it Stochastics and Dynamics,} 23(2023)2350013.

\bibitem{bs3} Z. W. Bezemek and K. Spiliopoulos: Moderate deviations for fully coupled multiscale weakly interacting particle systems, https://arxiv.org/abs/2202.08403.

\bibitem{BDM2}
\newblock A.~{Budhiraja}, P.~{Dupuis}, and V.~{Maroulas}:
\newblock {Variational representations for continuous time processes.}
\newblock {\em {Ann. Inst. Henri Poincar\'e, Probab. Stat.}}, 47(2011), 725-747.

\bibitem{cepa1} E. C\'epa: \'Equations diff\'erentielles stochastiques multivoques, in: S\'em. Prob. XXIX, in: Lecture Notes in Math., 1995, pp. 86-107.

\bibitem{cepa2} E. C\'epa: Probleme de Skorohod Multivoque, {\it Ann. Prob.}, 26(1998), 500-532.

\bibitem{cw} Z.-Q. Chen and J. Wu: Averaging principle for stochastic variational inequalities with application to PDEs with nonlinear Neumann conditions, {\it Journal of Differential Equations}, 328(2022)157-201.

\bibitem{de} P. Dupuis and R. Ellis: {\it A Weak Convergence Approach to the Theory of Large Deviations}, Wiley, New York, 1997.

\bibitem{ds} P. Dupuis and K. Spiliopoulos: Large deviations for multiscale diffusion via weak convergence methods, {\it Stoch. Process. Appl.}, 122(2012)1947-1987.

\bibitem{flqz} K. Fang, W. Liu, H. Qiao and F. Zhu: Asymptotic behaviors of small perturbation for multivalued McKean-Vlasov stochastic differential equations, {\it Applied Mathematics and Optimization}, 88(2023)22.

\bibitem{ghl} J. Gao, W. Hong and W. Liu: Small noise asymptotics of multi-scale McKean-Vlasov stochastic dynamical systems, {\it Journal of Differential Equations}, 364(2023)521-575.

\bibitem{gq} J. Gong  and H. Qiao: The stability for multivalued McKean-Vlasov SDEs with non-Lipschitz coefficients, https://arxiv.org/abs/2106.12080.

\bibitem{hlls} W. Hong, S. Li, W. Liu, X. Sun: Central limit type theorem and large deviations for multi-scale McKean-Vlasov SDEs, https://arxiv.org/abs/2112.08203.

\bibitem{lw} S. C. Leite, R. J. Williams: A constrained langevin approximation for chemical reaction networks, {\it Ann. Appl. Probab.}, 29 (3) (2019) 1541-1608.

\bibitem{rl} R. Liptser: Large deviations for two scaled diffusions, {\it Probab. Theory Relat. Fields}, 106(1996)71-104.

\bibitem{lwx} Y. Li, F. Wu and L. Xie: Poisson equation on Wasserstein space and diffusion approximations for McKean-Vlasov equation, http://arxiv.org/abs/2203.12796.

\bibitem{msz} A. Matoussi, W. Sabbagh and T. Zhang: Large deviation principles of obstacle problems for quasilinear stochastic PDEs, {\it Appl. Math. Optim.}, 83 (2021)849-879.

\bibitem{kp} R. Kumar and L. Popovic: Large deviations for multi-scale jump-diffusion processes, {\it Stoch. Process. Appl.}, 127(2017)1297-1320.

\bibitem{ku} H. J. Kushner: Large deviations for two-time-scale diffusions with delays, {\it Applied Mathematics and Optimization}, 62(2010)295-322.

\bibitem{aap} A. A. Puhalskii: On large deviations of coupled diffusions with time scale separation, {\it Ann. Probab.}, 44(2016)3111-3186.

\bibitem{q1} H. Qiao: Asymptotic behaviors of multiscale multivalued stochastic systems with small noises, http://arxiv.org/abs/2306.06922.

\bibitem{q2} H. Qiao: Limit theorems of invariant measures for multivalued McKean-Vlasov stochastic differential equations, {\it Journal of Mathematical Analysis and Applications}, 528(2023)127532.

\bibitem{qg} H. Qiao and  J. Gong: Backward multivalued McKean-Vlasov SDEs and associated variational inequalities, {\it Discrete and Continuous Dynamical Systems-S}, 16(2023)819-845.

\bibitem{qw1} H. Qiao and W. Wei: Strong approximation of nonlinear filtering for multiscale McKean-Vlasov stochastic systems, http://arxiv.org/abs/2206.05037.

\bibitem{qw2} H. Qiao and W. Wei: Weak approximation of nonlinear filtering for multiscale McKean-Vlasov stochastic systems, http://arxiv.org/abs/2212.00240.

\bibitem{RW} Y. Ren and J. Wang: Large deviation for mean-field stochastic differential equations with subdifferential operator, {\it Stoch. Ana. Appl.} 34 (2016), 318-338.

\bibitem{rwz} J. Ren, J. Wu and H. Zhang: General large deviations and functional iterated logarithm law for multivalued stochastic differential equations, {\it J. Theor. Probab.},  28(2015)550-586.

\bibitem{rwzx} J. Ren, J. Wu and X. Zhang: Exponential ergodicity of non-Lipschitz multivalued stochastic differential equations, {\it Bull. Sci. Math}, 134(2010)391-404.

\bibitem{rxz} J. Ren, S. Xu and X. Zhang: Large deviations for multivalued stochastic differential equations, {\it J. Theor. Probab} 23 (2010), 1142-1156.

\bibitem{rsx} M. R\"{o}ckner, X. Sun and Y. Xie: Strong convergence order for slow-fast McKean-Vlasov stochastic differential equations, {\it Annales de I'I.H.P.Probabilit\'{e}s et statistiques}, 57(2021)547-576.

\bibitem{ks1} K. Spiliopoulos: Fluctuation analysis and short time asymptotics for multiple scales diffusion processes, {\it Stochastic and Dynamics}, 14(2014)1350026.

\bibitem{ks2} K. Spiliopoulos: Large deviations and importance sampling for systems of slow-fast motion, {\it Appl. Math. Optim.}, 67(2013)123-161.

\bibitem{av1} A. Y. Verernnikov: On an averaging principle for systems of stochastic differential equations, {\it Mat. Sb.}, 181(2)(1990)256-268 (in Russian); translation in: {\it Math. USSR Sb.}, 69(1)(1991)271-284.

\bibitem{av2} A.Y. Veretennikov, On large deviations in the averaging principle for SDEs with a full dependence, correction, {\it arXiv:} math/0502098v1 [math.PR] (2005). initial article in {\it Annals of Probability}, 27(1999)284-296.

\bibitem{av3} A. Y. Verernnikov: On large deviations for SDEs with small diffusion and averaging, {\it Stoch. Process. Appl.}, 89(2000)69-79.

\bibitem{XLLM} J. Xu, J. Liu, J. Liu and Y. Miao: Strong averaging principle for two-time-scale stochastic McKean-Vlasov equations, {\it Applied Mathematics and Optimization}, 84(2021)837-867.

\bibitem{zh} H. Zhang: Moderate deviation principle for multivalued stochastic differential equations. {\it Stochastic and Dynamics}, 20(2020)1-30.

\bibitem{ZXCH} X. Zhang: Skorohod problem and multivalued stochastic evolution equations in Banach spaces. {\it Bull. Sci. Math}, 131(2007)175-217.

\end{thebibliography}

\end{document}
