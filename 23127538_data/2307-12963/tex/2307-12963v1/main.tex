\NeedsTeXFormat{LaTeX2e}


\documentclass[12pt]{amsart}
%%%%%%%%%%%%%%%%%%%%%%%%%%%%%%%%%%%%%%%%%%%%%%%%%%%%%%%%%%%%%%%%%%%%%%%%%%%%%%%%%%%%%%%%%%%%%%%%%%%%%%%%%%%%%%%%%%%%%%%%%%%%
\usepackage{cases}
\usepackage{amsthm}
\usepackage{amsmath}
\usepackage{amscd}
\usepackage{graphicx}
\usepackage{float}
\usepackage[mathscr]{eucal}
\usepackage[colorlinks,linkcolor=blue,citecolor=blue, pdfstartview=FitH]{hyperref}


\setcounter{MaxMatrixCols}{10}
%TCIDATA{OutputFilter=LATEX.DLL}
%TCIDATA{Version=4.10.0.2345}
%TCIDATA{LastRevised=Thursday, June 25, 2020 23:07:33}
%TCIDATA{<META NAME="GraphicsSave" CONTENT="32">}

\input xy
\xyoption{all} \numberwithin{equation}{section}
\setlength{\headheight}{8pt} \setlength{\textheight}{22.5cm}
\setlength{\textwidth}{16cm} \setlength{\oddsidemargin}{0cm}
\setlength{\evensidemargin}{0cm} \setlength{\topmargin}{0cm}



\begin{document}

\title[On the asymptotic expansions I]{On the asymptotic expansions of various quantum invariants I: the colored Jones polynomial of twist knots at the root of unity $e^{\frac{2\pi\sqrt{-1}}{N+\frac{1}{2}}}$}


\author[Qingtao Chen and Shengmao Zhu]{Qingtao Chen and
Shengmao Zhu}

\address{Division of Science \\
New York University Abu Dhabi \\
Abu Dhabi \\
United Arab Emirates}
\email{chenqtao@nyu.edu}, \email{chenqtao@hotmail.com}



\address{Department of Mathematics \\
Zhejiang Normal University,  \\
Jinhua Zhejiang,  321004, China }
\email{szhu@zju.edu.cn}

\begin{abstract}
This is the first article in a series devoted to the study of the asymptotic expansions of various quantum invariants related to the twist knots. In this paper,  by using the saddle point method developed by Ohtsuki, we obtain an asymptotic expansion formula for the colored Jones polynomial of twist knots $\mathcal{K}_p$ with $p\geq 6$ at the root of unity $e^{\frac{2\pi\sqrt{-1}}{N+\frac{1}{2}}}$. 
\end{abstract}

\maketitle


\theoremstyle{plain} \newtheorem{thm}{Theorem}[section] \newtheorem{theorem}[%
thm]{Theorem} \newtheorem{lemma}[thm]{Lemma} \newtheorem{corollary}[thm]{%
Corollary} \newtheorem{proposition}[thm]{Proposition} \newtheorem{conjecture}%
[thm]{Conjecture} \theoremstyle{definition}
\newtheorem{remark}[thm]{Remark}
\newtheorem{remarks}[thm]{Remarks} \newtheorem{definition}[thm]{Definition}
\newtheorem{example}[thm]{Example}

%mathfrak

%\newcommand{\bS}{{\mathbb S}}
%%%%%%%%%%%%%%%%%%%%%%%%%%%%%%%%%%%%%%%%%%%%%%%%%%%%%%%%%%%%%%

%%%%%%%%%%%%%%%%%%%%%%%%%%%%%%%%%%%%%%%%%%%%%%%%%%%%%%%%%%%%%%%%%%%%%%%%%%%%%%%%%%%%%%%%%%%%%%%%
%Abstract                                            %
%%%%%%%%%%%%%%%%%%%%%%%%%%%%%%%%%%%%%%%%%%%%%%%%%%%%%%%%%%%%%%%%%%%%%%%%%%%%%%%%%%%%%%%%%%%%%%%%

%\setcounter{tocdepth}{5} \setcounter{page}{1}

\tableofcontents
\newpage

\section{Introduction}
%\subsection{History of the volume conjecture}
In this series of articles, we study the asymptotic expansions of various quantum invariants at different roots of unit and make a connection between them. 

This work is motivated by the volume conjectures, let us briefly review the background.   In \cite{Kash95}, by using the quantum dilogarithm function, R. Kashaev defined a link invariant $\langle \mathcal{L} \rangle_{N}$  for a link $\mathcal{L}$, which depends on a positive integer $N$. Furthermore, in \cite{Kash97}, he conjectured that for any hyperbolic link $\mathcal{L}$, the asymptotics at $N\rightarrow \infty$ of $|\langle \mathcal{L}\rangle_N|$ gives its volume, i.e. 
 \begin{align}
     2\pi \lim_{N\rightarrow \infty}\frac{\log |\langle\mathcal{L}\rangle_N|}{N}=vol(S^3\setminus \mathcal{L})
 \end{align}
where $vol(S^3\setminus \mathcal{L})$ denotes the hyperbolic volume of the complement of $\mathcal{L}$ in $S^3$, and gave evidence for the conjecture. In  \cite{MuMu01}, H. Murakami and J. Murakami proved that for any link $\mathcal{L}$, Kashaev's invariant 
$\langle \mathcal{L}\rangle_N$ is equal to $N$-th normalized colored Jones polynomial  evaluated at the root of unity $e^{\frac{2\pi \sqrt{-1}}{N}}$, which is written as $J_{N}(\mathcal{L};e^{\frac{2\pi \sqrt{-1}}{N}})$, and they extended Kashaev's conjecture as follows 
\begin{align} \label{formula-original-volume}
    2\pi \lim_{N\rightarrow \infty} \frac{\log |J_N(\mathcal{L}; e^{\frac{2\pi \sqrt{-1}}{N}})|}{N}=vol(S^3 \setminus \mathcal{L}),
\end{align}
where $vol(S^3\setminus \mathcal{L})$ denotes the simplicial volume of the complement of $\mathcal{L}$ in $S^3$. This is usally called the (Kashaev-Murakami-Murakami) volume conjecture. Furthermore, as a complexification of the volume conjecture, it is conjectured in \cite{MMOTY02} that, for a hyperbolic link $\mathcal{L}$, 
    \begin{align}
    2\pi \lim_{N\rightarrow \infty} \frac{\log J_N(\mathcal{L};e^{\frac{2\pi \sqrt{-1}}{N}})}{N}=vol(S^3\setminus \mathcal{L})+\sqrt{-1}cs(S^3\setminus \mathcal{L}),
\end{align}
for an appropriate choice of a branch of the logarithm, where $cs$ denotes the Chern-Simons invariant \cite{Mey86}. 
From the viewpoint of the $SL(2,\mathbb{C})$ Chern-Simons theory, S. Gukov conjectured in \cite{Guk05} that the asymptotic expansion of $J_{N}(\mathcal{K};e^{\frac{2\pi\sqrt{-1}}{k}})$ of a hyperbolic knot as $N,k\rightarrow \infty$ fixing $u=\frac{N}{k}$ is presented by the following form, 
\begin{align} \label{formula-generalvolume}
    J_{N}(\mathcal{K};e^{\frac{2\pi\sqrt{-1}}{k}})\sim e^{N\zeta}N^{\frac{3}{2}}\omega\left(1+\sum_{i=1}^{\infty}\kappa_i\left(\frac{2\pi\sqrt{-1}}{N}\right)^i\right)
\end{align}
for some scalars $\zeta,\omega,\kappa_i$ depending on $\mathcal{K}$ and $u$, also see \cite{DGLZ09,GH08}. Moreover, T. Ohtuski showed when $\mathcal{K}$ is a hyperbolic knot with up to 7 crossings \cite{Oht16,OhtYok18,Oht17},  the asymptotic expansions of the Kashaev invariant is presented by the following form 
\begin{align} \label{formula-Kashaevexpansion}
    \langle \mathcal{K}\rangle_{N}=e^{N\zeta}N^{\frac{3}{2}}\omega(\mathcal{K})\left(1+\sum_{i=1}^d\kappa_i(\mathcal{K})\left(\frac{2\pi\sqrt{-1}}{N}\right)^i+O\left(\frac{1}{N^{d+1}}\right)\right),
\end{align}
for any $d$, where $\omega(\mathcal{K})$ and $\kappa_i(\mathcal{K})$'s are some scalars. 

The volume conjecture has been rigorously proved for some particular knots and links such as torus knots \cite{KT00,DKash07}, the figure-eight knot \cite{AndHan06}, Whitehead doubles of $(2,p)$-torus knots \cite{Zheng07}, positive ierated torus knots \cite{Van08}, the $5_2$ knot \cite{Oht16}, the knots with 6 crossings \cite{OhtYok18}, the knots with 7 crossings \cite{Oht17} and some links \cite{GL05,Van08,Van08-2,YY10,Zheng07}, see also \cite{Mur10} for a review. 

On the other hand, it is known that the quantum invariant of a closed 3-manifold at $q=e^{\frac{2\pi\sqrt{-1}}{N}}$ is of polynomial order as $N\rightarrow \infty$. However, the first author and T. Yang  \cite{CY18} observed that  Reshetikhin-Turaev invariants and Turaev-Viro invariants at $q=e^{\frac{4\pi\sqrt{-1}}{r}}$, for $r\geq 3$ an odd, are of exponential order as $r\rightarrow \infty$. Furthermore,  they proposed the volume conjecture for Reshetikhin-Turaev invariants and Turaev-Viro invariants. 

In \cite{DKY18},  Detcherry, Kalfagianni and Yang  gave a formula relating the Turaev-Viro invariants of the
complement of a link $\mathcal{L}$ in $S^3$ to the values of the colored Jones polynomials of $\mathcal{L}$. By using this formula, they proved Chen-Yang's volume conjecture \cite{CY18} for the figure-eight knot and Borromean rings.
In addition, they proposed the following 
\begin{conjecture}[\cite{DKY18}, Question 1.7] \label{conjecture-DKY}
    For a hyperbolic link $\mathcal{L}$ in $S^3$, we have
   \begin{align}
       \lim_{N\rightarrow \infty}\frac{2\pi}{N} \log|J_{N}(\mathcal{L}; e^{\frac{2\pi\sqrt{-1}}{N+\frac{1}{2}}})|=vol(S^3\setminus \mathcal{L}).
   \end{align} 
\end{conjecture}



The asymptotic behavior of  $J_{N}(\mathcal{L}; e^{\frac{2\pi\sqrt{-1}}{N+\frac{1}{2}}})$  is not predicted
either by the original volume conjecture (\ref{formula-original-volume})  or by its generalizations (\ref{formula-generalvolume}).  Moreover, Conjecture \ref{conjecture-DKY}  seems somewhat surprising, since a result in \cite{GL11,CLZ15} has stated that for any positive integer $k$, $J_{N}(\mathcal{L};e^{\frac{2\pi\sqrt{-1}}{N+k}})$ grows only polynomially in $N$. Conjecture \ref{conjecture-DKY} has been proved for figure-eight knot and Borromean ring in \cite{DKY18},  we also refer to \cite{Wong19} for an extended version of Conjecture \ref{conjecture-DKY}. 

The purpose of this paper is to study Conjecture \ref{conjecture-DKY} for the twist knot $\mathcal{K}_p$. We investigate the asymptotic expansion for the normalized $N$-th colored Jones polynomial  $J_N(\mathcal{K}_p;e^{\frac{2\pi \sqrt{-1}}{N+\frac{1}{2}}})$ instead. Furthermore, in a subsequent paper \cite{CZ23-2}, we present two asymptotic expansions  for the normalized $N$-th colored Jones polynomials of twist knots at the more general root unity $e^{\frac{2\pi \sqrt{-1}}{N+\frac{1}{M}}}$ with $M\geq 2$, and at the root of unity  $e^{\frac{2\pi \sqrt{-1}}{N}}$ respectively. Moreover, the asymptotic expansion for  Reshetikhin-Turaev invariants of closed hyperbolic 3-manifolds obtained by integral surgery along the twist knot at the root of unity   $e^{\frac{4\pi\sqrt{-1}}{r}}$ will be given in \cite{CZ23-3}. Finally, the last article \cite{CZ23-3} in this series is devoted to the asymptotic expansion for the Turaev-Viro invariants of the complements of the twist knots in $S^3$.

Let $V(p,t,s)$ be the potential function of the colored Jones polynomial for the twist knot $\mathcal{K}_p$ given by formula (\ref{formula-potentialfunction00}). 
By Proposition \ref{prop-critical}, there exists a unique critical point $(t_0,s_0)$ of $V(p,t,s)$. Let $x_0=e^{2\pi\sqrt{-1}t_0}$ and $y_0=e^{2\pi\sqrt{-1}s_0}$, we put
\begin{align}
  \zeta(p)&=V(p,t_0,s_0)\\\nonumber
  &=\pi \sqrt{-1}\left((2p+1)s_0^2-(2p+3)s_0-2t_0\right)\\\nonumber
    &+\frac{1}{2\pi\sqrt{-1}}\left(\text{Li}_2(x_0y_0)+\text{Li}_2(x_0/y_0)-3\text{Li}_2(x_0)+\frac{\pi^2}{6}\right)  
\end{align}
and 
\begin{align}
    \omega(p)&=\frac{\sin (2\pi s_0)e^{2\pi\sqrt{-1}t_0}}{(1-e^{2\pi\sqrt{-1}t_0})^{\frac{3}{2}}\sqrt{\det Hess(V)(t_0,s_0)}}\\\nonumber
    &=\frac{(y_0-y_0^{-1})x_0}{-4\pi (1-x_0)^\frac{3}{2}\sqrt{H(p,x_0,y_0)}}
\end{align}
with
\begin{align}
    H(p,x_0,y_0)&=\left(\frac{-3(2p+1)}{\frac{1}{x_0}-1}+\frac{2p+1}{\frac{1}{x_0y_0}-1}+\frac{2p+1}{\frac{1}{x_0/y_0}-1}-\frac{3}{(\frac{1}{x_0}-1)(\frac{1}{x_0y_0}-1)}\right.\\\nonumber
    &\left.-\frac{3}{(\frac{1}{x_0}-1)(\frac{1}{x_0/y_0}-1)}+\frac{4}{(\frac{1}{x_0y_0}-1)(\frac{1}{x_0/y_0}-1)}\right).
\end{align}

Then, we have
\begin{theorem}  \label{theorem-main}
  For $p\geq 6$, the asymptotic expansion of the colored Jones polynomial $J_N(\mathcal{K}_p;e^{\frac{2\pi \sqrt{-1}}{N+\frac{1}{2}}})$ is given by the following form
    \begin{align}
       J_N(\mathcal{K}_p;e^{\frac{2\pi \sqrt{-1}}{N+\frac{1}{2}}})&=(-1)^{p+1}\frac{4\pi e^{\frac{1}{4}\pi\sqrt{-1}}(N+\frac{1}{2})^{\frac{1}{2}}}{\sin\frac{\pi}{2N+1}}\omega(p)e^{(N+\frac{1}{2})\zeta(p)}\\\nonumber
       &\cdot\left(1+\sum_{i=1}^d\kappa_i(p)\left(\frac{2\pi\sqrt{-1}}{N+\frac{1}{2}}\right)^i+O\left(\frac{1}{(N+\frac{1}{2})^{d+1}}\right)\right),
    \end{align}
    for $d\geq 1$, where $\omega(p)$ and $\kappa_i(p)$ are constants determined by $\mathcal{K}_p$.
\end{theorem}
By Lemma \ref{lemma-volume}, we know that
\begin{align}
     2\pi \zeta(p)=vol(S^3\setminus \mathcal{K}_p)+cs(S^3\setminus \mathcal{K}_p) \mod \pi^2\sqrt{-1}\mathbb{Z}.
\end{align}
\begin{corollary}
    For $p\geq 6$, we have
    \begin{align}
        \lim_{N\rightarrow \infty}\frac{2\pi}{N}\log J_N(\mathcal{K}_p;e^{\frac{2\pi \sqrt{-1}}{N+\frac{1}{2}}})=vol(S^3\setminus \mathcal{K}_p)+cs(S^3\setminus \mathcal{K}_p) \mod \pi^2\sqrt{-1}\mathbb{Z}.
    \end{align}
\end{corollary}
Hence we proved Conjecture \ref{conjecture-DKY}.  





\begin{example}
    For $p=100$, we compute that 
    \begin{align}
           t_0&=0.8237997818-0.1280592525\sqrt{-1}, \\\nonumber
           s_0&=0.5050124998-0.00001256317546\sqrt{-1}, \\\nonumber
           x_0&=1.000001243-1.999752031\sqrt{-1}, \\\nonumber
           y_0&=-0.9995829910-0.03149174478\sqrt{-1}, \\\nonumber
           2\pi \zeta(100)&=3.6636144-1043.809608\sqrt{-1}.
    \end{align}
\end{example}
\begin{remark}
    We need the condition $p\geq 6$ in Theorem \ref{theorem-main} which makes sure the volume $vol(S^3\setminus \mathcal{K}_p)$ is not too small, so we can construct the required homotopy and verify the assumptions of the saddle point method successfully. We remark that our method can also work for the cases of $p\leq -1$ with some exceptions.     
\end{remark}



We use the saddle point method developed by Ohtsuki in a series papers  \cite{Oht16,  Oht17, Oht18, OhtYok18}  to prove Theorem \ref{theorem-main}.   An outline of the proof is follows.  First, we write the colored Jones polynomial of the twist knot $J_N(\mathcal{K}_p;e^{\frac{2\pi\sqrt{-1}}{N+\frac{1}{2}}})$  as a summation of Fourier coefficients with the help of  quantum dilogarithm function and the Poisson summation formula. Next, we will show that infinite terms of these Fourier coefficients can be neglected in the sense that they can be sufficiently small order at $N\rightarrow \infty$, we obtain formula (\ref{formula-Poission-after}). Then we estimate the remained finite terms of Fourier coefficients by using the saddle point method. Finally, we find that only two main Fourier coefficients will contribute to the asymptotic expansion formula.  Hence we finish the proof Theorem  \ref{theorem-main}. 





The paper is organized as follows. In Section \ref{section-preliminaries}, we fix the notations and review the related materials that will be used in this paper. In Section \ref{section-potentialfunction}, we compute the potential function for the colored Jones polynomials of the twist knot $\mathcal{K}_p$ and obtain Proposition \ref{prop-coloredJonespotential}. In Section  \ref{Section-poissonsummation}, we prove Proposition \ref{prop-fouriercoeff}  which expresses the colored Jones polynomial of the twist knot 
$J_N(\mathcal{K}_p;e^{\frac{2\pi\sqrt{-1}}{N+\frac{1}{2}}})$  as a summation of Fourier coefficients by Poisson summation formula.
In Section \ref{Section-Asympoticexpansion}, 
we first show that infinite terms of the Fourier coefficients can be neglected. Then we estimate the remained finite terms of Fourier coefficients by using the saddle point method, we obtain that only two main Fourier coefficients will contribute to the final form of the  asymptotic expansion.  Hence we finish the proof Theorem  \ref{theorem-main}.  Section \ref{Section-App} is devoted to the proof of several lemmas which will be used in previous sections.


\textbf{Acknowledgements.} 

The first author would like to thank Nicolai Reshetikhin, Kefeng Liu and Weiping Zhang for bringing him to this area and a lot of discussions during his career, thank Francis Bonahon,   Giovanni Felder and Shing-Tung Yau for his continuous encouragement, support and discussions, and thank Jun Murakami and Tomotada Ohtsuki for their helpful discussions and support. He also want to thank Jørgen Ellegaard Andersen, Sergei Gukov, Thang Le, Gregor Masbaum,  Rinat Kashaev, Vladimir Turaev and Hiraku Nakajima for their support, discussions and interests, and thank Yunlong Yao who built him solid analysis foundation twenty years ago.

  

 
The second author would like to thank Kefeng Liu and Hao Xu for bringing him to this area when he was a graduate student at CMS of Zhejiang University, and for their constant encouragement and helpful discussions since then.


\section{Preliminaries} \label{section-preliminaries}
\subsection{Colored Jones polynomials}
In this subsection, we review the definition of the colored Jones polynomials and fix the notations.  
 Let $M$ be an oriented 3-manifold, the Kauffman bracket skeim module $\mathcal{K}(M)$ is the free $\mathbb{Z}[A^{\pm 1}]$-module generated by isotopy classes of framed links in $M$ modulo the submodule generated by  the Kauffman bracket skein relation:

(1) Kauffman bracket skein relation: 
% Figure environment removed

(2) Framing relation: 
% Figure environment removed


 The Kauffman bracket $\langle \mathcal{L}\ \rangle$ of a framed link $\mathcal{L}$ in $S^3$ gives a map from $\mathcal{K}(S^3)$ to $\mathbb{Z}[A^{\pm 1}]$. We use the normalization that the bracket of the empty link is 1. 

The Kauffman bracket skein module of the solid torus $S^1\times D^2$ is given by $\mathbb{Z}[A^{\pm 1}][z]$. Usually, we denote this skein module by $\mathcal{B}$. Here $z$ is given by the framed link $S^1\times J$, where $J$ is a small arc lies in the interior of $D^2$, and $z^n$ means $n$-parallel copies of $z$. 

The twist map $t: \mathcal{B}\rightarrow \mathcal{B}$ is a map induced by a full right handed twist on the solid torus. There exists a basis $\{e_i\}_{i\geq 0}$ of $\mathcal{B}$, which are eigenvectors of the twist map $t$ (see e.g \cite{BHMV92}).  $\{e_i\}_{i\geq 0}$ can be defined recursively by 
\begin{align}
    e_0=1, \ e_1=z,  \ e_i=ze_{i-1}-e_{i-2}. 
\end{align}
Moreover, the $e_i$ satisfies
\begin{align}
\langle e_i \rangle&=(-1)^i\frac{A^{2(i+1)}-A^{-2(i+1)}}{A^2-A^{-2}}  \\
t(e_i)&=\mu_i e_i
\end{align}
where $\mu_i=(-1)^iA^{i^2+2i}$ is also called the framing factor. Throughout this paper, we make the  convention  
\begin{align} \label{formula-qconvention}
    q=A^{4}, \ \{n\}=q^{\frac{n}{2}}-q^{-\frac{n}{2}} \ \text{for an integer } \ n. 
\end{align}


\begin{definition}
Given a knot $\mathcal{K}$ with zero framing, the {\em $N$-th colored Jones polynomial} $J_{N}(\mathcal{K};q)$ of $\mathcal{K}$ is defined to be the Kauffman bracket of $\mathcal{K}$ cabled by $(-1)^{N-1}e_{N-1}$, i.e.
\begin{align}
    \bar{J}_{N}(\mathcal{K};q)=(-1)^{N-1}\langle\mathcal{K}(e_{N-1}) \rangle
\end{align}
where the factor of $(-1)^N$ is included such that for the unknot $U$, $\bar{J}_{N}(U;q)=[N]$. Furthermore, the {\em normalized $N$-th colored Jones polynomial} of $\mathcal{K}$ is defined as
\begin{align}
    J_{N}(\mathcal{K};q)=\frac{\langle \mathcal{K}(e_{N-1})\rangle}{\langle e_{N-1}\rangle}.
\end{align}
\end{definition}




We consider the twist knot $\mathcal{K}_p$ illustrated in Figure 1,
% Figure environment removed
where the index $2p$ represents $2p$ crossings (half-twists). For example, $\mathcal{K}_{-1}=4_1$, $\mathcal{K}_1=3_1$, $\mathcal{K}_2=5_2$.


By using the Kauffman bracket skein theory \cite{BHMV92,MV94}, Masbaum \cite{Mas03} rederived the cyclotomic expansion formula for the colored Jones polynomial of the twist knot $\mathcal{K}_p$ due to Habiro \cite{Hab08}.    

\begin{proposition}
The normalized $N$-th colored Jones polynomial of the twist knot $\mathcal{K}_p$ is given by 
\begin{align} \label{formula-coloredJonestwist}
J_N(\mathcal{K}_{p};q)=\sum_{k=0}^{N-1}\sum_{l=0}^{k}(-1)^lq^{\frac{k(k+3)}{4}+pl(l+1)}\frac{\{k\}!\{2l+1\}}{\{k+l+1\}!\{k-l\}!}\prod_{i=1}^k(\{N+i\}
\{N-i\}).
\end{align}
\end{proposition}

\subsection{Dilogarithm and Lobachevsky functions}
Let $\log: \mathbb{C}\setminus (-\infty,0]\rightarrow \mathbb{C}$ be the standard logarithm function defined by 
\begin{align}
    \log z=\log |z|+\sqrt{-1}\arg z
\end{align}
with $-\pi <\arg z<\pi$. 

The dilogarithm function $\text{Li}_2: \mathbb{C}\setminus (1,\infty)\rightarrow \mathbb{C}$ is defined by 
\begin{align}
    \text{Li}_2(z)=-\int_0^{z}\frac{\log(1-x)}{x}dx
\end{align}
where the integral is along any path in $\mathbb{C}\setminus (1,\infty)$ connecting $0$ and $z$, which is holomorphic in $\mathbb{C}\setminus [1,\infty)$ and continuous in $\mathbb{C}\setminus (1,\infty)$. 

The dilogarithm function satisfies the following properties 
\begin{align}
    \text{Li}_2\left(\frac{1}{z}\right)=-\text{Li}_2(z)-\frac{\pi^2}{6}-\frac{1}{2}(\log(-z) )^2.
\end{align}
In the unit disk $\{z\in \mathbb{C}| |z|<1\}$,  $\text{Li}_2(z)=\sum_{n=1}^{\infty}\frac{z^n}{n^2}$, and on the unit circle 
\begin{align}
 \{z=e^{2\pi \sqrt{-1}t}|0 \leq t\leq 1\},    
\end{align}
we have
\begin{align}
    \text{Li}_2(e^{2\pi\sqrt{-1} t})=\frac{\pi^2}{6}+\pi^2t(t-1)+2\pi \sqrt{-1}\Lambda(t)
\end{align}
where 
\begin{align} \label{formula-Lambda(t)}
\Lambda(t)=\text{Re}\left(\frac{\text{Li}_2(e^{2\pi \sqrt{-1}t})}{2\pi \sqrt{-1}}\right)=-\int_{0}^{t}\log|2 \sin \pi t|d t 
\end{align}
for $t\in \mathbb{R}$. The function $\Lambda(t)$ is an odd function which has period $1$ and satisfies 
$
\Lambda(1)=\Lambda(\frac{1}{2})=0.
$



Furthermore, we have the follow estimation for the function $$\text{Re}\left(\frac{1}{2\pi\sqrt{-1}}\text{Li}_2\left(e^{2\pi\sqrt{-1}(t+X\sqrt{-1})}\right)\right)$$ with $t,X\in \mathbb{R}$.   
\begin{lemma} (see Lemma 2.2 in \cite{OhtYok18}) \label{lemma-Li2}
    Let $t$ be a real number with $0<t<1$. Then there exists a constant $C>0$ such that 
\begin{align}
    \left\{ \begin{aligned}
         &0  &  \ (\text{if} \ X\geq 0) \\
         &2\pi \left(t-\frac{1}{2}\right)X & \ (\text{if} \ X<0)
                          \end{aligned} \right.-C<\text{Re}\left(\frac{1}{2\pi\sqrt{-1}}\text{Li}_2\left(e^{2\pi\sqrt{-1}(t+X\sqrt{-1})}\right)\right)
\end{align}
\begin{align*}
    <\left\{ \begin{aligned}
         &0  &  \ (\text{if} \ X\geq 0) \\
         &2\pi \left(t-\frac{1}{2}\right)X & \ (\text{if} \ X<0)
                          \end{aligned}\right.+C.
\end{align*}.
\end{lemma}


\subsection{Quantum dilogrithm functions}
For a positive integer $N$, we set $\xi_N=e^{\frac{2\pi\sqrt{-1}}{N+\frac{1}{2}}}$. We introduce the holomorphic function $\varphi_N(t)$ for $\{t\in
\mathbb{C}| 0< \text{Re} t < 1\}$, by the following integral
\begin{align}
\varphi_N(t)=\int_{-\infty}^{+\infty}\frac{e^{(2t-1)x}dx}{4x \sinh x
\sinh\frac{x}{N+\frac{1}{2}}}.
\end{align}
Noting that this integrand has poles at $n\pi \sqrt{-1} (n\in
\mathbb{Z})$, where, to avoid the poles at $0$, we choose the
following contour of the integral
\begin{align}
\gamma=(-\infty,-1]\cup \{z\in \mathbb{C}||z|=1, \text{Im} z\geq 0\}
\cup [1,\infty).
\end{align}

\begin{lemma}  \label{lemma-varphixi}
The function $\varphi_N(t)$ satisfies 
\begin{align}
    (\xi_{N})_n&=\exp \left(\varphi_N\left(\frac{1}{2N+1}\right)-\varphi_N\left(\frac{2n+1}{2N+1}\right)\right)   \  \   \left(0\leq n\leq N\right), \\
    (\xi_{N})_n&=\exp \left(\varphi_N\left(\frac{1}{2N+1}\right)-\varphi_N\left(\frac{2n+1}{2N+1}-1\right)+\log 2\right)   \  \   \left(N< n\leq 2N\right).
\end{align}
\end{lemma}

\begin{lemma} \label{lemma-varphixi2}
    We have the following identities:
\begin{align}
    \varphi_N(t)+\varphi_N(1-t)&=2\pi \sqrt{-1}\left(-\frac{2N+1}{4}(t^2-t+\frac{1}{6})+\frac{1}{12(2N+1)}\right),\\ 
    \varphi_N\left(\frac{1}{2N+1}\right)&=\frac{2N+1}{4\pi\sqrt{-1}}\frac{\pi^2}{6}+\frac{1}{2}\log \left(\frac{2N+1}{2}\right)+\frac{\pi \sqrt{-1}}{4}-\frac{\pi \sqrt{-1}}{6(2N+1)},\\
    \varphi_N\left(1-\frac{1}{2N+1}\right)&=\frac{2N+1}{4\pi\sqrt{-1}}\frac{\pi^2}{6}-\frac{1}{2}\log \left(\frac{2N+1}{2}\right)+\frac{\pi \sqrt{-1}}{4}-\frac{\pi \sqrt{-1}}{6(2N+1)}.
\end{align}
\end{lemma}
The function $\varphi_N (t)$ is closely related to the dilogarithm function as follows.
\begin{lemma} \label{lemma-varphixi3}
    (1)For every $t$ with $0<Re t<1$, 
    \begin{align}
        \varphi_N(t)=\frac{N+\frac{1}{2}}{2\pi \sqrt{-1}}\text{Li}_2(e^{2\pi\sqrt{-1}t})
 -\frac{\pi \sqrt{-1}e^{2\pi\sqrt{-1}t}}{6(1-e^{2\pi\sqrt{-1}t})}\frac{1}{2N+1}+O\left(\frac{1}{(N+\frac{1}{2})^3}\right).
    \end{align}
    (2) For every $t$ with $0<Re t<1$, 
    \begin{align}
        \varphi_N'(t)=-\frac{2N+1}{2}\log(1-e^{2\pi\sqrt{-1}t})+O\left(\frac{1}{N+\frac{1}{2}}\right)
    \end{align}
    (3) As $N\rightarrow \infty$, $\frac{1}{N+\frac{1}{2}}\varphi_N(t)$ uniformly converges to $\frac{1}{2\pi\sqrt{-1}}\text{Li}_2(e^{2\pi\sqrt{-1}t})$ and $\frac{1}{N+\frac{1}{2}}\varphi'_N(t)$ uniformly converges to $-\log(1-e^{2\pi\sqrt{-1}t})$ on any compact subset of $\{t\in \mathbb{C}|0<Re z<1\}$. 
\end{lemma}
See the literature, such as \cite{Oht16,CJ17,WongYang20-1} for the proof of Lemma \ref{lemma-varphixi}, \ref{lemma-varphixi2}, \ref{lemma-varphixi3}.


 \subsection{Saddle point method}
We need to use the following version of saddle point method as illustrated in \cite{Oht18}.
\begin{proposition}[\cite{Oht18}, Proposition 3.1]  \label{proposition-saddlemethod}
   Let $A$ be a non-singular symmetric complex $2\times 2$ matrix, and let $\Psi(z_1,z_2)$ and $r(z_1,z_2)$ be holomorphic functions of the forms, 
   \begin{align}
    \Psi(z_1,z_2)&=\mathbf{z}^{T}A\mathbf{z}+r(z_1,z_2), \\\nonumber
    r(z_1,z_2)&=\sum_{i,j,k}b_{ijk}z_iz_jz_k+\sum_{i,j,k,l}c_{ijkl}z_iz_jz_kz_l+\cdots
   \end{align}
   defined in a neighborhood of $\mathbf{0}\in \mathbb{C}$. The restriction of the domain 
   \begin{align} \label{formula-domain0}
       \{(z_1,z_2)\in \mathbb{C}^2| \text{Re}\Psi(z_1,z_2)<0\}  
   \end{align}
   to a neighborhood of $\mathbf{0}\in \mathbb{C}^2$ is homotopy equivalent to $S^1$. Let $D$ be an oriented disk embeded in $\mathbb{C}^2$ such that $\partial D$ is included in the domain (\ref{formula-domain0}) whose inclusion is homotopic to a homotopy equivalence to the above $S^1$ in the domain (\ref{formula-domain0}). Then we have the following asymptotic expansion
\begin{align}
    \int_{D}e^{N\psi(z_1,z_2)}dz_1dz_2=\frac{\pi}{N\sqrt{\det(-A)}}\left(1+\sum_{i=1}^d\frac{\lambda_i}{N^i}+O(\frac{1}{N^{d+1}})\right),
\end{align}
   for any $d$, where we choose the sign of $\sqrt{\det{(-A)}}$ as explained in Proposition \cite{Oht16}, and $\lambda_i$'s are constants presented by using coefficients of the expansion $\Psi(z_1,z_2)$, such presentations are obtained by formally expanding the following formula, 
\begin{align}
    1+\sum_{i=1}^{\infty}\frac{\lambda_i}{N^i}=\exp\left(Nr\left(\frac{\partial }{\partial w_1},\frac{\partial }{\partial w_2}\right)\right)\exp\left(-\frac{1}{4N}(w_1,w_2)A^{-1}\binom{w_1}{w_2}\right)|_{w_1=w_2=0}.
\end{align}
\end{proposition}
For a proof of the Proposition \ref{proposition-saddlemethod},  see \cite{Oht16}. 
\begin{remark}[\cite{Oht18}, Remark 3.2]  \label{remark-saddle}
    As mentioned in Remark 3.6 of \cite{Oht16}, we can extend Proposition \ref{proposition-saddlemethod} to the case where $\Psi(z_1,z_2)$ depends on $N$ in such a way that $\Psi(z_1,z_2)$ is of the form 
    \begin{align}
        \Psi(z_1,z_2)=\Psi_0(z_1,z_2)+\Psi_1(z_1,z_2)\frac{1}{N}+R(z_1,z_2)\frac{1}{N^2}. 
    \end{align}
    where $\Psi_i(z_1,z_2)$'s are holomorphic functions independent of $N$, and we assume that $\Psi_0(z_1,z_2)$ satisfies the assumption of the Proposition and $|R(z_1,z_2)|$ is bounded by a constant which is independent of $N$.  
\end{remark}

   

 




\section{Computation of the potential function} \label{section-potentialfunction}
This section is devoted to the computation of potential function for the colored 
Jones polynomial $J_{N}(\mathcal{K}_p;q)$ at the root of unity $\xi_N$. 

We introduce the following $q$-Pochhammer symbol
\begin{align}
    (q)_n=\prod_{i=1}^{n}(1-q^i).
\end{align}
Then we have
\begin{align}
\ \{n\}!=(-1)^nq^{\frac{-n(n+1)}{4}}(q)_n.
\end{align}
By formula (\ref{formula-coloredJonestwist}),  we obtain
\begin{align}
J_{N}(\mathcal{K}_p;q)
=&\sum_{k=0}^{N-1}\sum_{l=0}^k(-1)^{k+l}q^{pl(l+1)+\frac{l(l-1)}{2}-Nk+\frac{k(k+1)}{2}+k}\\\nonumber
&\cdot\frac{(1-q^{2l+1})}{(1-q^N)}\frac{(q)_k(q)_{N+k}}{(q)_{k+l+1}(q)_{k-l}(q)_{N-k-1}}. 
\end{align}
Computing at the root of unity $\xi_N$, we get   
\begin{align}  \label{formula-sumkl}
 J_{N}(\mathcal{K}_p;\xi_N)=&\sum_{k=0}^{N-1}\sum_{l=0}^k\frac{(-1)^{k+l+1}\sin \frac{2\pi(2l+1)}{2N+1}}{\sin \frac{\pi }{2N+1}}\\\nonumber
 &\cdot q^{(p+\frac{1}{2})l^2+(p+\frac{1}{2})l+\frac{k^2}{2}+2k+\frac{3}{4}}\frac{(q)_k(q)_{N+k}}{(q)_{k+l+1}(q)_{k-l}(q)_{N-k-1}}|_{q=\xi_N}. 
\end{align}




Now, we study the following term
\begin{align}
   &(-1)^{l-k-1}q^{(p+\frac{1}{2})l^2+(p+\frac{1}{2})l+\frac{k^2}{2}+2k+\frac{3}{4}}\frac{(q)_k(q)_{N+k}}{(q)_{k+l+1}(q)_{k-l}(q)_{N-k-1}}|_{q=\xi_N} .
\end{align}

By using Lemma \ref{lemma-varphixi},  we obtain 
\begin{align}
    &\frac{(\xi_N)_k(\xi_N)_{N+k}}{(\xi_N)_{k+l+1}(\xi_N)_{k-l}(\xi_N)_{N-k-1}}\\\nonumber
    &=\exp\left(\varphi_N\left(\frac{2(k+l+1)+1}{2N+1}\right)+\varphi_N\left(\frac{2(k-l)+1}{2N+1}\right)+\varphi_N\left(1-\frac{2(k+1)}{2N+1}\right)\right.\\\nonumber
    &\left.-\varphi_N\left(\frac{2k+1}{2N+1}\right)-\varphi_N\left(\frac{2k}{2N+1}\right)-\varphi_N\left(\frac{1}{2N+1}\right)+\log 2\right) \\\nonumber
    &=\exp\left(\varphi_N\left(\frac{2k+2l+3}{2N+1}\right)+\varphi_N\left(\frac{2k-2l+1}{2N+1}\right)-\varphi_N\left(\frac{2k+1}{2N+1}\right)\right.\\\nonumber
    &\left.-\varphi_N\left(\frac{2k}{2N+1}\right)-\varphi_N\left(\frac{2k+2}{2N+1}\right)+\frac{(2N+1)\pi\sqrt{-1}}{24 }-\frac{\pi \sqrt{-1}}{4}-\frac{1}{2}\log \frac{2N+1}{2}\right.\\\nonumber
    &\left.+\pi \sqrt{-1}\left(\frac{1}{3(2N+1)}-\frac{2(k+1)^2}{2N+1}+(k+1)-\frac{2N+1}{12}\right)+\log 2\right),
\end{align}
for $0<k+l+1\leq N$, and
\begin{align}
    &\frac{(\xi_N)_k(\xi_N)_{N+k}}{(\xi_N)_{k+l+1}(\xi_N)_{k-l}(\xi_N)_{N-k-1}}\\\nonumber
    &=\exp\left(\varphi_N\left(\frac{2(k+l+1)+1}{2N+1}-1\right)+\varphi_N\left(\frac{2(k-l)+1}{2N+1}\right)\right.\\\nonumber
    &\left.+\varphi_r\left(1-\frac{2(k+1)}{2N+1}\right)-\varphi_N\left(\frac{2k+1}{2N+1}\right)-\varphi_N\left(\frac{2k}{2N+1}\right)-\varphi_N\left(\frac{1}{2N+1}\right)\right)\\\nonumber
    &=\exp\left(\varphi_N\left(\frac{2k+2l+3}{2N+1}-1\right)+\varphi_N\left(\frac{2k-2l+1}{2N+1}\right)-\varphi_N\left(\frac{2k+1}{2N+1}\right)\right.\\\nonumber
    &\left.-\varphi_N\left(\frac{2k}{2N+1}\right)-\varphi_N\left(\frac{2k+2}{2N+1}\right)+\frac{(2N+1)\pi\sqrt{-1}}{24 }-\frac{\pi \sqrt{-1}}{4}-\frac{1}{2}\log \frac{2N+1}{2} \right.\\\nonumber
    &\left.+\pi \sqrt{-1}\left(\frac{1}{3(2N+1)}-\frac{2(k+1)^2}{2N+1}+(k+1)-\frac{2N+1}{12}\right)\right),
\end{align}
 for $N<k+l+1\leq 2N$. 

Therefore, we obtain 
\begin{align}
    &(-1)^{l-k-1}\xi_N^{(p+\frac{1}{2})(l^2+l)+\frac{k^2}{2}+2k+\frac{3}{4}}\frac{(\xi_N)_k(\xi_N)_{N+k}}{(\xi_N)_{k+l+1}(\xi_N)_{k-l}(\xi_N)_{N-k-1}}\\\nonumber
    &=2\exp (N+\frac{1}{2})\left(\pi \sqrt{-1}\left((2p+1)\left(\frac{2l+1}{2N+1}\right)^2+\frac{4(2k+1)}{(2N+1)^2}-\frac{6p+7}{3(2N+1)^2}\right.\right.\\\nonumber&\left.\left.+\frac{2l+1}{2N+1}-\frac{3}{2(2N+1)}\right)+\frac{1}{N+\frac{1}{2}}\varphi_N\left(\frac{2k+2l+3}{2N+1}\right)+\frac{1}{N+\frac{1}{2}}\varphi_N\left(\frac{2k-2l+1}{2N+1}\right)\right.\\\nonumber
    &\left.-\frac{1}{N+\frac{1}{2}}\varphi_N\left(\frac{2k+1}{2N+1}\right)-\frac{1}{N+\frac{1}{2}}\varphi_N\left(\frac{2k}{2N+1}\right)-\frac{1}{N+\frac{1}{2}}\varphi_N\left(\frac{2k+2}{2N+1}\right)\right.\\\nonumber
    &\left.-\frac{\pi\sqrt{-1}}{12 }-\frac{1}{2N+1}\log \frac{2N+1}{2} \right)
\end{align}
for $0<k+l+1\leq N$, and
\begin{align}
    &(-1)^{l-k-1}\xi_N^{(p+\frac{1}{2})(l^2+l)+\frac{k^2}{2}+2k+\frac{3}{4}}\frac{(\xi_N)_k(\xi_N)_{N+k}}{(\xi_N)_{k+l+1}(\xi_N)_{k-l}(\xi_N)_{N-k-1}}\\\nonumber
    &=\exp (N+\frac{1}{2})\left(\pi \sqrt{-1}\left((2p+1)\left(\frac{2l+1}{2N+1}\right)^2+\frac{4(2k+1)}{(2N+1)^2}-\frac{6p+7}{3(2N+1)^2}\right.\right.\\\nonumber&\left.\left.+\frac{2l+1}{2N+1}-\frac{3}{2(2N+1)}\right)+\frac{1}{N+\frac{1}{2}}\varphi_N\left(\frac{2k+2l+3}{2N+1}-1\right)\right.\\\nonumber
    &\left.+\frac{1}{N+\frac{1}{2}}\varphi_N\left(\frac{2k-2l+1}{2N+1}\right)-\frac{1}{N+\frac{1}{2}}\varphi_N\left(\frac{2k+1}{2N+1}\right)-\frac{1}{N+\frac{1}{2}}\varphi_N\left(\frac{2k}{2N+1}\right)\right.\\\nonumber
    &\left.-\frac{1}{N+\frac{1}{2}}\varphi_N\left(\frac{2k+2}{2N+1}\right)-\frac{\pi\sqrt{-1}}{12 }-\frac{1}{2N+1}\log \frac{2N+1}{2} \right)
\end{align}
for $N<k+l+1\leq 2N$. 

Now we set
\begin{align}
    t=\frac{2k+1}{2N+1}, s=\frac{2l+1}{2N+1}
\end{align}
and define the functions $\tilde{V}_N(p,t,s)$ and $\delta(t,s)$ as follows. 

(1) If $0<s<1$, $0<t\pm s<1$, then 
\begin{align*}
    \tilde{V}_N(p,t,s)&=\pi \sqrt{-1}\left((2p+1)s^2+s+\frac{4}{2N+1}t-\frac{6p+7}{3(2N+1)^2}-\frac{3}{2(2N+1)}\right)\\\nonumber
    &+\frac{1}{N+\frac{1}{2}}\varphi_N\left(t+s+\frac{1}{2N+1}\right)+\frac{1}{N+\frac{1}{2}}\varphi_N\left(t-s+\frac{1}{2N+1}\right)\\\nonumber
    &-\frac{1}{N+\frac{1}{2}}\varphi_N\left(t\right)-\frac{1}{N+\frac{1}{2}}\varphi_N\left(t-\frac{1}{2N+1}\right)-\frac{1}{N+\frac{1}{2}}\varphi_N\left(t+\frac{1}{2N+1}\right)\\\nonumber
    &-\frac{\pi\sqrt{-1}}{12 }-\frac{1}{2N+1}\log \frac{2N+1}{2},
\end{align*}
and $\delta(t,s)=2$. 

(2) If $0<t<1$, $0<t-s<1$ and $1<t+s<2$,  then 
\begin{align*}
    \tilde{V}_N(p,t,s)&=\pi \sqrt{-1}\left((2p+1)s^2+s+\frac{4}{2N+1}t-\frac{6p+7}{3(2N+1)^2}-\frac{3}{2(2N+1)}\right)\\\nonumber
    &+\frac{1}{N+\frac{1}{2}}\varphi_N\left(t+s+\frac{1}{2N+1}-1\right)+\frac{1}{N+\frac{1}{2}}\varphi_N\left(t-s+\frac{1}{2N+1}\right)\\\nonumber
    &-\frac{1}{N+\frac{1}{2}}\varphi_N\left(t\right)-\frac{1}{N+\frac{1}{2}}\varphi_N\left(t-\frac{1}{2N+1}\right)-\frac{1}{N+\frac{1}{2}}\varphi_N\left(t+\frac{1}{2N+1}\right)\\\nonumber
    &-\frac{\pi\sqrt{-1}}{12 }-\frac{1}{2N+1}\log \frac{2N+1}{2},
\end{align*}
and $\delta(t,s)=1$. 

Based on the above calculations, we obtain
\begin{align} \label{formula-coloredJonesPotential1}
    J_{N}(\mathcal{K}_p;\xi_N)&=\sum_{k=0}^{N-1}\sum_{l=0}^k\frac{\sin \frac{2\pi (2l+1)}{2N+1}}{\sin\frac{\pi}{2N+1}}\delta\left(\frac{2k+1}{2N+1},\frac{2l+1}{2N+1}\right)e^{(N+\frac{1}{2})\tilde{V}_N\left(\frac{2k+1}{2N+1},\frac{2l+1}{2N+1}\right)}\\\nonumber
    &=\sum_{k=0}^{N-1}\sum_{l=0}^k\frac{\sin \frac{2\pi (2l+1)}{2N+1}}{\sin\frac{\pi}{2N+1}}\delta\left(\frac{2k+1}{2N+1},\frac{2l+1}{2N+1}\right)\\\nonumber
    &\cdot e^{(N+\frac{1}{2})\left(\tilde{V}_N\left(\frac{2k+1}{2N+1},\frac{2l+1}{2N+1}\right)-2\pi\sqrt{-1}\frac{2k}{2N+1}-2(p+2)\pi\sqrt{-1}\frac{2l}{2N+1}\right)}. 
\end{align}
For convenience, we introduce the function $V_N(p,x,y)$ which is determined by the following formula  
\begin{align}
    &\tilde{V}_N(p,t,s)-2\pi\sqrt{-1}(t-\frac{1}{2N+1})-2(p+2)\pi\sqrt{-1}(s-\frac{1}{2N+1})\\\nonumber
    &=V_N(p,t,s)+\pi\sqrt{-1}\frac{4p+9}{2(2N+1)}-\frac{1}{2N+1}\log \frac{2N+1}{2}. 
\end{align}
Note that the functions $\tilde{V}_N(p,t,s)$ and $V_N(p,t,s)$ are defined on the region 
    \begin{align}
    D=\{(t,s)\in \mathbb{R}^2| 0<t<1, 0<s<1,  0< t-s<1\}.
\end{align}


From formula (\ref{formula-coloredJonesPotential1}),  we finally obtain
\begin{proposition}   \label{prop-coloredJonespotential}
The normalized $N$-th colored Jones polynomial of the twist $\mathcal{K}_p$ at the root of unity $\xi_N$ can be computed as
\begin{align} \label{formula-coloredJonespotential2}
     J_N(\mathcal{K}_{p};\xi_N)=\sum_{k=0}^{N-1}\sum_{l=0}^k g_N(k,l)
\end{align}
with
\begin{align}
    g_N(k,l)&=(-1)^pe^{\frac{\pi\sqrt{-1}}{4}}\frac{1}{\sqrt{(N+\frac{1}{2})}\sin\frac{\pi}{2N+1}}\sin \frac{2\pi (2l+1)}{2N+1}\\\nonumber
    &\cdot \delta\left(\frac{2k+1}{2N+1},\frac{2l+1}{2N+1}\right)e^{(N+\frac{1}{2})V_N\left(p,\frac{2k+1}{2N+1},\frac{2l+1}{2N+1}\right)},
\end{align}
where the function $\delta(t,s)$ and the function $V_N(p,t,s)$ are  given by  

(1) If $0<t<1$, $0<t\pm s<1$, then $\delta(t,s)=2$ and
\begin{align}
    V_N(p,t,s)&=\pi \sqrt{-1}\left((2p+1)s^2-(2p+3)s+\left(\frac{4}{2N+1}-2\right)t-\frac{6p+7}{3(2N+1)^2}\right)\\\nonumber
    &+\frac{1}{N+\frac{1}{2}}\varphi_N\left(t+s+\frac{1}{2N+1}\right)+\frac{1}{N+\frac{1}{2}}\varphi_N\left(t-s+\frac{1}{2N+1}\right)\\\nonumber
    &-\frac{1}{N+\frac{1}{2}}\varphi_N\left(t\right)-\frac{1}{N+\frac{1}{2}}\varphi_N\left(t-\frac{1}{2N+1}\right)\\\nonumber
    &-\frac{1}{N+\frac{1}{2}}\varphi_N\left(t+\frac{1}{2N+1}\right)-\frac{\pi\sqrt{-1}}{12 }.
\end{align}
(2)  If $0<t<1$, $0<t-s<1$ and $1<t+s<2$, then $\delta(t,s)=1$ and 
\begin{align}
    V_N(p,t,s)&=\pi \sqrt{-1}\left((2p+1)s^2-(2p+3)s+\left(\frac{4}{2N+1}-2\right)t-\frac{6p+7}{3(2N+1)^2}\right)\\\nonumber
    &+\frac{1}{N+\frac{1}{2}}\varphi_N\left(t+s+\frac{1}{2N+1}-1\right)+\frac{1}{N+\frac{1}{2}}\varphi_N\left(t-s+\frac{1}{2N+1}\right)\\\nonumber
    &-\frac{1}{N+\frac{1}{2}}\varphi_N\left(t\right)-\frac{1}{N+\frac{1}{2}}\varphi_N\left(t-\frac{1}{2N+1}\right)\\\nonumber
    &-\frac{1}{N+\frac{1}{2}}\varphi_N\left(t+\frac{1}{2N+1}\right)-\frac{\pi\sqrt{-1}}{12 }.
\end{align}
\end{proposition}

We introduce the potential function for the twist knot $\mathcal{K}_p$ as 
\begin{align} \label{formula-potentialfunction00}
    &V(p,t,s)=\lim_{N\rightarrow\infty}V_{N}(p,t,s)=\pi \sqrt{-1}\left((2p+1)s^2-(2p+3)s-2t\right)\\\nonumber
    &+\frac{1}{2\pi\sqrt{-1}}\left(\text{Li}_2(e^{2\pi\sqrt{-1}(t+s)})+\text{Li}_2(e^{2\pi\sqrt{-1}(t-s)})-3\text{Li}_2(e^{2\pi\sqrt{-1}t})+\frac{\pi^2}{6}\right). 
\end{align}










\section{Poisson summation formula} \label{Section-poissonsummation}
In this section, with the help of Poisson summation formula, we write the formula (\ref{formula-coloredJonespotential2}) as a sum of integrals.  First, according to formulas (\ref{formula-coloredJonestwist}) and (\ref{formula-coloredJonespotential2}), we have
\begin{align}
g_{N}(k,l)&=(-1)^lq^{\frac{k(k+3)}{4}+pl(l+1)}\frac{\{k\}!\{2l+1\}}{\{k+l+1\}!\{k-l\}!}\prod_{i=1}^k(\{N+i\}
\{N-i\})\\\nonumber
&=(-1)^lq^{\frac{k(k+3)}{4}+pl(l+1)}\frac{\{2l+1\}}{\{N\}}\frac{\{k\}!\{N+k\}!}{\{k+l+1\}!\{k-l\}!\{N-k-1\}!}. 
\end{align}

By Lemmas \ref{lemma-varphixi}, \ref{lemma-varphixi2}, \ref{lemma-varphixi3} and formula (\ref{formula-Lambda(t)}), we obtain 
\begin{align}
    \log \left|\{n\}!\right|=-(N+\frac{1}{2})\Lambda\left(\frac{2n+1}{2N+1}\right)+O(\log (2N+1))
\end{align}  
for any integer $0<n<2N+1$ and at $q=\xi_N=e^{\frac{2\pi \sqrt{-1}}{N+\frac{1}{2}}}$.
So we have 
\begin{align}
    &\log|g_N(k,l)|\\\nonumber
    &=-(N+\frac{1}{2})\Lambda\left(\frac{2k+1}{2N+1}\right)-(N+\frac{1}{2})\Lambda\left(\frac{2N+2k+1}{2N+1}\right)+(N+\frac{1}{2})\Lambda\left(\frac{2(k+l+1)+1}{2N+1}\right)\\\nonumber
    &+(N+\frac{1}{2})\Lambda\left(\frac{2(k-l)+1}{2N+1}\right)+(N+\frac{1}{2})\Lambda\left(\frac{2(N-k-1)+1}{2N+1}\right)+O(\log (2N+1)) \\\nonumber
    &=-(N+\frac{1}{2})\Lambda\left(\frac{2k+1}{2N+1}\right)-(N+\frac{1}{2})\Lambda\left(\frac{2k}{2N+1}\right)+(N+\frac{1}{2})\Lambda\left(\frac{2k+2l+3}{2N+1}\right)\\\nonumber
    &+(N+\frac{1}{2})\Lambda\left(\frac{2k-2l+1}{2N+1}\right)-(N+\frac{1}{2})\Lambda\left(\frac{2k+2}{2N+1}\right)+O(\log (2N+1)), 
\end{align}
where in the second ``=" we have used the properties of the function $\Lambda(t)$. 
We put
\begin{align}
    v_N(t,s)&=\Lambda\left(t+s+\frac{1}{2N+1}\right)+\Lambda\left(t-s+\frac{1}{2N+1}\right)\\\nonumber
    &-\Lambda\left(t-\frac{1}{2N+1}\right)-\Lambda\left(t\right)-\Lambda\left(t+\frac{1}{2N+1}\right), 
\end{align}
then we obtain 
\begin{align}
    |g_N(k,l)|=e^{(N+\frac{1}{2})v_N\left(\frac{2k+1}{2N+1},\frac{2l+1}{2N+1}\right)+O(\log (2N+1))}. 
\end{align}

We define the function
\begin{align}
    v(t,s)=\Lambda(t+s)+\Lambda(t-s)-3\Lambda\left(t\right).
\end{align}
Note that $\left(\frac{2k+1}{2N+1},\frac{2l+1}{2N+1}\right)\in D=\{(t,s)\in \mathbb{R}^2| 1< t+s< 2, 0< t-s<1, \frac{1}{2}< t<1\}$ for $0\leq k,l\leq N-1$. So we may assume 
the function $v(t,s)$ is defined on the region $D$.
We set
    \begin{align}
    D'_0=\{0.02 \leq t-s\leq 0.7, 1.02 \leq t+s\leq 1.7, 0.2 \leq s\leq 0.8,0.5\leq t\leq 0.909\}.
\end{align}

Let $\zeta_{\mathbb{R}}(p)$ be the real part of the critical value $V(p,t_0,s_0)$, see formula (\ref{formula-zetaR(p)}) for its precise definition.  

Then we have
\begin{lemma} \label{lemma-regionD'0}
The following domain
    \begin{align} \label{formula-domain}
        \left\{(t,s)\in D| v(t,s)> \frac{3.509}{2\pi }\right\}
    \end{align}
    is included in the region $D'_0$.
\end{lemma}
\begin{proof}
    See Appendix \ref{appendix-0} for a proof. 
\end{proof}

\begin{remark}
We can take $\varepsilon>0$ small enough (such as $\varepsilon=0.00001$), and set 
\begin{align}
    D'_\varepsilon=\left\{0.02+\varepsilon \leq t-s\leq 0.7-\varepsilon, 1.02+\varepsilon \leq t+s\leq 1.7-\varepsilon,\right.\\\nonumber \left. 0.2+\varepsilon \leq s\leq 0.8-\varepsilon,0.5+\varepsilon\leq t\leq 0.909-\varepsilon\right\},
\end{align}
then the region (\ref{formula-domain}) can also be included in the region $D'_{\varepsilon}$. 
\end{remark}



\begin{proposition} \label{prop-gkl}
For $p\geq 6$ and  $(\frac{2k+1}{2N+1},\frac{2l+1}{2N+1})\in D\setminus D'_0$,  we have
\begin{align}
    |g_{N}(k,l)|<O\left(e^{(N+\frac{1}{2})\left(\zeta_{\mathbb{R}}(p)-\epsilon\right)}\right)
\end{align}
for some sufficiently small $\epsilon>0$.
\end{proposition}
\begin{proof}
 For $(\frac{2k+1}{2N+1},\frac{2l+1}{2N+1})\in D\setminus D'_0$, since $v_{N}(t,s)$ converges uniformly to $v(t,s)$, by Lemma \ref{lemma-regionD'0}, we obtain
    \begin{align} \label{formula-gkl}
    |v_N\left(\frac{2k+1}{2N+1},\frac{2l+1}{2N+1}\right)|\leq \frac{3.509}{2\pi}<\frac{1}{2\pi}\left(v_8-\frac{49\pi^2}{64p^2}\right)  <\zeta_{\mathbb{R}}(p)-\epsilon
    \end{align}
    for some small $\epsilon>0$, where in the second $``<"$ of (\ref{formula-gkl}), we have used $p\geq 7$, and in the third $``<"$ of (\ref{formula-gkl}), we have used Lemma \ref{lemma-volumeestimate}.  
In particular, for $p=6$ we have 
$|v_N\left(\frac{2k+1}{2N+1},\frac{2l+1}{2N+1}\right)|\leq \frac{3.509}{2\pi}<\zeta_{\mathbb{R}}(6)-\epsilon$
    for some small $\epsilon>0$, since by a straightforward computation, $\zeta_{\mathbb{R}}(6)=\frac{3.5889}{2\pi }$. 
\end{proof}




For a sufficiently small $\varepsilon$, we take a smooth bump function $\psi$ on $\mathbb{R}^2$ such that
$\psi(t,s)=1$ on $(t,s)\in D'_{\varepsilon}$,  $0<\psi(t,s)<1$ on $(t,s)\in D'_0\setminus D'_{\varepsilon}$, $\psi(t,s)=0$ for $(t,s)\notin D'_0$.  
Let 
\begin{align}
    h_N(k,l)=\psi\left(\frac{2k+1}{2N+1},\frac{2l+1}{2N+1}\right)g_N(k,l).
\end{align}
Then  by Proposition \ref{prop-gkl},  for $p\geq 6$, we have 
\begin{align}  \label{formula-JN}
    J_N(\mathcal{K}_p;\xi_N)=\sum_{(k,l)\in \mathbb{Z}^2}h_N(k,l)+O\left(e^{(N+\frac{1}{2})\left(\zeta_{\mathbb{R}}(p)-\epsilon\right)}\right).
\end{align}

  We recall the Poisson summation formula \cite{SS03} in 2-dimensional case which states that for any function $h$ in the Schwartz space on $\mathbb{R}^2$, we have 
\begin{align} \label{formula-Poisson}
    \sum_{(k,l)\in \mathbb{Z}^2}h(k,l)=\sum_{(m,n)\in \mathbb{Z}^2}\hat{h}(m,n)
\end{align}
where 
\begin{align}
    \hat{h}(m,n)=\int_{\mathbb{R}^2}h(u,v)e^{-2\pi \sqrt{-1}mu-2\pi \sqrt{-1}nv}dudv.
\end{align}

Note that $h_N$ is $C^{\infty}$-smooth and equals zero outside $D'_0$, it is in the Schwartz space on $\mathbb{R}^2$. The Poisson summation formula (\ref{formula-Poisson}) holds for $h_N$. 

By using change of variables $t=\frac{2k+1}{2N+1}, s=\frac{2l+1}{2N+1}$, 
we compute the Fourier coefficient $\hat{h}_{N}(m,n)$ as follows
\begin{align}
    &\int_{\mathbb{R}^2}h_N(k,l)e^{-2\pi \sqrt{-1}mk-2\pi \sqrt{-1}nl}dkdl\\\nonumber
    &=(-1)^{m+n}\left(N+\frac{1}{2}\right)^2\\\nonumber
    &\cdot\int_{\mathbb{R}^2}h_N\left((N+\frac{1}{2})t-\frac{1}{2},(N+\frac{1}{2})s-\frac{1}{2}\right)e^{-2\pi \sqrt{-1}\frac{(2N+1)mt}{2}-2\pi \sqrt{-1}\frac{(2N+1)ns}{2}}dtds\\\nonumber
    &=(-1)^{m+n}\left(N+\frac{1}{2}\right)^2\frac{(-1)^pe^{\frac{\pi\sqrt{-1}}{4}}}{\sqrt{N+\frac{1}{2}}\sin\frac{\pi}{2N+1}}\\\nonumber
    &\cdot\int_{\mathbb{R}^2}\psi(t,s)\sin(2\pi s)e^{(N+\frac{1}{2})\left(V_N\left(p,t,s\right)-2\pi\sqrt{-1}mt-2\pi\sqrt{-1}ns\right)}dtds,
\end{align}
Therefore, applying the Poisson summation formula (\ref{formula-Poisson}) to (\ref{formula-JN}), we obtain
\begin{proposition} \label{prop-fouriercoeff}
For $p\geq 6$, the normalized $N$-th colored Jones polynomial of the twist knot $\mathcal{K}_{p}$ is given by 
\begin{align} \label{formula-fouriercoeff}
      J_N(\mathcal{K}_p;\xi_N)=\sum_{(m,n)\in \mathbb{Z}^2}\hat{h}_N(m,n)+O\left(e^{(N+\frac{1}{2})\left(\zeta_{\mathbb{R}}(p)-\epsilon\right)}\right),
\end{align}
where
  \begin{align}
      \hat{h}_N(m,n)=&(-1)^{p+m+n}e^{\frac{\pi\sqrt{-1}}{4}}\frac{\left(N+\frac{1}{2}\right)^{\frac{3}{2}}}{\sin\frac{\pi}{2N+1}} \int_{D'_0}\psi(t,s)\sin(2\pi s)e^{(N+\frac{1}{2})\left(V_N\left(p,t,s;m,n\right)\right)}dtds
\end{align}
with
\begin{align}
V_N\left(p,t,s;m,n\right)=V_N\left(p,t,s\right)-2\pi\sqrt{-1}mt-2\pi\sqrt{-1}ns,
\end{align}
and 
\begin{align}
    &V_N(p,t,s)\\\nonumber
    &=\pi \sqrt{-1}\left((2p+1)s^2-(2p+3)s+\left(\frac{4}{2N+1}-2\right)t-\frac{6p+7}{3(2N+1)^2}-\frac{1}{12}\right)\\\nonumber
    &+\frac{1}{2N+1}\varphi_N\left(t+s+\frac{1}{2N+1}-1\right)+\frac{1}{2N+1}\varphi_N\left(t-s+\frac{1}{2N+1}\right)\\\nonumber
    &-\frac{1}{2N+1}\varphi_N\left(t\right)-\frac{1}{2N+1}\varphi_N\left(t-\frac{1}{2N+1}\right)-\frac{1}{2N+1}\varphi_N\left(t+\frac{1}{2N+1}\right).
\end{align}
\end{proposition}


\begin{lemma}
The following identity holds
    \begin{align} \label{formula-potientalsym}
          V_{N}(p,t,1-s;m,n)
          =V_{N}(p,t,s;m,-n-2)-2\pi \sqrt{-1}(n+1). 
    \end{align}
\end{lemma}
\begin{proof}
    By a straightforward computation, we obtain the following identity
    \begin{align}
        &\pi\sqrt{-1}\left((2p+1)(1-s)^2-(2p+2n+3)(1-s)+\left(\frac{4}{2N+1}-2m-2\right)t-\frac{1}{12}\right)\\\nonumber
        &=\pi\sqrt{-1}\left((2p+1)s^2-(2p+2(-n-2)+3)s)+\left(\frac{4}{2N+1}-2m-2\right)t-\frac{1}{12}\right)\\\nonumber
        &-2\pi \sqrt{-1}(n+1).
    \end{align}
    which immediately gives the formula (\ref{formula-potientalsym}).
\end{proof}




\begin{proposition} \label{prop-hmn}
    For any $m,n\in \mathbb{Z}$, we have
    \begin{align}
        \hat{h}_{N}(m,-n-2)=(-1)^{n}\hat{h}_{N}(m,n).
    \end{align}
\end{proposition}
\begin{proof}
    Since 
    \begin{align}
      &\psi(t,s)\sin(2\pi s)\exp\left((N+\frac{1}{2})V_N\left(p,t,1-s;m,n\right)\right) \\\nonumber
      &=\psi(t,s)\sin(2\pi s)\exp\left((N+\frac{1}{2})\left(V_{N}\left(p,t,s;m,-n-2\right)-2\pi\sqrt{-1}(n+1)\right)\right)\\\nonumber
      &=\psi(t,s)\sin(2\pi s)\exp\left((N+\frac{1}{2})V_N\left(p,t,s;m,-n-2\right)\right)(-1)^{n+1},
    \end{align}
 and   we remark that we can choose the bump function $\psi(t,s)$ satisfying $\psi(t,s)=\psi(t,1-s)$ since the region $D'_0$ is symmetric with respect to the line $s=\frac{1}{2}$. 

Then we have
\begin{align}
    &\int_{D'_0}\psi(t,s)\sin(2\pi s)\exp\left((N+\frac{1}{2})V_N\left(p,t,s;m,-n-2\right)\right)dtds\\\nonumber
    &=(-1)^{n+1}\int_{D'_0}\psi(t,s)\sin(2\pi s)\exp\left((N+\frac{1}{2})V_N\left(p,t,1-s;m,n\right)\right)dtds\\\nonumber
    &=(-1)^{n}\int_{D'_0}\psi(t,\tilde{s})\sin(2\pi \tilde{s})\exp\left((N+\frac{1}{2})V_N\left(p,t,\tilde{s};m,n\right)\right)dtd\tilde{s},
\end{align}
where in the third ``=", we have let $\tilde{s}=1-s$.  It follows that 
\begin{align}
        \hat{h}_{N}(m,-n-2)=(-1)^{n}\hat{h}_{N}(m,n).
    \end{align}
\end{proof}
\begin{corollary}
    We have  
    \begin{align} \label{formula-bigcancel}
        \hat{h}_{N}(m,-1)=0,
    \end{align}
    and 
    \begin{align}
    \hat{h}_{N}(m,-2)=\hat{h}_{N}(m,0).    
    \end{align}
\end{corollary}


\begin{remark}
The case (\ref{formula-bigcancel})  is the big cancellation. The first situation of such phenomenon of ``Big cancellation" happened in quantum invariants is discovered in the Volume Conjecture of the Turaev-Viro invariants by Chen-Yang \cite{CY18}. 
The hidden reason behind that was found and described as a precise statement of symmetric property of asymptotics of quantum 6j-symbol which is on the Poisson Summation level by Chen-Murakami which is Conjecture 3 in \cite{CJ17}.
To the best of our knowledge, this is the first time that such a phenomenon of big cancellation on the Poisson Summation level on the case of colored Jones polynomial is proved.
\end{remark}
 





\section{Asympototic expansions} \label{Section-Asympoticexpansion}

%\begin{theorem}
%    Let $n\geq 1$ be an integer, and $\gamma$ an $n$-dimensional smooth compact real sub-manifold of $\mathbb{C}^n$ with connected % boundary. We denote $z=(z_1,...,z_n)\in \mathbb{C}^n$ and $dz=dz_1\cdots dz_n$. Let $g(z)$ and $S(z)$ be two complex-valued functions which are analytic on a domain $D$ such that $\gamma^n\subset D\subset \mathbb{C}^n$. We consider the integral 
%    \begin{align}
%        F(\lambda)=\int_{\gamma}g(z)\exp(\lambda S(z))dz
%    \end{align}
%    with real parameter $\lambda\in \mathbb{R}$.  If $\max_{z\in \gamma}\text{Re}(S(z))$ is attained only at a point $z^0$, which is an interior point of $\gamma$ and a simple saddle point of $S$ (i.e. $\nabla S(z^0)=0$ and $\det Hess(z^0)\neq 0$).
%    Then as $\lambda\rightarrow +\infty$, we have the following asymptotic expansion 
%\begin{align}
 %   F(\lambda)=\left(\frac{2\pi}{\lambda}\right)^{\frac{n}{2}}\frac{g(z^0)}{\sqrt{\det Hess(S)(z^0)}}\exp(\lambda S(z^0))\left(1+O\left(\frac{1}{r}\right)\right).
%\end{align}
 %   where the $c_k$ are complex numbers and the choice of branch for the root $\det Hess(S)(z^0)$ depends on the orientation of the contour $\gamma$. 
 %   In particular, 
%\begin{align}
 %   \lim_{\lambda\rightarrow +\infty}\frac{F(\lambda)}{\lambda}=S(z^0).
%\end{align}   
%\end{theorem}

 
The goal of this section is to estimate each Fourier coefficients $\hat{h}_N(m,n)$ appearing in Proposition \ref{prop-fouriercoeff}. In Section \ref{subsection-preparation}, we establish some results which will be used in the later subsections.  In Section \ref{subsection-mneq0} 
we estimate the Fourier coefficients $\hat{h}_N(m,n)$ that can be neglected. We remark that this subsection actually is equivalent to the corresponding subsection devoted to the verification of the assumption of the Poisson summation formula in Ohtsuki's original papers such as \cite{Oht16,Oht18}. The  final result we obtain in this subsection is the formula (\ref{formula-Poission-after}). In Sections \ref{subsection-m=-1np} and \ref{subsection-m=0np}, we estimate the remained Fourier coefficients and find out that only two terms will contribute to the final form of the asymptotic expansion. At last,  we finish the proof of Theorem \ref{theorem-main} in Section \ref{subsection-final}.

\subsection{Some preparations} \label{subsection-preparation}
The aim of this subsection is to make some preparations that will be used in the later subsections.
We consider the following potential function for the twist knot $\mathcal{K}_p$
\begin{align}
    &V(p,t,s; m,n)=\pi \sqrt{-1}\left((2p+1)s^2-(2p+3+2n)s-(2m+2)t\right)\\\nonumber
    &+\frac{1}{2\pi\sqrt{-1}}\left(\text{Li}_2(e^{2\pi\sqrt{-1}(t+s)})+\text{Li}_2(e^{2\pi\sqrt{-1}(t-s)})-3\text{Li}_2(e^{2\pi\sqrt{-1}t})+\frac{\pi^2}{6}\right).
\end{align}
We define the function
\begin{align}
    &f(p,t,X,s,Y;m,n)=Re V(p,t+X\sqrt{-1},s+Y\sqrt{-1};m,n),
\end{align}
which will also be denoted by $f(X,Y;m,n)$ for brevity in the following. 

We have
\begin{align}
    \frac{\partial f}{\partial X}&=Re\left(\sqrt{-1}\frac{\partial }{\partial t}V(p,t+X\sqrt{-1},s+Y\sqrt{-1};m,n)\right)\\\nonumber
    &=-Im(-(2m+2)\pi \sqrt{-1}+3\log(1-x)-\log(1-xy)-\log(1-x/y))\\\nonumber
    &=-3arg(1-x)+arg(1-xy)+arg(1-x/y)+(2m+2)\pi,
\end{align}
and
\begin{align}
    \frac{\partial f}{\partial Y}&=Re\left(\sqrt{-1}\frac{\partial }{\partial s}V(p,t+X\sqrt{-1},s+Y\sqrt{-1};m,n)\right)\\\nonumber
    &=-Im(-(2p+3+2n)\pi \sqrt{-1}+(4p+2)\pi\sqrt{-1}s\\\nonumber
    &-\log(1-xy)+\log(1-x/y))\\\nonumber
    &=arg(1-xy)-arg(1-x/y)+(2p+3+2n)\pi-(4p+2)\pi s,
\end{align}
where we put $x=e^{2\pi\sqrt{-1}(t+X\sqrt{-1})}$ and $y=e^{2\pi\sqrt{-1}(s+\sqrt{-1}Y)}$.

Since $\frac{dx}{dX}=-2\pi x$, we compute
\begin{align}
    \frac{\partial^2 f}{\partial X^2}&=-Im\frac{\partial }{\partial X}\left(3\log(1-x)-\log(1-xy)-\log(1-x/y)\right)\\\nonumber
    &=-Im\left(\left(-\frac{3}{1-x}+\frac{y}{1-xy}+\frac{1/y}{1-x/y}\right)\frac{\partial x}{\partial X}\right)\\\nonumber
    &=2\pi Im\left(-\frac{3x}{1-x}+\frac{xy}{1-xy}+\frac{x/y}{1-x/y}\right)\\\nonumber
    &=2\pi Im\left(-\frac{3}{1-x}+\frac{1}{1-xy}+\frac{1}{1-x/y}\right).
\end{align}

Furthermore, we have
\begin{align}
     \frac{\partial^2 f}{\partial X\partial Y}&=2\pi Im\left(\frac{1}{1-xy}-\frac{1}{1-x/y}\right)
\end{align}
and 
\begin{align}
    \frac{\partial^2 f}{\partial Y^2}&=2\pi Im\left(\frac{1}{1-xy}+\frac{1}{1-x/y}\right).
\end{align}
Therefore, the Hessian matrix of $f$ is presented by 

\begin{align}
    2\pi \begin{pmatrix}
     3a+b+c & b-c \\
     b-c & b+c
    \end{pmatrix}
\end{align}

where we put 
\begin{align}
    a=-Im \frac{1}{1-x}, \ b=Im\frac{1}{1-xy}, \ c=Im \frac{1}{1-x/y}. 
\end{align}
More precisely, by direct computations, we obtain
\begin{align}
    3a&=-\frac{3\sin 2\pi t}{e^{2\pi X}+e^{-2\pi X}-2\cos(2\pi t)}, \\ \nonumber
    b&=\frac{\sin(2\pi (t+s))}{e^{2\pi(X+Y)}+e^{-2\pi(X+Y)}-2\cos (2\pi(t+s))}, \\\nonumber
    c&=\frac{\sin(2\pi (t-s))}{e^{2\pi(X-Y)}+e^{-2\pi(X-Y)}-2\cos (2\pi(t-s))}.
\end{align}
So if $\frac{1}{2}<t<1$, $1<t+s<\frac{3}{2}$ and $0<t-s<\frac{1}{2}$, we have that 
\begin{align}
    a>0, \ b>0, c>0,
\end{align}
which implies that the Hessian matrix of $f$ with respect to $X,Y$ is positive definite, i.e. we obtain 
\begin{lemma} \label{lemma-HessXY}
    On the region $D_{H}=\{(t,s)\in \mathbb{R}^2|\frac{1}{2}<t<1, 1<t+s<\frac{3}{2}, 0<t-s<\frac{1}{2}\}$, the Hessian matrix of $f$ is positive definite. 
\end{lemma}

\begin{lemma} \label{lemma-Vr}
    For any $L>0$, in the region 
    \begin{align}
        \{(t,s)\in  \mathbb{C}^2| (Re(t),Re(s))\in D'_0, |\text{Im}\  t|<L, |\text{Im} \ s|<L\} 
    \end{align}
    we have
    \begin{align}
        V_N(p,t,s;m,n)
        &=V(p,t,s;m,n)-\frac{1}{2N+1}\left(\log(1-e^{2\pi\sqrt{-1}(t+s)})\right.\\\nonumber
   &\left.+\log(1-e^{2\pi\sqrt{-1}(t-s)})-4\pi\sqrt{-1}t\right)+\frac{w_N(t,s)}{(2N+1)^2},
    \end{align}
    with $|w_N(t,s)|$ bounded from above by a constant independent of $N$. 
\end{lemma}
\begin{proof}
By using Taylor expansion, together with Lemma \ref{lemma-varphixi3}, we have 
\begin{align}
    &\varphi_N\left(t+s-1+\frac{1}{2N+1}\right)\\\nonumber
    &=\varphi_N(t+s-1)+\varphi'_N(t+s-1)\frac{1}{2N+1}\\\nonumber&+\frac{\varphi''_{N}(t+s-1)}{2}\frac{1}{(2N+1)^2}+O\left(\frac{1}{(2N+1)^2}\right)\\\nonumber
    &=\frac{N+\frac{1}{2}}{2\pi\sqrt{-1}}\text{Li}_2(e^{2\pi\sqrt{-1}(t+s)})-\frac{\pi\sqrt{-1}}{6(2N+1)}\frac{e^{2\pi\sqrt{-1}(t+s)}}{1-e^{2\pi\sqrt{-1}(t+s)}}\\\nonumber
    &-\frac{1}{2}\log(1-e^{2\pi\sqrt{-1}(t+s)})+\frac{\pi\sqrt{-1}}{2(2N+1)}\frac{e^{2\pi\sqrt{-1}(t+s)}}{1-e^{2\pi\sqrt{-1}(t+s)}}+O\left(\frac{1}{(2N+1)^2}\right)
    \end{align}

Then, we expand $\varphi_N\left(t-s+\frac{1}{2N+1}\right)$, $\varphi_N\left(t-\frac{1}{2N+1}\right)$ and $\varphi_N\left(t+\frac{1}{2N+1}\right)$ similarly,  we obtain 
\begin{align}
   &V_N(p,t,s;m,n)=V(p,t,s;m,n)\\\nonumber
   &-\frac{1}{2N+1}\left(\log(1-e^{2\pi\sqrt{-1}(t+s)})+\log(1-e^{2\pi\sqrt{-1}(t-s)})-4\pi\sqrt{-1}t\right)\\\nonumber
   &-\frac{\pi\sqrt{-1}}{3(2N+1)^2}\left(-2\frac{e^{2\pi\sqrt{-1}(t+s)}}{1-e^{2\pi\sqrt{-1}(t+s)}}-2\frac{e^{2\pi\sqrt{-1}(t-s)}}{1-e^{2\pi\sqrt{-1}(t-s)}}+3\frac{e^{2\pi\sqrt{-1}t}}{1-e^{2\pi\sqrt{-1}t}}+6p+7\right)\\\nonumber
   &+O\left(\frac{1}{(2N+1)^3}\right). 
\end{align}
Finally, we let 
\begin{align}
    &w_N(t,s)\\\nonumber
    &=-\frac{\pi\sqrt{-1}}{3}\left(-2\frac{e^{2\pi\sqrt{-1}(t+s)}}{1-e^{2\pi\sqrt{-1}(t+s)}}-2\frac{e^{2\pi\sqrt{-1}(t-s)}}{1-e^{2\pi\sqrt{-1}(t-s)}}+3\frac{e^{2\pi\sqrt{-1}t}}{1-e^{2\pi\sqrt{-1}t}}+6p+7\right)\\\nonumber
    &+O\left(\frac{1}{(2N+1)^3}\right),
\end{align}
and we finish the proof of Lemma \ref{lemma-Vr}. 
\end{proof}
We consider the critical point of $V(p,t,s)$, which is given by the solution of the following equations

\begin{align}  \label{equation-critical1}
    \frac{\partial V(p,t,s)}{\partial t}&=-2\pi\sqrt{-1}+3\log(1-e^{2\pi\sqrt{-1}t})\\\nonumber
    &-\log(1-e^{2\pi\sqrt{-1}(t+s)})-\log(1-e^{2\pi\sqrt{-1}(t-s)})=0,
\end{align}
\begin{align} \label{equation-critical2}
    \frac{\partial V(p,t,s)}{\partial s}&=(4p+2)\pi\sqrt{-1}s-(2p+3)\pi\sqrt{-1}\\\nonumber
    &-\log(1-e^{2\pi\sqrt{-1}(t+s)})+\log(1-e^{2\pi\sqrt{-1}(t-s)})=0. 
\end{align}




\begin{proposition}  \label{prop-critical}
 The critical point equations (\ref{equation-critical1}), (\ref{equation-critical2}) has a unique solution $(t_0,s_0)=(t_{0R}+X_0\sqrt{-1},s_{0R}+Y_0\sqrt{-1})$ with $(t_{0R},s_{0R})$ lies in the region $D'_0$.
\end{proposition}
\begin{proof}
    See Appendix \ref{appendix-1} for a proof.   
\end{proof}
Now we set 
$\zeta(p)$ to be the critical value of the potential function $V(p,t,s)$, i.e. 
\begin{align}
    \zeta(p)=V(p,t_0,s_0),
\end{align}
and set
\begin{align} \label{formula-zetaR(p)}
    \zeta_\mathbb{R}(p)=Re \zeta(p)=Re V(p,t_0,s_0). 
\end{align}

\begin{lemma} \label{lemma-volume}
For $p\geq 2$, we have the following formula
    \begin{align}
        2\pi \zeta(p)=vol(S^3\setminus \mathcal{K}_p)+\sqrt{-1}cs(\mathcal{S}^3\setminus \mathcal{K}_p)-(p+5)\pi^2\sqrt{-1}.
    \end{align}
\end{lemma}
\begin{proof}
    See Appendix \ref{appendix-volume} for a proof. 
\end{proof}

\begin{lemma} \label{lemma-volumeestimate}
When $p\geq 6$, we have the following estimation for $\zeta_{\mathbb{R}}(p)$
\begin{align}
    2\pi \zeta_{\mathbb{R}}(p)\geq v_8-\frac{49\pi^2}{64}\frac{1}{p^2},
\end{align}
where $v_8$ denotes the volume of the ideal regular octahedron, i.e. $v_8\approx 3.66386$.
\end{lemma}
\begin{proof}
    See Appendix \ref{appendix-2} for a proof. 
\end{proof}









We compute the Hessian matrix of the potential function $V(p,t,s)$ as follows. By formulas (\ref{equation-critical1}) and (\ref{equation-critical2}), we obtain 
\begin{align}
    \frac{\partial^2 V}{\partial t^2}=6\pi \sqrt{-1}\frac{-e^{2\pi\sqrt{-1}t}}{1-e^{2\pi\sqrt{-1}t}}+2\pi \sqrt{-1}\frac{e^{2\pi\sqrt{-1}(t+s)}}{1-e^{2\pi\sqrt{-1}(t+s)}}+2\pi \sqrt{-1}\frac{e^{2\pi\sqrt{-1}(t-s)}}{1-e^{2\pi\sqrt{-1}(t-s)}},
\end{align}
\begin{align}
    \frac{\partial^2 V}{\partial s^2}=(4p+2)\pi \sqrt{-1}+2\pi \sqrt{-1}\frac{e^{2\pi\sqrt{-1}(t+s)}}{1-e^{2\pi\sqrt{-1}(t+s)}}+2\pi \sqrt{-1}\frac{e^{2\pi\sqrt{-1}(t-s)}}{1-e^{2\pi\sqrt{-1}(t-s)}},
\end{align}
\begin{align}
    \frac{\partial^2 V}{\partial t\partial s}=2\pi \sqrt{-1}\frac{e^{2\pi\sqrt{-1}(t+s)}}{1-e^{2\pi\sqrt{-1}(t+s)}}-2\pi \sqrt{-1}\frac{e^{2\pi\sqrt{-1}(t-s)}}{1-e^{2\pi\sqrt{-1}(t-s)}}.
\end{align}

Let $x=e^{2\pi\sqrt{-1}t}$ and $y=e^{2\pi \sqrt{-1}s}$, then we obtain 
\begin{align} \label{formula-hessV}
    &\det(Hess(V))\\\nonumber
    &=\frac{\partial^2 V}{\partial t^2}\frac{\partial^2 V}{\partial s^2}-\left(\frac{\partial^2 V}{\partial t\partial s}\right)^2\\\nonumber
    &=(2\pi\sqrt{-1})^2\left(\frac{-3(2p+1)}{\frac{1}{x}-1}+\frac{2p+1}{\frac{1}{xy}-1}+\frac{2p+1}{\frac{1}{x/y}-1}-\frac{3}{(\frac{1}{x}-1)(\frac{1}{xy}-1)}\right.\\\nonumber
    &\left.-\frac{3}{(\frac{1}{x}-1)(\frac{1}{x/y}-1)}+\frac{4}{(\frac{1}{xy}-1)(\frac{1}{x/y}-1)}\right).
\end{align}

For convience, we let 
\begin{align}
    H(p,x,y)&=\frac{-3(2p+1)}{\frac{1}{x}-1}+\frac{2p+1}{\frac{1}{xy}-1}+\frac{2p+1}{\frac{1}{x/y}-1}-\frac{3}{(\frac{1}{x}-1)(\frac{1}{xy}-1)}\\\nonumber
    &-\frac{3}{(\frac{1}{x}-1)(\frac{1}{x/y}-1)}+\frac{4}{(\frac{1}{xy}-1)(\frac{1}{x/y}-1)}.
\end{align}


\subsection{Fourier coefficients $\hat{h}_N(m,n)$ that can be neglected} \label{subsection-mneq0}
Motivated by Lemma \ref{lemma-Li2}, we introduce the following  function for $(t,s)\in D$.

\begin{equation} \label{eq:2}
F(X,Y;m,n)=\left\{ \begin{aligned}
         &0  &  \ (\text{if} \ X+Y\geq 0) \\
         &\left((t+s)-\frac{3}{2}\right)(X+Y) & \ (\text{if} \ X+Y<0)
                          \end{aligned} \right.
                          \end{equation}
\begin{equation*}
 +\left\{ \begin{aligned}
         &0  &  \ (\text{if} \ X-Y\geq 0) \\
         &\left((t-s)-\frac{1}{2}\right)(X-Y) & \ (\text{if} \ X-Y<0)
                          \end{aligned} \right.
                          \end{equation*}
\begin{equation*}
+\left\{ \begin{aligned}
         &0  &  \ (\text{if} \ X\geq 0) \\
         &\left(\frac{3}{2}-3t\right)X & \ (\text{if} \ X<0)
                          \end{aligned} \right.  +\left(p+\frac{3}{2}+n-(2p+1)s\right)Y+(m+1)X
                          \end{equation*}
where we have used $t+s-\frac{3}{2}$ instead of $t+s-\frac{1}{2}$ in the first summation since in our situation $1< t+s<2$.                       


  Note that $F(X,Y;m,n)$ is a piecewise linear function,  we subdivide the plane $\{(X,Y)\in \mathbb{R}^2\}$ into eight regions to discuss the asymptotic property of this function.

We study the conditions such that $F(X,Y,m,n)$ has the following property: 
\begin{align} \label{formula-F(X,Y)}
    F(X,Y;m,n)\rightarrow \infty \ \text{as} \ X^2+Y^2\rightarrow +\infty.
\end{align}
  
(I) $X\geq Y\geq 0$, then
\begin{align}
    F(X,Y;m,n)&=\left(\left(p+\frac{3}{2}+n\right)-(2p+1)s\right)Y+(m+1)X.
\end{align}
 For $m\leq -2$, then $ F(X,Y;m,n)\rightarrow -\infty$ as $X\rightarrow +\infty$. 
For $m=-1$, when $s< \frac{p+\frac{3}{2}+n}{2p+1}$, then the property (\ref{formula-F(X,Y)}) holds. 
For $m\geq 0$, when $s<\frac{p+\frac{3}{2}+n+(m+1)}{2p+1}$, then the property (\ref{formula-F(X,Y)}) holds. 


(II) $Y\geq X\geq 0$, then
\begin{align}
     F(X,Y;m,n)=\left(t-s+\frac{1}{2}+m\right)X+(p+2+n-2ps-t)Y.
\end{align}
When $t+2ps<p+2+n$, we have
\begin{align}
     F(X,Y;m,n)&=\left(t-s+\frac{1}{2}+m\right)X+(p+2+n-2ps-t)Y\\\nonumber
    &\geq \left(p+\frac{5}{2}+m+n-(2p+1)s\right)X.
\end{align}
Hence, the property (\ref{formula-F(X,Y)}) holds in this case if  $\left(p+\frac{5}{2}+m+n\right)-(2p+1)s>0$. 


(III) $X+Y\geq 0$ and $X\leq 0$, then
\begin{align}
     F(X,Y;m,n)=\left(2-2t-s+m\right)X+(p+2+n-2ps-t)Y.
\end{align}
When $t+2ps<p+2+n$, then 
\begin{align}
     F(X,Y;m,n)&\geq \left(2-2t-s+m\right)X+(p+2+n-2ps-t)(-X) \\\nonumber
    &=(p+n-m+t-(2p-1)s)(-X),
\end{align}
if $p+n-m+t-(2p-1)s>0$, 
we obtain that the property (\ref{formula-F(X,Y)}) holds in this case. 

(IV) $X+Y\leq 0$ and $Y\geq 0$, then 
\begin{align}
    F(X,Y;m,n)=\left(\frac{1}{2}-t+m\right)X+\left(p+\frac{1}{2}+n-(2p-1)s\right)Y.
\end{align}
Since $\frac{1}{2}<t<1$, if $m\geq 1$, then $\frac{1}{2}-t+m>\frac{1}{2}$, by $-X\geq Y\geq 0$, we obtain 
$ F(X,Y;m,n)\rightarrow -\infty$ when $X\rightarrow -\infty$. 

If $m\leq 0$, then 
\begin{align}
     F(X,Y;m,n)&=\left(-\frac{1}{2}+t-m\right)(-X)+\left(p+\frac{1}{2}+n-(2p-1)s\right)Y\\\nonumber
    &\geq (p+n-m+t-(2p-1)s)Y.
\end{align}
when $p+n-m+t-(2p-1)s>0$, we can see that the property (\ref{formula-F(X,Y)}) holds in this case. 



(V) $X-Y\leq 0$ and $Y\leq 0$, then 
\begin{align}
     F(X,Y;m,n)=\left(\frac{1}{2}-t+m\right)X+\left(p+\frac{1}{2}+n-(2p-1)s\right)Y.
\end{align}
Since $\frac{1}{2}<t<1$, if $m\geq 1$, then $\frac{1}{2}-t+m>\frac{1}{2}$, by $-X\geq -Y\geq 0$, we obtain 
$ F(X,Y;m,n)\rightarrow -\infty$ when $X\rightarrow -\infty$. 

If $m\leq 0$, then
\begin{align}
     F(X,Y;m,n)&=\left(-\frac{1}{2}+t-m\right)(-X)+\left(p+\frac{1}{2}+n-(2p-1)s\right)Y\\\nonumber
    &\geq \left(-\frac{1}{2}+t-m\right)(-Y)+\left(p+\frac{1}{2}+n-(2p-1)s\right)Y\\\nonumber
    &=(-p-1-m-n+t+(2p-1)s)(-Y),
\end{align}
if $t+(2p-1)s-p-1-m-n>0$, it follows that the property (\ref{formula-F(X,Y)}) holds in this case. 

(VI) $X-Y\geq 0$ and $X\leq 0$, then 
\begin{align}
     F(X,Y;m,n)=\left(1+m+s-2t\right)X+\left(p+n+t-2ps\right)Y,
\end{align}
when $p+n+t-2ps<0$, and since $-Y\geq -X\geq 0$,  
 \begin{align}
     F(X,Y;m,n)&\geq \left(1+m+s-2t\right)X-\left(p+n-2ps+t\right)(-X)\\\nonumber
    &=(-p-1-m-n+t+(2p-1)s)(-X),
\end{align}
if $t+(2p-1)s-p-1-m-n>0$, it follows that the property (\ref{formula-F(X,Y)}) holds in this case. 



(VII) $X+Y\leq 0$ and $X\geq 0$, then 
   \begin{align}
    F(X,Y;m,n)=\left(t+s-\frac{1}{2}+m\right)X+\left(p+n-2ps+t\right)Y.
\end{align}

When $m=0$, then $t+s-\frac{1}{2}>0$, if $p+n-2ps+t\leq 0$, by $Y\leq 0$,   
it follows that the property (\ref{formula-F(X,Y)}) holds in this case, if $p+n-2ps+t>0$, by $-Y\geq X\geq 0$, 
if follows that $ F(X,Y;m,n)\rightarrow -\infty$, when $Y\rightarrow -\infty$. 

When $m=-1$, $F(X,Y;-1,n)=\left(t+s-\frac{3}{2}\right)X+\left(p+n-2ps+t\right)Y$, if $p+n-2ps+t<0$, then 
\begin{align}
    F(X,Y;-1,n)&\geq \left(t+s-\frac{3}{2}\right)X -\left(p+n-2ps+t\right)X\\\nonumber
    &=\left((2p+1)s-p-\frac{3}{2}-n\right)X,
\end{align}
it follows that if $(2p+1)s-p-\frac{3}{2}-n>0$, then the property (\ref{formula-F(X,Y)}) holds in this case.


(VIII) $X+Y\geq 0$ and $Y\leq 0$, then 
   \begin{align}
     F(X,Y;m,n)=(m+1)X+\left(p+\frac{3}{2}+n-(2p+1)s\right)Y,
\end{align}
When $m\leq -2$, then $ F(X,Y;m,n)\rightarrow -\infty$ as $X\rightarrow +\infty$. 
When $m=-1$, if $p+\frac{3}{2}+n-(2p+1)s>0$, then $F(X,Y)\rightarrow -\infty$ as $Y\rightarrow -\infty$. If $p+\frac{3}{2}+n-(2p+1)s<0$, then $ F(X,Y;m,n)\rightarrow +\infty$ as $Y\rightarrow -\infty$. When $m\geq 0$, by $X\geq -Y$


   \begin{align}
     F(X,Y;m,n)&\geq (m+1)(-Y)+\left(p+\frac{3}{2}+n-(2p+1)s\right)Y\\\nonumber
    &=\left(m-p-\frac{1}{2}-n+(2p+1)s\right)(-Y),
\end{align}
if $m-p-\frac{1}{2}-n+(2p+1)s>0$, we obtain that the property (\ref{formula-F(X,Y)}) holds in this case.



We obtain
\begin{lemma} \label{lemma-m-2m1}
For any $(t,s)\in D'_0$, 

(i) when $m\leq -2$, for a fixed  $Y\in \mathbb{R}$, $f(X,Y;m,n)$ is a decreasing  function with respect to $X$, and
\begin{align}
    f(X,Y;m,n)\rightarrow -\infty \ \text{as } \ X\rightarrow +\infty.
\end{align}

(ii) when $m\geq 1$, for a fixed  $Y\in \mathbb{R}$, $f(X,Y;m,n)$ is an increasing  function with respect to $X$, and
\begin{align}
    f(X,Y;m,n)\rightarrow -\infty \ \text{as} \ X\rightarrow -\infty. 
\end{align}
\end{lemma}
\begin{proof}
       From the above discussions, more precisely, by case (I), 
    we obtain that, for $m\leq -2$, then  $F(X,Y;m,n)\rightarrow -\infty$ as $X\rightarrow +\infty$. By Lemma \ref{lemma-Li2}, we have 
       \begin{align}
           2\pi F(X,Y;m,n)-C<f(X,Y;m,n)<  2\pi F(X,Y;m,n)+C
       \end{align}
       for some constant $C$. Hence, we obtain (i).  
    Similarly, for $m\geq 1$,  by case (IV), we obtain (ii). 
\end{proof}
 





\begin{proposition} \label{prop-mgeq1-leg-2}
    When $m\leq -2$ or $m\geq 1$, then for any $n\in \mathbb{Z}$, we have 
    \begin{align}
        \hat{h}_{N}(m,n)&=(-1)^{p+m+n}e^{\frac{\pi\sqrt{-1}}{4}}\frac{\left(N+\frac{1}{2}\right)^{\frac{3}{2}}}{\sin\frac{\pi}{2N+1}}\int_{D'_0}\psi(t,s)\sin(2\pi s)e^{(N+\frac{1}{2})V_N(p,t,s;m,n)}dtds\\\nonumber
        &=O\left(e^{(N+\frac{1}{2})(\zeta_{\mathbb{R}}(p)-\epsilon)}\right).
    \end{align}
\end{proposition}
\begin{proof}
    We note that $V_N(p,t,s;m,n)$ uniformly converges to $V(p,t,s;m,n)$, we show the existence of a homotopy $D'_{\delta}$ ($0\leq \delta\leq \delta_0$) between $D'_0$ and $D'_{\delta_0}$ and such that 
    \begin{align}
        D'_{\delta_0}\subset \{(t,s)\in \mathbb{C}^2| Re V(p,t,s;m,n)<\zeta_{\mathbb{R}}(p)-\epsilon\},  \label{formula-Ddelta}\\ 
        \partial D'_{\delta}\subset \{(t,s)\in \mathbb{C}^2| Re V(p,t,s;m,n)<\zeta_{\mathbb{R}}(p)-\epsilon\}, \label{formula-partialD}
    \end{align}

    For each fixed $(t,s)\in D'_0$,  we move $(X,Y)$ from $(0,0)$ along the flow $(-\frac{\partial f}{\partial X},0)$. Then by Lemma \ref{lemma-m-2m1}, the value of $Re V(p,t+X\sqrt{-1},s+Y\sqrt{-1};m,n)$ monotonically decreases and it goes to $-\infty$. As for (\ref{formula-partialD}), since $\partial D'_0\subset \{(t,s)\in \mathbb{C}^2| Re V(p,t,s)<\zeta_{\mathbb{R}}(p)-\epsilon\}$ and the value of $Re V$ monotonically decreases, hence (\ref{formula-partialD}) holds. As for (\ref{formula-Ddelta}), since the value of $Re V$ uniformly goes to $-\infty$ by Lemma \ref{lemma-m-2m1}, (\ref{formula-Ddelta}) holds for sufficiently large $\delta_0$. Therefore, such a required homotopy exists.  
\end{proof}

\begin{lemma} \label{lemma-m-1}
    For any $(t,s)\in D'_0$, when $m=-1$, 
 if $s>\frac{p+\frac{3}{2}+n}{2p+1}$, we have 
    \begin{align}
         f(X,Y;-1,n)\rightarrow -\infty\ \text{as}  \ Y\rightarrow  +\infty.
    \end{align}
if $s<\frac{p+\frac{3}{2}+n}{2p+1}$, we have 
    \begin{align}
         f(X,Y;-1,n)\rightarrow -\infty\ \text{as}  \ Y\rightarrow  -\infty.
    \end{align}
\end{lemma}
\begin{proof}
For $m=-1$, if $s>\frac{p+\frac{3}{2}+n}{2p+1}$, by case (I), we obtain 
\begin{align}
    f(X,Y,-1,n)\rightarrow -\infty \ \text{as} \ Y\rightarrow +\infty .
\end{align}
if $s<\frac{p+\frac{3}{2}+n}{2p+1}$, by case (VIII), we obtain 
\begin{align}
    f(X,Y,-1,n)\rightarrow -\infty \ \text{as} \ Y\rightarrow -\infty. 
\end{align}
\end{proof}

\begin{proposition} \label{prop-m-1}
    When  $m=-1$, then for any $n\geq p-1$ or $n\leq -(p+1)$, we have 
    \begin{align}
        \hat{h}_{N}(m,n)&=(-1)^{p+m+n}e^{\frac{\pi\sqrt{-1}}{4}}\frac{\left(N+\frac{1}{2}\right)^{\frac{3}{2}}}{\sin\frac{\pi}{2N+1}}\int_{D'_0}\psi(t,s)\sin(2\pi s)e^{(N+\frac{1}{2})V_N(p,t,s;m,n)}dtds\\\nonumber
        &=O\left(e^{(N+\frac{1}{2})(\zeta_{\mathbb{R}}(p)-\epsilon)}\right).
    \end{align}
\end{proposition}
\begin{proof}
    Since $(t,s)\in D'_0$, we have $0.2<s<0.8$. For $n\geq p-1$,  we have that 
    \begin{align}
        \frac{p+\frac{3}{2}+n}{2p+1}>\frac{2p+\frac{1}{2}}{2p+1}>s,
    \end{align} 
    since $p\geq 6$. 
    by Lemma \ref{lemma-m-1}, we obtain
    \begin{align}
         f(X,Y;-1,n)\rightarrow -\infty\ \text{as}  \ Y\rightarrow  +\infty.
    \end{align}
For $n\leq -(p+1)$,  we have that $\frac{p+\frac{3}{2}+n}{2p+1}<\frac{\frac{1}{2}}{2p+1}<s$ since $p\geq 6$. by Lemma \ref{lemma-m-1},  we obtain
    \begin{align}
         f(X,Y;-1,n)\rightarrow -\infty\ \text{as}  \ Y\rightarrow  -\infty.
    \end{align}
    Then,  for each fixed $(t,s)\in D'_0$,  we move $(X,Y)$ from $(0,0)$ along the flow $(0,-\frac{\partial f}{\partial Y})$.  The value of $Re V(p,t+X\sqrt{-1},s+Y\sqrt{-1};m,n)$ monotonically decreases and it goes to $-\infty$. So we can construct the homotopy  similar to the proof of Proposition \ref{prop-mgeq1-leg-2} and finish the proof of Proposition \ref{prop-m-1}.
\end{proof}

Now let us consider the Fourier coefficients with $m=0$.  After the discussion of asymptotic behavior for the function $F(X,Y;-1,n)$,    we introduce the following region for $n\in \mathbb{Z}$,
\begin{small}
    \begin{align}
    U_n=\left\{(t,s)\in D'_0|\frac{p+n+1-t}{2p-1}<s<\frac{p+n+t}{2p-1}, \frac{p+n+t}{2p}<s<\frac{p+2+n-t}{2p}\right\}.
\end{align}
\end{small}
We have
\begin{lemma} \label{lemma-UnnotUn}
For any $(t,s)\in D'_0$, 

(i) when $(t,s)\in U_n$, we have 
\begin{align}
    f(X,Y;n)\rightarrow \infty \ \text{as } \ X^2+Y^2\rightarrow +\infty.
\end{align}

(ii) when $(t,s)\notin U_n$, 
\begin{align}
    f(X,Y;n)\rightarrow -\infty \ \text{in some directions of} \  X^2+Y^2\rightarrow +\infty.
\end{align}
\end{lemma}
\begin{proof}
    See Appendix \ref{appendix-LemmaUn} for a proof. 
\end{proof}


\begin{lemma} \label{lemma-topbottom}
(i) The top point of $U_n$ is given by $(\frac{3p-2-n}{4p-1},\frac{1}{2}+\frac{5+4n}{2(4p-1)})$, the bottom point of $U_n$ is given by $(\frac{3p+n}{4p-1},\frac{1}{2}+\frac{3+4n}{2(4p-1)})$,

(ii) For $p\geq 6$, $U_0\subset D_{H}$.   
\end{lemma}
\begin{proof}
    Solving the linear equations
\begin{equation} 
\left\{ \begin{aligned}
         s&=\frac{p+n+t}{2p-1} \\
                s&=\frac{p+2+n-t}{2p}
                          \end{aligned} \right.
                          \end{equation}
and 
\begin{equation} \label{eq:2}
\left\{ \begin{aligned}
         s&=\frac{p+n+1-t}{2p-1} \\
                s&=\frac{p+n+t}{2p}
                          \end{aligned} \right.
                          \end{equation}
                          respectively, we obtain (i).
Solving the equations 
\begin{equation} 
\left\{ \begin{aligned}
         s=\frac{p+2-t}{2p} \\
               t+s=\frac{3}{2}
                          \end{aligned} \right.
                          \end{equation}
we obtain $t=\frac{2p-2}{2p-1}=1-\frac{1}{2p-1}>0.909$ since $p\geq 6$. Hence $U_0\subset D_{H}$. 
\end{proof}

\begin{proposition} \label{prop-m-0}
    When  $m=0$, then for any $n\geq p-1$ or $n\leq -(p+1)$, we have 
    \begin{align}
        \hat{h}_{N}(m,n)&=(-1)^{p+m+n}e^{\frac{\pi\sqrt{-1}}{4}}\frac{\left(N+\frac{1}{2}\right)^{\frac{3}{2}}}{\sin\frac{\pi}{2N+1}}\int_{D'_0}\psi(t,s)\sin(2\pi s)e^{(N+\frac{1}{2})V_N(p,t,s;m,n)}dtds\\\nonumber
        &=O\left(e^{(N+\frac{1}{2})(\zeta_{\mathbb{R}}(p)-\epsilon)}\right).
    \end{align}
\end{proposition}
\begin{proof}
    Since $(t,s)\in D'_0$, we have $0<s<1$. For $n\geq p-1$,  we have 
    \begin{align}
         \frac{1}{2}+\frac{3+4n}{2(4p-1)}\geq 1>s,
    \end{align}
and for $n\leq -(p+1)$, we have 
 \begin{align}
         \frac{1}{2}+\frac{5+4n}{2(4p-1)}\leq 0<s.
    \end{align}
So by Lemma \ref{lemma-topbottom}, we know that $U_{n}=\emptyset$. Therefore, by Lemma \ref{lemma-UnnotUn}, for 
$(t,s)\in D'_0$, we have
\begin{align}
    f(X,Y;n)\rightarrow -\infty \ \text{in some directions of } \ X^2+Y^2\rightarrow +\infty. 
\end{align}
Now, similar to the proof of Proposition \ref{prop-mgeq1-leg-2}, we can finish the proof of Proposition \ref{prop-m-0} 
\end{proof}






\begin{remark}
    Proposition \ref{prop-mgeq1-leg-2}, Proposition \ref{prop-m-1} and Proposition \ref{prop-m-0} show that the Fourier coefficients $\hat{h}_{N}(m,n)$  with $m,n$ satisfying the conditions in those Propositions can be neglected when we study the asymptotic expansion.  We can also prove Proposition \ref{prop-mgeq1-leg-2} and Proposition \ref{prop-m-1}  by using the method as showed in Ohtsuki's original paper \cite{Oht16,Oht18} (cf. the section of verifying the assumption of the Poisson summation formula for $V$).  
Hence, by formula (\ref{formula-fouriercoeff}), we obtain 
\begin{align} \label{formula-Poission-after}
      J_N(\mathcal{K}_p;\xi_N)=\sum_{n=-p}^{p-2}\hat{h}_N(-1,n)+\sum_{n=-p}^{p-2}\hat{h}_{N}(0,n)+O\left(e^{(N+\frac{1}{2})\left(\zeta_{\mathbb{R}}(p)-\epsilon\right)}\right).
\end{align}
So in the following, we only need to investigate the Fourier coefficients $\hat{h}_N(-1,n)$ and $\hat{h}_N(0,n)$ with $-p\leq n\leq p-2$. Note that, in Ohtsuki's work \cite{Oht16,Oht17,OhtYok18,Oht18}, after verifying the assumption of the Poisson summation formula,  only one Fourier coefficient (or two Fourier coefficients in \cite{Oht18}) remains to be considered. 
\end{remark}

\subsection{Fourier coefficients $\hat{h}_N(-1,n)$ with $-p\leq n\leq p-2$} \label{subsection-m=-1np}
For a fixed constant $c\in \mathbb{R}$, we introduce the subset 
\begin{align}
    D'_0(c)=\{(t,s)\in D'_0| s=c\}.
\end{align}
we have
\begin{proposition} \label{prop-saddleonedim3}
    For any $c\in [0.2,0.8]$ and $n\in\mathbb{Z}$, there is a constant $C$ independent of $c$, such that 
  \begin{align}
        |\int_{D'_0(c)} e^{(N+\frac{1}{2})V_N(p,t,c;-1,n)}dt|<Ce^{(N+\frac{1}{2})\left(\zeta_{\mathbb{R}}(p)-\epsilon\right)}. 
    \end{align}
\end{proposition}
\begin{proof}
    See Appendix \ref{appendix-onesaddle2} for a proof.
\end{proof}


\begin{proposition} \label{prop-m-1np}
    When  $m=-1$, for $-p\leq n\leq p-2$, we have 
    \begin{align}
        \int_{D'_0}\psi(t,s)\sin(2\pi s)e^{(N+\frac{1}{2})V_N(p,t,s;-1,n)}dtds=O\left(e^{(N+\frac{1}{2})(\zeta_{\mathbb{R}}(p)-\epsilon)}\right),
    \end{align}
    for some sufficiently small $\epsilon>0$.
\end{proposition}
\begin{proof}
      We note that, $V_N(p,t,s;-1,n)$ uniformly converges to $V(p,t,s;-1,n)$ on $D'_{0}$. So we only need to estimate the integral of  $V(p,t,s;-1,n)$ as follow:
       \begin{align} \label{formula-m-1}
           &|\int_{D'_0}\psi(t,s)\sin(2\pi s)e^{(N+\frac{1}{2})V(p,t,s;-1,n)}dtds|\\\nonumber
           &=|\int_{0.2}^{0.8}\int_{D'_0(c)}\psi(t,c)\sin(2\pi c)e^{(N+\frac{1}{2})V(p,t,c;-1,n)}dtdc|\\\nonumber
           &\leq \int_{0.2}^{0.8} |\int_{D'_0(c)}\psi(t,c)\sin(2\pi c)e^{(N+\frac{1}{2})V(p,t,c;-1,n)}dt|dc.
       \end{align}
By Proposition \ref{prop-saddleonedim3}, we obtain 
       \begin{align}
            |\int_{D'_0(c)}\psi(t,c)\sin(2\pi c)e^{(N+\frac{1}{2})V(p,t,c;-1,n)}dt|<Ce^{(N+\frac{1}{2})(\zeta_{\mathbb{R}}(p)-\epsilon)}
       \end{align}
      where $C$ is a constant independent of $c$ and $n$. Then formula (\ref{formula-m-1}) implies Proposition \ref{prop-m-1}. We remark that the above proof holds for any $n\in \mathbb{Z}$. 
\end{proof}




\subsection{Fourier coefficients $\hat{h}_N(0,n)$ with $0\leq n\leq p-2$} \label{subsection-m=0np}
First, we have
\begin{proposition} \label{prop-saddleonedim}
  For  $n\in \mathbb{Z}$ and for $c_{upper}(p)\leq c\leq 1$ or $0\leq c\leq 1-c_{upper}(p)$, there exists a constant $C$ independent of $c$, such that
    \begin{align}
        |\int_{D'_0(c)} e^{(N+\frac{1}{2})V_N(p,t,c;0,n)}dt|<Ce^{(N+\frac{1}{2})\left(\zeta_{\mathbb{R}}(p)-\epsilon\right)}. 
    \end{align}
\end{proposition}
\begin{proof}
    See Appendix \ref{appendix-onesaddle} for a proof. 
\end{proof}

We introduce the quantity
\begin{align} \label{formula-cupperp}
    c_{upper}(p)=\frac{1}{\pi}(v_8-2\pi\zeta_{\mathbb{R}}(p))^{\frac{1}{2}}+\frac{1}{2},
\end{align}
and set
\begin{align}
    c_{0}(p)=\frac{7}{8p}+\frac{1}{2}.
\end{align}
By Lemma \ref{lemma-volumeestimate},  we have $2\pi\zeta_{\mathbb{R}}(p)>v_{8}-\frac{49\pi^2}{64}\frac{1}{p^2}$. Then, by formula (\ref{formula-cupperp}), we obtain 
\begin{align}  \label{formula-cupper<c0}
    c_{upper}(p)\leq c_0(p).
\end{align}
We introduce the region
\begin{align} \label{formula-D''0}
    D''_0=\{(t,s)\in D'_0|1-c_0(p)\leq s\leq c_0(p)\}.
\end{align}

\subsubsection{Fourier coefficients $\hat{h}_N(0,n)$ with $1\leq n\leq p-2$}
\begin{lemma}  \label{lemma-UnD''}
    For $n\geq 1$, we have
\begin{align}
    U_n\cap D''_0=\emptyset.
\end{align}
\end{lemma}
\begin{proof}
    By Lemma \ref{lemma-topbottom}, the bottom point of $U_n$ is given by $(\frac{3p+n}{4p-1},\frac{1}{2}+\frac{3+4n}{2(4p-1)})$, clearly 
    \begin{align}
        \frac{1}{2}+\frac{3+4n}{2(4p-1)}>c_0(p)=\frac{1}{2}+\frac{7}{8p},
    \end{align}
    for $n\geq 1$. Hence $U_n\cap D''_0=\emptyset$ for $n\geq 1$.
\end{proof}


\begin{proposition}
 \label{prop-m0np}
    For $1\leq n\leq p-2$, we have 
    \begin{align}
        \int_{D'_0}\psi(t,s)\sin(2\pi s)e^{(N+\frac{1}{2})V_N(p,t,s;0,n)}dtds=O\left(e^{(N+\frac{1}{2})(\zeta_{\mathbb{R}}(p)-\epsilon)}\right),
    \end{align}
    for some sufficiently small $\epsilon>0$.
\end{proposition}



\begin{proof}
By formula (\ref{formula-cupper<c0}) and  Proposition \ref{prop-saddleonedim}, we have
\begin{align}
&\int_{D'_0}\psi(t,s)\sin(2\pi s)e^{(N+\frac{1}{2})V_N(p,t,s;0,n)}dtds\\\nonumber
&=\int_{D''_0}\sin(2\pi s)e^{(N+\frac{1}{2})V_N(p,t,s;0,n)}dtds+O(e^{(N+\frac{1}{2})(\zeta_{\mathbb{R}}(p)-\epsilon)}).     
\end{align}

So we only need to estimate the following integral 
\begin{align}
 \int_{D''_0}\psi(t,s)\sin(2\pi s)e^{(N+\frac{1}{2})V_N(p,t,s;0,n)}dtds.
\end{align}
Note that $V_N(p,t,s;0,n)$ uniformly converges to $V(p,t,s;0,n)$ on $D''_{0}$ by Lemma \ref{lemma-varphixi3}.
We show that there is a homotopy $D''_{\delta}$ $(0\leq \delta\leq \delta_0)$ between $D''_0$ and $D''_{\delta_0}$ such that 
\begin{align}
    &D''_{\delta_0}\subset \{(t,s)\in \mathbb{C}^2|Re V(p;t,s;0,n)<\zeta_\mathbb{R}(p)-\epsilon\}, \label{formula-Delta'} \\
    &\int_{\partial D''_{\delta}}\sin(2\pi s)e^{(N+\frac{1}{2})V(p,t,s;0,n)}dtds=O(e^{(N+\frac{1}{2})(\zeta_\mathbb{R}(p)-\epsilon)}), \label{formula-pDelta'}
\end{align}
for some sufficiently small $\epsilon>0$.

In the fiber of the projection $\mathbb{C}^2\rightarrow \mathbb{R}^2$ at $(t,s)\in D''_0$, we consider the flow from $(X,Y)=(0,0)$ determined by the vector field $\left(-\frac{\partial f}{\partial X},-\frac{\partial f}{\partial Y}\right)$, we have known that for $(t,s)\notin U_n$, the flow goes infinity. Hence by Lemma \ref{lemma-UnD''}, we obtain that, for $n\geq 1$, the flow goes to infinity. Then, the value of $Re V(p;t,s;0,n)$ monotonically decreases and goes to $-\infty$.  So we can choose $\delta_0$ large enough to make sure formula (\ref{formula-Delta'}) holds.  




As for (\ref{formula-pDelta'}), note that the boundary $\partial D''_0$ consists of $D'_0(c_{0}(p))$, $D'_0(1-c_{0}(p))$ and the partial boundaries of $D'_0$, denoted by $D'_{0b}$.  Hence $\partial D''_{\delta}$ consists of three parts denoted by 
\begin{align}
    \partial D''_{\delta}=A_1\cup A_2\cup B,
\end{align}
where $A_1$ and $A_2$ comes from the flows start at $(t,s)\in D'_0(c_{0}(p))$ and  $(t,s)\in D'_0(1-c_{0}(p))$ respectively, while $B$ comes from the flows start at $(t,s)\in D'_{0b}$.

By its definition, $D'_{0b}\subset \partial D'_{0}\subset \{(t,s)\in \mathbb{C}^2|Re V(p,t,s;0,n)<\zeta_\mathbb{R}(p)-\epsilon\}$, and the function $Re V(p,t,s;0,n)$ decreases under the flow, so we have 
\begin{align} \label{formula-integralB}
    \int_{B}\sin(2\pi s)e^{(N+\frac{1}{2})V(p,t,s;0,n)}dtds=O(e^{(N+\frac{1}{2})(\zeta_\mathbb{R}(p)-\epsilon)}).
\end{align}


By Proposition \ref{prop-saddleonedim}, the integral on $D'_0(c_{0}(p))$ and $D'_0(1-c_{0}(p))$ 
is also of order $O(e^{(N+\frac{1}{2})(\zeta_\mathbb{R}(p)-\epsilon}))$. By applying the saddle point method to the slices of the region $A_1\cup A_2$ as shown in Appendix \ref{appendix-onesaddle}, we can prove that
\begin{align} \label{formula-integralA1A2}
    \int_{A_1\cup A_2}\sin(2\pi s)e^{(N+\frac{1}{2})V(p,t,s;0,n)}dtds=O(e^{(N+\frac{1}{2})(\zeta_\mathbb{R}(p)-\epsilon)}).
\end{align}
Combining formulas (\ref{formula-integralB}) and (\ref{formula-integralA1A2}) together, we prove (\ref{formula-pDelta'}).
Hence, the required homotopy exists.  
\end{proof}







 
    
\subsubsection{Fourier coefficient $\hat{h}_N(0,0)$}



\begin{lemma}  \label{lemma-U0D''0}
    For $p\geq 6$,  we have
\begin{align}
    U_0\subset D''_0\subset D_H.
    \end{align}
\end{lemma}
\begin{proof}
By Lemma \ref{lemma-topbottom},   the top and bottom points of the region $U_0$ are given by
     $(\frac{3p-2}{4p-1},\frac{1}{2}+\frac{5}{2(4p-1)})$ and $U_n$ is given by $(\frac{3p}{4p-1},\frac{1}{2}+\frac{3}{2(4p-1)})$
     respectively.  Then we have
     \begin{align} \label{formula-<c0<}
     c_{0}(p)=\frac{1}{2}+\frac{7}{8p}>\frac{1}{2}+\frac{5}{2(4p-1)}.
     \end{align}
    Hence $U_0\subset D''_0$. 
    
    
    The intersection point of the lines $s=c_{0}(p)=\frac{1}{2}+\frac{7}{8p}$ and $t+s=\frac{3}{2}$ is given by $(1-\frac{7}{8p},\frac{1}{2}+\frac{7}{8p})$, clearly, for $p\geq 10$, $1-\frac{7}{8p}>0.909$. Therefore, $D''_{0}\subset D_H$ for $p\geq 10$. 

In particular, for $p=6$, by formula (\ref{formula-cupperp}),  we have $c_{upper}(6)=0.5871$. Let $c_0(6)=c_{upper}(6)$, then the formula (\ref{formula-<c0<}) also holds for $p=6$, hence $U_0\subset D''_0$. 
The intersection point of the lines $s=c_{0}(6)=0.5871$ and $t+s=\frac{3}{2}$ is given by $(0.9129,0.5871)$, clearly, $0.9129>0.909$. Therefore, $D''_{0}\subset D_H$ also holds for $p=6$. Similarly, we can also show  $D''_{0}\subset D_H$ also holds for $p=7,8,9$. Hence we finish the proof of Lemma \ref{lemma-U0D''0}.
    
\end{proof}


\begin{proposition} \label{prop-m0n0} 
For $p\geq 6$,  we have
\begin{align}
    \hat{h}_N(0,0)&=\frac{(-1)^pe^{\frac{1}{4}\pi\sqrt{-1}}(N+\frac{1}{2})^{\frac{3}{2}}}{\sin \frac{\pi}{2N+1}}\int_{D'_0}\psi(t,s)\sin(2\pi s)e^{(N+\frac{1}{2})V_N(p,t,s)}dtds\\\nonumber
    &=\frac{(-1)^{p+1}2\pi e^{\frac{1}{4}\pi\sqrt{-1}} (N+\frac{1}{2})^{\frac{1}{2}}}{\sin \frac{\pi}{2N+1}}\omega(p)e^{(N+\frac{1}{2})\zeta_{\mathbb{R}}(p)}\\\nonumber
&\cdot\left(1+\sum_{i=1}^d\kappa_i(p)\left(\frac{2\pi\sqrt{-1}}{N+\frac{1}{2}}\right)^i+O\left(\frac{1}{(N+\frac{1}{2})^{d+1}}\right)\right),
    \end{align}
    for $d\geq 1$, where $\omega(p)$ and $\kappa_i(p)$ are constants determined by $\mathcal{K}_p$.
\end{proposition}
\begin{proof}
By Proposition \ref{prop-saddleonedim}, we have
\begin{align}
    \hat{h}_N(0,0)&=\frac{(-1)^pe^{\frac{1}{4}\pi\sqrt{-1}}(N+\frac{1}{2})^{\frac{3}{2}}}{\sin \frac{\pi}{2N+1}}\int_{D'_0}\psi(t,s)\sin(2\pi s)e^{(N+\frac{1}{2})V_N(p,t,s)}dtds\nonumber\\
    &=\frac{(-1)^pe^{\frac{1}{4}\pi\sqrt{-1}}(N+\frac{1}{2})^{\frac{3}{2}}}{\sin \frac{\pi}{2N+1}}\int_{D''_0}\sin(2\pi s)e^{(N+\frac{1}{2})V_N(p,t,s)}dtds \label{formula-integralD''}\\\nonumber
    &+O\left(e^{(N+\frac{1}{2})\left(\zeta_{\mathbb{R}}(p)-\epsilon\right)}\right),
\end{align}
where   $D''_0$ is given in (\ref{formula-D''0}). 


We will verify the conditions of Proposition \ref{proposition-saddlemethod} for saddle point method in Proposition \ref{propostion-checksaddle}. By Lemma \ref{lemma-Vr} and  Remark \ref{remark-saddle},   
we can apply the Proposition \ref{proposition-saddlemethod} to the above integral (\ref{formula-integralD''}). Let $(t_0,s_0)$ be the critical point of $V(p,t,s)$, we obtain that 


\begin{align}
  &\int_{D''_0}\sin(2\pi s)\exp\left((N+\frac{1}{2})V_N(p,t,s)\right)dtds\\\nonumber
        &=\frac{2\pi}{2N+1}\frac{2\alpha(t_0,s_0)}{\sqrt{\det Hess(V)(t_0,s_0)}}e^{(N+\frac{1}{2})\zeta_{\mathbb{R}}(p)}\\\nonumber
        &\left(1+\sum_{i=1}^d\kappa_i(p)\left(\frac{2\pi\sqrt{-1}}{N+\frac{1}{2}}\right)^i+O\left(\frac{1}{(N+\frac{1}{2})^{d+1}}\right)\right)   
\end{align}
 where the function  
\begin{align}
    \alpha(t,s)=&\psi(t,s)\sin (2\pi s)\\\nonumber
    &\cdot e^{-\frac{1}{2}\left(\log(1-e^{2\pi\sqrt{-1}(t+s)})+\log(1-e^{2\pi\sqrt{-1}(t-s)})-4\pi\sqrt{-1}t\right)}.
\end{align}
and the determinant of the Hessian matrix at $(t_0,s_0)$ is given by the formula (\ref{formula-hessV})
\begin{align}
    &\det Hess(V)(t_0,s_0)=(2\pi\sqrt{-1})^2H(p,x_0,y_0)
\end{align}
where 
\begin{align}
    H(p,x_0,y_0)&=\left(\frac{-3(2p+1)}{\frac{1}{x_0}-1}+\frac{2p+1}{\frac{1}{x_0y_0}-1}+\frac{2p+1}{\frac{1}{x_0/y_0}-1}-\frac{3}{(\frac{1}{x_0}-1)(\frac{1}{x_0y_0}-1)}\right.\\\nonumber
    &\left.-\frac{3}{(\frac{1}{x_0}-1)(\frac{1}{x_0/y_0}-1)}+\frac{4}{(\frac{1}{x_0y_0}-1)(\frac{1}{x_0/y_0}-1)}\right),
\end{align}
with $x_0=e^{2\pi\sqrt{-1}t_0}$ and $y_0=e^{2\pi\sqrt{-1}s_0}$. 


 


Since $(t_0,s_0)$ is the critical point of $V(p,t,s)$, then it satisfies the identity
\begin{align}
    &-\frac{1}{2}\left(\log(1-e^{2\pi\sqrt{-1}(t_0+s_0)})+\log(1-e^{2\pi\sqrt{-1}(t_0-s_0)})-4\pi\sqrt{-1}t_0\right)\\\nonumber
    &=\pi\sqrt{-1}-\frac{3}{2}\log(1-e^{2\pi\sqrt{-1}t_0})+2\pi\sqrt{-1}t_0,
\end{align}
we obtain
\begin{align}
    \alpha(t_0,s_0)=\frac{\sin (2\pi s_0)e^{2\pi\sqrt{-1}t_0}}{(1-e^{2\pi\sqrt{-1}t_0})^{\frac{3}{2}}}.
\end{align}
Therefore, we have
\begin{align}
        \hat{h}_{N}(0,0)&=\frac{(-1)^{p+1}2\pi e^{\frac{1}{4}\pi\sqrt{-1}} (N+\frac{1}{2})^{\frac{1}{2}}}{\sin \frac{\pi}{2N+1}}\omega(p) e^{(N+\frac{1}{2})V(p,t_0,s_0)}\\\nonumber
        &\left(1+\sum_{i=1}^d\kappa_i(p)\left(\frac{2\pi\sqrt{-1}}{N+\frac{1}{2}}\right)^i+O\left(\frac{1}{(N+\frac{1}{2})^{d+1}}\right)\right),
    \end{align}
    where  
\begin{align}
   \omega(p)&=\frac{\sin (2\pi s_0)e^{2\pi\sqrt{-1}t_0}}{(1-e^{2\pi\sqrt{-1}t_0})^{\frac{3}{2}}\sqrt{\det Hess(V)(t_0,s_0)}}\\\nonumber
   &=\frac{(y_0-y_0^{-1})x_0}{-4\pi (1-x_0)^\frac{3}{2}\sqrt{H(p,x_0,y_0)}}.
\end{align}
\end{proof}
\begin{proposition} \label{propostion-checksaddle}
    When we apply Proposition \ref{proposition-saddlemethod} (saddle point method)  to the integral (\ref{formula-integralD''}), the assumptions of Proposition \ref{proposition-saddlemethod} holds. 
\end{proposition}
\begin{proof}
We note that, by Lemma \ref{lemma-varphixi3}, $V_N(p,t,s)$ uniformly converges to the $V(p,t,s)$ on $D''_0$ as $N\rightarrow \infty$. Hence,  we only need to verify the assumptions of the saddle point method for $V(p,t,s)$.  
 We show that there exists a homotopy $D''_\delta$ ($0\leq \delta\leq 1 $) between $D''_0$ and $D''_1$ such that 
\begin{align}
 &(t_0,s_0)\in D''_1,  \label{saddle-1} \\ 
    &D''_1-\{(t_0,s_0)\}\subset \{(t,s)\in \mathbb{C}^2|Re V(p,t,s)<\zeta_\mathbb{R}(p)\}, \label{saddle-2} \\
    & \int_{\partial D''_{\delta}} \sin(2\pi s)e^{(N+\frac{1}{2})V(p,t,s)}dtds=O\left(e^{(N+\frac{1}{2})(\zeta_{\mathbb{R}}(p)-\epsilon)}\right). \label{saddle-3}
\end{align}


   In the fiber of the projection $\mathbb{C}^2\rightarrow \mathbb{R}^2$ at $(t,s)\in D_0''$, we consider the flow from $(X,Y)=(0,0)$ determined by the vector field $(-\frac{\partial f}{\partial X},-\frac{\partial f}{\partial Y})$. By Lemma \ref{lemma-U0D''0}, together with Lemma \ref{lemma-UnnotUn} and Lemma \ref{lemma-HessXY}, the convex neighborhood $U_0$ of $(t_0,s_0)$ satisfies the following holds. 
 \begin{itemize}
        \item[(1)] If $(t,s)\in U_0$, then $f$ has a unique minimal point, and the flow goes there.
        \item[(2)] If $(t,s)\in D''_0\setminus U_0$, then the flow goes to infinity. 
    \end{itemize}
We put $\mathbf{g}(t,s)=(g_1(t,s),g_2(t,s))$ to be the minimal point of $(1)$. In particular, $|\mathbf{g}(t,s)|\rightarrow \infty$ as $(t,s)$ goes to $\partial U_0$. Further, for a sufficiently large $R>0$, we stop the flow when $|\mathbf{g}(t,s)|=R$. We construct the revised flow $\hat{\mathbf{g}}(t,s)$, by putting $\hat{\mathbf{g}}(t,s)=\mathbf{g}(t,s)$ for $(t,s)\in U_0$ with $|\mathbf{g}(t,s)|<R$, otherwise, by putting $|\hat{\mathbf{g}}(t,s)|=(R,R)$. 

 We define the ending of the homotopy by
\begin{align}
    D''_1=\{(t,s)+\hat{\mathbf{g}}(t,s)\sqrt{-1}|(t,s)\in D''_0\}. 
\end{align}
Further, we define the internal part of the homotopy by setting it along the flow from $(t,s)$ determined by the vector field $\left(-\frac{\partial f}{\partial X},-\frac{\partial f}{\partial Y}\right)$. 

We show (\ref{saddle-1}) and (\ref{saddle-2}) as follows. We consider the function
\begin{align}
    h(t,s)=Re V(t,s,\hat{\mathbf{g}}(t,s)). 
\end{align}
If $(t,s)\notin U$, by (2), $-h(t,s)$ is sufficiently large (because we let $R$ be sufficiently large), hence (\ref{saddle-2}) holds in this case. Otherwise, $(t,s)\in U$, in this case, $\hat{\mathbf{g}}(t,s)=\mathbf{g}(t,s)$. It is shown from the definition of $\hat{g}(t,s)$ that 
\begin{align}
    \frac{\partial Re V}{\partial X}=\frac{\partial Re V}{\partial Y}=0 \ \text{at} \ (X,Y)=\mathbf{g}(t,s),
\end{align}
which implies 
\begin{align}
    Im\frac{\partial V}{\partial t}=Im \frac{\partial V}{\partial s}=0 \ \text{at} \ (t,s)+\mathbf{g}(t,s)\sqrt{-1}.
\end{align}
On the other hand, 
    \begin{align}
        \frac{\partial h}{\partial t}=Re\frac{\partial V}{\partial t}, \ \frac{\partial h}{\partial s}=Re \frac{\partial V}{\partial s} \ \text{at} \ (t,s)+\mathbf{g}(t,s)\sqrt{-1}.
    \end{align}
Therefore, when  $(t,s)+\mathbf{g}(t,s)\sqrt{-1}$ is a critical point of $V$, $(t,s)$ is a critical point of $h(t,s)$. Hence by Proposition \ref{prop-critical}, $h(t,s)$ has a unique maximal point at $(t_0,s_0)$. Therefore, (\ref{saddle-1}) and (\ref{saddle-2}) holds. 


We show (\ref{saddle-3}) as follows. Note that the boundary of $\partial D''_0$ consists of $D'_0(c_{0}(p))$, $D'_0(1-c_{0}(p))$ and the partial boundaries of $D'_0$, denoted by $D'_{0b}$.  Hence $\partial D''_{\delta}$ consists of three parts denoted by 
\begin{align}
    \partial D''_{\delta}=A_1\cup A_2\cup B,
\end{align}
where $A_1$ and $A_2$ comes from the flows start at $(t,s)\in D'_0(c_{0}(p))$ and  $(t,s)\in D'_0(1-c_{0}(p))$ respectively, while $B$ comes from the flows start at $(t,s)\in D'_{0b}$.

By its definition, $D'_{0b}\subset \partial D'_{0}\subset \{(t,s)\in \mathbb{C}^2|Re V(p,t,s;0,n)<\zeta_\mathbb{R}(p)-\epsilon\}$, and the function $Re V(p,t,s;0,n)$ decreases under the flow, so we have 
\begin{align} \label{formula-integralB2}
    \int_{B}\sin(2\pi s)e^{(N+\frac{1}{2})V(p,t,s;0,n)}dtds=O(e^{(N+\frac{1}{2})(\zeta_\mathbb{R}(p)-\epsilon)}).
\end{align}


By Proposition \ref{prop-saddleonedim}, the integral on $D'_0(c_{0}(p))$ and $D'_0(1-c_{0}(p))$ 
is also of order $O(e^{(N+\frac{1}{2})(\zeta_\mathbb{R}(p)-\epsilon}))$. By applying the saddle point method to the slices of the region $A_1\cup A_2$ as shown in Appendix \ref{appendix-onesaddle}, we can prove that
\begin{align} \label{formula-integralA1A22}
    \int_{A_1\cup A_2}\sin(2\pi s)e^{(N+\frac{1}{2})V(p,t,s;0,n)}dtds=O(e^{(N+\frac{1}{2})(\zeta_\mathbb{R}(p)-\epsilon)}).
\end{align}
Combining formulas (\ref{formula-integralB2}) and (\ref{formula-integralA1A22}) together, we prove (\ref{formula-pDelta'}).


 
By (\ref{saddle-1}) (\ref{saddle-2}) and (\ref{saddle-3}), the required homotopy exists. Hence the assumptions of Proposition \ref{proposition-saddlemethod} holds when we apply the saddle point method to the integral (\ref{formula-integralD''}).  
\end{proof}




\subsection{Final proof} \label{subsection-final}

Now we can finish the proof of Theorem \ref{theorem-main} as follows.
\begin{proof}
By using formula (\ref{formula-Poission-after}), together Proposition  \ref{prop-m-1np}, Proposition \ref{prop-m0np} and Proposition \ref{prop-m0n0} together, we obtain
    \begin{align}
        J_{N}(\mathcal{K}_p;\xi_N)&=2\hat{h}_{N}(0,0)+O(e^{(N+\frac{1}{2})(\zeta_{\mathbb{R}}(p)-\epsilon)})
        \\\nonumber
        &=(-1)^{p+1}\frac{4\pi e^{\frac{1}{4}\pi\sqrt{-1}}(N+\frac{1}{2})^{\frac{1}{2}}}{\sin\frac{\frac{\pi}{2}}{N+\frac{1}{2}}}\omega(p)e^{(N+\frac{1}{2})\zeta(p)}\\\nonumber
&\cdot\left(1+\sum_{i=1}^d\kappa_i(p)\left(\frac{2\pi\sqrt{-1}}{N+\frac{1}{2}}\right)^i+O\left(\frac{1}{(N+\frac{1}{2})^{d+1}}\right)\right),
    \end{align}
    for $d\geq 1$, where $\omega(p)$ and $\kappa_i(p)$ are constants determined by $\mathcal{K}_p$.
\end{proof}






















\section{Appendices} \label{Section-App}
\subsection{Proof of Lemma \ref{lemma-regionD'0}}   \label{appendix-0}
We define the function
\begin{align}
    v(t,s)=\Lambda(t+s)+\Lambda(t-s)-3\Lambda\left(t\right).
\end{align}
We set
    \begin{align}
    D&=\{(t,s)\in \mathbb{R}^2| 1< t+s< 2, 0< t-s<1, \frac{1}{2}< t<1\}, \\\nonumber
    D'_0&=\{0.02 \leq t-s\leq 0.7, 1.02 \leq t+s\leq 1.7, 0.2 \leq s\leq 0.8,0.5\leq t\leq 0.909\}.
\end{align}
Then we have
\begin{lemma} 
The following domain
    \begin{align} 
        \left\{(t,s)\in D| v(t,s)>\frac{3.509}{2\pi}\right\}
    \end{align}
    is included in the region $D'_0$.
\end{lemma}
\begin{proof}
Recall the definition of the function $\Lambda(t)$:
\begin{align}
  \Lambda(t)=Re\left(\frac{1}{2\pi\sqrt{-1}}\text{Li}_2(e^{2\pi\sqrt{-1}t})\right). 
\end{align}
We have $\Lambda'(t)=-\log 2\sin(\pi t)$ and $\Lambda''(t)=-\pi \cot(\pi t)$ for $t\in [0,1]$. In the following, we only describe how to obtain the upper bound 0.909 of $t$ in the definition of $D'_0$, the other bounds can be derived similarly.   

For a fixed $t\in (\frac{1}{2},1)$, we regard $v(t,s)$ as function of $s$, it follows that 
\begin{align}
    v_s(t,s)=\Lambda'(t+s)-\Lambda'(t-s)=\log\frac{\sin(\pi(t-s))}{\sin(\pi(t+s-1))}. 
\end{align}
The critical point is given by $s=\frac{1}{2}$. Furthermore, $v_{ss}(t,\frac{1}{2})<0$, i.e. for any fixed $t\in (\frac{1}{2},1)$, 
as a function of $s$, $v(t,s)$ takes the maximal value at $s=\frac{1}{2}$.  

As a function of $t$, 
\begin{align}
    v(t,\frac{1}{2})=\Lambda(t+\frac{1}{2})+\Lambda(t-\frac{1}{2})-3\Lambda(t)=2\Lambda(t+\frac{1}{2})-3\Lambda(t).
\end{align}
On can compute that there is a point $t_0\approx 0.8$ which is a maximal point of $v(t,\frac{1}{2})$, and such that   $v(t,\frac{1}{2})$ increases (resp. decreases) on the interval $(\frac{1}{2},t_0)$ (resp. the interval $(t_0,1)$). 
We compute that 
$2\pi v(0.909,\frac{1}{2})=3.4589<3.509$. By above analysis, if $(t,s)\in D$ such that $v(t,s)>\frac{3.509}{2\pi }$, then $t<0.909$. 
\end{proof}

\subsection{Proof of Lemma \ref{lemma-UnnotUn}} \label{appendix-LemmaUn}

Based on Lemma \ref{lemma-Li2}, in order to study the asymptotic behaviour of the function 
\begin{align}
    f(X,Y;n)=ReV(p,t+X\sqrt{-1},s+Y\sqrt{-1}), 
\end{align}
we introduce the following  function
\begin{equation} \label{eq:2}
F(X,Y;n)=\left\{ \begin{aligned}
         &0  &  \ (\text{if} \ X+Y\geq 0) \\
         &\left((t+s)-\frac{3}{2}\right)(X+Y) & \ (\text{if} \ X+Y<0)
                          \end{aligned} \right.
                          \end{equation}
\begin{equation*}
 +\left\{ \begin{aligned}
         &0  &  \ (\text{if} \ X-Y\geq 0) \\
         &\left((t-s)-\frac{1}{2}\right)(X-Y) & \ (\text{if} \ X-Y<0)
                          \end{aligned} \right.
                          \end{equation*}
\begin{equation*}
+\left\{ \begin{aligned}
         &0  &  \ (\text{if} \ X\geq 0) \\
         &\left(\frac{3}{2}-3t\right)X & \ (\text{if} \ X<0)
                          \end{aligned} \right.  +\left(p+\frac{3}{2}+n-(2p+1)s\right)Y+X
                          \end{equation*}
where we use $t+s-\frac{3}{2}$ instead of $t+s-\frac{1}{2}$ in the first summation since in our situation $1< t+s<2$.                       


  Note that $F(X,Y;n)$ is a piecewise linear function,  we subdivide the plane $\{(X,Y)\in \mathbb{R}^2\}$ into eight regions to discuss this function. We study the conditions such that $F(X,Y,n)$ has the following property: 
\begin{align} \label{formula-F(X,Y)2}
    F(X,Y;n)\rightarrow \infty \ \text{as} \ X^2+Y^2\rightarrow \infty.
\end{align}

  
(I) $X\geq Y\geq 0$, then
\begin{align}
    F(X,Y;n)&=\left(\left(p+\frac{3}{2}+n\right)-(2p+1)s\right)Y+X\\\nonumber
  &\geq \left(\left(p+\frac{5}{2}+n\right)-(2p+1)s\right)Y.  
\end{align}
When $s<\frac{p+\frac{5}{2}+n}{2p+1}$, then the property (\ref{formula-F(X,Y)2}) holds. 


(II) $Y\geq X\geq 0$, then
\begin{align}
     F(X,Y;n)=\left(t-s+\frac{1}{2}\right)X+(p+2+n-2ps-t)Y.
\end{align}
When $t+2ps<p+2+n$, we have
\begin{align}
     F(X,Y;n)&=\left(t-s+\frac{1}{2}\right)X+(p+2+n-2ps-t)Y\\\nonumber
    &\geq \left(p+\frac{5}{2}+n-(2p+1)s\right)X.
\end{align}
Hence, the property (\ref{formula-F(X,Y)2}) holds in this case if  $\left(p+\frac{5}{2}+n\right)-(2p+1)s>0$. 


(III) $X+Y\geq 0$ and $X\leq 0$, then
\begin{align}
     F(X,Y;n)=\left(2-2t-s\right)X+(p+2+n-2ps-t)Y.
\end{align}
When $t+2ps<p+2+n$, then 
\begin{align}
     F(X,Y;n)&\geq \left(2-2t-s\right)X+(p+2+n-2ps-t)(-X) \\\nonumber
    &=(p+n+t-(2p-1)s)(-X).
\end{align}
If $p+n+t-(2p-1)s>0$, 
we obtain that the property (\ref{formula-F(X,Y)2}) holds in this case. 

(IV) $X+Y\leq 0$ and $Y\geq 0$, then 
\begin{align}
     F(X,Y;n)&=\left(\frac{1}{2}-t\right)X+\left(p+\frac{1}{2}+n-(2p-1)s\right)Y\\\nonumber
     &=\left(-\frac{1}{2}+t\right)(-X)+\left(p+\frac{1}{2}+n-(2p-1)s\right)Y\\\nonumber
    &\geq (p+n+t-(2p-1)s)Y.
\end{align}
When $p+n+t-(2p-1)s>0$, we can see that the property (\ref{formula-F(X,Y)2}) holds in this case. 



(V) $X-Y\leq 0$ and $Y\leq 0$, then 
\begin{align}
     F(X,Y;n)=\left(\frac{1}{2}-t\right)X+\left(p+\frac{1}{2}+n-(2p-1)s\right)Y.
\end{align}
Since $\frac{1}{2}<t<1$, 
\begin{align}
     F(X,Y;n)&=\left(-\frac{1}{2}+t\right)(-X)+\left(p+\frac{1}{2}+n-(2p-1)s\right)Y\\\nonumber
    &\geq \left(-\frac{1}{2}+t\right)(-Y)+\left(p+\frac{1}{2}+n-(2p-1)s\right)Y\\\nonumber
    &=(-p-1-n+t+(2p-1)s)(-Y),
\end{align}
if $t+(2p-1)s-p-1-n>0$, it follows that the property (\ref{formula-F(X,Y)2}) holds in this case. 

(VI) $X-Y\geq 0$ and $X\leq 0$, then 
\begin{align}
     F(X,Y;n)=\left(1+s-2t\right)X+\left(p+n+t-2ps\right)Y.
\end{align}
When $p+n+t-2ps<0$, and since $-Y\geq -X\geq 0$,  
 \begin{align}
     F(X,Y;n)&\geq \left(1+s-2t\right)X-\left(p+n-2ps+t\right)(-X)\\\nonumber
    &=(-p-1-n+t+(2p-1)s)(-X),
\end{align}
if $t+(2p-1)s-p-1-n>0$, it follows that the property (\ref{formula-F(X,Y)2}) holds in this case. 



(VII) $X+Y\leq 0$ and $X\geq 0$, then 
   \begin{align}
    F(X,Y;n)=\left(t+s-\frac{1}{2}\right)X+\left(p+n-2ps+t\right)Y,
\end{align}
since $t+s-\frac{1}{2}>0$, if $p+n-2ps+t< 0$, by $Y\leq 0$,   
it follows that the property (\ref{formula-F(X,Y)2}) holds in this case. 




(VIII) $X+Y\geq 0$ and $Y\leq 0$, then 
   \begin{align}
     F(X,Y;n)=X+\left(p+\frac{3}{2}+n-(2p+1)s\right)Y.
\end{align}
By $X\geq -Y$,
   \begin{align}
     F(X,Y;n)&\geq (-Y)+\left(p+\frac{3}{2}+n-(2p+1)s\right)Y\\\nonumber
    &=\left(-p-\frac{1}{2}-n+(2p+1)s\right)(-Y),
\end{align}
if $-p-\frac{1}{2}-n+(2p+1)s>0$, we obtain that the property (\ref{formula-F(X,Y)2}) holds in this case.

Given $n\in \mathbb{Z}$, we introduce the region $U_n$ as follows.
\begin{small}
    \begin{align}
    U_n=\left\{(t,s)\in D'_0\bigg|\frac{p+n+1-t}{2p-1}<s<\frac{p+n+t}{2p-1}, \frac{p+n+t}{2p}<s<\frac{p+2+n-t}{2p}\right\}.
\end{align}
\end{small}
\begin{remark} \label{remark-Un}
   For $n\geq p-1$ or $n\leq -(p+1)$, $U_n=\emptyset$. Indeed, for $n\geq p-1$, we have 
       $\frac{p+n+1-t}{2p-1}>\frac{p+2+n-t}{2p}$ on $D'_0$, and for $n\leq -(p+1)$, we have 
        $\frac{p+n+t}{2p}>\frac{p+n+t}{2p-1}$ on $D'_0$. Hence $U_n=\emptyset$.  
\end{remark}



From above discussions, together with  Lemma \ref{lemma-Li2}, we have
\begin{lemma} 
For any $(t,s)\in D'_0$, 

(i) when $(t,s)\in U_n$, we have 
\begin{align}
    f(X,Y;n)\rightarrow \infty \ \text{as } \ X^2+Y^2\rightarrow +\infty.
\end{align}

(ii) when $(t,s)\notin U_n$, 
\begin{align}
    f(X,Y;n)\rightarrow -\infty \ \text{in some directions of} \ X^2+Y^2\rightarrow +\infty  
\end{align}
\end{lemma}
\begin{proof}
When $-p\leq n\leq p-2$, $U_n\neq \emptyset$.  For any $(t,s)\in U_n$, it satisfies  $\frac{p+\frac{1}{2}+n}{2p+1}<s<\frac{p+\frac{5}{2}+n}{2p+1}$. It follows that if $(t,s)\in U_n$, then all the inequalities in the eight region of the above analysis for the function $F(X,Y;n)$ 
are satisfied. Hence, for $(t,s)\in U_n$, we have  $F(X,Y;n)\rightarrow \infty \ \text{as } \ X^2+Y^2\rightarrow +\infty.$ So Lemma \ref{lemma-Li2}, we obtain (i).  

As to (ii), if $(t,s)\notin U_n$, then $(t,s)$  doesn't satisfy at least one of the inequality in the definition of $U_n$.  For example, suppose $s\geq \frac{p+1+n-t}{2p-1}$, then $p+1+n-t-(2p-1)s\geq 0$. In the following, we will show there exists a direction of $X^2+Y^2\rightarrow +\infty$ such that $f(X,Y;n)\rightarrow -\infty$. 

For $p+1+n-t-(2p-1)s>0$, by the case (V) in the previous discuss for the function $F(X,Y;n)$, when $X-Y\leq 0$ and $Y\leq 0$, we have 
\begin{align}
    F(X,Y;n)=\left(\frac{1}{2}-t\right)X+\left(p+\frac{1}{2}+n-(2p-1)s\right)Y.
\end{align}
Let $X=Y\rightarrow -\infty$, then $F(Y,Y;n)=(p+1+n-t-(2p-1)s)Y\rightarrow -\infty$ when $p+1+n-t-(2p-1)s>0$. So by Lemma \ref{lemma-Li2}, we obtain $f(X,Y;n)\rightarrow -\infty$ in the direction $X=Y\rightarrow -\infty$. 

For $p+1+n-t-(2p-1)s=0$, we consider the case 
(VI), when $X-Y\geq 0$ and $X\leq 0$, we have
\begin{align}
    F(X,Y;n)&=(1-2t+s)X+(p+n+t-2s)Y\\\nonumber
    &=(1-2t)X+(p+n+t)Y+s(X-2pY)\\\nonumber
    &<(1-2t)X+(p+n+t)Y+\frac{p+1+n-t}{2p-1}(X-2pY)\\\nonumber
    &=(2t-1-\frac{p+n+1-t}{2p-1})(Y-X)<0.
\end{align}


So we have proved (ii) for the case $s\geq \frac{p+1+n-t}{2p-1}$. It is easy to check the other cases in this way similarly and we finish the proof.  
\end{proof}


















\subsection{Proof of Proposition \ref{prop-critical}} \label{appendix-1}
In this section, we give the proof of Proposition \ref{prop-critical}. We establish some Lemmas first. 


\begin{lemma} \label{lemma-t0s0}
    Suppose $(t_0,s_0)$ is a critical point of $V(p,t,s)$ with $(Re(t_0),Re(s_0))\in D'_0$,  then we have  $(Re(t_0),Re(s_0))\in U_0$. 
\end{lemma}
\begin{proof}
Suppose $t_0=t_{0R}+X_0\sqrt{-1}$ and $s_0=s_{0R}+Y_0\sqrt{-1}$ is a critical point of $V(t,s)$ with $(t_{0R},s_{0R})\in D'_0$, i.e. 
\begin{equation} \label{formula-criticalpoint}
\left\{ \begin{aligned}
         &\frac{\partial V}{\partial t}(t_0,s_0)=0,  \\
         & \frac{\partial V}{\partial s}(t_0,s_0)=0.
                          \end{aligned}\right.
\end{equation}
We will prove that $(t_{0R},s_{0R})\in U_0$. Note that 
\begin{align}
    \frac{\partial f}{\partial X}&=\frac{\partial Re V(t+X\sqrt{-1},s+Y\sqrt{-1})}{\partial X}\\\nonumber
    &=Re \left(\sqrt{-1}\frac{\partial V(t+X\sqrt{-1},s+Y\sqrt{-1})}{\partial t}\right)\\\nonumber
    &=-Im \left(\frac{\partial V}{\partial t}\right),
\end{align}
similarly, we have
\begin{align}
    \frac{\partial f}{\partial Y}=-Im \left(\frac{\partial V}{\partial s}\right). 
\end{align}
According to the equation (\ref{formula-criticalpoint}), we obtain 
\begin{align} \label{equation:critical}
    \frac{\partial f}{\partial X}(t_{0R}+X_0\sqrt{-1},s_{0R}+Y_0\sqrt{-1})&=0,  \\\nonumber
    \frac{\partial f}{\partial Y}(t_{0R}+X_0\sqrt{-1},s_{0R}+Y_0\sqrt{-1})&=0.
\end{align}

By a straightforward computation, we obtain 
\begin{align}
    \frac{\partial f}{\partial X}&=-3arg(1-e^{2\pi\sqrt{-1}(t+X\sqrt{-1})})+arg(1-e^{2\pi\sqrt{-1}\left(t+s+(X+Y)\sqrt{-1}\right)})\\\nonumber
    &+arg(1-e^{2\pi\sqrt{-1}\left(t-s+(X-Y)\sqrt{-1}\right)})+2\pi,
\end{align}
\begin{align}
    \frac{\partial f}{\partial Y}&=arg(1-e^{2\pi\sqrt{-1}\left(t+s+(X+Y)\sqrt{-1}\right)})-arg(1-e^{2\pi\sqrt{-1}\left(t-s+(X-Y)\sqrt{-1}\right)})\\\nonumber
    &+(2p+3)\pi-(4p+2)\pi s.
\end{align}
Since $\frac{1}{2}<t<1$, we have
\begin{align}
    0<arg(1-e^{2\pi\sqrt{-1}(t+X\sqrt{-1})})<2\pi\left(t-\frac{1}{2}\right)
\end{align}
which implies
\begin{align}
    2\pi\left(-3t+\frac{3}{2}\right)<-3 arg(1-e^{2\pi\sqrt{-1}(t+X\sqrt{-1})})<0.
\end{align}

For $0<t+s-1<\frac{1}{2}$, then 
\begin{align}
    2\pi\left(t+s-\frac{3}{2}\right)<arg(1-e^{2\pi\sqrt{-1}(t+s+\sqrt{-1}(X+Y))})<0,
\end{align}
and for $\frac{1}{2}<t+s-1<1$, then 
\begin{align}
    0<arg(1-e^{2\pi\sqrt{-1}(t+s+\sqrt{-1}(X+Y))})<2\pi\left(t+s-\frac{3}{2}\right),
\end{align}
in particular, if $t+s-1=\frac{1}{2}$, then $arg(1-e^{2\pi\sqrt{-1}(t+s+\sqrt{-1}(X+Y))})=0$.

Furthermore, for $0<t-s<\frac{1}{2}$, then  
\begin{align}
    2\pi\left(t-s-\frac{1}{2}\right)<arg(1-e^{2\pi\sqrt{-1}\left(t-s+(X-Y)\sqrt{-1}\right)})<0
\end{align}
and for  $\frac{1}{2}<t-s<1$, then 
\begin{align}
    0<arg(1-e^{2\pi\sqrt{-1}\left(t-s+(X-Y)\sqrt{-1}\right)})<2\pi\left(t-s-\frac{1}{2}\right)
\end{align}
in particular, if $t-s=\frac{1}{2}$, then $arg(1-e^{2\pi\sqrt{-1}\left(t-s+(X-Y)\sqrt{-1}\right)})=0$.

For the convenience, we introduce the following three regions: 
\begin{itemize}
    \item[(i)] $\frac{1}{2}<t<0.909$, $0<t+s-1<\frac{1}{2}$, $0<t-s<\frac{1}{2}$, 
    \item[(ii)] $\frac{3}{4}<t<0.909$, $\frac{1}{2}\leq t+s-1<1$, $0<t-s<\frac{1}{2}$,
    \item[(iii)] $\frac{3}{4}<t<0.909$, $0< t+s-1<\frac{1}{2}$, $\frac{1}{2}\leq t-s<1$.
\end{itemize}



If $(t_{0R},s_{0R})$ lies in the region (ii), we have
\begin{align}
    \frac{\partial f}{\partial X}>2\pi\left(-2t_{0R}-s_{0R}+2\right), \ \frac{\partial f}{\partial X}<2\pi\left(t_{0R}+s_{0R}-\frac{3}{2}\right)
\end{align}
and 
\begin{align}
    \frac{\partial f}{\partial Y}>2\pi\left(p+\frac{3}{2}-(2p+1)s_{0R}\right), \ \frac{\partial f}{\partial Y}<2\pi\left(p+\frac{1}{2}-(2p-1)s_{0R}\right)
\end{align}
Therefore, if the equation (\ref{formula-criticalpoint}) has a solution $(t_0,s_0)$ with $(t_{0R},s_{0R})$ in region (ii), then $(t_{0R},s_{0R})$ satisfies
\begin{align}
    s_{0R}>\frac{p+\frac{3}{2}}{2p+1}\ \text{and} \ s_{0R}<\frac{p+\frac{1}{2}}{2p-1}. 
\end{align}
since on region (ii),  $t_{0R}+s_{0R}\geq \frac{3}{2}$, then 
\begin{align}
 t_{0R}\geq \frac{3}{2}-s_{0R}>\frac{3}{2}-\frac{p+\frac{1}{2}}{2p-1}=0.909   
\end{align}
which contradicts that $(t_{0R},s_{0R})$ lies in region (ii). 




If $(t_{0R},s_{0R})$ lies in the region (ii), we have
\begin{align}
    \frac{\partial f}{\partial X}>2\pi\left(-2t_{0R}+s_{0R}+1\right), \ \frac{\partial f}{\partial X}<2\pi\left(t_{0R}-s_{0R}-\frac{1}{2}\right)
\end{align}
and 
\begin{align}
    \frac{\partial f}{\partial Y}>2\pi\left(p+\frac{1}{2}-(2p-1)s_{0R}\right), \ \frac{\partial f}{\partial Y}<2\pi\left(p+\frac{3}{2}-(2p+1)s_{0R}\right).
\end{align}
Hence, if the equation (\ref{formula-criticalpoint}) has a solution $(t_0,s_0)$ with $(t_{0R},s_{0R})$ in region (iii), then $(t_{0R},s_{0R})$ satisfies
\begin{align}
    s_{0R}<\frac{p+\frac{3}{2}}{2p+1}\ \text{and} \ s_{0R}>\frac{p+\frac{1}{2}}{2p-1}
\end{align}
which is impossible since $\frac{p+\frac{3}{2}}{2p+1}<\frac{p+\frac{1}{2}}{2p-1}$. 

Therefore, we obtain that if $(t_{0R},s_{0R}) \in D'_0$, then $(t_{0R},s_{0R})$ lies in region (i), hence lies in $D_H$. 
Then $(t_{0R},s_{0R})$ must lie in $U_0$. Since if $(t_{0R},s_{0R})\in D_H\setminus U_0$, by Lemma \ref{lemma-HessXY}, 
the Hessian matrix of $f(X,Y)$ is positive definite on $(t_{0R},s_{0R})$,  and $(X_0,Y_0)$ is a critical point of $f(X,Y)$, hence 
$(X_0,Y_0)$ is a minimal point of $f(X,Y)$. On the other hand, by Lemma \ref{lemma-UnnotUn}, we have that $f(X,Y)$ goes to $-\infty$ in some directions of $X^2+Y^2\rightarrow +\infty$. It is a contradiction. 
Hence, we prove that $(t_{0R},s_{0R})\in U_0$. 
\end{proof}
    




\begin{lemma} \label{lemma-AB}
    Let $A$ and $B$ be two real symmetric matrices. If $\det(A)\neq 0$, then we have 
    \begin{align}
        \det(A+\sqrt{-1}B)\neq 0.
    \end{align}
\end{lemma}
\begin{proof}
Since $\det(A)\neq 0$, we have
    \begin{align}
        \det(A+\sqrt{-1}B)=\det(A)\cdot\det(I+\sqrt{-1}A^{-1}B).
    \end{align}
    Moreover,   $A^{-1}B$ is a real symmetric matrix which has only real eigenvalues, it follows that
    $\det(I+\sqrt{-1}A^{-1}B)\neq 0$. 
\end{proof}

\begin{lemma} \label{lemma-hess}
    Suppose $V(z,w)$ is an analytic function on $z=t+\sqrt{-1}X$ and $w=s+\sqrt{-1}Y$. Define the function
    \begin{align}
        f(t,X,s,Y)=Re V(z,w).
    \end{align}
    Then we have 
    \begin{align}
      &f_{tt}=ReV_{zz},  \  f_{ts}=ReV_{zw}, \  f_{st}=ReV_{wz}, \ f_{ss}=Re V_{ww} \\\nonumber
      &f_{XX}=-ReV_{zz}, \ f_{XY}=-Re V_{zw}, \ f_{YX}=-Re V_{wz}, \ f_{YY}=-Re V_{ww} \\ \nonumber
      &f_{tX}=-Im V_{zz},  \  f_{tY}=-Im V_{zw}, \  f_{Xt}=-Im V_{zz}, \ f_{Yt}=-Im V_{wz} \\\nonumber
      &f_{sX}=-Im V_{zw}, \ f_{sY}=-Im V_{ww}, \ f_{Xs}=-Im V_{wz}, \ f_{Ys}=-Im V_{ww}. 
    \end{align}
\end{lemma} 
\begin{proof}
   Let $f(t,X,s,Y)=Re V(z,w), g(t,X,s,Y)=Im V(z,w)$.  
Then the Cauchy-Riemann equation gives
\begin{align}
    \frac{\partial V}{\partial \bar{z}}=0, \ \frac{\partial V}{\partial \bar{w}}=0.
\end{align}
It follows that
\begin{align}
    f_{t}=g_X, \ f_X=-g_t, \ f_{s}=g_{Y}, \ f_{Y}=-g_{s}. 
\end{align}
Therefore, 
\begin{align}
    f_{tt}=-f_{XX}=g_{tX}, \ f_{ss}=-f_{YY}=g_{sY} \\\nonumber
    g_{tt}=-g_{XX}=-f_{tX}, \ g_{ss}=-g_{YY}=-f_{sY}.
\end{align}
As an example, we compute
\begin{align}
    V_{zz}&=f_{zz}+ig_{zz}\\\nonumber
    &=\frac{1}{4}\left((\partial_t-\sqrt{-1}\partial_X)^2f+\sqrt{-1}(\partial_t-\sqrt{-1}\partial_X)^2g\right)\\\nonumber
    &=\frac{1}{4}\left(\left(f_{tt}-f_{XX}+2g_{tX}\right)+\sqrt{-1}(g_{tt}-g_{XX}-2f_{tX})\right)\\\nonumber
    &=f_{tt}-\sqrt{-1}f_{tX}.
\end{align}
Hence, we obtain $ReV_{zz}=f_{tt}$ and $Im V_{zz}=-f_{tX}$. The other identities can be proved similarly.   
\end{proof}


\begin{lemma} \label{lemma-Hess(f)}
    Let $Hess(f)$ be the Hessian matrix  of $f$,  if 
\begin{align}
    \begin{vmatrix}
    f_{XX} & f_{XY} \\
    f_{XY}  &  f_{YY}
    \end{vmatrix}\neq 0, 
\end{align}
then we have 
    \begin{align}
        \det(Hess(f))=\begin{vmatrix}
    f_{tt} & f_{ts} & f_{tX} & f_{tY} \\
     f_{st} & f_{ss} & f_{sX} & f_{sY} \\
    f_{Xt} & f_{Xs} & f_{XX} & f_{XY} \\
    f_{Yt} & f_{Ys} & f_{YX} & f_{YY}
    \end{vmatrix}>0. 
    \end{align}
\end{lemma}
\begin{proof}
By Lemma \ref{lemma-hess}, we have
\begin{align}
Hess(f)&=\begin{pmatrix}
    f_{tt} & f_{ts} & f_{tX} & f_{tY} \\
     f_{st} & f_{ss} & f_{sX} & f_{sY} \\
    f_{Xt} & f_{Xs} & f_{XX} & f_{XY} \\
    f_{Yt} & f_{Ys} & f_{YX} & f_{YY}
    \end{pmatrix}\\\nonumber
    &=\begin{pmatrix}
    ReV_{zz} & ReV_{zw} & -ImV_{zz}  & -ImV_{zw} \\
    ReV_{zw}  &  ReV_{ww} &  -ImV_{zw} & -ImV_{ww} \\
     -ImV_{zz} &  -ImV_{zw}  & -ReV_{zz}  & -ReV_{zw} \\
    -ImV_{zw} &  -ImV_{ww} & -ReV_{zw}   & -ReV_{ww} 
    \end{pmatrix}  \\\nonumber
    &=\begin{pmatrix}
    \overline{V_{zz}} & \overline{V_{zw}} & 0  & 0 \\
    \overline{V_{zw}}  &  \overline{V_{ww}} &  0 & 0 \\
     -ImV_{zz} &  -ImV_{zw}  & -V_{zz}  & -V_{zw} \\
    -ImV_{zw} &  -ImV_{ww} & -V_{zw}   & -V_{ww}. 
    \end{pmatrix}    
\end{align}


Therefore, we obtain 
\begin{align} \label{formula-appendix-hessf}
    \det(Hess(f))=\det\begin{pmatrix}
    \overline{V_{zz}} & \overline{V_{zw}} \\
    \overline{V_{zw}}  &  \overline{V_{ww}}
    \end{pmatrix}\cdot \det\begin{pmatrix}
    V_{zz} & V_{zw} \\
    V_{zw}  &  V_{ww}
    \end{pmatrix}=|\det\begin{pmatrix}
    V_{zz} & V_{zw} \\
    V_{zw}  &  V_{ww}
    \end{pmatrix}|^2       
\end{align}

Furthermore, 
\begin{align}
    \begin{pmatrix}
    V_{zz} & V_{zw} \\
    V_{zw}  &  V_{ww}
    \end{pmatrix}&= \begin{pmatrix}
    ReV_{zz}+\sqrt{-1}Im V_{zz} & ReV_{zw}+\sqrt{-1}Im V_{zw} \\
    ReV_{zw}+\sqrt{-1}Im V_{zw}  &  ReV_{ww}+\sqrt{-1} Im V_{ww}
    \end{pmatrix}\\\nonumber
    &=\begin{pmatrix}
    ReV_{zz} & ReV_{zw} \\
    ReV_{zw}  &  ReV_{ww}
    \end{pmatrix}+\sqrt{-1}\begin{pmatrix}
    Im V_{zz} & Im V_{zw} \\
    Im V_{zw}  &  Im V_{ww}
    \end{pmatrix}
\end{align}

Let 
\begin{align}
A=-\begin{pmatrix}
    f_{XX} & f_{XY} \\
    f_{XY}  &  f_{YY}
    \end{pmatrix}=\begin{pmatrix}
    ReV_{zz} & ReV_{zw} \\
    ReV_{zw}  &  ReV_{ww}
    \end{pmatrix}, \ B=\begin{pmatrix}
    Im V_{zz} & Im V_{zw} \\
    Im V_{zw}  &  Im V_{ww}
    \end{pmatrix},
    \end{align}
then both $A$ and $B$ are real positive matrices, and $\det(A)\neq 0$. Hence by Lemma \ref{lemma-AB},  we obtain 
\begin{align}
    \det \begin{pmatrix}
    V_{zz} & V_{zw} \\
    V_{zw}  &  V_{ww}
    \end{pmatrix}\neq 0. 
\end{align}
Finally, by formula (\ref{formula-appendix-hessf}), we obtain 
\begin{align}
    \det(Hess(f))>0. 
\end{align}
\end{proof}

Suppose $X(t,s)$ and $Y(t,s)$ be a solution to the following equations


\begin{equation} 
\left\{ \begin{aligned}
         \frac{\partial f}{\partial X}(t,X(t,s),s,Y(t,s))&=0, \\\nonumber
                  \frac{\partial f}{\partial Y}(t,X(t,s),s,Y(t,s))&=0.
                          \end{aligned} \right.
                          \end{equation}



\begin{lemma} 
    Let $h(t,s)=f(t,X(t,s),s,Y(t,s))$, then we have
    \begin{align} \label{formula-htt}
     h_{tt}=\frac{\begin{vmatrix}
    f_{tt} &  f_{tX} & f_{tY} \\
    f_{Xt} &  f_{XX} & f_{XY} \\
    f_{Yt} & f_{YX} & f_{YY}
    \end{vmatrix}}{\begin{vmatrix}
    f_{XX} & f_{XY} \\
    f_{XY}  &  f_{YY}
    \end{vmatrix}}=\frac{\begin{vmatrix}
    -f_{XX} &  f_{tX} & f_{tY} \\
    f_{Xt} &  f_{XX} & f_{XY} \\
    f_{Yt} & f_{YX} & f_{YY}
    \end{vmatrix}}{\begin{vmatrix}
    f_{XX} & f_{XY} \\
    f_{XY}  &  f_{YY}
    \end{vmatrix}},
\end{align}
    and
    \begin{align} \label{formula-htthss}
        h_{tt}h_{ss}-h_{ts}^2=\frac{\begin{vmatrix}
    f_{tt} & f_{ts} & f_{tX} & f_{tY} \\
     f_{st} & f_{ss} & f_{sX} & f_{sY} \\
    f_{Xt} & f_{Xs} & f_{XX} & f_{XY} \\
    f_{Yt} & f_{Ys} & f_{YX} & f_{YY}
    \end{vmatrix}}{\begin{vmatrix}
    f_{XX} & f_{XY} \\
    f_{XY}  &  f_{YY}
    \end{vmatrix}}.
    \end{align}
\end{lemma}
\begin{proof}
Since $f_X(t,X(t,s),s,Y(t,s))=0$ and $f_Y(t,X(t,s),s,Y(t,s))=0$, we have 
\begin{align}
    h_t&=f_t+f_XX_t+f_YY_t=f_t,\\\nonumber
h_s&=f_s+f_XX_s+f_YY_s=f_s,
\end{align}
and then
\begin{align} \label{formula-htthtshss}
h_{tt}&=f_{tt}+f_{tX}X_t+f_{tY}Y_t, \\\nonumber
h_{ts}&=f_{ts}+f_{tX}X_t+f_{tY}Y_s, \\\nonumber
h_{ss}&=f_{ss}+f_{sX}X_s+f_{sY}Y_s.
\end{align}

Moreover, from $f_X(t,X(t,s),s,Y(t,s))=f_Y(t,X(t,s),s,Y(t,s))=0$, we obtain 
\begin{align}
    f_{Xt}+f_{XX}X_t+f_{XY}Y_t&=0, \\\nonumber
    f_{Yt}+f_{YX}X_t+f_{YY}Y_t&=0.
\end{align}
It follows that
\begin{align} \label{formula-XtYt}
    X_t=\frac{f_{XY}f_{tY}-f_{YY}f_{tX}}{\begin{vmatrix}
f_{XX} & f_{XY}  \\
f_{XY} & f_{YY}  \\
\end{vmatrix}}, \  Y_t=\frac{f_{XY}f_{tY}-f_{XX}f_{tX}}{\begin{vmatrix}
f_{XX} & f_{XY}  \\
f_{XY} & f_{YY}  \\
\end{vmatrix}}.
\end{align}
Similarly, 
\begin{align} \label{formula-XsYs}
    X_s=\frac{f_{XY}f_{sY}-f_{YY}f_{sX}}{\begin{vmatrix}
f_{XX} & f_{XY}  \\
f_{XY} & f_{YY}  \\
\end{vmatrix}}, \  Y_s=\frac{f_{XY}f_{sY}-f_{XX}f_{sX}}{\begin{vmatrix}
f_{XX} & f_{XY}  \\
f_{XY} & f_{YY}  \\
\end{vmatrix}}.
\end{align}



Combining formulas (\ref{formula-htthtshss}), (\ref{formula-XtYt}) and (\ref{formula-XsYs}) together, we obtain 
(\ref{formula-htt}) and (\ref{formula-htthss})
\end{proof}


\begin{lemma} \label{lemma-htt<0}
    On the region 
    \begin{align}
      D_{H}=\{(t,s)\in \mathbb{R}^2|\frac{1}{2}<t<1, 1<t+s<\frac{3}{2}, 0<t-s<\frac{1}{2}\}, 
    \end{align}
    we have 
    \begin{align}
        h_{tt}<0. 
    \end{align}
\end{lemma}
\begin{proof}
    By formula (\ref{formula-htt}), we only need to show that 
    \begin{align} 
    \begin{vmatrix}
    -f_{XX} &  f_{tX} & f_{tY} \\
    f_{Xt} &  f_{XX} & f_{XY} \\
    f_{Yt} & f_{YX} & f_{YY}
    \end{vmatrix}<0. 
\end{align}

By Lemma \ref{lemma-HessXY}, on the region $D_{H}$, we have $f_{XX}>0$ and $\begin{vmatrix}
     f_{XX} & f_{XY} \\
     f_{YX} & f_{YY}
    \end{vmatrix}>0$. 
    For convenience, we set 
    \begin{align}
     \Delta=\begin{pmatrix}
     f_{XX} & f_{XY} \\
     f_{YX} & f_{YY}
    \end{pmatrix}, \   I_2=\begin{pmatrix}
     1 & 0 \\
     0 & 1
    \end{pmatrix} \ \text{and} \ \alpha=(f_{tX}, f_{tY}).  
    \end{align}  
    By using the identity
    \begin{align}
     \begin{pmatrix}
     -f_{XX} & \alpha \\
     \alpha^T & \Delta
    \end{pmatrix}=\begin{pmatrix}
     1 & 0 \\
      0 & \Delta
    \end{pmatrix}\cdot\begin{pmatrix}
     1 & \alpha \\
     0 & I_2
    \end{pmatrix}\cdot\begin{pmatrix}
     -f_{XX}-\alpha\Delta^{-1}\alpha^{T} & 0 \\
     \Delta^{-1}\alpha^{T} & I_2
    \end{pmatrix},
    \end{align}
we obtain 
\begin{align} 
    \begin{vmatrix}
    -f_{XX} &  f_{tX} & f_{tY} \\
    f_{Xt} &  f_{XX} & f_{XY} \\
    f_{Yt} & f_{YX} & f_{YY}
    \end{vmatrix}=\begin{vmatrix}
     -f_{XX} & \alpha \\
     \alpha^T & \Delta
    \end{vmatrix}=|\Delta|\cdot (-f_{XX}-\alpha\Delta^{-1}\alpha^{T}).
\end{align}

Since $|\Delta|>0$, $f_{XX}>0$ and $\alpha\Delta^{-1}\alpha^{T}>0$ by the positiveness of the matrix $\Delta$, it follows that
\begin{align}
    \begin{vmatrix}
    -f_{XX} &  f_{tX} & f_{tY} \\
    f_{Xt} &  f_{XX} & f_{XY} \\
    f_{Yt} & f_{YX} & f_{YY}
    \end{vmatrix}<0.
\end{align}  
\end{proof}

Now, let us finish the proof of Proposition \ref{prop-critical}. 
\begin{proof}
For $p\geq 6$, by Lemma \ref{lemma-topbottom}, we have $U_0\subset D_{H}$. By Proposition \ref{lemma-HessXY}, we get 
\begin{align}
    \begin{vmatrix}
f_{XX} & f_{XY}  \\
f_{XY} & f_{YY}  \\
\end{vmatrix}>0
\end{align}
on $U_0$. Hence, by Lemma \ref{lemma-htt<0} and Lemma \ref{lemma-Hess(f)}, we obtain 
\begin{align}
    h_{tt}<0 \ \text{and}  \ \det(Hess(h))=h_{tt}h_{ss}-h_{ts}^2>0.
\end{align}

On the other hand, by our construction of $h(t,s)$, we know that as $(t,s)$ goes to the boundary of the region $U_{0}$, $h(t,s)$ goes to a value less than $\frac{3.509}{2\pi}$ by Lemma \ref{lemma-regionD'0}.  Therefore, $h(t,s)$ has a unique maximal point which  is a  critical point denoted by $(t_{0R},s_{0R})$ lying in the region $U_0$.   
    Let $(t_0,s_0)=(t_{0R}+\sqrt{-1}X(t_{0R},s_{0R}),s_{0R}+\sqrt{-1}Y(t_{0R},s_{0R}))$, then 
    $(t_0,s_0)$ is the unique critical point of $V(p,t,s)$ with $(Re(t_0),Re(s_0))=(t_{0R},s_{0,R})\in U_0$. 

Furthermore, by Lemma \ref{lemma-t0s0}, we know that if $(t'_0,s'_0)$ is another critical point of the potential function $V(p,t,s)$ with  $(t'_{0R},s'_{0R})\in D'_{0}$, then $(t'_{0R},s'_{0R})\in U_0$. Therefore, by the above argument, it follows that $(t'_0,s'_0)=(t_0,s_0)$. Hence, there is a unique critical point $(t_0,s_0)$ of $V(p,t,s)$ in the region $D'_0$.     
\end{proof}
\subsection{Proof of Lemma \ref{lemma-volume}} \label{appendix-volume}
First, applying the standard work of \cite{NZ85,Yoshida85} to the hyperbolic equation for the complement of the twist knot in $S^3$, we obtain 
\begin{proposition}
The hyperbolic gluing equation for $\mathcal{K}_{p}$ can be written in the following form
\begin{align} \label{formula-hyperequ}
    \log(w-1)+2(-1+2q)\log
(-\frac{1}{w})+\log(-\frac{1}{w}-1)=(3-2p)\pi\sqrt{-1}
\end{align}
with
\begin{align}
 \log w+\log(-\frac{1}{w})=-\pi\sqrt{-1}.   
\end{align}
Suppose $w_0$ is a solution of the above equation, then we have
\begin{align}
   &vol(S^3\setminus \mathcal{K}_{p})+\sqrt{-1}cs((S^3\setminus \mathcal{K}_{p})\\\nonumber
&=\sqrt{-1}\left(R(w_0)+R(-\frac{1}{w_0})+R(\frac{1}{1-w_0})+R(\frac{w_0}
{w_0+1})\right)-\frac{\pi}{2}(2\pi\sqrt{-1}\\\nonumber
&+\frac{2\pi\sqrt{-1}}{p}+\frac
{\log(w_0-1)+2\log(-\frac{1}{w_0})-\log(-\frac{1}{w_0}-1)+\pi\sqrt{-1}}
{p})\operatorname{mod}\pi^{2}\sqrt{-1}\mathbb{Z}
\end{align}
where 
\begin{align}
  R(U)=\frac{1}{2}\log U\log(1-U)+Li_{2}(U).   
\end{align}
\end{proposition}
Now, we going to prove Lemma \ref{lemma-volume}.
\begin{proof}
Recall that the potential function for $\mathcal{K}_p$ is given by
\begin{align} 
    &V(p,t,s)=\pi \sqrt{-1}\left((2p+1)s^2-(2p+3)s-2t\right)\\\nonumber
    &+\frac{1}{2\pi\sqrt{-1}}\left(\text{Li}_2(e^{2\pi\sqrt{-1}(t+s)})+\text{Li}_2(e^{2\pi\sqrt{-1}(t-s)})-3\text{Li}_2(e^{2\pi\sqrt{-1}t})+\frac{\pi^2}{6}\right). 
\end{align}
Suppose $t_0,s_0$ are the critical point of $V(p,t,s)$. I.e. $t_0,s_0$ satisfies equations (\ref{equation-critical1}) and (\ref{equation-critical2}).  Set $x_0=e^{2\pi \sqrt{-1}t_0}$ and $y_0=e^{2\pi\sqrt{-1}s_0}$, then we obtain
\begin{align} \label{formula-tran1}
    x_0=\sqrt{y}_0-\frac{1}{\sqrt{y}_0}+1. 
\end{align}
Furthermore, comparing with hyperbolic equation (\ref{formula-hyperequ}), we have  
\begin{align} \label{formula-tran2}
    w_0=\frac{1}{\sqrt{y}_0}.
\end{align}
 

Now we are going to prove the following identity
\begin{align}
    2\pi V(p,t_0,s_0)=vol(S^3\setminus \mathcal{K}_p)+\sqrt{-1}cs(\mathcal{S}^3\setminus \mathcal{K}_p)-(p+5)\pi^2\sqrt{-1}
\end{align}
which is just the statement of Lemma \ref{lemma-volume}.
First,
\begin{align}
&Vol(K_{p})+\sqrt{-1}CS(K_{p})+2\pi^{2}\sqrt{-1}\\\nonumber
    &=\sqrt{-1}\left(\frac{1}{2}\log w_0\log(1-w_0)+Li_{2}(w_0)+\frac{1}{2}
\log(-\frac{1}{w_0})\log(1+\frac{1}{w_0})+Li_{2}(-\frac{1}{w_0})\right.\\\nonumber
&\left.+\frac{1}{2}
\log\frac{1}{1-w_0}\log(1-\frac{1}{1-w_0})+Li_{2}(\frac{1}{1-w_0})
+\frac{1}{2}\log\frac{w_0}{w_0+1}\log(1-\frac{w_0}{w_0+1})+Li_{2}(\frac{w_0}
{w_0+1})\right)\\\nonumber
&-\frac{\pi}{2}(-2\pi\sqrt{-1}+\frac{2\pi\sqrt{-1}}{p}+\frac
{\log(w_0-1)+2\log(-\frac{1}{w_0})-\log(-\frac{1}{w_0}-1)+\pi\sqrt{-1}}{p})
\end{align}

$w_0$ satisfies the equation (\ref{formula-hyperequ}), we have
\begin{align}
    p=\frac{\log(w_0-1)+2\log(-\frac{1}{w_0})-\log(-\frac{1}{w_0}-1)+3\pi
\sqrt{-1}}{2(2\log(-\frac{1}{w_0})+\pi\sqrt{-1})}
\end{align}


Then, we obtain
\begin{align}
&vol(S^3\setminus \mathcal{K}_{p})+\sqrt{-1}cs(S^3\setminus{K}_{p})+2\pi^{2}\sqrt{-1}\\\nonumber
   &=\sqrt{-1}\left(\frac{1}{2}\log w_0\log(1-w_0)+Li_{2}(w_0)+\frac{1}{2}
\log(-\frac{1}{w_0})\log(1+\frac{1}{w})+Li_{2}(-\frac{1}{w_0})\right.\\\nonumber
&\left.+\frac{1}{2}
\log\frac{1}{1-w_0}\log(1-\frac{1}{1-w_0})+Li_{2}(\frac{1}{1-w_0})+\frac{1}{2}\log\frac{w_0}{w_0+1}\log\frac{1}{w+1}+Li_{2}(\frac{w_0}{w_0+1}
)\right)\\\nonumber
&-\pi(-\pi\sqrt{-1}+\frac{(2\pi\sqrt{-1}+\log(w_0-1)+2\log(-\frac{1}{w_0}
)-\log(-\frac{1}{w_0}-1)+\pi\sqrt{-1})(2\log(-1/w_0)+\pi\sqrt{-1})}{3\pi\sqrt
{-1}+\log(w_0-1)+2\log(-\frac{1}{w_0})-\log(-\frac{1}{w_0}-1)})
\end{align}
which now is a function  $w_0$, denoted by $F(w_0)$. 

On the other hand, since $(t_0,s_0)$ satisfies the critical equations (\ref{equation-critical1}) and (\ref{equation-critical2}), we have 
\begin{align}
    p=\frac{\left(  3\pi\sqrt{-1}-2\pi\sqrt{-1}s_{0}+\log(1-e^{2\pi\sqrt
{-1}(t_{0}+s_{0})})-\log(1-e^{2\pi\sqrt{-1}(t_{0}-s_{0})})\right)  }
{2(2s_{0}-1)\pi\sqrt{-1}}.
\end{align}

Therefore,
\begin{align}
    & 2\pi(V(p,t_0,s_0)+\frac{p+7}{2}\pi\sqrt{-1})\\\nonumber
&=2\pi^{2}\sqrt{-1}(-2t_{0}-s_{0}+\frac{11}{4})+\frac{(2s_{0}-1)\pi}{2}\left(
\log(1-e^{2\pi\sqrt{-1}(t_{0}+s_{0})})-\log(1-e^{2\pi\sqrt{-1}(t_{0}-s_{0})})\right) \\\nonumber
&+\frac{1}{\sqrt{-1}}\left(  \frac{\pi^{2}}{6}-3Li_{2}(e^{2\pi\sqrt{-1}t_{0}
})+Li_{2}(e^{2\pi\sqrt{-1}(t_{0}+s_{0})})+Li_{2}(e^{2\pi\sqrt{-1}(t_{0}
-s_{0})})\right).  
\end{align}

By using the formulas (\ref{formula-tran1}) and (\ref{formula-tran2}), we obtain 
\begin{align}
&2\pi(V(p,t_0,s_0)+\frac{p+7}{2}\pi\sqrt{-1})\\\nonumber
    &=\pi(-2\log(w_0-\frac{1}{w_0}+1)+2\log w_0+\frac{3\pi\sqrt{-1}}{2})+\pi
(-\frac{\log w_0}{\pi\sqrt{-1}}-\frac{1}{2})\\\nonumber
&\cdot\left(  \log(1-(w_0-\frac{1}
{w_0}+1)/w_0^{2})-\log(1-(w_0-\frac{1}{w_0}+1)w_0^{2})\right)  \\\nonumber
&+\frac{1}{\sqrt{-1}}\left(  \frac{\pi^{2}}{6}-3Li_{2}(w_0-\frac{1}{w_0}
+1)+Li_{2}((w_0-\frac{1}{w_0}+1)/w_0^{2})+Li_{2}((w_0-\frac{1}{w_0}+1)w_0^{2})\right)  
\end{align}
Which now is a function $w_0$, denoted by $G(w_0)$.

Finally, by some tedious calculations of the derivative, we can prove that $F(w_0)=G(w_0)$ and finish the proof of Lemma \ref{lemma-volume}
\end{proof}






























\subsection{Proof of Lemma \ref{lemma-volumeestimate}}  \label{appendix-2}
In this section, we prove the  Lemma \ref{lemma-volumeestimate} which gives the estimation of the critical value $\zeta_{\mathbb{R}}(p)$. 
\begin{proof}
    Recall that $\zeta_{\mathbb{R}}(p)$ is given by
\begin{align}
        \zeta_{\mathbb{R}}(p)=Re V(p,t_0,s_0), 
\end{align}
where $(t_0,s_0)$ is the unique solution of the equations
\begin{align}  
    \frac{\partial V(p,t,s)}{\partial t}&=-2\pi\sqrt{-1}+3\log(1-e^{2\pi\sqrt{-1}t})\\\nonumber
    &-\log(1-e^{2\pi\sqrt{-1}(t+s)})-\log(1-e^{2\pi\sqrt{-1}(t-s)})=0,
\end{align}
\begin{align} 
    \frac{\partial V(p,t,s)}{\partial s}&=(4p+2)\pi\sqrt{-1}s-(2p+3)\pi\sqrt{-1}\\\nonumber
    &-\log(1-e^{2\pi\sqrt{-1}(t+s)})+\log(1-e^{2\pi\sqrt{-1}(t-s)})=0. 
\end{align}

Putting $\gamma=\frac{1}{p}$, we regard $(t_0,s_0)$ as a function of $\gamma$, and the denote it by $(t(\gamma),s(\gamma))$,  then 
\begin{align}
    \zeta_{\mathbb{R}}(p)=Re V(p,t(\gamma),s(\gamma)). 
\end{align}

By expanding the above equations, we obtain the expansions of $t(\gamma)$ and $s(\gamma)$ as follows.
\begin{align}
    t(\gamma)&=\frac{\log(1-2\sqrt{-1})}{2\pi\sqrt{-1}}+1+\frac{(1+2\sqrt{-1})\pi}{40}\gamma^2+\frac{(3+\sqrt{-1})\pi}{80}\gamma^3\\\nonumber
    &+(\frac{180\pi+19\pi^3}{9600}-\frac{45\pi-4\pi^3}{4800}\sqrt{-1})\gamma^4+O(\gamma^5),\\\nonumber
    s(\gamma)&=\frac{1}{2}+\frac{1}{2}\gamma+\frac{1-\sqrt{-1}}{8}\gamma^2-\frac{\sqrt{-1}}{16}\gamma^3\\\nonumber
    &-(\frac{1}{61}+\frac{3+\pi^2}{192}\sqrt{-1})\gamma^4+O(\gamma^5). 
\end{align}
For brevity, we write $t(\gamma)$ and $s(\gamma)$ as
\begin{align}
    t(\gamma)&=\frac{\log(1-2\sqrt{-1})}{2\pi\sqrt{-1}}+1+\hat{t}(\gamma)\gamma^2, \\\nonumber
    s(\gamma)&=\frac{1}{2}+\frac{1}{2}\gamma+\hat{s}(\gamma)\gamma^2. 
\end{align}

By using Taylor expansion, we obtain 
\begin{align}
 2\pi V(p,t(\gamma),s(\gamma))&=v_8-(p+\frac{23}{4})\pi^2\sqrt{-1}-\pi^2 \sqrt{-1}\gamma-\pi^2\frac{1+\sqrt{-1}}{4}\gamma^2\\\nonumber
 &-\pi^2\frac{1}{8}\gamma^3-\pi^2\frac{6+\pi^2-6\sqrt{-1}}{192}\gamma^4+O(\gamma^5).   
\end{align}
Therefore
\begin{align}
    2\pi \zeta_{\mathbb{R}}(p)=2\pi Re V(p,t(\gamma),s(\gamma))=v_8-\pi^2\frac{1}{4}\gamma^2-\pi^2\frac{1}{8}\gamma^3-\pi^2 \frac{6+\pi^2}{192}\gamma^4+O(\gamma^5).
\end{align}
From this estimation, we obtain Lemma \ref{lemma-volumeestimate} holds. Actually, we can numerically verify Lemma \ref{lemma-volumeestimate} holds for $2\leq p\leq 1000$. As for $p\geq 1001$, one can prove Lemma \ref{lemma-volumeestimate} with basic but tedious calculations. 
\end{proof}

\subsection{Proof of Proposition \ref{prop-saddleonedim}}  \label{appendix-onesaddle}
For a fix constant $c\in \mathbb{R}$, we define the subset 
\begin{align}
    D'_0(c)=\{(t,s)\in D'_0| s=c\}.
\end{align}
We prove Proposition \ref{prop-saddleonedim} by proving Proposiiton \ref{prop-saddleonedim1} and Proposition \ref{prop-saddleonedim2} in the following.
\begin{proposition} \label{prop-saddleonedim1}
    For $c_{upper}(p)\leq c<1$ and $n\in \mathbb{Z}$, there exists a constant $C$ independent of $c$, such that
    \begin{align}
        |\int_{D'_0(c)} e^{(N+\frac{1}{2})V_N(p,t,c;0,n)}dt|<Ce^{(N+\frac{1}{2})\left(\zeta_{\mathbb{R}}(p)-\epsilon\right)}. 
    \end{align}
\end{proposition}
$D'_{0}(c)$ is a slice of the region $D'_0$, we will prove Proposition \ref{prop-saddleonedim} by using the saddle point method on $D'_0(c)$.
Recall that
\begin{align}
    &V(p,t,s; m,n)=\pi \sqrt{-1}\left((2p+1)s^2-(2p+3+2n)s-(2m+2)t\right)\\\nonumber
    &+\frac{1}{2\pi\sqrt{-1}}\left(\text{Li}_2(e^{2\pi\sqrt{-1}(t+s)})+\text{Li}_2(e^{2\pi\sqrt{-1}(t-s)})-3\text{Li}_2(e^{2\pi\sqrt{-1}t})+\frac{\pi^2}{6}\right),
\end{align}
so we have
\begin{align}
    \frac{\partial V(p,t,s; 0,n)}{\partial t}&=-2\pi\sqrt{-1}+3\log(1-e^{2\pi\sqrt{-1}t})\\\nonumber
    &-\log(1-e^{2\pi\sqrt{-1}(t+s)})-\log(1-e^{2\pi\sqrt{-1}(t-s)}),
\end{align}
and
\begin{align}
    \frac{\partial V(p,t,s; 0,n)}{\partial s}&=(4p+2)\pi\sqrt{-1}s-(2p+3+2n)\pi\sqrt{-1}\\\nonumber
    &-\log(1-e^{2\pi\sqrt{-1}(t+s)})+\log(1-e^{2\pi\sqrt{-1}(t-s)}).
\end{align}


\begin{proposition} \label{prop-critical1}
    Fixing $s=c\in [\frac{1}{2},1)$, as a function of $t$,  $V(p,t,c;0,n)$ has a unique critical point 
    $T_1(c)$ with $t_1(c)=Re(T_1(c))\in (\frac{1}{2},1)$.     
    \end{proposition} 
\begin{proof}
Consider the equation
\begin{align} \label{formula-equationonedim}
    \frac{dV(p,t,c;0,n)}{dt}&=-2\pi \sqrt{-1}+3\log(1-e^{2\pi\sqrt{-1}t})\\\nonumber
    &-\log(1-e^{2\pi\sqrt{-1}(t+c)})-\log(1-e^{2\pi\sqrt{-1}(t-c)})=0
\end{align}
which gives 
\begin{align}
    x^2-2x+3-C-\frac{1}{C}=0
\end{align}
where $x=e^{2\pi\sqrt{-1}t}$, $C=e^{2\pi\sqrt{-1}c}$.

So we obtain 
\begin{align}
    x=1\pm 2\sqrt{-1}\sin(\pi c). 
\end{align}
Let $T_\pm (c)$ be the solution determined by the equation 
\begin{align} \label{formula-equaitonT-1c}
    e^{2\pi\sqrt{-1}T_\pm (c)}=1\pm 2\sqrt{-1}\sin(\pi c). 
\end{align}
 From (\ref{formula-equaitonT-1c}), we have
\begin{align}
    T_\pm(c)=\frac{\log(1\pm 2\sqrt{-1}\sin(\pi c))}{2\pi\sqrt{-1}} +\mathbb{Z}.
\end{align}
Then 
\begin{align}
    Re(T_\pm(c))=\frac{arg(1\pm 2\sqrt{-1}\sin(\pi c))}{2\pi}+\mathbb{Z}. 
\end{align}

By $c\in [\frac{1}{2},1)$, we obtain 
\begin{align}
  0<arg(1+2\sqrt{-1}\sin(\pi c))<\arctan(2)<1.2, \\\nonumber
  -1.2<-\arctan(2)<arg(1-2\sqrt{-1}\sin(\pi c))<0. 
\end{align}
Therefore, one can see that only the solution $T_-(c)=\frac{\log(1-2\sqrt{-1}\sin(\pi c))}{2\pi\sqrt{-1}}+1$ satisfies that 
$Re(T_-(c))\in (\frac{1}{2},1)$. Moreover, by the following Lemma \ref{lemma-T1(c)iscritical}, we know that $T_-(c)$ satisfies the equation (\ref{formula-equationonedim}), so $T_-(c)$ is indeed a critical point of $V(p,t,c;0,n)$.  
In the following, we will denote $T_-(c)$ by $T_1(c)=t_1(c)+\sqrt{-1}X_1(c)$  for convenience.   
\end{proof}

\begin{lemma} \label{lemma-T1(c)iscritical}
$T_-(c)=\frac{\log(1-2\sqrt{-1}\sin(\pi c))}{2\pi\sqrt{-1}}+1$ satisfies the equation (\ref{formula-equationonedim}).
\end{lemma}
\begin{proof}
   The equation (\ref{formula-equationonedim}) is equivalent to 
   \begin{equation} 
\left\{ \begin{aligned}
        &x^2-2x+3-C-\frac{1}{C}=0, \\
                  &3arg(1-x)-arg(1-Cx)-arg(1-C^{-1}x)=2\pi,
                          \end{aligned} \right.
                          \end{equation}
where $x=e^{2\pi\sqrt{-1}t}$, $C=e^{2\pi\sqrt{-1}c}$.

Clearly, $x_0=e^{2\pi\sqrt{-1}T_-(c)}=1-2\sqrt{-1}\sin(\pi c)$ has admitted the first equation, and we have the equation
\begin{align} \label{formula-arguementequ}
    3arg(1-x_0)-arg(1-Cx_0)-arg(1-C^{-1}x_0)=2k\pi
\end{align}
for some $k\in \mathbb{Z}$.  In the following, we show $k=1$. Indeed, for $c\in [\frac{1}{2},1)$, we have
\begin{align}
    &3arg(1-x_0)=3arg(2\sqrt{-1}\sin (\pi c))=\frac{3}{2}\pi,  \\\nonumber
    &arg(1-Cx_0)\\\nonumber
    &=arg(1-2\sin(\pi c)\sin(2\pi c)-\cos(2\pi c)+\sqrt{-1}(2\sin(\pi c)\cos(2\pi c)-\sin(2\pi c)))\\\nonumber
    &\in [-\frac{\pi}{4},\frac{\pi}{2})\\\nonumber
    &arg(1-C^{-1}x_0),\\\nonumber
    &=arg(1+2\sin(\pi c)\sin(2\pi c)-\cos(2\pi c)+\sqrt{-1}(2\sin(\pi c)\cos(2\pi c)+\sin(2\pi c)))\\\nonumber
    &\in (-\frac{\pi}{2},\frac{\pi}{2}).
\end{align}
Therefore,  we obtain  
\begin{align}
    3arg(1-x_0)-arg(1-Cx_0)-arg(1-C^{-1}x_0)\in (\frac{\pi}{2},3\pi), 
\end{align}
which implies  $k=1$ in formula (\ref{formula-arguementequ}).  
\end{proof}



\begin{lemma}
We have the following identities:
\begin{align}
    Re\left(\log\left(1-e^{2\pi\sqrt{-1}(T_1(c)+c)}\right)\right)&=\log\left(4\sin(\pi c)\sin\left(\frac{\pi c}{2}\right)\right), \label{formula-iden1}\\
     Re\left(\log\left(1-e^{2\pi\sqrt{-1}(T_1(c)-c)}\right)\right)&=\log\left(4\sin(\pi c)\cos\left(\frac{\pi c}{2}\right)\right), \label{formula-iden2}\\
     Re\left(-\log\left(1-e^{2\pi\sqrt{-1}(T_1(c)+c)}\right)\right.&\left.+\log\left(1-e^{2\pi\sqrt{-1}(T_1(c)-c)}\right)\right)&=\log\left(\cot\left(\frac{\pi c}{2}\right)\right). \label{formula-iden3}
\end{align}
\end{lemma}
\begin{proof}
By straightforward computations, we obtain
\begin{align*}
    &Re\left(\log\left(1-e^{2\pi\sqrt{-1}(T_1(c)+c)}\right)\right)\\\nonumber
    &=Re\left(\log\left(1-(1- 2\sqrt{-1}\sin(\pi c))e^{2\pi\sqrt{-1}c}\right)\right)\\\nonumber
    &=Re\left(\log(1-2\sin(\pi c)\sin(2\pi c)-\cos(2\pi c)+\sqrt{-1}(2\sin(\pi c)\cos(2\pi c))-\sin(2\pi c))\right)\\\nonumber
    &=\frac{1}{2}\log\left(1-2\sin(\pi c)\sin(2\pi c)-\cos(2\pi c)^2+(2\sin(\pi c)\cos(2\pi c))-\sin(2\pi c))^2\right)\\\nonumber
    &=\log\left(4\sin(\pi c)\sin\left(\frac{\pi c}{2}\right)\right)
\end{align*}
and
\begin{align*}
    &Re\left(\log\left(1-e^{2\pi\sqrt{-1}(T_1(c)-c)}\right)\right)\\\nonumber
    &=Re\left(\log\left(1-(1- 2\sqrt{-1}\sin(\pi c))e^{-2\pi\sqrt{-1}c}\right)\right)\\\nonumber
    &=Re\left(\log(1+2\sin(\pi c)\sin(2\pi c)-\cos(2\pi c)+\sqrt{-1}(2\sin(\pi c)\cos(2\pi c))+\sin(2\pi c))\right)\\\nonumber
    &=\frac{1}{2}\log\left(1+2\sin(\pi c)\sin(2\pi c)-\cos(2\pi c)^2+(2\sin(\pi c)\cos(2\pi c))+\sin(2\pi c))^2\right)\\\nonumber
    &=\log\left(4\sin(\pi c)\cos\left(\frac{\pi c}{2}\right)\right),
\end{align*}
which prove the identities (\ref{formula-iden1}) and (\ref{formula-iden2}). Then the identity (\ref{formula-iden3}) follows from identities (\ref{formula-iden1}) and (\ref{formula-iden2})  immediately.
\end{proof} 







\begin{lemma} \label{lemma-ReVTc}
    As a function of $c\in [\frac{1}{2},1)$, $Re V(p,T_1(c),c;0,n)$ is a decreasing function of $c$. Furthermore, we have 
\begin{align}
    Re V(p,T_1(c),c;0,n)=2\left(\Lambda\left(\frac{c}{2}\right)-\Lambda\left(\frac{1}{2}-\frac{c}{2}\right)\right).
\end{align}
\end{lemma}

\begin{proof}
From the equation (\ref{formula-equaitonT-1c}), we obtain 
\begin{align}
    \frac{dT_1(c)}{dc}=\frac{-\cos(\pi c)}{e^{2\pi \sqrt{-1}T_1(c)}}=\frac{-\cos(\pi c)}{1-2\sqrt{-1}\sin(\pi c)}.
\end{align}
Then 
\begin{align}
    &\frac{d Re V(p,T_1(c),c;0,n)}{dc}\\\nonumber
    &=Re\left(\frac{\partial V(p,T_1(c),c;0,n)}{\partial t}\frac{dT_1(c)}{dc}+\frac{\partial V(p,T_1(c),c;0,n)}{\partial s}\frac{dc}{dc}\right)\\\nonumber
    &=Re\left(\frac{\partial V(p,T_1(c),c;0,n)}{\partial s}\right)\\\nonumber
    &=Re\left(-\log(1-e^{2\pi\sqrt{-1}(T_1(c)+c)})+\log\left(1-e^{2\pi\sqrt{-1}(T_1(c)-c)}\right)\right).
\end{align}
By identity (\ref{formula-iden3}), we obtain 
\begin{align}
    \frac{d Re V(p,T_1(c),c;0,n)}{dc}=\log\left(\cot\left(\frac{\pi c}{2}\right)\right)<0,
\end{align}
since $\cot\left(\frac{\pi c}{2}\right)<1$ for $\frac{1}{2}<c<1$. Hence $Re V(p,T_1(c),c;0,n)$ is a decreasing function. 

For $c\geq \frac{1}{2}$, we have
\begin{align}
    &Re V(p,T_1(c),c;0,n)-Re V\left(p,0,n,T_1\left(\frac{1}{2}\right),\frac{1}{2}\right)\\\nonumber
    &=\int_{\frac{1}{2}}^{c}\log \left(\cot\left(\frac{\pi \tau}{2}\right)\right)d\tau\\\nonumber
    &=\int_{\frac{1}{2}}^{c}\log \left(2\cos\left(\frac{\pi \tau}{2}\right)\right)d\tau-\int_{\frac{1}{2}}^{c}\log \left(2\sin\left(\frac{\pi \tau}{2}\right)\right)d\tau. 
\end{align}
Let $x=\frac{1}{2}(1-\tau)$, we obtain 
\begin{align}
    \int_{\frac{1}{2}}^{c}\log \left(2\cos\left(\frac{\pi \tau}{2}\right)\right)d\tau =-2\int_{\frac{1}{4}}^{\frac{1}{2}(1-c)}\log(2\sin \pi x)dx=2\left(\Lambda\left(\frac{1}{2}-\frac{c}{2}\right)-\Lambda\left(\frac{1}{4}\right)\right).
\end{align}
Let $y=\frac{\tau}{2}$, we obtain
\begin{align}
    \int_{\frac{1}{2}}^{c}\log \left(2\sin\left(\frac{\pi \tau}{2}\right)\right)d\tau=2\int_{\frac{1}{4}}^{\frac{c}{2}}\log(2\sin (\pi y))dy=-2\left(\Lambda\left(\frac{c}{2}\right)-\Lambda\left(\frac{1}{4}\right)\right).
\end{align}
Hence, for $c>\frac{1}{2}$, we have 
\begin{align}
     Re V(p,T_1(c),c;0,n)=2\left(\Lambda\left(\frac{c}{2}\right)-\Lambda\left(\frac{1}{2}-\frac{c}{2}\right)\right).
\end{align}
\end{proof}
Let $(t_0,s_0)=(t_{0R}+X_0\sqrt{-1},s_{0R}+Y_0\sqrt{-1})$ be a critical point of the potential function $V(p,t,s)$ as given in  Proposition \ref{prop-critical}.  

By the proof of Lemma \ref{lemma-volumeestimate} in Appendix \ref{appendix-2}, we have $0<\zeta_{\mathbb{R}}(p)<\frac{v_8}{2\pi}$.  We assume $c(p)$ is a solution to the following equation
\begin{align}
    Re V(p,T_1(c),c)=2 \left(\Lambda\left(\frac{c}{2}\right)+\Lambda\left(\frac{1}{2}-\frac{c}{2}\right)\right)=\zeta_{\mathbb{R}}(p).
\end{align}

\begin{lemma}
We have the following inequality:
\begin{align}  \label{formula-cleqcupp}
c(p)<c_{upper}(p).
\end{align}
\end{lemma}
\begin{proof}
Let $h(c)=\Lambda(\frac{c}{2})+\Lambda(\frac{1}{2}-\frac{c}{2})$, then
$2h(c(p))=\zeta_{\mathbb{R}}(p)$. 
 By Lemma \ref{lemma-convergent}, we have the following convergent power series
\begin{align}
    h(c)=2\Lambda\left(\frac{1}{4}\right)-\frac{\pi}{4}\left(c-\frac{1}{2}\right)^2-\frac{\pi^3}{48}\left(c-\frac{1}{2}\right)^4-\frac{\pi^5}{144}\left(c-\frac{1}{2}\right)^6-\cdots. 
\end{align}
Then we obtain 
\begin{align}
    2h(c_{upper}(p))&=\zeta_{\mathbb{R}}(p)-2\left(\frac{\pi^3}{48}\left(c_{upper}(p)-\frac{1}{2}\right)^4+\frac{\pi^5}{144}\left(c_{upper}(p)-\frac{1}{2}\right)^6+\cdots\right)\\\nonumber
    &<\zeta_{\mathbb{R}}(p)=2 h(c(p)).
\end{align}
Hence, by Lemma \ref{lemma-ReVTc}, we obtain $c(p)<c_{upper}(p)$. 
\end{proof}

As a consequence of formula (\ref{formula-cleqcupp}), we have
\begin{corollary} \label{coro-ReV1}
   For $c>c_{upper}(p)$, we have 
   \begin{align*}
   ReV(p,T_1(c),c)<ReV(p,T_1(c_{upper}(p)),c_{upper}(p))<ReV(p,T_1(c(p)),c(p))=\zeta_{\mathbb{R}}(p).
   \end{align*}
\end{corollary}
    



\begin{lemma} \label{lemma-convergent}
For $\frac{1}{2}<c<1$, we have the following convergent power series 
\begin{align}
    h(c)=2\Lambda\left(\frac{1}{4}\right)-\frac{\pi}{4}\left(c-\frac{1}{2}\right)^2-\frac{\pi^3}{48}\left(c-\frac{1}{2}\right)^4-\frac{\pi^5}{144}\left(c-\frac{1}{2}\right)^6-\cdots.
\end{align}
\end{lemma}
\begin{proof}
  We use the following power series expansion for $\sec(x)$,
\begin{align} \label{formula-sec}
    \sec(x)=1+\frac{1}{2}x+\frac{5}{24}x^4+\cdots, \ \text{for} \ |x|<\frac{\pi}{2}
    \end{align}
From  $h(c)=\Lambda(\frac{c}{2})+\Lambda(\frac{1}{2}-\frac{c}{2})$, we obtain 
\begin{align}
    h'(c)&=\frac{1}{2}\log \cot\left(\frac{\pi c}{2}\right), \\\nonumber
    h''(c)&=-\frac{\pi }{2\sin(\pi c)}=-\frac{\pi}{2}\sec\left(\pi\left(c-\frac{1}{2}\right)\right). 
\end{align}
    Since $0<\left(c-\frac{1}{2}\right)\pi<\frac{\pi}{2} $, by using formula (\ref{formula-sec}), we obtain 
    \begin{align}
        h'(c)&=\int_{\frac{1}{2}}^ch''(t)dt-h'\left(\frac{1}{2}\right)\\\nonumber
        &=-\frac{\pi}{2}\int_{\frac{1}{2}}^c \sec\left(\pi\left(t-\frac{1}{2}\right)\right)dt\\\nonumber
        &=-\frac{1}{2}\int_{0}^{\pi\left(c-\frac{1}{2}\right)}\sec(x) dx\\\nonumber
        &=-\frac{1}{2}\int_{0}^{\pi\left(c-\frac{1}{2}\right)}\left(1+\frac{1}{2}x^2+\frac{5}{24}x^4+\cdots\right)dx\\\nonumber
        &=-\frac{1}{2}\left(\pi\left(c-\frac{1}{2}\right)+\frac{1}{6}\left(\pi(c-\frac{1}{2})\right)^3+\frac{1}{24}\left(\pi(c-\frac{1}{2})\right)^5+\cdots\right)
    \end{align}
    Hence 
    \begin{align}
        h(c)&=\int_{\frac{1}{2}}^c h'(t)dt+h\left(\frac{1}{2}\right)\\\nonumber
        &=-\frac{1}{2}\int_{\frac{1}{2}}^c\left(\pi\left(t-\frac{1}{2}\right)+\frac{\pi^3}{6}\left(t-\frac{1}{2}\right)^3+\frac{\pi^5}{24}\left(t-\frac{1}{2}\right)^5\right)dt+2\Lambda\left(\frac{1}{4}\right)\\\nonumber
        &=2\Lambda\left(\frac{1}{4}\right)-\frac{1}{2\pi}\int_{0}^{\pi\left(c-\frac{1}{2}\right)}\left(x+\frac{1}{6}x^3+\frac{1}{24}x^5+\cdots\right)dx\\\nonumber
        &=2\Lambda\left(\frac{1}{4}\right)-\frac{\pi}{4}\left(c-\frac{1}{2}\right)^2-\frac{\pi^3}{48}\left(c-\frac{1}{2}\right)^4-\frac{\pi^5}{144}\left(c-\frac{1}{2}\right)^6-\cdots.
    \end{align}
\end{proof}

  

\begin{lemma} \label{lemma-1-dim-hess}
    For $t\in D'_0(c)$ and $n\in \mathbb{Z}$, we have 
    \begin{align}
        \frac{\partial^2 ReV(p,t+X\sqrt{-1},c;0,n)}{\partial X^2}>0.
    \end{align}
\end{lemma}
\begin{proof}
By straightforward computations,  we have
\begin{align}
&\frac{1}{2\pi}\frac{\partial^2 ReV(p,t+X\sqrt{-1},c;0,n)}{\partial X^2}\\\nonumber
&=-3\frac{\sin(2\pi t)}{e^{2\pi X}+e^{-2\pi X}-2\cos(2\pi t)}\\\nonumber
&+\frac{\sin(2\pi (t+c))}{e^{2\pi X}+e^{-2\pi X}-2\cos(2\pi (t+c))}+\frac{\sin(2\pi (t-c))}{e^{2\pi X}+e^{-2\pi X}-2\cos(2\pi (t-c))}.
\end{align}

Clearly, for $t\in D'_0(c)$, we have $\sin(2\pi t)<0$ and $\cos(2\pi c)\leq 0$ which imply that 
\begin{align}
 -3\frac{\sin(2\pi t)}{e^{2\pi X}+e^{-2\pi X}-2\cos(2\pi t)}>0,   
\end{align}
and 
\begin{align}
    \cos(2\pi c)(e^{2\pi X}+e^{-2\pi X})-2\cos(2\pi t)<2\cos(2\pi c)-2\cos(2\pi t)<0.
\end{align}
Hence
\begin{align}
    &\frac{\sin(2\pi (t+c))}{e^{2\pi X}+e^{-2\pi X}-2\cos(2\pi (t+c))}+\frac{\sin(2\pi (t-c))}{e^{2\pi X}+e^{-2\pi X}-2\cos(2\pi (t-c))}\\\nonumber
    &=2\sin(2\pi t)\frac{\cos(2\pi c)(e^{2\pi X}+e^{-2\pi X})-2\cos(2\pi t)}{(e^{2\pi X}+e^{-2\pi X}-2\cos(2\pi (t+c)))(e^{2\pi X}+e^{-2\pi X}-2\cos(2\pi (t-c)))}>0.
\end{align}
\end{proof}
\begin{lemma} \label{lemma-1-dim-finfty}
    For $t\in D'_0(c)$ and $n\in \mathbb{Z}$, we have 
    \begin{align} \label{formula-f-infty}
        ReV(p,t+X\sqrt{-1},c;0,n)\ \text{goes to $\infty$ uniformly}, \ \text{as} \ X^2\rightarrow \infty. 
    \end{align}
\end{lemma}
\begin{proof}
Based on Lemma \ref{lemma-Li2}, we introduce the following  function for $t\in D'_0(c)$
\begin{align}
    F(X;n)=\left\{ \begin{aligned}
         &X  &  \ (\text{if} \ X\geq 0), \\
         &\left(\frac{1}{2}-t\right)X & \ (\text{if} \ X<0).
                          \end{aligned} \right.
\end{align}
since $t>\frac{1}{2}$, we have 
\begin{align}
    F(X;n)\rightarrow \infty \ \text{as} \ X^2\rightarrow \infty,
\end{align}
and by Lemma \ref{lemma-Li2}, we obtain  
\begin{align}
    2\pi F(X;n)-C < ReV(p,t+X\sqrt{-1},c;0,n)< 2\pi F(X;n)+C,
\end{align}
which implies formula (\ref{formula-f-infty}). 
\end{proof}


Now, we can finish the proof of Proposition \ref{prop-saddleonedim1}.
\begin{proof}

We show that there exists a homotopy $D'_{\delta}(c)$ ($0\leq \delta\leq 1$) between $D'_{0}(c)$ and $D'_{1}(c)$ such that 
\begin{align}
    &(T_1(c),c)\in D'_{1}(c), \label{saddle1-1} \\  
    &D'_{1}(c)-\{(T_1(c),c)\}\subset \{t\in \mathbb{C}|Re V(p,t,c;0,n)<ReV(p,T_1(c),c)\}, \label{saddle1-2}\\
    &\partial D'_{1}(c)\subset \{t\in \mathbb{C}| Re V(p,t,c;0,n)<\zeta_{\mathbb{R}}(p)-\epsilon\}  \label{saddle1-3}.
\end{align}
In the fiber of the projection $\mathbb{C}\rightarrow \mathbb{R}$ at $(t,c)\in D'_0(c)$, we consider the flow from $X=0$ determined by the vector field $-\frac{\partial Re V}{\partial X}$, 
By Lemma \ref{lemma-1-dim-hess} and \ref{lemma-1-dim-finfty}, we obtain that,
    for $t\in D'_0(c)$, then $Re V$ has a unique minimal point, and the flow goes there. We put $g(t,c)$ to be the minimal point. We define the ending of the homotopy to be the set of the destinations of the flow
    \begin{align}
        D'_{1}(c)=\{t+g(t,c)\sqrt{-1}| t\in D'_{0}(c)\}. 
    \end{align}
Further, we define the internal part of the homotopy by setting it along the flows. 

We show $(\ref{saddle1-3})$ as follows, from the definition of $D'_{0}(c)$,
\begin{align}
    \partial D'_{0}(c)\subset \partial D'_0\subset \{t\in \mathbb{C}| Re V(p,t,c;0,n)< \zeta_{\mathbb{R}}(p)-\epsilon\}.   
\end{align}
Further, by the construction of the homotopy, $Re V(p,t,c;0,n)$ monotoincally decreases along the homotopy. Hence $(\ref{saddle1-3})$ holds. 

We show (\ref{saddle1-1}) and (\ref{saddle1-2}) as follows. Consider the following function
\begin{align}
    h(t,c)=Re V(p,t+g(t,c)\sqrt{-1},c;0,n).
\end{align}
It is shown from the definition of $g(t,c)$ that 
\begin{align}
    \frac{\partial Re V(p,t+g(t,c)\sqrt{-1},c;0,n)}{\partial X}=0 \ \text{at} \ X=g(t,c). 
\end{align}
Hence, we have
\begin{align}
    Im \frac{\partial V}{\partial t}=0 \  \text{at} \ t+g(t,c)\sqrt{-1}.  
\end{align}
Furthermore, we also have 
\begin{align}
    \frac{\partial h}{\partial t}=Re \frac{\partial V}{\partial t} \ \text{at} \ t+g(t,c)\sqrt{-1}. 
\end{align}
Therefore, when  $t+g(t,c)\sqrt{-1}$ is a critical point of $Re V$, \ $t$ is a critical point of $h(t,c)$. By Proposition \ref{prop-critical1}, $h(t,c)$ has a unique maximal point at $t=t_1(c)$.  Moreover, by Corollary $\ref{coro-ReV1}$,  the maximal value 
$$
h(t_1(c),c)=ReV(p,T_1(c),c)<ReV(p,T_1(c_{upper}(p),c_{upper}(p))=\zeta_{\mathbb{R}}(p)-\epsilon,
$$
for some small $\epsilon>0$. 
Therefore, (\ref{saddle1-1}) and (\ref{saddle1-2}) hold. The assumption of the saddle point method in one dimension is verified and  we finish the proof of Proposition \ref{prop-saddleonedim1}.  
\end{proof}

\begin{remark}
   By using the same method as illustrated above,  one can prove
\begin{proposition} \label{prop-saddleonedim2}
    For $0<c\leq 1-c_{upper}(p)$ and $n\in\mathbb{Z}$,  there exists a constant $C$ independent of $c$,  such that
    \begin{align}
        |\int_{D'_0(c)} e^{(N+\frac{1}{2})V_N(p,t,c;0,n)}dt|<C\left(e^{(N+\frac{1}{2})\left(\zeta_{\mathbb{R}}(p)-\epsilon\right)}\right). 
    \end{align}
\end{proposition}
Actually,  Proposition \ref{prop-saddleonedim2} can also be derived from the symmetric property of the function $Re V(p,t,c;0,n)$ with respect to the line $c=\frac{1}{2}$.  
\end{remark}

Therefore, combining Proposition \ref{prop-saddleonedim1} and Proposition \ref{prop-saddleonedim2} together, we obtain Proposition \ref{prop-saddleonedim}. 




\subsection{Proof of Proposition \ref{prop-saddleonedim3}} \label{appendix-onesaddle2}
This subsection is devoted to the proof Proposition \ref{prop-saddleonedim3}.

\begin{lemma} \label{lemma-1-dim-01}
  For $c\in (0,1)$, $t\in D'_0(c)$, $n\in \mathbb{Z}$, we have 
    \begin{align} 
        ReV(p,t+X\sqrt{-1},c;-1,n)\rightarrow 0, \ \text{as} \ X\rightarrow +\infty. 
    \end{align}
\end{lemma}
\begin{proof}
Note that 
\begin{align}
    ReV(p,t+X\sqrt{-1},c;-1,n)&=Re\left(\frac{1}{2\pi \sqrt{-1}}\left(\text{Li}_2(e^{-2\pi X}e^{2\pi\sqrt{-1}(t+c)})\right.\right.\\\nonumber
    &\left.\left.+\text{Li}_2(e^{-2\pi X}e^{2\pi\sqrt{-1}(t-c)})-3\text{Li}_2(e^{-2\pi X}e^{2\pi\sqrt{-1}t})\right)\right),
\end{align}
which is independent of $p$ and $n$. 
Then it is easy to see that
$ReV(p,t+X\sqrt{-1},c;-1,n)$ uniformly converges to $0$ as $X\rightarrow +\infty.$ 
\end{proof}

\begin{lemma} \label{lemma-1-dim-hess2}
    For $t\in D'_0(c)$, we have 
    \begin{align}
        \frac{\partial^2 ReV(p,t+X\sqrt{-1},c;-1,n)}{\partial X^2}>0, 
    \end{align}
\end{lemma}
\begin{proof}
    Note that we only consider the second order derivative of $X$ here, so 
    \begin{align}
        \frac{\partial^2 ReV(p,t+X\sqrt{-1},c;-1,n)}{\partial X^2}=\frac{\partial^2 ReV(p,t+X\sqrt{-1},c;0,n)}{\partial X^2}>0
    \end{align}
    by Lemma \ref{lemma-1-dim-hess}. 
\end{proof}
Now, we can finish the proof of Proposition \ref{prop-saddleonedim3}.
\begin{proof}
  We show the existence of a homotopy  $D'_{\delta}(c)$ ($0\leq \delta\leq \delta_0$) between $D'_{0}(c)$ and $D'_{\delta_0}(c)$ such that 
   \begin{align}
      D'_{\delta_0}(c)&\subset \{(t,c)\in D'_{0}(c)| ReV(p,t+X\sqrt{-1},c;-1,n)<\zeta_{\mathbb{R}}(p)-\epsilon\}, \label{formula-1dim-D} \\
         \partial D'_{\delta}(c)&\subset \{(t,c)\in D'_{0}(c)| ReV(p,t+X\sqrt{-1},c;-1,n)<\zeta_{\mathbb{R}}(p)-\epsilon\}.\label{formula-1dim-partial}
   \end{align}
  For each fixed $(t,c)\in D'_{0}(c)$, we move $X$ from 
$0$ along the flow $-\frac{\partial ReV(p,t+X\sqrt{-1},c;-1,n)}{\partial X}$, then by Lemma \ref{lemma-1-dim-01}, the value of   $ReV(p,t+X\sqrt{-1},c;-1,n)$ monotonically decreases and it goes to $0$. As for (\ref{formula-1dim-partial}), since $D'_{0\mathbb{C}}(c)\subset \{(t,c)\in D'_{0\mathbb{C}}(c)| ReV(p,t+X\sqrt{-1},c;-1,n)<\zeta_{\mathbb{R}}(p)-\epsilon\}$ and the value of $ReV$ monotonically decreases, hence (\ref{formula-1dim-partial}) holds. As for (\ref{formula-1dim-D}), since the value of $ReV$ monotonically goes to $0$ by Lemma \ref{lemma-1-dim-01}, (\ref{formula-1dim-D}) holds for sufficiently large $\delta_0$. Therefore, such a required homotopy exists and we prove Proposition \ref{prop-saddleonedim2}.

\end{proof}





\begin{thebibliography}{99}
\bibitem{AndHan06} J. E. Andersen and S. K. Hansen, {\em Asymptotics of the quantum invariants for surgeries on the
figure 8 knot}, J. Knot Theory Ramifications 15 (2006), 479–548.



\bibitem{AGP} F. Aribi, F. Guéritaud, E. Piguet-Nakazawa, {\em Geometric triangulations and the Teichmuller TQFT volume conjecture for twist knots}, arXiv:1903.09480.


\bibitem{BHMV92} C. Blanchet, N. Habegger, G. Masbaum, P. Vogel, {\em Three-manifold invariants derived from Kauffman bracket}, Topology 31 (1992), 685–699.

\bibitem{CLZ15} Q. Chen, K. Liu and S. Zhu, {\em Volume conjecture for SU(n)-invariants}, arXiv:1511.00658.


\bibitem{CJ17} Q. Chen and J. Murakami, {\em Asymptotics of quantum $6j$ symbols}, arxiv: 1706.04887.

\bibitem{CY18} Q. Chen and T. Yang, {\em Volume conjectures for the Reshetikhin-Turaev and the Turaev-Viro invariants}, Quantum Topol. 9 (2018), 419–460.


\bibitem{CZ23-2} Q. Chen and S. Zhu, {\em On the asymptotic expansions of various quantum invariants II: the colored Jones polynomial of twist knots at the root of unity $e^{\frac{2\pi\sqrt{-1}}{N+\frac{1}{M}}}$ and $e^{\frac{2\pi\sqrt{-1}}{N}}$}, in preparation, 2023.

\bibitem{CZ23-3} Q. Chen and S. Zhu, {\em  On the asymptotic expansion of various quantum invariants III:
the Reshetikhin-Turaev invariants of closed hyperbolic 3-manifolds obtained by integral surgery along the twist knot at the root of unity   $e^{\frac{4\pi\sqrt{-1}}{r}}$}, in preparation, 2023.


\bibitem{CZ23-4} Q. Chen and S. Zhu, {\em  On the asymptotic expansion of various quantum invariants IV:
the Turaev-Viro invariants of the twist knot complement in $S^3$  at the root of unity  $e^{\frac{4\pi\sqrt{-1}}{r}}$}, in preparation, 2023.

\bibitem{DGLZ09} T. Dimofte, S. Gukov, J. Lenells and D. Zagier,   {\em Exact results for perturbative Chern–Simons
theory with complex gauge group}, Commun. Number Theory Phys. 3 (2009), 363-443.

\bibitem{DKash07} J.Dubois and R. Kashaev, {\em On the asymptotic expansion of the colored Jones polynomial for torus
knots}, Math. Ann. 339 (2007), 757–782.

\bibitem{DKY18} R. Detcherry, E. Kalfagianni and T. Yang, \emph{Turaev-Viro invariants, colored Jones polynomials and
volume}, Quantum Topol. 9 (2018), no. 4, 775–813.

\bibitem{DK20} R. Detcherry and E. Kalfagianni, \emph{Gromov norm and Turaev-Viro invariants of 3-manifolds},   Ann. Sci.
\'Ec. Norm. Sup\'er. (4) , 53(6):1363–1391, 2020.

%\bibitem{Fedo} M. V. Fedoryuk, {\em Asymptotic Methods in Analysis, in Analysis I: Integral Representations
%nd Asymptotic Methods}, Springer-Verlag, 1989, pages 83–191.

\bibitem{GL11} S. Garoufalidis and T. Le \emph{Asymptotics of the colored Jones function of a knot},  Geom. Topol. 15 (2011), no. 4, 2135-2180.

\bibitem{GL05} S. Garoufalidis and T. T. Q. Le. {\em On the volume conjecture for small angles}. arXiv:math/0502163.

\bibitem{Guk05} S. Gukov. {\em Three-dimensional quantum gravity, Chern–Simons theory, and the A-polynomial}, Comm.
Math. Phys. 255 (2005), 577–627.

\bibitem{GH08} S. Gukov and H. Murakami, {\em $SL(2,\mathbb{C})$ Chern–Simons theory and the asymptotic behavior of the
colored Jones polynomial}, Lett. Math. Phys. 86 (2008), 79-98.

\bibitem{Hab08} K. Habiro, \emph{A unified Witten-Reshetikhin-Turaev invariant
for integral homology spheres}, Invent. Math. 171 (2008), no. 1,
1-81.

\bibitem{Hik07} K. Hikami, {\em Asymptotics of the colored Jones polynomial and the $A$-polynomial}, Nuclear Physics B 773 (2007) 184-202.


\bibitem{Kash95}  R.  Kashaev. {\em A link invariant from quantum
dilogarithm}, Modern Phys. Lett. A 10:19 (1995), 1409—1418.

\bibitem{Kash97} R. Kashaev, {\em The hyperbolic volume of knots from the quantum dilogarithm}, Lett. Math. Phys. 39
(1997), no. 3, 269–275.

\bibitem{KT00} R. M. Kashaev and O. Tirkkonen. {\em A proof of the volume conjecture on torus knots (Russian)}.
Zap. Nauchn. Sem. S.-Peterburg. Otdel. Mat. Inst. Steklov. (POMI) 269 (2000), Vopr. Kvant. Teor.
Polya i Stat. Fiz. 16, 262–268, 370; translation in J. Math. Sci. (N.Y.) 115 (2003), 2033–2036.


\bibitem{Lick} W. Lickorish, {\em The skein method for three-manifold invariants}, J. Knot Theory Ramifications 2
(1993), no. 2, 171–194.

\bibitem{MV94} G. Masbaum and P. Vogel, {\em 3-valent graphs and the Kauffman bracket}. Pacific J. Math, 164(2):361–381, 1994.

\bibitem{Mas03} G. Masbaum, {\em Skein-theoretical derivation of some formulas of
Habiro}. \textit{Algebraic \& Geometric Topology}, \textbf{3}
(2003), 537-55603.

\bibitem{Mey86} R. Meyerhoff. {\em Density of the Chern—Sim
ons invariant for hyperbolic 3-manifolds}. In Low dimensional topology and Kleinian groups (Coventry/Durham, 1984), pp. 217—239, Cambridge Univ.Press, Cambridge, 1986.

\bibitem{MuMu01} H. Murakami and J. Murakami, {\em The colored Jones polynomials and the simplicial volume of a knot},
Acta Math. 186 (2001), no. 1, 85–10.

\bibitem{Mur10} H.Murakami, {\em An introduction to the volume conjecture}. Interactions between hyperbolic geometry,
quantum topology and number theory, Contemp. Math. 541 (Amer. Math. Soc., Providence, RI, (2011), 1–40.

\bibitem{MMOTY02} H. Murakami, J. Murakami, M. Okmoto, T. Takata and Y. Yokota,  {\em Kashaev’s conjecture
and the Chern–Simons invariants of knots and links},  Experiment. Math. 11 (2002), 427–435.

\bibitem{NZ85} W. Neumann and D. Zagier, {\em Volumes of hyperbolic three-manifolds}, Topology 24 (1985), no.3, 307-332.

\bibitem{Oht16}  T. Ohtsuki, {\em On the asymptotic expansion of the Kashaev invariant of the $5_2$ knot}, Quantum Topol.
7 (2016), no. 4, 669–735.

\bibitem{Oht17} T. Ohtsuki, {\em On the asymptotic expansion of the Kashaev invariant of the hyperbolic knots with seven crossings},
Internat. J. Math. 28 (2017), no. 13, 1750096, 143 pp.


\bibitem{Oht18} T. Ohtsuki, {\em On the asymptotic expansion of the quantum SU(2) invariant at $q=exp(\frac{4\pi\sqrt{-1}}{N})$
for closed hyperbolic 3-manifolds obtained by integral surgery along the figure-eight knot}, Algebr. Geom. Topol. 18 (2018), no. 7, 4187–4274.

\bibitem{OhtYok18} T. Ohtsuki and Y. Yokota, {\em On the asymptotic expansion of the Kashaev invariant of the knots with 6 crossings}, Math. Proc. Camb. Phil. Soc. (2018), 165, 287–339.

\bibitem{SS03}  E. Stein and R. Shakarchi, {\em Fourier analysis, An introduction}, Princeton Lectures in Analysis, 1.
Princeton University Press, Princeton, NJ, 2003. xvi+311 pp. ISBN: 0-691-11384-X.

\bibitem{Takata} T. Takata, {\em On the asymptotic expansions of the Kashaev invariant of some hyperbolic knots with 8
crossings}, preprint.

\bibitem{Van08} R. Van Der Veen. {\em Proof of the volume conjecture for Whitehead chains}. Acta Math. Vietnam 33
(2008), 421–431.

\bibitem{Van08-2} R. Van Der Veen, {\em A cabling formula for the coloured Jones polynomial}, arXiv:0807.2679.

\bibitem{Wong19} K. H. Wong, {\em Asymptotics of some quantum invariants of the Whitehead chains}, arXiv: 1912.10638.

\bibitem{WongYang20-1} K. H. Wong and T. Yang, {\em On the Volume Conjecture for hyperbolic Dehn-filled 3-manifolds along
the figure-eight knot}, Preprint, arXiv:2003.10053.

\bibitem{WongYang20-2} K. H. Wong and T. Yang, {\em Relative Reshetikhin-Turaev invariants, hyperbolic cone metrics and
discrete Fourier transforms I}, Preprint, arXiv:2008.05045.




\bibitem{Yoshida85} T. Yoshida, {\em The $\eta$-invariant of hyperbolic 3-manifolds}, Invent. Math. 81 (1985), no. 3, 473–514.

\bibitem{YY10} M. Yamazaki and Y. Yokota, {\em On the limit of the colored Jones polynomial of a non-simple link}.
Tokyo J. Math. 33 (2010), 537–551.

\bibitem{Zheng07} H. Zheng, {\em Proof of the volume conjecture for Whitehead doubles of a family of torus knots}, Chin.
Ann.Math. Ser. B 28 (2007), 375–388.

\end{thebibliography}


\end{document}



