We provide an overview of state-of-the-art classical and quantum sampling techniques.

\subsection{Classical Uniform Random Sampling}
Uniform random sampling is an active field of research with various sampling approaches. 
Available samplers can be categorized into heuristic-~\cite{2018-Dutra-QuickSampler, 2021-Golia-CMSGen}, hashing-~\cite{2015-Chakraborty-UniformSatWitnessGeneration, 2020-Soos-UniGen3} and counting-based~\cite{2018-Sharma-KUS, 2018-Achlioptas-SPUR, 2019-Oh-SMARCH, 2022-Heradio-UniformAndScalableSAT-Sampling} samplers~\cite{2022-Heradio-UniformAndScalableSAT-Sampling}.

QuickSampler~\cite{2018-Dutra-QuickSampler} uses atomic mutations and heuristics for uniform random sampling. % STS
Another heuristic sampler, CMSgen~\cite{2021-Golia-CMSGen}, uses an SAT solver based on conflict-driven cause learning.
UniGen2~\cite{2015-Chakraborty-UniformSatWitnessGeneration} is a random hashing-based sampler that supports parallelization and it's runtime was improved in UniGen3~\cite{2020-Soos-UniGen3}.
KUS~\cite{2018-Sharma-KUS}, SPUR~\cite{2018-Achlioptas-SPUR}, SMARCH~\cite{2019-Oh-SMARCH} and BDDSampler~\cite{2022-Heradio-UniformAndScalableSAT-Sampling} are counting-based samplers.
Knuth's algorithm~\cite{2009-Knuth-Book} using Binary Decision Diagrams (BDDs) was applied to SPLs by Oh et al. in~\cite{2017-Oh-FindingNearOptimalConfigs}.
KUS uses the Deterministic Decomposable Negation Normal Form (d-DNNF) - a superset of ordered BDDs~\cite{2018-Sharma-KUS}.
SPUR and SMARCH rely on Thurley's \#-SAT solver~\cite{2006-Thurley-sharpSAT} which counts the number of satisfying assignments of a propositional formula (which are valid configurations in our case)~\cite{2018-Achlioptas-SPUR, 2019-Oh-SMARCH}.
BDDSampler uses ordered and reduced BDDs~\cite{2022-Heradio-UniformAndScalableSAT-Sampling}.

Comparisons of uniform random samplers showed a trade-off between scalability and uniformity~\cite{2018-Achlioptas-SPUR, 2018-Dutra-QuickSampler, 2018-Sharma-KUS, 2019-Chakraborty-Barbarik, 2022-Soos-QuantitativeTestingOfSamplers, 2019-Plazar-UniformSamplingAreWeThereYet, 2020-Heradio-UniformAndScalableSAT-Sampling, 2022-Heradio-UniformAndScalableSAT-Sampling}.
To the best of our knowledge, the time complexity for uniform random samplers is not fully understood yet and is an open question.
The results of Heradio et al.~\cite{2022-Heradio-UniformAndScalableSAT-Sampling} hint that additional aspects to the number of variables of the CNF influence sampling time, such as the number of clauses and complexity, and indicate an overall exponential growth.

\subsection{Quantum Sampling}

To the best of our knowledge, no other work discusses uniform random sampling using quantum algorithms in the way this paper does.
Hangleiter and Eisert~\cite{2022-Hangleiter} elaborate different random sampling schemes for quantum circuits and the computational advantage of using quantum computers for this.
They define sampling from a random quantum computation as quantum random sampling.
We are not interested in random quantum computations, but in (uniform) random samples from a constrained distribution.
These constraints can not be enforced by mere randomness, so our work does not relate to theirs at all.

