A software product line models the variability of a highly configurable system~\cite{clements2002software}.
It is common for one product line to have thousands of configurable features, which leads to a configuration space  consisting of potentially multiple millions of unique product configurations~\cite{sundermann2020evaluating}.
When new optional features are added to a product line, the size of the configuration space grows exponentially. 
As a result, quality assurance is challenging because exploring all valid configurations is not feasible. 
To overcome the challenge of testing large configuration spaces, practical approaches use algorithms to create samples with few representative configurations. 

\textit{Uniform random sampling} is a technique for sampling random solutions from propositional formulas~\cite{2019-Plazar-UniformSamplingAreWeThereYet, 2017-Oh-FindingNearOptimalConfigs, 2019-Oh-SMARCH}.
Uniform random sampling can be applied to a product line by translating the configuration space to a propositional formula in conjunctive normal form (CNF). 
A CNF consists of variables and clauses, where variables represent the configurable features of a product line, and clauses represent rules stating how features can be combined to create valid product configurations. 
Thus, every solution of the CNF represents one valid product configuration.
A uniform random sample is created by repeatably receiving a random solution of the CNF and thus creating a set of valid product configurations. 
%....... 
To be uniform, a sampling method must have the property that it selects all solutions with the same probability. 
%This property ensures that samples are not impacted by biases that are introduced through rare features in the configuration space. 
However, classical uniform random sampling faces two main challenges: 
1$)$ classical computers can only implement pseudo-randomness so that every created sample will have at least a small statistical bias~\cite{vadhan2012pseudorandomness} and 
2$)$ classical algorithms encounter scalability issues as they have to explore exponentially growing configuration spaces to create truly uniform samples~\cite{2019-Plazar-UniformSamplingAreWeThereYet}. 

Quantum computers can solve such issues because they are based on quantum mechanics, for which it is assumed that truly random processes can be executed through superposition and measurement~\cite{deshpande2020implications}.
Furthermore, quantum computers are theoretically proven to provide up to super-polynomial speedups for specific algorithmic problems~\cite{nielsen2002quantum}.

%OUR CONTRIBUTION
In this paper, we propose a method for quantum-based uniform random sampling of configurations from a software product line using Grover's algorithm~\cite{1996-Grover-Search}. 
Our approach to uniform random sampling on a quantum computer ensures that all solutions are measured with the same probability, making our samples uniform by design.
Additionally, the characteristics of quantum mechanics guarantee that no statistical bias is introduced through pseudo-randomness.
We evaluate this quantum-enabled method for uniform random sampling by showing its uniformity over multiple samples and investigating how it scales for different product lines and configuration spaces. 
We further discuss best and worst-case scenarios for the scalability of the approach. 
In summary, this paper makes the following contributions:
\begin{enumerate}
    \item We provide a method to encode a valid configuration space on a quantum computer to retrieve a random sample.
    \item We provide a detailed explanation and implementation of the quantum algorithm realizing our method for uniform random configuration sampling.
    \item We analyze the uniformity of the generated samples and discuss how current quantum hardware impacts results. 
    \item We evaluate the scalability of our method in terms of the resulting quantum circuit's size. %and depth scale are polynomial 
    for various product lines.
\end{enumerate}

%READER GUIDE:
The content of this paper is structured as follows:
Section~\ref{sec:background} provides background information regarding software product lines, quantum computing, and Grover's algorithm.
Section~\ref{sec:sota} introduces state-of-the-art techniques for uniform random sampling on classical and quantum computers.
Section~\ref{sec:contribution} describes the contribution of this paper: a new method for uniform random sampling that is based on quantum computing using Grover's algorithm.
Section~\ref{sec:eval} evaluates the method for uniform random sampling on multiple example product lines and discusses our insights and results. 
Section~\ref{sec:conclusion} concludes this paper and states possible future work. 