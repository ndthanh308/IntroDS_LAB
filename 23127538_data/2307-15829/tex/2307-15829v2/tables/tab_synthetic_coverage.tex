\begin{table}[!t]
    \centering
    \begin{adjustbox}{max width=\linewidth}
    \setlength{\tabcolsep}{4pt}
    {\small
    \begin{tabular}{lccccccc}
        \toprule
        Method & Input  & \multicolumn{6}{c}{PSNR $\uparrow$} \\
        Coverage &   & 10\% &  20\% &  30\%&  40\%&  50\%&  60\%\\
        \midrule
        MAT \cite{li_mat22cvpr} & I & 35.4937 & 31.4344 & 31.895 & 29.8665 & 27.7057 & 28.1765\\
        MISF \cite{li_misf22cvpr}& I & 35.6868 & 31.8720 & 31.7689 & 30.5100 &  28.3776 & 28.9148 \\
        PUT \cite{liu22cvpr} & I & 32.7979 & 28.6378 & 27.0988 & 26.1156 & 23.7555 & 23.5093 \\
        ZITS \cite{dong22cvpr}& I & 35.6619 &  31.9159 & 32.3340 &  30.4923 & 28.3828 & 28.9959 \\
        EF-SAI \cite{liao22cvpr}& I+E & 29.2994 & 27.9049 & 27.7354 & 26.2447 & 24.7267 & 24.6231\\
        E2VID \cite{Rebecq19cvpr}& E & 22.0048 & 19.8868 & 20.2063 & 18.1452 & 17.5353 & 17.5884 \\
        Ours (Acc. Method)& E & 28.0390 & 22.4906 & 20.6897 & 17.9743 & 17.0200 & 15.8527\\
        Ours (Learning) & I+E & \textbf{40.1286} &\textbf{ 36.3622} & \textbf{35.8476} & \textbf{33.2527} & \textbf{31.0083} & \textbf{31.1224}\\
        \bottomrule
    \end{tabular}}
    \end{adjustbox}
    \caption{Reconstruction performance  on our synthetic dataset in terms of PSNR divided according to the occlusion density of the test samples.}
    \label{tab:syn_coverage}
\end{table}
