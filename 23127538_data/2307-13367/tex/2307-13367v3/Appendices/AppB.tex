\section{Bayes factor and biases}\label{AppB}
Another way to quantify whether a certain eccentricity is well recovered in our analysis is by computing the Bayes factor. For this purpose, we need to compare two hypotheses  which are trying to explain the same eccentric signal. The Null hypothesis is that a circular template (here \textsc{TaylorF2}) is enough to accurately describe this signal. The eccentric hypothesis states that you need to have the eccentricity parameter in your template (here \textsc{TaylorF2Ecc}) to properly explain this signal. We then need to take the ratio of their evidence to compute the Bayes factor
\begin{equation}
    \mathcal{B}=\frac{Z_{\rm ecc}}{Z_{\rm no~ecc}}.
\end{equation}
If $\ln\mathcal{B}>8$ \citep{Lower2018,Thrane2019}, then we have a strong evidence that the given signal comes from an eccentric system rather than a circular one. 

To do this we inject eccentric signals using \textsc{TaylorF2Ecc} and recover them with \textsc{TaylorF2} to compute $Z_{\rm no~ecc}$ and \textsc{TaylorF2Ecc} to compute $Z_{\rm ecc}$ by sampling only intrinsic parameters. Since we are in a zero noise limit and high SNR limit with only eccentricity parameter different between two models, we can compute the Savage-Dickey ratio \citep{Dickey1971} to approximately get $\mathcal{B}$. We only need to use the Fisher matrix $\Gamma_{ij}$ from \textsc{TaylorF2Ecc} for a given injected eccentricity $e_{\rm inj}$:
\begin{equation}
    \mathcal{B}\approx\pi(e)\sqrt{\frac{2\pi \bar\Gamma}{\Gamma}}\exp{\left(\frac{1}{2}\frac{e^2_{\rm inj}}{\Gamma^{-1}_{ee}}\right)},
\end{equation}
where $\Gamma$ and $\bar\Gamma$ are determinants of Fisher matrices of all parameters and parameters except eccentricity, respectively, and $\Gamma^{-1}_{ee}$ is the value of the covariance on eccentricity. Here $\pi(e)=1/(0.2-10^{-6})$ due to an uniform prior.

In Fig.~\ref{fig:eccmin_lnBayes}, we show minimum measurable eccentricities if $\ln\mathcal{B}>8$ for a given $M_z$ and $q$. These results are almost consistent with Fig.~\ref{fig:min_ecc_MCMC}, although 
results slightly worsen due to a stricter criterion.

% Figure environment removed

We can also compute the bias induced in the estimation of $M_z$ and $q$ when fitting circular template to an eccentric signal. For this purpose, we compute the bias for a given parameter $\theta$, normalized by its standard deviation as
\begin{equation}
    \delta\theta[\sigma]=\frac{|\hat{\theta}_{\rm no~ecc}-\hat{\theta}_{\rm ecc}|}{\sigma^{\theta}_{\rm ecc}},
\end{equation}
where $\hat{\theta}$ denotes the highest likelihood point in the posterior distribution for the given model, and $\sigma^{\theta}_{\rm ecc}$ is the standard deviation of the eccentric model posterior of $\theta$. 

In Fig.~\ref{fig:Biases}, we show $\delta\theta[\sigma]$ for $M_z$ and $q$ as a function of varying eccentricity for a fixed $M_z=10^5~\MSun$ and $q=8$. Both biases are almost identical and as expected, grow rapidly as $e_{1\rm yr}$ becomes higher. For $e_{1\rm yr}=10^{-3.5}$, the bias in both parameters is $\approx0.4$ and for $e_{1\rm yr}=0.1$, $\delta\theta[\sigma]=120$. These results emphasize the need to included eccentricity during parameter estimation.

% Figure environment removed