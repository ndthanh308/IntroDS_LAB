%% Beginning of file 'sample631.tex'
%%
%% Modified 2022 May  
%%
%% This is a sample manuscript marked up using the
%% AASTeX v6.31 LaTeX 2e macros.
%%
%% AASTeX is now based on Alexey Vikhlinin's emulateapj.cls 
%% (Copyright 2000-2015).  See the classfile for details.

%% AASTeX requires revtex4-1.cls and other external packages such as
%% latexsym, graphicx, amssymb, longtable, and epsf.  Note that as of 
%% Oct 2020, APS now uses revtex4.2e for its journals but remember that 
%% AASTeX v6+ still uses v4.1. All of these external packages should 
%% already be present in the modern TeX distributions but not always.
%% For example, revtex4.1 seems to be missing in the linux version of
%% TexLive 2020. One should be able to get all packages from www.ctan.org.
%% In particular, revtex v4.1 can be found at 
%% https://www.ctan.org/pkg/revtex4-1.

%% The first piece of markup in an AASTeX v6.x document is the \documentclass
%% command. LaTeX will ignore any data that comes before this command. The 
%% documentclass can take an optional argument to modify the output style.
%% The command below calls the preprint style which will produce a tightly 
%% typeset, one-column, single-spaced document.  It is the default and thus
%% does not need to be explicitly stated.
%%
%% using aastex version 6.3
\documentclass[preprint]{aastex631}

%% The default is a single spaced, 10 point font, single spaced article.
%% There are 5 other style options available via an optional argument. They
%% can be invoked like this:
%%
%% \documentclass[arguments]{aastex631}
%% 
%% where the layout options are:
%%
%%  twocolumn   : two text columns, 10 point font, single spaced article.
%%                This is the most compact and represent the final published
%%                derived PDF copy of the accepted manuscript from the publisher
%%  manuscript  : one text column, 12 point font, double spaced article.
%%  preprint    : one text column, 12 point font, single spaced article.  
%%  preprint2   : two text columns, 12 point font, single spaced article.
%%  modern      : a stylish, single text column, 12 point font, article with
%% 		  wider left and right margins. This uses the Daniel
%% 		  Foreman-Mackey and David Hogg design.
%%  RNAAS       : Supresses an abstract. Originally for RNAAS manuscripts 
%%                but now that abstracts are required this is obsolete for
%%                AAS Journals. Authors might need it for other reasons. DO NOT
%%                use \begin{abstract} and \end{abstract} with this style.
%%
%% Note that you can submit to the AAS Journals in any of these 6 styles.
%%
%% There are other optional arguments one can invoke to allow other stylistic
%% actions. The available options are:
%%
%%   astrosymb    : Loads Astrosymb font and define \astrocommands. 
%%   tighten      : Makes baselineskip slightly smaller, only works with 
%%                  the twocolumn substyle.
%%   times        : uses times font instead of the default
%%   linenumbers  : turn on lineno package.
%%   trackchanges : required to see the revision mark up and print its output
%%   longauthor   : Do not use the more compressed footnote style (default) for 
%%                  the author/collaboration/affiliations. Instead print all
%%                  affiliation information after each name. Creates a much 
%%         4astro.ph.10551V         longer author list but may be desirable for short 
%%                  author papers.
%% twocolappendix : make 2 column appendix.
%%   anonymous    : Do not show the authors, affiliations and acknowledgments 
%%                  for dual anonymous review.
%%
%% these can be used in any combination, e.g.
%%
\usepackage{chngcntr}
\usepackage{multirow}
\usepackage{booktabs}
%% \documentclass[twocolumn,linenumbers,trackchanges]{aastex631}
%%
%% AASTeX v6.* now includes \hyperref support. While we have built in specific
%% defaults into the classfile you can manually override them with the
%% \hypersetup command. For example,
%%
%% \hypersetup{linkcolor=red,citecolor=green,filecolor=cyan,urlcolor=magenta}
%%
%% will change the color of the internal links to red, the links to the
%% bibliography to green, the file links to cyan, and the external links to
%% magenta. Additional information on \hyperref options can be found here:
%% https://www.tug.org/applications/hyperref/manual.html#x1-40003
%%
%% Note that in v6.3 "bookmarks" has been changed to "true" in hyperref
%% to improve the accessibility of the compiled pdf file.
%%
%% If you want to create your own macros, you can do so
%% using \newcommand. Your macros should appear before
%% the \begin{document} command.
%%
\newcommand{\vdag}{(v)^\dagger}
\newcommand\aastex{AAS\TeX}
\newcommand\latex{La\TeX}

\newcommand{\rough}[1]{{\color{red} \textit{#1}}}
\newcommand{\outline}[1]{{\color{magenta} #1}}
\newcommand{\sudip}[1]{{\color{red} #1}}
\newcommand{\yash}[1]{{\color{red} \textbf{#1}}}
\newcommand{\jeroen}[1]{{\color{brown} #1}}

\newcommand{\astr}{\textit{AstroSat}}
\newcommand{\nic}{\textit{NICER}}
\newcommand{\rxte}{\textit{RXTE}}

\newcommand{\src}{GX~340+0}
\newcommand{\sz}{$S_{\rm z}$}
%% Reintroduced the \received and \accepted commands from AASTeX v5.2
%\received{March 1, 2021}
%\revised{April 1, 2021}
%\accepted{\today}

%% Command to document which AAS Journal the manuscript was submitted to.
%% Adds "Submitted to " the argument.
%\submitjournal{PSJ}

%% For manuscript that include authors in collaborations, AASTeX v6.31
%% builds on the \collaboration command to allow greater freedom to 
%% keep the traditional author+affiliation information but only show
%% subsets. The \collaboration command now must appear AFTER the group
%% of authors in the collaboration and it takes TWO arguments. The last
%% is still the collaboration identifier. The text given in this
%% argument is what will be shown in the manuscript. The first argument
%% is the number of author above the \collaboration command to show with
%% the collaboration text. If there are authors that are not part of any
%% collaboration the \nocollaboration command is used. This command takes
%% one argument which is also the number of authors above to show. A
%% dashed line is shown to indicate no collaboration. This example manuscript
%% shows how these commands work to display specific set of authors 
%% on the front page.
%%
%% For manuscript without any need to use \collaboration the 
%% \AuthorCollaborationLimit command from v6.2 can still be used to 
%% show a subset of authors.
%
%\AuthorCollaborationLimit=2
%
%% will only show Schwarz & Muench on the front page of the manuscript
%% (assuming the \collaboration and \nocollaboration commands are
%% commented out).
%%
%% Note that all of the author will be shown in the published article.
%% This feature is meant to be used prior to acceptance to make the
%% front end of a long author article more manageable. Please do not use
%% this functionality for manuscripts with less than 20 authors. Conversely,
%% please do use this when the number of authors exceeds 40.
%%
%% Use \allauthors at the manuscript end to show the full author list.
%% This command should only be used with \AuthorCollaborationLimit is used.

%% The following command can be used to set the latex table counters.  It
%% is needed in this document because it uses a mix of latex tabular and
%% AASTeX deluxetables.  In general it should not be needed.
%\setcounter{table}{1}

%%%%%%%%%%%%%%%%%%%%%%%%%%%%%%%%%%%%%%%%%%%%%%%%%%%%%%%%%%%%%%%%%%%%%%%%%%%%%%%%
%%
%% The following section outlines numerous optional output that
%% can be displayed in the front matter or as running meta-data.
%%
%% If you wish, you may supply running head information, although
%% this information may be modified by the editorial offices.
%\shorttitle{AASTeX v6.3.1 Sample article}
%\shortauthors{Schwarz et al.}
%%
%% You can add a light gray and diagonal water-mark to the first page 
%% with this command:
%% \watermark{text}
%% where "text", e.g. DRAFT, is the text to appear.  If the text is 
%% long you can control the water-mark size with:
%% \setwatermarkfontsize{dimension}
%% where dimension is any recognized LaTeX dimension, e.g. pt, in, etc.
%%
%%%%%%%%%%%%%%%%%%%%%%%%%%%%%%%%%%%%%%%%%%%%%%%%%%%%%%%%%%%%%%%%%%%%%%%%%%%%%%%%
%\graphicspath{{./}{figures/}}
%% This is the end of the preamble.  Indicate the beginning of the
%% manuscript itself with \begin{document}.

\begin{document}

% \title{Explaining the evolution of the neutron star low-mass X-ray binary GX 340+0}
\title{\emph{AstroSat} view of the neutron star low-mass X-ray binary GX 340+0}

%% LaTeX will automatically break titles if they run longer than
%% one line. However, you may use \\ to force a line break if
%% you desire. In v6.31 you can include a footnote in the title.

%% A significant change from earlier AASTEX versions is in the structure for 
%% calling author and affiliations. The change was necessary to implement 
%% auto-indexing of affiliations which prior was a manual process that could 
%% easily be tedious in large author manuscripts.
%%
%% The \author command is the same as before except it now takes an optional
%% argument which is the 16 digit ORCID. The syntax is:
%% \author[xxxx-xxxx-xxxx-xxxx]{Author Name}
%%
%% This will hyperlink the author name to the author's ORCID page. Note that
%% during compilation, LaTeX will do some limited checking of the format of
%% the ID to make sure it is valid. If the "orcid-ID.png" image file is 
%% present or in the LaTeX pathway, the OrcID icon will appear next to
%% the authors name.
%%
%% Use \affiliation for affiliation information. The old \affil is now aliased
%% to \affiliation. AASTeX v6.31 will automatically index these in the header.
%% When a duplicate is found its index will be the same as its previous entry.
%%
%% Note that \altaffilmark and \altaffiltext have been removed and thus 
%% can not be used to document secondary affiliations. If they are used latex
%% will issue a specific error message and quit. Please use multiple 
%% \affiliation calls for to document more than one affiliation.
%%
%% The new \altaffiliation can be used to indicate some secondary information
%% such as fellowships. This command produces a non-numeric footnote that is
%% set away from the numeric \affiliation footnotes.  NOTE that if an
%% \altaffiliation command is used it must come BEFORE the \affiliation call,
%% right after the \author command, in order to place the footnotes in
%% the proper location.
%%
%% Use \email to set provide email addresses. Each \email will appear on its
%% own line so you can put multiple email address in one \email call. A new
%% \correspondingauthor command is available in V6.31 to identify the
%% corresponding author of the manuscript. It is the author's responsibility
%% to make sure this name is also in the author list.
%%
%% While authors can be grouped inside the same \author and \affiliation
%% commands it is better to have a single author for each. This allows for
%% one to exploit all the new benefits and should make book-keeping easier.
%%
%% If done correctly the peer review system will be able to
%% automatically put the author and affiliation information from the manuscript
%% and save the corresponding author the trouble of entering it by hand.

%\correspondingauthor{August Muench}
%\email{greg.schwarz@aas.org, gus.muench@aas.org}

\author[0000-0002-5967-8399]{Yash Bhargava}
\affiliation{Department of Astronomy and Astrophysics, Tata Institute of Fundamental Research, 1 Homi Bhabha Road, Colaba, Mumbai 400005, India}
\correspondingauthor{Yash Bhargava}
\email{yash.bhargava\_003@tifr.res.in}

\author[0000-0002-6351-5808]{Sudip Bhattacharyya}
\affiliation{Department of Astronomy and Astrophysics, Tata Institute of Fundamental Research, 1 Homi Bhabha Road, Colaba, Mumbai 400005, India}

\author[0000-0001-8371-2713]{Jeroen Homan}
\affiliation{Eureka Scientific, Inc., 2452 Delmer Street, Oakland, CA 94602, USA}

\author[0000-0002-5900-9785]{Mayukh Pahari}
\affiliation{Department of Physics, Indian Institute of Technology Hyderabad, IITH main road, Kandi 502284}

%% Note that the \and command from previous versions of AASTeX is now
%% depreciated in this version as it is no longer necessary. AASTeX 
%% automatically takes care of all commas and "and"s between authors names.

%% AASTeX 6.31 has the new \collaboration and \nocollaboration commands to
%% provide the collaboration status of a group of authors. These commands 
%% can be used either before or after the list of corresponding authors. The
%% argument for \collaboration is the collaboration identifier. Authors are
%% encouraged to surround collaboration identifiers with ()s. The 
%% \nocollaboration command takes no argument and exists to indicate that
%% the nearby authors are not part of surrounding collaborations.

%% Mark off the abstract in the ``abstract'' environment. 
\begin{abstract}

Understanding the spectral evolution along the `Z'-shaped track in the hardness-intensity diagram of Z-sources, which are a class of luminous neutron star low-mass X-ray binaries, is crucial to probe accretion processes close to the neutron star.
Here, we study the horizontal branch (HB) and the normal branch (NB) of the Z source \src\ using \astr\ data.
We find that the HB and the NB appear as two different types of X-ray intensity dips, which can appear in any sequence and with various depths.
Our $0.8-25$~keV spectra of dips and the hard apex can be modeled by the emissions from an accretion disk, a Comptonizing corona covering the inner disk, and the neutron star surface.
We find, as the source moves onto the HB the corona is replenished and energized by the disk and a reduced amount of disk matter reaches the neutron star surface.
We also conclude that quasi-periodic oscillations during HB/NB are strongly associated with the corona, and explain the evolution of strength and hard-lag of this timing feature using the estimated coronal optical depth evolution.

    
\end{abstract}

%% Keywords should appear after the \end{abstract} command. 
%% The AAS Journals now uses Unified Astronomy Thesaurus concepts:
%% https://astrothesaurus.org
%% You will be asked to selected these concepts during the submission process
%% but this old "keyword" functionality is maintained in case authors want
%% to include these concepts in their preprints.
\keywords{X-rays: binaries --- stars: individual (\src) --- accretion, accretion discs}

%% From the front matter, we move on to the body of the paper.
%% Sections are demarcated by \section and \subsection, respectively.
%% Observe the use of the LaTeX \label
%% command after the \subsection to give a symbolic KEY to the
%% subsection for cross-referencing in a \ref command.
%% You can use LaTeX's \ref and \label commands to keep track of
%% cross-references to sections, equations, tables, and figures.
%% That way, if you change the order of any elements, LaTeX will
%% automatically renumber them.
%%
%% We recommend that authors also use the natbib \citep
%% and \citet commands to identify citations.  The citations are
%% tied to the reference list via symbolic KEYs. The KEY corresponds
%% to the KEY in the \bibitem in the reference list below. 

\section{Introduction}\label{sec:intro}


% \outline{About Z-track sources}
%https://ui.adsabs.harvard.edu/abs/2014MNRAS.438.2784C/abstract Use this for ref

A neutron star (NS) low-mass X-ray binary (LMXB), viz., an NS accreting matter from a low-mass donor star, is a natural laboratory to study accretion processes in extreme conditions. These binaries can be classified into `Z' sources and `atoll' sources based on the evolution of spectral and temporal properties, as well as luminosity \citep{hasinger1989A&A...225...79H}. Hardness-intensity diagrams (HIDs) and color-color diagrams (CCDs) of these sources provide a simple model-independent way to probe the spectral evolution. Z-sources, which emit close to the Eddington luminosity,  show {\it Z}-shaped tracks in HIDs and CCDs. Such tracks can drift secularly over long duration \citep{hasinger1989A&A...225...79H,vdkReview2004astro.ph.10551V}. 
Z-sources can be further divided into two subclasses `Cyg-like' sources (Cyg~X-2, \src, GX~5-1) or `Sco-like' sources (Sco~X-1, GX~17+2, and GX~349+2) depending on the shape traced on the HID \citep{kuulkers1994A&A...289..795K,Homan2010ApJ...719..201H}. \emph{RXTE} observations of the transient sources XTE~J1701$-$462 and IGR J17480$-$2446, evolved through both Z and  atoll phases, have shown that the atoll and Z source (sub-)classes are probably strongly related to the mass accretion rate \citep{homan2007ApJ...656..420H, lin2009ApJ...696.1257L,Homan2010ApJ...719..201H,Chakrabortyetal2011}. 

Z-tracks, usually studied in $\sim 2-20$~keV band, can be subdivided into three branches; horizontal branch (HB), normal branch (NB), and flaring branch (FB) \citep{hasinger1989A&A...225...79H, vdkReview2004astro.ph.10551V}. Since the source typically moves along a single track, this motion can be parameterized by a single parameter \sz, which increases from the HB, via the NB, to the FB (with the HB/NB vertex (hard apex) set to \sz=1 and the NB/FB vertex (soft apex) set to \sz=2). However, the nature of the changes along the Z-track and what drives them is still not understood. \citet{hasinger1990} suggested that the evolution is driven by changes in the mass accretion rate, increasing from the HB to the FB, while \citet{Church2006A&A...460..233C} argued that the mass accretion rate increased from the NB to HB, with the FB being the result of changes in the thermonuclear burning rate. \citet{homan2002} suggested that the mass accretion rate may not change much at all along the Z-track and \citet{lin2009ApJ...696.1257L} proposed that motion along the Z-track could arise due to instabilities in the accretion disk.


Z-tracks suggest a distinctive and repetitive spectral evolution of Z-sources. Therefore, an explanation of such tracks should be a key to understand the accretion processes of this class of bright NS LMXBs. The spectral evolution of Z-sources has often been studied in terms of transitions from the FB to the NB and HB (and vice versa) with the main contribution to the 2$-$20~keV spectrum arising from a thermal and a non-thermal components \citep[e.g.,][]{mitsuda1989PASJ...41...97M, Church2006A&A...460..233C, church2012A&A...546A..35C}. According to  \citet{mitsuda1989PASJ...41...97M}, the soft thermal component arises due to a multicolor (disk) blackbody, while \citet{Church2006A&A...460..233C,Jackson2009A&A...494.1059J,balucinska2010A&A...512A...9B} preferred it to be a blackbody-like component, possibly from the boundary layer (BL) on the NS surface. 
The non-thermal component was assumed to arise from the Comptonized emission from a hot electron cloud, often referred to as a corona.  But at softer energies (0.5$-$2~keV), another component was necessary to describe the spectrum \citep{lavagetto2004NuPhS.132..616L,Seifina2013ApJ...766...63S}. The spectrum of Z-sources also have been interpreted as a combination of a thermal accretion disk, blackbody component and the Comptonized emission in which the both disk and blackbody are relatively hotter ($\sim$1.5 and $\sim$2.4 keV respectively) and the Comptonized emission is weaker with a break at $\sim$20~keV
\citep{lin2007ApJ...667.1073L, homan2007ApJ...656..420H, lin2009ApJ...696.1257L,Homan2010ApJ...719..201H, lin2012ApJ...756...34L}. 

Along the Z-track, HB shows kilo-Hertz quasi-periodic oscillations (kHz QPOs), HB oscillations (HBOs) in $\sim 25-50$ Hz and a low frequency broadband noise, and NB shows NB oscillations (NBOs) in $\sim 5-7$ Hz \citep{vanParadijs1988MNRAS.231..379V,Penninx1991MNRAS.249..113P,Jonker1998ApJ...499L.191J,Jonker2000ApJ...537..374J,sriram2011ApJ...743L..31S}. While various models have been proposed to explain these oscillations,
their origins are still unclear \citep[e.g.,][]{vdkReview2004astro.ph.10551V}.


\src\ is one of the brightest and well-studied Z-sources in the Galactic plane \citep{mitsuda1989PASJ...41...97M,Jonker1998ApJ...499L.191J, Jonker2000ApJ...537..374J, Gilfanov2003A&A...410..217G, lavagetto2004NuPhS.132..616L,Iaria2006ChJAS...6a.257I, balucinska2010A&A...512A...9B,Seifina2013ApJ...766...63S}. It shows a clear and repetitive Z-track with strong detections of HBOs and NBOs \citep{Jonker1998ApJ...499L.191J,Gilfanov2003A&A...410..217G,balucinska2010A&A...512A...9B}. The source can trace out a full Z-track, i.e. covering all the branches, on timescales as short as a few days \citep{Jonker2000ApJ...537..374J}. The source is highly absorbed with the neutral Hydrogen column density reported in the range $(6-12)\times$10$^{22}$~cm$^{-2}$ \citep{Iaria2006ChJAS...6a.257I,Church2006A&A...460..233C}.\citet{mitsuda1989PASJ...41...97M} described soft thermal component as an accretion disk while \citet{Church2006A&A...460..233C} described it as a single temperature blackbody. Both these prescriptions also required a non-thermal component which was modeled as a Comptonized emission. However, \citet{Seifina2013ApJ...766...63S} described the broadband spectrum from \emph{BeppoSAX} as a combination of a low temperature blackbody (from the accretion disk) and two Comptonization components (the softer one from the transition layer between the disk and the NS and the harder one from the NS surface). 
\citet{Church2006A&A...460..233C} explained the HB in terms of the reduction of blackbody area and Comptonized emission and an increase of the blackbody temperature. However, \citet{Seifina2013ApJ...766...63S} attributed the HB evolution to the change of the electron temperature of a Comptonizing transition layer.  

Thus, even decades after the first studies of Z-tracks, the cause of these distinctive tracks is not reliably known.
In this article, we attempt to explain the HB and the NB of \src\ by analyzing the \astr\ light curve and broadband spectra.




\section{Observation and Data Reduction}\label{sec:obs}

\astr\ \citep{Singh2014SPIE.9144E..1SS} observed the \src\ from September 19, 2020 to September 22, 2020 (observation ID A09\_134T01\_9000003896), corresponding to a total LAXPC exposure of 92.8~ks and a total SXT exposure of 29.6~ks. 
The observation covered HB and NB phases of the Z-track as seen in right panel of figure~\ref{fig:laxpc_hid}.
The \astr\ observation was conducted simultaneously with Large Area X-ray Proportional Counter (LAXPC) in Event Analysis mode \citep{Yadav2016SPIE.9905E..1DY,Yadav2017CSci..113..591Y} and Soft X-ray Telescope (SXT) in Photon Counting mode \citep{singh2016SPIE.9905E..1ES, Singh2017JApA...38...29S}.
LAXPC has three independent proportional counters but one of them suffered a gas leak early in the mission (LXP 30) and another has shown abnormal gain variations since early 2018 \citep[LXP 10;][]{Antia2021JApA...42...32A}.  Thus we use only LXP 20 for the analysis.
We reduce the LAXPC data using the pipeline \textsc{LAXPCsoftware22Aug15}\footnote{\url{http://astrosat-ssc.iucaa.in/uploads/laxpc/LAXPCsoftware22Aug15.zip}} \citep{Antia2021JApA...42...32A,Misra2021JApA...42...55M}. 
The software includes tools to reduce the level 1 data to level 2 data, calibration files, responses, etc.
We obtain the level 2 data and standard good time intervals (GTIs; which exclude the duration of South Atlantic Anomaly passage, etc.) using the tools from the above-mentioned pipeline. 
We also extract the energy-dependent light curves, spectra, background spectra and background light curves using the tools available in \textsc{LAXPCsoftware22Aug15}. We consider the data from layer 1 of the instrument due to its minimal background contribution. 

The processed SXT level 2 files were downloaded from Astrobrowse\footnote{\url{https://astrobrowse.issdc.gov.in/astro_archive/archive/Home.jsp}}. We extract the standard products (energy dependent lightcurves and spectra) using HEASARC (version 6.30.1) tool \texttt{xselect}. The source count rate is too low to have significant pile-up effects and thus a circular region of 15$\arcmin$ is used to extract the data products. We use the standard background and response files provided by the SXT payload operation centre (POC)\footnote{https://www.tifr.res.in/$\sim$astrosat\_sxt/index.html}. We modify the  standard ancillary response file to correct for the smaller source region using the tool provided by the POC. 


\section{Spectral and timing analysis}

\subsection{Demarcation of the HID zones}\label{ssec:demarc}

We investigate the evolution of the source by extracting energy dependent light curves (with 100~s bins) and mapping the count rate evolution to positions on the HID (see figure~\ref{fig:laxpc_hid} for description). We divide the HID into five different zones ensuring roughly similar length is covered by each zone on the track (see figure~\ref{fig:laxpc_hid}). Zone 1 and 2 correspond to the HB, Zone 3 and 4 correspond to the hard apex and Zone 5 corresponds to the NB.
We map the zones to the corresponding points in the light curve using different colors to probe the evolution of the source on the Z-track. To extract the HID zone resolved products, we identify the observed time intervals of each zone.
These intervals are used to extract the simultaneous energy spectra from SXT and LAXPC (see section~\ref{ssec:spec}) and power density spectra from LAXPC observations (see section~\ref{ssec:timing}). 
To determine the typical \sz\ range covered by our observation, we use the timing properties of the source as seen in the PDS (see section~\ref{ssec:timing}) and compare it with the source properties as seen by \citet{Jonker2000ApJ...537..374J}. The HBO frequency of Zone 1 would indicate that the \sz$\approx$0.6 for Zone 1 and transition from HBO to NBO (which happens in Zone 4 to 5) would place their \sz\ at $\approx$1.4. Hence we assume that the source is observed at \sz\ in the range 0.5$-$1.5 and divide the interval in table~\ref{tab:spec_timing_pars} accordingly. 


% Figure environment removed

\subsection{Spectral analysis of the HID zones}\label{ssec:spec}

We model\footnote{Using the standard {\it XSPEC} software (version 12.12.1): \url{https://heasarc.gsfc.nasa.gov/xanadu/xspec/}} the simultaneous SXT and LAXPC spectra jointly (with total exposure for all HID zones with LAXPC at 92.8~ks and SXT at 29.6~ks), using the $0.8-7$~keV range for SXT \citep{chakraborty2020MNRAS.498.5873C} and the $4-25$~keV range  for LAXPC \citep[the background dominates above $\sim 25$~keV;][]{Bhargava2019MNRAS.488..720B}. 
The spectra of the source are shown in figure~\ref{fig:sxt_lxp_spec}. The spectra from both instruments are rebinned to match the respective  spectral resolutions. We include 1.5\% systematic error in SXT spectra \citep{sridhar2019MNRAS.487.4221S} and 3\% systematic error in LAXPC spectra \citep{Bhargava2019MNRAS.488..720B}. We also note that the spectrum of different HID zones have a distinct spectral shape which is evident in the ratio of all the zone spectra to the spectra from Zone 3 (see figure~\ref{fig:sxt_lxp_spec} bottom-left panel). 

% Figure environment removed

To characterize the spectrum, we first note that the source is highly absorbed \citep{Iaria2006ChJAS...6a.257I, Church2006A&A...460..233C}. We initially model the spectrum with an absorbed cutoff powerlaw (\texttt{cutoffpl}).  We include the neutral hydrogen absorption using the {\it XSPEC} model \texttt{tbabs} with abundances as suggested by \citet{Wilms2000ApJ...542..914W} and the cross sections from \citet{Vern1996ApJ...465..487V}. We find that the spectrum of the source is consistent with a column density of $10^{23}$~cm$^{-2}$ \citep{Iaria2006ChJAS...6a.257I, Church2006A&A...460..233C}. Similar to \citet{mitsuda1989PASJ...41...97M, lavagetto2004NuPhS.132..616L, Church2006A&A...460..233C,lin2012ApJ...756...34L, Seifina2013ApJ...766...63S}, we find that the spectrum requires at least an additional thermal component. For example for zone 3, $\chi^2  \approx 979$ for 118 degrees of freedom (DoF) for an absorbed powerlaw ({\tt tbabs*cutoffpl} in {\it XSPEC}) improves to $\chi^2 \approx 294$ for 116 DoF for  an absorbed cutoff powerlaw plus a blackbody ({\tt tbabs*(cutoffpl+bbodyrad)} in {\it XSPEC}). 
However, the fitting is still not adequate, as there is a significant residual at lower energies.
We model this additional component with a multicolor disk blackbody, \texttt{diskbb}, which gives an acceptable $\chi^2 \approx 128$ for 114 DoF.
Thus, this spectral model comprises two thermal components and a phenomenological cutoff powerlaw component. Here, {\tt diskbb} and {\tt bbodyrad} are the softer ($\sim 0.3$~keV) and the harder ($\sim 1$~keV) thermal components, respectively. Note that $\chi^2$ increases from 128 to 148 (DoF~$= 114$), if we interchange the temperatures of {\tt diskbb}  and {\tt bbodyrad} and determine the best fit.

To have a physical picture of the non-thermal component, we replace the {\tt cutoffpl} with the thermal Comptonization model \texttt{Nthcomp} \citep{nthzdz1996MNRAS.283..193Z, nthzycki1999MNRAS.309..561Z}, with the seed photons provided by the soft disk. Hence, we tie the {\tt diskbb} temperature with the seed photon temperature of \texttt{Nthcomp}.
Thus, we use the {\it XSPEC} model \texttt{constant*tbabs*(diskbb+bbodyrad+Nthcomp)} \citep{lin2009ApJ...696.1257L, lin2012ApJ...756...34L}, which works well for all HID zones (see left panel of figure~\ref{fig:sxt_lxp_spec} and table~\ref{tab:spec_timing_pars}). 
We fix the absorption column density $n_{\rm H}$ to 10$^{23}$~cm$^{-2}$ \citep{mitsuda1989PASJ...41...97M, Church2006A&A...460..233C, dai2009ApJ...693L...1D, Seifina2013ApJ...766...63S, miller2016ApJ...822L..18M} for all the zones and find acceptable fits. 
Note that a free $n_{\rm H}$ does not improve the fits significantly. 
Moreover, the trends observed in parameters from the fits are independent of the value at which $n_{\rm H}$ is fixed. 
The evolution of the key parameters is shown in figure~\ref{fig:par_evol} while the best-fit spectral parameters are noted in table~\ref{tab:spec_timing_pars}. The optical depth of the Comptonizing corona can be computed from {\tt Nthcomp} index $\Gamma$ and electron temperature k$T_{\rm e}$ \citep{Bhargava2019MNRAS.488..720B}. 

 

% Figure environment removed


\begin{table}
    \centering
      \caption{Spectral and timing parameter values for all HID zones of \src.   }\label{tab:spec_timing_pars}
    \begin{tabular}{lll|rrrrr}
    \toprule
Component               & Parameter & Units  & \multicolumn{5}{c}{Zone (\sz\ range)} \\
                        &           &                        & 1 (0.5--0.66)   & 2 (0.66--0.83)   & 3 (0.83--1) & 4 (1--1.25)  & 5 (1.25--1.5) \\ 
\midrule
tbabs                   & $n_{\rm H}$        & cm$^{-2}$&                   \multicolumn{5}{c}{10$^{23}$$^a$} \\
\midrule
\multirow{3}{*}{diskbb} & $T_{\rm in}$       &  keV                       & $0.24\pm{0.02}$	&	$0.28\pm{0.02}$	&	$0.28\pm{0.01}$	&	$0.30_{-0.03}^{+0.01}$	&	$0.27\pm{0.02}$ \\
                        & Norm$^b$      & 10$^{5}$                        & $9.1_{-3.4}^{+12.4}$	&	$2.4_{-0.8}^{+1.5}$	&	$3.3\pm0.9$	&	$2.1_{-0.4}^{+2.0}$	&	$3.4_{-1.3}^{+2.5}$ \\ 
                        & Flux (3--20~keV)$^c$      &  & $(7\pm1)\times10^{-4}$ & $(25\pm3)\times10^{-4}$ & $(28\pm3)\times10^{-4}$ & $(40\pm4)\times10^{-4}$ & $(26\pm5)\times10^{-4}$  \\

\midrule
\multirow{4}{*}{bbodyrad} & $T_{\rm BB}$     &   keV                       & $1.14_{-0.02}^{+0.04}$	&	$1.15_{-0.02}^{+0.01}$	&	$1.13\pm{0.01}$	&	$1.14_{-0.01}^{+0.02}$	&	$1.14\pm{0.01}$ \\
                        &  Norm$^d$     &                         & $356_{-64}^{+40}$	&	$449\pm50$	&	$622\pm49$	&	$638_{-107}^{+24}$	&	$592\pm47$ \\
                        & Flux (0.01--100~keV)$^c$      &  & $0.65\pm{0.01}$ & $0.85\pm{0.01}$ & $1.10\pm{0.01}$ &  $1.18\pm{0.01}$ & $1.07\pm{0.01}$  \\
                        & Flux (3--20~keV)$^c$      &  & $0.45\pm{0.01}$ & $0.59\pm{0.01}$ & $0.75\pm{0.01}$ & $0.81\pm{0.01}$ & $0.73\pm{0.01}$  \\
\midrule
\multirow{4}{*}{Nthcomp} & $\Gamma$     &                         &$1.76_{-0.06}^{+0.11}$	&	$1.66\pm{0.08}$	&	$1.51\pm0.10$	&	$1.45_{-0.07}^{+0.23}$	&	$1.66\pm0.12$ \\
                        & k$T_{\rm e}$          &   keV                      & $3.21\pm{0.08}$	&	$3.14\pm{0.07}$	&	$2.98\pm{0.07}$	&	$2.91_{-0.05}^{+0.22}$	&	$3.05\pm{0.10}$ \\
                        & Flux (0.01--100~keV)$^c$      & & $1.26\pm{0.04}$ &  $1.08\pm{0.04}$ & $0.86\pm{0.03}$ &  $0.69\pm{0.04}$ & $0.79\pm{0.05}$ \\
                        & Flux (3--20~keV)$^c$      &  & $0.65\pm{0.01}$ & $0.63\pm{0.02}$ & $0.57\pm{0.01}$ & $0.48\pm{0.02}$ & $0.45\pm{0.02}$  \\
\midrule
 & Total Flux (3--20~keV)$^c$  & & $1.10\pm{0.06}$ & $1.23\pm{0.05}$ & $1.33\pm{0.04}$ & $1.30\pm{0.03}$ &  $1.19\pm{0.03}$ \\
\midrule
    & $\chi^2$ & & 93.5 & 91.3 & 128.2 &106.7 &83.6  \\
    & DoF & & 104 & 107 & 114 & 109 & 102 \\
\midrule
\multicolumn{8}{c}{Timing parameters$^e$}\\
\midrule
  & QPO frequency & Hz & $29.3\pm{0.1}$ & $36.8\pm{0.1}$ & 43.5$_{-0.2}^{+0.5}$ & 48.5$_{-0.5}^{+0.3}$ & $5.2\pm{0.1}$\\
 & QPO FWHM & Hz & $6.7\pm{0.2}$ & 7.3$\pm{0.4}$ & 15$_{-1.5}^{+0.5}$ & 9.3$_{-0.8}^{+2.7}$ & 2.3$_{-0.7}^{+0.4}$ \\
 & Fractional r.m.s. & \% & $7.2\pm{0.1}$ & $5.1\pm{0.1}$ & 4.18$_{-0.18}^{+0.05}$ & $2.2\pm{0.1}$ & 2.4$_{-0.4}^{+0.1}$ \\
& Phase lag & radian &  0.11$\pm0.05$ & 0.79$\pm0.18$ & 3.05$\pm0.02$ & --- & --- \\
& Time lag & ms &  0.58$\pm0.25$ & 3.43$\pm0.77$ & 11.15$\pm0.14$ & --- & --- \\
\bottomrule

    \end{tabular}
    \begin{flushleft}
        \astr\ SXT+LAXPC spectra are fitted with the {\it XSPEC} model \texttt{tbabs*(diskbb+bbodyrad+Nthcomp)} and the \texttt{Nthcomp} seed photon temperature is tied to the \texttt{diskbb} temperature.
        1$\sigma$ errors are shown. \\
        Notes:\\
        a: The parameter is frozen to the value.   \\
        b: The normalization is defined as $(R_{\rm in}/D_{\rm 10})^2$ cos($\theta$), where $D_{\rm 10}$ is source distance in units of 10~kpc, $R_{\rm in}$ is apparent inner disk radius in km and $\theta$ is the disk inclination angle.\\
        c: Unabsorbed flux in $10^{-8}$ erg cm$^{-2}$ s$^{-1}$ as estimated using \texttt{cflux} on the corresponding component. \\
        d: The normalization is defined as Norm = $(R/D_{\rm 10})^2$,  where $R$ implies the size of the blackbody region in km. \\
        e: The timing parameters are derived by fitting the PDS (figure~\ref{fig:pds_zonewise}) with phenomenological models and characterizing the oscillation with a Lorentzian function. The lags are computed between $3-7$~keV and $7-20$~keV with positive lags indicating that the hard band is lagging the soft band.
    \end{flushleft}
    
\end{table}


\subsection{Timing analysis}\label{ssec:timing}



We extract the power density spectra (PDSs) in $0.03-500$~Hz for all HID zones mentioned in section~\ref{ssec:demarc}. For the extraction of the PDS we use a custom software \textit{GHATS}\footnote{Developed by Tomaso Belloni.}.  The PDSs are shown in figure~\ref{fig:pds_zonewise}. This figure shows HBOs for zones 1--4 in $\sim 30-50$ Hz and NBOs for zone 5 in $\sim 5$
 Hz. Note that the PDS of zone 4 also indicates a broad feature at NBO frequencies.
The PDS is converted to \textit{XSPEC} readable format and is modeled with multiple Lorentzian functions (to characterize broad noise features and QPOs) and a 
constant (for the white noise).
Properties of these timing components, such as QPO peak frequency, full-width-half-maximum (FWHM) and strength (fractional r.m.s. amplitude), are estimated from modeling. 
We also extract the PDSs in various energy bands (in $3-4.5$, $4.5-6$, $6-10$ and $10-20$~keV) to trace the dependence of the QPO fractional r.m.s. amplitude on energy. We show the energy-dependence of QPO fractional r.m.s. amplitude in figure~\ref{fig:qpo_rms}.  
Besides, we compute the phase and time lags between $3-7$~keV and $7-20$~keV bands for strong QPOs (i.e., HBOs in zones 1--3) by averaging over the QPO FWHM (the lags are reported in table~\ref{tab:spec_timing_pars}). 

% Figure environment removed



% Figure environment removed


The timing features show a strong dependence on the position on the HID.  To gain an insight about how these features evolve with time, we depict the frequency--time image of \src\ in figure~\ref{fig:dyn_pds}. We also overplot the $3-20$~keV light curve to highlight the correlated evolution of count rates and the peak frequency of timing features. As the source goes onto the HB, the HBO frequency decreases and the feature becomes stronger.
The opposite happens when the source comes out of the HB.
The low frequency broadband noise becomes strong when the HBO frequency goes below $\sim 30$~Hz. This is consistently seen for two deep HBs, as well as for a shallow HBs. \

% Figure environment removed



\section{Results and Discussion}\label{sec:res_disc}

\subsection{Sampling of the Z-track}
Using observations of \src\ with \astr, we investigate source spectral and timing properties to understand its evolution along the Z-track. Our observation traces parts of the HB and the NB several times without going into the FB. The observation spanned roughly 3.5 days which is similar to the timescale on which the source has been found to trace out a complete Z-track in the past \citep{Jonker2000ApJ...537..374J}. This shows that the time scale on which full Z-tracks are traced out can vary in GX 340+0. Similar behavior was observed in the Z source GX 17+2 by \citet{lin2012ApJ...756...34L}. In their observation the source first lingered on the HB and NB for about six days and before tracing out a full Z-track within $\sim$1.5 days.


\subsection{HB and NB as the source excursions into dips}

In our observation, we find that 
\src\ spends most of its time in HID zones 3 and 4 (see figure~\ref{fig:laxpc_hid}), and occasionally makes excursions into HB and NB, which appear as dips in the light curves. The appearance of HB and NB excursions as dips in the light curves is a property that is common for Z sources in their Cyg-like Z phase, like \src. As can be seen for the Z sources analyzed in \citet{fridriksson2015}, as source evolve from Sco-like Z source behavior to Cyg-like Z source behavior, the count rate variations along the HB and NB become increasingly stronger.  The dips in \src\ are of two distinctly different types: with high hard color values ($>$0.375) implying the HB and with low hard color values implying the NB.
For our LAXPC observation, the source is observed around the hard apex, the HB and the NB for 60.5, 21.2 and 11.1 ks, respectively (excluding data gaps). 
The depth of the dips, corresponding to the length of the HB or the NB, can also vary apparently randomly. For example, figure~\ref{fig:laxpc_hid} shows shallow HB-dips when the source goes up to zone 2 and comes back to the hard apex.
In our observation, there are two deep and longer ($\sim 25$~ks, including the data gaps) HB-dips, but all NB dips are relatively shallow and short ($\sim 10$~ks, including the data gaps).
Note, when the source goes into a dip it gradually comes out to the hard apex in the same way, without jumping to another dip or branch \citep[also seen in other Z-sources,][etc.]{sriram2011ApJ...743L..31S,homan2007ApJ...656..420H, lin2012ApJ...756...34L}. This indicates well-defined changes in the accretion processes during a dip or a Z-track branch. Therefore, we analyze spectral properties during dips to probe the accretion processes. 



\subsection{Understanding the shape of the Z-track}\label{Understanding_Z-track}

A long-standing goal regarding Z-sources is to understand the shape of their tracks in the HID, particularly the vertices. Thus, before describing detailed spectral properties, we attempt to reproduce the Z-track of \src\ using our spectral components. 
This track appears in $3-20$~keV (figure~\ref{fig:laxpc_hid}), and only {\tt bbodyrad} and {\tt Nthcomp} significantly contribute in this energy range (table~\ref{tab:spec_timing_pars}).
Thus, relative flux values of {\tt bbodyrad} and {\tt Nthcomp} (see right panel of figure~\ref{fig:sxt_lxp_spec}) are expected to give rise to the Z-track of figure~\ref{fig:laxpc_hid}.
To test this, we compute the {\tt Nthcomp} flux to {\tt bbodyrad} flux ratio in the same energy range ($6-20$~keV) in which the hard color is computed. 
We also estimate the total flux in the same energy range ($3-20$~keV) as used to compute the intensity of HIDs (figure~\ref{fig:laxpc_hid}).
In figure~\ref{fig:flux_hid}, we plot this flux ratio versus total flux. This plot qualitatively reproduces (including the bend) the Z-track of \src\ (see figure~\ref{fig:laxpc_hid}), as shown in the left panel of figure~\ref{fig:flux_hid}. This implies our spectral decomposition can describe the motion of the source on the HID and explains how the HB and the NB originate from the evolution of and the interplay between blackbody and Comptonizing components.
 

\subsection{Spectral components and  evolution of spectral parameters}\label{spectral_component}

The three components, accretion disk, NS surface, and Comptonizing corona, used in our X-ray spectral analysis, are naturally expected for NS LMXBs \citep{lin2007ApJ...667.1073L, mukherjee2011MNRAS.411.2717M,lin2012ApJ...756...34L}.
All three components are significant (see section~\ref{ssec:spec}), which is a progress relative to the previous two-component models (see section~\ref{sec:intro}). 

Our three-component spectral model adequately fits spectra of the hard apex and HB/NB dips of \src\ (see section~\ref{ssec:spec}). This enables us to uniformly study the observed HB and NB using the evolution of spectral parameters. 
This evolution can be seen in figure~\ref{fig:par_evol} and table~\ref{tab:spec_timing_pars}.
We find that the disk inner edge is far from the NS in the HID zone~1 ($\sim$1100~km)\footnote{Assuming a distance of 11~kpc \citep{Penninx1993A&A...267...92P} and a inclination of 35$^\circ$ \citep{dai2009ApJ...693L...1D}.} and may be somewhat nearer for other zones (500$-$700~km). As expected, the trend of the {\tt diskbb} temperature (k$T_{\rm in}$) is opposite to that of {\tt diskbb} normalization \citep{frank_king_raine_2002}. The disk temperature is consistent with the estimates from \citet{Seifina2013ApJ...766...63S} but slightly lower than typical Z-source disk temperature \citep[e.g. GX 17+2;][]{lin2012ApJ...756...34L}. 
The blackbody component could naturally be explained in terms of a spreading layer of accreted matter on the NS surface. This implies that the disk material reaches the NS. Moreover, the Keplerian disk is believed to extend up to near the NS surface for a high-luminosity NS LMXB like \src.
But a large disk inner radius (see table~\ref{tab:spec_timing_pars}) means that the inner part of the disk is possibly hidden. This indicates that the corona covers the inner disk and hence gets seed photons from the disk. 
We find that the blackbody temperature remains almost constant throughout the HID zones. But, its normalization, which can be interpreted as proportional to the spreading layer area on the NS surface, increases from zone 1 to zone 3 (i.e., from the HB to the hard apex) and then remains almost same in the hard apex and the NB. The blackbody bolometric ($0.01-100$~keV) flux also increases from zone 1 to zone 3. 
But, all of {\tt Nthcomp} index ($\Gamma$), electron temperature (k$T_{\rm e}$) and bolometric flux decrease, while the optical depth ($\tau$) increases, from zone 1 to zone 3. The right panel of figure~\ref{fig:flux_hid} shows that the trend of the bolometric flux of the blackbody is exactly opposite to that of the Comptonizing corona. 
Besides, table~\ref{tab:spec_timing_pars} shows that the decrease in the blackbody flux causes the HB-dip in the $3-20$~keV range.

% Figure environment removed

\subsection{Spectro-Timing properties}\label{timing}

HID zones 1--4 show a low frequency broad noise and HBOs, while zone 5, and perhaps also zone 4, show NBOs (figure~\ref{fig:pds_zonewise}).
Note that simultaneous observations of HBOs and NBOs (such as indicated for our zone 4) were previously reported for various Z-sources \citep{Jonker2000ApJ...537..374J,Jonker2002MNRAS.333..665J,Homan2002ApJ...568..878H, sriram2011ApJ...743L..31S}.
Our timing analysis shows HBOs ($\sim 30-50$ Hz) and NBOs ($\sim 5$ Hz) are observed in the expected frequency ranges \citep[e.g.,][see also section~\ref{sec:intro}]{vanParadijs1988MNRAS.231..379V, Penninx1991MNRAS.249..113P, Jonker2000ApJ...537..374J}.
Table~\ref{tab:spec_timing_pars} shows that the HBO frequency decreases, but the strength increases, when the source goes into the HB-dip (i.e., zone 4 to zone 1; see also figures~\ref{fig:pds_zonewise} and \ref{fig:dyn_pds}). 
This trend reverses when the source comes out of the HB-dip (figure~\ref{fig:dyn_pds}). 
The broadband noise strengthens when the
HBO frequency decreases below 30 Hz (figure~\ref{fig:dyn_pds}). 

The strength of HBOs and NBOs in all HID zones significantly increases with energy (figure~\ref{fig:qpo_rms}), which is consistent with previous studies \citep[e.g., ][]{Penninx1991MNRAS.249..113P, Jonker2000ApJ...537..374J}. 
This trend, and particularly the fractional r.m.s. amplitude in the highest energy band of $10-20$~keV, cannot be explained if the HBOs/NBOs originate solely in either the accretion disk ({\tt diskbb}) or the NS surface ({\tt bbodyrad}; see the right panel of figure~\ref{fig:sxt_lxp_spec} for relative contributions of spectral components). Thus HBOs/NBOs should have a strong contribution from the Comptonizing corona ({\tt Nthcomp}). However, since the disk provides seed photons to the corona, the QPO frequencies might be determined by the disk properties. Hence the oscillation could originate in the accretion disk and enhance in strength in the corona. 
Note that a strong dependence of the oscillation r.m.s. with the coronal properties was extensively observed in black hole X-ray binaries \citep[e.g., ][]{kara2019Natur.565..198K,Bhargava2019MNRAS.488..720B, Rawat2023MNRAS.tmp..130R}.


We find that $7-20$~keV photons lag $3-7$~keV photons for HBOs and this lag drastically increases from zone 1 (deep HB-dip) to zone 3 (hard apex; see table~\ref{tab:spec_timing_pars}).
This could be because, as indicated by the rise of the coronal optical depth (see figure~\ref{fig:par_evol}), the number of Compton up-scattering interactions of photons increases from zone 1 to zone 3.
Such increasing scattering could also explain the decreasing strength of HBOs from the deep HB-dip to the hard apex.

\subsection{An explanation of the source behavior}

In our observation, \src\ primarily stays around the hard apex and occasionally goes ventures onto HB and NB for varying distances.
Since our NB excursions are short, we cannot reliably probe their origin.
But, we can study the processes involved in longer HB excursions.
As \src\ goes onto the HB, the bolometric flux and k$T_{\rm e}$ of the Comptonizing corona ({\tt Nthcomp}) increase and the bolometric flux and normalization (implying the area) of the spreading layer on the NS surface ({\tt bbodyrad}) decrease (see table~\ref{tab:spec_timing_pars}, figure~\ref{fig:par_evol} and the right panel of figure~\ref{fig:flux_hid}).
This indicates, as the source makes an excursion onto the HB, a lesser fraction of accretion disk matter reaches the NS surface, and more accreted matter  replenishes and energizes the corona. %The energizing of the corona could help replenish the energy of the corona which is critical for sustaining it for decades.  
This trend reverses when the source comes out of the HB. This happens for all HB excursions, regardless of their length.
Thus for longer HB excursions, the corona strengthens in $\sim 20$ ks and weakens in a similar time scale (figure~\ref{fig:laxpc_hid}).

When the corona is energized on the HB, it becomes hotter and its physical size perhaps increases, thus covering greater extent of the inner disk (implied by a higher {\tt diskbb} normalization for HID zone 1; table~\ref{tab:spec_timing_pars}). 
Therefore, the corona intercepts more disk photons and hence its bolometric flux increases. On the other hand, the optical depth of the corona decreases (figure~\ref{fig:par_evol}) implying  fewer Compton up-scattering interactions, which makes its spectrum softer (i.e., larger {\tt Nthcomp} $\Gamma$; see figure~\ref{fig:par_evol}).
As mentioned in section~\ref{timing}, these fewer up-scattering interactions also explains the decreased strength and hard-lag of HBOs on the HB. 




\section{Summary}

Here, we summarize the main points from the study of the Z-source \src\ using \astr\ data.
\begin{enumerate}
    \item As in other Cyg-like Z sources, we find that the HB and the NB appear as two different types of intensity dips from the hard apex level in $3-20$~keV. Such dips can appear in any sequence and with various depths. Thus, the source moved back and forth along the track several times. In our observation, it stays close to \sz=1 most of the time and makes occasional excursions onto the HB and NB.
    \item The spectra that we extracted along the parts of the Z-track covered by our observation can be explained by the emissions from an accretion disk, a Comptonizing corona covering the inner disk and a blackbody emitting spreading layer on the NS surface.
    \item When the source goes into the HB, the corona is energized by the disk and a relatively smaller amount of disk matter reaches the NS surface. Simultaneously, the coronal optical depth decreases, implying fewer Compton up-scattering of a disk photon. This enhances the strength and reduces the hard-lag of HBOs, which originate in the corona. This trend reverses when the source comes out of the HB-dip.
    
\end{enumerate}



\begin{acknowledgments}

This work makes use of data from the {\it AstroSat} mission of the Indian Space Research Organisation (ISRO), archived at Indian Space Science Data Centre (ISSDC). The article has used data from the SXT and the LAXPC developed at TIFR, Mumbai, and the {\it AstroSat} POCs at TIFR are thanked for verifying and releasing the data via the ISSDC data archive and providing the necessary software tools. 
This research has also made use of data and/or software provided by the High Energy Astrophysics Science Archive Research Center (HEASARC), which is a service of the Astrophysics Science Division at NASA/GSFC and the High Energy Astrophysics Division of the Smithsonian Astrophysical Observatory. 
The authors thank Tomaso Belloni for providing the tool {\it GHATS} for timing analysis. 
\end{acknowledgments}

%% To help institutions obtain information on the effectiveness of their 
%% telescopes the AAS Journals has created a group of keywords for telescope 
%% facilities.
%
%% Following the acknowledgments section, use the following syntax and the
%% \facility{} or \facilities{} macros to list the keywords of facilities used 
%% in the research for the paper.  Each keyword is check against the master 
%% list during copy editing.  Individual instruments can be provided in 
%% parentheses, after the keyword, but they are not verified.

\vspace{5mm}
\facilities{\astr\ \citep{Singh2014SPIE.9144E..1SS}}

%% Similar to \facility{}, there is the optional \software command to allow 
%% authors a place to specify which programs were used during the creation of 
%% the manuscript. Authors should list each code and include either a
%% citation or url to the code inside ()s when available.

 \software{GHATS, XSPEC \citep{xspec1996ASPC..101...17A}
            % astropy \citep{2013A&A...558A..33A,2018AJ....156..123A},  
          % Cloudy \citep{2013RMxAA..49..137F}, 
          % Source Extractor \citep{1996A&AS..117..393B}
          }


%
% \section{Gold Open Access}

% As of January 1st, 2022, all of the AAS Journals articles became open access, meaning that all content, past, present and future, is available to anyone to read and download. A page containing frequently asked questions is available at \url{https://journals.aas.org/oa/}.

% \section{Author publication charges} \label{sec:pubcharge}

% In April 2011 the traditional way of calculating author charges based on 
% the number of printed pages was changed.  The reason for the change
% was due to a recognition of the growing number of article items that could not 
% be represented in print. Now author charges are determined by a number of
% digital ``quanta''.  A single quantum is 350 words, one figure, one table,
% and one enhanced digital item.  For the latter this includes machine readable
% tables, figure sets, animations, and interactive figures.  The current cost
% for the different quanta types is available at 
% \url{https://journals.aas.org/article-charges-and-copyright/#author_publication_charges}. 
% Authors may use the online length calculator to get an estimate of 
% the number of word and float quanta in their manuscript. The calculator 
% is located at \url{https://authortools.aas.org/Quanta/newlatexwordcount.html}.

% \section{Rotating tables} \label{sec:rotate}

% The process of rotating tables into landscape mode is slightly different in
% \aastex v6.31. Instead of the {\tt\string\rotate} command, a new environment
% has been created to handle this task. To place a single page table in a
% landscape mode start the table portion with
% {\tt\string\begin\{rotatetable\}} and end with
% {\tt\string\end\{rotatetable\}}.

% Tables that exceed a print page take a slightly different environment since
% both rotation and long table printing are required. In these cases start
% with {\tt\string\begin\{longrotatetable\}} and end with
% {\tt\string\end\{longrotatetable\}}. Table \ref{chartable} is an
% example of a multi-page, rotated table. The {\tt\string\movetabledown}
% command can be used to help center extremely wide, landscape tables. The
% command {\tt\string\movetabledown=1in} will move any rotated table down 1
% inch. 

% A handy "cheat sheet" that provides the necessary \latex\ to produce 17 
% different types of tables is available at \url{http://journals.aas.org/authors/aastex/aasguide.html#table_cheat_sheet}.

% \section{Using Chinese, Japanese, and Korean characters}

% Authors have the option to include names in Chinese, Japanese, or Korean (CJK) 
% characters in addition to the English name. The names will be displayed 
% in parentheses after the English name. The way to do this in AASTeX is to 
% use the CJK package available at \url{https://ctan.org/pkg/cjk?lang=en}.
% Further details on how to implement this and solutions for common problems,
% please go to \url{https://journals.aas.org/nonroman/}.

%% For this sample we use BibTeX plus aasjournals.bst to generate the
%% the bibliography. The sample631.bib file was populated from ADS. To
%% get the citations to show in the compiled file do the following:
%%
%% pdflatex sample631.tex
%% bibtext sample631
%% pdflatex sample631.tex
%% pdflatex sample631.tex

\bibliography{ref}{}
\bibliographystyle{aasjournal}

%% This command is needed to show the entire author+affiliation list when
%% the collaboration and author truncation commands are used.  It has to
%% go at the end of the manuscript.
%\allauthors

%% Include this line if you are using the \added, \replaced, \deleted
%% commands to see a summary list of all changes at the end of the article.
%\listofchanges

\end{document}

% End of file `sample631.tex'.
