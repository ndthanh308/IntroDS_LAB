\documentclass[11pt]{amsart}
\usepackage{amsmath,amsfonts,amssymb,amsthm,mathscinet}
\usepackage{enumitem}
\usepackage{cite,hyperref}
\usepackage[noabbrev,capitalize]{cleveref}
\renewcommand{\epsilon}{\varepsilon}
\renewcommand{\phi}{\varphi}
\DeclareMathOperator{\alg}{alg}
\newcommand{\C}{{\mathbb C}}
\newcommand{\F}{{\mathbb F}}
\newcommand{\N}{{\mathbb N}}
\newcommand{\Z}{{\mathbb Z}}
\newcommand{\Fp}{\F_p}
\newcommand{\Fpu}{\F_p^*}
\newcommand{\Fq}{\F_q}
\newcommand{\Fu}{F^*}
\newcommand{\acF}{F^{\alg}}
\newcommand{\acFu}{F^{\alg*}}
\newcommand{\card}[1]{\left|{#1}\right|}
\newcommand{\cardF}{\card{F}}
\newcommand{\cardFu}{\card{\Fu}}
\newtheorem{theorem}{Theorem}[section]
\newtheorem{proposition}[theorem]{Proposition}
\newtheorem{lemma}[theorem]{Lemma}
\newtheorem{corollary}[theorem]{Corollary}
\theoremstyle{definition}
\newtheorem{definition}[theorem]{Definition}
\newtheorem{remark}[theorem]{Remark}

\begin{document}

\title[APN power functions with exponents expressed as fractions]{Almost perfect nonlinear power functions with exponents expressed as fractions}
\author[Katz, O'Connor, Pacheco, and Sapozhnikov]{Daniel J.\ Katz, Kathleen R.\ O'Connor, Kyle Pacheco, \\ and Yakov Sapozhnikov}\thanks{Daniel J.\ Katz and Kathleen R.\ O'Connor are with the Department of Mathematics, California State University, Northridge.  Kyle Pacheco and Yakov Sapozhnikov were with the Department of Mathematics, California State University, Northridge.  Yakov Sapozhnikov is with the Department of Mathematics and Statistical Science, University of Idaho.  This paper is based upon work supported in part by the National Science Foundation under Grants DMS-1500856, CCF-1815487, and CCF-2206454.}
\date{28 July 2023}

\begin{abstract}
Let $F$ be a finite field, let $f$ be a function from $F$ to $F$, and let $a$ be a nonzero element of $F$.
The discrete derivative of $f$ in direction $a$ is $\Delta_a f \colon F \to F$ with $(\Delta_a f)(x)=f(x+a)-f(x)$.
The differential spectrum of $f$ is the multiset of cardinalities of all the fibers of all the derivatives $\Delta_a f$ as $a$ runs through $F^*$.
The function $f$ is almost perfect nonlinear (APN) if the largest cardinality in the differential spectrum is $2$.
Almost perfect nonlinear functions are of interest as cryptographic primitives.
If $d$ is a positive integer, the power function over $F$ with exponent $d$ is the function $f \colon F \to F$ with $f(x)=x^d$ for every $x \in F$.
There is a small number of known infinite families of APN power functions.
In this paper, we re-express the exponents for one such family in a more convenient form.
This enables us to give the differential spectrum and, even more, to determine the sizes of individual fibers of derivatives.
\end{abstract}

\maketitle


\section{Introduction}

If $F$ is a finite field, then a {\it power function of $F$} is a function $f\colon F \to F$ with $f(x)=x^d$ where $d$ is a positive integer.
Power maps are useful as cryptographic primitives, as they can be used to introduce nonlinearity into systems and can be evaluated quickly.

This paper is especially concerned with power functions whose exponents are expressed as ratios, and to this end, we set a convention about how these ratios are interpreted.
First of all, it should be noted that if $d$ and $e$ are positive integers, and $x\mapsto x^d$ and $x\mapsto x^e$ are functions from a finite field $F$ to itself, then these functions equal each other if and only if $d \equiv e \pmod{\cardFu}$, so although exponents should be considered positive integers, it suffices to specify them up to congruence modulo $\cardFu$.
With this understood, our convention is that when a rational number $r$ is proposed as an exponent for a power function over the finite field $F$, the one should re-express this rational number in reduced form as $r=d_1/d_2$ with $\gcd(d_1,d_2)=1$ and then interpret $x \mapsto x^r$ over $F$ as $x \mapsto x^d$ where $d$ is a positive integer with $d \equiv d_1 d_2^{-1} \pmod{\cardFu}$; this is defined if and only if $\gcd(d_2,\cardFu)=1$ (otherwise we do not have $d_2^{-1}$ modulo $\cardFu$).

If $f\colon A \to B$ is any function and $b \in B$, then the {\it fiber of $f$ over $b$} is the set $f^{-1}(\{b\})=\{a \in A: f(a)=b\}$.
If $F$ is a field and $f\colon F \to F$ and $a \in \Fu$, then $\Delta_a f$ is the function from $F$ to $F$ with $(\Delta_a f)(x)=f(x+a)-f(x)$, and we call $\Delta_a f$ the {\it discrete derivative of $f$ in direction $a$}.
We define $\Delta=\Delta_1$, and if we omit mention of a direction, then the {\it discrete derivative of $f$} is just $\Delta f$ with $(\Delta f)(x)=f(x+1)-f(x)$.

If $F$ is a finite field, then for $(a,b) \in \Fu\times F$ {\it differential multiplicity of $f$ for $(a,b)$}, written $\delta_f(a,b)$, is the cardinality of the fiber of $\Delta_a f$ over $b$, i.e., $\delta_f(a,b)=(\Delta_a f)^{-1}(\{b\})=\{x \in F: f(x+a)-f(x)=b\}$.
The {\it differential spectrum of $f$} is the multiset of all $\delta_f(a,b)$ as $(a,b)$ runs through $\Fu\times F$.
Functions are resistant to differential cryptanalysis when all their differential multiplicities are small.
If $f$ is the power function $f(x)=x^d$, then it is not hard to show that $\delta_f(a,b)=\delta_f(1,b/a^d)$, so we can reconstruct all differential multiplicities from those of the form $\delta_f(1,c)$.
Therefore, we define the {\it reduced differential spectrum of $f$} as the multiset of all $\delta_f(1,c)$ as $c$ runs through $F$: it has the same elements as the full differential spectrum, but the number of instances of each value in the reduced spectrum is scaled by a factor of $1/\cardFu$.
Simply put, the reduced differential spectrum of $f$ is the multiset of cardinalities of the fibers of the discrete derivative of $f$.
There is a great deal of interest in functions where the largest differential multiplicity is $2$: these are called almost perfect nonlinear (APN) functions.

If we write $n_1 [a_1]+\cdots+n_t [a_t]$ where $t$ and $n_1,\ldots,n_t$ are nonnegative integers and $a_1,\ldots,a_t$ are distinct elements of some set, then this represents the multiset with $n_j$ instances of element $a_j$ for each $j \in \{1,\ldots,t\}$ (and nothing more).
If $f(x)=x^d$ is a power function over a finite field and $\cardF$ is even or $d$ is odd, then it is easy to check that every fiber of $\Delta f$ is closed under the involution $x\mapsto -1-x$, which has no fixed point if $\cardF$ is even and only one fixed point, $-1/2$, if $\cardF$ is odd.
This makes all fibers of the discrete derivative have even cardinality when $\cardF$ is even, and so an APN power function in characteristic $2$ has reduced differential spectrum $(\cardF/2) [0] + (\cardF/2) [2]$.
It also means that if $\cardF$ and $d$ are both odd, then the reduced differential spectrum of an APN power function must be $(\cardFu/2) [0] + 1 [1] + (\cardFu/2) [2]$.
But the differential spectrum for APN power functions with even exponents over fields of odd characteristic are more difficult to obtain.

The purpose of this paper is to examine the fibers of the discrete derivatives of an infinite family of APN power functions over finite fields of characteristic $3$.
To understand our theorem, one should recall that $\Fq$ denotes the finite field of order $q$, and that if $\Fq$ of odd characteristic, then the quadratic character of $\Fq$ maps quadratic residues to $1$ and quadratic nonresidues to $-1$.
We now state our main result.
\begin{theorem}\label{Victoria}
Let $n$ be an odd positive integer and $k$ a nonnegative even integer with $\gcd(n,k)=1$.  Let $F$ be the finite field of order $3^n$, and let $f\colon F \to F$ be the power function with exponent $(3^n+1)/(3^k+1)$.  Let $\eta$ be the quadratic character for $F$.  Then for $c \in F$, we have
\[
\card{(\Delta f)^{-1}(\{c\})}=
\begin{cases}
1 & \text{if $c \in \F_3$,} \\
1+\eta(1-c^{3^k+1}) & \text{otherwise.}
\end{cases}
\]
In particular, $f$ is an APN function with reduced differential spectrum
\[
((3^n-3)/2) [0] + 3 [1] + ((3^n-3)/2) [2].
\]  
\end{theorem}
The APN power functions described by this theorem were discussed under a different guise in \cite[Theorem 4.1]{Zha-Wang}, which only maintained that they are APN without determining the precise differential spectrum or the even more detailed information in \cref{Victoria} about which fibers of the discrete derivative are of which size.
To explain why we are speaking of essentially the same power functions, we say that two exponents $d$ and $e$ are {\it equivalent} over a finite field $F$ to mean that there is some $j \in \Z$ and $k \in \{1,-1\}$ such that $d\equiv p^j e^k \pmod{\cardFu}$, where we are only allowed to use $k=-1$ if $\gcd(e,\cardFu)=1$.  This gives an equivalence relation on positive integers.  For a long time it has been known (see, for example, \cite[pp.\ 475, 476]{Helleseth-Rong-Sandberg} and \cite[p.\ 1271]{Dobbertin-1999-Welch}) that when exponents are equivalent they produce the same differential spectrum. 
In \cref{Guido}, we show that the family of exponents treated in \cref{Victoria} are, up to equivalence, coextensive with those treated in \cite[Theorem 4.1]{Zha-Wang}.


\section{Preliminaries}

Throughout this paper $\N$ denotes the set of nonnegative integers.
We use $\Fq$ to denote the finite field of order $q$.
If $F$ is a field, then we use $\acF$ to denote an algebraic closure of $F$ and $\acFu$ to denote the unit group of the algebraic closure.

\subsection{Valuations}

\begin{definition}\label{pcount}
Let $n \in \Z$,  $k\in\N$, and let $p$ be a prime. Suppose that $p^k|n$ and $p^{k+1}\nmid n$. Then the \textit{$p$-adic valuation of $n$}, denoted $v_p(n)$ is, given by $v_p(n)=k$. If $n=0$ then $v_p(0)=\infty$.
\end{definition}

\begin{lemma}\label{Eric}
Let $a \in \Z$ with $a\equiv 1 \pmod{4}$ and let $j$ be a nonnegative integer.
Then we have \[v_2(a^j+1)=1\] and \[v_2(a^j-1)=v_2(a-1)+v_2(j).\]
\end{lemma}
\begin{proof}
Note that $a^j+1 \equiv 1^j+1 \equiv 2 \pmod{4}$, so $v_2(a^j+1)=1$.

For the second claim, if $j$ is odd, then write $a=1+2^m u$ for some $m \geq 2$ and some odd integer $u$.
Then note that $a^2=1+2^{m+1} u + 2^{2 m} u^2 \equiv 1 \pmod{2^{m+1}}$.
So then $a^j \equiv a \equiv 1+2^m \pmod{2^{m+1}}$, and so $a^j-1 \equiv 2^m \pmod{2^{m+1}}$, and so $v_2(a^j-1)=m=v_2(2^m u)=v_2(a-1)=v_2(a-1)+v_2(j)$.

Now we prove the second claim for even $j$ by induction on $j$, where the $j=0$ case is obvious.
Suppose that $j$ is even and positive and write $j=2 k$.
Then $v_2(a^j-1)=v_2(a^{2 k}-1)=v_2((a^k+1)(a^k-1))=v_2(a^k+1)+v_2(a^k-1)=1+v_2(a^k-1)=1+v_2(a-1)+v_2(k)=v_2(a-1)+v_2(2 k)=v_2(a-1)+v_2(j)$.
\end{proof}

\begin{lemma}\label{Francis}
Let $a \in \Z$ with $a \equiv 3 \pmod{4}$ and let $j$ be a nonnegative integer.  Then we have
\[
v_2(a^j+1) =
\begin{cases}
1 & \text{if $j$ is even,} \\
v_2(a+1) & \text{if $j$ is odd,}
\end{cases}
\]
and
\[
v_2(a^j-1) =
\begin{cases}
v_2(a+1)+v_2(j) & \text{if $j$ is even,} \\
1 & \text{if $j$ is odd.}
\end{cases}
\]
\end{lemma}
\begin{proof}
If $j$ is even, then $a^j+1\equiv (-1)^j+1 \equiv 1 +1 \equiv 2 \pmod{4}$, so $v_2(a^j+1)=1$.

If $j$ is odd, then write $a=-1+2^m u$ for some $m \geq 2$ and $u$ odd.
Note that $a^2 =1-2^{m+1} u+2^{2 m} u^2 \equiv 1 \pmod{2^{m+1}}$, so $a^j \equiv a \equiv -1+ 2^m \pmod{2^{m+1}}$, so that $a^j+1 \equiv 2^m \pmod{2^{m+1}}$, and so $v_2(a^j+1)=m=v_2(2^m u)=v_2(a+1)$.

If $j$ is odd, then $a^j-1 \equiv (-1)^j -1 \equiv -1-1 \equiv 2 \pmod{4}$, so $v_2(a^j-1)=1$.

We prove the claim about $v_2(a^j-1)$ when $j$ is even by induction on $j$, where the $j=0$ case is obvious.
If $j$ is even and positive, write $j=2 k$, and then $v_2(a^j-1)=v_2(a^{2 k}-1)=v_2((a^k-1)(a^k+1))=v_2(a^k-1)+v_2(a^k+1)$.
If $k$ is odd, then previous parts of this proof show that $v_2(a^k-1)+v_2(a^k+1)=1+v_2(a+1)=v_2(a+1)+v_2(2 k)=v_2(a+1)+v_2(j)$.
If $k$ is even, then the induction hypothesis and a previous part of this theorem show that $v_2(a^k-1)+v_2(a^k+1)=\left(v_2(a+1)+v_2(k)\right)+1=v_2(a+1)+v_2(2 k)=v_2(a+1)+v_2(j)$.
\end{proof}

\subsection{Greatest common divisors}

The following fact is well known, so we record it without proof.
\begin{lemma}\label{Hortense}
Let $a$ be an integer, and let $m$ and $n$ be nonnegative integers.  Then $\gcd(a^m-1,a^n-1)=a^{\gcd(m,n)}-1$.
\end{lemma}
The next fact is not as well known.
\begin{lemma}\label{Clementine}
Let $a$ be an integer, and let $m$ and $n$ be nonnegative integers with $n$ odd and $\gcd(m,n)=1$.  Then
\[
\gcd(a^m+1,a^n-1)=
\begin{cases}
1 &\text{ if $a$ is even} \\
2 &\text{ if $a$ is odd}.
\end{cases}
\]
Furthermore if $a$ is odd, then
\[
\gcd((a^m+1)/2,a^n-1)=
\begin{cases}
1 &\text{ if $a \equiv 1 \pmod{4}$ or $m$ is even} \\
2 &\text{ if $a \equiv 3 \pmod{4}$ and $m$ is odd}.
\end{cases}
\]
\end{lemma}
\begin{proof}
Note that
$\gcd(a^m+1,a^n-1) \mid \gcd((a^m+1)(a^m-1),a^n-1)$, and
\begin{align*}
\gcd((a^m+1)(a^m-1),a^n-1)
&= \gcd(a^{2m}-1,a^n-1) \\
&=a^{\gcd(2m,n)}-1 \\
&=a-1,
\end{align*}
where we have used \cref{Hortense} in the penultimate equality and the fact that $n$ is odd and coprime to $m$ in the final equality.
Therefore $\gcd(a^m+1,a^n-1) \mid a-1$.  Because $\gcd(a^m+1,a^n-1) \mid a^m+1$, we have $\gcd(a^m+1,a^n-1) \mid \gcd(a-1, a^m+1)$.  But $a^m+1 \equiv 1^m+1 \equiv 2 \pmod{a-1}$, so $\gcd(a-1, a^m+1)=\gcd(a-1,2)$, which is $1$ if $a$ is even and is $2$ if $a$ is odd. So we are done if $a$ is even.  If $a$ is odd then $\gcd(a^m+1,a^n-1) \mid 2$, but both $a^m+1$ and $a^n-1$ are even, so $\gcd(a^m+1,a^n-1)=2$.

Now suppose that $a$ is odd.  Note that $\gcd((a^m+1)/2,a^n-1) \mid \gcd(a^m+1,a^n-1)=2$.  If $a \equiv 1 \pmod{4}$ or $m$ is even, then $(a^m+1)/2$ is odd by Lemma \ref{Eric} or \ref{Francis}, so that $\gcd((a^m+1)/2,a^n-1)=1$.  If $a \equiv 3 \pmod{4}$ and $m$ is odd, then $(a^m+1)/2$ is even by \cref{Francis} (as is $a^n-1$), so that $\gcd((a^m+1)/2,a^n-1)=2$.
\end{proof}

\subsection{Half field, conjugation, and the unit circle}

\begin{definition}[half field]
If $F$ is a finite field of characteristic $p$ and order $q=p^n$ with $n$ even, then the {\it half field of $F$}, denoted $H_F$, is the unique subfield of $F$ with $[F:H_F]=2$, i.e., the subfield of order $p^{n/2}$.
\end{definition}
\begin{definition}[conjugation]
If $F$ is a finite field of characteristic $p$ and order $q=p^n$ with $n$ even, then the {\it conjugation map on $F$} is the power map $x\mapsto x^{p^{n/2}}$ of $F$.  This is an automorphism of $F$ of order $2$ that whose fixed field is the half field $H_F$.  The {\it conjugate of $x$} is $x^{p^{n/2}}$, and is often denoted $\bar{x}$.  Thus, if $x \in \acF$, then $x \in H_F$ if and only if $\bar{x}=x$ and $x \in F$ if and only if $\bar{\bar{x}}=x$.
\end{definition}
\begin{definition}[unit circle]\label{Ursula}
If $F$ is a finite field of characteristic $p$ and order $q=p^n$ with $n$ even, then the {\it unit circle of $F$}, denoted $U_F$, is the set $\{x \in F: x^{p^{n/2}}=x^{-1}\}$.  Equivalently, it is the set of elements in $F$ (or in $\acF$) whose conjugates (using the conjugation map of $F$) equal their inverses.  It is also the unique subgroup of $\Fu$ (or $\acFu$) of order $p^{n/2}+1$, and it is a cyclic subgroup generated by $\alpha^{p^{n/2}-1}$ if $\alpha$ is a primitive element of $F$.
\end{definition}
\begin{lemma}\label{Lenny}
Let $F$ be a finite field of characteristic $p$ and order $p^n$ with $n$ even.  Then $H_F\cap U_F=\{1,-1\}$.
\end{lemma}
\begin{proof}
Let $x\mapsto \bar{x}$ denote the conjugation map on $F$.
Then $H_F\cap U_F$ is the set of all $x \in F$ with both $\bar{x}=x$ and $\bar{x}=x^{-1}$, so that a necessary condition for being in the intersection is that $x=x^{-1}$, i.e., $x\in\{-1,1\}$.
This condition is also clearly sufficient since $1$ and $-1$ are self-conjugate and self-inverse.
\end{proof}

\subsection{Quadratic residues and nonresidues}

For a finite field $F$ of odd characteristic, a {\it quadratic residue of $F$} is an element of the form $a^2$ with $a \in \Fu$ and a {\it quadratic nonresidue of $F$} is an element of $\Fu$ that is not a quadratic residue.
The $0$ element is considered neither a quadratic residue nor a quadratic nonresidue.
There are $\cardFu/2$ quadratic residues and $\cardFu/2$ quadratic nonresidues in $F$.
If $a \in F$, then
\[
a^{\cardFu/2}=
\begin{cases}
0 & \text{if $a=0$,} \\
1 & \text{if $a$ is a quadratic residue,} \\
-1 & \text{if $a$ is a quadratic nonresidue.}
\end{cases}
\]
The {\it extended quadratic character of $F$} is the map $\eta\colon F \to \C$, that maps $0$ to $0$, quadratic residues to $1$, and quadratic nonresidues to $-1$.  Thus, if $F$ is of characteristic $p$, then $\eta(a)=a^{(q-1)/2}$ modulo $p$ for every $a \in F$.
We therefore have $\eta(a b)=\eta(a)\eta(b)$ for every $a,b \in F$.  If  we remove $0$ from the domain and codomain of $\eta$, we obtain the {\it quadratic character of $F$}, which is a homomorphism from the multiplicative group of $F$ onto the multiplicative subgroup $\{1,-1\}$ of $\C$.
\begin{lemma}\label{Quentin}
Let $F$ be a finite field of characteristic $p$ and order $p^n$.  The element $-1$ is a quadratic nonresidue in $F$ if $p\equiv 3 \pmod{4}$ and $n$ is odd; otherwise $-1$ is a quadratic residue.
\end{lemma}
\begin{proof}
We know that $-1$ is a quadratic nonresidue in $F$ if and only if $(-1)^{(p^n-1)/2}=-1$, which happens if and only if $(p^n-1)/2$ is odd.  By Lemmas \ref{Eric} and \ref{Francis}, this occurs if and only if $p\equiv 3 \pmod{4}$ and $n$ is odd.
\end{proof}
\begin{lemma}\label{Muriel}
Let $F$ be a finite field of odd characteristic $p$ and order $p^n$ and let $\eta$ be the extended quadratic character of $F$.  Then we have $\eta(1-c^2)=0$ if and only if $c\in\{1,-1\}$.
\begin{enumerate}[label=(\roman*)]
\item If $p\equiv 3 \pmod{4}$ and $n$ is odd, then there are $(p^n-1)/2$ values of $c \in F \smallsetminus\{1,-1\}$ (including $c=0$) such that $\eta(1-c^2)=1$, and for the remaining $(p^n-3)/2$ values of $c \in F\smallsetminus\{1,-1\}$ we have $\eta(1-c^2)=-1$.
\item Otherwise, there are $(p^n-3)/2$ values of $c \in F \smallsetminus\{1,-1\}$ (including $c=0$) such that $\eta(1-c^2)=1$, and for the remaining $(p^n-1)/2$ values of $c \in F\smallsetminus\{1,-1\}$ we have $\eta(1-c^2)=-1$.
\end{enumerate}
\end{lemma}
\begin{proof}
The statements about what happens when $c\in\{1,-1,0\}$ are clear.
Consider the character sum
\begin{align*}
\sum_{c \in F\smallsetminus\{1,-1\}} \eta(1-c^2)
& = \sum_{c \in F} \eta(1-c^2) \\
& = \sum_{b \in F} \eta(1-(b+1)^2) \\
& = \sum_{b \in F} \eta(-b^2-2 b) \\
& = \sum_{b \in \Fu} \eta(-b^2-2 b) \\
& = \sum_{b \in \Fu} \eta(b^{-2}) \eta(-2 b-b^2) \\
& = \sum_{b \in \Fu} \eta(-2 b^{-1} -1) \\
& = \sum_{c \in F\smallsetminus\{-1\}} \eta(c) \\
& = -\eta(-1)+ \sum_{c \in F} \eta(c) \\
& = -\eta(-1),
\end{align*}
where we used the fact that there are equally many quadratic residues and quadratic nonresidues in the last equality.

If $p\equiv 3 \pmod{4}$ and $n$ is odd, then \cref{Quentin} tells us that $-\eta(-1)=1$.  Then our character sum tells us that there is one more quadratic residue than quadratic nonresidue among the $p^n-2$ elements $1-c^2$ as $c$ runs through $F\smallsetminus\{1,-1\}$.  This means there must be $(p^n-1)/2$ quadratic residues and $(p^n-3)/2$ quadratic nonresidues.

In all other cases, \cref{Quentin} tells us that $-\eta(-1)=-1$.   Then our character sum tells us that there is one more quadratic nonresidue than quadratic residue among the $p^n-2$ elements $1-c^2$ as $c$ runs through $F\smallsetminus\{1,-1\}$.  This means there must be $(p^n-3)/2$ quadratic residues and $(p^n-1)/2$ quadratic nonresidues.
\end{proof}

The following is a congruence used in determining whether certain elements are quadratic residues or nonresidues over prime fields.
\begin{lemma}\label{Molly}
Let $p$ be an odd prime and $j$ a nonnegative integer.  Then $(p^j-1)/2 \equiv j(p-1)/2 \pmod{p-1}$.
\end{lemma}
\begin{proof}
What we want to prove is equivalent to $p^j-1 \equiv j (p-1) \pmod{2(p-1)}$.
Notice that $p^j-1 =(p-1) \sum_{k=0}^{j-1} p^k$, which is $p-1$ times a sum of $j$ odd numbers, so it is an even (resp., odd) multiple of $p-1$ if $j$ is even (resp., odd).  The same can be said of $j(p-1)$, so they match modulo $2(p-1)$.
\end{proof}

\subsection{Permuting fibers}

The following two lemmas result will be used to compute fiber sizes for the discrete derivative of a power function by computing fiber sizes of more tractable functions
\begin{lemma}\label{Fedor}
Let $A$ and $B$ be sets, let $\pi$ be a permutation of $B$, let $g\colon A \to B$ and $f=\pi \circ g$.  Then for each $b \in B$ we have $f^{-1}(\{b\})=g^{-1}(\{\pi^{-1}(b)\})$, so that $|f^{-1}(b)|=|g^{-1}(\{\pi^{-1}(b)\})|$.  Thus, the multiset of cardinalities of fibers of $f$ is the same as the multiset of cardinalities of fibers of $g$.
\end{lemma}
\begin{proof}
We know that $a \in f^{-1}(\{b\})$ if and only if $b=\pi(g(a))$, which is true if and only if $\pi^{-1}(b)=g(a)$, which is true if and only if $a \in g^{-1}(\{\pi^{-1}(b)\})$.  Since $\pi$ is a permutation of $B$, we know that $\pi^{-1}(b)$ runs through $B$ as $b$ runs through $B$, so the multiset of cardinalities of fibers of $f$ is the same as that of $g$.
\end{proof}
\begin{lemma}\label{Leonid}
Let $A$ and $B$ be sets, let $\pi$ be a permutation of $A$, let $g\colon A \to B$, and let $f=g \circ \pi$.  Then for each $b \in B$ we have $f^{-1}(\{b\})=\pi^{-1}(g^{-1}(\{b\}))$, so that $|f^{-1}(\{b\})|=|g^{-1}(\{b\})|$.  Thus, the multiset of cardinalities of fibers of $f$ is the same as the multiset of cardinalities of fibers of $g$.
\end{lemma}
\begin{proof}
We know that $a \in f^{-1}(\{b\})$ if and only if $b=g(\pi(a))$, which is true if and only if $\pi(a) \in g^{-1}(\{b\})$, which is true if and only if $a \in \pi^{-1}(g^{-1}(\{b\}))$.  Since $\pi$ is a permutation, the cardinalities of $\pi^{-1}(g^{-1}(\{b\}))$ and $g^{-1}(\{b\})$ are the same.
\end{proof}


\subsection{Power functions}

\begin{lemma}\label{Herbert}
Let $F$ be a finite field of order $q$ with primitive element $\alpha$, $d$ a positive integer, and $f\colon F \to F$ with $f(x)=x^d$.  Let $g=\gcd(d,\cardFu)$.  Let $K$ be the unique cyclic subgroup of $\Fu$ of order $g$, which is generated by $\alpha^{(q-1)/g}$.  For $c \in F$ we have
\begin{enumerate}[label=(\roman*)]
\item If $c=0$, then $f^{-1}(\{c\})=\{0\}$.
\item If $c$ is the $g$th power of some element of $\Fu$, then $f^{-1}(\{c\})$ is a coset of $K$ in $\Fu$, and in particular, $f^{-1}(\{1\})=K$.
\item Otherwise, $f^{-1}(\{c\})$ is empty.
\end{enumerate}
In particular, $f$ is a permutation if and only if $g=1$.
\end{lemma}
\begin{proof}
It is clear that $f^{-1}(\{0\})=\{0\}$, so assume $c \in \Fu$.
If we restrict the domain and codomain of $f$ to $\Fu$, it is easy to verify that we get an endomorphism of the abelian group $\Fu$.
The kernel of this endomorphism (which is the fiber over $1$) consists of those elements $b \in \Fu$ with $b^d=1$, or equivalently those elements $b\in\acFu$ with $b^{q-1}=b^d=1$.
These, in turn, are the elements $b\in\acFu$ with $b^g=1$, which is the unique cclic subgroup of order $g$ in $\acFu$, which is also the unique subgroup of order $g$ in $\Fu$ and is generated by $\alpha^{(q-1)/g}$.
The fibers of a homomorphism are empty sets and cosets of the kernel.
So each fiber $f^{-1}(\{c\})$ for $c\in \Fu$ is of cardinality $0$ or $g$.
There must be $(q-1)/g$ nonempty fibers of cardinality $g$ in addition to the singleton fiber over $0$.
So the image of our endomorphism must be the unique subgroup of order $(q-1)/g$ in $\Fu$, which is the set of $g$th powers of elements of $\Fu$.
Any element $c$ in this image has $f^{-1}(\{c\})$ equal to some coset of $K$, and for all other $c\in\Fu$, the fiber $f^{-1}(\{c\})$ is empty.

Other than the fiber of $f$ over $0$ (which is $\{0\}$), all nonempty fibers of $f$ have cardinality $g$.
So $f$ is injective if and only if $g=1$, and since $f$ is a function from a finite set to itself, this means that $f$ is a permutation if and only if $g=1$.
\end{proof}

\begin{corollary}\label{Yolanda}
Let $F$ be a finite field of order $3^n$ with $n$ odd, let $k$ be an even nonnegative integer with $\gcd(n,k)=1$, let $e_2=(3^k-1)/2$, and let $g \colon F \to F$ be the power map $g(x)=x^{e_2}$.  Then $\gcd(e_2,\cardFu)=2$ and for each $c \in F$, we have
\[
|g^{-1}(\{c\})|=
\begin{cases}
1 & \text{if $c=0$,} \\
2 & \text{if $c$ is a quadratic residue in $\Fu$,}\\
0 & \text{otherwise.}
\end{cases}
\]
If $a \in F$, then the fiber of $g$ that contains $a$ is equal to $\{a,-a\}$.
\end{corollary}
\begin{proof}
We have $\gcd(e_2,\cardFu) \mid \gcd(3^k-1,3^n-1) = 3^{\gcd(k,n)}-1=2$ by \cref{Hortense}.
On the other hand, both $3^n-1$ and $(3^k-1)/2$ are even by \cref{Francis}, so $\gcd(e_2,\cardFu)=2$.
The rest follows by \cref{Herbert} since the unique subgroup of order $2$ in $F$ is $\{1,-1\}$.
\end{proof}


\section{Relation to a known family of exponents}\label{Guido}

Consider the family of power functions discussed in \cref{Victoria} given by $x \mapsto x^{(3^n+1)/(3^k+1)}$ over $\F_{3^n}$ where $n$ is odd, $k$ is even, and $\gcd(n,k)=1$.
(Note that $\gcd((3^k+1)/2,3^n-1)=1$ by \cref{Clementine}, so that our exponent can be interpreted as $((3^n+1)/2)((3^k+1)/2)^{-1}$ modulo $3^n-1$ following the convention in the Introduction.)
We claim that the exponents in this family of power functions are, up to the equivalence defined in the Introduction, the same as those in a theorem of Zha and Wang \cite[Theorem 4.1]{Zha-Wang}.
We now state Zha and Wang's theorem.
\begin{theorem}[Zha--Wang, 2010]\label{Zenobia}
Let $F$ be the finite field of order $3^n$.
Let $d$, $m$ and $k$ be integers with $d$ even and positive, $m$ nonnegative, and $k$ odd.
Suppose that $\gcd(m,n)=1$, that $2m<n$, and that $(3^m+1)d-2=k(3^n-1)$.
Let $f: F \rightarrow F$ be the power function $f(x)=x^d$.
Then $f$ is APN.
\end{theorem}
We now prove the equivalence of our exponents and those of Zha and Wang.
\begin{lemma}\label{Jeanenne}
If $n, m, k$ and $d$ are integers with $n$ and $d$ positive, where $d$ is even, $k$ is odd, $\gcd(m,n)=1$, $2m<n$ and $(3^m+1)d-2=k(3^n-1)$, then $n$ is odd and $d$ is equivalent to $((3^n+1)/2) ((3^j+1)/2)^{-1} \pmod{3^n-1}$ over $\F_{3^n}$ for some even $j \in \N$ with $\gcd(j,n)=1$.  Conversely, if $n$ is an odd integer and $d \equiv ((3^n+1)/2) ((3^j+1)/2)^{-1} \pmod{3^n-1}$ where $j \in \N$ is even with $\gcd(j,n)=1$, this $d$ is equivalent to an even exponent $d'$ that satisfies the equation $(3^m+1)d'-2=k(3^n-1)$ for some $k,m \in \N$ with $k$ odd, $\gcd(m,n)=1$ and $2m<n$.
\end{lemma}
\begin{proof}
Suppose we have some even exponent $d$ over $\F_{3^n}$ satisfying 
\begin{equation}\label{Jane}
(3^m+1)d-2=k(3^n-1) 
\end{equation} for some $k,m \in \N$ with $k$ odd, $\gcd(m,n)=1$, and $2m<n$.
Then notice that since both $3^m+1$ and $d$ are even, we have $(3^m+1) d-2 \equiv 2 \pmod{4}$.  Thus, \eqref{Jane} forces $3^n-1 \equiv 2 \pmod{4}$, which forces $n$ to be odd.
Next note that $d$ satisfies the following
\begin{equation}\label{Tabitha}
    \left(\frac{3^m+1}{2}\right)d \equiv \frac{3^n+1}{2} \pmod{3^n-1}
\end{equation} 
because $d$ satisfies $(3^m+1)d=2+(2 \ell +1)(3^n-1)$ for $\ell=(k-1)/2 \in \Z$, which means that $d$ satisfies $((3^m+1)/2)d =\ell (3^n-1)+(3^n+1)/2$.

If $m$ is even, then $\gcd((3^m+1)/2,3^n-1)=1$ by \cref{Clementine}. So then $(3^m+1)/2$ is invertible modulo $3^n-1$, and \eqref{Tabitha} becomes $d \equiv ((3^n+1)/2)((3^m+1)/2)^{-1} \pmod{3^n-1}$ with $m$ even and $\gcd(m,n)=1$.

On the other hand, if $m$ is odd, then let $e=3^m d$ (which is even since $d$ is even).
Then we have $3^{n-m} e = 3^n d \equiv d \pmod{3^n-1}$.  Thus, substituting $3^{n-m} e$ in for $d$ in \eqref{Tabitha} satisfies \eqref{Tabitha}.  Doing so gives us
\[
    \left(\frac{3^m+1}{2}\right)  3^{n-m} e \equiv \frac{3^n+1}{2} \pmod{3^n-1},
\]
which is
\[
    \left(\frac{3^n+3^{n-m}}{2}\right)  e \equiv \frac{3^n+1}{2} \pmod{3^n-1},
\]
which is
\[
    \left(\frac{3^n-1}{2}+\frac{3^{n-m}+1}{2}\right)  e \equiv \frac{3^n+1}{2} \pmod{3^n-1},
\]
which is
\[
    \left(\frac{3^{n-m}+1}{2}\right)  e \equiv \frac{3^n+1}{2} \pmod{3^n-1}
\]
because $e$ is even, and an even multiple of $(3^n-1)/2$ is $0$ modulo $3^n-1$.
Then since $n-m$ is even (because $n$ is odd and $m$ is odd), we have  $\gcd((3^{n-m}+1)/2,3^n-1)=1$ by \cref{Clementine}.  Therefore so we can invert $(3^{n-m}+1)/2$ to get that $e \equiv ((3^n+1)/2)((3^{n-m}+1)/2)^{-1} \pmod{3^n-1}$.  Since $d$ is equivalent to $e$, then $d$ is equivalent to $((3^n+1)/2)/((3^{n-m}+1)/2)$ with $n-m$ even and $\gcd(n-m,n)=\gcd(m,n)=1$.

Conversely, suppose that $n$ is odd and $d \equiv ((3^n+1)/2)((3^m+1)/2)^{-1} \pmod{3^n-1}$ where $m$ is even and $\gcd(m,n)=1$.  Then $d$ must be even because the numerator, $(3^n+1)/2$, is even by \cref{Francis} and the inverse of the denominator, $((3^m+1)/2)^{-1}$, must be odd since every invertible element modulo $3^n-1$ is relatively prime to $3^n-1$.  Now write $m=\lambda n + m'$ where $\lambda, m' \in \Z$ with $0 \leq m'< n$.  Note that $\gcd(m',n)=\gcd(m,n)=1$.  Then
\begin{align*}
\left(\frac{3^m+1}{2}\right)d - \left(\frac{3^{m'}+1}{2}\right)d
&= \left(\frac{3^m-3^{m'}}{2}\right)d\\
&= \left(\frac{3^{m'}(3^{\lambda n}-1)}{2}\right)d\\
&= 3^{m'}\left(\frac{d}{2}\right)\left((3^n)^\lambda-1\right)\\
&\equiv 3^{m'}\left(\frac{d}{2}\right)\left(1^{\lambda}-1\right) \pmod {3^n-1}\\
&\equiv 0 \pmod{3^n-1},
\end{align*}
where the first congruence uses the fact that $d/2$ is an integer since $d$ is even.
Therefore
\begin{align*}
\left(\frac{3^{m'}+1}{2}\right)d &\equiv \left(\frac{3^{m}+1}{2}\right)d \pmod{3^n-1}\\
& \equiv \left(\frac{3^{m}+1}{2}\right)\left(\frac{3^n+1}{2}\right) \left(\frac{3^m+1}{2}\right)^{-1} \pmod{3^n-1}\\
& \equiv \left(\frac{3^n+1}{2}\right) \pmod{3^n-1}.
\end{align*}
So there is some $\ell \in \Z$ such that
\[
\left(\frac{3^{m'}+1}{2}\right) d = \frac{3^n+1}{2} + \ell(3^n-1),
\]
and so
\begin{equation}\label{Mary}
(3^{m'}+1) d -2 =(2 \ell+1)(3^n-1)
\end{equation}
with $\gcd({m'},n)=1$.

If $m'<n/2$, let $d'=d$ so that $d'$ is equivalent to $d$.  Then $d'$ is an even exponent that satisfies \eqref{Mary} with $\gcd(m',n)=1$ and $2m'<n$.

On the other hand, if $m'>n/2$, let $m''=n-m'$ and note that $0<m''<n/2$ and $\gcd(m'',n)=\gcd(n-m',n)=\gcd(m',n)=1$.  Then \eqref{Mary} becomes
\[
(3^{m'}+3^n-(3^n-1)) d -2 =(2 \ell+1)(3^n-1),
\]
which becomes
\[
(3^{m'}+3^n) d -2 =(2 \ell+d+1)(3^n-1).
\]
Let $k=2 \ell +d+1$, which is odd because $d$ is even.  Also let $d'=3^{m'}d$, which is even because $d$ is even.  Note that $d'$ is equivalent to $d$.  Then our last equation becomes
\begin{equation}\label{Jerry}
(3^{m''}+1) d' -2 =k(3^n-1).
\end{equation}
Then $d'$ is an even exponent that satisfies \eqref{Jerry} with $k$ odd, $\gcd(m'',n)=1$ and $2m''<n$.
\end{proof}

\section{Fibers of the discrete derivative}

\begin{lemma}\label{Xavier}
Let $F$ be a finite field and let $\pi\colon F \to F$ with $\pi(x)=x^{\cardF-2}+1$.
Then $\pi$ is a permutation of $F$ with $\pi(0)=1$ and $\pi(x)=x^{-1}+1$ for $x\not=0$.
\end{lemma}
\begin{proof}
The values of $\pi(0)$ and $\pi(x)$ for nonzero $x$ are clear, and since $x^{-1}+1$ can never equal $1$ and $x^{-1}+1=y^{-1}+1$ if and only if $x=y$, we see that $\pi$ is an injective function from a finite set into itself, hence a permutation.
\end{proof}
\begin{lemma}\label{Aaron}
Let $F$ be a finite field, $d$ a positive integer, and $f\colon F \to F$ the power map $f(x)=x^d$.
Let $\pi\colon F \to F$ be defined by $\pi(x)=x^{\cardF-2}+1$.
Let $f_1 \colon F \to F$ be defined by $f_1(1)=1$ and $f_1(x)=(x^d-1)/(x-1)^d$ for $x\not=1$.
Then $\Delta f=f_1\circ \pi$.
\end{lemma}
\begin{proof}
We have $(f_1\circ\pi)(0)=f_1(1)=1=f(1)-f(0)=(\Delta f)(0)$.
For $x \in \Fu$, by \cref{Xavier} we have
\begin{align*}
(f_1\circ \pi)(x)
& = \frac{(x^{-1}+1)^d-1}{((x^{-1}+1)-1)^d} \\
& = \frac{(x^{-1}+1)^d-1}{x^{-d}} \\
& = (x+1)^d-x^d \\
& = (\Delta f)(x). \qedhere
\end{align*}
\end{proof}

\begin{lemma}\label{Beatrice}
Let $F$ be a finite field and let $d_1$ and $d_2$ be integers with $\gcd(d_2,\cardFu)=1$.
Let $f_1 \colon F \to F$ be defined by $f_1(1)=1$ and $f_1(x)=(x^d-1)/(x-1)^d$ for $x\not=1$, where $d$ is the exponent $d_1/d_2$ over $F$.
Let $\sigma\colon F \to F$ with $\sigma(x)=x^{d_2}$.
Then $\sigma$ is a permutation of $F$.
Let $f_2\colon F \to F$ with $f_2(1)=1$ and
\[
f_2(x)=\frac{(x^{d_1}-1)^{d_2}}{(x^{d_2}-1)^{d_1}}
\]
for $x\not=1$.
Then $f_2$ is a defined function from $F$ to $F$ with $f_2=\sigma \circ f_1 \circ \sigma$.
\end{lemma}
\begin{proof}
Notice that the denominator in the definition of $f_2$ is $(\sigma(x)-1)^{d_1}$, which is zero if and only if $\sigma(x)=1$, which is true if and only if $x=1$ because $\sigma$ is a permutation by \cref{Herbert} since we assumed that $\gcd(d_2,\cardFu)=1$.  This makes $f_2$ defined.
Notice that $(\sigma\circ f_1\circ\sigma)(1)=(\sigma\circ f_1)(1)=\sigma(1)=1=f_2(1)$.
On the other hand if $x \in F$ with $x\not=1$, then
\begin{align*}
(\sigma \circ f_1 \circ \sigma)(x)
& = \left(\frac{x^{d_2 d}-1}{(x^{d_2}-1)^d}\right)^{d_2} \\
& = \frac{(x^{d_1}-1)^{d_2}}{(x^{d_2}-1)^{d_1}} \\
& = f_2(x).\qedhere
\end{align*}
\end{proof}
\begin{lemma}\label{Celia}
Let $F$ be a finite field and let $d_1$ and $d_2$ be integers with $\gcd(d_2,\cardFu)=1$.
Let $f_2\colon F \to F$ be the function from \cref{Beatrice} with $f_2(1)=1$ and
\[
f_2(x)=\frac{(x^{d_1}-1)^{d_2}}{(x^{d_2}-1)^{d_1}}
\]
for $x\not=1$.
Let $\pi\colon F \to F$ be defined by $\pi(x)=x^{\cardF-2}+1$.
Let $f_3\colon F \to F$ be given by 
\[
f_3(x)=\frac{((x+1)^{d_1}-x^{d_1})^{d_2}}{((x+1)^{d_2}-x^{d_2})^{d_1}}.
\]
Then $f_3$ is a defined function from $F$ to $F$ with $f_3 =f_2 \circ \pi$.
\end{lemma}
\begin{proof}
Note that $x\mapsto x^{d_2}$ is a permutation of $F$ by \cref{Herbert} since $\gcd(d_2,\cardFu)=1$.  Thus, the denominator in the definition of $f_3$ is nonzero for every $x \in F$, making $f_3$ a defined function.
We have $(f_2 \circ \pi)(0)=f_2(1)=1=f_3(0)$.
If $x \in \Fu$, then by \cref{Xavier} we have
\begin{align*}
(f_2 \circ \pi)(x)
& = \frac{((x^{-1}+1)^{d_1}-1)^{d_2}}{((x^{-1}+1)^{d_2}-1)^{d_1}} \\
& = \frac{((x+1)^{d_1}-x^{d_1})^{d_2}}{((x+1)^{d_2}-x^{d_2})^{d_1}} \\
& = f_3(x). \qedhere
\end{align*}
\end{proof}

\begin{lemma}\label{Dorothy}
Let $F$ be a finite field of odd characteristic and let $d_1$ and $d_2$ be integers with $\gcd(d_2,\cardFu)=1$.
Let $f_3\colon F \to F$ be the function from \cref{Celia} with
\[
f_3(x)=\frac{((x+1)^{d_1}-x^{d_1})^{d_2}}{((x+1)^{d_2}-x^{d_2})^{d_1}}.
\]
Let $\tau \colon F \to F$ with $\tau(x)=(x-2)/4$.
Let $f_4 \colon F \to F$ with
\[
f_4(x)=\frac{((x+2)^{d_1}-(x-2)^{d_1})^{d_2}}{((x+2)^{d_2}-(x-2)^{d_2})^{d_1}}.
\]
Then $f_4$ is a defined function from $F$ to $F$ with $f_4=f_3 \circ \tau$.
\end{lemma}
\begin{proof}
Note that $x\mapsto x^{d_2}$ is a permutation of $F$ by \cref{Herbert} since $\gcd(d_2,\cardFu)=1$.  Thus, the denominator in the definition of $f_4$ is nonzero for every $x \in F$ because $x+2\not=x-2$ in $F$ since $F$ is of odd characteristic.  This makes $f_4$ a defined function.
Note that for any $x \in F$, we have
\begin{align*}
(f_3 \circ \tau)(x)
& = f_3((x-2)/4) \\
& = \frac{\left(((x+2)/4)^{d_1}-((x-2)/4)^{d_1}\right)^{d_2}}{\left(((x+2)/4)^{d_2}-((x-2)/4)^{d_2}\right)^{d_1}} \\
& = \frac{((x+2)^{d_1}-(x-2)^{d_1})^{d_2}}{((x+2)^{d_2}-(x-2)^{d_2})^{d_1}} \\
& = f_4(x). \qedhere
\end{align*}
\end{proof}
\begin{remark}
The maps $\pi$, $\sigma$, and $\tau$ that appear in Lemmas \ref{Aaron}--\ref{Dorothy} are all permutations, so that by Lemmas \ref{Leonid} and \ref{Fedor}, the functions $\Delta f$, $f_1$, $f_2$, $f_3$, and $f_4$ from Lemmas \ref{Aaron}--\ref{Dorothy} all have the same multiset of cardinalities of fiber sizes.  Thus, the differential spectrum of $\Delta f$ (which is the discrete derivative of a power function) can be determined by examining $f_1$, $f_2$, $f_3$, or $f_4$.
\end{remark}

\section{A pair of parameterizations that doubly cover the finite field}

\begin{lemma}\label{Simon}
Let $K$ be a field and let $x,y \in K^*$.
Then $x+x^{-1}=y+y^{-1}$ if and only if $x \in \{y,y^{-1}\}$.
\end{lemma}
\begin{proof}
The ``if'' direction is clear.
On the other hand, if $x+x^{-1}=y+y^{-1}$, then we have
\[
x-y=y^{-1}-x^{-1}=\frac{x-y}{x y},
\]
and so
\[
(x-y)\left(1-\frac{1}{x y}\right)=0.
\]
Since $K$ is a field, this implies that either $x-y=0$ or $1-1/(x y)=0$.
\end{proof}
\begin{lemma}\label{Elaine}
Let $F$ be a finite field of odd characteristic.
Let $\mu\colon \acFu \to \acF$ with $\mu(x)=x+x^{-1}$.
Let $x \in F$ and let $y$ be a square root of $x^2-4$ in $\acF$.
Then $x\not\in\{y,-y\}$ and $2/(x+y)=(x-y)/2$ and $\mu^{-1}(\{x\}) = \{(x+y)/2,2/(x+y)\}=\{(x+y)/2,(x-y)/2\}$.
\end{lemma}
\begin{proof}
Let $\theta=(x+y)/2$.
We note that $x\not\in\{y,-y\}$, because that would make $x^2=y^2=x^2-4$, which makes $0=-4$, which is absurd in a field of odd characteristic.
So $\theta\not=0$.
Thus, we can write $\theta^{-1}=2/(x+y)=2(x-y)/(x^2-y^2)=2(x-y)/4=(x-y)/2$.
Now note that $\theta+\theta^{-1}=(x+y)/2+(x-y)/2=x$, so $\theta \in \mu^{-1}(\{x\})$, and so by \cref{Simon}, $\mu^{-1}(\{x\})=\{\theta,\theta^{-1}\}$.
\end{proof}

\begin{lemma}\label{Eugene}
Let $F$ be a finite field of odd characteristic, let $E$ be the quadratic extension of $F$, and let $U_E$ be the unit circle of $E$.
Let \begin{align*}
H & = \{c \in F\smallsetminus\{2,-2\}: c^2-4 \text{ is a quadratic residue}\}\\
I & = \{c \in F\smallsetminus\{2,-2\}: c^2-4 \text{ is a quadratic nonresidue}\}.
\end{align*}
Let $\mu\colon \acFu \to \acF$ with $\mu(x)=x+x^{-1}$.
Then
\begin{enumerate}[label=(\roman*)]
\item We have $\mu^{-1}(\{2\}) = \{1\}$ and $\mu(\{1\})=\{2\}$.
\item We have $\mu^{-1}(\{-2\}) = \{-1\}$ and $\mu(\{-1\})=\{-2\}$.
\item If $c\in H$, then $\mu^{-1}(\{c\})$ is a set of cardinality $2$ contained in $\Fu\smallsetminus\{1,-1\}$.  Furthermore, $\mu^{-1}(H)=\Fu\smallsetminus\{1,-1\}$ and $\mu(\Fu\smallsetminus \{1,-1\})=H$.
\item If $c \in I$, then $\mu^{-1}(\{c\})$ is a set of cardinality $2$ contained in $U_E\smallsetminus\{1,-1\}$.  Furthermore, $\mu^{-1}(I)=U_E\smallsetminus\{1,-1\}$ and $\mu(U_E\smallsetminus \{1,-1\})=I$.
\item We have $\mu(\Fu\cup U_E)=F$ and $\mu^{-1}(F)=\Fu\cup U_E$.
\end{enumerate}
\end{lemma}
\begin{proof}
The first two claims two follow immediately from \cref{Elaine} and easy calculations.
The first sentence of the third claim follows from \cref{Elaine}.
We will prove the rest of the third claim later.
For the first sentence of the fourth claim, let $\bar{v}$ be shorthand for $v^{\card{F}}$, the conjugate of $v$ in $E$.  Suppose that $c^2-4$ does not have a square root in $F$, so then it does have one in $E\smallsetminus F$, and so by \cref{Elaine}, we have some $\eta \in E\smallsetminus F$ such that $\mu^{-1}(\{c\})=\{\eta,\eta^{-1}\}$, a set of cardinality $2$.
Notice that $c=\bar{c}$ since $c \in F=H_E$.
So $c=\bar{c}=\bar{\eta}+\bar{\eta}^{-1}$, and so $\bar{\eta} \in \mu^{-1}(\{c\})=\{\eta,\eta^{-1}\}$.
But we cannot have $\bar{\eta}=\eta$, because that would force $\eta \in H_E=F$.
So $\bar{\eta} = \eta^{-1}$, which means that $\eta \in U_E$.
Thus, $\eta^{-1} \in U_E$ since $U_E$ is a subgroup of $E^*$, and so $\mu^{-1}(\{c\}) \subseteq U_E$, and furthermore $1$ and $-1$ cannot be in $\mu^{-1}(\{c\})$, since that would force $c \in \{2,-2\}$, making $c^2-4=0$, which is not a quadratic residue in $F$.  This finishes the proof of the first sentence of the fourth claim.

For the final claim, note that what we have proved so far establishes that $\mu^{-1}(F)= \bigcup_{c \in F} \mu^{-1}(\{c\}) \subseteq \Fu\cup U_E$.
Since fibers of a function over different elements are disjoint and we know the size of each fiber by the previous four claims, we have $|\mu^{-1}(F)|=\sum_{c \in F} |\mu^{-1}(\{c\})|=2|F|-2$, which is exactly equal to $\cardFu+|U_E|-|\Fu\cap U_E|$ (see \cref{Ursula} and \cref{Lenny}).  So $|\mu^{-1}(F)|=|\Fu\cup U_E|$, and so the containment we have already proved becomes the equality $\mu^{-1}(F)=\Fu\cup U_E$.  This implies $\mu(\Fu\cup U_E) \subseteq F$.  What we proved in the previous paragraph implies that for every $c \in F$, the fiber $\mu^{-1}(\{c\})$ is a nonempty subset of $\Fu\cup U_E$, so $c \in \mu(\Fu\cup U_E)$, and thus $F\subseteq \mu(\Fu\cup U_E)$.  Thie completes the proof that $\mu(\Fu\cup U_E)=F$.

Now we return to the second sentences of the third and fourth claims.
Let $G=\{2,-2\}$, so that $\{G,H,I\}$ is a partition of $F$.
In view of \cref{Lenny}, we can partition $\Fu\cup U_E$ into three sets: $G'=\{1,-1\}$, $H'=\Fu\smallsetminus\{1,-1\}$, and $I'=U_E\smallsetminus\{1,-1\}$.
The first two claims of this lemma along with the first sentences of the third and fourth claims show that $\mu^{-1}(G)\subseteq G'$, $\mu^{-1}(H) \subseteq H'$, and $\mu^{-1}(I)\subseteq I'$.  If any of these three containments were proper, then we would have $\mu^{-1}(F)=\mu^{-1}(G)\cup \mu^{-1}(H) \cup \mu^{-1}(I)$ strictly contained in $G' \cup H' \cup I'=\Fu\cup U_E$, contradicting the fifth claim, which we have already proved.
Thus, $\mu^{-1}(G)=G'$, $\mu^{-1}(H)=H'$ and $\mu^{-1}(I)=I'$.
This in turn proves that $\mu(G') \subseteq G$, $\mu(H')\subseteq H$ and $\mu(I')\subseteq I$.
If any of these three containments were proper, then we would have $\mu(\Fu\cup U_E)=\mu(G')\cup \mu(H') \cup \mu(I')$ strictly contained in $G \cup H \cup I=F$, contradicting the fifth claim, which we have already proved.
\end{proof}

\begin{proposition}\label{Eusebius}
Let $F$ be a finite field of odd characteristic, let $E$ be the quadratic extension of $F$, and let $U_E$ be the unit cirlce of $E$.
Let $\kappa \colon \Fu \to F$ with $\kappa(x)=x+x^{-1}$ and $\lambda \colon U_E \to F$ with $\lambda(x)=x+x^{-1}$ (this map is defined by \cref{Eugene}).
Let $\phi \colon F \to F$, let $\chi=\phi\circ\kappa$ and let $\psi=\phi\circ\lambda$.
For each $c \in F$, we have
\[
|\phi^{-1}(\{c\})|=\frac{|\chi^{-1}(\{c\})|+|\psi^{-1}(\{c\})|}{2}.
\]
\end{proposition}
\begin{proof}
We let 
\begin{align*}
G & = \{2,-2\} \\
H & = \{c \in F\smallsetminus\{2,-2\}: c^2-4 \text{ is a quadratic residue}\}\\
I & = \{c \in F\smallsetminus\{2,-2\}: c^2-4 \text{ is a quadratic nonresidue}\},
\end{align*}
so that $\{G,H,I\}$ is a partition of $F$.
Let $\mu \colon \acFu \to \acF$ with $\mu(x)=x+x^{-1}$, which makes $\kappa$ and $\lambda$ restrictions of $\mu$, concerning which \cref{Lenny} and \cref{Eugene} reveal that
\begin{align}
\begin{split}\label{Samuel}
\kappa^{-1}(G) = \lambda^{-1}(G) = \mu^{-1}(G) & = \{-1,1\} \\
\kappa^{-1}(H) = \mu^{-1}(H) & = \Fu\smallsetminus\{1,-1\} \\
\lambda^{-1}(H) & = \emptyset \\
\lambda^{-1}(I) = \mu^{-1}(I) & = U_E \smallsetminus\{1,-1\} \\
\kappa^{-1}(I) & = \emptyset.
\end{split}
\end{align}
Let $c \in F$, let $\Phi=\phi^{-1}(\{c\})$, let $X=\chi^{-1}(\{c\})$, and let $\Psi=\psi^{-1}(\{c\})$.
Since $\Phi \subseteq F$, we know that $\Phi$ is the disjoint union of $\Phi \cap G$, $\Phi \cap H$, and $\Phi \cap I$.
Now
\begin{align*}
X
& =\chi^{-1}(\{c\}) \\
& =\kappa^{-1}(\phi^{-1}(\{c\})) \\
& =\kappa^{-1}(\Phi) \\
& =\kappa^{-1}((\Phi\cap G)\sqcup(\Phi\cap H)\sqcup(\Phi \cap I)) \\
& =\kappa^{-1}(\Phi\cap G)\sqcup \kappa^{-1}(\Phi\cap H)\sqcup\kappa^{-1}(\Phi \cap I) \\
& = \mu^{-1}(\Phi\cap G)\sqcup \mu^{-1}(\Phi\cap H)\sqcup \emptyset,
\end{align*}
where the last step uses \eqref{Samuel}.
Since \cref{Eugene} shows that fibers of $\mu$ over points in $G$ are of cardinality $1$ and all other fibers of $\mu$ are of cardinality $2$, we have
\begin{align}\label{Charlene}
|X| =|\Phi\cap G|+2|\Phi\cap H|.
\end{align}
Similarly, 
\begin{align*}
\Psi
& =\psi^{-1}(\{c\}) \\
& =\lambda^{-1}(\phi^{-1}(\{c\})) \\
& =\lambda^{-1}(\Phi) \\
& =\lambda^{-1}((\Phi\cap G)\sqcup(\Phi\cap H)\sqcup(\Phi \cap I)) \\
& =\lambda^{-1}(\Phi\cap G)\sqcup \lambda^{-1}(\Phi\cap H)\sqcup\lambda^{-1}(\Phi \cap I) \\
& =\mu^{-1}(\Phi\cap G)\sqcup \emptyset\sqcup\mu^{-1}(\Phi \cap I),
\end{align*}
where the last step uses \eqref{Samuel}, and from what we know about the sizes of the fibers of $\mu$, we obtain
\[
|\Psi| =|\Phi\cap G|+2|\Phi\cap I|,
\]
which we add to \eqref{Charlene} and divide by $2$ to obtain
\[
\frac{|X|+|\Psi|}{2} = |\Phi\cap G|+|\Phi\cap H|+|\Phi\cap I|.
\]
Since $\{G,H,I\}$ is a partition of $F$ and $\Phi\subseteq F$, this gives $(|X|+|\Psi|)/2=|\Phi|$.
\end{proof}

\section{Proof of Theorem \ref{Victoria}}

\begin{lemma}\label{Oscar}
Let $p$ be a prime and let $j$ be a nonnegative integer.
Then the polynomial $(x+1)^{p^j+1}-(x-1)^{p^j+1} \in \F_p[x]$ is equal to $2(x^{p^j}+x)$.
\end{lemma}
\begin{proof}
We have
\begin{align*}
(x+1)^{p^j+1}-(x-1)^{p^j+1}
& = (x+1)^{p^j}(x+1)-(x-1)^{p^j}(x-1) \\
& = (x^{p^j}+1)(x+1)-(x^{p^j}-1)(x-1) \\
& = (x^{p^j+1}+x^{p^j}+x+1)-(x^{p^j+1}-x^{p^j}-x+1)\\
& = 2 (x^{p^j}+x).\qedhere
\end{align*}
\end{proof}
\begin{lemma}\label{Rita}
Let $p$ be a prime and let $j$ be a nonnegative integer.
Then the Laurent polynomial $(x+x^{-1}+2)^{(p^j+1)/2}-(x+x^{-1}-2)^{(p^j+1)/2} \in \F_p[x,x^{-1}]$ is equal to $2 x^{-(p^j+1)/2} (x^{p^j}+x)$.
\end{lemma}
\begin{proof}
We note that $(x+x^{-1}+2)^{(p^j+1)/2}-(x+x^{-1}-2)^{(p^j+1)/2}$ is equal to 
\begin{align*}
\frac{(x^2+2 x+1)^{(p^j+1)/2}-(x^2-2 x+1)^{(p^j+1)/2}}{x^{(p^j+1)/2}} 
& = \frac{(x+1)^{p^j+1}-(x-1)^{p^j+1}}{x^{(p^j+1)/2}} \\
& = \frac{2(x^{p^j}+x)}{x^{(p^j+1)/2}},
\end{align*}
where the last equality uses \cref{Oscar}.
\end{proof}

\begin{lemma}\label{Katherine}
Let $F$ be a finite field of odd characteristic $p$, let $j$ and $k$ be nonnegative integers, let $d_1=(p^j+1)/2$ and $d_2=(p^k+1)/2$, and suppose that $\gcd(d_2,\cardFu)=1$.
Let $f_4 \colon F \to F$ with
\[
f_4(x)=\frac{((x+2)^{d_1}-(x-2)^{d_1})^{d_2}}{((x+2)^{d_2}-(x-2)^{d_2})^{d_1}},
\]
from \cref{Dorothy}.
Let $\mu\colon \acFu \to \acF$ with $\mu(x)=x+x^{-1}$.
Let $E$ be the quadratic extension of $F$ and let $U_E$ be the unit circle of $E$.
If $x \in \Fu\cup U_E$, then $x^{p^k}+x\not=0$ and 
\[
(f_4\circ\mu)(x) = \sigma \frac{(x^{p^j}+x)^{d_2}}{(x^{p^k}+x)^{d_1}}
\]
where $\sigma=1$ if $j$ and $k$ have the same parity, if $p \equiv 1 \pmod{8}$, or if $p\equiv 7 \pmod{8}$; otherwise $\sigma=-1$.
\end{lemma}
\begin{proof}
Let $x \in \Fu\cup U_E$.
We know that $(f_4\circ \mu)(x)$ is defined because $\mu(\Fu\cup U_E)=F$ by \cref{Eugene} and $\gcd(d_2,\cardFu)=1$ makes $f_4$ defined on all of $F$ by \cref{Dorothy}.  We have
\begin{align*}
(f_4\circ\mu)(x)
& = f_4(x+x^{-1}) \\
& = \frac{((x+x^{-1}+2)^{d_1}-(x+x^{-1}-2)^{d_1})^{d_2}}{((x+x^{-1}+2)^{d_2}-(x+x^{-1}-2)^{d_2})^{d_1}},
\end{align*}
and we know the denominator does not vanish since $x+x^{-1} \in F$ and $x\mapsto x^{d_2}$ is a permutation of $F$ by \cref{Herbert} because $\gcd(d_2,\cardFu)=1$.
\cref{Rita} shows that the denominator is equal to $2^{d_1} x^{-d_1 d_2} (x^{p^k}+x)^{d_1}$, so we know that $x^{p^k}+x\not=0$.
\cref{Rita} also shows that the numerator is equal to $2^{d_2} x^{-d_1 d_2} (x^{p^j}+x)^{d_2}$, and so
\[
(f_4\circ\mu)(x) = 2^{d_2-d_1} \frac{(x^{p^j}+x)^{d_2}}{(x^{p^k}+x)^{d_1}}.
\]
Notice that $d_2-d_1 = (p^k-p^j)/2=(p^k-1)/2-(p^j-1)/2$, which by \cref{Molly} is congruent to $(k-j)(p-1)/2$ modulo $p-1$.
Thus, if $j$ and $k$ have the same parity, then $2^{d_2-d_1}=1$ by Fermat's little theorem.
So assume $j$ and $k$ are of opposite parity henceforth, and then $2^{d_2-d_1} = 2^{(p-1)/2}$, which is $1$ if $2$ is a quadratic residue in $\Fpu$ or $-1$ if $2$ is a quadratic nonresidue in $\Fpu$.
By a supplement to quadratic reciprocity, $2$ is a quadratic residue in $\Fpu$ if $p\equiv 1$ or $7 \pmod{8}$, but is a quadratic nonresidue if $p\equiv 3$ or $5 \pmod{8}$.
\end{proof}
\begin{lemma}\label{Thomas}
Let $F$ be a finite field of odd characteristic $p$, let $j$ and $k$ be nonnegative integers, let $d_1=(p^j+1)/2$ and $d_2=(p^k+1)/2$, and suppose that $\gcd(d_2,\cardFu)=1$.
Let $f_4 \colon F \to F$ with
\[
f_4(x)=\frac{((x+2)^{d_1}-(x-2)^{d_1})^{d_2}}{((x+2)^{d_2}-(x-2)^{d_2})^{d_1}},
\]
from \cref{Dorothy}.
Let $\mu\colon \acFu \to \acF$ with $\mu(x)=x+x^{-1}$.
Let $E$ be the quadratic extension of $F$ and let $U_E$ be the unit circle of $E$.
Let $\tau=-1$ if $p\equiv 3\pmod{4}$ and $j$ is odd; otherwise let $\tau=1$.
If $x \in \Fu\cup U_E$, then 
\begin{align*}
(f_4\circ\mu)(1/x) & = (f_4\circ\mu)(x) \\
(f_4\circ\mu)(-x) & = \tau(f_4\circ\mu)(x) \\
(f_4\circ\mu)(-1/x) & = \tau(f_4\circ\mu)(x).
\end{align*}
\end{lemma}
\begin{proof}
Let $\sigma$ be as defined in \cref{Katherine}.
Let $x \in \Fu\cup U_E$.
Since $\mu(1/x)=\mu(x)$, it is clear that $(f_4\circ\mu)(1/x) = (f_4\circ\mu)(x)$.
We have $-x \in \Fu\cup U_E$ since $\Fu$ and $U_E$ are groups under multiplication and contain $-1$.
Then \cref{Katherine} tells us that
\begin{align*}
(f_4\circ\mu)(-x)
& = \sigma \frac{((-x)^{p^j}+(-x))^{d_2}}{((-x)^{p^k}+(-x))^{d_1}} \\
& = (-1)^{d_2-d_1} \sigma \frac{(x^{p^j}+x)^{d_2}}{(x^{p^k}+x)^{d_1}} \\
& = (-1)^{d_2+d_1} (f_4\circ\mu)(x) \\
& = (-1)^{1+d_1} (f_4\circ\mu)(x),
\end{align*}
where the last equality uses the fact that $d_2$ is odd because $\gcd(d_2,\cardFu)=1$.
By Lemmas \ref{Eric} and \ref{Francis}, we know that $d_1=(p^j+1)/2$ is even if $p\equiv 3 \pmod{4}$ and $j$ is odd; otherwise $d_1$ is odd.
Thus, we see that $(-1)^{1+d_1}=\tau$, and thus $(f_4\circ\mu)(-x)=\tau (f_4\circ\mu)(x)$.  The final relation follows from the two that we have already proved.
\end{proof}
\begin{lemma}\label{Irene}
Let $F$ be a finite field of order $q=3^n$ with $n$ odd, let $k$ be an even nonnegative integer, let $d_1=(3^n+1)/2$ and $d_2=(3^k+1)/2$.
Let $f_4 \colon F \to F$ with
\[
f_4(x)=\frac{((x+2)^{d_1}-(x-2)^{d_1})^{d_2}}{((x+2)^{d_2}-(x-2)^{d_2})^{d_1}},
\]
from \cref{Dorothy}.
Let  $\kappa\colon \Fu \to F$ with $\kappa(x)=x+x^{-1}$ and $e_2=(p^k-1)/2$.
For $x \in \Fu$, we have
\[
(f_4\circ\kappa)(x)=\left(\frac{x}{x^{e_2}+x^{-e_2}}\right)^{(q-1)/2} \cdot \frac{1}{x^{e_2}+x^{-e_2}},
\]
which is nonzero, and the term $\left(x/(x^{e_2}+x^{-e_2})\right)^{(q-1)/2} \in \{1,-1\}$.
\end{lemma}
\begin{proof}
First of all, note that $\gcd(d_2,\cardFu)=1$ by \cref{Clementine}, which makes $d_2$ odd and makes $f_4$ is defined by \cref{Dorothy}.  
Let $x \in \Fu$.  According to \cref{Katherine},
\[
(f_4\circ\kappa)(x) = -\frac{(x^q+x)^{d_2}}{(x^{3^k}+x)^{d_1}}.
\]
Since $x\in \Fu$, we have $x^q=x$ for all $x$ in the domain.  So
\begin{align*}
(f_4\circ\kappa)(x)
& = -\frac{(x+x)^{d_2}}{(x^{3^k}+x)^{d_1}} \\
& = -\frac{(2 x)^{d_2}}{(x^{3^k}+x)^{d_1}} \\
& = -\frac{(-x)^{d_2}}{(x^{3^k}+x)^{d_1}} \\
& = \frac{x^{d_2}}{(x^{3^k}+x)^{d_1}} && \text{because $d_2$ is odd} \\
& = \frac{x^{d_2}}{(x^{d_2+e_2}+x^{d_2-e_2})^{d_1}} \\
& = \frac{x^{d_2}}{x^{d_1 d_2} (x^{e_2}+x^{-e_2})^{d_1}} \\
& = \frac{x^{(1-d_1) d_2}}{(x^{e_2}+x^{-e_2})^{d_1}} \\
& = \frac{x^{-d_2(q-1)/2}}{(x^{e_2}+x^{-e_2})^{(q+1)/2}}.
\end{align*}
Since $x^{q-1}=1$ and $d_2$ is odd, we have $x^{-d_2(q-1)/2}=x^{(q-1)/2}$, so
\begin{align*}
(f_4\circ\kappa)(x)
& = \frac{x^{(q-1)/2}}{(x^{e_2}+x^{-e_2})^{(q+1)/2}}\\
& = \left(\frac{x}{x^{e_2}+x^{-e_2}}\right)^{(q-1)/2} \cdot \frac{1}{x^{e_2}+x^{-e_2}}.
\end{align*}
For $x \in \Fu$ the term $\left(x/(x^{e_2}+x^{-e_2})\right)^{(q-1)/2}$ is an element of $\Fu$ raised to the $(q-1)/2$ power, so it is in $\{1,-1\}$, and the term $1/(x^{e_2}+x^{-e_2})$ cannot be zero, so $(f_4\circ\kappa)(x)\not=0$.
\end{proof}
\begin{corollary}\label{Augustine}
Let $F$ be a finite field of order $q=3^n$ with $n$ odd, let $k$ be an even nonnegative integer, let $d_1=(3^n+1)/2$ and $d_2=(3^k+1)/2$.
Let $f_4 \colon F \to F$ with
\[
f_4(x)=\frac{((x+2)^{d_1}-(x-2)^{d_1})^{d_2}}{((x+2)^{d_2}-(x-2)^{d_2})^{d_1}},
\]
from \cref{Dorothy}.
Let  $\kappa\colon \Fu \to F$ with $\kappa(x)=x+x^{-1}$ and $e_2=(p^k-1)/2$.
For $x \in \Fu$, we have
\[
\left((f_4\circ\kappa)(x)\right)^2=\frac{1}{\left(x^{e_2}+x^{-e_2}\right)^2},
\]
which is nonzero.
\end{corollary}
\begin{lemma}\label{William}
Let $F$ be a finite field of order $q=3^n$ with $n$ odd, let $k$ be an even nonnegative integer with $\gcd(n,k)=1$, let $d_1=(3^n+1)/2$, and let $d_2=(3^k+1)/2$.
Let $f_4 \colon F \to F$ with
\[
f_4(x)=\frac{((x+2)^{d_1}-(x-2)^{d_1})^{d_2}}{((x+2)^{d_2}-(x-2)^{d_2})^{d_1}},
\]
from \cref{Dorothy}.
Let  $\kappa\colon \Fu \to F$ with $\kappa(x)=x+x^{-1}$.
For $a \in \Fu$, the fiber of $f_4\circ\kappa$ that contains $a$ is $\{a,1/a\}$.
\end{lemma}
\begin{proof}
Let $a,b \in \Fu$ and suppose that $b$ is in the same fiber of $f_4\circ\kappa$ as $a$.
Then $(f_4\circ\kappa(b))^2=((f_4\circ\kappa)(a))^2$, so that by \cref{Augustine} we have $1/(b^{e_2}+b^{-e_2})^2 = 1/(a^{e_2}+a^{-e_2})^2$ where $e_2=(p^k-1)/2$.
Thus, $b^{e_2}+b^{-e_2} = \sigma (a^{e_2}+a^{-e_2})=(\sigma a^{e_2})+(\sigma a^{e_2})^{-1} $ for some $\sigma \in \{1,-1\}$.

If $\sigma=-1$, then by \cref{Simon}, we have $b^{e_2} \in \{-a^{e_2},-a^{-e_2}\}$, i.e., either $(b/a)^{e_2}=-1$ or else $(b a)^{e_2}=-1$.
Since $e_2$ is even by \cref{Francis}, this makes $-1$ a square in $\Fu$, contradicting \cref{Quentin}.

Thus, we must have $\sigma=1$, and so \cref{Simon} shows that $b^{e_2} \in \{a^{e_2},a^{-e_2}\}$.
So \cref{Yolanda} implies that $b \in \{a,-a,1/a,-1/a\}$.
\cref{Thomas} shows that if $b$ were in $\{-a,-1/a\}$, then $(f_4\circ\kappa)(b)=-(f_4\circ\kappa)(a)$, which is not equal to $(f_4\circ\kappa)(a)$ (because outputs of $f_4\circ\lambda$ are nonzero by \cref{Irene}); this would contradict our assumption that $b$ and $a$ are in the same fiber of $f_4\circ\kappa$.
Thus, we conclude that $b \in \{a,1/a\}$.
This shows that the fiber containing $a$ is a subset of $\{a,1/a\}$.
And in fact, the fiber containing $a$ must contain $1/a$ by \cref{Thomas}, so the fiber containing $a$ is $\{a,1/a\}$.
\end{proof}
\begin{proposition}\label{Irving}
Let $F$ be a finite field of order $q=3^n$ with $n$ odd, let $k$ be an even nonnegative integer with $\gcd(n,k)=1$, let $d_1=(3^n+1)/2$, and let $d_2=(3^k+1)/2$.
Let $f_4 \colon F \to F$ with
\[
f_4(x)=\frac{((x+2)^{d_1}-(x-2)^{d_1})^{d_2}}{((x+2)^{d_2}-(x-2)^{d_2})^{d_1}},
\]
from \cref{Dorothy}.
Let  $\kappa\colon \Fu \to F$ with $\kappa(x)=x+x^{-1}$.
Let $\eta\colon F \to \C$ be the extended quadratic character of $F$.
For $c \in \F$, we have
\[
\card{(f_4\circ\kappa)^{-1}(\{c\})}=
\begin{cases}
0 & \text{if $c=0$,}\\
1+\eta(1-c^2) & \text{otherwise}.
\end{cases}
\]
\end{proposition}
\begin{proof}
By \cref{Irene}, we know that $(f_4\circ\kappa)^{-1}(\{0\})=\emptyset$.
Let $c \in \Fu$.  Since $(f_4\circ\kappa)(-x)=-(f_4\circ\kappa)(x)$ for all $x \in \Fu$ by \cref{Thomas}, we know that the fiber over $c$ is nonempty if and only if the fiber over $-c$ is nonempty.
Thus, the fiber over $c$ is nonempty if and only if there is some $a \in \Fu$ such that $((f_4\circ\kappa)(a))^2=c^2$.
By \cref{Augustine}, this happens if and only if there is some $a \in \Fu$ such that $(a^{e_2}+a^{-e_2})^{-2}=c^2$, which happens if and only if there is some $a \in \Fu$ such that $a^{e_2}+a^{-e_2} \in \{c^{-1},-c^{-1}\}$.
This happens if and only if there is some $a\in\Fu$ such that at least one of the quadratic polynomials $x^2-c^{-1} x + 1$ and $x^2+c^{-1} x+1$ has a root at $a^{e_2}$.
Since both these polynomials have discriminant $c^{-2}-4=c^{-2}-1$, they each have $1+\eta(c^{-2}-1)=1+\eta(1-c^2)$ roots in $F$.
So if $1+\eta(1-c^2)=0$, then the fiber over $c$ is empty.
By \cref{Muriel}, this means that there are at least $(p^n-3)/2$ empty fibers over elements $c \in \Fu$.
Along with the empty fiber over $0$, this means that there are at least $(p^n-1)/2$ empty fibers, and so there are at most $(p^n+1)/2$ nonempty fibers.
The nonempty fibers form a partition of the domain $\Fu$ of $\kappa$, and \cref{William} shows us that $\{1\}$ and $\{-1\}$ are fibers and that all other nonempty fibers are of cardinality $2$.  Thus, if there were strictly fewer than $(p^n+1)/2$ nonempty fibers, then there would be two fibers of size $1$ and fewer than $(p^n-3)/2$ fibers of size $2$; then the union of all the fibers would have fewer than $2(1)+2((p^n-3)/2)=p^n-1$ elements, which is absurd.
So there must be precisely $(p^n+1)/2$ nonempty fibers: $\{1\}$, $\{-1\}$, and $(p^n-3)/2$ fibers of cardinality $2$.
Using \cref{Irene}, it is not hard to calculate $(f_4\circ\kappa)(1)=1$ and $(f_4\circ\kappa)(-1)=-1$, so we see that the fiber over $c$ is indeed of cardinality $1+\eta(1-c^2)$ when $1+\eta(1-c^2)=1$.
We have already accounted for the fiber over $0$, for the fibers over all $c\in\Fu$ with $1+\eta(1-c^2)=0$ (there are $(p^n-3)/2$ of these and they are empty), and for the fibers over all $c\in\Fu$ with $1+\eta(1-c^2)=1$ (there are $2$ of these and they are of cardinality $1$).  There are $(p^n-3)/2$ remaining values of $c \in \Fu$: by \cref{Muriel} these are all $c\in \Fu$ with $1+\eta(1-c^2)=2$.  The fibers over these $c$ must be the $(p^n-3)/2$ fibers of cardinality $2$, so these fibers also have cardinality equal to $1+\eta(1-c^2)$.
\end{proof}

\begin{lemma}\label{Heloise}
Let $F$ be a finite field of order $q=3^n$ with $n$ odd, let $k$ be an even nonnegative integer, let $d_1=(3^n+1)/2$ and $d_2=(3^k+1)/2$.
Let $f_4 \colon F \to F$ with
\[
f_4(x)=\frac{((x+2)^{d_1}-(x-2)^{d_1})^{d_2}}{((x+2)^{d_2}-(x-2)^{d_2})^{d_1}},
\]
from \cref{Dorothy}.
Let $E$ be the quadratic extension of $F$ and $U_E$ the unit circle of $E$.
Let $\lambda\colon U_E \to F$ with $\lambda(x)=x+x^{-1}$.
For $x \in U_E$, we have
\[
(f_4\circ\lambda)(x) = -\frac{(x+x^{-1})^{d_2}}{(x^{3^k}+x)^{d_1}}.
\]
\end{lemma}
\begin{proof}
Let $x \in U_E$.  According to \cref{Katherine}, we have
\begin{align*}
(f_4\circ\lambda)(x)
&  = -\frac{(x^{3^n}+x)^{d_2}}{(x^{3^k}+x)^{d_1}} \\
& = -\frac{(x^{-1}+x)^{d_2}}{(x^{3^k}+x)^{d_1}},
\end{align*}
where $x^{3^n}=x^{-1}$ because $x \in U_E$.
\end{proof}

\begin{lemma}\label{Julia}
Let $F$ be a finite field of order $q=3^n$ with $n$ odd, let $k$ be an even nonnegative integer, let $d_1=(3^n+1)/2$, and let $d_2=(3^k+1)/2$.
Let $f_4 \colon F \to F$ with
\[
f_4(x)=\frac{((x+2)^{d_1}-(x-2)^{d_1})^{d_2}}{((x+2)^{d_2}-(x-2)^{d_2})^{d_1}},
\]
from \cref{Dorothy}.
Let $E$ be the quadratic extension of $F$ and let $U_E$ be the unit circle of $E$.
Let $\lambda\colon U_E \to F$ with $\lambda(x)=x+x^{-1}$.
For $x \in U_E$, we have $x^{p^k}+x\not=0$ and
\[
\left((f_4\circ\lambda)(x)\right)^2 = 1+ \left(\frac{x^{3^k+1}-1}{x^{3^k}+x}\right)^2.\]
\end{lemma}
\begin{proof}
We use the notation $\tilde{v}=v^{3^k}$ throughout this proof.
Let $x \in U_E$.  Then $x^{3^k}+x\not=0$ by \cref{Katherine}, and by \cref{Heloise} we have
\begin{align*}
((f_4\circ\lambda)(x))^2
& = \frac{(x+x^{-1})^{2 d_2}}{(x^{3^k}+x)^{2 d_1}} \\
& = \frac{(x+x^{-1})^{3^k+1}}{(\tilde{x}+x)^{q+1}} \\
& = \frac{(x+x^{-1}) (x+x^{-1})^{3^k}}{(\tilde{x}+x)(\tilde{x}+x)^q} \\
& = \frac{(x+x^{-1}) (\tilde{x}+\tilde{x}^{-1})}{(\tilde{x}+x)(\tilde{x}^q+x^q)} \\
& = \frac{(x+x^{-1}) (\tilde{x}+\tilde{x}^{-1})}{(\tilde{x}+x)(\tilde{x}^{-1}+x^{-1})},
\end{align*}
where the last equality uses the fact that $x,\tilde{x} \in U_E$, which is a group of order $q+1$.
Multiply both numerator and denominator by the nonzero quantity $x\tilde{x}$ to obtain
\begin{align*}
((f_4\circ\lambda)(x))^2
& = \frac{(x^2+1) (\tilde{x}^2+1)}{(\tilde{x}+x)^2} \\
& = \frac{(x\tilde{x})^2+x^2+\tilde{x}^2+1}{(\tilde{x}+x)^2} \\
& = \frac{(x\tilde{x}-1)^2+2x\tilde{x}+x^2+\tilde{x}^2}{(\tilde{x}+x)^2} \\
& = \frac{(x\tilde{x}-1)^2+(x+\tilde{x})^2}{(\tilde{x}+x)^2} \\
& = 1+ \left(\frac{x\tilde{x}-1}{\tilde{x}+x}\right)^2. \qedhere
\end{align*}
\end{proof}

\begin{lemma}\label{Nicholas}
Let $F$ be a finite field of order $q=3^n$ with $n$ odd and let $k$ be an even nonnegative integer.
Let $E$ be the quadratic extension of $F$ and let $U_E$ be the unit circle of $E$.
If $x \in U_E$, then
\[
\left(\frac{x^{3^k+1}-1}{x^{3^k}+x}\right)^2
\]
is an element of $F$ that is not a quadratic residue in $F$.
\end{lemma}
\begin{proof}
Let $x \in U_E$.
Note that $x^{3^k}+x\not=0$ by \cref{Katherine}.
Let $y=\left(\frac{x^{3^k+1}-1}{x^{3^k}+x}\right)^2$.
If $y=0$, then it is in $F$ but is not a quadratic residue in $F$.
So henceforth assume that $y\not=0$ and we shall show that $y$ is a quadratic nonresidue in $F$.  We calculate
\begin{align*}
y^{(q-1)/2} 
& = \left(\frac{x^{3^k+1}-1}{x^{3^k}+x}\right)^{q-1} \\
& = \left(\frac{x^{3^k+1}-1}{x^{3^k}+x}\right)^q \left(\frac{x^{3^k}+x}{x^{3^k+1}-1}\right) \\
& = \left(\frac{x^{-(3^k+1)}-1}{x^{-3^k}+x^{-1}}\right) \left(\frac{x^{3^k}+x}{x^{3^k+1}-1}\right),
\end{align*}
where we have used the fact that $x^q=x^{-1}$ because $x \in U_E$.
Then we simplify to get
\begin{align*}
y^{(q-1)/2} 
& = \left(\frac{1-x^{3^k+1}}{x+x^{3^k}}\right) \left(\frac{x^{3^k}+x}{x^{3^k+1}-1}\right) \\
& = -1.
\end{align*}
This shows both that $y\in F$ (because $y^{q-1}=1$) and that $y$ is a quadratic nonresidue in $F$.
\end{proof}

\begin{lemma}\label{Uriah}
Let $F$ be a finite field of order $q=3^n$ with $n$ odd and let $k$ be an even nonnegative integer.
Let $E$ be the quadratic extension of $F$ and let $U_E$ be the unit circle of $E$.
Let $x,y \in U_E$ with
\[\frac{x^{p^k+1}-1}{x^{p^k}+x} = \frac{y^{p^k+1}-1}{y^{p^k}+y}.\]
Then $x \in \{y,-1/y\}$.
\end{lemma}
\begin{proof}
We use the notation $\tilde{v}=v^{3^k}$ in this proof.
Assume $x\not\in \{y,-1/y\}$ to show contradiction.
We have
\[
(x\tilde{x}-1) (\tilde{y}+y) = (y\tilde{y}-1) (\tilde{x}+x),
\]
and so
\[
x\tilde{x}\tilde{y} + x\tilde{x}y-\tilde{y}-y = y\tilde{y}\tilde{x} + y\tilde{y}x-\tilde{x}-x.
\]
Thus,
\[
\tilde{x}\tilde{y}(x-y) + (x-y) =-x y(\tilde{x}-\tilde{y}) - (\tilde{x}-\tilde{y})
\]
and so 
\[
(\tilde{x} \tilde{y}+1)(x-y)=-(x y+1) (\tilde{x}-\tilde{y}),
\]
in other words,
\[
(x y+1)^{3^k} (x-y) = - (x y+1) (x-y)^{3^k},
\]
and since we are assuming $x \not\in \{y,-1/y\}$, we obtain
\begin{equation}\label{Wolfgang}
\left(\frac{x y+1}{x- y}\right)^{3^k-1} = -1.
\end{equation}
Let $\beta$ be a primitive element of the quadratic extension $E$ of $F$.
Since $x,y \in U_E \subseteq E$ and since $x\not=-1/y$, we know that $(x y+1)/(x-y) \in E^*$, so there is some $t \in \Z$ such that $(x y+1)/(x-y)=\beta^t$.
Also, since $|E|=q^2=3^{2 n}$, we can write $-1=\beta^{(q^2-1)/2}$.
Thus, \eqref{Wolfgang} says that
\[
\beta^{(3^k-1) t} = \beta^{(q^2-1)/2},
\]
and so
\[
(3^k-1)t \equiv \frac{q^2-1}{2} \pmod{q^2-1}.
\]
We know that $q^2=3^{2 n}$ with $n$ odd, so \cref{Francis} says that $v_2(3^{2 n} -1)=v_2(3+1)+v_2(2 n)=2+1=3$, so $8\mid q^2-1$ but $16\nmid (q^2-1)$.
So our last congruence implies
\[
(3^k-1)t \equiv 4 \pmod{8}
\]
Since $k$ is even, \cref{Francis} shows that $v_2(3^k-1)=v_2(3+1)+v_2(k) \geq 3$, so $3^k-1 \equiv 0 \pmod{8}$, and so
\[
0 \equiv 4 \pmod{8},
\]
which is a contradiction.
\end{proof}
\begin{lemma}\label{Isaiah}
Let $F$ be a finite field of order $q=3^n$ with $n$ odd and let $k$ be an even nonnegative integer.
Let $E$ be the quadratic extension of $F$ and let $U_E$ be the unit circle of $E$.
Let $x,y \in U_E$ with
\[\frac{x^{p^k+1}-1}{x^{p^k}+x} = -\frac{y^{p^k+1}-1}{y^{p^k}+y}.\]
Then $x \in \{-y,1/y\}$.
\end{lemma}
\begin{proof}
Let $z=-y$.  Then
\begin{align*}
\frac{x^{p^k+1}-1}{x^{p^k}+x}
& = -\frac{(-z)^{p^k+1}-1}{(-z)^{p^k}+(-z)} \\
& = \frac{z^{p^k+1}-1}{z^{p^k}+z}.
\end{align*}
Now $x, z \in U_E$ since $U_E$ is a group and $-1 \in U_E$.
So by \cref{Uriah}, we know that $x \in \{z,-1/z\}=\{-y,1/y\}$.
\end{proof}

\begin{lemma}\label{Madeleine}
Let $F$ be a finite field of order $q=3^n$ with $n$ odd, let $k$ be an even nonnegative integer, let $d_1=(3^n+1)/2$ and $d_2=(3^k+1)/2$.
Let $f_4 \colon F \to F$ with
\[
f_4(x)=\frac{((x+2)^{d_1}-(x-2)^{d_1})^{d_2}}{((x+2)^{d_2}-(x-2)^{d_2})^{d_1}},
\]
from \cref{Dorothy}.
Let $E$ be the quadratic extension of $F$ and let $U_E$ be the unit circle of $E$.
Let $\lambda\colon U_E \to F$ with $\lambda(x)=x+x^{-1}$.
The fiber $(f_4\circ\lambda)^{-1}(\{0\})$ is equal to the set $\{i,1/i\}=\{i,-i\}$ of primitive fourth roots of unity in $U_E$.
\end{lemma}
\begin{proof}
By \cref{Heloise}, an $x \in U_E$ has $(f_4\circ\lambda)(x)=0$ if and only if $x^{-1}+x=0$, which is true if and only if $x^2+1=0$, whose solutions are the primitive fourth roots of unity, which exist in $U$ because $4|q+1=3^n+1$ by \cref{Francis} because $n$ is odd.
\end{proof}

\begin{lemma}\label{Jeremy}
Let $F$ be a finite field of order $q=3^n$ with $n$ odd, let $k$ be an even nonnegative integer, let $d_1=(3^n+1)/2$ and $d_2=(3^k+1)/2$.
Let $f_4 \colon F \to F$ with
\[
f_4(x)=\frac{((x+2)^{d_1}-(x-2)^{d_1})^{d_2}}{((x+2)^{d_2}-(x-2)^{d_2})^{d_1}},
\]
from \cref{Dorothy}.
Let $E$ be the quadratic extension of $F$ and $U_E$ the unit circle of $E$.
Let $\lambda\colon U_E \to F$ with $\lambda(x)=x+x^{-1}$.
For $a \in U_E$, the fiber of $f_4\circ\lambda$ that contains $a$ is $\{a,1/a\}$.
\end{lemma}
\begin{proof}
Let $a,b \in U_E$ and suppose that $b$ is in the same fiber of $f_4\circ\lambda$ as $a$.
Since $a$ and $b$ are in the same fiber of $f_4\circ\lambda$, we have $(f_4\circ\lambda(a))^2=(f_4\circ\lambda(b))^2$.
So by \cref{Julia}, we have
\[
1+\left(\frac{b^{p^k+1}-1}{b^{p^k}+b}\right)^2 = 1+\left(\frac{a^{p^k+1}-1}{a^{p^k}+a}\right)^2,
\]
so that there is some $\sigma \in \{1,-1\}$ such that
\[
\frac{b^{p^k+1}-1}{b^{p^k}+b} =\sigma \cdot \frac{a^{p^k+1}-1}{a^{p^k}+a}.
\]
Then by Lemmas \ref{Uriah} and \ref{Isaiah}, this shows that $b \in \{a,-1/a,-a,1/a\}$.

If $b \in \{-a,-1/a\}$, then \cref{Thomas} shows that $(f_4\circ\lambda)(b)=-(f_4\circ\lambda)(a)$, and since we assumed that $a$ and $b$ are in the same fiber of $f_4\circ\lambda$, this makes $(f_4\circ\lambda)(b)=(f_4\circ\lambda)(a)=0$.
By \cref{Madeleine}, this forces $a,b \in \{i,-i\}$, where $i$ is a primitive fourth root of unity, and then $\{a,1/a\}=\{i,-i\}$ and so $b \in \{a,1/a\}$.

So we see that $b \in \{a,1/a\}$ always.
Thus, the fiber of $f_4\circ\lambda$ containing $a$ is a subset of $\{a,1/a\}$.
And in fact, the fiber is $\{a,1/a\}$ because \cref{Thomas} shows that $f_4\circ\lambda(1/a)=f_4\circ\lambda(a)$.
\end{proof}

\begin{proposition}\label{Edward}
Let $F$ be a finite field of order $q=3^n$ with $n$ odd, let $k$ be an even nonnegative integer, let $d_1=(3^n+1)/2$ and $d_2=(3^k+1)/2$.
Let $f_4 \colon F \to F$ with
\[
f_4(x)=\frac{((x+2)^{d_1}-(x-2)^{d_1})^{d_2}}{((x+2)^{d_2}-(x-2)^{d_2})^{d_1}},
\]
from \cref{Dorothy}.
Let $E$ be the quadratic extension of $F$ and $U_E$ the unit circle of $E$.
Let $\lambda\colon U_E \to F$ with $\lambda(x)=x+x^{-1}$.
Let $\eta$ be the extended quadratic character of $F$.
For $c \in F$, we have $\card{(f_4\circ\lambda)^{-1}(\{c\})}=1+\eta(1-c^2)$.
\end{proposition}
\begin{proof}
Let $c \in F$.
Since $(f_4\circ\kappa)(-x)=-(f_4\circ\kappa)(x)$ for all $x \in U_E$ by \cref{Thomas}, we know that the fiber over $c$ is nonempty if and only if the fiber over $-c$ is nonempty.
Thus, the fiber over $c$ is nonempty if and only if there is some $a \in U_E$ such that $((f_4\circ\kappa)(a))^2=c^2$.
By \cref{Julia}, this happens if and only if there is some $a \in U_E$ such that
\[
1+ \left(\frac{a^{3^k+1}-1}{a^{3^k}+a}\right)^2=c^2.
\]
So the fiber over $c$ is nonempty if and only if there is some $a \in U_E$ with
\[
\left(\frac{a^{3^k+1}-1}{a^{3^k}+a}\right)^2=c^2-1.
\]
Now \cref{Nicholas} tells us that the left-hand side is not a quadratic residue in $c$.
Thus, the fiber over $c$ is certainly empty if $\eta(c^2-1)=1$.
Since $-1$ is a quadratic nonresidue in $F$ by \cref{Quentin}, this is equivalent to saying that the fiber over $c$ is empty if $1+\eta(1-c^2)=0$.
In other words, if $1+\eta(1-c^2)=0$, then the fiber over $c$ has cardinality $1+\eta(1-c^2)$.
By \cref{Muriel}, this means that there are at least $(p^n-3)/2$ empty fibers over elements $c \in F$.
Thus, there are at most $(p^n+3)/2$ nonempty fibers over elements $c \in F$.
The nonempty fibers form a partition of the domain $U_E$ of $\lambda$, and \cref{Jeremy} shows us that $\{1\}$ and $\{-1\}$ are fibers and that all other nonempty fibers are of cardinality $2$.
Thus, if there were strictly fewer than $(p^n+3)/2$ nonempty fibers, then there would be two fibers of size $1$ and fewer than $(p^n-1)/2$ fibers of size $2$; then the union of all the fibers would have fewer than $2(1)+2((p^n-1)/2)=p^n+1$ elements, which is absurd.
So there must be precisely $(p^n+3)/2$ nonempty fibers: $\{1\}$, $\{-1\}$, and $(p^n-1)/2$ fibers of cardinality $2$.
Using \cref{Heloise}, it is not hard to calculate $(f_4\circ\kappa)(1)=1$ and $(f_4\circ\kappa)(-1)=-1$, so we see that the fiber over $c$ is indeed of cardinality $1+\eta(1-c^2)$ when $1+\eta(1-c^2)=1$.
We have already accounted for the fibers over all $c\in F$ with $1+\eta(1-c^2)=0$ (there are $(p^n-3)/2$ of these and they are empty) and all $c\in\Fu$ with $1+\eta(1-c^2)=1$ (there are $2$ of these and they are of cardinality $1$).  There are $(p^n-1)/2$ remaining values of $c \in F$, and by \cref{Muriel} these are all $c\in F$ with $1+\eta(1-c^2)=2$.  The fibers over these $c$ must be the $(p^n-1)/2$ fibers of cardinality $2$, so these fibers also have cardinality equal to $1+\eta(1-c^2)$.
\end{proof}

\begin{proposition}\label{Abigail}
Let $F$ be a finite field of order $q=3^n$ with $n$ odd, let $k$ be an even nonnegative integer with $\gcd(k,n)=1$, let $d_1=(3^n+1)/2$, and let $d_2=(3^k+1)/2$.
Let $f_4 \colon F \to F$ with
\[
f_4(x)=\frac{((x+2)^{d_1}-(x-2)^{d_1})^{d_2}}{((x+2)^{d_2}-(x-2)^{d_2})^{d_1}},
\]
from \cref{Dorothy}.
Let $\eta$ be the quadratic character of $F$.
For $c \in F$, we have
\[
\card{f_4^{-1}(\{c\})}=
\begin{cases}
1 & \text{if $c \in \F_3$,} \\
1+\eta(1-c^2) & \text{otherwise.}
\end{cases}
\]
\end{proposition}
\begin{proof}
Extend the quadratic character $\eta$ to map $0$ to $0$.
Define $\kappa$ and $\lambda$ as in \cref{Eusebius}, which then tells us that for each $c \in F$, we have
\[
\card{f_4^{-1}(\{c\})}=\frac{\card{(f_4\circ\kappa)^{-1}(\{c\})}+\card{(f_4\circ\lambda)^{-1}(\{c\})}}{2},
\]
and then use Lemmas \ref{Irving} and \ref{Edward} to see that
\[
\card{f_4^{-1}(\{c\})}=\begin{cases}
1  & \text{if $c=0$} \\
1+\eta(1-c^2) & \text{otherwise}.
\end{cases}
\]
Then note that $\F_3=\{0,1,-1\}$, so that if $c \in \F_3$, then our last formula gives us $\card{f_4^{-1}(\{c\})}=1$.
For $c \in F\smallsetminus\F_3$, we have $\card{f_4^{-1}(\{c\})}=1+\eta(1-c^2)$ and since $1-c^2\not=0$, we are never evaluating $\eta$ at $0$ in this case, so the statement of this proposition needs only speak of the quadratic character, not the extended one.
\end{proof}

\begin{lemma}\label{Gilda}
Let $F$ be a finite field of order $q=3^n$ with $n$ odd, let $k$ be an even nonnegative integer with $\gcd(k,n)=1$, let $d_1=(3^n+1)/2$, and let $d_2=(3^k+1)/2$.
Let $f_3\colon F \to F$ be given by 
\[
f_3(x)=\frac{((x+1)^{d_1}-x^{d_1})^{d_2}}{((x+1)^{d_2}-x^{d_2})^{d_1}},
\]
from \cref{Celia}.
Let $\eta$ be the quadratic character of $F$.
For $c \in F$, we have
\[
\card{f_3^{-1}(\{c\})}=
\begin{cases}
1 & \text{if $c \in \F_3$,} \\
1+\eta(1-c^2) & \text{otherwise.}
\end{cases}
\]
\end{lemma}
\begin{proof}
\cref{Dorothy} tells us that $f_3=f_4\circ\tau^{-1}$, where $f_4$ is as in \cref{Abigail} and $\tau$ is a permutation of $F$.
Thus, the result follows from \cref{Abigail} and \cref{Leonid}.
\end{proof}
\begin{lemma}\label{Timothy}
Let $F$ be a finite field of order $q=3^n$ with $n$ odd, let $k$ be an even nonnegative integer with $\gcd(k,n)=1$, let $d_1=(3^n+1)/2$, and let $d_2=(3^k+1)/2$.
Let $f_2 \colon F \to F$ be the function from \cref{Beatrice} defined by $f_2(1)=1$ and
\[
f_2(x)=\frac{(x^{d_1}-1)^{d_2}}{(x^{d_2}-1)^{d_1}}
\]
for $x\not=1$.
Let $\eta$ be the quadratic character of $F$.
For $c \in F$, we have
\[
\card{f_2^{-1}(\{c\})}=
\begin{cases}
1 & \text{if $c \in \F_3$,} \\
1+\eta(1-c^2) & \text{otherwise.}
\end{cases}
\]
\end{lemma}
\begin{proof}
\cref{Celia} tells us that $f_2=f_3\circ\pi^{-1}$, where $f_3$ is as in \cref{Gilda} and $\pi$ is a permutation of $F$.  Thus, the result follows from Lemmas \ref{Gilda} and \ref{Leonid}.
\end{proof}
\begin{lemma}\label{Ulrich}
Let $F$ be a finite field of order $q=3^n$ with $n$ odd and let $k$ be an even nonnegative integer with $\gcd(k,n)=1$.
Let $f_1 \colon F \to F$ be the function from \cref{Aaron} defined by $f_1(1)=1$ and $f_1(x)=(x^d-1)/(x-1)^d$ for $x\not=1$ with $d=(3^n+1)/(3^k+1)$.
Let $\eta$ be the quadratic character of $F$.
For $c \in F$, we have
\[
\card{f_1^{-1}(\{c\})}=
\begin{cases}
1 & \text{if $c \in \F_3$,} \\
1+\eta(1-c^{3^k+1}) & \text{otherwise.}
\end{cases}
\]
\end{lemma}
\begin{proof}
Let $d_1=(3^n+1)/2$, let $d_2=(3^k+1)/2$, and let $\sigma \colon F \to F$ with $\sigma(x)=x^{d_2}$, the permutation of $F$ defined in \cref{Beatrice}, which tells us that $f_1=\sigma^{-1} \circ f_2 \circ \sigma^{-1}$, where $f_2$ is as in \cref{Timothy}.
For each $c \in F$, Lemmas \ref{Fedor} and \ref{Leonid} tell us that $|f_1^{-1}(\{c\})|=|(f_2\circ\sigma^{-1})^{-1}(\{\sigma(c)\})|=|f_2^{-1}(\{\sigma(c)\})|$.
Now we use the cardinalities of fibers of $f_2$ from \cref{Timothy}.
Since $\sigma$ maps $\F_3=\{0,1,-1\}$ onto itself, we have $|f_1^{-1}(\{c\})|=1$ for every $c \in \F_3$.
For $c\in F\smallsetminus \F_3$, we have $|f_1^{-1}(\{c\})|=|f_2^{-1}(\{\sigma(c)\})|=1+\eta(1-\sigma(c)^2)=\eta(1-c^{3^k+1})$.
\end{proof}
We now restate and prove \cref{Victoria}.
\begin{theorem}\label{Victor}
Let $n$ be an odd positive integer and $k$ a nonnegative even integer with $\gcd(n,k)=1$.  Let $F$ be the finite field of order $3^n$, and let $f\colon F \to F$ be the power function with exponent $(3^n+1)/(3^k+1)$.  Let $\eta$ be the quadratic character for $F$.  Then for $c \in F$, we have
\[
\card{(\Delta f)^{-1}(\{c\})}=
\begin{cases}
1 & \text{if $c \in \F_3$,} \\
1+\eta(1-c^{3^k+1}) & \text{otherwise.}
\end{cases}
\]
In particular, $f$ is an APN function with reduced differential spectrum
\[
((3^n-3)/2) [0] + 3 [1] + ((3^n-3)/2) [2].
\]
\end{theorem}
\begin{proof}
Let $d_1=(3^n+1)/2$ and $d_2=(3^k+1)/2$ and note that $\gcd(d_2,\cardFu)=1$ by \cref{Clementine} and our power function is $x\mapsto x^d$ where $d$ is the exponent $d_1/d_2$ over $F$.
From \cref{Aaron} we know that $\Delta f=f_1\circ\pi$ for $f_1$ as in \cref{Ulrich} and some permutation $\pi$ of $F$, so the formula for sizes of fibers of $\Delta f$ follows from Lemmas \ref{Ulrich} and \ref{Leonid}.
Since \cref{Clementine} tells us that $\gcd(3^k+1,3^n-1)=2$, \cref{Herbert} shows us that for $c \in F$, we have $c^{3^k+1}=1$ if and only if $c \in \{1,-1\}$.
Thus, $1+\eta(1-c^{3^k+1}) \in \{0,2\}$ for all $c \in F\smallsetminus \F_3$.
Therefore, $\Delta f$ has precisely three fibers of cardinality $1$, and each of the $3^n-3$ remaining fibers is of cardinality $0$ or $2$.
Since the nonempty fibers partition the domain (which is $F$) of $\Delta f$, there must be $(3^n-3)/2$ fibers of cardinality $2$ and just as many of cardinality $0$.
\end{proof}

\begin{thebibliography}{HRS99}

\bibitem[Dob99]{Dobbertin-1999-Welch}
Hans Dobbertin.
\newblock Almost perfect nonlinear power functions on {${\rm GF}(2^n)$}: the
  {W}elch case.
\newblock {\em IEEE Trans. Inform. Theory}, 45(4):1271--1275, 1999.

\bibitem[HRS99]{Helleseth-Rong-Sandberg}
Tor Helleseth, Chunming Rong, and Daniel Sandberg.
\newblock New families of almost perfect nonlinear power mappings.
\newblock {\em IEEE Trans. Inform. Theory}, 45(2):474--485, 1999.

\bibitem[ZW10]{Zha-Wang}
ZhengBang Zha and XueLi Wang.
\newblock Power functions with low uniformity on odd characteristic finite
  fields.
\newblock {\em Sci. China Math.}, 53(8):1931--1940, 2010.

\end{thebibliography}


\end{document}

