\documentclass[aps,notitlepage,10pt]{revtex4-2}

\usepackage{mathrsfs}
\usepackage{bbm}
\usepackage{amsmath}
\usepackage{amssymb}
\usepackage{amsthm}
\usepackage{graphicx}
\usepackage{MnSymbol}
\usepackage{mathtools}
\usepackage{stmaryrd}
\usepackage{bbold}
\usepackage[italicdiff]{physics}
\usepackage{chemformula}

\renewcommand{\theequation}{S\arabic{equation}}
\renewcommand{\thefigure}{S\arabic{figure}}
\renewcommand{\thesection}{S\arabic{section}}
%\renewcommand{\qedsymbol}{$\blacksquare$}

\makeatletter
\usepackage{hyperref}
\usepackage{color}
\definecolor{supcol}{RGB}{100,50,100}
\definecolor{eqcol}{RGB}{100,10,200}
\hypersetup{
	colorlinks,
	citecolor=supcol,
	linkcolor=eqcol,
	urlcolor=supcol
}

\allowdisplaybreaks

\newtheorem{theorem}{Theorem}
\newtheorem{claim}[theorem]{Claim}
\newtheorem{proposition}[theorem]{Proposition}
\newtheorem{definition}[theorem]{Definition}
\newtheorem{lemma}[theorem]{Lemma}
\newtheorem{corollary}[theorem]{Corollary}
\newtheorem{conjecture}[theorem]{Conjecture}

\newcommand{\bms}{\mathbbmss}
\newcommand{\mca}{\mathcal}
\newcommand{\mbb}{\mathbb}
\newcommand{\mrm}{\mathrm}
\newcommand{\msf}{\mathsf}
\newcommand{\mfr}{\mathfrak}
\newcommand{\msc}{\mathscr}
\newcommand{\mbm}[1]{\boldsymbol{#1}}
\newcommand{\sop}[1]{\llbracket #1 \rrbracket}
\newcommand{\react}[2]{\overset{#1}{\underset{#2}{\rightleftharpoons}}}
\newcommand{\sign}{\text{sgn}}
\newcommand{\sectionprl}[1]{{\em #1}\/.---}

\newcommand{\sx}{{\hat{\sigma}^x}}
\newcommand{\sy}{{\hat{\sigma}^y}}
\newcommand{\sz}{{\hat{\sigma}^z}}


\newcommand{\DetEq}{1}
\newcommand{\GenSL}{5}
\newcommand{\ConSL}{6}
\newcommand{\VeloUpB}{9}
\newcommand{\BosonVtUp}{18}

\begin{document}
\title{Supplemental Material: Brownian yet Non-Gaussian Heat Engine}
\author{I. Iyyappan}
\email[email: ]{iyyappan@imsc.res.in}
\affiliation{The Institute of Mathematical Sciences,\\ CIT Campus, Taramani, Chennai 600113, India.\\
Homi Bhabha National Institute, Training School Complex, Anushakti Nagar, Mumbai 400094, India.}
\author{Jetin E. Thomas}
\email[email: ]{jetinthomas@gmail.com}
\author{Sibasish Ghosh}
\email[email: ]{sibasish@imsc.res.in}
\affiliation{The Institute of Mathematical Sciences,\\ CIT Campus, Taramani, Chennai 600113, India.\\
Homi Bhabha National Institute, Training School Complex, Anushakti Nagar, Mumbai 400094, India.}

\begin{abstract}
In this supplemental material, we gave the details of the analytical and numerical results of our study. The derivation for average mobility, and the definition for time-periodic steady-state. The plots for average thermodynamic quantities such as input heat, ejected heat, and change in internal energy. We plotted the power and efficiency with error bars. Total non-Gaussian parameter per cycle time. Finally, we also presented our results for the free Brownian and Brownian yet non-Gaussian diffusion.
\end{abstract}

\pacs{}
\maketitle
\tableofcontents
\section{The average mobility of the fluctuation mobility medium}
The Fokker-Planck equation for the Langevin Eq. (7) becomes
\begin{equation}\label{i}
\frac{\partial f(\chi,t)}{\partial t}= -\frac{\partial }{\partial \chi} \left[-\frac{\chi}{\tau'} -\frac{\sigma^{2}}{2}\frac{\partial}{\partial \chi}\right]f(\chi,t). \tag{S2}
\end{equation}
Here, $f(\chi,t)$ is the probability distribution of $\chi$ at time $t$. Solving the Eq. (\ref{i}) with the initial value $\chi(0)\equiv\chi_0$, we get the following probability distribution 
\begin{equation}\label{k}
f(\chi,t|\chi_0,0)=\frac{1}{\sqrt{\pi \sigma^{2}\tau '\left(1-\exp\left[\frac{-2t}{\tau '}\right]\right)}}\exp\left[\frac{-\left(\chi-\chi_0\exp\left[-\frac{t}{\tau '}\right]\right)^{2}}{\sigma^{2}\tau '\left(1-\exp\left[\frac{-2t}{\tau '}\right]\right)}\right] . \tag{S3}
\end{equation}
Now, the $\chi_0$ is drawn randomly from the below steady-state  distribution 
\begin{equation}\label{j}
f(\chi_0)=\frac{1}{\sqrt{\pi\sigma^{2}\tau'}}\mbox{exp}\left(-\frac{\chi_0^{2}}{\sigma^{2}\tau'}\right).  \tag{S4}
\end{equation}
Taking the average over $\chi_0$ \cite{besL01}
\begin{equation}
f(\chi,t)=\int_{-\infty}^{\infty}f(\chi,t|\chi_0,0)f(\chi_0)d\chi_0. \tag{S5}
\end{equation}
We get the following probability distribution
\begin{equation}\label{prob}
f(\chi)=\frac{1}{\sqrt{\pi\sigma^{2}\tau'}}\mbox{exp}\left(-\frac{\chi^{2}}{\sigma^{2}\tau'}\right). \tag{S6}
\end{equation}
Using Eq. (\ref{prob}), we get the average mobility as
\begin{equation}\label{s5}
\langle \mu\rangle=\langle \chi^{2}\rangle=\frac{1}{2}\sigma^{2}\tau'. \tag{S7}
\end{equation}

\section{Time-periodic steady-state}
When we perform a Stirling cycle for a given cycle time $\tau$ to an ensemble of Brownian particles. The final position $x(\tau)$ (with $x(0)=0$) is a random variable. To get the efficiency independent of $x(0)$, first, we need to run many cycles until $\langle x(n\tau)^{2}\rangle$ and $\langle x([n+1]\tau)^{2}\rangle$ becomes equal, where $n$ is the number of cycles performed. After this, the probability distribution satisfies the following condition, $p_{ss}(x,t)=p_{ss}(x, t+\tau)$, which is called the time-periodic steady-state (TPSS) \cite{ran042,gom643,kum032}. If the ensemble of Brownian particles once reaches a TPSS, then it will no longer have the memory of its initial position $x(0)=0\,\mu m$(micrometer). Brownian particles in a homogeneous medium reach a steady state after a few cycles. However, we don't know in the beginning, the number of cycles required to reach the steady state. Therefore, we keep track of the $x^{2}$ at the end of each cycle and compare this with that of the previous cycle. We keep the precision 
\begin{equation}
\Omega\equiv \langle x([n+1]\tau)^{2} \rangle- \langle x(n\tau)^{2} \rangle < 10^{-6}.	
\end{equation}
Once the above condition is satisfied, we start to calculate the average quantities like work, heat supplied, and ejected heat for the cycles after this difference reaches a lower value than the fixed precision value. For cycles after this, the $x(m\tau)^{2}$ and the other average quantities at the end of each cycle behave like random variables fluctuating about some average values. We consider these values obtained at the end of cycles after reaching steady-state as identically distributed random variables. Consequently, the random variable of the average of the cumulative sum of the $x^{2}$ of each particle at the end of cycles after reaching steady-state obeys the central limit theorem. Thus, the standard deviation of this random variable for the $N$ particles decreases for each added cycle after reaching the steady state. We fixed another precision of $0.0007$ for this standard deviation which will enable us to calculate the average quantities with the desired statistical accuracy. Hence, the definition of the precision that determines the statistical accuracy from the central limit theorem is
\begin{equation}
\sqrt{\frac{1}{N}\sum_{i}^{N}\left(\frac{\sum_{m=1}^{N_{cycst}}x_{i}(m\tau)^{2}}{N_{cycst}}-\frac{\sum_{i=1}^{N}\sum_{m=1}^{N_{cycst}}x_{i}(m\tau)^{2}}{N*N_{cycst}}\right)^{2}} < 0.0007.	
\end{equation} 
where $N_{cycst}$ is the number of cycles traversed after reaching the steady-state and $m=1$ is the first cycle after reaching the steady state. We set $\Omega<10^{-7}$ for cycle times $0.01$ to $0.1$, and $\Omega<10^{-6}$ for cycle times $0.2$ to $10$. However, due to the limitation of our computational facility, we have used $\Omega<10^{-5}$ for cycle times $20$ to $50$. 
 
\section{The average input heat, ejected heat, and change in internal energy}
The average input heat is plotted in Fig. (\ref{fig:hci}a). It shows that $Q_h$ decreases monotonically with $\tau$ and saturates at a larger cycle time. We find that the Brownian heat engine (BHE) absorbs the larger input heat as compared with the Brownian yet non-Gaussian heat engine (BNGHE). The average ejected heat is plotted in Fig. (\ref{fig:hci}b). $Q_c$ increases monotonically with cycle time and saturates at a larger $\tau$. Again, the BHE ejects the larger heat as compared with the BNGHE. The average change in internal energy after completing a cycle is plotted as a function of $\tau$ in Fig. (\ref{fig:hci}c). It shows that the mean change in internal energy fluctuates around zero and it has to be noted that this fluctuation is a hundred times smaller than that of $Q_h$ and $Q_c$ values. The fluctuation in $\Delta U$ is due to the finite value of $dt=10^{-5}$ in our numerical integration of Eqs. (5) -(7) and (13) as well as due to the less number of trajectories (in our case $10^{5}$). See also Fig. (3) in Ref. \cite{bli070} for further explanations.  
% Figure environment removed

\section{Power, and efficiency with error bars}
We plotted the power and efficiency with error bars for a cycle time $\tau=5$ to $\tau=50$ (so that the error bars are visible) and we only considered black (BHE) and blue (BNGHE) curves in Fig. (\ref{fig:pe_eb}). We observe that the BNGHE efficiency becomes higher than the BHE, it is because of numerical error, not a physical phenomenon. Our reasoning as follows, diffusing diffusivity system requires (at least) $10^{6}$ trajectories to minimize the fluctuations at a reasonable level (see Figs. \ref{fig:msd} (c), and  (d) for the difference) which is unfeasible for heat engine with our computational facility (where only $10^{5}$ trajectories are considered). However, Chechkin \textit{et. al} \cite{che021} showed that the diffusing diffusivity system approaches the Gaussian distribution of position at a very large time (see Figs. (1) and (3) of Ref. \cite{che021}). Although $10^{6}$ trajectories are considered in Ref. \cite{che021}, we could still see the fluctuation in \textit{Kurtosis} (see Figs. (3) of Ref. \cite{che021}, and also Figs. \ref{fig:msd} (c), and (d) for our case). The average work, and input heat depend on the $\langle x(t)^{2}\rangle$ through Eqs. (3) and (4). When Brownian yet non-Gaussian diffusion and Brownian diffusion have the same Gaussian distribution of position at a larger cycle time will lead only to the same power and efficiency. Therefore, the difference visible between the BHE and BNGHE performance in Fig. (\ref{fig:pe_eb}) is solely a numerical error.  
% Figure environment removed

\section{The average position square}
To understand the BHE and BNGHE work output, we plotted the $\langle x(t)^{2} \rangle$ during the cycle times at $\tau=0.01,\;0.1,\;1,\; \mbox{and}\;10$ in Fig. (\ref{fig:x2}). We find that for a cycle time, $\tau=0.01$, the $\langle x(t)^{2} \rangle$ keeps increasing even after the isothermal compression process starts (see Fig. \ref{fig:x2}a) which makes the total work positive. As we know work is the path variable, we get different work for different cases. Fig. (\ref{fig:x2} c, and d) shows that the black, and blue curves merge which gives rise to roughly equal work outputs. The asymmetry of $\langle x(t)^{2} \rangle$ which we (particularly) observe in Figs. (\ref{fig:x2}c) and (\ref{fig:x2}d) are the reasons behind the negative work. It has to be noted that the area under the curve is directly proportional to the average work (see Eq. (3)) with the -ve sign for time $t=0$ to $t=\tau/2$ and +ve for $t=\tau/2$ to $t=\tau$.
% Figure environment removed

\section{Total non-Gaussian parameter per cycle time}
To find the effect of non-Gaussian position distribution on the performance of a stochastic heat engine, we calculate the total non-Gaussian parameter per cycle time using the following equation
\begin{equation}\label{ingp}
\Pi(x)=\frac{\int_{0}^{\tau}\Gamma(x,t)dt}{\tau}.
\end{equation}
We plotted Eq. (\ref{ingp}) in Fig. (\ref{fig:ingp}). We find that for heat engine at $\tau=1$, and $\tau=10$ the higher $\Pi(x)$ decreases the stochastic heat engine performance (see Fig. (2)).
% Figure environment removed

\section{Free: Brownian and Brownian yet non-Gaussian diffusion}
To analyze the free diffusion in heterogeneous and homogeneous thermal baths, we substitute $\lambda(t)=0$ in Eqs. (5), and (13), respectively, of the main paper. Therefore, our Eqs. (6)-(8) becomes Eqs. (19a)-(19c) of Ref. \cite{che021}. We numerically calculated the mean-squared displacement (Fig. (\ref{fig:msd}a)), mean mobility (Fig. (\ref{fig:msd}b)), non-Gaussian parameter $\Gamma(x)$  (Fig. (\ref{fig:msd}c) with $10^{5}$ trajectories), and $\Gamma(x)$  (Fig. (\ref{fig:msd}d) with $10^{6}$ trajectories) are plotted as a function of time. Fig. (\ref{fig:msd}c) shows that $\Gamma(x)$ fluctuates highly with time. Fig. (\ref{fig:msd}a)) shows that the Brownian diffusion and Brownian yet non-Gaussian diffusion (with three sets of $\tau'$ and $\sigma$) are obeying the normal diffusion condition $\langle x^{2}(t) \rangle \propto t$. Fig. (\ref{fig:msd}b) shows average mobility fluctuations mainly depend on the value of $\sigma$. To show we need a very high number of trajectories for Brownian yet non-Gaussian diffusion \cite{che021}, $\Gamma(x)$ with $10^{6}$ trajectories is plotted in Fig. (\ref{fig:msd}d). The high fluctuation of $\Gamma(x)$ in Fig. (\ref{fig:msd}c) is reduced in Fig. (\ref{fig:msd}d). However, it is not feasible with our computing facility to consider the $10^{6}$ trajectories for the Brownian yet non-Gaussian heat engine which takes an enormous time to reach the time-periodic steady-state. Therefore, we considered only $10^{5}$ trajectories for both BHE and BNGHE for the qualitative study.

% Figure environment removed
The probability distributions of $x$ are plotted at time, $t=1$, and $t=10$ in Fig. (\ref{fig:freeprob}), and it can be compared with the Fig. (1) of Ref. \cite{spo117}. It shows that the Brownian yet non-Gaussian diffusion consist of fast and slow-moving Brownian particles when we compare with the normal Brownian diffusion. At the bottom, the probability distributions away from the Gaussian (black curve) are due to the fast-moving Brownian particles. At the top, probability distributions above the Gaussian curves are due to the slow-moving Brownian particles as mentioned in Ref. \cite{spo117}. 

% Figure environment removed

\begin{thebibliography}{10}%
\bibitem{besL01}
B. Besga, F. Faisant, A. Petrosyan, S. Ciliberto, and S. N. Majumdar, Phys. Rev. E \textbf{104}, L012102 (2021).
\bibitem{ran042}
S. Rana, P. S. Pal, A. Saha, and A. M. Jayannavar, Phys. Rev. E \textbf{90}, 042146 (2014).
\bibitem{gom643}
J. R. Gomez-Solano, Front. Phys. \textbf{9}, 643333 (2021).
\bibitem{kum032}
A. Kumari, P. S. Pal, A. Saha, and S. Lahiri, Phys. Rev. E \textbf{101}, 032109 (2020).
\bibitem{bli070}
V. Blickle, T. Speck, L. Helden, U. Seifert, and C. Bechinger, Phys. Rev. Lett. \textbf{96}, 070603 (2006).
\bibitem{che021}
A. V. Chechkin, F. Seno, R. Metzler, and I. M. Sokolov, Phys. Rev. X \textbf{7}, 021002 (2017).
\bibitem{spo117}
V. Sposini \textit{et. al}, Phys. Rev. Lett. \textbf{132}, 117101 (2024); Phys. Rev. E \textbf{109}, 034120 (2024).

\end{thebibliography}%
\end{document}