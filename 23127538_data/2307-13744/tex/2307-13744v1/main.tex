
\documentclass[10pt]{article} % For LaTeX2e
%\usepackage{tmlr}
% If accepted, instead use the following line for the camera-ready submission:
%\usepackage[accepted]{tmlr}
% To de-anonymize and remove mentions to TMLR (for example for posting to preprint servers), instead use the following:
\usepackage[preprint]{tmlr}

% Optional math commands from https://github.com/goodfeli/dlbook_notation.
%%%%% NEW MATH DEFINITIONS %%%%%
\newtheorem{property}{Property}
\newtheorem{definition}{Definition}
\newtheorem{theorem}{Theorem}
\newtheorem{lemma}{Lemma}
\newtheorem{corollary}{Corollary}
\DeclarePairedDelimiter\abs{\lvert}{\rvert}
\DeclarePairedDelimiter\norm{\lVert}{\rVert}
\makeatletter
\let\oldabs\abs
\def\abs{\@ifstar{\oldabs}{\oldabs*}}
\let\oldnorm\norm
\def\norm{\@ifstar{\oldnorm}{\oldnorm*}}
\makeatother

% Mark sections of captions for referring to divisions of figures
\newcommand{\figleft}{{\em (Left) }}
\newcommand{\figcenter}{{\em (Center) }}
\newcommand{\figright}{{\em (Right)}}
\newcommand{\figtop}{{\em (Top) }}
\newcommand{\figbottom}{{\em (Bottom) }}
\newcommand{\captiona}{{\em (a) }}
\newcommand{\captionb}{{\em (b) }}
\newcommand{\captionc}{{\em (c) }}
\newcommand{\captiond}{{\em (d) }}

% Highlight a newly defined term
\newcommand{\newterm}[1]{{\bf #1}}


\def\figref#1{figure~\ref{#1}}
\def\Figref#1{Figure~\ref{#1}}
\def\twofigref#1#2{figures \ref{#1} and \ref{#2}}
\def\quadfigref#1#2#3#4{figures \ref{#1}, \ref{#2}, \ref{#3} and \ref{#4}}
\def\secref#1{section~\ref{#1}}
\def\Secref#1{Section~\ref{#1}}
\def\twosecrefs#1#2{sections \ref{#1} and \ref{#2}}
\def\secrefs#1#2#3{sections \ref{#1}, \ref{#2} and \ref{#3}}
\def\eqref#1{equation~\ref{#1}}
\def\Eqref#1{Equation~\ref{#1}}
% A raw reference to an equation---avoid using if possible
\def\plaineqref#1{\ref{#1}}
% Reference to a chapter, lower-case.
\def\chapref#1{chapter~\ref{#1}}
% Reference to an equation, upper case.
\def\Chapref#1{Chapter~\ref{#1}}
% Reference to a range of chapters
\def\rangechapref#1#2{chapters\ref{#1}--\ref{#2}}
% Reference to an algorithm, lower-case.
\def\algref#1{algorithm~\ref{#1}}
% Reference to an algorithm, upper case.
\def\Algref#1{Algorithm~\ref{#1}}
\def\twoalgref#1#2{algorithms \ref{#1} and \ref{#2}}
\def\Twoalgref#1#2{Algorithms \ref{#1} and \ref{#2}}
% Reference to a part, lower case
\def\partref#1{part~\ref{#1}}
% Reference to a part, upper case
\def\Partref#1{Part~\ref{#1}}
\def\twopartref#1#2{parts \ref{#1} and \ref{#2}}

% Random variables
\def\reta{{\textnormal{$\eta$}}}
\def\ra{{\textnormal{a}}}

% Random vectors
\def\rvepsilon{{\mathbf{\epsilon}}}
\def\rvtheta{{\mathbf{\theta}}}
\def\rva{{\mathbf{a}}}

% Elements of random vectors
\def\erva{{\textnormal{a}}}
\def\ervb{{\textnormal{b}}}

% Random matrices
\def\rmA{{\mathbf{A}}}
\def\rmB{{\mathbf{B}}}

% Elements of random matrices
\def\ermA{{\textnormal{A}}}
\def\ermB{{\textnormal{B}}}

\def\fvec{{\mathbf{f}}}
\def\bff{{\mathbf{f}}}
\def\bfg{{\mathbf{g}}}
% Vectors
\def\vzero{{\bm{0}}}
\def\vone{{\bm{1}}}
\def\vmu{{\bm{\mu}}}
\def\vtheta{{\bm{\theta}}}
\def\va{{\bm{a}}}
\def\vb{{\bm{b}}}
\def\vc{{\bm{c}}}
\def\vd{{\bm{d}}}
\def\ve{{\bm{e}}}
\def\vf{{\bm{f}}}
\def\vg{{\bm{g}}}
\def\vh{{\bm{h}}}
\def\vi{{\bm{i}}}
\def\vj{{\bm{j}}}
\def\vk{{\bm{k}}}
\def\vl{{\bm{l}}}
\def\vm{{\bm{m}}}
\def\vn{{\bm{n}}}
\def\vo{{\bm{o}}}
\def\vp{{\bm{p}}}
\def\vq{{\bm{q}}}
\def\vr{{\bm{r}}}
\def\vs{{\bm{s}}}
\def\vt{{\bm{t}}}
\def\vu{{\bm{u}}}
\def\vv{{\bm{v}}}
\def\vw{{\bm{w}}}
\def\vx{{\bm{x}}}
\def\vy{{\bm{y}}}
\def\vz{{\bm{z}}}

% Matrix
\def\mA{{\bm{A}}}

% Tensor
\DeclareMathAlphabet{\mathsfit}{\encodingdefault}{\sfdefault}{m}{sl}
\SetMathAlphabet{\mathsfit}{bold}{\encodingdefault}{\sfdefault}{bx}{n}
\newcommand{\tens}[1]{\bm{\mathsfit{#1}}}
\def\tA{{\tens{A}}}
\def\tB{{\tens{B}}}
\def\tC{{\tens{C}}}
\def\tD{{\tens{D}}}
\def\tE{{\tens{E}}}
\def\tF{{\tens{F}}}
\def\tG{{\tens{G}}}
\def\tH{{\tens{H}}}
\def\tI{{\tens{I}}}
\def\tJ{{\tens{J}}}
\def\tK{{\tens{K}}}
\def\tL{{\tens{L}}}
\def\tM{{\tens{M}}}
\def\tN{{\tens{N}}}
\def\tO{{\tens{O}}}
\def\tP{{\tens{P}}}
\def\tQ{{\tens{Q}}}
\def\tR{{\tens{R}}}
\def\tS{{\tens{S}}}
\def\tT{{\tens{T}}}
\def\tU{{\tens{U}}}
\def\tV{{\tens{V}}}
\def\tW{{\tens{W}}}
\def\tX{{\tens{X}}}
\def\tY{{\tens{Y}}}
\def\tZ{{\tens{Z}}}


% Graph
\def\gA{{\mathcal{A}}}
\def\gB{{\mathcal{B}}}
\def\gC{{\mathcal{C}}}
\def\dataset{{\mathcal{D}}}
\def\gE{{\mathcal{E}}}
\def\gF{{\mathcal{F}}}
\def\fourier{{\mathcal{F}}}
\def\gG{{\mathcal{G}}}
\def\gH{{\mathcal{H}}}
\def\gI{{\mathcal{I}}}
\def\gJ{{\mathcal{J}}}
\def\gK{{\mathcal{K}}}
\def\gL{{\mathcal{L}}}
\def\loss{{\mathcal{L}}}
\def\gM{{\mathcal{M}}}
\def\gN{{\mathcal{N}}}
\def\normal{{\mathcal{N}}}
\def\gaussian{{\mathcal{N}}}
\def\gO{{\mathcal{O}}}
\def\gP{{\mathcal{P}}}
\def\gQ{{\mathcal{Q}}}
\def\gR{{\mathcal{R}}}
\def\gS{{\mathcal{S}}}
\def\gT{{\mathcal{T}}}
\def\gU{{\mathcal{U}}}
\def\uniform{{\mathcal{U}}}
\def\gV{{\mathcal{V}}}
\def\gW{{\mathcal{W}}}
\def\gX{{\mathcal{X}}}
\def\gY{{\mathcal{Y}}}
\def\gZ{{\mathcal{Z}}}

\def\algebra{{\mathscr{A}}}
\def\borel{{\mathscr{B}}}
\def\manifold{{\mathscr{M}}}

% Sets
\def\sA{{\mathbb{A}}}
\def\sB{{\mathbb{B}}}
\def\complex{{\mathbb{C}}}
\def\sD{{\mathbb{D}}}
\def\expectation{{\mathbb{E}}}
\newcommand{\E}{\mathbb{E}}
\def\sF{{\mathbb{F}}}
\def\sG{{\mathbb{G}}}
\def\sH{{\mathbb{H}}}
\def\sI{{\mathbb{I}}}
\def\sJ{{\mathbb{J}}}
\def\sK{{\mathbb{K}}}
\def\sL{{\mathbb{L}}}
\def\sM{{\mathbb{M}}}
\def\natural{{\mathbb{N}}}
\def\sO{{\mathbb{O}}}
\def\sP{{\mathbb{P}}}
\def\rational{{\mathbb{Q}}}
\def\real{{\mathbb{R}}}
\newcommand{\R}{\mathbb{R}}
\def\sS{{\mathbb{S}}}
\def\sphere{{\mathbb{S}}}
\def\sT{{\mathbb{T}}}
\def\sU{{\mathbb{U}}}
\def\sV{{\mathbb{V}}}
\def\sW{{\mathbb{W}}}
\def\sX{{\mathbb{X}}}
\def\sY{{\mathbb{Y}}}
\def\integer{{\mathbb{Z}}}
\def\indicator{{\mathbbm{1}}}

% Entries of a matrix
\def\emLambda{{\Lambda}}
\def\emA{{A}}
\def\emB{{B}}
\def\emC{{C}}
\def\emD{{D}}
\def\emE{{E}}
\def\emF{{F}}
\def\emG{{G}}
\def\emH{{H}}
\def\emI{{I}}
\def\emJ{{J}}
\def\emK{{K}}
\def\emL{{L}}
\def\emM{{M}}
\def\emN{{N}}
\def\emO{{O}}
\def\emP{{P}}
\def\emQ{{Q}}
\def\emR{{R}}
\def\emS{{S}}
\def\emT{{T}}
\def\emU{{U}}
\def\emV{{V}}
\def\emW{{W}}
\def\emX{{X}}
\def\emY{{Y}}
\def\emZ{{Z}}
\def\emSigma{{\Sigma}}

% entries of a tensor
% Same font as tensor, without \bm wrapper
\newcommand{\etens}[1]{\mathsfit{#1}}
\def\etLambda{{\etens{\Lambda}}}
\def\etA{{\etens{A}}}
\def\etB{{\etens{B}}}
\def\etC{{\etens{C}}}
\def\etD{{\etens{D}}}
\def\etE{{\etens{E}}}
\def\etF{{\etens{F}}}
\def\etG{{\etens{G}}}
\def\etH{{\etens{H}}}
\def\etI{{\etens{I}}}
\def\etJ{{\etens{J}}}
\def\etK{{\etens{K}}}
\def\etL{{\etens{L}}}
\def\etM{{\etens{M}}}
\def\etN{{\etens{N}}}
\def\etO{{\etens{O}}}
\def\etP{{\etens{P}}}
\def\etQ{{\etens{Q}}}
\def\etR{{\etens{R}}}
\def\etS{{\etens{S}}}
\def\etT{{\etens{T}}}
\def\etU{{\etens{U}}}
\def\etV{{\etens{V}}}
\def\etW{{\etens{W}}}
\def\etX{{\etens{X}}}
\def\etY{{\etens{Y}}}
\def\etZ{{\etens{Z}}}

\def\ceil#1{\lceil #1 \rceil}
\def\floor#1{\lfloor #1 \rfloor}
\def\eps{{\epsilon}}

\newcommand{\pder}[1]{\frac{\partial}{\partial #1}}

\newcommand{\half}{\frac{1}{2}}
\newcommand{\limNinf}{\lim_{N \to \infty}}
\newcommand{\limTzero}{\lim_{\tau \to 0}}


\newcommand{\cmark}{\ding{51}}
\newcommand{\xmark}{\ding{55}}

\newcommand{\layer}{\mathcal{H}}
\newcommand{\defeq}{\triangleq}
%\newcommand{\defeq}{vcentcolon=}
\newcommand{\domain}{\Omega}
\newcommand{\grad}{\nabla}

\newcommand{\cin}{c_{\rm{in}}}
\newcommand{\cout}{c_{\rm{out}}}
\newcommand{\intdomain}{\int_{\domain}}
\newcommand{\network}{\gT}
\newcommand{\subnet}{\gK}
\newcommand{\map}{\gR} %\gR

\newcommand{\innerproduct}[2]{\langle #1, #2 \rangle}
\newcommand{\mcsum}[1][j]{\frac{1}{N}\sum_{#1=1}^N}

\newcommand{\inrspace}[1][c]{\gF_{#1}}

\DeclareMathOperator*{\argmax}{arg\,max}
\DeclareMathOperator*{\argmin}{arg\,min}

\let\ab\allowbreak


% Handy macros...

% latex short form notation
\newcommand{\tnm}[1]{{\tablenotemark{#1}}}
\newcommand{\tnt}[2]{{\tablenotetext{#1}{#2}}}
\newcommand{\fig}[2]{{fig.\,{#1}{#2}}}
\newcommand{\tab}[1]{{table\,{#1}}}
\newcommand{\eqn}[1]{{eq.\,{#1}}}

%places
\newcommand{\Hawaii}{{Hawai`i}}

%abbreviations
\newcommand{\eg}{{\it e.g.}}
\newcommand{\ie}{{\it i.e.}}
\newcommand{\etc}{{\it etc.}}
\newcommand{\etal}{{\it et~al.}}
\newcommand{\adhoc}{{\it ad~hoc}}
\newcommand{\insitu}{{\it in situ}}
\newcommand{\apriori}{{\it a~priori}}
\newcommand{\postfacto}{{\it post~facto}}

%math
\newcommand{\half}{{\frac{1}{2}}}
\newcommand{\mean}[1]{\langle{#1}\rangle}
\newcommand{\dif}{\mathrm{d}}
%\newcommand{\iint}[2]{\int\!\!\!\int\limits_{#1}^{#2}}
%\newcommand{\iiint}[2]{\int\!\!\!\int\limits_{#1}^{#2}\!\!\!\int}
\newcommand{\overbar}[1]{\mkern 1.5mu\overline{\mkern-1.5mu#1\mkern-1.5mu}\mkern 1.5mu}
\newcommand{\oforder}{\mathcal{O}}
% for scientific notation.  e.g.  x \times 10^y
\newcommand{\sci}[2]{{{#1}\times10^{#2}}}

%orbits
\newcommand{\xbar}{\bar x}
\newcommand{\xyz}{(x,y,z)}
\newcommand{\xyzdot}{(\dot{x},\dot{y},\dot{z})}
\newcommand{\aei}{(a,e,i)}
\newcommand{\qei}{(q,e,i)}
\newcommand{\aeiH}{(a,e,i,H)}
\newcommand{\qeiD}{(q,e,i,D)}
\newcommand{\OoM}{(\Omega,\omega,M)}
\newcommand{\vo}{\vec{o}}
\newcommand{\vx}{\vec{x}}
\newcommand{\vxdot}{\vec{\dot x}}
\newcommand{\vxavg}{\mean{\vec{x}}}
\newcommand{\vy}{\vec{y}}
\newcommand{\vyavg}{\mean{\vec{y}}}
\newcommand{\vz}{\vec{z}}
\newcommand{\vzavg}{\mean{\vec{z}}}
\newcommand{\Havg}{\mean{H}}
\newcommand{\deltav}{\Delta v}
\newcommand{\node}{\Omega}
\newcommand{\aperi}{\omega}
\newcommand{\lperi}{\tilde\omega}

%editing
\newcommand{\todo}[2]{{\color{red}\bf #1 - #2}}
%\newcommand{\todo}[2]{{}}
\newcommand{\note}[1]{{\color{blue}{\bf #1}}}
%\newcommand{\note}[1]{{}}
\newcommand{\XXX}{{\color{red}\bf XXX}}
% these commands kind of mimic \citep and \citet but are not available yet (na)
\newcommand{\citepna}[1]{{{\color{red} (#1)}}}
\newcommand{\citetna}[1]{{{\color{red} #1}}}

%miscellaneous
\newcommand{\sn}{$S/N$}
\newcommand{\SN}{$S/N$}
%\newcommand{\th}{$^{th}$}  % already defined somewhere else?
\newcommand{\rd}{$^{rd}$}
\newcommand{\st}{$^{st}$}
\newcommand{\nd}{$^{nd}$}
\newcommand{\fManx}{{\mathrm{f}_\mathrm{Manx}}}
\newcommand{\digesttwo}{{\texttt{digest2}}}

%molecules
\newcommand{\HtwoO}{{H$_2$O}}
\newcommand{\CO}{{CO}}
\newcommand{\COtwo}{{CO$_2$}}

%units
\newcommand{\arcdeg}{{^{\circ}}}
\newcommand{\arcmin}{^{\prime}}
\newcommand{\arcsec}{^{\prime\prime}}
\newcommand{\h}{^\mathrm{h}}
\newcommand{\m}{^\mathrm{m}}
\newcommand{\s}{^\mathrm{s}}
\newcommand{\Mpc}{\,\mathrm{Mpc}}
\newcommand{\kpc}{\,\mathrm{kpc}}
\newcommand{\pc}{\,\mathrm{pc}}
\newcommand{\au}{\,\mathrm{au}}
\newcommand{\km}{\,\mathrm{km}}
\newcommand{\kph}{\,\mathrm{km}/\mathrm{h}}
\newcommand{\kps}{\,\mathrm{km}\,\mathrm{s}^{-1}}
\newcommand{\ft}{\,\mathrm{ft}}
\newcommand{\meter}{\,\mathrm{m}}
\newcommand{\cm}{\,\mathrm{cm}}
\newcommand{\mm}{\,\mathrm{mm}}
\newcommand{\um}{\,\mu \mathrm{m}}
\newcommand{\nm}{\,\mathrm{nm}}
\newcommand{\rad}{\,\mathrm{rad}}
\newcommand{\rms}{\,\mathrm{(rms)}}
\newcommand{\anno}{\,\mathrm{a}}
\newcommand{\yr}{\,\mathrm{yr}}
\newcommand{\Myr}{\,\mathrm{Myr}}
\newcommand{\Gyr}{\,\mathrm{Gyr}}
\newcommand{\Day}{\,\mathrm{day}}
\newcommand{\days}{\,\mathrm{d}}
\newcommand{\dayperyear}{\,\mathrm{d}/\mathrm{yr}}
\newcommand{\vrk}{\,\mathrm{vrk}}
\newcommand{\degrees}{\,\mathrm{deg}}
\newcommand{\hours}{\,hours}
\newcommand{\hour}{\,\mathrm{h}}
\newcommand{\hourperday}{\,\mathrm{h}/\mathrm{d}}
\newcommand{\minute}{\,\mathrm{min}}
\newcommand{\second}{\,\mathrm{s}}
\newcommand{\mps}{\,\meter\,\second^{-1}}
\newcommand{\Hz}{\,\mathrm{Hz}}
\newcommand{\mags}{\,\mathrm{mag}}
\newcommand{\K}{\,\mathrm{K}}
\newcommand{\J}{\,\mathrm{J}}
\newcommand{\N}{\,\mathrm{N}}
\newcommand{\kg}{\,\mathrm{kg}}
\newcommand{\g}{\,\mathrm{g}}
\newcommand{\AMU}{\,\mathrm{AMU}}
\newcommand{\W}{\,\mathrm{W}}
\newcommand{\MW}{\,\mathrm{MW}}
\newcommand{\degC}{\arcdeg\mathrm{C}}
\newcommand{\degK}{\mathrm{K}}
\newcommand{\Jy}{\,\mathrm{Jy}}
\newcommand{\mJy}{\,\mathrm{mJy}}
\newcommand{\Mearth}{\,\mathrm{M}_\oplus}

%asteroids
\newcommand{\asteroid}[2]{{({#1})\,{#2}}}
\newcommand{\designation}[2]{{{#1}\,{#2}}}
\newcommand{\TC}{{2008\,TC$_3$}}
\newcommand{\RH}{{2006\,RH$_{120}$}}
\newcommand{\Bennu}{{(101955)\,Bennu}}

%comets
\newcommand{\Sthree}{{C/2014$\,$S3$\,$PANSTARRS}}
\newcommand{\Uone}{{1I/2017 U1 (‘Oumuamua)}}


% from
% PS-1 Latex template v. 7.0 C. Stubbs June 14 2011 (with thanks to Tex-gurus Michael Wood-Vasey, Gautham Narayan, and Ryan Foley)

%filters
\newcommand{\gps}{\ensuremath{g_{\rm P1}}}
\newcommand{\rps}{\ensuremath{r_{\rm P1}}}
\newcommand{\ips}{\ensuremath{i_{\rm P1}}}
\newcommand{\zps}{\ensuremath{z_{\rm P1}}}
\newcommand{\yps}{\ensuremath{y_{\rm P1}}}
\newcommand{\wps}{\ensuremath{w_{\rm P1}}}
\newcommand{\grizy}{\gps\rps\ips\zps\yps}
\newcommand{\JHK}{\ensuremath{JHK}}
\newcommand{\V}{\ensuremath{V}}

% Pan-STARRS
\newcommand{\PS}{\protect \hbox {Pan-STARRS}}
\newcommand{\PSone}{\protect \hbox {Pan-STARRS1}}
\newcommand{\PStwo}{\protect \hbox {Pan-STARRS2}}
\newcommand{\PSfour}{\protect \hbox {Pan-STARRS4}}
\newcommand{\knownserver}{{\tt known\_server}}

% journals
%\newcommand{\aap}{\protect \hbox {\it Astronomy and Astrophysics}}
%\newcommand{\aaps}{\protect \hbox {\it Astronomy and Astrophysics, Supplement Series}}
%\newcommand{\pasp}{\protect \hbox {\it PASP}}
%\newcommand{\aj}{\protect \hbox {\it Astronomical Journal}}
%\newcommand{\apj}{\protect \hbox {\it Astrophysical Journal}}
%\newcommand{\apjl}{\protect \hbox {\it Astrophysical Journal, Letters to the Editor}}
%\newcommand{\apjs}{\protect \hbox {\it Astrophysical Journal, Supplement Series}}
%\newcommand{\nat}{\protect \hbox {\it Nature}}
%\newcommand{\planss}{\protect \hbox {\it Planet.~Space~Sci.}}
%\newcommand{\icarus}{\protect \hbox {\it Icarus}}
%\newcommand{\jrasc}{\protect \hbox {\it Journal of the Royal Astronomical Society of Canada}}
%\newcommand{\mnras}{\protect \hbox {\it Monthly Notices of the RAS}}
%\newcommand{\jgr}{\protect \hbox {\it Journal of Geophysical Research}}
%\newcommand{\pasj}{\protect \hbox {\it Publications of the Astronomical Society of Japan}}
%\newcommand{\ssr}{\protect \hbox {\it Space Science Reviews}}


\newcommand\aj{AJ}%        % Astronomical Journal 
\newcommand\psj{{PSJ}}%       % Planetary Science Journal
\newcommand\araa{{ARA\&A}}%  % Annual Review of Astron and Astrophys 
\renewcommand\apj{{ApJ}}%    % Astrophysical Journal 
\newcommand\apjl{{ApJL}}     % Astrophysical Journal, Letters 
\newcommand\apjs{{ApJS}}%    % Astrophysical Journal, Supplement 
\renewcommand\ao{{ApOpt}}%   % Applied Optics 
\newcommand\apss{{Ap\&SS}}%  % Astrophysics and Space Science 
\newcommand\aap{{A\&A}}%     % Astronomy and Astrophysics 
\newcommand\aapr{{A\&A~Rv}}%  % Astronomy and Astrophysics Reviews 
\newcommand\aaps{{A\&AS}}%    % Astronomy and Astrophysics, Supplement 
\newcommand\azh{{AZh}}%       % Astronomicheskii Zhurnal 
\newcommand\baas{{BAAS}}%     % Bulletin of the AAS 
\newcommand\icarus{{Icarus}}% % Icarus
\newcommand\jaavso{{JAAVSO}}  % The Journal of the American Association of Variable Star Observers
\newcommand\jrasc{{JRASC}}%   % Journal of the RAS of Canada 
\newcommand\memras{{MmRAS}}%  % Memoirs of the RAS 
\newcommand\mnras{{MNRAS}}%   % Monthly Notices of the RAS 
\renewcommand\pra{{PhRvA}}% % Physical Review A: General Physics 
\renewcommand\prb{{PhRvB}}% % Physical Review B: Solid State 
\renewcommand\prc{{PhRvC}}% % Physical Review C 
\renewcommand\prd{{PhRvD}}% % Physical Review D 
\renewcommand\pre{{PhRvE}}% % Physical Review E 
\renewcommand\prl{{PhRvL}}% % Physical Review Letters 
\newcommand\pasp{{PASP}}%     % Publications of the ASP 
\newcommand\pasj{{PASJ}}%     % Publications of the ASJ 
\newcommand\qjras{{QJRAS}}%   % Quarterly Journal of the RAS 
\newcommand\skytel{{S\&T}}%   % Sky and Telescope 
\newcommand\solphys{{SoPh}}% % Solar Physics 
\newcommand\sovast{{Soviet~Ast.}}% % Soviet Astronomy 
\newcommand\ssr{{SSRv}}% % Space Science Reviews 
\newcommand\zap{{ZA}}%       % Zeitschrift fuer Astrophysik 
\renewcommand\nat{{Nature}}%  % Nature 
\newcommand\iaucirc{{IAUC}}% % IAU Cirulars 
\newcommand\aplett{{Astrophys.~Lett.}}%  % Astrophysics Letters 
\newcommand\apspr{{Astrophys.~Space~Phys.~Res.}}% % Astrophysics Space Physics Research 
\newcommand\bain{{BAN}}% % Bulletin Astronomical Institute of the Netherlands 
\newcommand\fcp{{FCPh}}%   % Fundamental Cosmic Physics 
\newcommand\gca{{GeoCoA}}% % Geochimica Cosmochimica Acta 
\newcommand\grl{{Geophys.~Res.~Lett.}}%  % Geophysics Research Letters 
\renewcommand\jcp{{JChPh}}%     % Journal of Chemical Physics 
\newcommand\jgr{{J.~Geophys.~Res.}}%     % Journal of Geophysics Research 
\newcommand\jqsrt{{JQSRT}}%   % Journal of Quantitiative Spectroscopy and Radiative Trasfer 
\newcommand\memsai{{MmSAI}}% % Mem. Societa Astronomica Italiana 
\newcommand\nphysa{{NuPhA}}%     % Nuclear Physics A 
\newcommand\physrep{{PhR}}%       % Physics Reports 
\newcommand\physscr{{PhyS}}%        % Physica Scripta 
\newcommand\planss{{Planet.~Space~Sci.}}%  % Planetary Space Science 
\newcommand\procspie{{Proc.~SPIE}}%      % Proceedings of the SPIE 

\newcommand\actaa{{AcA}}%  % Acta Astronomica
\newcommand\caa{{ChA\&A}}%  % Chinese Astronomy and Astrophysics
\newcommand\cjaa{{ChJA\&A}}%  % Chinese Journal of Astronomy and Astrophysics
\newcommand\jcap{{JCAP}}%  % Journal of Cosmology and Astroparticle Physics
\newcommand\na{{NewA}}%  % New Astronomy
\newcommand\nar{{NewAR}}%  % New Astronomy Review
\newcommand\pasa{{PASA}}%  % Publications of the Astron. Soc. of Australia
\newcommand\rmxaa{{RMxAA}}%  % Revista Mexicana de Astronomia y Astrofisica

%% added feb 9, 2016
\newcommand\maps{{M\&PS}}% Meteoritics and Planetary Science
\newcommand\aas{{AAS Meeting Abstracts}}% American Astronomical Society Meeting Abstracts
\newcommand\dps{{AAS/DPS Meeting Abstracts}}% American Astronomical Society/Division for Planetary Sciences Meeting Abstracts
\usepackage{bm}
\usepackage{amsmath}
\usepackage{amssymb}
\usepackage{mathtools}
\usepackage{amsthm}
\usepackage{booktabs}
\usepackage{tablefootnote}
\usepackage{subcaption}
\usepackage{hyperref}
\usepackage{url}
\usepackage{wrapfig}
\usepackage{algorithm}

\let\classAND\AND
\let\AND\relax
\usepackage{algorithmic}
\let\algoAND\AND
\let\AND\classAN
\AtBeginEnvironment{algorithmic}{\let\AND\algoAND}


\title{mL-BFGS: A \underline{M}omentum-based L-BFGS for Distributed Large-Scale Neural Network Optimization}

% Authors must not appear in the submitted version. They should be hidden
% as long as the tmlr package is used without the [accepted] or [preprint] options.
% Non-anonymous submissions will be rejected without review.

\author{\name Yue Niu \email yueniu@usc.edu \\
      \addr Department of Electrical and Computer Engineering\\
      University of Southern California \\
      \name Zalan Fabian \email zfabian@usc.edu \\
      \addr Department of Electrical and Computer Engineering \\
      University of Southern California \\
      \name Sunwoo Lee \email sunwool@inha.ac.kr\\
      \addr Department of Computer Science and Engineering, \\
      Inha University \\
      \name Mahdi Soltanolkotabi \email soltanol@usc.edu \\
      \addr Department of Electrical and Computer Engineering \\
      University of Southern California \\
      \name Salman Avestimehr \email avestime@usc.edu \\
      \addr Department of Electrical and Computer Engineering \\
      University of Southern California
      }

% The \author macro works with any number of authors. Use \AND 
% to separate the names and addresses of multiple authors.

\newcommand{\fix}{\marginpar{FIX}}
\newcommand{\new}{\marginpar{NEW}}

\def\month{July}  % Insert correct month for camera-ready version
\def\year{2023} % Insert correct year for camera-ready version
\def\openreview{\url{https://openreview.net/forum?id=9jnsPp8DP3}} % Insert correct link to OpenReview for camera-ready version


\begin{document}

\theoremstyle{plain}
\newtheorem{theorem}{Theorem}
\newtheorem{proposition}{Proposition}
\newtheorem{lemma}{Lemma}
\newtheorem{corollary}{Corollary}
\theoremstyle{definition}
\newtheorem{definition}{Definition}

\theoremstyle{remark}
\newtheorem{remark}{Remark}
\newtheorem{assumption}{AS}

\maketitle

\begin{abstract}
Quasi-Newton methods still face significant challenges in training large-scale neural networks due to additional compute costs in the Hessian related computations and instability issues in stochastic training.
A well-known method, L-BFGS that efficiently approximates the Hessian using history parameter and gradient changes, suffers convergence instability in stochastic training.
So far, attempts that adapt L-BFGS to large-scale stochastic training incur considerable extra overhead, which offsets its convergence benefits in wall-clock time.
In this paper, we propose \method{}, a lightweight momentum-based L-BFGS algorithm that paves the way for quasi-Newton (QN) methods in large-scale distributed deep neural network (DNN) optimization. 
\method{} introduces a nearly cost-free momentum scheme into L-BFGS update and greatly reduces stochastic noise in the Hessian, therefore stabilizing convergence during stochastic optimization.
For model training at a large scale, \method{} approximates a block-wise Hessian, thus enabling distributing compute and memory costs across all computing nodes.
We provide a supporting convergence analysis for \method{} in stochastic settings.
To investigate \method{}’s potential in large-scale DNN training, we train benchmark neural models using \method{} and compare performance with baselines (SGD, Adam, and other quasi-Newton methods). 
Results show that \method{} achieves both noticeable iteration-wise and wall-clock speedup.
\end{abstract}

\section{Introduction}\label{sec:intro}
\section{Introduction}

% Figure environment removed

Reinforcement Learning from Human Feedback (RLHF) has recently been used to great effect to align pretrained large language models (LLMs) to human preferences, optimizing for desirable qualities like harmlessness and helpfulness~\citep{bai2022training} and achieving state-of-the-art results across a variety of natural language tasks~\citep{openai2023gpt4}. %RLHF approaches fundamentally rely on collecting pairs of LLM outputs $(o_1, o_2)$ from a shared prompt $p$, with a human indicating which output in each pair is better on a specified attribute.
% A fundamental component of RLHF is a preference model derived from human labels, typically formatted as pairs of LLM outputs $(o_1, o_2)$ generated from a shared prompt $p$.

A standard RLHF procedure fine-tunes an initial unaligned LLM using an RL algorithm such as PPO~\citep{schulman2017proximal}, optimizing the LLM to align with human preferences. %\violet{not sure whether we need to provide this detail in the intro, especially this has nothing to do with our contribution.} % i feel like this context is useful later when e.g. explaining that context distillation is SFT
RLHF is thus critically dependent on a reward model derived from human-labeled preferences, typically \textit{pairwise preferences} on LLM outputs $(o_1, o_2)$ generated from a shared prompt $p$. % and labeled by humans. 

However, collecting human pairwise preference data, especially high-quality data, may be expensive and time consuming at scale. To address this problem, approaches have been proposed to obtain labels without human annotation, such as Reinforcement Learning from AI Feedback (RLAIF) and context distillation. 

\iffalse
raising the question of whether we can generate high-quality data for RLHF without using human labeling. %accurately-labeled preference pairs $(o_1, o_2)$
%, motivating model alignment approaches that aim to generate accurately-labeled preference pairs $(o_1, o_2)$ without human involvement. 
Two major categories of such approaches are . 
\fi

RLAIF approaches (e.g.,~\citet{bai2022constitutional}) simulate human pairwise preferences by scoring $o_1$ and $o_2$ with an LLM (Figure \ref{fig:rlcd_differences} center); the scoring LLM is often the same as the one used to generate the original pairs $(o_1, o_2)$. Of course, the resulting LLM pairwise preferences will be somewhat noisier compared to human labels. However, this problem is exacerbated by using the same prompt $p$ to generate both $o_1$ and $o_2$, causing $o_1$ and $o_2$ to often be of very similar quality and thus hard to differentiate (e.g., Table~\ref{tab:rlaif_bad_example}). Consequently, training signal can be overwhelmed by label noise, yielding lower-quality preference data. 

% While it avoids human labeling efforts, it has weakness. First, LLM preference labels will naturally be somewhat noisier compared to human labels. Furthermore, since the same prompt $p$ is used to generate both $o_1$ and $o_2$, their quality is often very similar and hard to differentiate (See Table~\ref{tab:rlaif_bad_example}). As a result, training signals can be overwhelmed by label noise, yielding lower-quality preference data. 

Meanwhile, context distillation methods (e.g., \citet{sun2023principle}) create more training signal by modifying the initial prompt $p$. 
%to create more significant training signal. 
The modified prompt $p_+$ typically contains additional context encouraging a \textit{directional attribute change} in the output $o_+$ (Figure \ref{fig:rlcd_differences} right). However, context distillation methods only generate a single output $o_+$ per prompt $p_+$, which is then used for supervised fine-tuning, losing the pairwise preferences which help RLHF-style approaches to 
%rather than using a RLHF-style preference model to 
derive signal from the contrast between outputs. 
Multiple works have observed that RL approaches using preference models for pairwise preferences can substantially improve over supervised fine-tuning by itself when aligning LLMs~\citep{ouyang2022training,dubois2023alpacafarm}. 

% conduct alignment by running supervised fine-tuning on model outputs $o_+$ generated from a modified prompt $p_+$. $p_+$ typically contains additional context encouraging desirable attributes (Figure \ref{fig:rlcd_differences} right), such as in \citet{sun2023principle}. However, multiple works have observed that RLHF-style approaches can substantially improve over supervised fine-tuning by itself when aligning LLMs~\citep{ouyang2022training,dubois2023alpacafarm}. 

Therefore, while both RLAIF and context distillation approaches have already been successfully applied in practice to align language models, we posit that it may be even more effective to combine the key advantages of both. That is, we will use RL with \textit{pairwise preferences}, while also using modified prompts to encourage \textit{directional attribute change} in outputs. %In particular, we will adapt the RLAIF data generation process with two different prompts rather than a single $p$, modifying both prompts similarly to context distillation. %\violet{this motivation is a little unexciting. I think we can more specifically discuss the potential benefits of our approach, like the benefits from RL: exploration/data generation; benefits from contrast. I don't think we get too much benefits from context distillation since we switched to the RL framework.} 

Concretely, we propose \oursfull{} (\ours{}). 
\ours{} generates preference data as follows. Rather than producing two i.i.d.\ model outputs $(o_1, o_2)$ from the same prompt $p$ as in RLAIF, \ours{} creates two variations of $p$: a \textit{positive prompt} $p_+$ similar to context distillation which encourages directional change toward a desired attribute, and a \textit{negative prompt} $p_-$ which encourages directional change \textit{against} it (Figure \ref{fig:rlcd_differences} left). We then generate model outputs $(o_+, o_-)$ respectively, and automatically label $o_+$ as preferred---that is, \ours{} automatically ``generates'' pairwise preference labels by construction. %, without further post hoc labeling.\violet{should make it clearer that our approach `generates' labels by construction} 
We then follow the standard RL pipeline of training a preference model followed by PPO. 

Compared to RLAIF-generated preference pairs $(o_1, o_2)$ from the same input prompt $p$, there is typically a clearer difference in the quality of $o_+$ and $o_-$ generated using \ours{}'s directional prompts $p_+$ and $p_-$, which may result in less label noise. %which may result in better training signal for the preference model. 
That is, intuitively, \ours{} exchanges having examples be \textit{closer to the classification boundary} for much more \textit{accurate labels} on average. Compared to standard context distillation methods, on top of leveraging pairwise preferences for RL training, \ours{} can derive signal not only from the positive prompt $p_+$ which improves output quality, but also from the negative prompt $p_-$ which degrades it. %\ours{} is not learning to imitate $o_+$, but to distill the \textit{contrast} between $o_+$ and $o_-$. 
Positive outputs $o_+$ don't need to be perfect; they only need to contrast with $o_-$ on the desired attribute while otherwise following a similar style.

% \todo{discuss our method and why intuitively it may be better.}

We evaluate the practical effectiveness of \ours{} through both human and automatic evaluations on three tasks, aiming to improve the ability of LLaMA-7B~\citep{touvron2023llama} to generate harmless outputs, helpful outputs, and high-quality story outlines. %\ours{} outperforms both RLAIF and context distillation baselines in pairwise comparisons on 
As shown in Sec. \ref{sec:experiments}, \ours{} substantially outperforms both RLAIF and context distillation baselines in pairwise comparisons when simulating preference data with LLaMA-7B, while still performing equal or better when simulating with LLaMA-30B. 
%On all three tasks, \ours{} substantially outperforms both RLAIF and context distillation baselines in pairwise comparisons---by a margin of at least 9\% and often more than 30\%---validating our method's efficacy. 
We will release all code at a later date, although in any case \ours{} is fairly easy to implement by modifying any reference RLAIF codebase. %We release all code at \todo{github link}.



\section{Preliminaries}\label{sec:prem}
% In this paper we consider a typical empirical loss minimization problem of the form $\underset{\theta}{\min}\text{ }\mathcal{L}(\btheta, \mathcal{X}) := \frac{1}{N}\sum_{i=1}^N\ell(\btheta, \bm{x}_i)$,
%\begin{equation*}\label{eq::Loss}
%    \underset{\theta}{\min}\text{ }\mathcal{L}(\btheta, \mathcal{X}) := \frac{1}{N}\sum_{i=1}^N\ell(\btheta, \bm{x}_i),
%\end{equation*}
% where $\btheta$ denotes the parameters of the model to be optimized, and $\mathcal{X}=\left \{ \bm{x}_i \right \}_{i=1}^N$ are the training data where $\bm{x}_i$ consists of both features and labels. 
The risk function $\mathcal{L}(\btheta, \mathcal{X})$ in Eq (\ref{eq:loss}) is usually optimized through a form of gradient descent as:
\begin{equation}\label{eq:ParamUpdateQN}
    \btheta_{t+1} = \btheta_{t} - \eta_{t}\cdot \HI\cdot\bm{g}_t,
\end{equation}
where $\eta_{t}$ denotes step size (learning rate) at iteration $t$ and $\HI$ is a gradient pre-conditioner.
% that uses the local gradients $\bm{g}_t=\nabla_{\btheta} \mathcal{L}(\btheta_t)$ directly to update the model parameters via iterates of the form  
%\begin{equation}\label{eq::ParamUpdateGD}
%    \btheta_{t+1} = \btheta_{t} - \eta_{t}\bm{g}_t,
%\end{equation}
%where $\eta_{t}$ denotes the step size (learning rate) at iteration $t$. However, such GD updates are typically slow especially for ill-conditioned problems \citep{convexOpt}. To speed up the convergence, often second-order methods are used. In particular, Quasi-Newton (QN) methods find an approximate Hessian inverse $\HI$ to pre-condition the gradient vector and apply the following update to minimize the loss:
In stochastic training, gradients are evaluated on a mini-batch input $\tX_t \subseteq \mathcal{X}$, namely $\bm{g}_t=\nabla_{\btheta} \mathcal{L}(\btheta_t, \tX_t)$. 

If $\HI$ is an identity matrix, the update above is reduced to SGD, whereas if $\HI$ is a diagonal matrix, it becomes an adaptive training algorithm such as Adagrad \citep{2011_JMLR_AdaGrad} or Adam \citep{2014_arXiv_Adam}. 
To further improve convergence performance, esp. in an ill-conditioned problem \citep{convexOpt}, it is desired to incorporate more second-order information into $\HI$ as done in quasi-Newton (QN) methods.

A prime challenge in QN methods is the evaluation of $\Hm$ and in particular its inverse. 
A well-known Broyden–Fletcher–Goldfarb–Shanno (BFGS) algorithm \citep{BFGS} addresses the challenge by formulating the Hessian inverse as a minimization problem:
\begin{equation}\label{eq:BFGSformulation}
\begin{split}
    & \min_{\HI} \quad\left \| \HI - \HI_{k-1}\right \|^2, \\
    & \text{s.t.} \quad\HI\cdot \bm{y}_k = \bm{s}_k,\quad \HI \text{ is symmetric},
\end{split}
\end{equation}
where $\bm{s}_k = \btheta_k - \btheta_{k-1}$ denotes the parameter changes, and $\bm{y}_k = \bm{g}_k-\bm{g}_{k-1} $ the gradient changes in two consecutive updates \footnote{$k$ rather than $t$ is used in the equation as parameter/gradient used might be different from the one in  Eq (\ref{eq:ParamUpdateQN})}.
By imposing the \emph{secant} condition during minimization, BFGS gradually attains the curvature information close to the real Hessian. \cite{2021_SIAM_Superlinear} establishes that BFGS converges to the real Hessian by a greedy strategy of choosing $(\vs_k, \vy_k)$.

Knowing $\HI_{k-1}$, the current $\HI$ is obtained via:
\begin{equation}\label{eq::BFGSupdate}
    \HI_k = (I - \rho_k\bm{y}_k\bm{s}_k^T)^T\HI_{k-1}(I-\rho_k\bm{y}_k\bm{s}_k^T)+\rho_k\bm{s}_k\bm{s}_k^T,
\end{equation}
where $\rho_k = \frac{1}{\bm{y}_k^T \bm{s}_k}$. 
Hence the Hessian inverse $\HI$ is constructed in an iterative manner with no need to compute the Hessian matrix.
% \red{Given such an update rule, methods such as Greedy BFGS \citep{2021_SIAM_Superlinear} show $\HI$ can converge to the real Hessian at a linear rate.}
% By carefully choosing $\bm{s}_k$, $\HI_k$ will converge to the real Hessian inverse at a linear rate for any strongly convex function \citep{2021_SIAM_Superlinear}. 

%In real-world problems, $\btheta$ usually consists of millions of parameters. As a result, it is infeasible to store the whole $\HI_k$ matrix with $O(\Dim{\btheta}^2)$ memory cost. 
To simplify computation in the Hessian-vector product, $\HI$ in BFGS is stored in the form of a sequence of history vectors $\left \{ \bm{y}_i \right \}$ and $\left \{ \bm{s}_i \right \}$. The matrix-vector product $\HI_k\cdot\bm{g}_t$ is replaced by a sequence of fast vector-vector products as shown in Algorithm~\ref{alg:HessianVecProd} (See Appendix \ref{appx:hessianvec}).
Furthermore, a limited-memory BFGS, L-BFGS \citep{1980_LBFGS} is usually adopted that only uses several latest history vectors when approximating the Hessian inverse.
% Furthermore, the number of history vectors ($\bm{y}_i$ and $\bm{s}_i$) that need to be stored for the Hessian approximation constantly increases in each iteration as the optimization continues, and will quickly exhaust our finite available hardware resources.
%To limit memory utilization and compute cost, a limited-memory version of BFGS, L-BFGS \cite{1980_LBFGS} is proposed that only uses the latest $M$ history vectors when approximating the Hessian inverse. The complete algorithm is shown in Algorithm~\ref{alg::HessianVecProd}.

\section{\method{}}\label{sec:method}
\section{Methodology}
\label{sec:method}

\subsection{Overview}
\label{sec:method_fmwk}

As shown in~\cref{fig:method_fmwk}, the proposed unsupervised MOT framework is trained with the widely-used contrastive learning technique~\cite{chen2020simple,he2020momentum}. 
\lk{Specifically, for multi-object tracking}, objects within the tracklet ($\boldsymbol{k}_{+}$) should be pulled together and different tracklets ($\boldsymbol{k}_{-}$) should be separated. It can be mathematically formulated as:

\begin{equation}
% \begin{split}
    \mathcal{L}_{cl}( \boldsymbol{q}; \boldsymbol{k}_{+}; \boldsymbol{k}_{-} )= 
    - \log \frac{\exp(\boldsymbol{q} \cdot \boldsymbol{k}_{+} / \epsilon)}{\sum_{i}\exp(\boldsymbol{q} \cdot \boldsymbol{k}_{i} / \epsilon)}
  \label{eq:method_nce}
% \end{split}  
\end{equation}

\noindent where $\mathcal{L}_{cl}$ denotes the InfoNCE~\cite{oord2018representation} loss function, and $\epsilon$ is the temperature hyper-parameter~\cite{wu2018unsupervised}. 
Within a video, following the unsupervised tracking fashion~\cite{liu2022online,shuai2022id}, the positive and negative keys mainly come from two sources, \ie pseudo-labeled historical frame and self-augmented frame. 

\lk{However, two issues occur: (1) the uncertainty reduces the accuracy of pseudo-tracklets and (2) the randomly augmented samples fail to learn the inter-frame consistency.} 
We argue the above issues are not independent. 
\lk{By leveraging the uncertainty in turn,} the accurate pseudo-tracklets can guide the qualified positive and negative augmentations.

To address these two issues, we propose an uncertainty-aware pseudo-tracklet labeling strategy in \cref{sec:method_uoap}, which integrates a verification-and-rectification mechanism into the tracklet generation. Our method significantly improves the accuracy of pseudo-identities, especially in long-term interval. 
Then we propose a tracklet-guided augmentation strategy in \cref{sec:method_ada_aug}, which brings the temporary information into spatial augmentation. The augmented samples simulates the objects' motion. A hierarchical uncertainty-based sampling strategy is proposed for hard sample mining. More details are described in the following section.


\subsection{Uncertainty-aware Tracklet-Labeling}
\label{sec:method_uoap}

Accurate pseudo tracklet is critical in \liuk{learning feature consistency}. 
However, without manual annotation, \lk{the aggravated uncertainty makes} the tracklet-labeling a huge challenge due to various interference factors, including similar appearance among objects, frequent object cross and occlusions, \etc. 
\lk{In fact, the uncertainty can also be leveraged to improve the pseudo-accuracy in turn.}
In this section, we propose an \textbf{U}ncertainty-aware \textbf{T}racklet-\textbf{L}abeling (\textbf{UTL}) strategy for better pseudo-tracklets.

Given an input video sequence $V = \{I^{1}, I^{2}, \cdots, I^{N}\}$, each frame $I^{t}$ is annotated with the bounding boxes $B^{t} = \{b_{1}^{t}, b_{2}^{t}, \cdots, b_{M^{t}}^{t}\}$ of $M^{t}$ objects in $t_{th}$ frame, where $b_{i}^{t} = (cx_{i}^{t}, cy_{i}^{t}, w_{i}^{t}, h_{i}^{t})$ is the center coordinate and shape of the $i_{th}$ object $o_{i}^{t}$. As shown in~\cref{fig:method_fmwk}, \mywork~generates accurate pseudo-tracklets in four main steps:

1) \textbf{Association}. For a certain object $o_{i}^{t}$ in frame $I^{t}$, the $\ell_2$-normalized representation $\boldsymbol{f}_{i}^{t}$ can be expressed as $\boldsymbol{f}_{i}^{t} = {\phi}(I^{t}, b_{i}^{t})$, 
% \begin{equation}
%   \boldsymbol{f}_{i}^{t} = {\phi}(I^{t}, b_{i}^{t})
%   % / {\Vert {\phi}(I^{t}, b_{i}^{t}) \Vert}_{2}
%   \label{eq:method_feat}
% \end{equation}
where the embedding encoder is denoted as $\phi$.

To associate the objects in frame $I^{t}$ with the objects or trajectories in previous $I^{t \minus 1}$, a similarity matrix is constructed with their appearance embeddings:

\begin{equation}
  \boldsymbol{C} \in \mathbb{R}^{M^{t} \times M^{t \minus 1}}, \;
  c_{i,j} = {\boldsymbol{f}_{i}^{t}} \cdot  \boldsymbol{f}_{j}^{t \minus 1}
  \label{eq:method_matrix}
\end{equation}

\noindent where $c_{i,j}$ represents the cosine similarity between the $i_{th}$ object in frame $I^{t}$ and the $j_{th}$ object (or trajectory) in frame $I^{t \minus 1}$. Then the Hungarian algorithm~\cite{kuhn1955hungarian} is adopted to generate the identity association results.

2) \textbf{Verification}. However, the appearance representations are sometimes unreliable, especially in the unsupervised scenario. To solve this issue, an uncertainty metric is proposed to evaluate the association after the first stage.

% For an object $o_{i}^{t}$ in frame $I^{t}$, the similarities against the $M^{t \minus 1}$ objects in the previous frame can be expressed as:

% \begin{equation}
%   \boldsymbol{s}_{i} = \boldsymbol{C}_{i} = [c_{i,1}, c_{i,2}, \cdots, c_{i,M^{t \minus 1}}]^T
%   \label{eq:method_svec}
% \end{equation}

% Inspired by margin-based OOD detection~\cite{hendrycks2016baseline}, we assume that the assignment ($o_{i}^{t} \!\sim\! o_{j}^{t \minus 1}$) in the association stage is not convincing under the following circumstances:

% \begin{itemize}
%     \setlength{\itemsep}{0pt}
%     \item The assigned similarity between $o_{i}^{t}$ and $o_{j}^{t \minus 1}$ is relatively low (\ie, $c_{i,j} < m_1$).
%     \item The second-highest similarity with others ($c_{i,j_{2}}$) is close to the assigned $o_{j}^{t \minus 1}$ (\ie, $c_{i,j} - c_{i,j_{2}} < m_2$).
% \end{itemize}

% Based on these assumptions, we define an association-level uncertainty metric, which is formulated as:



Object association can be viewed as multi-category classification.
And confidence-score has been proved efficient and effective on detecting mis-classified examples~\cite{hendrycks2016baseline}.
Inspired by this, we propose to detect the mis-associated objects through the similarity-scores.


Given an object $o_{i}^{t}$ associated with $o_{j}^{t \minus 1}$ in the previous frame based on \cref{eq:method_matrix}, the association ($o_{i}^{t} \!\sim\! o_{j}^{t \minus 1}$) is unconvincing in two cases: 
1) the assigned similarity $c_{i,j}$ is relatively low (\eg, partial occlusion or motion blur) and 
2) there are other objects whose similarities are close to the assigned $c_{i,j}$ (\eg, similar appearance or indistinguishable embedding).
It can be formulated as:

\begin{equation}
  c_{i,j} < m_1; \quad c_{i,j_{2}} > c_{i,j} - m_2
  \label{eq:method_margin}
\end{equation}


\noindent 
where $m_1,m_2$ are constant margins. Only the second-highest similarity with others ($c_{i,j_{2}}$) is considered for simplicity.
In an ideal association, $c_{i,j}$ should be close to 1 and $c_{i,j_{2}}$ close to 0.
We thus proposed to estimate the association \lk{risk} as:

% \sigma_{i,j} = - \left( 
% \log c_{i,j} + \log \left( 1 - c_{i,j_{2}} \right)
% + \overline{\log \left( 1 - c_{i,l} \right) }
% \right)  
\begin{equation}
  \sigma_{i,j} = - \log c_{i,j} - \log \left( 1 - c_{i,j_{2}} \right)
  \label{eq:method_energy}
\end{equation}

Detailed derivation process refers to the supplementary materials.
Combining with \cref{eq:method_margin} and \cref{eq:method_energy} , an adaptive threshold is proposed:

\begin{equation}
  % \gamma_{i,j} = -\log \left( 1 + m_2 - c_{i,j} \right) -\log m_1 \left( 1 - m_3 \right)
  \gamma_{i,j} =  -\log m_1 - \log \left( 1 + m_2 - c_{i,j} \right)
  \label{eq:method_border}
\end{equation}

As shown in~\cref{fig:method_verify}, when the \lk{risk} $\sigma_{i,j}$ is higher than the threshold $\gamma_{i,j}$, the assignment ($o_{i}^{t} \!\sim\! o_{j}^{t \minus 1}$) should be re-considered. 
\lk{The \textbf{association uncertainty} is quantified as:}

\begin{equation}
  \delta_{i,j} = \sigma_{i,j} - \gamma_{i,j}
  \label{eq:method_uncertain}
\end{equation}

The results are not sensitive to the exact margins. We set $m_1 = 0.5$ and $m_2 = 0.05$ for a slightly better performance.
% More experimental details are shown in the supplementary materials.

The uncertain pairs after the verification stage and unmatched objects after the association stage are gathered as uncertain candidates for the rectification stage.


3) \textbf{Rectification}. 
The rectification stage is performed among the uncertain candidate. The similarities between two adjacent frames are no longer convincing.
% due to irregular motion, severe occlusion, and so on. 
More information should be taken into account, including motion \lk{estimation} and appearance \lk{variation} within a tracklet. 
% Specifically, intersection-over-union (IoU)~\cite{bewley2016simple} is the widely-used motion metric. At the same time, the tracklet embeddings can provide complementary appearance information.

For the uncertain candidates, \mywork~constructs another similarity matrix for the secondary rectification. 
First, \lk{the motion constraints should be relaxed}, so the association shares overlap \lk{higher than} $\beta$ 
% in adjacent frames 
\lk{are preserved}. 
Second, \lk{the appearance should not vary extremely fast}, so we adopt the averaged similarity between object $o_{i}^{t}$ and tracklet $trk_{j} = \{o_{j}^{t \minus K}, \cdots, o_{j}^{t \minus 1}\}$ within previous $K$ frames. 
In this stage, we solve the sub-problem of global identity assignments, which can be formulated as:

\begin{equation}
\begin{split}
  \boldsymbol{C}^\prime \in \mathbb{R}^{{M^{t}}^\prime \times {M^{t \minus 1}}^\prime} & \\
  c^\prime_{i,j} = \left( \frac{1}{K} \sum_{\hat{t} = t \minus K}^{t \minus 1} {\boldsymbol{f}_{i}^{t}} \cdot  \boldsymbol{f}_{j}^{\hat{t}} \right) 
            \times \mathbb{I} & \left( \text{IoU} \left( b_{i}^{t}, b_{j}^{t \minus 1} \right) > \beta \right) 
  \label{eq:method_recti}
\end{split}
\end{equation}

\noindent where $\mathbb{I}(*)$ is the indicator function. Then the match set is updated based on the Hungarian algorithm.

\lk{
\textit{Remark.} Our core contribution is the uncertainty-based verification mechanism, rather than the specific rectification, which shall be adjusted in practice. Empirically we set $\beta=0.1$ and $K=5$.
}

% Figure environment removed

4) \textbf{Propagation}. The pseudo-tracklets are propagated frame-by-frame. As shown in~\cref{fig:method_reidacc}, our strategy brings \lk{consistently} accurate pseudo-identities, \lk{\eg, reaching 97\% accuracy across 100 frames}.
% The pseudo-tracklets are progressively updated during the training process.
The long-term intra-tracklet consistency is successfully maintained.
% by the accurate pseudo-identities.

It is worth mentioning that the \lk{verification and rectification} stages can be naturally applied to the inference process to boost the performance, \lk{which does not conflict with existing association methods}.

\subsection{Tracklet-Guided Augmentation}
\label{sec:method_ada_aug}

The accurate pseudo-tracklets can guide the sample augmentation in the unsupervised MOT framework.
To learn the \liuk{inter-frame consistency}~\cite{chen2020simple,zhang2021fairmot}, good training samples should be diverse and \liuk{temporal-aware}. 
However, as illustrated in~\cref{fig:method_ada_aug}, existing methods usually treat augmentation and multi-object tracking as two isolated tasks, leading to ineffective augmentations. 
Instead, this paper utilizes the tracklet's spatial displacements to guide the augmentation process. 
According to a properly selected anchor pair, the proposed strategy makes the augmented frames aligned to the historical frames, simulating realistic tracklet movements. The proposed method concurrently focuses on the hard negative samples.
Details \lk{of the \textbf{T}racklet-\textbf{G}uided \textbf{A}ugmentation (TGA)} are given below.

% Figure environment removed

We introduce the temporal information into spatial transformation. 
First, given a current frame $I^{t}$ with $M^{t}$ objects, we select a source-anchor object $o_{a}^{t}$, whose bounding box is denoted as $b_{a}^{t} = (cx_{a}^{t}, cy_{a}^{t}, w_{a}^{t}, h_{a}^{t})$. Then, we choose a target-anchor $o_{a}^{t \minus \tau}$ in $(t \minus \tau)_{th}$  historical frame from the pseudo-tracklet $trk_{a} = \{o_{a}^{t_0}, o_{a}^{t_1}, \cdots, o_{a}^{t}\}$. 
Finally, to augment the current $I^{t}$ to align with historical $I^{t \minus \tau}$,  a tracklet-guided affine transformation can be expressed as:

\begin{equation}
  \begin{bmatrix}
      x^{t \minus \tau} \\ y^{t \minus \tau} \\ 1
  \end{bmatrix}
  =
  \boldsymbol{M}_{t}^{t \minus \tau}
  \begin{bmatrix}
      x^{t} \\ y^{t} \\ 1
  \end{bmatrix}
  =
  \begin{bmatrix}
      m_{11} & m_{12} & m_{13} \\
      m_{21} & m_{22} & m_{23} \\
      0      & 0      & 1
  \end{bmatrix}
  \begin{bmatrix}
      x^{t} \\ y^{t} \\ 1
  \end{bmatrix}
  \label{eq:method_affine}
\end{equation}

\noindent where $x^*,y^*$ are spatial coordinates, and $\boldsymbol{M}_{t}^{t \minus \tau}$ can be solved by direct linear transform (DLT) algorithm ~\cite{detone2016deep}. 
% with the corner locations of the anchor pair $(o_{a}^{t} \!\sim\! o_{a}^{t \minus \tau})$. 
Then an augmented frame $\tilde{I}^{t}$ is generated based on the tracklet-guided affine transformation with perspective jitter, which can be expressed as $\tilde{I}^{t} = \mathcal{T}\left(I^{t}, M_{t}^{t \minus \tau} \right)$.
% \begin{equation}
%   \tilde{I}^{t} = \mathcal{T}\left(I^{t}, M_{t}^{t \minus \tau} \right)
%   \label{eq:method_aug}
% \end{equation}

Intuitively, a proper anchor-selection is vitally important for our augmentation strategy. 

First, the identity accuracy of anchor pair $(o_{a}^{t} \!\sim\! o_{a}^{t \minus \tau})$ is important. In other words, the consistency of anchor tracklet $trk_{a}$ should be guaranteed. We thus design a tracklet-level uncertain metric based on the propagated association-level uncertainty defined in \cref{eq:method_uncertain}, which is formulated as:

\begin{equation}
  \Omega_{i} = \frac{1}{n} \sum_{s=t_0}^{t} \exp (\delta_{i}^{s})
  % \Omega_{i} = \sqrt[n]{\sigma_{i}^{t_0} \cdot \sigma_{i}^{t_1} \cdots \sigma_{i}^{t}}
  \label{eq:method_tenergy}
\end{equation}

\noindent where $\Omega_{i}$ represents the uncertainty of tracklet $trk_{i}$, \lk{and $n$ is the tracklet length}.
An uncertainty-based sampling strategy is designed to select the source anchor $o_{a}^{t}$ (along with the anchor $trk_{a}$) from the $M^{t}$ objects in frame $I^{t}$, which can be formulated as:

\begin{equation}
  p\left(a=i \mid t \right) 
  % = softmax\left(-\Omega_{i}\right)
  = \frac{\exp{\left(-\Omega_{i}\right)}}{\sum_{\hat{i}=1}^{M^{t}}\exp{\left(-\Omega_{\hat{i}}\right)}}
  \label{eq:method_sel_an_src}
\end{equation}

\noindent where $p\left(a=i \mid t \right)$ represents the probability to choose the $i_{th}$ tracklet $trk_{i}$ as the anchor $trk_{a}$.
The uncertain tracklet with high $\Omega$ is less likely to be selected, avoiding dramatic augmentations from erroneous pseudo-tracklets.

Second, hard negative samples matters in discriminablity learning. We tend to choose an indistinguishable (or, high uncertain) target anchor $o_{a}^{t \minus \tau}$ along the tracklet $trk_{i}$. The selection probability can be formulated as:

\begin{equation}
  p\left(\pi=t \minus \tau \mid a \right) 
  = \frac{\exp{\left(\delta_{a}^{t \minus \tau}\right)}}{\sum_{\hat{\tau}=t_0}^{t-1}\exp{\left(\delta_{a}^{t-\hat{\tau}}\right)}}
  \label{eq:method_sel_an_tgt}
\end{equation}

\lk{A visualization example are displayed in the supplementary material to illustrate the hierarchical sampling process.}

Compared with conventional random transformation, the proposed tracklet-guided augmentation is well-directed and tracking-related. 
\lk{Together with accurate pseudo-tracklets, \mywork~successfully improves the inter-frame consistency, as shown in \cref{fig:method_disc_vis}. }


% Figure environment removed

% \subsection{Momentum Memory Dictionary}
% \label{sec:method_md}


%To reuse the encoded samples from the intermediate mini-batches, we maintain a queue for each video in the memory dictionary by enqueueing the $M^{t}$ objects in the current frame and removing the oldest samples.
%In representation learning, high-quality negative samples play an essential role~\cite{chen2020simple,he2020momentum}. However, existing unsupervised trackers only take negative samples from adjacent frames, augmented frames, and the current frame itself. The lack of negative sample diversity prevents trackers from learning discriminative representations. \mywork~adopts a momentum dictionary mechanism to alleviate this problem.

%As shown in~\cref{fig:method_fmwk}, we build a memory dictionary for each \textit{independent} video input during training. Given an input image $I^{t}$ from video $V$, we randomly sample a number of negative object samples from other videos in the memory dictionary, so as to enrich the negative sample diversity. To reuse the encoded samples from the intermediate mini-batches, we maintain a queue for each video in the memory dictionary by enqueueing the $M^{t}$ objects in the current frame and removing the oldest samples.

\section{Theoretical Guarantees}\label{sec:theorem}
In this section, we first prove that \method{} achieves a linear convergence rate under non-convex settings with proper assumptions.
Then, in the second part of this section, we delve into the compute and memory costs in \method{}, and show its benefits in wall-clock convergence compared to other baseline methods: stochastic L-BFGS, and KFAC.

\subsection{Convergence Analysis}
We assume the risk function $\Loss$ satisfies the following conditions:
\begin{assumption}\label{as:diff}
$\Loss(\btheta)$ is twice continuously differentiable.
\end{assumption}

\begin{assumption}\label{as:smooth}
$\ell_i(\btheta)$ is $\Lambda$-smooth for $1\leq i \leq N$, $\Lambda > 0$: 
$\forall \btheta_1, \btheta_2$, $\norm{\nabla\ell_i(\btheta_2)-\nabla\ell_i(\btheta_1)}\leq \Lambda\norm{\btheta_2-\btheta_1}$.
\end{assumption}

\begin{assumption}\label{as:pl}
$\Loss(\btheta)$ is $\lambda$-PL: it satisfies Polyak-Lojasiewicz (PL) condition for a constant $\lambda > 0$: $\norm{\grad{L}(\btheta)}^2 \geq \lambda\Loss(\btheta)$.
\end{assumption}

The smooth condition in AS \ref{as:smooth} is commonly used in analyzing convergence in practical optimization. 
In addition, noting that compared to typical strong convexity assumptions, the PL condition in AS \ref{as:pl} applies to a more general setting \citep{PL}.
Strong convexity implies the PL condition, but not vice versa. In AS \ref{as:pl}, we relax the constraint and only require the gradient variance to be lower bounded. 

With the assumptions, we present the convergence theorem as follows. Proofs are deferred to Appendix \ref{appx:proof}.
\begin{theorem}\label{theorem:convergence}
Assume AS \ref{as:diff}-\ref{as:pl} hold at each iteration $t$ of \method{} with mini-batch input $\tX_t$ where each sample is randomly sampled from  $\mathcal{X}$ with replacement, then the expectation of $\Loss(\btheta_{t})$ satisfies
\begin{center}
    $E_{\tX_{t}}[\Loss(\btheta_{t})] \leq \alpha_{t-1} E_{\tX_{t-1}}[\Loss(\btheta_{t-1})]$,
\end{center}
where $\alpha_{t-1} = 1-\eta_{t-1}\lambda\xi + \eta_{t-1}^2\Lambda^2\Xi^2$. $\xi$ and $\Xi$ denotes the lower and upper bound of the $\HI$.% and the expectations are with respect to the minibatches $\mathcal{S}_t$ and $\mathcal{S}_{t+1}$. %With $0<\eta_t < \frac{1}{2\lambda\xi}$, SLIM-QN converges to minimum $\mathcal{L}_{\otheta}$ in a linear rate.
\end{theorem}

By choosing $\eta_{t-1}$ such that $\alpha_{t-1} < 1$, \method{} converges at a linear rate. 
The convergence rate matches the best rate in stochastic QN optimizations. It is worth mentioning that no convergence rate beyond linear is observed in stochastic L-BFGS optimizations.
Theorem~\ref{theorem:convergence} is not aimed to push the theoretical convergence limit. Instead, it investigates the effects of momentum in the Hessian and block-diagonal approximation.
In addition, as mentioned earlier, Theorem~\ref{theorem:convergence} also applies to convex settings, as strong convexity implies $\norm{\grad{L}(\btheta)}^2$ is lower bounded by $\Loss(\btheta)$ for an appropriate $\lambda > 0$ and hence AS \ref{as:pl} holds.


\subsection{Complexity Analysis}
In this section, we analyse the compute and memory cost of SGD, KFAC\citep{2016_ICLR_distKFAC}, stochastic L-BFGS (we call it sL-BFGS)\citep{sLBFGS} and \method{}.
As the main motivation, reducing the complexities of QN methods is crucial for their deployment in real large-scale neural network optimization.

Given a model with parameter $\btheta \in \mathbb{R}^d$, we use $\Cfb$ and $\Mfb$ to represent the compute and memory cost of a forward/backward (Fwd/Bwd) pass with a batch size of $b=1$. Furthermore,  $\Copt$ denotes the compute cost of model updates (Opt) which consists of gradient reduction, computing and apply the update $\Delta\btheta$.

Table~\ref{tab::complexity} summarizes the total compute and memory cost of SGD, KFAC, sL-BFGS and \method{} in a general distributed system with $p$ workers. 
Compared to SGD, during the forward and backward passes, \method{} needs to additionally compute $\Mp{}$, $\Mg{}$, for which the complexity increases linearly with model size ($d$). 
The main extra compute \method{} introduces is the Hessian-vector product, in which we need to iterate over $\left \{ \bm{s}_i \right \}_{i=1}^M$ and $\left \{ \bm{y}_i \right \}_{i=1}^M$, as shown in  Alg~\ref{appx:hessianvec}. The complexity increases linearly with the number of history vectors and model size ($2Md$). However, it only adds average cost of $O(\frac{2Md}{p})$ on each worker.
Compared to $O(b\Cfb)$ complexity in forward and backward passes, such costs are relatively marginal.
% However, these added computations are marginal compared to computations in forward and backward pass ($b\Cfb$).

As a comparison, KFAC adds significant additional computations through 1) possible multiple backward passes to update factors ($\gamma b\Cfb$ with $\gamma\geq 1$), 2) matrix inversion ($\sum(l_i^3+(\frac{d_i}{l_i})^3)$) for every $T$ iterations, and 3) Matrix-vector products ($2\sum(l_i+\frac{d_i}{l_i})d_i$). 
% If the Fisher matrix is updated more frequently (that is for small $L$), then the amortized cost for matrix inversion is even more striking. 
On the other hand, sL-BFGS also resorts to computation-intensive operations including full-batch gradients and a separate large batch to estimate the Hessian, which respectively adds amortized costs of $O(b\Cfb)$ and $\frac{1}{T}b_H\Cfb$. With data parallelism, it requires each worker to locally perform gradient conditioning, which adds another cost of $O(2Md)$ in total. 
% As a comparison, it is almost cost-free to update the approximate Hessian in SLIM-QN.

As for memory usage, \method{} mainly needs $O(2Md)$ to store history vectors. The amortized costs of each worker are $O(\frac{2Md}{p})$.
In practice, $M$ is set to be $10\sim 20$, which ensures that memory usage is manageable in \method{}.
sL-BFGS needs $O(Md)$ storage for $\sv{i},\yv{i}$ in total, and amortized cost of $O(\frac{1}{T}b_H\Mfb)$ for additional backward passes.
While KFAC needs $O(2\sum (l_i^2+(\frac{d_i}{l_i})^2))$ to store sub-matrices and their inverse, where the actual memory footprint hinges on model architectures. 
\begin{table*}[!htb]
\vspace{-2mm}
\caption{\footnotesize Computations and Memory in SGD, KFAC, sL-BFGS and \method{}}
\label{tab::complexity}
\centering
\small
\begin{tabular}{@{}ccccc@{}}
\toprule
 & SGD & KFAC & sL-BFGS & \method \\ \midrule
 \multicolumn{5}{c}{Per-Node Computation} \\ \midrule
 \multicolumn{1}{c|}{Fwd\&Bwd}& $O(b\Cfb)$ &$O(d+\gamma b\Cfb+\frac{1}{T}\sum(l_i^3+(\frac{d_i}{l_i})^3))$ & $O(d+2b\Cfb+\frac{1}{T}b_H\Cfb)$ & $O(\frac{d}{p}+b\Cfb)$ \\
 \multicolumn{1}{c|}{Opt}& $O(d)$ & $O(d+2\sum(l_i+\frac{d_i}{l_i})d_i)$ & $O(d+2Md)$ & $O(d+\frac{2Md}{p})$ \\ \midrule
 \multicolumn{5}{c}{Per-Node Memory} \\ \midrule
 \multicolumn{1}{c|}{Fwd\&Bwd}& \multicolumn{1}{c}{$O(b\Mfb)$} & \multicolumn{1}{c}{$O(d+b\Mfb)$} & $O(d + b\Mfb + \frac{1}{T}b_H\Mfb)$ & $O(\frac{d}{p}+b\Mfb)$\\
 \multicolumn{1}{c|}{Opt}& \multicolumn{1}{c}{$O(d)$} & \multicolumn{1}{c}{$O(d+2\sum(l_i^2+(\frac{d_i}{l_i})^2))$} & $O(d+2Md)$ & $O(d+\frac{2Md}{p})$\\ \bottomrule
\end{tabular}
\small
\begin{itemize}
    \item $b$: per-node batch size. $b_H$: batch size for the Hessian approx. $p$: \#workers. $T$: Hessian update period.
    \item  $d_i$: \#params in  $i$-th layer. $l_i$: \#input neurons in $i$-th layer. $M$: max \#history vectors.
\end{itemize}
\vspace{-0.4cm}
\end{table*}

Table \ref{tab:resnet_costs} lists detailed amortized costs of different optimizers on ResNet-50/ImageNet in a distributed system. Compared to KFAC and sL-BFGS, \method{} significantly reduces compute costs in approximating the Hessian and gradient conditioning. Due to the efficient distributed design, memory consumption on each worker is also reduced compared to sL-BFGS.
\begin{table}[!htb]
\vspace{-2mm}
\caption{\footnotesize Computation (MACs) and memory costs of different optimizers on ResNet-50/ImageNet ($b=64$, $T=20$; $M=10$; $b_H=1024$, $\gamma=1$, $p=8$). ``B" denotes billion, and ``M" denotes million.}
\label{tab:resnet_costs}
\centering
\begin{tabular}{c|cccc}
\toprule
 & SGD & KFAC & sL-BFGS & \method{}  \\
\midrule
Fwd/Bwd & \multicolumn{4}{c}{769B} \\
\midrule
$\HI$ Compute & -  & 570B & 1414B & 52M\\
Opt & 26M & 156B & 4B & 546M \\
$\HI$ Memory & - & 308M & 4B & 520M \\
\bottomrule
\end{tabular}
\vspace{-.4cm}
\end{table}

\section{Experiments}\label{sec:exp}
We conduct various experiments on computer vision (CV) problems involving datasets such as CIFAR-10, CIFAR-100 and ImageNet. We choose SGD and Adam as the main baselines since it is widely used in these tasks and maintains the best training performance. 
We also compare with another quasi-Newton method, KFAC, in large-scale model training. 
We separately tune hyperparameters for each optimizer to ensure it achieves the best validation accuracy.

We use a single GPU server with 8 Nvidia Quadro RTX 5000 GPUs to simulate a distributed system, where each GPU is used as a worker to perform forward and backward passes, and model updates. Furthermore, each worker is also assigned with one Hessian block to compute the Hessian inverse and gradient conditioning. The current implementation is based on PyTorch. We set lower and upper thresholds of damping $\sigma_L, \sigma_H$ to be $0.01, 1.5$ in all experiments to smooth the Hessian approximation.

\subsection{Experiments on CIFAR-10/CIFAR-100}
We first evaluate \method{} on two small-scale problems: CIFAR-10 and CIFAR-100, and demonstrate the convergence advantage of \method{} compared to SGD and Adam. 
The models used are ResNet-18 and DeiT-Tiny \cite{DeiT}, where DeiT-Tiny is an efficient image transformer with 12 layers, 3 attention heads, and hidden and MLP dimension of 192.
% The ViT model is based on Vision Transformer model \citep{ViT}, with 6 layers, 8 attention heads, a patch size of 16, and both hidden and MLP dimension of 512 for a total of about 10M parameters.

For ResNet18, we divide it into 4 blocks such that each block consists of 2 \emph{resblocks} (\cite{2016_CVPR_ResNet}). The linear layer for classification is packed into the last block.  For DeiT-Tiny, due to the small model size, we choose to approximate the whole Hessian. 

Hyperparameters are tuned to achieve the best validation accuracy. Details are provided in Appendix \ref{appx:hparam:cifar}.

Figure \ref{fig:cifar} shows training loss. We observe that \method{} achieves a much faster convergence rate compared to SGD and ADAM. 
Table \ref{tab:acc:cifar} lists the validation accuracy on CIFAR-10 and CIFAR-100. 
We note that \method{} also achieves similar accuracy as SGD. On the other hand, as in Figure  \ref{fig:cifar}, although ADAM has a convergence rate close to \method{}, the validation accuracy are much lower in all experiments compared to \method{}.
Therefore, we obverse \method{} not only deliver faster convergence, but also achieves good generalization performance.
% Figure environment removed

\begin{table}[!htb]
\caption{\footnotesize Validation accuracy of ResNet-18 and Deit-Tiny on CIFAR-10/100 using SGD, ADAM, and \method{}. }
\label{tab:acc:cifar}
\centering
\small
\begin{tabular}{ccc|ccc|ccc}
\toprule
\multicolumn{3}{c|}{ResNet-18/CIFAR-10} & \multicolumn{3}{c|}{ResNet-18/CIFAR-100} & \multicolumn{3}{c}{DeiT-Tiny/CIFAR-100}\\
\midrule
 SGD & ADAM & \method{} & SGD & ADAM & \method{} & SGD & ADAM & \method{} \\
 $94.1\pm 0.1$ & $92.7\pm 0.1$ & $93.9\pm 0.1$ & $75\pm 0.15$ & $72.2\pm 0.16$ & $74.4 \pm 0.1$ & $80.5\pm 0.2$ & $75.3\pm 0.3$ & $79.9 \pm 0.2$ \\
\bottomrule
\end{tabular}
\vspace{-.4cm}
\end{table}

\subsection{Experiments on ImageNet}
ImageNet has been the gold standard for evaluating the performance of optimizers. It consists of $\sim$1.2M training and $\sim$50K test images, categorized into 1000 classes. 
We follow the standard data pre-processing procedure, where each image is first resized to $256\times 256$, and randomly cropped to $224\times 224$ and flipped horizontally. Each image is then normalized using pre-computed mean and variance.
%\subsubsection{ResNet-50}

\textbf{ResNet-50} --
When approximating the Hessian, we divide ResNet-50 into 8 blocks such that each block consists of 2 \emph{resblocks}. Similar to ResNet-18 in CIFAR-10, the linear layer is packed into the last block.
Figure~\ref{fig:resnet_imagenet} shows iteration-wise convergence on ResNet-50 using SGD, Adam, KFAC and \method{}. 
Detailed hyperparameter settings are provided in Appendix \ref{appx:hparam:imagenet}. Compared to Adam and SGD, \method{} enjoys much faster per-iteration convergence. Such fast convergence is also reflected in the validation dataset (Figure \ref{fig:resnet_imagenet_val}). Furthermore, it also generalizes well on the validation set, and finally reaches comparable validation accuracy to SGD.
% Figure environment removed

\begin{table}[!htb]
\caption{\footnotesize Validation accuracy of ResNet-50 on ImageNet using SGD, ADAM, KFAC, and \method{}. }
\label{tab:runtime}
\centering
\small
\begin{tabular}{cc|cc|cc|cc}
\toprule
\multicolumn{2}{c|}{SGD} & \multicolumn{2}{c|}{ADAM} & \multicolumn{2}{c|}{\method{}} & \multicolumn{2}{c}{KFAC}\\
\midrule
 Acc & Time/epoch & Acc & Time/epoch & Acc & Time/epoch & Acc & Time/epoch  \\
 $74.9\pm 0.11$ & 7.8min & $73.95\pm 0.1$ & 7.8min & $74.6\pm 0.13$ & 7.9min & $74.5\pm 0.1$ & 17min \\
\bottomrule
\end{tabular}
\end{table}

The benefit of \method{} is even more striking in terms of wall-clock time. 
As listed in Figure \ref{fig:resnet_imagenet_time} and Table \ref{tab:runtime}, due to light compute costs, the per-epoch runtime of \method{} is almost the same as SGD and Adam. 
On the other hand, for KFAC, while it delivers fast per-iteration convergence compared to SGD and ADAM, the wall-clock performance is significantly diminished by its additional compute costs. The per-epoch runtime is $>2\times$ more than \method{}.

\iffalse
\begin{table}[!htb]
\caption{Wall-clock time for each optimizer to reach the optimal accuracy. \method{} needs the least time to reach the optimal accuracy. }
\label{tab:runtime}
\centering
\begin{tabular}{c|ccc|cc}
\toprule
 & \multicolumn{3}{c|}{ResNet50} & \multicolumn{2}{c}{ViT}\\
\midrule
 Opt. & SGD & KFAC & \method{} & SGD & \method{} \\
 Time (h)& 31.2 & 50.5 & 22.7 & 12.5 & 10.4 \\
\bottomrule
\end{tabular}
\vspace{-.4cm}
\end{table}
\fi

\subsection{Ablation Study: The Effects of Momentum and Damping}\label{subsec:ablation}
In this section, we give more insight into the effects of momentum and damping used in \method{}. To this end, we ablate two critical components in \method{}: momentum and damping in the Hessian approximation, and then use the ablated version to train ResNet-18 on CIFAR-10. We focus on CIFAR-10 since we observed more convergence instability on this dataset compared to others. 

Figure~\ref{fig:ablation} shows convergence using the ablated \method{} with only momentum (black), with only damping (purple), and with no momentum or damping (red). Due to stochastic noise, the ablated version of \method{} without momentum/damping (vanilla L-BFGS) diverges easily in the early stages. 
With momentum (black), the whole optimization is significantly stabilized. However, it still fails to converge when there is a radical change in the loss landscape (for example, when learning rate decays). 
With damping (purple), the Hessian approximation is effectively restrained, especially when such sudden changes in the loss landscape happen. 
It is interesting to observe that while damping prevents divergence, the whole training is still largely affected by stochastic noise. Notable fluctuation in the loss is commonly observed during training. As a comparison, the complete \method{} (blue) effectively addresses these issues achieving much more stable convergence.
% Figure environment removed

\section{Related Works}\label{sec:related}
While SGD is widely used in many machine learning tasks, other forms of optimization have also been investigated extensively in the past years. Among these attempts, designing optimizers with preconditioned gradients is one of the most promising areas.

Adaptive methods such as Adam, AdaGrad, AdaDelta \citep{2014_arXiv_Adam, 2011_JMLR_AdaGrad, AdaDelta} construct a diagonal matrix by incorporating knowledge from the past gradients. Such a diagonal matrix adaptively adjusts the learning rate for each parameter.
For instance, AdaGrad uses a large learning rate for those irrelevant features (small gradients), and small learning for those relevant ones (large gradients).  Adams further uses the first and second moment of gradients to adjust the learning rate. 

Besides diagonal preconditioning matrices, constructing block diagonal matrices or even full matrices has received increasing attention in recent years. Methods such as Shampoo \citep{Shampoo} approximate the full matrix version of AdaGrad as a block-diagonal matrix to incorporate more curvature information during optimization. Similarly, a well-known KFAC method \citep{2015_ICML_KFAC} approximates the Fisher information matrix as a block-diagonal matrix. L-BFGS methods \cite{2016_PMLR_LBFGS} on the other hand directly construct the full Hessian matrix as a preconditioner during optimization. These preconditioners have been empirically proved to achieve fast convergence compared to SGD, as well as adaptive methods. Authors in \cite{2020_NIPS_KFAC-LBFGS} further adopt L-BFGS to efficiently compute matrix inversion in KFAC. 

Variance reduction in the preconditioning matrix is also crucial to ensure stable optimization. The algorithm in \cite{2016_PMLR_LBFGS} adopts a separate large batch of data to estimate current curvature. On the other hand, VITE \cite{VITE} chooses to use a pivot parameter together with full-batch gradients to reduce variance in the Hessian approximation.

\section{Conclusion}\label{sec:conclusion}

\section{Conclusion}

In this work, we present \texttt{vox2vec} --- a self-supervised framework for voxel-wise representation learning in medical imaging. Our method expands the contrastive learning setup to the feature pyramid architecture allowing to pre-train effective representations in full resolution. By pre-training a FPN backbone to extract informative representations from unlabeled data, our method scales to large datasets across multiple task domains. We pre-train a FPN architecture on more than 6500 CT images and test it on various segmentation tasks, including different organs and tumors segmentation in three setups: linear probing, non-linear probing, and fine-tuning. Our model outperformed existing methods in all regimes. Moreover, \texttt{vox2vec} establishes a new state-of-the-art result on the linear and non-linear probing scenarios. 

Still, this work has a few limitations to consider. We plan to investigate further how the performance of \texttt{vox2vec} scales with the increasing size of the pre-training dataset and the pre-trained architecture size. Another interesting research direction is exploring the effectiveness of \texttt{vox2vec} in the domain adaptation and few-shot learning scenarios.



\iffalse
In an attempt to encourage standardized notation, we have included the
notation file from the textbook, \textit{Deep Learning}
\cite{goodfellow2016deep} available at
\url{https://github.com/goodfeli/dlbook_notation/}.  Use of this style
is not required and can be disabled by commenting out
\texttt{math\_commands.tex}.


\centerline{\bf Numbers and Arrays}
\bgroup
\def\arraystretch{1.5}
\begin{tabular}{p{1in}p{3.25in}}
$\displaystyle a$ & A scalar (integer or real)\\
$\displaystyle \va$ & A vector\\
$\displaystyle \mA$ & A matrix\\
$\displaystyle \tA$ & A tensor\\
$\displaystyle \mI_n$ & Identity matrix with $n$ rows and $n$ columns\\
$\displaystyle \mI$ & Identity matrix with dimensionality implied by context\\
$\displaystyle \ve^{(i)}$ & Standard basis vector $[0,\dots,0,1,0,\dots,0]$ with a 1 at position $i$\\
$\displaystyle \text{diag}(\va)$ & A square, diagonal matrix with diagonal entries given by $\va$\\
$\displaystyle \ra$ & A scalar random variable\\
$\displaystyle \rva$ & A vector-valued random variable\\
$\displaystyle \rmA$ & A matrix-valued random variable\\
\end{tabular}
\egroup
\vspace{0.25cm}

\centerline{\bf Sets and Graphs}
\bgroup
\def\arraystretch{1.5}

\begin{tabular}{p{1.25in}p{3.25in}}
$\displaystyle \sA$ & A set\\
$\displaystyle \R$ & The set of real numbers \\
$\displaystyle \{0, 1\}$ & The set containing 0 and 1 \\
$\displaystyle \{0, 1, \dots, n \}$ & The set of all integers between $0$ and $n$\\
$\displaystyle [a, b]$ & The real interval including $a$ and $b$\\
$\displaystyle (a, b]$ & The real interval excluding $a$ but including $b$\\
$\displaystyle \sA \backslash \sB$ & Set subtraction, i.e., the set containing the elements of $\sA$ that are not in $\sB$\\
$\displaystyle \gG$ & A graph\\
$\displaystyle \parents_\gG(\ervx_i)$ & The parents of $\ervx_i$ in $\gG$
\end{tabular}
\vspace{0.25cm}


\centerline{\bf Indexing}
\bgroup
\def\arraystretch{1.5}

\begin{tabular}{p{1.25in}p{3.25in}}
$\displaystyle \eva_i$ & Element $i$ of vector $\va$, with indexing starting at 1 \\
$\displaystyle \eva_{-i}$ & All elements of vector $\va$ except for element $i$ \\
$\displaystyle \emA_{i,j}$ & Element $i, j$ of matrix $\mA$ \\
$\displaystyle \mA_{i, :}$ & Row $i$ of matrix $\mA$ \\
$\displaystyle \mA_{:, i}$ & Column $i$ of matrix $\mA$ \\
$\displaystyle \etA_{i, j, k}$ & Element $(i, j, k)$ of a 3-D tensor $\tA$\\
$\displaystyle \tA_{:, :, i}$ & 2-D slice of a 3-D tensor\\
$\displaystyle \erva_i$ & Element $i$ of the random vector $\rva$ \\
\end{tabular}
\egroup
\vspace{0.25cm}


\centerline{\bf Calculus}
\bgroup
\def\arraystretch{1.5}
\begin{tabular}{p{1.25in}p{3.25in}}
% NOTE: the [2ex] on the next line adds extra height to that row of the table.
% Without that command, the fraction on the first line is too tall and collides
% with the fraction on the second line.
$\displaystyle\frac{d y} {d x}$ & Derivative of $y$ with respect to $x$\\ [2ex]
$\displaystyle \frac{\partial y} {\partial x} $ & Partial derivative of $y$ with respect to $x$ \\
$\displaystyle \nabla_\vx y $ & Gradient of $y$ with respect to $\vx$ \\
$\displaystyle \nabla_\mX y $ & Matrix derivatives of $y$ with respect to $\mX$ \\
$\displaystyle \nabla_\tX y $ & Tensor containing derivatives of $y$ with respect to $\tX$ \\
$\displaystyle \frac{\partial f}{\partial \vx} $ & Jacobian matrix $\mJ \in \R^{m\times n}$ of $f: \R^n \rightarrow \R^m$\\
$\displaystyle \nabla_\vx^2 f(\vx)\text{ or }\mH( f)(\vx)$ & The Hessian matrix of $f$ at input point $\vx$\\
$\displaystyle \int f(\vx) d\vx $ & Definite integral over the entire domain of $\vx$ \\
$\displaystyle \int_\sS f(\vx) d\vx$ & Definite integral with respect to $\vx$ over the set $\sS$ \\
\end{tabular}
\egroup
\vspace{0.25cm}

\centerline{\bf Probability and Information Theory}
\bgroup
\def\arraystretch{1.5}
\begin{tabular}{p{1.25in}p{3.25in}}
$\displaystyle P(\ra)$ & A probability distribution over a discrete variable\\
$\displaystyle p(\ra)$ & A probability distribution over a continuous variable, or over
a variable whose type has not been specified\\
$\displaystyle \ra \sim P$ & Random variable $\ra$ has distribution $P$\\% so thing on left of \sim should always be a random variable, with name beginning with \r
$\displaystyle  \E_{\rx\sim P} [ f(x) ]\text{ or } \E f(x)$ & Expectation of $f(x)$ with respect to $P(\rx)$ \\
$\displaystyle \Var(f(x)) $ &  Variance of $f(x)$ under $P(\rx)$ \\
$\displaystyle \Cov(f(x),g(x)) $ & Covariance of $f(x)$ and $g(x)$ under $P(\rx)$\\
$\displaystyle H(\rx) $ & Shannon entropy of the random variable $\rx$\\
$\displaystyle \KL ( P \Vert Q ) $ & Kullback-Leibler divergence of P and Q \\
$\displaystyle \mathcal{N} ( \vx ; \vmu , \mSigma)$ & Gaussian distribution %
over $\vx$ with mean $\vmu$ and covariance $\mSigma$ \\
\end{tabular}
\egroup
\vspace{0.25cm}

\centerline{\bf Functions}
\bgroup
\def\arraystretch{1.5}
\begin{tabular}{p{1.25in}p{3.25in}}
$\displaystyle f: \sA \rightarrow \sB$ & The function $f$ with domain $\sA$ and range $\sB$\\
$\displaystyle f \circ g $ & Composition of the functions $f$ and $g$ \\
  $\displaystyle f(\vx ; \vtheta) $ & A function of $\vx$ parametrized by $\vtheta$.
  (Sometimes we write $f(\vx)$ and omit the argument $\vtheta$ to lighten notation) \\
$\displaystyle \log x$ & Natural logarithm of $x$ \\
$\displaystyle \sigma(x)$ & Logistic sigmoid, $\displaystyle \frac{1} {1 + \exp(-x)}$ \\
$\displaystyle \zeta(x)$ & Softplus, $\log(1 + \exp(x))$ \\
$\displaystyle || \vx ||_p $ & $\normlp$ norm of $\vx$ \\
$\displaystyle || \vx || $ & $\normltwo$ norm of $\vx$ \\
$\displaystyle x^+$ & Positive part of $x$, i.e., $\max(0,x)$\\
$\displaystyle \1_\mathrm{condition}$ & is 1 if the condition is true, 0 otherwise\\
\end{tabular}
\egroup
\vspace{0.25cm}


\subsubsection*{Broader Impact Statement}
In this optional section, TMLR encourages authors to discuss possible repercussions of their work,
notably any potential negative impact that a user of this research should be aware of. 
Authors should consult the TMLR Ethics Guidelines available on the TMLR website
for guidance on how to approach this subject.

\subsubsection*{Author Contributions}
If you'd like to, you may include a section for author contributions as is done
in many journals. This is optional and at the discretion of the authors. Only add
this information once your submission is accepted and deanonymized. 

\subsubsection*{Acknowledgments}
Use unnumbered third level headings for the acknowledgments. All
acknowledgments, including those to funding agencies, go at the end of the paper.
Only add this information once your submission is accepted and deanonymized. 
\fi

\bibliography{main}
\bibliographystyle{tmlr}

\appendix
\section{Appendix}
\begin{comment}
\section{System Architecture}
\label{appendix:architecture}
\system has a novel modularized system architecture with three key components: 
\emph{StreamManager}, 
\emph{TxnManager} and \emph{TxnScheduler}. 
These components are instantiated in each thread locally.
The execution outline of \system is presented in Algorithm~\ref{alg:algo}.
Transactional stream processing is continuous and potentially never ends (Line 1$\sim$8).
The dependency resolution and execution of state transactions are separated into two non-overlapping phases by punctuations~\cite{Tucker:2003:EPS:776752.776780} (Line 2 and 5), which guarantees that no subsequent input event will have a smaller timestamp. 
Effectively, a batch of state transactions is collected during the first phase, and processed during the second phase.

In the first phase (i.e., stream processing phase), 
the \emph{StreamManager} conducts preprocessing for every input event ($e$). Similar to some prior works~\cite{tstream}, state transactions may be issued but not immediately processed during preprocessing (Line 3).
The \emph{pre\_processing} and \emph{post\_processing} functions are exposed as APIs to users.
The \emph{TxnManager} handles dependency resolution (Line 4) among state transactions and insert decomposed operations to construct a \tpg. We discuss the detailed two-phase \tpg construction process in Section~\ref{subsec:construction}.

In the second phase  (i.e., transaction processing phase), 
the \emph{TxnManager} is first involved again to refine (Line 6) the constructed \tpg with further dependency resolution.
The \emph{TxnScheduler} 
schedules operations for concurrent execution based on the constructed \tpg according to the three dimensions of scheduling decisions (Line 7). 
In particular, a scheduling decision model $M$ is instantiated based on the constructed \tpg (Line 14).
\textbf{\circled{1}} Guided by $M$, execution threads adopt an exploration strategy (Section~\ref{subsec:explore}) to explore the constructed \tpg for operations available to be scheduled constrained by dependencies. 
\textbf{\circled{2}} 
During exploration, one or multiple operations may be treated as the 
% basic 
unit of scheduling (Section~\ref{subsec:granularity}). 
Subsequently, \textbf{\circled{3}} every thread executes operation(s) in the unit of scheduling with various abort handling mechanisms (Section~\ref{subsec:abort_handling}).
Only when state transactions are processed (i.e., committed or aborted) can the associated input events be postprocessed (Line 8) by the \emph{StreamManager} based on transaction processing results.
\end{comment}

\begin{comment}
\begin{algorithm}
\footnotesize
    \KwData{$e$ \tcp{Input event}}
    \KwData{$txn_{ts}$ \tcp{State transaction}}
    \KwData{$G$ \tcp{The currently constructed TPG}}
    \While{!finish processing of input streams}{
        \eIf(\tcp*[h]{Phase 1}){\text{$e$ is not a $punctuation$}}{
                $txn_{ts}$ $\gets$ PRE\_Processing($e$)\;
                \textbf{TPG\_Construction}($G$, $txn_{ts}$)\; 
          }(\tcp*[h]{Phase 2}){
                \textbf{TPG\_Refinement}($G$)\; 
                \textbf{TXN\_Scheduling}($G$)\; 
                POST\_Processing()\;
          }
    }
    
    \SetKwFunction{FMain}{TPG\_Construction}
    \SetKwProg{Fn}{Function}{:}{}
    \Fn{\FMain{$G$, $txn_{ts}$}}{
        $O_{1..k}$ $\gets$ \textbf{Partition} $txn_{ts}$\;
        \ForEach{\text{operation $O_{i}$ $\in$ $O_{1..k}$}}{
            \textbf{Identify} its \ld\;
            $G$ $\gets$ $G$ + $O_{i}$ \;
        }
    }
    \SetKwFunction{FMain}{TPG\_Refinement}
    \SetKwProg{Fn}{Function}{:}{}
    \Fn{\FMain{$G$}}{
        \ForEach{\text{vertex $e_{i}$ $\in$ $G$}}{
            \textbf{Identify} its \td, \pd\;
        }
    }
    
    \SetKwFunction{FMain}{TXN\_Scheduling}
    \SetKwProg{Fn}{Function}{:}{}
    \Fn{\FMain{$G$}}{
        $M$ $\gets$ Instantiated with $G$;\tcp{A decision model}
        \While{!finish scheduling of $G$
        }{
          \textbf{\circled{2}} $Scheduling Unit$ $\gets$ \textbf{\circled{1}} \emph{Explore}($G$, $M$)\; 
            \textbf{\circled{3}} \emph{Execute with Abort Handling} ($Scheduling Unit$)\; 
        }
    }
  \caption{Execution Outline of \system}
  \label{alg:algo}
\end{algorithm}
\end{comment}

\end{document}
