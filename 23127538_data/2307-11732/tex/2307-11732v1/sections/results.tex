\raggedbottom
\section{Empirical Results}
\label{sec:results}

\subsection{Standard auctions}
\label{sec:results:textbook}

We first consider a standard, symmetric auction environment with a single query, so $Q = \{q^*\}$, and pay-per-impression pricing, so all CTRs are equal to 1. We consider both a small auction with $N = 2$ bidders, and a more realistic one with $N=10$ bidders. Values are uniformly distributed on a grid $B$ in the interval $[0,1]$. Formally, given the grid $B$, for all $i$, we let $\mathcal T_i = B$, $V_i(b, q^*) = b$, $\mathit{CTR}_i(b, q^*) = 1$, and $F_i(b) = \frac{1}{|B|}$ for all $b \in B$. Furthermore, we allow all bids $b_i$ on the grid $B$.

In a second-price auction, it is a dominant strategy for every bidder to bid their value. In a first-price auction, with $N$ bidders, a continuum of values (uniformly distributed on $[0,1]$) and bids, the unique symmetric Bayesian Nash equilibrium is for each bidder to bid a fraction $\frac{N-1}{N}$ of their value. Furthermore, Myerson's revenue equivalence applies, so the expected revenue for the advertising service is the same under both formats, namely $\frac{N-1}{N+1}$. We want to compare these theoretical predictions with the output of our learning algorithms with a fine enough grid $B$, and a long enough horizon $T$.

We choose $B = \{\frac{i}{20} : i = 0,\ldots, 20\}$ and $T = 1,000,000$ for \verb|EXP3-IX|. We also choose the tuning parameters optimally, as described in \citet{lattimore2020bandit} (Theorem 12.1, Chapter 12).  Table \ref{tab:standard-EXP3IX} displays revenues for the advertising service in first- and second-price auctions under \verb|EXP3-IX|, averaged over the last $10\%$ of the learning period, i.e., $100,000$ iterations, and 5 different runs of the algorithm, as well as the standard deviation of revenues across different runs.\footnote{Throughout the paper, we draw a fixed sequence of type realizations for all bidders. This ensures that the only randomness is from the algorithms' choice of action.}
\begin{table}[ht]
    \caption{Standard single-item auction, \texttt{EXP3-IX}}
    \label{tab:standard-EXP3IX}
    \vskip 0.1in
    \centering
    \begin{tabular}{|l|c|c|}
    \hline
    Auction format & Mean Revenue & Std. Dev.  \\ \hline
    First price, $N=2$ & $0.3474$ & $0.0205$ \\
    Second price, $N=2$ & $0.3346$ & $0.0016$ \\ \hline
    First price, $N=10$ & $0.780$ & $0.0095$ \\
    Second price, $N=10$ & $0.7723$ & $0.0171$ \\ \hline
    \end{tabular}
\end{table}
Table \ref{tab:standard-Hedge} reports the results for \verb|Hedge|. We choose $T = 400,000$ and $\eta = 0.02$. We averaged over 5 different runs of the algorithm and over the last $40,000$ time periods.
% 
\begin{table}[H]
    \caption{Standard single-item auction, \texttt{Hedge}.}
    \label{tab:standard-Hedge}
    \vskip 0.1in
    \centering
    \begin{tabular}{|l|c|c|}
    \hline
    Auction format & Mean Revenue & Std. Dev.  \\ \hline
    First price, $N=2$ & $0.3201$ & $0.0018$ \\
    Second price, $N=2$ & $0.3256$ & $0.0018$ \\ \hline
    First price, $N=10$ & $0.8305$ & $0.0011$ \\
    Second price, $N=10$ & $0.8334$ & $0.0008$ \\ \hline
    \end{tabular}
    \vskip -0.1in
\end{table}

The key take-away is that expected revenues are close to the theoretical value $\frac13$ for $N=2$ under both $\texttt{Hedge}$ and $\texttt{Exp3-IX}$. With $N=10$, $\texttt{Hedge}$ again approximates the theoretical value $0.\overline{81}$, while $\texttt{Exp3-IX}$ is not as close. Revenues do not vary much across different runs. Even when $N=2$, \verb|EXP3-IX| required a considerably larger number of periods to achieve similar results
as \verb|Hedge|. 

\subsection{Soft floors}
\label{sec:results:SFRP}

\citet{zeithammer2019soft} shows that, in a symmetric auction for a single object, bid functions are monotonic. As a consequence, the revenue equivalence theorem \citep{myerson1981optimal} applies,\footnote{Intuitively, under different pricing rules, bidding behavior is also different, in a way that exactly offsets differences in the way prices are computed.} and introducing soft floors in second-price auctions do not affect either the final allocation or the advertising service's revenues. He then demonstrates by way of examples that, with asymmetric bid distributions, revenues in a second-price auction with a soft floor can be either higher or lower than in a standard second-price auction. In this section, we show that, in a more realistic environment in which types are multi-dimensional and bidders can choose which queries to target, different auction formats can yield different revenues even when bidder types are drawn from the same distribution.

Given a soft floor equal to $s$, the price is determined as follows; we describe the case of equal click-through rates for simplicity. Let $b_{(1)}$ and $b_{(2)}$ be the first- and, respectively, second-highest bids. If $b_{(2)} \geq s$, then $p = b_{(2)}$, as in a standard second-price auction. If $b_{(1)} \geq s > b_{(2)}$, then $p = s$, as if $s$ was a standard reserve price (``hard floor''). Crucially, if $s > b_{(1)}$, then the high bidder still wins the auction (on the contrary, with a standard reserve price, the seller would keep the object) and $p = b_{(1)}$. 

For completeness, we first consider the symmetric, single-query environment of \S\ref{sec:results:textbook} and simulate an auction with a soft floor equal to $s = 0.5$. Consistently with \citet{zeithammer2019soft}, we find that soft floors have virtually no impact on revenues.\footnote{Results are as follows (compare with Tables \ref{tab:standard-EXP3IX} and \ref{tab:standard-Hedge}): \texttt{Hedge} with $N=2$ bidders yields average revenues equal to $0.324$ (standard deviation $0.0024$); \texttt{Hedge} with $N=10$ yields $0.834$ ($7e-6$); \texttt{EXP3-IX} with $N=2$ yields $0.377$ ($0.032$); \texttt{EXP3-IX} with $N=10$ yields $0.774$ ($0.0176$).}

Next, we consider a multi-query environment, which is beyond the scope of \citet{zeithammer2019soft}. We assume $N = 3$ bidders, all with the same set of possible types $\mathcal T_1 = \mathcal T_2 = \mathcal T_3 =  \{1,2,3\}$. There are two queries, so $Q = \{1, 2\}$. Values and click-through rates for all bidders are as described in Table \ref{tab:SFRP}. Thus, for example, $V_i(2,1) = 0.25$ and $\mathit{CTR}_i(1, 2) = 0.1$. Both queries are equally likely, and all types are also equally likely. To clarify, these parameters are artificial and purposely chosen to illustrate the point. We let $B = \{\frac{i}{20} \: : \: i = 0,\ldots,20\}$ and $T = 1,000,000$.\footnote{We also ran these simulations with higher values of $T$, and similar patterns emerge.}
%
\begin{table}[ht]
\caption{Values and Click-Through Rates for all bidders}
\vskip 0.1in
\label{tab:SFRP}
\centering
\begin{tabular}{|c|c|c|c|c|}
\hline
$\tau_i$ & $V_i(\tau_i, 1)$ & $\mathit{CTR}_i(\tau_i, 1)$ &  $V_i(\tau_i, 2)$ & $\mathit{CTR}_i(\tau_i, 2)$ \\ \hline
1 & 0.5 & 0.3 & 0.25 & 0.1 \\
2 & 0.25 & 0.1 & 1 & 0.1 \\
3 & 0.25 & 0.1 & 1 & 0.2 \\ \hline
\end{tabular}
\end{table}

Table \ref{tab:SFRP-results-EXP3IX-ming} reports expected revenues per impression (where expectations are taken over search queries, bidder types, and click-through rates) for \verb|EXP3-IX|, averaged over 5 runs. Table \ref{tab:SFRP-results-Hedge-ming} reproduces the results for \verb|Hedge|, with $T = 400,000$.

\begin{table}[ht]
    \caption{Multi-query auctions and soft floors, \texttt{EXP3-IX} output for $T = 1\mathrm M$, averaged over 5 runs.}
    \vskip 0.1in
    \label{tab:SFRP-results-EXP3IX-ming}
    \centering
    \begin{tabular}{|l|c|c|}
    \hline
    Auction format & Mean Revenue & Std. Dev.  \\ \hline
    First price &  $0.0830$ & $0.0007$ \\
    Second price & $0.0509$ & $0.0008$ \\ 
    Soft floor $s=0.65$ & $0.0813$ & $0.0007$ \\ \hline
    \end{tabular}
\end{table}

\begin{table}[ht]
    \caption{Multi-query auctions and soft floors, \texttt{Hedge} output for $T = 400 \mathrm K$, averaged over 5 runs.}
    \vskip 0.1in
    \label{tab:SFRP-results-Hedge-ming}
    \centering
    \begin{tabular}{|l|c|c|}
    \hline
    Auction format & Mean Revenue & Std. Dev.  \\ \hline
    First price &  $0.0691$ & $0.0016$ \\
    Second price & $0.0857$ & $0.0001$ \\ 
    Soft floor $s=0.65$ & $0.0741$ & $0.0061$ \\ \hline
    \end{tabular}
\end{table}

Our first key finding is that, as anticipated, the three auction formats yield different expected revenues.  Soft-floor reserve prices can impact revenues; thus, our simulations provide some support to this common industry practice. Under \texttt{Hedge}, revenues are higher in soft-floor reserve price auctions than in first-price auctions, but second-price auctions perform best. With \texttt{EXP3-IX}, second-price auctions do not fare as well; the highest revenues come from first-price auctions, with soft-floor pricing behind. 

A second key finding is that \texttt{EXP3-IX} results do not align with \texttt{Hedge} even when run for longer periods. Each panel in Figure \ref{fig:hedge-SP} shows the frequency of bids in the final $40,000$ periods of the learning algorithm, summed over 5 runs, divided by type and targeting clause (i.e., queries actually targeted). The main take-away point is that every type eventually learns to choose a specific targeting clause \emph{and} for the most part also places fairly concentrated bids. 
% Figure environment removed

Now compare with Figure \ref{fig:EXP3IX-SP}, which summarizes predicted bids for bidder 1 under \texttt{EXP3-IX}, with Figure \ref{fig:hedge-SP} above (again we sum over 5 runs and only look at the last $100,000$ periods).

% Figure environment removed

Type 1 occasionally bids on one or both queries, showing little convergence even after one million runs. \texttt{EXP3-IX} requires more experimentation, as it only learns from played actions, while \texttt{Hedge} converges faster but assumes learning about unplayed actions. This supports the idea that leveraging knowledge of the pricing mechanism enhances performance. \texttt{Hedge} outperforms due to its reliance on full reward information. We suggest that incorporating heuristics into \texttt{EXP3-IX} may partially address this imbalance; we leave this to future work.

\subsection{Inferring values and bid shading}
\label{sec:results:bid-shading-prod}

Next, we use our approach to infer the distribution of bidders' values from the observed distribution of bids. We utilize \texttt{Hedge} for this inference procedure, due to the bid dispersion observed with \texttt{EXP3-IX} (\S\ref{sec:results:SFRP}).

% The analysis is based on aggregated bid data for two specific shopper queries in an e-commerce setting. To limit the amount of data, we focused on a specific locale, ad placement, and time window. One query ``record sleeves'' averaged 8 bids per auction, relatively low compared to the other high-traffic query ``iPhone 14 case'', which averaged 23 bids.

The analysis is based on aggregated bid data for two specific shopper queries in an e-commerce setting, one characterized by low traffic and the other by high traffic. The data aggregation process converts all bids into a bid per impression, so we set all click-through rates to 1. Thus, we apply our analysis to a symmetric environment with a single query, unit click-through rates, and two different scenarios, one with a low number of bidders (low traffic) and one with a high number of bidders (high traffic). 
% $N=8$ or $N=23$ bidders.

We normalize all bids to lie in a grid $B$ in the interval $[0,4]$. Specifically, we take $B = \{\frac{i}{10} : i = 0,\ldots,40\}$. We then define the set of types to be quantiles of the given bid distribution, with a step size of $0.1$. We initialize the procedure by setting the \emph{inferred} value distribution to the \emph{observed} \underline{bid} distribution. We then apply multiple iterations of the following procedure. First, we run the learning algorithm and derive a \emph{predicted} bid distribution. 

Then, we adjust inferred values for each quantile. To do so, we use the following heuristic. Suppose that, for a given quantile, the currently inferred value is $v$, the observed bid is $b^o$, and the predicted bid is $b^p$. This means that the predicted extent of bid shading (reducing one's bid below one's value) is $\sigma = \frac{b^p}{v}$. We then update $v$ according to $v\leftarrow v + \alpha \left(  \frac{b^o}{\sigma} - v \right)$. Finally, since the inferred values are associated with increasing quantiles, we apply a ``flattening'' step to ensure that they are indeed increasing. This completes one inference iteration. Intuitively, a bidder with value $\frac{b^o}{\sigma}$ who bids by applying a shading factor of $\sigma$ will bid exactly $b^o$, the actually observed bid for this quantile. We then adjust the inferred $v$ in the direction of $\frac{b^o}{\sigma}$, 
applying a learning rate adjustment $\alpha=0.2$. 

Figure \eqref{fig:record-sleeve-prod}  represents the inferred values of bidders for 
% the low-traffic search query, ``record sleeve'',
the low-traffic search query,
under the three pricing rules we analyzed in \S \ref{sec:results:SFRP}: first price, second price, and soft floor with a reserve price of $\$0.65$. We chose these pricing rules arbitrarily, to demonstrate their impact on the inference process. We used 8 iterations of the inference procedure and 3 runs of the learning algorithm per iteration, each with $T = 500,000$, averaging bids over the last $50,000$ periods.

As anticipated, we observe bid shading in the first price auction, as well as within the lower to middle quantiles in the case of the soft floor. The second price auction also displays bid shading at lower quantiles. We hypothesize that this deviation from theoretical prediction is due to a lack of learning amongst low-valuation types: players with low valuation win rarely, so the feedback they receive is coarse on most periods and hence insufficient to converge to bidding one's value.\footnote{We observe the same pattern of deviations from truthful bidding for low types in the second-price auction studied in \S \ref{sec:results:textbook}.}
% Figure environment removed
% \vskip -0.3in
% Figure environment removed

Figure \ref{fig:bid-shading-prod} analyzes the 
% high-traffic query, ``iPhone 14 case''. 
high-traffic query. 
We increased the length of the simulation to $T=800,000$ periods. As one can see, even with a large number of bidders,
% (recall that we use $N=23$), 
inferred values converge in a few iterations.\footnote{We used a realistic pricing function, but for confidentiality reasons we are unable to provide details.}


