\raggedbottom
\section{Learning Models}
\label{sec:learning}

The two main learning models we analyze are \texttt{Hedge} (also referred to as \texttt{Exponential Weights} or \texttt{Multiplicative Weights}) algorithm \citep{auer1995gambling,freund1997decision} and \texttt{EXP3-IX} \citep{kocak2014EXP3IX}. We fix a grid $B$ of possible bid values. To streamline notation, for each bidder $i$ and time period $t$, we let $a_{i,\tau_i,t}$ denote the pair $(b_i, c_i) \in B \times 2^Q \equiv A$ of bid and targeting clause chosen by bidder $i$ of type $\tau_i$ in period $t$. Relative to standard expositions of these algorithms \citep[e.g.]{lattimore2020bandit}, we add incomplete information and incorporate the draw of different queries in each iteration. 

Each bidder type learns from their own observations, not those of other types of the same bidder, to accommodate incomplete information. Learning is not conditioned on the unobserved realized query, reflecting the reality of online ad auctions.

\subsection{Model Setup for Auction Environment}
We use the notation in Table \ref{tab:Notation-for-auction} to describe the auction environment and learning algorithms. Suppose each bidder $i$ bids $b_{i}$ on clause
$c_{i}$, and prices per query are $p=(p_{q})$. Denote by $c_{iq}$ the indicator function that equals 1 if and only if $q \in c_i$. Then bidder $i$'s
expected utility if she wins the auction for the (non-empty) set of queries $\mathcal Q \subseteq \{0,\ldots,Q-1\}$, given her value and CTR for each query $q \in \mathcal Q$ as $v_{iq}$ and $CTR_{i,q}$ respectively, is
% 
\begin{equation}
\label{eq:utility}
EU_{i}(b_i,c_i,p)=\sum_{q \in \mathcal Q} F_Q(q) \cdot CTR_{iq}\cdot c_{iq}\cdot\left[v_{iq}-p_{q}\right],
\end{equation}
where the expectation is due to the random arrival of queries ($F_Q(q)$) and clicks ($CTR_{iq}$). In case of ties, for simplicity of implementation, we just divide the winner's surplus $v_{iq}-p_{q}$ by $N$, rather than (as we should)
the number of tying bidders. Losing bidders get 0 expected utility. Since the \texttt{EXP3-IX} algorithm requires rewards in $[0,1]$, we define
the \textbf{normalized reward} of bidder $i$ as
\vskip -0.1in
\begin{equation}
    \label{eq:reward-norm}
    r_{i}=\frac{EU_{i}(b_i,c_i,p)-(0-b_{\max})}{(v_{\max}-0)-(0-b_{\max})}=\frac{EU_{i}(b_i,c_i,p)+b_{\max}}{v_{\max}+b_{\max}}.
\end{equation}
% 
\vskip -0.1in
\begin{table}[H]
\caption{\label{tab:Notation-for-auction}Notation for online ad auction}
\begin{centering}
\vskip 0.15in
\footnotesize
\begin{tabular}{|c|c|}
\hline 
Object & Notation\tabularnewline
\hline 
\hline 
Bidders & $i=0,\ldots,N-1$\tabularnewline
\hline 
Time horizon & $T$\tabularnewline
\hline 
Bids & $b_{i}\in B \equiv \{0,...,b_{\max}\}$\tabularnewline
\hline 
Queries & $q\in\{0,...,Q-1\}$\tabularnewline
\hline 
Clause & $c_{i}=(c_{iq})_{q=0}^{Q-1}\in\{0,1\}^{Q}$\tabularnewline
\hline 
Value per bidder per query & $v_{iq}\in\{0,...,v_{\max}\}$\tabularnewline
\hline 
CTR per bidder per query & $CTR_{iq}\in[0,1]$\tabularnewline
\hline 
Normalized reward per bidder & $r_i\in[0,1]$\tabularnewline
\hline
\end{tabular}
\par\end{centering}
\vskip -0.1in
\end{table}
This is because the EU of a bidder cannot exceed the maximum value
of the slot for any query if she were to get it for free, i.e., $v_{\max}-0$;
and it is always at least as large as getting a worthless slot and
paying the maximum bid for it, i.e., $0-b_{\max}$. 

In our auction setting, we apply the \texttt{Hedge} and \texttt{EXP3-IX} algorithms, described in the Appendix. Bidders independently use these to update their actions based on observed rewards. \texttt{Hedge} assumes bidders observe rewards for all possible actions, leading to faster learning but less realistic scenarios. \texttt{EXP3-IX} assumes observation of the chosen action's reward, providing a slower but more realistic learning process. Although \texttt{Hedge} is more computationally efficient and less dispersed, a deeper investigation of these algorithms' distributional characteristics is suggested. We chose \texttt{EXP3-IX} over \texttt{EXP3} due to its tighter bounds on realized regret, despite \texttt{EXP3}'s good expected regret guarantees. 

\subsection{Convergence to Equilibrium}
\label{sec:coarseBCE}

Since both \texttt{Hedge} and \texttt{EXP3-IX} ensure that each type will have vanishing regret in the limit as $T \to \infty$, Lemma 10 in \citet{hartline2015coarseBCE} implies that the resulting dynamics will converge to a version of correlated equilibrium for games with incomplete information. We now formally define this equilibrium notion for the class of games we consider. 

For every bidder $i$, let $\mathbf s_i: \mathcal T_i \to B \times \{0,1\}^Q$ be a \emph{strategy} for bidder $i$: for each type $\tau_i \in \mathcal T_i$, a strategy specifies a bid and a targeting clause: $\mathbf s_i(\tau_i) = (b_i,c_i)$. Let $\mathbf S_i$ be the set of all  such strategies for $i$, and define the Cartesian product sets $\mathbf S = \prod_i \mathbf S_i$ and $\mathbf S_{-i} = \prod_{j \neq i} \mathbf S_j$ as usual. For $\mathbf s \in \mathbf S$ and $\tau \in \prod_i \mathcal T_i$, by convention $\mathbf s(\tau) = \left(\mathbf s_i(\tau_i) \right)_{i \in N}$; similarly,  $\mathbf s_{-i}(\tau_{-i}) = \left(\mathbf s_j(\tau_j) \right)_{j \neq i}$.

To describe the outcome of the auction, for every query $q$, let $\mathbf p_q : (B \times \{0,1\}^Q)^N \to \mathbb R$ denote the pricing rule of the auction, which associates the price charged to the winner with any vector of bids and targeting clauses. Furthermore, for each $i = 0,\ldots,N-1$ and $q \in Q$, let $\mathbf w_{iq} : (B \times \{0,1\}^Q)^N \to [0,1]$ denote the probability that bidder $i$ wins the auction for query $q$ given the bids and clauses submitted by all players. We assume that, for every query $q$ and tuple $(b,c) \in (B \times \{0,1\}^Q)^N$, $\sum_i \mathbf w_{iq}(b,c) = 1$  and $\mathbf w_{iq}(b,c)>0$ only if $c_{iq} = 1$; that is, in order to win the auction for a query, the bidder must have included it in her targeting clause.

We can then define the \emph{payoff} of bidder $i$'s type $\tau_i$, given a tuple $(b,c) \in (B \times \{0,1\}^Q)^N$ of bids and targeting clauses for every player, as
\[
U_i(b,c; \tau_i) = \sum_{q=0}^{Q-1} F_Q(q) \cdot CTR_{iq}(\tau_i)\cdot \mathbf w_{iq}(b,c) \cdot\left[V_{iq}(\tau_i)-\mathbf p_{q}(b,c)\right].
\]
With these definitions, a probability distribution $\sigma \in \Delta(\mathbf S)$ is a \textbf{coarse Bayes Correlated Equilibrium} (coarse BCE) if, for every $i = 0,\ldots,N-1$, $(b_i,c_i) \in B \times \{0,1\}^Q$, and $\tau_i \in \mathcal T_i$,
\[
\mathrm E_{\mathbf s \sim \sigma, \tau_j \sim F_j: j \neq i} U_i(\mathbf s(\tau); \tau_i)) \geq \mathrm E_{\mathbf s \sim \sigma, \tau_j \sim F_j: j \neq i} U_i\left( (b_i, c_i), \mathbf s_{-i}(\tau_{-i}); \tau_i\right).
\]
That is, for each type $\tau_i$, there is no single action $(b_i,c_i)$ that improves upon her payoff if she conforms to the profile $\sigma$. This differs from the notion of a (non-coarse) Bayes Correlated Equilibrium \citep{bergemann2016bayes} in that it does not rule out more sophisticated deviations, in which type $\tau_i$ chooses different actions $b_i,c_i$ for each recommended action $\mathbf s_i(\tau_i)$ in the support of the marginal of the equilibrium distribution $\sigma$ on $i$'s strategy space $\mathbf S_i$.  

