% \raggedbottom
\section{Auction Model and Learning}
% \label{sec:model}

This section introduces the auction model we analyze. For any finite set $\mathcal{X}$, we denote by $\Delta(\mathcal{X})$ the set of probability distributions on $\mathcal{X}$.

\subsection{Auction Model}
\label{sec:model}

We fix a single ad slot, or position, within the ad service's or publisher's web page. 
Time is discrete and indexed by $t = 1,2,\ldots,T$. In each time period $t$, a user visits the web page and submits a query, represented by a point $q$ in a finite set $Q$. The probability that query $q$ is submitted at time $t$ is given by the probability distribution $F_Q \in \Delta(Q)$. The ad service then runs an auction to determine which ad to show in the given slot.

There are $N$ (potential) bidders, whom we also refer to as advertisers, indexed by $i = 1,\ldots,N$. Each bidder $i$ is characterized by a \emph{type} $\tau_i$ drawn at the beginning of each period from a finite set $\mathcal T_i$ according to a distribution $F_i \in \Delta(\mathcal T_i)$. We also fix functions $V_i : \mathcal T_i \times Q \to \mathbb R$ and $\mathit{CTR}_i : \mathcal T_i \times Q \to [0,1]$ such that, whenever bidder $i$'s type is $\tau_i \in \mathcal T_i$ and the shopper query is $q \in Q$, $i$'s \emph{value per click} is $V_i(\tau_i, q)$ and the ad \emph{click-through rate} is $\mathit{CTR}_i(\tau_i, q)$; these functions are in practice represented by matrices. 

In every period, each bidder $i$ simultaneously submits a \emph{bid} $b_i \in \mathbb R$ and a \emph{targeting clause} $c_i \subset Q$. The interpretation is that a targeting clause is the subset of possible shopper queries that bidder $i$ wants to bid on: the bidder participates in the auction only if the user's query $q$ is an element of $c_i$. The timing is as follows: first, bidder $i$ observes her type $\tau_i$; then, she places a bid $b_i$ and a targeting clause $c_i$; finally, the user query $q \in Q$ is realized. Thus, bidders \emph{cannot} tailor their bid to the specific user query in a given period. On the other hand, their bids can and will in practice depend upon their realized type. The winner of the auction is the bidder with the highest \emph{score}, i.e., the product of their bid and their click-through rate, provided they targeted the realized query. Formally, if the realized user query is $q$ and each bidder $i$ has type $\tau_i$, bids $b_i$, and targets $c_i$, then bidder $i$'s score is $s_i = b_i \cdot \mathit{CTR}_i(\tau_i, q)$ if $q \in c_i$ and $0$ otherwise; the winner is any $i \in \arg \max_{j \in N : q \in c_j} s_j$. Ties are broken randomly.

If the realized user query is $q$, bidder $i$ wins the auction, and the charged \emph{price per click} is $p$, then the average reward (payoff) to bidder $i$ with type $\tau_i$ is
$\mathit{CTR}_i(\tau_i, q) \cdot \left[ V_i(\tau_i, q) - p \right]$. Finally, if bidder $i$'s targeting clause $c_i$ does not include the realized $q$, or if it does but $b_i$ is not the winning bid, her payoff is 0. 

The price charged to the winner depends upon the auction rules; we consider different cases in \S \ref{sec:results}, so we defer the specifics until then. We also specify what bidders observe at the end of each period in \S \ref{sec:learning}, as that is a function of the learning algorithm under consideration. However, we maintain throughout that bidders \emph{only} observe their own reward, and not others' rewards or bids.

The concept of ``type'', as introduced by \citet{harsanyi1967games}, is key to model games with incomplete information: in every period, each bidder $i$ knows the distribution $F_j$ of her opponents $j$, but not their realized type. In one common interpretation, there is a population of potential bidders for every bidder role $i$, each with distinct value per click and click-through rate; in each period $t$, and for every bidder role $i$, a specific element of the corresponding population is drawn according to $F_i$. A symmetric environment is one where  the set of bidder types and their distribution are the same for every bidder; we use this in \S \ref{sec:results} for aggregate-level bidding behavior analysis. One alternative interpretation is that there is a single player (a firm) for every bidder role $i$; her type can represent the fact that the firm may be selling a range of goods, with per-period variations in values and click-through rates reflecting product-specific margins and conversion rates, perhaps due to cost variability, promotions, etc.\footnote{In either interpretation, formally, types are indices for the functions $V_i$ and $\mathit{CTR}_i$ with no intrinsic meaning.}

Types can also be used to model (indirectly) a \emph{random number of bidders}. Suppose that, for each bidder (or for a subset of bidders), we define a type $\tau_i^0$ who has zero value and click-through rate for every query. Then, in our learning algorithms, this bidder type will eventually stop bidding. Hence, out of $N$ potential bidders, only a subset may be active, the ones whose types are different from $\tau_i^0$. We assume type distributions are independent across bidders, although this could be relaxed as our learning algorithms do not rely on independence.

\emph{Standard auctions} are a special case of our model in which there is a single query: this can be captured by assuming that the set $Q$ of queries is a singleton, say $\{q^*\}$. Thus, a bidder's type $\tau_i$ fully characterizes her value for the object, and one may as well take $V_i$ to be the identity: $V_i(\tau_i, q^*) = \tau_i$. Traditional ``textbook'' auctions do not consider click-through rates; this can be captured by assuming that each function $\mathit{CTR}_i(\cdot)$ is identically equal to 1. However, pay-per-click auctions are common in advertising, so we consider them ``standard'' as well.

Targeting clauses provide flexibility in ad auctions. For instance, queries about USB chargers can vary in specificity and suggest differing shopper expertise. This results in varied values and click-through rates for advertisers, causing them to target different query types. For example, a wide-range product advertiser might target ``USB charger'' and ``USB type C charger'', while an Apple-specific advertiser might target ``iPhone charger''.
