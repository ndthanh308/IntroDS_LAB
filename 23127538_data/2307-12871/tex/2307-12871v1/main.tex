\documentclass[conference]{IEEEtran4PSCC}
%\documentclass[conference]{IEEEtran}
\renewcommand{\arraystretch}{2}
\IEEEoverridecommandlockouts
% The preceding line is only needed to identify funding in the first footnote. If that is unneeded, please comment it out.
\usepackage{cite}
\usepackage{amsmath,amssymb,amsfonts}
\usepackage{algorithmic}
\usepackage{graphicx}
\usepackage{textcomp}
\usepackage[caption=false,font=footnotesize]{subfig}
\usepackage{color}
\usepackage{xcolor}
\usepackage{cite}
\usepackage{amsmath}
\usepackage{amssymb}
\usepackage{amsthm}
\usepackage{graphicx}
%\usepackage[draft]{graphicx}
\newtheorem{theorem}{Theorem}
\newtheorem{prop}{Proposition}
\newcommand{\indep}{\rotatebox[origin=c]{90}{$\models$}}
\usepackage{epstopdf}
\usepackage{comment}
\usepackage{multirow}
\usepackage{paralist}
\usepackage{lineno}
\usepackage{mdwlist}
\usepackage{eurosym}\DeclareGraphicsExtensions{.pdf,.png,.jpg}
\usepackage{breqn}
\usepackage{makecell}
\usepackage{soul}
\usepackage{empheq}
\graphicspath{{./pic/}}
\usepackage[super]{nth}
\usepackage{subfig}
\usepackage{bm}
\usepackage[normalem]{ulem}
\usepackage{xcolor}
\def\BibTeX{{\rm B\kern-.05em{\sc i\kern-.025em b}\kern-.08em
    T\kern-.1667em\lower.7ex\hbox{E}\kern-.125emX}}

\newcommand\upj{\mathord{\mathrm{j}}}
\usepackage{accents}
\newcommand{\ts}{\textsuperscript}

\begin{document}
\title{Topology-aware Piecewise Linearization of the AC Power Flow through Generative Modeling}
\author{\IEEEauthorblockN{Young-ho Cho and Hao Zhu}
\IEEEauthorblockA{Chandra Department of Electrical and Computer Engineering \\
The University of Texas at Austin\\
Austin, TX, USA \\
\{jacobcho, haozhu\}@utexas.edu}
}
\maketitle



\begin{abstract}
    Effective power flow modeling critically affects  the ability to efficiently solve large-scale grid optimization problems, especially those with topology-related decision variables.
    %Due to the nonlinearity of the active and reactive power flows, the corresponding optimization problems have high model complexities, which leads hard to obtain the optimal solution.
    %To reduce the model complexity, we propose a power flow generative model that approximates the common terms that active and reactive power flows have in common.
    In this work, we put forth a generative modeling approach to obtain a piecewise linear (PWL) approximation of AC power flow by training a simple neural network model from actual data samples. By using the ReLU activation, the NN models can produce a PWL mapping from the input voltage magnitudes and angles to the output power flow and injection. Our proposed generative PWL model uniquely accounts for the nonlinear and topology-related couplings of power flow models, and thus it can greatly improve the accuracy and consistency of output power variables. Most importantly, it enables to reformulate the nonlinear power flow and line status-related constraints into mixed-integer linear ones, such that one can efficiently solve grid topology optimization tasks like the AC optimal transmission switching (OTS) problem.  Numerical tests using the IEEE 14- and 118-bus test systems have demonstrated the modeling accuracy of the proposed PWL approximation using a generative approach, as well as its ability in enabling competitive  OTS solutions at very low computation order. 
\end{abstract}
	
	
	
\begin{IEEEkeywords}
Generative modeling, piecewise linear approximation, nonlinear AC power flow, grid topology optimization.
\end{IEEEkeywords}
	
\thanksto{\protect\rule{0pt}{0mm}
This work has been supported by NSF Grants 1802319 and 2130706.}

	
	


\section{Introduction}


\IEEEPARstart{E}{ffective} power flow modeling is critical  for analyzing and optimizing large-scale power systems for efficient and reliable grid operations. With worldwide energy transitions and decarbonization, grid optimization tasks  are increasingly  challenged by e.g., uncertainty factors, extreme conditions, and fast computation needs. Meanwhile, optimizing the grid topology for increased flexibility  has been advocated in problems like optimal transmission switching (OTS) \cite{fisher2008optimal}, adaptive islanding \cite{trodden2013optimization}, and post-disaster restoration  \cite{chen2017sequential}. Hence, it is important to develop effective power flow models that can facilitate the accurate and fast solutions for grid optimization problems, especially those with the combinatorial topology variables.   
%Due to the nonlinear power flows, the power system optimization problems have a high model complexity, leading to a computational burden and difficulty in obtaining optimal solutions.
%Furthermore, the grid topology optimization problems are extremely hard to solve since the topology-related decision variables have to be considered with the power flows. 
%Hence, those problems implement the approximation models of nonlinear power flow to obtain the solutions.







%\IEEEPARstart{R}{ecently}, we have faced plenty of power system optimization problems to maintain the system stable and reliable. For instance, we need to solve optimal power flow~\cite{liu2022topology}, optimal transmission switching~\cite{fisher2008optimal}, and unit commitment~\cite{padhy2004unit}. Those optimization problems typically have nonlinear AC power flow constraints. The nonlinear AC power flows make a problem non-convex, leading to a high-complexity model with a computational burden and difficulty obtaining optimal solutions.
%methods for efficient grid analysis and management

%DC, lin AC, piecewise lin AC
There exist significant efforts in developing approximation models for the nonlinear AC power flow.
Notably, linearized power flow models have been advocated due to their simplicity, such as the well-known DC model~\cite{stott2009dc}, or the first-order approximation at an operating point for better accuracy~\cite{coffrin2014linear}.
As linear models are very limited by their generalizability across all possible operation region~\cite{trodden2013optimization},
one straightforward extension is the piecewise linear (PWL) approximation approach by using multiple operating points; see e.g.,~\cite{brown2020transmission}. By and large, there is a trade-off between accuracy and complexity for these model-based approaches, while the number and location of operating points could be difficult to select. 
%Although the PWL model can approximate the nonlinear power flows with a consistent error level, selecting the operating points for approximation is still challenging.



%In~\cite{stott2009dc}, DC approximated power flow was used for  constraints of the AC power flows. However, since DC approximation ignores reactive powers, we cannot guarantee the determined solutions are feasible for power system operation~\cite{baker2021solutions}.
%a linear approximation of AC power flows was proposed to deal with the active and reactive power flows.
%In~\cite{coffrin2014linear}, the small-angle approximation was used to approximate the cosine and sine functions in the active and reactive power flows.
%However, the linear approximation of the cosine function incurs a significant error among the nonlinear terms in the AC power flows~\cite{trodden2013optimization}.
%To solve this problem, the cosine function was through the piecewise linear functions with manually selected approximation points in~\cite{brown2020transmission}. However, finding the required number of approximation points and their corresponding values is not trivial.


%NN AC
%There exist some efforts using data-driven models that can train the appropriate approximation points through datasets.
%By eliminating the need for manual selection of the operating points, 
To tackle these issues, data-driven approaches have been recently advocated as an alternative  for PWL modeling. Trained from realistic power flow scenarios, machine learning models such as $K$-plane regression~\cite{chen2021data} and neural networks (NNs) \cite{kody2022modeling,chen2023efficient} have shown good power flow approximation capabilities.
In particular, ReLU-based NNs can be used to construct simple yet accurate PWL power flow models by incorporating the power flow Jacobian information \cite{kody2022modeling}. Similar ideas have been explored in the constraint learning framework \cite{chen2023efficient} but for general grid operational constraints. Interestingly, these PWL models under the ReLU activation allow to reformulate grid optimization problems with nonlinear power flow as mixed-integer linear programs (MILPs), for which there exist efficient off-the-shelf solvers.  For example, successful applications to  unit commitment and distribution management have been considered. 
%However, the model is directly derived from the dataset without consideration of underlying physics related to voltages and power flows.
%To capture the underlying physics, the constraint learning model was proposed to replicate the power system operation constraints~\cite {chen2023efficient}.
%Although the constraint learning model makes the problem easy to solve, the model can only utilize specific problems with the trained constraints.
%In~\cite{kody2022modeling}, the PWL modeling that surrogates power flows was suggested to update the Jacobian matrix, which has the gradient information at each operating point.
%Once the surrogate model is trained, we can replace the power flow constraints of any grid optimization problems.
%Since the ReLU activation function can disable negative inputs, we can choose the desired linear region based on the activation statuses of the ReLU functions.
%However, each activation function considers the active and reactive powers individually, making it fail to explore the coupling between active and reactive powers.
%However, the proposed model fails to explore the coupling between active and reactive powers.
%Moreover, since early works focus power flows on the fixed topology, the models cannot be used for the grid topology optimization problems.
Albeit the success, these existing approaches mainly build on an end-to-end learning framework that does not consider the underlying physical models of power flow. In addition, none of them has yet considered a flexible grid topology. 



%Meanwhile, the ReLU activation function in the neural network can separate the piecewise linear regions.

%\textcolor{blue}{(plz rewrite the following based on my abstract (like after the first sentence). Here you can have more details on commone nonlinear terms, the four-layer and loss function  design, but plz make sure to capture the impotant components of our work. I can't believe you finish the discussions of your design in just the first sentence!!!! )}
In this paper, we put forth  a generative modeling approach to obtain the PWL approximation of AC power flow that is directly applicable to complex grid optimization problems.
With the ReLU activation, we  design the NN architecture to uniquely explore the generative structure of AC power flow.  %function to piecewise linearly map the relation between input voltages and output power flows and injections.
With the voltage and angle inputs, our proposed NN model first predicts the nonlinear terms that are common to all power variables in the first two layers, and then transforms these common terms to all line flows and power injections with two more layers. All the layers will be jointly trained to ensure an excellent consistency among all variables. This way, the proposed PWL model is generated in accordance with the power flow physics, while able to incorporate flexible topology connectivity.
%as part of the design. 
%The PWL model uniquely deals with the nonlinear and topology-related couplings of active and reactive power flows in the first two layers.
%The PWL model tries to generate the power flows for all lines and injections at all nodes through the fixed weight parameters in the last two layers.
Thanks to our proposed NN design,  we can cast grid optimization tasks like OTS %also reformulate the PWL model and binary line status relations 
into an MILP form for efficient solutions.
%In this paper, we adopt a single trainable hidden layer neural network to synthesize the piecewise linear (PWL) model, which approximates the nonlinear common terms in the active and reactive power flows.
%Then, we transform the neural network to the mixed integer linear program (MILP) constraints.
%We also implement the line-switching status binary variables to consider any possible line connectivity.
We use the IEEE 14- and 118-bus test systems to validate the proposed PWL approximation in terms of improving power flow modeling accuracy over non-generative, as well as an excellent optimality/feasibility performance in solving AC-OTS. 
%In addition, the optimal transmission switching (OTS) results that utilize the PWL functions are verified with the result of the ground truth solver.
%Finally, we expect the proposed PWL model can well approximate the nonlinear AC power flows, and implementing the PWL model can obtain the optimal solution while reducing a computational burden.
%Our contributions are as follows:
%\begin{enumerate}
%    \item We generate the PWL model to approximate nonlinear power flows while retaining the underlying couplings between active and reactive power flows.
%    \item We propose the formulation steps to attain an MILP for the PWL model and line status-related constraints to solve grid topology optimization problems.
%\end{enumerate}

%\textcolor{blue}{(rewrite the following, is common term even clear based on what you described above?and why linear approximation is important for sec 2??  PWL not used? could you do some serious editing of your write-up???
%what is application of considering binary variables? )}
The rest of the paper is organized as follows. Section~\ref{sec:lin} provides the nonlinear AC power flow modeling.
In Section~\ref{sec:piecewise}, we develop the PWL model that uses a neural network and discuss the formulation steps to attain a mixed-integer program.
Section~\ref{sec:sim} provides the simulation set-up for the IEEE 14- and 118-bus test systems and presents the numerical comparisons and validations for the proposed scheme, along with some concluding remarks.




\section{Nonlinear AC Power Flow Modeling} \label{sec:lin}

We first present the nonlinear AC power flow modeling for transmission systems, while introducing the relevant topology status variables and coupling terms useful for the discussions later on.
Consider a transmission system consisting of $n$ buses collected in the set $\mathcal{N} := \{1,\dots, n\}$ and $\ell$ lines (including transformers) in $\mathcal{L} := \{(i, j)\} \subset \mathcal{N} \times \mathcal{N}$. For each bus $i\in\mathcal{N}$, let $V_i\angle\theta_i$ denote the complex nodal voltage phasor, and  $\{P_i,Q_i\}$ denote the active and reactive power injections, respectively. For each line $(i,j)\in\mathcal{L}$, let $\theta_{ij}:= \theta_i-\theta_j$ denote the angle difference between bus $i$ and $j$, and $\{P_{ij},Q_{ij}\}$ denote the active and reactive power flows from bus $i$ to $j$; and similarly for $\{P_{ji},Q_{ji}\}$ from bus $j$ to $i$.
In addition, the line's series and shunt admittance values are respectively denoted by $y_{ij}= g_{ij}+\upj b_{ij}$ and $y_{ij}^{sh}=g^{sh}_i+\upj b^{sh}_i$.
%The power flows from bus $j$ to $i$ also need to be considered, and $\{P_{ji},Q_{ji}\}$ share the system parameters with $\{P_{ij},Q_{ij}\}$.


By defining a binary variable $\epsilon_{ij} \in \{0,1\}$ to indicate the status for each line $(i,j)\in\mathcal{L}$ (0/1: off/on),  the nodal power balance at bus $i$  per the Kirchhoff's law becomes
\begin{subequations}\label{powerinj}
\begin{align}
P_{i} &= \textstyle \sum_{(i,j) \in \mathcal{L} } ~\epsilon_{ij} P_{ij},\\
Q_{i} &= \textstyle \sum_{(i,j) \in \mathcal{L} } ~\epsilon_{ij} Q_{ij}.
\end{align}
\end{subequations}
Note that these binary variables $\{\epsilon_{ij}\}$ will be important for formulating topology-related grid optimization tasks such as the optimal transmission switching problem as detailed later on. Without any topology changes, they can be fixed at $\epsilon_{ij}=1$. 
For each line $(i,j)\in\mathcal{L}$, the power flows relate to the angle difference $\theta_{ij}$ and nodal voltages $\{V_i, V_j\}$, and in the case of transformer, its tap ratio $a_{ij}$, as given by%~\cite{jegatheesan2008newton} 
%
\begin{subequations}
\label{powerflow}
\begin{align}
    P_{ij}&= V^2_{i} (\frac{ g_{ij}}{ a^2_{ij}}+ g^{sh}_i) - \frac{ V_{i}  V_{j}}{ a_{ij}} ( g_{ij}\cos  \theta_{ij}+ b_{ij}\sin  \theta_{ij}), \label{powerflow_1}\\
    Q_{ij}&= \! - V^2_{i} (\frac{ b_{ij}}{ a^2_{ij}}+ b^{sh}_i) \! -\! \frac{ V_{i}  V_{j}}{ a_{ij}} ( g_{ij}\sin  \theta_{ij} \!- \!b_{ij}\cos  \theta_{ij}).
\end{align}
\end{subequations}
For the transmission lines, we can simply set  $a_{ij}=1$. As for a transformer, the tap ratio is typically set within the range of $[0.9, 1.1]$ and it only affects the primary-to-secondary direction.  Thus, for the power flows in the secondary-to-primary direction, one can use $a_{ij}=1$ in \eqref{powerflow}.


%\subsection{Active and Reactive Power Flow Coupling}
One advantage of our proposed piecewise linear (PWL) approximation is to leverage the underlying coupling among active and reactive power flows. To this end, let us denote the three nonlinear terms in  \eqref{powerflow} by  
\[\gamma_i := V^2_i,~\rho_{ij}:= V_i  V_j \cos  \theta_{ij},~\mathrm{and}~\pi_{ij}:= V_i  V_j \sin  \theta_{ij}.\]
To form the bi-directional power flows $\{P_{ij},Q_{ij},P_{ji},Q_{ji}\}$ per line $(i,j)$,  we only need  the nonlinear terms $\{\gamma_i,\gamma_j,\rho_{ij},\pi_{ij}\}$, and the mapping between the two groups of variables is simply linear. %, as 
%\begin{subequations}\label{powerflow_n}
%\begin{align}
%     P_{ij}&= \gamma_i (\frac{ g_{ij}}{ a^2_{ij}}+ g^{sh}_i)- \rho_{ij} \frac{ g_{ij}}{ a_{ij}} -  \pi_{ij} \frac{ b_{ij}}{ a_{ij}}, \label{powerflow_1_n}\\
%     Q_{ij}&=- \gamma_i  (\frac{ b_{ij}}{ a^2_{ij}}+ b^{sh}_i)+ \rho_{ij} \frac{ b_{ij}}{ a_{ij}}-  \pi_{ij} \frac{ g_{ij}}{ a_{ij}}.\label{powerflow_2_n}
%\end{align}
%\end{subequations}
%Note that the power flows $\{P_{ji}, Q_{ji}\}$ would also depend on the same terms $\{\rho_{ij},\pi_{ij}\}$, in addition to $\gamma_j$. 
%This slight reduction in the dimensionality would enable 
To represent the resultant linear relation of \eqref{powerflow} in a matrix-vector form,  let us concatenate all the power flow variables in $\bm z^{pf} \in \mathbb{R}^{4\ell}$, and all the injection ones in $\bm z^{inj}\in \mathbb{R}^{2n}$. In addition, let $\bm \gamma \in \mathbb{R}^{n}$, $\bm \rho \in \mathbb{R}^{\ell}$, and $\bm \pi \in \mathbb{R}^{\ell}$ denote the respective vectors for the three groups of common terms. This way, we have 
%Then, the power flows $\bm Z^{pf} \in \mathbb{R}^{2m}$ and injections $\bm Z^{inj}\in \mathbb{R}^{2n}$ can be calculated through the common terms and the fixed-weight matrices.
\begin{subequations}\label{pow}
\begin{align}
    \bm z^{pf} &= \bm W^{\gamma} \bm \gamma + \bm W^{\rho} \bm \rho + \bm W^{\pi} \bm \pi,\label{powa}\\
    \bm z^{inj} &= \bm W^{\psi} \bm z^{pf}
\label{powb}
\end{align}
\end{subequations}
where the weight matrices $\{\bm W^{\gamma}, \bm W^{\rho}, \bm W^{\pi}, \bm W^{\psi}\}$ are of appropriate dimension  given by the known line parameters and line status variables in \eqref{powerflow} and \eqref{powerinj}, respectively.
Clearly, the three groups of  common terms are sufficient for fully  generating  all power flow and injection quantities, and our proposed generative modeling will work by predicting these terms as the first step. 


\subsection{Linear Approximation}
\label{sec:linappx}
We discuss the linear approximation for the common terms, which will be used by the proposed PWL models. 
%The common terms consist of the bus voltage magnitude and angle difference.
%Due to the nonlinearity arising from the common terms, the problems that utilize the power flow have high model complexity.
%We need to reduce the model complexity by linearly approximating the power flows to make solvers find optimal and feasible solutions quickly.
Linear approximation is a basic approach to deal with power flow nonlinearity, thanks to its simplicity and reasonable accuracy within a small region of the operating point. We will consider the first-order approximation method to attain a linearized modeling, while there also exist other popular methods such as fixed-point method~\cite{simpson2017theory}. 

For the squared voltage term, it can be approximated by  $\hat \gamma_i=  2V_i - 1$, $\forall i\in\mathcal{N}$, based on a flat-voltage value of $V_o$. Of course, this linearized model can be improved by using the exact operating point if different from the flat-voltage profile. As shown in~\cite{trodden2013optimization}, the former already attains a very high accuracy for power systems with well-regulated bus voltages within the p.u. range of [0.94, 1.06]. Thus, for simplicity, this linear representation of $\hat {\bm{\gamma}}$ will be adopted in this work. 


Nonetheless, the other two terms $\bm \rho$ and $\bm \pi$ are more complicated to approximate than $\bm \gamma$ due to the presence of angle differences. To this end, we consider the first-order approximation for the former at the operating point. 
To simplify the notation, 
let us use $[\bm \rho;~\bm \pi] = \bm f(\bm x) \in \mathbb R^{2\ell}$ to represent the nonlinear mapping from the input $\bm x$, which consists of the voltage magnitude $\bm V =\{V_i\}_{i\in\mathcal N} \in \mathbb R^{n}$ and the angle difference  $\bm \theta =\{\theta_{ij}\}_{(i,j)\in\mathcal{L}} \in \mathbb R^{\ell}$.
%The corresponding outputs of $\bm f$ are $[\bm \rho; \bm \pi] \in \mathbb R^{2m}$.
With a fixed operating point denoted by $\bm x_o$,  the first-order approximation becomes  
\begin{align}
    [\hat{\bm \rho}; \hat{\bm \pi}]&= \bm f(\bm x_o)+ \bm J(\bm x_o)(\bm x - \bm x_o) \nonumber\\
    &= \bm f(\bm x_o)+ \bm J(\bm x_o)\Delta \bm x \label{linearf}
\end{align}
where $\bm J(\bm x_o)$ denotes the Jacobian matrix of $\bm f(\bm x)$ evaluated at  $\bm x_o$, while we use $\Delta \bm x := \bm x -\bm x_o$ for simplicity. 
The ensuing section will build upon the linear model in \eqref{linearf} by using a data-driven approach to improve the approximation accuracy.  
%The approximated $\bm \gamma$, $\bm \rho$, and $\bm \pi$ are represented as $\hat{\bm \gamma}$, $\hat {\bm \rho}$, and $\hat {\bm \pi}$.


% Figure environment removed

\section{PWL Approximation via Generative Modeling} \label{sec:piecewise}

Our proposed PWL model uses a two-layer neural network (NN) to first approximate the common nonlinear terms in  $[\bm \rho;\bm \pi]$, followed by two additional linear layers to generate the power flow and injection variables. 
As illustrated in Fig.~\ref{structure}, using the input voltage and angle difference in $\bm x =[\bm V;\bm\theta]$, the first two layers will rely on the ReLU activation functions to form the best PWL model for $[\hat{\bm \rho}; \hat{\bm \pi}]$  by adjusting the NN parameters. In addition, the last two layers use the fixed weight parameters from \eqref{powa} to generate the power flows for all lines, and accordingly, use \eqref{powb} to generate the power injection at all nodes. Note that the squared voltage $\hat{\bm \gamma}$ used by the third layer of generating power flow is based on the simple linearized model as described in Sec.~\ref{sec:linappx}. Thus, the proposed approximation fully matches the power flow relations and coupling among different terms, in an efficient and generative fashion. 

Using the ReLU activation, the first two layers can effectively produce a PWL mapping that can improve the accuracy of the linearized model in \eqref{linearf}. The ReLU function is defined by $\sigma(\cdot)$ where outputs the entry-wise maximum between the input value and 0.
\begin{comment}
\begin{align}
\begin{split}
    \Delta \bm y &=\bm J(\bm x_o)\Delta \bm x + \bm w_2 (\bm w^\top_1 \Delta \bm x+ b)\\
    &=\Big(\bm J(\bm x_o) + \bm w_2 \bm w^\top_1 \Big) \Delta \bm x + \bm w_2 b\\
    &=\bm J(\bm x_1) \Delta \bm x+ \Big(\bm J(\bm x_1)(\bm x_o - \bm x_1) + \bm r \Big)\\
    &=\bm J(\bm x_1)\Big(\Delta \bm x + (\bm x_o - \bm x_1)\Big) + \bm r.
\end{split}
\end{align}
\end{comment}
%Intuitively, when the activation status of ReLU function stays unchanged within a certain region of input values (i.e., varying $\bm x$ in the region does not affect the activation), its functional output enjoys the same linear relation with the input in that region.
Intuitively, when the activation status of ReLU function stays unchanged within a certain region of input values, its functional output enjoys the same linear relation with the input in that region.
Therefore, one can view that the combination of ReLU activation status would divide the whole input space into multiple smaller regions, within each it boils down to a purely linear function. Thus, the overall function over the whole input space becomes a PWL one. In this sense, the number of linear regions will grow exponentially with the number of activation functions, and it would be challenging to search for all possible combinations. Therefore, we will train the NN parameters within the first two layers from generated data samples that can best select the activation status and linear regions from the data. 

%We can show that the proposed neural network can be interpreted as piecewise linear (PWL) functions through two key assumptions.
To concretely connect the NN model with PWL functions, we first consider a simple case of two linear regions obtained by using two different operating points, namely $\bm x_o$ and $\bm x_1$, as
\begin{align}\nonumber
    [\hat{\bm \rho}; \hat{\bm \pi}]& =   \bm f(\bm x_o) + \Delta \bm y,~\mathrm{with}\\
    \label{dy} \Delta \bm y &= \begin{cases}
    \bm J(\bm x_o)\Delta\bm x, & \bm x \in \mathcal{R}_o \\[1\jot]
    \bm J(\bm x_1)(\bm x - \bm x_1) + \bm r, & \bm x \in \mathcal{R}_1
    \end{cases} 
\end{align}
where $\mathcal{R}_q$ represents the linear region corresponding to $\bm x_q$, and  the residue in $\mathcal{R}_1$ is given by $ \bm r := \bm f(\bm x_1) - \bm f(\bm x_o)$. 


To recover the NN structure for \eqref{dy}, we follow \cite{eckart1936approximation} to assume that the two Jacobian matrices therein are different by a low-rank component. Specifically, we assume that 
%As an example, $\bm J(\bm x_1)$ can express as the sum of $\bm J(\bm x_o)$ and the weight matrix $\bm W$ that can make a transition.
%Then, we use the second assumption, which is that $\bm W$ can be approximated through a low-rank surrogate matrix generated by the product of $\bm w_2$ and $\bm w_1$.
\begin{align}
    \bm J(\bm x_1) \approxeq \bm J(\bm x_o)+ \bm w_2 \bm w^\top_1 \label{Jac2}
\end{align}
where both $\bm w_1$ and $\bm w_2$ consist of the NN weight parameters, that can be of much lower dimension than the size of Jacobian matrix.
For simplicity, we consider both of them to be vectors with $\bm w_1 \in \mathbb R^{n+\ell}$ and $\bm w_2 \in \mathbb R^{2\ell}$, and thus the difference term in \eqref{Jac2} becomes a rank-one matrix. This will be expanded to a higher-rank case later by using weight matrices. 
%We will expand column vectors $\{\bm w_1, \bm w_2\}$ to matrices $\{\bm W_1, \bm W_2\}$, which makes the rank of the difference term becomes higher than one later.
%Note that the number of column vectors in $\{\bm W_1, \bm W_2\}$ represents the number of the required activation functions.
Interestingly, the simplification in \eqref{Jac2} allows to unify the two scenarios in \eqref{dy} by using one ReLU activation function, as given by 
\begin{align}
    \Delta \bm y \approxeq \bm J(\bm x_o)\Delta \bm x + \bm w_2 \sigma(\bm w^\top_1 \Delta \bm x+ b)\label{one_q}
\end{align}
where $b$ is a scalar bias parameter.
When the ReLU function is not activated, it becomes the linear model in $\mathcal R_o$ of \eqref{dy}. Otherwise, upon the activation of $\sigma(\cdot)$ the resultant linear model should approach the one in $\mathcal R_1$ by recognizing the relation between the two Jacobian matrices in \eqref{Jac2}, as given by
\begin{align}
    \Delta \bm y \approxeq \Big(\bm J(\bm x_o)+ \bm w_2 \bm w^\top_1 \Big)\Delta \bm x + \bm w_2 b.
\end{align}
In addition, to match the offset term in $\mathcal R_1$,  we would need to have $\bm w_2 b \approxeq \bm J(\bm x_1)(\bm x_o - \bm x_1) + \bm r$.
In general, the two-layer form in \eqref{one_q} may not fully express or match the two-region linearized model at the two operating points as in \eqref{dy}. Nonetheless, \eqref{one_q} definitely constitutes as a PWL approximation for the underlying $\bm f(\bm x)$ function. In particular, the single ReLU activation in \eqref{one_q} has led to 2 linear regions for the resultant PWL model. 
%\begin{align}
%\begin{split}
%    \Delta \bm y &=\bm J(\bm x_1)\Big(\Delta \bm x + (\bm x_o - \bm x_1)\Big) + \bm r\\
%    %&= \bm J(\bm x_1)\Delta \bm x+ \bm J(\bm x_1)(\bm x_o - \bm x_1) + \bm r\\
%    &=\Big(\bm J(\bm x_o)+ \bm w_2 \bm w^\top_1\Big)\Delta \bm x+ \Big(\bm J(\bm x_1)(\bm x_o - \bm x_1) + \bm r \Big)\\
%    &=\Big(\bm J(\bm x_o)+ \bm w_2 \bm w^\top_1\Big)\Delta \bm x+ \bm w_2 b.
%\end{split}
%\end{align}
%For the scalar input case in \eqref{one_q}, the ReLU activation status would lead to $2^1$ different linear regions.
%If $(\bm w^\top_1 \Delta \bm x+ b)$ is negative, the second summand in \eqref{one_q} would disappear and it becomes  the linear approximation at $\bm x_o$ in $\mathcal R_o$ of \eqref{piecewise_eq}.
%Otherwise, if $(\bm w^\top_1 \Delta \bm x+ b)$ is positive, then \eqref{one_q} represents the linear function approximated at $\bm x_1$ in $\mathcal R_1$, as
\begin{comment}
\begin{align}
\begin{split}
    \Delta \bm y &=\bm J(\bm x_o)\Delta \bm x + \bm w_2 (\bm w^\top_1 \Delta \bm x+ b)\\
    &=\Big(\bm J(\bm x_o) + \bm w_2 \bm w^\top_1 \Big) \Delta \bm x + \bm w_2 b\\
    &=\bm J(\bm x_1) \Delta \bm x+ \Big(\bm J(\bm x_1)(\bm x_o - \bm x_1) + \bm r \Big)\\
    &=\bm J(\bm x_1)\Big(\Delta \bm x + (\bm x_o - \bm x_1)\Big) + \bm r.
\end{split}
\end{align}
\end{comment}
%Thus, we can regard the equation in \eqref{one_q} as PWL with two approximation points $\bm x_o$ and $\bm x_1$.


The simple case of two linear regions can be expanded to encompass more complex PWL model by increasing the number of ReLU activation functions. If the first layer has $q$ ReLU functions with different linear transformations as the input, it is possible  to generate a PWL model with up to $2^q$ linear regions. % we need $q$ number of the activation functions with
This way, the weight parameters form the two matrices $\bm W_1 \in \mathbb R^{(n+\ell)\times q}$ and $\bm W_2 \in \mathbb R^{2\ell\times q}$, as well as the bias vector $\bm b \in \mathbb R^{q}$. The number of linear regions is related to the combination of activation status for all $q$ ReLU functions. 
%Since we have $2^q$ combinations based on the activation status of each ReLU function, we can divide $2^q$ linear regions.
%The rank of weight parameters ${\bm W_1, \bm W_2}$ is increased from 1 in \eqref{one_q} to $q$.
The larger $q$ is, the more expressive the corresponding PWL model becomes, at the price of more model parameters to consider. This makes it difficult to determine the weight parameters using model-based linearization as in \eqref{dy}, motivating us to train these parameters from generated power flow samples. In general, the latter can be designed to reflect the realistic operating points and the statistical variability around them, and thus the resultant PWL model could outperform a model-based approach by pre-selecting the points for linearization. 




Before presenting the training loss, recall that the full generative model in Fig.~\ref{structure} includes the first two layers for obtaining the nonlinear terms and two fixed-weight layers for power variables, as given by %to make those linear regions enough to approximate the power flows and injections, as
\begin{subequations}\label{NN_pf}
\begin{align}
    \bm z^{(1)} &= \sigma(\bm W^\top_1 \Delta \bm x + \bm b),\label{NN_pf1}\\
    \bm z^{(2)} &= \bm f(\bm x_o) + \Big( \bm J(\bm x_o)\Delta \bm x +  \bm W_2 \bm z^{(1)} \Big),\label{NN_pf2}\\
    \bm z^{(3)} &= \bm W^{\gamma} \hat{\bm \gamma} + [\bm W^{\rho};\bm W^{\pi}]\bm z^{(2)},\label{NN_pf3}\\
    \bm z^{(4)} &= \bm W^{\psi} \bm z^{(3)}\label{NN_pf4}
\end{align}
\end{subequations}
where the first two layers generalize the simple case of \eqref{one_q} to $q$ ReLU functions, and the last two layers follow from \eqref{pow}. When we generate random power flow data, the actual values for both $\bm f(\bm x)= [\bm \rho; \bm \pi]$ and $[\bm z^{pf}; \bm z^{inj}]$ can be obtained and using all of them for the loss function could effectively maintain the relations among the corresponding predicted values in \eqref{NN_pf}. Specifically, we can use the Euclidean distance to form the following loss function 
%In order to train the proposed neural network, we use the mean squared error between the actual function values and the approximated values.
%Moreover, we match not only the values of the common terms but also the active and reactive power flows and injections calculated by the approximated common terms.
%Thus, the loss function consists of the difference between the actual and approximated nonlinear function values and the active and reactive power flows and injections, as
\begin{align}
    \mathcal{L}({\bm W_1, \bm W_2, \bm b})= &\| \bm f(\bm x) - \bm z^{(2)}  \|^2_2 \nonumber \\
    &+\lambda \left\| [\bm z^{pf}; \bm z^{inj}] -  [\bm z^{(3)}; \bm z^{(4)}] \right\|^2_2 \label{update}
\end{align}
where $\lambda>0$ denotes a regularization hyperparameter to balance the error terms among the layers. The average loss will be used to aggregate different samples, yielding the total training loss objective to minimize.
After training, the proposed PWL model can fully generate the power flows and injections with linear transformations.
On the other hand, \eqref{NN_pf1} is not a linear transformation due to the ReLU function.
We also face nonlinear constraints when considering the binary line status variables with power flows.
Hence, we will work to reformulate the ReLU function and the line status-related constraints into mixed-integer linear forms.



%Due to the regularization terms, training complexity is increased, which leads to longer training time per epoch.
%However, the loss function can decide the proper gradients to tune the parameters, which leads to fewer epochs needed to converge.
%Thus, the total training time is decreased ultimately.
%Moreover, the neural network can train the underlying relation between power flows and injections.
%This joint training helps set the piecewise linear regions for the common terms related to the power flows and injections, not only for the value itself.




%\begin{align}
%\begin{split}
%    \Delta \bm y &= (\bm J(\bm x_o)+ \bm w_2 \bm w^\top_1)\Delta \bm x+ \bm w_2 b\\
%    &=\bm J(\bm x_o)\Delta \bm x+ \bm w_2 \bm w^\top_1\Delta \bm x+ \bm w_2 b\\
%    &=\bm J(\bm x_o)\Delta \bm x+ \bm w_2 (\bm w^\top_1\Delta \bm x+ b).
%\end{split}
%\end{align}

%We can use the ReLU activation function to separate two linear functions whose approximation points are $\bm x_o$ and $\bm x_1$ as    

%Finally, we can interpret the piecewise linear functions as a neural network with a single trainable hidden layer.

%To expand the linear function in (\ref{linearf}) to piecewise linear functions, we need to set the piecewise linear regions by selecting the approximation points. However, finding the required number of approximation points and their corresponding values is not trivial. The neural networks can search the appropriate approximated points by training the dataset.


%Since the ReLU activation function can disable negative values, we use the ReLU function to choose the desired piece.
%The second and third hidden layers are connected with the fixed-weight matrices $\{\bm W^{\gamma}, \bm W^{\rho}, \bm W^{\pi}, \bm W^{\psi}\}$ that calculate power flows and injections.
%The neural network with $q$ number of activation functions can be represented as 



%\subsection{PWL function interpretation of the neural network}

\subsection{Mixed-integer Linear Formulation for the PWL Model}

The proposed PWL models allow for formulating the nonlinear power flow equations into mixed-integer linear forms and thus enable efficient solutions for grid optimization problems  involving topology variables. We will present the formulation steps to attain a mixed-integer linear program (MILP) for the PWL models, and also for dealing with the binary line status relation in \eqref{powerinj}.


To adopt the PWL model in \eqref{NN_pf} into an MILP, the main issue lies in the ReLU function of \eqref{NN_pf1}, as all other transformations are just linear ones.  
To tackle the ReLU function, we will use a relaxation technique based on the big-M tightening method~\cite{griva2009linear}.
%We collaborate with a binary variable $\beta_k$ and big-M bounds $(\underline M_k,\bar M_k)$ to define $z^{(1)}_k$, where $k$ denotes the index of the activation functions.
For the $k$-th entry $z^{(1)}_k$ in \eqref{NN_pf1}, we will approximate it by introducing a binary variable $\beta_k$, and its upper/lower bounds $\{\bar M_k,\underline M_k\}$. 
The two bounds can be determined through an off-line optimization procedure~\cite{grimstad2019relu}. After determining these bounds and denoting the input in \eqref{NN_pf1} by $\hat {\bm z}^{(1)} = \bm W^\top_1 \Delta \bm x + \bm b$, the big-M method asserts that each $z^{(1)}_k$ can be reformulated by using four linear inequality constraints, as given by
\begin{subequations}\label{MILP}
\begin{align}
    0 &\leq  z^{(1)}_k \leq \bar M_k  \beta_k, \label{beta0}\\
    \hat {z}^{(1)}_k &\leq z^{(1)}_k \leq \hat {z}^{(1)}_k - \underline M_k (1- \beta_k), \label{beta1}
\end{align}
\end{subequations}
The binary variable $\beta_k$ critically relates to the ReLU activation status based on the input $ \hat{z}^{(1)}_k$.
If the input $\hat {z}^{(1)}_k > 0$, then  the constraints in  \eqref{beta1} enforce $\beta_k$  to be one such that $z^{(1)}_k=\hat {z}^{(1)}_k$ holds exactly.
Otherwise, if $\hat {z}^{(1)}_k \leq 0$, the constraints in \eqref{beta0} enforce $\beta_k$ to be zero to yield $z^{(1)}_k=0$.
This way, the output $z^{(1)}_k$ from \eqref{MILP} exactly attains the ReLU-based output in \eqref{NN_pf1}. Thus, with accurate upper/lower bounds,  the big-M method allows for an equivalent reformulation  of \eqref{NN_pf} into an MILP form. 


%Let $\{{\bar{ P}}_{ij}, {\bar{ Q}}_{ij}\}$ denote the upper bounds and $\{{\underline{ P}}_{ij}, {\underline{ Q}}_{ij}\}$ denote the lower bounds of the active and reactive power flows.
%Similarly, let $\{{\bar{ P}}_{i}, {\bar{ Q}}_{i}\}$ denote the upper bounds and $\{{\underline{ P}}_{i}, {\underline{ Q}}_{i}\}$ denote the lower bounds of the active and reactive power injections.
%We also need $\{{\bar{ V}}_{i}, {\underline{ V}}_{i}\}$ and $\{{\bar{ \theta}}_{i}, {\underline{ \theta}}_{i}\}$ for the upper and lower bounds of the voltage magnitude and angle.


%Determining tight yet accurate upper/lower bounds is important for the big-M based MILP solution accuracy and runtime \cite{grimstad2019relu}.
%To this end, we can set up an optimization problem to select these values according to the power flow outputs. 
%\textcolor{teal}{Specifically, suppose we are able to obtain the upper/lower limits of the active/reactive line flows, as denoted by $\{{\bar{\bm z}}^{pf},{\underline{\bm z}}^{pf}\}$.} In addition, suppose we know the bounds for active/reactive power injections, voltage magnitudes, and angle differences, all denoted in the similar way.
%This way, we can determine the bounds for  $z^{(1)}_k$  by solving
%\begin{subequations}\label{bigM}
%\begin{align}
%    \underline M_k(\bar M_k) &= \min (\max)  \quad  \hat {z}^{(1)}_k\\
%    \textrm{s.t.}  \qquad &\eqref{NN_pf2}-\eqref{NN_pf4},\\
%    {\underline{\bm z}}^{pf} &\leq  \bm z^{(3)} \leq {\bar{\bm z}}^{pf},\forall (i,j) \in \mathcal{L}\\
%    {\underline{\bm z}}^{inj} &\leq  \bm z^{(4)} \leq {\bar{\bm z}}^{inj}, \forall i \in \mathcal{N}\\
%    \underline{V}_i &\leq  V_i \leq \bar{V}_i, \forall i \in \mathcal{N}\\
%    \underline{\theta}_{ij} &\leq  \theta_{ij} \leq \bar{\theta}_{ij},\forall (i,j) \in \mathcal{L}.
%\end{align}
%\end{subequations}
%\textcolor{teal}{This linear optimization problem tries to find the minimum and maximum value of $\hat {z}^{(1)}_k$ while satisfying the power system operation constraints.
%This way, the minimum and maximum values regard the tightest big-M bounds since $\hat{\bm z}^{(1)}_k$ always lies between them.
%The big-M bounds regard the constants once the optimal values are found by iteratively solving the problem $2q$ times by changing the index and objective.}




%\subsection{Switching status of the transmission line}
Similarly, we formulate the line status-related constraints into an MILP form.
%The last nonlinearity arises from calculating the nodal power balance while considering the line status binary variable of corresponding power flows.
We face the multiplication of continuous variables $\{P_{ij}$, $ Q_{ij}\}$ and binary variables $ \epsilon_{ij}$ that are denoted as $\hat{ P}_{ij} := \epsilon_{ij}  P_{ij}$ and $\hat{ Q}_{ij} := \epsilon_{ij}  Q_{ij}$.
To tackle this multiplication term, we will use the McCormick relaxation technique~\cite{mccormick1976computability} derived from the big-M tightening method.
After attaining the upper/lower bounds of active and reactive power flows $\{{\bar{ P}}_{ij},{\underline{ P}}_{ij}\}$ and $\{{\bar{ Q}}_{ij},{\underline{ Q}}_{ij}\}$, each $\{\hat{ P}_{ij},\hat{ Q}_{ij}\}$ can be reformulated by using four linear inequality constraints, as given by
\begin{subequations}\label{sw}
\begin{align}
    {\underline{ P}}_{ij}  \epsilon_{ij} &\leq \hat{ P}_{ij} \leq {\bar{ P}}_{ij}  \epsilon_{ij}, \label{sw0_P}\\
    {\underline{ Q}}_{ij}  \epsilon_{ij} &\leq \hat{ Q}_{ij} \leq {\bar{ Q}}_{ij}  \epsilon_{ij}, \label{sw0_Q}\\
     P_{ij} + {\bar{ P}}_{ij} ( \epsilon_{ij}-1) &\leq \hat{ P}_{ij} \leq  P_{ij} + {\underline{ P}}_{ij} ( \epsilon_{ij}-1), \label{sw1_P}\\
     Q_{ij} + {\bar{ Q}}_{ij} ( \epsilon_{ij}-1) &\leq \hat{ Q}_{ij} \leq  Q_{ij} + {\underline{ Q}}_{ij} ( \epsilon_{ij}-1). \label{sw1_Q}
\end{align}
\end{subequations}
The outputs $\{\hat{ P}_{ij},\hat{ Q}_{ij}\}$ critically relate to $\epsilon_{ij}$.
If $ \epsilon_{ij}$ is equal to zero, then  the constraints in \eqref{sw0_P} and \eqref{sw0_Q} enforce $\hat{ P}_{ij}$ and $\hat{ Q}_{ij}$ to be zero.
Otherwise, if $ \epsilon_{ij}$ is equal to one, then  the constraints in \eqref{sw1_P} and \eqref{sw1_Q} enforce $\hat{ P}_{ij}$ and $\hat{ Q}_{ij}$ to be $ P_{ij}$ and $ Q_{ij}$, respectively.
This way, the output $\hat{ P}_{ij}$ and $\hat{ Q}_{ij}$ from \eqref{sw} exactly attain the power flows based on the binary line status.
Thus, the McCormick relaxation allows for an equivalent reformulation of the multiplication terms of power flows and line status variables into an MILP form.
%This completes the MILP formulation of the PWL-based approximation model and line status-related constraints.






\section{Numerical Studies} \label{sec:sim}


We have implemented the proposed generative modeling approach on the IEEE 14-bus and 118-bus test cases~\cite{IEEE_case_ref}, to compare its performance in power flow modeling and grid topology optimization.
The NN training has been performed in PyTorch with Adam optimizer on a regular laptop with Intel\textsuperscript{\textregistered} CPU @ 2.70 GHz, 32 GB RAM, and NVIDIA\textsuperscript{\textregistered} RTX 3070 Ti GPU @ 8GB VRAM.
We have formulated the OTS problem through Pyomo~\cite{hart2011pyomo} and used the Groubi optimization solver~\cite{gurobi} for the resultant MILPs.




%\subsection{PWL Model Training}
To train the proposed NN-based PWL models in Fig.~\ref{structure}, 
we generate 10,000 samples from the actual power flow model, with the outputs of common nonlinear terms $\{\bm \gamma, \bm \rho, \bm \pi\}$, as well as line flows and nodal injections.
For each sample, we generate uniformly distributed voltage magnitudes within the range of [0.94,~1.06] p.u., 
and similarly for the angle, which randomly varies within $[-\pi/6, \pi/6]$ radians around the initial operating point.
For the reference bus, namely \texttt{Bus 1} in the 14-bus or \texttt{Bus 69} in the 118-bus system, we fix its voltage magnitude and angle at default values.
For the first two trainable layers in Fig.~\ref{structure}, we use $q=25$ and $q=75$ ReLU activation functions, respectively for the two systems. 
%Accordingly, the weight parameters for the 14-bus system with 20 lines are $\bm W_2 \in \mathbb R^{40\times 25}$, $\bm W_1 \in \mathbb R^{34\times 25}$, and $\bm b \in \mathbb R^{25}$, while  for the 118-bus one with 186 lines are $\bm W_2 \in \mathbb R^{372\times 75}$, $\bm W_1 \in \mathbb R^{304\times 75}$, and $\bm b \in \mathbb R^{75}$.
The parameters are trained through the backpropagation using the loss function in \eqref{update} with $20\times10^3$ epochs and a learning rate of $2.5\times10^{-3}$.
For the training of the PWL model, we separate 90\% of the data set as training and 10\% as testing.
The PWL model only trains the training dataset and validates the loss through the testing dataset.
%\textcolor{blue}{(plz discuss how training/testing samples are separated. did you use a diff subset to show the results in the next part??) }



%\section{Simulation} \label{sec:sim}
%In this section, we validate the approximation performance of the piecewise linear functions using the neural network. Furthermore, we obtain the optimal transmission line for the flexible topology.

\begin{comment}
% Figure environment removed
\end{comment}



% Figure environment removed


% Figure environment removed





\subsection{AC Power Flow Approximation}
We first validate the AC power flow modeling performance of our PWL-based approximation. We compare the proposed generative modeling approach using the common nonlinear term prediction step (indicated by Gen) with the existing work \cite{kody2022modeling} that directly predicts the  power flow variables (indicated by Direct). The latter directly uses a two-layer NN of ReLU activation to output the line power flow, with the structure given by %that matches the difference of the AC power flows based on the initial operating point, as
%\begin{subequations}
\begin{align}\label{dNN_pf}
    %\bm z^{(1)} &= \sigma(\bm W^\top_1 \Delta \bm x + \bm b),\\
    %[P_{ij}; Q_{ij};P_{ji}; Q_{ji}] 
    \bm z ^{pf}&\leftarrow \bm J(\bm x_o)\Delta \bm x +  \bm W_2 \bm z^{(1)}.
\end{align}
%\end{subequations}
Note that we use the same number of activation functions for both types of models.

We compare the approximation error between the predicted and actual line power flow values as  normalized by the line capacity.
Figs.~\ref{ACPF14} and~\ref{ACPF118} show the box plots of the normalized prediction error percentages of both active and reactive line flows, respectively for the  14- and 118-bus systems. Note that both the average error and the maximum error, out of all transmission lines in each system are included for comparisons. 
Each box plot shows the median values as midlines, the first and third quartiles as boxes, maximum values as horizon bars, and some outliers.  %consists of the entire power flow approximation errors of a single scenario.
Clearly, the proposed generative model is of better accuracy in predicting the line flows than the direct method, especially for the reactive power parts.
%(What is the top bar of all box plots (right below the dots) mean? this seems to be a notable improvement in all cases.)
Notably, the proposed method has shown significant improvements in terms of reducing the maximum values of errors in all cases.
These results have verified the benefits of incorporating the underlying coupling between active and reactive power flows considered by our proposed NN design. 
%Especially, reactive power flow approximation results show a higher error than active power flows.
%Direct\_PL generates the active and reactive power flows through the different neurons in the hidden layer.
%Thus, the active and reactive power flows have independent piecewise linear regions due to the activation function on each neuron.
%It leads to the decoupled approximation result for active and reactive power.
%However, the active and reactive power flows have underlying couplings.
%Since Common\_PL generates the common terms, the active and reactive power flows can be decided by the same piecewise linear regions with similar error ranges.

    
Furthermore, we also compare the error performance in predicting the active and reactive power injections, which can be formed directly from the line flows using \eqref{powb}.
%We calculate the root mean square error (RMSE) of power injection across all buses in each scenario.
Without any normalization basis, Fig.~\ref{ACPFinj} instead shows the box plots for the root mean square error (RMSE) in predicting the injected power vectors in both test systems. 
Similar results have been observed for predicting the injections, with even more noticeable improvements in both active and reactive power values. This is because our proposed NN model in Fig.~\ref{structure} has directly accounted for the power flow coupling, and thus its joint training process would achieve high consistency with nodal power balance. Thanks to the generative structure of the underlying NN design, our proposed PWL models can improve the accuracy and consistency in the resultant power flow approximation. 

%Since the approximation errors are propagated, the active and reactive power injections calculated by Direct\_PL have a large error range.
%However, the active and reactive power injections between the actual and the Common\_PL generated values are matched in a small error range.
%Therefore, the proposed generative model well represents the AC power flows and power injections.


% Figure environment removed


\subsection{OTS Applications}
We adopt the proposed PWL models in solving the OTS problem using the 118-bus system.
The objective function and operational constraints are set up similarly  to the typical optimal power flow (OPF) problem. Additionally, OTS allows for line switching under the constraint of a total switching budget of $\alpha$ lines, given by
\[\textstyle \sum_{(i,j) \in \mathcal{L} } ~\epsilon_{ij} \geq \ell-\alpha.\]
The number $\alpha$ is typically no greater than 5-10. 
Hence, the computation complexity of OTS is much higher than that of OPF, due to the integer line status variables. 
We introduce the proposed PWL models into the AC-OTS formulation by replacing the power flow constraints by \eqref{MILP}, as well as the line status constraints by \eqref{sw}.
%Through these substitutions, we can reformulate 
This way, the resultant MILP problem can be efficiently solved  with solvers like Gurobi.


%We validate the solutions obtained by the piecewise linear approximated OTS (PLOTS) from the perspective of 
We test the performance of the proposed PWL model-based OTS solutions.
We compare it with the OTS solutions using the DC- and AC- power flow models, both provided by the open-source platform~\cite{8442948}. To compare across different OTS methods, we re-run the AC-OPF problem after fixing the topology with their line-switching decision outputs, using the MATPOWER~\cite{zimmerman2010matpower} solver. This way, we can compare the metrics in terms of the objective costs (for optimality), as well as the percentage rates of infeasibility and constraint violations (for feasibility), using the corresponding AC-OPF outputs. For the two feasibility measures, the infeasible solution rates measure the percentage of infeasible solutions over all solutions, while the constraint violation rates are based on  the percentage of over-limit voltage magnitude and angle over the infeasible solutions.
%To check the optimality of the solutions, we compare the line-switching decisions obtained from the different OTS formulations.
%We run the ACOPF after fixing the system topology derived by the OTS through , regarded as the ground truth solver.
Table~\ref{PLOTS} lists these metrics and also the computation time for each of the three OTS methods with a switching budget $\alpha$ equal to 1 or 3. Note that the computation time corresponds to solving the OTS optimization problem, not the follow-up AC-OPF one. 
A total of 1,000 power flow scenarios by having nodal demand  uniformly distributed within [50\%, 200\%] of the initial demands~\cite{IEEE_case_ref} have been used to compute the average of all these four metrics. For the objective cost, the AC-OTS method has been used as a baseline (normalized to be 100\%), and thus the other two OTS methods using approximate models attain higher percentage values. Nonetheless, the proposed PWL model only slightly increases the objective cost by less than 2\%, while attaining exactly the same infeasiblity metrics as the AC-based OTS solutions. Notably, our model achieves a highly competitive performance and also great efficiency, as its computation time is almost a tenth of the AC-OTS one. In particular, the PWL model has allowed for a very low computation complexity in the order of DC-OTS one. But the latter leads to significantly worse feasibility performance, with an almost order of magnitude higher of infeasiblity rates than PWL-based OTS.  Thanks to its high modeling accuracy, our proposed PWL model can greatly simplify the computation for grid topology optimization tasks by using the MILP reformulation trick, while approaching the ideal optimality/performance performance.  


\begin{table}[t!]
\caption{Comparison of the optimality and feasibility of the solutions of the AC, PWL, and DCOTS.}
\begin{center}
\begin{tabular}{c|c|c|c|c}
\Xhline{3\arrayrulewidth}
  &  {\makecell[c]{Switching\\Budget}}  & AC & PWL & DC \\ \hline
\multirow{2}{*}{\makecell[c]{Objective\\ Cost (\%)}} & $\alpha=1$ &   100\%       &    101.90\%    &   102.67\% \\ \cline{2-5}
 &  $\alpha=3$ &  100\%       &    101.94\%    &   104.21\% \\ \hline
\multirow{2}{*}{\makecell[c]{Infeasible\\Solution (\%)}} & $\alpha=1$ & 0.20\% & 0.20\% & 2.10\% \\ \cline{2-5}
 & $\alpha=3$ & 0.20\% & 0.20\% & 2.80\% \\ \hline
\multirow{2}{*}{\makecell[c]{Constraint\\Violation (\%)}} & $\alpha=1$ & 0.33\%     &  0.33\%   & 1.32\%   \\ \cline{2-5}
 & $\alpha=3$ & 0.33\%     &  0.33\%   & 2.31\%   \\ \hline
\multirow{2}{*}{\makecell[c]{Computation\\Time (s)}}  & $\alpha=1$ &    52.13 s    &  5.37 s  & 3.41 s \\ \cline{2-5}
  & $\alpha=3$ &    57.67 s    &  6.45 s  & 3.89 s \\
\Xhline{3\arrayrulewidth}
\end{tabular}\label{PLOTS}
\end{center}
\end{table}


%While the objective value of PLOTS is slightly worse than ACOTS, PLOTS produces lower generation costs than DCOTS.
%Furthermore, ACOTS faces a more computational burden than PLOTS.
%Thus, we ensure that PLOTS can obtain optimal solutions.
 
%We also compare the feasibility of the solution from the perspective of the ACOPF.
%Since the DCOTS and PLOTS approximate the power flows, we need to check whether the systems have feasible solutions for the optimal power flow problem utilizing the nonlinear AC power flows.
%Table~\ref{PLOTS} also compares the percentage of infeasible solutions and the corresponding constraint violations.
%The PLOTS can obtain more feasible solutions than the DCOTS since it accurately generates nonlinear AC power flows.
%We rarely confront infeasible power system operation schedules and constraint violations because the PLOTS rarely produce voltage magnitude and angle violations.
%The PLOTS faces infeasible solutions only if the ACOTS also has infeasible solutions.
%All these results have demonstrated that the power system optimization problem using the piecewise linear functions generates optimal and feasible solutions.



%\section{Conclusion} \label{sec:con}
%This paper presents the power flow generative model to approximate the nonlinear active and reactive power flows.Through the neural network, we explore the underlying physics of the system topology and coupling between the active and reactive power flows. We linearize the common terms that active and reactive power flows commonly have through the Jacobian matrix.We utilize the neural network to update the Jacobian matrix for the adjacent piecewise linear regions and use big-M bounds to  the ReLU activation functions in the hidden layer.Moreover, we adopt the McCormick  to consider the transmission line status.We validate the proposed generative model by comparing the outputs with the actual active and reactive power flow.Finally, the optimal transmission switching problem adopting the piecewise linear functions has been validated with the ground truth solver, corroborating the optimality and feasibility of the proposed scheme.

To sum up, we have designed a NN-based PWL approximation model for AC power flow with a good balance between model complexity and accuracy. Through its generative design, the proposed PWL models not only account for the underlying power flow coupling, but also allow for highly competitive solutions for  topology-aware grid optimization problems. 
Our future research directions include improving the scalability of our proposed PWL models in large-scale power systems, as well as considering more generalized topology-aware grid optimization tasks like restoration and adaptive islanding.


\ifCLASSOPTIONcaptionsoff
\newpage
\fi
	
\bibliographystyle{IEEEtran}
\bibliography{Ref.bib}


    
\end{document}
