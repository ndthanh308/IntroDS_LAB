\section{Summary and Recommendations}\label{recommendation}
In this section we identify a series of best practices and recommendations \textcolor{red}{for} designing a well-incentivized FL experimental setting base on our findings and insights in Section~\ref{study}.

\textbf{Level of statistical heterogeneity and local epochs.} The level of statistical heterogeneity and \textcolor{red}{the} number of local epochs determine the incentives for global and personalized FL. Generally, when the level of statistical heterogeneity is low, global FL is more incentivized than personalized FL, and vice versa (Figures~\ref{fig2:a} and~\ref{fig3:a}). Additionally, increasing the number of local epochs can make personalized FL more incentivized (Figures~\ref{app:fig-incentive-dir} and~\ref{app:fig-incentive-skew}). We have identified the well-incentivized settings for each FL approach in Table~\ref{tab:incentivized-settings}.

\begin{table}[t]
\caption{Well-incentivized settings for personalized and global FL approaches.}
\label{tab:incentivized-settings}
\centering
\resizebox{.6\linewidth}{!}{
\begin{tabular}{l|cc|cc|}
            \toprule
\multirow{2}{*}{Local Epoch} & \multicolumn{2}{c|}{Non-IID Label Dir} & \multicolumn{2}{c|}{Non-IID Label Skew}\\
          \cmidrule{2-3} \cmidrule{4-5}
 & pFL & gFL & pFL & gFL \\
            \midrule
$E=1$  & $\alpha < 0.3$ & $\alpha > 0.3$ & $p < 0.8$ & $p > 0.8$\\
            \midrule   
$E=5$  & $\alpha < 0.3$ & $\alpha > 0.3$ & $p < 0.8$ & $p > 0.8$\\
            \midrule
$E=10$  & $\alpha < 0.5$ & $\alpha > 0.5$ & $p < 0.8$ & $p > 0.8$\\
            \midrule
$E=20$  & $\alpha < 0.5$ & $\alpha > 0.5$ & $p < 0.9$ & $p > 0.9$\\
            \midrule
\end{tabular}
}
\end{table}

\textbf{Type of statistical heterogeneity.} We observe that the nature of label skew type of statistical heterogeneity favors personalized FL over a wider range of heterogeneity levels compared to label Dir (Figures~\ref{fig2:a},~\ref{fig3:a},~\ref{app:fig-incentive-dir} and~\ref{app:fig-incentive-skew}). Additionally, we observe that the impact of local epochs on performance is more pronounced for label Dir type of statistical heterogeneity compared to label skew (Figures~\ref{fig2:b},~\ref{fig2:c},~\ref{fig3:b} and~\ref{fig3:c}). To provide a comprehensive perspective on an algorithm's success in the presence of statistical heterogeneity, we recommend researchers conduct experiments with both types of statistical heterogeneity.

\textbf{Sample rate.} This variable plays a crucial role in evaluating an algorithm's performance under statistical heterogeneity. 
%Choosing a high sample rate ($C > 0.4$) can mask the effect of statistical heterogeneity and misrepresent an algorithm's ability to handle it, while a low sample rate ($C < 0.1$) can hinder convergence due to a lack of enough models to average (Figure~\ref{fig5}). 
\textcolor{red}{Choosing a high sample rate ($C > 0.4$) can mask the effect of statistical heterogeneity, thus misrepresenting an algorithm's true ability to handle it. Additionally, it can lead to inaccurate representations of an algorithm's capability to handle the stochasticity resulting from random device selection and the inherent errors caused by model averaging approximation. On the other hand, a low sample rate ($C < 0.1$) may hinder convergence due to insufficient models for averaging and high errors caused by model averaging approximation (Figure~\ref{fig5}).} To avoid these pitfalls, we recommend researchers use a sample rate of $0.1 \leq C \leq 0.4$ for their experiments.
% choosing a high sample rate ($C > 0.4$) can mask the effect of statistical heterogeneity and misrepresent an algorithm's true ability to handle it on one hand and lead to inaccurate representation of an algorithm's ability to handle the stochasticity resulting from random device selection and the inherent errors caused by model averaging approximation on the other hand. while a low sample rate ($C < 0.1$) can hinder convergence due to a lack of enough models to average and high errors caused by model averaging approximation. 

\textbf{Summary.} To ensure consistency, comparability, and meaningful results in FL experiments, we have compiled a set of recommended settings in Table~\ref{tab:recommended-settings}. We encourage researchers to adopt these settings, as well as the evaluation metrics outlined in Section~\ref{def:gfl}, for their experiments. This will facilitate more consistent and fair comparisons with SOTA algorithms, and eliminate concerns about evaluation failures and the impact of various experimental settings.

\begin{table}[ht]
\caption{Recommended settings for pFL and gFL approaches.}
\label{tab:recommended-settings}
\centering
\resizebox{1.0\linewidth}{!}{
\begin{tabular}{c|ccccc}
            \toprule
Approach & Type of Heterogeneity & Level of Heterogeneity & Local Epoch & Number of Clients & Sample Rate\\
            \midrule
\multirow{2}{*}{pFL} &  Label Dir & $\alpha \in [0.01, 0.3]$ & $\{1, 5, 10, 20\}$ & $\{20, 100\}$ & $\{0.1, 0.2, 0.3, 0.4\}$ \\
            \cmidrule{2-6}
  & Label Skew & $p \in \{2, 3, 4, 5\}$ & $\{1, 5, 10, 20\}$ & $\{20, 100, 500\}$ & $\{0.1, 0.2, 0.3, 0.4\}$ \\
            \midrule
\multirow{2}{*}{gFL} &  Label Dir & $\alpha \in (0.3, 1]$ & $\{1, 5, 10, 20\}$ & $\{20, 100\}$ & $\{0.1, 0.2, 0.3, 0.4\}$ \\
            \cmidrule{2-6}
  & Label Skew & $p \in \{8, 9\}$ & $\{1, 5, 10, 20\}$ & $\{20, 100, 500\}$ & $\{0.1, 0.2, 0.3, 0.4\}$ \\
  \bottomrule
\end{tabular}
}
\vspace{-0.5cm}
\end{table}

\begin{comment}
\begin{table}[ht]
\caption{Recommended settings for pFL and gFL approaches.}
\label{tab:recommended-settings}
\centering
\resizebox{0.9\linewidth}{!}{
\begin{tabular}{c|ccccc}
            \toprule
Approach & Type of Heterogeneity & Level of Heterogeneity & Local Epoch & Number of Clients & Sample Rate\\
            \midrule
\multirow{2}{*}{pFL} &  Label Dir & $\alpha \in [0.01, 0.3]$ & $E \in \{1, 5, 10, 20\}$ & $N \in \{20, 100\}$ & $S \in \{0.1, 0.2, 0.3, 0.4\}$ \\
            \cmidrule{2-6}
  & Label Skew & $p \in \{2, 3, 4, 5\}$ & $E \in \{1, 5, 10, 20\}$ & $N \in \{20, 100, 500\}$ & $S \in \{0.1, 0.2, 0.3, 0.4\}$ \\
            \midrule
\multirow{2}{*}{gFL} &  Label Dir & $\alpha \in [0.3, 1]$ & $E \in \{1, 5, 10, 20\}$ & $N \in \{20, 100\}$ & $S \in \{0.1, 0.2, 0.3, 0.4\}$ \\
            \cmidrule{2-6}
  & Label Skew & $p \in \{8, 9\}$ & $E \in \{1, 5, 10, 20\}$ & $N \in \{20, 100, 500\}$ & $S \in \{0.1, 0.2, 0.3, 0.4\}$ \\
  \bottomrule
\end{tabular}
}
\end{table}
\end{comment}