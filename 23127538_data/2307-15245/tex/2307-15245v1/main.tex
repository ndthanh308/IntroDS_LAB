\documentclass[journal]{IEEEtai}

%\usepackage[colorlinks,urlcolor=blue,linkcolor=blue,citecolor=blue]{hyperref}

\usepackage{inputenc}
%\usepackage[T1]{fontenc}
\usepackage{scalerel}
 \usepackage{xcolor}

\let\labelindent\relax

\usepackage[margin=1in]{geometry}
% \usepackage[math]{kurier}
% \usepackage[sc]{mathpazo}                   
% \renewcommand{\sfdefault}{kurier}

\usepackage{url}            
\usepackage{cite}
\usepackage{hyperref}

\definecolor{darkpastelgreen}{rgb}{0.01, 0.75, 0.24}
	\definecolor{cadmiumgreen}{rgb}{0.0, 0.42, 0.24}
\definecolor{armygreen}{rgb}{0.29, 0.33, 0.13}
\hypersetup{colorlinks,linkcolor={blue},citecolor={darkpastelgreen},urlcolor={red}}  


%\usepackage[draft]{hyperref}
\usepackage{booktabs}       % professional-quality tables
\usepackage{amsfonts}       % blackboard math symbols
\usepackage{nicefrac}       % compact symbols for 1/2, etc.
\usepackage{microtype}      % microtypography
\usepackage{url}
\usepackage{array}

\usepackage{amsmath}
\usepackage{color}
\usepackage{amssymb}
\usepackage{amsthm}
\usepackage{pifont}% http://ctan.org/pkg/pifont
\usepackage[linesnumbered,ruled,vlined]{algorithm2e}

\usepackage{paralist}
\usepackage{dsfont}

\usepackage[square,sort,comma,numbers]{natbib}
\bibliographystyle{plainnat}

\usepackage{csquotes}
\usepackage{comment}
\usepackage{mathtools}

\usepackage{microtype}
\usepackage{graphicx}
\usepackage{booktabs} % for professional tables
\usepackage{dirtytalk}
\usepackage{subfiles}

\usepackage{caption}
\usepackage{subcaption}

%\usepackage{breqn}
%\usepackage[square,numbers]{natbib}
\usepackage{xspace}
\usepackage{forloop}
\usepackage{multirow}
%\usepackage{booktabs}
\usepackage{algorithm,algorithmic,refcount}
\usepackage{comment}

% \usepackage[pdftex,dvipsnames,table]{xcolor}
% \usepackage[colorinlistoftodos,prependcaption,textsize=small]{todonotes}
\usepackage{graphicx}
\usepackage[shortlabels]{enumitem}
\usepackage{bbm}
\usepackage{float}
\usepackage[most]{tcolorbox}

%\usepackage{minipage}

\usepackage{footnote}
\usepackage{etoolbox}
\BeforeBeginEnvironment{tcolorbox}{\savenotes}
\AfterEndEnvironment{tcolorbox}{\spewnotes}

% \DeclareMathOperator*{\argmax}{arg\,max}
% \DeclareMathOperator*{\argmin}{arg\,min}

% \makeatletter
% \def\@makefnmark{%
%   \leavevmode
%   \raise.9ex\hbox{\fontsize\sf@size\z@\normalfont\tiny\@thefnmark}}
% \makeatother

\newcommand\calF{\mathcal{F}}
\newcommand\calG{\mathcal{G}}
\newcommand\calM{\mathcal{M}}
\newcommand\calV{\mathcal{V}}
\newcommand\calU{\mathcal{U}}
\newcommand\calW{\mathcal{W}}
\newcommand\calP{\mathcal{P}}
\newcommand\calD{\mathbb{D}}
%%%%%%%%%%%%%%%%%
%% macros introduced by Luke 
\newcommand\mydef[1]{{\bf\em #1}}
%%%%%%%%%%%%%%%%%

\newcommand{\numviparams}{{| \lambda |}}
\newcommand{\scoreaccvars}[1]{s_1^{#1}, \ldots, s_{\numviparams}^{#1}}
\newcommand{\scoreaccvar}[2]{s_{#1}^{#2}}
\newcommand{\isdeterm}[1]{\text{Deterministic}({#1})}


\newcommand{\expect}[1]{\mathbb{E}\left[{#1}\right]}
\newcommand{\var}[1]{\mathbb{V}\left[ {#1} \right]}
\newcommand{\expectdist}[2]{\mathbb{E}_{#1}\left[ {#2} \right]}
\newcommand{\vardist}[2]{\mathbb{V}_{#1}\left[ {#2} \right]}
\newcommand{\cov}[2]{\mathbb{C}\text{ov}[{#1}][{#2}]}
\newcommand{\covv}[1]{\mathbb{C}\text{ov}[{#1}]}
\newcommand{\corr}[1]{\mathbb{C}\text{orr}[{#1}]}

\newcommand{\fix}[1]{\mathit{fix}\left({#1}\right)}
\newcommand{\sbr}[1]{\left\llbracket {#1} \right\rrbracket}
\newcommand{\ctxtype}[3]{{#1} \cong_\text{ctx} {#2} : {#3}}
\newcommand{\bigstep}[3]{{#1} \Downarrow_{#2} {#3}}


% PCF types
\newcommand{\bool}{\mathit{bool}}
\newcommand{\nat}{\mathit{nat}}

\newcommand{\ctx}[1]{\mathcal{C}\left[ {#1}\right] }
\newcommand{\pcft}[1]{\text{PCF}_{#1}}

\newcommand{\nfl}{\mathbb{N}_\bot}
\newcommand{\bfl}{\mathbb{B}_\bot}

% PCF constructs
\newcommand{\succc}[1]{\mathbf{succ}({#1})}
\newcommand{\succcn}[2]{\mathbf{succ}^{#1}({#2})}
\newcommand{\zero}{\mathbf{0}}
\newcommand{\zerotest}[1]{\mathbf{zero}\left({#1}\right)}
\newcommand{\pred}[1]{\mathbf{pred}\left( {#1} \right)}
\newcommand{\predn}[2]{\mathbf{pred}^{#1}\left( {#2} \right)}
\def\solvable{\#}

\newcommand{\true}{\mathbf{true}}
\newcommand{\false}{\mathbf{false}}
\newcommand{\pcffix}[1]{\mathbf{fix}\left({#1}\right)}
\newcommand{\pcffn}[3]{\mathbf{fn}~{#1}:{#2}\mathpunct{.}{#3}}
\newcommand{\pairtype}[2]{{#1} * {#2}}
\newcommand{\pairexp}[2]{\mathbf{pair}({#1}, {#2})}
\newcommand{\leftexp}[1]{\mathbf{left}({#1})}
\newcommand{\rightexp}[1]{\mathbf{right}({#1})}

\newcommand{\RationalPos}{\mathbb{Q}^{+}}

\newcommand{\meas}[1]{\mathbb{M}\left( {#1} \right) }
\newcommand{\integ}[1]{\sbr{#1}_I}

\newcommand{\notbigstep}[2]{{#1}~\cancel{\Downarrow}_{#2}}
\newcommand{\subtrace}[3]{{#1}^{{#2} \ldots {#3}}}
\newcommand{\supp}[1]{\textsf{supp}\left({#1}\right)}
\newcommand{\dom}[1]{\textsf{Dom}\left({#1}\right)}
\newcommand{\suppk}[2]{\textsf{Supp}^{#1}\left({#2}\right)}
\newcommand{\tracespace}{\bigcup_{n \in \mathbb{N}}[0, 1]^n}
\newcommand{\generictracespace}{\mathbb{T}}
\newcommand{\nnreals}{\mathbb{R}_{\geq 0}}
\newcommand{\posreals}{\mathbb{R}_{> 0}}
\newcommand{\reals}{\mathbb{R}}

\newcommand{\unrollkM}[2]{\textsf{unroll}_{#1}\left({#2}\right)}
\newcommand{\nphmcint}[5]{\Psi_\textsf{NP}\left({#1}, {#2}, {#3}, {#4}, {#5}\right)}

%SPCF constructs
\newcommand{\spcfvalues}{\Lambda^0_v}

\newcommand{\prevalueM}[1]{\textsf{value}^{-1}_{#1}(\spcfvalues{})}
\newcommand{\num}[1]{\underline{#1}}

% \theoremstyle{definition}
% \newtheorem{thm}{Theorem}
% \newtheorem{lem}{Lemma}
% \newtheorem{defn}{Definition}
% \newtheorem{conj}{Conjecture}
% \newtheorem{prop}{Proposition}

%\theoremstyle{definition}
%\newtheorem{defn}{Definition}[section]
%\newtheorem{example}[defn]{Example}
%
%
%\theoremstyle{plain}
%\newtheorem{thm}{Theorem}[section]
%\newtheorem{lem}[thm]{Lemma}
%\newtheorem{cor}[thm]{Corollary}
%\newtheorem{conj}[thm]{Conjecture}
%\newtheorem{prop}[thm]{Proposition}
%\newtheorem{remark}[thm]{Remark}

%% Proofs
%\let\oldproof\proof
%\renewcommand{\proof}{\color{blue}\oldproof}


\definecolor{codegreen}{rgb}{0,0.6,0}
\definecolor{codegray}{rgb}{0.5,0.5,0.5}
\definecolor{codepurple}{rgb}{0.58,0,0.82}
\definecolor{backcolour}{rgb}{0.95,0.95,0.92}

\lstdefinestyle{myStyle}{
    belowcaptionskip=1\baselineskip,
    breaklines=true,
    frame=none,
    basicstyle=\footnotesize\ttfamily,
    keywordstyle=\bfseries\color{green!40!black},
    commentstyle=\itshape\color{purple!40!black},
    identifierstyle=\color{blue},
    backgroundcolor=\color{gray!10!white},
    %backgroundcolor=\color{backcolour}, 
    numberstyle=\tiny\color{codegray},
    stringstyle=\color{codepurple},
    breakatwhitespace=false,                          
    keepspaces=true,                 
    numbers=left,       
    numbersep=5pt,                  
    showspaces=false,                
    showstringspaces=false,
    showtabs=false,                  
    tabsize=2,
}

% argmin/argmax
\DeclareMathOperator*{\argmax}{arg\,max}
\DeclareMathOperator*{\argmin}{arg\,min}

% Concatenation of lists
\newcommand\doubleplus{+\kern-1.3ex+\kern0.8ex}

% Program configurations
\newcommand{\tuple}[1]{\ensuremath{\langle #1 \rangle}}
% Rule based definitions
\newcommand{\Rule}[4][]{\ensuremath{\inferrule*[lab={\hypertarget{#2}{(\TirName{#2})}},#1]{#3}{#4}}}

% Calligraphic symbols
\newcommand{\calI}{{\mathcal I}} 
\newcommand{\calT}{{\mathcal T}}

%  Macro for new Y operator.
\newcommand{\yBounded}[3]{\mu^{#1}_{#2}\rvert_{#3}}

%%%%%%%%%%%%%%%%%
 
%%%%%%%%%%%%%%%%%

\newcommand{\expv}{\mathbb{E}}

\newcommand{\combTr}[2]{\left[\begin{matrix}
		#1\\
		#2
	\end{matrix} \right]}

\newcommand{\exType}[2]{\left\{\begin{matrix}
		#1\\
		#2
	\end{matrix} \right\}}
\newcommand{\myint}[1]{ [#1]}
\newcommand{\Uniform}{\ensuremath{\mathrm{Uniform}}}
\newcommand{\Normal}{\ensuremath{\mathrm{normal}}}
\DeclareMathOperator{\abs}{abs}
\DeclareMathOperator{\pdf}{pdf}

\newcommand{\intConf}[1]{\lceil#1\rceil}
\newcommand{\tr}{\boldsymbol{t}}

\newcommand{\sample}{\tt{sample}}
%\newcommand{\fix}{\texttt{fix}}
%\newcommand{\num}[1]{\underline{#1}}
\newcommand{\myif}{\texttt{if}}
\newcommand{\mylet}{\texttt{let} \, }
\newcommand{\myin}{\, \texttt{in} \,}
\newcommand{\mythen}{\, \texttt{then} \,}
\newcommand{\myelse}{\, \texttt{else} \,}
\newcommand{\score}{\tt{score}}
\newcommand{\tick}{\tt{tick}}

\newcommand{\term}{\tt{term}}
\newcommand{\pv}{\mathbf{v}}
\newcommand{\rv}{\mathbf{r}}

\newcommand{\interval}{\mathfrak{I}}

\newcommand{\typeReal}{\textbf{\textsf{R}}}

\newcommand{\symbolInt}{\myint{\cdot}}

\newcommand{\LambdaInterval}{\Lambda_{\interval}}
\newcommand{\LambdaSymbolic}{\Lambda_{\text{sym}}}

\newcommand{\toIntervalTerm}[1]{#1^{2\interval}}

%Others
\newcommand{\Sset}{\mathbb{S}}
\newcommand{\Iset}{\mathbb{I}}
\newcommand{\Rset}{\mathbb{R}}
\newcommand{\Nset}{\mathbb{N}}
\newcommand{\Zset}{\mathbb{Z}}

\newcommand{\Term}{\mathbb{T}}
\newcommand{\prob}{\mathbb{P}}
\newcommand{\expt}{\mathbb{E}}


\newcommand{\Leb}{\tt{Leb}}
\newcommand{\Red}{\tt{Red}}
\newcommand{\cost}{\text{cost}}

%\newcommand{\intervalab}[2]{\underline{[#1,#2]}}
\newcommand{\intervalab}{\underline{[a,b]}}
\newcommand{\interI}{\mathcal{I}}
\newcommand{\trans}{\mathcal{T}}

\newcommand{\iv}{\mathbb{I}}

% Programming language constructs
\newcommand{\lit}[1]{\underline{#1}}
\newcommand{\letIn}[1]{\mathsf{let}\,{#1}\,\mathsf{in}\,}
\newcommand{\fixLam}[2]{\mu {#1} {#2}.}
\newcommand{\ifElse}[3]{\mathsf{if} (#1 \le \num{0}) \, {#2} \,\mathsf{else}\, {#3}}

%%Basic notions
\newcommand{\pspace}{(\Omega,\mathcal{F},\probm)}
\newcommand{\probm}{\mathbb{P}}
\newcommand{\condexpv}[2]{{\expt}{\left[{#1} \mid {#2}\right]}}

\newcommand{\stdConf}[1]{(#1)}
%\newcommand{\intConf}[1]{\lceil#1\rceil}
%\newcommand{\intConf}[1]{(#1)}
%\newcommand{\symConf}[1]{\langle\!\langle  #1 \rangle\!\rangle}
%\newcommand\symPath[1]{(#1)}
\newcommand{\symPath}[1]{\langle\!\langle  #1 \rangle\!\rangle}
\newcommand\symConf[1]{(#1)}

\newcommand{\ifSimple}[3]{\mathsf{if}(#1, #2, #3)}
%\newcommand{\ifElse}[3]{\mathsf{if} (#1 \le 0) \, \allowbreak {#2} \, \allowbreak \mathsf{else}\, {#3}}
%\newcommand{\ifElse}[3]{\ifSimple{#1}{#2}{#3}}

%\newcommand{\trace}{\mathsf{s}}
%
%\newcommand\defn[1]{{\bf \em #1}}
\newcommand{\traces}{\mathbb{T}}
%
%\newcommand{\stdConf}[1]{(#1)}
%%\newcommand{\intConf}[1]{\lceil#1\rceil}
%\newcommand{\intConf}[1]{(#1)}
%%\newcommand{\symConf}[1]{\langle\!\langle  #1 \rangle\!\rangle}
%%\newcommand\symPath[1]{(#1)}
%\newcommand{\symPath}[1]{\langle\!\langle  #1 \rangle\!\rangle}
%\newcommand\symConf[1]{(#1)}

\newcommand{\valueSem}[1]{\mathsf{val}_{#1}} % value (semantics)
\newcommand{\weightSem}[1]{\mathsf{wt}_{#1}} % weight (semantics)
\newcommand{\measureSem}[1]{\llbracket #1 \rrbracket}
\newcommand{\posterior}{\mathsf{posterior}}


%%%%%%%%%
% 
%%%%%%%%
\newcommand{\loc}{\ell}
\newcommand{\locs}{\mathit{L}}
\newcommand{\blocs}{\mathit{L}_{\mathrm{b}}}

\newcommand{\iflocs}{\mathit{L}_{\mathrm{if}}}
\newcommand{\looplocs}{\mathit{L}_{\mathrm{while}}}

\newcommand{\alocs}{\mathit{L}_{\mathrm{a}}}
\newcommand{\wlocs}{\mathit{L}_{\mathrm{w}}}
\newcommand{\rlocs}{\mathit{L}_{\mathrm{r}}}
\newcommand{\Alocs}[1]{\mathit{L}_{\mathrm{A}}^{\mathsf{#1}}}
\newcommand{\Dlocs}{\mathit{L}_{\mathrm{nd}}}
\newcommand{\transitions}{{\rightarrow}}

%%% 
\newcommand{\plocs}{\mathit{L}_{\mathrm{p}}}
\newcommand{\tlocs}{\mathit{L}_{\mathrm{t}}}

\newcommand{\lin}{\loc_\mathrm{init}}
\newcommand{\lout}{\loc_\mathrm{out}}
\newcommand{\val}[1]{\mbox{\sl Val}_{#1}}

\newcommand{\pvars}{V_\mathrm{p}}
\newcommand{\rvars}{V_{\mathrm{r}}}
\newcommand{\pre}{\mathrm{pre}}

\newcommand{\sle}{\sqsubseteq}
\newcommand{\sge}{\sqsupseteq}

\newcommand{\lfp}{\mathrm{lfp}}
\newcommand{\gfp}{\mathrm{gfp}}

\newcommand{\rdvarjdis}{\mathcal D}
\newcommand{\sampset}{\textit{supp}}

\newcommand{\upd}{\mbox{\sl upd}}
\newcommand{\wet}{\mbox{\sl wt}}
\newcommand{\transset}{\mathfrak T}
\newcommand{\valin}{\pv_{\mathrm{init}}}
\newcommand{\ret}{\mbox{\sl ret}}

\newcommand{\win}{w_{\mathrm{init}}}

\newcommand{\sampdpd}{\overline{\Upsilon}}

\newcommand{\outmap}{\text{O}}
\newcommand{\sat}[1]{\langle #1 \rangle}
\newcommand{\monoid}{\mbox{\sl Monoid}}
\newcommand{\handelmanformat}{(\dagger)}

\newcommand{\trunc}{\mathcal{B}}

\newcommand{\ewt}{\mbox{\sl ewt}}
\newcommand{\statemap}{\text{St}}

\newcommand{\valrd}{{\mathbf{r}}}
\newcommand{\frmloc}{\ell^{\mathrm{src}}}
\newcommand{\toloc}{\ell^{\mathrm{dst}}}

\newcommand{\monomials}{\mathbf{M}}
%% \setcounter{secnumdepth}{0}

\definecolor{red}{gray}{0}

\begin{document}


%\title{FedZoo-Bench: A Practical Recipe for Federated Learning with Non-IID Data Experimental Design} 


\title{A Practical Recipe for Federated Learning Under Statistical Heterogeneity Experimental Design} 


\author{Mahdi Morafah, Weijia Wang, and Bill Lin
\thanks{M. Morafah, W. Wang and B. Lin are all with Electrical and Computer Engineering Department of University of California San Diego, USA (e-mail address: \{mmorafah, wweijia, billlin\}@eng.ucsd.edu).}
\thanks{Corresponding author: Mahdi Morafah.}
\thanks{}
}
%\thanks{This paragraph will include the Associate Editor who handled your paper.}}

\markboth{Journal of IEEE Transactions on Artificial Intelligence, Vol. 00, No. 0, Month 2023}
{M. Morafah \MakeLowercase{\textit{et al.}}: IEEE Journals of IEEE Transactions on Artificial Intelligence}

\maketitle

\begin{abstract}
Federated Learning (FL) has been an area of active research in recent years. There have been numerous studies in FL to make it more successful in \textcolor{red}{the} presence of data heterogeneity. However, despite the existence of many publications, the \textcolor{red}{state} of progress in the field is unknown. Many of the works use inconsistent experimental settings and there \textcolor{red}{are} no comprehensive studies on the effect of FL-specific experimental variables on the results and practical insights for a more comparable and consistent FL experimental setup. Furthermore, \textcolor{red}{the} existence of several benchmarks and confounding variables \textcolor{red}{has} further complicated the issue of inconsistency and ambiguity. In this work, we present the first comprehensive study on the effect of FL-specific experimental variables in relation to each other and performance results, \textcolor{red}{bringing} several insights and recommendations for designing a meaningful and well-incentivized FL experimental setup. We further aid the community by releasing FedZoo-Bench, an open-source library based on PyTorch with pre-implementation of 22 state-of-the-art methods\footnote{We will continue the effort to extend FedZoo-Bench by implementing more methods and adding more features. Any contributions to FedZoo-Bench would be greatly appreciated as well.}, and a broad set of standardized and customizable features available at~\url{https://github.com/MMorafah/FedZoo-Bench}. We also provide a comprehensive comparison of several state-of-the-art (SOTA) methods to better understand the current state of the field and existing limitations.
\end{abstract}

\begin{IEEEImpStatement}
Federated Learning aims to train a machine learning model using the massive decentralized data available at IoT and mobile devices, and different data centers while maintaining data privacy. However, despite {the} existence of numerous works, the state of progress in the field is not well-understood. Papers use different methodologies and experimental setups that \textcolor{red}{are} hard to compare and examine the effectiveness of methods in more general settings. Moreover, the effect of federated learning experimental design factors such as local epochs, and sample rate on the performance results have remained unstudied in the field. Our work comprehensively studies the effect of experimental design factors in federated learning, provides suggestions and insights, introduces FedZoo-Bench with \textcolor{red}{the} pre-implementation of 22 state-of-the-art algorithms under a unified setting, and finally measures the state of progress in the field. The studies and findings discussed in our work can significantly help the federated learning field by providing a more comprehensive understanding of the impact of experimental design factors, facilitating the design of better performing algorithms, and enabling \textcolor{red}{a} more accurate evaluation of the effectiveness of different methods. 
\end{IEEEImpStatement}

\begin{IEEEkeywords}
Benchmark, Data Heterogeneity, Experimental Design, Federated Learning, Machine Learning, Non-IID Data.
\end{IEEEkeywords}

%\newpage

%
\usepackage{amsmath}
\usepackage{mathtools}
\usepackage{thmtools}
\usepackage{cancel}
\usepackage{wrapfig}


\newcommand{\cmark}{\ding{51}}%
\newcommand{\xmark}{\ding{55}}%

\usepackage{tcolorbox}
\tcbset{boxsep=0mm,boxrule=0pt,colframe=white,arc=0mm,left=0.5mm,right=0.5mm}

\newcommand\SC{\mathcal{S}}
\newcommand{\antonio}[1]{{\color{magenta} Antonio: ``#1''}}

\DeclareMathOperator*{\argmax}{arg\,max}
\DeclareMathOperator*{\argmin}{arg\,min}

\newcommand{\theHalgorithm}{\arabic{algorithm}}

\usepackage[capitalize,noabbrev]{cleveref}


\DeclareMathOperator{\LRU}{LRU}

\newcommand\A{\mathbf{A}}
\newcommand\V{\mathbf{V}}
\newcommand\B{\mathbf{B}}
\newcommand\C{\mathbb{C}}
\newcommand\Exp{\mathbb{E}}
\newcommand\R{\mathbb{R}}
\newcommand\calM{\mathcal{M}}
\newcommand\calR{\mathcal{R}}
\newcommand\rank{\operatorname{rank}}
\newcommand\eps{\varepsilon}
\newcommand\h{h}
\newcommand\bound{b}
\DeclareMathOperator{\tr}{tr}
\DeclareMathOperator{\vect}{vec}
\DeclareMathOperator{\diag}{diag}




The problem of the presence or absence of phase transition is central in statistical mechanics. To prove the existence of phase transition, the standard idea is to define a notion of contour and use \textit{Peierls' argument} \cite{Peierls.1936}. In the usual Ising model \cite{Ising_25}, particles of the system interact only with their nearest-neighbors. On ferromagnetic long-range Ising models \cite{Anderson_Yuval_69}, there is interaction between each pair of spins in the lattice. The Hamiltonian of the model is given formally by
\begin{equation*}
    H(\sigma) = - \sum_{x,y\in \Z^d}J_{xy}\sigma_x\sigma_y,
\end{equation*}
where $J_{xy}=J|x-y|^{-\alpha}$, $J>0$, $\alpha > d$. It is well-known that the Peierls' argument in dimension 2 implies phase transition for Ising models with nearest-neighbors or long-range interactions when $d\geq 2$, using correlation inequalities. For the unidimensional lattice, it was known that short-range models do not present phase transition. In the long-range case, a different behavior was expected depending on the exponent $\alpha$ (see \cite{Kac_Thompson_69}), but the problem was challenging since contours were first created as multidimensional objects.

In dimension $d=1$, phase transition was proved first in 1969 by Dyson \cite{Dyson.69}, for $\alpha \in (1,2)$, by proving phase transition in an auxiliary model and then using correlation inequalities. In 1982, Fr{\"o}hlich and Spencer \cite{Frohlich.Spencer.82} introduced a notion of one-dimensional contours and then applied the Peierls' argument to show phase transition for the critical value $\alpha = 2$. These contours were inspired by the multiscale techniques previously introduced to study the Berezinskii-Kosterlitz-Thouless transition in two-dimensional continuous spin systems \cite{FS81}. Later, Cassandro, Ferrari, Merola and Presutti  \cite{Cassandro.05} extended the contour argument previously available for $\alpha=2$ to exponents $\alpha\in (3-\frac{\ln 3}{\ln 2}, 2)$, with the additional restriction that the nearest-neighbor interaction is strong, i.e.,  ${J(1)\gg 1}$; this restriction was removed for a subclass of interactions in \cite{Bissacot.Endo.18}. Further results were obtained using contour arguments, such as the decay of correlations, cluster expansions, phase transition with random interactions, etc; some references with these results are \cite{ Cassandro.Merola.Picco.17, Cassandro.Merola.Picco.Rozikov.14, Imbrie.82, Imbrie.Newman.88, Johansson.91}. 

In the multidimensional setting ($d\geq 2$), Ginibre, Grossmann, and Ruelle, in \cite{Ginibre.Grossmann.Ruelle.66}, proved the phase transition for $\alpha > d+1$, using an enhanced version of Peierls' argument and the usual contours. Park proposed a different notion of contour for long-range systems in \cite{Park.88.I, Park.88.II}, extending the Pirogov-Sinai theory available for short-range interactions assuming $\alpha > 3d+1$, although he can also consider Potts models with his methods. Some results in the literature suggest that truly long-range effects appear only when $d < \alpha \leq d+1$, see for instance, \cite{Biskup_Chayes_Kivelson_07}. Recently, Affonso, Bissacot, Endo and Handa \cite{Affonso.2021}, inspired by the ideas from Fr{\"o}hlich and Spencer in \cite{FS81, Frohlich.Spencer.82}, introduced a version of multiscale multidimensional contour and proved phase transition by a contour argument in the whole region $\alpha > d$. They can consider long-range Ising models with deterministic decaying fields, first introduced in the context of nearest-neighbor interactions in \cite{Bissacot_Cioletti_10}. For these models, the lack of analyticity of the free energy does not imply phase transition since these models have the same free energy as the models with zero field. It is expected that fields decaying slowly imply uniqueness. In this setting, a contour argument is useful for proofs of phase transitions as well for uniqueness, some papers with models with deterministic decaying fields are \cite{Aoun_Ott_Velenik_23, Bissacot_Cass_Cio_Pres_15, Bissacot.Endo.18, Cioletti_Vila_2016}.

The Random Field Ising model (RFIM) \cite{Imry.Ma.75} is the nearest-neighbor Ising model with an additional external field acting on each site $(h_x)_{x\in\Z^d}$ that is a family of i.i.d. Gaussian random variable with mean 0 and variance 1. Formally, the Hamiltonian of the model is given by
\begin{equation*}
    H(\sigma) = - \sum_{\substack{x,y\in \Z^d \\|x-y|=1}}J\sigma_x\sigma_y  - \varepsilon\sum_{x\in\Z^d}h_x\sigma_x,
\end{equation*}
where $J>0$, $\varepsilon>0$, $\alpha > d$ and $d \geq 1$. A detailed account of the history of the phase transition problem for this model, as well as detailed proofs, was given in \cite{Bovier.06}. Here we present a brief overview.

During the 1980s, the question of the specific dimension where phase transition for the RFIM should happen attracted much attention and was a topic of heated debate. Two convincing arguments were dividing the physics community. One of them, due to Imry and Ma \cite{Imry.Ma.75}, was a non-rigorous application of the Peierls' argument together with the use of the isoperimetric inequality. The key idea of Peierls' argument is to define a notion of contour and calculate the energy cost of "erasing" each contour, i.e., the energy cost of flipping all spins inside the contour. When there is no external field, that energy necessary to flip the spins in a region $A\subset \Z^d$ is of the order of the boundary $|\partial A|$. When we add an external field, we get an extra cost depending on this field. Imry and Ma argued that this cost should be approximately $\sqrt{|A|}$, which is smaller than $|\partial A|$ for all regions only when $d\geq 3$, so this should be the region where phase transition occurs. The other argument, due to Parisi and Sourlas \cite{Parisi.Sourlas.79}, based on dimensional reduction, predicted that the $d$-dimensional RFIM would behave like the $d-2$-dimensional nearest-neighbor Ising model, therefore presenting phase transition only when $d\geq 4$. 

The question was settled by two celebrated papers showing that Imry and Ma's prediction was correct. First, in 1988, Bricmont and Kupiainen \cite{Bricmont.Kupiainen.88} showed that there is phase transition almost surely in $d\geq3$, for low temperatures and variance $\varepsilon$ small enough. Their proof uses a rigorous renormalization group analysis for the short-range case and it is considered involved. Still, they claimed that the result works for any model with a suitable contour representation and centered sub-gaussian external field. Later on, Aizenman and Wehr \cite{Aizenman.Wehr.90} proved uniqueness for $d\leq 2$. For detailed proofs of these results, we refer the reader to \cite{Bovier.06} (see also \cite{Berretti.85, Camia.18, Frohlich.Imbre.84,  Klein.Masooman.97} for more uniqueness results). 

Recently, Ding and Zhuang, see \cite{Ding2021}, provided a simpler proof of the phase transition, not using RGM. And in  \cite{Ding.Liu.Xia.22}, Ding, Liu and Xia proved that if $\beta_c(d)$ is the critical inverse of the temperature of the Ising model with no field, for all $\beta>\beta_c(d)$ there exists a critical value $\varepsilon_0(d, \beta)$ such that the RFIM with $\varepsilon \leq \varepsilon_0$ presents phase transition. 

In the present paper, we are considering a long-range Ising model with a random field, whose Hamiltonian is given formally by
\begin{equation*}
    H(\sigma) = - \sum_{x,y\in \Z^d}J_{xy}\sigma_x\sigma_y - \varepsilon\sum_{x\in\Z^d}h_x\sigma_x,
\end{equation*}
where $J_{xy}=J|x-y|^{-\alpha}$, $J, \varepsilon>0$, $\alpha > d$ and $h_x\in\mathbb{R}$, $d\geq 3$.
Until now, the only known result in the long-range setting is for the one-dimensional long-range Ising model with a random field, by Cassandro, Orlandi, and Picco \cite{Cassandro.Picco.09}. They used the contours of \cite{Cassandro.05} to show the phase transition for the model when $\alpha\in (3-\frac{\ln 3}{\ln 2}, \frac{3}{2})$, under the assumption $J(1) \gg 1$. We stress that, as remarked by Aizenman, Greenblatt, and Lebowitz \cite{Aizenman_Greenblatt_Lebowitz_2012}, although their argument does not work for the whole region for the exponent $\alpha$, the phase transition holds for values close to the critical value $\alpha=3/2$, since by the Aizenman-Wehr theorem we know that there is uniqueness for $\alpha>3/2$.

The argument from Ding and Zhuang in \cite{Ding2021}, for $d\geq3$, involves controlling the probability of a bad event, which is closely related to controlling the quantity $$\sup_{\substack{0\in A\subset\Z^d \\ A \text{ connected }}}\frac{\sum_{x\in A}h_x}{|\partial A|},$$ known as the greedy animal lattice normalized by the boundary. The greedy animal lattice normalized by the size, instead of the boundary, was extensively studied for general distributions of $(h_x)_{x\in\Z^d}$, see \cite{Cox_Gandolfi_Griffin_Kesten_93, Gandolfi_Kesten_94, Hammond_06, Martin_02}. When we normalize by the boundary, an argument by Fisher, Fr\"{o}hlich and Spencer \cite{FFS84} shows that the expected value of the greedy animal lattice is constant. In dimension $d=2$, the expected value is not finite, see \cite{Ding.Wirth.20}. The supremum is taken over connected regions containing the origin since the interiors of the usual Peierls contours are of this form.


For the long-range model, the interior of contours is not necessarily connected. In fact, long-range contours may have considerably large diameters with respect to their size, so their interiors can be very sparse. To avoid this, we define contours, strongly inspired by the $(M,a,r)$-partition in \cite{Affonso.2021}, using a multiscaled procedure that assures that the contours have no cluster with small density.  With them, we generalize the arguments by Fisher-Fr\"{o}hlich-Spencer \cite{FFS84}, and prove that the expected value of the greedy animal lattice is constant, even considering regions not necessarily connected in the supremum. Then, we prove the phase transition for $d\geq 3$. The main result of this paper is the following.
\begin{theorem*}Given $d\geq 3$, $\alpha>d$, there exists $\beta_c\coloneqq\beta(d, \alpha)$ and $\varepsilon_c\coloneqq\varepsilon(d, \alpha)$ such that, for $\beta >\beta_c$ and $\varepsilon\leq \varepsilon_c$, the extremal Gibbs measures $\mu_{\beta, \varepsilon}^+$ and $\mu_{\beta, \varepsilon}^-$ are distinct, that is, $\mu_{\beta, \varepsilon}^+ \neq \mu_{\beta, \varepsilon}^-$ $\mathbb{P}$-almost surely. Therefore the long-range random field Ising model presents phase transition.
\end{theorem*}

This paper is divided as follows. In Section 2, we define the model and the contours, and suitable generalizations to the constructions in \cite{Affonso.2021} are introduced.  In Section 3, we define two bad events of the external field and prove that they occur with a small probability.  In Section 4, we present the proof of the phase transition.
\section{RELATED WORKS}
\textbf{Imitation learning.}
Imitation learning is learning optimal control policies from the demonstrations from the expert policy executor, such as human expert \cite{schneider_analytic_2018}. Due to its advantage of removing the necessity for extensive interaction with the environment and delicate reward design, imitation learning have been widely used in reinforcement learning \cite{rashidinejad2021bridging, zhu2018reinforcement, sun2018truncated,DBLP:conf/cdc/CosnerYA22,DBLP:conf/cdc/StrongLB22,DBLP:conf/cdc/ZhangHLLZ21,DBLP:conf/cdc/GiammarinoP21,DBLP:conf/cdc/ChenPS021}. One of the major issues in imitation learning is the distributinoal shift problem where the learned policy shows unjustifiable behaviour due to encountering unseen stats during its execution. Existing works show that the distributional shift problem in imitation learning comes from causal misidentification or spurious causal relationship between the data and the policy \cite{de2019causal,DBLP:journals/corr/abs-2106-03443}. 

\textbf{Causal Inference.}
Causal Inference is the well-known problem that deduces the relationships between causes and effects among the variables \cite{DBLP:conf/cdc/Li021,DBLP:conf/cdc/XieKB020,DBLP:conf/cdc/PauliGBA21,DBLP:conf/cdc/ParkP22}. Most existing work take one of the approaches called ``causal discover'',
to find its relationship from recorded observations under some constraints \cite{https://doi.org/10.48550/arxiv.1605.08179,LouizosEtAl_arxiv17,Maathuis2010PredictingCE}.

Regarding with causal inference, one challenge in the imitation learning is confounding, which is also known as $\textit{causal delusion}$ \cite{de2019causal}. Namely, standard imitation learning does not assume the existence of latent variables that may affect the expert behavior but not captured by observations. Ortega et al.\cite{pedro2021CI} propose a interventional distribution in a causal structure model and claim that an imitator needs to refer the distribution during the policy learning. Tien et al. \cite{Tien2022} show a systematic study of causal confusion that learns the preference of reward functions which is provided by the human expert. Vuorio et al.\cite{Yuorio2022} propose learning a latent-conditional policy, where the environment dynamics is influnced by the latent variable. Pim de Haan et al. \cite{DBLP:journals/corr/abs-1905-11979} proposed causal graph parameterized policy learning that maps from multiple causal graphs to policies.

\vspacebeforesection
\section{Background}
\label{sec:background}

In this section, we provide the necessary background information to ensure a comprehensive understanding of the attack described in this paper. We start with a description of the Distributed Hash Table (DHT) used by IPFS, followed by its content resolution mechanisms. We also detail techniques for network size estimation, necessary for our attack detection and mitigation mechanisms.

\vspacebeforesection
\subsection{IPFS DHT}
\label{sec:kad_dht}

We review the features of the Kademlia DHT~\cite{maymounkov2002kademlia} and its \texttt{libp2p} implementation~\cite{libp2p_github} that are the most relevant to our attack.
To participate in the DHT, each peer generates a public/private key pair and derives an identity $\peerid \in \{0,1\}^{256}$ as the hash of its public key.
Ideally, each peer generates a random key pair and, therefore, peer IDs are distributed uniformly and independently over the space $\{0,1\}^{256}$.
While honest nodes follow this rule, malicious nodes may generate and choose from an arbitrary number of key pairs.
Each peer maintains a routing table consisting of $m=256$ buckets.
The $i$-th bucket contains the addresses of up to $k=20$ peers whose peer IDs share a common prefix of exactly $i$ bits with the peer's own peer ID. 

%
A new participant node joins the IPFS network by contacting one of the hardcoded bootstrap nodes. This bootstrap node provides the new node with some initial peers allowing it to join the DHT. The new node uses this information to perform a walk through the DHT towards its own peer ID.
The walk allows to: \textit{(i)}~make sure that there is no other node in the network with the same ID; \textit{(ii)}~discover new peers and fill the newcomer's DHT routing table. At the same time, the newcomer establishes \bitswap~\cite{de2021accelerating} connections to a subset of encountered peers (usually around 300 of them). The core role of the \bitswap protocol is to enable bilateral content transfer and to play the role of a cache for recently-accessed content.

The main DHT operation $\Call{GetClosestPeers}{\key}$ returns the $k=20$ closest peers to $\key$. 
%
In Kademlia, the distance between two keys $x$ and $y$ in the key space is given by $x \oplus y \in \{0,...,2^{256}-1\}$, where $\oplus$ denotes the bitwise XOR operation on the keys; the resulting binary string is interpreted as an integer.
%
When a client wants to find the peers with IDs closest to $\key$, it sends a request to the $\alpha=3$ peers in its routing table whose peer IDs are closest to $\key$. Each of these peers returns the $k$ closest peers to $\key$ in its own routing table and the addresses of these peers. 
%
The client again sends a request to the $\alpha$ peers closest to $\key$, among peers in its routing table and those whose addresses it just received. This process repeats until the client does not find any more peers closer to $\key$.
Due to network churn and imperfect routing tables, we observed in our experiments that successive calls to $\Call{GetClosestPeers}{\key}$ do not always return the same set of $k=20$ peers (we provide more details in \Cref{sec:evaluation}, \Cref{fig:20closest}). This is an important limitation affecting our attack.

\vspacebeforesection
\subsection{Content Resolution in IPFS}
\label{sec:ipfs}

IPFS is a content-centric network.
It allows its participant to request files without specifying their location. 
%
Content is indexed by content IDs $\cid \in \{0,1\}^{256}$ that are derived from a hash of that content.
Both peer IDs and CIDs are used as keys in the DHT.
Each node can play the role of a \provider, \downloader, or \resolver. 
The process of content advertisement and resolution is illustrated in \Cref{fig:add_get_provider}.

%
When a \provider wishes to publish content with a given $\cid$ on IPFS, it creates a \emph{provider record} that contains $cid$ and the \provider's address.
During a $\Call{Provide}{\cid}$ operation, the \provider first uses $\Call{GetClosestPeers}{\cid}$ to locate the $k=20$ peers with their peer IDs closest to $\cid$, 
%
and then sends them a $\mathsf{PutProvider}$ message including the provider record (\Cref{fig:add_get_provider}(a)).
We call the peers that hold provider records for $\cid$ the \emph{resolvers} for $\cid$.

Each CID can have several \providers. In fact, by default, each IPFS client becomes a provider for each piece of content it downloads for a fixed amount of time (12h, 24h, or 48h depending on the client version or custom configuration). As a result, the system provides an auto-scaling feature with supply automatically rising with demand.

%
When a \downloader wishes to fetch a piece of content, it first sends a request to all its \bitswap peers. If none of them has the content, the \downloader uses the DHT-based resolution system. We stress that the \bitswap protocol plays the supporting role of a cache in the dissemination of popular files. However, the mechanism does not provide reliable content resolution, in particular for new or less popular content. %

When \bitswap unstructured search fails, the \downloader resolves $\cid$ using $\Call{FindProviders}{\cid}$. This operation uses a DHT walk identical to that of $\Call{GetClosestPeers}{\cid}$ to find $k$ \resolvers but also queries encountered nodes for a provider record for $\cid$ (\Cref{fig:add_get_provider}(b)). The process terminates when either 20 \providers have been found, or all \resolvers have been asked. Querying all encountered nodes (\ie, not only the designated \resolvers) is useful because some of the encountered nodes may have a provider record in their cache.
%

Upon receiving a provider record, the client connects to the address specified in the provider record to retrieve the actual content (\Cref{fig:add_get_provider}(c)).
Provider records are not authenticated, and therefore malicious \providers may respond with incorrect provider records (or may not respond at all). However, the integrity of the content is preserved because the hash of the retrieved content can be verified against its $\cid$.
%


%

\input{img/add_get_provider.tex}

\vspacebeforesection
\subsection{Network Size Estimator}
\label{sec:netsize}

The number of nodes in a decentralized system is generally unknown due to the avoidance of centralized membership management.
This number is nonetheless useful for optimizations, deciding on individual node configurations, or security mechanisms.
Various methods were proposed for the decentralized estimation of unstructured and structured networks~\cite{eli-sohl-dht-size-estimation,kostoulas2005decentralized, manku2003symphony}.
We use in this work a mechanism developed initially by Protocol Labs as part of a mechanism for decreasing the latency of publishing content in IPFS~\cite{network-size-estimation-notion,network-size-estimation-github-pr}.

%
%
%
%
%
%
%
%
%
%

Each node in the DHT refreshes its routing table periodically (every $10$ minutes in \texttt{libp2p}). 
For this, the node samples $m$ random keys (one for each bucket of its routing table)
%
and queries the DHT to obtain the $k=20$ closest peer IDs to each key.
Using these, the node then computes the average distance between each one of these keys $\key_j$ for $j=1,\dots,m$ and their $i$-th closest peer ID for $i=1,...,k$ (with $m=256$ and $k=20$).
\begin{equation}
    \label{equ:avg-dist}
    \overline{D}_i = \frac{1}{m} \sum_{j=1}^m \operatorname{dist}(\key_j, \peerid_{j}^{(i)})
\end{equation}
where $\peerid_{j}^{(i)}$ is the $i$-th closest peer ID to $\key_j$.
With $N$ peers in the DHT and peer IDs uniformly distributed in the hash space, the expected distance between a $\key$ and its $i$-th closest peer ID is $\frac{2^{256}i}{N+1}$. The node then runs a least square regression to compute the value of $N$ for which the expected distances best fit the empirical average distances, \ie,
\begin{equation}
    \label{equ:netsize-least-squares}
    \hat{N} = \arg\min_{N} \sum_{i=1}^k \left(\overline{D}_i - \frac{2^{256}i}{N+1}\right)^2.
\end{equation}
The resulting estimate $\hat{N}$ can be computed in closed form.
%

When a node starts running, it must perform DHT queries for a few random keys to initialize its network size estimate. 
Since a larger number of queries will result in higher accuracy, making more queries than what is needed to initialize one's routing table is recommended.
Thereafter, keeping the estimate up-to-date does not require any excess DHT queries beyond what is already used for refreshing the routing table as this is done frequently (every 10 minutes).

While the network size estimate has a stochastic variance resulting from the probability distribution of the honest peer IDs, it is hard for an attacker to bias the estimate significantly. Since the estimator uses the density of peer IDs around keys chosen uniformly at random, the adversary would require numerous Sybil nodes (on the order of the whole network size) to significantly affect the peer ID density around those keys.

%\section{Findings From The Literature}
In this section we provide our findings from the literature after reviewing a corpus of 200 papers. 
\subsection{Overview of Our Corpus}

\subsection{Non-IID Partitioning}
In the existing literature, the majority of works have focused on label distribution skew case. We find the following ways to simulate label distribution skew from the prior arts: 
\begin{itemize}
    \item \textbf{Label skew($p\%$):} in this case $p\%$ of the total classes are being randomly selected for each clients; then, each class data points are being partitioned amongst the clients who have that class.
    \item \textbf{Label Dir($\alpha$):} in this case each client draws a random vector from $\textit{Dir}(\alpha)$ for their class data points; then, each class data points are being partitioned amongst clients based on each clients' proportions.
    \item \textbf{Random shard($p$):} in this case, dataset is being partitioned into shards based on the number of clients, where each shard has only one class; then, each client receives $p$ random shards. In this scenario, clients have at most $p$ classes. 
    \item \textbf{Quantity Dir($\alpha$):} in this case each client draws a random number from $\textit{Dir}(\alpha)$ for their data points; then, data points are being randomly partitioned amongst clients based on each clients' data proportions.
    \item \textbf{Affine Distribution Shift:}
    \item \textbf{Feature noise:} 
\end{itemize}
Figure~\ref{} shows the demographic percentage of Non-IID simulation ways used in our corpus. Leaf also provides a simulated Non-IID partitions based the natural inherent Non-IID.

\subsection{Motivation for Personalization}

\subsection{Evaluation Metrics}

\subsection{Confounding Variables}

\subsection{Differences in Methodology}

\subsection{Hyper-parameters and Settings}
In fact we find that prior works compare with each other under different settings and hyper-parameters. Figure~\ref{} shows 
\begin{itemize}
    \item Dataset and Architecture demographic
    \item Accuracy results demographic 
    \item Non-IID setting demographic
\end{itemize}

\section{Comprehensive Study on FL Experimental Variables}\label{study}
%\subsection{Overview and Setup}
\begin{comment}
In the previous sections, we discussed that there is a lack of consistent and experimental design settings in the literature and prior works tend to create the settings in a way to satisfy their assumptions and show better results. In summary, prior works tend to 
\begin{itemize}
    \item make it unclear about the exact experimental settings, and evaluation metrics
    \item do not follow a exact experimental settings universally
    \item not consider the effect of confounding variables in their results.
\end{itemize}

These obstacles have made it difficult for a fair and comprehensive evaluation of the proposed methods and identify the-state-of-the-art baselines. To identify the best experimental design settings and suggestions, we perform a comprehensive study on different experimental design factors and bring our findings in this section.
\end{comment} 
%In this section, we conduct a comprehensive study on the impact of various experimental variables in FL.
\vspace{-0.2cm}

\textbf{Overview.}  To design an effective FL experiment, it is crucial to understand how the FL-specific variables which are clients' data partitioning (type and level of statistical heterogeneity), local epochs ($E$), sample rate ($C$), and communication rounds ($T$) interact with each other and can affect the results. While communication rounds ($T$) serve as the equivalent of epochs in traditional centralized training and primarily determine the training budget, the other variables have a more direct impact on performance. Hence, we focus our analysis on these variables, in relation to each other and performance results, and evaluation metric failure and derive new insights for the FL community to design meaningful and well-incentivized FL experiments.

%Our analysis of these variables, in relation to each other and performance results, provides new insights and best practices for the FL community to design meaningful and well-incentivized FL experiments and avoid evaluation metric failure.
%We also identify the causes of evaluation metrics failure for each FL approach and bring our suggestions on how to avoid them. The study that we provide in this section derives new insights for the broader FL community. 
%We also bring our findings and suggestions on how to control the effect of each variables for a fair experimental design.

\textbf{Baselines.} We use three key baselines in our study: \emph{FedAvg}~\cite{mcmahan2017communication}, the standard FL baseline that has been widely used in the existing literature and can serve as a good representative for the global FL approach~\footnote{\label{foot1}We show in Section~\ref{sec:comparison} that the performance of this baseline is competitive to the SOTA methods.}; \emph{FedAvg + Fine-Tuning (FT)}~\cite{jiang2019improving}, a simple personalized FL baseline that has been shown to perform well in practice and can serve as a representative for the personalized FL approach~\footref{foot1}; and \emph{SOLO}, a solo (local only) training of each client on their own dataset without participation in federation, which serves as a baseline to evaluate the benefit of federation under different experimental conditions. 
%Our results show that the performance of these baselines is competitive with state-of-the-art methods, indicating that our studies can be generalized to both global and personalized FL approaches.

%We use three important baselines for our study in this section. \emph{FedAvg}~\cite{mcmahan2017communication} which is the standard FL baseline that has been compared with in almost the entire of existing works and can be used as a good representative for the global FL approach~\footnote{\label{foot1}We show in Section~\ref{sec:comparison} that the performance of this baseline is competitive to the SOTA methods and therefore, our studies can be generalized to global and personalized FL approaches.}.
%widely used in the real world applications~\cite{apple-siri}. \emph{FedAvg + Fine-Tuning (FT)}~\cite{jiang2019improving} which is the simplest personalized federated learning baseline and has been shown performing well in practice~\cite{wang2019federated,sim2021robust}. This baseline can be also served as a good representative for the personalized FL approach~\footref{foot1}. \emph{SOLO} which is a simple solo (local only) training of each client on their own dataset without participation in federation and can be competitive when clients have enough data or the statistical heterogeneity level is very high. This serves as a good comparison baseline to identify if federation has enough incentive under a specific experimental setting. 

\textbf{Setup.} We use CIFAR-10~\cite{krizhevsky2009learning} dataset and LeNet-5~\cite{lecun1998gradient} architecture which has been used in the majority of existing works. We fix number of clients ($N $), communication rounds ($R$), and sample rate ($C$) to $100$, $100$, \text{and} $0.1$ respectively; unless specified otherwise. We use SGD optimizer with \textcolor{red}{a} learning rate of $0.01$, and momentum of $0.9$~\footnote{This optimizer and learning rate have been used in some works~\cite{collins2021exploiting, li2021federated, vahidian2022efficient} under a similar setup and we also find that it works the best for our studies.}. We use this base setting for all of our experimental \textcolor{red}{studies} in this section. The reported results are the average results over 3 independent and different runs for a more fair and robust assessment.
% This is general setting which has been used in the majority of existing works as well.

% Figure environment removed


\subsection{Evaluation Metric}
\vspace{-0.1cm}
\textcolor{red}{The evaluation metric} for performance is a critical factor in making a fair and consistent assessment in FL. However, the way in which evaluation metrics are calculated in the current FL literature is often ambiguous and varies significantly between papers. In this part we focus on identifying the causes of evaluation metric failures for each FL approach and bring our suggestions for avoiding them.


\textbf{Global FL.} The evaluation metric is the performance of \textcolor{red}{the} global model on the test dataset at the server. We find that the causes for evaluation failures are (1) the round used to report the result and (2) the test data percentage used to evaluate the model. Figure~\ref{fig1:a} shows the global model accuracy over the last $10$ rounds on Non-IID Label Dir($0.1$) partitioning. We can see there is a maximum variation of $7\%$ in the results based on which round to pick for reporting the result. Also, the difference of \textcolor{red}{the} final round result with the average bar is about $4\%$. This shows that the round used to report the result is important and to have a more robust metric \textcolor{red}{for} these variations, it is better to report the average results over a number of rounds. Figure~\ref{fig1:b} shows the variations of the reported result for the same model using different test data percentages. This clearly shows that using different test data points can cause bias in the evaluation. To avoid the mentioned failures and have a more reliable evaluation metric we suggest the following definition:

\begin{tcolorbox}[colback=white!5!white,colframe=black!75!black]\label{def:gfl}
  \textbf{Definition 1} (global FL evaluation metric). We define the average performance of \textcolor{red}{the} global model on the entire test data at the server (if available) over the last $[C \cdot N]$ communication rounds as the evaluation metric for global FL approach~\footnote{\label{foot2}We use this metric definition for all of our experiments.}.
\end{tcolorbox}
\vspace{-0.2cm}
\textbf{Personalized FL.} The evaluation metric is the final average performance of all participating clients on their local test data. The factor which can cause evaluation failure is the local test data percentage used to evaluate each client's model. \textcolor{red}{Prior works have allocated different amounts of data as local test sets to individual clients.} Figure~\ref{fig1:c} shows the variability of the reported results for the same clients under different local test data percentages on Non-IID Label Dir($0.1$) partitioning. This highlights that the use of a randomly selected portion for test data can lead to inaccurate and biased evaluations based on the selected data points be easy or hard.
%\footnote{Previous works, use different percentage of test data for each client when partitioning the dataset across clients.} 
To avoid the mentioned failure in the evaluation metric we suggest the following definition:
\vspace{-0.1cm}
\begin{tcolorbox}[colback=white!5!white,colframe=black!75!black]\label{def:pfl}
  \textbf{Definition 2} (personalized FL evaluation metric). We define the average of \textcolor{red}{the} final performance of all the clients on their entire local test data (if available) as the evaluation metric for personalized FL approach~\footref{foot2}\textsuperscript{,}\footnote{\textcolor{red}{The entire local test data consists of allocating all available test samples belonging to the classes owned by the client.}}.
\end{tcolorbox}

\subsection{Statistical Heterogeneity and Local Epochs} \label{sec:heterogeneity-localepoch}
In this part, we focus our study to understand how different \textcolor{red}{levels and types} of statistical heterogeneity together with local epochs can affect the results and change the globalization and personalization incentives in FL.

% Figure environment removed

\textbf{Level of statistical heterogeneity.} Figure~\ref{fig2:a} illustrates how the performance of the baselines for a fixed label Dir type of statistical heterogeneity with 10 local epochs varies with the level of statistical heterogeneity. As the level of statistical heterogeneity decreases, the global FL approach becomes more successful than the personalized one. The vertical line in the plot indicates the approximate boundary between the incentives of the two FL perspectives. We can see that in the extreme Non-IID case ($\alpha=0.05$), neither of the FL approaches \textcolor{red}{is} motivated, as the performance of the SOLO baseline is competitive. Additionally, from $\alpha=0.8$ onwards, the global FL approach seems to perform \textcolor{red}{close} to the end of the spectrum at $\alpha=\infty$, which is IID partitioning. Furthermore, we find that the incentives for globalization and personalization can vary with changes in the number of local epochs for a fixed type of statistical heterogeneity (see Section~\ref{appendix:incentives} for more results).

\textbf{Local epochs.} Figures~\ref{fig2:b}, and~\ref{fig2:c} show how the performance of FedAvg and FedAvg + FT for a fixed label Dir type of statistical heterogeneity varies with different \textcolor{red}{levels} of statistical heterogeneity and \textcolor{red}{the} number of local epochs. Figure~\ref{fig2:b} suggests that FedAvg favors fewer local epochs to achieve higher performance. However, Figure~\ref{fig2:c} suggests that FedAvg + FT favors more local epochs for achieving better results but no more than $5$ or $10$ depending on the level of statistical heterogeneity. Our findings support the observation in~\cite{karimireddy2020scaffold} that client drift can have a significant impact on performance, and increasing the number of local epochs amplifies this effect in the results.
%Generally, for both approaches in FL it is desirable to have more local training while not hurting the results which is an aspect that researchers need to consider in their algorithm design.

%Local epoch of $\{5, 10\}$, and $\{1, 5\}$ seems to work really good for Non-IID Label Dir, and Non-IID Label Skew respectively. Another observation is that excessive fine-tuning more than $10$ local epochs can hurt the results. 
% In FL, it is desirable to have more local training and fewer communication rounds while not hurting the results. Increasing the number of local epochs can magnify the effect of client drift which is an aspect that researchers need to consider to handle in their algorithm desing.

% Figure environment removed

\textbf{Type of statistical heterogeneity.} Figure~\ref{fig3} illustrates the results of an experiment similar to that of Figure~\ref{fig2}, but with a label skew type of statistical heterogeneity. Figure~\ref{fig3:a} shows the performance of the baselines with fixed $10$ local epochs at different levels of statistical heterogeneity. Comparing this figure with Figure~\ref{fig2:a} reveals that this type of heterogeneity favors personalization over globalization across a wider range of heterogeneity levels. Figures~\ref{fig3:b} and~\ref{fig3:c} show the performance of FedAvg and FedAvg + FT with different levels of statistical heterogeneity and local epochs. Comparing these figures with Figures~\ref{fig2:b} and~\ref{fig2:c} for the label Dir type of statistical heterogeneity reveals that this type of statistical heterogeneity is less affected by an increase in local epochs at each level of heterogeneity. This highlights another finding that the effect of client drift may vary for different types of statistical heterogeneity (see Section~\ref{appendix:incentives} for more results).

%\textbf{Summary and Suggestions.} We observe that level of heterogeneity and local epochs are related all together and can affect personalization and globalization incentives. In particular, we notice that the effect of client drift phenomenon is more sever for Non-IID Label Dir partitioning compared to Non-IID Label Skew and Non-IID Label Skew partitioning favors personalized FL approach on a wider spectrum of level of heterogeneity compared to Non-IID Label Dir. Table~\ref{tab:incentivized-settings} identifies the well-incentivized settings for each FL approach.

% \begin{tcolorbox}[colback=white!5!white,colframe=black!75!black]
%   \textbf{Finding 1} (Global FL incentivized settings). We find the following (Non-IID, Epoch) combinations are well incentivized settings for global FL:
%   \begin{itemize}
%       \item \textbf{Dir:} $(\alpha >= 0.5, E \in \{1, 5, 10, 20\})$ 
%       \item \textbf{Label Skew:} $(\rm{ls} >= 8, E \in \{1, 5, 10, 20\})$, except $(\rm{ls} = 8, E = 20)$
%   \end{itemize}
% \end{tcolorbox}

% \begin{tcolorbox}[colback=white!5!white,colframe=black!75!black]
%   \textbf{Finding 2} (Personalized FL incentivized settings). We find the following (Non-IID, Epoch) combinations are well incentivized settings for personalized FL:
%   \begin{itemize}
%       \item \textbf{Dir:} $(0.05 < \alpha < 0.5, E \in \{1, 5, 10, 20\})$ 
%       \item \textbf{Label Skew:} $(2 <= \rm{ls} < 8, E \in \{1, 5, 10, 20\})$, except $(\rm{ls} = 8, E = 20)$
%   \end{itemize}
% \end{tcolorbox}

\begin{comment}
\subsection{Number of Communication Rounds}
Figure~\ref{fig4:a}, \ref{fig4:b}, and~\ref{fig4:c}, \ref{fig4:d} show the effect of number of communication rounds on Non-IID Dir(0.1) and Non-IID Label Skew (30\%) for different number of local epochs. We observe that more communication rounds can improve the results of FedAvg and FedAvg+FT for $1$ local epoch. However, more communication rounds hurt the results for higher local epochs. 

\begin{tcolorbox}[colback=white!5!white,colframe=black!75!black]
  \textbf{Finding 3} (Effect of communication rounds). We find that higher number of communication rounds indeed leads to better results for 1 local epochs. However, for more than 1 local epochs more number of communication rounds can hurt the results due to magnification of client drift effect. Also, hyper-parameter tuning becomes more difficult for more than 1 local epochs along with higher than 250 number of communication rounds.\footnote{We observe similar exists for other level of heterogeneity.}
\end{tcolorbox}

% Figure environment removed

\end{comment}

\subsection{Sample Rate}
Figures \ref{fig5:a}, \ref{fig5:b} and \ref{fig5:c},~\ref{fig5:d} demonstrate the impact of sample rate and local epochs on the performance for label Dir($0.1$) and label skew ($30$\%) types of statistical heterogeneity, respectively. Our observations show that \textbf{increasing the sample rate (i.e., averaging more models) can effectively \textcolor{red}{mitigate} the \textcolor{red}{negative} impact of statistical heterogeneity on performance}\footnote{\textcolor{red}{Model averaging is a standard component of many FL algorithms.}}. \textcolor{red}{Additionally, it is essential to consider that averaging a higher number of models (i.e., a high sample rate) reduces the approximation error associated with the averaged models across all clients\footnote{\textcolor{red}{It is also worth mentioning that generally high sample rates are not favorable, as they would increase the communication cost of the FL algorithms.}}.} We find that sample rates of $C >= 0.4$ can significantly reduce the effect of heterogeneity on the performance, while very small sample rates of $C < 0.1$ result in poor performance. Based on these findings, we suggest using a sample rate in the range of $0.1 <= C < 0.4$ for experimental design to accurately evaluate an algorithm's success in \textcolor{red}{the} presence of data heterogeneity.

% \begin{tcolorbox}[colback=white!5!white,colframe=black!75!black]
%   \textbf{Finding 4} (Effect of sample rate). We find that higher sample rate can indeed improve the results for all number of local epochs. Moreover, this improvement is more for lower than 5 local epochs. Also, hyper-parameter tuning becomes more difficult for higher than 0.5 number of local epochs.\footnote{We observe similar exists for other level of heterogeneity.}
% \end{tcolorbox}

% Figure environment removed

\begin{comment}
\subsection{Number of Clients}
Figure~\ref{fig6:a}, and~\ref{fig6:b} show the effect of number of clients on Non-IID Dir(0.1) and Non-IID Label Skew (30\%) for different number of local epochs with a fixed sample rate $C=0.1$. Increasing the number of clients with a fixed sample rate has increased the performance results. This happens due to the effect averaging more models as the number of clients increases. In Figure~\ref{fig6:b}, we have plotted the results with adaptive sample rate. In this figure we adaptively adjusted the sample rate with the number of clients so that the number of models being averaged at the server be between 8 to 15. This plot shows that the results are the same.  

% \begin{tcolorbox}[colback=white!5!white,colframe=black!75!black]
%   \textbf{Finding 5} (Effect of sample rate). We find that higher sample rate can indeed improve the results for all number of local epochs. Moreover, this improvement is more for lower than 5 local epochs. Also, hyper-parameter tuning becomes more difficult for higher than 0.5 number of local epochs.\footnote{We observe similar exists for other level of heterogeneity.}
% \end{tcolorbox}

% Figure environment removed


% Figure environment removed

\end{comment}
\section{Summary and Recommendations}\label{recommendation}
In this section we identify a series of best practices and recommendations \textcolor{red}{for} designing a well-incentivized FL experimental setting base on our findings and insights in Section~\ref{study}.

\textbf{Level of statistical heterogeneity and local epochs.} The level of statistical heterogeneity and \textcolor{red}{the} number of local epochs determine the incentives for global and personalized FL. Generally, when the level of statistical heterogeneity is low, global FL is more incentivized than personalized FL, and vice versa (Figures~\ref{fig2:a} and~\ref{fig3:a}). Additionally, increasing the number of local epochs can make personalized FL more incentivized (Figures~\ref{app:fig-incentive-dir} and~\ref{app:fig-incentive-skew}). We have identified the well-incentivized settings for each FL approach in Table~\ref{tab:incentivized-settings}.

\begin{table}[t]
\caption{Well-incentivized settings for personalized and global FL approaches.}
\label{tab:incentivized-settings}
\centering
\resizebox{.6\linewidth}{!}{
\begin{tabular}{l|cc|cc|}
            \toprule
\multirow{2}{*}{Local Epoch} & \multicolumn{2}{c|}{Non-IID Label Dir} & \multicolumn{2}{c|}{Non-IID Label Skew}\\
          \cmidrule{2-3} \cmidrule{4-5}
 & pFL & gFL & pFL & gFL \\
            \midrule
$E=1$  & $\alpha < 0.3$ & $\alpha > 0.3$ & $p < 0.8$ & $p > 0.8$\\
            \midrule   
$E=5$  & $\alpha < 0.3$ & $\alpha > 0.3$ & $p < 0.8$ & $p > 0.8$\\
            \midrule
$E=10$  & $\alpha < 0.5$ & $\alpha > 0.5$ & $p < 0.8$ & $p > 0.8$\\
            \midrule
$E=20$  & $\alpha < 0.5$ & $\alpha > 0.5$ & $p < 0.9$ & $p > 0.9$\\
            \midrule
\end{tabular}
}
\end{table}

\textbf{Type of statistical heterogeneity.} We observe that the nature of label skew type of statistical heterogeneity favors personalized FL over a wider range of heterogeneity levels compared to label Dir (Figures~\ref{fig2:a},~\ref{fig3:a},~\ref{app:fig-incentive-dir} and~\ref{app:fig-incentive-skew}). Additionally, we observe that the impact of local epochs on performance is more pronounced for label Dir type of statistical heterogeneity compared to label skew (Figures~\ref{fig2:b},~\ref{fig2:c},~\ref{fig3:b} and~\ref{fig3:c}). To provide a comprehensive perspective on an algorithm's success in the presence of statistical heterogeneity, we recommend researchers conduct experiments with both types of statistical heterogeneity.

\textbf{Sample rate.} This variable plays a crucial role in evaluating an algorithm's performance under statistical heterogeneity. 
%Choosing a high sample rate ($C > 0.4$) can mask the effect of statistical heterogeneity and misrepresent an algorithm's ability to handle it, while a low sample rate ($C < 0.1$) can hinder convergence due to a lack of enough models to average (Figure~\ref{fig5}). 
\textcolor{red}{Choosing a high sample rate ($C > 0.4$) can mask the effect of statistical heterogeneity, thus misrepresenting an algorithm's true ability to handle it. Additionally, it can lead to inaccurate representations of an algorithm's capability to handle the stochasticity resulting from random device selection and the inherent errors caused by model averaging approximation. On the other hand, a low sample rate ($C < 0.1$) may hinder convergence due to insufficient models for averaging and high errors caused by model averaging approximation (Figure~\ref{fig5}).} To avoid these pitfalls, we recommend researchers use a sample rate of $0.1 \leq C \leq 0.4$ for their experiments.
% choosing a high sample rate ($C > 0.4$) can mask the effect of statistical heterogeneity and misrepresent an algorithm's true ability to handle it on one hand and lead to inaccurate representation of an algorithm's ability to handle the stochasticity resulting from random device selection and the inherent errors caused by model averaging approximation on the other hand. while a low sample rate ($C < 0.1$) can hinder convergence due to a lack of enough models to average and high errors caused by model averaging approximation. 

\textbf{Summary.} To ensure consistency, comparability, and meaningful results in FL experiments, we have compiled a set of recommended settings in Table~\ref{tab:recommended-settings}. We encourage researchers to adopt these settings, as well as the evaluation metrics outlined in Section~\ref{def:gfl}, for their experiments. This will facilitate more consistent and fair comparisons with SOTA algorithms, and eliminate concerns about evaluation failures and the impact of various experimental settings.

\begin{table}[ht]
\caption{Recommended settings for pFL and gFL approaches.}
\label{tab:recommended-settings}
\centering
\resizebox{1.0\linewidth}{!}{
\begin{tabular}{c|ccccc}
            \toprule
Approach & Type of Heterogeneity & Level of Heterogeneity & Local Epoch & Number of Clients & Sample Rate\\
            \midrule
\multirow{2}{*}{pFL} &  Label Dir & $\alpha \in [0.01, 0.3]$ & $\{1, 5, 10, 20\}$ & $\{20, 100\}$ & $\{0.1, 0.2, 0.3, 0.4\}$ \\
            \cmidrule{2-6}
  & Label Skew & $p \in \{2, 3, 4, 5\}$ & $\{1, 5, 10, 20\}$ & $\{20, 100, 500\}$ & $\{0.1, 0.2, 0.3, 0.4\}$ \\
            \midrule
\multirow{2}{*}{gFL} &  Label Dir & $\alpha \in (0.3, 1]$ & $\{1, 5, 10, 20\}$ & $\{20, 100\}$ & $\{0.1, 0.2, 0.3, 0.4\}$ \\
            \cmidrule{2-6}
  & Label Skew & $p \in \{8, 9\}$ & $\{1, 5, 10, 20\}$ & $\{20, 100, 500\}$ & $\{0.1, 0.2, 0.3, 0.4\}$ \\
  \bottomrule
\end{tabular}
}
\vspace{-0.5cm}
\end{table}

\begin{comment}
\begin{table}[ht]
\caption{Recommended settings for pFL and gFL approaches.}
\label{tab:recommended-settings}
\centering
\resizebox{0.9\linewidth}{!}{
\begin{tabular}{c|ccccc}
            \toprule
Approach & Type of Heterogeneity & Level of Heterogeneity & Local Epoch & Number of Clients & Sample Rate\\
            \midrule
\multirow{2}{*}{pFL} &  Label Dir & $\alpha \in [0.01, 0.3]$ & $E \in \{1, 5, 10, 20\}$ & $N \in \{20, 100\}$ & $S \in \{0.1, 0.2, 0.3, 0.4\}$ \\
            \cmidrule{2-6}
  & Label Skew & $p \in \{2, 3, 4, 5\}$ & $E \in \{1, 5, 10, 20\}$ & $N \in \{20, 100, 500\}$ & $S \in \{0.1, 0.2, 0.3, 0.4\}$ \\
            \midrule
\multirow{2}{*}{gFL} &  Label Dir & $\alpha \in [0.3, 1]$ & $E \in \{1, 5, 10, 20\}$ & $N \in \{20, 100\}$ & $S \in \{0.1, 0.2, 0.3, 0.4\}$ \\
            \cmidrule{2-6}
  & Label Skew & $p \in \{8, 9\}$ & $E \in \{1, 5, 10, 20\}$ & $N \in \{20, 100, 500\}$ & $S \in \{0.1, 0.2, 0.3, 0.4\}$ \\
  \bottomrule
\end{tabular}
}
\end{table}
\end{comment}
\section{FedZoo-Bench}\label{fedzoo}
We introduce FedZoo-Bench, an open-source library based on PyTorch that facilitates experimentation in federated learning by providing researchers with a comprehensive set of standardized and customizable features such as training, Non-IID data partitioning, fine-tuning, performance evaluation, fairness assessment, and generalization to newcomers, for both global and personalized FL approaches. Additionally, it comes pre-equipped with a set of models and datasets, \textcolor{red}{and} pre-implemented 22 different SOTA methods, allowing researchers to quickly test their ideas and hypotheses. FedZoo-Bench is a powerful tool that empowers researchers to explore new frontiers in federated learning and \textcolor{red}{enables} fair, consistent, and reproducible research in federated learning. We have provided more details on the implemented algorithms and available features in Appendix Section~\ref{sec:app-fedzoo}.



% our codebase has many implementations
% previous benchmarks put their effort on to make FL runnable on different platforms and implement few algorithms or only FedAvg. 
% extension of previous benchmarks to cutomizable settings is not easy or clear thus not convenient for academic research purposes. not flexible to custom implementations and research
% researchers need to compare with other baselines and usually the baselines have been implemented in different codebases for a fixed setting. Fedzoo-Bench eliminated the need for researchers to spend time re-implement the methods in their codebase to test on their customized setting. 
% By providing these features in a standardized and easy-to-use format, FedZoo-Bench enables researchers to focus on their research and not on the technicalities of implementation.
% Federated learning is a rapidly growing field with the potential to revolutionize the way we approach machine learning. However, the complex nature of federated learning can make it challenging for researchers to implement their ideas and study its full potential.

% All of the existing FL benchmarks put their focus on making different platforms to be able to perform federated learning. They mostly implement few algorithms (many just FedAvg) using their platform to realize that and their platform has less flexibility to be extended to different experimental settings for academic research purposes. However, FedZoo-Bench contains the implementation of more than XX different algorithms in PyTorch with easy extension to custom algorithm implementation and experimental setups. We defer discussions of FedZoo-Bench's implementation and use case for different settings to the project's documentation.

%Despite the extensive amount of works have been done for both FL approaches, the state of progress is not well understood. This issue comes from lack of understanding in FL experimental design setups and consistent evaluation of methods under a common experimental setup and implementation.
\subsection{Comparison of SOTA methods}\label{sec:comparison}
In this section, we present a comprehensive experimental comparison of several SOTA FL methods using FedZoo-Bench. We evaluate their performance, fairness, and generalization to new clients \textcolor{red}{in} a consistent experimental setting. Our experiments aim to provide a better understanding of the current progress in FL research.

\textbf{Training setting.} Following our recommended settings indicated in Table~\ref{tab:recommended-settings}, we choose two different training settings presented in Table~\ref{tab:training-setting} to conduct the experimental comparison of the baselines for each FL approach. We run each of the baselines 3 times and report the average and standard deviation of the results. 
%For more details of the hyperparameters used for each baseline, see Appendix Section~\ref{sec:app-hyperparameters}.

\begin{table}[hb]
\caption{Training settings}
\centering
\resizebox{0.5\textwidth}{!}{
\begin{tabular}{c|ccccccc}
\toprule
Setting           & Dataset & Architecture & clients & sample rate & local epochs & partitioning & communication rounds  \\
\midrule
gFL \#1 & CIFAR-10 & LeNet-5 & 100 & 0.1 & 5 & Non-IID Label Skew(80\%) & 100\\
\midrule
gFL \#2 & CIFAR-100 & ResNet-9 & 20 & 0.2 & 10 & Non-IID Label Dir(0.5) & 100 \\
\midrule
pFL \#1 & CIFAR-10 & LeNet-5 & 100 & 0.1 & 10 & Non-IID Label Skew(30\%) & 100\\
\midrule
pFL \#2 & CIFAR-100 & ResNet-9 & 20 & 0.2 & 10 & Non-IID Label Dir(0.1) & 100 \\
\bottomrule
\end{tabular}
}
\label{tab:training-setting}
\end{table}

\begin{table}[ht]
\caption{Performance and fairness comparison for personalized FL baselines.}
\begin{center}
\begin{subtable}[c]{.8\linewidth}
\caption{Performance Comparison}
\label{tab:pfl-acc}
\centering
\resizebox{0.8\linewidth}{!}{
\begin{tabular}{l|c|c}
            \toprule
Algorithm & Setting (pFL \#1) & Setting (pFL \#2)\\
            \midrule
FedAvg + FT~\cite{jiang2019improving}  & $69.26\pm0.42$ & $49.03\pm0.40$  \\
            \midrule         
LG-FedAvg~\cite{liang2020think}  & $54.03\pm2.16$ & $25.30\pm0.50$\\
            \midrule  
PerFedAvg~\cite{fallah2020personalized}  & $76.03\pm0.31$ & $2.29\pm0.07$ \\
            \midrule
IFCA~\cite{ghosh2020efficient}  & $64.84\pm3.41$ & $46.73\pm1.82$ \\
            \midrule
Ditto~\cite{li2021ditto}  & $70.97\pm1.27$ & $48.16\pm3.25$ \\
            \midrule
FedPer~\cite{arivazhagan2019federated}  & $64.64\pm0.45$ & $42.87\pm1.66$ \\
            \midrule
FedRep~\cite{collins2021exploiting}  & $54.99\pm 3.16$ & $29.39\pm0.31$ \\
            \midrule
APFL~\cite{deng2020adaptive}  & $68.91\pm0.59$ & $55.18\pm0.65$ \\
            \midrule
pFedMe~\cite{t2020personalized}  & $10.00\pm0.98$ & $2.10\pm0.30$ \\
            \midrule
CFL~\cite{sattler2020clustered}  & $16.83\pm1.6$ & $1.52\pm0.17$ \\
            \midrule
SubFedAvg~\cite{vahidian2021personalized} & $69.95\pm1.34$ & $49.83\pm1.09$ \\
            \midrule
PACFL~\cite{vahidian2022efficient}  & $67.78\pm0.11$ & $50.11\pm1.10$ \\
            \bottomrule
\end{tabular}
}
\vspace{0.2cm}
\end{subtable}
\end{center}
\begin{center}
\begin{subtable}[c]{.8\linewidth}
\caption{Fairness Comparison}
\label{tab:pfl-fairness}
\centering
\resizebox{0.8\linewidth}{!}{
\begin{tabular}{l|c|c}
            \toprule
Algorithm & Setting (pFL \#1) & Setting (pFL \#2) \\
            \midrule
FedAvg + FT~\cite{jiang2019improving}  & $8.05\pm0.32$ & $5.40\pm0.77$  \\
            \midrule       
LG-FedAvg~\cite{liang2020think}  & $12.79\pm0.64$ & $5.11\pm0.38$\\
            \midrule  
PerFedAvg~\cite{fallah2020personalized}  & $7.32\pm0.27$ & $--$ \\
            \midrule 
IFCA~\cite{ghosh2020efficient}  & $10.56\pm2.75$ & $5.03\pm0.07$ \\
            \midrule  
Ditto~\cite{li2021ditto}  & $7.50\pm0.37$ & $4.10\pm0.73$ \\
            \midrule  
FedPer~\cite{arivazhagan2019federated}  & $7.64\pm0.59$ & $6.19\pm0.59$ \\
            \midrule  
FedRep~\cite{collins2021exploiting}  & $9.18\pm0.50$ & $5.19\pm0.73$ \\
            \midrule  
APFL~\cite{deng2020adaptive}  & $8.22\pm0.96$ & $4.60\pm1.15$ \\
            \midrule
pFedMe~\cite{t2020personalized}  & $--$ & $--$ \\
            \midrule
SubFedAvg~\cite{vahidian2021personalized} & $7.44\pm0.46$ & $4.66\pm0.56$ \\
            \midrule
CFL~\cite{sattler2020clustered} & $--$ & $--$ \\
            \midrule
PACFL~\cite{vahidian2022efficient} & $8.82\pm0.71$ & $4.68\pm0.42$ \\
            \bottomrule
\end{tabular}
}
\end{subtable}
\end{center}
\end{table}

\textbf{Performance comparison.} We use the evaluation metrics outlined in Section~\ref{def:gfl} to compare the performance results. \vspace{-0.1cm}
\begin{itemize}
    \item \textbf{Global FL.} Table~\ref{tab:gfl-acc} shows the performance results of 6 different global FL methods. As it is noticeable Scaffold and FedProx \textcolor{red}{have} given the best results in settings (gfl \#1) and (gfl \#2), respectively and FedDF has given the worst results in both settings. FedAvg which is the simplest method has appeared to be competitive in both settings and even better than 4 other algorithms in setting (gfl \#2).
    \item \textbf{Personalized FL.} In Table~\ref{tab:pfl-acc}, we present the performance results of 12 different personalized FL methods. Similar to global FL methods (Table~\ref{tab:gfl-acc}), we observe that each method performs differently in different settings. No single method consistently achieves the best results across all settings. For example, PerFedAvg performs well in setting (pfl \#1), but poorly in setting (pfl \#2). Additionally, CFL and pFedMe perform poorly in both settings. On the other hand, FedAvg + FT, the simplest baseline, performs fairly well in both settings and is competitive or even superior to several other methods.
\end{itemize} 

\textbf{Fairness comparison.} Fairness is another important aspect \textcolor{red}{of the} personalized FL approach. We use the fairness metric mentioned in~\cite{li2021ditto, mohri2019agnostic} which is the standard deviation of final local test accuracies. Table~\ref{tab:pfl-fairness} shows the fairness comparison of the methods. SubFedAvg and Ditto have achieved the best fairness results in (pfl \#1) and (pfl \#2) settings, respectively. FedAvg + FT also demonstrated competitive fairness results in both settings. For algorithms having poor results we did not report the fairness results as they would not be meaningful.

\textbf{Generalization to newcomers.} To evaluate the generalization capabilities of personalized FL methods to newcomers, we reserve 20\% of the clients as newcomers and train the FL models using the remaining 80\% clients. While the adaptation process for many methods is not explicitly clear, we follow the same procedure as in~\cite{marfoq2021federated, vahidian2022efficient} and allow the newcomers to receive the trained FL model and perform local fine-tuning. For methods like PACFL that have a different adaptation strategy, we follow their original approach. Table~\ref{tab:pfl-newcommers} shows the results of this evaluation.

% \textbf{Suggested settings.} In the previous section, we analyzed effects of different experimental design variables in FL. Now, we are ready to propose a unified FL settings  to compare algorithms. To have a comprehensive and fair comparison which reflects how all the baselines perform under different FL variables, we propose the following settings: 
% \begin{itemize}
%     \item Personalized FL: 
%     \begin{itemize}
%         \item Accuracy: \{Dir(0.1), Skew(3)\}, Epoch=\{10, 5\}, Comm=100, Clients=100, sample rate = 0.1
%         \item Communication: \{Dir(0.1), Skew(3)\}, Epoch=1, Comm=250, Clients=100, sample rate = 0.1
%         \item Sample rate: \{Dir(0.1), Skew(3)\}, Epoch=5, Comm=100, Clients=100, sample rate = 0.2, 0.5
%     \end{itemize}
%     \item Global FL
%     \begin{itemize}
%         \item Accuracy: \{Dir(0.5), Skew(9)\}, Epoch=\{10, 5\}, Comm=100, Clients=100, sample rate = 0.1
%         \item Communication: \{Dir(0.5), Skew(9)\}, Epoch=1, Comm=250, Clients=100, sample rate = 0.1
%         \item Sample rate: \{Dir(0.5), Skew(9)\}, Epoch=5, Comm=100, Clients=100, sample rate = 0.2, 0.5
%     \end{itemize}
% \end{itemize}
% Since, (dataset, architecture) is also important in the experiments, we suggest (CIFAR-10, LeNet-5) and (CIFAR-100, Resnet-9) for our study.

% \begin{table}
% \begin{center}
% \caption{pFL accuracy comparison}
% \label{tab:pfl-acc}
% \centering
% \resizebox{0.5\linewidth}{!}{
% \begin{tabular}{l|cc|cc|c}
%             \toprule
% \multirow{2}{*}{Algorithm} & \multicolumn{2}{c|}{(CIFAR-10, LeNet-5, Skew(30\%))} & \multicolumn{2}{c|}{(CIFAR-100, ResNet-9, Dir(0.1))} &  \multirow{2}{*}{Rank}\\
%           \cmidrule{2-3} \cmidrule{4-5}
%  & Epoch=5 & Epoch=10 & Epoch=5 & Epoch=10 \\
%             \midrule
% FedAvg+  & $XX\pm XX$ & $XX\pm XX$ & $XX\pm XX$ & $XX\pm XX$\\
%             \midrule         
% FedAvg+  & $XX\pm XX$ & $XX\pm XX$ & $XX\pm XX$ & $XX\pm XX$\\
%             \midrule  
% FedAvg+  & $XX\pm XX$ & $XX\pm XX$ & $XX\pm XX$ & $XX\pm XX$\\
%             \midrule  
% FedAvg+  & $XX\pm XX$ & $XX\pm XX$ & $XX\pm XX$ & $XX\pm XX$\\
%             \midrule  
% FedAvg+  & $XX\pm XX$ & $XX\pm XX$ & $XX\pm XX$ & $XX\pm XX$\\
%             \midrule  
% FedAvg+  & $XX\pm XX$ & $XX\pm XX$ & $XX\pm XX$ & $XX\pm XX$\\
%             \midrule  
% FedAvg+  & $XX\pm XX$ & $XX\pm XX$ & $XX\pm XX$ & $XX\pm XX$\\
%             \midrule  
% FedAvg+  & $XX\pm XX$ & $XX\pm XX$ & $XX\pm XX$ & $XX\pm XX$\\
%             \midrule  
% FedAvg+  & $XX\pm XX$ & $XX\pm XX$ & $XX\pm XX$ & $XX\pm XX$\\
%             \midrule  
% \end{tabular}
% }
% \end{center}
% \end{table}

\textbf{Discussion.} The experimental comparison between several SOTA methods for each FL approach outlined in this section highlights the progress that has been made in FL research. While we can see that many methods have improved compared to the simple FedAvg and FedAvg + FT baselines for global and personalized FL approaches, respectively, there are also some limitations that are worth noting:
\begin{itemize}
    \item There is no method that consistently performs the best across all experimental settings. Furthermore, for the personalized FL approach, a method may achieve good fairness results but lack generalization to newcomers. Thus, evaluating FL methods from different perspectives and developing algorithms that can provide a better trade-off is crucial.
    \item Despite the existence of numerous works for each FL approach, the performance of the simple methods of FedAvg and FedAvg + FT are still competitive or even better compared to several methods. Thus, there is a need for new methods that can achieve consistent improvements across different types of statistical heterogeneity.
    \item Fairness and generalization to newcomers are two important aspects of the personalized FL approach that are often overlooked in the literature and \textcolor{red}{are} only focused on performance improvement. Therefore, it is crucial to consider these aspects in addition to performance improvement when designing new personalized FL methods.
\end{itemize}

\begin{minipage}[c][][b]{0.5\textwidth}
\centering
\captionof{table}{gFL accuracy comparison}
\label{tab:gfl-acc}
\resizebox{0.6\linewidth}{!}{
\begin{tabular}{l|c|c}
            \toprule
Algorithm & Setting (gFL \#1) & Setting (gFL \#2)\\
            \midrule
FedAvg~\cite{mcmahan2017communication}  & $44.89\pm0.20$ & $56.47\pm0.57$\\
            \midrule
FedProx~\cite{li2020federated}  & $46.01\pm0.46$ & $56.85\pm0.36$\\
            \midrule
FedNova~\cite{wang2020tackling}  & $44.59\pm0.60$ & $53.20\pm0.33$\\
            \midrule
Scaffold~\cite{karimireddy2020scaffold}  & $56.85\pm1.06$ & $51.71\pm0.65$\\
            \midrule
FedDF~\cite{lin2020ensemble} & $27.43\pm 2.32$ & $30.24\pm0.26$ \\
            \midrule
MOON~\cite{li2021model} & $45.60\pm0.31$ & $50.23\pm0.55$ \\
            \bottomrule
\end{tabular}
}
\vspace{0.2cm}
\end{minipage}
\begin{minipage}[c][][b]{0.5\textwidth}
\centering
\captionof{table}{pFL generalization to newcomers}
\label{tab:pfl-newcommers}
\resizebox{0.6\linewidth}{!}{
\begin{tabular}{l|c|c}
            \toprule
Algorithm & Setting (pFL \#1) & Setting (pFL \#2)\\
            \midrule
FedAvg + FT~\cite{jiang2019improving} & $64.19\pm4.64$ & $37.14\pm0.43$ \\
            \midrule
LG-FedAvg~\cite{liang2020think}   & $40.39\pm17.98$ & $21.22\pm2.56$ \\
            \midrule  
PerFedAvg~\cite{fallah2020personalized}   & $74.97\pm1.10$ & $2.22\pm0.20$ \\
            \midrule  
IFCA~\cite{ghosh2020efficient}   & $62.64\pm1.03$ & $14.84\pm1.86$ \\
            \midrule
Ditto~\cite{li2021ditto}   & $62.55\pm3.10$ & $38.96\pm0.26$ \\
            \midrule
FedPer~\cite{arivazhagan2019federated}   & $65.3\pm2.41$ & $35.66\pm1.61$ \\
            \midrule
FedRep~\cite{collins2021exploiting}   & $64.50\pm0.62$ & $23.85\pm1.49$ \\
            \midrule
APFL~\cite{deng2020adaptive}   & $66.38\pm1.25$ & $39.52\pm1.11$ \\
            \midrule
SubFedAvg~\cite{vahidian2021personalized} & $63.54\pm1.42$ & $30.81\pm1.28$ \\
            \midrule
PACFL~\cite{vahidian2022efficient} & $68.54\pm1.33$ & $36.50\pm1.42$ \\
            \bottomrule
\end{tabular}
}
\end{minipage}

%\subsection{Computation Cost}
%\subsection{Communication Cost}

\begin{comment}
\begin{minipage}[c]{0.5\textwidth}
\centering
\begin{table}
\captionof{table}{gFL accuracy comparison}
\label{tab:gfl-acc}
\centering
\resizebox{0.5\linewidth}{!}{
\begin{tabular}{l|c|c}
            \toprule
Algorithm & Setting (gFL \#1) & Setting (gFL \#1)\\
            \midrule
FedAvg~\cite{mcmahan2017communication}  & $44.89\pm0.20$ & $56.47\pm0.57$\\
            \midrule
FedProx~\cite{li2020federated}  & $46.01\pm0.46$ & $56.85\pm0.36$\\
            \midrule
FedNova~\cite{wang2020tackling}  & $44.59\pm0.60$ & $53.20\pm0.33$\\
            \midrule
Scaffold~\cite{karimireddy2020scaffold}  & $56.85\pm1.06$ & $51.71\pm0.65$\\
            \midrule
FedDF~\cite{lin2020ensemble} & $27.43\pm 2.32$ & $30.24\pm0.26$ \\
            \midrule
MOON~\cite{li2021model} & $45.60\pm0.31$ & $50.23\pm0.55$ \\
            \bottomrule
\end{tabular}
}
\end{table}
\begin{table}
\captionof{table}{pFL generalization to newcomers comparison}
\label{tab:pfl-newcommers}
\centering
\resizebox{0.5\linewidth}{!}{
\begin{tabular}{l|c|c}
            \toprule
Algorithm & Setting (pFL \#1) & Setting (pFL \#2)\\
            \midrule
FedAvg + FT~\cite{jiang2019improving} & $64.19\pm4.64$ & $37.14\pm0.43$ \\
            \midrule         
LG-FedAvg~\cite{liang2020think}   & $40.39\pm17.98$ & $21.22\pm2.56$ \\
            \midrule  
PerFedAvg~\cite{fallah2020personalized}   & $74.97\pm1.10$ & $2.22\pm0.20$ \\
            \midrule  
IFCA~\cite{ghosh2020efficient}   & $62.64\pm1.03$ & $14.84\pm1.86$ \\
            \midrule
Ditto~\cite{li2021ditto}   & $62.55\pm3.10$ & $38.96\pm0.26$ \\
            \midrule
FedPer~\cite{arivazhagan2019federated}   & $65.3\pm2.41$ & $35.66\pm1.61$ \\
            \midrule
FedRep~\cite{collins2021exploiting}   & $XX\pm XX$ & $XX\pm XX$ \\
            \midrule
APFL~\cite{deng2020adaptive}   & $XX\pm XX$ & $XX\pm XX$ \\
            \midrule
SubFedAvg~\cite{vahidian2021personalized} & $XX\pm XX$ & $XX\pm XX$ \\
            \midrule
PACFL~\cite{vahidian2022efficient} & $XX\pm XX$ & $XX\pm XX$ \\
            \bottomrule
\end{tabular}
}
\end{table}
\end{minipage}

\end{comment}
\section{Conclusion and Future Work}
In this work, I design corruption-robust algorithms for the Lipschitz contextual search problem. I present the \emph{agnostic checking} technique and demonstrate its effectiveness in designing corruption-robust algorithms. There are several open problems for future research. First, in the algorithm I propose for pricing loss, the schedule for agnostic checks is fixed upfront. Can the learner design an adaptive checking schedule for the pricing loss? Second, this work assumes the learner has knowledge of the Lipschitz constant $L$. Can the learner design efficient no-regret algorithms without knowledge of $L$? 
%\section{Algorithms}


\begin{algorithm}[H]
\caption{Federated Learning}
\label{alg:FedAvg}

% \DontPrintSemicolon
% \SetCommentSty{mycommfont}
% \SetAlgoLined
% \SetKwInOut{Req}{Require}
% \SetKwFunction{FClient}{ClientUpdate}
% \SetKwProg{Fn}{}{:}{}

\begin{algorithmic}[1]
\REQUIRE number of clients ($N$), sampling rate ($C\in(0,1]$), number of communication rounds ($T$), local dataset of client $k$ ($D_k$), number of local epoch ($E$), number of local batch size ($B$), learning rate ($\eta$).
%\SetKwProg{Fn}{Server Executes}{:}{}
\STATE \textbf{Server Executes:}
\STATE Initialize the server model with $\btheta_g^0$ \;
\FOR {each round $t = 0, 1, \ldots, T-1$}
\STATE $m \leftarrow {\rm{max}}(C \cdot N,1)$ \;
\STATE $S_t \leftarrow $(random set of m clients) \; %\tcp*{set of $m$ available clients}
\FOR {each client $k \in S_t$ \rm{\textbf{in parallel}}}
\STATE $\btheta^{t+1}_{k}\leftarrow {\mathtt{ClientUpdate}}(k; \btheta^t_{g} )$ \; 
%$\theta^{t+1}_{g}=\sum_{k \in S_t }{|D_{k}|\theta^{t+1}_{k}}  /\sum_{k \in S_t }{|D_{k}|}$
\ENDFOR
\STATE $\btheta^{t+1}_{g}=\mathtt{ModelFusion}(\{\btheta_k \}_{k \in S_t})$
\ENDFOR
\vspace{1mm}
\FUNCTION {$\mathtt{ClientUpdate}(k, \btheta^t_{g})$}
\STATE $\btheta^{t}_k \leftarrow \btheta^t_{g}$ \;
\STATE $\mathcal{B} \leftarrow$ ({randomly splitting $D_k^{train}$ into batches of Size $B$})\;
\FOR {each local epoch $ \in \{1, \ldots, E\}$}
\FOR {each batch $\mathbf{b} \in \mathcal{B}$}
\STATE $\btheta^{t}_k \leftarrow \btheta^{t}_k - \eta \nabla_{\btheta^{t}_k} \ell(f_k(\bx; \btheta^{t}_k), y)$\;
\ENDFOR
\ENDFOR
\STATE $\btheta^{t+1}_k \leftarrow \btheta^{t}_k$\;
\ENDFUNCTION
% \Fn{\FClient{$k, \btheta^t_{g}$}}{
%     \STATE $\btheta^{t}_k \leftarrow \btheta^t_{g}$ \;
%     \STATE $\mathcal{B} \leftarrow$ ({randomly splitting $D_k$ into batches of Size $B$})\;
%     \For {each local epoch $i = 1, \dots, E$} {
%         \For {each batch $\mathbf{b} \in \mathcal{B}$} {
%             \STATE $\btheta^{t}_k \leftarrow \btheta^{t}_k - \eta \nabla_{\btheta^{t}_k} f_k(\btheta^{t}_k; \mathbf{b})$\;
%         }
%     }
%     \STATE $\btheta^{t+1}_k \leftarrow \btheta^{t}_k$\;
%     \KwRet $\btheta^{t+1}_{k}$\;
% }
\end{algorithmic}
\end{algorithm}


% \begin{algorithm}[H]
% \caption{FedAvg}
% \label{alg:FedAvg}

% % \DontPrintSemicolon
% % \SetCommentSty{mycommfont}
% % \SetAlgoLined
% % \SetKwInOut{Req}{Require}
% % \SetKwFunction{FClient}{ClientUpdate}
% % \SetKwProg{Fn}{}{:}{}

% \begin{algorithmic}[1]
% \STATE \textbf{Require:} number of clients ($N$), sampling rate ($C\in(0,1]$), number of communication rounds ($T$), local dataset of client $k$ ($D_k$), number of local epoch ($E$), number of local batch size ($B$), learning rate ($\eta$).
% %\SetKwProg{Fn}{Server Executes}{:}{}
% \STATE \textbf{Server Executes:}
% \STATE $\quad$ Initialize the server model with $\btheta_g^0$ \;
% \FOR {each round $t = 0, 1, \dots, T-1$}
% \STATE $\quad$ $m \leftarrow {\rm{max}}(C \cdot N,1)$ \;
% \STATE $\quad$ $S_t \leftarrow $(random set of m clients) \; %\tcp*{set of $m$ available clients}
% $\quad$ \FOR {each client $k \in S_t$ \rm{\textbf{in parallel}}}
% \STATE $\quad$ $\btheta^{t+1}_{k}\leftarrow {\mathtt{ClientUpdate}}(k; \btheta^t_{g} )$ \; 
% %$\theta^{t+1}_{g}=\sum_{k \in S_t }{|D_{k}|\theta^{t+1}_{k}}  /\sum_{k \in S_t }{|D_{k}|}$
% \ENDFOR
% \STATE $\quad$ $\btheta^{t+1}_{g}=\mathtt{ModelFusion}(\{\btheta_k \}_{k \in S_t})$
% \ENDFOR
% \vspace{1mm}
% \FUNCTION {$\mathtt{ClientUpdate}(k, \btheta^t_{g})$}
% \STATE H
% \ENDFUNCTION
% % \Fn{\FClient{$k, \btheta^t_{g}$}}{
% %     \STATE $\btheta^{t}_k \leftarrow \btheta^t_{g}$ \;
% %     \STATE $\mathcal{B} \leftarrow$ ({randomly splitting $D_k$ into batches of Size $B$})\;
% %     \For {each local epoch $i = 1, \dots, E$} {
% %         \For {each batch $\mathbf{b} \in \mathcal{B}$} {
% %             \STATE $\btheta^{t}_k \leftarrow \btheta^{t}_k - \eta \nabla_{\btheta^{t}_k} f_k(\btheta^{t}_k; \mathbf{b})$\;
% %         }
% %     }
% %     \STATE $\btheta^{t+1}_k \leftarrow \btheta^{t}_k$\;
% %     \KwRet $\btheta^{t+1}_{k}$\;
% % }
% \end{algorithmic}
% \end{algorithm}

% \begin{algorithm}[H]
% \caption{FedProx}
% \label{alg:FedProx}

% \DontPrintSemicolon
% \SetCommentSty{mycommfont}
% \SetAlgoLined
% \SetKwInOut{Req}{Require}
% \SetKwFunction{FClient}{ClientUpdate}

% \Req{number of clients ($N$), sampling rate ($C\in(0,1]$), number of communication rounds ($T$), 
% local dataset of client $k$ ($D_k$), number of local epoch ($E$), number of local batch size ($B$), learning rate ($\eta$), FedProx parameter ($\mu$)\;}
% \SetKwProg{Fn}{Server Executes}{:}{}
% \Fn{}{
% Initialize the server model with $\theta_g^0$ \;
% \For {each round $t = 0, 1, \dots, T-1$} {
%  $m \leftarrow {\rm{max}}(C \cdot N,1)$\;
%  $S_t \leftarrow $(random set of m clients)\; %\tcp*{set of $m$ available clients}
% \For  {each client $k \in S_t$ \rm{\textbf{in parallel}}}
% { 
% $\theta^{t+1}_{k}\leftarrow {\mathtt{ClientUpdate}}(k; \theta^t_{g} )$ \; 
% }

% $\theta^{t+1}_{g}=\sum_{k \in S_t }{|D_{k}|\theta^{t+1}_{k}}  /\sum_{k \in S_t }{|D_{k}|}$
% }
% }
% %\Indm
% \SetKwProg{Fn}{}{:}{}
% \Fn{\FClient{$k, \theta^t_{g}$}}{
%      $\theta^{t}_k \leftarrow \theta^t_{g}$\;
%      $\mathcal{B} \leftarrow$ ({randomly splitting $D_k$ into batches of Size $B$})\;
%     \For {each local epoch $i = 1, \dots, E$} {
%         \For {each batch $\mathbf{b} \in \mathcal{B}$} 
%         {
%              $\theta^{t+1}_k \leftarrow \argmin_{\theta_k} \mathbb{E} \left[ f(\theta_k; \mathbf{b}) \right] + \frac{\mu}{2} ||\theta_k - \theta^t_k||^2$\;
%         }
%     }
%     $\theta^{t+1}_k \leftarrow \theta^{t}_k$\;
%     \KwRet $\theta^{t+1}_{k}$\;
% }
% \end{algorithm}



%\appendix
\section*{APPENDIX}
\appendices
\textbf{Organization.} We organize the supplementary materials as follow: 
\begin{itemize}
    \item In Section~\ref{app:sec-additional}, we present addition experimental results to complete our analysis in Section~\ref{study} of the main paper.
    \item In Section~\ref{app:recommendation}, we discuss an experimental checklist to facilitate an easier comparison of FL methods.
    \item In Section~\ref{sec:app-fedzoo}, we provide more details about the available algorithms, datasets, architectures and data partitionings in FedZoo-Bench.
    %\item In Section~\ref{sec:app-hyperparameters}, we brings the details of the hyperparameters we used for each method in our experiments of Section~\ref{sec:comparison}. 
\end{itemize}
\section{Additional Results}\label{app:sec-additional}
%In this section we present additional experimental results for completeness of our study.
\subsection{Globalization and Personalization Incentives} \label{appendix:incentives}
The additional results in this part complements the results discussed in Section~\ref{sec:heterogeneity-localepoch}. Comparing Figures~\ref{app:fig-incentive-dir} and~\ref{app:fig-incentive-skew} further corroborates our finding mentioned in Section~\ref{sec:heterogeneity-localepoch} that Non-IID Label Skew partitioning has higher incentive for personalization compared to the other type of heterogeneity. Moreover, increase of local epochs has incentivized personalization more for both types of heterogeneity.
% Figure environment removed

% Figure environment removed

\begin{comment}
\subsection{Experimental Settings}


\begin{table}[h]
\caption{Hyper-parameters used for each figure}
\centering
\resizebox{0.8\textwidth}{!}{
\begin{tabular}{c|cccccccccccc}
\toprule
Figure           & Dataset & Architecture & clients & sample rate & epochs & communication rounds & optimizer & learning rate & momentum  & partitioning & mics \\
\midrule
Figure 1 & XX & XX & XX & XX & XX & XX & XX & XX & XX & XX & XX \\
\midrule
Figure 1 & XX & XX & XX & XX & XX & XX & XX & XX & XX & XX & XX \\
\midrule
Figure 1 & XX & XX & XX & XX & XX & XX & XX & XX & XX & XX & XX \\
\midrule
Figure 1 & XX & XX & XX & XX & XX & XX & XX & XX & XX & XX & XX \\
\midrule

\bottomrule
\end{tabular}
}
\label{tab:hyperparameters-figs-tabs}
\end{table}
\end{comment}

\section{Experimental Checklist} \label{app:recommendation}
To facilitate an easier comparison between FL methods in future studies, we recommend the following checklist:
\begin{itemize}
    \item Make sure that the used experimental setting is meaningful and well-incentived for the considered FL approach. 
    \item State the exact setting including local epochs, sample rate, number of clients, type of data partitioning, level of heterogeneity, communication rounds, dataset, architecture, evaluation metrics, any pre-processing on the dataset, any learning rate scheduling if used, and initialization.
    \item Report the average results over at least $3$ independent and different runs.
    \item Mention the hyperparameters used to obtain the results.
\end{itemize}

\section{FedZoo-Bench}\label{sec:app-fedzoo}
We introduced FedZoo-Bench in Section~\ref{fedzoo}. In this section we provide more details about the available features in FedZoo-Bench. For more information on FedZoo-Bench's implementation and use cases for different settings, refer to the project's documentation at~\url{https://github.com/MMorafah/FedZoo-Bench}.
\subsection{Available Baselines}
\begin{itemize}
    \item \textbf{Global FL} (7 algorithms)
    \begin{itemize}
        \item FedAvg~\cite{mcmahan2017communication}
        \item FedProx~\cite{li2020federated}
        \item FedNova~\cite{wang2020tackling}
        \item Scaffold~\cite{karimireddy2020scaffold}
        \item FedDF~\cite{lin2020ensemble}
        \item MOON~\cite{li2021model}
        \item FedBN~\cite{li2021fedbn}
    \end{itemize}
    \item \textbf{Personalized FL} (15 algorithms)
    \begin{itemize}
        \item FedAvg + FT~\cite{jiang2019improving}
        \item LG-FedAVg~\cite{liang2020think}
        \item PerFedAvg~\cite{fallah2020personalized}
        \item FedPer~\cite{arivazhagan2019federated}
        \item FedRep~\cite{collins2021exploiting}
        \item Ditto~\cite{li2021ditto}
        \item APFL~\cite{deng2020adaptive}
        \item IFCA~\cite{ghosh2020efficient}
        \item SubFedAvg~\cite{vahidian2021personalized}
        \item pFedMe~\cite{t2020personalized}
        \item CFL~\cite{sattler2020clustered}
        \item PACFL~\cite{vahidian2022efficient}
        \item MTL~\cite{smith2017federated}
        \item FedEM~\cite{marfoq2021federated}
        \item FedFOMO~\cite{zhang2020personalized}
    \end{itemize}
\end{itemize}
Additionally, FedZoo-Bench can be easily used for other variations of FedAvg~\cite{reddi2020adaptive} and different choice of optimizers.

\subsection{Available Datasets}
\begin{itemize}
    \item MNIST~\cite{mnist}
    \item CIFAR-10~\cite{cifar}
    \item CFIAR-100
    \item USPS~\cite{usps}
    \item SVHN~\cite{svhn}
    \item CelebA~\cite{celeba}
    \item FMNIST~\cite{fmnist}
    \item FEMNIST~\cite{femnist}
    \item Tiny-ImageNet~\cite{tiny}
    \item STL-10~\cite{stl-10}
\end{itemize}

\subsection{Available Architectures}
\begin{itemize}
    \item MLP as in FedAvg \cite{mcmahan2017communication}
    % should we give a specific architecture of mlp?
    \item LeNet-5 \cite{lenet}
    \item ResNet Family \cite{resnet}
    \item ResNet-50
    % I think resnet-50 is more like part of resnet family than resnet-9. Maybe we should put resnet-9 here?
    \item VGG Family \cite{vgg}
\end{itemize}

\subsection{Available Data Partitionings}
\begin{itemize}
    \item IID
    \item Non-IID Label Dir~\cite{hsu2019measuring}
    \item Non-IID Label Skew~\cite{li2021federated}
    \item Non-IID Random Shard~\cite{mcmahan2017communication}
    \item Non-IID Quantity Skew~\cite{li2021federated}
\end{itemize}

\begin{comment}
\section{Hyperparameters}\label{sec:app-hyperparameters}
In this section, we detail the hyperparameters used for each algorithm for our experiments in Section~\ref{sec:comparison}.
\subsection{Architectures}
Tables \ref{tab:lenet5} and \ref{tab:resnet9} summarize LeNet-5 and ResNet-9 architectures we used in the experiments, respectively.
\subsection{Global FL}
\subsubsection{Setting (gFL \#1)}
We fix local batch size to $10$ and use SGD optimizer for all the baselines. Herein, we detail the hyperparameter used for each method:
\begin{itemize}
    \item FedAvg~\cite{mcmahan2017communication}
    \begin{itemize}
        \item $\mathtt{learning\_rate=0.01}$
        \item $\mathtt{momentum=0.9}$
    \end{itemize}
    \item FedProx~\cite{li2020federated}
    \begin{itemize}
        \item $\mathtt{learning\_rate=0.01}$
        \item $\mathtt{momentum=0.9}$
        \item $\mathtt{\mu=0.001}$
    \end{itemize}
    \item FedNova~\cite{wang2020tackling}
    \begin{itemize}
        \item $\mathtt{learning\_rate=0.01}$
        \item $\mathtt{momentum=0.9}$
    \end{itemize}
    \item Scaffold~\cite{karimireddy2020scaffold}
    \begin{itemize}
        \item $\mathtt{learning\_rate=0.001}$
        \item $\mathtt{momentum=0.9}$
    \end{itemize}
    \item MOON~\cite{li2021model}
    \begin{itemize}
        \item $\mathtt{learning\_rate=0.01}$
        \item $\mathtt{momentum=0.9}$
        \item $\mathtt{use\_project\_head=false}$
        \item $\mathtt{\tau=0.5}$
        \item $\mathtt{\mu=0.1}$
    \end{itemize}
    \item FedDF~\cite{lin2020ensemble}
    \begin{itemize}
        \item $\mathtt{learning\_rate=0.01}$
        \item $\mathtt{momentum=0.9}$
        \item $\mathtt{distillation\_dataset=CIFAR{-}100}$
        \item $\mathtt{distillation\_T=1}$
        \item $\mathtt{distillation\_epoch=1}$
        \item $\mathtt{distillation\_batch\_size=128}$
        \item $\mathtt{distillation\_optimizer=Adam}$
        \item $\mathtt{distillation\_learning\_rate=0.0001}$
    \end{itemize}
\end{itemize}
\subsubsection{Setting (gFL \#2)}
We fix local batch size to $64$ and use SGD optimizer for all the baselines. Herein, we detail the hyperparameter we used for each method:
\begin{itemize}
    \item FedAvg~\cite{mcmahan2017communication}
    \begin{itemize}
        \item $\mathtt{learning\_rate=0.01}$
        \item $\mathtt{momentum=0.9}$
    \end{itemize}
    \item FedProx~\cite{li2020federated}
    \begin{itemize}
        \item $\mathtt{learning\_rate=0.01}$
        \item $\mathtt{momentum=0.9}$
        \item $\mathtt{\mu=0.001}$
    \end{itemize}
    \item FedNova~\cite{wang2020tackling}
    \begin{itemize}
        \item $\mathtt{learning\_rate=0.01}$
        \item $\mathtt{momentum=0.9}$
    \end{itemize}
    \item Scaffold~\cite{karimireddy2020scaffold}
    \begin{itemize}
        \item $\mathtt{learning\_rate=0.01}$
        \item $\mathtt{momentum=0.9}$
    \end{itemize}
    \item MOON~\cite{li2021model}
    \begin{itemize}
        \item $\mathtt{learning\_rate=0.01}$
        \item $\mathtt{momentum=0.9}$
        \item $\mathtt{use\_project\_head=false}$
        \item $\mathtt{\tau=0.5}$
        \item $\mathtt{\mu=0.1}$
    \end{itemize}
    \item FedDF~\cite{lin2020ensemble}
    \begin{itemize}
        \item $\mathtt{learning\_rate=0.01}$
        \item $\mathtt{momentum=0.9}$
        \item $\mathtt{distillation\_dataset=tiny\_imagenet}$
        \item $\mathtt{distillation\_T=1}$
        \item $\mathtt{distillation\_epoch=1}$
        \item $\mathtt{distillation\_batch\_size=128}$
        \item $\mathtt{distillation\_optimizer=Adam}$
        \item $\mathtt{distillation\_learning\_rate=0.0001}$
    \end{itemize}
\end{itemize}
\subsection{Personalized FL}
\subsubsection{Setting (pFL \#1)}
We fix local batch size to $10$ and use SGD optimizer for all the baselines. Herein, we detail the hyperparameter we used for each method:
\begin{itemize}
    \item FedAvg + FT~\cite{jiang2019improving}
    \begin{itemize}
        \item $\mathtt{learning\_rate=0.01}$
        \item $\mathtt{momentum=0.9}$
        \item $\mathtt{ft\_epoch=20}$
    \end{itemize}
    \item LG-FedAvg~\cite{liang2020think}
    \begin{itemize}
        \item $\mathtt{learning\_rate=0.01}$
        \item $\mathtt{momentum=0.9}$
        \item $\mathtt{global\_layers=last\_two\_layers}$
    \end{itemize}
    \item PerFedAvg~\cite{fallah2020personalized}
    \begin{itemize}
        \item $\mathtt{learning\_rate=0.01}$
        \item $\mathtt{momentum=0.9}$
        \item $\mathtt{\beta=0.001}$
    \end{itemize}
    \item FedPer~\cite{arivazhagan2019federated}
    \begin{itemize}
        \item $\mathtt{learning\_rate=0.01}$
        \item $\mathtt{momentum=0.9}$
        \item $\mathtt{global\_layers=except\_last\_two\_layers}$
    \end{itemize}
    \item FedRep~\cite{collins2021exploiting}
    \begin{itemize}
        \item $\mathtt{learning\_rate=0.01}$
        \item $\mathtt{momentum=0.9}$
        \item $\mathtt{global\_layers=except\_last\_two\_layers}$
    \end{itemize}
    \item Ditto~\cite{li2021ditto}
    \begin{itemize}
        \item $\mathtt{learning\_rate=0.01}$
        \item $\mathtt{momentum=0.9}$
        \item $\mathtt{\lambda=0.8}$
    \end{itemize}
    \item APFL~\cite{deng2020adaptive}
    \begin{itemize}
        \item $\mathtt{learning\_rate=0.01}$
        \item $\mathtt{momentum=0.9}$
        \item $\mathtt{\alpha=0.75}$
    \end{itemize}
    \item IFCA~\cite{ghosh2020efficient}
    \begin{itemize}
        \item $\mathtt{learning\_rate=0.01}$
        \item $\mathtt{momentum=0.9}$
        \item $\mathtt{nclusters=2}$
    \end{itemize}
    \item SubFedAvg~\cite{vahidian2021personalized}
    \begin{itemize}
        \item $\mathtt{learning\_rate=0.01}$
        \item $\mathtt{momentum=0.9}$
        \item $\mathtt{pruning\_percent=5}$
        \item $\mathtt{pruning\_target=50}$
        \item $\mathtt{dist\_thresh=0.0001}$
        \item $\mathtt{acc\_thresh=55}$
    \end{itemize}
    \item pFedMe~\cite{t2020personalized}
    \begin{itemize}
        \item $\mathtt{learning\_rate=0.01}$
        \item $\mathtt{momentum=0.9}$
        \item $\mathtt{K=5}$
        \item $\mathtt{\lambda=15}$
        \item $\mathtt{personal\_lr=0.09}$
    \end{itemize}
    \item CFL~\cite{sattler2020clustered}
    \begin{itemize}
        \item $\mathtt{learning\_rate=0.01}$
        \item $\mathtt{momentum=0.9}$
        \item $\mathtt{\epsilon_1=0.4}$
        \item $\mathtt{\epsilon_2=1.6}$
    \end{itemize}
    \item PACFL~\cite{vahidian2022efficient}
    \begin{itemize}
        \item $\mathtt{learning\_rate=0.01}$
        \item $\mathtt{momentum=0.9}$
    \end{itemize}
\end{itemize}

\subsubsection{Setting (pFL \#2)}
We fix local batch size to $10$ and use SGD optimizer for all the baselines. Herein, we detail the hyperparameter we used for each method:
\begin{itemize}
    \item FedAvg + FT~\cite{jiang2019improving}
    \begin{itemize}
        \item $\mathtt{learning\_rate=0.01}$
        \item $\mathtt{momentum=0.9}$
        \item $\mathtt{ft\_epoch=20}$
    \end{itemize}
    \item LG-FedAvg~\cite{liang2020think}
    \begin{itemize}
        \item $\mathtt{learning\_rate=0.01}$
        \item $\mathtt{momentum=0.9}$
        \item $\mathtt{global\_layers=last\_two\_layers}$
    \end{itemize}
    \item PerFedAvg~\cite{fallah2020personalized}
    \begin{itemize}
        \item $\mathtt{learning\_rate=0.01}$
        \item $\mathtt{momentum=0.9}$
        \item $\mathtt{\beta=0.001}$
    \end{itemize}
    \item FedPer~\cite{arivazhagan2019federated}
    \begin{itemize}
        \item $\mathtt{learning\_rate=0.01}$
        \item $\mathtt{momentum=0.9}$
        \item $\mathtt{global\_layers=except\_last\_two\_layers}$
    \end{itemize}
    \item FedRep~\cite{collins2021exploiting}
    \begin{itemize}
        \item $\mathtt{learning\_rate=0.01}$
        \item $\mathtt{momentum=0.9}$
        \item $\mathtt{global\_layers=except\_last\_two\_layers}$
    \end{itemize}
    \item Ditto~\cite{li2021ditto}
    \begin{itemize}
        \item $\mathtt{learning\_rate=0.01}$
        \item $\mathtt{momentum=0.9}$
        \item $\mathtt{\lambda=0.8}$
    \end{itemize}
    \item APFL~\cite{deng2020adaptive}
    \begin{itemize}
        \item $\mathtt{learning\_rate=0.01}$
        \item $\mathtt{momentum=0.9}$
        \item $\mathtt{\alpha=0.75}$
    \end{itemize}
    \item IFCA~\cite{ghosh2020efficient}
    \begin{itemize}
        \item $\mathtt{learning\_rate=0.01}$
        \item $\mathtt{momentum=0.9}$
        \item $\mathtt{nclusters=2}$
    \end{itemize}
    \item SubFedAvg~\cite{vahidian2021personalized}
    \begin{itemize}
        \item $\mathtt{learning\_rate=0.01}$
        \item $\mathtt{momentum=0.9}$
        \item $\mathtt{pruning\_percent=5}$
        \item $\mathtt{pruning\_target=50}$
        \item $\mathtt{dist\_thresh=0.0001}$
        \item $\mathtt{acc\_thresh=55}$
    \end{itemize}
    \item pFedMe~\cite{t2020personalized}
    \begin{itemize}
        \item $\mathtt{learning\_rate=0.01}$
        \item $\mathtt{momentum=0.9}$
        \item $\mathtt{K=5}$
        \item $\mathtt{\lambda=15}$
        \item $\mathtt{personal\_lr=0.09}$
    \end{itemize}
    \item CFL~\cite{sattler2020clustered}
    \begin{itemize}
        \item $\mathtt{learning\_rate=0.01}$
        \item $\mathtt{momentum=0.9}$
        \item $\mathtt{\epsilon_1=0.4}$
        \item $\mathtt{\epsilon_2=1.6}$
    \end{itemize}
    \item PACFL~\cite{vahidian2022efficient}
    \begin{itemize}
        \item $\mathtt{learning\_rate=0.01}$
        \item $\mathtt{momentum=0.9}$
    \end{itemize}
\end{itemize}

 \begin{table}[htbp]
\footnotesize
\centering
\caption{The details for LeNet-5 architecture.}
\begin{tabular}{l|l}
\toprule
\multicolumn{1}{l|}{\textbf{Layer}} & \multicolumn{1}{c}{\textbf{Details}} \\
\midrule
\multirow{3}{*}{layer 1} & Conv2d(i=3, o=6, k=(5, 5), s=(1, 1)) \\
                         & ReLU() \\
                         & MaxPool2d(k=(2, 2)) \\
\midrule
\multirow{3}{*}{layer 2} & Conv2d(i=6, o=16, k=(5, 5), s=(1, 1)) \\
                         & ReLU() \\
                         & MaxPool2d(k=(2, 2)) \\
\midrule
\multirow{2}{*}{layer 3} & Linear(i=400 (256 for FMNIST), o=120) \\
                         & ReLU() \\
\midrule
\multirow{2}{*}{layer 4} & Linear(i=120, o=84) \\
                         & ReLU() \\
\midrule
layer 5    & Linear(i=84, o=10 (100 for CIFAR-100, and 40 for Mix-4)) \\
\bottomrule
\end{tabular}
\label{tab:lenet5}
\end{table}

\begin{table}[htbp]
\footnotesize
\centering
\caption{The details for ResNet-9 architecture.}
\resizebox{0.7\linewidth}{!}{
\begin{tabular}{l|l|l}
\toprule
\multicolumn{1}{l|}{\textbf{Block}} & \multicolumn{1}{c|}{\textbf{Details}} & \multicolumn{1}{l}{\textbf{Input}} \\
\midrule
\multirow{3}{*}{block 1}    & Conv2d(i=3, o=64, k=(3, 3), s=(1, 1))    & \multirow{3}{*}{image} \\
                            & BatchNorm(64)                    & \\
                            & ReLU()                                   & \\
\midrule
\multirow{4}{*}{block 2}    & Conv2d(i=64, o=128, k=(3, 3), s=(1, 1))  & \multirow{4}{*}{block 1} \\
                            & BatchNorm(128)                   & \\
                            & ReLU()                                   & \\
                            & MaxPool2d(k=(2, 2))                      & \\
\midrule
\multirow{6}{*}{block 3}    & Conv2d(i=128, o=128, k=(3, 3), s=(1, 1)) & \multirow{6}{*}{block 2} \\
                            & BatchNorm(128)                   & \\
                            & ReLU()                                   & \\
                            & Conv2d(i=128, o=128, k=(3, 3), s=(1, 1)) & \\
                            & BatchNorm(128)                   & \\
                            & ReLU() \\
\midrule
\multirow{4}{*}{block 4}    & Conv2d(i=128, o=256, k=(3, 3), s=(1, 1)) & \\ %\multirow{4}{*}{block 2 + block 3} \\
                            & BatchNorm(256)                   & block 2 + \\
                            & ReLU()                                   & block 3 \\
                            & MaxPool2d(k=(2, 2))                      & \\
\midrule
\multirow{4}{*}{block 5}    & Conv2d(i=256, o=512, k=(3, 3), s=(1, 1)) & \multirow{4}{*}{block 4} \\
                            & BatchNorm(512)                   & \\
                            & ReLU()                                   & \\
                            & MaxPool2d(k=(2, 2))                      & \\
\midrule
\multirow{6}{*}{block 6}    & Conv2d(i=512, o=512, k=(3, 3), s=(1, 1)) & \multirow{6}{*}{block 5} \\
                            & BatchNorm(512)                   & \\
                            & ReLU()                                   & \\
                            & Conv2d(i=512, o=512, k=(3, 3), s=(1, 1)) & \\
                            & BatchNorm(512)                   & \\
                            & ReLU() \\
\midrule
\multirow{2}{*}{classifier} & MaxPool2d(k=(4, 4))                      & block 4 + \\ %\multirow{2}{*}{block 4 + block 5} \\
                            & Linear(i=512, o=100)                     & block 5\\
\bottomrule
\end{tabular}
}
\label{tab:resnet9}
\end{table}

\end{comment}

\newpage
\clearpage
%\section*{References}
%\bibliography{Mahdi-Fed}
\bibliography{Non_IID, extra}
%\bibliography{extra}
% \newpage
% \clearpage

\begin{IEEEbiography}
    [{% Figure removed}]{Mahdi Morafah}
received the BS in Electrical Engineering from the Tehran Polytechnic University, Tehran, Iran, in 2019, and the MS degree in Electrical and Computer Engineering from the University of California, San Diego, in 2021. He is currently pursuing the PhD degree in Electrical and Computer Engineering at the University of California, San Diego. His research interests include Machine Learning and Deep Learning, Distributed Learning, Federated Learning, Continual Learning, and Optimization. 
\end{IEEEbiography}

\begin{IEEEbiography}
    [{% Figure removed}]{Weijia Wang}
received the B.S. degree in Electrical Engineering from Zhejiang University, Hangzhou, China, in 2016 and the M.S. degree in Electrical and Computer Engineering from the University of California, San Diego, in 2018. He is currently pursuing the Ph.D. degree in Electrical and Computer Engineering at the University of California, San Diego. His research interest is machine learning and deep learning, including the compression and acceleration of deep convolutional neural networks, algorithms of meta-learning and federated learning, and explainable artificial intelligence.
\end{IEEEbiography}

\begin{IEEEbiography}
    [{% Figure removed}]{Bill Lin}
received the BS, MS, and the PhD degrees in electrical engineering and computer sciences from the University of California, Berkeley in 1985, 1988, and 1991, respectively. He is a Professor in Electrical and Computer Engineering at the University of California, San Diego, where he is actively involved with the Center for Wireless Communications (CWC), the Center for Networked Systems (CNS), and the Qualcomm Institute in industry-sponsored research efforts. His research has led to over 200 journal and conference publications, including a number of Best Paper awards and nominations. He also holds 5 awarded patents. He has served as the General Chair and on the executive and technical program committee of many IEEE and ACM conferences, and he has served as an Associate or Guest Editor for several IEEE and ACM journals as well.
\end{IEEEbiography}

\end{document}
