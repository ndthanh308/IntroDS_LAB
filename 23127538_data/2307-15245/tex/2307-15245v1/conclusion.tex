\section{Conclusion and Future Works}\label{sec:conclusion}
In this paper, we present a thorough examination of key variables that influence the success of FL experiments. Firstly, we provide new insights and analysis of the FL-specific variables in relation to each other and performance results by running several experiments. We then use our analysis to identify recommendations and best practices for a meaningful and well-incentivized FL experimental design. We have also developed FedZoo-Bench, an open-source library based on PyTorch that provides a comprehensive set of standardized and customizable features, different evaluation metrics, and implementation of 22 SOTA methods. FedZoo-Bench facilitates a more consistent and reproducible FL research. Lastly, we conduct a comprehensive evaluation of several SOTA methods in terms of performance, fairness, and generalization to newcomers using FedZoo-Bench. We hope that our work will help the FL community to better understand the state of progress in the field and encourage a more comparable and consistent FL research.

%In this paper, we present a thorough examination of key variables that influence the success of FL experiments. We provide new insights and analysis of the FL-specific variables in relation to each other and performance results, and bring recommendations and best practices for a meaningful and well-incentivized FL experimental design. To further facilitate a more consistent and reproducible FL research, we have developed FedZoo-Bench, an open-source library based on PyTorch that provides a comprehensive set of standardized and customizable features, different evaluation metrics, and pre-implementation of 22 SOTA methods. We also present a comprehensive evaluation of performance, fairness, and generalization to newcomers of several SOTA methods using FedZoo-Bench. We hope that our work will help the FL community to better understand the state of progress in the field and encourage a more comparable FL research.

% In this paper, we present a thorough examination of key variables that influence the success of FL experiments. We provide new insights and analysis of the FL-specific variables in relation to each other and performance results, and bring recommendations and best practices for a meaningful and well-incentivized FL experimental design. We have developed FedZoo-Bench, an open-source library based on PyTorch that provides a comprehensive set of standardized and customizable features, different evaluation metrics, and pre-implementation of 22 SOTA methods. FedZoo-Bench facilitates a more consistent and reproducible FL research. Lastly, we conduct a comprehensive evaluation of performance, fairness, and generalization to newcomers of several SOTA methods using FedZoo-Bench. We hope that our work will help the FL community to better understand the state of progress in the field and encourage a more comparable FL research.

In future work, we plan to expand our study to other domains such as natural language processing and graph neural networks to understand how FL experimental settings behave in those areas \textcolor{red}{and to assess its versatility and applicability across different problem domains.} Additionally, we will continue to improve our benchmark by implementing more algorithms and adding new features. \textcolor{red}{We also have plans to establish an open leaderboard using FedZoo-Bench, enabling systematic evaluations of FL methods across a wide variety of datasets and settings.} Based on the comparison results presented in Section~\ref{sec:comparison}, we also believe that \textcolor{red}{the} development of new algorithms for both global and personalized FL approaches to achieve even greater improvement and more consistent results across different experimental settings would be an exciting future avenue. Also, more studies on evaluation metrics and \textcolor{red}{the} development of new metrics that can better assess different aspects of an FL algorithm would be a valuable future work.

%In future work, we plan to expand our study to other domains such as natural language processing and graph neural networks to understand how FL experimental variables behave for other ML tasks. Additionally, we will continue to improve our benchmark by implementing more algorithms and adding new features. Based on the comparison results presented in Section~\ref{sec:comparison}, development of new algorithms for both global and personalized FL approaches that can achieve even greater improvement and consistent results across different experimental settings would be an exciting avenue for future research. Also, further research and studies on evaluation metrics and the development of new metrics that capture different aspects of federated learning would be valuable contributions to the field.