\section{Background}
\label{sec:Background}

\subsection{Temporal Supersampling}
\label{subsec:temporal}
As mentioned in Section~\ref{sec:Introduction}, temporal super-sampling relies on the fact that most of the content
remains the same from frame to frame. A significant portion of a frame corresponds to at least a portion of either
the previous, future, or both the frames~\cite{TemporalSuper}. This correspondence can be find out using optical flow
vectors that describe the velocities of pixels within a frame~\cite{opticalflow}.

\subsubsection{Interpolation Vs Extrapolation}
\label{subsubsec:inter_vs_extra}

As the name suggests, {\em interpolation} predicts a frame in between two already rendered neighboring frames. Whereas,
extrapolation predicts frames based on the past frame(s) without considering future frame(s).
Figure~\ref{inter_vs_extra} explains these two algorithms in detail. In the figure, we observe that both interpolation
and extrapolation introduce some latency in the system, which is their own operational latency. Both processes double the
frame rate by generating a new frame after each rendered frame and display the frame in the following order $0, 0.5, 1,
1.5, 2, ..$. However, interpolation introduces an additional latency. Let us consider the newly generated frame with
suffix $1.5$. In interpolation, the frame $I_{1.5}$  is generated using two frames $F_1$ and $F_2$. Since frame
$I_{1.5}$ needs to be displayed before $F_2$, it waits for $F_2$, holds it, starts interpolating $I_{0.5}$, and first
displays the interpolated frame, and then $F_2$. Hence, the input latency becomes interpolation cost + the time before
displaying $F_2$. Whereas in the case of extrapolation, the new frame, $E_{1.5}$ is generated only based on the past frame
$F_1$ and all the frames are displayed at the very next refresh cycle.

% Figure environment removed

\subsection{Image Warping}
\label{warping}
Image warping is a reprojection technique that maps all locations in one image to locations in a second image. It can be
used to distort the original image in a way that serves a certain purpose. It can be used to perform various tasks such
as correcting image distortion as well as for creative purposes like morphing ~\cite{beier1992feature}. One such task is
frame prediction by warping the current frame to predict a future frame ~\cite{Niklaus_2020_CVPR}. The accuracy of using
warping on frame prediction depends on how well we understand the motion between the two frames. Most modern approaches
use machine learning and Deep Neural Networks (DNNs) to estimate this motion ~\cite{jin2023enhanced} ~\cite{extranet}.

\subsection{Reinforcement Learning}
\label{RL}
Reinforcement Learning (RL) ~\cite{sutton1999reinforcement} is a machine learning-based technique that uses information about
the environment and the feedback from its actions to learn an action inside the environment. It has been derived from
reinforcement theory ~\cite{luthans1999reinforce}, which argues that human behavior is a direct result of the
consequences of one's actions. In machine learning, this technique generally doesn't require any labeled data but
requires the problem to be formulated as an actor in an environment defined by a tuple of the state space, action space, and
associated rewards. The state space defines all the legal states for the actor to be in; however, note that most
RL problems operating using incomplete information -- the state space does not capture all aspects of the environment
completely . The {\em action space} is the
collection of all the actions that the actor is allowed to take in a state. The rewards are the gains/loss associated
with each action in a particular state.






