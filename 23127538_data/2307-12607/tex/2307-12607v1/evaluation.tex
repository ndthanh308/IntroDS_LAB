\section{Results and Analysis}
\label{sec:Evaluation}

\subsection{Performance Analysis}

\subsubsection{Comparison of Various Possible Decision Paths}
\label{perf_with_scen}

We showed the possible decision paths in Table~\ref{tab:pos_sce}. In this section,
we compare the quality of the generated scenes.
The results are shown in
Table~\ref{perf_scenario}. We make the following observations from the results:\\ 

\circled{1} For most of the
benchmarks, \exwarp performs the best. Even if it is not the best, the values are comparable with the rest of the
methods. \\ 
\circled{2} There is an 18.02\% and 6.58\% increase in PSNR and SSIM in \exwarp, respectively, as compared to  $S_1$ --
pure extrapolation-based method. \\ 
\circled{3} When compared to $S_6$, a pure warping-based method, there is a 9.24\%
and 0.04\% increase in PSNR and SSIM, respectively.

\begin{table}[!htb]
\footnotesize
\begin{center}
	
\resizebox{0.99\columnwidth}{!}{
  \begin{tabular}{|l|l|l|l|l|l|l|l|l|}
    \hline
%    \rowcolor{gray}
 \multicolumn{2}{|c|}{ \multirow{2}{*}{\textbf{Scenes}}}  &\multicolumn{6}{c|}{\textbf{Scenarios}} &  \multirow{2}{*}{\textbf{\exwarp}}\\
   
 \multicolumn{2}{|c|}{}   &   $\mathbf{S_1}$ & $\mathbf{S_2}$ & $\mathbf{S_3}$ & $\mathbf{S_4}$ & $\mathbf{S_5}$ & $\mathbf{S_6}$ & \\
    \hline
 \multirow{7}{*}{\rotatebox[origin=c]{90}{\textbf{PSNR(dB)}}}  
   &  LB &  20.63  & 20.58 & 20.05  & 23.91 & 21.43 & 25.01 & \textbf{28.65}\\
    \cline{2-9}
   & TR &19.82 & 19.18 & 19.11  & 22.60 & 20.19 & \textbf{23.68} & 17.30\\
   \cline{2-9}
    & VL & 20.33 & 19.92 &  24.90  & 36.36 & 32.68 & 43.70 & \textbf{44.66}\\
    \cline{2-9}
  &  TN &  18.29 & 17.56 &  16.94 & 19.71 & 19.97 & 17.39 & \textbf{24.10}\\
   
  & TN2 &   \textbf{18.70} & 16.85 & 15.60   & 17.98 & 17.35 & 15.06 & 17.12\\
  
  & TN3 &  \textbf{19.26}  & 17.14 &  15.64  & 17.91 & 16.98 & 14.79 & 17.01\\
  \cline{2-9}
  &  SL &  29.06 & 28.86 & 30.11  & 31.38 & 32.10 & 30.93 & \textbf{32.62} \\
   
  & SL2 &  31.17 & 29.20 &  28.20  & 29.97 & 26.83 & 27.59 & \textbf{32.33}\\
  
  & SL3 &  30.99 & 29.40 & 27.83 & 28.52 & 26.43 & 26.90 & \textbf{32.03}\\
  \hline
  \hline
    \multirow{7}{*}{\rotatebox[origin=c]{90}{\textbf{SSIM}}} 
    &  LB & 0.64 & 0.63 & 0.55 & 0.76 & 0.60 & 0.81 & \textbf{0.87}\\
    \cline{2-9}
   & TR &  0.63 & 0.59 & 0.51 & \textbf{0.69} & 0.54 & 0.71 & 0.55\\
   \cline{2-9}
    & VL &  0.80 & 0.77 & 0.69  & 0.92 & 0.75 & 0.99 & \textbf{0.99} \\
    \cline{2-9}
  &  TN &  0.75 & 0.73 & 0.64  & 0.78 & 0.79 & 0.65 & \textbf{0.81}\\
   
  & TN2 &  \textbf{0.78} & 0.71  & 0.61  & 0.70 & 0.66 & 0.56 & 0.69\\
  
  & TN3 &  \textbf{0.82} & 0.74 &  0.62  & 0.72 & 0.66 & 0.57 & 0.71\\
  \cline{2-9}
  &  SL  & 0.88 & 0.87 & 0.77 & 0.92 & \textbf{0.94} & 0.78 & 0.93\\
   
  & SL2  & 0.88 & 0.85 &  0.72  & \textbf{0.88} & 0.71 & 0.87 & 0.87 \\
  
  & SL3  & \textbf{0.88}  & 0.86 &   0.72 & 0.87 & 0.72 & 0.87 & 0.87 \\
  \hline
  \end{tabular}
  }
 \end{center}
 \caption{Performance comparison across all scenarios (decision paths)}
\label{perf_scenario}
\vspace{-6mm}
\end{table}

\subsubsection{Comparison with the State-of-the-Art}
\label{perf_inter}
In this section, we compare the performance of our proposed model with two interpolation-based methods and 
ExtraNet. The interpolation-based methods are Softmax Splatting~\cite{softmax} and EMA-VFI~\cite{EMA}. 
All three methods are DNN-based techniques. Softmax splatting uses forward warping; it uses forward and backward motion
flow (reprojection). However, in this approach, multiple pixels may map to the same
target location in frame $F_t$. Softmax splatting uses a modified softmax layer, which takes the frame's depth data to
handle this ambiguity. EMA-VFI uses a transformer network to perform frame interpolation. We show the performance of
these methods in Table~\ref{perf_inter_extra}. For PSNR, \exwarp is the best for 4/9 benchmarks and the
second best for two. There is a large difference only in the case of LB and TR. For the SSIM metric, both ExtraNet and
\exwarp do well. 



\begin{table}[!h]
\footnotesize
\begin{center}
	
\resizebox{0.99\columnwidth}{!}{
  \begin{tabular}{|l|l|l|l|l|l|l|}
    \hline
%    \rowcolor{gray}
 \multicolumn{2}{|c|}{ \multirow{2}{*}{\textbf{Scenes}}}  &\multicolumn{2}{c|}{\textbf{Interpolation}} & \multicolumn{2}{c|}{\textbf{Extrapolation}} \\
   
 \multicolumn{2}{|c|}{}    &\textbf{EMA-VFI} &   \textbf{Softmax Splatting} & \textbf{ExtraNet} & \textbf{\textit{\exwarp}}  \\
    \hline
 \multirow{7}{*}{\rotatebox[origin=c]{90}{\textbf{PSNR(dB)}}}  
   &  LB  & \textbf{49.52}  & 48.74 & 20.63  &  28.65\\
    \cline{2-6}
   & TR &  \textbf{24.60} & 23.42 & 19.82 &  17.30 \\
   \cline{2-6}
    & VL &20.86 & 20.54  &  20.33  & \textbf{44.66} \\
    \cline{2-6}
  &  TN &  14.40 & 13.84 & 18.29  & \textbf{24.11} \\
   
  & TN2 &  13.42 & 13.47 &   \textbf{18.70} &  17.12 \\
  
  & TN3 &  14.15 & 14.53 &  \textbf{19.27}  & 17.01 \\
  \cline{2-6}
  &  SL &  28.57 & 24.07 & 29.06  &  \textbf{32.62}\\
   
  & SL2 &  24.58 & 22.74 &  31.16  & \textbf{32.33} \\
  
  & SL3 &  32.53 & \textbf{34.95} &  30.99  & 32.03 \\
  \hline
  \hline
    \multirow{7}{*}{\rotatebox[origin=c]{90}{\textbf{SSIM}}} 
    &  LB & \textbf{0.99}  & \textbf{0.99}  & 0.64 & 0.87 \\
    \cline{2-6}
   & TR &  \textbf{0.97} & 0.95 & 0.63 & 0.55  \\
   \cline{2-6}
    & VL &  0.94 & 0.93 & 0.80   & \textbf{0.99} \\
    \cline{2-6}
  &  TN & \textbf{0.82} & 0.77 & 0.75  & 0.81 \\
   
  & TN2 &  0.71 & 0.59 &   \textbf{0.78} & 0.69 \\
  
  & TN3 &  0.75 & 0.64  &  \textbf{0.82}  & 0.71 \\
  \cline{2-6}
  &  SL &  0.61 & 0.26 &  0.88 & \textbf{0.93} \\
   
  & SL2 &  0.40  & 0.27 & \textbf{0.88}   & 0.87 \\
  
  & SL3 &  0.75 & 0.83 &  \textbf{0.88}  & 0.87 \\
  \hline
  \end{tabular}
  }
 \end{center}
 \caption{Performance comparison with the state-of-the-art}
\label{perf_inter_extra}
\vspace{-6mm}
\end{table}


\subsection{Frame rate (FPS)}
In this section, we plot the final frame rate achieved using \exwarp for each benchmark. As mentioned in
Section~\ref{rlmodel}, the original frame rate was 30 fps. The results are shown in Figure~\ref{fps}. The insights from
the results are as follows:\\ 

\circled{1} The effective upsampled frame rate for all the benchmarks is more than 100 fps. We compute this based on the
number of new frames that we actually manage to insert. The more we extrapolate, lower is this figure. \\
\circled{2} The average frame rate across benchmarks is almost 110 fps, hence the supersampling factor is nearly 4 for
our proposed method. Note that this is more than all state-of-the-art work.

% Figure environment removed

\subsection{Warping vs Extrapolation}
Our proposed model, \exwarp, predicts the best method between warping and extrapolation. In this section, we plot the
prediction pattern of our predictor. The results are shown in Figure~\ref{actions}. The observations from the results
are as follows:\\ \circled{1} For most of the benchmarks, the ratio between warping and extrapolation is 80:20 except
\textit{VIL}.\\ \circled{2} The average percentage for warping across benchmarks is 75.86\%.

% Figure environment removed

\subsection{Synthesis Results}
We used the popular tool NNGen~\cite{nngen} to generate a baseline Verilog code for our neural network.
We then made modifications to it and manually tuned it.  We used
the Cadence Genus Tool (TSMC 28 nm technology) to synthesize the design and obtain the power, area and timing numbers.
Table~\ref{pred_lat} shows the area and power overheads of the hardware predictor. The total area is 0.12 $mm^2$, which
is negligible. Also, the latency, 6.2 ns, is insignificant. 


\begin{table}[!h]
\vspace{-1mm}
\footnotesize
\begin{center}
	
\resizebox{0.99\columnwidth}{!}{	
\begin{tabular}{ |l|p{45mm}|l|} 

\hline
\rowcolor{gray}
\textbf{Parameter} &
\textbf{Value} \\ 
\hline\hline
Tool & Cadence Genus, 28 $nm$\\
\hline
Area & 0.12 $mm^2$\\
\hline
Power & 9.12 $mW$  \\
\hline
Latency & 6.2 $ns$ \\
\hline

\end{tabular}
}
\end{center}
\caption{Overheads of the hardware predictor}
\label{pred_lat}
\vspace{-4mm}
\end{table}



