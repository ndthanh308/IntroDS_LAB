\section{Methodology}
\label{sec:methodology}
\subsection{Overview}
\label{sec:overview}
We propose to insert three new frames at time instances $t+0.25$, $t+0.5$, $t+0.75$ between any two consecutive frames $F_t$ and $F_{t+1}$. As mentioned in Section~\ref{sec:Introduction}, warping and extrapolation are two options for synthesizing these new frames. Based on these two algorithms, multiple scenarios are possible (refer to Figure~\ref{fig_overview}).  

%Now, we have two choices for the remaining two frames- warping and extrapolation. For the first frame $P_{t+0.25}$, extrapolation is not an option because if we start extrapolating frame $F_{t}$ at time $t$, the extrapolation takes at least $t+0.5$ time. So, the remaining choice to generate $P_{t+0.25}$ is warping $F_t$. We may use warping and extrapolation for $P_{t+0.75}$. If we use extrapolation, we need to finish extrapolating $P_{t+0.75}$ prior before $t+0.75$; we start extrapolating $P_{t+0.25}$ at time $t+0.25$. Since warping takes less time, we warp the latest frame $P_{t+0.25}$ instead of using $P_{t+0.5}$.

% Figure environment removed
\subsection{Problem Formulation}
\label{problem}
Given two rendered frames $F_t$ and $F_{t+1}$, our goal is to insert $n$ new frames between these two frames without using $F_{t+1}$, where $n$ $\in$ $[1,3]$. We represent these frames as $P_i$, where $i$ falls within the interval $[0,n]$ and the frame $P_i$ is displayed on the screen at time $t+(i/4)$. The value of $n$ depends on the algorithm used for generating $P_i$. We have two options - warping and extrapolation - we store them in a set A, denoted by $\{W, E\}$. 
 In the span of time from $t$ to $t+1$, we may have to choose an algorithm from set $A$ at most five times, denoted as $d_1$, $d_2$, $d_3$, $d_4$, and $d_5$, at $t$, $t+0.25$, and $t+0.5$ based on the current state of the system. The decision at time $t$, $d_1$, is whether to display the warped $F_1$ or wait for the extrapolated frame. Since the extrapolated frame would be available at time $t+0.5$, the last displayed frame i.e. $F_t$ remains on the screen; it also skips the decision at $t+0.25$. We show the complete flow based on various selections made at different time instances in Figure~\ref{flow}. Based on these five decisions, six scenarios for frames $P_i$s sre possible (refer to Table~\ref{pos_sce})

% Figure environment removed



\begin{table}[]
\footnotesize
\begin{center}
\resizebox{0.99\columnwidth}{!}{
\begin{tabular}{| l |l | l| l| l|} 

\hline
\textbf{Scenario} &
$\mathbf{P_1}$ & $\mathbf{P_2}$ & $\mathbf{P_3}$ & \textbf{Frame rate}\\
&  &  & & \textbf{upsampling} \\ 
\hline\hline
$\mathbf{S_1}$ & $F_1$  & Extrapolated $F_1$  & Extrapolated $F_1$ & 2$\times$\\
 &  (No new frame)  &  &  (No new frame) &   \\ \hline
$\mathbf{S_2}$ & $F_1$   & Extrapolated $F_1$  & Warped $P_2$ & 3$\times$ \\
 & (No new frame) &  &  &   \\ \hline
$\mathbf{S_3}$ & Warped $F_1$ & Extrapolated $F_1$ & Extrapolated $P_1$ & 4$\times$\\  \hline
$\mathbf{S_4}$ & Warped $F_1$ & Extrapolated $F_1$ & Warped $P_2$ & 4$\times$\\  \hline
$\mathbf{S_5}$ & Warped $F_1$ & Warped $P_1$ & Extrapolated $P_1$ & 4$\times$\\  \hline
$\mathbf{S_6}$ & Warped $F_1$ & Warped $P_1$ & Warped $P_2$ & 4$\times$\\
\hline
\end{tabular}
}
\end{center}
%\caption{Details of the baseline system}
\caption{Possible scenarios based on the type of synthesized frames- $P_1$, $P_2$, and $P_3$}
\label{pos_sce}
\end{table}








