\documentclass[runningheads]{llncs}

\newcommand\calF{\mathcal{F}}
\newcommand\calG{\mathcal{G}}
\newcommand\calM{\mathcal{M}}
\newcommand\calV{\mathcal{V}}
\newcommand\calU{\mathcal{U}}
\newcommand\calW{\mathcal{W}}
\newcommand\calP{\mathcal{P}}
\newcommand\calD{\mathbb{D}}
%%%%%%%%%%%%%%%%%
%% macros introduced by Luke 
\newcommand\mydef[1]{{\bf\em #1}}
%%%%%%%%%%%%%%%%%

\newcommand{\numviparams}{{| \lambda |}}
\newcommand{\scoreaccvars}[1]{s_1^{#1}, \ldots, s_{\numviparams}^{#1}}
\newcommand{\scoreaccvar}[2]{s_{#1}^{#2}}
\newcommand{\isdeterm}[1]{\text{Deterministic}({#1})}


\newcommand{\expect}[1]{\mathbb{E}\left[{#1}\right]}
\newcommand{\var}[1]{\mathbb{V}\left[ {#1} \right]}
\newcommand{\expectdist}[2]{\mathbb{E}_{#1}\left[ {#2} \right]}
\newcommand{\vardist}[2]{\mathbb{V}_{#1}\left[ {#2} \right]}
\newcommand{\cov}[2]{\mathbb{C}\text{ov}[{#1}][{#2}]}
\newcommand{\covv}[1]{\mathbb{C}\text{ov}[{#1}]}
\newcommand{\corr}[1]{\mathbb{C}\text{orr}[{#1}]}

\newcommand{\fix}[1]{\mathit{fix}\left({#1}\right)}
\newcommand{\sbr}[1]{\left\llbracket {#1} \right\rrbracket}
\newcommand{\ctxtype}[3]{{#1} \cong_\text{ctx} {#2} : {#3}}
\newcommand{\bigstep}[3]{{#1} \Downarrow_{#2} {#3}}


% PCF types
\newcommand{\bool}{\mathit{bool}}
\newcommand{\nat}{\mathit{nat}}

\newcommand{\ctx}[1]{\mathcal{C}\left[ {#1}\right] }
\newcommand{\pcft}[1]{\text{PCF}_{#1}}

\newcommand{\nfl}{\mathbb{N}_\bot}
\newcommand{\bfl}{\mathbb{B}_\bot}

% PCF constructs
\newcommand{\succc}[1]{\mathbf{succ}({#1})}
\newcommand{\succcn}[2]{\mathbf{succ}^{#1}({#2})}
\newcommand{\zero}{\mathbf{0}}
\newcommand{\zerotest}[1]{\mathbf{zero}\left({#1}\right)}
\newcommand{\pred}[1]{\mathbf{pred}\left( {#1} \right)}
\newcommand{\predn}[2]{\mathbf{pred}^{#1}\left( {#2} \right)}
\def\solvable{\#}

\newcommand{\true}{\mathbf{true}}
\newcommand{\false}{\mathbf{false}}
\newcommand{\pcffix}[1]{\mathbf{fix}\left({#1}\right)}
\newcommand{\pcffn}[3]{\mathbf{fn}~{#1}:{#2}\mathpunct{.}{#3}}
\newcommand{\pairtype}[2]{{#1} * {#2}}
\newcommand{\pairexp}[2]{\mathbf{pair}({#1}, {#2})}
\newcommand{\leftexp}[1]{\mathbf{left}({#1})}
\newcommand{\rightexp}[1]{\mathbf{right}({#1})}

\newcommand{\RationalPos}{\mathbb{Q}^{+}}

\newcommand{\meas}[1]{\mathbb{M}\left( {#1} \right) }
\newcommand{\integ}[1]{\sbr{#1}_I}

\newcommand{\notbigstep}[2]{{#1}~\cancel{\Downarrow}_{#2}}
\newcommand{\subtrace}[3]{{#1}^{{#2} \ldots {#3}}}
\newcommand{\supp}[1]{\textsf{supp}\left({#1}\right)}
\newcommand{\dom}[1]{\textsf{Dom}\left({#1}\right)}
\newcommand{\suppk}[2]{\textsf{Supp}^{#1}\left({#2}\right)}
\newcommand{\tracespace}{\bigcup_{n \in \mathbb{N}}[0, 1]^n}
\newcommand{\generictracespace}{\mathbb{T}}
\newcommand{\nnreals}{\mathbb{R}_{\geq 0}}
\newcommand{\posreals}{\mathbb{R}_{> 0}}
\newcommand{\reals}{\mathbb{R}}

\newcommand{\unrollkM}[2]{\textsf{unroll}_{#1}\left({#2}\right)}
\newcommand{\nphmcint}[5]{\Psi_\textsf{NP}\left({#1}, {#2}, {#3}, {#4}, {#5}\right)}

%SPCF constructs
\newcommand{\spcfvalues}{\Lambda^0_v}

\newcommand{\prevalueM}[1]{\textsf{value}^{-1}_{#1}(\spcfvalues{})}
\newcommand{\num}[1]{\underline{#1}}

% \theoremstyle{definition}
% \newtheorem{thm}{Theorem}
% \newtheorem{lem}{Lemma}
% \newtheorem{defn}{Definition}
% \newtheorem{conj}{Conjecture}
% \newtheorem{prop}{Proposition}

%\theoremstyle{definition}
%\newtheorem{defn}{Definition}[section]
%\newtheorem{example}[defn]{Example}
%
%
%\theoremstyle{plain}
%\newtheorem{thm}{Theorem}[section]
%\newtheorem{lem}[thm]{Lemma}
%\newtheorem{cor}[thm]{Corollary}
%\newtheorem{conj}[thm]{Conjecture}
%\newtheorem{prop}[thm]{Proposition}
%\newtheorem{remark}[thm]{Remark}

%% Proofs
%\let\oldproof\proof
%\renewcommand{\proof}{\color{blue}\oldproof}


\definecolor{codegreen}{rgb}{0,0.6,0}
\definecolor{codegray}{rgb}{0.5,0.5,0.5}
\definecolor{codepurple}{rgb}{0.58,0,0.82}
\definecolor{backcolour}{rgb}{0.95,0.95,0.92}

\lstdefinestyle{myStyle}{
    belowcaptionskip=1\baselineskip,
    breaklines=true,
    frame=none,
    basicstyle=\footnotesize\ttfamily,
    keywordstyle=\bfseries\color{green!40!black},
    commentstyle=\itshape\color{purple!40!black},
    identifierstyle=\color{blue},
    backgroundcolor=\color{gray!10!white},
    %backgroundcolor=\color{backcolour}, 
    numberstyle=\tiny\color{codegray},
    stringstyle=\color{codepurple},
    breakatwhitespace=false,                          
    keepspaces=true,                 
    numbers=left,       
    numbersep=5pt,                  
    showspaces=false,                
    showstringspaces=false,
    showtabs=false,                  
    tabsize=2,
}

% argmin/argmax
\DeclareMathOperator*{\argmax}{arg\,max}
\DeclareMathOperator*{\argmin}{arg\,min}

% Concatenation of lists
\newcommand\doubleplus{+\kern-1.3ex+\kern0.8ex}

% Program configurations
\newcommand{\tuple}[1]{\ensuremath{\langle #1 \rangle}}
% Rule based definitions
\newcommand{\Rule}[4][]{\ensuremath{\inferrule*[lab={\hypertarget{#2}{(\TirName{#2})}},#1]{#3}{#4}}}

% Calligraphic symbols
\newcommand{\calI}{{\mathcal I}} 
\newcommand{\calT}{{\mathcal T}}

%  Macro for new Y operator.
\newcommand{\yBounded}[3]{\mu^{#1}_{#2}\rvert_{#3}}

%%%%%%%%%%%%%%%%%
 
%%%%%%%%%%%%%%%%%

\newcommand{\expv}{\mathbb{E}}

\newcommand{\combTr}[2]{\left[\begin{matrix}
		#1\\
		#2
	\end{matrix} \right]}

\newcommand{\exType}[2]{\left\{\begin{matrix}
		#1\\
		#2
	\end{matrix} \right\}}
\newcommand{\myint}[1]{ [#1]}
\newcommand{\Uniform}{\ensuremath{\mathrm{Uniform}}}
\newcommand{\Normal}{\ensuremath{\mathrm{normal}}}
\DeclareMathOperator{\abs}{abs}
\DeclareMathOperator{\pdf}{pdf}

\newcommand{\intConf}[1]{\lceil#1\rceil}
\newcommand{\tr}{\boldsymbol{t}}

\newcommand{\sample}{\tt{sample}}
%\newcommand{\fix}{\texttt{fix}}
%\newcommand{\num}[1]{\underline{#1}}
\newcommand{\myif}{\texttt{if}}
\newcommand{\mylet}{\texttt{let} \, }
\newcommand{\myin}{\, \texttt{in} \,}
\newcommand{\mythen}{\, \texttt{then} \,}
\newcommand{\myelse}{\, \texttt{else} \,}
\newcommand{\score}{\tt{score}}
\newcommand{\tick}{\tt{tick}}

\newcommand{\term}{\tt{term}}
\newcommand{\pv}{\mathbf{v}}
\newcommand{\rv}{\mathbf{r}}

\newcommand{\interval}{\mathfrak{I}}

\newcommand{\typeReal}{\textbf{\textsf{R}}}

\newcommand{\symbolInt}{\myint{\cdot}}

\newcommand{\LambdaInterval}{\Lambda_{\interval}}
\newcommand{\LambdaSymbolic}{\Lambda_{\text{sym}}}

\newcommand{\toIntervalTerm}[1]{#1^{2\interval}}

%Others
\newcommand{\Sset}{\mathbb{S}}
\newcommand{\Iset}{\mathbb{I}}
\newcommand{\Rset}{\mathbb{R}}
\newcommand{\Nset}{\mathbb{N}}
\newcommand{\Zset}{\mathbb{Z}}

\newcommand{\Term}{\mathbb{T}}
\newcommand{\prob}{\mathbb{P}}
\newcommand{\expt}{\mathbb{E}}


\newcommand{\Leb}{\tt{Leb}}
\newcommand{\Red}{\tt{Red}}
\newcommand{\cost}{\text{cost}}

%\newcommand{\intervalab}[2]{\underline{[#1,#2]}}
\newcommand{\intervalab}{\underline{[a,b]}}
\newcommand{\interI}{\mathcal{I}}
\newcommand{\trans}{\mathcal{T}}

\newcommand{\iv}{\mathbb{I}}

% Programming language constructs
\newcommand{\lit}[1]{\underline{#1}}
\newcommand{\letIn}[1]{\mathsf{let}\,{#1}\,\mathsf{in}\,}
\newcommand{\fixLam}[2]{\mu {#1} {#2}.}
\newcommand{\ifElse}[3]{\mathsf{if} (#1 \le \num{0}) \, {#2} \,\mathsf{else}\, {#3}}

%%Basic notions
\newcommand{\pspace}{(\Omega,\mathcal{F},\probm)}
\newcommand{\probm}{\mathbb{P}}
\newcommand{\condexpv}[2]{{\expt}{\left[{#1} \mid {#2}\right]}}

\newcommand{\stdConf}[1]{(#1)}
%\newcommand{\intConf}[1]{\lceil#1\rceil}
%\newcommand{\intConf}[1]{(#1)}
%\newcommand{\symConf}[1]{\langle\!\langle  #1 \rangle\!\rangle}
%\newcommand\symPath[1]{(#1)}
\newcommand{\symPath}[1]{\langle\!\langle  #1 \rangle\!\rangle}
\newcommand\symConf[1]{(#1)}

\newcommand{\ifSimple}[3]{\mathsf{if}(#1, #2, #3)}
%\newcommand{\ifElse}[3]{\mathsf{if} (#1 \le 0) \, \allowbreak {#2} \, \allowbreak \mathsf{else}\, {#3}}
%\newcommand{\ifElse}[3]{\ifSimple{#1}{#2}{#3}}

%\newcommand{\trace}{\mathsf{s}}
%
%\newcommand\defn[1]{{\bf \em #1}}
\newcommand{\traces}{\mathbb{T}}
%
%\newcommand{\stdConf}[1]{(#1)}
%%\newcommand{\intConf}[1]{\lceil#1\rceil}
%\newcommand{\intConf}[1]{(#1)}
%%\newcommand{\symConf}[1]{\langle\!\langle  #1 \rangle\!\rangle}
%%\newcommand\symPath[1]{(#1)}
%\newcommand{\symPath}[1]{\langle\!\langle  #1 \rangle\!\rangle}
%\newcommand\symConf[1]{(#1)}

\newcommand{\valueSem}[1]{\mathsf{val}_{#1}} % value (semantics)
\newcommand{\weightSem}[1]{\mathsf{wt}_{#1}} % weight (semantics)
\newcommand{\measureSem}[1]{\llbracket #1 \rrbracket}
\newcommand{\posterior}{\mathsf{posterior}}


%%%%%%%%%
% 
%%%%%%%%
\newcommand{\loc}{\ell}
\newcommand{\locs}{\mathit{L}}
\newcommand{\blocs}{\mathit{L}_{\mathrm{b}}}

\newcommand{\iflocs}{\mathit{L}_{\mathrm{if}}}
\newcommand{\looplocs}{\mathit{L}_{\mathrm{while}}}

\newcommand{\alocs}{\mathit{L}_{\mathrm{a}}}
\newcommand{\wlocs}{\mathit{L}_{\mathrm{w}}}
\newcommand{\rlocs}{\mathit{L}_{\mathrm{r}}}
\newcommand{\Alocs}[1]{\mathit{L}_{\mathrm{A}}^{\mathsf{#1}}}
\newcommand{\Dlocs}{\mathit{L}_{\mathrm{nd}}}
\newcommand{\transitions}{{\rightarrow}}

%%% 
\newcommand{\plocs}{\mathit{L}_{\mathrm{p}}}
\newcommand{\tlocs}{\mathit{L}_{\mathrm{t}}}

\newcommand{\lin}{\loc_\mathrm{init}}
\newcommand{\lout}{\loc_\mathrm{out}}
\newcommand{\val}[1]{\mbox{\sl Val}_{#1}}

\newcommand{\pvars}{V_\mathrm{p}}
\newcommand{\rvars}{V_{\mathrm{r}}}
\newcommand{\pre}{\mathrm{pre}}

\newcommand{\sle}{\sqsubseteq}
\newcommand{\sge}{\sqsupseteq}

\newcommand{\lfp}{\mathrm{lfp}}
\newcommand{\gfp}{\mathrm{gfp}}

\newcommand{\rdvarjdis}{\mathcal D}
\newcommand{\sampset}{\textit{supp}}

\newcommand{\upd}{\mbox{\sl upd}}
\newcommand{\wet}{\mbox{\sl wt}}
\newcommand{\transset}{\mathfrak T}
\newcommand{\valin}{\pv_{\mathrm{init}}}
\newcommand{\ret}{\mbox{\sl ret}}

\newcommand{\win}{w_{\mathrm{init}}}

\newcommand{\sampdpd}{\overline{\Upsilon}}

\newcommand{\outmap}{\text{O}}
\newcommand{\sat}[1]{\langle #1 \rangle}
\newcommand{\monoid}{\mbox{\sl Monoid}}
\newcommand{\handelmanformat}{(\dagger)}

\newcommand{\trunc}{\mathcal{B}}

\newcommand{\ewt}{\mbox{\sl ewt}}
\newcommand{\statemap}{\text{St}}

\newcommand{\valrd}{{\mathbf{r}}}
\newcommand{\frmloc}{\ell^{\mathrm{src}}}
\newcommand{\toloc}{\ell^{\mathrm{dst}}}

\newcommand{\monomials}{\mathbf{M}}

\begin{document}

\title{Counterfactual Explanations for Graph Classification Through the Lenses of Density}

\titlerunning{Counterfactual Explanations for Graph Classification}

\author{Carlo Abrate\inst{1}\orcidID{0009-0003-8604-9699} \and
Giulia Preti\inst{1}\orcidID{0000-0002-2126-326X} \and
Francesco Bonchi\inst{1,2}\orcidID{0000-0001-9464-8315}}

\authorrunning{C. Abrate et al.}

\institute{
    CENTAI Institute, Turin, Italy \and
    EURECAT, Barcelona, Spain\\
    \email{\{carlo.abrate,giulia.preti,bonchi\}@centai.eu}
}

\maketitle \sloppy              % typeset the header of the contribution

\begin{abstract}
Counterfactual examples have emerged as an effective approach to produce simple and understandable post-hoc explanations. In the context of graph classification,
previous work has focused on generating counterfactual explanations by manipulating the most elementary units of a graph, i.e., removing an existing edge, or adding a non-existing one. In this paper, we claim that such language of explanation might be too fine-grained, and turn our attention to some of the main characterizing features of real-world complex networks, such as the tendency to close triangles, the existence of recurring motifs, and the organization into dense modules. We thus define a general
\emph{density-based counterfactual search} framework to generate instance-level counterfactual explanations for graph classifiers, which can be instantiated with different notions of dense substructures. In particular, we show two specific instantiations of this general framework: a method that searches for counterfactual graphs by opening or closing triangles, and a method driven by maximal cliques.
We also discuss how the general method can be instantiated to exploit any other notion of dense substructures, including, for instance, a given taxonomy of nodes.
We evaluate the effectiveness of our approaches in 7 brain network datasets and compare the counterfactual statements generated according to several widely-used metrics. Results confirm that adopting a semantic-relevant unit of change like density is essential to define versatile and interpretable counterfactual explanation methods.

%\keywords{Explainability  \and Counterfactual Explanations \and Graph Classification \and Brain Networks.}
\end{abstract}
%
%
\vspace{-0.1in}
\section{Introduction}
\vspace{-0.1in}
% With recent advances in natural language processing (NLP), large language models (e.g., ChatGPT and chatbot) are increasingly common use. As an essential and fundamental part of language models, the text classification task has a wide range of applications, such as content moderation, sentiment analysis, fraud detection, and spam filtering. However, text classification models are vulnerable to adversarial attacks, deliberately manipulating the model output by modifying the input text. \cite{RealWorl22:online}

With recent advances in natural language processing (NLP), large language models (e.g., ChatGPT \cite{chatgpt:online} and Chatbots \cite{zhang2020dialogpt,shuster2020dialogue,roller2021recipes}) have become increasingly popular and widely deployed in practice. Wherein, text classification plays an important role in language models, and it has a wide range of applications, including content moderation, sentiment analysis, fraud detection, and spam filtering~\cite{9TextCla71:online,AIDocume25:online}. Nevertheless, text classification models are vulnerable to word-level adversarial attacks, which imperceptibly manipulate the words in input text to alter the output~\cite{ren2019generating,maheshwary2021generating,jin2020bert,garg2020bae,lee2022query,li2021contextualized,feng2018pathologies}. These attacks can be exploited maliciously to spread misinformation, promote hate speech, and circumvent content moderation~\cite{wu2019misinformation}.

% \BW{To mitigate such attacks on text classification models, a series of empirical defenses have been proposed~\cite{pruthi2019combating,zhufreelb,jones2020robust,zhou2021defense,dongtowards,bao2021defending,huang2022word,yang2022robust,yoo2022detection,moon2022gradmask,mosca2022suspicious,azizi2021t,liu2022piccolo,zhang2022improving,qi2021onion,minh2022textual,DBLP:conf/iclr/MiyatoDG17}, where the state-of-the-art is based on adversarial training~\cite{DBLP:conf/iclr/MiyatoDG17} or data augmentation~\cite{jones2020robust,zhou2021defense}. 
% % For instance, 
% % Miyato et al.~\cite{DBLP:conf/iclr/MiyatoDG17} proposed to apply  adversarial training to the text domain. 
% % Zhou et al.~\cite{zhou2021defense} augments training data by creating virtual texts that mix the embedding of the original word with its synonyms' embedding via Dirichlet sampling. 
% However, these methods are shown to be broken by adaptive adversaries~\cite{jin2020bert}. 
% % For instance, \cite{} show that the defense 
% To end the cat-and-mouse game, in the past three years, researchers start focusing on certified defenses~\cite{jia2019certified,huang2019achieving,ye2020safer,zeng2021certified,wang2021certified}, i.e., defenses with provable robustness guarantees against the worst-case attacks. Among different certified defense...}

To defend against such attacks, numerous techniques have been proposed to improve the robustness of language models, especially for text classification models. For instance, adversarial training~\cite{goodfellow2014explaining,madrytowards,dongtowards} retains the model using both clean and adversarial examples to enhance the model performance; feature detection~\cite{yoo2022detection,mozes2021frequency} checks and discards detected adversarial inputs to mitigate the attack; and input transformation~\cite{wang2021natural, yang2022robust} processes the input text to eliminate possible perturbations. However, these empirical defenses are only effective against specific adversarial attacks and can be broken by adaptive attacks~\cite{jin2020bert}.
% (without formal guarantees). New adversarial examples in unseen attacks or adaptive attacks may explicitly break the defense.
% Thus, the defender becomes caught in a cycle with the adversary.
% robustness and generalization of ML models can be improved by crafting high-quality adversaries and including them in the training data.

One promising way to win the arms race against unseen or adaptive attacks %escape the attack-defense cycle 
is to provide provable robustness guarantees for the model. This line of work aims to develop certifiably robust algorithms that ensure the model's predictions are stable over a certain range of adversarial perturbations. Among different certified defense methods, randomized smoothing~\cite{cohen2019certified,zhang2020black,li2022sok} does not impose any restrictions on the model architecture and achieves acceptable accuracy for large-scale datasets. This method injects random noise, sampled from a smoothing distribution, into the input data during training to smoothen the classifier. The smoothed classifier will make a consistent prediction for a perturbed test instance (with noise) as the original class label. %under the smoothing distribution and ultimately outputs the most probable results. 
% Although randomized smoothing methods have been successfully applied to image inputs, applying them to text inputs remains fairly challenging.
Despite the successful application of randomized smoothing in protecting vision models, applying these methods to safeguard language models remains fairly challenging.
% We summarize four challenges for certifying the text classification model:
% While existing works on adversarial examples have succeeded in the image domain, it is still challenging to deal with text data due to its discrete nature. 

\begin{table*}[]
\centering
\setlength\tabcolsep{3pt}
% \scriptsize
\caption{Comparison of certified defense methods for NLP robustness against textual adversarial attacks.}
\vspace{-0.07in}
\resizebox{\linewidth}{!}{
\begin{threeparttable}
\begin{tabular}{@{}lcccccccc@{}}
\toprule
\multirow{2}{*}{Method} & \multirow{2}{*}{Model architecture} & \multicolumn{4}{c}{Adversarial operations (smoothing distribution / $\ell_p$ perturbation)} & Certified & Uni- & Accuracy \\
%\cline{3-6}
 &  & Substitution & Reordering & Insertion & Deletion & radius / $\rad$ & versality & (large-scale data) \\
\midrule
IBP-trained~\cite{jia2019certified} & LSTM/Att. layer & \ding{51}& \ding{51}& \ding{51}& \ding{51} & \ding{55} & \ding{55} & Low \\
POPQORN~\cite{ko2019popqorn} & RNN/LSTM/GRU & \ding{51}& \ding{51}& \ding{51}& \ding{51} & \ding{55} & \ding{55} & Low \\
Cert-RNN~\cite{du2021cert} & RNN/LSTM & \ding{51}& \ding{51}& \ding{51}& \ding{51} & \ding{55} & \ding{55} & Low \\
DeepT~\cite{bonaert2021fast} & Transformer & \ding{51}& \ding{51}& \ding{51}& \ding{51} & \ding{55} & \ding{55} & Low \\
SAFER~\cite{ye2020safer} & Unrestricted & \ding{51} (Uniform / $\ell_0$) & \ding{55} & \ding{55} & \ding{55} & \ding{51} Practical & \ding{55} & High \\
% WordDP~\cite{wangw2021certified} & Unrestricted & \ding{51} (Uniform) & \ding{55} & \ding{55} & \ding{55} & \ding{51} (1) & \ding{55} & High \\
RanMASK~\cite{zeng2021certified} & Unrestricted & \ding{51} (Uniform / $\ell_0$) & \ding{55} & \ding{55} & \ding{55} & \ding{51} & \ding{55} & High\\
CISS~\cite{zhao2022certified} & Unrestricted & \ding{51} (Gaussian / $\ell_2$) & \ding{55}& \ding{55}& \ding{55} & \ding{51} & \ding{55} & High \\
\midrule
Text-CRS (Ours) & Unrestricted & \ding{51} (Staircase / $\ell_1$) & \ding{51} (Uniform / $\ell_1$) & \ding{51} (Gaussian / $\ell_2$) & \ding{51} (Bernoulli / $\ell_0$) & \ding{51} Practical & \ding{51} & $>$  SOTA \\
\bottomrule
\end{tabular}
\begin{tablenotes}
    \item {1. The model architectures applicable to the first four methods have size restrictions, i.e., the number and size of layers cannot be too large.}
    \item {2. "Practical" means that the certified radius can correspond to the $\rad$-word level of perturbation. (We propose four practical certified radii.)}
    % \item {3. We introduce four certified radii under different smoothing distributions.}
\end{tablenotes}
\end{threeparttable}
}
\label{tab:intro}
\vspace{-0.23in}
\end{table*}


% And because they do not consider the numerical relationships between words
First, due to the discrete nature of the text data, numerical $\ell_1$ or $\ell_2$-norms cannot be directly used to measure the distance between texts. Without considering word embeddings,
% (e.g., may need encoding of words before distance calculation). 
previously certified defenses in the NLP domain, such as SAFER~\cite{ye2020safer} and WordDP~\cite{wang2021certified2}, are limited to certifying robustness against $\ell_0$ perturbations generated by synonym substitution attacks. Also, their assumption of uniformly distributed synonyms is impractical, leading to relatively low certified accuracy. Second, text classification models are vulnerable to a range of word-level operations that result in various perturbations. For instance, the word insertion operation introduces random words in the lexicon, while the word reordering operation causes positional permutation. These diverse perturbations can deceive text classification models successfully~\cite{garg2020bae,jin2020bert,lee2022query,li2021contextualized,feng2018pathologies}. To our best knowledge, there exist 
no certified defense methods 
% have been investigated 
against these word-level perturbations. 
% can provide certifiable robustness bounds for text classification models against these word-level perturbations. 
Third, significant absolute differences between adversarial and clean texts may exist due to word-level operations, while conventional  
% Conversely, 
randomized smoothing can only ensure the model's robustness against perturbations within a small radius. 
% Thus, the substantial difference may exceed the bounds of classical smoothing methods. 
Prior works~\cite{fischer2020certified,li2021tss,alfarra2022deformrs,hao2022gsmooth, liu2021pointguard} for images address this challenge by providing robustness guarantees against semantic transformations (e.g., rotation, scaling, shearing). However, they cannot be directly applied to the NLP domain because texts and words have a more heterogeneous discrete domain, and word insertion and deletion are new semantic transformations not involved in the image domain. %Firstly, unlike image pixels, numerous words have a greater heterogeneous discrete domain. Secondly, word insertion and deletion are new semantic transformations not involved in the image domain. 


% Figure environment removed

%\BW{We need to mention somewhere that our Text-CRS can address the three challenges in the existing works.}
In this paper, we present Text-CRS, the first generalized certified robustness framework against common word-level textual adversarial attacks (i.e., synonym substitution, reordering, insertion, and deletion) via randomized smoothing (see Figure\,\ref{fig:Text-CRS}). Text-CRS 
%encompasses most of the fundamental operations that occur in textual adversarial attacks (i.e., synonym substitutions, reorderings, insertions, and deletions), and 
\emph{certifies the robustness in the word embedding space} without imposing restrictions on model architectures, and demonstrates \emph{high universality} and \emph{superior accuracy} compared to state-of-the-art (SOTA) methods (see Table\,\ref{tab:intro}). 
% instead of the discrete text space. 
%Particularly, we consider incorporating word embeddings instead of d
%\BW{To our best knowledge, we take the first cut to?} 
%we consider numerical relationships between words rather than discrete relationships by incorporating the word embedding vectors into the certification. 
% Furthermore, by partitioning the operations' input space into permutation and embedding spaces, 
Specifically, we first model word-level adversarial attacks as combinations of permutations and embedding transformations. 
%\BW{Here we briefly describe the permutation and embedding transformations for two attacks, e.g., Synonym substitution and word insertion?}
%\BW
{For instance, synonym substitution attacks transform the original words' embeddings with the synonyms' embeddings, while word reordering attacks transform the word orderings with certain word permutations.} 
Then, the {word-level adversarial attacks} can be guaranteed to be certified robust (``\emph{practical}'') if their corresponding permutation and embedding transformation are certified. 

%\BW
To this end, we develop customized theorems (see Section~\ref{sec:theorem}) to derive the certified robustness against each attack.
% Furthermore, to improve the certified accuracy/radius,  % theoretically,
In each theorem, we use an 
appropriate 
% novel theorems that apply well-designed 
noise distribution for our smoothing distributions in order to fit different word-level attacks. 
% transformations in word-level adversarial operations. 
% Specifically, 
% in Text-CRS, 
For instance, unlike existing works~\cite{ye2020safer,huang2022word} that use the uniform distribution, we propose to use the Staircase-shape distribution~\cite{geng2015staircase} to simulate the synonym substitutions, which ensures that semantically similar synonyms are more likely to be substituted. %fixing the relative importance of each position in a sentence to the prediction is impossible due to differences in model architectures and datasets; thus, 
% \BW{address the uniform sampling issue..}
Moreover, we use a uniform distribution to simulate the word reordering. %as it is difficult to determine the relative importance of the words.
For the word insertion with a wide range of inserted words, we inject Gaussian noise into the embeddings. For the word deletion, the embedding vector of each word is either kept or deleted (i.e., set to be zero). %(simulate deleted words). 
%the deleted words are set to be all-zero vectors. 
Hence, we use the Bernoulli distribution to simulate the status of each word. 
%randomize the word deletions with a certain probability. 
Our certified radii for the four attacks are then derived based on the corresponding noise distributions. 
% For each of the four adversarial operations, we theoretically derive the certified radius based on the corresponding randomization. 


%In summary, Text-CRS provides certified robustness for the four fundamental operations without imposing restrictions on model architectures and demonstrates superior accuracy compared to SOTA methods, as outlined in Table~\ref{tab:intro}.  %\hl{briefly discuss Table 1} \xy{(done). The IBP-based method and the RS-based method have no restrictions on the operation and model architecture, respectively. So should I define universality as having no restrictions on both operations and models? I only compared Text-CRS with SAFER and CISS in the evaluation.} \YH{we need to be careful here about over-claiming. It seems good if considering both of these as universal....and then we can add universal to the title, and revise "general" to "universal"}
% Our framework provides provable robustness for text classification models against four operations. 



To further improve the certified accuracy/radius, 
% Finally, 
we also propose a training toolkit incorporating three optimization techniques designed for training. For instance, instead of using isotropic Gaussian noises that can lead to distortion in the word embedding space, we propose to use an anisotropic Gaussian noise and optimize it to enlarge the certified radius.  
%for text classification models, including three simple yet effective training methods to improve the certified accuracy and robustness bound. 

%Notably, these word-level adversarial operations can essentially be treated as word insertions. As a result, our word insertion theorem also applies to the other three word-level adversarial operations, while our refined theorems are more effective for the other three operations. Furthermore, we provide theoretical robustness bounds of models trained with Text-CRS to allow the model owners (defenders) to assess the model's robustness against different operations. 



%We conduct extensive experiments to evaluate our framework Text-CRS on two types of models (i.e., LSTM~\cite{LSTM} and BERT~\cite{devlin-etal-2019-bert}) and three datasets with different tasks. We offer certified accuracy under different certified radii for word-level adversarial operations and certified accuracy against five SOTA adversarial attacks (i.e., TextFooler~\cite{jin2020bert}, WordReorder\cite{moradi2021evaluating}, SynonymInsert\cite{morris2020textattack}, BAE-Insert\cite{garg2020bae}, InputReduction\cite{feng2018pathologies}). Besides, we evaluate and analyze the universality of the word insertion theorem. Our framework outperforms SOTA methods in synonym substitution and provides the first certified robustness results for other word-level adversarial operations. 
% where the adversarial perturbation may exceed the radius threshold.


Thus, our major contributions are summarized below: 

\begin{itemize}[leftmargin=*]
\setlength\itemsep{0.3em}
    \item To our best knowledge, we propose the first generalized framework Text-CRS to certify the robustness for text classification models against four fundamental word-level adversarial operations, covering most word-level textual adversarial attacks. Also, the certification against word insertion can be \emph{universally} applied to other operations. %\hl{mention universality? and enhance?}

    \item We provide novel robustness theorems based on Staircase, Uniform, Gaussian, and Bernoulli smoothing distributions against different operations. We also theoretically derive certified robustness bounds for each operation. 

    % \item We propose an enhanced training toolkit, including three simple yet effective methods to improve certified accuracy and robustness bounds.

    \item To study the deceptive potential of adversarial texts, we apply ChatGPT to assess whether adversarial texts can be crafted to be semantically similar to clean texts.

    \item We conduct extensive experiments to evaluate Text-CRS, including our enhanced training toolkit, on three real datasets (AG’s News, Amazon, and IMDB) with two NLP models (LSTM and BERT). The results show that Text-CRS effectively handles five representative adversarial attacks and achieves an average certified accuracy of $81.7\%$, which is a $64\%$ improvement over SOTA methods. Text-CRS also significantly outperforms SOTA on the substitution operation. Besides, it provides new benchmarks for certified robustness against the other three operations. 
    % (AG’s News~\cite{zhang2015character}, Amazon~\cite{mcauley2013hidden}, and IMDB~\cite{maas2011learning}) with two NLP models (LSTM~\cite{LSTM} and BERT~\cite{devlin-etal-2019-bert})
    % On the certification against synonym substitution, Text-CRS significantly outperforms the SOTA methods. 
\end{itemize}


% !TEX root = ../main.tex
\section{Related Work}\label{sec:sota}
%Explainable Artificial Intelligence (XAI) is receiving growing attention in the last few years. Multiple methods have been proposed, and many taxonomies have been defined. Since interpretable models are explained-by-design, the focus of XAI is on black-box classifiers that are not transparent by design. Therefore, black-box models need post-hoc explanation methods for multiple reasons: fairness and ethics, scientific-advance, and regulatory and audit of models.
Post-hoc explanation methods have become essential for understanding the behavior of black-box machine learning models. One such method is counterfactual explanations~\cite{Wachter2017}, which produces example-based explanations by means  of a counterfactual instance for each instance being classified.
More specifically, counterfactual explanations need to exhibit two key characteristics: they must be \emph{similar} to the original instance while being classified in the opposite class of the original instance.
Numerous methods have been proposed to generate counterfactual explanations that possess these critical characteristics~\cite{guidotti2022counterfactual}.

\spara{Explanations for Graph Classifiers.}
There has been a growing interest in addressing the challenge of explaining graph classifiers, resulting in a surge of the number of proposed methods, providing either local or global explanations.
A recent survey \cite{DBLP:journals/corr/abs-2012-15445} categorizes the main (local or) instance-level techniques into four main classes.
\textit{Gradient/feature-based methods}, such as SA and Guided BP ~\cite{baldassarre2019explainability}, and CAM and Grad-CAM~\cite{pope2019explainability}, aim to evaluate the relevance of each feature in the classification task.
\textit{Perturbation-based methods}, such as GNNExplainer~\cite{DBLP:journals/corr/abs-1903-03894}, PGExplainer~\cite{luo2020parameterized}, ZORRO~\cite{funke2020hard}, GraphMask~\cite{schlichtkrull2020interpreting}, RC-Explainer~\cite{wang2020causal}, SubgraphX~\cite{yuan2021explainability}, measure the impact of the perturbation of the input features on the output of the classifier, to detect the most important features.
Among them, GraphShap~\cite{perotti2022graphshap} generates graph-level explanations by ranking a set of input motifs according to their Shapley values.

\textit{Decomposition methods} for graph neural networks, such as LRP~\cite{baldassarre2019explainability}, Excitation BP~\cite{pope2019explainability} and GNN-LRP~\cite{schnake2020higher}, generate feature importance scores by back-propagating decomposed prediction scores to the input layer of the network.
\textit{Surrogate methods}, such as GraphLime~\cite{huang2022graphlime}, RelEx~\cite{zhang2021relex}, and PGM-Explainer~\cite{vu2020pgm}, fit an interpretable model in the neighborhood of the input graph.

Only a few works provide (global or) model-level explanations.
Among them, XGNN~\cite{yuan2020xgnn} is based on graph generation.

\spara{Counterfactual Explanations for Graph Classifiers.}
DBS and OBS~\cite{countg} propose heuristics to locally perturb a generic input graph.
Specifically, they consider two types of modifications: edge addition and edge removal.
The counterfactual explanations are found using a bidirectional search approach that first identifies a feasible counterfactual graph, and then modifies the candidate graph to make it more similar to the input graph.
% The method applied to a set of graph can also provide insights on the model and class level logic of the black-box classifier.
On the other hand, targeted approaches have been proposed for molecular graphs~\cite{DBLP:journals/corr/abs-2102-03322,wellawatte2022model}, which are graphs where nodes represent atoms and edges are bonds.

CF-GNNExplainer~\cite{DBLP:journals/corr/abs-2102-03322} is a counterfactual version of GNNExplainer~\cite{DBLP:journals/corr/abs-1903-03894} that returns relevant subgraphs as explanations. This method removes edges using a matrix sparsification technique that  minimizes the number of edges changed.
MMACE~\cite{wellawatte2022model} generates counterfactuals for molecular graphs by exploring the chemical space vis the Superfast Traversal, Optimization, Novelty, Exploration and Discovery (STONED) method. In addition, the method uses DBSCAN to generate multiple counterfactuals.

%This paper proposes a novel method, called Density-based Counterfactual Search, for discovering relevant counterfactual graphs. This method is capable of generating counterfactual statements that use density as the basic building block of the vocabulary of the explanation. This creates a coherent language for many graph structures, including brain networks, and provides a powerful tool for generating meaningful and interpretable explanations.
% Moreover, as in \cite{countg}, we provide a human readable statement useful as simple and specific explanation of a single classification output.

% \mpara{Counterfactual Explanations Vs Adversarial Attack}
% Given a black-box model $f(\cdot) \in \mathcal{F}$ and a data point $x \in \mathcal{X}$, both Counterfactual Explanation (CE) methods and Adversarial Explanation (AE) methods aim to generate a modified version $x' \in \mathcal{X}$ of $x$ that is classified by $f$ in the opposite class, i.e., $f(x') = 1 - f(x)$.
% % \footnote{To simplify the discussion we consider a simple binary classification problem, the multiclass case can be extended without important considerations.}.
% \cite{browne2020semantics} highlights an apparent paradox: \textit{what is the difference between CEs and AEs}?
% % Therefore, there is no clear consensus on the difference and boundaries between CEs and AEs \cite{pawelczyk2022exploring,freiesleben2022intriguing,browne2020semantics}.
% The first difference is in their purpose. AEs are perturbations to the input that change the output of the model.
% In contrast, CEs are modified versions of the input that aim to explain and interpret the black-box model.
% The second distinguishing factor between AEs and CEs lies in the \textit{distance} between the original and perturbed instances. While AEs strive to go unnoticed by the classifier, CEs are modifications that aim to be sparse~\cite{freiesleben2022intriguing}.
% % In fact, the optimization function can be generally defined in the same way \cite{freiesleben2022intriguing}:
% % $$
% % \argmin_{x' \in \mathcal{X}} d_{\mathcal{X}}(x,x') + \lambda d_{\mathcal{F}}(f(x'),f(x))
% % $$
% % Where $x \in \mathcal{X}$ is the input data point, $f(\cdot)$ is the black-box function, $x' \in \mathcal{X}$ is the adversarial or counterfactual instance, the first term $d_{\mathcal{X}}(x,x')$ measures the difference between the original and perturbed instance, meanwhile, $d_{\mathcal{F}}(f(x'),f(x))$ quantifies the difference in the predicted output.
% % In \cite{Wachter2017}, the difference between CEs and AEs is in the definition of the distance $d_{\mathcal{X}}(\cdot,\cdot)$ between the input data point. Therefore, they define different measure of distance to both minimize the change of the feature values and the number of perturbed features. However, this point cannot be considered as sufficient for explanation, for instance in computer vision the change in a single pixel can be used to flip the classifier enough \cite{su2019one}.
% %grath2018interpretable
% %However, the latter simplification based on the distance is not totally accepted, "do not fundamentally change the nature of the method" and so do not justify the explanatory divide \cite{browne2020semantics}.
% %The common denominator is that the change in the input should be small.
% The third difference~\cite{Wachter2017,browne2020semantics} relates to the "possible world" hypothesis: AEs may be data points that lie outside the data distribution (infeasible "world"), while CEs aim to be feasible data points similar to the original instances.
% %to write better
% The last distinguishing factor is interpretability: CEs aim to be concise, minimal, and human-understandable, while AEs are unconstrained. 

This work proposes a more general framework for counterfactual graph generation that goes beyond existing approaches such as DBS, OBS~\cite{countg} and CF-GNNExplainer~\cite{DBLP:journals/corr/abs-2102-03322}.
While previous works primarily focused on modifying the structure of the original graph by adding or removing one edge at a time, our framework provides more fine-grained control over the graph modifications, as it operates on the dense and sparse regions of the graph.
This opens up possibilities for generating counterfactuals for various scenarios.
\section{Preliminaries}\label{sec:problem}
Given a set of nodes $V$ we denote $\mathcal{G}$ the set of all possible graphs defined over $V$. Given one such graphs $G = (V,E) \in \mathcal{G}$, a \emph{subgraph} of $G$ is a graph $H = (V_H, E_H)$ such that $V_H \subseteq V$ and $E_H \subseteq E$.
A subgraph $H$ is a \emph{k-clique} iff $|V_H| = k$ and $E_H = V_H \times V_H$.
The \emph{density}\footnote{Density is usually defined as the number of edges over the number of possible edges. W.l.o.g. we omit the denominator.} $\delta(H)$ of a subgraph $H$ is defined as its number of edges, i.e., $\delta(H) = |E_H|$.

We assume we are given a binary graph classification model $f: \mathcal{G} \rightarrow \{0,1\}$, that assigns a label in $\{0,1\}$ to each graph in $\mathcal{G}$.
We assume that $f$ \textbf{(i)} is a trained machine learning model whose internal structure is not known (black-box model), \textbf{(ii)} can be queried at will, and \textbf{(iii)} does not change from one query to the other one (i.e., it is static).

Given a specific graph $G \in \mathcal{G}$ a \emph{counterfactual graph} is another graph $G' \in \mathcal{G}$ such that  $f(G) = 1 - f(G')$. Depending on the domain at hand, several desired properties might guide the search for counterfactuals, such as, e.g., \emph{similarity} between the original and the counterfactual instance, \emph{sparsity} (the change affects only a few features), \emph{efficiency} (the search should be fast), and the \emph{feasibility} (to generate a feasible instance).
We will discuss some of these measures in Section \ref{sec:results}.
For the moment, we only need to define the \emph{distance} $\textsf{d}(G, G')$ between two graphs $G$ and $G'$ as the symmetric difference between their edge sets:
\begin{equation}\label{eq:xor}
\textsf{d}(G, G') = |E \setminus E'| + |E' \setminus E|\enspace.
\end{equation}


We next introduce a novel framework to generate counterfactual graphs based on the manipulation of dense substructures, which become the fundamental units of the vocabulary of the explanations produced.

 
%Specifically, the framework leverages cliques as the unit of change in the input graph, improving the edge-based method proposed in \cite{countg}.
%We show that dense structures encode more information on the graph, and as such, they are more appropriate for generating human-interpretable explanations for this kind of data.
%Moreover, our framework can provide a set of diverse explanations, which has been demonstrated in previous studies to be advantageous for users~\cite{mothilal2020explaining}.

% To generate better counterfactual, depending on the context there are additional desiderata for $G'$,

% An example of density-based counterfactual explanations generable by our framework for a brain network is the following:
% \begin{displayquote}
%     \emph{A patient $X$ is classified as \texttt{Autism Spectrum Disorder}; however if the Posterior Fossa area had more connections between its neuronal units, $X$ would have been classified as \texttt{Typically Developed}.}
% \end{displayquote} 
%%%%%%%%%%%
% Check the algorithms
% variation of the method in modularity: backward, triangles, ...
% add other benchmark method
%%%%%%%%
\section{Density-based Counterfactual Search} \label{sec:dcs}
We next introduce our general \emph{Density-based Counterfactual Search} framework (\dcs), which builds instance-level counterfactual explanations by iteratively searching for sparse regions to densify and for dense regions to sparsify.
Pseudocode of {\dcs} is provided in \Cref{alg:db_counter}.

\begin{algorithm}[thb]
\caption{\dcs}
\small
\label{alg:db_counter}
    \begin{algorithmic}[1]
    \Require Binary Graph Classifier $f$; Graph $G$
    \Ensure Counterfactual $G'$
    \State $G' \gets G$
    \While{$f(G) = f(G')$}
        \State \textsc{Densify} a sparse region in $G'$
        \State \textsc{Sparsify} a dense region in $G'$
    \EndWhile
    \State \Return $G'$
    \end{algorithmic}
\end{algorithm}

The algorithm iteratively changes the input graph $G$ until the modified graph $G'$ is classified in the opposite class of $G$.
At each iteration, it adds a dense structure to a sparse region in $G'$ and removes a dense structure from a dense region in $G'$.
Since two different regions of the graph undergo changes at each iteration, $G'$ gradually diverges from the original graph $G$ as the number of iterations increases.
In generating counterfactual graphs, a commonly desired objective is to produce graphs that closely resemble the input graphs. This is because such counterfactual graphs are more likely to provide a concise and interpretable explanation.
For this reason, the algorithm returns the counterfactual graph found in the smallest number of iterations.

\Cref{alg:db_counter} can accommodate any definition of a dense substructure.
In the rest of this section, we introduce two alternative approaches for defining the operations of densification and sparsification.
The first approach, \tri, is based on triadic closure; while the second approach, \cli, is based on maximal cliques.
%In the context of brain networks, \cite{countg} proposed a counterfactual generation method based on the most fundamental unit of change, i.e., the edge.
%In this work, we adapted this method to ensure comparability within the density-based search framework.
%For a comprehensive analysis, please refer to \Cref{sec:local} and \Cref{sec:global}.

% \subsection{Edge-based Search }
% As proposed in a previous work \cite{countg}, the edge-level semantic of change can be adapted as smallest unit of change for graph.
% We refer to \cite{countg} for more details and consideration.

\subsection{Triangle-based Counterfactual Search}\label{sec:tri}
The \emph{Triangle-based Counterfactual Search} ({\tri}) is illustrated in \Cref{alg:TSearch}.
In addition to the original graph $G$ and the classifier $f$, {\tri} takes as input a sorted list of candidate edges to remove $E_-$ (to destroy triangles) and a sorted list of candidate edges to add $E_+$ (to create triangles) in the counterfactual graph. These lists are prepared using \Cref{alg:TScore} (discussed below).
\tri iterates over the two lists until the graph becomes a counterfactual graph for $G$.
At each iteration $i$, it selects the next best edge to add ($e_+$) and to remove ($e_-$) from the current graph $G_i$.
If all the possible wedges have been closed, or if all the possible triangles have been opened (i.e. there are no more edges available in either $E_{-}$ or $E_{+}$), but $G_i$ is still classified in the same class as $G$, the algorithm returns $\emptyset$, indicating that a counterfactual could not be found.

\begin{algorithm}[thb]
    \small
    \caption{\tri}
    \label{alg:TSearch}
    \begin{algorithmic}[1]
    \Require Binary Graph Classifier $f$; Graph $G$
    \Require Sorted lists of candidate edges to remove $E_{-}$ and to add $E_{+}$
    \Ensure Counterfactual $G'$ if found; $\emptyset$ otherwise
    \State $G_0 \gets G$; $i \gets 1$
    \While{$f(G_{i-1}) = f(G)$ \textbf{and} $i \leq \min\left(|E_{-}|, |E_{+}|\right)$}
        \State $e_{-} \gets \Call{nextBest}{E_{-}}$; $e_{+} \gets \Call{nextBest}{E_{+}}$
        \State $E_i \gets E_{i-1} \setminus \{e_{-}\} \cup \{e_{+}\}$; $i \gets i + 1$
    \EndWhile
    \If{$f(G_i) \neq f(G)$} \Return $G_i$
    \Else{ \Return $\emptyset$ }
    \EndIf
    \end{algorithmic}
\end{algorithm}

Given a graph $G$, \Cref{alg:TScore} first computes a score for each feasible edge $(u,v) \in V\times V$, and then partitions the edges into two lists: a list of existing edges that could be removed ($E_{-}$) and a list of non-existing edges that could be added ($E_{+}$).
Finally, the algorithm sorts both lists based on the number of triangles in $G$ that contain the vertices of each edge.
In particular, $E_{-}$ is sorted in ascending order, while $E_{+}$ is sorted in descending order.
This sorting strategy ensures that {\tri} adds triangles in sparse regions of $G$, and removes triangles from the dense regions.

\begin{algorithm}[thb]
    \small
    \caption{\textsc{Triangle Score}}
    \label{alg:TScore}
    \begin{algorithmic}[1]
    \Require Graph $G$
    \Ensure Sorted lists of candidate edges to remove $E_{-}$ and to add $E_{+}$
    \State $\mathsf{T}(v) \gets$ number of triangles including $v$, $\forall v \in V$
    \State $E_{-} \gets E_{+} \gets \emptyset$
    \For{$(v,u) \in V\times V$}
    %\inote{Not clear how you use $s_v$ and $s_u$}
        \State $s_v \gets \mathrm{T}(v)$; $s_u \gets \mathrm{T}(u)$
	\If{$(v,u) \in E$}
		$E_{-} \gets E_{-} \cup \{(s_v+s_u,(v,u))\}$
        \Else
            { $E_{+} \gets E_{+} \cup \{(s_v+s_u,(v,u))\}$ }
%\inote{Not clear how do you ensure that these edges close triangles}
	\EndIf
    \EndFor
    \State $Sort_s (E_{-})$ in ascending order of score;
    \State $Sort_s (E_{+})$ in descending order of score;
    \State \Return $(E_{-}, E_{+})$
    \end{algorithmic}
\end{algorithm}

\subsection{Clique-based Counterfactual Search}\label{sec:cli}
The \emph{Clique-based Counterfactual Search} ({\cli}), illustrated in \Cref{alg:DS}, follows a structure similar to \Cref{alg:db_counter} but employs several heuristics to speed up the search for a counterfactual graph $G'$ for $G$.
The algorithm receives in input two additional parameters: the maximum number of iterations $\textsf{max}_I$, and the list of nodes in $G$ ranked according to a metric that gives more importance to nodes that belong to dense regions in $G$.
At each iteration $i$, the algorithm adds a clique to a sparse region of the current graph $G_i$ around the next worst node in the ranking $\bar{n}_h$ and removes a maximal clique from a dense region in $G_i$ around the next best node in the ranking $\bar{n}_l$.
The algorithm terminates when either $G_i$ is classified in the opposite class of $G$ or the maximum number of iterations $\textsf{max}_I$ is reached.
The densification of a sparse region is carried out by \Cref{alg:HS} (\textsc{DensifyCLI}) and the sparsification by \Cref{alg:LS} (\textsc{SparsifyCLI}).
In the following, a clique is represented by its set of vertices, as its set of edges is the set of all the possible edges between such nodes.

\begin{algorithm}[thb]
    \small
    \caption{\cli}
    \label{alg:DS}
    \begin{algorithmic}[1]
    \Require Binary Graph Classifier $f$; Graph $G$
    \Require Max Num Iteration $\textsf{max}_I$; Sorted list of vertices $\bar{V}$
    \Ensure Counterfactual $G'$ if found; $\emptyset$ otherwise
    \State $D[v] \gets 0$, $\forall v \in V$
    \State $G_0 \gets G$; $\mathcal{L} \gets \emptyset$; $i \gets 1$
    \While{$f(G_{i-1}) = f(G)$ \textbf{and} $i \leq \textsf{max}_I$}
        \State $\bar{n}_l,\bar{n}_h \gets \textsc{nextBest}(\bar{V}), \textsc{nextWorst}(\bar{V})$
	\State $G_i \gets \Call{SparsifyCLI}{G, G_{i-1}, \bar{n}_l, D, \mathcal{L}}$
	\State $d_h\gets 0$; $s \gets d_l \gets \textsf{d}(G_{i-1}, G_{i})$\label{line:xor}
	\While{$f(G_i) = f(G)$ \textbf{and} $d_l > d_h$}
            \State $s \gets s - d_h$
		\State $G^h \gets \Call{DensifyCLI}{G_{i},\bar{n}_h,D,s}$
		\State $d_h \gets \textsf{d}(G_{i}, G^h)$; $G_{i} \gets G^h$\label{line:xor2}
	\EndWhile
	\State $i \gets i + 1$		
    \EndWhile
    \If{$f(G_i) \neq f(G)$} \Return $G_i$
    \Else{ \Return $\emptyset$ }
    \EndIf
    \end{algorithmic}
\end{algorithm}

Procedure~\textsc{SparsifyCLI} identifies a maximal clique in the input graph $G$ surrounding a given node $n$, and removes all the edges in that clique in the candidate counterfactual graph $G'$.
The algorithm operates by identifying the largest clique in $G$ including $n$ that has the lowest overlap with the cliques removed in prior iterations.
By choosing the largest, lowest-overlap clique, the algorithm sparsifies one of the densest regions in $G'$.
Note that the cliques considered by the algorithm are found in the original graph $G$, and some of their edges may have already been removed from $G'$ in previous iterations of \textsc{SparsifyCLI}.
After the desired clique $C$ has been identified, the algorithm removes all its edges from $G'$ and stores $C$ in the set of cliques removed $\mathcal{L}$. Finally, the counts associated with each node in the clique are incremented by 1.

\begin{algorithm}[thb]
    \small
    \caption{\textsc{SparsifyCLI}}
    \label{alg:LS}
    \begin{algorithmic}[1]
    \Require Graph $G$; Candidate Counterfactual $G'$
    \Require Node $n$; Dictionary $D$; Set of Cliques $\mathcal{L}$
    \Ensure $G'$ with a clique removed
    \State $\mathcal{C} \gets$ cliques in $G$ including $n$
    \If{$\mathcal{L} = \emptyset$}
	\Return largest clique in  $\mathcal{C}$
    \EndIf
    \State $O \gets \emptyset$
    \For{$C \in \mathcal{C}$}
	\State $o \gets \max_{L \in \mathcal{L}}{(|C \cap L|)}$; $O \gets O \cup \{(o, C)\}$
    \EndFor
    \State $C \gets $ clique in $O$ with lowest value $o$
    \State $\mathcal{L} \gets \mathcal{L} \cup \{C\}$
    \State $D[v] \gets D[v] + 1$ for each $v$ in $C$\label{line:dict}
    \State \Return $G' \setminus \{$ edges in $C \,\}$
    \end{algorithmic}
\end{algorithm}

\begin{algorithm}[thb]
    \caption{\textsc{DensifyCLI}}
    \label{alg:HS}
    \begin{algorithmic}[1]
    \Require Graph $G$; Node $n$; Dictionary $D$; Size $s$
    \Ensure $G$ with a clique added
    \State $\Gamma_G(n) \gets$ 2-hop neighborhood of $n$ in $G$ sorted according to $D$ (ascending)\label{line:2hop}
    \State $W \gets V \setminus \Gamma_G(n)$ sorted according to $D$ (ascending)\label{line:others}
    \State $C \gets \Call{concat}{\Gamma_G(n), W}$; $V_s \gets $ first $s$ nodes in $C$
    \State $D[v] \gets D[v] - 1$ for each $v \in V_s$
    \State \Return $G \cup \{$ edges between the nodes in $V_s \,\}$
    \end{algorithmic}
\end{algorithm}

Procedure~\textsc{DensifyCLI} is iteratively called until the dense region added to the candidate counterfactual graph has at least as many edges as the dense region removed by Algorithm~\ref{alg:LS}~\footnote{In our experiments we constrained the max deviation between the number of edges added and removed in terms of the max number of nodes $b$ that a clique added can have with respect to the number of nodes in the clique removed, and set $b = 10$.}.
This ensures that the size of the counterfactual is similar to that of the original graph.
At each iteration, the algorithm identifies a sparse region of size $s$ around the given node $n$ and adds all the possible edges between the $s$ nodes.
The size of the region $s$ is determined by subtracting the number of edges added in the previous iterations from the number of edges removed by Algorithm~\ref{alg:LS}.
To avoid densifying a region that has just been sparsified, the algorithm selects a sparse region involving nodes that are not present in many cliques added in previous iterations.
To achieve this, the algorithm uses a dictionary $D$, which keeps track of the number of times each node has been part of a clique added to the candidate counterfactual.
Given a node $n$, \textsc{DensifyCLI} sorts both the 2-hop neighborhood $\Gamma_G(n)$ of $n$ and the rest of the vertices $W$ according to their counts in $D$ (Algorithm~\ref{alg:HS} lines~\ref{line:2hop}-\ref{line:others}).
Then, it adds to $G$ the clique $C$ consisting of the first $s$ nodes in the concatenation between $\Gamma_G(n)$ and $W$, and updates $D$ by decreasing the counts associated with the nodes in $C$.



\spara{Further customizability of the framework.}
In \Cref{alg:DS}, we made specific design choices that, however, can be customized and adapted to meet the needs of the application at hand.

Firstly, \Cref{alg:DS} receives as input the list $\bar{V}$  of nodes in $G$ ranked according to a metric that prioritizes nodes in dense regions for \Cref{alg:LS} and nodes in sparse regions for \Cref{alg:HS}. In our implementation of \cli\ used in the experiments in Section \ref{sec:results} we sort the nodes in $\bar{V}$ based on the number of triangles in which each node participates.
However, alternative measures could be used to rank the nodes, such as the node clustering coefficient or other features at node level.
The key is to select a ranking that allows for traversing in one direction to be a good heuristic for sparsifying regions while traversing in the other direction is a good heuristic for densifying regions.

When domain-specific information is available, it is important to customize the algorithm to take such information into consideration. In the case of brain networks, for instance, 
nodes can be partitioned into well-defined and distinct regions (i.e. the brain lobes).
In \Cref{sec:results}, we explore a variation of \cli that selects the regions to sparsify/densify based on the brain lobes, which we refer to as \rcli.
The {\rcli} algorithm uses a two-level ranking strategy that first ranks the brain lobes according to the density of the subgraph induced by their nodes (regions having higher density ranked higher) and then, within each region, ranks the nodes according to the number of triangles in which they participate.
This two-level ranking allows \rcli to conduct the edge changes within a lower number of brain regions and thus generate more interpretable explanations.

%Secondly, in \Cref{alg:HS} we identify a sparse region around a node $n$ in the 2-hop neighborhood of $n$, with the aim of increasing the density of a region that is at least partially connected. However, in some scenarios, it may be preferable to connect nodes that are currently disconnected.

Secondly, for the sake of \emph{feasibility} of the counterfactual, i.e., keeping its density similar to the original graph, we alternate between 
\Cref{alg:LS} and \Cref{alg:HS}, meaning that we call them the same number of times.
However, while this strategy is effective in most cases, it may not be the optimal approach.
%This approach allows the algorithm to move dense structures from one region of the counterfactual to another.
%However, if the original graph is very sparse, a preferable strategy would be to call \Cref{alg:HS} more often, whereas if the input is dense, \Cref{alg:LS} may be called more frequently or using more edge in the change.
An alternative approach might let the density of the original graph govern the calls to \Cref{alg:LS} and \Cref{alg:HS}, so that when the graph is very sparse, \Cref{alg:HS} is called more often than \Cref{alg:LS}, and the other way around.
\section{Experimental Evaluation} \label{sec:results}
We next showcase the application of density-based counterfactuals in the context of brain networks, highlighting the high interpretability of such counterfactuals.

\subsection{Brain Networks}
Brain networks can be constructed using non-invasive techniques such as Functional Magnetic Resonance Imaging (fMRI) in resting-state patients.
By measuring blood flow, fMRI exploits the link between neural activity and blood flow and oxygenation, to associate a time series of activation scores at voxel level.
The voxels' signals are parcellated into Regions of Interest (ROIs) (\textit{nodes of the graph}) using specific templates, such as the Automated Anatomical Labeling (AAL)~\cite{tzourio2002automated} or the 200~\cite{craddock2012whole} parcellation scheme.
Then, interactions between ROIs (\textit{edges of the graph}) are identified by looking at the correlation between the corresponding time series.
Finally, relevant interactions are selected by applying a threshold (edge pruning), to obtain the brain's functional connectome.
ROIs can be further aggregated into areas associated with the lobes of the brain. As discussed before, this aggregation can be exploited to express interpretable density-based counterfactual explanations.

We consider seven publicly available brain network datasets.

\spara{AUT} is a dataset gathered within the Autism Brain Image Data Exchange (ABIDE) \cite{craddock2013neuro} project. This dataset includes brain network data from 49 patients with Autism Spectrum Disorder (ASD, condition group) and 52 Typically Developed (TD, control group) patients, all under the age of 9 years old.

\spara{BIP} dataset about lithium response in type I bipolar disorder patients~\cite{sani2018association}.

\spara{ADHD, ADHDM} come from the Multimodal Treatment of Attention Deficit Hyperactivity Disorder project\footnote{\url{http://fcon\_1000.projects.nitrc.org/indi/ACPI/html/acpi\_mta\_1.html}}, which investigated the impact of cannabis use on adults with or without a childhood diagnosis of ADHD. In the ADHD dataset, subjects are labeled as either ``ADHD'' or ``TD'', while in the ADHDM dataset, they are labeled as ``Marijuana use'' or ``Marijuana not used''.
  
\spara{OHSU, PEK, KKI}\footnote{\url{https://github.com/GRAND-Lab/graph_datasets}} are datasets constructed for three brain classification tasks: Attention Deficit Hyperactivity Disorder classification (OHSU), Hyperactive Impulsive classification (PEK), and gender classification (KKI)~\cite{pan2016task}.

\smallskip

Data is preprocessed following the literature for converting time series to correlation matrices \footnote{See \url{http://preprocessed-connectomes-project.org/abide/dparsf.html} for AUT and BIP, and \url{https://ccraddock.github.io/cluster_roi/atlases.html} for ADHD and ADHDM. For OHSU, PEK, and KKI, the data was already preprocessed.}.
To generate the graph dataset, correlation matrices are transformed into adjacency matrices by setting edges when the correlation between the two nodes is higher than a fixed threshold.
The threshold is selected based on the distribution of the correlation matrix values, using the 90th percentile for ADHD and AUT, and 80th for BIP. All the preprocessed graph datasets are available in our repository\footnote{\url{https://github.com/carlo-abrate/Counterfactual-Explanations-for-Graph-Classification-Through-the-Lenses-of-Density.git}}.

\Cref{tab-data} reports, for each dataset, the number of networks, the percentage of networks in class 1 (since we deal with binary classification, we report values for one class only), the total number of vertices and edges in the networks, and the accuracy and F1 score of the binary classifier trained on the dataset (see below).



\begin{table}[t]
    \centering
    \caption{Num. of graphs $|\mathcal{G}|$, percentage of graphs in class $1$, num. of nodes $|V|$, average num. of edges per graph $\mathrm{avg}|E|$, and accuracy $\textsf{ACC}$ and $\textsf{F1}$ score of the binary classifier, for each dataset.}
	\label{tab-data}
	\begin{tabular}{ccccccc}
	\toprule
	\textbf{Dataset} & \textbf{$|\mathcal{G}|$} & \textbf{$|Y_{=1}|$}  & \textbf{$|V|$} & \textbf{$\mathrm{avg}|E|$} & \textsf{ACC} & \textsf{F1} \\
	\midrule
	%AUT-CS & CS + KNN & 101 & $48 \%$ & 116 & 665 & 0.80 & 0.80 \\
	AUT & 101 & $48 \%$ & 116 & 665 & 0.92 & 0.90 \\
		%BIP & SF + KNN & 118 & $46 \%$ & 116 & 1334 & 0.64 & 0.52 \\
        BIP & 118 & $46 \%$ & 116 & 667 & 0.66 & 0.54 \\
	ADHD & 123 & $32 \%$ & 116 & 667 & 0.80 & 0.59 \\
	ADHDM & 123 & $50 \%$ & 116 & 667 & 0.93 & 0.93 \\
	OSHU & 79 & $56 \%$ & 190 & 199 & 0.68 & 0.72 \\
	PEK & 85 & $42 \%$ & 190 & 77 & 0.71 & 0.58 \\
	KKI & 83 & $55 \%$ & 190 & 48 & 0.66 & 0.68 \\
	\bottomrule
    \end{tabular}
\vspace{3mm}
\end{table}

\mpara{Classifier.}
The proposed framework is model-agnostic, making it suitable for explaining any kind of binary classifier.
In our experiments, we consider a binary classifier designed for graph classification that exploits the \emph{Spectral Features} (SF)~\cite{sfclassifier} of the graph to determine class memberships.
Let $A \in \{0,1\}^{|V|\times |V|}$ be the adjacency matrix of the graph, $D$ be the diagonal matrix of node degrees, and $L = I - D^{-1/2} A D^{-1/2}$ be the normalized Laplacian of $A$.
The SF of the graph is a vector consisting of the $k$ smallest positive eigenvalues of $L$, sorted in ascending order.
We trained a KNN classifier with various parameter settings and selected the optimal configuration based on the accuracy (\textsf{ACC}) and \textsf{F1} score using 5-fold cross-validation. The values of \textsf{ACC} and \textsf{F1} of the configurations selected are reported in the last two columns of \Cref{tab-data}.

\subsection{Metrics}\label{sec:metrics}
Various metrics have been proposed to evaluate the quality of counterfactual explanations \cite{guidotti2022counterfactual}. The selection of which measures to prioritize over others depends on factors such as the data type, the black-box models considered, and the vocabulary used to formulate the counterfactual statements.
We consider three measures specifically proposed to evaluate graph counterfactuals~\cite{prado2022survey}.

\spara{Flip rate:}
measures the percentage of graphs in the dataset for which the algorithm was able to find a counterfactual explanation~\cite{mothilal2020explaining,prado2022survey}.

\spara{Edit distance:} 
measures how \emph{different} is a graph $G$ from its counterfactual $G'$, and, in our case, is defined as the ratio between the symmetric difference of the edge sets of $G$ and $G'$ (\Cref{eq:xor}) and $|E \cup E'|$:
\[
d_{\%}(G,G') = \frac{\mathsf{d}(G,G')}{|E \cup E'|}\,.
\]
\spara{Calls:}
run-time complexity of a counterfactual search method measured in terms of the number of calls to the black-box model ($C$).

\subsection{Baselines}
We compare the performance of \tri, \cli, and \rcli against three baseline methods. The first baseline, \edg~\cite{countg}, utilizes an edge-based language to generate counterfactual explanations. The second baseline, \dataset, is an instance-level counterfactual search method proposed in \cite{guidotti2022counterfactual}. This method searches for the closest graph in the dataset that is classified by the black-box model in the opposite class and returns it as a counterfactual explanation for the input graph. 

Following~\cite{countg} we also equip the \rcli and \dataset methods with a \emph{backward search} phase which tries to refine the counterfactual found by 
modifying the edges in the symmetric difference between the edge set of the input graph and that of the counterfactual graph, with the aim of reducing the distance between the two graphs. The resulting methods are named \rclibw and \datasetbw respectively.

All the methods are implemented in Python and the code is made publicly available\footnote{\url{https://github.com/carlo-abrate/Counterfactual-Explanations-for-Graph-Classification-Through-the-Lenses-of-Density}} together with the datasets used in our analysis, and a supplemental material document containing further experimental results.
%\subsection{Validation of \dcs}
% \mynote[from=Carlo]{there are a multiple synthetic datasets that are used as benchmarks, maybe it's better to use them instead of ours?}
%Synthetic data Generation and results
We validate {\dcs} in two synthetic datasets equipped with a white-box binary classifier.
The graph generation process takes in input a set of nodes $V$ and a number of graphs $N$.
The set of nodes is partitioned into two subsets $S_0$ and $S_1$ of equal size, and then, for each class $i \in \{0, 1\}$, $N/2$ graphs are generated according to the following procedure:
\textbf{(i)} select one (1SG) or two (2SG) subgroups of nodes in $SG \subseteq S_i$, 
\textbf{(ii)} build $c$ random cliques among nodes within the same subgroup,
\textbf{(iii)} apply a Barabási–Albert preferential attachment algorithm \cite{albert2002statistical} with parameters $m$, $p$, and $q$, to add structure within $S_{1-i}$ and between $S_i$ and $S_{1-i}$
\footnote{We set $N_{1SG} = 100$ and $N_{1SG} = 200$; $SG_{1SG} = 1/4 \cdot |V|$ and $SG_{2SG} = 1/8 \cdot |V|$; $c_{1SG} = 10$ and $c_{2SG} = 20$; $m = SG$, $p = 5$ and $q = 0.7$.}.

This way, for each $i$, we obtain graphs with dense regions in $S_i$, but sparse in $S_{1-i}$.
\ref{fig:synt} illustrates the dense regions in the graphs for each class, in terms of triangle count at node level. The upper chart shows the average triangle count for the synthetic dataset 1SG, while the lower chart shows the same quantity for 2SG.

% Figure environment removed

Then, we build a white-box classifier using the same logic of the graph generation process, to achieve perfect accuracy.
\mynote[from=Carlo]{Is it fine to use this very simple white box classifier? Is it better to train a black-box model?}
%Explain in the appendix the classifier algorithm

\begin{table}[thb]
	\centering
	\caption{Quartiles of symmetric difference (percentage) $d_{\%}$, and number of iterations $I$, for the synthetic datasets 1SG and 2SG.}
	\label{tab1}
	\begin{tabular}{ccrrrrr}
		\toprule
		\textbf{Dataset} &  \textbf{Measure} & \textbf{Q0} & \textbf{Q1} & \textbf{Q2} & \textbf{Q3} & \textbf{Q4} \\ 
		\midrule
		\multirow{2}{*}{1SG} 
		& $d_{\%}$ & 18.5 & 39.3 & 55.5 & 80 & 100 \\
		%& $d_{cl}$ & 0.5 & 0.65 & 0.7 & 0.77 & 0.92 \\
		% & Cl. sort & -2.01 & 0.14 & 0.38 & 0.51 & 0.71 \\
		& $I$ & 1 & 2 & 3 & 6 & 21 \\
      \hline
		\multirow{2}{*}{2SG} 
		& $d_{\%}$ & 42.1 & 81.6 & 100 & 100 & 100 \\
		%& $d_{cl}$ & 0.43 & 0.52 & 0.57 & 0.61 & 0.73 \\
		% & Cl. sort & -3.52 & -1.33 & -0.35 & 0.3 & 0.68 \\
		& $I$ & 2 & 3 & 6 & 9 & 19 \\
		\bottomrule
	\end{tabular}
\end{table}

\ref{tab1} summarizes the results of the validation of \dcs.
In details, it reports the quartiles of the symmetric differences (percentage of edges modified) $d_{\%}$, the closeness $d_{cl}$ between the original and the counterfactual graphs, and the number of iterations $I$, for both 1SG and 2SG.
The flip rate is $100\%$ in both datasets. 
However, the symmetric difference $d_{\%}$ is quite high, especially in the 2SG dataset, where in more that half of the cases of the edges are changed.

\ref{fig:hm_bi} illustrates the evolution of the dense regions in one graph from the 2SG dataset at each iteration of \dcs.


% Figure environment removed

\mynote[from=Giulia]{Elaborate on the results in \ref{tab1} and \ref{fig:hm_bi}.}


\begin{table*}[thb]
	\centering
	\footnotesize
	\caption{}
	\label{tab:iterations}
	\begin{tabular}{c c c ccccc}
		\toprule
		&  &  & \multicolumn{5}{c}{Iterations} \\
	   \cmidrule(lr){4-8}
	   \textbf{Data} & 		
	   \textbf{Type} & 		
	   \textbf{FR0/FR1} & 
	   \textbf{min} & \textbf{25-p} & \textbf{50-p} & \textbf{75p} & \textbf{max} \\
    \midrule
AUT & EDG & 100.0/84.8 & 1 & 24.0 & 40.0 & 57.25 & 2000 \\
AUT & DAT+BW & 100.0/100.0 & 79 & 129.0 & 149.0 & 168.0 & 313 \\
AUT & TRI & 100.0/100.0 & 1 & 9.0 & 22.0 & 52.0 & 610 \\
AUT & CLI & 90.9/100.0 & 2 & 2.0 & 3.0 & 17.0 & 60 \\
AUT & RCLI & 96.4/93.5 & 3 & 6.0 & 11.0 & 83.0 & 671 \\
AUT & DAT & 100.0/100.0 & 102 & 102.0 & 102.0 & 102.0 & 102 \\
  \hline
BIP & EDG & 70.0/100.0 & 1 & 9.0 & 25.0 & 51.0 & 2000 \\
BIP & DAT+BW & 100.0/100.0 & 61 & 127.0 & 201.5 & 384.5 & 1477 \\
BIP & TRI & 100.0/100.0 & 1 & 1.0 & 6.0 & 25.0 & 595 \\
BIP & CLI & 100.0/100.0 & 2 & 2.0 & 6.5 & 8.75 & 20 \\
BIP & RCLI & 94.0/100.0 & 3 & 10.0 & 153.0 & 257.0 & 675 \\
BIP & DAT & 100.0/100.0 & 119 & 119.0 & 119.0 & 119.0 & 119 \\
  \hline
ADHD & EDG & 75.0/100.0 & 1 & 8.0 & 21.5 & 36.0 & 2000 \\
ADHD & DAT+BW & 100.0/100.0 & 73 & 108.0 & 141.0 & 215.0 & 635 \\
ADHD & TRI & 100.0/100.0 & 1 & 2.0 & 5.0 & 19.0 & 330 \\
ADHD & CLI & 100.0/100.0 & 2 & 2.0 & 5.0 & 9.0 & 49 \\
ADHD & RCLI & 95.0/98.4 & 3 & 5.0 & 12.0 & 45.0 & 622 \\
ADHD & DAT & 100.0/100.0 & 124 & 124.0 & 124.0 & 124.0 & 124 \\
\hline
ADHDM & EDG & 33.7/100.0 & 1 & 13.75 & 60.5 & 2000.0 & 2000 \\
ADHDM & DAT+BW & 100.0/100.0 & 71 & 193.0 & 372.0 & 606.0 & 1238 \\
ADHDM & TRI & 98.1/100.0 & 1 & 5.0 & 185.0 & 282.0 & 664 \\
ADHDM & CLI & 40.4/100.0 & 2 & 2.0 & 39.0 & 60.0 & 60 \\
ADHDM & RCLI & 60.6/100.0 & 3 & 8.0 & 307.0 & 548.0 & 632 \\
ADHDM & DAT & 100.0/100.0 & 124 & 124.0 & 124.0 & 124.0 & 124 \\
\hline
OHSU & EDG & 100.0/86.7 & 2 & 38.0 & 67.0 & 117.0 & 2000 \\
OHSU & DAT+BW & 100.0/100.0 & 40 & 100.0 & 149.0 & 196.0 & 889 \\
OHSU & TRI & 61.8/57.8 & 1 & 23.0 & 49.0 & 95.0 & 627 \\
OHSU & CLI & 52.9/86.7 & 2 & 29.0 & 60.0 & 82.0 & 97 \\
OHSU & DAT & 100.0/100.0 & 80 & 80.0 & 80.0 & 80.0 & 80 \\
\hline
PEK & EDG & 100.0/91.3 & 1 & 2.0 & 5.0 & 12.0 & 2000 \\
PEK & DAT+BW & 100.0/100.0 & 10 & 26.0 & 60.0 & 95.0 & 387 \\
PEK & TRI & 88.7/95.7 & 1 & 1.0 & 4.0 & 15.0 & 492 \\
PEK & CLI & 100.0/91.3 & 2 & 2.0 & 4.0 & 20.0 & 97 \\
PEK & DAT & 100.0/100.0 & 86 & 86.0 & 86.0 & 86.0 & 86 \\
\hline
KKI & EDG & 89.7/100.0 & 1 & 1.0 & 2.0 & 10.0 & 2000 \\
KKI & DAT+BW & 100.0/100.0 & 5 & 16.0 & 27.0 & 51.0 & 224 \\
KKI & TRI & 89.7/70.5 & 1 & 1 & 3.0 & 17.0 & 234 \\
KKI & CLI & 89.7/100.0 & 2 & 2.0 & 4.0 & 20.0 & 97 \\
KKI & DAT & 100.0/100.0 & 84 & 84.0 & 84.0 & 84.0 & 84 \\
  \bottomrule
	\end{tabular}
\end{table*}



%%%%%%%%%%%%%%%%%%%%%%%%%%%%%%%%%%%%%%%%%%%%%%%%%%%%%%%%%%%%%%%%%%%%%%%%%%%%%%%%%%%%%%%%%%%
\subsubsection{Configuration of {\cli}}\ref{sec:settings}
\mynote[from=Carlo]{Do we want to keep it? To be updated and aligned with current notation.}{}
\Cref{tab:res} reports the performance of {\cli} in AUT-CS, for different parameter configurations.
It displays the flip Rrate $\mathsf{fr}$, and the percentile of number of iteraton $I$ and distance $d_{\%}$.
The configurations considered are the following:
\begin{itemize}
    \item Node Ranking Strategy $R$: $\mathsf{t}$ sorts the nodes based on the number of triangles, and $\mathsf{eg}$ based on the eigenvector centrality~\cite{10.5555/1809753};
    \item Sorting Strategy for $D$: $0$ means no sorting, $l_{+1}$ means the use of a vector that weights node in \Cref{alg:LS}, and $h_{-1}$ means the use of a vector that weights node in \Cref{alg:HS}.
\end{itemize}

We observe that the quartiles of the two distributions are similar across the configurations, whereas the flip rate $\mathsf{fr}$ shows an increase of $10\%$ between the two extreme settings. We register the following results:
\begin{itemize}
	\item The Node Sorting for Fill Search $D_n$ performs better with only the weights in the Empty Search $l_{+}$.
	\item $k$ shows not much difference in performance: a bit of budget but not too much.
	\item $NS$ the eigenvector centrality sorting $Eg$ reaches the best $\mathsf{fr}$.
\end{itemize}

\begin{table}[t]
    \centering
    \small
    \caption{Flip rate $\mathsf{fr}$, and quartiles of computation time $I$ and closeness $d_{\%}$, for different parameter configurations of {\cli}, in AUT-CS.}
    \label{tab:res}
    \begin{tabular}{c ccccc ccccc ccc}
	\toprule
	& \multicolumn{5}{c}{$I$} & \multicolumn{5}{c}{$d_{\%}$} & & &\\
	\cmidrule(lr){2-6} \cmidrule(lr){7-11}
		$\mathsf{fr}$ & 
		\textbf{Q0} & \textbf{Q1} & \textbf{Q2} & \textbf{Q3} & \textbf{Q4} & 
		\textbf{Q0} & \textbf{Q1} & \textbf{Q2} & \textbf{Q3} & \textbf{Q4} &
		\textbf{$D$} & \textbf{$k$} & \textbf{$R$} \\
		\midrule
		73.3 
		& 1 & 2 & 6 & 15 & 56 
		& 1.9 & 9.3 & 19.2 & 27.3 & 48.7 
		& \textbf{$0$} & 2 & $\mathsf{t}$\\
		68.3 
		& 1 & 2 & 8 & 15 & 58 
		& 1.9 & 9.3 & 19.7 & 29.9 & 55.8 
		& $l_{+1}$ & \textbf{4} & $\mathsf{t}$\\
		76.2 
		& 1 & 2 & 6 & 16 & 56 
		& 1.9 & 9.3 & 19.7 & 28.3 & 54.6 
		& $l_{+1}$ & 2 & $\mathsf{t}$\\
		78.2 
		& 1 & 2 & 7 & 18 & 56 
		& 1.9 & 10 & 20.1 & 32.8 & 53 
		& $l_{+1}$ & \textbf{0} & $\mathsf{t}$\\
		\textbf{79.2} 
		& 1 & \textbf{1} & \textbf{5} & 14 & 57 
		& 2.7 & \textbf{7.3} & 20.6 & 32 & 53.4 
		& $l_{+1}$ & 2 & \textbf{$\mathsf{eg}$}\\
		71.3 
		& 1 & 2 & 6 & 13 & 58 
		& 1.9 & 8.5 & 18.8 & \textbf{26.4} & 53 
		& \textbf{$l_{+1}$,$h_{-1}$} & 2 & $\mathsf{t}$\\
		\bottomrule
	\end{tabular}
\end{table}




%%%%%%%%%%%%%%%%%%%%%%%%%%%%%%%%%%%%%%%%%%%%%%

The counterfactual graph can be visualized by coloring the edges in the symmetric difference with different colors, as shown in \ref{fig:local_1} (top).
This picture shows the brain connectome of a counterfactual explanation for a graph in ADHDM dataset in the class ``not used'', where the edges in $\{E \setminus E'\}$ (edges removed, for a total of $21$) are denoted in red, and the edges in $\{E' \setminus E\}$ (edges added, for a total of $16$) are denoted in blue.
Each node represents a voxel, and different node colors indicate different areas of the brain.

% Figure environment removed


The bar plot in \ref{fig:local_1} (bottom) shows which region of the brain the edges changed belong to.
In details, it reports the ratio of edges changed by region and by type of change (edges added $E^+$ and edges removed $E^-$).
This plot allow us to analyze which parts of the brain should be sparsified/densified to obtain a brain network classifiable in the ``Marijuana use'' class.
As we can see, we need to remove edges between the Frontal and Parietal Lobe, and add edges in the Posterior Fossa.

% Figure environment removed

% Figure environment removed

\ref{fig:local_2} and \ref{fig:local_3} display the the brain connectome of counterfactual graphs in the TD class for a graph in AUT-CS (\ref{fig:local_2}) and in ADHD (\ref{fig:local_3}).
In the former, the symmetric difference with the original graph is 63 (32 edges removed and 31 added); while in the latter, the symmetric difference is 41 (21 edges removed and 20 added).

We can see that the brain with ASD has fewer connections in the Posterior Fossa than the brain with ADHD, as such region includes most of the edges added (resp. removed) to find the corresponding counterfactual graph.
Moreover, voxels in the Central Structures of the ADHD brain are more interconnected than those in a TD brain.


\subsection{Qualitative analysis}\label{sec:local}
\begin{comment}
%% Old version with 3 raw, where each row is "brain+bar,brain+bar"

% Figure environment removed
\end{comment}

% Figure environment removed

In this section, we compare the counterfactual graphs generated by three instantiations of \dcs, namely \tri, \cli, and \rcli, with those produced by the three baseline methods, \edg, \dataset, and \datasetbw, for specific patients in three datasets. All the results presented pertain to brain networks for which the classifier accurately predicted the class.

\mpara{AUT Dataset.}
\Cref{fig:graph_counterfactual_methods} shows the counterfactual graphs for patient $9$ in AUT. This patient is classified as ``Autism Spectrum Disorder''.
For each method, the left figure shows the connectome of the patient overlaid on the brain glass schematics, where ROIs are projected onto a 2D space of the image, and different colors represent ROIs in different brain lobes. Blue edges indicate edges added to the counterfactual graph, while red edges denote edges removed from the counterfactual graph.
In addition to the connectome visualization, the right barplots illustrate the distribution of changes among brain lobes.
For each brain lobe, the bars report the percentage of the nodes involved in the added (blue) and removed (red) edges that belong to that lobe. By examining these barplots, we can gain insights into which brain regions are most affected by each method's counterfactual graph generation process.
We first observe that each method perturbed different regions of the brain.
This is due to the fact that the set of changes identified by each method depends on a range of factors, including the method's underlying assumptions, its optimization criteria, and its specific implementation.
The choice of counterfactual generation method should take into account the specific properties of the input data and the desired goals of the counterfactual analysis.
One of the desiderata is interpretability.
In general, the larger the number of regions changed and the more homogeneously the changes are distributed within the regions, the less human-interpretable the counterfactual explanation becomes.
Of the methods examined, \dataset produced the most complex explanation, with almost the same number of edges added and removed from each brain lobe.
The \datasetbw method provides a partial solution to this issue by removing edges mainly from the Parietal Lobe and the Temporal Lobe, which reduces the heterogeneity of edge removals across the brain lobes. However, the edge additions still span across many regions, limiting the interpretability of the solution.
The \edg method suffers from similar limitations in that its counterfactual graph involves changes spanning across most of the brain lobes.
In contrast, \tri and \cli provide simpler explanations, as they perturbed a lower number of regions and concentrated most of the changes in the same regions.
Specifically, \tri mainly sparsified the Occipital Lobe and densified the Insula \& Cingulate Gyri, while \cli sparsified only the Frontal Lobe and added most of the edges in the Posterior Fossa.
This results in a more focused and interpretable explanation.
In fact, the output of \cli can be summarized by the following simple \textit{counterfactual statement}:
\begin{displayquote}
\emph{Patient X is classified as Autism Spectrum Disorder. If X's brain had less activation in the \textsc{\color{red} Frontal Lobe} and more co-activation in the \textsc{\color{blue} Posterior Fossa}, \textsc{\color{blue} Insula Cingulate Gyri}, and the \textsc{\color{blue} Temporal Lobe} then X would have been classified as Typically Developed.}
\end{displayquote}

% % Figure environment removed

%We can represent the output of the \dcs as local explanation of the classification of the original graph.

%As discussed in \ref{sec:intro}, \dcs uses a unit of change between the original and the counterfactual graph, based on the dense and sparse regions of the graph.

%\subsection{Global Explanation for Brains Regions}
%Counterfactuals are also effective at providing an overview of the internal logic of black-box classifiers, that is, at giving global explanations.
%We aggregate the edge changes performed by \dcs to generate the counterfactual graph of each graph in the dataset.
%\ref{table:100} shows the frequency of change by type, region, and class (class 0 on the left and class 1 on the right), for AUT-CS (top) and ADHDM (bottom).
\begin{comment}
%%% Old version on 3 row for brain and only 1 for bar

% Figure environment removed
\end{comment}
% Figure environment removed

\mpara{BIP Dataset.}
\Cref{fig:example} shows the counterfactual graphs and the distributions of edges changed among the brain lobes, for patient $9$ in the BIP dataset.
This patient is classified as ``Typically Developed''.
This example serves to confirm the effectiveness of \tri and \cli in generating more compact and interpretable explanations. Specifically, \tri produces a counterfactual graph that closely resembles the input network, with only 10 edges added and 10 edges removed. On the other hand, \cli concentrates its changes in two specific regions: the Parietal Lobe (with connections removed) and the Posterior Fossa (with connections added).
The output of \tri can be summarized by the following simple \textit{counterfactual statement}:
\begin{displayquote}
\emph{Patient X is classified as Typically Developed. If X's brain had less activation in the \textsc{\color{red} Parietal Lobe} and the \textsc{\color{red} Insula Cingulate Gyri}, and more co-activation in the \textsc{\color{blue} Posterior Fossa}, then X would have been classified as Bipolar.}
\end{displayquote}
Given the Cerebellum's crucial role in emotional regulation, it's worth noting that the Posterior Fossa, which houses the Cerebellum, is an important area of study in bipolar disorder research~\cite{ewald2022posterior,kim2013posterior,minichino2014role}.

In contrast, \edg and \dataset generate sparser explanations that involve all the brain regions, making them more complex. The same is true for the backward search method (\datasetbw).
Based on these results, we can conclude that the baseline methods are less effective at producing counterfactual explanations that are consistent with the terminology used to describe the organization of the brain.

\mpara{ADHD Dataset.}
As a last example, \Cref{fig:example2} depicts the counterfactual graphs and the distributions of edges changed among the brain lobes, for patient $22$ in the ADHD dataset, who is classified by ADHD.
In this case, the counterfactuals generated by \tri and \cli are quite similar, with edges removed from the Occipital Lobe and added primarily in the Temporal Lobe. Additionally, \cli adds connections in the Posterior Fossa, while \tri adds them in the Frontal Lobe. It's worth noting that several studies on the structural and functional neuroimaging of ADHD patients have shown alterations in Occipital Regions~\cite{wu2020role,soros2017inattention}.
The output of \cli can be summarized by the following simple \textit{counterfactual statement}:
\begin{displayquote}
\emph{Patient X is classified as ADHD. If X's brain had less activation in the \textsc{\color{red} Occipital Lobe}, and more co-activation in the \textsc{\color{blue} Posterior Fossa} and the \textsc{\color{blue} Temporal Lobe}, then X would have been classified as Typically Developed.}
\end{displayquote}


% Figure environment removed

% Figure environment removed

\subsection{Quantitative Comparison}\label{sec:global}
We next present a comparison of the various counterfactual generation methods, using the metrics outlined in \Cref{sec:metrics}.


\Cref{fig:distance_methods} reports the distribution of $d_{\%}$ and $C$ values (the latter in logarithmic scale) at the class level for each method across three datasets (AUT, BIP, and ADHDM). Results for the remaining datasets can be found in the supplementary material shared in our repository.
Computation complexity, which is measured as the number of calls to the black-box classifier, varies across the different methods tested, with \dataset and \datasetbw being the most time-consuming due to the need to compare the input network with each graph classified in the opposite class. We note that \datasetbw requires slightly more calls to the oracle $C$ because it also performs a backward search.
\cli tends to find solutions more quickly than the other methods due to its tendency to make larger changes in the regions of the network, whereas \edg and \tri may require more iterations (and thus calls to the oracle) to achieve the same change.

We next examine the proximity of the counterfactual graphs to the corresponding input networks. As we observed in the previous section, the explanations generated by \dataset differ significantly from the input networks, as it searches for counterfactuals among the graphs in the dataset (which can vary considerably from each other) rather than perturbing the network itself.
The application of the backward search on top of \dataset results in counterfactuals that are much closer to the original networks compared to other methods. 
Interestingly, applying a backward search after \rcli does not significantly alter the resulting counterfactuals, suggesting that these solutions are more robust than those generated by \dataset.
Finally, since both \tri and \edg change a few edges at each iteration, the corresponding distributions of symmetric differences are comparable.

Methods such as \cli and \rcli operate on the maximal cliques in the network, which causes them to change a larger number of edges at each iteration, resulting in counterfactuals that are more distant than those obtained by \edg and \tri.
It is important to stress that \cli and \rcli, by design, are expected to induce larger changes when producing a counterfactual as they use a coarser-grain vocabulary in the explanation (dense regions), w.r.t. the fine-grain approaches of \edg and \tri. As motivated in \Cref{sec:intro}, we aim to have explanations at the level of regions (in which the changes are concentrated), because these are more interpretable for the domain expert than a simple list of flipped edges.

%Despite the increased distance, these methods produce more interpretable counterfactual statements due to the changes being concentrated in the same regions of the brain.




We finally report the flip rate per class (class 0/class 1) for each dataset (columns) and each method (rows) in \Cref{tab:fr}. 
We remind that \rcli was tested only in the datasets where the brains' parcellations were available (i.e., all but OHSU, PEK, and KKI).
By definition, the flip rate of \dataset is always 100\%, as it picks the closest counterfactual among the graphs in the database.
The other methods, instead, did not achieve a perfect score, as they were run for a fixed number of iterations.
In particular, \tri is run for at most $\min\left(|E_-|, |E_+|\right)$ iterations, \cli and \rcli for at most $\max_I =200$ iterations, and \edg for st most 2000 iterations. 
We observe that \tri converges to a counterfactual more frequently than the other methods, even in the unbalanced ADHD dataset. However, it struggles in the three sparsest networks (OHSU, PEK, KKI), likely because triadic closure is less observable in these graphs, while \edg adds and removes edges more indiscriminately, which allows it to eventually find a counterfactual even in these cases. 
Finally, \cli strikes a balance between \edg and \tri, as it acts on maximal cliques and can thus remove and add cliques even in sparser graphs (where cliques are just edges).




\begin{table}[t!]
    \centering
    \caption{Flip rate (class 0/ class 1), for each method and each dataset. Flip rate for \datasetbw and \rclibw are not reported as they are the same as \dataset and \rcli.}
    \vspace{-2mm}
    \label{tab:fr}
    \begin{tabular}{cccccccc}
    \toprule
    %\multicolumn{7}{c}{Data} \\
    %\cmidrule(lr){2-8}
    \textbf{Method} &
    \textbf{AUT} & 		
    \textbf{BIP} & 		
    \textbf{ADHD} & 		
    \textbf{ADHDM} & 		
    \textbf{OHSU} & 		
    \textbf{PEK} &
    \textbf{KKI} \\
    \midrule
    EDG & 100/85  & 70/100  & 75/100  & 74/100  & 100/87  & 100/91  & 90/100 \\
    TRI & 100/100  & 100/100  & 100/100  & 98/100  & 62/58  & 89/96  & 90/71 \\
    CLI & 91/100  & 100/100  & 100/100  & 56/100  & 53/87  & 100/91  & 90/100 \\
    RCLI & 96/93  & 94/100  & 95/98  & 61/100  & -  & - & -\\
    DATA & 100/100  & 100/100  & 100/100  & 100/100  & 100/100  & 100/100  & 100/100 \\
    %DATA+BW & 100/100  & 100/100  & 100/100  & 100/100  & 100/100  & 100/100  & 100/100 \\
    \bottomrule
    \end{tabular}
    \vspace{2mm}
\end{table}


\section{Conclusions and Future Work}

As demonstrated by HIV, HCV, and now cancer, combination therapy is a critical option for disease treatment. Yet, difficulties arise in regards to understanding drug-drug interactions and patient-specific genetic differences. To tackle this, we show that encoder-only language models are effective for drug synergy prediction. We then build on these results by proposing SynerGPT, a decoder model with a novel training strategy for in-context learning which can produce strong results for few-shot drug synergy prediction. We additionally show that the model context can be optimized using non-linear black-box approaches, which has exciting implications for the design of a standardized drug synergy testing panel for creating patient-specific synergy datasets. Finally, we explore a novel task of inverse design using desired drug synergy tuples. Performance on this challenging task is low for unknown drugs; nonetheless, it shows promise for future work that may enable personalized drug discovery. %

\paragraph{Limitations}
While we are able to achieve strong performance without additional cellular or drug data, our approach is very much a black box akin to most deep learning methods. Future work will still likely want to integrate external database features. However, they will likely need to be integrated in a more thoughtful manner in order to ensure an actual benefit. It would also likely be interesting for future work to investigate the internal connections language models are learning and what it might mean for understanding the fundamental biology of how cellular pathways interact. It is also worth noting that designing molecules using drug synergy tuples is a somewhat atypical task, so there may exist a wall in terms of the information content inherent in the context. 
While we do analysis by separating model performance into different tissue types in this work (as done in multiple prior studies), we note that for future research it is likely too limiting and simplistic to separate cell lines into tissues types.



\clearpage
% BibTeX users should specify bibliography style 'splncs04'.
% References will then be sorted and formatted in the correct style.
\bibliographystyle{splncs04}
\bibliography{bibfile,biblio}

\end{document}
