\section{Conclusions and Future Work} \label{sec:conclusions}
We introduced a general framework, called \emph{density-based counterfactual search} (\dcs), for generating instance-level counterfactual explanations for graph classifiers using the alteration of dense substructures. 
This framework identifies the most informative regions of the graphs and manipulates them by adding or removing dense structures until a counterfactual is found. 
The modularity of the framework allows users to customize their counterfactual search based on their specific needs.
We instantiated \dcs in two special cases: \tri and \cli. In \tri, we search for counterfactual graphs by opening or closing triangles, while in \cli, we move to a counterfactual search driven by maximal cliques.
Additionally, we showed a variation of \cli, called \rcli, which leverages the brain's parcellation to rank the nodes and encourage changes within the same lobes of the brain. This variation generates more interpretable explanations for brain networks.
%We evaluated \dcs in seven brain networks to prove the effectiveness of \tri, \cli, and \rcli in generating concise human-interpretable explanations.

As further work, we plan to address the feasibility and robustness constraints, pivotal in many counterfactual search scenarios. The feasibility constraint arises because, for certain types of data, some counterfactuals may not be feasible or may not exist at all. For example, not all the counterfactuals generated for molecule graphs may be chemically feasible structures. On the other hand, robustness to noise, i.e., when small perturbations to the counterfactual do not change its predicted class, makes the counterfactual explanation more trustworthy and is thus a desirable characteristic.
