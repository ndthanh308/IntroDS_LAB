\subsection{Qualitative analysis}\label{sec:local}
\begin{comment}
%% Old version with 3 raw, where each row is "brain+bar,brain+bar"

% Figure environment removed
\end{comment}

% Figure environment removed

In this section, we compare the counterfactual graphs generated by three instantiations of \dcs, namely \tri, \cli, and \rcli, with those produced by the three baseline methods, \edg, \dataset, and \datasetbw, for specific patients in three datasets. All the results presented pertain to brain networks for which the classifier accurately predicted the class.

\mpara{AUT Dataset.}
\Cref{fig:graph_counterfactual_methods} shows the counterfactual graphs for patient $9$ in AUT. This patient is classified as ``Autism Spectrum Disorder''.
For each method, the left figure shows the connectome of the patient overlaid on the brain glass schematics, where ROIs are projected onto a 2D space of the image, and different colors represent ROIs in different brain lobes. Blue edges indicate edges added to the counterfactual graph, while red edges denote edges removed from the counterfactual graph.
In addition to the connectome visualization, the right barplots illustrate the distribution of changes among brain lobes.
For each brain lobe, the bars report the percentage of the nodes involved in the added (blue) and removed (red) edges that belong to that lobe. By examining these barplots, we can gain insights into which brain regions are most affected by each method's counterfactual graph generation process.
We first observe that each method perturbed different regions of the brain.
This is due to the fact that the set of changes identified by each method depends on a range of factors, including the method's underlying assumptions, its optimization criteria, and its specific implementation.
The choice of counterfactual generation method should take into account the specific properties of the input data and the desired goals of the counterfactual analysis.
One of the desiderata is interpretability.
In general, the larger the number of regions changed and the more homogeneously the changes are distributed within the regions, the less human-interpretable the counterfactual explanation becomes.
Of the methods examined, \dataset produced the most complex explanation, with almost the same number of edges added and removed from each brain lobe.
The \datasetbw method provides a partial solution to this issue by removing edges mainly from the Parietal Lobe and the Temporal Lobe, which reduces the heterogeneity of edge removals across the brain lobes. However, the edge additions still span across many regions, limiting the interpretability of the solution.
The \edg method suffers from similar limitations in that its counterfactual graph involves changes spanning across most of the brain lobes.
In contrast, \tri and \cli provide simpler explanations, as they perturbed a lower number of regions and concentrated most of the changes in the same regions.
Specifically, \tri mainly sparsified the Occipital Lobe and densified the Insula \& Cingulate Gyri, while \cli sparsified only the Frontal Lobe and added most of the edges in the Posterior Fossa.
This results in a more focused and interpretable explanation.
In fact, the output of \cli can be summarized by the following simple \textit{counterfactual statement}:
\begin{displayquote}
\emph{Patient X is classified as Autism Spectrum Disorder. If X's brain had less activation in the \textsc{\color{red} Frontal Lobe} and more co-activation in the \textsc{\color{blue} Posterior Fossa}, \textsc{\color{blue} Insula Cingulate Gyri}, and the \textsc{\color{blue} Temporal Lobe} then X would have been classified as Typically Developed.}
\end{displayquote}

% % Figure environment removed

%We can represent the output of the \dcs as local explanation of the classification of the original graph.

%As discussed in \ref{sec:intro}, \dcs uses a unit of change between the original and the counterfactual graph, based on the dense and sparse regions of the graph.

%\subsection{Global Explanation for Brains Regions}
%Counterfactuals are also effective at providing an overview of the internal logic of black-box classifiers, that is, at giving global explanations.
%We aggregate the edge changes performed by \dcs to generate the counterfactual graph of each graph in the dataset.
%\ref{table:100} shows the frequency of change by type, region, and class (class 0 on the left and class 1 on the right), for AUT-CS (top) and ADHDM (bottom).
\begin{comment}
%%% Old version on 3 row for brain and only 1 for bar

% Figure environment removed
\end{comment}
% Figure environment removed

\mpara{BIP Dataset.}
\Cref{fig:example} shows the counterfactual graphs and the distributions of edges changed among the brain lobes, for patient $9$ in the BIP dataset.
This patient is classified as ``Typically Developed''.
This example serves to confirm the effectiveness of \tri and \cli in generating more compact and interpretable explanations. Specifically, \tri produces a counterfactual graph that closely resembles the input network, with only 10 edges added and 10 edges removed. On the other hand, \cli concentrates its changes in two specific regions: the Parietal Lobe (with connections removed) and the Posterior Fossa (with connections added).
The output of \tri can be summarized by the following simple \textit{counterfactual statement}:
\begin{displayquote}
\emph{Patient X is classified as Typically Developed. If X's brain had less activation in the \textsc{\color{red} Parietal Lobe} and the \textsc{\color{red} Insula Cingulate Gyri}, and more co-activation in the \textsc{\color{blue} Posterior Fossa}, then X would have been classified as Bipolar.}
\end{displayquote}
Given the Cerebellum's crucial role in emotional regulation, it's worth noting that the Posterior Fossa, which houses the Cerebellum, is an important area of study in bipolar disorder research~\cite{ewald2022posterior,kim2013posterior,minichino2014role}.

In contrast, \edg and \dataset generate sparser explanations that involve all the brain regions, making them more complex. The same is true for the backward search method (\datasetbw).
Based on these results, we can conclude that the baseline methods are less effective at producing counterfactual explanations that are consistent with the terminology used to describe the organization of the brain.

\mpara{ADHD Dataset.}
As a last example, \Cref{fig:example2} depicts the counterfactual graphs and the distributions of edges changed among the brain lobes, for patient $22$ in the ADHD dataset, who is classified by ADHD.
In this case, the counterfactuals generated by \tri and \cli are quite similar, with edges removed from the Occipital Lobe and added primarily in the Temporal Lobe. Additionally, \cli adds connections in the Posterior Fossa, while \tri adds them in the Frontal Lobe. It's worth noting that several studies on the structural and functional neuroimaging of ADHD patients have shown alterations in Occipital Regions~\cite{wu2020role,soros2017inattention}.
The output of \cli can be summarized by the following simple \textit{counterfactual statement}:
\begin{displayquote}
\emph{Patient X is classified as ADHD. If X's brain had less activation in the \textsc{\color{red} Occipital Lobe}, and more co-activation in the \textsc{\color{blue} Posterior Fossa} and the \textsc{\color{blue} Temporal Lobe}, then X would have been classified as Typically Developed.}
\end{displayquote}


% Figure environment removed