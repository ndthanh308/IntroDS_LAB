\subsection{Validation of \dcs}
% \mynote[from=Carlo]{there are a multiple synthetic datasets that are used as benchmarks, maybe it's better to use them instead of ours?}
%Synthetic data Generation and results
We validate {\dcs} in two synthetic datasets equipped with a white-box binary classifier.
The graph generation process takes in input a set of nodes $V$ and a number of graphs $N$.
The set of nodes is partitioned into two subsets $S_0$ and $S_1$ of equal size, and then, for each class $i \in \{0, 1\}$, $N/2$ graphs are generated according to the following procedure:
\textbf{(i)} select one (1SG) or two (2SG) subgroups of nodes in $SG \subseteq S_i$, 
\textbf{(ii)} build $c$ random cliques among nodes within the same subgroup,
\textbf{(iii)} apply a Barabási–Albert preferential attachment algorithm \cite{albert2002statistical} with parameters $m$, $p$, and $q$, to add structure within $S_{1-i}$ and between $S_i$ and $S_{1-i}$
\footnote{We set $N_{1SG} = 100$ and $N_{1SG} = 200$; $SG_{1SG} = 1/4 \cdot |V|$ and $SG_{2SG} = 1/8 \cdot |V|$; $c_{1SG} = 10$ and $c_{2SG} = 20$; $m = SG$, $p = 5$ and $q = 0.7$.}.

This way, for each $i$, we obtain graphs with dense regions in $S_i$, but sparse in $S_{1-i}$.
\ref{fig:synt} illustrates the dense regions in the graphs for each class, in terms of triangle count at node level. The upper chart shows the average triangle count for the synthetic dataset 1SG, while the lower chart shows the same quantity for 2SG.

% Figure environment removed

Then, we build a white-box classifier using the same logic of the graph generation process, to achieve perfect accuracy.
\mynote[from=Carlo]{Is it fine to use this very simple white box classifier? Is it better to train a black-box model?}
%Explain in the appendix the classifier algorithm

\begin{table}[thb]
	\centering
	\caption{Quartiles of symmetric difference (percentage) $d_{\%}$, and number of iterations $I$, for the synthetic datasets 1SG and 2SG.}
	\label{tab1}
	\begin{tabular}{ccrrrrr}
		\toprule
		\textbf{Dataset} &  \textbf{Measure} & \textbf{Q0} & \textbf{Q1} & \textbf{Q2} & \textbf{Q3} & \textbf{Q4} \\ 
		\midrule
		\multirow{2}{*}{1SG} 
		& $d_{\%}$ & 18.5 & 39.3 & 55.5 & 80 & 100 \\
		%& $d_{cl}$ & 0.5 & 0.65 & 0.7 & 0.77 & 0.92 \\
		% & Cl. sort & -2.01 & 0.14 & 0.38 & 0.51 & 0.71 \\
		& $I$ & 1 & 2 & 3 & 6 & 21 \\
      \hline
		\multirow{2}{*}{2SG} 
		& $d_{\%}$ & 42.1 & 81.6 & 100 & 100 & 100 \\
		%& $d_{cl}$ & 0.43 & 0.52 & 0.57 & 0.61 & 0.73 \\
		% & Cl. sort & -3.52 & -1.33 & -0.35 & 0.3 & 0.68 \\
		& $I$ & 2 & 3 & 6 & 9 & 19 \\
		\bottomrule
	\end{tabular}
\end{table}

\ref{tab1} summarizes the results of the validation of \dcs.
In details, it reports the quartiles of the symmetric differences (percentage of edges modified) $d_{\%}$, the closeness $d_{cl}$ between the original and the counterfactual graphs, and the number of iterations $I$, for both 1SG and 2SG.
The flip rate is $100\%$ in both datasets. 
However, the symmetric difference $d_{\%}$ is quite high, especially in the 2SG dataset, where in more that half of the cases of the edges are changed.

\ref{fig:hm_bi} illustrates the evolution of the dense regions in one graph from the 2SG dataset at each iteration of \dcs.


% Figure environment removed

\mynote[from=Giulia]{Elaborate on the results in \ref{tab1} and \ref{fig:hm_bi}.}


\begin{table*}[thb]
	\centering
	\footnotesize
	\caption{}
	\label{tab:iterations}
	\begin{tabular}{c c c ccccc}
		\toprule
		&  &  & \multicolumn{5}{c}{Iterations} \\
	   \cmidrule(lr){4-8}
	   \textbf{Data} & 		
	   \textbf{Type} & 		
	   \textbf{FR0/FR1} & 
	   \textbf{min} & \textbf{25-p} & \textbf{50-p} & \textbf{75p} & \textbf{max} \\
    \midrule
AUT & EDG & 100.0/84.8 & 1 & 24.0 & 40.0 & 57.25 & 2000 \\
AUT & DAT+BW & 100.0/100.0 & 79 & 129.0 & 149.0 & 168.0 & 313 \\
AUT & TRI & 100.0/100.0 & 1 & 9.0 & 22.0 & 52.0 & 610 \\
AUT & CLI & 90.9/100.0 & 2 & 2.0 & 3.0 & 17.0 & 60 \\
AUT & RCLI & 96.4/93.5 & 3 & 6.0 & 11.0 & 83.0 & 671 \\
AUT & DAT & 100.0/100.0 & 102 & 102.0 & 102.0 & 102.0 & 102 \\
  \hline
BIP & EDG & 70.0/100.0 & 1 & 9.0 & 25.0 & 51.0 & 2000 \\
BIP & DAT+BW & 100.0/100.0 & 61 & 127.0 & 201.5 & 384.5 & 1477 \\
BIP & TRI & 100.0/100.0 & 1 & 1.0 & 6.0 & 25.0 & 595 \\
BIP & CLI & 100.0/100.0 & 2 & 2.0 & 6.5 & 8.75 & 20 \\
BIP & RCLI & 94.0/100.0 & 3 & 10.0 & 153.0 & 257.0 & 675 \\
BIP & DAT & 100.0/100.0 & 119 & 119.0 & 119.0 & 119.0 & 119 \\
  \hline
ADHD & EDG & 75.0/100.0 & 1 & 8.0 & 21.5 & 36.0 & 2000 \\
ADHD & DAT+BW & 100.0/100.0 & 73 & 108.0 & 141.0 & 215.0 & 635 \\
ADHD & TRI & 100.0/100.0 & 1 & 2.0 & 5.0 & 19.0 & 330 \\
ADHD & CLI & 100.0/100.0 & 2 & 2.0 & 5.0 & 9.0 & 49 \\
ADHD & RCLI & 95.0/98.4 & 3 & 5.0 & 12.0 & 45.0 & 622 \\
ADHD & DAT & 100.0/100.0 & 124 & 124.0 & 124.0 & 124.0 & 124 \\
\hline
ADHDM & EDG & 33.7/100.0 & 1 & 13.75 & 60.5 & 2000.0 & 2000 \\
ADHDM & DAT+BW & 100.0/100.0 & 71 & 193.0 & 372.0 & 606.0 & 1238 \\
ADHDM & TRI & 98.1/100.0 & 1 & 5.0 & 185.0 & 282.0 & 664 \\
ADHDM & CLI & 40.4/100.0 & 2 & 2.0 & 39.0 & 60.0 & 60 \\
ADHDM & RCLI & 60.6/100.0 & 3 & 8.0 & 307.0 & 548.0 & 632 \\
ADHDM & DAT & 100.0/100.0 & 124 & 124.0 & 124.0 & 124.0 & 124 \\
\hline
OHSU & EDG & 100.0/86.7 & 2 & 38.0 & 67.0 & 117.0 & 2000 \\
OHSU & DAT+BW & 100.0/100.0 & 40 & 100.0 & 149.0 & 196.0 & 889 \\
OHSU & TRI & 61.8/57.8 & 1 & 23.0 & 49.0 & 95.0 & 627 \\
OHSU & CLI & 52.9/86.7 & 2 & 29.0 & 60.0 & 82.0 & 97 \\
OHSU & DAT & 100.0/100.0 & 80 & 80.0 & 80.0 & 80.0 & 80 \\
\hline
PEK & EDG & 100.0/91.3 & 1 & 2.0 & 5.0 & 12.0 & 2000 \\
PEK & DAT+BW & 100.0/100.0 & 10 & 26.0 & 60.0 & 95.0 & 387 \\
PEK & TRI & 88.7/95.7 & 1 & 1.0 & 4.0 & 15.0 & 492 \\
PEK & CLI & 100.0/91.3 & 2 & 2.0 & 4.0 & 20.0 & 97 \\
PEK & DAT & 100.0/100.0 & 86 & 86.0 & 86.0 & 86.0 & 86 \\
\hline
KKI & EDG & 89.7/100.0 & 1 & 1.0 & 2.0 & 10.0 & 2000 \\
KKI & DAT+BW & 100.0/100.0 & 5 & 16.0 & 27.0 & 51.0 & 224 \\
KKI & TRI & 89.7/70.5 & 1 & 1 & 3.0 & 17.0 & 234 \\
KKI & CLI & 89.7/100.0 & 2 & 2.0 & 4.0 & 20.0 & 97 \\
KKI & DAT & 100.0/100.0 & 84 & 84.0 & 84.0 & 84.0 & 84 \\
  \bottomrule
	\end{tabular}
\end{table*}



%%%%%%%%%%%%%%%%%%%%%%%%%%%%%%%%%%%%%%%%%%%%%%%%%%%%%%%%%%%%%%%%%%%%%%%%%%%%%%%%%%%%%%%%%%%
\subsubsection{Configuration of {\cli}}\ref{sec:settings}
\mynote[from=Carlo]{Do we want to keep it? To be updated and aligned with current notation.}{}
\Cref{tab:res} reports the performance of {\cli} in AUT-CS, for different parameter configurations.
It displays the flip Rrate $\mathsf{fr}$, and the percentile of number of iteraton $I$ and distance $d_{\%}$.
The configurations considered are the following:
\begin{itemize}
    \item Node Ranking Strategy $R$: $\mathsf{t}$ sorts the nodes based on the number of triangles, and $\mathsf{eg}$ based on the eigenvector centrality~\cite{10.5555/1809753};
    \item Sorting Strategy for $D$: $0$ means no sorting, $l_{+1}$ means the use of a vector that weights node in \Cref{alg:LS}, and $h_{-1}$ means the use of a vector that weights node in \Cref{alg:HS}.
\end{itemize}

We observe that the quartiles of the two distributions are similar across the configurations, whereas the flip rate $\mathsf{fr}$ shows an increase of $10\%$ between the two extreme settings. We register the following results:
\begin{itemize}
	\item The Node Sorting for Fill Search $D_n$ performs better with only the weights in the Empty Search $l_{+}$.
	\item $k$ shows not much difference in performance: a bit of budget but not too much.
	\item $NS$ the eigenvector centrality sorting $Eg$ reaches the best $\mathsf{fr}$.
\end{itemize}

\begin{table}[t]
    \centering
    \small
    \caption{Flip rate $\mathsf{fr}$, and quartiles of computation time $I$ and closeness $d_{\%}$, for different parameter configurations of {\cli}, in AUT-CS.}
    \label{tab:res}
    \begin{tabular}{c ccccc ccccc ccc}
	\toprule
	& \multicolumn{5}{c}{$I$} & \multicolumn{5}{c}{$d_{\%}$} & & &\\
	\cmidrule(lr){2-6} \cmidrule(lr){7-11}
		$\mathsf{fr}$ & 
		\textbf{Q0} & \textbf{Q1} & \textbf{Q2} & \textbf{Q3} & \textbf{Q4} & 
		\textbf{Q0} & \textbf{Q1} & \textbf{Q2} & \textbf{Q3} & \textbf{Q4} &
		\textbf{$D$} & \textbf{$k$} & \textbf{$R$} \\
		\midrule
		73.3 
		& 1 & 2 & 6 & 15 & 56 
		& 1.9 & 9.3 & 19.2 & 27.3 & 48.7 
		& \textbf{$0$} & 2 & $\mathsf{t}$\\
		68.3 
		& 1 & 2 & 8 & 15 & 58 
		& 1.9 & 9.3 & 19.7 & 29.9 & 55.8 
		& $l_{+1}$ & \textbf{4} & $\mathsf{t}$\\
		76.2 
		& 1 & 2 & 6 & 16 & 56 
		& 1.9 & 9.3 & 19.7 & 28.3 & 54.6 
		& $l_{+1}$ & 2 & $\mathsf{t}$\\
		78.2 
		& 1 & 2 & 7 & 18 & 56 
		& 1.9 & 10 & 20.1 & 32.8 & 53 
		& $l_{+1}$ & \textbf{0} & $\mathsf{t}$\\
		\textbf{79.2} 
		& 1 & \textbf{1} & \textbf{5} & 14 & 57 
		& 2.7 & \textbf{7.3} & 20.6 & 32 & 53.4 
		& $l_{+1}$ & 2 & \textbf{$\mathsf{eg}$}\\
		71.3 
		& 1 & 2 & 6 & 13 & 58 
		& 1.9 & 8.5 & 18.8 & \textbf{26.4} & 53 
		& \textbf{$l_{+1}$,$h_{-1}$} & 2 & $\mathsf{t}$\\
		\bottomrule
	\end{tabular}
\end{table}




%%%%%%%%%%%%%%%%%%%%%%%%%%%%%%%%%%%%%%%%%%%%%%

The counterfactual graph can be visualized by coloring the edges in the symmetric difference with different colors, as shown in \ref{fig:local_1} (top).
This picture shows the brain connectome of a counterfactual explanation for a graph in ADHDM dataset in the class ``not used'', where the edges in $\{E \setminus E'\}$ (edges removed, for a total of $21$) are denoted in red, and the edges in $\{E' \setminus E\}$ (edges added, for a total of $16$) are denoted in blue.
Each node represents a voxel, and different node colors indicate different areas of the brain.

% Figure environment removed


The bar plot in \ref{fig:local_1} (bottom) shows which region of the brain the edges changed belong to.
In details, it reports the ratio of edges changed by region and by type of change (edges added $E^+$ and edges removed $E^-$).
This plot allow us to analyze which parts of the brain should be sparsified/densified to obtain a brain network classifiable in the ``Marijuana use'' class.
As we can see, we need to remove edges between the Frontal and Parietal Lobe, and add edges in the Posterior Fossa.

% Figure environment removed

% Figure environment removed

\ref{fig:local_2} and \ref{fig:local_3} display the the brain connectome of counterfactual graphs in the TD class for a graph in AUT-CS (\ref{fig:local_2}) and in ADHD (\ref{fig:local_3}).
In the former, the symmetric difference with the original graph is 63 (32 edges removed and 31 added); while in the latter, the symmetric difference is 41 (21 edges removed and 20 added).

We can see that the brain with ASD has fewer connections in the Posterior Fossa than the brain with ADHD, as such region includes most of the edges added (resp. removed) to find the corresponding counterfactual graph.
Moreover, voxels in the Central Structures of the ADHD brain are more interconnected than those in a TD brain.

