\subsection{MaNGA survey}\label{s21}

We construct a parent sample from the sample of the MaNGA survey \citep{Bundy2015}, an integral field unit survey in the SDSS-IV \citep{Blanton2017}.
This survey was carried out with the SDSS 2.5-meter telescope \citep{Gunn2006} equipped with a fiber feed system \citep{Drory2015} and the Baryon Oscillation Spectroscopic Survey (BOSS) spectrograph \citep{Smee2013} covering 3600 – 10300 \AA\ at $R\sim2200$.
The target galaxies of the MaNGA survey were selected from the NASA-Sloan atlas (NSA {\tt v1\_0\_1}).
See \citet{Yan2016} for the survey design.
As detailed in \citet{Wake2017}, the MaNGA sample consists of the following three samples selected with redshifts, magnitudes, and colors to achieve a uniform number density distribution of $M_i$ and additional targets selected for specific sciences.
\begin{itemize}
    \item {\bf Primary Sample}: Galaxies whose $1.5R_e$ area is covered by the IFU.
    \item {\bf Secondary Sample}: Galaxies whose $2.5R_e$ area is covered by the IFU.
    \item {\bf Color-Enhanced Sample}: Selected to fill in the region on the ${\rm NUV}-i$ versus $M_i$ color-magnitude diagram poorly sampled in the Primary and Secondary Samples. Galaxies in this region include low-luminosity red galaxies, high-luminosity blue galaxies, and GVs. The IFU covers their $1.5R_e$, the same as the Primary Sample. The Primary Sample and the Color-Enhanced Sample are combined as the Primary+ Sample.
    \item {\bf Ancillary Targets}\footnote{\url{https://www.sdss4.org/dr17/manga/manga-target-selection/ancillary-targets/}}: Additionally selected 977 targets, such as luminous AGN, galaxies in voids, pairs and mergers, dwarf galaxies, blight cluster galaxies, and so on.
\end{itemize}
These galaxies are distributed over $5\times10^8\leq M^*/M_\odot \leq 3\times10^{11}$ at $0.01\leq z\leq0.15$.
In this study, we use the IFU data of 11,273 datacubes in the SDSS DR17 \citep{Abdurro'uf2022} reduced with the version 3.1.1 MaNGA Data Reduction Pipeline (DRP: \cite{Law2016,Law2021}), which performs flux calibration, sky subtraction, correction of the Galactic dust attenuation, and coadding individual exposures to make the final 3D datacubes of each galaxy.

Assuming that the RG fraction, the number ratio of RGs to total galaxies, is $f_{\rm RG}\sim 10\%$ as suggested by \citet{Nelson2018, Behroozi2019, Chauke2019,Tacchella2022}, we expect that the entire MaNGA sample includes $N\sim10^3$ RGs, enabling a reliable statistical analysis.
Besides, because all selected RGs have IFU spectroscopic data, we can also analyze the rejuvenation mechanism with spatially resolved characteristics.

\subsection{Removing AGN, mergers, and pair galaxies}\label{s22}
In this study, we do not use galaxies hosting active galactic nuclei (AGN) because {\sc Prospector} has no AGN template other than dust torus emission in IR wavelengths \citep{Leja2018}.
\citet{Comerford2020} published an AGN catalog selected from the 8th MaNGA Product Launch (MPL-8) sample.
However, since this sample ($N=6261$ galaxies) is smaller than the DR17 sample we use, we extend the selection method used by \citet{Comerford2020} to the DR17 MaNGA sample to exclude AGN.
In total, we remove 406 AGN from the whole sample.

Merger and pair galaxies are also excluded using the {\tt MANGA TARGET3} information, a flag for ancillary targets, because it is difficult to create a 1D spectrum of a target galaxy by subtracting the contribution from its companion(s) in the 1D-flattening process described in section \ref{s24}.
We remove 119 galaxies that have either {\tt TARGET3\_PAIR\_SIM}, {\tt TARGET3\_PAIR\_RECENTER}, {\tt TARGET3\_PAIR\_2IFU}, or {\tt TARGET3\_PAIR\_ENLARGE} flag.

\subsection{Crossmatching with photometric catalogs}\label{s23}
For SED fitting, we use both photometric data from UV to MIR and IFU-flattened 1D spectroscopic data (see Section \ref{s24}) to obtain more accurate SFHs than the existing SED fitting results, such as Pipe3D \citep{Sanchez2016a, Sanchez2016b, Sanchez2018} and FIREFLY \citep{Goddard2017, Parikh2018}.
For UV (FUV and NUV) photometry from Galaxy Evolution Explorer (GALEX) and optical ($ugriz$) photometry from the SDSS DR8, we use Sérsic fluxes summarized in the NSA and correct them for Galactic extinction using \citet{Schlegel2018}.
We also use IR photometry (W1, W2, W3, and W4) from the unWISE catalog \citep{Schlafly2019}, which is deeper than the ALLWISE catalog due to using coadd images and a forced photometry method \citep{Lang2016}.
Furthermore, because the forced photometry is based on galaxy positions and profiles from the SDSS, the unWISE photometry is more consistent with the NSA photometry described above.
By crossmatching MaNGA galaxies with the unWISE catalog 
with a maximum separation of 10 arcsec, we select 8857 sources with valid photometry (i.e., without missing values) as the main parent sample.
The $S/N$ distribution for each band and the redshift distribution of the sample are shown in figure~\ref{fig;sample_summary}.
We find that crossmatching with the unWISE catalog does not affect the redshift distribution significantly.

% Figure environment removed

\subsection{1D flattening of datacubes}\label{s24}
To transform 3D IFU data (with two spatial and one spectral dimensions) into a form that can be used for SED fitting, for each galaxy, we sum the spectra of all spatial positions to create a 1D spectrum in the following manner.

First, based on {\tt MANGA\_DRP3PIXMASK} information, we exclude invalid data with {\tt FORESTAR} or {\tt DONOTUSE} mask.
{\tt FORESTAR} means that the spaxel is contaminated by a foreground star while {\tt DONOTUSE} means that the data of the spaxel are unreliable for some reason.
We do not use a spaxel if it has even one wavelength bin masked with either flag.
The mask\_1 panel in figure~\ref{fig;1dflat} shows the positions removed by this operation in dark blue, and the mask\_2 panel shows the wavelengths removed by this operation (the value of 0 indicates that the wavelength has been removed).

Then, we remove spaxels that either have an invalid value at least one wavelength bin or are located near the edge of the field of view, i.e., at a distance from the center being greater than 0.8 times the maximum distance of all spaxels.
The locations removed by this operation are shown in dark blue in figure~\ref{fig;1dflat}, mask\_3.

In addition, we mask three wavelength ranges [5500~\AA, 5600~\AA], [5850~\AA, 5950~\AA], and [6250~\AA, 6450~\AA] because luminous airglow emission lines contaminate.
These wavelength ranges are shown in dark gray in the spectral and $S/N$ wavelength dependence diagrams in figure~\ref{fig;1dflat}.

Finally, we coadd the spaxels not masked by the above operations to produce 1D-spectrum data.
We calculate the errors in 1D spectra by assuming a normal distribution of errors for each spaxel.
An example coadded 1D-spectrum and its $S/N$ are shown in the second and third lines in figure~\ref{fig;1dflat}.

As mentioned in section~\ref{s21}, the size of the MaNGA field of view varies depending on the sample type.
The field of view covers $1.5R_e$ for the Primary Sample and $2.5R_e$ for the Secondary Sample, where $R_e$ is the effective radius.
However, we do not perform aperture correction according to sample type.
This is because {\sc Prospector} has the function of photospectral correction as explained in section~\ref{s312}.
Note that if there is non-negligible radial dependence in spectra, e.g., strong star formation only outside $1.5R_e$, there may be a bias between the Primary Sample and the Secondary Sample.

% Figure environment removed

\subsection{Value-added catalogs}\label{s25}
The MaNGA sample has various value-added catalogs\footnote{\url{https://www.sdss4.org/dr17/data_access/value-added-catalogs/}} (VAC).
Another advantage of using MaNGA data set is those VACs, and we can discuss the characteristics of RGs in more detail and more easily with such catalogs.

In this study, we use two VACs.
First, we use ``MaNGA PyMorph DR17 photometric catalog'' \citep{Dominguez2022} in section~\ref{s54} to discuss the morphology of the selected RGs.
Second, we use ``MaNGA Pipe3D value-added catalog: Spatially resolved and integrated properties of galaxies for DR17'' (Pipe3D, \cite{Sanchez2022}) in appendix to validate the results of our SED fitting.
We describe the detail of each catalog and the used values in the individual sections.

