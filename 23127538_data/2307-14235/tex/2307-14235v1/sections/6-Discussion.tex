\subsection{RGs are important players in galaxy evolution}\label{dis:importance}
One of the key questions in the study of rejuvenation is the importance of RGs in galaxy evolution, especially how much of the total stellar mass was formed by rejuvenation and whether or not rejuvenation has a significant contribution to cosmic star formation.
\citet{Chauke2019} have reported that about 10\% of the total stellar mass of rejuvenated galaxies is formed in the rejuvenation event and that RGs contribute only about 0.3\% of the cosmic SFR density (CSFRD) over $z \sim 0.7-1.5$.
\citet{Tacchella2022} have reported that, at most, only about 10\% of the total mass is formed in rejuvenation events.

To address this question, first, we calculate $C_j$, the contribution of each current galaxy population $j$ ($j=$nRSFG, nRGV, nRQG, and RG) to the CSFRD in the last 100 Myr, the timescale of rejuvenation in our definition.
Here, $C_j$ is defined as the ratio of the total $M^*$ formed in $j$-type galaxies in the last 100 Myr to the total $M^*$ formed in all galaxies in the last 100 Myr,  
\begin{equation*}
    C_j = \frac{\sum_{i\in j}w_i M^*_{i,{\rm 100~Myr}}}{\sum_{i\in {\rm All}}w_i M^*_{i,{\rm 100~Myr}}},
\end{equation*}
where $M^*_{i,{\rm 100~Myr}}$ is the formed stellar mass in a galaxy $i$ in the last 100 Myr and $w_i$ is the volume weight of this galaxy given in the MaNGA catalog.
The mass dependence of $C_j$ is shown in the upper figure~\ref{fig;formed_mass}.
The sum of $C_{\rm RG}$ for all $M^*$ bins is about $17\%$, meaning that RGs contribute about $17\%$ to the CSFRD in the recent 100 Myr, about two orders of magnitude larger than the value obtained by \citet{Chauke2019} for rejuvenated galaxies at $z\sim0.8$.
If the detail of rejuvenation events is similar at $z\sim1$ and $z\sim0$, this result implies that rejuvenation events have higher importance at lower $z$.
However, this difference may be explained by the difference in the definition of rejuvenation and the difference in the parent sample (they search for RGs only from QGs).
It may also be attributed to the $z$ dependence of the CSFRD because the CSFRD at $z\sim1$ is about one order of magnitude higher than at $z\sim0$ (e.g., \cite{Madau2014}).
Note also that we cannot rule out the possibility that $C_j$ is overestimated or underestimated in this study due to the contamination or completeness discussed in section~\ref{s43}.

Next, we calculate $\mu$, the ratio of the formed stellar mass in the last 100 Myr, $M^*_{\rm 100~Myr}$, to the total stellar mass $M^*$ for each RG (same as the discussion in section~\ref{s42}).
The lower panel of figure~\ref{fig;formed_mass} shows $\mu$ as a function of $M^*$.
We find that rejuvenation events increase $M^*$ only by a few percent at most, with $\sim 0.1\%$ on average.
This result roughly agrees with \citet{Akhshik2021} reporting that $0.5\%$ of $M^*$ is formed in 100-Myr-timescale rejuvenation events at $z=1.883$.
This agreement may suggest that the detail of rejuvenation events does not change with redshift.

Note that the values of $\mu$ obtained here are consistent with the result for mock data presented in section~\ref{s42} that our method can select galaxies with $\mu\sim10^{-3}$ with high completeness.
This consistency suggests that our RG sample is not significantly contaminated.
On the other hand, we cannot rule out the possibility that $\mu \sim 10^{-3}$ is obtained just because our method fails to select RGs with other $\mu$ values.
To do a more solid statistical discussion, it will be important to evaluate contamination and completeness using a larger mock sample covering wider parameter ranges or maybe cosmological simulation data and to develop a more robust selection method.

The very small increase in $M^*$ by rejuvenation obtained here is primarily because the rejuvenation timescale, 100-Myr, is too short to increase $M^*$ significantly.
However, if galaxies experienced rejuvenation events many times in the past, rejuvenation may be an important channel of galaxy mass growth.
Although we cannot directly detect such past rejuvenation events for our sample because of insufficient time resolution of the reconstructed SFHs, we discuss the possibility of multiple rejuvenation events in section~\ref{dis:once_or_many}.

% Figure environment removed


\subsection{Multiple rejuvenation scenario}\label{dis:once_or_many}
Whether rejuvenation can occur more than once in a galaxy's lifetime is still unclear.
\citet{Nelson2018} have found that $\sim10\%$ and $\sim1\%$ of massive passive galaxies in the TNG100 simulation have experienced one and more than one rejuvenation event, respectively.
In this subsection, we roughly estimate $N_{\rm rej}$, the number of times galaxies have experienced rejuvenation, using $f_{\rm RG}$ obtained in section \ref{s51}.

We simply assume that $f_{\rm RG}$ is $\sim8\%$ (the value obtained in section~\ref{s51}) over $z\sim0-1$. 
This assumption comes from the fact that previous studies of $z \lesssim 1$ galaxies have obtained similar $f_{\rm RG}$ values \citep{Tacchella2022,Chauke2019}.
We also estimate the timescale of rejuvenation $\tau_{\rm rej}$, the time from restarting star formation to stopping the resumed star formation, to be $\tau_{\rm rej} \left(10^8~{\rm yr}-10^{7.5}~{\rm yr}\right) / 0.21 \sim 320~{\rm Myr}$ because about 21\% of the RGs with $t_{\rm rej}=10^8~{\rm yr}$ returned to QGs at $10^{7.5}~{\rm yr}$.
Note that this calculation implicitly assumes a constant probability that the resumed star formation will end within a certain time.

With these assumption and estimate, the averaged $N_{\rm rej}$ for all galaxies between $z\sim0-1$ can be calculated as $N_{\rm rej} = \left(t|_{z=1}-t|_{z=0}\right)f_{\rm RG}/\tau_{\rm rej}\sim2.1$, suggesting that galaxies have experienced rejuvenation twice at $z \lesssim1$ on average.
Because $f_{\rm RG}$ increases with stellar mass as shown in figure~\ref{fig;SFRvsM}, massive galaxies can have larger $N_{\rm rej}$.
For example, the $N_{\rm rej}$ of $M^*>10^{11}M_\odot$ galaxies, with $f_{\rm RG}\sim18\%$, can be as high as $N_{\rm rej}\sim4.4$, meaning that such massive galaxies are likely to have experienced multiple rejuvenation events.

As shown in figure~\ref{fig;SFRvsM}, the path of a rejuvenation event in the SFR-$M^*$ plane is almost a vertical ascent.
Considering that the time scale of rejuvenation is $\tau_{\rm rej}\sim0.3~{\rm Gyr}$ and that rejuvenation may occur many times, essentially all rejuvenation events have to end up with re-quenching.
Therefore, we infer that massive galaxies are frequently moving up and down on the massive side of the SFR-$M^*$ plane.

To verify whether rejuvenation can occur many times, we use the reconstructed SFHs of current QGs to calculate the time from the last quenching $t_{\rm lq}$, the time since the galaxy last entered the QG regime from the SFG or GV regime.
In this verification, we consider two patterns of quenching, ``long quenching'' and ``mini quenching''.

In the ``long quenching'', it is assumed that galaxies never (at least for a longer time than $\tau_{\rm rej}$) rejuvenate once quenched.
This corresponds to the popular galaxy evolution scenario in that SFGs become QGs.
For simplicity, we assume that the probability of ``long quenching'' for SFGs does not change with cosmic time.
This assumption is reasonable because the cosmological evolution of quenching possibility can be ignored for our focusing timescale; e.g., if an observed object has $z\sim0.03$, even the second-to-last age bin only reaches about $z\sim0.2$, and the quenching rate at $z\sim0.2$ is about only two times higher than that at $z\sim0.03$ according to \citet{Peng2010}.

In contrast, ``mini quenching'' is temporary quenching that is terminated by rejuvenation.
Recently, \citet{Dome2023, Looser2023b} have reported that high-$z$ (low-mass) galaxies can experience mini-quenching.
\citet{Dome2023} have found the duration of a mini-quenching event to be $20-40$ Myr at $z\sim 7$, being comparable to the dynamical time of galaxies at this redshift.
Because the dynamical time scales as $t_{\rm dyn} \propto \rho^{-1/2} \propto \left(1+z\right)^{-3/2}$, we expect the duration of a mini-quenching event of local galaxies to be $\sim$400-900~Myr, which is indeed similar to the rejuvenation timescale $\tau_{\rm rej}$ of our RGs.
If galaxies can have multiple rejuvenation events, quenching other than the last mini quenching (i.e., the long quenching and the previous mini quenchings) will not be reflected in $t_{\rm lq}$ because $t_{\rm lq}$ only focuses on the last quenching event.
Therefore, the distribution of $t_{\rm lq}$ is expected to have an excess on a shorter time scale than $\tau_{\rm rej}$.

In the histogram of $t_{\rm lq}$ and the number normalized by the duration of those time bins shown in figure~\ref{fig;last_quenching}, we indeed find an excess with $t_{\rm lq}\lesssim\tau_{\rm rej}$.
This excess can be explained by considering that the information on mini-quenching that occurred at $t_l\gtrsim\tau_{\rm rej}$ is overwritten by rejuvenation and the next mini-quenching and hence that only mini-quenching at $t_l\lesssim\tau_{\rm rej}$ contributes to the histogram.
However, this interpretation is inconsistent with the fact that rejuvenation occurs at $z > 0$ as well, e.g., $z\sim0.8$ \citep{Chauke2019, Tacchella2022} and $z=1.88$ \citep{Akhshik2021}.
We conclude that galaxies, especially massive ones, have likely experienced multiple rejuvenation events.

% Figure environment removed

\subsection{Why are they disk-like?}\label{s63}
As shown in figures~\ref{fig;morph}~(a) and (c), RGs have a more disk-like or rotational-supported morphology distribution than nRQGs, even though RGs were QGs before rejuvenation by definition.
In this section, we discuss the reason for this morphological difference between RGs and the nRQGs.
Here, we propose the following two hypotheses to explain the disk-like structure of RGs.

\begin{enumerate}
    \item {\bf Selective rejuvenation (SR)}\\
    As high as one-third $(=1259/3806)$ of the QGs in our sample have disk-like morphology, i.e., $V/\sigma>1$. 
    Furthermore, it is well known that some spiral galaxies are anemic spirals without star formation (e.g., \cite{vandenBergh1976,Shimakawa2022}).
    \citet{Fudamoto2022} have found the reddest spiral galaxies at $z\sim1-3$ from JWST/NIRCam images and reported that one of them is well-described as a passive galaxy.
    If only disk-like QGs, or passive spiral galaxies, can rejuvenate, and if their morphology does not change during rejuvenation, then RGs will have a disk-like morphology.
    This is ``selective'' rejuvenation, where only disk-like QGs can rejuvenate.
    
    \item {\bf Disk-forming rejuvenation (DR)}\\
    The disk-like morphology distribution can be explained if elliptical-like QGs form a disk during rejuvenation.
    This scenario does not prohibit the presence of RGs that evolved from disk-like QGs.
\end{enumerate}

\noindent 
We argue that the DR scenario is unlikely because of two serious problems.
First, it is extremely difficult to change morphology in a short period of $\tau_{\rm rej}\sim320~{\rm Myr}$.
Although \citet{Diaz2018} claim that elliptical galaxies can form a disk by merging with gas-rich satellite galaxies, it takes a Gyr time scale, much longer than $\tau_{\rm rej}$.
Besides, as shown in figure~\ref{fig;SFRvsM}, galaxies increase the stellar mass only by $\sim 0.1\%$ in a single rejuvenation event.
It is challenging to make a noticeable disk with such a small amount of mass increase.
Second, there are significant differences in morphologies between RGs and nRQGs (figure~\ref{fig;morph}).
These differences, especially in kinetics morphology $V/\sigma$, may not be explained by simple star formation in the outer part of galaxies due to rejuvenation.
For these reasons, we conclude that the DR scenario is unlikely.

We then test the SR scenario using the $M^*$ and $n_{{\rm dens}, 1h^{-1}~{\rm Mpc}}$ plane.
As shown in figure~\ref{fig;SFRvsM}, $M^*$ does not change significantly before and after rejuvenation.
Because $n_{{\rm dens}, 1h^{-1}~{\rm Mpc}}$ should be almost unchanged before and after rejuvenation, the spatial distribution of RGs and their progenitors in this plane should be the same.
By comparing the distribution of RGs, disk-like nRQGs, and elliptical nRQGs, we can test the SR scenario.

In the SR scenario, only disk-like QGs are the progenitors of RGs.
Because not all disk-like QGs may be able to rejuvenate, the distribution of disk-like QGs does not necessarily have to be the same as of RGs.
However, the distribution of disk-like QGs must cover that of RGs.
In contrast, elliptical QGs can take a significantly different distribution from RGs.

Figure~\ref{fig;match} shows the $M^*$ and $n_{{\rm dens}, 1h^{-1}~{\rm Mpc}}$ distributions of RGs, disk-like nRQGs, and elliptical nRQGs. 
Here, we define galaxies with the same $V/\sigma$ distribution with RGs as disk-like nRQGs and construct a sample of disk-like nRQGs by selecting an nRQG with the nearest $V/\sigma$ for each RG.
On the other hand, we define galaxies with $V/\sigma<0.5$ as elliptical nRQGs.
We find that 
all RGs are within the area where disk-like nRQGs are found, while a significant fraction of RGs are outside the area where elliptical nRQGs are found.
This result is consistent with the SR scenario.
Note that this result does not rule out the existence of RGs that evolved from elliptical-like QGs.
An interesting feature in this figure is that the distribution of disk-like nRQGs has a secondary peak around $\log\left(M^*/M_\odot\right)\sim 10$ and $\log\left(n_{{\rm dens}, 1h^{-1}~{\rm Mpc}}\right) \sim 2.5$ where no RGs exist. 
This may imply that low-mass sources in high-density regions cannot rejuvenate, consistent with the discussion in section~\ref{s55}.
In a subsequent paper, we will further discuss the mechanism of rejuvenation, including the conditions in which rejuvenation occurs.


% Figure environment removed

Finally, we propose further tests for the two scenarios focusing on the mechanisms expected in these scenarios.
One possible mechanism with the SR scenario is that the remaining gas in disk-like QGs is used for rejuvenation.
In this case, disk-like QGs must have a higher molecular gas fraction $f_{\rm gas}$ than normal QGs, and the star formation in RGs must be driven by an increase in star formation efficiency (SFE).
Another possible mechanism in the SR scenario is that gas-poor disk-like QGs obtain gas to restart star formation.
In this case, we expect lower or similar $f_{\rm gas}$ in disk-like QGs. 
In any case, the SR scenario can be tested using a combination of $f_{\rm gas}$ and SFE data.
Additionally, the two scenarios can be verified by analyzing the spatially resolved SFH and kinetics of gases in RGs; e.g., evidence of gas inflows suggests the DR scenario.
Thus, a specially-resolved discussion of $f_{\rm gas}$, SFE, and kinetics using radio observation like ALMA and IFU data from MaNGA will help us constrain the mechanism and verify the SR and DR scenarios.
These hypotheses can also be tested by using simulation data to examine the progenitors of the RGs.

\subsection{The contribution of AGN}
In this study, we exclude 520 objects in section~\ref{s22} as AGN or merging galaxies.
This number amounts to only $6\%$ of the size of the parent sample and also smaller than the number of selected RGs.
Therefore, AGN activities as detected in MaNGA data are unlikely to be the primary driver of rejuvenation.
Note, however, that the parent sample may have some undetected AGNs and merging sources.
It is also possible that past AGN activities or mergers that are not detected from current observations could trigger rejuvenation because rejuvenation started $10^{7.5}~{\rm yr}$ or $10^8~{\rm yr}$ ago.
