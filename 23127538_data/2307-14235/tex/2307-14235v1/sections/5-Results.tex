\subsection{RG fraction}\label{s51}
Our method described in section~\ref{s3} selects 1071 RGs out of the $N=8857$ parent sample; thus, the RG fraction calculated by simply dividing the number of RGs by the number of all galaxies is about 12\%. 
Considering the volume weight $w$ provided in the MaNGA targeting catalog \citep{Wake2017} for the Primary+ sample and the secondary sample ({\tt ESRWEIGHT}), we calculate the intrinsic RG fraction, $f_{\rm RG}$, as,
\begin{equation}
    f_{\rm RG} = \frac{\sum_{i\in{\rm RG}}w_i}{\sum_{i\in{\rm All}}w_i}, \label{eq;f_rg_1}
\end{equation}
and obtain $f_{\rm RG}\simeq 8.1\%$.
Furthermore, when restricted to galaxies with $f\sim0.001$ and considering the completeness discussed in section~\ref {s43}, the true RG fraction is $f_{\rm RG}\sim12\%$.
Note, however, that due to the distribution differences between mock galaxies and real galaxies, it is difficult to estimate the true RG fraction using our results and completeness from the mock data, especially when considering RGs with all $f$ values.

As shown in section~\ref{s22}, we have excluded 520 objects as AGN or merging galaxies from the parent sample.
If these galaxies have higher $f_{\rm RG}$ than our parent sample, the $f_{\rm RG}$, including these galaxies, will also be higher than the above value.

\subsection{Images and spectra}\label{s52}
Figure~\ref{fig;cutouts-MaNGAvsHSC} compares images of the selected RGs constructed from MaNGA datacubes with HSC-SSP PDR3 (\cite{Aihara2022}) $gri$ composite images.
HSC images are deeper and cover a larger area than MaNGA datacubes.
We find that although RGs were QGs before rejuvenation, many RGs have a disk structure.
The morphologies of RGs will be analyzed in detail in section~\ref{s54}.

% Figure environment removed

The upper panels of figure~\ref{fig;stacked-spec} show stacked normalized spectra of current QGs, GVs, and SFGs, separated by whether they are RGs or not; i.e., comparison of rejuvenated QGs (RQGs) with non-rejuvenation QGs (nRQGs), rejuvenating GVs (RGVs) with non-rejuvenation GVs (nRGVs), and rejuvenating SFGs (RSFGs) with non-rejuvenation SFGs (nRSFGs).
Then, to compare the spectra of RGs with those of non-rejuvenation galaxies (nRGs), we calculate a normalized residual spectrum, $\Delta F_\lambda\left(\lambda\right)$, as,
\begin{equation}
    \Delta F_\lambda\left(\lambda\right) = \frac{F_{\lambda,{\rm RG}}\left(\lambda\right) - F_{\lambda,{\rm nRG}}\left(\lambda\right)}{F_{\lambda,{\rm nRG}}\left(\lambda\right)},
\end{equation}
as shown in the lower panels of figure~\ref{fig;stacked-spec}.
Additionally, we examine the distribution of galaxies on the well-studied ${\rm Dn4000}$ versus the ${\rm H\delta}$ equivalent widths (EW) plane in the lowest panels.
We calculate ${\rm Dn4000}$ as follows,
\begin{equation}
    {\rm Dn4000} = \frac{\int_{4000~{\rm \AA}}^{4100~{\rm \AA}} f_\nu~{\rm d}\lambda}{\int_{3850~{\rm \AA}}^{3950~{\rm \AA}} f_\nu~{\rm d}\lambda}.
\end{equation}

Firstly, it is found that the spectral difference between RGs and non-rejuvenation galaxies is less than the $1\sigma$ range, consistent with the previous results from observations by \citet{Chauke2019} and simulated spectra by \citet{Zhang2022}.
As also mentioned in \citet{Zhang2022}, this result suggests that it is difficult to select RGs without an SED fitting method.

Focusing on QGs, panel~(a), $\Delta F_\lambda$ is close to 0 for wavelengths longer than $4000~{\rm \AA}$, but increases to around 0.1 for shorter wavelengths. 
This indicates that RQGs have a slightly weaker $4000~{\rm \AA}$ break than nRQGs.
Similarly, in the ${\rm Dn4000}$ versus ${\rm H\delta}$~EW plane, RQGs have similar EWs to nRQGs but smaller ${\rm Dn4000}$.
This is probably due to the presence of a young stellar population, with a smaller ${\rm Dn4000}$ value, formed in the rejuvenation event.
Also seen is a strong excess in the ${\rm H\alpha}$ emission line, which also indicates recent star formation.

In the case of GVs, panel~(b), $\Delta F_\lambda$ is found to decrease with decreasing wavelength.
This is because RGVs retain a larger population of old stars formed before quenching than nRGVs.
RGVs have a slightly stronger $4000~{\rm \AA}$ break than nRGVs, but not as pronounced as in QGs.
This tendency is also seen in the ${\rm Dn4000}$ versus ${\rm H\delta}$~EW plane, i.e., there is no significant difference in the distributions of RGVs and nRGVs.
Looking at $\Delta F_\lambda$ for each line, H recombination lines (Balmer lines) have larger $\Delta F_\lambda$ than the continuum.
This is likely because RGVs have stronger Balmer emission lines due to active recent star formation.
In contrast, neutral metal lines like Na{\sc i} $\lambda\lambda5885, 5889$ (D1, D2 lines) and Mg{\sc i}$\lambda5175$ have lower $\Delta F_\lambda$ than the continuum, probably because of an aging stellar population formed before quenching.

In the case of SFGs, panel~(c), we can see a stronger decreasing trend of $\Delta F_\lambda$ than in GVs.
This is because RSFGs have a larger amount of old stellar population formed before quenching than nRSFGs.
Similar to GVs, RSFGs have stronger Balmer emission lines and absorption lines of neutral metals than nRSFGs, suggesting the presence of recent star formation and an older stellar population before quenching. 

% Figure environment removed

\subsection{$M^*$ and SFR distributions}\label{s53}
Figure~\ref{fig;SFRvsM}~(a) shows that the distribution of RGs in the SFR-$M^*$ plane is biased toward high $M^*$ and SFR values, with their median SFR being higher than those of nRSFGs and nRGVs.
We also find that the median $M^*$ of RGs is comparable to that of nRGVs but lower than that of nRQGs.
Note that although RGs have a similar distribution to nRGVs in the 1D distributions of SFR and $M^*$, the distribution of RGs in the SFR-$M^*$ plane is closer to that of SFGs than of nRGVs, as indicated by the black contours in figure~\ref{fig;SFRvsM}~(a).

Figure~\ref{fig;SFRvsM}~(a) also gives an insight into the path in the SFR-$M^*$ plane during rejuvenation events.
Since we focus on recent rejuvenation events within $10^8$~yr (by our definition), even if the sSFR increases to about $10^{-9}~{\rm yr^{-1}}$ due to rejuvenation, the ratio of the formed mass during the rejuvenation to the total stellar mass is only $10^{-1}$.
This means that galaxies move upward almost vertically in the SFR-$M^*$ plane by rejuvenation.
This result is consistent with \citet{Chauke2019}.

% Figure environment removed

Figure~\ref{fig;SFRvsM}~(b) shows the distribution of RG fractions in the SFR versus $M^*$ plane.
The RG fraction is found to be higher for GVs and massive ($M^*>10^{11}M_\odot$) SFGs.
The RG fraction calculated with equation~(\ref{eq;f_rg_1}) for only galaxies with $M^*>10^{11}M_\odot$ is about 15\%, indicating that RGs are more abundant on the high mass side.
These results are consistent with previous studies, which compared rejuvenated galaxies with QGs and found that RGs have a smaller mass than QGs (e.g., \cite{Chauke2019, Tacchella2022}).

\subsection{Morphologies}\label{s54}
In order to compare morphologies between RGs and other galaxies, we show in figure~\ref{fig;morph} the distributions of the $r$-band Sérsic index $n_r$, the $r$-band effective radius $R_{e,r}$, the bulge-to-total flux ratio $B/T$, and the ratio of rotational velocity $V$ to velocity dispersion $\sigma$, $V/\sigma$.
For $n_r$, $R_{e,r}$, and $B/T$, we use the PyMorph catalog \citep{Dominguez2022}, which summarizes photometric morphological parameters obtained from $g$, $r$, and $i$ images of MaNGA DR17 galaxies.
This PyMorph catalog has two types of results, one fit by a Sérsic profile and the other by a Sérsic+exponential profile, with {\tt FLAG\_FIT} parameter indicating which result is reliable: 
{\tt FLAG\_FIT}=0 indicates that both fits are reliable; 
{\tt FLAG\_FIT}=1 indicates that only the Sérsic fit is successful while the Sérsic+exponential profile fit may not be reliable; 
{\tt FLAG\_FIT}=2 indicates that only the Sérsic+exponential fit is successful while the Sérsic profile fit may not be reliable; 
{\tt FLAG\_FIT}=3 indicates both fits may be unreliable.
Using this parameter, we construct a different sample depending on the morphological parameter to be examined.
We use only galaxies with {\tt FLAG\_FIT}=1 when discussing $n_r$, only galaxies with {\tt FLAG\_FIT}=0 or 2 when discussing $B/T$, and only galaxies with {\tt FLAG\_FIT}=0, 1, or 2 when discussing $r_{e,r}$.
For the kinematic parameters, $V$ and $\sigma$, we use the DAPall catalog (\cite{Westfall2019, Belfiore2019}).

First, we focus on the photometric morphology parameters.
From figure~\ref{fig;morph}~(a) and (b), we find that RGs have a very different morphology from nRQGs (the $p$-value of Kolmogorov-Smirnov test is $p\ll0.05$); RGs show more a disk-like morphology (lower $n_r$ and $B/T$) than nRQGs and a more elliptical-like morphology (higher $n_r$) than nRSFGs.
From figure~\ref{fig;morph}~(c), we find that RGs tend to have a larger radius than nRSFGs and a smaller radius than nRQGs. 
Similar to the $n_r$ and $B/T$ distributions, RGs have almost the same $R_{e,r}$ distribution as nRGVs.
Based on the above results, we conclude that the RGs have a photometric parameter distribution between nRQGs and nRSFGs, similar to nRGVs.

We can also see similar trends in the kinematic morphology parameters.
Figure~\ref{fig;morph}~(d) shows that RGs have an intermediate distribution of $V/\sigma$ between nRSFGs and nRQGs.
Particularly, RGs have a more rotation-supported morphology than nRQGs, even though the mass distribution of nRQGs is not significantly different from that of the RGs.
RGs also show a similar $V/\sigma$ distribution to nRGVs, but nRGVs have a slightly more rotational-supported than RGs.

% Figure environment removed

\subsection{Environment}\label{s55}
To examine the environmental dependence of RGs, we cross-match our parent sample with the group catalog of \citet{Tempel2012} that applied a modified friends-of-friends method to the SDSS DR8, obtaining 7585 galaxies with environmental information.
We consider those with the richness $n_{\rm rich} =1 $ to be in the field or void environment, while those with $n_{\rm rich}>4$ to be in the cluster environment.

Figure~\ref{fig;dens} compares among galaxy types the distribution of normalized environmental densities within $r=1h^{-1}~{\rm Mpc}$ that are calculated by \citet{Tempel2012} from galaxy luminosity density fields in the same manner as \citet{Liivamagi2012}.
We find that the density distribution of RGs is intermediate between those of QGs and SFGs, being similar to GVs.
This trend is similar to the morphology trend found in section~\ref{s54} and consistent with previous studies such as \citet{Schawinski2007, Thomas2010, Chauke2019} in that RGs tend to be in lower-density environments than QGs. 

% Figure environment removed

Next, we draw a phase space diagram (\cite{Bertschinger1985}) for galaxies in the cluster environment ($n_{\rm rich}>4$), with the clustocentric radius $r_{\rm sep}$ normalized by virial radius $r_{\rm vir}$ being the horizontal axis and the clustocentric velocity in the line-of-sight direction $v_{\rm LOS}$ normalized the velocity dispersion in the cluster $\sigma_{\rm cl}$ being the vertical axis, using the group catalog by \citet{Tempel2012}.
Following \citet{Rhee2017}, we classify galaxies into five regions, A, B, C, D, and E, according to their positions on the phase-space diagram as shown in figure~\ref{fig;ps}.
\citet{Rhee2017} have reported that region A is dominated by interlopers and galaxies that have fallen into the cluster recently.
In contrast, region E is dominated by galaxies that have been in the cluster for a long time.

Table~\ref{tab;frac_env} shows $f_{\rm RG}$ for the five regions and the field.
We find that $f_{\rm RG}$ is higher in region A and the field than in regions B-D and E.
We also calculate $f_{\rm RG/nRQG}$, the ratio of the number of RGs to the number of nRQGs.
As RGs were QGs before rejuvenation, we consider $f_{\rm RG/nRQG}$ as an indicator of the fraction of QGs that caused rejuvenation, i.e., how likely rejuvenation occurs in QGs.
$f_{\rm RG/nRQG}$ is found to be highest in the field and decreases toward the cluster center in the order of regions A, B-D, and E.
This suggests that rejuvenation is more likely to occur in galaxies in the field and galaxies that have just entered a cluster than in galaxies that have been in a cluster for a long time.

\begin{table*}[]
\caption{The number of each type of galaxies for each environment. 
The error in the fraction only includes the Poisson error.
}
\label{tab;frac_env}
\begin{tabular}{ll | lllll | ll}\hline\hline
                             &       & \multicolumn{5}{c|}{Number}     & \multicolumn{2}{c}{Fraction} \\
                             &       & All & RG & nRSFG & nRGV & nRQG & $f_{\rm RG}$: RG/All       & $f_{\rm RG/nRQG}$; RG/nRQG       \\\hline
\multicolumn{2}{c|}{All Environment}              & 6743 & 731 & 2831 & 376 & 2805 & $10.8\pm0.4\%$ & $26.1\pm1.1\%$ \\
\multicolumn{2}{c|}{Field ($N_{\rm rich}=1$)} & 2063 & 221 & 1179 & 122 & 541 & $10.7\pm0.8\%$ & $40.9\pm3.3\%$ \\
\multirow{3}{*}{Cluster ($N_{\rm rich}>4$)} & A     & 1056 & 129 & 362 & 65 & 500 & $12.2\pm1.1\%$ & $25.8\pm2.5\%$ \\
                             & B-D   & 1294 & 102 & 326 & 69 & 797 & $7.9\pm0.8\%$ & $12.8\pm1.3\%$ \\
                             & E     & 272 & 20 & 38 & 13 & 201 & $7.4\pm1.7\%$ & $10.0\pm2.3\%$\\\hline
\end{tabular}
\end{table*}





% Figure environment removed

