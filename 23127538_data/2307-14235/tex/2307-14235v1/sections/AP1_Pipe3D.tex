Several studies have estimated galaxy parameters for the MaNGA sample with SED fitting, although their results cannot be used to select RGs because they are based on parametric SFHs.
In this subsection, to further demonstrate the validity of our SED fitting results with {\sc Prospector}, we compare the estimated basic parameters with existing SED fitting results in the Pipe3D DR17 catalog \citep{Sanchez2022}.
This catalog summarizes the results of a stellar population and ionized gas analysis of the MaNGA data set using the pyPipe3D pipeline \citep{Lacerda2022}.
Note that the true values of individual parameters, like assumed mock values in section \ref{s41}, are unavailable and hence that we cannot know which SED fitting results are closest to the truth.
However, we expect our results to be generally consistent with the values in the Pipe3D catalog.

% Figure environment removed

% Figure environment removed

The Pipe3D catalog has two types of SFR; {\tt log\_SFR\_ssp} obtained from simple stellar population analysis and {\tt log\_SFR\_Ha} obtained from the Ha flux.
We use both SFRs in the Pipe3D catalog in the comparison.
Lines 1 and 2 of figure~\ref{fig;pipe3d} show that the SFRs estimated with {\sc Prospector} strongly correlate with those from the Pipe3D catalog for GVs and SFGs.
Indeed, Spearman's correlation coefficient calculated for all galaxies is $\rho\simeq0.75$ for SSP-based SFRs and $0.88$ for H$\alpha$-based ones.
We note, however, that the SFRs by {\sc Prospector} are systematically lower by about 0.6 dex.
In contrast, the correlation is very weak for QGs, especially those with ${\rm SFR}\lesssim 10^{-1}M_\odot~{\rm yr^{-1}}$.

Line 1 of figure~\ref{fig;pipe3d-2} shows that 
the stellar masses by {\sc Prospector} strongly correlate with those in the Pipe3D catalog ({\tt log\_Mass}), with a Spearman's correlation coefficient calculated for all types of galaxies of $\rho\simeq0.96$, although {\sc Prospector} tends to give slightly higher masses for QGs and GVs.
We also find that SFGs with higher sSFR in {\sc Prospector} tend to have lower $M^*$ than in the Pipe3D.

In our RG selection, we classify a galaxy into an SFG, GV, or QG at each of the eight age bins using the sSFR and $t_H$ of the bin.
Because the redshift range of galaxies in the parent sample is very narrow ($0.01\le z\le0.15$), the differences in $t_H$ at each age bin among the galaxies are less than $\sim 15\%$.
Therefore, comparing {\sc Prospector}'s sSFRs in the most recent age bin with the Pipe3D's values provides a sensitive test for the validity of our RG selection.

Lines 3 and 4 of figure~\ref{fig;pipe3d} compare sSFRs 
for each galaxy type.
For the whole sample, we obtain a high Spearman's correlation coefficient of $\rho\simeq0.84$ (vs. SSP-based sSFR) and $0.91$ (H$\alpha$-based sSFR).
However, as seen from the figure, the correlation in a given galaxy type is not so strong, being weaker than that for SFR and $M^*$.
This is mainly because the classification of SFGs, GVs, and QGs is based on sSFR estimated by {\sc Prospector}; thus, the distribution in each panel has a cutoff at an sSFR.

Line 2 of figure~\ref{fig;pipe3d-2} shows that the $A_{\rm V}$ estimated with {\sc Prospector} are weakly correlated with those from the Pipe3D catalog ({\tt Av\_ssp\_Re}) with a Spearman's correlation coefficient (for all galaxies) of $\rho\simeq0.48$.
The correlation is weakest for QGs.
Similarly, we find  a very weak correlation for $Z$ ({\tt ZH\_LW\_Re\_fit} in Pipe3D) in line 3 of figure~\ref{fig;pipe3d-2}.


In summary, the values of SFR, $M^*$, and sSFR by {\sc Prospector} correlate strongly with those from the Pipe3D catalog despite the presence of systematic offsets.
Therefore, we conclude that  our results are reliable enough to select RGs.
The criteria for RG selection are based solely on {\sc Prospector}' outputs.
This means that the systematic offsets from  Pipe3D's values will not significantly affect our classification.