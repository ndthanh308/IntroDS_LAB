In this paper, we have presented the {\sc Hinotori} project that aims at understanding the nature of RGs using the SFH.
As the first step of this project, we have performed a flexible SED fitting analysis using {\sc Prospector} on 8857 MaNGA galaxies with photometric and spectroscopic data and constructed a sample of 1071 RGs, the largest ever SFH-selected RG sample.
Our main results are summarized as follows:
\begin{itemize}
    \item Tests with mock data show that our selection method can select $f\sim0.001$ RGs with high completeness ($\sim67\%$).
    \item The RG fraction is $f_{\rm RG}=8.1\%$ after volume-weight correction.
    The selected RGs contribute $\sim 20\%$ of the CSFRD in the recent 100~Myr.
    \item From a rough estimate of $N_{\rm rej}$, we find that single galaxies can rejuvenate multiple times. 
    Multiple rejuvenations may significantly contribute to the mass assembly of galaxies, while a single rejuvenation increases the stellar mass only $\sim 0.1\%$.
    \item RGs have more disk-like morphology than nRQGs, suggesting the ``selective rejuvenation'' scenario where disk-like QGs rejuvenate more likely than elliptical QGs.
\end{itemize}

The next goal of {\sc Hinotori} is to specify the mechanism of rejuvenation.
We plan to analyze the IFU data of the selected RGs in the future.
Using simulation data is also an effective way to constrain the rejuvenation mechanism.
Furthermore, radio observations of molecular gas with, e.g., ALMA will allow us to discuss the star formation activity in RGs in more detail.
We also plan to examine the relationship between AGN activity and rejuvenation (c.f., \cite{Martin-Navarro2022}).

Recent studies have found high-$z$ low-$M^*$ QGs ($M^*\sim5\times10^8 M_\odot$ at $z\sim7.3$, \cite{Looser2023a}) and post-starburst galaxies ($M^*\sim4\times10^7 M_\odot$ at $z\sim5.2$, \cite{Strait2023}) in JWST data.
These findings may imply that high-$z$ low-mass QGs evolve into more massive galaxies at lower $z$ through repeated rejuvenation events (c.f., \cite{Looser2023a, Strait2023, Dome2023}).
To evaluate the importance and the role of rejuvenation in galaxy evolution, we have to explore RGs in a wide redshift range.
Besides, the $z$-dependence of $f_{\rm RG}$ will place stronger constraints on the possibility of multiple rejuvenations.
High-$z$ RG exploration needs a deep infrared spectroscopic survey with such as JWST/NIRSpec, TAO/SWIMS, Roman/WFI, and Euclid/NISP.
The development of a low-cost RG selection method (e.g., \cite{Zhang2022}) is another effective way to select RG candidates for follow-up spectroscopic observation and advance RG research.
As our results with mock data suggest, the outshining problem is serious when we reconstruct the SFHs of galaxies with strong rejuvenation.
Spatial resolved (or pixel-by-pixel) SED fitting may be able to solve this problem as \citet{Sorba2018} and \citet{Gimenez-Arteaga2023} suggested.


