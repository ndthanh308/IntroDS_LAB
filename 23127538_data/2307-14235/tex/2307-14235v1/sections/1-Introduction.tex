Galaxies segregate into two distinct groups in the star-formation rate (SFR) - stellar mass ($M^*$) diagram: ``star-forming galaxies'' (SFGs) with active star formation and ``quiescent galaxies'' (QGs) with low-to-no star formation (e.g., \cite{Renzini2015, Feldmann2017}).
Because the fraction of QGs increases with cosmic time (e.g., \cite{Ilbert2013, Muzzin2013, Davidzon2017, Weaver2022}), a widely-accepted simple galaxy evolution scenario is that SFGs stop star formation (quenching) and become QGs (e.g., \cite{Faber2007, Peng2010}).
Galaxies in between SFGs and QGs are called ``green valley'' galaxies (GVs).
GVs are widely thought to be galaxies in transition from SFGs to QGs (e.g., \cite{Salim2014,Angthopo2020}).

Recent progress in the spectral energy distribution (SED) fitting technique (\cite{Walcher2011} for review) has enabled us to search for and analyze galaxies with unique star-formation histories (SFHs), such as ``rejuvenating galaxies", which have been overlooked in previous studies. Rejuvenating galaxies have resumed their star formation after being quiescent (e.g., \cite{Trayford2016,Cleland2021,Zhang2022}).
We may need to update the simple evolution scenario from SFGs to QGs to explain rejuvenating galaxies.

Some studies (e.g., \cite{Chauke2019,Mancini2019,Tacchella2022}) have also focused on ``rejuvenated galaxies''.
The difference between ``rejuvenating galaxies'' and ``rejuvenated galaxies'' is that the latter have terminated their resumed star-formation and have been quiescent again.
Since both populations have experienced rejuvenation events, this paper deals with both populations and collectively refers to them as ``rejuvenation galaxies'' (RGs).
We do not use the abbreviation ``RG'' when distinguishing between ``rejuvenated galaxies'' and ``rejuvenating galaxies''.

RGs have been studied since the 2000s. 
Early studies, which are limited to the low-$z$ universe, have selected early-type galaxies (ETGs) with recent star formation as RGs by using UV detection (\cite{Kaviraj2007,Donas2007,Schawinski2007}), spectral and photometric features (\cite{Treu2005,Cleland2021}), or a stellar population synthesis (\cite{Thomas2010}).
The fraction of RGs (or ETGs with recent star formation) in previous samples ranges from $10\%$ to $30\%$ depending on the selection method and the properties of the parent samples.
Recent developments in SED fitting methods and spectroscopic surveys in the intermediate-$z$ universe enabled selecting $0.6\lesssim z\lesssim 2$ RGs based on their SFHs or stellar ages reconstructed by SED fitting (\cite{Belli2017,Gobat2017,Carnall2019,Chauke2019,Mancini2019,Akhshik2021,Tacchella2022,Paspaliaris2023}).

{\bf The mechanism and cause of rejuvenation} and {\bf the role of rejuvenation in galaxy evolution} are major open questions. On the cause of rejuvenation, some studies have suggested that rejuvenation is related to mergers. For example, 
\citet{Kaviraj2009} have used a simulation to show that a minor merger can explain the recent star formation of ETGs. Other studies have also suggested that gas accretion onto elliptical galaxies after merging may cause rejuvenation and enables rejuvenated ellipticals to acquire a disk and turn into red spirals or S0 galaxies (e.g., \cite{Mapelli2015,Diaz2018,Hao2019,Himansh2022}). 
\citet{Paspaliaris2023} have argued that HI gas, as found in some ETGs by \citet{Thom2012}, could cause rejuvenation.
Note, however, that \citet{Park2022} have suggested that red disk-like QGs cannot be described with the above merger-rejuvenation scenario.
Besides, \citet{Martin-Navarro2022} have reported that many type-I active galactic nuclei (AGN) had experienced rejuvenation, suggesting the relationship between rejuvenation events and AGN activities.

Understanding the rejuvenation mechanism will deepen our knowledge of the star formation activity in a galaxy.
In particular, we can consider RGs as former QGs that could not maintain a quiescent state.
Comparing RGs with QGs enables us to discuss the quenching process, especially the conditions needed to maintain a quiescent state.

{\bf The role of rejuvenation in galaxy evolution} also remains to be evaluated.
\citet{Chauke2019} and \citet{Tacchella2022} have reported that, at most, only about 10\% of the total mass is formed in a rejuvenation event.
However, we can detect only recent rejuvenation events because the time resolution of the reconstructed star formation history of a galaxy decreases with look-back time.
Thus, we may underestimate the contributions of rejuvenation events to the total star-formation activities.
The relationship between RGs and GVs is also relevant to the role of rejuvenation.
GVs are widely thought to be in the quenching phase.
However, this simple view needs to be modified if a significant fraction of GVs are RGs.
Furthermore, \citet{Mancini2019} reported that the bending of the star-forming main sequence (SFMS) in the high-mass end is caused by RGs.

Despite the importance of RGs, previous studies have failed to discuss the nature of RGs statistically.
This is mainly because the previous studies are based on either small samples of RGs or biased parent samples (e.g., RGs are selected from only QGs).
For example, \citet{Tacchella2022} have found that RGs reside in massive dark halos based on nine RGs selected from 161 QGs. 
\citet{Cleland2021} have detected no significant environmental dependence of rejuvenation fraction in a sample of $\simeq 350$ RGs. However, these results need to be confirmed with a larger RG sample and with a more inclusive parent sample not limited to QGs. In addition, different definitions of RGs among the previous studies have also made it difficult to compare their results with each other and to compare them with simulation results (e.g., \cite{Kaviraj2009,Trayford2016,Pandya2017,Nelson2018,Behroozi2019,Alarcon2022}).

We launch the {\sc Hinotori}\footnote{star-formation History INvestigatiOn TO find RejuvenatIon. Hinotori means phoenix in Japanese.} project to solve those problems.
{\sc Hinotori} aims to understand the nature of RGs using well-constrained SFHs.
Selecting RGs with reliable SFHs is the most direct selection method.
As the first step of {\sc Hinotori}, we select RGs from a large ($N\sim10^4$) parent sample of $z\simeq 0$ galaxies covering all morphological types.
To reconstruct the SFH for individual galaxies, we run the Bayesian SED fitting library {\sc Prospector} \citep{Leja2017, Johnson2021} on their spectroscopic and UV to MIR photometric data.
In this paper, we present the most extensive catalog to date of SFH-selected RGs and make the first secure statistical analysis of RGs.

Section~\ref{s2} describes the data and how to make the input data for {\sc Prospector}.
Section~\ref{s3} describes the SED fitting method and the selection method of RGs.
In Sections~\ref{s4} and \ref{s5}, we show the results from a mock sample and real data, respectively.
We discuss the nature of RGs in Section~\ref{s6} and present a conclusion and prospects for future RG studies in Section~\ref{s7}.
Throughout this work, we assume the cosmological parameters from the WMAP-9 \citep{Hinshaw2013}.
