We run {\sc Prospector} with mock data to verify that our method works well.
Since SED fitting with a non-parametric SFH has been pointed out to have prior dependence \citep{Leja2019}, we also discuss prior dependence.

\subsection{Making mock data}\label{s41}
We make mock data at $z=0.1$ with {\sc Prospector}.
Since we also use {\sc Prospector} in the SED fitting (see section~\ref{s31} for the fitting process), we cannot evaluate the validity of the model spectra in this validation.
The purpose of using a mock sample is only to discuss whether our fitting process can reconstruct the correct SFH and select RGs even with noise.

We construct two mock samples, a single-peak-SFH sample and a double-peak-SFH sample, based on the delayed-tau SFH, one of the popular parametric SFHs.

{\bf \underline{Double-peak-SFH mock sample:}}
To examine if our selection method can select RGs and distinguish RGs from non-rejuvenation galaxies (nRGs) with RG-like SFHs, we make mock data with double-peak SFHs that are a superposition of two delayed-tau models as follows,
\begin{align}
    {\rm SFH}\left(t\right) = C_1 &\left(t_1-t_l\right) \exp\left(-\frac{t_1-t_l}{\tau_1}\right) + \nonumber \\ 
    &
    \begin{cases}
      C_2 \left(t_2-t_l\right) \exp\left(-\frac{t_2-t_l}{\tau_2}\right)  & \left(t_l>t_1\right) \\
      0 & \left(t_l >t_2\right),
    \end{cases}
    \label{eq:sfh}
\end{align}
where $\tau_1$ ($\tau_2$) is the time scale of the first (second) delayed tau model, $t_1$ ($t_2$ [$<t_1$]) is the lookback time when the first (second) delayed tau star formation starts, and $C_1$ ($C_2$) is a coefficient of the first (second) delayed tau model.
We calculate $C_1$ and $C_2$ from $f$, the ratio of the formed mass in the second delayed tau SFH to the total $M^*$.
We fix $t_1$ as $0.95 t_H$, and $M^*$ as $10^{11} M_\odot$.
In equation~(\ref{eq:sfh}), the first term shows the typical galaxy evolution scenario of quenching after active star formation. The second term corresponds to a rejuvenation or pseudo-rejuvenation event.
Note that the first term has a negative value in $t_l>t_1$; thus, we calculate the SFH only for $t_l>t_1$.

To examine various RGs and pseudo RGs, we generate 144 SFHs as a combination of the following parameters,
\begin{itemize}
    \item {\bf $f$}: $10^{-1}$, $10^{-2}$, $10^{-3}$, $10^{-4}$
    \item {\bf $\tau_1$}: 1 Gyr, 1.5 Gyr, 2 Gyr
    \item {\bf $\left(t_H-t_2\right)/\tau_2$}: 1, 10, 50
    \item {\bf $t_H-t_2$}: $10^{7}$ yr, $10^{7.5}$ yr, $10^{8}$ yr, $10^{8.5}$ yr. 
\end{itemize}

The specific star-formation histories (sSFHs) of the mock data are shown in figure~\ref{fig;mock_RG}.
Note that although these mock galaxies have two star-formation peaks, not all mock galaxies are RGs.
For example, if we apply our definition introduced in section~\ref{s32}, some galaxies were SFGs or GVs just before the start of the second star formation, and these are not classified as RGs.
Besides, we should also note that some galaxies with $f\sim0.1$ have extremely high SFRs such as $\log\left({\rm SFR}/M_\odot\ {\rm yr^{-1}}\right)\gtrsim3$.

% Figure environment removed

We calculate mock SEDs with the above modeled SFHs.
We add dust and nebular emissions to the stellar continuum.
We fix $\log\left(Z/Z_\odot\right)$ at $-0.5$ and $\hat{\tau}_2$ at $ 0.3$, which are similar to the estimated values for our galaxy sample as shown in appendix.
We assume a \citet{Chabrier2003} initial mass function as in the case of real data.
Then, we use the mock SEDs to calculate mock photometric and spectral data by adding noise following the $S/N$ values shown in the table~\ref{tab:sn}.
These $S/N$ are mainly based on the median $S/N$ for the main sample, but we fix the mock $S/N$ to $20$ for $S/N>20$ and $5$ for $S/N>5$ WISE photometry to account for possible systematic errors, such as aperture correction errors.
Thus, our mock data are more conservative than the real data in terms of $S/N$.

\begin{table*}[]
\caption{Median $S/N$ for the main sample and the assumed $S/N$ in making mock observation data.}
\begin{tabular}{llllllllllllll}
       & \multicolumn{2}{c}{GALEX}                         & \multicolumn{5}{c}{SDSS}                                                                                              & \multicolumn{4}{c}{WISE}                                                                          & \multicolumn{2}{c}{MaNGA}                          \\
       & \multicolumn{1}{c}{FUV} & \multicolumn{1}{c}{NUV} & \multicolumn{1}{c}{$u$} & \multicolumn{1}{c}{$g$} & \multicolumn{1}{c}{$r$} & \multicolumn{1}{c}{$i$} & \multicolumn{1}{c}{$z$} & \multicolumn{1}{c}{W1} & \multicolumn{1}{c}{W2} & \multicolumn{1}{c}{W3} & \multicolumn{1}{c}{W4} & \multicolumn{1}{c}{blue} & \multicolumn{1}{c}{red} \\\hline\hline
Median & 5.1                     & 17.6                    & 42.9                  & 290.1                 & 357.6                 & 341.3                 & 167.6                 & 485.7                  & 155.7                  & 17.8                   & 4.5                    & 23.0                     & 46.6                    \\
Mock   & 5.1                     & 17.6                    & 20                    & 20                    & 20                    & 20                    & 20                    & 5                      & 5                      & 5                      & 4.5                    & 20                       & 20                     
\end{tabular}
\label{tab:sn}
\end{table*}

{\bf \underline{Single-peak-SFH mock sample:}}
To discuss the possibility that galaxies without rejuvenation or a second star-formation event are erroneously selected, we also generate mock data with single peak SFHs using the single delayed-tau model as,
\begin{align}
    {\rm SFH}\left(t\right) = C_1 &\left(t_1-t_l\right) \exp\left(-\frac{t_1-t_l}{\tau_1}\right).
    \label{eq:sfh2}
\end{align}
This time, we also change $t_1$: the lookback time of starting star formation.
We generate 25 SFHs as a combination of the following parameters,
\begin{itemize}
    \item {\bf $\tau_1$}: $2.00\times10^{8}$ yr, $3.56\times10^{8}$ yr, $6.32\times10^{8}$ yr, $1.12\times10^{9}$ yr, $2.00\times10^{9}$ yr
    \item {\bf $t_H-t_1$}: $1.25\times10^{9}$ yr, $2.19\times10^{9}$ yr, $3.84\times10^{9}$ yr, $6.74\times10^{9}$ yr, $1.18\times10^{10}$ yr,
\end{itemize}
where $\tau_1$ is equally sampled in log space between $2\times10^8$ yr and $2\times10^9$ yr, and $t_1-t_l$ is equally sampled in log space between $0.1t_H$ and $0.95t_H$.

The sSFHs of the single-peak-SFH mock sample are shown in figure~\ref{fig;mock_nRG}.
They have a wide range of current sSFRs over the SFR, GV, and QG regimes and have never experienced a second active star formation, including rejuvenation.

% Figure environment removed

In the calculation of mock SEDs, we make the same assumptions for the metallicity, dust extinction, and IMF as for the double-peak-SFH mock sample. 

\subsection{Basic parameters}\label{s42}
Using the settings described in section~\ref{s31}, we run {\sc Prospector} on the mock galaxies and estimate their SFH and other parameters.

We compare the estimated parameters with the assumed ones in the mock data in figure~\ref{fig:mock_param}.
Figures~\ref{fig:mock_param}~(a), (b), and (c) shows the distribution of $M^*$, ${\rm Z}$, and $\hat{\tau}_{\rm 2, dust}$, respectively.
We find that {\sc Prospector} reproduces the values of these parameters well.
We also find that there are no large differences between the results for the two priors.

Next, we test how well the assumed SFHs are reconstructed by calculating mean SFRs in three bins: $t_l<100~{\rm Myr}$, $100~{\rm Myr}<t_l<1000~{\rm Myr}$, and $t_l>1000~{\rm Myr}$ (figure~\ref{fig:mock_param}~(d), (e), and (f), respectively).
Figure~\ref{fig:mock_param}~(d) compares ${\rm SFR_{100~Myr}}$, the SFR in $t_l<100~{\rm Myr}$, showing a strong correlation between the mock and estimated values.
For galaxies with low assumed SFRs ($\log{\rm SFR_{100~Myr}}\lesssim-0.5$), the continuity prior reproduces the mock values better than the Dirichlet prior.
However, both priors overestimate the SFRs by about 0.3-0.4 dex.

Figure~\ref{fig:mock_param}~(e) compares ${\rm SFR_{100-1000~Myr}}$, the SFR in $100~{\rm Myr}<t_l<1000~{\rm Myr}$, finding a similarly strong correlation.
For galaxies with medium assumed SFRs ($\log{\rm SFR_{100-1000~Myr}}\sim0$), the continuity prior often underestimates ${\rm SFR_{100-1000~Myr}}$.
On the other hand, for galaxies with low assumed SFRs ($\log{\rm SFR_{100-1000~Myr}}\sim-2$), the Dirichlet prior often overestimates ${\rm SFR_{100-1000~Myr}}$.

Figure~\ref{fig:mock_param}~(f) compares ${\rm SFR_{before~1000~Myr}}$, the SFR in $t_l>1000~{\rm Myr}$, and finds a weak correlation in both priors. 
This study does not focus on past star formations.
However, because we define RGs with sSFHs, underestimation of the old stellar population may cause  an overestimation of the current sSFR and affect our selection.
We discuss this problem in section \ref{s43}.

The distribution of estimated metallicities (figure~\ref{fig:mock_param}~(b)) shows a small peak with lower metallicities ($\log\left({\rm Z^*}/{\rm Z_\odot}\right)\lesssim-0.8$) in addition to the primary peak around the assumed value ($\log\left({\rm Z^*}/{\rm Z_\odot}\right)\sim-0.5$).
The galaxies around the small peak have underestimated masses ($\log\left(M^*/M_\odot\right)\lesssim10.7$).
The double-peak-SFH mock sample have 16 galaxies with $\log\left({\rm Z^*}/{\rm Z_\odot}\right)<-0.8$, and 15 of them ($\sim94\%$) have $t_H-t_2=10^7~{\rm yr}$.
Besides, ten and six (about 63 and 38\%) have $f=0.1$ and $0.01$, respectively.
The ${\rm SFR_{before~1000~Myr}}$ of these galaxies are greatly underestimated.
It is likely that {\sc Prospector} cannot reconstruct old star formation due to an extremely large contribution from recent star formation, i.e., the young stellar population.
Thus, {\sc Prospector} may underestimate the stellar mass and metallicity of galaxies with very recent and active resumed star formation.
However, since many $f=0.1$ SFHs have extremely high SFRs over $10^3M_\odot~{\rm yr}^{-1}$ for low-$z$ galaxies, this underestimation is not a serious problem in applying to real data.
However, this underestimation may need to be considered to search for RGs in the high-z universe 
because they have low stellar masses and high SFRs.

% Figure environment removed

To evaluate how well {\sc Prospector} can detect rejuvenation events, we calculate $\mu$, the ratio of the formed mass in the recent 100~Myr to the total stellar mass.
Note that $\mu$ is different from $f$ because we calculate $\mu$ by dividing the formed mass in the recent 100~Myr by the total $M^*$, while calculating $f$ by dividing the formed mass in the second delayed-tau model by the total $M^*$.
Although estimating $f$ from reconstructed SFHs is challenging, $\mu$ can be easily calculated.
Figure~\ref{fig:mock_mu} compares estimated $\mu$ with assumed ones.
We find that {\sc Prospector} tends to overestimate $\mu$ about 0.5~dex.
This tendency is mainly 
due to the overestimation of ${\rm SFR_{100~Myr}}$ (about 0.4~dex higher).

% Figure environment removed

\subsection{RG selection}\label{s43}
The correlation between $p_{\rm RG, con}$ and $p_{\rm RG, Dir}$ for mock and real galaxies is shown in figure~\ref{fig:mock_rg}~(a).
We classify mock galaxies into RGs and nRGs in the same manner as for real galaxies using reconstructed SFHs.
These mock RGs and nRGs are shown in purple and orange, respectively, in figure~\ref{fig:mock_rg}~(a).
We find that almost all the mock RGs have $p_{\rm RG, con} >p_{\rm RG, Dir}$.

For real galaxies, a high Spearman's correlation coefficient of $\rho\simeq 0.79$ is obtained.
However, the value drops to $\rho\simeq 0.38$ if limited to sources with intermediate $p_{\rm RG}$ ($0.1<p_{\rm RG,con}<0.9$), suggesting that the high coefficient value for all galaxies is caused by very RG-like ($0.9<p_{\rm RG}$) and very non-RG-like ($p_{\rm RG}<0.1$) galaxies.
We note that both priors are relatively consistent with each other in that they can correctly select both very RG-like and very non-RG-like objects.

Despite the high correlation coefficient, real galaxies (and mock galaxies) are not evenly distributed around the equality line; objects with medium $p_{\rm RG, con}$ values tend to be lower than the line.
This distribution may be explained as follows.
As seen in section~\ref{s42}, the continuity prior tends to give lower ${\rm SFR}_{\rm 100-1000~Myr}$ and higher ${\rm SFR}_{\rm before~100~Myr}$ than the Dirichlet prior, especially for low SFR objects.
This means that the continuity prior tends to reconstruct more V-shaped, or more RG-like, SFHs than the Dirichlet prior.
Thus, $p_{\rm RG,con}$ is likely to be higher than $p_{\rm RG,Dir}$ especially for low to medium $p_{\rm RG, con}$ objects, 
which are often located around the boundary of QGs and GVs.

Considering these trends in the mock and main parent samples, we set up three regions (A), (B), and (C) in the $p_{\rm RG, Dir}$ - $p_{\rm RG, con}$ plane as candidate regions for RG selection and discuss which region selects RGs best.
Regions (A), (B), and (C) are defined as, 
\begin{itemize}
    \item (A) :$p_{\rm RG,Dir}>0.5$ and $p_{\rm RG,con}>0.8$,
    \item (B) :$p_{\rm RG,Dir}<0.4$ and $p_{\rm RG,con}>0.6$,
    \item (C) :$p_{\rm RG,Dir}<0.2$ and $p_{\rm RG,con}<0.2$.
\end{itemize}
The SFHs and sSFHs of mock RGs and mock nRGs in each region are shown in figure~\ref{fig:mock_rg}~(b).
The parameters of the single-peak-SFH mock sample in each region are also shown in table~\ref{tab:params_region}.

% Figure environment removed

In region (A), we have 12 mock galaxies, and six of them are classified as RGs with mock SFHs.
All six RGs have $f=0.001$ and $t_H-t_2=10^7~{\rm yr},10^{7.5}~{\rm yr}$.
Similarly, all the mock nRGs in region (A) also have $f=0.001$ and $t_H-t_2=10^7~{\rm yr}$.
From these results, galaxies in region (A) are likely to have experienced a very recent second SF with $f\sim0.001$.

Then, in region (B), we have 18 mock galaxies, and five of them are classified as RGs with mock SFHs.
As far as we can judge from the mock data, the contamination rate in region (B) is 72\%, which is higher than in region (A) (50\%).
Region (B) has more mock galaxies with $f=0.01$ and $t_H-t_2=10^{7.5}~{\rm yr}$ than region (A).

Finally, in region (C), we have 106 mock galaxies, and 21 of them are classified as RGs with mock SFHs.
The contamination rate is 80\%, the highest in the three regions.
Most single-peaked-SFH mock galaxies ($22/25$) are distributed in region (C), meaning that our method can correctly diagnose single-peaked-SFH galaxies as nRGs.
We find that mock RGs with $f=0.0001, 0.01,$ and $0.1$ are concentrated in region (C) while there are no mock RGs with $f=0.001$ there.
Besides, most of the mock RGs with $t_H-t_2\gtrsim10^8~{\rm yr}$ are in region (C).
Although our method focuses on rejuvenation within $10^8~{\rm yr}$, rejuvenation events starting from $t_l=10^8~{\rm yr}$ may not be detected in our method.

Based on these results, we choose the region for RG selection.
If we use region (A) to select RGs, the contamination rate of RG selection in mock data is 50\% ($6/12$).
However, the contaminants, i.e., nRGs in region (A), also have a recent second star formation with $f=0.001$.
Besides, region (A) does not include single-peaked-SFH mock galaxies.
Therefore, region (A) can select RGs and RG-like secondary-star-formation galaxies with $f\sim0.001$. 
Since the parameter distribution of mock galaxies should not be the same as the distribution of real galaxies, we cannot calculate the exact contamination rate.
We can only affirm the possibility of selecting such contaminants.

Next, the completeness of RG selection using region (A) is found to be 17\% ($6/36$).
However, similar to the above discussion, we cannot know the exact completeness with our mock data.
For example, as shown in figure~\ref{fig:mock_rg}(b), the mock SFHs in regions (B) or (C) have extremely high SFRs like ${\rm SFR}\gtrsim10^2M_\odot~{\rm yr^{-1}}$.
In the low-$z$ Universe, galaxies with such extremely high SFRs are extremely rare, and the settings of mock data may not be realistic.
If we focus only on galaxies with $f=0.001$, the contamination is $67\%$ ($6/9$).
The results with mock data suggest that using region (A), we can detect weak rejuvenation events with high completeness, although maybe missing galaxies with strong rejuvenation.
From these results, we decide to use region (A).

As for contamination, region (A) may select galaxies with a second star formation whose progenitors are not QGs but GVs or SFGs.
As for completeness, region (A) can miss galaxies that have experienced strong rejuvenation with $f\gtrsim0.01$ or weak rejuvenation with $f\sim0.0001$.
However, region (A) can select RGs with $f\sim0.001$ with high completeness.

We use the results with the continuity prior in the remaining part of this paper.
This is mainly because the continuity prior has a stronger correlation in ${\rm SFR_{100~Myr}}$ (figure~\ref{fig:mock_param} (d)) and in $\mu$ (figure~\ref{fig:mock_mu}).
Besides, $p_{\rm RG, con}$ tends to be higher for $f=0.001$ mock RGs than $p_{\rm RG, Dir}$, suggesting that the continuity prior is better in reproducing $f=0.001$ rejuvenation events.
Note that our selection method uses both $p_{\rm RG, con}$ and $p_{\rm RG, Dir}$ as introduced in section~\ref{s32} and discussed above.

The low completeness of our selection method comes from difficulties in estimating non-parametric SFHs.
For large-$f$ galaxies ($f\gtrsim0.01$), the light from the young stellar population overwhelms that from the old stellar population, i.e., the outshining problem (c.f., \cite{Maraston2010,Sorba2018,Gimenez-Arteaga2023,Narayanan2023}).
In contrast, for small-$f$ galaxies ($f\sim0.0001$), the light from the young stellar population is buried by that from the old stellar population.
It will be important to reduce this uncertainty as much as possible by improving the method, e.g., improving the SED fitting technique and searching for indices to find RGs easily.
The latter will become especially important in the future as data are increasing rapidly.
Because the completeness of selecting $f=0.001$ mock RGs is high, we focus on $f\sim0.001$ RGs in this study.
We will discuss the effects of this incompleteness in section~\ref{s6}.

Again, note that because the mock data are generated with {\sc Prospector}, the comparison in this section cannot demonstrate the validity of model spectra.
To further demonstrate the validity of our SED fitting results with {\sc Prospector}, we also checked that the basic results, such as $M^*$ and SFR, estimated with {\sc Prospector} are strongly correlated with an existing SED fitting result, the Pipe3D DR17 catalog \citep{Sanchez2022}.
The detail of this comparison is introduced in appendix.

\begin{landscape}
\begin{table}[]
\caption{The number of mock galaxies with each mock parameter in each region.}\label{tab:params_region}
\begin{tabular}{ll | llll | lll | lll | llll}\hline\hline
                                         &                    & \multicolumn{4}{c}{$f$}     & \multicolumn{3}{|c|}{$\tau_1$}                               & \multicolumn{3}{|c|}{$\left(t_H-t_2\right)/\tau_2$} & \multicolumn{4}{|c}{$t_H-t_2$}                            \\
                                         &   & 0.0001 & 0.001   & 0.01   & 0.1    & 1 Gyr     & 1.5 Gyr                  & 2 Gyr                   & 1         & 10       & 50      & $10^{7}$ yr      & $10^{7.5}$ yr    & $10^{8}$ yr       & $10^{8.5}$ yr                   \\\hline
\multicolumn{1}{c}{\multirow{7}{*}{All}} & All                & 36  & 36    & 36     & 36     & 48     & 48                    & 48                    & 48        & 48       & 48      & 36     & 36     & 36      & 36                    \\
\multicolumn{1}{c}{}                     & \multirow{2}{*}{A} &0& 12      & 0      & 0      & 6      & 3                     & 3                     & 4         & 4        & 4       & 9      & 3      & 0       & 0                     \\
\multicolumn{1}{c}{}                     &                    &(0\%)& (33\%)  & (0\%)  & (0\%)  & (13\%) & (6.3\%)               & (6.3\%)               & (8.3\%)    & (8.3\%)   & (8.3\%)  & (25\%) & (8.3\%) & (0\%)   & (0\%)                 \\
\multicolumn{1}{c}{}                     & \multirow{2}{*}{B} &0& 2       & 10     & 6      & 5      & 8                     & 5                     & 2         & 9        & 7       & 3      & 14     & 1       & 0                     \\
\multicolumn{1}{c}{}                     &                    &(0\%)& (5.6\%) & (28\%) & (17\%) & (10\%) & (17\%)                & (10\%)                & (4.2\%)   & (19\%)   & (15\%)  & (8.3\%) & (39\%) & (2.8\%) & (0\%)                 \\
\multicolumn{1}{c}{}                     & \multirow{2}{*}{C} &36& 17      & 23     & 30     & 33     & 33                    & 39                    & 38        & 32       & 35      & 23     & 15      & 32      & 35                    \\
\multicolumn{1}{c}{}                     &                    &(100\%)& (47\%)  & (64\%) & (83\%) & (69\%) & (69\%)                & (81\%)                & (79\%)    & (67\%)   & (73\%)  & (64\%) & (42\%) & (89\%)  & (97\%)                \\\hline
\multirow{7}{*}{RG}                      & All                &9& 9       & 9      & 9      & 36     & 0                     & 0                     & 12         & 12        & 12       & 12      & 12      & 12       & 0                     \\
                                         & \multirow{2}{*}{A} &0& 6       & 0      & 0      & 6      & 0                     & 0                     & 2         & 2        & 2       & 3      & 3      & 0       & 0                     \\
                                         &                    &(0\%)& (67\%)  & (0\%)  & (0\%)  & (17\%) & - & - & (17\%)    & (17\%)   & (17\%)  & (25\%) & (25\%) & (0\%)   & - \\
                                         & \multirow{2}{*}{B} &0& 1       & 2      & 2      & 5      & 0                     & 0                     & 0         & 3        & 2       & 0      & 4      & 1       & 0                     \\
                                         &                    &(0\%)& (13\%)  & (22\%) & (22\%) & (14\%) & - & - & (0\%)     & (25\%)   & (17\%)  & (0\%)  & (33\%) & (8.3\%)  & - \\
                                         & \multirow{2}{*}{C} &9& 0       & 5      & 7      & 21     & 0                     & 0                     & 8         & 6        & 7       & 9      & 4      & 8       & 0                     \\
                                         &                    &(100\%)& (0\%)   & (56\%) & (78\%) & (58\%) & - & - & (67\%)    & (50\%)   & (58\%)  & (75\%) & (33\%) & (67\%)  & -\\\hline
\end{tabular}
\end{table}
\end{landscape}