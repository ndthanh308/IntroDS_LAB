% This is main.tex, a sample chapter demonstrating the
% LLNCS macro package for Springer Computer Science proceedings;
% Version 2.21 of 2022/01/12
%
\documentclass[runningheads]{llncs}
\usepackage[]{todonotes}
\usepackage[linesnumbered,lined,ruled,commentsnumbered]{algorithm2e}
\usepackage{amssymb}
\usepackage{float}
\usepackage{enumitem}

\setlist[itemize]{noitemsep}
\setlist[enumerate]{noitemsep}
%

\usepackage[T1]{fontenc}
% T1 fonts will be used to generate the final print and online PDFs,
% so please use T1 fonts in your manuscript whenever possible.
% Other font encondings may result in incorrect characters.
%
\usepackage{graphicx}
\usepackage{hyperref}
% Used for displaying a sample figure. If possible, figure files should
% be included in EPS format.
%
% If you use the hyperref package, please uncomment the following two lines
% to display URLs in blue roman font according to Springer's eBook style:
%\usepackage{color}
%\renewcommand\UrlFont{\color{blue}\rmfamily}
%

\newcommand{\draft}[1]{\textcolor{gray}{#1}}

\def\sectionautorefname{Section}
\def\algorithmautorefname{Algorithm}

\begin{document}
%
    \title{Hybrid Genetic Search for Dynamic Vehicle Routing with Time Windows}

    \author{Mohammed Ghannam\inst{1,2}\orcidID{0000-0001-9422-7916} \and \\
    Ambros Gleixner\inst{1,2}\orcidID{0000-0003-0391-5903}}

    \institute{HTW Berlin, Treskowallee 8, 10318 Berlin, Germany \\
    \and
    Zuse Institute Berlin, Takustr. 7, 14195 Berlin, Germany  \\
    \email{\{ghannam, gleixner\}@htw-berlin.de}
    }
%
    \maketitle              % typeset the header of the contribution
%

    \begin{abstract}
        The dynamic vehicle routing problem with time windows (DVRPTW) is a generalization of the classical VRPTW to an online setting,
        where customer data arrives in batches and real-time routing solutions are required.
        In this paper we adapt the Hybrid Genetic Search (HGS) algorithm, a successful heuristic for VRPTW,
        to the dynamic variant.
        We discuss the affected components of the HGS algorithm including giant-tour representation, cost computation,
        initial population, crossover, and local search.
        Our approach modifies these components for DVRPTW, attempting to balance solution quality and constraints on future customer arrivals.
        To this end, we devise methods for comparing different-sized solutions, normalizing costs, and accounting for
        future epochs that do not require any prior training.
        Despite this limitation, computational results on data from the EURO meets NeurIPS Vehicle Routing Competition 2022
        demonstrate significantly improved solution quality over the best-performing baseline algorithm.

        \keywords{Vehicle Routing  \and Online Optimization \and Metaheuristics \and Genetic Algorithms}

    \end{abstract}


    \section{Introduction}

    Vehicle routing has attracted a lot of attention in the past decades due to its prevalence in many
    branches of industry.
    One of the most prominent variants is the \emph{Capacitated Vehicle Routing problem with Time
    Windows} (VRPTW).
    The problem is defined on a directed graph $G = (V,A)$ where the set of nodes $V$ represents the customers
    and the set of arcs $A$ represents possible connections between nodes.
    Throughout the paper, nodes and customers are used interchangeably.
    Arc weights $c_a > 0$ are defined for all $a \in A$ to represent the cost for driving along the arc.
    For each node~$i$ a time window $[s_i, t_i]$ is given that represents the earliest and latest time of
    arrival at customer $i$. % a_i for earliest arrival time clashes with a for arcs
    A demand $d_i > 0$ is associated with each customer $i$ and
    the total demand of all customers in one route should not exceed a maximum capacity $K$.
    The goal is to minimize the total cost of the routes while ensuring feasibility.

    Exact approaches in the literature are mostly based on the branch-price-and-cut
    paradigm~\cite{costaExactBranchPriceandCutAlgorithms2019}.
    Current state-of-the art algorithms are able to solve most instances of up to $200$ customers to proven optimality~\cite{pecinNewEnhancementsExact2017}.
    In practice, however, the size of the instances can be much larger and optimality guarantees are often not required.
    % Although quite impressive, the optimality guarantee is hardly ever needed in practice, and the computational effort
    % required is hard to justify.
    Therefore, there has been much interest in efficient heuristic and meta-heuristic algorithms that can
    find high-quality solutions within a reasonable amount of time.
    For VRPTW, meta-heuristics have proved highly effective; finding optimal or near-optimal solutions within seconds
    and can scale to problems with thousands of nodes.
    In the $12^{th}$ DIMACS challenge~\cite{dimacs2021cvrp}, the \emph{Hybrid Genetic Search}
    (HGS)~\cite{vidalHybridGeneticAlgorithm2012} meta-heuristic in an implementation by Kool et al.~\cite{koolHybridGeneticSearch}
    was the top performing submission with respect to primal solution quality.

    In the recent \textit{EURO meets NeurIPS Vehicle Routing Competition}~\cite{euromeetsneurips2022}, a generalization
    of VRPTW was posed.
    The \emph{Dynamic VRPTW} (DVRPTW) is defined over epochs $t \in {1, 2, \ldots, T}$; with each epoch $t$
    a set of customers is given.
    The task is to select \emph{directly after each epoch}, i.e., without knowledge on future epochs, a subset of customers
    to dispatch and a feasible route for these customers.
    All customers appearing in any epoch must be dispatched eventually, so when a customer is not chosen in some epoch
    $t'$, then it will reappear in all future epochs~$t > t'$ until it is dispatched.

    Additionally, due to time window constraints, some nodes need to be dispatched in the current epoch.
    These are marked as ``\emph{must-go}''.
    In particular, all customers appearing in the last epoch must be dispatched.
    The objective is to minimize the total drive times of all routes computed in all epochs.
    The original VRPTW can be seen as a special case of DVRPTW with only one epoch.

    In this work, we introduce a general method for DVRPTW based on the state-of-the-art meta-heuristic HGS.
    %
    First, let us note that any method for VRPTW can be lifted
    trivially to the dynamic setting if we specify a strategy that
    chooses a fixed subset of customers to be dispatched at each
    epoch: simply perform, at each epoch, the VRPTW method on the
    fixed subset of customers in order to find high-quality routes.
    %
    In our experiments we include several baselines that follow precisely this scheme, calling HGS at each epoch on a fixed set of customers.
    
    By contrast, the motivation for our work is to extend HGS's solution management and local search techniques in order to decide \emph{in an integrated fashion}
    which customers to dispatch and which routes to use for serving the chosen customers.
    %
    The resulting algorithm DHGS is called at each epoch and does not require any pre-training on existing data.
%%     Certainly, the dependency of DVRPTW on online data makes it attractive for the application of advanced techniques from machine learning.
%%     % nature of the data calls for make approaches and the \draft{vastness} of the solution space makes it a good fit
%%     % for Machine Learning (ML) based approaches.
%%     However, although machine learning algorithms have been shown to be useful for some routing problems~\cite{koolDeepPolicyDynamic2021}, they tend to
%%     require prior training and can be prone to \textit{overfitting}.
%%     The recent study~\cite{santanaNeuralNetworksLocal2023} gives an example where the resulting improvement does not justify the added complexity.
%%     Additionally, due to their complexity it can become hard to reason about their performance.
%%     Ambros: Maybe it is easier to not start justifying our approch.

    The paper is organized as follows. \autoref{sec:hgs} presents a succinct description of the HGS algorithm.
    Our adaptions to DVRPTW are described in \autoref{sec:adapting}.
    In \autoref{sec:comp} we investigate how these adaptations perform computationally by comparing them to different baselines, and give concluding remarks.

    % ---------–––––––––––––––––––––––---------–––––––––––––––––––––––---------–––––––––––––––––––––––


    \section{Hybrid Genetic Search}\label{sec:hgs}

    Hybrid Genetic Search was first introduced by Vidal et
    al.~\cite{vidalHybridGeneticAlgorithm2012} for
    vehicle routing and has proven to be effective on many variants of
    the problem~\cite{vidalUnifiedSolutionFramework2014}.
    %
    As a genetic algorithm, it aims to balance solution quality and
    diversity and uses the corresponding metaphors of referring to
    solutions as \textit{individuals}, the set of solutions as a
    \textit{population}, and \textit{fitness} to refer to a measure of
    solution's quality.

    The algorithm maintains two separate sets for feasible and infeasible solutions.
    In the beginning, an \textit{initial population} is created using multiple construction heuristics.
    At each iteration, two individuals are chosen based on their fitness and a \textit{crossover operator} is
    applied to them in order to create a new individual.
    This individual is then improved using \textit{local search} operators and added back to the corresponding population.

    The controlled evaluation of infeasible solutions is one of HGS's important ingredients, as optimal solutions are likely
    to exist near the border of feasibility.
    This evaluation is performed through penalizing the violation of constraints.
    The (penalized) cost of a solution is composed of two parts: one for the total cost of the routes,
    the other for infeasibility of the routes.
    This is then combined with a measure of diversity to represent the overall fitness of the solution.

    The overall HGS algorithm is presented in \autoref{alg:hgstw}.
    For the purposes of a compact presentation, many of the details of the algorithm are not mentioned here. For a detailed description we refer to \cite{vidalHybridGeneticAlgorithm2012}.
    The parts relevant to our work are detailed in the next section along with our adaptation to DVRPTW.

    \vspace*{-1.5em}

    %\IncMargin{1em}
    \begin{algorithm}[th]
        initialize population with random solutions improved by local search\;
        \While{number of iterations without improvement and time within limits}{
            select parent solutions $P_1$ and $P_2$\;
            apply all crossover operators on $P_1$ and $P_2$ and choose offspring $C$ with minimum penalized cost\;
            improve offspring $C$ using local search\;
            insert $C$ into the respective subpopulation based on feasiblity\;
            \If{C is infeasible}{
                with $50\%$ probability, repair $C$ (local search) and insert it into respective subpopulation\;
            }
            \If{maximum subpopulation size reached}{
                select survivors\;
            }
            adjust penalty coefficients for infeasibility\;
        }
        \Return best feasible solution\;
        \caption{The HGS Algorithm (adapted from~\cite{vidalHybridGeneticAlgorithm2012}).}\label{alg:hgstw}
    \end{algorithm}%\DecMargin{1em}

    \vspace*{-3em}

    \section{Adapting HGS for the Dynamic VRPTW}\label{sec:adapting}

    In this section we present our \emph{Dynamic HGS} (DHGS) algorithm to be called at each epoch of a DVRPTW instance.
    The starting point of the algorithm is the HGS implementation by Kool et
    al.~\cite{koolHybridGeneticSearch} for VRPTW, which in turn relies
    on the open-source implementation of HGS for capacitated vehicle
    routing by Vidal~\cite{vidalHybridGeneticSearch2022}.
    In the following, we discuss specific parts of HGS along with their adaptation to DHGS.

    \emph{Solution Representation.} In HGS, solutions are represented using the so-called \emph{giant-tour}
    representation.
    Routes in a solution are concatenated without specifying the visits to depots.
    From this representation, optimal positions for the depots can be computed using a linear time
    SPLIT algorithm~\cite{vidalTechnicalNoteSplit2016}.
    While for VRPTW all giant tours must consist of the same number of nodes, a solution for one epoch of DVRPTW can include only a subset of the nodes.
    Hence, for the dynamic setting, we allow giant tours to have variable size.

%    TAKES TOO MUCH SPACE!
%    See an example in \autoref{fig:giant-tour}.

%    % Figure environment removed

    \emph{Initial Population.} At the outset, HGS creates an initial set of solutions
    via multiple construction heuristics.
    Most of the initial solutions are generated using the \textit{random} strategy;
    A random ordering of the customers is generated, then split into routes using~\cite{vidalTechnicalNoteSplit2016}.
    Local search is then applied and the individual is added to the corresponding population.
    In DHGS, we construct the initial population in a similar manner by generating random orderings of the customers.
    Subsequently, we remove a subset of the optional customers, each with fixed probability.
    In our implementation we use probabilities $p \in \{0, 0.1, 0.5, 1 \}$ to create (at most) four solutions from each random ordering.

    \emph{Crossover.} This operation of combining two (parent) solutions into a new (offspring) solution has
    the goal of exploring a new subspace of solutions.
    The new solution should be sufficiently different from the parent solutions while maintaining reasonable quality.
    HGS uses the \emph{Ordered Crossover} (OX)~\cite{oliverStudyPermutationCrossover1987} operator,
    which copies a random contiguous fragment of the giant-tour representation of the first parent solution and
    then completes it
    with the missing nodes in the order in which they appear in the second parent.
    In DHGS we apply the same technique on solutions of variable size.
    Note that this never lead to solutions of smaller size.
    While this could in theory affect diversity negatively, this effect
    is hopefully mitigated by the \emph{delete} local search operator explained below.

    \emph{Fitness.} In HGS, the fitness of a solution (individual) is a score that reflects the cost and (in)feasibility of a solution.
    The difficulty in the dynamic setting is to find a good proxy for the quality of routes on the chosen set of customers \emph{plus}
    the effect of this choice on future epochs.

    First, in order to avoid that solutions with fewer nodes appear to have better cost, we normalize the cost with respect to
    the number of nodes chosen $n$.
%    and the other is removing a constant cost for
%    each customer served (as this would mean less limitation for the future epochs).
    Second, to reflect the effect on future epochs, we add a \emph{lateness} measure $S_{late}$ calculated as the sum of end of the
    time windows for all dispatched customers; this favor customers that are more likely to become must-go
    nodes in the near future.
    Third, we modify the normalization by dividing by a number representing the progress with respect to the
    total number of epochs
    in order to incentivize the algorithm to dispatch more nodes as it approaches the last epoch.
    To summarize we calculate fitness as
    \[
        \frac{p_{dist} \cdot S_{dist} + p_{cap} \cdot S_{cap} + p_{time} \cdot S_{time} + p_{late} \cdot
        S_{late}}{(\frac{t-1}{T} + 1) \cdot (n + 1)},
    \]
    where $S_{dist}$ is the total distance of the routes, $t$ is the index of the current epoch, $T$ is the total number of epochs, $S_{cap}$ and $S_{time}$ are the penalized infeasibility
    measures computed as in HGS for capacity and time window constraints, respectively.
    The non-negative weights.
    In our implementation $p_{late}$ is set to $100$ and $p_{dist}$, $p_{cap}$, and $p_{time}$ are managed by HGS as described
    in~\cite{vidalHybridGeneticAlgorithm2012};
    they are updated to balance the search between increasing feasibility of certain violated constraints
    and improving cost.

    \emph{Local Search.}
    Local search heuristics are an essential ingredient of HGS to improve solutions found from the initial population and crossover.
    They work by applying local improvement operators until no improvement can be achieved.
    These operators range from simple relocate-and-swap steps to more complex modifications.
    HGS includes multiple local search heuristics, that are applied in the order of increasing computational complexity.
    We add three new operators to
    allow for manipulating the set of dispatched nodes in a solution:
    \begin{itemize}
        \item \texttt{delete} removes an optional customer if this improves feasibility.
        \item \texttt{add} inserts a customer if this does not affect the cost significantly.
        \item \texttt{swap-out} exchanges a customer that is part of the solution with one that is not used if it
        improves the original HGS' score $S_{dist} + S_{cap} + S_{time}$.
    \end{itemize}
    We run the new heuristics first to modify the set of dispatched customers; subsequently, we improve the routes using the
    original HGS heuristics.


    % ---------–––––––––––––––––––––––---------–––––––––––––––––––––––---------–––––––––––––––––––––––

    \vspace*{-0.5em}
    \section{Computational Results}\label{sec:comp}

    Our DHGS implementation is based on the HGS solver for VRPTW provided for the
    EURO meets NeurIPS Vehicle Routing Competition~\cite{euromeetsneurips2022}.
    We evaluate the algorithm on the full set of 250~instances released in the competition against the following
    baselines provided by the competition organizers:
    \begin{itemize}
        \item \texttt{greedy} dispatches all customers in each epoch.
        \item \texttt{lazy} dispatches only the must-go customers.
        \item \texttt{random} dispatches the set of must-go customers in addition to some optional customers
        chosen at random.
        \item \texttt{oracle} has access to all future information and solves the problem in hindsight as one large VRPTW\@.
        The oracle solution is not guaranteed to be a global optimum since it uses the HGS heuristic; still, it can be viewed as an approximate lower bound for the best solution quality that can be achieved.
        \item \texttt{supervised} is an imitation learning approach based on graph neural networks trained on the oracle
        solver's decisions.
        \item \texttt{dqn} is a deep-Q neural network trained using reinforcement learning.
        % dynamo can be called DHGS and is not a baseline, so not in this list 
    \end{itemize}
    These baselines all use HGS as a subroutine and differ only in the strategy to choose the subset of customers to
    dispatch in each epoch.

    Experiments were conducted on a cluster of identical machines equipped with Intel(R) Xeon(R) Gold 5122
    processors with 3.6GHz and 96GB of RAM; each chip has 4~cores, but experiments were run in single-threaded mode.
    The time limit for each solver is $5$ minutes per epoch.
    The \texttt{oracle} strategy runs the \texttt{greedy} strategy first to
    get a good initial solution then is given $10$ minutes to run HGS on all nodes from all epochs.
    All solvers are initialized with the trivially feasible singleton solution, i.e., a collection of routes
    visiting only a single node and returning to the depot.

    The results are summarized in \autoref{fig:box}.
    On average, DHGS improves the cost objective by 7.82\% when the
    median cost is compared to that of the top
    performing baselines \texttt{greedy} and \texttt{dqn}.
    The gap to the \texttt{oracle} strategy is 10.37\%.
    In addition, the raw performance data also shows that the size of the instance does not affect the relative ranking of the methods.

    While DHGS outperforms all other strategies on average and in most cases, its performance shows higher variance.
    On $7$ out of the $250$ instances, DHGS (and only DHGS) achieves better quality than the oracle
    baseline.
    However, on $19$ instances DHGS is also worse than \texttt{greedy} and
    \texttt{dqn}, and on $3$ instances it has the highest cost of all strategies.
    In these $3$ cases, there is an epoch where DHGS gets stuck at the trivial initial solution, which uses a separate
    vehicle for each customer.

    \vspace*{-2em}
    % Figure environment removed


%   REMOVED BECAUSE DOESN'T GIVE MUCH INSIGHT
%    \begin{table}[h]
%        \centering
%        \section{Experimental Results}\label{sec:results}
    \subsection{General Results}
        The basic ResSAN model is used to determine reference results which our expanded model can be compared to as it is structurally similar to ResLAN but does not possess the Lidar adaptive components of it. Further, we compare with the full-size PackNet-SAN and the unmodified NLSPN architecture. 
        As it can be seen from Tab.\,\ref{tab:sota-results}, our LiDAR-adaptive ResLAN achieves competitive performance compared to state-of-the-art standard depth completion methods, which are specialized to the unfiltered 64-beam-LiDAR. The performance differences are in the range of a few centimetres in terms of MAE, which is acceptable given the practical advantage that ResLAN can generalize to different beam patterns as will be shown below.

        Furthermore, we compared the architectures for a set of three different input types that contained 64, 32 or 16 LiDAR channels using both filter types on the metrics from the KITTI benchmark. The NLSPN model was trained for the standard depth completion task and then evaluated with different input data. As for the ResSAN models, we trained one model for each input type and tested it for the corresponding one which serve serve as the \emph{Baseline} in Tab.\,\ref{tab:overall-results}. Our ResLAN model was jointly trained for all three settings. As listed in Tab.\,\ref{tab:overall-results}, the ResLAN models outperform the challenging baseline in all metrics for FOV filtering and all but one for sparse filtering. This implies that our LiDAR adaptive model is able to outperform dedicated models in case of very sparse input depth. Fig.\,\ref{fig:comp-plot} shows this is indeed the case for 32 and even more for 16 channels. For FOV-filtered inputs with 16 channels, the ResLAN exhibits approx. $10\%$ smaller MAE than the baseline. As for the NLSPN, it becomes apparent that it is not capable of generalizing to other input types since it shows clearly worse results. The difference is especially pronounced for the FOV filtering where on average more than every fourth predicted pixel is more than $25 \%$ deviating from the ground truth\,($\delta_{1.25}$). Therefore, using a weight-adapting network in combination with differently filtered input depths allows us to train models that outperform their non-adaptive counterparts.

        \begin{table}[]
            \centering
    	    \small
            \vspace{0.4cm}
            \caption{\textbf{Depth estimation result for standard depth completion} when the ResSAN model was only trained for 64 channels and the ResLAN model for multiple tasks. The PackNet-SAN and NLSPN models were trained with the setup that was also used for our model architecture.}
            \footnotesize
            \setlength{\tabcolsep}{5pt}
            \begin{tabular}{@{}lrrrrl@{}}
            \toprule
            \multicolumn{6}{c}{\textbf{Standard LiDAR Depth Completion}}                                                                                                                         \\ \midrule
            \multicolumn{1}{l|}{Method}          & RMSE $\downarrow$            & MAE  $\downarrow$            & iRMSE $\downarrow$             & iMAE $\downarrow$ & $\delta_{1.25}$ $\uparrow$ \\
            \multicolumn{1}{l|}{}                & \multicolumn{1}{l}{{[}mm{]}} & \multicolumn{1}{l}{{[}mm{]}} & \multicolumn{1}{l}{{[}1/km{]}} & {[}1/km{]}        &                            \\ \midrule
            \multicolumn{1}{l|}{PackNet-SAN}     &  914                            &  298                            &  2.78                              &  1.4                 &  99.65 \%                          \\
            \multicolumn{1}{l|}{NLSPN}           &  \textbf{889}                            &   \textbf{263}                           &  \textbf{2.62}                              &   \textbf{1.3}                &   \textbf{99.61} \%                         \\ \midrule
            \multicolumn{1}{l|}{ResSAN (Ours)}   & 948                             &  275                            &  2.75                              &    1.4               &   99.58 \%                         \\
            \multicolumn{1}{l|}{ResLAN (Ours)} &   969                           &  283                            &   2.83                             &   1.4                &  99.56 \%                          \\ \bottomrule
            \end{tabular}
            \vspace{0.2cm}
            \label{tab:sota-results}
        \end{table}

        \begin{table}[]
    	    \centering
    	    \small
    	    \caption{\textbf{Depth estimation results of the two baseline setups and the explicit and implicit ResSAN} when evaluated on a combination of 16, 32 and 64 channel depth inputs. Please note that Specialist Methods need to train three specialized networks, one for each of the three types of inputs while our method only uses one network.}
            \footnotesize
            \setlength{\tabcolsep}{4.8pt}
            \begin{tabular}{@{}lrrrrl@{}}
                \toprule
                \multicolumn{6}{c}{\textbf{Sparse Channel Filter}}                                                                                                                                  \\ \midrule
                \multicolumn{1}{l|}{Method}        & RMSE $\downarrow$            & MAE  $\downarrow$            & iRMSE $\downarrow$             & iMAE $\downarrow$ & $\delta_{1.25}$ $\uparrow$  \\
                \multicolumn{1}{l|}{}              & \multicolumn{1}{l}{{[}mm{]}} & \multicolumn{1}{l}{{[}mm{]}} & \multicolumn{1}{l}{{[}1/km{]}} & {[}1/km{]}        &                             \\ \midrule
                \multicolumn{1}{l|}{NLSPN}         &  1396                            &  437                            & 5.54                               &  2.2                 &  98.82 \%                           \\
                \multicolumn{1}{l|}{Baseline}      & \textbf{1207}                             &  381                            & 4.41                               &  1.8                 &  \textbf{99.37} \%                           \\
                \multicolumn{1}{l|}{ResLAN (Ours)} &  1215                            &  \textbf{378}                            &  \textbf{4.27}                              &  \textbf{1.7}                 &  99.31 \%                           \\ \toprule
                \multicolumn{6}{c}{\textbf{Field-of-View Filter}}                                                                                                                                   \\ \midrule
                \multicolumn{1}{l|}{Method}        & RMSE $\downarrow$            & MAE  $\downarrow$            & iRMSE $\downarrow$             & iMAE $\downarrow$ & $\delta_{1.25}$ $\uparrow$ \\
                \multicolumn{1}{l|}{}              & \multicolumn{1}{l}{{[}mm{]}} & \multicolumn{1}{l}{{[}mm{]}} & \multicolumn{1}{l}{{[}1/km{]}} & {[}1/km{]}        &                             \\ \midrule
                \multicolumn{1}{l|}{NLSPN}         &  2738                            &  1702                            & 12.3                              &  4.3                 &  74.69 \%                           \\
                \multicolumn{1}{l|}{Baseline}      &  1556                            &  525                            &  6.8                              &  3.0                 & 98.14 \%                            \\
                \multicolumn{1}{l|}{ResLAN (Ours)} &  \textbf{1548}                            &  \textbf{519}                            &  \textbf{6.44}                              &  \textbf{2.8}                 & \textbf{98.52 \%}                            \\ \bottomrule
            \end{tabular}
            \label{tab:overall-results}
        \end{table}

        
        
        % Figure environment removed
        
        % Figure environment removed

    \subsection{Filter Effects}
        Comparing the effect of the two different types of depth input filters on the model performance, it becomes apparent that FOV filtering is the more challenging task. In that setting, reducing LiDAR channels is more detrimental to the performance than sparse filtering as it creates regions where no depth information is available. Effectively, the model is forced to perform depth prediction in these regions. These effects are highlighted in the depth images in Fig.\,\ref{fig:dense-maps} where the effect of a 16-channel sparse depth filter and a 16-channel FOV can be compared.

    \subsection{Generalization Capabilities}
        We trained three models for both filter types eaach, so the combinations and number of filtered depth inputs they receive are different. This serves the purpose of testing the generalization capabilities of the ResLAN architecture as well as the robustness to different filter settings. After training, the models were evaluated for the depth input settings they were trained for, as well as for ones they weren't exposed to. Overall, ResLAN shows good generalization capabilities. As one can gather from Fig.\,\ref{fig:explicit-comp} and Fig.\,\ref{fig:implicit-comp}, the consequences of slightly varying sets of input depth settings are limited. The most considerable deviations can be seen when the model is tasked to extrapolate. For instance, the model $\{64, 32, 16\}$ shows a noticeably higher MAE for eight-channel depth inputs than the model that was trained for it. Similar behaviour can be seen for the FOV filtering case as well for the model $\{64, 48, 32\}$ when tasked to generalize for a 16-channel input. There is no such pronounced effect for generalization tasks that lie between two filter settings the model was trained for. At most, it can be observed that models that were trained for a smaller range of filter values perform slightly better than ones that have to cover a wider range. The number of filter settings used in a fixed range does not relevantly influence the model performance, as can be seen, when comparing the two models in Fig.\,\ref{fig:implicit-comp}, which are both trained for a range of 64 to 32 channels but one with three filter settings and the other one with five.
    
    % Figure environment removed
    
    
    % Figure environment removed
%        \caption{Example Table}
%        \label{tab:example}
%    \end{table}
%


    % ---------–––––––––––––––––––––––---------–––––––––––––––––––––––---------–––––––––––––––––––––––

    \vspace*{-1.5em}

    To conclude, the results demonstrate that the adapted DHGS algorithm can perform favorably against different baselines, providing an effective
    general method for DVRPTW that does not require any pre-training.
    However, the dependency of DVRPTW on online data also makes it a natural test case for the application of machine learning.
    The results show that this is not straightforward.
    The \texttt{supervised} baseline, although trained on the \texttt{oracle}
    results, performs worse than the trivial \texttt{greedy} strategy.
    The reinforcement learning baseline \texttt{dqn} effectively learned the \texttt{greedy} strategy.

    However, many of the top performing solvers in the competition were based on machine learning
    and demonstrate that more sophisticated approaches can successfully harness the potential of machine learning~\cite{kleopatra,team_sb,milestogobeforewesleep}.
    Some follow the simple scheme of applying HGS as a subroutine after performing ML-based customer selection.
    Hence, one interesting question for future research is if combining these ideas with DHGS can
    improve performance and robustness further.

%%     \section{Conclusion}\label{sec:conc}

%%     We showed that the adapted DHGS algorithm can perform favorably against different baselines, providing an effective
%%     general method for DVRPTW that does not require any pre-training.
%%     However, the dependency of DVRPTW on online data certainly makes it attractive for the application of advanced techniques from machine learning.
%%     Many of the top performing solvers in the competition were based on machine learning~\cite{kleopatra,team_sb,milestogobeforewesleep},
%%     some simply applying ``static'' HGS as a subroutine after performing ML-based customer selection.
%%     Hence, one interesting question for future research is whether a combination of these ideas with DHGS can be used to
%%     improve performance and robustness further.
%%     However, although machine learning algorithms have been shown to be useful for some routing problems, they tend to
%%     require prior training and can be prone to \textit{overfitting}.
%%     \draft{The recent study~\cite{santanaNeuralNetworksLocal2023} gives an example where the resulting improvement does
%%     not justify the added complexity.}
%%     Furthermore, the intricate nature of these algorithms can hinder our ability to understand and reason about their performance, and they can generally be more challenging to fine-tune.
%%     In this work, we presented a simple general algorithm for the DVRPTW without the requirement of prior training.


    % ---------–––––––––––––––––––––––---------–––––––––––––––––––––––---------–––––––––––––––––––––––


    \bibliographystyle{splncs04}
    \bibliography{main}

\end{document}