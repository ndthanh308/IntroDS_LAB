% \section{Introduction}
% Figure environment removed

Authors often pair charts and caption text together to convey information about data (e.g., in news articles, academic papers, and reports). For example, in Figure~\ref{fig:teaser}, the peak around 1981 is visually prominent. The caption text also emphasizes portions of the data by referring to specific points or ranges of data values. The caption text \textit{`peak in 1981'} emphasizes the visually prominent peak. When the chart and caption text emphasize the same aspects of the data, as in this example, people tend to remember those aspects of the data as takeaways~\cite{cheng2022captions,kim2021towards,xiong2019curse}; when they emphasize different aspects of the data (e.g., the caption text emphasizes the \textit{`dip between 2008 and 2012'}, but the fall is not visually prominent in the chart), people remember the visually prominent data features in the chart, not the data emphasized in the text~\cite{cheng2022captions,kim2021towards}
and question the credibility of the information in the chart and text~\cite{kong2019trust}.

But how often do authors really create charts and caption text that emphasize the same aspects of data? Analyzing time-series charts and their captions in real-world publications (Section~\ref{sec:survey}), we find that professional authors match chart and caption emphasis about 65\% of the time and mismatches in the remaining 35\%.
In a survey of chart-caption pairs on Tableau Public~\cite{tableau2022public}, a community for the general public to share visualizations created using Tableau Software~\cite{tableau2022tableau}, we find that mismatches are even more common among the general public.

To help authors convey their messages effectively, we present \toolname{}, a caption-and-chart checker tool that takes time-series charts
as input. The tool highlights the visually emphasized chart features and the data features emphasized in the caption. The user can then decide how to update the caption or the chart to better align their emphasis.
Our tool follows the model of spell- and grammar-checkers~\cite{google2022check,grammarly2022grammarly,language2022language}, facilitating the process of locating potential mismatches between chart and caption text,
while leaving the final decision of how to resolve the issue to the author. 
In this work, we focus on time-series line charts as a first step towards more general tools as they are among the most common type of charts on the Web~\cite{battle2018beagle}.

Our tool includes a \textit{time-series prominent feature detector} that identifies the visually prominent features of the chart. It also includes a \textit{text reference extractor} that identifies data emphasized in the text. The \toolname{}  interface is designed so that users can then compare the chart-emphasized data features with the text-emphasized data features (Figure~\ref{fig:teaser}).
For the prominent feature detector, we introduce a new $\varepsilon$-persistence technique based on the Ramer-Douglas-Peucker line simplification algorithm for approximating visual prominence. 
The chart-text reference extractor uses heuristics based on example analysis, 
modules within the Stanford CoreNLP toolkit~\cite{chang2012sutime,chen2014fast,finkel2005incorporating,manning2014stanford}, and BERT embeddings~\cite{devlin-etal-2019-bert}
to find time references and data descriptions in the caption text and matches them with chart data.
\toolname~visualizes the results of these components to the user as a part of a chart-caption authoring interface shown in Figure~\ref{fig:teaser}.

We evaluated the two components using the charts from Kim et al.~\cite{kim2021towards}.
We find that our time-series prominent feature detector outperforms a state-of-the-art method for prominent feature detection~\cite{hullman2013contextifier} while performing at the level of crowdsourced prominent features~\cite{kim2021towards}.
For human-written captions on the real-world charts in the corpus, we find that our text reference extractor correctly identifies text references to charts for 63.41\% of the sentences.
Finally, through a user study, we find that \toolname{} is both useful and easy to use when authoring charts and captions.

In summary, the contributions of this work include: 
\begin{itemize}[nolistsep]
    \item a survey of what charts and their captions in the real-world emphasize;
    \item \toolname{}, a caption-and-chart checker tool that guides authors to create charts and captions whose emphasis match chart emphasis; and
    \item algorithms for detecting visually prominent features in time-series line charts and text references to chart features. 
\end{itemize}