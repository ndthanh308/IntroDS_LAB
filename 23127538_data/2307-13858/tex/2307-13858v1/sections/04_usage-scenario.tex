\section{Usage Scenario}
\label{sec:usage-scenario}

% Figure environment removed

We first describe a usage scenario of \toolname{} to motivate the design of the tool and illustrate how it guides the user when writing a text caption for a chart.
Tess, a policymaker, wants to add a chart showing the real home price index data to a presentation she plans to give to her fellow policymakers. As a part of her presentation, she would like to make a case for building more homes based on past data. To ensure that she effectively gets her message across, she decides to use the \toolname{} tool.

\vspace{0.05in}
\noindent\textit{\textbf{Viewing visually prominent features in the chart.}} (Figure~\ref{fig:user-interface}a above textbox) Tess starts by loading the data into \toolname{}. The time-series chart shows the real home price index over time. The tool displays the visually prominent features in the chart as orange circles ({\color[RGB]{223,123,51}$\CIRCLE$}) and bars ({\scalebox{1.5}[1]{\color[RGB]{223,123,51}$\blacksquare$}}) just above the chart. 
\toolname{} show  the recent low point in 1997 and the rising segment afterward as the top two most prominent features (darker shade of orange {\color[RGB]{223,123,51}$\blacksquare$}), followed by the global minimum point around 1920 and two other less prominent features (lighter shades of orange {\color[RGB]{238,189,141}$\blacksquare$}).

\vspace{0.05in}
\noindent\textit{\textbf{Typing a basic caption.}} (Figure~\ref{fig:user-interface}a) Seeing the time range of the chart, she types \textit{``The chart shows the real home price index between 1890 and 2006.''} in the text box below the chart and hits the \textsc{[Shift-Enter]} command to run text analysis and assess text emphasis with respect to the chart.  
After the code runs to completion and the textbox is re-enabled, she sees that there is no change in the region above the chart.
Based on this information, she understands that the text she typed in has no reference to any specific features in the chart.

\vspace{0.05in}
\noindent\textit{\textbf{Typing caption text that matches the most prominent visual feature.}} (Figure~\ref{fig:user-interface}b) Tess then returns to thinking about the most prominent feature. 
She hovers over the orange circle ({\color[RGB]{223,123,51}$\CIRCLE$}) closest to the bottom and the orange bar ({\scalebox{1.5}[1]{\color[RGB]{223,123,51}$\blacksquare$}}) right above it to see where they lie on the chart.
She observes the spike after 1997 in the context of the chart and convinces herself that the spike is not only visually prominent but also that she can point to this spike as proof of the dire situation of the housing market.
She decides to cover the feature in her caption and types
\textit{``The housing prices have skyrocketed starting around 1997 and we need to act.''} After hitting \textsc{[Shift-Enter]} to run analysis on the text, the interface highlights the phrase \textit{`skyrocketed starting around 1997'} in blue {\color[RGB]{153,185,211}$\blacksquare$} and displays a blue circle ({\color[RGB]{153,185,211}$\CIRCLE$}) on the year 1997, the endpoint explicitly mentioned in the text, and a bar ({\scalebox{1.5}[1]{\color[RGB]{153,185,211}$\blacksquare$}}) starting in the year 1997 to show the reference between the chart and caption.
The interface also highlights the top two prominent chart features in green {\color[RGB]{97,158,58}$\blacksquare$} to indicate that Tess has matched the emphasis in the caption she has written so far.

\vspace{0.05in}
\noindent\textit{\textbf{Typing caption text with an error.}} (Figure~\ref{fig:user-interface}c)
Tess now looks for falling trends in the chart and discovers one between 1894 and 1921. She looks up the reason for the fall and thinks that it would support her message well; she types, \textit{``Looking back, they declined since \textbf{1984} with an increased housing supply as manufactured homes became available to the public.''}
Tess hits \textsc{[Shift-Enter]} and this time, she is surprised to see a red squiggly underline 
(% Figure removed)
under the phrase \textit{`declined since 1984'} in her caption. When she hovers over the text \textit{`declined since 1984'}, she sees that the time segment that she is referring to (red bar ({\scalebox{1.5}[1]{\color[RGB]{207,145,147}$\blacksquare$}}) above the chart in Figure~\ref{fig:user-interface}c) is not the one she intended and soon realizes that she mistyped \textit{`1984'} instead of \textit{`1894'}. 

% Figure environment removed

\vspace{0.05in}
\noindent\textit{\textbf{Pushing through with caption text about a less prominent feature.
}} (Figure~\ref{fig:user-interface}d)
Tess fixes the typo and completes her caption by adding the sentence \textit{``A similar supply-side solution is what we need.''} 
She then confirms the change by pressing \textsc{[Shift-Enter]}. 
She finds that the detected time range has been revised to the one she initially intended, but she realizes that there is a blue squiggly underline (% Figure removed)
under the same phrase \textit{`declined since 1894'}.
She sees that the red bars ({\scalebox{1.5}[1]{\color[RGB]{207,145,147}$\blacksquare$}}) matched to the phrase do not match with any of the top five prominent features but decides that she wants to push forward with her caption.
Before finalizing her caption, she looks over the unmatched prominent features, still shown in orange ({\color[RGB]{223,123,51}$\CIRCLE$}), and thinks about whether there are additional features she should describe in her caption.
She sees that the third most prominent feature corresponding to the global minimum in 1921 is the other end point of the downward trend she just wrote about and decides that she does not need to describe the point explicitly.
Finally, Tess looks over the fourth and the fifth most prominent features but decides that they are both irrelevant to her needs and not very prominent based on the lighter shade of orange {\color[RGB]{238,189,141}$\blacksquare$}.
She concludes the chart-caption authoring process.