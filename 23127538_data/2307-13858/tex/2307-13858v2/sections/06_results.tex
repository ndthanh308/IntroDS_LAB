\section{Results and Evaluation}

% Figure environment removed

Figure~\ref{fig:results} shows results of running the \toolname{} tool on various charts and captions.
The tool's components function accurately and provide correct guidance in Figure~\ref{fig:results}a-c, but err in cases shown in Figure~\ref{fig:results}d.
To understand how well each component of \toolname{} performs, we ran an evaluation on the time-series prominent feature detector and the text reference extractor.

\subsection{Evaluation: Time-Series Prominent Feature Detector} 
To evaluate our time-series prominent feature detector, we used Kim et al.'s~\cite{kim2021towards} corpus of 43 synthetic and real-world examples.
We used their crowdsourced prominent features as a gold feature set to compare against.
As another baseline for comparison, we also generated chart features using a state-of-the-art prominent feature detector, Contextifier~\cite{hullman2013contextifier}, which uses the value and the first derivative at each point to compute the visual saliency of each point. 

The features we generated with \toolname{} matched the gold feature set far better than features generated by Contextifier.
On average, $1.47$ ($49\%$) of \toolname's top three prominent features matched the gold features, whereas only $0.81$ ($27\%$) of Contextifier's top three prominent features matched the gold features. 
We further analyzed the variance among crowd workers who labeled the gold feature set. That is, we compared each crowd worker's labeling of the top-three features against the average crowd workers' labels of these features and found a match of $1.72$ ($57\%$) of the time. This suggests that there is 
high variation among what people think of as the top three visually prominent features. 


We include further analysis of our prominent feature detection algorithm in the supplemental material.

\subsection{Evaluation: Text Reference Extractor}
To understand how reliably our text reference extractor identifies text references to chart features, we ran our text reference extractor on 24 chart-caption pairs with 82 sentences collected through a user study (Section~\ref{sec:user-study}). 
Two of the authors of this paper independently reviewed each of the captions at sentence level along with the participant's stated intended message to determine the references included in each sentence.
The two authors discussed any mismatches and asked a third-party visualization expert for interventions whenever the conflicts could not be resolved between the two authors.

Based on the labels, we ran our text reference extractor on the captions and classified errors in each sentence as one of the following:
\begin{itemize}[nolistsep,leftmargin=*]
    \item \textit{False negative} (FN): the tool failed to extract an existing reference (e.g., Figure~\ref{fig:results}c Sentence 3).
    \item \textit{False positive} (FP): the tool extracted a reference that is non-existent.
    \item \textit{Intention mismatch} (IM): the tool correctly extracts the reference based on what is said in the sentence but detects a feature that is different from the author's intention (e.g., Figure~\ref{fig:results}d Sentence 1; the endpoint of the range is underspecified in the text, but the participant's intended endpoint is clearly 2008, not the end of the chart).
\end{itemize}
A sentence can include several of these errors and is considered correct if it includes no errors.
For three sentences that included apparent typos (one written with the baseline tool (Figure~\ref{fig:results}a), two written with \toolname{} tool), we considered them correct if the text reference extractor correctly identified the mentioned time range.


Of the 81 sentences, the text reference extractor correctly identified the references in 57 sentences (70\%), included FN errors in 22 sentences (27\%), FP errors in 4 sentences (5\%), and IM errors in 2 sentences (2\%). Four of the sentences included multiple errors and were double-counted into multiple categories.
We include further analysis of the error cases in the supplemental material.