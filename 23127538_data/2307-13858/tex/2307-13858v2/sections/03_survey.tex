\section{A Survey of Line Charts in the Wild}
\label{sec:survey}
To understand whether charts and captions emphasize the same information in practice, we conducted a survey of chart-text pairs in various real-world sources.
We specifically looked at charts and captions written by professionals and circulated through various publishing venues, as well as charts and captions published on Tableau Public~\cite{tableau2022public}, a community of professional and non-professional authors for sharing charts and captions created with Tableau Software~\cite{tableau2022tableau}, a tool that suggests basic captions about data fields and chart encoding to its users.

\subsection{Dataset}

For chart-caption pairs written by authors, we sampled a total of 250 chart-article pairs (189 unique articles) from various publishing venues to obtain a representative set.
The venues include news media (New York Times~\cite{nyt2022nyt}, The BBC~\cite{bbc2022bbc}, Vox~\cite{vox2023vox}), poll reports (Pew Research~\cite{pew2022pew}), governmental and intergovernmental organization reports (US Treasury~\cite{us2023treasury}, International Monetary Fund~\cite{imf2023imf}, International Labour Organization~\cite{un2023ilo}, etc.), and academic articles (Nature~\cite{springer2023nature}).
In addition, we sampled 30 chart-article pairs (21 unique charts) from Tableau Public. We programmatically scraped the Web for articles published up to two years from the collection date and filtered for line charts using the chart classification pipeline from Poco and Heer~\cite{poco2017reverse}.
Because Pew Research did not allow scraping without permission, we performed the collection process manually for that site.
We include the links to these articles in the supplemental material.

\subsection{Analysis Method}
We performed analysis on the chart-article pairs in two steps; we identified (1) chart emphasis and then (2) text emphasis.

\vspace{0.05in}
\noindent\textbf{\textit{Step 1: Identify chart emphasis.}}
For each chart image in the dataset, two of the authors of this paper independently labeled the visually prominent features.
Afterward, the authors merged their labels and whenever the labels did not match, the two authors shared their reasoning for their annotations with one another to arrive at a consensus. If no consensus was reached, the authors used the opinion of a third individual who works in visualization research but is not a co-author to determine the prominent features of the chart. At this stage, the authors did not read any of the text of the article to avoid biasing their perception of visually prominent features~\cite{xiong2019curse}.

\vspace{0.05in}
\noindent\textbf{\textit{Step 2: Identify text emphasis.}}
The two authors read through the article text to identify all paragraphs that mention any information about the chart by looking for mentions of the phrases in the chart title, axis names, or chart numbering (e.g., `\textit{Figure X},' `\textit{Chart X}').
They each independently perused the sentences in each of the paragraphs, chart titles, and textual annotations within the charts to identify any references to prominent features and non-prominent features.
The authors again discussed the annotations to arrive at a consensus, using the third individual for arbitration when necessary to resolve conflicts. 


\subsection{Results}
% Figure environment removed

Figure~\ref{fig:survey-results} shows the results of our analysis.
We observed a visible dividing line between chart captions written by professional authors circulated through publishing organizations and chart-caption pairs on Tableau Public.

\vspace{0.05in}
\noindent\textbf{\textit{Professionals often match emphasis but occasionally do not.}}
We found that 65\% chart-caption pairs made by professional authors match emphases (blue {\color[RGB]{140,158,212}$\blacksquare$} segment in the top bar of Figure~\ref{fig:survey-results}); the chart emphasizes the author's message in the text, and the text explains the visually prominent features in the chart.
Yet, emphasis mismatches occur regularly; in 35\% of the chart-caption pairs, the chart emphasizes a feature in the data different from the text, and the text either describes features that are not visually prominent (26\%; first orange {\color[RGB]{224,164,122}$\blacksquare$} bar in Figure~\ref{fig:survey-results}) or only describes how the chart encodes information (9\%; first gray {\color[RGB]{192,192,192}$\blacksquare$} bar in Figure~\ref{fig:survey-results}).
Such mismatches suggest that there is some room for improvement, even in professionally authored documents.

\vspace{0.05in}
\noindent\textbf{\textit{Tableau Public authors often write basic captions.}}
In Tableau Public, the overwhelming majority (93\%) of the text descriptions for charts are basic captions that do not describe the specific features in the charts (gray {\color[RGB]{192,192,192}$\blacksquare$} segment in the bottom  bar of Figure~\ref{fig:survey-results}). 
We hypothesize that this result is due in part because the Tableau Public authoring software defaults to providing  basic captions. Authors do not have to put in the extra effort required to discuss the features visible in the charts. 
Unfortunately, prior work has found that readers find such basic captions of little use~\cite{lundgard2021accessible} and that the basic captions play no role in helping readers understand what they should take away~\cite{kim2021towards}.
Hence, the high ratio of basic captions indicates that these authors could benefit from further support and guidelines for authoring charts and text.