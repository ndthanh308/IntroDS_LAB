\documentclass[prl,twocolumn,amsmath,amssymb,superscriptaddress]{revtex4-2}

\usepackage{bm}                     
\usepackage{appendix}
\usepackage[latin1]{inputenc}
\usepackage{dcolumn}      
\usepackage{bbm}
\usepackage{color}
\usepackage{xcolor}

\usepackage[mathscr]{eucal}
\usepackage{latexsym}
\usepackage{amsbsy}
\usepackage{float}
\usepackage{graphicx}
\usepackage{epsfig}
\usepackage{subfigure}
\usepackage{wrapfig}
\usepackage{epsf}
\usepackage{float}
\usepackage[normalem]{ulem}
\usepackage{qcircuit}

\definecolor{darkblue}{HTML}{004D6B}
\definecolor{darkred}{HTML}{8c1515}
\definecolor{darkgreen}{HTML}{006400}
\usepackage{hyperref}
\hypersetup{ colorlinks=true, urlcolor=darkblue, citecolor=darkred,
    linkcolor=darkblue, breaklinks }

\newcommand{\ba}{\begin{array}}
\newcommand{\ea}{\end{array}}
\newcommand{\be}{\begin{equation}}
\newcommand{\ee}{\end{equation}}
\newcommand{\bea}{\begin{eqnarray}}
\newcommand{\eea}{\end{eqnarray}}


%\newcommand{\new}[1]{\textcolor{blue}{#1}}
\begin{document}

\title{Approaching ideal rectification in superconducting diodes through multiple Andreev reflections}

\author{A. Zazunov}
\affiliation{Institut f\"ur Theoretische Physik, Heinrich-Heine-Universit\"at, D-40225  D\"usseldorf, Germany}
\author{J. Rech}
\affiliation{Aix Marseille Univ., Universit\'e de Toulon, CNRS, CPT, Marseille, France}
\author{T. Jonckheere}
\affiliation{Aix Marseille Univ., Universit\'e de Toulon, CNRS, CPT, Marseille, France}
\author{B. Gr{\'e}maud}
\affiliation{Aix Marseille Univ., Universit\'e de Toulon, CNRS, CPT, Marseille, France}
\author{T. Martin}
\affiliation{Aix Marseille Univ., Universit\'e de Toulon, CNRS, CPT, Marseille, France}
\author{R. Egger}
\affiliation{Institut f\"ur Theoretische Physik, Heinrich-Heine-Universit\"at, D-40225  D\"usseldorf, Germany}

\date{\today}

\begin{abstract}
We analyze the rectification properties of voltage-biased Josephson junctions exhibiting the superconducting diode effect. Taking into account multiple Andreev reflection (MAR) processes in our scattering theory, we consider a
short weak link of arbitrary transparency between two helical superconductors with finite
Cooper pair momentum $2q$.  In equilibrium, the diode efficiency is bounded from above in this 
model, with maximal efficiency $\eta_0\approx 0.4$. Out of equilibrium, we find a rich subharmonic structure in the current-voltage curve.  For high transparency and 
low bias voltage $V$, the rectification efficiency $\eta(V)$ approaches the ideal value $\eta=1$ for
$q\xi\to 1$ (with coherence length $\xi$). 
\end{abstract}
\maketitle

\emph{Introduction.---}Starting with the work by Ando \emph{et al.}~in 2020 \cite{Ando2020}, 
a surge of experiments reported evidence for the superconducting diode effect (SDE)  \cite{Lyu2021,Bauriedl2022,Baumgartner2022,Pal2022,Lin2022,Wu2022a,Jeon2022,Turini2022,Sundaresh2023,Mazur2023,Anwar2023,Banerjee2023,Ghosh2023,Hou2023a,Costa2023}, 
see Ref.~\cite{Nadeem2023} for a review.  Even though microscopic mechanisms behind the SDE are not yet 
fully understood, many aspects have been clarified by theoretical works   \cite{Edelstein1996,Hu2007,Reynoso2008,Zazunov2009,Misaki2021,Ilic2022,He2022,Zhang2022,Davydova2022,Kokkeler2022,Daido2022a,Daido2022b,Tanaka2022,Zinkl2022,Cheng2023,Ikeda2023,He2023a,Lu2023,Fu2023,Wang2023a,Yuan2023b,Legg2023,Nakamura2023,Picoli2023}, and the hope is that practically useful device applications will emerge soon. In essence, the SDE amounts to an asymmetry between the (absolute value of the) 
critical supercurrent flowing to the right ($I_{c+}>0$) and to the left ($I_{c-}<0$).
Assuming $|I_{c-}|<I_{c+}$, the
SDE efficiency is defined by $\eta_0 =(I_{c+}+I_{c-})/(I_{c+}-I_{c-})$,
where a dissipationless supercurrent $I$ can only flow to the right if $|I_{c-}|<|I|<I_{c+}$.  
We consider the intrinsic SDE in a single Josephson junction (the SDE is also possible in junction-free bulk 
superconductors, see, e.g., Refs.~\cite{Ando2020,Bauriedl2022,Nadeem2023}, and
in more complex multi-junction devices \cite{Souto2022,Fominov2022,Paolucci2023,Gupta2023,Ciaccia2023,Zhang2023expA}),
where the conditions for the anomalous Josephson effect 
\cite{Buzdin2008,Reynoso2008,Zazunov2009,Reynoso2012,Brunetti2013,Dolcini2015,Szombati2016,Qin2017}
have to be met. In particular, time-reversal and inversion symmetries must be broken.  In addition, 
the equilibrium current-phase relation must contain contributions from higher harmonics \cite{Reynoso2008,Zazunov2009,Bauriedl2022,Baumgartner2022}.

We here study the out-of-equilibrium behavior of
Josephson junctions exhibiting the SDE in equilibrium, 
in particular,  the DC current-voltage ($I$-$V$) curve of a voltage-biased
intrinsic Josephson diode.  At low temperatures,  
MAR processes \cite{Klapwijk1982,Bratus1995,Averin1995,Cuevas1996} 
can then provide the dominant transport mechanism, especially for subgap voltages $e|V|<2\Delta$
(with pairing gap $\Delta$).  We focus on junctions with a single (or a few uncoupled) 
channels, where the impedance is of order $h/e^2$ and thus much larger than the typical
impedance of the external circuit. We then do not have to account for
the self-consistent dynamics of the phase difference and voltage across the junction. 
Previous studies of nonequilibrium transport in Josephson diodes have considered weakly damped 
low-impedance junctions \cite{Misaki2021,Fominov2022,Trahms2023,Steiner2023} 
or externally driven junctions \cite{Paaske2023}, but MAR effects have not been addressed.
Our theory predicts a characteristic voltage-dependent 
rectification pattern, $I(-V)\ne -I(V)$, quantified by the efficiency parameter
\begin{equation}\label{efficiency}
    \eta(V) = \frac{I(V)+I(-V)}{I(V)-I(-V)},
\end{equation}
where $\eta(V)$ is especially large in the subgap regime. 
For the conventional case without SDE, MAR causes a subharmonic structure,
i.e., singular features in the nonlinear conductance for $eV=2\Delta/n$ with integer $n$ \cite{Klapwijk1982,Bratus1995,Averin1995}.   
For Josephson diodes, we predict an even richer subharmonic structure which determines
the rectification characteristics and might provide precious information about 
the microscopic mechanisms generating the SDE. 
Our central finding is that the efficiency $\eta(V)$ can approach the ideal limit of full rectification with $\eta=1$
at low voltages, even though $\eta_0\alt 0.4$ for the SDE efficiency in equilibrium for the model 
considered below.  Significant rectification efficiency persists also away from the ideal conditions discussed below, including the case of voltages well above the pairing gap.
 
It is well known that the SDE can arise from magnetochiral effects \cite{Rikken2001,Tokura2018,Morimoto2018,Legg2022} 
in noncentrosymmetric superconductors \cite{Edelstein1996}. 
An alternative mechanism arises from the finite Cooper pair
 momentum $2q$ in a helical superconductor \cite{Davydova2022,Yuan2022,Pal2022,Lin2022,Yuan2022,Banerjee2023}.
We here consider a weak link connecting two helical superconductors with identical
$q$, where the current-phase relation computed in Ref.~\cite{Davydova2022} implies the SDE. 
The simplicity of the corresponding model allows us to determine the full $I$-$V$ curve without approximations from  scattering theory, where
known results \cite{Bratus1995,Averin1995,Zazunov2006} are recovered for $q=0$.    
A detailed account  is given in Ref.~\cite{PRB}, where we also address the SDE efficiency $\eta_0$ in equilibrium and the $I$-$V$ curve for an NS
contact between a normal metal and a helical superconductor.  
In this Letter, we summarize the salient features of the theory and discuss the rectification efficiency $\eta(V)$ of the Josephson diode model with $q\ne 0$.
 
\emph{Model.---}For finite Cooper pair momentum $2q$,
the order parameter of an $s$-wave BCS superconductor oscillates in space, 
$\Delta(x)=\Delta e^{2iqx}$,
where $q\ne 0$ may originate from the interplay of the spin-orbit interaction with
a Zeeman field in superconducting films \cite{Daido2022a,He2022,Yuan2022,Levichev2023},
or from magnetic proximity and/or Meissner effects \cite{Davydova2022}. 
In either case, time-reversal and inversion symmetries are broken for $q\ne 0$.  
Following Ref.~\cite{Davydova2022}, we study a short single-channel weak link between two superconducting banks with the 
same pairing gap and Cooper pair momentum. The coherence length is $\xi=\hbar v_F/\Delta$ with Fermi velocity $v_F$.
For definiteness, we assume $0 \le q \xi < 1$ since the superconductor becomes 
gapless for $q\xi \ge 1$.  

Linearizing the band dispersion around the Fermi momentum points $\pm k_F$ with $k_F\xi\gg 1$, 
the Hamiltonian is expressed in terms of effectively one-dimensional quasiclassical Nambu spinor envelopes,
$\psi_{\pm}(x,t)=(\psi^{}_{\pm,\uparrow},\psi^\dagger_{\pm,\downarrow})^T$,
for right- and left-movers having momenta $\pm k_F+k$ with $|k|\ll k_F$,
resp., where $x<0$ ($x>0$) refers to the left (right) superconducting bank.
Gauging away the $e^{2iqx}$ factor from the order parameter, 
the Bogoliubov-de~Gennes (BdG) Hamiltonian for $x\ne 0$ follows as (we often put $\hbar= 1$) \cite{Davydova2022}
\begin{equation}\label{BdG}
H_{\rm BdG}  =  -iv_F \sigma_z\tau_z \partial_x + v_F q \sigma_z\tau_0 + \Delta\sigma_0\tau_x,
\end{equation} 
where we use Pauli matrices $\tau_{x,y,z}$ (and identity $\tau_0$) in Nambu space
and $\sigma_{x,y,z,0}$ in chiral (right-left mover) space.
Defining the bispinor $\Psi(x,t)=(\psi_+,\psi_-)^T$ in chiral space,
modeling the weak link as normal-conducting constriction with length much shorter than $\xi$ and 
transmission probability ${\cal T}$, and using the phase difference $\varphi(t)=2eVt$ 
across the junction, we arrive at a matching condition 
connecting the bispinors on the left ($x=0^-$) and right ($x=0^+$) side, see also Refs.~\cite{Zazunov2005,Nazarov2009,Zazunov2014,Ackermann2023},
\begin{equation}\label{BC}
\Psi(0^-,t) = \frac{1}{\sqrt{\cal T}} (\sigma_0+r\sigma_x) \,  e^{i\tau_z eV t} \, \Psi(0^+,t),
\end{equation}
with the reflection amplitude $r=\sqrt{1-{\cal T}}$.  

% Figure environment removed

\emph{Spectral properties.---}Consider first a ballistic junction (${\cal T}=1$) at zero voltage, where Eq.~\eqref{BC} is automatically fulfilled by continuous wave functions
and one recovers the ``bulk'' case. BdG eigenstates have conserved energy $E$ and chirality
$\alpha=\sigma_z=\pm$, where Eq.~\eqref{BdG} gives a quasiparticle dispersion with four branches in total, see Fig.~\ref{fig1},
\begin{equation}\label{bulkenergy}
  E(k) = \pm \sqrt{(v_F k)^2+ \Delta^2}  + \alpha v_F q.
\end{equation} 
From Eq.~\eqref{bulkenergy}, there are \emph{two} positive threshold energies for the quasiparticle continuum, $\Delta_\pm=\Delta\pm v_Fq$, which are Doppler shifted away from $\Delta$ due to the finite Cooper pair momentum.  

For transparency ${\cal T}<1$, Andreev bound state solutions localized near the junction at $x=0$ can then only exist for energies inside both spectral gaps, $|E|<\Delta_-$.
On the other hand, for $|E|>\Delta_+$, propagating continuum states are possible along both directions, while for $\Delta_-<|E| <\Delta_+$, we have mixed-character states which can 
freely propagate along one direction but are evanescent along the other. 
As an important technical step forward, we specify BdG solutions that apply 
in a unified manner to all three energy regions.  We choose a formulation that can be 
leveraged to describe scattering states for finite voltage.
We first observe that for $x\ne 0$, using the Doppler-shifted energy $E_\alpha= E-\alpha v_F q$ for an $\alpha$-mover ($\alpha=\pm$) with energy $E$, 
Eq.~\eqref{BdG} implies that  electron ($e$) and hole ($h$) type states have the Nambu spinor structure 
\begin{equation}\label{tildepsi}
\tilde \psi_{\alpha,e}(E) = \frac{1}{\sqrt2}\left(\begin{array}{c} 1 \\ \rho(E_\alpha) \end{array}\right),
\quad \tilde \psi_{\alpha,h}(E) =\tau_x \tilde\psi_{\alpha,e}(E), 
\end{equation}
with the Andreev reflection amplitude
\begin{equation}\label{tildegamma} 
\rho(E)=  \left\{\begin{array}{cc}  
{\rm sgn}(E)\, \frac{|E|- \sqrt{E^2-\Delta^2}}{\Delta}, & |E|\ge \Delta,\\ & \\
\frac{E-i \sqrt{\Delta^2-E^2}}{\Delta}, & |E| <\Delta.
\end{array} \right. 
\end{equation} 
The states \eqref{tildepsi} describe arbitrary energies and are very useful for the description of outgoing (scattered) states, even 
though they satisfy unconventional normalization conditions.  On the other hand,
incident states, $\psi_{\alpha,e/h}(E)$, should satisfy the standard normalization condition
$\psi_{\alpha,e/h}^\dagger(E)\cdot \psi_{\alpha,e/h}^{}(E)=1$.
Since incident states are only defined for $|E_\alpha|>\Delta$, we can simply obtain them
from Eq.~\eqref{tildepsi} by including a normalization factor,
$\psi_{\alpha,e/h}(E)=\sqrt{\frac{2}{1+\rho^2(E_\alpha)}} \,\tilde\psi_{\alpha,e/h}(E)$.

\emph{MAR scattering states.---}We next construct scattering states for the finite-voltage case taking into account MAR processes.  Typical MAR trajectories in energy space (``MAR ladder'') are shown in Fig.~\ref{fig1}.  We consider an incident $\alpha$-mover which is an electron or hole like quasiparticle with energy $E$ in the respective continuum, $|E_\alpha|>\Delta$. 
For each step of the MAR ladder sketched in Fig.~\ref{fig1}, 
the energy of an electron changes by $\pm eV$ for right- or left-movers 
when traversing the normal junction region, and similarly the energy shift for holes is $\mp eV$.
The energy $E_n$ of an outgoing (reflected or transmitted) state may therefore involve the 
emission or absorption of an integer number of  $eV$ quanta,  $E_n=E+neV$ with integer $n$.  
Noting that there are four possible types of incident states, labeled by $s\in \{1,2,3,4\}$ depending on whether an electron- or a hole-type state is injected from the left or from the right side, we obtain a general \emph{Ansatz}
for  MAR scattering states. For $x=0^\pm$, the corresponding bispinor states have the form 
\begin{eqnarray} \nonumber
\Psi_E(0^-,t) &=&e^{-iEt}  \left( \begin{array}{c} \delta_{s,1}\, \psi_{+,e}(E) \\ \delta_{s,2} \, \psi_{-,h}(E)\end{array} \right) \\
\nonumber &+& \sum_n e^{-iE_n t}\left( \begin{array}{c} a_n\tilde\psi_{+,h}(E_n) \\ b_n \tilde\psi_{-,e}(E_n)
\end{array} \right), \\ \label{MARAnsatz}
\Psi_E(0^+,t) &=&e^{-iEt}  \left( \begin{array}{c} \delta_{s,3} \,\psi_{+,h}(E) \\ \delta_{s,4} \,\psi_{-,e}(E)\end{array} \right)\\
&+&\nonumber \sum_n e^{-iE_n t}
\left( \begin{array}{c} c_n \tilde\psi_{+,e}(E_n) 
\\ d_n \tilde\psi_{-,h}(E_n)\end{array} \right).
\end{eqnarray}
Keeping the incident quasiparticle energy $E$ and scattering channel $s$ implicit,  
the complex-valued scattering amplitudes $(a_n,b_n,c_n,d_n)$ appearing in the outgoing spinor ($\tilde \psi_{\alpha,e/h})$ contributions 
in Eq.~\eqref{MARAnsatz} are determined from the matching conditions in Eq.~\eqref{BC}.
They are also indicated in Fig.~\ref{fig1}.  As a result, the scattering amplitudes
satisfy a set of recurrence relations encoding the MAR ladder, see Ref.~\cite{PRB} for their explicit form.  

Given a solution of the recurrence relations, using the Fermi function $n_F(E)=1/(e^{E/T}+1)$ and superconducting density of states factors ($\Theta$ is the Heaviside function),
$\nu_{\alpha=\pm} (E) =  \frac{|E_{\alpha}|}{\sqrt{E_{\alpha}^2-\Delta^2}}
\Theta(|E_{\alpha}|-\Delta),$
the $I$-$V$ characteristics follows as
\begin{equation}\label{MARcur}   
I(V) = \frac{e}{2h} \sum_{\alpha=\pm}\int dE\, n_F(E)\nu_\alpha(E) 
    I_\alpha(r,E)+ (r\to -r),
\end{equation}
with the reflection amplitude $r$  in Eq.~\eqref{BC} and the current matrix elements 
 \begin{eqnarray}\nonumber
I_\alpha (r,E) &=&  \sum_{{\rm odd}\,n} \Bigl [ 
|c^{}_{\alpha,n}|^2 \left(1+|\rho(E+neV-v_Fq)|^2\right) \\
&-&  |d_{\alpha,n}|^2 \left(1+|\rho(E+neV+v_Fq)|^2\right)
 \Bigr]. \label{iam}
 \end{eqnarray}
By taking advantage of symmetry relations connecting the solutions incident from the left side ($s=1,2$) to 
those incident from the right side ($s=3,4$), the latter solutions are contained in Eq.~\eqref{MARcur} through 
the term with $r\to -r$.  The index $\alpha$ in Eq.~\eqref{iam} then corresponds to $s=1$ (for $\alpha=+$) and  $s=2$ (for $\alpha=-$).  
The current expression in Eq.~\eqref{MARcur}  affords a transparent physical interpretation.  Summing over all scattering channels $s$ and integrating
over all incident energies $E$, the
weight of the corresponding incident state in the current is determined by the product of the Fermi function, the density of states, 
and a current matrix element.  The latter follows by summing over all orders $n$ of the MAR ladder, where current contributions
only arise  for odd $n$.  At given order $n$,  electrons $(\propto |c_{\alpha,n}|^2)$ and holes $(\propto |d_{\alpha,n}|^2)$ enter with 
opposite sign, where the corresponding Doppler-shifted energy $E_n\mp v_Fq$ appears in the Andreev reflection amplitude $\rho(E)$.
 
For arbitrary system parameters, which are represented by the four dimensionless quantities $q\xi$, ${\cal T}$, $k_BT/\Delta$, and $eV/\Delta$, 
the rectification efficiency $\eta(V)$ in Eq.~\eqref{efficiency} follows from Eq.~\eqref{MARcur} 
after a numerical solution of the recurrence relations.  The order at which the recurrence relations can be truncated is given by $n_{\rm max}\sim 2\Delta/e|V|$. 
It is thus numerically difficult to reach extremely low voltages for high transparency ${\cal T}\alt 1$, where the MAR ladder in Fig.~\ref{fig1} includes a very large number of round trips in the junction region.  Our code accurately reproduces the $q=0$ results for $I_{q=0}(V)$ reported in Ref.~\cite{Averin1995}.  Another check passed by our code
comes from the ballistic limit ${\cal T}=1$, where the recurrence relations can be solved analytically for arbitrary other parameter values.  We next describe the corresponding results in the zero-temperature limit.

\emph{Ballistic limit.---}For ${\cal T}=1$, the matching conditions \eqref{BC} 
as well as the BdG Hamiltonian conserve chirality, $\sigma_z=\alpha=\pm$, and the 
recurrence relations admit a closed solution  \cite{PRB}.
We find that $I(V)$ for $q\ne 0$ is related to the known $q=0$ curve $I_{q=0}(V)$ \cite{Averin1995}  by
a simple shift,
\begin{equation}\label{currdopplershift}
    I(V) = I_{q=0}(V)+ \frac{4e\Delta}{h}  q\xi.  
\end{equation}
This shift has a clear physical interpretation: it is the current carried by Cooper pairs with finite momentum $2q$ and charge $2e$.  The simple decomposition \eqref{currdopplershift} only 
applies in the ballistic limit where chirality is conserved.   
The rectification efficiency  then follows for arbitrary voltage from Eq.~\eqref{efficiency} as
\begin{equation}\label{rectball}
    \eta(V,q\xi,{\cal T}=1) = \frac{4e\Delta}{h} \frac{q\xi}{I_{q=0}(V)},
\end{equation}
where the dependence on $q\xi$ is a simple proportionality.  Since we consider the regime $0\le q\xi<1$,
maximal efficiency is reached for $q\xi \to 1$.  This is 
in contrast to the equilibrium SDE case, where the maximal SDE efficiency $\eta_0$ is found for 
$q\xi\approx 0.9$, with $\eta_0\approx 0.4$ \cite{Davydova2022,PRB}.
In both cases, however, the optimal conditions for rectification correspond to full junction
transparency with ${\cal T}=1$.

For $e|V|\gg \Delta$, the Ohmic result of the corresponding normal-normal contact is approached, $I_0(V)\approx (2e^2/h)V$, implying $\eta(V) \simeq  2q\xi \frac{\Delta}{eV}$.
On the other hand, for $V\to 0$, using $I_{q=0}(V\to 0)\approx (4e \Delta/h)\,{\rm sgn}(V)$ \cite{Averin1995}, Eq.~\eqref{rectball} implies $\eta(V)\simeq q\xi.$ 
For $q\xi\to 1$, one approaches the ideal rectification limit since the MAR-induced current $I_0$ now
precisely cancels the finite-momentum Cooper pair current for $V<0$, i.e, $I(V<0)=0$ in Eq.~\eqref{currdopplershift},
while both currents add for $V>0$ to give $I(V>0)=8e\Delta/h$.  As a result, we have $\eta(V)=1$.
We conclude that MAR processes can generate highly efficient superconducting diode 
behavior in the deep subgap regime $e|V|\ll \Delta$.   As we show next, this enhancement of $\eta(V)$ compared to the equilibrium value $\eta_0$ (for otherwise identical parameters) is also found  for non-ideal transparency.



% Figure environment removed
% Figure environment removed


\emph{Subharmonic structure.---}In Fig.~\ref{fig2}, we show numerical results for 
$\eta(V)$ for different values of $(q\xi,{\cal T})$, again assuming the zero-temperature limit.  
We observe an overall increase of $\eta(V)$ with increasing Cooper pair 
momentum $2q$ and/or junction transparency ${\cal T}$.
The efficiency is particularly large in the subgap regime $eV\alt 2\Delta$, where we also observe a subharmonic structure with peaks or dips.  These MAR features are more clearly visible 
in the derivative $d\eta(V)/dV$ (bottom panel in Fig.~\ref{fig2}).
Apart from the standard $q=0$ MAR features at $2\Delta/eV=n$ (integer $n$),
which are also observed for $q\ne 0$ and follow from MAR trajectories as 
drawn in the upper panel in Fig.~\ref{fig1}, we also find 
resonances or antiresonances corresponding to the Doppler-shifted pairing gaps $\Delta_\pm$ (indicated by arrows in Fig.~\ref{fig2}). 
The corresponding transitions are naturally explained from the MAR ladder picture shown in the lower panel of Fig.~\ref{fig1}, where the presence of normal reflection $r\ne 0$ enables
MAR trajectories between states near the same type of spectral gap ($\pm\Delta_+$ or $\pm\Delta_-$) where $\nu_\alpha(E)$ has sharp peaks.  Let us also note that for $eV\gg \Delta$, the rectification efficiency is given by 
\begin{equation}\label{Adef}
\eta(eV\gg\Delta,q\xi,{\cal T}) \simeq A(q\xi,{\cal T}) \frac{\Delta}{eV}.
\end{equation}
The dimensionless coefficient $A=A(q\xi,{\cal T})$ is illustrated
in Fig.~\ref{fig3}, with $A=2q\xi$ for ${\cal T}=1$ from the analytical solution. 
For $q\xi \alt 1$, our numerical results suggest $A(q\xi,{\cal T})\approx 2q\xi{\cal T}$.
  
\emph{Conclusions.---}We have studied a model for a voltage-biased Josephson diode with finite Cooper pair momentum.  For voltages in the subgap regime, we find that MAR processes allow for large rectification efficiencies, far beyond those found in equilibrium.  In particular,  the ideal one-way rectification limit could be 
reached in principle.  Similar low-bias efficiency enhancements may also be found for Josephson diodes based on other physical mechanisms. We hope that future experimental and theoretical work will shed light on this intriguing question.


\begin{acknowledgments}
We thank Liang Fu for discussions. 
We acknowledge funding by the Deutsche Forschungsgemeinschaft (DFG, German Research Foundation) under Grant No.~277101999 - TRR 183 (project C01), Grant No.~EG 96/13-1, 
and under Germany's Excellence Strategy - Cluster of Excellence Matter and Light for Quantum Computing (ML4Q) EXC 2004/1 - 390534769.
This work received support from the French government under the France 2030 investment plan, as part of the Initiative d'Excellence d'Aix-Marseille Universit\'e - A*MIDEX, through the institutes IPhU (AMX-19-IET-008) and AMUtech (AMX-19-IET-01X).
\end{acknowledgments}

%\bibliographystyle{aipnum4-1}
\bibliography{sup}
\end{document}