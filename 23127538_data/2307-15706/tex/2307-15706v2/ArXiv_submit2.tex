\documentclass[graphics,floatfix, footinbib,tightenlines,nobibnotes, aps, prb, twocolumn]{revtex4-1}

\usepackage{amsmath,amssymb,wasysym}
\usepackage{graphicx}% Include figure files
\usepackage{dcolumn}% Align table columns on decimal point
\usepackage{bm}% bold math
\usepackage{braket}
\usepackage{verbatim}
\usepackage{float}
\usepackage{bbold}
\usepackage{color}
\usepackage{xcolor}
\usepackage{relsize}
\usepackage{amsthm}
\usepackage{enumerate}
\usepackage{soul,xcolor}
\usepackage[T1]{fontenc}
\usepackage{adjustbox}
\usepackage[colorlinks=true ,urlcolor=blue,urlbordercolor={0 1 1}]{hyperref}
\newcommand{\<}{\langle}
\newcommand{\e}{\varepsilon}
\newcommand{\up}{\uparrow}
\newcommand{\down}{\downarrow}
\renewcommand{\>}{\rangle}
\renewcommand{\(}{\left(}
\renewcommand{\)}{\right)}
\renewcommand{\[}{\left[}
\renewcommand{\]}{\right]}
\renewcommand{\v}[1]{\mathbf{#1}} % \v -> vector (bf)
\newcommand{\dslash}{d \hspace{-0.8ex}\rule[1.2ex]{0.8ex}{.1ex}}
\newcommand{\bs}[1]{\boldsymbol{#1}}
\renewcommand{\d}{\partial}
\newcommand{\del}{\nabla}
\renewcommand{\div}{\nabla\cdot}
\newcommand{\curl}{\nabla\times}
\newcommand{\eps}{\epsilon}
%special commands for this paper
\newcommand{\tv}{\tau_v}
\newcommand{\tvb}{\tau_{\bar{v}}}
\newcommand{\T}{\mathcal{T}}
\newcommand{\CT}{\mathcal{CT}}
\newcommand{\tCT}{\tilde{\mathcal{CT}}}
\newcommand{\be}{\begin{equation}}
\newcommand{\ba}{\begin{align}}
\newcommand{\ee}{\end{equation}}
\newcommand{\bea}{\begin{eqnarray}}
\newcommand{\eea}{\end{eqnarray}}
\newcommand{\beq}{\begin{equation}}
\newcommand{\eeq}{\end{equation}}
\newcommand{\beqn}{\begin{eqnarray}}
\newcommand{\eeqn}{\end{eqnarray}}
\newcommand{\HH}{{\cal H}}
\newcommand{\RR}{{\mathcal R}}
\newcommand{\CC}{\mathcal{C}}
\newcommand{\p}{\partial}
\newcommand{\s}{\sigma}
\newcommand{\lam}{\lambda}
\newcommand{\la}{\langle}
\newcommand{\ra}{\rangle}
\newcommand{\lb}{\left[}
\newcommand{\rb}{\right]}
\newcommand{\lp}{\left(}
\newcommand{\rp}{\right)}
\newcommand{\Tr}{{\rm \, Tr\,}}
\newcommand{\Z}{\mathbb{Z}}
\newcommand{\LL}{\mathcal{L}}
\newcommand {\ep}{\epsilon}
\newcommand{\TexBlue}[1]{ {\color{blue} #1}}
\newcommand{\ashvin}[1]{ {\color{red} #1}}
\newcommand{\yahui}[1]{ {\color{blue} #1}}

\newcommand{\hb}[1]{{\color{blue}{{#1}}}} 
\renewcommand{\hat}[1]{{\widehat #1}}
\renewcommand{\Re}{{\rm \, Re\,}}
\renewcommand{\Im}{{\rm \, Im\,}}
\def\nn{\nonumber\\}
\newcommand{\red}{}
\newcommand{\blue}{}
\newcommand{\redhighlight}{\color{red}}


\newcommand*{\yhz}[1]{\textcolor{red}{#1}}




\usepackage{bm}

\newcommand{\fix}[1]{{\color{blue}{{#1}}}} % needs to be fixed

\usepackage{array}
\newcolumntype{L}[1]{>{\raggedright\arraybackslash}p{#1}}
\newcolumntype{C}[1]{>{\centering\arraybackslash}p{#1}}
\newcolumntype{R}[1]{>{\raggedleft\arraybackslash}p{#1}}
\usepackage{multirow}






%\usepackage{pstricks, pst-node, pst-plot}
\allowdisplaybreaks

%\newcommand{\Z}{\mathbb{Z}}
\newcommand{\id}{\mathds{1}}
\newcommand{\x}{\mathbb{\times}}
\newcommand{\lv}{\langle}
\newcommand{\rv}{\rangle}
\newcommand{\R}{\mathbb{R}}
\newcommand{\Q}{\mathbb{Q}}
\newcommand{\C}{\mathbb{C}}
\newcommand{\Aut}{\textrm{Aut}}
\newcommand{\normal}{\trianglelefteq}
%\newcommand{\Tr}{\textrm{Tr}}
\newcommand*\diff{\mathop{}\!\mathrm{d}}
\newcommand*\Diff[1]{\mathop{}\!\mathrm{d^#1}}
\newcommand{\sgn}{\textrm{sgn}}




\begin{document}

\title{Type II t-J model and shared super-exchange coupling from Hund's rule in superconducting La$_3$Ni$_2$O$_7$ }

\author{Hanbit  Oh}
\author{Ya-Hui Zhang}
\email{yzhan566@jhu.edu}
\affiliation{William H. Miller III Department of Physics and Astronomy, Johns Hopkins University, Baltimore, Maryland, 21218, USA}

\date{\today}

\begin{abstract}
Recently, a 80 K superconductor was discovered in La$_3$Ni$_2$O$_7$ under high pressure. 
Density function theory (DFT) calculations identify $d_{x^2-y^2}$, $d_{z^2}$ as the active orbitals on the bilayer square lattice 
with a $d^{8-x}$ configuration of of Ni per site. 
One naive expectation is to describe this system in terms of a two-orbital t-J model. 
However, we emphasize the importance of the Hund's coupling $J_H$ and the $x=0$ limit should be viewed as a spin-one Mott insulator.
Especially, the significant Hund's coupling share the inter-layer super-exchange $J_\perp$ of the $d_{z^2}$ orbital to the $d_{x^2-y^2}$ orbital, an effect that cannot be captured by conventional perturbation or mean-field approaches.
In this study, we first explore the limit where the $d_{z^2}$ orbital is Mott localized, dealing with an one-orbital bilayer t-J model focused on the $d_{x^2-y^2}$ orbital. 
Notably, we find that strong inter-layer pairing survives up to $x=0.5$ hole doping driven by the transmitted $J_\perp$, which explains the existence of high Tc superconductor in the experiment at this doping level.
Next, we uncover the more realistic situation where the $d_{z^2}$ orbital is slightly hole doped and cannot be simply integrated out. 
We take the $J_H\rightarrow +\infty$ limit and propose a type II t-J model with four \textit{spin-half} singlon ($d^7$) states and three \textit{spin-one} doublon ($d^8$) states. 
Employing a parton mean field approach, we recover the similar results as in the one-orbital t-J model, but now with the effect of the $J_\perp$ automatically generated.
Our calculations demonstrate that the pairing strength decreases with the hole doping $x$ and $x=0.5$ is likely larger than the optimal doping. We propose future experiments to electron dope the system to further enhance $T_c$. 
 \end{abstract}

\maketitle
\textit{Introduction}:  Recently a superconductor with $T_c=80$K was found in La$_3$Ni$_2$O$_7$ under high pressure\cite{sun2023signatures}, following  previous discoveries of  superconductivity in nickelate Nd$_{1-x}$ Sr$_x$NiO$_2$ \cite{li2019superconductivity} and also in Nd$_6$Ni$_5$O$_{12}$\cite{pan2022superconductivity} at ambient pressure. The discovery has triggered many  experimental\cite{liu2023electronic,hou2023emergence} and theoretical\cite{luo2023bilayer,zhang2023electronic,yang2023possible,sakakibara2023possible,gu2023effective,shen2023effective,wu2023charge,christiansson2023correlated,liu2023s,hou2023emergence,liu2023electronic,cao2023flat} studies.
The average valence of Ni is in $d^{8-x}$ with $x=0.5$\cite{sun2023signatures}. Density functional theory (DFT) calculations identify a bilayer square lattice structure with active $d_{x^2-y^2}$ and $d_{z^2}$  orbitals, which we label as $d_1$ and $d_2$ in the following. The density (summed over spin) per site is estimated to be $n_1\approx 1-x=0.5$ and $n_2 \approx 1$, so that the $d_{z^2}$ orbital is close to Mott localization. Due to a large inter-layer hybridization of the $d_{z^2}$ orbital, we expect that it just forms a rung singlet when $n_2=1$.    The $d_{z^2}$ orbital has a small intra-layer hopping, thus we do not expect a strong superconductivity from it.  Then one may expect that superconductivity origins from the $d_{x^2-y^2}$ orbital. But the $d_{x^2-y^2}$ orbital is at hole doping level of $50\%$. According to the phase diagram of cuprates, it should be in the overdoped Fermi liquid phase.  A major goal of this paper is to identify the minimal model to describe the nickelate superconductor and also find a mechanism for the material to superconductor at such a large hole doping.


One important ingredient we identify is the Hund's coupling $J_H$ between the $d_{z^2}$ and the $d_{x^2-y^2}$ orbital. Due to the $J_H$ coupling, the $x=0$ limit should be viewed as a spin-one Mott insulator formed by Ni$^{2+}$. The strong Hund's coupling $J_H$ aligns the spin of the two orbitals at each site, then the large inter-layer spin coupling $J_\perp$ of the $d_{z^2}$ orbital is shared to the $d_{x^2-y^2}$ orbital. Therefore, when $n_2=1$, we can ignore the Mott localized $d_{z^2}$ orbital (which is in a gapped rung-singlet phase) and phenomenologically consider a bilayer one-orbital t-J model for $d_{x^2-y^2}$ only.  The model has  a large inter-layer spin coupling $J_\perp$ but without inter-layer hopping $t_\perp$, a new situation not possible in the usual one-orbital Hubbard model. Through a slave-boson mean field calculation, we find that a large $J_\perp$ disfavors the familiar $d_{x^2-y^2}$ pairing at the $J_\perp=0$ limit and the system forms a strong s-wave superconductor with dominant inter-layer pairing. The pairing strength decreases with the hole doping level $x$. But with a sufficiently large $J_\perp$, the pairing survives at $x=0.5$, which explains the superconductor at this hole doping level in the experiment. We note that a previous work has discussed quantitative renormalization effects of the Hund's coupling in flattening the bands\cite{cao2023flat}, but the effect we identify here is qualitatively distinct and completely new. To our best knowledge the possibility  of strong inter-layer pairing for the $d_{x^2-y^2}$ orbital due to Hund's rule coupling to a rung-singlet phase of the $d_{z^2}$ orbital has not been discussed previously.




The above treatment of `integrating' out the $d_{z^2}$ orbital is not very rigorous. Also, in the real system the $d_{z^2}$ orbital may also be slightly hole doped. To be more precise and to enable the doping of the $d_{z^2}$ orbital, we propose a bilayer type II t-J model to describe the low energy physics. The model is a generalization of a model proposed one of us before\cite{zhang2020type,zhang2021fractional}. Basically we take the large $J_H$ limit and restrict to a Hilbert space with four spin 1/2 singlon ($d^7$) states and three spin-one doublon ($d^8$) states.  Inter-orbital $J_H$ disappears in the model with the cost of non-trivial constraint.  The type II t-J model can be understood to describe the low energy physics of doping a spin-one Mott insulator\cite{zhang2022pair} with doped hole in a spin 1/2 state. The model has two important parameters: the total hole doping level $x$ and energy splitting $\Delta$ between the two orbitals to tune the relative doping of the two orbitals.  In the large $\Delta$ limit, we have $n_2=1$ and $d_{z^2}$ is Mott localized and forms a rung singlet. We propose a parton mean field theory to deal with the type II t-J model. In the simple large $\Delta$ limit, in mean field level we reach a bilayer one-orbital t-J model for an emergent `$d_{x^2-y^2}$' orbital in the mean field level. In this model, we can automatically get a large  $J_\perp/t$ from our parton mean field theory, justifying our previous phenomenological treatment. From a direct mean field calculation of the type II t-J model, we find s-wave inter-layer pairing at $x=0.5$ similar to the one-orbital t-J model before. 

\textit{Bilayer two-orbital model}: We  start from a two-orbital t-J model on a bilayer square lattice, Fig.~\ref{fig:1} (a), which has the following Hamiltonian,
\begin{eqnarray}
    H&=& H_{K}+ J^x_\parallel \sum_l \sum_{\langle ij \rangle} \vec S_{i;l;1}\cdot \vec S_{i;l;1}+J^z_\perp \sum_{i} \vec S_{i;t;2}\cdot \vec S_{i;b;2} \notag \\
&+ &U'\!\sum_i n_{i;1}n_{i;2}-2J_H\!\sum_i (\vec S_{i;1}\cdot \vec S_{i;2}+\frac{1}{4}n_{i;1}n_{i;2}), \label{e1} 
\end{eqnarray}
and
\begin{eqnarray}
    H_{K}&=&-t^x_\parallel \sum_{l ,\sigma} \sum_{\langle i,j \rangle} (P d^\dagger_{i;l;1;\sigma}d_{j;l;1;\sigma} P+H.c.)\notag \\ 
    &-&t^z_\parallel\sum_{l,\sigma }  \sum_{\langle i,j \rangle} (P d^\dagger_{i;l;2;\sigma}d_{j;l;2;\sigma} P+H.c.) \notag \\ 
    &-&t^{xz}_\parallel \sum_{l,\sigma }\sum_{\langle ij \rangle} ((-1)^{s_{ij}} P d^\dagger_{i;l;1;\sigma }d_{j;l;2;\sigma } P+H.c.) \notag \\ 
    &-&t_\perp^z \sum_i (P d^\dagger_{i;t;2;\sigma}d_{i;b;2;\sigma} P+H.c.) +\Delta \!\sum_i (n_{i;1}-n_{i;2}) \notag ,  
\end{eqnarray}
where $P$ is the projection operator to remove the double occupancy of each orbital. Here, $l=t,b$ labels the layer index, and $\sigma=\uparrow,\downarrow$ is for the spin index. We dub $d_1, d_2$ for the $d_{x^2-y^2}$ and $d_{z^2}$ orbital respectively. The hopping parameters are estimated $t_{\parallel}^{x}=0.485$, $t_{\parallel}^{z}=0.110$, $t_{\parallel}^{xz}=0.239$, 
$t_{\perp}^{z}=0.635$ by DFT \cite{luo2023bilayer}.  
% $t_{\perp}^{x}=-0.005$, 
$s_{ij}=1$ for the $x$ bond and $s_{ij}=-1$ for the $y$ bond.     For simplicity, we only keep intra-layer $J^{x}_{\parallel}$ for the $d_{x^2-y^2}$ orbital and the inter-layer $J^{z}_{\perp}$ for the $d_{z^2}$ coupling. 
$U'$ is inter-orbital repulsion and $J_H$ is the Hund's coupling.  $n_{i;a}$ is the density for orbital $a=1,2$. $\vec S_{i;a}$ is the spin operator for orbital $a=1,2$ summed over spin.  We also ignore the $n_i n_j$ term in the $J$ coupling.
In Fig.~\ref{fig:1}, we illustrate the system and the model. On average we have $n=2-x$ number of electron (summed over spin) per site with $x\approx 0.5$ in the experiment. We have $n_1 \approx 0.5$ and $n_2 \approx 1$.

% Figure environment removed

\textit{Bilayer one-orbital t-J model}: 
We first consider the limit where the $d_{2}$ orbital is Mott localized with pinned $n_2=1$.  In this limit, $d_{2}$ orbitals form a rung-singlet insulator due to large $J_\perp$ and may be integrated out and one can focus on an one-orbital t-J model with the $d_{1}$ orbital. However, we emphasize that the gapped $d_{2}$ degree of freedom still plays an important role due to the Hund's coupling. A large Hund's coupling enforces the two orbitals to form a spin-triplet at each site. Within the restricted Hilbert space, the spins of the two orbitals align and the inter-layer spin-spin coupling $J_\perp^z$ also induce anti-ferromagnetic coupling of the $d_{1}$ orbital (see the Inset of Fig.~\ref{fig:1}(a)). Basically only the orbital symmetric part, $J^{x}_{\perp}=J^{z}_{\perp}$, can persist in the restricted Hilbert space. 
Consequently, we should consider a significant inter-layer $J_\perp$ also for the $d_{x^2-y^2}$ orbital, though there is no inter-layer hopping.

Motivated by the above considerations, we now consider an effective one-orbital t-J model for the $d_{x^2-y^2}$ orbital, 
\begin{align}
    H_{\mathrm{eff}} &=
-t_{\parallel}^{x} \sum_{l, \sigma}\sum_{ \langle i,j \rangle}
P\left(
d_{i;1;l,\sigma }^{\dagger}d_{1;l;\sigma }
\right) P+H.c. \notag\\
& +J_{\parallel}^{x}\sum_{l}\sum_{ \langle  i, j \rangle}\vec{S}_{i;l;1}
\cdot \vec{S}_{i;l;1}
+  J_{\perp}^{z}\sum_{i}
\vec{S}_{i;t;1} \cdot \vec{S}_{i;b;1}
\label{eq:one-orbital t-Jmodel}
\end{align}
Hereafter, shorthand notation $t=t^{x}_{\parallel},J_{\parallel}=J^{x}_{\parallel}$, and $J_{\perp}=J^{z}_{\perp}$ are used, unless otherwise stated. Note that the model above is quite unconventional in the sense that we have a large $J_\perp$ but no inter-layer hopping $t_\perp$, compared to other existing models \cite{PhysRevB.95.214509}. This is impossible in the standard t-J model usually with $J<t$. We note a similar model (dubbed as mixed dimensional t-J model) has been proposed in the cold atom context but only out of equilibrium\cite{bohrdt2022strong,hirthe2023magnetically}. 

We then employ the standard U(1) slave-boson mean-field theory\cite{lee2006doping} and represent the electronic operator as, $d_{i;l;1;\sigma}^{\dagger} = f_{i;l;\sigma}^{\dagger} b_{i;l}$ with the constraint $n_{i;l;f}+n_{i;l;b}=1$ (see the Supplemental Material (SM) for details). In mean field level, we decouple the following order parameters from the $J$ terms: the hopping terms $\chi_{\parallel;ij,\sigma}^{l} = 2 \langle f_{i;l;\sigma}^{\dagger}f_{j;l;\sigma}\rangle$, $\chi_{\perp; i;\sigma}= 2\langle f_{i;t;\sigma}^{\dagger}f_{i;b;\sigma}\rangle$ and  the pairing terms $\Delta_{\parallel;ij}^{l} =  
2s^{ij}\langle f_{i;l;\uparrow}f_{j;l;\downarrow}\rangle$, $\Delta_{\perp;i} =2 \langle f_{i;t;\uparrow}f_{i;b,\downarrow}\rangle$.
We obtain these order parameters from self consistent calculations. We fix $t_{\parallel}=1$ and $J_{\parallel}=1/2$ and vary the $J_\perp$ and the doping $x$ in the range $0\leq x \leq 1/2$.  



Here we summarize our numerical results. In the limit of small $J_\perp$, the model reproduces the well-known behaviors of the single-layer t-J model, with the famous $d_{x^2-y^2}$ pairing within each layer. As the strength of $J_{\perp}$ is gradually increased, there is a first order transition after which we find s-wave pairing with dominated inter-layer pairing, as illustrated in Fig.\ref{fig:2} (a) (also see SM for more details).  With a large enough $J_{\perp}$ (for example, $J_\perp/t$>0.5), the value of $|\Delta_{\perp}|$ remains survives to the large hole doping regime with $x\simeq 0.5$. We note that the normal Fermi-surfaces are completely gapped in the  s-wave pairing phase, while there are nodes in the d-wave pairing, as depicted in Fig.\ref{fig:2} (d). $J_\perp/t>0.5$ is quite reasonable given that $J_\perp$ origins from the super-exchange of the $d_{2}$ orbital which has a large inter-layer coupling. Thus we expect a s-wave inter-layer paired superconductor in the experimental regime even with a $50\%$ hole doping. We emphasize that it is important to have large $J_\perp$ but with the inter-layer hopping $t_\perp=0$. For example, one can imagine a conventional bilayer t-J model for the $d_{z^2}$ orbital with $t_\perp>J_\perp$. In Fig.\textcolor{red}{S1} (d) in SM, we show that a large $t_\perp$ term suppresses the pairing because the hopping disfavors inter-layer spin-singlet Cooper pair. Therefore the unusual model we consider here for the $d_{x^2-y^2}$ orbital host stronger pairing than the usual t-J model.

% Figure environment removed

\textit{Type II t-J model}: The importance of the Hund's coupling in sharing the super-exchange $J$ has been demonstrated in the simple case of $n_2=1$ per site.
In this limit, the $d_{2}$ orbital is orbital-selective Mott localized and forms rung-singlet.
Then we just ignore $d_2$ and deal with a one-orbital model and take the transmission of $J_\perp$ by hand.  
However, this approach is not very rigorous and needs a justification. Moreover, in real system, the $d_{2}$ orbital is likely to be slightly hole doped with $n_2<1$. Then the $d_2$ orbital should be kept in the low energy model. In this case, we need to deal with the full two-orbital model in Eq.~\ref{e1}. However, $U'$ and $J_H$ are large and cannot be treated in perturbation or mean field level. Especially, there is no good way to capture the effect of sharing the J terms between the two orbitals from the Hund's coupling. Apparently, a new model and a new method is called for to describe the realistic regimes with two active orbitals and a strong Hund's coupling.

To address this challenging problem, we take a non-perturbative approach. We first take $U',J_H$ to be large and project to a restricted Hilbert space. This leads to a generalization of the type II t-J model proposed by one of us in Ref.\onlinecite{zhang2020type}. 
We only keep four singlon ($d^7$) states and three spin-triplet doublon ($d^8$) states. First, at each site $i$, the four singlon states can be labeled as, $\ket{a \sigma}=d^\dagger_{a;\sigma}\ket{G}$ where $\ket{G}$ is defined as a vacuum states where all $t_{2g}$ orbitals are fully filled with $a=1,2$ and $\sigma=\uparrow, \downarrow$.
Meanwhile, the three spin-triplet doublon states are written as, $\ket{-1}=d^\dagger_{1\downarrow}d^\dagger_{2\downarrow}\ket{G}$, $\ket{0}=\frac{1}{\sqrt{2}}(d^\dagger_{1\uparrow}d^\dagger_{2\downarrow}+d^\dagger_{1\downarrow}d^\dagger_{2\uparrow})\ket{G}$ and $\ket{1}=d^\dagger_{1\uparrow}d^\dagger_{2\uparrow}\ket{G}$. Here, we ignore the site index $i$ for simplicity. The spin-singlet doubly occupied states is penalized by a large $J_H$ and is removed from the Hilbert space.


Now, we project the electron operator inside this $4+3=7$ dimensional Hilbert space, 
\begin{eqnarray}
    d_{i;l;1\uparrow}&= \prod_{j<i}(-1)^{n_j}  \big(\ket{2\uparrow}_{il}\bra{1}_{i\l}+\frac{1}{\sqrt{2}}\ket{2\downarrow}_{il}\bra{0}_{il}\big),
    \label{eq:p_electron_1}
\notag \\
    d_{i;l;1\downarrow}&= \prod_{j<i}\!(-1)^{n_j}  \big(\ket{2\downarrow}_{il}\bra{-1}_{il}\!+\frac{1}{\sqrt{2}}\ket{2\uparrow}_{il}\bra{0}_{il}\big),
     \label{eq:p_electron_2}
\notag \\
    d_{i;l;2\uparrow}&= -\!\prod_{j<i}\!(-1)^{n_j}  \big(\ket{1\uparrow}_{il}\bra{1}_{i\;l}+\!\frac{1}{\sqrt{2}}\ket{1\downarrow}_{il}\bra{0}_{il}\big),
     \label{eq:p_electron_3}
\notag \\
        d_{i;l;2\downarrow}&= -\!\prod_{j<i}(-1)^{n_j}  \big(\ket{1\downarrow}_{il}\bra{-1}_{il}\!+\!\frac{1}{\sqrt{2}}\ket{1\uparrow}_{il}\bra{0}_{il}\big)
     \label{eq:p_electron_4}
\end{eqnarray}
where $\prod_{j<i}(-1)^{n_j}$ is the Jordan-Wigner string. 
The spin operators for the \textit{spin-1/2} singlon state are $ \vec s_{i;a}=\frac{1}{2}\sum_{\sigma \sigma'} \ket{a\sigma}_i \vec \sigma_{\sigma \sigma'}\bra{a\sigma'}_i$ with $\vec \sigma$ as the Pauli matrices. the spin operators for the \textit{spin-one} doublon states are written as $ \vec S_i=\sum_{\alpha,\beta=-1,0,1} \vec T_{\alpha \beta]}\ket{\alpha}_i \bra{\beta}_i$. Here we have $  $, $  T_x=\frac{1}{\sqrt{2}}\begin{pmatrix} 0 & 1 & 0 \\ 1 & 0 & 1 \\ 0 & 1 & 0 \end{pmatrix}$ and $ T_y=\frac{1}{\sqrt{2}}\begin{pmatrix} 0 & -i & 0 \\ i & 0 & -i \\ 0 & i & 0 \end{pmatrix}$ in the $\ket{1},\ket{0},\ket{-1}$ basis.



The type II t-J model Hamiltonian is 
\begin{eqnarray}
     H=H_{K}&+&   J^x_\parallel \sum_l \sum_{\langle ij \rangle} \vec s_{i;l;1}\cdot \vec s_{i;l;1}+J^z_\perp \sum_{i} \vec s_{i;t;2}\cdot \vec s_{i;b;2} \notag \\
&+& J_{sd}^\parallel \sum_l \sum_{\langle ij \rangle} (\vec s_{i;l;1}\cdot \vec S_{i;l} +\cdot \vec S_{i;l}\cdot \vec s_{j;l;1}) \notag \\ 
&+&J_{sd}^\perp 
 \sum_i (\vec s_{i;t;2}\cdot \vec S_{i;b}+\vec S_{i;t}\cdot \vec s_{i;b;2})\notag \\
&+&J_{dd}^\parallel \sum_l \sum_{\langle ij \rangle}\vec S_{i;l}\cdot \vec S_{j;l}+J_{dd}^\perp \sum_i \vec S_{i;t}\cdot \vec S_{i;b} ,
\label{eq:type_II_t_J_main}
\end{eqnarray}
where $H_K$ is the same as in Eq.~\ref{e1}, except that the above projected electron operators are in the $4+3=7$ Hilbert space as defined above. We have $J_{sd}^\parallel=\frac{1}{2}J^x_\parallel$, $J_{sd}^\perp=\frac{1}{2} J^z_\perp$. $J_{dd}^\parallel=\frac{1}{4} J^x_\parallel$ and $J_{dd}^\perp=\frac{1}{4}J^z_\perp$. We are interested in the filling of $n_T=n_1+n_2=1+n=2-x$. If the number of sites is $N_S$, there are $(1-x) N_s$ number of doublon states and $x N_s$ number of singlon states. The energy splitting $\Delta$ in $H_K$ tunes the relative density of the two orbitals. In particular, if $\Delta$ is large and positive, we only need to keep two singlon states corresponding to the $d_2$ orbital.


\textit{Parton mean-field theory}: We employ the three-fermion parton construction\cite{zhang2020type} to deal with the type II t-J model. 
The four singlon states are constructed as  $\ket{a\sigma}_i=f^\dagger_{i;a\sigma}\ket{0}$, while the three S=1 doublons are created by $\ket{-1}_i=\psi^\dagger_{i;1\downarrow}\psi^\dagger_{i;2\downarrow}\ket{0}$, $\ket{0}_i=\frac{1}{\sqrt{2}}(\psi^\dagger_{i;1\uparrow}\psi^\dagger_{i;2\downarrow}-\psi^\dagger_{i;2\uparrow}\psi^\dagger_{i;1\downarrow})\ket{0}$ and $\ket{1}=\psi^\dagger_{i;1\uparrow}\psi^\dagger_{i;2\uparrow}\ket{0}$. We need to impose a local constraint at each site $i$: $n_{i;f}+n_{i;\psi_1}=1$, $n_{i;\psi_1}=n_{i;\psi_2}$ with $n_{i;f}=\sum_{a\sigma}f^\dagger_{i;a\sigma}f_{i;a\sigma}$ and $n_{i;\psi_a}=\sum_\sigma \psi^\dagger_{i;a\sigma}\psi_{i;a\sigma}$. 
On average, we have $n_f=x$ and $n_{\psi_1}=n_{\psi_2}=1-x$ with the convention $n_1+n_2=2-x$. We introduce the notation $\Psi_{i;\sigma}=(\psi_{i;1\sigma},\psi_{i;2\sigma})^T$, then there is another constraint: $\Psi^\dagger_i \vec \tau \Psi_i=0$ where $\vec \tau$ is Pauli matrix in the color space. This constraint enforces the two colors $a=1,2$ forms singlet, thus the spin is in a triplet due to fermion statistics\cite{zhang2020type}. This constraint gives a SU(2) gauge symmetry: $\Psi_i \rightarrow U_i \psi_i$ where $U_i \in SU(2)$ acting in the color space, rotating $\psi_1$ to $\psi_2$.




Within the parton construction, the projected electron operator is represented as,
$
d_{i;a\sigma}=\epsilon_{ab}f^\dagger_{i;b \sigma}\psi_{i;2 \sigma}\psi_{i;1\sigma}+\frac{1}{2} \epsilon_{ab} f^\dagger_{i;b\bar \sigma}(\psi_{i;2\downarrow}\psi_{i;1\uparrow}+\psi_{i;2\uparrow}\psi_{i;1\downarrow})
$. Here, $\epsilon_{ab}$ is the anti-symmetric tensor with $\epsilon_{12}=1$ and $\bar \sigma$ denotes the opposite spin of $\sigma$.
The singlon and doublon spin operators are now represented as, $\vec s_{i;a}=\frac{1}{2}\sum_{\sigma,\sigma'} f^\dagger_{i;a\sigma} \vec \sigma_{\sigma \sigma'} f_{i;a\sigma'}$ and $\vec S_i=\frac{1}{2} \sum_{a}\sum_{\sigma \sigma'} \psi^\dagger_{i;a\sigma}\vec \sigma_{\sigma \sigma'} \psi_{i;a\sigma'}$. 



Substituting all the above expressions, one can decouple the type II t-J model in Eq.~\ref{eq:type_II_t_J_main} and perform the self-consistent mean-field calculation. We provide all details in SM. In principle, one can have a phase diagram from tuning $\Delta$ and $x$. For simplicity, we her consider the large positive $\Delta$ limit, so that $n_2$ is pinned to be $1$, safely 
ignoring $f_1$ and keeping only the two singlon states occupied by $f_{2\sigma}$. 
This corresponds to orbital selective Mott localization of the $d_{z^2}$ orbital and now $d_{i;2\sigma}=0$ without the $f_{1}$ operator.
One important mean field decoupling is an on-site term, $\langle \psi^\dagger_{i;l;a\sigma}f_{i;l;2\sigma} \rangle =\frac{3}{4}\Phi_a$ for each spin $\sigma$ component. Due to the SU(2) gauge symmetry, we can always fix the gauge to choose $\Phi_2\neq 0$ while $\Phi_1=0$. Then $\langle \psi^\dagger_{i;l;2\sigma} f_{i;l;2\sigma} \rangle =3\Phi_2/4 \neq 0$ and we have $d_{i;l;1\sigma}\sim \frac{3}{4} \Phi_2^\dagger \psi_{i;l;1\sigma}$.  
Now $\psi_{i;l;1\sigma}$ can be identified as the electron operator of the $d_{x^2-y^2}$ orbital with density $n_{\psi_1}=1-x$, while $f_2$ and $\psi_2$ hybridize and form the same band with the total density $n_{f_2}+n_{\psi_2}=1$ per site. They just represent the localized spin moments of the $d_{z^2}$ orbital and form a rung singlet in the bilayer model due to the large $J^z_\perp$ term. 


In terms of the emergent `$d_{x^2-y^2}$' orbital $\psi_1$, an effective model can be derived from Eq.~\ref{eq:type_II_t_J_main} by substituting $d_{i;l_1\sigma}\sim \frac{3}{4} \Phi^\dagger_2 \psi_{i;l;1\sigma}$,
\begin{eqnarray}
      H_{\psi_1}&= &\sum_l \sum_{\langle ij \rangle} \Big[
      -\frac{9}{16} |\Phi_2|^2  t^x_\parallel \psi^\dagger_{i;l; 1\sigma} \psi_{i; l; 1\sigma}
      \nonumber
      \\
      &+&
      J^\parallel_{dd} \vec S_{i;l;\psi_1}\cdot \vec S_{j;l;\psi_1}
\Big]
      +J^\perp_{dd} \sum_i \vec S_{i;t;\psi_1}\cdot \vec S_{i;b;\psi_1}  
\end{eqnarray}
where $\vec S_{i;l;\psi_1}=\frac{1}{2} \psi^\dagger_{i;l;1\sigma} \vec{\sigma}_{\sigma \sigma'} \psi_{i;l;1\sigma'}$ is the spin operator of $\psi_1$.  The effective spin-spin coupling for this emergent $\psi_1$ orbital originates from the $J_{dd}$ coupling of the spin-one moments. As a result, the super-exchange of both $d_{z^2}$ and $d_{x^2-y^2}$ orbitals contribute to the $J$ coupling of this effective model. We have a large $J^\perp_{dd}=\frac{1}{4} J^z_\perp$ and large  $J^\parallel_{dd}=\frac{1}{4}J^x_\parallel$ for this emergent $\psi_1\sim d_1$ orbital, even though there is no inter-layer hopping. We also note an interesting effect of reducing the hopping by a factor of $|\Phi_2|^2$ ($|\Phi_2|<0.5$ from our calculation as in Fig \textcolor{red}{S2}(c) in SM).

 We perform a full self-consistent mean field calculation involving all $f_2, \psi_1, \psi_2$ orbitals. We confirm that $f_2, \psi_2$ just form a band insulator in agreement with a rung-singlet phase, while the $\psi_1$ orbital is at density $n_1=1-x$ and  gets intra-layer and inter-layer pairing terms as shown in Fig.~\ref{fig:2}(b-c). Note that we still use $t$, $J_\parallel$, and $J_\perp$ as abbreviation of $t^x_\parallel$, $J^x_\parallel$ and $J^z_\perp$, and set $t=1$, $J_\parallel=1/2$. Varying $J_\perp$, we again find a first order transition from the familiar d-wave to s-wave pairing with dominated inter-layer pairing. If we take a large $J_\perp$ such as  $J_\perp/t=1$, the s-wave pairing is still large at $x=0.5$. Overall, the results are qualitatively the same as the previous bilayer one-orbital t-J model (see Fig.~\ref{fig:2}(a)), justifying our previous treatment. However, now we achieve these results from a more precise approach of a microscopic model. The sharing of the super-exchange of one-orbital to the other orbital is automatically taken care of in our model and parton framework. 


\textit{Discussion}: The calculation in Fig.~\ref{fig:2} is limited to the large $\Delta$ regime with the orbital $d_{z^2}$ in a Mott localized state (forming a rung singlet). In the realistic system we may have  a smaller $\Delta$ and  the $d_{z^2}$ orbital may likely be slightly doped and also participates in the pairing. This will induce some quantitative effects: (1) $d_{z^2}$ orbital also contributes to superconductivity; (2) The effective hole doping level of the $d_{x^2-y^2}$ can get reduced even though the total hole doping level is fixed; (3) The inter-orbital hopping may further transmit the pairing of one orbital to the other orbital.  We note that two-orbital t-J model has been proposed and studied for La$_3$Ni$_2$O$_7$ (for example, see Ref.~\onlinecite{luo2023bilayer}), but the previous works all ignore the important effect of sharing the super-exchange $J$ coupling between the two-orbitals by the large Hund's coupling. We have demonstrated that this effect is crucial in the large $\Delta$ limit, so obviously it should not be ignored in the smaller $\Delta$ regime.  With both orbitals active, we also can not derive an one-orbital model simply by integrating the $d_{z^2}$ orbital. In this regime we believe the type II t-J model we propose here is the minimal model to capture all essential ingredients.  A phase diagram of $(\Delta,x)$ can be obtained by extending our parton mean field theory with $f_1$ orbital included, which we leave to a future work.






\textit{Conclusion}: In summary, we propose and study a bilayer type II t-J model for the superconducting La$_3$Ni$_2$O$_7$ under high pressure. We emphasize the important role of the Hund's coupling between the $d_{x^2-y^2}$ and the $d_{z^2}$ orbital, which enforces the $d^8$ state to be a spin-triplet.  Due to the Hund's rule, the super-exchange of one-orbital can be shared to the other orbital.  We propose a parton mean field treatment of the type II t-J model. In the limit that the $d_{z^2}$ is Mott localized and forms a rung singlet, we reach a bilayer one-orbital t-J model without inter-layer hopping, but with enhanced  inter-layer anti-ferromagnetic spin-spin coupling $J_\perp$ over intra-layer hopping $t$. Mean field theory then predicts a s-wave inter-layer paired superconductor even at hole doping $50\%$, in agreement with the experiment. In future, one natural extension is to tune the orbital splitting $\Delta$ in our type II t-J model to make the $d_{z^2}$ orbital also slightly hole doped.    We  also propose future experiments to  reduce $x$ through electron doping to search for an even higher $T_c$ than 80 K. 


\textit{Note added}: When finalizing the manuscript, we become aware of a preprint\cite{lu2023interlayer} which also studied a bilayer one-orbital t-J model with strong inter-layer $J_\perp$, which overlaps with the first part of our paper. 

\textit{Acknowledgement}: YHZ was supported by the National Science Foundation under Grant No. DMR-2237031. 
%\bibliography{refs}
%merlin.mbs apsrev4-1.bst 2010-07-25 4.21a (PWD, AO, DPC) hacked
%Control: key (0)
%Control: author (8) initials jnrlst
%Control: editor formatted (1) identically to author
%Control: production of article title (-1) disabled
%Control: page (0) single
%Control: year (1) truncated
%Control: production of eprint (0) enabled
\begin{thebibliography}{23}%
\makeatletter
\providecommand \@ifxundefined [1]{%
 \@ifx{#1\undefined}
}%
\providecommand \@ifnum [1]{%
 \ifnum #1\expandafter \@firstoftwo
 \else \expandafter \@secondoftwo
 \fi
}%
\providecommand \@ifx [1]{%
 \ifx #1\expandafter \@firstoftwo
 \else \expandafter \@secondoftwo
 \fi
}%
\providecommand \natexlab [1]{#1}%
\providecommand \enquote  [1]{``#1''}%
\providecommand \bibnamefont  [1]{#1}%
\providecommand \bibfnamefont [1]{#1}%
\providecommand \citenamefont [1]{#1}%
\providecommand \href@noop [0]{\@secondoftwo}%
\providecommand \href [0]{\begingroup \@sanitize@url \@href}%
\providecommand \@href[1]{\@@startlink{#1}\@@href}%
\providecommand \@@href[1]{\endgroup#1\@@endlink}%
\providecommand \@sanitize@url [0]{\catcode `\\12\catcode `\$12\catcode
  `\&12\catcode `\#12\catcode `\^12\catcode `\_12\catcode `\%12\relax}%
\providecommand \@@startlink[1]{}%
\providecommand \@@endlink[0]{}%
\providecommand \url  [0]{\begingroup\@sanitize@url \@url }%
\providecommand \@url [1]{\endgroup\@href {#1}{\urlprefix }}%
\providecommand \urlprefix  [0]{URL }%
\providecommand \Eprint [0]{\href }%
\providecommand \doibase [0]{http://dx.doi.org/}%
\providecommand \selectlanguage [0]{\@gobble}%
\providecommand \bibinfo  [0]{\@secondoftwo}%
\providecommand \bibfield  [0]{\@secondoftwo}%
\providecommand \translation [1]{[#1]}%
\providecommand \BibitemOpen [0]{}%
\providecommand \bibitemStop [0]{}%
\providecommand \bibitemNoStop [0]{.\EOS\space}%
\providecommand \EOS [0]{\spacefactor3000\relax}%
\providecommand \BibitemShut  [1]{\csname bibitem#1\endcsname}%
\let\auto@bib@innerbib\@empty
%</preamble>
\bibitem [{\citenamefont {Sun}\ \emph {et~al.}(2023)\citenamefont {Sun},
  \citenamefont {Huo}, \citenamefont {Hu}, \citenamefont {Li}, \citenamefont
  {Liu}, \citenamefont {Han}, \citenamefont {Tang}, \citenamefont {Mao},
  \citenamefont {Yang}, \citenamefont {Wang} \emph
  {et~al.}}]{sun2023signatures}%
  \BibitemOpen
  \bibfield  {author} {\bibinfo {author} {\bibfnamefont {H.}~\bibnamefont
  {Sun}}, \bibinfo {author} {\bibfnamefont {M.}~\bibnamefont {Huo}}, \bibinfo
  {author} {\bibfnamefont {X.}~\bibnamefont {Hu}}, \bibinfo {author}
  {\bibfnamefont {J.}~\bibnamefont {Li}}, \bibinfo {author} {\bibfnamefont
  {Z.}~\bibnamefont {Liu}}, \bibinfo {author} {\bibfnamefont {Y.}~\bibnamefont
  {Han}}, \bibinfo {author} {\bibfnamefont {L.}~\bibnamefont {Tang}}, \bibinfo
  {author} {\bibfnamefont {Z.}~\bibnamefont {Mao}}, \bibinfo {author}
  {\bibfnamefont {P.}~\bibnamefont {Yang}}, \bibinfo {author} {\bibfnamefont
  {B.}~\bibnamefont {Wang}},  \emph {et~al.},\ }\href@noop {} {\bibfield
  {journal} {\bibinfo  {journal} {Nature}\ ,\ \bibinfo {pages} {1}} (\bibinfo
  {year} {2023})}\BibitemShut {NoStop}%
\bibitem [{\citenamefont {Li}\ \emph {et~al.}(2019)\citenamefont {Li},
  \citenamefont {Lee}, \citenamefont {Wang}, \citenamefont {Osada},
  \citenamefont {Crossley}, \citenamefont {Lee}, \citenamefont {Cui},
  \citenamefont {Hikita},\ and\ \citenamefont
  {Hwang}}]{li2019superconductivity}%
  \BibitemOpen
  \bibfield  {author} {\bibinfo {author} {\bibfnamefont {D.}~\bibnamefont
  {Li}}, \bibinfo {author} {\bibfnamefont {K.}~\bibnamefont {Lee}}, \bibinfo
  {author} {\bibfnamefont {B.~Y.}\ \bibnamefont {Wang}}, \bibinfo {author}
  {\bibfnamefont {M.}~\bibnamefont {Osada}}, \bibinfo {author} {\bibfnamefont
  {S.}~\bibnamefont {Crossley}}, \bibinfo {author} {\bibfnamefont {H.~R.}\
  \bibnamefont {Lee}}, \bibinfo {author} {\bibfnamefont {Y.}~\bibnamefont
  {Cui}}, \bibinfo {author} {\bibfnamefont {Y.}~\bibnamefont {Hikita}}, \ and\
  \bibinfo {author} {\bibfnamefont {H.~Y.}\ \bibnamefont {Hwang}},\ }\href@noop
  {} {\bibfield  {journal} {\bibinfo  {journal} {Nature}\ }\textbf {\bibinfo
  {volume} {572}},\ \bibinfo {pages} {624} (\bibinfo {year}
  {2019})}\BibitemShut {NoStop}%
\bibitem [{\citenamefont {Pan}\ \emph {et~al.}(2022)\citenamefont {Pan},
  \citenamefont {Ferenc~Segedin}, \citenamefont {LaBollita}, \citenamefont
  {Song}, \citenamefont {Nica}, \citenamefont {Goodge}, \citenamefont {Pierce},
  \citenamefont {Doyle}, \citenamefont {Novakov}, \citenamefont
  {C{\'o}rdova~Carrizales} \emph {et~al.}}]{pan2022superconductivity}%
  \BibitemOpen
  \bibfield  {author} {\bibinfo {author} {\bibfnamefont {G.~A.}\ \bibnamefont
  {Pan}}, \bibinfo {author} {\bibfnamefont {D.}~\bibnamefont {Ferenc~Segedin}},
  \bibinfo {author} {\bibfnamefont {H.}~\bibnamefont {LaBollita}}, \bibinfo
  {author} {\bibfnamefont {Q.}~\bibnamefont {Song}}, \bibinfo {author}
  {\bibfnamefont {E.~M.}\ \bibnamefont {Nica}}, \bibinfo {author}
  {\bibfnamefont {B.~H.}\ \bibnamefont {Goodge}}, \bibinfo {author}
  {\bibfnamefont {A.~T.}\ \bibnamefont {Pierce}}, \bibinfo {author}
  {\bibfnamefont {S.}~\bibnamefont {Doyle}}, \bibinfo {author} {\bibfnamefont
  {S.}~\bibnamefont {Novakov}}, \bibinfo {author} {\bibfnamefont
  {D.}~\bibnamefont {C{\'o}rdova~Carrizales}},  \emph {et~al.},\ }\href@noop {}
  {\bibfield  {journal} {\bibinfo  {journal} {Nature materials}\ }\textbf
  {\bibinfo {volume} {21}},\ \bibinfo {pages} {160} (\bibinfo {year}
  {2022})}\BibitemShut {NoStop}%
\bibitem [{\citenamefont {Liu}\ \emph {et~al.}(2023{\natexlab{a}})\citenamefont
  {Liu}, \citenamefont {Huo}, \citenamefont {Li}, \citenamefont {Li},
  \citenamefont {Liu}, \citenamefont {Dai}, \citenamefont {Zhou}, \citenamefont
  {Hao}, \citenamefont {Lu}, \citenamefont {Wang} \emph
  {et~al.}}]{liu2023electronic}%
  \BibitemOpen
  \bibfield  {author} {\bibinfo {author} {\bibfnamefont {Z.}~\bibnamefont
  {Liu}}, \bibinfo {author} {\bibfnamefont {M.}~\bibnamefont {Huo}}, \bibinfo
  {author} {\bibfnamefont {J.}~\bibnamefont {Li}}, \bibinfo {author}
  {\bibfnamefont {Q.}~\bibnamefont {Li}}, \bibinfo {author} {\bibfnamefont
  {Y.}~\bibnamefont {Liu}}, \bibinfo {author} {\bibfnamefont {Y.}~\bibnamefont
  {Dai}}, \bibinfo {author} {\bibfnamefont {X.}~\bibnamefont {Zhou}}, \bibinfo
  {author} {\bibfnamefont {J.}~\bibnamefont {Hao}}, \bibinfo {author}
  {\bibfnamefont {Y.}~\bibnamefont {Lu}}, \bibinfo {author} {\bibfnamefont
  {M.}~\bibnamefont {Wang}},  \emph {et~al.},\ }\href@noop {} {\bibfield
  {journal} {\bibinfo  {journal} {arXiv preprint arXiv:2307.02950}\ } (\bibinfo
  {year} {2023}{\natexlab{a}})}\BibitemShut {NoStop}%
\bibitem [{\citenamefont {Hou}\ \emph {et~al.}(2023)\citenamefont {Hou},
  \citenamefont {Yang}, \citenamefont {Liu}, \citenamefont {Li}, \citenamefont
  {Shan}, \citenamefont {Ma}, \citenamefont {Wang}, \citenamefont {Wang},
  \citenamefont {Guo}, \citenamefont {Sun} \emph {et~al.}}]{hou2023emergence}%
  \BibitemOpen
  \bibfield  {author} {\bibinfo {author} {\bibfnamefont {J.}~\bibnamefont
  {Hou}}, \bibinfo {author} {\bibfnamefont {P.}~\bibnamefont {Yang}}, \bibinfo
  {author} {\bibfnamefont {Z.}~\bibnamefont {Liu}}, \bibinfo {author}
  {\bibfnamefont {J.}~\bibnamefont {Li}}, \bibinfo {author} {\bibfnamefont
  {P.}~\bibnamefont {Shan}}, \bibinfo {author} {\bibfnamefont {L.}~\bibnamefont
  {Ma}}, \bibinfo {author} {\bibfnamefont {G.}~\bibnamefont {Wang}}, \bibinfo
  {author} {\bibfnamefont {N.}~\bibnamefont {Wang}}, \bibinfo {author}
  {\bibfnamefont {H.}~\bibnamefont {Guo}}, \bibinfo {author} {\bibfnamefont
  {J.}~\bibnamefont {Sun}},  \emph {et~al.},\ }\href@noop {} {\bibfield
  {journal} {\bibinfo  {journal} {arXiv preprint arXiv:2307.09865}\ } (\bibinfo
  {year} {2023})}\BibitemShut {NoStop}%
\bibitem [{\citenamefont {Luo}\ \emph {et~al.}(2023)\citenamefont {Luo},
  \citenamefont {Hu}, \citenamefont {Wang}, \citenamefont {Wu},\ and\
  \citenamefont {Yao}}]{luo2023bilayer}%
  \BibitemOpen
  \bibfield  {author} {\bibinfo {author} {\bibfnamefont {Z.}~\bibnamefont
  {Luo}}, \bibinfo {author} {\bibfnamefont {X.}~\bibnamefont {Hu}}, \bibinfo
  {author} {\bibfnamefont {M.}~\bibnamefont {Wang}}, \bibinfo {author}
  {\bibfnamefont {W.}~\bibnamefont {Wu}}, \ and\ \bibinfo {author}
  {\bibfnamefont {D.-X.}\ \bibnamefont {Yao}},\ }\href@noop {} {\bibfield
  {journal} {\bibinfo  {journal} {arXiv preprint arXiv:2305.15564}\ } (\bibinfo
  {year} {2023})}\BibitemShut {NoStop}%
\bibitem [{\citenamefont {Zhang}\ \emph {et~al.}(2023)\citenamefont {Zhang},
  \citenamefont {Lin}, \citenamefont {Moreo},\ and\ \citenamefont
  {Dagotto}}]{zhang2023electronic}%
  \BibitemOpen
  \bibfield  {author} {\bibinfo {author} {\bibfnamefont {Y.}~\bibnamefont
  {Zhang}}, \bibinfo {author} {\bibfnamefont {L.-F.}\ \bibnamefont {Lin}},
  \bibinfo {author} {\bibfnamefont {A.}~\bibnamefont {Moreo}}, \ and\ \bibinfo
  {author} {\bibfnamefont {E.}~\bibnamefont {Dagotto}},\ }\href@noop {}
  {\bibfield  {journal} {\bibinfo  {journal} {arXiv preprint arXiv:2306.03231}\
  } (\bibinfo {year} {2023})}\BibitemShut {NoStop}%
\bibitem [{\citenamefont {Yang}\ \emph {et~al.}(2023)\citenamefont {Yang},
  \citenamefont {Liu}, \citenamefont {Wang},\ and\ \citenamefont
  {Wang}}]{yang2023possible}%
  \BibitemOpen
  \bibfield  {author} {\bibinfo {author} {\bibfnamefont {Q.-G.}\ \bibnamefont
  {Yang}}, \bibinfo {author} {\bibfnamefont {H.-Y.}\ \bibnamefont {Liu}},
  \bibinfo {author} {\bibfnamefont {D.}~\bibnamefont {Wang}}, \ and\ \bibinfo
  {author} {\bibfnamefont {Q.-H.}\ \bibnamefont {Wang}},\ }\href@noop {}
  {\bibfield  {journal} {\bibinfo  {journal} {arXiv preprint arXiv:2306.03706}\
  } (\bibinfo {year} {2023})}\BibitemShut {NoStop}%
\bibitem [{\citenamefont {Sakakibara}\ \emph {et~al.}(2023)\citenamefont
  {Sakakibara}, \citenamefont {Kitamine}, \citenamefont {Ochi},\ and\
  \citenamefont {Kuroki}}]{sakakibara2023possible}%
  \BibitemOpen
  \bibfield  {author} {\bibinfo {author} {\bibfnamefont {H.}~\bibnamefont
  {Sakakibara}}, \bibinfo {author} {\bibfnamefont {N.}~\bibnamefont
  {Kitamine}}, \bibinfo {author} {\bibfnamefont {M.}~\bibnamefont {Ochi}}, \
  and\ \bibinfo {author} {\bibfnamefont {K.}~\bibnamefont {Kuroki}},\
  }\href@noop {} {\bibfield  {journal} {\bibinfo  {journal} {arXiv preprint
  arXiv:2306.06039}\ } (\bibinfo {year} {2023})}\BibitemShut {NoStop}%
\bibitem [{\citenamefont {Gu}\ \emph {et~al.}(2023)\citenamefont {Gu},
  \citenamefont {Le}, \citenamefont {Yang}, \citenamefont {Wu},\ and\
  \citenamefont {Hu}}]{gu2023effective}%
  \BibitemOpen
  \bibfield  {author} {\bibinfo {author} {\bibfnamefont {Y.}~\bibnamefont
  {Gu}}, \bibinfo {author} {\bibfnamefont {C.}~\bibnamefont {Le}}, \bibinfo
  {author} {\bibfnamefont {Z.}~\bibnamefont {Yang}}, \bibinfo {author}
  {\bibfnamefont {X.}~\bibnamefont {Wu}}, \ and\ \bibinfo {author}
  {\bibfnamefont {J.}~\bibnamefont {Hu}},\ }\href@noop {} {\bibfield  {journal}
  {\bibinfo  {journal} {arXiv preprint arXiv:2306.07275}\ } (\bibinfo {year}
  {2023})}\BibitemShut {NoStop}%
\bibitem [{\citenamefont {Shen}\ \emph {et~al.}(2023)\citenamefont {Shen},
  \citenamefont {Qin},\ and\ \citenamefont {Zhang}}]{shen2023effective}%
  \BibitemOpen
  \bibfield  {author} {\bibinfo {author} {\bibfnamefont {Y.}~\bibnamefont
  {Shen}}, \bibinfo {author} {\bibfnamefont {M.}~\bibnamefont {Qin}}, \ and\
  \bibinfo {author} {\bibfnamefont {G.-M.}\ \bibnamefont {Zhang}},\ }\href@noop
  {} {\bibfield  {journal} {\bibinfo  {journal} {arXiv preprint
  arXiv:2306.07837}\ } (\bibinfo {year} {2023})}\BibitemShut {NoStop}%
\bibitem [{\citenamefont {W{\'u}}\ \emph {et~al.}(2023)\citenamefont {W{\'u}},
  \citenamefont {Luo}, \citenamefont {Yao},\ and\ \citenamefont
  {Wang}}]{wu2023charge}%
  \BibitemOpen
  \bibfield  {author} {\bibinfo {author} {\bibfnamefont {W.}~\bibnamefont
  {W{\'u}}}, \bibinfo {author} {\bibfnamefont {Z.}~\bibnamefont {Luo}},
  \bibinfo {author} {\bibfnamefont {D.-X.}\ \bibnamefont {Yao}}, \ and\
  \bibinfo {author} {\bibfnamefont {M.}~\bibnamefont {Wang}},\ }\href@noop {}
  {\bibfield  {journal} {\bibinfo  {journal} {arXiv preprint arXiv:2307.05662}\
  } (\bibinfo {year} {2023})}\BibitemShut {NoStop}%
\bibitem [{\citenamefont {Christiansson}\ \emph {et~al.}(2023)\citenamefont
  {Christiansson}, \citenamefont {Petocchi},\ and\ \citenamefont
  {Werner}}]{christiansson2023correlated}%
  \BibitemOpen
  \bibfield  {author} {\bibinfo {author} {\bibfnamefont {V.}~\bibnamefont
  {Christiansson}}, \bibinfo {author} {\bibfnamefont {F.}~\bibnamefont
  {Petocchi}}, \ and\ \bibinfo {author} {\bibfnamefont {P.}~\bibnamefont
  {Werner}},\ }\href@noop {} {\bibfield  {journal} {\bibinfo  {journal} {arXiv
  preprint arXiv:2306.07931}\ } (\bibinfo {year} {2023})}\BibitemShut {NoStop}%
\bibitem [{\citenamefont {Liu}\ \emph {et~al.}(2023{\natexlab{b}})\citenamefont
  {Liu}, \citenamefont {Mei}, \citenamefont {Ye}, \citenamefont {Chen},\ and\
  \citenamefont {Yang}}]{liu2023s}%
  \BibitemOpen
  \bibfield  {author} {\bibinfo {author} {\bibfnamefont {Y.-B.}\ \bibnamefont
  {Liu}}, \bibinfo {author} {\bibfnamefont {J.-W.}\ \bibnamefont {Mei}},
  \bibinfo {author} {\bibfnamefont {F.}~\bibnamefont {Ye}}, \bibinfo {author}
  {\bibfnamefont {W.-Q.}\ \bibnamefont {Chen}}, \ and\ \bibinfo {author}
  {\bibfnamefont {F.}~\bibnamefont {Yang}},\ }\href@noop {} {\bibfield
  {journal} {\bibinfo  {journal} {arXiv preprint arXiv:2307.10144}\ } (\bibinfo
  {year} {2023}{\natexlab{b}})}\BibitemShut {NoStop}%
\bibitem [{\citenamefont {Cao}\ and\ \citenamefont {Yang}(2023)}]{cao2023flat}%
  \BibitemOpen
  \bibfield  {author} {\bibinfo {author} {\bibfnamefont {Y.}~\bibnamefont
  {Cao}}\ and\ \bibinfo {author} {\bibfnamefont {Y.-f.}\ \bibnamefont {Yang}},\
  }\href@noop {} {\bibfield  {journal} {\bibinfo  {journal} {arXiv preprint
  arXiv:2307.06806}\ } (\bibinfo {year} {2023})}\BibitemShut {NoStop}%
\bibitem [{\citenamefont {Zhang}\ and\ \citenamefont
  {Vishwanath}(2020)}]{zhang2020type}%
  \BibitemOpen
  \bibfield  {author} {\bibinfo {author} {\bibfnamefont {Y.-H.}\ \bibnamefont
  {Zhang}}\ and\ \bibinfo {author} {\bibfnamefont {A.}~\bibnamefont
  {Vishwanath}},\ }\href@noop {} {\bibfield  {journal} {\bibinfo  {journal}
  {Physical Review Research}\ }\textbf {\bibinfo {volume} {2}},\ \bibinfo
  {pages} {023112} (\bibinfo {year} {2020})}\BibitemShut {NoStop}%
\bibitem [{\citenamefont {Zhang}\ and\ \citenamefont
  {Zhu}(2021)}]{zhang2021fractional}%
  \BibitemOpen
  \bibfield  {author} {\bibinfo {author} {\bibfnamefont {Y.-H.}\ \bibnamefont
  {Zhang}}\ and\ \bibinfo {author} {\bibfnamefont {Z.}~\bibnamefont {Zhu}},\
  }\href@noop {} {\bibfield  {journal} {\bibinfo  {journal} {Physical Review
  B}\ }\textbf {\bibinfo {volume} {103}},\ \bibinfo {pages} {115101} (\bibinfo
  {year} {2021})}\BibitemShut {NoStop}%
\bibitem [{\citenamefont {Zhang}\ and\ \citenamefont
  {Vishwanath}(2022)}]{zhang2022pair}%
  \BibitemOpen
  \bibfield  {author} {\bibinfo {author} {\bibfnamefont {Y.-H.}\ \bibnamefont
  {Zhang}}\ and\ \bibinfo {author} {\bibfnamefont {A.}~\bibnamefont
  {Vishwanath}},\ }\href@noop {} {\bibfield  {journal} {\bibinfo  {journal}
  {Physical Review B}\ }\textbf {\bibinfo {volume} {106}},\ \bibinfo {pages}
  {045103} (\bibinfo {year} {2022})}\BibitemShut {NoStop}%
\bibitem [{\citenamefont {Nakata}\ \emph {et~al.}(2017)\citenamefont {Nakata},
  \citenamefont {Ogura}, \citenamefont {Usui},\ and\ \citenamefont
  {Kuroki}}]{PhysRevB.95.214509}%
  \BibitemOpen
  \bibfield  {author} {\bibinfo {author} {\bibfnamefont {M.}~\bibnamefont
  {Nakata}}, \bibinfo {author} {\bibfnamefont {D.}~\bibnamefont {Ogura}},
  \bibinfo {author} {\bibfnamefont {H.}~\bibnamefont {Usui}}, \ and\ \bibinfo
  {author} {\bibfnamefont {K.}~\bibnamefont {Kuroki}},\ }\href {\doibase
  10.1103/PhysRevB.95.214509} {\bibfield  {journal} {\bibinfo  {journal} {Phys.
  Rev. B}\ }\textbf {\bibinfo {volume} {95}},\ \bibinfo {pages} {214509}
  (\bibinfo {year} {2017})}\BibitemShut {NoStop}%
\bibitem [{\citenamefont {Bohrdt}\ \emph {et~al.}(2022)\citenamefont {Bohrdt},
  \citenamefont {Homeier}, \citenamefont {Bloch}, \citenamefont {Demler},\ and\
  \citenamefont {Grusdt}}]{bohrdt2022strong}%
  \BibitemOpen
  \bibfield  {author} {\bibinfo {author} {\bibfnamefont {A.}~\bibnamefont
  {Bohrdt}}, \bibinfo {author} {\bibfnamefont {L.}~\bibnamefont {Homeier}},
  \bibinfo {author} {\bibfnamefont {I.}~\bibnamefont {Bloch}}, \bibinfo
  {author} {\bibfnamefont {E.}~\bibnamefont {Demler}}, \ and\ \bibinfo {author}
  {\bibfnamefont {F.}~\bibnamefont {Grusdt}},\ }\href@noop {} {\bibfield
  {journal} {\bibinfo  {journal} {Nature Physics}\ }\textbf {\bibinfo {volume}
  {18}},\ \bibinfo {pages} {651} (\bibinfo {year} {2022})}\BibitemShut
  {NoStop}%
\bibitem [{\citenamefont {Hirthe}\ \emph {et~al.}(2023)\citenamefont {Hirthe},
  \citenamefont {Chalopin}, \citenamefont {Bourgund}, \citenamefont
  {Bojovi{\'c}}, \citenamefont {Bohrdt}, \citenamefont {Demler}, \citenamefont
  {Grusdt}, \citenamefont {Bloch},\ and\ \citenamefont
  {Hilker}}]{hirthe2023magnetically}%
  \BibitemOpen
  \bibfield  {author} {\bibinfo {author} {\bibfnamefont {S.}~\bibnamefont
  {Hirthe}}, \bibinfo {author} {\bibfnamefont {T.}~\bibnamefont {Chalopin}},
  \bibinfo {author} {\bibfnamefont {D.}~\bibnamefont {Bourgund}}, \bibinfo
  {author} {\bibfnamefont {P.}~\bibnamefont {Bojovi{\'c}}}, \bibinfo {author}
  {\bibfnamefont {A.}~\bibnamefont {Bohrdt}}, \bibinfo {author} {\bibfnamefont
  {E.}~\bibnamefont {Demler}}, \bibinfo {author} {\bibfnamefont
  {F.}~\bibnamefont {Grusdt}}, \bibinfo {author} {\bibfnamefont
  {I.}~\bibnamefont {Bloch}}, \ and\ \bibinfo {author} {\bibfnamefont {T.~A.}\
  \bibnamefont {Hilker}},\ }\href@noop {} {\bibfield  {journal} {\bibinfo
  {journal} {Nature}\ }\textbf {\bibinfo {volume} {613}},\ \bibinfo {pages}
  {463} (\bibinfo {year} {2023})}\BibitemShut {NoStop}%
\bibitem [{\citenamefont {Lee}\ \emph {et~al.}(2006)\citenamefont {Lee},
  \citenamefont {Nagaosa},\ and\ \citenamefont {Wen}}]{lee2006doping}%
  \BibitemOpen
  \bibfield  {author} {\bibinfo {author} {\bibfnamefont {P.~A.}\ \bibnamefont
  {Lee}}, \bibinfo {author} {\bibfnamefont {N.}~\bibnamefont {Nagaosa}}, \ and\
  \bibinfo {author} {\bibfnamefont {X.-G.}\ \bibnamefont {Wen}},\ }\href@noop
  {} {\bibfield  {journal} {\bibinfo  {journal} {Reviews of modern physics}\
  }\textbf {\bibinfo {volume} {78}},\ \bibinfo {pages} {17} (\bibinfo {year}
  {2006})}\BibitemShut {NoStop}%
\bibitem [{\citenamefont {Lu}\ \emph {et~al.}(2023)\citenamefont {Lu},
  \citenamefont {Pan}, \citenamefont {Yang},\ and\ \citenamefont
  {Wu}}]{lu2023interlayer}%
  \BibitemOpen
  \bibfield  {author} {\bibinfo {author} {\bibfnamefont {C.}~\bibnamefont
  {Lu}}, \bibinfo {author} {\bibfnamefont {Z.}~\bibnamefont {Pan}}, \bibinfo
  {author} {\bibfnamefont {F.}~\bibnamefont {Yang}}, \ and\ \bibinfo {author}
  {\bibfnamefont {C.}~\bibnamefont {Wu}},\ }\href@noop {} {\enquote {\bibinfo
  {title} {Interlayer coupling driven high-temperature superconductivity in
  la$_3$ni$_2$o$_7$ under pressure},}\ } (\bibinfo {year} {2023}),\ \Eprint
  {http://arxiv.org/abs/2307.14965} {arXiv:2307.14965 [cond-mat.supr-con]}
  \BibitemShut {NoStop}%
\end{thebibliography}%


\onecolumngrid
\newpage
\clearpage

\setcounter{equation}{0}
\setcounter{figure}{0}
\setcounter{table}{0}
\setcounter{page}{1}

\maketitle 
\makeatletter
\renewcommand{\theequation}{S\arabic{equation}}
\renewcommand{\thefigure}{S\arabic{figure}}
\renewcommand{\thetable}{S\arabic{table}}

\begin{center}
\vspace{10pt}
\textbf{\large Supplemental Material for ``Type II t-J model and shared antiferromagnetic spin coupling from Hund's rule in superconducting La$_3$Ni$_2$O$_7$''}
\end{center} 
\begin{center} 
{Hanbit Oh and Ya-Hui Zhang$^{\ \textcolor{red}{*}}$}\\
\emph{William H. Miller III Department of Physics and Astronomy, \\
Johns Hopkins University, Baltimore, Maryland, 21218, USA}
\vspace{5pt}
\end{center}


\section{One-orbital t-J model and slave-boson theory}
We start from the one-orbital Hamiltonian, 
\begin{align}
    H &=
-t_{\parallel}^{x} \sum_{l, \sigma}\sum_{ \langle i,j \rangle}
P\left(
d_{i;1;l,\sigma }^{\dagger}d_{1;l;\sigma }
\right) P+H.c. \notag\\
& +J_{\parallel}^{x}\sum_{l}\sum_{ \langle  i, j \rangle}\vec{S}_{i;l;1}
\cdot \vec{S}_{i;l;1}
+  J_{\perp}^{z}\sum_{i}
\vec{S}_{i;t;1} \cdot \vec{S}_{i;b;1},
\end{align}
and perform the mean field theory employing the slave boson representation, $d^{\dagger}_{i;l,1,\sigma}= f^{\dagger}_{i;l;\sigma}b_{i;l}$.
Assuming $\langle b_i \rangle =\sqrt{x}$, after the mean-field decoupling, the mean-field Hamiltonian is given by,
\begin{eqnarray}
H^{MF}_{SB}&=&
-t_{\parallel}
\sum_{l,\sigma,\langle i,j\rangle } \left( f_{i;l;\sigma}^{\dagger}
f_{j;l;\sigma}
+h.c.
\right)
 -t_{\perp} \sum_{\sigma, i}
\left(
f^{\dagger}_{i;t;\sigma}
f_{i;b;\sigma}
+h.c.
\right) 
\\
 &&
  +D_{\parallel}
\sum_{l, \langle i, j \rangle}
\left( 
s_{ij}(f^{\dagger}_{i;l;1;\uparrow}
f^{\dagger}_{j;l;1;\downarrow}
-f^{\dagger}_{i;l;1;\downarrow}
f^{\dagger}_{j;l;1;\uparrow})
+h.c.
\right) \nonumber \\
&&
  +D_{\perp}
\sum_{i}
\left( 
f^{\dagger}_{i;t;\uparrow}
f^{\dagger}_{i;b;\downarrow}
-f^{\dagger}_{i;t;\downarrow}
f^{\dagger}_{i;b;\uparrow}
+h.c.
\right), \nonumber
\label{eq:S_sb}
\end{eqnarray}
with the coefficients,
\begin{eqnarray*}
    t_{\parallel} &=& x t_{\parallel}^{x}+ \frac{3}{8}J_{\parallel}^{x}\chi_{\parallel}
    ,\quad
    t_{\perp} =\frac{3}{8}
    J_{\perp}^{z} \chi_{\perp},\\
    D^{\parallel}&=&
    \frac{3}{8} 
    J^{x}_{\parallel}
    \Delta_{\parallel}^{d},
    \quad
    D^{\perp} = \frac{3}{8} J^{z}_{\perp}\Delta_{\perp}^{s}.
\end{eqnarray*}
There are 4 mean field order parameters,
   \begin{eqnarray}
        \chi_{\parallel}&= &  \sum_{\sigma} \langle 
  f^{\dagger}_{j;l;\sigma}
 f_{i;l;\sigma}
    \rangle  
,\quad
 \chi_{\perp}
    =
     \sum_{\sigma} \langle 
    f^{\dagger}_{i;t;\sigma}
    f_{i;b;\sigma}
    \rangle
   ,\\
   \Delta_{\parallel} &=&
    \langle 
    s^{ij}(f_{i;l;\uparrow}
   f_{j;l;\downarrow}
    -f_{i;l;\downarrow}
    f _{j;l;\uparrow})
    \rangle 
    ,\quad
   \Delta_{\perp}=
    \langle 
    f_{i;t;\uparrow}
    f_{j;b;\downarrow}
    -f_{i;t;\downarrow}
    f _{j;b;\uparrow}
    \rangle. 
\end{eqnarray}
Moreover, the chemical potential should be fixed for conserving the particle number, $n=\sum_{k,l} \langle f^{\dagger}_{k;l;\sigma}
f_{k;l;\sigma}
\rangle =1-x$. In Fig.\ref{fig:S1}, we plot $(\Delta_{\parallel}, \Delta_{\perp})$ upon doping with a fraction $x$ of holes. 

% Figure environment removed



\section{Type II t-J model and Three-fermion parton theory}
We start from the type II t-J model introduced in Eq.\textcolor{red}{4}. Considering the large $\Delta$ limit, the singlon is formed by only $d_{2}$ orbital, thus the Hilbert space is restricted into
$P_{0} = P- \ket{1,\uparrow}
\bra{1,\uparrow}-
\ket{1,\downarrow}
\bra{1,\downarrow}
$. 
In this Hilbert space, electron operators of $d_2$ orbital itself become zero, thus the kinetic Hamiltonian can be expressed in terms of $d_{1}$ orbital, 
\begin{eqnarray}
       H&=&-t^x_\parallel 
    \sum_{l,\sigma,\langle i,j \rangle} (P_{0}d^\dagger_{i;l;1;\sigma}d_{j;l;1;\sigma} P_{0}+h.c.)\\ 
&&+   J^x_\parallel  \sum_{l,\langle i,j \rangle} \vec s_{i;l;1}\cdot \vec s_{j;l;1}
+J^{dd}_\parallel  \sum_{l, \langle i,j \rangle}\vec S_{i;l}\cdot \vec S_{j;l}
+ J^{sd}_\parallel  \sum_{l,\langle i,j \rangle} (\vec s_{i;l;1}\cdot \vec S_{j;l} +\cdot \vec S_{i;l}\cdot \vec s_{j;l;1}) \notag \\ 
&&+J^z_\perp \sum_{i} \vec s_{i;t;2}\cdot \vec s_{i;b;2} +J^{dd}_\perp \sum_i \vec S_{i;t}\cdot \vec S_{i;b} 
+J^{sd}_\perp 
 \sum_i (\vec s_{i;t;2}\cdot \vec S_{i;b}+\vec S_{i;t}\cdot \vec s_{i;b;2})
\nonumber\end{eqnarray}

Here we use the following three-fermion decomposition,
\begin{eqnarray}
d_{i;l;1;\sigma}^{\dagger}&=&
(\psi_{i;l;1;\sigma}^{\dagger}
\psi_{i;l;2;\sigma}^{\dagger})
f_{i;l;2;\sigma}
+\frac{1}{2} (\psi_{i;l;1\uparrow}^{\dagger}\psi_{i;2;l;\downarrow}^{\dagger}
+\psi_{i;1;l;\downarrow}^{\dagger}\psi_{i;2;l;\uparrow}^{\dagger})
f_{i;l; 2;\bar \sigma}
,\\
d_{j;l;1;\sigma}&=&f^\dagger_{j;l;2;\sigma}(\psi_{j;l;2;\sigma}\psi_{j;l;1;\sigma})+\frac{1}{2} f^\dagger_{j;l; 2;\bar \sigma}(\psi_{j;l;2;\downarrow}\psi_{j;l;1;\uparrow}+\psi_{j;l;2;\uparrow}\psi_{j;l;1;\downarrow}).
\label{eq:three_fermion_parton_appendix}
\end{eqnarray}
Employing the standard decoupling principle, the mean-field Hamiltonian is given by 

\begin{eqnarray}
H^{MF}_{TF}&=&
-t_{f;2}
\sum_{l,\sigma,\langle i,j\rangle } \left( f_{i;l;2;\sigma}^{\dagger}
f_{j;l;2;\sigma}
+h.c.
\right)
-
\sum_{a,c=1,2}
t_{\psi;ac}
\sum_{l,\sigma, \langle i,j \rangle}
\left(\psi_{i;l;a;\sigma}^{\dagger}
\psi_{j;l;c;\sigma}
+h.c.\right)
\\
&& -\sum_{a=1,2}
C_{a}^{0}
\sum_{l,\sigma, i}
\left( 
f^{\dagger}_{i;l;2;\sigma}
\psi_{i;l;a;\sigma}
+\psi^{\dagger}_{i;l;a;\sigma}
f_{i;l;2;\sigma}
+h.c.
\right)\nonumber \\
& & -t_{f}^{\perp} \sum_{\sigma, i}
\left(
f^{\dagger}_{i;t;2;\sigma}
f_{i;b;2;\sigma}
+h.c.
\right)
 -\sum_{a,c=1,2}
t_{\psi;ac}^{\perp}
\sum_{\sigma, i}
\left(
\psi^{\dagger}_{i;t;a;\sigma}
\psi_{i;b;c;\sigma}
+h.c.
\right)\nonumber\\
& & -\sum_{a=1,2}
C_{a}^{\perp} \sum_{\sigma, i}
\left(
f^{\dagger}_{i;t;2;\sigma}
\psi_{i;b;a;\sigma}
+
\psi^{\dagger}_{i;t;a;\sigma}
f_{i;b;2;\sigma}
+h.c.
\right)\nonumber\\
 &&
  +D_{\psi;1}
\sum_{l, \langle i, j \rangle}
\left( s_{ij}(
\psi^{\dagger}_{i;l;1;\uparrow}
\psi^{\dagger}_{j;l;1;\downarrow}
-\psi^{\dagger}_{i;l;1;\downarrow}
\psi^{\dagger}_{j;l;1;\uparrow})
+h.c.
\right)\nonumber \\
&&
  +D_{\psi;1}^{\perp}
\sum_{i}
\left( 
\psi^{\dagger}_{i;t;1;\uparrow}
\psi^{\dagger}_{i;b;1;\downarrow}
-\psi^{\dagger}_{i;t;1;\downarrow}
\psi^{\dagger}_{i;b;1;\uparrow}
+h.c.
\right) \nonumber
\\
&& -\mu_{f} \sum_{l,\sigma, i}
f^{\dagger}_{i;l;a;\sigma}
f_{i;l;a;\sigma}
 -\sum_{a=1,2} \mu_{a} \sum_{l,\sigma, i}
 \psi^{\dagger}_{i;l;a;\sigma}
 \psi_{i;l;a;\sigma},\nonumber\label{eq:S_tf}
\end{eqnarray}
with the coefficients,
\begin{eqnarray*}
    t_{\psi;11}&=&
    t_{\parallel}^{x}
    \left[
    \frac{3}{8}
    \chi_{f}\chi_{\psi;22}
    -\frac{9}{16}
    \Phi_{2}^{0}
    \Phi_{2}^{0}
    \right]
+ \frac{3}{8}J^{dd}_{\parallel}\chi_{\psi;11}, 
    \\
    t_{\psi;22}&=&
     t_{\parallel}^{x}
    \left[
    \frac{3}{8}
    \chi_{f}\chi_{\psi;11}
    \right]
+ \frac{3}{8}J^{dd}_{\parallel}\chi_{\psi;22}, 
\quad
    t_{f;2} =t_{\parallel}^{x}
    \left[
    \frac{3}{8}\left( 
    \chi_{\psi;11}\chi_{\psi;22}
    \right)
    \right]
    ,\quad
    C_{2}^{0} =
 t_{\parallel}^{x}
    \left[
    -\frac{9}{8}\Phi_{2}^{0}\chi_{\psi;11}
    \right], \\
    t^{\perp}_{\psi;11}&= &\frac{3}{8}J^{dd}_{\perp}\chi_{\psi;11},
    \quad 
    t^{\perp}_{\psi;22}=\frac{3}{8}J^{dd}_{\perp}\chi_{\psi;22}
    ,\quad 
t_{f}^{\perp} =
\frac{3}{8}J_{\perp}^{z} \chi_{f}^{\perp}
    ,\quad 
C_{2}^{\perp} =
\frac{3}{8}J_{\perp}^{sd} 
\Phi_{2}^{\perp},
\end{eqnarray*}
and
\begin{eqnarray*}
    D_{\psi;1} 
    &=&  \frac{3}{8}J^{dd}_{\parallel} \Delta_{\psi;1}
    ,\quad
    D_{\psi;1}^{\perp}
  = \frac{3}{8}J^{dd}_{\perp} \Delta_{\psi;1}^{\perp}.
\end{eqnarray*}

% Figure environment removed

 

There are 10 mean-field order parameters in total for constructing a mean-field Hamiltonian, 
    \begin{eqnarray}
        \chi_{\psi;aa}&= &  \sum_{\sigma} \langle 
    \psi^{\dagger}_{j;l;a;\sigma}
    \psi_{i;l;a;\sigma}
    \rangle  
    ,\quad
    \chi_{f} =\sum_{\sigma} \langle 
    f^{\dagger}_{j;l;2;\sigma}
    f_{i;l;2;\sigma}
    \rangle  
    ,\quad
    \Phi_{2}^{0} =\sum_{\sigma} \langle 
    \psi^{\dagger}_{i;l;2;\sigma}
    f_{i;l;2;\sigma}
    \rangle,
\\
 \chi^{\perp}_{\psi;aa} 
    &=& 
     \sum_{\sigma} \langle 
    \psi^{\dagger}_{i;t;a;\sigma}
    \psi_{i;b;a;\sigma}
    \rangle
    ,\quad 
        \chi_{f}^{\perp}= \sum_{\sigma} \langle 
    f^{\dagger}_{i;t;2;\sigma}
    f_{i;b;2;\sigma}
    \rangle  
,\quad
    \Phi_{2}^{\perp}
    =
    \sum_{\sigma} \langle 
    \psi^{\dagger}_{i;t;2;\sigma}
    f_{i;b;2;\sigma}
    \rangle,\\
   \Delta_{\psi;1} &=&
    \langle 
    s^{ij}(\psi_{i;l;1;\uparrow}
    \psi_{j;l;1;\downarrow}
    -\psi_{i;l;1;\downarrow}
    \psi _{j;l;1;\uparrow})
    \rangle 
    ,\quad
   \Delta_{\psi;1}^{\perp}=
    \langle 
    \psi_{i;t;1;\uparrow}
    \psi_{j;b;1;\downarrow}
    -\psi_{i;t;1;\downarrow}
    \psi _{j;b;1;\uparrow}
    \rangle. 
\end{eqnarray}
Note that $t_{\psi;12}=C^{0}_{1}=C^{\perp}_{1}=\chi_{\psi;12}=\Phi_{1}^{0}=\Phi_{1}^{\perp} =0$, and $J_{sd}^\parallel=\frac{1}{2}J^x_\parallel$, $J_{sd}^\perp=\frac{1}{2} J^z_\perp$, $J_{dd}^\parallel=\frac{1}{4} J^x_\parallel$, $J_{dd}^\perp=\frac{1}{4}J^z_\perp$.  
Together with the order parameters, one should impose the constraints on the number of fermion $n_{\psi;1}=n_{\psi;1}=1-x$, and $n_{f}=x$, where the particle numbers are defined as, 
\begin{eqnarray*}
n_{\psi;a}=\sum_{k,l} \langle \psi^{\dagger}_{k;l;a;\sigma}
\psi_{k;l;a;\sigma}
\rangle ,\quad
n_{f}=\sum_{k,l} \langle f^{\dagger}_{k;l;2;\sigma}
f_{k;l;2;\sigma}\rangle.    
\end{eqnarray*}
In Fig.\ref{fig:S2}, we plot $(\Delta^{\parallel}_{\psi;1}, \Delta^{\perp}_{\psi;1},\Phi^{0}_{2})$ upon doping with a fraction $x$ of holes. 
Moreover in Fig.\ref{fig:S3}, we illustrate the physical meaning of the three fermions in our parton construction. With a non-zero $\Phi=\Phi^0_2$, the  $\psi_{1}$ orbital can be identified as the $d_{1}$ orbital from Eq.~\ref{eq:three_fermion_parton_appendix}. At the same time, $\psi_{2}$,$f$ together form a localized $d_{2}$ orbital with total density $n_{i;2}+n_{i;f}=1$ per site. In our bilayer model they form a gapped rung-singlet phase.



% Figure environment removed


\end{document}
