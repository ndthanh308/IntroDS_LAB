\documentclass[graphics,floatfix, footinbib,tightenlines,nobibnotes, aps, prb, twocolumn]{revtex4-1}

\usepackage{amsmath,amssymb,wasysym}
\usepackage{graphicx}% Include figure files
\usepackage{dcolumn}% Align table columns on decimal point
\usepackage{bm}% bold math
\usepackage{braket}
\usepackage{subcaption}
\usepackage{verbatim}
\usepackage{float}
\usepackage{bbold}
\usepackage{color}
\usepackage{xcolor}
\usepackage{relsize}
\usepackage{amsthm}
\usepackage{enumerate}
\usepackage{soul,xcolor}
%\usepackage[utf8]{inputenc}
\usepackage[T1]{fontenc}
\usepackage{adjustbox}
\usepackage[colorlinks=true ,urlcolor=blue,urlbordercolor={0 1 1}]{hyperref}
%\newcommand{\<}{\langle}
\newcommand{\e}{\varepsilon}
\newcommand{\up}{\uparrow}
\newcommand{\down}{\downarrow}
%\renewcommand{\>}{\rangle}\
\renewcommand{\(}{\left(}
\renewcommand{\)}{\right)}
\renewcommand{\[}{\left[}
\renewcommand{\]}{\right]}
\renewcommand{\v}[1]{\mathbf{#1}} % \v -> vector (bf)
\newcommand{\dslash}{d \hspace{-0.8ex}\rule[1.2ex]{0.8ex}{.1ex}}
\newcommand{\bs}[1]{\boldsymbol{#1}}
\renewcommand{\d}{\partial}
\newcommand{\del}{\nabla}
\renewcommand{\div}{\nabla\cdot}
\newcommand{\curl}{\nabla\times}
\newcommand{\eps}{\epsilon}
%special commands for this paper
\newcommand{\tv}{\tau_v}
\newcommand{\tvb}{\tau_{\bar{v}}}
\newcommand{\T}{\mathcal{T}}
\newcommand{\CT}{\mathcal{CT}}
\newcommand{\tCT}{\tilde{\mathcal{CT}}}
\newcommand{\be}{\begin{equation}}
\newcommand{\ba}{\begin{align}}
\newcommand{\ee}{\end{equation}}
\newcommand{\bea}{\begin{eqnarray}}
\newcommand{\eea}{\end{eqnarray}}
\newcommand{\beq}{\begin{equation}}
\newcommand{\eeq}{\end{equation}}
\newcommand{\beqn}{\begin{eqnarray}}
\newcommand{\eeqn}{\end{eqnarray}}
\newcommand{\HH}{{\cal H}}
\newcommand{\RR}{{\mathcal R}}
\newcommand{\CC}{\mathcal{C}}
\newcommand{\p}{\partial}
\newcommand{\s}{\sigma}
\newcommand{\lam}{\lambda}
\newcommand{\la}{\langle}
\newcommand{\ra}{\rangle}
\newcommand{\lb}{\left[}
\newcommand{\rb}{\right]}
\newcommand{\lp}{\left(}
\newcommand{\rp}{\right)}
\newcommand{\Tr}{{\rm \, Tr\,}}
\newcommand{\Z}{\mathbb{Z}}
\newcommand{\LL}{\mathcal{L}}
\newcommand {\ep}{\epsilon}
\newcommand{\TexBlue}[1]{ {\color{blue} #1}}
\newcommand{\ashvin}[1]{ {\color{red} #1}}
\newcommand{\yahui}[1]{ {\color{blue} #1}}

%\newcommand{\hb}[1]{{\color{violet}{{#1}}}} 


%\renewcommand{\vec}[1]{{\bf #1}}
\renewcommand{\hat}[1]{{\widehat #1}}
\renewcommand{\Re}{{\rm \, Re\,}}
\renewcommand{\Im}{{\rm \, Im\,}}
\def\nn{\nonumber\\}
\newcommand{\red}{}
\newcommand{\blue}{}
\newcommand{\redhighlight}{\color{red}}


\newcommand*{\yhz}[1]{\textcolor{red}{#1}}



\usepackage{bm}

\newcommand{\hb}[1]{{\color{red}{{#1}}}} % Hanbit's edit


\newcommand{\fix}[1]{{\color{blue}{{#1}}}} % needs to be fixed

\usepackage{array}
\newcolumntype{L}[1]{>{\raggedright\arraybackslash}p{#1}}
\newcolumntype{C}[1]{>{\centering\arraybackslash}p{#1}}
\newcolumntype{R}[1]{>{\raggedleft\arraybackslash}p{#1}}
\usepackage{multirow}






%\usepackage{pstricks, pst-node, pst-plot}
\allowdisplaybreaks

%\newcommand{\Z}{\mathbb{Z}}
\newcommand{\id}{\mathds{1}}
\newcommand{\x}{\mathbb{\times}}
\newcommand{\lv}{\langle}
\newcommand{\rv}{\rangle}
\newcommand{\R}{\mathbb{R}}
\newcommand{\Q}{\mathbb{Q}}
\newcommand{\C}{\mathbb{C}}
\newcommand{\Aut}{\textrm{Aut}}
\newcommand{\normal}{\trianglelefteq}
%\newcommand{\Tr}{\textrm{Tr}}
\newcommand*\diff{\mathop{}\!\mathrm{d}}
\newcommand*\Diff[1]{\mathop{}\!\mathrm{d^#1}}
\newcommand{\sgn}{\textrm{sgn}}




\begin{document}
\widetext

\title{Type II t-J model and shared antiferromagnetic spin coupling from Hund's rule in  superconducting La$_3$Ni$_2$O$_7$ }

\author{Hanbit  Oh$^1$}
\author{Ya-Hui Zhang$^2$}
\email{yzhan566@jhu.edu}
\affiliation{$^1$Department of Physics, KAIST, Daejeon, 34126, Republic of Korea}
\affiliation{$^2$William H. Miller III Department of Physics and Astronomy, Johns Hopkins University, Baltimore, Maryland, 21218, USA}

\date{\today}

\begin{abstract}
Recently a 80 K superconductor was observed in La$_3$Ni$_2$O$_7$ under high pressure. Density function theory (DFT) calculations identify $d_{x^2-y^2}$ and $d_{z^2}$ as two active orbitals and a bilayer square lattice structure. The averange valence of Ni is $d^{8-x}$ with $x=0.5$ per site.  Naively one may expect a description in terms of a two-orbital t-J model. However, there should be significant inter-orbital repulsion $U'$ and Hund's coupling $J_H$ larger than the bare value of $t$ and $J$. Especially the Hund's coupling can share the inter-layer super-exchange $J_\perp$ of $d_{z^2}$ to $d_{x^2-y^2}$, an effect beyond any perturbative and mean field treatment. In the limit that $d_{z^2}$ is Mott localized, we integrate it out and deal with a bialyer t-J model for $d_{x^2-y^2}$ only. We find strong inter-layer pairing due to the transmitted $J_\perp$ which can survive to $50\%$ hole doping relevant to the experiment.  In real system we expect that $d_{z^2}$ orbital will also be slightly hole doped and can not be simply ignored.  To deal with this situation, we take the $J_H\rightarrow +\infty$ limit and propose a type II t-J model with four singlon ($d^7$) states and three spin-triplet doublon ($d^8$) states. Through a  parton mean field treatment of the constrained Hilbert space, we derive the bilayer one-orbital t-J model for an emergent '$d_{x^2-y^2}$' orbital with significant $J_\perp$, justifying our phenomenological treatment.  The type II t-J model can also describe the regime where the $d_{z^2}$ orbital is also slightly hole doped through tuning an orbital energy splitting $\Delta$. From our calculation the pairing strength decreases with the hole doping $x$ and $x=0.5$ is likely larger than the optimal doping. We propose future experiments to electron dope the system to further enhance $T_c$.
 \end{abstract}

\maketitle

\textbf{Introduction} High temperature superconductivity is arguably one of the most important problems in condensed matter physics. High Tc cuprates have been intensively studied for decades. Although there is still no well-established theory for the pairing mechanism , it is widely accepted that an one-orbital Hubbard or t-J model based on Cu $d_{x^2-y^2}$ orbital forms the basic description of the essential physics\cite{lee2006doping}. An interesting question is whether we can find a high Tc superconductor beyond the familiar one-orbital t-J model. 

Recently a superconductor with $T_c=80$K was found in La$_3$Ni$_2$O$_7$ under high pressure\cite{sun2023signatures}, following a previous discovery of a $T_c\approx 20$ K superconductor in nickelate Nd$_{1-x}$ Sr$_x$NiO$_2$ at ambient pressure\cite{li2019superconductivity}. The discovery has triggered many  experimental\cite{liu2023electronic,hou2023emergence} and theoretical\cite{luo2023bilayer,zhang2023electronic,yang2023possible,sakakibara2023possible,gu2023effective,shen2023effective,wu2023charge,christiansson2023correlated,liu2023s,hou2023emergence,liu2023electronic} studies.
The average valence of Ni is in $d^{8-x}$ with $x=0.5$\cite{sun2023signatures}. Density functional theory (DFT) calculations identify a bilayer square lattice structure with active $d_{x^2-y^2}$ and $d_{z^2}$  orbitals, which we label as $d_1$ and $d_2$ in the following. The density (summed over spin) per site is estimated to be $n_1\approx 1-x=0.5$ and $n_2 \approx 1$, so that the $d_{z^2}$ orbital is close to a Mott localized state. Due to a large inter-layer hybridization of the $d_{z^2}$ orbital, we expect that it just forms a rung singlet when $n_2=1$.    The $d_{z^2}$ orbital is close to Mott localization and has a small intra-layer hopping, thus we do not expect a strong superconductivity from it.  Then one may expect that superconductivity origins from the $d_{x^2-y^2}$ orbital. But the $d_{x^2-y^2}$ orbital is at hole doping level of $50\%$. According to the phase diagram of cuprates, it should be in the overdoped Fermi liquid.  A major goal of this paper is to find a mechanism for the material to superconduct at such a large hole doping.

One important ingredient we identify is the Hund's coupling between the $d_{z^2}$ and the $d_{x^2-y^2}$ orbital.  We know that there should be a large inter-layer antiferromagnetic coupling $J^z_\perp$ for the $d_{z^2}$ orbital, while the inter-layer hopping of $d_{x^2-y^2}$ should be negligible. However, the strong Hund's coupling $J_H$ align the spin of the two orbitals at each site, then the $J_\perp$ of the $d_{z^2}$ orbital is shared to $d_{x^2-y^2}$ orbital.  Therefore, when $n_2=1$,  we can ignore the insulating $d_{z^2}$ orbital and phenomenologically consider a bilayer one-orbital t-J model for $d_{x^2-y^2}$, but with a large inter-layer spin coupling $J_\perp$ despite the absence of inter-layer hopping $t_\perp$.  We perform a slave boson mean field treatment of the model. We find that a large $J_\perp$ disfavors the familiar $d_{x^2-y^2}$ pairing at the $J_\perp=0$ limit and the system forms a s wave superconductor with dominant inter-layer pairing in the large $J_\perp$ regime. The pairing strength decreases with the hole doping level $x$. But with a sufficiently large $J_\perp$, the pairing survives at $x=0.5$, which may explain the superconductor at this hole doping level.


The above treatment of `integrating' out the $d_{z^2}$ orbital is not very rigorous. Also, in the real system the $d_{z^2}$ orbital may also be slightly hole doped. To be more rigorous and to enable the doping of the $d_{z^2}$ orbital, we propose a bilayer type II t-J model to describe the low energy physics. The model is a generalization of a model proposed one of us before\cite{zhang2020type}. Basically we take the large $J_H$ limit and restrict to a Hilbert space with four singlon ($d^7$) states and three spin-triplet doublon ($d^8$) states.  Inter-orbital $J_H$ disappears in the model with the cost of non-trivial constraint.  The model has two important parameters: the total hole doping level $x$ and energy splitting $\Delta$ between the two orbitals to tune the relative doping of the two orbitals.  In the large $\Delta$ limit, we have $n_2=1$ and $d_{z^2}$ is Mott localized and forms a rung singlet. We propose a parton mean field theory to deal with the type II t-J model. In the simple large $\Delta$ limit, in mean field level we reach a bilayer one-orbital t-J model for an emergent `$d_{x^2-y^2}$' orbital. In this model, we can automatically get large a $J_\perp/t$, justifying our phenomenological treatment.  Therefore strong inter-layer pairing is possible in this model even at $x=0.5$.  In future we hope to extend the analysis to smaller $\Delta$ with $n_2=1-\delta$, so the $d_{z^2}$ orbital also has a small hole doping $\delta$ and get mobile.  On the experimental side, we propose to electron dope the system to reduce the total hole doping $x$, which may further enhance the $T_c$.



\textbf{Two-orbital model} We  start from a two-orbital t-J model on a bilayer square lattice(see Fig.~\ref{fig:1}:


\begin{align}
    H&=-t^x_\parallel \sum_{l ,\sigma} \sum_{\langle i,j \rangle} (P d^\dagger_{i;l;1;\sigma}d_{j;l;1;\sigma} P+H.c.)\notag \\ 
    &~~~-t^z_\parallel\sum_{l,\sigma }  \sum_{\langle i,j \rangle} (P d^\dagger_{i;l;2;\sigma}d_{j;l;2;\sigma} P+H.c.) \notag \\ 
    &~~~-t^{xz}_\parallel \sum_{l,\sigma }\sum_{\langle ij \rangle} ((-1)^{s_{ij}} P d^\dagger_{i;l;1;\sigma }d_{j;l;2;\sigma } P+H.c.) \notag \\ 
    &~~~-t_\perp^z \sum_i (P d^\dagger_{i;t;2;\sigma}d_{i;b;2;\sigma} P+H.c.)  \notag \\
   &~~~ +\Delta \sum_i (n_{i;1}-n_{i;2}) \notag \\
& ~~~+   J^x_\parallel \sum_l \sum_{\langle ij \rangle} \vec S_{i;l;1}\cdot \vec S_{i;l;1}+J^z_\perp \sum_{i} \vec S_{i;t;2}\cdot \vec S_{i;b;2} \notag \\
&~~~+U'\sum_i n_{i;1}n_{i;2}-2J_H\sum_i (\vec S_{i;1}\cdot \vec S_{i;2}+\frac{1}{4}n_{i;1}n_{i;2}),  \label{e1}
\end{align}
where $P$ is the projection operator to remove the double occupancy of each orbital. Here, $l=t,b$ labels the layer index, and $\sigma=\uparrow,\downarrow$ is for the spin index. We dub $d_1, d_2$ for the $d_{x^2-y^2}$ and $d_{z^2}$ orbital respectively. Estimated hopping parameters are shown in Table.~\ref{table:1}. Here, we assume only the $d_{z^2}$ orbital has inter-layer hopping $t_\perp^{z}$ and thus only the $J_\perp^{z}\simeq \frac{t^{z}_{\perp}}{4U}$ exists.  
$s_{ij}=1$ for the $x$ bond and $s_{ij}=-1$ for the $y$ bond.     For simplicity, we only keep intra-layer $J^{x}_{\parallel}$ for the $d_{x^2-y^2}$ orbital and the inter-layer $J^{z}_{\perp}$ for the $d_{z^2}$ coupling. 
In principle there are inter-orbital spin coupling and orbital-orbital coupling, which we ignore in this work. $U'$ is inter-orbital repulsion and $J_H$ is the Hund's coupling.  $n_{i;a}$ is the density for orbital $a=1,2$. $\vec S_{i;a}$ is the spin operator for orbital $a=1,2$. For simplicity we ignore inter-orbital super-exchange. We also ignore the $n_i n_j$ term in the $J$ coupling.


In Fig.~\ref{fig:1}, we illustrate our bilayer two orbital model and the electron configurations of two Ni atoms. On average we have $n=2-x$ number of electron (summed over spin) per site with $x\approx 0.5$ in the experiment.

\begin{table}[b]
    \centering
    \begin{tabular}{c|ccccc}
    \hline
    \hline
        Parameter& $t_{\parallel}^{x}$ & $t_{\parallel}^{z}$ &  $t_{\parallel}^{xz}$ & $t_{\perp}^{x}$ & $t_{\perp}^{z}$   \\ \hline
        Value&0.485& 0.110 &  0.239 &-0.005& 0.635  \\ 
        \hline
        \hline
    \end{tabular}
    \caption{Tight binding parameters of the bilayer two orbital model, Eq.\ref{e1}, estimated by DFT\cite{luo2023bilayer}. Note that the value of $J^{x}_{\perp}$ is much smaller than $J^{z}_{\perp}$.}
    \label{table:1}
\end{table}


% Figure environment removed

\textbf{Bilayer one orbital t-J model}
 It is crucial to account for the considerable inter-orbital repulsion $U'$ and Hund's coupling $J_H$, which are larger than the bare values of $t$ and $J$. 
Notably, large Hund coupling can facilitate the sharing of inter-layer exchange interaction between $d_{1}$ and $d_{2}$.
This is due to the ferromagnetic Hund coupling which aligns the spin between different orbital at same site (see Fig.~\ref{fig:1}(a)).
In this restricted Hilbert space where the two orbital forms a spin-triplet at each site, only the symmetric part under orbital interchange can persist and we can  argue that $J^{x}_{\perp}=J^{z}_{\perp}$.
As a consequence, inter-layer $J_\perp$ coupling plays a significant role even for $d_{1}$ orbital, despite that there is no inter-layer hopping. 


Motivated by the above observation, we consider a simple bialyer one-orbital t-J model for $d_{x^2-y^2}$ orbital: 

\begin{eqnarray}
    H_{\mathrm{eff}} &=& 
-t_{\parallel}^{x} \sum_{l, \sigma}\sum_{ \langle i,j \rangle}
P\left(
d_{i;1;l,\sigma }^{\dagger}d_{1;l;\sigma }
\right) P+H.c. \nonumber\\
&& +J_{\parallel}^{x}\sum_{l}\sum_{ \langle  i, j \rangle}
[\vec{S}_{i;l;1}
\cdot \vec{S}_{i;l;1}-\frac{1}{4}n_{i;l;1}n_{j;l;1}] \nonumber
\\
&&+  J_{\perp}^{z}\sum_{i}
[\vec{S}_{i;t;1} \cdot \vec{S}_{i;b;1}-\frac{1}{4}n_{i;t;1}n_{i;b;1}]\\ \nonumber
\end{eqnarray}
Hereafter, the shorthand notations for parameters $t_{\parallel}=t^{x}_{\parallel},J_{\parallel}=J^{x}_{\parallel}$, and $J_{\perp}=J^{z}_{\perp}$ are utilized.


To address the interacting model conveniently, we employ the U(1) slave boson theory. This representation involves decomposing electron operators into fermionic spinons ($f$) and holons ($b$) as $d_{i;l;1;\sigma}^{\dagger} = f_{i;l;\sigma}^{\dagger} b_{i;l}$. The mean field Hamiltonian can be decoupled using the mean field order parameter ansatz of hoppings and pairings ${\chi_{ij},\Delta_{ij}}$, which are defined as follows,
\begin{align}
\Delta_{\parallel;i,j}^{l} &=&   
2s^{ij}\langle f_{i;l;\uparrow}f_{j;l;\downarrow}\rangle, 
\ 
\chi_{\parallel;ij,\sigma}^{l} =  2 \langle f_{i;l;s}^{\dagger}f_{j;l;s}\rangle, 
\\
\Delta_{\perp;i,j} &=& 2 \langle f_{i;t;\uparrow}f_{i;b,\downarrow}\rangle,
%=2 \langle f_{i;b;\uparrow}f_{i;t;\downarrow}\rangle, 
\ 
\chi_{\perp; i;\sigma}= 2\langle f_{i;t;\sigma}^{\dagger}f_{i;b;\sigma}\rangle,
\end{align}\vspace{-20pt}
\begin{align}
\chi^{b}_{ij} = \langle b_{i;l}b_{j;l}^{\dagger}\rangle
=x.
\end{align}
Here, we omitted the magnetic ordering, as there is no signature on this system.
By decoupling the effective Hamiltonian as a mean-field Hamiltonian (See Eq.\ref{eq:h_mf}), we derive the self-consistency equation and its solution.

In Fig.\ref{fig:bialyer_t_J_results}, we illustrate the order parameter dependence on the filling $x=1-n$ at $t_{\parallel}=1$ and $J_{\parallel}=1/2$. In this parameter range, two pairing order parameters dominate, namely, the intra-layer $d$-wave pairing and the inter-layer $s$-wave pairing, as described in our ansatz. Moreover, we emphasize that we reach the self-consistent ansatz from general initial ansatz.
% and decouple the effective Hamiltonian as a mean-field Hamiltonian (see Appendix). Then it is straightforward to obtain the self-consistency equation and solution.
% In Fig.2, the order parameter dependence on the filling $x=1-n$
% at $t_{\parallel}^{x}=1$, and $J_{\parallel}^{x}=1/2$ is illustrated. 
% In this range, the two pairing order parameters, intra-layer $d-$wave pairing , and the inter-layer $s-$wave pairing are dominant, as described in the ansatz. 
% We also stress that our ansataz is not biased, meaning it can even be obtained from a general pairing approach. 

Here are a few remarks regarding the results.
First, in the limit of small interlayer coupling, the model reproduces the well-known behaviors of the t-J model, resulting in $d$-wave superconductivity within each layer. 
As the strength of $J_{\perp}$ is gradually increased, initially, there is no change in both order parameters. However, once it surpasses a critical value, the intra-layer $d$-wave pairing gets suppressed, while the inter-layer $s$-wave pairing is enhanced, as illustrated in Fig.\ref{fig:bialyer_t_J_results}. 
Second, the two types of pairings compete with each other, resulting in a first-order phase transition at a critical value $x=x_{c}$. Thus, there is no coexistence region where both phases simultaneously exist.
Interestingly, we also found that when $J_{\perp}$ reaches a significant magnitude, the value of $|\Delta_{\perp}|$ remains persistent across a wide region, even encompassing the high hole doping regime with $x\simeq 0.5$. 
Since the pairing in this case is $s$-wave type, the normal Fermi surface becomes fully gapped in all directions. This is in stark contrast to $d$-wave pairing, which gives rise to point nodes along the $\Gamma-M$ line, as clearly illustrated in Fig. \ref{fig:bialyer_t_J_results} (d).
Lastly, we have checked the robustness of the characteristic behaviors under layer asymmetry. We have explored a more generic case where two hopping constants $t_{\parallel}^{t}\neq t_{\parallel}^{b}$ are mismatched, and we have also incorporated next-nearest neighbor hoppings.


% Figure environment removed

\textbf{Type II t-J model} We have seen the importance of the Hund's coupling in sharing the super-exchange $J$ from one orbital to another orbital. In the simple limit that $n_2=1$ per site, the $d_{z^2}$ orbital is orbital-selective Mott localized and forms  rung-singlet due to large inter-layer coupling. Then we can integrate $d_2$ and deal with a one-orbital model and take the transmission of $J_\perp$ by hand.  However, this approach is not very rigorous and needs justification. Besides, in real system the $d_{z^2}$ orbital is likely to be slightly hole doped and $n_2<1$. Then $d_2$ orbital should be kept in the low energy model. In this case we need to deal with the two-orbital model in Eq.~\ref{e1}. However, $U'$ and $J_H$ are large and can not be treated in perturbation or mean field level. 

Here we take a non-perturbative approach. We first take $U',J_H$ to be large and project to a restricted Hilbert space.  This leads to a generalization of the type II t-J model proposed by one of us in Ref.\onlinecite{zhang2020type}. We only keep four singlon ($d^7$)  states and three spin-triplet doublon ($d^8$) states.    First, at each site $i$ the four singlon states can be labeled as $\ket{a \sigma}=d^\dagger_{a;\sigma}\ket{0}$ where $a=1,2$ and $\sigma=\uparrow, \downarrow$. Meanwhile, the three spin-triplet doublon states are: $\ket{-1}=d^\dagger_{1\downarrow}d^\dagger_{2\downarrow}\ket{0}$, $\ket{0}=\frac{1}{\sqrt{2}}(d^\dagger_{1\uparrow}d^\dagger_{2\downarrow}+d^\dagger_{1\downarrow}d^\dagger_{2\uparrow})\ket{0}$ and $\ket{1}=d^\dagger_{1\uparrow}d^\dagger_{2\uparrow}\ket{0}$. Here we ignore the site index $i$ for simplicity.


We can project the electron operator inside this $4+3=7$ dimensional Hilbert space:

\begin{equation}
    d_{i;1\uparrow}= \prod_{j<i}(-1)^{n_j}  \big(\ket{2\uparrow}_i\bra{1}_i+\frac{1}{\sqrt{2}}\ket{2\downarrow}_i\bra{0}_i\big)
    \label{eq:p_electron_1}
\end{equation}

\begin{equation}
    d_{i;1\downarrow}= \prod_{j<i}(-1)^{n_j}  \big(\ket{2\downarrow}_i\bra{-1}_i+\frac{1}{\sqrt{2}}\ket{2\uparrow}_i\bra{0}_i\big)
     \label{eq:p_electron_2}
\end{equation}


\begin{equation}
    d_{i;2\uparrow}= -\prod_{j<i}(-1)^{n_j}  \big(\ket{1\uparrow}_i\bra{1}_i+\frac{1}{\sqrt{2}}\ket{1\downarrow}_i\bra{0}_i\big)
     \label{eq:p_electron_3}
\end{equation}


\begin{equation}
    d_{i;2\downarrow}= -\prod_{j<i}(-1)^{n_j}  \big(\ket{1\downarrow}_i\bra{-1}_i+\frac{1}{\sqrt{2}}\ket{1\uparrow}_i\bra{0}_i\big)
     \label{eq:p_electron_4}
\end{equation}
where $\prod_{j<i}(-1)^{n_j}$ is the Jordan-Wigner string. 


We can also define the spin operators for the singlon state as: $ \vec s_{i;a}=\frac{1}{2}\sum_{\sigma \sigma'} \ket{a\sigma}_i \bra{a\sigma'}_i$.  The spin operators for the doublon states are defined as: $ \vec S_i=\sum_{\alpha,\beta=-1,0,1} \vec T_{\alpha \beta]}\ket{\alpha}_i \bra{\beta}_i$. Here we have $ T_z=\begin{pmatrix} 1 & 0 & 0 \\ 0 & 0 & 0 \\ 0 & 0 & -1 \end{pmatrix} $, $  T_x=\frac{1}{\sqrt{2}}\begin{pmatrix} 0 & 1 & 0 \\ 1 & 0 & 1 \\ 0 & 1 & 0 \end{pmatrix}$ and $ T_y=\frac{1}{\sqrt{2}}\begin{pmatrix} 0 & -i & 0 \\ i & 0 & -i \\ 0 & i & 0 \end{pmatrix}$ in the $\ket{1},\ket{0},\ket{-1}$ basis.




The type II t-J model is then defined as:

\begin{align}
    H&=-t^x_\parallel \sum_l  \sum_{\langle i,j \rangle} (P d^\dagger_{i;l;1}d_{j;l;1} P+H.c.)\notag \\ 
    &~~~-t^z_\parallel\sum_{l}  \sum_{\langle i,j \rangle} (P d^\dagger_{i;l;2}d_{j;l;2} P+H.c.) \notag \\ 
    &~~~-t^{xz}_\parallel \sum_l\sum_{\langle ij \rangle} ((-1)^{s_{ij}} P d^\dagger_{i;l;1}d_{j;l;2} P+H.c.) \notag \\ 
    &~~~-t_\perp^z \sum_i (P d^\dagger_{i;t;2}d_{i;b;2} P+H.c.)  \notag \\
   &~~~ +\Delta \sum_i (n_{i;1}-n_{i;2}) \notag \\
& ~~~+   J^x_\parallel \sum_l \sum_{\langle ij \rangle} \vec s_{i;l;1}\cdot \vec s_{i;l;1}+J^z_\perp \sum_{i} \vec s_{i;t;2}\cdot \vec s_{i;b;2} \notag \\
&~~~+ J_{sd}^\parallel \sum_l \sum_{\langle ij \rangle} (\vec s_{i;l;1}\cdot \vec S_{i;l} +\cdot \vec S_{i;l}\cdot \vec s_{j;l;1}) \notag \\ 
&~~~+J_{sd}^\perp 
 \sum_i (\vec s_{i;t;2}\cdot \vec S_{i;b}+\vec S_{i;t}\cdot \vec s_{i;b;2})\notag \\
&~~~+J_{dd}^\parallel \sum_l \sum_{\langle ij \rangle}\vec S_{i;l}\cdot \vec S_{j;l}+J_{dd}^\perp \sum_i \vec S_{i;t}\cdot \vec S_{i;b} 
\label{eq:type_II_t_J_main}
\end{align}
where $P$ is now the projection to the $4+3=7$ Hilbert space as defined above. We have $J_{sd}^\parallel=\frac{1}{2}J^x_\parallel$, $J_{sd}^\perp=\frac{1}{2} J^z_\perp$. $J_{dd}^\parallel=\frac{1}{4} J^x_\parallel$ and $J_{dd}^\perp=\frac{1}{4}J^z_\perp$.  We are interested in the filling that $n_T=n_1+n_2=1+n=2-x$. So if the number of sites is $N_S$, there are $n N_s$ number of doublon states and $(1-n) N_s$ number of singlon states.  The energy splitting $\Delta$ tunes the relative density of the two orbitals. If $\Delta$ is large and positive, we only need to keep two singlon states corresponding to the $d_2$ orbital.  In the following we also use the notation $x=2-n_T=1-n$ as the hole doping level away from the $d^8$ state.




\textbf{Parton mean field treatment} We employ the three-fermion parton construction\cite{zhang2020type} to deal with the type II t-J model.  We construct the four singlon states as  $\ket{a\sigma}_i=f^\dagger_{i;a\sigma}\ket{0}$.  The three S=1 doublons are created by $\ket{-1}_i=\psi^\dagger_{i;1\downarrow}\psi^\dagger_{i;2\downarrow}\ket{0}$, $\ket{0}_i=\frac{1}{\sqrt{2}}(\psi^\dagger_{i;1\uparrow}\psi^\dagger_{i;2\downarrow}-\psi^\dagger_{i;2\uparrow}\psi^\dagger_{i;1\downarrow})\ket{0}$ and $\ket{1}=\psi^\dagger_{i;1\uparrow}\psi^\dagger_{i;2\uparrow}\ket{0}$.  We need to impose a local constraint at each site $i$: $n_{i;f}+n_{i;\psi_1}=1$ and $n_{i;\psi_1}=n_{i;\psi_2}$.  Here we have $n_{i;f}=\sum_{a\sigma}f^\dagger_{i;a\sigma}f_{i;a\sigma}$ and $n_{i;\psi_a}=\sum_\sigma \psi^\dagger_{i;a\sigma}\psi_{i;a\sigma}$. On average we have $n_f=x$ and $n_{\psi_1}=n_{\psi_2}=1-x$ if we assume $n_1+n_2=2-x$. We introduce the notation $\Psi_{i;\sigma}=(\psi_{i;1\sigma},\psi_{i;2\sigma})^T$, then there is another constraint: $\Psi^\dagger_i \vec \tau \Psi_i=0$ where $\vec \tau$ is Pauli matrix in the color space.  This constraint enforces the two colors $a=1,2$ forms singlet, thus the spin is in a triplet due to fermion statistics\cite{zhang2020type}.  This constraint gives a SU(2) gauge symmetry: $\Psi_i \rightarrow U_i \psi_i$ where $U_i \in SU(2)$ acting in the color space, rotating $\psi_1$ to $\psi_2$.



The singlon and doublon spin operators can now be represented as: $\vec s_{i;a}=\frac{1}{2}\sum_{\sigma,\sigma'} f^\dagger_{i;a\sigma} \vec \sigma_{\sigma \sigma'} f_{i;a\sigma'}$ and $\vec S_i=\frac{1}{2} \sum_{a}\sum_{\sigma \sigma'} \psi^\dagger_{i;a\sigma}\vec \sigma_{\sigma \sigma'} \psi_{i;a\sigma'}$. $n_{i;1}-n_{i;2}=\sum_\sigma(f^\dagger_{i;1\sigma}f_{i;1\sigma}-f^\dagger_{i;2\sigma}f_{i;2\sigma})$  The projected electron operator can be represented as: 



\begin{equation}
    d_{i;a\sigma}=\epsilon_{ab}f^\dagger_{i;b \sigma}\psi_{i;2 \sigma}\psi_{i;1\sigma}+\frac{1}{2} \epsilon_{ab} f^\dagger_{i;b\bar \sigma}(\psi_{i;2\downarrow}\psi_{i;1\uparrow}+\psi_{i;2\uparrow}\psi_{i;1\downarrow})
\end{equation}
where $\epsilon_{ab}$ is the anti-symmetric tensor with $\epsilon_{12}=1$. $\bar \sigma$ is the opposite spin of $\sigma$.


We can rewrite it as

\begin{equation}
    d_{i;a\sigma}=-\frac{1}{2}\epsilon_{ab} f^\dagger_{i;b \sigma'}  \epsilon_{a' b'} \psi_{i;a' \sigma}\psi_{i;b'\sigma'}
\end{equation}
where we assumed Einstein summation convention.






We can substitute the above expressions into the type II t-J model in Eq.~\ref{eq:type_II_t_J_main} and then try to do self-consistent mean field calculation. Because we have two orbitals and two layers, the calculations need to include many mean field decoupling order parameters and are quite tedious. We leave a detailed analysis to future. Here we simply point out how we can understand the sharing of the super-exchange J from one orbital to another orbital.

For simplicity, we assume $\Delta$ is positive and large.  $\Delta \sum_i (n_{i;1}-n_{i;2})=\Delta \sum_i \sum_\sigma(f^\dagger_{i;1\sigma}f_{i;1\sigma}-f^\dagger_{i;2\sigma}f_{i;2\sigma})$ penalizes $f_1$.  In this limit we can ignore $f_1$ and only keep the two singlon states occupied by $f_{2\sigma}$. This corresponds to orbital selective Mott localization of $f_2$ because now $d_{i;2\sigma}=0$ without the $f_{1}$ operator.

In this simple case we only need to consider the hopping $t^x_\parallel$ for the $d_1$ orbital, and it is in the form:

\onecolumngrid

\begin{equation}
    H_K=-\frac{1}{4}t^x_\parallel \sum_l \sum_{\langle ij \rangle}  \epsilon_{a_1 b_1} \epsilon_{a_2 b_2} \psi^\dagger_{i;b_1 \sigma'_1} \psi^\dagger_{i;a_1\sigma}f_{i;2\sigma'_1} f^\dagger_{j;2\sigma'_2}\psi_{j;a_2\sigma}\psi_{j;b_2\sigma'_2} +H.c.
\end{equation}

\twocolumngrid


One obvious mean field decoupling is the on-site term $\langle \psi^\dagger_{i;a\sigma}f_{i;2\sigma} \rangle = \Phi_a$ for each spin $\sigma$. Due to the SU(2) gauge symmetry, we can always fix the gauge to choose $\Phi_2\neq 0$ while $\Phi_1=0$.   Then $\langle \psi^\dagger_{i;2\sigma} f_{i;2\sigma} \rangle =\Phi \neq 0$ and we have $d_{i;1\sigma}\sim \frac{1}{2} \Phi^\dagger \psi_{i;1\sigma}$.  Now $\psi_{i;1\sigma}$ can be identified as the electron operator with density $n_{\psi_1}=1-x$.  Meanwhile $f_2$ and $\psi_2$ form the same band with the total density $n_{f_2}+n_{\psi_2}=1$.  They just represent the localized spin moments and form a rung singlet in the bilayer model due to the large $J^z_\perp$ and $t_\perp^z$ term. 


We can then ignore $f_2, \psi_2$ and write down an effective model for the electron operator $\psi_1$:

\onecolumngrid 

\begin{equation}
    H_{\psi_1}=-\frac{1}{4} |\Phi|^2  t^x_\parallel \sum_l \sum_{\langle ij \rangle} \psi^\dagger_{i;l; 1\sigma} \psi_{i; l; 1\sigma}+J^\parallel_{dd} \sum_l \sum_{\langle ij \rangle} \vec S_{i;l;\psi_1}\cdot \vec S_{j;l;\psi_1}+J^\perp_{dd} \sum_i \vec S_{i;t;\psi_1}\cdot \vec S_{i;b;\psi_1}
\end{equation}

\twocolumngrid

where $\vec S_{i;l;\psi_1}=\frac{1}{2} \psi^\dagger_{i;l;1\sigma} \vec{\sigma}_{\sigma \sigma'} \psi_{i;l;1\sigma'}$ is the spin operator generated by $\psi_1$. We still have the constraint that there is no double occupancy for $\psi_1$.

One can see that in the parton mean field treatment of the type II t-J model, we can automatically generate the bilayer one-orbital t-J model and the inter-layer $J_{\perp}$.  Because $J^{\parallel}_{dd}=\frac{1}{4}J^x_{\parallel}$ and $J^{\perp}_{dd}=\frac{1}{4}J^z_\perp$, we have the effective $J_\parallel/t=\frac{1}{|\Phi|^2} \frac{J^x_\parallel}{t^x_\parallel}$ and $J_\perp/t=\frac{1}{|\Phi|^2} \frac{J^z_\perp}{t^x_\parallel}$. Due to the enhancement of the factor $\frac{1}{|\Phi|^2}$ ($|\Phi|<1$), we are in a regime with even larger $J_\perp/t$ than the bare value in the effective t-J model. Then according to Fig.~\ref{fig:bialyer_t_J_results}, we should expect a strong pairing even when $x=50\%$.


Right now  we assume a large $\Delta$ to always keep the orbital $d_{z^2}$ in a Mott localized state (forming a rung singlet).  Our formalism allows the extension to the region with smaller $\Delta$ so that $n_{f_2}=x-\delta$ and $n_{f_1}=\delta$. In this case the density of $d_2$ is $n_2=n_{f_2}+n_{\psi_2}=x-\delta+1-x=1-\delta$ and the density of $d_1$ is $n_1=n_{f_1}+n_{\psi_1}=\delta+1-x=1-x_{eff}$ with $x_{eff}=x-\delta$.  So $d_{z^2}$  is slightly hole doped with hole doping level $\delta$, while $d_{x^2-y^2}$ has a reduced hole doping level $x_{eff}=x-\delta$ and thus $x_{eff}$ can be smaller than $x=0.5$. In this case of course $d_{z^2}$ orbital can also be mobile and even superconducting.  In the mean field level, we again can decouple $ \langle \psi^\dagger_{i;a\sigma} f_{i;b\sigma} \rangle =\Phi_{ab}$. One natural choice is $\Phi_{ab}=\Phi_a \delta_{ab}$. Then we find $d_{i;a\sigma}\sim \frac{1}{2} \Phi^\dagger_{\bar a} \psi_a$ where $\bar a$ is the opposite orbital to the orbital $a$.  Again we can then write down an effective two-orbital t-J model, but without any $J_H$ or $U'$ term. Meanwhile the $d_{2}$ orbital can inherit a finite $J_\parallel$ and $d_1$ orbital can obtain a finite $J_\perp$ due to the $J_{dd}$ and $J_{sd}$ term. It is interesting to see whether the existence of mobile $d_{z^2}$ orbital can enhance the pairing of the $d_{x^2-y^2}$ orbital. We will have a detailed study of the type II t-J model tuning both $x$ and $\Delta$ in future.




\textbf{Summary} In summary, we propose a bilayer type II t-J model for the superconducting La$_3$Ni$_2$O$_7$ under high pressure. We emphasize the important role of the Hund's coupling between the $d_{x^2-y^2}$ and the $d_{z^2}$ orbital, which enforces the $d^8$ state to be a spin-triplet.  Due to the Hund's rule, the super-exchange of one-orbital can be shared to the other orbital.  We propose a parton mean field treatment of the type II t-J model. In the limit that the $d_{z^2}$ is Mott localized and forms a rung singlet, we reach a bilayer one-orbital t-J model without inter-layer hopping, but with enhanced  inter-layer anti-ferromagnetic spin-spin coupling $J_\perp$ over intra-layer hopping $t$. Mean field theory then predicts a s-wave inter-layer paired superconductor even at hole doping $50\%$, in agreement with the experiment. In future, one natural extension is to tune the orbital splitting $\Delta$ in our type II t-J model to make the $d_{z^2}$ orbital also slightly hole doped.  We expect a rich phase diagram  through tuning $\Delta$ and the total hole doping level $x$.   We  also propose future experiments to  reduce $x$ through electron doping to search for an even higher $T_c$ than 80 K. 


\textbf{Note} When finalizing the manuscript, we become aware of a preprint\cite{lu2023interlayer} which also studied a bilayer one-orbital t-J model with strong inter-layer $J_\perp$, which overlaps with the first part of our paper. 

\textbf{Acknowledgement} YHZ was supported by the
National Science Foundation under Grant No. DMR-2237031. 

\bibliography{refs}


\appendix

\onecolumngrid


\section{Details on One orbital t-J model}
% We consider the effective t-J model of $d_{x^{2}-y^{2}}$ orbital defined in the bilayer square lattice,  
% \begin{eqnarray}
% H_{\mathrm{t-J}}
% &= &
% -t_{1}^{x} \sum_{l, \sigma, \langle i,j \rangle}
% c_{1,l,\sigma }^{\dagger}(i)c_{1,l,\sigma }(j)
% +h.c. -\mu_{0} \sum_{l, \sigma, i}n_{l,i} \nonumber
% \\
% && +J \sum_{l, \langle  i, j \rangle}
% [\bm{S}_{l}(i) \cdot \bm{S}_{l}(j)-\frac{1}{4}n_{l,i}n_{l,j}]
% +  J_{\perp}\sum_{i}
% [\bm{S}_{t}(i) \cdot \bm{S}_{b}(i)-\frac{1}{4}n_{t,i}n_{b,i}],\label{e1}
% \end{eqnarray}
% where $l\in t,b$ denotes the top/bottom layer and $\sigma\in \uparrow, \downarrow$ is for the spin index with $\bm{S}_{l}(i)= \frac{1}{2} c^{\dagger}_{1,l,\sigma }(i) \bm{\sigma}_{\sigma,\sigma '}c_{1,l,\sigma' }(i)$, and $n_{l,i}=\sum_{\sigma} 
% c_{1,l,\sigma}^{\dagger}(i)
% c_{1,l,\sigma}(i)$.

\iffalse
\begin{align}
d_{i;1\sigma}&=f^\dagger_{i;2 \sigma}\psi_{i;2 \sigma}\psi_{i;1\sigma}+\frac{1}{2} f^\dagger_{i;2\bar \sigma}(\psi_{i;2\downarrow}\psi_{i;1\uparrow}+\psi_{i;2\uparrow}\psi_{i;1\downarrow}) \notag \\ 
d_{i;2\uparrow}&=-f^\dagger_{i;1\uparrow}\psi_{i;2\uparrow}\psi_{i;1\uparrow}-\frac{1}{2} f^\dagger_{i;1\downarrow}(\psi_{i;2\downarrow}\psi_{i;1\uparrow}+\psi_{i;1\downarrow}\psi_{i;2\uparrow}) \notag \\ 
d_{i;2\downarrow}&=-f^\dagger_{i;1\downarrow}\psi_{i;2\downarrow}\psi_{i;1\downarrow}-\frac{1}{2} f^\dagger_{i;1\uparrow}(\psi_{i;2\downarrow}\psi_{i;1\uparrow}+\psi_{i;1\downarrow}\psi_{i;2\uparrow})  
\end{align}


\fi




We now study the model Hamiltonian, Eq.\ref{e1} by performing the mean field theory.
It is convenient to employ the slave boson representation of electron 
%for imposing the no-double occupancy constraint.
with the decomposition, $c^{\dagger}_{i;l,1,\sigma}(i)= f^{\dagger}_{i;l;\sigma}{l,\sigma}b_{i;l}$.
% where $\sum_{\langle i,j\rangle }=\sum_{i}(j=i)+$. 
The mean-field approximation with the ansatz, 
\begin{eqnarray}
\Delta_{\parallel,l}^{x} &=&   2\langle f_{i;l;\uparrow}f_{i+x;l;\downarrow}\rangle, 
\quad 
\Delta_{\parallel,l}^{y} =   2\langle f_{i;l;\uparrow}f_{i+y;l;\downarrow}\rangle, 
\quad
\Delta_{\perp} =2 \langle f_{i;t;\uparrow}f_{i;b;\downarrow}\rangle,
\quad
\\
\chi_{ij,s}^{l} &=&2 \langle f_{i;l;s}^{\dagger}f_{j;l;s}\rangle, 
\quad
\chi_{i,s}^{\perp} = 2\langle f_{i;t;s}^{\dagger}f_{i;b;s}\rangle,
\quad 
\chi^{b}_{ij} = \langle b_{i;l} b_{j;l}^{\dagger}\rangle
=x,
\end{eqnarray}
gives the mean field Hamiltonian,
\begin{eqnarray}
H_{\mathrm{MF}}
&=&
-t_{\parallel}^{x} \sum_{l, \sigma}\sum{\langle i,j \rangle}
\big[ 2\chi^{b}
f^{\dagger}_{i;l;\sigma}f_{j;l;\sigma}
-\chi^{b}\chi^{l}
\big] 
+ h.c.
-\mu \sum_{l, \sigma, i}
f^{\dagger}_{i;l;\sigma}  f_{i;l;\sigma} 
\nonumber 
\\
&& -\frac{J_{\parallel}}{2}    \sum_{l}\sum{i}
\big[
\frac{\Delta^{x}_{\parallel,l}}{2}\big[ -f^{\dagger}_{i;l;\uparrow}f^{\dagger}_{i+x;l;\downarrow}
\big]+
\frac{\Delta^{x}_{\parallel,l}}{2}
\big[-f^{\dagger}_{i+x;l;\uparrow} f^{\dagger}_{i;l;\downarrow}\big]
+\mathrm{H.c.} -2\left|\frac{\Delta^{x}_{\parallel,l}}{2}\right|^2
\big]
\nonumber
\\
&& -\frac{J_{\parallel}}{2}  \sum_{l}\sum{i}
\big[
\frac{\Delta^{y}_{\parallel,l}}{2}\big[ -f^{\dagger}_{i;l;\uparrow}f^{\dagger}_{i+y;l;\downarrow}
\big]+
\frac{\Delta^{y}_{\parallel,l}}{2}
\big[-f^{\dagger}_{i+y;l;\uparrow} f^{\dagger}_{i;l;\downarrow}\big]
+\mathrm{H.c.} -2\left|\frac{\Delta^{y}_{\parallel,l}}{2}\right|^2
\big]
\nonumber
\\
&& -\frac{J_{\parallel}}{2}\sum_{l,i,\sigma}
\big[
\frac{\chi^{l}}{2} f^{\dagger}_{i+x;l;\sigma}f_{i;l;\sigma}
+\mathrm{H.c.}
-\left|\frac{\chi^{l}}{2} \right|^2
\big]
+
\big(
x\leftrightarrow y
\big)
\nonumber
\\ 
%inter
&& -\frac{J_{\perp}}{2}\sum_{l, i }
\frac{\Delta_{\perp}}{2}
\big[ -f^{\dagger}_{i;t;\uparrow}f^{\dagger}_{i;b;\downarrow}
\big]+
\frac{\Delta^{\perp}}{2}
\big[-f^{\dagger}_{i;b;\uparrow}
f^{\dagger}_{i;t,\downarrow}\big]
+\mathrm{H.c.} -2\left|\frac{\Delta_{\perp}}{2} \right|^2
\nonumber
\\
&& -\frac{J_{\perp} }{2}
\sum_{l,\sigma,i }
\big[
\frac{\chi_{\perp}}{2} f^{\dagger}_{i;b;\sigma}f_{i;t;\sigma}
+\mathrm{H.c.} 
-\left|\frac{\chi_{\perp}}{2}\right|^2
\big]
\label{eq:h_mf}
\end{eqnarray}
%% Here



Introducing the Nambu basis $\Psi_{k}^{T} = (c_{t,k,\uparrow},c_{t,-k,\downarrow},c_{b,k,\uparrow},c_{b,-k,\downarrow})$ further simplifies the Hamiltonian in the momentum space,
\begin{eqnarray}
H_{\mathrm{MF}} &= &
\sum_{k} 
\Psi^{\dagger}_{k}
\hat{M}(k)
\Psi_{k}
+F_{0},
\end{eqnarray}
with
\begin{eqnarray}
    \hat{M}(k) &=&
    \left(
    \begin{array}{cc|cc}
         \epsilon_{t}& \overline{\Delta}_{t}&\epsilon_{
         \perp}&\overline{\Delta}_{\perp} \\
         \overline{\Delta}_{t}^*  & -\epsilon_{t}
         &\overline{\Delta}_{\perp}^* &-\epsilon_{\perp}\\
         \hline
        \epsilon_{\perp}^{*}& \overline{\Delta}_{\perp}
        &
        \epsilon_{b}&\overline{\Delta}_{b} \\
         \overline{\Delta}_{\perp}^*  & -\epsilon_{\perp}^{*}
         &\overline{\Delta}_{b}^* &-\epsilon_{b}
         \\
    \end{array}
    \right)
\end{eqnarray}
and
\begin{eqnarray}
    \epsilon_{l}(k) 
    &=& -[2t_{1}^{x}x +\frac{J_{\parallel}}{2}\chi^{l}]
    \Phi_{s} (k) 
    -\mu,
    \quad
       \epsilon_{\perp} (k)
   =-\frac{J_{\perp}^{x}}{4}\chi_{\perp}^{*}, \label{e12}
    \\
\overline{\Delta}^{l}_{\parallel}(k)
    &=&
   - \frac{J_{\parallel} }{2}[\Delta^{l}_{\parallel,x}\cos k_{x}
    +\Delta^{l}_{\parallel,y}\cos k_{y}], 
  \quad
    \overline{\Delta}_{\perp}  = -\frac{J_{\perp}^{x}}{4}\Delta_{\perp
    },
 \end{eqnarray}
and
\begin{eqnarray}
    F_{0} &=& \sum_{l,k} \epsilon_{l} (k)
    + \sum_{l}
    t_{1}^{x} \chi^{b} \chi^{l}
+\frac{J_{\parallel}}{2}\left(\frac{|\Delta^{l}_{\parallel,x}|^2}{2} +\frac{|\Delta^{l}_{\parallel,y}|^2 }{2}+|\chi^{l}|^2\right)
+\frac{J_{\perp}}{4}\left(|\Delta_{\perp}|^2 +|\chi_{\perp}|^2\right).
\end{eqnarray}

 % \begin{eqnarray}
 %     E_{\alpha}(k) &=& \Big[\epsilon_{\parallel}(k)^2
 %     +\overline{\Delta}_{\perp}^2
 %     + \overline{\Delta}_{\parallel,x}^2 
 %    + \overline{\Delta}_{\parallel,y}^2 
 %    +
 %     2\cos\theta \overline{\Delta}_{\parallel,x}
 %     \overline{\Delta}_{\parallel,y}
 %      \\
 %     &&
 %     +\alpha 2 \overline{\Delta}_{\perp}
 %     (\overline{\Delta}_{\parallel,x}
 %     \cos \left( \frac{\theta-\theta_{t}-\theta_{b}}{2}\right)
 %        +\overline{\Delta}_{\parallel,y}
 %     \cos \left(\frac{\theta+\theta_{t}+\theta_{b}}{2}\right)
 %     )
 %     \Big]^{1/2}.
 % \end{eqnarray}
 % At zero temperature, the mean field free energy becomes the ground state energy, 
 
 % \subsubsection{Self-consistent equations}
The self-consistency equation can be obtained bythe expectation value of each order parameters. Due to the translational symmetry, one can find that
\begin{eqnarray}
x&=& 1-\frac{1}{2} \sum_{l,k,\sigma}
\langle
f^{\dagger}_{l,\sigma}(k)
f_{l,\sigma}(k)
\rangle \label{e17}
,\\
\chi^{l} &=& 
\sum_{k,\sigma}\Phi_{s}(k)
\langle
f^{\dagger}_{l,\sigma}(k)
f_{l,\sigma}(k)
\rangle
,\\
\chi^{\perp} &=& \sum_{k,\sigma}
\langle
f^{\dagger}_{t,\sigma}(k)
f_{b,\sigma}(k)
\rangle
,\\
\Delta^{l}_{d} &=& \sum_{k} \Phi_{d}(k)
\langle f_{l,\uparrow}(k)f_{l,\downarrow}(-k)-f_{l,\downarrow}(k)f_{l,\uparrow}(-k)\rangle
 \\
\Delta^{\perp} &=& \sum_{k} \langle f_{t,\uparrow}(k)f_{b,\downarrow}(-k)-f_{t,\downarrow}(k)f_{b,\uparrow}(-k)\rangle \label{e21}
\end{eqnarray}
with $\Phi_{s,d}(k)=\cos k_x \pm \cos k_y$.
% To evaluate $\langle f^{\dagger} f\rangle$,$\langle ff\rangle$, we introduce the unitary transformation $ \Psi(k) =\mathcal{U}(k) \Gamma(k)$ with 
% \begin{eqnarray}
%    \quad 
%     \mathcal{U}^{\dagger} \hat{M}(k) \mathcal{U}(k)
%     = D(k),
% \end{eqnarray}
% where $\Gamma(D)$ is diagonalized basis (matrix) of the BdG Hamiltonian. In appendix, we provide the exact form of self-consistency equations, Eqs. (\ref{e17}-\ref{e21}) in terms of the component of $\mathcal{U}(k)$.
 % \subsubsection{Numerical results}
For solving the self-consistency solutions for $\{x,\chi,\Delta\}$, we used the iteration methods with a convergence tolerance $\sim 10^{-5}$.
The numerical integration in the momentum space is performed by using $100 \times 100 k$ points in Brillouin zone. 






\end{document}
\section{Derivation of the type II t-J model}


The above model is hard to analyze due to the large $U', J_H$ which should be larger than $t$ and $J$ terms. We already know the important effect of $J_H$ on sharing the $J$ term of one orbital to the other orbital. Meanwhile $U'$ suppresses the simultaneous ocuupation of the two orbitals at the same site. Both these two effects are beyond simple perturbative or mean field treatment. Therefore the two-orbital t-J model with large U' and $J_H$ may not be very useful.


We propose a non-perturbative treatment of the large $U'$ and $J_H$ by further restricting the Hilbert space.  Let us consider the total filling $n=n_1+n_2 =1+x$ with $x\in (0,1)$.  Then in the large $U'$ regime, if we label the number of sites as $N_s$, there must be $(1-x)N_S$ number of singly occupied sites and the rest of $x N_S$ sites have both orbitals occupied.  Note that the empty site is forbidden: if we want to add one empty site, we need to increase one doubly occupied site which cost a large $U'$. Therefore, in the large $U'$ regime, we can restrict to singly occupied states and doubly occupied states, which correspond to d$^7$ and $d^8$ configuration respectively. We will call them singlon and doublon states.   Then let us further add the large Hund's coupling $J_H$. Obviously $J_H$ favors the doublon state to be in the $S=1$ triplet. 

In the end we can just keep four singlon states and three spin-triplet doublon states.   First, at each site $i$ the four singlon states can be labeled as $\ket{a \sigma}=d^\dagger_{a;\sigma}\ket{0}$ where $a=1,2$ and $\sigma=\uparrow, \downarrow$. Meanwhile, the three spin-triplet doublon states are: $\ket{-1}=d^\dagger_{1\downarrow}d^\dagger_{2\downarrow}\ket{0}$, $\ket{0}=\frac{1}{\sqrt{2}}(d^\dagger_{1\uparrow}d^\dagger_{2\downarrow}+d^\dagger_{1\downarrow}d^\dagger_{2\uparrow})\ket{0}$ and $\ket{1}=d^\dagger_{1\uparrow}d^\dagger_{2\uparrow}\ket{0}$. Here we ignore the site index $i$ for simplicity.


We can project the electron operator inside this $4+3=7$ dimensional Hilbert space:

\begin{equation}
    d_{i;1\uparrow}= \prod_{j<i}(-1)^{n_j}  \big(\ket{2\uparrow}_i\bra{1}_i+\frac{1}{\sqrt{2}}\ket{2\downarrow}_i\bra{0}_i\big)
    \label{eq:p_electron_1}
\end{equation}

\begin{equation}
    d_{i;1\downarrow}= \prod_{j<i}(-1)^{n_j}  \big(\ket{2\downarrow}_i\bra{-1}_i+\frac{1}{\sqrt{2}}\ket{2\uparrow}_i\bra{0}_i\big)
     \label{eq:p_electron_2}
\end{equation}


\begin{equation}
    d_{i;2\uparrow}= -\prod_{j<i}(-1)^{n_j}  \big(\ket{1\uparrow}_i\bra{1}_i+\frac{1}{\sqrt{2}}\ket{1\downarrow}_i\bra{0}_i\big)
     \label{eq:p_electron_3}
\end{equation}


\begin{equation}
    d_{i;2\downarrow}= -\prod_{j<i}(-1)^{n_j}  \big(\ket{1\downarrow}_i\bra{-1}_i+\frac{1}{\sqrt{2}}\ket{1\uparrow}_i\bra{0}_i\big)
     \label{eq:p_electron_4}
\end{equation}
where $\prod_{j<i}(-1)^{n_j}$ is the Jordan-Wigner string. 


We can also define the spin operators for the singlon state as:

    
\begin{equation}
    \vec s_{i;a}=\frac{1}{2}\sum_{\sigma \sigma'} \ket{a\sigma}_ibra{a\sigma'}_i
\end{equation}

The spin operators for the doublon states are defined as:

\begin{equation}
    \vec S_i=\sum_{\alpha,\beta=-1,0,1} \vec T_{\alpha \beta]}\ket{\alpha}_i \bra{\beta}_i
\end{equation}
with
\begin{equation}
    T_z=\begin{pmatrix} 1 & 0 & 0 \\ 0 & 0 & 0 \\ 0 & 0 & -1 \end{pmatrix}
\end{equation}

\begin{equation}
    T_x=\frac{1}{\sqrt{2}}\begin{pmatrix} 0 & 1 & 0 \\ 1 & 0 & 1 \\ 0 & 1 & 0 \end{pmatrix}
\end{equation}

\begin{equation}
    T_y=\frac{1}{\sqrt{2}}\begin{pmatrix} 0 & -i & 0 \\ i & 0 & -i \\ 0 & i & 0 \end{pmatrix}
\end{equation}
in the $\ket{1}, \ket{0}, \ket{-1}$ basis. 

Note that if we restrict to the three doublon states, the original operator $\vec S_{i;a}=\frac{1}{2} \vec S_i$.  From this relationship, we can easily derive the following type II t-J model within this $7$ dimensional Hilbert space:

\begin{align}
    H&=-t^x_\parallel \sum_l  \sum_{\langle i,j \rangle} (P d^\dagger_{i;l;1}d_{j;l;1} P+H.c.)\notag \\ 
    &~~~-t^z_\parallel\sum_{l}  \sum_{\langle i,j \rangle} (P d^\dagger_{i;l;2}d_{j;l;2} P+H.c.) \notag \\ 
    &~~~-t^{xz}_\parallel \sum_l\sum_{\langle ij \rangle} ((-1)^{s_{ij}} P d^\dagger_{i;l;1}d_{j;l;2} P+H.c.) \notag \\ 
    &~~~-t_\perp^z \sum_i (P d^\dagger_{i;t;2}d_{i;b;2} P+H.c.)  \notag \\
   &~~~ +\Delta \sum_i (n_{i;1}-n_{i;2}) \notag \\
& ~~~+   J^x_\parallel \sum_l \sum_{\langle ij \rangle} \vec s_{i;l;1}\cdot \vec s_{i;l;1}+J^z_\perp \sum_{i} \vec s_{i;t;2}\cdot \vec s_{i;b;2} \notag \\
&~~~+ J_{sd}^\parallel \sum_l \sum_{\langle ij \rangle} (\vec s_{i;l;1}\cdot \vec S_{i;l} +\cdot \vec S_{i;l}\cdot \vec s_{j;l;1})+J_{sd}^\perp 
 \sum_i (\vec s_{i;t;2}\cdot \vec S_{i;b}+\vec S_{i;t}\cdot \vec s_{i;b;2})\notag \\
&~~~+J_{dd}^\parallel \sum_l \sum_{\langle ij \rangle}\vec S_{i;l}\cdot \vec S_{j;l}+J_{dd}^\perp \sum_i \vec S_{i;t}\cdot \vec S_{i;b} 
\label{eq:type_II_t_J_appendix}
\end{align}
where $P$ is now the projection to the $4+3=7$ Hilbert space as defined in Eq.~\ref{eq:p_electron_1} to Eq.~\ref{eq:p_electron_4}. We have $J_{sd}^\parallel=\frac{1}{2}J^x_\parallel$, $J_{sd}^\perp=\frac{1}{2} J^z_\perp$. $J_{dd}^\parallel=\frac{1}{4} J^x_\parallel$ and $J_{dd}^\perp=\frac{1}{4}J^z_\perp$.
