%% This is file `sample-manuscript.tex',
%% generated with the docstrip utility.
%%
%% The original source files were:
%%
%% samples.dtx  (with options: `manuscript')
%% 
%% IMPORTANT NOTICE:
%% 
%% For the copyright see the source file.
%% 
%% Any modified versions of this file must be renamed
%% with new filenames distinct from sample-manuscript.tex.
%% 
%% For distribution of the original source see the terms
%% for copying and modification in the file samples.dtx.
%% 
%% This generated file may be distributed as long as the
%% original source files, as listed above, are part of the
%% same distribution. (The sources need not necessarily be
%% in the same archive or directory.)
%%
%%
%% Commands for TeXCount
%TC:macro \cite [option:text,text]
%TC:macro \citep [option:text,text]
%TC:macro \citet [option:text,text]
%TC:envir table 0 1
%TC:envir table* 0 1
%TC:envir tabular [ignore] word
%TC:envir displaymath 0 word
%TC:envir math 0 word
%TC:envir comment 0 0
%%
%%
%% The first command in your LaTeX source must be the \documentclass
%% command.
%%
%% For submission and review of your manuscript please change the
%% command to \documentclass[manuscript, screen, review]{acmart}.
%%
%% When submitting camera ready or to TAPS, please change the command
%% to \documentclass[sigconf]{acmart} or whichever template is required
%% for your publication.
%%
%%
%%\documentclass[manuscript, screen, review]{acmart}
\documentclass[manuscript]{acmart}
\usepackage{multirow}
\usepackage{hyperref}
\usepackage{graphicx}
\usepackage{subcaption}
\usepackage{hyperref}

%%
%% \BibTeX command to typeset BibTeX logo in the docs
\AtBeginDocument{%
  \providecommand\BibTeX{{%
    Bib\TeX}}}

%% Rights management information.  This information is sent to you
%% when you complete the rights form.  These commands have SAMPLE
%% values in them; it is your responsibility as an author to replace
%% the commands and values with those provided to you when you
%% complete the rights form.
\copyrightyear{2023}
\acmYear{2023}
\setcopyright{acmlicensed}\acmConference[RecSys '23]{Seventeenth ACM Conference on Recommender Systems}{September 18--22, 2023}{Singapore, Singapore}
\acmBooktitle{Seventeenth ACM Conference on Recommender Systems (RecSys '23), September 18--22, 2023, Singapore, Singapore}
\acmPrice{15.00}
\acmDOI{10.1145/3604915.3608839}
\acmISBN{979-8-4007-0241-9/23/09}


%%
%% Submission ID.
%% Use this when submitting an article to a sponsored event. You'll
%% receive a unique submission ID from the organizers
%% of the event, and this ID should be used as the parameter to this command.
%%\acmSubmissionID{123-A56-BU3}

%%
%% For managing citations, it is recommended to use bibliography
%% files in BibTeX format.
%%
%% You can then either use BibTeX with the ACM-Reference-Format style,
%% or BibLaTeX with the acmnumeric or acmauthoryear sytles, that include
%% support for advanced citation of software artefact from the
%% biblatex-software package, also separately available on CTAN.
%%
%% Look at the sample-*-biblatex.tex files for templates showcasing
%% the biblatex styles.
%%

%%
%% The majority of ACM publications use numbered citations and
%% references.  The command \citestyle{authoryear} switches to the
%% "author year" style.
%%
%% If you are preparing content for an event
%% sponsored by ACM SIGGRAPH, you must use the "author year" style of
%% citations and references.
%% Uncommenting
%% the next command will enable that style.
%%\citestyle{acmauthoryear}

%%
%% end of the preamble, start of the body of the document source.
\begin{document}

%%
%% The "title" command has an optional parameter,
%% allowing the author to define a "short title" to be used in page headers.
\title{Widespread Flaws in Offline Evaluation of Recommender Systems}

%%
%% The "author" command and its associated commands are used to define
%% the authors and their affiliations.
%% Of note is the shared affiliation of the first two authors, and the
%% "authornote" and "authornotemark" commands
%% used to denote shared contribution to the research.
\author{Bal\'{a}zs Hidasi}
%%\authornote{Both authors contributed equally to this research.}
\email{balazs.h@taboola.com}
%%\orcid{1234-5678-9012}
\author{\'{A}d\'{a}m Tibor Czapp}
%%\authornotemark[1]
\email{adam-tibor.c@taboola.com}
\affiliation{%
  \institution{Gravity R\&D, a Taboola company}
  \streetaddress{Vill\'{a}nyi \'{u}t 40/B}
  \city{Budapest}
  \country{Hungary}
  \postcode{1113}
}

%%
%% By default, the full list of authors will be used in the page
%% headers. Often, this list is too long, and will overlap
%% other information printed in the page headers. This command allows
%% the author to define a more concise list
%% of authors' names for this purpose.
\renewcommand{\shortauthors}{Hidasi and Czapp}

%%
%% The abstract is a short summary of the work to be presented in the
%% article.
\begin{abstract}
  \begin{abstract}
Graph Neural Networks (GNNs) have proven to be effective in processing and learning from graph-structured data.
However, previous works mainly focused on understanding single graph inputs while many real-world applications require pair-wise analysis for graph-structured data (e.g., scene graph matching, code searching, and drug-drug interaction prediction).
To this end, recent works have shifted their focus to learning the interaction between pairs of graphs.
Despite their improved performance, these works were still limited in that the interactions were considered at the node-level, resulting in high computational costs and suboptimal performance.
To address this issue, we propose a novel and efficient graph-level approach for extracting interaction representations using co-attention in graph pooling. 
Our method, Co-Attention Graph Pooling (CAGPool), exhibits competitive performance relative to existing methods in both classification and regression tasks using real-world datasets, while maintaining lower computational complexity.

\end{abstract}
\end{abstract}

%%
%% The code below is generated by the tool at http://dl.acm.org/ccs.cfm.
%% Please copy and paste the code instead of the example below.
%%
\begin{CCSXML}
<ccs2012>
<concept>
<concept_id>10002951.10003317.10003347.10003350</concept_id>
<concept_desc>Information systems~Recommender systems</concept_desc>
<concept_significance>500</concept_significance>
</concept>
</ccs2012>
\end{CCSXML}

\ccsdesc[500]{Information systems~Recommender systems}

%%
%% Keywords. The author(s) should pick words that accurately describe
%% the work being presented. Separate the keywords with commas.
\keywords{recommender systems, offline evaluation, evaluation setups, evaluation flaws}

%%\received{20 February 2007}
%%\received[revised]{12 March 2009}
%%\received[accepted]{5 June 2009}

%%
%% This command processes the author and affiliation and title
%% information and builds the first part of the formatted document.
\maketitle

\section{Introduction}

% Figure environment removed

Reinforcement Learning from Human Feedback (RLHF) has recently been used to great effect to align pretrained large language models (LLMs) to human preferences, optimizing for desirable qualities like harmlessness and helpfulness~\citep{bai2022training} and achieving state-of-the-art results across a variety of natural language tasks~\citep{openai2023gpt4}. %RLHF approaches fundamentally rely on collecting pairs of LLM outputs $(o_1, o_2)$ from a shared prompt $p$, with a human indicating which output in each pair is better on a specified attribute.
% A fundamental component of RLHF is a preference model derived from human labels, typically formatted as pairs of LLM outputs $(o_1, o_2)$ generated from a shared prompt $p$.

A standard RLHF procedure fine-tunes an initial unaligned LLM using an RL algorithm such as PPO~\citep{schulman2017proximal}, optimizing the LLM to align with human preferences. %\violet{not sure whether we need to provide this detail in the intro, especially this has nothing to do with our contribution.} % i feel like this context is useful later when e.g. explaining that context distillation is SFT
RLHF is thus critically dependent on a reward model derived from human-labeled preferences, typically \textit{pairwise preferences} on LLM outputs $(o_1, o_2)$ generated from a shared prompt $p$. % and labeled by humans. 

However, collecting human pairwise preference data, especially high-quality data, may be expensive and time consuming at scale. To address this problem, approaches have been proposed to obtain labels without human annotation, such as Reinforcement Learning from AI Feedback (RLAIF) and context distillation. 

\iffalse
raising the question of whether we can generate high-quality data for RLHF without using human labeling. %accurately-labeled preference pairs $(o_1, o_2)$
%, motivating model alignment approaches that aim to generate accurately-labeled preference pairs $(o_1, o_2)$ without human involvement. 
Two major categories of such approaches are . 
\fi

RLAIF approaches (e.g.,~\citet{bai2022constitutional}) simulate human pairwise preferences by scoring $o_1$ and $o_2$ with an LLM (Figure \ref{fig:rlcd_differences} center); the scoring LLM is often the same as the one used to generate the original pairs $(o_1, o_2)$. Of course, the resulting LLM pairwise preferences will be somewhat noisier compared to human labels. However, this problem is exacerbated by using the same prompt $p$ to generate both $o_1$ and $o_2$, causing $o_1$ and $o_2$ to often be of very similar quality and thus hard to differentiate (e.g., Table~\ref{tab:rlaif_bad_example}). Consequently, training signal can be overwhelmed by label noise, yielding lower-quality preference data. 

% While it avoids human labeling efforts, it has weakness. First, LLM preference labels will naturally be somewhat noisier compared to human labels. Furthermore, since the same prompt $p$ is used to generate both $o_1$ and $o_2$, their quality is often very similar and hard to differentiate (See Table~\ref{tab:rlaif_bad_example}). As a result, training signals can be overwhelmed by label noise, yielding lower-quality preference data. 

Meanwhile, context distillation methods (e.g., \citet{sun2023principle}) create more training signal by modifying the initial prompt $p$. 
%to create more significant training signal. 
The modified prompt $p_+$ typically contains additional context encouraging a \textit{directional attribute change} in the output $o_+$ (Figure \ref{fig:rlcd_differences} right). However, context distillation methods only generate a single output $o_+$ per prompt $p_+$, which is then used for supervised fine-tuning, losing the pairwise preferences which help RLHF-style approaches to 
%rather than using a RLHF-style preference model to 
derive signal from the contrast between outputs. 
Multiple works have observed that RL approaches using preference models for pairwise preferences can substantially improve over supervised fine-tuning by itself when aligning LLMs~\citep{ouyang2022training,dubois2023alpacafarm}. 

% conduct alignment by running supervised fine-tuning on model outputs $o_+$ generated from a modified prompt $p_+$. $p_+$ typically contains additional context encouraging desirable attributes (Figure \ref{fig:rlcd_differences} right), such as in \citet{sun2023principle}. However, multiple works have observed that RLHF-style approaches can substantially improve over supervised fine-tuning by itself when aligning LLMs~\citep{ouyang2022training,dubois2023alpacafarm}. 

Therefore, while both RLAIF and context distillation approaches have already been successfully applied in practice to align language models, we posit that it may be even more effective to combine the key advantages of both. That is, we will use RL with \textit{pairwise preferences}, while also using modified prompts to encourage \textit{directional attribute change} in outputs. %In particular, we will adapt the RLAIF data generation process with two different prompts rather than a single $p$, modifying both prompts similarly to context distillation. %\violet{this motivation is a little unexciting. I think we can more specifically discuss the potential benefits of our approach, like the benefits from RL: exploration/data generation; benefits from contrast. I don't think we get too much benefits from context distillation since we switched to the RL framework.} 

Concretely, we propose \oursfull{} (\ours{}). 
\ours{} generates preference data as follows. Rather than producing two i.i.d.\ model outputs $(o_1, o_2)$ from the same prompt $p$ as in RLAIF, \ours{} creates two variations of $p$: a \textit{positive prompt} $p_+$ similar to context distillation which encourages directional change toward a desired attribute, and a \textit{negative prompt} $p_-$ which encourages directional change \textit{against} it (Figure \ref{fig:rlcd_differences} left). We then generate model outputs $(o_+, o_-)$ respectively, and automatically label $o_+$ as preferred---that is, \ours{} automatically ``generates'' pairwise preference labels by construction. %, without further post hoc labeling.\violet{should make it clearer that our approach `generates' labels by construction} 
We then follow the standard RL pipeline of training a preference model followed by PPO. 

Compared to RLAIF-generated preference pairs $(o_1, o_2)$ from the same input prompt $p$, there is typically a clearer difference in the quality of $o_+$ and $o_-$ generated using \ours{}'s directional prompts $p_+$ and $p_-$, which may result in less label noise. %which may result in better training signal for the preference model. 
That is, intuitively, \ours{} exchanges having examples be \textit{closer to the classification boundary} for much more \textit{accurate labels} on average. Compared to standard context distillation methods, on top of leveraging pairwise preferences for RL training, \ours{} can derive signal not only from the positive prompt $p_+$ which improves output quality, but also from the negative prompt $p_-$ which degrades it. %\ours{} is not learning to imitate $o_+$, but to distill the \textit{contrast} between $o_+$ and $o_-$. 
Positive outputs $o_+$ don't need to be perfect; they only need to contrast with $o_-$ on the desired attribute while otherwise following a similar style.

% \todo{discuss our method and why intuitively it may be better.}

We evaluate the practical effectiveness of \ours{} through both human and automatic evaluations on three tasks, aiming to improve the ability of LLaMA-7B~\citep{touvron2023llama} to generate harmless outputs, helpful outputs, and high-quality story outlines. %\ours{} outperforms both RLAIF and context distillation baselines in pairwise comparisons on 
As shown in Sec. \ref{sec:experiments}, \ours{} substantially outperforms both RLAIF and context distillation baselines in pairwise comparisons when simulating preference data with LLaMA-7B, while still performing equal or better when simulating with LLaMA-30B. 
%On all three tasks, \ours{} substantially outperforms both RLAIF and context distillation baselines in pairwise comparisons---by a margin of at least 9\% and often more than 30\%---validating our method's efficacy. 
We will release all code at a later date, although in any case \ours{} is fairly easy to implement by modifying any reference RLAIF codebase. %We release all code at \todo{github link}.
\section{Design flaws in evaluation setups}\label{sec:flaws}

In this section we go through the main steps of designing offline evaluation and point out potential pitfalls and flaws.

Recommender systems are utilized in various scenarios covering a wide spectrum of domains, use-cases, data and goals, which might require completely different approaches. Therefore, step zero is task definition, since the offline evaluation is designed around the task, not vice versa. The actual first design step is deciding on the evaluation methodology and metrics. Selecting options that are inappropriate for the task undermines the whole evaluation. 

The two main methodologies differ in how they model user preference. Under \emph{behavior prediction}, user preferences are described by both interactions with the recommender and organic events. Model performance is measured by how well it can predict user behavior. The research community has been utilizing this setup since the early days of implicit feedback based recommenders (e.g.~\cite{hu2008collaborative}). This methodology suffers from the lack of negative feedback. It is also imperfect since flawless behavior prediction would recommend the same items that the user would have interacted with on their own. However, it is still a satisfactory setup because the behavior modeling capability of an algorithm loosely correlates with online performance. \emph{Interaction prediction} only considers the interactions with the recommender. Positive/negative feedback is generated when a user does/doesn't interact with a recommended item. The traditional use-case of this methodology has been CTR prediction~\cite{guo2017deepfm}. The main challenge is that reliance on bandit feedback results in a closed feedback loop, making it hard to estimate the uplift of a new algorithm through off-policy evaluation. Counterfactual evaluation~\cite{saito2021counterfactual} aims at alleviating the problem, but still does not scale up to the large state--action space of recommenders. 

The two main options for metrics are ranking/IR (e.g.~recall@N, MRR@N, NDCG@N, etc.) and classification metrics (e.g.~AUC, accuracy, etc.). Behavior prediction often uses the former and interaction prediction the latter, but this is not a necessity. Auxiliary metrics (e.g.~diversity, novelty, serendipity) can be used along with accuracy metrics, and recent research shows that serendipity can have a significant impact on online performance~\cite{chen2021values}. 

Common flaws of subsequent design steps are discussed in detail. These flaws are generic, but for demonstration purposes, we chose the next item prediction of session-based recommenders~\cite{hidasi2015session} as the task, behavior prediction as the appropriate methodology, and recall@N and MRR@N as offline metrics. Most of our experiments utilize the official implementation\footnote{\url{https://github.com/hidasib/GRU4Rec}} of GRU4Rec\cite{hidasi2015session,hidasi2018recurrent}, because the speed-optimized code allows for quick experimentation on both small and large datasets. The code of our experiments, hyperparameters and additional results are publicly available\footnote{\url{https://github.com/hidasib/recsys_eval_flaws}}.

\subsection{Dataset--task mismatch}\label{ssec:flaw-data}

The next step is choosing datasets. For the sake of reproducibility, public datasets should be used along with or instead of proprietary ones. This is where a common flaw can happen. Not every dataset is appropriate for every task or evaluation methodology. Data might be transformed to somewhat accommodate a task, e.g.~rating data was often used as implicit feedback (e.g.~\cite{pilaszy2010fast,hidasi2012fast}), when public implicit feedback datasets were not available. However, the validity of this transformation can be questionable. If all ratings are treated as implicit feedback, negative feedback is knowingly treated as positive; on the other hand, if only high ratings are treated as implicit feedback, it becomes artificially less noisy than feedback in real-life datasets. Nowadays more and more datasets~\cite{requena2020shopper,kang2018self,wan2018item} are made available for a wide array of recommendation tasks. However, many researchers still use the same datasets regardless of the task.

Sequential recommendation -- i.e.~when the collections of events (e.g.~user histories, sessions) are treated as sequences -- only makes sense if the data has sequential patterns. E.g.~sessions of similar topics happening within a few days often have similar patterns because the service with which the users interact sets them on similar paths. \cite{petrov2022systematic} summarized that the most commonly (e.g.~\cite{kang2018self,sun2019bert4rec,huang2018improving,zhou2020s3}) used datasets for evaluating sequential recommenders are rating datasets\footnote{ML10M: \url{https://grouplens.org/datasets/movielens/10m/}; Steam: \url{https://cseweb.ucsd.edu/~jmcauley/datasets.html\#steam\_data}; Yelp: \url{https://www.yelp.com/dataset}; Amazon: \url{http://jmcauley.ucsd.edu/data/amazon/links.html}} (MovieLens, Steam, Yelp, Amazon (Beauty)) wherein user rating histories are treated as sequences. However, the presence of sequential patterns is questionable, because the time of rating is disjoint from the time of interacting with the item. E.g.~the user might skip rating  items they have no strong opinion on.
  
\begin{table*}
  \caption{Basic statistics of train/test splits and event collision rate of the datasets}
  \label{tab:data_stat}
  \footnotesize
  \begin{tabular}{l|rrr|rrr|r||rr}
    \toprule
    \multirow{2}{*}{Dataset} & \multicolumn{3}{c|}{Training set} & \multicolumn{3}{c|}{Test set} & \multirow{2}{*}{\#Items} & \multicolumn{2}{c}{Event time collisions}\\
    & \multicolumn{1}{c}{\#Events} & \multicolumn{1}{c}{\#Sequences} & \multicolumn{1}{c|}{\#Days} & \multicolumn{1}{c}{\#Events} & \multicolumn{1}{c}{\#Sequences} & \multicolumn{1}{c|}{\#Days} & &  \multicolumn{1}{c}{Proportion} & \multicolumn{1}{c}{Event\%}\\
    \midrule
    Amazon (Beauty) & 724,440 & 215,595 & 4,907 & 30,191 & 11,452 & 56 & 38,606 & 31.89\% & 33.03\% \\
    MovieLens10M & 9,861,612 & 69,141 & 5,054 & 99,022 & 737 & 56 & 10,066 & 17.83\% & 27.33\% \\
    Steam & 4,856,479 & 900,878 & 2,582 & 46,039 & 16,916 & 56 & 12,229 & 7.67\% & 13.49\% \\
    Yelp & 5,583,947 & 810,015 & 6,091 & 15,437 & 5,183 & 91 & 132,895 & 0.05\% & 0.06\% \\
    \midrule
    Rees46 & 67,575,203 & 10,190,006 & 60 & 1,054,210 & 166,841 & 1 & 172,756 & 0.03\% & 0.04\% \\
    Coveo & 1,411,113 & 165,673 & 17 & 52,501 & 7,748 & 1 & 10,868 & 0.00\% & 0.00\% \\
    RetailRocket & 750,832 & 196,234 & 131 & 29,148 & 8,036 & 7 & 36,824 & 0.05\% & 0.05\% \\
    \bottomrule
  \end{tabular}
\end{table*}

The aforementioned datasets are compared to three session datasets\footnote{Rees46: \url{https://www.kaggle.com/datasets/mkechinov/ecommerce-behavior-data-from-multi-category-store}; Coveo: \url{https://github.com/coveooss/shopper-intent-prediction-nature-2020}; Retailrocket: \url{https://www.kaggle.com/datasets/retailrocket/ecommerce-dataset}} -- Rees46, Coveo~\cite{requena2020shopper} and RetailRocket -- to investigate the presence of sequential patterns. Data preprocessing is as follows (see Table~\ref{tab:data_stat} for basic statistics):
\begin{enumerate}
    \item Session datasets might contain multiple event types. If so, only the one corresponding to item views is kept.
    \item Sequences are corresponding to user histories in rating datasets, precomputed sessions in Coveo, and sessions with one hour session gap in Rees46 and RetailRocket.
    \item Subsequent repeating items are removed from the sequences, e.g.~$(i,i,j)\rightarrow(i,j)$, but $(i,j,i)$ is unchanged. These are not informative for recommenders because recommending the same item to the user is not useful.
    \item The dataset is iteratively filtered for sequences shorter than 2 and items occurring less than 5 times until there is no change to the dataset. Sequences consisting of one item are useless for this task. The weak item support filter is applied because of the collaborative filtering nature of the algorithms.
    \item Train/test splits are time based. Split time is set so that the size of the test data is satisfactory, but it is at least one day before the last event of the dataset. The test set consists of sequences that started after the split time. The train set consists of events that happened before the split time, i.e.~sequences extending over are cut off.
\end{enumerate}
Analyzing the data reveals that the resolution of timestamps is one day for the Amazon and Steam datasets, which might result in event collision, i.e.~two or more events of the same user having the same timestamp. The order of the events of an event collision can not be determined, and thus their sequence is unreliable and might be changed unintentionally during training or testing. The two rightmost columns in Table~\ref{tab:data_stat} show the proportion of collisions to all user--timestamp pairs and the proportion of participating events to all events. Beside Amazon and Steam, MovieLens10M also has many event collisions, which is extremely problematic, because this dataset comes presorted by user and item ID by default. This means that if user $A$ originally had a rating sequence on a set of items (e.g.~$k\rightarrow i \rightarrow j$), while user $B$ rated the same items in a different order (e.g.~$j\rightarrow k\rightarrow i$), then timestamp collision and the default sorting by item ID together produces the same sequence ($i\rightarrow j\rightarrow k$) for both. This is not the original sequence of user $A$ or $B$ but introduces two instances of an artificial sequence that did not even occur. On a larger scale, this phenomenon introduces artificial sequential patterns even if the data was not sequential originally.

\begin{table*}
    \caption{Recommendation accuracy using the same model with and without sequence modelling}
    \label{tab:seq_vs_noseq}
    \footnotesize
    \begin{tabular}{l|llll|llll|rrrr}
        \toprule
        Dataset & \multicolumn{4}{c|}{Model w/ sequence modelling} & \multicolumn{4}{c|}{Model w/o sequence modelling} &  \multicolumn{4}{c}{Relative change} \\
        & \multicolumn{2}{c}{Recall@N} & \multicolumn{2}{c|}{MRR@N} & \multicolumn{2}{c}{Recall@N} & \multicolumn{2}{c|}{MRR@N} & \multicolumn{2}{c}{Recall@N} & \multicolumn{2}{c}{MRR@N} \\
        & \multicolumn{1}{c}{N=5} & \multicolumn{1}{c}{N=20} & \multicolumn{1}{c}{N=5} & \multicolumn{1}{c|}{N=20} & \multicolumn{1}{c}{N=5} & \multicolumn{1}{c}{N=20} & \multicolumn{1}{c}{N=5} & \multicolumn{1}{c|}{N=20} & \multicolumn{1}{c}{N=5} & \multicolumn{1}{c}{N=20} & \multicolumn{1}{c}{N=5} & \multicolumn{1}{c}{N=20} \\    
        \midrule
        Rees46 & 0.3010 & 0.5293 & 0.1778 & 0.2008 & 0.2594 & 0.4785 & 0.1474 & 0.1694 & -13.80\% & -9.58\% & -17.09\% & -15.67\% \\
        Coveo & 0.1496 & 0.3135 & 0.0852 & 0.1010 & 0.1289 & 0.2678 & 0.0734 & 0.0868 & -13.83\% & -14.59\% & -13.85\% & -14.05\% \\
        Retailrocket & 0.3237 & 0.5186 & 0.1977 & 0.2175 & 0.2747 & 0.4652 & 0.1613 & 0.1806 & -15.13\% & -10.30\% & -18.42\% & -16.97\% \\
        Amazon (Beauty) & 0.0784 & 0.1319 & 0.0527 & 0.0579 & 0.0779 & 0.1271 & 0.0531 & 0.0579 & -0.71\% & -3.61\% & 0.86\% & 0.00\% \\
        MovieLens10M & 0.1728 & 0.3264 & 0.1062 & 0.1211 & 0.1276 & 0.2440 & 0.0763 & 0.0875 & -26.18\% & -25.23\% & -28.16\% & -27.68\% \\
        Steam & 0.1117 & 0.2371 & 0.0662 & 0.0781 & 0.1035 & 0.2208 & 0.0622 & 0.0735 & -7.38\% & -6.87\% & -5.99\% & -5.96\% \\    
        Yelp & 0.0702 & 0.1627 & 0.0371 & 0.0457 & 0.0657 & 0.1625 & 0.0353 & 0.0445 & -6.46\% & -0.12\% & -4.78\% & -2.51\% \\
        \bottomrule
    \end{tabular}
\end{table*}

Table~\ref{tab:seq_vs_noseq} shows recommendation accuracy of a sequential recommender (GRU4Rec) and the same algorithm with the sequence modeling part (GRU) replaced with a feedforward layer. Hyperparameters are optimized separately for the two algorithms and seven datasets, since replacing the GRU layer might alter the optimal parameters. A validation set  -- created from the full training set using the same process as the train/test split -- is used for optimization. Then the algorithm is retrained on the full training set using the optimal parameters, and performance is measured on the test set.

Results show that sequence modeling is not important for the Amazon and Yelp datasets\footnote{Small differences in offline metrics don't translate to any change in online performance due to the proxy nature of the offline setup.}, indicating the lack of sequential patterns and that they are unfit for evaluating sequential algorithms. Sequence modeling has a small positive impact on Steam, despite the daily resolution of the timestamp due to the specific user behavior of its domain. The rest of the datasets greatly benefit from session modeling, suggesting the presence of sequential patterns. However, in the case of MovieLens10M, these are artificial patters resulting from event collisions and presorting.

\subsection{Overzealous preprocessing}\label{ssec:flaw-preproc}

Real-life data is often noisy due to data collection errors, unusual user behavior, bot traffic, etc. Data preprocessing can partially eliminate this noise, enabling better modeling. E.g.~step (3) of the preprocessing described in section~\ref{ssec:flaw-data} and step (2) for Rees46. Preprocessing can also be used to adapt the dataset to the selected task and evaluation setup. E.g.~step (1) and (4) of our preprocessing. However, it is important to consider if preprocessing affects (a) how the outcome of the experiment can be interpreted; (b) whether the results are directly comparable with earlier work; (c) and if the experiment is still suitable to support the articulated claims. 

Modifying only the training set usually does not hurt the generality of claims about model performance, but direct comparison with earlier work and interpretation of the results might become non-trivial, since changes might affect algorithms differently. E.g.~shrinking the training window and using more recent data might benefit recommenders~\cite{tan2016improved} until the uplift from reduced concept drift is balanced out by the degradation from training on less data. Offline evaluation is already biased towards memorization type algorithms, e.g.~neighbor methods. This bias is stronger on smaller training windows due to more static user behavior and because generalization requires more data. Thus, reduced training window is likely to have more severe effect on more complex algorithms with better generalization capabilities. Figure~\ref{fig:gru_vs_sknn} demonstrates this by measuring the performance of the model-based GRU4Rec and the neighbor-based V-SkNN~\cite{ludewig2018evaluation} algorithms using training windows of varying sizes\footnote{Hyperparameters are optimized separately for each training window size and algorithm.}. While the performance of both models decrease as the training window shortens, the model-based method is affected more severely: GRU4Rec outperforms V-SkNN by $5.7\%$ ($8.9\%$) in recall@5 (recall@20) when both are trained on the full training set, but performs $29.6\%$ ($25.1\%$) worse when trained on the last 14 days. Unfortunately, connections between properties of training sets and their effect on the correlation between offline and online performance is not known, thus there is no ``right way'' to set the window size. However, biases like this should be considered when designing an evaluation setup.

Modifying the test set creates an entirely new evaluation setup, and heavy changes might result in less general performance claims. E.g.~evaluating on test users with more than 200 events is informative on the performance on established users only. Real-life recommenders should work for every user, but they can contain multiple algorithms. Therefore, evaluating on certain subsets is not a flaw, if the validity of claims is clearly communicated. Making only the most necessary preprocessing steps is advisable, because the stronger the filtering, the less general the claims can be. Unfortunately, this advice is often ignored~\cite{tang2018personalized,huang2018improving,he2017neural,he2016fast,wang2019neural}. 

\subsection{Information leaking through time}\label{ssec:flaw-split}

\begin{figure*}[!ht]
    \centering
    \begin{subfigure}[b]{.45\textwidth}
        \includegraphics[width=\textwidth]{images/gru_vs_vsknn_flat.pdf}
        \Description[Accuracy measurements over different training windows]{Recall at 5 and 20 values measured for memorization and generalization type algorithms over different training windows.}
        \caption{The effect of using only recent data on the recommendation accuracy of model and neighbor based methods}
        \label{fig:gru_vs_sknn}
    \end{subfigure}
    \begin{subfigure}[b]{.45\textwidth}
        \includegraphics[width=\textwidth]{images/new_rules_flat.pdf}
        \Description[New rule proportion stops decreasing]{After a quick drop, the proportion of new rules stabilizes at a high level.}
        \caption{Proportion of $i\rightarrow j$ item transitions observed first on day $N$ to the number of unique sequences of the same day}
        \label{fig:new_rules}
    \end{subfigure}
    \caption{}
    \label{fig:combined_figure}
\end{figure*}

The next step in the design process is the train/test split. Proper evaluation requires eliminating access to any information that would not be available during inference. E.g.~training examples must not be used for evaluation. Recommenders should also not have access to future information during evaluation. However, some commonly used splitting strategies (e.g.~random split, leave-one-out) inherently enable information leaking through time.

Random splits are appropriate if the preference of users is considered to be long-term and mostly static. This is not the case due to changes in the item catalog, shifting user interest and external factors (e.g.~promotions). Figure~\ref{fig:new_rules} shows the proportion of $i\rightarrow j$ item transitions observed first on day $N$ and the number of unique sequences on the same day. While it declines, it stabilizes on fairly high values, indicating constantly changing user behavior. This concept-drift has been studied by the research community~\cite{gama2014survey,tsymbal2004problem}. Overlapping train/test splits lessen the need of modeling the concept drift, and provide an easier, unrealistic problem for which less generalization is needed. Therefore,  its effect is not uniform over different types of algorithms. 

\begin{figure*}[!h]
    \centering
    \begin{subfigure}[b]{.45\textwidth}
        \includegraphics[width=\textwidth]{images/overlapratio_lo1_vs_time_flat.pdf}
        \Description[Bar plot of shared rules]{Bar plot of shared rules between training and test sets with leave-one-out having higher overlap}
        \caption{Leave-one-out and time based split}
        \label{fig:overlap_lo1_time}
    \end{subfigure}
    \begin{subfigure}[b]{.45\textwidth}
        \includegraphics[width=\textwidth]{images/overlapratio_random_vs_timebased_l1o_flat.pdf}\Description[Bar plot of shared rules]{Bar plot of shared rules between training and test sets with random split having higher overlap}
        \caption{Leave-one-out on random vs. most recent sessions}
        \label{fig:overlap_random_vs_time_lo1}
    \end{subfigure}
    \caption{Proportion of the $i\rightarrow j$ test item transitions that are shared with the training set}
    \label{fig:dataset_split}
\end{figure*}

\emph{Leave-one-out} splitting is often used in the evaluation of sequential recommenders~\cite{kang2018self,sun2019bert4rec,huang2018improving,zhou2020s3,li2020time}. The last event of the training sequences are moved from the train set to the test set. Sequences end at different times, thus the two sets overlap in time. Figure~\ref{fig:dataset_split} shows the concept drift between train and test sets as the proportion of the $i\rightarrow j$ test item transitions that are shared with the train set\footnote{Note that the exact value also depends on the proportion of the training and test sets, therefore the size of the test sets was matched for both comparisons.} for leave-one-out and time based splits (\ref{fig:overlap_lo1_time}); and two variants of the leave-one-out strategy (\ref{fig:overlap_random_vs_time_lo1}) using the most recent and random sequences. Concept drift is significantly smaller for non-overlapping splits of Rees46 and RetailRocket. Coveo has similar proportions for both splits due to consisting of only 18 days of data, which is too short for concept drift to be too prevalent. This clearly indicates that by not requiring strict separation by time, the evaluation setup becomes compromised by information leaking through time, and thus the algorithms are evaluated on a somewhat easier problem than what they would face in a production system.

\subsection{Negative sampling during testing}\label{ssec:flaw-sampling}

Negative item sampling is a common and severe flaw of the final step, testing. Recently,~\cite{krichene2020sampled} discussed how sampling introduces bias to measurements and its potential impact on the comparison of algorithms, while~\cite{canamares2020target,dallmann2021case} demonstrated that it can change the performance based ordering of models. Thus, results achieved with sampling are not reliable. Besides confirming the result of earlier work, we demonstrate how the bias introduced by negative sampling can change the ordering of models, demonstrate its impact, as well as discuss alternatives and whether sampling is needed at all.

This flaw is rooted in the transition from error metrics to IR metrics from a decade ago that brought along a significant increase in the amount of compute needed for evaluation. Opposed to rating prediction that needs only a few score computations per test case, ranking requires scoring all items for every test case, since test items are treated as relevant ones that need to be ranked against other irrelevant items. Different alternatives~\cite{bellogin2011precision} were proposed including ranking over all items and ranking the target(s) against a number of sampled negative items. Ranking over all items is how real-life recommenders work, but nevertheless, negative sampling also became a widespread practice, utilized in well cited papers of prestigious conferences (e.g.~\cite{kang2018self,sun2019bert4rec,huang2018improving,zhou2020s3,he2017neural,cai2022aspect,rashed2022context}).

\begin{figure*}[!h]
    \centering
    \includegraphics[width=\textwidth]{images/sampling_strategies.pdf}
    \Description[Bar plot of recall measurements]{Bar plot of recall measurements showing that sampling strategies result in overestimating accuracy metrics.}
    \caption{Comparison of the strength of various negative samples of 100 items and no sampling.}
    \label{fig:sampling_strategies}
\end{figure*}

Metrics focus on the top few items, thus negative items should be highly ranked for sampled and full ranking results to be similar. Uniform sampling is not able to provide strong samples if the size of the item catalog is significantly larger than the number of samples. The probability of a target item ranked $R$ in the full ranking of $N$ items making it into the recommendation list of $C$ items ($C<R$) when ranked against $S$ negative samples is $\sum_{i=0}^{C-1}{\binom{R-1}{i}\binom{N-R}{S-i}/\binom{N-1}{S}}$. If the target item is ranked $1,490^{\textrm{th}}$ among $10,000$ items or $14,878^{\textrm{th}}$ among $100,000$, it still has more than $90\%$ chance to make it into the top $20$ with $100$ uniform samples. Weak negative samples are easily distinguishable from the target and thus their use results in the severe overestimation of offline accuracy metrics. Using stronger samples provides a more accurate estimation of the accuracy of the non sampled setup. The strength of negative samples -- i.e.~how hard it is to distinguish them from the target -- is depicted on Figure~\ref{fig:sampling_strategies} for 8 sampling strategies:
\begin{itemize}
    \item \textbf{Sampling strategies:} Sampling probability is either uniform or proportional to the items' support. Popular items are considered to be stronger samples, since recommenders tend to rank them higher. 
    \item \textbf{Most popular items:} The target items are ranked against the $100$ most popular items.
    \item \textbf{Similar/close items:} Soft upper bound. Requires at least as much compute as full ranking. Items are selected based on the (cosine) similarity or closeness of their embeddings to that of the target item.
    \item \textbf{Weak baselines:} Soft lower bound. Sampling with probability proportional to reciprocal item support, and items with least similar or farthest embeddings to the target item's embedding. 
\end{itemize}
Figure~\ref{fig:sampling_strategies} indicates that even the computationally expensive most similar and closest approaches can not produce strong item sets consistently. Popularity sampling and most popular items are better than uniform sampling, but these approaches are still not able to estimate the true rank of the target. This can falsify the results, because (1) the difference between the performance of two algorithms can vanish due to the lack of challenging negative items. (2) The relative performance at recommendation list length $M$ shifts to length $N(\ll M)$ because sampling makes it easier to push the target item up on the list. Since the relative performance of two models might depend on the length of the recommendation list, this can also change their ordering for relevant list lengths. 

\begin{figure*}[!h]
    \centering
    \begin{subfigure}[b]{.32\textwidth}
        \includegraphics[width=\textwidth]{images/coveo_all_random_10_random_0.0_random_0.1_recall_diff_large_tall.pdf}
        \Description[Recall values at different list length]{Recall values at different list length. Algorithm ranking changes at smaller list length as sample size decreases.}
        \caption{Coveo -- Recall@N}
        \label{fig:coveo_recall}
    \end{subfigure}
    \begin{subfigure}[b]{.32\textwidth}
        \includegraphics[width=\textwidth]{images/retailrocket_all_random_10_random_1_random_0.0_random_0.1_recall_diff_large_tall.pdf}
        \Description[Recall values at different list length]{Recall values at different list length. Algorithm ranking changes at smaller list length as sample size decreases.}
        \caption{Retailrocket -- Recall@N}
        \label{fig:retailrocket_recall}
    \end{subfigure}
    \begin{subfigure}[b]{.32\textwidth}
        \includegraphics[width=\textwidth]{images/rees46_optuna_mrr_xe_noembed_localsearch100_bprmax_embed_all_random_10_random_1_random_0.0_recall_diff_large_tall.pdf}
        \Description[Recall values at different list length]{Recall values at different list length. Algorithm ranking changes at smaller list length as sample size decreases.}
        \caption{Rees46 -- Recall@N -- AB}
        \label{fig:rees46_recall_AB}
    \end{subfigure}
    \begin{subfigure}[b]{.32\textwidth}
        \includegraphics[width=\textwidth]{images/rees46_localsearchbig_bprmax_constrained_localsearch100_bprmax_embed_all_random_10_random_1_random_0.0_recall_diff_large_tall.pdf}
        \Description[Recall values at different list length]{Recall values at different list length. Algorithm ranking changes at smaller list length as sample size decreases.}
        \caption{Rees46 -- Recall@N -- BC}
        \label{fig:rees46_recall_BC}
    \end{subfigure}
    \begin{subfigure}[b]{.32\textwidth}
        \includegraphics[width=\textwidth]{images/rees46_optuna_mrr_xe_noembed_localsearch100_bprmax_embed_localsearchbig_bprmax_constrained_all_random_10_random_1_random_0.0_combined_recall_diff_large_tall.pdf}
        \Description[Recall values at different list length]{Recall values at different list length. Algorithm ranking changes at smaller list length as sample size decreases.}
        \caption{Rees46 -- Recall@N -- ALL}
        \label{fig:rees46_recall_ABC}
    \end{subfigure}
    \begin{subfigure}[b]{.32\textwidth}
        \includegraphics[width=\textwidth]{images/rees46_optuna_mrr_xe_noembed_localsearch100_bprmax_embed_localsearchbig_bprmax_constrained_all_random_10_random_1_random_0.0_combined_mrr_diff_large_tall.pdf}
        \Description[MRR values at different list length]{MRR values at different list length. Algorithm ranking changes as sample size decreases.}
        \caption{Rees46 -- MRR@N -- ALL}
        \label{fig:rees46_mrr}
    \end{subfigure}
    \caption{Accuracy as the function of recommendation list length, with and without sampling}
    \label{fig:metrics_at_n}
\end{figure*}

Figure~\ref{fig:metrics_at_n} demonstrates this point by showing recall@N as the function of $N$ for pairs of models for full ranking ($100\%$), and for randomly sampling $10\%$, $1\%$, $0.1\%$ of the item catalog or $100$ items. The ordering of models is not static over $N$, and it might change multiple times (marked by red dots). Decreasing sample size moves these intersections to the left, pushing the change in model ordering to lower $N$ values. Changes to the relative performance are drastic: model A on Coveo (\ref{fig:coveo_recall}) outperforms model B by $8.1\%$ in recall@20, but A underperforms B by $4.7\%$ when $100$ negative samples are used. While the ordering of model A and B does not change for Retailrocket (\ref{fig:retailrocket_recall}), $14.3\%$ uplift in recall@20 vanishes and drops to $1.5\%$. The biggest change can be observed between model A and B on Rees46 (\ref{fig:rees46_recall_AB}) where the uplift in recall@1 (and MRR@1) drops from $49.5\%$ to $-1.6\%$. Since MRR is even more sensitive to the top of the list, this affects relative MRR based performance for any $N$: the true order of models A, B and C on Rees46 (\ref{fig:rees46_mrr}) is $A>C\gg B$, but with $100$ samples it reads as $B>A\gg C$.

Computational complexity of evaluation via full ranking depends on the number of test recommendations and the size of the item catalog. Most public dataset are quite small, and even the bigger ones have item catalogs of easily manageable sizes. One test on Rees46 requires $887,369$ recommendation lists over $172,756$ items each that is executed in only $205.1$ seconds on an A30 GPU. Even before GPUs, evaluation on public datasets of the time could be executed in a few minutes using optimized code. If a dataset is really too large for quick experimentation, the sampling of test recommendations (e.g.~users) gives more representative results. Certain scoring models are not designed for full ranking. If one such model must be evaluated on ranking, generating candidate sets via another algorithm can solve the problem. However, this limits the generality of claims about its performance, because currently there are no universally accepted candidate set generators.

\section{Related Work}
\paragraph{Instruction Tuning}

Instruction tuning regulates LLMs to accurately comprehend and interpret natural language instructions. Prior works in this field focus on reformulating NLP tasks using the templates of instructions. \citet{wei2021finetuned} pioneered to show that fine-tuning LLMs on large collections of tasks formatted in instructions enables the model to generalize to unseen tasks in a zero-shot manner. Since then, there is a surge of interest in manually constructing high-quality instruction datasets by first reformulating the formats of existing NLP datasets and then merging them~\citep{mishra2022cross, bach2022promptsource, ye2021crossfit, ouyang2022training}. Another line of study~\citep{longpre2023flan,iyer2022opt} demonstrates that scaling the number of training tasks and task diversity can further enhance the model's generalization performance. However, all these works directly mix all the existing instruction datasets while ignoring the potential issue of format inconsistency. Instead of investigating the number and diversity of training instructions, we instead explore an under-explored facet, i.e., the instruction format of instruction tuning, and investigate its impact on generalization.

\paragraph{Data Augmentation}
Besides manually curating instruction tuning datasets, \citet{honovich2022unnatural} show that fine-tuning LLMs with machine-generated instruction tuning data achieves excellent performance compared with human-written data, indicating that data augmentation is an effective method to enhance the data quantity and task diversity, which overcomes the time-consuming issues of human annotation. Recently, \citet{alpaca, peng2023instruction, ding2023enhancing} adopt machine-annotation method~\citep{wang2022self} to generate real-world human instructions (rather than instructions that describe NLP tasks) and model responses based on powerful LLMs such as ChatGPT. Similarly, in this paper, we also leverage  LLMs for automatic format transfer and data augmentation. Since real-world instructions are quite diverse and hard to annotate their formats, we instead focus on instructions that describe NLP tasks to rigorously study the effects of instruction format. We believe the derived findings can potentially be applied to real-world instructions in the future.

\paragraph{Synthetic Data Denoising}
Generative models are commonly utilized for data augmentation~\citep{alpaca}. However, these synthetic datasets are not always as reliable as those human-annotated ones, and filtering out noisy examples can boost the model performance~\citep{le2020adversarial}. Recent studies have suggested different approaches for denoising. For instance, \citet{yang-etal-2020-generative,fang2022pseudoreasoner} adopted influence functions~\citep{koh2017understanding} to evaluate the quality of the synthetic data; \citet{wang-etal-2022-promda} employ the NLU Consistency Filtering~\citep{koh2017understanding} to filter out low-quality samples.
In our research, we utilized LLMs for instruction format transfer, which may introduce noise throughout the process. To overcome this challenge, we adopted a simple and effective perplexity scoring strategy to denoise our auto-constructed dataset (\cref{denoising_method}).

%******************************************************************************%
%******************************    CONCLUSION   *******************************%
%******************************************************************************%
\section{Conclusion}
In this study, a novel temperature control method called APaTheCSys was evaluated in real space equipment, operated under ground conditions. 
After an exploration of 5 hours, APaTheCSys was able to maintain a simulated high load for 12 hours without exceeding an imposed threshold of 55 degrees Celsius starting from an idle temperature of 50 degrees Celsius by adjusting the number of cpus available for processing and their frequency dynamically.
The gents must interact with the environment several times before coming up with a viable policy, particularly at the beginning, during the cold start, as the initial performance of agents on real hardware was not better than using a random policy. 

In this study, a condensed simulated environment was used for the initial validation of the algorithm. Future experimentation should address the utility of using these reduced environments to pretrain the policy to develop acceptable initial behavior, then allowing the agent to refine the solution in response to actual environmental conditions.

Since it is not possible to adjust the computing power to regulate the temperature of the equipment if no workload is present, other traditional thermal control systems should be used in conjunction to ensure the nominal temperature conditions of the equipment and avoid damage to sensitive components. In this context, APaTheCSys should be seen as an auxiliary thermal control method that maximizes the efficiency with which equipment is used in proportion to the dynamic conditions of the surrounding environment.

The experimentation was confined to controlling the processing power of the CPUs contained in the SoC used, but activities such as turning off other equipment or changing their energy modes might be included in the set of actions to regulate the temperature of the subsystems. However, the addition of too many dimensions to the action and state spaces increases complexity and might make it difficult for learning models to converge.

Overall, the method could be suitable for small payloads from satellites, spacecraft, or even planetary probes on upcoming space exploration missions, provided they are capable of hosting onboard intelligence and have the ability to control the hardware from a software interface. And this, like other active lines of research such as autonomous systems for collision avoidance, docking or active debris removal constitutes a further step towards autonomous robotic missions.

Finally, APaTheCSys will be evaluated under space conditions in the IMAGIN-e future space edge computing mission, that will be hosted in the ISS.


\begin{acks}
  The work leading to these results received funding from \grantsponsor{PIACIKFI}{National Research, Development and Innovation Office, Hungary}~~ under grant agreement number \grantnum{PIACIKFI}{2020-1.1.2-PIACI-KFI-2021-00289}.

  The authors would also like to thank Domonkos Tikk for his valuable support as the project leader of the above-mentioned R\&D grant.
\end{acks}

\bibliographystyle{ACM-Reference-Format}
\bibliography{references}

\end{document}
\endinput
