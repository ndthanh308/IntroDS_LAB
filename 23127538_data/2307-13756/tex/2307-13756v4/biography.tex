%!TEX root = geotrans.tex

% biography section
% 
% If you have an EPS/PDF photo (graphicx package needed) extra braces are
% needed around the contents of the optional argument to biography to prevent
% the LaTeX parser from getting confused when it sees the complicated
% \includegraphics command within an optional argument. (You could create
% your own custom macro containing the \includegraphics command to make things
% simpler here.)
%\begin{IEEEbiography}[{% Figure removed}]{Michael Shell}
% or if you just want to reserve a space for a photo:
\vspace{-1.15cm}
\begin{IEEEbiography}[{% Figure removed}]{Jingjia Shi} is a PhD student at the College of Computer, National University of Defense Technology (NUDT), Changsha, China. Her research interests focus on learning based 3D vision, including structure-aware reconstruction, pose estimation, and 3D representation learning.
\end{IEEEbiography}
\vspace{-1.15cm}
\begin{IEEEbiography}[{% Figure removed}]{Shuaifeng Zhi}  is currently a Lecturer (Assistant Professor) at the College of Electronic Science and Technology, National University of Defense Technology (NUDT), Changsha, China. He received his Ph.D. degree in Computing Research at the Dyson Robotics Laboratory, Imperial College London, UK, in 2021. He was a 6-month visiting student in 5GIC, University of Surrey, UK, in 2015. His current research interests focus on robot vision, particularly on scene understanding, neural scene representation, and semantic SLAM. He also serves on the editorial board of The Visual Computer.
\end{IEEEbiography}
\vspace{-1.15cm}
\begin{IEEEbiography}[{% Figure removed}]{Kai Xu}
is a Professor at the College of Computer, NUDT, where he received his Ph.D. in 2011. He conducted visiting research at Simon Fraser University and Princeton University. His research interests include geometric modeling and shape analysis, especially on data-driven approaches to the problems in those directions, as well as 3D vision and its robotic applications. He has published over 80 research papers, including 20+ SIGGRAPH/TOG papers. He has co-organized several SIGGRAPH Asia courses and Eurographics STAR tutorials. He serves on the editorial board of ACM Transactions on Graphics, Computer Graphics Forum, Computers \& Graphics, and The Visual Computer. He also served as program co-chair of CAD/Graphics 2017, ICVRV 2017 and ISVC 2018, as well as PC member for several prestigious conferences including SIGGRAPH, SIGGRAPH Asia, Eurographics, SGP, PG, etc. His research work can be found in his personal website: www.kevinkaixu.net.
\end{IEEEbiography}

% You can push biographies down or up by placing
% a \vfill before or after them. The appropriate
% use of \vfill depends on what kind of text is
% on the last page and whether or not the columns
% are being equalized.

% \vfill

% Can be used to pull up biographies so that the bottom of the last one
% is flush with the other column.
%\enlargethispage{-5in}

