%% bare_jrnl_compsoc.tex
%% V1.4b
%% 2015/08/26
%% by Michael Shell
%% See:
%% http://www.michaelshell.org/
%% for current contact information.
%%
%% This is a skeleton file demonstrating the use of IEEEtran.cls
%% (requires IEEEtran.cls version 1.8b or later) with an IEEE
%% Computer Society journal paper.
%%
%% Support sites:
%% http://www.michaelshell.org/tex/ieeetran/
%% http://www.ctan.org/pkg/ieeetran
%% and















%% http://www.ieee.org/

%%*************************************************************************
%% Legal Notice:
%% This code is offered as-is without any warranty either expressed or
%% implied; without even the implied warranty of MERCHANTABILITY or
%% FITNESS FOR A PARTICULAR PURPOSE!
%% User assumes all risk.
%% In no event shall the IEEE or any contributor to this code be liable for
%% any damages or losses, including, but not limited to, incidental,
%% consequential, or any other damages, resulting from the use or misuse
%% of any information contained here.
%%
%% All comments are the opinions of their respective authors and are not
%% necessarily endorsed by the IEEE.
%%
%% This work is distributed under the LaTeX Project Public License (LPPL)
%% ( http://www.latex-project.org/ ) version 1.3, and may be freely used,
%% distributed and modified. A copy of the LPPL, version 1.3, is included
%% in the base LaTeX documentation of all distributions of LaTeX released
%% 2003/12/01 or later.
%% Retain all contribution notices and credits.
%% ** Modified files should be clearly indicated as such, including  **
%% ** renaming them and changing author support contact information. **
%%*************************************************************************


% *** Authors should verify (and, if needed, correct) their LaTeX system  ***
% *** with the testflow diagnostic prior to trusting their LaTeX platform ***
% *** with production work. The IEEE's font choices and paper sizes can   ***
% *** trigger bugs that do not appear when using other class files.       ***                          ***
% The testflow support page is at:
% http://www.michaelshell.org/tex/testflow/


\documentclass[10pt,journal,compsoc]{IEEEtran}

% packages
\usepackage{graphicx}
\usepackage{amsmath}
\usepackage{amssymb}
\usepackage{booktabs}

\usepackage{multirow}
\usepackage{xcolor}
\usepackage{array}
\usepackage{multirow}
\usepackage{mathtools}
\usepackage{overpic}
\usepackage{stmaryrd}
\usepackage{url}
\usepackage{xspace}
\usepackage[ruled]{algorithm2e}

% \usepackage[pagebackref,breaklinks,colorlinks]{hyperref}
\usepackage{hyperref}
\usepackage{times}
\usepackage{epsfig}
\usepackage{bbm}
\usepackage{bbding}
\usepackage{subfigure}
\usepackage{amsthm}
\usepackage{makecell}
\usepackage{mathrsfs}

\newcommand{\compresslist}{%
 \setlength{\itemsep}{0pt}%
 \setlength{\parskip}{0pt}%
 \setlength{\parsep}{0pt}%
 }
\usepackage[capitalize]{cleveref}
\crefname{section}{Sec.}{Secs.}
\Crefname{section}{Section}{Sections}
\crefname{table}{Tab.}{Tabs.}
\Crefname{table}{Table}{Tables}
\crefname{figure}{Fig.}{Figs.}
\Crefname{figure}{Figure}{Figures}
\crefname{equation}{Eq.}{Eqs.}
\Crefname{equation}{Equation}{Equations}
\crefname{appendix}{Appx.}{Appxs.}
\Crefname{Appendix}{Appendix}{Appendices}
\crefname{algorithm}{Alg.}{Algs.}
\Crefname{algorithm}{Algorithm}{Algorithms}

\ifCLASSOPTIONcompsoc
  \usepackage[nocompress]{cite}
\else
  \usepackage{cite}
\fi


% my command
%!TEX root = geotrans.tex

\hyphenation{sinusoi-dal ac-cepts acc-epts Trans-former Co-FiNet CoFi-Net sce-narios scena-rios domi-nated Al-though Alth-ough gene-rate genera-te hie-rarchical em-bedding Geo-metric gene-rates}

% match
\DeclareMathOperator*{\argmax}{\arg\!\max}
\DeclareMathOperator*{\argmin}{\arg\!\min}

% abbr
\makeatletter
\DeclareRobustCommand\onedot{\futurelet\@let@token\@onedot}
\def\@onedot{\ifx\@let@token.\else.\null\fi\xspace}

\def\eg{\emph{e.g}\onedot}
\def\Eg{\emph{E.g}\onedot}
\def\ie{\emph{i.e}\onedot}
\def\Ie{\emph{I.e}\onedot}
\def\cf{\emph{cf}\onedot}
\def\Cf{\emph{Cf}\onedot}
\def\etc{\emph{etc}\onedot}
\def\vs{\emph{vs}\onedot}
\def\wrt{w.r.t\onedot}
\def\dof{d.o.f\onedot}
\def\iid{i.i.d\onedot}
\def\wolog{w.l.o.g\onedot}
\def\etal{\emph{et al}\onedot}
\makeatother

% algorithm
\SetKwInOut{KwParam}{Parameter}

% format
\newcommand{\todo}[1]{\textcolor{red}{#1}}
\newcommand{\secname}[1]{\textcolor{red}{#1}}
\newcommand{\head}[1]{\noindent\textbf{#1}}
\newcommand{\tight}[1]{\hspace{1pt}{#1}{\hspace{1pt}}}
\newcommand{\medium}[1]{\hspace{2pt}{#1}{\hspace{2pt}}}
\newcommand{\wide}[1]{\hspace{3pt}{#1}{\hspace{3pt}}}
\newcommand{\slfrac}[2]{\left.#1\middle/#2\right.}
\newcommand{\tabincell}[2]{\begin{tabular}{@{}#1@{}}#2\end{tabular}}
\newcolumntype{P}[1]{>{\centering\arraybackslash}p{#1}}
\newlength{\wdth}
\newcommand{\strike}[1]{\settowidth{\wdth}{#1}\rlap{\rule[.5ex]{\wdth}{.4pt}}#1}
\newcommand{\ptitle}[1]{\noindent\textbf{#1}\hspace{5pt}}
% \newcommand\notsotiny{\@setfontsize\notsotiny{6}{7}}

\newcommand{\kx}[1]{{\color{red}#1}}
\newcommand{\tofix}[1]{{\color{red}[Fix: #1]}}
\newcommand{\num}{{\color{blue}[X]~}}
\newcommand{\supl}[1]{{\color{black}\emph{#1}}}
\newcommand{\fix}[1]{{\color{blue}#1}}
\newcommand{\paraspace}{\vspace{0pt}}
% \newcommand{\fix}[1]{{#1}}
% \newcommand{\revised}[1]{{\color{blue}#1}}
\newcommand{\revised}[1]{#1}

% \newcommand{\jiajia}[1]{\textcolor{blue}{#1}}
\newcommand{\jiajia}[1]{\textcolor{black}{#1}}


% document
\begin{document}

% \title{PlaneRecTR++: Unified Query Learning for Simultaneous Relative Pose Estimation and Planar Reconstruction}

\title{PlaneRecTR++: Unified Query Learning for Joint 3D Planar Reconstruction and Pose Estimation}


% PlaneRecTR++, Plane Query Learning for single stage/end-to-end 3D Planar Reconstruction, Simultaneous?, Matching-Free? Relative Pose Estimation and Planar Reconstruction
% Plane Query Learning for single stage/end-to-end 3D Planear Reconstruction 
% Unified Query Learning focomr Simultaneous Relative Pose Estimation and Planar Reconstruction
% PlaneRecTR++: Unified Plane Query Learning for Simultaneous Relative Pose Estimation and 3D Reconstruction
%
%
% author names and IEEE memberships
% note positions of commas and nonbreaking spaces ( ~ ) LaTeX will not break
% a structure at a ~ so this keeps an author's name from being broken across
% two lines.
% use \thanks{} to gain access to the first footnote area
% a separate \thanks must be used for each paragraph as LaTeX2e's \thanks
% was not built to handle multiple paragraphs
%
%
%\IEEEcompsocitemizethanks is a special \thanks that produces the bulleted
% lists the Computer Society journals use for "first footnote" author
% affiliations. Use \IEEEcompsocthanksitem which works much like \item
% for each affiliation group. When not in compsoc mode,
% \IEEEcompsocitemizethanks becomes like \thanks and
% \IEEEcompsocthanksitem becomes a line break with idention. This
% facilitates dual compilation, although admittedly the differences in the
% desired content of \author between the different types of papers makes a
% one-size-fits-all approach a daunting prospect. For instance, compsoc
% journal papers have the author affiliations above the "Manuscript
% received ..."  text while in non-compsoc journals this is reversed. Sigh.

\author{
Jingjia~Shi*, Shuaifeng~Zhi*$^\dag$, Kai~Xu$^\dag$% <-this % stops a space
% \IEEEcompsocitemizethanks{
% \IEEEcompsocthanksitem The first two authors contributed equally to this work.
% \IEEEcompsocthanksitem  Shuaifeng Zhi and Kai Xu are corresponding authors..
% }% <-this % stops an unwanted space
\thanks{* The first two authors contributed equally to this work.}
\thanks{$^\dag$ Shuaifeng Zhi and Kai Xu are corresponding authors.}
\thanks{Jingjia Shi, Shuaifeng Zhi and Kai Xu are with National University of Defense Technology, Changsha, China.}
}

% note the % following the last \IEEEmembership and also \thanks -
% these prevent an unwanted space from occurring between the last author name
% and the end of the author line. i.e., if you had this:
%
% \author{....lastname \thanks{...} \thanks{...} }
%                     ^------------^------------^----Do not want these spaces!
%
% a space would be appended to the last name and could cause every name on that
% line to be shifted left slightly. This is one of those "LaTeX things". For
% instance, "\textbf{A} \textbf{B}" will typeset as "A B" not "AB". To get
% "AB" then you have to do: "\textbf{A}\textbf{B}"
% \thanks is no different in this regard, so shield the last } of each \thanks
% that ends a line with a % and do not let a space in before the next \thanks.
% Spaces after \IEEEmembership other than the last one are OK (and needed) as
% you are supposed to have spaces between the names. For what it is worth,
% this is a minor point as most people would not even notice if the said evil
% space somehow managed to creep in.



% The paper headers
% \markboth{Journal of \LaTeX\ Class Files,~Vol.~14, No.~8, August~2015}%
% {Qin \MakeLowercase{\textit{et al.}}: Learning Fast and Robust Point Cloud Registration with Geometric Transformer}
% \markboth{Submitted To IEEE Transactions on Pattern Analysis and Machine Intelligence}%
% {Shi \MakeLowercase{\textit{et al.}}: PlaneRecTR++}
% The only time the second header will appear is for the odd numbered pages
% after the title page when using the twoside option.
%
% *** Note that you probably will NOT want to include the author's ***
% *** name in the headers of peer review papers.                   ***
% You can use \ifCLASSOPTIONpeerreview for conditional compilation here if
% you desire.



% The publisher's ID mark at the bottom of the page is less important with
% Computer Society journal papers as those publications place the marks
% outside of the main text columns and, therefore, unlike regular IEEE
% journals, the available text space is not reduced by their presence.
% If you want to put a publisher's ID mark on the page you can do it like
% this:
%\IEEEpubid{0000--0000/00\$00.00~\copyright~2015 IEEE}
% or like this to get the Computer Society new two part style.
%\IEEEpubid{\makebox[\columnwidth]{\hfill 0000--0000/00/\$00.00~\copyright~2015 IEEE}%
%\hspace{\columnsep}\makebox[\columnwidth]{Published by the IEEE Computer Society\hfill}}
% Remember, if you use this you must call \IEEEpubidadjcol in the second
% column for its text to clear the IEEEpubid mark (Computer Society jorunal
% papers don't need this extra clearance.)



% use for special paper notices
%\IEEEspecialpapernotice{(Invited Paper)}



% for Computer Society papers, we must declare the abstract and index terms
% PRIOR to the title within the \IEEEtitleabstractindextext IEEEtran
% command as these need to go into the title area created by \maketitle.
% As a general rule, do not put math, special symbols or citations
% in the abstract or keywords.
\IEEEtitleabstractindextext{%
\begin{abstract}

The Fast Reciprocal Square Root Algorithm is a well-established approximation technique consisting of two stages: first, a coarse approximation is obtained by manipulating the bit pattern of the floating point argument using integer instructions, and second, the coarse result is refined through one or more steps, traditionally using Newtonian iteration but alternatively using improved expressions with carefully chosen numerical constants found by other authors. The algorithm was widely used before microprocessors carried built-in hardware support for computing reciprocal square roots. At the time of writing, however, there is in general no hardware acceleration for computing other fixed fractional powers. This paper generalises the algorithm to cater to all rational powers, and to support any polynomial degree(s) in the refinement step(s), and under the assumption of unlimited floating point precision provides a procedure which automatically constructs provably optimal constants in all of these cases. It is also shown that, under certain assumptions, the use of monic refinement polynomials yields results which are much better placed with respect to the cost/accuracy tradeoff than those obtained using general polynomials. Further extensions are also analysed, and several new best approximations are given.

\end{abstract}

}


% make the title area
\maketitle


% To allow for easy dual compilation without having to reenter the
% abstract/keywords data, the \IEEEtitleabstractindextext text will
% not be used in maketitle, but will appear (i.e., to be "transported")
% here as \IEEEdisplaynontitleabstractindextext when the compsoc
% or transmag modes are not selected <OR> if conference mode is selected
% - because all conference papers position the abstract like regular
% papers do.
\IEEEdisplaynontitleabstractindextext
% \IEEEdisplaynontitleabstractindextext has no effect when using
% compsoc or transmag under a non-conference mode.



% For peer review papers, you can put extra information on the cover
% page as needed:
% \ifCLASSOPTIONpeerreview
% \begin{center} \bfseries EDICS Category: 3-BBND \end{center}
% \fi
%
% For peerreview papers, this IEEEtran command inserts a page break and
% creates the second title. It will be ignored for other modes.
\IEEEpeerreviewmaketitle


% Figure environment removed

\section{Introduction}
Automatic 3D reconstruction of clothed humans using image inputs has gained increasing significance due to its potential applications in a wide array of AR/VR scenarios. High-fidelity reconstructions typically depend on sophisticated capture systems, which are developed with dense camera arrays~\cite{collet2015high,joo2015panoptic,joo2018total}, programmable light-stages~\cite{Vlasic2009, guo2019relightables}, and depth sensors~\cite{newcombe2011kinectfusion,DoubleFusion,BodyFusion,dou2016fusion4d,newcombe2015dynamicfusion}. However, stringent capture environments equipped with complex hardware pose significant challenges for consumer-level applications.


In this context, considerable research effort has been dedicated to developing methods that allow for more flexible capture configurations, such as utilizing a few RGB inputs. Among these works, learning implicit functions \cite{iccv2020PIFu, saito2020pifuhd, hong2021stereopifu} has proven effective in achieving highly detailed reconstructions by integrating the advancements of deep neural networks. These methods employ large multi-layer perceptrons (MLPs) to predict the occupancy probability or truncated signed distance function (TSDF) value of every queried 3D point based on its associated local feature, which is extracted from images. They can recover a continuous surface at arbitrary resolutions without topology restrictions.


However, in typical MLP-based implicit networks, the occupancy or TSDF value at each location is solved independently with planar image features, rendering them less capable of addressing challenging cases such as occlusions. Consequently, these methods suffer from generalization and robustness issues, particularly when tackling strong occlusions caused by large motion or multiple interacting humans. 
Some follow-up studies  \cite{zheng2021deepmulticap,zheng2021pamir,huang2020arch} utilize an extra geometric model, SMPL~\cite{Loper2015}, to improve robustness by introducing strong shape priors. 
Their success typically relies on the assumption of geometrical similarity \cite{huang2020arch} between the shape prior and target reconstruction, making them intractable for handling complex cases with loose clothes and sensitive to errors in SMPL model fitting.



%\ping{this paragraph sounds like `TSDF is better than MLP/SMPL, and we use TSDF to solve the problem'. But in Sec 3, we are telling a different story, saying `MLP needs a 3D convolutional encoder'. We need to make these two sections consistent.}\sicong{I think in this paragraph we claim that the TSDF}


%We opt for Trucated Signed Distance Funtion (TSDF) volumetric representations as they are naturally suitable for convolution operations, which have shown remarkable performance for learning hierarchical features on 2D visual perception tasks \cite{SunXLW19}. 
%Meanwhile, TSDF also describes the gradual geometry change around shape surface, which is not reflected by occupancy volume. 

We instead revisit the 3D volumetric representation and resort to 3D convolutional neural networks (CNNs) for feature learning, due to their impressive performance in feature learning and the ability to incorporate spatial context. However, volumetric methods and 3D convolution involve discretization, which might raise concerns regarding whether a discretized volume can preserve subtle geometric details as continuous representations learned in implicit functions. We investigate the relationship between volume resolution and quantization error on synthetic data by converting target mesh objects to TSDF volumes, as shown in Figure~\ref{fig:quantization_error}. We observe that the quantization errors are significantly reduced by increasing volume resolution and become nearly negligible when reaching a relatively high resolution (e.g., 512 or higher). In other words, achieving fine-detailed reconstruction is not supposed to be restricted by the use of volume representations as long as a proper volume resolution is utilized. Therefore, we present a method with high-resolution feature volumes, e.g., 256 and 512, while traditional volumetric methods \cite{varol18_bodynet,gilbert2018volumetric} are often limited to much lower resolutions, such as 32 or 128.



On the other hand, an increase in volume resolution may lead to a cubic growth of memory overhead \cite{8100085}. Reducing memory costs while guaranteeing the granularity of volumetric representations is necessary for pursuing high-quality reconstruction. Thus, we adopt a coarse-to-fine approach and cull away irrelevant voxels to build a sparse high-resolution feature volume. At the coarse level, the network computes an initial TSDF by applying a U-Net with sparse 3D CNN \cite{3DSemanticSegmentationWithSubmanifoldSparseConvNet} on the sparse feature volume, which is carved by a visual hull. Through our experiments, it turns out that more than 95\% of the volume grids are discarded by the visual hull culling, making the sparse 3D CNN efficient. At the fine level, the network focuses on a narrow band near the zero-level set of the initial TSDF and discretizes the narrow band with smaller voxels. By employing this narrow-band culling, we further shrink the sampling space, resulting in a relatively small range of grid numbers (usually 300K--500K in our experiments) even with a high volume resolution of 512. The remaining voxels in the narrow band are associated with features that fuse high-frequency information from the computed normal maps upon the low-frequency shape from the coarse level to compute the TSDF at high resolution. The final mesh is then extracted from the TSDF using the Marching-Cube algorithm ~\cite{Lorensen87marchingcubes}.
% Different from the u-net sturcture to preserve global topology context, we then apply a shallow 3dcnn to compute the final TSDF $D_{final}$ which contain more local geometry detail.




% \ping{this paragraph can be expanded. It is an important contribution and often ignored by other works. stress on the novel idea of regressing blending weights instead of colors}

In addition to geometry, high-quality mesh texture is also a crucial factor contributing to visual appearance. Directly computing a color field in 3D space, as in \cite{iccv2020PIFu}, struggles to capture high-frequency texture details, while the neural radiance field (NeRF) \cite{yu2020pixelnerf} or the DoubleField~\cite{shao2022doublefield} require expensive per-instance optimization and are often unstable for sparse input images. In contrast, we adopt an image-based rendering approach to compute a texture atlas map, which is efficient and widely supported in existing computer graphics tools. 
Specifically, we compute a blending weight at each 3D point on the mesh surface to determine its color as a weighted average of the colors at its image projections. The blending weights can be computed at a relatively coarse resolution, e.g., 512 volume resolution in our case, and leave texture details to the high-resolution images, such as 1K or 2K. Unlike previous methods that generate blurry texturing results under sparse input, our method generalizes well on both synthetic and real data with just a few input views. 
Figure~\ref{fig:teaser} shows two examples reconstructed by our method. Despite the challenging garment, pose, and occlusion, our method recovers faithful shape, normal, and texture on the right.

%with a wide variety of poses and clothing styles, and it is also adaptive to handle input image with arbitrary resolutions.
%\sicong{For this concern we claim that when the resolution of dicretized volume meets certain threshold (which is 256 in our experiment), the quantization error can be neglected.} 



In summary, the main contributions of this paper are as follows:
\begin{itemize}
\vspace{-0.1in}
  \item 
  We revisit the 3D volumetric representation and demonstrate that it can support clothed human reconstruction with equal or even better performance compared to implicit representation. 
  \item 
  We develop a memory and computation-efficient method for high-resolution volumetric reconstruction using sophisticated sparse 3D CNN, coarse-to-fine estimation, and voxel culling by visual hull and narrow bands. 
  \item 
  We introduce a novel method to compute a texture atlas map, which captures rich appearance details from high-resolution input images.
  \item 
  We achieve impressive results on standard benchmark datasets Twindom and MultiHuman, significantly reducing the point-2-surface (P2S) precision to approximately 0.2cm from just six input views, with more than $50\%$ error reduction compared to the state-of-the-art methods, including DoubleField~\cite{shao2022doublefield} and PIFuHD~\cite{saito2020pifuhd}.
\end{itemize}
\section{Related Work}
\label{appsec: related work}
Bayesian causal discovery literature has primarily focused on inference in linear models with closed-form posteriors or marginalized parameters. Early works considered sampling directed acyclic graphs (DAGs) for discrete~\cite{cooper1992bayesian, madigan1995bayesian, heckerman2006bayesian} and Gaussian random variables~\cite{friedman2003being, tong2001active} using Markov chain Monte Carlo (MCMC) in the DAG space. However, these approaches exhibit slow mixing and convergence~\cite{eaton2012bayesian,grzegorczyk2008improving}, often requiring restrictions on number of parents~\cite{kuipers2017partition}. %Alternative exact dynamic programming methods are limited to small settings~\cite{koivisto2012advances}. 

Recent advances in variational inference~\cite{zhang2018advances} have facilitated graph inference in DAG space, with gradient-based methods employing the NOTEARS DAG penalty \cite{zheng2018dags}.\cite{annadani2021variational} samples DAGs from autoregressive adjacency matrix distributions, while \cite{lorch2021dibs} utilizes Stein variational approach \cite{liu2016stein} for DAGs and causal model parameters. \cite{cundy2021bcd} proposed a variational inference framework on node orderings using the gumbel-sinkhorn gradient estimator \cite{mena2018learning}. \cite{deleu2022bayesian,nishikawa2022bayesian} employ the GFlowNet framework \cite{bengio2021gflownet} for inferring the DAG posterior. Most methods, except\cite{lorch2021dibs} are restricted to linear models, while \cite{lorch2021dibs} has high computational costs and lacks DAG generation guarantees compared to our method.
% at least quadratic scaling complexity, both with respect to the number of nodes (due to the DAG penalty) as well as number of posterior samples. Our proposed approach instead has linear complexity with respect to number of posterior samples and does not require any additional DAG penalty.     

In contrast, \emph{quasi-Bayesian} methods, such as DAG bootstrap \cite{friedman2013data}, demonstrate competitive performance. DAG bootstrap resamples data and estimates a single DAG using PC \cite{spirtes2000causation}, GES \cite{chickering2002optimal}, or similar algorithms, weighting the obtained DAGs by their unnormalized posterior probabilities. Recent neural network-based works employ variational inference to learn DAG distributions and point estimates for nonlinear model parameters \cite{charpentier2022differentiable,geffner2022deep}.
\section{Secure Design of \puma}\label{sec:design}
In this section, we first present an overview of \puma, and present the protocols for secure $\gelu$ , $\softmax$, embedding, and $\layernorm$ used by \puma. Note that the linear layers such as matrix multiplication are straightforward in replicated secret sharing, so we mainly describe our protocols for non-linear layers in this manuscript.

\subsection{Overview of \puma}\label{sec:overview}
To achieve secure inference of Transformer models, \puma\ defines three kinds of roles: one model owner, one client, and three computing parties. The model owner and the client  provide their models or inputs to the computing parties (i.e., $P_0$, $P_1$, and $P_2$) in a secret-shared form, then the computing parties execute the MPC protocols and send the results back to the client. Note that the model owner and client can also act as one of the computing party, we describe them separately for generality. \eg, when the model owner acts as $P_0$, the client acts as  $P_1$, a third-party dealer acts as $P_2$, the system model becomes the same with \mpcformer~\citep{li2023mpcformer}.

During the secure inference process, a key invariant is maintained: For any layer, the computing parties always start with 2-out-of-3 replicated secret shares of the previous layer's output and the model weights, and end with 2-out-of-3 replicated secret shares of this layer's output. As the shares do not leak any information to each party, this ensures that the layers can be sequentially combined for arbitrary depths to obtain a secure computation scheme for any Transformer-based model.
%The main focus of \puma\ is to reduce the computation and communication costs between the computing parties while maintaining the desired level of security. 



\iffalse
\textbf{Threat Model.}
Following previous works~\citep{aby3,li2023mpcformer},
\puma\ resists a semi-honest (a.k.a., honest-but-curious) adversary in honest-majority~\citep{lindell2009proof}, where the adversary passively corrupts no more than one computing party. Such an adversary follows the protocol specification exactly, but may try to learn more information than permitted. Please note that \puma\ cannot protect against the extraction of information from the inference results, and the examination of mitigating solutions (\eg, differential privacy~\citep{abadi2016deep}) falls outside the scope of this study.
\fi 

\subsection{Protocol for Secure GeLU}\label{sec:gelu}
Most of the current approaches view the $\gelu$ function as a composition of smaller functions and try to optimize each piece of them, making them to miss the
chance of optimizing the private $\gelu$ as a whole. Given the $\gelu$ function:
\begin{equation}\label{eq:gelu}
\begin{split}
    \gelu(x) &= \frac{x}{2} \cdot \left(1 + \tanh \left( \sqrt{\frac{2}{\pi}} \cdot \left(x + 0.044715 \cdot x^3 \right) \right) \right)\\
    &\approx x\cdot \mathsf{sigmoid}(0.071355\cdot x^3 + 1.595769\cdot x) 
\end{split},
\end{equation}
these approaches~\citep{hao2022iron,characmpctranformer} focus either on designing efficient protocols for function $\tanh$
or using the existing MPC protocols of exponentiation and reciprocal for $\mathsf{sigmoid}$. 

However, none of current approaches have utilized the fact that $\gelu$ function is almost linear on the two sides (\ie, $\gelu(x)\approx 0$ for $x<-4$ and $\gelu(x)\approx x$ for $x>3$). 
Within the short interval $[-4,3]$ of $\gelu$,
we suggest a piece-wise approximation of low-degree polynomials is a more efficient and easy-to-implement choice for its secure protocol. Concretely, our piece-wise low-degree polynomials are shown as equation~(\ref{eq:geluapprox}):
\begin{equation}\label{eq:geluapprox}
\gelu(x)=
\begin{cases}
0, & x<-4 \\
F_0(x), & -4 \le x < -1.95 \\
F_1(x), & -1.95 \le x \le 3 \\
x, & x >3
\end{cases},
\end{equation}
where polynomials $F_0()$ and $F_1()$ are computed by library $\mathsf{numpy.ployfit}$\footnote{\url{https://numpy.org/doc/stable/reference/generated/numpy.polyfit.html}} as equation~(\ref{eq:f0f1}). Surprsingly, the above simple poly fit works very well and our $\mathsf{max\ error}< 0.01403$, $\mathsf{median\ error}< 4.41e-05$, and $\mathsf{mean\ error}< 0.00168$.
\begin{equation}\label{eq:f0f1}
\begin{cases}
F_0(x) &= -0.011034134030615728 x^3 -0.11807612951181953 x^2 \\
&- 0.42226581151983866 x -0.5054031199708174\\
F_1(x) &= 0.0018067462606141187x^6 -0.037688200365904236 x^4 \\
&+ 0.3603292692789629x^2 + 0.5x + 0.008526321541038084
\end{cases}
\end{equation}

Formally, given secret input $\share{x}$, our secure $\gelu$ protocol $\Pi_{\gelu}$ is constructed as algorithm~\ref{protocol:gelu}. 
\iffalse
\begin{itemize}
    \item The parties jointly compute
$\share{b_0}^2 = \Pi_{\mathsf{LT}}(\share{x}, 4)$,
$\share{b_1}^2 = \Pi_{\mathsf{LT}}(\share{x}, -1.95)$, and
$\share{b_2}^2 = \Pi_{\mathsf{LT}}(3, \share{x})$.

\item  Then, each $P_i$ locally compute
$\share{b_3}^2 = \share{b_1}^2 \oplus \share{b_2}^ \oplus 1$ and
$\share{b_4}^2 = \share{b_0}^2 \oplus \share{b_1}^2$

\item Finally, the parties compute and return 
$\share{b_2}^2 \cdot \share{x} + \share{b_4}^2 \cdot F_0(\share{x}) + \share{b_3}^2 \cdot F_1(\share{x})$, where polynomials $(F_0, F_1)$ can be computed easily using secure addition and multiplication (and its variants, \eg, secure square)~\citep{spu}. 
\end{itemize}
\fi 

\begin{algorithm}[tp]
\caption{Secure $\gelu$ Protocol $\Pi_{\mathsf{GeLU}}$}\label{protocol:gelu}
\begin{algorithmic}[1]
\REQUIRE
$P_i$ holds the 2-out-of-3 replicate secret share $\share{x}_i$ for $i\in \{0,1,2\}$ 
\ENSURE
$P_i$ gets the 2-out-of-3 replicate secret share $\share{y}_i$ for $i\in \{0,1,2\}$, where $y=\gelu(x)$.

\STATE $P_0$, $P_1$, and $P_2$ jointly compute
\begin{equation*}
\begin{split}
&\shareb{b_0} = \Pi_{\mathsf{LT}}(\share{x}, -4),~~~\vartriangleright b_0 = 1\{x<-4\}\\
&\shareb{b_1} = \Pi_{\mathsf{LT}}(\share{x}, -1.95),~~~\vartriangleright b_1 = 1\{x<-1.95\} \\
&\shareb{b_2} = \Pi_{\mathsf{LT}}(3, \share{x}),~~~~~~\vartriangleright b_2 = 1\{3<x\}
\end{split}
\end{equation*}
and compute 
$\shareb{z_0} = \shareb{b_0} \oplus \shareb{b_1}$,
$\shareb{z_1} = \shareb{b_1} \oplus \shareb{b_2} \oplus 1$, and $\shareb{z_2}=\shareb{b_2}$. Note that $z_0 = 1\{-4\le x < -1.95\}$, $z_1 = 1\{-1.95\le x\le 3\}$, and $z_2 = 1\{x>3\}$.

\STATE Jointly compute $\share{x^2} = \Pi_{\mathsf{Square}}(\share{x})$, $\share{x^3} = \Pi_{\mathsf{Mul}}(\share{x}, \share{x^2})$, $\share{x^4} = \Pi_{\mathsf{Square}}(\share{x^2})$, and $\share{x^6} = \Pi_{\mathsf{Square}}(\share{x^3})$.

\STATE Computing polynomials $\share{F_0(x)}$ and $\share{F_1(x)}$ based on $\{\share{x}, \share{x^2}, \share{x^3}, \share{x^4}, \share{x^6}\}$ as equation~(\ref{eq:geluapprox}) securely.


\RETURN$\share{y} = \Pi_{\mathsf{Mul_{BA}}}(\shareb{z_0}, \share{F_0(x)}) + \Pi_{\mathsf{Mul_{BA}}}(\shareb{z_1}, \share{F_1(x)})+\Pi_{\mathsf{Mul_{BA}}}(\shareb{z_2}, \share{x})$.

\end{algorithmic}
\end{algorithm}



\subsection{Protocol for Secure Softmax}\label{sec:secureatten}

In the function $\attention(\Q,\K,\V)=
\softmax(\Q \cdot \K^\mathsf{T} + \M) \cdot \V$, where $\M$ can be viewed as a bias matrix, the key challenge is computing function $\softmax$. For the sake of numerical stability, the $\softmax$ function is computed as
\begin{equation}\label{eq:softmax}
    \softmax(\x)[i]=\frac{\exp(\x[i] - \bar{x} - \epsilon)}{\sum_i \exp(\x[i] - \bar{x} - \epsilon)},
\end{equation}
where $\bar{x}$ is the maximum element of the input vector $\x$. 
For the normal plaintext softmax, $\epsilon=0$. For a two-dimension matrix, we apply equation~(\ref{eq:softmax}) to each of its row vector.

Formally, our detailed secure protocol  $\Pi_{\softmax}$ is illustrated in algorithm~\ref{protocol:softmax}, where we propose two optimizations:
\begin{itemize}
\item 
For the first optimization, we set $\epsilon$ in equation~\ref{eq:softmax} to a tiny and positive
value, e.g., $\epsilon =
10^{-6}$, so that the inputs to exponentiation
in equation~\ref{eq:softmax} are all negative. We exploit the negative operands
for acceleration. Particularly, we compute the exponentiation using the Taylor series~\citep{tan2021cryptgpu} with a simple clipping
\begin{equation}\label{eq:negexp}
\mathsf{negExp}(x) = \begin{cases}
    0, &x < T_{\exp} \\
    (1+\frac{x}{2^t})^{2^t}, &x\in [T_{\exp},0].
\end{cases}
\end{equation}
Indeed, we apply the less-than for the branch $x < T_{\exp}$
The division by $2^t$ can be achieved using
$\Pi_{\mathsf{Trunc}}^t$ since the input is already negative. Also, we can
compute the power-of-$2^t$ using $t$-step sequences of square function $\Pi_{\mathsf{square}}$ and $\Pi_{\mathsf{Trunc}}^f$. Suppose our MPC program uses
$18$-bit fixed-point precision. Then we set $T_{\exp}=-14$ given $\exp(-14) < 2^{-18}$, and empirically set $t = 5$.


\item 
Our second optimization is to reduce the number of divisions, which ultimately saves computation and communication costs.
To achieve this, for a vector $\x$ of size $n$, we have replaced the operation $\mathsf{Div}(\x, \mathsf{Broadcast}(y))$ with $\x \cdot  \mathsf{Broadcast}(\frac{1}{y})$, where $y=\sum_{i=1}^n\x[i]$. By making this replacement, we effectively reduce $n$ divisions to just one reciprocal operation and $n$ multiplications.
This optimization is particularly beneficial in the case of the $\softmax$ operation. The $\frac{1}{y}$ in the $\softmax$ operation is still large enough to maintain sufficient accuracy under fixed-point values. As a result, this optimization can significantly reduce the computational and communication costs while still providing accurate results.
\end{itemize}

\begin{algorithm}[tp]
\caption{Secure $\softmax$ Protocol $\Pi_{\softmax}$}\label{protocol:softmax}
\begin{algorithmic}[1]
\REQUIRE
$P_i$ holds the 2-out-of-3 replicate secret share $\share{\x}_i$ for $i\in \{0,1,2\}$, and $\x$ is a vector of size $n$. 
\ENSURE
$P_i$ gets the 2-out-of-3 replicate secret share $\share{\y}_i$ for $i\in \{0,1,2\}$, where $\y=\softmax(\x)$.

\STATE $P_0$, $P_1$, and $P_2$ jointly compute
$\shareb{\mathbf{b}} = \Pi_{\mathsf{LT}}(T_{\exp}, \share{\x})$ and the maximum $\share{\bar{x}} = \Pi_{\mathsf{Max}}(\share{\x})$.

\STATE Parties locally computes $\share{\hat{\x}} = \share{\x} - \share{\bar{x}} - \epsilon$, and jointly compute $\share{\z_0} = 1+  \Pi_{\mathsf{Trunc}}^t(\share{\hat{\x}})$.

\FOR{$j=1,2,\dots, t$}
\STATE $\share{\z_j} = \Pi_{\mathsf{Square}}(\share{\z_{j-1}})$.
\ENDFOR

\STATE Parties locally compute $\share{z} = \sum_{i=1}^n \share{\z[i]}$ and jointly compute $\share{1/z} = \Pi_{\mathsf{Recip}}(\share{z})$.

\STATE Parties jointly compute $\share{\z / z} = \Pi_{\mathsf{Mul}}(\share{\z}, \share{1/z})$

\RETURN $\share{\y} = \Pi_{\mathsf{Mul}_{\mathsf{BA}}}( \shareb{\mathbf{b}}, \share{\z / z})$.

\end{algorithmic}
\end{algorithm}

\subsection{Protocol for Secure Embedding}\label{sec:embed}


The current secure embedding procedure described in~\citep{li2023mpcformer} necessitates the client to  generate a one-hot vector using the token $\tokenid$ locally. This deviates from a plaintext Transformer workflow where the one-hot vector is generated inside the model. As a result, they have to carefully strip off the one-hot step from the pre-trained models, and add the step to the client side, which could be an obstacle for deployment. 



To address this issue, we propose a secure embedding design as follows. Assuming that the token $\tokenid\in [n]$ and all embedding vectors are denoted by $\E= (\e_1^T, \e_2^T, \dots, \e_n^T)$, the embedding can be formulated as $\e_{\tokenid} = \mathbf{E}[\tokenid]$. Given $(\tokenid, \E)$ are in secret-shared fashion, our secure embedding protocol $\Pi_{\mathsf{Embed}}$ works as follows:
\begin{itemize}
    \item The computing parties securely compute the one-hot vector $\shareb{\mathbf{o}}$ after receiving $\share{\tokenid}$ from the client. Specifically, $\shareb{\mathbf{o}[i]}=\Pi_{\mathsf{Eq}}(i,\share{\tokenid})$ for $i\in [n]$.
    \item The parties can compute the embedded vector via $\share{\e_{\tokenid}} = \Pi_{\mathsf{Mul_{BA}}}(\share{\E}, \shareb{\mathbf{o}})$, where  does not require secure truncation.
\end{itemize}
In this way, our $\Pi_{\mathsf{Embed}}$ does not require explicit modification of the workflow of plaintext Transformer models, at the cost of more $\Pi_{\mathsf{Eq}}$ and $\Pi_{\mathsf{Mul_{BA}}}$ operations. 



\subsection{Protocol for Secure LayerNorm}\label{sec:seclayernorm}
Recall that given a vector $\x$ of size $n$, $\layernorm(\x)[i] =  \gamma \cdot \frac{\x[i]-\mu}{\sqrt{\sigma}} + \beta$, where $(\gamma, \beta)$ are trained parameters, $\mu = \frac{\sum_{i=1}^n \x[i]}{n}$, and $\sigma = \sum_{i=1}^n (\x[i] - \mu)^2$. In MPC, the key challenge is the evaluation of the divide-square-root $\frac{\x[i]-\mu}{\sqrt{\sigma}}$ formula. To securely evaluate this formula, CrypTen sequentially executes the MPC protocols of square-root, reciprocal, and multiplication. However, we observe that $\frac{\x[i]-\mu}{\sqrt{\sigma}}$ is equal to $(\x[i]-\mu)\cdot \sigma^{-1/2}$. And in the MPC side, the costs of computing the inverse-square-root $\sigma^{-1/2}$ is similar to that of the square-root operation~\citep{rSqrt}. Besides, inspired by the second optimization of \S~\ref{sec:secureatten}, we can first compute $\sigma^{-1/2}$ and then $\mathsf{Broadcast}(\sigma^{-1/2})$ to support fast and secure $\layernorm(\x)$. And our formal protocol $\Pi_{\layernorm}$ is shown in algorithm~\ref{protocol:layernorm}.

\begin{algorithm}[tp]
\caption{Secure $\mathsf{LayerNorm}$ Protocol $\Pi_{\mathsf{LayerNorm}}$}\label{protocol:layernorm}
\begin{algorithmic}[1]
\REQUIRE
$P_i$ holds the 2-out-of-3 replicate secret share $\share{\x}_i$ for $i\in \{0,1,2\}$, and $\x$ is a vector of size $n$. 
\ENSURE
$P_i$ gets the 2-out-of-3 replicate secret share $\share{\y}_i$ for $i\in \{0,1,2\}$, where $\y=\mathsf{LayerNorm}(\x)$.

\STATE $P_0$, $P_1$, and $P_2$ compute $\share{\mu} = \frac{1}{n}\cdot \sum_{i=1}^n\share{\x[i]}$ and $\share{\sigma} = \sum_{i=1}^n \Pi_{\mathsf{Square}}(\share{\x} - \share{\mu})[i]$.

\STATE Parties jointly compute $\share{\sigma^{-1/2}} = \Pi_{\mathsf{rSqrt}}(\share{\sigma})$.

\STATE Parties jointly compute $\share{\mathbf{c}} = \Pi_{\mathsf{Mul}}((\share{\x} - \share{\mu}), \share{\sigma^{-1/2}})$

\RETURN $\share{\y} = \Pi_{\mathsf{Mul}}(\share{\gamma}, \share{\mathbf{c}}) + \share{\beta}$.

\end{algorithmic}
\end{algorithm}
\input{figures/planerectr_network}

% \section{PlaneRecTR(Single View Plane Recovery)}
\section{Intra-Frame Plane Query Learning}
\label{sec:intra_frame_method}

In this section, we present the details of intra-frame plane query learning, \ie, PlaneRecTR. We start by first introducing its architecture in Section~\ref{sec:model}, and then discuss the training process and loss functions in Section \ref{sec:training}. Finally, we describe the inference process of recovering 3D planes from a single view in Section \ref{sec:planeinference}.


%-------------------------------------------------------------------------
% \subsection{Unified Query Learning for Plane Recovery}
% \subsection{PlaneRecTR Module}
\subsection{Transformer-based Unified Query Learning for Single-view Plane Recovery}
\label{sec:model}


Inspired by the recent successes of DETR \cite{carion2020end} and Mask2Former \cite{cheng2022masked} in object detection and instance segmentation, we find that, it is not impossible to tackle the challenging monocular planar reconstruction task using a \textit{single}, \textit{compact} and \textit{unified} framework, thanks to the merits of query-based reasoning to enable joint modeling of multiple tasks.

As shown in Figure \ref{fig:planerectr_network}, intra-frame plane query learning component consists of three main modules: (1) A pixel-level module to learn dense pixel-wise deep embedding of the input colour image. (2) A Transformer-based unified query learning module to jointly predict, for each of $N$ learnable plane queries, its corresponding plane embeddings $\mathcal{E}_\text{plane}$ as well as four target properties, including plane classification probability $p_i$, plane parameter $n_i$, mask embedding, and depth embedding ($i\in[1,2,..., N]$). Specifically, $p_i$ is the probability to judge whether the $i^\text{th}$ query corresponds to a plane or not; $n_i\doteq \tilde{n_i}/d_i \in \mathbb{R}^3$,  where $\tilde{n_i} \in \mathbb{R}^3$ is its plane normal and $d_i$ is the distance from the $i^\text{th}$ plane to camera center, \ie, offset. (3) A plane-level module to generate plane-level mask $m_i$ and plane-level depth $d_i$ through mask and depth embedding ($i\in[1,2,..., N]$). We then remove non-plane query hypothesis while combining the remaining ones for the final image-wise plane recovery. These three modules will be described in details bellow.



\paraspace
\ptitle{Pixel-Level Module.}
Given an image of size $H \times W$ as input, we use the pre-trained ResNet-50 \cite{He:etal:CVPR2016} as the backbone model to extract dense image feature maps, unless otherwise mentioned. Similar to Mask2Former, a multi-scale convolutional pixel decoder is used to produce a set of dense feature maps with four scales, denoted as follows:
\begin{equation}
\begin{aligned}
\mathbb{F}&=\{F_{1}\in \mathbb{R}^{C_{1}\times H/32\times W/32}, F_{2}\in \mathbb{R}^{C_{2}\times H/16\times W/16}, \\
& F_{3}\in \mathbb{R}^{ C_{3}\times H/8\times W/8}, \mathcal{E}_\text{pixel}\in \mathbb{R}^{ C_{\mathcal E}\times H_{\mathcal E}\times W_{\mathcal E}}\}.
\end{aligned}
\end{equation}

Among these, the first three feature maps $\{F_{1}, F_{2}, F_{3}\}$ are fed to the Transformer module, while the last one $\mathcal{E}_\text{pixel}$, a dense per-pixel embedding of resolution $H_{\mathcal E}=H/4$ and $W_{\mathcal E}=W/4$, is exclusively used for computing plane-level binary masks and plane-level depths.

\paraspace
\ptitle{Transformer Module.}
We use the Transformer decoder with masked attention proposed in \cite{cheng2022masked}, which computes \textbf{unified plane embeddings} $\mathcal{E}_\text{plane} \in \mathbb{R}^{N \times C_{\mathcal E}}$ from above mentioned multi-scale feature maps $\{F_{1}, F_{2}, F_{3}\}$ and $N$ learnable plane queries. The predicted $\mathcal{E}_\text{plane}$ are then independently projected to four target properties by four different multi-layer perceptrons (MLPs). Overall, the Transformer module predicts required planar attributes through $N$ plane queries (upper left part in Figure \ref{fig:planerectr_network}).

\paraspace
\ptitle{Plane-Level Module.}
 As shown in the upper right part of Figure \ref{fig:planerectr_network}, we obtain a dense plane-level binary mask $m_i \in [0, 1]^{H_{\mathcal E} \times W_{\mathcal E}}$/depth prediction $d_i \in \mathbb{R}^{H_{\mathcal E} \times W_{\mathcal E}}$ by a dot product between the $i^\text{th}$ mask/depth embedding and the dense per-pixel embedding $\mathcal{E}_\text{pixel}$ from previous two modules, respectively.
 
Please note that we have also investigated the idea of learning two individual pixel decoders respectively for semantic and depth dense embeddings, in order to separately predict planar masks and depths. However, we did not observe a clear performance improvement, and therefore stick to current set-up with a shared $\mathcal{E}_\text{pixel}$ for the sake of efficiency and simplicity.

We finally obtain $N$ plane-level predictions $\{y_i=(p_i,n_i, m_i, d_i)\}_{i=1}^N$ by the plane-level module, and each plane-level prediction contains all the necessary information to recover a possible 3D plane.


%-------------------------------------------------------------------------
\subsection{Training Objective and Configuration}
\label{sec:training}
% \paraspace
\ptitle{Plane-level Depth Training.}
Previous methods tend to predict a global image-wise depth by separate network branches to calculate plane offset \cite{Liu:etal:CVPR2019:Planercnn} or use depth as additional cues to formulate segmentation masks \cite{Yu:etal:CVPR2019:PlaneAE,Tan:etal:ICCV2021:Planetr}.
In contrast, our method tries to achieve mutual benefits between planar semantic and geometric reasoning, 
we leverage joint query learning to unify all components of plane recovery in a concise multi-task manner. As a result, we explicitly predict dense plane-level depths, binary-masks, plane probabilities and parameters from a shared feature space, which is produced and refined via attention mechanism of the Transformer.

% Therefore we additionally add plane-level Depth prediction Task in the training phase, and just add an MLP and dot product operation in the implementation.
\paraspace
\ptitle{Bipartite Matching.}
During training, one important step is to build optimal correspondences between $N$ predicted planes and $M$ ground truth planes ($N \geq M$).
% We achieve this following \cite{cheng2022masked} using bipartite matching by searching for a permutation $\hat{\sigma}$ (where $\sigma(i)$ indicates the matched index of the predict planes to the ground truth $\hat{s}_{p}^{i}$) by minimizing a matching defined cost function $D$:
Following bipartite matching of \cite{cheng2022masked, Tan:etal:ICCV2021:Planetr}, we search for a permutation $\hat{\sigma}$ 
% (where $\sigma(i)$ indicates the matched index of the predict planes to the ground truth $\hat{s}_{p}^{i}$) 
by minimizing a matching defined cost function $D$:
\begin{equation}
\label{eq: bi-matching}
\hat{\sigma}=\underset{\sigma}{\arg \min } \sum_{i=1}^{N} D\left(\hat{y}_{i}, y_{\sigma(i)}\right), 
\end{equation}
\vspace{-0.5cm}
\begin{align}
\label{eq: bi_matching2}
% D=-\omega_{1} \, p_{\sigma(i)}\left( \hat{p}_{i}\right)
% &+ \mathbbm{1}_{\{\hat{p}_{i}=1\}}  \, \omega_{2} \, L_{1}\left(\hat{{n}}_{i}, {n}_{\sigma(i)}\right) \nonumber\\
% &+ \mathbbm{1}_{\{\hat{p}_{i}=1\}}  \, \omega_{3} \, L_{1}\left(\hat{{d}}_{i}, {d}_{\sigma(i)}\hat{m}_{i}\right)  \nonumber\\
% &+ \mathbbm{1}_{\{\hat{p}_{i}=1\}}  \, \omega_{4} \, L_{ce}
% \nonumber\\
% &+ \mathbbm{1}_{\{\hat{p}_{i}=1\}}  \, \omega_{5} \, L_{dice},
% D=\mathbbm{1}_{\{\hat{p}_{i}=1\}} \left[& -\omega_{1} \, p_{\sigma(i)}
% + \omega_{2} L_{1}\left(\hat{{n}}_{i}, {n}_{\sigma(i)}\right) \nonumber \right. \\
% &+ \omega_{3} L_{1}\left(\hat{{d}}_{i}, {d}_{\sigma(i)}\hat{m}_{i}\right)  \nonumber\\
% &+ \omega_{4} L_{ce} \nonumber\\
% & \left. +  \omega_{5} L_{dice} \right],
% D=\mathbbm{1}_{\{\hat{p}_{i}=1\}} \Big[
% & -\omega_{1} \, p_{\sigma(i)}
% + \omega_{2} L_{1}\left(\hat{{n}}_{i}, {n}_{\sigma(i)}\right) \nonumber\\
% & + \omega_{3} L_{1}\left(\hat{{d}}_{i}, {d}_{\sigma(i)}\hat{m}_{i}\right) + \omega_{4} L_{ce}  +  \omega_{5} L_{dice} \Big],
D= & \mathbbm{1}_{\{\hat{p}_{i}=1\}} \Big[
 -\omega_{1} \, p_{\sigma(i)}
+ \omega_{2} L_{1}\left(\hat{{n}}_{i}, {n}_{\sigma(i)}\right) \nonumber\\
& + \omega_{3} L_{1}\left(\hat{{d}}_{i}, {d}_{\sigma(i)}\hat{m}_{i}\right) + \omega_{4} L_{ce}  +  \omega_{5} L_{dice} \Big],
\end{align}
where $\hat{y}_{i} = (\hat{p}_i,\hat{n}_i, \hat{m}_i, \hat{d}_i)$ are the $i^\text{th}$ ground-truth plane attributes, we augment the ground truth instances with non-planes where $\hat{p}_{i}=0$ if $i\textgreater M$; $\sigma(i)$ indicates the matched index of the predicted planes to the ground truth $\hat{y}_{i}$;$\mathbbm{1}$ is an indicator function taking 1 if $\hat{p}_{i}=1$ is true and 0 otherwise; $\omega_{1}, \omega_{2}, \omega_{3}, \omega_{4}$ and $\omega_{5}$ are weighting terms and set to 2, 1, 2, 5, 5, respectively. 
% Here we additional consider of influence of mask and depth quality using an $L_1$ depth loss and segmentation dice loss $L_{dice}$, respectively.
Here we additionally consider the influence of mask and depth quality using a mask binary cross-entropy loss $L_{ce}$ \cite{cheng2022masked}, a mask dice loss $L_{dice}$ \cite{milletari2016v} and an $L_1$ depth loss, respectively.

\paraspace
\ptitle{Loss Functions.}
After bipartite matching, the final training objectives $L$ is composed of following four parts:
\vspace{-0.3cm}
\begin{equation}
% \small 
\mathcal{L} = 
\sum\limits_{i=1}^{M}\left( \lambda\mathcal{L}_{\text {cls }}^{(i)} + \mathcal{L}_{\text {param }}^{(i)} + \mathcal{L}_{\text{mask}}^{(i)} + \lambda\mathcal{L}_{\text {depth}}^{(i)} \right),
\end{equation}
where $\lambda$ is a weighting factor and is set to 2 in this paper. $\mathcal{L}_{\text {cls}}$ and $\mathcal{L}_{\text {param}}$ are a plane classification loss and a plane parameter loss, in a similar form to previous work \cite{Tan:etal:ICCV2021:Planetr}.
% \begin{equation}
% \mathcal{L}_{\text {cls }}^{(i)} = -\log p_{{\hat{\sigma}}(i)}\left( \hat{p}_{i}\right).
% \end{equation}
% \begin{align}
% \label{eq:param}
% \mathcal{L}_{\text {parm }}^{(i)} =
% & \mathbbm{1}_{\{\hat{p}_{i}=1\}}   \,          L_{1}\left(\hat{{n}}_{i}, {n}_{{\hat{\sigma}}(i)}\right) + \nonumber\\
% & \mathbbm{1}_{\{\hat{p}_{i}=1\}}   \,  \beta_{cos} \left(1- \cos\left(\hat{{n}}_{i}, {n}_{{\hat{\sigma}}(i)}\right)\right) + \nonumber\\
% & \mathbbm{1}_{\{\hat{p}_{i}=1\}}   \,  \beta_{q} \sum\limits_{q\in Q_{i}} \| {n}_{{\hat{\sigma}}(i)}^{T} q - 1 \|,
% \end{align}
% where $Q_{i}$ is the set of 3D points calculated from pixels belonging to ground truth plane instance via the ground truth depth map. $\beta_{cos} = 5$ and $\beta_q = 2$ in this paper.

However, different from PlaneTR \cite{Tan:etal:ICCV2021:Planetr}, the left two loss terms in our paper $\mathcal{L}_{\text{mask}}$ and $\mathcal{L}_{\text {depth}}$ are designed to explicitly learn dense planar masks and depths. Specifically,
we introduce the plane segmentation mask prediction loss, as a combination of a cross-entropy loss and a dice loss:
% \begin{align}
% % \label{eq:mask}
% \mathcal{L}_{\text {mask }}^{(i)} =
% & \mathbbm{1}_{\{\hat{p}_{i}=1\}}   \,  \beta_{1} L_{ce} + \nonumber\\
% & \mathbbm{1}_{\{\hat{p}_{i}=1\}}   \,  \beta_{2} L_{dice},
% \end{align}
\begin{equation}
% \label{eq:mask}
\mathcal{L}_{\text {mask }}^{(i)} =
 \mathbbm{1}_{\{\hat{p}_{i}=1\}}   \,  \beta_{1} L_{ce} + \nonumber
 \mathbbm{1}_{\{\hat{p}_{i}=1\}}   \,  \beta_{2} L_{dice},
\end{equation}
where $\beta_{1} = 5$ and $\beta_{2} = 5$.
The depth prediction loss is in a typical $L_1$ form, penalizing the discrepancy of depth value within planar regions:
\begin{equation}
% \label{eq:mask}
\mathcal{L}_{\text {depth }}^{(i)} =
 \mathbbm{1}_{\{\hat{p}_{i}=1\}}  L_{1}\left(\hat{{d}}_{i}, {d}_{\sigma(i)}\hat{m}_{i}\right).
\end{equation}

%-------------------------------------------------------------------------
\subsection{Inference Process of Monocular 3D Plane Recovery}
\label{sec:planeinference}

% sjj: argmin->argmax
In this section, we briefly discuss how to recover 3D planes from our plane-level predictions. For the $N$ plane-level predictions $\{y_i\}_{i=1}^N$ predicted by the network, we first drop non-plane candidates according to the plane classification probability $p_i$, leading to a valid subset of $K$ planes (\ie, $K\leq N$). 
% The final planar segmentation mask of the input image is obtained by, for each pixel position of planar region, the most probable plane index based on segmentation mask prediction, \ie, $\mathop{\arg\max}\limits_{i}\{m_i\}_{i=1}^{K}$.
For each pixel within the planar regions, we calculate 
the most likely plane index
$\mathop{\arg\max}\limits_{i}\{m_i\}_{i=1}^{K}$ to obtain its mask id and thus obtain the final image-wise segmentation mask.



In this simple and efficient manner, we can use the network predictions to create 3D plane reconstruction of the input frame. Note that plane-level depths are not involved during inference and we use plane parameters and segmentation to infer planar depths. 
We experimentally found that this design also leads to more structural and smooth geometric predictions than that relying on direct depth predictions.
\input{figures/crossplaneattn_network}
\section{Inter-Frame Plane Query Learning}
\label{sec:inter_frame_method}
%-------------------------------------------------------------------------
In this section, we \jiajia{present} how we extend our \jiajia{monocular} framework PlaneRecTR to a novel multi-view setup, while still retaining the virtue of query-based learning. 
We introduce an inter-frame query learning component on top of unified plane embeddings of per-frame. A plane aware cross attention layer is proposed to achieve inter-frame plane interactions, within which dual softmax \cite{sun2021loftr, rockwell20228posevit} and bilinear attention \cite{kim2018bilinear, rockwell20228posevit} mechanisms are used to align the intermediate attention structure with a plane correspondence matrix and enable plane-level feature fusion between views.  
Most importantly, we further modify the key, query, and value forms of standard multi-head attention \cite{Vaswani:etal:NIPS2017} to effectively utilize the complete representation of unified plane embeddings, which better accommodates multi-view plane properties without requiring any additional inputs such as position encoding \cite{sun2021loftr, rockwell20228posevit}. This simple adjustment guarantees that our attention structure truly accomplishes plane matching, allowing the network to spontaneously focus on genuine paired planes from two views.

As illustrated in Figure \ref{fig:crossplaneattn_network}, we employ two plane-aware cross attention layers and an MLP head to construct a simple and lightweight pose regression module. A plane-aware attention layer first takes two-view plane embeddings $\mathcal{E}_\text{plane}^1$ and $\mathcal{E}_\text{plane}^2$ as input, and actively learns their correspondences within the network. Subsequently, our model directly regresses a relative camera pose from probabilistic paired plane embeddings. Conceptually, this entire process aligns with the logical framework of solutions based on classical two-view geometry \cite{hartley2003multiple}, 
thereby offering enhanced interpretability. 


\subsection{Cross Attention for Unified Plane Embeddings}
\label{sec:plane_cross_attention}

\ptitle{Preliminaries: Standard Cross Attention.} 
The Transformer cross attention layer \cite{Vaswani:etal:NIPS2017} updates the input value term by mapping query from the $i$-th input and key-value pair from another $j$-th input through a weighted summation, typically using a scaled dot-product similarity function $\operatorname{S}(\cdot,\cdot)$. In addition, attention often employs a multi-head strategy \cite{Vaswani:etal:NIPS2017} where query, key, and value (denoted $Q_i$, $K_j$, $V_j$ 
$\in \mathbb{R}^{N \times C_{\mathcal E}}$, respectively)
are divided, along channel dimensions, into $N_h$ segments $\{q_i^h\}_{h=1}^{N_h}$, $\{k_j^h\}_{h=1}^{N_h}$, $\{v_j^h\}_{h=1}^{N_h} \in \mathbb{R}^{N \times \frac{C_{\mathcal E}}{N_h}}$, in order to enhance the expressiveness and diversity of results without incurring extra computational costs. 

The multi-head similarity function ($\operatorname{S}$) and  cross attention layer ($\operatorname{MCA}$) are defined in Equations \ref{eq:standard_crossattn1} and \ref{eq:standard_crossattn2}:
\begin{small}
\begin{equation}
% \setlength\abovedisplayskip{2pt}%shrink space
% \setlength\belowdisplayskip{3pt}
\label{eq:standard_crossattn1}
% \rm{Attention}(Q_i, K_j, V_j) = \rm{softmax}( \frac{Q_iK_j^T}{\sqrt{d_k}} )V_j
\operatorname{S}(q_i^h, k_j^h) =  \operatorname{softmax}\left(\frac{q_i^h{k_j^h}^T}{\sqrt{C_{\mathcal E}/N_h}}, 1\right),
\end{equation}
\end{small}
\begin{small}
\begin{align}
\label{eq:standard_crossattn2}
\operatorname{MCA}(Q_i, K_j, V_j) = \operatorname{Linear}(\operatorname{Concat}(\{\operatorname{S}(q_i^h, k_j^h)v_j^h\}_{h=1}^{N_h})),
\end{align}
\end{small}where $\operatorname{softmax}(\cdot, k)$ applies softmax operation across the $k$-th axis; $\operatorname{Linear}$ and $\operatorname{Concat}$ mean linear projection and channel-wise concatenation, respectively.
Our \jiajia{method} targets equations \ref{eq:standard_crossattn1} and \ref{eq:standard_crossattn2} for intuitive and efficient plane-specific modifications, achieving implicit plane matching and direct pose regression.

\paraspace
% \ptitle{Dual Softmax}
\ptitle{Plane Correspondence Probability Function.}
To overcome the limitations of the multi-stage two-view plane reconstruction paradigm \cite{jin2021sparseplanes, agarwala2022planeformers, tan2023nopesac}, here we specifically devise an inter-frame correspondence attention structure to learn reliable plane embeddings, which \jiajia{enables} the network to autonomously acquire probabilities of plane correspondence and conduct pose inference in a single forward pass. This design also eliminates the dependency on either ground truth correspondence supervision or initial pose.




Specifically, in contrast to the similarity function in Equation \ref{eq:standard_crossattn1},
%employed for calculating the similarity attention matrix in the standard cross-attention layer.   
we utilize a dual-softmax operation instead of a single softmax on the \emph{unsplit} query and key embeddings $Q_i, K_j$, aiming to keep integral embedding information when constructing plane-wise correspondence probability, as shown in Equation \ref{eq:dual_softmax} below.
\begin{equation}
\label{eq:dual_softmax}
% \setlength\abovedisplayskip{0.5pt}%shrink space
% \setlength\belowdisplayskip{0.5pt}
\small
\operatorname{C}(Q_i, K_j) =\operatorname{softmax}(\frac{Q_iK_j^T}{\sqrt{C_{\mathcal E}}} , 1) \odot \operatorname{softmax}(\frac{Q_iK_j^T}{\sqrt{C_{\mathcal E}}} , 2)\jiajia{,}
\end{equation}
We compute a 2D correspondence matrix $\operatorname{C}(Q_i, K_j)$,  where the element at the $m^{th}$ row and $n^{th}$ column $\operatorname{C}_{mn}(Q_i, K_j)$ denotes the probability that the $m^{th}$ plane embedding from the $i^{th}$ image $I_i$ corresponds to the $n^{th}$ plane embedding from  $j^{th}$ image $I_j$, indicating their likelihood of representing the same plane instance. For the task of two-view plane reconstruction, we stick to configurations of $\{i=1,j=2\}$ and $\{i=2,j=1\}$.


The noteworthy aspect lies in our simple modification to the input query and key's formats, which yield benefits that better align with the characteristics of plane instance. The key $K_j$ and query $Q_i$ in our model are derived from a linear mapping of the unified plane embeddings $\mathcal{E}_\text{plane}$ obtained by intra-frame query learning. We want to highlight the practical significance of an intact $\mathcal{E}_\text{plane}$ representing plane entities. Specifically, we preserve the integrity of $Q_i$ and $K_j$ rather than dividing them into multiple heads to fully leverage the representation power of unified plane embeddings encoding comprehensive information (geometry, appearance, location, context, etc.). This facilitates learning genuine correspondence distribution between planes instead of only abstractly capturing similarities among different subspace representations at various positions.

This simple design has been experimentally validated (Section \ref{sec:sparseview_ablation} and \ref{sec:unified_plane_emb}) to significantly increase the discriminative multi-view consistency of the unified plane embeddings $\mathcal{E}_\text{plane}$, which is of much higher quality than previous methods. Consequently, it facilitates the integration of real plane pairs' information without initial pose for direct precise pose estimation and enables explicit utilization in achieving accurate fusion of plane meshes from two views, thereby completing the overall reconstruction.

\paraspace
% \ptitle{Bilinear Cross Attention}
\ptitle{Inter-frame Plane Aware Cross Attention.}
\label{sec:bilinear_attntion}
The standard cross attention offers an efficient approach to selectively utilize one of the input plane sequences, but overlooks the interaction between two inputs. Therefore, we adopt bilinear attention \cite{kim2018bilinear, rockwell20228posevit} to incorporate planar information from both views through the plane correspondence probability distribution $\operatorname{C}(Q_i, K_j)$:
\begin{align}
\label{eq:bilinear_attentin}
% \setlength\abovedisplayskip{2pt}%shrink space
% \setlength\belowdisplayskip{2pt}
% \rm{BiAttention}(Q_i, K_j, V_i, V_j) =V_i^T \rm{dual\_softmax}( Q_i, K_j)V_j
\operatorname{PCA}(Q_i, K_j, V_i, V_j) = &\operatorname{Linear}(\operatorname{Concat}( \nonumber\\
& \{(v_i^h)^T \operatorname{C}(Q_i, K_j)v_j^h\}_{h=1}^{N_h} ))\jiajia{,}
\end{align}

Unlike the query and key terms, the value is still  divided into $N_h$ segments along the feature dimension, maintaining the advantages of multi-head attention. The value segments share the correspondence attention (Equation \ref{eq:dual_softmax}) from \emph{unsplit} query and key, allowing our model to selectively attend to actual corresponding plane embedding pairs across distinct sub-spaces.


As shown in Figure \ref{fig:crossplaneattn_network}, two parallel plane aware cross attention layers (Equation \ref{eq:bilinear_attentin}) are used to capture integrated features of the corresponding planes from $I_1$ to $I_2$ as well as from $I_2$ to $I_1$. It is worth noting that pose ViT \cite{rockwell20228posevit} employs the identical  features on both sides of bilinear attention matrix. However, we intuitively choose to cross-place embedding sequences of distinct images to model correspondences, thereby enhancing learning efficiency and yielding improved results. We will validate various design disparities through subsequent ablation studies in Section \ref{sec:sparseview_ablation}. 



\subsection{Pose Regression}
\label{sec:pose_loss}
Each plane aware cross attention layer ultimately outputs a feature map of size $N_h\times \frac{C_{\mathcal E}}{N_h} \times \frac{C_{\mathcal E}}{N_h}$. The corresponding plane feature maps from two parallel cross attention layers are concatenated and then mapped to a relative camera pose $T=(t, q) \in SE(3)$ using a simple MLP with two hidden layers. Here, $t \in \mathbb{R}^3$ represents translation in real units, and $q \in \mathbb{R}^4$ denotes a unit quaternion representing rotation transformation satisfying $\Vert q \Vert = 1 $.

We use lietorch \cite{teed2021lietorch} to calculate the geodesic distance $\mathcal{G} \in \mathbb{R}^6$ between the predicted pose $T$ and the ground truth pose $T^{\ast}$ as the loss for backpropagation:
\begin{equation}
% \setlength\abovedisplayskip{2pt}%shrink space
% \setlength\belowdisplayskip{2pt}
\mathcal{G} (T, T^{\ast}) = 
\operatorname{Log} ( T.\operatorname{inv()} \cdot  T^{\ast}),
\end{equation}
\begin{equation}
% \setlength\abovedisplayskip{2pt}%shrink space
% \setlength\belowdisplayskip{2pt}
\mathcal{L}_{pose} = \lambda_t \Vert \mathcal{G}_{1:3} \Vert + \lambda_q \Vert \mathcal{G}_{4:6} \Vert\jiajia{.}
\end{equation}
where $\lambda_t = 5$, $\lambda_q = 15$.
It should be noted that $\mathcal{L}_{pose}$ serves as the sole objective for our inter-frame component. Without requiring correspondence supervision, the proposed framework allows for the discovery of plane correspondence, and transforms abstract similarity attention distribution within network into a concrete probabilistic distribution of plane correspondences. Furthermore, the bilinear structure effectively utilizes integrated features of inter-frame planes, leading to accurate pose recovery without relying on external initial poses.


\subsection{Inferring Sparse Views Planar Reconstruction}
% Monocular planes recovery

% Plane matching

% Planar fusion
After intra-frame plane query learning, we can independently recover the 3D plane sets of two images in their respective camera coordinate systems using the corresponding unified plane embeddings as described in Section \ref{sec:model} and \ref{sec:planeinference}.

% Following inter-frame plane query learning, our unified plane embeddings capture multi-view consistency and the probability distribution computation for plane correspondence enables efficient plane matching within the attention layer.
During inference, we extract the learned correspondence matrix $\operatorname{C}(Q_i, K_j)$ from the network and filter out low-probability correspondences using a threshold $\theta$. We then employ the mutual nearest neighbor (MNN) criterion \cite{sun2021loftr} to obtain a hard assignment between the two plane sets. Like previous methods for sparse view planar reconstruction \cite{jin2021sparseplanes,agarwala2022planeformers,tan2023nopesac}, based on above estimated camera pose and plane matching results, we transform the plane attributes into canonical viewpoint for final reconstruction and evaluation.  Specifically, we merge normals, offsets, and textures of paired monocular 3D planes \cite{jin2021sparseplanes} to achieve a geometrically precise and smooth reconstruction using sparse views. Those paired planes whose deviations in merged normals or offsets exceed predefined thresholds are removed during inference. 

\section{Experiments}
% \haizhou{Follow the same way of introduction as we did in Section2.}
% \noindent In this section, we will introduce datasets and experimental setups that we used. Then we evaluate our method, other self-supervised methods, and supervised methods under different distribution shifts (\ie, concept shifts and covariate shifts) under common settings (\ie, transductive, inductive settings). It has to note that we focus on node-level tasks (\eg, node classification) in this work. As for graph-level tasks, we leave it as our future work and some simple experiments can be found in Appendix~\ref{app:graph_classification}. 
In this section, we first introduce the experimental setup including datasets, training, and evaluation protocol in Section~\ref{sec:dataset}~and~\ref{sec:unsupervised}. 
% Next, we present our experimental setup and conduct extensive experiments to evaluate our method in Section~\ref{sec:unsupervised}. 
We then perform an ablation study to demonstrate the effectiveness of each proposed component in Section~\ref{sec:ablation}. 
Additionally, we analyze the impact of important hyper-parameters in Section~\ref{sec:sensitivity}. 
Subsequently, we integrate our method with various encoding models, showcasing the model-agnostic nature of our recipe in Section~\ref{sec:other_models}. 
Finally, we provide some qualitative results such as feature visualization in Section~\ref{sec:vis}.
It is important to note that we focus on node-level tasks (\eg, node classification) in this work. As for graph-level tasks, we leave it as our future work, while some simple experiments are also provided in Appendix~\ref{app:graph_classification}.

\subsection{Datasets}\label{sec:dataset}
There exist some benchmarks for evaluating graph out-of-distribution generalization~\cite{good,ji2022drugood,gds}. 
Among them, GOOD~\cite{good} is the most representative and comprehensive benchmark that curates more diverse graph datasets with diverse tasks, including single/multi-task graph classification, graph regression, and node classification involving more distribution shifts (\ie, concept shifts and covariate shifts). Hence in this work, we follow the evaluation protocol proposed in \cite{good}. Furthermore, we validate the effectiveness of our method in the datasets (\ie, Amazon-Photo, Elliptic) that are used in EERM~\cite{eerm}. The statistics and detailed introduction to these datasets can be found in Table~\ref{tab:dataset} and Appendix~\ref{app:datasets}.

\begin{table*}[htp]
\caption{The descriptions of datasets. ``Domain-Level'' means splitting by graphs, ``Time-Aware'' denotes splitting according to chronological order.``Word'' and ``Degree'' represent splitting according to word diversity and node degree respectively. ``Language'' means splitting by user language, suggesting the prediction should not be impacted by the language the user use. ``University'' denotes splitting according to the domain university, implying that the prediction of webpages should be based on word contents and link connections rather than university features. ``Color'' means that nodes are split according to node differences in covariate shift and color-label correlations in concept shift.}
\label{tab:dataset}
\centering
\begin{tabular}{cccccccc}
\toprule
Datasets     & Network Type        & \#Nodes & \#Edges & \#Attributes &\#Classes& Train/Val/Test Split     & Metric   \\
% Cora         & Artificial Transformation & 2,703   &         &              &         &                      & Accuracy \\
Amazon-Photo\footnotemark
             & Co-purchasing network      & 7,650   & 119,081   & 755          & 10      & Domain-Level         & Accuracy \\
Elliptic\footnotemark  
             & Bitcoin transactions       & 203,769 & 234,355   & 165          & 2       & Time-Aware           & F1-Score \\
GOOD-Cora    & Scientific publications    & 19,793  & 126,842   & 8,710         & 70      & Word/Degree          & Accuracy \\
% GOOD-Arxiv   & arXiv papers               & 169,343 & 2,315,598 & 128          & 40      & Time/Degree          & Accuracy \\
GOOD-Twitch  & Gamer network              & 34,120  & 892,346   & 128          & 2       & Language             & ROC-AUC  \\
GOOD-CBAS    & A BA-house graph           & 700     & 3,962     & 4             & 4       & Color                & Accuracy \\
GOOD-WebKB   & Webpage network            & 617     & 1,138     & 1,703         & 5       & University           & Accuracy \\
\bottomrule
\end{tabular}
\end{table*}
\footnotetext[5]{This dataset is adopted from~\cite{yang2016revisiting}. \cite{eerm} constructs ten graphs with different environment id’s for each graph.} 
\footnotetext[6]{The original is available on \hyperlink{https://www.kaggle.com/ellipticco/elliptic-data-set}{https://www.kaggle.com/ellipticco/elliptic-data-set}}

\subsection{Unsupervised Representation Learning}\label{sec:unsupervised}
\subsubsection{Transductive Setting}~\label{sec:trans}
% \noindent\textbf{Baselines.}\quad We conduct experiments with 12 baselines which consist of three categories: supervised methods and self-supervised generative methods, self-supervised contrastive methods. Specifically, we compare with three supervised baselines: empirical risk minimization~(ERM)~\cite{erm}, invariant risk minimization (IRM)~\cite{irm}, and a recent proposed graph OOD method dubbed EERM~\cite{eerm}. We also compare various unsupervised node-level representation learning methods: three self-supervised generative methods including GAE~\cite{gae}, VGAE~\cite{gae}, GraphMAE~\cite{gmae} and seven self-supervised contrastive methods: DGI~\cite{dgi}, MVGRL~\cite{mvgrl}, GRACE~\cite{grace}, RoSA~\cite{rosa}, BGRL~\cite{bgrl}, COSTA~\cite{costa}, SwAV~\cite{swav}. The descriptions of these methods can be found in Appendix~\ref{app:baselines}.
In this subsection, we focus on validating our proposed algorithm under the transductive setting, where the test nodes will participate in message passing~\cite{gilmer2017neural} during training following~\cite{good}. 

\noindent\textbf{Baselines.} We conduct experiments with 12 baselines from three categories: (i)~supervised methods, including empirical risk minimization~(\textbf{ERM})~\cite{erm}, invariant risk minimization (\textbf{IRM})~\cite{irm}, and a recent proposed graph OOD method \textbf{EERM}~\cite{eerm}; (ii)~self-supervised generative methods including Graph Autoencoder (\textbf{GAE})~\cite{gae}, Variational Graph Autoencoder (\textbf{VGAE})~\cite{gae}, Self-Supervised Masked Graph Autoencoders (\textbf{GraphMAE})~\cite{gmae}; (iii)~self-supervised contrastive methods including Deep Graph Infomax (\textbf{DGI})~\cite{dgi}, Contrastive Multi-View Representation Learning on Graphs (\textbf{MVGRL})~\cite{mvgrl}, Deep Graph Contrastive Representation Learning (\textbf{GRACE})~\cite{grace}, A Robust Self-Aligned Framework for Node-Node Graph Contrastive Learning (\textbf{RoSA})~\cite{rosa}, Bootstrapped Representation Learning on Graphs (\textbf{BGRL})~\cite{bgrl}, Covariance-Preserving Feature Augmentation for Graph Contrastive Learning (\textbf{COSTA})~\cite{costa}, Unsupervised Learning of Visual Features by Contrasting Cluster Assignments (\textbf{SwAV})~\cite{swav}. The detailed descriptions of these baselines can be found in Appendix~\ref{app:baselines}.

\noindent\textbf{Experimental setup.} We use the same graph encoder across different datasets for a fair comparison following~\cite{good}. We use grid search to find other hyper-parameters (\eg, learning rate, epochs) for different methods. For all experiments, we select the best checkpoints for ID and OOD tests according to results on ID and OOD validation sets following~\cite{good}, respectively. Experimental details and hyper-parameter selections are provided in Appendix~\ref{app:hyper}. For evaluating unsupervised methods, a linear classifier will be built on the frozen trained encoder after finishing pre-training. The reported results are the mean performance with standard deviation after 10 runs following~\cite{good}.

\noindent\textbf{Analysis.}\quad Based on the experimental results listed in Table~\ref{tab:trans_concept} and \ref{tab:trans_covariate}, we can draw the following conclusions: firstly, we find strong self-supervised methods (\eg, GRACE, BGRL, COSTA) are more robust to distribution shifts (concept shift in Table~\ref{tab:trans_concept} and covariate shift in Table~\ref{tab:trans_covariate}) compared to supervised methods. For instance, on GOOD-CBAS and GOOD-WebKB datasets, GRACE surpasses the best supervised method by large margins (over 6\% absolute improvement). Interestingly, we find the methods designed for OOD generalization (\ie, IRM) and graph OOD generalization (\ie, EERM) do not attain superior performance than the standard ERM on most of the datasets. For example, EERM shows superior OOD performance compared to ERM in only one experiment, and IRM outperforms ERM in four out of ten experiments across the conducted evaluations. This phenomenon is also observed in \cite{good,ahuja2020empirical,rosenfeld2021risks}, showcasing the challenge of achieving invariant prediction in non-Euclidean graph settings. 

Furthermore, our method surpasses other SOTA self-supervised methods on the OOD test set of all datasets by a considerable margin while achieving comparable performance in the in-distribution test set. For instance, on small datasets such as GOOD-CBAS and GOOD-WebKB, our method outperforms GRACE\footnote{MARIO is built up on GRACE according to our recipe. So, we make a comparison with GRACE here.} by over 2\% absolute accuracy on the OOD test set. On larger datasets such as GOOD-Cora and GOOD-Twitch, our method still outperforms other methods which shows its superiority. For instance, under covariate shift, MARIO surpasses other methods by over 7\% absolute accuracy on the GOOD-Twitch OOD test set. These statistics prove the effectiveness of our design.


\begin{table*}[htp]
\caption{Experimental results of all methods under concept shift. The bold font means the top-1 performance and the underline represents the second performance across the unsupervised methods. 'ID' represents in-distribution test performance and 'OOD' means out-of-distribution test performance. (OOM: out-of-memory on a GPU with 24GB memory)}
\label{tab:trans_concept}
\centering
\scalebox{0.95}{
\begin{tabular}{l|cc|cc|cc|cc|cc}
\toprule
\toprule
\multirow{3}{*}{concept shift} & \multicolumn{4}{c|}{GOOD-Cora}                   & \multicolumn{2}{c|}{GOOD-CBAS} & \multicolumn{2}{c|}{GOOD-Twitch} & \multicolumn{2}{c}{GOOD-WebKB} \\
                           & \multicolumn{2}{c}{word} & \multicolumn{2}{c|}{degree}& \multicolumn{2}{c|}{color}    & \multicolumn{2}{c|}{language}   & \multicolumn{2}{c}{university} \\
                           & ID         & OOD         & ID          & OOD          & ID            & OOD           & ID             & OOD            & ID            & OOD            \\
\midrule
ERM                        & 66.38±0.45 & 64.44±0.18  & 68.60±0.40  & 60.76±0.34   & 89.79±1.39    & 83.43±1.19    & 80.80±1.00     & 56.92±0.92     & 62.67±1.53    & 26.33±1.09     \\
IRM                        & 66.42±0.41 & 64.29±0.31  & 68.57±0.35  & 61.45±0.24   & 89.64±1.21    & 82.29±1.14    & 78.87±1.04     & 59.30±1.79     & 62.67±1.10    & 26.88±1.42     \\
EERM                       & 65.10±0.44 & 62.45±0.19  & 66.95±0.44  & 56.58±0.25   & 79.07±2.12    & 64.50±1.01    & OOM            & OOM            & 62.50±2.01    & 28.07±3.23      \\
\midrule
% Random-Init                & 37.53±1.74 & 32.12±1.24  & 37.82±1.71  & 27.74±1.14   &               &               &                &                & 60.33±2.21    & 27.07±1.70     \\
GAE                        & 60.65±0.89 & 58.00±0.55  & 62.59±1.11  & 53.44±0.80   & 75.28±1.36    & 68.07±2.05    & 81.25±0.81     & 51.51±1.05     & 62.17±3.34    & 25.78±1.85     \\
VGAE                       & 63.19±0.53 & 60.35±0.47  & 61.65±0.66  & 54.28±0.28   & 76.50±0.50    & 59.07±0.56    & 80.46±0.53     & 55.56±4.53     & 62.50±2.38    & 24.40±2.57     \\
GraphMAE                   & \underline{66.44±0.46} & \underline{64.87±0.30}  & 67.95±0.46  & 59.41±0.39   & 89.14±0.89    & 82.93±0.93    & 80.05±0.64     & 59.38±1.49     & 61.83±3.37    & 29.27±2.15     \\
DGI                        & 63.33±0.56 & 60.71±0.49  & 65.93±1.02  & 55.83±0.53   & 91.22±1.47    & 85.00±1.66    & 80.05±0.87     & 59.16±1.88     & 61.83±2.83    & 28.63±1.92      \\
MVGRL                      & OOM        & OOM         & OOM         & OOM          & 88.57±1.15    & 76.50±1.17    & OOM            & OOM            & 62.00±3.79    & 28.26±4.20     \\
GRACE                      & 65.61±0.61 & 63.92±0.44  & \textbf{68.59±0.35}  & 60.15±0.45   & 92.00±1.39    & 88.64±0.67    & \textbf{83.43±0.63}     & \underline{60.45±1.46}     & 64.00±3.43    & \underline{34.86±3.43}  \\
RoSA                       & 64.06±0.67 & 62.44±0.39  & 67.07±0.65  & 57.68±0.44   & 90.78±2.27    & 85.93±2.14    & 82.39±0.42     & 57.45±2.16     & 64.17±4.10    & 32.20±2.15     \\
BGRL                       & 65.18±0.43 & 63.43±0.45  & 66.83±0.80  & 59.63±0.38   & 92.36±1.16    & 87.14±1.60    & 82.52±0.60     & 55.48±1.48     & 63.67±2.33    & 31.47±3.43     \\
COSTA                      & 65.05±0.80 & 62.37±0.45  & 66.76±0.87  & 55.73±0.36   & \underline{93.50±2.62}    & \underline{89.29±3.11}    & 83.15±0.30 & 55.03±3.22     & 61.66±2.58    & 32.39±2.13 \\
% ArCL                       &            &             & 67.64±0.57  & 59.71±0.44   &               &               &                &                & 65.00±3.94    & 35.41±1.97 \\      
SwAV                       & 62.22±0.53 & 59.79±0.53  & 64.65±0.94  & 55.06±0.39   & 89.00±0.79    & 81.72±0.66    & \underline{83.32±0.15}     & 59.69±1.97     & \underline{65.17±3.76}    & 29.36±2.01    \\
\midrule
MARIO                       & \textbf{67.11±0.46} & \textbf{65.28±0.34}  & \underline{68.46±0.40}  & \textbf{61.30±0.28}   & \textbf{94.36±1.21}    & \textbf{91.28±1.10}    & 82.31±0.54     & \textbf{63.33±1.72}     & \textbf{65.67±2.81}    & \textbf{37.15±2.37}     \\
\bottomrule
\end{tabular}}
\end{table*}

\begin{table*}[htp]
\caption{Experimental results of all methods under covariate shift. The bold font means the top-1 performance and the underline represents the second performance across the unsupervised methods. 'ID' represents in-distribution test performance and 'OOD' means out-of-distribution test performance. (OOM: out-of-memory on a GPU with 24GB memory)}
\label{tab:trans_covariate}
\centering
\scalebox{0.95}{
\begin{tabular}{l|cc|cc|cc|cc|cc}
\toprule
\toprule
\multirow{3}{*}{covariate shift} & \multicolumn{4}{c|}{GOOD-Cora}                                   & \multicolumn{2}{c|}{GOOD-CBAS} & \multicolumn{2}{c|}{GOOD-Twitch} & \multicolumn{2}{c}{GOOD-WebKB} \\
                           & \multicolumn{2}{c}{word} & \multicolumn{2}{c|}{degree}& \multicolumn{2}{c|}{color}    & \multicolumn{2}{c|}{language}   & \multicolumn{2}{c}{university} \\
                           & ID         & OOD         & ID          & OOD          & ID            & OOD           & ID             & OOD            & ID            & OOD            \\
\midrule
ERM                        & 70.50±0.41 & 64.69±0.33  & 72.46±0.49  & 55.53±0.50   & 92.00±3.08    & 77.57±1.29    & 70.98±0.41     & 49.35±5.09     & 39.34±1.79    & 14.52±3.14   \\
IRM                        & 70.48±0.26 & 64.53±0.57  & 71.98±0.34  & 53.72±0.46   & 90.86±2.41    & 78.86±1.67    & 69.81±0.95     & 49.11±2.82     & 38.52±3.30    & 13.97±2.80     \\
EERM                       & OOM        & OOM         & OOM         & OOM          & 65.00±2.57    & 57.43±3.60    & OOM            & OOM            & 46.07±4.55    & 27.40±7.65     \\
\midrule
GAE                        & 56.63±0.79 & 48.93±0.93  & 66.30±0.88  & 34.01±0.87   & 73.00±2.16    & 60.86±3.01    & 67.24±1.23     & 47.65±2.49     & 45.08±6.32    & 28.02±6.29    \\
VGAE                       & 62.02±0.66 & 54.12±0.86  & 69.41±0.57  & 44.20±1.29   & 62.29±2.04    & 63.29±1.11    & 66.99±1.43     & \underline{50.48±4.58}     & 48.85±4.68    & 20.87±6.69     \\
GraphMAE                   & 68.14±0.43 & 64.00±0.33  & \textbf{73.36±0.56}  & 53.75±0.55   & 67.28±3.03    & 67.28±1.49    & 68.84±1.20     & 48.02±2.79     & 48.03±4.34    & 30.00±8.09     \\
DGI                        & 60.85±0.75 & 57.03±0.67  & 68.97±0.41  & 41.75±0.88   & 69.57±4.09    & 59.71±3.43    & 68.43±1.05     & 44.83±1.61     & 48.52±5.04    & 21.11±7.50     \\
MVGRL                      & OOM        & OOM         & OOM         & OOM          & 65.00±1.94    & 64.15±0.77    & OOM            & OOM           & \textbf{54.10±5.39}    & 16.59±6.51     \\
GRACE                      & \underline{68.77±0.33} & \underline{64.21±0.41}  & 72.69±0.34  & \underline{56.10±0.63}   & \underline{93.57±1.83}    & \underline{89.29±3.40}    & \underline{71.12±0.87} & 46.21±1.54 & 49.67±5.82    & 28.10±4.68    \\
RoSA                       & 68.19±0.56 & 62.48±0.61  & 71.04±0.62  & 52.72±0.79   & 84.71±4.14    &79.14±3.51     & 70.58±0.36     & 45.83±1.72     & 52.30±4.24    & \underline{34.24±7.92}     \\
BGRL                       & 67.23±0.43 & 61.33±0.36  & 72.11±0.39  & 49.15±0.73   & 89.00±2.56    & 79.86±3.29    & \textbf{71.43±0.53}     & 43.86±0.94     & 51.80±5.55    & 30.32±7.61    \\
COSTA                      & 65.28±0.60 & 60.33±0.53  & 70.65±0.62  & 54.03±0.28   & 92.29±1.59    & 82.71±2.74    & 69.29±1.37     & 49.07±2.13     & 50.49±3.01    & 29.84±4.75   \\
SwAV                       & 63.29±1.01 & 56.98±0.94  & 70.27±0.73  & 43.00±0.52   & 89.57±1.12    & 81.43±1.69    & 69.19±0.93     & 49.37±2.96     & 49.84±4.82    & 30.55±6.72   \\
\midrule
MARIO                       & \textbf{69.99±0.54} & \textbf{65.06±0.34}  & \underline{72.73±0.43}  & \textbf{57.73±0.45}  & \textbf{94.57±2.46}    & \textbf{91.00±2.48}     & 68.31±0.78 & \textbf{57.37±1.37}     & \underline{53.94±3.23}    & \textbf{35.24±4.98}   \\
\bottomrule
\end{tabular}}

\end{table*}

\subsubsection{Inductive Setting}
In this subsection, we conduct experiments under the inductive settings, where the test nodes are kept unseen during training. This setting is more suitable for domain generalization.
% But we think it is more convincing that conduct experiments under inductive settings which means test nodes are unseen during training. This setting is more appropriate for domain generalization.

\noindent\textbf{Baselines:} For GOOD-WebKB and GOOD-CBAS datasets, we adopt ERM, IRM, GraphMAE, and GRACE as our baselines. And for Amazon-Photo and Elliptic datasets, we select ERM, EERM, and GRACE as our baselines.

\noindent\textbf{Experimental setup:} For GOOD-WebKB and GOOD-CBAS datasets, we use the same model configuration in Section~\ref{sec:trans}.
% Besides, we add experiments on Amazon-Photo dataset~\cite{yang2016revisiting} and Elliptic~\cite{elliptic} dataset in this subsection. 
For Amazon-Photo dataset~\cite{yang2016revisiting} and Elliptic~\cite{elliptic} dataset, they consist of many snapshots (training data and testing data use different snapshots) which are naturally inductive. For Amazon-Photo dataset, we use 2-layer GCN~\cite{gcn} as the encoder and for elliptic dataset, we use 5-layer GraphSAGE~\cite{sage} as encoder following~\cite{eerm}.

% Figure environment removed

\noindent\textbf{Analysis:}
According to Figure~\ref{fig:amazon},\ref{fig:elliptic},\ref{fig:ind_con},\ref{fig:ind_cov}, we can draw following conclusions:
firstly, based on Figure~\ref{fig:amazon}, it is evident that our method outperforms other representative supervised and self-supervised methods on all test graphs (T1$\sim$T8). This superiority is reflected in the larger median value of our method compared to others. For instance, MARIO achieves over a 3\% absolute improvement compared to ERM in terms of the mean value of eight median values. Additionally, our method demonstrates higher stability across different random initializations, as indicated by the closer proximity of the first and third quartile values to the median value~(\eg, the difference of first and third quartile values of ERM, EERM, GRACE and MARIO are 4.2, 3.3, 6.7 and 1.0 on T8 respectively which indicates MARIO is much more stable than other methods). Furthermore, our method exhibits consistent performance across different graphs (\eg, The standard deviation of median values on T1$\sim$T8 for ERM, EERM, GRACE, and MARIO are 0.4, 1.1, 1.2, and 0.3, respectively.), indicating its robustness to environmental variations and its ability to extract invariant features: $g(G^e) \approx g(G^{e'})$ for all $e, e' \in \mathcal{E}^\text{train}$. In summary, our method showcases enhanced OOD generalization capabilities.
% $g(G^e)g(G^e^\prime)$ where $any e, e^\prime in \mathcal{E}^{train}$

Secondly, from the results presented in Figure~\ref{fig:elliptic}, we can observe that our method averagely harvests 10.9\% absolute improvement over GRACE and 12.5\% absolute improvement over EERM in terms of F1 scores on Elliptic dataset. This demonstrates the effectiveness of our method in handling distribution shifts and improving performance compared to existing approaches. It is worth noting that GRACE's performance worsens over time, indicating its inability to handle distribution shifts effectively. In contrast, our method consistently achieves better F1 scores, except for T9, which is caused by the dark market shutdown occurred after T7~\cite{elliptic}. The emergence of such an event introduces significant variations in data distributions, which subsequently results in performance degradation for all methods. Indeed, this event serves as an unpredictable external factor that introduces significant challenges for models trained on limited training data. The results indicate that the performance heavily depends on available training data. Nonetheless, our approach outperforms other methods even in such an extreme case. This highlights the effectiveness of our method in addressing distribution shifts and improving generalization performance.

Finally, based on the observations from Figure~\ref{fig:ind_con} and Figure~\ref{fig:ind_cov} MARIO demonstrates the best performances on both ID and OOD test sets for GOOD-WebKB and GOOD-CBAS datasets, under both concept shift and covariate shift. Notably, MARIO outperforms other methods by more than 3\% and 10\% absolute improvement on GOOD-WebKB and GOOD-CBAS, respectively, under covariate shift. We can draw similar conclusions as discussed in Section~\ref{sec:trans}. Even under the inductive setting, our method continues to demonstrate excellent OOD generalization capabilities and achieves comparable or even improved in-distribution test performance. These statistical results further validate the effectiveness of our method in handling distribution shifts and enhancing generalization performance.

Overall, the observations we have made provide strong evidence of the great capacity of our method for handling distribution shifts, validating its effectiveness and potential for real-world applications.



% Figure environment removed

% Figure environment removed


% Figure environment removed


\subsection{Ablation Studies}\label{sec:ablation}
\noindent Table~\ref{tab:aba} provides a detailed analysis of the effect of each component according to our proposed recipe for improving OOD generalization in graph contrastive learning. Let's examine the different variants of our method and their impact on performance.
Specifically, MARIO~(w/o ad) represents MARIO without  adversarial augmentation. MARIO~(w/o cmi) denotes we only maximize the mutual information between positive pairs without considering conditional mutual information. MARIO~(w/o cmi, ad) means a vanilla graph contrastive method that is similar to GRACE. 

From Table~\ref{tab:aba}, we can find MARIO~(w/o cmi) lags far behind MARIO on OOD test set which demonstrates appropriately minimizing the redundant information (\ie, conditional mutual information) is essential to improve OOD generalization of GCL methods. And adversarial augmentation can also boost OOD generalization because it can approximately serve as a supermum operator to learn more invariant features  discussed in Section~\ref{sec:aug}. Based on the analysis of these variants, it is evident that the proposed improvements on data augmentation and contrastive loss in the recipe are both effective in enhancing graph OOD generalization. Each component contributes to the overall performance improvement, and their combination leads to a stronger self-supervised graph learner in terms of graph OOD generalization. 

In short, the findings from Table~\ref{tab:aba} support the rationale behind your proposed recipe and provide empirical evidence of the effectiveness of each proposed component. By incorporating these enhancements, our method achieves superior performance in handling distribution shifts and improving graph OOD generalization in graph contrastive learning.
\begin{table*}[htp]
\caption{Ablation studies for MARIO by masking each component.}
\label{tab:aba}
\centering
\scalebox{0.9}{
\begin{tabular}{l|cc|cc|cc|cc|cc}
\toprule
\toprule
\multirow{3}{*}{concept shift} & \multicolumn{4}{c|}{GOOD-Cora}                       & \multicolumn{2}{c|}{GOOD-CBAS} & \multicolumn{2}{c|}{GOOD-Twitch} & \multicolumn{2}{c}{GOOD-WebKB} \\
                           & \multicolumn{2}{c}{word} & \multicolumn{2}{c|}{degree}& \multicolumn{2}{c|}{color}    & \multicolumn{2}{c|}{language}   & \multicolumn{2}{c}{university} \\
                           & ID         & OOD         & ID          & OOD          & ID            & OOD           & ID             & OOD            & ID            & OOD            \\
\midrule
MARIO                      & \textbf{67.11±0.46} & \textbf{65.28±0.34}  & \textbf{68.46±0.40}  & \textbf{61.30±0.28}      & \textbf{94.36±1.21}  & \textbf{91.28±1.10}    & 82.31±0.54     & \textbf{63.33±1.72}     & \textbf{65.67±2.81}    & \textbf{37.15±2.37}     \\
MARIO(w/o ad)              & 66.23±0.53 & 64.02±0.18  & 67.88±0.38  & 60.46±0.29   & 93.21±1.25    & 90.29±0.91    & 82.42±0.73     & 60.50±1.02     & 64.83±2.83    & 36.51±3.25    \\
MARIO(w/o cmi)             & 65.32±0.60 & 63.51±0.32  & 68.14±0.32  & 61.19±0.34   & 94.15±1.23    & 90.57±1.96    & \textbf{82.51±0.56}     & 61.41±2.63     & 64.50±4.35    & 35.78±2.53     \\
MARIO(w/o cmi, ad)         & 64.67±0.55 & 63.11±0.32  & 67.95±0.65  & 60.01±0.57   & 93.36±1.66    & 89.64±1.73    & 81.90±0.75     & 60.12±1.60     & 64.17±3.67    & 34.13±2.38     \\
\bottomrule
\end{tabular}}
\end{table*}
% & 65.32±0.60 & 63.51±0.32 exchange 64.67±0.55 & 63.11±0.32
% 68.14±0.32       id ood test: 60.95±0.43       ood ood test: 61.19±0.34


\subsection{Sensitivity Analysis}\label{sec:sensitivity}
\noindent In this subsection, we will analyze some important hyper-parameters of our method. We conduct sensitivity analysis on GOOD-WebKB dataset with concept shift, we chose two sensitive hyper-parameters (\ie, the coefficient $\gamma$ of condition mutual information in Equation~\ref{equ:cmi} and the number of prototypes $|C|$ in Equation~\ref{equ:pq}). The coefficient of CMI range in $[0.001, 0.01, 0.1, 0.5, 1]$ and the number of prototypes $|C|$ ranges in $[10, 50, 100, 200, 300]$. From Figure~\ref{fig:sensitivity}, we can observe that $\gamma$ reaches 0.1 and $|C|$ reaches 100 or 200 can achieve the best OOD test accuracy. Both higher and lower values of $\gamma$ result in suboptimal performance. This finding aligns with previous research such as DIB~\cite{dib}, indicating that an appropriate compression level is crucial for achieving optimal performance. Extremely high or low compression values are not ideal. 

Regarding the number of prototypes $|C|$, based on the results shown in Figure~\ref{fig:sensitivity}, it is found that setting $|C|=100$ leads to the best performance in terms of OOD test accuracy. This choice provides a moderate number of pseudo labels, which is beneficial for the learning process. 

Based on the sensitivity analysis, we determined that setting $\gamma=0.1$ and $|C|=100$ on most datasets. These hyperparameter values strike a balance between compression level and the number of prototypes, resulting in improved graph OOD generalization.
% Figure environment removed


\subsection{Integrated with Other Models}\label{sec:other_models}
% Figure environment removed

\begin{table}[htp]
\caption{Results of different learning approaches with different encoding models (\ie, GCN, GraphSAGE, GAT).}
\label{tab:others}
\centering
\scalebox{0.9}{
\begin{tabular}{cc|cc|cc}
\toprule
\toprule
\multirow{3}{*}{Model}& \multirow{3}{*}{Method} & \multicolumn{2}{c|}{GOOD-CBAS} & \multicolumn{2}{c}{GOOD-WebKB} \\
                & & \multicolumn{2}{c|}{color}    & \multicolumn{2}{c}{university} \\
                &   & ID          & OOD         & ID          & OOD            \\
\midrule
\multirow{3}{*}{GCN} 
&ERM               & 89.79±1.39 & 83.43±1.19  &  62.67±1.53 & 26.33±1.09         \\
&GRACE             & 92.00±1.39 & 88.64±0.67  &  64.00±3.43 & 34.86±3.43        \\
&MARIO             & 94.36±1.21 & 91.28±1.10  &  65.67±2.81 & 37.15±2.37        \\ \bottomrule
\multirow{3}{*}{SAGE} 
&ERM               & 95.07±1.51 & 75.14±1.19  & 73.67±2.08  & 46.33±3.42       \\
&GRACE             & 95.29±1.11 & 74.43±2.36  & 70.50±5.06  & 49.54±3.83        \\
&MARIO             & 96.00±1.07 & 76.29±3.01  & 71.00±3.82  & 51.74±4.63        \\ \bottomrule
\multirow{3}{*}{GAT} 
&ERM               & 78.64±3.63 & 72.93±2.64  & 61.33±3.71  & 28.99±2.63        \\
&GRACE             & 84.57±1.79 & 78.36±1.60  & 59.50±2.36  & 35.78±3.26        \\
&MARIO             & 84.93±1.95 & 80.43±1.89  & 62.17±4.78  & 38.17±3.10        \\
\bottomrule
\end{tabular}}
\end{table}



\noindent In the subsection, we demonstrate the model-agnostic nature of the recipe by integrating it with various graph neural network (GNN) models, including GCN, GraphSAGE, and GAT.

From Table~\ref{tab:others}, it can be observed that regardless of the specific GNN model used as the encoder, our method consistently achieves the best performance on the OOD test set. This indicates the effectiveness and robustness of our method across different GNN models.
By achieving superior performance across different GNN models, MARIO demonstrates its versatility and ability to improve the OOD generalization of various graph neural models. This highlights the broad applicability and effectiveness of our recipe in enhancing the performance of different GNN encoders.

Furthermore, we integrate our recipe with other GCL methods in Appendix~\ref{app:other_methods}. The results demonstrate our recipe can boost the OOD generalization ability of various GCL methods which means our recipe can serve as a plug-in for many current classical GCL methods.

% Figure environment removed

\subsection{Visualization}\label{sec:vis}
\subsubsection{Metric Score Curves}
We present metric score curves for ERM and MARIO, including training, ID validation, ID testing, OOD validation, and OOD testing accuracy, in Figure~\ref{fig:curve2}. Notably, MARIO demonstrates superior convergence with approximately 10\% absolute improvement on the OOD test set compared to ERM. Furthermore, MARIO effectively narrows the performance gap between in-distribution and out-of-distribution performance, showcasing its efficacy in enhancing OOD generalization for graph data. More metric score curves can be found in Appendix~\ref{app:curves}.


\subsubsection{Feature Visualization}
In order to assess the quality of learned embeddings, we adopt t-SNE~\cite{tsne} to visualize the node embedding on GOOD-Cora dataset (concept shift in word domain) using random-init of GCN, EERM, GRACE, and MARIO, where different classes have different colors in Figure~\ref{fig:vis}. For clarity, we select eight classes with the largest number of nodes to enhance the informativeness and interpretability of the visualization. We can observe that the 2D projection of node embeddings learned by MARIO has a better separation of clusters, which indicates the model can help learn representative features for downstream tasks. It has to note that we depict both ID nodes and OOD nodes in the same figure. 

Besides, we also separately visualize ID nodes and OOD nodes in the different figures in the Appendix~\ref{app:feature}. And we can find MARIO performs a clearer separation of clusters whether on ID nodes or OOD nodes compared to other methods.



\section{Sparse View Experiments}
\label{sec:corrattn_experiments}

\paraspace
\ptitle{Datasets.}
We adopt the setup of various existing works \cite{jin2021sparseplanes, tan2023nopesac} to ensure a fair comparison, 
\jiajia{which includes two large-scale sparse view datasets as benchmarks.}

\textbf{ScanNetv2 Dataset.}
% We utilized the indoor ScanNetv2 \cite{Dai:etal:CVPR2017, Dai:scannetv2web} video dataset, which includes plane annotations generated by \cite{Liu:etal:CVPR2019:Planercnn}. 
We use a more challenging sparse view split of the ScanNetv2 dataset created by \cite{Dai:etal:CVPR2017,Dai:scannetv2web,Liu:etal:CVPR2019:Planercnn,tan2023nopesac}, consisting of 17,237/4,051 image pairs from 1,210/303 non-overlapping scenes for training/testing. 
% We denote it as ScanNetv2 to make a distinction between the ScanNetv1 in Section \ref{sec:experiments} for monocular plane recovery. 
Different to ScanNetv1 for monocular plane recovery, ScanNetv2 was proposed to contain a lower overlapping ratio between frames and more complex camera rotation distributions. The average overlap ratio of adjacent frames were 20.6\% and 18.6\% in the training and test sets, respectively. The image size is 640 × 480 for all baselines.

\textbf{MatterPort3D Dataset.}
In contrast to ScanNetv2, the sparse view version of the MatterPort3D dataset\cite{chang2017matterport3d} is created in a semi-simulated manner, whose RGB images are rendered from approximated 3D planar meshes during the generation process of plane annotations \cite{jin2021sparseplanes}, thereby mitigating real-world impact like lighting variation. 
However, these rendered RGB images consist solely of planes along with numerous small erroneous facets caused by approximated 3D planar meshes, increasing \jiajia{the difficulty for accurate plane detection. Therefore, evaluations on MatterPort3D may provide unfair advantages to all two-stage baselines \cite{jin2021sparseplanes, agarwala2022planeformers, tan2023nopesac} with a dense pixel-based pose initialization, and cause challenges to plane-based methods (like ours). We have confirmed such findings in subsequent experiments and regard MatterPort3D mainly as a complement benchmark compared to ScanNetv2 dataset.} The training set and testing set consist of 31,932 and 7,996 image pairs, respectively, exhibiting an overlap ratio of about 21.0\% \cite{tan2023nopesac}. The image size is also kept to 640 × 480.

\paraspace
\ptitle{Evaluation Metrics.}
To evaluate relative camera pose, we use geodesic distance and euclidean distance to measure rotation and translation error, respectively. We also employ three popular statistical measurements: the median to reflect overall prediction accuracy, the mean to account for large outlier errors, and the percentage of errors below a certain threshold.

To assess the overall planar reconstruction performance, we initially present the predicted results obtained from monocular images on the above datasets, utilizing two metrics: plane segmentation and plane recovery recall from Section \ref{sec:experiments}. For the merged 3D reconstruction results derived from two adopted views, we employ the average precision (AP) metric \cite{jin2021sparseplanes}, treating each reconstructed 3D plane as a detection target for evaluation.    A true positive of 3D plane detection necessitates three conditions: (i) (Mask) an intersection-over-union value to the ground-truth mask $\geq 0.5$;    (ii) (Normal) arccos value between the predicted and ground truth normal $\leq \alpha$ where $\alpha \in \{30^{\circ}, 15^{\circ}, 5^{\circ}\}$; and (iii) (Offset) absolute difference between predicted and ground truth offset $\leq \beta$ where $\beta \in \{1\text{m}, 0.5\text{m}, 0.2\text{m}\}$.
To further assess the network's capability in implicitly learning plane-aware correspondences during reconstruction, Section \ref{sec:unified_plane_emb} presents the number of true positives (TP) and calculates precision, recall, and F-score for plane correspondences.

\subsection{Baselines and their Variants}
% \paraspace
% \ptitle{Baselines and their Improved Variants.}
3D planar reconstruction from sparse views demands the resolution of multiple sub-tasks, encompassing relative camera pose estimation, monocular planes recovery and multi-view plane matching.  We conducted a comparative analysis between representative baselines in terms of both overall results and intermediate outcomes.


\paraspace
\ptitle{Public Baselines} We compared our method with several state-of-the-art and leading sparse views planar reconstruction methods, including SparsePlanes \cite{jin2021sparseplanes}, PlaneFormers \cite{agarwala2022planeformers}, and NOPE-SAC \cite{tan2023nopesac} which are all multi-stage approaches. Specifically, all these baselines begin by estimating one or more initial camera poses from learned dense features of two input frames, and subsequently refine these camera poses using predicted planes via post optimization algorithms\cite{jin2021sparseplanes},  SIFT-like keypoints \cite{jin2021sparseplanes}, or separate network models\cite{agarwala2022planeformers, tan2023nopesac}.
Therefore, their primary contribution usually lies in an elaborated pose refinement module while treating monocular plane prediction and initial camera pose prediction as given knowledge from external and distinct modules. 
These facts are also our main differences by performing two-view planar reconstruction in a purely end-to-end manner from images without any external priors and matching supervision. 

As to the evaluation of camera pose estimation, we additionally consider two popular baselines SuperGlue \cite{sarlin2020superglue} and Pose ViT \cite{rockwell20228posevit}. SuperGlue is a competitive graph-based point matching network that leverages the estimated essential matrix with a RANSAC solver to obtain the relative camera pose. Therefore, SuperGlue inherits the intrinsic scale ambiguity of the Essential Matrix,  so we solely focus on comparing the rotation error during its evaluation. Pose Vit uses embeddings of tokenized image patches extracted from a ViT \cite{vit2021} and follows the principle of the Eight-Point Algorithm \cite{hartley1997defense} via another ViT module to directly estimate relative camera pose between two input images. Our proposed PlaneRecTR++ draws inspiration from its design philosophy but brings several important modifications to better fit unified query learning as well as end task. 
Specifically, we choose to use plane-aware embeddings instead of those from raw image patches,  even without requiring the Quadratic Position Encodings used in \cite{rockwell20228posevit}. Our method also introduces distinct differences in the attention structure, enhancing interpretability and yielding better performance.  



\begin{table}
% \small
% \renewcommand\arraystretch{1.1}
\centering
\caption{Plane segmentation on the ScanNetV2 and MatterPort3D datasets. }
\vspace{-5pt}
\resizebox{0.95\linewidth}{!}{ 
    % \begin{center}
    % \setlength{\tabcolsep}{1.3mm}f
    \begin{tabular}{c|ccc|ccc}
    \toprule
        \multirow{2}{*}{Method} & \multicolumn{3}{c|}{ScanNetV2} & \multicolumn{3}{c}{MatterPort3D} \\ 
                                & VI~$\downarrow$ & RI~$\uparrow$ & SC~$\uparrow$ & VI~$\downarrow$ & RI~$\uparrow$ & SC~$\uparrow$\\\midrule
        PlaneTRP \cite{Tan:etal:ICCV2021:Planetr, tan2023nopesac}          & 1.291 & 0.880 & 0.716 &1.458  & 0.897 & 0.683 \\ \midrule
        % PlaneMaskParamTR            & 0.694 & 0.937 & 0.855 & 1. & 0. &0. \\
        Ours (Monocular)       & 0.781  & \textbf{0.939} & 0.835 & \textbf{0.920}  & \textbf{0.934} & \textbf{0.773}  \\
        Ours
   &\textbf{0.777}  & 0.938  & \textbf{0.836} & 0.946 & 0.931 & 0.766  \\
    \bottomrule

    \end{tabular}
}
    % \end{center}
\vspace{-5pt}
\label{tab:planerectr++_scannetv2_mp3d_seg}
\end{table}


\begin{table}[t]
\centering
\centering
\caption{Per-pixel/plane recalls on the ScanNetv2 and MatterPort3D datasets.}
% \large
\LARGE
\resizebox{1.0\linewidth}{!}
{    \renewcommand{\arraystretch}{1.3} % Default value: 1
    \begin{tabular}{ccccccc}
    \toprule
        \multirow{3}{3cm}{\centering {}\\{\huge Method}} & \multicolumn{4}{c}{Per-Pixel/Per-Plane Recalls~$\uparrow$ } & \multicolumn{2}{c}{Plane Parameters}  \\ 
         \cmidrule(l{15pt}r{15pt}){2-5}
        &\multicolumn{2}{c}{Depth}&\multicolumn{2}{c}{Normal} & \multicolumn{2}{c}{Estimation Errors~$\downarrow$}\\
        \cmidrule(l{15pt}r{15pt}){2-3}\cmidrule(l{15pt}r{15pt}){4-5}\cmidrule(l{15pt}r{15pt}){6-7}

         & @0.10 m& @0.60 m& @$5^{\circ}$ & @$30^{\circ}$ & Normal (°) & Offset (mm)\\\midrule

        \multicolumn{7}{c}{ScanNetv2 dataset} \\
        \midrule
         
        PlaneTRP\cite{Tan:etal:ICCV2021:Planetr, tan2023nopesac} & 14.17/10.97  & 64.73/52.20 & 37.38/27.59 &67.52/55.97  & 14.95 & 237.90 \\\midrule
        Ours (Monocular)& 24.27/18.91& \textbf{80.66}/\textbf{64.98} & 57.71/41.15 &\textbf{82.11}/\textbf{67.27} & \textbf{10.11} & 193.33  \\
        Ours & \textbf{25.51}/\textbf{19.72} &80.58/64.89  & \textbf{58.81}/\textbf{41.29}& 82.07/67.12  &10.32 & \textbf{191.09} \\\midrule

        \multicolumn{7}{c}{MatterPort3D dataset} \\
        \midrule
        PlaneTRP\cite{Tan:etal:ICCV2021:Planetr, tan2023nopesac} & 25.87/20.56 & 61.40/56.96 & 53.89/\textbf{47.51} & 64.85/62.89 & 10.67 & 390.38 \\\midrule
        Ours (Monocular) & \textbf{31.36}/\textbf{24.57} & \textbf{71.13}/\textbf{63.96}& 56.41/44.86&  \textbf{73.78}/67.79&8.64& 390.65\\
        Ours & 28.51/22.86 & 69.96/63.69 & \textbf{58.00}/46.91& 72.84/\textbf{67.99} & \textbf{8.47} & \textbf{384.59}\\
        \bottomrule
    \end{tabular}
}
\vspace{-5pt}
\label{tab:planerectr++_scannetv2_mp3d_recall}
\end{table}
% \noindent \textbf{Plane segmentation and geometry prediction.}

\paraspace
\ptitle{Improved Baseline Variants} The monocular plane predictor of SparsePlanes and PlaneFormers is based on PlaneRCNN\cite{Liu:etal:CVPR2019:Planercnn}, while \jiajia{NOPE-SAC} utilizes an improved version of the leading PlaneTR \cite{Tan:etal:ICCV2021:Planetr} (denoted as PlaneTRP) to recover monocular planes. Motivated by \cite{tan2023nopesac}, to make fair comparisons, we include improved baseline variants by replacing the PlaneRCNN backbone of SparsePlanes and PlaneFormers with a more powerful PlaneTRP module, termed SparsePlanes-TRP and PlaneFormers-TRP.


% ### Move paragraph below to single-view 7.2
% We also compare the monocular plane predictions of our PlaneRecTR++ with PlaneTRP \cite[nopsesac] for single view evaluation to isolate the contribution of our intra???-frame plane query learning. Among them, we present our monocular plane prediction results obtained exclusively during the monocular pretraining phase and after the joint training phase, respectively denoted as Ours(Monocular) and Ours.  
% In the appendix, similarly, we also use the plane predictions of our PlaneRecTR++ and input them to other baseline methods for experiments, but find no significant improvement in results.



\begin{table*}[t!]
\centering
\caption{Comparison of relative camera pose on the ScanNetv2 dataset and the Matterport3D dataset.} 
\vspace{-5pt}
\resizebox{0.9\linewidth}{!}{ 
    \begin{tabular}{l|ccccc|ccccc}
    \toprule
        \multicolumn{1}{c|}{\multirow{2}{*}{Method}} & \multicolumn{5}{c|}{Translation} & \multicolumn{5}{c}{Rotation} \\ 
                                & Med.~$\downarrow$ & Mean~$\downarrow$ & ($\le$1m)~$\uparrow$ & ($\le$0.5m)~$\uparrow$& ($\le$0.2m)~$\uparrow$ & Med.~$\downarrow$ & Mean~$\downarrow$ & ($\le 30^{\circ}$)~$\uparrow$ & ($\le 15^{\circ}$)~$\uparrow$ & ($\le 10^{\circ}$)~$\uparrow$\\
        
        \midrule
        \multicolumn{10}{c}{ScanNetv2 dataset} \\
        \midrule
        SuperGlue~\cite{sarlin2020superglue} & - & - & - & - & - & 10.90 & 31.00 & 67.8\% & 56.0\% & 48.4\%\\
        % \textcolor{black}{LoFTR~\cite{LoFTR}} & - & - & - & - & - & \textbf{5.49} & 27.13 & 71.0\% & 63.3\% & 58.4\%\\
        NOPE-SAC Init.~\cite{tan2023nopesac}& 0.48 &0.72 &77.7\% &51.9\% &16.5\% &14.68 &26.75 &73.7\% &51.0\% &34.4\% \\
        Pose ViT~\cite{rockwell20228posevit}& 0.43 & 0.68 & 79.9\%  & 56.1\% & 18.7\% & 11.65 & 24.28  & 78.0\% & 59.7\% & 43.1\% \\
        \midrule
        SparsePlanes~\cite{jin2021sparseplanes}  & 0.56& 0.81& 73.7\% & 44.6\% & 10.7\% & 15.46& 33.38& 70.5\% & 48.7\% & 28.0\% \\
        PlaneFormers~\cite{agarwala2022planeformers} & 0.55 & 0.81 & 75.3\% & 45.5\% & 11.3\% & 14.34 & 32.08 & 73.2\% & 52.1\% & 32.3\% \\\midrule
        SparsePlanes-TRP~\cite{jin2021sparseplanes, Tan:etal:ICCV2021:Planetr} &  0.57 &  0.82 &  73.4\% &  43.6\% & 10.1\% & 14.57 & 32.36 & 72.8\% & 51.2\% & 30.1\% \\
        PlaneFormers-TRP~\cite{agarwala2022planeformers,Tan:etal:ICCV2021:Planetr} & 0.53 & 0.79 & 76.2\% & 47.0\% & 11.4\% & 13.81 & 31.58 & 74.5\% & 54.1\% & 33.6\% \\

        NOPE-SAC & 0.41  & 0.65  & 82.1\% &  59.2\%  &  20.9\% & 8.29  & 22.30  &  82.4\%  &  73.0\%  &  59.2\% \\ 
        \midrule
        NOPE-SAC Ref.~\cite{tan2023nopesac}  & 0.57  & 0.81  & 75.4\% & 43.2\% & 7.3\%   &14.91 &37.05  & 66.8\%   & 50.3\% &  34.8\% \\
        PlaneRecTR++ (ours) & \textbf{0.24}  & \textbf{0.46}  & \textbf{88.6\%}  & \textbf{76.3\%}   &  \textbf{43.2\%} & \textbf{4.30}  & \textbf{17.16}  &  \textbf{87.6\%}  &  \textbf{84.1\%}  & \textbf{79.7\%} \\
        % PlaneRecTR++-SPL (ours) & \textbf{0.24}  & \textbf{0.46}  & \textbf{88.3\%}  &  \textbf{76.4\%}  &  \textbf{43.4\%} & \textbf{4.36}  & \textbf{18.64}  &  \textbf{86.7\%}  &  \textbf{83.9\%}  &  \textbf{78.8\%} \\ 
        \midrule
        
        \multicolumn{10}{c}{MatterPort3D dataset} \\\midrule
        SuperGlue~\cite{sarlin2020superglue} & - & - & - & - & - & 3.88 & 24.17 & 77.8\% & 71.0\% & 65.7\%\\
        % \textcolor{black}{LoFTR~\cite{LoFTR}} & - & - & - & - & - & 5.85 & 33.13 & 67.0\% & 61.0\% & 57.0\% \\
        NOPE-SAC Init.~\cite{tan2023nopesac}& 0.69 & 1.08 & 65.0\% & 37.0\% & 10.1\% & 11.16 & 21.49 & 81.3\% & 60.5\% & 46.5\% \\
        Pose ViT~\cite{rockwell20228posevit}& 0.64 & 1.01 & 67.4\% & 39.9\% & 11.6\% & 8.01 & 19.13 & 85.4\% & 70.8\% & 57.8\% \\
        \midrule
        SparsePlanes~\cite{jin2021sparseplanes} & 0.63& 1.15 & 66.6\% & 40.4\% & 11.9\% & 7.33 & 22.78& 83.4\% & 72.9\% & 61.2\% \\
        PlaneFormers~\cite{agarwala2022planeformers} & 0.66 & 1.19 & 66.8\% & 36.7\% & 8.7\% & 5.96 & 22.20 & 83.8\% & 77.6\% & 68.0\% \\\midrule
        SparsePlanes-TRP~\cite{jin2021sparseplanes,Tan:etal:ICCV2021:Planetr} & 0.61 & 1.13 & 67.3\% & 41.7\% & 12.2\% & 6.87 & 22.17 & 83.8\% & 74.5\% & 63.3\% \\
        PlaneFormers-TRP~\cite{agarwala2022planeformers,Tan:etal:ICCV2021:Planetr} & 0.64 & 1.17 & 67.9\% & 38.7\% & 8.9\% & 5.28 & 21.90 & 83.9\% & 79.0\% & 70.8\% \\
        NOPE-SAC~\cite{tan2023nopesac} & 0.52  &  0.94  &  73.2\%  &   48.3\%  &   16.2\% & 2.77   &   \textbf{14.37}  &  \textbf{89.0\%}   &  \textbf{86.9\%}  &  \textbf{84.0\%} \\
        \midrule
        NOPE-SAC Ref.~\cite{tan2023nopesac}  &   1.53  & 1.92  &31.2\% & 11.8\%  & 2.5\%  & 3.88 &    27.95  & 76.8\%   &74.1\% & 70.9\%\\
        PlaneRecTR++ (ours) & \textbf{0.39}  & \textbf{0.86}  & \textbf{77.6\%}  & \textbf{58.5\%}   &  \textbf{24.3\%} & \textbf{2.60}  & 21.19  &  84.6\%  &  81.2\%  & 78.2\% \\
        % PlaneRecTR++-SPL (ours) & \textbf{0.38}  & \textbf{0.85}  & \textbf{78.3\%}  &  \textbf{60.1\%}  &  \textbf{26.0\%} & \textbf{2.43}  & 20.33  &  85.2\%  &  82.2\%  &  79.8\% \\ 
        
        
        \bottomrule
    \end{tabular}
}
\vspace{-5pt}
\label{tab:pose_comparison}
\end{table*}


\begin{table*}[!t]
\centering
\caption{Average Precision (AP) of 3D plane reconstruction given mask IoU ($\geq$0.5), normal angle error, and offset distance error. `All' means we consider all three conditions. `-Offset' and `-Normal' mean we ignore the offset and the normal conditions respectively.}
\vspace{-2mm}
\resizebox{0.8\linewidth}{!}{ 
    \begin{tabular}{l|ccc|ccc|ccc}
    \toprule
        \multicolumn{1}{c|}{\multirow{2}{*}{Method}} & \multicolumn{3}{c|}{Offset$\le$1m,~Normal$\le 30^{\circ}$} & \multicolumn{3}{c|}{Offset$\le$0.5m,~Normal$\le 15^{\circ}$} & \multicolumn{3}{c}{Offset$\le$0.2m,~Normal$\le 5^{\circ}$} \\
        & All & -Offset & -Normal & All & -Offset & -Normal & All & -Offset & -Normal \\\midrule
        \multicolumn{10}{c}{ScanNetv2 dataset} \\\midrule
        % SparsePlanes~\cite{jin2021planar} w/o Con. & 33.20& 34.12& 40.74& 22.89& 25.62& 33.67& 3.03& 4.52& 17.17 \\ 
        SparsePlanes~\cite{jin2021sparseplanes}  & 33.08& 34.12& 40.51& 21.69& 25.59& 32.20& 2.52&  4.50& 14.85\\
        PlaneFormers~\cite{agarwala2022planeformers} & 34.64& 35.47& 41.37& 24.48& 27.19& 34.69& 3.93& 5.52& 18.58\\\midrule
        % SparsePlanes-TRP~\cite{jin2021planar,planeTR} w/o Con. & 35.56& 36.51& 42.14& 26.01& 29.61& 35.12& 3.96& 6.10& 18.59\\  
        SparsePlanes-TRP~\cite{jin2021sparseplanes,Tan:etal:ICCV2021:Planetr}  & 35.32& 36.50& 41.92& 24.71& 29.55& 33.50& 3.21& 6.07& 15.32\\
        PlaneFormers-TRP~\cite{agarwala2022planeformers,Tan:etal:ICCV2021:Planetr} & 36.82& 37.87& 43.01& 27.41& 30.72& 36.31& 4.83& 7.02& 19.94\\
        NOPE-SAC \cite{tan2023nopesac}& 39.61 & 40.45 & 44.04 & 31.39 & 35.07 & 38.05 & 6.76 & 10.12 & 21.50 \\
        \midrule
        PlaneRecTR++ (ours) & \textbf{51.08}  & \textbf{51.73} & \textbf{55.08} & \textbf{44.24} &  \textbf{46.20} & \textbf{51.25} & \textbf{18.19} & \textbf{21.68} &  \textbf{36.86}\\
        % PlaneRecTR++-SPL (ours) & 49.73  & 50.52 & 53.62  & 43.04  & 44.98  & 49.85 &17.59  & 21.02  & 35.80 \\
        \midrule

        
        
        \multicolumn{10}{c}{MatterPort3D dataset} \\\midrule
        % SparsePlanes~\cite{jin2021planar} w/o Con. & 35.87 & 42.13 & 38.80 & 23.36 & 35.34 & 27.48 & 8.07 & 17.28 & 12.99 \\ 
        SparsePlanes~\cite{jin2021sparseplanes} & 36.02 & 42.01 & 39.04 & 23.53 & 35.25 & 27.64 & 6.76 & 17.18 & 11.52\\
        PlaneFormers~\cite{agarwala2022planeformers} & 37.62 & 43.19 & 40.36 & 26.10 & 36.88 & 29.99 & 9.44 & 18.82 & 14.78 \\\midrule
        % SparsePlanes-TRP~\cite{jin2021planar,planeTR} w/o Con. & 39.91 & 46.50 & 42.53 & 27.37 & 40.79 & 31.03 & 9.99 & 22.80 & 14.64 \\
        SparsePlanes-TRP~\cite{jin2021sparseplanes,Tan:etal:ICCV2021:Planetr} & 40.35 & 46.39 & 43.03 & 27.81 & 40.65 & 31.38 & 9.02 & 22.80 & 13.66 \\
        PlaneFormers-TRP~\cite{agarwala2022planeformers,Tan:etal:ICCV2021:Planetr} & 41.87 & 47.50 & 44.43 & 30.78 & 42.82 & 34.03 & 12.45 & 25.98 & 17.34 \\
        NOPE-SAC~\cite{tan2023nopesac} &  43.29 & 49.00 & 45.32 & 32.61 & \textbf{44.94} & 35.36 & \textbf{14.25} & \textbf{30.39} & 18.37 \\
        \midrule
        PlaneRecTR++ (ours) & \textbf{45.22}  &  \textbf{49.71}  & \textbf{48.23} & \textbf{34.66} & 43.90 &  \textbf{38.85} & 13.70 & 26.03  & \textbf{19.69} \\
        % PlaneRecTR++-SPL (ours) & 44.83  & 49.13 & 47.80  &\textbf{34.78}  & 43.92  & 38.67 & 14.12 &27.02  & \textbf{19.85} \\
        
        \bottomrule
    \end{tabular}
}
\vspace{-5pt}
\label{tab:3DPlane}
\end{table*}

\paraspace
\ptitle{NOPE-SAC and its Variants} 
The NOPE-SAC framework comprises four distinct modules\jiajia{: (1) A} Monocular Plane Predictor (PlaneTRP); \jiajia{(2) An attention} network for \jiajia{dense pixel-based} pose initialization; \jiajia{(3) A supervised} differentiable plane matching module based on optimal transport; \jiajia{(4) A neural RANSAC network for plane-level pose refinement.}
To provide a more comprehensive comparison in pose estimation to the leading NOPE-SAC \cite{tan2023nopesac}, we further introduce its two variants to better isolate the contribution of key modules.

\textbf{NOPE-SAC Init.} refers to the camera pose initialization network to kick off its overall procedure, \ie, module (2) above, which relies solely on dense pixel features. We present its pose accuracy to indicate the quality of learned pose prior \cite{tan2023nopesac}, acting as a similar role to external pose predictor in \cite{jin2021sparseplanes, agarwala2022planeformers}. 

\textbf{NOPE-SAC Ref.} \jiajia{refers to NOPE-SAC without module (2) and mainly relies on the pose refinement module for pose estimation. NOPE-SAC Ref. performs plane matching without initial pose and directly estimates relative pose from sparser views through the neural RANSAC network. Specifically, NOPE-SAC Ref. sets the initial pose to an identity matrix and cuts off initial pose related geometric score function during plane matching, while maintaining the appearance score function intact. Despite retaining a multi-stage approach,  NOPE-SAC Ref. shares consensus with ours in that it relies solely on \textit{monocular plane features} for plane matching and pose regression from \textit{input sparse views}.}


In contrast to NOPE-SAC and other existing baselines, one thing worth highlighting is that our unified plane embedding demonstrates multi-view consistency and implicitly accomplishes plane matching, \textit{without any external supervision for plane matching or the need for initial pose assistance}. In Section \ref{sec:unified_plane_emb}, we compare our method with optimal transport (OT) within NOPE-SAC\cite{tan2023nopesac} requiring external priors.

\subsection{Implementation Detail}


The two-phase training process of sparse views reconstruction, as described in Section \ref{sec:overview}, begins with pre-training the model on the Scannetv2 and Matterport3d datasets following Section \ref{sec:monocular_imp_detail}. Subsequently, we employ nearly identical training configurations to jointly train the entire model with 42 epochs and a 10-fold reduction in loss for monocular planes.

\input{figures/2view_rec}

\subsection{Evaluation of Monocular Planes}
\label{sec:eval_mono}
In this section, we compare the monocular plane predictions to first highlight the effectiveness of our intra-frame plane query learning component after joint optimization. In Table \ref{tab:planerectr++_scannetv2_mp3d_seg} and Table \ref{tab:planerectr++_scannetv2_mp3d_recall}, we initially evaluated the performance for single-view plane segmentation and geometry, similar to Section \ref{sec:experiments}. 

On both datasets, our method demonstrated superior monocular plane prediction accuracy compared to the state-of-the-art PlaneTRP \cite{Tan:etal:ICCV2021:Planetr, tan2023nopesac}. Furthermore, as discussed in Section \ref{sec:corrattn_experiments}, the MatterPort3D dataset \cite{chang2017matterport3d} presents a greater challenge for plane prediction compared to the ScanNetv2 dataset, leading to a moderate performance degradation for all methods.




We also present our monocular plane prediction results exclusively from the monocular pre-training phase (\ie, without the joint optimization stage using pose losses), denoted as Ours (Monocular). There is almost indistinguishable difference in terms of the monocular prediction accuracy (rows 2-3 of Table \ref{tab:planerectr++_scannetv2_mp3d_seg} and Table \ref{tab:planerectr++_scannetv2_mp3d_recall}). This observation signifies the feasibility of our overall pipeline that even though our unified plane embeddings have incorporated evident multi-view consistency for pose estimation (Section \ref{sec:pose_evaluation}), there still remains a comparable ability for monocular plane prediction after the comprehensive joint training phase.




\input{figures/2view_rec_compare}



% In the appendix, similarly, we also use the plane predictions of our PlaneRecTR++ and input them to other baseline methods for experiments, but find no significant improvement in results.

\subsection{Relative Camera Pose Evaluation}
\label{sec:pose_evaluation}



\paraspace
\ptitle{Quantitative Results.} In Table \ref{tab:pose_comparison}, our PlaneRecTR++ demonstrates superior performance in camera pose \textit{across all metrics} on the realistic ScanNetv2 dataset compared to other methods. On the MatterPort3D dataset, our approach achieves overall comparable results to the \jiajia{leading NOPE-SAC}, exhibiting better translation accuracy while slightly lagging behind in rotation estimation.

\jiajia{The relatively poor rotation performance can primarily be attributed to the simulated characteristics of input images on the MatterPort3D dataset, which contains numerous erroneous tiny planes that challenge plane detection of all methods (see Section \ref{sec:eval_mono}) but enhance cross-view photometric consistency during rendering (see dataset introduction in Section \ref{sec:corrattn_experiments}).  
During inference on MatterPort3D, the dense pixel-based NOPE-SAC Init. and other pose initialization networks within baseline methods \cite{jin2021sparseplanes, agarwala2022planeformers, tan2023nopesac} could provide a superior initial rotation, transforming the input sparse views into closer views and thus lowering the difficulty of following plane-level pose predictions.
} 

\jiajia{It is noteworthy that even on this challenging MatterPort3D dataset, our median rotation error remains smaller than that of NOPE-SAC, which indicates smaller typical prediction errors.}



\paraspace
\ptitle{Qualitative Results.} 
Figure \ref{fig:2view_rec} visually illustrates the relative camera pose estimates of our PlaneRecTR++ from two different viewpoints (last two columns) on the ScanNetV2 and MatterPort3D datasets.  Our method can accurately recover a precise relative camera pose from input sparse views, even in scenarios with extremely low image overlap (rows 3-8), without relying on initial pose estimation and explicit corresponding plane pairs.
Figure \ref{fig:2view_rec_compare} presents the predicted pose comparison of ours and NOPE-SAC\cite{tan2023nopesac}, showing that our method recovers a more accurate relative camera pose than the leading baseline.


\subsection{3D Planar Reconstruction Evaluation}


\paraspace
\ptitle{Quantitative Results.}
The numerical evaluation on the final 3D reconstruction is shown in Table \ref{tab:3DPlane}.  Our unified single-stage approach demonstrates superior performance compared to all other multi-stage methods, particularly exhibiting a significant enhancement in the reconstruction accuracy on the real-world dataset ScanNetv2.

\paraspace
\ptitle{Qualitative Results.} 
Figure \ref{fig:2view_rec} presents the visualization of the 3D plane reconstruction results achieved by PlaneRecTR++ on the ScanNetv2 \cite{Dai:etal:CVPR2017, Dai:scannetv2web} and the MatterPort3D \cite{chang2017matterport3d} datasets. Note that our model implicitly learns plane matching during pose inference, and we further extract and process the probability distributions of plane correspondences from our model (see 3rd column). \jiajia{Our method exhibits superior performance in plane matching and reconstruction even in the presence of extremely sparse views, and is capable of identifying and exploiting the discontinuous ground (rows 5,7,8).}  In Figure \ref{fig:2view_rec_compare}, a \jiajia{further} comparison between our proposed PlaneRecTR++ and the current state-of-the-art NOPE-SAC \cite{tan2023nopesac} is provided on both datasets, demonstrating that our method yields more precise plane reconstructions and recovers more accurate relative camera poses from sparse views.
Even in more challenging scenarios characterized by inconsistent brightness (column 1,3), confusion caused by symmetrical repetitive patterns (column 6) and so on, our method can implicitly acquire robust correspondences for reconstruction and pose recovery, which is outperforms NOPE-SAC that relies on initial pose prior and matching supervision.



\begin{table*}[t!]
\centering
\caption{Ablation studies for the different model designs in pose estimation of PlaneRecTR++.}
\vspace{-5pt}
\resizebox{0.9\linewidth}{!}{ 
    \begin{tabular}{ccc|ccccc|ccccc}
    \toprule
        \multicolumn{3}{c|}{Settings} & \multicolumn{5}{c|}{Translation} & \multicolumn{5}{c}{Rotation} \\ 
        CE & QKNum. & VNum. & Med.~$\downarrow$ & Mean~$\downarrow$ & ($\le$1m)~$\uparrow$& 
        ($\le$0.5m)~$\uparrow$& ($\le$0.2m)~$\uparrow$ &
        Med.~$\downarrow$ & Mean~$\downarrow$ & ($\le 30^{\circ}$)~$\uparrow$ & ($\le 15^{\circ}$)~$\uparrow$ & ($\le 10^{\circ}$)~$\uparrow$ \\
        \midrule
        
        \multicolumn{13}{c}{ScanNetv2 dataset} \\
        \midrule
        $\checkmark$& 1 & 4 &\textbf{0.24}  & \textbf{0.46}  & \textbf{88.6\%}  & \textbf{76.3\%}   &   \textbf{43.2\%} &  \textbf{4.30}  & 17.16  &  \textbf{87.6\%}  &   \textbf{84.1\% } &  \textbf{79.7\%}\\
          & 1 & 4 &0.28 & 0.50 & 87.4\% & 72.3\% & 37.2\% & 5.24  &  17.09 &  85.9\%&  80.4\%&  73.0\% \\
        % $\checkmark$&  & 4 &  0.25 & 0.47 & 88.5\% & 75.1\% & 41.2\% & 4.44  & \textbf{16.69} & \textbf{87.8}\%  & 83.9\% & 78.7\% \\
        $\checkmark$& 4 & 4 &  0.25 & 0.47 & 88.3\% & 74.7\% & 40.2\% & 4.45  & \textbf{16.81} & 87.4\%  & 83.7\% & 78.4\% \\
        $\checkmark$& 1 & 1 &  0.27 & 0.49 & 87.9\% & 72.7 \% & 38.2 \% & 5.13  & 17.15  &  86.3\% & 80.9\% & 73.3\% \\
        \midrule

        \multicolumn{13}{c}{MatterPort3D dataset} \\
        \midrule
       $\checkmark$& 1 & 4 &  \textbf{0.39}  & \textbf{0.86}  & \textbf{77.6\%}  & \textbf{58.5\%}   &  \textbf{24.3\%} & \textbf{2.60}  & \textbf{21.19}  &  \textbf{84.6\%} &  \textbf{81.2\%}  & \textbf{78.2\%} \\
        & 1 & 4&  0.49 & 0.98 & 72.4\% & 50.7\% & 18.4\% & 4.39  &  25.72 &  80.0 \%&  74.0\%&  69.2\% \\
        $\checkmark$& 4 & 4  &  0.42 & 0.88 & 76.1\% & 56.5\% & 23.4\% & 3.06  & 21.90 & 83.6\%  & 79.6\% & 76.1\% \\
        $\checkmark$& 1 & 1  &  0.51 & 1.00 & 71.5\% & 49.5\% & 17.5\% & 4.70  & 26.06 & 79.4\% & 72.7\% & 67.1\% \\
         
        \bottomrule
    \end{tabular}
}
\label{tab:pose_model_ablation}
\vspace{-5pt}
\end{table*}


\begin{table*}[t!]
\centering
\caption{Ablation studies for the different model designs in plane correspondences and 3D reconstruction of PlaneRecTR++.}
\vspace{-2mm}
\resizebox{0.8\linewidth}{!}{ 
    \begin{tabular}{ccc|cccc|ccc|ccc}
    \toprule
        \multicolumn{3}{c|}{Settings} & \multicolumn{4}{c|}{Correspondence} & \multicolumn{3}{c|}{Offset$\le$1m,~Normal$\le 30^{\circ}$} & \multicolumn{3}{c}{Offset$\le$0.2m,~Normal$\le 5^{\circ}$} \\ 
        CE & QKNum. & VNum. &Precision & Recall & F-score &TP  & All & -Offset & -Normal  & All & -Offset & -Normal \\
        \midrule
        
        \multicolumn{13}{c}{ScanNetv2 dataset} \\
        \midrule
        $\checkmark$& 1 & 4  & \textbf{0.576} & \textbf{0.552} & \textbf{0.564}  & \textbf{10192} & \textbf{51.08} &\textbf{51.73} & \textbf{55.08} &\textbf{18.19} & \textbf{21.68} &  \textbf{36.86} \\ 
          & 1 & 4  &0.540  &0.518 & 0.529 & 9572  & 48.48 & 49.07 & 53.03 & 15.32 &18.63  &33.50\\ 
        $\checkmark$& 4 & 4  & 0.566 & 0.547 & 0.556 & 10102 &50.43 &51.06 & 54.50 & 17.77 & 21.35 & 36.05\\ 
        $\checkmark$& 1 & 1 & 0.562  & 0.538 & 0.550  & 9938 & 49.96 &50.57 &54.62 & 15.50& 18.66& 34.57  
        \\
        \midrule

        \multicolumn{13}{c}{MatterPort3D dataset} \\
        \midrule
       $\checkmark$& 1 & 4 & \textbf{0.540} & \textbf{0.476} & \textbf{0.506} & \textbf{20630} & \textbf{45.22}  &  \textbf{49.71}  & \textbf{48.23} & \textbf{13.70} & \textbf{26.03}  & \textbf{19.69} \\
          & 1 & 4  & 0.518 & 0.460 & 0.487 & 19915 & 42.26 & 46.85 & 45.86 & 10.60 &21.04 &16.98
  \\
        $\checkmark$& 4 & 4 & 0.530 & 0.469 & 0.498 & 20302 & 44.53 & 48.69 & 47.63 & 13.32 & 24.95 &19.51 \\ %
        $\checkmark$& 1 & 1  &  0.503& 0.447 & 0.473 & 19358 & 43.01 & 47.46 & 46.56 &12.18 &22.20 & 18.45
\\ 
         
        \bottomrule
    \end{tabular}
}
\label{tab:corr_ablation}
\vspace{-2mm}
\end{table*}


\paraspace
\ptitle{Inference Time.} 
We calculate the average inference time of the latest neural methods for joint planar reconstruction and pose estimation on a NVIDIA TITAN V GPU. 
Our single-stage PlaneRecTR++ ($0.258$ s) achieves higher inference speed than previous multi-stage NOPE-SAC ($0.274$ s) and PlaneFormers ($4.257$ s), thanks to the exemption from external pose initialization modules \cite{jin2021sparseplanes, agarwala2022planeformers, tan2023nopesac} and iterative refinement modules \cite{agarwala2022planeformers}.
\subsection{Ablation Studies of Model Designs}
\label{sec:sparseview_ablation}


We conducted extensive ablation studies to investigate the contributions of each design choice, particularly within our plane aware cross attention layer in Section \ref{sec:plane_cross_attention}. We focus on experimenting with the following two aspects: (1) cross embedding structure (CE) within the bilinear attention mechanism, and (2) the specialized design of query, key and value subdivision.





\paraspace
\ptitle{Cross Embeddings.}
Our method differs from Pose ViT \cite{rockwell20228posevit}, which also employs bilinear attention, in that we cross-place plane embeddings of different input images on both sides of the bilinear attention matrix, while Pose ViT places visual features and positional encodings of the same input image, which experimentally yields better results as shown in \cite{rockwell20228posevit}. We consider that our cross plane embeddings placement follows a more intuitive discipline and better performance. To validate our idea, we follow  \cite{rockwell20228posevit} and shift our plane aware cross attention layer's structure to the same embedding placement strategy.  The experimental findings (rows 1, 2 of Table \ref{tab:pose_model_ablation} and \ref{tab:corr_ablation}) show a significant degradation in performance without proposed cross embeddings set-up.

We believe the key reason for the differences between ours and Pose ViT lies in whether the network truly learns plane correspondences. In both methods, it is widely anticipated that the similarity attention matrix would serve as the function for the object assignment matrix. 
However, only our plane aware cross attention design empowers the network to effectively execute authentic plane-level matching, considering that it is less clear to generate plausible patch-wise correspondences of two sparse views.  Consequently, in our method, cross embedding placement naturally yields superior performance compared to Pose ViT, wherein this prerequisite is not met and may even impede efficient passing of visual features through cross-embedding attention.

\begin{table*}[t]
    \centering
    \def\arraystretch{1.0}
    \resizebox{0.9\linewidth}{!}{
    \setlength\tabcolsep{0pt}
    \footnotesize
    \renewcommand{\arraystretch}{0.0}
    \begin{tabular}{ccccc}
         $I_{\text{c}}$ & \methodname & PT & SNP & SNP+  \\
                        
            % % Figure removed &
            % % Figure removed &
            % % Figure removed &
            % % Figure removed &
            % % Figure removed \\
            
            % Figure removed &
            % Figure removed &
            % Figure removed &
            % Figure removed &
            % Figure removed \\
            
            % Figure removed &
            % Figure removed &
            % Figure removed &
            % Figure removed &
            % Figure removed \\
            
            % Figure removed &
            % Figure removed &
            % Figure removed &
            % Figure removed &
            % Figure removed \\
         
            % Figure removed &
            % Figure removed &
            % Figure removed &
            % Figure removed &
            % Figure removed  \\
            
            % Figure removed &
            % Figure removed &
            % Figure removed &
            % Figure removed &
            % Figure removed  \\
            
            % Figure removed &
            % Figure removed &
            % Figure removed &
            % Figure removed &
            % Figure removed  \\
            
            % % Figure removed &
            % % Figure removed &
            % % Figure removed &
            % % Figure removed &
            % % Figure removed  \\
    \end{tabular}
    }
    \vspace{2mm}
    \captionof{figure}{Next predicted stroke probability distribution. For each method, we show the probability that a given pixel will be occupied by a predicted stroke for the given context. \methodname~predicts strokes that tend to focus on the same object as the context stroke and its outline emerges from the probability distribution.}
    \label{fig:heatmap}
    \vspace{-7mm}
\end{table*}

\input{figures/multiview_3dres}

\paraspace
% \ptitle{One Piece Key/Query.}
\ptitle{Query, Key and Value Designs.} We show the effectiveness of our query, key and value designs by evaluating two model variants on pose, plane correspondences and reconstruction, respectively. 

In  Table \ref{tab:pose_model_ablation}, on both datasets, pose accuracy of PlaneRecTR++ with the unsplit query and key design is more precise when maintaining VNum. is 4 (rows 1, 3). In Figure \ref{subfig:qk1v4} and \ref{subfig:qkv4}, the highlighted areas of the plane correspondence attention matrix $\rm{C}(Q_i, K_j)$ using our unsplit key and query, align well with the ground truth correspondence. However, the distribution of 4 similarity attention matrices, each computed using one of the 4 split query and key segments, does not effectively capture the ground truth pattern. Though the combination of 4 similarity matrices can approximate actual plane correspondence distribution, it is still inferior to ours caused by more introduced noisy matches with high probabilities. 
We consider its potential in capturing real correspondences via a post-hoc evaluation, where we select the one head with the highest matching accuracy to ground truth and compare it with our method.  As presented in rows 1, 3 of two datasets in Table \ref{tab:corr_ablation},  even after carefully selecting the best possible matches from the split query and key pairs, the performance is still inferior to ours adopting unsplit query and key pairs.

Moreoever, in Table \ref{tab:pose_model_ablation} and \ref{tab:corr_ablation}, when the key and query are guaranteed to be complete, the accuracy of all metrics with VNum. $=4$ still surpasses that with VNum. $=1$ (rows 1, 4), indicating a positive contribution from the partition of value term.


On the whole, we have validated that our design not only retains the advantages of multi-head attention in standard Transformer, but also effectively captures the distribution of plane correspondence and further enhancing model performance.





% \paraspace
% \ptitle{Model Designs.}


% \paraspace
% \ptitle{Pose Regression}

% \subsection{Unified Plane Embedding Study}
\vspace{-0.1cm}
\subsection{Studies of Unified Plane Embedding}
\label{sec:unified_plane_emb}

During the inter-frame plane query learning stage, we have actually conducted several experiments to enhance the capability of input plane embedding with auxiliary knowledge. Such attempts include concatenating cosine positional encoding \cite{sun2021loftr}, quadratic positional encoding of plane center \cite{rockwell20228posevit}, plane parameter encoding or plane appearance embedding along with original plane embedding. We also explored to filter out plane embedding sequence using their plane probability $p_i$, or to incorporate several self-attention layers \cite{rockwell20228posevit, Vaswani:etal:NIPS2017} to promote contextual features, or to introduce an explicit view consistency loss of planes and pose during training.
%(see Appendix?).
However, none of these variants yielded any obvious improvement in the current model's performance.

It became evident that our unified plane embedding, achieved through a simple combination of intra-frame and inter-frame plane query learning, already encompassed adequate information to address the task of sparse views planar reconstruction.

\input{figures/tsne}
\begin{table}
% \small
% \renewcommand\arraystretch{1.1}
\centering
\caption{Comparison of Plane correspondence between supervised optimal transport (OT) using GT Correspondence, initial pose, and Ours. 
%-P represents a simple screening of correspondence by the final estimated pose and plane geometry.
% Because the final correspondences have been filtered through pose and plane parameters, 
Their raw correspondence without post-filtering is denoted as "-R".
}
\resizebox{1.0\linewidth}{!}{ 
    % \begin{center}
    % \setlength{\tabcolsep}{1.3mm}f
    \begin{tabular}{ccc|cccc}
    \toprule
        % \multirow{2}{*}{Method\_Backbone}  &\multicolumn{6}{c}{(Planar Depth Accuracy)} \\ 
        Method & Corr. Sup. & Init. Pose & Precision &Recall  &F-score &TP  \\ 
       \midrule

         \multicolumn{7}{c}{ScanNetv2 dataset} \\
        \midrule
        OT-R \cite{sarlin2020superglue, tan2023nopesac} &$\checkmark$ & & 0.305 &  0.438  & 0.359 & 8087 \\ %26543     |   18479
        OT-R \cite{sarlin2020superglue, tan2023nopesac} &$\checkmark$ &$\checkmark$ & 0.443  &0.480 & 0.461 &8873  \\ % &   20039   & 18479
        Ours (Monocular)-R &  & & 0.382 & 0.367 & 0.374 & 6786 \\ 
        Ours-R & & & \textbf{0.534} & \textbf{0.562} & \textbf{0.547} & \textbf{10390} \\ %& 19642& 18479 
        % PlaneRecTR++-SPL & & 0.515 & 0.541 & 0.527 & 9988 & 19404 & 18479 \\ 
         \midrule
        
        % OT \cite{} &$\checkmark$ & & 0.541 & 0.409  &0.466 &7552 \\ %13960     |   18479 
        OT \cite{sarlin2020superglue, tan2023nopesac} &$\checkmark$ &$\checkmark$    & 0.473    & 0.467   &  0.470   &8627  \\    %&   18240     &  18479
        Ours & & & \textbf{0.576} & \textbf{0.552} & \textbf{0.564}  & \textbf{10192} \\ %& 17777  & 18479
        % PlaneRecTR++-SPL-P   &   &0.569 & 0.529 & 0.548 & 9771 & 17183 & 18479 \\
        \midrule

         \multicolumn{7}{c}{MatterPort3D dataset} \\
        \midrule
        OT-R \cite{sarlin2020superglue, tan2023nopesac} &$\checkmark$ & &0.371 & 0.500  &0.426   & 21670 \\
        OT-R \cite{sarlin2020superglue, tan2023nopesac} &$\checkmark$ &$\checkmark$ & \textbf{0.499}    &  \textbf{0.515}   &    \textbf{0.507}  &  \textbf{22285} \\ % &     44685    &  43301
        Ours (Monocular)-R & & & 0.316 &  0.283 &  0.298 & 12238 \\ 
        Ours-R & & & 0.456 &  0.509 & 0.481 & 22022 \\ %  & 43014 & 43301
        % PlaneRecTR++-SPL & &0.485 & 0.496 & 0.491 & 21495 & 44286 & 43301 \\ 
         \midrule
        
        % OT \cite{} &$\checkmark$ & & 0.534 &0.467 &  0.498  & 20208 \\
        OT \cite{sarlin2020superglue, tan2023nopesac} &$\checkmark$ &$\checkmark$   & 0.531    &  \textbf{0.501}   &   \textbf{0.515}   & \textbf{21677}  \\     % &    40817     &   43301
        Ours & & &\textbf{0.540} &0.476 &0.506 & 20630 \\  %  & 38212 & 43301
        % PlaneRecTR++-SPL-P   &   & \textbf{0.558} & 0.466 & 0.508 & 20162 & 36103 & 43301 \\
         \bottomrule
        % Ours-R101               & 0.793 & 0.923 & 0.832  \\
    \end{tabular}
}
\vspace{-5pt}
\label{tab:corr_compare}
\end{table}

\paraspace
\ptitle{Consistent Planar Attributes across Frames.}
After the initial single view training and following comprehensive sparse view training, rows 2,3 of Table \ref{tab:planerectr++_scannetv2_mp3d_seg} and Table \ref{tab:planerectr++_scannetv2_mp3d_recall} exhibit comparable performance in monocular plane detection on two datasets.   In the rows 3,4 of Table \ref{tab:corr_compare}, the former relies solely on similar appearance features from a single view and achieves poor matching results, whereas the latter computes a reasonable and precise plane correspondence. In Figure \ref{fig:tsne}, despite only being trained on input sparse views, our unified query embeddings exhibit promising consistency across more frames without the need for ground truth correspondence supervision. \jiajia{Figure \ref{fig:multiview_rec} exhibits our qualitative results that conspicuously outperform the leading NOPE-SAC on multiple views ($\geq 3$ views).}



% \paraspace
% \ptitle{Monocular Visualization}

% poaitinal encoding, context???

\paraspace
\ptitle{Implicit Plane Matching.}
Compared with the differentiable optimal transport (OT) \cite{sarlin2020superglue} method from NOPE-SAC \cite{tan2023nopesac}, our approach owns the following advantages:
(1) Most importantly, we skip the requirement of pose initialization in all previous methods \cite{jin2021sparseplanes, agarwala2022planeformers, tan2023nopesac}. This means that there is no need for us to convert plane parameters into the same coordinate system before effectively utilizing planar geometry for matching.  We believe that this is a crucial factor for constructing a single-stage method.
(2) We do not require explicit supervision using \jiajia{the} ground truth correspondences. Instead, only through pose supervision, our carefully designed simple network structure actively learns multi-view consistency for plane embedding.  In a single forward pass, our method implicitly performs plane matching and probabilistically synthesizes pairwise plane features for pose prediction.  
It does not explicitly perform plane matching and input hard plane pairs to a pose refinement network \cite{jin2021sparseplanes, tan2023nopesac}.
(3) The correspondence attention matrix formed by our network can be processed using a simple MNN (maximum nearest neighbor) operation to obtain a hard plane assignment matrix.  The accuracy of this assignment matrix is comparable or even higher than previous methods with supervisions, as shown in Table \ref{tab:corr_compare}.


% \section{Experimental Results}\label{sec:results}
    \subsection{General Results}
        The basic ResSAN model is used to determine reference results which our expanded model can be compared to as it is structurally similar to ResLAN but does not possess the Lidar adaptive components of it. Further, we compare with the full-size PackNet-SAN and the unmodified NLSPN architecture. 
        As it can be seen from Tab.\,\ref{tab:sota-results}, our LiDAR-adaptive ResLAN achieves competitive performance compared to state-of-the-art standard depth completion methods, which are specialized to the unfiltered 64-beam-LiDAR. The performance differences are in the range of a few centimetres in terms of MAE, which is acceptable given the practical advantage that ResLAN can generalize to different beam patterns as will be shown below.

        Furthermore, we compared the architectures for a set of three different input types that contained 64, 32 or 16 LiDAR channels using both filter types on the metrics from the KITTI benchmark. The NLSPN model was trained for the standard depth completion task and then evaluated with different input data. As for the ResSAN models, we trained one model for each input type and tested it for the corresponding one which serve serve as the \emph{Baseline} in Tab.\,\ref{tab:overall-results}. Our ResLAN model was jointly trained for all three settings. As listed in Tab.\,\ref{tab:overall-results}, the ResLAN models outperform the challenging baseline in all metrics for FOV filtering and all but one for sparse filtering. This implies that our LiDAR adaptive model is able to outperform dedicated models in case of very sparse input depth. Fig.\,\ref{fig:comp-plot} shows this is indeed the case for 32 and even more for 16 channels. For FOV-filtered inputs with 16 channels, the ResLAN exhibits approx. $10\%$ smaller MAE than the baseline. As for the NLSPN, it becomes apparent that it is not capable of generalizing to other input types since it shows clearly worse results. The difference is especially pronounced for the FOV filtering where on average more than every fourth predicted pixel is more than $25 \%$ deviating from the ground truth\,($\delta_{1.25}$). Therefore, using a weight-adapting network in combination with differently filtered input depths allows us to train models that outperform their non-adaptive counterparts.

        \begin{table}[]
            \centering
    	    \small
            \vspace{0.4cm}
            \caption{\textbf{Depth estimation result for standard depth completion} when the ResSAN model was only trained for 64 channels and the ResLAN model for multiple tasks. The PackNet-SAN and NLSPN models were trained with the setup that was also used for our model architecture.}
            \footnotesize
            \setlength{\tabcolsep}{5pt}
            \begin{tabular}{@{}lrrrrl@{}}
            \toprule
            \multicolumn{6}{c}{\textbf{Standard LiDAR Depth Completion}}                                                                                                                         \\ \midrule
            \multicolumn{1}{l|}{Method}          & RMSE $\downarrow$            & MAE  $\downarrow$            & iRMSE $\downarrow$             & iMAE $\downarrow$ & $\delta_{1.25}$ $\uparrow$ \\
            \multicolumn{1}{l|}{}                & \multicolumn{1}{l}{{[}mm{]}} & \multicolumn{1}{l}{{[}mm{]}} & \multicolumn{1}{l}{{[}1/km{]}} & {[}1/km{]}        &                            \\ \midrule
            \multicolumn{1}{l|}{PackNet-SAN}     &  914                            &  298                            &  2.78                              &  1.4                 &  99.65 \%                          \\
            \multicolumn{1}{l|}{NLSPN}           &  \textbf{889}                            &   \textbf{263}                           &  \textbf{2.62}                              &   \textbf{1.3}                &   \textbf{99.61} \%                         \\ \midrule
            \multicolumn{1}{l|}{ResSAN (Ours)}   & 948                             &  275                            &  2.75                              &    1.4               &   99.58 \%                         \\
            \multicolumn{1}{l|}{ResLAN (Ours)} &   969                           &  283                            &   2.83                             &   1.4                &  99.56 \%                          \\ \bottomrule
            \end{tabular}
            \vspace{0.2cm}
            \label{tab:sota-results}
        \end{table}

        \begin{table}[]
    	    \centering
    	    \small
    	    \caption{\textbf{Depth estimation results of the two baseline setups and the explicit and implicit ResSAN} when evaluated on a combination of 16, 32 and 64 channel depth inputs. Please note that Specialist Methods need to train three specialized networks, one for each of the three types of inputs while our method only uses one network.}
            \footnotesize
            \setlength{\tabcolsep}{4.8pt}
            \begin{tabular}{@{}lrrrrl@{}}
                \toprule
                \multicolumn{6}{c}{\textbf{Sparse Channel Filter}}                                                                                                                                  \\ \midrule
                \multicolumn{1}{l|}{Method}        & RMSE $\downarrow$            & MAE  $\downarrow$            & iRMSE $\downarrow$             & iMAE $\downarrow$ & $\delta_{1.25}$ $\uparrow$  \\
                \multicolumn{1}{l|}{}              & \multicolumn{1}{l}{{[}mm{]}} & \multicolumn{1}{l}{{[}mm{]}} & \multicolumn{1}{l}{{[}1/km{]}} & {[}1/km{]}        &                             \\ \midrule
                \multicolumn{1}{l|}{NLSPN}         &  1396                            &  437                            & 5.54                               &  2.2                 &  98.82 \%                           \\
                \multicolumn{1}{l|}{Baseline}      & \textbf{1207}                             &  381                            & 4.41                               &  1.8                 &  \textbf{99.37} \%                           \\
                \multicolumn{1}{l|}{ResLAN (Ours)} &  1215                            &  \textbf{378}                            &  \textbf{4.27}                              &  \textbf{1.7}                 &  99.31 \%                           \\ \toprule
                \multicolumn{6}{c}{\textbf{Field-of-View Filter}}                                                                                                                                   \\ \midrule
                \multicolumn{1}{l|}{Method}        & RMSE $\downarrow$            & MAE  $\downarrow$            & iRMSE $\downarrow$             & iMAE $\downarrow$ & $\delta_{1.25}$ $\uparrow$ \\
                \multicolumn{1}{l|}{}              & \multicolumn{1}{l}{{[}mm{]}} & \multicolumn{1}{l}{{[}mm{]}} & \multicolumn{1}{l}{{[}1/km{]}} & {[}1/km{]}        &                             \\ \midrule
                \multicolumn{1}{l|}{NLSPN}         &  2738                            &  1702                            & 12.3                              &  4.3                 &  74.69 \%                           \\
                \multicolumn{1}{l|}{Baseline}      &  1556                            &  525                            &  6.8                              &  3.0                 & 98.14 \%                            \\
                \multicolumn{1}{l|}{ResLAN (Ours)} &  \textbf{1548}                            &  \textbf{519}                            &  \textbf{6.44}                              &  \textbf{2.8}                 & \textbf{98.52 \%}                            \\ \bottomrule
            \end{tabular}
            \label{tab:overall-results}
        \end{table}

        
        
        % Figure environment removed
        
        % Figure environment removed

    \subsection{Filter Effects}
        Comparing the effect of the two different types of depth input filters on the model performance, it becomes apparent that FOV filtering is the more challenging task. In that setting, reducing LiDAR channels is more detrimental to the performance than sparse filtering as it creates regions where no depth information is available. Effectively, the model is forced to perform depth prediction in these regions. These effects are highlighted in the depth images in Fig.\,\ref{fig:dense-maps} where the effect of a 16-channel sparse depth filter and a 16-channel FOV can be compared.

    \subsection{Generalization Capabilities}
        We trained three models for both filter types eaach, so the combinations and number of filtered depth inputs they receive are different. This serves the purpose of testing the generalization capabilities of the ResLAN architecture as well as the robustness to different filter settings. After training, the models were evaluated for the depth input settings they were trained for, as well as for ones they weren't exposed to. Overall, ResLAN shows good generalization capabilities. As one can gather from Fig.\,\ref{fig:explicit-comp} and Fig.\,\ref{fig:implicit-comp}, the consequences of slightly varying sets of input depth settings are limited. The most considerable deviations can be seen when the model is tasked to extrapolate. For instance, the model $\{64, 32, 16\}$ shows a noticeably higher MAE for eight-channel depth inputs than the model that was trained for it. Similar behaviour can be seen for the FOV filtering case as well for the model $\{64, 48, 32\}$ when tasked to generalize for a 16-channel input. There is no such pronounced effect for generalization tasks that lie between two filter settings the model was trained for. At most, it can be observed that models that were trained for a smaller range of filter values perform slightly better than ones that have to cover a wider range. The number of filter settings used in a fixed range does not relevantly influence the model performance, as can be seen, when comparing the two models in Fig.\,\ref{fig:implicit-comp}, which are both trained for a range of 64 to 32 channels but one with three filter settings and the other one with five.
    
    % Figure environment removed
    
    
    % Figure environment removed
\section{Conclusion and Future Work}
In this work, I design corruption-robust algorithms for the Lipschitz contextual search problem. I present the \emph{agnostic checking} technique and demonstrate its effectiveness in designing corruption-robust algorithms. There are several open problems for future research. First, in the algorithm I propose for pricing loss, the schedule for agnostic checks is fixed upfront. Can the learner design an adaptive checking schedule for the pricing loss? Second, this work assumes the learner has knowledge of the Lipschitz constant $L$. Can the learner design efficient no-regret algorithms without knowledge of $L$? 
\section*{Acknowledgment}
\label{sec:ack}
%------------------------------------------------------------------------------
We would like to thank the anonymous reviewers.
This work has been partially supported by NSF projects CCF-2217070 and CNS-1909769, the Applications Driving Architectures (ADA) Research
Center, a JUMP Center co-sponsored by SRC and DARPA, and by funding and equipment gifts from VMware and Intel.


% Can use something like this to put references on a page
% by themselves when using endfloat and the captionsoff option.
\ifCLASSOPTIONcaptionsoff
  \newpage
\fi


% trigger a \newpage just before the given reference
% number - used to balance the columns on the last page
% adjust value as needed - may need to be readjusted if
% the document is modified later
%\IEEEtriggeratref{8}
% The "triggered" command can be changed if desired:
%\IEEEtriggercmd{\enlargethispage{-5in}}

% references section

% can use a bibliography generated by BibTeX as a .bbl file
% BibTeX documentation can be easily obtained at:
% http://mirror.ctan.org/biblio/bibtex/contrib/doc/
% The IEEEtran BibTeX style support page is at:
% http://www.michaelshell.org/tex/ieeetran/bibtex/
%\bibliographystyle{IEEEtran}
% argument is your BibTeX string definitions and bibliography database(s)
%\bibliography{IEEEabrv,../bib/paper}
%
% <OR> manually copy in the resultant .bbl file
% set second argument of \begin to the number of references
% (used to reserve space for the reference number labels box)

\bibliographystyle{IEEEtran}
% \bibliography{geotrans}
\bibliography{robotvision}





\begin{IEEEbiography} [{% Figure removed}]{Yizhou Lu}
\Xhide{Biography text here.}
obtained his B.S. degree in Geochemistry from Nanjing University, China, in 2017, followed by M.S. degrees in Materials Science and Engineering and Computer Sciences from the University of Wisconsin-Madison, WI, USA, in 2019 and 2023, respectively. Currently pursuing a Ph.D. degree with the Electrical and Computer Engineering Department at the University of Wisconsin-Madison, his research interests encompass computational imaging and information theory.
\end{IEEEbiography}




\begin{IEEEbiography} [{% Figure removed}]{Trevor Seets}
\Xhide{Biography text here.}
received his B.S. and M.S. degree in electrical engineering from the University of Wisconsin-Madison in 2019 and 2023, respectively. He is now a Ph.D. student under Professor Andreas Velten at UW-Madison. His research interests include computational optics, statistical signal processing, and imaging.
\end{IEEEbiography}

\begin{IEEEbiography} [{% Figure removed}]{Felipe Gutierrez Barragan}
\Xhide{Biography text here.}
is a Senior Software Engineer at Ubicept. He received his B.S. (2016), M.S. (2019), and Ph.D. (2022) in Computer Sciences from the University of Wisconsin-Madison.  His Ph.D. research focused on practical modifications to SPAD-based and indirect time-of-flight 3D cameras to reduce power consumption and data bandwidth while preserving or increasing their precision. His research interests include computational imaging, computer vision, and machine learning.
\end{IEEEbiography}

\begin{IEEEbiographynophoto}{Ehsan Ahmadi} earned his B.S. in Computer Software Engineering from Iran University of Science and Technology, Tehran, Iran, and his M.S. in Electrical and Electronics Engineering from Southern Illinois University, Carbondale, Illinois. Currently, he is pursuing a Ph.D. in Computer Engineering at UW-Madison, focusing on Fluorescence Lifetime Imaging Microscopy (FLIM).



    
\end{IEEEbiographynophoto}



\begin{IEEEbiography} [{% Figure removed}]{Andreas Velten}
\Xhide{Biography text here.}
received the B.A. degree in physics from the Julius Maximilian University of Wurzburg, Wurzburg, Germany, in 2003, and the Ph.D. degree in physics from the University of New Mexico, Albuquerque, NM, in 2009. From 2010 to 2012, he was a postdoctoral research associate with the Massachusetts Institute of Technology, Cambridge, MA, USA, where he was working on high speed imaging systems that can look around
a corner using scattered light. From 2013 to 2016, he was an associate scientist with the Laboratory for Optical and Computational Instrumentation, University of Wisconsin–-Madison, Madison WI, USA, working in optics, computational imaging, and medical devices. Since 2016, he has been an associate professor with the Biostatistics and Medical Informatics, Electrical and Computer Engineering Department, University of Wisconsin–-Madison. His research focuses on performing multidisciplinary work in applied computational optics and imaging.
\end{IEEEbiography}



% that's all folks
\end{document}
