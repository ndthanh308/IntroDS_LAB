% \usepackage{amsmath}
\usepackage{amsfonts}
\usepackage{amssymb}
\usepackage[switch]{lineno}

%% SET PAPER TITLE HERE
% \newcommand{\papertitle}{Single-Pixel Computer Vision in Low-Light}
% \newcommand{\papertitle}{Single-Pixel Computer Vision under Poisson Noise}
% \newcommand{\papertitle}{Single-Pixel Inference under Poisson Noise}
% \newcommand{\papertitle}{When does optical coding improve performance of single-photon cameras?}

% \newcommand{\papertitle}{Coded Imaging under Poisson Noise}
\newcommand{\papertitle}{Sparsity-Aware Coding for Single Photon Sensitive Vision Using Selective Sensing}
%\renewcommand{\papertitle}{Sparsity is All You Need: Improved Coding with Selective Sensing}

%% Additional packages included, and also any commands defined
\usepackage{enumitem,kantlipsum}

\usepackage{siunitx}

% Fun colors to use: https://en.wikibooks.org/wiki/LaTeX/Colors
\usepackage[dvipsnames]{xcolor} 
% \hypersetup{colorlinks,linkcolor={blue},citecolor={RoyalBlue},urlcolor={red}} 
\newcommand{\rnote}[1]{\textcolor{red}{#1}}
\newcommand{\bnote}[1]{\textcolor{blue}{#1}}
%\newcommand{\rnote}[1]{{\begin{color}{red}#1\end{color}}}
%\newcommand{\bnote}[1]{{\begin{color}{blue}#1\end{color}}}

\newcommand{\norm}[1]{\left\lVert#1\right\rVert}

\newenvironment{myenum}
{ \begin{enumerate}
    \setlength{\itemsep}{0pt}
    \setlength{\parskip}{0pt}
    \setlength{\parsep}{0pt}     }
{ \end{enumerate}                } 

\newenvironment{myitem}
{ \begin{itemize}
    \setlength{\itemsep}{0pt}
    \setlength{\parskip}{0pt}
    \setlength{\parsep}{0pt}     }
{ \end{itemize}                } 


\def\E{\mathsf{E}}
\def\Var{\mathsf{Var}}
% \DeclareMathOperator*{\argmax}{arg\,max}
% \DeclareMathOperator*{\argmin}{arg\,min}
\newcommand*{\defeq}{\mathrel{\vcenter{\baselineskip0.5ex \lineskiplimit0pt
                     \hbox{\scriptsize.}\hbox{\scriptsize.}}}%
                     =}

\newcommand{\taud}{\tau_\text{d}}

\newcommand*{\nolink}[1]{%
  {\protect\NoHyper#1\protect\endNoHyper}%
}

% customized styles

% \input{\homedir/Main/custom_sty_edit}
% re-define the equation environment
\let\oldequation\equation%
\let\endoldequation\endequation%
% \renewenvironment{equation}%
%   {\linenomath\oldequation}{\endoldequation\endlinenomath}%
\expandafter\let\expandafter\oldequationstar\csname equation*\endcsname%
\expandafter\let\expandafter\endoldequationstar\csname endequation*\endcsname%
\renewenvironment{equation*}%
  {\linenomath\oldequationstar}{\endoldequationstar\endlinenomath}%
  

\newcommand{\B}{\boldsymbol}
\newcommand{\IR}{\mathbb{R}}
\newcommand{\IZ}{\mathbb{Z}}

\newcommand{\diag}{\text{diag}}

\usepackage{subcaption}

\newcommand{\totphoton}{\mathfrak{N}}
\newcommand{\photon}{\mathfrak{n}}
% \def\totphoton{\left(\rho\tau\right)}

\newcommand{\totexposure}{\tau} %total expo time
\newcommand{\exposure}{\mathfrak{t}} %expo time for each mask
\newcommand{\flux}{\rho} %photons per second



\newcommand{\Xalign}[1]{\begin{aligned}#1\end{aligned}}

\newcommand{\Xequa}[1]{\begin{equation}\Xalign{#1}\end{equation}}




% \def\Xhide#1{\iffalse#1\fi}
\def\Xhide#1{}

\def\noisy#1{\tilde{#1}}
\def\AGN{\mathcal{N}}
\def\Poisson{\mathcal{P}}
\def\Poisson{\mathrm{Poisson}}

% \usepackage{svg}
\usepackage{graphicx} % include PDF figures
\def\Xpolish#1#2{#2}


\def\bnote#1{\textcolor{blue}{#1}}
\def\rnote#1{\textcolor{red} {#1}}
\def\ynote#1{\textcolor{orange}{#1}}
\def\enote#1{\textcolor{violet}{#1}}
\def\YLnote{\rnote}
\def\FGnote#1{\textcolor{brown}{#1}}

\def\ynote#1{\color{orange} #1 \color{black}}
\def\AVnote{\ynote}
\def\TSnote{\bnote}

\def\Bingnote#1{\color{purple} \textit{{#1}} \color{black}}
% \usepackage{pifont}
% \def\checkednote#1{\ding{51} #1}
% \usepackage{soul}
% \usepackage[normalem]{ulem}
% \usepackage{ulem, lipsum}
% \newcommand\xoutpars[1]{\let\helpcmd\xout\parhelp#1\par\relax\relax}
% \newcommand\soutpars[1]{\let\helpcmd\sout\parhelp#1\par\relax\relax}
% \long\def\parhelp#1\par#2\relax{%
%   \helpcmd{#1}\ifx\relax#2\else\par\parhelp#2\relax\fi%
% }

\usepackage[normalem]{ulem}
\def\checkednote#1{\sout{#1}}

%%%%%%network plot
\usepackage{tikz}
\usetikzlibrary{shapes.geometric}
\usetikzlibrary {shapes.misc}
\usepackage{caption}
\usepackage{amsmath,amsfonts}

%%
% \usepackage{slashbox}

%%%% before submit: uncomment the following, OR changing the value of \peerreview in output.tex 
\ifpeerreview
    % \linenumbers\linenumbersep 15pt\relax    
    \def\checkednote#1{} % remove comments
    \def\YLnote{}
\else
    % \linenumbers \linenumbersep 15pt\relax  
\fi
%%%%%%%%%%%%%%%%%%%%%%%%%%%%%%%%%%%%%%%%%%

\def\YLnote{}
\def\checkednote#1{}

\pagenumbering{Roman}

\usepackage{printlen}
\usepackage{layouts}
% \usepackage{unicode-math}




\iffalse
% this is the older version
\noindent \textbf{\bnote{Notes for Paper Contributors:}}

\begin{itemize}
    \item To define new commands or include additional packages, add to \textit{include.tex}. If we have to change latex template this minimizes effort.
    \item To add new sections, add them under \textit{sections} and then edit \textit{main.tex}.
    \item For examples of how to add Figures, Tables, see \textit{examplelatex\_iccp2020\_template}.
    \item One method to create bibtex entries is by going to google scholar, searching paper title, clicking on the cite icon, clicking on Bibtex, and copying and pasting the google scholar bibtex to \textit{references.bib}
\end{itemize}

\rnote{
% 1. Make noiseless classification result 
2. Select data for ONN experiment 
3. Replace Random Hadamard by Compressed Sensing ?
3. Split decoder and encoder in ONN
4. Show regularization improves all masks (equally), then I can use random truncated 5. Supplementary includes how the experiment was done 6. The name is 
\textbf{Selective Sensing} 7. Use Seaborn theme 
}


\medskip
\medskip
\medskip
\medskip
\medskip
\medskip

%%%%%%%%% BODY TEXT

% The first section title should be wrapped inside a \IEEEraisesectionheading as follows.


% \IEEEraisesectionheading{
%   \section{Introduction}\label{sec:1_intro}
% }


% \printlength\linewidth
\printinunitsof{in}\prntlen{\linewidth}

\noindent\rule{\linewidth}{0.4pt}
\fi

    \YLnote{\checkednote{1. Replace Low frequency Hadamard with random frequency Hadamard 2. Fulfill the paragraphs left}}

\section{Introduction}\label{sec:1_intro}
\section{Introduction}

% Figure environment removed

Reinforcement Learning from Human Feedback (RLHF) has recently been used to great effect to align pretrained large language models (LLMs) to human preferences, optimizing for desirable qualities like harmlessness and helpfulness~\citep{bai2022training} and achieving state-of-the-art results across a variety of natural language tasks~\citep{openai2023gpt4}. %RLHF approaches fundamentally rely on collecting pairs of LLM outputs $(o_1, o_2)$ from a shared prompt $p$, with a human indicating which output in each pair is better on a specified attribute.
% A fundamental component of RLHF is a preference model derived from human labels, typically formatted as pairs of LLM outputs $(o_1, o_2)$ generated from a shared prompt $p$.

A standard RLHF procedure fine-tunes an initial unaligned LLM using an RL algorithm such as PPO~\citep{schulman2017proximal}, optimizing the LLM to align with human preferences. %\violet{not sure whether we need to provide this detail in the intro, especially this has nothing to do with our contribution.} % i feel like this context is useful later when e.g. explaining that context distillation is SFT
RLHF is thus critically dependent on a reward model derived from human-labeled preferences, typically \textit{pairwise preferences} on LLM outputs $(o_1, o_2)$ generated from a shared prompt $p$. % and labeled by humans. 

However, collecting human pairwise preference data, especially high-quality data, may be expensive and time consuming at scale. To address this problem, approaches have been proposed to obtain labels without human annotation, such as Reinforcement Learning from AI Feedback (RLAIF) and context distillation. 

\iffalse
raising the question of whether we can generate high-quality data for RLHF without using human labeling. %accurately-labeled preference pairs $(o_1, o_2)$
%, motivating model alignment approaches that aim to generate accurately-labeled preference pairs $(o_1, o_2)$ without human involvement. 
Two major categories of such approaches are . 
\fi

RLAIF approaches (e.g.,~\citet{bai2022constitutional}) simulate human pairwise preferences by scoring $o_1$ and $o_2$ with an LLM (Figure \ref{fig:rlcd_differences} center); the scoring LLM is often the same as the one used to generate the original pairs $(o_1, o_2)$. Of course, the resulting LLM pairwise preferences will be somewhat noisier compared to human labels. However, this problem is exacerbated by using the same prompt $p$ to generate both $o_1$ and $o_2$, causing $o_1$ and $o_2$ to often be of very similar quality and thus hard to differentiate (e.g., Table~\ref{tab:rlaif_bad_example}). Consequently, training signal can be overwhelmed by label noise, yielding lower-quality preference data. 

% While it avoids human labeling efforts, it has weakness. First, LLM preference labels will naturally be somewhat noisier compared to human labels. Furthermore, since the same prompt $p$ is used to generate both $o_1$ and $o_2$, their quality is often very similar and hard to differentiate (See Table~\ref{tab:rlaif_bad_example}). As a result, training signals can be overwhelmed by label noise, yielding lower-quality preference data. 

Meanwhile, context distillation methods (e.g., \citet{sun2023principle}) create more training signal by modifying the initial prompt $p$. 
%to create more significant training signal. 
The modified prompt $p_+$ typically contains additional context encouraging a \textit{directional attribute change} in the output $o_+$ (Figure \ref{fig:rlcd_differences} right). However, context distillation methods only generate a single output $o_+$ per prompt $p_+$, which is then used for supervised fine-tuning, losing the pairwise preferences which help RLHF-style approaches to 
%rather than using a RLHF-style preference model to 
derive signal from the contrast between outputs. 
Multiple works have observed that RL approaches using preference models for pairwise preferences can substantially improve over supervised fine-tuning by itself when aligning LLMs~\citep{ouyang2022training,dubois2023alpacafarm}. 

% conduct alignment by running supervised fine-tuning on model outputs $o_+$ generated from a modified prompt $p_+$. $p_+$ typically contains additional context encouraging desirable attributes (Figure \ref{fig:rlcd_differences} right), such as in \citet{sun2023principle}. However, multiple works have observed that RLHF-style approaches can substantially improve over supervised fine-tuning by itself when aligning LLMs~\citep{ouyang2022training,dubois2023alpacafarm}. 

Therefore, while both RLAIF and context distillation approaches have already been successfully applied in practice to align language models, we posit that it may be even more effective to combine the key advantages of both. That is, we will use RL with \textit{pairwise preferences}, while also using modified prompts to encourage \textit{directional attribute change} in outputs. %In particular, we will adapt the RLAIF data generation process with two different prompts rather than a single $p$, modifying both prompts similarly to context distillation. %\violet{this motivation is a little unexciting. I think we can more specifically discuss the potential benefits of our approach, like the benefits from RL: exploration/data generation; benefits from contrast. I don't think we get too much benefits from context distillation since we switched to the RL framework.} 

Concretely, we propose \oursfull{} (\ours{}). 
\ours{} generates preference data as follows. Rather than producing two i.i.d.\ model outputs $(o_1, o_2)$ from the same prompt $p$ as in RLAIF, \ours{} creates two variations of $p$: a \textit{positive prompt} $p_+$ similar to context distillation which encourages directional change toward a desired attribute, and a \textit{negative prompt} $p_-$ which encourages directional change \textit{against} it (Figure \ref{fig:rlcd_differences} left). We then generate model outputs $(o_+, o_-)$ respectively, and automatically label $o_+$ as preferred---that is, \ours{} automatically ``generates'' pairwise preference labels by construction. %, without further post hoc labeling.\violet{should make it clearer that our approach `generates' labels by construction} 
We then follow the standard RL pipeline of training a preference model followed by PPO. 

Compared to RLAIF-generated preference pairs $(o_1, o_2)$ from the same input prompt $p$, there is typically a clearer difference in the quality of $o_+$ and $o_-$ generated using \ours{}'s directional prompts $p_+$ and $p_-$, which may result in less label noise. %which may result in better training signal for the preference model. 
That is, intuitively, \ours{} exchanges having examples be \textit{closer to the classification boundary} for much more \textit{accurate labels} on average. Compared to standard context distillation methods, on top of leveraging pairwise preferences for RL training, \ours{} can derive signal not only from the positive prompt $p_+$ which improves output quality, but also from the negative prompt $p_-$ which degrades it. %\ours{} is not learning to imitate $o_+$, but to distill the \textit{contrast} between $o_+$ and $o_-$. 
Positive outputs $o_+$ don't need to be perfect; they only need to contrast with $o_-$ on the desired attribute while otherwise following a similar style.

% \todo{discuss our method and why intuitively it may be better.}

We evaluate the practical effectiveness of \ours{} through both human and automatic evaluations on three tasks, aiming to improve the ability of LLaMA-7B~\citep{touvron2023llama} to generate harmless outputs, helpful outputs, and high-quality story outlines. %\ours{} outperforms both RLAIF and context distillation baselines in pairwise comparisons on 
As shown in Sec. \ref{sec:experiments}, \ours{} substantially outperforms both RLAIF and context distillation baselines in pairwise comparisons when simulating preference data with LLaMA-7B, while still performing equal or better when simulating with LLaMA-30B. 
%On all three tasks, \ours{} substantially outperforms both RLAIF and context distillation baselines in pairwise comparisons---by a margin of at least 9\% and often more than 30\%---validating our method's efficacy. 
We will release all code at a later date, although in any case \ours{} is fairly easy to implement by modifying any reference RLAIF codebase. %We release all code at \todo{github link}.


\section{Related Work}
\label{sec:2_related}
\textbf{Coding under Poisson noise}. In single-pixel imaging, coding allows the capture of a two-dimensional image with a single-pixel sensor \cite{harwit1979hadamard}. Hadamard matrices are considered the optimal coding scheme for multiplexing \cite{harwit1979hadamard, cossairt2012does, wuttig2005optimal} in systems with only additive gaussian noise. \Xhide{However, when Poisson noise is predominant, we should not use any coding \cite{harwit1979hadamard, cossairt2012does}. When the noise is Poisson distributed, coding with sparse priors such as compresses sensing are not recommended \cite{willet2009CSPoisson, willett2011poisson, vanden2019various}.}
{In many recent projects, optimization efforts focus on algorithmic enhancements for reconstruction, often maintaining the use of random coding strategies \cite{goyal2016performance}  \YLnote{need more here}, though this type of matrices are not recommended under Poisson noise\cite{willet2009CSPoisson,  willett2011poisson, vanden2019various}. While some approaches, such as Feature Specific Imaging (FSI), optimize sensing matrices based on learnable priors \cite{neifeld2003dual, neifeld2003FSI, neifeld2014optimizing}, these methods explicitly do not consider Poisson noise. Thus the challenge of optimizing masks under Poisson noise persists \cite{neifeld2014optimizing}. Additionally, attempts at matrix optimization focusing on minimal mutual coherence lack a foundation in the Poisson noise model, as highlighted in this study \cite{mordechay2014matrixRIP}.}



% \textbf{Task Specific Imaging}

\textbf{End-to-end optimization}. This method refers to training hardware and software networks for image processing pipelines \cite{diamond2021dirty, zhang2021deep}. \YLnote{In many previous projects, this idea was usually implemented without considering Poisson noise \cite{hinojosa2021learning, dun2020learned, metzler2020deep, chang2019deep, onzon2021neural, spall22hybrid_training} or without optimizing masks under Poisson noise during training \cite{tseng2021differentiable, diamond2021dirty, rego2022deep, duarte2008CS, nature2022ONN}. Rego et. al froze the sensing matrix as a pinhole without optimizing it \cite{rego2022deep}}
Wang et al. \cite{nature2022ONN} successfully implemented a neural network model for handwritten number classification on an optical device with limited photon budget, demonstrating the potential for AI-assisted optimization of coding schemes in CI. 
\YLnote{However, the Poisson noise was considered only in model testing where the most robust model was picked from a set of hyper-parameter combinations \cite{nature2022ONN}. } Our contribution is developing a noise-included training approach within a neural network model to find the optimal masks under Poisson noise.

% \textcolor{blue}{Add section on feature specific imaging}

\textbf{Feature Specific Imaging}. Feature-specific imaging is a type of imaging system that directly measures linear features of the object irradiance distribution, instead of forming a conventional image and then extracting features from it \cite{neifeld2003FSI, neifeld2003dual}. This approach can provide higher feature fidelity and lower detector count than conventional imaging, especially for applications that require relatively few features \cite{neifeld2003FSI, neifeld2003dual, neifeld2014optimizing}. This technique can be viewed as a variant of compressed sensing, wherein the sensing matrix is determined based on prior information \cite{neifeld2014optimizing}. Nevertheless, its performance and implications under Poisson noise conditions remain relatively unexplored in the existing literature, warranting further discussion and investigation.


\iffalse

\YLnote{For the feature specific imaging, here are some paper to cite}

\YLnote{yizhou notes start}
\textbf{Mark A. Neifeld and Premchandra Shankar, "Feature-specific imaging," Appl. Opt. 42, 3379-3389 (2003)} Feature-specific imaging is a type of imaging system that directly measures linear features of the object irradiance distribution, instead of forming a conventional image and then extracting features from it. This approach can provide higher feature fidelity and lower detector count than conventional imaging, especially for applications that require relatively few features.  Poisson noise when the mean and variance of the photon count are equal. The paper shows that feature-specific imaging does not provide any advantage over conventional imaging in the presence of shot noise, because the shot-noise-limited SNR is inherent in the irradiance collected by the instrument and does not depend on the energy incident on any particular detector. Therefore, we can infer that feature-specific imaging would not work well with Poisson noise either, unless the mean and variance of the photon count are very different. \YLnote{However, that is where we come up with hardware PCA. We can then further reduce the text by citing Neifeld for the PCA part. }

\textbf{Abhijit Mahalanobis and Mark Neifeld, "Optimizing measurements for feature-specific compressive sensing," Appl. Opt. 53, 6108-6118 (2014)} "Although our analysis assumes that the noise is independent of the signal, this is not the case when Poisson noise is present. While a formal treatment of the effects of signal-dependent noise is beyond the scope of the paper, we compare the behavior of optimized masks to the normalized PCA in signal-dependent noise." It also has imcomplete future work "Other types of noise models will also be considered, such as signal-dependent shot noise." It seems it hasn't been studied after this paper. We could say that our contribution is designing a method to extend feature specific imaging in poisson noise model.

\YLnote{Interesting conclusions listed below}  

"FSI is a form of CS in which the measurement kernels are not random, but are based on prior knowledge of the information we are interested in sensing. Working in this FSI framework, we have developed a methodology for designing a set of masks that satisfy the photon constraint and are optimum for making measurements that minimize the reconstruction MSE in the presence of noise. "

"Of course, the primary reason for CS is to make fewer measurements than the number of pixels in the reconstructed image, and to collect the information more efficiently than a conventional image. One might intuitively expect that utilizing more features (or measurements) in the reconstruction process will yield a smaller reconstruction error. We demonstrated, however, that the photon constraint limits the number of masks that can be used at a particular SNR to reduce the reconstruction MSE. In noisy conditions, the MSE initially decreases as the number of measurements is increased, but then increases when measurements that contain more noise than signal information are included." \YLnote{It aligns with Rebecca's work.}

\textbf{W. Van den Broek, B. W. Reed, A. Béché, A. Velazco, J. Verbeeck and C. T. Koch, "Various Compressed Sensing Setups Evaluated Against Shannon Sampling Under Constraint of Constant Illumination," in IEEE Transactions on Computational Imaging, vol. 5, no. 3, pp. 502-514, Sept. 2019, doi: 10.1109/TCI.2019.2894950.}  show that for the investigated sensing matrices and in the absence of read-out noise, i.e. with only Poisson noise present, compressed sensing does not raise the amount of Fisher information in the recordings above that of a Shannon sampled signal.





\YLnote{yizhou notes end}

\ynote{andreas notes start}
Other refrerences:
M. Mordechay and Y. Y. Schechner, "Matrix optimization for poisson compressed sensing," 2014 IEEE Global Conference on Signal and Information Processing (GlobalSIP), Atlanta, GA, USA, 2014, pp. 684-688, doi: 10.1109/GlobalSIP.2014.7032205.
“Considering that the measurements are noisy, it is important to
avoid energy loss, since measurements with low energy have low
signal-to-noise ratio (SNR).”
We prove this assumption wrong in this paper ... \YLnote{ONN reconstruction}

Photon-noise: is a single-pixel camera better than point scanning? A signal-to-noise ratio analysis for Hadamard and Cosine positive modulation, Camille Scotté, Frédéric Galland1 and Hervé Rigneault
Published 7 June 2023 • © 2023 The Author(s). Published by IOP Publishing Ltd
Journal of Physics: Photonics, Volume 5, Number 3 Citation Camille Scotté et al 2023 J. Phys. Photonics 5 035003 DOI 10.1088/2515-7647/acc70b
Looking at SNR for different codes. Mostly confirms prior results. Finds that snr is improved for brightest pixels, but degraded for others. \YLnote{Only $x_i \ge k \bar{x} = \frac{k\sum_{j=1}^N x_j}{N}$ can get improved. $k$ is usually $2, 4, 16$, which means only bright pixels get reconstruction gain. E.g. cosmophotography. For normal images, that usually means the general performance degrades!}

D. Shin, J. H. Shapiro and V. K. Goyal, "Performance Analysis of Low-Flux Least-Squares Single-Pixel Imaging," in IEEE Signal Processing Letters, vol. 23, no. 12, pp. 1756-1760, Dec. 2016, doi: 10.1109/LSP.2016.2617329.
Finds bounds for noise. Probably agrees with paper above.
This is very interesting. AT very low rates the non negativity constraint can improve our estimate for multiplexing so it may become better than poisson. The paper doesn’t prove that it is better. Just says the proof for the general case is not valid and shows an experiment where an improvement is shown. Could we use the same constraint in convex optimization? What does this say about other constrained or regularized optimizations?
\ynote{andreas notes end}


\iffalse
better arrangement
coding under Poisson: mention hadamard book, ollie and becca's paper show that coding under Poiss is not ideal
mask optimization: people optimize the masks under wrong noise model, we proposed a better way for Poiss 

\fi


\YLnote{In summary, we can have a list of literature by their category
\begin{enumerate}
    \item Open more pixels doesn't help under Poisson noise (no compression): \cite{harwit1979hadamard, neifeld2003FSI, cossairt2012does, scotte2022photon_noise}
    \item Regularization doesn't help (usually comes with compression):
    \cite{willet2009CSPoisson, willett2011poisson, vanden2019various}. Their suggestion is to use fewer measurements as more measurements give rise to a greater reconstruction error upper bound. BTW, ONN are can have many measurements with low rank, which doesn't contradict to this conclusion.
    \item Feature specific imaging is designed for gaussian noise especially. It is a good way for feature preserving and reconstruction. No follow-up about the poisson. That is why we start this project.
    \item When doing a classification task, we give some higher compression by labelling. E.g, gender recognition. We only need one mask to tell the gender in the most ideal case. \textbf{It will be good to find some paper directly saying this.} As the data can be highly compressible, we can avoid dense measurements, and thus have lower error bound. Reconstruction is just a linear operation, so it doesn't matter a lot if we use Rebecca's conclusion about the upper bound, even though it is for reconstruction. 
        \begin{itemize}
        \item Think about reconstructing an image about the night sky. After using TV \textbf{re-binning}, we have two colors, black for background, and white for stars. It can be treated as a classification task as well! That is why rebecca proposed some algorithms with TV stuff: to minimize the possible amount of measurements.
    \end{itemize}
    Neifeld uses features learned from training data to do reocnstruction, that is how our idea of directly classification comes from. If we want to apply the ONN for reconstruction, it may be useful to read "RecDNN: deep neural network for image reconstruction from limited view projection data"
\end{enumerate}
}
\fi


\AVnote{\checkednote{Regularization, CS, might help but noise bounds are very unfavorable (cite becca). \\ In most of the papers, people optimize the algorithm. In only very few people try to optimize the masks. No paper where masks are optimize with poisson noise. \\ In this paper we are looking at how to optimize the measurement under poisson noise. \\ We find that optimizing the measurement can have profound improvements to the performance of single pixel vision approaching the perfonance of ideal mutipixel cameras in particular for Poisson noise dominated signals like visible light. }}




% \section{Single-Pixel Image Formation Model}
\section{Background}
\label{sec:3_background}
% !TEX root = ../main.tex

% \section{Background}

% \subsection{Badminton Match Analysis and Coaching}
% match analysis
% Professional badminton players compete at the highest level, and the outcomes of their matches are often decided by the narrowest of margins. 
% As a result, players and coaches have come to rely heavily on match analysis as a means of identifying key patterns and strategies that can be used to gain a competitive advantage against their opponents. 
% % coaching
% Different from training, coaching provides crucial guidance for players to develop training and playing strategies by extracting and prioritizing valuable insights from match analyses. 
% The success of badminton coaching relies on several factors, including the coach's expertise, resources available to collect and analyze data, and the amount and quality of the coaching time with the player.
% \jui{This paragraph seems largely overlapping with intro. Consider removing or at least shortening}

% rule: https://olympics.com/en/news/badminton-guide-how-to-play-rules-olympic-history
% A badminton match consists of the best-of-three games. 
% A game is won by the first side to win 21 points or two clear points if the score is 20-20. 
% Each point is earned by winning a rally, which involves multiple shots (strokes). 
% % The side that won the previous rally serves first in the next rally. 
% % A point is scored when the bird (shuttlecock) hits the ground in the opponent's court.
% Players change sides after each game and at the midway point of the third game, when one side reaches 11 points. 
% A break is allowed at the midpoint of each game. A typical match takes 40-50 minutes, but the duration can vary depending on the players and circumstances. \jui{the only useful thing in this paragraph seems to be the duration of a typical game. Can we just add it to the intro and remove this paragraph?}

% \subsection{CV Techniques}

% \subsubsection{Pose and shape estimation}
% Using statistical models and a top-down approach, CLIFF (Carrying Location Information in Full Frames)\cite{li-2022} performs a 3D human pose and shape estimation from a single RGB image. The first step is the detection of players. This is done using YoloV3 \cite{yolov3}, a fast and accurate model that predicts object class and bounding boxes by dividing the image into a grid of cells. Then, the prediction of the SMPL parameters from image features is done using regression-based methods. This estimation is used to obtain the human body meshes through a linear function, resulting in a complete 3D representation of the players. \jui{Why do we need these details? Isn't it just applying off-the-shelf technique? If so, we should focus on WHY we need the pose and for the implementation details, just cite the paper. Same for the trajectory reconstruction; my view is we need only a very short description of what we did there, too.}

% \section{Methodologies}
% We applied a user-centered design process to develop VIRD and involved target users (i.e., professional coaches and players) at every design stage. 
% We interviewed experts to identify the design requirements (Sec.\ref{sec:goal_task_analysis}}), 
% and iterate our design decision based on three rounds of  user testing (Sec.\ref{sec:vird}), 
% and finally, we evaluated the use of our tool in analyzing actual matches in a user study (Sec.\ref{sec:user-study}).
% Note that due to the specific domain we target, i.e., high-performance badminton coaching, our design is guided by a small number of domain experts involved in the study. While this is the nature of professional sports, we discuss the implications of conducting research with professional athletes in Sec.~\ref{sec:proathletes}.
% \jui{This is important. However, if we remove 3.1-3.2 (or combine them with intro, then we can put this at the end of the intro as well?}

\section{Methods}
\label{sec:4_methods}

\matmethods{

\subsection*{Participants}
We recruited 26 adult participants from a local university (age: 23.5 ± 2.3 years; 6 females). All were right-handed with normal hearing and vision (with eyeglasses). Each participant received 10 USD compensation and was asked to wear headphones during the study. Three participants were excluded for not wearing headphones, and two were excluded due to incomplete data from technical issues, leaving 21 participants for analysis.
We used a within-subject design where each participant completed 12 randomized sessions combining 3 feedback types (\textit{silence}, \textit{stationary}, \textit{filter}) x 2 durations (short, long) x 2 conditions (with/without distraction). Session order randomization ensured variability. The University of California San Diego Institutional Review Board approved the study, and we obtained written informed consent from all participants beforehand.

\subsection*{Attention Task}
We employed a modified three-choice vigilance task (3CVT) to measure attention \cite{meghdadi2021eeg}. Participants were presented with three geometrical shapes: a target upward triangle, non-target downward triangle, and diamond distractor. Each of the ten trials per session displayed one shape followed by a random time interval. Shapes followed a 4:3:3 (target:non-target:distractor) ratio, with order shuffled to prevent bias.
We made two key modifications to the original 3CVT \cite{meghdadi2021eeg}. First, we introduced short (2-5 s) and long (25-35 s) intervals between shapes, manipulating difficulty. Longer durations required extended focus, thus increasing difficulty. Second, rather than random locations, shapes appeared at the center, enabling direct measurement of continuous attention and response times. Gaze deviations from center indicated lapsed attention, thus increasing response times. Participants pressed left/right arrows for target/non-target shapes, allowing the response time measurement.

\subsection*{Tactile Bodily Gaze Map}
The eyerofeedback system delivered vibratory stimuli to users' wrists and ankles based on eye movement directions. We divided the screen into four areas - Upper Left, Upper Right, Lower Left, Lower Right (Fig.~\ref{f1}) - establishing the following mapping: Upper Left $\rightarrow$ Left Wrist, Upper Right $\rightarrow$ Right Wrist, Lower Left $\rightarrow$ Left Ankle, Lower Right $\rightarrow$ Right Ankle. As users shifted gaze across areas, corresponding vibrations were triggered on their body. For instance, gazing upper left induced left wrist vibration, allowing users to sense their eye movements and regulate attention accordingly.

Vibratory tactile stimuli were delivered via 3D printed black wristbands housing vibration motors (Fig. S1 in Appendix). An Arduino Uno \cite{WinNT} controlled the 1 Hz vibration motors from PC commands. Adjustable wristbands using Velcro ensured comfortable wearing on both wrists and ankles.

Eye gaze data was collected using WebGazer \cite{papoutsaki2016webgazer}. Participants underwent pre-study calibration by clicking dots at various screen locations. Notably, we did not capture or save any face images during data collection. The gaze data contained timestamps for synchronization and (x,y) coordinates of eye movements on-screen.

\subsection*{Feedback Conditions}

We designed three feedback types: \textit{silence}, \textit{stationary}, and \textit{filter}. The \textit{silence} feedback served as the control condition with no tactile feedback. In the \textit{stationary} feedback condition, tactile feedback (eyerofeedback) was consistently provided to the users' entire body based on their real-time eye movements. The \textit{filter} feedback was a modified version of eyerofeedback, which was only triggered when users' eye movement distance exceeded a certain threshold. This decision was based on preliminary observations indicating that the \textit{stationary} feedback might introduce additional distractions to users, as tactile stimuli occurred continuously with any eye movement \cite{horvath2010distraction}. Consequently, the \textit{filter} feedback aimed to mitigate this potential distraction and warranted further investigation.
Specifically, we designated a centered sub-area spanning half the screen width/length (Fig.~\ref{f1}). Only eye movements beyond this sub-area's boundaries triggered tactile stimuli, allowing perception of gaze-pattern eyerofeedback.


\subsection*{Experiment Apparatus and Instruction}

The experimenter began by introducing the basic procedures of the study and addressing any questions or concerns participants had, ensuring their full comprehension of the information provided in the consent form. Subsequently, the experimenter assisted participants in wearing the four wristbands on their wrists and ankles. To minimize any additional pressure, the experimenter did not observe the participants or the screen during the formal study. However, if participants encountered any issues or had questions during the study, the experimenter was available to provide assistance.

Participants were informed that they would be completing an attention task and would receive tactile stimuli on their wrists and ankles, corresponding to their eye movements. They were instructed that the only way to reduce or eliminate the feedback was to maintain focus on the center of the screen. Prior to the formal study, participants were given an explanation of the attention task and provided with an opportunity to practice, allowing them to become familiar with the task.

After each session, participants were instructed to take a minimum of 1-minute rest and complete a questionnaire (Fig.~\ref{f2}) before proceeding to the next session. If participants felt fatigued from the previous study session, they were encouraged to take a longer rest period. Additionally, participants were required to perform a new eye tracking calibration after each session to minimize potential drift in WebGazer.

It is important to note that participants were instructed to wear the wristbands containing the vibration motors throughout the entire study, even during sessions when the feedback type was set to \textit{silence}. Finally, participants were instructed to prioritize accuracy in the attention task, ensuring they pressed the correct key on the keyboard when a shape appeared, and then attempting to respond as quickly as possible.


\subsection*{Data and Code Availability}

All data and codes needed to reproduce the findings and analysis are available at github: https://github.com/songlinxu/Eyerofeedback.
}

\showmatmethods{}




\section{Results}
% \label{sec:5_results}
\section{Experimental prospects for \Afb  and \Rq at ILC250 and ILC500}
\label{sec:results}

Three different scenarios have been studied: reconstruction without TPC kaon ID, a reconstruction using TPC Kaon ID (via \dEdx) for charge measurement as well as adding \dEdx in the flavour tagging and a reconstruction using TPC Kaon ID (via \dNdx) for charge measurement as well as adding \dNdx in the flavour tagging, being the later an estimation as described in section \ref{dNdxsection}. These three scenarios are covered for both 250 and 500 GeV, for the cases of $P_{\mathrm{e^{-}e^{+}}}=(-0.8,+0.3)$ and $P_{\mathrm{e^{-}e^{+}}}=(+0.8,-0.3)$. A comparison for each case using only statistical uncertainties has also been plotted. The results are summarised in Fig. \ref{fig:results}.


% Figure environment removed




\section{Discussion and Limitations}
\label{sec:6_discussion}
\checkednote{\subsection{Definition of a measurement}}
In compressed sensing and coding is typically defined as a measurement of an analog flux through some type of coding projection or in our example a mask. As has been shown by multiple works, this analog model of light does not account for the poisson noise inherent in any real measurements and leads to counter intuitive behavior of coding approaches.

If we instead think of our system as measuring photons through different codes, the code behavior makes intuitive sense. A photon measured through a mask with many open pixels carries less information about the scene than one captured through a raster mask because our measurement is ambiguous regarding which pixel in the mask was the origin of the photon. In a raster mask every photon can be uniquely assigned to one pixel. In essence, more photons do not equal more information.

The implication of this well documented problem become ever more important in the age of low noise and photon counting cameras where Poisson noise dominates all measurements. It is wide reaching since the projection process we study here in a specific coding experiment is part of the design of any camera. In other words: Any camera or vision system has to project data from a high dimensional scene space down into a lower dimensional sensor space where it encounters Poisson noise and then uses those noisy measurements to make inferences about the scene. 

\Xpolish{This paper has shown that the challenges for computational imaging under Poisson noise. Algorithms based on the AGN noise assumption are problematic on the modern sensors. However, if the task requires no reconstruction but direct feature extraction, the Selective Sensing using the Optical Neural Networks model can find the optimal coding methods. We have shown the feasibility of the Selective Sensing via simulations and experiments, and it demonstrated a promising classification performance on the MNIST handwritten number dataset. Also, it is robust in the application scenarios where their noise level is hard to estimate. Furthermore, the Selective Sensing provides an motivation for proposed optical ANNs or ANNs with optical layers which allow us to optimize the coding schemes wherever the Poisson noise happens. }
{Our paper highlights the challenges of computational imaging under Poisson noise and its impact on algorithms based on the AGN noise assumption. We find that for compressible measurements, and especially tasks that involve direct feature extraction instead of signal reconstruction, a Selective Sensing approach using task optimized codes provides a viable coding solution. Through simulations and experiments, we demonstrate the feasibility of Selective Sensing and its promising classification performance on the MNIST handwritten number dataset. It is also robust in application scenarios with difficult-to-estimate noise levels. Our ONN method represents a method that can generate these selective measurements. Furthermore, Selective Sensing motivates the development of optical ANNs or ANNs with optical layers to globally optimize imaging systems.}

\Xpolish{On the other hand, there are some limitations in our project. First, we employed the AGN model and reparameterization trick for the model training, which is only an approximation to the noise at the sensor. Second, our test set in the experiments only contains 10 numbers, which may not be representative enough. There is also an inconsistency as the model was trained by simulated data but tested with experimental data. Last but not the least, the Photon Distribution Factor rescales the masks $\B{M}$, but its value changes during the training and it is not evolved in the back-propagation. In general, the optimization of the ONN model still has some defects and we still need to improve the results by using better optimization methods and more experimental data.}
{Despite the promising results of our project, there are some limitations that must be acknowledged. First, we used a Gaussian noise model with reparameterization to train our model, which is only an approximation of the actual quantization noise at the sensor. Second, our test set consisted of only 10 numbers, which may not provide a comprehensive evaluation of the model's performance. Additionally, we noted an inconsistency in that the model was trained using simulated data but tested with experimental data. Lastly, the Photon Distribution Factor rescales the masks $\B{M}$, but its value changes during training and is not evolved during back-propagation. These limitations highlight the need for further improvements in the optimization of the ONN model, such as using more advanced optimization methods and larger sets of experimental data. Our work highlights the importance of the integration of imaging hardware and signal processing. In single photon accurate imaging systems, comprehensibility and sparsity of the data can be exploited to far greater effect during the measurement, as opposed to post processing.}



% Any acknowledgments to only be included in camera ready
% \ifpeerreview \else
% \section*{Acknowledgments}
% The authors would like to thank...
% \fi

\clearpage


\bibliographystyle{IEEEtran}
\bibliography{\homedir/Main/bibtex.bib}
% \bibliography{hardware_pca_onn/Main/bibtex}

\ifpeerreview \else
%%%% For the camera ready version, please fill out this
%%%% biography. Your camera ready should be within a 12 page limit
%%%% including acknowledgments, references and biography.

% If you have an EPS/PDF photo (graphicx package needed) extra braces are
% needed around the contents of the optional argument to biography to prevent
% the LaTeX parser from getting confused when it sees the complicated
% \includegraphics command within an optional argument. (You could
% create your own custom macro containing the \includegraphics command
% to make things simpler here.)
% \begin{IEEEbiography}[{% Figure removed}]{Michael Shell}
% or if you just want to reserve a space for a photo:

% 


\begin{IEEEbiography} [{% Figure removed}]{Yizhou Lu}
\Xhide{Biography text here.}
obtained his B.S. degree in Geochemistry from Nanjing University, China, in 2017, followed by M.S. degrees in Materials Science and Engineering and Computer Sciences from the University of Wisconsin-Madison, WI, USA, in 2019 and 2023, respectively. Currently pursuing a Ph.D. degree with the Electrical and Computer Engineering Department at the University of Wisconsin-Madison, his research interests encompass computational imaging and information theory.
\end{IEEEbiography}




\begin{IEEEbiography} [{% Figure removed}]{Trevor Seets}
\Xhide{Biography text here.}
received his B.S. and M.S. degree in electrical engineering from the University of Wisconsin-Madison in 2019 and 2023, respectively. He is now a Ph.D. student under Professor Andreas Velten at UW-Madison. His research interests include computational optics, statistical signal processing, and imaging.
\end{IEEEbiography}

\begin{IEEEbiography} [{% Figure removed}]{Felipe Gutierrez Barragan}
\Xhide{Biography text here.}
is a Senior Software Engineer at Ubicept. He received his B.S. (2016), M.S. (2019), and Ph.D. (2022) in Computer Sciences from the University of Wisconsin-Madison.  His Ph.D. research focused on practical modifications to SPAD-based and indirect time-of-flight 3D cameras to reduce power consumption and data bandwidth while preserving or increasing their precision. His research interests include computational imaging, computer vision, and machine learning.
\end{IEEEbiography}

\begin{IEEEbiographynophoto}{Ehsan Ahmadi} earned his B.S. in Computer Software Engineering from Iran University of Science and Technology, Tehran, Iran, and his M.S. in Electrical and Electronics Engineering from Southern Illinois University, Carbondale, Illinois. Currently, he is pursuing a Ph.D. in Computer Engineering at UW-Madison, focusing on Fluorescence Lifetime Imaging Microscopy (FLIM).



    
\end{IEEEbiographynophoto}



\begin{IEEEbiography} [{% Figure removed}]{Andreas Velten}
\Xhide{Biography text here.}
received the B.A. degree in physics from the Julius Maximilian University of Wurzburg, Wurzburg, Germany, in 2003, and the Ph.D. degree in physics from the University of New Mexico, Albuquerque, NM, in 2009. From 2010 to 2012, he was a postdoctoral research associate with the Massachusetts Institute of Technology, Cambridge, MA, USA, where he was working on high speed imaging systems that can look around
a corner using scattered light. From 2013 to 2016, he was an associate scientist with the Laboratory for Optical and Computational Instrumentation, University of Wisconsin–-Madison, Madison WI, USA, working in optics, computational imaging, and medical devices. Since 2016, he has been an associate professor with the Biostatistics and Medical Informatics, Electrical and Computer Engineering Department, University of Wisconsin–-Madison. His research focuses on performing multidisciplinary work in applied computational optics and imaging.
\end{IEEEbiography}


% insert where needed to balance the two columns on the last page with
% biographies
%\newpage

% if you will not have a photo at all:
%\begin{IEEEbiographynophoto}{John Doe}
%Biography text here.
%\end{IEEEbiographynophoto}

% You can push biographies down or up by placing
% a \vfill before or after them. The appropriate
% use of \vfill depends on what kind of text is
% on the last page and whether or not the columns
% are being equalized.
%\vfill

\fi


