\Xpolish{Though we are able to find the ideal $\B{M}$ based on the tasks we want to accomplish, the optimization problem of $\B{M}$ may not be convex due to the following two constraints.}
{\Xpolish{While we can identify the optimal $\B{M}$ tailored to the tasks at hand, it's crucial to acknowledge that the optimization problem for $\B{M}$ might not exhibit convexity. This is attributed to the presence of two constraining factors. Consequently, achieving globally optimal masks may not be universally feasible within the scope of this problem.}{The implementation of sensing matrix $\B{M}$ usually involves the following constraining factors.}}

\begin{enumerate}
    \item\label{constraint:photonNumber} \textbf{Flux-preserving} \cite{willet2009CSPoisson}\Xhide{\bnote{Y: Should I use the same term as Rebecca?}}. The single-pixel camera model involves the allocation of available photons among masks, as discussed in \cite{neifeld2003dual}. It is important to ensure that the mask basis $\B{M}$ does not produce additional photons through improper entries \cite{neifeld2003dual}. Mathematically, $\sup \sum_{i=1}^m M_{ij} = 1, \forall j \in \{1,2,\dots,N\}$ \cite{neifeld2003dual}.\Xhide{\bnote{Include what an improper entry is!} \bnote{A: this is not something we need to mention here. its obvious that the measurements have to be normalized correctly}}    
    \item\label{constraint:negativeEntry} \textbf{Positivity-preserving} \cite{willet2009CSPoisson}. It is not possible to physically implement negative values for the masks $\B{M}$, as demonstrated in \cite{neifeld2003dual, willet2009CSPoisson}. In this project, we employed the dual-rail approach as outlined in the work of Neifeld et al where the positive and negative entries of a mask give rise to two separate measurements \cite{neifeld2003dual}.
\end{enumerate}