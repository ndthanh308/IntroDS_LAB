\iffalse
\IEEEPARstart{T}{his} demo file is intended to serve as a ``starter
file'' for ICCP 2020 submissions produced under
\LaTeX~\cite{kopka-latex} using IEEEtran.cls version 1.8b and later.
\fi

\AVnote{\checkednote{Lets try to make the text about machine vision system optimization. Could start with: While we have the ability to mass produce optical components of increasing complexity, creation of human interpretable imaging by closely following the optics of the human eye is still the design principle of most vision hardware. We use a single pixel camera relying on optical coding as a test bed system to optimize. This system can produce images, but allows great freedom to perform alternative measurements while being easily described mathematically.}}

% \YLnote{\IEEEPARstart{D}{}espite advances in mass-producing complex optical components, most vision systems still follow the template of human vision where the optics create a two dimensional image that is then digitized and processed. A more general imaging system that has been extensively studied is the single-pixel camera that relies on masks to perform coded measurements from which images are computed. This setup also offers flexibility for alternative measurements, with straightforward mathematical descriptions.} 


% While we have the ability to mass-produce optical components of increasing complexity, the design principle for most vision hardware still closely follows the optics of the human eye to create human-interpretable imaging. However, the more generalized vision systems can also perform alternative measurement which can be easily described in mathematical ways and is more perceptually meaningful than just producing images. We use a single-pixel camera, which can be seen as a system in this kind,  relying on masks to perform coded measurements as a test bed for hardware optimization.

% can depend on direct measurement with visual perception as well. We use a single-pixel camera relying on masks to perform coded measurements from which images are computed as a test bed to optimize, as it not only can produce images but also allows great freedom to perform alternative measurements while being easily described mathematically.

\IEEEPARstart{W}{}hile we possess the capability to mass-produce optical components of increasing complexity, the design principle for most vision hardware still closely mimics the optics of the human eye to create human-interpretable imaging. However, more generalized vision systems can also perform alternative coded measurements that are not only easily described mathematically but also potentially more perceptually meaningful than merely producing images. We employ a single-pixel camera as a test bed for hardware optimization, which exemplifies such a system, relying on masks to conduct optically-coded measurements. 

% Optical coding, also known as multiplexing, is a popular technology in computational imaging (CI) due to its potential to achieve enhanced signal-to-noise ratio (SNR) and greater light throughput than point by point measurements\cite{mitra2014can}. If ignoring the noise, coding is represented as 
%  \begin{equation}
%     \label{eqn:NoiselessMeasurement}
%     \begin{aligned}
%         \B{y} &=  \B{M x} ,
%     \end{aligned}
% \end{equation}
% where $\B{x}$ is the signal vector, $\B{M}$ is the sensing matrix, and $\B{y}$ is the measured vector. 

% Maybe: For optical coding or multiplexing an image is prjoected onto a mask and only light transmitted to the mask is collected by a single large sensor pixel. A measurement consists of a set of measurements expressed as a vector y with a set of masks. The vectiruzed masks form the rows of the measurement matrix M.

% Optical coding, or multiplexing, is a popular technology projecting an image onto a mask and collecting the light transmitting it by a large senor pixel. It potentially achieves enhanced signal-to-noise ratio (SNR) and greater light throughput than point by point measurements\cite{mitra2014can}. A measurement consists of a set of measurements expressed as a vector $\B{y}$ with a set of masks vectorized as the rows of the measurement matrix $\B{M}$. If ignoring the noise, coding is represented as 

Optical coding, or multiplexing, is a popular technology that projects an image onto a mask and collects the light transmitted through it using a large sensor pixel. It potentially achieves an enhanced signal-to-noise ratio (SNR) and greater light throughput compared to point-by-point measurements \cite{mitra2014can}. A measurement vector $\B{y}$ consists of a set of measured photon counts, with the corresponding masks vectorized as the rows of the sensing matrix $\B{M}$. Ignoring the noise, coding can be represented as
 \begin{equation}
    \label{eqn:NoiselessMeasurement}
    \begin{aligned}
        \B{y} &=  \B{M x} ,
    \end{aligned}
\end{equation}
where $\B{x}$ is the signal vector. 

Decoding refers to the process of recovering the initial signal $\B{x}$ from $\B{y}$. The recovery quality of $\B{x}$, however, also depends on the conditioning of $\B{M}$ and the light throughput of $\B{M}$~\cite{mitra2014can}. Suboptimal $\B{M}$ have the potential to significantly degrade the performance of the vision system during the recovery process, as highlighted by Mitra et al. \cite{mitra2014can}. The optimization of coding schemes regarding $\B{M}$ is intricately linked to the characteristics of noise. 
%---- delete
\iffalse
{
Coding schemes, typically formulated under the assumption of signal-independent Additive Gaussian Noise (AGN), offer a mathematically tractable framework. 
However, AGN, while suitable for modeling certain noise types introduced by measurement devices, inadequately represents \Xpolish{Poisson noise arising intrinsically due to the discrete and random nature of photon measurements}{\YLnote{\Xpolish{photon noise which arises intrinsically due to the discrete and random nature of photon measurements and is commonly approximated by Poissonian variables despite that it is indeed super-Possionian. However, we continue to use the term "Poisson noise" to maintain consistency with the terminology commonly found in the literature.}{Photon noise arises intrinsically due to the discrete and random nature of photon measurements and is commonly approximated by Poissonian variables, even though it is actually super-Poissonian. However, we continue to use the term "Poisson noise" to maintain consistency with the terminology commonly found in the literature.}}} \AVnote{\checkednote{Lets call it photon noise. Photon noise is more complex and Poisson distribution is just an approximation. Then we can explain later that the key feature to model is the fact that the variance depends on the mean.}}\cite{yang2015poisson}Past works usually distinguish between low-light regimes, where AGN introduced by imperfect sensors dominates, and bright-light regimes, where the physical photon noise dominates \cite{mitra2014can}. With advances in sensor design, even the sensor noise of low-end camera sensors has been reduced to a few photons per pixel. This reduction is significant enough that photon noise becomes the primary noise factor in the vast majority of imaging applications. }
\fi
%----
\YLnote{
Coding schemes, typically formulated under the assumption of signal-independent Additive Gaussian Noise (AGN) which arises from imperfect sensors and is introduced from the low-light regimes in many past works, offer a mathematically tractable framework. With advances in sensor design, even the sensor noise of low-end camera sensors has been reduced to a few photons per pixel. This reduction is significant enough that photon noise becomes the primary noise factor in the vast majority of imaging applications \cite{cossairt2012does}. Photon noise is intrinsically signal-dependent \cite{cossairt2012does} and usually approximated by Poisson noise, even though it is actually \textbf{super-Poissonian} \cite{mandel1959fluctuations}. 
% In this work, we focus on its signal-dependent feature and propose 
% arises intrinsically due to the discrete and random nature of photon measurements and is commonly approximated by Poissonian variables. 
}
\YLnote{Unfortunately, the effects of changing the noise model are not subtle as designs that are optimal under AGN conditions often fail to approximate optimal designs under photon noise.
%
The intensity dependence of photon noise renders the camera a nonlinear system, necessitating nonlinear optimization methods. Linear optics are only truly linear in the limit of infinite photons, where quantum noise vanishes. Consequently, the linear approximation frequently used is not applicable at finite light levels.}\AVnote{\checkednote{Add here: Unfortunately, the effect of changing the noise model are not subtle. Designs optimal under AGN are usually not even approximations of optimal designs under photon noise. In fact, the intensity dependence of the noise turns the camera into a nonlinear system requiring nonlinear optimization approaches. Linear optics are only truly linear at the infinite photon limit where quantum noise disappears. The linear approximation commonly used is not applicable at any finite light level. }} 
\AVnote{\checkednote{ While this issue affects any camera optic it has been studied extensively in the context of single pixel and compressed sensing cameras that rely on coding. For example, it has been shown, that while a single pixel camera provides performance advantages under AGN, in the case of  Poisson noise that more closely resembles photon noise, coding does not ....}}  \YLnote{\Xpolish{It has been shown that in that case without the use of priors for regularization, coding does not provide an improvement over a simple sequential scan}{ While this issue impacts all camera optics, it has been extensively studied in the context of single-pixel and compressed sensing cameras that rely on coding techniques. }
For instance, although single-pixel cameras demonstrate performance advantages under AGN, coding does not offer improvements over a simple sequential scan when dealing with \Xpolish{Poisson noise \cite{harwit1979hadamard, swift1976hadamard, willet2009CSPoisson, scotte2022photon_noise, vanden2019various}, which more closely resembles photon noise.}{photon noise, or Poisson noise \cite{harwit1979hadamard, swift1976hadamard, willet2009CSPoisson, scotte2022photon_noise, vanden2019various}.}
}
{
Mitra et al. \cite{mitra2014can} develop a data driven prior to design optimized codes for image reconstructions that can markedly enhance the reconstruction performance of coding schemes under Poisson noise \cite{mitra2014can}. \YLnote{An important conclusion is that if the signal exhibits sparsity and thereby  compressibility, signal optimized coding can improve recovery performance under signal-dependent noise and the amount of improvement is directly related to the comprehensibility of the measured data. However, recovery is typically not the ultimate goal for modern vision systems.}
}{
\YLnote{Since vision tasks, such as classification, operate in embeddings created from images their data should be inherently more compressible than their upstream imaging tasks. Therefore, if signal compressibility is the key, skipping the intermediate imaging step and directly focusing on coding for better feature extraction for subsequent vision tasks should yield better performance.} To show this, our study implements a simple end to end vision system that can be optimized  under different noise assumptions. We illustrate why AGN is not suitable for optimizing coding under \Xpolish{Poisson}{photon} noise and show that a coding scheme optimized for a specific vision task and for the correct noise model can perform close to the optimal non-coding camera even when \Xhide{only }photon noise is present.
}

Our contributions are as listed.
\begin{itemize} 
\item {We introduce the methodology of Selective Sensing (SS), which encompasses coding techniques specifically designed to extract data-driven priors selectively. \YLnote{These coding methods can be seamlessly integrated with models for downstream tasks, providing a novel approach to efficiently incorporate learned priors into various applications.\checkednote{\textbf{sales pitch?}}}}
\item \Xpolish{We provide a model to {optimize $\B{M}$} for Poisson noise using neural network model. This model is end to end vision model with classification as the performance metric}{
We provide an end-to-end vision model using classification as the performance metric, which is more perceptually meaningful than mean-squared-error (MSE), to optimize $\B{M}$.
}
\end{itemize}

Our model is extendable to more general camera optics and different vision tasks in future work.