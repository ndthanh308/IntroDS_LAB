\iffalse
\IEEEPARstart{T}{his} demo file is intended to serve as a ``starter
file'' for ICCP 2020 submissions produced under
\LaTeX~\cite{kopka-latex} using IEEEtran.cls version 1.8b and later.
\fi

\IEEEPARstart{W}{}hile we possess the capability to mass-produce optical components of increasing complexity, the design principle for most vision hardware still closely mimics the optics of the human eye to create human-interpretable imaging where the lens projects an image that is then post-processed to extract patterns.
More generalized vision systems can perform many alternative measurements that are potentially more perceptually meaningful than merely producing images. Designing such a system requires co-designing hardware and digital processing with the correct noise model.
To explore the design, we employ a single-pixel camera as a toy model for joint vision system optimization. 
The single pixel camera uses the concept of optical coding, or multiplexing, which involves projecting an image onto a mask and collecting the transmitted light with a large sensor pixel.
% We employ a single-pixel camera as a test bed for hardware optimization, which exemplifies such a system, relying on masks to conduct optically-coded measurements.
% Optical coding, or multiplexing, is a popular technology that projects an image onto a mask and collects the light transmitted through it using a large sensor pixel. 
The system can potentially achieve an enhanced signal-to-noise ratio (SNR) and greater light throughput by coding compared to point-by-point measurements \cite{mitra2014can}.
In this context, a measurement vector $\B{y}$ is formed from a set of measured photon counts, with the corresponding masks vectorized as the rows of the sensing matrix $\B{M}$.
Ignoring the noise, coding can be represented as
 \begin{equation}
    \label{eqn:NoiselessMeasurement}
    \begin{aligned}
        \B{y} &=  \B{M x} ,
    \end{aligned}
\end{equation}
where $\B{x}$ is the signal vector. 

Decoding refers to the process of recovering the initial signal $\B{x}$ from $\B{y}$.
The recovery quality of $\B{x}$, however, also depends on the conditioning of $\B{M}$ and the light throughput of $\B{M}$~\cite{mitra2014can}.
Suboptimal $\B{M}$ have the potential to significantly degrade the performance of the vision system during the recovery process, as highlighted by Mitra et al. \cite{mitra2014can}. %The optimization of coding schemes regarding $\B{M}$ is intricately linked to the characteristics of noise. 

% \subsection*{Transitioning from Gaussian to Photon Noise Models}
Code design is often performed under the assumption of signal-independent Additive Gaussian Noise (AGN) which arises from imperfect sensors.
With advances in sensor design, even the sensor noise of low-end camera sensors has been reduced to a few photons per pixel.
This reduction is significant enough that non-additive photon noise becomes the primary noise factor in the vast majority of imaging applications \cite{cossairt2012does}. Photon noise follows a Super Poissonian distribution~\cite{mandel1959fluctuations}. For passive cameras operated with incoherent ambient light the noise can usually be approximated as Poissonian.

% \subsection*{Inapplicability of Gaussian Noise for Poisson Mask Optimization}
The transition from AGN to more resalistic photon noise  fundamentally alters the designs of vision systems.
%Linear optics are only truly linear in the limit of infinite photons, where quantum noise vanishes.
%The intensity dependence of photon noise renders the camera a nonlinear system, necessitating nonlinear optimization methods. 
%Consequently, the linear approximation frequently used for AGN is not applicable at finite light levels.
%While this issue impacts all camera optics, it has been extensively studied in the context of single-pixel and compressed sensing cameras that rely on coding techniques. 
For instance, although single-pixel cameras with random masks provides performance advantages under AGN, but does not offer improvements over a simple sequential scan when dealing with  Poisson noise \cite{harwit1979hadamard, swift1976hadamard, willet2009CSPoisson, scotte2022photon_noise, vanden2019various}.
Furthermore, the optimal coding designs under AGN conditions fail to deliver optimal performance under photon noise. 

In response to this challenge, Mitra et al. \cite{mitra2014can} develop a data driven prior to design optimized codes for image reconstructions that can markedly enhance the reconstruction performance of coding schemes under Poisson noise \cite{mitra2014can} by designing measurements specifically targeting sparsity or compressibility in the data. 
%An important conclusion from their work is that if the signal exhibits sparsity and thereby compressibility, signal optimized coding can improve recovery performance under signal-dependent noise and the amount of improvement is directly related to the comprehensibility of the measured data. 

However, recovery is typically not the ultimate goal for modern vision systems.
Since vision tasks, such as classification, operate in embeddings created from images, their data should be inherently more compressible than their upstream imaging tasks allowing for more effective measurement designs. 
%Therefore, if signal compressibility is the key, skipping the intermediate imaging step and directly focusing on coding for better feature extraction for subsequent vision tasks should yield better performance.
To show this, our study implements a simple end-to-end vision system that can be optimized  under different noise assumptions. 
We illustrate why AGN is not suitable for optimizing coding under \Xpolish{Poisson}{photon} noise and show that a coding scheme optimized for a specific vision task and for the correct noise model can perform close to the optimal non-coding camera even when \Xhide{only }photon noise is present.


Our contributions are as listed.
\begin{itemize} 
\item {We introduce the methodology of Selective Sensing (SS), which encompasses coding techniques specifically designed to extract data-driven priors selectively for \YLnote{various downstream tasks.} \Xhide{\YLnote{These coding methods can be seamlessly integrated with models for downstream tasks, providing a novel approach to efficiently incorporate learned priors into various applications.\checkednote{\textbf{sales pitch?}}}}}
\item {
We provide an end-to-end vision model using classification as the performance metric, which is more perceptually meaningful than mean-squared-error (MSE), to optimize $\B{M}$.
}
\item {
% \textbf{We train the model under the physically accurate signal-dependent photon noise, as opposed to its approximated version of Poisson noise, to enhance the robustness of our model.}
% We extend Poisson noise to an alternative approximation that places greater emphasis on signal-dependency property, using the reparameterization trick to effectively trace the gradient and train the model to be near optimal when tested under Poisson noise.
We extend Poisson noise to an alternative approximation that prioritizes the signal-dependency property, using the reparameterization trick to effectively trace the gradient and train the model to achieve near-optimal performance when evaluated under Poisson noise conditions.
}
\end{itemize}

Our model is extendable to more general camera optics and different vision tasks in future work.