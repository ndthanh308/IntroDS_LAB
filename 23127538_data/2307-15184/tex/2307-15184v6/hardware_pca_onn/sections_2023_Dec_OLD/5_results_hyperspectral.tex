%%%%%%%%%%%%%%%%%%%%%%%%%%%%%%%%%%%%%%%%%%%%%%%%%%%%%%%%%%%
\subsection{Simulated Experiment on Hyper-Spectral Data}

\Xpolish{For this experiment, we use the Indian Pines dataset \cite{PURR1947}, which was acquired by employing Airborne Visible / Infrared Imaging Spectrometer (AVIRIS) on June 12, 1992 over the Purdue University Agronomy farm northwest of West Lafayette and the surrounding area.
This dataset is composed of $145 \times 145$ pixels and 224 spectral reflectance bands in the wavelength range 0.4 to 2.5  $\mu m$
In Fig. \ref{fig:pines_dataset}, the classes of the Indian Pines dataset are depicted, along with the respective counts within the dataset. The figure showcases an illustration of one channel (band 12) for all pixels, along with the ground truth of semantic labeling.}
{
In this experiment, we utilized the Indian Pines dataset \cite{PURR1947}, which was acquired through the Airborne Visible/Infrared Imaging Spectrometer (AVIRIS) on June 12, 1992, covering the Purdue University Agronomy farm northwest of West Lafayette and its surrounding area. Comprising $145 \times 145$ pixels and 224 spectral reflectance bands in the wavelength range of 0.4 to 2.5 $\mu m$, this dataset provides a comprehensive view of the study area. Fig. \ref{fig:pines_dataset} illustrates the classes within the Indian Pines dataset, presenting the counts associated with each class. The figure further features a representation of one channel (band 12) for all pixels, along with the ground truth of semantic labeling.
}

\Xpolish{In this experiment, we zero-padded each spectrum from 224 to 256 to align with the Hadamard mask. Spectral data are normalized within the range of $[0,1]$. The scanner-classifier module from Fig. \ref{fig:model_config} was utilized, with modifications made to the ONN classification layers to match the sizes of the new dataset. Each classification was performed for 2000 epochs, with a learning rate of $5 \times 10^{-3}$ and batch size of 5000.
Figure \ref{fig:pines_result} shows the classification rates for the Indian Pines dataset varying light level from $10^{1}$ to $10^{10}$.
% $10^-1$ to $10e^10$.
% To ensure accuracy, each classification was repeated 5 times, and the average was computed as the final result. 
}
{
In this experiment, spectral data were zero-padded from 224 to 256 to align with the Hadamard mask and \YLnote{\Xpolish{normalized}{re-scaled}} to the range $[0,1]$. We utilized the scanner-classifier module from Fig. \ref{fig:model_config}, adapting the ONN classification layers to the new dataset sizes. Each classification ran for 2000 epochs, with a learning rate of $5 \times 10^{-3}$ and a batch size of 5000. Figure \ref{fig:pines_result} illustrates the classification rates for the Indian Pines dataset across varying light levels from $10^1$ to $10^{10}$.
}


% by employing an 80-megapixel RGB camera and a hyperspectral CMOS sensor from a height of approximately 5,000 feet above the Rochester Institute of Technology’s university campus as shown in figure \ref{fig:aerorit_dataset}. The dataset includes a hyperspectral dataset with almost seven-million-pixel annotations. We utilized surface reflectance as our dataset, which was calculated from the calibrated radiance image. Each pixel in the dataset contains a spectrum of size 51, and it is labeled for different categories including cars, vegetation, buildings, water, and roads as shown in figure \ref{fig:AeroRIT_dataset_labels }. It is worth noting that the cars category is notably under-represented compared to the other classes. Based on the observations depicted in figure \ref{fig:AeroRIT_Histogram}, it is evident that spectral data corresponding to certain labels, such as water, exhibit minimal scattering, and demonstrate a uniform distribution pattern. Conversely, labels such as buildings, which possess varying colors, exhibit a greater degree of scattering in their spectra.



% \printinunitsof{in}\prntlen{\textwidth}



% Figure environment removed


% % Figure environment removed

% Figure environment removed


% % Figure environment removed

% % Figure environment removed


% Figure environment removed