\subsection{Constraints Implementation}

\Xpolish{This section introduces our methods for the optical implementations for the model. To construct a valid model, the Flux-preserving and Positivity-preserving constraints has been satisfied by the following techniques.}
{In this section, we outline our methods for implementing the model optically. In order to ensure that our model is valid, we have implemented techniques that satisfy both the Flux-preserving and Positivity-preserving constraints. By using these techniques, we can ensure that the model accurately represents the underlying physical processes and produces results that are both reliable and interpretable.}

\subsubsection{Photon Distribution Factor} \label{sssection:PDF}
\Xhide{\bnote{A: This section shouldn't be here}}

\Xpolish{The optimization of $\B{M}$ can be challenging, as it is subject to constraint \ref{constraint:photonNumber}.\Xhide{ Specifically, $\B{M}$ must belong to a subset $S$ of $\IR^{m \times N}$, which can vary depending on the specific case \bnote{<- not sure why this sentence is here}. To address this issue \bnote{the issue of belonging to a subset?}, } Here, we propose using a scale factor as a solution. This allows us to select any values for $\B{M}$ without being limited by the constraint \ref{constraint:photonNumber}\Xhide{\bnote{It would be nice to name the constraints}}. The Photon Distribution Factor, $\lambda$, is thus introduced to ensure compliance with constraint \ref{constraint:photonNumber}. As a scale factor, it is calculated as \Xequa{\lambda = \frac{\flux \totexposure}{N \sum_{k=1}^m v_k}},\Xhide{ \bnote{this should be in an equation line}} where $\flux$ represents the number of photons per second projected onto the DMD, $\totexposure$ is the total exposure time for all masks, $N$ represents the number of pixels, and $v_k$ represents the greatest value of the $k_\text{th}$ mask. The maximum value $v_k$ is selected as its corresponding pixel is open for longer than any other one in the $k_\text{th}$ mask, and the exposure time of this mask depends solely on $v_k$. Furthermore, the total exposure time $\totexposure$ is proportional to $\sum_{k=1}^m v_k$ and the total number of photons from the field of view is $\totexposure \sum_{i=1}^N \frac{\flux}{N} x_i$.  }
{Optimizing $\B{M}$ subject to constraint \ref{constraint:photonNumber} can be challenging. To overcome this limitation, we propose the use of a scale factor, known as the Photon Distribution Factor $\lambda$, which allows us to select any values for $\B{M}$ without violating constraint \ref{constraint:photonNumber}. The Photon Distribution Factor is calculated as \Xequa{\lambda = \frac{\flux \totexposure}{N \sum_{k=1}^m v_k}}, where $\flux$ is the number of photons per second projected onto the DMD, $\totexposure$ is the total exposure time for all masks, $N$ is the number of pixels, and $v_k$ is the greatest value of the $k_\text{th}$ mask. The maximum value $v_k$ is chosen as its corresponding pixel is open for longer than any other pixel in the $k_\text{th}$ mask, and the exposure time of this mask depends solely on $v_k$. Moreover, the total exposure time $\totexposure$ is proportional to $\sum_{k=1}^m v_k$, while the total number of photons from the field of view is $\totexposure \sum_{i=1}^N \frac{\flux}{N} x_i$.}

\Xhide{\rnote{Maybe in appendix} \bnote{ Could just state the first and last lines of the math and move the proof to the appendix but still state this ensures constraint one is true}
% Suppose we use an arbitrary matrix $\{\B{a}_1, \dots, \B{a}_m \} \in \IR^{N \times m}$ for measurement, the total number of photons detected is
When an arbitrary measurement matrix, $\{\B{a}_1, \dots, \B{a}_m \} \in \IR^{N \times m}$, is used, the total number of detected photons can be expressed as $\sum_{i=1}^m \lambda \B{a}_i^\top \B{x}$ and the following inequality always holds:
\begin{equation}
\begin{aligned}
    \sum_{i=1}^m \lambda \B{a}_i^\top \B{x} &\le \sum_{i=1}^m \lambda \sum_{j=1}^N v_i x_j\\
    &= \sum_{i=1}^m \sum_{j=1}^N \frac{\flux \totexposure}{N \sum_{k=1}^m v_k} v_i x_j \\
    &= \frac{\flux \totexposure}{N} \sum_{j=1}^N x_j  \frac{\sum_{i=1}^m v_i}{\sum_{k=1}^m v_k} \\
    &= \totexposure \sum_{j=1}^N \frac{\flux}{N} x_j
\end{aligned}.
\end{equation}
% Therefore, by the Photon Distribution Factor, the constraint \ref{constraint:photonNumber} is never violated.
}

\Xpolish{By utilizing the Photon Distribution Factor, constraint \ref{constraint:photonNumber} is guaranteed to be never violated. Then we can re-formulate model in Equation \ref{eqn:NoiselessMeasurement} as the following equation
 \begin{equation} \label{eqn:NoisyMeasurement}
    \begin{aligned}
        \tilde{\B{y}} &=  \frac{1}{\lambda}f_r(\lambda \B{ M x})
    \end{aligned}
\end{equation}
, where $f_r$ generates random numbers with expectation $\lambda \B{ M x}$ to simulate the noise and gives rise to the measurement with noise \bnote{Be consistent with language throughout the text. "noisy-version-measurement" is probably not the best name} $\tilde{\B{y}}$.}
{By utilizing the Photon Distribution Factor, we can guarantee that constraint \ref{constraint:photonNumber} is never violated. With this in mind, we can re-formulate the model in Equation \ref{eqn:NoiselessMeasurement} into the following equation,
\Xequa{\tilde{\B{y}} &=  \frac{1}{\lambda}f_r(\lambda \B{ M x})}.
Here, $f_r$ generates random numbers with an expected value of $\lambda \B{ M x}$ to simulate noise and produce a measurement with noise, denoted as $\tilde{\B{y}}$.}

\subsubsection{Dual-Rail Technique for Negative Entries}\label{sssection:DualBranch}
\Xpolish{\Xhide{\ynote{this could be part of a methods section later on or go to the supplement. Shouldn't be here.}\bnote{Consistent language "Rail" or "Branch" or "Path"} }
The dual-rail technique is introduced for constraint \ref{constraint:negativeEntry} \cite{neifeld2003dual}. In it\Xhide{ \bnote{ might be better to call it a "technique"}}, one measurement needs two complementary branches where one measures with all positive entries, $\B{M}^+$, and the other measures with all negative entries, $\B{M}^-$ \cite{neifeld2003dual}. }
{To satisfy constraint \ref{constraint:negativeEntry}, the dual-rail technique \cite{neifeld2003dual} employs two complementary measurement branches: one measures with all positive entries, denoted as $\B{M}^+$, while the other measures with all negative entries, denoted as $\B{M}^-$. By doing so, the dual-rail technique effectively converts the constraint on negative entries to a constraint on the difference between the measurements made by the two branches.}

% Figure environment removed


\Xpolish{\bnote{Many of these equations should be on their own line for readability and to reference them later. We have 11 pages so we can do this easily.}
To create the two branches, the mask $\B{M}$ should be split into $\B{M}^+ = \max(\B{M}, 0)$ and $\B{M}^- = \max(-\B{M}, 0)$. The superscripts of the parameters indicate the branches they belong to. Intuitively, we have $\lambda \tilde{\B{y}} = \lambda \tilde{\B{y}}^+ - \lambda \tilde{\B{y}}^-$ where $\tilde{\B{y}}^+$ and $\tilde{\B{y}}^-$ are photon counts with noise measured by $\B{M}^+$ and $\B{M}^-$ respectively. The two-rail trick also changes the expression of the  Photon Distribution Factor. In the case that each branch has a PMT, 
\Xequa{\lambda = \frac{\flux \totexposure}{N \sum_{k=1}^m |v_k|}}
where $|\cdot|$ takes the absolute value. In the case of using only one PMT, it is worth noting that the Photon Distribution Factor changes to 
\Xequa{\lambda = \frac{\flux \totexposure}{N (\sum_{k=1}^m v_k^+ + \sum_{k=1}^m v_k^-)}}
as one PMT measures with $\B{M}_k^+$ and $\B{M}_k^-$ in turn to accomplish measuring $\B{M}_{k}$. \iffalse as there are approximately two times of measurements. \fi \rnote{figure}All the parameters in this updated Photon Distribution Factor are of the same meaning as in section \ref{sssection:PDF}.}{To implement the two branches, we split the mask $\B{M}$ into two components: $\B{M}^+ = \max(\B{M}, 0)$ and $\B{M}^- = \max(-\B{M}, 0)$, with the superscripts indicating the respective branches. This enables us to express $\lambda \tilde{\B{y}}$ as the difference between the photon counts with noise measured by $\B{M}^+$ and $\B{M}^-$, denoted $\lambda \tilde{\B{y}}^+$ and $\lambda \tilde{\B{y}}^-$, respectively. Notably, the two-rail approach also modifies the expression for the Photon Distribution Factor. In the case that each branch has a PMT, 
\Xequa{\lambda = \frac{\flux \totexposure}{N \sum_{k=1}^m |v_k|}}, 
where $|\cdot|$ takes the absolute value. In the case of using only one PMT, it is worth noting that the Photon Distribution Factor changes to 
\Xequa{\lambda = \frac{\flux \totexposure}{N (\sum_{k=1}^m v_k^+ + \sum_{k=1}^m v_k^-)}},
since one PMT measures with both $\B{M}_k^+$ and $\B{M}_k^-$ sequentially to obtain the measurement result for $\B{M}_{k}$. All the parameters in this updated Photon Distribution Factor are of the same meaning as in section \ref{sssection:PDF}, and $v_k^+$ and $v_k^-$ stand for the maximum values of $\B{M}^+$ and $\B{M}^-$ respectively. 
} 