\subsubsection{Spectral Datasets and Performance Evaluation}


% \rnote{How well this model can work w/o noise}

% \subsubsection{Performance Evaluation}


\Xpolish{The performances of the models were evaluated through their average classification rates. To ensure a thorough assessment, the model was tested five times independently, each time with a randomly split dataset for 70\% training and 30\% validation. During each assessment, the model generated noisy photon counts, which were then used to train the classifier until the validation rate reached a plateau or declined. The best validation rate achieved during the training process was chosen as the representative performance upper bound for each assessment. The overall performance of the model was determined by averaging the results of all five assessments.}
{The model performances were assessed based on their average classification rates, which were calculated after five independent tests. In each test, the dataset was randomly split into 70\% for training and 30\% for validation. The model generated noisy photon counts during each assessment, and the training continued until the validation rate reached a plateau or declined. The highest validation rate obtained during the training process was considered as the representative upper bound for each assessment. The overall model performance was determined by averaging the results from all five assessments. This comprehensive evaluation ensured the reliability and accuracy of the model performance.}

