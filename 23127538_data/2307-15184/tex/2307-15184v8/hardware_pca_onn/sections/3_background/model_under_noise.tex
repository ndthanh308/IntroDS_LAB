\Xpolish{Two noise models were investigated in this project. The first one is the AGN, or read noise, mainly stemming from thermal vibration of atoms at sensors. In this noise model, the equation \ref{eqn:NoiselessMeasurement} turns into 
\Xequa{
    \noisy{\B{y}} = \B{Mx} + \B{\epsilon} 
},
where $\B{\epsilon} \sim \AGN(\B{0}, \sigma^2\B{I})$ and the $\sigma$ is the standard deviation. The other is the Poisson noise, or shot noise, \Xhide{model for photon-counting systems\rnote{A:shot noise doesn't just exist in photon counting systems} \cite{willet2009CSPoisson} }originating from the statistical nature of photons \cite{boyat2015review} with the following form
\Xequa{
    \noisy{\B{y}} \sim \Poisson(\B{Mx})
}. Due to the constraint \ref{constraint:negativeEntry}, no entries in $\B{M}$ can be negative.}
{This project investigates two noise models. The first one is AGN \Xpolish{or read noise}{related to dark current or read noise from imperfect sensor materials}, which mainly originates from thermal vibrations of atoms at sensors. In this noise model, Eq. \ref{eqn:NoiselessMeasurement} becomes:
\Xequa{
\label{eqn:gaussian_measurement}
\noisy{\B{y}} = \B{Mx} + \B{\epsilon},
}
 where $\B{\epsilon} \sim \AGN(\B{0}, \sigma^2\B{I})$, and $\sigma$ is the standard deviation. The other noise model is photon noise, which arises from the statistical nature of photons \cite{boyat2015review}. 
 % In a majority of previous work, it is approximated by Poisson noise and the measurement equation for this noise model usually obeys a Poisson distribution:
 {
 In most prior research, this is approximated as Poisson noise, and the measurement equations associated with this noise model typically follow a Poisson distribution:
 }
\begin{equation}
\label{eqn:poisson_measurement}
\begin{aligned}
\noisy{\B{y}} &\sim \Poisson(\B{Mx})\\
\Pr(\noisy{{y}}_i &=  k) = \frac{y_i^k \exp(-y_i)}{k!}
\end{aligned} \quad,
\end{equation}
where $y_i = \sum_{j=1}^N M_{ij} x_j$ \cite{willet2009CSPoisson}. It should be noted that the constraint \ref{constraint:negativeEntry} in Eq. \ref{eqn:poisson_measurement} prohibits negative entries in $\B{M}$.
}
% \YLnote{However, it is important to note that photon noise is inherently super-Poissonian though it is often approximated by Poissonian variables only when light-beam degeneracy is small \cite{mandel1959fluctuations}.}
% The reason that the Poisson noise is not good enough for this project is not limited to the fact that this model is not physically accurate by the context.
{
% Though Poisson noise is the best approximation of photon noise in this moment, the inadequacy of the Poisson noise model for this project extends beyond its lack of physical accuracy with regard to photon noise in this context.
% While Poisson noise serves as the most effective approximation of photon noise at this stage, its limitations for this project go beyond mere physical inaccuracies. 
While Poisson noise is the best way to approximate photon noise at this stage, its limitations for this project extend beyond just physical inaccuracies. 
}
% The non-linearity of photon noise introduces complexities in the design and optimization of systems affected, and the approximation by Poissonian variables usually gives rise to the non-existence of gradients. 
The nonlinearity of photon noise adds complexities to the design and optimization of affected systems, and the use of Poisson random functions eliminates gradients, making it difficult to optimize the hardware components in the end-to-end model.
\Xpolish{To address these shortcomings, in this project, we focus on the signal-dependency of photon noise and adopt a quasi-classical model that uses a Gaussian variable with intensity-dependent variance formulated as}{
To address these shortcomings, this project focuses on the signal dependency of photon noise and adopts a quasi-classical model known as MLGAUSS that uses a Gaussian variable with \Xpolish{intensity-dependent variance}{mean-variance equivalence} \cite{selwood2022coded_aperture_imaging}, formulated as
}
\begin{equation} \left\{
    \begin{aligned}
        \noisy{\B{y}} &= \B{Mx} + \B{J},\\
        \B{J} &=  (\B{Mx})^{\circ \frac{1}{2}} \odot \B{\epsilon},\\
        \B{\epsilon} &\sim \mathcal{N}(0, \B{I}),
    \end{aligned} \right. \quad .
\end{equation}
 \Xpolish{
 which has been introduced by Cossairt et al. and Shin et al. \cite{cossairt2012does, shin2013low} and \Xpolish{introduced}{further discussed} in section \ref{sssection:ONN Optimization under Poisson Noise}.
 }{
 \Xhide{
 This model was introduced by Cossairt et al. and Shin et al. \cite{cossairt2012does, shin2013low} and is further discussed in section \ref{sssection:ONN Optimization under Poisson Noise}.
    }
 } 
 It substitutes conventional Poisson approximations whenever a gradient computation is required, yet the term "Poisson" is retained for consistency with other classical models involving no hardware optimization, facilitating straightforward comparisons. 
 While this approach represents an alternative version of approximated photon noise besides Poisson noise, it is utilized exclusively during model training. 
 For model testing, all models employ the Poisson noise model, as gradient computation is no longer necessary.
 \Xhide{Although there is an inconsistency between the noise models used for training and testing, this discrepancy is not significant as long as we adhere to the signal-dependency property, which allows for optimal or near-optimal solutions.
 Supporting this perspective, the simulated and experimental results demonstrate consistent performance improvement across both stages.}

\Xhide{
Two noise models were investigated in this project. The first one is the 
% Additive White Gaussian Noise (AWGN)
AGN
, or read noise,  mainly stemming from thermal vibration of atoms at sensors \iffalse the imperfectness of sensors\fi \cite{boyat2015review} with the following form
% \subsubsection{Gaussian Noise Model}
\Xequa{
    \Xalign{
        \lambda\tilde{\B{y}^+} &\sim \mathcal{N}(\lambda \B{M}^+ \B{x} , \sigma^2 \B{I})\\
        \lambda\tilde{\B{y}^-} &\sim \mathcal{N}(\lambda \B{M}^- \B{x} , \sigma^2 \B{I})
    },
} 
% \subsubsection{Poisson Noise Model}
where the $\sigma$ is the standard deviation. The other is the Poisson noise, or shot noise, \Xhide{model for photon-counting systems\rnote{A:shot noise doesn't just exist in photon counting systems} \cite{willet2009CSPoisson} }originating from the statistical nature of photons \cite{boyat2015review} with the following form
\Xequa{
    \Xalign{
        \lambda\tilde{\B{y}^+} &\sim \mathcal{P}(\lambda \B{M}^+ \B{x})\\
        \lambda\tilde{\B{y}^-} &\sim \mathcal{P}(\lambda \B{M}^- \B{x})
    }.
}   
\bnote{This paragraph is somewhat confusing, and some of the ideas should maybe be before the equations.}  
Although we can obtain a negative number of photons from the difference between the two branches, the Poisson noise cannot be directly applied to these values. Otherwise, it violates the nature of Poisson distributions and the fact that Poisson noise appears in the sensors. On the contrary, noise should be considered in both branches independently.  
\bnote{Mention that this comes from the relevant equation in the dual branch section. }
Therefore, in the AGN model, the measurement $\lambda \tilde{\B{y}} \sim \mathcal{N}(\lambda \B{Mx}, 2\sigma^2 \B{I})$ with the dual-rail trick. But in the Poisson noise model, the measurement $\lambda \tilde{\B{y}} \sim \text{Skellam}(\lambda \B{M}^+ \B{x}, \lambda \B{M}^- \B{x})$ \cite{hwang2007skellam, gan2017skellam}. 
}

% \subsection{Scan Strategies with Different Priors}

% \bnote{We should maybe focus less on the term "single-pixel imaging" that is the way we show our results but maybe it obscures our main point.}

% \Xhide{\ynote{We already talked about single pixel imaging in the beginning. I think the figure could be introduced here along with the math model}
% Single-pixel imaging is a technique that involves both computational and non-computational methods. }