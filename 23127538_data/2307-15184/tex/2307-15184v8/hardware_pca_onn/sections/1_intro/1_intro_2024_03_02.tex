% The very first letter of the paper is a 2 line initial drop letter
% followed by the rest of the first word in caps.
% 
% form to use if the first word consists of a single letter:
% \IEEEPARstart{A}{demo} file is ....
% 
% form to use if you need the single drop letter followed by
% normal text (unknown if ever used by the IEEE):
% \IEEEPARstart{A}{}demo file is ....
% 
% Some journals put the first two words in caps:
% \IEEEPARstart{T}{his demo} file is ....
% 
% Here we have the typical use of a "T" for an initial drop letter
% and "HIS" in caps to complete the first word.

\iffalse
\IEEEPARstart{T}{his} demo file is intended to serve as a ``starter
file'' for ICCP 2020 submissions produced under
\LaTeX~\cite{kopka-latex} using IEEEtran.cls version 1.8b and later.
\fi


\IEEEPARstart{O}{}ptical coding, also known as multiplexing, is a \Xpolish{prevalent}{popular} technology in computational imaging (CI) due to its potential to achieve enhanced signal-to-noise ratio (SNR) and greater light throughput thank point by point measurements\cite{mitra2014can}. The recovery quality of images, however, also depends on the conditioning of the coding matrices \cite{mitra2014can}. Suboptimal coding matrices have the potential to significantly degrade the performance of the vision system during the recovery process, as highlighted by Mitra et al. \cite{mitra2014can}. The optimization of coding schemes is intricately linked to the characteristics of noise, a facet that has not been extensively emphasized previously. Coding schemes, typically formulated under the assumption of signal-independent Additive Gaussian Noise (AGN), offer a mathematically tractable framework. However, AGN, while suitable for modeling certain noise types introduced by measurement devices, inadequately represents Poisson noise \cite{yang2015poisson} arising intrinsically due to the discrete and random nature of light measurements. The optimization behaviors of coding manifest notable distinctions under low-light conditions where Gaussian noise dominates and high-light conditions where Poisson noise prevails \cite{mitra2014can}, suggesting coding and compressed sensing approaches designed for effectiveness under the AGN or noiseless conditions unsuitable under Poisson noise. Existing literature strongly discourages their utilization in such situations \cite{harwit1979hadamard, swift1976hadamard, willet2009CSPoisson, scotte2022photon_noise, vanden2019various}. \YLnote{This issue was first identified by Mitra et al \cite{mitra2014can} but has not received significant attention.} \Xpolish{As Mitra et al. highlighted, a data-driven prior can significantly improve the reconstruction performance of coding schemes under Poisson noise \cite{mitra2014can}, the cost function based on L2 error metric lacks fundamental perceptual information \cite{mitra2014can}. In this project, the primary objectives is to better identify the role of data-driven prior in computer vision tasks on a more meaningful metric, and further fix the limitations when learning that prior in provious models.}{
As emphasized by Mitra et al., leveraging a data-driven prior can markedly enhance the reconstruction performance of coding schemes under Poisson noise \cite{mitra2014can}. However, the cost function based on the L2 error metric lacks essential perceptual information \cite{mitra2014can}. Our study aims to illustrating why AGN is not suitable for optimizing coding under Poisson noise by direct comparison, elucidating the nuanced impact of a data-driven prior on computer vision tasks using more meaningful metrics, and addressing limitations encountered in previous models during the learning \Xhide{process }of such priors and the optimization of coding schemes.
}

\Bingnote{{Clarity and Focus: Ensure the introduction clearly states the problem and the proposed solution. It should succinctly explain why Poisson noise is significant and how it affects imaging systems.}}

% Figure environment removed

The \textbf{single-pixel-camera} \Xpolish{demonstrated}{shown} in Fig. \ref{fig:SinglePixelImaging} is a \Xpolish{typical}{popular} example of \Xpolish{this model}{coding}, and serves as the primary example in this project. \YLnote{It employs a rapid single-pixel detector in conjunction with a Digital Micromirror Device (DMD) to sequentially multiplex a sensing matrix by modulating various patterns on the DMD \cite{mitra2014can}.} Supposing the image representation of the field of view (FOV) $\B{x} \in \IR^{N \times 1}$ consists of $N \in \IZ^+$ pixels and is measured by $m \in \IZ^+$ \Xpolish{projections}{measurements, whose photon counts are denoted as $\B{y} \in \IR^{m \times 1}$}, 
  the single-pixel-imaging process, if ignoring noise, can be expressed as the following equation 
 \begin{equation}
    \label{eqn:NoiselessMeasurement}
    \begin{aligned}
        \B{y} &=  \B{M x} ,
    \end{aligned}
\end{equation}
where \Xhide{$\B{x} \in \IR^{N \times 1}$ is the image representation of the field of view, }$\B{M} \in \IR^{m \times N}$ is a set of sensing masks linearly projecting the \Xpolish{FOV}{$\B{x}$} onto the sensor pixel\Xhide{, and $\B{y} \in \IR^{m \times 1}$ are the corresponding measured {flux} levels or photon counts }\cite{willet2009CSPoisson}. Physically, $\B{M}$ can be implemented by directing or blocking different parts of the incoming light from $\B{x}$ with masks \cite{raskar2009computational} and averaging them on sensors that digitize the detected flux levels or photon counts $\B{y}$. If the sensing matrix $\B{M}$ is full-rank, \Xpolish{this model accurately characterizes conventional single-pixel imaging scenarios}{the \Xpolish{FOV}{$\B{x}$} can be reconstructed by $\B{M}^{-1}\B{y}$}. However, even when $m < N$, \Xhide{the model remains effective through the integration of regularization techniques, specifically }{we can still recover the \Xpolish{FOV}{$\B{x}$} via} compressed sensing. \YLnote{While Hadamard matrices are considered effective for $\B{M}$, their suitability diminishes in the presence of data-dependent noise \cite{mitra2014can, harwit1979hadamard}.} In this work, we will study the performance of this camera for different choices of $\B{M}$ under different noise models. While a compressed sensing system that reconstructs an image fits this paradigm, we will focus on highly compressible problems such as character recognition. In these problems the differences between the common linear Gaussian noise model of the camera and the real Poisson noise model are most apparent.

\Bingnote{Provide a brief background on existing coding schemes and their limitations with Poisson noise. This sets the stage for the importance of your work.}

Our contributions are as listed.
\begin{itemize} 
% \item We analyze the performance of coding techniques for  using toy applications in single-pixel-imaging, compare the behavior of popular approaches for both additive Gaussian and Poisson noise individually, and develop coding strategies that work under Poisson noise. 
\item Similar to the conclusion of Cossairt et al. \cite{cossairt2012does}, Mitra et al. \cite{mitra2014can}, Harwit et. al. \cite{harwit1979hadamard}, Swift et. al. \cite{swift1976hadamard}, and Raginsky et. al. \cite{willet2009CSPoisson}, we find that general prior free  coding \YLnote{using Hadamard and random matrices} provides no benefit over a trivial raster scanning approach under Poisson noise using an identity matrix as $\B{M}$  {\cite{scotte2022photon_noise, wuttig2005optimal, schumann2002SNR_poisson, larson1974hadamard_poisson, streeter2009optical_poisson}}. We find, however, that in applications like compressed sensing and pattern recognition, where the collected data is compressible, similar benefits \Xpolish{to those seen in compressed sensing and computational imaging}{in classification rate which is more perceptually meaningful metric} can be seen if appropriate adjustments to the measurement matrices are made. 
\item We assert that coding can yield substantial performance improvements under Poisson noise \textit{only} when codes used in hardware capture are optimized specifically for downstream vision tasks, which extend beyond naive image reconstruction to include promising classification.
\item \Xpolish{We propose the methodology of selective sensing (SS) which includes coding methods selectively extracting the data-driven priors and can be directly combined with models for downstream tasks.}{We introduce the methodology of Selective Sensing (SS), which encompasses coding techniques specifically designed to extract data-driven priors selectively. These coding methods can be seamlessly integrated with models for downstream tasks, providing a novel approach to efficiently incorporate learned priors into various applications.}
\item We provide a model to \Xpolish{create masks optimized}{optimize $\B{M}$} for Poisson noise using a neural network. \YLnote{We find optimizing measurements significantly enhances single-pixel vision performance, particularly in Poisson noise-dominated signals like visible light, approaching the efficacy of ideal multipixel vision systems.} 
Addressing the limitations identified by Mitra et al. \cite{mitra2014can}, this model presents the following advantages.
    \begin{itemize}
        \item The optimization of coding relies on the gradient descent algorithm, leading to the determination of a globally optimal matrix $\B{M}$ rather than its greedy approximation.
        \item Noise is incorporated during the training process to simulate real-world conditions, deviating from the assumption of a one-dimensional affine noise model.
        \item The training process is accelerated and computationally more efficient due to the binary constraint being addressed even in real experiments. Additionally, the prior is learned from the entirety of the light fields and spectral data, rather than from patches.
        \item The contribution of a data-driven prior to performance enhancement is more accurately discerned when evaluating this model using the classification rate, a metric that holds superior interpretability in terms of visual perceptual meaning compared to the reconstruction error.
    \end{itemize}
\end{itemize}
