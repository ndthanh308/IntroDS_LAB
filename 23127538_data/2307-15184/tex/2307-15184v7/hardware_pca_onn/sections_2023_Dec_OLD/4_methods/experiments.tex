\subsection{Hardware Experiments}

\Xpolish{Besides the simulations, an experiment is also conducted to enhance our conclusion. The DMD we use is the DLP 7000 purchased from Texas Instruments and the PMT is the PMA Series from PicoQuant. This experiment consisted of a hardware-end data capture and a software-end classifier training. \Xhide{The steps are introduced in the following sections.}}
{To further support our conclusions, we conducted an experiment in addition to simulations. The experiment utilized a DMD (digital micromirror device), specifically the DLP 7000 model from Texas Instruments, and a PMT (photomultiplier tube) from PicoQuant, specifically the PMA Series. The experimental setup involved both hardware-end data acquisition and software-end classifier training.}

% \subsubsection{Data preparation}

\Xpolish{Before data acquisition, we transformed and re-calibrated the MNIST images into what the PMT exactly "observed", and trained the classifier with this new data. This step is important if we don't use the hardware-end data for training.}
{To ensure that the classifier was accurately trained using the PMT data, we performed a transformation and re-calibration of the MNIST images to match what the PMT would actually "observe" during data acquisition. This step is crucial when training the classifier without the use of hardware-end data. By implementing this transformation and re-calibration process, we were able to accurately simulate the imaging conditions that the PMT would encounter during actual data acquisition, allowing for more reliable and robust classification results.}

\Xpolish{First, we conducted a Raster scan on a sample from MNIST data and each mask in this scan was exposed for 1 second. The light level had been adjusted and the brightest pixel in this experiment generated about 1000 photon records per second. The captured data vector was then reshaped into a 32 by 32 image, which can be treated as a linearly transformed version of the raw image.}
{First, we performed a Raster scan of a sample from the MNIST dataset, with each mask exposed for 1 second. The light intensity was adjusted to ensure that the brightest pixel generated approximately 1000 photon counts per second. The data captured during this experiment was then reshaped into a 32 by 32 image, which can be regarded as a linearly transformed version of the original raw image.}

% Figure environment removed

\Xpolish{Then, we estimated the linear transformation between the digital data and exprimental data, and then transformed all digital data in the same way so that the new data are similar to the sensor's observation. This transformed data was then input into the code modules used for simulation to train the software-end classifier. For the selective sensing, the masks were also trained in this step.}
{Then, we estimated the linear transformation between the digital and experimental data to ensure that the digital data closely matched the sensor's observations. After applying this transformation to all digital data, we used it to train the software-end classifier through simulation. Additionally, in this step, the masks used for Selective Sensing were also trained.} \YLnote{The masks with decimals use the fraction of a super pixel \cite{spall22hybrid_training} The ONN training is called \textit{in silico} method (Wright et al) \cite{spall22hybrid_training, wright2021backprop}}

\Xpolish{Finally, we uploaded the masks as the pattern sequences onto the DMD to acquire the data $\noisy{\B{y}}$. The testset consisted of ten images and each handwritten number appeared exactly once. The exposure time for each mask was 1 second. Since all photon records preserved the arrival time, we could change the noise level by choosing random intervals during the whole exposure time and counting the photons within it. }
{After generating the mask pattern sequences, we uploaded them onto the DMD to acquire the data, denoted as $\noisy{\B{y}}$. Our test set consisted of ten handwritten images, each of which appeared exactly once during the experiment. To control the noise level in the data, we varied 
% the number of photons recorded within 
lengths of random intervals throughout the 1-second exposure time for each mask. By preserving the arrival times of all photons, we were able to systematically adjust the level of noise in the acquired data.}


% Figure environment removed