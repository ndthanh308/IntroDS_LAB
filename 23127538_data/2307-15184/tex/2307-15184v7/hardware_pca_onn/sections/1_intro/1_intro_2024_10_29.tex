% outline
\iffalse

1. why we need co-design of hardware and software, and and use single pixel camera to demonstrate

2. what is a single pixel camera, what is coding, decoding; a bad coding can degrade the performance

3. a key factor in finding optimal coding is noise/ if no noise, no coding is needed

4. coding for poisson noise is different from coding for AGN

5. -

\fi


\iffalse
\IEEEPARstart{T}{his} demo file is intended to serve as a ``starter
file'' for ICCP 2020 submissions produced under
\LaTeX~\cite{kopka-latex} using IEEEtran.cls version 1.8b and later.
\fi


\IEEEPARstart{W}{}hile we can mass-produce optical components of increasing complexity, most vision hardware still closely mimics the optics of the human eye to create easily interpretable images. 
In this approach, the lens projects an image that is then post-processed to extract patterns.
However, more generalized vision systems can perform alternative measurements that may be more perceptually meaningful than simply producing images. 
Designing such systems requires a collaborative effort to integrate hardware and digital processing algorithms, all while considering the appropriate noise model.
The significance of this approach is particularly illustrated by the single-pixel camera in this context.

The single pixel camera uses the concept of optical coding, or multiplexing, which involves projecting an image onto a mask and collecting the transmitted light with a large sensor pixel.
The system can potentially achieve an enhanced signal-to-noise ratio (SNR) and greater light throughput by coding compared to point-by-point measurements \cite{mitra2014can}.
In this context, a measurement vector $\B{y}$ is formed from a set of measured photon counts, with the corresponding masks vectorized as the rows of the sensing matrix $\B{M}$.
Ignoring the noise, coding can be represented as
 \begin{equation}
    \label{eqn:NoiselessMeasurement}
    \begin{aligned}
        \B{y} &=  \B{M x} ,
    \end{aligned}
\end{equation}
where $\B{x}$ is the signal vector. 
Decoding refers to the process of recovering the initial signal $\B{x}$ from $\B{y}$.
The recovery quality of $\B{x}$
significantly % , however, also 
depends on the conditioning of $\B{M}$ and the light throughput of $\B{M}$~\cite{mitra2014can}.
Suboptimal $\B{M}$ have the potential to significantly degrade the performance of the vision system during the recovery process, as highlighted by Mitra et al. \cite{mitra2014can}. 
%The optimization of coding schemes regarding $\B{M}$ is intricately linked to the characteristics of noise. 

While a greater light throughput, or a lower noise level, can improve the recovery, the improper choice of the noise model can also give rise to an suboptimal coding at specific noise levels. 
Coding design is often performed under the assumption of signal-independent Additive Gaussian Noise (AGN) which arises from imperfect sensors.
With advances in sensor design, even the sensor noise of low-end camera sensors has been reduced to a few photons per pixel.
This reduction is significant enough that non-additive photon noise becomes the primary noise factor in the vast majority of imaging applications \cite{cossairt2012does}. Photon noise follows a super-Poissonian distribution~\cite{mandel1959fluctuations}. For passive cameras operated with incoherent ambient light the noise can usually be approximated as Poissonian.

The transition from AGN to more realistic photon noise fundamentally alters the designs of vision systems.
For instance, although single-pixel cameras with random masks provide performance advantages under AGN, they do not 
outperform %improvements over 
a simple sequential scan when dealing with Poisson noise \cite{harwit1979hadamard, swift1976hadamard, willet2009CSPoisson, scotte2022photon_noise, vanden2019various}.
Moreover, the optimal coding designs under AGN conditions fail to deliver optimal performance under 
Poisson %photon 
noise at the same noise levels.

To address this issue, Mitra et al. \cite{mitra2014can} develop a data-driven prior to design optimized codes for image reconstructions. 
Their approach significantly enhances the performance of coding schemes under Poisson noise by designing measurements specifically targeting sparsity or compressibility in the data \cite{mitra2014can}. 
% It can markedly enhance the reconstruction performance of coding schemes under Poisson noise \cite{mitra2014can} by designing measurements specifically targeting sparsity or compressibility in the data. 
However, image recovery is typically not the ultimate goal for modern vision systems.
Since vision tasks, such as classification, operate in embeddings created from images, their data should be inherently more compressible than their upstream imaging tasks allowing for more effective measurement designs. 
To show this, we implement a simple end-to-end vision system that can be optimized under different noise assumptions. 
We illustrate why AGN is not suitable for optimizing coding under \Xpolish{Poisson}{photon} noise and show that a coding scheme optimized for a specific vision task and for the correct noise model can perform close to the optimal non-coding camera even when \Xhide{only }photon noise is present.

Our contributions are as listed.
\begin{itemize} 
\item {We introduce the methodology of Selective Sensing (SS), which encompasses coding techniques specifically designed to extract data-driven priors selectively for \YLnote{various downstream tasks.} \Xhide{\YLnote{These coding methods can be seamlessly integrated with models for downstream tasks, providing a novel approach to efficiently incorporate learned priors into various applications.\checkednote{\textbf{sales pitch?}}}}}
\item {
We provide an end-to-end vision model using classification as the performance metric, which is more perceptually meaningful than mean-squared-error (MSE), to optimize $\B{M}$.
}
\item {
We propose a hybrid optimization method for non-differentiable Poisson noise that employs AGN with signal-dependent variance and the reparameterization trick to estimate gradients during training, while switching to Poisson noise in the testing stage for model assessment.
}
\end{itemize}

Our model is extendable to more general camera optics and different vision tasks in future work.