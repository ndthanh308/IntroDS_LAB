% \bnote{Don't use the paper start unless it is the large first letter unless it the paper start.... }Multiplexing \bnote{multiplexing has not been mentioned yet. stick with either coding or multiplexing.} %is a method that 

\bnote{Few projects optimize masks for Poisson noise}
\bnote{list how they find the masks for Poisson}


In single-pixel imaging, coding allows the capture of a two-dimensional image with a single-pixel sensor. \bnote{\checkednote{ not constrained to single independent noise just probed for that case.}} Hadamard matrices are considered the optimal coding scheme for multiplexing \cite{harwit1979hadamard, cossairt2012does, wuttig2005optimal} in systems with only additive gaussian noise.
\Xhide{ \bnote{Combine these two sentences?} However, under Poisson noise a trivial raster scan, where one pixel of the mask opens at a time provides the best performance even though it transmits only a small fraction of the light \cite{wuttig2005optimal, harwit1979hadamard, cossairt2012does}.} \checkednote{Previous studies have shown that there is an upper limit to the performance gain of Hadamard multiplexing when Poisson noise is present.} However, incorporating task-specific priors can be beneficial for reconstructions \cite{cossairt2012does, harwit1979hadamard}. \ynote{do these papers really show this? That seems like it is our contribution} \Xhide{\bnote{We are citing mostly just the cossairt2012 paper for all these claims, can we add more citations to the different claims?}. \ynote{Cossairt takes these results from the hadamard transform optics book. We should cite that.}}

% but it is still unclear how to surpass this limit on photon counting systems where Poisson noise is predominant. 



Priors are usually used to enhance the image quality \cite{cossairt2012does, nayer2009prior, Levin2007prior} but the extent of this improvement depends on prior types and masks\Xhide{\ynote{what is imaging coding??}}\cite{cossairt2012does}. Compressed sensing, which utilizes sparsity priors, is a proposed method for improving the reconstruction performance \Xhide{of photon counting systems~\rnote{Is this true?? This seems to be contradicting beccas paper.}}\cite{duarte2008CS}. Nevertheless, the performance attainable under Poisson noise is bounded, significantly lower than under additive gaussina noise, and is heavily influenced by the number of measurements taken and thereby the resolution of the reconstructed image\cite{willet2009CSPoisson}. As such, compressed sensing is a far more challenging task when the measured data are Poisson distributed \cite{willet2009CSPoisson}. Coding strategies and theoretical frameworks developed for additive Gaussian noise are not necessarily usable under Poisson noise.

\Xpolish{While image reconstruction is a common task performed by an optical imaging system, it is often an intermediate step for performing downstream computer vision tasks such as classification or segmentation. Extracting features from reconstructed signals is sub-optimal compared to direct linear feature extraction during measurements \cite{neifeld2003FSI}, and the extracted features hardly improve the reconstruction \cite{neifeld2003dual}. Notably, the Mean Squared Error (MSE) is a common metric for evaluating image reconstruction but may not necessarily reflect visual image quality \cite{cossairt2012does, wang2004MSEmetric}. \Xhide{\ynote}{In many cases the most meaningful way to quantify the quality of the collected imaging data is to apply it to a downstream vision task.} Therefore, we explore 
% alternative metrics that better capture the perceptual features and 
optimal masks for feature specific imaging \cite{neifeld2003dual}. \Xhide{\rnote{FSI doesn't work under Poisson \cite{neifeld2003dual}}}
One inspiring solution is the task-specific imaging via Optical Neural Networks \cite{nature2022ONN, spall22hybrid_training}. Wang et al. \cite{nature2022ONN}\Xhide{\ynote{citation}} successfully implemented a neural network model for handwritten number classification on an optical device with limited photon budget, demonstrating the potential for AI-assisted optimization of coding schemes in CI. }
{While image reconstruction is a common task performed by an optical imaging system, it is often an intermediate step for performing downstream computer vision tasks such as classification or segmentation \cite{davenport2007smashed}. Extracting features from reconstructed signals is sub-optimal compared to direct linear feature extraction during measurements \cite{neifeld2003FSI}, and the extracted features hardly improve the reconstruction \cite{neifeld2003dual}. Notably, the Mean Squared Error (MSE) is a common metric for evaluating image reconstruction but may not necessarily reflect visual image quality \cite{cossairt2012does, wang2004MSEmetric}. Therefore, we explore 
% alternative metrics that better capture the perceptual features and 
optimal masks for feature specific imaging \cite{neifeld2003dual}. The principal component analysis (PCA) is a classical method for multispectral data correspondence analysis and classification \cite{rodarmel2002PCA, carr1999PCA, jensen1996PCA, gonzales1987PCA, schowengerdt2006PCA}. Another inspiring solution is the task-specific imaging via end-to-end optimization such as Optical Neural Networks \cite{nature2022ONN, spall22hybrid_training}. 
\YLnote{This idea was usually implemented without considering Poisson noise \cite{hinojosa2021learning, dun2020learned, metzler2020deep, chang2019deep, onzon2021neural, spall22hybrid_training} or training under Poisson noise \cite{tseng2021differentiable, diamond2021dirty, rego2022deep}. Rego et. al froze the sensing matrix as a pinhole without optimizing it.}
Wang et al. \cite{nature2022ONN} successfully implemented a neural network model for handwritten number classification on an optical device with limited photon budget, demonstrating the potential for AI-assisted optimization of coding schemes in CI. }
% Wang et al. reported a successful implementation of a neural networks model on optical device for handwritten numbers classification with limited photon budget \cite{nature2022ONN}, which gives a inspiring example in searching task-specific optimal coding scheme for the CI with the assistance of AI .
\YLnote{However, the Poisson noise was considered only in model testing where the most robust model was picked from a set of hyper-parameter combinations \cite{nature2022ONN}. } 


% \rnote{Our contribution}





% Raginsky et al proved the 

%see
%http://vigir.missouri.edu/~gdesouza/Research/Conference_CDs/IEEE_CVPR_2007/data/papers/0194.pdf