\Xhide{Supposing the image of the field of view consists of $N \in \IZ^+$ pixels and is measured by $m \in \IZ^+$ masks\Xhide{\bnote{We have not introduced the language mask yet.}}, 
  the single-pixel-sensing process, if ignoring noise, can be expressed as the following equation 
 \begin{equation}
    \label{eqn:NoiselessMeasurement}
    \begin{aligned}
        \B{y} &=  \B{M x}
    \end{aligned}
\end{equation}
, where $\B{x} \in \IR^{N \times 1}$ is the image representation of the field of view, $\B{M} \in \IR^{m \times N}$ is the sensing masks\Xhide{or in other words, weighing design (WD) \cite{harwit1979hadamard} \bnote{not sure if language from 1979 is useful. I have not heard of weighing design},} linearly projecting the field of view, and $\B{y} \in \IR^{m \times 1}$ is the corresponding photon counts \cite{willet2009CSPoisson}.}

\Xpolish{
As a typical CI technique, the mathematical foundation of single pixel imaging follows the equation \ref{eqn:NoiselessMeasurement}. However, the different choices of the sensing matrix $\B{M}$ give rise to different coding types (strategies) introduced later. However, $\B{M}$ yields to two constraints when physically implemented. 
}{
As a representative technique in the CI, the mathematical underpinning of single-pixel camera is encapsulated by Eq. \ref{eqn:NoiselessMeasurement}. The diverse selections of the sensing matrix $\B{M}$ lead to distinct coding strategies, as elaborated upon subsequently. \Xhide{Nonetheless, the physical realization of $\B{M}$ introduces two inherent constraints.}
}
\YLnote{Not a good place for this}
\begin{enumerate}
    \item\label{constraint:photonNumber} \textbf{Flux-preserving} \cite{willet2009CSPoisson}\Xhide{\bnote{Y: Should I use the same term as Rebecca?}}. The single-pixel camera model involves the allocation of available photons among masks, as discussed in \cite{neifeld2003dual}. It is important to ensure that the mask basis $\B{M}$ does not produce additional photons through improper entries \cite{neifeld2003dual}. Mathematically, $\sup \sum_{i=1}^m M_{ij} = 1, \forall j \in \{1,2,\dots,N\}$ \cite{neifeld2003dual}.\Xhide{\bnote{Include what an improper entry is!} \bnote{A: this is not something we need to mention here. its obvious that the measurements have to be normalized correctly}}
    \item\label{constraint:negativeEntry} \textbf{Positivity-preserving} \cite{willet2009CSPoisson}. It is not possible to physically implement negative values for the masks $\B{M}$, as demonstrated in \cite{neifeld2003dual, willet2009CSPoisson}. 
\end{enumerate}

\AVnote{the rest of this section can be removed. the figure should stay}
\YLnote{The sensing matrix $\B{M}$ can be generated using either computational or non-computational imaging methods (i.e. Raster scan). The key difference between the two is that computational imaging involves additional steps of optical coding and computational decoding to capture and process the image \cite{cossairt2012does}, while non-computational method, i.e. Raster scan, involves measuring each pixel sequentially \cite{duarte2008CS}. In the case of Raster scan, $\B{M} = \B{I}$, which is an identity matrix. In contrast, computational methods measure a combination of pixels simultaneously and can be further classified based on the type of masks used. For example, Hadamard matrices composed of $\pm 1$ have been shown to be the optimal coding strategy for reconstruction under the AGN \cite{harwit1979hadamard}. A useful feature of computational methods is their compressibility. If the signal is sparse in the subspace of $\B{M}$, it is likely that only a few vectors of $\B{M}$ are needed.}

% Figure environment removed

\YLnote{It is better to intro this at the end of this section or \textbf{move it to methods}}
\Xpolish{
Another problem to be investigated is the noise. To compare measurements, we assume that each measurement is performed using the same amount of light at the camera aperture and over the same total time.
}{
Another critical issue for investigation is noise level. In order to establish meaningful comparisons between measurements, it is assumed that each measurement is conducted under consistent conditions, including uniform light intensity at the camera aperture and an equal duration of exposure.
}