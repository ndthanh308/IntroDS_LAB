Optical coding is widely used in computational imaging systems and is a good approach for designing vision systems. However, most coding methods are developed assuming additive Gaussian noise, while modern optical imaging systems are mainly affected by Poisson noise. Previous studies have highlighted the significant differences between these noise models and proposed coding optimization algorithms for image recovery under Poisson noise. They concluded that the compressibility arising from data variance is crucial for image recovery under Poisson noise. This makes a strong case for the design of end-to-end vision systems that avoid image formation, since the data-driven vision tasks, typically downstream of imaging, is more compressible than imaging itself. In this project, we propose a coding strategy by jointly optimizing an entire vision system, including measurement and inference, using the classification accuracy as a metric. We demonstrate the importance of incorporating Poisson noise in optimizing even the simplest vision systems and propose an approach to achieve it.

% Abstract Single-pixel computer vision under poisson noise

 % Optical coding is widely used in the design of computational imaging systems. While most coding approaches are developed with additive Gaussian noise models, most real optical imaging systems noise is dominated by Poisson noise due to light quantization. It has been observed in previous studies that code performance varies significantly between these two noise models. Therefore the design of an imaging approach, including the derivation of suitable codes, lenses, or cameras should take into account Poisson noise. In this work, we use the single pixel camera as a toy model to derive \Xpolish{capture strategies}{coding schemes} that are feasible in Poisson noise limited environments. While our work has implications for all optical vision systems, we focus on a system designed for highly compressive tasks like character recognition where the effect of Poisson noise is most apparent. We show that in such a case, our optimized capture approach can approach the performance of an ideal full resolution camera.


\iffalse
 \Xpolish{Optical coding is widely used in the design of computational imaging systems. While most coding approaches are developed with additive Gaussian noise models, most real optical imaging systems noise is dominated by Poisson noise. Previous studies identified the significant difference between these two noise models and proposed coding optimization algorithms for image recovery under Poisson noise, but this algorithms suffer from limitation of lacking perceptual information, not globally optimal and low computation efficiency. While our work has implications for all optical vision systems addressing these limitations, we focus on a system designed for highly compressive tasks like character recognition where the effect of Poisson noise is most apparent. We show that in such a case, our optimized capture approach can approach the performance of an ideal full resolution camera.
 }{
 Optical coding is widely used in computational imaging system design. Despite coding methods often being tailored to additive Gaussian noise models, real optical imaging systems predominantly experience Poisson noise. Existing studies recognize the substantial disparity between these noise models and propose optimization algorithms for image recovery under Poisson noise. However, these algorithms exhibit limitations, including a lack of perceptual information, suboptimality, and low computational efficiency. Our work, with broad implications for optical vision systems, focuses on addressing these limitations, particularly in a highly compressive task such as character recognition. We demonstrate that our optimized capture approach, in such scenarios, can approximate the performance of an ideal full-resolution camera.
 }
\fi
 % keep original one


% Optical coding is widely used in computational imaging systems and is a good model problem for the design of vision systems as a whole. However, most coding methods are developed assuming additive Gaussian noise, while modern optical imaging systems are mainly affected by Poisson noise. Previous studies have highlighted the significant differences between these noise models and proposed coding optimization algorithms for image recovery under Poisson noise. They concluded that the compressibility arising from data variance is crucial for image recovery under Poisson noise. This makes a strong case for the design of end to end vision systems that avoid image formation. Since the data used vision tasks, typically downstream of imaging, is more compressible than imaging itself. In this project, we propose a coding strategy by jointly optimizing an entire vision system, including measurement and inference, using the classification rate as a metric. We demonstrate the importance of modeling Poisson noise in even the simplest vision systems and propose an approach to achieve this.

%  Maybe we could say, while there has been much work on algorithms that reconstruct images from measurements corrupted by poisson noise, much less work focuses on the joint optimization of the actual measurement masks, and even fewer works consider Poisson noise in the process. Another issue is the lack of an effective metric to quantify image quality or its usefulness for a downstream vision task.

% In this work we address this problem by studying the optimization of n entire vision system that includes measurement and inference. We show the importance of modeling Poisson noise in the design of even the most simple vision systems and provide an approach for doing so.

% Mar 8, 2024 2:12 PM