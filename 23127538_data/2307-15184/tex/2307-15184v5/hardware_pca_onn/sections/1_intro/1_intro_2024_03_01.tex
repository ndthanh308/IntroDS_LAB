% \IEEEPARstart{T}{}

\Xpolish{Optical coding, or multiplexing, is a common technology in computational imaging (CI) as it is believed to be able to acquire a better SNR with greater light throughput \cite{mitra2014can}. However, the recovery of the image also depends on the conditioning of coding matrices as a non-optimal coding matrix can even degrade the vision system's recovery performance \cite{mitra2014can}.}{
\IEEEPARstart{O}{}ptical coding, also known as multiplexing, is a prevalent technology in computational imaging (CI) due to its potential to achieve enhanced signal-to-noise ratio (SNR) and greater light throughput \cite{mitra2014can}. The successful recovery of images, however, also depends on the conditioning of the coding matrices. Suboptimal coding matrices have the potential to significantly degrade the performance of the vision system during the recovery process, as highlighted by Mitra et al. \cite{mitra2014can}. The optimization of coding schemes is intricately linked to the characteristics of noise, a facet that has not been extensively emphasized previously. Coding schemes, typically formulated under the assumption of signal-independent Additive Gaussian Noise (AGN), offer a mathematically tractable framework. However, AGN, while suitable for modeling certain noise types introduced by measurement devices, inadequately represents Poisson noise \cite{yang2015poisson} arising intrinsically due to the discrete and random nature of light measurements. The optimization behaviors of coding manifest notable distinctions under low-light conditions where Gaussian noise dominates and high-light conditions where Poisson noise prevails \cite{mitra2014can}, suggesting coding and compressed sensing approaches, designed for effectiveness under the AGN or noiseless conditions unsuitable. The literature discourages their use in such scenarios \cite{harwit1979hadamard, swift1976hadamard, willet2009CSPoisson, scotte2022photon_noise, vanden2019various}. In this project, the primary objective is to optimize coding schemes, with a particular emphasis on selecting the appropriate noise model.
}



The \textbf{single-pixel-camera} \Xpolish{demonstrated}{shown} in Fig. \ref{fig:SinglePixelImaging} is a \Xpolish{typical}{popular} example of \Xpolish{this model}{coding}. Supposing the image representation of the field of view (FOV) $\B{x} \in \IR^{N \times 1}$ consists of $N \in \IZ^+$ pixels and is measured by $m \in \IZ^+$ \Xpolish{projections}{measurements, whose photon counts are denoted as $\B{y} \in \IR^{m \times 1}$}, 
  the single-pixel-imaging process, if ignoring noise, can be expressed as the following equation 
 \begin{equation}
    \label{eqn:NoiselessMeasurement}
    \begin{aligned}
        \B{y} &=  \B{M x} ,
    \end{aligned}
\end{equation}
where \Xhide{$\B{x} \in \IR^{N \times 1}$ is the image representation of the field of view, }$\B{M} \in \IR^{m \times N}$ is a set of sensing masks linearly projecting the field of view onto the sensor pixel\Xhide{, and $\B{y} \in \IR^{m \times 1}$ are the corresponding measured {flux} levels or photon counts }\cite{willet2009CSPoisson}. Physically, $\B{M}$ can be implemented by directing or blocking different parts of the incoming light from $\B{x}$ with masks \cite{raskar2009computational} and averaging them on sensors that digitize the detected flux levels or photon counts $\B{y}$. If the sensing matrix $\B{M}$ is full-rank, \Xpolish{this model accurately characterizes conventional single-pixel imaging scenarios}{the \Xpolish{FOV}{$\B{x}$} can be reconstructed by $\B{M}^{-1}\B{y}$}. However, even when $m < N$, \Xhide{the model remains effective through the integration of regularization techniques, specifically }{we can still recover the \Xpolish{FOV}{$\B{x}$} via} compressed sensing.