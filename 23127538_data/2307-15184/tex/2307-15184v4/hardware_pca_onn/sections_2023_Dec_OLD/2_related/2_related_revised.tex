\textbf{Single Pixel Imaging}. In single-pixel imaging, coding allows the capture of a two-dimensional image with a single-pixel sensor \cite{harwit1979hadamard}. Hadamard matrices are considered the optimal coding scheme for multiplexing \cite{harwit1979hadamard, cossairt2012does, wuttig2005optimal} in systems with only additive gaussian noise.

\textbf{Imaging Priors}. Priors are usually used to enhance the image quality \cite{cossairt2012does, nayer2009prior, Levin2007prior} but the extent of this improvement depends on prior types and masks \cite{cossairt2012does}. 

\textbf{Compressed Sensing}. Compressed sensing, which utilizes sparsity priors, is a proposed method for improving the reconstruction performance \Xhide{of photon counting systems~\rnote{Is this true?? This seems to be contradicting beccas paper.}}\cite{duarte2008CS}. Nevertheless, the performance attainable under Poisson noise is bounded, significantly lower than under additive gaussina noise, and is heavily influenced by the number of measurements taken and thereby the resolution of the reconstructed image\cite{willet2009CSPoisson}. As such, compressed sensing is a far more challenging task when the measured data are Poisson distributed \cite{willet2009CSPoisson}. Coding strategies and theoretical frameworks developed for additive Gaussian noise are not necessarily usable under Poisson noise.

\textbf{Non-reconstruction classification}. While image reconstruction is a common task performed by an optical imaging system, it is often an intermediate step for performing downstream computer vision tasks such as classification or segmentation \cite{davenport2007smashed}. Extracting features from reconstructed signals is sub-optimal compared to direct linear feature extraction during measurements \cite{neifeld2003FSI}, and the extracted features hardly improve the reconstruction \cite{neifeld2003dual}. Notably, the Mean Squared Error (MSE) is a common metric for evaluating image reconstruction but may not necessarily reflect visual image quality \cite{cossairt2012does, wang2004MSEmetric}.

% \textbf{Metric for imaging}. \YLnote{MSE is not a good metric}

\textbf{Principal Component Analysis}. The principal component analysis (PCA) aims to extract significant features in another orthogonal space \cite{abdi2010PCA}. It is a classical method for multispectral data correspondence analysis and classification \cite{rodarmel2002PCA, carr1999PCA, jensen1996PCA, gonzales1987PCA, schowengerdt2006PCA}.

\textbf{End-to-End optimization}. This method refers to training hardware and software networks for image processing pipelines \cite{diamond2021dirty, zhang2021deep}. \Xhide{One typical archetecture is Optical Neural Networks \cite{nature2022ONN, spall22hybrid_training, caulfield1989ONN}.} \YLnote{In many previous projects, this idea was usually implemented without considering Poisson noise \cite{hinojosa2021learning, dun2020learned, metzler2020deep, chang2019deep, onzon2021neural, spall22hybrid_training} or without optimizing masks under Poisson noise \cite{tseng2021differentiable, diamond2021dirty, rego2022deep, duarte2008CS, nature2022ONN}. Rego et al. froze the sensing matrix as a pinhole without optimizing it \cite{rego2022deep}}
Wang et al. \cite{nature2022ONN} successfully implemented a neural network model for handwritten number classification on an optical device with limited photon budget, demonstrating the potential for AI-assisted optimization of coding schemes in CI. 
\YLnote{However, the Poisson noise was considered only in model testing where the most robust model was picked from a set of hyper-parameter combinations \cite{nature2022ONN}. } 