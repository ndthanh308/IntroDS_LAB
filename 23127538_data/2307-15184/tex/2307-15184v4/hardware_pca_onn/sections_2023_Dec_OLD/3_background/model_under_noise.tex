\Xpolish{Two noise models were investigated in this project. The first one is the AGN, or read noise, mainly stemming from thermal vibration of atoms at sensors. In this noise model, the equation \ref{eqn:NoiselessMeasurement} turns into 
\Xequa{
    \noisy{\B{y}} = \B{Mx} + \B{\epsilon} 
},
where $\B{\epsilon} \sim \AGN(\B{0}, \sigma^2\B{I})$ and the $\sigma$ is the standard deviation. The other is the Poisson noise, or shot noise, \Xhide{model for photon-counting systems\rnote{A:shot noise doesn't just exist in photon counting systems} \cite{willet2009CSPoisson} }originating from the statistical nature of photons \cite{boyat2015review} with the following form
\Xequa{
    \noisy{\B{y}} \sim \Poisson(\B{Mx})
}. Due to the constraint \ref{constraint:negativeEntry}, no entries in $\B{M}$ can be negative.}
{This project investigates two noise models. The first one is the additive Gaussian noise (AGN) or read noise, which mainly originates from thermal vibrations of atoms at sensors. In this noise model, Eq. \ref{eqn:NoiselessMeasurement} becomes:
\Xequa{
\noisy{\B{y}} = \B{Mx} + \B{\epsilon},
}
 where $\B{\epsilon} \sim \AGN(\B{0}, \sigma^2\B{I})$, and $\sigma$ is the standard deviation. The other noise model is the Poisson noise or shot noise, which arises from the statistical nature of photons \cite{boyat2015review}. The measurement equation for this noise model is:
\Xequa{
\noisy{\B{y}} \sim \Poisson(\B{Mx}).
\label{eqn:poisson_measurement}
} It should be noted that the constraint \ref{constraint:negativeEntry} in Eq. \ref{eqn:poisson_measurement} prohibits negative entries in $\B{M}$. 
}

\Xhide{
Two noise models were investigated in this project. The first one is the 
% Additive White Gaussian Noise (AWGN)
AGN
, or read noise,  mainly stemming from thermal vibration of atoms at sensors \iffalse the imperfectness of sensors\fi \cite{boyat2015review} with the following form
% \subsubsection{Gaussian Noise Model}
\Xequa{
    \Xalign{
        \lambda\tilde{\B{y}^+} &\sim \mathcal{N}(\lambda \B{M}^+ \B{x} , \sigma^2 \B{I})\\
        \lambda\tilde{\B{y}^-} &\sim \mathcal{N}(\lambda \B{M}^- \B{x} , \sigma^2 \B{I})
    },
} 
% \subsubsection{Poisson Noise Model}
where the $\sigma$ is the standard deviation. The other is the Poisson noise, or shot noise, \Xhide{model for photon-counting systems\rnote{A:shot noise doesn't just exist in photon counting systems} \cite{willet2009CSPoisson} }originating from the statistical nature of photons \cite{boyat2015review} with the following form
\Xequa{
    \Xalign{
        \lambda\tilde{\B{y}^+} &\sim \mathcal{P}(\lambda \B{M}^+ \B{x})\\
        \lambda\tilde{\B{y}^-} &\sim \mathcal{P}(\lambda \B{M}^- \B{x})
    }.
}   
\bnote{This paragraph is somewhat confusing, and some of the ideas should maybe be before the equations.}  
Although we can obtain a negative number of photons from the difference between the two branches, the Poisson noise cannot be directly applied to these values. Otherwise, it violates the nature of Poisson distributions and the fact that Poisson noise appears in the sensors. On the contrary, noise should be considered in both branches independently.  
\bnote{Mention that this comes from the relevant equation in the dual branch section. }
Therefore, in the AGN model, the measurement $\lambda \tilde{\B{y}} \sim \mathcal{N}(\lambda \B{Mx}, 2\sigma^2 \B{I})$ with the dual-rail trick. But in the Poisson noise model, the measurement $\lambda \tilde{\B{y}} \sim \text{Skellam}(\lambda \B{M}^+ \B{x}, \lambda \B{M}^- \B{x})$ \cite{hwang2007skellam, gan2017skellam}. 
}

% \subsection{Scan Strategies with Different Priors}

% \bnote{We should maybe focus less on the term "single-pixel imaging" that is the way we show our results but maybe it obscures our main point.}

% \Xhide{\ynote{We already talked about single pixel imaging in the beginning. I think the figure could be introduced here along with the math model}
% Single-pixel imaging is a technique that involves both computational and non-computational methods. }