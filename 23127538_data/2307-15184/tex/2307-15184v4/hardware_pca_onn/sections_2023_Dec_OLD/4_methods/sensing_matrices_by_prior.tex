\subsection{Sensing Matrices by Priors}

\Xpolish{In the last section, we mentioned the masks $\B{M}$ can be generated by different coding strategies. 
In order to effectively determine the appropriate coding strategies for a given scenario, it is necessary to evaluate the various strategies based on their priors. Image priors, which can vary in type, can enhance the robustness of models in the presence of noise and improve overall image quality \cite{cossairt2012does, levin2007coded}. It should be noted that all of the strategies discussed in this paper possess three distinct prior levels. }
{In the preceding section, we discussed the generation of masks $\B{M}$ using different coding strategies. Evaluating the efficacy of these strategies for a given scenario requires a systematic assessment of their respective priors. Image priors come in varying types and can enhance model robustness in noisy environments and improve overall image quality \cite{cossairt2012does, levin2007coded}. It is important to note that all the strategies outlined in this paper possess three distinct prior levels.}
% \bnote{suggest bolding the priors}
\begin{itemize}
    \item \textbf{Null prior} (data is not compressible or needs to be reconstructed without priors). \Xpolish{Coding strategies in this category require no information about the data. They usually generate full rank matrices as $\B{M}$. In this project, we chose Raster Scan (RS) and Hadamard Basis (HB) for performance evaluation. $\B{M}$ for RS is an identity matrix while it is a Hadamard matrix for HB. On top of these two stratgies, we also include Impulse Imaging (II) which is the optimal measurement that can only be achieved with a sensor array. In our toy example this would be achieved by a camera sensor that captures the image so no coding is necessary. Mathematically, the sensing matrix $\B{M}$ of II is also an identity matrix, but each mask in II has $N$ times longer exposure time compared with RS.}
    {Coding strategies in this category require no information about the data. They usually generate full rank matrices as $\B{M}$. In this project, we chose Raster Scan (RS), Binary Random Basis (BR), and Hadamard Basis (HB) for performance evaluation. $\B{M}$ for RS is an identity matrix while it is a Hadamard matrix for HB. BR employs random values for each pixel in the mask, i.e  $\B{M} \in \{0,1\}^{N\times N}$. On top of these strategies, we also include Impulse Imaging (II) which works as a golden standard in this project. II is an ideal case assuming it can capture the whole data spectrum simultaneously. Mathematically, the sensing matrix $\B{M}$ of II is also an identity matrix, but each mask in II has $N$ times longer exposure time compared with RS.}
    \item \textbf{Reconstruction prior} (data is moderately compressible). \Xpolish{Coding strategies in this category are based on the fact that noise has greater influence on the high-frequency components in any images. We used Low-Frequency-Hadamard Basis which only keeps the low frequency vectors in a Hadamard matrix to study this prior. In this paper, it is labeled as Truncated Hadamard (TH).}
    {One common strategy in coding for images is to take advantage of the fact that noise tends to have a greater impact on high-frequency components. To investigate this phenomenon, we utilized a Low-Frequency-Hadamard Basis approach, in which only the low-frequency vectors of a Hadamard matrix are retained. This approach is referred to as Truncated Hadamard (TH) in this paper.}
    \item \textbf{Task-Specific prior or Selective Sensing prior} (data is extremely compressible). \Xpolish{Coding strategies requires much information from the data and tasks, and their corresponding sensing matrices $\B{M}$ vary from task to task. One strategy in this category is called Hardware PCA (Principal Components Analysis). It computes the most principal components in the training data, and constructs the masks with these components. Another interesting strategy is Selective Sensing (SS) whose sensing matrix $\B{M}$ is not pre-defined nor fixed, but is optimized within a Neural Networks model during the training. This optimization method is called Optical Neural Networks (ONN) in this project.}
    {Coding strategies are an essential component of signal processing that involve gathering task-specific information from data. In order to design these strategies, sensing matrices $\B{M}$ are developed, which can vary depending on the task at hand. One such approach is the Hardware Principal Components Analysis (PCA) technique, which identifies the most principal components in the training data and uses them to construct masks. Another strategy\Xhide{ is the Selective Sensing (SS), which} uses a sensing matrix $\B{M}$ that is not pre-defined or fixed, but rather is optimized using a Neural Networks model during training. In this project, this optimization method is referred to as Optical Neural Networks (ONN).}
\end{itemize}

\Xpolish{In this study, we examined the impact of various priors on a single-pixel imaging system in terms of\Xhide{ both \textbf{reconstruction}} and classification. Firstly, we explained our the methods for imeplementing the constraints in the single-pixel imaging model. Then, we simulated the effect of different prior types on classification performance using the MNIST dataset. Subsequently, we tested the feasibility of these strategies by experiments. \Xhide{Specifically, we first conducted a theoretical analysis to compare the reconstruction capabilities of the HB and RS in the presence of noise. Subsequently, we simulated the effect of different prior types on classification performance using the MNIST dataset. \bnote{is our theoretical analysis novel?} \rnote{I thought it was novel before read the hadamard book}}}
{In this study, we examine the impact of various priors on a single-pixel imaging system in terms of classification. Firstly, we explain our methods for implementing constraints in the single-pixel imaging model. Then, we simulate the effect of different prior types on classification performance using the MNIST dataset. Finally, we test the feasibility of these strategies through experiments.}



\Xhide{\bnote{The first and second sentences here are somewhat confusing due to the "Proposed model" language. We have not proposed a model yet, so unclear if it is our model or someone else's.} \rnote{This model is proposed by Rebecca, but I believe there is some error on the negative entries} In equation \ref{eqn:NoiselessMeasurement}, the parameter regarding the number of photons is an integral part of the mathematical representation of $\B{x}$. To further examine the mask basis under varied light levels, we propose a revised photon-counting model that decouples $\B{M}$ from the number of photons. In the following sections, we detail the development of this new model and provide insights into its practical implementation to avoid violating these constraints.} 