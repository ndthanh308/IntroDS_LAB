\iffalse
\IEEEPARstart{T}{his} demo file is intended to serve as a ``starter
file'' for ICCP 2020 submissions produced under
\LaTeX~\cite{kopka-latex} using IEEEtran.cls version 1.8b and later.
\fi


\IEEEPARstart{O}{}ptical coding, also known as multiplexing, is a popular technology in computational imaging (CI) due to its potential to achieve enhanced signal-to-noise ratio (SNR) and greater light throughput than point by point measurements\cite{mitra2014can}. If ignoring the noise, coding is represented as 
 \begin{equation}
    \label{eqn:NoiselessMeasurement}
    \begin{aligned}
        \B{y} &=  \B{M x} ,
    \end{aligned}
\end{equation}
where $\B{x}$ is the signal vector, $\B{M}$ is the sensing matrix, and $\B{y}$ is the measured vector. Decoding refers to the process of recovering the initial signal $\B{x}$ from $\B{y}$. The recovery quality of $\B{x}$, however, also depends on the conditioning of the $\B{M}$ besides the light throughput \cite{mitra2014can}. Suboptimal $\B{M}$ have the potential to significantly degrade the performance of the vision system during the recovery process, as highlighted by Mitra et al. \cite{mitra2014can}. The optimization of coding schemes regarding $\B{M}$ is intricately linked to the characteristics of noise. Coding schemes, typically formulated under the assumption of signal-independent Additive Gaussian Noise (AGN), offer a mathematically tractable framework. However, AGN, while suitable for modeling certain noise types introduced by measurement devices, inadequately represents Poisson noise \cite{yang2015poisson} arising intrinsically due to the discrete and random nature of photon measurements. 
{Past works usually distinguish between low-light regimes, where AGN introduced by imperfect sensors dominates, and bright-light regimes, where the physical photon noise dominates \cite{mitra2014can}. With advances in sensor design, even the sensor noise of low-end camera sensors has been reduced to a few photons per pixel. This reduction is significant enough that photon noise becomes the primary noise factor in the vast majority of imaging applications.}
It has been shown that in that case without the use of priors for regularization, coding does not provide an improvement over a simple sequential scan \cite{harwit1979hadamard, swift1976hadamard, willet2009CSPoisson, scotte2022photon_noise, vanden2019various}. 
{
Mitra et. al. \cite{mitra2014can} develop a data driven prior to design optimized codes for image reconstructions that can markedly enhance the reconstruction performance of coding schemes under Poisson noise \cite{mitra2014can}. \YLnote{An important conclusion is that if the signal exhibits sparsity and thereby  compressibility, signal optimized coding can improve recovery performance under Poisson noise and the amount of improvement is directly related to the comprehensibility of the measured data. However, recovery is typically not the ultimate goal for modern vision systems.}
}{
\YLnote{Since vision tasks, such as classification, operate in embeddings created from images their data should be inherently more compressible than their upstream imaging tasks. Therefore, if signal compressibility is the key, skipping the intermediate imaging step and directly focusing on coding for better feature extraction for subsequent vision tasks should yield better performance.} To show this, our study implements a simple end to end vision system that can be optimized  under different noise assumptions. We illustrate why AGN is not suitable for optimizing coding under Poisson noise and show that a coding scheme optimized for a specific vision task and for the correct noise model can perform close to the optimal non-coding camera even when only photon noise is present.
}


Our contributions are as listed.
\begin{itemize} 

\item {We introduce the methodology of Selective Sensing (SS), which encompasses coding techniques specifically designed to extract data-driven priors selectively. \YLnote{These coding methods can be seamlessly integrated with models for downstream tasks, providing a novel approach to efficiently incorporate learned priors into various applications.\checkednote{\textbf{sales pitch?}}}}
\item \Xpolish{We provide a model to {optimize $\B{M}$} for Poisson noise using neural network model. This model is end to end vision model with classification as the performance metric}{
We provide an end-to-end vision model using classification as the performance metric, which is more perceptually meaningful than mean-squared-error (MSE), to optimize $\B{M}$.
}

\end{itemize}