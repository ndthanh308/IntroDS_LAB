
\subsection{Theoretical Investigation for Null-Prior Reconstruction }

Suppose $\B{I}$ is the identity matrix. Given the noisy photon counts $\tilde{\B{y}}$ and the full-rank mask basis $\B{M}$, we can reconstruct the field of view $\tilde{\B{x}} = \B{M}^{-1} \tilde{\B{y}}$.
Based on Parseval's identity, the covariance matrix of a vector $\B{b}$ is $\B{\Sigma}_{\B{b}\B{b}} = c^2 \sigma^2 \B{I}$ if $\B{b} = \B{W a}$ where $\B{\Sigma}_{\B{a}\B{a}} = \sigma^2 \B{I}$ and 
% $\B{W}$ is $c$ times an orthonormal matrix
$\B{W} \B{W}^\top = c^2 \B{I}$
. For RS, we have $\lambda = \frac{\totphoton}{N^2}$ and $\lambda \tilde{\B{y}} \sim \mathcal{N}(\lambda \B{I}\B{x}, \sigma^2 \B{I})$. Thus, the reconstructed field of view is 
\begin{equation}
    \label{eqn:RS_G_reconstruction}
    \begin{aligned}
        \tilde{\B{x}} &\sim \mathcal{N}\left(\B{x}, \frac{\sigma^2 N^4}{{\totphoton}^2} \B{I}\right)
    \end{aligned}.
\end{equation}
For HB, we have $\lambda = \frac{\totphoton}{2N^2}$  and $\lambda \tilde{\B{y}} \sim \mathcal{N}(\lambda \B{H}\B{x}, 2\sigma^2 \B{I})$ due to the dual branch trick mentioned in section \ref{sssection:DualBranch}. For $\B{H}^{-1}$, $c = \sqrt{\frac{1}{N}}$. The reconstructed field of view is 
\begin{equation}
    \label{eqn:HB_G_reconstruction}
    \begin{aligned}
        \tilde{\B{x}} &\sim \mathcal{N}\left(\B{x}, \frac{8 \sigma^2 N^3}{{\totphoton}^2} \B{I}\right)
    \end{aligned}.
\end{equation}
% Therefore, when considering the AWGN model, using the Hadamard matrix can improve the SNR of the reconstructed field of view when $N$ is large.
Therefore, using the Hadamard matrix in the AWGN model can improve the signal-to-noise ratio (SNR) of the reconstructed field of view when the number of elements, $N$, is large. 

When the noise is Poisson distributed, the covariance matrix of the noisy photon counts depends on the choice of $\B{M}$. Suppose $\sigma_j$ is the standard deviation of $\lambda \tilde{y}_j$, the $j_\text{th}$ measured noisy photon counts. For the RS, we have $\sigma_j^2 = \lambda x_j$ and $\lambda = \frac{\totphoton}{N^2}$. The covariance matrix of the reconstructed field of view is thus 
\begin{equation}
    \label{eqn:RS_P_reconstruction}
    \begin{aligned}
        \B{\Sigma}_{\tilde{\B{x}} \tilde{\B{x}}} &= \frac{N^2}{\totphoton}\diag\{x_1, x_2, \cdots, x_N\}
    \end{aligned}.
\end{equation}
% For HB, we have $\lambda = \frac{\totphoton}{2N^2}$ and $\sigma_j^2 = \lambda \sum_{k=1}^N |H_{jk}| x_k$ by the Skellam Distribution in section \ref{sssection:DualBranch}. Since Hadamard matrices only includes $\pm 1$ entries, we further have $\sigma_j^2 = \lambda \sum_{k=1}^N x_k$ and the covariance matrix  

In the case of the HB, we can express the rate parameter $\lambda$ as $\lambda = \frac{\totphoton}{2N^2}$, as demonstrated in section \ref{sssection:DualBranch} using the Skellam Distribution. Additionally, we can express the variance of each element in the sum as $\sigma_j^2 = \lambda \sum_{k=1}^N |H_{jk}| x_k$. As Hadamard matrices only contain entries of $\pm 1$, this reduces to $\sigma_j^2 = \lambda \sum_{k=1}^N x_k$. Using this information, we can construct the covariance matrix
\begin{equation}
    \label{eqn:HB_P_reconstruction}
    \begin{aligned}
        \B{\Sigma}_{\tilde{\B{x}} \tilde{\B{x}}} &= \frac{2N \sum_{k=1}^N x_k}{\totphoton}  \B{I}
    \end{aligned}.
\end{equation}
% Therefore, the mean squared error (MSE) of HB is 2 times that of RS under Poisson noise. This factor of 2 arises from the dual branch trick. In other words, HB can never exceed RS in reconstruction error even when measuring with 2 PMTs simultaneously.
The results of our analysis reveal that the mean squared error (MSE) of the HB method is twice that of the RS method when the noise is Poisson distributed. This difference is attributed to the dual branch trick utilized in the HB method. Despite the use of two PMTs in the HB method, it is unable to surpass the reconstruction error of the RS method. These findings demonstrate the limitations of the HB method in comparison to the RS method under Poisson noise conditions.

\subsection{Simulated Results}
