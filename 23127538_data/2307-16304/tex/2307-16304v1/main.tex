\documentclass{article}


% if you need to pass options to natbib, use, e.g.:
%     \PassOptionsToPackage{numbers, compress}{natbib}
% before loading neurips_2022


% ready for submission



% to compile a preprint version, e.g., for submission to arXiv, add add the
% [preprint] option:

%\usepackage{neurips_2023}
\usepackage[preprint]{neurips_2023}


% to compile a camera-ready version, add the [final] option, e.g.:
%     \usepackage[final]{neurips_2022}


% to avoid loading the natbib package, add option nonatbib:
%    \usepackage[nonatbib]{neurips_2022}


\usepackage[utf8]{inputenc} % allow utf-8 input
\usepackage[T1]{fontenc}    % use 8-bit T1 fonts
\usepackage{hyperref}       % hyperlinks
\usepackage{url}            % simple URL typesetting
\usepackage{booktabs}       % professional-quality tables
\usepackage{amsfonts}       % blackboard math symbols
\usepackage{nicefrac}       % compact symbols for 1/2, etc.
\usepackage{microtype}      % microtypography
\usepackage{xcolor}         % colors
\usepackage{amsmath}
\usepackage{amssymb}
\usepackage{mathtools}
\usepackage{amsthm}
\usepackage{graphicx}
\usepackage{caption}
\usepackage{subcaption}
\usepackage{svg}
\usepackage{bm}
\theoremstyle{plain}
\usepackage{wrapfig}
\usepackage{ushort}
\newtheorem{theorem}{Theorem}[section]
\newtheorem{lemma}[theorem]{Lemma}
\newtheorem{property}[theorem]{Property}
\newtheorem{assumption}{Assumption}
\newtheorem{hypothesis}{Hypothesis}
\newtheorem{definition}[theorem]{Definition}
\newtheorem{corollary}{Corollary}[theorem]
\theoremstyle{definition}
\DeclareMathOperator*{\argmax}{\text{arg\,max}}
\DeclareMathOperator*{\argmin}{\text{arg\,min}}
\newcommand{\ve}[1]{\bm #1}
\newcommand{\mat}[1]{\text{\bf #1}}
\newcommand{\Set}[1]{\mathcal #1}
\newcommand{\R}{{\mathbb R}}%
\usepackage[noend]{algpseudocode}
\usepackage{algorithm}
\usepackage{setspace}
%\usepackage{algorithm2e}
%\SetKwComment{Comment}{/* }{ */}
%\newcommand{\atcp}[1]{\tcp*[r]%{\makebox[\commentWidth]{#1\hfill}}}

%\newlength\algowd
%\def\savewd#1{\setbox0=\hbox{#1\hspace{.7in}}\algowd=\wd0\relax#1}
%\newcommand\algolines[2]{\savewd{#1}%
% \tcp*{\parbox[t]{\dimexpr\algowidth-\algowd}{#2}}}
%\RestyleAlgo{ruled}
\title{You Shall not Pass: the Zero-Gradient Problem in Predict and Optimize for
Convex Optimization}


% The \author macro works with any number of authors. There are two commands
% used to separate the names and addresses of multiple authors: \And and \AND.
%
% Using \And between authors leaves it to LaTeX to determine where to break the
% lines. Using \AND forces a line break at that point. So, if LaTeX puts 3 of 4
% authors names on the first line, and the last on the second line, try using
% \AND instead of \And before the third author name.


\author{%
  Grigorii Veviurko \\
  Delft University of Technology\\
  \texttt{g.veviurko@tudelft.nl} \\
  \And
  Wendelin Boehmer \\
  Delft University of Technology\\
  \texttt{j.w.bohmer@tudelft.nl} \\
  \And
  Mathijs de Weerdt \\
  Delft University of Technology\\
  \texttt{m.m.deweerdt@tudelft.nl} \\
}


\begin{document}


\maketitle


\begin{abstract}
Predict and optimize is an increasingly popular decision-making paradigm that employs machine learning to predict unknown parameters of optimization problems. Instead of minimizing the prediction error of the parameters, it trains predictive models using task performance as a loss function. In the convex optimization domain, predict and optimize has seen significant progress due to recently developed methods for differentiating optimization problem solutions over the problem parameters. This paper identifies a yet unnoticed drawback of this approach -- the zero-gradient problem -- and introduces a method to solve it. The suggested method is based on the mathematical properties of differential optimization and is verified using two real-world benchmarks.
\end{abstract}


% ######################################################
% Cycling
% ######################################################
To promote sustainability, cities worldwide are promoting a transition to public transportation and active transportation. From these, cycling has proven to provide numerous advantages, including benefits to health \cite{gotschi2016cycling}, economy \cite{clifton2013examining}, and reduction of carbon emissions \cite{NEVES2019130}. Despite these benefits, cycling numbers remain predominantly low in some cities. In contrast, barriers to cycling include hilliness, lack of cycling infrastructure, or appropriate bike storage or parking. Yet, the main deterrent to cycling relates to safety concerns \cite{aldred2015investigating, lawson2013perception, felix2019maturing}. If cyclists feel unsafe or are afraid to cycle, they will prefer other means of transportation. 


% ######################################################
% Perception of Safety
% ######################################################
Thus, for cities aiming to boost cycling numbers and the effectiveness of such strategies, it is increasingly important to understand what affects individuals' perceptions. Perception of cycling safety research explores how individuals subjectively experience cycling accident risk and what fears and events negatively impact one's perception of being involved in a cycling accident. Current research shows that infrastructure layout, fear of traffic, and distracted cycling are some aspects that influence this perception \cite{heinen2010commuting}. Most research focuses on surveys and in-loco and post-riding interviews to compare factors influencing perceptions \cite{sanders2015perceived}. Even though these approaches are vital to understanding cycling perception of safety, they need to be more scalable over space or time due to their high cost (human resources, time, and money). This prevents any analysis of perceptions over time, and qualitative non-scalable data analysis hampers any comparative study across cities or countries.


% !BIB TS-program =
\documentclass[12pt]{article}
\usepackage{color}
\usepackage{amsfonts,amssymb,amsmath}
\usepackage[export]{adjustbox}
\makeatletter
\setlength{\@fptop}{0pt}
 \makeatother
\usepackage{graphicx}
\usepackage[T1]{fontenc}
\usepackage[numbers,sort&compress]{natbib}
\graphicspath{ {./images/} } \textheight 9in \textwidth  6.5in
\topmargin -1cm \oddsidemargin -0.1in \evensidemargin -0.1in
\marginparwidth 17.57mm
%\renewcommand{\baselinestretch}{1.55}
\newcounter{tempeq}
\begin{document}
\title{\textsf{Enhancing the performance of an open quantum battery by adjusting its velocity}}
\author{B. Mojaveri\thanks{Email: bmojaveri@azaruniv.ac.ir; bmojaveri@gmail.com},
\hspace{2mm}R. Jafarzadeh Bahrbeig\thanks{Email:
r.jafarzadeh86@gmail.com},\hspace{2mm}M. A. Fasihi
\thanks{Email: ma-fasihi@azarunic.ac.ir}, and S. Babanzadeh\thanks{Email: s.babanzadeh@azaruniv.ac.ir}\\
{\small {Department of Physics, Azarbaijan Shahid Madani University,
PO Box 51745-406, Tabriz, Iran \,}}} \maketitle
\begin{abstract}
The performance of open quantum batteries (QBs) is severely limited
by decoherence due to the interaction with the surrounding
environment. So, protecting the charging processes against
decoherence is of great importance for realizing QBs. In this work
we address this issue by developing a charging process of a
qubit-based open QB composed of a qubit-battery and a qubit-charger,
where each qubit moves inside an independent cavity reservoir. Our
results show that, in both the Markovian and non-Markovian dynamics,
the charging characteristics, including the charging energy,
efficiency and ergotropy, regularly increase with increasing the
speed of charger and battery qubits. Interestingly, when the charger
and battery move with higher velocities, the initial energy of the
charger is completely transferred to the battery in the Markovian
dynamics. In this situation, it is possible to extract the total
stored energy as work for a long time. Our findings show that open
moving-qubit systems are robust and reliable QBs, thus making them a
promising candidate for experimental implementations.\\\\
{\bf Keywords:} Open quantum batter, Markovian and non-Markovian
charging process, Ergotropy, Atomic motion.
\end{abstract}
\section{introduction}
In recent years, with advancements in quantum thermodynamics, there
has been a radical change of perspective in the framework of energy
manipulation based on the electrochemical principles. The
possibility to create an alternative and efficient energy storage
device at small scale introduces the concept of the quantum battery
(QB), which was proposed by Alicki and Fennes in the 2013's
\cite{Alicki}, and  subsequently became into a significant field of
research. As their name indicates, QBs are finite dimensional
quantum systems that are able to temporarily store energy in their
quantum degrees of freedom for later use. The fundamental strategy
for developing the idea of QBs is based on their non-classical
features such as quantum coherence, entanglement and many-body
collective behaviors that can be cleverly exploited to achieve more
efficient and faster charging processes than the macroscopic
counterpart \cite{Strasberg, Vinjanampathy, Goold, Campisi,
Gelbwaser, Horodecki}. A QB is charged based on an interaction
protocol between QB itself with either an external field or a
quantum system which serves as a charger. It is then discharged into
a consumption hub based on the same protocol. When the battery
enters into an interaction with the charger, it transitions from a
lower energy level into the higher ones and will be charged. So far,
a variety of powerful charging protocols have been proposed in
different platforms, including two-level systems \cite{Farin, Zhang,
Fus}, harmonic oscillators \cite{Cata}, and hybrid light-matter
systems \cite{Maze, Manzo, Cond}. Some proposals have been also
devoted to implement QBs based on the two-level systems such as
trapped ions \cite{Forn, Lv}, cold atoms \cite{Bau} and
superconducting qubits \cite{Devoret}.

 Due to the fact that a real quantum system inevitably interacts with
its environment, studying QBs from the open quantum systems
perspective is attracting considerable interest. The interaction of
a QB with its surrounding environments causes the leakage of the
coherence of battery to the environment, leading to decoherence
effect in the battery. Such an adverse effect often plays a negative
role in the charging and discharging performance of QBs \cite{Camp,
Farin1, Carega}. Decoherence brought during the charging process
tends to lead QBs to a non-active (passive) equilibrium state in
which work extracting from the QBs is often impossible \cite{Barra}
in a cyclic unitary process. The environmental-induced noises also
affect QBs that are disconnected from both charger and consumption
hub and cause self-discharging of that QBs \cite{San0, Pedro,
Salimi}. Therefore, designing a more robust battery against the
environmental dissipations is valuable step for implementation of
QBs in the real-life. Recently, researchers have devoted efforts not
only to studying the effect of the environment on QBs, but also to
exploit non-classical effect as well as to developing open system
protocols to stabilize the charging cycle performance through
quantum control techniques. For example, Kamin et al \cite{Kamin1}
studied the charging performance of a qubit-based QB charged by the
mediation of a non-Markovian environment. They revealed the
non-Markovian property is beneficial for improving charging cycle
performance. In Ref. \cite{Squeezing}, the authors studied dynamics
of a continuous variable QB coupled weakly to the squeezed thermal
reservoir and managed to control the performance of the charging
process by boosting the quantum squeezing of reservoir. A feasible
route for harnessing loss-free dark states for stabilizing the
stored energy of a qubit-based open QB has been introduced in
\cite{Dark}. In addition to the above considerations, several other
protocols have been developed to protect the charging cycle of QBs
such as feedback control method \cite{Mitch, Shao, Ios}, convergent
iterative algorithm \cite{Borhan}, Bang-Bang modulation of the
intensity of an external Hamiltonian \cite{Franc}, inhiring an
auxiliary quantum system \cite{Behzadi}, modulating the detuning
between system and reservoir \cite{Yu0}, stimulated Raman adiabatic
passage technique \cite{Baris}, engineering quantum environments
\cite{Segal}, etc.

 On the other hand, according to the previous studies on
the Markovian and non-Markovian dynamics of open two-qubit systems,
translational motion of qubits provides novel insights for
stabilizing qubit-qubit entanglement against the environmental
induced dissipations by suitably adjusting the velocities of the
qubits \cite{Epjp0, morteza0, Chao0, sare0, Golkar1, Epjp1, MPLA,
Wang00}. We want here to use this safeguard capability of the
motional properties to improve the charging cycle performance of the
open qubit-based QBs. For this end, we consider a moving-biparticle
system composed of a qubit-battery and a qubit-charger that
independently interacts with their local environments. The battery
qubit here is charged with the help of the dipole-dipole interaction
with the charger qubit. We will investigate how the translational
motion of qubits affects the charging process of QB. Our results
show that translational motion of qubits always plays a constructive
role in protecting QB from decay induced by the environment. This
work is organized as follows: in Sec. 2, we introduce and describe
several figures of merit for characterizing the performance of QBs.
In Sec. 3, we illustrate our model and obtain explicit expressions
for the reduced density matrix of the QB and the charger. In Sec. 4
we present the results of our numerical simulations in the context
of their physical significance. Finally, Sec. 5 concludes this
paper.
\section{Figures of Merit}
Let us consider a QB modeled as a quantum system with d-dimensional
Hilbert space $\mathcal{H}$ and Hamiltonian $H_B$ such that
\renewcommand\theequation{\arabic{tempeq}\alph{equation}}
\setcounter{equation}{-1}
\addtocounter{tempeq}{1}\begin{eqnarray}\label{Bat}
H_B=\sum_{i=1}^{d} \varepsilon_i
|\varepsilon_i\rangle\langle\varepsilon_i|,
\end{eqnarray}
with non-degenerate energy levels $\varepsilon_i \leq
\varepsilon_{i+1}$. Internal energy of QB is given by $Tr(\rho_B
H_B)$, where $\rho_B$ is the state of the battery. Charging a QB
means brings the quantum system from a lower energy state $\rho_B$
to a higher energy state $\rho_B^\prime$, while discharging refers
to the inverse process, i.e., brings the quantum system from a
higher energy state $\rho_B^\prime$ to a lower one
$\rho_B^{\prime\prime}$:
\renewcommand\theequation{\arabic{tempeq}\alph{equation}}
\setcounter{equation}{-1}
\addtocounter{tempeq}{1}\begin{eqnarray}\label{den}
\texttt{Tr}\left\{\left(\rho_B^\prime-\rho_B\right) H_B\right\}\geq0,\qquad\qquad\qquad\qquad charging \nonumber \\
\texttt{Tr}\left\{\left(\rho_B^{\prime\prime}-\rho_B^\prime\right)
H_B\right\}\geq0.\qquad\qquad\qquad\quad \;\;discharging
\end{eqnarray}
Therefore, in a charging process, the actual stored energy of QB at
time $t$, regarding the initial energy, can be expressed as follows
\cite{Alicki}
\renewcommand\theequation{\arabic{tempeq}\alph{equation}}
\setcounter{equation}{-1} \addtocounter{tempeq}{1}\begin{equation}
\Delta E_B=\texttt{Tr}\{\rho_B(t) H_B\}-\texttt{Tr}\{\rho_B(0)
H_B\}.
\end{equation}
A complete converting the stored energy into valuable work is
impossible without dissipation of heat according to the second law
of thermodynamics. The maximum amount of energy extracted from a
given quantum state $\rho_B=\sum_{i} r_i |r_i\rangle\langle r_i|$,
($ r_i \geq r_{i+1}$) through a cyclic unitary operation is called
ergotropy \cite{Allahverdyan}. This quantity can be defined as
\cite{Allahverdyan, Franc0, Cakmak0}
\renewcommand\theequation{\arabic{tempeq}\alph{equation}}
\setcounter{equation}{-1}
\addtocounter{tempeq}{1}\begin{equation}\label{ergo}
\mathcal{W}=\texttt{Tr}\{\rho_B
H_B\}-\texttt{min}_U\,\texttt{Tr}\{U\rho_B U^{\dagger} H_B\},
\end{equation}
where the minimization is taken over all possible unitary
transformations acting locally on such system. It has been shown in
\cite{Allahverdyan} that no work can be extracted from the passive
counterpart of $\rho_B$ with the form $\sigma_{\rho_B}=\sum_{i} r_i
|\varepsilon_i\rangle\langle\varepsilon_i|$. The unique unitary
transformation $U=\sum_i |\varepsilon_i\rangle\langle r_i|$ on the
$\rho$ minimizes $\texttt{Tr}(U\rho_B U^{\dagger} H_B)$, and when
inserted in Eq. (\ref{ergo}) yields the following expression for the
ergotropy
\renewcommand\theequation{\arabic{tempeq}\alph{equation}}
\setcounter{equation}{-1} \addtocounter{tempeq}{1}\begin{equation}
\mathcal{W}=\sum_{i,j} r_j \varepsilon_i\left(|\langle
r_j|\varepsilon_i\rangle|^2-\delta_{ij}\right).
\end{equation}
In order to quantify the amount of extractable energy, the
efficiency $\eta$ is defined as the ratio between the ergotropy
$\mathcal{W}$ and the total charging energy $\Delta E_B$
\renewcommand\theequation{\arabic{tempeq}\alph{equation}}
\setcounter{equation}{-1} \addtocounter{tempeq}{1}\begin{equation}
\eta=\frac{\mathcal{W}}{\Delta E_B}.
\end{equation}% Figure environment removed
\section{Open Moving-Quantum Battery}
The open QB under consideration is composed of an atomic two-qubit
system, the qubit $A$ as a charger and the qubit $B$ as a quantum
battery, coupled to each other trough the dipole-dipole interaction.
The battery and charger qubits coupled locally to two independent
zero-temperature cavity reservoirs (see Fig. 1). We assume that each
qubit moves along the $z$-axis of its cavity at a constant
non-relativistic speed $v$. For simplicity we neglect here any
scattering \cite{Engl} or trapping \cite{Haro} effects and consider
the translational motion of the atom qubits being classically. Under
the dipole and rotating wave approximation, the entire system is
ruled by Hamiltonian (setting $\hbar=1$)
\renewcommand\theequation{\arabic{tempeq}\alph{equation}}
\setcounter{equation}{-1} \addtocounter{tempeq}{1}\begin{equation}
H=H_0+H_{int},
\end{equation}
with
\renewcommand\theequation{\arabic{tempeq}\alph{equation}}
\setcounter{equation}{-1}
\addtocounter{tempeq}{1}\begin{eqnarray}\label{Ham}
&&\hspace{-1.15cm}
H_0=H_A+H_B+H_{R_A}+H_{R_B}=\sum_{j=A,B}\left(\frac{\omega_0}{2}
\sigma_{z}^{j} + \sum_{k}\omega_{k}^j a_{k}^{j\dag} a_{k}^j\right),\nonumber\\
&&\hspace{-1.2cm}H_{int}=H_{A-B}+H_{A-R_A}+H_{B-R_B}=D\left(\sigma_{+}^{A}\sigma_{-}^{B}+\sigma_{-}^{A}
\sigma_{+}^{B}\right) +\sum_{j=A,B}\sum_{k} f_k^j(z)
\left(\mathfrak{g}_{k}^j \sigma_{+}^{j} a^j_k +H.c.\right).
\end{eqnarray}
Here, H.c. stands for Hermitian conjugate, $\sigma_z^j$,
$\sigma_+^j$, and $\sigma_-^j$ $(j=A,B)$ are, respectively, the
population inversion, raising and lowering operators of the $j$th
qubit with transition frequency $\omega_0$. $a_k^{j\dagger}$ and
$a^j_k$ are, respectively, the creation and annihilation operators
of the $k$th mode of the cavity reservoir $j$ with the frequency
$\omega_k^j$. Also, $D$ is coupling constant of the dipole-dipole
interaction between the battery and charger qubits, and
$\mathfrak{g}_{k}^j$ is the coupling constant between the $j$th
qubit and $k$th mode of in the cavity reservoir $j$. The effect of
translation motion of the battery and charger qubits has been
included in the model by introducing the $z$-dependent shape
function $f_k^j(z)$ in the Hamiltonian $H_{int}$. When the battery
and charger qubits are moving with same constant velocity $v$, the
shape function $f_k^j(z=vt)$ can be taken into account as
\renewcommand\theequation{\arabic{tempeq}\alph{equation}}
\setcounter{equation}{-1} \addtocounter{tempeq}{1}\begin{equation}
f_k^j(z)=\sin[\omega_k^j(\beta t-\Gamma)],\qquad\qquad j=A,B
\end{equation}
where, $\Gamma=L/c$ with $L$ being the size of the cavity. Also,
$\beta=v/c$ where $c$ refers to the speed of light in the vacuum
space. This particular form of the shape function can be obtained by
imposing an appropriate boundary condition on the cavity reservoirs
\cite{Lenard, morteza0}. Here we describe the translational motion
of both battery and charger qubits by classical mechanics ($z=vt$).
To this end, we will choose the values of the parameters in such a
way that the de Broglie wavelength of qubit $\lambda_B$ is
significantly smaller than the wavelength $\lambda_0$ associated
with the resonant transition $\omega_0=\omega_n$ ($\omega_n$ is the
central frequency of the cavity field mode) \cite{mortezapour,
Cook}. Furthermore, we consider a situation in which the photon
momentum is relatively small than the atomic momentum and thus we
neglect the atomic recoil caused by the interaction with the
electric field \cite{Wilkens}. In the optical regime, to ignore the
atomic recoil and consider the translational motion of atoms as
classical, the velocity of qubits should be $v\gg 10^{-3}$
\cite{morteza0}.

In the interaction picture (IP) generated by the unitary
transformation $U=e^{-iH_0t}$, the Hamiltonian (\ref{Ham}) can be
written as follows
\renewcommand\theequation{\arabic{tempeq}\alph{equation}}
\setcounter{equation}{-1}
\addtocounter{tempeq}{1}\begin{eqnarray}\label{HIP}
&&\hspace{-1.5cm}H_{IP}=D\left(\sigma_{+}^{A}
\sigma_{-}^{B}+\sigma_{-}^{A} \sigma_{+}^{B}\right)+
\sum_{j=A,B}\sum_{k} f_k^j(z)\left(\mathfrak{g}_{k}^j \sigma_{+}^{j}
a_k^{j} e^{i(\omega_0-\omega_k^j)t}+\mathfrak{g}_k^{j \ast}
\sigma_{-}^{j}a_{k}^{j\dag} e^{-i(\omega_0-\omega_k^j)t}\right).
\end{eqnarray}
It is straightforward to show that the total excitation operator
$\hat{\mathcal{N}}=\sum_{j=A,B}\left(\sum_k\hat{a_k}^{j\dagger}\hat{a_k}^j+
\frac{1}{2}\hat{\sigma}_{z}^{j}\right)+1$, commutes with the total
Hamiltonian, i.e. $[H,\hat{\mathcal{N}}]=0$ and therefor it is the
constant of the motion. This allows us to decompose Hilbert space of
the entire qubit-cavity system,
$\mathcal{H}=\mathcal{H}_q\otimes\mathcal{H}_R$ spanned by the basis
$\{\left|i_A,j_B\right\rangle\otimes\left|n_1,n_2, ...,n_k,
...\right\rangle_{R_A}|_{n_1,n_2,...=0}^{\infty}
\otimes\left|m_1,m_2, ...,m_k,
...\right\rangle_{R_B}|_{m_1,m_2,...=0}^{\infty}\}$
$\left(i,j=e,g\right)$ into the excitation subspaces, as follows
\renewcommand\theequation{\arabic{tempeq}\alph{equation}}
\setcounter{equation}{-1} \addtocounter{tempeq}{1}
\begin{eqnarray}
&&\hspace{-14mm} \mathcal{H}=\oplus_{n=0}^{\infty} \mathcal{H}_{n}.
\end{eqnarray}
As a result of this decomposition, the dynamics of the entire
qubit-reservoir system can be restricted to the excitation subspaces
labeled by the total excitation number $n$. Here we are interested
to explore dynamics of the entire system in the single-excitation
subspace $\mathcal{H}_1$ spanned by vectors
$\{\left|g_A,g_B\right\rangle\otimes\left|1_k\right\rangle_{R_A}\left|0_k\right\rangle_{R_B}|_{k=0}^\infty,
\left|g_A,g_B\right\rangle\otimes\left|0_k\right\rangle_{R_A}\left|1_k\right\rangle_{R_B}|_{k=0}^\infty,
\left|e_A,g_B\right\rangle\otimes\left|0_k\right\rangle_{R_A}\left|0_k\right\rangle_{R_B},
\left|g_A,e_B\right\rangle\otimes\left|0_k\right\rangle_{R_A}\left|0_k\right\rangle_{R_B}\}$
in which the single excitation is either in one of the qubits or in
the k-th mode of one of cavity reservoirs. We consider a normalized
initial state of entire qubit-reservoir as a superposition of
$\left|e_A,g_B\right\rangle\left|0_k\right\rangle_{R_A}\left|0_k\right\rangle_{R_B}$
and
$\left|g_A,e_B\right\rangle\left|0_k\right\rangle_{R_A}\left|0_k\right\rangle_{R_B}$
states with the following form
\renewcommand\theequation{\arabic{tempeq}\alph{equation}}
\setcounter{equation}{-1}
\addtocounter{tempeq}{1}\begin{eqnarray}\label{sai0}
|\Psi(0)\rangle=\big[c_1(0) |e_{A},g_{B}\rangle +c_2(0)
|g_{A},e_{B}\rangle\big]\otimes |0\rangle_{R_A}|0\rangle_{R_B}.
\end{eqnarray}
For times $t>0$, we expand the state vector $|\Psi(t)\rangle$ in
terms of the vector basis of the single-excitation subspace
$\mathcal{H}_1$ as
\renewcommand\theequation{\arabic{tempeq}\alph{equation}}
\setcounter{equation}{-1}
\addtocounter{tempeq}{1}{\footnotesize\begin{eqnarray}\label{sai}
&&\hspace{-3.5cm}\left|\Psi(t)\right\rangle=\big[c_1(t)\left |e_{A},
g_B\right\rangle +c_2(t) \left|g_A, e_B\right\rangle\big] \otimes
\left|0_k\right\rangle_{R_A}\left|0_k\right\rangle_{R_B}
\nonumber\\
&&\hspace{-2.35cm}+\left|g_A, g_B\right\rangle\otimes\sum_{k}
\big[d_{k}(t)\left|1_k\right\rangle_{R_A}\left|0_k\right\rangle_{R_B}+d_{k}^{\prime}(t)
\left|0_k\right\rangle_{R_A}\left|1_k\right\rangle_{R_B}\big],
\end{eqnarray}}
where the time-dependent amplitudes satisfy the normalization
requirement
\renewcommand\theequation{\arabic{tempeq}\alph{equation}}
\setcounter{equation}{-1} \addtocounter{tempeq}{1}\begin{eqnarray}
\sum_{i=1}^2|c_i(t)|^2+\sum_k(|d_{k}(t)|^2+|d_{k}^{\prime}(t)|^2)=1.
\end{eqnarray}
By taking the partial traces over the field modes and subsystem A
(B), the reduced time-dependent density operator for the battery
(charger) in the $\{\left|e\right\rangle, \left|g\right\rangle\}$
basis is obtained as
\renewcommand\theequation{\arabic{tempeq}\alph{equation}}
\setcounter{equation}{0} \addtocounter{tempeq}{1}\begin{eqnarray}
&&\hspace{-2cm}\rho_A(t)=|c_1(t)|^2\left|e_A\right\rangle\left\langle
e_A\right|-\left(1-|c_1(t)|^2\right)\left|g_A\right\rangle\left\langle
g_A\right|\label{rob2},\\
&&\hspace{-2cm}\rho_B(t)=|c_2(t)|^2\left|e_B\right\rangle\left\langle
e_B\right|-\left(1-|c_2(t)|^2\right)\left|g_B\right\rangle\left\langle
g_B\right|\label{rob1}.
\end{eqnarray}

 Inserting Eq. (\ref{sai}) into the time dependent Schr\"{o}dinger
equation $H_{IP}|\Psi(t)\rangle=i\frac{d}{d t}|\Psi(t)\rangle$, with
$H_{IP}$ given in (\ref{HIP}), leads to the following set of
differential equations for time-dependent amplitudes
\renewcommand\theequation{\arabic{tempeq}\alph{equation}}
\setcounter{equation}{0} \addtocounter{tempeq}{1}\begin{eqnarray}
&&\hspace{-4cm}i\dot{c_1}(t)=D c_2(t)+\sum_{k} \mathfrak{g}_{k}^A
f_k^A(z)d_{k}(t)e^{i(\omega_0-\omega_{k}^A)}\label{c1t},\\
&&\hspace{-4cm}i\dot{c_2}(t)= D c_1(t)+ \sum_{k} \mathfrak{g}_{k}^B
f_k^B(z)d_{k}^{\prime}(t)e^{i(\omega_0-\omega_{k}^B)}\label{c2t},\\
&&\hspace{-4cm}i\dot{d}_{k}(t)=\mathfrak{g}_k^{A\ast}f_k^A(z)
c_1(t)e^{-i(\omega_0-\omega_{k}^A)t},\label{d1t}\\
&&\hspace{-4cm}i\dot{d}_{k}^{\prime}(t)=
\mathfrak{g}_k^{B\ast}f_k^B(z)
c_2(t)e^{-i(\omega_0-\omega_{k}^B)t}\label{d2t}.
\end{eqnarray}
By integrating Eqs. (\ref{d1t}) and (\ref{d2t}) with the initial
condition $d_{k}(0)=0$ and $d_{k}^{\prime}(0)=0$ and putting their
solutions, respectively, in Eqs. (\ref{c1t}) and (\ref{c2t}), we get
the following integro-differential equations for the amplitudes
$c_1(t)$ and $c_2(t)$
\renewcommand\theequation{\arabic{tempeq}\alph{equation}}
\setcounter{equation}{0} \addtocounter{tempeq}{1}\begin{eqnarray}
&&\hspace{-2cm}\dot{c_1}(t)=-iDc_2(t)+\int_{0}^{t}F_A(t-t^\prime)c_1(t^\prime)dt^\prime,\label{mt}\\
&&\hspace{-2cm}\dot{c_2}(t)=-iDc_1(t)+\int_{0}^{t}F_B(t-t^\prime)c_2(t^\prime)dt^\prime,\label{nt}
\end{eqnarray}
where
\renewcommand\theequation{\arabic{tempeq}\alph{equation}}
\setcounter{equation}{0} \addtocounter{tempeq}{1}\begin{eqnarray}
&&\hspace{-2cm}F_{A}(t-t^\prime)=\sum_{k} |\mathfrak{g}_{k}^A|^2
e^{i(\omega_0-\omega_{k}^A)(t-t^\prime)}\sin[\omega_k^A(\beta^A
t-\Gamma)]\sin[\omega_k^A(\beta^A t^\prime-\Gamma)],\\
&&\hspace{-2cm}F_{B}(t-t^\prime)=\sum_{k} |\mathfrak{g}_{k}^B| ^2
e^{i(\omega_0-\omega_{k}^B)(t-t^\prime)}\sin[\omega_k^B(\beta^B
t-\Gamma)]\sin[\omega_k^B(\beta^B t^\prime-\Gamma)],
\end{eqnarray}
are the memory correlation function of the reservoirs $A$ and $B$,
respectively. For simplicity, we suppose
$F_{A}(t-t^\prime)=F_{B}(t-t^\prime)=F(t-t^\prime)$. In the limit of
a large number of modes ( in the continuum limit ), the correlation
function $F(t-t^\prime)$ takes the following form
\renewcommand\theequation{\arabic{tempeq}\alph{equation}}
\setcounter{equation}{-1}
\addtocounter{tempeq}{1}\begin{equation}\label{kernel}
F(t-t^\prime)=\int d\omega J(\omega)
e^{i(\omega_0-\omega)(t-t^\prime)}\sin[\omega(\beta
t-\Gamma)]\sin[\omega(\beta t^\prime-\Gamma)],
\end{equation}
in which $J(\omega)$ is the spectral density of the cavity
reservoirs and has the Lorentzian form \cite{Lenard, Breuer0}
\renewcommand\theequation{\arabic{tempeq}\alph{equation}}
\setcounter{equation}{-1}
\addtocounter{tempeq}{1}\begin{equation}\label{lorentz}
J(\omega)=\frac{1}{2\pi}\frac{\gamma\lambda^2}{(\omega_0-\omega-\Delta)^2+\lambda^2},
\end{equation}
where $\lambda$ defines the spectral width of the coupling which is
connected to the memory time $\tau_E$ by the relation
$\tau_E=\lambda^{-1}$ and $\gamma$ refers to the qubit-environment
coupling strength which is related to the relaxation time scale
$\tau_R$ by $\tau_R \approx \gamma^{-1}$. Also $\Delta$ is the
detuning of $\omega_0$ and the central frequency of the cavity. The
weak and strong coupling regimes can be distinguished by comparing
$\tau_E$ and  $\tau_R$, in other words with an increasing
$\frac{\tau_E}{\tau_R}=\frac{\gamma}{\lambda}$ ratio, the
interaction will transition into a strong coupling or a non-Markovian regime \cite{Breuer0}.\\
By inserting the Eq. (\ref{lorentz}) into the Eq. (\ref{kernel}) and
after some calculations, in the continuum limit ($\Gamma \rightarrow
\infty$), the correlation function is simplified as
\renewcommand\theequation{\arabic{tempeq}\alph{equation}}
\setcounter{equation}{-1}
\addtocounter{tempeq}{1}\begin{equation}\label{ft}
F(t-t^\prime)=\frac{\gamma \lambda}{4} \cosh[\beta
\overline{\lambda}(t-t^\prime)] e^{-(\lambda-i\Delta) |t-t^\prime|}
\end{equation}
with $\overline{\lambda}=\lambda+i(\omega_0-\Delta)$.\\
In view of (\ref{ft}), taking the Laplace transformations of both
sides of the differential Eqs. (\ref{mt}) and (\ref{nt}) and using
the convolution property
$\mathcal{L}[\int_{0}^{t}\mathbf{A}(t-t^\prime) \mathbf{B}(t^\prime)
dt^\prime]=\mathbf{A}(s)\mathbf{B}(s)$ yields
\renewcommand\theequation{\arabic{tempeq}\alph{equation}}
\setcounter{equation}{0} \addtocounter{tempeq}{1}\begin{eqnarray}
&&\hspace{-2cm}sc_1(s)-c_1(0)=-iDc_2(s)-F(s)c_1(s),\label{ms}\\
&&\hspace{-2cm}sc_2(s)-c_2(0)=-iDc_1(s)-F(s)c_2(s),\label{ns}
\end{eqnarray}
where the functions $c_1(s)$ and $c_2(s)$ are the Laplace
transformations of the $c_1(t)$ and $c_2(t)$, respectively, and
$F(s)$ is the Laplace transforms of $F(t-t^\prime)$ which has the
following explicit form
\renewcommand\theequation{\arabic{tempeq}\alph{equation}}
\setcounter{equation}{-1} \addtocounter{tempeq}{1}\begin{eqnarray}
F(s)=\frac{\gamma\lambda}{4}\frac{s+\overline{\lambda}}{(s+\overline{\lambda})^2-\beta^2\overline{\lambda}\,^2}.
\end{eqnarray}
By reformulating the Eqs. (\ref{ms}) and (\ref{ns}), we get a
general solution for $c_1(s)$ and $c_2(s)$ as follows
\renewcommand\theequation{\arabic{tempeq}\alph{equation}}
\setcounter{equation}{0} \addtocounter{tempeq}{1}\begin{eqnarray}
&&\hspace{-2cm}c_1(s)=\frac{s+F(s)}{\big(s+F(s)\big)^2+D^2}c_1(0)-i\frac{D}{(s+F(s))^2+D^2}c_2(0),\\
&&\hspace{-2cm}c_2(s)=\frac{s+F(s)}{\big(s+F(s)\big)^2+D^2}c_2(0)-i\frac{D}{(s+F(s))^2+D^2}c_1(0).
\end{eqnarray}
In continuation, by applying the inverse Laplace transformation on
the both side of the above equations, we obtain finally $c_1(t)$ and
$c_2(t)$, as
\renewcommand\theequation{\arabic{tempeq}\alph{equation}}
\setcounter{equation}{0} \addtocounter{tempeq}{1}\begin{eqnarray}
&&\hspace{-2cm}c_1(t)=\frac{1}{2}\bigg(c_1(0)\Re(\mathcal{M}(t))-ic_2(0)\Im(\mathcal{M}(t))\bigg)\label{ct12},\\
&&\hspace{-2cm}c_2(t)=\frac{1}{2}\bigg(c_2(0)\Re(\mathcal{M}(t))-ic_1(0)\Im(\mathcal{M}(t))\bigg)\label{ct122},
\end{eqnarray}
where, $\Re(x)$ ($\Im(x)$) is real (imaginary) part of $x$, and
\renewcommand\theequation{\arabic{tempeq}\alph{equation}}
\setcounter{equation}{-1} \addtocounter{tempeq}{1}\begin{equation}
\mathcal{M}(t)=\sum_{i,j,k=1}^3\varepsilon_{ijk}\frac{ e^{q_it}
(q_j-q_k)\bigg((q_i+\overline{\lambda})^2-\beta
^2\overline{\lambda}^2\bigg)}{\prod_{i=1}^{3}\prod_{j=i+1}^{3}(q_i-q_j)},
\end{equation}
with $\varepsilon_{ijk}$ is the Levi-Civita symbol and $q_i (i=  1,
2, 3)$ are the roots of
\renewcommand\theequation{\arabic{tempeq}\alph{equation}}
\setcounter{equation}{-1} \addtocounter{tempeq}{1}\begin{equation}
q^3+q^2(2 \overline{\lambda}-i \text{D} )+q \left(\frac{\gamma
\lambda }{4}+\overline{\lambda} (\overline{\lambda}-2 i
\text{D})-\beta ^2\overline{\lambda}^2\right)+\frac{\gamma  \lambda
\overline{\lambda}}{4}+i \text{D} \overline{\lambda}^2\left(\beta
^2-1\right)=0.
\end{equation}

 With substitution (\ref{ct12}) and (\ref{ct122}), respectively, into the reduced density matrices
(\ref{rob1}) and (\ref{rob2}), and then using the $\Delta
E_{A(B)}=\texttt{Tr}\{\rho_{A(B)}(t)
H_{A(B)}\}-\texttt{Tr}\{\rho_{A(B)}(0) H_{A(B)}\}$ , the internal
energy of the charger and battery are deduced as
\renewcommand\theequation{\arabic{tempeq}\alph{equation}}
\setcounter{equation}{-1} \addtocounter{tempeq}{1}\begin{equation}
\Delta
E_A=\omega_0\left(|c_1(t)|^2-|c_1(0)|^2\right),\quad\quad\Delta
E_B=\omega_0\left(|c_2(t)|^2-|c_2(0)|^2\right).
\end{equation}
On the other hand, one can obtain ergotropy of the battery by
substitution Eq. (\ref{rob1}) with Eq. (\ref{ergo}). So, we have
\renewcommand\theequation{\arabic{tempeq}\alph{equation}}
\setcounter{equation}{-1} \addtocounter{tempeq}{1}\begin{equation}
 W_B=\omega_0\left(2|c_2(t)|^2-1\right)\Theta
\left(|c_2(t)|^2-\frac{1}{2}\right),
\end{equation}
where $\Theta(x-x_0)$ is the Heaviside function, which satisfies
$\Theta(x-x_0)=0$ for $x<x_0$, $\Theta(x-x_0)=\frac{1}{2}$ for
$x=x_0$ and $\Theta(x-x_0)=1$ for $x>x_0$.
% Figure environment removed
% Figure environment removed
\section{Numerical Results and Discussion}
In this section, we will analyze the charging dynamics of the
introduced open moving-battery in the weak and strong coupling
regimes. In particular, we explore the role of the movement of QB on
the dynamical behavior of performance indicators including stored
energy, ergotropy and efficiency. In our following analysis, we
choose the optical regime parameters \cite{Hood, Pinkse} and
consider that qubit transition frequency as
$\omega_0=1.5\times10^{9}\lambda$. In what follows, we consider an
initial condition in which the battery is initially empty and the
charger has the maximum energy, i.e. $c_1(0)=0$, $c_2(0)=1$.
% Figure environment removed

 In Fig. 2, we plot the Markovian and non-Markovian dynamics of the stored energy $\Delta E_B$
for the initial state
$\left|\Psi(0)\right\rangle=\left|g\right\rangle_{A}\left|e\right\rangle_{B}\otimes
\left|0\right\rangle_{R_B}\left|0\right\rangle_{R_B}$, by
considering different values of the QB speed $\beta$. In panel (a),
the battery is charged in the Markovian dynamics with
$(\gamma=0.1\lambda)$, while in panel (b), it is charged in a
non-Markovian dynamics with $(\gamma=20\lambda)$. Here we consider a
situation at which the charger and battery's qubits are both in
resonance with the reservoir modes by setting $\Delta=0$. According
to this figure, the positive impact of the translational motion of
the charger and batter's qubits in controlling the stored energy of
battery is clearly visible in both Markovian and non-Markovian
charging processes. As can be seen in both Figs. 2(a) and (b), when
the charger and battery's qubits are at rest inside their cavity
reservoirs, the stored energy in the battery $\Delta E_B$ decays
into zero at sufficiently long times. However the rate of these
decays decreases regularly by gradual growth of the qubit velocity,
and therefore the energy stored in the battery and consequently the
charging process is strongly protected from the environmental
noises. Comparing Fig. 2(a) with Fig. 2(b) clearly reveals a
fundamental difference between Markovian and non-Markovian charging
processes. The maximal amount of stored energy in the Markovian
charging process is more than those of the non-Markovian charging
process. The reason stems from the nature of the qubit-cavity
coupling. In the non-Markovian charging process, the coupling
strength of charger's qubit to the cavity modes is greater than its
coupling to the battery's qubit, therefore, the initial internal
energy of charger has more tendency to evolve toward the reservoir
than to the battery. Moreover, since the motional effect of QB has
been included in battery-cavity and charger-cavity coupling
strength, it seems that increasing speed of QB decreases the
charger-cavity coupling strength in favor of to charger-battery
coupling strength, which increases the energy stored in the battery.

In order to get more insight to this area and a deeper understanding
of the relationship between the charger and battery energy, in Fig.
2 we have illustrated the energy stored in the battery at the end of
charging process as well as the energy that the charger loses at the
same time. Here $\Delta E_B$ and $|\Delta E_A|$ have been plotted as
a function of the dimensionless time $\lambda t$ for the qubit
velocities $\beta=0$ and $\beta=0.7\times 10^{-9}$ in the Markovian
and non-Markovian regimes. In the non-Markovian charging process,
$|\Delta E_A|$ is much more than $\Delta E_B$ for a given $\beta$ as
shown in Fig. 3(b). This implies that the internal energy of the
charger is not completely transferred to the battery. Fig. 3(b) also
shows that, when the charger and battery's qubits are at rest inside
their cavity reservoirs, the charger's qubit immediately loses a
large amount of its initial energy without being transferred to the
battery. However, increasing the qubit velocity (decreasing the
ratio of charger-cavity coupling strength to charger-battery
coupling strength) during the non-Markovian process, decreases the
initial loss-rate of the charger, and therefore improves the energy
transfer in the charging processes.

The relationship between the charger and battery energy in the
Markovian charging process is drastically different from that in the
non-Markovian charging process. One can infer from Fig. 3(a) that,
for the static battery-charger system ($\beta=0$), the total energy
of the charger can be transferred to the battery in the Markovian
short-charging process, where we have $|\Delta E_A|=\Delta E_B$.
Interestingly, when the qubits move with the velocity
$\beta=0.7\times10^{-9}$, $|\Delta E_A|=\Delta E_B$ holds at any
charging time. So, we conclude again that a robust Markovian
charging against the arisen dissipation can be achieved, when the
qubits move with higher velocities.
% Figure environment removed

 In the following, we examine the influence of translational motion
of the battery-charger system on the dynamics of ergotropy. In Fig.
4, we plot $W/W_{max}$ as a function of $\lambda t$ for the
different values of $\beta$ in the Markovian (Fig. 4(a)) and
non-Markovian (Fig. 4(b)) regimes. Our numerical results in Fig.
4(a) and (b) illustrate that, the effect of translational motion of
QB on the ergotropy is also constructive in both Markovian and
non-Markovian regimes. Fig. 4(b) shows that, in the non-Markovian
regime, in the cases of stationary ($\beta=0$) and slowly moving
($\beta=3\times10^{-9}$) qubits, we are not able to extract useful
work from the QB, but in this regime a considerable work can be
extracted, as the qubits move with a higher velocity
($\beta=0.8\times10^{-9}$). Our numerical results in Fig. 4(a)
illustrate that, the effect of translational motion of QB on the
ergotropy is more considerable in the Markovian case. We observe
that, in the Markovian regime, increasing the speed of QB $\beta$
(decreasing the qubit-reservoir coupling) not only boosts the
ergotropy, but also increases the number of time zones in which work
can be extracted. Accordingly, a strong robust charging process can
be established in the higher speed limit, in which the extractable
work approaches to its maximum value.

 Finally, we examine the effect of translational motion
of QB on the Markovian and non-Markovian charging efficiency. The
results for Markovian and non-Markovian charging processes are
presented in Fig. 5(a) and 5(b), respectively. Here we consider the
same parameter values as Fig. 4. Comparing Figs. 4 and 3 reveals
that both ergotropy and efficiency are positively affected by the
translational motion of QB. However the efficiency is influenced
more than the ergotropy; the amount of increment in efficiency is
more than the ergotropy in both Markovian and non-Markovian charging
processes.
\section{Outlook and summary}
To summarize, we proposed a mechanism for robust charging process of
an open qubit-based quantum battery (QB) whose robustness can be
well controlled by the translational motion of the charger and
battery in both Markovian and non-Markovian dynamical regimes. Both
the battery and charger's qubits move with a same speed inside two
separated identical environments, and are directly coupled by the
dipole-dipole interaction. We showed that the stored energy,
ergotropy and efficiency of the moving QB regularly increased with
the gradual growth of the charger and battery speed, thereby
improving its charging performance. The constructive role of the
translational movement of QB in controlling the charging process
arises from the attachment of qubits velocities to the
qubit-reservoir coupling strength (see Eq. (\ref{Ham})). According
to the adopted charging protocol, a weak qubit-reservoir coupling is
required for a strongly robust charging process which can be
fulfilled by adjusting $\beta$ to the higher velocities.

 Our results represent a further control strategy to have a robust QB with
a natural implementation in cavity-QED context. The strategy can be
easily implemented also in the circuit-QED setups where the qubit
position slowly varies linearly with time and also the qubit-cavity
interaction is tuned through a sinusoidal position-dependent
coupling \cite{Jones}.

  In perspective, we believe that this strategy can be used
to control the performance of the discharging of a qubit-based QB to
an available consumption hub. Further efforts in this field can be
devoted to use the proposed strategy for improving the performance
of the two-photon based charging process where the moving-QB is
coupled with a cavity reservoir by means of a two-photon
relaxation.\\\\
\textbf{\large{Data availability}}\\ The datasets used and analysed
during the current study available from the corresponding author on
reasonable request.
\begin{thebibliography}{99}
\bibitem{Alicki} R. Alicki and M. Fannes, Entanglement boost for extractable work from ensembles of quantum batteries, Phys. Rev. E 87, 042123 (2013).
\bibitem{Strasberg} P. Strasberg, G. Schaller, T. Brandes, and M. Esposito, Quantum and information thermodynamics: A unifying framework based on repeated interactions, Phys. Rev. X 7, 021003 (2016).
\bibitem{Vinjanampathy} S. Vinjanampathy and J. Anders, Quantum thermodynamics, Cont. Phy. 57, 545 (2016).
\bibitem{Goold} J. Goold, M. Huber, A. Riera, L. del Rio, and P. Skrzypczyk, The role of quantum information in thermodynamics: a topical review, J. Phys. A: Math. Theor. 49, 143001 (2016).
\bibitem{Campisi} M. Campisi, P. H\"{a}nggi, and P. Talkner, Colloquium: Quantum fluctuation relations: Foundations and applications, Rev. Mod. Phys. 83, 1653 (2011).
\bibitem{Gelbwaser} D. Gelbwaser-Klimovsky, W. Niedenzu and G. Kurizki, Thermodynamics of quantum systems under dynamical control, Adv. At. Mol. Opt. Phys., 64, 329 (2015).
\bibitem{Horodecki} M. Horodecki and J. Oppenheim,Fundamental limitations for quantum and nanoscale thermodynamics, Nature Comm. 4, 2059 (2013).
\bibitem{Farin} D. Farina, G. M. Andolina, A. Mari, M. Polini and V. Giovannetti, powerful charging of quantum batteries, Phys. Rev. B 99, 035421 (2019).
\bibitem{Zhang} Y-Y. Zhang, T-R. Yang, L. Fu and X. Wang, Powerful harmonic charging in a quantum battery, Phys. Rev. E 99, 052106 (2019).
\bibitem{Fus} L. Fusco, M. Paternostro, and G. D. Chiara, Work extraction and energy storage in the Dicke model, Phys. Rev. E 94, 052122 (2016).
\bibitem{Cata} R. R. Rodriguez, B. Ahmadi, P. Mazurek, S. Barzanjeh, R. Alicki and P. Horodecki, catalysis in charging quantum batteries, Phys. Rev. A 107, 042419 (2023).
\bibitem{Maze} J. Carrasco, J. R. Maze, C. Hermann-Avigliano and F. Barra, collective enhancement in dissipative quantum batteries, Phys. Rev. E. 105, 064119 (2022).
\bibitem{Manzo} M. Gumberidze, M. Kol\'{a}r and R. filip, Measurement induced Synthesis of coherent Quantum Batteries, Sci. Rep 9, 19628 (2019).
\bibitem{Cond} D. Ferraro, M. Campisi, G. M. Andolina, V. Pellegrini and M. Polini, High-power collective charging of a solid-state quantum battery, Phys. Rev. Lett. 120, 117702 (2018).
\bibitem{Forn} P. Forn-D\'{\i}laz, J. J. Garc\'{\i}la-Ripoll, B. Peropadre, J.-L. Orgiazzi, M. A. Yurtalan, R. Belyansky, C. M. Wilson, and A. Lupascu, Ultrastrong coupling of a single artificial atom to an electromagnetic continuum in the nonperturbative regime, Nat. Phys. 13, 39 (2016).
\bibitem{Lv} Bruzewicz, C.D.; Chiaverini, J.; McConnell, R.; Sage, J.M. Trapped-Ion Quantum Computing: Progress and Challenges. Appl. Phys. Rev. 2019, 6, 021314..
\bibitem{Bau} K. Baumann, C. Guerlin, F. Brennecke, and T. Esslinger, The dicke quantum phase transition with a superfluid gas in an optical cavity, Nature (London) 464, 1301 (2010)
\bibitem{Devoret} Devoret, M.H.; Schoelkopf, R. J. Superconducting Circuits for Quantum Information: An Outlook. Science 2013, 339, 1169
\bibitem{Farin1} D. Farina, G. M. Andolina, A. Mari, M. Polini, and V. Giovannetti, Charger-mediated energy transfer for quantum batteries: Anopen-system approach. Phys. Rev. B 99, 035421 (2019).
\bibitem{Camp} C. Ou, R. V. Chamberlin and S. Abe, Lindbladian operators, von Neumann entropy and energy conservation in time-dependent quantum open systems, Physica A 466, 450 (2017).
\bibitem{Carega} M. Carrega, A. Crescente, D. Ferraro, and M. Sassetti, Dissipative dynamics of an open quantum battery. New J. Phys. 22, 083085 (2020).
\bibitem{Barra} F. Barra, Dissipative charging of a quantum battery, Phys. Rev. Lett. 122, 210601 (2019).
\bibitem{San0} A. C. Santos, Quantum advantage of two-level batteries in
self-discharging process, Phys. Rev. E 103, 042118 (2021).
\bibitem{Pedro} L. P. Garcia-Pintos, A. Hamma, A. del Campo, Fluctuations in extractable work bound the charging power of quantum batteries. Phys. Rev. Lett. 125, 040601 (2020).
\bibitem{Salimi} F. H. Kamian, F. T. Tabesh, S. Salimi, F. Kheirandish, and A. C. Santos, Non-markovian effects on charging and selfdischarging processes of quantum batteries, New J. Phys. 22, 083007 (2020).
\bibitem{Kamin1} F. T. Tabesh, F. H. Kamin, and S. Salimi, Environmentmediated charging process of quantum batteries, Phys. Rev. A 102, 052223 (2020).
\bibitem{Squeezing} F. Centrone, L. Mancino, M. Paternostro, Charging batteries with quantum squeezing, https://doi.org/10.48550/arXiv.2106.07899.
\bibitem{Dark} J. Q. Quach and W. J. Munro, Using dark states to charge and stabilise open quantum batteries, Phys. Rev. Applied 14, 024092 (2020).
\bibitem{Mitch} M. T. Mitchison, J. Goold and J. Prior, Charging a quantum battery with linear feedback control, Quantum 5, 500 (2021).
\bibitem{Shao} Y. Yao and X. Q. Shao, Phys. Rev. E Optimal charging of open spin-chain quantum batteries via homodyne-based feedback control, 106, 014138 (2022).
\bibitem{Ios} S. Borisenok, Ergotropy of quantum battery controlled via target attractor feedback, J. Appl. Phys. 12, 43 (2020).
\bibitem {Borhan} R. R. Rodriguez, B. Ahmadi, G. Suarez, P. Mazurek, S. Barzanjeh, P. Horodecki, Optimal quantum control of charging quantum batteries, arXiv:2207.00094 [quant-ph].
\bibitem{Franc} F. Mazzoncini, V. Cavina, G. M. Andolina, P. A. Erdman and V. Giovannetti, Optimal control methods for quantum batteries, Phys. Rev. A 107 (2023) 032218.
\bibitem{Behzadi} N. Behzadi and H. Kassani, Mechanism of controlling robust and stable charging of open quantum batteries, J. Phys. A: Math. Theor. 55, 425303 (2022).
\bibitem{Yu0} J. L. Li, H. Z. Shen and X. X. Yi, Quantum batteries in non-Markovian reservoirs, Opt. Lett 21, 5614 (2022).
\bibitem{Baris} A. C. Santos, B. \c{C}akmak, S. Campbell and N.T. Zinner, Stable adiabatic quantum batteries, Phys. Rev. E 100, 032107 (2019).
\bibitem{Segal} J. Liu, D. Segal, Boosting quantum battery performance by structure engineering, arXiv:2104.06522 [quant-ph].
\bibitem{Epjp0} J. Taghipour, B. Mojaveri and A. Dehghani, Witnessing entanglement between two two-level atoms coupled to a leaky cavity via two-photon relaxation, Eur. Phys. J. Plus 137, 772 (2022).
\bibitem{morteza0} A. Mortezapour, M. A. Borji, and R. L. Franco, Protecting entanglement by adjusting the velocities of moving qubits inside non-Markovian environments, Laser Phys. Lett 14, 055201 (2017).
\bibitem{Chao0} W. Chao and F. Mao-Fa, The entanglement of two moving atoms interacting with a single-mode field via a three-photon process, Chin. Phys. B 19, 020309 (2010).
\bibitem{sare0} S. Golkar and M. K. Tavassoly and A. Nourmandipour, Entanglement dynamics of moving qubits in a common environment, J. Opt. Soc. Am. B 37, 400 (2020).
\bibitem{Golkar1} S. Golkar and M. K. Tavassoly And A. Nourmandipour, Qubit movement-assisted entanglement swapping, Chin. Phys. B. 29, 050304 (2020).
\bibitem{Epjp1} B. Mojaveri, A. Dehghani and J. Taghipour, Control of entanglement, single excited-state population and memory-assisted entropic uncertainty of two qubits moving in a cavity by using a classical driving field, Eur. Phys. J. Plus 137, 1065 (2022).
\bibitem{MPLA} J. Taghipour, B. Mojaveri and A. Dehghani, Witnessing entanglement between two two-level atoms moving inside a leaky cavity under classical control, Mod. Phys. Lett. A 37, 2250141 (2022).
\bibitem{Wang00} Q. Wang, R. Liu, H. M. Zou, D. Long and J. Wang, Entanglement dynamics of an open moving-biparticle system driven by classical-field, Phys. Scr. 97, 055101, (2022).
\bibitem{Allahverdyan} A. E. Allahverdyan, R. Balian and T. M. Nieuwenhuizen, Maximal work extraction from finite quantum systems. Eur. phys. Lett 67, 565 (2004).
\bibitem{Franc0} G. Francica, J. Goold, F. Plastina, and M. Paternostro, Daemonic ergotropy: enhanced work extraction from quantum correlations, npj Quantum Inf. 3, 12 (2017).
\bibitem{Cakmak0} B. \c{C}akmak, Ergotropy from coherences in an open quantum system, Phys. Rev. E 102, 042111 (2020).
\bibitem{Engl} B.G. Englert, J. Schwinger, A.O. Barut and M.O. Scully, Reflecting slow atoms from a micromaser field, Eur. Phys. Lett 14, 25 (1991).
\bibitem{Haro} S. Haroche, M. Brune and J.M. Raimond, Trapping atoms by the vacuum field in a cavity, Eur. Phys. Lett 14, 19 (1991).
\bibitem{Lenard} C. Leonardi and A. Vagliea, Non-markovian dynamics and spectrum of a moving atom strongly coupled to the field in a damped cavity, Opt. Commun 97, 130 (1993).
\bibitem{mortezapour} F. Nosrati, A. Mortezapour and R. Lo Franco, Validating and controlling quantum enhancement against noise by the motion of a qubit, Phys. Rev. A. 101, 012331 (2020).
\bibitem{Cook} R. J. Cook, Atomic motion in resonant radiation: An application of Ehrenfest's theorem, Phys. Rev. A. 20, 224 (1979).
\bibitem{Wilkens} M. Wilkens, Z. Bialynicka-Birula and P. Meystre, Spontaneous emission in a Fabry-P\'{e}rot cavity: The effects of atomic motion, Phys. Rev. A. 45, 477 (1992).
\bibitem{Breuer0} H. P. Breuer and F. Petruccione, \textit{The Theory of Open Quantum Systems} (Oxford University Press, Oxford, New York, 2002).
\bibitem{Hood} C. J. Hood et al., The Atom-Cavity Microscope: Single Atoms Bound in Orbit by Single Photons, Science 287, 1447 (2000).
\bibitem{Pinkse} P. W. H. Pinkse et al., Trapping an atom with single photons, Nature 404, 365 (2000).
\bibitem{Jones} P. J. Jones, J. A. M. Huhtam\"{a}ki, K. Y. Tan and M. M\"{o}tt\"{o}nen, Tunable electromagnetic environment for superconducting quantum bits, Sci. Rep. 3, 1987 (2013).
\end{thebibliography}
\end{document}

% ######################################################
% Pairwise Comparisons
% ######################################################
Studying such perceptions has traditionally been carried out using direct rating methods (users assign a score to each event or situation). This procedure requires a well-defined scale and user training and is particularly difficult to conduct when events or conditions substantially differ from one another \cite{perez2017practical}, which is the case when analyzing real-world environments. In contrast, using pairwise comparisons (users compare two situations and choose one of the two) is often simpler and faster to set up, well-suited for non-expert participants \cite{perez2017practical}, and presents lower measurement error compared to direct ratings \cite{shah2015estimation}. With this in mind, we employ pairwise comparisons to analyze cycling safety perceptions. Moreover, we draw current practice and knowledge from other research areas (e.g., sports outcome prediction and preference learning) about pairwise comparisons and how algorithms can be used to study cycling safety perceptions, something unexplored in cycling safety research. This paves the way to scale safety perception studies and ubiquitously understand how individuals perceive cycling risk.


% ######################################################
% Gap, Objectives & Contributions
% ######################################################
The main contributions of this paper are as follows. First, we draw knowledge from other research areas about pairwise comparisons and apply them to studying cycling safety perceptions. This novel approach
uses a survey showcasing images of two road environments and asking users which one they find safer, if any. % We use respondents' answers to compare different methodologies previously applied to sports prediction and preference learning, showcasing how these can be directly applied to our main goal: understanding cycling perception of safety.
With the respondents' answers, we compare different methodologies, previously applied to sports prediction and preference learning, and show how these can be directly applied to our main goal: understanding cycling perception of safety. Lastly, we draw from these results to objectively classify cycling environments based on urban characteristics and cycling environments. 


% ######################################################
% Outline of article
% ######################################################
We divide the article as follows. In the next section, we explore the current literature on pairwise comparisons and how traditional rating methods unravel such data. In Section \ref{sec:survey}, we detail our pairwise comparison survey and present different algorithms to rate cycling environments. Next, in Section \ref{sec:ranking}, we present the methodology, overviewing all pairwise ranking algorithms and environment classification. Section \ref{sec:results} presents the results and highlights what environments are perceived as safer or riskier. Finally, Section \ref{sec:conclusions} concludes the paper and draws possible paths forward.
\section{Predict and optimize}
In this section, we provide an overview of the existing research in the predict and optimize domain. Then, we define the P\&O problem and introduce the solution approach that we are going to investigate later in this manuscript.
\subsection{Related work}
To the best of our knowledge, the predict and optimize framework was first introduced by \citet{Elmachtoub2017}. They consider optimization problems with linear objectives and derive a convex approximation of the task performance function. Then, they optimize the prediction model by using sub-gradients of this approximation. Later, this method was extended onto combinatorial problems by \citet{emir2019}. Several other approximations were introduced in other studies focusing on combinatorial problems. \citet{Vlastelica2019} derive a differentiable piecewise-linear approximation for the task performance; \citet{Berthet2020-rp} employ stochastic perturbations to approximate derivative of combinatorial problems.

Unlike in the combinatorial case, continuous convex optimization problems do allow exact differentiation of the loss function.  The sequence of works \citep{Amos2017}, \citep{Agrawal2019cp}, \citep{Agrawal2019} developed a differential optimization technique to compute the derivative of convex optimization problems. In their latest work \citep{Agrawal2019}, the authors delivered a general method that allows differentiating disciplined convex programs \citep{Grant2006}. This result gave rise to new applications of P\&O to convex optimization: \citet{Uysal2021-el} applied convex differential optimization to the risk budgeting portfolio optimization problem; \citet{Wang2020} utilized it to learn surrogate models for predict and optimize; \citet{Donti2017} applied the method to three different real-world benchmarks. Moreover, several studies applied differential optimization to predict and optimize for other problem classes. In \cite{Wilder2019-ai}, it was used in linear optimization via constructing a quadratic approximation of the problem. Later, \citet{Mandi2020} improved upon this result by using logarithmic approximations. \citet{Ferber2020-aj} combined a similar idea with the cutting plane approach and used differential optimization in combinatorial problems.

Outside of predict and optimize, differential optimization also has found several applications. \citet{Chen2021} used it to train reinforcement learning agents in the action space with convex constraints, and \citet{Agrawal2019-ru}, employed it for tuning model predictive control algorithms.

While the benefits of the differential optimization approach to predict and optimize are numerous, it is still not fully understood. It was reported in several studies \cite{Vlastelica2019,Wilder2019-ai}, that the gradient of a linear problem is zero everywhere, except for the finite set of points where it is undefined. Since any linear problem is convex, this observation suggests that the gradients of convex problems should be also thoroughly investigated.

\subsection{Problem formulation}
In this section, we introduce the P\&O problem. We refer readers to \cite{Elmachtoub2017} for further details.
In predict and optimize, we solve optimization problems of the form
\begin{equation}
   \argmax_{x} f(x, w) \text{ s. t. } x\in\mathcal{C},
   \label{eq:true-problem}
   \tag{True problem}
\end{equation}
where $x\in\mathbb{R}^n$ is the decision variable, $w\in\mathbb{R}^u$ is a vector of unknown parameters, ${f: \mathbb{R}^n \times \mathbb{R}^u \to \mathbb{R}}$ is the objective function, and $\mathcal{C}$ is the feasibility region. The defining feature of this problem is that the parameters $w$ are unknown at the moment when the decision must be made. Therefore, the true optimization problem is under-defined and cannot be solved directly.

One way to deal with the unknown parameters $w$ is to use a prediction $\hat{w}$ instead. Then, the decision can be computed by solving the following problem, which we refer to as the internal problem:
\begin{equation}
   x^\ast(\hat{w}) = \argmax_{x} f(x, \hat{w}) \text{ s. t. } x\in\mathcal{C}.
   \tag{Internal problem}
\end{equation}
A commonly made assumption is that instead of $w,$ we observe a feature vector $o$ that contains some information about $w.$ Also, we have a  dataset $\mathcal{D}=\{(o_k, w_k)\},$ e.g., of historical data, which we can use to learn the relation between $w$ and $o.$ This setup enables using ML models to compute the prediction. We denote the prediction model by $\phi_\theta$, and thus we have $\hat{w} = \phi_\theta(o)$. 

The problem described above is not specific to predict and optimize. What separates the P\&O paradigm from earlier works is the approach to training the model $\phi_\theta.$ In the past, machine learning models would be trained to predict $w$ as accurately as possible, e.g., in \cite{mukhopadhyay2017prioritized}. However, the parameter prediction error is merely an artificial objective and our true goal is to derive a decision $x$ that maximizes the task performance $f(x, w).$ The main goal of the P\&O approach is to utilize this objective for training the model $\phi_\theta.$ The task performance achieved by $\phi_\theta$ on the dataset $\mathcal{D}$ can be quantified by the following loss function:
\begin{equation}
   L(\theta) = -\frac{1}{|\mathcal{D}|}\sum_{(o, w) \in \mathcal{D}} f\Big(
   x^\ast\big(\phi_\theta(o)\big), w\Big)
   \label{eq:po-loss}
\end{equation}
Most machine learning algorithms for training models are based on computing the gradient of the loss function (\cite{kiefer1952stochastic}).  To train $\phi_\theta$ with a gradient-based algorithm, we need to differentiate $L$ over $\theta,$ and hence we need to compute the gradient $\nabla_{\theta}f\Big(
   x^\ast(\hat{w}), w\Big),$ where $\hat{w}=\phi_\theta(o).$ Applying the chain rule, it can be decomposed into three terms:
\begin{equation}
  \nabla_{\theta}f\Big(x^\ast(\hat{w}), w\Big) = 
  \nabla_{x}f\big(x^\ast(\hat{w}), w\big)\;
  \nabla_{\hat{w}} x^\ast(\hat{w})\;
  \nabla_{\theta} \hat{w}.
  \label{eq:chain-rule}
\end{equation}
The second term, $\nabla_{\hat{w}}x^\ast(\hat{w}),$ is the Jacobian of the solution of the optimization problem over the prediction $\hat{w}.$ An exact method to compute this Jacobian was introduced in \cite{Agrawal2019}, but it has never been thoroughly analyzed. 
In the next section, we show that $\nabla_{\hat{w}}x^\ast(\hat{w})$ has a large null space, thereby causing the total gradient in Eq.~\ref{eq:chain-rule} to be zero even outside of the optimum.
\section{Differentiable optimization}
In this section, we study the derivative of convex optimization programs over the parameters of the objective function. We show that the gradient in Eq.~\ref{eq:chain-rule} is often zero outside of the optimum, and hence it causes gradient-following methods to get stuck in suboptimal solutions. In the second part of this section, we introduce a method to solve this problem. 

Without loss of generality, we consider a single instance of the problem, i.e.,~one sample $(o, w)\in\mathcal{D}.$ Everywhere in this section, we denote the prediction by $\hat{w}=\phi_\theta(o).$ Then, the decision is computed as a solution of the internal optimization problem defined as follows:
\begin{equation}
    x^\ast(\hat{w}) = \argmax_{x} f(x, \hat{w}) \text{ s.t. } x\in\mathcal{C}.
    \label{eq:int}
\end{equation}
We use $\hat{x}$ to denote the value of $x^\ast(\hat{w})$ for a given prediction $\hat{w}.$ As we are interested in convex optimization problems, we make the following assumptions:
% f(x) is convex
\begin{assumption} 
The objective function $f(x, w)$ is concave and twice continuously differentiable in $x$ for any ${w}.$ \end{assumption}
% C is convex, i.e., it is defined as g_i(x) <= 0 for some convex g_i.
\begin{assumption}
    The feasibility region $\mathcal{C}$ is convex, i.e., $\{\mathcal{C}=\{x|g_i(x)\leq 0, i=1,\dots,l\},$ where $g_i(x)$ are convex differentiable functions. Moreover, for any $x\in\mathcal{C},$ the gradients $\{\nabla_{x}g_i(x)|g_i(x)=0\}$ of the active constraints are linearly independent. \footnote{As is, Assumption 2 does not allow equality constraints. For clarity, we use this formulation in the main body of the paper. In the appendix, we show that our results hold for the equality constraints as well.}
\end{assumption}
Additionally, we make an assumption about how $f$ depends on $w$, which holds for many real-world problems, including linear and quadratic optimization problems.

\begin{assumption}
    The objective function $f(x, w)$ is twice continuously differentiable in $w.$
\end{assumption}

Throughout this paper, we use derivatives of different objects. For clarity, we first provide an overview of them: the gradient of the true objective function over the decision, $\nabla_{x} f(\hat{x},w);$ the Jacobian of the decision over the prediction, $\nabla_{\hat{w}}x^\ast(\hat{w});$ the Jacobian of the prediction over the ML model parameters, $\nabla_{\theta}\hat{w};$ and the gradient of the predicted objective in the internal problem, $\nabla_x f(x, \hat{w}).$ In the next section, we establish some crucial properties of the Jacobian $\nabla_{\hat{w}}x^\ast(\hat{w}).$

\subsection{The zero-gradient theorem}
We begin by investigating the relation between the values of the function $x^\ast(\hat{w})$ and the gradient of the internal objective, $\nabla_{x}f(x, \hat{w}).$
Let $n_i:=\nabla_{x}g_i(\hat{x}),\,i=1,\dots,\,l$ be the normal vectors of the constraints at $\hat{x},$ Then, the KKT conditions \cite{kkt} at $\hat{x}$ state that there exist real values $\alpha_1,\ldots,\alpha_l$ such that the following holds:
\begin{equation*}
\nabla_{x}f(\hat{x}, \hat{w}) = \sum_{i=1}^{l}\alpha_in_i,\quad
\alpha_ig_i(\hat{x})=0, \quad \alpha_i \geq 0, \quad g_i(\hat{x})\leq0,\quad i=1,\dots,l.    
\end{equation*}

%$$g_i(\hat{x})\leq0, i=1,\dots,l.$$
Under Assumptions 1 and 2, the KKT multipliers $\alpha_i$ are uniquely defined by $\hat{w}$ and $\hat{x}.$ Thus, as $\hat{x}$ is defined by $\hat{w},$ we sometimes write $\alpha_i(\hat{w})$ to emphasize that it is, in fact, a function of $\hat{w}.$
To provide a geometrical perspective on the KKT conditions, we introduce the following definition:

\begin{definition} 
Let $x\in\mathcal{C}$ and let ${I(x)=\{i|g_i(x)=0\}}$ be the set of indices of the constraints active at $x.$ Let $n_i=\nabla_{x} g_i(x),\,\forall i\in I(x)$, be the normal vectors of these constraints. The \textnormal{gradient cone},
$ G(x):=\Big\{\sum_{i\in I}\alpha_in_i|\alpha_i\geq 0 \Big\}, $ 
is the positive linear span of normal vectors $n_i.$
\end{definition}

Combining the KKT conditions with Definition 3.1, we immediately arrive at the following property:
% For x inside C, G(x) is degenerate

\begin{property}
    Let $x\in\mathcal{C}$ and let $\nabla_{x}f(x, \hat{w})$ be the internal gradient at $x.$ Then, $x$ is a solution to the problem in Eq. \ref{eq:int} if and only if \,$\forall i \in I(x),\exists \alpha_i\geq 0,$ such that 
    $\nabla_{x}f(x,\hat{w})=\sum_{i\in I(x)} \alpha_in_i\in G(x),$
    where $I(x)$ is the set of indices of active constraints, $I(x)=\{i|g_i(x)=0\}.$
\end{property}

While trivial, this property provides a geometrical interpretation of the problem. Effectively, a point $x$ is a solution to the problem in Eq. \ref{eq:int} if and only if the internal gradient at this point lies inside its gradient cone. Figure \ref{fig:zg_cone} illustrates this property. 

Before studying the Jacobian $\nabla_{\hat{w}}x^\ast(\hat{w}),$ we first need to address the question of when this Jacobian exists. Sufficient conditions for existence are given in \citet{fiacco1976sensitivity}. Under Assumptions 1-3, these conditions can be reformulated as follows:

\begin{lemma}[Theorem 2.1 in \citet{fiacco1976sensitivity}]
    Let Assumptions 1-3 hold and let 
    \begin{equation*}
    \nabla_{x}f(\hat{x},\hat{w})=\sum_{i\in I(\hat{x})}\alpha_i (\hat{w})n_i
    %,\quad\alpha_i(\hat{w})\geq0, \;\forall i\in I(\hat{x})\,,
    \end{equation*}
    be the representation of the internal gradient with the normals of the active constraints. Then, suppose that the \textnormal{strict complementary slackness condition} holds, i.e., $\alpha_i(\hat{w})>0,\,\forall i\in I(\hat{x}).$
    Then, the Jacobian $\nabla_{\hat{w}}x^\ast(\hat{w})$ exists at $\hat{w}.$ Moreover, $\alpha_i(\cdot)$ is continuous around $\hat{w}$ for any $i\in I(\hat{x}).$
\end{lemma}

%\begin{wrapfigure}{l}{0.5 \textwidth}
% Figure environment removed
%\end{wrapfigure}

Proof of this lemma is given in \citet{fiacco1976sensitivity}. This result establishes that strict complementary slackness is sufficient for the Jacobian $\nabla_{\hat{w}}x^\ast(\hat{w})$ to exist. 
In most cases, the points that violate strict complementary slackness form a zero-measure set and hence can be neglected in practice. 

Now, we have all the necessary tools to describe the structure of the Jacobian $\nabla_{\hat{w}}x^\ast(\hat{w}).$
Suppose that the strict complementary slackness condition holds at $\hat{x}$ and hence the Jacobian exists. 
Assume that we perturb $\hat{w}$ and obtain $\hat{w}'.$ Let  $\hat{x}'=x^\ast(\hat{w}')$ denote the solution corresponding to $\hat{w}'.$ What can be said about $\hat{x}'?$ Strict complementary slackness implies that the constraints active at $\hat{x}$ will remain active at $\hat{x}'$ if the difference $\|\hat{w}' - \hat{w}\|_2^2$ is small enough. Therefore, the decision $\hat{x}'$ can only move within the tangent space of $\mathcal{C}$ at $\hat{x}$, i.e., orthogonally to all $n_i,\, i\in I(\hat{x}.)$  Hence, when more constraints are active, $\hat{x}'$ can move in less directions. Formally, we obtain the following lemma:

\begin{lemma}
Suppose that the strict complementary slackness conditions hold at $\hat{x}$ and let $\nabla_{x}f(\hat{x}, \hat{w})=\sum_{i\in I(\hat{x})}\alpha_in_i,$ $ \alpha_i> 0,\; \forall i \in I(\hat x)$
be the internal gradient. Let 
$\mathcal{N}(\hat{x})=span(\{n_i \,|\, i\in I(\hat{x})\})$
be the linear span of the gradient cone.
Then $\mathcal{N}(\hat{x})$ is contained in the left null space of $\nabla_{\hat{w}} x^\ast(\hat{w}),$ i.e., $v\,\nabla_{\hat{w}}x^\ast(\hat{w})=0,\,\forall v\in\mathcal{N}(\hat{x})$
\end{lemma}
The formal proof of this result can be found in the appendix. Lemma 3.4 is very important, as it specifies in what directions $x^\ast(\hat{w})$ \textit{can move} as a consequence of changing $\hat{w}.$ Now, the first term in the chain rule in Eq. \ref{eq:chain-rule}, $\nabla_{x}f(\hat{x}, w),$ specifies in what directions $x^\ast(\hat{w})$ \textit{should} move in order for the true objective to increase. Naturally, if these directions are contained in the null space of $\nabla_{\hat{w}}x^\ast(\hat{w}),$ then the total gradient in Eq.~\ref{eq:chain-rule} is zero. This observation constitutes the main theorem of this paper -- the zero-gradient theorem.

% THE THEOREM!
\begin{theorem}[Zero-gradient theorem] Let $\hat{w}$ be a prediction, and let $\hat{x}$ be the solution of the internal optimization problem defined in Eq. \ref{eq:int}. Suppose that the strict complementary slackness conditions hold at $\hat{x}$ and let
$\mathcal{N}(\hat{x})=span(\{n_i \,|\, i\in I(\hat{x})\})$
be the linear span of the gradient cone at $\hat{x}.$
Then, 
$\nabla_{x} f(\hat{x}, w)\in\mathcal{N}(\hat{x})\implies
\nabla_{\theta} f(\hat{x}, w) = 0.$
\end{theorem}

The proof of this theorem is obtained by simply applying Lemma 3.4 to the chain rule in Eq.~\ref{eq:chain-rule}. 
The theorem claims that the gradient of the P\&O loss in Eq. \ref{eq:po-loss} can be zero in the points outside of the optimal solution. Hence, any gradient-following method ``shall not pass'' these points.
In particular, the zero-gradient phenomenon happens in such points $\hat{x}$ where the true gradient $\nabla_{x} f(\hat{x}, w)$ is contained in the space $\mathcal{N}(\hat{x})$ spanned by the gradient cone $G(\hat{x}).$ As the dimensionality of this space grows with the number of active constraints, the zero-gradient issue is particularly important for problems with a large number of constraints. 
In the worst case, $\mathcal{N}(\hat{x})$ can be as big as the whole decision space $\R^n,$ thereby making the total gradient $\nabla_{\theta}f(\hat{x}, w)$ from Eq.~\ref{eq:chain-rule} zero for any value of the true gradient $\nabla_{x}f(\hat{x}, w)$.
In the following sections, we introduce a method that resolves the zero-gradient problem and provides theoretical guarantees for its performance.

\subsection{Quadratic programming approximation}
The fundamental assumption of the predict and optimize framework is that training $\phi_{\theta}$ using the task performance loss is better than fitting it to the true values of $w.$ Hence, the models trained with predict and optimize might output $\hat{w}$ that is significantly different from the true $w$ and yet produces good decisions. Taking this argument one step further, we claim that the objective function $f(x, \hat{w})$ in the internal optimization problem in Eq. \ref{eq:int} does not need to be the same as the true objective $f(x, w).$ In particular, we suggest computing decisions using a simple quadratic program (QP):
\begin{equation}
    x^\ast_{QP}(\hat{w}) = \argmax_{x} - \|x-\hat{w}\|^2_2 \text{ s.t. } x\in\mathcal{C}.
    \label{eq:qp}
\end{equation}

The reasons for this choice are manyfold. First, the internal objective $f_{QP}(x, \hat{w})=- \|x-\hat{w}\|^2_2,$ is strictly concave and hence $x^\ast_{QP}(\hat{w})$ is always uniquely-defined. Moreover, the range of $x_{QP}(\hat{w})$ is $\mathcal{C},$ i.e., $
\forall x\in\mathcal{C},\,\exists \hat{w}$ such that $x=x^\ast_{QP}(\hat{w}).$ Hence, it can represent any optimal solution. However, the most important property of QP is that its Jacobian is very simple, which we explain below.

The problem in Eq.~\ref{eq:qp} has a simple geometrical interpretation: the point $x=\hat{w}$ is the unconstrained maximum of $f_{QP}(x, \hat{w})$ and $x^\ast_{QP}(\hat{w})$ is its Euclidean projection on the feasibility set $\mathcal{C},$ see Figure~\ref{fig:qp}. To compute the Jacobian $\nabla_{\hat{w}}\,x^\ast_{QP},$ we need to understand how perturbations of $\hat{w}$ affect $x^\ast_{QP}.$ Employing the geometrical intuition above, we obtain the following lemma:

\begin{lemma}
    Let $\hat{w}$ be a prediction and $\hat{x}$ be the optimal solution of the QP problem defined in Eq. \ref{eq:qp}. Let the strict complementary slackness condition hold and let $\{n_i|i\in I(\hat{x})\}$ be the normals of the active constraints. Let
    $
        \{e_j|j=1,\ldots,n-|I(\hat{x}|) \}
    $
    be an orthogonal complement of vectors $\{n_i| i \in I(\hat{x})\}$ to a basis of $\R^n.$ Then, the representation of the Jacobian $\nabla_{\hat{w}}x_{QP}(\hat{w})$ in the basis $\{n_i\}\cup \{e_j\}$ is a diagonal matrix. Its first $|I(\hat{x})|$ diagonal entries are zero, and the others are one.
\end{lemma} 

Proof of this lemma can be found in the appendix. Lemma 3.6 implies that the Jacobian $\nabla_{\hat{w}}x_{QP}(\hat{w})$ has a simple form and can be easily computed by hand. While providing computational benefits, this approach does not address the zero-gradient problem. In the next section, we introduce a method to compute an approximate of the Jacobian $\nabla_{\hat{w}}x_{QP}(\hat{w})$ that has a strictly one-dimensional null space. Combined with the QP approximation, it is guaranteed to at least not decrease the task performance.

\subsection{Local smoothing}
% Figure environment removed
We identified a fundamental issue of differential optimization -- the zero-gradient problem. We showed that the null space of the Jacobian $\nabla_{\hat{w}}x(\hat{w})$ depends on the number of constraints active at $\hat{x}.$ Generally, this number can be as large as the number of optimized variables $n,$ and the gradient-descent algorithms can get stuck in certain points on the boundary of the feasibility region.

%A potential solution to this issue is to approximate the feasibility region $\mathcal{C}$ in such a way that all gradient cones become one-dimensional (see Figure \ref{fig:smooth-c}). 
%It is known that any convex set can be approximated with a convex polytope \cite{bronstein2008approximation}, and a convex polytope can be approximated with a smooth convex set \cite{ghomi2004}. By combining these results we could approximate $\mathcal{C}$ with a smooth $\mathcal{C}'$. Then, the null space of the Jacobian of the resulting problem over $\mathcal{C}'$ would be strictly one-dimensional. Therefore, it is possible to derive an arbitrarily close approximation of the problem that suffers from the zero-gradient problems much less. 

% Theoretically, this approach seems promising but it might be hard to use in practice. 
In this section, we propose a simple way to modify the feasibility region -- we smooth $\mathcal{C}$ locally around the point for which we compute the Jacobian, thereby ensuring that its null space becomes one dimensional. First, we define a method for the general setup, without imposing any assumptions on the optimization problem. Then, we demonstrate that combined with the QP approximation from Section 3.2, this smoothing approach has theoretical guarantees.

We begin with the general case -- the problem in Eq. \ref{eq:int}.
%Assume that it has a unique solution for any $\hat{w}.$ Then, let $\hat{w}$ be a prediction and let $\hat{x}$ denote the optimal decision.
Let
$\nabla_{x}f(\hat{x},\hat{w})=\sum_{i\in I(\hat{x})}\alpha_in_i$ be the internal gradient at $\hat{x}$ for some $\alpha_i\geq 0,\; \forall i \in I(\hat x).$ Then, we introduce the following definition:
\begin{definition}
    Let $r>0$ be a positive real number. Let $c=\hat{x} - r\frac{\nabla_{x}f(\hat{x},\hat{w})}{\|\nabla_{x}f(\hat{x},\hat{w})\|_2}.$
    \textnormal{The local $r$-smoothed feasibility region},
    $\mathcal{C}_r(\hat{x},\hat{w}):=\{y|y\in\R^n, \|y - c\|_2\leq r\},$
    is a ball of radius $r$ around $c.$
    \textnormal{The local $r-$smoothed problem} $P_r(\hat{x}, \hat{w})$ with parameters $\hat{x}, \hat{w}$ is defined as 
    ${x^\ast_r(\hat{w}):=\argmax_{x\in\mathcal{C}_r(\hat{x}, \hat{w})}f(x, \hat{w}).}$
\end{definition}
Figure \ref{fig:qp} shows an example of the local $r-$smoothed problem. Now, let $\hat{x}_r=x^\ast_r(\hat{w})$ denote the solution of $P_r(\hat{x}, \hat{w})$. By construction, the internal gradient at $\hat{x}_r$ lies in the one-dimensional gradient cone, and hence, by Property 3.2, $\hat{x}_r=\hat{x}.$
The main purpose of smoothing is to approximate the gradient in Eq. \ref{eq:chain-rule} by substituting $\nabla_{\hat{w}}x^\ast(\hat{w})$ with $\nabla_{\hat{w}}x^\ast_r(\hat{w}).$ We highlight that the decisions are still computed using the non-smoothed problem $x^\ast(\hat{w})$ and $x^\ast_r(\hat{x}, \hat{w})$ is used exclusively to perform the gradient update step. In other words, we use the following expression to compute the gradient:
\begin{equation}
    \nabla_{\theta}f(x^\ast(\hat{w}), w) \approx \nabla_{x}f\big(\hat{x}, w\big)\; \nabla_{\hat{w}}x^\ast_r(\hat{w})\; \nabla_{\theta}\hat{w}
\end{equation}
It is worth mentioning that the strict complementary slackness in the original problem is a stronger condition than the strict complementary slackness on $P_r(\hat{x}, \hat{w}).$ Therefore, the Jacobian of the $r-$smoothed problem can exist even for predictions $\hat{w}$ where the true Jacobian does not.

Generally, the efficiency of $r-$smoothing depends on the form of the internal problem in Eq. \ref{eq:int}. Below, we show that combining $r-$smoothing with the QP approximation has guarantees on its performance. First, we notice that Lemma 3.6 prescribes the Jacobian of the $r-$smoothed QP problem:
\begin{property}
Let $\hat{x}=x^\ast_{QP}(\hat{w})$ be a decision derived via QP. Suppose that the complementary slackness conditions hold for $P_r(\hat{x}, \hat{w})$ and let $e_1=\nabla_{x}f_{QP}(\hat{x}, \hat{w})$ be the internal gradient. Let $\{e_2,\ldots,e_{n}\}$ be a complement of $e_1$ to an orthogonal basis of $\R^n.$
Then, the Jacobian
$\nabla_{\hat{w}}x^\ast_r(\hat{w})$ of the local $r-$smoothed problem expressed in the basis $\{e_1, e_2,\ldots,e_{n}\}$ is a diagonal matrix. Its first entry is zero, others are ones.
\end{property}
As $C_r(\hat{x}, \hat{w})$ is defined by a single constraint, the null space of $\nabla_{\hat{w}}x^\ast_r(\hat{x},\hat{w})$ is always one-dimensional. Hence, the zero-gradient problem can only occur when the internal gradient $\nabla_{x}f_{QP}(\hat{x}, \hat{w})$ and the true gradient $\nabla_{x}f(\hat{x}, w)$ are exactly collinear. Hence, we expect $r-$smoothing to significantly improve upon the zero-gradient problem. Next, we show that the $r-$smoothed Jacobian is actually a good approximation. In the following theorem, we demonstrate that the local $r-$smoothing of the QP approach indeed yields a ``good'' direction for the gradient steps.
\begin{theorem}
    Let $\hat{x}=x^\ast_{QP}(\hat{w})$ be the decision obtained via QP and let $\nabla_{\hat{w}}x^\ast_r(\hat{w})$ be the Jacobian of the $r-$smoothed QP problem. Let $\Delta\hat{w}=\nabla_{x}f(\hat{x},w)\;\nabla_{\hat{w}}x^\ast_r(\hat{w})$ be the prediction perturbation obtained by using this Jacobian and let $\hat{w}'(t)=\hat{w} + t\Delta\hat{w}$ be the updated prediction.
    Then, for $t\to0^+,$ using $\hat{w}'(t)$ results in a non-decrease in the task performance. In other words,
    $f\big(x^\ast_{QP}(\hat{w}'(t)), w\big)\geq f\big(x^\ast_{QP}(\hat{w}), w\big).$
\end{theorem}
Interestingly, this result does not depend on $r.$ However, this is to be expected -- no matter the radius of $\mathcal{C}_r,$ the Jacobian of $P_r(\hat{x}, \hat{w})$ is still the same by Lemma 3.6. 
Theorem 3.10 shows that using $r-$smoothing together with the QP approximation results in analytically computable Jacobian that has a strictly one-dimensional null space. Therefore, we are much less likely to encounter the zero-gradient problem when using this approximation. 
However, the resulting one-dimensional null space contains
the only direction that can move the prediction $\hat{w}$, and hence the decision $\hat{x},$ inside $\mathcal{C}$. This might become crucial, for example, when the optimal solution with respect to the true objective lies in the interior of $\mathcal{C}.$ To resolve this problem, we use the projection distance regularization method first suggested in \cite{Chen2021}. Specifically, we add a penalty term
\begin{equation}
    p(\hat{w}) = \alpha\|\hat{x} -\hat{w}\|_2^2,
    \label{eq:reg}
\end{equation}
where $\alpha\in\R^+$ is a hyperparameter. Minimizing this term, we push $\hat{w}$ along the null-space of the Jacobian towards the feasibility region and eventually move $\hat{x}$ inside $\mathcal{C}.$
\subsection{The training process}
In this section, we summarize the results of Sections 3.1-3.3 and describe the final algorithm we use to solve the P\&O problems. For each problem instance $(o, w),$ we first compute the prediction, $\hat{w}=\phi_{\theta}(o),$ and the decision using the QP approximation method, $\hat{x}=x^\ast_{QP}(\hat{w}).$ Then, we obtain the achieved objective value, $f(\hat{x}, w).$ During training, we update the model parameters $\theta$ by performing the steps described in Algorithm \ref{alg:training}.

\begin{algorithm}[h!]
\setstretch{1.4}
\caption{}
\label{alg:training}
\begin{algorithmic}
\For{$(o,w)\in\mathcal{D}$}
 \State $\hat{x}\gets x_{QP}^\ast\big(\phi(o)\big)$\Comment{Compute the decision}
 \State $f_x\gets \nabla_{x}f(\hat{x}, w)$\Comment{Compute the true gradient}
 \State $\hat{f}_x\gets\nabla_{x}f(\hat{x}, \hat{w})$\Comment{Compute the internal gradient}
 \State $f^0 \gets \frac{f_x^\top \hat{f}_x}{\|\hat{f}_x\|_2}$ \Comment{Project the true gradient on the null space of $\nabla_{\hat{w}}x_r^\ast(\hat{w})$}
 \State $\Delta\hat{w}\gets f_x\,\nabla_{\hat{w}}x^\ast_r(\hat{w})= f_x - f^0.$ \Comment{Compute the prediction perturbation}
 \State $\Delta\hat{w}^{reg}\gets2\alpha(\hat{x}-\hat{w})$\Comment{Compute the anti-gradient of the penalty from Eq. \ref{eq:reg}}
 \State$\Delta\theta \gets (\Delta\hat{w} + \Delta\hat{w}^{reg}) \nabla_{\theta}\,\phi_\theta(o)$\Comment{Approximate the total gradient}
 \State $\theta \gets \theta + \eta \Delta\theta$\Comment{Perform the gradient step of size $\eta$}
 \EndFor
\end{algorithmic}
\end{algorithm}


% In this section, we demonstrate the effectiveness of $\TextMani$ in scarce data regimes.
We evaluate $\TextMani$ in various cases presenting sparse data with 
% two
different tasks: long-tail classification in \Sref{sec4.1}, evenly distributed scarce data classification in \Sref{sec4.2}, and few-shot object detection in \Sref{sec4.3}.
We also conduct additional studies demonstrating the effectiveness of the design of our method and the versatility of $\TextMani$
% and ablation study on attributes
% regarding the sampling distribution and the minimum of the
% number of attribute samples for feature manipulation and
% mixing weight $\alpha$ range 
in \Sref{sec4.4}.
Additional experimental results and details can be found in the supplementary material.



\begin{table}
\centering
\resizebox{1.0\linewidth}{!}{
    \footnotesize
    \begin{tabular}{@{}l C{16mm}C{16mm}C{16mm}@{}}
        \toprule
        \multirow{2}[2]{*}{\textbf{(a) Augmentation}} & \multicolumn{3}{c}{\textbf{Imbalance Factor (IF)}} \\ 
        \cmidrule{2-4}
        & \textbf{100} & \textbf{50} & \textbf{10} \\ 
        \midrule
        Baseline           & 38.39 & 43.33 & 59.29 \\
        $\TextMani$ (CLIP) & 40.65 (\blue{+2.26}) & 46.48 (\blue{+3.15}) & 60.17 (\blue{+0.88}) \\ 
        $\TextMani$ (BERT) & 41.10 (\blue{+2.71}) & \textbf{47.17 (\blue{+3.84})} & 60.67 (\blue{+1.38}) \\
        $\TextMani$ (GPT-2) & \textbf{41.20 (\blue{+2.81})} & 46.93 (\blue{+3.60}) & 60.94 (\blue{+1.65}) \\
        \midrule
        Cutout~\cite{devries2017improved}      & 37.51 & 42.28 & 59.26 \\
         + $\TextMani$    & 40.35 (\blue{+2.84}) & 45.48 (\blue{+3.20}) & \textbf{61.31 (\blue{+2.05})} \\
        \cmidrule{1-4}
        Cutmix~\cite{yun2019cutmix}      & 37.93 & 43.34 & 59.30\\
         + $\TextMani$    & 40.22 (\blue{+2.29}) & 45.36 (\blue{+2.02}) & 61.30 (\blue{+2.00})\\
        \cmidrule{1-4}
        Mixup~\cite{zhang2017mixup}       & 36.75 & 40.77 & 57.50 \\
         + $\TextMani$     & 38.40 (\blue{+1.65}) & 43.33 (\blue{+2.56}) & 59.80 (\blue{+2.30})\\
        \cmidrule{1-4}
        ManiMixup~\cite{verma2019manifold}   & 35.72 & 40.51 & 55.26 \\ 
         + $\TextMani$ & 38.60 (\blue{+2.88}) & 43.22 (\blue{+2.71}) & 59.35 (\blue{+4.09})\\
        \midrule
        \midrule
        \multirow{2}[2]{*}{\textbf{(b) Augmentation}} & \multicolumn{3}{c}{\textbf{Set of Classes (IF=100)}} \\
        \cmidrule{2-4}
        & \textbf{Many} & \textbf{Medium} & \textbf{Few}  \\ 
        \midrule
        Baseline           & 71.11 & 38.42 & 3.00 \\
        $\TextMani$ (CLIP) & 71.14 (\blue{+0.03}) & 40.28 (\blue{+1.86}) & 7.53 (\blue{+4.53}) \\
        $\TextMani$ (BERT) & 70.22 (\orange{-0.89}) & 40.73 (\blue{+2.31}) & 9.41 (\blue{+6.41}) \\
        $\TextMani$ (GPT-2) & 70.60 (\orange{-0.51}) & 40.61 (\blue{+2.19}) & \textbf{9.93 (\blue{+6.93})} \\
        \midrule
        Cutout               & 71.54 & 35.94 & 1.06 \\
        + $\TextMani$ & 71.94 (\blue{+0.83}) & \textbf{40.97 (\blue{+2.55})} & 4.03 (\blue{+3.03}) \\
        \cmidrule{1-4}
        Cutmix                & 72.02 & 37.17 & 0.90 \\
         + $\TextMani$  & 72.37 (\blue{+0.35}) & 40.80 (\blue{+3.63}) & 3.90 (\blue{+3.00}) \\
        \cmidrule{1-4}
        Mixup               & 71.97 & 33.62 & 0.36 \\
         + $\TextMani$ & 71.97 (+0.00) & 36.77 (\blue{+3.15}) & 1.83 (\blue{+1.47})\\
        \cmidrule{1-4}
        ManiMixup               & 72.97 & 29.51 & 0.70 \\
         + $\TextMani$ & \textbf{73.20 (\blue{+0.23})} & 36.80 (\blue{+7.29})& 0.76 (\blue{+0.06})\\
        \bottomrule
    \end{tabular}
    }
    \caption{Long-tail classification results (\%) on CIFAR-100-LT with ResNet18.
    (a) The accuracy with respect to the different imbalance factors, \ie, IF=$\{100, 50, 10\}$.
    (b) The accuracy of each class set with IF=$100$.
    Baseline contains random horizontal flip, random crop and rotation, and normalization, applied in all experiments.
    $\TextMani$ without parenthesis uses CLIP for the text encoder.
    }
    \label{tab:CIFAR-100-LT}
    \vspace{4mm}
% \end{table}
% \begin{table}
    \centering
    \resizebox{1.0\linewidth}{!}{\scriptsize
    \begin{tabular}{@{\,\,}l@{\quad}C{5mm}C{9mm}C{9mm}C{9mm}c@{\,\,}}
         \toprule
         \textbf{Aug.} & \textbf{CBS} & \textbf{All} & \textbf{Many} & \textbf{Medium} & \textbf{Few}\\
         \midrule
         Baseline  &            & 38.39 & 71.11 & 38.42 & 3.00 \\
         Cutmix    & \checkmark & 38.23 & \textbf{71.77} & 37.79 & 1.90 \\
         Mixup     & \checkmark & 38.73 & 71.60 & 37.64 & 3.16 \\ 
         ManiMixup & \checkmark & 38.56 & 71.25 & 37.88 & 2.80 \\
         \TextMani & & \textbf{40.65} & 71.14 & \textbf{40.28} & \textbf{7.53} \\
         \bottomrule
    \end{tabular}
    }
    % \vspace{-3mm}
    \caption{Comparison 
    % of Top-1 accuracy (\%) 
    to label mix-based augmentations with class-balanced sampling (CBS) on CIFAR-100-LT with IF=100. 
    % with ResNet18.
    CBS samples two classes first and then samples data in each classes.
    }
    \label{tab:balanced_sampling}
    % \vspace{-3mm}
\end{table}


% textmani for all(Many,mid,few) groups, 30 sampling
% \begin{table}
% \centering
% \resizebox{1.0\linewidth}{!}{
%     \footnotesize
%     \begin{tabular}{@{}ll C{11mm}C{11mm}C{11mm}C{11mm}@{}}
%         \toprule
%         \multirow{2}[2]{*}{\textbf{Method}} &\multirow{2}[2]{*}{\textbf{Aug.}} & \multicolumn{3}{c}{\textbf{Set of Classes}} & \multirow{2}[2]{*}{\textbf{All}} \\ 
%         \cmidrule{3-5}
%         && \textbf{Many} & \textbf{Medium} & \textbf{Few} & \\ 
%         \midrule
%         \multirow{2}{*}{LWS~\cite{kang2019decoupling}}
%           & Baseline    & \textbf{67.65} & 37.52 & 6.03 & 44.84\\
%           & $\TextMani$ & 67.00 & \textbf{39.32} & \textbf{9.49} & \textbf{45.92}\\
%         \midrule 
%         \multirow{2}{*}{cRT~\cite{kang2019decoupling}}
%           & Baseline    & \textbf{67.58} & 38.74 & 9.30 & 45.84 \\ 
%           & $\TextMani$ & 66.91 & \textbf{40.86} & \textbf{13.01} & \textbf{47.10} \\
%           % & $\TextMani$ & 63.61 & \textbf{47.30} & \textbf{24.71} & \textbf{50.50} \\
%         \bottomrule
%     \end{tabular}
% }
% \caption{Long-tail classification accuracy (\%) on ImageNet-LT with ResNext50.
% % LWS represents weight scale learning method, and cRT is a decoupled classifier learning method.
% Baseline stands for the basic augmentation with random horizontal flip and random resized crop.
% % The results show the impact of each augmentation in terms of class sets.
% % \textbf{Bold} indicates the best results in each class set.
% % ResNet-18  
% The models are trained with the batch size of 128.
% }
% \label{tab:ImageNet-LT}
% \end{table}




\subsection{Long-tail Classification}\label{sec4.1}
\paragraph{Experimental Setting}
We compare $\TextMani$ with the mix-based augmentations on the CIFAR-100-LT~\cite{cui2019class} and ImageNet-LT~\cite{liu2019large} datasets, where LT stands for long-tailed distribution.
% The long-tail datasets
They are artificially truncated to have a long-tail from each original dataset, CIFAR-100~\cite{krizhevsky2009learning} and ImageNet-2012~\cite{deng2009imagenet}.
Long-tail datasets usually have three sets of classes: Many-shot (more than 100 images), Medium-shot (20-100 images), and Few-shot (less than 20 images).

For CIFAR-100-LT, we 
% can
control the imbalance factor (IF) \cite{chu2020feature} computed as the ratio of samples in the head to tail class, $N_1/N_K$, where $N_k=\left| \mathcal{D}_k\right|$, and $\mathcal{D}_k$ is the set of samples belonging to the class $k\in\{1,\cdots,K\}$.
% $IF = \max(\{N_i\})/\min(\{N_i\})$, where $N_i$ is the number of training samples of the $i$-th class.
A larger value of the IF represents a more severe imbalance in data, which is more challenging.
We evaluate the performance according to different IFs of 100, 50, and 10.
% We use 100, 50, and 10 for the IF, confirming the performance along the IF value.

We utilize ResNet18 as the baseline on CIFAR-100-LT and ResNext50 on ImageNet-LT.
We use the validation set of the original datasets to measure the Top-1 accuracy.
Note that we apply each augmentation on all the samples without carefully selecting a set of classes in \Tref{tab:CIFAR-100-LT}.




\paragraph{Results}
\moon{\Tref{tab:CIFAR-100-LT} presents the long-tail classification results on CIFAR-100-LT,
% of $\TextMani$, mix-based augmentations, and their combination with $\TextMani$
% are in \Tref{tab:CIFAR-100-LT}.
which show consistent improvement with
% when additionally applying
$\TextMani$.
Also, $\TextMani$ with various text encoders achieves analogous improvement trend regardless of the imbalance factor but marginal degradation on Many class of IF=100 when using general language model, BERT and GPT-2.
While the performance gain is from leaking pre-trained language information, it is surprising and a virtue that the language models never exposed to any image can improve the visual recognition performance.
% It implies that while extracted information from the general language model is helpful for Medium or Few classes due to their scarcity, there is 
% In \Tref{tab:CIFAR-100-LT} for CIFAR-100-LT, we compare the effect of $\TextMani$, mix-based augmentations, and their combination with $\TextMani$.
% Our method achieves the largest improvement among the single usage of augmentation methods in all the imbalance factor conditions regardless of the text encoder type.
% The results show that $\TextMani$ is more effective than other mix-based augmentation when the data distribution has a long-tail, 
In comparison to single usage of mix-based augmentations, our method shows higher accuracy
because of uniform effects of $\TextMani$ on samples regardless of class imbalance.
The mix-based methods, on the other hand, sample two data points from the total dataset, where the probability that a tail class sample contributes to a resulting augmented sample is very low. 
Even with class-balanced sampling on mixed-based augmentation in \Tref{tab:balanced_sampling}, $\TextMani$ performs better, further demonstrating our effectiveness.
}
% The result of \Tref{tab:balanced_sampling} shows that our method performs better even with advantage on mix-based augmentation, and verifies our effectiveness.

\begin{table}
\centering
\resizebox{1.0\linewidth}{!}{
    \footnotesize
    \begin{tabular}{@{}l C{11mm}C{11mm}C{11mm}C{11mm}@{}}
        \toprule
        \multirow{2}[2]{*}{\textbf{Augmentation}} & \multicolumn{3}{c}{\textbf{Set of Classes}} &\multirow{2}[2]{*}{\textbf{Total}}\\
        % & \multirow{2}[2]{*}{\textbf{All}} \\ 
        \cmidrule{2-4}
        & \textbf{Many} & \textbf{Medium} & \textbf{Few}  \\ 
        \midrule
        Baseline & 85.34 & 70.47 & 42.80 & 72.24 \\
        $\TextMani$ & \textcolor{CarnationPink}{85.40} & 71.75 & \textcolor{CarnationPink}{48.49}& \textbf{\textcolor{RedViolet}{73.68}}\\ 
        \cmidrule{1-5}
        Cutout~\cite{devries2017improved} & 85.02 & 70.32 & 42.91 & 72.07 \\ 
         + $\TextMani$ & 85.33 & 71.70 & \textbf{\textcolor{RedViolet}{48.54}}& \textcolor{CarnationPink}{73.65}\\
         \cmidrule{1-5}
        Cutmix~\cite{yun2019cutmix} & 84.85 & 69.90 & 35.82 & 70.77 \\
         + $\TextMani$  & 85.30 & \textcolor{CarnationPink}{71.93} & 47.04 & 73.52 \\
        \cmidrule{1-5}
        Mixup~\cite{zhang2017mixup} & 84.96 & 70.27 & 40.20 & 71.63 \\
         + $\TextMani$ & 84.55 & 69.95 & 35.72 & 70.66 \\
        \cmidrule{1-5}
        ManiMixup~\cite{verma2019manifold} & 84.84 & 69.97 & 38.83 & 71.24 \\
         + $\TextMani$ & \textbf{\textcolor{RedViolet}{85.42}} & \textbf{\textcolor{RedViolet}{71.98}} & 46.65 & 73.54 \\
        \bottomrule
    \end{tabular}
}
\caption{
Long-tail classification results (\%) on ImageNet-LT with ViT, and color the value as \textbf{\textcolor{RedViolet}{best}} and \textcolor{CarnationPink}{second best}.
Baseline contains random horizontal flip, random resize crop, color jitter, and normalization, applied in all experiments.
% \textbf{Bold} indicates the best results in each set of classes, and the value in the parentheses for the improvement or degradation compared to the Baseline.
}
\label{tab:ImageNet-LT_vit}
\end{table}

\begin{table}
    \centering
    \resizebox{1.0\linewidth}{!}{\scriptsize
    \begin{tabular}{@{\,}l@{\quad\quad}C{9mm}C{9mm}C{9mm}C{9mm}@{\,}}
         \toprule
         \textbf{Method} & \textbf{Many} & \textbf{Medium} & \textbf{Few} & \textbf{All}\\
         \midrule
         LWS~\cite{kang2019decoupling} & \textbf{\textcolor{RedViolet}{63.34}} & \textcolor{CarnationPink}{48.08} & \textcolor{CarnationPink}{27.19}& \textcolor{CarnationPink}{51.14}  \\
         cRT~\cite{kang2019decoupling} & 61.80 & 46.20 & 27.40 & 49.60 \\
         cRT+\TextMani & \textcolor{CarnationPink}{62.74} & \textbf{\textcolor{RedViolet}{48.60}} & \textbf{\textcolor{RedViolet}{29.67}} & \textbf{\textcolor{RedViolet}{51.47}} \\
         \bottomrule
    \end{tabular}
    }
    % \vspace{-3mm}
    \caption{Long-tail classification results (\%) on ImageNet-LT with ResNext50. 
    We compare with LWS, cRT, and $\TextMani$ on cRT, and color the value as \textbf{\textcolor{RedViolet}{best}} and \textcolor{CarnationPink}{second best}.
    % The highest accuracy is colored with \textcolor{RedViolet}{purple}, the second one is \textcolor{CarnationPink}{pink}, and the third one is \textcolor{gray}{gray}.
    % The models are trained with a batch size of 512.
    }\vspace{3mm}
    \label{tab:ImageNetLT_512}
\end{table}


Particularly in \Tref{tab:CIFAR-100-LT}(b), the mix-based methods have degraded performance in the Medium and Few-shot classes, while our $\TextMani$ improves performance.
% Although $\TextMani$ with BERT and GPT-2 have marginally degraded accuracy on the Many class, increments on other classes are larger, especially in the Few-shot class.
Combining the mix-based methods with $\TextMani$ improves overall performance, but the tendency to sacrifice the Medium and Few-shot classes is the same as before combining.
Additionally, while Cutout has performance degradation due to the information loss~\cite{yun2019cutmix}, it is not affected by skewness due to no mix between inter-classes; thus, the performance is higher than the mix-based one in the long-tailed distribution.


We also evaluate our $\TextMani$ on the large-scale dataset ImageNet-LT. 
In \Tref{tab:ImageNet-LT_vit}, the best and second best results are with $\TextMani$, which demonstrate that our augmentation method is also effective in the large-scale long-tailed data distribution, consistent with the CIFAR-100-LT results in \Tref{tab:CIFAR-100-LT}.
The improvement with $\TextMani$ implies the importance of intra-class perturbation, which can uniformly affect the samples regardless of the skewness of the class distribution.
% While the probability of getting a tail class sample is getting lower when using the mix-based methods, $\TextMani$ uniformly affects the samples regardless of the skewness of the class distribution.

In \Tref{tab:ImageNetLT_512}, we compare with LWS~\cite{kang2019decoupling}, cRT~\cite{kang2019decoupling}, and $\TextMani$ on cRT. 
LWS and cRT are one of effective methods in recent long-tailed recognition.
% This is the same experiment with Table~\textcolor{blue}{3} in the main paper but with a different batch size of 512.
The result shows that $\TextMani$ on cRT achieves the best results compared to the counterparts in all classes except for the Many class, wherefrom ours achieves second best.
% This is consistent with the result in Table~\textcolor{blue}{3} in the main paper regardless of different batch sizes.
Overall, $\TextMani$ improves well-established works, \eg, LWS, and cRT, and it demonstrates the compatibility of our method.
% The overall results show that our $\TextMani$ improves 
% % is capable of boosting
% well-established works, \eg, LWS, and cRT.


% \Tref{tab:ImageNet-LT} shows the results on ImageNet-LT. 
% We adopt two variants with top accuracy, Learnable Weight Scaling (LWS) and classifier Re-Training (cRT), from a decoupling methodology~\cite{kang2019decoupling} as the baselines.\footnote{The baseline results are reproduced with the authors' code, and the reproduced value would differ from the one in the original paper due to different experimental environments, \eg, the number of GPUs and batch size.}
% % Decoupled representation trained with
% % LWS learns the scaling factor for classifier weights, and representations and classifier weights are fixed with the same weights in both experiments for a fair comparison.
% Compared to the LWS baseline, $\TextMani$ achieves gain in performance, which indicates that $\TextMani$ is a scalable method not only effective in neural network training with skewed class distribution but also in scaling factor learning.
% % cRT fine-tunes the classifier with class-balanced sampling. 
% \moon{Compared to the cRT baseline, $\TextMani$ also shows an improvement.
% The overall results show that our $\TextMani$ is capable of boosting well-established works, and is also very flexible yet favorable since it can be attached to any rich representations along with a noticeable gain in performance.
% }
% The cRT baseline scores were reproduced following the best configuration and the one with $\TextMani$ shows improvement. 
% The overall results show that our method is very flexible yet favorable 
% % accurate
% since it can be attached to any rich representations along with a noticeable
% % significant
% gain in performance.
% The results show that $\TextMani$ is capable of boosting well-established work.







\begin{table}
\centering
\resizebox{1.0\linewidth}{!}{
\footnotesize
        \begin{tabular}{@{\,}l C{16mm}C{16mm} @{\,}} 
        \toprule
        % \multirow{2}[2]{*}{\textbf{Model}} & 
        \multirow{2}[2]{*}{\textbf{Augmentation}} & \multicolumn{2}{c}{\textbf{Acc.}} \\
        \cmidrule{2-3}
        &  \textbf{Top-1} & \textbf{Top-5}\\
        \midrule
        % \multirow{8}{*}{ResNet18}
          Baseline    & 31.10 & 59.14 \\
          Cutout      & 32.03 & 60.53 \\
          Cutmix      & 32.43 & 61.04 \\
          Mixup       & 32.72 & 62.47 \\
          ManiMixup   & 33.74 & 63.29 \\
        %   $\TextMani$ 30sample & 32.89 & 60.77 \\
        %   $\TextMani$ maxsample & \textbf{34.52} & \textbf{65.74} \\
          $\TextMani$ & \textbf{34.52 (\blue{+3.42})} & \textbf{65.74 (\blue{+6.60})} \\
          \midrule
        %  \cmidrule{2-4}
          Cutout + $\TextMani$    & 33.91 (\blue{+2.81}) & 61.58 (\blue{+2.44}) \\
          Cutmix + $\TextMani$    & 35.61 (\blue{+4.51}) & 63.82 (\blue{+4.68}) \\
          Mixup + $\TextMani$     & 37.97 (\blue{+6.87}) & 66.75 (\blue{+7.61}) \\
          ManiMixup + $\TextMani$ & \textbf{38.02 (\blue{+6.92})} & \textbf{67.28 (\blue{+8.14})} \\

        % conditions
        % 10% cifar100
        % vit-tiny-patch16-224
        % randn fuction
        % common
            % attr: color, size
            % epochs 200, schedule=50,100, 150
            % init_lr=0.1, gamma 0.1, 0.1, 0.1
            % momentum=0.9, decay=5e-5
            % base_aug
                % random horizontal flip
                % random crop size=32, padding=2
                % normalize
        % mixup
            % mixup_alpha 1.0
        % cutout
            % cutout 16
        % cutmix
            % mixup_alpha 1.0
            % cutmix prob 0.5
        % textmani
            % 30, max sample
            % 0.1 scale
            % nsample true
        % textmani + alpha
            % 30 sample
            % nsample
            % 0.1 scale

        \bottomrule
    \end{tabular}
    }
    \caption{
    Classification results (\%) on CIFAR-100-10\% with ResNet18.
    Baseline represents random horizontal flip, random crop, and normalization, basically applied in all experiments.
    % \textbf{Bold} stands for the best results among those with the same number of added augmentations except for Basic, and 
    The parentheses stands for the improvement compared to the Baseline.
    }
    \label{tab:cifar100_10}
    \vspace{3mm}
% \end{table}
% \begin{table}
% \centering
\resizebox{0.95\linewidth}{!}{
\footnotesize
        \begin{tabular}{@{\,}l C{16mm}C{16mm} @{\,}} 
        \toprule
        % \multirow{2}[2]{*}{\textbf{Model}} & 
        \multirow{2}[2]{*}{\textbf{Augmentation}} & \multicolumn{2}{c}{\textbf{Acc.}} \\
        \cmidrule{2-3}
        &  \textbf{Top-1} & \textbf{Top-5}\\
        \midrule
          Baseline    & 65.37 & 89.82 \\
          Cutout      & 69.17 & 91.12 \\
          Cutmix      & 69.82 & 91.76 \\
          Mixup       & 67.54 & 90.23 \\
          $\TextMani$ & \textbf{70.81 (\blue{+5.44})} & \textbf{92.37 (\blue{+2.55})} \\
        %   $\TextMani-30sample$ & 70.81 & 92.37 \\
        %   $\TextMani-maxsample$ & \textbf{71.21} & 92.40 \\
          \midrule
          Cutout + $\TextMani$    & 69.71 (\blue{+4.34}) & 91.32 (\blue{+1.50}) \\
          Cutmix + $\TextMani$    & \textbf{71.05 (\blue{+5.68})} & \textbf{92.22 (\blue{+2.40})} \\
          Mixup + $\TextMani$     & 70.56 (\blue{+5.19}) & 91.58 (\blue{+1.76}) \\
        
        % conditions
        % 10% cifar100
        % resnet50
        % randn fuction
        % common
            % pretrained
            % interpolation 224x224
            % attr: color, size
            % epochs 200, schedule=50,100, 150
            % init_lr=0.1, gamma 0.1, 0.1, 0.1
            % momentum=0.9, decay=5e-4
            % base_aug
                % random horizontal flip
                % random crop size=32, padding=2
                % normalize
        % manimixup
            % mixup_alpha 2.0
        % mixup
            % mixup_alpha 1.0
        % cutout
            % cutout 16
        % cutmix
            % cutmix prob 0.5
            % mixup alpha 1.0
        % textmani
            % 30, max samples
        % textmani + alpha
            % 30 samples
            % nsample True
            % scale min 0.1
            
            
        % \midrule
        % \textbf{ViT-T} & \\
        % \midrule
        %   Vanilla & & \\
        %   TextMani & & \\
        % \midrule
        % \textbf{ViT-S} & \\
        % \midrule
        %   Vanilla & & \\
        %   TextMani & & \\
        \bottomrule
    \end{tabular}
    }
    \caption{Classification results (\%) on CIFAR-100-10\% with VIT-Tiny. %pretrain
    % Input image is interpolated to 224x224, 
    % Basic augmentation contains random horizontal flip, random crop, and normalization, which are Basically applied in all experiments.
    The configuration follows \Tref{tab:cifar100_10}.
    % \textbf{Bold} stands for the best results,
    % among those with the same number of added augmentations except for Basic, 
    The parentheses stands for the improvement compared to the Baseline.
    }
    \label{tab:cifar100-10_preVIT}
\end{table}


\subsection{Evenly Distributed Scarce Data Classification}\label{sec4.2}
\paragraph{Experimental Setting}
For evaluating the effectiveness of $\TextMani$ on the scarce dataset, we use 10\% data of the CIFAR-100~\cite{krizhevsky2009learning} and Tiny-ImageNet~\cite{le2015tiny} datasets, named CIFAR-100-10\% and Tiny-ImageNet-10\%, respectively.
CIFAR-100 has 100 classes with 500 training images per class, but we only use randomly sampled 50 images per class.
% The evaluation set is the same as the CIFAR-100 test set containing 10k images. 
Tiny-ImageNet is a subset of ImageNet-1k~\cite{russakovsky2015imagenet} with 100k images and 200 classes, but we use 10k images (50 images per class) for simulating 
% constructing the 
a small dataset.
\moon{Note that the evaluation set is same with those of the original datasets.}
% The evaluation set is the same as the original Tiny-ImageNet one having 10k images.
% , and the validation metrics are Top-1 and 5 accuracies.


The baseline models of scarce data classification are ResNet18~\cite{he2016deep} and ViT-Tiny~\cite{dosovitskiy2020image}.
Due to the space limit, 
% we present the ResNet18 and Vit-Tiny results in this section. 
% the results of other models and 
% the 
details of training can be found in the supplementary material.


\paragraph{Results}
\moon{We demonstrate the effectiveness of $\TextMani$ compared to mix-based augmentations on evenly distributed scarce datasets.}
% We compare the effectiveness of $\TextMani$ and other augmentation methods on the evenly distributed scarce datasets.
As in \Tref{tab:cifar100_10} for CIFAR-100-10\%, $\TextMani$ \moon{outperforms}
% is more effective than 
other methods when a single augmentation is used.
Furthermore, the effect is amplified when our method and mix-based methods are combined, \moon{with particularly good compatibility with Manifold Mixup.}
% and the compatibility with ManiMixup is particularly good.
The results demonstrate the importance of intra-class semantic perturbation along with inter-class in scarce data settings.
\moon{This tendency is also observed with another baseline architecture in \Tref{tab:cifar100-10_preVIT}, and datasets in \Tref{tab:TinyImagenet_10}, implying that
% which implies
$\TextMani$ is model-agnostic to be applied.}
% This tendency is also observed in both the ViT-Tiny (\Tref{tab:cifar100-10_preVIT}) and ResNet18 (\Tref{tab:TinyImagenet_10}) cases.
% evaluated on the Tiny-ImageNet-10\% dataset.
% The results also imply that $\TextMani$ is model-agnostic to be applied.
\moon{The overall results demonstrate the potential of $\TextMani$ to enrich the visual feature space using text modalities and develop more accurate and robust models in scarce data regimes.}










\begin{table}
\centering
% \resizebox{0.8\linewidth}{!}{
% epoch 500 : max sample 10, lr 0.2, attribute color+size
% textmani + other augmentation 은 max sample 10보다 30이 훨씬 좋아서 일단 table에는 30개 sampling으로 작성헀습니다.
\footnotesize
\resizebox{1.0\linewidth}{!}{%
        \begin{tabular}{@{\,}l C{16mm}C{16mm} @{\,}} 
        \toprule
        % \multirow{2}[2]{*}{\textbf{Model}} & 
        \multirow{2}[2]{*}{\textbf{Augmentation}} & \multicolumn{2}{c}{\textbf{Acc.}} \\
        \cmidrule{2-3}
         & \textbf{Top-1} & \textbf{Top-5}\\
        \midrule
        % \multirow{8}{*}{ResNet18}
            Baseline    & 25.94 & 50.53\\
            Cutout      & 26.41 & 50.28\\
            Cutmix      & 25.94 & 49.67\\
            Mixup       & 29.34 & \textbf{54.10}\\
            ManiMixup   & 28.43 & 53.25\\
            $\TextMani$ & \textbf{29.37 (\blue{+3.43})} & 52.37 (\blue{+1.84})\\ 
        %   \cmidrule{2-6}
            \midrule
            Cutout + $\TextMani$    & 29.14 (\blue{+3.20})	& 52.60 (\blue{+2.07}) \\
            Cutmix + $\TextMani$    & 29.86 (\blue{+3.92}) & 54.31 (\blue{+3.78}) \\
            Mixup + $\TextMani$     & 31.15 (\blue{+5.21}) & 56.71 (\blue{+6.18}) \\
            ManiMixup + $\TextMani$ & \textbf{32.39 (\blue{+6.35})} & \textbf{58.25 (\blue{+7.72})} \\
        \bottomrule
    \end{tabular}
    }
    \caption{Classification results on Tiny-ImageNet-10\% with ResNet18. 
    % The results imply Top-1 and 5 accuracies (\%) on the test set.
    The configuration follows \Tref{tab:cifar100_10}.
    % \textbf{Bold} represents the best results, and 
    The parentheses represents the improvement compared to the Baseline.
    }
    \label{tab:TinyImagenet_10}
\end{table}




\subsection{Few-shot Object Detection}\label{sec4.3}
\paragraph{Experimental Setting}
We evaluate $\TextMani$ on the PASCAL VOC~\cite{everingham2010pascal} and MS-COCO~\cite{lin2014microsoft} datasets with a few-shot divison following Wang~\etal~\cite{wang2020frustratingly}.
% ~\cite{qiao2021defrcn}.
For VOC, we have three random splits, which have different divisions into 15 base classes and 5 novel classes among the 20 total classes, and $K=1, 2, 3, 5, 10$ objects are sampled from the novel classes.
We utilize the VOC2007 test set for evaluation with AP50 metrics and train with the combination of the VOC2007 and VOC2012 train/val set.
For COCO, the base classes are disjoint with VOC classes while the remaining classes are used as novel classes, and $K=1, 3, 5, 10, 30$ objects are sampled from the novel classes for few-shot fine-tuning.
We use 5k images from the validation set in COCO for evaluation with mAP metrics and the rest for training.

% The model is trained with the base classes first, and then fine-tuned with the novel classes.
% (few-shot object detection; FSOD setting) or with both base and novel classes (Generalized few-shot object detection; G-FSOD setting).
\moon{The baseline~\cite{yan2019meta} is the Faster R-CNN~\cite{ren2015faster} trained with the base classes first and then fine-tuned 
% the model 
with the novel classes.}
% TFA~\cite{wang2020frustratingly} using
% of the FSOD and G-FSOD 
% , denoted as FRCN, 
% which is the standard model for the object detection task.
% and Decoupled Faster R-CNN~\cite{qiao2021defrcn}, which are denoted as FRCN and DeFRCN, respectively.
\moon{$\TextMani$ is applied to the novel class samples during the fine-tuning stage.}
% at each baseline.
Following the 
% As following 
prior studies, all the reported results are averaged over 10 repeated runs.
% All the results are reproduced based on the DeFRCN~\cite{qiao2021defrcn} code.

\begin{table}
\centering
\resizebox{1.0\linewidth}{!}{
\footnotesize
        \begin{tabular}{@{\,}cl ccccc @{\,}} 
        \toprule
        \multirow{2}[2]{*}{\textbf{Split}} & \multirow{2}[2]{*}{\textbf{Aug.}} & \multicolumn{5}{c}{\textbf{$K$- shot}} \\
        \cmidrule{3-7}
        & & \textbf{1} & \textbf{2} & \textbf{3} & \textbf{5} & \textbf{10} \\
        \midrule
        \multirow{3}{*}{All}
        & Baseline     & 12.82 & 16.65 & 20.04 & 20.64 & 23.19 \\
        & \multirow{2}{*}{$\TextMani$} & \textbf{17.74} & \textbf{22.40} & \textbf{23.37} & \textbf{25.09} & \textbf{24.22} \\
        & & \textbf{(\blue{+4.92})} & \textbf{(\blue{+5.75})} & \textbf{(\blue{+3.33})} & \textbf{(\blue{+4.45})} & \textbf{(\blue{+1.03})} \\
        \midrule
        \midrule
        \multirow{3}{*}{1}
        & Baseline     & 15.11 & 18.82 & 22.61 & 21.97 & 23.74 \\
        & \multirow{2}{*}{$\TextMani$} & \textbf{21.94} & \textbf{26.44} & \textbf{23.66} & \textbf{25.88} & \textbf{25.14} \\
        & & \textbf{(\blue{+6.83})} & \textbf{(\blue{+7.62})} & \textbf{(\blue{+1.05})} & \textbf{(\blue{+3.91})} & \textbf{(\blue{+1.40})} \\
        \midrule
        \multirow{3}{*}{2}
        & Baseline     & 10.86 & 14.22 & 18.67 & 19.34 & 22.49 \\
        & \multirow{2}{*}{$\TextMani$} & \textbf{14.64} & \textbf{18.49} & \textbf{23.28} & \textbf{23.06} & \textbf{24.44} \\
        & & \textbf{(\blue{+3.78})} & \textbf{(\blue{+4.27})} & \textbf{(\blue{+4.61})} & \textbf{(\blue{+3.72})} & \textbf{(\blue{+1.95})} \\
        \midrule
        \multirow{3}{*}{3}
        & Baseline     & 12.49 & 16.90 & 18.84 & 20.61 & 23.35 \\
        & \multirow{2}{*}{$\TextMani$} & \textbf{16.65} & \textbf{22.26} & \textbf{23.16} & \textbf{26.33} & \textbf{25.08} \\
        & & \textbf{(\blue{+4.16})} & \textbf{(\blue{+5.36})} & \textbf{(\blue{+4.32})} & \textbf{(\blue{+5.72})} & \textbf{(\blue{+1.73})} \\
        \bottomrule
        
    % gfsod
    % novel AP50 
    % textmani 30 samples
    % scale 0.1
    
    % defrcn vanilla take only 2 repeat yet
    \end{tabular}
    }
    \caption{Few-shot object detection results (AP50) on VOC. % FSOD setting.
    % The second column represents the types of augmentation methods applied.
    % Baseline stands for no augmentation methods applied.
    % \textbf{Bold} indicates the best results in each model, and 
    The value in the parentheses indicates the improvement compared to the Baseline of each split set.
    }
    \label{tab:voc}
\end{table}


\begin{table}
\centering
\resizebox{1.0\linewidth}{!}{
\footnotesize
        \begin{tabular}{@{\,} l C{10mm}C{10mm}C{10mm}C{10mm}C{10mm} @{\,}} 
        \toprule
        % \multirow{2}[2]{*}{\textbf{Model}} &
        \multirow{2}[2]{*}{\textbf{Aug.}} & \multicolumn{5}{c}{\textbf{$K$- shot}} \\
        \cmidrule{2-6}
        & \textbf{1} & \textbf{3} & \textbf{5} & \textbf{10} & \textbf{30} \\
        \midrule
        % \multirow{2}{*}{FRCN}
         Baseline      & 3.43 & 4.66 & 6.10 &  9.11 & 12.78 \\
         \multirow{2}{*}{$\TextMani$}
         & \textbf{5.39} & \textbf{6.47} & \textbf{7.80} & \textbf{10.03} & \textbf{13.60} \\
         & \textbf{(\blue{+1.96})} & \textbf{(\blue{+1.81})} & \textbf{(\blue{+1.70})} & \textbf{(\blue{+0.92})} & \textbf{(\blue{+0.82})} \\
        % \midrule
        % \multirow{2}{*}{DeFRCN}
        % & Vanilla     & 4.63 & 12.31 & 16.06 & 18.56 & 22.43 \\
        % & $\TextMani$ & & & & & \\
        \bottomrule
    \end{tabular}
    }
    \caption{Few-shot object detection results (mAP) on COCO. % FSOD setting. 
    The configuration follows \Tref{tab:voc}.
    % \textbf{Bold} indicates the best results in each model, and 
    % The parentheses indicates the improvement compared to the Baseline.
    }
    \label{tab:coco}
\end{table}

% \begin{table}
% \centering
% \caption{GFSOD experimental results (mAP) on the COCO dataset.
% }
% \resizebox{1.0\linewidth}{!}{
% \footnotesize
%         \begin{tabular}{@{}ll ccccc @{}} 
%         \toprule
%         \multirow{2}[2]{*}{\textbf{Model}} & \multirow{2}[2]{*}{\textbf{Aug.}} & \multicolumn{5}{c}{\textbf{$K$- shot mAP}} \\
%         \cmidrule{3-7}
%         & & \textbf{1} & \textbf{3} & \textbf{5} & \textbf{10} & \textbf{30} \\
%         \midrule
%         \multirow{2}{*}{FRCN}
%         & Vanilla        & 1.17 (15.51) & 2.72 (14.90) & 3.96 (14.74) & 5.26 (15.15) & 7.32 (15.19) \\
%         & $\TextMani$  & & & & & \\
%         \midrule
%         \multirow{2}{*}{DeFRCN}
%         & Vanilla       & 4.49 (23.73) & 10.30 (26.46) & 13.29 (27.67) & 16.42 (29.51) & 20.70 (31.23) \\
%         & $\TextMani$ & & & & & \\
%         \bottomrule
%     \end{tabular}
%     }
%     \label{tab:coco_g}
% \end{table}


\paragraph{Results}
% Our evaluation on the FSOD task validates that $\TextMani$ can also be applied in the detection task.
% , we evaluate our method on 
Note that we apply $\TextMani$ only on the classification head; thus, the quality of the regressed bounding boxes will remain 
% be the
similar 
% the same
as before applying $\TextMani$.
As shown in \Tref{tab:voc} for VOC and \Tref{tab:coco} for COCO, $\TextMani$ improves the AP by improving only the classification accuracy, where the result 
% follows
is in a similar line to
the analysis~\cite{borji2019empirical} that classification error weighs more than localization error.
The improvement is clearer when $K$ is low.
\moon{The results demonstrate the applicability of $\TextMani$ to enhance the classification accuracy of detection models.}






\begin{table}
    \centering
    \resizebox{1.0\linewidth}{!}{\scriptsize
    \begin{tabular}{@{\,}c@{\,\,\,}l@{\,\,}c@{\quad}c@{\quad}c@{\quad}c@{\quad}c@{\quad}c@{\,}}
         \toprule
         & \textbf{Aug.} & \textbf{Many} & \textbf{Medium} & \textbf{Few}& \textbf{IF=100} & \textbf{IF=50} & \textbf{IF=10}\\
         \midrule
         \multirow{3}{*}{(a)}
         & Baseline   & 71.11 & 38.42 & 3.00 & 38.39 & 43.33 & 59.29  \\ 
         & Random     & \textbf{71.37} & 38.55 & 2.90 & 38.43 & 43.28 & 60.39 \\
         & \TextMani  & 70.22 & \textbf{40.73} & \textbf{9.41} & \textbf{41.10} & \textbf{47.17} & \textbf{60.67} \\
         \midrule
         \multirow{2}{*}{(b)}
         & Direct.    & 71.34 & 38.64 & 4.32 & 38.66 & 43.44 & 59.82 \\
         & Concat.    & 68.02 & 35.82 & 5.35 & 36.98 & 42.68 & 59.44 \\
         \bottomrule
    \end{tabular}
    }
    \caption{Comparison to (a) random perturbation, and
    (b) direct text and concatenated embeddings on CIFAR-100-LT. 
    }
    \label{tab:random}
\end{table}




\subsection{Further Analyses}\label{sec4.4}
% In this section, we further demonstrate the compatibility of our $\TextMani$ with the linear-probed model and conduct an ablation study on attributes.

\paragraph{Random Baseline}
In \Tref{tab:random}-{\color{blue}(a)}, we compare our method with the Random baseline.
We randomly sample a vector from a Normal distribution $\mathcal{N}(0,1)$ and use it instead of the difference vector, \ie, augmenting visual features with random perturbations on the same manifold of visual features.

The result shows that the Random baseline improves performance by serving as intra-perturb, but marginal compared to our method considering semantics additionally, which implies that semantic information embedded in the difference vector guides the augmentation more effective direction rather than random.


\paragraph{Effectiveness of Difference Vectors}
While we use the difference vectors by subtracting the embeddings with and without attribute words for \TextMani, there could be another way to extract the attribute information.
In \Tref{tab:random}-{\color{blue}(b)}, we compare with counterparts, direct text embedding (Direct.) and concatenated embeddings (Concat.).
For the Direct method, we use the text embedding computed from the attribute word directly instead of the difference vector.
For the Concat method, we concatenate the text embeddings from with and without attribute words, \eg, [``bull''$\|$``red bull''], and use it instead of the difference vector.

The results show that using difference vector (\TextMani) outperforms using direct text embedding or concatenated embeddings, and imply that remaining contextual information after subtraction plays an important role in doing intra-perturbation in a semantic way.
% We further discuss the reason for the better performance in the following questions.
Although the word ``blue'' can function as both an adjective and a noun, its exact role in a sentence cannot be determined solely based on the word itself.
Our intention of subtraction is for attribute words to act as a modifier in the sentence motivated by word analogy.
When we computed the cosine similarity, embeddings derived directly from ``red'' and those obtained from the difference exhibited low similarity because they \emph{contain different contextual information} despite the same origin of a word.
% Also, the surrounding meaning can be left even after subtraction to some extent; the class label information would be reflected in the difference vector, which makes the subtle differences (L464-465) in Fig.~{\color{blue}4}.
% Also, the results imply that remaining contextual information after subtraction plays an important role in doing intra-perturbation in a semantic way.



% \begin{table}
\begin{wraptable}{r}{0.46\linewidth}
    \centering
    \vspace{-3mm}
    \resizebox{0.85\linewidth}{!}{\scriptsize
    \begin{tabular}{@{\,}l@{\,\,\,}c@{\,}}
         \toprule
         \textbf{Model} & \textbf{LP-Full}\\
         \midrule
         VL-LTR     & 61.04 \\
         +\TextMani & \textbf{61.82} \\
         \bottomrule
    \end{tabular}
    }
    % \vspace{-3mm}
    \caption{\moon{Comparison between the SOTA model with and without $\TextMani$ during linear probing on CIFAR-100.}
    % Comparison between linear-probed SOTA method and applying our method to it on CIFAR-100.
    % with ResNet18.
    }
    \label{tab:sota}
    \vspace{-2mm}
\end{wraptable}
% \end{table}
\paragraph{Linear Probing with Advanced Models}
\moon{Further demonstrating the compatibility of $\TextMani$, we apply our method during linear probing of the model.}
% when linear-probe the model.
In \Tref{tab:sota}, we test VL-LTR~\cite{tian2022vl}, the state-of-the-art model in long-tail classification, on CIFAR-100.
In \Tref{tab:CLIP_baseline}, we use a CLIP image encoder~\cite{radford2021learning} with various architectures as the baseline model and linear-probe the model on both 10\% and full data of CIFAR-100.
The results demonstrate that $\TextMani$ is compatible with linear-probed CLIP and VL-LTR models.


\begin{table}
    \centering
    \resizebox{0.9\linewidth}{!}{\scriptsize
    \begin{tabular}{@{\ \ }l@{\quad}l@{\quad}c@{\quad}c@{\quad}c@{\ \ }}
         \toprule
         \textbf{CLIP Arch.} & \textbf{Aug.} & \textbf{ZS} & \textbf{LP-10\%} & \textbf{LP-Full}\\
         \midrule
         \multirow{2}{*}{ResNet50} 
         & Baseline  & 39.47 & 50.18 & 63.64 \\
         & \TextMani &   -   & \textbf{52.83} & \textbf{64.17} \\
         \midrule
         \multirow{2}{*}{ResNet101} 
         & Baseline  & 45.17 & 57.37 & 68.60 \\
         & \TextMani &   -   & \textbf{59.49} & \textbf{69.12} \\
         \midrule
         \multirow{2}{*}{ViT-B} 
         & Baseline  & 58.21 & 73.30 & \textbf{79.99} \\
         & \TextMani &   -   & \textbf{73.35} & 79.58 \\
         \bottomrule
    \end{tabular}
    }
    % \vspace{-3mm}
    \caption{Classification results (\%) of CLIP with zero-shot (ZS) and linear-probe (LP) on Full and 10\% CIFAR-100. 
    We apply our $\TextMani$ to the linear-probed CLIP.
    % with ResNet18.
    }
    \label{tab:CLIP_baseline}
    % \vspace{-6mm}
\end{table}






\begin{wraptable}{r}{0.37\linewidth}
    \centering
    % \vspace{-2mm}
    \resizebox{0.75\linewidth}{!}{\scriptsize
    \begin{tabular}{@{\,}c@{\,\,\,}c@{\,\,\,}c@{\,}}
         \toprule
         \textbf{Color} & \textbf{Size} & \textbf{Acc.}\\
         \midrule
                    & & 31.10 \\
         \checkmark & & 33.48 \\
         & \checkmark & 33.89 \\ 
         \checkmark & \checkmark & \textbf{34.52} \\
         \bottomrule
    \end{tabular}
    }
    % \vspace{-3mm}
    \caption{Ablation study on the attributes with CIFAR-100-10\%.
    % with ResNet18.
    }
    \label{tab:ablation_attr}
    \vspace{-2mm}
\end{wraptable}
\paragraph{Ablation Study on Attributes}
\moon{In $\TextMani$, we have considered color and size attributes.}
% We consider color and size attributes in experiments of $\TextMani$.
To confirm the effect of each attribute, we conduct an ablation study on attributes in \Tref{tab:ablation_attr}.
The result shows that while each attribute brings non-trivial gain, using both brings more gain.
We believe that there are additional attributes we could use and a more effective method for selecting appropriate attributes, 
% such as prompt suggestion~\cite{pratt2022does}, 
but leave it for future work.











\section{Conclusion}\label{sec:conclusion}
In this work, we design a new type of tuning method, termed regularized mask tuning, that masks the network parameters under a learnable selection. 
Specifically, we first identify a set of parameters that are key to a given downstream task, then attach a binary mask to this parameter, and finally optimize these masks on the downstream data with the parameters frozen.
When updating the mask, we introduce a novel gradient dropout strategy to regularize the parameter selection, to prevent the model from forgetting and overfitting.
% Meanwhile, our method is synergistic with most existing prompt tuning methods and provides the capacity to customize parameter settings based on downstream needs.
Extensive experiments demonstrate that our method consistently outperforms existing methods and is synergistic with them.
Future work will explore applying mask tuning to other visual tasks such as segmentation.
% \todo{why not add the mask in the text encoder?}
\bibliographystyle{plainnat}
\bibliography{references}
\newpage
\appendix
\section{Proofs}
\begin{proof}[Proof of Lemma 3.4]
Let $\Delta\hat{w}$ denote an arbitrary direction and let $d=\nabla_{\hat{w}}\,x^\ast(\hat{w})\,\Delta\hat{w}$ be the corresponding directional derivative of the decision. The existence of $d$ is guaranteed by the strict complementary slackness conditions and Lemma 3.3.
Let $t\to0^+.$ Then, we have
\begin{equation*}
\hat{x}'(t):=x^\ast(\hat{w}+t\Delta\hat{w})=\hat{x}+td+o_x(t),
\end{equation*} where $o_x(t)$ is the ``little $o$'' notation, i.e., $\lim_{t \to 0^+} \frac{\|o_x(t)\|_2}{t}=0.$ To prove the lemma, we first want to show that $d^\top n_i =0,\;\forall i \in I(\hat{x}).$ Then, we will show that it implies the lemma's claim.

By definition, $n_i=\nabla_{x}g_i(\hat{x}).$ Then, since $g_i(\cdot)$ is differentiable and $g_i(\hat{x})=0,\,\forall i\in I(\hat{x}),$ we have the following first-order approximation for $g_i\big(\hat{x}'(t)\big):$
\begin{equation*}    g_i\big(\hat{x}'(t)\big)=g_i\big(\hat{x}+td+o(t)\big)=g_i(\hat{x}) + tn_i^\top d  + o_g(t) = tn_i^\top d + o_g(t).
\end{equation*}
Since $\hat{x}'$ is the solution of the internal optimization problem, the inequality $g_i(\hat{x}'(t))\leq 0$ holds. Hence, the equation above implies that $n_i^\top d \leq 0.$
Now, we want to show that, in fact, $n_i^\top d = 0.$ For a proof by contradiction, suppose that $n_i^\top d < 0.$
Then, by definition of $o_g(t)$, there exists $\epsilon>0,$ such that 
\begin{equation*}
    0< t<\epsilon\implies g_i\big(\hat{x}'(t)\big)<0.
\end{equation*}
Now, we will to show that $g_i\big(\hat{x}'(t)\big)<0$ contradicts the complementary slackness condition at $\hat{x}.$ From Lemma 3.3, we know the KKT multiplier, $\alpha'_i(t):=\alpha_i(\hat{w}+t\Delta\hat{w}),$ is a continuous function of $t.$ On the one hand, from the KKT conditions, we know that $g_i\big(\hat{x}'(t)\big)<0\implies \alpha'_i(t)=0.$ Therefore, $\alpha'_i(t)=0$ for $t<\epsilon.$ Hence, we have
\begin{equation*}
    \lim_{t \to 0^+} \alpha'_i(t)=0.
\end{equation*} 
On the other hand, the continuity implies that
$\lim_{t \to 0^+} \alpha'_i(t)=\alpha'_i(0)=\alpha_i$ and, due to strict complementary slackness, $\alpha_i>0.$ Hence, we also have
\begin{equation*}
    \lim_{t \to 0^+} \alpha'_i(t)>0.
\end{equation*}

We arrived at a contradiction and therefore can claim that ${d^\top n_i=0}$ for all $n_i.$ 
Since ${\{n_i|i\in I(\hat{x})\}}$ is a basis of $\mathcal{N}(\hat{x}),$ this implies that for any direction $v\in\mathcal{N}(\hat{x})$ and for any $\Delta\hat{w},$ we have $v^\top\,\nabla_{\hat{w}}\,x^\ast(\hat{w})\,\Delta\hat{w}=0.$ 
In other words, vector $v^\top\,\nabla_{\hat{w}}\,x^\ast(\hat{w})$ is orthogonal to the whole space of $\hat{w}$ and hence it must be zero, $v^\top\,\nabla_{\hat{w}}\,x^\ast(\hat{w})=0,\,
\forall v\in\mathcal{N}(\hat{x}).$
Hence $\mathcal{N}(\hat{x})$ is contained in the left null space of $\nabla_{\hat{w}}\,x^\ast(\hat{w}).$
\end{proof}

\begin{proof}[Proof of Lemma 3.6]
First, consider the case when the unconstrained maximum $\hat{w}$ is in the interior of $\mathcal{C}.$ By definition of $x^\ast_{QP},$ it means that $\hat{x}=x^\ast_{QP}(\hat{w})$ is also in the interior of $\mathcal{C}$ and $\hat{x}=\hat{w}$. Then, $x^\ast_{QP}$ is the identity function around $\hat{w},$ and hence 
$x^\ast_{QP}(\hat{w}+\Delta\hat{w})=x(\hat{w})+\Delta\hat{w}$ for small enough $\Delta\hat{w}.$ Hence, $\nabla_{\hat{w}}x^\ast_{QP}(\hat{w})=I.$ Since no constraints are active in this case ($I(\hat{x})=\emptyset$), the lemma's claim holds.

Now, consider the case when some constraints are active, and thus $\hat{x}$ lies on the boundary of $\mathcal{C}.$ 
To get the exact form of the Jacobian $\nabla_{x}\,x_{QP}^\ast(\hat{w}),$ we will compute $\lim_{t\to0}x^\ast_{QP}(\hat{w}+t\Delta\hat{w})$ for all possible $\Delta\hat{w}.$ 
As in the QP case the predictions $\hat{w}$ lie in the same space as $\hat{x}$, we can do it first for $\Delta\hat{w}\in\mathcal{N}(\hat{x})$ and then for $\Delta\hat{w}\perp\mathcal{N}(\hat{x}).$

\paragraph{1. ${\Delta\hat{w}\in\mathcal{N}(\hat{x}).}$} For $\Delta\hat{w}\in\mathcal{N}(\hat{x}),$ we want to show that the corresponding directional derivative is zero. We begin by computing the internal gradient $\nabla_{x}f_{QP}(\hat{x}, \hat{w}):$ 
\begin{equation*}
\nabla_{x}f_{QP}(\hat{x}, \hat{w})=-\nabla_{x}\,\|x-w\|^2_2 = 2(\hat{w}-\hat{x}).    
\end{equation*}
Using this formula, we can write the internal gradient for the perturbed prediction $\hat{w}+t\Delta\hat{w}$ at the same point $\hat{x}$:
\begin{equation*}
    \nabla_{x}f_{QP}(\hat{x}, \hat{w} + t\Delta\hat{w}) = \nabla_{x}f_{QP}(\hat{x}, \hat{w}) + 2t\Delta\hat{w}. 
\end{equation*}
By definition, $\mathcal{N}(\hat{x})$ is a linear span of the vectors $\{ n_i|i\in I(\hat{x})\}.$ Hence, since $\Delta\hat{w}\in\mathcal{N}(\hat{x}),$ it can be expressed as 
\begin{equation*}
\Delta\hat{w}=\sum_{i\in I(\hat{x})}\delta_in_i,\quad\delta_i\in\R.
\tag{$\ast$}
\end{equation*}
 
By Property 3.2, the internal gradient has the following representation: 
\begin{equation*}
    \nabla_{x}f_{QP}(\hat{x}, \hat{w}) = \sum_{i\in I(\hat{x})}\alpha_in_i,\quad \alpha_i>0.
    \tag{$\ast\ast$}
\end{equation*}
Then, combining $(\ast)$ and $(\ast\ast),$ we obtain
\begin{equation*}
 \nabla_{x}f_{QP}(\hat{x}, \hat{w} + t\Delta\hat{w}) = \nabla_{x}f_{QP}(\hat{x}, \hat{w}) + 2t\Delta\hat{w} = \sum_{i\in I(\hat{x})}(\alpha_i + 2t\delta_i)n_i
\end{equation*}
Since $\alpha_i>0,\,\forall i \in I(\hat{x})$, there exists $\epsilon>0,$ such that $\alpha_i -2t\delta_i > 0$ for $|t|<\epsilon.$ Therefore, $\nabla_{x}f_{QP}(\hat{x}, \hat{w} + t\Delta\hat{w})$ lies in the gradient cone of $\hat{x},$ and hence, by Property 3.2, $x_{QP}^\ast(\hat{w}+t\Delta\hat{w})=\hat{x}$ for $|t|<\epsilon.$ Therefore, the directional derivative of $x_{QP}^\ast(\hat{w})$ along $\Delta\hat{w}\in\mathcal{N}(\hat{x})$ is zero.
\paragraph{2. $\Delta\hat{w}\perp\mathcal{N}(\hat{x}).$ } Next, let $\Delta\hat{w}$ be orthogonal to $\mathcal{N}(\hat{x}).$ We begin with the first order approximation of $\hat{x}'(t):$
\begin{equation*}
    \hat{x}'(t)=\hat{x} + td + o(t).
\end{equation*} 
From the proof of Lemma 3.3, we can know that $d\perp \mathcal{N}.$ By definition of $x^\ast_{QP}$, we know that 
$\hat{x}$ is the point on $\mathcal{C}$ closest to $\hat{w}.$ Likewise, $\hat{x}'(t)$ is the point on $\mathcal{C}$ closest to $\hat{w}+t\Delta\hat{w}.$ Hence, $d=\Delta\hat{w}.$ Therefore, for any $\Delta\hat{w}\perp\mathcal{N},$ the directional derivative of $x_{QP}(\hat{w})$ along $\Delta\hat{w}$ is one. 

So, we have shown that 
\begin{equation*}
    \nabla_{\hat{w}}\,x^\ast_{QP}(\hat{w})\,\Delta\hat{w}=\begin{cases}
0 &\text{for } \Delta\hat{w}\in\mathcal{N}(\hat{x}) \\
\Delta\hat{w} & \text{for }  \Delta\hat{w}\perp\mathcal{N}(\hat{x}).
\end{cases}
\end{equation*}
Therefore, the lemma is proven.
\end{proof} 

\begin{proof}[Proof of Theorem 3.9]
First, we want to construct an orthogonal basis $\{e_1,\ldots e_n\}$ of $\R^n$ that will greatly simplify the calculations. We start by including the internal gradient in this basis, i.e., we define $e_1=\nabla_{x}f_{QP}(\hat{x}, \hat{w}).$ Then, let $I(\hat{x})=\{i|g_i(\hat{x})=0\}$ be the set of indices of the active constraints of the original problem and let $\mathcal{N}(\hat{x})=span(\{n_i|i\in I(\hat{x})\})$ be a linear span of their normals. By the liner independence condition from Assumption 2, $dim\big(\mathcal{N}(\hat{x})\big)=|I(\hat{x})|.$ Moreover, by Property 3.2, we know that $e_1\in\mathcal{N}(\hat{x}).$ Then, we can choose vectors $e_2,\ldots, e_{|I(\hat{x})|}$ that complement $e_1$ to an orthogonal basis of $\mathcal{N}(\hat{x}).$
The remaining vectors $e_{|I(\hat{x})| +1},\ldots,e_n,$ are chosen to complement $e_1,\ldots,e_{|I(\hat{x})|}$ to an orthogonal basis of $\R^n$. The choice of this basis is motivated by Lemma 3.6: $e_1$ is a basis of the null-space of the $r-$smoothed Jacobian, $e_1,\ldots,e_{|I(\hat{x})|}$ form a basis of the null space of the true QP Jacobian, and the remaining vectors form a basis of space in which we can move $x^\ast_{QP}(\hat{w}).$

For brevity, let $f_x=\nabla_{x}f(\hat{x}, w)$ denote the true gradient vector. By definition,
${\Delta\hat{w}=f_x\,\nabla_{\hat{w}}x^\ast_r(\hat{x}, \hat{w})}$
is obtained via the $r-$smoothed problem.
From Property 3.8, we know that $\Delta\hat{w}$ is a projection of $f_x$ on the vectors $e_2,\ldots,e_n.$ Then, since $e_1,\ldots,e_n$ is an orthogonal basis, we have
\begin{equation*}
    \Delta\hat{w}=\sum_{i=2}^n\beta_ie_i,\quad \beta_i=f_x^\top e_i,\, i=2,\ldots,n.
\end{equation*}

Now, let's see how this $\Delta\hat{w}$ affects the true decision $x^\ast_{QP}(\hat{w} + t\Delta\hat{w})$ for $t\to 0^+.$ First, we have a first-order approximation 
\begin{equation*}
x^\ast_{QP}(\hat{w} + t\Delta\hat{w})=\hat{x} + td + o(t),
\end{equation*}
for some $d\in\R.$ From Lemma 3.6, we know that $d$ is actually a projection of $\Delta\hat{w}$ onto the vectors $e_{|I(\hat{x})|+1},\ldots,e_n.$ 
Therefore, we have
\begin{equation*}
    x^\ast_{QP}(\hat{w} + t\Delta\hat{w})=\hat{x} + \sum_{i=|I(\hat{x}|+1}^n\beta_ie_i + o(t).
\end{equation*}

Finally, the change in the true objective can be expressed as
\begin{align*}
    f\Big(x^\ast_{QP}(\hat{w}+t\Delta\hat{w}), w\Big)-f\Big(x^\ast_{QP}(\hat{w}), w\Big)=tf^\top_x \Big(\sum_{i=|I(\hat{x}|+1}^n\beta_ie_i\Big) + o(t)=\\=t\sum_{i=|I(\hat{x}|+1}^n\beta_if^\top_xe_i
+o(t)= t\sum_{i=|I(\hat{x}|+1}^n\beta_i ^2 +o(t)\geq 0.
\end{align*}
    
Therefore, perturbing prediction along $\Delta\hat{w}$ does not decrease the true objective $f(\hat{x}, w),$ and hence \begin{equation*}
    f\big(x^\ast_{QP}(\hat{w}+t\Delta\hat{w}), w\big)\geq f\big(x^\ast_{QP}(\hat{w}), w\big) 
\end{equation*}
for $t\to 0^+.$
\end{proof}

\section{Equality constraints}
Assumption 2 postulates that for any $x\in\mathcal{C},$ the gradients of active constraints, $\{\nabla_{x}g_i(x)|g_i(x)=0\},$ are linearly independent. Now, suppose we include equality constraints in our problem. e.g., we have a constraint $g^{eq}(x)\leq0$ and $-g^{eq}(x)\leq 0$ for some $g.$ Clearly, the gradients of $g^{eq}(x)$ and $-g^{eq}(x)$ violate the independence assumption. However, we claim that it does not affect our results.
Let $\hat{w}$ and $\hat{x}$ be a prediction and a corresponding decision and let $n^{eq}=\nabla_{x}\,g^{eq}(\hat{x}).$ Suppose the equality constraint $g^{eq}(\hat{x})=0$ is active. Let $I(\hat{x})$ be the set of indices of the active constraints \textit{not including} $g^{eq}(x).$ Then, we have a representation of the internal gradient, 
\begin{equation*}
    \nabla_{x}f(\hat{x},\hat{w})=\alpha^{eq}_1n^{eq} - \alpha^{eq}_2n^{eq} + \sum_{i\in I(\hat{x})}\alpha_in_i.
\end{equation*}
Suppose that $\alpha^{eq}_1\neq\alpha_2^{eq},$ e.g., without loss of generality, $\alpha^{eq}_1>\alpha^{eq}_2.$ Then, 
\begin{equation*}
    \nabla_{x}f(\hat{x},\hat{w})=(\alpha^{eq}_1-\alpha^{eq}_2)n^{eq} + \sum_{i\in I(\hat{x})}\alpha_in_i
\end{equation*} and hence removing the constraint $-g^{eq}(x)\leq 0$ would not change the optimality of $\hat{x}.$ The remaining problem would satisfy complementary slackness and hence would have all the properties demonstrated in Section 3. Therefore, for the case with equality constraints, we need to extend the complementary slackness conditions by demanding $\alpha_1^{eq}\neq\alpha_2^{eq}.$ 

\section{Experimental details}
In this section, we provide the details of the experiments reported in the paper. All experiments were conducted on a machine with 32gb RAM and NVIDIA GeForce RTX 3070. The code is written in Python 3.8, and neural networks are implemented in \textit{PyTorch} 1.12. For methods requiring differentiation of optimization problems (those, without $r-$smoothing), we use the implementation by Agrawal et.\ al [2019a]. The code can be found at \textit{placeholder for GitHub link. For the reviewers, the code is submitted through OpenReview.}
\subsection{Portfolio optimization problem}
\begin{table}[t]
\centering
\begin{tabular}{l|l}
%\cline{2-2}
& Search space \\ \hline
\multicolumn{1}{l|}{\textit{Learning rate}} 
&$\{5\times10^{-6}, 10^{-5}, 2\times10^{-5}, 5\times10^{-5}, 10^{-4}, 5\times10^{-4}\}$  \\ %\hline

\multicolumn{1}{l|}{\textit{Batch size}} & $\{1, 2, 4, 8, 32\}$    \\ %\hline
\multicolumn{1}{l|}{\textit{Proj. distance weight} $\alpha$ from Eq. (6)} & 
$\{0.001, 0.0025, 0.005, 0.01, 0.05, 0.1, 1\}$   \\ %\hline
%\multicolumn{1}{|l|}{\textit{Training epochs}} &  \\ \hline
\multicolumn{1}{l|}{$x_{shift}$} & $\{0, .1, 1\}$ \\ %\hline
\multicolumn{1}{l|}{$x_{scale}$} & $\{0.1, 1, 5\}$ \\ %\hline
\end{tabular}
\caption{Search space for different hyperparameters for the portfolio optimization problem}
\label{tab:po_search}
\end{table}
In the portfolio optimization problem, the predictor $\phi_\theta$ is represented by a fully connected neural network with two hidden layers of 100 neurons each, and \textit{ReLU} activation functions. The output layer has no activation function. Instead, the output of the neural network is scaled by the factor $x_{scale}$ and shifted by $x_{shift}$.  For the methods using the QP approximation, the output layer predicts only $\hat{w}$ and consists of 200 neurons, one per the decision variable $x_i$. For the method that uses the original problem formulation and predicts both $\hat{p}$ and $\hat{Q}$, the output layer additionally predicts a $200\times200$ matrix $L$ and then sets $\hat{Q}:=(0.9L+0.1I)(0.9L+0.1I)^\top$, where $I$ is the identity matrix.

For training, we used the \textit{Adam} optimizer from PyTorch, with custom learning rate and otherwise default parameters.
The values of different hyperparameters were determined by a grid search procedure, summarized in Table~\ref{tab:po_search}.
The values used in the experiments are reported in Table~\ref{tab:po_results}. These values may vary across experiments with different $\lambda'$s. For each $\lambda,$ however, the four studied methods use the same values of the hyperparameters (the only exception is the projection distance weight $\alpha,$ which is always zero for methods without regularization).

\begin{table}[t]
\centering
%\begin{tabular}{l|l|l|l|l|l|l|l}
\begin{tabular}{lrrrrrrrr}
%\cline{2-7}
                         & $\lambda=2$&$\lambda=1$&$\lambda=0.5$&$\lambda=0.25$&$\lambda=0.1$&$\lambda=0$ \\ \hline
\multicolumn{1}{l|}{\textit{Learning rate}} 
&$10^{-5}$ &$10^{-5}$ &$2\times10^{-5}$&$2\times10^{-5}$ & 
$2\times10^{-5}$& $5\times10^{-5}$\\ %\hline

\multicolumn{1}{l|}{\textit{Batch size}} & $1$ &$1$&$1$&$1$&$1$&$1$    \\ %\hline
\multicolumn{1}{l|}{\textit{Penalty weight} $\alpha$ from Eq. (6)} &$0.1$&$0.1$&$0.02$&$0.005$&$0.005$&$0.0025$    \\ %\hline
\multicolumn{1}{l|}{\textit{Training epochs}} &$180$&$180$&$180$&$180$&$180$&$180$    \\ %\hline
\multicolumn{1}{l|}{$x_{shift}$} &$1$&$1$&$1$&$1$&$1$&$1$  \\ %\hline
\multicolumn{1}{l|}{$x_{scale}$} &$0.1$&$0.1$&$0.1$&$0.1$&$0.1$&$0.1$\\ %\hline
\end{tabular}
\caption{Best performing values of the hyperparameters for the portfolio optimization problem with different $\lambda'$s}
\label{tab:po_results}
\end{table}

\subsection{Optimal power flow problem}
\begin{table}[h]
\centering
\begin{tabular}{l|l}
%\cline{2-2}
& Search space \\ \hline
\multicolumn{1}{l|}{\textit{Learning rate}} 
&$\{5\times10^{-6}, 10^{-5}, 2\times10^{-5}, 5\times10^{-5}, 10^{-4}, 5\times10^{-4}\}$  \\ %\hline

\multicolumn{1}{l|}{\textit{Batch size}} & $\{1, 2, 4, 8, 32\}$    \\ %\hline
\multicolumn{1}{l|}{\textit{Proj. distance weight} $\alpha$ from Eq. (6)} & 
$\{0.0001, 0.00025, 0.0005, 0.001, 0.005, 0.01, 0.1\}$   \\ %\hline
%\multicolumn{1}{|l|}{\textit{Training epochs}} &  \\ \hline
\multicolumn{1}{l|}{$x_{shift}$} & $\{0, .1, 1\}$ \\ %\hline
\multicolumn{1}{l|}{$x_{scale}$} & $\{0.1, 1, 5\}$ \\ %\hline
\end{tabular}
\caption{Search space for different hyperparameters in the DC OPF problem}
\label{tab:opf_search}
\end{table}
\paragraph{Data generation process.}
Data for the DC OPF problem is generated artificially. First, we randomly generate a grid topology, see Figure~\ref{fig:grid} for an example. For each line, its admittance is set to $6S$. Nodal voltages are bounded between 325V and 375V, and the reference node has a fixed voltage of $v_0=350V.$ The demand in loads (power upper-bound), generators capacity (power lower-bound), and line current limits are sampled randomly from the following normal distributions: $\mathcal{N}(8000, 2500)\times$watt-hour, $\mathcal{N}(-14000, 2500)\times$watt-hour, $\mathcal{N}(25, 5)\times$ampere.
The coefficients $w$ are also sampled form the normal distributions: $\mathcal{N}(1.2, 1)$ for loads, and $\mathcal{N}(0.8, 0.1)$ for generators. Finally, all values are normalized such that $v_0$ becomes $7V$ (surprisingly, it performed better numerically than scaling $v_0$ to $1V$).
The observations $o$ consist of the true coefficients $w,$ demand of the loads, the capacity of the generators, and line current limits.

The predictor in the optimal power flow problem is the same as the one in the portfolio optimization, except for its hidden layers consisting of 256 neurons and using \textit{LeakyReLU} activation functions. The hyperparameters search space and final values are reported in Tables~\ref{tab:opf_search},\ref{tab:opf_results}

\begin{table}[t]
\centering
%\begin{tabular}{l|l|l|l|l|l|l|l}
\begin{tabular}{lr}
%\cline{2-7}
                         & Value  \\ \hline
\multicolumn{1}{l|}{\textit{Learning rate}} 
&$5\times10^{-5}$ \\ %\hline
\multicolumn{1}{l|}{\textit{Batch size}} 
&$1$ \\ %\hline
\multicolumn{1}{l|}{\textit{Penalty weight} $\alpha$ from Eq. (6)}
&$0.0001$ \\ %\hline
\multicolumn{1}{l|}{\textit{Training epochs}} 
&$250$ \\ %\hline
\multicolumn{1}{l|}{$x_{shift}$}
&$7$ \\ %\hline
\multicolumn{1}{l|}{$x_{scale}$} 
&$1$ \\ %\hline
\end{tabular}
\caption{Best performing values of the hyperparameters for the DC OPF problem}
\label{tab:opf_results}
\end{table}

% Figure environment removed

\end{document}
