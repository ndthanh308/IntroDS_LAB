\section{Discussion}
We studied the local convergence properties of the projected power method in a general class of structured PCA problems. We also established the fundamental limits of estimation in this family of problems, and studied a general family of initialization methods.
%to tackle the problem of recovering the first principal component with an underlying linear structure obscured by stochastic noises. 
Our work generalizes these statistical and algorithmic results from vanilla sparse PCA to this more general class of models. 
%To our knowledge, this paper is one of the first to give local geometric convergence for the projected power method under structural assumptions. 
We specialized our results to two commonly used notions of structure---given by tree and path sparsity---showing end-to-end estimation algorithms accompanied by evidence of computational hardness.
%Also, it is the first to provide end-to-end analysis on two widely-used typical examples of linearly structured PCA.  \\

Let us close with some potential questions for future investigation. The first is to generalize these 
%
results to other forms of structured PCA
%for fundamental limits and local geometric convergence results to the convexly/conely structured PCA problem, which was raised from the success of the conely projected power method for non-sparse PCA in 
\citep{yi2020non}. Another natural direction is consider more than a single principal component. Progress has been made towards establishing general fundamental limits in these settings~\citep{cai2021optimal}; there are also natural analogs for the projected power method in these settings and it would be interesting to analyze it under a general structural assumption along with statistical and computational limits.