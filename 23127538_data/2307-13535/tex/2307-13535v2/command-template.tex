%!TEX root = IEEETSP-main-paper.tex

% The command above makes the PDF compilable from this file if you're using SublimeText

%%%%%%%%%%%%%%%%%%%%%%%%%%%%%%%%%%%%%%%%%%%%%%%%%%%%%%%%%%%%%%%%%%%%%%
% Some basic packages

\usepackage{comment}
\usepackage{epsf}
\usepackage{graphicx}
\usepackage{subfigure}
\usepackage{bbm}
\usepackage{color}
\usepackage{nicefrac}
\usepackage{mathtools}
\usepackage{amsfonts}
%\usepackage{amsthm}
\usepackage{amsmath, bm}
\usepackage{amssymb}
\usepackage{textcomp}
\usepackage{xcolor}

% package for algorithm
% \usepackage[ruled, vlined, lined, commentsnumbered]{algorithm2e}
\usepackage{algorithm}
\usepackage{algpseudocode} 
%\newcommand{\theHalgorithm}{\arabic{algorithm}}

%\RequirePackage[colorlinks,citecolor={blue!60!black},urlcolor={blue!70!black},linkcolor={red!60!black},breaklinks,hypertexnames=false]{hyperref} %Adds links to references (without boxes)


% Alternative minipage environment, renders better in terms of spacing

\newenvironment{fminipage}%
  {\begin{Sbox}\begin{minipage}}%
  {\end{minipage}\end{Sbox}\fbox{\TheSbox}}
  
\newenvironment{algbox}[0]{\vskip 0.2in
\noindent 
\begin{fminipage}{6.3in}
}{
\end{fminipage}
\vskip 0.2in
}

% no indent for paragraph
% \setlength{\parindent}{0pt}

%%%%%%%%%%%%%%%%%%%%%%%%%%%%%%%%%%%%%%%%%%%%%%%%%%%%%%%%%%%%%%%%%%%%%%
% MACROS BELOW

%%%%%%%

% {Theorem, Proposition, Lemma, Corollary} numbered sequentially
% throughout the paper, not by section number and not combined
%%%%%%%%%%%%%%%%%%%%%%%%%%%%%%%%%%%%%%%%%%%%%%%%%%%%%%%%%%%%%%%%%%%%%%
%             PROOF, THEOREM, and FRIENDS
\newcommand{\BlackBox}{\rule{1.5ex}{1.5ex}}  % end of proof
\ifdefined\proof
    \renewenvironment{proof}{\par\noindent{\bf Proof\ }}{\hfill\BlackBox\\[2mm]}
\else
    \newenvironment{proof}{\par\noindent{\bf Proof\ }}{\hfill\BlackBox\\[2mm]}
\fi
\newtheorem{theorem}{Theorem}
\newtheorem{example}{Example}
\newtheorem{assumption}{Assumption}
\newtheorem{proposition}{Proposition}
\newtheorem{lemma}{Lemma}
\newtheorem{corollary}{Corollary}
\newtheorem{definition}{Definition}
\newtheorem{remark}{Remark}
\newtheorem{conjecture}{Conjecture}
\newtheorem{claim}{Claim}

%%%%%%%%%%%
%References (natbib): Use \citet and \citep to take full advantage

\usepackage{natbib}
\setcitestyle{round}

%\setcitestyle{numbers, square}

%%%%%%%%%%%




%%%%%%%%%%%%%%%%%%%%%%%%%%%%%%%%%%%%%%%%%%%%%%%%%%%%%%%%%%%%%%%%%%%%%%%
% Widebar command
\newlength{\widebarargwidth}
\newlength{\widebarargheight}
\newlength{\widebarargdepth}
\DeclareRobustCommand{\widebar}[1]{%
 \settowidth{\widebarargwidth}{\ensuremath{#1}}%
 \settoheight{\widebarargheight}{\ensuremath{#1}}%
 \settodepth{\widebarargdepth}{\ensuremath{#1}}%
 \addtolength{\widebarargwidth}{-0.3\widebarargheight}%
 \addtolength{\widebarargwidth}{-0.3\widebarargdepth}%
 \makebox[0pt][l]{\hspace{0.3\widebarargheight}%
   \hspace{0.3\widebarargdepth}%
   \addtolength{\widebarargheight}{0.3ex}%
   \rule[\widebarargheight]{0.95\widebarargwidth}{0.1ex}}%
 {#1}}



%%% New version of \caption puts things in smaller type, single-spaced 
%%% and indents them to set them off more from the text.
\makeatletter
\long\def\@makecaption#1#2{
       \vskip 0.8ex
       \setbox\@tempboxa\hbox{\small {\bf #1:} #2}
       \parindent 1.5em 
       \dimen0=\hsize
       \advance\dimen0 by -3em
       \ifdim \wd\@tempboxa >\dimen0
               \hbox to \hsize{
                       \parindent 0em
                       \hfil 
                       \parbox{\dimen0}{\def\baselinestretch{0.96}\small
                               {\bf #1.} #2
                               %%\unhbox\@tempboxa
                               } 
                       \hfil}
       \else \hbox to \hsize{\hfil \box\@tempboxa \hfil}
       \fi
       }
\makeatother



%% COMMENTING commands: insert more for more than two collaborators

\long\def\comment#1{}

\newcommand{\blue}[1]{\textcolor{blue}{#1}}
\newcommand{\apcomment}[1]{{\bf{{\blue{{AP --- #1}}}}}}

\newcommand{\red}[1]{\textcolor{red}{#1}}
\newcommand{\mqcomment}[1]{{\bf{{\red{{MQ --- #1}}}}}}
\newcommand{\gwcomment}[1]{{\bf{{\red{{GW --- #1}}}}}}


% Some vector/matrix norms. Convention is that we use triple lines for matrix 
% norms and double lines for vector norms.
\newcommand{\matnorm}[3]{|\!|\!| #1 | \! | \!|_{{#2}, {#3}}}
\newcommand{\matsnorm}[2]{|\!|\!| #1 | \! | \!|_{{#2}}}
\newcommand{\vecnorm}[2]{\| #1\|_{#2}}

% Some useful and common operators
% Independence symbol
\newcommand\indpt{\protect\mathpalette{\protect\independenT}{\perp}} 
\def\independenT#1#2{\mathrel{\rlap{$#1#2$}\mkern2mu{#1#2}}}
\newcommand{\defn}{\coloneqq} %This is a properly aligned definition symbol
\DeclareMathOperator*{\minimize}{minimize}
\DeclareMathOperator{\modd}{mod}
\DeclareMathOperator{\diag}{diag}
\DeclareMathOperator{\Var}{var}
\DeclareMathOperator{\cov}{cov}
\DeclareMathOperator*{\argmin}{argmin}
\DeclareMathOperator*{\argmax}{argmax}
\DeclareMathOperator{\floor}{floor}
\DeclareMathOperator{\vol}{vol}
\DeclareMathOperator{\rank}{rank}
\DeclareMathOperator{\card}{card}
\DeclareMathOperator{\range}{range}
\DeclareMathOperator{\dist}{\rho}



% Specific norms
\newcommand{\nucnorm}[1]{\ensuremath{\matsnorm{#1}{\footnotesize{\mbox{nuc}}}}}
\newcommand{\fronorm}[1]{\ensuremath{\matsnorm{#1}{\tiny{\mbox{F}}}}}
\newcommand{\opnorm}[1]{\ensuremath{\matsnorm{#1}{\tiny{\mbox{op}}}}}
\newcommand{\eucnorm}[1]{\left\|#1 \right\|}
\newcommand{\linfnorm}[1]{\left\|#1 \right\|_\infty}

% Bracket notation to make life easier when writing long expressions
\newcommand{\paren}[1]{\left\{ #1 \right\} }
% Inner product
\newcommand{\inprod}[2]{\ensuremath{\langle #1 , \, #2 \rangle}}

% The macros below are useful if you see these quantities changing over the course of writing
\newcommand{\numdim}{\ensuremath{d}}
\newcommand{\numobs}{\ensuremath{n}}

% Distances between distributions
\newcommand{\kull}[2]{\ensuremath{D_{\mathsf{KL}}(#1\; \| \; #2)}}
\newcommand{\KL}{\ensuremath{D_{\mathsf{KL}}}}
\newcommand{\TV}{\ensuremath{d_{\mathsf{TV}}}}
\newcommand{\Hel}{\ensuremath{d_{\mathsf{hel}}}}

% Probability
\newcommand{\EE}{\ensuremath{{\mathbb{E}}}}
\newcommand{\Prob}{\ensuremath{{\mathbb{P}}}}
\newcommand{\ind}[1]{\ensuremath{{\mathbb{I}\left\{ #1 \right\}}}}

%Numbering (use with \overset to label steps)
\newcommand{\1}{\ensuremath{{\sf (i)}}}
\newcommand{\2}{\ensuremath{{\sf (ii)}}}
\newcommand{\3}{\ensuremath{{\sf (iii)}}}
\newcommand{\4}{\ensuremath{{\sf (iv)}}}
\newcommand{\5}{\ensuremath{{\sf (v)}}}
\newcommand{\6}{\ensuremath{{\sf (vi)}}}

%Eigenvector / eigenvalue related notation
\newcommand{\eig}[1]{\ensuremath{\lambda_{#1}}}
\newcommand{\eigmax}{\ensuremath{\eig{\max}}}
\newcommand{\eigmin}{\ensuremath{\eig{\min}}}


% Distributions
\newcommand{\NORMAL}{\ensuremath{\mathcal{N}}}
\newcommand{\BER}{\ensuremath{\mbox{\sf Ber}}}
\newcommand{\BIN}{\ensuremath{\mbox{\sf Bin}}}
\newcommand{\Hyp}{\ensuremath{\mbox{\sf Hyp}}}

% Script letters
\newcommand{\Aspace}{\ensuremath{\mathcal{A}}}
\newcommand{\Xspace}{\ensuremath{\mathcal{X}}}
\newcommand{\Yspace}{\ensuremath{\mathcal{Y}}}
\newcommand{\Zspace}{\ensuremath{\mathcal{Z}}}
\newcommand{\Fspace}{\ensuremath{\mathcal{F}}}

% Population and sample loss functions
\newcommand{\loss}{\ensuremath{\mathcal{L}}}
\newcommand{\lossn}[1]{\ensuremath{\mathcal{L}^n \left( #1 \right)}}

% True parameter
\newcommand{\thetastar}{\ensuremath{\theta^*}}
% Estimates
\newcommand{\thetahat}{\ensuremath{\widehat{\theta}}}
\newcommand{\thetabar}{\ensuremath{\overline{\theta}}}
\newcommand{\thetatil}{\ensuremath{\widetilde{\theta}}}

% Order notation
\newcommand{\order}{\ensuremath{\mathcal{O}}}
\newcommand{\ordertil}{\ensuremath{\widetilde{\order}}}
\newcommand{\Omegatil}{\ensuremath{\widetilde{\Omega}}}


% Macro for Structured PCA
\newcommand{\ESest}{\widehat{\bm{v}}_{\mathsf{ES}}}
\newcommand{\EST}{{\tt Est}}
\newcommand{\DET}{{\tt Det}}
\newcommand{\subindex}{j}


