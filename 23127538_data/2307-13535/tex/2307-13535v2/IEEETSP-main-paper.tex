 \documentclass[11pt,twoside]{article}

\usepackage{geometry}
\geometry{letterpaper, margin=1in}

%!TEX root = IEEETSP-main-paper.tex

% The command above makes the PDF compilable from this file if you're using SublimeText

%%%%%%%%%%%%%%%%%%%%%%%%%%%%%%%%%%%%%%%%%%%%%%%%%%%%%%%%%%%%%%%%%%%%%%
% Some basic packages

\usepackage{comment}
\usepackage{epsf}
\usepackage{graphicx}
\usepackage{subfigure}
\usepackage{bbm}
\usepackage{color}
\usepackage{nicefrac}
\usepackage{mathtools}
\usepackage{amsfonts}
%\usepackage{amsthm}
\usepackage{amsmath, bm}
\usepackage{amssymb}
\usepackage{textcomp}
\usepackage{xcolor}

% package for algorithm
% \usepackage[ruled, vlined, lined, commentsnumbered]{algorithm2e}
\usepackage{algorithm}
\usepackage{algpseudocode} 
%\newcommand{\theHalgorithm}{\arabic{algorithm}}

%\RequirePackage[colorlinks,citecolor={blue!60!black},urlcolor={blue!70!black},linkcolor={red!60!black},breaklinks,hypertexnames=false]{hyperref} %Adds links to references (without boxes)


% Alternative minipage environment, renders better in terms of spacing

\newenvironment{fminipage}%
  {\begin{Sbox}\begin{minipage}}%
  {\end{minipage}\end{Sbox}\fbox{\TheSbox}}
  
\newenvironment{algbox}[0]{\vskip 0.2in
\noindent 
\begin{fminipage}{6.3in}
}{
\end{fminipage}
\vskip 0.2in
}

% no indent for paragraph
% \setlength{\parindent}{0pt}

%%%%%%%%%%%%%%%%%%%%%%%%%%%%%%%%%%%%%%%%%%%%%%%%%%%%%%%%%%%%%%%%%%%%%%
% MACROS BELOW

%%%%%%%

% {Theorem, Proposition, Lemma, Corollary} numbered sequentially
% throughout the paper, not by section number and not combined
%%%%%%%%%%%%%%%%%%%%%%%%%%%%%%%%%%%%%%%%%%%%%%%%%%%%%%%%%%%%%%%%%%%%%%
%             PROOF, THEOREM, and FRIENDS
\newcommand{\BlackBox}{\rule{1.5ex}{1.5ex}}  % end of proof
\ifdefined\proof
    \renewenvironment{proof}{\par\noindent{\bf Proof\ }}{\hfill\BlackBox\\[2mm]}
\else
    \newenvironment{proof}{\par\noindent{\bf Proof\ }}{\hfill\BlackBox\\[2mm]}
\fi
\newtheorem{theorem}{Theorem}
\newtheorem{example}{Example}
\newtheorem{assumption}{Assumption}
\newtheorem{proposition}{Proposition}
\newtheorem{lemma}{Lemma}
\newtheorem{corollary}{Corollary}
\newtheorem{definition}{Definition}
\newtheorem{remark}{Remark}
\newtheorem{conjecture}{Conjecture}
\newtheorem{claim}{Claim}

%%%%%%%%%%%
%References (natbib): Use \citet and \citep to take full advantage

\usepackage{natbib}
\setcitestyle{round}

%\setcitestyle{numbers, square}

%%%%%%%%%%%




%%%%%%%%%%%%%%%%%%%%%%%%%%%%%%%%%%%%%%%%%%%%%%%%%%%%%%%%%%%%%%%%%%%%%%%
% Widebar command
\newlength{\widebarargwidth}
\newlength{\widebarargheight}
\newlength{\widebarargdepth}
\DeclareRobustCommand{\widebar}[1]{%
 \settowidth{\widebarargwidth}{\ensuremath{#1}}%
 \settoheight{\widebarargheight}{\ensuremath{#1}}%
 \settodepth{\widebarargdepth}{\ensuremath{#1}}%
 \addtolength{\widebarargwidth}{-0.3\widebarargheight}%
 \addtolength{\widebarargwidth}{-0.3\widebarargdepth}%
 \makebox[0pt][l]{\hspace{0.3\widebarargheight}%
   \hspace{0.3\widebarargdepth}%
   \addtolength{\widebarargheight}{0.3ex}%
   \rule[\widebarargheight]{0.95\widebarargwidth}{0.1ex}}%
 {#1}}



%%% New version of \caption puts things in smaller type, single-spaced 
%%% and indents them to set them off more from the text.
\makeatletter
\long\def\@makecaption#1#2{
       \vskip 0.8ex
       \setbox\@tempboxa\hbox{\small {\bf #1:} #2}
       \parindent 1.5em 
       \dimen0=\hsize
       \advance\dimen0 by -3em
       \ifdim \wd\@tempboxa >\dimen0
               \hbox to \hsize{
                       \parindent 0em
                       \hfil 
                       \parbox{\dimen0}{\def\baselinestretch{0.96}\small
                               {\bf #1.} #2
                               %%\unhbox\@tempboxa
                               } 
                       \hfil}
       \else \hbox to \hsize{\hfil \box\@tempboxa \hfil}
       \fi
       }
\makeatother



%% COMMENTING commands: insert more for more than two collaborators

\long\def\comment#1{}

\newcommand{\blue}[1]{\textcolor{blue}{#1}}
\newcommand{\apcomment}[1]{{\bf{{\blue{{AP --- #1}}}}}}

\newcommand{\red}[1]{\textcolor{red}{#1}}
\newcommand{\mqcomment}[1]{{\bf{{\red{{MQ --- #1}}}}}}
\newcommand{\gwcomment}[1]{{\bf{{\red{{GW --- #1}}}}}}


% Some vector/matrix norms. Convention is that we use triple lines for matrix 
% norms and double lines for vector norms.
\newcommand{\matnorm}[3]{|\!|\!| #1 | \! | \!|_{{#2}, {#3}}}
\newcommand{\matsnorm}[2]{|\!|\!| #1 | \! | \!|_{{#2}}}
\newcommand{\vecnorm}[2]{\| #1\|_{#2}}

% Some useful and common operators
% Independence symbol
\newcommand\indpt{\protect\mathpalette{\protect\independenT}{\perp}} 
\def\independenT#1#2{\mathrel{\rlap{$#1#2$}\mkern2mu{#1#2}}}
\newcommand{\defn}{\coloneqq} %This is a properly aligned definition symbol
\DeclareMathOperator*{\minimize}{minimize}
\DeclareMathOperator{\modd}{mod}
\DeclareMathOperator{\diag}{diag}
\DeclareMathOperator{\Var}{var}
\DeclareMathOperator{\cov}{cov}
\DeclareMathOperator*{\argmin}{argmin}
\DeclareMathOperator*{\argmax}{argmax}
\DeclareMathOperator{\floor}{floor}
\DeclareMathOperator{\vol}{vol}
\DeclareMathOperator{\rank}{rank}
\DeclareMathOperator{\card}{card}
\DeclareMathOperator{\range}{range}
\DeclareMathOperator{\dist}{\rho}



% Specific norms
\newcommand{\nucnorm}[1]{\ensuremath{\matsnorm{#1}{\footnotesize{\mbox{nuc}}}}}
\newcommand{\fronorm}[1]{\ensuremath{\matsnorm{#1}{\tiny{\mbox{F}}}}}
\newcommand{\opnorm}[1]{\ensuremath{\matsnorm{#1}{\tiny{\mbox{op}}}}}
\newcommand{\eucnorm}[1]{\left\|#1 \right\|}
\newcommand{\linfnorm}[1]{\left\|#1 \right\|_\infty}

% Bracket notation to make life easier when writing long expressions
\newcommand{\paren}[1]{\left\{ #1 \right\} }
% Inner product
\newcommand{\inprod}[2]{\ensuremath{\langle #1 , \, #2 \rangle}}

% The macros below are useful if you see these quantities changing over the course of writing
\newcommand{\numdim}{\ensuremath{d}}
\newcommand{\numobs}{\ensuremath{n}}

% Distances between distributions
\newcommand{\kull}[2]{\ensuremath{D_{\mathsf{KL}}(#1\; \| \; #2)}}
\newcommand{\KL}{\ensuremath{D_{\mathsf{KL}}}}
\newcommand{\TV}{\ensuremath{d_{\mathsf{TV}}}}
\newcommand{\Hel}{\ensuremath{d_{\mathsf{hel}}}}

% Probability
\newcommand{\EE}{\ensuremath{{\mathbb{E}}}}
\newcommand{\Prob}{\ensuremath{{\mathbb{P}}}}
\newcommand{\ind}[1]{\ensuremath{{\mathbb{I}\left\{ #1 \right\}}}}

%Numbering (use with \overset to label steps)
\newcommand{\1}{\ensuremath{{\sf (i)}}}
\newcommand{\2}{\ensuremath{{\sf (ii)}}}
\newcommand{\3}{\ensuremath{{\sf (iii)}}}
\newcommand{\4}{\ensuremath{{\sf (iv)}}}
\newcommand{\5}{\ensuremath{{\sf (v)}}}
\newcommand{\6}{\ensuremath{{\sf (vi)}}}

%Eigenvector / eigenvalue related notation
\newcommand{\eig}[1]{\ensuremath{\lambda_{#1}}}
\newcommand{\eigmax}{\ensuremath{\eig{\max}}}
\newcommand{\eigmin}{\ensuremath{\eig{\min}}}


% Distributions
\newcommand{\NORMAL}{\ensuremath{\mathcal{N}}}
\newcommand{\BER}{\ensuremath{\mbox{\sf Ber}}}
\newcommand{\BIN}{\ensuremath{\mbox{\sf Bin}}}
\newcommand{\Hyp}{\ensuremath{\mbox{\sf Hyp}}}

% Script letters
\newcommand{\Aspace}{\ensuremath{\mathcal{A}}}
\newcommand{\Xspace}{\ensuremath{\mathcal{X}}}
\newcommand{\Yspace}{\ensuremath{\mathcal{Y}}}
\newcommand{\Zspace}{\ensuremath{\mathcal{Z}}}
\newcommand{\Fspace}{\ensuremath{\mathcal{F}}}

% Population and sample loss functions
\newcommand{\loss}{\ensuremath{\mathcal{L}}}
\newcommand{\lossn}[1]{\ensuremath{\mathcal{L}^n \left( #1 \right)}}

% True parameter
\newcommand{\thetastar}{\ensuremath{\theta^*}}
% Estimates
\newcommand{\thetahat}{\ensuremath{\widehat{\theta}}}
\newcommand{\thetabar}{\ensuremath{\overline{\theta}}}
\newcommand{\thetatil}{\ensuremath{\widetilde{\theta}}}

% Order notation
\newcommand{\order}{\ensuremath{\mathcal{O}}}
\newcommand{\ordertil}{\ensuremath{\widetilde{\order}}}
\newcommand{\Omegatil}{\ensuremath{\widetilde{\Omega}}}


% Macro for Structured PCA
\newcommand{\ESest}{\widehat{\bm{v}}_{\mathsf{ES}}}
\newcommand{\EST}{{\tt Est}}
\newcommand{\DET}{{\tt Det}}
\newcommand{\subindex}{j}






\begin{document}

\begin{center}

  {\bf{\LARGE{Do algorithms and barriers for sparse principal \\
  \mbox{component analysis extend to other structured settings?}}}}
  
\vspace*{.2in}

{\large{
\begin{tabular}{ccc}
Guanyi Wang$^{\ddagger}$, Mengqi Lou$^{\star}$, Ashwin Pananjady$^{\star, \dagger}$
\end{tabular}
}}

\vspace*{.2in}

\begin{tabular}{c}
%\apcomment{Guanyi update affiliation.}
Department of Industrial Systems Engineering and Management$^\ddagger$, National University of Singapore \\
Schools of Industrial and Systems Engineering$^\star$ and Electrical and Computer Engineering$^\dagger$ \\
Georgia Institute of Technology
\end{tabular}

\vspace*{.2in}
\today
\vspace*{.2in}

\begin{abstract}
We study a principal component analysis problem under the spiked Wishart model in which the structure in the signal is captured by a class of union-of-subspace models. This general class includes vanilla sparse PCA as well as its variants with graph sparsity. With the goal of studying these problems under a unified statistical and computational lens, we establish fundamental limits that depend on the geometry of the problem instance, and show that a natural projected power method exhibits local convergence to the statistically near-optimal neighborhood of the solution. We complement these results with end-to-end analyses of two important special cases given by path and tree sparsity in a general basis, showing initialization methods and matching evidence of computational hardness. Overall, our results indicate that several of the phenomena observed for vanilla sparse PCA extend in a natural fashion to its structured counterparts.
\end{abstract}
\end{center}

%\begin{IEEEkeywords}
%principal component analysis, structured sparsity, nonconvex iterative optimization, computational hardness. 
%\end{IEEEkeywords}

%%%%%%%%%%%%%%%%%%%%%%%%%%%%%%%%%%%%%%%%%%%%%%%%%%%%%%%%%%%%%%%%%%%%%%%%%


% Figure environment removed

\section{Introduction}
Automatic 3D reconstruction of clothed humans using image inputs has gained increasing significance due to its potential applications in a wide array of AR/VR scenarios. High-fidelity reconstructions typically depend on sophisticated capture systems, which are developed with dense camera arrays~\cite{collet2015high,joo2015panoptic,joo2018total}, programmable light-stages~\cite{Vlasic2009, guo2019relightables}, and depth sensors~\cite{newcombe2011kinectfusion,DoubleFusion,BodyFusion,dou2016fusion4d,newcombe2015dynamicfusion}. However, stringent capture environments equipped with complex hardware pose significant challenges for consumer-level applications.


In this context, considerable research effort has been dedicated to developing methods that allow for more flexible capture configurations, such as utilizing a few RGB inputs. Among these works, learning implicit functions \cite{iccv2020PIFu, saito2020pifuhd, hong2021stereopifu} has proven effective in achieving highly detailed reconstructions by integrating the advancements of deep neural networks. These methods employ large multi-layer perceptrons (MLPs) to predict the occupancy probability or truncated signed distance function (TSDF) value of every queried 3D point based on its associated local feature, which is extracted from images. They can recover a continuous surface at arbitrary resolutions without topology restrictions.


However, in typical MLP-based implicit networks, the occupancy or TSDF value at each location is solved independently with planar image features, rendering them less capable of addressing challenging cases such as occlusions. Consequently, these methods suffer from generalization and robustness issues, particularly when tackling strong occlusions caused by large motion or multiple interacting humans. 
Some follow-up studies  \cite{zheng2021deepmulticap,zheng2021pamir,huang2020arch} utilize an extra geometric model, SMPL~\cite{Loper2015}, to improve robustness by introducing strong shape priors. 
Their success typically relies on the assumption of geometrical similarity \cite{huang2020arch} between the shape prior and target reconstruction, making them intractable for handling complex cases with loose clothes and sensitive to errors in SMPL model fitting.



%\ping{this paragraph sounds like `TSDF is better than MLP/SMPL, and we use TSDF to solve the problem'. But in Sec 3, we are telling a different story, saying `MLP needs a 3D convolutional encoder'. We need to make these two sections consistent.}\sicong{I think in this paragraph we claim that the TSDF}


%We opt for Trucated Signed Distance Funtion (TSDF) volumetric representations as they are naturally suitable for convolution operations, which have shown remarkable performance for learning hierarchical features on 2D visual perception tasks \cite{SunXLW19}. 
%Meanwhile, TSDF also describes the gradual geometry change around shape surface, which is not reflected by occupancy volume. 

We instead revisit the 3D volumetric representation and resort to 3D convolutional neural networks (CNNs) for feature learning, due to their impressive performance in feature learning and the ability to incorporate spatial context. However, volumetric methods and 3D convolution involve discretization, which might raise concerns regarding whether a discretized volume can preserve subtle geometric details as continuous representations learned in implicit functions. We investigate the relationship between volume resolution and quantization error on synthetic data by converting target mesh objects to TSDF volumes, as shown in Figure~\ref{fig:quantization_error}. We observe that the quantization errors are significantly reduced by increasing volume resolution and become nearly negligible when reaching a relatively high resolution (e.g., 512 or higher). In other words, achieving fine-detailed reconstruction is not supposed to be restricted by the use of volume representations as long as a proper volume resolution is utilized. Therefore, we present a method with high-resolution feature volumes, e.g., 256 and 512, while traditional volumetric methods \cite{varol18_bodynet,gilbert2018volumetric} are often limited to much lower resolutions, such as 32 or 128.



On the other hand, an increase in volume resolution may lead to a cubic growth of memory overhead \cite{8100085}. Reducing memory costs while guaranteeing the granularity of volumetric representations is necessary for pursuing high-quality reconstruction. Thus, we adopt a coarse-to-fine approach and cull away irrelevant voxels to build a sparse high-resolution feature volume. At the coarse level, the network computes an initial TSDF by applying a U-Net with sparse 3D CNN \cite{3DSemanticSegmentationWithSubmanifoldSparseConvNet} on the sparse feature volume, which is carved by a visual hull. Through our experiments, it turns out that more than 95\% of the volume grids are discarded by the visual hull culling, making the sparse 3D CNN efficient. At the fine level, the network focuses on a narrow band near the zero-level set of the initial TSDF and discretizes the narrow band with smaller voxels. By employing this narrow-band culling, we further shrink the sampling space, resulting in a relatively small range of grid numbers (usually 300K--500K in our experiments) even with a high volume resolution of 512. The remaining voxels in the narrow band are associated with features that fuse high-frequency information from the computed normal maps upon the low-frequency shape from the coarse level to compute the TSDF at high resolution. The final mesh is then extracted from the TSDF using the Marching-Cube algorithm ~\cite{Lorensen87marchingcubes}.
% Different from the u-net sturcture to preserve global topology context, we then apply a shallow 3dcnn to compute the final TSDF $D_{final}$ which contain more local geometry detail.




% \ping{this paragraph can be expanded. It is an important contribution and often ignored by other works. stress on the novel idea of regressing blending weights instead of colors}

In addition to geometry, high-quality mesh texture is also a crucial factor contributing to visual appearance. Directly computing a color field in 3D space, as in \cite{iccv2020PIFu}, struggles to capture high-frequency texture details, while the neural radiance field (NeRF) \cite{yu2020pixelnerf} or the DoubleField~\cite{shao2022doublefield} require expensive per-instance optimization and are often unstable for sparse input images. In contrast, we adopt an image-based rendering approach to compute a texture atlas map, which is efficient and widely supported in existing computer graphics tools. 
Specifically, we compute a blending weight at each 3D point on the mesh surface to determine its color as a weighted average of the colors at its image projections. The blending weights can be computed at a relatively coarse resolution, e.g., 512 volume resolution in our case, and leave texture details to the high-resolution images, such as 1K or 2K. Unlike previous methods that generate blurry texturing results under sparse input, our method generalizes well on both synthetic and real data with just a few input views. 
Figure~\ref{fig:teaser} shows two examples reconstructed by our method. Despite the challenging garment, pose, and occlusion, our method recovers faithful shape, normal, and texture on the right.

%with a wide variety of poses and clothing styles, and it is also adaptive to handle input image with arbitrary resolutions.
%\sicong{For this concern we claim that when the resolution of dicretized volume meets certain threshold (which is 256 in our experiment), the quantization error can be neglected.} 



In summary, the main contributions of this paper are as follows:
\begin{itemize}
\vspace{-0.1in}
  \item 
  We revisit the 3D volumetric representation and demonstrate that it can support clothed human reconstruction with equal or even better performance compared to implicit representation. 
  \item 
  We develop a memory and computation-efficient method for high-resolution volumetric reconstruction using sophisticated sparse 3D CNN, coarse-to-fine estimation, and voxel culling by visual hull and narrow bands. 
  \item 
  We introduce a novel method to compute a texture atlas map, which captures rich appearance details from high-resolution input images.
  \item 
  We achieve impressive results on standard benchmark datasets Twindom and MultiHuman, significantly reducing the point-2-surface (P2S) precision to approximately 0.2cm from just six input views, with more than $50\%$ error reduction compared to the state-of-the-art methods, including DoubleField~\cite{shao2022doublefield} and PIFuHD~\cite{saito2020pifuhd}.
\end{itemize} 

%!TEX root = main-paper-template.tex


\section{Problem setting, background, and examples}\label{sec:setting-background}

Throughout this paper, we operate under the spiked Wishart model. Assume that our data set consists of $n$ i.i.d. samples $\{\bm{x}_i\}_{i = 1}^n$ drawn from a $d$-dimensional Gaussian distribution with zero-mean and covariance
\begin{align}
    \bm{\Sigma} := \lambda \bm{v}_* \bm{v}_*^{\top} + \bm{I}_{d \times d}. \notag
\end{align}
For brevity, we use $\mathcal{D}(\lambda; \bm{v}_*) := \mathcal{N}(\bm{0}_d, \lambda \bm{v}_* \bm{v}_*^{\top} + \bm{I}_{d \times d})$ to denote the distribution of each $\bm{x}_i$.
Here $\lambda > 0$ represents the \emph{strength of the signal}, and $\bm{v}_*$ is a $d$-dimensional, unit-norm \emph{ground truth} vector that we wish to estimate. In addition to the unit norm condition, we also assume the inclusion $\bm{v}^* \in \mathcal{M}$, where $\mathcal{M}$ is a known union of subspaces satisfying a certain \emph{union of linear structures} assumption defined below.


\begin{definition} \label{cond:linear-structure}
\textbf{Union of linear structures condition.} Let $\mathcal{B} := \{\phi_1, \ldots, \phi_d\}$ be an orthonormal basis of $\mathbb{R}^d$ and $\mathcal{L} := \left\{ L_1, \ldots, L_M \right\}$ be a collection of $M$ distinct linear subspaces such that for each $m \in [M]$, we have $L_m = \text{span}(\mathcal{B}_m)$ for some $\mathcal{B}_m \subseteq \mathcal{B}$. We say set $\mathcal{M}$ obeys the union of linear structures condition if
\begin{align*}
    \mathcal{M} := \bigcup_{m = 1}^M L_m,
\end{align*}
i.e., $\mathcal{M}$ is the union of all linear subspaces in $\mathcal{L}$.
\end{definition}

% \apcomment{deleted a paragraph here; will move to later.}
%\iffalse
%By a sufficiency argument, the problem can be phrased as one of estimating $\bm{v}_*$ from the sample covariance matrix $\widehat{\bm{\Sigma}} = \frac{1}{n} \sum_{i =1}^n \bm{x}_i \bm{x}_i^\top$. We use 
%\begin{align}
%    \widehat{\bm{v}}_{\mathsf{ES}} := \argmax_{\bm{v}} ~  \bm{v}^{\top} \widehat{\bm{\Sigma}} \bm{v} ~~\text{s.t.}~~ \|\bm{v}\|_2 = 1, ~~ \bm{v} \in \mathcal{M} \label{eq:linearly-structured-PCA}
%\end{align} 
%to denote the estimator of $\bm{v}_*$ from exhaustive search by seeking the leading principal component within the set $\mathcal{M}$.
%\fi

%\apcomment{Put notation for identity matrix in background notation.} \gwcomment{Done.}
%In short, define above distribution as $\mathcal{D}(\lambda; \bm{v}_*) := \mathcal{N}(\bm{0}_d, \bm{\Sigma}) = \mathcal{N}(\bm{0}_d, \lambda \bm{v}_* \bm{v}_*^{\top} + \bm{I}_{d \times d})$. 

% As alluded to in the introduction, we also place a structural assumption on $\bm{v}_*$ in addition to a unit-norm constraint. We call this the \emph{linear structure condition} for the rest of the paper. 
%\apcomment{I think a more natural way to say this is that $\bm{v}_*$ is in the union-of-subspaces $\mathcal{M} = \cup_{i = 1}^M L_i$ and $\mathcal{M}$ satisfies the linear structure condition. Overall, $\mathcal{M}$ is an important object and should be defined up front.} \gwcomment{Update the linear structure condition wrt $\mathcal{M}$ as your comments.}
%The linearly structured PCA is a generalized PCA which assumes that the true PC $\bm{v}_*$ (vector with ``hidden data'') is a unit $\ell_2$-norm vector and satisfies the below-mentioned \emph{linear structure condition}. 

% \begin{definition} \label{def:wishart-model}
% \textbf{Wishart Model.} We say $\{\bm{x}_i\}_{i = 1}^n$ satisfies the Wishart Model if and only if the samples are i.i.d. generated from the $d$-dimensional Gaussian distribution with zero-mean and covariance
% \begin{align}
%     \bm{\Sigma} := \lambda \bm{v}_* \bm{v}_*^{\top} + \bm{I}_{d \times d}, \notag
% \end{align}
% where $\lambda$ represents the strength of the signal, $\bm{v}_*$ satisfies the linear structure condition~\ref{cond:linear-structure}, and $\bm{I}_{d \times d}$ denotes the $d$-by-$d$ identity matrix. In short, define above distribution as $\mathcal{D}(\lambda; \bm{v}_*) := \mathcal{N}(\bm{0}_d, \bm{\Sigma}) = \mathcal{N}(\bm{0}_d, \lambda \bm{v}_* \bm{v}_*^{\top} + \bm{I}_{d \times d})$. 
% \end{definition}

% removed this paragraph because the optimization problem comes later, when you prove guarantees for it.

% Therefore, the goal of the linearly structured PCA is to estimate the unknown true PC $\bm{v}_*$ via a given set of noisy samples $\{\bm{x}_i\}_{i = 1}^n$ generated from the Wishart Model with linear structure condition \ref{cond:linear-structure}. In order to estimate $\bm{v}_*$, it is natural to consider the following optimization problem~\eqref{eq:linearly-structured-PCA}
% \begin{align}
%     \begin{array}{rlll}
%         \widehat{\bm{v}} := \argmax_{\bm{v}} & \bm{v}^{\top} \widehat{\bm{\Sigma}} \bm{v}  \\
%         \text{s.t.} & \|\bm{v}\|_2 = 1 \\
%         & \bm{v} \in \mathcal{M} := \bigcup_{m = 1}^M L_m
%     \end{array} \label{eq:linearly-structured-PCA}
% \end{align}
% and use the optimal solution $\widehat{\bm{v}} = \widehat{\bm{v}}(\bm{x}_1, \ldots, \bm{x}_n)$ as an estimator for the unknown true PC $\bm{v}_*$, where $\widehat{\bm{\Sigma}} := \frac{1}{n} \bm{X} \bm{X}^{\top}$ is the sample covariance matrix with $\bm{X} := (\bm{x}_1 ~|~ \cdots ~|~ \bm{x}_n) \in \mathbb{R}^{d \times n}$, and $\mathcal{M} := \bigcup_{m = 1}^M L_m$ is the union of linear subspaces $L_1, \ldots, L_M \in \mathcal{L}$ as given in the linear structure condition~\ref{cond:linear-structure}.

\begin{remark} \label{rem:eq-structured}
It is worth noting that the union of linear structures condition in Definition~\ref{cond:linear-structure} resembles a structured sparsity condition. Indeed, using the rotation invariance of the Gaussian distribution, the problem of estimating $\bm{v}_*$ from observations $\{\bm{x}_i\}_{i = 1}^n$ is \emph{statistically} equivalent to estimating the structured-sparse vector $\bm{\Phi}^\top \bm{v}_*$ from $\{ \bm{\Phi}^\top \bm{x}_{i} \}_{i = 1}^n$, where $\bm{\Phi} \in \mathbb{R}^{d \times d}$ is an orthonormal matrix with columns $\phi_1, \ldots, \phi_d$. However, the two problems may not be \emph{computationally} equivalent when $\bm{\Phi}$ is unknown. We provide an example in Appendix~\ref{app:time-consuming-case} to illustrate that if an efficient projection oracle onto the union of subspaces $\mathcal{M}$ is accessible, then it is more computationally efficient to estimate the vector $\bm{v}_*$ directly, rather than to estimate $\bm{\Phi}^\top \bm{v}_*$ from $\{ \bm{\Phi}^\top \bm{x}_{i} \}_{i = 1}^n$ by first computing $\bm{\Phi}$. Accordingly, the rest of the paper assumes that $\bm{\Phi}$ is unknown, and that we have access to a projection oracle onto the union of subspaces $\mathcal{M}$.
\end{remark}
%The attentive reader may notice that the linear structure condition in Definition~\ref{cond:linear-structure} resembles a structured sparsity condition. 
%	Indeed, using the rotation invariance of the Gaussian distribution, estimating $\bm{v}_*$ from observations $\{\bm{x}_i\}_{i = 1}^n$ is statistically equivalent to estimating the structured-sparse vector $\bm{\Phi}^\top \bm{v}_*$ from $\{ \bm{\Phi}^\top \bm{x}_{i} \}_{i = 1}^n$ with $\bm{\Phi} \in \mathbb{R}^{d \times d}$ the orthonormal matrix with columns $\phi_1, \ldots, \phi_d$. While the above two problems are \emph{statistically} equivalent when $\bm{\Phi}$ is known, they may not be \emph{computationally} equivalent when $\bm{\Phi}$ is unknown. To illustrate, we provide an example (see Example~\ref{exmp:time-consuming-case} of Appendix~\ref{app:time-consuming-case}) in which the linear structure vector $\bm{v}_*$ can be estimated  faster than estimating $\bm{\Phi}^\top \bm{v}_*$ from $\{ \bm{\Phi}^\top \bm{x}_{i} \}_{i = 1}^n$ without computing $\bm{\Phi}$, if an efficient projection oracle onto the union of subspaces $\mathcal{M}$ is accessible. Accordingly, the rest of this paper assumes the linear structure condition in Definition~\ref{cond:linear-structure} with a general, unknown $\bm{\Phi}$, combined with an efficient, known projection oracle. 



%{\color{orange}
% Attentive readers may note that the linear structure condition in Definition~\ref{cond:linear-structure} resembles a sparsity setting if the size of $\mathcal{B}_m$ is upper bounded by some fixed sparsity parameter $k$ for all $1\leq m\leq M$. Indeed, the transformed ground truth $\bm{\Phi}^\top \bm{v}_*$ is at most $k$-sparse. Using the rotation invariance of the Gaussian distribution, it is \emph{statistically} equivalent to estimate the sparse vector $\bm{\Phi}^\top \bm{v}_*$ from observations $\{ \bm{\Phi}^\top \bm{x}_{i} \}_{i = 1}^n$. Hence, once one can compute $\bm{\Phi}$ exactly, the linear structure setting can be reduced to the sparsity setting. However, computing $\bm{\Phi}$ is \emph{computationally} more expensive than using the projection oracle (see Definition~\ref{assump:exact-proj}) under some cases. To illustrate, we provide an Example~\ref{exmp:time-consuming-case} which shows that the running time of computing $\bm{\Phi}$ is $O(d^{3})$ while the projection oracle takes $O(d^{2})$. Combined with the results on local convergence of the proposed method (Theorem~\ref{thm:convergence}), the above illustration indicates that $\bm{v}_*$ can be more efficiently estimated via proposed method than computing $\bm{\Phi}$. Accordingly, the rest of this paper assumes the linear structure condition in Definition~\ref{cond:linear-structure} with general, unknown $\bm{\Phi}$.} \apcomment{What exactly does this mean? What do you have oracle access to?} \gwcomment{Modified the above two sentences, please have a check.}

% Indeed, every linear subspace $L_m \in \mathcal{L}$ is the span of a subset of vectors in $\mathcal{B}$ with $L_m = \textup{span}(\mathcal{B}_m)$ for some $\mathcal{B}_m \subseteq \mathcal{B}$. Let $\bm{\Phi} = [\phi_1\;|\cdots|\;\phi_{d}] \in \mathbb{R}^{d \times d}$.


% While the two problems are statistically equivalent when $\bm{\Phi}$ is known, they may not be equivalent \emph{computationally} when $\bm{\Phi}$ is unknown. To illustrate, we provide an example of an instance (in Example~\ref{exmp:time-consuming-case} of Appendix~\ref{app:time-consuming-case}) {\color{orange}that runs much faster than computing the matrix $\bm{\Phi}$ exactly, once the projection oracle (step 4 of Algorithm~\ref{alg:PPM}) is efficient\footnote{Actually, in some typical examples, such exact projection is very fast.}.}
%in which a projection oracle onto the union of subspaces $\mathcal{M}$ can be implemented much faster than the time taken to find the matrix $\bm{\Phi}$.
%Combined with our results in the sequel showing local convergence of the projected power method, this indicates that $\bm{v}_*$ can be recovered without having to compute $\bm{\Phi}$. 

%By the rotation invariance of the Gaussian distribution, it is thus statistical
% It is easy to observe that if the unknown basis $\bm{\Phi}$ is achievable from $\mathcal{L}$, by rotation invariance of the Gaussian distribution, the linear structure assumption~\ref{cond:linear-structure} is equivalent to the structured sparsity. Thus one natural question is: 

%\apcomment{Please move the following two paragraphs with your time consuming example to an appendix.}\gwcomment{Sure, moved the following discussions to the appendix.}
% \begin{center}
%     % \emph{Whether $\bm{\Phi}$ necessary to know for the projection $\Pi_{\mathcal{M}}$ in the projected power method?} 
%     \emph{Whether $\bm{\Phi}$ necessary to recover the true PC $\bm{v}_*$ for linearly structured PCA?}
% \end{center}
% The answer to the above question is No in two parts.
% In the first part, as presented later, the convergence result of the projected power method does not need to know $\bm{\Phi}$. In the second part, we further claim that, in some of the cases, computing $\bm{\Phi}$ from $\mathcal{L}$ is much more time-consuming compared with computing a projection onto the set $\mathcal{M}$.\\ 


\subsection{Examples of linearly structured PCA}

Clearly, vanilla sparse PCA is covered by our formulation. We instantiate the union of linear structures assumption with two other examples.

\subsubsection{Example 1: Tree-Sparse PCA} \label{sec:TS-PCA-intro}

Motivated by applications in signal and image processing and computer graphics \citep{baraniuk2010model}, a particular model for the underlying signal is \emph{tree sparsity} in an underlying basis. In particular, consider the following simplified model for tree-sparsity with one-dimensional signals and binary wavelet trees as a typical such instance. We require some notation to introduce it formally.

%Wavelet decompositions are widely-used in data compression, signal/image processing, pattern recognition, and computer graphics \citep{baraniuk2010model}. In addition, large coefficients of the wavelet basis in the above applications cluster along the branches of trees due to the multi-scale nature of wavelets. Such property motivates the requirement of connected tree sparsity structures for the wavelet coefficients. {\color{orange}In this example, we focuses on the following \emph{tree sparsity structure} (with one-dimensional signals and binary wavelet trees) as a typical instance to decompose a covariance matrix constructed from wavelet representations of images.} 
%As illustrated in \citet{baraniuk2010model}, this paper focuses on one-dimensional signals and binary wavelet trees. Then we study the following \emph{tree sparsity structure} as a typical instance to decompose a covariance matrix constructed from wavelet representations of images.

% Using the fact that the large coefficients form connected trees leads to performance improvements when working with wavelet representations of images. Another example of tree sparsity occurs in genomics, in which features in a supervised learning problem can be arranged in a hierarchy \citep{kim2012tree}. 

% As illustrated in \citet{baraniuk2010model}, in this paper, we also focus on one-dimensional signals and binary wavelet trees, but still the results for tree-sparsity extend directly to $d$-dimensional signals and $2^d$-ary wavelet trees. 
\iffalse
\apcomment{Need to be more specific about why this makes sense in the PCA context. Are you trying to decompose a covariance matrix constructed from images?} \gwcomment{Add some discussions on why using this tree sparsity structure.}
\fi

%\paragraph{Tree Sparsity Structure and Tree-Sparse PCA.}
Given a natural number $h$, a \textit{complete binary tree} or $\mathsf{CBT}$ of size $d = 2^h - 1$ is given by the following construction. 
%\apcomment{You are using $L$ for your subspaces. Use different notation.} \gwcomment{Replace $L$ by $h$.} a fixed integer parameter. In particular, the complete binary tree $\text{CBT}$ has
Create $h$ levels $\{1, \ldots, h\}$, with $2^{\ell - 1}$ nodes in $\ell$-th level. Index each node from $1$ to $d$, top to bottom and left to right in the following way. The root node $r_{\mathsf{CBT}}$ of $\mathsf{CBT}$ has index $1$, and for any node with index $i \in \{2, \ldots, 2^{h - 1} - 1\}$, its parent is the node with index $\lfloor \frac{i}{2} \rfloor$ and its children are the nodes with indices $2i, 2i + 1$. Define the collection of vertex sets
\[
\mathcal{T}^k := \left\{T: |T| = k, ~ \text{root node }1 \in T, ~ \text{the subgraph of $\mathsf{CBT}$ induced by $T$ is connected} \right\}.
\]
Abusing notation slightly, consider a bijection between the coordinates of any $d$-dimensional vector and the vertices of a $\mathsf{CBT}$.
The vector $\bm{v}_*$ is said to be $k$-\emph{tree-sparse} if $\mathsf{supp}(\bm{v}_*) \in \mathcal{T}^k$.

%where, for any binary subtree $T$, we use $\text{supp}(T)$ to denote the set of vertex index (i.e., support set) in $T$. \\

Therefore, tree-sparse PCA is a specific example of union of linear structures in our formulation. To see this, let $\bm{e}_i \in \mathcal{S}^{d - 1}$ denote the $i$-th standard basis vector in $\mathbb{R}^d$, and set
\begin{align*}
    & ~ \mathcal{B} := \{\bm{e}_1, \ldots, \bm{e}_d\}, \text{ and } ~ \mathcal{L} := \left\{L = \mathsf{span}(\{\bm{e}_i\}_{i \in T}) ~|~ T \in \mathcal{T}^k \right\}
\end{align*}
in Definition~\ref{cond:linear-structure}. 
% More generally, the linear structure condition allows tree-sparsity to exist in any (possibly unknown) basis.

\subsubsection{Example 2: Path-Sparse PCA} \label{sec:PS-PCA-intro}

Another commonly used variant of linearly structured PCA is path-sparse PCA~\citep{asteris2015stay}, in which the support set of $\bm{v}_*$ forms a path on an underlying directed acyclic graph $G = (V,E)$. For a vertex $v$ in this graph, let $\delta_{\text{out}}(v)$ denote the out-neighborhood of $v$.

\begin{definition}
\textbf{$(d,k)$-Layered Graph.} 
A directed acyclic graph $G = (V,E)$ is a $(d,k)$-layered graph if
\begin{itemize}
    \item
     $V = \{v_s, v_t\} \cup \widetilde{V}$ such that $|\widetilde{V}| = d-2$ and $v_s,v_t \notin \widetilde{V}$.
    \item $\widetilde{V} = \cup_{i=1}^{k} V_i$ where $V_i \cap V_j = \emptyset$ for all $i \neq j \in [k]$ and $|V_1| =  \cdots =  |V_k| = \frac{d - 2}{k}$. 
    \item $\delta_{\text{out}}(v) = V_{i + 1}$ for all $v \in V_{i}$ and $i = 1, \ldots, k - 1$, and 
        \item $\delta_{\text{out}}(v_s) = V_1$ and $\delta_{\text{out}}(v) = \{v_t\}$ for all $v \in V_k$. 
\end{itemize}
\end{definition}

%\paragraph{Path Sparsity Structure and Path-Sparse PCA.} 
Let $G = (V,E)$ be a $(d,k)$-layered graph and we define the collection of vertex sets
\begin{align}
 \mathcal{P}^k := \big\{ P \subseteq V ~|~ v_s,v_t \in P \text{ and } |P\cap V_i|=1 \text{ for any }i\in [k]\big\}. 
\end{align}
Once again, we consider the natural bijection between the coordinates of any $d$-dimensional vector and the vertices of a $(d, k)$-layered graph, and a vector $\bm{v}_*$ is said to be $k$-\emph{path-sparse} if $\mathsf{supp}(\bm{v}_*) \in \mathcal{P}^k$. It is straightforward to see that the set of all $k$-\emph{path-sparse} vectors satisfies the union of linear structures condition in Definition~\ref{cond:linear-structure} with 
\begin{align*}
    & ~ \mathcal{B} := \{\bm{e}_1, \ldots, \bm{e}_d\}, \text{ and } ~ \mathcal{L} := \left\{L = \mathsf{span}(\{\bm{e}_i\}_{i \in P}) ~|~ P \in \mathcal{P}^k \right\}.
\end{align*}
%\apcomment{Specify $\mathcal{B}$ and $\mathcal{L}$?} \gwcomment{Done.}

\subsection{Notation} \label{sec:notations}
We use $\bm{I}_{d \times d}$ to denote the $d$-by-$d$ identity matrix, and $\lambda_i(\bm{M})$ to denote the $i$-th largest eigenvalue of a symmetric matrix $\bm{M}$. 
%Let $\bm{P}_L \in \mathbb{R}^{d \times d}$ be the projection matrix onto a subspace $L$. 
We use $\bm{X} := [\bm{x}_1 ~|~ \cdots ~|~ \bm{x}_n]^{\top} \in \mathbb{R}^{n \times d}$ to denote the sample matrix where the $i$-th row of $\bm{X}$ is the $i$-th sample $\bm{x}_i$. The sample covariance matrix is given by $\widehat{\bm{\Sigma}} := \frac{1}{n} \bm{X}^{\top} \bm{X}$, and we let 
\begin{align} \label{eq:noise-def}
\bm{W} := \widehat{\bm{\Sigma}} - \bm{\Sigma}
\end{align}
denote the $d \times d$ matrix of noise. For any linear subspace $L \subseteq \mathbb{R}^d$ and its projection matrix $\bm{P}_L \in \mathbb{R}^{d \times d}$, we use $\widehat{\bm{\Sigma}}_L := \bm{P}_L^{\top} \widehat{\bm{\Sigma}} \bm{P}_L$ to denote the sample covariance matrix restricted to the subspace $L$. We also use the analogous notation $\bm{\Sigma}_L := \bm{P}_L^{\top} \bm{\Sigma} \bm{P}_L$ and $\bm{W}_L := \bm{P}_L^{\top} \bm{W} \bm{P}_L$. 
%\apcomment{We should define $\mathcal{M}$ in a more prominent way, without needing an optimization 
We index the subspaces $L_1, \ldots, L_M$ in some consistent lexicographic order.
We reserve the notation $\mathcal{M} := \bigcup_{m = 1}^M L_m$ to denote the set containing $\bm{v}_*$, and the notation $L_* \in \{L_{1}, \dots, L_{M}\}$ to denote the specific linear subspace that contains $\bm{v}_*$, with ties broken lexicographically.
% As mentioned in optimization problem~\eqref{eq:linearly-structured-PCA}, we use $\mathcal{M} := \bigcup_{m = 1}^M L_m$ as the union of all possible candidate linear subspaces. 
We let $\mathcal{S}^{d - 1}:= \{\bm{v} \in \mathbb{R}^{d}: \|\bm{v}\|_2 = 1\}$ denote the unit $\ell_2$-sphere in $d$-dimensional Euclidean space. For any subspace $L$, let  
\[
\widehat{\bm{v}}_L := \argmax_{\bm{v} \in \mathcal{S}^{d-1}} \bm{v}^{\top} \widehat{\bm{\Sigma}}_L \bm{v} = \argmax_{\bm{v} \in \mathcal{S}^{d-1} \cap L} \bm{v}^{\top} \widehat{\bm{\Sigma}} \bm{v}.
\]
be the leading eigenvector of the restricted sample covariance $\widehat{\bm{\Sigma}}_L$. For an arbitrary symmetric matrix $\bm{M} \in \mathbb{R}^{d \times d}$ and set $S \subseteq \mathbb{R}^d$, define the scalar 
\begin{align}\label{eqs:rho}
    \rho(\bm{M}, S) := \max_{\|\bm{v}\|_2 = 1,\bm{v} \in S} \big| \bm{v}^{\top} \bm{M} \bm{v} \big|.
\end{align}
For two sequences of non-negative reals $\{f_n\}_{n \geq 1}$ and $\{g_n\}_{n \geq 1}$, we use $f_n \gtrsim g_n$ to indicate that there is a universal positive constant $C$ such that $f_n \leq C g_n$ for all $n \geq 1$. We also use standard order notation $f_n = O(g_n)$ to indicate that $f_n \lesssim g_n$ and $f_n = \tilde{O}(g_n)$ to indicate that $f_n \lesssim g_n \ln^c n$ for some universal constant $c$. We say that $f_n = \Omega(g_n)$ (resp. $f_n = \tilde{\Omega}(g_n)$) if
$g_n = \Omega(f_n)$ (resp. $g_n = \Omega(f_n)$). We use $f_n = \Theta(g_n)$ (resp. $f_n = \tilde{\Theta}(g_n)$) if $f_n = O(g_n)$ and $f_n = \Omega(g_n)$ (resp. $f_n = \tilde{O}(g_n)$ and $f_n = \tilde{\Omega}(g_n)$). We say that $f_n = o(g_n)$ (resp. $f_n = \tilde{o}(g_n)$) when $\lim_{n \rightarrow \infty} f_n / g_n = 0$ (resp. $\lim_{n \rightarrow \infty} f_n / (g_n \ln^c n) = 0$ for some universal constant $c$). We also use $f_n = \omega(g_n)$ to indicate that $\lim_{n \rightarrow \infty} f_n / g_n = \infty$. Throughout, we use $c, c_1, c_2, \ldots$ and $C, C_1, C_2, \ldots$ to denote universal positive constants, and their values may change from line to line. 

% We say that $f_n = \Omega(g_n)$ (resp. $f_n = \tilde{\Omega}(g_n)$) if $f_n = O(g_n)$ (resp. $f_n = \tilde{O}(g_n)$).


% \paragraph{In Section~\ref{sec:lower-bound},}let $\mathcal{B}(\mathcal{L}) := \{\mathcal{B}_1, \ldots, \mathcal{B}_{M}\}$ be the collection of sub-basis with respect to each linear subspace in $\mathcal{L}$. For each $i \in [d]$, let $\mathcal{B}(\phi_i) := \{\mathcal{B}_m \in \mathcal{B}(\mathcal{L}) ~|~ \phi_i \in \mathcal{B}_m \}$ be the collection of sub-basis which contains $\phi_i$. We use $|\mathcal{B}(\phi_i)|$ to denote the number of sub-basis that contains $\phi_i$ in $\mathcal{B}(\mathcal{L})$. Define $i_* := \argmax_{i \in [d]} |\mathcal{B}(\phi_i)|$ be the index with respect to the maximum size/cardinality among $\mathcal{B}(\phi_1), \ldots, \mathcal{B}(\phi_d)$. Define the set of characteristic vectors $\mathcal{Z}$ as
% \begin{align}
%     \mathcal{Z} := \left\{ \bm{z} \in \{0,1\}^d ~|~ \mathrm{supp}(\bm{z}) \in \mathrm{supp}(\mathcal{B}(\phi_{i_*})) \right\}, \notag
% \end{align}
% where $\mathrm{supp}(\mathcal{B}(\phi_{i_*})) := \left\{ \mathrm{supp}(\mathcal{B}_m) ~|~ \mathcal{B}_m \in \mathcal{B}(\phi_{i_*}) \right\}$ and $\mathrm{supp}(\mathcal{B}_m)$ denotes the support of $\mathcal{B}_m$, i.e., the indices of $\{\phi_1, \ldots, \phi_d\}$ that contained in $\mathcal{B}_m$. Given $\bm{z}, \bm{z}' \in \{0, 1\}^d$, let $\delta_H(\bm{z}, \bm{z}') := |\{i: \bm{z}_i \neq \bm{z}_i' \}|$ be the Hamming distance between $\bm{z}$ and $\bm{z}'$. Under Assumption~A\ref{assump:minimax-assumption}.1 listed below, for any $\xi \in (0,1)$, let $\mathcal{Z}_{\xi} \subseteq \mathcal{Z}$ be a subset of $\mathcal{Z}$ such that $\delta_H(\bm{z}, \bm{z}') > 2 (1 - \xi) k$ for all $\bm{z} \neq \bm{z}' \in \mathcal{Z}_{\xi}$. For any integer $r \geq 0$, let 
% \begin{align}
%     \mathcal{N}_H(\bm{z}; r) := \left\{ \bm{z}' \in \mathcal{Z} ~|~ \delta_H(\bm{z}, \bm{z}') \leq r \right\} \notag
% \end{align}
% be the neighborhood of $\bm{z}$ in $\mathcal{Z}$ with Hamming ball distance $\delta_H$ at most $r$. 
 
% \paragraph{In Section~\ref{sec:PPM-EM},}\label{para:notation-4} given $T$ the total number of iterations of Algorithm~\ref{alg:PPM}, we use $\bm{v}_{t + 1}$ to denote the solution obtained in $t$-th iteration. For $t = 0, 1, \ldots, T - 1$, define $L^{(t)} \in \mathcal{L}$ the linear subspaces that contains $\bm{v}_t$, and define the linear subspace $F^{(t)} := \text{conv}(L^{(t)} \cup L^{(t + 1)} \cup L_*)$ as the convex hull of $L^{(t)}, L^{(t + 1)}, L_*$. Decomposing the vector $\bm{v}_t$ over $\widehat{\bm{v}}_{F^{(t)}}$ and its orthonormal direction $\widehat{\bm{v}}_{\bot}$, we have $\bm{v}_t = \alpha_t \widehat{\bm{v}}_{F^{(t)}} + \beta_t \widehat{\bm{v}}_{\bot}$, where $\alpha_t := \langle \bm{v}_t, \widehat{\bm{v}}_{F^{(t)}}\rangle, \beta_t := \langle \bm{v}_t, \widehat{\bm{v}}_{\bot}\rangle, \alpha_t^2 + \beta_t^2 = 1, \|\widehat{\bm{v}}_{F^{(t)}}\|_2 = \|\widehat{\bm{v}}_{\bot}\|_2 = 1,$ and $\langle \widehat{\bm{v}}_{F^{(t)}}, ~ \widehat{\bm{v}}_{\bot} \rangle = 0$. We use $\lambda_i$ as the $i$-th eigenvalue of the covariance matrix $\bm{\Sigma}$, $\widehat{\lambda}_i := \lambda_i(\widehat{\bm{\Sigma}})$ as the $i$-th eigenvalue of the sample covariance matrix $\widehat{\bm{\Sigma}}$, and $\widehat{\lambda}_{i}^{F^{(t)}} := \lambda_{i}(\widehat{\bm{\Sigma}}_{F^{(t)}})$ as a shorthand for the $i$-th eigenvalue of $\widehat{\bm{\Sigma}}_{F^{(t)}}$. We denote $\kappa := \lambda_2 / \lambda_1 = 1 / (1 + \lambda)$ as the signal-to-noise ratio of the true covariance matrix $\bm{\Sigma}$ and $\widehat{\kappa}_{F^{(t)}} := \widehat{\lambda}_2^{F^{(t)}} / \widehat{\lambda}_{\max}^{F^{(t)}}$ as the signal-to-noise ratio of the restricted (in linear subspace $F^{(t)}$) sample covariance matrix $\widehat{\bm{\Sigma}}_{F^{(t)}}$. Moreover, define 
% \begin{align}
%     F^* := \argmax_F \rho(\bm{W}, {F}) ~~\text{s.t.}~~ F = \text{conv}(L_{m_1} \cup L_{m_2} \cup L_{m_3}), ~ \forall ~ m_1 \neq m_2 \neq m_3 \in [M], \notag
% \end{align}  
% be the union of three linear subspaces in $\mathcal{L}$ with respect to the maximum $\rho(\bm{W}, {F})$.

% \paragraph{In Section~\ref{sec:specific-examples},} we use $\mathcal{T}^k, \mathcal{P}^k$ to denote the structure set (i.e., feasible set $\mathcal{M}$ of \eqref{eq:linearly-structured-PCA}) of tree/path sparse PCA, repsectively. 
 





 

%!TEX root = main-paper-template.tex

\section{General results} \label{sec:general-results}

In this section, we present our general results for union of linearly structured PCA, covering both fundamental limits of estimation and local convergence properties of a projected power method. Recall the notation $\rho(\bm{M}, S)$ from Eq.~\eqref{eqs:rho} for any symmetric matrix $\bm{M} \in \mathbb{R}^{d \times d}$ and set $S \subseteq \mathbb{R}^d$. We let
\begin{align}\label{exhaustive-search-estimator} 
    \ESest := \argmax_{\bm{v} \in \mathcal{M} \cap \mathcal{S}^{d-1}} \bm{v}^{\top} \widehat{\bm{\Sigma}} \bm{v}
\end{align} 
denote the general exhaustive search estimator.
%\apcomment{Put notation for $\rho$ here instead of above. Otherwise recall explicitly.}

\subsection{Fundamental limits of estimation} \label{sec:fund-limits}

\iffalse
\apcomment{Some pointers:
\begin{itemize}
    \item Is there a reason to have Prop 1 and Prop 2 stated in the main text? This is something you need only for the proof right?
    \item I would state the setup for the minimax lower bound up front (including Assumption 1) and then state a {\bf single theorem} for fundamental limits having two parts. Part (a) is the upper bound that does not need further assumptions. Part (b) is the lower bound, which holds under Assumption 1.
    \item After this, you can discuss both parts of the theorem and compare with existing literature/parallels with sparse PCA.
    \item If we state things in this fashion, we don't need two subsubsections here for upper and lower bounds.
\end{itemize}
}
\fi
We begin by studying the fundamental limits of estimation for linearly structured PCA, without computational considerations. These serve as baselines for the results to follow. We first introduce some notation before presenting main results. Recall $\mathcal{L} = \{L_1, \ldots, L_M\}$, the collection of $M$ linear subspaces, and subsets of bases $\mathcal{B}_m \subseteq \mathcal{B} = \{\phi_1, \ldots, \phi_d\}$ such that $L_m = \text{span}(\mathcal{B}_m)$. For each $m \in [M]$, define the characteristic vector $\bm{z}_{m} \in \{0,1\}^d$ of each subset $\mathcal{B}_m$ as follows
\begin{align}\label{definition-z-m}
    \bm{z}_{m}(i) := \left\{
    \begin{array}{lll}
        1 & \text{if } \quad \phi_i \in \mathcal{B}_m \\
        0 & \text{if } \quad \phi_i \notin \mathcal{B}_m
    \end{array}
    \right., \quad \text{for all} \; i \in [d],
\end{align}
where $\bm{z}_{m}(i)$ is the $i$-th entry of $\bm{z}_{m}$. We further define 
\begin{align}\label{definition-i-star}
    i_* := \argmax_{i \in [d]} \sum_{m=1}^{M}\bm{z}_{m}(i) 
\end{align}
as the index with the most ones among $\{\bm{z}_m\}_{m = 1}^M$, breaking ties lexicographically. In words, this is the index of the basis vector that appears in the most subspaces. Now let 
\begin{align}\label{definition-Z-star}
    \mathcal{Z}_{*} := \left\{ \bm{z}_{m} \in \{\bm{z}_1, \ldots, \bm{z}_M\} ~ | ~ \bm{z}_{m}(i_*) = 1 \right\}.
\end{align}
be the set of characteristic vectors with $\bm{z}_{m}(i_*) = 1$. For any fixed integer $r \geq 0$ and characteristic vector $\bm{z} \in \{\bm{z}_{m}\}_{m=1}^{M}$, we use
\begin{align*}
    \mathcal{N}_H(\bm{z}; r) := \left\{ \bm{z}' \in \mathcal{Z}_{*} \;|\; \delta_H(\bm{z}, \bm{z}') \leq r \right\} 
\end{align*} 
to denote the neighborhood of $\bm{z}$ in $\mathcal{Z}_{*}$ with Hamming ball distance $\delta_H(\bm{z}, \bm{z}') := |\{i: \bm{z}(i) \neq \bm{z}'(i) \}|$ at most $r$. We further state Assumption~\ref{assump:minimax-assumption} for the minimax lower bound.


\begin{assumption} \label{assump:minimax-assumption}
%\textbf{Additional Assumptions for the Linear Structure Condition~\ref{cond:linear-structure}.} Assume the following conditions hold.
This assumption has two parts:
\begin{description}
    \item[(a)] For some $k \leq d$, each linear subspace $L_m = \text{span}(\mathcal{B}_m) \in \mathcal{L}$ satisfies $|\mathcal{B}_m| = k$. 
    \item[(b)] There exists $\xi \in [3/4,1)$ such that
    \begin{align} 
        \frac{|\mathcal{Z}_{*}|}{\max_{\bm{z} \in \mathcal{Z}_{*}}|\mathcal{N}_H(\bm{z}; 2(1-\xi)k)|} \geq 16. \label{eq:minimax-assump}
    \end{align}
\end{description}
\end{assumption}

Assumption~\ref{assump:minimax-assumption}(a) is clearly satisfied by vanilla sparse PCA, tree-sparse PCA, and path-sparse PCA. For a general $\mathcal{L}$, one can always set $k = \max_{m \in [M]} |\mathcal{B}_m|$. Assumption~\ref{assump:minimax-assumption}(b), on the other hand, controls the ratio of the sizes between the largest neighborhood $\mathcal{N}_H(\bm{z};2(1-\xi) k)$ (among $\bm{z} \in \mathcal{Z}_*$) and $\mathcal{Z}_*$. Geometric intuition for this assumption will be provided shortly.
%Without such control (Inequality~\eqref{eq:minimax-assump}), there are at least $|\mathcal{Z}_*|/ 16$ linear subspaces overlapping on more than $2(1-\xi)k$ bases. As a result, the size of the packing set for $\mathcal{M}$ approaches the size of the packing set for these heavily overlapped bases, resulting in a decrease in the order of the minimax lower bounds. 
It is worth noting that the specific constant $16$ in Ineq.~\eqref{eq:minimax-assump} is arbitrary, and any constant greater than $2$ can be used. We choose $16$ for simplicity and convenience in presenting the subsequent theoretical results (Theorem~\ref{thm:fund-limits} part (b)). 
%Without such control (Inequality~\eqref{eq:minimax-assump}), there are at least $|\mathcal{Z}_*|/ 16$ linear subspaces overlapping on more than $2(1-\xi)k$ bases. As a result, the size of the packing set for $\mathcal{M}$ approaches the size of the packing set for these heavily overlapped bases, resulting in a decrease in the order of the minimax lower bounds. Moreover, we would like to point out that the right hand side term in Inequality~\eqref{eq:minimax-assump} can be changed to arbitrary constant greater than $2$. We pick $16$ in Inequality~\eqref{eq:minimax-assump} only for constant simplicity that will be presented later in part (b) of Theorem~\ref{thm:fund-limits}.


%Without Inequality~\eqref{eq:minimax-assump}, if the size of $|\mathcal{N}_H(\bm{z}; 2(1-\xi)k)|$ becomes excessively large, at least $|\mathcal{Z}_*|/ 16$ linear subspaces may overlap on more than $2(1-\xi)k$ bases. 
%\begin{itemize}
%	\item {\color{orange}Note that the right hand side term can be changed to arbitrary constant greater than $2$.  We pick $16$ here only for constant simplicity that will be presented later in part (b) of Theorem~\ref{thm:fund-limits}.}
%%	The right hand side term $16$ only influences the constant term of minimax lower bound provided in part (b) of Theorem~\ref{thm:fund-limits}. Indeed, such right hand side term can be taken any constant greater than $2$ to achieve similar minimax bounds aforementioned. We pick $16$ here only for constant simplicity presented in part (b) of Theorem~\ref{thm:fund-limits}.
%	\item  Without such control, if the size of $|\mathcal{N}_H(\bm{z}; 2(1-\xi)k)|$ becomes excessively large, at least $|\mathcal{Z}_*|/ 16$ linear subspaces may overlap on more than $2(1-\xi)k$ bases. As a result, the size of the packing set for $\mathcal{M}$ approaches the size of the packing set for these heavily overlapped bases, resulting in a decrease in the order of the minimax lower bounds.
%\end{itemize}

%\apcomment{I don't understand the following sentence.} \gwcomment{Trying to claim that these linear subspaces should not be clustered too close to each other.}

%Assumption~\text{A\ref{assump:minimax-assumption}.2} places a condition on the local structure around every linear subspace in $\mathcal{L}$; in particular, the size of the neighborhood $\mathcal{N}_H(\bm{z}; k/2)$.  If the size of neighborhoods $\mathcal{N}_H(\bm{z}; k/2)$ are too large, there exists inear subspaces that intersect most of the remaining linear subspaces on at least $k/2$ bases.
% {\color{orange}Thus, such intersections can be ``roughly'' viewed as ``clusters'' of linear subspaces in $\mathcal{L}$. Due to these clusters, intuitively, the size of the packing set for the the union of whole subspaces $\mathcal{M}$ is close to the size of the packing set for these clusters, which reduce the order of the resulting minimax lower bounds.}
% \apcomment{I don't understand the following sentence.} \gwcomment{Trying to claim that these linear subspaces should not be clustered too close to each other.} 
% {\color{orange}As a result, in the extreme case, the packing set of the above linear subspace becomes ``almost'' the packing set of $\mathcal{M}$.} \\
% As a result, it is unlikely to construct a packing set $\mathcal{V}_{\epsilon}$ presented later in Proposition~\ref{prop:g-fano-method} with desire cardinality to show lower bound. 

We are now poised to state the main result of this subsection. Recall that $L_*$ denotes the subspace containing the vector $\bm{v}_*$. Let $\widehat{L} \in \mathcal{L}$ be the linear subspace such that $\ESest \in \widehat{L}$ (once again breaking ties lexicographically) and let $\widehat{F} := \mathsf{conv}(\widehat{L} \cup L_*)$. 

\begin{theorem}\label{thm:fund-limits}
Suppose the linear structure condition in Definition~\ref{cond:linear-structure} holds.

\noindent (a) Let $\ESest$ be defined in equation~\eqref{exhaustive-search-estimator}. Without loss of generality, suppose $\langle \bm{v}_*, \widehat{\bm{v}}_{\mathsf{ES}} \rangle \geq 0$.  Then for all $\bm{v}_* \in \mathcal{S}^{d - 1} \cap \mathcal{M}$, we have
\begin{subequations}
    \begin{align} \label{eq:fund-limit-a}
        \|\widehat{\bm{v}}_{\mathsf{ES}} - \bm{v}_*\|_2 \leq \frac{2\sqrt{2}}{\lambda} \rho(\bm{W}, \widehat{F}).
    \end{align}

\noindent (b) Let $\xi \in [3/4,1)$ such that Assumption~\ref{assump:minimax-assumption} holds. 
%let $\mathcal{D}^{(n)}(\lambda; \bm{v}_*) := \mathcal{D}(\lambda; \bm{v}_*)^{\otimes n}$ be the product measure over the $n$ independent draws of samples from $\mathcal{D}(\lambda; \bm{v}_*)$. With the additional Assumption~\ref{assump:minimax-assumption}, for 
We have the minimax lower bound
%here exists $\bm{v}_* \in \mathcal{S}^{d - 1} \cap \mathcal{M}$ such that for any estimator $\widehat{\bm{v}}$, 
    \begin{align}
        &\inf_{\widehat{\bm{v}}} \; \sup_{\bm{v}_* \in \mathcal{S}^{d - 1} \cap \mathcal{M}} \mathbb{E} \left[ \left\| \widehat{\bm{v}} \widehat{\bm{v}}^{\top} - \bm{v}_* \bm{v}_*^{\top}  \right\|_F \right]  \notag\\
        & \qquad \geq \frac{\sqrt{2(1 - \xi)}}{4} \min\Bigg\{1, ~~ \sqrt{\frac{1 + \lambda}{8 \lambda^2}} \sqrt{\frac{ \log \left(|\mathcal{Z}_*|\right) - \log \big( \max_{\bm{z} \in \mathcal{Z}_{*}}|\mathcal{N}_H(\bm{z}; 2(1 - \xi) k)| \big)}{n}} \Bigg\}. \label{eq:fund-limit-b}
    \end{align}
    \end{subequations}
    Here, the infimum is taken over all measurable functions of the observations $\{ \bm{x}_i \}_{i = 1}^n$, which are drawn i.i.d. from the distribution $\mathcal{D}(\lambda; \bm{v}_*)$.
\end{theorem}

Theorem~\ref{thm:fund-limits}(a) provides a deterministic upper bound on the $\ell_2$ error between the estimate $\widehat{\bm{v}}_{\mathsf{ES}}$ and the ground truth $\bm{v}_*$, showing that this error can be bounded on the order $\rho(\bm{W}, \widehat{F})$ for any fixed $\lambda$. 
%The statement of the theorem should be compared with that of~\citet{yuan2013truncated} for vanilla sparse PCA---our contribution is to modify the proof technique to show that a projected power method for union-of-linear attains a similar guarantee. 
We provide the proof of this result in Section~\ref{app:fund-limits-up}. While the result is deterministic, we will see that Eq.~\eqref{eq:fund-limit-a} nearly matches the minimax lower bound Eq.~\eqref{eq:fund-limit-b} for our special cases of interest. Consequently, we use Theorem~\ref{thm:fund-limits}(a) as a heuristic baseline to assess the performance of efficient algorithms. 
 
%We will shortly use this result as a baseline to assess the performance of efficient algorithms.


%The proof of Part (b) of Theorem~\ref{thm:fund-limits} is presented in the Appendix~\ref{app:fund-limits-lb}. Comparing with existing minimax lower of classical sparse PCA and path-sparse PCA, 

On its own, Theorem~\ref{thm:fund-limits}(b) provides a minimax lower bound that depends on the local structure of $\mathcal{M}$ around any choice of ground truth $\bm{v}_*$. 
%-- the number of linear subspaces that shares at least $2(1 - \xi)k$ same bases -- of $\mathcal{L}$. 
The proof uses the generalized Fano inequality \citep{verdu1994generalizing}, and we construct a rich packing set $\mathcal{V}_{\epsilon}$ in $\mathcal{S}^{d - 1} \cap \mathcal{M}$ (i.e., $\mathcal{V}$ in Proposition~\ref{prop:g-fano-method} of Appendix~\ref{app:fund-limits-lb}) such that the points in $\mathcal{V}_{\epsilon}$ are $\mathcal{O}(\epsilon)$ separated in some appropriate distance measure. In contrast to existing proofs for sparse PCA \citep{vu2012minimax} and path PCA \citep{asteris2015stay}, the set $\mathcal{V}_{\epsilon}$ here is constructed so that there exists a common support index (i.e., the index $i_*$, defined in Eq~\eqref{definition-i-star}) for every point $\bm{v} \in \mathcal{V}_{\epsilon}$ that one can use to construct the packing. 
%(in the order of $O(\sqrt{1 - \epsilon^2})$) to ensure the condition~\eqref{cond:g-fano-method} in the generalized Fano method.  \apcomment{Wasn't there a structured subspace paper by Cai and Ma (with both upper and lower bounds) that we should compare to?} \gwcomment{Done, add as follows.} 
%
%\gwcomment{Very important, compare difference between (10b) and Cai et al. (2021)'s minimax result. Also compare with Tong Zhang upper bound for sparse PCA.}
%{\color{orange}
On a related note, a paper by \citet{cai2021optimal} studies the minimax risk of a general structured principal subspace estimation problem, including vanilla sparse PCA as a special case. These bounds are phrased in terms of critical inequalities that arise from local packing numbers (see~\cite{yang1999information,wainwright2019high}). Our lower bound instead takes a more global approach, which we show suffices for union-of-linear structure.
%In particular, the minimax lower bound presented in \cite{cai2021optimal} is characterized by the entropy measure of the size of some local packing set. 
%\gwcomment{(1) upper bound (2) Cai gives a minimax bound with local packing set, we can do the similar thing. But here, we consider in a more global perspective, ...}
In particular, the minimax lower bound~\eqref{eq:fund-limit-b} is controlled by the relative ratio between $|\mathcal{N}_H(\bm{z};2(1-\xi) k)|$ and $|\mathcal{Z}_*|$: Our assumption in Ineq.~\eqref{eq:minimax-assump} avoids the scenario that many linear subspaces heavily overlap on a few bases.  
%A recent paper by \cite{cai2021optimal} studies minimax lower bounds for structured principal subspace estimation. The key distinction in our case (under Definition~\ref{cond:linear-structure}) is that $\mathcal{N}_H(\bm{z}; r)$ emerges as a natural way to quantify how heavily clustered the linear subspaces are (see Ineq.~\eqref{eq:minimax-assump}). Without the Ineq.~\eqref{eq:minimax-assump}, there are at least $|\mathcal{Z}_*|/ 16$ linear subspaces overlapping on more than $2(1-\xi)k$ bases. As a result, the size of the packing set for $\mathcal{M}$ approaches the size of the packing set for these heavily overlapped bases, resulting in a decrease in the order of the minimax lower bounds. 
%}



Finally, it is instructive to note that Theorem~\ref{thm:fund-limits}(b) recovers the existing minimax lower bound for vanilla sparse PCA [Theorem 2.1, \cite{vu2012minimax}]. 
% the minimax lower bound in Theorem~\ref{thm:fund-limits}(b) with the existing minimax lower bound for sparse PCA [Theorem 2.1, \cite{vu2012minimax}]. 
 Indeed, supposing that $d \gg k$ and applying Theorem~\ref{thm:fund-limits}(b) for sparse PCA, we obtain the known lower bound
\begin{align} \label{eq:SPCA-known}
    \inf_{\widehat{\bm{v}}} \; \sup_{ \bm{v_*} \in \mathcal{S}^{d - 1}: \| \bm{v_*} \|_0 \leq k} \mathbb{E} \;\left[ \left\| \widehat{\bm{v}} \widehat{\bm{v}}^{\top} - \bm{v}_* \bm{v}_*^{\top}  \right\|_F \right] \gtrsim  \min\bigg\{1,  \sqrt{\frac{1 + \lambda}{8 \lambda^2}} \sqrt{\frac{k \log d}{n}} \bigg\}. 
\end{align}
The proof of inequality~\eqref{eq:SPCA-known} (from Theorem~\ref{thm:fund-limits}(b)) is provided in Section~\ref{min-max-pca-proof} for completeness. In Section~\ref{sec:specific-examples}, we provide novel corollaries for tree-sparse PCA and path-sparse PCA.

%\apcomment{No need to state the following as formal corollary. Just state in text.} \gwcomment{Done.}
%%%%%%%%%%%%%%%%%%%%%%%%%%%%%%%%%%%%%%%%%%%%%%%%%%%%%%%%%%%%%
%%%%%%%%%%%%%%%%%%%%%%%%%%%%%%%%%%%%%%%%%%%%%%%%%%%%%%%%%%%%%
%%%%%%%%%%%%%%%%%%%%%%%%%%%%%%%%%%%%%%%%%%%%%%%%%%%%%%%%%%%%%





%!TEX root = IEEETSP-main-paper.tex

\subsection{A locally convergent projected power method} \label{sec:PPM-EM} 

In Section~\ref{sec:fund-limits}, we studied the fundamental limits of the problem, where our upper bounds were achieved by the exhaustive search estimator $\ESest$.
%\apcomment{Please give this a name. Possibly $\widehat{\bm{v}}_{\mathsf{ES}}$ for exhaustive search. Also use a macro for it so it is easy to type out.} \gwcomment{Done.} for exhaustive search.
Given the computational challenge of searching over every linear subspace $L \in \mathcal{L}$, we propose the following iterative projected power method (Algorithm~\ref{alg:PPM}) and show that with access to a suitable exact projection oracle, it locally converges to a statistical neighborhood of the ground truth.


\begin{definition}[Exact projection] \label{assump:exact-proj}
For all $\bm{v} \in \mathbb{R}^d$, let 
\[ \Pi_{\mathcal{M}} (\bm{v}) := \argmin_{\bm{v}' \in \mathcal{M}}\; \|\bm{v}' - \bm{v}\|_2 = \argmin_{\bm{v}' \in L_{m},m \in [M]} \; \| \bm{v}' - \bm{v}\|_2,
\]
where ties between subspaces are broken lexicographically.
\end{definition}
Owing to the tie-breaking rule, this projection is always unique. As we will see in Section~\ref{sec:specific-examples}, an exact projection oracle $\Pi_{\mathcal{M}}$ can be constructed efficiently (in time nearly logarithmic in $M$) in some specific examples of union-of-linearly structured PCA. 
% \apcomment{Breaking ties lexicographically removes the need for any of the older additional remarks.}
%
\iffalse
\begin{remark}
For any $\bm{v} \in \mathbb{R}^d$, let $L_{\Pi} \in \mathcal{L}$ be the linear subspace that contains $\Pi_{\mathcal{M}}(\bm{v})$. When the linear subspace $L_{\Pi} \in \mathcal{L}$ is known, the exact non-convex projection $\Pi_{\mathcal{M}}$ is equivalent to the classical Euclidean projection onto $L_{\Pi}$, i.e., $\Pi_{\mathcal{M}} (\bm{v}) = \argmin_{\bm{\theta} \in L_{\Pi} } \|\bm{\theta} - \bm{v}\|_2.$ 
\end{remark} 
\begin{remark}
\textbf{Uniqueness of the Projection $\Pi_{\mathcal{M}}$.}
To ensure the uniqueness of $\Pi_{\mathcal{M}}$, when there are two linear subspace $L_{m_1} \neq L_{m_2} \in \mathcal{L}$ with index $m_1 < m_2$ that satisfy
\begin{align}
    \min_{\bm{u}_{m_1} \in L_{m_1}} \|\bm{u}_{m_1} - \bm{v}\|_2 = \min_{\bm{u}_{m_2} \in L_{m_2}} \|\bm{u}_{m_2} - \bm{v}\|_2 \leq \argmin_{\bm{u}_{m} \in L_{m}} \|\bm{u}_{m} - \bm{v}\|_2  ~~ \forall m \in [M], \notag 
\end{align}
the exact projection $\Pi_{\mathcal{M}}$ projects $\bm{v}$ onto the linear subspace with smaller index $m$, i.e., in this case, $\Pi_{\mathcal{M}}$ projects $\bm{v}$ onto the linear subspace $L_{m_1}$. 
\end{remark}
\fi
%
We are now in a position to present the projected power method, described formally in Algorithm~\ref{alg:PPM}.


\begin{algorithm}
\caption{Projected Power Method}
\label{alg:PPM}
\textbf{Input:} Sample covariance matrix $\widehat{\bm{\Sigma}}$.
\begin{algorithmic}[1]
\State \textbf{Initialize} with a vector $\bm{v}_0 \in \mathcal{M} \cap \mathcal{S}^{d - 1}$. 
\For{$t = 0, 1, \ldots, T - 1$}
\State Compute $\tilde{\bm{v}}_{t + 1} = \widehat{\bm{\Sigma}} \bm{v}_t / \big\|\widehat{\bm{\Sigma}} \bm{v}_t \big\|_2$. \label{alg:PPM-compute}
\State Project $\bm{v}_{t + 1}^{\mathcal{M}} = \Pi_{\mathcal{M} }(\tilde{\bm{v}}_{t + 1})$. 
% \apcomment{Split this into two steps, one of which explicitly uses the projection onto $\mathcal{M}$. Define additional vector if necessary.} \gwcomment{Got it.} 
\label{alg:PPM-project}
\State Normalize to unit sphere $\bm{v}_{t + 1} = \frac{ \bm{v}_{t + 1}^{\mathcal{M}} }{ \|\bm{v}_{t + 1}^{\mathcal{M}} \|_2} \in \mathcal{M} \cap \mathcal{S}^{d - 1}$. 
\EndFor
\end{algorithmic}
\textbf{Output:} $\bm{v}_T$.  
\end{algorithm} 

\iffalse
\begin{remark}
Notice that, the projection used in step~\eqref{alg:PPM-project} of Algorithm~\ref{alg:PPM} is $\Pi_{\mathcal{M} \cap \mathcal{S}^{d - 1}}: \mathcal{S}^{d - 1} \mapsto \mathcal{M} \cap \mathcal{S}^{d - 1}$, which exactly projects any unit vector onto the intersection of the non-convex set $\mathcal{M}$ and unit $\ell_2$-ball, i.e., for any $\bm{v} \in \mathcal{S}^{d - 1}$, 
\begin{align}
    \Pi_{\mathcal{M} \cap \mathcal{S}^{d - 1}} (\bm{v}) := \argmin_{\bm{\theta} \in \mathcal{M} \cap \mathcal{S}^{d - 1}} \|\bm{\theta} - \bm{v}\|_2. \notag
\end{align}
In Appendix~\ref{app:prop-decom-covar}, we show that such projection $\Pi_{\mathcal{M} \cap \mathcal{S}^{d - 1}}$ used in the projected power method (Algorithm~\ref{alg:PPM}) can be constructed using the given exact projection $\Pi_{\mathcal{M}}$. 
\end{remark}

\begin{remark}
\textbf{Uniqueness of the Projection $\Pi_{\mathcal{M} \cap \mathcal{S}^{d - 1}}$.} The uniqueness of $\Pi_{\mathcal{M}}$ ensures the uniqueness of $\Pi_{\mathcal{M} \cap \mathcal{S}^{d - 1}}$. 
\end{remark}
\fi

\iffalse
%We then introduce notations for Algorithm~\ref{alg:PPM} and Theorem~\ref{thm:convergence}. \apcomment{You haven't yet defined the algorithm. Weird to give notation for it.} \gwcomment{Have moved to current position.}  Given $T$ the total number of iterations of Algorithm~\ref{alg:PPM}, we use $\bm{v}_{t + 1}$ to denote the solution obtained in $t$-th iteration. For $t = 0, 1, \ldots, T - 1$, define
Let $L^{(t)} \in \mathcal{L}$ denote the linear subspace that contains $\bm{v}_t$ (with ties broken lexicographically), and define the linear subspace $F^{(t)} := \text{conv}(L^{(t)} \cup L^{(t + 1)} \cup L_*)$ as the convex hull of $L^{(t)}, L^{(t + 1)}$ and $L_*$.
%, where $L_* \in \mathcal{L}$ is the linear subspace that contains the ground truth $\bm{v}_*$. 
Decomposing the vector $\bm{v}_t$ over the subspace $F^{(t)}$ and its orthonormal complement, as %$\widehat{\bm{v}}_{F^{(t)}}$ and its orthonormal direction $\widehat{\bm{v}}_{\bot}$, we have 
$\bm{v}_t = \alpha_t \widehat{\bm{v}}_{F^{(t)}} + \beta_t \widehat{\bm{v}}_{\bot}$, where $\widehat{\bm{v}}_{F^{(t)}}$ and $\widehat{\bm{v}}_{\bot}$ are unit norm vectors and 
\[
\alpha_t := \langle \bm{v}_t, \widehat{\bm{v}}_{F^{(t)}}\rangle, \quad \beta_t := \langle \bm{v}_t, \widehat{\bm{v}}_{\bot}\rangle, , \text{ with } \alpha_t^2 + \beta_t^2 = 1.
\]
We use the shorthand $\lambda_i:= \lambda_i(\bm{\Sigma})$, $\widehat{\lambda}_i := \lambda_i(\widehat{\bm{\Sigma}})$, and $\widehat{\lambda}_{i}^{F^{(t)}} := \lambda_{i}(\widehat{\bm{\Sigma}}_{F^{(t)}})$. Let $\kappa := \lambda_2 / \lambda_1 = 1 / (1 + \lambda)$ denote the signal-to-noise ratio of the true covariance matrix $\bm{\Sigma}$ and $\widehat{\kappa}_{F^{(t)}} := \widehat{\lambda}_2^{F^{(t)}} / \widehat{\lambda}_{\max}^{F^{(t)}}$ denote the analogous quantity for the sample covariance matrix restricted to the linear subspace $F^{(t)}$. Using the notation $\rho(\bm{M}, S)$ and $\rho(\bm{M}, \mathcal{M})$ from Eq.~\eqref{eqs:rho}, define
\begin{align*}
    F^* := \argmax_F \rho(\bm{W}, {F}) ~~\text{s.t.}~~ F = \text{conv}(L_{m_1} \cup L_{m_2} \cup L_{m_3}), ~ \forall ~ m_1 \neq m_2 \neq m_3 \in [M].
\end{align*}  
\apcomment{Don't use align with notag. align* should be used in such cases.} \gwcomment{Got it, modified.}

Let 
\begin{align*}
    c_1(\lambda) := \lambda - 2 \rho(\bm{W}, F^*), \;\;\text{ and }  \;\; c_2(\lambda) \in \left(0, ~ \left(\frac{1}{2} - \frac{\rho(\bm{W}, F^*)}{2 c_1(\lambda)}\right) \frac{1 + \lambda - \rho(\bm{W}, {F^*})}{1 + \rho(\bm{W}, {F^*})} - 1 \right), 
\end{align*}
be two strictly positive parameters, i.e., 
\begin{align*}
    \lambda > \max\left\{2 \rho(\bm{W}, F^*), ~ 2 \frac{1 + \rho(\bm{W}, F^*)}{1 - \rho(\bm{W}, F^*)/c_1(\lambda)} + \rho(\bm{W}, F^*) - 1\right\}, 
\end{align*}
and define for convenience the scalars
\begin{align*}
    \alpha_* : = 2 \cdot (1 + c_2(\lambda)) \cdot \frac{\lambda_2(\bm{\Sigma}) + \rho(\bm{W}, {F^*})}{\lambda_1(\bm{\Sigma}) - \rho(\bm{W}, {F^*})}, \quad 
\mu^* := \frac{2}{\alpha^*} \frac{\lambda_2(\bm{\Sigma}) + \rho(\bm{W}, {F^*})}{\lambda_1(\bm{\Sigma}) - \rho(\bm{W}, {F^*})}, \;\;\text{ and }  \;\;
\epsilon^* := \frac{3 \rho(\bm{W}, F^{*})}{c_1(\lambda)}.
\end{align*}

\begin{theorem} \label{thm:convergence}
Using the notations above, suppose that $c_1(\lambda)$ and $c_2(\lambda)$ are strictly positive, thereby ensuring that $0 < \alpha_* < 1 - \frac{\rho(\bm{W}, {F^*})}{c_1(\lambda)}$ and $\mu^* < \frac{1}{1 + c_2(\lambda)}$. Suppose Algorithm~\ref{alg:PPM-equ} is run from an initialization $\bm{v}_0 \in \mathcal{M} \cap \mathcal{S}^{d - 1}$ satisfying $\langle \bm{v}_0, \bm{v}_* \rangle \geq \alpha_* + \frac{\rho(\bm{W}, {F^*})}{c_1(\lambda)}$. Then either $\|\bm{v}_* - \bm{v}_0 \|_2 \leq \frac{\epsilon^*}{1 - \mu^*}$, or for all $t \geq 0$, we have
\begin{align}
    \left\|\bm{v}_* - \bm{v}_{t}\right\|_2 \leq & ~ (\mu^*)^t \left\|\bm{v}_* - \bm{v}_{0}\right\|_2 + \frac{\epsilon^*}{1 - \mu^*}.
\end{align}
\end{theorem}
\noindent The proof of Theorem~\ref{thm:convergence} can be found in Appendix~\ref{app:PPM-EM}. 
\fi




Using the notation $\rho(\bm{M}, S)$ from Eq.~\eqref{eqs:rho}, we define 
$
F^{*} := \argmax_{F} \rho\big(\bm{W}, F \big)\;\text{s.t.}\; F = \text{conv}\big(L_{m_1} \cup  L_{m_2} \cup L_{m_3} \big),\; \forall \;  m_1,m_2,m_3 \in [M].
$
We now state the definition of a ``good region"; Theorem~\ref{thm:convergence} to follow shows that once Algorithm~\ref{alg:PPM} is initialized in this region, it will converge geometrically to a neighborhood of the ground truth $\bm{v}_*$.
\begin{definition}[Good region]\label{good-region}
For eigen gap $\lambda > 2 \rho(\bm{W}, F^{*})$, we define the good region 
\begin{align*} 
&\mathbb{G}(\lambda) = \big\{ \bm{v} \in \mathcal{M} \cap \mathcal{S}^{d - 1}: \langle \bm{v}, \bm{v}_{*} \rangle \geq t_1(\lambda) \big\},\quad \text{where}
\\& \quad t_1(\lambda):= \frac{4}{\lambda + 1 - \rho(\bm{W}, F^{*})} + \frac{5\rho(\bm{W}, F^{*})}{\lambda - 2\rho(\bm{W}, F^{*})}.
\end{align*}
\end{definition}
\noindent Note that $\mathbb{G}(\lambda)$ becomes a larger set as $\lambda$ increases. To ensure that such a good region is non-empty, it is necessary to have $t_1(\lambda) < 1$. In our proof of convergence (see Appendix~\ref{app:PPM-EM}), we require that the good region is not just non-empty but large enough. In particular, we require the eigengap $\lambda$ to be large enough so that
\begin{align}\label{eigen-gap-condition}
     t_2(\lambda) := \frac{4}{\lambda + 1 - \rho(\bm{W}, F^{*})} + \frac{10\rho(\bm{W}, F^{*})}{\lambda - 2\rho(\bm{W}, F^{*})} < 1. 
\end{align}
Note that this automatically ensures that $t_1(\lambda) < 1$ since $t_{2}(\lambda) > t_1(\lambda)$. 
%{\color{orange}These quantities are chosen to ensure the validitiness of our convergence result, see proof~\ref{app:PPM-EM} for details.}
%{\color{orange}Note that the second term of Inequality~\eqref{eigen-gap-condition} is slightly greater than the second term of the lower bound of $\langle \bm{v}_0, \bm{v}_{*} \rangle$ in Good region. It is designed to ensure the validness of our convergence result, see proof~\ref{app:PPM-EM} for details.}
We are now poised to state our main result for this subsection.
\begin{theorem} \label{thm:convergence}
Suppose the eigengap satisfies $\lambda > 2 \rho(\bm{W}, F^{*})$, and condition~\eqref{eigen-gap-condition} holds. Suppose in Algorithm~\ref{alg:PPM-project} the initialization satisfies $\bm{v}_{0} \in \mathbb{G}(\lambda)$. Then for all $t\geq 1$, we have
\begin{align}\label{ineq-thm-convergence}
    \| \bm{v}_{t+1} - \bm{v}_{*} \|_2 \leq \frac{1}{2^{t}} \cdot \| \bm{v}_{0} - \bm{v}_{*} \|_2 + \frac{6\rho(\bm{W}, F^{*})}{\lambda - 2\rho(\bm{W}, F^{*})}.
\end{align}

\end{theorem}
%\noindent The proof of Theorem~\ref{thm:convergence} can be found in Section~\ref{app:PPM-EM}.

 

The proof of Theorem~\ref{thm:convergence} can be found in Section~\ref{app:PPM-EM}. Once (a) an exact projection oracle $\Pi_{\mathcal{M}}$ is accessible; and (b) an initial vector $\bm{v}_0$ is in the good region $\mathbb{G}(\lambda)$, Theorem~\ref{thm:convergence} ensures a deterministic convergence result. 
%{\color{red}once the noise $\rho(\bm{W}, F^{*})$ is given.} 
Note that the result parallels that of~\cite{yuan2013truncated} for vanilla sparse PCA, where $\rho(\bm{W}, F^{*}) = O(\sqrt{k\log d / n})$. The key additional technique that we use to control the error accumulated at each iteration is based on an ``equivalent replacement'' step; see Section~\ref{proof-lemma-equivalence-of-v}. 
%which follows the framework of local convergence presented in \citet{yuan2013truncated}. The key difference is a so-called ``equivalent replacement'' step that is used to control the noise involved in each iteration, see Section~\ref{app:PPM-EM} for details.  

The projected power method was also recently analyzed by~\citet{liu2021generative} for PCA with generative models when given access to an exact projection oracle.
%. . While there may be some similarities between our work and theirs, such as the use of projection oracle and 
While they also proved local geometric convergence results given access to a sufficiently correlated initialization, there are significant differences in the assumptions of that paper and our own. First, our work imposes the union-of-linear structure assumption on the principal component, which is an altogether different structural assumption from a generative model. Given this, our proof techniques differ significantly from those of~\citet{liu2021generative}. Second, %and we provide fundamental limits of estimation for this problem. Additionally, our work 
we present a computationally efficient initialization method and matching evidence of computational hardness for two prototypical examples; see below. %Let us now turn to proving these results. 

%two prototypical examples, including exact projection oracles, initialization methods, and hardness results, which complement the theoretical results.

\iffalse
Theorem~\ref{thm:convergence} is a deterministic convergence result.
In the case where $c_1(\lambda)$ and $c_2(\lambda)$ are further bounded below by universal constants, it shows that a natural variant of power method guarantees local geometric convergence to the ground truth parameter $\bm{v}_*$ within a neighborhood of radius $\epsilon^* = \mathcal{O}(\rho(\bm{W}, F^*))$. This result relies on (1) an exact projection oracle $\Pi_{\mathcal{M}}$ being accessible; and (2) an initial vector $\bm{v}_0 \in \mathcal{M} \cap \mathcal{S}^{d - 1}$ that satisfies the initialization condition $\langle \bm{v}_0, \bm{v}_* \rangle \geq \alpha_* + \frac{\rho(\bm{W}, {F^*})}{\lambda - 2 \rho(\bm{W}, {F^*})}$. 
%Here, we only state the convergence result concerning the general restricted statistical error $\rho(\bm{W}, F^*)$. As we will see in Section~\ref{sec:specific-examples}, this general error $\rho(\bm{W}, F^*)$ will be replaced by some specific non-asymptotic upper bounds under tree/path sparsity samples, respectively. For people with independent interests, 
The proof of Theorem~\ref{thm:convergence}, follows the framework of local convergence in \citet{yuan2013truncated}. The key difference is a so-called ``equivalent replacement'' step that is used to control the noise involved in each iteration, see Appendix~\ref{app:prop-decom-covar} for details.
% In comparison with the local convergence result of sparse PCA \citep{yuan2013truncated}, the initialization condition and the local geometric convergence are of a similar format. The only difference is the requirement of the exact projection oracle $\Pi_{\mathcal{M}}$ which is accessible in sparse PCA.  \\
% The projected power method was also studied in recent work for generative PCA by \cite{liu2021generative}, under a different structural assumption on the principal component. 
\apcomment{Please rewrite the following text; it does not read well.}
{\color{orange}
A recent work from \cite{liu2021generative} studies the similar projected power method for the so-called generative PCA problem. In the present paper, the structural assumption we placed on the principal component is distinct from theirs, and the proof techniques we used is also different. We further give the fundamental limits of estimation for union-of-linearly structured PCA, which are not covered in \cite{liu2021generative}. Moreover, end-to-end analysis (which includes efficient initialization methods and hardness results) is proposed for two prototypical examples. In contrast, the initial oracle assumed in \cite{liu2021generative} may not be easily implementable under their assumptions for generative  PCA.  
}
\fi

% In particular, [Assumption 2, \cite{liu2021generative}] is not required in the linear structure condition; (2) and linear subspaces in our setting have different dimensions, unlike their feasible set $G(\mathbb{B}^k_2(1)) = \{G(\bm{x}) ~|~ \bm{x} \in \mathbb{B}^k_2(1)\}$ which is intrinsic $k$-dimensional with $\mathbb{B}^k_2(1) = \{\bm{x} \in \mathbb{R}^k ~|~ \|\bm{x}\|_2 \leq 1\}$. Although the local geometric convergence result in \cite{liu2021generative} for the projected power method is similar, the proof techniques are different. Besides the above algorithmic results, we also give a general form of minimax lower bounds for the linearly structured PCA (see Section~\ref{sec:fund-limits}), study the computational hardness (instead of the information-theoretical lower bound) of two prototypical examples, and propose initialization methods to complete the story (see Section~\ref{sec:specific-examples}). 

%Recall that Theorem~\ref{thm:convergence} requires an initialization $\bm{v}_0$ in the good region $\mathbb{G}(\lambda)$. To complete the story, we provide such an initialization method (see Algorithm~\ref{alg:initialization} in Section~\ref{sec:initial-method-general}) that works when given a projection oracle, with an additional assumption holds true. \mqcomment{I think we can move the initialization result to the main text.}

\subsection{Initialization method} \label{sec:initial-method-general}

Recall that Theorem~\ref{thm:convergence} requires an initialization $\bm{v}_0$ in the good region $\mathbb{G}(\lambda)$. In this subsection, we provide such an initialization method (see Algorithm~\ref{alg:initialization}) that works when given a projection oracle, provided the following assumption holds. 
%\apcomment{Just making sure: We don't need any independence assumption between $v_0$ and the data?}

%whose output ensures the required initialization condition of $\bm{v}_0$.

% \begin{assumption} \label{assump:M-set-initial}
% Suppose the linear structure set $\mathcal{M}$ can be represented as $\mathcal{M} = \{\bm{v} \in \mathbb{R}^d ~|~ \|\bm{v}\|_0 = k\} \cap \{\textup{additional-structure}\}$, i.e., an intersection between set of $k$-sparsity vectors and some additional structure constraints. 
% \end{assumption}

\begin{assumption}\label{assump:M-set-initial}
The set $\mathcal{M}$ in  satisfies
\[
    \mathcal{M} \subseteq \{\bm{v} \in \mathbb{R}^{d} : \|\bm{v}\|_{0} = k\},\quad \text{where} \quad k \in \mathbb{N}.
\]
\end{assumption}


Assumption~\ref{assump:M-set-initial} is not guaranteed by Definition~\ref{cond:linear-structure}, but includes many typical examples. For instance, the sets $\mathcal{T}^k$ and $\mathcal{P}^k$ for tree-sparse or path-sparse PCA, respectively, satisfy Assumption~\ref{assump:M-set-initial} in addition to union-of-linear structure. Moreover, if the orthonormal matrix $\bm{\Phi}$ is known, one can reformulate the problem as  estimating the structured-sparse vector $\bm{\Phi}^{\top} \bm{v}_{*}$ from observations $\{\bm{\Phi}^{\top} \bm{x}_i\}_{i = 1}^n$ (see Remark~\ref{rem:eq-structured}).

%\apcomment{Fill in once the previous discussion about structured sparsity is made into a remark.}\gwcomment{Got it.}
% In this part, we propose an initialization method (Covariance Thresholding with Projection, see Algorithm~\ref{alg:initialization}) for the projected power method (Algorithm~\ref{alg:PPM}). 

% We start with the notations used in this subsection. Let $C_1, C_2, C_3 > 0$ be three numerical constants. Define $\tau_* := C_1 \max\{\lambda, ~ 1\} \sqrt{\log (d/k^2)}$. Set the thresholding parameter
% \begin{align*}
%     \tau := \left\{
%     \begin{array}{lll}
%         \tau_* & \textup{when  } \tau_* \leq \sqrt{\log d} / 2, ~ k^2 \leq d / e, \\
%         C_2 \tau_* & \textup{when  } \tau_* \geq \sqrt{\log d} / 2, ~ k^2 \leq d / e, \\
%         0 & \textup{otherwise.}
%     \end{array}
%     \right. ~~ . 
% \end{align*}
% Recall the set $F^*$, strict positive parameters $c_1(\lambda)$, $c_2(\lambda)$, and parameters $\alpha_*$, $\mu^* < 1 / (1 + c_2(\lambda))$, $\epsilon^*$ given in Section~\ref{sec:PPM-EM}.

\begin{algorithm}
\caption{Initialization Method -- Covariance Thresholding with Projection Oracle}
\label{alg:initialization}
\textbf{Input.} $\{\bm{x}_i\}_{i = 1}^{n}$, parameter $k \in \mathbb{N}$, thresholding parameter $\tau$ and exact projection $\Pi_{\mathcal{M}}$. 
\begin{algorithmic}[1]
\State Compute covariance matrix $\widehat{\bm{\Sigma}} = \sum_{i = 1}^n \bm{x}_i \bm{x}_i^{\top} / n$. 
\State Set the soft-thresholding matrix $\widehat{\bm{G}}(\tau)$ as:
\begin{align*}
&\text{If} \quad \quad~~~ \widehat{\bm{\Sigma}}_{ij} - [\bm{I}_d]_{ij} \geq \tau/\sqrt{n}, \quad ~~ \text{then} \quad [\widehat{\bm{G}}(\tau)]_{ij} = \widehat{\bm{\Sigma}}_{ij} - [\bm{I}_d]_{ij} - \tau/\sqrt{n};\\
&\text{else if} \quad \widehat{\bm{\Sigma}}_{ij} - [\bm{I}_d]_{ij} \leq -\tau/\sqrt{n}, \quad \text{then} \quad [\widehat{\bm{G}}(\tau)]_{ij} = \widehat{\bm{\Sigma}}_{ij} - [\bm{I}_d]_{ij} + \tau/\sqrt{n};
\\&\text{else} \quad \quad [\widehat{\bm{G}}(\tau)]_{ij} = 0.
\end{align*}
% \begin{align*}
%     [\widehat{\bm{G}}(\tau)]_{ij} := \left\{
%     \begin{array}{llll}
%         \widehat{\bm{\Sigma}}_{ij} - [\bm{I}_d]_{ij} - \tau/\sqrt{n} \\
%         \text{~ if } ~ \widehat{\bm{\Sigma}}_{ij} - [\bm{I}_d]_{ij} \geq \tau/\sqrt{n}, \\
%         0 \\
%         \text{~ if } ~ \left| \widehat{\bm{\Sigma}}_{ij} - [\bm{I}_d]_{ij} \right| < \tau / \sqrt{n}, \\
%         \widehat{\bm{\Sigma}}_{ij} - [\bm{I}_d]_{ij} + \tau/\sqrt{n} \\
%         \text{~ if } ~ \widehat{\bm{\Sigma}}_{ij} - [\bm{I}_d]_{ij} \leq - \tau/\sqrt{n}, 
%     \end{array}
%     \right. ~~ .
% \end{align*}
\State Compute  $\widehat{\bm{v}}_{\textup{soft}} := \max_{\|\bm{v}\|_2 = 1} \bm{v}^{\top} \widehat{\bm{G}}(\tau) \bm{v}$ as the leading eigenvector of $\widehat{\bm{G}}(\tau)$.
\State Project $\bm{v}_0 :=  \Pi_{\mathcal{M}}(\widehat{\bm{v}}_{\textup{soft}}) / \|\Pi_{\mathcal{M}}(\widehat{\bm{v}}_{\textup{soft}})\|_2$.
\end{algorithmic}
\textbf{Return} $\bm{v}_0 \in \mathcal{S}^{d - 1} \cap \mathcal{M}$. 
\end{algorithm}  
 
% Thus the output $\bm{v}_0 = \Pi_{\mathcal{M}}(\widehat{\bm{v}}_{\textup{soft}})$ of Algorithm~\ref{alg:initialization} satisfies the following theorem. 

\begin{theorem} \label{thm:initialization-method}
Suppose Assumption~\ref{assump:M-set-initial} holds and $k^2 \leq d/e$. There exists a tuple of universal, positive constants $(C_1,C_2,C_3,C)$ such that the following holds. Suppose $n\geq \max\{C\log d,k^{2}\}$ and let $\tau_{*} := C_1 \max\{\lambda, 1\} \sqrt{\log (d/k^2)}$. Set the thresholding level according to 
\begin{align}\label{initia-threshold-tau}
    \tau := \left\{
    \begin{array}{lll}
        \tau_* & \textup{when  } \tau_* \leq \sqrt{\log d} / 2, \\
        C_2 \tau_* & \textup{when  } \tau_* \geq \sqrt{\log d} / 2, \\
        0 & \textup{otherwise.}
    \end{array}
    \right. 
\end{align}
Then for any $0<c_{0} <1$, if
\begin{align*}
    n \geq  n_0(c_0) := \frac{18 C_3 \max\{\lambda^2, ~ 1\} k^2}{2(1 - c_0)^2 \lambda^2} \log(d/k^2),
\end{align*}
then the initial vector $\bm{v}_0 \in \mathcal{S}^{d - 1} \cap \mathcal{M}$ obtained from Algorithm~\ref{alg:initialization} satisfies $\langle \bm{v}_0, \bm{v}_* \rangle \geq c_0$ with probability $1 - C'\exp(- \min\{\sqrt{d}, n\}/C')$ for some positive constant $C'$.

%\gwcomment{Here the contraintuitive behavior of term $d$ (as discussed below) is due to the setting of $\tau_* = O(\sqrt{\log (d/k)}) \leq \sqrt{\log d} / 2$. Based on such setting, as $d$ increasing, the RHS probability bound in [formulation (104), Covariance Thresholding] decreases due to a Chernoff-Hoeffding type bound, which will be used to control [Prop 13, Covariance Thresholding].} \\ 
%
%\gwcomment{To Ashwin: In the paper Covariance Thresholding for SPCA, this error probability $o(1)$ is of the form $C'\exp(- \min\{\sqrt{d}, n\}/C')$. However, we find that such error probability contradicts to our intuition. In particular, the error probability $C'\exp(- \min\{\sqrt{d}, n\}/C')$ increases as the dimension $d$ decreases (when $n^2 \geq d \geq n \geq k^2$), i.e., the lower dimension $d$, the easier estimation task for SPCA.} 
\end{theorem}

%\gwcomment{Write carefully the event $\bm{v}_0$ are dependent on the data $\bm{X}$, and $t_2(\lambda) < \frac{1}{2}$ (some constant) with high prob.}\\ 
%\gwcomment{Now written in the following logic: (1) Set $c_0 = 7/8$. (2) Say Theorem 3 shows $\langle \bm{v}_0, \bm{v}_{*} \rangle \geq 7/8$ whp. (3) Show that $7/8 > t_2(\lambda)$ whp. (4) $v_0$ is in good region.}

The proof of Theorem~\ref{thm:initialization-method}, which builds on existing results in \cite{deshpande2016sparse}, can be found in Appendix~\ref{app:initialization}. Let us now show that the output of this algorithm serves as a valid initialization for the projected power method, since this is not immediate given that the event $\mathcal{E}_1 = \{\langle \bm{v}_0, \bm{v}_{*} \rangle \geq c_0\}$  depends on the samples $\{\bm{x}_i\}_{i = 1}^n$.
%
%
%Observe that for any $0 < c_0 < 1$, the event $\{\langle \bm{v}_0, \bm{v}_{*} \rangle \geq c_0\}$  depends on the sample $\{\bm{x}_i\}_{i = 1}^n$. 
Recall the quantities $t_1(\lambda)$ and $t_2(\lambda)$ in Definition~\ref{good-region} and Eq.~\eqref{eigen-gap-condition}, respectively. 
In Theorem~\ref{thm:initialization-method}, set $c_0 := t_2(\lambda)$ and recall that $t_2(\lambda) > t_2(\lambda)$ by definition. Suppose $\lambda \geq 5$ for convenience. Then it can be shown that the event $\mathcal{E}_2 = \{ t_1(\lambda) < t_2(\lambda) = c_0 < 7/8\} \subseteq \{\rho(\bm{W}, F^*) < 9/400\}$ occurs with probability at least $1 - C' \exp(- \min\{\sqrt{d}, n\} / C')$. 
%Here, this probabilistic bound is a trivial corollary based on the similar noise upper bound for vanilla sparse PCA, as also mentioned in Remark 6.6 of \citet{deshpande2016sparse}. 
%
%\mqcomment{Why this event happen with high probability? Can we cite the proof?}
%
%and suppose that $\lambda \geq 5$ and $n \geq n_0(7/8)$. Then %some calculations show that with high probability \apcomment{How high?}, we have that 
%the event
%%Setting $c_0 = 7/8$ and thus the following event for $t_2(\lambda)$
%%\begin{align} \label{eq:c_0-upper-bound}
% $   \mathcal{E}_2 = \{ t_1(\lambda) <  7/8\}$
%%\end{align}
%occurs with high probability 
%holds with high probability when $\lambda \geq 5$ and $n \geq n_0(7/8)$.
Consequently, on the high probability event $\mathcal{E}_1 \cap \mathcal{E}_2$, we have that the initialization $\bm{v}_{0}$ obtained by Algorithm~\ref{alg:initialization} satisfies $\bm{v}_0 \in \mathbb{G}(\lambda)$. The projected power method can thus be employed after this initialization to guarantee convergence to a small neighborhood of $\bm{v}_*$.

%the good region condition. {\color{orange}Although t

A key feature of Theorem~\ref{thm:initialization-method} is the lower bound $n_0 = \Theta(k^2 \log(d / k^2) )$ on the number of samples required for the Algorithm~\ref{thm:initialization-method} to succeed. 
Note that this is of a strictly %in Theorem~\ref{thm:initialization-method} is 
larger order than the number of samples required information-theoretically even for vanilla sparse PCA---this is a well-known phenomenon. In the next section, we show that even with the additional structure afforded by tree and path sparsity, this larger sample size is in some sense necessary for computationally efficient algorithms.


% by showing matching evidence of computation hardness in these problems.
%the minimax lower bound proposed in Corollary~\ref{coro:TS-PCA-fund-limits}(b), Section~\ref{sec:examples-SDP-hard} shows that $O(k^2)$ number of samples is necessary in term of SDP hardness for tree-sparse PCA; while Section~\ref{sec:examples-average-hard} proves that $O(k^2)$ number of samples is necessary in term of average-case hardness for path-sparse PCA. Furthermore, we would like to point that the above probability $1 - C'\exp(- \min\{\sqrt{d}, n\}/C')$ ensures the high probability condition $1 - o(1)$ as $d, n \rightarrow \infty$.} \\
%\gwcomment{Move to Section 3.3 after the result of initialization methiod.} 
%{\color{orange} 
%Here, we would like to point out a recent study \cite{liu2021generative} on the projected power method for PCA with generative models. While there may be some similarities between our work and theirs, such as the use of projection oracle and local geometric convergence results for power methods, there are also significant differences in the assumptions and goals of the two papers. Our work focuses on the linear structure assumption for the principal components, which is a different structural assumption from those in \cite{liu2021generative}, and we provide fundamental limits of estimation for this problem. Additionally, our work presents an end-to-end analysis of two prototypical examples, including exact projection oracles, initialization methods, and hardness results, which complement the theoretical results.
%a recent study \cite{liu2021generative}. This paper explores the projected power method for PCA problem in which its ground truth follows a generative model, and prove a local geometric convergence result under appropriate conditions with an accessible exact projection oracle. In comparison, our paper focuses on a different structural assumption on the principal components and employs novel proof techniques. Additionally, we provide fundamental limits of estimation for linearly structured PCA. Later, Section~\ref{sec:specific-examples} further presents an end-to-end analysis, which encompasses tractable exact projection oracles, efficient initialization methods and hardness results, for two prototypical examples.
%}


%\subsection{Compare with some recent works}
%\gwcomment{Consider whether add this subsection or not.}

%%%%%%%%%%%%%%%%%%%%%%%%%%%%%%%%%%%%%%%%%%%%%%%%%%%%%%%%%%%
%%%%%%%%%%%%%%%%%%%%%%%%%%%%%%%%%%%%%%%%%%%%%%%%%%%%%%%%%%%
%%%%%%%%%%%%%%%%%%%%%%%%%%%%%%%%%%%%%%%%%%%%%%%%%%%%%%%%%%%
%%%%%%%%%%%%%%%%%%%%%%%%%%%%%%%%%%%%%%%%%%%%%%%%%%%%%%%%%%




 

%!TEX root = IEEETSP-main-paper.tex

\section{End-to-end analysis for specific examples} \label{sec:specific-examples}

In this section, we provide end-to-end analyses for path-sparse and tree-sparse PCA, including results on their information-theoretic limits of estimation as well as the performance of the projected power method when initialized using covariance thresholding. We complement these with what may be considered as the main results of this section: matching suggestions of computational hardness. 
%in these problems. 

\subsection{Path-Sparse PCA} \label{sec:PS-PCA}

\subsubsection{Fundamental limits for Path-Sparse PCA} \label{sec:limits-PS-PCA}

Recall the notation $\mathcal{P}^k$ as the structure set of path-sparse PCA from Section~\ref{sec:PS-PCA-intro}. We write $\bm{v} \in \mathcal{P}^k$ if the support set satisfies $\mathsf{supp}(\bm{v}) \in \mathcal{P}^k$. We use 
\begin{align}
    \widehat{\bm{v}}_{\textsf{PS}} := \argmax_{\bm{v}} ~ \bm{v}^{\top} \widehat{\bm{\Sigma}} \bm{v} ~~\text{s.t.}~~ \bm{v} \in \mathcal{S}^{d - 1} \cap \mathcal{P}^k \label{eq:PS-PCA-est}
\end{align}  
to denote the corresponding estimate from exhaustive search. 
%Similarly to $F^*$, define 
%$
%P^{*} := \argmax_F \rho(\bm{W}, {F}) \; \text{s.t.}\; F = \text{conv}\big(L_{m_1} \cup L_{m_2} \cup L_{m_3} \big),\; \forall\; L_{m_1}, L_{m_2}, L_{m_3} \in \mathcal{P}^k.
%$
%\mqcomment{we seems didn't use the definition of $P^{*}$ in the main text.}

\begin{corollary}\label{coro:PS-PCA-fund-limits}
%Let $\mathcal{P}^k$ be defined as in Section~\ref{sec:PS-PCA-intro}. 
There exists a pair of positive constants $(c, C)$ such that the following holds.
\noindent (a) Without loss of generality, assume $\langle \bm{v}_*, \widehat{\bm{v}}_{\mathsf{PS}} \rangle \geq 0$. Then for any $c_1>0$ and $\bm{v}_* \in \mathcal{S}^{d - 1} \cap \mathcal{P}^k$, we have 
\[
 \big\|\widehat{\bm{v}}_{\mathsf{PS}} - \bm{v}_*\big\|_2 \leq C \left( \frac{1 + \lambda}{\lambda} \right) \sqrt{\frac{3 (\ln d - \ln k)k + c_1k}{n}}
 \] 
 with probability at least $1 - 2\exp(- c_1k)$.

\noindent (b) Suppose that $d\geq 16 k^2$ and $k\geq 4$. Then we have the minimax lower bound 
\begin{align*}
&\inf_{\widehat{\bm{v}}} \; \sup_{\bm{v}_* \in \mathcal{S}^{d - 1} \cap \mathcal{P}^k} \mathbb{E}\left[ \left\| \widehat{\bm{v}} \widehat{\bm{v}}^{\top} - \bm{v}_* \bm{v}_*^{\top}  \right\|_F \right] \geq  c \cdot \min\bigg\{1,  \sqrt{\frac{1 + \lambda}{8 \lambda^2}} \sqrt{\frac{k \cdot \big( \frac{\ln d}{2} - \ln k \big)}{n}} \bigg\}.
\end{align*}
Here, the infimum is taken over all measurable functions of the observations $\{ \bm{x}_i \}_{i = 1}^n$ drawn i.i.d. from the distribution $\mathcal{D}(\lambda; \bm{v}_*)$.
\end{corollary}

Corollary~\ref{coro:PS-PCA-fund-limits} is proved in Section~\ref{proof-coro-PS-PCA-fund-limits} of the supplementary material, and is based on Theorem~\ref{thm:fund-limits}. In particular, Corollary~\ref{coro:PS-PCA-fund-limits}(a) gives an upper bound on the estimation error of $\widehat{\bm{v}}_{\textsf{PS}}$ by showing that the statistical noise term\footnote{As expected, this term does not differ significantly from the corresponding term for vanilla sparse PCA, since the number of sparsity patterns for path sparse PCA $|\mathcal{P}^k|$ is on the order $(d/k)^k$.} $\rho(\bm{W}, P^*)$ is of the order $(\lambda + 1) \sqrt{k \cdot ( \ln d - \ln k)/n}$. 
%{\color{red}Compared with the vanilla sparse PCA, its logarithmic term changes from $\log d$ to $\log(d / k)$ due to the path sparsity structure.}
The minimax lower bound obtained in Corollary~\ref{coro:PS-PCA-fund-limits}(b) is of the same order as the minimax lower bound given in [Theorem 1, \cite{asteris2015stay}] with the outer degree parameter $|\Gamma_{\text{out}}(v)| = (d - 2) / k$. 


\subsubsection{Local convergence and initialization} \label{sec:initialization-PS-PCA}
\paragraph{Exact projection oracle} We build the exact projection oracle for path-sparse PCA $\Pi_{\mathcal{P}^k}$ by picking the component with the largest absolute value in each partition (layer) for a given $(d,k)$-layered graph $G$. The formal procedure is given in Algorithm~\ref{alg:projection-PSPCA} (see Section~\ref{app:initial-examples}), and has running time $O(d)$.  


\begin{corollary} \label{coro:PS-PCA-PPM}
 Suppose the initiatialization $\bm{v}_0$ in Algorithm~\ref{alg:PPM} satisfies $\bm{v}_{0} \in \mathcal{P}^{k} \cap \mathcal{S}^{d-1}$ and $\langle \bm{v}_0,\bm{v}_* \rangle \geq 1/2$. There exists a tuple of universal positive constants $(c,C_1,C_2,C_3)$ such that for $\lambda \geq C_1$, $n\geq C_2k\ln(d)$,  and all $t \geq 1$, the iterate $\bm{v}_t$ from Algorithm~\ref{alg:PPM} satisfies 
 \[
 \| \bm{v}_{t} - \bm{v}_{*} \|_2 \leq \frac{1}{2^t} \cdot \| \bm{v}_{0} - \bm{v}_{*} \|_2 +  C_3 \sqrt{\frac{k(2\ln d - \ln k)}{n}},
 \] 
 with probability at least $1-\exp(-ck)$. 
\end{corollary}
Corollary~\ref{coro:PS-PCA-PPM} is proved in Section~\ref{proof-coro-PS-PCA-PPM}
%. {\color{orange}
%It is proved by 
by applying Theorem~\ref{thm:convergence}. %and showing
%\begin{align*}
%    \rho(\bm{W}, P^*) \lesssim  (1+\lambda)\sqrt{\frac{k\ln d + ck}{n}}
%\end{align*}
%holds with probability at least $1 - \exp(- ck)$.}
%where $P^*$ above is defined similar to $F^*$ with respect to path sparsity set $\mathcal{P}^k$. 
%\apcomment{Define $P^*$ before by making explicit reference to $F^*$.} \gwcomment{Done, at the beginning of Section 4.1.1.}
%

The final problem is to obtain an initialization $\bm{v}_0$.
To do so, note that the set $\mathcal{P}^k$ satisfies Assumption~\ref{assump:M-set-initial}, leading to the following corollary of Theorem~\ref{thm:initialization-method}.

%We set $\mathcal{M} = \mathcal{P}^k$ and use the exact projection oracle of path-sparse PCA $\Pi_{\mathcal{P}^k}$ \footnote{The exact projection $\Pi_{\mathcal{P}^k}$ can be established by picking the component of largest absolute value in each partition (layer) for a given $(d,k)$-layered graph $G$.} (with running time $O(d)$) in both Algorithm~\ref{alg:PPM-project} and Algorithm~\ref{alg:initialization}.
\begin{corollary} \label{coro:initial-PS-PCA}
Assume $k^2 \leq d / e$. There exists a pair of universal positive constants $(C, C')$ such that if $n\geq \max\{C\log d,k^{2}\}$ and $n \geq C'\max \big\{1,\lambda^{-2} \big\} \log(d/k^2) k^{2}, $ then the initial vector $\bm{v}_0 \in \mathcal{S}^{d - 1} \cap \mathcal{P}^k$ obtained from Algorithm~\ref{alg:initialization} satisfies $\langle \bm{v}_0, \bm{v}_* \rangle \geq 7/8$ with probability $1 - C'\exp(- \min\{\sqrt{d}, n\}/C')$. 
%\gwcomment{Add some discussions in next paragraph. This $7/8$ is selected to be the same with the $7/8$ presented in the discussion of Theorem 3 for initialization.}
\end{corollary}

%It is straightforward to see that Corollary~\ref{coro:initial-PS-PCA} follows from Theorem~\ref{thm:initialization-method} by specifying $c_0 = 0.5$. \apcomment{It was $7/8$ before. Please stay consistent. In fact, I would remove this sentence.}
%The proof of Corollary~\ref{coro:initial-PS-PCA} is derived from Theorem~\ref{thm:convergence} in Appendix~\ref{app:initial-methods}. 
%The proof of Corollary~\ref{coro:initial-PS-PCA} exactly follows the proof of Theorem~\ref{thm:initialization-method} by setting $c_0$ presented in Theorem~\ref{thm:initialization-method} as $7/8$. Thus, we omit the proof of Corollary~\ref{coro:initial-PS-PCA} in appendix. 
In words, Corollary~\ref{coro:initial-PS-PCA} provides an initialization method whose outputs can be used for the general projected power method (Algorithm~\ref{alg:PPM}) for path-sparse PCA when the number of samples\footnote{The constant $7/8$ in $\langle \bm{v}_0, \bm{v}_* \rangle \geq 7/8$ can be replaced by any positive constant within $(0, 1)$  provided it ensures the good region condition $\langle \bm{v}_0, \bm{v}_* \rangle > t_2(\lambda)$.} $n \gtrsim k^2 \log(d/k^2)$. 

As previously mentioned, there is a gap between the condition $n \gtrsim k$ required for Corollary~\ref{coro:PS-PCA-PPM} and the stronger condition above. We will now show evidence that $k^{2}$ samples are necessary. In particular, %in Section~\ref{sec:examples-average-hard}, 
we will show that no randomized polynomial-time algorithm can ``solve" (i.e. produce a consistent estimate for) path-sparse when $n \ll k^2$, provided we assume the  average-case hardness of the secret-leakage planted clique problem. This can be regarded as the main takeaway for path-sparse PCA: The additional structure has minimal effect on its statistical and computational limits.


%%%%%%%%%%%%%%%%%%%%%%%%%%%%%%%%%%%%%%%%%%%%%%%%%%%%%%%%%%%%%%%%%%%%%%%%%%%%%%%%%%%%%%%%%%%%%%
%%%%%%%%%%%%%%%%%%%%%%%%%%%%%%%%%%%%%%%%%%%%%%%%%%%%%%%%%%%%%%%%%%%%%%%%%%%%%%%%%%%%%%%%%%%%%%
%%%%%%%%%%%%%%%%%%%%%%%%%%%%%%%%%%%%%%%%%%%%%%%%%%%%%%%%%%%%%%%%%%%%%%%%%%%%%%%%%%%%%%%%%%%%%%

\subsubsection{Average-Case Hardness of Path Sparse PCA}\label{sec:examples-average-hard}


%\apcomment{Please add the setup for computational lower bound here.} \gwcomment{Done.}
%\gwcomment{Partially done. (Will discuss with Ashwin and Mengqi)}
%\gwcomment{Need: 
%\begin{itemize}
%	\item Def 5 
%	\item Def 6 K-PC
%	\item Def hardness. What it means to be hard? K-PC is hard
%	\item Coro: if K-PC is hard, then Path sparse PCA is hard
%\end{itemize}
%}

This section focuses on the average-case hardness of the path sparse PCA, which is obtained via a reduction from the $K$-partite planted clique (PC) detection problem, which is in turn conjectured to be hard. 
%{\color{blue}Due to the page limit, we leave the following necessary definitions, e.g., Secret Leakage $\text{PC}_{\mathcal{D}}$ Detection Problem, $K$-Partite Planted Clique Detection Problem, $K$-Partite PC Hardness Conjecture, Qualified Estimator$(\epsilon)$, to Appendix~\ref{app:definition-examples}.}

\begin{definition} \label{defn:SL-PC}
\textbf{Secret Leakage $\text{PC}_{\mathcal{D}}$ Detection Problem,} \cite{brennan2020reducibility}. Given a distribution $\mathcal{D}$ on $K$-subsets of $[N]$, let $\mathcal{G}_{\mathcal{D}}(N, K, 1/2)$ be the distribution on $N$-vertex graphs sampled by first sampling $G \sim \mathcal{G}(N, 1/2)$ and $S \sim \mathcal{D}$ independently and then planting a $K$-clique on the vertex set $S$ in $G$. The secret leakage $\text{PC}_{\mathcal{D}}$ detect problem $\text{PC}_{\mathcal{D}}(N, K, 1/2)$ is defined as the resulting hypothesis testing problem between 
\[
	H_0: ~ G \sim \mathcal{G}(N, 1/2) \quad \text{and} \quad H_1: ~ G \sim \mathcal{G}_{\mathcal{D}}(N, K, 1/2).
\] 
\end{definition}

Now consider the following $K$-partite PC as a special case of the secret leakage $\text{PC}_{\mathcal{D}}$ detection problem. 

\begin{definition} \label{def:KPC}
\textbf{$K$-Partite Planted Clique Detection Problem (with source and terminal).} The $K$-partite planted clique detection problem $K\text{-PC}(N, K, 1/2)$ is a special case of the secret leakage planted clique detection problem $\text{PC}_{\mathcal{D}}(N, K, 1/2)$. Here the vertex set of $G$ has two special vertices: source and terminal, and the remaining vertices are evenly partition into $K$ parts of size $(N - 2) / K$. The distribution $\mathcal{D}$ always picks source, terminal and uniformly picks one element at random in each part.  
\end{definition}

Like the well-known planted clique conjecture, the $K$-Partite PC problem $K\text{-PC}(N, K, 1/2)$ is believed to satisfy the following hardness conjecture. 

\begin{conjecture} \label{conj:KPC-hardness}
\textbf{$K$-Partite PC Hardness Conjecture}, restatement of \cite{brennan2020reducibility}.
Suppose that $\{\mathcal{A}_N\}$ is a sequence of randomized polynomial time algorithms $\mathcal{A}_N: \mathcal{G}_N \to \{0,1\}$ and $K_N$ is a sequence of positive integers satisfying that $\limsup_{N \rightarrow \infty} \log_N K_N < 1/2$ with $\mathcal{G}_N$ the set of graphs with $N$ nodes. Then if $G$ is an instance of $K\text{-PC}(N, K_N, 1/2)$, it holds that $\liminf_{N \rightarrow \infty} \left( \mathbb{P}_{H_0}[\mathcal{A}_N(G) = 1] + \mathbb{P}_{H_1}[\mathcal{A}_N(G) = 0] \right) \geq 1. $
\end{conjecture}

\begin{definition}\label{def:qual-est}
\textbf{Qualified Estimator.} 
%\gwcomment{We have $\| \widehat{\bm{v}} - \bm{v}_* \|_2 < \frac{1}{4}$ implies $\| \widehat{\bm{v}} / \|\widehat{\bm{v}} \|_2 - \bm{v}_* \|_2 < \sqrt{2 - 2\sqrt{15/16}} < 1/2$. Modify the definition of qualified estimator due to this issue.}
A qualified estimator $\widehat{\bm{v}}(n, d_n, k_n, \lambda_n, \epsilon)$ for path-sparse PCA is a sequence of functions $\EST_n: \mathbb{R}^{d_n \times n} \to \mathbb{R}^{d_n}$ mapping $\{\bm{x}_i\}_{i = 1}^n \mapsto \widehat{\bm{v}}$ such that if the set of samples $\{\bm{x}_i\}_{i = 1}^n$ are drawn i.i.d. from $\mathcal{D}(\lambda_n, \bm{v}_*)$ for some $\bm{v}_* \in \mathcal{S}^{d_n - 1} \cap \mathcal{P}^{k_n}$ then $\liminf_{n \to \infty} \Pr \left\{ \| \widehat{\bm{v}} - \bm{v}_* \|_2 < \frac{1}{4} \right\} \geq \frac{1}{2} + \epsilon$ for some fixed $0 < \epsilon < 1/2$. 
\end{definition}

%\apcomment{Discussion of why this notion is reasonable. Also add comment saying that the init/end-to-end estimator is qualified provided $n \gtrsim k^2$.} \gwcomment{Done.}
From this point onward, we do not make $\epsilon$ explicit when referring to a qualified estimator. It suffices for the reader to think of it as a small positive constant that does not depend on $n$.
Geometrically, a qualified estimator $\widehat{\bm{v}}$ exhibits proximity to the ground truth $\bm{v}_* \in \mathcal{S}^{d_n - 1} \cap \mathcal{P}^{k_n}$ with probability at least $1/2 + \epsilon$ as $n \rightarrow \infty$. Note that Definition~\ref{def:qual-est} does not require explicit control on the behavior of $\widehat{\bm{v}}$ for a general vector $\bm{v}_* \notin \mathcal{S}^{d_n - 1} \cap \mathcal{P}^{k_n}$. 


%We require some definitions to make this precise.
%Before stating the main result of the reduction, we start with a generalization of the well-known planted clique detection problem (PC, see Definition~\ref{defn:PC}) -- Secret Leakage PC detection problem. Compared with the PC, the random graph $G$ in the Secret Leakage PC comes with some information about the vertex set of the planted clique. 


%Later, in the proof of Proposition~\ref{prop:reduction-PathPCA}, we present an approach to address a specific case when $\bm{v}_* = \bm{0}_{d_n}$ for ``null hypothesis'' of detection problem of the path-sparse PCA. 

%Back to the proposed methods in Section~\ref{sec:PS-PCA}, based on the 
It is also worth noting (using Corollary~\ref{coro:PS-PCA-PPM} and Corollary~\ref{coro:initial-PS-PCA} and the corresponding algorithms) 
%the exact projection (Algorithm~\ref{alg:projection-PSPCA}), the initialization (Algorithm~\ref{alg:initialization}), and the projected power method (Algorithm~\ref{alg:PPM}) 
that our end-to-end  estimator for path-sparse PCA is a polynomial-time computable qualified estimator provided $n \geq C k^2 \log(d /k)$ and $\lambda = \Omega(1)$. 
%Notably, such a resulting qualified estimator $\widehat{\bm{v}}$ always resides within the feasible set $\mathcal{S}^{d_n - 1} \cap \mathcal{P}^k$ and ensures polynomial running time in $n$.

\begin{proposition} \label{prop:reduction-PathPCA}
There exists a universal constant $c > 0$ such that the following holds. Let $1/2 \leq \beta <1$ and $0 < \epsilon < 1/2$ be fixed. Here, we use integer $\subindex$ as our index parameter. Suppose the sequence of parameters $\{(k_{\subindex}, d_{\subindex}, \lambda_{\subindex}, \tau_{\subindex})\}_{ \subindex \in \mathbb{N}}$ is in the parameter regime 
\[k_{\subindex} = \lceil \subindex^{\beta} \rceil, \quad d_{\subindex} = \subindex, \quad \lambda_{\subindex} = \frac{k_{\subindex}^2}{\tau_{\subindex} \cdot \subindex } \cdot \frac{(\log 2)^2}{4 (6 \log (\subindex) + 2 \log 2)},
\] 
%\begin{align} \label{eq:regime-of-int}
%	k_n = \lceil n^{\beta} \rceil, \quad d_n = n, \quad \lambda_n = \frac{k_n^2}{\tau_n n} \cdot \frac{(\log 2)^2}{4 (6 \log n + 2 \log 2)}
%\end{align}
where $\tau_{\subindex}$ is an arbitrarily slowly growing function of $\subindex$. If the K-Partite PC hardness conjecture (Conjecture~\ref{conj:KPC-hardness}) holds, then there is no qualified estimator $\widehat{\bm{v}}(n_{\subindex}, d_{\subindex}, k_{\subindex}, \lambda_{\subindex}, \epsilon)$ running in time polynomial in $d_{\subindex}$ when the sample size $n_{\subindex}$ satisfies $n_{\subindex} \leq c \left( \frac{k_{\subindex}^2}{2 \tau_{\subindex} \log k_{\subindex}} \right).$ 
%\mqcomment{the sequence $n$ and sample size $n$ is confusing and what is $N_n$?}
\end{proposition}
%
%\apcomment{Discuss in words what the implications are.}
The proof of Proposition~\ref{prop:reduction-PathPCA} is given in Section~\ref{app:reduction-PathPCA} of the supplementary material.
%which is based on an existing average-case reduction method proposed in \cite{brennan2018reducibility}. 
%
%In words, Proposition~\ref{prop:reduction-PathPCA} provides an average-case hardness lower bound on the sample size required to compute a nontrivial solution for path sparse PCA w
In particular, when the eigengap satisfies\footnote{This can be ensured for dimension $d_j = j$ growing such that $\frac{k_j}{\tau_j d_j \log d_j} = \Theta(1)$.} $\lambda = \Theta(1)$, it shows that $n = \widetilde{\Omega} \left(k^2\right)$ is necessary for computationally efficient estimation. 
%unless the $K$-Partite PC Hardness (Conjecture~\ref{conj:KPC-hardness}) is not true. Therefore, the number of samples $O(k^2 \log(d / k))$ needed in Corollary~\ref{coro:initial-PS-PCA} for path-sparse PCA of our end-to-end qualified estimator is tight comparing with this average-case hardness lower bound, ignoring logarithm terms, which completes the story. 
%implies that the estimation problem of path sparse PCA has an average-case hardness lower bound on the sample size $\left( O \left( \frac{k^2}{2 \tau \log k}\right) = \tilde{\Theta}(k^2)\right)$ with eigengap with a known constant $\lambda = \Theta(1)$ satisfying the eigengap condition (Inequality~\eqref{eigen-gap-condition}), unless the $K$-Partite PC Hardness Conjecture~\ref{conj:KPC-hardness} is not ture. Thus the samples needed for path-sparse PCA in Corollary~\ref{coro:initial-PS-PCA} is tight comparing with this average-case hardness lower bound ignoring logarithm terms.


%\begin{definition} 
%\textbf{$K$-Partite PC Hardness Conjecture, }\cite{brennan2020reducibility}.  There is no (randomized) polynomial time algorithm (over $N, K$) solving $K\text{-PC}(N, K, 1/2)$ when $K = o(\sqrt{N})$. 
%\end{definition}


%\apcomment{remove from here.}
%Moreover, we give a formal definition of the estimation problem of path sparse PCA that we would like to reduce the $K$-Partite PC detection problem (Definition~\ref{def:KPC}) to. 

%{\color{orange}
%\gwcomment{Need to discuss: Should we add $\bm{0}_d$ as a choice for $\bm{v}_*$? Since $\bm{X} \sim N(\bm{0}_d, \bm{I}_{d \times d})$ is also a possible instance for detector $\EST$.} 
%%\begin{definition}
%%\textbf{Estimation problem of path sparse PCA.} The estimation problem of path sparse PCA $\text{E-PSPCA}(n, k, d, \lambda)$ is defined as follows: Given a set of $n$ i.i.d. samples $\bm{X}$ drawn from $\mathcal{D}(\lambda; \bm{v}_*)$ with a unknown ground truth $\bm{v}_* \in \{\mathcal{S}^{d - 1} \cap \mathcal{P}^k\} \cup {\color{red}\{\bm{0}_d\}}$, the task is to find a vector $\widehat{\bm{v}} \in \{\mathcal{S}^{d - 1} \cap \mathcal{P}^k\} \cup {\color{red}\{\bm{0}_d\}}$ such that $\|\bm{v}_* - \widehat{\bm{v}}\|_2 < \frac{1}{2}$ holds with probability at least $1/2 + \epsilon$ for some fixed $0 < \epsilon < 1/2$.    
%%\end{definition}
%
%\gwcomment{Claim that our PPM is a qualified estimator.}
%Here we would like to mention that the proposed projected power method (Algorithm~\ref{alg:PPM}) succeeds for this estimation problem $\text{E-PSPCA}(n, k, d, \lambda)$ when the parameters $n, k, d, \lambda$ satisfy the conditions in Theorem~\ref{thm:convergence} with a good initialization $\bm{v}_0$. The estimation for $\bm{0}_d$ zero ground truth could be done by simply comparing the objective value of the optimization problem~\eqref{exhaustive-search-estimator} with a lower bound, see Algorithm~\ref{alg:D-from-E} for details. 
%
%
%\gwcomment{Done. Move the detection problem of PSPCA to the proof of Proposition 1. The statement of Proposition 1 is updated. But I am not sure if this statement is well-polished...}
%	
%Given parameters $n, k, d, \lambda$, we use $\mathcal{X}_n$ to denote the collection of all instances for $\text{E-PSPCA}(n, k, d, \lambda)$, and $\EST$ to denote the family of all possible estimator of path sparse PCA. Now we are ready to present the main result in this section.}



\iffalse
\begin{proposition} \label{prop:reduction-PathPCA}
%Let $\beta \in [1/2, ~ 1)$. There is a sequence $\{(k_n, d_n, \mu_n, \lambda_n, \tau(n))\}_{n \in \mathbb{N}}$ of parameters such that 
%\begin{enumerate}
%	\item The parameters are in the regime
%		\begin{align*}
%			k_n = \lceil n^{\beta} \rceil, \quad d_n = n, \quad \mu_n = \frac{\log 2}{2\sqrt{6 \log n + 2 \log 2}}, \quad \lambda_n = \frac{k_n^2}{\tau n} \cdot \mu_n^2
%		\end{align*}
%		with $\tau = \tau(n) \rightarrow \infty$, i.e., an arbitrarily slowly growing function of $n$. 
%	\item Let $0 < \epsilon < 1/2$ be fixed and let $\bm{X}_n$ be an instance of $\text{D-PSPCA}(n, k_n, d_n, \lambda_n)$. There is no sequence of randomized polynomial-time computable functions $\EST_n: \mathcal{X}_n \to \mathbb{R}^d$ that solves the $\text{E-PSPCA}(n, k_n, d_n, \lambda_n)$ with probability at least $1/2 + \epsilon$, assuming the K-Partite PC hardness conjecture (Conjecture~\ref{conj:KPC-hardness}). 
%\end{enumerate}
%Therefore, given the K-Partite PC hardness conjecture (Conjecture~\ref{conj:KPC-hardness}), by setting $\beta = 1/2$, there is no sequence of randomized polynomial-time computable functions $\EST_n: \mathcal{X}_n \to \mathbb{R}^d$ that solves the $\text{E-PSPCA}(n, k_n, d_n, \lambda_n)$ when $n \ll O(k^2)$ with a known $\lambda_n = O(1/(\tau \log n)) = \tilde{\Theta}(1)$. Moreover, by setting $k_n := \lceil c n^{1/2} \tau^{1/2} (\log n)^{1/2} \rceil$\footnote{This setting is in the parameter regime for $k_n = \lceil n^{\beta} \rceil$ with $\beta \in [1/2, 1)$.} for some constant $c$, there is no sequence of randomized polynomial-time computable functions $\EST_n: \mathcal{X}_n \to \mathbb{R}^d$ that solves the $\text{E-PSPCA}(n, k_n, d_n, \lambda_n)$ when $n \ll O \left( \frac{k^2}{2 \tau \log k}\right) = \tilde{\Theta}(k^2)$ with a known constant $\lambda = \lambda_n = \Theta(1)$ satisfying the eigengap condition (Inequality~\eqref{eigen-gap-condition}).

Let $\beta \in [1/2, ~ 1)$ and $0 < \epsilon < 1/2$ be fixed. Let the sequence $\{(k_n, d_n, \mu_n, \lambda_n, \tau(n))\}_{n \in \mathbb{N}}$ of parameters is in the regime
\begin{align} \label{eq:regime-of-int}
	k_n = \lceil n^{\beta} \rceil, \quad d_n = n, \quad \mu_n = \frac{\log 2}{2\sqrt{6 \log n + 2 \log 2}}, \quad \lambda_n = \frac{k_n^2}{\tau n} \cdot \mu_n^2
\end{align}
with $\tau = \tau(n) \rightarrow \infty$, i.e., an arbitrarily slowly growing function of $n$. Assume the K-Partite PC hardness conjecture (Conjecture~\ref{conj:KPC-hardness}) holds. By setting $k_n := \lceil c n^{1/2} \tau^{1/2} (\log n)^{1/2} \rceil$\footnote{This setting is in the parameter regime for $k_n = \lceil n^{\beta} \rceil$ with $\beta \in [1/2, 1)$.} for some constant $c$, there is no sequence of randomized polynomial-time computable functions $\EST_n: \mathcal{X}_n \to \mathbb{R}^d$ that solves the $\text{E-PSPCA}(N_n, k_n, d_n, \lambda_n)$ with probability at least $1/2 + \epsilon$, when the number of samples $N_n \ll O \left( \frac{k_n^2}{2 \tau \log k_n}\right) = \tilde{\Theta}(k_n^2)$ with a known constant $\lambda = \lambda_n = \Theta(1)$ satisfying the eigengap condition (Inequality~\eqref{eigen-gap-condition}).
\iffalse
we have the following results:
\begin{itemize}
	\item By setting $k_n := \lceil n^{1/2} \rceil$, i.e., $\beta = 1/2$, there is no sequence of randomized polynomial-time computable functions $\EST_n: \mathcal{X}_n \to \mathbb{R}^d$ that solves the $\text{E-PSPCA}(N_n, k_n, d_n, \lambda_n)$ with probability at least $1/2 + \epsilon$, when the number of samples $N_n \ll O(k_n^2)$ with a known $\lambda_n = O(1/(\tau \log n)) = \tilde{\Theta}(1)$.
	\item By setting $k_n := \lceil c n^{1/2} \tau^{1/2} (\log n)^{1/2} \rceil$\footnote{This setting is in the parameter regime for $k_n = \lceil n^{\beta} \rceil$ with $\beta \in [1/2, 1)$.} for some constant $c$, there is no sequence of randomized polynomial-time computable functions $\EST_n: \mathcal{X}_n \to \mathbb{R}^d$ that solves the $\text{E-PSPCA}(N_n, k_n, d_n, \lambda_n)$ with probability at least $1/2 + \epsilon$, when the number of samples $N_n \ll O \left( \frac{k_n^2}{2 \tau \log k_n}\right) = \tilde{\Theta}(k_n^2)$ with a known constant $\lambda = \lambda_n = \Theta(1)$ satisfying the eigengap condition (Inequality~\eqref{eigen-gap-condition}). 
\end{itemize}
\fi
\end{proposition}
\fi




%\begin{proof}
%%\apcomment{Make notation consistent.}
%%\gwcomment{This proof will be moved to Section~\ref{app:reduction-PathPCA}.}
%%\gwcomment{The proof has been updated.}
%For the proof, one should think of the estimator and parameters as indexed by the natural number $n$, although we drop this explicit dependence for clarity of exposition.
%We begin by introducing formal definition for the detection problem for path sparse PCA. Recall the definition of a qualified estimator (Definition~\ref{def:qual-est}).
%%, respectively.  
%
%\begin{definition}
%\textbf{Detection problem for path sparse PCA.} Suppose $\bm{v}_*$ is $\bm{0}_d$ with probability $1/2$ and an arbitrary vector in the set $\mathcal{S}^{d - 1} \cap \mathcal{P}^k$ with probability $1/2$. The detection problem for path sparse PCA -- $\text{D-PSPCA}(n, k, d, \lambda)$ is defined as the resulting hypothesis testing problem 
%\begin{align*}
%	H_0: \bm{X} \sim \mathcal{D}(0; \bm{0}_d)^{\otimes n} ~~~~\text{and}~~~~  H_1: \bm{X} \sim \mathcal{D}(\lambda; \bm{v}_*)^{\otimes n}.
%\end{align*}
%%with a unknown ground truth $\bm{v}_* \in \mathcal{S}^{d - 1} \cap \mathcal{P}^k$. 
%\end{definition}
%
%%\begin{definition}
%%\textbf{Estimation problem for path sparse PCA.} The estimation problem of path sparse PCA -- $\text{E-PSPCA}(n, k, d, \lambda)$ is to find a qualified estimator$(\epsilon)$ denoted by $\widehat{\bm{v}}$ which additionally guarantees that
%%	\begin{align*}
%%		\liminf_{n \to \infty} \Pr \left\{  \widehat{\bm{v}} = \bm{0}_{d_n} \right\} \geq \frac{1}{2} + \epsilon ,
%%	\end{align*} 
%%	if $\bm{v}_* = \bm{0}_d$. % with some fixed $\epsilon$ used in the qualified estimator. 
%%%defined as follows: Given a set of $n$ i.i.d. samples $\bm{X}$ drawn from $\mathcal{D}(\lambda; \bm{v}_*)$ with a unknown ground truth $\bm{v}_* \in \{\mathcal{S}^{d - 1} \cap \mathcal{P}^k\} \cup {\color{red}\{\bm{0}_d\}}$, the task is to find a vector $\widehat{\bm{v}} \in \{\mathcal{S}^{d - 1} \cap \mathcal{P}^k\} \cup {\color{red}\{\bm{0}_d\}}$ such that $\|\bm{v}_* - \widehat{\bm{v}}\|_2 < \frac{1}{2}$ holds with probability at least $1/2 + \epsilon$ for some fixed $0 < \epsilon < 1/2$.    
%%\end{definition} 
%%\apcomment{Why do you need to define the estimation problem at all? You directly construct a detector it seems?}
%
%We show our average-case hardness result by contradiction, i.e., we assume there exists a randomized polynomial-time qualified estimator. We then transform it into a good detector for the path-sparse PCA problem and eventually the secret leakage planted clique problem. The proof of Proposition~\ref{prop:reduction-PathPCA} can be separated into four parts: 
%
%\begin{enumerate}
%	\item \textbf{Constructing a randomized polynomial time algorithm for detection based on an estimation algorithm for path sparse PCA.} 
%	
%	Suppose $\EST_n: (\mathbb{R}^{d_n})^n \to \mathbb{R}^{d_n}$ is a sequence of randomized polynomial time functions for the assumed qualified estimator. 
%	
%%	\gwcomment{To be general, let the successful prob of nontrivial detector wp $> 1/2$} 
%%	$\EST_n$ maps an instance $\bm{X}$ of $n$ samples for path sparse PCA problem to an estimation $\widehat{\bm{v}} \in \mathcal{S}^{d - 1} \cap \mathcal{P}^k$ which satisfies the following property: {\color{orange}when we have $\lambda = \Theta(1)$ some known constant}, and $\bm{X} \sim \mathcal{D}(\lambda; \bm{v}_*)^{\otimes n}$ with 
%%		\begin{align*}
%%			\bm{v}_* \in \text{UPSPCA}^k := \left\{ \left. \bm{v} \in \left\{0, \frac{1}{\sqrt{k}} \right\}^d ~\right|~ \bm{v} \in \mathcal{S}^{d - 1} \cap \mathcal{P}^k \right\},  
%%		\end{align*} 
%%		$\mathcal{R}_n$ ensures that $\widehat{\bm{v}} \in \text{UPSPCA}^k$ and $\|\bm{v}_* - \widehat{\bm{v}}\|_2 < \frac{1}{2}$ holds with probability $1 - o(1)$ as $n \rightarrow \infty$. 
%
%
%	Given $\EST_n$, we construct a detector $\DET_n: \EST \times (\mathbb{R}^{d_n})^n \times \mathbb{R} \to \{0,1\}$ of $\text{D-PSPCA}(n, k, d, \lambda)$, which maps a tuple of three components including an estimator function $\EST_n$, an instance $\bm{X}$ and a known eigengap $\lambda$ to  a detection algorithm that outputs one of the two hypotheses; this is presented in Algorithm~\ref{alg:D-from-E}. %\apcomment{Add unit normalization of $\widehat{v}$.} 
%%	\gwcomment{DONE. We cannot simply add unit normalization. Since $\| \widehat{\bm{v}} - \bm{v}_* \|_2 \leq \frac{1}{2}$ does not imply $\| \widehat{\bm{v}} / \|\widehat{\bm{v}} \|_2 - \bm{v}_* \|_2 \leq \frac{1}{2}$ when $\|\widehat{\bm{v}} \|_2 < 1$ and $\bm{v}_* \in \mathcal{S}^{d_n} \cap \mathcal{P}^k$. Modify the definition of qualified estimator due to this issue.}
%	
%%	based on $\EST_n$ for the detection problem of path sparse PCA with unit signals $\bm{v}_* \in \{0, 1/\sqrt{k}\}^d$ as presented in Algorithm~\ref{alg:D-from-E}. 
%
%\begin{algorithm}
%\caption{Detector $\DET_n$ of path sparse PCA}
%\label{alg:D-from-E}
%\textbf{Input:} A function $\EST_n$, an instance $\bm{X}$ drawn from $\text{D-PSPCA}(n,k,d,\lambda)$, and an eigengap $\lambda$.  
%\begin{algorithmic}[1]
%\State Compute the estimation $\widehat{\bm{v}} := \EST_n(\bm{X})$ of path sparse PCA. \label{alg:D-from-E-1}
%\State Normalize the estimation 
%	\begin{align*}
%		\tilde{\bm{v}} := \left\{
%		\begin{array}{lll}
%			\widehat{\bm{v}} / \|\widehat{\bm{v}}\|_2 & \text{ if } \widehat{\bm{v}} \neq \bm{0}_{d_n} \\
%			\text{any point in $\mathcal{S}^{d_n} \cap \mathcal{P}^k$} & \text{ if } \widehat{\bm{v}} = \bm{0}_{d_n}
%		\end{array}
%		\right. .
%	\end{align*} \label{alg:D-from-E-2}
%\State Set the sample covariance matrix $\widehat{\bm{\Sigma}} := \frac{1}{n} \bm{X} \bm{X}^{\top}$. \label{alg:D-from-E-3}
%\If{$\tilde{\bm{v}}^{\top} \widehat{\bm{\Sigma}} \tilde{\bm{v}} \geq 1 + \lambda/4 - (1 + \lambda) \sqrt{k \ln d / n}$} \label{alg:D-from-E-4}
%\State \textbf{Output:} $1$, i.e., $\bm{X}$ is from $H_1$.
%\Else 
%\State \textbf{Output:} $0$, i.e., $\bm{X}$ is from $H_0$.
%\EndIf
%\end{algorithmic}
%\end{algorithm} 
%
%	The detector $\DET_n$ clearly runs in polynomial time. Next, we show that for any instance $\bm{X}$ of $\text{D-PSPCA}(n, k, d, \lambda)$ and a known eigengap $\lambda$ (satisfying the conditions required in Theorem~\ref{thm:convergence} and Theorem~\ref{thm:initialization-method}), $\DET_n$ satifies
%	\begin{align} \label{eq:key-claim}
%		\liminf_{n \rightarrow \infty} \left( \mathbb{P}_{H_0}[\DET_n(\bm{X}) = 1] + \mathbb{P}_{H_1}[\DET_n(\bm{X}) = 0] \right) < 1. 
%	\end{align}
%
%%	maps an instance $\bm{X}$ to an estimation $\widehat{\bm{v}} \in \mathcal{S}^{d - 1} \cap \mathcal{P}^k$ such that $\|\bm{v}_* - \widehat{\bm{v}}\|_2 < \frac{1}{2}$ holds with probability greater than $\frac{1}{2} + \epsilon$ for any $0 < \epsilon < 1/2$. \footnote{Recall that our proposed projected power method for path sparse PCA ensures estimation task with probability at least $1 - \exp(- ck) > \frac{1}{2} + \epsilon$.}.
%		
%%		Set the sample covariance matrix $\widehat{\bm{\Sigma}} := \frac{1}{n} \bm{X} \bm{X}^{\top}$. If $\widehat{\bm{v}}^{\top} \widehat{\bm{\Sigma}} \widehat{\bm{v}} \geq 1 + \lambda/4 - (1 + \lambda) \sqrt{k \ln d / n}$, then $\DET_n(\bm{X}) = 1$, i.e., $\bm{X}$ is generated from hypothesis $H_1$. Otherwise, we have $\mathcal{DE}_n(\bm{X}) = 0$, i.e., $\bm{X}$ is generated from null hypothesis $H_0$. 
%		
%		To establish claim~\eqref{eq:key-claim}, note that the ``if criteria'' (Step-\eqref{alg:D-from-E-3} in Algorithm~\ref{alg:D-from-E}) is based on the following two-part calculation. 
% On the one hand, suppose the instance $\bm{X}$ is drawn from $\mathcal{D}(\lambda; \bm{v}_*)^{\otimes n}$ for $\bm{v}_* \in \mathcal{S}^{d - 1} \cap \mathcal{P}^k$. Consider the orthogonal decomposition of $\tilde{\bm{v}}$ given by $\tilde{\bm{v}} = \alpha \bm{v}_* + \beta \bm{v}_{\perp}$ with $\|\bm{v}_*\|_2 = \|\bm{v}_{\perp}\|_2 = 1, ~ \langle \bm{v}_*, \bm{v}_{\perp} \rangle = 0, \alpha^2 + \beta^2 = 1$. Based on the definition of the qualified estimator, the normalized vector $\|\tilde{\bm{v}} - \bm{v}_*\|_2$ satisfies $\| \widehat{\bm{v}} / \|\widehat{\bm{v}} \|_2 - \bm{v}_* \|_2 < \sqrt{2 - 2\sqrt{15/16}} < 1/2$. Consequently, we have $\alpha > 1/2$, and therefore the objective satisfies
%		\begin{align*}
%			\tilde{\bm{v}}^{\top} \widehat{\bm{\Sigma}} \tilde{\bm{v}} = & ~ (\alpha \bm{v}_* + \beta \bm{v}_{\perp})^{\top} \widehat{\bm{\Sigma}}  (\alpha \bm{v}_* + \beta \bm{v}_{\perp}) \\
%			= & ~ \alpha^2 \bm{v}_*^{\top} \bm{\Sigma} \bm{v}_* + \beta^2 \bm{v}_{\perp}^{\top} \bm{\Sigma} \bm{v}_{\perp} + 2 \alpha \beta \bm{v}_{*}^{\top} \bm{\Sigma} \bm{v}_{\perp} + \widehat{\bm{v}}^{\top} \bm{W} \widehat{\bm{v}} \\
%			\geq & ~ \alpha^2 (1 + \lambda) + \beta^2 + 0 + \rho(\bm{W}, P^*) \\
%			\geq & ~ 1 + \frac{\lambda}{4} - (1 + \lambda) \sqrt{k \ln d / n}, 
%		\end{align*}
%		where the final inequality holds due to $\rho(\bm{W}, P^*) \leq (1 + \lambda) \sqrt{k \ln d / n}$ with probability $1 - \exp(- ck)$.  
%		On the other hand, if the instance $\bm{X}$ is drawn from $\mathcal{D}(0; \bm{0}_d)^{\otimes n}$, then the objective satisfies
%		\begin{align*}
%			\tilde{\bm{v}}^{\top} \widehat{\bm{\Sigma}} \tilde{\bm{v}} = & ~ \tilde{\bm{v}}^{\top} (\bm{I}_d + \bm{W}) \tilde{\bm{v}} \\
%			\leq & ~ 1 + \rho(\bm{W}, P^*) \\
%			\leq & ~ 1 + (1 + \lambda) \sqrt{k \ln d / n} \\
%			< & ~ 1 + \frac{\lambda}{4} - (1 + \lambda) \sqrt{k \ln d / n},
%		\end{align*}
%		where the final strict inequality holds in the parameter regime of interest~\eqref{eq:regime-of-int}. Assuming that $n$ is sufficiently large and combining the above two cases implies 
%		\begin{align*}
%			\mathbb{P}_{H_0}[\DET_n(\bm{X}) = 1] \leq & ~ \frac{1}{2} - \epsilon + \exp(- ck), \\
%			\mathbb{P}_{H_1}[\DET_n(\bm{X}) = 0] \leq & ~ \frac{1}{2} - \epsilon + \exp(- ck), 
%		\end{align*} 
%		and therefore, 
%		\begin{align*}
%			\liminf_{n \rightarrow \infty} \left( \mathbb{P}_{H_0}[\DET_n(\bm{X}) = 1] + \mathbb{P}_{H_1}[\DET_n(\bm{X}) = 0] \right) < 1. 
%		\end{align*}
%		%For computational complexity, since Step-\eqref{alg:D-from-E-1} -- computing the estimation $\widehat{\bm{v}}$, Step-\eqref{alg:D-from-E-2} -- normalizing $\widehat{\bm{v}}$ for $\tilde{\bm{v}}$, Step-\eqref{alg:D-from-E-3} -- setting the sample covariance matrix $\widehat{\bm{\Sigma}}$, and Step-\eqref{alg:D-from-E-4} -- computing objective value $\tilde{\bm{v}}^{\top} \widehat{\bm{\Sigma}} \tilde{\bm{v}}$, take polynomial running time by our assumption, the detector $\DET_n$ is still a randomized polynomial-time algorithm. 
%	
%	\item \textbf{Reduction from $K$-Partite PC detection problem to path sparse PCA detection problem.} 
%		
%%		\gwcomment{SPCA-RECOVERY method \cite{brennan2018reducibility}: 
%%		\begin{itemize}
%%			\item BC-Recovery: maps $\textbf{PDS}_R(n, k, 1/2 + \rho, 1/2)$ to $\text{BC}_R(n,k,\mu)$ with $\mu = \frac{\log(1 + 2 \rho)}{2\sqrt{6 \log n + 2 \log 2}}$ (structure preserving)
%%			\item Random-Rotation: maps $\text{BC}_R(n,k,\mu)$ to $\text{UBSPCA}(n, k, d, \theta)$ with $d = n$ and $\theta = \frac{k^2 \mu^2}{\tau n}$. 
%%		\end{itemize}
%%		When $\rho = 1/2$, we have $\textbf{PDS}_R(n, k, 1, 1/2) = \textbf{PC}_R(n, k, 1/2)$. Instance from the recovery problem can be used for the detection problem, and vice versa.
%%		}
%		
%		%As a brief summary for this part, 
%		In this step, we use the average-case reduction method -- SPCA-RECOVERY [Figure 19, \citep{brennan2018reducibility}] to be our reduction method. Recall \cite{brennan2018reducibility} show that SPCA-RECOVERY maps an instance of planted clique problem $G \sim \text{PC}(n,k,1/2)$ to an instance $\bm{X}$ of vanilla sparse PCA detection problem (with ground truth $\bm{v}_*$ takes values in discrete set $\{0, 1 / \sqrt{k}\}$) approximately under total variance distance. Note that SPCA-RECOVERY ensures a preserving property (See Remark~\ref{remark:str-pre} for details). In words, SPCA-RECOVERY maps an instance $G \sim \text{K-PC}(n,k,1/2)$ to an instance $\bm{X} \sim \text{D-PSPCA}(n, k, d, \lambda)$ approximately under total variance distance while maintaining the structure, i.e., mapping rows associated with planted $k$-clique to rows associated with the corresponding path.
%
%		\begin{remark} \label{remark:str-pre}
%		\textbf{Structure preserving property of \textup{SPCA-RECOVERY}.} Given any $G \sim K\text{-PC}(n, k, 1/2)$ as an input instance of \textup{SPCA-RECOVERY}, \textup{SPCA-RECOVERY} maps the rows in the adjacency matrix $\bm{A}(G)$ concerning the planted $k$-clique of $G$ to the rows corresponding to the support set of the underlying path structure for the sample matrix (instance) $\bm{X} = [\bm{x}_1 ~|~ \cdots ~|~ \bm{x}_n]^{\top} \sim \text{D-PSPCA}(n, k, d, \lambda)$. 
%		\end{remark} 
%		
%%		\gwcomment{Proof for this part starts here... (1) define planted clique for a fixed clique structure (2) fix this clique, we have 2018 SPCA-RE maps a PC instance to a SPCA instance with corresponding support set}
%		Armed with these tools, we now present a quantitative analysis of the total variance distance that arises from our reductions. Let us start by recalling the parameter regime~\eqref{eq:regime-of-int} with some additional notation. Let $\beta \geq 1/2$. For path sparse PCA detection problem, define parameters 
%		\begin{align*}
%			k_n = \lceil n^{\beta} \rceil, \quad \rho_n = \frac{1}{2}, \quad, d_n = n, \quad \mu_n = \frac{\log 2}{2\sqrt{6 \log n + 2 \log 2}}, \quad \lambda_n = \frac{k_n^2}{\tau_n n} \cdot \frac{(\log 2)^2}{4 (6 \log n + 2 \log 2)},
%		\end{align*}
%		where $\tau_n \rightarrow \infty$ as $n \rightarrow \infty$, i.e., an arbitrarily slowly growing function of $n$. Let $\varphi_n = \text{SPCA-RECOVERY}$ be the reduction method. We use graph $G_n$ to denote an instance of $\text{K-PC}(n,k_n,1/2)$, and use $\bm{X}_n = \varphi_n (G_n)$ to be the output of $\text{SPCA-RECOVERY}$ with input $G_n$. To be concise, we use $\mathcal{L}(\bm{X}_n)$ to denote the distribution of a given instance $\bm{X}_n$.
%		
%		
%		Suppose $G_n \sim \mathcal{G}_{\mathcal{D}}(n, k_n, 1/2)$, is drawn from the $H_1$ hypothesis of $\text{K-PC}(n,k_n,1/2)$. Let $\bm{v}_*$ denote the unit vector supported on indices with respect to the clique in $G_n$ with nonzero entries equal to $1/\sqrt{k_n}$. Using\footnote{It is easy to observe that our parameter regime of $(n, \mu_n, \rho_n, \tau(n))$ satisfies the conditions presented in Lemma 6.7 and Lemma 8.2 in \cite{brennan2018reducibility}.} Lemma 6.7 and Lemma 8.2 in \cite{brennan2018reducibility}, we have
%		\begin{align*}
%			\text{d}_{\text{TV}} \left( \mathcal{L}(\bm{X}_n), \mathcal{D}(\lambda; \bm{v}_*) \right) \leq O\left( \frac{1}{\sqrt{\log n}} \right) + \frac{2 (n + 3)}{\tau n - n - 3} \rightarrow 0 \quad \text{as} \quad n \rightarrow \infty. 
%		\end{align*}
%		On the other hand, if an instance $G_n \sim \mathcal{G}_{\mathcal{D}}(n, 1/2)$, is drawn from the $H_0$ hypothesis of $\text{K-PC}(n,k_n,1/2)$. Still using Lemma 6.7 and Lemma 8.2 in \cite{brennan2018reducibility}, we also have 
%		\begin{align*}
%			\text{d}_{\text{TV}} \left( \mathcal{L}(\bm{X}_n), \mathcal{D}(0; \bm{0}_d) \right) \leq O\left( \frac{1}{\sqrt{\log n}} \right) + \frac{2 (n + 3)}{\tau n - n - 3} \rightarrow 0 \quad \text{as} \quad n \rightarrow \infty.
%		\end{align*}
%		Combining the above two cases ensures the reduction from $K$-Partite PC detection problem to path sparse PCA detection problem as we desired. 
% 		
% 		
% 		%		Due to the structure preserving property of \textup{SPCA-RECOVERY}, each part of the vertex set $V$ in graph $G \sim K\text{-PC}(n, k, 1/2)$ can be viewed as the vertex set of a layer in the layered graph with size $(n - 2) / k$. The planted $k$-clique (with one vertex in each part) in $G \sim K\text{-PC}(n, k, 1/2)$ corresponds to a path of length $k$ (excluding source and terminal) in the layered graph. 
%		
%	\item \textbf{Constructing a detection algorithm for $K$-Partite PC based on a detection algorithm for path sparse PCA.} 
%	
%		
%		Recall that there exists a sequence of detector algorithms $\DET_n$ that solves the detection problem of path-sparse PCA ($\text{D-PSPCA}(n, k, d, \lambda)$) as described above. Observe that under $H_1$ hypothesis,
%		\begin{align*}
%			\left|\mathbb{P}_{\bm{X} \sim \mathcal{L}(\bm{X}_n)}[\DET_n(\bm{X}) = 1] - \mathbb{P}_{\bm{X} \sim \mathcal{D}(\lambda; \bm{v}_*)}[\DET_n(\bm{X}) = 1] \right| \leq \text{d}_{\text{TV}} \left( \mathcal{L}(\bm{X}_n), \mathcal{D}(\lambda; \bm{v}_*) \right) \rightarrow 0 \quad \text{as } n \rightarrow \infty. 
%		\end{align*}
%		Since 
%		\begin{align*}
%			\mathbb{P}_{\bm{X} \sim \mathcal{D}(\lambda; \bm{v}_*)}[\DET_n(\bm{X}) = 1] = \mathbb{P}_{H_1}[\DET_n(\bm{X}) = 1] \geq \frac{1}{2} + \epsilon
%		\end{align*} 
%		for sufficiently large $n$ by the definition of the detector $\DET_n$, it follows that 
%		\begin{align*}
%			\mathbb{P}_{\bm{X} \sim \mathcal{L}(\bm{X}_n)}[\DET_n \circ \varphi_n(G_n) = 1] = \mathbb{P}_{\bm{X} \sim \mathcal{L}(\bm{X}_n)}[\DET_n (\bm{X}_n) = 1] \geq \frac{1}{2} + \frac{\epsilon}{2} 
%		\end{align*}
%		for sufficiently large $n$. On the other hand, under $H_0$ hypothesis, 
%		\begin{align*}
%			\left|\mathbb{P}_{\bm{X} \sim \mathcal{L}(\bm{X}_n)}[\DET_n(\bm{X}) = 0] - \mathbb{P}_{\bm{X} \sim \mathcal{D}(0; \bm{0}_d)}[\DET_n(\bm{X}) = 0] \right| \leq \text{d}_{\text{TV}} \left( \mathcal{L}(\bm{X}_n), \mathcal{D}(0; \bm{0}_d) \right) \rightarrow 0 \quad \text{as } n \rightarrow \infty. 
%		\end{align*}
%		Still 
%		\begin{align*}
%			\mathbb{P}_{\bm{X} \sim \mathcal{D}(0; \bm{0}_d)}[\DET_n(\bm{X}) = 0] = \mathbb{P}_{H_0}[\DET_n(\bm{X}) = 0] \geq \frac{1}{2} + \epsilon
%		\end{align*} 
%		holds for sufficiently large $n$ by the definition of the detector $\DET_n$. Then 
%		\begin{align*}
%			\mathbb{P}_{\bm{X} \sim \mathcal{L}(\bm{X}_n)}[\DET_n \circ \varphi_n(G_n) = 0] = \mathbb{P}_{\bm{X} \sim \mathcal{L}(\bm{X}_n)}[\DET_n (\bm{X}_n) = 0] \geq \frac{1}{2} + \frac{\epsilon}{2}. 
%		\end{align*}
%		Combining the above two results together implies
%		\begin{align*}
%			\liminf_{n \rightarrow \infty} \left( \mathbb{P}_{G_n \sim \mathcal{G}(n, 1/2)}[\DET_n \circ \varphi_n(G_n) = 1] + \mathbb{P}_{G_n \sim \mathcal{G}(n, k_n, 1/2)}[\DET_n \circ \varphi_n(G_n) = 0] \right) \leq 1 - \epsilon, 
%		\end{align*}
%		i.e., the detection method works.
%		
%		
%	\item \textbf{Putting together the pieces.}
%
%		Putting the above three parts together, we have used a polynomial-time qualified estimator to construct a sequence of functions $\DET_n \circ \varphi_n$ that detects K-Partite PC (Definition~\ref{def:KPC}) in polynomial time. This contradicts the K-Partite PC conjecture (Conjecture~\ref{conj:KPC-hardness}). Therefore, no such sequence of qualified estimator functions $\EST_n$ exists when the sequence of parameters $\{(k_n, d_n, \lambda_n, \tau_n)\}_{n \in \mathbb{N}}$ is in the proposed regime.
%		%, i.e., 
%	%\begin{align*}
%	%	k_n := \lceil c n^{1/2} \tau_n^{1/2} (\log n)^{1/2} \rceil, \quad d_n := n,\quad
%	%	\lambda_n := \frac{k_n^2}{\tau n} \cdot \mu_n^2 = \left\lceil \frac{c^2 (\log 2)^2}{24 + \frac{8\log 2}{\log n}} \right\rceil = \Theta \left( 1 \right),  
%	%\end{align*}
%	which completes the proof of the theorem.
%	%\footnote{Recall that, by letting $c \geq 12/\log 2$, we have the eigengap $\lambda \geq 5$, which could be further used to ensure the good region condition as required in Corollary~\ref{coro:initial-PS-PCA}.} %\apcomment{Explicitly have a point ``4. Putting together the steps''}
%	
%%		Assuming the Hardness Assumption for Planted Clique Problem [Definition~\ref{defn:PC}], the above reduction SPCA-RECOVERY ensures that there is no randomized polynomial-time algorithm solving $\text{UBSPCA}(k,n,n,\lambda)$ with $n \ll k^2$ when $\lambda = \tilde{\Theta}(1)$. Now in Algorithm , let $\text{Path-UBSPCA}(k + 2, n, d, \lambda)$ be a special case of UBSPCA problem with signal $\bm{v}_* \in \textup{UBS}_k \cap \mathcal{P}^k$, where $\mathcal{P}^k$ is the path structure set. The \textup{SPCA-RECOVERY} satisfies the following property. 
%		
%		
%	
%%		Recall that the sequence of detectors $\mathcal{DE}_n$ is still a sequence of randomized polynomial time algorithms for path sparse PCA. Consider any instance $G \sim K\text{-}PC(N, K, 1/2)$ and corresponding $\bm{X}$ obtained by the reduction \textup{SPCA-RECOVERY} with input $G$. Once 
%%		\begin{align*}
%%			\left\{ 
%%			\begin{array}{llll}
%%				\mathcal{DE}_n(\bm{X}) = 1 & \text{ A path sparse clique is detected,} \\
%%				\mathcal{DE}_n(\bm{X}) = 0 & \text{ No path sparse clique is detected.}
%%			\end{array}
%%			\right.
%%		\end{align*}
%%		Due to the structure preserving property of \textup{SPCA-RECOVERY}, a path sparsity signal corresponds with a planted clique in graph $G$. Thus $\mathcal{DE}_n(\bm{X}) = 1$ implies a planted clique of $G$ for $K$-Partite PC, and vice versa. Then we have
%%		\begin{align*}
%%			\mathcal{DE}_n \circ \textup{SPCA-RECOVERY}: \mathcal{G}_n \to \{0, 1\} 
%%		\end{align*}
%%		a randomized polynomial time detector for $K$-Partite PC detection problem, which contradicts to the Conjecture~\ref{conj:KPC-hardness}. \gwcomment{be formal as the conjecture 1 for K-PC.}
%%		
%%		
%%		Therefore, there is no randomized polynomial time algorithms for the recovery problem of path sparse PCA. 
%\end{enumerate}
%
%
%\end{proof}

































%!TEX root = IEEETSP-main-paper.tex

%\section{End-to-end analysis for specific examples} \label{sec:specific-examples}
%
%In this section, we provide an end-to-end analysis for tree-sparse and path-sparse PCA, including results on their information-theoretic limits of estimation as well as the performance of the projected power method when initialized using covariance thresholding. We complement these with some matching suggestions of computational hardness in these problems. 

\subsection{Tree-sparse PCA} \label{sec:TS-PCA}

\subsubsection{Fundamental limits for Tree-Sparse PCA} \label{sec:limits-TS-PCA}

Recall the notation $\mathcal{T}^k$ as the set of all rooted binary subtrees in the underlying \emph{complete binary tree} from Section~\ref{sec:TS-PCA-intro}. We write $\bm{v} \in \mathcal{T}^k$ if the support set of $\bm{v}$ satisfies $\mathsf{supp}(\bm{v}) \in \mathcal{T}^k$. Let
\begin{align}
    \widehat{\bm{v}}_{\mathsf{TS}} := \argmax_{\bm{v}} ~ \bm{v}^{\top} \widehat{\bm{\Sigma}} \bm{v} ~~\text{s.t.}~~ \bm{v} \in \mathcal{S}^{d - 1} \cap \mathcal{T}^k \label{eq:TS-PCA-est}
\end{align} 
denote the estimator obtained from exhaustive search. 
%Define $T^{*} := \argmax_F \rho(\bm{W}, {F}) \quad \text{s.t.}\quad F = \text{conv}(L_{m_1} \cup L_{m_2} \cup L_{m_3}), ~ \text{for all} ~ L_{m_1} \neq L_{m_2} \neq L_{m_3} \in \mathcal{T}^k.$ \mqcomment{We seems didn't use the definition of $T^{*}$ in the main text.}

\begin{corollary}\label{coro:TS-PCA-fund-limits}
%Let $\mathcal{T}^k$ be defined as in Section~\ref{sec:TS-PCA-intro}. 
There exists a pair of positive constants $(c, C)$ such that the following holds. \\
\noindent (a) Without loss of generality, suppose $\langle \bm{v}_*, \widehat{\bm{v}}_{\mathsf{TS}} \rangle \geq 0$. Then for any $c_1 > 0$ and $\bm{v}_* \in \mathcal{S}^{d - 1} \cap \mathcal{T}^k$, we have
\[  \big\|\widehat{\bm{v}}_{\mathsf{TS}} - \bm{v}_*\big\|_2 \leq  C \left( \frac{1 + \lambda}{\lambda}\right) \cdot \sqrt{\frac{(3 + \ln 2 + c_1)k}{n}}
\]
with probability at least $1 - 2\exp(- c_1k)$. \\
\noindent (b) We have the minimax lower bound 
\begin{align*}
&\inf_{\widehat{\bm{v}}} \; \sup_{\bm{v}_* \in \mathcal{S}^{d - 1} \cap \mathcal{T}^k} \mathbb{E}\left[ \left\| \widehat{\bm{v}} \widehat{\bm{v}}^{\top} - \bm{v}_* \bm{v}_*^{\top}  \right\|_F \right] \geq c \cdot  
    \min \Bigg\{\frac{1}{4\sqrt{\log k}}, ~~ \frac{1}{4}\sqrt{\frac{1 + \lambda}{8 \lambda^2}} \sqrt{\frac{k/\log k}{n}} \Bigg\}.
\end{align*}
Here, the infimum is taken over all measurable functions of the observations $\{ \bm{x}_i \}_{i = 1}^n$ drawn i.i.d. from the distribution $\mathcal{D}(\lambda; \bm{v}_*)$.
\end{corollary}
Corollary~\ref{coro:TS-PCA-fund-limits} is proved in Section~\ref{proof-coro-TS-PCA-fund-limits}. The term
%
%Corollary~\ref{coro:TS-PCA-fund-limits}(a) evaluates the term 
$\sqrt{k/n}$ arises from evaluating the cardinality of the set  $\mathcal{T}^k$ in tree-sparse PCA. In particular, 
%this term is of the order $O(\sqrt{k/n})$, since the size of 
we have $|\mathcal{T}^k| \leq (2e)^k / (k + 1)$  \citep{baraniuk2010model}, and taking logarithms results in a logarithmic factor gain over vanilla sparse PCA. Corollary~\ref{coro:TS-PCA-fund-limits}(b) provides a minimax lower bound of $\Omega(\sqrt{k / (n \log k)})$ for tree-sparse PCA, which has a logarithm gap $\sqrt{1/\log k}$ compared with the upper bound in Corollary~\ref{coro:TS-PCA-fund-limits}(a). This gap is small for small $k$, but we conjecture that it can be eliminated.

\begin{remark} \label{remark:logd-saving-TSPCA}
Compared with the fundamental limits for vanilla sparse PCA, the upper bounds for tree-sparse PCA in Corollary~\ref{coro:TS-PCA-fund-limits} save a factor $\log d$, which parallels the 
%Such a difference is obtained due to tighter upper bounds for the noise $\rho(\bm{W}, \mathcal{T}^k)$ and term $\max_{\bm{z} \in \mathcal{Z}_{*}}|\mathcal{N}_H(\bm{z}; k/2)|$ (see Theorem~\ref{thm:fund-limits}) in the tree-sparse case. In particular, we would like to point out that the tree sparsity structure can at most reduce the sample complexity/rate by an $O(\log d)$ factor, which parallels the 
model-based compressed sensing literature. The saving could be significant in practice when $d$ is large (see Figures~\ref{figure-tree-sparse} to follow)---indeed, this is one of the successes behind model-based compressive sensing. 
\end{remark}

% Figure environment removed


% \apcomment{At this point, we need a single figure with two subfigures. Condense the current text into a caption for the main figure. The subfigures will then be (a) and (b) of the big figure. They will have their own captions.}

% This section aims to demonstrate the efficiency and advantages {\color{blue}(a saving of the logarithm term $\log d$, see Remark~\ref{remark:logd-saving-TSPCA})} of using exact tree sparsity projection by comparing the performances of proposed methods (i.e., initialize with Algorithm~\ref{alg:initialization} -- Covariance Thresholding, followed by Algorithm~\ref{alg:PPM} -- Projected Power Method) with or without using exact tree sparsity projection. 

% We begin with the settings and procedures of our simulation experiments. Given the sample dimension $d = 2^L - 1$, sparsity $k$, and eigengap $\lambda$, we choose a particular tree sparsity support set $T_* \in \mathcal{T}^k$ and set the ground truth vector $\bm{v}_*$ as $ [\bm{v}_*]_i = \pm \frac{1}{\sqrt{k}}$ if $i \in T_*$ and $[\bm{v}_*]_i = 0$ if $i \notin T_*$. For each tuple of $(d, k, \lambda)$ and a pre-determined number of samples $n$, we generate 50 independent sample sets $\bm{X}^{(1)}, \ldots, \bm{X}^{(50)}$. Each sample set $\bm{X}^{(\ell)}$ has $n$ i.i.d. samples from the distribution $\mathcal{D}(\lambda, \bm{v}_*)$ based on the Wishart model as mentioned in Section~\ref{sec:setting-background}. That is to say, $\bm{X}^{(1)} := \left\{ \bm{x}_1^{(1)}, \ldots, \bm{x}_n^{(1)} \right\} \sim \mathcal{D}^{n}(\lambda, \bm{v}_*), \ldots, \bm{X}^{(50)} := \left\{ \bm{x}_1^{(50)}, \ldots, \bm{x}_n^{(50)} \right\} \sim \mathcal{D}^{n}(\lambda, \bm{v}_*)$. We then run Algorithm~\ref{alg:initialization} as our initialization method, followed by Algorithm~\ref{alg:PPM} based on every sample set. 

% The numerical comparison is between whether an exact tree sparsity projection or a classical $k$-sparsity projection is used in both Algorithm~\ref{alg:initialization} and Algorithm~\ref{alg:PPM}. The performance is measured by two metrics. The first metric is the point distance $\|\bm{v}_T - \bm{v}_*\|_2$ of the final iteration as we discussed in Theorem~\ref{thm:convergence}. Moreover, for people of independent interest, we further consider the support recovery property of the compared two methods using the success probability of support recovery, i.e., $\mathbb{P} \left[ \widehat{T}^{\mathcal{A}}(\bm{X}) = T_* \right]$, where $\widehat{T}^{\mathcal{A}}(\bm{X})$ denotes the support set obtained by the proposed methods $\mathcal{A}$ based on sample set $\bm{X}$. To be precise, we have $\mathcal{A} \in \{$Methods with k-sparse proj, Methods with tree-sparse proj$\}$ and $\bm{X} \in \left\{\bm{X}^{(1)}, \ldots, \bm{X}^{(50)} \right\}$. Therefore, given a fixed set of parameters (i.e., $(d, k, \lambda, n)$), the empirical success probability for performance measure can be computed by $\mathbb{P} \left[ \widehat{T}^{\mathcal{A}}(\bm{X}) = T_* \right] = \frac{1}{50} \sum_{\ell = 1}^{50} \bm{1}(\widehat{T}^{\mathcal{A}}(\bm{X}^{(\ell)}) = T_*).$ The numerical results are listed in Figure~\ref{fig:comparison-point-dist} and Figure~\ref{fig:comparison-prob-supp}. It is easy to observe that methods using tree-sparse projection outperform methods without using tree-sparse projection under all parameter settings we proposed. 

%This gap is due to the challenge of obtaining a tight upper bound for the term $\max_{\bm{z} \in \mathcal{Z}_{*}}|\mathcal{N}_H(\bm{z}; k/2)|$ (see Theorem~\ref{thm:fund-limits}) in the tree-sparse case, and we believe that a tighter bound would close such a gap.

\subsubsection{Local convergence and initialization} \label{sec:initialization-TS-PCA}

%Note that both the projected power method and initialization algorithm require an exact projection oracle.

%{\color{orange}
%Here, we propose the results of tractable exact projection onto $\mathcal{T}^k$, local geometric convergence for projected power method, and initialization method of tree sparse PCA. 
\paragraph{Exact projection oracle} We use the projection method proposed in \cite{cartis2013exact} as our tractable exact projection oracle $\Pi_{\mathcal{T}^k}$ for tree sparse PCA. This oracle has running time $O(kd)$. 
With our projection oracle in hand, we can now state our corollaries for the projected power method for tree sparse PCA.

%\gwcomment{Put projection oracle first.}

\begin{corollary} \label{coro:TS-PCA-PPM}
 Suppose in Algorithm~\ref{alg:PPM} that the initialization $\bm{v}_{0} \in \mathcal{T}^{k} \cap \mathcal{S}^{d-1}$ satisfies $\langle \bm{v}_0,\bm{v}_* \rangle \geq 1/2$. There exists a tuple of universal positive constants $(c,C_1,C_2, C_3)$ such that for $\lambda \geq C_1$, $n\geq C_2k$ and all $t \geq 1$, the iterate $\bm{v}_t$ from Algorithm~\ref{alg:PPM-project} satisfies 
 \[ 
 \| \bm{v}_{t} - \bm{v}_{*} \|_2 \leq \frac{1}{2^t} \cdot \| \bm{v}_{0} - \bm{v}_{*} \|_2 + C_3 \sqrt{\frac{k}{n}},
 \]
 with probability at least $1-\exp(- ck)$. 
%\apcomment{Is this for all $t$ or for each $t$? From discussion below it seems like a statement uniformly for all $t$. Please state properly.} \gwcomment{The above conditions holds for all $t$. The high probability condition is used to control $\rho(\mathcal{M}, F)$, which is decided when $\hat{\bm{\Sigma}}$ is given.
\end{corollary}
Corollary~\ref{coro:TS-PCA-PPM} is proved in Section~\ref{proof-coro-TS-PCA-PPM} from Theorem~\ref{thm:convergence}. 
%and by showing that
%\begin{align*}
%    \rho(\bm{W}, T^*) \leq (\lambda+1) \sqrt{\frac{(2 + \ln 2 + c)k}{n}} 
%\end{align*}
%holds with probability at least $1 - \exp(- ck)$, where $T^*$ above is defined similar to $P^*$ with respect to the tree sparsity set $\mathcal{T}^k$. \apcomment{This last sentence is weird. Improve definition of $F^*$, maybe by calling it $T^*$?} \gwcomment{Done}
%under current tree-sparse example.} {\color{orange}
%Based on $\Pi_{\mathcal{T}^k}$, we also have the following corollary of initialization method.}
We can also use the exact projection oracle $\Pi_{\mathcal{T}^k}$ to obtain the following corollary for our initialization method.
%\apcomment{Switch order. Local convergence first. Maintain parallel flow with main results.}\gwcomment{TODO.}  


\begin{corollary} \label{coro:initial-TS-PCA}
Assume $k^2 \leq d / e$. There exists a pair of universal positive constants $(C, C')$ such that if  $n\geq \max\{C\log d,k^{2}\}$ and $n \geq C'\max \big\{1,\lambda^{-2}\big\} \log(d/k^2) k^2,$ then Algorithm~\ref{alg:initialization} returns an initial vector $\bm{v}_0 \in \mathcal{S}^{d - 1} \cap \mathcal{T}^k$ satisfying $\langle \bm{v}_0, \bm{v}_* \rangle \geq 1/2$ with probability $1 - C'\exp(- \min\{\sqrt{d}, n\}/C')$. %$ for some positive constant $C'$}.
\end{corollary} 
Like Corollary~\ref{coro:initial-PS-PCA}, it is straightforward to see that Corollary~\ref{coro:initial-TS-PCA} follows from Theorem~\ref{thm:initialization-method} by specifying $c_0 = 1/2$. 

% We begin with the settings and procedures of our simulation experiments. Given the sample dimension $d = 2^L - 1$, sparsity $k$, and eigengap $\lambda$, we choose a particular tree sparsity support set $T_* \in \mathcal{T}^k$ and set the ground truth vector $\bm{v}_*$ as $ [\bm{v}_*]_i = \pm \frac{1}{\sqrt{k}}$ if $i \in T_*$ and $[\bm{v}_*]_i = 0$ if $i \notin T_*$. For each tuple of $(d, k, \lambda)$ and a pre-determined number of samples $n$, we generate 50 independent sample sets $\bm{X}^{(1)}, \ldots, \bm{X}^{(50)}$. Each sample set $\bm{X}^{(\ell)}$ has $n$ i.i.d. samples from the distribution $\mathcal{D}(\lambda, \bm{v}_*)$ based on the Wishart model as mentioned in Section~\ref{sec:setting-background}. That is to say, $\bm{X}^{(1)} := \left\{ \bm{x}_1^{(1)}, \ldots, \bm{x}_n^{(1)} \right\} \sim \mathcal{D}^{n}(\lambda, \bm{v}_*), \ldots, \bm{X}^{(50)} := \left\{ \bm{x}_1^{(50)}, \ldots, \bm{x}_n^{(50)} \right\} \sim \mathcal{D}^{n}(\lambda, \bm{v}_*)$. We then run Algorithm~\ref{alg:initialization} as our initialization method, followed by Algorithm~\ref{alg:PPM} based on every sample set. 

% The numerical comparison is between whether an exact tree sparsity projection or a classical $k$-sparsity projection is used in both Algorithm~\ref{alg:initialization} and Algorithm~\ref{alg:PPM}. The performance is measured by two metrics. The first metric is the point distance $\|\bm{v}_T - \bm{v}_*\|_2$ of the final iteration as we discussed in Theorem~\ref{thm:convergence}. Moreover, for people of independent interest, we further consider the support recovery property of the compared two methods using the success probability of support recovery, i.e., $\mathbb{P} \left[ \widehat{T}^{\mathcal{A}}(\bm{X}) = T_* \right]$, where $\widehat{T}^{\mathcal{A}}(\bm{X})$ denotes the support set obtained by the proposed methods $\mathcal{A}$ based on sample set $\bm{X}$. To be precise, we have $\mathcal{A} \in \{$Methods with k-sparse proj, Methods with tree-sparse proj$\}$ and $\bm{X} \in \left\{\bm{X}^{(1)}, \ldots, \bm{X}^{(50)} \right\}$. Therefore, given a fixed set of parameters (i.e., $(d, k, \lambda, n)$), the empirical success probability for performance measure can be computed by $\mathbb{P} \left[ \widehat{T}^{\mathcal{A}}(\bm{X}) = T_* \right] = \frac{1}{50} \sum_{\ell = 1}^{50} \bm{1}(\widehat{T}^{\mathcal{A}}(\bm{X}^{(\ell)}) = T_*).$ The numerical results are listed in Figure~\ref{fig:comparison-point-dist} and Figure~\ref{fig:comparison-prob-supp}. It is easy to observe that methods using tree-sparse projection outperform methods without using tree-sparse projection under all parameter settings we proposed. 



% \begin{subfigure}
% \centering
    % % Figure removed
    % % Figure removed
    % % Figure removed
    % \caption{Plot of the value of point distances $\|\bm{v}_T - \bm{v}_*\|_2$
     % verse the number of samples $n = \{20, 40, \ldots, 200\}$. For all three panels, we set $\lambda = 3$. We set $(d, k) = (255, 9)$, $(d, k) = (511, 10)$ and $(d, k) = (1023, 13)$ respectively for the three panels. The two curves in each panel correspond to the averaged values of point distance of the proposed methods with or without using exact tree sparse projection; the shaded parts represent the empirical standard deviations computed from 50 repetitions for each dot/cross in the panel, respectively. As we can observe, the point distances for the methods using tree-sparse projection are always smaller than the methods without tree-sparse projection, which demonstrate the efficient of using tree sparse projection. 
    % }\label{fig:comparison-point-dist}
% \end{subfigure}
% \hfill
% \begin{subfigure}
    % \centering
    % % Figure removed
    % % Figure removed
    % % Figure removed
    % \caption{Plot of the success probability $\mathbb{P} \left[ \widehat{T}^{\mathcal{A}}(\bm{X}) = T_* \right]$ 
    % verse the number of samples $n = \{20, 40, \ldots, 200\}$. We set $(d, k) = (255, 9)$, $(d, k) = (511, 10)$ and $(d, k) = (1023, 13)$ respectively for the three panels. The two curves in each panel correspond to the empirical probability of success of the proposed methods with or without using exact tree sparse projection. It is easy to observe that the phase transition for the methods using tree-sparse projection is earlier (with fewer samples) than the methods without tree-sparse projection.
    % }
    % \label{fig:comparison-prob-supp}
% \end{subfigure}






Corollary~\ref{coro:initial-TS-PCA} shows that provided $n = \Omega(k^{2})$, the output $\bm{v}_0 \in \mathcal{S}^{d - 1} \cap \mathcal{T}^k$ satisfies the initialization condition required for the subsequent projected power method to succeed. Putting these two results together, we have produced an end-to-end and computationally efficient algorithm that produces a statistically efficient solution provided $n = \Omega(k^{2})$. The next section is concerned with the question of whether the condition $n = \Omega(k^{2})$ is necessary for polynomial-time algorithms. %applied for tree-sparse PCA with  samples.  \apcomment{Following discussion should be used to motivate lower bound. Changing order of local/init will help.} \gwcomment{Move to the first paragraph of Section 4.1.3.}
%Although, the number of samples $O(k^2)$ required in Corollary~\ref{coro:initial-TS-PCA} is greater than the minimax lower bound proposed in Corollary~\ref{coro:TS-PCA-fund-limits}(b), Section~\ref{sec:examples-SDP-hard} shows that $O(k^2)$ is sharp up to a logarithm term of SDP hardness.


\subsubsection{SDP Hardness for Tree Sparse PCA.} \label{sec:examples-SDP-hard}


To understand the aforementioned gap in sample size, we now provide a computational lower bound for a class of SDP solutions to tree-sparse PCA, showing that they require on the order of $k^2$ samples.

To make things formal, we consider the following subclass of tree sparse PCA problems: every entry of the $k$ tree-sparse ground truth unit vector $\bm{v}_*$ only takes one of the values $\{0, \pm k^{-1/2}\}$. With knowledge of this side information in addition to tree sparsity, the natural choice of exhaustive estimator is given by the maximizer of the following optimization problem:
\begin{align}\label{re-opt-tree-sparse}
	\begin{array}{rllll}
		\max_{\bm{v}} & \bm{v}^{\top} \widehat{\bm{\Sigma}} \bm{v} \\
		\text{s.t.} & \|\bm{v}\|_2^2 = 1,\;\|\bm{v}\|_0 = k \\
		& \bm{v}(i)^2 \leq \bm{v}(\lfloor i / 2 \rfloor)^2\;\text{ for all } 2\leq i \leq d. 
	\end{array}	
\end{align}
%\begin{align}\label{re-opt-tree-sparse}
%%\widehat{\bm{v}}_{\mathsf{TS-side}} := 
%\max_{\bm{v}} & ~ \;\; \bm{v}^{\top} \widehat{\bm{\Sigma}} \bm{v} \\
%\text{s.t.} & ~ \|\bm{v}\|_2^2 = 1,\;\|\bm{v}\|_0 = k \notag \\
%& ~ \bm{v}(i)^2 \leq \bm{v}(\lfloor i / 2 \rfloor)^2,\;\text{ for all } 2\leq i \leq d. \notag
%\end{align} 
%\apcomment{removed $\widehat{\bm{v}}_{\mathsf{TS-side}}$ definition. Define once again if needed in proof.} \gwcomment{Done. $\widehat{\bm{v}}_{\mathsf{TS-side}}$ is not used in the proof.}
%to denote the estimator of TS-side from exhaustive search. 
The natural semidefinite programming (SDP) relaxation of the program~\eqref{re-opt-tree-sparse} is then given by
\begin{align}\label{SDP-tree-sparse-PCA}
	\begin{array}{rllll}
		\mathsf{SDP}(\widehat{\bm{\Sigma}}) = \max_{\bm{M} \in \mathbb{R}^{d \times d}} & \sum_{i=1}^{d} \sum_{j=1}^{d} \widehat{\bm{\Sigma}}_{ij} \bm{M}_{ij} \\
		\text{s.t.} & \sum_{i=1}^{d} \bm{M}_{ii}^{2} = 1 \\
		& \sum_{i=1}^{d}\sum_{j=1}^{d} |\bm{M}_{ij}| \leq k \\
		& \bm{M} \succeq \bm{0}_{d \times d} \\
		& \bm{M}_{ii} \leq \bm{M}_{\lfloor i / 2 \rfloor\lfloor i / 2 \rfloor} \text{ for all }\; 2\leq i\leq d.
	\end{array}
\end{align}
%\begin{align}\label{SDP-tree-sparse-PCA}
%\begin{split}
%    \mathsf{SDP}(\widehat{\bm{\Sigma}}) = \max_{\bm{M} \in \mathbb{R}^{d \times d}} &\sum_{i=1}^{d} \sum_{j=1}^{d} \widehat{\bm{\Sigma}}_{ij} \bm{M}_{ij} 
%    \\  \text{s.t.} \quad &\sum_{i=1}^{d} \bm{M}_{ii}^{2} = 1,\; \sum_{i=1}^{d}\sum_{j=1}^{d} |\bm{M}_{ij}| \leq k \\
%    & \bm{M} \succeq \bm{0}_{d \times d}
%    \\ & \bm{M}_{ii} \leq \bm{M}_{\lfloor i / 2 \rfloor\lfloor i / 2 \rfloor} \; \text{ for all }\; 2\leq i\leq d. 
%\end{split}
%\end{align}

It is well-known that for vanilla sparse PCA, the SDP attains the best-known sample complexity among all polynomial time algorithms. Proving a lower bound for this class of algorithms is thus powerful---when this subclass of low-degree estimators fails at the indicated threshold, it suggests a natural hardness result.

\begin{proposition}\label{prop:TS-SDP-hard}
    Suppose data $\bm{X}$ are drawn from the distribution $\mathcal{D}(\lambda; \bm{v}_*)$ with ground truth $\bm{v}_*$ given by a $k$ tree-sparse unit vector with every entry of taking one of the values in the set $\{0, \pm k^{-1/2}\}$.
    There exists a tuple of universal positive constants $(c,c_1,C,C_1)$ such that for $c_1d \leq n \leq C_1 d,\; n \leq ck^{2}$ and $1 \leq \lambda \leq \frac{d}{Cn}$, the optimal solution $\bm{M}_*$ of the SDP relaxation~\eqref{SDP-tree-sparse-PCA} satisfies $\big\| \bm{M}_* - \bm{v}_{*}\bm{v}_{*}^{\top}\big\|_{2} \geq \frac{1}{5}$ with probability at least $1- \tilde{c} d^{-\tilde{c}}$ for some constant $\tilde{c} \geq 1$.
\end{proposition}


In words, Proposition~\ref{prop:TS-SDP-hard} shows that unless the number of samples satisfies $n \geq C' k^2$ for some positive constant $C' \geq c$, the optimal solution $\bm{M}_*$ of the SDP relaxation~\eqref{SDP-tree-sparse-PCA} fails to estimate the ground truth consistently, even with the side information that its entries take only one of three values. The proof of Proposition~\ref{prop:TS-SDP-hard} can be found in Section~\ref{proof-prop-TS-SDP-hard}, and builds on the techniques proposed in [Section 4, \cite{ma2015sum}]. 
%Moreover, in Section~\ref{proof-prop-TS-SDP-hard}, we show that when the number of samples $n \ll k^2$, the \emph{detection task} of SDP relaxation~\eqref{SDP-tree-sparse-PCA} for TS-side (and tree-sparse PCA) also fails.  
%In particular, we prove that the optimal solution $\bm{M}_*$ is not too close with the desired rank-1 matrix $\bm{v}_* \bm{v}_*^{\top}$. Proposition~\ref{prop:TS-SDP-hard} ensures that when the number of samples $n \ll k^2$, optimal solution $\bm{M}_*$ of SDP relaxation~\eqref{SDP-tree-sparse-PCA} for tree-sparse PCA fails to estimate the ground truth. 
%
%Moreover, for people with independent interests, in Section~\ref{proof-prop-TS-SDP-hard}, we show that when the number of samples $n \ll k^2$, the \emph{detection task} of SDP relaxation~\eqref{SDP-tree-sparse-PCA} for tree-sparse PCA also fails. 
%}


%\section{Numerical Experiments for Tree Sparsity} 
%\apcomment{Create a section (not subsection) called Numerical Experiments.} \gwcomment{I am wondering whether this section numerical experiment should be placed under the Section IV: end-to-end analysis for specific examples}







%%%%%%%%%%%%%%%%%%%%%%%%%%%%%%%%%%%%%%%%%%%%%%%%%%%%%%
%%%%%%%%%%%%%%%%%%%%%%%%%%%%%%%%%%%%%%%%%%%%%%%%%%%%%%
%%%%%%%%%%%%%%%%%%%%%%%%%%%%%%%%%%%%%%%%%%%%%%%%%%%%%%
%%%%%%%%%%%%%%%%%%%%%%%%%%%%%%%%%%%%%%%%%%%%%%%%%%%%%%
\iffalse
\subsubsection{SDP Hardness for Tree Sparse PCA.} \label{sec:examples-SDP-hard}

We use the following Assumption~\ref{assump:discrete-entries} to simplify the proof of SDP hardness. 

\begin{assumption} \label{assump:discrete-entries}
\textbf{Discrete Entries.} Suppose that $\bm{v}_*$ has discrete entries, namely that $[\bm{v}_*]_i \in \left\{0, \pm 1 / \sqrt{k} \right\}$ for all $i \in [d]$, where $k$ denotes the number of non-zero entries. 
\end{assumption} 

Based on Assumption~\ref{assump:discrete-entries}, we scale the variables $\bm{v}$ up by a factor of $\sqrt{k}$ for simplicity, i.e., each entry of $\bm{v}$ now takes value from $\left\{0, \pm 1 \right\}$. The \emph{re-scaled estimation} $\widehat{\bm{v}}_{\mathsf{TS}}'$ for tree-sparse PCA can be therefore formulated explicitly as follows: 
\begin{align}
    \widehat{\bm{v}}_{\mathsf{TS}}' := \argmax_{\bm{v}} ~  \frac{1}{k} \bm{v}^{\top} \widehat{\bm{\Sigma}} \bm{v} ~~\text{s.t.}~~ \|\bm{v}\|_2^2 = k, ~~ \bm{v} \in \mathcal{T}^k = \left\{ \bm{v} \left|
    \begin{array}{llll}
        \|\bm{v}\|_0 = k \\
        \bm{v}_{i}^2 \leq \bm{v}_{\lfloor i / 2 \rfloor}^2 ~~ \forall i \in \{2, \ldots, d\}
    \end{array} \right.
    \right\},  \label{eq:TS-PCA-est-2}
\end{align} 
where the constraint $\bm{v}_{i}^2 \leq \bm{v}_{\lfloor i / 2 \rfloor}^2, ~ \forall i \in \{2, \ldots, d\}$ ensures that $\text{supp}(\bm{v})$ represents the support set of a connected subtree of $\text{CBT}$ with root node $1$. 
Here, it is easy to observe that the original estimation $\widehat{\bm{v}}_{\mathsf{TS}}$ of tree-sparse PCA satisfies $\widehat{\bm{v}}_{\mathsf{TS}} = \widehat{\bm{v}}_{\mathsf{TS}}' / \sqrt{k}$ by re-scaling the factor $\sqrt{k}$ back. 

Recall the number of samples $n = O(k^2 / \lambda^2)$ required for the initialization method is greater than the minimax lower bound from Part (b) of Corollary~\ref{coro:TS-PCA-fund-limits}. In this part, we show that $O(k^2 / \lambda^2)$ samples required for the initialization method is almost ``tight'' in some sense. In particular, consider the SDP relaxation of the optimization for exhaustive search~\eqref{eq:TS-PCA-est-2} of the tree sparse PCA, 
\begin{align}
    \begin{array}{rllll}
        \text{SDP}(\widehat{\bm{\Sigma}}) := \max_{M \in \mathbb{R}^{d \times d}} & \frac{1}{k} \sum_{i,j = 1}^d \widehat{\bm{\Sigma}}_{ij} M(\bm{v}_i \bm{v}_j) \\
        \text{s.t.} & \sum_{i = 1}^d M(\bm{v}_i^2) = k \\
        & \sum_{i,j = 1}^d |M(\bm{v}_i\bm{v}_j)| \leq k^2 \\
        & M(\bm{v}_i^2) \leq M(\bm{v}_{\lfloor i / 2 \rfloor}^2) & \forall i \in \{2, \ldots, d\} \\
        & M \succeq \bm{0}_{d \times d} 
    \end{array},  \label{eq:SDP-tree-sparsity}
\end{align}
where $M: \mathbb{R}^d \times \mathbb{R}^d \mapsto \mathbb{R}$ can be viewed as a ``pseudo moment''. The Proposition~\ref{thm:TS-estimation} shows that the optimal solution $M^*$ of the SDP relaxation~\eqref{eq:SDP-tree-sparsity} fails to estimate the principal component $\bm{v}_*$ when the number of samples $n$ is smaller than the following lower bounds. 
 
% Based on the SDP relaxation~\eqref{eq:SDP-tree-sparsity}, the next proposition (Proposition~\ref{thm:TS-estimation}) shows that the estimation problem for the tree-sparse PCA fails when the number of samples $n$ is smaller than the corresponding lower bounds.
% The proof of the Proposition~\ref{thm:TS-estimation} uses the techniques proposed in [Section 4, \cite{ma2015sum}]. In particular, we prove that the optimal solution $M^*$ obtained by solving  \eqref{eq:SDP-tree-sparsity} is not too close with the desired rank-1 matrix $\bm{v}_* \bm{v}_*^{\top}$.  

% In particular, we construct a feasible ``pesudo-moment'' $M(\cdot)$ of the SDP relaxation~\eqref{eq:SDP-tree-sparsity} that: when samples $\{\bm{x}^i\}_{i = 1}^n$ are i.i.d. generated from the $H_0$, the $\text{SDP}(\widehat{\bm{\Sigma}}) \geq 1 + \lambda$ holds with high probability, i.e., the detection task fails (see Appendix~\ref{app:TS-PCA} for details). As a result, we could obtain a lower bound on the number of samples for the detection problem of this SDP relaxation. \gwcomment{1. Should we say the following two results as Propositions? 2. Should we remove the result about the \textit{detection} problem (i.e., previous paragraph defining the detection problem and Proposition 1) to Appendix?}

% For the estimation problem, we prove that the optimal solution $M^*$ obtained by solving  \eqref{eq:SDP-tree-sparsity} is not too close with the desired rank-1 matrix $\bm{v}_* \bm{v}_*^{\top}$. 

\begin{proposition}\label{thm:TS-estimation} 
For any constant $B$, there exists absolute constants $C$ and $r$ such that for $\lambda \leq B/2, ~ Bn \geq d \geq 2 \lambda n$ and $o(d) \geq k \geq C \sqrt{n} \log^r d$, suppose the data $\bm{X}$ is drawn from the distribution $\mathcal{D}(\lambda; \bm{v}_*)$ with $\bm{v}_* \in \mathcal{S}^{d - 1} \cap \mathcal{T}^k$, then with high probability $(1 - d^{-10})$ over the randomness of the data, the optimal solution $M^*$ satisfies $\|\frac{1}{k} M^* - \bm{v}_* \bm{v}_*^{\top}\|_2 \geq \frac{1}{5}$. 
\end{proposition}

The proof of Proposition~\ref{thm:TS-estimation} can be found in Appendix~\ref{app:TS-estimation}, which uses the techniques proposed in [Section 4, \cite{ma2015sum}]. In particular, we prove that the optimal solution $M^*$ is not too close with the desired rank-1 matrix $\bm{v}_* \bm{v}_*^{\top}$. Proposition~\ref{thm:TS-estimation} ensures that when the number of samples $n \ll k^2$, optimal solution $M^*$ of SDP relaxation~\eqref{eq:SDP-tree-sparsity} for tree-sparse PCA fails to estimate the ground truth. Moreover, for people with independent interests, in Appendix~\ref{app:TS-detection}, we show that when the number of samples $n \ll k^2$, the \emph{detection task} of SDP relaxation~\eqref{eq:SDP-tree-sparsity} for tree-sparse PCA also fails. 

\fi
%%%%%%%%%%%%%%%%%%%%%%%%%%%%%%%%%%%%%%%%%%%%%%%%%%%%%%
%%%%%%%%%%%%%%%%%%%%%%%%%%%%%%%%%%%%%%%%%%%%%%%%%%%%%%
%%%%%%%%%%%%%%%%%%%%%%%%%%%%%%%%%%%%%%%%%%%%%%%%%%%%%%
%%%%%%%%%%%%%%%%%%%%%%%%%%%%%%%%%%%%%%%%%%%%%%%%%%%%%%

\iffalse 
\subsection{Path-Sparse PCA} \label{sec:PS-PCA}

%\apcomment{Make changes to this section in parallel to suggested changes from the tree-sparse section.}

Recall the notation $\mathcal{P}^k$ as the structure set of path-sparse PCA from Section~\ref{sec:PS-PCA-intro}. We write $\bm{v} \in \mathcal{P}^k$ if the support set satisfies $\mathsf{supp}(\bm{v}) \in \mathcal{P}^k$. 


\subsubsection{Fundamental statistical limits}

Let
\begin{align}
    \widehat{\bm{v}}_{\textsf{PS}} := \argmax_{\bm{v}} ~ \bm{v}^{\top} \widehat{\bm{\Sigma}} \bm{v} ~~\text{s.t.}~~ \bm{v} \in \mathcal{S}^{d - 1} \cap \mathcal{P}^k \label{eq:PS-PCA-est}
\end{align}  
denote the corresponding estimate from exhaustive search.


\apcomment{I want the technical writing in this subsection and the next to look EXACTLY parallel to tree-sparse case. It currently does not. You have Proposition 2 instead of corollary, and there is impreciseness in result statements.}
\gwcomment{The following corollaries are revised.}

\begin{corollary}\label{coro:PS-PCA-fund-limits}
{\color{orange}Let $\mathcal{P}^k$ be defined as in Section~\ref{sec:PS-PCA-intro}. There is a pair of positive constants $(c, C)$ such that the following is true.

%Based on Theorem~\ref{thm:fund-limits}, suppose the union-of-linear structures condition in Definition~\ref{cond:linear-structure} holds.
\noindent (a) Without loss of generality, assume $\langle \bm{v}_*, \widehat{\bm{v}}_{\mathsf{PS}} \rangle \geq 1/2$. Then for any $c > 0$ and $\bm{v}_* \in \mathcal{S}^{d - 1} \cap \mathcal{P}^k$, we have
\begin{subequations}
{\color{orange}
\begin{align*}
    \big\|\widehat{\bm{v}}_{\mathsf{PS}} - \bm{v}_*\big\|_2 \leq C \left( \frac{1 + \lambda}{\lambda} \right) \sqrt{\frac{3 (\ln d - \ln k)k + ck}{n}} 
\end{align*}
with probability at least $1 - 2\exp(- ck)$.}

\noindent (b) Suppose that $d\geq 16 k^2$ and $k\geq 4$. Then we have the minimax lower bound
\begin{align*}
    \inf_{\widehat{\bm{v}}} \; \sup_{\bm{v}_* \in \mathcal{S}^{d - 1} \cap \mathcal{P}^k} \mathbb{E}\left[ \left\| \widehat{\bm{v}} \widehat{\bm{v}}^{\top} - \bm{v}_* \bm{v}_*^{\top}  \right\|_F \right] \geq  c \cdot \min\bigg\{1,  \sqrt{\frac{1 + \lambda}{8 \lambda^2}} \sqrt{\frac{k \cdot \big( \frac{\ln d}{2} - \ln k \big)}{n}} \bigg\}. 
\end{align*}
\end{subequations}
Here, the infimum is taken over all measurable functions of the observations $\{ \bm{x}_i \}_{i = 1}^n$ drawn i.i.d. from the distribution $\mathcal{D}(\lambda; \bm{v}_*)$.}
\end{corollary}

Corollary~\ref{coro:PS-PCA-fund-limits} is proved in Section~\ref{proof-coro-PS-PCA-fund-limits}. Corollary~\ref{coro:PS-PCA-fund-limits} studies fundamental limits for path-sparse PCA based on Theorem~\ref{thm:fund-limits}. In particular, Corollary~\ref{coro:PS-PCA-fund-limits}(a) gives an upper bound of the estimation $\widehat{\bm{v}}_{\textsf{PS}}$, where the statistical noise term $\rho(\bm{W}, F^*)$ is of the order $(\lambda + 1) \sqrt{k \cdot ( \ln d - \ln k)/n}$ under the path sparsity case. In comparison with existing result, the minimax lower bound obtained in Corollary~\ref{coro:PS-PCA-fund-limits}(b) is in the same order of the minimax lower bound given in [Theorem 1, \cite{asteris2015stay}] with the outer degree parameter $|\Gamma_{\text{out}}(v)| = (d - 2) / k$. 

\iffalse
Corollary~\ref{coro:PS-PCA-fund-limits} studies fundamental limits for path-sparse PCA based on Theorem~\ref{thm:fund-limits}. In particular, Corollary~\ref{coro:PS-PCA-fund-limits}(a) gives an upper bound of the estimation $\widehat{\bm{v}}_{\textsf{PS}}$, where the statistical noise term $\rho(\bm{W}, F^*)$ is of the order $\sqrt{(1 + \lambda) / 8 \lambda^2}\sqrt{k \ln d / n}$ under the path sparsity case. The proof of Corollary~\ref{coro:PS-PCA-fund-limits}(a) can be found in Appendix~\ref{app:ub-coro}. Corollary~\ref{coro:PS-PCA-fund-limits}(b) provides a minimax lower bound for the tree-sparse PCA under a mild condition $d \geq \max\left\{4k, ~ 2^{H(\xi)/\xi} k^{1/\xi} \right\}$. To illustrate the above claim, by setting $\xi = 7/8 \in (3/4, 1)$, we only request $d \geq \max\{4k, 2^{8H(7/8)/7} k^{8/7} \} = O(k^{8/7})$. In comparison with existing result, the minimax lower bound obtained in Corollary~\ref{coro:PS-PCA-fund-limits}(b) is in the same order of the minimax lower bound given in [Theorem 1, \cite{asteris2015stay}] with the outer degree parameter $|\Gamma_{\text{out}}(v)| = (d - 2) / k$. The proof of Corollary~\ref{coro:PS-PCA-fund-limits}(b) is given in Appendix~\ref{app:coro-minimax-lb}. 
\fi

\subsubsection{Initialization and local convergence} 

{\color{orange}Like Section~\ref{sec:TS-PCA}, we first establish the exact projection oracle.\\

\noindent \textbf{Exact projection oracle:} The exact projection oracle of path-sparse PCA $\Pi_{\mathcal{P}^k}$ is built by picking the component of the largest absolute value in each partition (layer) for a given $(d,k)$-layered graph $G$ (with running time $O(d)$). Using such projection oracle, we can now state our corollaries for the projected power method for path sparse PCA.}

\begin{corollary} \label{coro:PS-PCA-PPM}
 Suppose in Algorithm~\ref{alg:PPM-project} the initialization $\bm{v}_{0} \in \mathcal{T}^{k} \cap \mathcal{S}^{d-1}$ and $\langle \bm{v}_0,\bm{v}_* \rangle \geq 1/2$. There are universal positive constants $C_1,C_2,C_3$ and $c$ such that for $\lambda \geq C_1$, $n\geq C_2k\ln(d)$,  {\color{orange}and all $t \geq 1$, the iterate $\bm{v}_t$  in Algorithm~\ref{alg:PPM-project} satisfies }
% the iterates $\{\bm{v}_{t}\}_{t=0}^{T}$ in Algorithm~\ref{alg:PPM-project} satisfies
\begin{align*}
    \| \bm{v}_{t} - \bm{v}_{*} \|_2 \leq \frac{1}{2^t} \cdot \| \bm{v}_{0} - \bm{v}_{*} \|_2 +  C_3 \sqrt{\frac{k(2\ln d - \ln k)}{n}}
\end{align*}
with probability at least $1-\exp(-ck)$. 
\end{corollary}
Corollary~\ref{coro:PS-PCA-PPM} is proved in Section~\ref{proof-coro-PS-PCA-PPM}. It is proved by applying Theorem~\ref{thm:convergence} and showing
\begin{align*}
    \rho(\bm{W}, P^*) \lesssim  (1+\lambda)\sqrt{\frac{k\ln d + ck}{n}}
\end{align*}
holds with probability at least $1 - \exp(- ck)$, where $P^*$ above is defined similar to $F^*$ with respect to path sparsity set $\mathcal{P}^k$.

{\color{orange} 
We are now ready to propose the result of initialization for path sparse PCA using $\Pi_{\mathcal{P}^k}$.} 

%We set $\mathcal{M} = \mathcal{P}^k$ and use the exact projection oracle of path-sparse PCA $\Pi_{\mathcal{P}^k}$ \footnote{The exact projection $\Pi_{\mathcal{P}^k}$ can be established by picking the component of largest absolute value in each partition (layer) for a given $(d,k)$-layered graph $G$.} (with running time $O(d)$) in both Algorithm~\ref{alg:PPM-project} and Algorithm~\ref{alg:initialization}.
{\color{orange}
\begin{corollary} \label{coro:initial-PS-PCA}
Assume $k^2 \leq d / e$. There exists universal constant $(C, C')$ such that if $n\geq \max\{C\log d,k^{2}\}$ and 
\begin{align*}
    n \geq \frac{C'\max\{\lambda^2,1\} k^2}{\lambda^2} \log(d/k^2), 
\end{align*}
the initial vector $\bm{v}_0 \in \mathcal{S}^{d - 1} \cap \mathcal{P}^k$ obtained from Algorithm~\ref{alg:initialization} satisfies $\langle \bm{v}_0, \bm{v}_* \rangle \geq 1/2$ with probability $1 - C'\exp(- \min\{\sqrt{d}, n\}/C')$ for some positive constant $C'$.
\end{corollary}}

It is straightforward to see that Corollary~\ref{coro:initial-PS-PCA} follows from Theorem~\ref{thm:initialization-method} by specifying $c_0 = 0.5$. 
%The proof of Corollary~\ref{coro:initial-PS-PCA} is derived from Theorem~\ref{thm:convergence} in Appendix~\ref{app:initial-methods}. 
Like Corollary~\ref{coro:initial-TS-PCA}, Corollary~\ref{coro:initial-PS-PCA} provides an initialization method whose outputs can be used for the general projected power method (Algorithm~\ref{alg:PPM}) for path-sparse PCA when the number of samples $n \geq O(k^2)$. Still, there is a gap in sample complexity between the minimax lower bound and what we obtained in Corollary~\ref{coro:initial-PS-PCA}. In the next Section~\ref{sec:examples-average-hard}, we show no randomized polynomial-time algorithm solving path-sparse with $n \ll k^2$ assuming the average-case hardness of the planted clique problem and its variants. 
\fi 

%%%%%%%%%%%%%%%%%%%%%%%%%%%%%%%%%%%%%%%%%%%%%%%%%%%%%%%%%%%%%%%%%%%%%%%%%%%%%%%%%%%%%%%%%%%%%%
%%%%%%%%%%%%%%%%%%%%%%%%%%%%%%%%%%%%%%%%%%%%%%%%%%%%%%%%%%%%%%%%%%%%%%%%%%%%%%%%%%%%%%%%%%%%%%
%%%%%%%%%%%%%%%%%%%%%%%%%%%%%%%%%%%%%%%%%%%%%%%%%%%%%%%%%%%%%%%%%%%%%%%%%%%%%%%%%%%%%%%%%%%%%%



\iffalse
\subsubsection{Average-Case Hardness of Path Sparse PCA.}\label{sec:examples-average-hard}


\apcomment{I have not gone through this subsection. From a glance, it looks like it could be significantly polished.} \gwcomment{Partially done. (Will discuss with Ashwin and Mengqi)}

This section focuses on the average-case hardness of path sparse PCA problem. Before stating the main result of reduction, we start with a generalization of the well-known planted clique (PC) problem (see Definition~\ref{defn:PC}) -- Secret Leakage PC. Compared with the PC problem, the random graph $G$ in the Secret Leakage PC comes with some information about the vertex set of the planted clique. 

% \gwcomment{The following result follows from [\cite{brennan2019optimal}, \cite{brennan2020reducibility}]. So, fairly speaking, it cannot be viewed as our main result... What we did here is applied their result to path-sparse PCA. The reason of giving this section is to show that the number samples used in the initialization method is somehow "tight" from average-case hardness.} 

%\begin{definition} \label{defn:PC}
%\textbf{Planted Clique Problem (PC).} The task of planted clique is to find the vertex set of a $K$-clique planted uniformly at random in an $N$-vertex Erd\"{o}s-Renyi random graph $G$, which can be represented as a testing problem $\text{PC}(N, K, 1/2)$ between the following two hypothesises 
%\begin{align}
%    & H_0: ~ G \sim \mathcal{G}(N, 1/2) ~~~~\text{and}~~~~ H_1: ~ G \sim \mathcal{G}(N, K, 1/2), \notag
%\end{align}
%where $\mathcal{G}(N, 1/2)$ denotes the $N$-vertex Erd\"{o}s-Renyi random graph with edge density $1/2$ and $\mathcal{G}(N, K, 1/2)$ the distribution resulting from planting a $K$-clique uniformly at random in $\mathcal{G}(N, 1/2)$.
%\end{definition}

%A generalization of the planted clique problem (PC) -- Secret Leakage PC is proposed in \cite{brennan2020reducibility}, where the random graph $G$ comes with some information about the vertex set of the planted clique. 

\gwcomment{Need: 
\begin{itemize}
	\item Def 5 
	\item Def 6 K-PC
	\item Def hardness. What it means to be hard? K-PC is hard
	\item Coro: if K-PC is hard, then Path sparse PCA is hard
\end{itemize}
}

\begin{definition} \label{defn:SL-PC}
\textbf{Secret Leakage $\text{PC}_{\mathcal{D}}$,} \cite{brennan2020reducibility}. Given a distribution $\mathcal{D}$ on $k$-subsets of $[N]$, let $\mathcal{G}_{\mathcal{D}}(N, K, 1/2)$ be the distribution on $N$-vertex graphs sampled by first sampling $G \sim \mathcal{G}(N, 1/2)$ and $S \sim \mathcal{D}$ independently and then planting a $K$-clique on the vertex set $S$ in $G$. Let $\text{PC}_{\mathcal{D}}(N, K, 1/2)$ denote the resulting hypothesis testing problem between  
\begin{align}
    & H_0: ~ G \sim \mathcal{G}(N, 1/2) ~~~~\text{and}~~~~ H_1: ~ G \sim \mathcal{G}_{\mathcal{D}}(N, K, 1/2). \notag
\end{align} 
\end{definition}
\gwcomment{detection problem}

Now consider the following $K$-partite PC as a special case of the Secret Leakage $\text{PC}_{\mathcal{D}}$. 

\begin{definition} \label{def:KPC}
\textbf{$K$-Partite Planted Clique (with source and terminal).} The $K$-partite planted clique problem $K\text{-PC}(N, K, 1/2)$ is a special case of the secret leakage planted clique problem $\text{PC}_{\mathcal{D}}(N, K, 1/2)$. Here the vertex set of $G$ has two special vertices: source and terminal, and the remaining vertices are evenly partition into $K$ parts of size $(N - 2) / K$. The distribution $\mathcal{D}$ always picks source, terminal and uniformly picks one element at random in each part.  
\end{definition}

In order to simplify the analysis of average-case hardness, we use the following variant ofsparse PCA -- uniformly biased sparse PCA (UBSPCA) proposed in \cite{brennan2018reducibility}.  

\begin{definition} \label{def:UBSPCA}
\textbf{UBSPCA Problem,} \cite{brennan2018reducibility}. Consider the following simple hypothesis testing variant $\text{UBSPCA}(k,n,d,\lambda)$ of sparse PCA with hypothesis: 
\begin{align*}
    H_0: ~ \bm{X} \sim \mathcal{N}(\bm{0}_d, \bm{I}_d)^{\otimes n} ~~~~\text{and}~~~~ H_1: ~ \bm{X} \sim \mathcal{N}(\bm{0}_d, \lambda \bm{v}_* \bm{v}_*^{\top} + \bm{I}_d)^{\otimes n}, 
\end{align*}
where $\bm{v}_*$ is in the uniformly biased set $\textup{UBS}_k := \left\{ \left. \bm{v} \in \left\{ 0, 1/\sqrt{k}\right\}^d ~ \right| ~ \|\bm{v}\|_2 = 1, ~ \|\bm{v}\|_0 = k  \right\}.$
\end{definition}


{\color{blue}
The reduction of path sparse PCA is based on an existing average-case reduction method SPCA-RECOVERY which was proposed in [Figure 19, Section 8, \cite{brennan2018reducibility}]. In this section, we omit the theoretical guarantees of SPCA-RECOVERY given by \cite{brennan2018reducibility}. Reader with independent interests could see Appendix~\ref{app:reduction-PathPCA} (Theorem~\ref{thm:reduction-UBSPCA}) for details. 

\gwcomment{Maybe move the following commented Theorem to appendix?} 
\iffalse
The following Theorem~\ref{thm:reduction-UBSPCA} is a restatement of the theoretical results for SPCA-RECOVERY given by \cite{brennan2018reducibility}. 

\begin{theorem} \label{thm:reduction-UBSPCA}
\textbf{Restate of [Theorem 8.7, \cite{brennan2018reducibility}].} Let $\alpha \in \mathbb{R}, \beta \in (0,1)$, and suppose there exists a sequence $\{(N_i, K_i, D_i, \lambda_i)\}_{i = 1}^{\infty}$ of parameters such that: 
\begin{enumerate}
    \item The parameters are in the regime $d = \Theta(N), \lambda = \tilde{\Theta}(N^{- \alpha})$ and $k = \tilde{\Theta}(N^{\beta})$ or equivalently,
    \begin{align*}
        & \lim_{i \rightarrow \infty} \log (\lambda_i^{-1} / \log N_i) = \alpha & \lim_{i \rightarrow \infty} \log (K_i / \log N_i) = \beta
    \end{align*}
    \item If $\alpha > 0, \beta > 1/2$, then the following holds. Let $\epsilon > 0$ be fixed and let $\bm{X}_i$ be an instance of $\text{UBSPCA}(K_i,N_i,D_i,\lambda_i)$. There is no sequence of randomized polynomial-time computable functions $\phi_i: \mathbb{R}_n: \mathbb{R}^{D_i \times N_i} \to \binom{[N_i]}{k}^2$ such that for all sufficiently large $i$ the probability that $\phi_i(\bm{X}_i)$ is exactly the pair of latent row and column supports of $\bm{X}_i$ is at least $\epsilon$, assuming the PC recovery conjecture. 
\end{enumerate}
Therefore, given the PC recovery conjecture, the computational boundary for $\text{UBSPCA}(k,n,d,\lambda)$ 
in the parameter regime $\theta = \tilde{\Theta}(n^{- \alpha})$ and $k = \tilde{\Theta}(n^{\beta})$ is $\alpha^* = 0$ when $\beta > 1/2$. 
\end{theorem}
\fi

As a brief summary of Theorem~\ref{thm:reduction-UBSPCA}, the SPCA-RECOVERY maps an instance of planted clique problem $G \sim \text{PC}(n,k,1/2)$ to an instance $\bm{X} \sim \text{UBSPCA}(k,n,d,\lambda)$ approximately under total variance distance with $\lambda = \tilde{\Theta}(k^2/n), d = n$. Assuming the Hardness Assumption for Planted Clique Problem [Definition~\ref{defn:PC}], the above reduction SPCA-RECOVERY ensures that there is no randomized polynomial-time algorithm solving $\text{UBSPCA}(k,n,n,\lambda)$ with $n \ll k^2$ when $\lambda = \tilde{\Theta}(1)$.}\\

Let $\text{Path-UBSPCA}(k + 2, n, d, \lambda)$ be the path-sparse PCA problem with signal $\bm{v}_* \in \textup{UBS}_k \cap \mathcal{P}^k$ with $\textup{UBS}_k$ defined in Definition~\ref{def:UBSPCA}. Combined with previous results, we have

\begin{corollary} \label{coro:reduction-PathPCA}
Under the same parameter regime presented in Theorem~\ref{thm:reduction-UBSPCA}, 
\begin{itemize}
    \item The SPCA-RECOVERY maps an instance $G \sim K\text{-PC}(n, k, 1/2)$ directly to an instance $\bm{X} \sim \text{Path-UBSPCA}(k + 2, n, d, \lambda)$ approximately under total variance distance with $\lambda = \tilde{\Theta}(k^2/n), d = n$. 
    \item Assuming the Hardness Assumption [Conjecture2 \& Conjecture 3, \cite{brennan2020reducibility}] for $K$-Partite Planted Clique Problem [Definition~\ref{def:KPC}], the above reduction SPCA-RECOVERY ensures that there is no randomized polynomial-time algorithm solving  $\text{Path-UBSPCA}(k + 2, n, d, \lambda)$ with $n \ll k^2$ when $\lambda = \tilde{\Theta}(1)$, which matches the number of samples $n$ required in Corollary~\ref{coro:initial-PS-PCA} (ignoring logarithm term). 
\end{itemize}
\end{corollary}
The proof of Corollary~\ref{coro:reduction-PathPCA} is presented in Section~\ref{app:reduction-PathPCA}. Corollary~\ref{coro:reduction-PathPCA} shows that the $\text{Path-UBSPCA}(k + 2, n, d, \lambda)$ has an average-case hardness lower bound on the sample size ($n \geq c k^2$ for some constant $c$). Since $\text{Path-UBSPCA}(k + 2, n, d, \lambda)$ is a special case of the path sparse PCA, the above result also ensures an average-case hardness lower bound for the path sparse PCA. 

\iffalse
Let us start with an existing average-case reduction method SPCA-RECOVERY that proposed in [Figure 19, Section 8, \cite{brennan2018reducibility}]. Based on [Theorem 8.7, \cite{brennan2018reducibility}], the SPCA-RECOVERY maps an instance of planted clique problem $G \sim \text{PC}(n,k,1/2)$ to an instance $\bm{X} \sim \text{UBSPCA}(k,n,d,\lambda)$ approximately under total variance distance with $\lambda = \tilde{\Theta}(k^2/n), d = n$. Assuming the Hardness Assumption for Planted Clique Problem [Definition~\ref{defn:PC}], the above reduction SPCA-RECOVERY ensures that there is no randomized polynomial-time algorithm solving $\text{UBSPCA}(k,n,n,\lambda)$ with $n \ll k^2$ when $\lambda = \tilde{\Theta}(1)$.\\


In order to show average-case hardness for path-sparse PCA, we would like to point out that: The SPCA-RECOVERY Algorithm is designed to \emph{preserve the structure} during the reduction. 
\begin{itemize}
    \item \textbf{Observation.} Note that: each part of the vertex set $V$ in graph $G \sim K\text{-PC}(n, k, 1/2)$ can be viewed as the vertex set of a layer in the layered graph with size $(n - 2) / k$. Thus the planted $k$-clique (with one vertex in each part) in $G \sim K\text{-PC}(n, k, 1/2)$ corresponds to a path of length $k$ (excluding source and terminal) in the layered graph.
    
    \item \textbf{Structure Preserving.} Based on the structure preserving property and above observation, given $G \sim K\text{-PC}(n, k, 1/2)$ as input instance of SPCA-RECOVERY, the rows in the adjacency matrix $\bm{A}(G)$ concerning the planted $k$-clique of $G$ transfer/map to the rows corresponding to the support set of the underlying path of the path-sparse PCA for the sample matrix $\bm{X} = (\bm{x}_1 ~|~ \cdots ~|~ \bm{x}_n)^{\top}$ as the output of $\textup{SPCA-RECOVERY}(G)$.
    % \gwcomment{This is what I learned from \cite{brennan2019optimal}. However, since the Algorithm is very complex, I did not check the proof detail. Or we may refer to the result in \cite{brennan2018reducibility} for sparse PCA by restricting $\lambda \asymp \sqrt{k^2 / n \log n}$ with $k = K = w(\sqrt{N})$.} 
\end{itemize}

In conclusion, using [Theorem 8.7, \cite{brennan2018reducibility}] directly, the SPCA-RECOVERY maps an instance $G \sim K\text{-PC}(n, k, 1/2)$ directly to an instance $\bm{X} \sim \text{Path-UBSPCA}(k + 2, n, d, \lambda)$ approximately under total variance distance with $\lambda = \tilde{\Theta}(k^2/n), d = n$. Here we use $\text{Path-UBSPCA}(k + 2, n, d, \lambda)$ to denote the path-sparse PCA problem with signal $\bm{v}_* \in \textup{UBS}_k \cap \mathcal{P}^k$ with $\textup{UBS}_k$ defined in Definition~\ref{def:UBSPCA}. Similarly, assuming the Hardness Assumption [Conjecture2 \& Conjecture 3, \cite{brennan2020reducibility}] for $K$-Partite Planted Clique Problem [Definition~\ref{def:KPC}], the above reduction SPCA-RECOVERY ensures that there is no randomized polynomial-time algorithm solving  $\text{Path-UBSPCA}(k + 2, n, d, \lambda)$ with $n \ll k^2$ when $\lambda = \tilde{\Theta}(1)$, which matches the number of samples $n$ required in Corollary~\ref{coro:initial-PS-PCA} (ignoring logarithm term). 


\begin{remark} 
Notice that $\text{Path-UBSPCA}(k + 2, n, d, \lambda)$ of samples from SPCA-RECOVERY is a special case of path-sparse PCA defined in Section~\ref{sec:setting-background}. Moreover, as also mentioned in \cite{brennan2018reducibility}, the lower bound on the number of samples are only tight when $\lambda = \tilde{\Theta}(1)$. 
\end{remark}
\fi 

\fi




\section{Discussion}
\label{sec: discussion}
\kmsdelete{In this work} We study \kmsreplace{Fairness-Aware PAC learning}{Fair-ERM} in the malicious noise model, and  in some cases allow 
the learner to maintain optimal overall accuracy despite the signal in Group $B$ being almost entirely washed out.
%when we allow learners to use the
%$\PQ$ randomized expansion of the hypothesis class $\mathcal{H}$
In particular we show that different fairness constraints have fundamentally different behavior in the presence of Malicious Noise, in terms of the amount of accuracy loss that a given level of Malicious Noise could cause a fairness-constrained learner to incur. 
The key to achieving our results, which are more optimistic than those in \cite{lampert}, is allowing for improper learners using the (P,Q)-randomized expansions of the given class $\mathcal{H}$.
%We \kmsreplace{present a picture of the}{prove upper and lower bounds on}
%accuracy loss for a range of fairness notions, given \kmsreplace{this simple randomization step.}{learning over $\PQ$.
%In general our results indicate Fair-ERM (given learning over $\PQ$) is more robust than claimed in \cite{lampert}.
The type of smoothness we create by using $\PQ$ seems to be a natural property that is likely shared by many natural hypothesis classes.

Fairness notions are motivated as a response to learned disparities when there is \kmsdelete{data corruption or} systemic error affecting \kmsdelete{the data for}
one group. 
Fairness notions are supposed to mitigate this by ruling out classifiers that have worse performance on a sub-group. 
This can peg both classifiers at a lower level of performance \kmsdelete{(e.g that the lower subgroup)} in order to \emph{motivate} \cite{hardt16} improving the data collection or labelling process to obtain more reliable performance. 
%So in \kmsreplace{some}{a} sense, sensitivity of the fairness notion to poor sub-group performance caused by malicious noise is the \textit{point} of fairness constraints! 
However, it also desirable that fairness constraints perform gracefully when subject to Malicious Noise because fairness constraints will be used in contexts where the data is unreliable and noisy and this might not be known to the learner.
This tension, exposed by our work, motivates 
%a revisiting of fairness notions from first principles approach and trying to axiomatize the 
%desired properties of a fairness intervention a la cryptography and privacy. \footnote{Work in multi-calibration \cite{multicalib} is a viable direction for this problem but it is unclear how 
%that and related notions behave with unreliable data. }
on going work studying the sensitivity level of fairness constraints. 
%If we we are to take a view, if a classifier is deployed 



%%%%%%%%%%%%%%%%%%%%%%%%%%%%%%%%%%%%%%%%%%%%%%%%%%%%%%%%%%%%%%%%%%%%%%%%%

\subsubsection*{Acknowledgements}
GW was supported by the National University of Singapore under AcRF Tier-1 grant (A-8000607-00-00) 22-5539-A0001. ML and AP were supported in part by the NSF under grants CCF-2107455 and DMS-2210734 and by research awards/gifts from Adobe, Amazon, and Mathworks. We are grateful to the Simons Institute for the Theory of Computing for their hospitality, where part of this work was performed.


%\subsubsection*{References}
\bibliographystyle{abbrvnat}
%\bibliographystyle{apalike}
\bibliography{reference}

%%%%%%%%%%%%%%%%%%%%%%%%%%%%%%%%%%%%%%%%%%%%%%%%%%%%%%%%%%%%%%%%%%%%%%%%%
 
%\newpage
%%~ 
%%\newpage
%%\onecolumn

\appendix  
\begin{comment}
\section{System Architecture}
\label{appendix:architecture}
\system has a novel modularized system architecture with three key components: 
\emph{StreamManager}, 
\emph{TxnManager} and \emph{TxnScheduler}. 
These components are instantiated in each thread locally.
The execution outline of \system is presented in Algorithm~\ref{alg:algo}.
Transactional stream processing is continuous and potentially never ends (Line 1$\sim$8).
The dependency resolution and execution of state transactions are separated into two non-overlapping phases by punctuations~\cite{Tucker:2003:EPS:776752.776780} (Line 2 and 5), which guarantees that no subsequent input event will have a smaller timestamp. 
Effectively, a batch of state transactions is collected during the first phase, and processed during the second phase.

In the first phase (i.e., stream processing phase), 
the \emph{StreamManager} conducts preprocessing for every input event ($e$). Similar to some prior works~\cite{tstream}, state transactions may be issued but not immediately processed during preprocessing (Line 3).
The \emph{pre\_processing} and \emph{post\_processing} functions are exposed as APIs to users.
The \emph{TxnManager} handles dependency resolution (Line 4) among state transactions and insert decomposed operations to construct a \tpg. We discuss the detailed two-phase \tpg construction process in Section~\ref{subsec:construction}.

In the second phase  (i.e., transaction processing phase), 
the \emph{TxnManager} is first involved again to refine (Line 6) the constructed \tpg with further dependency resolution.
The \emph{TxnScheduler} 
schedules operations for concurrent execution based on the constructed \tpg according to the three dimensions of scheduling decisions (Line 7). 
In particular, a scheduling decision model $M$ is instantiated based on the constructed \tpg (Line 14).
\textbf{\circled{1}} Guided by $M$, execution threads adopt an exploration strategy (Section~\ref{subsec:explore}) to explore the constructed \tpg for operations available to be scheduled constrained by dependencies. 
\textbf{\circled{2}} 
During exploration, one or multiple operations may be treated as the 
% basic 
unit of scheduling (Section~\ref{subsec:granularity}). 
Subsequently, \textbf{\circled{3}} every thread executes operation(s) in the unit of scheduling with various abort handling mechanisms (Section~\ref{subsec:abort_handling}).
Only when state transactions are processed (i.e., committed or aborted) can the associated input events be postprocessed (Line 8) by the \emph{StreamManager} based on transaction processing results.
\end{comment}

\begin{comment}
\begin{algorithm}
\footnotesize
    \KwData{$e$ \tcp{Input event}}
    \KwData{$txn_{ts}$ \tcp{State transaction}}
    \KwData{$G$ \tcp{The currently constructed TPG}}
    \While{!finish processing of input streams}{
        \eIf(\tcp*[h]{Phase 1}){\text{$e$ is not a $punctuation$}}{
                $txn_{ts}$ $\gets$ PRE\_Processing($e$)\;
                \textbf{TPG\_Construction}($G$, $txn_{ts}$)\; 
          }(\tcp*[h]{Phase 2}){
                \textbf{TPG\_Refinement}($G$)\; 
                \textbf{TXN\_Scheduling}($G$)\; 
                POST\_Processing()\;
          }
    }
    
    \SetKwFunction{FMain}{TPG\_Construction}
    \SetKwProg{Fn}{Function}{:}{}
    \Fn{\FMain{$G$, $txn_{ts}$}}{
        $O_{1..k}$ $\gets$ \textbf{Partition} $txn_{ts}$\;
        \ForEach{\text{operation $O_{i}$ $\in$ $O_{1..k}$}}{
            \textbf{Identify} its \ld\;
            $G$ $\gets$ $G$ + $O_{i}$ \;
        }
    }
    \SetKwFunction{FMain}{TPG\_Refinement}
    \SetKwProg{Fn}{Function}{:}{}
    \Fn{\FMain{$G$}}{
        \ForEach{\text{vertex $e_{i}$ $\in$ $G$}}{
            \textbf{Identify} its \td, \pd\;
        }
    }
    
    \SetKwFunction{FMain}{TXN\_Scheduling}
    \SetKwProg{Fn}{Function}{:}{}
    \Fn{\FMain{$G$}}{
        $M$ $\gets$ Instantiated with $G$;\tcp{A decision model}
        \While{!finish scheduling of $G$
        }{
          \textbf{\circled{2}} $Scheduling Unit$ $\gets$ \textbf{\circled{1}} \emph{Explore}($G$, $M$)\; 
            \textbf{\circled{3}} \emph{Execute with Abort Handling} ($Scheduling Unit$)\; 
        }
    }
  \caption{Execution Outline of \system}
  \label{alg:algo}
\end{algorithm}
\end{comment}


\end{document}
