\documentclass[10pt,a4paper,leqno]{article}
%\usepackage{showkeys}
\usepackage{amsfonts,dsfont,amsmath,amsthm,enumerate,color}
\usepackage{graphicx}
\usepackage{fullpage}
\usepackage{authblk}
\usepackage{bold-extra}

 
\definecolor{orange}{rgb}{1,0.25,0}
\definecolor{darkgreen}{rgb}{0.0, .5, 0.13}


\usepackage[colorlinks=true,linkcolor=blue,citecolor=red, pdfborder={0 0 0}]{hyperref}

%\usepackage{pxfonts}
%
%\usepackage{bm,amsthm,amsopn,amssymb,dsfont, oldgerm,yhmath}
%
\setlength{\parindent}{0cm}

\newcommand{\vertiii}[1]{{\left\vert\kern-0.25ex\left\vert\kern-0.25ex\left\vert #1 
		\right\vert\kern-0.25ex\right\vert\kern-0.25ex\right\vert}}


\renewcommand{\i}{{\rm i}}

\renewcommand{\b}[1]{\textbf{#1}}



\newcommand{\CC}{\mathbb{C}}
\newcommand{\NN}{\mathbb{N}}
\newcommand{\RR}{\mathbb{R}}
\newcommand{\ZZ}{\mathbb{Z}}

\theoremstyle{plain}
\newtheorem{theorem}{Theorem}[section]
\newtheorem{lemma}[theorem]{Lemma}
\newtheorem{proposition}[theorem]{Proposition}
\newtheorem{corollary}[theorem]{Corollary}

\theoremstyle{definition}
\newtheorem{definition}[theorem]{Definition}
\newtheorem{example}[theorem]{Example}

\theoremstyle{plain}
\newtheorem{remark}[theorem]{Remark}


\numberwithin{equation}{section}

 
\newcommand\ep{\varepsilon}
\newcommand\od[2]{\frac{d #1}{d #2}}
\newcommand\pd[2]{\tfrac{\displaystyle\partial #1}{\displaystyle\partial #2}}
\newcommand{\Ep}{\epsilon}
%\title{A general framework for the homogenisation of high-contrast problems}
\title{\vspace{-1.8cm}  Uniform asymptotics 
for a family of degenerating variational problems and error estimates in homogenisation theory  
%Uniform asymptotics for a family of degenerating variational problems with applications to error estimates  in homogenisation theory. 
%	\\ \\  Asymptotics for a class of degenerating variational problems with applications to error estimates  in homogenisation theory
}



\author[1]{\vspace{-.2cm}Shane Cooper}
\author[1]{Ilia V. Kamotski}
\author[1]{Valery P. Smyshlyaev\vspace{-.3cm}}

%\affil[1]{\footnotesize Department of Mathematical Sciences, Durham University, Mountjoy Centre, Stockton Rd, Durham, DH1 3LE.}
\affil[1]{\footnotesize Department of Mathematics, University College London, Gordon Street, London WC1E 6BT, UK.}

%\affil{First Institution/Department, Affiliation, City, Country }{FIRSTAFF}
%\affil{Second Institution/Department, Affiliation, City, Country }{SECONDAFF}
%\author{Shane Cooper\footnote{Department of Mathematical Sciences, Durham University, Mountjoy Centre, Stockton Rd, Durham, DH1 3LE. Email: shane.a.cooper@durham.ac.uk}, Ilia Kamotski\footnote{Department of Mathematics, University College London, 25 Gordon Street, London, WC1H 0AY. Email: i.kamotski@ucl.ac.uk} and Valery Smyshlyaev\footnote{}}

\renewcommand{\t}{\theta}
\renewcommand{\l}{\langle}
\renewcommand{\r}{\rangle}

\newcommand{\N}{\mathcal{N}_\t}
\newcommand{\T}{\mathcal{T}_\theta}
\newcommand{\M}{\mathcal{M}_\t}
\newcommand{\Vs}{V^\star}
\newcommand{\Ws}{W^\star}

\newcommand{\PW}{P_{W^\star_\t}}
\newcommand{\PV}{P_{V_\t}}

\DeclareMathOperator{\ourN}{\mathtt{N}}

%\newcommand{\H}{\mathcal{H}}

\newcommand{\be}{\begin{equation}}
\newcommand{\ee}{\end{equation}}
\begin{document}

\maketitle
\begin{abstract}
We study %the uniform asymptotics of solutions to 
an abstract family of asymptotically degenerating variational problems. 
Those are natural generalisations of families of problems emerging 
upon application of  
%in the course of 
%reformulating, via 
a rescaled Floquet-Bloch-Gelfand transform to resolvent problems for highly oscillatory high-contrast elliptic PDEs.  
%of a `micro-resonant' type.  
An asymptotic analysis of these  problems leads us to 
%deriving effective equations  for highly oscillatory degenerating linear PDE systems. A 
a hierarchy of approximation results with uniform operator-type error estimates under various assumptions, 
satisfied by 
%based on generic characteristics of the variational problems observed in diverse 
specific examples. Associated spectral problems are considered, and we provide  approximations of the spectrum  in terms of the spectrum of a certain `bivariate' operator which appears an abstract generalisation of 
the two-scale limit operators %emerging via two-scale convergence methods 
for highly oscillatory high-contrast PDEs.  
An explicit  description of the limit spectrum in the abstract setting is provided, and tight error estimates on the distance between the original and limit spectra are established. 
Our generic approach 
%starts as the well-established spectral method, of classical homogenisation, but the techniques we develop do not rely on analytic perturbation theory and this 
allows us to readily consider a wide class of asymptotically degenerating problems including but also going beyond high-contrast highly oscillatory PDEs. 
The obtained results are illustrated by various %highly oscillatory PDE 
examples.

\end{abstract}
%-------------------------------------------------------------------
\section{Introduction}

One of the main motivations for the present study comes from the desire for construction of accurate approximations of two-scale type, with controllably small errors, 
 for mathematical models with strong interaction between micro and macro-scales. An interest in such models comes from the fact that they are 
often capable of displaying certain non-standard and unusual physical effects in an asymptotically explicit way, which often clarifies the nature and the microscopic mechanism of the observable macroscopic effects. 

One class of such models includes two-scale Partial Differential Equations (PDEs) with high-contrast coefficients, or in other words asymptotically degenerating PDEs. 
Consider for example a simple scalar model of time-harmonic wave propagation in an $\ep$-periodic medium described by 
\begin{equation} 
\label{th-hc}
{\rm div}\big(a_{\ep,\delta}(x) \nabla u_\ep\big)  \,\,+\,\,  \rho\,\omega^2\, u_\ep \,\,=\,\, 0, \quad \, x=(x_1,x_2,...,x_n)\,\in\mathbb{R}^n, \,\, n\ge 1.  
	\end{equation}
Here $\omega>0$ is angular frequency, $\rho$ is (for simplicity) a positive constant, and 
 $a_{\ep,\delta}(x)=a_\delta\left(x/\ep\right)$ where  $a_\delta(y)=1-\chi(y)+\delta\chi(y)$ and $\chi$ is the characteristic function of a set of isolated ``soft'' 
inclusions $1$-periodic with respect to each variable $x_j$, $j=1,2,...,n$, surrounded by a ``stiff'' connected matrix. 

When both $\ep$ and $\delta$ are small, a critical scaling is $\delta\sim\ep^2$, which in the context of the wave propagation model \eqref{th-hc} is known to be 
a ``micro-resonant'' scaling: frequencies producing order-one wavelengths in the matrix material would produce an order-$\ep$ wavelength in the inclusion material i.e. 
those comparable with the inclusions' size, or in other words such frequencies are comparable to the resonant frequencies of the inclusions. 
This is reflected in the two-scale asymptotics of certain Bloch wave solutions to \eqref{th-hc}: $u_\ep(x)\sim u_0(x,x/\ep)$, where $u_0(x,y)$ is a function of 
only macroscopic variable $x$ in the matrix but is also a function of the microscopic variable $y$ in the resonating inclusions. 
This leads to a limit two-scale system for $u_0$ which 
displays such effects as band-gap opening near the resonances, as was probably first formally observed in a similar context in \cite{AurBonn85} and then made rigorous in 
\cite{HeLi} and \cite{Zhi2000,Zhi2005}. 
The $\delta\sim\ep^2$ scaling plays a similarly key role in the so-called double porosity type diffusion models, for related earlier studies see e.g. 
\cite{FeKh80,Ar90,Pa91,All,Sa99}.
The continued intensive interest in studying such degenerate models is largely due to the fact that the %effective equations like 
related two-scale approximations indeed possess a wealth of interesting and unusual properties. 
For an incomplete list of related other works we mention 
%`novel'  properties; i.e. properties that are not present  in the homogenised equations for the (classical) non-degenerate setting.  Such properties have been shown in various specific settings to be practically desirable, see 
\cite{Av08,BaKaSm,Be,BoFe,Br,CaEd05,cherd,ChErKi,ChKiVeZu23,Co,IVKVPS18,IVKVPS19,ZhiPa,VPS}. % for a very short, non-exhaustive, list of such works. 

Problem \eqref{th-hc} is mathematically a spectral problem for operator $\mathcal{A}u=\,-\,{\rm div}A_{\ep}\nabla u_\ep$ in Hilbert space $L^2\left(\RR^n\right)$, 
with spectral parameter $\lambda=\rho\,\omega^2$. (Here we regard $A_\ep(x):=a_{\ep,\,\ep^2}(x)$ i.e. we set $\delta=\ep^2$.) 
A key for the analysis of \eqref{th-hc}, including of its asymptotic properties as $\ep\to 0$, lies in analysing the related resolvent problem  
\begin{equation} 
\label{hom}
	-\,{\rm div}A_{\ep} \nabla u_\ep  \,\,+\,\,   u_\ep\,\, =\,\, F, \ \,\,\,\, F\in \,L^2\left(\RR^n\right). 
\end{equation}
Indeed, as we will see in this work too, asymptotic behaviour of a wide class of resolvent problems like \eqref{hom} plays crucial role  for the 
properties of the related spectral problems as $\ep\to 0$. 
Moreover, they also hold keys for analysing related evolution problems, both parabolic and hyperbolic, cf. e.g. \cite{Pas05,ZhP07,IVKVPS19}. 

This all motivates efforts on constructing tractable but accurate approximations first to the solutions of \eqref{hom}. 
In other words, this leads to a natural and until recently largely open question: 
 %possess the same desirable properties (as their limits) for finite $\ep$ (rather than just when $\ep$ tends to zero); or to put this question another way: 
can one establish tractable leading-order approximations for the equations like \eqref{hom}, %, with respect to $\ep$,  for 
with small error estimates for the solutions for small $\ep$, uniformly with respect to $F$ (in various norms)? 


In the context of non-degenerate PDEs or classical homogenisation (e.g. for problem \eqref{hom} which corresponds to a fixed $\delta>0$ in \eqref{th-hc}), error estimates for the approximation given by the related homogenised equations are by now well-known, and various approaches exist to establish them. We shall 
not even attempt to provide a review of all these different methods here 
except to mention one approach of particular relevance to this work: the so-called {\it spectral method}, see for example \cite{Zh89,BiSu,ZhSpectr} and also \cite{CoVa97}. Our approach here is conceptually somewhat similar but we believe bears at the same time fundamental novelties allowing to obtain new results for wide classes of examples,  see Section \ref{sec:examples}. 
Namely, our method does not employ analytic perturbation theory but (as clarified below) performs instead a robust generic asymptotic analysis near 
%the related key 
certain degeneracy points. % in (an abstract analog of) the `quasi-periodicity' parameter. 
This is what ultimately allows us to successfully analyse the more general degenerate problems with relative ease.


As far as degenerate problems are concerned, there has been some progress recently in obtaining approximations with error estimates 
specifically for the high-contrast problem like the ``double-porosity'' one above with $\delta=\ep^2$ in \eqref{hom}.  
A leading-order approximation with $L^2$ error estimates was established in \cite{ChCo}, for multiple spatial dimensions, using the spectral method as a basis; however the techniques employed therein appear problem specific and not readily generalisable. 
 In  one-spatial dimension double porosity models, leading-order approximations with error estimates were obtained in \cite{ChChCo} and \cite{ChKi} by different approaches. 
 We emphasise here that upon applying our  general method to the above key example of double porosity model, see Section \ref{e.dp}, we provide novel operator estimates between the initial and  two-scale limit operators. 
A crucial role is played here, see Theorem \ref{thm.2scOpRes}, by a new two-scale interpolation operator 
$\mathcal{I}_\ep:L^2\left(\RR^n\right)\rightarrow L^2\left(\RR^n\times\square\right)$, where $\square$ is the $y$-periodicity cell. 
$\mathcal{I}_\ep$ is an $L^2$-isometry and a key tool for recasting, for any final $\ep>0$, an input function $F(x)$ of \ref{hom} as  a two-scale function $f_0(x,y)$ which in turn serves 
as an input for the two-scale limit problem and which ultimately ensures the desired operator error estimate, see \eqref{dpcompe3}, with further implications e.g. for 
error estimates on Floquet-Bloch eigenvalues and eigenfunctions, Theorem \ref{EVestDP} and Remark \ref{EFestDP}. 
Operator $\mathcal{I}_\ep$ appears to be a 
 novel two-scale analogue of classical 
Whittaker-Shannon interpolation formula and replaces in a sense the unfolding operator \cite{CDG} which seems insufficient for such high-contrast problems, 
see Remarks \ref{RemShann} and \ref{ShannVsUnf}. 
%and \ref{e:idp} for the multi- and one- spatial dimensional problem respectively.
%Our work in this context also extends the results in \cite{HeLi}.
 
 
%   e.g.
%% are periodic with $A_{\ep}(\cdot)$ being a bounded matrix and $\rho_\ep(\cdot)$  a posistive function that 
%$\exists c_\ep, d_\ep >0$ such that 
%\[
%c_\ep  \int_{\RR^n} |\nabla u|^2 \le \int_{\RR^n} A_\ep \nabla u \cdot \overline{\nabla u}, \qquad \text{and} \qquad  \int_{\RR^n}  \rho_\ep |u|^2 \le d_\ep\int_{\RR^n} |u|^2  , \qquad   \forall u \in C^\infty_0(\RR^n)
%\]
%with $\lim_{\ep \rightarrow 0} c_\ep =0$ and or $\lim_{\ep \rightarrow +\infty} d_\ep = + \infty$. 




%$=a_1(\tfrac{\cdot}{\ep} )+\ep^2a_2(\tfrac{\cdot}{\ep} )$ with $a_1,a_2$  non-negative  measurable periodic matrix-functions such that the sum is uniformly positive and bound:
%\[\gamma| \xi|^2 \le  \big(a_1(y) + a_2(y)\big)\xi\cdot \xi\leq \gamma^{-1}|\xi|^2,\ \   \forall \xi, y \in \mathbb{R}^n,
%\]
%for some $\gamma>0$, see \cite{VPS} and references therein. Well-studied examples include the  non-degenerate `classical' homogenisation problem, when $a_2 \equiv 0$, and  the high-contrast or ``double porosity" model, when $a_1=\chi $, $a_2 = 1-\chi$ for $\chi$ the characteristic function of a connected periodic set, see \cite{Zhi2000,Ar90,All}. 



%Answering this question, for a large class of problems,  is one of the outcomes of this article.




That all being said, one of the main aims of this article is  not to rediscover or improve the previously obtained %mentioned available 
results. Rather, we shall  demonstrate that a large class of the above mentioned problems (as well as many others) are all examples of  one particular 
generic asymptotically degenerating abstract variational problem. As such, their leading-order asymptotics are all particular instances of an asymptotic 
approximation for that general variational problem. In this article we derive the leading-order asymptotics for this abstract problem with error estimates. We then specify the underlying abstract objects to provide  asymptotics (with operator-type error estimates) for various problems of interest.
% and apply the results  to derive asymptotics with error for various problems of interest.


As a way to motivate the general problem we  recall that the starting point in the spectral method, used in the above-mentioned non-degenerate PDE setting i.e. 
with \eqref{hom} where $A_\ep(x)=a_\delta(x/\ep)$ with a fixed $\delta>0$, is following. 
Apply the rescaling $x \mapsto \ep y$ and then the Floquet-Bloch-Gelfand transform (see Section \ref{e.class}) to arrive at the family of problems on Sobolev 
space $H^1_{per}(\square)$ of $\square$-periodic functions 
\begin{equation}
	\left\{ \ \begin{aligned} & \text{For each $\t \in [-\pi,\pi]^n$, find $u_{\ep,\theta} \in H^1_{per}(\square)$ such that} \\
		&	-\ep^{-2} ( \nabla + \i \theta ) \cdot a_\delta(y) (\nabla + \i \theta)  u_{\ep,\t}  +  u_{\ep, \t}= f, 
	\end{aligned}
	\right.\label{hom2}
\end{equation}
where $f$ is the transform of (rescaled) $F$. 


% put this in the abstract: -----
%The spectral method then proceeds as follows:   for $P_\t$ the spectral projection  onto $[0,r)$ for the non-negative self-adjoint operator $\mathcal{L}_\t = -( \nabla + \i \theta ) \cdot a_1(y) (\nabla + \i \theta)$, on $L^2$ periodic functions,  one has   $u_{\ep,\t} = (\ep^{-2}  \mathcal{L}_\t + I)^{-1} F =  (\ep^{-2}  \mathcal{L}_\t + I)^{-1}  P_{\t} F + w_{\ep,\t}$ with  $\| w_{\ep,\t} \|_{L^2} \le C \ep \| F\|_{L^2}$ for some constant $C$ that depends only on $r$. Thus, to establish the leading-order asymptotics of $u_{\ep,\t}$ with error estimate uniform in $\t$ (that in turn provides an approximation for $u_\ep$ with error by inverse transforms) one need only look at the structure of the spectral projector $P_\t$ for small $r$; this is done using analytic perturbation theory.
%Our approach differs to the classical method in two key ways: first we do not require the special structure  for the rescaled transformed problem \eqref{hom2}; we will study the more general family of variational problems:
%The second key  difference in our approach lies in the techniques we use; compared to  the previous studies, we do not rely on analytic perturbation theory (or even spectral theory).
%simultaneously provide the leading-order approximation, with respect to $\ep>0$,   and derive  error estimates, uniform in $\theta\in \Theta$, for the solution $u_{\ep,\t}$ to 

%Now, in this article, 
Next, we observe that the weak formulation of problem \eqref{hom2} is of the following more abstract type: 
\begin{equation}
	\left\{ \ \begin{aligned} & \text{For each $\ep >0$ and  $\t \in \Theta$, find $u_{\ep,\theta} \in H$ such that} \\
		&	\ep^{-2}
		a_\t\left(u_{\ep,\theta} ,\,\tilde u \right) \,\,+\,\, b_\t(u_{\ep, \theta},\tilde u) \,\,=\,\, \l  f,\tilde u\r,  \quad \forall \tilde u \in H,
	\end{aligned}
	\right.\label{pr}
\end{equation}
where $\Theta \subset \mathbb{R}^n$ is compact, $H$ a complex Hilbert space, $f$ a bounded linear functional on  $H$, $a_\t$ and $b_\t$ are non-negative bounded sesquilinear forms such that $a_\t + b_\t$ is a family of uniformly equivalent inner products  on $H$, and  $a_\t$ is Lipschitz-continuous in $\t$ (see Section \ref{sec:pf} for the precise details). Notice that in many of our motivating examples the `singular' forms  
$a_\t$  have non-trivial degeneracy subspaces $V_\t = \left\{ u \in H \,\, | \,\, a_\t(u,u) =0\right\}$, 
%need not be  an inner product on $H$, 
that is why we refer to such variational problems as %(asymptotically) 
{\it degenerate}.


The formulation \eqref{pr} does not just cover the above classical or double-porosity type settings \eqref{hom2} (for $\delta>0$ fixed and $\delta=\ep^2$ respectively, with corresponding $a_\t$ and $b_\t$), but also a much wider class of interesting problems. For example it %\eqref{pr} 
 includes among others, as illustrated by examples in Section \ref{sec:examples}, models as diverse as:  

-- `Inverted' high-contrast model (Section \ref{e:idp}), with resulting approximation (accompanied by error bounds) by an infinite contrast `stiff problem' 
rather than any two-scale one; 

-- Problems with concentrated perturbations (Section \ref{sec:concpert}) where $\t$ can be not (only) the quasi-periodicity parameter; 

-- Problems with `weakly bonded' imperfect interfaces (Section \ref{sec:impint}); 

-- Elasticity with `partially degenerating' inclusions (Section \ref{e.pdelast}); 

-- Schrodinger's equations  with a strong periodic magnetic field (Section \ref{magnschrod}); 
%\[
%-(\nabla-i\ep^{-1}b(\tfrac{\cdot}{\ep} ))^*(\nabla-i\ep^{-1}b(\tfrac{\cdot}{\ep} )) u + u = f;
%\]

-- Differential-difference equations (Section \ref{sec:nonloc}). 

Further examples which are not covered in the present work but also fall into the abstract framework of \eqref{pr} include: 
a wide class of partially degenerating high-contrast PDE systems
%, i.e.  \eqref{hom} with $A_\ep = a_1(\tfrac{\cdot}{\ep}) + \ep^2 a_0(\tfrac{\cdot}{\ep})$ where $a_0 + a_1$ is positive and bounded (this model generalises the classical and double-porosity model, 
(see \cite{IVKVPS13} for some further related details); 
 homogenisation problems on periodic quantum graphs and their generalisations, cf. e.g. \cite{IVKVPS19};  
problems in thin domains; 
problems on discrete periodic lattices; 
and some higher-order differential and pseudo-differential operators. 
We emphasise here that we do not generally require the forms $a_\t$ and $b_\t$ to  be generated by differential operators, nor do  we require $\t$ to 
necessarily be the  Floquet-Bloch parameter or even for $H$ to be a function space. This suggests possible far-reaching consequences of the present approach that 
can go  even further beyond the scope of the examples outlined  above. 

%The second key  difference in our approach lies in the techniques we use; compared to  the previous studies, we do not rely on analytic perturbation theory (or even spectral theory). 
%In this article we shall provide leading-order approximations, with respect to $\ep$, for $u_{\ep,\t}$ with error estimates that are uniform in $\t \in \Theta$ and $f$( in dual norm). This results are shown to hold for a small number of assumptions that are readily verifiable in practice and hold for all our motivating examples.

The article is organised as follows. In Section \ref{sec:pf} we formulate the abstract problem and introduce our main assumption of the article that we call a  \textit{spectral gap} condition; namely, for every $\t\in\Theta$ the form $a_\t$ is coercive on the orthogonal complement $W_\t$ of its null-space $V_\t$.  This condition is a far-reaching generalisation of a `key assumption' introduced in \cite{IVKVPS13} found to be important in establishing the two-scale  convergence to homogenisation limits for a general class of partially degenerating elliptic PDE systems of type \eqref{hom} in general domains. 

In Section \ref{s:uniforma}, we show that if  the null-space $V_\t$ is Lipschitz continuous with respect to $\t$ then $a_\t$ is uniformly coercive 
in $\t$ on $W_\t$  % (up to its kernel $V_\t$) 
and, consequently,  the leading-order approximation  simply comes from  `projecting'  problem \eqref{pr} onto $V_\t$. This simple result not only forms the basis for further investigation, it appears applicable to certain physically relevant models, for example,  the { \it inverted } double porosity model, see Example \ref{e:idp}, and in the study of certain polarisations of electromagnetic waves in Photonic Crystal Fibres, cf. \cite{Cothesis,CoKaSmPCF}. 

In Section \ref{section:discV} we study the case of discontinuous $V_\t$. 
  This situation is typical in  motivating examples  such as the above %both 
	classical and 
high-contrast problems \eqref{hom}, and  corresponds to loss of the uniformity of the spectral gap which requires %as a result 
 a much more subtle asymptotic analysis of the solutions near such a singular point. 
  In Theorem \ref{thm1.all} we construct a leading-order approximation to problems \eqref{pr}  when the null-space $V_\t$ possesses an isolated singularity (say at the origin $\t=0$) that is removable in the following sense:  there exists a subspace $V_\star$ such that 
$	
	V^\star_\t  = \left\{ \begin{array}{cc}
		V_\t, & \t \neq 0, \\ V_\star, & \t = 0
	\end{array} \right.  
$ {is Lipschitz continuous.} 
The resulting approximation is simpler than \eqref{pr} but still depends on $\ep$ and $\t$. In Section 
%Unlike in the Lipschitz $V_\t$ case, the leading-order approximation is no-longer $\ep$-independent; however  this dependence is much simplified. Indeed, in Sections \ref{section:discV}-
\ref{sec.2dif}, we provide an approximation with even simpler self-similar $\ep$ and $\t$ dependencies via their ratio $\t/\ep=:\xi$. This is done by approximating 
 the forms $a_\t$ and $b_\t$ for small $\t$, which can be performed under  additional mild regularity assumptions on $a_\t$ and $b_\t$ and some $\t$-nondegeneracy condition for the spectral gap, that are readily observed in practical examples. 
%nder some additional mild regularity assumptions on $a_\t$ and $b_\t$ that are readily observed in practice. 

This leads to one of the main results, Theorem \ref{thm.IKunifest2}, that provides a uniform approximation to the two-parameter solution $u_{\ep,\t}$ of \eqref{pr} in terms of a solutions to a one-parameter family of variational problems on the smaller space $V_0$ with quadratic forms $a^{\rm h}_\xi + b_0$, $\xi \in \RR^n$. 
%family  of forms $\ep^{-2} a_\t + b_\t$  on $H$  can be uniformly approximated by the much-simplified family of forms $a^{\rm h}_{\t/\ep} + b_0$ on the smaller space $V_0$.  
Here the non-negative form $a^{\rm h}_\xi\left(z,\tilde z\right)$ acts on an even smaller `defect subspace' $Z$ describing the gap between $V_0$ and $V_\star$, and 
is quadratic %a second-order homogeneous  polynomial 
in $\xi$. 
Form $a_\xi^{\rm h}$ appears to  generalise the homogenised matrix in classical homogenisation problems, and $Z$ 
%cokernel of $a^{\rm h}_\xi$ (which only appears  when $V_\t$ is discontinuous) 
is found to be finite-dimensional under a stronger version of the spectral gap condition which typically holds in practical examples, see Section \ref{sec.newKA}. 
The significance of the dependence of the approximating problem only  on single parameter $\xi=\t/\ep$ manifests itself  in Section \ref{s:resolv} were $\ep$-independent approximations in terms of an abstract version of a `two-scale' limit operator, with principal symbol  $a^{\rm h}_\xi$,  are constructed.  
% This for example means that our approximation in the classical setting is a system of finitely many algebraic equations whose coefficients  depend only on the ratio $\t / \ep$. 

Section \ref{s:resolv} focuses on the associated abstract spectral problem
\begin{equation}
	\left\{ \ \begin{aligned} & \text{For each $\ep >0$ and  $\t \in \Theta, \,\,\,$ find $\lambda_{\ep,\t}\in [0,\infty)$,  
	$\,u_{\ep,\theta} \in H\backslash \{0\}$ such that} \\
		&	\ep^{-2}\,
		a_\t\left(u_{\ep,\theta},\,\tilde u \right) \,\,+\,\, b_\t\left(u_{\ep, \theta},\,\tilde u\right) \,\,=\,\, 
		\lambda_{\ep,\t}\,\, d_\t\left(u_{\ep,\t},\,\tilde u\right)  \quad \, \forall \phi \in H,
	\end{aligned}
	\right.\label{prspec}
\end{equation}
where $d_\t$ is a positive compact sesquilinear form satisfying some isometry conditions that are trivially observed in examples. Using the results of the previous sections we establish that the union of the spectra $\bigcup_{\t \in \Theta} \lambda_{\ep,\t} $  converges in appropriate sense, with rate $\ep$. Moreover, we show  that  its limit is the spectrum of certain self-adjoint `bivariate'  operator, which is an abstract version of two-scale limit operator 
 and is related to the `homogenised' form $a^{\rm h}_\xi \,+\, b_0$ via Fourier transform (i.e. $\xi \,\mapsto\, \i\,\nabla$).  This bivariate operator is a second-order constant-coefficient differential operator in the Bochner space $L^2\left(\RR^n ;\, \overline{V_0} \right)$ and generalises  the two-scale %homogenised 
limit operators found for 
some high-contrast models via  two-scale convergence. Amongst other things, the spectrum of the bivariate operator is characterised in terms of certain operator-valued function (generalising in some way the Zhikov's $\beta$-function introduced in \cite{Zhi2000,Zhi2005}). This is turn provides an asymptotic characterisation, with error estimates, for gaps in  $\bigcup_{\t \in \Theta} \lambda_{\ep,\t}$ which in particular leads  to new estimates for the gaps in the Floquet-Bloch spectrum in specific examples. 

Section \ref{sec:examples} applies our abstract results to study various  problems of interest. We first  re-establish, in the new-light, some  known results in the classical setting, then provide some previously known and several new results for various high-contrast and some other degenerating problems. Each of the problems is picked not only for their wider relevance, but also to demonstrate a particular feature and breadth of the article's main assumptions and results.

\section{Problem formulation}
\label{sec:pf}
An abstract setup for the general class of problems under consideration in this article is as follows. 
Let  $H$ be a separable complex Hilbert space with a family of inner products $(\cdot,\cdot)_\t$ parametrised by $\t$ varying in a compact subset $\Theta$ of $\RR^n$, 
$n\geq 1$.  We assume throughout that the norms  $\| \cdot \|_\t : = (\cdot,\cdot)_\t^{1/2}$ are uniformly equivalent in $\t$, i.e. 
\begin{equation}
\label{as.b1}
\text{$\exists\, K > 0$ such that} \ \|u\|_{\t_1} \,\,\le\,\, K_{}\, \|u\|_{\t_2}, \quad \forall u \in H,\ \forall\, \t_1,\t_2 \in \Theta.
\end{equation}
We also suppose throughout that the inner products have the following structure
\begin{equation}\label{astructure}
(u,\tilde{u})_\t = a_\t(u,\tilde{u}) + b_\t(u,\tilde{u}), \qquad u,\tilde{u} \in H,
\end{equation} 
where $a_\t$ and $b_\t$ are non-negative sesquilinear forms\footnote{In a complex Hilbert space $H$, for a non-negative sesquilinear form 
%That is 
$\mathfrak{b}:H\times H\to\CC$, $\mathfrak{b}[u]:= \mathfrak{b}(u, u)$ is non-negative real $\forall u\in H$. 
%by definition, $\forall u_1, u_2, \tilde{u}_1, \tilde{u}_2\in H$, $\forall \lambda_1, \lambda_2, \mu_1, \mu_2\in \CC$, 
%$\mathfrak{b}(\lambda_1 u_1+\lambda_2 u_2, \tilde{u}_1)= \lambda_1\mathfrak{b}( u_1 , \tilde{u}_1)+\lambda_2\mathfrak{b}( u_2 , \tilde{u}_1)$, 
%$\mathfrak{b}(u_1, \mu_1 \tilde{u}_1+\mu_2 \tilde{u}_2)=\overline{\mu_1}\mathfrak{b}(u_1,  \tilde{u}_1)+\overline{\mu_2}\mathfrak{b}(u_1,  \tilde{u}_2)$ (with bar denoting the complex conjugate), and 
%$\mathfrak{b}[u_1]:= \mathfrak{b}(u_1, u_1)$ is non-negative real. 
It is then straightforward to see that $\mathfrak{b}$ is complex-Hermitian or 
 symmetric, with Cauchy-Schwarz and triangle inequalities held, i.e. $\mathfrak{b}(u,\tilde u) = \overline{\mathfrak{b}(\tilde u,u)}$, 
$|\mathfrak{b}(u,\tilde u)|\le \mathfrak{b}^{1/2}[u]\,\mathfrak{b}^{1/2}[\tilde u]$, 
$\mathfrak{b}^{1/2}[u+\tilde u]\le \mathfrak{b}^{1/2}[u] + \mathfrak{b}^{1/2}[\tilde u], 
 \ \forall\, u,\tilde u \in H$. 
(Here $\mathfrak{b}^{1/2}[u]:=\left(\mathfrak{b}[u]\right)^{1/2}$.) 
We shall also be using throughout the following simple implications, ``squared'' triangle inequalities: 
$\mathfrak{b}[u_1+u_2]\le 2\,\mathfrak{b}[u_1] + 2\,\mathfrak{b}[u_2]$, 
$\mathfrak{b}[u_1+u_2+u_3]\le 3\,\mathfrak{b}[u_1] + 3\,\mathfrak{b}[u_2]+3\,\mathfrak{b}[u_3]$, $\forall\,u_1$, $u_2$, $u_3\in H$. 
}
 on $H$. Furthermore, we assume that $a_\t$ is Lipschitz continuous with respect to $\t$ in the following sense: there exists   $L_{a} > 0$ such that
\begin{align}
\label{ass.alip}
&\big| a_{\t_1}(u,\tilde{u}) - a_{\t_2}(u,\tilde{u}) \big| \,\,\le\,\, L_{a} \big| \t_1 - \t_2 \big|\, {\|u\|_{\t_1}}{\|\tilde{u}\|_{\t_1}}, \quad \forall u,\tilde{u} \in H,\, \forall \t_1,\t_2 \in \Theta.
\end{align}
We consider a general class of problems reducible to the following common abstract variational form. 
For any given  $0<\ep <1 $, $\t \in \Theta$, and $f \in H^*$, 
\begin{equation}
\label{p1}
\left\{ \ \begin{aligned} & \text{find $u_{\ep,\theta} \in H$ such that} \\
& \ep^{-2} a_\t\left(u_{\ep,\theta} ,\tilde{u}\right) \,\,+\,\, b_\t \left(u_{\ep,\theta},\tilde{u}\right) \,\,\,=\,\,\, \l f,\tilde{u}\r, \quad \forall \tilde{u} \in H.  
\end{aligned}
\right.
\end{equation}
 Here $H^*$ is the dual space of $H$ and $\l\cdot,\cdot\r$ denotes the duality pairing. 
Clearly, for any fixed $\ep>0$, $\t \in \Theta$,
\begin{equation}\label{Aform}
A_{\ep,\t}(\cdot,\cdot) \,\,: =\,\, \ep^{-2} a_\t(\cdot,\cdot) \,+\, b_\t(\cdot,\cdot) 
\end{equation} is an equivalent inner product for $H$, and therefore problem \eqref{p1}  is well-posed by the Riesz theorem. \\
Our main aim is to establish asymptotic approximations  of the solution $u_{\ep,\t}$ with respect to small $\ep$ that are uniform in an appropriate 
sense in both $\t$ and $f$. 

For each $\t$, we introduce the set of ``degeneracy'' or the kernel of $a_\t$ 
\begin{equation}
\label{spaceV}
V_\theta : = \big\{ v \in H \,\,\, \big| \,\,\, a_\t[v] = 0 \big\},
\end{equation}
where henceforth  $\mathfrak{b}[v]:=\mathfrak{b}(v,v)$ for a sesquilinear form $\mathfrak{b}$. 
Notice that,  
as $a_\t$ is non-negative, 
% and $V_\t$ is its kernel, 
it immediately follows from Cauchy-Schwartz inequality that 
%\footnote{Indeed, if $v\in V_\t$ and $t\in \CC\backslash\{0\}$, 
%$a_\t[v+tu]/|t|=\overline{t}a_\t(v,u)/|t|+t a_\t(u,v)/|t|+|t|a[u]\geq 0$; passing in this inequality to the limit as (complex) $t\to 0$ implies $a_\t(v,u)=\overline{a_\t(u,v)}=0$.} 
%Indeed, 
\begin{equation}\label{Vkersesa}
a_\t(v,u) \,=\, a_\t(u,v)\, =\, 0, \quad \forall v \in V_\t,\,\,\, \forall u \in H.
\end{equation}
%$a_\t(v,u)=a_\t(u,v)=0$ for all $u\in H$. 
Further, since the sesquilinear form $a_\t$ is %non-negative and  
bounded with respect to the norm $\|\cdot\|_\t$, \eqref{Vkersesa} implies that $V_\t$ is a closed linear subspace of $H$. 
% one has the equivalent definition
%\begin{equation}\label{azeroonV}
%%a_\t(v,u) = 0, \qquad \forall v \in V_\t,\, \forall u \in H.
%V_\theta  = \{ v \in H \, | \,  a_\t(v,u) = 0\  \forall u \in H \big\}.
%\end{equation}
Let  $W_\theta$, another closed linear subspace of $H$, be the orthogonal complement of $V_\theta$ in $H$ with respect to the inner product $(\cdot,\cdot)_\t$: 
\begin{equation}
W_\t\,:=\,\big\{w\in H\,\,\, \big|\,\,\, (w,v)_\t=0, \,\, \forall v\in V_\t\big\}. 
\label{2.6-w}
\end{equation} 
The main assumption of the article is the following pointwise in $\t$ (spectral) {\bf  gap condition:}
\begin{equation}\tag{H1}
\label{KA}
\ \begin{aligned}
& \text{$\forall \,\t \in \Theta$, $\exists\, \nu_\theta>0$ such that  $\forall w \in W_\theta$ the inequality 
$a_\t [w] \,\,\ge\,\, \nu_\t \|w\|_\t^2\,\,$ holds}.
\end{aligned} 
\end{equation}
%Henceforth, we shall consider $\nu_\t$ to be the optimal positive constant, that is
%\begin{equation}
%\label{nut}
%\nu_\t : = \inf_{w \in W_\t \backslash \{ 0 \}} \frac{a_\t[w]}{\|w\|_\t^2}.
%\end{equation}
%$\t\in \Theta$ and $f\in H^*$. 
%  and 
% i.e.
%\begin{equation}
%\label{ass.acont}
%0 \le a_\t[u] \le K_a \|u\|_\t^2, \quad \forall u \in H,\, \forall \t \in \Theta,
%\end{equation} for some constant $K_a >0$. 
\begin{remark}
\label{spgap}
To clarify why \eqref{KA} can be called a spectral gap condition, notice that 
for every $\t\in\Theta$ the form $a_\t$ defines a non-negative bounded self-adjoint operator in $H$. Condition (H1) implies together with \eqref{astructure} that the spectrum of this operator is contained in 
$\{0\}\cup [\nu_\t,1]$, in particular if both $V_\t$ and $W_\t$ are nontrivial $(0,\nu_\t)$ is in the gap of the spectrum. 
In Section \ref{s:resolv} and in examples of Section \ref{sec:examples}, $a_\t$ will specify possibly unbounded self-adjoint operator in a bigger Hilbert space ${\cal H}$ into which $H$ is densely (compactly) embedded, for which \eqref{KA} will still be implying presence of a spectral gap. 
\end{remark}
\begin{remark}
		\label{r.oldkafromnewka}  
		In a wide class of examples (see Section \ref{sec:examples}) one can readily verify that the following further strengthening 
		(see Proposition \ref{prop.kaequiv}) of condition \eqref{KA} holds. 
		There exists $C >0 $ and a  non-negative sesquilinear form $c$, $\|\cdot\|_\t$-compact (see Section \ref{sec.newKA} for the precise definition) for all $\t\in \Theta$, such that
		\begin{equation}\tag{H1$^\prime$}
		\label{KA2.1}
		\|w\|_\t^2 \le C a_\t[w] + c[w], \quad \forall w \in W_\t, \; \forall \t \in \Theta.
		\end{equation}
In particular, we will see that  \eqref{KA2.1} is self-evident in the context of classical homogenisation. In Section \ref{sec.newKA}, we shall see that \eqref{KA2.1} does not only imply \eqref{KA} but has other important implications.
\end{remark}
\section{The case of a continuous $V_\t$ (uniform spectral gap)}\label{s:uniforma}
As we shall see, the asymptotics of the solution to \eqref{p1} crucially depend on certain continuity properties of $V_\t$ with respect to $\t$. We begin with the  simple case of the spectral gap $\nu_\t$ being uniform in $\t$ and then we shall characterise this condition in terms of the continuity of $V_\t$. 
\subsection{The case of uniform in $\t$ gap}
The spaces $V_\t$ and $W_\t$ are not only orthogonal with respect to $(\cdot,\cdot)_\t$,   they are also orthogonal with respect to $A_{\ep,\t}$ (cf. \eqref{Aform},  \eqref{astructure} and \eqref{Vkersesa}). 
 Furthermore, it is clear that the restriction of $A_{\ep,\t}$ to $V_\t$ is $b_\t$. Consequently, 
by choosing in \eqref{p1} as the test functions first $\tilde{u}=\tilde{v}\in V_\t$ and then 
$\tilde{u}=\tilde{w}\in W_\t$, the problem uncouples as follows: 
$u_{\ep,\t} = v_\t + w_{\ep,\t}$, where   $v_\t \in V_\t$ solves
\begin{equation}
\label{thmcontv.vprob}	b_\t\left(v_\theta,\tilde{v}\right) \,\,=\,\, \l f\,,\tilde{v}\r, \quad \forall \tilde{v} \in V_\theta,
\end{equation}
and $w_{\ep,\t} \in W_\t$ solves
\begin{equation}
\label{wprob} 
A_{\ep,\t}\left(w_{\ep,\t}\, , \widetilde{w}\right) \,=\, 
\ep^{-2}a_\t\left(w_{\ep,\t} \,, \widetilde{w}\right) \,+\,	b_\t\left(w_{\ep,\theta}\,,\widetilde{w}\right) 
\,\,=\,\, \l f\,,\widetilde{w}\r, \quad \forall \widetilde{w} \in W_\theta.
\end{equation}
Problems \eqref{thmcontv.vprob} and \eqref{wprob} are well-posed in their own right: in particular, \eqref{thmcontv.vprob} is well-posed as $b_\theta$ is coercive on 
$V_\theta$:  for all $v\in V_\t$, $b_\theta[v]=b_\theta[v]+a_\theta[v]=\|v\|_\theta^2$ (recalling also that both $V_\t$ and $W_\t$ are closed in $H$). 
 
Now, in a standard way, setting in \eqref{wprob}  $\widetilde{w}=w_{\ep,\t}$ and recalling the {spectral gap condition} \eqref{KA}, 
\[
A_{\ep,\t}[w_{\ep,\t}]\,=\, \l f,w_{\ep,\t}\r\,\,\le\,\, \| f\|_{* \t}\,\|w_{\ep,\t}\|_\t\,\,\le\,\, 
\| f\|_{* \t}\, \nu_\t^{-1/2}a_\t^{1/2}[w_{\ep,\t}]\,\,\le\,\,  \|f\|_{* \t}\,\, \nu_\t^{-1/2}\ep\, A_{\ep,\t}^{1/2}[w_{\ep,\t}], 
\]
where
\begin{equation}
\label{fstar}
\| f \|_{* \t} \,\,\,: =\,\, \sup_{u \in H \backslash \{0 \}} \frac{| \l f,u \r |}{{\|u\|_\t}}.
\end{equation} 
As a result, for $w_{\ep,\t}=u_{\ep,\t}-v_\t$, 
%$\| w_{\ep,\t}\|_{\t} \le \ep \nu_\t^{-1/2} \sqrt{A_{\ep,\t}[w_{\ep,\t}]}$. Therefore,
\begin{equation}\label{1}
A_{\ep,\t}[w_{\ep,\t}]\,=\,\ep^{-2} a_\t[w_{\ep,\theta}] + b_\t[w _{\ep,\theta}] \,\,\le\,\,    \ep^2\, \nu_\t^{-1}\| f\|_{* \t}^2, \quad \ \ \forall \ep>0. 
\end{equation}
 Moreover, another application of \eqref{KA} and \eqref{1} give 
\begin{equation}\label{1-2}
 \|w _{\ep,\theta}\|_\t^2\,\,\le\,\, \nu_\t^{-1}\,a_\t[w _{\ep,\theta}]\,\,\le\,\, \nu_\t^{-1}\ep^2 A_{\ep,\t}[w _{\ep,\theta}]\,\,
 \le  \,\,  \ep^4\, \nu_\t^{-2}\,\| f\|_{* \t}^2.
\end{equation} 
If the spectral gap is uniform in $\t$, 
regarding \eqref{thmcontv.vprob} as an approximate problem,  
\eqref{1} and \eqref{1-2} immediately provide  the following error estimates. 
\begin{theorem}\label{thm:contV}
%Fix $\ep >0$, $\theta \in \Theta$, $f \in H^*$. Let $u_{\ep,\theta} \in H$ be the  solution to problem \eqref{p1} and   $v_\theta $ the solution to \eqref{thmcontv.vprob}. 
Assume that \begin{equation}
\label{bddspec}
%\nu : =	\inf_{\theta \in \Theta} \nu_\t   >0,
\exists\, \nu >0\ \text{ such that }\ a_\t[w]\,\ge\,\,\nu \|w\|_\t^2, \quad \forall w \in W_\t, \, \forall \t \in \Theta.
\end{equation}
 Then %, for all $\ep>0$ and 
 for $u_{\ep,\theta} \in H$ the  solution to \eqref{p1} and   $v_\theta\in V_\t$ the solution to \eqref{thmcontv.vprob}, % the following inequalities hold:
\begin{align}
\label{errorcontinuouscase}
\ep^{-2} a_\t\left[u_{\ep,\theta} - v_\theta\right] \,+\, b_\t\left[u_{\ep,\theta} - v_\theta\right]  \,\,\,\le\,\,\,    \ep^2\, \nu^{-1}\,\| f\|_{* \t}^2, \\ 
\label{errorcontinuouscase2}
\left\| u_{\ep,\theta} - v_\theta \right\|^2_\t  \,\,\,\le\,\,\,    \ep^4\, \nu^{-2}\,\| f\|^2_{* \t}.
\end{align}
\end{theorem}
\begin{remark}\label{rem3.2}
Theorem \ref{thm:contV} also holds with $\Theta$ replaced by any of its subsets $\Theta'$ such that 
%for any subset $\Theta'$ of $\Theta$ such that 
assumption \eqref{bddspec} is satisfied only on $\Theta'$ rather than on the whole of $\Theta$. 
Also, the theorem and its proof remain valid for all $\ep>0$ (i.e. not only for $0<\ep<1$, as assumed above). 
%with $\Theta$ replaced by $\Theta'$. 
These simple observations will be useful for some of our subsequent constructions. 
\end{remark}
\begin{remark}
	\label{r.normfindependentoft}
Note that while the right-hand-sides of \eqref{errorcontinuouscase} and \eqref{errorcontinuouscase2} formally depend on $\t$, 
this dependence is easily removed by \eqref{as.b1} :
$
\| f \|_{* \t_1} \le {K} \| f \|_{* \t_2}, \ \forall \t_1,\t_2 \in \Theta.
$ 
Therefore \eqref{errorcontinuouscase} and \eqref{errorcontinuouscase2} provide desired error estimates for small 
$\ep$, which are uniform in both $\t$ and $f$. 
\end{remark}
\subsection{A characterisation of  forms $a_\t$ with uniform gap condition}
\label{sec.contV}
In applications,  the direct verification of \eqref{bddspec} can be complicated. An equivalent but often more transparent condition relies on a notion of continuity of the degeneracy subspace $V_\t$ in $\t$ that we shall introduce now. 
Namely, we say that $V_\t$ is Lipschitz continuous with respect to $\t$ on $\Theta$ if  
\begin{equation}
\label{VtLip}
\exists L_V >0\ \text{ such that }\ 
\forall \t_1,\t_2 \in \Theta, \ \forall v_{1} \in V_{\t_1}, \quad 
\inf_{v_{2} \in V_{\t_2}} \left\|v_{1} - v_{2}\right\|_{\t_2} \,\,\le\,\, L_V \left| \t_1 - \t_2 \right|\, \|v_{1}\|_{\t_1}.
\end{equation} 


%Note we clearly see from  \eqref{ass.acont} that $\nu_\t$ is bounded on $\Theta$:
%$\nu_\t \le K_a, \quad \forall \t \in \Theta.$
%Denoting   $P_{V_{\t}}$, $P_{W_{\t}}$ to be respectively the orthogonal projections onto $V_\t$, $W_\t$ with respect to the inner product $b_\t$; we 
By the identity $
\left\|P_{W_{\t_2}} v_{1}\right\|_{\t_2} = \inf_{v_2 \in V_{\t_2}} \| v_1 - v_2\|_{\t_2}
$, where $P_{W_\t} : H \rightarrow W_\t$ is the orthogonal projection on $W_\t$ with respect to $( \cdot,\cdot)_\t$,  the inequality in \eqref{VtLip} can be 
equivalently rewritten as 
\begin{equation}
\label{VtLip2}
\left\|P_{W_{\t_2}} v_{1}\right\|_{\t_2} \,\,\le\,\, L_V | \t_1 - \t_2 |\, \|v_1\|_{\t_1}, \quad \forall v_{1} \in V_{\t_1},  \ \forall \t_1,\t_2 \in \Theta.
\end{equation} 
The following result establishing, under assumption \eqref{KA},  the equivalence of the uniformity of the gap property \eqref{bddspec} and of the continuity of 
$V_\t$ property \eqref{VtLip} holds. An intuition behind is that the $\t$-continuity property \eqref{ass.alip} of $a_\t$ implies certain regular behaviour 
 of the related spectra, see 
Remark \ref{spgap}. So, as long as the spectral gap remains uniformly positive, the zero eigenspace $V_\t$ can vary with $\t$ only continuously, while if the uniformity is 
violated on $\t$ approaching a point $\t_0$ this can be only be due to an instant addition of a non-trivial subspace to $V_\t$ at $\t=\t_0$.
\begin{theorem}\label{thm.contVequiv} Assume \eqref{KA}. Then  \eqref{bddspec} holds if and only if  \eqref{VtLip} holds.
\end{theorem}
\begin{proof}
			\emph{ Proof of \eqref{bddspec} $\hspace{-5pt}\implies\hspace{-5pt}$ \eqref{VtLip} }. 
%			To prove \eqref{VtLip} it is sufficient to show that for arbitrary fixed $v_0 \in V_{\t_0}$  there exists $v_1 \in V_{\t_1}$ such that
%			\begin{equation}
%			\label{pcontV.e1}
%			b_{\t_0}[v_0-v_1] \le L_V | \t_0 - \t_1 | b_{\t_0}[v_0]. \end{equation}
Fix $\t_1, \t_2 \in \Theta$ and $v_1 \in V_{\t_1}$. By \eqref{bddspec}, the definition of $P_{W_{\t_2}}$, \eqref{Vkersesa} (first for $\t=\t_2$, 
$v=v_1-P_{W_{\t_2}}v_1\in V_{\t_2}$,  $u = P_{W_{\t_2}}v_1$, and then 
for $\t=\t_1$, $v=v_1$, $u = P_{W_{\t_2}}v_1$), and \eqref{ass.alip} we obtain  
\[
\begin{aligned}
\| P_{W_{\t_2}} v_1 \|_{\t_2}^2\,\, & \le\,\,\, \nu^{-1} a_{\t_2} [P_{W_{\t_2}} v_1]\,\,=\,\, \nu^{-1} a_{\t_2} \left( P_{W_{\t_2}}v_1, v_1\right) \,\,=\,\, 
\nu^{-1}  \Big( a_{\t_2} \left( P_{W_{\t_2}}v_1, v_1\right) -  a_{\t_1} \left( P_{W_{\t_2}}v_1, v_1\right) \Big) \\
&\le\,\,\, \nu^{-1} L_a |\t_1 - \t_2 | \| P_{W_{\t_2}} v_1 \|_{\t_1}\| v_1 \|_{\t_1}.
\end{aligned}
\]
Hence, after an application of \eqref{as.b1}, \eqref{VtLip2} holds with $L_V =  \nu^{-1} L_aK$ and 
therefore so does \eqref{VtLip}. 

{\emph{Proof of \eqref{VtLip} $\hspace{-5pt}\implies\hspace{-5pt}$	\eqref{bddspec}}}. Let us suppose that \eqref{bddspec} does not hold. Then there exists a convergent sequence $\t_n \in \Theta$ with limit $\t_0\in\Theta$, and a sequence $w_n \in W_{\t_n}$ such that $\| w_n \|_{\t_n} =1$ and $\lim_{n}a_{\t_n}[w_n] =0$. Now 
\begin{equation}\label{6520e}
1 \,=\,   \left\| w_n \right\|_{\t_n}^2 \,\,=\,\,  \left( P_{V_{\t_0} } w_n , w_n \right)_{\t_n}\,+\,\left( P_{W_{\t_0} } w_n , w_n \right)_{\t_n},
\end{equation}
where $P_{V_{\t_0}}$ is the orthogonal projector on $V_{\t_0}$ with respect to 
$(\cdot,\cdot)_{\t_0}$. We will show that each term on the right of \eqref{6520e} 
converges to zero. First we observe that by \eqref{as.b1}
\[
\left| ( P_{W_{\t_0} } w_n , w_n )_{\t_n}\right| \,\,\le\,\, \left\| P_{W_{\t_0}} w_n \right\|_{\t_n} \left\| w_n \right\|_{\t_n}\,\, =\,\, 
\left\| P_{W_{\t_0}} w_n \right\|_{\t_n} \,\,\le\,\, K\,\left\| P_{W_{\t_0}} w_n \right\|_{\t_0},
\]
and we claim that $\lim_n \| P_{W_{\t_0}} w_n \|_{\t_0} =0$. Indeed, by  \eqref{KA} for $\t=\t_0$, \eqref{Vkersesa} and \eqref{ass.alip} we have
\[
\begin{aligned}
\left\| P_{W_{\t_0}} w_n \right\|_{\t_0}^2\,\,\, & \le\,\, \nu_0^{-1}\, a_{\t_0}\left[ P_{W_{\t_0}} w_n\right] \,\,=\,\, \nu_0^{-1} a_{\t_0}\left[ w_n\right]  
\,\,\le\,\, \nu_0^{-1} a_{\t_n}\left[ w_n\right]  \,+\, \nu_0^{-1} L_a\, |\t_n - \t_0 |\, \left\| w_n \right\|_{\t_n}^2 \\
&=\,\,  \nu_0^{-1} a_{\t_n}\left[ w_n\right]  \,+\, \nu_0^{-1} L_a\, \left|\t_n - \t_0 \right|\,\to\,0\, \mbox{ as } n\to\infty.
\end{aligned}
\]
Thus the last  term in \eqref{6520e} converges to zero. On the other hand, by the definition of $P_{W_{\t_n}}$ and assumed \eqref{VtLip2} 
(for  $\t_1 = \t_0 $, $\t_2 = \t_n$ and $v_1=  P_{V_{\t_0}} w_n $) we compute
\[
\begin{aligned}
%\| P_{V_{\t_0}} w_n \|_{\t_n}^2 &=
\left|\left( P_{V_{\t_0}} w_n ,w_n\right)_{\t_n}\right|\,\,\,  &=\,\,
%|(  P_{W_{\t_n}} P_{V_{\t_0}} w_n ,w_n)_{\t_n} | \le \|P_{W_{\t_n}} P_{V_{\t_0}} w_n\|_{\t_n}  \| w_n \|_{\t_n}  \le L_V |\t_n - \t_0|\|P_{V_{\t_0}} w_n \|_{\t_0} \| w_n \|_{\t_n} .
\left|\left(  P_{W_{\t_n}} P_{V_{\t_0}} w_n ,w_n\right)_{\t_n} \right| \,\,\le\,\, 
\left\|P_{W_{\t_n}} P_{V_{\t_0}} w_n\right\|_{\t_n}\,  \left\| w_n \right\|_{\t_n} \,\,=\,\,  \left\|P_{W_{\t_n}} P_{V_{\t_0}} w_n\right\|_{\t_n} \\
& \le\,\, L_V |\t_n - \t_0|\,\left\|P_{V_{\t_0}} w_n \right\|_{\t_0}.
%\\&\le    L_V  |\t_n - \t_0| \| w_n \|_{\t_0} \| w_n \|_{\t_n}   \le  K  L_V  |\t_n - \t_0|  \| w_n \|_{\t_n}^2 =  K  L_V  |\t_n - \t_0|.
\end{aligned}
\]
Clearly, $\left\|P_{V_{\t_0}} w_n \right\|_{\t_0}  \le   \left\| w_n \right\|_{\t_0} \le K \left\| w_n \right\|_{\t_n}=K$, and therefore the first term on the righ hand side of \eqref{6520e} also converges to zero. Whence, we arrive at the contradiction in \eqref{6520e}, and so \eqref{bddspec} holds.
%
%On the other hand $\lim_n \| P_{V_{\t_0}} w_n \|_{\t_0} =0$. Indeed
%\[
%\begin{aligned}
%\| P_{V_{\t_0}} w_n \|_{\t_n}^2 &=( P_{V_{\t_0}} w_n ,P_{V_{\t_0}} w_n)_{\t_n} =( P_{V_{\t_0}} w_n ,w_n - P_{W_{\t_0}} w_n)_{\t_n} =( P_{V_{\t_0}} w_n , w_n)_{\t_n}  - ( P_{V_{\t_0}} w_n ,P_{W_{\t_0}} w_n)_{\t_n} \\
%& = ( P_{W_{\t_n}} P_{V_{\t_0}} w_n , w_n)_{\t_n}  - ( P_{V_{\t_0}} w_n ,P_{W_{\t_0}} w_n)_{\t_n} 
%\end{aligned}
%\]
%and 
%\[
%\|  P_{W_{\t_n}} P_{V_{\t_0}} w_n \|_{\t_n} \le L_V |\t_n - \t_0 |  \| P_{V_{\t_0}} w_n \|_{\t_0} \le  L_V |\t_n - \t_0 |  \| w_n \|_{\t_0} \le  L_V |\t_n - \t_0 | K  \| w_n \|_{\t_n} = K  L_V |\t_n - \t_0 | 
%\]
\end{proof}
\begin{remark}\label{rem.merelycont}
The above proof demonstrates that an analogous version of Theorem \ref{thm.contVequiv} can be proved if we merely require both $a_\t$ and $V_\t$ to be say 
H\"{o}lder continuous (rather than Lipschitz continuous) in $\t$, with appropriate 
modification of \eqref{ass.alip} and \eqref{VtLip}. 
Also, a `local' analogue of Theorem \ref{thm.contVequiv} holds, i.e. when in both \eqref{bddspec} and  \eqref{VtLip} $\Theta$ is replaced by its closed subset $\Theta'$. 
\end{remark}
\section{The case of discontinuous $V_\theta$ (non-uniform gap)}
\label{section:discV}
Typically, in applications (see Section \ref{sec:examples}) the space $V_\t$  violates \eqref{VtLip} and has isolated discontinuities.  
Moreover, these are special approximations near those discontinuity points which often characterise the key asymptotic properties of the solutions. 
Henceforth, we consider this situation and begin with an analysis in the neighbourhood of a given %interior  
point $\t_0\in \Theta$ that, without loss of generality, we consider to be the origin\footnote{We comment in passing that most of the analysis and results that follow extend in a straightforward manner to the case when the discontinuity set  is an arbitrary set of isolated %interior 
points. This is because most of our subsequent methods are local in nature.} $\t_0=0$.
\subsection{Local estimates}
While Theorem \ref{thm:contV} is not anymore directly applicable, the %method of proof 
idea behind it is. 
 The essential part of Theorem \ref{thm:contV}
was to identify two $\ep$-independent complementary subspaces of $H$, $V_\t$ and $W_\t$, that are orthogonal with respect to $A_{\ep,\t}$ whilst having $a_\t$ uniformly coercive on one of them. 
Staring with the latter, as 
a consequence of \eqref{KA} held at $\t=\theta_0=0$ together with the continuity of $a_\t$ due to \eqref{ass.alip}, $W_0$  remains available as a subspace of $H$ with the uniform coercivity condition still held 
in a small enough neighbourhood of $\theta_0$ :
% Indeed, the following result.

\begin{proposition}
	\label{Wt2coerc}  
Assume \eqref{KA}.
% that \eqref{as.b1},  \eqref{ass.alip} for $\t_1 = 0$ and \eqref{KA} for $\t =0$ hold. 
Then
% For  $|\theta|\leq \tfrac{1}{2}\nu_0L_a^{-1}$ the form $a_\t$ is uniformly coercive on $W_0$, that is
%	Namely one has
	\begin{equation}
	\label{C2}
	\frac{\nu_0}{2K^2}  \big\| w_0 \big\|_\t^2\,\,\, \le\,\,\, a_\t \left[w_0\right], \quad \forall w_0 \in  W_0, 
	\ \forall \t  \in \Theta\ \text{such that}\ \, |\theta|\,\le\, \tfrac{1}{2}\nu_0L_a^{-1}.
	\end{equation}
\end{proposition}
\begin{proof}
For $w_0\in W_0$, 
as follows from  \eqref{KA} (for $\theta=0$), \eqref{ass.alip}  
	(for $\theta_1=0$ and $\theta_2=\theta$) and $ L_a  |\t| < \tfrac{1}{2}\nu_0$, 
	\[
	\nu_0 \left\| w_0 \right\|_0^2  \,\,\,\le\,\, a_0[w_0] \,\,\le \,\, a_\t\left[w_0\right] \,+ \,
	L_a|\t|\left\| w_0\right\|_0^2\,\,
	\le\,\, a_\t[w_0]\,+ \,\,
	\tfrac{1}{2}\, \nu_0 \left\| w_0\right\|_0^2\,,
	\]
	which implies 
	\begin{equation}
	\label{coercv0t}
	\tfrac{\nu_0}{2} \| w_0 \|_0^2  \,\, \le\,\, a_\t[w_0], \quad \forall w_0\in W_0. 
	%, \quad \forall w_0\in W_0. 
	\end{equation} 
The latter,  along with \eqref{as.b1}, implies \eqref{C2}.
\end{proof}	
% In what follows,  $B_r : = \{ \t \in \Theta \, : \,  | \t| < r\}$. 
Turning now to the orthogonality issue, 
recall that, at $\theta=0$, $V_0$ is the orthogonal complement of $W_0$ with respect to $A_{\ep,\t}$. However for $\t\neq 0$, in general, the direct sum decomposition 
$H = V_0 \,\dot{+}\, W_0$ is not anymore orthogonal with respect to $A_{\ep,\t}$. Nevertheless, it is possible to partially rectify this as follows. The idea is, for small enough 
$\theta\neq 0$, to ``correct'' $V_0$ slightly to maintain the desired orthogonality at least to the main order in small $\ep$, i.e. with regards to the ``singular''  
part $a_\t$  
of $A_{\ep,\t}$. 
%correct this decomposition to make it orthogonal. 
To that end, given $v_0\in V_0$, seek a ``corrector'' $\N v_0 \in W_0$ such that for $\M v_0:=v_0+\N v_0$, 
\begin{equation}\label{IliaN2}
a_\t\left( \M  v_0 , w_0 \right) \,\,=\,\, 0,  \qquad \forall v_0 \in V_0, \ \ \forall w_0 \in W_0,
\end{equation}
i.e. so that $\M V_0$ and $W_0$ are ``orthogonal with respect to $a_\t$''. 
Equivalently, we seek  %$v_0 \mapsto \N v_0$ 
$\N v_0 \in W_0$ solving 
\begin{equation}\label{IliaN}
a_\t\left( \N  v_0 , \widetilde w_0 \right) \,\,= \,\,-\,\, a_\t\left(v_0,\widetilde w_0\right), \quad \forall\, \widetilde w_0 \in W_0. 
\end{equation}
The above problem is well-posed for $|\theta|\leq \tfrac{1}{2} \nu_0 L_a^{-1}$ by Proposition \ref{Wt2coerc}, 
and determines a linear map $\N  : V_0 \rightarrow W_0$.  
Let us show that the following estimate holds: 
%In particular
\begin{equation}\label{Nbound}
\left\| \N v_0\right\|_0\,\, \le\,\, 2{L_a}{\nu_0}^{-1} |\t|\, \left\| v_0\right\|_0, \quad \forall v_0 \in V_0, \ \    |\theta|\,\le\, \frac{1}{2} \nu_0\, L_a^{-1}. 
\end{equation} 
Indeed, it follows from   \eqref{coercv0t}, \eqref{IliaN},  \eqref{ass.alip} and \eqref{Vkersesa} (for $\t =0$):
\[
\tfrac{\nu_0}{2} \left\| \N v_0 \right\|_0^2 \,\le\, a_\t\left[\N v_0\right] \,=\, -\, a_\t\left(v_0,\N v_0\right) 
\,\le\, \big|a_0(v_0,\N v_0) \big|\, +\, L_a |\t| \left\|  v_0\right\|_0\, \left\| \N v_0\right\|_0 \,=\,  L_a |\t|\, \left\| v_0\right\|_0 \,\left\| \N v_0\right\|_0.
\]
Additionally, since 
$\N V_0 \subseteq W_0$, it readily follows that  $H = \M V_0 \dot{+}W_0$, i.e. $H$ is a direct sum\footnote{Indeed, for $u\in H$, with unique 
$\widetilde{v}_0\in V_0$ and $\widetilde{w}_0\in W_0$ such that $u=\widetilde{v}_0+\widetilde{w}_0$, seek 
${v}_0\in V_0$ and ${w}_0\in W_0$ so that $u=\M v_0+w_0= v_0+ (\N v_0+w_0)$. Hence $v_0=\widetilde{v}_0$ and $w_0=\widetilde{w}_0-\N \widetilde{v}_0$. 
Also, on this way, $\M v_0+w_0=0$ implies $v_0=0$ and $w_0=- \N v_0=0$.}of $\M V_0$ and $W_0$. 
Further, from the orthogonality of $V_0$ and $W_0$, and \eqref{Nbound}, 
\begin{equation}\label{Mpositive}
\left\| v_0\right\|_0 \,\,\le\,\, \left\| \M v_0\right\|_0\,\,\le\,\,2\,\left\| v_0\right\|_0, \quad \forall v_0 \in V_0, 
\ \    |\theta|\,\le\, \frac{1}{2} \nu_0 L_a^{-1}. 
\end{equation}
Notice also that \eqref{Mpositive} implies that $\M V_0$ is  closed in $H$.
Such properties allow us, for sufficiently small $\t$, to construct a desirable approximation to the solution $u_{\ep,\t}$ of variational problem \eqref{p1} by restricting it 
% $A_{\ep,\t}$ 
to $\M V_0$ which is ``almost orthogonal'' to $W_0$. Indeed, the following modification of Theorem \ref{thm:contV} holds.
\begin{theorem}
	\label{lmm1}
	Assume \eqref{KA}. Consider  $f \in H^*$, %$\ep \in (0, (2\sqrt{2}K)^{-1} \nu_0^{1/2})$,
	and $ \t \in \Theta$, $|\t| <  \nu_0/ (2L_a )$. %if $L_\star = 0$ or $r_0 = \min \{ \nu_0/ (2L_a ), 1/(KL_\star)\}$ otherwise. 
	Let $u_{\ep,\t}$ solve  \eqref{p1}, 
	and $v_0 \in V_0$ solve
	\begin{equation}
	\label{w1prob0}
	\ep^{-2} a_\t\big(  \M v_0 \,,\, \M \tilde{v} \big) \,\,+\,\, b_\t\left( \M v_0, \M \tilde{v}\right) \,\,=\,\, 
	\left\langle f,\, \M \tilde{v} \right\rangle, \quad \forall \tilde{v} \in V_0.
	\end{equation}
	Then problem \eqref{w1prob0} is well-posed, and the following error estimates hold:
	\begin{align}
	\label{ik43000}
	\ep^{-2}a_\t\big[u_{\ep,\theta} - \M v_0 \big]\,\,+\,\,	
	b_\t\big[ u_{\ep,\theta} -  \M v_0 \big]\,\,\, \le \,\,\,   8K^2 \nu_0^{-1}  \ep^2\,\|f\|_{*\t}^2, \\
	\label{s4ep2bd}
	b_\t\big[ u_{\ep,\theta} - \M v_0 \big]\,\,\,\le\,\,\, 16K^4 \nu_0^{-2}  \ep^4\,\|f\|_{*\t}^2.
	\end{align}
\end{theorem}  
\begin{proof}
The well-posedness of \eqref{w1prob0} follows e.g. from its left-hand side specifying an equivalent inner product on $V_0$, as implied by \eqref{Mpositive}. 
For the difference $r : = u_{\ep,\t} - \M v_0$, the left-hand-side of \eqref{ik43000} equals $A_{\ep,\t}[r]$ (see \eqref{Aform}) and expanding this out (and dropping the subscripts in notation) gives $A[r] = A\left(u_{\ep,\t},\, r\right) - A\left(\M v_0 , r\right)$. 
Note $u_{\ep,\t} = \M v + w$ for some unique $v \in V_0$  and $w \in W_0$, and hence $r = \M v_r  + w$ where $v_r := v - v_0 \in V_0$, and so 
\[
A[r]\,\, = \,\, A\left(u_{\ep,\t},\, r\right) \,-\, A\left(\M v_0 ,\, \M v_r\right) \,-\, A\left(\M v_0, w\right).
\]
Now,  \eqref{p1} gives $A\left(u_{\ep,\t},r\right) = \langle f, r\rangle $, \eqref{w1prob0} gives $A\left(\M v_0 ,\M v_r\right) = \left\langle f, \M v_r \right\rangle $  
and the 
almost-orthogonality due to \eqref{IliaN2} implies
% the orthogonality relation
%\begin{equation}\label{IliaN4.1}
%a_\t[\M v_0] + a_\t[w] = a_\t[\M v_0 + w],
%\end{equation}
% $a_\t(\M v_0,w) = 0$ which leads to 
 $ A\left(\M v_0, w\right)  = b_\t\left(\M v_0, w\right)$. Therefore,    
\begin{flalign*}
A[r]\, & \,=\,\,  \langle f, r \rangle \,-\, \left\l f, \M v_r\right\r -  b_\t(\M v_0, w)  \,\,=\,\, 
\langle f,  w \rangle \,-\, b_\t\left(\M v_0,  w\right) \,\,\le\,\, \Big( \| f\|_{*\t} \,+\, b_\t^{1/2}\left[ \M v_0\right] \Big) \|  w\|_\t,
\end{flalign*}
via \eqref{fstar}, and 
having applied  Cauchy-Schwarz inequality to $b_\t$ and then recalling \eqref{astructure}. Setting  $\tilde{v} = v_0$ in  \eqref{w1prob0} and recalling that $\ep <1$ implies 
$b_\t \left[\M v_0 \right] \le\, \left\| \M v_0\right\|^2_\t\, \le 
A\left[\M v_0\right] \,\le\, \| f\|_{*\t}\,\left\| \M v_0\right\|_\t \le 
\| f\|^2_{*\t}$ 
and as a result $ A[r] \le 2 \| f\|_{*\t} \|  w\|_\t$. 
So for proving \eqref{ik43000} one needs to bound $\| w\|_\t$ in terms of $A[r]$. 
Now  \eqref{C2} gives $\tfrac{\nu_0}{2K^2} \|  w \|_\t^2 \le a_\t[ w]$, and noticing that  $a_\t(\M v_r,w) = 0$ 
by \eqref{IliaN2} and $r =\M v_r + w$ implies 
\begin{equation}\label{IliaN4.1}
 a_\t[w]\,\, \le\,\, a_\t[r]. 
\end{equation}
Therefore, $
%\begin{equation}\label{IliaN5.1}
\tfrac{\nu_0}{2K^2} \|  w \|_\t^2   \le  a_\t[r]  \le \ep^2 A[r], 
%\end{equation} 
$ and as a result 
\begin{equation}\label{IliaN5}
\|  w\|_\t^2 \,\,\,\le\,\,\,  {2K^2}{\nu_0}^{-1} \ep^2 A[r], 
\end{equation}
implying \eqref{ik43000}.  \\
It remains to prove  \eqref{s4ep2bd}, whose left-hand-side equals $b_\t[ r]$. Then, from \eqref{IliaN5} and \eqref{ik43000} it suffices to show that 
\begin{equation}\label{IliaN6}
b_\t[r ]\,\,\le\,\, \| w\|_\t^2\,.
\end{equation} 
Since \eqref{w1prob0} is a restriction of \eqref{p1} to $\M V_0$, $r$ is orthogonal to $\M V_0$ with respect to $A$ (indeed setting $\tilde{u} = \M \tilde{v}$ in \eqref{p1} and subtracting \eqref{w1prob0}  gives $A\left(r,\M \tilde{v}\right) =0$ for any $\tilde{v} \in V_0$). Consequently, since $w = r - \M v_r$, one infers $A[r] \le A[w]$. This inequality along with \eqref{IliaN4.1} implies $b_\t[r] \le b_\t [w]$ which clearly implies  \eqref{IliaN6}, see \eqref{astructure}. The proof is complete. 
 \end{proof}
We finish this subsection with a comparison between Theorem \ref{lmm1} and Theorem \ref{thm:contV}.
 In the continuous case we restrict variational problem \eqref{p1} to the subspace $V_\t$, but  for the general (possibly discontinuous) case 
this may be not anymore sufficient for maintaining the same order of the approximation's accuracy, and 
we restrict instead  (locally near $\theta=\theta_0=0$) to $\M V_0$. 
% In the continuous case we restricted \eqref{p1} to the space $V_\t$ while general (possibly discontinuous) considerations provided the restriction to $\M V_0$. 
 From this
% perspective it is clear that the intersection $V_\t \cap \M V_0$ is non-trivial. In fact, it turns out that $V_\t$ is a subset of $\M V_0$.
 observation one may  expect that $V_\t$ is a subset of  $\M V_0$.  
Indeed, the following proposition holds. 
\begin{proposition}
\label{lemVtsubMV0} 
Assume \eqref{KA}. Let $\t \in \Theta$, $\,|\t| \le\, \nu_0 / (2L_a)$. Then
\begin{gather}\label{250520e2}
V_\t \,\,\subseteq\,\, \M V_0.   \\
\text{In fact, %therefore 
} \hspace{\textwidth} \nonumber \\
v_\t \,\,=\,\, \M P_{V_0} v_\t, \quad \forall v_\t \in V_\t.\label{250520e1}
\end{gather}
\end{proposition}
\begin{proof}
For \eqref{250520e2}, for any fixed $v_\t\in V_\t$ we need to find $v_0\in V_0$ such that $v_\t=\M v_0=v_0+\N v_0$. 
As $\N v_0\in W_0$, necessarily, $v_0=P_{V_0} v_\t$. Hence, for both \eqref{250520e2} and \eqref{250520e1}, it remains 
to show that $v_\t-P_{V_0} v_\t=\N P_{V_0} v_\t$. Clearly $v_\t-P_{V_0} v_\t=P_{W_0} v_\t\in W_0$, and 
%\eqref{IliaN} holds with $\N v_0$ replaced by $w_0$ as follows: 
\[
%a_\t\left(w_0,\widetilde w_0\right)\,\,=\,
\,a_\t\big(v_\t-P_{V_0} v_\t,\, \widetilde w_0\big)\,\,=\,\,-\,\,
a_\t\left(P_{V_0} v_\t, \widetilde w_0\right), %\,=\, -\,\,a_\t\left(v_0, \widetilde w_0\right), 
\quad \forall \widetilde w_0\in W_0 
\]
(having used \eqref{Vkersesa} for $v_\t\in V_\t$).  
Hence, by \eqref{IliaN}, $v_\t-P_{V_0} v_\t  %w_0=\N v_0
=\N P_{V_0} v_\t$, which completes the proof. 
\end{proof}
\begin{remark}\label{rem.s3.1}
	Observe that the proofs in this subsection only require that \eqref{KA} holds at %the discontinuity 
	$\t=0$, that \eqref{ass.alip} holds for $\t_1=0$ and that \eqref{as.b1} holds in a neighbourhood of $\t =0$. 
\end{remark}
\subsection{On the class of $V_\t$ with  a removable singularity}\label{ssRSing}
{ So far we have made no assumptions on the nature of the singularity of $V_\t$, and so 
	one could apply Theorem \ref{lmm1} %further 
	for any singularity (and even for any non-singular point). 
%	 assuming anything about the singularity 
%(only requiring technical assumptions as in Section \ref{sec.2dif}). 
However, in a large class of relevant examples 
(Section \ref{sec:examples}), the singularity of $V_\t$ has some additional structures which allow to 
significantly simplify the sought approximations further. 
%Indeed, the uniform asymptotics provided in Theorem \ref{lmm1}  are valid for arbitrary $V_\t$.
In the remainder of the article we mostly focus on developing Theorem \ref{lmm1} further  for such singularities. Namely, we assume that $\t_0=0$ is not an isolated point 
of $\Theta$ and 
$V_\t$ has a {\it removable singularity} at $\t_0$ in the following sense:
%$V_\t$ has a removable singularity in the following sense:
there exists a closed subspace $V_\star%\ne V_0
$ of $H$ and constant $L_\star\ge0$ such that (cf. \eqref{VtLip} )
%\[
%\Vs_\t = \left\{ \begin{array}{cc}
%V_\t & \t \neq 0, \\[.5em] V_\star & \t = 0,
%\end{array} \right. 
%\]
%is Lipshitz continuous, i.e. there exists $L_\star\ge0$ such that 
%\begin{equation}\tag{H2}
%\label{contVs}
%\inf_{v_2 \in \Vs_{\t_2}} \| v_1 - v_2 \|_{\t_2} \le L_\star | \t_1 - \t_2 | \| v_1 \|_{\t_1}, \quad \forall v_1 \in \Vs_{\t_1}, \ \forall \t_1, \t_2 \in \Theta.
%\end{equation}
\begin{equation}\tag{H2}
\label{contVs}
\Vs_\t \,\,:=\, \left\{ \begin{array}{cc}
V_\t & \t \neq 0, \\[.5em] V_\star & \t = 0,
\end{array} \right.   \quad \text{satisfies} \quad  
\inf_{v_2 \in \Vs_{\t_2}} \left\| v_1 - v_2 \right\|_{\t_2} \,\,\le\,\, L_\star | \t_1 - \t_2 |\, 
\left\| v_1 \right\|_{\t_1}, \quad \forall v_1 \in \Vs_{\t_1}, \ \, \forall\, \t_1, \t_2 \in \Theta,
\end{equation} 
or equivalently satisfies
\begin{equation}%\label{H2$^\prime$}
\label{contVs2}
\| P_{W^*_{\theta_2}}v_1  \|_{\t_2} \,\,\le\,\, L_\star | \t_1 - \t_2 |\, \left\| v_1 \right\|_{\t_1}, \quad \forall\, v_1 \in \Vs_{\t_1}, \ \, \forall \t_1,\, \t_2 \in \Theta, 
\end{equation}  
where $W^*_{\theta}$ is the orthogonal complement of $V^*_\theta$ in $H$ with respect to $(\cdot,\cdot)_\theta$. 
Note that \eqref{contVs} formally covers also the case without singularity when $V_\star=V_0$. 
\begin{remark}\label{constV}
	For a wide class of examples (cf. Section \ref{sec:examples}) $V_\theta$  is independent of $\theta$ away from the discontinuity, i.e. $V_\theta = V$ for $\theta \neq 0$. In this situation \eqref{contVs} trivially holds with $V_\star = V$ and $L_\star = 0$.
\end{remark}
% In preparation for what follows, we shall provide in this subsection a convenient alternative representation for $\M V_0$ based on \eqref{contVs}. \\
First, we observe that $
V_\star \subset V_0.
$
Indeed, for $v_\star\in V_\star$ and $\t\in\Theta$, $\t\neq 0$ and $\t\to 0$, by \eqref{ass.alip}, the definition of $P_{W_\t}$, \eqref{astructure} and  \eqref{contVs2} (for $\t_1 =0$ and $\t_2=\t$) we obtain 
\[
a_0[v_\star] \,=\, \lim_{\t \rightarrow 0} a_\t[v_\star] \,=\, \lim_{\t \rightarrow 0} a_\t\left[P_{W_\t^\star} v_\star\right] \,\,\le\,\,
  \lim_{\t \rightarrow 0}\, \left\|P_{W_\t^\star} v_\star\right\|_\t^2 \,\,\le\,\, \lim_{\t \rightarrow 0} L_\star^2  |\t|^2  \left\|  v_\star\right\|_0^2 \,\,=\,\, 0.
%, \quad \forall v_\star \in V_\star
\]
A key role in our subsequent constructions will be played by a ``defect'' subspace $Z$ of $V_0$, which characterises the 
discontinuity gap between $V_\star$ and $V_0$. Namely, let $Z$ be 
a closed linear subspace of $V_0$, such that 
\begin{gather}\label{spaceZ}
%Z= V_0 \cap W_\star.
V_0 \,\,=\,\, V_\star\, \dot{+}\,Z, \\
\textrm{and for some constant $0\leq K_Z<1$, } \hspace{\textwidth} \nonumber\\ 
\label{VZorth}
\big|(v_\star, z)_0 \big| \,\,\le\,\, K_Z \left\| v_\star \right\|_0 \left\| z\right\|_0, \qquad \forall v_\star \in V_\star, \,\, \forall z \in Z, \\ 
\textrm{or equivalently} \hspace{\textwidth} \nonumber\\ 
\label{VZorth2}
{\left(1-K_Z^2\right)}^{1/2}  \left\| v_\star\right\|_0  \,\,\le\,\, \left\| v_\star +z \right\|_0, \qquad \forall v_\star \in V_\star, \,\, \forall z \in Z.
\end{gather} 
%for some constant $0 \le K_Z < 1$.
\begin{remark}
\label{zorth}
Note that such $Z$ always exist, 
in particular $Z$  could  be the orthogonal complement of $V_\star$ in the Hilbert space $\big(V_0,\,(\cdot,\cdot)_0\big)$, 
in which case $K_Z=0$. 
In the regular case  $V_\star=V_0$, trivially $Z=\{0\}$. 
\end{remark}
Now we are ready to provide an alternative representation of $\M V_0$ in terms of  $V_\t^\star$ and $Z$, which is useful for a further simplification of the 
approximating problem \eqref{w1prob0} as the singular form $a_\t$ vanishes on $V_\t^\star$.  
The following technical 
lemma, important for our consequent constructions, holds. 
% that will be useful in Section \ref{sec.2dif}. 

\begin{lemma}\label{EquivdefcontVs}
	Assume \eqref{KA} and \eqref{contVs}. If  $ \t\in \Theta$ satisfies $KL_\star |\t| < \tfrac{1}{3}\left(1- K_Z\right)$ then 
\begin{gather}\label{V+ZCoercive}
\big\| v_\t^\star + z + w_0  \big\|_0^2 \,\,\,\ge\,\, \frac{1-K_Z}{3}\Big( \| v_\t^\star \|_0^2 \,+\, \|  z\|_0^2 \,+\, \| w_0\|_0^2 \Big) \,, \quad \forall v_\t^\star \in V_\t^\star,\,\, \forall z\in Z, \,\, \forall w_0 \in W_0\,;\\ 
\text{and if additionally $| \t | <   \nu_0/(2L_a)$ then } \hspace{\linewidth} \nonumber \\
	\label{frame}
	\M V_0 \,\,=\,\, V_\t^\star \,\,\,\dot{+}\,\, \M Z.  
%	\\
%\text{Furthermore, if $| \t | < \tfrac{1}{2} \nu_0L_a^{-1}$ then } \hspace{\linewidth} \nonumber \\
	\end{gather}
\end{lemma}
\begin{proof}[Proof of \eqref{V+ZCoercive}]
%{\it Proof of \eqref{V+ZCoercive}}. 
Set $\kappa_0 := 1-K_Z$, so $0<\kappa_0\leq 1$. Assumption \eqref{VZorth} and the fact that $W_0$ is orthogonal to 
$V_0 = V_\star \dot{+}Z$ imply  
\begin{equation}\label{kappa0}
 \big\|v_\star+z+w_0\big\|_0^2\,\,\ge\,\, \kappa_0\Big(\|v_\star\|_0^2\,\,+\,\|z\|_0^2\,\,+\, \| w_0 \|_0^2\Big), \quad  \forall v_\star \in V_\star,\, \,\forall z\in Z, \, \,\forall w_0 \in W_0.
\end{equation} 
Thus, for any $v_\t^\star \in V_\t^\star,\,  z\in Z$ and $ w_0 \in W_0$, with $W_*:=W_0^*$, 
\begin{flalign*}
 \big\|v_\t^\star+z+w_0\big\|_0^2\,\,\,&\ge\,\, \tfrac{1}{2}\left\| P_{V_\star}v_\t^\star+z+w_0\right\|_0^2\,-
\left\|P_{W_\star}v_\t^\star  \right\|_0^2 \,\,\ge\,\, \tfrac{\kappa_0}{2}\left( \left\|  P_{V_\star}v_\t^\star\right\|_0^2\,+\|z\|_0^2\,+\left\|w_0\right\|_0^2\right)\,-\,
\left\|P_{W_\star}v_\t^\star  \right\|_0^2\\
 &=\,\,\tfrac{\kappa_0}{2} \Big(\|v_\t^\star\|_0^2\,+\|z\|_0^2 \,+ \| w_0\|_0^2\Big)\,-\,\left(\tfrac{\kappa_0}{2}+1\right)\left\|P_{W_\star}v_\t^\star  \right\|_0^2\,.
\end{flalign*}
Now, \eqref{contVs2} (for  $\t_1 =\t$, $\t_2 =0$), \eqref{as.b1} and the assumption on $|\t|$ gives 
\[
(\tfrac{\kappa_0}{2}+1)\left\| P_{W_\star}v_\t^\star\right\|_0^2  \,\,\,\,\le\,\,\,\, 
\left(\tfrac{\kappa_0}{2}+1\right)\left(KL_\star|\t|\right)^2\, \left\| v_\t^\star\right\|_0^2\,\,\,\,\le\,\,\,\, 
\left(\tfrac{\kappa_0}{2}+1\right)\left(\tfrac{\kappa_0}{3}\right)^2\,\left\| v_\t^\star\right\|_0^2 \,\,\,\,\le\,\,\, \tfrac{1}{6}\kappa_0\, \left\| v_\t^\star\right\|_0^2,
\]
and \eqref{V+ZCoercive} follows. 


{\it Proof of \eqref{frame}}. The inclusion $V_\t^\star \subseteq V_\t$ and  \eqref{250520e2} show $V_\t^\star + \M Z \subseteq \M V_0$. Furthermore, \eqref{V+ZCoercive} for $w_0 = \mathcal{N}_\t z$ together with the closedness of $\M Z$ (following e.g. 
from \eqref{Mpositive} ) 
 implies that this sum is a direct sum and closed. 

	It remains to show that $V_\t^\star \,\dot{+}\, \M Z$ is not a proper subset of $\M V_0$.  If it were, there would exist a non-zero $v_0 =v_\star + z$, $v_\star \in V_\star$, $z \in Z$, such that $\M v_0 $ is orthogonal (with respect to $(\cdot,\cdot)_0$) to  $V_\t^\star\, \dot{+}\, \M Z$. 
Seeking a contradiction to this orthogonality, a natural choice of an element of $V_\t^\star \,\dot{+}\, \M Z$ expected to be close to 
$\M v_0 =\M v_\star + \M z$ is $u=P_{V_\t^\star} v_\star + \M z$. Since   \eqref{250520e1} gives
$P_{V_\t^\star}  v_\star  =\M P_{V_0}P_{V_\t^\star}  v_\star=\M P_{V_0}v_\star-\M P_{V_0}P_{W_\t^\star}  v_\star= 
\M v_\star-\M P_{V_0}P_{W_\t^\star}  v_\star$ (as $v_\star\in V_0$), 
%and, therefore, 
we conclude that $\M v_0 $ is orthogonal to $u=\M v_\star-\M P_{V_0}P_{W_\t^\star}  v_\star +\M z=
\M v_0-\M  P_{V_0}P_{W_\t^\star}v_\star$.
%	 Recalling that $\M = I + \N$ and ${\rm Ran}\, \N \subseteq W_0$ we see that $v_0$ is also orthogonal to $Z$. 
%In particular,  $\M v_0 =\M v_\star + \M z$ (where $v_\star \in V_\star$, $z \in Z$)  is perpendicular to $ P_{V_\t^\star} v_\star  + \M z$. 
Now, by \eqref{Mpositive} and the latter orthogonality, one has
%\[
%\| v_0\|_0 \le \| \M v_\star + \M z\|_0,
%\]
%and so
% and so (recalling \eqref{Mpositive})
\[
\left\| v_0 \right\|_0 \,\,\le\,\, \left\| \M v_0 \right\|_0  \,\,\le\,\, \left\| \M v_0- u\right\|_0
% = \| \M(v_\star - P_{V_0} P_{V_\t^\star} v_\star) \|_0 
\,\,=\,\,  \left\| \M P_{V_0} P_{W_\t^\star} v_\star\right\|_0.
\]
%where in the last equality we used $P_{V_0} P_{W_\t^\star}v_\star = v_\star - P_{V_0} P_{W_\t^\star}v_\star$. 
%This assertion  along with \eqref{Mpositive} gives $\| v_0 \|_0 \le \|  \M P_{V_0} P_{W_\t^\star} v_\star \|_0.$ 
Further, by the second inequality in  \eqref{Mpositive}, % (for $|\t| < \tfrac{1}{2} \nu_0 L_a^{-1}$), 
the properties of $P_{V_0}$, \eqref{as.b1}  and \eqref{contVs2} (for $\t_1 =0$, $\t_2 = \t$) %we compute 
\[
\left\| \M P_{V_0}P_{W_\t^\star}  v_\star \right\|_0  \,\,\le\,\, 2\, \left\| P_{V_0} P_{W_\t^\star}  v_\star \right\|_0 \,\,\le\,\,  
2\, \left\| P_{W_\t^\star}  v_\star \right\|_0
\,\,\le\,\, 2KL_\star |\t|\,  \left\| v_\star \right\|_0. 
%\le  2 K L_\star |\t|  \| v_\star \|_0.
\]
Moreover \eqref{VZorth2} gives $\left\|v_\star\right\|_0 \le \left(1- K_Z^2\right)^{-1/2} \left\| v_0 \right\|_0$. 
Consequently  $\left\| v_0 \right\|_0 \le 2 K L_\star |\t|\,  \left(1-K_Z^2\right)^{-1/2} \left\| v_0 \right\|_0$. 
This along with the assumed restriction on $\t$ and the inequality 
$\left(1-K_Z^2\right)^{-1/2} \le \left(1-K_Z\right)^{-1}$ lead to $\left\|v_0\right\|_0 =0$, which is a contradiction.
\end{proof}
We now present a global approximation  for the case of $V_\t$ with a removable singularity at $\t=0$. Let 
\begin{equation}\label{r0}
\text{ $r_0 =\nu_0 / (2L_a)$ if $L_\star = 0$ or $r_0 = \min\Big\{\,\nu_0 / \left(2L_a\right)\,,\, \left(1-K_Z\right)/ \left(3KL_\star\right)\,\Big\}$ otherwise,}
\end{equation}
and fix positive $r_1 \le r_0$. The direct sum representation \eqref{frame} %of Lemma \ref{EquivdefcontVs} 
and Theorem \ref{lmm1} show that, for $|\t | < r_1$, the solution $u_{\ep,\t}$  to \eqref{p1} is approximated in terms of the solution $v_0$ of the simplified problem \eqref{w1prob0} by 
$\mathcal{M}_\t v_0=v_{\ep,\t} \,+\, \mathcal{M}_\t z_{\ep,\t}$, with unique   
$v_{\ep,\t}\in V_\t^\star$ and $z_{\ep,\t}\in Z$. Recalling \eqref{Vkersesa}, we conclude that 
$\left(v_{\ep,\t}, \,z_{\ep,\t}\right)$ is the unique solution to 
\begin{equation}\label{coupledbest}
\ep^{-2} a_\t\big(  \M z_{\ep,\t},\, \M \tilde{z}\big) \,\,+\,\, b_\t\big(v_{\ep,\t} + \M z_{\ep,\t},\, \tilde{v} + \M \tilde{z}\big) \,\,=\,\, 
\left\langle f, \tilde{v} + \M \tilde{z} \right\rangle, \quad \forall \tilde{v} \in V^\star_\t,\, \,\forall \tilde{z}\in Z. 
\end{equation}
% $r_0 =\nu_0 / (2L_a)$ if $L_\star = 0$ or $r_0 = \min\{\nu_0 / (2L_a), (1-K_Z)/ (3KL_\star)\}$ otherwise. 
Furthermore, the solution $v_{\t}$ to \eqref{thmcontv.vprob} approximates $u_{\ep,\t}$ outside this neighbourhood of the origin: indeed \eqref{contVs} implies that $V_\t$ is Lipschitz continuous on the compact set $\Theta_{r_1} = \{ \t \in \Theta \, : \,  | \t | \ge r_1 \}$; therefore Theorem \ref{thm.contVequiv} 
(applied for $\Theta$ replaced by $\Theta_{r_1}$, cf. Remark \ref{rem.merelycont}) implies that the assumption of Theorem \ref{thm:contV} (namely \eqref{bddspec}) holds on $\Theta_{r_1}$ with a  positive constant
$
\nu(r_1) = \inf_{\t \in \Theta_{r_1}} \nu_\t.
$ More precisely, we have proved the following result:
\begin{theorem}\label{thm1.all}
	Assume \eqref{KA}--\eqref{contVs} and let $0 < r_1 \le r_0$ (see \eqref{r0}). Consider  $f \in H^*$,	$u_{\ep,\t}$ the solution to   \eqref{p1}  and an approximation 
 	\[
 u_{\ep,\t}^{(0)} : = \left\{ \begin{array}{ll}
 v_{\ep,\t} + \M z_{\ep,\t} & |\t| < r_1, \\
 v_{\t} & |\t| \ge r_1,
 \end{array} \right.
 \] 
where 
% $r_1 \le r_0 = \min\{\nu_0 / (2L_a), (1-K_Z)/ (2KL_\star)\}$,
 the pair $\left(v_{\ep,\t},\,  z_{\ep,\t}\right)\in V_\t^\star \times Z$ solves \eqref{coupledbest} and 
$v_\t \in V_\t=V_\t^\star$ solves \eqref{thmcontv.vprob}.
Then, the following error estimates hold for all $\t\in\Theta$ and $0<\ep<1$:
	\begin{align}
	\label{ik430}
	\ep^{-2}a_\t\left[u_{\ep,\theta} - u_{\ep,\t}^{(0)}\right]\,\,+\,\,	
	b_\t\left[ u_{\ep,\theta} -u_{\ep,\t}^{(0)}\right] \,\,\,\le \,\,\,   C_1\ep^2\,\|f\|_{*\t}^2, \qquad C_1= \max\left\{8K^2 \nu_0^{-1},\, 1/\nu(r_1)\right\},\\
	\label{ik430.ep2}
	b_\t\left[ u_{\ep,\theta} -  u_{\ep,\t}^{(0)}\right] \,\,\,\le \,\,\,   C_2\ep^4\|f\|_{*\t}^2, \qquad C_2= \max\left\{16K^4 \nu_0^{-2},\,1/\nu^2(r_1)\right\}.
	\end{align}
\end{theorem} }
We emphasise that the ``inner'' approximate problem \eqref{coupledbest} provides a significant further 
simplification compared to \eqref{w1prob0}, as its singular part $a_\t$ is now a form on the defect subspace $Z$ only. 
\section{Further refinements of Theorem \ref{thm1.all} }
\label{sec.2dif}
The results of Section \ref{section:discV} approximate the original problem by simpler ones, for example Theorem \ref{thm1.all} reduces problem \eqref{p1} on $H$ to those 
on the generally smaller subspaces $V_\t^\star$ and $Z$. However, the price to pay for this %reduction 
is the more complicated dependence on $\t$, in particular through the operator $\M$. The purpose of this section is to simplify the approximate problems further, in particular their 
dependence on $\t$, as much as possible under certain readily verifiable additional assumptions. 
%As we will see, t
In particular, 
this ultimately allows to approximate the original problem by one which in turn leads in Section \ref{s:resolv} to construction of an abstract version of a two-scale limit operator, possessing certain important properties which is in turn illustrated by new results for a number of examples in Section \ref{sec:examples}. 
% that are readily verifiable in a wide class of examples. 

{
We begin by noting that \eqref{V+ZCoercive} implies that $V_\t^\star$ and $Z$ form a closed direct sum in $H$ for small enough $\t$. Furthermore, one can see that $b_\t$ generates an equivalent norm on $V_\t^\star \,\dot{+}\,Z$. Indeed, since  $Z \subseteq V_0$, we use   \eqref{ass.alip} and  \eqref{V+ZCoercive} to obtain 
$a_\t[z] \,\le\, \tfrac{3L_a}{1-K_Z} |\t|\, \left\|  v_\t^\star + z\right\|_0^2$, and consequently, via \eqref{astructure}, 
\begin{equation}\label{btnormonV+Z}
\left\| v_\t^\star + z\right\|_\t^2 \,\,\le\,\, 2\, b_\t\left[v_\t^\star + z\right], \quad \forall v_\t^\star \in V_\t^\star, \,\,\forall z \in Z,\, \ \forall \t \in \Theta, \,\,\,
|\t| \,\le\, r_1\,  :=\, \min \left\{ r_0,\,  \tfrac{1-K_Z}{6K^2L_a}\, \right\}, 
\end{equation}
with $r_0$ given by \eqref{r0}. 
\subsection{Case of quadratically degenerating spectral gap width}\label{sec.quadnu}
Here, we additionally assume that there exists  $\gamma >0$ such that, for   $\nu_\t$ defined in \eqref{KA}, 
\begin{equation}
\tag{H3}\label{distance}
	\nu_{\t} \,\,\ge\,\, \gamma\, |\t|^2, \quad \forall\, \t \in \Theta, \quad \mbox{i.e. } \ a_\t[w]\,\,\ge\,\,\gamma\,|\t|^2\,\|w\|_\t^2, \quad \forall w\in W_\t. 
\end{equation}
This is a generalisation of the well-known property of %quadratic 
non-degeneracy of homogenised matrix in classical homogenisation. 
Condition \eqref{distance} allows us to characterise the non-degeneracy of the form $a_\t\left[\M\,\, \cdot\,\right]$ on $Z$.
% for non-zero $\t$.
\begin{proposition}\label{p.nondegZ}
	Assume \eqref{KA}--\eqref{distance} and $\t \in \Theta$, $|\t| < r_0$, for  $r_0$  as in Theorem \ref{thm1.all}, 
	see \eqref{r0}. Then 
\begin{equation}\label{150620.e1}
a_\t\left[\M z\right] \,\,\ge\,\,  \nu_\star|\t|^2\, \left\| z \right\|_0^2, \quad \forall z \in Z, \quad 
\mbox{ with } \, \nu_\star \,=\,  \tfrac{\gamma (1-K_Z)}{3K^2}.
\end{equation}
\end{proposition}
\begin{proof}
 It is enough to consider the case $\t \neq 0$. 	Then $W_\t = W_\t^\star$ and \eqref{KA} implies 
$
a_\t\left[\M z\right] = a_\t\left[P_{W^\star_\t}\M z\right] \ge 
%\nu_\t \| P_{W_\t}\M z\|_\t^2
 \nu_\t \left\| P_{W^\star_\t}\M z\right\|_\t^2.
$ Moreover,     \eqref{V+ZCoercive} for $w_0 = \N z$ and $v_\t^\star = -\,P_{V_\t^\star} \M z$ implies 
$
\left\| P_{W^\star_\t}\M z\right\|_0^2 \ge \tfrac{1-K_Z}{3} \| z \|_0^2.
$ 
Now, \eqref{150620.e1} readily follows upon recalling \eqref{as.b1} and \eqref{distance}.
\end{proof}
%
\begin{corollary} By \eqref{as.b1}, \eqref{Nbound} and \eqref{150620.e1} one has 
\begin{equation} \label{Zcoerc}
\| \N  {z}\|_\t^2 \,\,\le\,\, \kappa_1^2 a_\t[\M {z}], \quad \forall {z} \in Z, \quad
% \kappa_1 = \tfrac{2 {K L_a}}{\nu_0 \sqrt{\nu_{\star}}}, 
\kappa_1\,=\,  2\, {K L_a} \nu_0^{-1} \nu_{\star}^{-1/2}.
% \quad \nu_\star K^{-4}  \tfrac{1-K_Z}{2}\tfrac{\nu_0^2}{4L_a^2 }
\end{equation}
\end{corollary}
The next theorem demonstrates that the extra assumption \eqref{distance} and specifically its implication \eqref{Zcoerc} allow us to further simplify approximate problem \eqref{coupledbest} by removing $\M$ 
(i.e replacing it by unity operator) in both 
$b_\t$ and $f$ terms (but not in the $a_\t$ term). 

 \begin{theorem}\label{thm.trunc1}
	Assume \eqref{KA}--\eqref{distance}, and consider the objects as in Theorem \ref{thm1.all} with $r_1$ as in \eqref{btnormonV+Z}. Then, for each $\t \in \Theta,$	 $|\t| < r_1 $, there exists a unique solution  $v+z \in V_\t^\star \dot{+} Z$ to
\begin{equation}
\label{p.trunc1}
\ep^{-2} a_\t\big(  \M z, \M \tilde{z}\big) \,+\, b_\t(v + z, \tilde{v} + \tilde{z}\big) \,=\, \langle f, \tilde{v} + \tilde{z} \rangle, \quad \forall\, \ \tilde{v} +\tilde{z} \in V_\t^\star \dot{+}  Z.
\end{equation}
Furthermore, the following error estimates hold:
	\begin{align}
	\label{trunc1.e2}
	\ep^{-2}a_\t\big[u_{\ep,\theta} -(v+ \M z)\big]\,+\,	b_\t\big[ u_{\ep,\theta} -(v + \M z)\big] \,\, \le\,\,    C_3\ep^2\|f\|_{*\t}^2, \qquad C_3\, = \,2 C_1 + 12 \kappa_1^2.
%	\qquad C_2= \max\{8K^2 \nu_0^{-1},1/\nu(r_1)\},
	\\
	\label{trunc1.e3}
	\ep^{-2}a_\t\big[u_{\ep,\theta} -(v+ \M z)\big]\,+\,	b_\t\big[ u_{\ep,\theta} - (v+ z)\big] \,\,\le\,\,    C_4\ep^2\|f\|_{*\t}^2, \qquad 
C_4\,=\, 2C_3 +\kappa_1^2, 
	\end{align}
	where $C_1$ is given by \eqref{ik430} and $\kappa_1$ is given by \eqref{Zcoerc}. 
\end{theorem} 
%{\color{blue}
%Note that 
%\[
%\big((v,z ), (\tilde{v},\tilde{z})\big)_\t : = a_\t(\M z,\M \tilde{z})+	 b_\t(v + z, \tilde{v} + \tilde{z}\big) , \quad \forall\, (v,z) , (\tilde{v},\tilde{z}) \in V_\t^\star \times Z, 
%\]
%defines an inner product on $V_\t^\star \times Z$. Indeed, if $a_\t[\M z] +	 b_\t[v + z]=0$ then $\M z\in V_\t^\star\cap \M  Z=\{ 0\}$, so $z = 0$ and consequently $v =0 $. Now, by Riesz's theorem \eqref{p.trunc1} is well-posed. Set $ \vertiii{(v,z)}_\t : = \big((v,z ), (v,z)\big)_\t^{1/2}$,  for every element of $f\in H^\star$ define $\vertiii{f}_{\star\t} : = \sup_{(v,z)\in V_\t^\star \times Z} \frac{\langle f, v+z\rangle}{\vertiii{(v,z)}_\t}$.
%}
\begin{proof} For any $\ep>0$ and $\t\in\Theta$, the sesquilinear form 
%	left-hand-side of  \eqref{p.trunc1}, denoted by $B(v+z,\tilde{v}+\tilde{z})$,
\begin{equation}\label{Bform}
B(v+z,\tilde{v}+\tilde{z}) : = \ep^{-2} a_\t\big(  \M z, \M \tilde{z}\big) + b_\t(v + z, \tilde{v} + \tilde{z}\big), \quad  v, \tilde{v} \in V_\t^\star,\ z, \tilde{z} \in Z,
\end{equation} is bounded and coercive  in the Hilbert space $(V_\t^\star \dot{+} Z, (\cdot,\cdot)_\t)$ 
(see \eqref{btnormonV+Z}, \eqref{Mpositive} and \eqref{V+ZCoercive}) and so problem \eqref{p.trunc1} is well-posed.
Furthermore, setting $\tilde{v} = v$ and $\tilde{z} = z$ in \eqref{p.trunc1} and utilising \eqref{btnormonV+Z} gives $B[v+z]
%\ep^{-2} a_\t[ \M z] + b_\t[v + z] 
 \le  \| f\|_{*\t} \sqrt{2 b_\t[v + z]}$ from which we can readily deduce
\begin{equation}\label{18.09.20e1}
B[v+z]
%\ep^{-2} a_\t[ \M z] +  b_\t[v + z] 
 \le 2  \| f\|_{*\t}^2 , \quad \text{and} \quad \ep^{-2} a_\t[ \M z] \, \le \,
\|f\|_{*\t}\sqrt{2 b_\t[v + z]}\,\,-\,\,b_\t[v+z] \,\le \,
\tfrac{1}{2}\| f\|_{*\t}^2.
\end{equation}

{\it Proof of \eqref{trunc1.e2}.} Notice that the left-hand-side of \eqref{trunc1.e2} equals 
$A_{\ep,\t}\big[u_{\ep,\t} - (v+\M z)\big]$ (see \eqref{Aform}) and that Theorem \ref{thm1.all}, 
see \eqref{ik430}, states $A_{\ep,\t}\big[u_{\ep,\t} - (v_{\ep,\t} + \M z_{\ep,\t})\big] \le C_1 \ep^2  \| f\|_{*\t}^2$  for $v_{\ep,\t} + z_{\ep,\t}$ the solution to \eqref{coupledbest}. Thus,  it 
remains to bound $A_{\ep,\t}[ r_v + \M r_z ]$ where $r_v := v_{\ep,\t} - v$ and $r_z := z_{\ep,\t} - z$.
%is sufficient to show \begin{equation}\label{18.09.20e2}
%A_{\ep,\t}[ r_v + \M r_z ] \le   6 \kappa_1^2  \ep^2\| f\|_{*\t}^2, \quad \text{for $r_v = v_{\ep,\t} - v$ and $r_z = z_{\ep,\t} - z$. }
%\end{equation}
Subtracting  \eqref{p.trunc1} from \eqref{coupledbest} for $\tilde{v} = r_v$ and $\tilde{z}=r_z$ gives 
\[
\ep^{-2}a_\t[\M r_z]\,+\,b_\t\left(v_{\ep,\t}+\M z_{\ep,\t}, r_v\,+\, \M r_z\right)\,-\,b_\t(v+z, r_v+r_z)
\,=\,\langle f,\, \N r_z \rangle, 
\] 
which upon further direct calculation (and noticing $a_\t[\M r_z]=a_\t[r_v+\M r_z]$) yields
\[
A_{\ep,\t}\left[ r_v + \M r_z \right] \,=\, \langle f,\, \N r_z \rangle \,-\,
b_\t\left(\N z, r_v + \M r_z\right) \,-\, b_\t\left(v+ z,\, \N r_z\right).
%& \le  \| f\|_{*\t} \|  \N r_z \|_\t + b_{\t}[ \N z]^{1/2}b_\t[ r_v + \M r_z]^{1/2} + b_\t[ v+z]^{1/2} b_\t[\N r_z]^{1/2}
\]
From this identity, along with  \eqref{Zcoerc}, Cauchy-Schwarz inequality, and \eqref{astructure}, we obtain 
\begin{flalign*}
A_{\ep,\t}[ r_v + \M r_z ] & \le\,\,   \kappa_1 \left( \| f\|_{*\t} a_\t^{1/2}[ \M r_z] \,+\, a_\t^{1/2}[\M z]b_\t^{1/2}[ r_v + \M r_z]  \,+\,
b_\t^{1/2}[ v+z] a_\t^{1/2}[\M r_z] \right)  \\
& \le\,\, \ep\,  \kappa_1 \left( \| f\|_{*\t} A_{\ep,\t}^{1/2}[ r_v + \M r_z ] \,+\,
% ( \ep^{-2}a_\t[ \M z] +  b_\t[v + z]  )^{1/2} 
B^{1/2}[v+z] A_{\ep,\t}^{1/2}[ r_v + \M r_z ] \right).  
%\le 3 \kappa_1  \| f\|_{*\t} (A_{\ep,\t}[ r_v + \M r_z ]^{1/2}.
\end{flalign*}
(In the last inequality, along with the definitions \eqref{Aform} and \eqref{Bform} for $A_{\ep,\t}$ and $B$ respectively and the fact that 
$a_\t\left[r_v\right]=0$, 
discrete Cauchy-Schwarz inequality was also used.) 
%a simple algebraic inequality $(a_1b_2)^{1/2}+(b_1a_2)^{1/2} \le (a_1+b_1)^{1/2}(a_2+b_2)^{1/2}$, $\forall a_1,b_1,a_2,b_2\ge 0$ was also used.) 
This along with the first inequality in \eqref{18.09.20e1} gives %\eqref{18.09.20e2}, 
$A_{\ep,\t}[ r_v + \M r_z ] \le   6 \kappa_1^2  \ep^2\| f\|_{*\t}^2$,  
and \eqref{trunc1.e2} follows via \eqref{ik430} and the triangle-type inequality. 

{\it Proof of \eqref{trunc1.e3}}. From \eqref{trunc1.e2} we only need showing 
$b_\t[\N z]\,\le \, \tfrac{1}{2} \ep^2 \kappa_1^2 \| f\|_{*\t}^2$ 
% \tfrac{2 K^2 L_a^2}{\nu_\star \nu_0^2} \| f\|_{*\t}^2,
and this follows from \eqref{Zcoerc} and 
%\eqref{Nbound}, \eqref{150620.e1} and
 the second inequality in \eqref{18.09.20e1}. 
\end{proof}}
\subsection{Case of $a_\t$ with  additional regularity}\label{sec.atreg}
While \eqref{distance} was sufficient for removing $\M$ from $b_\t$ and the right-hand-side (cf. problems \eqref{coupledbest} and \eqref{p.trunc1}), in general one cannot remove $\M$  from $a_\t$. 
However in the majority of examples, Section \ref{sec:examples}, $a_\t$ has an additional regularity in $\t$ which allows one to approximate $a_\t(\M\, \cdot\,,\,\M\,\cdot)$ up to quadratic terms in small $\t$ and thereby further simplify problem  \eqref{p.trunc1}.

In this subsection we assume that $\theta_0=0$ is an interior point of $\Theta$, and 
$a_\t$ additionally satisfies the following ``differentiability'' properties with respect to $\t$ at $\t=0$. 
There exist sesquilinear maps $a'_0: V_0 \times H \rightarrow \CC^n$ and $a_0'': V_0 \times V_0 \rightarrow \mathbb{C}^{n\times n}$, 
i.e. vector-valued and matrix-valued maps respectively,
%{\rm Sym}_n (\CC)$, the space of symmetric square matrices over $\CC$, 
%: Z \times H \rightarrow \mathbb{C}^{d}$,  $a''_0: Z \times Z \rightarrow \mathbb{C}^{d\times d}$ and positive constants $K_{1a''}, K_{2a''}$ 
%{\color{red}remove symmetric} 
such that 
\begin{equation}\tag{H4}\label{H4}
\left\{ \hspace{.5em} \begin{aligned}
&\big| a_\t(v,u) - a'_0(v,u)\cdot \t \big| \,\,\le\,\, K_{a'} | \t|^2\, \|v\|_0\,\|u\|_0,  \qquad \forall v  \in V_0, \, \,\forall u\in H,\, \,\forall \t\in\Theta;\\
&\big|a_\t(v,\tilde{v}) - a''_0(v,\tilde{v})\t\cdot \t \big| \,\,\le\,\, K_{a''} |\t|^3\, \|v\|_0\,\|\tilde{v}\|_0,  \qquad \forall v, \tilde{v} \in V_0,\,\,\, \forall \t\in\Theta,
\end{aligned} \right.
\end{equation}
for some %positive 
non-negative 
constants $K_{a'}, K_{a''}$. Notice that \eqref{H4} and \eqref{ass.alip} gives  
$| a'_0(v,u) \cdot \theta| \le \left(L_a | \t|+K_{a'}|\t|^2\right) \|v\|_0\|u\|_0$ for all $\t\in\Theta$. 
Dividing this by $|\t|$ and taking 
the limit $\t\to 0$ yields
% there exist constants $K_{2a'},$ $K_{2a''}$ such that  
%\[
%\frac{| a'_0(v,u) \cdot \theta| }{| \t| \|v\|_0\|u\|_0} \le \frac{| a_\t(v,u) - a_0(v,u) | }{| \t| \|v\|_0\|u\|_0} + \frac{| a_\t(v,u) - a_0(v,u)- a'_0(v,u) \cdot \theta| }{| \t| \|v\|_0\|u\|_0} \le L_a + K_{a'}| \t|
%\]
\begin{equation}\label{a'a''cont}
%\left\{ \hspace{.5em} \begin{aligned}
| a'_0(v,u) \cdot \theta| \le L_a | \t| \|v\|_0\|u\|_0, \\
%& | a''_0(v,\tilde{v})\t\cdot\t | \le K_{2a''} |\t|^2 \|v\|_0\|\tilde{v}\|_0,
%\end{aligned} \right. \hspace{.94cm} 
\qquad \forall v \in V_0,\, \forall u \in H,\, \forall \t\in\mathbb{R}^n.
\end{equation}
Notice that the non-negativity of $a_\t$ implies 
%and in particular 
$a'_0(v,\tilde v)=0$, $\forall v,\tilde v\in V_0$. \\
We now demonstrate that assertion \eqref{H4} allows us to approximate $\N$ near $\t=0$ by some $N_\t$ which is linear in $\t$. 
To that end, in problem \eqref{IliaN} defining $\N$, approximate $a_\t$ on its left hand side by $a_0$ and $a_\t$ on 
the right hand side according to \eqref{H4} by $a_0'(v_0,\widetilde w_0)\cdot\t$. As a result, for each $\t\in \mathbb{R}^n$, 
we define  $N_\t: V_0 \rightarrow W_0$ so that $N_\t v$ for $v\in V_0$ is  a solution to 
\begin{equation}
\label{cell:prob2}
a_0( N_\t v , \widetilde{w}_0)\, =\, -\,\,a'_{0}(v, \widetilde{w}_0) \cdot \t, \qquad \forall \widetilde{w}_0 \in W_0.
\end{equation}
The unique solvability of \eqref{cell:prob2} is ensured by \eqref{KA}  and \eqref{a'a''cont}; in particular, one has  
\begin{equation}\label{Nbdd}
\| N_\t v \|_0\, \le \, {L_a}{\nu_0}^{-1}|\t| \,\, \| v\|_0, \quad \forall v \in V_0.
\end{equation}
As the right hand side of \eqref{cell:prob2} is linear in $\t$, $N_\theta v = \theta\cdot N v$, 
where  $N : V_0 \rightarrow [W_0]^n$ is a bounded linear mapping. 
The following proposition establishes closeness of $N_\t$ to $\N$ for small $\t$. 
\begin{proposition}\label{truncN}
Assume \eqref{KA}, \eqref{H4} and  $\t \in \Theta,\,\, |\t| \le \nu_0/(2 L_a)$. Then, the following inequality holds:
\begin{equation}\label{curlNtrunc}
\left\| \N v\,- \, N_\t v\right\|_0 \,\, \le\,\, \kappa_2 |\t|^2\, \| v\|_0, \quad \forall v\in V_0, \quad 
\mbox{ with } \, \kappa_2 = 
\nu_0^{-1} \big( 2L_a^2 \nu_0^{-1}  + K_{a'}\big).
\end{equation}

\end{proposition}
\begin{proof} For $R =\N v - N_\t v \in W_0$, by \eqref{IliaN} and \eqref{cell:prob2} we compute 
\[
a_0[R]  
%=a_0( \N v, R)   - a_0(   \t \cdot N v, R)  
\,\,=\,\, a_0\left( \N v, R\right) - a_0\left( N_\t v, R\right)
\,\,=\,\,   a_0\left( \N v, R\right) - a_\t\left( \N v, R\right)    -  a_\t( v, R) \,+\,  a'_{0}(v, R) \cdot \t.
\]
Now, \eqref{ass.alip} and \eqref{Nbound} give  $|   a_0( \N v, R) - a_\t( \N v, R) | \le  2 L_a^2 \nu_0^{-1} |\t|^2 \|  v\|_0 \| R\|_0 $, 
and the first inequality in \eqref{H4}  gives $\big|   a_\t( v, R) -  a'_{0}(v, R) \cdot \t\big| \le  K_{a'} |\t|^2\, \| v\|_0 \|R\|_0$. Therefore $a_0[R] \le \nu_0 \kappa_2 |\t|^2\| v\|_0 \|R\|_0$
%\[
%a_0[R] \le (2 L_a^2 \nu_0^{-1} |\t|^2 \|+  K_{a'} |\t|^2)\| v\|_0 \|R\|_0,
%\]
which along with \eqref{KA} gives \eqref{curlNtrunc}. 
%\begin{flalign*}
%\nu_0 \|  R \|_0^2 \le a_0[R] & = a_0(   \t \cdot N v, R)  - a_0( \N v, R)  =   -a'_{0}(v, R) \cdot \t  -  a_\t( \N v, R) +  a_\t( \N v, R) - a_0( \N v, R) \\
%& =  -a'_{0}(v, R) \cdot \t  +  a_\t( v, R) +  a_\t( \N v, R) - a_0( \N v, R).
%\end{flalign*}
%
%which along with \eqref{Nbound} proves the result for $C_{11} = K_{a'}/\nu_0 + 2L_a^2/ \nu_0^2 $. 
\end{proof}
Now, we are in a position  to further approximate $a_\t\left( \M v,\M \tilde v \right)$ as entering the approximations in e.g. 
Theorem \ref{thm.trunc1}, see \eqref{p.trunc1}. 
To that end, recalling first that $\M = I + \N$ and applying \eqref{IliaN2} for $v_0=v$ and 
$w_0=\N \tilde v\in W_0$, and similarly for  $v_0=\tilde v$ and 
$w_0=\N  v$, 
we observe  that $a_\t\left(\M v,\, \N \tilde{v}\right)=a_\t\left(\N v, \M \tilde{v}\right)=0$, and hence via 
$a_\t(v, \tilde v)=a_\t\left(\M v-\N v,\,\M\tilde v-\N\tilde v\right)$, 
\begin{equation}\label{amnorth}
a_\t(\M v, \M \tilde{v}) \,=\, 
a_\t( v,  \tilde{v})\,-\, a_\t( \N v, \N  \tilde{v}).
\end{equation}
Now, according to \eqref{H4} approximate $a_\t( v,  \tilde{v})$ by $a''_0(v,\tilde{v}) \t \cdot \t$, and  
$a_\t( \N v, \N  \tilde{v})$ by $a_0( N_\t v, N_\t \tilde{v})$. 
As a result, $a_\t(\M v, \M \tilde{v})$ is approximated by  the   sesquilinear form 
$a^{\rm h}_\t: V_0 \times V_0 \rightarrow \CC$, $\t \in \mathbb{R}^n$,  given by
\begin{equation}\label{defhom.form}
a^{\rm h}_\t(v,\tilde{v})\,\, : =\,\, 
a''_0(v,\tilde{v}) \t \cdot \t \,-\, a_0(  N_\t v, N_\t \tilde{v}) \,\,=\,\, 
a''_0(v,\tilde{v}) \t \cdot \t \,-\, a_0\big( \t \cdot Nv,\, \t \cdot N \tilde{v}\big), \quad \forall v, \tilde{v} \in V_0,
\end{equation}
which is a quadratic form in $\t$. We call $a_\t^h$ a ``homogenised'' form, for reasons to become clear later.  
The following proposition establishes the closeness of this approximation, and a $|\t|^2$-coercivity of $a^h_\t$ on $Z$. 
\begin{proposition}\label{prop.ahom}
		Assume \eqref{KA}--\eqref{H4}. Then, the following inequalities hold:
\begin{gather}
\label{atrunc.maine1}
\Big| a_\t(\M v , \M \tilde{v} ) \,-\, a^h_\t(v,\tilde{v}) \Big| \,\,\,\le\,\,\, \kappa_3\, |\t|^3\, \| v\|_0 \| \tilde{v} \|_0, \quad \forall v, \tilde{v} \in V_0, \,\, \,\forall \t \in \Theta,\,\, |\t| <  \nu_0/(2 L_a);\\
\label{ahcoercive}
a^h_\t[z] \,\,\,\ge\,\,\, \nu_\star |\t|^2\, \| z \|_0^2, \quad \forall z \in Z, \ \ \forall \t \in \mathbb{R}^n, 
\end{gather}
where $\kappa_3 =K_{a''} + {\nu_0}^{-1}{L_a}K_{a'} + L_a \kappa_2$, and $\nu_\star$ is given in  \eqref{150620.e1}. 
\end{proposition}
\begin{proof}[Proof of \eqref{atrunc.maine1}.]
%{\it Proof of \eqref{atrunc.maine1}.}	
From \eqref{amnorth} and \eqref{defhom.form}, 
\begin{flalign*}
a_\t(\M v , \M \tilde{v} ) - a^h_\t(v,\tilde{v})\,=\, 
\big[a_\t(v,\tilde v)-a''_0(v,\tilde{v})\t\cdot\t \big] \,+ \, 
\left[ a_0(  N_\t v, N_\t \tilde{v})-a_\t( \N v, \N  \tilde{v}) \right]. 
\end{flalign*}
Note that the second inequality in \eqref{H4} provides the desired estimate for  the first bracketed term on the right. Let us consider the second term: applying  \eqref{cell:prob2} and \eqref{IliaN}, 
\[
a_0(  N_\t v, N_\t \tilde{v})\,-\,a_\t( \N v, \N  \tilde{v})\,=\, 
-\, a_0'(v , N_\t \tilde{v})\cdot\t \,\,+\,\, a_\t(v , \N \tilde{v}) \,\, = 
\]
\[
\ \ \ \ \ \ \ \ \ \ 
\big\{a_\t\left(v , N_\t \tilde{v}\right)\,-\,a_0'\left(v , N_\t \tilde{v}\right)\cdot\t \big\} \, + \, 
a_\t\left(v , \N \tilde{v}-N_\t \tilde{v}\right). 
\]
By the first inequality in \eqref{H4} and \eqref{Nbdd} we obtain 
\[
|a_\t(v , N_\t \tilde{v}) - a_0'(v , N_\t \tilde{v})\cdot \t | \,\le\, K_{a'} |\t|^2 \|  v\|_0 \| N_\t \tilde{v} \|_0 
\,\le\,  {L_a}{\nu_0}^{-1} K_{a'}  | \t|^3 \| v\|_0 \| \tilde{v}\|_0.
\]
Also, by \eqref{ass.alip} and \eqref{curlNtrunc} we deduce that 
\[ %begin{flalign*}
|a_\t(v , \N \tilde{v}-N_\t \tilde{v})| \,\le\, L_a |\t| \| v\|_0 \|  \N \tilde{v} - N_\t \tilde{v}\|_0 \,\le\, L_a \kappa_2 |\t|^3 \| v\|_0 \| \tilde{v} \|_0. 
%= | a_0(\N v - \t \cdot N v,\N \tilde{v}) + a_0(\t \cdot N v,\N \tilde{v} - \t \cdot N \tilde{v})|   \\
% & \le \sqrt{ a_0[\N v - \t \cdot N v] }\sqrt{ a_0[\N \tilde{v}] } + \sqrt{ a_0[\t \cdot N v] } \sqrt{ a_0[\N \tilde{v} - \t \cdot N \tilde{v}] } \\
%& \le  \|\N v - \t \cdot N v \|_0 \| \N \tilde{v}\|_0 + \| \t \cdot N v\|_0  \| \N \tilde{v} - \t \cdot N \tilde{v}\|_0 \\
%&\le 3\tfrac{L_a}{\nu_0} \kappa_2 |\t|^3 \| v\|_0 \| \tilde{v}\|_0.
\] %end{flalign*}
Combining the above estimates yields \eqref{atrunc.maine1}. \\
%\sqrt{a_0[\N v - \t \cdot N v]} \sqrt{a_0[\N \til]} 
%\[
% a_\t( \N v,  \N \tilde{v}) -  a_0( \t \cdot N v ,  \t \cdot N \tilde{v}) = -  a_\t(v, \N \tilde{v}) + a_0'( v , \t \cdot N \tilde{v}) = -  a_\t(v, \N \tilde{v}) + a_0'( v , \N \tilde{v})   + a_0'( v , \t \cdot N \tilde{v} - \N \tilde{v}) 
%\]
{\it Proof of \eqref{ahcoercive}.}	  For fixed $z\neq 0$, $a^{h}_\t[z]$ as defined by \eqref{defhom.form} 
% and $\nu_\star|\t|^2 \tfrac{1-K_Z}{K^2} \| z\|_0^2$ are
is  quadratic in $\t$, and so for each fixed ``direction'' $\t|\t|^{-1}$ ($\t\neq 0$) 
  the ratio $ a^{h}_\t[z] /  \left(|\t|^2 \| z\|_0^2\right)$ is independent of $|\t|$. Moreover, we recall that $0$ in an interior point  of $\Theta$. So to prove \eqref{ahcoercive}, for a chosen $\t\neq 0$ we bound the ratio via passing to the limit as $|\t|\to 0$ along the corresponding direction and 
successively using  \eqref{atrunc.maine1} and   \eqref{150620.e1},  as follows:  
\[
\frac{a_{\t}^{h}[ z]}{|{\t}|^2\| z\|_0^2}  \,\,=\,\,
\lim_{|\t| \rightarrow 0}\frac{a_{\t}^{h}[ z]}{|{\t}|^2\| z\|_0^2}  
\,\,= \,\, \lim_{|\t|\rightarrow 0}  \frac{a_{\t}[\M z]}{|{\t}|^2\| z\|_0^2} \,\, \ge \,\, \lim_{|\t|\rightarrow 0}  \nu_\star\,=\, \nu_\star.\qedhere
\]
\end{proof}
Now, we are ready to further simplify approximate problem \eqref{p.trunc1}. 
\begin{theorem}\label{thm.maindiscthm}
	Assume \eqref{KA}--\eqref{H4}  and let $\t \in \Theta$, $\,|\t| < r_1$ for $r_1$ as in \eqref{btnormonV+Z}. Then, there exists a unique solution  $v^h+z^h \in V_\t^\star \dot{+} Z$ to
	\begin{equation}
	\label{z3prob}
	\ep^{-2} a^{\rm h}_{\t}\left(z^h,\, \tilde{z}\right) \,\,+\,\, b_\t\left(v^h+z^h,\, \tilde{v}+ \tilde{z}\right)\,\,\, =\,\,\, 
	\left\langle f,\, \tilde{v} +\tilde{z} \right\rangle, \quad \,\, \forall\, \tilde{v} +\tilde{z} \in V_\t^\star \dot{+}  Z.
	\end{equation}
	Furthermore, $v^h + (I + N_\t) z^h$ approximates $u_{\ep,\t}$, the solution to \eqref{p1}, in the following sense: 
	\begin{equation}
	\label{final1}
	\ep^{-2}a_\t\big[u_{\ep,\theta} - \left(v^h + (I+ N_\t) z^h\right)\big]\,+\,	
	b_\t\big[ u_{\ep,\theta} -\left(v^h + (I+N_\t)z^h\right)\big] \,\,\,\le\,\,\,    C_5\,\ep^2\,\|f\|_{*\t}^2,
	\end{equation}
	for $C_5 = 3 C_4 +3K^4 \kappa_2^2 \nu_\star^{-2} + \tfrac{3}{2}K^2 L_a^2 \nu_0^{-2}  \nu_\star^{-1}  + 3K^2 \kappa_3^2 \nu_\star^{-3}$.	 
	Moreover, $v^h +z^h$ approximates $u_{\ep,\t}$ as follows:
	\begin{equation}
	b_\t\big[ u_{\ep,\theta} -(v^h +z^h)\big] \,\,\,\le\,\,\,    C_6\,\ep^2\,\|f\|_{*\t}^2, \quad C_6 = 2C_4 + 2K^2 \kappa_3^2 \nu_\star^{-3}.
%	2C_5 + \tfrac{K^2 L_a^2}{ \nu_0^{2}  \nu_\star}.
 \label{final2}
	\end{equation}
\end{theorem}
\begin{proof}
Since $a^{\rm h}_\t$ is  bounded (e.g. via \eqref{defhom.form}, \eqref{H4} and \eqref{Nbdd}) 
and non-negative  on $Z$ (see Proposition \ref{prop.ahom}), by arguing as in the beginning of the proof of Theorem \ref{thm.trunc1}, it follows that \eqref{z3prob} is well-posed and
\begin{equation}\label{zhgoodnear0}
	\ep^{-2} a^{\rm h}_\t[z^h] \, \le \, \tfrac{1}{2} \| f\|_{*\t}^2.
\end{equation}
As a further preparation, we need a more refined estimate for $z^h$ rather than the one implied by e.g. \eqref{zhgoodnear0} with 
\eqref{ahcoercive}. 
To that end, we 
set in \eqref{z3prob} $\tilde{z}=z^h$ and  $\tilde{v}= -\,P_{V_\t^\star} z^h$, and note that 
$\tilde{v} + \tilde{z} = P_{W_\t^\star} z^h$  and 
(as for $v \in V_\t^\star$ and $w\in W_\t^\star$, $b_\t(v,w)= (v,w)_\t = 0$) 
that $b_\t \left(v^h + z^h, P_{W_\t^\star} z^h\right) = b_\t\left[P_{W_\t^\star} z^h\right]$. This  gives
\[
\ep^{-2} a^h_\t[z^h] + b_\t[P_{W_\t^\star} z^h] = \langle f, P_{W_\t^\star}z^h \rangle \,\le\, \| f\|_{*\t} \| P_{W_\t^\star}z^h \|_\t 
 \,\le\, \| f\|_{*\t}\, \|z^h \|_\t \,\le\, K  \| f\|_{*\t}\, \| z^h \|_0. 
\]
%Furthermore \eqref{btnormonV+Z} ( for $v_\t^\star = - P_{V_\t^\star} z^h$, $z = z^h$) gives $\| P_{W_\t^\star}z^h \|_\t^2 \le 2 b_\t[P_{W_\t^\star}z^h]$.  Therefore, we deduce that 
Along with \eqref{ahcoercive}  this yields  
\begin{equation}\label{100720.e2}\ep^{-2} |\t|^2 \| z^h \|_0 \,\le\,  {K}{\nu_\star}^{-1}  \| f\|_{*\t}.
% \quad \forall \t \in \Theta.
\end{equation}

{\it Proof of \eqref{final1}.} With the aim of exploiting \eqref{trunc1.e3}, 
decompose the argument in the square brackets on left-hand-side of \eqref{final1} in two slightly different ways as follows:
\begin{eqnarray*}
u_{\ep,\theta} - \left(v^h + (I+ N_\t) z^h\right)&=&\big[u_{\ep,\theta} - (v + \M  z)\big] \,+ \,
\left[\left(v-v^h\right) +\M \left(z-z^h\right)\right] \,+\, \left( \N z^h - N_\t z^h\right)= \nonumber \\
&&\big[u_{\ep,\theta} - (v +   z)\big] \,\,+ \,\,
\left[\left(v-v^h\right) + \left(z-z^h\right)\right]\,\, -\,\, N_\t z^h. 
\end{eqnarray*}
Applying e.g. a squared triangle inequality to the first of the above decompositions for the $a_\t$-term on the 
left-hand-side of \eqref{final1} and to the second decomposition for the $b_\t$-term results in bounding the whole 
left-hand side of \eqref{final1} 
%By \eqref{trunc1.e3}  
from above by
\[
3\Bigl(\ep^{-2}a_\t\big[u_{\ep,\theta} -(v+ \M z)\big]+	b_\t\big[ u_{\ep,\theta} - (v+z)\big] \Bigr) 
+  3B\left[\left(v-v^h\right) +\left(z-z^h\right)\right]
+3\ep^{-2} a_\t\big[\N z^h - N_\t z^h \big]+3b_\t[N_\t z^h],
\]
where $v+z$ solves \eqref{p.trunc1} and $B$ is given by \eqref{Bform}. By \eqref{trunc1.e3} the first term is bounded by  
$3C_4 \ep^2 \| f\|_{*\t}^2$.
 By \eqref{as.b1}, \eqref{curlNtrunc} and \eqref{100720.e2}, the third term is  bounded by $3K^4 \kappa_2^2 \nu_\star^{-2} \ep^2 \|f \|_{*\t}^2$. By \eqref{Nbdd}, \eqref{ahcoercive} and \eqref{zhgoodnear0}, the last term is bounded by 
$\tfrac{3}{2}K^2 L_a^2 \nu_0^{-2}  \nu_\star^{-1}  \ep^2 \| f\|_{*\t}^2$. 
  So it remains to bound the second term. By subtracting \eqref{z3prob} from \eqref{p.trunc1} 
	(both with $\tilde{v}=v-v^h$ and $\tilde{z}=z-z^h$)  we deduce that
\[
B\left[v-v^h +z-z^h\right]
%= \langle f, v+z -(v^h+z^h) \rangle - \ep^{-2} a_\t(\M z^h , \M (z-z^h) ) - b_\t(v^h+ z^h , v+z-(v^h+z^h) ) 
= \ep^{-2} a^h_\t(z^h,z-z^h)  - \ep^{-2} a_\t(\M z^h , \M (z-z^h) ).
\]
Now, by sequentially applying  \eqref{atrunc.maine1}, \eqref{150620.e1} and \eqref{100720.e2} we obtain  
\[
\begin{aligned}
B\left[v-v^h +z-z^h\right] & \,\le\, \ep^{-2}  | \t|^3  \kappa_3 \|z^h\|_0 \,\| z-z^h\|_0 \,  \le \, \ep^{-2}| \t|^2  \kappa_3 \|z^h\|_0 \,   \nu_\star^{-1/2 }a_\t^{1/2}\left[\M(z-z^h)\right] \\
& \le \, K \nu_\star^{-3/2} \kappa_3 \| f\|_{*\t}\,  a_\t^{1/2}\left[\M(z-z^h)\right] \,\le\,  \ep K \nu_\star^{-3/2} \kappa_3 \| f\|_{*\t}  B^{1/2}[v-v^h +z-z^h].
\end{aligned}
\] 
Thus 
\begin{equation}\label{280920e1}
 B\left[v-v^h +z-z^h\right] \, \le \, K^2 \kappa_3^2 \nu_\star^{-3} \ep^2 \| f\|_{*\t}^2
\end{equation}
 and \eqref{final1} follows by combining the above bounds.

{\it Proof of \eqref{final2}.} Since the left-hand-side of \eqref{final2} is bounded by 
$2b_\t\big[u_{\ep,\t} - (v+z)\big] + 2b_\t\big[v+z - (v^h+z^h)\big]$ then the desired inequality immediately follows from \eqref{trunc1.e3} and \eqref{280920e1}. 
\end{proof}
%We now present the main theorem of the section that provides a global uniform approximation to $u_{\ep,\t}$. 
%\begin{theorem}\label{thm.unifest}
%%	\label{thm1.all}
%	Assume \eqref{KA}-\eqref{H4}. Consider  $f \in H^*$,	$u_{\ep,\t}$ the solution to   \eqref{p1}  and $(v^h,z^h) \in V_\t^\star \times Z$  to \eqref{z3prob}. 
%Then, $v^h + (I +\t  \cdot N) z^h$ approximates $u_{\ep,\t}$ in following sense: 
%%\begin{equation}
%%	\label{final1}
%%\ep^{-2}a_\t[u_{\ep,\theta} - (v^h + (I+\t \cdot N) z^h)]+	b_\t[ u_{\ep,\theta} -(v^h + (I+\t\cdot N)z^h)] \le    C_5\ep^2\|f\|_{*\t}^2,
%%\end{equation}
%%for some positive $C_5$ independent of $\ep$, $\t$ and $f$. Moreover, $v^h +z^h$ approximates $u_{\ep,\t}$ as follows:
%%	\begin{equation}
%%		b_\t[ u_{\ep,\theta} -(v^h +z^h)] \le    C_6\ep^2\|f\|_{*\t}^2, \quad C_6 = \max\{ C_4,  r_1^{-2}  K^2\nu_\star^{-1}\big(1 +K^2  \nu_\star^{-1} L_a^2 \big)  \}. \label{final2}
%%	\end{equation}
%\end{theorem}
%\begin{proof}{\it Step 1)} Here we will show \eqref{final1} and \eqref{final2} hold for $|\t| < r_1$. Note that \eqref{final2} follows directly from \eqref{eq.maindiscthm}. 
%%	 Moreover, the identity $\M = I +\N$ and triangle inequalities show that the left-hand-side of \eqref{final1} is bounded by
%%\begin{equation*}
%%2\big( 	\ep^{-2}a_\t[u_{\ep,\theta} - \M  z^h)]+	b_\t[ u_{\ep,\theta} -(v^h + z^h)] \big)  + 2 \big( \ep^{-2}a_\t[ \N  z^h - \t \cdot N z^h]+	b_\t[ \t \cdot N z^h] \big).
%%\end{equation*}	
%	Moreover, \eqref{eq.maindiscthm} and the identity $\M = I +\N$ imply that \eqref{final1} follows if 
%\begin{gather}
%%		\ep^{-2}a_\t[ \N z^h - \t \cdot N z^h]\le     K^4 \kappa_2^2 \nu_\star^{-2} \ep^2 \|f\|_{*\t}^2;  \label{240720e.1}\\	
%%				b_\t[  \t \cdot N z^h] \le    \tfrac{K^2L_a^2}{\nu_\star \nu_0^2}\big( 1 +\tfrac{K^2 L_a^2}{\nu_\star}  \big)\ep^2\|f\|_{*\t}^2.\label{240720e.2}
%		\ep^{-2}a_\t[ \N z^h - \t \cdot N z^h] +  b_\t[  \t \cdot N z^h]  \le   C \ep^2 \|f\|_{*\t}^2,  \label{240720e.0}
%%		 \tfrac{K^2L_a^2}{\nu_\star \nu_0^2}\big( 1 +\tfrac{K^2 L_a^2}{\nu_\star}  \big)
%\end{gather}
%for an appropriate $C$. By \eqref{as.b1}, \eqref{curlNtrunc} and \eqref{100720.e2} we compute
%\begin{equation}\label{240720e.1}
%\ep^{-2}a_\t[ \N z^h - \t \cdot N z^h]  \le K^2 \ep^{-2} \| \N z^h - \t \cdot N z^h\|_0^2 \le K^2	\kappa_2^2 \ep^{-2} |\t|^4 \| z^h\|_0^2 \le \ep^2 K^4 \kappa_2^2 \nu_\star^{-2} \| f\|_{*\t}^2.
%\end{equation}
%It remains to bound $b_\t[\t\cdot Nz^h]$ which we shall do for all $\t$. Observe that $\| v^h + z^h \|_\t^2  =b_\t[ v^h + z^h ] + a_\t(v^h+z^h,z^h)$ and by \eqref{ass.alip}, one has  
%\begin{flalign*}
%  a_\t(v^h+z^h,z^h) \le  KL_a |\t| \| v^h+z^h\|_\t \|z^h\|_0 \le  \tfrac{1}{2}  \| v^h+z^h\|_\t^2 + \tfrac{1}{2} K^2L_a^2 |\t|^2 \|z^h\|_0^2,
%\end{flalign*} 
%thus
%%which along with \eqref{ahcoercive} gives
%\[
%%\| v^h + z^h \|_\t^2 \le 2b_\t[ v^h + z^h ] +  K^2L_a^2 \nu_\star^{-1} a^h_\t[z^h].
%\| v^h + z^h \|_\t^2 \le 2b_\t[ v^h + z^h ] +  K^2L_a^2 |\t|^2 \|z^h\|_0^2.
%\]
%Then, from the above inequality  and setting $(\tilde{v},\tilde{z}) = (v^h,z^h)$ in \eqref{z3prob} we deduce that
%\begin{flalign*}
%\ep^{-2} a^h_\t[z^h] + b_\t[v^h + z^h] &= \langle f, v^h+z^h \rangle \le \| f\|_{*,\t} \| v^h + z^h\|_\t \le  \| f\|_{*,\t}  ( \sqrt{2b_\t[ v^h + z^h ] } + \sqrt{K^2L_a^2 |\t|^2 \|z^h\|_0^2} ) \\
%& \le \tfrac{1}{2} \| f\|_{* \t}^2 + b_\t[v^h+z^h] +\tfrac{1}{2} K^2 L_a^2 \nu_\star^{-1}\ep^{2} \| f\|_{* \t}^2  +\tfrac{1}{2} \ep^{-2} \nu_\star  |\t|^2 \|z^h\|_0^2,
%\end{flalign*}
%which along with \eqref{ahcoercive} gives
%\begin{equation}\label{scon}
%\ep^{-2} \nu_\star  |\t|^2 \|z^h\|_0^2 \le \big( 1 +K^2 L_a^2 \nu_\star^{-1}\ep^{2}  \big) \| f\|_{*\t}^2, \quad \forall \t \in \Theta.
%\end{equation}
%Therefore,  \eqref{as.b1}, \eqref{Nbdd} and \eqref{scon} give
%$
%b_\t[  \t \cdot N z^h] \le    \tfrac{K^2L_a^2}{\nu_\star \nu_0^2}\big( 1 +\tfrac{K^2 L_a^2}{\nu_\star}\ep^2  \big)\ep^2\|f\|_{*\t}^2,
%$ 
%which along with  \eqref{240720e.1} gives \eqref{240720e.0} (upon recalling $\ep <1$).
%
%{\it Step 2)} We now show \eqref{final1} and \eqref{final2} hold for $|\t| \ge  r_1$.  Let us prove \eqref{final1}. Due to \eqref{ik430} it is enough to show 
%\begin{equation}\label{130720.e1}
%	\ep^{-2}a_\t[ v_\t  - (v^h+ z^h + \t \cdot N z^h)]+	b_\t[  v_\t  - (v^h+ z^h + \t \cdot N z^h) ]
%%	 +	b_\t[  v_\t  - (v^h+ z^h) ]
%\le 2 \tfrac{K^4}{r_1^2\nu_\star^2} (\tfrac{1}{r_1^2}+ \tfrac{L_a^2}{\nu_0^2} )\ep^2 \|f\|_{*\t}^2.
%\end{equation}
%Notice that by setting $\tilde{z} = 0$ in \eqref{z3prob} shows $P_{V_\t^\star}(v^h + z^h) = v_\t$ and so $v^h + z^h = v_\t + P_{W_\t^\star}z^h$. Therefore 
%\begin{flalign*}
%&	\ep^{-2}a_\t[ v_\t  - (v^h+ z^h + \t \cdot N z^h)]+	b_\t[  v_\t  - (v^h+ z^h + \t \cdot N z^h) ] \\
%&  = \ep^{-2}a_\t[  P_{W_\t^\star}z^h +  \t \cdot N z^h]+	b_\t[  P_{W_\t^\star}z^h + \t \cdot N z^h]  \le \ep^{-2} \| P_{W_\t^\star}z^h +  \t \cdot N z^h \|_\t^2  \le  2 K^2(1 +\tfrac{L_a^2}{\nu_0^2}|\t|^2) \ep^{-2}\| z^h \|_0^2,
%\end{flalign*}
%where the last inequality used \eqref{Nbdd}.
%%the left-hand-side of  \eqref{130720.e1} equals
%%$
%%\ep^{-2}a_\t[  P_{W_\t^\star}z^h +  \t \cdot N z^h]+	b_\t[  P_{W_\t^\star}z^h + \t \cdot N z^h]
%%% \le2 \tfrac{K^4}{r_1^2\nu_\star^2} (\tfrac{1}{r_1^2}+ \tfrac{L_a^2}{\nu_0^2} ) \ep^2 \|f\|_{*\t}^2.
%%$
%%which is bounded from above  by $ \ep^{-2} \| P_{W_\t^\star}z^h +  \t \cdot N z^h \|_\t^2$. This in turn is bounded by $ 2 K^2(1 +\tfrac{L_a^2}{\nu_0^2}|\t|^2) \ep^{-2}\| z^h \|_0^2  $ (cf. \eqref{Nbdd}). 
%Finally,  by 
% \eqref{100720.e2}, we note that  the last term is bounded by $2 K^4(1 +\tfrac{L_a^2}{\nu_0^2}|\t|^2)  |\t|^{-4}  \nu_\star^{-2}  \ep^2 \|f\|_{*\t}^2$. 
%% \[
%% \ep^{-2}a_\t[  P_{W_\t^\star}z^h +  \t \cdot N z^h]+	b_\t[  P_{W_\t^\star}z^h + \t \cdot N z^h]  \le 2 K^4(1 +\tfrac{L_a^2}{\nu_0^2}|\t|^2)  |\t|^{-4}  \nu_\star^{-2}  \ep^2 \|f\|_{*\t}^2.
%% \]
%Hence, \eqref{130720.e1} holds since  $|\t| \ge r_1$.
%
%For the proof of \eqref{final2}, we argue as above to see that we need only bound $b_\t[P_{W_\t^\star}z^h]$ from above. However, for $|\t| \ge r_1$, we see from \eqref{scon} that this is bounded by $r_1^{-2}\nu_\star^{-1}K^2\big( 1 +K^2 L_a^2 \nu_\star^{-1}\ep^{2}  \big) \ep^2 \| f\|_{*\t}^2$.  The proof is complete. 
%\end{proof}
\subsection{Case of continuous $b_\t$}
The last in our hierarchy of simplified problems, problem \eqref{z3prob}, has two main advantages: restriction to a smaller 
subspace $V_\t^*\dot{+}Z$, and replacement of the singular form $a_\t$ by the ``homogenised form'' $a_\t^h$  which has a quadratic dependence on $\t$ (as well as is restricted further to the defect subspace $Z$ only). The dependence of $b_\t$ on 
$\t$ remains so far unspecified, however as we will later see, if it were possible to approximate it for small $\t$ by a 
$\t$-independent $b_0$ that would provide significant additional benefits for properties of the approximate problem. 
In particular, as we will see in Section \ref{s:resolv}, such an approximate problem will 
motivate construction of an 
abstract version of a two-scale limit operator with important further implications. 


To that end, we make here the following additional assumption: $\Theta$ is connected, and $b_\t$ is Lipschitz continuous at $\t=0$ i.e. there exists $L_b\ge 0$ such that
\begin{equation}\label{H5}\tag{H5}
\big|b_\t(v, \tilde{v}) \,-\,b_0( v, \tilde{v} )\big| \,\, \le \,\, L_b |\t|\, \| v\|_0\, \| \tilde{v} \|_\t, \quad \forall v, \,\tilde{v} \in V_0,\,\, \forall\, \t \in \Theta.
\end{equation}
%This additional assumption will allow us to further simplify the $\t$-dependence of problem \eqref{z3prob}. 

First, we observe that \eqref{H5} implies 
%the following lemma generally holds, providing 
existence of a transfer operator $\mathcal{E}_\t$ which plays an important role by allowing to state the forthcoming approximate problem on $\t$-independent subspace 
$V_\star\dot{+}Z$. 
\begin{lemma}\label{propeth}
Conditions \eqref{contVs} and \eqref{H5} 
% ensure that $V_\t^\star$  and $V_\star$ are isomorphic, and condition \eqref{H5} ensure that one can find an isomorphism that is close to  identity near the origin.
imply $\forall\t\in\Theta$ 
%the existence of an isomorphism between $V_\star$ and $V_\t^\star$ that is close to  identity near the origin.
 existence of a bijection $\mathcal{E}_\t : V_\star \rightarrow V_\t^\star$ such that 
\begin{gather}
b_\t\left(\mathcal{E}_\t v,\mathcal{E}_\t\tilde{v}\right)\,\,=\,\,b_0( v,\tilde{v}), \quad \forall v,\tilde{v}\in V_\star,  \label{Eprop1} \\
\text{and} \hspace{\textwidth} \nonumber\\
\big\vert b_\t(\mathcal{E}_\t v,  z)\,-\,b_0( v,z)\big\vert \,\, \le \,\, K_b\, |\t| \, \| v\|_0 \, \| z \|_0, \quad \forall v \in V_\star, \forall z \in Z, \label{Eprop2}
\end{gather}
for some constant $K_b \ge0$ independent of $\t$. %\begin{remark}\label{constV}
%In the context of differential operators with periodic coefficients, if $V_\theta = V$ for $\theta \neq 0$ then $\mathcal{E}_\t$ is just simply multiplication by $exp(\i \t \cdot y)$, see Examples.
%\end{remark}
\end{lemma} 
In most of the relevant examples (Section \ref{sec:examples}), such a $\mathcal{E}_\t$ will be naturally identified. 
In its abstract form, a proof of the lemma is given in the Appendix A.  Notice that \eqref{Eprop2} is equivalent to 
\begin{equation}\label{Eprop3}
\big\vert b_\t(z, v')-\,b_0\left( z,\mathcal{E}_\t^{-1}v'\right)\big\vert\,\,\le\,\, K_b\, |\t|\, \| v'\|_\t \| z \|_0,\quad \forall v' \in V_\t^\star, \,\forall z \in Z.
\end{equation}
%\begin{proposition}\label{PVbi}
%	Assume \eqref{contVs}. Then $P_{V_\star} : V_\t^\star \rightarrow V_\star$ is a bijection when $K L_\star |\t| < 1$.
%\end{proposition}
%\begin{proof} Fix $v_\t \in V_\t^\star$. By \eqref{contVs}, \eqref{as.b1} we have  $  \| P_{W_\star} v_\t \|_0 \le K L_\star |\t| \| v_\t \|_0$ and so 
%	\begin{equation}\label{22.09.20e1}
%	\| P_{V_\star} v_\t \|_0^2 =\| v_\t \|_0^2  - \| P_{W_\star} v_\t \|_0^2  \ge (1 - ( K L_\star |\t| )^2) \| v_\t \|_0^2. 
%	\end{equation}
%	So $P_{V_\star}$ is injective and the range of $P_{V_\star}$ is closed. It remains to prove that $ P_{V_\star} V_\t^\star$ is not a proper subset of $V_\star$. Suppose there exists $0 \neq v_\star \in V_\star$ such that $ v_\star$ is orthogonal to  $ P_{V_\star} V_\t^\star$ with respect to the $(\cdot,\cdot)_0$.  Then, $v_\star = P_{V_\t^\star} v_\star  + P_{W_\t^\star} v_\star $ and $(v_\star , P_{V_\t^\star} v_\star)_0 = (v_\star , P_{V_\star}P_{V_\t^\star} v_\star)_0 = 0$. Furthermore, \eqref{contVs} and \eqref{as.b1}  gives $\|  P_{W_{_\t^\star}}v_\star \|_0 \le K L_\star |\t| \| v_\star \|_0$. Consequently,  we compute 
%	\[
%	\| v_\star \|_0^2 = (v_\star , P_{V_\t^\star} v_\star )_0 + ( v_\star , P_{W_\t^\star} v_\star)_0  = ( v_\star , P_{W_\t^\star} v_\star)_0\le  \| v_\star \|_0 \| P_{W_\t^\star} v_\star\|_0  \le K L_\star |\t| \| v_\star \|_0^2,
%	\]
%	which leads to the contradiction $\| v_\star \|_0 = 0$ for $K L_\star |\t|  <1$. Hence $P_{V_\star} V_\t^\star = V_\star$. 
%\end{proof}
Our aim is to simplify further the last approximate problem \eqref{z3prob} by stating it 
on the $\t$-independent subspace $V_\star \dot{+} Z$ instead of $V_\t^\star \dot{+} Z$, as 
well as approximating $b_\t$ by $b_0$. The former can be achieved via the above transfer 
operator $\mathcal{E}_\t$ by replacing in \eqref{z3prob} $v^h$ and $\tilde v$ 
(both in  $V_\t^\star$) by respectively $\mathcal{E}_\t v$ and $\mathcal{E}_\t\tilde v$, with both $v$ and $\tilde v$ now in $V_\star$. For the latter, the hope is to  use \eqref{H5}. 
Properties \eqref{Eprop1} and \eqref{Eprop2} of $\mathcal{E}_\t$ suggest that it can actually be dropped from the $b_0$-term, but not from the right-hand side, see the resulting simplified approximate problem \eqref{IKz3prob88} below. 
As a result, 
the following theorem providing an approximation to the original problem \eqref{p1} by the simplified problem holds, and is of special importance to us. 
\begin{theorem}\label{thm.IKunifest}
%	\label{thm1.all}
	Assume \eqref{KA}--\eqref{H5} and consider  $f \in H^*$, $\,\t \in \Theta$, 
	$\,|\t| <r_1$ for $r_1$ as in \eqref{btnormonV+Z}, and	$u_{\ep,\t}$ the solution to   \eqref{p1}, 
	and let $\mathcal{E}_\t: V_\star \to V_\t^\star$ be as in Lemma \ref{propeth} i.e. such that \eqref{Eprop1} and \eqref{Eprop2} hold. 
	Then, there exists a unique  solution $v + z \,\in\, V_\star \dot{+} Z$  to 
\begin{equation}
	\label{IKz3prob88}
	\ep^{-2} a^{\rm h}_{\t}(z, \tilde{z}) \,\,+\,\, b_0(v+z, \tilde{v}+ \tilde{z}) \,\,\,=\,\,\, 
	\left\langle f,\, \mathcal{E}_\t\tilde{v} +\tilde{z} \right\rangle, \quad \forall\, \tilde{v} + \tilde{z}\, \in\, V_\star \dot{+}  Z\,,
	\end{equation}
and, there exist  constants $C_7$ and $C_8$, independent of $\ep$, $r_1$, $\t$ and $f$, such that
	\begin{eqnarray}
			\label{IKfinal1}
\ep^{-2} a_\t\big[u_{\ep,\theta}  - \big(\mathcal{E}_\t v + (I+N_\t)z\big)\big] \,\,+\,\,
b_\t\big[u_{\ep,\theta}  - \big( \mathcal{E}_\t v + (I+ N_\t) z\big)\big] &\, \le \,& C_7\, \ep^2 \| f\|_{*\t}^2; \\
		b_\t\big[\, u_{\ep,\theta} \,-\,\left(\mathcal{E}_\t v +z\right)\,\big] &\, \le \,&    
		C_8\,\ep^2\|f\|_{*\t}^2.
		%\quad c =
		%\max\{ C_4,  r_1^{-2}  K^2\nu_\star^{-1}\big(1 +K^2  \nu_\star^{-1} L_a^2 \big)  \}. 
		\label{IKfinal2}
	\end{eqnarray}
\end{theorem}
\begin{proof} 
	The method of proof is quite similar to that of Theorem \ref{thm.maindiscthm}, and so we will be slightly less detailed. 
The assertions $b_0[\cdot] = \| \cdot \|_0^2$ on $V_\star \dot{+}Z$, \eqref{VZorth2}, and the  fact that 
$a^h_\t$ is bounded and non-negative 
on $Z$ %, see \eqref{ahcoercive}, 
 imply that the sesquilinear form given by the left-hand-side of \eqref{IKz3prob88}  is bounded and coercive on $V_\star \dot{+}  Z$ and hence, as $\mathcal{E}_\t$ is  clearly bounded, problem \eqref{IKz3prob88} is well-posed. Furthermore, taking in  
\eqref{IKz3prob88} $\tilde v=v$ and $\tilde z=z$, and using \eqref{ahcoercive}, \eqref{VZorth2} and \eqref{Eprop1}, and denoting by $C$ a positive constant independent of $\t, \ep$ and $f$ whose precise value may change from line to line, we first obtain 
%\begin{equation}\label{530est}
$\left(\ep^{-2} |\t|^2 + 1\right) \| z\|_0^2+\|v\|^2_0 \le C \|f\|_{*\t}\left(\|z\|_0+\|v\|_0\right)$. 
%\end{equation} 
This immediately bounds $\|v\|_0^2$ and  $\|z\|_0^2$ by $C \,\|f\|_{*\t}^2$, and as a result so also  $\ep^{-2} |\t|^2\|z\|_0^2$. 
We also obtain an additional estimate for $z$ analogous to \eqref{100720.e2} by taking in \eqref{IKz3prob88} 
 $\tilde{z}=z$ and  $\tilde{v}= -\,P_{V_\star} z$, concluding that $\ep^{-4} |\t|^4\|z\|_0^2$ is also  
bounded by $C \,\|f\|_{*\t}^2$, cf. the derivation of 
\eqref{100720.e2}. Combining the above estimates, we obtain 
\begin{equation} \label{est1234} 
\Big(\ep^{-4} |\t|^4 + \ep^{-2} |\t|^2 + 1\Big) \| z\|_0^2\,+\,\|v\|^2_0\,\,\le \,\, \kappa_4 \|f\|_{*\t}^2,
%	\quad c_1 =  (1+K^2) \max\{ \nu_\star^{-1}, 2(1-K_Z)^{-1} \}.
\end{equation}
with some $\kappa_4>0$ independent of $\t$, $\ep$ and $f$.  %For $|\t|$ away from the origin we see that\eqref{IKfinal1}, \eqref{final2} hold  from \eqref{est1234} that .
%For $|\t| <r_1$ \eqref{IKfinal1}-\eqref{IKfinal2}  will follow from Theorem \ref{thm.maindiscthm} if we bound
%\begin{equation}\label{18082e2}
%\ep^{-2}a_{\t}[\M z^h -(I+\t \cdot N) (z^h-z)] + b_\t[ v^h+z^h -(\mathcal{E}_\t v +z)]  +   b_\t[\t \cdot Nz] 
%\end{equation}
% for $(v^h,z^h)$ the solution to \eqref{z3prob}. 
% Now,
% \[
%  \ep^{-2}a_{\t}[\M z^h - (I+\t \cdot N) z] \le 2 \ep^{-2} a_\t[\M (z^h - z) ]+ 2\ep^{-2} a_\t[\N z - \t \cdot N z]
%  \]
%  and by \eqref{curlNtrunc} and \eqref{Nbdd}
%  \[
%  2\ep^{-2} a_\t[\N z - \t \cdot N z]+   b_\t[\t \cdot Nz]  \le2 K^2 \kappa_2^2 \ep^{-2} |\t|^4 \| z\|_0^2 + K^2 L_a^2 \nu_0^{-2} \| z\|_0^2,
%  \]
%Let $r_1$ be as in Theorem \ref{thm.maindiscthm}, then we have two cases to consider. 
%
%{\textit{Case 1:} $|\t| < r_1$} 
By Theorem \ref{thm.maindiscthm} to prove both \eqref{IKfinal1} and \eqref{IKfinal2} we only need bounding related difference terms:  
\begin{equation}\label{18082e2}
\ep^{-2}a_{\t}\big[ N_\t(z^h-z)\big] \,+\,  b_\t\big[ N_\t(z^h-z)\big]  \,+\,
 \ep^{-2}a_{\t}\big[z^h-z\big] \,+\,  b_\t\big[ v^h+z^h -(\mathcal{E}_\t v +z)\big],  
\end{equation}
where $v^h+z^h$ is the solution to \eqref{z3prob}. 
Now, (recalling $\ep <1$) by \eqref{Nbdd} and \eqref{ahcoercive}, 
\[
\ep^{-2}a_{\t}\big[  N_\t (z^h-z)\big] \,+\, b_{\t}[ N_\t (z^h-z)]  \,\,\le \,\, 
{K^2  L_a^2}{\nu_0^{-2} \nu_\star^{-1}} \ep^{-2}  a^{\rm h}_\t\big[z^h-z\big].
\]
Next, \eqref{IliaN2},  \eqref{atrunc.maine1} and \eqref{Nbound} first show that  
$\ep^{-2} a_\t\big[z^h-z\big] = \ep^{-2} a_\t\big[\M (z^h-z)\big] + 
\ep^{-2} a_\t\big[\N (z^h-z)\big]$ is bounded by a multiple of 
$\ep^{-2}a^{\rm h}_\t\big[z^h-z\big] +\ep^{-2} |\t|^3 \left\| z^h-z\right\|_0^2+
\ep^{-2} |\t|^2 \left\| z^h-z\right\|_0^2$. 
Then, via $|\t|^3\le \left(|\t|^2+|\t|^4\right)/2$ and bounding the resulting $|\t|^2$-terms via the $a^{\rm h}_\t$-term using 
\eqref{ahcoercive}, we conclude that $\ep^{-2} a_\t\big[z^h-z\big]$  is bounded by a multiple of 
\[
\ep^{-2}a^{\rm h}_\t\big[z^h-z\big] \,+\, \ep^{-2} |\t|^4 
\Big( \| z^h\|_0^2 \,+\, \|z\|_0^2\Big).
\]
Consequently, via \eqref{100720.e2} and \eqref{est1234}, we appropriately bound \eqref{18082e2} if we bound 
% we show 
\begin{equation}\label{18.08.20e1}
\ep^{-2} a^h_\t\big[z^h - z\big] \,+\, b_\t\big[ v^h+z^h -\left(\mathcal{E}_\t v +z\right)\big].
%leq  \kappa_5 \ep^2\|f\|_{*\t}^2,
\end{equation}
%for some $\kappa_5 >0$ independent of $\t, \ep$ and $f$. 
To this end, we shall demonstrate that replacing in \eqref{z3prob} $v^h+z^h$ by 
$\mathcal{E}_\t v +z$ produces a small error on the right-hand side. For $\tilde{v} \in V_\t^\star$ and $\tilde{z} \in Z$, utilising \eqref{Eprop1}, 
 we deduce 
\begin{flalign*}
b_\t\big(\mathcal{E}_\t v+z, \tilde{v}+ \tilde{z}\big) & = 
\,\,b_\t\big(\mathcal{E}_\t v,\, \mathcal{E}_\t \mathcal{E}_\t^{-1}\tilde{v}\big) \,+\,
b_\t\big(\mathcal{E}_\t v,  \tilde{z}\big)\,+\,b_\t\big(z, \tilde{v}+ \tilde{z}\big) \\
& =\,\,  
b_0\big( v,\,\mathcal{E}_\t^{-1}\tilde{v}\big) +b_\t\big(\mathcal{E}_\t v,  \tilde{z}\big)+
b_\t\big(z, \tilde{v}+ \tilde{z}\big)
\,\,=\,\,
b_0( v+z,\, \mathcal{E}_\t^{-1}\tilde{v}+\tilde{z})\,\, +\,\, J,
\end{flalign*}
where	$J=\big(b_\t(\mathcal{E}_\t v,  \tilde{z})-b_0( v,\tilde{z}) \big)+\big(b_\t(z, \tilde{v})-b_0( z,\mathcal{E}_\t^{-1}\tilde{v})\big)+\big( b_\t(z,  \tilde{z})
-b_0( z,\tilde{z})\big)$. Thus, via \eqref{IKz3prob88}, 
\begin{equation}
\label{IKz3prob89}
\ep^{-2} a^{\rm h}_{\t}(z, \tilde{z}) \,+\, b_\t(\mathcal{E}_\t v+z, \tilde{v}+ \tilde{z})
\,\, =\,\, \langle f,   \tilde{v} +\tilde{z} \rangle \,+J, \,\,\, \quad \forall\, (\tilde{v}, \tilde{z}) \in V_\t^\star \times  Z.
\end{equation}
Note $J$ is small. Indeed, \eqref{Eprop2}, \eqref{Eprop3}, \eqref{H5} and estimates \eqref{est1234} provide the following bound: 
	\begin{equation}\label{Jbound}
	|J|\,\le \, \kappa_5\ep \|f\|_{*\t}\Big(\ep^{-2} |\t|^2 \|\tilde{z}\|_0^2\,+\,\|\tilde{v}\|_0^2\Big)^{1/2}, \quad \,\, \forall \tilde{v} \in V_\t^\star,\, \, \forall \tilde{z} \in Z,
	\end{equation}
for some $\kappa_5 >0$ independent of $\t, \ep$ and $f$. 
Comparing \eqref{IKz3prob89}  with \eqref{z3prob}	we conclude that
\begin{equation}
\label{IKz3prob899}
\ep^{-2} a^{\rm h}_{\t}\big(z-z^h, \tilde{z}\big) \,+\, 
b_\t\big(\mathcal{E}_\t v+z-(v^h+z^h),\, \tilde{v}+ \tilde{z}\big) \,\, =\,\,J,\quad \  \forall\, (\tilde{v}, \tilde{z}) \in V_\t^\star \times  Z.
\end{equation}
Finally, to bound \eqref{18.08.20e1}, we can set in \eqref{IKz3prob899} $\tilde z=z-z^h$ and $\tilde v= \mathcal{E}_\t v-v^h$, 
and then use \eqref{Jbound} followed by \eqref{ahcoercive}, \eqref{V+ZCoercive} and \eqref{btnormonV+Z}. 
\end{proof}

We end this subsection %Section \ref{sec.2dif} 
by noting that one can  produce a global in $\t$  approximation to $u_{\ep,\t}$  by combining  an approximation for $|\t | <r_1$, given by Theorem \ref{thm.trunc1}, \ref{thm.maindiscthm} or \ref{thm.IKunifest}, with the approximation $v_\t$ for $|\t| \ge r_1$ given by Theorem  \ref{thm1.all} (with $1/\nu(r_1)  \le  \gamma^{-1} r_1^{-2}$, cf. \eqref{distance}). 
However, it turns out that the solution $v+z$ to \eqref{IKz3prob88} is well-defined  also 
% approximations for $|\t| <r_1$ make sense 
when $|\t| \ge r_1$, and can be seen to still approximate $u_{\ep,\t}$ up to leading order. % for this range of $\t$ as well. 
Such global approximations %might be less accurate but 
%possess certain specific advantages 
will play a vital role for some of our subsequent constructions in Section \ref{s:resolv}, as well as in some examples of Section \ref{sec:examples}. %We illustrate this point with 
Following theorem holds. 
\begin{theorem}\label{thm.IKunifest2}
	%	\label{thm1.all}
	Assume \eqref{KA}--\eqref{H5} and consider  $f \in H^*$, $\t \in \Theta$,	$u_{\ep,\t}$ the solution to   \eqref{p1}. Then, there exists a unique  solution $v + z \in V_\star \dot{+} Z$  to 	\eqref{IKz3prob88}, 
%	\begin{equation}
%	\label{IKz3prob88}
%	\ep^{-2} a^{\rm h}_{\t}(z, \tilde{z}) + b_0(v+z, \tilde{v}+ \tilde{z}) = \langle f, \mathcal{E}_\t\tilde{v} +\tilde{z} \rangle \quad \forall\, \tilde{v} + \tilde{z} \in V_\star \dot{+}  Z,
%	\end{equation}
	and there exist constants $C_{9}$ and $C_{10}$, independent of $\ep$, $\t$ and $f$, such that
	\begin{eqnarray}
	\label{IKfinal3}
	%\begin{aligned}
		\ep^{-2} a_\t\big[u_{\ep,\theta}  \,-\, \big(\mathcal{E}_\t v + (I+N_\t)z\big)\big] 
		\,\,+\,\, b_\t\big[u_{\ep,\theta} \, -\, \big( \mathcal{E}_\t v + (I+N_\t) z\big)\big]  % \hspace{1cm} \\  \hspace{2cm} 
		\,&\le& \, C_{9}\, \ep^2\, \| f\|_{*\t}^2,  \\ 
	\label{IKfinal3-2}	
		b_\t\big[ u_{\ep,\theta} \,-\, \left(\mathcal{E}_\t v +z\right)\big]
		\,&\le& \, C_{10}\, \ep^2 \,\| f\|_{*\t}^2.		
	%\end{aligned}
% \\
%	b_\t[ u_{\ep,\theta} -(\mathcal{E}_\t v +z)] \le    C_9\ep^2\|f\|_{*\t}^2.
%	%\quad c =
%	%\max\{ C_4,  r_1^{-2}  K^2\nu_\star^{-1}\big(1 +K^2  \nu_\star^{-1} L_a^2 \big)  \}. 
%	\label{IKfinal4}
	\end{eqnarray}
\end{theorem}
\begin{proof} Due to Theorem \ref{thm.IKunifest} we only need to consider the case $|\t| \ge r_1$. Note that the well-posedness of \eqref{IKz3prob88} and estimates \eqref{est1234} presented in the proof of Theorem \ref{thm.IKunifest} remain valid for all $\t \in %\mathbb{R}^n$. 
\Theta$. 
In particular, for $|\t| \ge r_1$, 
\eqref{est1234} implies 
\begin{equation}\label{281020e1}
\| z\|_0^2 \le \kappa_4 r_1^{-4} \ep^4 \| f\|_{*\t}^2,
\end{equation}
and so to prove \eqref{IKfinal3} and \eqref{IKfinal3-2} we only need bounding the difference 
$
	\ep^{-2} a_\t[u_{\ep,\theta}  - \mathcal{E}_\t v] + b_\t[u_{\ep,\theta}  - \mathcal{E}_\t v].
$
Now by recalling Theorem \ref{thm:contV} we see that it remains to bound the difference
 \[
 \ep^{-2} a_\t[ v_\t - \mathcal{E}_\t v] \, + \, b_\t[v_\t  - \mathcal{E}_\t v] \, =\, b_\t[v_\t  - \mathcal{E}_\t v], 
 \]
where $v_\t\in V_\t=V_\t^*$ solves \eqref{thmcontv.vprob}. 
 Setting $\tilde{z} = 0$ in \eqref{IKz3prob88}  and utilising \eqref{Eprop1} implies that $\mathcal{E}_\t v \in V_\t^\star$ solves
 \[
 b_\t(\mathcal{E}_\t v , \tilde{v}) = \langle f, \tilde{v} \rangle - b_0(z, \mathcal{E}_\t^{-1} \tilde{v}), \quad \forall \tilde{v} \in V_\t^\star. 
 \]
 Comparing this to  \eqref{thmcontv.vprob} and using \eqref{281020e1} implies  $ b_\t[v_\t  - \mathcal{E}_\t v]  \le\kappa_4 r_1^{-4} \ep^4 \| f\|_{*\t}^2$, completing the proof.
\end{proof}

\subsection{A strengthening of condition \eqref{KA} 
%	and its consequences
}
\label{sec.newKA}
In conclusion of this section, we provide a sufficient condition for \eqref{KA} which on the one hand is quite simple to  
verify for a broad class of examples, and on the other hand assures an important additional property of finite dimensionality 
of the defect subspace $Z$.  The latter provides a substantial further simplification, as 
  the singular form $a^h_\t[z]$ that appears in the approximate problem \eqref{z3prob}  can then be represented as a finite dimensional 
	matrix (the homogenised matrix). 

Recall \eqref{KA2.1} from Remark \ref{r.oldkafromnewka}, that is  there exists $C >0 $ and a  non-negative sesquilinear form $c$, $\|\cdot\|_\t$-compact\footnote{A form $c$ is $\|\cdot\|_\t$-compact if every sequence $\{u_n\}$, bounded in $\|\cdot\|_\t$, has a convergent subsequence $\left\{u_{n_k}\right\}$ with respect to $c$, i.e. $c\left[u_{n_k}-u \right]\to 0$ as $k\to \infty$ for some $u$.} for all $\t\in \Theta$, such that
\begin{equation}\tag{H1$^\prime$}
\label{KA2}
\|w\|_\t^2\,\,\, \le\,\,\, C a_\t[w] \,\,+\,\, c[w], \quad \forall w \in W_\t, \; \, \forall \t \in \Theta.
\end{equation}
The next result follows from standard arguments that we present here for the reader's convenience.
% show that this assertion is stronger than \eqref{KA}. 
%which we will present here for the readers convenience.
% (see Proposition \eqref{prop.kaequiv}). 
%The main result of this subsection is that \eqref{KA2} implies that $Z$ (see \eqref{spaceZ}) is a finite dimensional vector space. 
\begin{proposition}
	\label{prop.kaequiv}
Assertion \eqref{KA2} implies \eqref{KA}.
\end{proposition}
\begin{proof} Suppose  \eqref{KA} does not hold for some $\t \in \Theta$. Then there exists a sequence $w_n \in W_\theta$  such that $a_\t[w_n] < \tfrac{1}{n} \| w_n \|_\t^2$. 
Notice that \eqref{KA2} implies $c[w_n]>0$ for $n>C$, so we can assume $c[w_n]=1$. 
Then   \eqref{KA2} implies  $w_n$ is bounded in $H$. Consequently, 
up to a discarded subsequence, $\lim_n c[w_n-u]=0$ for some $u\in H$. 
Moreover (possibly up to another subsequence) $w_n$ weakly converges to some $w \in H$. 
The $\|\cdot \|_\t$-compactness of $c$ implies that $c$ is bounded in $H$ (i.e. $c[u] \le C'\|u\|^2_\t$, $\forall u\in H$,  
for some $C'>0$). 
%$w_n$ weakly converges, up to a discarded subsequence, to some $w \in H$. 
%Moreover, the $\|\cdot \|_\t$-compactness of $c$ implies that $c$ is bounded in $H$, and (possibly up to another subsequence) $\lim_n c[w_n-u]=0$ for some $u\in H$. 
Hence, $\forall\, \tilde u\in H$, $c(u-w, \tilde u)= \lim_n c(w_n-w,\tilde u)- \lim_n c(w_n-u,\tilde u)=0$, and so $c[u-w]=0$. 
Therefore $c[w] = c[u] = \lim_n c[w_n] = 1$. 
We  now demonstrate that $w\in W_\t \cap V_\t = \{0\}$ which contradicts the fact $c[w]=1$. Clearly $w \in W_\t$ since $W_\theta$ is weakly closed (being an orthogonal complement). On the other hand, since $a_{\t}$ is non-negative and bounded in $H$ (see \eqref{astructure}) it is clearly weakly lower semi-continuous, and therefore, $a_\t[w] \le \liminf_n a_\t[w_n]$ =0, i.e. $w \in V_\theta$. The proof is complete. 
% Hence $w=0$ which  
%contradicts $c_\t[w]=1$ and so \eqref{KA} must hold.
\end{proof}
One advantage of \eqref{KA2} is that 
%in practice, for a broad class of relevant examples, 
it provides a direct means to verify \eqref{KA}. 
We finally turn the other important implication of \eqref{KA2}, the finite dimensionality of the defect subspace $Z$.  
\begin{proposition}
	Assume \eqref{KA2} and \eqref{contVs}. Then any space $Z$ satisfying \eqref{spaceZ}-\eqref{VZorth} is finite dimensional.
\end{proposition}
\begin{proof}
	To prove $Z$ is finite dimensional we show first that since $c$ is $\|\cdot\|_0$-compact it is sufficient to prove that there exists some $0\neq \t \in \Theta$ and $\kappa >0$, such that 
	\begin{equation}
	\label{Zfinite}
	\|z\|_0^2 \,\,\,\le \,\,\, \kappa\, c\big[ P_{W_\t^\star}z\big], \quad \forall z \in Z.
	\end{equation}
%	for some positive $\kappa$ independent of $z$ and $\t$.
Indeed, for a bounded sequence $\{z_n\}$ in $Z$, $\{w_n\}:=\left\{P_{W_\t^\star}z_n\right\}$ is also bounded. 
Hence (up to a subsequence) 
 $c[w_n-u]\to 0$ for some $u\in H$, and so $c[w_n-w_m]\to 0$ as $m,n\to\infty$. 
Then \eqref{Zfinite} implies $\{z_n\}$ is a Cauchy sequence and hence 
 $z_n\to z\in Z$. 
So every bounded sequence in $Z$ has a convergent subsequence and hence $Z$ must be finite-dimensional. 

Let us now show \eqref{Zfinite}. Fixing $z \in Z$, for small enough $\t\in\Theta\backslash\{0\}$, 
by \eqref{V+ZCoercive} for $v_\t^\star = - P_{V_\t^\star} z$ and $ w_0 =0$, \eqref{as.b1} and \eqref{KA2},  we obtain  
	\begin{equation}
	\label{Zfiniteaddede1}
\tfrac{1}{3}(1-K_Z) \| z\|_0^2 \,\,\le\,\,  \left\| P_{W_\t^\star} z \right\|_0^2\,\,\le\,\, K^2 \left\| P_{W_\t^\star}z \right\|_\t^2 
\,\,\le\,\, K^2\, \big( C a_\t[z] \,+\, c[P_{W_\t^\star}z]\big).
	\end{equation}
	Now it remains to note that
%	 $a_\t[\M z] = a_\t[z] - a_\t[\N z]$ (see problem \eqref{IliaN}) and this along with
 \eqref{ass.alip} gives $a_\t[z] \le L_a |\t|\, \| z\|_0^2$.  Hence, for small enough $\t$, 
%one sees that 
\eqref{Zfiniteaddede1} implies \eqref{Zfinite}.
\end{proof}

\section{Approximations with uniform error estimates for related operators and their spectra} % of spectrum %for $\ep^{-2} a_\t + b_\t(\cdot,\cdot)$ 
%with uniform error estimates} 
\label{s:resolv}
In this section, we develop certain approximations for general classes of self-adjoint operators generated by the forms 
$A_{\ep,\t}\,=\,\ep^{-2} a_\t \,+\, b_\t$ and for their spectra, with uniform in $\t\in \Theta$ error estimates as $\ep\to 0$. \\
%For each $\t \in \Theta$, 
An abstract setup for wide classes of examples, see Section \ref{sec:examples}, is as follows. 
Let $\mathcal{H}$ be a complex separable Hilbert space with a family of uniformly equivalent inner products $d_\t$  
for each $\t \in \Theta$, i.e. 
\begin{equation}
\label{vs61}
d_{\t_1}[u] \,\,\le\,\, K_d\, d_{\t_2}[u], \quad \forall u \in \mathcal{H},\, \ \ \forall \t_1, \t_2 \in \Theta,  
\quad \text{ for some } K_d>0. 
\end{equation} 
Assume that 
%(\cdot,\cdot): \mathcal{H} \times \mathcal{H} \rightarrow \CC$, 
%such that 
$H$ is a compactly embedded\footnote{ That is, for any sequence $\{u_n\}\in H\subset \mathcal{H}$ with bounded $\|u_n\|_\t$, up to a subsequence, $d_\t[u_n-u]\to 0$ for some 
$u\in \mathcal{H}$.} dense subset of $\mathcal{H}$. Furthermore, we assume that 
\begin{equation}
\label{ik2}
d_\t[u] \,\,\le\,\, b_\t[u], \quad \forall u \in H,\, \ \forall \t \in \Theta.
\end{equation} 
%We characterise here the  $\t$-uniform  $\ep$-asymptotics of the spectrum of the self-adjoint operator  
Consider the self-adjoint operator $\mathcal{L}_{\ep,\theta}$  in $\mathcal{H}$ with inner product $d_\t$, generated 
according to the standard Friedrichs extension procedure by the (non-negative, closed, densely-defined) sesquilinear form $A_{\ep,\t}$ with the form domain $H$. 
Note that $\mathcal{L}_{\ep,\theta}$ has compact resolvent and therefore has a discrete spectrum which consists of 
%we are interested in analysing the asymptotic behaviour, as $\ep \rightarrow 0$, of the 
the sequence of positive real eigenvalues $\{\lambda_{\ep,\theta}^{(k)}\}_{k\in \NN}$ (which may accumulate only at infinity if $\mathcal{H}$ is infinite-dimensional) labelled in ascending order and repeated according to multiplicity:
\[
1 \le \lambda_{\ep,\theta}^{(1)} \le \lambda_{\ep,\theta}^{(2)} \le \ldots \le \lambda_{\ep,\theta}^{(k)} \le \lambda_{\ep,\theta}^{(k+1)} \le \ldots
\]
In this section we provide approximations to the operators $\mathcal{L}_{\ep,\theta}$ with corresponding quantitative asymptotic approximations for the eigenvalues $\lambda^{(k)}_{\ep,\t}$, %as well as for the associated eigenfunctions, 
for small $\ep >0$, that are uniform in $\t\in \Theta$. 

Throughout this section we use the following notation. For  a linear subset $\mathcal{U}$  of $ \mathcal{H}$,  $\overline{\mathcal{U}}$ denotes the closure of $\mathcal{U}$ in $\mathcal{H}$  and $\mathcal{P}_{\overline{\mathcal{U}}}^\t$ is the orthogonal  projection onto $\overline{\mathcal{U}}$ with respect to  $d_\t$.  
%Whenever  $\overline{\mathcal{U}}$ is considered to be an Hilbert space, it is equipped with the inner product $d_\t$. 
Where appropriate, we use the notation $(\mathcal{V},d%_\t
)$ to denote the Hilbert space formed by equipping the vector space 
$\mathcal{V}$ with the inner product $d%_\t
$. We denote the spectrum of a linear operator $\mathbf{L}$ by $\text{Sp}\, \mathbf{L}$. 

%Fix  $z=\{ z^{(m)}\}_{m=1}^M$ a basis for $Z$. 
\subsection{The case of continuous $V_\t$}\label{s.spcontV}
In this subsection we suppose that the assumption of Theorem \ref{thm:contV} holds.  Consider original problem \eqref{p1}  with the functional $f$ given by 
\begin{equation}
\label{ik3}
\left\l f,\tilde u\right\r  \,\,:=\,\,d_\t\left(g,\tilde u\right), \quad  \forall \tilde u \in H,
\end{equation}
for any  $g\in \mathcal{H}$. 
Notice that by \eqref{ik2} $f\in H^*$, and the solution $u_{\ep,\t}$ to \eqref{p1} is in the domain $ \text{dom}\,\mathcal{L}_{\ep,\theta}\subset H$ of operator $\mathcal{L}_{\ep,\theta}$ 
and $\mathcal{L}_{\ep,\theta}u_{\ep,\t}=g$. 
Then Theorem \ref{thm:contV}, in particular \eqref{errorcontinuouscase2}, along with  \eqref{ik2}, \eqref{ik3} and \eqref{fstar}, for the solution $v_\t$ of the approximate problem  
\eqref{thmcontv.vprob},  imply 
\begin{equation}
\label{ik4}
d_\t[u_{\ep,\t}-v_\t] \,\,\le\,\, \left\| u_{\ep,\t}-v_\t\right\|^2_\t 
\,\,\le\,\, \ep^4\nu^{-2} \| f\|_{*\t}^2 \,\,\le\,\,  \ep^4\,\nu^{-2} d_\t[g], \quad \, \forall \t \in \Theta.
\end{equation}
Let us give an operator-theoretic interpretation of \eqref{ik4}. 
Let $\mathbf{B}_\t$ be the self-adjoint operator in Hilbert space 
$(\overline{V_\t},d_\t)$,  
 generated by the closed positive sesquilinear form $b_\t$ with form domain $V_\theta$.
In particular,    
\begin{equation}
\label{ikddd} d_\t\left(\mathbf{B}_\t v,\tilde{v}\right)\,\,=\,\,b_\t\left( v,\tilde{v}\right),\quad  \forall v\in \text{dom}\, \mathbf{B}_\t , \ \ \forall \tilde{v}\in V_\t, 
\end{equation} 
where $\text{dom}\, \mathbf{B}_\t\subset V_\t$ is the domain of $\mathbf{B}_\t$. 
Then, via \eqref{ik3} and \eqref{ikddd}, the approximate problem \eqref{thmcontv.vprob} 
can be rewritten as 
$ \mathbf{B}_\t v_\t=\mathcal{P}_{\overline{V_\t}}^\t g$,  and so \eqref{ik4} can  
equivalently be re-stated as the following norm-operator estimate: 
\begin{equation}
\label{ik7}	\left\|\,\mathcal{L}_{\ep,\theta}^{-1}\,\,-\,\, \mathbf{B}_\t^{-1}  \mathcal{P}_{\overline{V_\t}}^\t\, \right\|_{(\mathcal{H},d_\t)\rightarrow (\mathcal{H},d_\t)}\,\,\,\le\,\, \,\ep^2\,\nu^{-1}.
\end{equation}
Next we observe that $\mathbf{B}_\t^{-1}  \mathcal{P}_{\overline{V_\t}}^\t $ is compact, non-negative and self-adjoint in $(\mathcal{H},d_\t)$.
Therefore, the %non-zero 
spectrum of $\mathbf{B}_\t^{-1}  \mathcal{P}_{\overline{V_\t}}^\t $  consists of real non-negative eigenvalues with only a possible accumulation point at zero. Let us put these %non-negative 
eigenvalues in descending  order:
\[
\alpha_{\theta}^{(1)} \geq\alpha_{\theta}^{(2)} \geq  \alpha_{\theta}^{(3)} \geq \ldots 
\]
(Here we are assuming for definiteness  
that $\text{dim }\mathcal{H}=\infty$.) 
Now, the key standard step is in noticing that the operator estimate \eqref{ik7} implies similar estimates for the spectra via the 
 min-max principle (see e.g. \cite{ReeSim}). Namely, uniformly for all 
 $k\in \mathbb{N}$ and $\t \in \Theta$, 
\begin{equation}
\label{ik8}
\bigl| 1/\lambda_{\ep,\theta}^{(k)} \, - \, \alpha_{\theta}^{(k)} \bigr| \,\,\,\le\,\,\, \ep^2\nu^{-1}.
\end{equation}
Finally, we notice that all non-zero eigenvalues of $\mathbf{B}_\t^{-1}  \mathcal{P}_{\overline{V_\t}}^\t$ are the inverses of the eigenvalues of  $\mathbf{B}_\t$ and vice versa. 
Therefore we have the following relations between the eigenvalues $\{\mu_{\theta}^{(k)}\}$  of $\mathbf{B}_\t$ and  the eigenvalues 
$\lambda_{\ep,\theta}^{(k)}$ of $\mathcal{L}_{\ep,\theta}$:
\be\Big| 1/\lambda_{\ep,\theta}^{(k)} \,-\,  1/\mu_{\theta}^{(k)} \Big| 
\,\,\le\,\, \ep^2\nu^{-1}, \quad \ \forall k\in \NN, \ \ \forall \t \in \Theta, \qquad \text{if $ \ \dim  V_\theta\ =\infty$,}\label{ik889.0}
\end{equation}
 or
\begin{equation}\label{ik889}
\Big| 1/\lambda_{\ep,\theta}^{(k)} -  1/\mu_{\theta}^{(k)} \Big| \,\,\le\,\, \ep^2\nu^{-1},\quad 
\Big| 1/\lambda_{\ep,\theta}^{(p)} \Big| \,\,\le\,\, \ep^2\nu^{-1} ,\quad \forall k \le N ,\  \forall p\ge N+1, \ \forall \t \in \Theta, \ \text{if $\dim V_\theta\ =N$. }
\end{equation}
Inequalities \eqref{ik7}, \eqref{ik889.0} and \eqref{ik889} provide the desired estimates on the closeness for small $\ep$ of the ``resolvents'' and of the spectra of the exact and approximate operators, $\mathcal{L}_{\ep,\theta}$ and $\mathbf{B}_\t$ respectively, uniform in $\t$. 


\subsection{The case of discontinuous $V_\theta$ with removable singularities}\label{spectralasymptoticsDiscV}
Here, we suppose that the assumptions of Theorem \ref{thm.maindiscthm} hold, establishing the closeness of the solution $u_{\ep,\t}$ to the original problem \eqref{p1} to the solution 
$v^h+z^h\in \Vs_\t\,\dot{+}\, Z$ to the approximate problem \eqref{z3prob}.
We shall follow the pattern of the previous subsection, aiming first at recasting \eqref{z3prob} 
in an operator form. 
%First we observe that  the analogue of $V_\t$ is $\Vs_\t +Z$ and we assume that $\Vs_\t\cap Z =\{0\}$
% and $V_\t^\star \dot{+} Z$ is closed in $H$ 
%for all $\t$\footnote{ This assumption is not very restrictive since it always holds for small $\t$,  by \eqref{V+ZCoercive} and \eqref{Nbound}, and  for large $\t$ the spectrum of $\mathcal{L}_{\ep,\theta}$ can be analysed as in Section \ref{s.spcontV}, see Remark \ref{r.ikth77}.}.   Then, one can  
To that end, for $\t \in \Theta,$ $\,| \t | < r_1$, define on $\Vs_\t\,\dot{+}\, Z$ an inner product $s$ %given 
by
\begin{equation}
\label{ikspectr200} s(v+z,\tilde{v}+\tilde{z}) \,\,: =\,\,\ep^{-2} a_\t^{\rm h}(z,\tilde{z}) \,+\, b_\t(v+z,\tilde{v}+\tilde{z}), \quad   \forall v,\,\tilde{v}\in \Vs_\t,\, \  \, \forall z,\tilde{z}\in Z.
\end{equation}
%here $u=z+v$, $\tilde{u}=\tilde{z}+\tilde{v}$ and $z,\tilde{z}\in Z$, $v,\tilde{v}\in {V}_\t^*$. 
As follows for example from \eqref{btnormonV+Z} and \eqref{ahcoercive}, 
%there exists $c_\t>0$ such that
%\[
%c_\t\big( \| z\|_\t^2 + \|v\|_\t^2 \big)\le   s[v+z], \quad \forall v\in V_\t^\star, \, \forall z \in Z,
%\]
%and, therefore, 
$V_\t^\star\,\dot{+}\,Z$ endowed with the inner product $s$ is a Hilbert space that is continuously embedded in $H$ and therefore compactly embedded in $\mathcal{H}$; we denote this Hilbert space by $\mathbf{H}$.
Set $\mathcal{H}_\t : = \left(\overline{\Vs_\t\,\dot{+}\, Z}, \,d_\t\right)$  and  
let     $\mathbf{L}_{\ep,\theta}: {\rm dom}\, \mathbf{L}_{\ep,\theta} %\subseteq  \mathbf{H}
\rightarrow \mathcal{H}_\t$ be the self-adjoint operator in $\mathbf{H}$ 
%in  $\mathcal{H}_\t$
generated by the closed positive sesquilinear form $s$ with the form domain  $\Vs_\t\,\dot{+}\, Z$. The spectrum  of $\textbf{L}_{\ep,\theta}$ consists of positive isolated eigenvalues (which may only accumulate at infinity if $V_\t^\star\,\dot{+}\,Z$ is infinite-dimensional).  

Consider problem \eqref{p1} with  functional $f$ given by \eqref{ik3}. Then, for the solution to 
the approximate problem \eqref{z3prob}, 
 $v^h+z^h \,=\, \mathbf{L}_{\ep,\theta}^{-1}\,\mathcal{P}_{\mathcal{H}_\t}^\t g$. 
%Notice that $(v,z)$ solves \eqref{z3prob} and therefore by  
By Theorem \ref{thm.maindiscthm} (see \eqref{final2}) and \eqref{ik2} one has 
\[
d_\t\left[u_{\ep,\t}-(v^h+z^h)\right] \,\,\le \,\, C_6\, \ep^2 d_\t[g], 
\]
	which can be rewritten in the operator language as
	\begin{equation}
	\label{ik37}	\left\|\,\mathcal{L}_{\ep,\theta}^{-1}\,-\, \mathbf{L}_{\ep,\theta}^{-1}\mathcal{P}_{\mathcal{H}_\t}^\t\,\right\|_{(\mathcal{H},d_\t)\rightarrow (\mathcal{H},d_\t)}\,\,\,\le\,\,\, C_6^{1/2}\ep,  
	\end{equation}
	where $\mathbf{L}_{\ep,\theta}^{-1}\,\mathcal{P}_{\mathcal{H}_\t}^\t$ is a self-adjoint operator in $(\mathcal{H},d_\t)$. 
Arguing then as in the previous subsection we arrive at the following result. 
\begin{theorem}\label{ikthm}
Assume \eqref{KA}--\eqref{H4}. Let $\{ \lambda^{(k)}_{\ep,\theta}\}_{k\in \NN}$ and $\{\Lambda^{(k)}_{\ep,\theta}\}_{k\in \NN}$ be the eigenvalues of the operators 
$\mathcal{L}_{\ep,\theta}$ and $\mathbf{L}_{\ep,\theta}$ respectively. Then
\begin{gather*}
%\label{ik88}
\Big| 1/\lambda_{\ep,\theta}^{(k)} \,- \, 1/\Lambda_{\ep,\theta}^{(k)} \Big| \,\,\le\,\,  C_6^{1/2}\ep, \quad \forall k\in \NN,\  \,\forall\, \t \in \Theta, \ |\t| < r_1, \quad 
\text{ if $\,\,\dim  \left(\Vs_\t \,\dot{+}\,Z\right)\ =\infty$, } \\
%\end{equation*}
\text{or} \hspace{\textwidth}\\
%\begin{equation*}
%\label{ik89}
\Big| 1/\lambda_{\ep,\theta}^{(k)} -  1/\Lambda_{\ep,\theta}^{(k)} \Big| \,\le\, C_6^{1/2} \ep,\ \  
\Big| 1/\lambda_{\ep,\theta}^{(p)} \Big| \le C_6^{1/2} \ep, \ \ \forall k\le N, \, \forall   p\ge N+1,\, \forall \t \in \Theta,\, |\t| < r_1, \ \text{if $\dim \left( \Vs_\t\dot{+} Z\right) =N.$}
\end{gather*}
\end{theorem}
%; we shall provide this description now. Let $\mathbf{B}_{\theta}$ be the self-adjoint operator in ${\overline{\Vs_\t}}$  generated by the closed positive sesquilinear form  $(\cdot,\cdot)_\t$ with form domain $\Vs_\t$, in particular 
%\begin{equation}\label{ikss88}  c(\mathbf{B}_{\theta}v,\tilde{v})=(v,\tilde{v})_\t,\ \ \forall v\in \text{dom}\,\mathbf{B}_{\theta},\ \ \forall \tilde{v}\in \Vs_\t. 
%\end{equation}
%%$\begin{remark} \label{ikrem5}If one replaces $V_\t$ with  $\overline{V}_\t$, see   \eqref{contVs}, in the definition of $\mathbf{L}_{\ep,\theta}$ and $\Lambda^{(k)}_{\ep,\theta}$ then the condition $\t\neq 0$ in Theorem \ref{ikthm} may be omitted. 
%%\end{remark}
%Then the following result follows directly from Theorem \ref{thm1neigh}.
%\begin{theorem} \label{ikth77}  If $\mu_\theta^{(k)}$ are eigenvalues of the operator $\mathbf{B}_\t$ (arranged in ascending order) and $|\t|>r>0$, then
%\begin{equation}\label{ik888}
%\big| 1/\lambda_{\ep,\theta}^{(k)} -  1/\mu_{\theta}^{(k)} \big| \le \frac{\ep^2K_b}{\nu_*r^2}\ ,\  \forall k\in \NN,
%\end{equation}
%if $\dim Z\overset{\cdot}{+} V_\theta\ =\infty$, and
%\begin{equation}\label{ik889}
%\Big| 1/\lambda_{\ep,\theta}^{(k)} -  1/\mu_{\ep,\theta}^{(k)} \Big| \le \frac{\ep^2K_b}{\nu_*r^2}\ ,\  \ \Big| 1/\lambda_{\ep,\theta}^{(p)} \Big| \le \frac{\ep^2K_b}{\nu_*r^2},\ \ \forall k=1,..,N ,\  p=N+1,..., 
%\end{equation}
%if $\dim Z\overset{\cdot}{+} V_\theta\ =N$. Here $\nu_*$ is the constant defined in (H3).
%\end{theorem}
%
%
%
%\begin{theorem} \label{ikth77}  Let $\{\alpha_\theta^{(k)}\}_{k\in \NN}$ be the eigenvalues (in ascending order) of the operator $\mathbf{D}_\t$ (cf. \eqref{ikddd}) and $|\t|>r>0$, then
%	\begin{equation}\label{ik888}
%	\big| 1/\lambda_{\ep,\theta}^{(k)} -  1/\alpha_{\theta}^{(k)} \big| \le \ep^2{\nu_*^{-1}r^{-2}}K_c\ ,\  \forall k\in \NN, \qquad \text{	if $\dim V_\theta\ =\infty$,}
%	\end{equation}
% and
%	\begin{equation}\label{ik889}
%	\big| 1/\lambda_{\ep,\theta}^{(k)} -  1/\alpha_{\ep,\theta}^{(k)} \big| \le \ep^2{\nu_*^{-1}r^{-2}}K_c, \qquad   \big| 1/\lambda_{\ep,\theta}^{(p)} \big| \le \ep^2{\nu_*^{-1}r^{-2}}K_c,\ \ \forall k\le N,\  \forall p\ge N+1, 
%	\end{equation}
%	if $\dim V_\theta\ =N$. Here $\nu_*$ is the constant defined in (H3).
%\end{theorem}
%----------------------------------------------- this may be not needed----------------------------------------
%
%Let us look at the eigenvalues of $\mathbf{L}_{\ep,\t}$. Min-max principle  implies
%\begin{equation} \label{ikLambda}\Lambda^{(k)}_{\ep,\theta}=\inf_{T\subset Z\overset{\cdot}{+} {V}_\t^* :\,\dim T= k} \ \ \sup_{ u\in T} \ \frac{s(u,u)}{c(u,u)}=\end{equation}
%$$\inf_{T\subset Z\overset{\cdot}{+} {V}_\t^* :\,\dim T= k}\  \ \ \sup_{ z\in Z,\, v\in \overline{V}_\t:\, z+v\in T}\  \ \frac{\mathfrak{b}_{\ep,\t}^{\rm trunc}(z,z)+(v,v)_\t}{c(z+v,z+v)}=
%$$
%$$\inf_{T\subset Z\overset{\cdot}{+} {V}_\t^* :\,\dim T= k}\  \ \ \sup_{ z\in Z,\, v\in \overline{V}_\t:\, z+v\in T}\  \ \frac{\ep^{-2}\Big(a_0''(z,z)\t \cdot\t-a_0(\t\cdot Nz,\t\cdot Nz)\Big) +(z,z)_\t+(v,v)_\t}{c(z+v,z+v)}\ .$$
%Similar formula for double-porosity model was derived in.... 
%
%It may appear that formulas for the eigenvalues of the operator $\mathbf{L}_{\ep,\theta}$ (which can be interpreted as the  approximation of  $\mathcal{L}_{\ep,\theta}$) have the same level of complexity as the expressions for the  eigenvalues of operator 
%$\mathcal{L}_{\ep,\theta}$ itself:
%\begin{equation}
%\label{iklambda} \lambda_{\ep,\theta}^{(k)}=\inf_{T\subset H :\,\dim T= k} \ \ \sup_{ u\in T} \ \frac{\ep^{-2}a_\t(u,u)+(u,u)_\t}{c(u,u)}\ .
%\end{equation}
%However, in practical cases, \eqref{ikLambda} is much simpler then \eqref{iklambda}. Indeed, if  $V_\t=0$ for $\t\neq 0$ and $Z$ is one dimensional space, generated say by some vector $z_0\in H$, $c[z_0]=1$ (this is what occur in the  problems of classical homogenisation of scalar differential operators, see Example) then \eqref{iklambda}  and \eqref{ik8} imply
%$$ \Big|\frac{1}{\lambda_{\ep,\theta}^{(1)}}-\frac{1}{\Lambda_{\ep,\theta}^{(1)}}\Big|\leq \ep C,\ \Big|\frac{1}{\lambda_{\ep,\theta}^{(k)}}\Big|\leq \ep C, \ k=2,3...,$$
%and
%$$\Lambda_{\ep,\theta}^{(1)}=\ep^{-2}A^{hom}\t\cdot\t+(z_0,z_0)_0,$$ where 
%$$A^{hom}\t\cdot\t=a_0''(z_0,z_0)\t \cdot\t-a_0(\t\cdot Nz_0,\t\cdot Nz_0), \forall \t\in \Theta.
%$$
%Matrix $A^{hom}$ appears in the context of homogenisation as {\it homogenised matrix} and describes effective properties of oscillatory media, see Example.
%
\begin{remark}\label{r.ikth77}
The approximations for the eigenvalues of $\mathcal{L}_{\ep,\t}$ given by Theorem \ref{ikthm} 
for $|\t|<r_1$ can be combined with 
%can be easily extended to hold for all $\t \in \Theta$ by arguing as in the proof of Theorem \ref{thm.IKunifest2}. However, 
the results of Section \ref{s.spcontV} for $|\t|\ge r_1$. 
%imply  a better  description of the eigenvalues of $\mathcal{L}_{\ep,\theta}$ for $\t$ separated from zero. 
Indeed, under \eqref{contVs}--\eqref{distance},  the estimates \eqref{ik889.0} and \eqref{ik889} hold  for $|\t| \ge r_1 >0$ with $\nu$ replaced by $\gamma r_1^2$ (as seen directly from the proof of Theorem  \ref{thm1.all} for $\nu(r_1) = \gamma r_1^2$ as implied by \eqref{distance}).	
	
%	Furthermore, if  $\mathcal{E}_\t$ exists (see \eqref{H5}, \eqref{H6}) then $\mathbf{D}_\t^{-1} = \mathcal{E}_\t (\mathbf{B}+I)^{-1} \mathcal{E}_\t^{*}$ and so the eigenvalues $\alpha^{(k)}_{\t} = \mu^{(k)} +1$, $k \in \mathbb{N}$, and, therefore,  $\lambda^{(k)}_{\ep,\t}$  are approximately independent of both  $\ep$ and $\t$.  Indeed, one has
%	%	\begin{equation}\label{ik88}
%	%\big| 1/\lambda_{\ep,\theta}^{(k)} -  1/\mu^{(k)} \big| \le  \nu_r K_d\ep^2, \quad \forall k\in \NN,\ \forall \t \in \Theta, \quad \text{ if $\dim  V_\star \dot{+}Z\ =\infty$, }
%	%\end{equation}
%	%and
%	\begin{equation}
%	%\label{ik89}
%	\big| \tfrac{1}{\lambda_{\ep,\theta}^{(k)}} -  \tfrac{1}{\mu^{(k)}+1} \big| \le \nu_r^{-1} \ep^2,\ \  \big| 1/\lambda_{\ep,\theta}^{(p)} \big| \le  \nu_r^{-1} \ep^2, \quad \forall k\le N , \, \forall   p\ge N+1,\, \forall \t \in \Theta, |\t| \ge r,
%	\end{equation}
%	where $N = {\rm dim}\, V_\star$ (which can be infinity).
\end{remark}	
\subsection{The case of Lipschitz continuous $b_\t$}\label{s.spbt}
{
Let us now suppose  the assumptions of Theorem \ref{thm.IKunifest2} hold, establishing the closeness 
of the solution $u_{\ep,\t}$ of the original problem \eqref{p1} to the approximations based on the solution 
$v+z\in V_\star\,\dot{+}\,Z$ of the simplified problem \eqref{IKz3prob88}. 
%Suppose  $\mathcal{H}_0$ is a complex separable Hilbert space with inner product $d$ and let $\mathcal{H}_\t$ be the Hilbert space 
With the aim of rewriting (a further approximation to) \eqref{IKz3prob88} and the resulting estimate 
\eqref{IKfinal3-2} in an operator form, notice first that the left-hand side of 
\eqref{IKz3prob88} 
 has the following important self-similarity property: since $a^h_\t$ is quadratic in $\t$ and 
$b_0$ is $\t$-independent, it depends on $\ep$ and $\t$ only via $\t/\ep=:\xi$ and 
in particular $\ep^{-2}a^h_\t=a^h_\xi$. 
  For each $\xi \in \mathbb{R}^n$, let  $\mathbb{L}_\xi$ be the  self-adjoint operator in 
	$\mathcal{H}_0 = \left(\overline{V_\star \,\dot{+\,}Z},\,d_0\right)$ generated by the following inner product on $V_\star \,\dot{+}\,Z$: 
\begin{equation}\label{Sform}
\mathbb{S}_\xi(v+z,\tilde{v}+\tilde{z}) \,\,: = \,\, a_\xi^{\rm h}(z,\tilde{z}) \,\,+\,\, b_0(v+z,\tilde{v}+\tilde{z}), \quad   \forall v +z, \ \tilde{v}+ \tilde{z}\in V_\star \,\dot{+}\, Z.
\end{equation}
Similarly to the previous subsections, for any $\xi\in\mathbb{R}^n$, $\mathbb{L}_\xi$ has a compact resolvent and hence a discrete positive spectrum which can only accumulate at infinity. \\
In addition, let us suppose  that $\mathcal{E}_\t$ (that satisfies \eqref{Eprop1}--\eqref{Eprop2}) and $d_\t$ also satisfy 
%$d_\t[\cdot]=d_0[\mathcal{E}_\t \cdot ]$.
\begin{equation}
\tag{H6}
\label{H6}
\begin{aligned}
&\text{ $\mathcal{E}_\t$ extends to a bijection  in $ \mathcal{H}$ such that 
$d_\t\left(\mathcal{E}_\t u,\mathcal{E}_\t\tilde u\right) = d_0( u,  \tilde u), \quad 
\forall u,\tilde u \in \mathcal{H}, \, \,\t \in \Theta,$}\\
& \mathcal{E}_0=I\,\, 
\text{and $\big\|\, \mathcal{E}_{\t_1} \,-\, \mathcal{E}_{\t_2} \,\big\|_{(\mathcal{H},d_0) \rightarrow (\mathcal{H},d_{\t_1})} \,\,\le\,\, K_e\, |\t_1-\t_2|, \quad \forall \t_1,\t_2 \in \Theta, \ \ $ for some $K_e >0$.}
\end{aligned}
% \text{ $\mathcal{E}_\t$ extends to a isomorphism  between $(\mathcal{H},d_0)$ and $(\mathcal{H},d_\t)$ such that }
%\| \mathcal{E}_\t - I \|_{(\mathcal{H},d_\t) \rightarrow (\mathcal{H},d_0)} \le K_e |\t| \quad \forall \t \in \Theta, \text{ for some $K_e >0$.}
\end{equation}
%Henceforth, we consider the following family of inner products on $\mathcal{H}$:
%In particular one has 
%\begin{equation}
%%\tag{H6}
%\label{defdt}
%% \text{ $\mathcal{E}_\t$ extends to a unitary operator in $ \mathcal{H}$ and
%%	$\| \mathcal{E}_\t - I \|_{\mathcal{H} \rightarrow \mathcal{H}} \le K_e |\t|$}, \forall \t \in \Theta, \text{ for some $K_e >0$.}
%d_\t(u,v) : = d_0(\mathcal{E}_\t^{-1} u, \mathcal{E}_\t^{-1} v), \quad \forall u,v \in \mathcal{H}, \, \t \in \Theta.
%\end{equation}
\begin{remark}
Assumption \eqref{H6} appears to be restrictive but, as we will see, this condition often trivially holds in examples, Section \ref{sec:examples}.
\end{remark}
Let the right-hand-sides of \eqref{p1} be again given by \eqref{ik3}. 
Aiming at recasting \eqref{IKfinal3-2} in an operator form, we observe that this is prevented by 
the presence of the transfer operator $\mathcal{E}_\t$, in both \eqref{IKfinal3-2} and in the 
right-hand side of 
\eqref{IKz3prob88}, in the ``$v$-terms'' but not in the ``$z$-terms''. 
As we will see, this can be rectified by replacing $\tilde z$ on the right-hand side of \eqref{IKz3prob88} by $\mathcal{E}_\t\tilde z$ as well as $z$ in \eqref{IKfinal3-2} by 
$\mathcal{E}_\t z$. On the one hand, as we will see, this introduces a small additional error in \eqref{IKfinal3-2}, but on the other hand allows to express the amended approximation $\mathcal{E}_\t (v+ z)$ 
in an operator form. Indeed, $v+z$ now solves the amended \eqref{IKz3prob88} which via \eqref{Sform} and \eqref{H6} reads 
$\mathbb{S}_{\t/\ep}(v+z,\tilde v+\tilde z)=
d_\t\big(g,\,\mathcal{E}_\t(\tilde v+\tilde z)\big)= 
d_0\big(\mathcal{E}_\t^{-1}g,\,\tilde v+\tilde z\big)$. 
Hence $\mathcal{E}_\t (v+ z)=
\mathcal{E}_\t \mathbb{L}_{\t / \ep}^{-1}\mathcal{P}_{\mathcal{H}_0}^0\mathcal{E}_\t^{-1}g$. 
Notice that \eqref{H6} implies that $\mathcal{E}^{-1}_\t$ is the adjoint of $\mathcal{E}_\t$ when the latter is 
considered as a (unitary)  map from   $(\mathcal{H}, d_0)$ to $(\mathcal{H},d_\t)$, so the 
emerging operator $\mathcal{E}_\t \mathbb{L}_{\t / \ep}^{-1}\mathcal{P}_{\mathcal{H}_0}^0\mathcal{E}_\t^{-1}$ is %indeed 
self-adjoint and non-negative in $(\mathcal{H}, d_\t)$. 
%that is
%\begin{equation}\label{Eprop4}
%d_\t(\mathcal{E}_\t u , v ) = d_0 (u ,\mathcal{E}_\t^\star v), \quad \forall u, v \in \mathcal{H}.
%\end{equation}
As a result, the following theorem holds.
\begin{theorem}\label{p.unitaryequiv} Assume \eqref{KA}--\eqref{H6}. For all $\t \in \Theta$ and $0<\ep <1$ one has:
	\begin{equation*}\label{splimSe.3}
\left\|\, \mathcal{L}_{\ep,\t}^{-1} \,\, -\,\, 
\mathcal{E}_\t\, \mathbb{L}_{\t / \ep}^{-1}\mathcal{P}_{\mathcal{H}_0}^0\mathcal{E}_\t^{-1}\,\right\|_{(\mathcal{H},d_\t) \rightarrow (\mathcal{H},d_\t)} \,\,\le\,\, C_{11} \,\ep,
 \quad 
C_{11} \,=\, C_{10}^{1/2} \,+\, \frac{1}{2} K_e \nu_\star^{-1/2} \left(\, 1 \,+\, 
\left(\tfrac{ 1+K^2}{2(1-K_Z)}\right)^{1/2} \,\right).
\end{equation*}
%for $C_{10}= C_{9}^{1/2} + \tfrac{1}{2} K_e \nu_\star^{-1/2} \big( 1 + (\tfrac{ 1+K^2}{1-K_Z})^{1/2} \big)$. 
\end{theorem}
\begin{proof}
Estimate \eqref{IKfinal3-2} of	Theorem \ref{thm.IKunifest2} informs us via \eqref{ik2} that 
$d_\t\big[ \mathcal{L}_{\ep,\t}^{-1}g - (\mathcal{E}_\t v + z)\big]\le C_{10} \ep^2 d_\t[ g]$, where $v+z$ is the solution to \eqref{IKz3prob88} with functional \eqref{ik3}.  It remains to bound $  \big(\mathcal{E}_\t v + z\big) - \mathcal{E}_\t \mathbb{L}_{\t / \ep}^{-1}\mathcal{P}_{\mathcal{H}_0}^0\mathcal{E}_\t^{-1}g$. 
	Let  $ v_1+z_1 =\mathbb{L}_{\t / \ep}^{-1}\mathcal{P}_{\mathcal{H}_0}^0 \mathcal{E}^{-1}_\t g$, that is (see  \eqref{Sform} and \eqref{H6}) $v_1+z_1 \in V_\star \dot{+} Z$ solves 
	\begin{equation}\label{v1z1prob}
	\ep^{-2} a_\t^{\rm h}(z_1,\tilde{z}) \,+\, b_0(v_1+z_1,\tilde{v}+\tilde{z}) \,=\, 
	d_\t\big(g,\, \mathcal{E}_\t(\tilde{v} + \tilde{z})\,\big),  
	\quad \forall \, \  \tilde{v} + \tilde{z} \in V_\star \dot{+} Z.
	\end{equation}
	Then 
	\begin{equation}\label{15.09.20e1}
	\big(\mathcal {E}_\t v +z\big) \,-\, \mathcal{E}_\t \mathbb{L}_{\t / \ep}^{-1}\mathcal{P}_{\mathcal{H}_0}^0\mathcal{E}_\t^{-1}g\,=\, \mathcal {E}_\t v +z - \mathcal{E}_\t(v_1+z_1) =   (I- \mathcal{E}_\t ) z +  \mathcal{E}_\t  \big(  v+z -(v_1 +  z_1) \big).
	\end{equation}
	% the triangle inequality and \eqref{IKfinal2} one has
	%% the difference between  $\mathcal{E}_\t v + z$ and $\mathcal{E}_\t(v_1+z_1) $. Notice that 
	%\[
	%%u_{\ep,\t} - \mathcal{E}_\t(v_1+z_1) = u_{\ep,\t} - (\mathcal{E}_\t v + z) +  (1- \mathcal{E}_\t ) z +  \mathcal{E}_\t  \big(  v+z -(v_1 +  z_1) \big),
	%d[ u_{\ep,\t} - \mathcal{E}_\t(v_1+z_1) ]^{1/2}\le C_8^{1/2} \ep d[g]^{1/2} +  d[ (1- \mathcal{E}_\t ) z]^{1/2}   + d[\mathcal{E}_\t  \big(  v+z -(v_1 +  z_1) \big) ]^{1/2}.
	%\]
	%where $(v,z)$ the solution to \eqref{IKz3prob88} with functional \eqref{ik3}. 
	%By Theorem \ref{thm.IKunifest}, \eqref{IKfinal2},
	% one has (see \eqref{ik2},\eqref{IKfinal2})  $d[u_{\ep,\t} - (\mathcal{E}_\t v + z)] \le C_8 \ep^2 d[g]$ where  $(v,z)$ solves \eqref{IKz3prob88} for the functional \eqref{ik3}. 
	%for the functional \eqref{ik3}, 
	%we need only bound  the last two terms.
	% $(1- \mathcal{E}_\t ) z +  \mathcal{E}_\t  \big(  v+z -(v_1 +  z_1) \big)$ for  $(v,z)$ the solution to \eqref{IKz3prob88} with functional \eqref{ik3}. 
	Let us bound $d_\t\big[(I- \mathcal{E}_\t ) z\big] $.  By \eqref{VZorth} one has 
	$2\left(1-K_Z\right)\big(b_0[v]+ b_0[z]\big) \le b_0[v+z]$ which combined with \eqref{ik2}, \eqref{as.b1}, \eqref{ahcoercive}, \eqref{IKz3prob88} and %\eqref{Eprop1} 
	\eqref{H6} 
	gives 
	\begin{flalign*}
	&\nu_\star\ep^{-2}|\t|^2d_0[z] +2\left(1-K_Z\right) \left(d_0[v]+ K^{-2} d_\t[z]\right) \,\le\, \ep^{-2} a^{\rm h}_\t[z] + b_0[v+z] \,=\, d_\t(g,\mathcal{E}_\t v + z) \\
	&\le\,\, d_\t^{1/2}[g] \left(d_\t^{1/2}[\mathcal{E}_\t v] + d_\t^{1/2}[z]\right)\,=\, 
	d_\t^{1/2}[g] \left(d_0^{1/2}[v] + d_\t^{1/2}[z]\right)\\
	& \le \, \tfrac{1}{8}\left(1-K_Z\right)^{-1} \left(1+  K^{2} \right)d_\t[g]  \,+ \,
	2\left(1-K_Z\right) \left(d_0[v] \,+\, K^{-2}d_\t[z] \right).
	\end{flalign*}
	Thus $\nu_\star \ep^{-2} |\t|^2 d_0[z]  \le \tfrac{1}{8}\left(1-K_Z\right)^{-1}(1+K^2) 
	d_\t[g] $  and therefore, via \eqref{H6}, one has  
	\[
	d_\t[(I-\mathcal{E}_\t) z] \,\le\,  K_e^2 |\t|^2 d_0[z] \, \le\,\tfrac{1}{8} K_e^2 \nu_\star^{-1}(1-K_Z)^{-1} (1+K^2) \ep^2 d_\t[g].
	\] 
	It remains to  bound the last term in \eqref{15.09.20e1} via $d_\t[g]$. By \eqref{H6} it is equivalent to bounding  $d_0\left[v+z -(v_1 +  z_1)\right]$. 
	%Note that, via \eqref{Sform}, $v_1+z_1$ solves 
	%\[
	%\ep^{-2} a_\t^{\rm h}(z_1,\tilde{z}) + b_0(v_1+z_1,\tilde{v}+\tilde{z}) = d(g, \mathcal{E}_\t(\tilde{v} + \tilde{z})) 
	%\quad \forall \tilde{v} + \tilde{z} \in V_\star \dot{+} Z.
	%\]
	Subtracting \eqref{v1z1prob}  from \eqref{IKz3prob88} (with $f$ from \eqref{ik3}, $\tilde{v} = v-v_1$ and $\tilde{z} = z-z_1$) and utilising \eqref{H6}, \eqref{ik2} and \eqref{ahcoercive} gives 
	%by \eqref{IKz3prob88} and the identity $\mathbb{S}_{\t /\ep} (v_1 + z_1, \tilde{v} + \tilde{z}) = d(g, \mathcal{E}_\t(\tilde{v} + \tilde{z}))$ $ \forall \tilde{v} + \tilde{z} \in V_\star \dot{+} Z$, we see, via \eqref{Sform} and \eqref{H6}, that 
	\[
	\ep^{-2} a^{\rm h}_\t \left[z-z_1\right] + b_0\left[v +z -( v_1 + z_1)\right] \,=\,
	d_\t\bigl(g, (I-\mathcal{E}_\t) (z-z_1) \bigr) \,\,\le 
	\]
	\[
	\ \ \ \ \ \ \quad \ \  d^{1/2}_\t[g]\,  K_e|\t|\,d^{1/2}_0\left[z-z_1\right] \,\,\,\le \,\,\, 
	K_e \,\nu_\star^{-1/2}\,  d^{1/2}_\t[g]
	\left(a^{\rm h}_\t[z-z_1]\right)^{1/2}.
	\]
	Therefore  $
	b_0\left[ v + z -(v_1 +z_1) \right] \le\tfrac{1}{4} \nu_\star^{-1}\, K_e^2\, \ep^2 \,d_\t[g]$ and so 
 $d_0\left[ v + z -(v_1 +z_1) \right] \le\tfrac{1}{4} \nu_\star^{-1}\, K_e^2 \,\ep^2 d_\t[g].$ Applying finally the triangle inequality completes the proof.  
\end{proof}
Theorem \ref{p.unitaryequiv}, together with the fact that 
$\mathcal{E}_\t: (\mathcal{H}, d_0)\rightarrow (\mathcal{H},d_\t)$ 
%$\mathcal{E}_\t: V_\star\rightarrow V_\t^\star$ 
is unitary by \eqref{H6}, 
provides the following analogue of Theorem \ref{ikthm}:
\begin{theorem}\label{ikthm2}
	Assume \eqref{KA}--\eqref{H6}. Let $\{ \lambda^{(k)}_{\ep,\theta}\}_{k\in \NN}$ and $\{\lambda^{(k)}_{\xi}\}_{k\in \NN}$ be the eigenvalues of the operators 
	$\mathcal{L}_{\ep,\theta}$ and $\mathbb{L}_{\xi}$ respectively. Then, 
	for some $C_{11}>0$ independent of $\ep$ and $\t$, 
	\begin{equation}\label{ik88}
	\left| 1/\lambda_{\ep,\theta}^{(k)} \,- \, 1/\lambda_{\theta/\ep}^{(k)} \right| \,\,\le\,\,  C_{11}\,\ep, \quad \forall k\in \NN,\ \ \forall \t \in \Theta, \quad 
	\text{ if $\ \ \ \dim  \left(V_\star \,\dot{+}\,Z\right)\ =\infty$, }
	\end{equation}
	or
	\begin{equation}\label{ik89}
	\left| 1/\lambda_{\ep,\theta}^{(k)} -  1/\lambda_{\theta/\ep}^{(k)} \right| \,\le\, C_{11}\ep,\ \  \left| 1/\lambda_{\ep,\theta}^{(p)} \right| \le  C_{11}\ep, \quad \forall k\le N , \, \forall   p\ge N+1,\, \forall \t \in \Theta, \ \text{	if $\ \dim  \left(V_\star\dot{+} Z\right)\ =N.$}
	\end{equation}
 \end{theorem}}
%Analysis of the  the collective spectrum 
%$\sigma_\ep : =\bigcup_{\t \in \Theta} {\rm Sp}\, \mathcal{L}^{-1}_{\ep,\t}$ is important from the point of view of applications and  
%Theorem \ref{ikthm2} gives an approximation to this set. However,  this approximation still depends on  the parameter $\ep$. Now we will correct this result to produce an $\ep$-independent approximation to $\sigma_\ep$.
We next aim at approximating  the collective spectrum 
$\bigcup_{\t \in \Theta} {\rm Sp}\, \mathcal{L}_{\ep,\t}=:{\rm Sp}_\ep$. 
The importance of ${\rm Sp}_\ep$ is due to the fact that in many examples 
(Section \ref{sec:examples}) operators $\mathcal{L}_{\ep,\t}$, $\t\in\Theta$, serve as fibers in a decomposition 
of an (appropriately transformed) original operator whose spectrum of interest is (the closure of)  
${\rm Sp}_\ep$. 
Theorem \ref{ikthm2} provides one approximation to ${\rm Sp}_\ep$, however,  this approximation still depends on  the parameter $\ep$. We will now rectify this 
%correct this result 
to produce an important $\ep$-independent approximation of the above collective spectrum. 
%set $\bigcup_{\t \in \Theta} {\rm Sp}\, \mathcal{L}_{\ep,\t}^{-1}$.
\begin{theorem}\label{t.collectivespec}
Assume \eqref{KA}--\eqref{H6}. Then 
\begin{equation}\label{symdist.e1}
{\rm dist}_s \Big(  \bigcup_{\t \in \Theta} {\rm Sp}\, \mathcal{L}_{\ep,\t}^{-1} \cup \{ 0\} , \bigcup_{\xi \in \mathbb{R}^n} {\rm Sp}\, \mathbb{L}^{-1}_\xi \cup \{ 0 \}  \Big)  \,\,\le\,\, C_{12} \,\ep,
\end{equation}
where for non-empty $X,Y\subset\mathbb{R}$, 
${\rm dist}_s(X,Y):=\max\bigl(\sup_{x\in X}{\rm dist}(x,Y),\, \sup_{y\in Y}{\rm dist}(y,X)\bigr)$ is the symmetric Hausdorff distance and $C_{12}$ is a positive constant independent of $\ep$. 
\end{theorem}
%{\color{blue} For $\t$
%\[
%{\rm dist}_S \big(  {\rm Sp}\, \mathcal{L}^{-1}_{\ep,\t},  \cup_{r\in [0,\infty)} {\rm Sp}\, \mathbb{L}^{-1}_{r\t} \big) \le C_{10} \ep,
%\]	
%}
Before proving the theorem, we state its corollary providing %for small enough $\ep$ 
an $\ep$-independent approximation of the set $\bigcup_{\t \in \Theta} {\rm Sp}\, \mathcal{L}_{\ep,\t}$ in any finite interval.
\begin{corollary}\label{c.collspec}
%Assume $V_0\neq \{0\}$. 
For every interval $ [a,b] \subset (-\infty,\infty)$   one has 
\begin{equation}
\label{cor67}
d_{[a,b]}\Big( \, \overline{ \bigcup_{\theta \in \Theta} {\rm Sp}\, \mathcal{L}_{\ep,\t} }, \ \  \overline{\bigcup_{\xi \in \mathbb{R}^n} {\rm Sp}\, \mathbb{L}_\xi } \, \Big) \,\,\le\,\, C_{b}\, \ep,
\quad \forall\, \  0<\ep<1,\\
%d\Big([a,b]  \cap \overline{ \bigcup_{\theta \in \Theta} {\rm Sp}\, \mathcal{L}_{\ep,\t} }, \overline{\bigcup_{\xi \in \mathbb{R}^n} {\rm Sp}\, \mathbb{L}_\xi } \Big)  \le b(b+1)C_{11} \ep.
\end{equation}
with a constant $C_b$ independent of $\ep$ and $a$. 
%ee \eqref{ikddd}). 
%$\ep_b=1$ if $b<1$, $\ep_b=\min\left\{1,\,1/(4 C_{13}b)\right\}$ if $b\ge 1$. 
In \eqref{cor67},  
$d_{[a,b]}(X,Y) := \max\big({\rm dist}([a,b]\cap X,Y), {\rm dist}([a,b]\cap Y,X)\big)$ 
where ${\rm dist}(X,Y):=\sup_{x\in X}{\rm dist}(x,Y)$ is the (non-symmetric) distance, 
and we adopt the convention that ${\rm dist}(\emptyset,A)={\rm dist}(A,\emptyset)
= 0$ for any set $A$. In particular, this %implies 
can be interpreted as 
that 
$\overline{ \bigcup_{\theta \in \Theta} {\rm Sp}\, \mathcal{L}_{\ep,\t} }$ converges when $\ep\to 0$ to $\overline{\bigcup_{\xi \in \mathbb{R}^n} {\rm Sp}\, \mathbb{L}_\xi }$ in the Fell topology (see e.g. \cite[p. 142]{Be}), with a ``rate'' specified by 
\eqref{cor67}.
%\[
%\text{If $[a,b] \cap \overline{\bigcup_{\xi \in \mathbb{R}^n} {\rm Sp}\, \mathbb{L}_\xi }\neq \emptyset\quad$ then $\quad d\Big([a,b]  \cap  \overline{\bigcup_{\xi \in \mathbb{R}^n} {\rm Sp}\, \mathbb{L}_\xi } ,  \overline{ \bigcup_{\theta \in \Theta} {\rm Sp}\, \mathcal{L}_{\ep,\t} }\Big) \le b(b+1)C_{11} \ep$;}
%\]
%\[
%\text{If $[a,b] \cap \overline{\bigcup_{\xi \in \mathbb{R}^n} {\rm Sp}\, \mathbb{L}_\xi } = \emptyset\quad$ then $\quad [a-C_{11} \ep, b+C_{11} \ep] \cap \overline{ \bigcup_{\theta \in \Theta} {\rm Sp}\, \mathcal{L}_{\ep,\t} } = \emptyset$;}
%\]
%\[
%{\rm dist}_S \Big([a,b] \cap \bigcup_{\theta \in \Theta} {\rm Sp}\, \mathcal{L}_{\ep,\t}  , [a,b] \cap \bigcup_{\xi \in \mathbb{R}^n} {\rm Sp}\, \mathbb{L}_\xi \Big) \le b^2 C_{11}\ep.
%\]
\end{corollary}
\begin{proof}%[Proof of Theorem \ref{t.collectivespec}] 
%where $C_b=0$ if $b< 1$, and e.g. $C_b=\max\left\{4C_{12}b, 1\right\}\max\left\{b,\lambda_0^{(1)}\right\}$ if $b\ge 1$  
%where $\lambda_0^{(1)}\ge 1$ is the first eigenvalue of operator $\mathbb{L}_0$ (i.e. of $\mathbb{L}_\xi$ with $\xi=0$). 
Since, for any $\t\in\Theta$ and $\xi\in \mathbb{R}^n$, 
${\rm Sp}\, \mathcal{L}_{\ep,\t}\cup {\rm Sp}\, \mathbb{L}_\xi \subset [1,\infty)$, 
for $b<1$ the left-hand side of \eqref{cor67} vanishes and 
so we can set $C_b=0$ for $b< 1$. 
Let $b\ge 1$, and let for some $\t\in\Theta$, $0<\ep<1$ and $k\in\mathbb{N}$, 
$1\le \lambda:=\lambda^{(k)}_{\ep,\t}\in [a,b]\cap {\rm Sp}\, \mathcal{L}_{\ep,\t}$. 
(The case of 
$\lambda:=\lambda^{(k)}_\xi\in [a,b]\cap {\rm Sp}\, \mathbb{L}_\xi$, $\xi\in\mathbb{R}^n$, is considered in a similar way.) 
Then, by \eqref{symdist.e1}, either $1/\lambda\le C_{12}\,\ep$ or for some $\xi\in\mathbb{R}^n$ and 
$l\in\mathbb{N}$, $\big|1/\lambda-1/\mu\big|\le 2\,C_{12}\,\ep$ where $\mu:=\lambda^{(l)}_{\xi}\ge 1$. 
Assuming first the latter, 
\[
|\lambda-\mu|\,=\,\left|\lambda^{-1}-\mu^{-1}\right|\lambda\mu\,\,\le\,\, 
2\,C_{12}\,\ep\, b \,\big(|\lambda-\mu|\,+\,b\big). 
\]
Therefore, if $\ep<\min\left\{1,\,(4 C_{12}b)^{-1}\right\}=:\ep_b$ then 
$2\,C_{12}\,\ep b< 1/2$ and it follows that $|\lambda-\mu|\le 4 C_{12}b^2\ep$. %\le C_b\ep$, implying \eqref{cor67}. 
Notice that, as $\lambda\le b$, the former case ($1/\lambda\le C_{12}\ep$) is not possible for $\ep<\ep_b$. 
On the other hand, if $\ep\ge\ep_b$, we notice that as $V_0\neq\{0\}$, $\lambda^{(1)}_0<+\infty$. 
(Similarly, from the variational principle, $\lambda^{(1)}_{\ep,0}\le\lambda^{(1)}_0<+\infty$.) 
So, taking instead $\mu=\lambda^{(1)}_0$ (similarly, $\mu=\lambda^{(1)}_{\ep,0}$), 
\[
|\lambda-\mu|\,\le\,\max\left\{b,\lambda^{(1)}_0\right\}\,\le\max\left\{b,\,\lambda^{(1)}_0\right\}\ep_b^{-1}\ep.%\,=\,C_b\ep. 
\]
So it would suffice, for $b\ge 1$,  to take 
$C_b=\max\left\{b,\,\lambda^{(1)}_0\right\}\ep_b^{-1}=
\max\left\{b,\lambda_0^{(1)}\right\}\max\left\{4C_{12}b, 1\right\}\ge 4 C_{12}b^2$. 
\end{proof}
%\begin{remark}
%If $H$ ({\it respectively} $V_\star \dot{+} Z$) is infinite-dimensional then $\tilde{\sigma}_\ep = \sigma_\ep$ ({\it respectively} $\tilde{\sigma}_0 = \sigma_0$).
%\end{remark}
\begin{proof}[Proof of Theorem \ref{t.collectivespec}] 
%	Set $\sigma_\ep = \bigcup_{\t \in \Theta} {\rm Sp}\, \mathcal{L}_{\ep,\t}^{-1} \cup  \{ 0 \} $, $\sigma_0 : = \bigcup_{\xi \in \mathbb{R}^n} {\rm Sp}\, \mathbb{L}^{-1}_\xi \cup  \{ 0 \}  $.
Note that \eqref{symdist.e1} follows if we establish that: 
\begin{gather}
\label{syd.e2} \text{for every } l_\ep \in \bigcup_{\t \in \Theta} {\rm Sp}\, \mathcal{L}_{\ep,\t}^{-1} \cup  \{ 0 \}  \text{ there exists } l_0 \in \bigcup_{\xi \in \mathbb{R}^n} {\rm Sp}\, \mathbb{L}^{-1}_\xi \cup  \{ 0 \}  \ \text{such that}\ | l_\ep - l_0 | \le C_{12} \ep; \\
\label{syd.e3} \text{for every }  l_0 \in \bigcup_{\xi \in \mathbb{R}^n} {\rm Sp}\, \mathbb{L}^{-1}_\xi \cup  \{ 0 \}  \text{ there exists }  l_\ep \in \bigcup_{\t \in \Theta} {\rm Sp}\, \mathcal{L}_{\ep,\t}^{-1} \cup  \{ 0 \}  \ \text{such that}\ | l_0 - l_\ep | \le C_{12} \ep.
\end{gather}
Now Theorem \ref{ikthm2} implies \eqref{syd.e2} and additionally implies %the inequality in 
\eqref{syd.e3} for all 
$l_0 \in  \bigcup_{\xi \in \ep^{-1} \Theta} {\rm Sp}\, \mathbb{L}^{-1}_\xi  \cup \{ 0 \}$. 
% that 
%\[
%{\rm dist}_S\, \big(\tilde{\sigma}_\ep, C_\ep\big) \le \mathcal{C}_{9} \ep, \quad \text{ where } \mathcal{C}_\ep := \big( \bigcup_{\xi \in \ep^{-1} \Theta} {\rm Sp}\, \mathbb{L}^{-1}_\xi \big) \cup \{ 0 \}.
%\]
Therefore, in order to establish \eqref{syd.e3} it remains to consider arbitrary  $l_0 \in {\rm Sp}\, \mathbb{L}^{-1}_\xi$,  $\xi \in \mathbb{R}^n \backslash \ep^{-1} \Theta$. 
Recalling that $\theta=0$ is assumed to be an interior point of $\Theta$, let 
 $0<\ep<1$ and let $\xi\in \mathbb{R}^n\backslash \ep^{-1}\Theta$ i.e. $|\xi|>\ep^{-1}R$ where $R>0$ is radius of the largest closed ball centred at the origin that is contained in $\Theta$. So, for small $\ep$, we are interested in 
approximating the spectrum 
${\rm Sp}\, \mathbb{L}^{-1}_\xi$ for large $|\xi|$. 

The plan is first to use for such a case of $\mathbb{L}^{-1}_\xi$ with a large $|\xi|$ 
the methods and results of  Section \ref{s.spcontV}, 
regarding $\tilde\ep=|\xi|^{-1}$ as a small parameter. 
As a result, we will see that the spectrum of $\mathbb{L}^{-1}_\xi$ is approximated by that of an 
operator whose spectrum is identical to the one of another operator approximating in turn 
$\mathcal{L}_{\ep,\t}^{-1}$. 

%and to related $\mathcal{L}_{\ep,\t}^{-1}$ (with both viewed as original operators), and to show that 
%The idea is that 
%the spectra of $\mathbb{L}^{-1}_\xi$ and of related $\mathcal{L}_{\ep,\t}^{-1}$ can 
%both be approximated by those of respective operators with an identical spectra.  
%this spectrum can be approximated 
%using the methods and results of  Section \ref{s.spcontV}. 

To that end, for any %$0<\ep<1$ and 
$\xi\in\mathbb{R}^n\backslash\{0\}$,  
%(hence with $|\xi|>\ep^{-1}R>R$), 
introduce ``wiggled'' objects as follows. Let $\tilde\ep:=|\xi|^{-1}$ and $\tilde\theta:= \xi/|\xi|$, and so 
$\xi=\tilde\t/\tilde\ep$ with $\tilde\ep>0$ and 
$\tilde\t\in\tilde\Theta= {S}^{n-1}$ 
the unit sphere %of radius $R$ 
in $\mathbb{R}^n$ centered at the origin. 
We also set 
\begin{equation}
\label{wigprobl}
\begin{aligned}
\widetilde{H} = V_\star \dot{+} Z, \quad %\widetilde{\Theta} = \mathbb{S}^n, 
\quad \widetilde{A}_{\tilde{\ep},\tilde{\theta}}[v+z]\,: = \,
\tilde{\ep}^{\,-2} \,a^{\rm h}_{\tilde{\theta}}[z] + b_0[v+z]\,=\,
\mathbb{S}_{\tilde\t/\tilde\ep}[v+z]\,=\,\mathbb{S}_\xi[v+z], \quad \tilde{\theta} \in \widetilde{\Theta},  
\end{aligned}
\end{equation}
i.e., for any $\tilde\t\in\widetilde{\Theta}$, we set 
$\tilde a_{\tilde\t}(v+z,\tilde v+ \tilde z) =a^{\rm h}_{\tilde{\theta}}(z, \tilde z)$,  
$\tilde b_{\tilde\t}(v+z,\tilde v+ \tilde z)=b_0(v+z,\tilde v+ \tilde z)$, with respective inner product  
$(v+z,\tilde v+ \tilde z)_{\tilde\t}= a^{\rm h}_{\tilde{\theta}}(z, \tilde z)+b_0(v+z,\tilde v+ \tilde z)$. 
Then $\widetilde V_{\tilde\t}=V_\star$ and 
$\widetilde W_{\tilde\t}=\bigl\{v+z\in V_\star \dot{+} Z \,\,\big\vert \,\, b_0(v+z, \tilde v)=0, \ \forall \tilde v\in V_\star\bigr\}$. 
Now we notice that for the above wiggled objects all 
the assumptions of Theorem \ref{thm:contV} %Section \ref{s.spcontV}  
hold. In particular, 
\eqref{bddspec} can be seen to hold as follows. 
%\eqref{ahcoercive} implies 
%\eqref{bddspec} with $\nu=\tilde\nu:=\nu_\star/(1+\nu_\star)$. 
%Indeed, 
For $\tilde\t\in\widetilde{\Theta}$ and $w=v+z\in \widetilde W_{\tilde\t}$, via \eqref{ahcoercive}, 
\[
a^{\rm h}_{\tilde{\theta}}[z]\,\,\ge\,\, \nu_\star \,\|z\|_0^2
\,\,=\,\,\nu_\star\,b_0[z]\,\,=\,\,\nu_\star\,b_0\big[(v+z)-v\big]
\,\,\ge\,\,
\nu_\star\,b_0[v+z].
\]
(In the last inequality we used that $w=v+z$ and $v$ are orthogonal with respect to $b_0$.) 
As a result, for any $0<\tilde\nu<1$, 
\[
\tilde a_{\tilde\t}[w]\,\,=\,\,a^{\rm h}_{\tilde{\theta}}[z]\,\,\ge\,\,
\tilde\nu\,a^{\rm h}_{\tilde{\theta}}[z]\,+\,
(1-\tilde\nu)\,\nu_\star \,b_0[v+z].  
\]
Hence choosing $\tilde\nu$ so that $\tilde\nu=(1-\tilde\nu)\,\nu_\star$, i.e. 
$\tilde\nu:=\nu_\star/(1+\nu_\star)$, implies 
\[
\tilde a_{\tilde\t}[w] 
\,\,\ge\,\,\tilde\nu\,\Big( a^{\rm h}_{\tilde{\t}}[z]+b_0[v+z]\,\Big)\,\,=\,\,
\tilde\nu\, \|w\|_{\tilde\t}^2, 
\]
and therefore \eqref{bddspec} holds with $\nu=\tilde\nu$. 

Next, for the assumptions of Section \ref{s.spcontV}, we set 
$\widetilde{\mathcal{H}}:=\overline{V_\star \dot{+} Z}$ with $d_{\tilde\t}=d_0$, $\forall\t\in\Theta$. 
So  $\widetilde{\mathcal{L}}_{\tilde{\ep},\tilde{\theta}}$ is the self-adjoint operator in  
$\left(\overline{V_\star \dot{+} Z}, d_0\right)$ 
generated by  $\tilde{A}_{\tilde{\ep},\tilde{\theta}}$ with form domain  $V_\star\dot{+} Z$.
Further, 
$\mathbf{\widetilde B}_{\tilde\t}$ becomes in this setting  the $\tilde\t$-independent 
self-adjoint operator $ \mathbf{B}$ in $\left(\overline{V_\star},d_0\right)$ generated by $b_0$ with form domain $V_\star$, i.e. (cf. \eqref{ikddd}), 
\begin{equation}
\label{ikddd2} 
d_0({\mathbf{B}} v,\tilde{v})\,\,=\,\,b_0( v,\tilde{v}),\quad  \forall v\in \text{dom}\, {\mathbf{B}}\subset V_\star , \ \ \forall \tilde{v}\in V_\star. 
\end{equation} 
Recall that Theorem \ref{thm:contV} holds for all $\ep>0$ (see Remark \ref{rem3.2}), 
and hence so are the results of Section \ref{s.spcontV}. 
Consequently,  via \eqref{ik889.0}--\eqref{ik889}, we have 
\begin{equation}\label{iliabc}
{\rm dist}_s \left( {\rm Sp}\, \widetilde{\mathcal{L}}^{-1}_{\tilde{\ep},\tilde{\theta}}\cup \{ 0 \} \,,\, {\rm Sp}\,  \mathbf{ B}^{-1}\cup \{ 0 \}\right)\,\, \le \,\,\tilde{\ep}^{2}\, \tilde\nu^{-1}, \quad  \forall\,\,\tilde{\ep}>0, \, \, \forall \,\tilde{\theta} \in {S}^{n-1}.
\end{equation}  
%and $ \mathbf{\widetilde B}$ is the self-adjoint operator in $\overline{V_\star}$ generated by $b_0$ with form domain $V_\star$.
 Notice that, see \eqref{wigprobl},  
$\mathbb{L}^{-1}_{ \xi} = \widetilde{\mathcal{L}}^{-1}_{\tilde\ep,\tilde\t}$, and therefore \eqref{iliabc} implies 
\begin{equation}\label{bdrylimspec}
{\rm dist}_s\left( {\rm Sp}\, \mathbb{L}^{-1}_{ \xi} \cup \{ 0 \} \,,\, {\rm Sp}\,  
\mathbf{B}^{-1}\cup \{ 0 \}\right) \,\,\le\,\,  |\xi|^{-2} \tilde\nu^{-1}, \quad   \forall \xi \in \mathbb{R}^n \backslash  \{ 0 \}. 
\end{equation}
In particular, for every  $l_0 \in {\rm Sp}\, \mathbb{L}^{-1}_\xi$, 
$\xi \in \mathbb{R}^n \backslash \ep^{-1} \Theta$ (hence $|\xi|>\ep^{-1}R$), 
there exists $\mu \in {\rm Sp}\,  \mathbf{B}^{-1}\cup \{ 0 \}$ such that 
\begin{equation}
\label{syd.e4}
\left|\, l_0 \,-\, \mu\, \right| \,\, \le \,\,|\xi|^{-2}\, \tilde\nu^{-1}\,\, < \,\,
 \ep^2\,R^{-2}\, \tilde\nu^{-1}.  
\end{equation}
%where $R$ is radius of the largest  closed ball centred at the origin that is  contained in $\Theta$. 

	On the other hand, from \eqref{ikddd} and \eqref{ikddd2} we observe via  
	\eqref{Eprop1} and \eqref{H6} that, for every $\t \in \Theta \backslash \{0\}$,  
	$\mathbf{B}_\t^{-1} = \mathcal{E}_\t  \mathbf{B}^{-1} \mathcal{E}_\t^{-1}$ and thus 
	${\rm Sp}\, \mathbf{B}_\t^{-1} ={\rm Sp}\,  \mathbf{B}^{-1}$.  Consequently, Remark \ref{r.ikth77} implies that
for the above $\mu \in {\rm Sp}\,  \mathbf{B}^{-1}\cup \{ 0 \}={\rm Sp}\,  \mathbf{B}_\t^{-1}\cup \{ 0 \}$ with any chosen $\t\in\Theta$ with $|\t| = R$, 
there exists  $l_\ep \in {\rm Sp}\, \mathcal{L}^{-1}_{\ep,\t} \cup \{0\}$,   such that  
\begin{equation}
\label{syd.e5}
\left| \,\mu \,-\, l_\ep \,\right|  \,\,\le\,\,  \ep^2 \gamma^{-1} R^{-2}.
\end{equation}
Thus, \eqref{syd.e4} and \eqref{syd.e5} imply that \eqref{syd.e3} holds for $l_0 \in \bigcup_{\xi \notin \ep^{-1} \Theta} {\rm Sp}\, \mathbb{L}^{-1}_\xi $ and the proof is complete. 
 \end{proof}


\subsection{Characterisation of the limit collective spectrum $\overline{{\bigcup_{\xi \in \mathbb{R}^n} {\rm Sp}\, \mathbb{L}_\xi}  }$}
\label{s.charlimsp}
As Corollary \ref{c.collspec} provides an approximation of the collective spectrum 
 for the original problem $\overline{\bigcup_{\t \in \Theta} {\rm Sp}\, \mathcal{L}_{\ep,\t}}$ in terms 
of the ``limit collective spectrum'' 
$\overline{{\bigcup_{\xi \in \mathbb{R}^n} {\rm Sp}\, \mathbb{L}_\xi}  }$, 
we aim here at characterising the latter limit spectrum. 

If $\lambda$ is an eigenvalue of $\mathbb L_{\xi}$ with some eigenvector $0\neq v+z\in V_\star \dot{+}Z$, 
then by \eqref{Sform} the spectral problem reads
 	\begin{equation}
	\label{lxispec}
	\mathbb{S}_\xi(v+z,\tilde v+\tilde{z}) \,  =\, 
 	a_\xi^{\rm h}(z,\tilde{z}) \,+\, b_0(v+z,\tilde v+\tilde{z}) \,  =\, \lambda\,\, d_0(v+z,\tilde{v} +\tilde{z}), \quad \forall\, \tilde{v} + \tilde{z} \in V_\star \dot{+} Z.  
% 	\text{ where } \mu =\lambda-1,
 	\end{equation} 
	For deriving  useful expressions for the above eigenvalues $\lambda$, it would help if it were possible to 
	eliminate $v$ from \eqref{lxispec} by expressing it in terms of a self-adjoint operator acting on $z$. 
	%uncouple in \eqref{lxispec} the problems for $v$ from that for $z$ by 
	%expressing means of a self-adjoint operator on $\overline{V_\star}$. 
	This is prevented by the coupling terms $b_0(z,\tilde v)$ and $d_0(z,\tilde v)$, and 
	we aim at overcoming this by first showing that the defect subspace $Z$ (satisfying   \eqref{spaceZ} and \eqref{VZorth}) can always be chosen so that
 	\begin{equation}\label{zvbd}
 	b_0(z, v_\star)\, =\, d_0(z,v_\star), \quad \forall  z \in Z, \ \ 
	\forall v_\star \in V_\star.
 	\end{equation}	
	In most of the specific examples, Section \ref{sec:examples}, \eqref{zvbd} naturally holds. 
	In our general abstract setting, we show that \eqref{zvbd} can always be achieved by selecting $Z$ appropriately. 
	With  this aim, we assume without loss of generality that 
 	\begin{equation}
	\label{kdass}
 	d_0[v_0] \, \le\, M_d\, b_0[v_0], \quad \forall v_0 \in V_0=V_\star\dot{+} Z, \quad 0<M_d <1.
 	\end{equation} 
	(This is a small strengthening of \eqref{ik2} for $\t=0$, which can always be achieved 
	by re-defining $d_\t$ via multiplying it by any positive constant factor smaller than 1, with 
	corresponding rescaling of the spectrum.) 
	
 Aiming at \eqref{zvbd}, let first $Z'$ 
%where $P_{W_*}$ is the orthogonal projector onto $W_*$ with respect to $(\cdot,\cdot)_0$ is 
be the orthogonal complement of $V_\star$ in $V_0$ 
(with respect to $b_0=(\cdot,\cdot)_0$ on $V_0$), cf 
Remark \ref{zorth}. So, for any 
	$z'\in Z'$ and $v_\star\in V_\star$, $b_0(z',v_\star)=0$ and 
	the idea is to construct $Z$ by ``correcting'' $Z'$ via 
adding to $z'$ some $Tz'\in V_\star$, i.e. to seek $Z\ni z=z'+Tz'$. Then \eqref{zvbd} reads 
$b_0(z' +Tz', v_\star)\, =\, d_0(z'+Tz',v_\star)$, which can be restated as a problem for  
$T : Z'\rightarrow V_\star$ as follows: 
\begin{equation}
	\label{tuprobl}
 	b_0(Tz',\tilde{v}) \,-\, d_0(Tz',\tilde{v}) \,\,=\,\, d_0(z',\tilde{v}) \,-\, b_0(z',\tilde{v})
	\,=\,\, d_0(z',\tilde{v}), \quad \forall \tilde{v} \in V_\star.
 	\end{equation}
Notice that problem \eqref{tuprobl} is well-posed on $V_\star$ due to the coercivity implied by 
\eqref{kdass}, and so uniquely 
determines	a $\|\cdot\|_0$-bounded injective linear operator 
$T : Z'\rightarrow V_\star$,  %$u \mapsto T w_* $ is the unique solution to
 	and $Z:= (I+ T) Z'$ satisfies  \eqref{zvbd}. 
	Show now that $V_0 = V_\star\dot{+}Z$. For any $v_0\in V_0$, seek $v_\star\in V_\star$ and $z\in Z$ i.e. 
$z'\in Z'$ such that for $z=z'+Tz'$, 
$v_0=v_\star+z=\left(v_\star+T z'\right)+ z'=
 P_{V_\star} v_0+P_{Z'} v_0$. 
This uniquely determines $z'=P_{Z'} v_0$, $v_\star=P_{V_\star} v_0-T P_{Z'} v_0$ 
and $z=v_0-v_\star$, so $V_0 = V_\star\,\dot{+}\,Z$.
%Also, if $v_0=0$, then $z'=0$, $v_\star=0$ and $z=0$, hence a direct sum. 
 
Furthermore, via \eqref{zvbd} and \eqref{kdass}, for all $z \in Z$ and $v_\star \in V_\star$, 
 	\[
 	\left|(v_\star,z)_0\right|\,=\,|b_0(v_\star,z) |\,=\, 
	|d_0(v_\star,z) |\,\,\le\,\, d_0^{1/2}[v_\star]\, d_0^{1/2}[z]\,\,\le \,\,  
	M_d\,  b_0^{1/2}[v_\star]\, b_0^{1/2}[z]
\,\,=\,\,M_d\,\|v_\star\|_0\,\|z\|_0. 
 	\]	
 	Therefore \eqref{spaceZ} and \eqref{VZorth} hold  with $K_Z=M_d$. 
	Notice that $Z$ is closed (and therefore also weakly closed) linear subspace, as follows e.g. from 
	the closedness of $Z'$ and the boundedness of $T$. 
	%eqref{VZorth2} implied by \eqref{VZorth}.
 	
 	Now, by \eqref{zvbd}, the spectral problem \eqref{lxispec} can be rewritten as 
 	\begin{flalign*}
 	a_\xi^{\rm h}(z,\tilde{z})\,+\,b_0(z,\tilde{z})\,+\,b_0(v,\tilde{v})\,+\,d_0(z,\tilde{v})\,+
	\,d_0(v,\tilde{z})\,\,=\,\,\lambda\,\, d_0(v+z,\tilde{v} +\tilde{z}), 
	\quad \forall\, \tilde{v} + \tilde{z} \in V_\star \dot{+} Z, 
% 	\text{ where } \mu =\lambda-1,
 	\end{flalign*} or equivalently,
 	\begin{gather}
 	\label{03.09.20e1} b_0(v,\tilde{v})  \,-\, \lambda d_0(v,\tilde{v})\, =\, (\lambda-1)\, d_0(z,\tilde{v} ), \quad \forall\, \tilde{v}\in V_\star; \\
 	a_\xi^{\rm h}(z,\tilde{z}) \, =\, -\, b_0(z,\tilde{z}) \,+\, \lambda\, d_0(z,\tilde{z}) \,+\, 
	(\lambda-1)\, d_0(v, \tilde{z}), \quad \forall\, \tilde{z} \in  Z. 
 	\label{03.09.20e2}
 	\end{gather}
 	It follows from \eqref{03.09.20e1}  that $v\in {\rm dom}\, \mathbf{B}$ where 
	$\mathbf{B}$ is  the operator introduced in Section \ref{s.spbt}, see \eqref{ikddd2}, 
	i.e. self-adjoint operator in $\overline{V_\star}$ generated by $b_0$ with form domain $V_\star$. 
	Furthermore,  if $\lambda \notin {\rm Sp}\, \mathbf{B}$,  $v=v_\lambda(z)$ is uniquely found from $z$: 
	\begin{equation} 
	\label{vfromz}
	v\,\,=\,\,v_\lambda(z)\,=\, (\lambda-1) (\mathbf{B}-\lambda I)^{-1} \mathcal{P}_{\overline{V_\star}}^0\,z. 
	%\ \ \mbox{for some } z\neq 0. 
	\end{equation}
	For such $\lambda$,  \eqref{03.09.20e2} implies that $z\in Z\backslash \{0\}$ solves
 	\begin{equation}\label{finallimitspectralproblem}
 	a_\xi^{\rm h}(z,\tilde{z}) \,\,=\,\, \beta_\lambda (z,\tilde{z}),  \quad \forall  \tilde{z} \in Z,
 	%+ b_0(z,\tilde{z}) -d(z,\tilde{z}) = d(\beta(\mu)z,\tilde{z}), \quad \forall  \tilde{z} \in Z,
 	\end{equation}
 where, for $\lambda\notin {\rm Sp}\, \mathbf{B}$, $\beta_\lambda : Z\times Z \rightarrow \CC$	is the sesquilinear form
\begin{equation}
\label{betadef}
\beta_\lambda(z,\tilde z)\,\,:=\,\,-\,\,b_0(z,\tilde z)\,+\,\lambda\, d_0(z,\tilde z)\,+\,
(\lambda-1)d_0\left(v_\lambda(z),\tilde z\right), 
\end{equation} 
where $v_\lambda(z)$ is the unique solution to \eqref{03.09.20e1}. 
Show that, for real $\lambda\notin {\rm Sp}\, \mathbf{B}$, 
form $\beta_\lambda$ is 
complex-Hermitian symmetric, i.e.  $\beta_\lambda(\tilde z,z)=\overline{\beta_\lambda(z,\tilde z)}$ for all $z,\tilde z\in Z$; 
in particular $\beta_\lambda[z]:=\beta_\lambda(z,z)$ is real-valued. 
Indeed, given $z,\tilde z\in Z$, 
by setting in \eqref{03.09.20e1} $z=\tilde z$ and $\tilde v=v_\lambda(z)$ and combing with \eqref{betadef} 
and using the symmetry of $b_0$ and $d_0$, we conclude 
\[
\beta_\lambda(z,\tilde z)\,=\,-\,b_0(z,\tilde z)\,+\,\lambda\, d_0(z,\tilde z)\,+\,
b_0\big(v_\lambda(z),v_\lambda(\tilde z)\big)\,-\,
\lambda\,\, d_0\big(v_\lambda(z),v_\lambda(\tilde z)\big)\,\,=\,\,
\overline{\beta_\lambda(\tilde z,z)}.  %\,\in\,\mathbb{R}. 
\]
For an equivalent operator interpretation of $\beta_\lambda$, from \eqref{betadef} and \eqref{vfromz}, 
\begin{equation}\label{betaform}
\beta_\lambda (z,\tilde{z}) \,\,=\,\, -\,\, b_0(z,\tilde{z}) \,+\,\, d_0\big(\beta(\lambda)z,\,\tilde{z}\big)
\end{equation} 
 for $\beta(\lambda) : \overline{Z} \rightarrow \overline{Z}$ the bounded linear operator
\begin{equation}\label{6.27-2}
	\beta(\lambda)  \,\,:=\,\,\lambda\, I \,+\, (\lambda-1)^2\, \mathcal{P}_{\overline{Z}}^0\,
	(\mathbf{B}-\lambda I)^{-1} \mathcal{P}_{\overline{V_\star}}^0\,.
\end{equation}
%Indeed, from \eqref{betaform} and \eqref{6.27-2}, it remains to see that so is 
%$d_0\left( \mathcal{P}_{\overline{Z}}^0\,
%	(\mathbf{B}-\lambda I)^{-1} \mathcal{P}_{\overline{V_\star}}^0\,z\,,\,z\right)$. Now, 
%	\[
%d_0\left( \mathcal{P}_{\overline{Z}}^0\,
%	(\mathbf{B}-\lambda I)^{-1} \mathcal{P}_{\overline{V_\star}}^0\,z\,,\,z\right)\,=\,
%		d_0\left(  
%	(\mathbf{B}-\lambda I)^{-1} \mathcal{P}_{\overline{V_\star}}^0\,z\,,\,z\right)\,=\, 
%	d_0\left(  
%	(\mathbf{B}-\lambda I)^{-1} \mathcal{P}_{\overline{V_\star}}^0\,z\,,\,\mathcal{P}_{\overline{V_\star}}^0z\right)
%	\]
%	which is real-valued. % as $(\mathbf{B}-\lambda I)^{-1}:\overline{V_\star}\rightarrow  \overline{V_\star}$ 
%	is self-adjoint for $\lambda\in\mathbb{R}\backslash {\rm Sp}\, \mathbf{B}$. 

 Now notice that, since $a^{\rm h}_\xi[z]$ is non-negative real for all $\xi$ and $z$,  \eqref{finallimitspectralproblem} immediately implies the following inclusion:
\begin{equation}\label{easyspinc}
\bigcup_{\xi \in \mathbb{R}^n} {\rm Sp}\, \mathbb{L}_\xi \,\subseteq\, \Bigl\{ \lambda \notin {\rm Sp}\, \mathbf{B} : \beta_{\lambda }[z] \ge 0 \text{ for some $0\neq z \in Z$} \Bigr\} \cup {\rm Sp}\, \mathbf{B}.
\end{equation}
%or
%\begin{equation*}
%\sigma_0 \subseteq \{ \mu \notin {\rm Sp}\, \mathbf{D}^{-1} : \beta_{1/\mu }[z] \ge 0 \text{ for some $0\neq z \in Z$} \} \cup {\rm Sp}\, \mathbf{D}^{-1},
%\end{equation*}
In fact we have a stronger assertion, providing the following important characterisation of the limit spectrum in terms of the form $\beta_\lambda$. 
\begin{theorem}\label{thm.limspecrep} The following characterisation of the collective limit spectrum holds:
\begin{equation*}
\overline{\bigcup_{\xi \in \mathbb{R}^n} {\rm Sp}\, \mathbb{L}_\xi} \, =\, \Bigl\{ \lambda \notin {\rm Sp}\, \mathbf{B} : \beta_{\lambda }[z] \ge 0 \text{ for some $0\neq z \in Z$} \Bigr\} \cup {\rm Sp}\, \mathbf{B}.
\end{equation*}
%
%Let $\Xi$ consist of the eigenvalues of $\mathbf{B}+I$ whose eigenspace is orthogonal to $Z$ with respect to $d$ (equivalently $b_0$). Then
%\[
%\bigcup_{\xi \in \mathbb{R}^n} {\rm Sp}\, \mathbb{L}_\xi ={ \{  1+ \mu \, 	| \,   \beta_\mu[z] \ge 0 \text{ for some $z \in Z \backslash \{ 0\}$} \} } \cup \Xi
%% {\rm Sp}\, (\mathbf{B}+I)
%\]
\end{theorem}
\begin{proof}
%	[Proof of Theorem \ref{thm.limspecrep}]

{{\it Step 1.} 
	Here we prove
	\begin{equation*}
	\overline{\bigcup_{\xi \in \mathbb{R}^n} {\rm Sp}\, \mathbb{L}_\xi }\,\subseteq\, 
	\mathcal{C} \,: =\,\Bigl\{ \lambda \notin {\rm Sp}\, \mathbf{B} : \beta_{\lambda }[z] \ge 0 \text{ for some $0\neq z \in Z$} \Bigr\} \cup {\rm Sp}\, \mathbf{B}.
	\end{equation*}
By \eqref{easyspinc} it is sufficient to show that	
	$\mathcal{C}$ is closed. Recalling that ${\rm Sp}\, \mathbf{B}$ is closed, let $\lambda_n \in \mathcal{C}$ be such that 
	$\lim_n \lambda_n = \lambda \notin {\rm Sp}\, \mathbf{B}$, and consider $n$ large enough so that  
	${\rm dist} ( \lambda_n , {\rm Sp}\, \mathbf{B}) \ge \delta >0$. Then 
	$\left\| (B-\lambda_n I)^{-1}\right\|_{(V_\star,d_0) \rightarrow (V_\star,d_0)}\leq \delta^{-1}$, and so 
	by \eqref{6.27-2} 
	$\| \beta(\lambda_n) \|_{(\mathcal{H},d_0) \rightarrow (\mathcal{H},d_0)} $ is bounded,  and there exists $z_n \in Z$, $d_0[z_n] =1$ and $\beta_{\lambda_n}[z_n] \ge 0$. 
Via \eqref{betaform} these assertions imply that $b_0[z_n]$ is bounded, and 
%%Since $\beta_{\lambda_n}[z_n] \ge 0$ and $\| \beta(\lambda_n) \|_{B(\mathcal{H})} \le \lambda + \lambda^2 / \delta $ then
%\[
%b_0[z_n] \le d[z_n] + d(\beta(\lambda_n) z_n,z_n)  \le 1 + \lambda + \lambda^2 / \delta .
%\]
so (up to a subsequence) $z_n$ converges weakly  in $H$, and strongly in $\mathcal{H}$, to some $z \in Z$, $d_0[z]=1$ and by weak lower semi-continuity of $b_0$, 
 $b_0[z] \le {\liminf}_n b_0[z_n]=:\underline{\lim}_n b_0[z_n]$.  We show that all these assertions imply  that 
$\beta_\lambda[z] \ge 0$. Indeed, for $\mathcal{R}_{\mu} :=(\mathbf{B} - \mu I)^{-1}$, we calculate via 
\eqref{betadef} and \eqref{vfromz} 
\[
\beta_{\lambda_n}[z_n]  \,= \,-\,\, b_0[z_n] \,+\,\lambda_n\, d_0[z_n]  \,+\, (\lambda_n -1)^2\,
d_0\left( \,\mathcal{R}_{\lambda_n} \mathcal{P}_{\overline{V_\star}}^0 z_n\,,\,z_n\,\right) \,\,=
\]
\[
\ \ \ \ \ \ \ \ 
\, -\,\, b_0[z_n]\, +\,\,\lambda_n\, d_0[z_n] \, +\, (\lambda_n -1)^2\,d_0\left( \,\mathcal{R}_{\lambda} \mathcal{P}_{\overline{V_\star}}^0 z_n\,,\,z_n\,\right) \,+\, (\lambda_n-\lambda)(\lambda_n -1)^2 
d_0\left(\, \mathcal{R}_{\lambda_n} \mathcal{R}_{\lambda} \mathcal{P}_{\overline{V_\star}}^0 z_n\,,\,z_n\,\right),  
\] 
having used in the last equality the standard resolvent identity 
$\mathcal{R}_{\lambda_n} - \mathcal{R}_{\lambda}=(\lambda_n-\lambda)\mathcal{R}_{\lambda_n} \mathcal{R}_{\lambda}$. 
By taking the limit superior (and recalling $\beta_{\lambda_n}[z_n] \ge 0$)  we obtain  
\[
0 \,\le\, \overline{\lim}_n\, \beta_{\lambda_n}[z_n] \,\le \,-\,\, \underline{\lim}_n b_0[z_n] \,+\,
\lambda d_0[z]  \,+\, (\lambda -1)^2d_0
\left(\, \mathcal{R}_{\lambda} \mathcal{P}_{\overline{V_\star}}^0\, z\,,\,z\,\right) \,\, \le 
\]
\[
\ \  \ \ \ \ \ \ \ \ \ \ \ \quad 
- \,\, b_0[z] \,+\,\lambda\, d_0[z] \, +\, (\lambda -1)^2
d_0\left(\, \mathcal{R}_{\lambda} \mathcal{P}_{\overline{V_\star}}^0 \,z\,,\,z\,\right) 
\,=\,\, \beta_\lambda[z].
\]
%Additionally, $\beta(\mu)$ is continuous with respect to  $\mu$ in $B(\mathcal{H})$ and, therefore, $d( \beta(\lambda_n) z_n,z_n)$ converges to  $d( \beta(\lambda) z,z)$. These assertions collectively imply that for every $\ep >0$ there exists $N(\ep)$ such for all $n \ge N(\ep)$
%\begin{gather*}
%d[z] \ge d[z_n] - \ep, \qquad d( \beta(\lambda) z,z) \ge d( \beta(\lambda_n) z_n,z_n) - \ep, \qquad b_0[z] \le b_0[z_n]+\ep. 
%\end{gather*} Thus
%\[
%\beta_{\lambda}[z]  = d[z] - b_0[z] + d( \beta(\lambda) z,z)\ge d[z_n] - b_0[z_n] + d( \beta(\lambda_n) z_n,z_n) - 3\ep = \beta_{\lambda_n}[z_n] -3 \ep \ge - 3\ep, \quad \forall \ep>0.
%\]
That is $\beta_\lambda[z]\ge 0$ and since  $0\neq z \in Z$ it follows that $\lambda \in \mathcal{C}$. Hence $\mathcal{C}$ is closed, as required. 

{\it Step 2.} Let us now prove
\begin{equation*}
\mathcal{C}
%\{ \lambda \notin {\rm Sp}\, \mathbf{D} : \beta_{\lambda }[z] \ge 0 \text{ for some $0\neq z \in Z$} \} \cup {\rm Sp}\, \mathbf{D} 
\subseteq \overline{\bigcup_{\xi \in \mathbb{R}^n} {\rm Sp}\, \mathbb{L}_\xi }.
\end{equation*}
We shall consider two cases. First, consider  
$\lambda \in {\rm Sp}\, \mathbf{B}\subset [1,+\infty)$. Notice that, by \eqref{bdrylimspec},  for every $\lambda \in {\rm Sp}\, \mathbf{B}={\rm Sp}\, \widetilde{\mathbf{B}}$ and for any $\xi\in\mathbb{R}^n$ such that 
$|\xi|>(\lambda/\tilde\nu)^{1/2}$ there is a $\lambda_\xi \in {\rm Sp}\, \mathbb{L}_{\xi} $  such that
$
\left| \lambda^{-1} - \lambda_\xi^{-1} \right| \le |\xi |^{-2} \tilde\nu^{-1}.
$
%and so for every $\mu \in {\rm Sp}\, \mathbf{D}$ and every $n\in \mathbb{N}$ there exists $\xi_n \in \mathbb{R}^n$, $|\xi_n| = n$, $\lambda_n \in {\rm Sp}\, \mathbb{L}_{\xi_n}^{-1} \cup \{ 0 \} $ such that
%\[
%| \mu^{-1} - l_n |  \le n^{-2} \nu_\star^{-1}.
%\]
Consequently $\lim_{|\xi| \rightarrow \infty} \lambda_\xi = \lambda$, i.e. $\lambda \in \overline{\bigcup_{\xi \in \mathbb{R}^n} {\rm Sp}\, \mathbb{L}_\xi }.$

Now, we  suppose $\lambda \in \mathcal{C} \backslash  {\rm Sp}\, \mathbf{B}$. Then, there exists  $0\neq z' \in Z$ such that $\beta_\lambda[z'] \ge 0$. Let us fix $\eta \in \mathbb{R}^n$, $|\eta|=1$, and  consider 
\[
k \,:=\, \sup_{z\in Z \backslash \{ 0\}} \,\frac{\beta_\lambda[z]}{a^{\rm h}_\eta [z]}.
\]
By the definition \eqref{betaform}--\eqref{6.27-2} of the form $\beta_\lambda$ implying that $\beta_\lambda[z]/\|z\|_0^2$ is bounded, and by  coercivity of $a^{\rm h}_\eta$, 
see \eqref{ahcoercive}, we see that $k$ is finite and, as $\beta_\lambda[z']\ge0$,  $k$ is non-negative. 
We aim at showing that the above supremum is attained by some point $z_0\in Z\backslash\{0\}$. If $k=0$ we can set 
$z_0=z'$. Let $k>0$, and let $z_n \in Z$, $d_0[z_n]=1$, be such that ${\beta_\lambda[z_n]}/{a^{\rm h}_\eta [z_n]}$ 
converges to $k$. Then, for large enough $n$, $\beta_\lambda[z_n]\ge 0$, and similarly to Step 1 we conclude from 
\eqref{betaform}--\eqref{6.27-2} that $b_0[z_n]$ is bounded. 
Then, by arguing further similarly to Step 1, we see that (up to a subsequence) $z_n$ weakly converges in $H$ to some $0\neq z_0 \in Z$ and that
$\beta_\lambda[z_0] \ge \overline{\lim}_n \beta_\lambda[z_n]$.
%for any $\ep>0$ we have $\beta_\lambda[z] \ge \beta_{\lambda}[z_n] - \ep$ for large enough $n$.
 Furthermore, since the sesqulinear form $a^{\rm h}_\eta$ is bounded and positive (see Proposition \ref{prop.ahom}) 
it is weakly lower semi-continuous and so one has  $a^{\rm h}_\eta[z_0] \le \underline{\lim}_n  a^{\rm h}_\eta [z_n]$. Thus we obtain 
 \[
k \,\,\,\ge\,\,\, \frac{\beta_\lambda[z_0]}{a^{\rm h}_\eta [z_0]} \,\,\,\ge\,\,\, \frac{ \overline{\lim}_n \beta_\lambda[z_n]}{\underline{\lim}_n  a^{\rm h}_\eta [z_n]}  \,\,\, \ge\,\,\, \overline{\lim}_n   \frac{ \beta_\lambda[z_n]}{ a^{\rm h}_\eta [z_n]} \,\, = \,\, k.
 \]
Therefore $k$ is attained by $z_0$, as desired. 
So the sesquilinear form $\widehat{A}(z,\tilde z):=k a^h_{\eta}(z,\tilde{z})-\beta_\lambda(z,\tilde{z})$ 
is non-negative on $Z$ (i.e. $\widehat{A}[z]\ge 0$, $\forall z\in Z$) and vanishes at $z_0\neq 0$, $A[z_0]=0$. 
Therefore, cf. \eqref{Vkersesa}, $\widehat{A}(z_0,\tilde z)=0$ for all $\tilde z\in Z$, i.e. 
\[
\beta_\lambda(z_0,\tilde{z})\, =\, k\, a^{\rm h}_{\eta}(z_0,\tilde{z})\,=\, a^{\rm h}_{k^{1/2}\eta}(z_0,\tilde{z}), \quad \forall \tilde{z} \in Z.
\] 
Hence, cf \eqref{finallimitspectralproblem}, 
$\lambda \in {\rm Sp}\, \mathbb{L}_{\xi}$ for $\xi={k^{1/2}} \eta$ with, according to \eqref{vfromz},  non-zero eigenvector 
$v_0\,+\,z_0\,=\,(\lambda-1)(\mathbf{B}-\lambda I)^{-1} \mathcal{P}_{\overline{V_\star}}^0 z_0 + z_0$. The proof is complete.
% Hence, $\mathcal{C} \backslash  {\rm Sp}\, \mathbf{B} \subseteq \overline{\bigcup_{\xi \in \mathbb{R}^n} {\rm Sp}\, \mathbb{L}_\xi }$.
}
%Let $\lambda \in \overline{\bigcup_{\xi \in \mathbb{R}^n} {\rm Sp}\, \mathbb{L}_\xi }$, then there exists sequences $\xi_n \in \mathbb{R}^n$ and $\lambda_n \in {\rm Sp}\, \mathbb{L}_{\xi_n} $ such that $\lim_n \lambda_n = \lambda$.
%
%If $\xi_n$ is bounded then, up to subsequence, $\lim_n \xi_n = \xi$ and the $\mathcal{H}$-normalised eigenfunctions $v_n + z_n$ weakly converge in $H$ to  some $v+z \neq 0$.  Since the components of $a_0''$ are continuous forms in $H\times H$, and the fact $N$ is bounded, then by the definition of $a^{\rm h}_\xi$ (see \eqref{defhom.form}) we see that there exists $C(\xi) >0$ such that  
%%$a^{\rm h}_\xi(z,\tilde{z})$ is uniformly in $z, \tilde{z}$ locally Lipschitz in $\xi$, that is
%\[
%|a^{\rm h}_\xi (z,\tilde{z}) - a^{\rm h}_{\xi_n} (z,\tilde{z}) | \le C(\xi) |\xi - \xi_n| \| z\|_0 \| \tilde{z}\|_0, \quad \forall z,\tilde{z} \in Z.
%\]
%This allows to pass to the limit in 
%\[
%a_{\xi_n}^{\rm h} (z_n, \tilde{z}) + b_0(v_n + z_n , \tilde{v} + \tilde{z}) = \lambda_n d( v_n + z_n, \tilde{v} + \tilde{z}), \quad \forall \tilde{v} \in V_\star, \tilde{z} \in Z,
%\]
%to show $\lambda \in {\rm Sp}\, \mathbb{L}_{\xi}$ with eigenfunction $v+z$.
%
%If $\xi_n$ is unbounded, then without loss of generality $\lim_n |\xi_n| = + \infty$ and  by \eqref{bdrylimspec}
%% there exists $\mu_n \in {\rm Sp}\, \mathbf{D}$ such that
%%\[
%%| \lambda_n^{-1} - \mu_n^{-1} | \le |\xi_n|^{-2} \nu_\star^{-1},
%%\]
%\[
%{\rm dist} (\lambda_n^{-1}, {\rm Sp}\, \mathbf{D}^{-1}) \le |\xi_n|^{-2} \nu_\star^{-1},
%\]
%and since $\lim_n \lambda_n^{-1}  = \lambda^{-1}$ then ${\rm dist}(\lambda^{-1}, {\rm Sp}\, \mathbf{D}^{-1}) =0 $. Thus $\lambda^{-1}  \in {\rm Sp}\, \mathbf{D}^{-1}$ and consequently $\lambda  \in {\rm Sp}\, \mathbf{D}$.
%
%
\end{proof}
%{\ }\\\\\\\\\\
%One important question from the point of view of applications is to establish if gaps are present in the  set\footnote{In the context of operators with periodic coefficients, this set is the spectrum of an operator.} $\mathbf{J}_\ep= \underset{\t,k}\bigcup \lambda_{\ep,\t}^{(k)}$. We shall address this question here under the assumptions of  Remark \ref{rem.diagspec}.
%
%
%Determining gaps in $\mathbf{J}_\ep$ is equivalent to asking if there is an interval $I_1$ such that $I_1\cap \text{Sp}\, \mathcal{L}_{\ep,\t}^{-1}=\emptyset$ for all $\t\in \Theta$. This question, in turn,  translates to determining if there is an interval $I_2$ such that $I_2\cap \text{Sp}\, (\mathbf{L}_{\ep,\t}^d)^{-1}=\emptyset$ for all $\t\in \Theta$, for sufficiently small $\ep$ (see \eqref{disspec.e1}). Equivalently,
% it is enough to establish the existence of an interval $I_3$ that satisfies $I_3\cap \mathrm{Sp}\, \mathbf{L}_{\ep,\t}^d=\emptyset$ for all $\t\in \Theta$ and all $\ep \in (0,\ep_0)$ for some $\ep_0$.
%
%Remark \ref{r.ikth77} implies that $\rm{Sp}\, \mathcal{L}_{\ep,\t}$ is   approximated by  $\rm{Sp}\, (\mathbf{B} +I)$, for $\t$ separated from zero. In our general setting, we do not have control over the location of $\rm{Sp}\, (\mathbf{B} +I)$  and, as such, gaps may not even exist. Thus one needs  additional hypotheses in order to establish the presence of the gaps. Following the situation that commonly occurs in examples\footnote{ In the ``double porosity" model, which describes media with isolated soft inclusions, say $Q_0$. Then $\mathcal{E}_\t$ stands for multiplication by $\exp{\mathrm{i}\t\cdot x}$ and $K_e = 1$. This operator is unitary in $L^2$ and is isomorphic on space  $H^1_0(Q_0)$, which plays the role of $V_\star$, see Example \ref{e.dp}. \label{ft.dp}}, see Section \ref{sec:examples}, we make the assumption:
%\begin{equation}\tag{H6}
%\label{ikn10}\left\{
%\hspace{10pt}\begin{aligned}
%&\text{for  each $\t\in \Theta$, $\Vs_\t=V_\star$ and there is an operator $\mathcal{E}_\t$,  unitary  in $ \mathcal{H}$ and isomorphic on $V_\star$,}  \\
%&\text{ and a constant $K_e>0$ independent of $\t$ such that } \\
%& (\mathcal{E}_\t v,\mathcal{E}_\t\tilde{v})_0=( v,\tilde{v})_\t, \quad \forall v,\tilde{v}\in V_\star, \qquad \text{and} \qquad  \|\mathcal{E}_\t-1\|_{\mathcal{H}\rightarrow \mathcal{H}}\leq K_e|\t|.
%\end{aligned} \right.
%\end{equation}
%Then, in particular, the spectrum $\mathrm{Sp}\, \mathbf{B}_\t$ is independent of $\t$, i.e. $\mathrm{Sp}\, \mathbf{B}_\t = \mathrm{Sp}\, \mathbf{B}_0$ for all $\t$,
%and
%\begin{equation}\label{ik9000}\big(\mathbf{B}_{0}-\lambda\big)^{-1}\mathcal{E}_\t= \mathcal{E}_\t\big(\mathbf{B}_\t-\lambda\big)^{-1}. 
%\end{equation}
%The following result provides a quantitative affirmation of  the existence  of gaps in $\mathbf{J}_\ep$.
%\begin{proposition}\label{ikthm2} Assume  there exists an interval $(\rho_1,\rho_2)$ such that the form  $\beta'_0(\lambda; \cdot, \cdot)$ is negative for $\lambda\in (\rho_1,\rho_2)$ and  $(\rho_1,\rho_2)\cap \emph{Sp}\, \mathbf{B}_0=\emptyset$. 
% Then, for
%	\begin{equation}\label{ik900}\ep<\nu_*^{1/2}\big(\rho_2^2d^{-1}  K_e K_d \big)^{-1}, \quad d : = {\rm dist}\{ (\rho_1,\rho_2) , {\rm Sp}\, \mathbf{B}_0 \},
%	\end{equation}
%	one has
%	\[(\rho_1,\rho_2)\cap \mathrm{Sp}\, \mathbf{L}_{\ep,\t}^d=\emptyset, \quad \forall\t\in \Theta.\]
%\end{proposition}   
%\begin{proof} Assume to the contrary that $\lambda \in (\rho_1,\rho_2)\cap \text{Sp}\, \mathbf{L}_{\ep,\t}^d$. Then 
%	$\lambda=\Lambda_{\ep,\t}^{(k)}$ for some $k$ and $\t$ and, due to \eqref{11.08.20e3},  there exists $0\neq z\in Z$
%	 such that
%\[
%	\ep^{-2} a_\t^{\rm h}[z] + b_0[z]=\beta'_\t(\lambda;z,z).
%	\]
%	Note  that $\t\neq 0$, indeed  the left-hand side in above equality is non-negative for all $\t$ while the right-hand side is negative for $\t=0$.
%	%For the case $\t\neq 0$, 
%	We intend to arrive at a contradiction by using the fact that the left-hand side is large, when $\ep$ is small, whilst the right-hand side remains bounded. Indeed \eqref{ahcoercive} implies
%	\begin{equation} \label{ik102}
%	(\nu_*\ep^{-2}|\t|^2+1)\| z\|^2_0 \le \ep^{-2} a_\t^{\rm h}[z] + b_0[z] = \beta'_{\t}(\lambda;z,z),
%	\end{equation}
%	and \eqref{b'formula} and \eqref{ik9000} imply
%	%\begin{flalign*}
%	%\label{ik500}
%	%&c(\beta_{\t}(\lambda)z,z)=\lambda c[z] +\lambda^{2}\, c\big((\mathbf{B}_\t-\lambda)^{-1}\mathcal{P}_{\overline{V_\star}}^0 z, \mathcal{P}_{\overline{V_\star}}^0 z\big)=  \lambda c[z] +\lambda^{2}\, c\big((\mathbf{B}_0-\lambda)^{-1} \mathcal{E}_\t\mathcal{P}_{\overline{V_\star}}^0 z,\mathcal{E}_\t\mathcal{P}_{\overline{V_\star}}^0 z\big) \\
%	%& = c(\beta_{0}(\lambda)z,z) + 2\mathrm{Re} \Big( \lambda^{2}\, c\big((\mathbf{B}_0-\lambda)^{-1} (\mathcal{E}_\t-1)\mathcal{P}_{\overline{V_\star}}^0 z, \mathcal{P}_{\overline{V_\star}}^0 z\big) \Big)+
%	%\lambda^{2}\, c\big((\mathbf{B}_0-\lambda)^{-1} (\mathcal{E}_\t-1)\mathcal{P}_{\overline{V_\star}}^0 z,(\mathcal{E}_\t-1)\mathcal{P}_{\overline{V_\star}}^0 z\big).
%	%\end{flalign*}
%	%The first term in the right hand side above  is not positive by assumption and the sum of the last two terms  is bounded by
%	%$$\lambda^2\|(\mathbf{B}_0-\lambda)^{-1}\|\Big(2+\|1-\mathcal{E}_\t\|\Big)\,
%	%\|1-\mathcal{E}_\t\|\,c[z],
%	%$$
%	%which in turn is no greater than 
%	%\[\lambda^2d^{-1} (2+|\t|K_e)|\t| K_ec[z].\]
%	%Then it follows from \eqref{ik2} and \eqref{ik102} that
%	%\begin{flalign*}
%	% (\tfrac{1}{2}\nu_*\ep^{-2}|\t|^2+1)\| z\|^2_\t & \le {\lambda^2}{d^{-1}}(2+|\t|K_e)|\t|K_ec[z] \le  {\rho_2^2}{d^{-1}}(2+|\t|K_e)|\t| K_eK_c \|z\|_\t^2 \\ & \le \big (1+\big({2\rho^4_2K_e^2K_c^2}{d^{-2}}+{\rho_2 ^2K_e^2K_c}{d^{-1}}\big)|\t|^2\big)\|z\|_\t^2,
%	%\end{flalign*}
%	%which contradicts \eqref{ik900}.
%	\begin{flalign*}
%	\label{ik500}
%	&\beta'_{\t}(\lambda;z,z)=\lambda d[z] +\lambda^{2}\, d\big((\mathbf{B}_\t-\lambda)^{-1}\mathcal{P}_{\overline{V_\star}}^0 z, \mathcal{P}_{\overline{V_\star}}^0 z\big)=  \lambda d[z] +\lambda^{2}\, d\big((\mathbf{B}_0-\lambda)^{-1} \mathcal{E}_\t\mathcal{P}_{\overline{V_\star}}^0 z,\mathcal{E}_\t\mathcal{P}_{\overline{V_\star}}^0 z\big) \\
%	& = \beta'_{0}(\lambda;z,z) +   \lambda^{2}\, d\big((\mathbf{B}_0-\lambda)^{-1} \mathcal{P}_{\overline{V_\star}}^0 z, (\mathcal{E}_\t - 1) \mathcal{P}_{\overline{V_\star}}^0 z\big)+
%	\lambda^{2}\, d\big((\mathbf{B}_0-\lambda)^{-1} (\mathcal{E}_\t-1)\mathcal{P}_{\overline{V_\star}}^0 z,\mathcal{E}_\t\mathcal{P}_{\overline{V_\star}}^0 z\big).
%	\end{flalign*}
%	The first term in the right hand side above  is negative, by assumption, and the sum of the last two terms  is bounded by
%	\[2 \lambda^2\|(\mathbf{B}_0-\lambda)^{-1}\|_{\mathcal{H} \rightarrow \mathcal{H}}\|\mathcal{E}_\t-1\|_{\mathcal{H} \rightarrow \mathcal{H}} d[z],
%	\]
%	which in turn is no greater than 
%	\[2\rho_2^2d^{-1}K_e |\t| d[z].\]
%	Then it follows from \eqref{ik102} and \eqref{ik2} that
%	\begin{flalign*}
%	\nu_*\ep^{-2}|\t|^2 \| z\|^2_0 + \|z\|_0^2 & \le 2\rho_2^2d^{-1} |\t| K_ed[z] \le ( \rho_2^4d^{-2} |\t|^2 K_e^2 K_d +K_d^{-1})  d[z] \le  \rho_2^4d^{-2} |\t|^2 K_e^2 K_d^2 \|z\|_0^2 + \|z\|_0^2,
%	% \\ & \le \big (1+\big({2\rho^4_2K_e^2K_c^2}{d^{-2}}+{\rho_2 ^2K_e^2K_c}{d^{-1}}\big)|\t|^2\big)\|z\|_\t^2,
%	\end{flalign*}
%	which contradicts \eqref{ik900}.
%\end{proof}
%{\color{blue}
%	Suppose $\mu_{0}^{(k)} < \lambda < \mu_{0}^{(k+1)}$.
%	
%	
%%	 and $\beta_0(\lambda;\cdot,\cdot)$ is negative. 
%\begin{flalign*}
%d[(\mathbf{B}_0-\lambda)^{-1} v] & =\sum_{m\in \NN}\frac{|d(v,\varphi^{(m)}_\theta)|^2}
%{\mu^{(m)}_0-\lambda} = \sum_{m=0}^{k}\frac{|d(v,\varphi^{(m)}_\theta)|^2}
%{\mu^{(m)}_0-\lambda}  + \sum_{m\ge k+1}\frac{|d(v,\varphi^{(m)}_\theta)|^2}
%{\mu^{(m)}_0-\lambda}  \le  \sum_{m\ge k+1}\frac{|d(v,\varphi^{(m)}_\theta)|^2}
%{\mu^{(m)}_0-\lambda}  \\
%& \le  (\mu_0^{(k+1)}-\lambda)^{-1} \sum_{m\ge k+1}{|d(v,\varphi^{(m)}_\theta)|^2} \le  (\mu_0^{(k+1)}-\lambda)^{-1}  d[v] 
%\end{flalign*}
%Note $(\mathbf{B}_0-\lambda)^{-1}$ is self-adjoint  and so (from identity for $\beta_\t'$ under (5.24))
%	\begin{flalign*}
%\beta'_{\t}(\lambda;z,z)& \le   \lambda^{2}\, d\big( \mathcal{P}_{\overline{V_\star}}^0 z,(\mathbf{B}_0-\lambda)^{-1} (\mathcal{E}_\t - 1) \mathcal{P}_{\overline{V_\star}}^0 z\big)+
%\lambda^{2}\, d\big((\mathbf{B}_0-\lambda)^{-1} (\mathcal{E}_\t-1)\mathcal{P}_{\overline{V_\star}}^0 z,\mathcal{E}_\t\mathcal{P}_{\overline{V_\star}}^0 z\big) \\
%& \le2 \lambda^2 d[z]^{1/2} d[(\mathbf{B}_0-\lambda)^{-1} (\mathcal{E}_\t-1)\mathcal{P}_{\overline{V_\star}}^0 z]^{1/2} \le 2 \lambda^2(\mu_0^{(k+1)}- \lambda)^{-1/2} d[z]^{1/2} d[(\mathcal{E}_\t-1) P_{\overline{V_\star}}z]^{1/2} \\ 
%& \le 2 \lambda^2(\mu_0^{(k+1)}- \lambda)^{-1/2} K_e |\t| d[z]
%	\end{flalign*}
%Hence
%\[
%	\nu_*\ep^{-2}|\t|^2 \| z\|^2_0 + \|z\|_0^2  \le (\lambda^4(\mu_0^{(k+1)}- \lambda)^{-1} K_e^2 |\t|^2 K_d^2 + 1) \|z\|_0^2 \le  (\rho_2^4(\mu_0^{(k+1)}- \mu_0^{(k)})^{-1} K_e^2 |\t|^2 K_d^2 + 1) \|z\|_0^2 
%\]
%and we have result (with no restriction on left end of gap) when $\ep < \nu_\star^{1/2} \rho_2^{-2}(\mu_0^{(k+1)}- \mu_0^{(k)})^{1/2} K_e^{-1} K_d^{-1}$.
%	
%%\begin{flalign*}
%%d\big((\mathbf{B}_0-\lambda)^{-1}\mathcal{E}_\t\mathcal{P}_{\overline{V_\star}}^0 z, \mathcal{E}_\t\mathcal{P}_{\overline{V_\star}}^0 z\big) & =\sum_{m\in \NN}\frac{|d(\mathcal{E}_\t\mathcal{P}_{\overline{V_\star}}^0 z,\varphi^{(m)}_\theta)|^2}
%%{\mu^{(m)}_0-\lambda} = \sum_{m=0 }^{k+1}\frac{|d(\mathcal{E}_\t\mathcal{P}_{\overline{V_\star}}^0 z,\varphi^{(m)}_\theta)|^2}
%%{\mu^{(m)}_0-\lambda} + \sum_{m \ge k+1 }\frac{|d(\mathcal{E}_\t\mathcal{P}_{\overline{V_\star}}^0 z,\varphi^{(m)}_\theta)|^2}
%%{\mu^{(m)}_0-\lambda} \\
%%& \le \sum_{m \ge k+1 }\frac{|d(\mathcal{E}_\t\mathcal{P}_{\overline{V_\star}}^0 z,\varphi^{(m)}_\theta)|^2}
%%{\mu^{(m)}_0-\lambda}  \le (\mu_0^{(k+1)}-\lambda)^{-1}\sum_{m \ge k+1 }{|d(\mathcal{E}_\t\mathcal{P}_{\overline{V_\star}}^0 z,\varphi^{(m)}_\theta)|^2} \\
%%& \le (\mu_0^{(k+1)}-\lambda)^{-1}d[\mathcal{E}_\t\mathcal{P}_{\overline{V_\star}}^0 z]
%%\end{flalign*}
%%So
%%\[
%%\beta'_{0}(\lambda;\mathcal{E}_\t\mathcal{P}_{\overline{V_\star}}^0z,\mathcal{E}_\t\mathcal{P}_{\overline{V_\star}}^0z) \le \lambda d[\mathcal{E}_\t\mathcal{P}_{\overline{V_\star}}^0z] +\lambda^2(\mu_0^{k+1}-\lambda)^{-1}d[\mathcal{E}_\t\mathcal{P}_{\overline{V_\star}}^0 z] 
%%\le (\mu_0^{(k+1)})^2 d[\mathcal{E}_\t\mathcal{P}_{\overline{V_\star}}^0z]
%%\]
%%\[
%%\beta'_{\t}(\lambda;z,z)= \lambda d[z] +\lambda^{2}\, d\big((\mathbf{B}_0-\lambda)^{-1} \mathcal{E}_\t\mathcal{P}_{\overline{V_\star}}^0 z,\mathcal{E}_\t\mathcal{P}_{\overline{V_\star}}^0 z\big)
%%\]
%}
%\begin{remark}\label{r.betasign}
\subsection{An approximation by a bivariate operator}\label{s.bivariate}


%In examples we have in mind, we shall consider differential  operators $\mathcal{L}_\ep$ in $L^2(\RR^n)$ that have  with rapidly oscillating coefficients with periodic reference cell $\ep \square$, $\square : = [0,1)$. In particular, we consider $\mathcal{L}_\ep$ such that after an application of the  rescaling mapping $\Gamma_\ep : L^2(\RR^n) \longrightarrow L^2(\RR^n)$,  $(\Gamma_\ep f)(y) :=  f(\ep y)$ and then the unitary Gelfand transform\footnote{The unitary Gelfand transform $U :L^2(\RR^n) \rightarrow L^2(\square^* \times \square)$ and its inverse are given, for example, as the continuous extension of the mappings
%	\[
%	Uf(\theta,y) : = (2\pi)^{-n/2}\sum_{h \in \ZZ^n} f(y+h) e^{-\i \theta \cdot (y+h)}, \qquad f \in C^\infty_0(\RR^n),
%	\]
%	%with inverse 
%	\[
%	U^{-1} g (x) =(2\pi)^{-n/2} \int_{\square^*} g(\t ,x   ) e^{\i \t \cdot x}\, {\rm d}\t , \quad g \in C ( \square^* ; C_{per} (\square)),
%	\]
%	see for example \cite{Gel} or \cite{Ku}.
%	%\cite{Gel} and \cite[Section 3.2]{BeLiPa}.
%} $U: L^2(\RR^n) \longrightarrow L^2(\Box^* \times \Box)$, the operator
%\[
%\mathcal{L}_{\ep,\t}^{-1} = U \Gamma_\ep \mathcal{L}_\ep^{-1} \Gamma_\ep^{-1} U^{-1},
%\]
%is generated by a problem of the form \eqref{p1} in $\mathcal{H} = L^2(\square)$ with $d_\t$ being the standard $L^2(\square)$ inner product and $\Theta = \square^* : = [-\pi,\pi)^n$. We now consider those operators that satisfy the assumptions of Theorem \ref{p.unitaryequiv}.
%and additionally for simplicity of exposition that $Z$ is $1$-dimensional, i.e. $Z = {\rm Sp}(\mathbf{e})$ where $\mathbf{e}$ is the constant function equal to $1$. Then the form $a^h$
%Up to no
%
%In this section we shall interpret the presence of $\xi$ as the introduction of a new variable in the approximation of $\mathcal{L}_{\ep,\t}$. This interpretation comes from considering $\mathbb{L}_\xi$ as the `bivariate' symbol of an operator defined on the Bochner space ..


Here we will provide alternative representations to the approximations in Theorem \ref{splimSe.3} for 
the ``resolvent'' $\mathcal{L}_{\ep,\t}^{-1}$  and in Corollary \ref{c.collspec} for the 
collective spectrum of the original operators $\mathcal{L}_{\ep,\t}$ 
in terms of those 
%the fibres and spectrum, respectively, 
of an % a `bivariate' 
operator defined on the Bochner space $L^2(\RR^n ; \mathcal{H}_0)$, 
i.e. on a separable Hilbert space-valued functional space with 
%with Hilbert space 
$\mathcal{H}_0 = \left(\overline{V_\star \dot{+}Z},\,d_0\right)$. %, cf Section \ref{s.spbt}. 
This operator, as examples in Section \ref{sec:examples} will illustrate, can be viewed as an abstract version 
of a two-scale limit operator. 
%These representations will be playing a key role in certain examples in Section \ref{sec:examples} where the operators $\mathcal{L}_{\ep,\t}$ are themselves 
%results of certain transforms (like scaling and Floquet-Bloch-Gelfand) of operators %$\mathcal{L}_{\ep}$ in initial physical models. 
Some basic facts from the theory of Bochner spaces, see e.g. \cite{Hytonen}, which are relevant to our exposition %here 
are collected in Appendix B. %\ref{appb}. 
With the right-hand-side given by \eqref{ik3} for any $g\in\mathcal{H}$, according to Theorem  \ref{splimSe.3} 
the solution $u_{\ep,\t}=\mathcal{L}_{\ep,\t}^{-1}g$ to the original problem \eqref{p1} is approximated by 
${u}^a_{\ep,\t}= 
\mathcal{E}_\t\, \mathbb{L}_{\t / \ep}^{-1}\mathcal{P}_{\mathcal{H}_0}^0\mathcal{E}_\t^{-1}g$. 
Denoting $h:=\mathcal{P}_{\mathcal{H}_0}^0\mathcal{E}_\t^{-1}g\in \mathcal{H}_0$, 
one observes that $v+z:=\mathbb{L}_{\t / \ep}^{-1}h\in \text{dom }\mathbb{L}_{\t / \ep}\subset V_\star\dot{+}Z=V_0$ depends on $\t$ and $\ep$ only via their ratio $\xi:=\theta/\ep\in\mathbb{R}^n$ where according to 
\eqref{defhom.form}  the dependence of $\mathbb{L}_\xi$ on $\xi$ is quadratic. 
The idea is to try and represent it via an appropriate (inverse) Fourier transformed problem with a transformed variable $x\in\mathbb{R}^n$ of $\xi$.  
To that end, first recall that according to \eqref{Sform} the above $v+z\in V_0$ is the solution to 
\begin{equation}
\label{Linvprobl}
a^{\rm h}_{\xi}(z,\tilde z)\,\,+\,\,b_0(v+z, \tilde v+\tilde z)\,\,=\,\,d_0(\,h\,,\, \tilde v+\tilde z), \quad 
\forall\,\, \tilde v+\tilde z\in V_\star\dot{+}Z; \quad \xi:=\t/\ep, \quad \t\in\Theta. 
\end{equation} 
Regard now $h=h(\t)$, $\t\in\Theta$, as belonging to the Bochner space $L^2(\Theta; \mathcal{H}_0)$ with the standard 
$\RR^n$-Lesbegue measure induced on $\Theta$. By extending $h(\t)$ by zero outside $\Theta$ for the whole of $\RR^n$,
we can assume $h\in L^2(\RR^n; \mathcal{H}_0)$. 
The quadratic dependence of 
%in the form $\mathbb{S}_\xi$ representing the left-hand side of \eqref{IKz3prob88}, see \eqref{Sform}, 
$a^h_\xi$ on $\xi$ in \eqref{defhom.form} can be represented as follows: 
for any $z, \tilde z\in Z$ and $\xi\in\mathbb{R}^n$, 
\begin{equation}
\label{ahgrad}
a^{\rm h}_\xi(z, \tilde z)\,\,=\,\,\sum_{j,k=1}^n a^{\rm h}_{jk}(z, \tilde z)\,\xi_j\,\xi_k\,\,=\,\,
\sum_{j,k=1}^n a^{\rm h}_{jk}\left(\xi_j z, \xi_k\tilde z\right). 
\end{equation} 
Here 
\begin{equation}
\label{ahij}
a^{\rm h}_{jk} \big( {z}, {\tilde{z}} \big)\,\,:=\,\, a_{0\, jk}''( z,\tilde z )\,-\, 
a_{0} ( N^j {z}, N^k {\tilde{z}}), 
% \quad \mathbf{u}(x) = ( u_1(x),u_2(x),\ldots, u_n(x)),\ u_i \in L^2(\RR^n ; Z).
\end{equation} 
where %$N \cdot \mathbf{{z}}:=\sum_{j=1}^n N^j z_j$ with 
$N^j:=N_{e^j}=e^j\cdot N$ and  
$e^1,...,e^n$ is the canonical basis in $\mathbb{R}^n$, i.e. $N^j=N_\t$ with $\t=e^j$, cf \eqref{cell:prob2}. 

Let us now, given $0<\ep<1$, make in \eqref{Linvprobl} a change of variable $\t \rightarrow \xi=\t/\ep\in\RR^n$, and 
 with the aim of formally integrating \eqref{Linvprobl} over $\RR^n$ in $\xi$ and recalling \eqref{ik2} 
assume $v,\tilde v\in L^2\left(\RR^n; V_\star\right)$. (Remind that we regard $V_\star$ and $Z$ as Hilbert spaces with norm $\|\cdot\|^2_0=b_0[\cdot]$.) 
Regarding $h\in L^2(\RR^n; \mathcal{H}_0)$ as arbitrary, and bearing in mind the 
boundedness of $a^{\rm h}_\xi$ in $z, \tilde z\in Z$ as well as its 
quadratic growth %of $a^{\rm h}_\xi$ 
in $\xi$, we need to take 
$z(\xi),\tilde z(\xi)\in L^2\left(\RR^n; Z\right)$ so that also 
$\xi_jz(\xi),\xi_k\tilde z(\xi)\in L^2\left(\RR^n; Z\right)$, $\forall j,k=1,...,n$. 
In other words, $z$ and $\tilde z$ can be said to belong to weighted Bochner space 
$L^2\left(\RR^n, {\langle\xi\rangle^2}{\rm d}\xi; Z\right)$ with the weight $\langle\xi\rangle^2:=1+|\xi|^2$. 
The resulting problem is to find $v\in L^2\left(\RR^n; V_\star\right)$ and 
$z\in L^2\big(\RR^n, {\langle\xi\rangle^2}{\rm d}\xi; Z\big)$, such that 
\begin{equation}
\label{Lproblint}
\int_{\RR^n}a^{\rm h}_{\xi}\left(z,\tilde z\right){\rm d}\xi\,\,+\,\int_{\RR^n}\,\,b_0(v+z, \tilde v+\tilde z){\rm d}\xi\,= \int_{\RR^n}d_0\big(\,h(\ep\xi), \tilde v+\tilde z\big){\rm d}\xi, \ \ \ 
\forall \tilde v+\tilde z\in L^2\big(\RR^n; V_\star\big)\dot{+}L^2\big(\RR^n, {\langle\xi\rangle^2}{\rm d}\xi; Z\big). 
\end{equation} 
(Notice that the above is obviously a direct sum: if 
$v\in L^2\left(\RR^n; V_\star\right)$ and 
$z\in L^2\big(\RR^n, {\langle\xi\rangle^2}{\rm d}\xi; Z\big)$ and $v+z=0$ in $\mathbb{H}_0:=L^2(\RR^n; \mathcal{H}_0)$, 
then $v(\xi)+z(\xi)=0$ for a.e. $\xi\in\RR^n$. Then, since $V_\star\dot{+}Z$ is a direct sum, 
$v(\xi)=z(\xi)=0$ for a.e. $\xi$, i.e. $v=z=0$ in $\mathbb{H}_0$.) 

We next argue that, for arbitrary $h\in \mathbb{H}_0$, problem \eqref{Lproblint} is  well-posed on its own right, and the form $\mathbb{A}$ on its left-hand side generates 
a self-adjoint operator $\mathbb{L}$ in Bochner (Hilbert) 
space $\mathbb{H}_0$. 
Indeed, the form is non-negative and has domain 
$\mathbb{D}=L^2\left(\RR^n; V_\star\right)\dot{+}L^2\left(\RR^n, {\langle\xi\rangle^2}{\rm d}\xi;\, Z\right)$ which, see  Proposition \ref{propb1} of Appendix B, is dense in 
$\mathbb{H}_0$ 
and is closed with respect to the form (Proposition \ref{propb2}). 
Since, as implied by \eqref{ik2}, ${\rm Sp}\, \mathbb{L}\subset [1,+\infty)$, for any  
$h\in \mathbb{H}_0$ problem \eqref{Lproblint} is well-posed and has a unique 
solution $v+z\in\mathbb{D}$. 
Moreover, given $0<\ep<1$, as shown in Proposition \ref{propb3}, \eqref{Lproblint} holds 
if and only if 
\eqref{Linvprobl} holds for a.e. $\t\in\Theta$. 
Therefore (see Definition \ref{defb4} and Proposition \ref{propb5}), the latter immediately implies that the newly defined operator $\mathbb{L}$ is a direct integral of $\mathbb{L}_\xi$  %$\xi\in\RR^n$, 
which serve as 
its fibers: $\mathbb{L}=\int_{\RR^n}^\oplus \mathbb{L}_\xi \, {\rm d}\xi$.  
Similarly, for the resolvents, $\mathbb{L}^{-1}=\int_{\RR^n}^\oplus \mathbb{L}^{-1}_\xi \, {\rm d}\xi$. 

We now aim at equivalently reformulating problem \eqref{Lproblint} in a Fourier transformed 
setting,  %(see Definition \ref{defb6}), 
i.e. for $\left(\check v+u\right):=\mathcal{F}^{-1}(v+z)$ where $v+z$ is the 
solution to \eqref{Lproblint} %for a fixed $0<\ep<1$ and 
with $h(\ep\xi)$ replaced for a moment by arbitrary $f\in \,\mathbb{H}_0=L^2(\RR^n; \mathcal{H}_0)$, i.e. 
$v+z=\mathbb{L}^{-1}f$, 
 and $\mathcal{F}^{-1}$ being  
the inverse Fourier transform in the sense of Definition \ref{defb6} for 
$\mathbb{H}=L^2\left(\RR^n; \left(\mathcal{H}, d_0\right)\,\right)$. 
% with $\mathcal{H}=\left(\mathcal{H}, d_0\right)$. %_0=\left(\overline{V_\star\dot{+}Z},d_0\right)$. 
As immediately follows from  \eqref{ftdef}, $\mathcal{F}$ restricted to the closed subspace 
$\mathbb{H}_0$ of $\mathbb{H}$ coincides with the 
Fourier transform as given by Definition \ref{defb6} directly for Bochner space $\mathbb{H}_0$. 
Then, 
$\check v+u=\mathcal{F}^{-1}\mathbb{L}^{-1}f= 
\mathcal{F}^{-1}\mathbb{L}^{-1}\mathcal{F} F= \mathcal{L}^{-1}F$, where 
$F:=\mathcal{F}^{-1}f\in \mathbb{H}_0$ and $\mathcal{L}:=\mathcal{F}^{-1}\mathbb{L}\,\mathcal{F}$. 
We will show that $\mathcal{L}$ is a self-adjoint operator in $\mathbb{H}_0$ generated by 
a form which is a formal (inverse) Fourier transform of the one in \eqref{Lproblint}. 

To that end, let $u,\tilde u\in H^1\big(\RR^n ; \left(Z, b_0\right)\big)$, see Definition \ref{defb7}, 
and introduce $a^{\rm h}\big( \nabla  u(x), \nabla\tilde{u}(x)\big)$ 
by formally replacing $\xi_j$ on the right-hand side of \eqref{ahgrad}  by $-i\,\partial_{x_j}$, i.e. by their Fourier-transformed counterparts: 
\[
a^{\rm h}\big( \nabla  u(x), \nabla\tilde{u}(x)\big)\,\,:=\,\,a^h_{-i\nabla}(u,\tilde u)\,\,:=\,\,
\sum_{j,k=1}^n a^{\rm h}_{jk}\left(\partial_{x_j} u, \partial_{x_k}\tilde u\right). 
\]
This motivates considering the %dense 
subspace 
$\check{\mathbb{D}}=H^1(\RR^n ; Z) \dot{+} L^2(\RR^n ;V_\star)$ of 
$\mathbb{H}_0$, 
%$L^2(\RR^n ; \mathcal{H}_0)$, 
on which we define the bivariate form
\begin{equation}\label{Q}
Q\big(u+v,\tilde{u} + \tilde{v}\big) \,\,: =\,\,\int_{\RR^n} a^{\rm h}\big( \nabla  u(x), \nabla\tilde{u}(x)\big)  \, {\rm d}x 
\,\,+\,\,  \int_{\RR^n} b_0\Big(u(x) + v(x),\, \tilde{u}(x) + \tilde{v}(x) \Big) \, {\rm d} x, 
\end{equation}
where $u,\tilde{u} \in H^1\big(\RR^n;\left(Z,b_0\right)\big)$ and 
$ v, \tilde{v} \in L^2\big(\RR^n;\left(V_\star,b_0\right)\big).$ 
%Here, cf \eqref{defhom.form}, 
%with form domain $H^1(\RR^n ; Z) \dot{+} L^2(\RR^n ;V_\star)$. 
% $a^{\rm h} : (Z)^n \times  (Z)^n \rightarrow \CC$ is the sesquilinear form
%\[
%a^{\rm h} \big( \mathbf{z}, \mathbf{\tilde{z}} \big): = \sum_{i,j=1}^n a_{0 ij}''( z_i,z_j )- a_{0} ( N \cdot \mathbf{z}, N \cdot \mathbf{\tilde{z}}), 
% \quad \mathbf{u}(x) = ( u_1(x),u_2(x),\ldots, u_n(x)),\ u_i \in L^2(\RR^n ; Z).
%\]
%where $N \cdot \mathbf{{z}}:=\sum_{j=1}^n N^j z_j$ with $N^j:=N_{e^j}=e^j\cdot N$ and  
%$e^1,...,e^n$ the canonical basis in $\mathbb{R}^n$, i.e. $N^j=N_\t$ with $\t=e^j$, cf \eqref{cell:prob2}. 

%The following important 
Lemma \ref{lemft} establishes that the form $Q$ specifies a self-adjoint ``bivariate'' operator $\mathcal{L}$ 
in Hilbert space $\mathbb{H}_0$,  which is a Fourier-transformed counterpart of $\mathbb{L}$, namely 
$\mathcal{L}=\mathcal{F}^{-1}\mathbb{L}\,\mathcal{F}$. %where $\mathcal{F}$ is the Fourier transform in $\mathbb{H}_0$. 
\begin{remark}
We shall see in the examples that $\mathcal{L}$ often coincides with the  %homogenised 
two-scale limit  operator, e.g. in the homogenisation theory for elliptic differential operators with rapidly oscillating high-contrast coefficients, 
see Section \ref{e.dp}. 
\end{remark}


%By the assumptions of Section \ref{sec.atreg} (cf. \eqref{ahcoercive}) it follows that $a^h$ is strongly elliptic in the following sense: 
%\[
%a^{\rm h}[\xi z] \,\ge\, \nu_\star|\xi|^2 b_0[ z], \quad \forall \xi \in \RR^n, \, \, \forall z \in Z.
%\]
%This ellipticity, along with the condition \eqref{VZorth}, ensures that $Q$ is an inner product on 
%Hilbert space $H^1(\RR^n ; Z) \dot{+} L^2(\RR^n ;V_\star)$. Indeed, this follows  from the known properties of the  Fourier transform $\mathcal{F}$ (see Appendix B) 
%which is a unitary transform on the Bochner space $L^2(\RR^n ; (\mathcal{H},d_0))$. Therefore, $Q$ defines a self-adjoint 
%``bivariate" operator %$\mathcal{L}$ 
%in $L^2(\RR^n ; \mathcal{H}_0)$.

Aiming at restating Theorem \ref{p.unitaryequiv} in terms of operator $\mathcal{L}$, 
we first observe that from Lemma \ref{lemft} and Proposition \ref{propb5} (see Definition \ref{defb4})
\begin{equation}
\label{lft}
\mathbb{L}_\xi^{-1} f(\xi) \,\,=\,\,  \big(\mathcal{F}\mathcal{L}^{-1}\mathcal{F}^{-1} f \big)(\xi) \quad 
for \,\,a.e.\ \xi, \quad f \in \mathbb{H}_0=L^2(\RR^n ; \mathcal{H}_0).
\end{equation}
Relation \eqref{lft} signifies the important fact that, while the bi-variate resolvent $\mathcal{L}^{-1}$ is 
generally not decomposable into a direct integral, its Fourier transformed counterpart 
$\mathcal{F}\mathcal{L}^{-1}\mathcal{F}^{-1}$ is. 
%\end{remark}
Further, we observe that the orthogonal projectors $\mathcal{P}_{\mathcal{H}_0}^0:(\mathcal{H},d_0) \rightarrow \mathcal{H}_0$ and  $\mathcal{P}:L^2(\RR^n ; (\mathcal{H},d_0)) \rightarrow L^2(\RR^n ; \mathcal{H}_0)$ are related by the identity 
(Proposition \ref{propb9}) 
\begin{equation}
\label{projft}
\mathcal{P}_{\mathcal{H}_0}^0f(\xi) = \big(\mathcal{F}  \mathcal{P}\mathcal{F}^{-1} f \big)(\xi) \quad 
for \,\,a.e.\  \xi, \quad f \in \mathbb{H}=L^2(\RR^n; (\mathcal{H},d_0)). %, 
\end{equation}
%where $\mathcal{F}$ is Fourier transform in $\mathbb{H}:=L^2\big(\RR^n ; (\mathcal{H},d_0)\big)$. 
%(Notice that on $\mathbb{H}_0=L^2\big(\mathbb{R}^n;\mathcal{H}_0\big)$ it coincides with the above 
%introduced definition of $\mathcal{F}$, as immediately follows from \eqref{ftdef}.) 
Next, introduce in $\mathbb{H}$ a normalised rescaling operator 
$\Gamma_{\ep} : \mathbb{H} \rightarrow \mathbb{H} $ and its inverse $\Gamma_{\ep}^{-1}$ by 
\begin{equation}
\label{gammaep}
\big(\Gamma_{\ep}f\big)(x)\,\,:=\,\,\ep^{n/2}f(\ep x), \ \ \ \ \ \ \ 
\big(\Gamma^{-1}_{\ep}f\big)(x)\,\,=\,\,\ep^{-n/2}f\left(\ep^{-1} x\right). 
\end{equation} 
Notice that $\Gamma_{\ep}$ and $\Gamma^{-1}_{\ep}$ are 
 unitary operators in $\mathbb{H}$. Then, via \eqref{lft} and \eqref{projft},  
\begin{equation}
\label{lpft}
 \mathbb{L}_{\t/\ep}^{-1} \mathcal{P}_{\mathcal{H}_0}^0 f (\t)\, =\, 
\Big( \Gamma_{\ep}^{-1}  \mathbb{L}^{-1} \mathcal{P}_{\mathcal{H}_0}^0  \Gamma_{\ep} f \Big)(\t)
\, =\, 
\Big( \Gamma_{\ep}^{-1} \mathcal{F} \mathcal{L}^{-1} \mathcal{P} \mathcal{F}^{-1} \Gamma_{\ep} f \Big)(\t), \ \ \ for\,\,  a.e. \  \t, \quad f \in 
L^2\big(\RR^n;(\mathcal{H},d_0)\big).
\end{equation} 
Finally, for reformulating Theorem \ref{p.unitaryequiv}, we recall that it approximates the exact solution 
$u_{\ep,\t}=\mathcal{L}_{\ep,\t}^{-1}g$, where according to \eqref{ik3} $g\in\mathcal{H}$ for any $\t\in\Theta$. 
We now assume $g\in L^2\big(\Theta;\left(\mathcal{H},d_0\right)\big)$, and 
comparing with the approximation 
$u^a_{\ep,\t}=\mathcal{E}_\t \mathbb{L}_{\t / \ep}^{-1}\mathcal{P}_{\mathcal{H}_0}^0\mathcal{E}_\t^{-1}g$ of 
Theorem \ref{p.unitaryequiv}  suggests taking in \eqref{lpft} $f=\chi\mathcal{E}^{-1}g$, where 
$\mathcal{E}$ 
 is given by $f(\t) \mapsto \mathcal{E}_\t f(\t),$ for a.e. $ \t \in \Theta$, and 
$\chi: L^2(\Theta ; (\mathcal{H},d_0)) \rightarrow  L^2(\RR^n ; (\mathcal{H},d_0))  $ 
is the %embedding 
operator of extension by zero outside $\Theta$. 
We notice that bounded operator $\mathcal{E}$ acts from $\mathbb{H}_\Theta:=L^2\big(\Theta ; (\mathcal{H},d_0)\big)$ into itself 
as, due to \eqref{H6}, $\mathcal{E}_\t$ is continuous in $\t$ and so is $\t$-(weakly) measurable. 
Further, in combination with \eqref{vs61}, \eqref{H6} assures that $d_\t$ is also continuous in $\t$ and hence $(u,\tilde u)=\int_\Theta d_\t\big(u(\t), \tilde u(\t)\big) {\rm d}\t$ forms an equivalent inner product in $\mathbb{H}_\Theta$. When endowed with such an inner product, we conveniently denote this 
space by $L^2\big(\Theta ; (\mathcal{H},d_\t)\big)$, and notice that because of \eqref{H6} operator $\mathcal{E}$ is 
unitary when viewed as acting from $L^2\big(\Theta;\left(\mathcal{H},d_0\right)\big)$ to 
$L^2\big(\Theta;\left(\mathcal{H},d_\t\right)\big)$. 
Assembling all this together and also noticing that 
$\Gamma_{\ep}^{-1} \mathcal{F}= \mathcal{F}\Gamma_{\ep}$ and $\mathcal{F}^{-1}\Gamma_{\ep}=\Gamma_{\ep}^{-1} \mathcal{F}^{-1}$, 
for the above approximation,  
$
u^a_{\ep,\t}\,\,=\,\,\mathcal{E}\, \chi^*\, \mathcal{F}\Gamma_{\ep}\,
\mathcal{L}^{-1}\mathcal{P}\, 
\Gamma_{\ep}^{-1} \mathcal{F}^{-1}\, \chi\,\mathcal{E}^{-1}g, 
$ 
where $\chi^*: L^2(\RR^n ; (\mathcal{H},d_0)) \rightarrow L^2(\Theta ; (\mathcal{H},d_0)) $  
is the operator of restriction from $\RR^n$ to $\Theta$ and is the adjoint of $\chi$. 
%is (weakly) measurable in $\t$, and acts from $L^2(\Theta ; (\mathcal{H},d_0))$ into itself.) 
%$\mathcal{E}: L^2(\Theta ; (\mathcal{H},d_0)) \rightarrow L^2(\Theta ; (\mathcal{H},d_0))$.) 
%Upon considering $L^2(\Theta;(\mathcal{H},d_0)) \subset L^2(\RR^n;(\mathcal{H},d_0))$ by trivially extending its elements by zero into $\RR^n \backslash \Theta$, 
As a result, Theorem \ref{p.unitaryequiv} implies the following.
\begin{theorem}\label{thm.bivariate} 
Assume \eqref{KA}--\eqref{H6}. Then, for $0<\ep <1$, one has
	\[
 d_\t^{1/2}\Big[\,\mathcal{L}_{\ep,\t}^{-1}\, g(\t) \,\,-\,\, \big(A_\ep^* \mathcal{L}^{-1}\mathcal{P} A_\ep g\big)(\t)\Big] 
\,\, \le\,\,  C_{11}\,\ep\,  d_\t^{1/2}\big[g(\t)\big] ,	\quad \forall g \in L^2\big(\Theta; (\mathcal{H},d_\t)\big), \quad 
for \,\,a.e.\ \t \in \Theta,
	\]
where  operator $A_\ep : L^2\big(\Theta ; (\mathcal{H},d_\t)\big) \rightarrow L^2\big(\RR^n ; (\mathcal{H},d_0)\big) $ is the composition 
$A_\ep : = \Gamma_\ep^{-1}\mathcal{F}^{-1} \, \chi\,\mathcal{E}^{-1} $, 
and 
$A_\ep^* : L^2\big(\RR^n ; (\mathcal{H},d_0)\big) \rightarrow L^2\big(\Theta ; (\mathcal{H},d_\t)\big) $ is its adjoint given by  
$A_\ep^* : = \mathcal{E}\, \chi^*\,  \mathcal{F}\Gamma_\ep$. 
%Here, $\chi: L^2(\RR^n ; (\mathcal{H},d_0)) \rightarrow L^2(\Theta ; (\mathcal{H},d_0)) $ is the embedding operator and 
%$\mathcal{E} : L^2(\Theta ; (\mathcal{H},d_0)) \rightarrow L^2(\Theta ; (\mathcal{H},d_0))$ is given by $f(\t) \mapsto \mathcal{E}_\t f(\t),$ a.e. $ \t \in \Theta$. 
Furthermore, %$A_\ep$ and $B_\ep$ are adjoint to each other and 
the following identities hold:
\begin{equation}
\label{abident}
A_\ep^* A_\ep  = I \quad \text{and } \quad A_\ep A_\ep^* = \Gamma_\ep^{-1}\mathcal{F}^{-1}  \chi_\Theta \mathcal{F}\Gamma_\ep = 
\mathcal{F}^{-1}\Gamma_\ep  \chi_\Theta \Gamma_\ep^{-1}\mathcal{F},
\end{equation} 
where $\chi_\Theta$ is operator of multiplication by the characteristic function of $\Theta$ in 
$L^2\big(\RR^n ; (\mathcal{H},d_0)\big)$. 
%\[
%(\mathcal{T}_\ep f) (x) = (2\pi)^{-n/2} \int_{} (\mathcal{E}_{ \t}^* f(\t)) e^{\i \t \cdot x} \, {\rm d}\t.
%\]
\end{theorem}
(Identities \eqref{abident} immediately follow via obvious relations $\chi^*\chi=I$ and 
$\chi\chi^*=\chi_\Theta$.) 
Notice that, by \eqref{abident}, $\,A_\ep A_\ep^*$ is the operator of projection onto 
$\ep^{-1}\chi_\Theta$ in the Fourier space. 
Notice also that all the above implies that the approximating operator $A_\ep^* \mathcal{L}^{-1}\mathcal{P} A_\ep g$ is self-adjoint in 
$L^2\big(\Theta ; (\mathcal{H},d_\t)\big)$, and is the form of a direct integral along its fibers as given for a.e. $\t\in\Theta$. 
\vspace{.15in}

Turning now to approximation of the collective spectrum of $\mathcal{L}_{\ep,\t}$ in terms of that of the 
bivariate operator $\mathcal{L}$, we first observe that because of the unitary equivalence of $\mathcal{L}$ and 
$\mathbb{L}$ (Lemma \ref{lemft}) the spectra of the latter two coincide. 
On the other hand, recalling that $\mathbb{L}$ is the direct integral of  $\mathbb{L}_\xi$, $\xi\in\RR^n$, 
notice that the eigenvalues $\lambda_k(\xi)$ of $\mathbb{L}_\xi$ continuously depend on $\xi$. 
(This directly follows e.g. from \eqref{Sform}, continuous dependence of $a^h_\xi$ on $\xi$ and the min-max arguments.)  
Then, by e.g. Theorem XIII.85 (d) of \cite{ReeSim}, the spectrum of $\mathbb{L}$ is the closure 
of the union of the spectra of $\mathbb{L}_\xi$. As a result, 
%Additionally, since the eigenvalues of $\mathbb{L}_\xi$ are continuous in $\xi$ one has the identity
$
{\rm Sp}\, \mathcal{L} = \overline{\bigcup_{\xi \in \RR^n} {\rm Sp} \, \mathbb{L}_\xi}
$
and Theorem \ref{thm.limspecrep} and Corollary \ref{c.collspec} give together the following result.
\begin{theorem}\label{bivariate.spec}
	One has 
	\[
	{\rm Sp}\, \mathcal{L} \,\,=\,\, \Big\{ \lambda \notin {\rm Sp}\, \mathbf{B} :\,\, \beta_{\lambda }[z] \ge 0 \text{ for some $0\neq z \in Z$} \Big\} \cup {\rm Sp}\, \mathbf{B}.
	\]
Furthermore, for every interval $[a,b] \subset (-\infty,\infty)$ one has 
\[
d_{[a,b]} \Big(\, \overline{\bigcup_{\theta \in \Theta} {\rm Sp}\, \mathcal{L}_{\ep,\t}}  \,,\,\,{\rm Sp}\, \mathcal{L}\Big) \,\,\le\,\, C_b\,\ep, \ \ \ \forall\, 0<\ep<1, 
\]
with $C_b$ as given in Corollary \ref{c.collspec}. In particular, if $(a,b)$ is a gap in the spectrum of $\mathcal{L}$, i.e. $(a,b) \cap {\rm Sp}\, \mathcal{L} = \emptyset$ then $[a +C_b \ep,b-C_b\ep]$ is in a gap of the collective spectrum $\overline{\bigcup_{\theta \in \Theta} {\rm Sp}\, \mathcal{L}_{\ep,\t}}$ 
when $\ep < (b-a)/(2C_b)$.
\end{theorem}
\begin{remark}
Under additional assumptions on the regularity of $b_\t$ and $d_\t$ in $\t$, one can identify $\mathcal{L}_{\ep,\t}$ and  $\overline{\bigcup_{\theta \in \Theta} {\rm Sp}\, \mathcal{L}_{\ep,\t} }$ respectively as the fibres and  spectrum of a decomposable operator 
$\mathcal{L}_\ep=\int_\Theta^\oplus \mathcal{L}_{\ep,\t} \, {\rm d}\theta$ 
acting in the space $L^2(\Theta; (\mathcal{H},d_0))$, see the examples section below.
\end{remark}
We end this section by discussing the possibility of gaps in the  spectrum ${\rm Sp}\, \mathcal{L}$. We know 
from Theorem \ref{bivariate.spec} that an interval $I$ is in a gap of ${\rm Sp}\, \mathcal{L}$ if, and only if, ${\rm Sp}\, \mathbf{B} \cap I = \emptyset$ and for every $\lambda \in I$ the form $\beta_\lambda$ is negative-definite on $Z$. When $Z$ is one-dimensional it is straightforward to verify the existence of such intervals.  Indeed, for such $Z$ and $\lambda\notin{\rm Sp}\, \mathbf{B}$, one has via \eqref{betadef} and \eqref{vfromz} %{03.09.20e1} 
 the representation 
\begin{equation} \label{iksign}
\beta_\lambda(z,\tilde{z})\,=\,-\,\, b_0(z,\tilde{z}) \,+\,\lambda\, d_0(z,\tilde{z}) \,+\,(\lambda-1)^{2}\sum_{k\in \NN}\,\frac{d_0(z,\varphi^{(k)})d_0(\varphi^{(k)},\tilde{z})}{\mu^{(k)}-\lambda}\,, \quad \forall \,z,\tilde{z} \in {Z},
\end{equation}
where $\{ \varphi^{(k)}\}_{k\in \NN}$ are the eigenvectors of $\mathbf{B}$, corresponding to the eigenvalues $\{ \mu^{(k)}\}_{k\in \NN}$, that form an  orthonormal basis in $\left(\overline{V_\star},\, d_0\right)$, assumed here for definiteness 
infinite-dimensional. Now we can see that  if 
$\mu^{(n)}$ is a single eigenvalue and 
$\varphi^{(n)}$ is not orthogonal to $Z$ then, for $0\ne z\in Z$, $\beta_\lambda[z]$ is positive just to the left, and negative just to the right, of $\mu^{(n)}$. Consequently, there is some interval to the left (respectively the right) of $\mu^{(n)}$  in ${\rm Sp}\, \mathcal{L}$ (respectively the  gap).  In general, there maybe infinitely many such eigenvalues and thus there are infinitely many spectral gaps.
%, and therefore, lead to arbitrarily many gaps in the set $\mathbf{I}_\ep$  (and therefore in $\mathbf{J}_\ep$ )  appearing as $\ep \rightarrow 0$. 
This situation occurs, for example, in the double-porosity type problem studied by V. Zhikov in \cite{Zhi2000,Zhi2005}, wherein $\mathcal{L}^{-1}$ coincides with the homogenised two-scale limit operator resolvent  $(\mathcal{L}_0 + I)^{-1}$ and $\beta_{\lambda-1}$ coincides with the Zhikov $\beta$-function, see Example \ref{e.dp}. 

The situation is more complicated when $Z$ is not one-dimensional. Whilst \eqref{iksign} still holds and one identifies intervals in ${\rm Sp}\, \mathcal{L}$ just to the left of eigenvalues $\mu^{(n)}$ with eigenvector $\varphi^{(n)}$ not orthogonal to $Z$, as there will always exist $0\ne z\in Z$ such that $d_0\left(z,\varphi^{(n)}\right)= 0$ it is not necessarily the case that the right of this point is in the gap. 
There may even be no gaps. For example, if ${\rm dim}\, Z >1$, ${\rm dim}\, V_\star =1$  and say $b_0$ coincides with $d_0$ on $Z$, then one can always find a $0\neq z \in Z$ such that  $\beta_\lambda[z]\ge 0$ for all $\lambda \ge 1$; thus ${\rm Sp}\, \mathcal{L} = [1,\infty)$. 
\section{Examples}
\label{sec:examples}
Here we aim at demonstrating the power and versatility of the abstract results obtained above by applying them to a diverse set of problems. 
We provide examples of homogenisation-type models that can be reformulated as problems of the type \eqref{p1} and satisfy (some of)  the main assumptions \eqref{KA}--\eqref{H6}. We shall begin our demonstration with the classical and 
by now ``almost'' classical (`high-contrast') homogenisation problems, where our approach already leads to some new results. 
Then, we shall study various other physically-relevant models of interest, each chosen to showcase the relevance and utility of the  main abstract assumptions and results.
\subsection{Uniformly elliptic PDEs with rapidly oscillating periodic coefficients}
\label{e.class}
We begin our examples with the classical homogenisation problem for elliptic PDE systems with rapidly oscillating periodic coefficients. 
For the convenience of the reader, we consider scalar PDEs  with matrix-valued coefficients and comment here that the subsequent exposition readily extends in a directly analogous manner for systems with tensor-valued coefficients. 
For the present example, it would suffice restricting the application of the above developed general theory up to and 
including 
Section \ref{sec.atreg}, based on assumption \eqref{KA}--\eqref{H4}. 
(The %spectral 
results of Section \ref{s:resolv} are also formally applicable although, in %sharp 
contrast with some of 
the subsequent ``non-classical'' examples where they play a key role, 
are of limited further value in the classical homogenisation.) 

Consider the following ``resolvent'' problem in the whole of $\mathbb{R}^n$: 
%\footnote{For the resolvent parameter $\lambda \neq 1$ see Remark \ref{rem.resolv} below.}
\begin{equation}
\label{differentialPDE}
\left\{ \begin{aligned}
& \text{Find $u_\ep \in H^1(\RR^n)$ such that} \\
&-\,{\rm div}\Big( A \left(\tfrac{x}{\ep} \right)\nabla u_\ep (x) \Big)\,\,+\,\,  u_\ep (x) \,\,=\,\, F(x), \qquad  \text{for \,\,a.e.}\  x \in \RR^n,
\end{aligned} \right.
\end{equation}
for a given $0<\ep<1$, $\,F \in L^2(\RR^n)$, and measurable possibly complex-valued $n\times n$ matrix $A$ that satisfies the following standard conditions of Hermitian symmetry, uniform ellipticity and boundedness: 
\be
\begin{aligned} \label{IKcond}
 \quad A = A^*:=\overline{A^T}, &   \quad  & \gamma_0^{-1} |\eta|^2\,\le\, A(y)\eta \cdot \overline{\eta}\,\le\, \gamma_0 |\eta|^2 \quad \text{a.e. } y \in \mathbb{R}^n, \, \forall \eta \in \CC^{n}, \ \text{for some constant $\gamma_0 \ge 1$.}
\end{aligned}
\ee
We assume that $A(y)$ is periodic with period $1$ with respect to each variable  $y_j$, $j=1,2,...,n$, 
that is $\Box=[-\frac{1}{2},\frac{1}{2}]^n$ is  the periodicity cell and $\Box^*=[-\pi,\pi]^n$ is the associated 
Bloch-dual cell (the Brillouin zone). Our goal is to construct approximations of $u_\ep $ with ``operator-type'' error bounds (in $L^2(\RR^n)$ and $H^1(\RR^n)$ 
norms) of order $\ep$, i.e. those that linearly depend on $\| F \|_{L^2(\RR^n)}$.  
%Clearly for this purpose we need only consider the case $\ep \in (0,1)$\footnote{Indeed for $\ep \ge 1$, it immediately follows that $
%\| u_\ep \|_{H^1(\RR^n)} \le \ep \gamma_0  \|f \|_{L^2(\RR^n)}.
%$
%}.

In this and most of subsequent examples a key role will be played by Floquet-Bloch or Gelfand transform  %closely related 
 transform, which reduces 
problems like \eqref{differentialPDE} to an equivalent one of the type \eqref{p1} via a decomposition into quasi-periodic functions. % via Gelfand transform or %closely related 
%Floquet-Bloch transform as follows. 
 Henceforth,  we shall mostly use for the latter 
the definitions and 
notation in a %n equivalent 
 form close to e.g. \cite{Ku}, \cite{ZhSpectr}. 
Namely, Gelfand (or Floquet-Bloch) transform $U :L^2(\RR^n) \rightarrow L^2(\square^* \times \square)$ and its inverse are unitary maps that can be defined, for example, as the continuous extensions of the ($L^2$-isometric) mappings
\be
\label{gt1}
UF(\theta,y) \,: =\, (2\pi)^{-n/2}\sum_{m\, \in\, \ZZ^n} F(y+m) e^{-\i\, \theta \cdot (y+m)}, \qquad F \in C^\infty_0(\RR^n),
\ee
%with inverse 
\be
\label{gt2}
U^{-1} G (x) \,=\,(2\pi)^{-n/2} \int_{\square^*} G(\t ,\,\{x\}   )\, e^{\i\, \t \cdot x}\, {\rm d}\t , \qquad 
G \in C \big( \square^* ;\, C_{per} (\square)\big),
\ee
cf
 e.g. \cite{Gel} or \cite{Ku}. 
($C_{per} (\square)$ is here the space of functions on $\square$ which allow a $\square$-periodic continuous extension on $\mathbb{R}^n$; $\{x\}\in\square$ denotes a ``fractional part'' of $x\in\mathbb{R}^n$: 
e.g. $\{x\}:=x-m$ for the unique $m\in\mathbb{Z}^n$ such that $x-m\in [-1/2,1/2)^n\subset\square$.) 


Fix $0<\ep <1$ and  %begin by incorporating  \eqref{differentialPDE} into the general framework presented in Section \ref{sec:pf}. 
apply to \eqref{differentialPDE} 
first 
the normalised (unitary)  rescaling operator $\Gamma_\ep : L^2(\RR^n) \longrightarrow L^2(\RR^n)$,  
$(\Gamma_\ep F)(y) :=  \ep^{n/2}F(\ep y)$, and then the %unitary 
Gelfand transform. 
%\footnote{
%\cite{Gel} and \cite[Section 3.2]{BeLiPa}.
%} $U: L^2(\RR^n) \longrightarrow L^2(\Box^* \times \Box)$.  
%which
It is a key property of the Gelfand transform that it reduces a PDE problem with periodic coefficients in $\mathbb{R}^n$ like \eqref{differentialPDE} into an 
equivalent  $\t$-parametrised family of problems on the periodic cell $\Box$. 
Namely via the properties of Gelfand transform, cf. in particular \eqref{gt2}, we determine that for a.e. $\theta \in \Box^*$,  
$u_{\ep,\theta}(\cdot) := U\Gamma_\ep u_\ep(\theta, \cdot)$ belongs to $H^1_{per}(\Box)$ the 
Hilbert space of functions in $H^1(\Box)$ admitting a locally-$H^1$  $\square$-periodic extension on $\mathbb{R}^n$, 
 and solves
\begin{equation}
\label{differentialPDE1}
-\,e^{-\i\, \theta \cdot y}\,\ep^{-2}\,{\rm div}\Big( A \left( y\right)\nabla\big( e^{\i \theta \cdot y} u_{\ep,\theta}(y)\,\big)\, \Big) \,\,+\,\, u_{\ep,\theta}(y) \,\,=\,\, F_{\ep,\theta}(y), \qquad  \text{a.e.}\  y \in \Box,
\end{equation}
where $F_{\ep,\theta}(\cdot) : = U\Gamma_\ep F(\theta, \cdot) \in L^2(\Box)$. The standard 
equivalent weak formulation of \eqref{differentialPDE1} is: 
\begin{equation}
\label{difformIK}
\left\{ \begin{aligned}
& \text{ Find $u_{\ep,\theta} \in H^1_{per}(\Box)$ the solution to} \\
&\ep^{-2} \int_\Box A(\nabla+\i\t)u_{\ep,\theta}\cdot \overline{(\nabla+\i\t)\tilde{u}} \,\,+\,\int_\Box u_{\ep,\theta}\overline{\tilde{u}} \,\,=\,\, \int_\Box F_{\ep,\theta}\overline{\tilde{u}}, \qquad \forall \tilde{u} \in H^1_{per}(\Box).
\end{aligned} \right.
\end{equation}
 Problem \eqref{difformIK}  is of the type \eqref{p1}. Indeed, with chosen Hilbert space $H=H^1_{per}(\Box)$,  it can be restated as: 
\begin{equation}
\label{difform}
\mbox{Find } u_{\ep,\t}\in H\,\,\mbox{ such that }
\ep^{-2}\, a_\t\left(u_{\ep,\theta}\,,\tilde{u}\right) \,+\, b_\t\left(u_{\ep,\theta}\,,\tilde{u}\right)
\,\, =\,\,\l f,\, \tilde{u}\r, \qquad \forall \tilde{u} \in H, \,\, a.e.\, \t \in \Theta,
\end{equation}
for  $\Theta=\Box^*$,  
$\,\,\left\l f, \tilde u \right\r \,: =\,   \int_\Box F_{\ep,\theta}\, \overline{\tilde u}\,$, 
\begin{equation}
\label{cforma}
\begin{aligned}
a_\t(u,\tilde{u}) \,:=\, \int_\Box A(\nabla+\i\t)u\cdot \overline{(\nabla+\i\t)\tilde{u}}\,,
%\big( A (\nabla+\i \t) u , (\nabla+\i \t) \tilde{u}\big) 
 \qquad \text{and} \qquad  b_\t(u,\tilde{u})\,:=\, \int_\Box u\,\overline{\tilde{u}}\,
% (u,\tilde{u})
,  \qquad u,\tilde{u} \in H^1_{per}(\Box).
\end{aligned}
\end{equation}
%where $(\cdot,\cdot)$ denotes the standard $L^2(\Box)$ scalar product. 
Recall that, according to \eqref{astructure}, for the inner products in $H$, 
$(u,\tilde u)_\t:=a_\t (u,\tilde u)+b_\t (u,\tilde u)$. 
Assumption \eqref{as.b1} then easily follows: %, e.g. via expansion into Fourier series on $\Box$. 
%Indeed, 
for $u\in H$, 
%$u(y)=\sum_{m\in\ZZ^n}c(m)e^{2\pi\i\, m\cdot y}$. 
%Then, 
via assumptions \eqref{IKcond} on the coefficients $A$, 
\begin{equation}
\label{2.1classhom}
a_{\t_1}[u]\le\gamma_0\left\|\left(\nabla+\i\t_1\right)u\right\|^2_{L^2(\Box)}\le
2\gamma_0\left\|\left(\nabla+\i\t_2\right)u\right\|^2_{L^2(\Box)}+
2\gamma_0\left|\t_1-\t_2\right|^2\|u\|^2_{L^2(\Box)}\le 
2\gamma_0^2a_{\t_2}[u]+8\pi\gamma_0\|u\|^2_{L^2(\Box)}, 
%\gamma_0^{-1}\sum_{m\in\ZZ^n}|c(m)|^2\left(\,|2\pi m+\t|^2+1\right)\,\,\le\,\,\,\|u\|_\t^2\,\,\,\le\, 
%\gamma_0\sum_{m\in\ZZ^n}|c(m)|^2\left(\,|2\pi m+\t|^2+1\right), 
\end{equation}
and %as $|\t|\le n^{1/2}\pi$ for $\t\in\Box^*$, for $|m|\ge n^{1/2}$ one has 
%$\pi|m|\le|2\pi m+\t|\le 3\pi|m|$. So $\|\cdot\|_\t$ are uniformly equivalent to the standard norm 
%in $H^1_{per}(\Box)$, implying 
\eqref{as.b1} holds e.g. with $K=\left(2\gamma_0^2+8\pi\gamma_0+1\right)^{1/2}$. 
 It is then straightforward to show that \eqref{ass.alip} is also satisfied. 
%for $K \le 1 + 2\pi \sqrt{n\gamma_0}$ and $L_a = \sqrt{\gamma_0} K$.
%
%\begin{flalign*} a_\t[u] =  \big( A (\nabla+\i \t) u , (\nabla+\i \t) \tilde{u}\big) = \big( A (\nabla+\i \t_2) u , (\nabla+\i \t) \tilde{u}\big)   + \big( A\i (\t-\t_2) u , (\nabla+\i \t) \tilde{u}\big) \\
% \le a_{\t}[u]^{1/2} \big( a_{\t_2}[u]^{1/2}  + 2\pi\sqrt{\gamma n}  \| u\| \big) \\
% \le2 \pi \sqrt{\gamma n}  (a_{\t}[u]^{1/2} + \| u \|)  \end{flalign*}

Let us next determine the spaces $V_\theta$ and $W_\theta$, cf. \eqref{spaceV} and \eqref{2.6-w}. 
Notice that the form $a_\t[\cdot]$ satisfies 
\be\label{IKaest}
a_\t[u] \,\ge\, \gamma_0^{-1}\int_\Box \big| (\nabla+\i\t) u\big|^2\,\,\ge\,\, \gamma_0^{-1}|\t|^2 \int_\Box | u|^2, \qquad \forall u \in H^1_{per}(\Box),\ \forall \t  \in \Box^*,
\ee
where the last inequality follows e.g. %the above 
via expansion into Fourier series on $\Box$. 
Indeed, for $u\in H$, 
$u(y)=\sum_{m\in\ZZ^n}c(m)e^{2\pi\i\, m\cdot y}$. 
Then $|2\pi m +\t|\ge|\t|$, $\forall\t\in\Theta$, 
$m\in\ZZ^n$. Therefore, from \eqref{spaceV} and \eqref{2.6-w} via \eqref{IKaest} and \eqref{cforma},  
%we determine that the spaces $V_\theta$ and $W_\theta$, cf. \eqref{spaceV}, are
\be
\label{vwclasshom}
\begin{aligned}
V_\theta = \left\{
\begin{array}{lr}
\{ 0 \}, & \theta \neq 0, \\[5pt]
{\rm Span} (\mathbf{e}), & \theta = 0,
\end{array} 
\right. & \qquad & W_\theta = \left\{
\begin{array}{lr}
H^1_{per}(\Box), & \theta \neq 0, \\[5pt]
H^1_{per, 0} \,: =\,\big\{ w\in H^1_{per}(\Box) \,\, \big| \,\, \int_\Box w = 0 \big\}, & \theta = 0.
\end{array} 
\right.,
\end{aligned}
\ee
where $\mathbf{e}=1$ is the constant unity function on $\Box$. 
We clearly  see that $V_\t$ is discontinuous with respect to $\t$  (only) at the origin,  and 
 we are in the context of Sections \ref{section:discV} and \ref{sec.2dif}. Now  let us demonstrate that the 
related main assumptions \eqref{KA}--\eqref{H4} %, which would suffice for the purposes for the present example, %trivially 
hold. % in this setting.
%\eqref{difform}-\eqref{cforma}.

\textbullet\, The proof of  \eqref{KA} follows from 
%By Proposition \ref{prop.kaequiv} we need only demonstrate \eqref{KA2.1},  and we observe
noticing  that the stronger assertion \eqref{KA2.1} (see Proposition \ref{prop.kaequiv}) trivially holds with $C=1$ upon choosing  ${c}= b_\t$ (recall that $b_\t$ is in fact $\t$-independent, see \eqref{cforma}, and $b_\t$ is 
$\|\cdot\|_\t$-compact by the Rellich compactness theorem).
% Indeed, 
%\[\int_{\Box} A (\nabla+\i \t) u \cdot \overline{(\nabla+\i \t) u} + \int_\Box |u|^2 \le  \int_{\Box} A (\nabla+\i \t) u \cdot \overline{(\nabla+\i \t) u}+\int_\Box |u|^2. 
%\]


\textbullet \, Hypothesis  \eqref{contVs} holds trivially for $V_\star = \{ 0\}$ with $L_\star = 0$;  see  Remark \ref{constV}.

\textbullet\, Let us show that hypothesis \eqref{distance} holds with $\gamma = \left(n\pi^2+\gamma_0\right)^{-1}$. Indeed, it follows from \eqref{IKaest} that, for $\t\ne 0$,    
\[
\|u\|_\t^2
\,\,=\,\,a_\t[u] \,+\,b_\t[u]\,\,
\le\,\, \big(1+ \gamma_0|\t|^{-2}\big)\,a_\t[u]\,, 
%\qquad \forall u \in H^1_{per}(\Box), \, \forall \t \in \Box^* \backslash \{ 0\},
\]
and consequently (recalling that $|\t|^2\le n\pi^2$ )
\[\nu_\theta\,\,:=\, \inf_{w \in W_\t \backslash \{ 0 \}}\, \frac{a_\t[w]}{\|w\|_\t^2}\,\,\ge\,\, 
\big(1+ \gamma_0|\t|^{-2}\big)^{-1}\,\,=\,|\t|^2\left(|\t|^2+\gamma_0\right)^{-1}\,\,\ge\,\,|\t|^2\left(n\pi^2+\gamma_0\right)^{-1}, 
\]
as required. 
Notice also that, for any $r>0$, Theorem \ref{thm:contV} (with $\Theta=\Box^*$ replaced by 
$ \Theta \cap \{ |\t | \ge r \}$, cf. Remark \ref{rem3.2}) implies 
\begin{equation}\label{coutside}
\ep^{-2} \gamma_0^{-1} \big\| (\nabla + \i \t) u_{\ep,\t} \big\|^2_{L^2(\Box)} \,+\, 
\big\| u_{\ep,\t}\big\|^2_{L^2(\Box)} \,\,\le\,\, \ep^2 \left(n \pi^2 + \gamma_0\right) |r|^{-2} 
\big\| U\Gamma_\ep F(\theta, \cdot)\big\|^2, \quad \mbox{a.e. }\,  \t \in \Box^*,\, |\t| \ge |r|.
\end{equation}
since $V_\t = \{ 0 \}$ for $\t \neq 0$.
% and  $\inf_{ \t \in \Theta_r } \nu_\t \ge r^2 (n\pi^2 + \gamma_0)$.

\textbullet \, Assumption \eqref{H4} is obviously satisfied with $K_{a'} = \gamma_0$, $K_{a''}=0$ and 
\begin{equation}
\label{IKa'}
\begin{aligned}
& a'_{0}(v,  u) \cdot \t: =\i\int_\square   A\, \t v \cdot \overline{ \nabla  u} 
%-i( \t\cdot A\nabla v, u)
,\qquad a''_{0}(v,  \tilde{v})\t \cdot \t:= \int_\square A\,\t\cdot\t\, v\, \overline{ \tilde{v}}.
\end{aligned}
\end{equation}

%\textbullet \, Assumption \eqref{H5} is obvious for $L_b = 0$ since $b_\t$ is independent of $\t$. 


As \eqref{KA}--\eqref{H4} hold we can apply our general theory and, in particular, we conclude that  Theorem \ref{thm.maindiscthm} holds. Let us now specify the objects appearing therein. In this setting $V^\star_\t = \{ 0 \}=:V_\star$ for all 
$\theta\in\Box$,  and 
%so $v^h = 0$. 
the space $Z$ in \eqref{spaceZ} is simply the one-dimensional space  $V_0 = {\rm Span} (\mathbf{e})$. 
Therefore, in the notation of Theorem \ref{thm.maindiscthm}, $v^h =0$,  $z^h=z_{\ep,\t}\textbf{e}$, where $z_{\ep,\t}\in \CC$,  and  \eqref{z3prob} becomes  the scalar linear algebraic equation
\begin{equation}
	\label{IKz3prob}
	\ep^{-2}\, a^{\rm h}_{\t}(\textbf{e}, \textbf{e})z_{\ep,\t} \,\,+\,\,
	b_\t(\textbf{e}, \textbf{e})\, z_{\ep,\t} \,\,=\,\, \l f, \textbf{e} \r\,.
	\end{equation}
Let us rewrite the coefficients of this equation in more explicit terms. Clearly, $\langle f, \mathbf{e} 
\rangle = \int_\square U\Gamma_\ep F(\theta, y)  \, {\rm d}y$ and $b_\t(\textbf{e}, \textbf{e})=1$. 
%$$ (\textbf{e}, \textbf{e})_0=a_0(\textbf{e},\textbf{e}) +(\textbf{e},\textbf{e})=\int_{\Box}  A \nabla \textbf{e}\cdot \overline{\nabla \textbf{e}}+\int_{\Box} \textbf{e}=1 .
%$$
As for the first coefficient, recalling \eqref{defhom.form} and \eqref{cell:prob2}:
 \begin{equation}
\label{ahom7.1}
a^{\rm h}_{\t}(\textbf{e}, \textbf{e})\,\,=\,\,a''_0(\textbf{e},\textbf{e})\, \t \cdot \t \,-\, 
a_0( N_\t\textbf{e} , N_\t \textbf{e})\,\,
 =\,\,a''_0(\textbf{e},\textbf{e})\, \t \cdot \t\,+\,a_0'\left( \textbf{e},\,N_\t\textbf{e}\right)\cdot\t,
\end{equation} 
where $N_\t\textbf{e}\in H^1_{per,0}(\Box)$ solves (via \eqref{cell:prob2} and \eqref{IKa'} and recalling 
$A=A^*$)
\begin{equation}
\label{cell:IKprob22}
a_0( N_\t\textbf{e} , w) \,=\, -\,\,
a_0'\left( \textbf{e},\,w\right)\cdot\t\,=\,-\,
\i\,\t\cdot \int_\Box \overline{\,A\nabla w}\,, \qquad \forall w \in H^1_{per,0}(\Box), \,\,\, \forall \t \in \RR^n.
\end{equation}
As a result, $N_\t\textbf{e}=\i\,\t\cdot\ourN$ where $\ourN\in H^1_{per,0}\left(\Box;\,\CC^n \right)$ solves 
%Upon adopting the notation  $\ourN:=-\i N \textbf{e}$,  \eqref{cell:IKprob22}
%takes the form:
\begin{equation}
\label{IKclasscor1}
%\left\{ \begin{aligned}
%& \text{  Find $\ourN \in H^1_{per,0}(\Box;\CC^n )$ such that} \\
%&
\int_{\Box} A\Big( \nabla (\t\cdot \ourN) + \t \Big) \cdot \overline{\nabla w} \,=\, 0\,, \qquad \forall w \in H^1_{per,0}(\Box),\ \, \forall \t \in \RR^n.
%\end{aligned} \right.
\end{equation}
Thus $\ourN$ is the classical corrector, see e.g. \cite{JKO}.
%\footnote{\label{fn.class}We note for future reference that \eqref{IKclasscor1} implies that $\ourN=0$ if $A$ is a constant matrix.}
\begin{remark} If we introduce the components of $\ourN$, $\ourN=( \ourN^1, \ldots ,  \ourN^n)$, then \eqref{IKclasscor1} can be equivalently rewritten in a more traditional form:
\begin{equation}
\label{IKclasscor}
\left\{ \begin{aligned}
& \text{  For $j=1,\ldots, n$, find $\ourN^{j} \in H^1_{per,0}(\Box)$ such that} \\
&\int_{\Box} A\bigl( \nabla \ourN^{j} + e^j \bigr) \cdot \overline{\nabla w }= 0 , \qquad \forall w \in H^1_{per,0}(\Box),
\end{aligned} \right.
\end{equation}
where $e^1,\ldots,e^n$ is the canonical basis in $\mathbb{R}^n$.
\end{remark}
Now we express  $a^{\rm h}_{\t}(\textbf{e}, \textbf{e})$ in terms of $\ourN$: via \eqref{ahom7.1} and \eqref{IKa'} and 
using $\mbox{Im}\left(a^{\rm h}_{\t}(\textbf{e}, \textbf{e})\right)=0$, 
\[
 a^{\rm h}_{\t}(\textbf{e}, \textbf{e})\,=\, a''_0(\textbf{e},\textbf{e})\, \t \cdot \t\,+\,
a_0'(\textbf{e}, N_\t\textbf{e})\cdot\t\,=\, 
%\int_\Box   b\,\t\cdot \t -\int_\Box   b\nabla (\t \cdot \ourN)\cdot\overline{\nabla(\t \cdot \ourN)}  \\ 
 %\int_\Box  b\,\theta\cdot \big(    \theta+\overline{\nabla(\theta \cdot \ourN)} \big)  =
\int_\Box  A \,  \theta\cdot   \theta \,+\,\i\int_\Box  A\,\t\cdot\overline{\nabla N_\t\textbf{e}}\,=\,
\int_\Box  A \,  \theta\cdot   \theta \,+\int_\Box  A\,\nabla (\theta \cdot \ourN)\cdot   \theta. 
\]
Thus we can represent $a^{\rm h}_{\t}(\textbf{e}, \textbf{e})$ as 
\be \label{IKahom55} a^{\rm h}_{\t}(\textbf{e}, \textbf{e})\,\,=\,\,  A^{\rm hom} \theta \cdot \theta,
\ee
 where $A^{\rm hom}$ is the classical homogenised matrix with components
\be\label{IKcoef}
 A^{\rm hom}_{ij}\,\, :=\,\, \int_\Box\Big( A_{ij}+\sum_{k=1}^n A_{ik}  \partial_{k}\ourN^{j}\Big)\,, \qquad  i,j \in \{1,\ldots,n\}.
 \ee
Matrix $A^{\rm hom}$ is well-known to be positive definite (which can also directly be seen from \eqref{IKahom55}, \eqref{ahcoercive}) and Hermitian (which can be checked using \eqref{IKcoef} and \eqref{IKclasscor}). 
%Finally the right hand side of \eqref{IKz3prob} looks quite abstract but in fact 
%$$(F_{\ep,\t}, \textbf{e})=\ep^{-d}\widehat{f}(\t\ep^{-1}),
%$$ 
%where $\widehat{f}$ is the unitary Fourier transform of $f$ (the right hand side of \eqref{differentialPDE}  ). 

Putting all of this together, we conclude from \eqref{IKz3prob}
\begin{equation}\label{classicalzsol}
z_{\ep,\t} 
% = \frac{\l f, \textbf{e}\r}{\ep^{-2} A^{\rm hom}\t\cdot\t + 1}
 \,\,=\,\, \frac{\int_\square U\Gamma_\ep F(\theta, y)  \, {\rm d}y}{\ep^{-2}A^{\rm hom}\t\cdot\t \,\,+\, 1}\,.
\end{equation}
%In final preparation for applying Theorem \ref{thm.maindiscthm}
Finally, using \eqref{fstar} and \eqref{cforma} 
we observe that $\|f\|_{*\t}\le \big\|U\Gamma_\ep F(\theta, \cdot) \big\|_{L^2(\Box)}$. 
Now from Theorem \ref{thm.maindiscthm} and \eqref{coutside}, 
%putting all of the above together 
we readily deduce the following result. 
\begin{proposition}
\label{thm.IKeclassexample} Consider the objects of  Theorem \ref{thm.maindiscthm}.
Let $u_{\ep,\theta}\in H^1_{per}(\Box)$ solve \eqref{differentialPDE1},   $z_{\ep,\t}$ be given by 
\eqref{classicalzsol}.  
%given in \eqref{classicalzsol}
Then for some $r_1>0$ 
and $\chi$ the characteristic function for the ball of radius $r_1$, 
%Define the approximation to $u_{\ep,\theta}$ by
%$$ u_{\ep,\theta}^{\rm approx} = \chi(\t) (1 + i\t \cdot \ourN) z_{\ep,\t},
%$$
%where $\chi$ is the characteristic function of $B_{r_1}$.
%Then 
%there exist positive constants $C_1, C_0$ independent of $\Ep$ and $\t$ such that: $\forall \t \in \Box^*,$	
\begin{gather}
	\label{eq.IKmaindiscthm}
\begin{aligned}
\ep^{-2}\Bigl\| (\nabla+\i \t) \Bigl( u_{\ep,\theta} \, -\, \chi(\t) \left(1 + \i\t \cdot \ourN\right) z_{\ep,\t}\Bigr)\Bigr\|^2_{L^2(\Box)}  \,\, +\,\,	\bigl\| u_{\ep,\theta}  \,-\,\chi(\t) \left(1 + \i\t \cdot \ourN\right) 	z_{\ep,\t}\bigr\|^2_{L^2(\Box)} \hspace{1.7 cm}\\  \le\,\,  \ep^2\,  c_0^2\,\bigl\|U\Gamma_\ep F(\theta, \cdot) \bigr\|^2_{L^2(\Box)}, 
\end{aligned} \\
\label{IKL2}
\big\| u_{\ep,\theta} \, -\,   \chi(\t)	z_{\ep,\t} \big\|_{L^2(\Box)} \,\le\,\,  \ep\, c_1\,\big\| U\Gamma_\ep F(\theta, \cdot) \big\|_{L^2(\Box)},
\end{gather}
for some positive constants $c_0, c_1$ independent of $\ep$, $\t$ and $F$. 
\end{proposition}
%Few comments are in order. The inequality \eqref{eq.IKmaindiscthm} is equivalent to  \eqref{eq.maindiscthm} but is not exactly the same.  
We now show that, via the inverse Gelfand and scaling transforms, inequalities  \eqref{eq.IKmaindiscthm} and \eqref{IKL2} 
provide respectively the desired  $H^1$ and $L^2$ 
operator estimates for certain approximations of  $u_\ep$, the solution to \eqref{differentialPDE}. 
To this end, recall that according to \eqref{IKL2} $\chi\, z_{\ep,\theta}$ serves as an approximation to the transformation 
$u_{\ep,\t}$ of the original solution $u_\ep$ of \eqref{differentialPDE} where $u_\ep=\Gamma^{-1}_\ep U^{-1}u_{\ep,\t}$. 
So set the (inverse) transformed approximation 
%Indeed, if we set 
\be
\label{IKappr45} 
u_\ep^{(0)} \,\,:=\,\, \Gamma^{-1}_\ep U^{-1} \,\chi\, z_{\ep,\theta},  
\ee
and notice that as $\chi(\t)z_{\ep,\t}$ does not depend on $y$, by \eqref{gt2} and \eqref{classicalzsol}, 
$u_\ep^{(0)}\in C^\infty(\RR^n)\cap H^1(\RR^n)$ and $\nabla u_\ep^{(0)}\in L^\infty(\RR^n)$. 
Next, for the ``corrector'' term in \eqref{eq.IKmaindiscthm}, by the properties of the Gelfand transform, cf. \eqref{gt2}, 
%and note that (by the properties of the Gelfand transform)
% and  \be\label{IKcorrep} 
% 	u_\ep^{(1)} := \Gamma^{-1}_\ep U^{-1} \chi (1 + \i\t \cdot \ourN) z_{\ep,\t} 
%% 	=
\begin{eqnarray}
\label{7.19-2}
\Gamma^{-1}_\ep U^{-1}\chi\, \i\,\t \cdot \ourN  	\, z_{\ep,\t}\,\,=\,\, 
\Gamma^{-1}_\ep\left(\ourN\cdot U^{-1}\,\i\,\t\,\chi\, z_{\ep,\t}\right)\,\,=\,\,
\Gamma^{-1}_\ep\left(\ourN\cdot \nabla\,U^{-1}\,\chi\, z_{\ep,\t}\right)\,\,= \nonumber \\ 
\,\, \ \ \ \ \ \ 
 \,\,(\tilde\Gamma^{-1}_\ep\ourN)\cdot   \left(\Gamma^{-1}_\ep \nabla U^{-1}\chi 	z_{\ep,\t}\right)
\,\,=\,\, 
\ep\,(\tilde\Gamma^{-1}_\ep\ourN)\cdot  \nabla \left(\Gamma^{-1}_\ep U^{-1}\chi 	z_{\ep,\t}\right)
\,=\, \ep\left(\tilde\Gamma^{-1}_\ep\ourN\right)\cdot  \nabla u^{(0)}_\ep,
% \Gamma^{-1}_\ep U^{-1} \chi (1 + \i\t \cdot \ourN) z_{\ep,\t}  = u_\ep^{(0)}+\Gamma^{-1}_\ep U^{-1}\chi \i\t \cdot \ourN  	\, z_{\ep,\t} 
%=u_\ep^{(0)}+ \ep(\Gamma^{-1}_\ep\ourN)\cdot  \nabla \left(\Gamma^{-1}_\ep U^{-1}\chi 	z_{\ep,\t}\right)
%=u_\ep^{(0)}+ \ep(\Gamma^{-1}_\ep\ourN)\cdot  \nabla u^{(0)}_\ep,
\end{eqnarray} 
where $\left(\tilde\Gamma_\ep^{-1} f\right)(y):=f(y/\ep)$ denotes ``ordinary'' rescaling. 
[In \eqref{7.19-2} we have used sequentially that: $\ourN$ does not depend on $\t$ and 
$U^{-1}\big(f(y)g\big)=f(y)U^{-1}g$,  
$\,U^{-1}\big(i\t\,f(\t)\big)=\nabla\left(U^{-1}f\right)$, 
$\,\Gamma_\ep^{-1}(fg)=\left(\tilde\Gamma_\ep^{-1}f\right)\Gamma_\ep^{-1}g$, 
and $\Gamma_\ep^{-1}(\nabla f)=\ep\nabla\left(\Gamma_\ep^{-1}f\right)$.]
%\ee 
%where the last inequality comes from the basic properties of the Gelfand transform,
As a result \eqref{eq.IKmaindiscthm} and \eqref{IKL2}, upon application of the $L^2$-unitary inverse 
Gelfand transform $U^{-1}$ and inverse rescaling $\Gamma_\ep^{-1}$ 
(and noticing that 
$\Gamma_\ep^{-1}U^{-1}\big(\,(\nabla+i\t)\,f\big)=\ep\,\nabla\left(\Gamma_\ep^{-1}U^{-1}f\right)\,$), 
% and \eqref{IKmod} 
lead to the following.
\begin{theorem}
\label{IK77}
Let $u_\ep$ solve \eqref{differentialPDE} and $u_\ep^{(0)}$
be given by  \eqref{IKappr45} with $z_{\ep,\t}$ specified by \eqref{classicalzsol}. Then 

\begin{gather}
\label{IKH1est}
\Bigl\Vert u_\ep \,-\, \Bigl(u_\ep^{(0)}+ \ep\big(\tilde\Gamma^{-1}_\ep\ourN\big)\cdot  \nabla u^{(0)}_\ep\Bigr)\,  \Bigr\Vert_{H^1(\mathbb{R}^n)} \,\le\,\, \ep\, c_0\, \Vert F \Vert_{L^2(\mathbb{R}^n)}, \\
\label{IKL2est}
\big\Vert u_\ep - u_\ep^{(0)}  \big\Vert_{L^2(\mathbb{R}^n)} \,\,\le\,\, \ep\, c_1\,\Vert F \Vert_{L^2(\mathbb{R}^n)}.
\end{gather}
\end{theorem}
The above theorem already provides constructive approximations of the solution to \eqref{differentialPDE}, however it is customary to relate these %approximations 
to the solution of the %so-called 
corresponding 
{\it homogenised equation}. We now provide this link. 
For the homogenised differential operator applied to $u^{(0)}_\ep$, 
by the standard properties of the scaling and Gelfand transformations 
together with  \eqref{IKappr45} and the fact that $\chi z_{\ep,\t}$ is independent of $y$,
\[
-\,{\rm div}\bigl( A^{\rm hom} \nabla u^{(0)}_\ep\bigr) \,=\, \Gamma_\ep^{-1} U^{-1} 
\Big(\,-\, \ep^{-2} (\nabla + \i \t) \cdot  A^{\rm hom} (\nabla + \i \t) \Big)U  \Gamma_\ep u^{(0)}_\ep\,=\,
\Gamma_\ep^{-1} U^{-1} 
\left(\ep^{-2}\t \cdot  A^{\rm hom} \t \right)\chi z_{\ep,\t}. 
\]
%
%\[
%(U L U^{-1}  f)(\t, \cdot) = L(\t) f(\t,\cdot) 
%\]
%
%\[
%\Big( U \Gamma_\ep -{\rm div}\big( A^{\rm hom} \nabla ) \Gamma_\ep^{-1} U^{-1} \Big) \phi(\t,\cdot) =} \big( \ep^{-2} (\nabla + \i \t) \cdot  A^{\rm hom} (\nabla + \i \t) \big) \phi(\t,\cdot), \quad \phi \in L^2(\square^* \times \square).
%\]
This together with 
%\eqref{IKappr45}, 
\eqref{classicalzsol} % and the fact that $z_{\ep,\t}$ is independent of $y$, 
implies that $u^{(0)}_\ep$ solves 
%\begin{flalign*}
%-{\rm div}\big( A^{\rm hom} \nabla {u^{(0)}_\ep} \big) u^{(0)}_\ep & = -{\rm div}\big( A^{\rm hom} \nabla {u^{(0)}_\ep} \big) \Gamma_\ep U^{-1} z_{\ep,\t} = U \Gamma_\ep \big( \ep^{-2} (\nabla +\i \t) \cdot A^{\rm hom}(\nabla + \i \t) + 1 \big) z_{\ep,\t}  \\
%& =  U \Gamma_\ep \big( \ep^{-2}  A^{\rm hom}\t \cdot \t  + 1 \big) z_{\ep,\t}  
%\end{flalign*}
%%To do this we simply determine what PDE the function $u^{(0)}_\ep$ solves.  
%Since $z_{\ep,\t}$ is independent of $y$ then we can rewrite \eqref{classicalzsol} as 
%\[
%\big( \ep^{-2} (\nabla +\i \t) \cdot A^{\rm hom}(\nabla + \i \t) + 1 \big) z_{\ep,\t}  = \int_\square U\Gamma_\ep f(\theta, y)  \, {\rm d}y,
%\]
%and so 
%\[
%\big( \ep^{-2} (\nabla +\i \t) \cdot A^{\rm hom}(\nabla + \i \t) + 1 \big) \chi z_{\ep,\t}  = \chi \int_\square U\Gamma_\ep f(\theta, y)  \, {\rm d}y,
%\]
%and applying inverse Gelfand and scaling gives 
\[
-\,{\rm div}\big( A^{\rm hom} \nabla {u^{(0)}_\ep} \big) + u^{(0)}_\ep\,\,=\,\,  
\Gamma_\ep^{-1} U^{-1}  \chi  
\int_\square U\,\Gamma_\ep F(\theta, y)  \, {\rm d}y.
\]
Now notice that the standard Fourier transform\footnote{The conventional Fourier transform in $L^2(\RR^n)$  
is here specified by
	\[
	\left(\mathcal{F}g\right)(\theta) : = (2\pi)^{-n/2}\int_{\mathbb{R}^n} e^{-\i \theta \cdot y}g(y)\,dy, \qquad g \in 
	L^2(\RR^n)\cap L^1(\RR^n).
	\]} $\mathcal{F}$ in $L^2\left(\RR^n\right)$ and the Gelfand transform, as directly follows from \eqref{gt1} 
	and \eqref{gt2}, are related by the identities: 
	$\int_\square U g(\theta, y)  \, {\rm d}y = \mathcal{F} g(\t)$, $g \in L^2(\RR^n)$, $\t \in \square^*$, and $U^{-1} (h \otimes \mathbf{e}) = \mathcal{F}^{-1} h$ for $h \in L^2(\RR^n)$ with support in $\square^*$.
Consequently, we determine that $u^{(0)}_\ep$ is the solution to 
\begin{equation}\label{Sep}
-\,{\rm div}\big( A^{\rm hom} \nabla {u^{(0)}_\ep} \big) \,+\, u^{(0)}_\ep\,=\, \mathcal{S}_\ep F,
\end{equation}
for the smoothing operator
 $\mathcal{S}_\ep : L^2(\RR^n) \rightarrow C^\infty(\RR^n) \cap H^1(\RR^n) \cap W^{1,\infty}(\RR^n)$ given by
\be
\label{7.22-2}
 \mathcal{S}_\ep F\,\,\, =\,\,\, 
 \Gamma^{-1}_\ep \mathcal{F}^{-1} \chi \mathcal{F}\Gamma_\ep \,F\,\, =\,\, 
\mathcal{F}^{-1} ( \tilde\Gamma_\ep \chi ) \mathcal{F}\,F.  
\ee 
(In the latter equality we have used that $\Gamma^{-1}_\ep \mathcal{F}^{-1}= \mathcal{F}^{-1}\Gamma_\ep$, 
$\mathcal{F}\Gamma_\ep=\Gamma_\ep^{-1}\mathcal{F}$ and 
$\Gamma_\ep \chi \Gamma_\ep^{-1}g=( \tilde\Gamma_\ep \chi )g\,$.) 
%\[
%\big( \ep^{-2}A^{\rm hom}\t\cdot\t + 1 \big) U \Gamma_\ep u^{(0)}_\ep = \chi  \int_\square U\Gamma_\ep f(\theta, y)  \, {\rm d}y = \chi \mathcal{F} \Gamma_\ep f
%\]
%then
%
%\[
%U^{-1} \big( \ep^{-2}A^{\rm hom}\t\cdot\t + 1 \big) U \Gamma_\ep u^{(0)}_\ep = U^{-1}  \chi \mathcal{F} \Gamma_\ep f = \mathcal{F}^{-1} \chi \mathcal{F} \Gamma_\ep f 
%\]
%so
%\[
%\Gamma_\ep^{-1} U^{-1} \big( \ep^{-2}A^{\rm hom}\t\cdot\t + 1 \big) U \Gamma_\ep u^{(0)}_\ep =  \Gamma_\ep^{-1} \mathcal{F}^{-1} \chi \mathcal{F} \Gamma_\ep f  = \mathcal{S}_\ep f
%\]
%
%
%
%Note that since $U \Gamma_\ep u^{(0)}_\ep$ is independent of $y$ then 
%\[
%\t_i U  \Gamma_\ep u^{(0)}_\ep = U \partial_i \Gamma_\ep  u^{(0)}_\ep  = \ep U \Gamma_\ep  \partial_i  u^{(0)}_\ep , 
%\]
%so 
%\[
%\Gamma_\ep^{-1} U^{-1} \big( \ep^{-2}A^{\rm hom}\t\cdot\t + 1 \big) U \Gamma_\ep u^{(0)}_\ep  = - {\rm div} A^{\rm hom} \nabla u^{(0)}_\ep.
%\]
%From \eqref{IKappr45} and \eqref{classicalzsol} we see that
%\[
%g
%\]
Let ${u}\in H^2(\RR^n)$ be the solution to the classical homogenised equation
\begin{equation}
\label{homeq}
-{\rm div}\big( A^{\rm hom} \nabla {u}(x) \big) \,+\, {u}(x) \,=\, F(x), \qquad x \in \RR^n. 
\end{equation}
Applying $\mathcal{S}_\ep$ to \eqref{homeq} and using the standard properties of the Fourier transform, it is easy to see 
that $u^{(0)}_\ep$ solving \eqref{Sep} and ${u}$ 
are related by the identity $u^{(0)}_\ep = \mathcal{S}_\ep u$. 
%It is now easy to see using the standard properties of the Fourier transform that $u^{(0)}_\ep$ and ${u}\in H^2(\RR^n)$ the solution to the classical homogenised equation
%\begin{equation}
%\label{homeq}
%-{\rm div}\big( A^{\rm hom} \nabla {u}(x) \big) \,+\, {u}(x) \,=\, F(x), \qquad x \in \RR^n,
%\end{equation}
%are related by the identity $u^{(0)}_\ep = \mathcal{S}_\ep u$. 
Further, let us show that one has the inequality
\be
\label{7.23-2}
\big\| u^{(0)}_\ep - u \big\|_{H^1(\RR^n)} \,\,\le\,\,\, \ep\, r^{-1}_1 \gamma_0\, \| F\|_{L^2(\RR^n)}.
\ee
Indeed, by \eqref{7.22-2}, the Plancherel identity and \eqref{homeq}, 
\be
\label{7.23-3}
\big\| u^{(0)}_\ep - u \big\|_{H^1(\RR^n)} \,=\, 
\Big\| (1+|\xi|^2)^{1/2}\big(\tilde\Gamma_\ep \chi-1\big)\mathcal{F}u(\xi)\Big\|_{L^2(\RR^n)}\,=\, 
\Big\| \frac{(1+|\xi|^2)^{1/2}}{A^{\rm hom}\xi\cdot\xi+1}
\left(\tilde\Gamma_\ep \chi-1\right)\mathcal{F}F(\xi)\Big\|_{L^2(\RR^n)}. 
\ee
Noticing that $\left\vert\tilde\Gamma_\ep \chi-1\right\vert$ vanishes for $|\xi|<\ep^{-1}r_1$ and equals -1 otherwise, and recalling 
that \eqref{IKcond} implies  $A^{\rm hom}\xi\cdot\xi\ge \gamma_0^{-1}|\xi|^2$ (see e.g. \cite{JKO}) 
leads to \eqref{7.23-2}. 
%(As $\gamma_0^{-1}$ is the ellipticity constant for $A^{\rm hom}$.) 

Combining \eqref{7.23-2} with inequalities \eqref{IKH1est} and \eqref{IKL2est} provides the following result.
\begin{proposition}
\label{homeqest}
 Let $u_\ep$ solve \eqref{differentialPDE} and $u$ solve \eqref{homeq}. Then 
	\begin{gather}
	\label{IKH1est...}
	\Bigl\Vert\, u_\ep \,-\, \Bigl(u+ \ep\big(\tilde\Gamma^{-1}_\ep\ourN\big)\cdot  \nabla 
	\mathcal{S}_\ep u\Bigr)\,\Bigr\Vert_{H^1(\mathbb{R}^n)} \,\,\le\,\,\, \ep\, \bigl(c_0+ r_1^{-1} \gamma_0\bigr)\, \Vert F \Vert_{L^2(\mathbb{R}^n)}, \\
	\label{IKL2est...}
	\big\Vert u_\ep - u  \big\Vert_{L^2(\mathbb{R}^n)} \,\,\le\,\,\, \ep\, 
	\bigl(c_1+ r_1^{-1} \gamma_0\bigr)\,\Vert F \Vert_{L^2(\mathbb{R}^n)}.
	\end{gather}
\end{proposition}
%
%
%
%
%
%where $A^{\rm hom}$ is defined in \eqref{IKcoef}.  Applying the Fourier transform $\mathcal{F}$\footnote{The Fourier transform  is fixed by
%	\[
%	\left(\mathcal{F}g\right)(\theta) : = (2\pi)^{-n/2}\int_{\mathbb{R}^n} e^{-\i \theta \cdot y}g(y)\,dy, \qquad g \in L^2(\RR^n).
%	\]} to \eqref{homeq} gives 
%\[
%(A^{\rm hom} \xi \cdot \xi +1)\mathcal{F} {u}(\xi) = \mathcal{F} f(\xi), \qquad \xi \in \mathbb{R}^n.
%\]
%Considering $\xi = \ep^{-1} \t$, $\t \in \square^*$, in the above equation and noting that $\Gamma_{\ep}^{-1} \mathcal{F} = \ep^n \mathcal{F} \Gamma_\ep$ gives
%\begin{equation}\label{home2}
%(\ep^{-2}A^{\rm hom}\t\cdot\t +1)\mathcal{F}\Gamma_\ep\textbf{u}(\t)=\mathcal{F} \Gamma_\ep f(\t), \qquad \t \in \square^*.
%\end{equation}
%As the Fourier and Gelfand transforms are related by the identity $\mathcal{F}g(\t) = \int_\square Ug(\t,y) \, {\rm d}y$, $\t \in \Box^*$, for any $g\in L^2(\mathbb{R}^n)$,
%%$$(\ep^{-2}b^{\rm hom}\t\cdot\t + \lambda)(U\Gamma_\ep\textbf{w},\textbf{e})=(U \Gamma_\ep f,\textbf{e})$$
%then comparing \eqref{home2} and  \eqref{classicalzsol} gives 
%$\mathcal{F}\Gamma_\ep\mathbf{u} = z_{\ep,\t}$ for $\t \in \square^*$, $| \t | < r_1$. Thus upon noting $\Gamma^{-1}_\ep \mathcal{F}^{-1}\chi \mathcal{F}\Gamma_\ep = \mathcal{F}^{-1}(\Gamma_\ep\chi )\mathcal{F}$, inequalities  \eqref{IKH1est}, \eqref{IKL2est} provide the $H^1$ and $L^2$ estimates
%\begin{gather}
%\label{ikh1}  
%\Vert u_\ep-{u} - \ep\ourN(\tfrac{\,\cdot}{\ep})\cdot  \nabla \mathcal{P}_\ep{u}\, \Vert_{H^1(\mathbb{R}^n)} \le \ep c_0 \Vert f \Vert_{L^2(\mathbb{R}^n)}, \\
%\label{bs} \Vert u_\ep - \mathcal{P}_\ep u \Vert_{L^2(\RR^n)} \le \ep c_1 \Vert f \Vert_{L^2(\RR^n)},
%\end{gather}
%for  $\mathcal{P}_\ep = \mathcal{F}^{-1}(\Gamma_\ep\chi )\mathcal{F}$. Furthermore, since $\| \phi - \mathcal{P}_\ep \phi \|_{L^2(\mathbb{R}^n)} \le \ep r_1^{-1}\| \nabla \phi \|_{L^2(\mathbb{R}^n)}$, $\phi \in H^1(\mathbb{R}^n)$, and $A^{\rm hom}$ is elliptic with constant $\gamma_0^{-1}$ then $\| u - \mathcal{P}_\ep u \|_{L^2(\mathbb{R}^n)} \le \ep r_1^{-1} \gamma_0^{1/2}/2 \| f\|_{L^2(\mathbb{R}^n)}$. Combining this inequality with \eqref{bs} we arrive at the $L^2$ estimate\footnote{ In general one can not  remove the smoothing operator $\mathcal{P}_\ep$ from \eqref{ikh1} unless the corrector $\ourN$ is a bounded function. It is known that $\ourN$ has the required regularity in the scalar setting but we shall not dwell on it here.}
%\begin{equation}
%\label{bs1}
%\Vert u_\ep -  u \Vert_{L^2(\RR^n)} \le \ep (c_1 + \tfrac{1}{2}r_1^{-1}\gamma_0^{1/2}) \Vert f \Vert_{L^2(\RR^n)}.
%% \\
%%\label{ikh1}  
%%\Vert u_\ep-{u} - \ep\ourN(\tfrac{\,\cdot}{\ep})\cdot  \nabla \mathcal{P}_\ep{u}\, \Vert_{H^1(\mathbb{R}^n)} \le \ep c_1 \Vert f \Vert_{L^2(\mathbb{R}^n)}.
%\end{equation}
%%provide the following comparsion.
%%gives the $L^2$ estimate:

The result of type \eqref{IKL2est...} was obtained for the first time in \cite{BiSu}, see also \cite{ZhL2}. The result  \eqref{IKH1est...} was  obtained in \cite{ZhH1}, \cite{ZhPasH1} although with different smoothing operator and in \cite{BSh1} with the same smoothing operator $\mathcal{S}_\ep$. 
Finally, we reiterate that the  operator and spectral 
results of Section \ref{s:resolv} are also formally applicable for this example (with $\mathcal{E}_\t =I$), but %, for this example, these results 
will  not be developed here as are of limited further value (since the spectrum of the homogenised operator has a simple 
structure with no gaps). 
This is in sharp contrast with the following example, %which follows and 
where the general results of Section \ref{s:resolv} 
play a key role. 

\subsection{High-contrast elliptic PDE  with rapidly oscillating periodic coefficients}\label{e.dp}
Here we  demonstrate our method's applicability to scalar high-contrast  elliptic PDEs with periodic coefficients. 
The present example has formed a key motivation for the general approach developed in the present work. 
We comment here that the approach below could be extended to a wider class of 
PDE systems, cf \cite{IVKVPS13}, in particular would be essentially the same for the analogous 
high-contrast problems of linear %; the details can be left to the reader
elasticity\footnote{The only substantive difference  in the case of linear elasticity is that we would need to replace the extension Proposition \ref{prp.zhiext} with analogous extension property in linear elasticity, see Proposition \ref{prp.zhiextelast} below.}.
% We emphasise that 
We assume here $n > 1$; the case $n=1$ is qualitatively different and much simpler, cf. Example \ref{e:idp} below.
%The asymptotics of spectrum (without error estimates) for this problem was studied in \cite{Zhi2005}.  
%Here, we  establish correct-type error estimates for the resolvent problems, as well as error estimates for the asymptotics of the spectrum. In particular, the existence and location of gaps in the spectrum for a given fixed $\ep$ is provided by  Proposition \ref{ikthm2}; indeed assumption \eqref{ikn10} trivially holds, see footnote \ref{ft.dp}. To the best of the authors' knowledge these results are new.
We follow the example of \cite{Zhi2005} and focus on the simplest geometric model:
\begin{equation}
\label{dp.differentialPDEdp}
\left\{ \begin{aligned}
& \text{Find $u_\ep \in H^1(\RR^n)$ such that} \\
&-\,{\rm div}\bigl( A_\ep \left(\tfrac{x}{\ep} \right) \nabla u_\ep(x)\, \bigr) \,\,+\,\, u_\ep(x) \,\,=\,\, F(x), \qquad x \in \RR^n,
\end{aligned} \right.
\end{equation}
for a given $F \in L^2(\RR^n)$ and $\Box$-periodic coefficients $A_\ep$ of the form
\[
A_\ep(y) = \left\{ 
\begin{array}{lr}
1  & y \in \Box \backslash B, \\[5pt]
\ep^2  & y \in B.
\end{array} \right.
\]
Here  the set $B$ has Lipschitz boundary,  $\overline{B} \subset \left(-\tfrac{1}{2},\tfrac{1}{2}\right)^n$ for simplicity, and $\square \backslash B$ is connected. 
%Here  the set $B$ has a Lipschitz boundary and is such that $\overline{B} \subset (0,1)^d$ $\Box \backslash B$ is connected.  The tensor coefficients  satisfy the following standard  conditions:
%\begin{equation}
%\label{dpcoeffs}
%\begin{aligned}
%A^{(r)}= (A^{(r)})^*,  & \quad  & \gamma^{-1} | \eta|^2 \le A^{(r)}\eta\cdot \eta \le \gamma |\eta|^2 \quad \forall y \in Q_r, \ \text{for some constant $\gamma \ge 1$. }
%\end{aligned}
%\end{equation}
%Again as in Example  \ref{e.class} we can without loss of generality consider $\ep <1$.

Following the steps in Example \ref{e.class}, %we can without loss of generality consider 
for $0<\ep <1$ after an application of the rescaling $\Gamma_\ep$ and the Gelfand transform ${U}$, we determine that $u_{\ep, \t} (\cdot) : = U \Gamma_\ep u_\ep(\theta, \cdot)\in H^1_{per}(\Box)$ for a.e. $\t \in \Theta : = \Box^*$, and solves 
\begin{equation}\label{dp.pde}
-\,e^{-\i \theta \cdot y}\ep^{-2}{\rm div}\Bigl( A_\ep \left( y\right)\nabla\left( e^{\i \theta \cdot y} u_{\ep,\theta}(y)\,\right)\,
\Bigr) \,\,+\,\, u_{\ep,\theta}(y) \,\,=\,\, U\Gamma_\ep F(\t,y), \qquad  \text{a.e.}\  y \in \Box. 
\end{equation}
This has the equivalent weak formulation
\begin{equation}
\label{dp.varp}
\begin{aligned}
\ep^{-2} \int_{\Box \backslash B} (\nabla + \i \t) u_{\ep,\t} \cdot \overline{(\nabla + \i \t) \tilde{u}} \,\,\,\,+\, \int_{B}  (\nabla + \i \t) u_{\ep,\t} \cdot \overline{(\nabla + \i \t) \tilde{u}} \,\,\,+ \int_{\Box} u_{\ep,\t} \overline{\tilde{u}} \,\,\, =\,\,\, \l f, \tilde{u}\r ,  \hspace{1.7cm}\\ \forall \tilde{u} \in H^1_{per}(\Box),
\end{aligned}
\end{equation}
 where 
\begin{equation}
\label{fdp}
\ \l f, \tilde{u}\r \,\,:=\,\, \int_\Box \,  U\Gamma_\ep F(\t,y)\,\, \overline{\tilde{u}(y)} \,\, {\rm d}y.
\end{equation} 
 We note \eqref{dp.varp}  is a problem of the form \eqref{p1} with $\ H :=H^1_{per}(\Box)$, $ \ \Theta : = \Box^*$,  
\begin{equation}
\label{abcdp}
\begin{aligned} 
&a_\t(u,\tilde{u}) := \int_{\Box \backslash B} (\nabla + \i \t) u \cdot \overline{(\nabla + \i \t) \tilde{u}}\,, \quad  \text{and} \quad     b_\t(u,\tilde{u}) : =  
\int_{B} (\nabla + \i \t) u \cdot \overline{(\nabla + \i \t) \tilde{u}}\,\,+ \int_\square u\overline{\tilde{u}}. 
\end{aligned}
\end{equation}
Then the same argument as in Example \ref{e.class}, cf \eqref{2.1classhom}, assures 
that assumptions \eqref{as.b1} and \eqref{ass.alip} hold.
% for $K \le 1 + 2\pi \sqrt{n}$ and $L_a = 1$. Let us determine the spaces $V_\t$, $W_\t$.
Next, for determining the subspaces $V_\t$ but also for some later purposes, we notice that 
 the assumptions on the `soft' phase $B$ ensure the following extension result (see e.g. \cite[Proposition 4.3]{Zhi2005}).
%[Lemma 3.2]{JKO}).
\begin{proposition}\label{prp.zhiext}
	There exists an extension operator $E : H^1(\Box \backslash B) \rightarrow H^1(\Box)$ with the following properties: $Eu|_{\Box \backslash B} = u$, $\| Eu \|_{H^1(\Box)} \,\,\,\le\,\,\, C_E \| u \|_{H^1(\Box \backslash B)}$,  and
	\begin{equation}\label{ZhExtension}
	\int_\Box |\nabla Eu|^2 \,\,\,\,\le\,\,\,\, C_E^2 \int_{\Box \backslash B} |\nabla u|^2, \quad \forall \,u \in H^1(\Box \backslash B),
	\end{equation}	
	with some constant $C_E>0$ independent of $u$.
\end{proposition}
Using Proposition \ref{prp.zhiext} we argue
\begin{align}
a_\t[u]\,=\,
 \int_{\Box \backslash B} | (\nabla + \i \t) u|^2  & =  \int_{\Box \backslash B} 
\left\vert \nabla (e^{\i \t \cdot y}  u)\right\vert^2 \,\ge\, 
C_E^{-2} \int_\Box \bigl\vert \nabla E \left(e^{\i \t \cdot y}  u\right)\bigr\vert^2 \,\ge\, 
C_E^{-2}|\, \t |^2 \int_\Box \left\vert E \left(e^{\i \t \cdot y}  u\right)\right\vert^2 
 \nonumber \\ \label{dpH1} & \ge\,\, C_E^{-2}|\, \t |^2 \int_{\Box \backslash B}|u|^2\,, \qquad  \forall u \in H^1_{per}(\Box \backslash B), \,\,\forall \t \in \Theta,
\end{align}
where the first inequality holds due to \eqref{ZhExtension}, the second (c.f. \eqref{IKaest}) by  expanding 
$e^{-\i \t \cdot y}E \left(e^{\i \t \cdot y}  u\right)\in H^1_{per}(\Box)$ in %terms of trigonometric polynomials 
Fourier series in $\Box$, 
and the last from the extension property $E \vert_{\Box \backslash B} = I$. Therefore, we deduce via \eqref{spaceV} that
\begin{equation}\label{hcV}
V_\theta = \left\{
\begin{array}{lr}
H^1_0(B), & \theta \neq 0, \\[5pt]
\Big\{ v \in H^1_{per}(\Box) \, \Big\vert \, \text{$v$ is constant in $\Box \backslash B$} \Big\} \,\,=\,\, \CC\overset{\cdot}{+} H^1_0(B), & \theta = 0, 
\end{array} 
\right.  \end{equation}
where elements of $H^1_0(B)$ are understood as those of $H^1_{per}(\Box)$ which are identically zero in 
%to be extended by zero into 
$\Box \backslash B$. Hence, recalling \eqref{2.6-w} and \eqref{astructure} with \eqref{abcdp}, 
\[
W_\theta = \left\{
\begin{array}{lr}
\Big\{ w\in H^1_{per}(\Box) \,\, \big\vert \,  -(\nabla+\i \t )\cdot (\nabla+\i \t ) w\, +\, w = 0 \, \text{ in } B \Big\}, & \theta \neq 0, \\[5pt]
\Big\{ w\in H^1_{per}(\Box) \,\, \big\vert \,  -\Delta w + w = 0 \, \text{ in } B \ \& \ \int_\Box w = 0 \Big\}, & \theta = 0.
\end{array} 
\right.
\]
Clearly, $V_\t$ is discontinuous with respect to $\t$ (only) at the origin. 
% and we are in the context of Sections \ref{section:discV} and \ref{sec.2dif}. 
Now, proceeding as in Example \ref{e.class}, let us demonstrate that the main assumptions \eqref{KA}--\eqref{H6} hold for the setting \eqref{abcdp} and identify the principal objects associated with each assumption.

\textbullet\, To prove  \eqref{KA} we shall demonstrate that the stronger assertion  \eqref{KA2} holds for $C=C_E^{2} $  and 
$c(u,\tilde{u}) : = C_E^2 \int_{\Box \backslash B} u \overline{\tilde{u}}$, with the latter clearly being $\|\cdot\|_\t$-compact by the Rellich theorem. 
For fixed $\t \in \Theta$ and $u\in H$, we find that $v:= u - e^{-\i\t\cdot y}E\left(e^{\i\t\cdot y}u\right) \in H^1_0(B) \subseteq V_\t$ satisfies
\[
\| u-v\|_\t^2\,=\,
\| u-v\|_{L^2(\Box )}^2\,+\,\| (\nabla +\i \t)(u-v) \|_{L^2(\Box)}^2  \,\,= 
\]
\begin{equation}
\label{h1prdp}
 \ \ \ \ \ \ \ \ \  
\left\| E \left(e^{\i \t \cdot y} u\right) \right\|_{H^1(\Box)}^2 \,\,\,\le\,\,\, C_E^2\,\,  \| e^{\i \t \cdot y} u \|_{H^1(\Box \backslash B)}^2 
\,\,=\,\, C_E^2  \int_{\Box \backslash B} \Big(\big| (\nabla + \i \t) u \big|^2 \,+\, |u|^2\Big).
\end{equation}
Since, for any $w\in W_\t$ and $v\in V_\t$, %$\|P_{W_{\t}} u\|_{\t} = \inf_{v \in V_{\t}} \| u - v\|_{\t},$  
$\|w\|_{\t} \le \| w- v\|_{\t}$,   
the above inequality with $u=w$ implies that
\begin{equation}\label{dpNewKA}
\begin{split}
\| w\|_\t^2  \,\,\le\,\, C_E^2\bigg( a_\t[w] + \int_{\Box \backslash B} |w|^2 \bigg), \qquad \forall w\in W_\t, \ \forall \t \in \Theta, 
\end{split}
\end{equation}
which establishes \eqref{KA2}. 
%This is \eqref{newKA} for $K_{\mathcal{I}} =C_E^{2} $  and $c_{\mathcal{I}}(u,\tilde{u}) : = C_E^2 \int_{\Box \backslash B} u \overline{\tilde{u}}$. Therefore \eqref{KA} holds (see Proposition \ref{prop.kaequiv}).


\textbullet \, The validity of  \eqref{contVs} is immediate for $V_\star = H^1_0(B)$ and $L_\star = 0$;  see  Remark \ref{constV}. Furthermore, one can choose the defect subspace $Z = {\rm Span}\, \{ \mathbf{e}\}$ where $ \mathbf{e} \in H^1_{per}(\square)$ is 
the constant unity: indeed, $V_0 = H^1_0(B) \dot{+} \CC$ implying \eqref{spaceZ}, and for $\phi \in H^1_0(B)$, % by H\"{o}lder's inequality, 
\begin{equation}
\label{4.17-dp}
\left|(\phi,\mathbf{e})_0\right| \,=\, \left|\int_B \phi\right| \,\,\le\,\,   | B |^{1/2} \left( \int_B |\phi|^2 \right)^{1/2} \le 
\,\,| B |^{1/2} \|\phi\|_0 \,\,=\,\, |B|^{1/2} \| \phi \|_0 \| \mathbf{e} \|_0 ,  
\end{equation}
i.e. %that is 
\eqref{VZorth} holds with $K_Z = |B|^{1/2} <1$.

\textbullet\, Assumption \eqref{distance} holds with $\nu_\star = C_E^{-2}(\pi^2 n + C_E^2)^{-1}$. Indeed, \eqref{dpNewKA} and \eqref{dpH1} imply for $w\in W_\t$ and $\t\neq 0$, 
%we compute
$
\| w \|_\t^2 
%\le C_E^2 \bigg( a_\t[w] + \int_{\Box \backslash B}|w|^2 \bigg) 
\le C_E^2 \big(1 + C_E^2 |\t|^{-2}\big) a_\t[w],
$
and consequently
\[
\nu_\theta\,\,=\,\, \inf_{w \in W_\t \backslash \{ 0 \}} \frac{a_\t[w]}{\|w\|_\t^2}\,\,\,\,\ge\,\,\,  
C_E^{-2} \big(1+ C_E^2|\t|^{-2}\big)^{-1}\,\,\,\ge\,\,\, |\t|^2\, C_E^{-2}\big(\pi^2 n + C_E^2\big)^{-1}.
\]


\textbullet \, Assumption \eqref{H4} is obviously satisfied, cf. \eqref{IKa'}, with  $K_{a'} =1$ , $K_{a''}=0$ and 
\begin{equation}
\label{dp.rega}
\begin{aligned}
& a'_{0}(v, u) \cdot \t\,\,: =\,\, \i  \int_{\Box \backslash B}   \t v \cdot \overline{ \nabla {u}} ,\qquad 
a''_{0}(v, \tilde{v})\,\t \cdot \t\,\,:=\,\, |\t|^2 \int_{\Box \backslash B} v \overline{\tilde{v}}.
\end{aligned}
\end{equation}
Now let us calculate $a^{\rm h}_\t[\mathbf{e}]$. Recalling \eqref{defhom.form}, \eqref{cell:prob2} and \eqref{dp.rega}, 
% and the fact $\textbf{e}_{Z} =1$ in $\Box \backslash B$,  
we obtain: %determine that:
\begin{equation}\label{dp.a'2}
a^{\rm h}_{\t}[\mathbf{e}]\,\,=\,\,a''_0[\mathbf{e}] \t \cdot \t \,+\, a_0'\left( \mathbf{e}, N_\t\mathbf{e}\right) \cdot \t 
\,\,=\, \int_{\Box \backslash B} |\t|^2 \,+\, \i \int_{\Box \backslash B} \t \cdot\overline{\nabla \left( N_\t \mathbf{e}\right)},
%=  \int_{\Box\backslash B}     \theta\cdot  \big( \t + \overline{\nabla (-\i \theta \cdot N \mathbf{e}_{dp})} \big) ,
%a''_0(\textbf{e},\textbf{e}) \t \cdot \t+a_0'( \t \cdot N\textbf{e} ,\textbf{e})\cdot\t,
\end{equation} 
where $N_\t\mathbf{e}\in W_0$ solves (see \eqref{cell:prob2})
\begin{equation}
\label{dp.cell1}
\int_{\Box \backslash B }  \nabla\left(N_\t\textbf{e}\right) \cdot \overline{\nabla w} \,\,=\,\, -\,\i \int_{\Box \backslash B} \t\cdot \overline{ \nabla w} , \qquad \forall w \in W_0, \,\,\, \forall \t \in \RR^n.
\end{equation}
It is clear (since $V_0 = \CC \dot{+} H^1_0(B)$ and $H^1_{per}(\square) = V_0 \oplus W_0$) that the above equality holds 
in fact for test functions $\phi \in H^1_{per}(\square)$. Therefore, 
\begin{equation}
\label{dp.correctorproblem}\begin{aligned}
& N_\t\mathbf{e}\,\,=\,\, \i \,\t \cdot \big(\ourN_1^{\rm pd},\ldots,\ourN_n^{{\rm pd}}\big) \ \ \text{in $\,\Box\backslash B$, where real-valued  } 
\text{  $\ourN_j^{{\rm pd}} \in H^1_{per}(\Box \backslash B)$, $j=1,...,n$, solve}  \\ 
& 
\int_{\Box \backslash B}  \nabla \ourN_j^{{\rm pd}}\cdot\, \overline{\nabla \phi }\,\,=\, - \int_{\Box \backslash B}  e^j \cdot \overline{\nabla \phi } , \qquad \forall \phi \in H^1_{per}(\Box \backslash B),
\end{aligned} 
\end{equation}
with $e^1,\ldots,e^n$ denoting the canonical basis in $\mathbb{R}^n$. Thus 
$\ourN^{\rm pd}  = (\ourN_1^{\rm pd},\ldots,\ourN_n^{\rm pd})$ is (up to an additive constant) the perforated domain corrector, see e.g. \cite[Section 3.1]{JKO}.
%	\footnote{We note for future reference that \eqref{IKclasscor1} implies that $\ourN=0$ if $A$ is a constant matrix.}
%\end{remark}
As a result  \eqref{dp.a'2} and \eqref{dp.correctorproblem} allow us to rewrite  $a^{\rm h}_{\t}[\mathbf{e}]$ as 
\be \label{dp.ahom} 
%a^{\rm trunc}_{\t}(\textbf{e}, \textbf{e})=  A^{\rm hom}_{dp} \theta \cdot \theta,
a^{\rm h}_{\t}[\mathbf{e}]\,\,=\,\,  A^{\rm hom}_{\rm pd} \theta \cdot \theta,
\ee
where $A^{\rm hom}_{\rm pd}$ is the perforated domain homogenised matrix with components
\be\label{dp.coef}
\{A^{\rm hom}_{\rm pd}\}_{ij} \,\,=\,\, \int_{\Box \backslash B} \delta_{ij} \,+ \int_{\Box \backslash B}  \partial_{i}\ourN_{j}^{\rm pd}, \qquad  i,j \in \{1,\ldots,n\}.
\ee
Matrix $A^{\rm hom}_{\rm pd}$ is well-known to be positive definite  and symmetric  (see e.g. \cite[Section 3.1]{JKO}); these can 
also be seen directly, respectively via \eqref{dp.ahom} and \eqref{ahcoercive}, and  \eqref{dp.coef} and \eqref{dp.correctorproblem} 
with $\phi=\ourN_{i}$.

\textbullet \, Assumption \eqref{H5} is immediate for $L_b = 1$: indeed, via \eqref{abcdp}, 
\[
\bigl| b_\t(u, \tilde{u}) - b_0 (u, \tilde{u})  \bigr|\,\,=\,\,  
\left|  \int_{B}  \i \t u \cdot \overline{(\nabla + \i \t) \tilde{u}} \,\,+   \int_{B} \nabla  u \cdot \overline{ \i \t  \tilde{u}}\, \right|  
\]
\[
\ \ \ \ \ \ \ \ 
\,\,\le\,\,
|\t| \Big( \| u \|_{L^2(B)} \| (\nabla + \i \t)\tilde{u} \|_{L^2(B)} \,+\,  \| \nabla u \|_{L^2(B)}  \| \tilde{u} \|_{L^2(B)} \Big)
\,\,\le\,\, |\t|\, \|u\|_0 \,\|\tilde{u}\|_\t, \qquad \forall u, \tilde{u} \in H^1_{per}(\square).
\]
Now, we can take as the operator $\mathcal{E}_\t$, see Lemma \ref{propeth}, multiplication by $e^{-\i \t \cdot y}$. Indeed, for $\phi, \tilde{\phi} \in H^1_0(B)=V_\star$,
\[
b_\t\left(e^{-\i \t \cdot y}\phi,e^{-\i \t \cdot y}\tilde{\phi}\right) \,:=\, \int_B (\nabla + \i \t) e^{-\i \t \cdot y}\phi \,\cdot\, \overline{(\nabla + \i \t) e^{-\i \t \cdot y}\tilde{\phi} } \,+ \int_\square e^{-\i \t \cdot y} \phi \overline{e^{-\i \t \cdot y}\tilde{\phi}} = \int_B \nabla \phi \cdot \overline  {\nabla \tilde{\phi}} + \int_\square  \phi \overline{ \tilde{\phi}} = b_0(  \phi,  \tilde{\phi}),
\]
i.e.  \eqref{Eprop1} holds. One can also readily verify \eqref{Eprop2} with $K_b = \sqrt{|B|(1+n/4)}$ as follows:  
\[
\Big| b_\t(e^{-\i\t\cdot y} \phi,\mathbf{e})  - b_0(\phi,\mathbf{e}) \Big| \,\,=\,\,
  \left| \int_B e^{-\i\t\cdot y}\nabla  \phi\cdot \overline{ \i \t } \,\, + \int_B  (e^{-\i \t \cdot y} -1) \phi \right| 
	\ \ 
	\le\,\, |\t| \,\| \nabla \phi\|_{L^2(B)} |B|^{1/2} \,\,+\, 
	\]
	\[
	\ \ \ \ \ 
	\sqrt{|B|n/4}\,|\t|\|\, \phi\|_{L^2(B)}  
\,\, \le\,\, \sqrt{|B|(1+n/4)}\, |\t|\,  \|\phi\|_0 \,\,=\,\, \sqrt{|B|(1+n/4)}\, |\t| \, \|\phi\|_0 \| \mathbf{e} \|_0, \quad \  \forall \phi \in H^1_0(B).
\]

\textbullet\, Finally, we can see that \eqref{H6} holds  for  $\mathcal{H} = L^2(\square)$ 
with $\t$-independent $d_\t=d_0$ the standard $L^2(\square)$ inner product and 
 $\mathcal{E}_\t$ multiplication by   $e^{-\i \t \cdot y}$. 
Indeed, $H=H^1_{per}(\square)$ is compactly embedded into and dense in $\mathcal{H}$ and \eqref{ik2} holds, and $K_e=\sqrt{n}/2$ in \eqref{H6}.
% i.e. $d_0(u,\tilde{u}) = \int_\square u \overline{\tilde{u}}$, then multiplication by clearly makes sense and  (by  \eqref{defdt})
%\begin{equation}\label{dpdtheta}
%d_\t( u,\tilde{u}) = \int_\square e^{-\i \t\cdot y} u(y) \overline{ e^{-\i \t\cdot y} \tilde{u}(y)} \, {\rm d}y = \int_\square  u(y) \overline{  \tilde{u}(y)} \, {\rm d}y , \qquad u,\tilde{u} \in L^2(\square), \, \t \in \square^*,
%\end{equation}
%i.e. $d_\t = d_0$ for all $\t \in \square^*$. 
% Additionally,
%\[
%\| \mathcal{E}_\t - I \|_{(\mathcal{H},d_\t) \rightarrow (\mathcal{H},d_0)}  = \sup_{u \in L^2(
%\square), \| u \|_{L^2(\square)}=1} \Big(\int_\square | (e^{-\i \t \cdot y} -1) u |^2\Big)^{1/2} \le |\t|.
%\]


%
%\textbullet \, Assertion \eqref{Eprop1} holds for the operator $\mathcal{E}_\t$ given as the multiplication by $e^{-\i \t \cdot y}$. Assumption \eqref{H6} holds for $\mathcal{H} = L^2(\square)$ with  inner product $d_0(u,\tilde{u})  = \int_\square u \overline{u}$ and 
As \eqref{KA}--\eqref{H6} hold we can apply %our method and conclude that all 
the results of Sections \ref{section:discV} - \ref{s:resolv}, and we % hold. 
detail below implications of the relevant approximation theorems for the present example. 
\subsubsection{Application of Theorem \ref{thm.IKunifest2}}
We begin with specifying the approximations given in Theorem \ref{thm.IKunifest2}. Therein,  $V_\star = H^1_0(B)$ and $Z ={\rm Span}\, \{ \mathbf{e}\}$ 
(recalling $\mathbf{e}(y) \equiv 1$) and consequently, $v = v_{\ep,\t}\in H^1_0(B)$,  $z = c_{\ep,\t} \mathbf{e}$, $c_{\ep,\t} \in \CC$, and problem \eqref{IKz3prob88}, 
via \eqref{dp.ahom}, \eqref{abcdp} and \eqref{fdp}, specialises to 
	\begin{equation}\label{dplimp1}
%\label{z3prob}
\begin{aligned}
\ep^{-2} \left(A^{\rm hom}_{\rm pd}\t \cdot \t\right) c_{\ep,\t}  \overline{\tilde{c}}\,\, + 
\int_B \nabla v_{\ep,\t}\cdot \overline{\nabla \phi} \,\,+ \int_\square (v_{\ep,\t}+c_{\ep,\t} ) \overline{(\phi + \tilde{c})}  
\,\, =\, \int_\square U \Gamma_\ep F(\t ,y) 
\overline{\bigl(e^{-\i \t \cdot y} \phi(y) + \tilde{c}\bigr)} \, {\rm d}y, \hspace{.5cm} \\ \qquad \forall\, \phi \in H^1_0(B) , \,\, \ \forall \tilde{c} \in \CC.
\end{aligned}
\end{equation} 
This can equivalently be re-written as 
\begin{equation}\label{zdp}
\left\{ \ \begin{aligned}
 \Big( \ep^{-2} A^{\rm hom}_{\rm pd}\t \cdot \t  +1\Big) c_{\ep,\t} \,\, + \int_B v_{\ep,\t}(y) \, {\rm d}y\,\,  & = \int_\square U \Gamma_\ep F(\t ,y) \, {\rm d}y, \qquad \t \in \square^*;\\
 -\,\Delta v_{\ep,\t}(y)\,  + v_{\ep,\t}(y)\,+ c_{\ep,\t}\,\, &=\,\, e^{\i \t \cdot y} \,U\Gamma_\ep F(\t,y), \qquad y \in B, 
\ \ \t \in \square^*.
% \label{vdp}
\end{aligned} \right.
\end{equation}
Applying Theorem \ref{thm.IKunifest2} 
and noticing that, by \eqref{fstar} and \eqref{fdp}, \eqref{abcdp},  
$\|f\|_{*\t}\le \big\|U\Gamma_\ep F(\t,\cdot)\big\|_{L^2(\square)}$, 
we conclude that  inequalities \eqref{IKfinal3} and 
\eqref{IKfinal3-2} imply the following result.
\begin{proposition}
	\label{prop:dp}
	Let $u_{\ep,\t}$ solve  \eqref{dp.pde} and $c_{\ep,\t}$, $v_{\ep,\t}$ solve \eqref{zdp}.	Then 
	\be
	\label{dp.mainest1}
	\begin{aligned}
		\ep^{-2}\int_{\Box \backslash B}  \Big\vert\big(\nabla+\i \t\big) \Big( u_{\ep,\t}(y)  - 
		\bigl(1 + \i\,\t \cdot \ourN^{\rm pd}(y) \bigr) c_{\ep,\t}\Big)\Big\vert^2  \,\,\,  +\,\int_{\Box \backslash B}  
		\bigl|u_{\ep,\theta}(y) - 
		\big(1 &+ \i\,\t  \cdot \ourN^{\rm pd}(y) \big)  c_{\ep,\t}\bigr|^2 \, {\rm d}y\hspace{.05\linewidth} \\&  
		\le\,\,\,  C_9\,\ep^2 \int_\square \big|U\Gamma_\Ep F(\t,y)\big|^2 \, {\rm d}y,
	\end{aligned}
	\ee
	\be\label{dp.mainest2}
	\int_\square  \bigl| u_{\ep,\theta}  \,-\, \left( 	c_{\ep,\t} +e^{-\i \t \cdot y}v_{\ep,\t}\right)   \bigr|^2  \,\,\le\,\,  
	\,C_{10}\,\ep^2 \int_\square |U\Gamma_\Ep F(\t,y)|^2 \, {\rm d}y. \hspace{.35\linewidth}
	\ee
\end{proposition} 
Inequalities  \eqref{dp.mainest1} and \eqref{dp.mainest2} provide   $L^2$ estimates for the corresponding approximations of $u_\ep$, the solution  to \eqref{dp.differentialPDEdp}, and its gradient on the ``stiff'' phase. Indeed, 
applying the inverse transforms to the approximation $c_{\ep,\theta} \mathbf{e}+e^{-\i \t \cdot y}v_{\ep,\t}$ to 
$u_{\ep,\theta}=U\Gamma_\ep u_\ep$ in \eqref{dp.mainest2}, set 
\be\label{dp.app1} 
u_\ep^{(0)} \,\,:=\,\, \Gamma^{-1}_\ep U^{-1} c_{\ep,\theta} \mathbf{e}  , \qquad 
v_{\ep}^{(0)} \,\,: =\,\, \Gamma_\ep^{-1}  U^{-1} e^{-\i \t \cdot y} v_{\ep,\t}. 
%+ \chi \alpha_{\Ep,\theta}\otimes \mathbf{e}_0)
\ee
Note that, cf \eqref{IKappr45} and \eqref{7.19-2}, %by \eqref{dplimp1} and 
as $c_{\ep,\t}\mathbf{e}$ is $y$-independent $u_\ep^{(0)}$ is smooth and 
\[
\Gamma^{-1}_\ep U^{-1}  \big(1 + \i\,\t \cdot \ourN^{\rm pd}\big) c_{\ep,\t} \mathbf{e}
%	 = w_\Ep^{(0)}+\Gamma^{-1}_\Ep U^{-1}\chi \i\t \cdot \ourN  	\, c_{\Ep,\t} 
\,=\,\,u_\ep^{(0)}+ \ep\left(\tilde\Gamma^{-1}_\ep\ourN^{\rm pd}\right)\cdot  \nabla u^{(0)}_\ep \quad 
\text{in $\,\RR^n\, \backslash\,  \overline{B_\ep}$, where  $ B_\ep \,:= \bigcup_{m\in \ZZ^n} \ep (B+m)$.} 
\]
Then  inequalities \eqref{dp.mainest1}, \eqref{dp.mainest2}, 
via the $L_2$-unitarity of the above inverse transform $\Gamma_\ep^{-1}U^{-1}$, 
  lead to the following theorem. 
\begin{theorem}
	\label{IK77-2}
	Let $u_\ep$ solve \eqref{dp.differentialPDEdp} and $u_\ep^{(0)}$,  $v_\ep^{(0)}$ be as in \eqref{dp.app1} 
	where $c_{\ep,\t}$, $v_{\ep,\t}$ solve \eqref{zdp}. Then 
	there exist  positive constants $c_0$ and $c_1$ independent of $\ep$  and of $F\in L^2(\mathbb{R}^n)$, such that
	\begin{gather}
	\label{dp.H1est}
	\bigl\Vert u_\ep \,-\,\bigl( u_\ep^{(0)}\,+\,\ep\,\ourN^{\rm pd}\left(\tfrac{\cdot}{\ep}\right)\cdot  \nabla u^{(0)}_\ep\bigr)   \bigr\Vert_{H^1(\RR^n\backslash \overline{B_\ep} )} \,\,\,\le\,\,\, c_0\,\ep\, \Vert F \Vert_{L^2(\mathbb{R}^n)}, \\
	\label{dp.L2est}
	\bigl\Vert u_\ep \,-\,\bigl(u_\ep^{(0)} \,+\, v^{(0)}_\ep\left(\tfrac{\cdot}{\ep}\right)\,\bigr)   \bigr\Vert_{L^2(\mathbb{R}^n)} \,\,\,\le\,\,\, c_1\,\ep\, \Vert F \Vert_{L^2(\mathbb{R}^n)}.
\end{gather}
\end{theorem}
An estimate similar to the $L^2$-estimate \eqref{dp.L2est} (more precisely  to that based on \eqref{final2} with $\langle f, \phi \rangle = \int_\square f \overline{\phi}$) was first derived in \cite{ChCo} by different means. 
The $H^1$-estimate \eqref{dp.H1est} is believed to be new. We remark that, like Proposition \ref{homeqest} re-expresses estimates \eqref{IKH1est} and \eqref{IKL2est} in terms of the solution of the homogenised equation \eqref{homeq}, estimates akin to %both 
\eqref{dp.H1est} and \eqref{dp.L2est} can be re-expressed in terms of appropriate solutions to the two-scale limit problem, cf. the next subsection. 
\subsubsection{Approximation via the two-scale limit operator and an associated two-scale interpolation operator}
%While $u^{(0)}_{\ep} +v^{(0)}_{\ep}$ provide an $L^2(\RR^n)$-approximation  to $u_\ep$, it is meaningful to compare this approximation with the two-scale homogenised limit of $u_\ep$. 
Let us recall, see e.g. \cite{Zhi2005,IVKVPS13},  that for problem \eqref{dp.differentialPDEdp} the following 
property of (strong) two-scale (pseudo-)resolvent convergence is held. 
If $F_\ep\in L^2(\RR^n)$ weakly or strongly two-scale converges 
as $\ep\to 0$ to 
$F_0\in L^2(\RR^n \times \square)$ then $u_\ep$ the solution to \eqref{dp.differentialPDEdp} (for $F= F_\ep$) respectively 
weakly or strongly two-scale converges to 
$u_0(x,y) = u(x) + v(x,y)$  the solution to the two-scale %homogenised 
limit system  $\mathcal{L}_0 u_0= \mathcal{P} F_0$. 
Here $\mathcal{L}_0$ is the self-adjoint  \textit{two-scale operator} in the closed subspace 
$L^2\big(\RR^n ;\, \mathbb{C}\,\dot{+}\,L^2(B)\big)\,=\,L^2(\RR^n) \,\dot{+}\, L^2\big(\RR^n ; L^2(B)\big)$ of 
$L^2\big(\RR^n ; L^2(\square)\big)=L^2(\RR^n\times \square)$
%= L^2(\RR^n; \CC \dot{+}L^2(B))$ 
generated by the \textit{two-scale  form}
\[
\begin{aligned}
Q_0(u+v,\phi+\psi) \,\,=\,\,	\int_{\RR^n} A^{\rm hom}_{\rm pd} \nabla u(x) \cdot \overline{\nabla \phi(x)} \, {\rm d}x \,+\, \int_{\RR^n}\int_B \nabla_y v(x,y) \cdot \overline{\nabla_y \psi(x,y)} \,\, {\rm d}y\,{\rm d}x, 
	%+  \int_{\RR^n}\int_B \big(u(x) + v(x,y) \big)\big( \phi(x) +\psi(x,y) \big) \, {\rm d}y{\rm d}x, \\
\end{aligned}
\] 
for $u,\phi \in H^1(\RR^n), v,\psi \in L^2(\RR^n ; H^1_0(B)),$
with dense form domain $H^1(\RR^n) \,\dot{+}\, L^2\big(\RR^n ; H^1_0(B)\big)$, and  $\mathcal{P}: L^2(\RR^n\times \square) \rightarrow L^2(\RR^n) \,\dot{+}\, L^2\big(\RR^n ; L^2(B)\big)$ is the orthogonal projection or simply
\be
\label{p-cal-dp}
\mathcal{P}g(x,y) =  \left\{ 
\begin{array}{lcr}
	g(x,y) & & x\in \RR^n, y \in B \\
	{|\square \backslash B|^{-1}} \int_{\square \backslash B} g(x,y') \, {\rm d}y'  & &  x\in \RR^n, y \in \square \backslash B.
\end{array}
\right.
\ee
Now we observe that the above objects are precisely those that appeared in Section \ref{s.bivariate} when specialised to the present example. 
Indeed, recall that 
$\mathcal{H}=L^2(\square)$  with inner product $d_\t(u,{\tilde{u}})  = \int_\square u \,\overline{\tilde{u}}$, 
and notice  via \eqref{abcdp} that \eqref{zvbd} holds. 
Further, according to Section \ref{s.bivariate}, 
%  $Z = \CC$, $V_\star = H^1_0(B)$, 
$\mathcal{H}_0 :=\overline{Z\dot{+}V_\star} =  \CC \,\dot{+}\, L^2(B)$, and
\begin{align*}
&\mathbb{H}\,=\,L^2\big(\RR^n ; (\mathcal{H},d_0)\big) = L^2\left(\RR^n \times \square\right), \qquad 
\mathbb{H}_0\,=\,L^2(\RR^n ; \mathcal{H}_0) = L^2(\RR^n) \,\dot{+}\, L^2(\RR^n ; L^2(B)), \\
&\text{and}  \quad
\check{\mathbb{D}}\,=\,H^1(\RR^n;Z) \,\dot{+}\, L^2(\RR^n;V_\star) \,=\, H^1(\RR^n) \,\dot{+}\, L^2\big(\RR^n ; H^1_0(B)\big),
\end{align*}
 all equipped with the standard norms. Therefore, comparing the above two-scale form $Q_0$ and the  bivariate form $Q$ (see \eqref{Q}) and recalling \eqref{dp.ahom} we find that 
 \[
 Q[u+v] \,\,\,=\,\,\, Q_0[u+v] \,\,+\,\, \| u+v \|_{L^2(\RR^n \times \square)}^2 
 \]  and so $\mathcal{L}=\mathcal{L}_0+I $ for  the  abstract bivariate operator $\mathcal{L}$ generated by $Q$ 
as introduced in Section \ref{s.bivariate}. 
Consequently, Theorem \ref{thm.bivariate} via a routine specialisation to the present setting yields the following. 
%substitution specialises in the present setting 
%to the following.
\begin{theorem}
	\label{thm.bivariatehc} 
For $0<\ep <1$, one has
	\[
	\Big\Vert\mathcal{L}_{\ep,\t}^{-1} g(\t) \,-\, \Big(A_\ep^* (\mathcal{L}_0+I)^{-1}\mathcal{P} A_\ep g\Big)(\t)\Big\Vert_{L^2(\square)}  \,\,\le\,\, 
	C_{11}\,\ep\,
	  \big\Vert g(\t)\big\Vert_{L^2(\square)} \, ,	\quad \forall g \in L^2(\square^* \times \square), \quad a.e.\ \t \in \square^*,
	\]
	where  $A_\ep : L^2(\square^* \times \square) \rightarrow L^2(\RR^n \times \square) $, $A_\ep=\Gamma_\ep^{-1}\mathcal{F}^{-1} \, \chi\,\mathcal{E}^{-1}$,   
	and its adjoint 
	$A_\ep^* : L^2(\RR^n \times \square ) \rightarrow L^2(\square^*\times\square )$, $A_\ep^*=\mathcal{E}\, \chi^*\, \Gamma_\ep^{-1} \mathcal{F}$, 
 are given by the continuous extensions of 
\begin{equation}
\label{Aep}
 A_\ep g(x,y) 
% = (2\pi)^{-n/2} \int_{\RR^n}   \chi(\ep \t) e^{\i \ep \t \cdot y}f(\ep \t,y) e^{\i \t \cdot x}\, {\rm d} \t  = (2\pi)^{-n/2} \int_{\ep \square^*}  e^{\i \ep \t \cdot y}f(\ep \t,y) e^{\i \t \cdot x}\, {\rm d} \t \]
 \,\,=\,\, (2\pi)^{-n/2} \ep^{-\,n/2} \int_{\square^*}  e^{\i\, \t \cdot y}g(\t,y) 
e^{\i \,\t\, \cdot\, \frac{x}{\ep}}\, {\rm d} \t, \quad x\in \mathbb{R}^n, \ y\in\square; 
 \end{equation}
 \begin{equation}
\label{Aepstar}
 A_\ep^* h(\t,y) \,\,=\,\,(2\pi)^{-\,n/2} \ep^{-\,n/2} e^{-\i\, \t \cdot y}\int_{\RR^n} h(x,y) 
e^{-\i \,\frac{\t}{\ep}\, \cdot\, x} \, {\rm d}x, \quad \t\in \square^*, \ y\in\square. 
 \end{equation} 
 
%	Furthermore, the following identities hold:
%	\[
%	A_\ep B_\ep  = I \quad \text{and } \quad B_\ep A_\ep = \mathcal{F}^{-1} \Gamma_\ep \chi \Gamma_\ep^{-1} \mathcal{F}.
%	\]
%	%\[
%	%(\mathcal{T}_\ep f) (x) = (2\pi)^{-n/2} \int_{} (\mathcal{E}_{ \t}^* f(\t)) e^{\i \t \cdot x} \, {\rm d}\t.
%	%\]
\end{theorem}
\vspace{.12in}

The above theorem can be re-stated in terms of an operator-type estimate for the original problem 
\eqref{dp.differentialPDEdp} as follows. Let 
$\mathcal{L}_\ep\,=\,-\,{\rm div}\bigl( A_\ep \left(\tfrac{x}{\ep} \right) \nabla \cdot \bigr) $ be the 
non-negative self-adjoint operator defined in a standard way in Hilbert space $L^2\left(\mathbb{R}^n\right)$. 
Then, for the solution of  \eqref{dp.differentialPDEdp}, $u_\ep=\left(\mathcal{L}_\ep+I\right)^{-1}F$. 
On the other hand, denoting $g=U\Gamma_\ep F$,  we observe via \eqref{fdp} and \eqref{ik3} that 
$u_{\ep,\t}=\mathcal{L}^{-1}_{\ep,\t}g$ for which in turn $u_{\ep,\t}=U\Gamma_\ep u_\ep$. 
Combining this all implies 
%Finally, upon recalling that
%for $(\mathcal{L}_\ep+I)^{-1}$ be the self-adjoint operator in $L^2(\RR^n)$  generated by mapping $f$ to  the solution $u_\ep$ to \eqref{dp.differentialPDEdp},
% we note that
$
\mathcal{L}_{\ep,\t}^{-1} =  U \Gamma_\ep (\mathcal{L}_\ep+I)^{-1}  \Gamma_\ep^{-1} U^{-1},
$
which due to the $L_2$-unitarity of $U$ and $\Gamma_\ep$ allows to recast  
%Consequently,  
Theorem \ref{thm.bivariatehc} in the following form.
\begin{theorem}\label{thm.2scOpRes}
	For $0<\ep<1$ one has 
	\begin{equation}
		\label{dpcompe3}
		%\label{dpcompe1}
		\bigl\Vert \left(\mathcal{L}_\ep+I\right)^{-1} \,-\,  
		\mathcal{I}_\ep^* \left(\mathcal{L}_0+I\right)^{-1} \mathcal{P} \mathcal{I}_\ep \bigr\Vert_{L^2(\RR^n) \rightarrow L^2(\RR^n)} \,\,\le\,\, C_{11}\, \ep.
	\end{equation}
	Here  $\mathcal{I}_\ep: L^2(\RR^n) \rightarrow L^2(\RR^n \times \square)$, which we call ``two-scale interpolation operator'' (see Remark \ref{RemShann} below),
	is a bounded operator 
	given by the composition 
	\be
	\label{2ScInterp}
	\mathcal{I}_\ep:\,\,=\,\,A_\ep\, U\, \Gamma_\ep\,\,=\,\,\Gamma_\ep^{-1}\mathcal{F}^{-1} \, \chi\,\mathcal{E}^{-1}\, U\, \Gamma_\ep\,,
	\ee
	which is 
	an $L^2$-isometry and the continuous extension of 
	\begin{equation}
	\label{7.54-1}
	\mathcal{I}_\ep F(x,y) \,\,=\,\,  \sum_{m\in \ZZ^n} F\big(\ep y + \ep m\big)\,\,  
	{\rm Sinc}\left( \frac{x}{\ep}  - m\right) \,; 
	\end{equation}
$\mathcal{I}_\ep^*: L^2(\RR^n \times \square) \rightarrow L^2(\RR^n) $ is the adjoint of $\mathcal{I}_\ep$ given by   
$\mathcal{I}_\ep^*=\Gamma_\ep^{-1} U^{-1} A_\ep^*=\Gamma_\ep^{-1} U^{-1}\mathcal{E}\, \chi^*\, \mathcal{F}\Gamma_\ep $, which is the continuous extension of 
\begin{equation}
\label{7.54-2}
\mathcal{I}_\ep^* u_0(x) \,\,=\,\,  \ep^{-n}\, \int_{\RR^n} 
u_0\left(s, \left\{ \frac{x}{\ep} \right\}\right) 
\,\,  
	{\rm Sinc}\left( \left[\frac{x}{\ep}\right]\,-\, \frac{s}{\ep}\right)
\,{\rm d}s\,, 
\end{equation}	
where $\{ p \}$ %=p-m\in\square$, $m\in\mathbb{Z}^n$, 
is the fractional part of $p\in\mathbb{R}^n$, and $[p]:=p-\{p\}$ is its ``entire part''. 
In \eqref{7.54-1} and \eqref{7.54-2} ${\rm Sinc}(z)$, $z\in\mathbb{R}^n$,  is the ($n$-dimensional normalised) sinc-function: 
\[
{\rm Sinc}(z)\,\,:=\,\,\prod_{j=1}^n\,\frac{\sin\left(\pi z_j\right)}{\pi z_j}, \ \ \ z\in \mathbb{R}^n. 
\]
%Moreover,
The range of $\mathcal{I}_\ep$ consists of all functions $f(x,y)\in L^2(\RR^n \times \square)$ whose Fourier transform in $x$ is supported in 
$[-\pi/\ep,\pi/\ep]^n$ for a.e. $y\in\Box$. Moreover, 
\begin{equation}
\label{calttstar}
\mathcal{I}_\ep^* \mathcal{I}_\ep \,=\, I, \quad \text{and} \quad \mathcal{I}_\ep \mathcal{I}_\ep^* \,=\, \mathcal{S}_{\ep},
\end{equation} 
{ where $\mathcal{S}_\ep$ is the smoothing operator as given by \eqref{7.22-2} (with $\chi$ replaced by the characteristic function of ${\square^*}$) 
applied to the first variable. } 
\end{theorem}
\begin{proof}
Operator 
$\mathcal{I}_\ep=\Gamma_\ep^{-1}\mathcal{F}^{-1} \, \chi\,\mathcal{E}^{-1}\, U\, \Gamma_\ep$ is an $L^2$-isometry from  
$L^2(\RR^n)$ to $L^2(\RR^n \times \square)$, as a superposition of $L^2$-norm preserving operators. 
Therefore, at a dense subspace, e.g. $C_0^\infty(\mathbb{R}^n)\ni F$, combining \eqref{Aep} with \eqref{gt1} and \eqref{gammaep} we obtain 
\[
\mathcal{I}_\ep F(x,y)\,:=\,A_\ep\, U\, \Gamma_\ep F(x,y)\,=\,\, (2\pi )^{-\,n}   \sum_{m\in \ZZ^n} F(\ep y + \ep m)  
	\int_{\square^*}   e^{\i\,  \t\, \cdot\, \left( \frac{x}{\ep}  - m\right)}\, {\rm d} \t, 
\]
which yields \eqref{7.54-1}. 
Similarly, combining \eqref{Aepstar} with \eqref{gt2} and \eqref{gammaep} gives 
\[
\mathcal{I}_\ep^* u_0(x) \,=\, \Gamma_\ep^{-1} U^{-1} A_\ep^*u_0(x)\,=\,  (2\pi\ep)^{-\,n} \int_{\RR^n} 
u_0\left(s, \left\{ \tfrac{x}{\ep} \right\}\right) \left( 
\int_{\square^*}e^{\i\,\t\,\cdot\,\left(\left[\frac{x}{\ep}\right]- \frac{s}{\ep}\right)} \,  {\rm d}\t \right){\rm d}s, 
\]
yielding \eqref{7.54-2}. 
Finally, \eqref{calttstar} immediately follows via \eqref{abident} and \eqref{7.22-2}. 
\end{proof}
\begin{remark}
\label{RemShann}
Operator $\mathcal{I}_\ep$ plays a key role of $L^2$-isometrically 
transferring, for any $\ep>0$, an input function $F(x)$ from $L^2(\mathbb{R}^n)$ into corresponding two-scale 
function $\mathcal{I}_\ep F(x,y)$ in $L^2(\RR^n \times \square)$. 
The latter serves in turn as the input for the two-scale limit problem, whose solution $u_0(x,y)$ is converted by the adjoint 
$\mathcal{I}_\ep^*$ back into a function of $x$. 
The whole point is that such a procedure delivers an approximate self-adjoint solution operator, which is the inverse of the two-scale limit operator preceded by the 
projection operator $\mathcal{P}$ and flanked by the transfer operator $\mathcal{I}_\ep$ and its adjoint, 
delivering the operator-normed error estimate \eqref{dpcompe3}. 
Interestingly, \eqref{7.54-1} appears to be a two-scale version of the Whittaker–Shannon interpolation formula, see e.g. \cite{Higgins} for a %detailed 
review. 
In this respect, operator $\mathcal{I}_\ep$ can be viewed as a new two-scale interpolation operator. 
Indeed for regular enough $F$, given $y\in\Box$, if $x=\ep l$ for $l\in \mathbb{Z}^n$ then \eqref{7.54-1} implies $\mathcal{I}_\ep F(x,y)=F(x+\ep y)$, interpolating for 
other $x\in\mathbb{R}^n$ between the values of $F$ on the $\ep$-periodic lattice containing $\ep y$ (i.e. on the lattice 
$\ep\mathbb{Z}^n+\ep y$ of all the points with the chosen ``phase'' $y$). 
In particular, the following 
can be derived directly from the above definition of $\mathcal{I}_\ep$ and is also immediately implied by the classical 
Whittaker-Kotelnikov-Nyquist-Shannon sampling theorem (see e.g. \cite{Higgins}). -- If the right hand side $F$ is itself a two-scale function, i.e. 
$F_\ep(x)=\Phi(x,x/\ep)$ where $\Phi(x,y)$
is sufficiently regular, $\square$-periodic in $y$ and its Fourier transform in $x$ is uniformly for a.e. $y$ compactly supported in an origin-centred cube $Q$ of size $2R$, 
i.e. $Q=[-R,R]^n$, then for all 
%there exists $\ep_0>0$ 
 $0<\ep<\pi R^{-1}$, $\big(\mathcal{I}_\ep F_\ep\big)(x,y)=\Phi(x+\ep y, y)$.  
On the other hand it is easy to see from \eqref{2ScInterp} that, for any $F\in L^2\left(\RR^n\right)$, $\left(\mathcal{I}_\ep F\right)(x,y)$ 
automatically has the above property of uniformly compact support of the $x$-Fourier transforms with $R=\pi/\ep$, 
which property is inherited by the solution $u_0(x,y)$ of the two-scale limit problem. 
%It is easy to see that the above property of uniformly compact support of Fourier transforms of the right hand side of the limit problem is 
%inherited by its solution $u_0(x,y)$. 
For such $u_0$ it follows in turn from \eqref{7.54-2} that 
$\mathcal{I}_\ep^* u_0(x)=u_0\left(x-\ep y, y\right)$ where $y:=\{x/\ep\}$. 
\end{remark}
\begin{remark} 
\label{ShannVsUnf}
It appears that, in contrast to the classical homogenisation (Example \ref{e.class} above), an interpolation operator is necessary for recasting the input $F$ as 
a two-scale function (to serve in turn as the input for the two-scale limit problem). 
In this respect, 
it would be interesting to compare the new interpolation operator $\mathcal{I}_\ep$ with the periodic unfolding operator, see e.g. \cite{CDG}. 
The former appears to resemble the latter in certain respects, although to 
differ in one key aspect which may be particularly significant when the limit operator remains intrinsically two-scale like in the present high-contrast problem. 
Such problems, in contrast to the intrinsically ``low-frequency'' classical homogenisation, require controlling higher frequencies, and the interpolation operator 
$\mathcal{I}_\ep$ maybe a more suitable tool than the unfolding precisely for the latter. 
Indeed, denoting by $\mathcal{U}_\ep: L^2(\RR^n) \rightarrow L^2(\RR^n \times \square)$ the $L^2$-isometric unfolding operator, for sufficiently regular $F$ we have 
$\left(\mathcal{U}_\ep F\right)(x,y):=F\big(\ep\,[x/\ep]\,+\,\ep y\big)$, see \cite{CDG}. This means that, for regular enough $F$, both $\mathcal{U}_\ep$ and 
$\mathcal{I}_\ep$ produce on the same $\ep$-periodic lattice 
$\ep\, \mathbb{Z}^n$ exactly the same values $F(x+\ep y)$, however interpolate between those in different ways. 
Namely, while $\mathcal{U}_\ep$ simply extends the latter value for the whole of the related $\ep$-cell $x\in \ep l+\ep\square$ in piecewise constant way, 
$\mathcal{I}_\ep$ 
smoothly interpolates between the above points according to \eqref{7.54-1}. 
The former may introduce spurious higher frequency contributions due to the discontinuities between the piecewise constant values in $x$, while the latter 
interpolates so that the resulting function is smooth in $x$. 
The key point is that it is the above new two-scale Whittaker 
interpolation operator $\mathcal{I}_\ep$ which achieves a desired approximation with a strong operator norm error estimate \eqref{dpcompe3}, uniformly valid in $\ep$ and $F$. 
\end{remark}
\begin{remark}
One potential disadvantage of the above constructed operator $\mathcal{I}_\ep$, suffered in fact also by $\mathcal{U}_\ep$,  is that for a given $x$ e.g.  
$x\in \ep\,\mathbb{Z}^n$, even for smooth $F(x)$ both produce a discontinuity in $y$  in the values of 
$(\mathcal{I}_\ep F)(x,y)=F(x+\ep y)$ on the ``opposite'' sides of the periodicity cell $\square$. We notice however that this can be eliminated by a slight modification in the choice of the operator $\mathcal{E}_\t$. 
Namely, keeping it as the multiplication by $e^{-\i\, \t \cdot y}$ for $y\in B$ i.e. in the inclusion, set it to be identity (i.e. a multiplication by unity) in 
the surrounding `matrix' $\square\backslash B$. 
One can then easily check that all the assumptions in Lemma \ref{propeth} and hypothesis \eqref{H6} remain valid. 
Therefore a minor modification of Theorem \ref{thm.2scOpRes} holds, where $\mathcal{I}_\ep F$ is still given by \eqref{7.54-1} for $y\in B$, however  with $y$ additionally subtracted in the argument of 
$\mbox{Sinc}$ for $y\in \square\backslash B$: 
\[
\mathcal{I}_\ep F(x,y) \,\,=\,\,  \sum_{m\in \ZZ^n} F\big(\ep y + \ep m\big)\,\,  
	{\rm Sinc}\left(\, \frac{x}{\ep}  \,-\, m\,-\,y\right), \ \ \ y\in \square\backslash B. 
\]
As a result, given a regular $F$, while for $y$ in the inclusion $\left(\mathcal{I}_\ep F\right)(x,y)$ remains unchanged, for $y$ in the matrix and 
for $x\in \ep\,\mathbb{Z}^n+\ep y$ it is simply 
$F(x)$ with Whittaker interpolation in between. 
Since the isolated inclusions do not intersect the boundary of the periodicity cell $\square$, no discontinuities are anymore introduced on artificial 
boundaries like that of $\square$. 
Notice however that, for both choices of $\mathcal{I}_\ep$, according to \eqref{dpcompe3} $\mathcal{I}_\ep F$ is immediately followed by the projection 
operator $\mathcal{P}$, see \eqref{p-cal-dp}. 
One can see that, for the modified choice of $\mathcal{I}_\ep$ the approximation delivered as a result by \eqref{dpcompe3} has a particularly simple form. 
Namely, given any $F\in L^2\left(\RR^n\right)$, with related $u_0= \left(\mathcal{L}_0+I\right)^{-1} \mathcal{P} \mathcal{I}_\ep F$ one has 
$u_0(x,y)=u(x)+v(x,y)$ and $\mathcal{I}_\ep^*u_0(x)=u(x)$ in the matrix phase $\RR\backslash B_\ep$ while 
$\mathcal{I}_\ep^*u_0(x)=u_0\big(x-\ep\{x/\ep\}, \,\ep\{x/\ep\}\big)$ in the inclusion phase $B_\ep= \bigcup_{m\in \ZZ^n} \ep (B+m)$. 
\end{remark}

\subsubsection{Estimates on the rate of convergence of the spectrum} 
Finally, the following important result on the approximation of spectra holds by adopting Theorem \ref{bivariate.spec} to the present example. 
%orollary \ref{c.collspec} and Theorem \ref{thm.limspecrep} together give the following result.
\begin{theorem}
\label{EVestDP}
For every real $b$ there exists a non-negative constant $C_b$ such that 
	for every interval $[a,b] \subset (-\infty,\infty)$ one has 
	\begin{equation}
	\label{specest}
	d_{[a,b]} \Big( {\rm Sp}\,  \mathcal{L}_\ep,{\rm Sp}\, \mathcal{L}_0\Big) \,\,\,\le \,\,\,C_b\,\ep,  \ \ \ \ \forall \ \ 0<\ep<1. 
	\end{equation}
	Further, for the spectrum of the above two-scale limit operator $\mathcal{L}_0$, 
	\[
	{\rm Sp}\, \mathcal{L}_0 \,\,=\,\, \Big\{ \lambda \notin {\rm Sp}\, \left(-\Delta_{H^1_0(B)}\right) \,:\, \beta_{\rm B}(\lambda) \,\ge\, 0  \Big\} \,\cup\, 
	{\rm Sp}\,\left(-\Delta_{H^1_0(B)}\right).
	\]
Here $-\Delta_{H^1_0(B)}$ is the Dirichlet Laplacian on the inclusion $B$ and 
$
\beta_{\rm B}(\lambda) = \lambda + \lambda^2 \int_B \Big( -\Delta_{H^1_0(B)} \,-\, \lambda\,\Big)^{-1} \mathbf{e} 
$ 
is the $\beta$-function associated with $B$ which was probably first introduced by Zhikov, see e.g. \cite{Zhi2000,Zhi2005}. 
	In particular, when $(a,b)$ is a gap in  ${\rm Sp}\, \mathcal{L}_0$  then $[a +\,C_b\, \ep,\, b\,-\,C_b\,\ep]$ is in a gap of  $\,{\rm Sp}\, \mathcal{L}_\ep$ when $\ep \,<\, (b-a)/(2C_b)$.
\end{theorem}
The proof immediately follows from Theorem \ref{bivariate.spec} upon noting the following specialisations for the present example: 
\[
{\rm Sp}\, \mathcal{L}_\ep  \,=\, \overline{\bigcup_{\theta \in \Theta} {\rm Sp} \, \mathcal{L}_{\ep,\t}}\,, \qquad \mathbf{B} \,=\, -\,\Delta_{H^1_0(B)}+ I, \qquad \text{and} \quad \beta_\lambda[\mathbf{e}] \, = \,\beta_{\rm B}(\lambda - 1),
\]
see \eqref{betaform}--\eqref{6.27-2}. 
Notice further that \eqref{iksign} specialises to 
\[
\beta_B(\lambda)\,\,=\,\,\lambda\,\,+\,\,\lambda^2\, \sum_{m=1}^\infty\,\frac{\left\vert\int_B\phi_m(y){\rm d} y\right\vert^2}{\lambda_m\,-\,\lambda}\, ,
\]
where $\lambda_m$ and $\phi_m$, $m=1,2,...$, are respectively all the simple eigenvalues and the $L^2$-orthonormalised eigenfunctions of  
Dirichlet Laplacian $-\Delta_{H^1_0(B)}$ on the inclusion $B$. 
This implies, cf. \cite{Zhi2000,Zhi2005}, that the spectrum of the limit operator $\mathcal{L}_0$ typically has infinitely many gaps. 
It was shown in \cite{Zhi2005} that the Floquet-Bloch spectrum of the original 
operator $\mathcal{L}_\ep$ converges to that of $\mathcal{L}_0$ in the sense of Hausdorff. 
The estimate \eqref{specest} provides a new result on the rate of the convergence of the spectrum. 
We remark that it, as well as \eqref{dpcompe3}, appears to improve recent results of \cite{ChErKi,ChKiVeZu23}: while the latter references have similar 
estimates in terms a certain $\ep$-dependent approximate operator $\mathcal{L}_\ep^{app}$, the above approach of ours allows to construct an 
approximation with the desired error estimates in terms the $\ep$-independent two-scale limit operator $\mathcal{L}_0$.  

\begin{remark} 
\label{EFestDP}
The general spectral results of Section \ref{s:resolv} imply also error estimates for convergence of corresponding eigenfunctions, and in the present example of 
the Bloch waves. 
Not attempting here any detailed investigation of this, we remark that Theorems \ref{splimSe.3} and \ref{ikthm2} in combination with general methods found e.g. 
in \cite{VishLus} imply the following. Let $\xi\in\mathbb{R}^n$ and $k\in\mathbb{N}$ and let $\lambda^{(k)}_\xi$ be (for simplicity) a simple eigenvalue of 
$\mathbb{L}_\xi$ with associated eigenfunction $\psi_\xi^{(k)}=z_\xi^{(k)}+v_\xi^{(k)}$ where $z_\xi^{(k)}\in\CC$ and $v_\xi^{(k)}\in H^1_0(B)$. 
Then there exists $0<\ep_0\le 1$ and $\delta>0$ such that for all $0<\ep<\ep_0$ and for $\t=\ep\xi\in\square^*$ there exists a single isolated 
eigenvalue $\lambda^{(k)}_{\ep,\t}$ 
of $\mathcal{L}_{\ep,\t}$ in the $\delta$-neighbourhood of $\lambda^{(k)}_\xi$ and associated eigenfunction $\varphi^{(k)}_{\ep,\t}\in H^1_{per}(\square)$ such that 
\be
\label{eigfunest}
\left\|\,\varphi^{(k)}_{\ep,\t}\,\,-\,\,\psi_\xi^{(k)}\,\right\|_{L^2(\square)}\,\,\,\le\,\,\,C\,\ep,
\end{equation}
with a constant $C>0$ independent of $\ep$. 
A converse property also holds: for an appropriate sequence of eigenvalues $\lambda^{(k)}_{\ep,\t}$  it necessarily converges as $\ep\to 0$ to $\lambda^{(k)}_\xi$ 
with associated eigenfunctions, up to scalar pre-factors, obeying \eqref{eigfunest}. 
As $\,e^{\i\t\cdot y}\varphi^{(k)}_{\ep,\t}(y)$ is a $\t$-quasiperiodic Bloch wave associated with the original (rescaled) operator, 
\eqref{eigfunest} implies approximation of the latter by $\,e^{\i\t\cdot y}\psi_{\t/\ep}^{(k)}(y)$, where $\psi_{\t/\ep}^{(k)}$ is explicitly found from the two-scale limit 
problem. 
More detailed analysis, in particular of any uniformity properties of \eqref{eigfunest} with respect to $\xi$ and $k$, may deserve a separate investigation. 
\end{remark} 

%{\color{blue}
% Let
%$A_\ep$  be a sequence of non-negative self-adjoint operators  in $L^2(\RR^n)$, $A$ a self-adjoint operator in $H$ a closed subspace of $L^2(\RR^n\times \square)$ and let $P:L^2(\RR^n\times\square) \rightarrow H$ be the orthogonal projection.  
%
%We recall the following notion of strong two-scale resolvent convergence introduced by Zhikov:
%\begin{flalign*}
%&\text{$A_\ep$ strong two-scale resolvent converges to $A$ if, and only if, } \\
%& (A_\ep+I)^{-1} f_\ep\ \text{strongly two-scale converges to } (A+I)^{-1} Pf_0\ \text{when }  f_\ep \text{ strongly two-scale coverges to } f_0.
%\end{flalign*}
%Now, 
%%we observe that  two-scale convergence in $L^2(\RR^n)$  is equivalent to convergence in $L^2(\RR^n\times\square)$ under the action of $\mathcal{T}_\ep$.  More precisely,
% it is a straight-forward task to show that a bounded sequence $u_\ep \in L^2(\RR^n)$ weakly (strongly) two-scale converges to $u_0 \in L^2(\RR^n \times \square)$ if, and only if, $\mathcal{T}_\ep u_\ep$ weakly (strongly) converges to $u_0$ in $L^2(\RR^n \times \square)$. 
%
% Therefore, strong two-scale convergence can be equivalently reformalated as:
% \begin{flalign*}
% 	&\text{$A_\ep$ strong two-scale resolvent converges to $A$ if, and only if, } \\
% 	& \mathcal{T}_\ep (A_\ep+I)^{-1} \mathcal{T}_\ep^* g_\ep \longrightarrow (A+I)^{-1} Pg_0\ \text{in $L^2(\RR^n \times \square)$ when } g_\ep \rightarrow g_0\  \text{in $L^2(\RR^n \times \square)$}.
% \end{flalign*}
%This equivalence allows for a natural notion of norm-resolvent two-scale convergence. 
%\begin{definition}
%
%$A_\ep$ is said to  norm-resolvent two-scale converge to $A_0$ if, and only if,
%\[
%\lim_{\ep \rightarrow 0} \| \mathcal{T}_\ep (A_\ep +I)^{-1} \mathcal{T}_\ep^* - (A_0+I)^{-1} P \|_{L^2(\RR^n \times \square)} = 0.
%\]
%\end{definition}
%From  Theorem \ref{thm.2scOpRes} we can see  that $\mathcal{L}_\ep$ norm-resolvent two-scale converges to $\mathcal{L}_0$. Indeed, as $\mathcal{T}_\ep$ is an isometry and $\mathcal{T}_\ep \mathcal{T}^*_\ep = \mathcal{S}_\ep$ is a projection then
%\begin{flalign*}
%\| \mathcal{T}_\ep  (\mathcal{L}_\ep+I)^{-1} \mathcal{T}_\ep^*g -  \mathcal{S}_\ep  (\mathcal{L}_0+I)^{-1} \mathcal{P} \mathcal{S}_\ep g \|_{L^2(\RR^n) \rightarrow L^2(\RR^n)}  &= \|  (\mathcal{L}_\ep+I)^{-1} \mathcal{T}_\ep^*g-    \mathcal{T}_\ep^*(\mathcal{L}_0+I)^{-1} \mathcal{P} \mathcal{S}_\ep g \|_{L^2(\RR^n) \rightarrow L^2(\RR^n)}  \\
%%&= \| \mathcal{T}_\ep  (\mathcal{L}_\ep+I)^{-1} -   (\mathcal{L}_0+I)^{-1} \mathcal{P} \|_{L^2(\RR^n) \rightarrow L^2(\RR^n)} 
%& \le C_{10} \ep \| \mathcal{T}_\ep^* g \|_{L^2(\RR^n)} \le C_{10} \ep \| g \|_{L^2(\RR^n \times \square)}.
%\end{flalign*}
%}
\subsection{`Inverted'  high-contrast model}
\label{e:idp}
Here we provide an example for which the form $a_\t$ has $\t$-regular associated spaces $V_\theta$, that is the assumption \eqref{bddspec} of Theorem  \ref{thm:contV} of Section \ref{s:uniforma} holds. A simple example of such a case  is the `inverted high-contrast' problem: the case where the roles of the isolated inclusions and connected matrix sets $B$ and $\square \backslash B$ respectively in Example \ref{e.dp} above are switched. 
As we will see, this results in an approximation of the original problem, with error bounds, in terms of a limit problem which in contrast to the previous two examples is not anymore homogenised or two-scale but is instead with an ``infinite'' contrast inclusions
 (e.g. with rigid inclusions in case of linear elasticity). 
We will consider here slightly less general elliptic systems\footnote{This could be routinely extended to the case of linear elasticity for example, by  appropriate modifications in the ellipticity conditions \eqref{ivass} and in the related extension operator, cf. Example \ref{e.pdelast} below.},  i.e. we set 
$H = H^1_{per}\left(\square;\, \CC^m\right)$, $m\ge 1$, $\Theta = \square^\star : = [-\pi,\pi]^n$, $n \ge 1$, 
$$
\begin{aligned}
a_\t(u,\tilde u) = \int_{B} a^{(0)} \big(\nabla u +\i\t \otimes u \big) : \overline{\big(\nabla \tilde u+\i\t\otimes\tilde u\big) }, & \ & 
b_\t(u,\tilde u) =  \int_{\square \backslash B} a^{(1)} \big(\nabla u+\i\t\otimes u\big) : \overline{\big(\nabla\tilde u+\i\t\otimes\tilde u\big)} +  
\int_\square u\cdot\overline{\tilde u}. 
%, \quad u \in H^1(\square;\CC^N),
\end{aligned}
$$
Here    $a^{(0)}$ and $a^{(1)}$ are Hermitian $\square$-periodic tensor-valued bounded coefficients that satisfy
\begin{equation}\label{ivass}
\begin{aligned}
\gamma_0^{-1} \int_{\square \backslash B} | \nabla \phi|^2 & \,\,\le\,\, \int_{\square \backslash B}  a^{(1)} \nabla \phi : \overline{\nabla \phi} \,\, &\le&\,\,
 \,\,\gamma_0 \int_{\square \backslash B} | \nabla \phi|^2,  \quad &\phi& \in H^1(\square \backslash B;\, \CC^m), \\
\gamma_0^{-1} \int_{ B} | \nabla \phi|^2 & \,\,\le\,\, \int_{B}  a^{(0)} \nabla \phi : \overline{\nabla \phi}  \,\,&\le&\,\,\,\, \gamma_0 \int_{B} | \nabla \phi|^2, \quad &\phi& \in H^1(B;\, \CC^m),
\end{aligned}
\end{equation}
for some constant $\gamma_0\ge 1$. 
In this setting, for each $\theta \in\square^\star$ and $u\in H$, one has
\[
a_\t[u] \,\,=\,\, \int_B a^{(0)} \nabla \big( e^{\i \t \cdot y} u\big) : \overline{\nabla \big( e^{\i \t \cdot y} u\big) } \,\,\ge\,\, 
\gamma_0^{-1} \int_B \big\vert \nabla \left( e^{\i \t \cdot y} u \right) \big\vert^2
\]
whence, assuming for simplicity $B$ connected, 
\[
V_\theta \,=\, \Big\{ v \in H^1_{per}\big(\square;\,\CC^m\big) \, \,\Big\vert \, \, \text{$v(y) = e^{-i\t\cdot y} c$, $\,y \in B$, for some constant $c \in \CC^m$} \Big\},
\]
and $W_\t$ is the orthogonal complement of $V_\t$ in $H=H^1_{per} (\square ;\, \CC^m)$ with respect to the inner product $a_\t + b_\t$. 
Let us now show that \eqref{KA} holds uniformly on $\square^\star$, i.e. condition \eqref{bddspec} is satisfied. 
\begin{proposition}
	\label{vcontinverdp} There exists a constant $\nu >0$ independent of $\t\in\Theta=\square^\star$ such that
\[
\nu \big( a_\t[w] \,+\, b_\t[w]\big)\,\,\,\le\,\,\, a_\t[w], \qquad \forall w \in W_\t, \ \ \ \forall\t\in\Theta. 
\]	
\end{proposition}
\begin{proof}
Let $E:H^1(B) \rightarrow H_0^1(\square)$ be a Sobolev extension, cf. Proposition \ref{prp.zhiext}, 
and for any fixed $u \in H^1_{per}(\square;\,\CC^m)$ and $\t \in \square^*$, consider 
$v = u - e^{-\i\t\cdot y} E \big( e^{\i\t\cdot y} u - |B|^{-1}\int_B e^{\i\t\cdot y} u(y) \, {\rm d}y\big)$ 
with the extension $E$ acting component-wise. Clearly $v \in V_\t$ and we readily estimate 
%\begin{flalign*}
\[
\|u\,-\,v\|_\t^2\,\,=\,\,
a_\t[u - v ]+b_\t[u-v] \,\,\le\,\, \gamma_0 \left\|  E \left( e^{\i\t\cdot y} u - |B|^{-1}\int_B e^{\i\t\cdot y} u(y) \, {\rm d}y\right) \right\|_{H^1(\square)}^2  \,\,\,\le 
\]
\[
 \gamma_0 C_E^2 \left\|   e^{\i\t\cdot y} u - |B|^{-1}\int_B e^{\i\t\cdot y} u(y) \, {\rm d}y \right\|_{H^1(B)}^2\,\,
\le\,\, \gamma_0\, C_E^2\, C_B^2 \Big\|\,  \nabla \left(  e^{\i\t\cdot y} u\right)\, \Big\|_{L^2(B)}^2 \,\,\le\,\, \gamma_0^2\,\, C_E^2\,\, C_B^2\,\, a_\t[u], 
\]
%\end{flalign*}
where $C_E$ and $C_B$, respectively, are the $H^1$-operator norm of $E$ and the  Poincar\'{e}-Wirtinger  (Poincar\'{e} inequality with mean) constant for domain $B$. Hence,
 for $u=w\in W_\t$, $\|w\|_\t^2\,\le\, \|w-v\|_\t^2\,\le\, \gamma_0^2\, C_E^2\, C_B^2\, a_\t[w]$, and 
 the desired inequality holds with $\nu = \big(\gamma_0\, C_E\, C_B\big)^{-2}$.
\end{proof}
Consequently, for these class of problems, the main approximation %asymptotic analysis 
result is given by Theorem \ref{thm:contV}. 
Employing like in the previous examples the scaling and Gelfand transforms, 
it implies in the present context the following.
% Indeed, we saw from Proposition \ref{prop:contV} that Proposition \ref{vcontinverdp} implies the form $a$ is uniformly coercive on $[-\pi,\pi]^d$, see Remark \ref{rem.contv}.  Theorem \ref{thm:contV}  implies the following result.
\begin{theorem}
	For fixed $0<\ep <1$ let $u_\ep \in H^1(\RR^n;\,\RR^m)$ solve  the elliptic PDE system
	\begin{equation}
	\label{inv-hc-prob}
	-\,{\rm div}\Big(a_\ep\left(\tfrac{x}{\ep}\right)\nabla u_\ep(x) \Big) \,\,+\,\, u_\ep(x)\,\, =\,\, F(x), \quad x \in \RR^n,
	\end{equation}
where $a_\ep(y) \,=\, \chi_{B}(y)a^{(0)}(y)\, +\, \ep^2\big(1- \chi_B(y)\big) a^{(1)}(y)$, $\chi_B$ the characteristic function of $B$,  
$F \in L^2\left(\RR^n;\,\RR^m\right)$.
% where
%	$$
%	a_\ep(y) = \left\{ \begin{array}{lr}
%	\ep^2 a^{(1)}(y) & y \in Q_1, \\
%	a^{(0)}(y) & y \in Q_0,
%	\end{array}
%	\right.
%	$$ 
%	for $a^{(r)}$ satisfying  \eqref{ivass} (or \eqref{dpcoeffs}).
	% Then, for $u_{\ep,\theta}(\cdot) = \mathcal{U}\mathcal{T}_\ep u_\ep(\theta,\cdot)$ and $v_{\theta} \in V_\theta$ the solution to
	%$$
	%\int_{Q_1} a^{(1)} \nabla v_\theta \cdot \overline{\nabla \phi} + \lambda \int_Q v_\theta \cdot \overline{\phi} = \int_Q  \mathcal{U}\mathcal{T}_\ep f(\theta,\cdot)\cdot \overline{\phi}, \qquad \forall \phi \in V_\theta,
	%$$ 
	% the following inequality
	%$$
	%\Vert u_{\ep,\theta} - v_\theta \Vert_{[H^1(Q)]^n} \le \kappa \ep^4 \Vert  \mathcal{U}\mathcal{T}_\ep f(\theta,\cdot) \Vert_{[L^2(Q)]^n}, 
	%$$
	%holds for some constant $\kappa>0$ that depends on $a^{(0)}$, $a^{(1)}$ and $\lambda$ only.
	Consider $v_{\ep}(\t,\cdot) \in V_\theta$ the (unique) solution %, for a.e. $\t\in\square^\star$, 
	to
	\[
	\int_{\square \backslash B} a^{(1)}(y) \big(\nabla v_{\ep}(\t,y)+\i\t\otimes v_{\ep}(\t,y)\big)  : \overline{\big(\nabla \phi(y)+\i\t\otimes \phi(y)\big)} \, {\rm d}y \,+\,  \int_\square v_{\ep}(\theta,y) \cdot \overline{\phi(y)}  \, {\rm d}y \,=\, \int_\square   U\Gamma_\ep F(\theta,y)\cdot \overline{\phi(y)} \, {\rm d}y, 
	\]
%	or equivalently 	\begin{equation*} \left\{ 
%	\begin{aligned}
%		- (\nabla + \i \t )\cdot  a^{(1)} (\nabla + \i \t\otimes) v_\ep + v_\ep &= U \Gamma_\ep f, \quad  y \in \square \backslash B,\\
%\int_{\partial B} e^{\i \t \cdot y}  a^{(1)} (\nabla + \i \t \otimes) v_\ep \cdot \nu+ \int_B e^{\i \t \cdot y} v_\ep  &= \int_{ B} e^{\i \t \cdot y} U\Gamma_\ep f, 
%	\end{aligned} \right. \qquad \t \in \square^*.
%		\end{equation*}
for all $\phi \in V_\theta$, a.e. $ \t \in \square^\star$. Then, for the approximation $u^{(0)}_\ep:= \Gamma_\ep^{-1} U^{-1} v_\ep$, 
inequality \eqref{errorcontinuouscase2} implies the following: 
\begin{equation}
\label{inv-dp-est}
% \gamma_0^{-1} \int_{F_0^\ep} |	\nabla u_{\ep}|^2 +
 \ep^2 \gamma_0^{-1} \int_{\RR^n}   \big|	\nabla u_{\ep} \,-\, \nabla u^{(0)}_\ep \big|^2  \,\,+\,\, 
\int_{\RR^n} \big|u_\ep \,-\, u^{(0)}_\ep \big|^2 \,\,\, \le\,\,\, \nu^{-2} \ep^4 \int_{\RR^n} |F|^2.
\end{equation}
	Notice that $\nabla u^{(0)}_\ep$ 
	vanishes in the inclusions $B_\ep := \bigcup_{l \in \ZZ^n} \ep (B+l)$ 	
	i.e. has support in $\RR^n \backslash \overline{B_\ep}$, and in particular one has  
	$  \int_{B_\ep}   \left|	\nabla u_{\ep} \right|^2   \,\le\, \gamma_0 \nu^{-2} \ep^2 \int_{\RR^n} |F|^2.$
\end{theorem}
%\subsection{An example from wave propagation in photonic crystal fibres}
%Here, we present a non-trivial example, of a form $a[\cdot]$, where the space $V_\theta$ is continuous with respect to $\theta$. This example appears in the context of time-harmonic wave solutions to the Maxwell system for two-dimensional periodic dielectric materials that occupy the whole space $\RR^3$.
To the authors' knowledge, the above result is not found in previous %existing 
literature. 
%Amongst other things, this result  clearly provides uniform in $\t$ asymptotics in $\ep$ of the spectrum (see  Section \ref{s.spcontV}). 
In dimension $n=1$, the inverted high-porosity model is equivalent to the one-dimensional double-porosity model. The quantitative homogenisation of the  scalar ($m=1$) one-dimensional double-porosity model was studied, by different means,  in the works 
%\cite{ChChCo, ChCo3, }, with $L^2(\RR)$ estimates being obtained by different means in 
\cite{ChChCo, ChKi}. 
\begin{remark}
We remark that, for any fixed $\ep>0$, the above approximation $ u_{\ep}^{(0)}$ appears to be the solution of a ``stiff'' problem associated with 
\eqref{inv-hc-prob}, cf. e.g. \cite{JKO} \S 4.2. 
Namely, if one makes in \eqref{inv-hc-prob} change of variable $y=x/\ep$ and introduces contrast $\delta:=\ep^2$ then it becomes 
$-\,{\rm div}_y\big(a^{(\delta)}\left(y\right)\nabla_y u_\ep(\ep y) \big) \,+\, u_\ep(\ep y)\,\, =\,F(\ep y)$, where 
$a^{(\delta)}(y) \,=\, \delta^{-1}\chi_{B}(y)a^{(0)}(y)\, +\, \big(1- \chi_B(y)\big) a^{(1)}(y)$. 
Now if, for a fixed $\ep$, one takes a ``stiff inclusion'' limit $\delta\to 0$, then one can see that 
$u_\ep^{(0)}(\ep y)$ is the limit of the solution $u_\ep(\ep y)$. 
Therefore \eqref{inv-dp-est} can be viewed as providing %, upon the re-scaling, 
new operator-type estimates for the solution of the %stiff 
problem for large 
contrast $\delta$ in terms of its stiff limit with an infinite contrast as $\delta\to 0$. 
Amongst other things, this also %result  clearly 
provides %uniform in $\t$ asymptotics in $\ep$ 
an approximation with error estimates 
for the Floquet-Bloch spectrum of the high-contrast problem (see  Section \ref{s.spcontV}) in terms of that for the limit stiff problem. 
\end{remark} 

%	\begin{align*}
%- (\nabla + \i \t)\cdot  a^{(1)} (\nabla + \i \t) v_\ep + v_\ep = \mathcal{U} \Gamma_\ep f, \quad  y \in \square \backslash B, \\
%\int_{\partial B} e^{\i \t \cdot y} a^{(1)} (\nabla + \i \t) v_\ep \cdot n = \int_{ B} e^{\i \t \cdot y} \mathcal{U} \Gamma_\ep f
%\end{align*}
\subsection{A problem with 
concentrated perturbations %/ of `strange term' type
}
\label{sec:concpert}
In this section we demonstrate that the parameter $\t$ does not necessarily have to come from the Gelfand  transform only. 
Let $F\in L^2(\mathbb{R}^3)$, $0<\ep<1$, $0\le\delta\le 1/2$, and $B_r(x)$ denote the ball of radius $r$ centred at $x$, with $B_r(0)$ denoted by $B_r$. Consider the problem, in the weak form,
\begin{equation}
\label{IKs}
\left\{ \begin{aligned}
& \text{ Find $U_{\ep,\delta} \in H^1(\mathbb{R}^3)$ the solution to} \\
& \int_{\mathbb{R}^3} \nabla U_{\ep,\delta}\cdot \overline{\nabla\phi} \,\,+\,\,\ep^{-2}\delta^{-1}\sum_{j\in \mathbb{Z}^3} \int_{B_{\delta\ep}(\ep j)}U_{\ep,\delta}\,\overline{ \phi}\,\,+\,\int_{\mathbb{R}^3}  U_{\ep,\delta}\, \overline{\phi} \,\,\, =\,\, \int_{\mathbb{R}^3} F\, \overline{\phi} , \qquad \forall \phi \in H^1(\mathbb{R}^3),
\end{aligned} \right.
\end{equation}
where  we make a convention that if $\delta=0$ the singular term (i.e. the second term on the left hand side of \eqref{IKs}) is absent. 
Related problems with different scalings for ``concentrated perturbations'' %with respect to $\delta$ 
where considered, for example,  in \cite{GoNa92,Na93}.

Our aim is to construct, for small $\ep$, approximations to the solution $U_{\ep,\delta}$ which would be uniform in $\delta$. 
The idea here is to reduce problem \eqref{IKs} to the general form \eqref{p1} by regarding $\delta$ as another component in the abstract parameter $\t$. 
Namely, let 
$\theta=(k,\delta)%=\big(\xi_1,\xi_2,\xi_3,\delta\big)
\in \Theta=\square^\star\times[0,1/2]\subset\mathbb{R}^4$, where 
$k%=\left(\xi_1,\xi_2,\xi_3\right)
\in \square^\star=[-\pi,\pi]^3$ is the usual Floquet-Bloch quasiperiodicity parameter. 
Then, 
as in the preceding examples, after rescaling and application of Gelfand transform,  we arrive at the equivalent problem: 
%for $u_{\ep,\theta}=U\Gamma_\ep$:
\begin{equation}
\label{IKs4}
\left\{ \begin{aligned}
& \text{ Find $u_{\ep,\theta} \in H^1_{per}(\Box)$, $\Box=[-1/2,1/2]^3$, the solution to} \\
& \ep^{-2}\int_{\Box} (\nabla+ik) u_{\ep,\theta}\cdot \overline{(\nabla+ik)\phi} \,\,\,+\,\ep^{-2}\delta^{-1} \int_{B_{\delta}}u_{\ep,\theta}\,\overline{ \phi}
\,\,\,+\,\int_{\Box}  u_{\ep,\theta} \,\overline{\phi} \,\,\, =\,\, \int_{\Box} g_{\ep,k} \,\overline{\phi} , \quad \forall \phi \in H^1_{per}(\Box), 
\end{aligned} \right.
\end{equation}
where 
%Here, $\xi=\left(\xi_1,\xi_2,\xi_3\right)\in \square^\star=[-\pi,\pi]^3$ is the usual Floquet-Bloch quasiperiodicity parameter, and let now 
%$\theta=(\xi,\delta)=\big(\xi_1,\xi_2,\xi_3,\delta\big)\in \Theta=\square^\star\times[0,1/2]\subset\mathbb{R}^4$; 
$g_{\ep,k} = {U} \Gamma_{\ep}F(k ,\cdot)$.  
%is the Floquet transform of the rescaled $f$ 
Thus %we see 
\eqref{IKs4} is of the form \eqref{p1} with
$H=H^1_{per}(\Box)$, $n=4$, $ \langle f,\tilde{u}\rangle= \int_{\Box} g_{\ep,k} \overline{\tilde{u}}$,
%$$ (u,v)_\theta=\int_{\Box} (\nabla+i\xi) u\cdot \overline{(\nabla+i\xi)v}+\int_{\Box}  u \overline{v},\ \ \langle f,v\rangle= \int_{\Box} F_{\ep,\xi} \overline{v}
%$$
%and
\begin{equation}
\label{conc-mass-ab}
a_\theta\left(u,\tilde{u}\right)\,\,=\,\,\int_{\Box} (\nabla+ik) u\cdot \overline{(\nabla+ik)\tilde{u}} \,\,+\,\,
\delta^{-1} \int_{B_{\delta}}u\,\overline{\tilde{u}}, \quad \text{ and} \quad b_\t\left(u,\tilde{u}\right) \,=\, \int_{\square} u \,\overline{\tilde{u}}.
\end{equation}
%for $\delta\neq 0$. Original problem is defined only for nonzero $\delta$, in order to use our approach we need to define $a_\theta$ for all $\theta\in\Theta$. We put 
%$$a_{\xi,0}(u,v)=\int_{\Box} (\nabla+i\xi) u\cdot \overline{(\nabla+i\xi)v}.$$
To check \eqref{as.b1} we notice that, by  H\"older inequality and standard Sobolev embeddings, there exists $c_0 >0$ such that 
%, for $0 \le \delta \le \gamma \le \tfrac{1}{2}$, one has    
\begin{equation}\label{GNSineq}
\delta^{-1}\int_{B_\delta }|\phi|^2\,\,\le\,\, \delta^{-1} \Big(\int_\square |\phi|^6\Big)^{1/3} \left| B_\delta\right|^{2/3} \,\,=\,\, 
\left(\tfrac{4}{3}\pi\right)^{2/3} \,\delta\, \|\phi\|_{L^6(\square)}^2\,\,\le\,\, c_0\, \delta\, \|\phi\|_{H^1(\square)}^2, \quad 
\forall\delta>0, \ \forall \phi \in H^1_{per}(\square).
\end{equation}
This, together with the arguments as in Section \ref{e.class}, cf \eqref{2.1classhom}, 
implies that 
\eqref{as.b1} holds. Further, as $\delta\le |\t|$ inequality \eqref{GNSineq} also implies that $a_\t$ is  Lipschitz in $\t$ at the origin, i.e. 
\eqref{ass.alip} is satisfied 
for $\t_1 = 0$, $\t_2\in\Theta$. 
%that $a_\t$ is not Lipschitz in the sense \eqref{ass.alip} at the origin
%; indeed by \eqref{GNSineq} (for $\phi = e^{\i \t \cdot y} u$, $u \in H^1_{per}(\square)$)  one readily deduces 
%\begin{equation}\label{SV.atLip}
%| a_\t(u,v) - a_0(u,v) | \le ( 1 + c_0) |\t| \| u \|_\t \| v \|_\t, \quad  \forall u,v \in H^1_{per}(\square), \, \forall  \t \in \Theta.
%\end{equation}
Notice however that \eqref{ass.alip} does not hold globally on $\Theta$ (it can be shown by estimates similar to \eqref{GNSineq} that $a_\t$ is 
merely $\tfrac{2}{3}$-H\"{o}lder continous in $\delta$ and hence in $\t$), and that \eqref{H4} fails to hold for similar reasons. 
Still, we can proceed here with our general method in its relevant parts. 

First notice that as follows from \eqref{conc-mass-ab} the  spaces $V_\theta$ and $W_\theta$ are  as in the classical setting (Example \ref{e.class}, 
see \eqref{vwclasshom}, with $\mathbf{e}$ denotes the identical unity function):
\[
\begin{aligned}
V_\theta = \left\{
\begin{array}{lr}
\{ 0 \}, & \theta \neq 0, \\[5pt]
{\rm Span} (\mathbf{e}), & \theta = 0,
\end{array} 
\right. & \qquad & W_\theta = \left\{
\begin{array}{lr}
H^1_{per}(\Box), & \theta \neq 0, \\[5pt]
H^1_{per, 0} : =\left\{ u\in H^1_{per}(\Box) \, \big\vert \, \int_\Box u = 0 \right\}, & \theta = 0.
\end{array} 
\right.
\end{aligned}
\]
Observe next that the key condition \eqref{KA}  holds: this can be seen by noting \eqref{KA2} is obviously valid with $C=1$ and $c[u]=\int_{\square} |u|^2$. 
Hypothesis \eqref{contVs} is trivially satisfied (with $V^\star_\t=\{0\}$ and $L_\star=0$), and 
moreover we will prove at the end of the subsection that \eqref{distance} also holds. 

In applying our abstract results based on hypotheses \eqref{KA}--\eqref{distance}, certain care needs to be exercised as some of these results may be based on 
global version of \eqref{ass.alip} while in the present example the latter is only assured when $\t_1=0$. 
In any  case, our aim here is also to show that our earlier theorems may already lead to meaningful approximations, 
and thanks to Remark \ref{rem.s3.1} 
Theorem \ref{lmm1} is applicable %(see Remark \ref{rem.s3.1}), 
and states that $u_{\ep,\t}$ is approximated  when $|\t| < \nu_0 / (2L_a)$ by $\M v_0$, where $v_0 \in V_0$ is the solution to \eqref{w1prob0}. 
On the other hand, 
 due to \eqref{distance},  for each $r>0$ \eqref{bddspec} is satisfied on $\Theta_r := \{ \t \in \Theta : | \t | \ge r\}$ with $\nu = \gamma r^2$. Therefore,  Theorem \ref{thm:contV} applies when $\t \in \Theta_r$  (see Remark \ref{rem3.2}), and states in this setting that the solution $u_{\ep,\t}$ to 
\eqref{IKs4} satisfies \eqref{errorcontinuouscase} and \eqref{errorcontinuouscase2} with $v_\t=0$ and $\nu = \gamma r^2$. 
%\begin{equation}\label{SVoutest}
%\begin{aligned}
%	\ep^{-2} \Big(  \int_\square | (\nabla+ \i \xi) u_{\ep,\t} |^2 + \delta^{-1} \int_{B_\delta} |u_{\ep,\t}|^2\Big) + \int_\square | u_{\ep,\t}|^2 \le \frac{\ep^2}{\gamma r^2}\int_\square   |F_{\ep,\xi}|^2, \quad \forall \t \in \Theta_r, \\
%\hspace{-3cm}\text{and} \hspace{3cm}	\ \int_\square | u_{\ep,\t}|^2 \le\frac{\ep^4}{\gamma^2 r^4} \int_\square |F_{\ep,\xi}|^2, \quad \forall \t \in \Theta_r.
%\end{aligned}
%\end{equation}


%On the other hand, Remark \ref{rem.s3.1}  
%\eqref{GNSineq} 
%assures  that Theorem \ref{lmm1} is applicable, %(see Remark \ref{rem.s3.1}), 
%and states that $u_{\ep,\t}$ is approximated  when $|\t| < \nu_0 / (2L_a)$ by $\M v_0$, where $v_0 \in V_0$ is the solution to \eqref{w1prob0}. 
Specialising \eqref{IliaN2} and \eqref{IliaN} to the present example, $v_0=z_{\ep,\t}\,\mathbf{e}$ for some $z_{\ep,\t}\in\CC$, 
$\M v_0=v_0+\N v_0 = z_{\ep,\t}(\mathbf{e}+\ourN_\t) $ where 
$\ourN_\t:=\N\mathbf{e} \in H^1_{per,0}$ is the unique solution to 
\begin{equation}
	\label{IKs15}
	a_\t( \ourN_\t, \widetilde{w}_0)
	%	 =  -a_{\t}(1, \widetilde{w}_0)
	\,=\,-\,\,\delta^{-1} \int_{B_{\delta}}\overline{\widetilde{w}_0}, \qquad \forall \widetilde{w}_0 \in H^1_{per,0}. 
\end{equation} 
Equation \eqref{w1prob0} reduces then to an algebraic equation for $z_{\ep,\t}$ as follows. Setting $\tilde v=\tilde z\mathbf{e}$, $\tilde z\in\CC$, 
and e.g. using \eqref{amnorth}, one obtains: 
$
a_\t\left(\M v_0,\M\tilde v\right)\,=\,\Big( |k|^2\,+\,4\pi\delta^2/3 \,-\, a_\t[\ourN_\t]\Big)z_{\ep,\t}\,\overline{\tilde z}$, %\ \ \ 
$\,\,\, b_\t\left(\M v_0,\M\tilde v\right)\,=\,\Big( 1\,+\, \|\ourN_\t\|^2_{L^2(\square)}\Big)z_{\ep,\t}\,\overline{\tilde z}$, %\ \ \ 
$\,\,\, \left\langle f, \M\tilde v\right\rangle\,=\,  \left(\int_{\Box} g_{\ep,k}(y) \overline{\left(1+\ourN_\t(y)\right)} {d} y\right)\overline{\tilde z}$. 
Then \eqref{w1prob0} results in  
\begin{equation}\label{SVz}
	z_{\ep,\t} \,\,=\,\, \frac{  \int_{\Box}  g_{\ep,k}(y) \overline{(1+\ourN_\t(y))} \, {\rm d} y}
	{\ep^{-2} \Big( |k|^2+4\pi\delta^2/3 - a_\t[\ourN_\t]\Big) \,+\, 
	1\,+\, \|\ourN_\t\|^2_{L^2(\square)}}\,, 
\end{equation}
where  $\ourN_\t \in H^1_{per,0}$ is the solution to the ``cell problem'' \eqref{IKs15}. 
(Notice that the bracketed term in the denominator coincides with $a_\t\left[\M\mathbf{e}\right]$ which via \eqref{distance} is bounded from below 
by e.g. $\gamma|\t|^2$.) 
 
The above approximates $u_{\ep,\t}$ when $\t \in \Theta,$ $|\t| < \nu_0/(2L_a)$. 
Namely, for 
\[
A_{\ep,\t}\big(u,\tilde u\big)\,:=\,\ep^{-2}a_\theta\left(u,\tilde{u}\right)+b_\t\left(u,\tilde{u}\right)\,=\,
\ep^{-2}\left(
\int_{\Box} (\nabla+ik) u\cdot \overline{(\nabla+ik)\tilde{u}} \,+\,
\delta^{-1}\int_{B_{\delta}}u\,\overline{\tilde{u}}\right)\,+\, \int_{\square} u \,\overline{\tilde{u}}, 
\]
\eqref{ik43000}--\eqref{s4ep2bd} imply: 
\begin{equation}
\label{thm42strange}
A_{\ep,\t}\Big[u_{\ep,\t}\,-\,z_{\ep,\t}(\mathbf{e}+\ourN_\t)  \Big]\,\le\,c_1\ep^2\left\Vert g_{\ep,k}\right\Vert_{L^2(\Box)}^2, \quad \text{and} \quad 
\big\Vert u_{\ep,\theta} \,-\,z_{\ep,\t}(\mathbf{e}+\ourN_\t)\big\Vert_{L^2(\square)} \,\le\,   c_1 \ep^2 
\left\Vert g_{\ep,k}\right\Vert_{L^2(\Box)}, \ \ \forall \, |\t|<\,\tfrac{\nu_0}{2L_a},
\end{equation} 
with some constant $c_1>0$ independent of $\ep$, $\t$ and $F$. 
% read
%\begin{equation}\label{SVInnest}
%	\begin{aligned}
%\ep^{-2}\big(  \int_\square | (\nabla+ \i \xi) (u_{\ep,\theta} -(1+\ourN_\t)z_{\ep,\t} ) |^2 + \delta^{-1} \int_{B_\delta} |u_{\ep,\theta} -(1+\ourN_\t)z_{\ep,\t} |^2\Big)   + \| u_{\ep,\theta} -(1+\ourN_\t)z_{\ep,\t}\|_{L^2(\square)}^2
%%	a_\t[w_{\ep,\theta} -(1+\N\textbf{e} )z_{\ep,\t} ]+	\| w_{\ep,\theta} - (1+\N\textbf{e} )z_{\ep,\t} \|_\t^2 
%\hspace{0cm}\\
%\hspace{0\textwidth} + \| u_{\ep,\theta} -(1+\ourN_\t)z_{\ep,\t}\|_{L^2(\square)}^2\le    8K^2 \nu_0^{-1}  \ep^2\|F_{\ep,\xi}\|_{L^2(\Box)}^2, \quad \forall \t \in \Theta, |\t| \le \nu_0 / (2L_a),\\
%\text{and} \hspace{1cm} \| u_{\ep,\theta} -(1+\ourN_\t)z_{\ep,\t}\|_{L^2(\square)} \le    4K^2 \nu_0^{-1}  \ep^2\|F_{\ep,\xi}\|_{L^2(\Box)}, \quad \forall \t \in \Theta, |\t| \le \nu_0 / (2L_a). \hspace{1cm}
%	\end{aligned}
%\end{equation}
In principle this, together with  
\eqref{errorcontinuouscase}--\eqref{errorcontinuouscase2} for $|\t|\ge r_0 = \nu_0/(2L_a)$,  
provides us after the inverse Gelfand and scaling transforms with an approximation to the 
solution $U_{\ep,\t}$ of \eqref{IKs} for small $\ep$, uniform with respect to both $F$ and $\delta$. 
This is quite inexplicit though as requires in particular solving the cell problem \eqref{IKs15} for a range of $k$ and $\delta$. 
However, we can construct a more explicit further approximation of \eqref{SVz} as follows.  

First notice that $\ourN_\t$ is small for small $\delta$. 
Indeed, by \eqref{IKs15} with $\tilde w_0=\ourN_\t$,  $a_\t\left[\ourN_\t\right] \,\le\,\big| \delta^{-1}\int_{B_{\delta}}  \ourN_\t \big|$. 
On the other hand, arguing similarly to \eqref{GNSineq} and employing 
 the Poincar\'{e}-Wirtinger inequality we observe that 
\begin{equation}\label{SVPoinMean}
\Big| \delta^{-1}\int_{B_{\delta}}  \phi_0 \Big| \,\,\le\,\, c_2 \,\delta^{3/2}\, \left\| \nabla \phi_0\right\|_{L^2(\square)}, \quad \ \ \ \forall \phi_0 \in H^1(\square), \,\, \int_\square \phi_0 =0,
\end{equation}
for some $c_2 > 0$. Thus, by sequentially using \eqref{SVPoinMean}, \eqref{conc-mass-ab} and \eqref{coercv0t} one has  
%$\,\le\, 
$a_\t\left[\ourN_\t\right] \,\le\, 2\,c_2^2\,\nu_0^{-1} \delta^3 $, and recalling \eqref{C2} also 
$\|\ourN_\t\|^2_{L^2(\square)}  \,\le\, 4\,c_2^2\,K^2\nu_0^{-2} \delta^3$. 


So estimating, in terms of both 
$|\t|=\left(\delta^2+|k|^2\right)^{1/2}$ and $\ep$, the error of neglecting in \eqref{SVz} all the terms containing $\ourN_\t$,  
and recalling \eqref{distance} for bounding from below the denominator of \eqref{SVz}, 
% and then optimising with respect to $0< t:=|\t|/\ep<+\infty$ 
we obtain  
\begin{equation}\label{SVz2}
\Big| z_{\ep,\t} \,-\, c_{\ep,\t}\Big| \,\,\le\,\, c_3\, %\ep 
\frac{\ep^{-2}|\t|^3}{\left(\ep^{-2}|\t|^2+1\right)^2}
\left\| g_{\ep,k} \right\|_{L^2(\square)}, \quad \text{where}  \quad 	
c_{\ep,\t}\,\, =\,\, \frac{  \int_{\Box} g_{\ep,k}(y) \, {\rm d} y}{\ep^{-2} \big( |k|^2+4\pi\delta^2/3  \big) \,+ \,1}\,, \quad 
g_{\ep,k}=U\Gamma_\ep F(k,\cdot), 
\end{equation}
 and $c_3$ is some positive constant independent of $\ep$, $\delta$, $k$ and $F$. 
Now replace in \eqref{thm42strange} the approximation $z_{\ep,\t}(\mathbf{e}+\ourN_\t)$ by $c_{\ep,\t}\mathbf{e}$. As a result, for the first estimate, 
\begin{equation}
\label{innerest}
A_{\ep,\t}\big[u_{\ep,\t}\,-\,c_{\ep,\t}\mathbf{e}  \big]\,\,\le\,\,\,3\, A_{\ep,\t}\big[u_{\ep,\t}\,-\,z_{\ep,\t}(\mathbf{e}+\ourN_\t)  \big]\,+\,
3 \,A_{\ep,\t}\big[\left(z_{\ep,\t}-c_{\ep,\t}\right)\mathbf{e}  \big]\,+\,
3 \,A_{\ep,\t}\big[z_{\ep,\t}\ourN_\t  \big]. 
\end{equation}
With the first term on the right hand side bounded by \eqref{thm42strange}, for the second term via \eqref{SVz2} 
\[
A_{\ep,\t}\big[\left(z_{\ep,\t}-c_{\ep,\t}\right)\mathbf{e}  \big]=
\left\vert z_{\ep,\t}-c_{\ep,\t}\right\vert^2\left[\ep^{-2}\left(|k|^2+\frac{4}{3}\pi\delta^2\right)+1\right]\le
\frac{4}{3}\pi c_3^2
\frac{\ep^{-4}|\t|^6}{\left(\ep^{-2}|\t|^2+1\right)^3}
\left\| g_{\ep,k} \right\|^2_{L^2(\square)}\le
c_4\ep^2\left\| g_{\ep,k} \right\|^2_{L^2(\square)}, 
\]
with constant $c_4=4\pi c_3^2/3$. 
%>0$ independent of $\ep$, $\t$ and $F$. 
(In the last inequality we used that for $0\le t:=|\t|/\ep<+\infty$, $t^6/(1+t^2)^3<1$.) 
Finally, for the last term in \eqref{innerest}, via \eqref{SVz} together with the above estimates for $a_\t\left[\ourN_\t\right]$ and 
$\left\|\ourN_\t\right\|_{L^2(\square)}$, 
\[
A_{\ep,\t}\big[z_{\ep,\t}\ourN_\t  \big]\,=\,|z_{\ep,\t}|^2\left(\ep^{-2}a_\t\left[\ourN_\t\right]+\left\|\ourN_\t\right\|^2_{L^2(\square)}\right)\,\le\,
c_5\,\frac{\left\| g_{\ep,k} \right\|^2_{L^2(\square)}}{\Big(\ep^{-2}|\t|^2\,\,+\,\,1\Big)^2}\left(\ep^{-2}\delta^3+\delta^3\right)\,\le\,
c_6\,\ep\left\| g_{\ep,k} \right\|^2_{L^2(\square)},
\]
with positive constants $c_5$ and $c_6$ independent of $\ep$, $\t$ and $F$ (having used in the last inequality the boundedness of $t^3/(1+t^2)^2$, 
$0\le t<+\infty$). 
Combining the above we bound \eqref{innerest}, for $|\t|<r_0:=\nu_0/(2L_a)$, by a constant times $\ep\left\| g_{\ep,k} \right\|^2_{L^2(\square)}$. 
On the other hand, for $\t\ge r_0$, it immediately follows from \eqref{SVz2} 
that $\big|c_{\ep,\t}\big| \le  r_0^{-2} \ep^2 \| g_{\ep,k} \|_{L^2(\square)}$ and as a result 
$A_{\ep,\t}\left[c_{\ep,\t}\mathbf{e}\right]=\left|c_{\ep,\t}\right|^2\left(\ep^{-2}a_{\t}\left[\mathbf{e}\right]+1\right)$ is bounded 
by a constant times $\ep^2\left\| g_{\ep,k} \right\|^2_{L^2(\square)}$. 
Also, for $|\t|\ge r_0$,  by 
\eqref{errorcontinuouscase} and 
\eqref{distance} $ A_{\ep,\t}\left[u_{\ep,\t}\right] \le \gamma^{-1} r_0^{-2}  \ep^2 \|  g_{\ep,k} \|_{L^2(\square)}^2$. 
As a result, the left hand side of \eqref{innerest} is bounded by a constant times $\ep^2\left\| g_{\ep,k} \right\|^2_{L^2(\square)}$ for $|\t|\ge r_0$. 

Repeating the above arguments for the second estimate in \eqref{thm42strange} with the approximation $z_{\ep,\t}(\mathbf{e}+\ourN_\t)$ again replaced 
by $c_{\ep,\t}\mathbf{e}$, we observe that the corresponding estimate is dominated by the term analogous to the second term on the right hand side of \eqref{innerest}. 
Namely, recalling \eqref{SVz2}, 
\[
\big\|\left(z_{\ep,\t}-c_{\ep,\t}\right)\mathbf{e}  \big\|_{L^2(\square)}\,=\,
\left\vert z_{\ep,\t}-c_{\ep,\t}\right\vert\,\,\le\,\,
c_3\,   
\frac{\ep^{-2}|\t|^3}{\left(\ep^{-2}|\t|^2+1\right)^2}
\left\| g_{\ep,k} \right\|_{L^2(\square)}\,\,\le\,\,
c_7\,\ep\left\| g_{\ep,k} \right\|_{L^2(\square)}, 
\]
with $c_7>0$ independent of $\ep$, $\t$ and $F$. 

Combining all the above estimates we deduce that 
 \[ %begin{equation*}
 %	\label{SVLimest1}
%\begin{aligned}
\ep^{-2}\left(  \int_\square \big\vert (\nabla+ \i k) \left(u_{\ep,\theta} -c_{\ep,\t}\mathbf{e} \right) \big\vert^2 \,+\, 
\delta^{-1}\int_{B_\delta} \big\vert u_{\ep,\theta} -c_{\ep,\t}\mathbf{e} \big\vert^2\right)  \,\, +\,\, 
\big\Vert u_{\ep,\theta} -c_{\ep,\t}\mathbf{e}\big\Vert_{L^2(\square)}^2 \,\,\,\le\,\,\,   c_8\, \ep\,
\left\Vert g_{\ep,k}\right\Vert_{L^2(\Box)}^2, 
% \quad \forall \t \in \Theta,\\
\]
\begin{equation}
\label{eststgt}
 \text{and} \hspace{3cm} \big\Vert u_{\ep,\theta} -c_{\ep,\t}\mathbf{e}\big\Vert_{L^2(\square)} \,\,\,\le\,\,\,   c_8\, \ep 
\left\Vert g_{\ep,k}\right\Vert_{L^2(\Box)}, \quad \quad g_{\ep,k}=U\Gamma_\ep F(k,\cdot), \quad \quad \forall \t \in \Theta, \hspace{4cm}
%\end{aligned}
\end{equation} %\end{equation*}
for some constant $c_8>0$ independent of $\ep$, $\t=(k,\delta)$ and $F$. 
Comparing the above estimates with \eqref{thm42strange}, we observe that replacing the approximation $z_{\ep,\t}(\mathbf{e}+\ourN_\t)$ by the simplified 
one, $c_{\ep,\t}\mathbf{e}$, results in a ``one power of $\ep$'' loss in the accuracy.

Then, arguing as in Example \ref{e.class} (cf \eqref{SVz2} with \eqref{classicalzsol} leading to \eqref{Sep} ), we deduce that the 
approximation ${U}_{\ep,\delta}^{(0)} = \Gamma_\ep^{-1} U^{-1} c_{\ep,\t}\mathbf{e}$ to the exact solution 
${U}_{\ep,\delta} = \Gamma_\ep^{-1} U^{-1} u_{\ep,\t}$ of the original problem \eqref{IKs}, 
%for $\xi \in \square^*$ where 
%$\mathbf{w}_{\delta / \ep} \in H^1(\mathbb{R}^3)$ 
is itself the solution to 
 \begin{equation}\label{SVlim}
 \left(-\,\Delta\,\, +\,\frac{4}{3}\pi\,\frac{\delta^2}{\ep^2}\,\,+\,1\right){U}_{\ep,\delta}^{(0)}\,\,=\,\, \mathcal{S}_\ep F, \quad  \text{in} \ \mathbb{R}^3, 
 \end{equation}
with $\mathcal{S}_\ep$ given by \eqref{7.22-2} where $\chi$ stands for the characteristic function of $\square^*$. 
Further, estimates \eqref{eststgt} imply similar estimates for ${U}_{\ep,\delta}^{(0)}$, cf. \eqref{IKH1est}--\eqref{IKL2est}: 
%Consequently, upon taking the inverse Gelfand and scaling in \eqref{SVLimest1} we demonstrate that 
 \begin{equation}\label{SVLimest2}
\begin{aligned}
\left\Vert U_{\ep,\delta}  \,-\,   U_{\ep,\delta}^{(0)} \right\Vert_{H^1(\RR^3)}  \,\,\le\,\,   c_8\, \ep^{1/2}\,
\|F\|_{L^2(\RR^3)}, \quad \text{and} \quad 
\left\Vert U_{\ep,\delta}  \,-\,  % \mathcal{S}_\ep 
U_{\ep,\delta}^{(0)} \right\Vert_{L^2(\RR^3)}  \,\,\le\,\,   c_8\, \ep\,\|F\|_{L^2(\RR^3)}. 
\end{aligned}
\end{equation}
%where we recall $\mathcal{S}_\ep = \mathcal{F}^{-1} (\Gamma_\ep \chi) \mathcal{F}$ for $\chi$ the characteristic function of $\square^*$. 
Finally, as \eqref{7.22-2} implies (cf. \eqref{7.23-3} leading to \eqref{7.23-3}) that 
$\| \mathcal{S}_\ep F -F \|_{H^{-1}(\RR^3)} \le  \ep \pi^{-1} \| F\|_{L^2(\RR^3)}$, it follows that the estimates analogous to \eqref{SVLimest2} remain valid if 
 $  \mathcal{S}_\ep$ is removed in \eqref{SVlim}. We collect all of the above observations to state the following theorem.   
 %where
%\begin{equation}\label{SVz2}
%	c_{\ep,\t} = \frac{  \int_{\Box} \mathcal{U} \Gamma_\ep f(,\xi,y) \, \rm{d} y}{\ep^{-2} ( |\xi|^2+4\pi\delta^2/3  ) + 1}.
%\end{equation}
% and we shall demonstrate that one can establish, via Theorems \ref{thm:contV} and \ref{lmm1}, the following result.
\begin{theorem}
	\label{IKVolcompareclassicalh1}
Let $U_{\ep,\delta}$ solve \eqref{IKs} and, for each $\alpha \in [0,\infty)$, let  $U_{\alpha}\in H^1(\mathbb{R}^3)$ solve 
\[\left(-\,\Delta \,\,+\,\,\frac{4}{3}\pi\, \alpha^2\,\,+1\,\,\right)U_{\alpha}\,\,=\,\,F \quad \text{in} \ \mathbb{R}^3. 
\]
Then there exists a positive constant $c$ independent of $\ep$, $\delta$ and $F$ such that 
\begin{equation}\label{SVfinalEst}
\left\Vert U_{\ep,\delta} \,-\, U_{\delta/\ep} \right\Vert_{H^1(\mathbb{R}^3)}\,\, \le\,\,c\, \ep^{1/2}\,
 \Vert F \Vert_{L^2(\mathbb{R}^3)},\quad \text{and} \quad   
\left\Vert U_{\ep,\delta} \,-\, U_{\delta/\ep} \right\Vert_{L_2(\mathbb{R}^3)}\,\, \le\,\,c\, \ep\,\Vert F \Vert_{L^2(\mathbb{R}^3)}, 
\end{equation}
for all $0<\ep<1$ and $0 \le \delta \le 1  / 2$.
\end{theorem}
\begin{remark}
Notice that  estimates \eqref{SVfinalEst}, uniform in both $\ep$ and $\delta$, hold in particular for the `critical scaling' $\delta = O(\ep)$. 
For example, for $\delta=\ep$  Theorem \ref{IKVolcompareclassicalh1} states that $u_\ep$, the solution to the concentrated perturbation problem 
\[
 \int_{\mathbb{R}^3} \nabla u_{\ep}\cdot \overline{\nabla\phi} \,\,+\,\,\ep^{-3}\sum_{j\in \mathbb{Z}^3} \int_{B_{\ep^2}(\ep j)}u_{\ep}\overline{ \phi}
\,\,+\,\,\int_{\mathbb{R}^3}  u_{\ep}\overline{\phi} \,\, =\,\, \int_{\mathbb{R}^3} F \overline{\phi} , \qquad \forall \phi \in H^1(\mathbb{R}^3),
\] 
i.e. with $\ep$-periodic inclusions of size $\ep^2$ and of ``density'' $\ep^{-3}$, 
is approximated with operator-type error estimates 
by $u_0 $ the solution to the %`strange term' 
averaged 
problem
\[ \big(-\,\Delta \,\,+\,\, \mu \,\,+\,\,1\big)u_0\,\,=\,\,F, \quad \text{in} \ \mathbb{R}^3,
\]
where $\mu = 4\pi / 3$. % is the capacity volume of the unit ball in $\RR^3$. 
Indeed, \eqref{SVfinalEst} gives 
\[
\Vert u_{\ep} \,-\, u_0 \Vert_{H^1(\mathbb{R}^3)} \,\,\le\,\, c\,\ep^{1/2} \Vert F \Vert_{L^2(\mathbb{R}^3)},\  \quad 
\Vert u_{\ep} \,-\, u_0 \Vert_{L_2(\mathbb{R}^3)} \,\,\le\,\, c\,\ep \Vert F \Vert_{L^2(\mathbb{R}^3)}.
\]
\end{remark}
We conclude this example with the proof of \eqref{distance}. 
\begin{proof}
For proving \eqref{distance} it is sufficient to show that 
\begin{equation*}
%	\label{SV.bddspec}
\| (\nabla + \i k) u\|_{L^2(\square)}^2 \,\,+\,\, \delta^{-1} \int_{B_{\delta}} | u|^2 \,\, \ge\,\, 
\tilde\gamma \, \bigl(|k|^2 \,+\,\delta^2\bigr)\| u \|^2_{L^2(\square)}, \quad \forall u \in H^1_{per}(\square),\, \ \  \forall \,(k,\delta) \in \Theta,
%\ c_3 = (2+\tfrac{9c_1^2}{4\pi^2}) ^{-1},
\end{equation*}
for some $\tilde\gamma>0$. Clearly, cf. e.g. \eqref{IKaest} and from the triangle inequality, 
%$(\nabla + \i \xi) u\|_{L^2(\square)} \ge   |\xi| \| u \|_{L^2(\square)}$, $\forall\xi\in \square^*$. and 
\[
\| (\nabla + \i k) u\|_{L^2(\square)} \,\ge \,  |k|\, \| u \|_{L^2(\square)}, \quad \text{and}\quad  
\| (\nabla + \i k) u \|_{L^2(\square)} \, +\, |k|\| u \|_{L^2(\square)}  \,\ge \,\| \nabla u \|_{L^2(\square)} , \quad   \forall u \in H^1_{per}(\square), \, \  \forall k \in \square^*.
\]
Combining these implies $3\,\| (\nabla + \i k) u\|_{L^2(\square)}\,\ge\, |k|\, \| u \|_{L^2(\square)}+\| \nabla u \|_{L^2(\square)}$, and 
so it suffices to show that 
\begin{equation}\label{SV.bddspec}
	\| \nabla u\|_{L^2(\square)}^2 \,+\, \delta^{-1} \int_{B_{\delta}} | u|^2 \,\, \ge \,\,c\,\delta^2\,\| u \|^2_{L^2(\square)}, \quad \forall u \in H^1_{per}(\square),\quad  \forall \delta \in (0, 1/2],
	%\ c_3 = (2+\tfrac{9c_1^2}{4\pi^2}) ^{-1},
\end{equation}
with some %appropriate 
$c>0$. 

To prove \eqref{SV.bddspec}, 
for $u = c+u_0 $  with $c =\int_\square u$ and so $\int_\square u_0 =0$, using \eqref{SVPoinMean} we obtain  
\begin{flalign*}
&\delta^{-1}\int_{B_{\delta}} | u|^2 \,\,\ge\,\, \frac{4}{3} \pi\, \delta^2|c|^2 \,\,+\,\,
 2\,\delta^{-1}\, {\rm Re} \int_{B_{\delta}}  u_0 \,\overline{c} \,\,  \ge\,\, 
\frac{4}{3} \pi \delta^2|c|^2   -\, 2\,c_2\delta^{3/2}|c| \, \| \nabla u \|_{L^2(\square)}\,\,\ge \\ 
\quad & \frac{2}{3} \pi \delta^2|c|^2  \, -\, \frac{3\,c_2^2}{2\pi} \,\delta\,  \| \nabla u \|_{L^2(\square)}^2  
\,\,=\,\, \frac{2}{3} \pi \delta^2\left(|c|^2 \,+\, \| \nabla u \|_{L^2(\square)}^2\right) \,-\,  
\left( \frac{2}{3} \pi \delta^2 \,+\, \frac{3\,c_2^2}{2\pi} \delta \right) \| \nabla u \|_{L^2(\square)}^2. 
\end{flalign*}
This implies  (as $\delta \le 1/2$) that 
\[
\left(\frac{\pi}{6} \,+\, \frac{3\,c_2^2}{4\,\pi}\right)  \| \nabla u \|_{L^2(\square)}^2 \,\,+\,\, 
\delta^{-1} \int_{B_{\delta}} | u|^2 \,\,\,\ge\,\,\, \frac{2}{3}\, \pi\, \delta^2
\left(\left\vert\int_\square u\right\vert^2 \,+\, \| \nabla u \|_{L^2(\square)}^2\right), \quad \forall u \in H^1(\square).
\]
Then, after application of the Poincar\'{e}-Wirtinger inequality, one arrives at \eqref{SV.bddspec}.
%Upon setting $\phi = e^{\i \xi\cdot y} u$ and utilising \eqref{IKaest}, we deduce that 
%\[
%(\tfrac{2\pi}{3}+\tfrac{3c_1^2}{4\pi}) \| (\nabla + \i \xi) u\|_{L^2(\square)}^2 + \delta^{-1} \int_{B_{\delta}} | u|^2  \ge \tfrac{2\pi}{3} \Big( |\xi|^2 \| u \|_{L^2(\square)}^2 + \delta^2\Big| \int_\square e^{\i \xi \cdot y} u \Big|^2 \Big) \quad \forall u \in H^1_{per}(\square).
%\]
%Finally, since $\big| \int_\square e^{\i \xi \cdot y} u \big|^2 \ge \tfrac{1}{2} \Big| \int_\square u \Big|^2 - |\xi|^2 \int_\square |u|^2$ (and $\| u\|_{L^2(\square)}^2= \| u-\int_\square u\|_{L^2(\square)}^2 + \big| \int_\square u\big|^2$) then \eqref{SV.bddspec} holds for $\gamma = (2+{9c_1^2}/{4\pi^2}) ^{-1}$. 
\end{proof}
\subsection{Periodic inclusions with imperfect interfaces}
\label{sec:impint}
Here we consider a problem where the space $V_\t$ has a removable singularity at the origin but $V_\t$ is not piece-wise constant, i.e. \eqref{contVs} holds but the conditions of Remark \ref{constV} do not\footnote{Another such example, in the context of linear elasticity, is Example \ref{e.pdelast} below.}. Such a situation arises, for example, %by studying 
in composites whose inclusions are not in perfect contact with the surrounding `matrix'.


Let %the dimension $n\ge 2$ and 
the reference inclusion $B\subset\square$ be as in Example \ref{e.dp}, $B_\ep = \bigcup_{z\in\ZZ^n}\ep (B+z)$ be the set 
of associated $\ep$-periodic inclusions 
and its complement $M_\ep :=\RR^n \backslash \overline{B_\ep}$ be the connected matrix, and 
%Denote $\chi_B$ the characteristic function of $%\overline{B}$ %(periodically extended to $\RR^n$) 
%and 
%$\chi_1 : = 1 - \chi_0$,
let $n_\ep$ be the outer unit normal to  the interface $I_\ep=\partial B_\ep$. For %fixed 
$0<\ep <1$, consider the problem: 
Given $F\in L^2(\RR^n)$, find in the matrix and in the inclusions  
 %$u_\ep = \big(1 - \chi_0(\tfrac{\cdot}{\ep})\big) u_1^\ep + \chi_0(\tfrac{\cdot}{\ep}) u_2^\ep$ where
 $u_1^\ep$ and $u_2^\ep$ respectively, such that 
\begin{equation}
\label{impintclass}
\begin{aligned}
-\, \Delta\, u_1^\ep \,+\, u_1^\ep \,\,=\,\, F \ \text{in $M_\ep$}, \quad  - \,\Delta\, u_2^\ep \,+\, u_2^\ep \,\,=\,\, F \ \text{in $B_\ep$}, 
\quad \partial_{n_\ep} u_1^\ep \,=\, \partial_{n_\ep} u_2^\ep \,=\, \ep\, \left(u_2^\ep \,-\, u_1^\ep\right) \ \text{on $I_\ep$},
\end{aligned}
\end{equation}
where $\partial_{n_\ep}$ denotes the normal derivative. 
This problem admits equivalent variational formulation: Find $u_\ep \in W_\ep := L^2(\RR^n) \cap  H^1(M_\ep)  \cap H^1(B_\ep)$ such that
%=\{ u \in L^2(\RR^n) \, \big | \, u|_{\ep F_1} \in H^1(\ep F_1) \text { and } u|_{\Ep F_0} \in H^1(\Ep F_0) \big\}$ such that
\begin{equation}\label{p.ic}
\int_{M_\ep} \nabla u_\ep \cdot \overline{\nabla \phi}\,\, + \int_{B_\ep} \nabla u_\ep \cdot \overline{\nabla \phi}\,\, + 
\ep \int_{I_\ep} \left[ u_\ep\right]_\ep \overline{\left[{\phi}\right]_\ep}\,\, + \int_{\RR^n}  u_\ep\, \overline{\phi} \,\,\,=\,\, 
\int_{\RR^n} F\, \overline{\phi}, \quad \forall \phi \in W_\ep,
\end{equation}
where $[u]_\ep$ denotes the jump in $u$ across $I_\ep$, i.e. $[u]_\ep := T_\ep^+ u - T_\ep^- u$ where 
$T^+_\ep : H^1\left(M_\ep\right) \rightarrow L^2\left(I_\ep\right)$ and 
$T^-_\ep : H^1\left(B_\ep\right) \rightarrow L^2\left(I_\ep\right)$ are the trace operators. 
%{\color{red} a sentence on physical origin/relevance of such a model}.

We take our usual approach and restate problem \eqref{p.ic} in the form \eqref{p1} via the transforms $\Gamma_\Ep$ and $U$ (see Example \ref{e.class}). 
Then $u_{\ep,\t} := U \Gamma_\ep u_\ep(\t,\cdot)$  is the solution to
\begin{equation*}
	\ep^{-2} a_\t \left(u_{\ep,\t}, \tilde{u}\right)\, +\, b_\t\left(u_{\ep,\t},\tilde{u}\right)  \,\,=\,\, \l f,\tilde{u} \r, \quad 
	\forall \tilde{u} \in H, \,\,a.e.\, \theta \in \Theta,
\end{equation*}
where  $u_{\ep,\t}\in H =  L^2(\Box) \cap H^1_{per}(\Box\backslash B) \cap H^1(B)$,  $\Theta := \Box^*$, $\l f, \tilde{u} \r := (U \Gamma_\ep F(\t,\cdot),\tilde{u})$ for $(\cdot,\cdot)$ the standard $L^2(\Box)$ inner product, 
\begin{equation}
\label{abimpint}
\begin{aligned}
%&H =  L^2(\Box) \cap H^1_{per}(\Box\backslash B) \cap H^1(B)
%%= \big\{ u \in L^2(\Box) \, \big| \, u|_{\Box \backslash B} \in H^1_{per}(\Box\backslash B) \text { and } u|_{B} \in H^1(B) \big\},
% \\
a_\t[u] \,\,=\,\, \int_{\Box \backslash B} \big| (\nabla + \i \t ) u \big|^2 &\,+\,  
\int_{B} \big| (\nabla + \i \t ) u \big|^2, \quad \text{and } \quad b_\t[u] \,=\, b[u]\,\, :=\,\, \int_{\partial{B}} \big|\, [u]\, \big|^2\,+\, 
\int_\Box |u|^2,
% \quad u \in H.
\end{aligned}
\end{equation}
where  
%$J[u] : =  \int_{\partial{B}} \big| [u] \big|^2$ where
 $[u]$ denotes the jump in $u$ across the interface $\partial B$.  
% Clearly, $\| u\|_\t$ (see \eqref{astructure}) is an equivalent norm on $H$ due to the properties of the trace operator.

Let us now show that all our general assumptions hold. It is clear, cf. \eqref{2.1classhom}, that the basic assumptions  \eqref{as.b1} and \eqref{ass.alip}  hold.  Next note that, 
denoting $\chi_B$ the characteristic function of the inclusion $B$, 
\begin{equation}
\label{vwimpint}
V_\t \,=\, \left\{ 
\begin{array}{lr}
{\rm Span}\,  \left(e^{-\i \t \cdot y}\chi_B\right) & \t \neq 0, \\ 
\vspace{-.08in} \\ 
 {\rm Span}\, \left(  \mathbf{e}, \, \chi_B \right)  & \t = 0,
\end{array}
\right. \quad 
W_\t \,=\, \left\{ 
\begin{array}{lr}
\Big\{ w \in H \, \big| \, \int_{\partial B} [w] e^{\i \t \cdot y}  \,= \, \int_B we^{\i \t \cdot y}  \Big\}  & \t \neq 0, \\ 
\vspace{-.08in} \\
\Big\{ w \in H \, \big| \, \int_{\partial B} [w]  \,=\, \int_{ B} w \text{ and }  \int_{\Box }  w =0    \Big\}  & \t = 0.
\end{array}
\right.
\end{equation}
Therefore $V_\t$ has a discontinuity at $\t=0$ and %non-trivially depends on 
varies with 
$\t$ in $\Box^* \backslash \{ 0 \}$. 
Now check that for this example \eqref{KA}--\eqref{H6} all hold.


\textbullet \, {\it Proof of \eqref{KA}.} 
Standard arguments show that
%({\color{red} cite Peetre/Tartar equivalence lemma?})
%\[
%\| u \|_H^2 : =  \| \nabla u \|_{L^2(\Box \backslash B)}^2 + \| \nabla u \|_{L^2(B)}^2 + \bigg|  \int_{\Box \backslash B} u \bigg|^2 +  \bigg|  \int_{\partial B} [u] + \int_{B} u \bigg|^2,
%\]
%defines an equivalent norm on $H$. In particular, 
there exists a constant $C_{B}>0$ such that
%(independent of $\t$ by the above assertion and \eqref{as.b1})
\begin{equation}\label{ic.h1'e1}
b[\phi] \,\,\le\,\, C_{B} \left( a_0[\phi] \,+\, \bigg|  \int_{\Box \backslash B} \phi \bigg|^2 \,+\,  
\bigg|  \int_{B} \phi \,-\, \int_{\partial B} [\phi] \bigg|^2 \right), \quad \forall \phi \in L^2(\square) \cap H^1(
\square \backslash B) \cap H^1(B).
\end{equation}
Indeed, if $\phi_n$ with $b\left[\phi_n\right]=1$ are such that \eqref{ic.h1'e1} is violated with $C_B$ replaced by $n$, then $a_0\left[\phi_n\right]\to 0$ and so 
$\left\{\phi_n\right\}$ are bounded in $H^1(\square \backslash B)$ and $H^1(B)$. So, up to a subsequence, $\left\{\phi_n\right\}$ converges  
$H^1$-weakly and  $L^2$-strongly to some $\phi_0$. Then $b\left[\phi_0\right]= 1$  by the $L^2$-compactness of the trace operators, 
and by the weak lower-semicontinuity $a_0\left[\phi_0\right]= 0$ 
so $\phi_0\in V_0$ i.e. 
$\phi_0=c_1 +c_2\chi_B$. On the other hand, by the compactness, for $\phi=\phi_0$ both other terms on the right hand side of \eqref{ic.h1'e1} 
are zero, which implies $c_1=c_2=0$ i.e. 
$b\left[\phi_0\right]= 0$ which is a contradiction. 

We then show that \eqref{KA2} holds for $c(u,\tilde{u}) =\, C_{B}\, |\Box \backslash B| \int_{\Box \backslash B} u \overline{\tilde{u}}$ and $C = C_{B}+1$. Indeed for fixed $\t \in \Theta$ and  $w \in W_\t$, see \eqref{vwimpint}, $\int_{B} w\,e^{\i \t \cdot y} \, - \int_{\partial B} \left[w\,e^{\i \t \cdot y}\right]  =0$ 
and so \eqref{ic.h1'e1} for $\phi = e^{\i \t \cdot y} w$ gives
\begin{equation}\label{ic.h1proof}
\begin{aligned}
b[w] & \,\,=\,\, b\left[e^{\i \t \cdot y} w\right]\,\,\le\,\, C_{B} \left( a_0\left[e^{\i \t \cdot y} w\right] \,+\, \bigg|  \int_{\Box \backslash B}  e^{\i \t \cdot y} w \bigg|^2\right) \,\,\le\,\,  C_{B}  \left( a_\t[w] \,+\,  |\Box \backslash  B|  \int_{\Box \backslash B} | w |^2 \right).
\end{aligned}
\end{equation}
Hence \eqref{KA2} holds (as $c[\cdot]$ is clearly $\|\cdot\|_\t$-compact), and so \eqref{KA} holds too by Proposition \ref{prop.kaequiv}.

\textbullet \,  {\it Proof of \eqref{contVs}.} We set $V_\star={\rm Span}\, \left(\chi_B\right)$, and so 
$V^\star_\t : = {\rm Span}\, \left(e^{-\i \t \cdot y}\chi_0\right)$ for all $\t \in \Theta= \Box^*$. 
Then, 
%is clearly the continuous part of $V_\t$. Indeed, 
for given $V^\star_{\t_1} \ni v_1 = c_1 e^{-\i \t_1 \cdot y}\chi_B $, $c_1\in \CC$,  set $v_2 =c_1 e^{-\i \t_2 \cdot y}\chi_B \in V_{\t_2}^\star$ and so 
$\left| v_1(y) - v_2(y) \right| \le\,\left(n^{1/2}/2\right) \left| \t_1 - \t_2\right|\, \left| v_1(y) \right|$, $y \in \Box$. Therefore, since for 
$i=1,2$, 
$a_{\t_i}\left[ v_i\right]=0$ and $\left[v_i\right] = \,-\,v_i$,   
\[
\bigl\| v_1 - v_2\bigr\|_{\t_2}^2 \,\,=\,\, \left|\t_1-\t_2\right|^2 \int_\square \left| v_1 \right|^2 \,\,+\, 
\int_{\partial{B}} \left| v_1 - v_2\right|^2 \,\,+\, \int_\square \left|v_1 - v_2\right|^2\,\,\,\le\,\, \left(\frac{n}{4}+1\right) \left|\t_1-\t_2\right|^2\, 
\left\| v_1 \right\|^2_{\t_1},
\]
and so \eqref{contVs} holds %for  $V_\star = {\rm span}\, ( \{\chi_0\})$ 
with $L_\star = \left({n}/{4}+1\right)^{1/2}$. Furthermore, %notice  that 
one can naturally choose  $Z = {\rm Span}\, (\mathbf{e})$ so that  \eqref{spaceZ} and \eqref{VZorth} 
hold with $K_Z=|B|/\big(|B|+|\partial B|\big)^{1/2} < 1$). 


%as \eqref{spaceZ} and \eqref{VZorth} for $K_Z = ( 1 + | \square \backslash B |  / | \partial B|)^{-1/2} ( 1 + | B |  / | \partial B|)^{-1/2}$ obviously hold.
% $K_Z = | \partial B| / (( | \partial B| + | \square \backslash B | )^{1/2} ( | \partial B| + | B |)^{1/2})$.

\textbullet \, Assumption \eqref{distance} 
follows by applying \eqref{ic.h1proof} and then \eqref{dpH1}, with e.g. 
%is immediate, for 
$\gamma =\left(1+C_{B}\right)^{-1}\left(n\pi^2  + |\Box \backslash B| C_E^2\right)^{-1} $. %, from \eqref{ic.h1proof} and \eqref{dpH1}.

\textbullet \, Assumption \eqref{H4} is obviously satisfied, see \eqref{abimpint}, with  $K_{a'}=1, K_{a''}=0$, and 
\begin{equation}
\label{impint.a'}
\begin{aligned}
& a'_{0}(v, u) \cdot \t\,\, :=\,\,
%\int_{\Box \backslash B}  \nabla u \cdot \overline{ \i \t \tilde{u}}+
 \int_{\Box \backslash B}  \i\, \t v \cdot \overline{ \nabla u} \,+\,  \int_{B}  \i\, \t v \cdot \overline{ \nabla u},
% + \int_{B}  \nabla u \cdot \overline{ \i \t \tilde{u}} + \int_{B}  \i \t v \cdot \overline{ \nabla u}  ,
\quad \text{and} \quad 
& a''_{0}\,\left(v, \tilde{v}\right)\,\t \cdot \t\,\,:=\,\, |\t|^2 \int_{\Box } v\, \overline{\tilde{v}}.
%+ |\t|^2 \int_{B} v \overline{\tilde{v}}.
\end{aligned}
\end{equation}



\textbullet \, Assumption \eqref{H5} trivially holds for $L_b =0$ since $b_\t$ is independent of $\t$.  Furthermore, it is clear that \eqref{Eprop1} and \eqref{Eprop2} with $K_b = n^{1/2}/2$ hold for $\mathcal{E}_\t$ defined as multiplication by $e^{-\i\, \t\cdot y}$.

\textbullet \,  Finally, \eqref{H6} also trivially holds  with $\mathcal{H} = L^2(\square)$, $d_\t$ the standard $L^2$-inner product, $\mathcal{E}_\t$ multiplication by $e^{-\i \t\cdot y}$ on $L^2(\square)$ and $K_e = n^{1/2}/2$. Notice that the above choice of 
$Z={\rm Span}\, (\mathbf{e})$ satisfies \eqref{zvbd}.

As all the main assumptions \eqref{KA}--\eqref{H6} hold, our general results are applicable to the present example, in particular Theorems \ref{thm.IKunifest2} and Theorem \ref{ikthm2}. We will illustrate below the spectral results related to the latter, leaving it to the reader specialising any of our other general results to the present setting. 

Notice first that for the solution $u_\ep$ to \eqref{impintclass}, equivalently \eqref{p.ic}, $u_\ep=\left(\mathcal{L}_\ep+I\right)^{-1}F$, where $\mathcal{L}_\ep$ is 
the self-adjoint operator in $L^2(\RR^n)$, with  standard inner product, which is generated by the form 
\[
Q_\ep\left(u,\,\tilde u\right)\,\,=\,\,\int_{M_\ep} \nabla u\cdot\overline{\nabla \tilde u} \,\,+\, 
\int_{B_\ep} \nabla u\cdot\overline{\nabla \tilde u}\,\,\,+\, \ep \int_{I_\ep} \left[ u_\ep\right]\,\overline{\left[ \tilde u_\ep\right]}, 
\]
with the form domain $W_\ep$. We are interested in  the spectrum  ${\rm Sp}\, \mathcal{L}_\ep$   of $\mathcal{L}_\ep$. 

  Upon  applying the unitary rescaling $\Gamma_\ep$ and the Gelfand transform $U$, it follows that the spectrum ${\rm Sp}\, \mathcal{L}_\ep$ is equal to 
	$\overline{\bigcup_{\t \in \square^*}\mathbf{L}_{\ep,\t} }$ where $\mathbf{L}_{\ep,\t} $ is  the self-adjoint operator generated in $L^2(\square)$ by the form
 \[
q_{\ep,\t}\left(u,\,\tilde u\right)\,\,=\,\,
 \ep^{-2} \left(  \int_{\Box \backslash B}  (\nabla + \i \t) u \cdot\overline{\left(\nabla+\i \t \right) \tilde u}\,\,\,  +\,  
\int_{B} (\nabla + \i \t) u \cdot\overline{\left(\nabla+\i \t \right) \tilde u} \right)\,\,\,+\,
\int_{\partial B}[u]\,\,\overline{\tilde u} 
 \]
 with the form domain $H = L^2(\square) \cap H^1_{per}( \square \backslash B) \cap H^1(B)$.  The spectrum of $\mathbf{L}_{\ep,\t}$ consists of countably many nonnegative real eigenvalues $\{ \lambda_{\ep,\t}^{(k)}\}_{k \in \NN}$ labelled in the increasing order accounting for their multiplicity. The functions $E^{(k)}_\ep(\t) : = \lambda_{\ep,\t}^{(k)}$, $\t\in\square$, are the spectral band functions of $\mathcal{L}_\ep$. 
%  We will provide the  leading-order asymptotics of the bands $E^{(k)}_\ep$ and approximate ${\rm Sp}\, \mathcal{L}_\ep$  with error.
 
% We begin by noting, upon setting $\mathcal{H} = L^2(\square)$ and  $d_\t = d$, that  $\mathbf{L}_{\ep,\t} =  \mathcal{L}_{\ep,\t} - I$ for $\mathcal{L}_{\ep,\t}$ is as in Section \ref{s:resolv}. Consequently,


In short, Theorem \ref{ikthm2} provides the asymptotics of the spectral bands $E^{(k)}_\ep$  in terms of the eigenvalues of  $\mathbb{L}_{\t / \ep} $ which, in turn, describes  the approximation of ${\rm Sp} \, \mathcal{L}_\ep$ in terms of $\overline{\bigcup_{\xi \in \ep^{-1} \square^*} {\rm Sp}\, \left(\mathbb{L}_\xi-I\right)}$ or, via Corollary \ref{c.collspec}, by  $\overline{\bigcup_{\xi \in \RR^n } {\rm Sp}\, \left(\mathbb{L}_\xi-I\right)}$. Finally, we have the characterisation of $\overline{\bigcup_{\xi \in \RR^n } {\rm Sp}\, \mathbb{L}_\xi}$ given in Theorem \ref{thm.limspecrep} thus completing our asymptotic analysis with error 
estimates for  
${\rm Sp}\, \mathcal{L}_\ep$ and its spectral band functions. 

Now, we follow the above steps with more detail. First we need to specify the limit operator $\mathbb{L}_\xi$ given by  the form \eqref{Sform}. 
%First note, for the sake of spectral analysis we shall choose $Z$ to be such that \eqref{zvbd} holds. In this setting, that results in setting $Z = {\rm Span}\,( \{ 1\})$.  
To  determine the homogenised form $a^{\rm h}$, defined by \eqref{defhom.form}, we first need the corrector $N_\t=\t\cdot N$, as specified by \eqref{cell:prob2}, on $Z$. 
For this, one can see from \eqref{cell:prob2} and \eqref{impint.a'}, cf. \eqref{dp.cell1}, that (up to an element of $V_0$) 
$\mathbf{e} \mapsto  (N\mathbf{e})(y) =  \i\,\left(1-\chi_B\right) \ourN^{\rm pd}(y) - \i\, y \chi_B $ 
 where $\ourN^{\rm pd}$ solves \eqref{dp.correctorproblem}. As a result $a^{\rm h}$, which is on $Z$ fully determined by $a^{\rm h}_\t [\mathbf{e}]$, 
is found, cf \eqref{dp.a'2}, to be in the form 
$a^{\rm h}_\t [\mathbf{e}] = A^{\rm hom}_{\rm pd} \t \cdot \t$ where $A^{\rm hom}_{\rm dp}$ is the perforated domain homogenised matrix 
  given by \eqref{dp.coef} in Example \ref{e.dp}. 
%Considering $w$ that is zero in $\square \backslash B$

Further,  in this setting we have $V_\star + Z \,=\, \overline{V_\star + Z} \,=\, \mathcal{H}_0 = \big\{ c_1 \,+\,c_2\,\chi_B  \,\, \big| \,\, c_1,c_2 \in \CC \big\}$. 
Putting all this together, form \eqref{Sform} specialises to 
\[
\begin{aligned}
& \mathbb{S}_\xi \big(c_1+c_2 \chi_B \,,\, \tilde{c}_1 + \tilde{c}_2 \chi_B\big) \,\,=\,\, 
M_\xi\left( \begin{matrix}c_1 \\ c_2  \end{matrix} \right) \cdot  \overline{\left( \begin{matrix} \tilde{c}_1 \\ \tilde{c}_2  \end{matrix} \right)}, \quad \text{ where } M_\xi \, =\,\left( \begin{matrix}
	A^{\rm hom}_{\rm pd} \xi \cdot \xi + 1 & |B| \\ |B| & |\partial  B| + |B| 
\end{matrix} \right), \\
&d_0\big(c_1+c_2 \chi_B ,\, \tilde{c}_1 + \tilde{c}_2 \chi_B\big) \,\,=\,\, D\left( \begin{matrix}c_1 \\ c_2  \end{matrix} \right) \cdot  \overline{\left( \begin{matrix} \tilde{c}_1 \\ \tilde{c}_2  \end{matrix} \right)}, \quad \text{ where } D \, =\,\left( \begin{matrix}
1 & |B| \\ |B| &  |B| 
\end{matrix} \right). 
\end{aligned}
 \]
As a result, see \eqref{lxispec}, the eigenvalues of $\mathbb{L}_\xi$ are the solutions of the generalised (real-valued) eigenvalue problem
 \[
 M_\xi c \,\,=\,\, \lambda\, D c \quad \text{ for some non-trivial $c \in \RR^2$.}
 \]
%Whence, $\mathcal{H}_0$ is a $2$-dimensional vector space and   $\mathbb{L}_\xi$ is isomorphic to the matrix $M_\xi$.

Hence $\lambda$ are roots of the polynomial ${\rm det} \left( M_\xi -\lambda D\right) = 
{\rm det} \Big( {\rm diag}\left( A^{\rm hom}_{\rm dp} \xi \cdot \xi ,\, |\partial B|\right)  \,-\,(\lambda -1)D\Big)$, 
however we shall determine $\lambda$ by using the representation
%the  analysis   of Section \ref{s.charlimsp}. 
\eqref{finallimitspectralproblem} with $\beta_\lambda$ given by \eqref{betaform}--\eqref{6.27-2}.  
We note that $V_\star = \overline{V_\star} = {\rm Span}\, ( \chi_B )$,  $b_0[\chi_B] = |\partial B| + |B|$ and 
$d_0[\chi_B] = |B|$. Therefore, the operator $\mathbf{B}$ (defined in \eqref{ikddd2}, i.e. as the operator in $\overline{V_\star}$ with inner product $d_0$ generated by $b_0$ with form domain $V_\star$) is simply the multiplication by $1 + |\partial B | / |B|$. 
Hence ${\rm Sp}\, \mathbf{B} = \{ 1 + \mu_0 \}$, where $\mu_0 := |\partial B| / |B|$.
Further, $\overline{Z} = Z = {\rm Span}\, ( \mathbf{e} )$, $b_0[\mathbf{e}] =  d_0[\mathbf{e}] =1$, $\mathcal{P}^0_{\overline{V_\star}}\, \mathbf{e} =  \chi_B$,  
$(\mathbf{B}-\lambda I)^{-1}$ is multiplication by $(1 + \mu_0 - \lambda)^{-1}$ 
and $\mathcal{P}^0_{\overline{Z}}\chi_B=|B|\mathbf{e}$. 
Thus, via \eqref{6.27-2}, $d_0\big(\beta(\lambda) \mathbf{e},\,\mathbf{e}\big) = \lambda + (\lambda-1)^2 (1 + \mu_0 - \lambda)^{-1}|B|$ and so 
(see \eqref{betaform}) 
\be
\label{fimpint}
\beta_\lambda[\mathbf{e}] \,\,=\,\, \Phi(\lambda - 1), \quad \text{for}\quad \Phi(\mu) \,\,:=\,\,
 \frac{\mu}{\mu_0 - \mu} \Big(\mu_ 0 \,\,-\,\, \mu \big(1 \,-\, |B|\big)\,\Big). 
\ee
Then, from \eqref{finallimitspectralproblem}, for each $\xi$ the eigenvalues $\lambda$ of $\mathbb{L}_\xi$ are the two solutions of  the dispersion relation
\begin{equation}
\label{22.12.20e1}
A^{\rm hom}_{\rm pd} \xi \cdot \xi  \,\,=\,\, \Phi(\lambda - 1),
\end{equation}
i.e. $\lambda - 1$ are the roots of quadratic polynomial 
$p(\mu) = |\square \backslash B| \mu^2 - \mu\big(A^{\rm hom} \xi \cdot \xi + \mu_0\big) + \mu_0 A^{\rm hom} \xi \cdot \xi$.  
(Note that for every $\xi\in\mathbb{R}^n$, $p(\mu)$ has two distinct nonnegative roots, and 
$p(\lambda-1) = |B|^{-1} {\rm det} \left( M_\xi -\lambda D\right)$.) 
%{\rm det}\big( {\rm diag}(A^{\rm hom}_{\rm dp} \xi \cdot \xi, |\partial B|\big) - \mu D\big)$). 

Theorem \ref{ikthm2}, wherein consistently with our notation in the present example $\mathcal{L}_{\ep,\t}= \mathbf{L}_{\ep,\t}+I$, now implies the following for the spectral bands of $\mathcal{L}_\ep$:
\begin{theorem}	Let $\left\{ \lambda^{(k)}_{\ep,\theta}\right\}_{k\in \NN}$ be the eigenvalues of $\mathbf{L}_{\ep,\t}$ and $1 \le \Lambda^{(1)}_\xi < \Lambda^{(2)}_\xi$  the eigenvalues of $\mathbb{L}_\xi$, equivalently the roots of \eqref{22.12.20e1}. Then, there exists a positive constant $C$ independent of $\ep$, $\t$ and $k$ such that
 	\begin{gather}
	\label{est1.eigrotballs}
 	\Big| 1/\big(\lambda_{\ep,\theta}^{(k)}+1\big) \,-\,  1/\Lambda_{\theta / \ep}^{(k)}  \Big|\,\, \le\,\, C\ep, \ k = 1, 2,  \quad \text{and} \quad    
	\Big| 1/\big(\lambda_{\ep,\theta}^{(k)} +1\big)\Big| \,\,\le\,\, C\ep, \quad \forall  k \ge 3, \qquad \forall \t \in \square^*.
% 	\label{est2.eigrotballs}	\big| 1/\lambda_{\ep,\theta}^{(1)} -  1/(1 + \mu_0) \big| \le \ep^2\nu_\star^{-1}r^{-2},\qquad \big| 1/\lambda_{\ep,\theta}^{(k)} \big| \le \ep^2\nu_\star^{-1}r^{-2},\quad \forall  k\ge 2, \forall \t \in \Theta, |\t| \ge r>0.
 	\end{gather}
 \end{theorem}
 By direct  inspection of \eqref{22.12.20e1} we can determine the behaviour of $\Lambda^{(k)}_{\xi}$ in $\xi$. Indeed, we see from \eqref{fimpint} that 
$F$ has zeros at $\mu=0$ and $\mu=\mu_0 / (1 - |B|) = \mu_0 + \mu_1$ where $\mu_1 = |\partial B| / |\square \backslash B|>0$, and blows up (indicating at a ``resonance'') 
at $\mu=\mu_0$. Further, 
$F$  is non-negative and strictly increases from $0$ to $+\infty$ on $\left[0,\,\mu_0\right)$ and on $\left[ \mu_0 + \mu_1,\, +\infty\right)$, 
and $F$ is negative on $(-\infty,\, 0)$ and $\left(\mu_0,\, \mu_0 + \mu_1\right)$. Therefore, 
as $A^{\rm hom}_{\rm pd}$ is positive definite, 
it follows from \eqref{22.12.20e1} that $\Lambda_{\xi}^{(k)}$ are strictly increasing in $|\xi|$ with $0 \le\Lambda_{\xi}^{(1)} -1 < \mu_0$  and 
$\mu_0 + \mu_1 \le \Lambda^{(2)}_\xi -1 < +\infty$.  

Now, we can use $F$ to approximate ${\rm Sp}\, \mathcal{L}_\ep$. Indeed, recalling 
$
{\rm Sp}\, \mathcal{L}_\ep = \overline{\bigcup_{\t \in \Theta} {\rm Sp}\, \left( \mathbf{L}_{\ep,\t}\right)},
$
Theorem \ref{thm.limspecrep} for the present $\beta_\lambda$ and $\mathbf{B}$ gives 
\[
\overline{\bigcup_{\xi \in \RR^n} {\rm Sp}\, ( \mathbb{L}_\xi - I)} \,\,=\,\, {\big\{ \mu  \, \big| \, F(\mu) \ge 0 \big\}} \cup \left\{ \mu_0 \right\} 
\,\,=\,\,\overline{\big\{ \mu  \,\, | \,\, \Phi(\mu) \ge 0 \big\}} \,\,=\,\,  \left[0,\,\mu_0\right] \,\cup\, 
\left[\mu_0+\mu_1,\,+\infty\right), 
\]
i.e. $\left(\mu_0, \,\mu_0+\mu_1\right)$ is a gap in the limit collective spectrum $\overline{\bigcup_{\xi \in \RR^n} {\rm Sp}\, ( \mathbb{L}_\xi - I)}$. 
Combining this with %Theorem \ref{bivariate.spec} 
Corollary \eqref{c.collspec} 
provides the following results on the structure of ${\rm Sp}\, \mathcal{L}_\ep$:
\begin{theorem}\label{5.3.21e1}
	For every interval $ [a,b] \subset (-\infty,\infty)$  there exists $C_b\geq 0$, such that
	\begin{flalign*}
		d_{[a,b]}\Big({\rm Sp}\, \mathcal{L}_{\ep}\,,\,[0,\mu_0] \,\cup\, [\mu_0+\mu_1,\,+\infty) \Big) \,\,\,\le\,\,\, C_b \ep, \quad \forall\, 0<\ep<1.
		%d\Big([a,b]  \cap \overline{ \bigcup_{\theta \in \Theta} {\rm Sp}\, \mathcal{L}_{\ep,\t} }, \overline{\bigcup_{\xi \in \mathbb{R}^n} {\rm Sp}\, \mathbb{L}_\xi } \Big)  \le b(b+1)C_{11} \ep.
	\end{flalign*}
%for $C_b = b(b+1) C_{11}$. 
In particular, if  $\ep < \mu_1/(2C_b)$ then 
%$[a,b] \subset (\mu_0,\mu_0+\mu_1)$  then 
$\bigl[\mu_0 +C_b \ep,\,\mu_0+\mu_1-C_b\ep\bigr]$  is in a gap in the spectrum ${\rm Sp}\, \mathcal{L}_{\ep}$. % when $\ep < (b-a)/(2C_b)$.
\end{theorem}
We finish this example by observing that, routinely specialising to the present context the constructions of Section \ref{s.bivariate} leads to the associated 
 bivariate operator %given in Section \ref{s.bivariate} is 
of the form $\mathcal{L} = \mathcal{L}_0 + I$, where $\mathcal{L}_0$ is the operator in 
$L^2\left(\RR^n\right) \,\dot{+}\, L^2\big(\RR^n;\, {\rm Span}\,( \chi_B)\big)$, equipped with the standard $L^2\left(\RR^n \times \square\right)$ inner product,  generated by the form
\be
\label{2scformimpint}
\mathbb{Q}\big( u + v\chi_B,\, \tilde{u} + \tilde{v} \chi_B\big) \,\,\,: =\,
\int_{\RR^n} A^{\rm hom}_{\rm pd} \nabla u(x) \cdot\overline{ \nabla \tilde{u}(x)} \, {\rm d}x \,\,+\,\,  
|\partial B|\int_{\RR^n} v(x)\, \overline{\tilde{v}(x)}\, {\rm d}x,
%+ \int_{\RR^n} 
%\left( \begin{matrix}
%1 & |B| \\ |B| & |\partial B| + |B| 
%\end{matrix} \right) \left( \begin{matrix}
%u(x) \\ v(x)
%\end{matrix} \right) \cdot \overline{ \left( \begin{matrix}
%	\tilde{u}(x) \\ \tilde{v}(x)
%\end{matrix} \right)}  \, {\rm d}x
\ee
with the form domain $H^1\left(\RR^n\right) \,\dot{+}\, L^2\left(\RR^n;\, {\rm Span}\,\left( \chi_B\right)\right)$. 
%Clearly, $\mathcal{L}_0$ is isomorphic to the operator on $L^2(\RR^n) \times L^2(\RR^n)$ generated by the form
%\[
%\mathcal{Q}\big((u,v),(\tilde{u},\tilde{v})\big) : = \int_{\RR^n} A^{\rm hom}_{\rm dp} \nabla u(x) \cdot\overline{ \nabla \tilde{u}(x)} \, {\rm d}x +  |\partial B|\int_{\RR^n} v(x) \overline{\tilde{v}(x)}\, {\rm d}x,
%\]
%with form domain $H^1(\RR^n) \times L^2(\RR^n)$. 
Adjusting the derivation leading to Theorem \ref{thm.2scOpRes} for the present example, we have the following result.

\begin{theorem}
%	\label{thm.2scOpRes}
	For $0<\ep<1$ one has 
	\begin{equation}
		\label{dpcompe3-2}
		%\label{dpcompe1}
		\left\|\, \left(\mathcal{L}_\ep+I\right)^{-1} \,\,-\,\,  \mathcal{I}_\ep^* \left(\mathcal{L}_0+I\right)^{-1} \mathcal{P} \mathcal{I}_\ep \,\right\|_{L^2(\RR^n) \rightarrow L^2(\RR^n)} \,\,\,\le\,\,\, C_{11} \ep,
	\end{equation}
where the two-scale interpolation operator $\mathcal{I}_\ep$ and its adjoint $\mathcal{I}_\ep^*$ are given by \eqref{7.54-1} and \eqref{7.54-2} respectively, and 
$\mathcal{P}: L^2\left(\RR^n \times \square\right) \rightarrow L^2\left(\RR^n\right) \dot{+} L^2\big(\RR^n;\, {\rm Span}\,( \chi_B)\big)$ is the orthogonal projection. 
\end{theorem}
\begin{remark}
Notice that, as follows from \eqref{2scformimpint}, for $f_0 \in L^2(\RR^n \times \square)$, $(\mathcal{L}_0 + I)^{-1} \mathcal{P} f_0 = u(x) + v(x) \chi_0(y)$ where $(u,v) \in H^1(\RR^n) \times L^2(\RR^n)$ solve the coupled system
\[
\left\{\ \begin{aligned}
- \,{\rm div}\, A^{\rm hom}_{\rm dp} \nabla u(x)  \,+\, u(x)\, +\, |B| v(x) \,\,=\, \int_\square f_0(x,y) \, {\rm d}y, \quad x \in \RR^n ;\\
|B| u(x) \,+\, \big( |\partial B| \,+\, |B|\big) v(x)\,\, =\, \int_B f_0(x,y) \, {\rm d}y,  \quad x \in \RR^n.
\end{aligned} \right.
\]
We remark in passing that the above is nothing but the two-scale %homogenised 
limit system for the original problem \eqref{impintclass}. 
%$u_\ep$ solving the 
% = (1 - \chi_0(\tfrac{\cdot}{ \ep})) u^\ep_1 + \chi_0(\tfrac{\cdot}{ \ep}) u^\ep_2$. 
Furthermore, Theorem \ref{5.3.21e1} immediately implies the following estimate on the closeness of the spectra of the original and the limit problems: 
\begin{flalign*}
	d_{[a,b]}\big({\rm Sp}\, \mathcal{L}_\ep \,,\, {\rm Sp}\, \mathcal{L}_0 \big) \,\,\le\,\, C_b \ep.
\end{flalign*}
\end{remark}

% isomorphic to the operator in $H^1(\RR^n) \times L^2(\RR^n)$ generated by the form
%\[
%c
%\]


%%
%%
%we can apply our method and in particular Theorem  \ref{thm.maindiscthm} holds. To specify the objects therein, let us consider $e_Z = \chi_1 + \gamma \chi_2$, where $\gamma$ is such that  $e_Z$ is orthogonal to $
%V^\star_0 = {\rm span}(\{  \chi_2 \})$ with respect to $(\cdot,\cdot)_0$. Then 
%\[
%Z = {\rm span}(\{e_Z\}), \quad \&   \quad  a^{\rm trunc}_\t [e_Z] = A^{\rm hom}_{p} \t \cdot \t,
%\] 
%where $A^{\rm hom}_p$ is given by \eqref{dp.coef} in Example \ref{e.dp}. To see this last equality note that $a^{\rm trunc}_\t(\chi_2, \cdot) \equiv 0$ since $a_0'(\chi_2 ,\cdot) \equiv 0$, $a''_0(\chi_2,\cdot) \equiv 0$, plus note $ \t \cdot N \chi_1 = - \i \t \cdot \ourN^{(p)}$ in $\square \backslash B$ for $\ourN^{(p)}$ provided in  \eqref{dp.correctorproblem}. We leave it to the reader to follow the example of Example \ref{e.class} and \ref{e.dp} and spell out Theorem \ref{thm.maindiscthm} precisely in terms of $Z = {\rm span}(\{ e_Z \})$ and $V^\star_\t = {\rm span}(\{e^{\i\t\cdot y}\chi_2\})$. Let us focus on the asymptotics of the spectrum given by Section \ref{s:resolv}, in particular Theorem \ref{ikthm} and Remark \ref{r.ikth77}.
%
%In this context,  $\mathcal{H} = L^2(\square)$,  $c$ is the standard $L^2$ inner product $(\cdot,\cdot)$, $K_c =1$ (see \eqref{ik2}) and $Z \overset{\cdot}{+} V^\star_\t= {\rm span}(\{e_z,e^{\i \t \cdot y}\chi_2\})$. 
%%Note that since $Z \overset{\cdot}{+} V^\star_\t$ is two-dimensional  we need only  consider the first two eigenvalues of $\mathcal{L}_{\ep,\t}$ (the self-adjoint operator in $L^2(\Box)$ generated by the form $\ep^{-2} a_\t[\cdot] + \| \cdot \|_\t^2$ ). To this end notice that, for this example, 
%The form $s$ (given in \eqref{ikspectr200}) is equivalent to
%\begin{equation}
%\label{sformballs}
%s\big( (c,d),(\tilde{c},\tilde{d}) \big) : = \ep^{-2} A^{\rm hom}_p \t \cdot \t c \cdot\overline{\tilde{c}}+ \big( c e_Z + de^{\i \t \cdot y}\chi_2 , \tilde{c} e_Z + \tilde{d}  e^{\i \t \cdot y}\chi_2\big)_0 , \quad (c, d), (\tilde{c},\tilde{d} ) \in \CC^2,
%\end{equation}
%and $\mathbf{L}_{\ep,\t}$ (the self-adjoint operator generated by $s$ on $Z \overset{\cdot}{+}V^\star_\t$) has spectrum 
%$\{ \Lambda^{(1)}_{\ep,\t},  \Lambda^{(2)}_{\ep,\t}\}$ where
%\begin{equation}\label{balleigs.e1}
%\begin{aligned}	\Lambda^{(1)}_{\ep,\t} & = \min_{(c,d) \in \CC^2} \frac{ s[(c,d)]}{ \| c e_Z + de^{\i \t \cdot y}\chi_2\|^2 },  \qquad 
%%= 1 + \mu^{(1)}_{\ep,\t}, 
%%\qquad \mu^{(1)}_{\ep,\t} : = \min_{(c,d) \in \CC^2} \frac{\ep^{-2} A^{\rm hom}_p \t \cdot \t |c|^2 +  J[c e_Z + d  e^{\i \t \cdot y}\chi_2 ]}{ \| c e_Z + de^{\i \t \cdot y}\chi_2\|^2 }, \\
%\Lambda^{(2)}_{\ep,\t} & = \max_{(c,d) \in \CC^2} \frac{s[(c,d)]}{ \| c e_Z + de^{\i \t \cdot y}\chi_2\|^2 }.
%%= 1 + \mu^{(2)}_{\ep,\t}, \qquad  \mu^{(2)}_{\ep,\t} : = \max_{(c,d) \in \CC^2} \frac{\ep^{-2} A^{\rm hom}_p \t \cdot \t |c|^2 +  J[c e_Z + d  e^{\i \t \cdot y}\chi_2 ]}{ \| c e_Z + de^{\i \t \cdot y}\chi_2\|^2 }.
%\end{aligned}
%\end{equation}
%Note, for $\t \neq 0$, $\mathbf{D}_\t$ (the self-adjoint operator in $\overline{V_\t} = {V_\t} = {\rm span}(\{e^{-\i\t\cdot y}\chi_2\}) $  generated by the closed positive sesquilinear form $(\cdot,\cdot)_\t$ with form domain $V_\theta$) has spectrum $\{ 1+ \mu_0 \}$ for $\mu_0 = |\partial B| / |B|$ where $|\partial B|$ and $|B|$ respectively denote the surface area and volume of $B$.
%
%Theorem \ref{ikthm} and Remark \ref{r.ikth77} state, in this setting, the following.
%\begin{theorem}	Consider $\{ \lambda^{(k)}_{\ep,\theta}\}_{k\in \NN}$ the eigenvalues of $\mathcal{L}_{\ep,\t}$, $\mu_0 = |\partial B| / |B|$, and $\Lambda^{(1)}_{\ep,\t}$, $\Lambda^{(2)}_{\ep,\t}$  in \eqref{balleigs.e1}. Then, there exists a positive constant $C$ independent of $\ep$, $\t$ and $k$ such that
%	\begin{gather}\label{est1.eigrotballs}
%	\big| 1/\lambda_{\ep,\theta}^{(i)} -  1/\Lambda_{\ep,\theta}^{(i)}  \big| \le C\ep, \qquad    \big| 1/\lambda_{\ep,\theta}^{(n)} \big| \le C\ep, \quad \ i = 1,2,\, \forall  n \ge 2, \, \forall \t \in \Theta, \\
%\label{est2.eigrotballs}	\big| 1/\lambda_{\ep,\theta}^{(1)} -  1/(1 + \mu_0) \big| \le \ep^2\nu_\star^{-1}r^{-2},\qquad \big| 1/\lambda_{\ep,\theta}^{(k)} \big| \le \ep^2\nu_\star^{-1}r^{-2},\quad \forall  k\ge 2, \forall \t \in \Theta, |\t| \ge r>0.
%	\end{gather}
%\end{theorem}
%On the one hand, inequality \eqref{est2.eigrotballs} provides better estimates outside a given neighbourhood of the origin $\t=0$ and in particular informs us that the first spectral band function $E^1_\ep(\t) := \lambda^{(1)}_{\ep,\t}$ is essentially  flat. While on the other hand, inequality \eqref{est1.eigrotballs} provides a continuous-in-$\t$ description of the asymptotics of $\lambda^{(1)}_{\ep,\t}$, $\lambda^{(2)}_{\ep,\t}$ on the whole of $\Theta$. This description for the all $\t$ can be used, for example,  to determine if $\min\limits_{\t\in \Theta} \lambda^{(2)}_{\ep,\t} > \max\limits_{\t\in \Theta} \lambda^{(1)}_{\ep,\t}$, i.e. to determine if there exists a spectral gap. 
%
%Before establishing that a spectral gap exists, we mention that the $\t$-continuous approximations $\Lambda^{(i)}_{\ep,\t}$ can be simplified further. Indeed, we demonstrate at the end of this example that \begin{equation}\label{ballseig.final}
%| 1 / \Lambda^{(i)}_{\ep,\t} - 1/ (1 +\mu^{(i)}_{\ep,\t}) | 
%%\le \tfrac{3}{2} \| e_Z\| \ep + | 1 / \Lambda^{(i)}_{\ep,\t} - 1/ \tilde{\Lambda}^{(i)}_{\ep,\t}  |
% \le \big(\tfrac{3}{2} \| e_Z\|  + \|e_Z\|_0 \big) \ep, \qquad i = 1,2, \, \forall \t \in \Theta,
%\end{equation}
%where
%%\[
%%\begin{aligned}	M^{(1)}_{\ep,\t} & = \min_{(c,d) \in \CC^2} \frac{ \ep^{-2} A^{\rm hom}_p \t \cdot \t |c|^2 + \|c e_Z + d\chi_2\|_0^2}{ \| c e_Z + d\chi_2\|^2 }  = 1 + \mu^{(1)}_{\ep,\t}, \\
%%M^{(2)}_{\ep,\t} & = \max_{(c,d) \in \CC^2}\frac{ \ep^{-2} A^{\rm hom}_p \t \cdot \t |c|^2 + \|c e_Z + d\chi_2\|_0^2}{ \| c e_Z + d\chi_2\|^2 }   = 1 + \mu^{(2)}_{\ep,\t},
%%\end{aligned}
%%\]
%\[
% \mu^{(1)}_{\ep,\t} : = \min_{(c,d) \in \CC^2\backslash \{ 0\}} \frac{\ep^{-2} A^{\rm hom}_p \t \cdot \t |c|^2 +  J[c e_Z + d  \chi_2 ]}{ \| c e_Z + d\chi_2\|^2 }, \ \& \  \mu^{(2)}_{\ep,\t} : = \max_{(c,d) \in \CC^2\backslash \{ 0\}} \frac{\ep^{-2} A^{\rm hom}_p \t \cdot \t |c|^2 +  J[c e_Z + d  \chi_2 ]}{ \| c e_Z + d\chi_2\|^2 }. 
% \]
%Now \eqref{est1.eigrotballs}  and \eqref{ballseig.final} imply that there exists a constant $C>0$ such that
%\begin{equation}
%\label{est3.eigrotballs}
%\big| 1/\lambda_{\ep,\theta}^{(i)} -  1/(1 +\mu^{(i)}_{\ep,\t})  \big| \le C\ep,\qquad i = 1,2, \, \forall \t \in \Theta.
%\end{equation}
%Let us compute  $\mu^{(i)}_{\ep,\t}$ . Note, if $\mu_i : =  \mu^{(i)}_{\ep,\t}$  is attained by  $ u_i = c_i e_Z + d_i  \chi_2 = c_i\chi_1 + (\gamma c_i+d_i)\chi_2$  then $\mu_i, \mathbf{x}_i=(c_i,\gamma c_i+d_i)\in \CC^2$ solve the eigenvalue-eigenvector problem 
%\[
%\left( \begin{matrix}
%\ep^{-2} A^{\rm hom}_p \t \cdot \t  & 0 \\ 0 & |\partial B | 
%\end{matrix} \right) \mathbf{x}_i = \mu_i \left( \begin{matrix}
%1 & |B| \\ |B| & |B| 
%\end{matrix} \right) \mathbf{x}_i.
%\]
%%Consequently
%%\[
%%\begin{aligned}
%%&\mu^{(1)}_{\ep,\t} = \frac{|\partial B| +\tfrac{1}{\ep^2}  A^{\rm hom}_p \t \cdot \t |B|}{2 |B| |\Box \backslash B|} \left( 1 - \sqrt{1-4\ep^2 \frac{|\partial B||B||\Box \backslash B|  A^{\rm hom}_p \t \cdot \t}{(\ep^2 |\partial B| +  A^{\rm hom}_p \t \cdot \t|B|)^2}}\right), \\
%%&\mu^{(2)}_{\ep,\t} = \frac{|\partial B| +\ep^{-2}  A^{\rm hom}_p \t \cdot \t |B|}{2 |B| |\Box \backslash B|} \left(1 + \sqrt{1-4\ep^2 \frac{|\partial B||B||\Box \backslash B|  A^{\rm hom}_p \t \cdot \t}{(\ep^2 |\partial B| +  A^{\rm hom}_p \t \cdot \t|B|)^2}}\right).
%%\end{aligned}
%%\]
%That is 
%\[
%\mu^{(i)}_{\ep,\t} = \Phi_\pm( \ep^{-2} A^{\rm hom}_p \t \cdot \t), \quad \text{ for } \Phi_\pm(x) = \frac{\mu_0 + x \pm \sqrt{(\mu_0 + x - 2 | \Box \backslash B| \mu_0)^2 + 4 |B|| |\Box \backslash B|\mu_0^2}}{2 |\Box \backslash B|}.
%% \Phi_\pm(x) = \frac{\mu_0 + x \pm \sqrt{(\mu_0 + x)^2 - 4 | \Box \backslash B| \mu_0x}}{2 |\Box \backslash B|}.
%\]
%Since $(\mu_0 - 2 | \Box \backslash B| \mu_0)^2 + 4 |B|| |\Box \backslash B|\mu_0^2 = \mu_0^2$ then $\Phi_-(0)=0$ and $\Phi_+(0)=\tfrac{|\partial B|}{|B||\Box \backslash B|}  = \mu_0 + \tfrac{|\partial B|}{|\Box \backslash B|} $. Moreover, it is clear that $\Phi_{\pm}$ are strictly increasing on $[0,\infty)$ and $\Phi_-$ asymptotes to $ \mu_0 = \tfrac{|\partial B|}{|B|}$ as $x \rightarrow +\infty$ while  $\lim_{x \rightarrow +\infty}\Phi_+(x) = +\infty$. Consequently,\[
%\mu^{(1)}_{\ep,\t} < \tfrac{|\partial B|}{|B|}, \quad \text{and} \quad \tfrac{|\partial B|}{|B|} + \tfrac{|\partial B|}{|\Box \backslash B|} = \mu^{(2)}_{\ep,0} \le \mu^{(2)}_{\ep,\t}, \quad \forall \t \in \Theta,\]
%and the following analogue of Theorem \ref{ikthm20} holds.
%\begin{theorem}\label{thm.rotballsgap}
%Let $C$ be the constant in \eqref{est3.eigrotballs}. Then 
% $
%\Big( \tfrac{|\partial B|}{| B|} + C\ep, \tfrac{|\partial B|}{| B|} + \tfrac{|\partial B|}{|\Box \backslash B|} - C\ep\Big) 
%$
%is a gap of length $\frac{|\partial B|}{|\Box \backslash B|} - 2C \ep$ in the collective spectrum $\bigcup\limits_{\t,k} \lambda^{(k)}_{\ep,\t}$.
%\end{theorem}
%
%Let us finish with the proof of  \eqref{ballseig.final}. Fix $\t\in \theta$ and consider the form
%
%\[
%s_1\big( (c,d),(\tilde{c},\tilde{d}) \big) : = \ep^{-2} A^{\rm hom}_p \t \cdot \t c \cdot\overline{\tilde{c}}+  \big( c e_Z + d\chi_2 , \tilde{c} e_Z + \tilde{d}  \chi_2\big)_0, \qquad (c,d), (\tilde{c}, \tilde{d}) \in \CC^2.
%\]
%then, for $u : =  c e_Z + e^{\i\t\cdot y}d \chi_2$, $u_1 : =  c e_Z + d \chi_2$, 
%\[
%\begin{aligned}
%| s[(c,d)] - s_1[(c,d)] | & = \big| \big( (1-e^{\i \t\cdot y} ) c e_Z, u_1 \big)_0 + \big(  u, (1-e^{\i \t\cdot y} ) c e_Z\big)_0 \big| \le |\t| |c| \| e_Z\|_0 (\|u_1 \|_0 + \|  u \|_0) \\ 
%& \le \| e_Z\|_0\ep (\ep^{-2} A^{\rm hom}_p \t \cdot \t |c|^2)^{1/2} (\|u_1 \|_0 + \|  u \|_0) \le \tfrac{\| e_Z\|_0}{2} \ep \big( s[(c,d)] + s_1[(c,d)] \big),
%\end{aligned}
%\]
%where in the last inequality we recall the form $s$ in \eqref{sformballs}.
%Therefore, it readily follows that 
%\begin{equation}\label{ballseig.proofe1}
%\big| \Lambda^{(i)}_{\ep,\t} - \tilde{\Lambda}^{(i)}_{\ep,\t} \big| \le \tfrac{\| e_Z\|_0}{2} \ep \big( {\Lambda}^{(i)}_{\ep,\t} + \tilde{\Lambda}^{(i)}_{\ep,\t} \big), \quad  i=1,2,
%\end{equation}
%where $\tilde{\Lambda}^{(1)}_{\ep,\t}$, $\tilde{\Lambda}^{(2)}_{\ep,\t}$ are the minimum and maximum respectively of $s_1[(c,d)] / \| c e_Z + de^{\i \t \cdot y}\chi_2\|^2$ over $\CC^2\backslash \{ 0\}$. Similarly, we compute
%\[
%\begin{aligned}
%\big| \|u \|^2 - \|u_1\|^2 \big| & = \big| \big( (1 - e^{\i \t\cdot y} c e_Z , u_1 \big) + \big( u,(1 - e^{\i \t\cdot y} c e_Z \big) \big| \le |\t| |c| \| e_Z\| (\| u_1\| + \|u \|) \\ & \le \| e_Z\| \ep ( \tfrac{1}{2} s[(c,d)]+ \| u_1\| ), 
%\end{aligned}
%\]
%and so
%\begin{equation}\label{ballseig.proofe2}
%\big| 1 /\tilde{\Lambda}^{(i)}_{\ep,\t} - 1/ M^{(i)}_{\ep,\t} \big| \le \| e_Z\| \ep \big( \tfrac{1}{2} + 1/ M^{(i)}_{\ep,\t} \big), \quad  i=1,2,
%\end{equation}
%where $M^{(1)}_{\ep,\t}$, $M^{(2)}_{\ep,\t}$ are the minimum and maximum respectively of $s_1[(c,d)] / \| c e_Z + d\chi_2\|^2$ over $\CC^2\backslash \{ 0\}$. Note $M^{(i)}_{\ep,\t} = 1 + \mu^{(i)}_{\ep,\t}$, $i=1,2$. Therefore, \eqref{ballseig.proofe1}, \eqref{ballseig.proofe2}, and the fact all the numbers $\tilde{\Lambda}^{(i)}_{\ep,\t}$, $\Lambda^{(i)}_{\ep,\t}$, $M^{(i)}_{\ep,\t}$ are larger than 1, imply \eqref{ballseig.final}.
%

%\subsubsection{Approximation and effective equations}
%We demonstrated above that the assumptions  \eqref{KA}-\eqref{distance} hold.
%% and in particular that the approximation theorem Theorem \ref{thm.maindiscthm} holds. 
%Consequently,  Theorem \ref{thm:contV} informs us that the solution $v_\t\in H^1_0(B)$ to
%\begin{equation}\label{dphom.e0}
%-(\nabla + \i\t) \cdot (\nabla+\i\t) v_\t + v_\t = F_{\Ep,\t} \quad \text{ in $B$},
%\end{equation}
%is an approximation for $w_{\Ep,\t}$ the solution to \eqref{dp.varp} 
%for $\t \in \Box^* \backslash B_r$. Indeed, we have the estimate 
%\begin{equation}\label{dp.outsideest}
%\| (\nabla + \i \t) (w_{\Ep,\t} - v_\t )\|^2 + \| w_{\Ep,\t} - v_\t \|^2 \le  \Ep^4 (\nu^\star r^2)^{-1} \| F_{\Ep,\t} \|^2.
%\end{equation}
%In turn, Theorem \ref{thm.maindiscthm}  provides the most-reduced diagonalised approximation to $w_{\Ep,\t}$ in a neighbourhood of the discontinuity $\t = 0$. Therefore, in principle,  upon 
%applying the inverse transform $\Gamma_\Ep^{-1} U^{-1}$, we arrive at an approximation to $w_\Ep$ the solution to \eqref{dp.differentialPDEdp}.  However, in high-contrast problems (that is those with genuine non-trivial $V^\star_\t$) the $\Ep$-dependence of these approximations is not immediately explicit. To arrive at effective equations for \eqref{dp.differentialPDEdp} with a more explicit $\Ep$ dependence it is more convenient to consider a different approximation to $w_{\Ep,\t}$ in $B_r$.  Precisely, in place of Theorem \ref{thm.maindiscthm} we use the following result.
%\begin{proposition} \label{twoscaleapprox}Fix $\t \in \Box^*$, let $w_{\Ep,\t}$ solve \eqref{dp.varp} and  $\alpha_\t \in \CC$, $v \in L^2(\Box^*;H^1_0(B))$ (uniquely) solve
%	\begin{gather}
%	\Ep^{-2} A^{hom}_{dp} \t \cdot \t\, \alpha_\t + \alpha_\t + (v(\t,\cdot) ,1) = (U \Gamma_\Ep f(\t,\cdot), 1),  \label{dphom.e3}\\ 
%	-(\nabla_y + \i \t) \cdot (\nabla_y + \i \t) v(\t,y) + \alpha_\t + v(\t,y) = U \Gamma_\Ep f(\t,y), \quad y \in B.\label{dphom.e4}
%	\end{gather}
%	Then, there exist positive constants $r$, $C_1, C_0$ independent of $\ep$ and $\t$ such that
%	\begin{gather}
%	\label{eq.maindiscthmdp1}
%	\Ep^{-2}a_\t[w_{\Ep,\theta} -(1  - \i \t \cdot \ourN_{dp} ) \alpha_\t ]+	\| w_{\Ep,\theta} -  (1  - \i \t \cdot \ourN_{dp} ) \alpha_\t   \|^2_{L^2(\Box \backslash B)} \le  \Ep^2  C_1\| f\|^2, \\
%	\label{eq.maindiscthmdp2}
%	\| (\nabla + \i \t) (w_{\Ep,\t} - \alpha_\t - v(\t,\cdot) )\|^2 + \| w_{\Ep,\t} - \alpha_\t - v(\t,\cdot) \|^2   \le  \Ep^2  C_0\| f\|^2,
%	\end{gather}
%	for all $\t \in B_r$, $\Ep>0$.
%\end{proposition}
%The proof of Proposition \ref{twoscaleapprox} is at the end of the section. Now, utilising the above approximation we readily find the following effective equations for \eqref{dp.differentialPDEdp}.
%\begin{theorem} Let $w_\Ep$ solve \eqref{dp.differentialPDEdp} and the pair $(w,v_0) \in H^1(\RR^d) \times  H^1_0(F_0)$, $F_0 : = \cup_{z \in \ZZ^d} (B+z)$, solve
%	\begin{gather}\label{dphom.final1}
%	- {\mathrm{div}}_x\, A^{hom}_{dp} \nabla_x w(x) + w(x) +  \mathcal{P}_\Ep v_0(\tfrac{x}{\Ep})  = \mathcal{P}_\Ep f(x), \quad x\in \mathbb{R}^d, \\ 
%	-\Delta_y v_0(y) + w(\Ep y) + v_0(y) = f(\Ep y), \quad y \in F_0, \label{dphom.final2}
%	\end{gather}
%	where we recall the smoothing operator $\mathcal{P}_\Ep : = \Gamma^{-1}_\Ep \mathcal{F}^{-1} \chi \mathcal{F} \Gamma_\Ep= \mathcal{F}^{-1} (\Gamma_\Ep \chi) \mathcal{F}$, $\mathcal{F}$ denotes the Fourier transform and $\chi$ is the characteristic function of $B_r$. 	Then, there exists constant $c_0,$ $c_1$ independent of $f$ and $\Ep$ such that
%	\begin{align*}
%	%\label{dp.H1est}
%	&	\Vert w_\Ep - w - \Ep \ourN_{dp}(\tfrac{\cdot}{\Ep}) \cdot \nabla w  \Vert_{H^1(F^1_\Ep)} \le \Ep c_1 \Vert f \Vert_{L^2(\mathbb{R}^d)},  \qquad F^1_\Ep : = \RR^d \backslash \Ep F_0, \\
%	%\label{dp.L2est}
%	&	\Vert w_\Ep - w - v_0(\tfrac{\cdot}{\Ep})  \Vert_{L^2(\mathbb{R}^d)} \le \Ep c_0 \Vert f \Vert_{L^2(\mathbb{R}^d)}.
%	\end{align*}
%\end{theorem}
%\begin{proof}
%	From \eqref{dphom.e0} and Proposition \ref{twoscaleapprox} we have the approximation 
%	\[
%	\chi(\t)(1 + \i \t \cdot  \ourN_{dp}) \alpha_\t +\chi(\t) v(\t,y) + (1-\chi(\t))v_\t(y), \quad \t \in \Box^*, y \in Q.
%	\]
%	We can concisely write $u^{ap}$ and \eqref{dphom.e0},\eqref{dphom.e3}-\eqref{dphom.e4} as $(1 + \i \t \cdot N_{dp})\tau + \mathbf{v}$, where  $\tau(\t) = \chi(\t) \alpha_\t$, $\mathbf{v}(\t,y) = \chi(\t) v(\t,y) + (1-\chi(\t))v_\t(y)$, and 
%	\begin{gather}
%	\ep^{-2} A^{hom}_{dp} \t \cdot \t\, \tau(\t) + \tau(\t) + \chi(\t)(\mathbf{v}(\t,\cdot) ,1) = \chi(\t) (U \Gamma_\ep f(\t,\cdot), 1), \quad \t \in \mathbb{R}^d, \label{dphom.e5}\\ 
%	-(\nabla_y + \i \t) \cdot (\nabla_y + \i \t) \mathbf{v}(\t,y) + \tau(\t) + \mathbf{v}(\t,y) = U \Gamma_\ep f(\t,y), \quad y \in B, \t \in \Box^*.\label{dphom.e6}
%	\end{gather}
%	The error estimates \eqref{dp.outsideest},\eqref{eq.maindiscthmdp1},\eqref{eq.maindiscthmdp2} are compactly written as
%	\begin{equation}\label{dp.compactest}	
%	\begin{aligned}
%	&\Ep^{-2}a_\t[w_{\Ep,\theta} - (1 + \i \t \cdot  \ourN_{dp}) \tau]+	\| w_{\Ep,\theta} - (1 + \i \t \cdot  \ourN_{dp}) \tau \|^2_{L^2(\Box \backslash B)} \le  \Ep^2  C_1\| f\|^2, \\
%	&\| (\nabla + \i \t) (w_{\Ep,\t} -   \tau  - \mathbf{v})\|^2 + \| w_{\Ep,\t} -  \tau  - \mathbf{v} \|^2  \le  \Ep^2  c_0\| f\|^2, \quad c_0 = \max\{ C_0, (\nu^* r^2)^{-1} \}.
%	\end{aligned}
%	\end{equation}
%	
%	Let us show that  $v_0 := U^{-1} \mathbf{v}$ and  $w := \Gamma_\Ep^{-1} \mathcal{F}^{-1} \tau$.
%	Since  $( U g(\t,\cdot) ,1) = \mathcal{F} g(\t)$, for $g \in L^2(\RR^d)$,  then upon setting $v = U v_0$, and  $\tau = \mathcal{F}\Gamma_\Ep w$ in equation \eqref{dphom.e5} gives
%	\[
%	\Ep^{-2}A^{hom}_{dp} \t \cdot \t\, \mathcal{F} \Gamma_\Ep w + \mathcal{F}\Gamma_\Ep w + \chi \mathcal{F} v_2  = \chi \mathcal{F}\Gamma_\Ep f,
%	\]
%	then     taking inverse of $\mathcal{F}$ followed by the inverse $\Gamma_\ep$  above gives \eqref{dphom.final1}. Furthermore,  $(\partial_{y_j} + \i \t_j) U v_0 = U \partial_{j} v_0$, $j=1,\dots,d$ and since $\mathbf{v} = U v_0$ then equation \eqref{dphom.e6} becomes
%	\[
%	- U \Delta v_0 + \tau + U v_0 = U \Gamma_\ep f.
%	\]
%	Moreover, $U^{-1} \tau = \mathcal{F}^{-1}\tau$ since ${\rm supp}\, \tau \subset \Box^*$ and so $U^{-1} \tau =  \Gamma_\ep w$. Therefore, taking the inverse Gelfand in the above equation gives \eqref{dphom.final2}.
%	
%	
%	Recalling the forms \eqref{abcdp}, the error estimates follow from \eqref{dp.compactest} and the identities 
%	\[
%	\begin{aligned}
%	&\Gamma_\Ep^{-1} U^{-1} (\tau + \mathbf{v})(x)= \Gamma_\Ep^{-1} U^{-1} ( U \Gamma_\Ep w + U v_0 )(x)= w(x) + v_0(\tfrac{x}{\Ep}),\\
%	&\Gamma_\Ep^{-1} U^{-1} (1 + \i \t \cdot  \ourN_{dp}) \tau (x) =\Gamma_\Ep^{-1} U^{-1} (1 + \i \t \cdot  \ourN_{dp}) U \Gamma_\Ep w(x) = w(x) + \Ep  \ourN_{dp}(\tfrac{x}{\Ep}) \cdot \nabla w(x).
%	\end{aligned}
%	\]
%\end{proof}
%
%\begin{proof}[Proof of Proposition \ref{twoscaleapprox}]
%	The proof follows by our method exposed in Sections \ref{section:discV} and \ref{sec.2dif} and as such we shall only sketch the proof. 
%	
%	Let $u_{\ep,\t}$ be the solution to \eqref{p1} and    $z_1 \in Z$, $v_1 \in V^\star_\t$ solve	
%	\begin{equation}
%	\label{dp.z1prob}
%	\ep^{-2} a_\t\big(  \M z_1, \M \tilde{z}\big) + \big(v_1 + \PW \M z_1, \tilde{v} + \PW \M \tilde{z}\big)_\t = \langle f, \tilde{v} + \PW \M \tilde{z} \rangle, \quad \forall \tilde{z}\in Z, \, \forall \tilde{v} \in V^\star_\t.
%	\end{equation}
%	Then
%	\[
%	\ep^{-2} a_\t[u_{\ep,\t} -  \PW \M z_1] + \| u_{\ep,\t} - v_1 - \PW \M z_1 \|_\t^2  \le C \ep^2 \| f \|^2_{*\t}. 
%	\]
%	This result follows from arguing as in Section \ref{section:discV} Theorem \ref{thm1}. Next, we argue as in Section \ref{sec.2dif} and provide  appropriate truncations to the forms present in \eqref{dp.z1prob}. Indeed, by Proposition \ref{prop.ggg}, its clear that for $z_2 \in Z$, $v_2 \in V^\star_\t$ that solve
%	\begin{equation}
%	\label{dp.z2prob}
%	\ep^{-2} a_\t^{\rm trunc}\big(  z_2, \tilde{z}\big) + \big(v_2 + z_2, \tilde{v} +  \tilde{z}\big)_\t = \langle f, \tilde{v} + \tilde{z} \rangle, \quad \forall \tilde{z}\in Z, \, \forall \tilde{v} \in V^\star_\t,
%	\end{equation}
%	we have
%	\[
%	\ep^{-2} a_\t[ \PW \M z_1 - (I + \i \t \cdot N) z_2] + \|v_1 - \PW \M z_1 - v_2 - (I + \t\cdot N)z_2) \|_\t^2  \le C \ep^2 \| f \|^2_{*\t}. 
%	\]
%	Lastly, in the particular context of double porosity, recall the fact that $Z + V_\t^\star = \CC + H^1_0(B)$ and \eqref{dp.ahom}: the identification of $a^{\rm trunc}_\t$ with $A^{\rm hom}_{dp}$. Then we consider $\alpha \in \CC$, $v_3 \in H^1_0(B)$ that solve 
%	\begin{equation}\label{dphom.e2}
%	\ep^{-2} A^{\rm hom}_{dp} \t \cdot \t \alpha \cdot \overline{ {\tilde \alpha}} + b_\t^{(1)}(\alpha + v_3, \tilde{\alpha} + \tilde{v}) = \l f, \tilde{\alpha} + \tilde{v}\r, \quad \forall \tilde{\alpha} \in \CC, \, \tilde{v} \in  H^1_0(B),
%	\end{equation}
%	for
%	\[
%	b^{(1)}_\t(\alpha + v, \tilde{\alpha} + \tilde{v}) : = (\alpha+ v, \tilde{\alpha} + \tilde{v}) + \big(  (\nabla + \i \t) v, (\nabla + \i \t) \tilde{v} \big),
%	\]
%	and we can argue as in Section \ref{sec.2dif} to demonstrate that
%	\[
%	\ep^{-2} a_\t[ (I + \i \t \cdot N) (z_2 - \alpha \mathbf{e})] + \|v_2 -v_3 + (I + \t\cdot N)(z_2 - \alpha\mathbf{e})) \|_\t^2  \le C \ep^2 \| f \|^2_{*\t}, 
%	\]
%	recalling $\mathbf{e}$ is the function equal to $1$. Indeed, this inequality will readily follow from the identity
%	\begin{equation}\label{dphom.e1}
%	b^{(1)}_\t({\alpha} + {v},\tilde{\alpha} + \tilde{v})  - b_\t({\alpha} + {v},\tilde{\alpha} + \tilde{v})  = - \big(\i \t {a},  (\nabla + \i \t)(\tilde{a} + \tilde{v})\big) - \big((\nabla + \i \t) {v},\i\t \tilde{\alpha}\big).
%	\end{equation}
%	Note that \eqref{dphom.e2} has a unique solution since we readily see from  \eqref{dphom.e1} that $b^{(1)}_\t$ is an equivalent inner product on $\CC+H^1_0(B)$ for small enough $\t$.  Now recall the objects in, and directly above, \eqref{abcdp}). Upon setting $\tilde{v} =0$ in \eqref{dphom.e2} gives
%	\[
%	\alpha = \frac{ ( \mathcal{U}\Gamma_\Ep f(\t,\cdot), 1) - (v(\t,\cdot),1)}{(\Ep^2 -1) A^{\rm hom}_{dp} \t \cdot \t + 1},
%	\]
%	and  by arguing as in Example \ref{e.class}  (to justify \eqref{classicalEpwiggle}) we deduce that
%	\[
%	|\alpha - \alpha_\t | \le \Ep^2 \| f \|.
%	\]
%	Finally,  by the above inequality, \eqref{dphom.e4} and setting $\tilde{\alpha} =0$ in \eqref{dphom.e2} and the above inequality we can deducate that $v(\theta,\cdot)$ is close $v_3$, i.e.
%	\[
%	\| (\nabla + \i \t)( v_3 - v(\t,\cdot)) \|^2 + \|  v_3 - v(\t,\cdot) \|^2 \le \Ep^4 \| f\|^2.
%	\] The proof of the proposition is complete. 
%\end{proof}
%%------------------------------------------------------------------------------------------------------------------------------------------------------------------------------------------------------------------------------------------------------------------------------------------------------------------------------------------------------------------------------------------------------------------------------------------------------------------------------------------------------------------------------------------------------------------------------------------------------------------------------------------------------------------------------------------------------------------------------------------------------------------------------------------------------
%\subsection{Double-porosity model of linear elasticity}
%\label{s.edp}
%Here, we consider the high-contrast  resolvent problem: 
%\begin{equation}
%\label{differentialPDEdp}
%\left\{ \begin{aligned}
%& \text{Find $u_\Ep \in H^1(\RR^d;\RR^d)$ such that} \\
%&-{\rm div}\big( C_\Ep \left(\tfrac{x}{\Ep} \right) e(u_\Ep)(x) \big) + u_\Ep(x) = f(x), \qquad x \in \RR^d, \qquad e(u)_{ij} = \tfrac{1}{2}\big( u_{i,j}+u_{j,i} \big),
%\end{aligned} \right.
%\end{equation}
%for a given $\lambda \in (0,\infty)$, $f \in L^2(\RR^d;\RR^d)$ and periodic tensor coefficients $C_\Ep$ of the form
%$$
%C_\Ep(y) = \left\{ 
%\begin{array}{lr}
%C^{(1)}(y) & y \in Q_1, \\[5pt]
%\Ep^2 C^{(0)}(y) & y \in Q_0.
%\end{array} \right.
%$$
%Here $Q_r$, $r=0,1$, are subsets of the unit square $\Box=[0,1]^d$ such that  $\Box = Q_0 \cup Q_1$, with $Q_0$ having a Lipschitz boundary and additionally  $\RR^d \backslash \cup_{z \in \ZZ } (Q_0+z)$ forms a connected set.  The tensor coefficients  satisfy the following standard  conditions:
%\begin{equation}
%\label{dpcoeffs}
%\begin{aligned}
%%& C^{(r)}_{ijkl} \in L^\infty(\Box_r), \qquad 
%&C^{(r)}_{ijkl} = C^{(r)}_{klij} = C^{(r)}_{klji}, \quad i,j,k,l\in \{1,\ldots,d\},  \\
%& \gamma^{-1} | \Xi|^2 \le C^{(r)}_{ijkl}(y)\Xi_{kl}\Xi_{ij} \le \gamma |\Xi|^2 \quad \forall y \in Q_r, \   \text{for all symmetric $d \times d$ matrices $\Xi$,}
%\end{aligned}
%\end{equation}
%for some constant $\gamma \ge 1$.  Again, arguing as in the above example it is clear that without loss of generality we can suppose $\Ep <1$.
%
%
%
%
%As in Example \ref{e.class}, following the rescaling $\Gamma_\Ep$ and then applying the Gelfand transform $\mathcal{U}$, we deduce that $u_{\Ep, \t} (\cdot) : = U \Gamma_\Ep u_\Ep(\Theta, \cdot)$, for a.e. $\t \in \Theta : = \Box^*$, solves
%\begin{equation}
%\label{dpel.varp}
%\Ep^{-2} \int_{Q_1} C^{(1)} e(e^{\i \t \cdot y}u_{\Ep,\t}) : \overline{e(e^{\i \t \cdot y}\phi)} + \int_{Q_0} C^{(0)} e(e^{\i \t \cdot y}u_{\Ep,\t}) : \overline{e(e^{\i \t \cdot y}\phi)}  + \int_{\Box} u_{\Ep,\t} \cdot \overline{\phi}  =  \int_{\Box} F_{\Ep,\t} \cdot \overline{\phi}, \qquad \forall \phi \in H^1_{per}(\Box;\CC^d),
%\end{equation}
%for $F_{\Ep,\t}(\cdot) : = U\Gamma_\Ep F(\t,\cdot)$. Now, as in the above example we can recast \eqref{dp.varp} in the form \eqref{p1} for: $\ep : = \frac{1-\Ep^2}{\Ep^2}$, $\Theta : = \Box^*$, $H :=H^1_{per}(\Box;\CC^d)$, $u_{\ep,\t} : = u_{\Ep,\t}$, $F_{\ep,\t} : = F_{\Ep,\t}$ and the forms
%\begin{equation}
%\label{abcdpELast}
%\begin{aligned}
%a_\t(u,\tilde{u}) := \int_{Q_1} C^{(1)} e( u) : \overline{e( \tilde{u})}, & \quad & (u,\tilde{u})_\t : = a_\t(u,\tilde{u}) + \int_{Q_0} C^{(0)} e( u): \overline{e( \tilde{u})} + \int_Q u\overline{\tilde{u}}. 
%\end{aligned}
%\end{equation}
%%and $c[\cdot]$ taken to be the standard $L^2$ norm. 
%Note $(\cdot,\cdot)_\t$ is an inner product on $H^1$ due to Korn's inequality and the assumptions on $C^{(i)}$. 
%
%Now  let us demonstrate that the main assumptions \eqref{KA}-\eqref{distance} hold in this setting. 
%
%\textbullet\, Proof of the {\bf spectral gap condition}  \eqref{KA}. 
%The assertion
%$$
%a[u] \ge \nu \int_{Q_1} | e( u)|^2, \qquad \forall u \in [H^1(Q)]^d,
%$$
%which holds due to the assumptions on $a^{(1)}$ implies that the space $V_\theta$, cf. \eqref{spaceV}, takes the form 
%$$
%\begin{aligned}
%V_\theta = \left\{
%\begin{array}{lr}
%[H^1_0(Q_0)]^d, & \theta \neq 0, \\[5pt]
%\CC^d\overset{\cdot}{+} [H^1_0(Q_0)]^d, & \theta = 0.
%\end{array} 
%\right. 
%\end{aligned}
%$$
%Henceforth, we understand $H^1_0(Q_0)$ to be a subset of $H^1(Q)$ by extending the elements of $H^1_0(Q_0)$ by zero into $Q_1$.
%The first non-trivial result is that the {\bf key assumption} \eqref{KA} holds. This is the following result.
%\begin{proposition}
%\label{KAdp}
%There exists a constant $C>0$ such that $\forall \theta \in [-\pi,\pi)^d$, $\forall w \in W_\theta,$  
%$$
%\Vert w \Vert_{[H^1(Q)]^d} \le C \big( \Vert  e(w) \Vert_{[L^2(Q_1)]^{d\times d}} + \Vert w \Vert_{[L^2(Q)]^d} \big).
%$$
%\end{proposition}
%\begin{proof}
%Fix $w \in W_\theta$ and first note that $\Vert w \Vert_{[H^1(Q)]^d} = \inf_{v \in V_\theta} \Vert w - v \Vert_{[H^1(Q)]^d}$. We construct a candidate $v$ as follows: let the mapping $E:H^1(Q_1) \rightarrow H^1(Q)$ denote a Sobolev of extension, i.e. 
%$$
%\begin{aligned}
%Eu = u \text{ in $Q_1$}, & \quad \text{and $\exists \, C_E>0$ such that} & \Vert Eu \Vert_{H^1(Q)} \le C_E  \Vert u \Vert_{H^1(Q_1)} \quad \forall u \in H^1(Q_1).
%\end{aligned}
%$$
%Then, setting $v:= w - Ew$ we see that $v \in V_\theta$ and 
%$$
%\Vert w \Vert_{[H^1(Q)]^d} \le \Vert w - v \Vert_{[H^1(Q)]^d} = \Vert Ew \Vert_{[H^1(Q)]^d} \le C_E \Vert w \Vert_{[H^1(Q_1)]^d}.
%$$
%To prove the desired result it remains to utilise the Korn's inequality in $H^1(Q_1)$.
%\end{proof}
%------------------------------------------------------------------------------------------------------------------------------------------------------------------------------------------------------------------------------------------------------------------------------------------------------------------------------------------------------------------------------------------------------------------------------------------------------------------------------------------------------------------------------------------------------------------------------------------------------------------------------------------------------------------------------------------------------------------------------------------------------------------------------------------------------
%We readily determine that $V_\theta$ is continuous, in the sense \eqref{contV}, everywhere except at $\theta = 0$, cf. Section \ref{section:discV}. As in the classical example above, $V_\theta$ is constant for $\theta \neq 0$ and consequently the mapping $P_\star : = \mathrm{id}_{H^1_{\#}} - P \vert_{H^1_{\#}}$ satisfies \eqref{noosc}, where $P : [H^1(Q)]^d \rightarrow [H^1_0(Q_0)]^d$ is the orthogonal projection, cf Remark \ref{rem:clasproj}.
%
%
% Now, following the construction in Section \ref{sec:suffconds}, we easily identify $W_\star = [H^1_{\#}(Q)]^d \ominus [H^1_0(Q_0)]^d$ and therefore 
%$$
%Z = V_0 \cap \big( [H^1_0(Q_0)]^d \big)^\perp = \big\{ u \in [H^1_{\#}(Q)]^d \, | \, \text{$u$ constant in $Q_1$ and $b(u,v)=0$ for all $v \in [H^1_0(Q_0)]^d $} \big\}.
%$$
%Notice that for a given $c \in \CC^d$ there exists a unique $z \in Z$ such that $z$ equals $c$ on $Q_1$, i.e. $z\vert_{Q_1}=c$. Indeed, suppose that there exists $z_1, z_2 \in Z$ such that $z_1\vert_{Q_1}=z_2\vert_{Q_1}=c$, then $z_1-z_2 \in [H^1_0(Q_0)]^d$; yet $z_1-z_2 \in \big([H^1_0(Q_0)]^d\big)^\perp$ and therefore $z_1-z_2=0$.
%
%We now build a basis for $Z$ as follows: for each Euclidean basis vector $e_i$, $i=1,\ldots,d$, let $\hat{z}^{(i)} \in Z$ be the element such that $\hat{z}^{(i)}\vert_{Q_1} = e_i$. Introducing the functions $N_{\hat{z}^{(i)}}^{(j)} \in W_0$, $j=1,\ldots,d$, that solve \eqref{cell:prob}, that is the solutions to
%\[
%\int_{Q_1} a^{(1)} \big( \nabla N_{\hat{z}^{(i)}}^{(j)} + e_i \otimes e_j  \big) \cdot \overline{\nabla \phi} =0, \qquad \forall \phi \in W_0,
%\]
%we find that the matrix $A^{\rm hom}_{\hat{z}}(\theta)$, given by Definition \ref{def:ahomtheta}, takes the form
%\[
%A^{\rm hom}_{\hat{z}}(\theta)_{rs} = \sum_{i,j=1}^d \{ a^{\rm hom}_{\rm dp}\}_{ripj} \theta_i \theta_j, \quad r,s \in \{1,\ldots,n\}
%\]
%where $a^{\rm hom}_{dp}$ is the well-known double-porosity homogenised tensor:
%\[
%\{ a^{\rm hom}_{\rm dp} \}_{risj} =  \int_{Q_1} \big( a^{(1)}_{sjkl} \partial_{l}\{N^{(i)}_{\hat{z}_r}\}_{k} + a_{sjri}^{(1)} \big), \qquad r,s \in \{1,\ldots,d\},\ i,j \in \{1,\ldots,d\}.
%\]
%Note, as in the above example, we can take $M^{\rm approx}_{\ep, \hat{z}}(\theta) = \ep^{-2} A^{\rm hom}_{\hat{z}}(\theta)$, cf. Section \ref{ahombhomexample}. 
%
%
%Let us show that \eqref{distance2} holds for sufficiently small $\theta$. W utilise the fact that there exists a Sobolev extension $J : H^1(Q_1) \rightarrow H^1(Q)$ with the additional property that  $\Vert e( Ju )\Vert_{L^2(Q)} \le C \Vert e(u) \Vert_{L^2(Q_1)}$ for some $C>0$ \footnote{This fact follows in a similar way to the scalar anaolgue  demonstrated in \cite[Chapter 3]{JKO}. Indeed, the proof follows by 
%	utilising a Sobolev extension $E$, as in the proof of Proposition \ref{KAdp}, and constructing  for $u \in H^1(Q_1)$ the mapping $ Ju : = \mathcal{R}u + E( u - \mathcal{R} u)$. Here $\mathcal{R}$ is the projection onto the rigid body motion in $H^1(Q_1)$.
%}. Furthermore, by integration by parts it is clear that a Korn inequality of the first kind holds for elements of $H^1_\theta$: $\Vert \nabla u \Vert_{L^2(Q)} \le C \Vert e(u) \Vert_{L^2(Q)}$, $ \forall u \in H^1_\theta(Q)$.
%From these facts and \eqref{qppoin} we compute for $ w \in W_\theta$ that
%$
%| \theta|^2 \int_{Q_1} | w|^2 =   |\theta|^2 \int_{Q_1} | J w|^2 \le  C \Vert \nabla Jw \Vert_{L^2(Q)}^2 \le C \int_{Q} | e(Jw) |^2 \le C \int_{Q_1} | e(w)|^2 .
%$
%That is
%\[
%| \theta|^2 \int_{Q_1} | w|^2  \le C \int_{Q_1} | e(w)|^2, \qquad \forall w \in W_\theta.
%\]
%To finish the proof of \eqref{distance2} it remains to bound $\Vert w \Vert_{L^2(Q_0)}$ from above by $\Vert w \Vert_{H^1(Q_1)}$. This assertion readily follows since for $w \in W_\theta$ we find $w - Jw \in V_\theta$ and so $b(w,w-Jw)=0$, whence $b[w] \le b[Jw]  \le C \Vert w\Vert_{H^1(Q_1)}^2$.
%
%%$a^{\rm hom}_{dp}$ is known to be elliptic, then assertion \eqref{posonw1}  holds. 
%%That is, cf. Proposition \ref{prop:posdistequivtoahompos}, 
%We have demonstrated that the hypotheses of Theorem \ref{thm:ahom} hold, and we therefore deduce the following result.
%\begin{theorem}
%\label{thm:dp}
%	Let $u_\ep$ be the unique solution to problem \eqref{differentialPDEdp}, and consider $u_{\ep,\theta}(\cdot)=\mathcal{U}\mathcal{T}_\ep u_\ep(\theta, \cdot)$. Let $\hat{z}=(\hat{z}^{(1)},\ldots,\hat{z}^{(d)})$ where 
%	\[
%	\begin{aligned}
%\hat{z}^{(i)} = e_i \ \text{on $Q_1$}, \quad \& \quad \int_{Q_0} a^{(0)} e(\hat{z}^{(i)}) \cdot \overline{ e( \varphi)} + \lambda \int_{Q_0} \hat{z}^{(i)} \cdot \overline{\varphi} = 0, \quad \forall \varphi \in [H^1_0(Q_0)]^d.
%	\end{aligned}
%	\]
%	 Consider
% $\alpha_{\ep,\theta} \in \CC^d$ the solution to the problem
%\[
%\tfrac{1}{\ep^2} a^{\rm hom}_{{\rm dp}}(\alpha_{\ep,\theta} \otimes \theta) \cdot \overline{\widetilde{\alpha} \otimes \theta} + b(\alpha_{\ep,\theta} \cdot \hat{z}, \widetilde{\alpha} \cdot \hat{z}) = c(\mathcal{U}\mathcal{T}_{\ep}f(\theta,\cdot),\widetilde{\alpha} \cdot \hat{z}), \qquad \widetilde{\alpha} \in \CC^d, 
%\]
%and let $v_{\ep,\theta}  \in [H^1_0(Q_0)]^d$ be the solution to
%	\[
%	\int_{Q_0} a^{(0)} e(v_{\ep,\theta}) \cdot \overline{e(\varphi)} + \lambda \int_{Q_0} v_{\ep,\theta} \cdot \overline{\varphi} = \int_{Q_0}\mathcal{U}\mathcal{T}_{\ep}f(\theta,\cdot)\overline{\varphi}, \qquad \forall \varphi \in [H^1_0(Q_0)]^d.
%	\]
%
%	
%Let $r = \min{\{ \tfrac{1}{4C_2}, r_0, \tfrac{1}{2}r_1 \} }$ be as in Theorem \ref{thm:ahom}  and Set 
%\[
%u^{\rm approx}_{\ep, \theta} : = \left\{ \begin{array}{lr}
%e^{\i \theta \cdot y} (\alpha_{\ep,\theta} \cdot \hat{z}  + \i \theta \cdot N{\alpha_{\ep,\theta} \cdot \hat{z}} ) + v_{\ep,\theta} & |\theta| < r, \\
%v_{\ep,\theta}, & |\theta| \ge r. 
%\end{array}  \right. .
%\] Then, there exists $\kappa >0$ such that $\forall \ep >0$, $\forall\, \theta \in [-\pi,\pi)^d$, the following inequalities hold.
%\begin{enumerate}[(i)]
%\item{$\tfrac{1}{\ep^{2}} \Vert  e (u_{\ep,\theta} - u^{\rm approx}_{\ep,\theta}) \Vert_{[L^2(Q_1)]^{d\times d}}^2	+ \Vert  e(u_{\ep,\theta} - u^{\rm approx}_{\ep,\theta}) \Vert_{[L^2(Q_0)]^{d\times d}}^2+  \Vert u_{\ep,\theta} - u^{\rm approx}_{\ep,\theta} \Vert_{[L^2(Q)]^d}^2  \\ \hspace*{\fill}\le \kappa \ep^2 \Vert \mathcal{U}\mathcal{T}_\ep f(\theta, \cdot) \Vert_{[L^2(Q)]^d}^2.
%	$}
%\item{$\Vert u_{\ep,\theta} -  (e^{\i \theta \cdot y}\alpha_{\ep,\theta} \cdot \hat{z} + v_{\ep,\theta})\Vert_{[L^2(Q)]^d}^2 	 \le \kappa \ep^2 \Vert \mathcal{U}\mathcal{T}_\ep f(\theta, \cdot) \Vert_{[L^2(Q)]^d}^2.
%$}
%\end{enumerate}
%\end{theorem}
%Inequality (i) is a new result which gives for the first time a corrector-type error estimates in homogenisation of double-porosity systems. Inequality (ii) provides for the first time (order-sharp) operator norm-resolvent estimates for double-porosity systems. In the context of scalar double-porosity problems, order-sharp operator estimates were first demonstrated by different means  in \cite{ChCo}. 
%
%%{ \ }
%%
%%We conclude this example by comparing the asymptotics of $u_\ep$ present in Theorem \ref{thm:dp} to its two-scale limit, which was first determined in series of works \cite{Zhi2000,Zhi2005} for the scalar case and \cite{ZhiPa} for the system case in the context of linear elasticity. 
%%The two-scale limit $u_0(x,y) =u(x) + v_0(x,y)$ of the solution $u_\ep(x)$ to \eqref{differentialPDEdp} is found, in the works cited above, to solve the system: {Find $(u,v_0) \in [H^1(\RR^d)]^n\times[L^2(\RR^d;H^1_0(Q_0)]^n$ the solution to}
%%\begin{equation*}
%%%\label{twoscaledp}
%%\left\{
%%\begin{aligned}
%%& - {\rm div}_x \big( a^{\rm hom}_{\rm dp} \nabla_x u(x)\big) + \lambda \left( u(x) + \int_{Q_0} v_0(x,y)\, {\rm d}y \right) = f(x), \qquad x \in \RR^d, \\
%%& - {\rm div}_y \big( a^{(0)}(y) \nabla_y v_0(x,y)\big) + \lambda \big( u(x) + v_0(x,y) \big) = f(x), \qquad x \in \RR^d, y \in Q_0.
%%\end{aligned}
%%\right.
%%\end{equation*}
%%Equivalently,   $u_0 \in [H^1(\RR^d)]^n\overset{\cdot}{+} L^2(\RR^d; [H^1_0(Q_0)]^n)$ solves the  variational problem
%%\begin{multline*}
%%\tfrac{1}{|Q_1|}\int_{\RR^d} \int_{Q_1} a^{\rm hom}_{\rm dp} \nabla_x u_0(x,y) \cdot \overline{ \nabla_x \phi_0(x,y)} \, {\rm d}y{\rm d}x +  \int_{\RR^d}\int_Q a^{(0)}(y) \nabla_y u_0(x,y) \cdot \overline{\nabla_y \varphi_0(x,y)} \, {\rm d}y{\rm d}x  \\
%%+ \lambda \int_{\RR^d} \int_Q u_0(x,y) \cdot \overline{ \phi_0(x,y)} \, {\rm d}y{\rm d}x =  \int_{\RR^d} \int_Q f(x) \cdot \overline{\phi_0(x,y)} \, {\rm d}y{\rm d}x, 
%%\qquad \forall \phi_0 \in [H^1(\RR^d)]^n\overset{\cdot}{+} L^2(\RR^d; [H^1_0(Q_0)]^n).
%%\end{multline*} 
%%
%%To simplify the exposition and aid the delivery of the most important details, we shall focus our attention on scalar  double-porosity  equations, i.e. set $n=1$.
%%
%%Let $z$ be the element of $Z$ such that $z = 1$ in $Q_1$.  Then $\phi_0 \in  H^1(\RR^d)\overset{\cdot}{+} L^2(\RR^d; H^1_0(Q_0))$ if, and only if, $\phi_0(x,y) = \hat{\phi}(x) z(y) + \phi_1(x,y)$ for some  $\hat{\phi} \in H^1(\RR^d)$, $\phi_1 \in L^2(\RR^d; H^1_0(Q_0))$. 
%%
%%Let us rewrite the above variational problem in terms of the forms $a$, $b$ and $c$ in \eqref{abcdp}. Recalling that $b(z,\varphi) = 0$ $\forall \varphi \in H^1_0(Q_0)$, we deduce that $u_0(x,y) = \hat{u}(x) z(y) + u_1(x,y)$, where  $\hat{u} \in H^1(\RR^d)$ solves
%%\begin{equation}
%%\begin{aligned}
%%& \int_{\RR^d} a^{\rm hom}_{\rm dp} \nabla \hat{u}(x) \cdot \overline{ \nabla \hat{\phi}(x)} \, {\rm d}x +  \int_{\RR^d} B^{\rm hom}_{\hat{z}} \hat{u}(x) \overline{\hat{\phi}(x)} \, {\rm d}x =  \int_{\RR^d} \rho f(x)   \overline{\hat{\phi}(x)} {\rm d}x, \qquad \forall \hat{\phi} \in H^1(\RR^d),  \label{twoscaledp}\\
%%\end{aligned}
%%\end{equation}
%%for $\rho = \int_{Q} z$, and $u_1 \in L^2(\RR^d;H^1_0(Q_0))$ solves
%%\[
%%\begin{aligned}
%%& \hspace{0pt}\int_{\RR^d}b\big( u_1(x,\cdot), \phi_1(x,\cdot)\big) \, {\rm d}x =  \int_{\RR^d}  c\big( f(x) ,\phi_1(x,\cdot) \big) \, {\rm d}x, \qquad \forall \phi_1 \in L^2(\RR^d; H^1_0(Q_0)).
%%\end{aligned}
%%\]
%%It follows that $u_1(x,y) =  V(y) f(x)$, where $V \in H^1_0(Q_0)$ solves
%%\[
%% \quad b(v,\varphi) = c(1,\varphi), \quad \forall \varphi \in H^1_0(Q_0).
%%\]
%%The relationship between $v_{\ep,\theta}$, of Theorem \ref{thm:dp} and $u_1$ is now clear. Indeed, for the invertible operator $B_0: H^1_0(Q_0) \subset L^2(Q_0) \rightarrow L^2(Q_0)$ with the action $v \mapsto f$ if, and only if, $b(v,\varphi) = c(f,\varphi)$, $\forall \varphi \in H^1_0(Q_0)$,  we deduce that $f = \mathcal{T}^{-1}_\ep \mathcal{U} B_0 v_{\ep,\theta}$,  and then 
%%\begin{equation}
%%\label{twoscaleoscillations}
%%w(x,y) = V(y) \mathcal{T}^{-1}_\ep \mathcal{U}^{-1}  B_0 v_{\ep,\theta}(x), \quad x \in \RR^d, y \in Q_0.
%%\end{equation}
%%
%%
%%
%%Let us analyse the relationship between $\hat{u}$ and the constant $\alpha^{\rm hom}_{\ep,\theta}$ in  Theorem \ref{thm:dp}. As $a^{\rm hom}_{\rm dp}$ is a positive constant matrix,  $\hat{u}$ solves a classical problem of the type discussed  in example \ref{e.class}: Indeed, this is seen to be so by noting that for
%%\[
%%a[u] : = \int_{\RR^d} a^{\rm hom}_{\rm{dp}} \nabla u \cdot \overline{\nabla u}, \qquad b[u] = a[u] +  \int_{\RR^d} B^{\rm hom}_{\hat{z}} u \cdot \overline{u}, \quad c[u] : =  \int_{\RR^d} | u|^2, \quad u \in [H^1(\RR^d)]^n,
%%\]
%%then $\hat{u}_{\ep,\theta} = \mathcal{U}\mathcal{T}_\ep \hat{u}(\theta,\cdot)$ solves the problem
%%\[
%%\tfrac{1-\ep^2}{\ep^2}a(\hat{u}_{\ep,\theta}, \phi) + b(\hat{u}_{\ep,\theta},\phi) = c{\rho f, \phi}, \quad \forall \phi \in H^1_\theta(Q).
%%\]
%%Now it is clear that  $V_\theta = \{ 0 \}$, $\theta \neq 0$, $V_0 = \CC$, and that all the hypotheses of  Theorem \ref{thm:ahom} hold. Therefore,  for each $\ep>0$, $\theta \in [-\pi,\pi)^d$, we infer from Theorem \ref{thm:ahom} that   $\alpha_{\ep,\theta} \in \CC$ the solution to
%%\[
%%\tfrac{1-\ep^2}{\ep^2} a^{\rm hom}_{\rm{dp}} \alpha_{\ep,\theta} \otimes \theta \cdot \overline{\beta \otimes \theta} + B^{\rm hom}_{\hat{z}} \alpha_{\ep,\theta} \cdot \overline{\beta} = \int_Q  \mathcal{U}\mathcal{T}_{\ep,\theta} \rho f (\theta , y)  \, {\rm d}y \cdot \overline{\beta}, \qquad \forall \beta \in \CC.
%%\]
%%satisfies the inequality 
%%\[
%%\Vert \hat{u}_{\ep,\theta} - \alpha_{\ep,\theta}\Vert_{[L^2(Q)]^n} \le \kappa \ep^2 \Vert \mathcal{U}\mathcal{T}_{\ep,\theta} f (\theta , \cdot)\Vert_{[L^2(Q)]^n}
%%\]
%% for some $\kappa >0$ independent of $\ep$ and $\theta$.
%%
%%Let us finish with commenting that
%%\[
%%\alpha^{\rm hom}_{\ep,\theta} = \rho^{-1} \alpha_{\ep,\theta}.
%%\]
%%Indeed, this is immediate from the equations that these constants satisfy once we establish that  $\rho$ is a non-zero. This claim is demonstrated by noting that if $0=\rho = \int z$ then $z$ would be a non-zero element of $V_0 = \CC \overset{\cdot}{+}H^1_0(Q_0)$ that is orthogonal to $\CC^n$ and $H^1_0(Q_0)$. We have proved the following result.
%%\begin{theorem}
%%	\label{thm:dptwo}
%%	Let $u_\ep$ solve \eqref{differentialPDEdp}. Consider the functions $\alpha^{\rm hom}_{\ep,\theta}$, $z$ given in Theorem \ref{thm:dp}. Let $u_0$ to be the two-scale limit of $u_\ep$. Then, the following assertions hold.
%%	\begin{enumerate}[(i)]
%%\item{ Let $B_0$ be the unbounded operator given in \ref{twoscaleoscillations}. Then, 
%%	%	the identities
%%	\[
%%	f=  \mathcal{T}^{-1}_\ep \mathcal{U}^{-1} B_0 v_{\ep,\theta}.
%%	\]
%%	%and
%%}
%%\item{Let $\hat{u}$ solve \eqref{twoscaledp}. Then,
%%\[
%%u_0(x,y) = \hat{u}(x)z(y)  + \mathcal{T}^{-1}_\ep \mathcal{U}^{-1} B_0 v_{\ep,\theta}(x)V(y), \quad x \in \RR^d, y \in Q.
%%\] }
%%\item{ The inequality
%%	\[
%%	\int_{\RR^d}\int_Q | u_0(x,y) - z(y)\mathcal{T}_\ep^{-1} \mathcal{U}^{-1} \rho \alpha^{\rm hom}_{\ep,\theta}(x)  + V(y) \mathcal{T}^{-1}_\ep \mathcal{U}^{-1} B_0 v_{\ep,\theta}(x)  |^2 \, {\rm d}y {\rm d}x \le \kappa \ep^2 \Vert f\Vert_{L^2(\RR^d)}
%%	\]	
%%holds for some constant $\kappa$ independent of $\ep$ and $f$.	}
%%	\end{enumerate}
%%
%%\end{theorem}
%
%\subsection{`Inverse' double-porosity model}
%\label{e:idp}
%Here, we consider an example where the form $a$ on the space $W_\theta$ is uniformly coercive and therefore is in the class of problems analysed in Section \ref{s:uniforma}. A simple example of such a case  is the `inverse double-porosity' problem: the case where the roles of the sets $Q_0$ and $Q_1$ in Section \ref{s.edp} are reversed, i.e. we set
%$$
%\begin{aligned}
%a[u] = \int_{Q_0} a^{(0)} \nabla u \cdot \overline{\nabla u}, & \quad & b[u] = \int_{Q_1} a^{(1)} \nabla u \cdot \overline{\nabla u} +  \int_{Q_0} a^{(0)} \nabla u \cdot \overline{\nabla u} + \lambda  \int_Q |u|^2, \qquad u \in [H^1(Q)]^n,
%\end{aligned}
%$$
%where, $n \ge 1$, and   $a^{(0)}$, $a^{(1)}$ satisfy
%\begin{equation}
%\label{ivass}
%\begin{aligned}
%& a^{(r)}_{ijkl} \in L^\infty(Q_r), \qquad a^{(r)}_{ijkl} = a^{(r)}_{klij} , \quad i,k \in \{ 1,\ldots,n\}, \  j,l\in \{1,\ldots,d\},  \\
%& a^{(r)}_{ijkl}(y)\Xi_{kl}\Xi_{ij} \ge \nu |\Xi|^2 \quad \forall y \in Q_r, \   \text{for all $n \times d$ matrices $\Xi$}.
%\end{aligned}
%\end{equation}
%Or, for linear elasticity we would simply take $a^{(i)}$ as in Section \ref{s.edp}.
%
%
% In this setting, for each $\theta \in[-\pi,\pi)^d$, $d \ge 1$, one has
%$$
%V_\theta = \{ v \in [H^1_{\theta}(Q)]^n \, | \, \text{$v$ is constant on $Q_0$} \}
%$$
%and, by arguing as in the proof of Proposition \ref{KAdp} with the simple modification of $\ E \ $ now being taken to be a Sobolev extension from $H^1(Q_0)$ to $H^1_0(Q)$, the \textbf{key assumption} \eqref{KA} is readily shown to hold. Let us now prove that $V_\theta$ is continuous in the sense \eqref{contV}, i.e.  the following result holds.
%\begin{proposition}
%	\label{vcontinverdp}
%For every $\theta \in [-\pi,\pi]^d$, the space $V_\theta$ is continuous in the sense \eqref{contV}.
%\end{proposition}
%\begin{proof}
%Let  $\chi \in C^\infty_0(Q)$ be such that $\chi \equiv 1$ on $Q_0$. Fix $\theta \in [-\pi,\pi]^d$ and $v \in V_\theta$. It is clear that $v = c\chi + w$ where $c \in \CC^n$, $w \in \mathcal{V}_\theta = \{ u \in [H^1_{\theta}(Q)]^n \, | \, \text{$u \equiv 0$ on $Q_0$} \}$.
%
%Now, consider $\{ \theta_m\}_{, \in \NN}$ such that $\lim\limits_{m\rightarrow \infty} \theta_m = \theta$. Then,
%% for fixed $v \in V_\theta$, $v=\chi c+w$, for some $c \in \CC^n$, $w \in \mathcal{V}_\theta$, and
% we construct $v_m = \chi c + e^{\i (\theta_m - \theta)\cdot y}w$. It immediately follows that $v_m \in V_{\theta_m}$ and $v_m$ converges strongly to $v$ in $[H^1(Q)]^n$ as $m \rightarrow \infty$.
%\end{proof}
%A direct consequence of this result is that, for these class of problems, the main asymptotic analysis result is given by Theorem \ref{thm:contV}. Indeed, we saw from Proposition \ref{prop:contV} that Proposition \ref{vcontinverdp} implies the form $a$ is uniformly coercive on $[-\pi,\pi]^d$, see Remark \ref{rem.contv}.  Theorem \ref{thm:contV}  implies the following result.
%\begin{theorem}
%For fixed $\ep >0$ let $u_\ep \in [H^1(\RR^d)]^n$ solve the problem
%$$
%-{\rm div}\big(a_\ep\left(\tfrac{x}{\ep}\right)\nabla u_\ep(x) \big) + \lambda u(x) = f(x), \quad x \in \RR^d,
%$$
%for a given $\lambda \in (0,\infty)$, $f \in [L^2(\RR^d)]^n$, where
%$$
%a_\ep(y) = \left\{ \begin{array}{lr}
%\ep^2 a^{(1)}(y) & y \in Q_1, \\
%a^{(0)}(y) & y \in Q_0,
%\end{array}
%\right.
%$$ 
%for $a^{(r)}$ satisfying  \eqref{ivass} (or \eqref{dpcoeffs}).
%% Then, for $u_{\ep,\theta}(\cdot) = \mathcal{U}\mathcal{T}_\ep u_\ep(\theta,\cdot)$ and $v_{\theta} \in V_\theta$ the solution to
%%$$
%%\int_{Q_1} a^{(1)} \nabla v_\theta \cdot \overline{\nabla \phi} + \lambda \int_Q v_\theta \cdot \overline{\phi} = \int_Q  \mathcal{U}\mathcal{T}_\ep f(\theta,\cdot)\cdot \overline{\phi}, \qquad \forall \phi \in V_\theta,
%%$$ 
%% the following inequality
%%$$
%%\Vert u_{\ep,\theta} - v_\theta \Vert_{[H^1(Q)]^n} \le \kappa \ep^4 \Vert  \mathcal{U}\mathcal{T}_\ep f(\theta,\cdot) \Vert_{[L^2(Q)]^n}, 
%%$$
%%holds for some constant $\kappa>0$ that depends on $a^{(0)}$, $a^{(1)}$ and $\lambda$ only.
% Then, for $v_{\ep,\theta} \in V_\theta$ the solution to
% $$
% \int_{Q_1} a^{(1)} \nabla v_{\ep,\theta} \cdot \overline{\nabla \phi} + \lambda \int_Q v_{\ep,\theta} \cdot \overline{\phi} = \int_Q  \mathcal{U}\mathcal{T}_\ep f(\theta,\cdot)\cdot \overline{\phi}, \qquad \forall \phi \in V_\theta,
% $$ 
% the following inequality
% $$
% \Vert u_{\ep} - \mathcal{T}_\ep^{-1} \mathcal{U}^{-1} v_{\ep,\theta} \Vert_{[H^1(\mathbb{R}^d)]^n} \le \kappa \ep \Vert  f \Vert_{[L^2(\mathbb{R}^d)]^n}, 
% $$
% holds for some constant $\kappa>0$ that depends on $d$, $a^{(0)}$, $a^{(1)}$ and $\lambda$ only.
%\end{theorem}
%%\subsection{An example from wave propagation in photonic crystal fibres}
%%Here, we present a non-trivial example, of a form $a[\cdot]$, where the space $V_\theta$ is continuous with respect to $\theta$. This example appears in the context of time-harmonic wave solutions to the Maxwell system for two-dimensional periodic dielectric materials that occupy the whole space $\RR^3$.
%To the authors knowledge, this result has not previously been found in literature.
%
%In dimension $d=1$, the inverse double-porosity model is equivalent to the one-dimensional double-porosity model. The scalar one-dimensional double porosity model has been studied in the works \cite{ChChCo, ChCo3, ChKi}, with order-sharp operator norm-resolvent estimates (from $L^2$ to $L^2$) being obtained by different means in \cite{ChChCo, ChKi}. 
%
\subsection{An example  with a `partial' high-contrast}
\label{e.pdelast}
Here we consider an % non-trivial 
example of a high-contrast linear elasticity problem with so-called `partial degeneracy'. 
Consider the following resolvent problem:
\begin{equation}
\label{elastres}
\left\{ 
\begin{aligned}
& \text{Find $u_\ep \in \left[H^1(\RR^3)\right]^3$ such that} \\
& -{\rm div}\, \sigma_\ep(u_\ep) \,+\, u_\ep \,\,=\,\, F \in [L^2(\RR^3)]^3,
\end{aligned}
\right.
\end{equation}
where the matrix is assumed stiff but the $\ep$-periodic inclusions are stiff in compression but soft in shear. 
Namely, for 
%a given $f \in [L^2(\RR^3)]^3$ and
 the stress-strain constitutive relation, % of a linear elastic isotropic body with `weakly' compressible inclusions: 
\[
\sigma_\ep(u_\ep)(x) \,\,=\,\, \lambda\left(\tfrac{x}{\ep}\right) \big({\rm div}\,u_\ep\big) I \,\, +\,\, 2\mu_\ep\left(\tfrac{x}{\ep}\right) e(u_\ep), \ 
x \in \RR^3, \qquad e(u)= \tfrac{1}{2}\big( \nabla u + \nabla u^T \big),
\]
with $\square$-periodic Lam\'{e} coefficients of the form
\[
\begin{aligned}
\lambda(y) \,\,=\,\, \left\{ \begin{array}{lr}
\lambda_1(y), & y \in \square \backslash B, \\[5pt]
\lambda_2(y), & y \in B,
\end{array} \right. & \qquad &\mu_\ep(y) = \left\{ \begin{array}{lr}
\mu_1(y), & y \in \square \backslash B, \\[5pt]
\ep^2 \mu_2(y), & y \in B.
\end{array} \right.
\end{aligned}
\]
Here as before  the reference inclusion set $B$ is assumed to have Lipschitz boundary,  
$\overline{B} \subset (-\tfrac{1}{2},\tfrac{1}{2})^n$ and $\square \backslash B$ is connected, and the coefficients $\lambda_i$, $\mu_i$, $i=1,2$, are 
uniformly positive and bounded. 
Now, we proceed as in the above examples, to find that $u_{\ep,\theta}%(\cdot) 
= U\Gamma_\ep u_\ep(\theta,\cdot)$ solves \eqref{p1} with the following identifications: 
$H = \left[H^1_{per}(\square)\right]^3$, $\Theta = [-\pi,\pi]^3$, 
$\l f, \tilde{u} \r = \int_\square  U \Gamma_\ep F(\theta,y) \cdot\overline{\tilde{u}(y)} \, {\rm d} y$, and
% $\l f, \tilde{u} \r : =( U \Gamma_\ep f(\theta,\cdot), \tilde{u})$, 
%for $(\cdot,\cdot)$ the standard $L^2$ inner product, and
\begin{eqnarray}
\label{atpd}
	a_\t[u]  &=& \int_{\square \backslash B} \Big(\lambda_1\big| {\rm div}\,  u + \i\, \t \cdot u\big|^2 \,+\, 2\mu_1 \big| e(u) + \i \,\t \odot u\big|^2 \Big) 
	\,\,+\,\, \int_{B}\lambda_2\, \big| {\rm div}\, u + \i\, \t \cdot u\big|^2,   \\
	\label{btpd}
	b_\t[u] &=&  \int_{B}  2\mu_2 \big| e(u) + \i\, \t \odot u\big|^2 \,\,+\, \int_\square |u|^2, \quad \quad 
	\t \odot u : = \tfrac{1}{2}\big(\t \otimes u + u \otimes \t\big).
\end{eqnarray}
Let us now check %demonstrate that 
the main abstract assumptions. % of the article hold.
We begin by recalling the elasticity theory variant of extension Proposition \ref{prp.zhiext}, whose proof we shall provide for the reader's convenience.

\begin{proposition}\label{prp.zhiextelast}
There exists an extension operator $E : \left[H^1(\Box \backslash B)\right]^3 \rightarrow \left[H^1(\Box)\right]^3$ with the following properties: $Eu|_{\Box \backslash B} = u$, $\| Eu \|_{\left[H^1(\Box)\right]^3} \,\,\le\,\, C_E \| u \|_{\left[H^1(\Box \backslash B)\right]^3}$, and
\begin{equation}\label{ZhExtensionelast}
	\int_\Box \big| e(Eu)\big|^2 \,\,\,\le\,\,\, C_E^2 \int_{\Box \backslash B} \big| e(u)\big|^2, 
\end{equation}	
for some constant $C_E>0$ independent of $u$.
\end{proposition}
\begin{proof}
For fixed $u \in \left[H^1(\square \backslash B)\right]^3$  let $Ru=c + d \times y$ %, $c,d \in \CC^3$, 
be the $[H^1(\square \backslash B)]^3$-projection of $u$ onto 
the subspace $\mathcal{R}=\left\{\tilde c + \tilde d \times y\,\big|\,\tilde c, \tilde d \in \CC^3\right\}$ of rigid body motions of $\square \backslash B$ 
(so in particular $\|Ru\|_{\left[H^1(\square \backslash B)\right]^3}\le \|u\|_{\left[H^1(\square \backslash B)\right]^3}$). 
Recall Korn inequality in the following form:
\begin{equation}
\label{kornineq}
\big\| u -\, Ru \big\|_{\left[H^1(\square \backslash B)\right]^3} \,\,\,\le\,\,\, 
C_K\, \big\| \,e(u)\, \big\|_{\left[L^2(\square \backslash B)\right]^{3\times3}}
\end{equation} 
for some $C_K>0$ independent of $u$. Let $P:[H^1(\square \backslash B)]^3 \rightarrow [H^1(B)]^3$ be the standard Sobolev extension 
(applied component-wise), i.e. $P u = u$ in $\square \backslash B$ and  there exists $C_P>0$ such that 
\begin{equation}\label{15.04.21}
	\| Pu \|_{\left[H^1(\square)\right]^3} \,\,\,\le\,\,\, C_P\, \| u \|_{ \left[H^1(\square \backslash B)\right]^3}, \qquad \forall 
	u \in \left[H^1(\square \backslash B)\right]^3.
\end{equation}
We construct $E$ as follows: $Eu := Ru + P(u - Ru)$, where $Ru=c + d \times y$ for all $y\in\square$. 
As $\mathcal{R}$ is finite-dimensional and a direct sum of `translational' ($d=0$) and 
`rotational' ($c=0$) subspaces, one can see that 
$c_1\|Ru\|_{\left[H^1(\square)\right]^3}\le |c|^2+|d|^2\le c_2 \|Ru\|_{\left[H^1(\square \backslash B)\right]^3}$ with positive constants $c_1$ and $c_2$ 
independent of $u$. 
It is then straightforward to check via \eqref{15.04.21} and \eqref{kornineq} 
that all the stated properties of $E$ hold.
\end{proof}
Now, the above uniform positivity assumptions on the Lam\'{e} coefficients imply that
\[
a_\t[u]\,\,\, \ge\,\,\, C\, 
\left( \int_\square \big| {\rm div}\, u + \i\, \t \cdot u \big|^2 \,\,+\, 
\int_{\square \backslash B} \big| e(u) +\, \i \,\t\odot u\big|^2 \right), \quad \forall u \in [H^1_{per}(\square)]^3,
\]
for some positive constant $C$. Next, by Proposition \ref{prp.zhiextelast}, and arguing as in \eqref{dpH1}, we obtain 
\begin{flalign*}
\int_{\square \backslash B} \big| e(u) +\, \i\, \t\odot u\big|^2 & \,\,=\, 
\int_{\square\backslash B} \big| e\left(e^{\i \t\cdot y} u\right)\big|^2 \,\,\ge\,\, 
C_E^{-2}  \int_{\square} \big| e\left(E\left(e^{\i \t\cdot y} u\right)\right)\big|^2 \,\,\ge\,\,
\frac{1}{2} C_E^{-2}  \int_{\square} \big| \nabla \left(E\left(e^{\i \t\cdot y} u\right)\right)\big|^2 \\
& \ge\,\,\frac{1}{2} C_E^{-2}\,|\t|^2  \int_{\square} \big| E\left(e^{\i \t\cdot y} u\right)\big|^2 \,\,\ge\,\,
\frac{1}{2} C_E^{-2}\,\,|\t|^2  \int_{\square \backslash B} |  u|^2,
\end{flalign*}
where the second inequality  follows e.g. from integration by parts and the $\square$-periodicity of $u$. 
Consequently, one has 
\begin{equation}
\label{PDelast:nondeg}
a_\t[u] \,\,\,\ge\,\,\, C\, \left( \int_\square \big| {\rm div}\, u + \i\, \t \cdot u \big|^2 \,\,+\,\,
\frac{1}{2}\, C_E^{-2}\, |\t|^2 \int_{\square \backslash B} | u|^2 \right), \quad \forall u \in \left[H^1_{per}(\square)\right]^3.
\end{equation}
Thus, from \eqref{atpd} and \eqref{PDelast:nondeg}, the space $V_\theta$  %in this setting 
is
\begin{equation}
	\label{aaspaceV}
	\begin{aligned}
		V_\theta = \left\{ \begin{array}{lr}
			\left\{ v \in \left[H^1_0(B)\right]^3 \,\,\, \big| \, \text{ ${\rm div}\, v + \i\, \t \cdot v=0$ in $B$}\right\}, & \theta \neq 0, \\[5pt]
			\left\{ v \in \left[H^1_{per }(\square)\right]^3 \, \, \big| \, \text{$v$ constant in $\square \backslash B$ and ${\rm div}\, v=0$ in $B$}\right\}, & \theta = 0.
		\end{array}
		\right.
	\end{aligned}
\end{equation}
As before, we regard $H^1_0(B)$ as a subspace of $H^1_{per}(\square)$ by extending by zero into $\square \backslash B$. 
For  proving the key spectral gap condition \eqref{KA}  we  shall be using the following `Sobolev Modification' lemma.
% introduce in \cite{Co}.
\begin{proposition}\label{SobModprop} 
	There exists linear operator $M : \left[H^1(\square)\right]^3 \rightarrow \left[H^1(\square)\right]^3$ and a positive constant $C_M$ such that
	\begin{equation*}
%		\label{modification}
%		\left\{
		\begin{aligned}
			& (i)\ \ \text{ $Mu = u$ in $\square \backslash B$;} \\
			& (ii)\ \ \text{${\rm div}\, Mu = {\rm div}\, u$ in $B$;} \\
			& (iii)\ \ \Vert Mu \Vert_{\left[H^1(\square)\right]^3}^2 \,\,\le\,\, 
			C_M\,  \Big( \| u \|_{\left[L^2(\square \backslash B)\right]^3}^2 \,+\,  \Vert e(u) \Vert_{\left[L^2(\square \backslash B)\right]^{3\times 3}}^2\,+\,
			\| {\rm div}\, u\|_{L^2(B)}^2 \Big),  \quad \forall u \in \left[H^1(\square)\right]^3.
%&(iv)\ \ \| e(Mu) \|_{[L^2(\square)]^3}^2 \le C_M  \| e(u) \|_{[L^2(\square \backslash B)]^3}^2     \qquad \forall u \in [H^1(\square)]^3.
		\end{aligned}
%		\right.
	\end{equation*}
%	There exists a constant $C>0$ such that
%\[
%a_\t[u] \ge C\Big( |\t|^2 \int_{\square \backslash B} |u|^2 + \int_B | {\rm div}\, u + \i \t \cdot u|^2 \Big), \quad \forall u \in [H^1_{per}(\square) ]^3, \, \forall \t \in [-\pi,\pi]^3.
%\]
\end{proposition}
\begin{proof}
%Consider $R : [H^1(\square \backslash B)]^3 \rightarrow \{ u \in [H^1(\square \backslash B)]^3 \, | \, e(u) = 0 \}$ be the projection onto rigid motion. Then, Korn's inequality states that there exists a constant  $C_K>0$ such that
%\begin{equation}
%\| u - Ru \|_{[H^1(\square \backslash B)]^3} \le C_K \| e(u) \|_{ [L^2(\square \backslash B)]^3}, \qquad \forall u \in [H^1(\square \backslash B)]^3.
%\end{equation}

%From the above two inequalities we find $Eu := Ru + P(u - Ru)$ satisfies $E u = u$ in $\square \backslash B$ and 
%\begin{equation}
%	\| Eu \|_{[H^1(\square)]^3} \le C_P \| u \|_{ [H^1(\square \backslash B)]^3}, \qquad \forall u \in [H^1(\square \backslash B)]^3.
%\end{equation}
It is well know that the divergence operator  ${\rm div}:\, [H^1_0(B)]^3 \rightarrow L^2_0(B) : = \{ f \in [L^2(B)]^3 \, | \, \int_{B} f =0 \}$ is surjective,  see for example \cite[ Chapter 1, Section 2.1]{Lady}. Moreover, 
\begin{equation}
	\label{divinverse}
	\left\{ \begin{aligned}
		& \text{there exists a linear map $U:L^2_0(B) \rightarrow [H^1_0(B)]^3$ such that 
			for each $f \in L^2_0(B)$ %there exists $U \in [H^1_0(B)]^3$ such that  
			}
		\\
		& {\rm div}\, Uf = f, \quad \text{and} \quad \Vert Uf \Vert_{\left[H^1(B)\right]^3} \,\,\le\,\, C_B \Vert f \Vert_{\left[L^2(B)\right]^3} \quad \text{for some $C_B>0$ independent of $f$.}
	\end{aligned} \right.
\end{equation}
Let us now construct $M$. For fixed $u \in \left[H^1(\square )\right]^3$, 
% consider $\tilde{u} = P u $ where $r$ is the projection of $u$ onto rigid body motion in $\square \backslash B$.  Now  
$u-Pu \in [H^1_0(B)]^3$ for $P$ as in \eqref{15.04.21} and so ${\rm div}( u-Pu) \in L^2_0(B)$.
Now let $f ={\rm div}( u -Pu)$ 
%consider $U \in[H^1_0(B)]^3$ the solution to \eqref{divinverse} for $f ={\rm div}( u -Pu)$ 
and set $M u=  Uf + Pu$ (where $Uf$ is continuously extended by zero into $\square \backslash B$). Now, the desired properties (i) and (ii) immediately follow by construction. For property (iii), %we compute
via \eqref{15.04.21} and \eqref{divinverse} we obtain obtain
\begin{flalign*}
	\Vert M u\Vert_{\left[H^1(\square)\right]^3}\,\,\, & \le\,\,\,  
	\Vert Pu \Vert_{\left[H^1(\square)\right]^3}\,+\, \Vert Uf \Vert_{\left[H^1(B)\right]^3}\,\, \le\,\, 
	C_P\,\Vert u \Vert_{\left[H^1(\square \backslash B)\right]^3} \,+\, C_B\, \Vert {\rm div} (u - Pu) \Vert_{L^2(B)} \\
	& \le\,\,\, C_P\left(1 + \sqrt{3}C_B\right) \Vert u \Vert_{\left[H^1(\square \backslash B)\right]^3} \,\,+\, C_B\, \Vert {\rm div}\, u  \Vert_{L^2(B)},
\end{flalign*}
and then  (iii)  follows from this last inequality and Korn's second inequality 
in $\square \backslash B$.
\end{proof}

Now,  one can readily prove that the main assumptions of the article hold by arguing as in the previous examples (in particular Example \ref{e.dp}).  We sketch the details below.

First, \eqref{as.b1} and \eqref{ass.alip} hold by essentially the same arguments as in the previous examples, cf. e.g.  \eqref{2.1classhom}.

\textbullet\, {\it Proof of \eqref{KA}}. 
We argue that \eqref{KA2} holds  with $c[w]$ chosen as a constant times $\int_{\square \backslash B} |w|^2$ (which is $\|\cdot\|_\t$-compact via Korn inequality). 
Indeed, for given $u \in \left[H^1_{per}(\square)\right]^3$, it follows by properties of $M$ that $v : = u - e^{-\i \t\cdot y}M(e^{\i \t \cdot y}u)$ belongs to $V_\t$. 
Hence, for any $w\in W_\t$ and $v$ chosen as above for $u=w$, we have $\|w\|_\t\le\|w-v\|_\t$ and arguing similarly to \eqref{h1prdp} we readily see via  Proposition \ref{SobModprop} that
\eqref{KA2} holds. %  with $c[w] : = \int_{\square \backslash B} |w|^2$. 

\textbullet\, {\it Proof of \eqref{contVs}}. It is straightforward to show that \eqref{contVs} holds for 
\begin{equation}\label{PDvstar}
V^\star_\t \,\,: =\,\, \left\{  v \in \left[H^1_0(B)\right]^3 \,\,\, \big| \,\,\, {\rm div}\, v + \i\, \t \cdot v = 0 \text{ in } B\right\}.
\end{equation}
 Indeed, for each $v_1 \in V^\star_{\t_1}$  it is sufficient to consider $v_2 = e^{\i (\t_1 - \t_2) \cdot y}\, v_1\in V^\star_{\t_2}$.

\textbullet \, {\it Proof of \eqref{distance}} follows from combining \eqref{KA2} where  $c[w] = k\int_{\square \backslash B} |w|^2$ with some $k>0$, 
and \eqref{PDelast:nondeg}. 

\textbullet \, {\it Proof of \eqref{H4}}. This is immediate  in the present setting with
\[
a_0'(v,u) \cdot \t \,\,=\,\, %2{\rm Re}\, \left( 
\int_{\square \backslash B} \lambda_1 \i\,  \t \cdot v\,\, \overline{{\rm div}\, u} \,\,\,+  
\int_{\square \backslash B} 2\mu_1 \i\, \t \odot v : \overline{e(u)} \,\,\,+ 
\int_B \lambda_2\, \i\, \t \cdot v\,\,\overline{{\rm div}\, u} %\right)
\]
and
\[a_0''(v,v) \t \cdot \t \,\,=\,\, %2{\rm Re}\, \left(  
\int_{\square \backslash B} \lambda_1 |\t \cdot v|^2 \,\,\,+ \int_{\square \backslash B}  2\mu_1 | \t \odot v|^2 \,\,\,+ \int_B \lambda_2 |\t \cdot v|^2. 
%\right).
\]
Now, from \eqref{aaspaceV} and \eqref{PDvstar} we see that one can choose the defect space $Z$  to be the $3$-dimensional vector space of constant functions. 
Indeed, with $V_\star:=V^\star_0$, \eqref{spaceZ} clearly holds, and \eqref{VZorth} holds with $K_Z=0$: 
\be
\label{kzpd}
\left(v_\star, z\right)_0\,\,=\,b_0\left(v_\star, z\right)\,\,=\,\int_B\,v_\star\cdot z\,\,=\,\,0, \quad \forall \,v_\star\in V_\star, \,\,\forall\,z\in Z, 
\ee
where the latter equality can be seen e.g. by integrating $(z\cdot y)\,{\rm div}\,v_\star$ over $B$ by parts. 
%is validated as in \eqref{4.17-dp}. 
It then routinely follows via a derivation similar to that leading to \eqref{dp.ahom} that 
\[
a^{\rm h}_\t(z,\tilde{z})\,\, =\,\, A^{\rm hom}_p z \odot \t \,:\, \overline{ \tilde{z} \odot \t}, \quad \,\, \forall z,\, \tilde{z} \in Z , \quad \forall \t \in \RR^3,
\] 
where $A^{\rm hom}_{p}$ is the homogenised tensor for the natural analogue of perforated elastic domain  for the present example. 
Namely, $A^{\rm hom}_{p}$ correspond to the periodic matrix-inclusion composite with $\lambda=\lambda_1$ and $\mu=\mu_1$ in the matrix $\square\backslash B$, and 
  with $\lambda=\lambda_2$ but 
zero shear modulus $\mu=0$ in the inclusion $B$. 

\textbullet\, The proof of \eqref{H5} is immediate from \eqref{btpd},  
and \eqref{H6} holds for $\mathcal{H} = \left[L^2(\square)\right]^3$, $d_\t$  the standard ($\t$-independent) 
$[L^2(\square)]^3$ inner product and $\mathcal{E}_\t$  multiplication by $e^{-\i\t\cdot y}$. In this setting, we observe that the bivariate operator (see Section \ref{s.bivariate}) is the (shifted for unity) two-scale homogenised limit operator, found in \cite{Co}, and therefore its spectrum is the semi-axis $[1,+\infty)$.


In what remains we shall specify our approximation given by general Theorem \ref{thm.IKunifest2} %thm.IKunifest} 
and provide some new results in the context of homogenisation theory 
for the present example. Estimate \eqref{IKfinal3-2} %IKfinal2} 
in particular implies
		\[
	\left\| u_{\ep,\t} \,-\, \left(z + e^{-\i \t \cdot y} v\right) \right\|_{\left[L^2(\square)\right]^3} \,\,\le\,\, 
	C_{10}^{1/2}\, \ep\, \| U\Gamma_\ep F(\t, \cdot) \|_{\left[L^2(\square)\right]^3}\,,
	\]
	where $z+v \in Z \,\dot{+}\, V_\star=V_0$ solves  \eqref{IKz3prob88}. Therefore, exploiting as in the earlier examples the $L^2$-unitarity of the scaling and 
	Gelfand transforms, one has 	
	\[
	\big\| u_{\ep} \,-\,\left( \mathbf{u}_\ep  + \mathbf{v}_\ep\right) \big\|_{[L^2(\RR^3)]^3} \,\,\le\,\, C_{10}^{1/2}\, \ep\, \| F\|_{\left[L^2(\RR^3)\right]^3},
	\]
	where $\mathbf{u}_\ep = \Gamma_\ep^{-1} U^{-1} z $ and $\mathbf{v}_\ep = \Gamma_\ep^{-1} U^{-1} e^{-\i\t\cdot y} v$. Let us determine the problems that $\mathbf{u}_\ep$ and $\mathbf{v}_\ep$ solve. 	First note, see \eqref{kzpd}, that $V_\star$ and $Z$ are  orthogonal with respect to $b_0$ and so problem \eqref{IKz3prob88} decouples. Thus,  $z \in Z$ solves
\begin{equation}\label{pdzprob}
A^{\rm hom}_p  z \odot \tfrac{\t}{\ep} \,:\, \overline{\tilde{z} \odot \tfrac{\t}{\ep}} \,\,+\,\, 
z \cdot \overline{\tilde{z}} \,\,\,=\,\,\, (U \Gamma_\ep F(\t,\cdot), \tilde{z})_{\left[L^2(\square)\right]^3}, \quad \forall \tilde{z} \in Z,
\end{equation}
and $v \in V_\star$ solves
\begin{equation}\label{pdVprob}
	\int_B 2\mu_2\, e(v) : \overline{e(\tilde{v})} \,\,+\, \int_B v \cdot \overline{\tilde{v}}  \,\,\,=\,\, 
	\int_B e^{\i \t \cdot y} U \Gamma_\ep F(\t,y)\cdot\overline{\tilde{v}(y)} \, {\rm d}y,  \quad \forall \tilde{v} \in V_\star.
\end{equation}
Now, similarly to Example \ref{e.class}, we take the inverse Gelfand and scaling transforms in  \eqref{pdzprob} to find  that 
$\mathbf{u}_\ep \in \left[H^1(\RR^3)\right]^3$ solves, cf. \eqref{Sep}, 
\[
-\, {\rm div}\, \sigma_0(\mathbf{u}_\ep) \,\,+\,\, \mathbf{u}_\ep\,\,\,=\,\,\, \mathcal{S}_\ep F 
\quad \text{in $\mathbb{R}^3$}, \qquad \sigma_0(u) = A^{\rm hom}_p e(u),
\]
for the smoothing operator $\mathcal{S}_\ep = \mathcal{F}^{-1} \chi\big(\tfrac{\cdot}{\ep}\big) \mathcal{F}$, where 
$\chi$ is the characteristic function of $\square^*$. Furthermore, as we saw in Example \ref{e.class}, cf \eqref{7.23-2},
one can remove $\mathcal{S}_\ep$ for $F\in [L^2(\RR^3)]^3$. Namely, one has
\[
\big\| \mathbf{u}_\ep - u_0 \big\|_{\left[L^2(\RR^3)\right]^3} \,\,\le\,\, \ep \pi^{-1} \gamma_0 \| F\|_{\left[L^2(\RR^3)\right]^3},
\]
where  $u_0 \in [H^1(\RR^3)]^3$ solves
\begin{equation}\label{pdhomp}
- \,{\rm div}\, \sigma_0(u_0)\,\, +\,\,u_0\,\,\,=\,\,\, F \quad \text{in $\mathbb{R}^3$},
\end{equation}
and $\gamma_0>0$ is a strong ellipticity constant of $A^{\rm hom}_p$ (i.e. 
$ A^{\rm hom}_p z \odot \xi : \overline{ {z} \odot \xi}\ge\gamma_0^{-1}|z|\,|\xi|$, $\forall z, \xi\in \CC^3$). 

Let us now turn to $\mathbf{v}_\ep$. By \eqref{pdVprob} and the properties of the Gelfand transform we conclude that $\mathbf{v}_\ep$ belongs to 
$V_\ep : = \left\{ u \in \left[H^1(\RR^3)\right]^3 \,\, \big| \, u = 0 \text{ in } \RR^3 \backslash B_\ep \text{ and } 
{\rm div}\, u = 0 \text{ in } B_\ep \right\}$, $B_\ep = \bigcup_{m \in \mathbb{Z}^3} \ep( B+m)$.
Moreover, from
% applying $\mathcal{E}_\t = e^{-\i\t\cdot y}$, $U$ and then $\Gamma_\ep$ to
 \eqref{pdVprob} we readily deduce that $\mathbf{v}_\ep$ solves
\begin{equation}\label{pdVprob2}
	\ep^2\int_{B_\ep} 2\,\mu_0\left(\tfrac{x}{\ep}\right)\, e(\mathbf{v}_\ep) \,:\, \overline{e(\tilde{v})}\,\,\, +\, 
	\int_{B_\ep} \mathbf{v}_\ep \cdot \overline{\tilde{v}}  \,\,\,=\,\, \int_{B_\ep} F\cdot \overline{\tilde{v}}   \quad \forall\, \tilde{v} \in V_\ep,
\end{equation}
which is nothing but a collection of Stokes problems on each inclusion $\ep(B+m)$ of $B_\ep$.
%Putting all this together, we see that the solution $u_\ep$ to \eqref{elastres} is approximated by $u_0 + \mathbf{v}_\ep$
In general,  $\mathbf{v}_\ep$ is not negligible and  the solution $u_\ep$ to \eqref{elastres} is approximated  up to leading order by $u_0 + \mathbf{v}_\ep$. 
However, if $F$ does not rapidly vary over $F_0^\ep$ (for simplicity if $\| F\|_{[H^1(F_0^\ep)]^3}$ is  bounded) we can see that
 $\mathbf{v}_\ep$ is $\ep$ small in $L^2$-norm. Indeed as, cf. \eqref{kzpd}, 
$\int_{\ep(B+m)} \tilde{v} = 0$ for each $m$, one has
 \[
 \big\| \mathbf{v}_\ep \big\|_{\left[L^2(\ep(B+m))\right]^3} \,\,\,\le\,\,\, 
\Big\| \, F \,-\, \tfrac{1}{|\ep(B+m)|}\int_{\ep(B+m)}F\,  \Big\|_{\left[L^2(\ep(B+m))\right]^3}\,\,\,\le\,\,\,   
\ep\, C_{B}\, \big\| \nabla f \big\|_{\left[L^2(\ep(B+m))\right]^{3\times 3}} 
 \]
 where $C_{B}$ is the Poincar\'{e}-Wirtinger constant for $B$.
%  Indeed, from \eqref{pdVprob2} we have
%\[
%\| \mathbf{v}_\ep \|_{[L^2(\RR^3)]^3} \le \| P_\ep f \|_{[L^2(\RR^3)]^3},
%\]
%where $P_\ep$ is the  $[L^2(\RR^3)]^3$ orthogonal projection onto $\overline{V_\ep}$, the closure of $V_\ep$ in $[L^2(\RR^3)]^3$ . Now, 
% elements of $\overline{V}_\ep$ clearly belong to $V_{\ep,h} : = \{ u \in [L^2(\ep(B+h))]^3 \, | \, u = 0 \text{ in } \ep(\square\backslash B+h)  \text{ and ${\rm div}\, u =0$}\}$. Since, constant functions belong to the $[L^2(\ep(B+h))]^3$-orthogonal complement of $V_{\ep,h}$ one has 
% 
%% are orthogonal to constant functions in $[L^2(\ep(B+h))]^3$. Thus,
% \begin{flalign*}
%  \| P_\ep f \|_{[L^2(\ep(B+h))]^3} & \le \| f - \tfrac{1}{|\ep(B+h)|}\int_{\ep(B+h)}f  \|_{[L^2(\ep(B+h))]^3} \le   \ep C_{PB} \| \nabla f \|_{[L^2(\ep(B+h))]^{3\times 3}} 
%%  & =  \ep C_{PB} \| \nabla f \|_{[L^2(F_0^\ep)]^{3 \times 3}} \le  \ep C_{PB} \| \nabla f \|_{[L^2(\square)]^{3 \times 3}} 
% \end{flalign*}
% for each $h \in \mathbb{Z}^3$,
%%functions that are supported on $F_0^\ep$ but constant on each $\ep(B+z)$,  $z \in \mathbb{Z}^3$,  are orthogonal to $\overline{V}_\ep$.  Thus, considering 
%%\[
%%f^\perp(x) := \sum_{z \in \ZZ^3} \chi_B\big( \tfrac{x-z}{\ep}\big) \frac{1}{|\ep(B+z)|}\int_{\ep(B+z)}f 
%%\]
%%for $\chi_B$ the characteristic function of $B \subset \RR^3$, we get 
%%\begin{flalign*}
%% \| P_\ep f \|_{[L^2(\square)]^3} & =   \| P_\ep (f-f^\perp) \|_{[L^2(\square)]^3} = \sum_{z\in \mathbb{Z}^3} \| f - \tfrac{1}{|\ep(B+z)|}\int_{\ep(B+z)}f  \|_{[L^2(\ep(B+z))]^3} \le  \sum_{z\in \mathbb{Z}^3}  \ep C_{PB} \| \nabla f \|_{[L^2(\ep(B+z))]^{3\times 3}} \\
%% & =  \ep C_{PB} \| \nabla f \|_{[L^2(F_0^\ep)]^{3 \times 3}} \le  \ep C_{PB} \| \nabla f \|_{[L^2(\square)]^{3 \times 3}} 
%%\end{flalign*}
%where $C_{PB}$ is the Poincar\'{e}-Wirtinger inequality for $B$. 

Putting all this together gives the following approximation results.
\begin{theorem}\label{PDapprox}
For $F \in \left[L^2(\RR^3)\right]^3$, the solution $u_\ep$ to   \eqref{elastres}, the solution  $u_0$ to the classical homogenised system \eqref{pdhomp} 
and the solution $\mathbf{v}_\ep$ to the inclusion Stokes problems \eqref{pdVprob2} satisfy the estimate
\[
\big\|u_\ep \,-\, \left(u_0+\mathbf{v}_\ep\right)\big\|_{\left[L^2(\RR^3)\right]^3} \,\,\,\le\,\,\, \ep\, C\, \big\| \,F\, \big\|_{\left[L^2(\RR^3)\right]^3}. 
\]
for the constant $C = {C_{10}^{1/2}} + \pi^{-1} \gamma_0$ that is independent of $\ep$ and $F$.
If additionally $F \in \left[H^1(\RR^3)\right]^3$, then 
\[
\big\|u_\ep \,-\, u_0\,\big\|_{\left[L^2(\RR^3)\right]^3} \,\,\,\le\,\,\, \ep\, C\, \big\| \,F\, \big\|_{\left[H^1(\RR^3)\right]^3}
\]
with $C = {C_{10}^{1/2}} + \pi^{-1} \gamma_0 + C_{B}$ independent of $\ep$ and $F$.
\end{theorem}


\subsection{Magnetic Schrodinger equation with a `strong' periodic field}
\label{magnschrod}
Up until now, in the examples, the space $V_\theta$ has either  been continuous in $\t$ or possessed an isolated discontinuity at the origin $\theta =0$. 
Here, we give a simple one-dimensional example demonstrating that  for certain physically motivated models 
it is possible to have isolated discontinuities appear at non-zero points in the $\theta$-space. 

For a given periodic magnetic field $A:C^1_{per}[0,1] \rightarrow \RR$, $\int_0^1 A(y)\,{\rm d}y \notin 2\pi \mathbb{Z}$,  a uniformly positive periodic potential $V \in L_{per}^\infty(0,1)$ and $F \in L^2(\mathbb{R})$ we consider the solution $u_\ep$ to the one-dimensional Magnetic-Schr\"{o}dinger equation
\[
 -\,\,  \left( \frac{d}{d x} \,-\,  \frac{\i}{\ep} A\left(\frac{x}{\ep}\right)\, \right)^2 u_\ep \,\,+\,\, V\left(\frac{x}{\ep}\right)u_\ep \,\,\,=\,\,\, F, 
\qquad x \in \RR,
\] 
with small parameter $0< \ep <1$. After the spatial rescaling $\Gamma_\ep$ and the Gelfand transform $U$, we find that $u_{\ep,\t}:= U \Gamma_\ep u_\ep(\t,\cdot)$ is the $[0,1]$-periodic solution to 
\[
- \,\, \ep^{-2}\left( \frac{d}{d y} \,+\, \i\, \t \, -\,  \i\, A(y)\right)^2 u_{\ep,\t} \,\,+\,\, V(y)u_{\ep,\t} \,\,\,=\,\,\, U\Gamma_\ep F(\t,\cdot), \quad 
a.e. \,  \ \t \in \Theta=[-\pi, \pi].
\]
The variational  formulation of this problem is of type \eqref{p1}  for $H = H^1_{per}(0,1)$, %$\Theta = [-\pi,\pi]$, 
$\l f, \tilde{u} \r = \int_0^1 U \Gamma_\ep F(\t,y)\,\overline{\tilde{u}(y)} \, {\rm d}y$,  and
%
%
%the spectral problem: For each $\theta \in [-\pi,\pi)$, find the eigenvalue-eigenfunction pairs $(\omega,v) \in (0,\infty) \times H^1_\theta(0,1)$   to
%$$
%\tfrac{1}{\ep^{2}}\big(-\i \tfrac{d}{dy} - A(y) \big)^2 v(y) = \omega v(y), \qquad y \in (0,1).
%$$
%That is, we aim to study the spectrum of the operator associated to the form $a : H^1_\theta \rightarrow \CC$, for each $\theta$, where
$$
a_\t[u] \,\,= \int_0^1 \Big|u'(y)\,+\, \i\, \t u(y) \,-\, \i\, A(y) u(y)\, \Big|^2\,{\rm d}y, \qquad 
b_\t[u] \,\equiv\, b_0[u]\,\, =\,\, \int_0^1 V(y)\, \big|u(y)\big|^2\,{\rm d}y, \qquad u \in H^1_{per}(0,1).
$$
Upon observing that
$$
a_\t[u] \,\,=\,\, \int_0^1 \left| \frac{d}{dy}\left(u(y)\exp \left[\i \t y -\i \int_0^yA(y'){\rm d}y'\right] \right)\right|^2 \, {\rm d}y,
$$
we can readily see that
%by an appropriate choice of the magnetic field $A$ we can produce discontinuities in the space $V_\theta$ for any $\theta$. Indeed, suppose 
\[
\begin{aligned}
V_\theta \,\,=\,\, \left\{ 
\begin{array}{lr}
\{ 0 \} & \theta \neq \theta_0, \\[5pt]
\Big\{ c\,\exp\Big[ -\,\i\,\Big(\,\t_0 y -\int_0^yA(y'){\rm d}y'\,\Big)\Big] \,\, \Big| \,\, c \in \CC\, \Big\} & \theta = \theta_0,
\end{array}
\right.
\end{aligned}
\]
where $\t_0\in (-\pi,\pi]\backslash \{0\} $ equals $\int_0^1 A(y){\rm d}y$ up to an integer multiple of $2\pi$. That is, $V_\theta$ is piecewise constant with isolated discontinuity at the point $\t_0$ determined by the mean value of the magnetic field $A$. Clearly, $\t_0$ can take any value in $\Theta$.
%there eists an $A$ such that $\int_0^1 A = \t_0$. 
%Note that 
Moreover, one can readily check that the assumptions \eqref{KA2}-\eqref{H6} hold with straightforward details left to the reader.

\subsection{A non-local example/differential-difference equation}
\label{sec:nonloc}
Here, we provide a simple example where the dependence of $a_\t$ on the Floquet-Bloch parameter $\t$ does not have to be quadratic (and not even polynomial). 
Consider $u_\ep \in H^1(\RR)$ the solution to one-dimensional problem 
\begin{equation}
\label{nonlocalp}
	\begin{aligned}
	\int_{\RR} A\left(\tfrac{x}{\ep}\right)  u'_\ep\,  \overline{\tilde{u}'}\,\,\, +\,\, \ep^{-2}	
	\int_{\RR} D\left(\tfrac{x}{\ep}\right) \big(  u_\ep(x + \ep) - u_\ep(x) \big)\,
	\overline{\big(  \tilde{u}(x + \ep) - \tilde{u}(x) \big)} \, {\rm d} x \,\,\,+\, 	
	\int_{\RR}  u_\ep\,\overline{\tilde{u}}\,\,\, =\, 	\int_{\RR} F\,\overline{\tilde{u}},  \\
	\forall \tilde{u} \in H^1(\RR),
	\end{aligned}
\end{equation}
for given $F\in L^2(\RR)$ and uniformly positive bounded $1$-periodic functions $A(y)$ and $D(y)$.  
We apply the usual scaling and Gelfand transforms, noticing via \eqref{gt1} that for a function with a shifted argument $\hat u(x)=u(x + \ep)$ one has 
$U \Gamma_\ep \hat u (\t,y)=e^{i\t}U \Gamma_\ep u (\t,y)$. 
As a result, 
we determine that $u_{\ep,\t} : = U \Gamma_\ep u_\ep (\t,\cdot) \in H^1_{per}(\square)$ solves a problem of the form \eqref{p1}
for $H = H^1_{per}(\square)$, $\square = [0,1]$, $\Theta = [-\pi,\pi]$, 
$\langle f , \tilde{u} \rangle =	\int_{\square} U \Gamma_\ep F(\t ,y)\,\overline{\tilde{u}}(y) \, {\rm d}y$, with the sesquilinear forms
\begin{equation}\label{nonlocalforms}
	\begin{aligned}
a_\t(u,\,\tilde u)\,\, =\,\, 	\int_{\square} A(y) \big(u'(y) +\, \i\, \t u(y)\big) \, \overline{ \big( \tilde u'(y) +\, \i\, \t \tilde u(y)\big)}\,{\rm d}y\,\, +  
\int_\square	 D(y)  \big|1 - e^{\i \t}\big|^2 u(y)\,\overline{\tilde u(y)}  \, {\rm d} y\,, \\ 
\quad \text{and} \quad 
b_\t(u,\,\tilde u)\,\, =\,\, 	\int_{\square}  u(y)\,\overline{\tilde u(y)}\,{\rm d}y.
	\end{aligned}
\end{equation}
Here, as in the classical homogenisation Example \ref{e.class}, we have 
\[
\begin{aligned}
	V_\theta = \left\{
	\begin{array}{lr}
		\{ 0 \}, & \theta \neq 0, \\[5pt]
		{\rm Span} (\mathbf{e}), & \theta = 0,
	\end{array} 
	\right. & \qquad & 
	W_\theta = \left\{
	\begin{array}{lr}
		H^1_{per}(\Box), & \theta \neq 0, \\[5pt]
		H^1_{per, 0} : =\left\{ u\in H^1_{per}(\Box) \,\, \big| \,\, \int_\Box u = 0 \right\}, & \theta = 0.
	\end{array} 
	\right.,
\end{aligned}
\]
Further, we can argue  with the exact same reasoning as in Example \ref{e.class} to show that \eqref{KA2}, \eqref{contVs} and \eqref{distance} all hold. In particular, we have $V_\star = \{ 0 \}$ and $Z = {\rm Span} ( \bf{e} )$. Next, 
since $1-e^{\i\t}=-\i\,\t+O(\t^2)$ as $\t\to 0$,   
it is clear that 
% since $|1 - e^{\i \t \cdot y}| \le \sqrt{n} \pi$ for $\t \in [-\pi,\pi]^n$, $y \in [0,1]^n$,  
 \eqref{H4} holds with  %$K_{a'} = \pi$, $K_{a''} = \pi^2 / 2$ and 
the forms 
	\[
	a_0'(v,u) \cdot \t \,\,=\,\, \i \int_\square A\,\t v \, \overline{ u'}, \qquad 
	a_0''(v,\tilde{v})\t \cdot \t \,\,=\,\,|\t|^2 \int_\square A\,  v \,\overline{\tilde{v}} \,\,+\,\, |\t|^2 \int_\square D\, v\, \overline{\tilde{v}} , \quad 
	\text{	$v, \tilde{v} \in V_0$, $u \in H^1_{per}(\square)$.}
	\]
As a result, Theorem \ref{thm.maindiscthm} is applicable. Arguing again as  in Example \ref{e.class}, 
since $a_0(u,\tilde u)=\int_\square A u'\,\overline{\tilde u'}$ 
we readily verify that for the corrector $N_\t$ defined by \eqref{cell:IKprob22}, $N_\t \bf{e}=\i\,\t\ourN$ where 
$\ourN$ is the solution to the classical corrector problem \eqref{IKclasscor} (for $n=1$), and  applying \eqref{ahom7.1} 
\[
a^h_{\xi}[\mathbf{e}] \,\,=\,\, \left(\,\left\langle A^{-1} \right\rangle^{-1}\,+\,\, \langle D \rangle\, \right) | \xi|^2,  \qquad \xi \in \RR,    
\]
where
% $A^{\rm hom} = \langle A^{-1} \rangle^{-1}$ is the classical homogenised coefficient (see \eqref{IKcoef} with $n=	1$) and 
$\langle h \rangle := \int_\square h(y)\,{\rm d}y$. 
Following further the pattern of Example \ref{e.class}, we observe that the 
solution $u_\ep$ to the original problem \eqref{nonlocalp} is approximated 
%via Theorem \ref{thm.maindiscthm} by the inverse Gelfand and scaling transformed solution of \eqref{IKz3prob}. The latter is further approximated 
in terms of the following homogenised problem, cf. \eqref{homeq}: 
\[
		-\,\,\left(\left\langle A^{-1} \right\rangle^{-1}\,\,+\,\, \langle D\rangle \right)u''\,\, +\,\, u \,\,=\,\,F. 
\]
As a result, estimates directly analogous to those in Proposition \ref{homeqest} hold, with one further refinement namely with the possibility of 
removing in the present example the smoothing operator $\mathcal{S}_\ep$ in the analogue of \eqref{IKH1est...}. 
(This follows from noticing  that in the present one-dimensional case $N'(y)$ is bounded, and then establishing the $L^2$-smallness of 
$N'(x/\ep)\big(\,(\mathcal{S}_\ep-I)u\big)'(x)$ via an argument similar to \eqref{7.23-3}.) 
Consequently,  
%following again the pattern of Example \ref{e.class}, Theorem \ref{thm.maindiscthm} eventually implies that the 
%following result.
%\begin{theorem} Let 
%solutions $u_\ep$ to the original problem \eqref{nonlocalp} and $u$  to the homogenised problem 
 the following estimates are satisfied: 
	\begin{gather*}
		\left\Vert u_\ep \,-\,\, \Big(u\,+\, \ep\ourN\left(\tfrac{\cdot}{\ep}\right)  u'\Big)\,  \right\Vert_{H^1(\mathbb{R})} \,\,\le\,\, \ep\, C\, 
		\big\Vert\, F\, \big\Vert_{L^2(\mathbb{R})}, \qquad \text{ and } \qquad 
		\big\Vert \,u_\ep \,-\, u\,  \big\Vert_{L^2(\mathbb{R})} \,\,\le\,\, \ep\, C\,\big\Vert \,F\,\big\Vert_{L^2(\mathbb{R})},
	\end{gather*}
with some constant $C$ independent of $\ep$ and $F$.
%\end{theorem}
%	 As $b_\t$ is independent of $\t$ then clearly \eqref{H5} holds for $L_b = 0$. 
%$C = 1$, $c[u] = \int_\square |u|^2$, $L_\star = 0$, $V_\star = \{ 0 \}$ and $\gamma

%  to quickly see this,  observe that this problem is a $\t_0$-shift of classical problem  in Example \ref{e.class}: $u_{\ep,\t} : = \exp(\i \t_0 y -\i \int^y A) (U\Gamma_\ep u_\ep)(\t,\cdot) \in H^1_{per}(0,1)$ solves 
%\[
%-\ep^{-2} \big(\tfrac{d}{dy} + \i (\t-\t_0)\big)^2 u_{\ep,\t}+ V u_{\ep,\t} = e^{\i(\t_0 y - \int^y A)}U\Gamma_\ep f(\t,\cdot) \in L^2(0,1) \qquad a.e.\ \t \in [-\pi+\t_0, \pi+\t_0].
%\]
%The details of reformulating our  main results in this context are left to the reader.
%
%
%
%
%
%
%Furthermore, we readily see that the main assumptions of our article hold for this problem since, after a simple transformation,  this problem is equivalent to $-u'' + Vu = e^{-\i \int_0^y A} f.$
%Now, bearing in mind the construction given in Section \ref{s.spconst}, we aim to show that condition \eqref{distance1} and, therefore, Theorem \ref{thm1} holds\footnote{Here, we are in the framework of problem \eqref{p1} for $c[\cdot]$ the usual $L^2$ norm, $b[\cdot]=a[\cdot] + c[\cdot]$ being taken as the equivalent $H^1$ norm.}. In particular, we demonstrate that the first eigenvalue $\lambda^{(1)}_\theta$ of the operator associated to the form $a[\cdot]$ on the space $W_\theta = H^1_\theta(0,1)$ for $\theta \neq \theta_0$, cf. \eqref{aeigs}, is 
%\begin{equation}
%\label{eigquadmagschr}
%\lambda^{(1)}_\theta = |\theta-\theta_0|^2, \qquad \forall \theta \neq \theta_0.
%\end{equation}
%Indeed, let $\theta \neq \theta_0$ and consider $w^{(1)}_\theta \in W_\theta = H^1_\theta$ the eigenfunction corresponding to $\lambda^{(1)}_\theta$, i.e. 
%$$
%\int_0^1 \big( \tfrac{d}{dx}w^{(1)}_\theta - \i A w^{(1)}_\theta  \big) \overline{\big( \tfrac{d}{dx}\phi - \i A \phi \big)} = \lambda^{(1)}_\theta \int_0^1 w^{(1)}_\theta \overline {\phi}, \qquad \forall \phi \in H^1_\theta(0,1).
%$$
%Equivalently, $w(y) = w^{(1)}_\theta(y)e^{-\i \int_0^y A}$ belongs to the space $H^1_{\theta-\theta_0}(0,1)$ and solves 
%$$
%\int_0^1 \tfrac{dw}{dx}\overline{\tfrac{d\varphi}{dx}} = \lambda^{(1)}_\theta \int_0^1 w \overline {\varphi}, \qquad \forall \varphi \in H^1_{\theta-\theta_0}(0,1).
%$$
%That is $\lambda^{(1)}_\theta$ is the first eigenvalue of the $(\theta-\theta_0)$-quasi-periodic Laplacian which is well-known to be given by \eqref{eigquadmagschr}.
%
%% as follows:
%%$$
%%\lambda^{(1)}_\theta = | \theta - \theta_0 |^2.
%%$$

%%\subsection{Thin Structures}
%
%%\subsection{Singular structures}(Following \cite{Zhi2000})
%%Let $\mu$ be a $Q$-periodic normalised Borel measure on $\mathbb{R}^d$. Consider the
%%% Sobolev space $H^1_\mu(\mathbb{R}^d)$ 
%%Hilbert space $H^1(\mathbb{R}^d,d\mu)$ defined as the closure of  $\{ (\phi,\nabla \phi) \, | \, \phi \in C^\infty_0(\RR^d) \}$ in $L^2_\mu \times [L^2_\mu(\RR^d)]^{d}$. For a fixed $u$, the elements $v$ of the pairs $(u,v)\in H^1(\mathbb{R}^d,d\mu)$ are called the gradients of $u$ and for a general measure $\mu$ these `gradients' are not unique. 
%%% Denoting $\Gamma_\mu(0)$ to be the set of gradients of zero, i.e.  $\Gamma_\mu(0) = \{ v \, | \,  (0,v) \in W_\mu(\mathbb{R}^d) \}$, we see that $\Gamma_\mu(0)$ is a closed linear subspace of $L^2_\mu(Q)$ and, therefore, $L^2_\mu(Q) = \Gamma_\mu(0) \oplus (\Gamma_\mu(0))^\perp$. One can readily show that for a given pair $ (u,v) \in W_\mu(\RR^d)$ the gradient $v$ of $u$  is uniquely determined in $(\Gamma_\mu(0))^\perp$. Therefore, we define the Sobolev space
%%%\[
%%%H^1(\RR^d, \mu) : = \{ u \in L^2_\mu(Q) \, | \, \exists w \in (\Gamma_\mu(0))^\perp \text{ such that } (u,w) \in W_\mu(\RR^d)   \}, 
%%%\]
%%%and for each $u \in H^1(\RR^d, \mu)$  we can
%%% introduce
%%%\footnote{One could then, in principle, use the quotient space $L^2_\mu(Q) \backslash \Gamma_\mu(0)$ to define the Sobolev space $H^1_\mu(\RR^d)$ as $\{ u\in L^2_\mu(Q) \, | \, \exists \phi_n \in C^\infty_0(Q) \text{ s.t.  $\phi_n \rightarrow u$ strongly in $L^2_\mu(Q)$ and $\nabla \phi_n \rightarrow v$ strongly in $(L^2_\mu(Q))^d$ to some $v\in (\Gamma_\mu(0))^\perp$} \}$} 
%%%the notation $\nabla_\mu w$ to denote this (unique) $w \in (\Gamma_\mu(0))^\perp$. 
%%% this unique component of any given gradient $v$ of $u$ and 
%%
%%For each $\ep >0$ we introduce the $\ep Q$-periodic Borel measure given by $\mu_\ep(B) = \ep^d \mu(\ep^{-1} B)$ for all Borel-measurable sets $B$.
%%We consider the resolvent problem: For fixed $f \in L^2_{\mu_\ep}(Q)$, find $(u_\ep,v_\ep) \in H^1(\RR^d,d\mu)$ the solution to 
%%\begin{equation}
%%\label{singe2}
%%\int_{\RR^d}  a(\tfrac{\cdot}{\ep}) v_\ep \cdot \overline{\nabla \phi }\, d{\mu_\ep}  + \lambda \int_{\RR^d}  u_\ep \overline{ \phi }\, d{\mu_\ep} =\int_{\RR^d}  f  \overline{ \phi }\, d{\mu_\ep}, \qquad \forall \phi \in C^\infty_0(\RR^d).
%%\end{equation} Here $\lambda >0$ and $a$ is a $Q$-periodic matrix-valued coefficients that
%%% with the usual boundedness  $a \in L^\infty(Q)$ with
%%satisfies the usual boundedness and ellipticity conditions: there exists $\nu >0$ such that
%% \begin{equation}
%% \label{mucoef}
%%\nu | \eta |^2 \le a(y) \eta \cdot \eta \le \nu^{-1} | \eta |^2, \qquad \forall \eta \in \CC^d, \ a.e.\ y \in Q. 
%% \end{equation}
%%The problem \eqref{singe2} is seen to be well-posed by an application of the Riesz representation theorem.
%%% provides a pair $(u,v) \in W_{\mu_\ep}(\mathbb{R}^d)$ that solves the problem
%%%\[
%%%\big( u, \phi \big)_{L^2_{\mu_\ep}(Q)} + \big(a(\tfrac{\cdot}{\ep}) v, \nabla \phi \big)_{(L^2_{\mu_\ep}(Q))^d} = \big( f, \phi \big)_{L^2_{\mu_\ep}(Q)}, \qquad \forall \phi \in C^\infty_0(\RR^d),
%%%\]
%%Furthermore,  for the solution $(u,v)$ it can be readily shown that
%%%$av \in (\Gamma_{\mu_\ep}(0))^\perp$ and consequently 
%%$v$ is uniquely defined, (cf \cite{Zhi2000}), and so we introduce the definition $\nabla_{\mu_\ep} u_\ep := v_\ep$.
%%% i.e. $v = \nabla_{\mu_\ep} u$. 
%% That is, $u_\ep$
%%%  consider $u_\ep \in H^1(\mathbb{R}^d,{\mu_\ep})$ 
%%is the (weak) solution to
%%\begin{equation}
%%\label{singweake1}
%%- \mathrm{div} ( a(\tfrac{\cdot}{\ep}) \nabla_{\mu_\ep} u_\ep) + \lambda u_\ep =f.
%%\end{equation} 
%%
%%
%%To incorporate problem \eqref{singweake1} (more precisely  \eqref{singe2}) in the framework presented in Section \ref{sec:pf} we argue as in the examples above: by first performing a unitary change of variables $x \mapsto \ep y$ and Bloch decomposition for PDEs defined on periodic Borel measure spaces that was introduced in \cite{ZhPaBloch}.
%%%\footnote{This transform can be arrived at by closing the Gelfand transform mapping described in Example \ref{e.class} in $L^2_\mu$.} 
%%%for more details. 
%%Indeed, following  an application of the unitary rescaling mapping $\mathcal{T_\ep} : L^2_{\mu_\ep}(\RR^d) \rightarrow L^2_\mu(\RR^d)$,  $\mathcal{T}_\ep f(y) = \ep^{d/2} f(\ep y)$ and then applying the unitary transform
%%\footnote{This transform is given for example as the continuous extension of the mapping
%%	$$
%%	\mathcal{U}f(\theta,y) : = \tfrac{1}{\sqrt{2\pi^d}}\sum_{z \in \ZZ^d} f(y -z) e^{\i \theta \cdot z}, \qquad f \in C^\infty_0(\RR^d).
%%	$$
%%to the space $L^2_\mu(\RR^d)$} $\mathcal{U}: L^2_\mu(\RR^d) \rightarrow L^2\big([-\pi,\pi)^d;L^2_\mu(Q)\big)$, $\theta \in [-\pi,\pi)^d$, it follows that $(u_{\ep,\theta} , v_{\ep,\theta} ) : =  \mathcal{U}_{\theta} \mathcal{T}_\ep ( u , \nabla_\mu u )$ solves
%%\begin{equation}
%%\label{singe2}
%%\ep^{-2}\int_{Q}  a v_{\ep,\theta} \cdot \overline{\nabla \phi }\, d{\mu}  + \int_{Q}  u_{\ep,\theta}  \overline{ \phi }\, d{\mu} =\int_{Q}  \mathcal{U}_\theta f(\ep \cdot)  \overline{ \phi }\, d{\mu}, \qquad \forall \phi \in C^\infty_\theta(Q),
%%\end{equation}
%%for $C^\infty_\theta(Q)$ denoting the space of infinitely smooth $\theta$-quasiperiodic functions, i.e. the set $\{ e^{\i \theta \cdot y} \psi \, | \, \psi \in C^\infty_{\#}(Q) \}$. Note that $(u_{\ep,\theta} , v_{\ep,\theta} )$ belongs to the Sobolev space $H^1_\theta(Q,d\mu)$ defined as the closure of $\{ (\phi, \nabla \phi) \, | \, \phi \in C^\infty_\theta(Q) \}$ in $L^2_\mu(Q)\times(L^2_\mu(Q))^d$, $\theta \in [-\pi,\pi)^d$. Furthermore, as per the discussion above, $v_{\ep,\theta}$ is
%% unique 
%%%  the unique  component of the gradients of u (in $H^1_{\theta}(Q,d\mu)$) that are orthogonal to the gradients of zero (also in $H^1_{\theta}(Q,d\mu)$). We 
%%and we use the definition $\nabla_{\mu}^{\theta} u_{\ep,\theta} : = v_{\ep,\theta}$. Therefore, setting $F_{\ep,\theta} : = \mathcal{U}_\theta f(\ep \cdot)$, we see that $u_{\ep,\theta}$ solves 
%%\begin{equation}
%%\label{difformmu}
%%\tfrac{1-\ep^2}{\ep^2} a\big( (u_{\ep,\theta}, \nabla^\theta_\mu u_{\ep,\theta}), (\phi, \nabla \phi) \big) + b\big( (u_{\ep,\theta}, \nabla^\theta_\mu u_{\ep,\theta}), (\phi, \nabla \phi) \big) = c(F_{\ep,\theta},\phi), \qquad \forall \phi \in C^\infty_\theta(Q),
%%\end{equation}
%%where $c[\cdot]$ is the standard $L^2_\mu(Q)$ norm,
%%\begin{equation}
%%\label{cformamu}
%%\begin{aligned}
%%a[(u,v)] = \int_{Q} a v \cdot \overline{v}\, d\mu, & \quad \mathrm{and } & b[(u,v)] = a[(u,v)] +  \lambda c[u], \qquad  (u,v) \in H^1(Q,d\mu),
%%\end{aligned}
%%\end{equation}
%%for $H^1(Q,d\mu)$ defined as the closure of $\{ (\phi, \nabla \phi) \, | \, \phi \in C^\infty(Q) \}$ in $L^2_\mu(Q)\times(L^2_\mu(Q))^d$. Note by the assumptions \eqref{mucoef} on the coefficients that $b$ is an equivalent norm for $H^1(Q,d\mu)$ and assumptions \eqref{cb} and \eqref{h1} hold.
%%
%%To study the structure of the spaces $V_\theta$ and $W_\theta$, see \eqref{spaceV}, we shall recall a standard property the measure $\mu$ is assumed to possess in the homogenisation of PDE \eqref{singweake1}; that is  $\mu$ is connected, i.e. if $(u,0) \in H^1(Q, d\mu)$  then $u$ is necessarily a constant.
%%%\begin{equation*}
%%%\label{muconnected}
%%%\text{  if $(u,0) \in H^1(Q, d\mu)$  then $u$ is necessarily a constant.}
%%%\end{equation*} 
%%Moreover, it is clear that $\mu$ is connected if the following Poincar\'{e} inequality holds: $\exists K_p > 0$ such that
%%\begin{equation}
%%\label{mupoincare}
%%\int_Q | \phi|^2 \, d\mu \le K_p \left(  \left| \int_Q \phi \, d\mu \right|^2 + \int_Q | \nabla \phi |^2 \, d\mu  \right), \qquad \forall \phi \in C^\infty(Q).
%%\end{equation}
%%
%%Notice that if $\mu$ is connected 
%%%\eqref{mupoincare} 
%%and \eqref{mucoef} holds then $V_\theta$ (cf. \eqref{spaceV}) and $W_\theta$ take the form
%%\begin{equation}
%%\label{muV}
%%\begin{aligned}
%%V_\theta = \left\{
%%\begin{array}{lr}
%%\{ 0 \} \times \{ 0 \}, & \theta \neq 0, \\[5pt]
%%\CC\times \{ 0 \}, & \theta = 0,
%%\end{array} 
%%\right., \quad \text{and } \quad  W_\theta = \left\{
%%\begin{array}{lr}
%%H^1_\theta(Q,d\mu), & \theta \neq 0, \\[5pt]
%%\{ u\in H^1_\#(Q,d\mu) \, | \, \int_Q u_1 \, d\mu= 0 \}, & \theta = 0.
%%\end{array} 
%%\right.
%%\end{aligned}
%%\end{equation}
%%%and, therefore,
%%%\begin{equation}
%%%\label{muW}
%%% W_\theta = \left\{
%%%\begin{array}{lr}
%%%H^1_\theta(Q,d\mu), & \theta \neq 0, \\[5pt]
%%%\{ u\in H^1_\#(Q,d\mu) \, | \, \int_Q u \, d\mu= 0 \}, & \theta = 0.
%%%\end{array} 
%%%\right.
%%%\end{equation}
%% As usual $H^1_{\#}(Q,d\mu)$ denotes $H^1_\theta(Q,d\mu)|_{\theta = 0}$. Notice also that the key assumption \eqref{KA} trivial holds for $K = \max\{1, \lambda\}$ in the setting \eqref{cformamu}.
%%% as the assumptions \eqref{mucoef} on the coefficients ensure that the form $a[\cdot] + c[\cdot]$ is an equivalent inner product to $b[\cdot]$. 
%%
%%
%%To apply the methodology presented in this article we require  Proposition \ref{KAimpliesOLDKA} to hold, and this result follows from \eqref{KA} and the additionally assumption that
%%\begin{equation}
%%\label{mucompact}
%%\{ u \in L^2_\mu(Q) \, | \, (u,v) \in H^1(Q,d\mu) \} \text{ is compactly embedded in } L^2_\mu(Q). 
%%\end{equation}
%%
%%
%%Let us now show that the assumptions of Theorem \ref{thm1} hold, namely that \eqref{distance1} holds. Indeed, \eqref{distance1} is an immediate consequence of the following result.
%%\begin{proposition}
%%\label{p:mudistance}
%%Assume that \eqref{mupoincare} holds. Then, for $\theta \in [-\pi,\pi)^d$ such that $ | \theta | \le 1/ \sqrt{(4 K_p)}$, the inequality 
%%\[
%%|\theta|^2 \int_Q | \phi|^2 \, d\mu \le\int_Q | \nabla \phi|^2 \, d\mu, \qquad \forall \phi \in C^\infty_\theta(Q).
%%\]
%%\end{proposition}
%%\begin{proof}
%%Fix $\phi \in C^\infty_\theta(Q)$ and set $\beta = \int_Q \phi e^{-\i \theta \cdot y} \, d\mu $, $\psi = \phi - \beta e^{\i \theta \cdot y}$. Then
%%\[
%%\phi = \beta e^{\i \theta \cdot y} + \psi, \quad \text{and} \quad \int_Q \psi   e^{-\i \theta \cdot y} \, d\mu = 0.
%%\]
%%Consequently, by \eqref{mupoincare} it follows that
%%\[
%%\int_Q |  \psi |^2 \, d\mu = \int_Q |  e^{-\i \theta \cdot y} \psi |^2 \, d\mu \le K_p \int_Q | \nabla (e^{-\i \theta \cdot y} \psi) |^2 \, d\mu \le 2  K_p \int_Q \big( |\theta|^2|\psi|^2 + | \nabla  \psi |^2 \big) \, d\mu.
%%\]
%%Now,  for $2 K_p|\theta|^2 \le 1/2$, we have
%%\[
%%\int_Q |  \psi |^2 \, d\mu \le 4K_p \int_Q | \nabla \psi |^2 \, d\mu.
%%\]
%%Note that $e^{\i \theta \cdot y}$ is orthogonal to $\psi$ and $\nabla \psi$ in $L^2_\mu(Q)$ and so we compute
%%\begin{flalign*}
%%\int_Q | \nabla \phi |^2 \, d\mu &= |\theta|^2 |\beta|^2 + \int_Q | \nabla \psi |^2 \, d\mu  \\& \ge |\theta|^2 |\beta|^2 + \tfrac{1}{4K_p} \int_Q |  \psi |^2 \, d\mu \\
%%& \ge  |\theta|^2 \big( |\beta|^2 + \int_Q | \psi |^2  \, d\mu \big) = |\theta|^2 \int_Q | \phi |^2  \, d\mu.
%%\end{flalign*} 
%%Hence the proposition is proved.
%%\end{proof}
%%
%%%Theorem \ref{thm1} reformulated in the setting \eqref{cformamu} takes the following form. 
%%%\begin{theorem}
%%%\label{muthm1}
%%%Assume that $\mu$ satisfies \eqref{mupoincare} and \eqref{mucompact}. Let $(W^1_\theta, W^2_\theta)$ be given by \eqref{claim1}-\eqref{claim3},  $f \in L^2(Q,d\mu)$, and $\ep \in (0, \tfrac{1}{4}(K(1+C_1))^{-1/2})$, $\theta \in \Pi$, $| \theta | < \min\big((4C_2)^{-1}, r_0, (4K_p)^{-1/2}\big)$. 
%%%Let $u_{\ep,\theta}$ solve  \eqref{difformmu},  and $u^{\rm int}_{\ep,\theta} \in V_\theta \overset{\cdot}{+}W^1_{\theta}$ be the solution to 
%%%\begin{equation}
%%%\label{aprojvplusw1}
%%%\ep^{-2} a(u^{\rm int}_{\ep,\theta}, \phi) + b(u^{\rm int}_{\ep,\theta}, \phi) = c(f, \phi), \quad \forall  \phi \in V_\theta \overset{\cdot}{+} W^1_\theta.
%%%\end{equation}
%%%
%%%Then, the inequality
%%%\begin{equation}
%%%\label{thm1eb}
%%%\ep^{-2}a[u_{\ep,\theta} -u^{\rm int}_{\ep,\theta} ]+	b[ u_{\ep,\theta} - u^{\rm int}_{\ep,\theta} ] \le C_{\rm int} \ep^2 c[f]
%%%\end{equation}
%%%holds with the constant 
%%%\[
%%%C_{\rm int} = 11 C_1 + 36 \big( KK_c(1+C_1) + C^{\,2}_2\nu^{-1} \big).
%%%\]
%%%\end{theorem}
%%Now that we have demonstrated that Theorem \ref{thm1} holds, let us
%%% end this example with 
%%show that the construction $(W^1_\theta,W^2_\theta)$ presented in Theorem \ref{prop:approxeigs} holds. Namely, we shall demonstrate that \eqref{noosc} holds for $P_\star = \mathrm{id}_{H^1_{\#}(Q,d\mu)}$.  Note that,
%% for a connected measure $\mu$, showing \eqref{noosc} for $P_\star = \mathrm{id}_{H^1_{\#}(Q,d\mu)}$ is equivalent to showing that
%%\[
%%\Vert (1 - e^{\i \theta \cdot y} ) \phi \Vert^2_{L^2_\mu(Q)} + \Vert \nabla [ (1 - e^{\i \theta \cdot y} ) \phi ]\Vert^2_{L^2_\mu(Q)} \le L_\star |\theta|^2 \big( \Vert \phi \Vert^2_{L^2_\mu(Q)} + \Vert \nabla \phi \Vert^2_{L^2_\mu(Q)} \big), \qquad \forall \phi \in C^\infty_{\#}(Q),
%%\]
%%and that this assertion is a trivial consequence of the properties of the exponential function. Hence \eqref{noosc}, and therefore Theorem \ref{prop:approxeigs}, holds if $\mu$ is connected (in particular if \eqref{mupoincare} holds).
%%
%%
%%As $P_\star = \mathrm{id}_{H^1_{\#}(Q,d\mu)}$, we determine, cf. \eqref{muV}, Proposition \ref{prop:pstar} (i) and (iii), that
%%\[
%%\begin{aligned}
%%W_\star = {\rm Im} P_\star = H^1_{\#}(Q,d\mu), & \quad & Z = \CC \times \{ 0 \}.
%%\end{aligned}
%%\] 
%%%and, in fact, Proposition \ref{prop:pstar} \eqref{pstar4}  holds for all $\theta \neq 0$:
%%%\[
%%%H^1_\theta(Q) = W_\theta = P_{W_\theta} e^{\i \theta \cdot y} \big( Z \oplus W_0 \big) = e^{\i \theta \cdot y} H^1_{\#}(Q).
%%%\]
%%Then, upon choosing  the basis $\hat{z}=(1,0)$ for $Z$, we follow the construction \eqref{w1w2p} in Section \ref{sec:suffconds} to obtain
%%\[
%%W^1_\theta = \left\{ \begin{array}{lr}
%%\{ e^{\i \theta \cdot y} \big( \alpha + \i \theta \cdot  N{\alpha}  \big) \, | \, \alpha \in \CC \}, & \theta \neq 0, \\
%%\{ 0 \}, & \theta =0,
%%\end{array} \right. \quad W^2_\theta = \{ e^{\i \theta \cdot y} w_0 \, | \, w_0 \in H^1_{\#}(Q), \int_Q w_0 = 0 \},
%%\]
%%where $N \alpha = \alpha( N^{(1)}, \ldots , N^{(d)} )$, for $N^{(j)} \in W_0$, the solutions to \eqref{cell:prob} for $z = (1,0)$, which takes the form: Find $N^{(j)} \in W_0 = \{ w \in H^1_{\#}(Q,d\mu) \, | \, \int_Q w_1 = 0 \}$ such that
%%\[
%%\int_{Q} a \big( N^{(j)}_2 + e_j ) \cdot \nabla \phi \, d\mu = 0, \qquad \forall \phi \in C^\infty_{\#}(Q), \int_Q \phi \, d\mu = 0,
%%\]
%%for $e_j$ is the $j$-th Euclidean basis vector. Now, from \eqref{cforma} it is  clear that since $\tfrac{1-\ep^2}{\ep^2} a + b = \tfrac{1}{\ep^2} a + \lambda c$, and consequently we can take 
%%\[
%%M^{\rm approx}_{\ep, \hat{z}}(\theta) = \ep^{-2} A^{\rm hom}_{\hat{z}}(\theta),
%%\]
%%cf. Section \ref{ahombhomexample}.  Moreover, from the above calculations we compute $A^{\rm hom}_{\hat{z}}(\theta)$, cf.  \eqref{def:ahomtheta}, to be
%%\[
%%A^{\rm hom}_{\hat{z}}(\theta) 
%%%= \int_Q a (\nabla (\i \theta \cdot N) + \i \theta)\cdot\overline{(\nabla (\i \theta \cdot N)  +  \i \theta)}= \int_Q a (\nabla (\i \theta \cdot N)  +  \i \theta)\cdot\overline{  \i \theta} 
%%=  a^{\rm hom} \theta \cdot \theta,
%%\]
%%where $a^{\rm hom}$ is the well-known homogenised matrix with components
%%\[
%%a^{\rm hom}_{ij} = \sum_{k=1}^d \int_Q  a_{ik} \big( \{ N^{(j)}_{2} \}_k +  \delta_{jk} \big) \, d\mu, \qquad  i,j \in \{1,\ldots,d\}.
%%\]
%%
%%
%%From Theorem \ref{thm1neigh}  and \ref{thm:ahom} we readily deduce in this setting the following result. 
%%\begin{theorem}
%%	\label{thm.eclassexample}
%%Suppose $\mu$ satisfies \eqref{mupoincare} and \eqref{mucompact}.	Let $u_{\ep,\theta}\in H^1_\theta(Q,d\mu)$ solve \eqref{difformmu}, and 
%%	\[
%%	\alpha_{\ep,\theta}  =  \big( \tfrac{1}{\ep^2} a^{\rm hom} \theta \cdot \theta + \lambda\big)^{-1}c( \mathcal{U}\mathcal{T}_\ep f(\theta, \cdot) ,1).
%%	\]
%%	
%%	Let $r = min{ \{ \tfrac{1}{4C_2}, r_0, \tfrac{1}{2r_1} \} }$ be as in Theorem \ref{thm:ahom} and set
%%	%$N$ to be the matrix whose $(i,j)$-th component is $N^{(j)}{e_i}$, and set 
%%	\[
%%	u^{\rm approx}_{\ep, \theta} : = \left\{ \begin{array}{ll}
%%	e^{\i \theta \cdot y} (\alpha_{\ep,\theta} +  \i \theta \cdot N  \alpha_{\ep,\theta}), & |\theta| < r, \\[2pt]
%%	0 & |\theta| \ge r.
%%	\end{array} \right. 
%%	\]
%%	Then, there exists $\kappa >0$ such that for $\forall \ep> 0$ and $\theta \in [-\pi,\pi)^d$, the following inequalities hold.
%%	\begin{enumerate}[(i)]
%%		\item{$
%%			\tfrac{1}{\ep^{2}} \Vert \nabla (u_{\ep,\theta} - u^{\rm approx}_{\ep,\theta}) \Vert^2_{L^2_\mu(Q)}	+   \Vert u_{\ep,\theta} - u^{\rm approx}_{\ep,\theta} \Vert^2_{L^2_\mu(Q)} \le \kappa \ep^2 \Vert \mathcal{U}\mathcal{T}_\ep f(\theta, \cdot) \Vert_{L^2_\mu(Q)}^2;$
%%		}
%%		\item{ $
%%			\Vert u_{\ep,\theta} - e^{\i \theta\cdot y} \alpha_{\ep,\theta} \Vert_{L^2_\mu(Q)}^2 	 \le \kappa \ep^2 \Vert \mathcal{U}\mathcal{T}_\ep f(\theta, \cdot) \Vert_{L^2_\mu(Q)}^2.
%%			$	}
%%	\end{enumerate}
%%\end{theorem}
%%
%%
%%Inequality (i) provides a novel corrector-type result for $u_\ep$ the solution to \eqref{singweake1}. Indeed, if we set $\widetilde{u}_\ep = \mathcal{T}^{-1}_\ep \mathcal{U}^{-1}  u^{\rm approx}_{\ep,\theta}$, then (i)  implies that
%%\[
%%\Vert u_\ep - \widetilde{u}_\ep \Vert_{H^1(\mathbb{R}^d,d\mu)} \le   \ep \sqrt{\kappa} \,\Vert f \Vert_{L^2_\mu(\mathbb{R}^d)}.
%%\]
%%Also, as in example \ref{e.class} (cf. Theorem \ref{compareclassical}), we have the following comparision to the standard homogenisation result.
%%\begin{theorem}
%%	\label{compareclassical}
%%Suppose the measure $\mu$ satisfies \eqref{mupoincare} and \eqref{mucompact}.	Let $u_\ep$ solve \eqref{singweake1} and consider $v \in H^1(\RR^d,d\mu)$ the  solution to 
%%	\begin{equation}
%%	\label{homeq}
%%	-{\rm div}\big( a^{\rm hom} \nabla_\mu v \big) + \lambda v = f.
%%	\end{equation}
%%Then, the inequality
%%\[
%%	\Vert u_\ep - v \Vert_{L^2_\mu(\RR^d)} \le C \ep \Vert f \Vert_{L^2_\mu(\RR^d)}
%%\] holds for some constant $\kappa>0$ independent of $\ep$ and $f$. 
%%\end{theorem}
%\section*{Acknowledgements}
%S. Cooper was supported by the EPSRC grant EP/M017281/1 (``Operator asymptotics, a new approach to length-scale interactions in metamaterials").



\appendix

%\section{Formal derivation of the limit problem (\ref{330})--(\ref{limsystem}).}
%\setcounter{section}{1}
%\renewcommand{\theequation}{\Alph{section}}
%\setcounter{equation}{0}
%\renewcommand{\theequation}{\mbox{\thesection.\arabic{equation}}}

\section*{Appendix A}\label{appa}
\setcounter{section}{1}
\renewcommand{\theequation}{\Alph{section}}
\setcounter{equation}{0}
\setcounter{theorem}{0}
\renewcommand{\theequation}{\mbox{\thesection.\arabic{equation}}}
We prove here Lemma \ref{propeth}, i.e. the existence of $\mathcal{E}_\t$ satisfying \eqref{Eprop1} and \eqref{Eprop2} (or equivalently \eqref{Eprop3}). We shall prove it in the following form:
\begin{proposition}\label{prop.Eexists} 
	Assume \eqref{contVs} and \eqref{H5}. Then, there exists a bijection $\mathcal{E}_\t : V_\star \rightarrow V_\t^\star$ such that
	\begin{gather}
	b_\t(  \mathcal{E}_\t  v_\star, \mathcal{E}_\t \tilde{v}_\star)=b_0(v_\star,\tilde{v}_\star), \quad \forall v_\star,\tilde{v}_\star\in V_\star; \label{Eunitary} \\
	\text{and there exists a constant $K'_b \ge 0$ such that } \nonumber  \hspace{\linewidth}\\
	%|b_\t(v_\t, v_0) - b_0(\mathcal{E}_\t v_\t , v_0) | \le K_b |\t| \| v_\t\|_0 \| v_0 \|_0, \quad \forall v_\t \in V_\t^\star, \forall v_0 \in V_0.  
	%\| \mathcal{E}_\t - I\| \label{Elip}
	\| \mathcal{E}_\t v_\star - v_\star \|_\t  \le K'_b |\t| \| v_\star\|_0, \quad  \forall v_\star \in V_\star. \label{Elip}
	\end{gather}
\end{proposition}
Notice that combining \eqref{H5} with \eqref{Elip} implies \eqref{Eprop2} and \eqref{Eprop3} 
(for $K_b = L_b + K K'_b$). 

Before proving this proposition, let us first demonstrate  that under \eqref{contVs} the subspace $V^\star_{\t_1}$ is isomorphic to $V_{\t_2}^\star$ when $\t_1$ and $\t_2$ are close. 
\begin{proposition}\label{PVbi}
	Assume \eqref{contVs}. Then, for $\t_1,\t_2\in\Theta$, $P({\t_1,\t_2}): V_{\t_1}^\star \rightarrow V_{\t_2}^\star$, given by $v \mapsto P_{V_{\t_2}^\star}  v$, is a bijection when $K L_\star |\t_1 - \t_2| < 1$.
\end{proposition}
\begin{proof} Fix $\t_1, \t_2 \in \Theta$, $K L_\star |\t_1 - \t_2| <1$, and 	$v_1  \in V_{\t_1}^\star$. 
By \eqref{contVs2} and \eqref{as.b1} we have  $  \| P_{W_{\t_2}^\star} v_1 \|_{\t_2} \le K L_\star |\t_1-\t_2|\, \| v_1 \|_{\t_2}$  and so 
	\begin{equation}\label{22.09.20e1}
	\| P_{V_{\t_2}^\star} v_1 \|_{\t_2}^2 \,\,=\,\, \| v_1 \|_{\t_2}^2  - \| P_{W_{\t_2}^\star} v_1 \|_{\t_2}^2  
	\,\,\ge\,\, \left(1 \,-\, \left( K L_\star |\t_1 - \t_2| \right)^2\right) \| v_1 \|_{\t_2}^2. 
	\end{equation}
This implies $P({\t_1,\t_2})$ is injective and has a closed range. It remains to prove that $ P_{V_{\t_2}^\star} V_{\t_1}^\star$ is not a proper subset of $V_{\t_2}^\star$. Suppose there exists $0 \neq v \in V_{\t_2}^\star$ such that $ v$ is orthogonal to  $ P_{V_{\t_2}^\star} V_{\t_1}^\star$ with respect to $(\cdot,\cdot)_{\t_2}$.  
Then, % $v = P_{V_{\t_1}^\star} v  + P_{W_{\t_1}^\star} v $ and
 $\big(v, P_{V_{\t_1}^\star} v\big)_{\t_2} = \big(v, P_{V_{\t_2}^\star}P_{V_{\t_1}^\star} v\big)_{\t_2} = 0$.
 Consequently,  we compute 
	\[
	\| v\|_{\t_2}^2 \,=\, \big(v , P_{V_{\t_1}^\star} v\big)_{\t_2} + \big( v, P_{W_{\t_1}^\star} v\big)_{\t_2} 
\, =\, 
	\big( v, P_{W_{\t_1}^\star} v\big)_{\t_2} \,\,\le \,\, \| v \|_{\t_2} \| P_{W_{\t_1}^\star} v\|_{\t_2} \, 
	\le \, K L_\star |\t_1-\t_2| \| v \|_{\t_2}^2,
	\]
where we have used  \eqref{as.b1} and \eqref{contVs2} in the last inequality.
% gives $\|  P_{W_{\t_1}^\star} v \|_{\t_1} \le  L_\star |\t_1 - \t_2| \| v \|_{\t_2}$.
This leads to the contradiction $\| v\|_{\t_2} = 0$ for $K L_\star |\t_1 - \t_2|  <1$. Hence $P_{V_{\t_2}^\star} V_{\t_1}^\star = V_{\t_2}^\star$. 
\end{proof}

\begin{proof}[Proof of Proposition \ref{prop.Eexists}]
	As $H$ is separable, the dimension (i.e. any basis) of $V_\t^\star$ is at most countable. Moreover, it
%	f the dimension of $V_\t^\star$ is finite for some $\t$ it 
follows from Proposition \ref{PVbi} that the dimension of $V_\t^\star$ is independent of $\t$ for close enough $\t_1$ and $\t_2$. Since $\Theta$ is assumed connected, $V_\t^\star$ and $V_\star$ are isomorphic for any $\t\in\Theta$, and in particular one can always find a $\mathcal{E}_\t$ which satisfies \eqref{Eunitary}. Moreover it is clear that, for any such $\mathcal{E}_\t$, and for any chosen $r>0$ \eqref{Elip} holds for $|\t| \ge r >0$ 
(with $K_b'$ replaced by $(1+K)/r$).  As such, we need only establishing \eqref{Elip} for the case 
$K L_\star |\t| <1/\sqrt{2}$, for which we construct below $\mathcal{E}_\t$ in a particular way.  
	
	
	%Let $\mathcal{V}_0$ and $\mathcal{V}_\t$ be 
	Consider the Hilbert spaces $(V_\star,b_0)$ and $(V_\t^\star, b_\t)$, and let $Q_\t : (V_\t^\star, b_\t) \rightarrow (V_\star,b_0)$ be the  inverse of $P(0,\t)$, which is bounded (see \eqref{22.09.20e1}),
	and let $Q_\t^* : (V_\star, b_0) \rightarrow (V_\t^\star,b_\t)$ be the adjoint of $Q_\t$.  
	Noticing that $Q_\t^*Q_\t: (V_\t^\star, b_\t) \rightarrow (V_\t^\star, b_\t)$ is bounded, self-adjoint and non-negative, 
	set $\mathcal{E}_\t = (Q_\t^* Q_\t)^{1/2} P(0,\t)$ and note that \eqref{Eunitary} holds: 
	\[
	b_\t(\mathcal{E}_\t v_\star,  \mathcal{E}_\t \tilde v_\star)= (\mathcal{E}_\t v_\star,  \mathcal{E}_\t \tilde v_\star)_\t= 
	\big((Q_\t^* Q_\t)^{1/2} P(0,\t) v_\star,\,  (Q_\t^* Q_\t)^{1/2} P(0,\t) \tilde v_\star\big)_\t \ \,= \,
	\]
	\[
	\ \ \ \ 
	\big((Q_\t^* Q_\t) P(0,\t) v_\star, \,   P(0,\t) \tilde v_\star\big)_\t\,=\, 
	\big(Q_\t P(0,\t) v_\star,\,  Q_\t  P(0,\t) \tilde v_\star\big)_0 = 
	( v_\star,   \tilde v_\star)_0\,=\,b_0( v_\star,   \tilde v_\star). 
	\]
	
	Let us prove \eqref{Elip}.	For arbitrary $v_\star \in V_\star$, $v_\t \in V_\t$, recall that 
	$\|v_\t\|_\t^2 = b_\t[v_\t]$, $\| v_\star \|_0^2 = b_0[v_\star]$. Now, since $(Q_\t^* Q_\t)^{1/2}$ is non-negative on $(V_\t^\star, b_\t)$ one has $\| v_\t \|_\t \le\| (I + (Q_\t^* Q_\t)^{1/2}) v_\t\|_\t$, which (upon setting $v_\t = P_{V^\star_\t}v_\star - \mathcal{E}_\t v_\star$) gives
%	 setting $v_\t = (I - (Q_\t^\star Q_\t)^{1/2})  P_{V_\t^\star} v_\star$, $v_\star \in V_\star$, gives 
\begin{flalign*}
%	\label{9.03.21}
\| P_{V^\star_\t}v_\star - \mathcal{E}_\t v_\star\|_\t  \le 
\| (I+(Q_\t^* Q_\t)^{1/2})\left(P_{V^\star_\t}v_\star - \mathcal{E}_\t v_\star\right) \|_\t	= 
	%		b_\t[   (I + (Q_\t^\star Q_\t)^{1/2})  (I - (Q_\t^\star Q_\t)^{1/2})P_{V_\t^\star} v_\star] =  
	\| (I-Q_\t^* Q_\t) P_{V^\star_\t} v_\star \|_\t	=\| P_{V^\star_\t} v_\star - Q_\t^* v_\star  \|_\t,
	 \, \forall v_\star \in V_\star.
%	=  b_\t[ (Q_\t^\star- P_{V_\t^\star} ) v_\star ]  =  b_\t[ (Q_\t^\star- P(0,\t)) v_\star ].
	%\le K^2 \| Q_\t^\star- Q_\t^{-1}) v_\star \|_0^2.
\end{flalign*}	
Combining the above inequality with \eqref{contVs2} gives
%Bounding the left-hand-side of the above inequality from below via \eqref{contVs2} gives
\begin{flalign}
		\label{9.03.21}
	\| v_\star - \mathcal{E}_\t v_\star\|_\t   
	= \| P_{W^\star_\t} v_\star +\left(P_{V^\star_\t} v_\star - \mathcal{E}_\t v_\star\right)\|_\t 
	\le L_\star |\t| \, \|v_\star\|_0 +  \| P_{V^\star_\t} v_\star  - Q_\t^* v_\star \|_\t,
	\quad \forall v_\star \in V_\star.
	%	=  b_\t[ (Q_\t^\star- P_{V_\t^\star} ) v_\star ]  =  b_\t[ (Q_\t^\star- P(0,\t)) v_\star ].
	%\le K^2 \| Q_\t^\star- Q_\t^{-1}) v_\star \|_0^2.
\end{flalign}	
It remains to estimate the difference $ P_{V^\star_\t}-Q^*_\t$ on $V_\star$. 
For this note that, for any $v_\t\in V_\t^*$, 
\begin{flalign*}
b_\t\big(P_{V^\star_\t}v_\star - Q^*_\t v_\star, v_\t \big) & \,=\,\, b_\t(v_\star, v_\t) \,-\, b_0 \left(v_\star,Q_\t v_\t \right) \\ 
&=\,\, b_\t\left(v_\star,v_\t - Q_\t v_\t\right) \,+\, b_\t\left(v_\star, Q_\t v_\t \right) \, -\,  
b_0\left(v_\star, Q_\t v_\t \right), \quad \forall v_\star \in V_\star, \, \forall v_\t \in V^\star_\t. 
\end{flalign*}
To estimate  the first term on the right, we use $P_{V_\t^\star} Q_\t v_\t=v_\t$ implying  $	Q_\t v_\t  - v_\t = P_{W_\t^\star} Q_\t v_\t$, and \eqref{contVs2}. To bound the difference of the remaining two terms  we use \eqref{H5}. Hence, one has
\[
|b_\t(P_{V^\star_\t}v_\star - Q^*_\t v_\star, v_\t )| \,\le\, \left(K^2 L_\star + L_b \right) |\t|\, \| v_\star \|_0 \| Q_\t v_\t\|_\t, \quad \forall v_\star \in V_\star, \, \forall v_\t \in V^\star_\t. 
\]
Now the above inequality and the bound $\| Q_\t v_\t \|_\t \le \sqrt{2} \| v_\t \|_\t$ (see \eqref{22.09.20e1} for $KL_\star |\t| \le 1/\sqrt{2}$)  give 
\[
 \| P_{V^\star_\t} v_\star  - Q_\t^* v_\star \|_\t \le \sqrt{2}\left(K^2 L_\star + L_b \right)|\t|\, \| v_\star \|_0, \quad \forall v_\star \in V_\star,
\]which along with \eqref{9.03.21} implies  \eqref{Elip} with $K_b'=\sqrt{2}\left(K^2 L_\star + L_b \right)$. 
% and \eqref{contVs2}  and \eqref{22.09.20e1} (for $KL_\star |\t| \le 1/\sqrt{2}$) 
%one has
%\[
%|b_\t(v_\star,v_\t - Q_\t v_\t) | \le K^2 L_\star |\t| \| v_\star \|_0 \| Q_\t v_\t \|_\t, 
%\]
%For the second term,
%Now by \eqref{H5} and \eqref{22.09.20e1} (for $KL_\star |\t| \le 1/\sqrt{2}$) one has 
%\[
%| b_\t(v_\star, Q_\t v_\t )  -  b_0(v_\star, Q_\t v_\t )| \le \sqrt{2}L_b|\t| \| v_\star \|_0 \| v_\t \|_\t, \quad \forall v_\star \in V_\star, \, \forall v_\t \in V^\star_\t. 
%\]
%Further, since $	Q_\t v_\t  - v_\t = P_{W_\t^\star} Q_\t v_\t$, then 
%%FurtheThe first  and second term  on the right is controlled by \eqref{contVs2} and \eqref{H5} respectively. As for the last term notice 
%%	\[
%%	\| P(0,\t) v_\star - v_\star \|_\t \le L_\star |\t| \| v_\star ||_0, \quad v_\star \in V^\star;
%%	\]
%%	and
%	\[
%	Q_\t v_\t  - v_\t  = ( I - P(0,\t)) Q_\t v_\t =  P_{W_\t^\star} Q_\t v_\t, \qquad v_\t \in V_\t^\star,
%	\]
%	and consequently, by \eqref{contVs2} and \eqref{22.09.20e1}, one has
%	$
%	\|Q_\t v_\t  - v_\t\|_0 \le \sqrt{2}K^2 L_\star |\t|\|v_\t\|_\t.
%%	\qquad v_\t \in V_\t^\star.
%$ Putting these assertions together gives
%\[
%\|P_{V^\star_\t}v_\star - Q^*_\t v_\star\|_\t \le K^2 ((1+\sqrt{2}) L_\star +  L_b )|\t| \|v_\star\|_0, \quad \forall v_\star \in V_\star,
%\]
%and combining with \eqref{9.03.21} gives  the desired estimate. 
%	
%	Whence $\| (I - (Q_\t^\star Q_\t)^{1/2}) P_{V_\t^\star} v_\star \|_\t \le K \| (Q_\t^\star- P(0,\t)) v_\star \|_0$. Now, \eqref{contVs2} states $P(0,\t)$ is Lipschitz  (at the origin in operator norm); furthermore as $P(0,\t)$ is Lipschitz, and  $Q_\t = P(0,\t)^{-1}$ is bounded, then $Q_\t$ is Lipschitz; finally \eqref{H5} and $Q_\t $ being Lipschitz imply that  $Q_\t^\star$ is Lipschitz and \eqref{Elip} follows. 
%	and we compute
%		\begin{flalign*}
%		b_\t[ (Q_\t^\star- P_{V_\t^\star} ) v_\star ] & = 
%	\end{flalign*}
%	\begin{flalign*}
%	b_\t ( (Q_\t^\star - I)v_\star  , v)  & = b_0 (  v_\star , (Q_\t - I) v ) + b_0 (  v_\star , v) -  b_\t (  v_\star , v) \\
%	&=  b_0 (  v_\star , (Q_\t - I) v ) + b_0 (  v_\star ,  P_{V_\t^\star}v) -  b_\t (  v_\star ,  P_{V_\t^\star}v) + b_\t (  v_\star ,  P_{W_\t^\star}v) 
%	\end{flalign*}
%	
%	\[
%	\| \mathcal{E}_\t v_\star - v_\star \| \le L_\star |\t| +  \| (Q_\t^* Q_\t - I) P_{V_\t^\star} v_\star  \| =  L_\star |\t| +  \| Q_\t^*  v_\star -    P_{V_\t^\star}  v_\star  \|
%	% \le \| Q_\t^* \| \| (Q_\t -  P_{V_\t^\star}^*)v_\star  \|
%	\]
%	\[
%	(( Q_\t^* Q_\t - I) P_{V_\t^\star} v_\star , v_\t )_\t = b_0 (v_\star, Q_\t v_\t) - b_\t(v_\star, v_\t) = b_\t (Q_\t^\star v_\star - v_\star, v_\t) 
%	\]
%	\[
%	(( Q_\t^* Q_\t - I) P_{V_\t^\star} v_\star , v_\t )_\t = b_0 (v_\star, Q_\t v_\t) - b_\t( P_{V_\t^\star }v_\star,  v_\t) 
%	\]
%	%
%	%\[
%	%(Q_\t^*  v_\star -    P_{V_\t^\star}  v_\star , \tilde{v}_\t)_\t = b_0(v_\star, Q_\t \tilde{v}_\t ) - b_\t(v_\star,\tilde{v}_\t)
%	%\]
%	% : V_\star \rightarrow   $A_\t : V_\t^\star \rightarrow V_\t^\star$, $v\mapsto A_\t v$ the solution to 
%	\begin{equation}\label{At}
%	b_\t (A_\t v, \tilde{v} ) = b_0 ( P_{V_\t^\star}^{-1} v ,P_{V_\t^\star}^{-1} \tilde{v}), \quad \forall \tilde{v} \in V_\t^\star. 
%	\end{equation}
%	Note $\eqref{At}$ is well-posed since 
%	% \eqref{H5} asserts that, for $L_b |\t| <1$, $b_\t$ is an equivalent inner product on $V_0$, and 
%	the right-hand-side of \eqref{At} is a continuous sesquilinear form on $V_\t^\star$.
%	
%	Setting $\mathcal{E}_\t : V_\t \rightarrow V_\star$, $\mathcal{E}_\t = A^{1/2}_\t P_{V_\star}$ it follows from \eqref{At} that \eqref{Eunitary} holds. 
%	
%	It remains to prove \eqref{Elip}. Now, 
%	\begin{flalign*}
%	b_\t(v_\t, v_0) - b_0(\mathcal{E}_\t v_\t , v_0) = b_\t(P_{W_\star}v_\t, v_0) + b_\t(P_{V_\star}v_\t, v_0) - b_0(P_{V_\star} v_\t , v_0)+b_0(P_{V_\star} v_\t -\mathcal{E}_\t v_\t , v_0).
%	\end{flalign*}
%	Note $|  b_\t(P_{W_\star}v_\t, v_0)| \le K^3 L_\star |\t| \| v_\t \|_0 \| v_0 \|_0$,  $|b_\t(P_{V_\star}v_\t, v_0) - b_0(P_{V_\star} v_\t , v_0) | \le L_b |\t| \| v_\t \|_0 \| v_0 \|_0$ and $|b_0(P_{V_\star} v_\t -\mathcal{E}_\t v_\t  , v_0)| \le \| (P_{V_\star} - \mathcal{E}_\t) v_\t\|_0 \| v_0\|_0$; so it remains to bound $P_{V_\star} - \mathcal{E}_\t v_\t = (I - A^{1/2}_\t) P_{V_\star} v_\t$.  Now, by \eqref{At} we compute 
%	\begin{flalign*}
%	b_0 ( ( A_\t-I)P_{V_\star} v_\t , \tilde{v}_\star )  & = b_\t ( (I -P_{V_\star}) v_\t , P^{-1}_{V_\star} \tilde{v}_\star  ) +  b_\t (    P_{V_\star} v_\t , (I -P_{V_\star}) P^{-1}_{V_\star} \tilde{v}_\star  )  + b_\t(P_{V_\star} v_\t, \tilde{v}_\star)- b_0(P_{V_\star} v_\t ,\tilde{v}_\star) \\
%	& \le K^2 \| P_{W_\star} v_\t \|_0  \|P^{-1}_{V_\star} \tilde{v}_\star    \|_0 +K^2 \| P_{V_\star}v_\t\|_0 \| P_{W_\star}P^{-1}_{V_\star} \tilde{v}_\star  \|_0+ L_b |\t| \| P_{V_\star}v_\t \|_0 \|  \tilde{v}_\star \|_0 \\
%	& \le 2K^3 L_\star|\t| \|v_\t\|_0 \| P^{-1}_{V_\star} \tilde{v}_\star \|_0  + L_b |\t| \| v_\t\|_0 \| \tilde{v}_\star \|_0.
%	\end{flalign*}
%	Now since $\| P^{-1}_{V_\star} \| \le 3/2$ for $KL_\star |\t| < 1/3$ (see \eqref{22.09.20e1}) then if $(4K^3 L_\star + 2L_b) |\t|  < 1$ inequality \eqref{Elip} holds. 
\end{proof}
%Now, we shall prove  Proposition \ref{p.unitaryequiv} holds for $\mathcal{E_\t}$ that additionally satisfying \eqref{H6}.

\section*{Appendix B}\label{appb}
\setcounter{section}{2}
%\renewcommand{\theequation}{\Alph{section}}
\setcounter{equation}{0}
\setcounter{theorem}{0}
\renewcommand{\theequation}{\mbox{\thesection.\arabic{equation}}}

We provide here some basic facts from theory of Bochner spaces, see e.g. \cite{ReeSim1,Hytonen}, as relevant and specialised 
to our setting in Section \ref{s.bivariate}, as well as justify some accompanying facts specific to our context. The latter follow quite standard arguments, but are still sketched here for the reader's convenience. 

Let $\mathcal{H}$ be a separable complex Hilbert space with inner product $(\cdot,\cdot)$ and associated 
norm $\|\cdot\|$. 
Bochner space $L^2(\RR^n;\mathcal{H})=:\mathbb{H}$ consists of all (Lebesgue measure zero equivalence classes of) weakly-measurable\footnote{i.e. $\forall\tilde u\in \mathcal{H}$, 
$\mathbb{R}^n\ni \xi \mapsto \big(u(\xi),\tilde u\big)\in \mathbb{C}$ is 
Lebesgue measurable} 
 maps 
$u:\RR^n\rightarrow \mathcal{H}$, %(with respect to Borel $\Sigma$-algebra on $H$), 
such that 
$\|u\|^2_\mathbb{H}\,:=\,\int_{\RR^n}\|u(\xi)\|^2{\rm d}\xi<\infty$. 
The latter defines the norm $\|u\|_\mathbb{H}$ in $\mathbb{H}$. 
With associated inner product 
\[
(u,\tilde u)_\mathbb{H}\,\,:=\,\int_{\RR^n}
\big(\,u(\xi)\,,\,\tilde u(\xi)\,\big)\,{\rm d}\xi,
\]
 $\mathbb{H}$ is known to become a (separable) Hilbert space. 
Similarly are defined 
$L^2(\Theta;\mathcal{H})$ for any measurable $\Theta\subset\RR^n$, with induced Lebesgue measure. 
\vspace{.07in}
% is, we mean the set of all functions $u:\Theta\rightarrow H$ 
%such that when extended by zero outside $\Theta$ for the whole of $\RR^n$ they belong to $L^2(\RR^n;H)$. 

Let us now show that the domain $\mathbb{D}$ of the form on the left-hand side of 
\eqref{Lproblint} is dense in $L^2(\RR^n;\mathcal{H}_0)$.
\begin{proposition}
\label{propb1}
$\mathbb{D}\,:=\,L^2\left(\RR^n; V_\star\right)\dot{+}
L^2\left(\RR^n, {\langle\xi\rangle^2}{\rm d}\xi;\, Z\right)$ is dense in 
$\mathbb{H}_0:=L^2(\RR^n;\mathcal{H}_0)$. 
\end{proposition} 
\begin{proof}
Let us show first that $\mathbb{V}_0:=L^2(\RR^n;V_0)$, where $V_0=V_\star\dot{+}Z$ is regarded as Hilbert space with 
inner product $b_0$, is dense in $\mathbb{H}_0$. 
Recall that both $V_0$ and $\mathcal{H}_0=\overline{V_0}$ are separable, and $V_0$ is compactly embedded into 
$\mathcal{H}_0$. 
Let $\lambda_0^{(k)}$, $k=1,2,...$, be the eigenvalues of $\mathbb{L}_0$, i.e. of $\mathbb{L}_\xi$ for $\xi=0$, cf. 
Corollary \ref{c.collspec}. Since the domain of $\mathbb{L}_0$ is in $V_0$, the associated eigenfunctions 
$\psi^{(k)}\in V_0\subset\mathcal{H}_0$, $k=1,2,...$, can be chosen to form 
an orthogonal basis in both $V_0$ and $\mathcal{H}_0$. 
Let $u\in\mathbb{H}_0$. Then, decomposing along this basis, for a.e. $\xi\in\RR^n$, 
$u(\xi)=\sum_{k=1}^\infty c_k(\xi)\psi^{(k)}$. 
Considering the truncated sums, $u^N(\xi):=\sum_{k=1}^N c_k(\xi)\psi^{(k)}$, clearly $u^N\to u$ in $\mathbb{H}_0$. 
On the other hand, since $b_0\left(\psi^{(k)},\psi^{(l)}\right)=
\delta_{kl}\lambda_0^{(k)}d_0\left[\psi^{(k)}\right]$, $u^N\in \mathbb{V}_0$ 
for any finite $N$. 
($\delta_{kl}=1$ if $k=l$ and $\delta_{kl}=0$ if $k\ne l$ is Kroneker symbol.) 
Hence $\mathbb{V}_0$ is dense in $\mathbb{H}_0$. 

Next we argue that $L^2\left(\RR^n,\langle\xi\rangle^2{\rm d}\xi; V_0\right)$ is in turn dense in $\mathbb{V}_0$. 
To see this, given $u\in\mathbb{V}_0$ with associated $u(\xi)\in V_0$ for a.e. $\xi$, we construct $u^N$ by  
setting $u^N(\xi):=\chi_{B_N}(\xi)u(\xi)$, where $\chi_{B_N}$ is the characteristic function of ball of radius 
$N$ in $\RR^n$ centered at the origin. It is easy to check that 
$u^N\in L^2\left(\RR^n,\langle\xi\rangle^2{\rm d}\xi;\, V_0\right)$ and $u^N\to u$ in $\mathbb{V}_0$ as $N\to\infty$, 
hence the stated density. 

Combining the above two density statements, we conclude that 
$L^2\left(\RR^n,\langle\xi\rangle^2{\rm d}\xi; V_0\right)$ is dense in $\mathbb{H}_0$. 
Finally, it remains to observe that $L^2\left(\RR^n,\langle\xi\rangle^2{\rm d}\xi; V_0\right)$ is a subset of 
$L^2\left(\RR^n; V_\star\right)\dot{+}L^2\left(\RR^n, {\langle\xi\rangle^2}{\rm d}\xi; Z\right)$. 
Hence the latter is dense in $\mathbb{H}_0$, as required. 
\end{proof}

\begin{proposition}
\label{propb2}
The form $\mathbb{A}$ defined by the left-hand side of \eqref{Lproblint} with domain $\mathbb{D}$ is closed. 
\end{proposition} 
\begin{proof}
Let $\left\{u_m\right\}_{m=1}^\infty\subset\mathbb{D}$, 
$u_m(\xi)=v_m(\xi)+z_m(\xi)$, be a Cauchy sequence with respect to $\mathbb{A}$, 
i.e. $\mathbb{A}\left[u_m-u_l\right]\to 0$ as $m,l\to\infty$. 
It follows from \eqref{ahcoercive} and \eqref{kappa0} that 
\[
\mathbb{A}\left[u_m-u_l\right]\,\ge\, 
\nu_\star\int_{\mathbb{R}^n}\,|\xi|^2\,b_0\big[z_m(\xi)-z_l(\xi)\big]{\rm d}\xi\,\,+\,
\left(1-K_Z\right)\int_{\mathbb{R}^n}
\Big(\,b_0\big[z_m(\xi)-z_l(\xi)\big]\,+\,b_0\big[v_m(\xi)-v_l(\xi)\big]\Big)\,{\rm d}\xi.
\]
The above implies that $\left\{v_m\right\}$ and $\left\{z_m\right\}$ are 
 Cauchy sequences in, respectively, $L^2\left(\mathbb{R}^n; V_\star\right)$ and 
$L^2\left(\mathbb{R}^n, \langle\xi\rangle^2; Z\right)$. 
It follows from the basic theory of Bochner spaces that both of these spaces are 
 complete, and hence there exist $v\in L^2\left(\mathbb{R}^n; V_\star\right)$ and 
$z\in L^2\left(\mathbb{R}^n, \langle\xi\rangle^2; Z\right)$ such that 
\[
\int_{\mathbb{R}^n}
\Big\{\left(1+|\xi|^2\right)b_0\big[z_m(\xi)-z(\xi)\big]\,\,+\,\,b_0\left[v_m(\xi)-v_l(\xi)\right]\Big\}\,\,{\rm d}\xi\,\,\,\rightarrow\,\,0. 
\]
For $u:=v+z\in \mathbb{D}$ this clearly implies that $\mathbb{A}\left[u_m-u\right]\to 0$, which 
completes the proof. 
\end{proof} 

\begin{proposition}
\label{propb3}
Let $h\in L^2\left(\Theta;\mathcal{H}_0\right)$ and regard it as an element of 
$L^2\left(\mathbb{R}^n;\mathcal{H}_0\right)$ by setting $h(\t)=0$ for $\t\notin\Theta$.  
Let $0<\ep<1$ and set $\xi=\t/\ep$. Then, for a.e. $\xi\in\mathbb{R}^n$, the unique solutions 
$v(\xi)+z(\xi)$ to \eqref{Lproblint} and \eqref{Linvprobl} coincide. 
\end{proposition} 
\begin{proof}
Let $v+z\in\mathbb{D}$ be the solution to \eqref{Lproblint}. 
Let $\tilde v_j+\tilde z_j\in V_\star\dot{+}Z$, $j=1,2,...$, form a dense set in 
$\left(V_\star\dot{+}Z,\, b_0\right)$. For any $j$, consider arbitrary 
$\varphi(\xi)\in C_0^\infty\left(\mathbb{R}^n\right)$ and set 
$\tilde v(\xi)=\tilde v_j\varphi(\xi)$ and $\tilde z(\xi)=\tilde z_j\varphi(\xi)$. Then from \eqref{Lproblint} 
\begin{equation}
\label{b3pf1}
\int_{\RR^n}\Big[\,a^{\rm h}_{\xi}\big(z(\xi),\tilde z_j\big) + 
b_0\big(v(\xi)+z(\xi),\, \tilde v_j+\tilde z_j\big)\,\,-\,\,
d_0\big(\,h(\ep\xi), \tilde v_j+\tilde z_j\big)\,\Big]\,\varphi(\xi)\,{\rm d}\xi\,\,
=\,\,0, \ \ \ 
\forall \varphi\in C_0^\infty\left(\mathbb{R}^n\right). %, \ \ \ \forall j=1,2,...\,. 
\end{equation}
Because of the density of %Since 
$C_0^\infty\left(\mathbb{R}^n\right)$, % is dense in $L^2\left(\mathbb{R}^n\right)$, 
the square bracket in \eqref{b3pf1} must vanish for a.e. $\xi\in\mathbb{R}^n$, for all 
$j\ge 1$. 
Finally, because of the density of $\left\{v_j+z_j\right\}$ in $V_\star\dot{+}Z$ (and hence 
also in $\mathcal{H}_0$), %\eqref{b3pf1} 
the above square bracket 
must vanish for a.e. $\xi\in\mathbb{R}^n$ for 
all $\tilde v+\tilde z\in V_\star\dot{+}Z$ which is equivalent to \eqref{Linvprobl}. 

The converse statement 
trivially follows from the uniqueness of the solutions to \eqref{Linvprobl} and \eqref{Lproblint}. 
%is trivial. Assuming \eqref{Linvprobl}, let $\tilde v+\tilde z\in \mathbb{D}$. 
%Then, for a.e. $\xi$,  $\tilde v(\xi)+\tilde z(\xi)\in V_\star\dot{+}Z$ and integrating 
%\eqref{Linvprobl} in $\xi$ over $\mathbb{R}^n$ yields \eqref{Lproblint}. 
\end{proof} 

We next recall the notion of a direct integral of operators and specialise it to the context of 
Section \ref{s.bivariate}. 
\begin{definition} \label{defb4} 
Let $\left(\mathcal{H}_0, d_0\right)$ be a complex Hilbert space and $\mathbb{L}$ be a self-adjoint operator in Bochner space $\mathbb{H}_0:=L^2\big(\mathbb{R}^n;\mathcal{H}_0\big)$. Let $\mathbb{L}_\xi$, 
$\xi\in\,\mathbb{R}^n$, be a family of self-adjoint operators in $\mathcal{H}_0$ with their 
spectra (say) contained in $[1,+\infty)$ and which are 
weakly-measurable\footnote{i.e. $\forall g,\tilde u\in \mathcal{H}_0$,  
$\xi\mapsto d_0\left(\mathbb{L}_\xi^{-1}g,\tilde u\right)$ is Lebesgue-measurable 
as a map from $\mathbb{R}^n$ to $\mathbb{C}$} in $\xi$. 
We say that $\mathbb{L}$ is a direct integral of $\mathbb{L}_\xi$ over $\xi\in\mathbb{R}^n$, 
denoted $\mathbb{L}=\int_{\RR^n}^\oplus \mathbb{L}_\xi \, {\rm d}\xi$, and $\mathbb{L}_\xi$ 
are fibers of $\mathbb{L}$, if 
\begin{enumerate}
\item[(i)] $u\in \mathbb{H}_0$ is in the domain ${\rm dom}\,\mathbb{L}$ of $\mathbb{L}$,  if 
and only if $u(\xi)\in {\rm dom}\,\mathbb{L}_\xi$ for a.e. $\xi\in\mathbb{R}^n$ and 
$\int_{\mathbb{R}^n}d_0\big[\mathbb{L}_\xi u(\xi)\big]\,{\rm d}\xi<+\infty$; 
\item[(ii)] $\forall u\in {\rm dom}\,\mathbb{L}$, 
$\big(\mathbb{L}u\big)(\xi)\,=\,\mathbb{L}_\xi \big(u(\xi)\big), \ \ \ \mbox{ for a.e. } \ 
\xi\in\mathbb{R}^n$. 
\end{enumerate}
\end{definition} 
\begin{proposition}
\label{propb5}
For $\mathbb{L}_\xi$, $\xi\in\mathbb{R}^n$, and $\mathbb{L}$ as defined in Sections 
\ref{s.spbt} and \ref{s.bivariate} respectively, 
\[
\mathbb{L}\,=\,\int_{\RR^n}^\oplus \mathbb{L}_\xi \, {\rm d}\xi, \ \ \ \mbox{and } \ \ 
\mathbb{L}^{-1}\,=\,\int_{\RR^n}^\oplus \mathbb{L}_\xi^{-1}\, {\rm d}\xi. 
\]
\end{proposition} 
\begin{proof}
First, $\mathbb{L}_\xi$ are weakly-measurable as for any $g\in \mathcal{H}_0$, 
$d_0\left[ \mathbb{L}_\xi^{-1}g\right]$ is continuous in $\xi$. 
As a brief sketch for proving the latter, consider $\xi_1,\xi_2\in\RR^n$ with associated $u_j=\mathbb{L}_{\xi_j}^{-1}g$, 
$j=1,2$.  Then, via \eqref{Sform},  $\mathbb{S}_{\xi_j}(u_j, u_1-u_2)=d_0(g,u_1-u_2)$, $j=1,2$. 
Hence, subtracting, $S_{\xi_1}[u_1-u_2]=S_{\xi_2}(u_2,u_1-u_2)-S_{\xi_1}(u_2,u_1-u_2)$. 
When $\xi_2\to\xi_1$, the latter difference form becomes small, see \eqref{Sform}. 
Hence, via standard arguments, $d_0[u_1-u_2]\le S_{\xi_1}[u_1-u_2]\to 0$ as $\xi_2\to\xi_1$, as required. 
The rest of the proof essentially follows that of Proposition \ref{propb3}. 
By definition, $v+z\in \mathbb{D}$ is in ${\rm dom}\,\mathbb{L}$ if there exists 
$H\in\mathbb{H}_0$ such that 
\begin{equation}
\label{b4pf1}
\mathbb{A}\big(v+z,\,\tilde v +\tilde z\big)\,\,=\,\,
\int_{\mathbb{R}^n}\,d_0\big(H(\xi),\,\tilde v +\tilde z\big) \, {\rm d}\xi, \ \ 
\ \forall \tilde v +\tilde z\in \mathbb{D},  
\end{equation}
where $\mathbb{A}$ is the form on the left-hand side of \eqref{Lproblint}. 
Arguing then as in the proof of Proposition \ref{propb3}, we conclude that for a.e. 
$\xi\in\mathbb{R}^n$, $\forall\, \tilde v+\tilde z\in V_\star\dot{+}Z$, 
$\,\,\mathbb{S}_\xi\big(v(\xi)+z(\xi),\,\tilde v +\tilde z\big)=
d_0\big(H(\xi),\tilde v +\tilde z\big)$ 
where the form $\mathbb{S}_\xi$ is given by \eqref{Sform}. 
The latter implies both conditions in Definition \ref{defb4}, so $\mathbb{L}$ is 
the direct integral of $\mathbb{L}_\xi$. 
For the inverses, condition $(i)$ in Definition \ref{defb4} trivially follows from the 
well-posedness of \eqref{Lproblint}. 
Also, if $H\in\mathbb{H}_0$ and $ u=\mathbb{L}^{-1}H$ then $(ii)$ implies 
$\mathbb{L}_\xi^{-1}\big(H(\xi)\big)=u(\xi)= 
\big(\mathbb{L}^{-1}H \big)(\xi)$ for  a.e. 
$\xi\in\mathbb{R}^n$, as required. 
\end{proof}
Fourier transform $\mathcal{F}$ is known to be a well-defined unitary operator in a Bochner space 
$L^2\left(\mathbb{R}^n; \mathcal{H}\right)$, together with its inverse $\mathcal{F}^{-1}$: 
\begin{definition}
\label{defb6}
Given $u\in L^2\left(\mathbb{R}^n; \mathcal{H}\right)=:\mathbb{H}$, $\mathcal{F}u=:\hat u$ and 
$\mathcal{F}^{-1}u=:\check u$ are such elements of $\mathbb{H}$ that, 
 respectively for a.e. $\xi\in\mathbb{R}^n$ and a.e. $x\in\mathbb{R}^n$, 
\begin{equation}
\label{ftdef}
\big(\hat u(\xi),\tilde u\big)\,=\,
(2\pi)^{-n/2}\int_{\mathbb{R}^n}e^{-{\rm i}x\cdot\xi}\big(u(x),\tilde u\big){\rm d}x, \ \ \ 
\big(\check u(x),\tilde u\big)\,=\,
(2\pi)^{-n/2}\int_{\mathbb{R}^n}e^{{\rm i}x\cdot\xi}\big(u(\xi),\tilde u\big){\rm d}\xi, \ \ \ 
\forall \tilde u\in \mathcal{H}. 
\end{equation} 
[The above integrals denote conventional (inverse) Fourier transforms 
%of $\big(u(\cdot),\tilde u\big)\
in $L^2\left(\mathbb{R}^n\right)\ni \big(u(\cdot),\tilde u\big)=:g$, i.e. by first defining the integrals 
for $g\in L^2\left(\mathbb{R}^n\right)\cap L^1\left(\mathbb{R}^n\right)$
and then extending by continuity.] 
Given $u$, the above $\hat u$ and $\check u$ are (uniquely) well-defined. 
Indeed, e.g. for $\hat u$, for a.e.  $x$ decompose $u(x)$ along an orthonormal basis $e^j$, 
$j=1,2,...,$ in $\mathcal{H}$: 
$u(x)=\sum_{j=1}^\infty c_j(x)e^j$ where $c_j=\big(u(\cdot),e^j\big)\in L^2\left(\mathbb{R}^n\right)$ and 
$\|u\|^2_{\mathbb{H}}=\sum_{j=1}^\infty\|c_j\|^2_{L^2(\mathbb{R}^n)}<\infty$. 
Then \eqref{ftdef} for $\tilde u=e^j$ 
implies $\hat u(\xi)=\sum_{j=1}^\infty \hat c_j(\xi)e^j$, where $\hat c_j\in L^2\left(\mathbb{R}^n\right)$ is 
the conventional Fourier transform of $c_j$. Then \eqref{ftdef} is held for arbitrary $\tilde u\in \mathcal{H}$ by 
linearity and continuity, and Plancherel theorem for $c_j$ and the orthogonality imply the unitarity of 
$\mathcal{F}$ and $\mathcal{F}^{-1}$: 
\begin{equation}
\label{planch}
\big( u,\,\tilde u\big)_{\mathbb{H}}\,\,=\,\,\big(\mathcal{F} u,\,\mathcal{F}\tilde u\big)_{\mathbb{H}}\,\,=\,\,
\big(\mathcal{F}^{-1} u,\,\mathcal{F}^{-1}\tilde u\big)_{\mathbb{H}}, \ \ \ \forall u,\tilde u\in\mathbb{H}. 
\end{equation}
\end{definition}
Bochner Sobolev space $H^1\left(\mathbb{R}^n; \mathcal{H}\right)$ can be defined in two equivalent ways, 
see e.g. \cite{ReeSim1} and \cite{Hytonen}: via generalised derivatives or via Fourier transform. 
Adopting the former, 
\begin{definition}
\label{defb7}
It is said that $u\in L^2\left(\mathbb{R}^n; \mathcal{H}\right)$ has a (first order) $L^2-$generalised derivative 
$\partial_{x_j}u\in L^2\left(\mathbb{R}^n; \mathcal{H}\right)$, $j=1,2,...,n$, if for all 
$\varphi\in C_0^\infty\left(\mathbb{R}^n\right)$ and all $\tilde u\in \mathcal{H}$, 
\[
\int_{\mathbb{R}^n}\big( u(x),\,\tilde u\big)\,\partial_{x_j}\varphi(x)\,{\rm d}x\,\,=\,\,
-\,\,\int_{\mathbb{R}^n}\big( \partial_{x_j}u(x),\,\tilde u\big)\,\varphi(x)\,{\rm d}x. 
\]
$H^1\left(\mathbb{R}^n; \mathcal{H}\right)$ is the space of all $u\in L^2\left(\mathbb{R}^n; \mathcal{H}\right)$ 
having all the first-order $L^2-$generalised derivatives. 
\end{definition}
It is known that $H^1\left(\mathbb{R}^n; \mathcal{H}\right)$ is a separable Hilbert space with inner 
product 
\[
\big(u,\,\tilde u\big)_{H^1\left(\mathbb{R}^n; \mathcal{H}\right)}\,\,:=\,\int_{\RR^n}\Big[\, 
\big(u(x),\tilde u(x)\,\big)\,+\,
\sum_{j=1}^n\Big(\,\partial_{x_j}u(x)\,,\,\partial_{x_j}\tilde u(x)\Big)\,\Big]\,{\rm d}x.
\]
Finally, $u\in H^1\left(\mathbb{R}^n; \mathcal{H}\right)$ if and only if 
$u\in L^2\left(\mathbb{R}^n; \mathcal{H}\right)$ and 
$\mathcal{F}u\in L^2\Big(\mathbb{R}^n,\langle\xi\rangle^2{\rm d}\xi;\, \mathcal{H}\Big)$, with obvious definition 
of the latter weighted Bochner space. 
For $u\in H^1\left(\mathbb{R}^n; \mathcal{H}\right)$, the gradient 
$\nabla u \in \Big(L^2\left(\mathbb{R}^n; H\right)\Big)^n$ is defined in a standard way, 
and (in the component-wise sense) 
$\mathcal{F}(\nabla u)(\xi)\,=\,{\rm i}\xi\,\mathcal{F}(u)(\xi)$. 

\begin{lemma}
\label{lemft}
Form $Q$ on domain $\check{\mathbb{D}}$ determines a self-adjoint operator $\mathcal{L}$ in Hilbert space 
$\mathbb{H}_0=L^2\left(\RR^n; \mathcal{H}_0\right)$, $\mathcal{H}_0=\left(\overline{V_0},\, d_0\right)$, 
$V_0=V_\star\dot{+}Z$. In fact, 
$\mathcal{L}=\mathcal{F}^{-1}\mathbb{L}\,\mathcal{F}$, 
where $\mathcal{F}$ is the Fourier transform in $\mathbb{H}_0$. 
\end{lemma}
\begin{proof}
Since $\mathcal{F}$ is a unitary operator in $\mathbb{H}_0$, see \eqref{planch}, it suffices to 
show that $\check{\mathbb{D}}=\mathcal{F}^{-1}\mathbb{D}$ and 
\begin{equation}
\label{ftforms}
Q\big(u+v,\tilde{u} + \tilde{v}\big) \,\,=\,\,\mathbb{A}\big(\,\mathcal{F}^{-1}(u+v),\, 
\mathcal{F}^{-1}\left(\tilde{u} + \tilde{v}\right)\big), \quad \forall\,\, u+v,\tilde u+\tilde v\in \mathbb{D}, 
\end{equation} 
where $\mathbb{A}$ is the form on the left-hand side of \eqref{Lproblint}. 
First notice that by properties of the Fourier transform %(Appendix B)  
 $\check{\mathbb{D}}=\mathcal{F}^{-1}\mathbb{D}$. 
%, and so $\check{\mathbb{D}}$ is dense in $\mathbb{H}_0$ by Proposition \ref{propb1}. 
%Let for some $f\in \mathbb{H}_0$, $v+z=\mathbb{L}^{-1}f$, i.e. \eqref{Lproblint} is satisfied 
%with $h(\ep\xi)$ replaced by $f(\xi)$, and let $F:=\mathcal{F}^{-1}f$. 
To prove \eqref{ftforms}, %we need to show that 
%We need to show that then 
%equivalently $\check v+u:=\mathcal{F}^{-1}(v+z)$ solves 
%\begin{equation}
%\label{formft}
%Q\big(u+\check{v},\, \tilde u+\tilde v\big)\,\,=\,\,
%\int_{\mathbb{R}^n} d_0\big( F(x),\, \tilde u(x)+\tilde v(x)\big)\,{\rm d}x, 
%\ \ \  \forall\, \tilde u+\tilde v\in \check{\mathbb{D}}. 
%\end{equation}
%Since $\mathcal{F}^{-1}\mathbb{D}=\check{\mathbb{D}}$, 
it would suffice to show that a variant 
of Plancherel theorem holds for all the forms entering \eqref{ahgrad}--\eqref{Lproblint}. 
Namely, $\forall\, v+z$, $\tilde v+\tilde z\in \mathbb{V}_0:=L^2\left(\mathbb{R}^n; \left(V_0, b_0\right)\right)$, 
$V_0=V_\star\dot{+}Z$, 
\begin{equation}
\label{plancher}
\int_{\mathbb{R}^n} \mathfrak{b}\big(v(\xi)+z(\xi),\, \tilde v(\xi)+\tilde z(\xi)\big)\,{\rm d}\xi\,\,=\,\,
\int_{\mathbb{R}^n} \mathfrak{b}\big(\mathcal{F}^{-1}v(x)+\mathcal{F}^{-1}z(x),\, 
\mathcal{F}^{-1}\tilde v(x)+\mathcal{F}^{-1}\tilde z(x)\big)\,{\rm d} x, 
\end{equation} 
for $\mathfrak{b}=%d_0$, 
b_0$ and $\mathfrak{b}=a^h_{jk}$, $j,k=1,...,n$. 
(For %$\mathfrak{b}=d_0$ one has to replace in \eqref{plancher} $v+z$ by $f\in\mathbb{H}_0$, and for 
$\mathfrak{b}=a^h_{jk}$ one has to set in \eqref{plancher} $v=\tilde v=0$.) 
%Recall that in \eqref{plancher} $\mathcal{F}$ denotes 
%Fourier transform in $\mathbb{H}_0=L^2\left(\RR^n; \mathcal{H}_0\right)$.
\vspace{.05in}

1. 
%For $d_0$, \eqref{plancher} immediately follows by the isometry of $\mathcal{F}$ in 
%$\mathbb{H}_0=L^2\left(\RR^n; \left(\overline{V_\star\dot{+}Z},d_0\right)\right)$, cf \eqref{planch}. 
%\vspace{.06in}
%2. 
Aiming at \eqref{plancher} for $\mathfrak{b}=b_0$, 
regard $\mathbb{V}_0$ as a Bochner space in its 
own and let $\mathcal{F}_b$ be the corresponding Fourier transform according to 
Definition \ref{defb6}. 
By the unitarity of $\mathcal{F}_b$ we immediately have
\[
\int_{\mathbb{R}^n} {b}_0\big(v(\xi)+z(\xi),\, \tilde v(\xi)+\tilde z(\xi)\big)\,{\rm d}\xi\,\,=\,\,
\int_{\mathbb{R}^n} {b}_0\big(\mathcal{F}_b^{-1}v(x)+\mathcal{F}_b^{-1}z(x),\, 
\mathcal{F}_b^{-1}\tilde v(x)+\mathcal{F}_b^{-1}\tilde z(x)\big)\,{\rm d} x,
\]
So it remains to show that $\mathcal{F}_bv=\mathcal{F}v$, $\forall v\in\mathbb{V}_0$. 
From \eqref{ik2}, by the Riesz theorem for Hilbert space $\mathcal{V}_0:=(V_0,b_0)$, there exists a linear map 
$T:V_0 \rightarrow V_0$ such that 
$d_0(v,\tilde v)=b_0(Tv,\tilde v)=b_0(v,T^*\tilde v)$, $\forall v,\tilde v\in V_0$, where $T^*$ is the 
adjoint of $T$ in $\mathcal{V}_0$. ($T$ is a bounded operator in $\mathcal{V}_0$, as follows e.g. from 
setting above $\tilde v= T v$ and using \eqref{ik2}). 
%
%Let now  $v,\tilde v\in \mathbb{V}_0=L^2\left(\mathbb{R}^n;\mathcal{V}_0\right)$. Then, using the boundedness of 
%$d_0$ and $T$ and 
%the isometry of $\mathcal{F}_b$, %and the above established identity $\mathcal{F}_b z= \mathcal{F} z$, $\forall z\in \mathbb{Z}\subset\mathbb{V}_0\subset\mathbb{H}_0$, 
%\begin{equation}
%\label{vs1}
%\int_{\mathbb{R}^n} d_0\big(v(x),\tilde v(x)\big)\,{\rm d}x\,\,=\,\,
%\int_{\mathbb{R}^n} b_0\big(T v(x),\tilde v(x)\big)\,{\rm d}x\,\,=\,\, 
%\int_{\mathbb{R}^n} b_0\Big(\mathcal{F}_b\left(T v\right)(\xi),\,\mathcal{F}_b\tilde v(\xi)\Big)\,{\rm d}\xi. 
%\end{equation}
%We next show that, for a.e. $\xi, \,$ 
%$\mathcal{F}_b\left(T v\right)(\xi)= T\left(\mathcal{F}_b v(\xi)\right)$. 
%Formula \eqref{ftdef} of Definition \ref{defb6} applied for $\mathcal{F}_b$, $u=T v$, and arbitrary $v'\in V_0$ 
%for $\tilde u$ states that, for a.e. $\xi$,  
%$b_0\big(\mathcal{F}_bT v(\xi), v'\big)=\mathcal{F}_c b_0\big(T v(\cdot), z'\big)(\xi)$ where 
%$\mathcal{F}_c$ is the conventional Fourier transform in $L^2\left(\mathbb{R}^n\right)$. 
%Then, with $T^*$ denoting the adjoint of $T$ in $\mathcal{V}_0$, and applying \eqref{ftdef} again, we obtain: 
%\begin{equation}
%\label{ttstar}
%b_0\big(\mathcal{F}_bT v(\xi), v'\big)\,\,=\,\,\mathcal{F}_c b_0\big( v(\cdot), T^*v'\big)(\xi)\,\,=\,\, 
%b_0\left(\mathcal{F}_b v(\xi), T^*v'\right)\,\,=\,\,b_0\big(T \mathcal{F}_b v(\xi), v'\big). 
%\end{equation}
%Since the above holds for arbitrary $v'\in V_0$, we conclude that 
%$\mathcal{F}_b\left(T v\right)(\xi)= T \left(\mathcal{F}_b v(\xi)\right)$, as desired. 
%Employing the latter in \eqref{vs1} and then using again the definition of $T$, we conclude 
%\[
%\int_{\mathbb{R}^n} d_0\big(v(x),\tilde v(x)\big)\,{\rm d}x\,=\,
%\int_{\mathbb{R}^n} b_0\Big(T\mathcal{F}_bv(\xi),\,\mathcal{F}_b\tilde v(\xi)\Big)\,{\rm d}\xi\,=\,
%\int_{\mathbb{R}^n} d_0\big(\mathcal{F}_bv(\xi),\mathcal{F}_b\tilde v(\xi)\big)\,{\rm d}\xi. 
%\]
%It remains to show that $\mathcal{F}_bv=\mathcal{F}v$, $\forall v\in\mathbb{V}_0$. %Now, 
%Arguing similarly to the above, 
Notice that formula \eqref{ftdef} of Definition \ref{defb6} states that, for a.e. $\xi$,  
$\big(\mathcal{F} u(\xi), \tilde u\big)=\mathcal{F}_c \big(u(\cdot), \tilde u\big)(\xi)$ where 
$\mathcal{F}_c$ is the conventional Fourier transform in $L^2\left(\mathbb{R}^n\right)$. 
So, for any 
$v\in\mathbb{V}_0\subset\mathbb{H}_0$ and $v'\in V_0\subset\mathcal{H}_0$, for a.e. $\xi$, 
%formula \eqref{ftdef} of Definition \ref{defb6} applied for $\mathcal{F}_b$, $u=v$, and arbitrary $v'\in V_0$ 
%for $\tilde u$ 
%states that, for a.e. $\xi$,  
%$b_0\big(\mathcal{F}_b v(\xi), v'\big)=\mathcal{F}_c b_0\big(v(\cdot), v'\big)(\xi)$ where 
%$\mathcal{F}_c$ is the conventional Fourier transform in $L^2\left(\mathbb{R}^n\right)$. 
\[
d_0\big(\mathcal{F}v(\xi),v'\big)=\mathcal{F}_c d_0\big(v(\cdot),v'\big)(\xi)= 
%\mathcal{F}_c b_0\big(Tv(\cdot),v'\big)(\xi)=
\mathcal{F}_c b_0\big(v(\cdot),T^*v'\big)(\xi)= 
b_0\big(\mathcal{F}_b v(\xi),T^*v'\big)=%b_0\big(T\mathcal{F}_b v(\xi),v'\big)= 
d_0\big(\mathcal{F}_bv(\xi),v'\big), \ \ \forall v'\in V_0. 
\]
As $V_0$ is dense in $\mathcal{H}_0$, the latter implies $\mathcal{F}_bv=\mathcal{F}v$ for a.e. $\xi$, as required. 
%
%Then, by the unitarity of $\mathcal{F}_b$ in $\mathbb{V}_0$, \eqref{plancher} holds with 
%$\mathcal{F}^{-1}$ replaced by $\mathcal{F}_b^{-1}$, cf. \eqref{planch}. 
%Hence, to show \eqref{plancher}, it would suffice to prove that 
%$\forall v\in \mathbb{V}_0$, $\mathcal{F}_bv=\mathcal{F}v$. 
%
%Let $\psi^{(k)}\in V_0\subset \mathcal{H}_0$, $k\ge 1$, be a common orthogonal basis in 
%$\mathcal{V}_0:=(V_0, b_0)$ and 
%$\mathcal{H}_0=\left(\overline{V_0},\, d_0\right)$ of the eigenfunctions of $\mathbb{L}_0$ with associated eigenvalues 
%$\lambda_0^{(k)}\ge 1$, see proof of Proposition \ref{propb1}. 
%Let $\psi^{(k)}$ be orthonormal in $\mathcal{H}_0$, and so $\psi_b^{(k)}:=\left(\lambda_0^{(k)}\right)^{-1/2}\psi^{(k)}$ is 
%orthonormal in $\mathcal{V}_0$. 
%For $v\in \mathbb{V}_0\subset\mathbb{H}_0$ by decomposing for a.e. $x$  along the basis in $\mathcal{V}_0$, 
%$v(x)=\sum_{k=1}^\infty c_k(x)\psi_b^{(k)}$ where $c_k\in L^2\left(\mathbb{R}^n\right)$ and 
%$\|v\|^2_{\mathbb{V}_0}=\sum_{k=1}^\infty\|c_k\|^2_{L^2(\mathbb{R}^n)}<\infty$. 
%Decomposing $v$ similarly along the basis in $\mathcal{H}_0$, 
%$v(x)=\sum_{k=1}^\infty c'_k(x)\psi^{(k)}=\sum_{k=1}^\infty c'_k(x)\left(\lambda_0^{(k)}\right)^{1/2}\psi_b^{(k)}$ for 
%a.e. $x$, where $c_k'\in L^2\left(\mathbb{R}^n\right)$ and 
%$\|v\|^2_{\mathbb{H}_0}=\sum_{k=1}^\infty\|c_k'\|^2_{L^2(\mathbb{R}^n)}<\infty$. 
%Now, as $b_0\left(\psi^{(k)},\tilde v\right)=\lambda_0^{(k)} d_0\left(\psi^{(k)},\tilde v\right)$, 
%$\forall \tilde v\in V_0$ 
%one has  for a.e. $x$,  
%\[
%c_k(x)=b_0\left(v(x),\psi_b^{(k)}\right)=\left(\lambda_0^{(k)}\right)^{-1/2} b_0\left(v(x),\psi^{(k)}\right)=
%\left(\lambda_0^{(k)}\right)^{1/2}d_0\left(v(x),\psi_0^{(k)}\right)=
%\left(\lambda_0^{(k)}\right)^{1/2}c'_k(x),   
%\]
%implying $c'_k(x)=\left(\lambda_0^{(k)}\right)^{-1/2}c_k(x)$. 
%Then, as follows from Definition \ref{defb6}, for a.e. $\xi$, 
%\[
%\mathcal{F}_b v(\xi)\,\,=\,\,\sum_{k=1}^\infty \widehat{c_k}(\xi)\psi_b^{(k)}\,\,=\,\, 
%\sum_{k=1}^\infty \widehat{c_k}(\xi)\left(\lambda_0^{(k)}\right)^{-1/2}\psi^{(k)}\,\,=\,\, 
%\sum_{k=1}^\infty \widehat{c_k'}(\xi)\psi^{(k)},  
%\]
%$\mathcal{F}_b v(\xi)=\sum_{k=1}^\infty \widehat{c_k}(\xi)\psi_b^{(k)}$ 
%for a.e. $\xi$ 
%where 
%$\widehat{c_k}$ and $\widehat{c'_k}$ are the conventional Fourier transform of $c_k$ and $c_k'$ respectively. 
%Decomposing now $v$ similarly along the basis in $\mathcal{H}_0$, 
%$v(x)=\sum_{k=1}^\infty c'_k(x)\psi^{(k)}=\sum_{k=1}^\infty c'_k(x)\left(\lambda_0^{(k)}\right)^{1/2}\psi_b^{(k)}$ for a.e. $x$, with $c_k'\in L^2\left(\mathbb{R}^n\right)$ and 
%$\|v\|^2_{\mathbb{H}_0}=\sum_{k=1}^\infty\|c_k'\|^2_{L^2(\mathbb{R}^n)}<\infty$, and 
%$\mathcal{F} v(\xi)=\sum_{k=1}^\infty \widehat{c_k'}(\xi)\psi^{(k)}=
%\sum_{k=1}^\infty \widehat{c'_k}(\xi)\left(\lambda_0^{(k)}\right)^{1/2}\psi_b^{(k)}$ 
%for a.e. $\xi$. 
%Next we observe that the above two decompositions of $v(x)$ coincide. Indeed, 
%By \eqref{ik2} the former series (converging for a.e. $\xi$ 
%to $\mathcal{F}_b v(\xi)$ with respect to the $b_0$-form) will also converge with respect to the $d_0$ form.
%The latter limit sum defines $\mathcal{F} v(\xi)$, and hence 
% and hence by the uniqueness of 
%the latter decomposition for a.e. $x$, $c'_k(x)=c_k(x)\left(\lambda_0^{(k)}\right)^{-1/2}$ for all $k$. 
%Hence $\widehat{c'_k}(\xi)=\widehat{c_k}(\xi)\left(\lambda_0^{(k)}\right)^{-1/2}$ for a.e. $\xi$, and as a result 
%$\mathcal{F}_b v=\mathcal{F} v$ as required. 
\vspace{.06in}

2. To prove \eqref{plancher} for $\mathfrak{b}=a^h_{jk}$ for any fixed $1\le j,k\le n$, we notice first that the form 
$a^h_{jk}$ is bounded in terms of $b_0$. Namely,  with some constant $C>0$, 
\begin{equation} 
\label{aijbdd} 
\left\vert a^h_{jk}(z,\tilde z)\right\vert \,\,\le\,\, C\, b_0^{1/2}[z]\,b_0^{1/2}[\tilde z], \ \ \ 
\forall\, z,\tilde z\in Z. 
\end{equation}
(This follows e.g. from \eqref{ahij}, \eqref{H4} and \eqref{Nbdd}.) 
Therefore, by the Riesz theorem for $\mathcal{Z}:=(Z,b_0)$, there exists a 
bounded linear map $T_{jk}:\mathcal{Z} \rightarrow \mathcal{Z}$ such that 
$a^h_{jk}(z,\tilde z)=b_0(T_{jk}z,\tilde z)$, $\forall z,\tilde z\in Z$. 
%Operator $T_{jk}$ is a bounded operator in $\mathcal{Z}$ (as follows e.g. from setting above 
%$\tilde z= T_{jk} z$ and using \eqref{aijbdd}). 
%The rest of the proof is essentially identical to the above one for $\mathfrak{b}=b_0$, together with 
%noticing 
Let now  $z,\tilde z\in \mathbb{Z}:=L^2\left(\mathbb{R}^n;\mathcal{Z}\right)$. Then, using the boundedness of 
$a^h_{jk}$ and $T_{jk}$, 
the isometry of $\mathcal{F}_b$, and the above established identity $\mathcal{F}_b z= \mathcal{F} z$, $\forall z\in \mathbb{Z}\subset\mathbb{V}_0\subset\mathbb{H}_0$, 
\begin{equation}
\label{vs1}
\int_{\mathbb{R}^n} a^h_{jk}\big(z(x),\tilde z(x)\big)\,{\rm d}x\,\,=\,\,
\int_{\mathbb{R}^n} b_0\big(T_{jk}z(x),\tilde z(x)\big)\,{\rm d}x\,\,=\,\, 
\int_{\mathbb{R}^n} b_0\Big(\mathcal{F}\left(T_{jk}z\right)(\xi),\,\mathcal{F}z(\xi)\Big)\,{\rm d}\xi. 
\end{equation}
Notice 
that $\mathcal{F} z(\xi)= \mathcal{F}_b z(\xi)\in Z$ for a.e. $\xi$ (as immediately follows %e.g. 
from 
\eqref{ftdef}). % for $\mathcal{F}_b$ with $\tilde u$ varying along the $b_0$-orthogonal complement of $Z$ in $V_0$.)  
We next show that, for a.e. $\xi$, 
$\mathcal{F}\left(T_{jk}z\right)(\xi)= T_{jk}\left(\mathcal{F} z(\xi)\right)$. 
Formula \eqref{ftdef} of Definition \ref{defb6} applied for $\mathcal{F}_b$, $u=T_{jk}z$, and arbitrary $z'\in Z$ 
for $\tilde u$ implies %states that, for a.e. $\xi$,  
$b_0\big(\mathcal{F}_bT_{jk} z(\xi), z'\big)=\mathcal{F}_c b_0\big(T_{jk} z(\cdot), z'\big)(\xi)$.  % where 
%$\mathcal{F}_c$ is the conventional Fourier transform in $L^2\left(\mathbb{R}^n\right)$. 
Then, with $T^*_{jk}$ denoting the adjoint of $T_{jk}$ in $\mathcal{Z}$, and applying \eqref{ftdef} again, we obtain: 
\begin{equation}
\label{ttstar}
b_0\big(\mathcal{F}T_{jk} z(\xi), z'\big)\,\,=\,\,\mathcal{F}_c b_0\big( z(\cdot), T_{jk}^*z'\big)(\xi)\,\,=\,\, 
b_0\left(\mathcal{F} z(\xi), T_{jk}^*z'\right)\,\,=\,\,b_0\big(T_{jk}\mathcal{F} z(\xi), z'\big). 
\end{equation}
Since the above holds for arbitrary $z'\in Z$, we conclude that 
$\mathcal{F}\left(T_{jk} z\right)(\xi)= T_{jk}\left(\mathcal{F} z(\xi)\right)$, as desired. 
Employing the latter in \eqref{vs1} and then using again the definition of $T_{jk}$, we conclude 
\[
\int_{\mathbb{R}^n} a^h_{jk}\big(z(x),\tilde z(x)\big)\,{\rm d}x\,=\,
\int_{\mathbb{R}^n} b_0\Big(T_{jk}\mathcal{F}z(\xi),\,\mathcal{F}\tilde z(\xi)\Big)\,{\rm d}\xi\,=\,
\int_{\mathbb{R}^n} a^h_{jk}\big(\mathcal{F}z(\xi),\mathcal{F}\tilde z(\xi)\big)\,{\rm d}\xi, 
\]
as required. The proof is complete. 
\end{proof} 

\begin{proposition}
\label{propb9}
Let $\mathcal{P}_{\mathcal{H}_0}^0:(\mathcal{H},d_0) \rightarrow \mathcal{H}_0$ and  
$\mathcal{P}:L^2\big(\RR^n ; (\mathcal{H},d_0)\big) \rightarrow L^2(\RR^n ; \mathcal{H}_0)$ be the orthogonal projections 
on the respective subspaces. Then
\begin{equation}
\label{b9statm}
\mathcal{P}_{\mathcal{H}_0}^0f(\xi) = \big(\mathcal{F}  \mathcal{P}\mathcal{F}^{-1} f \big)(\xi) \quad for\,\, a.e.\  \xi, \quad f \in L^2(\RR^n; (\mathcal{H},d_0)).
\end{equation} 
\end{proposition}
\begin{proof}
Given $f\in \mathbb{H}:=L^2\big(\RR^n ; (\mathcal{H},d_0)\big)$ let $g=\mathcal{F}^{-1}f\in 
\mathbb{H}$ and notice first that, for a.e. $x$, $(\mathcal{P}g)(x)=\mathcal{P}_{\mathcal{H}_0}^0g(x)$. 
Indeed, $h=\mathcal{P}g\in L^2(\RR^n ; \mathcal{H}_0)=:\mathbb{H}_0$ is such that, for any $\tilde h\in \mathbb{H}_0$, 
$
\int_{\mathbb{R}^n} d_0\big(g(x)-h(x),\,\tilde h(x)\big){\rm d}x=0,  
$
and the latter obviously holds for $h(x)=\mathcal{P}_{\mathcal{H}_0}^0g(x)$ for a.e. $x$. 
So \eqref{b9statm} is equivalent to 
$\mathcal{P}_{\mathcal{H}_0}^0\mathcal{F}g(\xi) = \big(\mathcal{F}  \mathcal{P}_{\mathcal{H}_0}^0 g(\cdot) \big)(\xi)$, 
for a.e. $\xi$. The latter can be proved by the argument identical to \eqref{ttstar}, 
%the proof of 
%$\mathcal{F}\left(T_{jk} z\right)(\xi)= T_{jk}\left(\mathcal{F} z(\xi)\right)$ in Lemma \ref{lemft}, 
with $T_{jk}$ replaced by (self-adjoint) $\mathcal{P}_{\mathcal{H}_0}^0$, $b_0$ by $d_0$ and $z'$ by $g'\in\mathcal{H}$. 
\end{proof}

\begin{thebibliography}{9}

\bibitem{Ar90} Arbogast, T., Douglas J. Jr. and Hornung, U., 1990. {\it Derivation of the double porosity model of single phase flow via homogenization theory.} 
SIAM J. Math. Anal. 21 (4), 823–836.

\bibitem{All}Allaire, G. {\it Homogenization and two-scale convergence}, SIAM J. Math. Anal. {\bf 23} (1992),
1482-1518.

\bibitem{AurBonn85} Auriault, J.-L., Bonnet, G., 1985. {\it Dynamique des composites elastiques periodiques.} 
Arch Mech [Archiwum Mechaniki Stosowanej] 37, 269--284.

\bibitem{Av08} Avila, A., Griso, G., Miara, B. and Rohan, E., 2008. {\it Multiscale modeling of elastic waves: theoretical justification and numerical simulation of band gaps}. Multiscale Model. Simul. 7 (1), 1–21

\bibitem{BaKaSm} Babych, N.O.,  Kamotski, I.V. and  Smyshlyaev, V.P., 2008. {\it  Homogenization of spectral problems in bounded domains with doubly high contrasts. Networks \& Heterogeneous Media}, 3 (3) : 413-436. 

\bibitem{BaPa} Bakhvalov, N. S. and Panasenko, G. P.  \emph{ Homogenization of processes in periodic media. Mathematical problems in the mechanics of composite materials.} Nauka, Moscow 1984; English transl., Kluwer, Dordrecht 1989.


%\bibitem{BeLiPa}{Bensoussan, A., Lions, J.-L., and Papanicolaou, G. C., 1978. {\it Asymptotic Analysis for Periodic Structures}, North-Holland}
\bibitem{Be}{Beer, G. Topologies on Closed and Closed Convex Sets, Kluwer Academic Publishers,Dordrecht, 1993.}

\bibitem{Bell} Bellieud M. {\it Torsion Effects in Elastic Composites with High Contrast.}SIAM J. Math. Anal.2010;41:2514–2553.

\bibitem{BiSu}Birman, M. Sh. and Suslina, T. A., 2004. {\it Second order periodic differential operators. Threshold
	properties and homogenisation.} {St. Petersburg. Math. J.} 15(5), pp. 639-714.

\bibitem{BSh1} Birman, M. Sh. and Suslina, T. A. 2006. {\it Averaging of periodic differential operators taking a corrector into account. Approximation of solutions in the Sobolev class $H^2(\mathbb{R}^d)$.} (Russian) Algebra i Analiz 18, no. 6, 1--130; translation in St. Petersburg Math. J. 18, no. 6, 857–955.

\bibitem{BoFe} Bouchitt\'{e}, G. and Felbacq, D. {\it Homogenization near resonances and	artificial magnetism from dielectrics.} C. R. Math. Acad. Sci. Paris 339{\bf 5}, (2004), 377–382.

\bibitem{Br} Briane M.,  2003. {\it Homogenization of high-conductivity periodic problems: application to a general distribution of one-directional fibers.}   SIAM J. Math. Anal. Vol. 35, no. 1, pp. 33-60.
%\bibitem{Br02} Briane, M., 2002. {\it Homogenization of non-uniformly bounded operators:critical barrier for nonlocal effects.} Arch. Ration. Mech. Anal. 164 (1),73–101.

\bibitem{CaEd05} Camar-Eddine, M. and Milton, G.W., 2005. {\it Non-local interactions in the homogenization closure of thermoelectric functionals.} Asymptotic Anal. 41 (3–4), 259–276.

\bibitem{cherd} Cherdantsev, M. I. {\it Spectral convergence for high contrast elliptic	periodic problems with a defect via homogenization.} Mathematika, Volume 55, Issue 1-2, (2009), pp. 29-57

\bibitem{ChChCo}{ Cherdantsev, M., Cherednichenko, K. and Cooper, S., 2017. {\it Extreme localisation of eigenfunctions to one-dimensional high-contrast periodic problems with a defect.} SIAM Journal on Mathematical Analysis, Vol. 50, No. 6, 5825-5856.
}	

\bibitem{ChCo}{Cherednichenko, K. D. and Cooper S., 2016. Resolvent estimates for high-contrast elliptic problems with periodic coefficients. {\it Archive for Rational Mechanics and Analysis}, 219(3), pp. 1061-1086.}

%\bibitem{ChCo2}{Cherednichenko, K. D., Cooper, S. 2016. Asymptotic behaviour of the spectra of systems of Maxwell equations in periodic composite media with high contrast. {\it  Available at https://arxiv.org/abs/1601.01305}}

%\bibitem{ChCo3}{Cherednichenko, K. D., Cooper, S., Guenneau, S. 2014.Spectral Analysis of One-Dimensional High-Contrast Elliptic Problems with Periodic Coefficients. {\it Multiscale Modeling \& Simulation: A SIAM Interdisciplinary journal}, 13(1), pp.72–98.}


\bibitem{ChKi}{Cherednichenko, K. D. and Kiselev, A., 2017. Norm-resolvent convergence of one-dimensional high-contrast periodic problems to a Kronig-Penney dipole-type model. {\it Communications in Mathematical Physics} 349, pp. 441-480.}


\bibitem{ChErKi}{Cherednichenko, K. D., Ershova, Yu. Yu., and Kiselev, A. V., 2020. Effective behaviour of critical-contrast PDEs: micro-resonances, frequency conversion, and time dispersive properties. I {\it Communications in Mathematical Physics} 375, pp. 1833–1884.}

\bibitem{ChKiVeZu23}{Cherednichenko, K., Kiselev, A., Vel\v{c}i\'{c}, I., and \v{Z}ubrini\'{c}, J., 2023. Effective behaviour of critical-contrast PDEs: micro-resonances, frequency conversion, and time dispersive properties. II {\it Arxiv preprint } https://arxiv.org/pdf/2307.01125.pdf }
%{\it Communications in Mathematical Physics} 375, pp. 1833–1884.}

\bibitem{CDG} Cioranescu, D., Damlamian, A. and Griso, G., 
{\it 
The periodic unfolding method in homogenization.}
SIAM J. Math. Anal., {\bf 40}, No. 6,  pp. 1585-1620.

\bibitem{CoVa97}Conca, C. and  Vanninathan, M. 1997 {\it 
Homogenization of Periodic Structures via Bloch Decomposition.}
SIAM Journal on Applied Mathematics, Vol. 57, No. 6,  pp. 1639-1659.

\bibitem{Cothesis} Cooper, S. {\it Two-scale homogenisation of partially degenerating PDEs with applications to photonic crystals
	and elasticity.} PhD Thesis, (2012), University of Bath. \newline 
	http://salc.azurewebsites.net/wp-content/uploads/2017/08/PhDThesissalc.pdf
	% {https://purehost.bath.ac.uk/ws/portalfiles/portal/187959425/UnivBath.PhD 2012 SAL-Cooper.pdf} 

\bibitem{Co}{ Cooper, S., 2013. Homogenisation and spectral convergence of a periodic elastic composite with weakly compressible inclusions. {\it Applicable Analysis},  93(7), pp.1401-1430. 
}

\bibitem{CoKaSmPCF}{ Cooper, S., Kamtoski, I. V. and Smyshlyaev, V. P. (2014). \emph{On band gaps in photonic crystal fibers}. 
\emph{Preprint:}	https://arxiv.org/pdf/1411.0238.pdf
}


\bibitem{FeKh80}{Fenchenko, V.N. and Khruslov, E.Ya., 1980.  {\it Asymptotic behaviour for the solutions of differential equations with strongly oscillating and degenerating coefficient matrix.} Dokl. Akad. Nauk Ukrain. SSR Ser. A4, 26–30.}



\bibitem{Gel} Gelfand, I. M. {\it Expansion in characteristic functions of an equation with periodic coefficients.} (Russian) Doklady Akad. Nauk SSSR (N.S.) {\bf 73}, (1950). 1117-1120.


\bibitem{GoNa92} Golovatyj, Yu, D, Nazarov, S.A. and O. A. Oleinik, 1992. {\it  Asymptotic expansions of eigenvalues and eigenfunctions in problems on oscillations of a medium with concentrated perturbations.} 
%Trudy Mat. Inst. Steklov., 192, Nauka, Moscow, 1990, 42–60; 
Proc. Steklov Inst. Math., 192, 43–63.

\bibitem{HeLi} Hempel, R. and Lienau, K. {\it 
Spectral properties of periodic media in the large coupling limit
Properties of periodic media.} Communications in Partial Differential Equations, Volume 25, 1999, Issue 7-8.

\bibitem{Higgins} Higgins,  J.R. {\it Five short stories about the cardinal series.} Bull. AMS {\bf 12}, (1985). 45-89.

\bibitem{Hytonen}{ Hyt\"{o}nen, T.,  Van Neerven, J., Veraar, M. and Weis, L., 2016. {\it Analysis in Banach spaces I,} Springer.}

\bibitem{JKO}{ Jikov, V. V.,  Kozlov, S. M. and Olejnik, O. A., 1994. {\it Homogenization of differential operators and integral functionals,} Springer-Verlag, Berlin.}

\bibitem{IVKVPS18} Kamotski, I. and Smyshlyaev, V. P. 2018. {\it Localized modes due to defects in high contrast periodic media via two-scale homogenization.} 
JJournal of Mathematical Sciences, 232 (3), 349-337. 

\bibitem{IVKVPS19} Kamotski, I. and Smyshlyaev, V. P. 2019. {\it Bandgaps in two-dimensional high-contrast periodic elastic beam lattice materials.} Journal of the Mechanics and Physics of Solids, Vol 123, 292-304. 

%\bibitem{KaSm1} Kamotski, I. V.,  Smyshlyaev, V. P. 2018.{\it Localised modes due to defects in	high contrast periodic media via homogenization.} Journal of Mathematical Sciences volume 232, 349–377.
\bibitem{IVKVPS13} Kamotski, I. V. and Smyshlyaev, V. P. 2019. {\it Two-scale homogenization for a general class of high contrast PDE systems with periodic coefficients.} Applicable Analysis, An International Journal, Volume 98, Issue 1-2.

\bibitem{Ku}{Kuchment, P. {\it An overview of periodic elliptic operators.}  Bull. Amer. Math. Soc. {\bf 53} (2016), 343-414.}

\bibitem{Lady}Ladyzhenskaya, O. A., 1969. {\it The Mathematical Theory of Viscous Flows, second edition}, Gordon and Breach, New York.

\bibitem{Na93}  Nazarov, S.A, 1993. {\it Interaction of concentrated masses in a harmonically oscillating spatial body with Neumann boundary conditions.} ESAIM: M2AN, Volume 27, Number 6,  777-799.

\bibitem{Ng}Nguetseng, G. 1989. {\it A general convergence result for a functional related to the theory of
	homogenization}, SIAM J. Math. Anal. {\bf 20}, 608-623.

\bibitem{Pa91} Panasenko, G.P., 1991. {\it Multicomponent homogenization of processes in strongly nonhomogeneous structures.} Math. USSR Sbornik 69 (1),143–153.

\bibitem{Pas05} Pastukhova, S.E., 2005. {\it On the convergence of hyperbolic semigroups in variable Hilbert spaces.} J Math Sci (N Y) 127, 2263–2283.

\bibitem{ZhiPa}{Pastukhova, S. E. and Zhikov, V. V.,} 2013. On gaps in the spectrum of the operator of elasticity theory on a high contrast periodic structure. {\it Journal of Mathematical Sciences}, 188(3), pp. 227-240.
%\bibitem{ZhPaBloch}{Pastukhova S. E., Zhikov, V. V.,} 2016. Bloch principle for elliptic differential operators with periodic coefficients. {\it Russian Journal of Mathematical Physics,} 23(2), pp. 257-277.

\bibitem{ReeSim1} Reed, M. and Simon, B., 1980. {\it Methods of Modern Mathematical Physics I: Analysis of Operators}, Academic Press.

\bibitem{ReeSim} Reed, M. and Simon, B., 1978. {\it Methods of Modern Mathematical Physics IV: Analysis of Operators}, Academic Press.

\bibitem{Sa99} Sandrakov, G.V., 1999. {\it Homogenization of elasticity equations with contrasting coefficients.} Sbornik Math. 190 (12), 1749–1806.

\bibitem{VPS}{Smyshlyaev, V. P., 2009. Propagation and localization of elastic waves in highly anisotropic periodic composites via two-scale homogenization. {\it Mechanics of Materials}, 41 (4 (Sp. Iss. SI)), pp. 434-447.
}

\bibitem{VishLus}{Vishik, M. I. and A. A. Lyusternik, A. A.} 1957. Regular degeneration and boundary layer for linear
differential equations with small parameter. {\it Usp. Mat. Nauk}, 12(5), pp. 3--122.

\bibitem{Zhi2000}{ Zhikov, V. V., 2000. On an extension of the method of two-scale convergence and its applications. {\it Sb.
Math.}, 191(7), pp. 973-1014.}

\bibitem{Zhi2005}{ Zhikov, V. V., 2005. On gaps in the spectrum of some divergence elliptic operators with periodic coefficients. {\it St. Petersburg Math. J.}, 16(5), pp.773-719.}

%\bibitem{ZhPaEstimates}{Zhikov, V. V., and Pastukhova, S. E.} (2016) \emph{Operator estimates in homogenization theory}, {Russian Mathematical Surveys}, {71}(3),  page {417} onwards.

\bibitem{Zh89}{Zhikov, V.V. 1989. {\it Spectral approach to asymptotic diffusion problems}, Differ. Uravn., 25:1, 44–50; Differ. Equ., 25:1, 33–39}
\bibitem{ZhSpectr} Zhikov, V.V. 2005. {\it Spectral method in homogenization theory} [in Russian], Tr. Mat. Inst. Steklova 250, 95–104; English transl.: Proc. Steklov Inst. Math. 250, 85–94.
\bibitem{ZhL2} Zhikov, V. V. 2005. {\it On operator estimates in homogenization theory.} (Russian) Dokl. Akad. Nauk 403, no. 3, 305–308.
\bibitem{ZhH1}{Zhikov, V.V. 2006. {\it Some estimates from homogenization theory}, Dokl. Math., 73:1, 96-99.}
\bibitem{ZhPasH1}{Zhikov, V.V. and Pastukhova, S.E. 2005. {\it On operator estimates for some problems in homogenization theory,} Russ. J. Math. Phys., 12:4, 515-524.}

\bibitem{ZhP07}{Zhikov, V.V. and Pastukhova, S.E. 2007. {\it On the Trotter-Kato theorem in a variable space.}   Funct Anal Appl., 41, 264–270.}

\end{thebibliography}
\end{document}
