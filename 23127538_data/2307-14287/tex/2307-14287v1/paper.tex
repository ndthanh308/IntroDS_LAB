\documentclass[conference]{IEEEtran}
\IEEEoverridecommandlockouts
\usepackage{cite}
\usepackage{amsmath,amssymb,amsfonts}
\usepackage{algorithmic}
\usepackage{booktabs}
\usepackage{graphicx}
\usepackage{textcomp}
\usepackage{svg}
\usepackage{xcolor}
\usepackage{colortbl}
\usepackage{multirow}
\usepackage{array}
\usepackage{threeparttable}
\usepackage{url}
\usepackage{float}
\usepackage{balance}

\newcolumntype{C}{>{\centering}p{0.045\textwidth}}
\newcolumntype{X}{>{\centering\arraybackslash}m{0.045\textwidth}}

\def\BibTeX{{\rm B\kern-.05em{\sc i\kern-.025em b}\kern-.08em
    T\kern-.1667em\lower.7ex\hbox{E}\kern-.125emX}}
    
\usepackage[hyperfootnotes=false]{hyperref}    

\newcommand{\algorithmautorefname}{Algorithm}
\def\sectionautorefname{Section}
\def\subsectionautorefname{Subsection}
\def\subsubsectionautorefname{Subsubsection}

\begin{document}

\title{Evaluation of Data Enrichment Methods for Distributed Stream Processing Systems}

\author{
\IEEEauthorblockN{Dominik Scheinert,
Fabian Casares,
Morgan K. Geldenhuys,
Kevin Styp-Rekowski,
and Odej Kao}
\IEEEauthorblockA{Technische Universit{\"a}t Berlin, Germany, \{firstname.lastname\}@tu-berlin.de}
}

\maketitle


\begin{abstract}
Stream processing has become a critical component in the architecture of modern applications. 
With the exponential growth of data generation from sources such as the Internet of Things, business intelligence, and telecommunications, real-time processing of unbounded data streams has become a necessity. 
DSP systems provide a solution to this challenge, offering high horizontal scalability, fault-tolerant execution, and the ability to process data streams from multiple sources in a single DSP job.
Often enough though, data streams need to be enriched with extra information for correct processing, which introduces additional dependencies and potential bottlenecks.

In this paper, we present an in-depth evaluation of data enrichment methods for DSP systems and identify the different use cases for stream processing in modern systems. 
Using a representative DSP system and conducting the evaluation in a realistic cloud environment, we found that outsourcing enrichment data to the DSP system can improve performance for specific use cases. 
However, this increased resource consumption highlights the need for stream processing solutions specifically designed for the performance-intensive workloads of cloud-based applications.

\end{abstract}

\begin{IEEEkeywords}
Distributed Stream Processing, Data Enrichment, Data Analysis, Resource Management, Cloud Computing
\end{IEEEkeywords}


\section{Introduction}
Deep learning models have been widely used in many applications.
For example, BERT~\citep{devlin_bert_2019}, GPT-3~\citep{brown_language_2020}, and T5~\citep{raffel_exploring_2020} achieved state-of-the-art~(SOTA) results on different natural language processing~(NLP) tasks. 
For computer vision~(CV), Transformer-like models such as ViT~\citep{dosovitskiy_image_2021} and Swin Transformer~\citep{liu_swin_2021} deliver excellent accuracy performance upon multiple tasks. 


At the same time, training deep learning models has been a critical problem troubling the community due to the long training time, especially for those large models with billions of parameters~\citep{brown_language_2020}. 
In order to enhance the training efficiency, researchers propose some manually designed parallel training strategies~\citep{narayanan_efficient_2021,shazeer_mesh-tensorflow_2018,xu_gspmd_2021}. 
However, selecting, tuning, and combining these strategies require extensive domain knowledge in deep learning models and hardware environments. With the increasing diversity of modern hardware architectures~\cite{flynn_very_1966,flynn_computer_1972} and the rapid development of deep learning models, these manually designed approaches are bringing heavier burdens to developers. 
Hence, \emph{automatic parallelism} is introduced to automate the parallel strategy searching for training models.


There are two main categories of parallelism in deep learning models: inter-layer parallelism~\citep{huang_gpipe_2019,narayanan_pipedream_2019,narayanan_memory-efficient_2021,fan_dapple_2021,li_chimera_2021,lepikhin_gshard_2021,du_glam_2022,fedus_switch_2022} and intra-layer parallelism~\citep{li_pytorch_2020,narayanan_efficient_2021,rasley_deepspeed_2020,fairscale_authors_fairscale_2021}. 
Inter-layer parallelism partitions the model into disjoint sets on different devices without slicing tensors. 
Alternatively, intra-layer parallelism partitions tensors in a layer along one or more axes and distributes them across different devices.


Current automatic parallelism techniques focus on optimizing strategies within these two categories. However, they treat these two categories separately. 
Some methods~\citep{zhao_vpipe_2022,jia_exploring_2018,cai_tensoropt_2022,wang_supporting_2019,jia_beyond_2019,schaarschmidt_automap_2021,liu_colossal-auto_2023} overlook potential opportunities for inter- or intra-layer parallelism, the others optimize inter- and intra-layer parallelism hierarchically and sequentially~\citep{narayanan_pipedream_2019,fan_dapple_2021,he_pipetransformer_2021,tarnawski_efficient_2020,tarnawski_piper_2021,zheng_alpa_2022}. 
As a result, current automatic parallelism techniques often fail to achieve the global optima and instead become trapped in local optima. 
Therefore, a unified inter- and intra-layer approach is needed to enhance the effectiveness of automatic parallelism.


This paper aims to find the optimal parallelism strategy while simultaneously considering inter- and intra-layer parallelism. 
It enables us to search in a more extensive strategy space where the globally optimal solution lurk. 
However, unifying inter- and intra-layer parallelism in automatic parallelism brings us two challenges. 
Firstly, to adopt a unified perspective on the inter- and intra-layer automatic parallelism, we should not formalize them with separate formulations as prior works. Therefore, how can we express these parallelism strategies in a unified formulation? 
Secondly, previous methods take a long time to obtain the solution with a limited strategy space. Therefore, how can we ensure that the best solution can be obtained in a reasonable time while expanding the strategy space?


To solve the above challenges, we propose UniAP. For the first challenge, UniAP adopts the mixed integer quadratic programming~(MIQP)~\citep{lazimy_mixed_1982} to search for the globally optimal parallel strategy automatically. 
It unifies the inter- and intra-layer automatic parallelism in a single MIQP formulation. 
For the second challenge, our complexity analysis and experimental results show that UniAP can obtain the globally optimal solution in a significantly shorter time.


The contributions of this paper are summarized as follows: 
\begin{itemize}
    \item We propose UniAP, the first framework to unify inter- and intra-layer automatic parallelism in model training.
    \item The optimal parallel strategies discovered by UniAP exhibit scalability on training throughput and strategy searching time.
    \item The experimental results show that UniAP speeds up model training on four Transformer-like models by up to 1.70$\times$ and reduces the strategy searching time by up to 16$\times$, compared with the SOTA method.
\end{itemize}

\section{Problem Analysis}
\label{sec:problem_analysis}
In this section, we first present our assumptions regarding data enrichment in DSP systems and then elaborate on general applicable use cases in this field.

\subsection{Assumptions}
The processing of unbounded data streams requires DSP systems to in theory execute indefinitely. 
As the near-to real-time processing of events is crucial for a wide range of applications, various requirements must be met by the DSP system depending on the application.
In this work, we therefore primarily focus on low-latency streaming jobs, but there are other relevant aspects that must be considered.

One such aspect is the reliability of a streaming job. 
System failures are common in large clusters, and for DSP jobs that are required to operate indefinitely, failures are inevitable, making it essential for the DSP system to guarantee exactly-once semantics and recovery from failures. 
Data enrichment methods should consequently take this into account and exhibit certain robustness.
Another important requirement is scalability, as the workload of a stream may change over time, leading to the need for adding new resources to avoid performance degradation or removing resources to optimize resource utilization.
Depending on the concrete design, this can also affect an employed data enrichment method.
Lastly, streaming applications and their underlying architectures can quickly lead to an increase in complexity, i.e., in light of the distributed execution graph, heterogeneous data sources, or data sinks.
This is further reinforced through support for libraries that enable ML or graph processing.

\subsection{Data Enrichment Use Cases}

Data processing in a DSP system often requires additional context for accurate analysis and interpretation. 
To achieve this, enrichment with additional data can be performed during the execution of the DSP job.
There are several reasons why data enrichment could be necessary. 
Firstly, the incoming data streams may lack the necessary information to provide accurate insights. 
For example, if a system is monitoring sensor data, it may be necessary to enrich the data with information about the location, time, or weather conditions to understand the context in which the data was collected.
In case of constrained network links as in IoT environments, this can furthermore reduce the size of the individual events and lower overall network overhead, as events remain compact in size until they reach the DSP system in the cloud, where they are eventually enriched.
Secondly, data enrichment can help to detect anomalies or patterns that may be hidden in the data. 
By adding more information to the data, it may be possible to identify patterns that were not apparent before, such as identifying fraud or predicting a failure before it occurs.
Thirdly, data enrichment can help to integrate data from multiple sources. 
In DSP systems, data may come from multiple sources, and integrating this data can be a complex process. 
By enriching the data with additional information, it may be possible to integrate data from different sources and provide a more complete picture of the target system.

Data enrichment during execution can vary greatly based on the use case, and the underlying architecture and priorities can be unique. 
Due to the diversity of use cases, there is no one-size-fits-all solution for enriching events in a DSP system. 
In order to carve out the advantages and disadvantages of particular solutions, we conduct a comprehensive evaluation of different enrichment methods.
For this endeavor, we derive the following broad use case categories, along the criteria of data availability, data volume, and time sensitivity:

\begin{itemize}
\item \underline{Simple Queries:} In many instances, a data enrichment use case includes common operations, for instance, a simple key-value database query, where the event contains a key and there is one value in the database that can be efficiently queried.
An example would be the location of a sensor in an IoT environment that could be queried by a key.
Other examples include API requests or inference services of ML applications.
We envision that for this category, all these examples have in common that the response time is fairly constant, yet loading the entirety of information is not possible, for example, due to capacity limitations (e.g. databases), service barriers (APIs of closed systems), or undeterministic data (e.g. time-sensitive information such as weather data).
\item \underline{Complex Queries:} This category resembles the first one, with the difference that queries or data formats are more complex, and hence response times can be fluctuating.
This is for instance the case for complex database queries that include multiple join statements (e.g. for fraud detection), or for API requests which trigger different behavior depending on the payload.
This increased complexity hinders the migration of information directly to the respective DSP system for accelerated data enrichment.
\item \underline{Finite Data Sources:} In certain scenarios, the information used for enriching streaming events might be compact in size and hence might qualify for a migration directly to the DSP system as embedded state. 
Examples range from small disclosed ML models, which generate a data output for each data input, to static sensor information.
While potentially beneficial for event latencies, additional challenges are raised with respect to state handling within the DSP system as well as resource management. 
\end{itemize}

The goal of our evaluation is to determine suitable enrichment methods for specific use cases originating from our defined use case categories, to allow for guidance, and to enable practitioners to make informed decisions.

\subsection{Enrichment Methods}

In the following, we discuss various methods of enriching events in state-of-the-art DSP systems. 
These methods serve as a baseline for investigating the previously identified categories of data enrichment use cases.\\
\textbf{Datasource Client.} This method connects to an external data source to access the data for single / batches of events, which can be performed either synchronously or asynchronously.
\begin{itemize}
    \item \underline{Synchronous}: A synchronous client is the simplest way to connect to an external data source. This method enriches each event with the result of a synchronous query to the data source. Although a blocking procedure, the advantage of this method is that it can be easily integrated into existing architectures, and most conventional databases provide a synchronous client. If a pattern recognition model is used for enrichment, this method can be applied if the model is executed in an external service.
    \item \underline{Asynchronous}: This method involves using an asynchronous client to connect to the external data source, allowing for parallel execution of queries and improved utilization of query times. This requires the availability of an asynchronous client library. If no such library is available, asynchronous queries can be simulated with a custom multi-threading implementation.
\end{itemize}
\textbf{Cache.} To reduce access to external and potentially slow data sources, a subset of the data can be cached for faster access and to reduce dependencies. The data format must be able to be cached, and the external data should not change frequently. For aggregation operations, caching can quickly become costly.
\begin{itemize}
    \item \underline{Local Caching}: This method caches a subset of the external data within the respective operation of the DSP system. In case of a cache miss, a query to the external data source is executed. This method reduces latency and the load on the external system, and can store non-serializable objects. The storage capacity of the local cache depends on the worker node's storage capacity.
    \item \underline{External In-Memory Database Cache}: This method caches a subset of the external data in an external in-memory database such as Redis. This creates an additional synchronous or asynchronous connection to the in-memory database, in addition to the connection to the disk-based data source. In case of a cache miss, an additional query to the disk-based data source must be executed. Although an entirely new system is additionally required, the advantage of this method is that the resources can be managed independently of the DSP system, allowing for caching of a larger amount of data. The external cache is also more transparent and modifiable, making it easier to keep it consistent with the disk-based database if necessary.
\end{itemize}
\textbf{Embedded State.} Caching methods maintain a connection to the external data source, leading to a direct dependency. To overcome this, this method involves loading the entire external data into the DSP system as a stream and treating it as another source. The events are then enriched by a join operation. This method requires that outsourcing external data is possible and that the amount of data is within the available resources of the DSP system. Modern systems such as Spark or Flink have an included state backend that can store large amounts of data using RocksDB. However, using a disk-based state in the DSP system can reduce performance. In-memory state is recommended for real-time processing, but requires a large amount of memory.
The embedded state can hence often be regarded as a special case of local caching.

These enrichment methods can be implemented in common DSP systems and serve as a foundation for our evaluation.
\section{Generalized Open-World Semi-Supervised Object Detection}
\label{sec:method}

% Figure environment removed

We propose an integrated framework for \textit{Generalized Open-World Semi-Supervised Object Detection} consisting of an Ensemble-Based OOD Explorer (Section \ref{sec:methodology:ood-explorer}) and an OOD aware semi-supervised pipeline (Section \ref{sec:methodology:ood-explorer}). The OOD Explorer plays a crucial role in identifying OOD objects, encompassing both localization and classification into ID or `'unknown' class. The OOD-aware semi-supervised learning pipeline ( Section \ref{sec:methodology:ssl-pipeline}) ensures that the model assimilates knowledge from both ID and OOD data without risking the forgetting of previously learned ID classes. This integrated approach aims to enhance the adaptability and robustness of object detection models in the open-world context. Figure \ref{fig-owssd-method} provides a visual summary of our proposed framework.

\subsection{Problem Formulation}\label{sec:OWSSD:Problem} 

For our task of generalized open-world semi-supervised object detection, we are given a small labeled dataset \textit{$D_l = \{(x_1,y_1),(x_2,y_2), \ldots, (x_n,y_n)\}$} and a large unlabeled dataset \textit{\textit{$D_u = \{u_1, u_2, \ldots, u_m\}$}}, where $\{x_i\}$ and $\{u_i\}$ are input images, and $\{y_i\}$ are annotations. \textit{$D_l$} consists of a set of ID categories denoted by \textit{$C_\mathrm{id}$} (\ie, \textit{\textit{$C_{l}$} = \textit{$C_\mathrm{id}$}}), and \textit{$D_u$} consists of both \textit{$C_\mathrm{id}$} and a set of OOD categories denoted by \textit{$C_\mathrm{ood}$} (\textit{\textit{i.e., $C_{u}$} = \textit{$C_\mathrm{id}$} $\cup$ \textit{$C_\mathrm{ood}$}}). Each annotation \textit{$y_i$} in \textit{$D_l$} includes a set of object bounding box coordinates and the corresponding class labels from \textit{$C_\mathrm{id}$}.

Our objective is to train a detection model on \textit{$D_l$} and \textit{$D_u$} jointly that is able to (1) localize and classify instances belonging to one of the ID categories \textit{$C_\mathrm{id}$}; and (2) identify instances belonging to \textit{$C_\mathrm{ood}$} and localizing them. Note that the model does not have access to any \textit{$C_\mathrm{ood}$} bounding box coordinates or labels. The evaluation is performed on a held-out dataset \textit{$D_v$} that consists of both objects from \textit{$C_\mathrm{id}$} and \textit{$C_\mathrm{ood}$} categories (\ie, \textit{\textit{$C_{v}$} = \textit{$C_\mathrm{id}$} $\cup$ \textit{$C_\mathrm{ood}$}}).

\subsection{Ensemble-Based OOD Explorer}
\label{sec:methodology:ood-explorer}

In the context of object detection, the integration of OOD data involves two primary tasks: OOD Classification, which distinguishes OOD data from ID data, and OOD Localization, which concurrently provides the precise location of OOD objects within the image. 

% Figure environment removed


\textbf{OOD Classification}. We propose an ensemble model that is trained using labeled ID data \textit{only} and without OOD samples. Our ensemble model comprises of multiple auto-encoder networks, each trained on samples from a specific ID class. Each model within the ensemble is dedicated to learning a lower-dimensional representation (encoding) for its respective category. 
An autoencoder network uses an encoder ($z=E_{\phi }(x), E_{\phi }:\mathcal {X}\rightarrow \mathcal {Z}$) to compress its input data and a decoder ($\hat{x}=D_{\theta }(z), D_{\theta }:\mathcal {Z}\rightarrow \mathcal {X}$)to reconstruct the original input. 

The autoencoder learns by minimizing the reconstruction error $R$, which gives a measure of how well the output was reconstructed compared to the input. $R$ is estimated with a dissimilarity function $d$, such that $d(x,\hat{x})$ measures how much $x$ differs from $\hat{x}$ ($R = d(x,D_{\theta }(E_{\phi }(x)))$)

When an auto-encoder model is trained only with a specific category of ID data, the reconstruction error, $R$, for samples of that category is low. At test time, when confronted with an OOD sample, the model is unable to accurately reconstruct the input, resulting in a higher reconstruction error. If this error value is less than a threshold value ($\mu$), then the sample belongs to the corresponding ID class which the model was trained on. This process is repeated for all auto-encoder models in the ensemble. Each model votes if the sample belongs to its class based on the observed $R$. A sample is deemed OOD if none of the models ``claim'' it. This simple heuristic illustrates how the decision boundary between ID and OOD samples is established (Figure \ref{fig:ensemble} shows this process).

This threshold is calibrated during training by observing the reconstruction error of ID categories. Since each model is responsible only for its own category, all other categories function as pseudo-OOD data for determining $\mu$. Finally, given that a single image may contain multiple objects of diverse categories, we leverage the box-level features of each object present in the image (Refer to the appendix for details on the training and test procedure).

As seen above, training an autoencoder network is inherently unsupervised as it does not require any information about the class label. By introducing an ensemble of such networks trained on individual categories, we introduce a specialization for each model. \emph{Our key insight is that this introduced specialization is a useful and necessary requirement for OOD detection with very limited labeled data.} When an autoencoder model is trained using the objects of a single category, the model learns the most salient and informative characteristics for that category. Training such an ensemble of models ensures that a model makes confident predictions when encountered with objects from the same ID category and results in a high reconstruction error when encountered with an object different from the one it was trained on. Section \ref{sec:expts} describes the variety of experiments we conduct which demonstrate that an ensemble model performs better compared to not only a common autoencoder model trained on all ID classes, but also a variety of state-of-the-art OOD detection algorithms. 

\textbf{OOD Localization}. In standard two-stage detectors like Faster R-CNN, object localization is achieved by using a Region Proposal Network (RPN) to generate object proposals, a selection of locations likely to contain objects. This network, however, is trained on a fixed set of ID classes and thus fails to generalize to novel OOD classes.

To address this challenge, we employ CutLER \citep{wang2023cut}, an unsupervised method that generates class-agnostic proposals. This characteristic proves advantageous in our open-world setting, as the proposals are not confined to in-distribution (ID) classes. We utilize the models pre-trained on unsupervised ImageNet, including both Cascade Mask RCNN and Mask RCNN. Subsequently, we filter the proposals based on a confidence score.

Additionally, we also analyzed two other localization methods for OOD localization, namely OLN \citep{kim2022learning} and MOST \citep{rambhatla2023most}. CutLER and MOST use the features extracted from a transformer \citep{vaswani2017attention} network trained with with a self-supervised manner proposed in DINO \citep{caron2021emerging} to localize multiple objects in an image. OLN \citep{kim2022learning}, on the other hand, estimates the objectness of a candidate region by relying on geometric cues such as location and shape of an object, regardless of its category. We show the performance of these methods when incorporated in our OOD Explorer and SSL pipelines in Section \ref{sec:expts:ablation}. 

\subsection{OOD-Aware Semi-Supervised Learning}
\label{sec:methodology:ssl-pipeline}

We now describe how the OOD samples identified by the OOD Explorer are introduced for training along with the ID samples, in our all {\em OOD-aware} semi-supervised learning framework. 

We adopt the Teacher-Student paradigm and use a two-stage training process (Fig.~\ref{fig-owssd-method}). In the first stage, we use the labeled data to train a \textbf{Teacher model}. This model then operates on weakly-augmented unlabeled images to generate bounding boxes and class predictions for the \textit{ID} classes. A subset of confident predictions that are higher than a pre-determined threshold are treated as pseudo-labels. 

In the open-world setting, such pseudo-labels could be highly noisy -- a novel OOD sample might be wrongly classified as an ID category. Therefore, we filter out the predictions made by the  Teacher model that have an Intersection over Union (IoU) score greater than a threshold $\tau$ with the OOD pseudo labels. 

In the second stage, a \textbf{Student model} is trained using both the labeled data and strongly augmented unlabeled data and the consistency regularization paradigm is used which enforces a model to output the same prediction for an unlabeled sample even after augmentation. The final loss is computed for the labeled samples and unlabeled samples (which includes both ID and OOD categories): $\mathcal{L} =~ \mathcal{L}_s + \lambda \mathcal{L}_{u} $

The supervised loss $\mathcal{L}_s$ is the standard loss for Faster RCNN network comprising of the the RPN classification loss (${L}_{cls}^{rpn}$), the RPN regression loss (${L}_{reg}^{rpn}$), the ROI classification loss (${L}_{cls}^{roi}$), and the ROI regression loss (${L}_{reg}^{roi}$). This loss is computed on the model prediction compared to the ground truth labels. In contrast, the unsupervised loss $\mathcal{L}_u$ is computed on the model prediction on a strongly augmented image compared to the pseudo-label, also consists of the four loss terms. $\lambda$ is a hyper-parameter that controls contribution of the unsupervised loss to the total loss. The details of data augmentation and hyper-parameter settings are described in Sec.~\ref{sec:expts}.

Note that since we are introducing a new OOD class to the model, instead of using the weights provided by the Teacher model, the Student model is retrained from scratch to alleviate the \textit{catastrophic forgetting} issue observed in incremental learning scenarios \citep{li2017learning}.

\section{Experiment Results}
\label{sec:experiment_results}

This section presents our experiments and associated results, and discusses our key findings.
To obtain a meaningful evaluation, each experiment was performed three times. 
The plots of the results always include the data from all three executions, whereas the mean of the three data points is highlighted. 

\subsection{Fraud Detection Use Case}

The fraud detection use case uses the default configuration of the \texttt{HashMapStateBackend} with a checkpointing interval of 10 minutes to avoid affecting performance as would be the case when using a performance-intensive state backend such as the \texttt{EmbeddedRocksDBStateBackend}. A window size of 10 seconds and a slide size of 5 seconds were chosen for the sliding window operation, balancing the detection of fraud and performance requirements. The chosen parameters ensure that no large state is required during execution, which could affect latency and distort results.

\subsubsection{Datasource Clients}

% Figure environment removed

The purpose of the first experiment was to compare the latency and throughput of synchronous and asynchronous Cassandra clients under different conditions. 
The experiment was conducted using throughput rates ranging from 1,000 to 2,200 events per second in increments of 100 events per second. 
The results, illustrated in~\autoref{fig:sync_async_latency}, showed that asynchronous enrichment had a fairly constant latency that was always lower than the latency of enrichment by synchronous queries. 
Synchronous enrichment showed a slight increase in latency from 1,600 events/s, and a significant increase from 1,900 events/s, reaching a latency of approximately 50 seconds at 2,200 events/s. 
This rapid increase in latency is due to the maximum throughput being reached, leading to the back pressure mechanism taking effect, delaying event processing. 
The comparison of the consumed rate of the two streaming jobs was measured using the Kafka metric \textit{records-consumed-rate} and is depicted in~\autoref{fig:sync_async_consumedrate}. 
The data indicated that synchronous enrichment consumes slightly fewer events per second than asynchronous enrichment. 
Synchronous enrichment had a maximum throughput of approximately 1,900 events/s and reached a load of 100\% after 50 minutes (~\autoref{fig:sync_busy_rates}). 
On the other hand, asynchronous enrichment had a lower load, and the busy values of the enrichment tasks only increased slightly and remained below 100ms per second (~\autoref{fig:async_busy_rates}). 
In the streaming job with synchronous enrichment, the enrichment task had the highest load and reached 100\% at 1,900 events/s.
In conclusion, the results showcase the limitations of synchronous enrichment, and while asynchronous enrichment comes out superior, it will as well face problems in light of substantially higher throughput rates due to its technical implications, underlining the need for caching strategies, as discussed next.

% Figure environment removed

\subsubsection{Caching Methods}

In the previous experiments, it has become clear that asynchronous enrichment allows both better latency and higher throughput than enrichment using synchronous database queries.
We consequently now evaluate enrichment methods that use a cache to store database records combined with asynchronous Cassandra queries. 
The evaluation was conducted under two factors - the generation of events and the size of caches. 
To maintain the uniformity of the evaluation, each transaction event was required to be unique and generated uniformly. 
Also, the number of cache entries had to be equal to the amount of entries of the external cache. 
A total maximum number of cache entries was defined, with the sum of all cache entries being equal to 24,000. 
A fixed throughput of 4,000 events per second was selected for the latency evaluation, with each execution running for 70 minutes. 
The results, depicted in~\autoref{fig:caches_latency}, indicate that caching using a preceding custom partitioner has the best latency and cache hit rate of 100\% (~\autoref{fig:caches_cachehit_rates}). 
With our particular experiment design, local caching without a preceding custom partitioner achieved a cache hit rate of only 50\% and lower latency than asynchronous enrichment without cache but worse latency than caching with a preceding custom partitioner.
The local cache size for each local cache was 3,000. 
In contrast, asynchronous enrichment without cache showed a more volatile latency, ranging approximately between 3.3s and 3.5s. 
Also, enrichment with an external Redis cache performed comparably bad with a volatile latency and, oftentimes, being slower than asynchronous enrichment without cached database entries. 
The reason for the highly fluctuating latency may be the overhead incurred by the additional asynchronous operator and associated network connections, along with the implementation of the asynchronous Redis client.
Although latency is poor compared to the other enrichment methods in this evaluation, an external cache has the advantage over local caches of not being affected when \textit{TaskManagers} fail and does not need to be refilled. 
In order to show how a streaming job with a local cache behaves in the case of a failure of all \textit{TaskManagers}, we conducted an experiment in which all \textit{TaskManager} processes were deleted every few minutes. 
We chose the streaming job with the enrichment method with local cache and preceding custom partitioner for the experiment. 
~\autoref{fig:cache_partition_failures_latency} shows how the latency behaves in the case of a failure of all \textit{TaskManagers} at once. 
It can be seen that in the beginning, after restarting the streaming job at minutes 4 and 10, the volatility is comparatively high and then decreases after a short time and the latency becomes constant again. 
This is due to the fact that the cache must first be filled again by database accesses. 
~\autoref{fig:cache_partition_failures_cachehit} shows the corresponding cache hit rates. 
It can be seen that after all \textit{TaskManagers} fail, the cache hit is back at 100\% after a short time. 
This is because a single local cache contains only 3.000 entries and with a throughput of 4.000 events/s it is filled again very timely.
Note that our chosen latency measurement does not account for waiting time of events at the streaming platform, which is why we do not see a spike of latency in~\autoref{fig:cache_partition_failures_latency} after complete failure even though real consumer lag is experienced.

% Figure environment removed

% Figure environment removed

\subsubsection{State}

As an alternative to enriching events using a cache, we also evaluated data stream enrichment using Flink’s HashMapStateBackend.
Latency is measured over a period of 70 minutes, with a fixed throughput of 4.000 events/s.
We evaluated the enrichment method using two different amounts of historical transaction events, 2,000 and 200,000, since the amount of data managed within the Flink cluster has an impact on latency.
This means that depending on the execution, the available amount of enrichment data is written to Kafka and is then read by Flink as another data stream to enrich the current events.
The results, illustrated in~\autoref{fig:stream_enrichment}, showed that as the amount of enrichment data increased, the latency also increased. 
However, both methods of enrichment using the state backend had lower latency and minor fluctuations compared to enrichment using asynchronous database queries. 
Increasing the amount of data could potentially result in increased latency, and a geographically distributed Flink cluster could further increase network overhead and latency.

\subsection{Log Analytics Use Case}

% Figure environment removed

For the log analytics use case, we evaluated the performance of embedded ML models in Flink with a focus on the number and size of models. 
The evaluation used pre-trained ResNet models, trained on the ImageNet dataset and provided by the ONNX community. 
ResNet models from the image classification domain were selected since different models are provided in terms of memory capacity. The context of the models does not fit into the area of log analytics, but the goal of this evaluation is to achieve a realistic performance analysis of embedded models.
The selected models were ResNet-18 with 18 layers and 44.7 MB memory size and ResNet-101 with 101 layers and 170.6 MB memory size. 
These models receive mini-batches of 3-channel RGB images of the form (N x 3 x H x W), where N is the batch size and H and W must be at least 224.
During the evaluation, the same image with a batch size of 1 and 224 for H and W was always selected as input. 
The models were stored in a local cache in Flink, which was flushed every four minutes to control the number of cache misses and create the same conditions for all executions. 
The throughput chosen for the evaluation was 1000 events/s, and the streaming job was run for 25 minutes with only 12 different keys that were normally distributed. 
For each key, the same model was used, which meant that when all caches were filled, the same model was cached 12 times.

The evaluation revealed that the Java Virtual Machine (JVM) Garbage Collector (GC) was slow in freeing the memory occupied by the ONNX session, which has a similar memory size as the models. 
As a result, the memory resources of a \textit{TaskManager} were quickly used up if new ONNX sessions were added several times per second, while removing old ones from the cache. 
This led to the \textit{TaskManager} crashing, and the entire streaming job had to be restarted. 
It was assumed that the amount of memory of the models was larger than the memory capacity of the \textit{TaskManagers}. 
To prevent the GC from being overwhelmed, the caches were flushed every four minutes, and the number of cache misses was controlled. 
The latencies of the embedded executions of the models were affected by the loading of the models from Google Cloud Storage and the subsequent creation of the ONNX session. 
The latency was highest at the beginning of the execution due to these processes.
The evaluation showed that the ONNX session creation time for the ResNet-18 model was almost up to half a second, while for the ResNet-101 model, it could take over a second. 
The prediction duration for the ResNet-101 model was also more than twice as long as that of the ResNet-18 model. 
The maximum execution times of these two processes for both models are shown in Table \ref{table:model_time_details}. 
The latencies of the ResNet-101 model, as depicted in~\autoref{fig:model_time_latency}, were higher than those of the ResNet-18 model, particularly during the periods when the cache was cleared every four minutes, increasing up to 3.6 seconds during these periods after the latencies normalized in the first few minutes.
Generally, we conclude that while good latencies can be obtained using embedded ML models, established frameworks such as Flink are not yet optimized for such usage, as indicated by our previous findings.

\subsection{Discussion}

\begin{table}
\centering
\caption{Model Details for Log Analytics Use Case}
\begin{tabular}{ccccc} 
 \toprule
 Model & Size & \shortstack{GCS\\Fetch Time} & \shortstack{Session\\Creation Time} & \shortstack{Prediction\\Time} \\
 \midrule
 resnet-18 & 44.7Mb & 871ms & 461ms & 90.8ms \\
 resnet-101 & 170.6Mb & 1663ms & 1098ms & 216.75ms \\
 \bottomrule
\end{tabular}
\label{table:model_time_details}
\end{table}

We presented the evaluation results of different enrichment methods executed in a real data infrastructure in GCP using Flink as a representative DSP system. 
These enrichment methods were to varying extents tested with two representative use cases, where the fraud detection use case can be associated with our defined categories of \emph{simple} and \emph{complex queries}, and the log analytics use case sheds light on a special case from category \emph{finite data sources}.
The experiments conducted to assess synchronous and asynchronous Cassandra queries showed that streaming jobs with asynchronous queries have better performance and are more resource-efficient. 
The evaluation of caching enrichment methods revealed that, for the investigated workloads, using a local cache rather than an external cache leads to better performance, but places a greater load on the resources of the respective DSP system. 
It should be noted that outsourcing data to a cache is not always possible due to the complexity of the database queries or the data structure. 
The use of state backend for enrichment demonstrated reliable enrichment of stream events, but the performance decreases with an increasing amount of enrichment data. 
Furthermore, running embedded ML models in Flink is not suitable for performance-heavy workloads in a single task, and multiple memory-intensive workloads tend to consume too much of available resources, causing the JVM Garbage Collector to free memory slowly.
All in all, and regardless of the details of the implementations, the results presented allow an assessment of in which case and at which approximate throughput rates certain methods of data enrichment are appropriate and what advantages and disadvantages they bring.
\section{Related Work}
\label{sec:related_work}

This section discusses related works on unifying batch and stream jobs, execution of ML models in and out of DSP systems, joining of stream and disk-based data, and event enrichment in modern DSP systems with disk-based databases. 
Its purpose is to contextualize the contribution of this work in enriching events in a DSP system with data from disk-based databases or generated data from a model.
\\
\textbf{Unified Batch and Stream Applications.} 
Service decoupling and the complexity of modern end-to-end data pipelines lead to an increasing overhead that may negatively impact performance. 
Arcon~\cite{MeldrumSKC0H19} and Neptune~\cite{GarefalakisKP19} address the unification of stream and batch processing to increase performance by providing an optimized common intermediate representation and dynamically prioritizing latency-critical jobs in unified stream and batch applications, respectively. 
While modern DSP systems offer the unified execution of stream and batch jobs, they cannot keep up with the query capabilities and memory sizes of modern databases. 
Enriching events during execution in a DSP system with additional data from a database leads to the merging of fast data streams and slow disk-based databases. 
Frameworks such as Kafka could be used in combination with a DSP system as data storage, but this is only applicable for a small number of use cases, as the query capabilities in Kafka are fairly limited compared to traditional databases. 
Therefore, there are no specific approaches to unifying large-scale data storage and latency-critical streaming applications to achieve more effective resource utilization and improved latency.\\
\textbf{Model Performance Evaluation.} 
The first performance evaluation study of model-serving integration tools in stream processing frameworks has been conducted in~\cite{HorchidanKKC22} by assessing the internal and external execution of a model in DSP systems. 
The integration of ML models assumes that the DSP system requires multiple models, which may exceed its storage capacity.
Additionally, the study considers different model sizes and addresses associated memory concerns. 
The work demonstrates that there are benefits to using integrated execution over external execution DSP frameworks, and that certain model formats offer superior performance.\\
\textbf{Data Warehouse Source Updates.}
Earlier works addressing the combination of high-speed data streams and slow disk-based databases involve active data warehousing. 
In this scenario, a data stream refers to quickly incoming events of source updates, which the data warehouse must process in real-time. 
One of the initial solutions to this issue is the MESHJOIN~\cite{PolyzotisSVSF08} algorithm, which has several variations and extensions~\cite{NaeemDWA10,NaeemDW12}. 
MESHJOIN fuses a high-speed data stream with a disk-based relationship, under the constraint of limited memory, using a hash-join. 
The algorithm scans the entire disk-based relationship sequentially at high speed, and the incoming events from the stream are processed in windows and then combined with the entries from the relationship. 
This approach distributes the expenses of the input-output operations across windows of stream events.\\
\textbf{Disk-based Database Enrichment.}
In~\cite{DerakhshanSS13}, an operator for DSP systems is proposed that enriches incoming stream events using a cache with data from a relational and disk-based database in a single node.
Depending on whether an incoming event causes a cache hit or a cache miss, it is processed in a thread for the respective category. 
Events causing cache misses are combined into batches and then used to query the database. 
Meanwhile, events causing cache hits are processed in parallel. After processing, the two sets of events are merged back into their original order. 
The paper reports higher throughput with this approach compared to a record-at-a-time approach. 
The experiments were conducted using a simulated DSP system and a MySQL database.
In~\cite{JeonLK19}, the authors describe a join between stream and disk-based data using a micro-batch model built on Spark Streaming~\cite{ZahariaDLHSS13} and MongoDB. 
The approach considers distributed execution of operators and assumes that the external disk-based data volume is larger than the storage capacity of the DSP system. 
To minimize database access, a cache is implemented in Spark that stores database entries in their own RDDs. 
If data is unavailable in the cache, a query is generated for multiple cache miss keys to reduce the number of queries. 
The authors also implemented a load-balancing mechanism by dynamically adjusting cache sizes in the DSP system to regulate the database load. 
In~\cite{KimL20c}, the authors extended~\cite{JeonLK19} to support similarity joins.

\section{Conclusion}

In this work, we present \texttt{vox2vec} --- a self-supervised framework for voxel-wise representation learning in medical imaging. Our method expands the contrastive learning setup to the feature pyramid architecture allowing to pre-train effective representations in full resolution. By pre-training a FPN backbone to extract informative representations from unlabeled data, our method scales to large datasets across multiple task domains. We pre-train a FPN architecture on more than 6500 CT images and test it on various segmentation tasks, including different organs and tumors segmentation in three setups: linear probing, non-linear probing, and fine-tuning. Our model outperformed existing methods in all regimes. Moreover, \texttt{vox2vec} establishes a new state-of-the-art result on the linear and non-linear probing scenarios. 

Still, this work has a few limitations to consider. We plan to investigate further how the performance of \texttt{vox2vec} scales with the increasing size of the pre-training dataset and the pre-trained architecture size. Another interesting research direction is exploring the effectiveness of \texttt{vox2vec} in the domain adaptation and few-shot learning scenarios.


\section*{Acknowledgments}
This work has been supported through grants by the Google Cloud Research Credits program (award GCP211695187).

\bibliographystyle{IEEEtran}
\bibliography{bib}

\end{document}
