\section{Conclusion}
\label{sec:conclusion}

This paper conducted an assessment of various data enrichment methods for DSP systems. 
To account for a broad range of practical scenarios, three categories of data enrichment use cases were identified, covering database queries of different complexity as well as the embedding of state, and corresponding enrichment methods were designed and executed in Apache Flink.
To realistically evaluate these methods, representative use cases were implemented and run in a data infrastructure in a public cloud.
Our results showcase the advantages and limitations of the investigated methods, underlining the necessity for both sophisticated caching strategies for conventional data as well as better integration of embedded ML models into existing or future DSP systems.


To build on the work presented in this paper, future research could focus on building a dynamic caching mechanism that adapts to the access speeds of sources and different workloads.
Another promising direction is to explore methods that are reflecting more specialized situations, such as using GPU acceleration or cloud-based methods like serverless.