\section{Representing each stablecoin type with the model}
\label{sec:type}

%\subsection{Modeling Intuition}

In this section, we describe how $v$ is characterized in various types of stablecoins. Here, $v$ and $v^\prime$ would be the same because all stablecoins analyzed in this paper have a pegging mechanism independent of time. 
First, to ease the reader's understanding, we start with an intuition justifying how we will represent each type of stablecoins with our model. While a stablecoin system pays users (more than) the target value (e.g., \$1) upon redemption to prevent its price from falling below the target value, the payment that users realize can be lower than the intended value, depending on the type of assets paid. 

For example, consider a stablecoin system that holds cryptocurrencies as reserves to maintain the peg. It is widely known that cryptocurrencies experience high price volatility, which can result in unforeseen stability-threatening events in these stablecoin systems. Even if the systems were to reward a user with \$1 in cryptocurrency, the user might not end up receiving assets worth \$1. That is because there exist several challenges, such as delayed price oracles, as well as the sheer time difference between transaction creation and execution. 
The challenges cause payment value discrepancies between users and the system. %??, which also leads to price volatility of stablecoins. %given that the pegging mechanism of stablecoins relies on arbitrageurs and high frequency traders. %are abundant in state of the art DeFi systems~\cite{zhou2021high,qin2022quantifying}. 
Indeed, we have experienced stablecoins depegging permanently because their oracle update interval was set to 10 minutes (cf. Iron~\cite{Analysis10:online}).
As a result, the stablecoin's price stability can be threatened if the collateral is a volatile asset. 

The situation deteriorates if a system uses assets with higher risk (e.g., endogenous cryptocurrencies) to back its stablecoin.
Such assets would receive greater downward market pressure from the stablecoin redemption process in which the system should release the assets on the market.
The downward pressure can lead to liquidation spirals and market panics of the reserve assets, leading to a significant price drop and swiftly pushing a stablecoin toward an unhealthy state. A famous example of such a downfall is the severe price drop of Luna (an endogenous asset in the UST system) due to significant UST redemptions by users in May 2022. 
In this case, even though the Luna system paid users \$1 based on its Luna price oracle, users did not expect to earn \$1 due to a rapid decline in Luna's price.

With these intuitions in mind, we characterize $v$ of various types of stablecoins. Table~\ref{tab:v} summarizes $v$ of stablecoins that will be described below. 

\begin{table}[!ht]
    \centering
    \begin{tabular}{c|>{\raggedright\arraybackslash}m{0.8\columnwidth}}
        %\hline
        Type & \multicolumn{1}{c}{$v$} \\
        \hline
        \hline
        Fiat & $1\,\,$ if $Q\leq V^f,\,\,$ otherwise $V^f/Q$ \\
        \hline
        Crypto & $r^c\left(Q, \theta\right)\,\,$ if $Q\leq V^c(\theta),\,\,$ otherwise $r^c\left(Q, \theta\right)\cdot V^c(\theta)/Q$ \\
        \hline
        Algo & $r^c\left(Q, \theta\right)$ \\
        \hline
        Over & $r^c\left(Q, \theta\right)\cdot o(\theta)\,\,$ if $0< D^L(\theta)$ or $0< D_{u_i}(\theta),\,\,$ otherwise 0  \\
        \hline
    \end{tabular}
    \caption{$v$ by stablecoin type}
    \label{tab:v}
\end{table}

\subsection{Fiat-collateralized stablecoins}

We first look at fiat-collateralized stablecoins such as USDT and USDC. 
In these stablecoins, the system issues coins and purchases them from the market at 1 in fiat currency to achieve the peg. 
However, taking coins out of the market is possible only when the system's fiat reserves are not exhausted. 
We let $Q$ and $V^f$ denote the total quantity of coins whose holders want to redeem at one point and the total value of the fiat reserves at the moment, respectively. 
If $Q\leq V^f$, users can always be paid 1, and $v$ is 1. 
On the other hand, if $Q> V^f,$ users have two cases where they receive 1 or not. Then, under the assumption that users are uniform, the expected value would be $V^f/Q$ because the probabilities of users to get 1 and 0 is $V^f/Q$ and $1-V^f/Q$, respectively. Therefore, in that case, the expected value of $v$ would be $V^f/Q$. 

\subsection{Crypto-collateralized stablecoins} 

These stablecoins\footnote{Note that they are different from over-collateralized stablecoins in this paper.} (e.g., USDN) are similar to fiat-collateralized stablecoins except that the reserves are stored in cryptocurrencies. 
Specifically, the pegging mechanism pays 1 in cryptocurrencies to users in return for taking back a stablecoin from them. 
However, unlike fiat collateral, the cryptocurrency price fluctuates, which can cause the discrepancy between the cryptocurrency price values that users and the system refer to due to non-zero transaction time and a discrete cryptocurrency price oracle update within the system. 
Thus, the value users earn can differ from 1.

Here, the cryptocurrency used as collateral is denoted by $c$.
Let $p^c_u$ and $p^c_s$ denote $c$'s price to which users and the system refer, respectively. 
In addition, the total value of cryptocurrency reserves is determined by the characteristics of the economy, which is a fundamental state $\theta$. 
Therefore, the total value of cryptocurrency reserves is denoted by a strictly increasing function $V^c(\theta)$ of $\theta$; naturally, the condition of greater asset value and higher asset growth is stronger fundamentals. 
Then $v$ is $p^c_u/ p^c_s$ in the case of $Q \leq V^c(\theta)$; users receive $1 /p^c_s$ cryptocurrencies by redeeming a stablecoin only when the crypto reserves are not depleted.

In fact, a ratio $p^c_u/p^c_s$ is for the fluctuation of $c$'s price; a greater drop in $c$'s price implies a smaller value of the ratio below 1. 
The price change of crypto collateral is affected by an economic situation that can be represented by the state of fundamentals, and also the stablecoin redemption action of users.
Note that in the redemption process, the system should pay the users, which implies that a part of the crypto collateral should be unlocked and flow into the market.
This can lower the cryptocurrency price by increasing its circulating supply and/or bringing other secondary market impacts. 
%The quantity of cryptocurrencies that will enter the market would be $1\cdot Q/p^c$, and therefore a ratio between the quantity of unlocked cryptocurrencies and its total supply would be $\frac{1\cdot Q}{p^c\cdot T^c}$, where $T^c$ indicates the total supply of the cryptocurrency of which reserves are composed.
Given this, we will denote $p^c_u/p^c_s$ by a function $r^c\left(Q, \theta\right)$, where $r^c$ is strictly decreasing for the first input $Q$ and strictly increasing for the second input $\theta$.
That is, $p^c_u/p^c_s$ decreases as users redeem more stablecoins in a short period, while $p^c_u/p^c_s$ increases as the economic condition is better. 
As a result, $v$ is expressed as $r^c\left(Q, \theta\right)$ when $Q \leq V^c(\theta).$

On the other hand, if $Q> V^c(\theta)$, users can consider two cases where they receive $r^c\left(Q, \theta\right)$ or not. Then because the probability of users to get the reward is $V^c(\theta)/Q$ under the assumption that users are symmetric, the expected value would be $$r^c\left(Q, \theta\right)\cdot \frac{V^c(\theta)}{Q}=r^c\left(Q, \theta\right)\times \frac{V^c(\theta)}{Q} + 0 \times \left(1-\frac{V^c(\theta)}{Q}\right).$$
Therefore, in this case, the value of $v$ that users expect is $r^c\left(Q, \theta\right)\cdot V^c(\theta)/Q$. 

Additionally, note that $r^c$ and $V^c$ depend on the cryptocurrency $c$ used as collateral; therefore, even for the same inputs, the values of $r^c$ and $V^c$ can differ by which cryptocurrency the system uses. 
For example, consider a system that employs a robust exogenous cryptocurrency such as Bitcoin or Ethereum. 
How many of these cryptocurrencies the system will unlock to the market in the redemption process does not significantly influence their price because their value comes from other external sources. Therefore, in that case, $Q$'s impact on $r^c$ can be negligible. 
On the contrary, it can become significant (i.e., $\Delta r^c(Q,\theta)/\Delta Q$ would be larger) when $c$'s value comes from the system's usage (e.g., endogenous cryptocurrencies designed to back stablecoins). 

\subsection{Algorithmic stablecoins}
This category of stablecoins uses an endogenous cryptocurrency that the system can directly mint and burn to stabilize the stablecoin price. 
%That is, a system should be able to control the supply of two cryptocurrencies, including its stablecoin. 
As the most representative example, UST falls into this category. 
An algorithmic stablecoin system sells and buys stablecoins to the market at the price $1$, where the payment is processed with its endogenous cryptocurrency (e.g., Luna in the Terra system). 
However, as mentioned when describing crypto-collateralized stablecoins, the value that users receive may be different from $1$. 
Then, similar to crypto-collateral stablecoins, $v$ is $r^c(Q, \theta)$.
A different point with crypto-collateralized stablecoins is that crypto-collateralized stablecoins have $v$ as $r^c\left(Q, \theta\right)$ only when the reserves are not exhausted, while $v$ always has this value in algorithmic stablecoins because they can always pay users by minting their endogenous cryptocurrencies. 
%Therefore, $\frac{1\cdot Q}{p^c(\theta)\cdot T^c}$ has a limit in crypto-collateralized stablecoins, but not in algorithmic stablecoins, which implies that algorithmic stablecoins can also have $v$ as almost zero.

\subsection{Over-collateralized stablecoins} 
This type of stablecoins containing USDX and DAI\footnote{More specifically, the current version of DAI has a mixed mechanism of crypto collateralization using other stablecoins and over-collateralization.} is also popular. 
They require users to deposit crypto collateral with a value greater than $1$ in the system when minting a stablecoin. %, to prevent a shortage of reserves needed to back the stablecoins. 
Here, we can consider the minted stablecoins as the user's debt. 
The stablecoin system returns the deposited collateral to its owner only when he/she redeems stablecoins. That is, in that case, only debtors can redeem their stablecoins and receive a value from the system. 
Alternatively, if a liquidation process for some specific collateral starts due to a drop in the collateral values, other users can redeem stablecoins (i.e., pay back stablecoin debts) on behalf of the debtor in return for taking the collateral.
Given this, $v$ should depend on whether users are stablecoin debtors or non-debtors.

First, let $D^L(\theta)$ denote the total quantity of stablecoins that users \textit{should redeem} to the system in the liquidation process at one point. 
If $D^L(\theta)$ is zero, it means that a liquidation process was not triggered for any collateral at the moment. 
Meanwhile, if $D^L(\theta)$ is equal to the circulating supply of the stablecoin, it suggests that a liquidation process was initiated for all collateral, encouraging users to redeem all stablecoins in the market. 
%$Q^L$ indicates the total quantity of stablecoins that users \textit{want to redeem} in the liquidation process at one point. 
We also use the notation $D_{u_i}(\theta)$ to denote the stablecoin debt of user $u_i$ that did not enter a liquidation process. %and $Q^D_{u_i}$ indicates a quantity of stablecoin that user $u_i$ now wants to redeem out of its debt.
Here, $D_{u_i}(\theta)$ would be an increasing function of $\theta$; under a bad economic condition, more collateral of that user would be at risk and enter liquidation. 
If $D_{u_i}(\theta)$ is zero, user $u_i$ cannot redeem its stablecoins unless a liquidation process exists for some collateral. Note that the fact that $D_{u_i}(\theta)=0$ implies that user $u_i$ is a non-debtor or its deposited collateral is in the liquidation progress. According to the definition of notations $D^L(\theta)$ and $D_{u_i}(\theta)$, their relationship is as follows: $D^L(\theta)+\sum_{u_i}D_{u_i}(\theta)$ should be the same as the total circulating supply of stablecoins. Therefore, when $D^L(\theta)$ is the circulating supply of the stablecoin, $D_{u_i}(\theta)$ would be zero for any user $u_i$.

Lastly, we consider the notation $o(\theta)$: Crypto collateral to be paid per stablecoin from the system to users has a value of $o(\theta)$ from the system's perspective.\footnote{Strictly speaking, $o(\theta)$ would be different by each pair of collateral and debt, but we simplify it by standardizing the value over a pair because it does not affect our main theoretical analysis and result.} 
For example, if the collateral value deposited by user $u_i$ is 1.5 times the stablecoin debt based on the system's price oracle, then $o(\theta)$ would be $1.5.$
The value of $o(\theta)$ would be greater than $1$ unless the state of fundamentals $\theta$ is too small, because the system requires collateral worth more than $1$ when issuing a stablecoin debt to a user.
As a result, $v$ of user $u_i$ would be expressed as follows: $r^c\left(Q, \theta\right)\cdot o(\theta)$ if $0< D^L(\theta)$ or $0<D_{u_i}(\theta)$, otherwise zero. 
That is, it means that users can be paid by redeeming stablecoins to the system, only when there are liquidation processes or they are stablecoin debtors whose deposited collateral did not enter a liquidation process. 