\section{Equilibrium for Each Stablecoin Type}
\label{sec:eq}

In this section, we will analyze the equilibria and price instability for each type of stablecoin. 
To focus on the equilibria changing with $v$ and $v^\prime$ (i.e., equilibria varied by a stablecoin design), we consider that users have little incentive to keep holding coins (i.e., $i(x)\approx x$).
%In other words, we do not consider the case where users expect the stablecoin price can be $1$ due to an economic uptrend even without a proper value of $v$ and $v^\prime.$
We also assume that values of the future status variables including the future fundamental state $\theta^\prime$, which $v^\prime$ depends on in stablecoins using cryptocurrencies as reserves, can approximate values of the current status variables such as the current fundamental state $\theta$ to simplify the equilibrium analysis by reducing the dimension of parameters. If we consider the future fundamental state as a variable $\theta^\prime$ independent of $\theta$, our results would be extended to two dimensions of fundamental states $[\theta, \theta^\prime]$.

We first present Figure~\ref{fig:range} that illustrates the result visually, which makes it possible to compare stablecoin designs intuitively.
In the figure, $\theta_{\max}$ and $\theta_{\min}$ indicate the maximum and minimum values of $\theta$, respectively. 
The blue bar represents a range of $\theta$ in which a unique pegging equilibrium exists so the peg is guaranteed. 
The yellow range of $\theta$ has multiple equilibria, including the pegging state of $p(M)=1$, which implies that the peg state is not guaranteed even though it can be reached. 
More specifically, in this range, the pegging state is a self-fulfilling equilibrium; users' belief totally determines the destiny of stablecoins.
Finally, in the red range of $\theta$, there is only a depegging equilibrium, which implies that the peg cannot be achieved. 
Therefore, the wider the blue range, the better the stablecoin design. 

A value of $\overline{\theta}$ indicates a lower bound for having a unique pegging equilibrium, and $\underline{\theta}$ is an upper bound for having only a depegging equilibrium. 
Therefore, according to Figure~\ref{fig:range}, stablecoins fully backed by fiat assets have $\overline{\theta}$ as $\theta_{\min}$. Stablecoins partially backed by fiat assets do not have a value of $\overline{\theta}$ and $\underline{\theta}$. 
Furthermore, an over-collateralized stablecoin does not have $\overline{\theta}$. 
In crypto-collateralized, algorithmic, and over-collateralized stablecoins, the values of $\overline{\theta}$ and $\underline{\theta}$ would depend on $c$. 
Specifically, if the system employs a more robust crypto asset in its pegging mechanism, the blue zone can widen by decreasing $\overline{\theta}$.
On a side note, a significant incentive function $i$ can also help expand the blue zone and narrow the yellow and red zones.

We now describe below the equilibrium state of stablecoin systems in detail.

% Figure environment removed

\subsection{Fiat-collateralized stablecoins}

As described in Section~\ref{sec:model}, fiat-collateralized stablecoins have $v$ as $1$ if $Q\leq V^f,$ otherwise $V^f/Q.$ 
The following theorem presents equilibria of fiat-collateralized stablecoins. 

\begin{theorem}
Fully backed fiat-collateralized stablecoins have a unique pegging equilibrium for any $\theta$. On the other hand, for any $\theta$, partially backed fiat-collateralized stablecoins have multiple equilibria including the pegging state due to users' self-fulfilling beliefs.
\label{thm:fiat}
\end{theorem}

According to Theorem~\ref{thm:fiat}, fully backed fiat-collateralized stablecoins can guarantee the peg. 
Meanwhile, a system partially backing its stablecoin has multiple equilibria, so it is difficult to predict the consequence.
The system can reach the pegging state, but it is not guaranteed.
In particular, it suffers from a self-fulfilling belief of users; 
if users believe only a few stablecoin holders will redeem their coins, the users with that expectation would not need to redeem their coins to the system right now, which fulfills the expectation by themselves and, in turn, results in price stabilization. 
On the other hand, let us assume that users believe that too many holders will redeem coins so that the reserves of the system cannot cover it. 
Then users should immediately redeem their coins to the system, which realizes the expectation by themselves and puts the stablecoin in danger by bringing about the depreciation of the stablecoin.
We present the proof of Theorem~\ref{thm:fiat} in Appendix~\ref{app:proof2}. 

\subsection{Crypto-collateralized \& algorithmic stablecoins}

Next, we look at crypto-collateralized and algorithmic stablecoins.
Crypto-collateralized stablecoins have a value of $v$ as $r^c\left(Q, \theta\right)$ if $Q\leq V^c(\theta)$, otherwise $r^c\left(Q, \theta\right)\cdot V^c(\theta)/Q.$ 
In algorithmic stablecoins, $v$ is $r^c(Q, \theta)$.
We present their equilibria in Theorem~\ref{thm:crypto}. 

\begin{theorem}
We denote the total market supply of stablecoins by $T^s.$
Then, in crypto-collateralized and algorithmic stablecoins, $\overline{\theta}$ is a value such that $r^c\left(T^s,\overline{\theta}\right)=1$, and $\underline{\theta}$ is a value such that $r^c\left(0,\underline{\theta}\right)=1.$
Here, for crypto-collateralized stablecoins, we assume that $V^c(\underline{\theta})\geq T^s$.
Then both crypto-collateralized and algorithmic stablecoins have a unique pegging equilibrium for any $\theta\geq\overline{\theta}$, multiple equilibria including the pegging state for any $\theta$ in the range $[\underline{\theta},\overline{\theta})$, and depegging equilibria for any $\theta<\underline{\theta}$. 
\label{thm:crypto}
\end{theorem}

%\noindent\textbf{Intuition behind Theorem~\ref{thm:crypto}:} 
Crypto-collateralized and algorithmic stablecoins can guarantee the peg under good economic conditions. 
However, under poor economic conditions, even if the reserves are sufficient to back the stablecoins fully, they would not be able to reach the pegging state because users who redeem their coins cannot, in effect, receive 1 due to a downward price fluctuation of the cryptocurrency, the payment medium.
In the mediocre economic status, there are multiple equilibria including the pegging state; the successful peg is up to users' belief in others' redemption actions because the stablecoin redemption affects whether a cryptocurrency price can be less than 1 in that economic condition. 

The values of $\overline{\theta}$ and $\underline{\theta}$ depend on $r^c$ according to Theorem~\ref{thm:crypto}; $\overline{\theta}$ and $\underline{\theta}$ are the values such that $r^c\left(T^s,\overline{\theta}\right)=1$ and $r^c\left(0,\underline{\theta}\right)=1$, respectively. Note that $\overline{\theta}$ is greater than $\underline{\theta}$ because $r^c$ decreases and increases as the first and second inputs increase, respectively. 
The more robust cryptocurrency the systems use to back stablecoins, the gentler the slope of $r^c$. 
That is, for a more robust cryptocurrency $c$, $r^c$ can be greater, so $\overline{\theta}$ and $\underline{\theta}$ become lower. 
For example, crypto-collateralized stablecoins can use Bitcoin, Ethereum, or even other stablecoins as their collateral to widen the blue zone.
Meanwhile, algorithmic stablecoins should use endogenous cryptocurrencies according to their protocol, which would have a narrower blue zone and a wider red zone. 
We present the proof of the theorem in Appendix~\ref{app:proof3}.

\subsection{Over-collateralized stablecoins}

Finally, we consider over-collateralized stablecoins.
The stablecoin has $v$ for user $u_i$ as follows: $r^c\left(Q, \theta\right)\cdot o(\theta)$ if $0< D^L(\theta)$ or $0< D_{u_i}(\theta)$, otherwise zero. 
Here, $o(\theta)$ is greater than $1$ as long as $\theta$ is not too low. 
Theorem~\ref{thm:over} presents equilibria of over-collateralized stablecoins. 

\begin{theorem}
We assume that $r^c(0,\theta)\cdot o(\theta)<1$ when $D^L(\theta)=T^s$.
Then over-collateralized stablecoins have multiple equilibria including the pegging state for any $\theta\geq\underline{\theta}$, and depegging equilibria for any $\theta<\underline{\theta}$, where $\underline{\theta}$ satisfies $r^c(0,\underline{\theta})\cdot o(\underline{\theta})=1$. Moreover, $\underline{\theta}$ of over-collateralized stablecoins is smaller than that for crypto-collateralized and algorithmic stablecoins.
\label{thm:over}
\end{theorem}

In over-collateralized stablecoins, stablecoin debtors can redeem their coins anytime, while non-debtors cannot unless a liquidation process starts for some collateral. 
Therefore, users' belief regarding redemption by debtors plays an important role in maintaining the peg; if users believe that many stablecoin debtors will redeem their coins, debtors with this expectation will redeem their coins now to settle their debt cheaper because they expect an increase in the stablecoin price due to other debtors' redemption. This leads to a rise in a stablecoin price by fulfilling their expectation by themselves. 
On the contrary, if users believe that only a few debtors will redeem their coins, debtors with the expectation do not need to redeem their coins immediately because they do not think the coin price will increase. 
Therefore, the stablecoin price will not increase by vindicating their decision.

On the other hand, if the collateral value is less than 1 due to severely bad economic conditions, the system would not be able to attain the pegging state. 

Moreover, where a system pays users more than 1 based on its price oracle, $v$ can be not less than 1 even with a price drop of cryptocurrencies. This makes $\underline{\theta}$ of over-collateralized stablecoins smaller than that for crypto-collateralized and algorithmic stablecoins. 
%
We present the proof of Theorem~\ref{thm:over} in Appendix~\ref{app:proof4}. 