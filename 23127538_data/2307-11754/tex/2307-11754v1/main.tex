
\documentclass[a4paper,UKenglish,cleveref, autoref, thm-restate]{lipics-v2021}

\bibliographystyle{plainurl}% the mandatory bibstyle


\usepackage{hyperref}
\usepackage{xspace}

%\settopmatter{printfolios=true}
\usepackage[T1]{fontenc}
\usepackage{graphicx, amsmath,bm}
\usepackage{colortbl}
\usepackage{hhline}
\usepackage[title]{appendix}
\usepackage{etoolbox}
\usepackage{xcolor}
\renewcommand\UrlFont{\color{blue}\rmfamily}
\usepackage{tikz, pifont}
\usepackage{stackengine}
\usepackage{subcaption}

\usepackage{makecell}
\usepackage{array}
\renewcommand{\qed}{\hfill\blacksquare}
\usepackage{setspace}

\definecolor{YellowGreen}{rgb}{0.75, 0.9, 0.36}

\usepackage{multirow}
\newif\ifshowcomment
\showcommenttrue
\ifshowcomment
    \newcommand{\dawn}[1]{\textsf{\color{red}{[{Dawn: #1}]}}}
    \newcommand{\yujin}[1]{\textsf{\color{blue}{[{Yujin: #1}]}}}
    \newcommand{\korn}[1]{\textsf{\color{magenta}{[{Kornrapat: #1}]}}} 
    \newcommand{\ag}[1]{\textsf{\color{red}{[{AG: #1}]}}} 
    \newcommand{\ariah}[1]{\textsf{\color{red}{[{Ariah: #1}]}}}
    \newcommand{\philipp}[1]{\textsf{\color{teal}{[{Philipp: #1}]}}}
    \newcommand{\todo}[1]{\textsf{\color{orange}{[{TO DO: #1}]}}}
\else
    \newcommand{\dawn}[1]{}
    \newcommand{\yujin}[1]{}
    \newcommand{\korn}[1]{}
    \newcommand{\ag}[1]{}
    \newcommand{\ariah}[1]{}
    \newcommand{\philipp}[1]{}
    \newcommand{\todo}[1]{}
\fi

% commands
\newcommand{\etal}{\textit{et al.\ }}


\title{What Drives the (In)stability of a Stablecoin?}

%\titlerunning{Dummy short title} %TODO optional, please use if title is longer than one line

% \author{
% {\rm Yujin Kwon\textsuperscript{1}, Kornrapat Pongmala\textsuperscript{1}, Kaihua Qin\textsuperscript{2}, Ariah Klages-Mundt\textsuperscript{3}, Philipp Jovanovic\textsuperscript{4}, Christine Parlour\textsuperscript{1}, Arthur Gervais\textsuperscript{4}, Dawn Song\textsuperscript{1}}\\
%  {\rm \textsuperscript{1}UC Berkeley 
%   \hspace{0.05in}
%   \textsuperscript{2}Imperial College London
%   \hspace{0.05in}
%   \textsuperscript{3}Cornell
%    \hspace{0.05in}
%    \textsuperscript{4}UCL}
%   \\
%   }


\author{Yujin Kwon}{UC Berkeley, USA}{yujinyujin9393@berkeley.edu}{ORCID}{}%{(Optional) author-specific funding acknowledgements}
\author{Kornrapat Pongmala}{UC Berkeley, USA}{kornrapatp@berkeley.edu}{ORCID}{}
\author{Kaihua Qin}{Imperial College London, UK}{kaihua.qin@imperial.ac.uk}{ORCID}{}
\author{Ariah Klages-Mundt}{Cornell, USA}{aak228@cornell.edu}{ORCID}{}
\author{Philipp Jovanovic}{UCL, UK}{p.jovanovic@ucl.ac.uk}{ORCID}{}
\author{Christine Parlour}{UC Berkeley, USA}{parlour@berkeley.edu}{ORCID}{}
\author{Arthur Gervais}{UCL, UK}{arthur@gervais.cc}{ORCID}{}
\author{Dawn Song}{UC Berkeley, USA}{dawnsong@cs.berkeley.edu}{ORCID}{}

%TODO mandatory, please use full name; only 1 author per \author macro; first two parameters are mandatory, other parameters can be empty. Please provide at least the name of the affiliation and the country. The full address is optional. Use additional curly braces to indicate the correct name splitting when the last name consists of multiple name parts.

%\author{Joan R. Public\footnote{Optional footnote, e.g. to mark corresponding author}}{Department of Informatics, Dummy College, [optional: Address], Country}{joanrpublic@dummycollege.org}{[orcid]}{[funding]}

\authorrunning{Kwon et al.} %TODO mandatory. First: Use abbreviated first/middle names. Second (only in severe cases): Use first author plus 'et al.'

% \Copyright{Jane Open Access and Joan R. Public} %TODO mandatory, please use full first names. LIPIcs license is "CC-BY";  http://creativecommons.org/licenses/by/3.0/

% \ccsdesc[100]{\textcolor{red}{Replace ccsdesc macro with valid one}} %TODO mandatory: Please choose ACM 2012 classifications from https://dl.acm.org/ccs/ccs_flat.cfm 

\keywords{Stablecoin, Game theory, Price stability, Data analysis} 
% \category{} %optional, e.g. invited paper

%\relatedversion{} %optional, e.g. full version hosted on arXiv, HAL, or other respository/website
%\relatedversiondetails[linktext={opt. text shown instead of the URL}, cite=DBLP:books/mk/GrayR93]{Classification (e.g. Full Version, Extended Version, Previous Version}{URL to related version} %linktext and cite are optional

%\supplement{}%optional, e.g. related research data, source code, ... hosted on a repository like zenodo, figshare, GitHub, ...
%\supplementdetails[linktext={opt. text shown instead of the URL}, cite=DBLP:books/mk/GrayR93, subcategory={Description, Subcategory}, swhid={Software Heritage Identifier}]{General Classification (e.g. Software, Dataset, Model, ...)}{URL to related version} %linktext, cite, and subcategory are optional

%\funding{(Optional) general funding statement \dots}%optional, to capture a funding statement, which applies to all authors. Please enter author specific funding statements as fifth argument of the \author macro.

%\acknowledgements{I want to thank \dots}%optional

\nolinenumbers %uncomment to disable line numbering

%Editor-only macros:: begin (do not touch as author)%%%%%%%%%%%%%%%%%%%%%%%%%%%%%%%%%%
% \EventEditors{John Q. Open and Joan R. Access}
% \EventNoEds{2}
% \EventLongTitle{42nd Conference on Very Important Topics (CVIT 2016)}
% \EventShortTitle{CVIT 2016}
% \EventAcronym{CVIT}
% \EventYear{2016}
% \EventDate{December 24--27, 2016}
% \EventLocation{Little Whinging, United Kingdom}
% \EventLogo{}
% \SeriesVolume{42}
% \ArticleNo{23}
%%%%%%%%%%%%%%%%%%%%%%%%%%%%%%%%%%%%%%%%%%%%%%%%%%%%%%

\begin{document}

\maketitle

\begin{abstract}
In May 2022, an apparent speculative attack, followed by market panic, led to the precipitous downfall of UST, one of the most popular stablecoins at that time. 
However, UST is not the only stablecoin to have been depegged in the past. Designing resilient and long-term stable coins, therefore, appears to present a hard challenge.

To further scrutinize existing stablecoin designs and ultimately lead to more robust systems, we need to understand where volatility emerges.
Our work provides a game-theoretical model aiming to help identify why stablecoins suffer from a depeg.
This game-theoretical model reveals that stablecoins have different price equilibria depending on the coin's architecture and mechanism to minimize volatility.
Moreover, our theory is supported by extensive empirical data, spanning $1$ year. To that end, we collect daily prices for 22 stablecoins and on-chain data from five blockchains including the Ethereum and the Terra blockchain.
%, which is necessary to lay the groundwork towards better stablecoin design principles
\end{abstract}


% Paper body
% Figure environment removed

\section{Introduction}
Automatic 3D reconstruction of clothed humans using image inputs has gained increasing significance due to its potential applications in a wide array of AR/VR scenarios. High-fidelity reconstructions typically depend on sophisticated capture systems, which are developed with dense camera arrays~\cite{collet2015high,joo2015panoptic,joo2018total}, programmable light-stages~\cite{Vlasic2009, guo2019relightables}, and depth sensors~\cite{newcombe2011kinectfusion,DoubleFusion,BodyFusion,dou2016fusion4d,newcombe2015dynamicfusion}. However, stringent capture environments equipped with complex hardware pose significant challenges for consumer-level applications.


In this context, considerable research effort has been dedicated to developing methods that allow for more flexible capture configurations, such as utilizing a few RGB inputs. Among these works, learning implicit functions \cite{iccv2020PIFu, saito2020pifuhd, hong2021stereopifu} has proven effective in achieving highly detailed reconstructions by integrating the advancements of deep neural networks. These methods employ large multi-layer perceptrons (MLPs) to predict the occupancy probability or truncated signed distance function (TSDF) value of every queried 3D point based on its associated local feature, which is extracted from images. They can recover a continuous surface at arbitrary resolutions without topology restrictions.


However, in typical MLP-based implicit networks, the occupancy or TSDF value at each location is solved independently with planar image features, rendering them less capable of addressing challenging cases such as occlusions. Consequently, these methods suffer from generalization and robustness issues, particularly when tackling strong occlusions caused by large motion or multiple interacting humans. 
Some follow-up studies  \cite{zheng2021deepmulticap,zheng2021pamir,huang2020arch} utilize an extra geometric model, SMPL~\cite{Loper2015}, to improve robustness by introducing strong shape priors. 
Their success typically relies on the assumption of geometrical similarity \cite{huang2020arch} between the shape prior and target reconstruction, making them intractable for handling complex cases with loose clothes and sensitive to errors in SMPL model fitting.



%\ping{this paragraph sounds like `TSDF is better than MLP/SMPL, and we use TSDF to solve the problem'. But in Sec 3, we are telling a different story, saying `MLP needs a 3D convolutional encoder'. We need to make these two sections consistent.}\sicong{I think in this paragraph we claim that the TSDF}


%We opt for Trucated Signed Distance Funtion (TSDF) volumetric representations as they are naturally suitable for convolution operations, which have shown remarkable performance for learning hierarchical features on 2D visual perception tasks \cite{SunXLW19}. 
%Meanwhile, TSDF also describes the gradual geometry change around shape surface, which is not reflected by occupancy volume. 

We instead revisit the 3D volumetric representation and resort to 3D convolutional neural networks (CNNs) for feature learning, due to their impressive performance in feature learning and the ability to incorporate spatial context. However, volumetric methods and 3D convolution involve discretization, which might raise concerns regarding whether a discretized volume can preserve subtle geometric details as continuous representations learned in implicit functions. We investigate the relationship between volume resolution and quantization error on synthetic data by converting target mesh objects to TSDF volumes, as shown in Figure~\ref{fig:quantization_error}. We observe that the quantization errors are significantly reduced by increasing volume resolution and become nearly negligible when reaching a relatively high resolution (e.g., 512 or higher). In other words, achieving fine-detailed reconstruction is not supposed to be restricted by the use of volume representations as long as a proper volume resolution is utilized. Therefore, we present a method with high-resolution feature volumes, e.g., 256 and 512, while traditional volumetric methods \cite{varol18_bodynet,gilbert2018volumetric} are often limited to much lower resolutions, such as 32 or 128.



On the other hand, an increase in volume resolution may lead to a cubic growth of memory overhead \cite{8100085}. Reducing memory costs while guaranteeing the granularity of volumetric representations is necessary for pursuing high-quality reconstruction. Thus, we adopt a coarse-to-fine approach and cull away irrelevant voxels to build a sparse high-resolution feature volume. At the coarse level, the network computes an initial TSDF by applying a U-Net with sparse 3D CNN \cite{3DSemanticSegmentationWithSubmanifoldSparseConvNet} on the sparse feature volume, which is carved by a visual hull. Through our experiments, it turns out that more than 95\% of the volume grids are discarded by the visual hull culling, making the sparse 3D CNN efficient. At the fine level, the network focuses on a narrow band near the zero-level set of the initial TSDF and discretizes the narrow band with smaller voxels. By employing this narrow-band culling, we further shrink the sampling space, resulting in a relatively small range of grid numbers (usually 300K--500K in our experiments) even with a high volume resolution of 512. The remaining voxels in the narrow band are associated with features that fuse high-frequency information from the computed normal maps upon the low-frequency shape from the coarse level to compute the TSDF at high resolution. The final mesh is then extracted from the TSDF using the Marching-Cube algorithm ~\cite{Lorensen87marchingcubes}.
% Different from the u-net sturcture to preserve global topology context, we then apply a shallow 3dcnn to compute the final TSDF $D_{final}$ which contain more local geometry detail.




% \ping{this paragraph can be expanded. It is an important contribution and often ignored by other works. stress on the novel idea of regressing blending weights instead of colors}

In addition to geometry, high-quality mesh texture is also a crucial factor contributing to visual appearance. Directly computing a color field in 3D space, as in \cite{iccv2020PIFu}, struggles to capture high-frequency texture details, while the neural radiance field (NeRF) \cite{yu2020pixelnerf} or the DoubleField~\cite{shao2022doublefield} require expensive per-instance optimization and are often unstable for sparse input images. In contrast, we adopt an image-based rendering approach to compute a texture atlas map, which is efficient and widely supported in existing computer graphics tools. 
Specifically, we compute a blending weight at each 3D point on the mesh surface to determine its color as a weighted average of the colors at its image projections. The blending weights can be computed at a relatively coarse resolution, e.g., 512 volume resolution in our case, and leave texture details to the high-resolution images, such as 1K or 2K. Unlike previous methods that generate blurry texturing results under sparse input, our method generalizes well on both synthetic and real data with just a few input views. 
Figure~\ref{fig:teaser} shows two examples reconstructed by our method. Despite the challenging garment, pose, and occlusion, our method recovers faithful shape, normal, and texture on the right.

%with a wide variety of poses and clothing styles, and it is also adaptive to handle input image with arbitrary resolutions.
%\sicong{For this concern we claim that when the resolution of dicretized volume meets certain threshold (which is 256 in our experiment), the quantization error can be neglected.} 



In summary, the main contributions of this paper are as follows:
\begin{itemize}
\vspace{-0.1in}
  \item 
  We revisit the 3D volumetric representation and demonstrate that it can support clothed human reconstruction with equal or even better performance compared to implicit representation. 
  \item 
  We develop a memory and computation-efficient method for high-resolution volumetric reconstruction using sophisticated sparse 3D CNN, coarse-to-fine estimation, and voxel culling by visual hull and narrow bands. 
  \item 
  We introduce a novel method to compute a texture atlas map, which captures rich appearance details from high-resolution input images.
  \item 
  We achieve impressive results on standard benchmark datasets Twindom and MultiHuman, significantly reducing the point-2-surface (P2S) precision to approximately 0.2cm from just six input views, with more than $50\%$ error reduction compared to the state-of-the-art methods, including DoubleField~\cite{shao2022doublefield} and PIFuHD~\cite{saito2020pifuhd}.
\end{itemize}
%\section{LCA overview and Method}
\label{sec:background}

 In this work, we follow a process-based LCA due to its consistency and ability to provide provides a more detailed and specific understanding of the life cycle. Other than process-based LCA, economic input-output LCA is the most common approach and a valuable tool for determining the impact of economic activities across sectors~\cite{EIO_LCA}. LCA considers the entire life cycle of a product, from raw material extraction to end-of-life disposal, in order to identify and address environmental impacts. One of the key aspects of LCA is defining the system boundaries, which determine which stages of the life cycle are included in the assessment. These boundaries can be narrow, focusing on specific parts of the product's life cycle, or broad, encompassing the entire supply chain. 

% Figure environment removed

Figure~\ref{fig:LCA_boundry} illustrates the boundaries and extent of our study. We present a thorough analysis of the inventory and energy, encompassing the manufacturing, assembly, and use phases of both CMOS and AQFP integrated circuits. The accounting of upstream inputs (e.g., raw materials extraction and transportation) has not been considered due to the high degree of uncertainty involved. Our investigation provides a detailed account of the life-cycle inventory and energy analysis which includes the following aspects: functional unit selection, the wafer manufacturing and fabrication phase, wafer cutting and yield analysis, assembly phase, and use phase, which will be discussed in the rest of this section. 

A. Functional unit selection: The life-cycle impacts of CMOS/AQFP circuits can be significantly influenced by the selection of the functional unit. To perform an ``apple-to-apple'' comparison and ensure the same level of functionality, we choose the single-core 32-bit CMOS RISC-V architecture for both CMOS and AQFP technologies as the functional unit. The AQFP RISC-V processor includes crucial components such as a decoder, 32-bit ALU, register file, controller, and L1 AQFP cache memory~\cite{AQFP_memory}. The total area of the AQFP RISC-V processor is obtained by the summation over component areas. The synthesis of the AQFP component circuits is based on the MIT-LL process~\cite{MITLL_process}.
Based on our synthesis, the clock frequency, power, and area for the AQFP processor are 5 GHz, \SI{41}{\micro\watt}, and 3.5~cm$^2$, respectively. Note that to ensure a fair evaluation, we compare today's RISC-V AQFP processors built out of micron-size AQFP devices with a 130 nm CMOS processor that has its fabrication process steps provided in~\cite{fabrication_steps}. Moreover, note that for state-of-the-art technology nodes (e.g., 7nm), the full process steps details are not available. We used Western Digital SweRV EH1 RISC-V processor features as a sample and used scaling equations provided in~\cite{scalling_formulas} to obtain the equivalent 130 nm chip features. For CMOS, the clock frequency, power, and area are 1 GHz, \SI{7.5}{\watt}, and 12.1~mm$^2$, respectively. 


B. The manufacturing processes for AQFP and CMOS circuits share some similarities, but there are also notable differences in materials and specific steps. AQFP technology typically employs simpler steps than those used in semiconductor fabrication plants. For CMOS, a complete inventory of fabrication steps and their associated energy requirements for a 300 mm wafer can be found in Reference~\cite{fabrication_steps}, where a total of 206 process steps are listed in Tables S3 to S17 of the supplementary materials. To evaluate the fabrication process for AQFP technology, we carefully reviewed the process presented in~\cite{Japan_process_steps}, and estimated a total of 216 steps and their corresponding energy values. The total energy required for each wafer is obtained by summing the energy values of all steps.   
 

C. Wafer cutting: After determining the energy required for fabricating a wafer calculate the fabrication yield. The yield for CMOS wafers is found (using Murphy’s model) to be 97.6\%, while the yield for AQFP wafers was estimated to be 85.2\%. Using yield, we calculate the number of functional dies for each technology. The energy for the fabrication of each die is calculated using the formula: Manufacturing Energy of Wafer / Number of Functional Dies. 

D. Assembly phase: In our analysis, we consider the energy required for the packaging and assembly of each die. We consider commonly used plastic packages. The energy usage in the packaging stage is 0.34kWh per cm\textsuperscript{2} of silicon~\cite{packaging_energy}.

E. Use phase: We assume the usage as the server for both technologies. AQFP and CMOS circuit lifetimes are considered to be 10 and 5 years, respectively. Large computing systems
require cooling. This is particularly important for superconducting electronics which operate at temperatures below 10 K. We consider the cooling energy cost of the superconducting circuit to be 400X larger than the energy generated by the circuit at the cryogenic temperature~\cite{400X_cooling}.    




% !TEX program = pdflatex
% !TEX root = main.tex


\section{The Model}

We represent a series of interactions between $N$ individuals as a sequence of weighted directed networks with adjacency matrix $A^t$ for $t=0,1,2,\ldots,T$. For each $t$, its entry $A_{ij}^t$ is the outcome of interactions $i \rightarrow j$ suggesting that $i$ is ranked above $j$. This allows both cardinal and ordinal inputs. For instance, in team sports, $A_{ij}^t$ could be the number of points by which team $i$ beat team $j$, or we could simply set $A_{ij}^t=1$ to indicate that $i$ won and $j$ lost. We can include the case where individuals interact multiple times at time $t$ by summing the corresponding entries.

We assume that the values of $A_{ij}^t$ are influenced by a vector of real-valued ranks $\v{s}^t=(s_{1}^t,\dots, s_{N}^t)$, where $s_i^t$ is $i$'s skill, strength or prestige at time $t$.
To model these interactions, we follow SpringRank's approach of imagining the network as a physical system~\cite{de2018physical}. Specifically, each node $i$ is embedded in $\mathbb{R}$ at position $s_i^t$, and each directed edge $i \rightarrow j$ becomes an oriented spring with a non-zero resting length and displacement $s_i^t-s_j^t$. Since we are free to rescale latent space and the energy scale, we set the spring constant and resting length to $1$. The spring corresponding to an edge $i \rightarrow j$ at time $t$ then has energy
\be\label{eqn:staticH}
H_{ij}(s_i^t,s_j^t)=\f{1}{2} \bup{s_i^t-s_j^t-1}^{2} \, .
\ee
If there were no other effects, the total energy of the system at time $t$ would then be 
\be\label{eqn:totalstaticH}
H^t(\v{s}^t) = \sum_{i,j=1}^{N} A_{ij}^t \,H_{ij}(s_i^t,s_j^t) \, .
\ee
If we determined $\v{s}^t$ by minimizing $H^t$ for each $t$ separately, we would simply be applying the static SpringRank model separately to each ``snapshot'' of the network. This would ignore all previous (and future) interactions, and ignore the hypothesis that ranks change smoothly from one time-step to the next.

% Figure environment removed

To model this smoothness, we also assume a dependence between ranks at successive time-steps. Specifically, we extend the Hamiltonian~\eqref{eqn:totalstaticH} with an extra term that models the \emph{self-interaction} between past and current ranks,
\begin{equation}\label{eqn:selfH}
\Hself^t(\v{s}^t,\v{s}^{t-1}) 
= \frac{\kself}{2} \sum_{i=1}^N (s_i^t-s_i^{t-1})^2 \, .
\end{equation}
This can be seen as a set of additional ``self-springs'' that connect the rank of each individual with its own previous rank. The spring constant $\kself$ parametrizes how smoothly we want the ranks to change from one step to the next. In inference terms, $\kself$ is a hyperparameter which we tune using cross-validation.

Summing over all time-steps $0 < t \le T$ and adding this to the pairwise interactions at each time-step then gives a total energy

\begin{align}\label{eqn:fullH}
\Htotal(\{\v{s}^t\}) = \sum_{t=0}^T H^t(\v{s}^t) + \sum_{t=1}^T \Hself^t(\v{s}^t,\v{s}^{t-1}) \, .
\end{align}
We call this the dynamical SpringRank Hamiltonian. The optimal ranks $\v{s}^0,\v{s}^1,\ldots,\v{s}^T$ are those that minimize it.


There are two ways to minimize $\Htotal$. One is to proceed in an online way, moving forward in time. In this approach, we use the static SpringRank model Eq.~\eqref{eqn:totalstaticH} to find the initial ranks $\v{s}^0$ by minimizing $H^0(\v{s}^0)$. As in Ref.~\cite{de2018physical}, the energy is unchanged if we add a constant to all the ranks; we can break this translational symmetry by setting the mean initial rank $(1/N) \sum_{i=1}^N v_i^0$ to zero.
Then, at each subsequent time-step $t \ge 1$, we update the ranks by taking into account both the new pairwise interactions and the self-springs connecting the ranks with their previous values. Namely, given $\v{s}^{t-1}$ and $A^t$, we find the ranks $\v{s}^t$ that minimize $H^t(\v{s}^t) + \Hself^t(\v{s}^t,\v{s}^{t-1})$.

Since this is a convex function of $\v{s}^t$, we can find its minimum by setting its gradient to zero, or equivalently by balancing all the forces $v_i^t$. This yields a system of linear equations:
\begin{align}\label{eqn:fullsolution}
\rup{ D^{out,t}+D^{in,t}- \bup{A^t + (A^t)^\dagger}+\kself\id} \,\v{s}^t
&=\rup{D^{out,t}-D^{in,t}}\v{1} \nonumber \\& +\kself\, \v{s}^{t-1} \, . 
\end{align}

Here 
$D^{out,t}$ and $D^{in,t}$ are diagonal matrices whose entries are the weighted out- and in-degrees $D^{out,t}_{ii}=\sum_{j}A^t_{ij}$ and $D^{in,t}_{ii}=\sum_{j}A^t_{ji}$; 
$\dagger$ denotes the transpose; 
$\id$ is the identity matrix; 
and $\v{1}$ is the all-ones vector.

The matrix on the left side of~\Cref{eqn:fullsolution} is invertible if $\kself > 0$. In particular, its eigenvector $\v{1}$ has eigenvalue $N \kself$. Thus for each $A^t$ and each $\v{s}^{t-1}$, Eq.~\eqref{eqn:fullsolution} has a unique solution $\v{s}^t$. Overall, Eq.~\eqref{eqn:fullsolution} is similar to the regularized version of SpringRank~\cite{de2018physical} with regularization parameter $\alpha= \kself$. However, unlike the static model, there is a term on the right-hand side containing the previous ranks $\v{s}^{t-1}$, creating a Markovian dependence between successive time-steps. We refer to this model as \dsrfull\ (\dsr).

Importantly the online DSR approach does not actually minimize $\Htotal$, instead solving a sequence of minimization problems, one for each time step. To minimize $\Htotal$ instead, we set $\nabla \Htotal(\v{s}^t) = 0$, solving for the minimizers $\v{s}^t$ over all $N(T+1)$ ranks simultaneously, yielding the following system of equations (SI \Cref{sec:h_total_derive}):

\begin{align}\label{eqn:h_total}
\rup{ D^{out,t}+D^{in,t} - \bup{A^t+(A^t)^\dagger} + 2\kself\id}\,\v{s}^t 
&=\rup{D^{out,t}-D^{in,t}}\v{1} \nonumber\\ 
& +\kself \,\bup{\v{s}^{t-1} + \v{s}^{t+1}} \, . 
\end{align}
This differs from \Cref{eqn:fullH} in that the right-hand side now includes both past and future ranks (which doubles the contribution of $\kself$ on the left). We remove the terms $\v{s}^{t-1}$ and $\v{s}^{t+1}$ for $t=0$ and $t=T$ respectively. This entire system has translational symmetry, since the energy Eq.~\eqref{eqn:fullH} remains the same if we add the same constant to all ranks at all times, but we can again break this symmetry by setting the mean rank to zero.

Additionally, in contrast to \Cref{eqn:fullsolution}, the ranks at $t$ now depend on both $t-1$ and $t+1$, which themselves depend on ranks at adjacent time-steps, so that ranks are affected by interactions in both the past and the future. In computer science, methods like this where the entire history is provided to the algorithm are called \emph{offline}, to distinguish them from \emph{online} approaches that update their results in real time as data becomes available. Thus we refer to this model as \nmdsrfull\ (\nmdsr).  

The cost of solving \Cref{eqn:fullsolution} for a single time-step is the same as static SpringRank with only one additional parameter to be tuned using cross-validation, and there are $T$ such $N$-dimensional equations to be solved successively. On the other hand, \Cref{eqn:h_total} requires solving a single  system of dimension $NT$, whose operator consists of $T$ blocks, each of dimension $N\times N$. While these two approaches feature numbers of non-zero entries that are fundamentally determined by the number of total edges across all time steps, the cost of solving \dsr vs \nmdsr will depend on the particular choice of linear solver~\cite{peng2021solving}.

Philosophically, Eqns.~\eqref{eqn:fullsolution} and~\eqref{eqn:h_total} are trying to do two different things. If we are given all the data $A^0,A^1,\ldots,A^T$ and we want to infer retrospectively how each individual's rank changed over time, it makes sense to include both past and future interactions as in~\eqref{eqn:h_total} so that $s_i^t$ is affected by $i$'s entire history. 

In contrast, \eqref{eqn:fullsolution} can be viewed as modeling each individual's perceived rank at the time, based only on the interactions that have occurred so far.

In principle, one could envisage other ways to formally incorporate an explicit dependence on  $\v{s}^{t-1}$ into the model, and we provide one example in SI \Cref{sec:sidynl}. However, we found that the approaches presented in this Section provide a natural interpretation, result in good prediction performance on both real and synthetic datasets (see \Cref{sec:results}) and are computationally scalable. 

We close this section with two possible extensions to these models. First, in some settings we might have timestamps $t$ that are not successive integers $0,1,\ldots,T$. In this case, if the time interval between two successive times is $\Delta t$, one could scale the spring constant of the self-springs between time-steps as $\kself/\Delta t$. This corresponds to the fact that if we have $\Delta$ identical springs in series, each of which is stretched by $(s^t-s^{t-1})/\Delta$, their total energy is $(1/2)(\kself/\Delta)(s^t-s^{t-1})^2$. The same expression applies if the timestamps are real-valued so that $\Delta$ is not an integer.

Second, if we believe that not just the ranks themselves but their rates of change behave smoothly over time, one could add a momentum term to the Hamiltonian which is quadratic in the discrete second derivative of the ranks. Since
\begin{gather*}
\left( (s^{t+1}-s^t) - (s^t-s^{t-1}) \right)^2
= \left( s^{t+1} - 2 s^t + s^{t-1} \right)^2 \\
= 2 (s^t-s^{t-1})^2 + 2 (s^{t+1}-s^t)^2 - (s^{t+1} - s^{t-1})^2 \, ,
\end{gather*}
this is equivalent to adding a repulsive force, i.e., a spring with negative spring constant, between ranks two time-steps apart. Note that the system nevertheless remains convex: this momentum term is positive semidefinite, so adding it to~\eqref{eqn:fullH} keeps the coupling matrix positive definite except for translational symmetry. Of course, these terms are second-order in time. In the online approach, one would have to determine $\v{s}^0$ from the static model, $\v{s}^1$ from the first-order model~\eqref{eqn:fullsolution}, and then use the model including this momentum term for $\v{s}^t$ for $t \ge 2$. We have not pursued this here, but it may make sense for certain datasets.


\subsection{Moving-window SpringRank}\label{subsec:mwsr}

Before we test the various versions of \dsrfull\ defined above, we consider a simpler model as a baseline. 
The simplest way to extend SpringRank to a dynamical context is to apply the static model to the interactions in a series of ``windows,'' where in each window we sum the interactions over a series of consecutive time-steps. For instance, we can compute $\v{s}^t$ for each $t$ by applying the static model to a window of width $\tau$, i.e., replacing $A^t$ with $\sum_{t'=t}^{t+\tau-1} A^{t'}$. Since these windows overlap, the resulting estimates $\v{s}^t$ will be smooth to some extent, even without imposing an explicit dependence between $\v{s}^t$ and $\v{s}^{t-1}$. We use this method, which we call \mwsrfull\ (\mwsr), as a baseline to compare with the dynamical models presented above.

Roughly speaking, a larger $\tau$ is like a larger self-spring constant $\kself$, since it induces more overlap between windows and thus a stronger correlation between the inferred ranks. However, like a decaying-history approach, \mwsr\ assumes a particular kernel for the importance of past time-steps: namely, that all $t'$ in the window are equally important. In contrast, \dsrfull\ infers the importance of past time-steps by coupling $\v{s}^t$ with $\v{s}^{t-1}$.

However, both models have a free parameter that needs to be tuned, i.e., $\kself$ and $\tau$. A shorter window $\tau$ or smaller spring constant $\kself$ allows the ranks to respond quickly to new interactions, while a longer window or larger spring constant more tightly couples nearby estimates. This trade-off suggests the existence of an optimal window length $\tau_{\opt}$. We tune $\tau$ using a cross-validation procedure as explained in SI \Cref{sisec:tuning}.


\subsection{Generative Model and Synthetic Data}
\label{sec:genmod}

Analogous to a model presented in~\cite{de2018physical}, we propose a probabilistic generative model for dynamic data. It takes as input the ranks $\v{s}^t$ and generates a sequence of weighted directed networks with adjacency matrix $A^t$ at time $t$. One can also imagine models that generate the ranks, for instance with a random walk with Gaussian steps whose log-probability is the self-spring Hamiltonian~\eqref{eqn:selfH}, but we treat $\v{s}^t$ as an input since we want the user of this model to have control over how the ground-truth ranks vary with time.  For instance, in our experiments below we generate synthetic data where the ranks vary sinusoidally.

The generative model has two real-valued parameters: a signal-to-noise ratio or inverse temperature $\beta$, and an overall density of edges $c$. Given the ranks $\v{s}^t$, it generates weighted, directed edges between each pair of nodes $i,j$ independently, as follows. The probability $P_{ij}^t(\beta)$ of $i$ ``beating'' $j$ at time $t$, giving a directed edge $i \to j$, is a logistic function as in~\cite{de2018physical} or the Bradley-Terry-Luce model~\cite{bradley1952,luce1959}:
\bea
\nonumber P_{ij}^t(\beta)=\frac{1}{1+\e^{-2\beta(s_i^t-s_j^t)}} \, .
\eea
The number of such edges, which gives the integer weight $A_{ij}^t$, is then drawn from a Poisson distribution whose mean $\lambda_{ij}^t$ is $cP^t_{ij}\,(\beta)$: 
\be
\label{generative_poiss}
A^t_{ij} \sim \Poi\left(\lambda_{ij}^t=\frac{c}{1+\e^{-2\beta(s_i^t-s_j^t)}}\right).
\ee
Since $P_{ij}^t(\beta) + P_{ji}^t(\beta)=1$, for any pair $i,j$ the total number of interactions $A_{ij}^t + A_{ji}^t$ is Poisson-distributed with mean $c$. The rank differences $s_i^t-s_j^t$ are used only to choose the directions of these edges. This  is equivalent to a model where we define a random multigraph where the number of edges between $i$ and $j$ is $\Poi(c)$, and then we choose the direction of each edge independently according to $P_{ij}^t$.

This is different from the generative model proposed in the static case in~\cite{de2018physical}. In that model the probability that $i$ and $j$ interact depends on $s_i-s_j$ so that nodes are more likely to interact if their ranks are fairly close. This is consistent with SpringRank's assumption that if $i$ beats $j$ then $j$ is below $i$, but not too far below it (since the springs have resting length $1$). This assumption makes sense for some datasets but not for others. By generating synthetic data without this dependence, our intent is to pose a greater challenge to SpringRank by modeling (for example) round-robin tournaments where every team plays each other.

\subsection{Model Evaluation}
\label{sec:testing}

Assessing a ranking model on real datasets is not straightforward since we do not know the true values of the underlying ranks. Nevertheless, we may measure the extent to which inferred ranks are accurate in the sense that they can predict the outcome of new observations. 

There are several performance metrics that can be used for prediction evaluation. From coarse-grained measures capable of predicting the likely winner to more fine-grained measures that also estimate odds, we consider four main metrics in our experiments, detailed in \Cref{sisec:evaluation}. We measure prediction performance using a cross-validation protocol where datasets are divided into training and test sets. The training set is used for hyperparameter tuning and parameter estimation while performance is evaluated on the test set. In order to preserve the chronological ordering of the data, the test set contains future observations, i.e., observations that chronologically follow those used in training. Hyperparameters for each method are tuned using grid-search in order to maximize the performance metrics as described in SI \Cref{sisec:tuning}.





%%% Local Variables:
%%% mode: latex
%%% TeX-master: "main"
%%% End:


\NewDocumentCommand{\simplechoice}{E_{i} o}{ \{ \TypeChoice_{#1}.S_{#1} \}_{\lowercase{#1}\in \IfNoValueTF{#2}{\uppercase{#1}}{\uppercase{#2}}} }

% ~ past indicator 
% #1: * mm
% #2: t'
% #3: O
\NewDocumentCommand{\Past}{s t' O{\Const}}{%
   \IfBooleanT{#1}{$}% mm wrap
   \downarrow\!\FmtMathOp{#3}{#2}%
   \IfBooleanT{#1}{$}% mm wrap
}

% ~ message type def
% #1: * mm
% #2: O l
% #3: t'
% #4: e_
% #5: O T
% #6: t'
% #7: e_
\NewDocumentCommand{\DefMsgType}{s O{\MsgLabel} t' e_ O{\DataType} t' e_}{%
   \IfBooleanT{#1}{$}% mm wrap
   \IfNoValueTF{#4}{\FmtMathOp{#2}{#3}}{\FmtMathOp{#2}{#3}_{#4}}
   \left\langle
   \IfNoValueTF{#7}{\FmtMathOp{#5}{#6}}{\FmtMathOp{#5}{#6}_{#7}}
   \right\rangle
   \IfBooleanT{#1}{$}% mm wrap
}

% ~ rec def def
\NewDocumentCommand{\TypRecDef}{s O{\RecDef} O{\TypRecLabel} e: t. O{\TypeS}}{%
   \IfBooleanT{#1}{$}% mm wrap
   {#2}{{#3}\IfValueT{#4}{^{#4}}}%
   \IfBooleanTF{#5}{.}{.#6}% proceeding
   \IfBooleanT{#1}{$}% mm wrap
}

% ~ rec call
\NewDocumentCommand{\TypRecCall}{s O{\TypRecLabel} e:}{%
   \IfBooleanT{#1}{$}% mm wrap
   {{#2}\IfValueT{#3}{^{#3}}}%
   \IfBooleanT{#1}{$}% mm wrap
}

% ~ ~ ~ ~ ~ ~
% ~ templates

% ~
% #1 : s  : mm wrap
% #2 : E1 : rec env
% #3 : t| : past const
% #4 : E2 : const 
% #5 : o  : index
% #6 : Oi : index set
% #7 : E3 : TypInteract
\NewDocumentCommand{\TypJudge}{s E1{{\RecEnv}} t| E2{{\Const}} o O{i} E3{{\TypInteract}}}{%
   \IfBooleanT{#1}{$}% mm wrap
   {#2};\,
   % * if of set?
   \IfNoValueTF{#5}{% * true
      % * if past?
      \IfBooleanTF{#3}{\Past[#4]}{#4}
      \;\Entails\,
      {#7}
   }{% * false
      % * if past?
      \IfBooleanTF{#3}{% * true
         \Past[{\SetOr_{\hspace{-0.5ex}\vspace{0.5ex}\scriptstyle{\AsSet[#5][#6]}}}{#4_{#5}}]
      }{% * false
         {\SetOr_{\hspace{-0.5ex}\vspace{0.5ex}\scriptstyle{\AsSet[#5][#6]}}}{#4_{#5}}
      }
      \;\Entails\,
      {#7[#5][#6]}
   }
   \IfBooleanT{#1}{$}% mm wrap
}

% ~
\NewDocumentCommand{\TypCond}{s t' e_ E1{\Const} E2{\RSet}}{%
   \IfBooleanT{#1}{$}% mm wrap
   \mkTup[\FmtMathOp{#4}{#2}_{#3}][\FmtMathOp{#5}{#2}_{#3}]
   \IfBooleanT{#1}{$}% mm wrap
}

% ~
\NewDocumentCommand{\TypAct}{s t' e_ E1{\TypComm} E2{\MsgType}}{%
   \IfBooleanT{#1}{$}% mm wrap
   \FmtMathOp{#4}{#2}_{#3}\FmtMathOp{\MsgLabel}{#2}_{#3}\left\langle\FmtMathOp{\DataType}{#2}_{#3}\right\rangle
   \IfBooleanT{#1}{$}% mm wrap
}

% ~
% #1 : mm wrap
% #2 : t| : single element?
% #3 : E1 : TypeAct args
% #4 : E2 : TypCond args
% #5 : E3 : TypeS
% #6 : t_ : single element subscript
% #7 : Oi : element subscript
% #8 : Oi : set of elements
% #9 : t~ : dual?
\NewDocumentCommand{\TypInteract}{s t| E1{{}} E2{{}} E3{{\TypeS}} t_ O{i} O{i} t~}{%
\IfBooleanT{#1}{$}% mm wrap
\IfBooleanTF{#2}{% * single element
\IfBooleanTF{#6}{% * subscripted
\IfBooleanTF{#9}{% * dual
\Dual[{\TypAct_{#7}#3}{\TypCond_{#7}#4}\,.{#5_{#7}}]%
}{% * not dual
{\TypAct_{#7}#3}{\TypCond_{#7}#4}\,.{#5_{#7}}%
}%
}{% * not subscripted
\IfBooleanTF{#9}{% * dual
   \Dual[{\TypAct#3}{\TypCond#4}\,.#5]%
}{% * not dual
   {\TypAct#3}{\TypCond#4}\,.#5%
}%
}%
}{% * set
\IfBooleanTF{#9}{% * dual
\InSet[\Dual[{\TypAct_{#7}#3}{\TypCond_{#7}#4}\,.{#5_{#7}}]][#7][#8]%
}{% * not dual
\InSet[{\TypAct_{#7}#3}{\TypCond_{#7}#4}\,.{#5_{#7}}][#7][#8]%
}%
}%
\IfBooleanT{#1}{$}% mm wrap
}

% ~ delegation
% #1 : mm wrap
% #2 : O Const
% #3 : t'
% #4 : e_
% #5 : O TypeS
% #6 : t'
% #7 : e_
\NewDocumentCommand{\TmpTypBTDelegate}{s O{\Const} t' e_ O{\TypeS} t' e_}{%
   \IfBooleanT{#1}{$}% mm wrap
   \mkTup[%
      \IfNoValueTF{#4}{\FmtMathOp{#2}{#3}}{\FmtMathOp{#2}{#3}_{#4}}
   ][%
      \IfNoValueTF{#7}{\FmtMathOp{#5}{#6}}{\FmtMathOp{#5}{#6}_{#7}}
   ]\,
   \IfBooleanT{#1}{$}% mm wrap
}

% ~ future satisfiability
% #1 : mm wrap
% #2 : O typeenv
% #3 : t'
% #4 : e_
% #5 : O resets
\NewDocumentCommand{\TmpTypFutureSat}{s E1{\TypEnv} t' e_ E2{{1{\Const}2{\RSet}}}}{%
\IfBooleanT{#1}{$}% mm wrap
{\Resets#5}%
\subseteq%
\IfNoValueTF{#4}{\FmtMathOp{#2}{#3}}{\FmtMathOp{#2}{#3}_{#4}}%
\IfBooleanT{#1}{$}% mm wrap
}


% ~ ~ ~ ~ ~ ~
% ~ generators
% #1 : mm wrap
% #2 : O typeS
% #3 : O const
% #4 : O tnv
% \NewDocumentCommand{\mkTypJudgement}{s O{\TypeS} O{\Const} O{\RecEnv}}{%
% 	\IfBooleanT{#1}{$}% mm wrap
% 	{#4};\,{#3}\,\Entails\,{#2}
% 	\IfBooleanT{#1}{$}% mm wrap
% }



\section{Equilibrium for Each Stablecoin Type}
\label{sec:eq}

In this section, we will analyze the equilibria and price instability for each type of stablecoin. 
To focus on the equilibria changing with $v$ and $v^\prime$ (i.e., equilibria varied by a stablecoin design), we consider that users have little incentive to keep holding coins (i.e., $i(x)\approx x$).
%In other words, we do not consider the case where users expect the stablecoin price can be $1$ due to an economic uptrend even without a proper value of $v$ and $v^\prime.$
We also assume that values of the future status variables including the future fundamental state $\theta^\prime$, which $v^\prime$ depends on in stablecoins using cryptocurrencies as reserves, can approximate values of the current status variables such as the current fundamental state $\theta$ to simplify the equilibrium analysis by reducing the dimension of parameters. If we consider the future fundamental state as a variable $\theta^\prime$ independent of $\theta$, our results would be extended to two dimensions of fundamental states $[\theta, \theta^\prime]$.

We first present Figure~\ref{fig:range} that illustrates the result visually, which makes it possible to compare stablecoin designs intuitively.
In the figure, $\theta_{\max}$ and $\theta_{\min}$ indicate the maximum and minimum values of $\theta$, respectively. 
The blue bar represents a range of $\theta$ in which a unique pegging equilibrium exists so the peg is guaranteed. 
The yellow range of $\theta$ has multiple equilibria, including the pegging state of $p(M)=1$, which implies that the peg state is not guaranteed even though it can be reached. 
More specifically, in this range, the pegging state is a self-fulfilling equilibrium; users' belief totally determines the destiny of stablecoins.
Finally, in the red range of $\theta$, there is only a depegging equilibrium, which implies that the peg cannot be achieved. 
Therefore, the wider the blue range, the better the stablecoin design. 

A value of $\overline{\theta}$ indicates a lower bound for having a unique pegging equilibrium, and $\underline{\theta}$ is an upper bound for having only a depegging equilibrium. 
Therefore, according to Figure~\ref{fig:range}, stablecoins fully backed by fiat assets have $\overline{\theta}$ as $\theta_{\min}$. Stablecoins partially backed by fiat assets do not have a value of $\overline{\theta}$ and $\underline{\theta}$. 
Furthermore, an over-collateralized stablecoin does not have $\overline{\theta}$. 
In crypto-collateralized, algorithmic, and over-collateralized stablecoins, the values of $\overline{\theta}$ and $\underline{\theta}$ would depend on $c$. 
Specifically, if the system employs a more robust crypto asset in its pegging mechanism, the blue zone can widen by decreasing $\overline{\theta}$.
On a side note, a significant incentive function $i$ can also help expand the blue zone and narrow the yellow and red zones.

We now describe below the equilibrium state of stablecoin systems in detail.

% Figure environment removed

\subsection{Fiat-collateralized stablecoins}

As described in Section~\ref{sec:model}, fiat-collateralized stablecoins have $v$ as $1$ if $Q\leq V^f,$ otherwise $V^f/Q.$ 
The following theorem presents equilibria of fiat-collateralized stablecoins. 

\begin{theorem}
Fully backed fiat-collateralized stablecoins have a unique pegging equilibrium for any $\theta$. On the other hand, for any $\theta$, partially backed fiat-collateralized stablecoins have multiple equilibria including the pegging state due to users' self-fulfilling beliefs.
\label{thm:fiat}
\end{theorem}

According to Theorem~\ref{thm:fiat}, fully backed fiat-collateralized stablecoins can guarantee the peg. 
Meanwhile, a system partially backing its stablecoin has multiple equilibria, so it is difficult to predict the consequence.
The system can reach the pegging state, but it is not guaranteed.
In particular, it suffers from a self-fulfilling belief of users; 
if users believe only a few stablecoin holders will redeem their coins, the users with that expectation would not need to redeem their coins to the system right now, which fulfills the expectation by themselves and, in turn, results in price stabilization. 
On the other hand, let us assume that users believe that too many holders will redeem coins so that the reserves of the system cannot cover it. 
Then users should immediately redeem their coins to the system, which realizes the expectation by themselves and puts the stablecoin in danger by bringing about the depreciation of the stablecoin.
We present the proof of Theorem~\ref{thm:fiat} in Appendix~\ref{app:proof2}. 

\subsection{Crypto-collateralized \& algorithmic stablecoins}

Next, we look at crypto-collateralized and algorithmic stablecoins.
Crypto-collateralized stablecoins have a value of $v$ as $r^c\left(Q, \theta\right)$ if $Q\leq V^c(\theta)$, otherwise $r^c\left(Q, \theta\right)\cdot V^c(\theta)/Q.$ 
In algorithmic stablecoins, $v$ is $r^c(Q, \theta)$.
We present their equilibria in Theorem~\ref{thm:crypto}. 

\begin{theorem}
We denote the total market supply of stablecoins by $T^s.$
Then, in crypto-collateralized and algorithmic stablecoins, $\overline{\theta}$ is a value such that $r^c\left(T^s,\overline{\theta}\right)=1$, and $\underline{\theta}$ is a value such that $r^c\left(0,\underline{\theta}\right)=1.$
Here, for crypto-collateralized stablecoins, we assume that $V^c(\underline{\theta})\geq T^s$.
Then both crypto-collateralized and algorithmic stablecoins have a unique pegging equilibrium for any $\theta\geq\overline{\theta}$, multiple equilibria including the pegging state for any $\theta$ in the range $[\underline{\theta},\overline{\theta})$, and depegging equilibria for any $\theta<\underline{\theta}$. 
\label{thm:crypto}
\end{theorem}

%\noindent\textbf{Intuition behind Theorem~\ref{thm:crypto}:} 
Crypto-collateralized and algorithmic stablecoins can guarantee the peg under good economic conditions. 
However, under poor economic conditions, even if the reserves are sufficient to back the stablecoins fully, they would not be able to reach the pegging state because users who redeem their coins cannot, in effect, receive 1 due to a downward price fluctuation of the cryptocurrency, the payment medium.
In the mediocre economic status, there are multiple equilibria including the pegging state; the successful peg is up to users' belief in others' redemption actions because the stablecoin redemption affects whether a cryptocurrency price can be less than 1 in that economic condition. 

The values of $\overline{\theta}$ and $\underline{\theta}$ depend on $r^c$ according to Theorem~\ref{thm:crypto}; $\overline{\theta}$ and $\underline{\theta}$ are the values such that $r^c\left(T^s,\overline{\theta}\right)=1$ and $r^c\left(0,\underline{\theta}\right)=1$, respectively. Note that $\overline{\theta}$ is greater than $\underline{\theta}$ because $r^c$ decreases and increases as the first and second inputs increase, respectively. 
The more robust cryptocurrency the systems use to back stablecoins, the gentler the slope of $r^c$. 
That is, for a more robust cryptocurrency $c$, $r^c$ can be greater, so $\overline{\theta}$ and $\underline{\theta}$ become lower. 
For example, crypto-collateralized stablecoins can use Bitcoin, Ethereum, or even other stablecoins as their collateral to widen the blue zone.
Meanwhile, algorithmic stablecoins should use endogenous cryptocurrencies according to their protocol, which would have a narrower blue zone and a wider red zone. 
We present the proof of the theorem in Appendix~\ref{app:proof3}.

\subsection{Over-collateralized stablecoins}

Finally, we consider over-collateralized stablecoins.
The stablecoin has $v$ for user $u_i$ as follows: $r^c\left(Q, \theta\right)\cdot o(\theta)$ if $0< D^L(\theta)$ or $0< D_{u_i}(\theta)$, otherwise zero. 
Here, $o(\theta)$ is greater than $1$ as long as $\theta$ is not too low. 
Theorem~\ref{thm:over} presents equilibria of over-collateralized stablecoins. 

\begin{theorem}
We assume that $r^c(0,\theta)\cdot o(\theta)<1$ when $D^L(\theta)=T^s$.
Then over-collateralized stablecoins have multiple equilibria including the pegging state for any $\theta\geq\underline{\theta}$, and depegging equilibria for any $\theta<\underline{\theta}$, where $\underline{\theta}$ satisfies $r^c(0,\underline{\theta})\cdot o(\underline{\theta})=1$. Moreover, $\underline{\theta}$ of over-collateralized stablecoins is smaller than that for crypto-collateralized and algorithmic stablecoins.
\label{thm:over}
\end{theorem}

In over-collateralized stablecoins, stablecoin debtors can redeem their coins anytime, while non-debtors cannot unless a liquidation process starts for some collateral. 
Therefore, users' belief regarding redemption by debtors plays an important role in maintaining the peg; if users believe that many stablecoin debtors will redeem their coins, debtors with this expectation will redeem their coins now to settle their debt cheaper because they expect an increase in the stablecoin price due to other debtors' redemption. This leads to a rise in a stablecoin price by fulfilling their expectation by themselves. 
On the contrary, if users believe that only a few debtors will redeem their coins, debtors with the expectation do not need to redeem their coins immediately because they do not think the coin price will increase. 
Therefore, the stablecoin price will not increase by vindicating their decision.

On the other hand, if the collateral value is less than 1 due to severely bad economic conditions, the system would not be able to attain the pegging state. 

Moreover, where a system pays users more than 1 based on its price oracle, $v$ can be not less than 1 even with a price drop of cryptocurrencies. This makes $\underline{\theta}$ of over-collateralized stablecoins smaller than that for crypto-collateralized and algorithmic stablecoins. 
%
We present the proof of Theorem~\ref{thm:over} in Appendix~\ref{app:proof4}. 
\lstMakeShortInline[columns=fixed]@
% Figure environment removed
\lstDeleteShortInline@

In this section, we describe how we collect examples for learning repair strategies without any version-controlled data. Specifically, we first detect \safeprogs and corresponding witnesses using \sawitnessfull (witnesses are sanitizers and guards that protect from vulnerabilities)  in Section~\ref{subsec:sa-witness}. Using these witness annotations, we generate unsafe programs and \textit{edits} from the \safeprog using a \textbf{witness-removal} step (Section ~\ref{subsec:witness-removal}). In the following, we define terminology for the \astree  data-structure we operate on. 


\astree refers to the abstract syntax tree representation of programs, augmented with data flow edges and annotations for sources, sinks, sanitizers, guards, witnesses etc. 
An \astree is a five-tuple 
$\langle \mathcal{N},\mathcal{V},\mathcal{T},\mathcal{E}, \mathcal{A} \rangle$, where:
\begin{enumerate}
\item
$\mathcal{N}=\{\mathit{id}_0,\ldots\mathit{id}_n\}$  is a set of nodes, where  $\mathit{id_i}\in\mathbb{N}$ for 
$ 0 \leq i \leq n$.
\item
$\mathcal{V}$ is a map from nodes to program snippets
represented as strings. For a node $n$, we have that $\mathcal{V}(n)$ is a string representing the code snippet associated with $n$
\item
$\mathcal{T}$ is a map from nodes to their types defined by 
 \sa~\cite{codeqlast}. For example, \callexpr is the type of a node representing a function call, \indexexpr is the type of a node representing an array index, and \blockstmt is the type of a node representing a basic block of statements.
\item
$\mathcal{E}$ is a set of directed edges.
Each edge is of the form $(n_1,n_2,\edgetype,z)$, where
$n_1$ is a source node, $n_2$ is a target node, 
$\edgetype \in \{\T{SynParent}, \T{SynChild}, \T{SemParent},
\T{SemChild} \}$ denotes the relationship from 
$n_1$ to $n_2$, as one of syntactic parent, syntactic child, semantic parent or semantic child,
and $z\in\mathbb{Z}$ is the index of $n_2$ among $n_1's$ children if this edge is a child edge, and $-1$ if the edge is a parent edge. 
\item
$\mathcal{A}$ is a set of annotations associated with each node. The annotations are from the set $\{\T{source},
\T{sink},\T{sanitizer},\T{guard}$,\T{witness}\}. We also refer to annotations using predicates or relations. For instance, for a node $n$, if an annotation  $\T{source}$ is present, we say that
the predicate $\T{source}(n)$ is true.
\end{enumerate}

%\setlength{\grammarindent}{5em} % increase separation between LHS/RHS

% Figure environment removed



A {\em traversal} or a {\em path} in an \astree is a sequence of edges $e_0,\ldots,e_{i-1},e_i,\ldots ,e_k$ such that the target node of $e_{i-1}$ is also the source node of $e_i$, for all $i\in\{1,\ldots,k\}$. That is, $e_{i-1}$ is of the form $(\_,n,\_,\_)$ and $e_i$ is of the form $(n,\_,\_,\_,\_)$. The source node of $e_0$ is the source of this path and the target node of $e_k$ is the target of the path.


\lstMakeShortInline[columns=fixed]@
%Note that these additional edges can capture long-range dependencies in programs. E.g. edge 4 in Figure ~\ref{fig:unsafememberex} links two nodes across the function boundaries. 
Figure~\ref{fig:example1-pdg} depicts a partial \pdg corresponding to the unsafe program in Figure~\ref{fig:unsafememberex}. Each oval corresponds to an \astree-node containing a type $\tau$ and an associated value. The dark edges denote the syntactic child edges. For example, the oval with value @foo(data)@ is an \astree-node with type \callexpr and has two children -- @foo@ and @data@, both with the type \varexpr. 
%Similarly, the \blockstmt node on the top refers to the function body between Line~\ref{lst:line:handlers-run} and Line~\ref{lst:line:handlers-run-end} in Figure ~\ref{fig:unsafememberex}. As the body of a function block can contain a variable number of children, we link to @handlers[callerId](data);@ as the k-th child of the \blockstmt. 
The semantic child edges are at the bottom in cyan. These edges correspond to the ones depicted in cyan in Figure ~\ref{fig:unsafememberex}. 
\lstDeleteShortInline@

%TODO:FIX THIS

%With this simplification, 
If $\prog$ is an \pdg then
we use  $\prog.\mathtt{source}$ to denote the source node, $\prog.\mathtt{sink}$ to denote the sink node, and $\prog.\mathtt{witness}$ to denote the witness node.
If the program has several sources, sinks and sanitizers then we generate a separate \pdg for each $(\mathtt{source},\mathtt{witness},\mathtt{sink})$ triple.
For a node $n$, its syntactic parent is $n.\mathtt{parent}$, syntactic children are $n.\mathtt{children}$, semantic parent is $n.\mathtt{semparent}$, and semantic children are $n.\mathtt{semchildren}$.

%\input{ql.tex}

\subsection{Static Analysis Witnessing}
\label{subsec:sa-witness}

\newcommand{\DMethodjudge}[1]{\texttt{#1(}\checknextarga}

% Figure environment removed

%\naman{TODO - sell this more as technique to work with any \sa tool ; our master query is a general framework implemented in \codeql that can work for any vulnerability -- easily extendable to other languages }
In this section, we show how to repurpose \sa tools to generate witnesses.
\sa tools perform dataflow analysis to check for rule-violations in programs. They use pattern matching to identify known sources, sinks, sanitizers, and guards. For commercial tools, these patterns are implemented (and continuously updated) manually by developers and encode this domain knowledge. Next, 
%these patterns are used to detect sources, sinks, sanitizers, and guards in programs and
\sa checks if there exists a flow between a source and a sink that does not cross a sanitizer or guard. We capture this formally in Figure~\ref{fig:judgements} (top two rules), and explain the notation used in it below.

\sa tools encode domain knowledge about the vulnerability by annotating nodes as \T{Source}, \T{Sink}, \T{Sanitizer}, and \T{Guard}. %These relations operate on the set of dataflow nodes in the programs.
So \DMethod{Source}{\I{n}}\ is true iff the node \I{n} is a \textit{source} node for a vulnerability. Next, \sa tools perform dataflow analysis by defining the relation \DMethod{SemChild}{$n_1$}{$n_2$}\ which is true iff there is a \taintpropedge between $n_1$ and $n_2$. Then the \DMethod{Vulnerability}{$n_1$}{$n_2$}\ relation can be defined as:
\begin{enumerate}
    \item $n_1$ and $n_2$ are source and sink nodes (\DMethod{Source}{$n_1$}\ and \DMethod{Sink}{$n_2$}\ are true)
    \item There exists a \textit{path} between $n_1$ and $n_2$ which is free of sanitizers or guards (\DMethod{SanGuardFree*}{$n_1$}{$n_2$}\ is true). A path is free of sanitizers and guards iff every \textit{edge} in the \textit{path} is free of sanitizers and guards. An edge between $n_1$ and $n_2$ is considered free of sanitizers and guards (\DMethod{SanGuardFree}{$n_1$}{$n_2$}\ is true) iff $(n_1, n_2, \_, \T{SemChild}) \in \mathcal{E}$ and neither of $n_1$ or $n_2$ is a sanitizer or a guard
\end{enumerate}

Here, we make the following observation - \emph{this domain knowledge present in these annotations and relations is helpful beyond just detecting vulnerabilities}. For instance, simply using the sanitizer relation allows us to query the different kinds of sanitizers domain experts have specified. We use this observation to discover \emph{\safeprogs} i.e., programs having a source, sink, and a sanitizer or guard that \textit{blocks} the \taintprop or, in simpler terms, make the program safe. In addition, we also detect the corresponding sanitizers or guards in the programs and refer to them as \textit{witnesses} because they serve as the evidence of making the program safe. We call this procedure \sawitnessfull (abbreviated as \sawitness). 
We define this as the \T{Witness} relation in Figure~\ref{fig:judgements} (bottom two rules). Specifically, \DMethod{Witness}{$n_1$}{$n_3$}{$n_2$}\ is defined as:
\begin{enumerate}
    \item $n_1$ and $n_2$ are source and sink nodes (\DMethod{Source}{$n_1$}\ and \DMethod{Sink}{$n_2$}\ are true)
    \item There exists a node $n_3$ such that it satisfies \DMethod{SanGuardInMid}{$n_1$}{$n_3$}{$n_2$}. \DMethod{SanGuardInMid}{$n_1$}{$n_3$}{$n_2$}\ is true iff there exists a \T{SemChild}
    %\naga{notation for flow inconsistent with (2) above} 
    path between $n_1$, $n_3$, between $n_3$ and $n_2$, with the additional constraint of $n_3$ being a sanitizer or guard. 
\end{enumerate}

The difference between the \T{Vulnerability} relation (which \sa populates) and \T{Witness} relations (which we want to find) is highlighted in {\color{red} red} and {\color{ForestGreen} green}. Notice that while defining the \T{Witness} relation, we simply use the existing relations that define the \T{Vulnerability} relation. Thus, we argue that \sawitness can be implemented on top of \sa by using the intermediate relations that \sa is computing.
%for every pair of source and sink, they track taint through a taint-flow analysis. If there is a flow from a source to a sink that does not go through a sanitizer or guard, then the source-sink pair is reported as vulnerable.

%We make the following observation - \emph{the patterns defined by experts encodes domain knowledge which can be used for use cases beyond just detecting vulnerabilities}. For instance, we can use the sanitizer patterns to search for all sanitizers in source-code. In this work, we use this idea to detect \safeprogs, which we define as programs having a source, sink, and a sanitizer or guard that blocks the \unsure{flow} or in other words, makes the program safe.  \naman{highlighted part of Figure somethings shows the difference between semantics of witnessing vs traditional semantics}

%We realize the following -- the set of patterns of sources, sinks, and sanitizers are useful beyond detecting vulnerabilities. We override the existing static analysis query that detects unsafe programs and use these encoded sanitizers for detecting sanitizers and guards in programs. Specifically, in the existing query that detects unsafe programs, we modify the taint-propagation steps to propagate taints through sanitizers and guards and then use static analysis to then find these dataflows containing sanitizers and guards. Thus, we can directly find the safe programs containing these \textit{witnesses} of safety. 
%Once such a dataset is collected, we use these witnesses to convert safe  to unsafe  and thus obtain paired examples for learning repair strategies (Section~\ref{subsec:witness-removal}). 

\lstMakeShortInline[columns=fixed]@
%We instantiate our \sawitness technique using \codeql~\cite{a}. It is an open-source \sa tool that allows implementing custom static analysis as queries in a high-level object-oriented extension of datalog. These queries usually contain a \Verb|select from where| statement that allows querying the program database. \codeql maintains these patterns of sources, sinks, sanitizers, and guards using \Verb"Configuration" classes. Consider an example of a simplified \Verb"Configuration" for \xss vulnerability in Figure~\ref{fig:configuration}. It defines a set of predicates @isSource@, @isSink@, @isSanitizer@, and @isGuard@. These predicates are written manually by \codeql authors and improved through rich community support\footnote{\url{https://github.com/github/codeql}}. With this configuration, vulnerabilities are reported by selecting source-sink pairs such that the @cfg.hasFlow@ predicate is true for the source, and the sink. This predicate is internally defined by \codeql and uses the patterns defined in the configuration to check for the presence of vulnerability-causing dataflows. %\spsays{Showing corresponding programs will be useful}

%Now, we demonstrate the static-analysis-witnessing approach for collecting examples of \safeprog and witnesses in Figure~\ref{fig:safe-configuration}. Specifically, we inherit from the existing configuration, using the same @isSource@ and @isSink@ predicates while overriding the @isSanitizer@ and @isGuard@ predicates to @none()@. This ensures that all the source and sink pairs are detected independent of the presence of sanitizers/guards between them. Finally, to detect our witnesses, we define the @isWitness@ predicate which uses the @isSanitizer@ and @isGuard@ predicates from the original configuration. Specifically, witnesses are defined as sanitizers/guards that lie between a source-sink pair. Finally, to report \safeprog and witnesses, the @cfg.hasFlow@ predicate is used to select all valid source-sink pairs and the corresponding witnesses are detected via the @isWitness@ predicate. Note that Figure~\ref{fig:configuration-vs-safe-configuration} depicts the key idea behind our approach in a simplified view. In practice, additional measures need to block the taint propagation internally and we share the actual \codeql queries used as part of the Appendix~\ref{app:codeql-queries}.


\subsection{Witness Removal}
\label{subsec:witness-removal}

We obtain \safeprogs and witnesses by applying \sawitness to a snapshot of a codebase. Recall that the witnesses block the flow between a source and a sink and thus help make programs  \textit{safe}. Hence, removing these witnesses will make the programs unsafe. Recall also that the witnesses are either sanitizing functions of the form @sanitize(taintedVar)@ or guards of the form @if checkSafe(taintedVar) {executeSink(taintedVar)}@. %Usually, they are used only for ensuring the safety of programs and are not critical to the functionality of programs. Therefore, 
We implement witness-removal perturbations  that precisely remove the guard-checks and sanitizer-functions. Note that our goal here is to generate unsafe programs and corresponding edits that enable learning repair strategies that insert such witnesses. So, while we generate the unsafe programs by perturbation, they should look structurally similar to natural unsafe programs written by the developers, otherwise the repair strategies learned on this artificially generated data through perturbations would not generalize to code in the wild. 
%At the same time, minor syntactic-semantic issues in parts of unsafe programs not directly relevant to the vulnerability or repair do not impact learning.
\lstDeleteShortInline@

% Figure environment removed

\lstMakeShortInline[columns=fixed]@

\input{witnessremoval.tex}

We use \rmSan and \rmGuard functions to programmatically remove the witnesses. A high-level sketch of these functions is illustrated in Figure~\ref{fig:remove-functions}. The functions use the structure of the corresponding \astree (node types $\tau$) to decide how to remove witnesses. Consider the \rmGuard function. It first computes the parent (\witnesspar) and grand-parent (\witnessparpar) of the witness guard condition. Then if the type of \witnesspar is \ifstmt (i.e., program is of the form @if (witness) body@ then we modify the \astree edge from \witnessparpar and \witnesspar to instead point to the body of the \ifstmt (index 1 child is body of \ifstmt). Similarly, if the type of \witnesspar is \binaryexpr with operator @&&@ (i.e. of the form @if (otherCond && guard)@ or @if (guard && otherCond)@) then we again modify the edge from \witnessparpar and \witnesspar to instead point to the non-guard child of \binaryexpr (@otherCond@ in the example). Note that since \binaryexpr has 3 children, the index of non-guard child is index of guard-child subtracted from 2. 
Figure~\ref{fig:witness-removal} depicts this removal on the \astree level, where the syntactic edges in red are removed and the syntactic edges in green are inserted.
In the end, the functions returns a tuple of the \pdg of the unsafe program ($\prog_{unsafe}$), \pdg of the safe program ($\prog_{safe}$)
and an edit object (\edit) which stores


\begin{enumerate}
    \item \astree for the removed witness (referred to as \editprog)
    \item location in the \pdg where the witness is removed (referred to as editloc
    %\naga{shouldn't it be editloc to be consistent with (1)?} 
    or \editloc)
    %\item an enum (\insertsc or \replace) depending on whether \concedit is inserted or replaced 
\end{enumerate}

Since $\prog_{unsafe}$ and edit-object can generate the safe program, we only propagate the unsafe programs and edits as the output of this step. Applying \rmGuard function to the safe program in Figure~\ref{fig:safememberex} removes the \ifstmt on Line~\ref{lst:line:fix-start} while preserving the @handlers[callerId](data);@ statement and in fact produces the unsafe program in Figure~\ref{fig:unsafememberex}. Additionally, it  returns the removed witness guard  @if handlers.hasOwnProperty(data.id){ ... }@ as the \editprog and \blockstmt (blue oval in Figure~\ref{fig:example1-pdg}) as the edit location \edit.editloc. Figure~\ref{fig:example1-editprog} shows the \astree for the \editprog containing the \ifstmt. 
The dashed line and dark circle correspond to the \textit{removed} \astree edge between the \blockstmt and the \expr @handlers[callerId](data)@. 

Note that Figure~\ref{fig:remove-functions} provides a high-level sketch of witness-removal and elides over implementation details that are required to make it work for real \js programs. We discuss these issues in the implementation section (Section~\ref{subsec:impl:witness-removal}).% and include the full implementation as part of supplementing source code\naga{we should make sure we are doing these, else remove this sentence}. 
%. In practice, we need implement such decisions more carefully to cover other traditional cases in which guards occur and we document them in the supplementing source code.
\lstDeleteShortInline@

%\naman{add examples $\dots$ } \spsays{do we re-run codeql on this generated bad program? -- NO (naman)}


%%auto-ignore
\pdfoutput=1

\documentclass[../main.tex]{subfiles}

\begin{document}

\section{Simulation}\label{sec:sim}

\subsection{Overview}

In this simulation exercise, we study the performance of integrated estimation of the median and Gini index in the context of Australian personal income. We shall compare the bias and variance of big-data-only, survey-only and integrated estimators across varying population sizes, maintaining a fixed sampling fraction. Estimators are compared under the joint distribution spanning both the design and superpopulation. 

In this case, our population consists of $12$ strata stratified by age and sex (males and females aged 24 and under, 25 to 34, 35 to 44, 45 to 54, 55 to 64 and 65 and over), where each stratum is comprised of scalar observations representing personal income. Our big-data set will emulate an administrative tax-return data set as could be sourced from a taxation department such as the Australian Taxation Office. Because not all Australian residents earning under the tax-free threshold of \$18,200 are required to submit a tax-return, we expect that such a data set would underrepresent this demographic, introducing selection bias that will be emulated by the big-data sampling mechanism in the simulation. Our simulated survey data set will represent an official survey conducted by a national statistics office, such as the Australian Bureau of Statistics' Survey of Income and Housing (SIH), and use stratified simple random sampling without replacement.\footnote{Note that the SIH typically uses a survey design different from what we consider here; see \citepref{ABSMD}.} We consider two survey sampling mechanisms: one in which units are sampled from the entire population, and a second in which only non-big-data units are sampled. 

\subsection{Simulation Design}

Firstly, we construct a superpopulation distribution from which an `Australia-like' population may be generated. This superpopulation distribution is a finite mixture with c.d.f.
\begin{equation}
F(y) = \sum_{h = 1}^{12} p_h F_h(y),
\end{equation}
where $p_h$ represents the proportion of stratum $h$ in the Australian population, and $F_h(y)$ the c.d.f.\ of stratum $h$. These proportions are derived from Table 4.1 of \citepref{ABSPI} by calculating the ratio of the counts (Column I) of each stratum for a given age (Column C) and sex (Column D) to the total (Column I, Row 28). Each c.d.f.\ $F_h(y)$ is constructed by interpolating a monotonic cubic smoothing spline through income frequency data from Graph 1 of \citepref{ABSDB}, using the method described in the `Interpolated CDFs' section of \citepref{Hippel2017}. Specifically, for each stratum $h$, we fit a cubic smoothing spline with knots $(y, F_h(y))$ at the points 
\begin{equation}
( 0, 0), \bigg ( \frac{52\eta_h b_1}{\eta}, \frac{ r_1}{100} \bigg ),  \bigg ( \frac{52\eta_h b_2}{\eta}, \frac{ \sum_{i=1}^2 r_i}{100} \bigg ), \ldots,  \bigg ( \frac{52\eta_h b_{57}}{\eta}, 1 \bigg ),
\end{equation}
where $\eta_h$ is the reported median of stratum $h$ (Table 4.1 Col.\ N of \citealppref{ABSPI}), $\eta$ is the reported population median (Table 4.1 Col.\ N of \citealppref{ABSPI}), $b_i$ is the upper bound of the $i$\textsuperscript{th} income bracket ($x$-axis) of Graph 1 of \citepref{ABSDB}\footnote{Note that \citepref{ABSDB} uses equivalised household income instead of personal income. We assume that the distribution is comparable. } and $r_i$ is the corresponding frequency from the 2019-20 financial year ($y$-axis). Bracket $b_{57}$ is set such that $\mathbb{E}[Y_h]= \mu_h$, where $\mu_h$ is the reported mean of stratum $h$ (Table 4.1 Col.\ S of \citealppref{ABSPI}). This approach also ensures the median of each stratum $h$ falls within the income bracket of the reported median $\eta_h$. The coefficients of each spline are computed by the Hyman method using the splinebins function of the R package `Binsmooth' (\citealppref{binsmooth}).

From this c.d.f.\ $F$, we calculate the superpopulation median, given by $F^{-1}(0.5)$, and the superpopulation Gini index, given by $\mathbb{E}[Y]^{-1}\int_0^\infty F(y)(1 - F(y)) \mathop{dy}$. We also estimate the asymptotic joint variance (see Theorem \ref{thm:desvar}) of the integrated median and Gini index estimators by generating $10^8$ observations from the superpopulation distribution and using these observations to calculate $\hat V ^\prime$ based on (\ref{eqn:vhatprimestrat}) and (\ref{eqn:vhatprimestratintegratedsrswor}).

We then perform a simulation consisting of the following steps:
\begin{enumerate}
	\item Generate $2 \times 10^6$ observations from the superpopulation distribution, given above.
	\item Construct $10$ populations of increasing sizes, the sizes equally spaced along the interval $[ 5 \times 10^5,  \ldots, 2 \times 10^6]$, where the population $U_k = \{1, 2, \ldots, N_k\}$ consists of the first $N_k$ individuals whose incomes were generated in Step 1. Denote $Y_{i}$ the income of the $i$\textsuperscript{th} individual.
	\item For each population $U_k$, sample $N_k/2$ individuals to comprise big data set $B_k$, where the probability of individual $i \in U_k$ being selected into $B_k$ is given by
	\begin {equation}
	\pi_i^b =
    \begin{cases}
        (|H_k| + 0.05|L_k| )^{-1} &  i \in H_k\\
        0.05(|H_k| + 0.05|L_k|)^{-1} &  i \in L_k
    \end{cases},
    \end{equation}
    where $ L_k = \{i \in U_k  \mid Y_i < 18200 \} $ and $ H_k = U_k \setminus L_k$. Note that we  $\pi_i^b$ would be unknown in practice and is not used to compute any statistics in this simulation.
	\item For each population $U_k$ and stratum $h$, sample
	\begin{equation}
	n_{k, h} =  n_k \frac{N_{k,h} S_{Y_{k,h}}}{\sum_{h=1}^{12} N_{k, h} S_{Y_{k, h}}}
	\end{equation}
	units without replacement to produce the set of sampled stratum-$h$ units units $A_{k,h}$ and the set of all surveyed units $A_k = \cup_{h=1}^{12} A_{k,h}$,  where $N_{k, h}$ is the population size of stratum $h$ from population $U_k$, $n_k = 10^{-3}N_k$,  and $S_{Y_{k, h}}^2$ is the sample variance of $Y_i$ across $i$ in stratum $h$ and population $U_k$. Within each stratum, units are selected with equal probability, yielding a first-order inclusion probability of $ \pi_i = n_{k,h}/N_{k, h}$ for a unit $i$ in stratum $h$. Given a fixed total sample size $n_{k}$, this allocation minimises the variance of the survey-only mean (see Section 3.7.4.\ i.\ of \citealppref{Sarndal1992}).
	\item Define $ U_k^{\prime} = U_k \setminus B_k$. Repeat Step 4., sampling instead from $U_k^{\prime}$ to form $A_k^{\prime}$. This is equivalent to treating the big data sample as a completely enumerated stratum, as described in Section \ref{eg:strat}.
	\item For $k \in \{1,2,\ldots,10\}$, use \eqref{eqn:pquantile} with $p = 0.5$ to calculate the following estimates of the median:
	\begin{itemize}
		\item An unweighted estimate using only $B_k$. 
		\item A Horvitz-Thompson-weighted estimate using only $A_k$.
		\item An integrated estimate using both $B_k$ and $A_k$, using the integrated weights in \eqref{eqn:wdi} alongside Horvitz-Thompson survey weights.
		\item An integrated estimate using both $B_k$ and $A_k^{\prime}$, using the integrated weights in \eqref{eqn:wdi} alongside Horvitz-Thompson survey weights.
	\end{itemize}
	\item Repeat Step 6.\ for the Gini index given by $\eqref{eqn:gini}$.
	\item Produce $10^4$ Monte-Carlo draws for each population size $N_k$ by repeating Steps 1 to 7.
	\item Across incomes given each $N_k$:
	\begin{enumerate}
	\item Calculate the sample variance of the four estimates for both the median and Gini index using the Monte-Carlo draws.
	\item Calculate the sample bias of the four estimates using the Monte-Carlo draws and the superpopulation median and Gini index. 
	\item Calculate approximate $95 \%$ confidence intervals for the bias and variance estimates, using the standard asymptotics for i.i.d.\ draws.
	\end{enumerate}
\end{enumerate}

\subsection{Results}

% Figure environment removed

% Figure environment removed

% Figure environment removed

% Figure environment removed

The sample bias of the median and Gini index estimates as a function of the population size are depicted in Figures \ref{fig:median_bias} and \ref{fig:gini_bias} respectively. Observe that the big-data-only estimate exhibits bias due to the sampling mechanism of the big data set outlined in Step 3. This mechanism overrepresents high-income earners, resulting in an overestimate of the median and an underestimate of the Gini index. Importantly, both of the integrated estimators yield a statistically insignificant bias for large populations, which is consistent with Theorem \ref{thm:normsup} and Remark \ref{rmk:clt}.

Figures \ref{fig:median_var} and \ref{fig:gini_var} depict the size-adjusted variance estimates for the Gini index and median, respectively. We see that the Monte-Carlo variance of the integrated estimators calculated in Step 9.\ are consistent with the asymptotic variance estimates described in Section \ref{sec:varest}. This demonstrates the utility of the asymptotic variance estimates for practical applications, particularly given that the asymptotic variances only very occasionally fall outside the confidence intervals for the corresponding Monte-Carlo variances for population and sample sizes consistent with those of surveys such as the SIH.

Figures \ref{fig:median_var} and \ref{fig:gini_var} also highlight the efficiency of the integrated estimators over the survey-only estimators in that they exhibit significantly lower variance. Additionally, we can see from both Figure \ref{fig:median_var} and Figure \ref{fig:gini_var} that the integrated estimate using probability sample $A_k^{\prime}$ outperforms in terms of variance the estimate using $A_k$ (with a variance of less than half). This illustrates that, in the presence of a pre-existing big data set, significant efficiency gains in terms of variance can be achieved by tailoring the survey design to exclude those units present in the big data set.

\if0\wholepaper {
	\bibliographypref{../library}
} \fi

\end{document}
\section{Discussion}
\label{sec: discussion}
\kmsdelete{In this work} We study \kmsreplace{Fairness-Aware PAC learning}{Fair-ERM} in the malicious noise model, and  in some cases allow 
the learner to maintain optimal overall accuracy despite the signal in Group $B$ being almost entirely washed out.
%when we allow learners to use the
%$\PQ$ randomized expansion of the hypothesis class $\mathcal{H}$
In particular we show that different fairness constraints have fundamentally different behavior in the presence of Malicious Noise, in terms of the amount of accuracy loss that a given level of Malicious Noise could cause a fairness-constrained learner to incur. 
The key to achieving our results, which are more optimistic than those in \cite{lampert}, is allowing for improper learners using the (P,Q)-randomized expansions of the given class $\mathcal{H}$.
%We \kmsreplace{present a picture of the}{prove upper and lower bounds on}
%accuracy loss for a range of fairness notions, given \kmsreplace{this simple randomization step.}{learning over $\PQ$.
%In general our results indicate Fair-ERM (given learning over $\PQ$) is more robust than claimed in \cite{lampert}.
The type of smoothness we create by using $\PQ$ seems to be a natural property that is likely shared by many natural hypothesis classes.

Fairness notions are motivated as a response to learned disparities when there is \kmsdelete{data corruption or} systemic error affecting \kmsdelete{the data for}
one group. 
Fairness notions are supposed to mitigate this by ruling out classifiers that have worse performance on a sub-group. 
This can peg both classifiers at a lower level of performance \kmsdelete{(e.g that the lower subgroup)} in order to \emph{motivate} \cite{hardt16} improving the data collection or labelling process to obtain more reliable performance. 
%So in \kmsreplace{some}{a} sense, sensitivity of the fairness notion to poor sub-group performance caused by malicious noise is the \textit{point} of fairness constraints! 
However, it also desirable that fairness constraints perform gracefully when subject to Malicious Noise because fairness constraints will be used in contexts where the data is unreliable and noisy and this might not be known to the learner.
This tension, exposed by our work, motivates 
%a revisiting of fairness notions from first principles approach and trying to axiomatize the 
%desired properties of a fairness intervention a la cryptography and privacy. \footnote{Work in multi-calibration \cite{multicalib} is a viable direction for this problem but it is unclear how 
%that and related notions behave with unreliable data. }
on going work studying the sensitivity level of fairness constraints. 
%If we we are to take a view, if a classifier is deployed 

\section{Conclusion and Future Work}
In this work, I design corruption-robust algorithms for the Lipschitz contextual search problem. I present the \emph{agnostic checking} technique and demonstrate its effectiveness in designing corruption-robust algorithms. There are several open problems for future research. First, in the algorithm I propose for pricing loss, the schedule for agnostic checks is fixed upfront. Can the learner design an adaptive checking schedule for the pricing loss? Second, this work assumes the learner has knowledge of the Lipschitz constant $L$. Can the learner design efficient no-regret algorithms without knowledge of $L$? 

\bibliography{references}

\appendix
\begin{comment}
\section{System Architecture}
\label{appendix:architecture}
\system has a novel modularized system architecture with three key components: 
\emph{StreamManager}, 
\emph{TxnManager} and \emph{TxnScheduler}. 
These components are instantiated in each thread locally.
The execution outline of \system is presented in Algorithm~\ref{alg:algo}.
Transactional stream processing is continuous and potentially never ends (Line 1$\sim$8).
The dependency resolution and execution of state transactions are separated into two non-overlapping phases by punctuations~\cite{Tucker:2003:EPS:776752.776780} (Line 2 and 5), which guarantees that no subsequent input event will have a smaller timestamp. 
Effectively, a batch of state transactions is collected during the first phase, and processed during the second phase.

In the first phase (i.e., stream processing phase), 
the \emph{StreamManager} conducts preprocessing for every input event ($e$). Similar to some prior works~\cite{tstream}, state transactions may be issued but not immediately processed during preprocessing (Line 3).
The \emph{pre\_processing} and \emph{post\_processing} functions are exposed as APIs to users.
The \emph{TxnManager} handles dependency resolution (Line 4) among state transactions and insert decomposed operations to construct a \tpg. We discuss the detailed two-phase \tpg construction process in Section~\ref{subsec:construction}.

In the second phase  (i.e., transaction processing phase), 
the \emph{TxnManager} is first involved again to refine (Line 6) the constructed \tpg with further dependency resolution.
The \emph{TxnScheduler} 
schedules operations for concurrent execution based on the constructed \tpg according to the three dimensions of scheduling decisions (Line 7). 
In particular, a scheduling decision model $M$ is instantiated based on the constructed \tpg (Line 14).
\textbf{\circled{1}} Guided by $M$, execution threads adopt an exploration strategy (Section~\ref{subsec:explore}) to explore the constructed \tpg for operations available to be scheduled constrained by dependencies. 
\textbf{\circled{2}} 
During exploration, one or multiple operations may be treated as the 
% basic 
unit of scheduling (Section~\ref{subsec:granularity}). 
Subsequently, \textbf{\circled{3}} every thread executes operation(s) in the unit of scheduling with various abort handling mechanisms (Section~\ref{subsec:abort_handling}).
Only when state transactions are processed (i.e., committed or aborted) can the associated input events be postprocessed (Line 8) by the \emph{StreamManager} based on transaction processing results.
\end{comment}

\begin{comment}
\begin{algorithm}
\footnotesize
    \KwData{$e$ \tcp{Input event}}
    \KwData{$txn_{ts}$ \tcp{State transaction}}
    \KwData{$G$ \tcp{The currently constructed TPG}}
    \While{!finish processing of input streams}{
        \eIf(\tcp*[h]{Phase 1}){\text{$e$ is not a $punctuation$}}{
                $txn_{ts}$ $\gets$ PRE\_Processing($e$)\;
                \textbf{TPG\_Construction}($G$, $txn_{ts}$)\; 
          }(\tcp*[h]{Phase 2}){
                \textbf{TPG\_Refinement}($G$)\; 
                \textbf{TXN\_Scheduling}($G$)\; 
                POST\_Processing()\;
          }
    }
    
    \SetKwFunction{FMain}{TPG\_Construction}
    \SetKwProg{Fn}{Function}{:}{}
    \Fn{\FMain{$G$, $txn_{ts}$}}{
        $O_{1..k}$ $\gets$ \textbf{Partition} $txn_{ts}$\;
        \ForEach{\text{operation $O_{i}$ $\in$ $O_{1..k}$}}{
            \textbf{Identify} its \ld\;
            $G$ $\gets$ $G$ + $O_{i}$ \;
        }
    }
    \SetKwFunction{FMain}{TPG\_Refinement}
    \SetKwProg{Fn}{Function}{:}{}
    \Fn{\FMain{$G$}}{
        \ForEach{\text{vertex $e_{i}$ $\in$ $G$}}{
            \textbf{Identify} its \td, \pd\;
        }
    }
    
    \SetKwFunction{FMain}{TXN\_Scheduling}
    \SetKwProg{Fn}{Function}{:}{}
    \Fn{\FMain{$G$}}{
        $M$ $\gets$ Instantiated with $G$;\tcp{A decision model}
        \While{!finish scheduling of $G$
        }{
          \textbf{\circled{2}} $Scheduling Unit$ $\gets$ \textbf{\circled{1}} \emph{Explore}($G$, $M$)\; 
            \textbf{\circled{3}} \emph{Execute with Abort Handling} ($Scheduling Unit$)\; 
        }
    }
  \caption{Execution Outline of \system}
  \label{alg:algo}
\end{algorithm}
\end{comment}

\end{document}
