\section{Concluding Remarks}
\label{sec:conclusion}

%Although stablecoins have been proposed to address the high price volatility of existing cryptocurrencies, many of them did not measure up to people's expectations.
%In particular, UST's downfall brought about far-reaching aftermath in all aspects of society, which has raised the alarm about stablecoin risks. 
%Its collapse was accelerated by the significantly rapid price drop of Luna that backs UST. 
%
In this paper, we developed a common theory to characterize the stability properties of many stablecoins, considering reserve asset types and redemption mechanisms.
%This paper attempts to identify why many stablecoins have often depreciated, considering asset types paid to users upon redemption;
A continuous price drop of the assets backing stablecoins can reduce the actual (or recognized) payoff that users get by redeeming stablecoins, which can contribute to peg stress even in fully backed systems. 
Stablecoin systems collateralized by volatile cryptocurrencies can often suffer from this, leading to a higher price instability level compared to fiat-collateralized systems. 
To avoid the depegging state, such systems may need to rely on additional incentive mechanisms, such as large interest rates that can motivate many users to keep holding their coins.
%Although our analyses find that collateral type is one of the potential factors influencing stablecoin price stability, it is far from being the only factor. 
% \yujin{Somewhat I think we should have one more sentence here given the flow?} yeah
%We believe that our study can provide a better understanding of existing stablecoin designs and guidelines for future stablecoin designs.
As stablecoins receive newfound scrutiny, our model helps to improve understanding of design differences and establishes guidelines for future stablecoin design.
