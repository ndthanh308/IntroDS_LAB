\section{Introduction}
\label{sec:intro}

Cryptocurrencies, such as Bitcoin and Ethereum, have gained tremendous global attention over the past few years. 
Despite their popularity, the significant price volatility of their native coins, such as BTC and ETH, has hampered their usage as a consistent value storage medium. 
To address this volatility issue, \emph{stablecoins} were proposed and, since their introduction, have taken up a central role in the Decentralized Finance (DeFi) ecosystem. At the time of writing, stablecoins have reached a market capitalization of over USD \$116 billion. 

The goal of a stablecoin is to offer a store of value with low volatility, i.e., the fluctuation of the price of a stablecoin should be minimized. Since the price of an asset is always expressed relatively to the value of another asset, one popular method to define a coin with a ``stable value'', is to link or \emph{peg} a stablecoin to the value of a government-issued fiat currency such as the US Dollar. Note that pegged stablecoins aim to remain stable relative to the target currency, but may fluctuate with respect to other assets. The widespread adoption of fiat-pegged stablecoins crafts a productive connection between traditional and decentralized finance ecosystems.

Stablecoins can achieve price stability by adopting a pegging mechanism similar to a traditional pegged currency. For national currencies to achieve fixed exchange rates, the central bank plays an important role. The central bank adjusts the market supply and demand to attain the target exchange rate by holding foreign reserves that can be exchanged for national currency. For example, if an exchange rate is less than the target, the central bank purchases national currency from the market with foreign reserves, which increases the exchange rate~\cite{lyons2023keeps}. 

However, the blockchain use of a stablecoin diversifies a stablecoin design, allowing for a more progressive, albeit often complicated, pegging mechanism. For example, they can rely on cryptocurrencies to maintain the peg, and some stablecoins even attempt to stabilize a price without relying on reserves consisting of exogenous assets.\footnote{It refers to assets that are run independently of the stablecoin system. For example, Bitcoin and Ethereum are one of the most representative exogenous crypto assets.} 

In broad strokes, we distinguish between \emph{collateralized} and \emph{algorithmic} stablecoins, while hybrid models may exist. Collateralized stablecoins safeguard their value by holding a reserve of exogenous assets that can be used to purchase stablecoins from users. 
On the other hand, algorithmic stablecoins aim to achieve a peg through algorithmically expanding and contracting supply without holding exogenous collateral to back their coins. The systems mint an endogenous cryptocurrency to purchase stablecoins from users.

The most intuitive design of the pegging mechanisms of stablecoins is to allow users to buy/sell stablecoins from/to the system at near the target price (e.g., 1 USD) to maintain the peg. If the market price deviates from the target price, users have a monetary incentive to perform arbitrage between the stablecoin system and the secondary market. Depending on the types of assets stored in the reserves, we can categorize collateralized stablecoins into \textit{fiat-collateralized} or \textit{crypto-collateralized}.
\emph{Algorithmic} stablecoins also adopt such a pegging mechanism design, but they use endogenous cryptocurrencies. 

Moreover, stablecoins can be designed with close ties to lending systems, and such stablecoins are referred to as \emph{over-collateralized}. In these coin systems, users deposit collateral of a higher value than the stablecoins issued to them, thereby becoming debtors. When users redeem the stablecoins (i.e., debts) to the system, they can get back the deposited collateral. If a position is at risk of turning into a bad debt due to a drop in the collateral value, the system allows other users to liquidate the position by paying back the stablecoins on behalf of the debtors and thereby getting the underlying collateral~\cite{qin2021empirical}. In this design, if the market price is less than the target price, users are expected to buy stablecoins from the market to settle (their own or others') debts at a discount, which raises the overall market price to the peg.

In fact, a crypto-collateralized stablecoin is a term often associated with a broad categorization of any stablecoin backed by cryptocurrencies. However, in this study, we further partition this broad categorization based on how redemption takes place.
In other words, we differentiate between crypto-collateralized and over-collateralized stablecoins while both use cryptocurrencies as collateral; crypto-collateralized stablecoins purchase and sell stablecoins to users at a target price to maintain the peg. On the other hand, over-collateralized systems apply a lending system.

%If the reserves of a crypto-collateralized system diminishes in value due to a price decline, they can replenish the reserves through the earned transaction fees or other profits.

% Figure environment removed

Although stablecoins are designed for relative price stability, price history shows that it is quite common for pegs to break. The Terra stablecoin, UST, which was one of the most representative algorithmic stablecoins, collapsed in the second week of May 2022 (see Figure~\ref{fig:binance-ust-usdt-price})~\cite{terra1,terra2}. 
As an algorithmic stablecoin, Terra uses an endogenous cryptocurrency, Luna, to maintain the peg of UST at 1 USD; the system mints 1 USD worth of Luna to the users for each UST they redeem to the system. 
Therefore, if the market price of UST falls below 1 USD, users should sell their coins to the system instead of the market for a higher profit. 
However, this redemption process has a risk; it inflates the market with the minted Luna, driving the price of Luna down, scaring users to sell their coins, and making each subsequent UST redemption mint even more Luna. 
As a result, a significant deviation from the target price could lead to a larger deviation.
This is commonly known as the \emph{death spiral}, which is what happened to Terra in May 2022. Another example of a USD-pegged stablecoin is Tether, where there have been doubts regarding the reserve coverage, raising concerns about bank run risks~\cite{klages2020stablecoins,arner2020stablecoins}.
Tether was even fined by the US Commodity Futures Trading Commission (CFTC) for making ``untrue or misleading statements and omissions'' regarding their reserves~\cite{tether_law,tether_fine}.
In short, many stablecoin designs contain weaknesses that may threaten price stability, and in some cases these have led to the collapse of their prices to almost zero~\cite{terra1,terra2,terra3,terra4,nubits}.  


\subsection{Implications of our work}


The statistically observed difference in price volatility of stablecoins~\cite{jarno2021does} and several collapse events including Terra UST beg the question of how mechanism design choices get translated into stablecoin price stability. 
Currently, it is unclear how and to what extent these design-dependent risks are reflected in the actual stablecoin price stability. 
Bearing the risks does not necessarily directly lead to actual price instability. 
For example, although many people point out a bank run risk\footnote{It can be triggered by a lack of reserves stored by a stablecoin system.} of Tether, we also often observe that it did not significantly affect its price in the real world. Even when Tether was fined multiple times due to some evidence that may suggest holding partial reserves, its price was still around \$1. 
Indeed, in general, fiat-collateralized stablecoins that store reserves in fiat money are regarded as more stable assets than other stablecoin types~\cite{jarno2021does}.
In contrast, many crypto-collateralized stablecoins that hold collateral in cryptocurrencies often depreciate below \$1 and return more slowly to the pegging state. 
In particular, considering when UST depegged, unlike the Tether case at that time, UST could not quickly return to the peg even though the market cap of Luna, which is the asset backing it, exceeded its market cap in the early stage of the catastrophic event~\cite{terra1,terra2,terra3,terra4}. 
These suggest that the relationship between design risks and price instability is not straightforward, which makes it hard to understand their inherent relationship clearly. 

The goal of this work is to assist in developing a solid, common theory that can model the relationship between stablecoin designs and price volatility and further compare the effects of their design differences. 
Our work helps to understand why stablecoins show different price volatility by modeling them in a common framework, taking into account the difference in asset types that various stablecoin systems use to purchase coins from users. The type of asset paid to users by a system can matter; a volatile asset can lower the user's recognized payment value due to a price fluctuation during the payment transaction process. We show that this factor would significantly affect a level of stablecoin price stability. In practice, when designing a stablecoin, developers can focus more on addressing risks related to a lack of reserves while little considering the discrepancy in payment values between users and the system~\cite{usdn,lusd,celo,kereiakes2019terra}. This suggests that the current stablecoin systems may be missing the critical point. 

\subsection{Related work}

Many works~\cite{klages2020stablecoins,arner2020stablecoins,kwon2021trilemma,moin2020sok,algorithmic} have analyzed the systematic risks of different stablecoin designs. 
As discussed in \cite{klages2020stablecoins}, traditional models of bank runs \cite{diamond1983bank}, runs on currency pegs \cite{morris1998unique}, and pegged redemption money market funds \cite{parlatore2016fragility} can be applied to understand the risks of various stablecoins; it is possible to reinterpret the pegging mechanism as the central bank, and so basic results about bank and currency runs can translate to many stablecoin settings.
Given this, \cite{routledge2022currency} investigates a way to avoid speculative attacks~\cite{morris1998unique} in partially backed stablecoins, and shows that optimal exchange rate policies rate can help avoid the risk. 
\cite{li2022money} also studies the optimal strategy of collateralized stablecoin systems to investigate whether stablecoins are sustainable given the risks. 

%discussed and created adaptations of these types of models specific to stablecoins.  
%routledge2022currency: an exchange policy rate can help avoid speculative attacks; who study speculative attacks on undercollateralized stablecoins and coordination failure~\cite{morris1998unique,goldstein2005demand}
%li2022money: 


%bank run = assets may cover liabilities, but some assets are illiquid and not all stablecoins can be redeemed at one point in time (but may over time)
%run on currency peg = assets partially cover liabilities and not all stablecoins can be redeemed for peg value
%Pegged redemption money market funds face similar issues if net asset value falls below the pegged redemption rate.

On the other hand, models explaining the price (in)stability of stablecoins are relatively sparse. Note that having risk in a stablecoin design does not always lead to price drops (e.g., the Tether case), and thus studying systemic risks alone may not be sufficient to understand stablecoins. \cite{klages2022while,klages2021stability} model stablecoins like DAI, where issuance is based on a market for leverage. In this context, they characterize deleveraging spirals that caused instability in DAI on ``Black Thursday'' in March 2020. 
\cite{lyons2023keeps} models how stablecoins backed by 100\% reserves, such as USDC, maintain price stability through arbitrage with minting and redemption. \cite{klagesmundt2022designing} explains a shape of redemption curves of several stablecoins and then designs a redemption curve that attains price stability.

Beyond few attempts to elucidate the price (in)stability of stablecoins, existing models are difficult to generalize due to the heterogeneity of the stablecoin design space.
Moreover, various distinctions about stablecoin models are not as developed, such as modeling the fact that reserve assets are not the currency target (e.g., many stablecoins hold reserves consisting of cryptocurrencies, not their price target). 
We address this by proposing a game-theoretical model motivated by \cite{morris1998unique} as a starting point. We then extend to include reserve assets that change in price and further by breaking down assets backing into exogenous and endogenous assets, which have different degrees of price volatility (this is encoded in the function $r^c$ in our model).
%We address this by proposing a new game-theoretical model to include reserve assets that change in price and further by breaking down assets backing into exogenous and endogenous assets, which have different degrees of price risk.

Having a common framework to analyze stablecoin price stability would be conducive not only to understanding the connection between stablecoin designs and price stability but also to comparing different stablecoin designs.
In addition, there is room to expand a common model to better characterize nuanced differences across different types of stablecoins.
This is the approach we take in encoding a model that is flexible enough to analyze redemption mechanisms across different types of stablecoins.
%which is the approach we take in analyzing redemption mechanisms in different stablecoins. 

%we do not have a general theoretical framework due to the 
%makes it difficult to build the general framework. %makes it difficult to build the general framework. 

\smallskip\noindent\textbf{Contributions.}
Here, we summarize the contributions of this paper below.

\begin{itemize}
    \item We build a game theoretical model to analyze and compare four different stablecoin designs, fiat-collateralized, crypto-collateralized, algorithmic, and over-collateralized stablecoins, under a general framework. 
    \item We analyze the game model for each stablecoin design and show that they result in different price equilibria, allowing us to relatively rank the price stability of stablecoin designs. 
    Specifically, we show that fully backed fiat-collateralized stablecoins can guarantee the peg by proving they have the unique pegging equilibrium for any economic condition. 
    On the other hand, our analysis shows that partially backed fiat-collateralized stablecoins have multiple equilibria including the pegging state; in particular, the pegging state becomes a self-fulfilling equilibrium. 
    The equilibria of crypto-collateralized, algorithmic, and over-collateralized stablecoins depend on economic situations. We show that crypto-collateralized and algorithmic stablecoins have the unique pegging equilibrium in a good economic environment, but there are only depegging equilibria in poor economic conditions. Even where they can hold sufficient reserves to back the stablecoins fully, they may have only depegging equilibria. 
    Lastly, while analyzing the equilibria of over-collateralized stablecoins, we find that they have multiple equilibria including the pegging state in most economic conditions.
    In other words, it may be difficult to rank crypto-collateralized, algorithmic, and over-collateralized stablecoins absolutely; in a good economic condition, crypto-collateralized and algorithmic stablecoins are better than over-collateralized stablecoins because crypto-collateralized and algorithmic stablecoins have the unique pegging equilibrium. But over-collateralized stablecoins can be superior to the two in poor economic conditions. 
    \item We collect and analyze the daily price of stablecoins and redemption transactions to empirically confirm our theories. We compare price stability by stablecoin type and find a statistical connection between a stablecoin price and the payment that users get when redeeming stablecoins.
    As a result, we find that a stablecoin price has overall fluctuated in agreement with our theory. 
\end{itemize}



% In over-collateralized stablecoins, redemption is limited to a group of users, unlike other types; only debtors can always redeem their stablecoins to the system, and users can have the opportunity to redeem coins only when there are liquidation processes. 
% This causes the price to be up to users' belief in the debtors' actions, bringing about multiple price equilibria. 

% maintain the peg by the user's payoff of not less than \$1, but in poor economic conditions, the peg can break because the payment medium (i.e., cryptocurrencies) would depreciate and the users' actual payoff can be less than \$1 by the price discrepancy between users and the system. %the time difference between prices to which the system and users refer


% that is, it can guarantee the peg regardless of economic conditions because users can always expect to get \$1 when redeeming stablecoins to the system due to the stable payment medium, fiat money.

% \smallskip
% \noindent\textbf{Empirical analysis:}


% As expected in the theories, we observe that fiat-collateralized systems and crypto-collateralized systems using other stablecoins have been relatively stable, but crypto-collateralized systems with non-stablecoins, algorithmic systems, and over-collateralized systems have been less stable. 
% We also find a strong correlation between the price fluctuation and the payoff fluctuation of users who redeem stablecoins. 
% Further, we show that many stablecoins have a significant correlation and causality between their price and the users' payoff. 

\smallskip\noindent
\textbf{Paper structure:}
The rest of the paper is organized as follows. Section~\ref{sec:model} proposes a general, common game-theoretical model to analyze different stablecoin designs. Section~\ref{sec:type} represents four types of stablecoins under the proposed game-theoretical framework. Then, we analyze the game and find equilibria in Section~\ref{sec:eq}. Section~\ref{sec:data} provides empirical analyses to support our theories. Lastly, we discuss stablecoin design challenges in Section~\ref{sec:discuss} and conclude in Section~\ref{sec:conclusion}.