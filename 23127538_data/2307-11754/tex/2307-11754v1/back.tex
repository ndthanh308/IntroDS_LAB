\section{How stablecoins work}
\label{sec:back}

In this section, we discuss the peg defense mechanism of each of the four categorizations of stablecoin in more detail as well as provide real world examples of each categorizations. While all four types of stablecoin rely on the redemption against a reserve, they differs in the composition of the reserves and how redemption is conducted.

\subsection{Fiat-collateralized stablecoins}
Fiat-collateralized stablecoins are stablecoins that can be redeemed directly as a corresponding fiat currency. A major portion of their reserves is made of fiat or fiat-equivalent assets, such as short-dated U.S. treasury bills. Other portion of the reserves could consist of loans or even precious metals. Some examples of fiat-collateralized stablecoins are USDT, USDC, and USDP. The key distinctions of fiat-collateralized stablecoin are as followed: 1) Their reserves consist mainly of fiat or fiat-equivalent assets. 2) Minting or redeeming them are done with 1:1 ratio with fiat currencies in terms of value.

\subsection{Crypto-collateralized stablecoins}
Crypto-collateralized stablecoin is a term often associated with a broad categorization of any stablecoin that can be redeemed into cryptocurrencies. However, in this study, we further partition this broad categorization based on how redemption takes place. In this study, crypto-collateralized stablecoins refers to stablecoins that possess the following 3 characteristics. 1) Their reserves are composed of cryptocurrencies. 2) Redemption returns a basket of cryptocurrencies with approximately 1:1 in value of the redeemed stablecoins. 3) The stablecoin protocol has no control over expansion and contraction of the supply of the redeemed currencies; the redeemed currencies are not minted and burned during redemption and creation of the stablecoins, respectively. Given the volatility of cryptocurrency market, it is possible that the reserve of a crypto-collateralized stablecoin to be worth less than the supply of the it. To compensate for the higher risk of a bank-run event, crypto-collateralized stablecoin protocols often implement another peg defense mechanism. Those that use only crypto-collateralization as the only peg defense mechanism either have their reserves consist of only other stablecoins and relies on the reserves' price stability or have a way to accrue more value to their reserves. 

\subsection{Algorithmic stablecoins}
Similar to crypto-collateralized stablecoins, algorithmic stablecoins can be redeemed for cryptocurrencies. Unlike crypto-collateralized stablecoins, however, algorithmic stablecoin protocols have control over the expansion and contraction of their reserve currencies. Upon redemption, the reserve currencies are minted at 1:1 value ratio as the redeemed stablecoins. Given this, algorithmic stablecoins are characterized by the following 3 characteristics: 1) Redemption returns a basket of cryptocurrencies with approximately 1:1 in value of the redeemed stablecoins. 2) Redeemed cryptocurrencies are minted as opposed from being transferred from the reserve. 3) The stablecoin protocol has control over the expansion and contraction of supply of the redeemed currency. The expansion and contraction of supply of the redeemed token allows a flow of economic value between the stablecoin and redeemed token. USDT an example of an algorithmic stablecoin.

\subsection{Over-collateralized stablecoins} 
Over-collateralized stablecoins is another type of stablecoins that returns a basket of cryptocurrencies upon redemption. Issuance and redemption of over-collateralized stablecoins rely on a lending protocol backbone, where a basket of cryptocurrencies can be deposited into the reserve to issue stablecoins valuing less than the deposited collateral, hence 'over-collateralized'. Unlike crypto-collateralized stablecoins, the value held in their reserves are guaranteed to be greater than or equal to the stablecoin supply in normal working conditions. This property is derived from liquidation events in the lending protocol backbone if the value of the deposited basket of cryptocurrencies risk falling below the outstanding stablecoin debt. In this paper, we characterize over-collateralized stablecoins with the following characteristics: 1) Their reserves are composed of cryptocurrencies. 2) Redemption returns a basket of cryptocurrencies valued higher than the redeemed stablecoins. 3) There exists a liquidation mechanism ensuring that the value of the reserve is higher than the stablecoin supply. 4) The stablecoin protocol has no control over expansion and contraction of the supply of the redeemed currencies.
%Most recently, Terra, which was one of the most representative stablecoins, collapsed. 

