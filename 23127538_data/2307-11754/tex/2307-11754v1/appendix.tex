\newpage

\section{Proofs}

\subsection{Proof of Theorem~\ref{thm:unique}}
\label{app:proof1}

First, we will show that the conditions that 
\begin{equation}
    \max\{v,i(v^\prime),i(p(M^\prime))\}>p(M) \text{ if } p(M)<1
    \label{eq:first}
\end{equation} and 
\begin{equation}
    \max\{v,i(v^\prime),i(p(M^\prime))\}\geq 1 \text{ if } p(M)=1
    \label{eq:second}
\end{equation} are sufficient to ensure the reachable and unique pegging equilibrium. 
We prove that, under this condition, the state $p(M)=1$ is an equilibrium. Consider user $u_i$ who decided not to sell its coins to the market. Then its payoff would be $v$ or $\max\{i(p(M^\prime)),i(v^\prime)\}.$ Due to Eq.~\eqref{eq:second}, both $v$ and $\max\{i(p(M^\prime)),i(v^\prime)\}$ are not smaller than $p(M)=1$.
Therefore, the rational choice of $u_i$ is not to change its action to selling its coins to the market. 
Moreover, considering users who decided to sell their coins to the market, even if they change their action, it would increase $p(M)$, but the maximum value of $p(M)$ is 1 according to our assumption described in Section~\ref{sec:model}. Therefore, $p(M)$ would also not change from $1$ in this case.
As a result, $p(M)=1$ is an equilibrium under the given conditions.

Next, we prove that $p(M)<1$ is not an equilibrium under the given conditions. Here, we consider user $u_j$ who decided to sell its coins to the market. Because of Eq.~\eqref{eq:first}, the user should change its action to redeeming coins to the system or to keeping holding coins to increase its payoff, which increases $p(M).$
Therefore, $p(M)<1$ cannot be an equilibrium, and $p(M)=1$ is a unique and reachable equilibrium. 

Moreover, we show that Eq.~\eqref{eq:first} is necessary to have the reachable, unique pegging equilibrium. 
We assume that there are some values of $p(M)$ ($<1$) such that $\max\{v,i(v^\prime),i(p(M^\prime))\}<p(M)$.
Here, we denote the value of $p(M)$ by $y.$
Then when $p(M)=y,$ it cannot reach $1$ because it is the optimal decision of users to sell coins in the market, which cannot decrease the market supply $M$ of the stablecoin. 
As a result, Eq.~\eqref{eq:first} is necessary to have a reachable and unique pegging equilibrium.
$\qed$

\subsection{Proof of Theorem~\ref{thm:fiat}}
\label{app:proof2}

When $T^s$ denotes the total supply of stablecoins, $Q$ is always less than or equal to $T^s.$
Then, in a fully backed fiat-collateralized system, $V^f$ is always greater than or equal to $Q$ because $V^f \geq  T^s$.
Therefore, according to Theorem~\ref{thm:unique}, the system has a unique equilibrium as a pegging state, because $v$ and $v^\prime$ that are 1 are always greater than $p(M)$ when $p(M)$ is less than 1. 

Meanwhile, a system partially backing its stablecoin (i.e., $V^f < T^s$) has multiple equilibria. 
If users believe that $Q$ will not be greater than $V^f$, they will act in the direction of inducing the state $p(M)=1$ because their rational action is not only redeeming but also holding coins. Note that in this case, users expect to earn $1$ even if they decide to keep their coins because $v^\prime$ would be expected to be $1.$ However, if they believe $Q$ will be greater than $V^f$, they will act in the direction of inducing the depegging state $p(M)<1$. In that case, their rational action is to redeem their coins to the system before $Q>V^f,$ which vindicates their beliefs by depleting reserves (i.e., $V^f=0$). This also leads to $v^\prime=0.$  
If $V^f/Q<e(\theta)$, the price $p(M)$ would reach $e(\theta)$. On the other hand, if $V^f/Q>e(\theta)$, users have a rational action as coin redemption, which decreases the value of $V^f/Q.$
As a result, the equilibrium price at that time would be $e(\theta).$
$\qed$

\subsection{Proof of Theorem~\ref{thm:crypto}}
\label{app:proof3}

To ensure the peg, a system needs to satisfy $r^c\left(Q, \theta\right)>p(M)$ for any $p(M)<1$ according to Theorem~\ref{thm:unique}. That is, $r^c\left(Q, \theta\right)\geq 1$ is needed.
Let $\overline{\theta}$ and  $\underline{\theta}$ be a value such that $r^c\left(T^s,\overline{\theta}\right)=1$ and $r^c\left(0,\underline{\theta}\right)=1$, respectively. 
For crypto-collateralized stablecoins, we assume that  $V^c(\theta)=T^s$ at $\theta=\theta^\circ$, where $\theta^\circ$ is less than $\overline{\theta}$ and $\underline{\theta}$ according to the assumption that $V^c(\underline{\theta})\geq T^s$.
For any $\theta\geq \overline{\theta},$ because $v$ and $v^\prime$ are always greater than $p(M)$ when $p(M)<1$, there exists a unique pegging equilibrium $p(M)=1$ according to Theorem~\ref{thm:unique}.  

On the other hand, in crypto-collateralized stablecoins, if $r^c\left(0, \theta\right)< 1,$ $v\,(=r^c(Q,\theta), \text{ or } r^c(Q,\theta) \cdot V^c(\theta)/Q \text{ in the case where } V^c(\theta)<Q)$ is less than $p(M)$ for some $p(M)<1$.
Therefore, for any $\theta<\underline{\theta}$, %the mechanism cannot increase $p(M)$ when $p(M)$ is between $\max\{v,i(p(M^\prime))\}$ and 1.
the optimal choice of users would be to sell their coins in the market when $p(M)=1$, which decreases the market price from 1. 
As a result, the pegging state $p(M)=1$ is not an equilibrium in that range where $\theta<\underline{\theta}$. 

Moreover, if $\theta<\underline{\theta}$, a depegging state becomes an equilibrium. 
In a crypto-collateralized stablecoin, for given $\theta\, (<\underline{\theta}),$ if $r^c(0,\theta)\leq e(\theta),$ the state $p(M)=e(\theta)$ would be an equilibrium because $v$ is always equal to or less than $e(\theta)$.
Moreover, for $\theta$ in the range $\theta^\circ \leq \theta < \underline{\theta}$, $v$ is always $r^c(Q,\theta)$. 
Then, within the given range of $\theta,$ if $r^c(0,\theta)> e(\theta),$ there is $y \,(\geq e(\theta))$ such that the state $p(M)=y$ becomes an equilibrium, where the value of $y$ depends on users' belief on $Q.$ For example, if users believe that a value of $Q$ will be $Q_1$ such that $r^c(Q_1,\theta)> e(\theta),$ the value of $y$ would be $r^c(Q_1,\theta).$ This is because users would not change their actions in the state $p(M)=r^c(Q_1,\theta)$, as the payoff for all actions is the same due to $p(M^\prime)=r^c(Q_1,\theta)$ according to the assumption that users' expected future can  approximate the current state. 
Alternatively, if users believe that $Q$ will have a value as $Q_2$ such that $r^c(Q_2,\theta)\leq e(\theta),$ the value of $y$ would be $e(\theta)$ because $v$ is not greater than $e(\theta).$ 

If $\theta< \theta^\circ$ and $r^c(0,\theta)> e(\theta),$ there are multiple depegging equilibria including  $p(M)=y \, (>e(\theta))$ and $p(M)=e(\theta)$, depending users' belief. 
If users believe that not many stablecoin holders will redeem their coins (i.e., $Q_1< V^c(\theta)$ and  $r^c(Q_1,\theta)> e(\theta)$), the users with the expectation do not need to redeem their coins right not, which induces the state $p(M)=r^c(Q_1,\theta).$
In contrast, if users believe that many holders try to redeem coins (i.e., $r^c(Q,\theta)\cdot V^c(\theta)/Q\leq e(\theta)$), the users' rational action is to redeem coins before $r^c(Q,\theta)\cdot V^c(\theta)/Q\leq e(\theta)$, which, in ends, leads to the state $p(M)=e(\theta)$. 

On the other hand, unlike crypto-collateralized stablecoins, we do not need to differentiate two ranges of $\theta$, 
$\theta^\circ \leq \theta < \underline{\theta}$ and $\theta < \theta^\circ$, to find depegging equlibria in algorithmic stablecoins for any $\theta<\underline{\theta}$ because $v$ is always $r^c(Q,\theta).$ Therefore, the algorithmic stablecoins also have multiple depegging equlibria according to the process above.
% the depegging equilibrium is the state of $p(M)=\max(r^c(Q,\theta), i(p(M^\prime)))$ for any $\theta < \underline{\theta}$.

Lastly, in the range $\underline{\theta}\leq\theta<\overline{\theta}$, systems have multiple equilibria, including the pegging state, depending on users' belief, because whether $r^c(Q, \theta)$ is less than 1 depends on $Q.$  Specifically, if users believe $Q$ is small enough so that $r^c(Q, \theta)\geq 1,$ $p(M)=1$ would be an equilibrium. Otherwise, a depegging state would be an equilibrium.
Therefore, users' belief in whether only a few holders will redeem their coins will determine the system's future. $\qed$

\subsection{Proof of Theorem~\ref{thm:over}}
\label{app:proof4}

Let $\underline{\theta}$ be a value satisfying $r^c(0,\underline{\theta})\cdot o(\underline{\theta})=1$.
The notation $\theta^\circ$ denotes the maximum value of $\theta$ such that $D^L(\theta)=T^s$. 
%Also, we use $\theta^\triangle$ to denote a value of $\theta$ such that $r^c(0,\theta)\cdot o(\theta)=1$. 
Note that $\underline{\theta}>\theta^\circ$, according to the assumption that $r^c(0,\theta)\cdot o(\theta)<1$ if $D^L(\theta)=T^s$. 
In other words, $D^L(\theta)<T^s$ for any $\theta\geq\underline{\theta}$.

In that range of $\theta\geq\underline{\theta}$, there are multiple equilibria including the pegging state, by users' self-fulfilling belief in others' redemption actions, because whether $v\,(=r^c(Q,\theta)\cdot o(\theta) \text{ or } 0)$ is less than 1 depends on $Q$ and whether a user is a \textit{good} stablecoin debtor. Here, good debtors mean users whose collateral did not enter a liquidation process.
%in that case, $Q$ would be $D^L(\theta)+Q_d$, where $Q_d$ is the total quantity of stablecoins that \textit{good} debtors redeem. 
To describe this in more detail, we first introduce the notation $\theta^\star$ that indicates a value of $\theta$ such that $r^c(T^s,\theta)\cdot o(\theta)=1.$ If $\theta<\theta^\star,$ whether $r^c(Q,\theta)\cdot o(\theta)$ is less than 1 depends on $Q,$ which leads to multiple equilibria including the self-fulfilling pegging equilibrium. This can be proven similar to Theorem~\ref{thm:crypto}. 

On the other hand, if $\theta\geq \theta^\star,$ whether $v$ is $r^c(Q,\theta)\cdot o(\theta)$ or $0$ is up to whether a user is a good stablecoin debtor. And users would decide their rational behavior depending on their belief in good debtors' redemption. Good debtors should redeem (i.e., pay back) their stablecoin debts optimally considering a coin price. After they redeem all of their debts or their collateral triggers liquidation, their $v$'s value becomes 0, changing the status of whether they are a good debtor. This makes the debtors choose coin redemption only when the action can give them the maximum benefit. Consequently, they would redeem coins when the price is low since the payoff difference between $v$ and $p(M)$ is large in that case. 
Given this, if good debtors believe that other debtors will redeem their coins now, they expect the coin price will increase, which encourages the debtors to redeem their coins now. Therefore, this vindicates their belief and makes the state $p(M)=1$ an equilibrium. Otherwise, if they believe that other debtors will not redeem their debts now, their rational action would be not to redeem coins now, which fulfills their belief.  As a result, for $\theta\geq \theta^\star,$ there are multiple equilibria, including the self-fulfilling pegging equilibrium. 

The value of $r^c(0,\theta)\cdot o(\theta)$ is less than 1 for any $\theta<\underline{\theta}.$
Therefore, in that case, $v\,(=r^c\left(Q, \theta\right)\cdot o(\theta) \text{ or } 0)$ is always less than $1$.
As shown in the proof of Theorem~\ref{thm:crypto}, it implies that there cannot be a pegging equilibrium. Moreover, similar to the proof of Theorem~\ref{thm:crypto}, it can be proven that there are depegging equilibria in that range of $\theta.$

Next, we show that $\underline{\theta}$ of over-collateralized stablecoins is smaller than that for crypto-collateralized and algorithmic stablecoins. In crypto-collateralized and algorithmic stablecoins, $\underline{\theta}$ is a value such that $r^c(0,\underline{\theta})=1$ according to Theorem~\ref{thm:crypto}.
In over-collateralized stablecoins, $\underline{\theta}$ is a value satisfying $r^c(0,\underline{\theta})\cdot o(\underline{\theta})=1$.
Because a collateral value would not drop below 1 if the corresponding asset appreciates, $r^c(0,\theta)<1$ if $o(\theta)<1.$
Therefore, $r^c(0,\theta)<1$ if $r^c(0,\theta)\cdot o(\theta)<1.$
This implies that $\theta$ should be less than $\underline{\theta}$ of crypto-collateralized and algorithmic stablecoins if it is less than $\underline{\theta}$ of over-collateralized stablecoins. 
As a result, over-collateralized stablecoins have a smaller value of $\underline{\theta}$ than that for crypto-collateralized and algorithmic stablecoins.
$\qed$

\newpage

%\noindent\begin{minipage}{\linewidth}

\section{Tables}
\label{app:tab}

\begin{table}[!ht]
\centering
\begin{tabular}{>{\centering\arraybackslash}m{0.11\textwidth}|>{\centering\arraybackslash}m{0.1\textwidth}||>{\centering\arraybackslash}m{0.15\textwidth}|>{\centering\arraybackslash}m{0.1\textwidth}||>{\centering\arraybackslash}m{0.15\textwidth}|>{\centering\arraybackslash}m{0.1\textwidth}}
        \multirow{2}{*}{\parbox{\linewidth}{\centering\vspace{1mm}Type} }&\multirow{2}{*}{\parbox{\linewidth}{\centering\vspace{1mm}Name}}& \multicolumn{2}{c||}{\makecell{Price deviation \\ from 1}}& \multicolumn{2}{c}{\makecell{Downward price \\ deviation from 1}}\\
         \cline{3-6}
         & & Value & Ranking & Value & Ranking\\
         \hline
         \hline
         \multirow{8}{*}{\parbox{\linewidth}{\centering Fiat}}&USDT & 6.1961$\times 10^{-4}$ & 3 & 2.7112$\times 10^{-4}$ & 1\\
         \cline{2-6}
         &USDC & 4.1747$\times 10^{-4}$ & 1 & 2.8842$\times 10^{-4}$ & 2\\
         \cline{2-6}
         &BUSD & 6.5501$\times 10^{-4}$ & 5 & 4.1341$\times10^{-4}$ & 4\\
         \cline{2-6}
         & TUSD & 4.8585$\times 10^{-4}$ &2 & 2.9634$\times 10^{-4}$ & 3 \\
         \cline{2-6}
         &USDP & 1.4753$\times 10^{-3}$ & 6 & 1.1181$\times 10^{-3}$ & 8 \\
         \cline{2-6}
         &GUSD & 6.5357$\times 10^{-3}$& 13 &5.6325$\times 10^{-3}$ & 15 \\
         \cline{2-6}
         &HUSD & 6.2961$\times10^{-4}$ & 4 & 4.4792$\times 10^{-4}$ & 5 \\
         \cline{2-6}
         &USDK & 3.0760$\times 10^{-3}$ & 8 & 1.5460$\times 10^{-3}$ & 9  \\
         \hline
         \hline
         Crypto-S$\bm{+}$Over &DAI & 1.5766$\times 10^{-3}$ & 7 & 9.3304$\times 10^{-4}$ & 7\\
         \hline
         \hline
        \multirow{5}{*}{\parbox{\linewidth}{\centering\vspace{2mm}Crypto-S}} &FRAX & 4.1901$\times10^{-3}$ & 10 & 7.7119$\times 10^{-4}$ & 6\\
         \cline{2-6}
         &FEI & 8.3035$\times10^{-3}$ & 15 & 7.9128$\times10^{-3}$ & 16\\
         \cline{2-6}
         &OUSD & 7.8163$\times 10^{-3}$ & 14 & 5.3287$\times 10^{-3}$ & 14\\
         \cline{2-6}
         &MUSD & 1.6690$\times 10^{-2}$ & 17 & 1.3002$\times 10^{-2}$&  17\\
         \cline{2-6}
         &RSV & 3.5518$\times 10^{-3}$ & 9 & 2.2750$\times 10^{-3}$ & 11 \\
         \hline
         \hline
          Crypto-NS$\bm{+}$Over & LUSD & 1.0255$\times 10^{-2}$ & 16 & 4.9520$\times 10^{-3}$& 13\\
         \hline
         \hline
          \multirow{2}{*}{\parbox{\linewidth}{\centering\vspace{0mm}Crypto-NS}}&USDN & 2.2982$\times 10^{-2}$ & 18 & 2.2933$\times 10^{-2}$ & 18\\
         \cline{2-6}
         &CUSD & 5.0588$\times 10^{-3}$ & 11 & 3.8918$\times 10^{-3}$  & 12 \\ 
         \hline
         \hline
         Algo &USTC &3.6189$\times10^{-2}$& 19 & 3.6120$\times10^{-2}$ & 19\\
         \hline
         \hline
         \multirow{4}{*}{\parbox{\linewidth}{\centering Over}}&USDX & 6.5899$\times 10^{-2}$ & 20 & 6.5859$\times 10^{-2}$ & 20\\
        \cline{2-6}
         &sUSD & 5.5589$\times 10^{-3}$ & 12 & 2.1421$\times 10^{-3}$ & 10\\
         \cline{2-6}
         &VAI & 1.1425$\times 10^{-1}$ & 22 & 1.1389$\times 10^{-1}$ & 22 \\
         \cline{2-6}
         &EOSDT & 1.0582$\times 10^{-1}$ & 21 & 9.2677$\times 10^{-2}$ & 21\\
         \hline
    \end{tabular}
    \begin{tabular}{p{\textwidth}}
    \\\vspace*{-5mm}
    \renewcommand{\arraystretch}{1} % this reduces the vertical spacing between rows
\linespread{1}\fontsize{9}{10.2}\selectfont
    *When running a t-test to rank a price deviation of stablecoins, there was no statistical significance (i.e., P-value $>$ 0.1) for the following 18 pairs out of 231 ($={}_{22}C_2$) pairs: (USDT,BUSD), (USDT, HUSD), (HUSD, BUSD), (USDP,DAI), (DAI, FRAX), (USDK,FRAX), (USDK, RSV), (RSV, FRAX), (FRAX, sUSD), (FRAX, CUSD), (CUSD, sUSD), (OUSD, USTC), (OUSD,FEI), (FEI, USTC), (LUSD, USTC), (MUSD, USTC), (USDN, USTC), (EOSDT, VAI)\\
\linespread{1}\fontsize{9}{10.2}\selectfont
    *When running a t-test to rank a downward price deviation of stablecoins, there was no statistical significance  (i.e., P-value $>$ 0.1) for the following 16 pairs out of 231 ($={}_{22}C_2$) pairs: (USDT,USDC), (USDT,TUSD),(USDC, TUSD),(BUSD, HUSD),(FRAX,DAI), 
    (DAI,USDP), (DAI, USDK), (USDP, USDK), (sUSD, RSV), (CUSD, LUSD), (LUSD,GUSD), (LUSD,OUSD), (OUSD, GUSD), (FEI, USTC), (MUSD, USTC), (USDN, USTC)
    \end{tabular}
    \vspace{2mm}
    \captionof{table}{Price volatility and downward price volatility of stablecoins given daily price data from May 13, 2021 to May 12, 2022.}
    \label{tab:price}    
\end{table}