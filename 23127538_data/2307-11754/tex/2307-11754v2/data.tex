\section{Empirical Analysis}
\label{sec:data}

We found that existing stablecoin designs have different price equilibria by their unique value of $v.$ % and $v^\prime$.
Here, we conduct an extensive empirical analysis using observational data to see whether actual stablecoin prices have fluctuated in agreement with our theory. To this end, we first compare the stability levels of stablecoins and examine the relationship between a stablecoin price and $v$. We publish the data in \cite{data}.

\subsection{Top 22 stablecoins}

In the analysis, we considered stablecoins that aim to be pegged at USD, and that existed for one year before the UST downfall (i.e., May 13, 2021, to May 12, 2022). 
We selected, based on market cap, the top 22 stablecoins among the ones satisfying the above criteria, and then collected their daily price data from CoinMarketCap~\cite{coinmarketcap}. 
Table~\ref{tab:coin} presents our analysis targets, their market cap, and some statistics of a daily price. 
In the table, the fiat-collateralized stablecoins are marked as \textit{Fiat}. \textit{Crypto-S} and \textit{Crypto-NS} indicate a crypto-collateralized system that uses other stablecoins and non-stablecoins as collateral, respectively. 
\textit{Algo} and \textit{Over} indicate an algorithmic and over-collateralized stablecoin, respectively. 
Some stablecoins combine different pegging mechanisms.
For example, DAI allows users to swap the coin with other stablecoins, such as USDC, in a 1:1 ratio, while having a loan mechanism to maintain the peg~\cite{dai_psm}.
LUSD also runs crypto-collateral and over-collateral mechanisms simultaneously~\cite{lusd_peg}. 

\begin{table}[!ht]
\centering
    \begin{tabular}{>{\centering\arraybackslash}m{0.13\textwidth}|>{\centering\arraybackslash}m{0.21\textwidth}|>{\centering\arraybackslash}m{0.12\textwidth}|>{\centering\arraybackslash}m{0.1\textwidth}|>{\centering\arraybackslash}m{0.1\textwidth}|>{\centering\arraybackslash}m{0.1\textwidth}} 
       Name & Type & Market Cap. & Avg. & Min & Max \\
         \hline
         \hline
         USDT & Fiat & \$67.56B & 1.0004 & 0.9959 & 1.0019 \\
         \hline
         USDC & Fiat & \$51.72B & 1.0000 & 0.9982 & 1.0016 \\
         \hline
         BUSD & Fiat & \$19.91B & 1.0001 & 0.9981 & 1.0037 \\
         \hline
         DAI & Crypto-S$\bm{+}$Over & \$6.87B & 1.0004 & 0.9878 & 1.0098 \\
         \hline
         TUSD & Fiat & \$1.62B & 1.0000 & 0.9985 & 1.0024\\
         \hline
         FRAX* & Crypto-S & \$1.48B & 1.0003 & 0.9871 & 1.0682  \\
         \hline
         USDP & Fiat & \$945.18M & 1.0001 & 0.9907 & 1.0060  \\
         \hline
         USTC** & Algo & \$637.41M & 0.9977 & 0.4086 & 1.0098  \\
         \hline
         USDN* & Crypto-NS & \$629.60M & 0.9866 & 0.7831 & 1.0157 \\
         \hline
         FEI & Crypto-S & \$422.23M & 0.9964 & 0.9468 & 1.0127  \\
         \hline
         GUSD & Fiat & \$363.84M  & 0.9972 & 0.9656 & 1.0319 \\
         \hline
         LUSD & Crypto-NS$\bm{+}$Over & \$174.97M & 1.0039 & 0.9515 & 1.0415 \\
         \hline
         HUSD & Fiat & \$160.19M & 1.0000 & 0.9976 & 1.0026  \\
         \hline
         USDX & Over & \$105.40M & 0.9676 & 0.6677 & 1.0203 \\
         \hline
         sUSD & Over & \$77.24M & 1.0002 & 0.9899 & 1.0276 \\
         \hline
         VAI & Over & \$54.92M & 0.8972 & 0.7408 & 1.0963 \\
         \hline
         CUSD & Crypto-NS & \$51.63M & 0.9991 & 0.9847 & 1.0230 \\
         \hline
         OUSD & Crypto-S & \$48.27M & 0.9985 & 0.9772 & 1.0481 \\
         \hline
         MUSD & Crypto-S & \$41.92M & 1.0024& 0.9111 & 1.0578  \\
         \hline
         RSV & Crypto-S & \$28.84M & 0.9996 & 0.9867 & 1.0196 \\
         \hline
         USDK & Fiat & \$28.82M & 1.0013 & 0.9795 & 1.0230 \\
         \hline
         EOSDT & Over & \$2.17M & 0.9625 & 0.4313 & 1.9084 \\
         \hline
    \end{tabular}
    \begin{tabular}{p{\textwidth}}
    \\\vspace*{-5mm}
    \renewcommand{\arraystretch}{1} % this reduces the vertical spacing between rows
    \linespread{1}\fontsize{9}{10.2}\selectfont
   *Although FRAX says that they are an algorithmic stablecoin, here we mark FRAX as Crypto-S, because it uses the algorithmic pegging mechanism only when the crypto-collateralized mechanism using other stablecoins does not work.\\
\linespread{1}\fontsize{9}{10.2}\selectfont
    **Here, USTC means UST before May 12, 2022. 
    \end{tabular}
    \vspace{2mm}
    \captionof{table}{Statistics of daily price data considering the time period from May 13, 2021, to May 12, 2022}
    \label{tab:coin}    
\end{table}


\subsection{Stability levels of stablecoins}
\label{subsec:stability}

Given that a time point corresponds to one value of $\theta$, we can compare the actual stability levels of stablecoins with the ones expected by our theory.
We evaluate the stability level of stablecoins by analyzing how close their price has been to \$1 (or how much their price has deviated from \$1).
We use two metrics to measure a (downward) price deviation from 1.
The metric to estimate a price deviation is defined as  $\sqrt{\frac{\sum_{i=0}^{N-1} (p_i-1)^2}{N}}$, where $p_i$ indicates one price data point in the data set. 
That is, it means a standard deviation of price from 1.
The metric to estimate downward price deviation is defined as $\sqrt{\frac{\sum_{i=0}^{N-1} (\min(p_i-1,0))^2}{N}}$, which implies a price deviation considering only when a price falls below 1.
For FRAX, unlike others, it aims to keep the price within the range of \$0.9933 to \$1.0033 rather than \$1.
Therefore, to calculate its price deviation and downward price deviation from the target range $[0.9933,1.0033]$, we apply $\sqrt{\frac{\sum_{i=0}^{N-1} \min_{x\in[0.9933,1.0033]}(p_i-x)^2}{N}}$ and $\sqrt{\frac{\sum_{i=0}^{N-1} (\min(p_i-0.9933,0))^2}{N}}$, respectively.
Moreover, we ran an independent t-test for all pairs of stablecoins to rank the stablecoins statistically.
Table~\ref{tab:price} in Appendix~\ref{app:tab} presents the results. 
Figures~\ref{fig:volatility} and \ref{fig:volatility2} show price deviation and downward price deviation from 1 by stablecoin type, respectively.

% Figure environment removed

Through the table and the figures, we can see that fiat-collateralized stablecoins (Fiat) and crypto-collateralized stablecoins (Crypto-S) using other stablecoins have been relatively stable. 
Over-collateralized stablecoins (Over) have been overall unstable, but there is also a large deviation across the samples. 
Note that the yellow zone that over-collateralized stablecoins possess largely in Figure~\ref{fig:range} has multiple equilibria, which makes it not easy to predict a price state.
On the other hand, because Crypto-NS and Algo categories have only one or two samples, it is difficult to compare them with other types meaningfully. 
However, we find that some algorithmic stablecoins have collapsed like UST or have changed their pegging mechanism; for example, Nubits have disappeared by not overcoming a severe peg break~\cite{nubits}. 
USDD also changed its original algorithmic pegging mechanism to the crypto-collateralized mechanism using combination of stablecoins and other cryptocurrencies~\cite{usdd,usdd2}.
Therefore, one may suspect that the general poor stability of algorithmic stablecoins results in the small sample number.

Another point we can see is that Crypto-S+Over and Crypto-NS+Over are observed to be more stable than Crypto-S and Crypto-NS, respectively.\footnote{Here, we would not claim the observation is statistically valid due to a small sample number.} In fact, this is also expected by our theoretical analysis. Theorem~\ref{thm:over} states that $\underline{\theta}$ of over-collateralized stablecoins is smaller than that for crypto-collateralized, which implies that crypto-collateralized stablecoins can reduce the red zone by combining with an over-collateralized mechanism. In addition, the combination of crypto-collateralized and over-collateralized systems does not affect the blue zone because all users can redeem their stablecoins in any case through the crypto-collateralized mechanism. As a result, the price stability of Crypto-S+Over and Crypto-NS+Over is empirically observed in agreement with our theory. 

% \begin{table}[]
%     \centering
%     \begin{tabular}{>{\centering\arraybackslash}m{0.1\textwidth}|>{\centering\arraybackslash}m{0.1\textwidth}||>{\centering\arraybackslash}m{0.05\textwidth}|>{\centering\arraybackslash}m{0.05\textwidth}|>{\centering\arraybackslash}m{0.05\textwidth}||>{\centering\arraybackslash}m{0.05\textwidth}|>{\centering\arraybackslash}m{0.05\textwidth}|>{\centering\arraybackslash}m{0.05\textwidth}||>{\centering\arraybackslash}m{0.06\textwidth}|>{\centering\arraybackslash}m{0.05\textwidth}|>{\centering\arraybackslash}m{0.05\textwidth}||>{\centering\arraybackslash}m{0.06\textwidth}|>{\centering\arraybackslash}m{0.05\textwidth}|>{\centering\arraybackslash}m{0.05\textwidth}||>{\centering\arraybackslash}m{0.06\textwidth}|>{\centering\arraybackslash}m{0.05\textwidth}|>{\centering\arraybackslash}m{0.05\textwidth}}
    
%         \multirow{2}{*}{\parbox{\linewidth}{\centering\vspace{1mm}Type} }&\multirow{2}{*}{\parbox{\linewidth}{\centering\vspace{1mm}Name}}& \multicolumn{3}{c||}{\$0.9}& \multicolumn{3}{c||}{\$0.95} & \multicolumn{3}{c||}{\$0.99} & \multicolumn{3}{c||}{\$0.999} & \multicolumn{3}{c}{\$1}\\
%          \cline{3-17}
%          & & Avg. & Min & Max & Avg. & Min & Max & Avg. & Min & Max & Avg. & Min & Max& Avg. & Min & Max\\
%          \hline
%          \hline
%          \multirow{8}{*}{\parbox{\linewidth}{\centering Fiat}}&USDT & 0 & 0 & 0 & 0 & 0 & 0 & 0 & 0 & 0 & 1.5 & 1 & 2 & 1.6 & 1 & 8\\
%          \cline{2-17}
%          &USDC & 0 & 0 & 0 & 0 & 0 & 0 & 0 & 0 & 0 & 1.0 & 1 & 1 & 2.5 & 1 & 13\\
%          \cline{2-17}
%          &BUSD & 0 & 0 & 0 & 0 & 0 & 0 & 0 & 0 & 0 & 1.2 & 1 & 2 & 2.2 & 1 & 6\\
%          \cline{2-17}
%          & TUSD & 0 & 0 & 0 & 0 & 0 & 0 & 0 & 0 & 0 & 1.0 & 1 & 1 & 2.2 & 1 & 10\\
%         \cline{2-17}
%         &USDP & 0 & 0 & 0 & 0 & 0 & 0 & 0 & 0 & 0 & 1.9 & 1 & 8 & 2.5 & 1 & 11\\
%          \cline{2-17}
%          &GUSD & 0 & 0 & 0 & 0 & 0 & 0 & 1.5 & 1 & 5 & 4.5 & 1 & 55 & 7.5 & 1 & 72\\
%          \cline{2-17}
%          &HUSD & 0 & 0 & 0 & 0 & 0 & 0 & 0 & 0 & 0 & 1.2 & 1 & 2 & 2.0 & 1 & 6\\
%          \cline{2-17}
%          &USDK & 0 & 0 & 0 & 0 & 0 & 0 & 1.0 & 1 & 1 & 1.6 & 1 & 7 & 3 & 1 & 13 \\
%          \hline
%          \hline
%          Crypto-S$\bm{+}$Over&DAI & 0 & 0 & 0 & 0 & 0 & 0 & 1.0 & 1 & 1 & 1.5 & 1 & 3 & 2.8 & 1 & 17\\
%          \hline
%          \hline
%         \multirow{5}{*}{\parbox{\linewidth}{\centering\vspace{2mm}Crypto-S}} &FRAX & 0 & 0 & 0 & 0 & 0 & 0 & 1.2 & 1 & 2 & 1.8 & 1 & 14 & 2.2 & 1 & 14\\
%          \cline{2-17}
%          &FEI & 0 & 0 & 0 & 1.0 & 1 & 1 & 2.2 & 1 & 8 & 5.5 & 1 & 55 & 2.9 & 1 & 53\\
%          \cline{2-17}
%          &OUSD & 0 & 0 & 0 & 0 & 0 & 0 & 2.1 & 1 & 5 & 8.4 & 1 & 55 & 10.3 & 1 & 106 \\
%          \cline{2-17}
%          &MUSD & 0 & 0 & 0 & 1.4 & 1 & 2 & 2.1 & 1 & 4 & 1.8 & 1 & 7 & 2.3 & 1 & 24 \\
%          \cline{2-17}
%          &RSV & 0 & 0 & 0 & 0 & 0 & 0 & 1.0 & 1 & 1 & 2.9 & 1 & 28 & 3.8 & 1 & 76\\
%          \hline
%          \hline
%          Crypto-NS$\bm{+}$Over &LUSD & 0 & 0 & 0 & 0 & 0 & 0 & 1.4 & 1 & 4 & 2.5 & 1 & 26 & 2.6 & 1 & 26\\
%          \hline
%          \hline
%           \multirow{2}{*}{\parbox{\linewidth}{\centering\vspace{0mm}Crypto-NS}}&USDN & 2.0 & 2 & 2 & 3.0 & 2 & 4 & 10.0 & 1 & 64 & 19.9 & 1 & 198 & 33.4 & 1 & 199\\
%          \cline{2-17}
%          &CUSD & 0 & 0 & 0 & 0 & 0 & 0 & 1 & 1 & 1 & 5.4 & 1 & 38 & 5.7 & 1 & 39 \\
%          \hline
%          \hline
%          Algo &USTC & 4.0 & 4 & 4 & 2.5 & 1 & 4 & 5.0 & 4 & 6 & 2.2 & 1 & 11 & 2.3 & 1 & 14\\
%          \hline
%          \hline
%          \multirow{4}{*}{\parbox{\linewidth}{\centering Over}}&USDX & 8.0 & 2 & 20 & 4.9 & 1 & 21 & 13.2 & 1 & 120 & 24.5 & 1 & 127 & 31.2 & 1 & 127\\
%         \cline{2-17}
%          &sUSD & 0 & 0 & 0 & 0 & 0 & 0 & 1.0 & 1 & 1 & 2.4 & 1 & 17 & 3.0 & 1 & 17\\
%          \cline{2-17}
%          &VAI & 7.6 & 1 & 78 & 69 & 1 & 164 & 176.5 & 142 & 211 & 118.3 & 1 & 211 & 118.3 & 1 & 211 \\
%          \cline{2-17}
%          &EOSDT & 1.7 & 1 & 10 & 2.3 & 1 & 19 & 3.3 & 1 & 32 & 4.3 & 1 & 32 & 4.7 & 1 & 32 \\
%          \hline
%     \end{tabular}
%     \vspace{2mm}
%     \caption{The period (days) to return to the peg given daily price data from May 13,
% 2021 to May 12, 2022.}
%     \label{tab:period}
% \end{table}

\subsection{Relationship between a price and $v$}

According to our theory, a stablecoin price should be depegged when $v$ is less than \$1, which should bring a correlation and causality between a price and $v$. 
We collected and explored actual redemption transactions of one year (05/13/2021$\sim$05/12/2022) to analyze the relationship between $v$ and a stablecoin price.
On-chain data collection was prioritized by high market cap, diversity of design, and ease of access to data.
In the process, we were able to collect on-chain data for a total of eleven stablecoins from five different blockchains, Ethereum, Terra, BSC, Celo, and Waves: DAI, FRAX, FEI, OUSD, MUSD, LUSD, USDN, CUSD, USTC\footnote{For USTC, we were able to collect redemption data for Oct 01, 2021 to May 12, 2022.}, sUSD, and VAI. 
The value of $v$ was calculated considering the quantity of the asset earned when users redeemed their stablecoins and the asset's market price. 
Because we were able to obtain only daily price data (i.e., end of day asset price) in USD, we used the last redemption transaction data for each day to reduce errors when calculating $v$.
On the other hand, for DAI and LUSD, we used the average value of $v$ for each day because they mix different pegging mechanisms. 
Here, note that we did not consider and subtract transaction fees by assuming that the difference between redemption and market transaction fees is negligible and by offsetting all the fees.\footnote{In fact, if we take into account transaction fees, the user payoff should be 1 minus transaction fees even when users sell a stablecoin in the exchange market.}
Therefore, for fiat-collateralized stablecoins, $v$ would be 1.
%taking into account transaction fees and protocol fees, with the assumption that users realized the collaterals into US Dollars at the end of the day that each redemption took place.

We first analyze the correlation between the downward $v$ deviation and the downward price deviation of stablecoins. 
Figure~\ref{fig:corr} shows that, considering 19 stablecoins (11 stablecoins, of which $v$ was actually collected, along with 8 fiat-collateralized stablecoins), there is a significantly strong correlation (Pearson's $\rho\approx$ 0.7150, p-value$\approx$0.0006, Bayes factor$\approx$0.0165\footnote{This value represents ``very strong evidence'' for the correlation~\cite{bayes}.}) between the downward $v$ deviation and the downward price deviation. 
%Note that fiat-collateralized stablecoins have zero $v$ deviation because $v$ is always \$1.

% Figure environment removed

\begin{table*}[ht]
    \centering
    \begin{tabular}{>{\centering\arraybackslash}m{0.12\textwidth}|>{\centering\arraybackslash}m{0.1\textwidth}||>{\centering\arraybackslash}m{0.1\textwidth}|>{\centering\arraybackslash}m{0.1\textwidth}||>{\centering\arraybackslash}m{0.1\textwidth}|>{\centering\arraybackslash}m{0.1\textwidth}}
        \multirow{2}{*}{\parbox{\linewidth}{\centering\vspace{1mm}Type} }&\multirow{2}{*}{\parbox{\linewidth}{\centering\vspace{1mm}Name}}& \multicolumn{2}{c||}{Correlation}& \multicolumn{2}{c}{Granger causality}\\
         \cline{3-6}
         & & Rho & P-value & F-stats.  & P-value\\
         \hline
         \hline
         Crypto-S$\bm{+}$Over&\cellcolor{YellowGreen}DAI & \cellcolor{YellowGreen}0.1499 & \cellcolor{YellowGreen}0.0136 & \cellcolor{YellowGreen}5.8753 & \cellcolor{YellowGreen}0.0160\\
         \hline\hline
          \multirow{4}{*}{\parbox{\linewidth}{\centering Crypto-S}}&FRAX & 0.1833 & 0.2576 & 1.1987 & 0.2809\\
          %https://app.frax.finance/
          \hhline{~|*5{-}|}
         &\cellcolor{YellowGreen}FEI & \cellcolor{YellowGreen}0.1845 & \cellcolor{YellowGreen}0.0028 &  \cellcolor{YellowGreen}5.2799& \cellcolor{YellowGreen}0.0224\\
         \hhline{~|*5{-}|}
         &OUSD & 0.0934 & 0.2097 & 0.9743 & 0.3250 \\
         \hhline{~|*5{-}|}
         &MUSD & -0.0986 & 0.0620 & 1.2057 & 0.2729\\
         \hline
         \hline
         Crypto-NS$\bm{+}$Over &\cellcolor{YellowGreen}LUSD &\cellcolor{YellowGreen}0.3248& \cellcolor{YellowGreen}$<$0.0001 & \cellcolor{YellowGreen}30.1870 & \cellcolor{YellowGreen}$<$0.0001\\
         \hline
         \hline
          \multirow{2}{*}{\parbox{\linewidth}{\centering\vspace{0mm}Crypto-NS}}&\cellcolor{YellowGreen}USDN &  \cellcolor{YellowGreen}0.4914 & \cellcolor{YellowGreen}$<$0.0001 & \cellcolor{YellowGreen}76.9957 & \cellcolor{YellowGreen}$<$0.0001 \\
         \hhline{~|*5{-}|}
         &\cellcolor{YellowGreen}CUSD & \cellcolor{YellowGreen}0.1341 & \cellcolor{YellowGreen}0.0104 & \cellcolor{YellowGreen}12.3066 & \cellcolor{YellowGreen}0.0005  \\ 
         \hline
         \hline
         Algo &\cellcolor{YellowGreen}USTC & \cellcolor{YellowGreen} 0.8366 & \cellcolor{YellowGreen} $<$0.0001 & \cellcolor{YellowGreen}88.7618 & \cellcolor{YellowGreen}$<$0.0001\\
         \hline
         \hline
         \multirow{2}{*}{\parbox{\linewidth}{\centering Over}} &\cellcolor{YellowGreen}sUSD & \cellcolor{YellowGreen}0.7677 & \cellcolor{YellowGreen}$<$0.0001 &\cellcolor{YellowGreen} 44.1318 & \cellcolor{YellowGreen}$<$0.0001\\
         \hhline{~|*5{-}|}
         &VAI & -0.0200 & 0.7560 & 9.9877 & 0.0018\\
         \hline
    \end{tabular}
    \vspace{2mm}
    \caption{Correlation and Granger causality between a stablecoin price and $v$}
    \label{tab:v_correlation}
\end{table*}

In addition, for each stablecoin, we performed the correlation analysis between $v$ and a stablecoin price. 
Specifically, we examined the relationship between the last value of $v$\footnote{Remind that, for DAI and LUSD that combine different pegging mechanisms, we used the average value of $v$ for each day. } and the closing stablecoin price for each day, and used the minimum value between $v$ and 1 and the minimum value between a price and 1 to consider a downward fluctuation. 
We also saw if $v$ has affected the stablecoin price through the Granger causality analysis.
Table~\ref{tab:v_correlation} presents the results, where we colored coins if they have significant correlation and causality.

We can see that the significant correlation and causality between $v$ and a price are manifested in most stablecoins. 
DAI, FEI, LUSD, USDN, CUSD, USTC, and sUSD showed a significant correlation and causality. %, and  showed only significant causality. 
Meanwhile, it was not observed for FRAX, OUSD, MUSD, and VAI. 
In particular, we find that stablecoins with relatively good stability of a price and $v$ do not show a strong correlation and causality; %when conducting correlation and causality analyses; 
overall, DAI and Crypto-S have a relatively high P-value in Granger causality.
This could be resulted from the deviations caused by noise or other factors being more apparent in coins with good downward stability of $v$.
Most representatively, we can think of fiat-collateralized stablecoins where $v$ is always 1. Definitely, their price deviations do not come from $v$. 

In addition, the existence of correlation and causality between $v$ and a price can be inconsistent across over-collateralized stablecoins; in our data, sUSD showed a significantly positive correlation and causality, but VAI showed an insignificant correlation. 
As described in Section~\ref{subsec:stability}, it can be difficult to predict consequences of over-collateralized stablecoins due to their wide yellow zone.

The last important point is that stablecoin systems with a popular and large incentive protocol have little correlation and causality between a price and $v$, which Theorem~\ref{thm:unique} points out.  
In fact, even though UST showed significant correlation and causality considering its collapse event, there was no significant correlation (Pearson's $\rho=$0.0421, P-value$=$0.5369) and causality (F$=$0.8426, P-value$=$0.3597) when only considering the period (Oct 01, 2021 to May 06, 2022) that an incentive protocol, Anchor, was greatly popular. 
Note that Anchor promises to give users nearly 20\% annual percentage yield (APY), and it even held about 75\% of the total UST market cap in some cases. 

We recognize the limitation of empirical analysis to show the true causality because there are not only two variables, $v$ and a price, in the real world. In fact, it is well known that deriving the true causality is really challenging.
By analyzing the stablecoin price along with many variables other than $v$, we will be able to examine whether the true causality between $v$ and the price exists.
Nevertheless, we believe that the results of the empirical analyses confirm our theory to some extent.