\section{Related Work}
\label{sec:related}
%\vspace{-3mm}

Models explaining the (in)stability of stablecoins are relatively sparse. \cite{klages2022while,klages2021stability} model stablecoins like Dai, where issuance is based on a market for leverage. In this context, they characterize deleveraging spirals that caused instability in Dai on `Black Thursday' in March 2020. \cite{lyons2020keeps} models how stablecoins backed by 100\% reserves, such as USDC, maintain stability through arbitrage with minting and redemption.

As discussed in \cite{klages2020stablecoins}, models of bank runs \cite{diamond1983bank}, runs on currency pegs \cite{morris1998unique}, and pegged redemption money market funds \cite{parlatore2016fragility} can be applied to understand many other types of stablecoins.
%bank run = assets may cover liabilities, but some assets are illiquid and not all stablecoins can be redeemed at one point in time (but may over time)
%run on currency peg = assets partially cover liabilities and not all stablecoins can be redeemed for peg value
%Pegged redemption money market funds face similar issues if net asset value falls below the pegged redemption rate.
\cite{routledge2022currency,li2020managing} discussed and created adaptations of these types of models specific to stablecoins.
These models can be applied to many types of decentralized stablecoins (e.g., algorithmic stablecoins), after reinterpreting the protocol as the central bank, as well as custodial stablecoins, and so basic results about bank and currency runs translate to many stablecoin settings.
In the context of these models, \cite{klagesmundt2022designing} characterize many stablecoins based on the shape of redemption curves and then design a redemption curve that attains properties informed by currency peg models.

Various distinctions about decentralized stablecoin models are not as developed, however, such as modeling the fact that reserve assets are not the currency target (e.g., USD is not an on-chain asset).
We address this by proposing a new game-theoretical model to include reserve assets that change in price and further by breaking down assets backing into exogenous and endogenous assets, which have different degrees of price risk.
In addition, there is room to expand a common model to better characterize nuanced differences across different types of stablecoins, which is the approach we take in analyzing redemption mechanisms in different stablecoins.