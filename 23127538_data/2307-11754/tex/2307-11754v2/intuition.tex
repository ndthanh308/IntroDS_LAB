\vspace{-3mm}
\section{Intuition}
\vspace{-3mm}

%Here, we describe intuitions of the stablecoin model before diving into it. %We starts with a simple idea. 
%Traditionally, people have believed that users would be paid \$1 by the system when redeeming stablecoins. However, it may not be true depending on the asset type that the system uses to pay the users. 
%
While typical stablecoin systems rely on paying users the target price upon redemption, the payment that users realize can be lower depending on the reserve type.
%Fiat-collateralized systems use fiat money, whereas other types, such as crypto-collateralized, algorithmic, and over-collateralized systems use cryptocurrencies. 
%For example, consider cryptocurrencies employed in other types except fiat-collateralized systems. The price of a cryptocurrency constantly fluctuates. Even if a system pays users \$1 based on its price oracle, the actual payoff of users can be less than \$1 when a cryptocurrency depreciates, because the time difference between prices to which the system and users refer would exist due to non-zero transaction time, a discrete price oracle update within the system, etc. 
For example, consider stablecoin systems with cryptocurrency reserves. With the price volatility of the reserves, even if a system pays users \$1 based on its price oracle, the actual payoff of users can be less than \$1 when a cryptocurrency depreciates, because the time difference between prices to which the system and users refer would exist due to non-zero transaction time, a discrete price oracle update within the system, etc. 
Due to the user payoff of less than \$1, arbitrage between the stablecoin system and the secondary market would not operate even if the market price is less than \$1, which leads to the depeg.

The situation can deteriorate if a system uses weak assets to back its stablecoin. 
The system releases the assets backing the stablecoin on the market when purchasing stablecoins from users. Given this, users' redemption can decrease the price of the asset backing the stablecoin, and this influence would become greater when the asset value is largely derived from the system's usage. 
For example, recall the severe price drop of Luna due to the significant UST redemption of users in the second week of May, 2022. Back then, even though the system paid users \$1 based on its Luna price oracle, users would not actually have expected to earn \$1 by a rapid drop in Luna's price. Note that its value relies on the system's usage rather than external sources because it is an endogenous asset. 

% To escape from the depegging state, a system exploiting cryptocurrencies would need to pay more than \$1 to users when buying stablecoins from them, which makes their actual payoff not less than \$1. 
% Alternatively, it would be necessary to incentivize users to keep their stablecoins and not sell them in the market.  
%
With these intuitions in mind, we propose a game-theoretical model to quantify the risks of stablecoins.