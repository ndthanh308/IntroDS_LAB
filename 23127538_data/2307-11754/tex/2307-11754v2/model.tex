\section{Model}
\label{sec:model}

In this section, we provide a possible avenue to model a stablecoin system using a game-theoretical framework. Based on the stablecoin model, we will analyze four types of stablecoins and quantify their price (in)stability degrees later. We first present Table~\ref{tab:par} that summarizes the parameters used throughout the paper.

\begin{table*}[!h]
    \centering
    \begin{tabular}{>{\centering\arraybackslash}m{0.2\textwidth}|>{\raggedright\arraybackslash}m{0.8\textwidth}}
     %\hline
        Notation & \multicolumn{1}{c}{Definition}  \\
        \hline
        \hline
        $\theta$ & A state of fundamentals \\
        \hline
        $e(\theta)$ & A reached stablecoin price for given economic state $\theta$ without any system intervention\\
        \hline
        $M$ (or $M^\prime$) & The total quantity of coins that users choose to sell to the market at the moment (or in the future) \\
        \hline
        $p(M)$ (or $P(M^\prime)$) & A stablecoin price for given $M$ (or $M^\prime$) \\
        \hline
        $v$  (or $v^\prime$) & A value that the system pays users who redeem their stablecoins at the moment (or in the future)\\
        \hline
        $i(\cdot)$ & An incentive function for users to keep holding their stablecoins\\
        \hline
        $Q$ & The total quantity of coins whose users want to redeem at one point\\
        \hline
        $V^f$ & The total value of fiat reserves in a fiat-collateralized stablecoin\\
        \hline
        $c$ & A cryptocurrency used to back stablecoins in crypto-collateralized, algorithmic, and over-collateralized systems\\
        \hline
        $p^c_u$ & $c$'s price to which users refer\\
        \hline
        $p^c_s$ & $c$'s price to which the system assumes \\
        \hline
        $V^c(\theta)$ & A total value of crypto reserves in crypto-collateralized stablecoins\\
        \hline
        $r^c\left(Q, \theta\right)$ & A ratio between $p^c_u$ and $p^c_s$ \\
        \hline
        $D^L(\theta)$ & The total quantity of stablecoins that users should redeem in the liquidation process of over-collateralized systems at one point\\
        \hline
%        $Q^L$ & The total quantity of stablecoins that users want to redeem in the liquidation process of over-collateralized systems at one point \\
%        \hline
        $D_{u_i}(\theta)$ & The stablecoin debt of user $u_i$ that did not enter the liquidation process in over-collateralized systems \\
        \hline
%        $Q^D_{u_i}$ & A quantity of stablecoins that user $u_i$ now wants to redeem out of its debt in an over-collateralized system \\
%        \hline
        $o(\theta)$ & The system's estimated value of cryptocurrencies that an over-collateralized system pays to users who redeem their stablecoins  \\
        \hline
    \end{tabular}
    \caption{List of parameters}
    \label{tab:par}
\end{table*}

\subsection{Game-theoretical framework}
\label{subsec:system}

If a stablecoin is to be pegged at 1, the pegging state is defined as $p^s=1$, where $p^s$ indicates the current market price of the stablecoin. 
%Therefore, even an infinitesimal deviation from the target price would be considered as a depegging state.
A pegging mechanism of the stablecoin system intervenes in the exchange market to maintain the peg. 
Considering that many stablecoins are currently suffering from downward price instability rather than upward price instability, in this paper, we focus on the mechanism that recovers a price when a stablecoin depreciates below 1. 
If the price falls below 1, the mechanism tries to decrease market supply and increase market demand by incentivizing users not to sell and to buy stablecoins in the market. 
Thus, designing a proper payoff function of users would be a key to achieving price stability. 
Note that we will not consider that the stablecoin system can halt the exchange market like a circuit breaker. 

If a pegging mechanism functions effectively, $p^s$ will eventually reach its target value of 1. However, in the absence of any system intervention in the market to adjust the coin price (i.e., if a pegging mechanism gives up a price recovery), its equilibrium price is determined by the economic characteristics, referred to as the fundamental state $\theta$~\cite{morris1998unique}. A higher value of $\theta$ signifies ``stronger fundamentals''. 
In other words, a high $\theta$ indicates a favorable economic condition where assets can appreciate. 
We express the coin price without system intervention as an increasing function $e(\theta)$ of $\theta$, where the value of $e(\theta)$ is assumed to be always below 1. % to focus on when a coin price drops below 1.

% Figure environment removed

In this paper, we model a one-shot game with many users. Note that, currently, numerous users trade stablecoins in the market. 
Users can transact their coins with each other in the exchange market and issue coins from or redeem them to the system according to the pegging mechanism. 
Because we focus on the mechanism that recovers a price below $1$, it is enough to consider only redemption as an interaction between users and the system. Figure~\ref{fig:model} shows the relationship between users, the market, and the system. 

On the supply side, stablecoin holders can select their actions among 1) selling their coins in the market, 2) redeeming coins to the system, and 3) holding coins continuously. 
If they decide to sell coins to the market, their payoff would be the current stablecoin price $p^s.$ If they choose to redeem coins to the system, they are paid a value $v$ by the system, where $v$ is characterized according to each stablecoin design.
Lastly, when they keep holding their coins, their payoff depends on the future value of the stablecoin and an incentive provided by the system for users to keep holding their stablecoins (e.g., savings interest for a stablecoin). 
Here, let $i(\cdot)$ denote an incentive function for users to keep holding stablecoins, and it has a stablecoin value as input of a function.\footnote{For simplicity, we will omit a discount factor or time discounting because it does not significantly affect the results.}
The future stablecoin value can be expressed as $\max\{p^{s\prime},v^\prime\}$, where $p^{s\prime}$ indicates the price at which users will trade coins in the market in the future and $v^\prime$ indicates the redemption value that users can get from the system when redeeming coins in the future.\footnote{Note that we omitted a time variable for simplicity.}  
%would be the stablecoin value that users recognize in the future.
As a result, the payoff of users who decided to keep holding their stablecoins can be expressed as $\max\{i(p^{s\prime}),i(v^\prime)\}$ (or it is also possible to be expressed as $i\left(\max\{p^{s\prime},v^\prime\right)\})$).\footnote{In fact, users cannot be sure about the future value of the stablecoin. Therefore, in a more advanced model setting, the expected payoff would be expressed as $\int_{x} i(x)f\left(\max\{p^{s\prime},v^\prime\}=x\right)dx$ (i.e., an expected value based on the user's expectation on the future stablecoin value), where $f(\max\{p^{s\prime},v^\prime\})$ is a probability density function based on user expectation on $\max\{p^{s\prime},v^\prime\}.$ However, in this work, we simplify it, assuming users believe a specific value of the future stablecoin price for sure, because the simplification does not change the results and implications.} 
Note that $i(x)$ is equal to $x$ when the system provides no incentive for users to keep their coins (e.g., zero savings interest). Meanwhile, the greater the incentives, the greater the value of the function $i$.
% On the other hand, if the mechanism fails to defend the peg, the price 
% As a result, the user payoff would be $i(s(v,\theta))$, if the pegging mechanism fails to maintain the peg while users keep their coins.  

In terms of the demand side, potential buyers can select their actions among 1) not buying coins in the market (i.e., not joining the stablecoin market), 2) buying coins in the market and immediately redeeming the coins to the system, and 3) buying coins in the market and keeping the coins in their pocket. Note that their actions are done in the one-shot game. 
In fact, their payoff equals the above described payoff (of stablecoin holders) minus $p^s$. Specifically, if they decide not to buy coins, their payoff would be 0. On the other hand, if they decide to go for the second action, their payoff would be $v-p^s$. Lastly, if they choose the third action, their payoff would be $\max\{i(p^{s\prime}),i(v^\prime)\}-p^s$.
Given this, the game analysis would be the same with considering only the payoff of stablecoin holders on the supply side. Therefore, for simplicity, we will consider only the payoff of stablecoin holders. 

According to the basic economic theory, the stablecoin price is determined by the market supply and demand. In our model, the stablecoin price is expressed as a function of only the market supply from the above paragraph. The market supply is related to the set of all users who decide to sell their coins to the market. When $M$ denotes the total quantity of coins that the users decide to sell to the market at the moment, $p^s$ can be expressed as $p(M),$ where $p$ indicates a decreasing function that converts from $M$ to a price. 
Therefore, $p(M)$ would decrease and increase if more users choose to sell their coins in the market (i.e., an increase in $M$) and choose other actions (i.e., a decrease in $M$), respectively. 
Similarly, we will express the future stablecoin price $p^{s\prime}$ as $p(M^\prime)$, denoting the future market supply by $M^\prime.$
Lastly, we assume that $p(\cdot)$ is always equal to or less than 1 to focus on when a stablecoin depreciates below 1.

In summary, in our model, the payoff function of users is as follows. 

\begin{equation}
\text{payoff}=\begin{cases}
p(M) (=p^s) &\text{if they sell coins to the market,}\\
\vspace{2mm}
v &\text{if they redeem coins to the system,}\\
\vspace{2mm}
\max\{i(p(M^\prime)),i(v^\prime)\}  &\parbox[m]{0.5\linewidth}{if they keep holding coins}
% i(1) &\parbox[m]{0.5\linewidth}{if they keep coins and the pegging mechanism works,}\\
% \vspace{1mm}
% i\big(s(v, \theta)\big) &\parbox[m]{0.5\linewidth}{if they keep coins and the pegging mechanism does not work}
\end{cases}  
\label{eq:pay}
\end{equation}

\subsection{A unique pegging equilibrium}
\label{subsec:unique}

A stablecoin system should achieve $p(M)=1.$ Adopting $v$ and $v^\prime$ appropriately is essential to stabilize the coin price because it affects the decision of rational users. 
To guarantee the peg, there should be a reachable, unique equilibrium in which $p(M)=1.$ 
Here, that the state $p(M)=x$ is an \textit{equilibrium} indicates that a user's rational decision on whether to keep or change its action to increase their expected payoff in that state cannot change the stablecoin price $p(M)$ (i.e., the stablecoin price is fixed at $x$). %In other words, users cannot increase their expected payoff by changing their actions in the equilibrium state $p(M)=x.$ 
Moreover, we say the state is \textit{reachable} if rational actions of users make the stablecoin price $p(M)$ return to $x$ in the case where $p(M)\not=x$; for example, if $p(M)$ is less than $x,$ rational users sell fewer coins to the market, which would recover $p(M)$ to $x$ by decreasing a value of $M.$ 

A sound stablecoin system should choose a proper value of $v$ and $v^\prime$ that equilibrates only the state $p(M)=1.$ 
The following theorem presents two sufficient conditions and one necessary condition to make the pegging state (i.e., $p(M)=1$) the reachable and unique equilibrium.

\begin{theorem}
\label{thm:unique}
To have a reachable and unique equilibrium as $p(M)=1,$ the following two conditions are sufficient:
$$\max\{v,i(v^\prime),i(p(M^\prime))\}>p(M) \text{ if } p(M)<1,$$ and 
$$\max\{v,i(v^\prime),i(p(M^\prime))\}\geq 1 \text{ if } p(M)=1.$$
Moreover, the first condition is necessary. 
\end{theorem}

The theorem says that to ensure a unique pegging equilibrium, we must design a stablecoin as follows: having a sufficiently high $v$ or $v^\prime$, or designing a (large) incentive function $i(\cdot)$ for users to keep holding their coins. 

Consider when there is little incentive (i.e., $i(x)\approx x$). Then Theorem~\ref{thm:unique} implies that, to guarantee the peg, either $v$, $v^\prime$, or $p(M^\prime)$ must be greater than $p(M)$ for any $p(M) <1.$ 
In the case where the market price of the stablecoin is less than the target price, if $v$ or $v^\prime$ is high enough to satisfy this, users would redeem their coins to the system or keep holding coins, instead of selling them now in the market, which decreases $M$ and increases the market price. 
On the one hand, if users expect the stablecoin price to increase (i.e., $p(M^\prime)>p(M)$) due to the promising economic situation, they would keep holding coins rather than selling them immediately in the market, which also naturally increases the market price. However, recall that we assumed that the stablecoin price without any system intervention in the market is $e(\theta)$ ($<1$) in Section~\ref{subsec:system}. %, to focus specifically on the pegging mechanism and its effect on stabilizing the stablecoin's price. 
Therefore, in this paper, we do not consider the situation where the price will naturally increase to 1 without any pegging mechanism by an economic uptrend. %(i.e., $p(M^\prime)>p(M)$ for any $p(M)<1$).

However, this is not the only way to ensure a reachable and unique equilibrium as the state that $p(M)=1.$
If a system can give users sufficient incentives to hold coins, it is also possible to, even with low $v$, $v^\prime$, and $p(M^\prime)$, make the pegging state the unique equilibrium.
Note that with a high value of the function $i$, $i(v^\prime)$ and $i(p(M^\prime))$ can be greater than $p(M).$ 
As great incentives, we can come up with large savings interest for a stablecoin. Intuitively, such incentives are conducive to a decrease in market supply necessary for a price increase.
We present the proof of Theorem~\ref{thm:unique} in Appendix~\ref{app:proof1}.
