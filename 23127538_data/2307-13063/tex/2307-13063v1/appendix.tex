\section{Combined Features from Multivariate Correlation Analysis}\label{app:multivariate_details}
We here show in Figure~\ref{fig:multi_example} a visual representation of correlating the combined features we obtained in Section~\ref{sec:multivariate_tidal} with the target feature $\bar\lambda$. From this figures we can clearly see the strong correlation of the combined feature obtained from Principal Component 0 with $\bar \lambda$, which we then used above to construct our \AU relation. A similar visual analysis can and should be performed to assist in any attempt to construct \AU relations using multivariate data analysis.

\section{The Special Case with strong Collinearity}\label{app:multivariate_special}
Unfortunately, the approach using multivariate statistical analysis we described in this work (cf. Section~\ref{sec:multivariate}) does not always produce conclusive results: in cases where there exist strong correlations between features, the conditions we formulated in Section~\ref{sec:multivariate_methodlogy} will not necessarily or sufficiently lead to the construction of \AU relations.

For instance, let's consider the case where we want to predict the compactness $C$ given the features $\bar M\omega$ and $\eta$. The principal component analysis leads to the loadings given in Table~\ref{tab:multi_example3}, and the associated combined features shown in Figure~\ref{fig:multi_example3}. As we can see, each corresponding combined feature is strongly correlated to $C$, however inspection of the loading does not necessarily yield any specific principal component for which $C$ has a significantly larger contribution.

\begin{table}[t]
\def\arraystretch{1.5}%
\centering
\begin{tabular}{c | r | r | r}
Component & $\eta$ & $\bar M \omega$& $C$ \\\hline
0 & \num{-0.57720817} & \num{0.57768953}& \num{-0.57715296}  \\
 1 & \num{-0.68722846} & \num{-0.03811131} & \num{0.72544095}\\
 2 & \num{-0.44107569} & \num{0.81536638}& \num{-0.37500653} \\
\end{tabular}
\caption{Loadings of features in each principal component shown in Figure~\ref{fig:multi_example3}}
\label{tab:multi_example3}
\end{table}

\begin{table}[t]
\def\arraystretch{1.5}%
\centering
\begin{tabular}{c | r | r | r | r }
Component & $\rho_c$ & $M$& $\bar I$ & $\lambda$ \\\hline
0 & \num{-0.51191746} & \num{-0.44792063}& \num{0.53917124} &  \num{0.49659036}   \\
 1 & \num{-0.29858968} & \num{0.77774975} & \num{-0.13208151} & \num{0.53712568} \\
 2 & \num{0.78973118} & \num{0.09031859} & \num{0.44106068}& \num{0.41669256} \\
 3 & \num{-0.15845937} & \num{0.43164219} & \num{0.70520517} & \num{-0.53968631}
\end{tabular}
\caption{Loadings of features in each principal component shown in Figure~\ref{fig:multi_example2}}
\label{tab:multi_example2}
\end{table}


\section{Counterexample for Multivariate Correlation Analysis}\label{app:multivariate_counter}
We now attempt to construct a \AU relation for $\lambda$, but this time using the features $M$, $\rho_c$ and $\bar I$. We again apply the principal component analysis on all 4 features. The resulting principal components are shown in Figure~\ref{fig:multi_example2}. The loadings of each feature corresponding to each principle component are given in Table~\ref{tab:multi_example2}.

As we can clearly see here, none of the combined features derived from the principal components are well correlated with $\lambda$. This is also reflected in the loadings: there is no principal component for which the feature $\lambda$ shows a significantly higher contribution than the other features.  As such, this example serves as a good example for the case where there is good correlation to be found.

% Figure environment removed

% Figure environment removed

% Figure environment removed
