\section{Bivariate Universal Relations}\label{sec:bivariate_relations}
In the following, we inspect the \AU relations found by the correlation measures we discussed in the previous section. For each relation, we will also indicate the correlation value obtained by each respective measure. This will allow us to inspect in which cases each of the correlation measures succeed or fail in correctly identifying features that are suited for \AU relations.

Since the features we correlate cover very different ranges of values, we will evaluate the quality of each proposed \AU relation through the \emph{average relative error} $\bar e$ given by
\begin{equation}
\bar e = \frac{1}{n} \sum_i \frac{\lvert \hat y_i - y_i \rvert}{\lvert y_i\rvert}
\end{equation}
where $\hat y_i$ is the value predicted by the \AU relation, and $y_i$ the actual data point.

In some cases, our automated approach will find an exponential relation between two feature that we are analyzing. We find that by instead fitting for the logarithm of the target feature we achieve better universality. In such cases, after performing the correlation analysis on the regular features, we therefore manually fit a polynomial relation between the logarithm of the target feature and the independent feature. Not that the correlation values, however, will still be given between the regular features, and not after applying the logarithm, as this is how the features are fed into the automatic method described in Section~\ref{sec:bivariate}.

A table summarizing all universal relations presented in this section can be found in the Conclusion~\ref{sec:conclusion}.%Appendix~\ref{sec:relation_table}. 
\subsection{Tidal Deformability Relations}
% Figure environment removed
In Figure~\ref{fig:bivariate_0} we show a \AU relation between the normalized tidal deformability $\bar\lambda$ and the normalized moment of inertia $\bar I$. This relation was also previously put forward by Yagi and Yunes~\cite{Yagi365} as part of their \emph{I-Love-Q} relations. The best fit for this relation is given by the function 
\begin{equation}
\bar I = \num{0.01860855} \log \bar\lambda^2 - \num{0.07584032}\log \bar\lambda + \num{0.33375708}\,.
\label{eq:bivariate_0}
\end{equation}
This relation achieves an average relative error of $\num{0.020266944213251526}$. The fact that the relation itself is non-linear is also reflected in the comparatively low correlation value of $\num{0.9108}$ by the Pearson correlation coefficient.

% Figure environment removed
In Figure~\ref{fig:bivariate_1} we show a \AU relation between the effective compactness $\eta$ and the logarithm of the normalized tidal deformability $\log \bar\lambda$. A similar relation was also previously proposed by us in the context of a binary neutron star merger connecting the pre-merger binary tidal deformability to the post-merger effective compactness~\cite{2021PhRvD.104b3005M}. 

This is a case in which the automated approach yields an exponential relation between $\eta$ and $\bar\lambda$, and as discussed above, we manually fit a polynomial relation for $\log \bar \lambda$, yielding the relation 
\begin{equation}
\log \bar \lambda = \num{-0.09253471}\eta^2 -\num{5.42472023} \eta +\num{13.60418765}%\num[exponent-mode=scientific]{1132453.4706765804} e^{\num{-5.80292988}\eta}\,.
\label{eq:bivariate_1}
\end{equation}
This relation achieves an an average relative error of $\num{0.007502394271055736}$. In this case, the originally exponential relation between the two features causes the Pearson correlation coefficient in particular to give a very low correlation value of $-\num{0.7626}$. In comparison, the other correlation measures still assign a fairly high correlation measures, however the Distance Correlation also begins to assign a lower value of $\num{0.9400}$.

% Figure environment removed
In Figure~\ref{fig:bivariate_2} we show a \AU relation between the compactness $C$ and the logarithm of the normalized tidal deformability $\log\bar\lambda$. Such a relation follows directly from the definition of $\bar\lambda$ in terms of the tidal Love number $k_2$, i.e.
\begin{equation}
\bar\lambda = \frac{\lambda}{M^5} = \frac{2}{3} k_2 \frac{R^5}{M ^5} =\frac{2}{3} k_2 C^5\,.
\end{equation}
%an naturally be obtained by combining the relations between normalized $f$-mode frequency and compactness $C$ put forward by Tsui and Leung~\cite{10.1111/j.1365-2966.2005.08710.x}, and the universal relation between $f$-mode and tidal deformability put forward by Chan et al.~\cite{chan2014multipolar}.
The automatic approach again finds an exponential relation between the features $C$ and $\bar\lambda$, and as before, we find that fitting for $\log\bar\lambda$ instead yields the more accurate, universal relations. The manual fit yields the relation 
%The best fit for this relation is given by the function
\begin{equation}
\log \bar \lambda =  \num{46.1226271} C^2 -\num{53.04505671} C + \num{13.63341181}%\num[exponent-mode=scientific]{259755.6187230266} e^{\num{-40.42696543} C}\,.
\label{eq:bivariate_2}
\end{equation}
This relation achieves a relative error of \num{0.019736506644986718}. As before, the regular features have a highly non-linear, exponential relation fore which the Pearson correlation measure assigns a low correlation value, even though we can observe a strong relation.

% % Figure environment removed

% \clearpage
\subsection{Astroseismological Relations}\label{sec:bivariate_osci}
%For the following relations, we will rely on data presented in~\cite{kruger2019fast,Kuan:2021jmk}. 
We here present some of the astroseismological, universal relations we were able to find for the $f$-mode and $g$-mode oscillation frequencies.

% Figure environment removed
In Figure~\ref{fig:bivariate2_0} we show a \AU relation between the compactness $C$ and the normalized $f$-mode frequency $\bar M \omega_f$. This relation was previously put forward by Tsui and Leung~\cite{PhysRevLett.95.151101}. The best fit for this relation is given by the function
\begin{equation}
\bar M \omega_f = \num{0.04212737} \log C^2+ \num{0.22183278} \log C + \num{0.3145776}\,.
\label{eq:bivariate2_0}
\end{equation}
This relation achieves an average relative error of \num{0.01148544725114905}. While the optimal fit is given by a logarithmic relation, visually the relation can generally be considered to be linear. As expected, in this case even the Pearson correlation coefficient assigns a high value, and the other correlation measures also identify strongly correlated features.

% Figure environment removed
In Figure~\ref{fig:bivariate2_1} we show a \AU relation between the normalized moment of inertia $\bar I$ and the normalized $f$-mode frequency $\bar M \omega_f$. This relation follows straight-forwardly by combining the relation by Tsui and Leung~\cite{10.1111/j.1365-2966.2005.08710.x} between the $f$-mode and compactness $C$, with the understanding that the compactness $C$ and effective compactness $\eta$ can often be used interchangeably in such general relativistic relations. However, to our knowledge, this is the first time that this relation is presented explicitly. 

The best fit for this relation is given by the function
\begin{equation}
\bar M \omega_f = \num{1.01634729} \log {\bar I}^2 - \num{0.93359431} \log \bar I + \num{1.53017764}\,.
\label{eq:bivariate2_1}
\end{equation}
This relation achieves an average relative error of \num{0.006879103949206297}. While the best fit is given by a logarithmic relation, it does not appear to be non-linear to an extreme degree. This is reflected by the correlation measures assigned by all correlation measures. However, even here, the Pearson correlation gives these features a comparatively low correlation value.

% Figure environment removed
Figure~\ref{fig:bivariate2_2} shows a \AU relation between the normalized tidal deformability $\bar \lambda$ and the normalized $f$-mode frequency $\bar M \omega_f$. This relation was also previously put forward by Chan et al.~\cite{chan2014multipolar}. The best fit for this relation is given by the function
\begin{equation}
\bar M \omega_f = 0.0003 \log {\bar \lambda}^2 - \num{0.01490679} \log \bar \lambda + \num{0.12746998}\,.
\label{eq:bivariate2_2}
\end{equation}
This relation achieves an average relative error of of \num{0.01425718984920659}. The highly non-linear, logarithmic relation between these features again causes the Pearson correlation coefficient to fail to detect the correlation between these features, and even the Distance Correlation assigns a comparatively small correlation value. 

% Figure environment removed
Figure~\ref{fig:bivariate2_3} shows a \AU relation between the effective compactness $\eta$ and the normalized $f$-mode frequency $\bar M \omega_f$. This relation was also previously put forward by Lau et al.~\cite{lau2010inferring} and Kr\"uger and Kokkotas~\cite{kruger2019fast}. The best fit for this relation is given by the function
\begin{equation}
\bar M \omega_f = \num{0.7134402} \eta^2 + \num{1.18096983} \eta - \num{0.42236486}\,.
\label{eq:bivariate2_3}
\end{equation}
This relation achieves an average relative error of \num{0.006629639167064198}. Visually, this relation again appears to be mostly linear, which is reflected by all correlation measures assigning a high correlation value.

% Figure environment removed
Figure~\ref{fig:bivariate3_0} shows a \AU relation between the average density $\tilde\rho$ and the $f$-mode frequency $\omega_f$. This relation was also previously put forward by Andersson and Kokkotas~\cite{PhysRevLett.77.4134,Andersson98}. The best fit for this relation is given by the function
\begin{equation}
\omega_f = \num{-2.19908806} \tilde\rho^2 + \num{0.98518922} \tilde\rho + \num{0.007272}\,.
\label{eq:bivariate3_0}
\end{equation}
This relation achieves an average relative error of \num{0.03499534511917183}. Again, the fact that this relation appears to be mostly linear is reflected in the fact that all correlation measures assign a fairly high correlation value to these two features.

% Figure environment removed
Figure~\ref{fig:bivariate2_5} shows a \AU relation between the normalized $f$-mode frequency $R \omega_f$ and the logarithm of the normalized $g$-mode frequency $\log \bar M\omega_{g_1}$. This relation was also previously put forward by Kuan et al.~\cite{Kuan:2021jmk}. As was the case for the relations in Equations~\eqref{eq:bivariate_1} and~\eqref{eq:bivariate_2}, the automatic method finds an exponential relation between the features $R \omega_f$ and $\bar M\omega_{g_1}$. As before, we find that manually fitting the relation for the logarithm $\log \bar M \omega_{g_1}$ gives the more accurate relation, yielding %The best fit for this relation is given by the function
\begin{equation}
\log \bar M\omega_{g_1} = \num{16.05167444} \left(R \omega_f\right)^2 -\num{5.32343308} R \omega_f+ \num{5.58908468}%\num{6.0010885033563595} e^{\num{10.3503197} R \omega_f}\,.
\label{eq:bivariate2_5}
\end{equation}
This relation achieves an average relative error of \num{0.004346355091302037}. Even though the automatic method finds an exponential relation between the original features, the non-linearity of the relation in this case is not as extreme. As such, even the Pearson correlation coefficient achieves a fairly high correlation value, however notably lower than the other correlation measures.

\subsection{Quantitative Comparison of Correlation Measures}
% Figure environment removed

We can perform a more quantitative analysis and comparison of the four different correlation measures by considering some specific performance measures commonly used in statistics. To define these, we first introduce a few notions for binary classifiers. We define them here in terms of our use case of identifying universally related neutron star features: A \emph{true positive} is a pair of features that is universally related, and also identified as such by a given correlation measure. The number of true positives is denoted by $\TP$.

A \emph{false positive} is a pair of features that is \emph{not} universally related, but classified as such. The number of false positives is denoted by $\FP$.

A \emph{false negative} is a pair of features that is universally related, but not classified as such. The number of false negatives is denoted by $\FN$.

A \emph{true negative} is a pair of features that is not universally related, and also not classified as such. The number number of true negatives is denoted by $\TN$.

Given these notions, we can now define performance measures that quantify how well our classifiers correctly label pairs of features. \emph{Recall}, or \emph{true positive rate} $\TPR$ is the rate at which the classifier correctly labels universally related pairs of features as universally related. It is given by
\begin{equation}
\TPR = \frac{\TP}{\TP + \FN}\,.
\end{equation}

\emph{Precision}, or \emph{positive predictive value} ($\PPV$), is the rate of pairs of features classified as universally related that are in fact universally related. It is given by
\begin{equation}
\PPV = \frac{\TP}{\TP + \FP}\,.
\end{equation}

Finally, the \emph{fallout}, or \emph{false positive rate} ($\FPR$), is the rate at which not related pairs of features are classified as being universally related. It is given by
\begin{equation}
\FPR = \frac{\FP}{\FP + \TN}\,.
\end{equation}

We can now compute the precision, recall and fallout for each correlation measure at a given classification threshold $\tau$, and compare how these performance develop with $\tau$. Ideally, we would like to achieve high recall, while keeping precision high, and fallout low.
% Figure environment removed

Typically, one considers the precision-recall and ROC-curves for a better understanding on how these quantities evolve with each other. The precision-recall curves plot the maximum precision achieved by a classifier for a required recall, and allow us to understand how accurate a positive prediction (i.e. classification as universal relation) is, given a specific correlation measure and classification threshold. We show the precision-recall curves for each correlation measure, and one combined plot, in Figure~\ref{fig:prec_rec}.  

The ROC (or receiver operating characteristic) curve plots the recall against the fallout of the classifier. This plot allows us to better understand how many incorrectly classified universal relations we should expect for a given recall requirement. The ROC curves for each correlation measure applied to each of the two data sets considered in this work can be found in Figure~\ref{fig:roc}. 

As we can see in all Figures, the standard Pearson correlation measure is outperformed by the other metrics significantly for most of the recall range. The distance correlation measures, in turn, is also outperformed by the mutual information based measures. Both the kernel-based mutual information measure, and the maximal information measure show high precision and low fallout for high recall values, identifying them as the preferred measures for the task of identifying universally related features.

Note that the above analyses were performed by manually labeling all feature pairs in our limited data set as either being universally related or not in order to obtain the true/false positive/negative counts. As such, the exact values for each performance measure will most likely vary with different datasets and labels. However, the difference in behavior of each correlation measure appears to be significant enough to warrant the conclusion drawn above.