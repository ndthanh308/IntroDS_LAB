\section{Introduction}\label{sec:introduction}
The successful detection of gravitational waves from binary neutron star (BNS) mergers through the LIGO-VIRGO detectors~\cite{2017PhRvL.119p1101A,2020ApJ...892L...3A} has opened a new avenue into probing and understanding the structure of neutron stars and will hopefully allow us to eventually uncover their true equation of state (EoS). 

Important tools for this task are EoS independent -- or (approximately) universal -- relations that allow for the inference of neutron star bulk parameters through information extracted from gravitational waves. Inspired by early work on such universal relations for isolated neutron stars~\cite{PhysRevLett.77.4134,Andersson98,PhysRevLett.95.151101,lau2010inferring,chan2014multipolar,Yagi365}, the last five years have also given rise to universal relations for binary neutron stars (BNS)~\cite{PhysRevLett.115.091101,PhysRevD.93.124051,PhysRevD.101.084006}: they relate features of the pre-merger neutron stars to the early post-merger remnant, primarily relying on numerical relativity simulations.

Following our own recent work on universal relations for BNS using perturbative calculations~\cite{2021PhRvD.104b3005M,2021FrASS...8..166K}, we found that, with the increasing number of features and amount of data that theoretical computations are able to produce, the traditional method of relying on physical intuition to find \AU relations might not always uncover all possible or the best \AU relations for a given scenario: instead, an automated approach fueled by statistical data analysis might yield better results in finding highly correlated features, and the best functional form to relate them with. A recent work by Soldateschi et al.~\cite{2021A&A...654A.162S} demonstrated the application of principal component analysis (PCA) to the construction of \AU relations with multiple independent variables. 

In this paper, we present applications of statistical data analysis methods from both bi- and multivariate statistics to find suitable sets of neutron star features that can be leveraged for accurate and EoS independent relations. To this end, we first analyze the statistical power of four different correlation measures -- Pearson Correlation, Distance Correlation~\cite{DistanceCor}, Mutual Information~\cite{MutualInf} and Maximal Information~\cite{MaxInf} -- in identifying pairs of features amenable to universal relations. We find that the conventional wisdom that Pearson Correlation only detects linearly correlated features also applies to the use case of finding bivariate \AU relations for neutron stars. Furthermore, we also find that mutual information based features are more suited for finding non-linear correlation between features, making them more useful for this application.  

In a second step, inspired by~\cite{2021A&A...654A.162S}, we investigate the application of principal component analysis (PCA) in constructing multivariate \AU relations, i.e. relations with multiple independent variables. We find this method suitable for constructing linear order \AU relations that combine several features of a neutron star to predict a target feature. Among others, we find the an entirely novel relation between the average density $\tilde\rho = \sqrt{M/R^3}$, compactness $C = M/R$ and the $f$-mode frequency $\omega_f$ of a neutron star of the form
\begin{equation}
\omega_f = 0.00017 \hat F^2 + \num{0.00647675} \hat F + \num{0.00276012}
\end{equation}
with 
\begin{equation}
\hat F = \num{6.9109313199999995} \frac{\tilde\rho}{0.04} - \num{1.71649574} \frac{C \tilde\rho}{0.01}
\end{equation}
which, when compared to the original relation between $\tilde\rho$ and $\omega_f$ derived by Andersson and Kokkotas~\cite{PhysRevLett.77.4134,Andersson98} inspired from Newtonian gravity, could be considered a first order correction for general relativity. In particular, it can be considered a step towards the well known general relativistic \AU relation between the $f$-mode frequency $\omega_f$ and the compactness $C$ put forward by Tsui and Leung~\cite{PhysRevLett.95.151101}.

We perform our analyses using two different data sets from the literature~\cite{2021PhRvD.104b3005M,Kuan:2021jmk}, exemplifying the generalizability of the methods discussed in this work. The results in this work present a first step towards a automated, statistical data analysis driven effort towards the identification and construction of \AU relations for neutron stars (and other objects of astrophysical interest). In a time where the amounts of theoretical model data for astrophysical objects is drastically increasing, we expect having such robust and automated methods available as tools will have a tremendous effect on the quality and quantity of \AU relations that will become available in the future. 

\medskip
\noindent
\textbf{Outline.} We begin by introducing the two data sets that we will base our analyses on in Section~\ref{sec:data}. We then introduce the bivariate approach to finding \AU relations in Section~\ref{sec:bivariate}, and discuss the found relations, and the implications for the relative statistical power of the analyzed correlation measures in Section~\ref{sec:bivariate_relations}. 

In Section~\ref{sec:multivariate}, we introduce the multivariate approach based on PCA for finding \AU relations, before we discuss some exemplary \AU relations we were able to construct in Section~\ref{sec:multivariate_relations}. We finally conclude our work and give an outlook into potential future directions in Section~\ref{sec:conclusion}.

Note that, unless stated otherwise, we will assume geometrized units in which $G = c = 1$ throughout this paper.
