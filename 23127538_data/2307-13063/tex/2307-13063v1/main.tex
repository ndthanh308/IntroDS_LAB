%\documentclass[11pt,a4paper, twocolumn]{article}
\documentclass[%
reprint,
superscriptaddress,
showkeys,
twoside,
%groupedaddress,
%unsortedaddress,
%runinaddress,
%frontmatterverbose, 
%preprint,
%preprintnumbers,
%nofootinbib,
%nobibnotes,
%bibnotes,
amsmath,
amssymb,
aps,
%pra,
prd,
%rmp,
%prstab,
%prstper,
%floatfix,
]{revtex4-2}
%\usepackage[demo]{geometry}%margin=0.5in
\usepackage{mathtools}
\usepackage{xspace}
\usepackage{siunitx}
\usepackage{journals}
\usepackage{multirow}
\usepackage{xcolor}
\usepackage[pdftex, pdftitle={Article}, pdfauthor={Author}, hidelinks]{hyperref}
\usepackage{tensor}
\usepackage{cancel}
\usepackage{subfig}
\usepackage[]{graphicx}%draft
\usepackage[utf8]{inputenc}

\sisetup{round-mode=places,round-precision=3}
\DeclareMathOperator{\cov}{cov}
\DeclareMathOperator{\dCor}{dCor}
\DeclareMathOperator{\dVar}{dVar}
\DeclareMathOperator{\dCov}{dCov}
\DeclareMathOperator{\MIC}{MIC}

\DeclareMathOperator{\TP}{TP}
\DeclareMathOperator{\FN}{FN}
\DeclareMathOperator{\FP}{FP}
\DeclareMathOperator{\TN}{TN}
\DeclareMathOperator{\TPR}{TPR}
\DeclareMathOperator{\PPV}{PPV}
\DeclareMathOperator{\FPR}{FPR}

\newcommand{\AU}{universal\xspace}
\usepackage{tabularx}

\widowpenalty10000
\clubpenalty10000

\begin{document}

\begin{abstract}
We present applications of statistical data analysis methods from both bi- and multivariate statistics to find suitable sets of neutron star features that can be leveraged for accurate and EoS independent -- or \emph{universal} -- relations. To this end, we investigate the ability of various correlation measures such as Distance Correlation and Mutual Information in identifying universally related pairs of neutron star features. We also evaluate relations produced by methods of multivariate statistics such as Principal Component Analysis to assess their suitability for producing universal relations with multiple independent variables.

As part of our analyses, we also put forward multiple entirely novel relations, including multivariate relations for the $f$-mode frequency of neutron stars with reduced error when compared to existing, bivariate relations. 
\end{abstract}
\title{Finding Universal Relations using Statistical Data Analysis}
 \author{Praveen Manoharan}
	\email{praveen.manoharan@uni-tuebingen.de}
	\affiliation{Theoretical Astrophysics, IAAT, University of T\"ubingen, 72076 T\"ubingen, Germany}
 \author{Kostas D. Kokkotas}
    \email{kostas.kokkotas@uni-tuebingen.de}
    \affiliation{Theoretical Astrophysics, IAAT, University of T\"ubingen, 72076 T\"ubingen, Germany}

\date{\today}

\keywords{keywords}	

\maketitle

\section{Introduction}
Current quantum hardware is unable to carry out universal quantum computations due to the buildup of errors that occur during the computation. 
The magnitude of the individual error is currently above the value that the Threshold Theorem requires in order to kick-start quantum error correction and fault-tolerant quantum computation~\cite[Section 10.6]{nielsen_chuang_2010}. 
Although the experimentally achieved fidelity rates are promising and the error bounds are inching closer to the required threshold, we will have to work for the foreseeable future with quantum hardware with errors that build-up during the computation.  This implies that we can only do a limited number of steps before the output of the computation has become completely uncorrelated with the intended one.

For fault-tolerant quantum computing, we repeat four steps: 
1) We apply a number of single and two-qubit quantum gates, in parallel whenever possible; 
2) We perform a syndrome measurement on a subset of the qubits; 
3) We perform fast classical computations to determine which errors have occurred and how to correct them; 
and, 4) We apply correction terms based on the classical computations.
We then repeat these four steps with a next sequence of gates. 
These four steps are essential to fault-tolerant quantum computing. 


The starting point of this work is to use the four steps outlined above, not to carry out error correction and fault-tolerant computation, but to enhance short, constant-depth, {\em uncorrected} quantum circuits that perform single qubit gates and {\em nearest-neighbor} two qubit gates. 
Since in the long run we will have to implement error-correction and fault-tolerant computation anyhow, and this is done by such a four-step process, why not make other use of this architecture? Moreover, on some of the quantum hardware platforms, these operations are already in place.
Embracing this idea we naturally arrive at the question: what is the computational power of \textit{low-depth} quantum-classical circuits organized as in the four steps outlined above? 
We thus investigate circuits that execute a small, ideally constant, number of stages, where at each stage we may apply, in parallel, single qubit gates and {\em nearest-neighbor} two qubit gates, followed by measurements, followed by low-depth classical computations of which the outcome can control quantum gates in later stages. 
It is not clear, at first, whether such circuits, especially with constant depth, can do anything remotely useful. 
But we will see that this is indeed the case: many quantum computations can be done by such circuits in constant depth. 
By parallelizing quantum computations in this way, we improve the overall computational capabilities of these circuits, as we do not incur errors on qubits that are idle, simply because qubits are not idle for a very long time. 
Furthermore, reducing the depth of quantum circuits, at the cost of increasing width, allows the circuit to be run faster even if errors occur.

The first usage of such a four-step layout, not to do error correction, but to perform computations, can be found in the paradigm of measurement-based quantum computing~\cite{gottesman1999demonstrating,raussendorf2001one,jozsa2006introduction,clark2007generalised}: 
A universal form of quantum computing where a quantum state is prepared and operations are performed by measuring qubits in different bases, depending on previous measurements and intermediate measurements.

\citeauthor{PhamSvore2013} were the first to formalize the four-step protocol for performing computations~\cite{PhamSvore2013}. They included specific hardware topologies by considering two-dimensional graphs for imposing constraints on qubit interactions. In their model, they develop circuits for particularly useful multi-qubit gates, including specifying costs in the width, number of qubits, depth, number of concurrent time steps, size, and total number of non-Identity operations.
As a result, they find an algorithm that factors integers in polylogarithmic depth.
\citeauthor{Browne:2011} showed that the main tool in the work by \citeauthor{PhamSvore2013}, the fan-out gate, can also be replaced by additional log-depth classical computations in the measurement-based quantum computing setting~\cite{Browne:2011}.

More recently, \citeauthor{Cirac:2021} introduced a scheme to implement unitary operations involving quantum circuits combined with Local Operations and Classical Communication ($\mathsf{LOCC}$) channels: $\mathsf{LOCC}$-assisted quantum circuits~\cite{Cirac:2021}. Similarly to the four-step scheme we just described, they allow for a short depth circuit to be run on the qubits, followed by one round of $\mathsf{LOCC}$, in which ancilla qubits are measured and local unitaries are applied based on the measurement outcomes. They show that in this model any 1D transitionally invariant matrix-product state (MPS) with fixed bond dimension is in the same phase of matter as the trivial state. Similar ideas can be found in~\cite{TVV_NonAbelianTopologicalOrder_2022, tantivasadakarn2021long}.

In this work, we introduce a new model, called \textit{Local Alternating Quantum-Classical Computations} ($\LAQCC$). In this model we alternate between running quantum circuits (constrained by locality), ending in the measurement of a subset of qubits, and fast classical computations based on the measurement results. The outcome of the classical computations are then used to control future quantum circuits. We allow for flexibility in this model, by giving different constraints to the power of both the quantum circuits and the classical circuits as well as the number of alternations between them. 
Most attention will be given to $\LAQCC$ containing quantum circuits of constant depth, classical circuits of logarithmic depth and at most a constant number of alternations between them. 
Any circuit constructed in this model is considered to be of constant depth. 
We restrict ourselves to logarithmic depth classical computations, as this is the first natural and non-trivial extension beyond constant-depth classical computations. 
Constant-depth classical computations do however also have an equivalent constant-depth quantum implementation.

The definition of $\LAQCC$ sharpens the original definition of \citeauthor{PhamSvore2013} by adding constraints to the intermediate classical computations. This allows us to bound the power of $\LAQCC$ from above. 

The main result of \citeauthor{Cirac:2021}, that 1D translational invariant MPS with fixed bond dimension can be prepared by $\mathsf{LOCC}$-assisted circuits, relies on local symmetries of the MPS. These symmetries allow them to prepare local states (on a constant number of qubits) and glue them together by doing one round of the appropriate entangling measurement and corrections, after which they run a round of local unitaries to get the desired result. This general scheme for preparing states that exhibit an MPS description with the appropriate local symmetries requires only geometrically local unitaries and one round of measurement and corrections an therefore is accessible in $\LAQCC$. Studying different local symmetries, known as Symmetry Protected Topological (SPT) phases of matter, to find measurement-based constant depth circuits for states is a broad ongoing field of research~\cite{TVV_NonAbelianTopologicalOrder_2022, tantivasadakarn2021long, smith2023deterministic}. 
All these schemes have a $\LAQCC$ implementation.

%$\LAQCC$-circuits also exist for general schemes of preparing local states, based on the local tensors, and gluing them together using one round of entangled measurement and corrections, based on the local symmetry. 
%The main result of \citeauthor{Cirac:2021}, that 1D translational invariant MPS with fixed bond dimension can be prepared by $\mathsf{LOCC}$-assisted circuits, relies heavily on local symmetries of the MPS and as a result also has an equivalent $\LAQCC$ implementation. 
%The corrections applied after the measurement round are local unitaries depending on the local symmetries of the MPS. 

 

%This general scheme of preparing local states, based on the local tensors, and gluing it together by doing one round of entangled measurement and corrections, based on the local symmetry, is accessible in $\LAQCC$.
Note however that \citeauthor{Cirac:2021} also suggest a circuit for the $W$-state.
This circuit uses sequentially and dependent measurement-based corrections of the ancilla qubits. 
These dependent measurements translate to sequential alternations between the quantum and classical circuits and therefore increase the total depth to linear depth, exceeding the constant-depth constraints imposed by $\LAQCC$-circuits. 

We study the power of the $\LAQCC$ model with respect to state preparation, showing that even with only constant quantum-depth and logarithmic classical depth it remains possible to prepare states with long-range entanglement.
Another surprising result is that it is unlikely that $\LAQCC$ circuits are classically simulatable. We show that any instantaneous quantum polynomial-time (IQP) circuit~\cite{Bremner2010,Shepherd2009} has an $\LAQCC$ implementation.
Classical simulation of IQP circuits implies the collapse of the polynomial hierarchy to the third level, which is not believed to be true~\cite{Bremner2017}. Therefore, we expect that $\LAQCC$ circuits are unlikely to be classically simulatable. We bound the power of $\LAQCC$ by showing that it is contained in $\QNC^1$, the class of polynomial-size, log-depth circuits.

Next, we also study the power that intermediate classical calculations can add to quantum computations, by considering a new model that alternates between polynomially many polynomial-depth quantum circuits and unbounded classical computations
We study this model by doing a complexity theoretical analysis, where we draw inspiration from the notions of complexity given by \citeauthor{RosenthalYuen:2022}, \citeauthor{MetgerYuen:2023}, and \citeauthor{Aaronson:2004}.
All three complexity notions are based on the notion of state preparation, instead of more traditional definition of complexity such as the decidability of a computational problem. 
The first two consider classes based on sequences of quantum states preparable by a polynomial-sized quantum circuit, where the circuits are uniformly generated by a computational class, for instance, the class $\mathsf{PSPACE}$, which results in the complexity class $\mathsf{StatePSPACE}$~\cite{RosenthalYuen:2022,MetgerYuen:2023}.
The third notion considers a relative complexity, where the complexity is measured between two given states, and is measured by the number of gates, from a given gate-set, required to transform one state in another state~\cite{Aaronson:2004}. 
For our definition of state preparation complexity, we drop the uniformity constraint from~\cite{RosenthalYuen:2022,MetgerYuen:2023} and define a class as $\mathsf{StateX}$, which refers to states preparable by circuits of type $\mathsf{X}$. 
As an example, if $\mathsf{X} = \QNC^0$, this results in the class $\mathsf{StateQNC^0}$, which is the set of states preparable from the $\ket{0}^n$ state by poly-size constant-depth circuits. 
This notion is similar to the relative complexity from~\cite{Aaronson:2004}, where one state is the  $\ket{0}^n$ state and instead of counting the number of gates we consider the set of states preparable by a fixed number of gates. Using this notion of complexity we show that any state preparable by an $\LAQCC^*$ circuit is also preparable by a $\mathsf{PostQPoly}$ circuit, the class of circuits of polynomial depth with an additional post-selection gate. 

All Clifford circuits have a constant-depth $\LAQCC$ implementation, implying that any stabilizer state can be implemented by a constant-depth $\LAQCC$ circuit, see Section~\ref{sec:clifford_circuits} for a proof of this statement. 
Efficient circuits for stabilizer states have been known already through measurement-based quantum computing. Therefore this paper focuses on the preparation of non-stabilizer states, and as a surprising result we find novel constant-depth protocols for four very natural classes of non-stabilizer states.
Despite the extensive research into these four classes of non-stabilizer states and the many applications of them, no efficient constant- or low-depth state preparation protocols are known yet. We specifically consider these four classes as they are all often used as initial states in other algorithms.

The first state is a uniform superposition over an arbitrary number of states. 
This state finds applications in many quantum algorithms, as they often start with a uniform superposition over multiple states. 
This superposition is often achieved by applying Hadamard gates to every qubit due to its simplicity to prepare. 
Yet, the analysis of many algorithms, such as Shor's algorithm~\cite{Shor:1997}, would benefit from a different initial superposition. 
The circuit to prepare the uniform superposition over an arbitrary number of states uses an exact version of Grover search as a subroutine, that turns a probabilistic circuit, with a known constant probability of success, into a deterministic circuit. 
We use the circuit for preparing a uniform superposition over an arbitrary number of states as a subroutine in the next two quantum state preparation protocols. 

The second state is the $W$-state, the uniform superposition over all computational basis states of Hamming-weight~$1$, a natural long-ranged entangled state that displays a fundamentally nonequivalent type of entanglement from the Greenberger–Horne–Zeilinger state~\cite{WState:2000}, for which $\LAQCC$-type constant-depth circuits were previously known~\cite{PhamSvore2013, Cirac:2021}. 
The $W$-state is often used as benchmark for new quantum hardware~\cite{Haffner2005,Neeley2010,GarciaPerez:2021}. 
A novel way to prepare the $W$-state therefore gives a new way to benchmark different quantum devices with each other. 
A circuit for preparing the $W$-state was given in~\cite{Cirac:2021}, but this implementation requires sequentially alternating measurements followed by local unitaries, which in the $\LAQCC$ model is not considered to be of constant depth. 
We improve this protocol by giving an $\LAQCC$ implementation of the $W$-state, based on a compress-uncompress method that links the one-hot and binary encoding of integers.

The third state considered is the Dicke state, a generalization of the $W$-state, a superposition over all computational basis states with Hamming-weight $k$~\cite{Dicke:1954}. 
Dicke states have relevance in various practical settings.
For instance, for quantum game theory~\cite{zdemir2007}, quantum storage~\cite{Bacon_Compress:2006,Plesch:2010}, quantum error correction~\cite{ouyang2014permutation}, quantum metrology~\cite{toth2012multipartite}, and quantum networking~\cite{prevedel2009experimental}. 
Dicke states have been used as a starting state for variational optimization algorithms, most notably Quantum Alternating Operator Ansatz (QAOA)~\cite{Hadfield2019}, to find solutions to problems such as Maximum k-vertex Cover~\cite{Brandhofer2022,cook2020quantum}.
The ground states of physical Hamiltonians describing one-dimensional chains tend to show a resemblance to Dicke states such as states resulting from the Bethe ansatz, making them an ideal starting state when investigating the ground state behavior of these Hamiltonians~\cite{TDL_BetheAnsatzDerivation:2010,B_ExcitedStateQuantumPhaseTransitions:2013,DickeTransitions:2021}. 
For instance, the algorithm by \citeauthor{van2021preparing}, who give an algorithm to prepare the Bethe ansatz eigenstates of the spin-1/2 XXZ spin chain, starts by first preparing a Dicke state~\cite{van2021preparing}. 
A Dicke-state preparation protocol based on the compress-uncompress methodology used in the $W$-state furthermore finds applications in entanglement distillation, where the entanglement of a large state is concentrated on only a few qubits. 
Efficient deterministic circuits for preparing Dicke states have been proposed by \citeauthor{bartschi2019deterministic}~\cite{bartschi2019deterministic, bartschi2022deterministic_short_depth}. 
They provide a quantum circuit of depth $\mathO(k \log(\frac{n}{k}))$, allowing arbitrary connectivity, to prepare a Dicke state, which they conjecture to be optimal when $k$ is constant. 
In this work, we provide a constant-depth $\LAQCC$ circuit below their conjectured bound already for constant $k$. 
However, this does not directly disprove their conjecture, as we allow for intermediate measurements and classical computations. 
More significantly, we even construct constant-depth $\LAQCC$ circuits for $k = \mathO(\sqrt{n})$ greatly improving their bound.
This construction extends the compress-uncompress method for the $W$-state combined with additional subroutines. 

We continue with a log-depth state preparation protocol for the Dicke-state for arbitrary $k$. 
This protocol implements an efficient transformation between the factoradic number representation and the combinatorial number representation of a positive integer. 
The combinatorial number representation relates directly to the Dicke state. 
The provided efficient transformation between number representation systems might be of independent interest. 

We conclude by modifying our protocol for preparing a Dicke-state to a protocol that prepares quantum many-body scar states in constant-depth. 
These states have low entanglement and longer coherence times than states with similar energy density.
These characteristics make many-body scar states interesting to analyze and relevant within physics.
Many-body scar states appear for instance in the AKLT model~\cite{AKLT:1987,MRBAR:2018,MRB:2018} and different spin models~\cite{SI:2019,MOBFR:2020}.
Known methods for preparing these states have polynomial-depth~\cite{Gustafson:2023}, whereas our circuit has constant depth. 

% We conclude by studying the power that intermediate classical calculations can add to quantum computations. 
% In this study, we define a new model that relaxes constant-depth quantum circuits to polynomial depth quantum circuits, log-depth classical calculations to unbounded classical computations and a constant number of alternations to a polynomial number of alternations. 
% We call this model $\LAQCC^*$. 
% We study this model by doing a complexity theoretical analysis, where we draw inspiration from the notions of complexity given by \citeauthor{RosenthalYuen:2022}, \citeauthor{MetgerYuen:2023}, and \citeauthor{Aaronson:2004}.
% All three complexity notions are based on the notion of state preparation, instead of more traditional definition of complexity such as the decidability of a computational problem. 
% The first two consider classes based on sequences of quantum states preparable by a polynomial-sized quantum circuit, where the circuits are uniformly generated by a computational class, for instance, the class $\mathsf{PSPACE}$, which results in the complexity class $\mathsf{StatePSPACE}$~\cite{RosenthalYuen:2022,MetgerYuen:2023}.
% The third notion considers a relative complexity, where the complexity is measured between two given states, and is measured by the number of gates, from a given gate-set, required to transform one state in another state~\cite{Aaronson:2004}. 
% For our definition of state preparation complexity, we drop the uniformity constraint from~\cite{RosenthalYuen:2022,MetgerYuen:2023} and define a class as $\mathsf{StateX}$, which refers to states preparable by circuits of type $\mathsf{X}$. 
% As an example, if $\mathsf{X} = \QNC^0$, this results in the class $\mathsf{StateQNC^0}$, which is the set of states preparable from the $\ket{0}^n$ state by poly-size constant-depth circuits. 
% This notion is similar to the relative complexity from~\cite{Aaronson:2004}, where one state is the  $\ket{0}^n$ state and instead of counting the number of gates we consider the set of states preparable by a fixed number of gates. Using this notion of complexity we show that any state preparable by an $\LAQCC^*$ circuit is also preparable by a $\mathsf{PostQPoly}$ circuit, the class of circuits of polynomial depth with an additional post-selection gate. 

\paragraph{Summary of results}
\begin{itemize}
    \item We give a new definition of a computational model that captures the power of the four step process: applying a constant number of layers of one- and two-qubit gates; performing a syndrome measurement; perform a fast classical computation determining corrections; apply corrections. We call this model \emph{Local Alternating Quantum Classical Computations}, or $\LAQCC$ for short. In this model we bound the allowed quantum operations, intermediate classical calculations, and number of rounds separately. In Section~\ref{sec:LAQCC_model} we define this model and give a list of operations based on results from literature contained in this computational model. In some of these operations we explicitly use that we allow for multiple, but at most constant, rounds  of corrections.
    \item  We show show that there exist $\LAQCC$ circuits that can not be weakly simulated in Section~\ref{sec:IQP_in_LAQCC}. We further show that for every $\LAQCC$ circuit there exists a $\QNC^1$ circuit simulating it perfectly, in Section~\ref{sec:LAQCC_in_QNC1}.
    \item We introduce a new type computational complexity for preparing states and show that the extension of $\LAQCC$ where we allow a polynomial number of rounds and unbounded classical computation, is contained in $\mathsf{PostQPoly}$, the class of polynomial circuits with post-selection, in Section~\ref{sec:Complexity results}.
    \item We show a protocol to prepare the uniform superposition state of size $q$ in $\LAQCC$ using $\mathO(\ceil{\log_2(q)}^2)$ qubits in Section~\ref{sec:superposition_modulo_q}. 
    \item We show a protocol to prepare the $W_n$ state in $\LAQCC$ using $\mathO(n\log(n))$ qubits in Section~\ref{sec:W_state_in_LAQCC}.
    \item We show two ways of preparing the Dicke-$(n,k)$ state. The first method is in $\LAQCC$, works up to $k = \mathO(\sqrt{n})$, uses $\mathO(n^2\log(n))$ qubits, and is found in Section~\ref{sec:dicke:small_k}. The second method is in $\LAQCC\text{-}\mathsf{LOG}$ (an extension of $\LAQCC$ allowing for logarithmic number of alterations instead of constant), works for any $k$, uses $\mathO(\text{poly}(n))$ qubits, and is found in Section~\ref{sec:Dicke_in_LAQCC_LOG}. 
    \item We extend on our $\LAQCC$ method of generating Dicke-$(n,k)$ states for $k = \mathO(\sqrt{n})$ and show a protocol to generate many-body scar states for a particular Hamiltonian in $\LAQCC$ (Section~\ref{sec:many_body_scar}). 
\end{itemize}
Summarized in a table, we provide the following state generation protocols:
\begin{table}[htb]
\centering
\begin{tabular}{l|l|l|l}
\textbf{State description} & \textbf{Width} & \textbf{Depth} & \textbf{Implementation}\\
\hline 
Uniform superposition mod $q$: $\frac{1}{\sqrt{q}} \sum_{i = 0}^{q-1}\ket{i}$ & $\mathO(\ceil{\log^2 q})$ & $\mathO(1)$ & Section~\ref{sec:superposition_modulo_q}\\

$W$-state: $\frac{1}{\sqrt{n}}\sum_{i = 0}^{n-1}\ket{e_i}$ & $\mathO(n \log n)$ & $\mathO(1)$ & Section~\ref{sec:W_state_in_LAQCC}\\

Dicke-$(n,k)$, $k = \mathO(\sqrt{n})$: $\binom{n}{k}^{-1/2}\sum_{x \in \{0,1\}^n: |x| = k} \ket{x}$ &  $\mathO(n^2\log n)$ & $\mathO(1)$ 
&Section~\ref{sec:dicke:small_k}\\

Dicke-$(n,k)$: $\binom{n}{k}^{-1/2}\sum_{x \in \{0,1\}^n: |x| = k} \ket{x}$ & $\mathO(\text{poly}(n))$ & $\mathO(\log n)$ &Section~\ref{sec:Dicke_in_LAQCC_LOG}\\

QMBS: $\ket{S_k} = \frac{1}{k! \sqrt{\mathcal N(n,k)}}(Q^\dagger)^k \ket{\Omega}$ &  $\mathO(n^2\log n)$ & $\mathO(1)$  &  Section~\ref{sec:many_body_scar}
\end{tabular}
\caption{Summary of state preparation protocols given in this paper.}
\label{tab:sate_prep}
\end{table}
In the entry for the quantum many-body scar state $Q$ denotes the raising operator and $\mathcal N(n,k)=\binom{n-k-1}{k}$. 
Section~\ref{sec:many_body_scar} will provide more details on the variables and the implementation. 

\paragraph{Organization of the paper}
\noindent We first introduce relevant preliminaries in Section~\ref{sec:preliminaries}. 
In Section~\ref{sec:LAQCC_model} we formally define the class of Local Alternating Quantum-Classical Computations ($\LAQCC$). We also show that any Clifford circuit can be implemented in constant depth $\LAQCC$ (a result based on a result from measurement-based quantum computing~\cite{jozsa2006introduction}). 
This result allows us to give many useful multi-qubit gates and routines in Section~\ref{sec:gates_created_in_LAQCC}. 
Beyond that we show that constant depth $\LAQCC$ circuits are contained in $\QNC^1$ and that any $\mathsf{IQP}$ circuit has an $\LAQCC$ implementation.
We conclude this section with an analysis of a more powerful instantiation of $\LAQCC$ and show an inclusion with respect to the class $\mathsf{PostQPoly}$, which is the class of circuits of polynomial depth with one additional post-selection gate. 
In Section~\ref{sec:state_prep_in_LAQCC} we give $\LAQCC$ circuit implementations for preparing the uniform superposition over an arbitrary number of states, the $W$-state and the Dicke state up to $k = \mathO(\sqrt{n})$. We furthermore give a log-depth circuit implementation for preparing the Dicke state for any $k$. We conclude by showing a $\LAQCC$ circuit for generating many body scar states of a particular type of Hamiltonian.


% Figure environment removed
\section{Neutron Star Data}\label{sec:data}
In this work we consider non-rotating neutron stars from a wide range of equations of state. We here give a brief description of the origin and shape of the data sets we utilize for our analysis. For a detailed treatment of the computation of the neutron star models we refer to the original work~\cite{Kuan:2021jmk,2021PhRvD.104b3005M}.
\subsection{Data Sets}
For our analyses, we utilize two different data sets that were used in previous publications: we will henceforth call \textbf{Data set A} the data that was put forward in~\cite{2021PhRvD.104b3005M,kruger2019fast}, while \textbf{Data set B} contains the data originally put forward in~\cite{Kuan:2021jmk}. Both data sets contain models of non-rotating stars of different EoSs, providing the values of a wide range of parameters of these neutron stars.

While Data set A only covers a subset of the EoSs considered in Data set B, it contains some additional features of non-rotating neutron stars that we can include in our analysis. For a comprehensive discussion of the EoSs covered in each data set we refer to each respective publication, however Figure~\ref{fig:EoS} gives and overview through the mass-radius relation of each EoS.

The main purpose of utilizing two different data sets is that it allows us to investigate in how far our qualitative observations regarding, e.g., the relative performance of each correlation measure, generalize to different data. To this end, we treat each data set independently, and do not merge the data to obtain one larger data set. By observing the same behavior independently in both data sets increases the confidence that the observations made here also generalize to other data.

%In the following, we give a brief description of the data and the features therein.
% \begin{itemize}
% \item structure of data
% \item how where they used in their respective papers
% \item why are they useful for this current work
% \end{itemize}

\begin{table}[t]
\def\arraystretch{1.5}%
\centering
\begin{tabularx}{0.8\columnwidth}{X | c | c}
Name & Symbol & Data set \\\hline
Gravitational Mass & $\overline{M} = M/M_\odot$ & A, B \\
Radius & $R$ & A, B \\
Square Root of\newline Average Density & $\tilde \rho = \sqrt{M/R^3}$ & A, B \\ 
Compactness & $C = M/R$ & A, B \\
Moment of Inertia & $\bar I = I/M^3$ & A, B \\
Effective Compactness & $\eta = \sqrt{M^3/I}$ & A \\\hline
$f$-mode frequency & $\omega_f = 2 \pi f_2$  & A, B\\
$g$-mode frequency & $\omega_{g_1} = 2 \pi f_{g_1}$ & B \\
Tidal Deformability & $\bar\lambda = \frac{\lambda}{M^5}$ & A, B \\
\end{tabularx}
\caption{Neutron star features considered in this paper. The last column indicates whether these features are available in Data sets A or B.}
\label{tab:features}
\end{table}

\subsection{Neutron Star Features}
The features considered in our analysis are obtained through either the direct integration of the TOV equations, or through first-order perturbation of the non-rotating neutron star models. The formal description on how these features are obtained are presented in the previous publications that introduced this data~\cite{2021PhRvD.104b3005M,Kuan:2021jmk}. We here summarize the properties of these features. Table~\ref{tab:features} gives an overview of all the features mentioned here.

The first group of features is comprised of macroscopic equilibrium features of the computed neutron star models. In a first step, this includes the \emph{gravitational mass} $M$ (typically normalized $\bar M$ = $M/M_\odot$, where $M_\odot$ is the solar mass), the \emph{radius} $R$ and the \emph{compactness} $C = M/R$. In a second step, we here also consider other neutron star features that have been identified in the literature as useful in the construction of \AU relations. This includes the 
%\emph{central energy density} $\rho_c$, which is an initial parameter for the integration of the neutron star model, the 
square-root of the \emph{average density} $\tilde \rho = \sqrt{M/R^3}$, the \emph{moment of inertia} $I$ (typically normalized $\bar I = I/M^3$) and \emph{effective compactness} $\eta = \sqrt{M^3/I} $ of the neutron star.

All of these equilibrium features we try to correlate to various perturbative features that are computed using linear perturbations: this includes the \emph{tidal deformability} $\lambda$ (typically normalized $\bar \lambda = \lambda/M^5$), the \emph{(angular) $f$-mode frequency} $\omega_f$ and the \emph{(angular) $g$-mode frequency} $\omega_{g_1}$ (we here only consider the first $g$-mode frequency for brevity, but keep the given notation to go along with the notation presented in~\cite{Kuan:2021jmk}). To keep in line with a commonly used notion in the literature~\cite{PhysRevLett.77.4134,Andersson98}, we will denote relations involving the latter as \emph{astroseismological} relations.
%are often called \emph{astroseismological} relations in the literature, and we will sometimes adopt this notion throughout this work.
% to its first overtone, or the second $g$-mode, $\omega_{g_2}$ that is also considered in~\cite{Kuan:2021jmk}). 

% \begin{itemize}
% \item more explicit definition of all used features
% \item relevance of said features for observations
% \item clarify used units
% \end{itemize}
\section{Bivariate Correlation Analysis}\label{sec:bivariate}
The simplest \AU relations try to directly relate two different features of neutron stars, i.e., they are \emph{bivariate} relations. We believe that by evaluating the correlation between different features, we can automate finding such bivariate relations to a high degree. The main issue, however, is identifying which correlation measure is best suited to the task of finding \AU relations (for neutron stars). 

In this section, we describe four different correlation measures that we consider viable for the identification of \AU relations, and describe how we utilize these measures for this purpose.

\subsection{Correlation Measures for Bivariate Relations}
The standard correlation measure that is typically considered when on talks about correlation is the \emph{Pearson correlation coefficient}. In the following, we briefly introduce both this measure, and 3 other correlation measures that are generally considered to perform better when evaluating non-linear correlation in particular. %We evaluate the ability of four different correlation measures to identify universally related features in neutron star model data. We will first briefly introduce each measure, and discuss them below.

\medskip
\noindent
\textbf{Pearson Correlation.}
The Pearson correlation coefficient $\rho$ of two random variables $X$ and $Y$ is given by
\begin{equation}
\rho(X,Y) = \frac{\cov(X,Y)}{\sigma_X \sigma_Y}
\label{eq:pearson}
\end{equation}
where $\cov(X,Y)$ is the covariance of the two random variables, and $\sigma_X$ and $\sigma_Y$ their standard deviations.

\medskip
\noindent
\textbf{Distance Correlation.}
The Distance correlation\cite{DistanceCor} $\dCor$ of two random variables $X$ and $Y$ is defined similarly to the Pearson correlation by 
\begin{equation}
\dCor(X,Y) = \frac{\dCov^2(X,Y)}{\sqrt{\dVar(X)\dVar(Y)}}
\label{eq:distance}
\end{equation}
where the standard notions of covariance and standard deviation are replaced by sample distance covariance $\dCov$ and distance standard deviation $\dVar$. 
While for covariance and standard deviation are computed based the distance of each sample from means $\bar X$ and $\bar Y$ of the random variables, $\dCov$ and $\dVar$ denote similar quantities that are instead based on the \emph{pairwise} distance of all samples.

\medskip
\noindent
\textbf{Mutual Information.}
The mutual information\cite{MutualInf} $I(X;Y)$ of two random variables $X$ and $Y$ is given by
\begin{equation}
I(X;Y) = \sum_{x,y} P_{XY}(x,y) \log \frac{P_{XY}(x,y)}{P_X(x) P_Y(y)}
\label{eq:mutual}
\end{equation}
where $P_{XY}$ is the joint probability distribution of $X$ and $Y$, and $P_X$ and $P_Y$ are the marginal distributions given by
\begin{equation}
P_X(x) = \sum_y P_{X,Y}(x,y)\,.
\end{equation}
Mutual information measures how much we can learn about one random variable $Y$ by having knowledge of another random variable $X$ (or vice versa), and is zero exactly when the two distribution are independent (i.e. knowledge about $X$ does not tell us anything about $Y$). As a quantity, it measures how many bits can be saved if we try to encode $Y$ while assuming knowledge of $X$ (in contrast to encoding $Y$ on its own without any further knowledge).

Technically the above definition for mutual information is for discrete variables, and our use case is centered around continuous random variables. However, in practice, the data vectors we use are discrete, and computational methods have to be used to obtain the sample distribution from the actual sample vectors. In this paper, we rely on the \textsc{mutual\_info\_regression} method implemented in the \textsc{sklearn} Python package.

\medskip
\noindent
\textbf{Maximal Information.}
Maximal information~\cite{MaxInf} is a direct extension of mutual information to continuous variables that puts a given pair of random variables X and Y into histograms of varying bin-sizes, computes the mutual information for each such histogram, and finally chooses the binning that maximizes the mutual information. That is, the maximal information coefficient $\MIC$ of two random variables $X,Y$ is given by
\begin{equation}
\MIC(X;Y) = \max_{B} \frac{I(X;Y)}{\log_2(\min(B_X, B_Y))}
\label{eq:mic}
\end{equation}
where $B$ is the total number of used bins (typically with some upper bound, cf.~\cite{MaxInf}), and $B_X$ and $B_Y$ are the number of bins used for $X$ and $Y$ respectively. To compute the maximal information between two vectors, we will utilize the \textsc{minepy} package for Python~\cite{10.1093/bioinformatics/bts707}.

\medskip
\noindent
\textbf{Comparison of Correlation Measures.}
The main issue with the more prominent Pearson correlation measure is that it only identifies linear correlation of features. While we can adjust to this to some degree by computing some function values of our features (i.e. computing some polynomials or exponential function on the features values), this can become fairly cumbersome in practice. In recent years, especially with the advent of Big Data and the necessity of finding non-linear correlations in various applications, the other above mentioned correlation measures have been developed~\cite{DistanceCor,MaxInf}. 
The main idea behind them is that instead of looking for a global, linear correlation, they instead approximate global correlation by finding local (linear correlation), i.e. correlation of data points that are in close proximity, and generalize it over the whole data set. This applies to both Distance correlation, which to some degree generalizes the Pearson correlation in such a manner, and Maximal Information, which directly generalizes the measure of Mutual Information.

A similar comparison has already been performed in the past by Clark~\cite{clark2013comparison}. They find that, in particular for non-linear relations, distance correlation and mutual/maximal information outperform Pearson correlation in identifying correlated variables. Our purpose for this work is to verify that the same observations can also be made for the use case of finding \AU relations in neutron star model data, and evaluate which correlation measure indeed performs best for this use case. 

\subsection{General Methodology}
Our general approach to evaluating the different correlation measures introduced above, and also for later automatically finding bivariate \AU relations, is the following:
\begin{enumerate}
\item Obtain neutron star model data with features $F_1, \ldots, F_n$ from theoretical/numerical computations.
\item Compute the pairwise correlation of all feature pairs using on of the above correlation measures. This provides us with the correlation matrix $\mathbf{M}$, where the entry $\mathbf{M}_{i,j}$ is the correlation between features $F_i$ and $F_j$.
\item Specify a correlation threshold $\tau$ above which we will consider feature pairs correlated, i.e. find all entries in $\mathbf{M}$ with
\begin{equation}
\mathbf{M}_{i,j} \geq \tau\,.
\end{equation}
This threshold will depend on the correlation measure used, and finding the best value for it is something we want to achieve here, but might need to be further explored in future work.
\item For each selected feature pair, choose a suitable model. Here, model denotes the expected functional relation between the two selected features. This can be, e.g., a linear, polynomial, exponential model, etc. Model selection is a notoriously difficult task in data analysis, and we will here simply choose to evaluate a number of preset templates for the functional relations, and choose the one with the best fit after the following step.
\item Fit the model to the given data to determine the coefficients of the best fit for the \AU relation.
\end{enumerate}



\section{Bivariate Universal Relations}\label{sec:bivariate_relations}
In the following, we inspect the \AU relations found by the correlation measures we discussed in the previous section. For each relation, we will also indicate the correlation value obtained by each respective measure. This will allow us to inspect in which cases each of the correlation measures succeed or fail in correctly identifying features that are suited for \AU relations.

Since the features we correlate cover very different ranges of values, we will evaluate the quality of each proposed \AU relation through the \emph{average relative error} $\bar e$ given by
\begin{equation}
\bar e = \frac{1}{n} \sum_i \frac{\lvert \hat y_i - y_i \rvert}{\lvert y_i\rvert}
\end{equation}
where $\hat y_i$ is the value predicted by the \AU relation, and $y_i$ the actual data point.

In some cases, our automated approach will find an exponential relation between two feature that we are analyzing. We find that by instead fitting for the logarithm of the target feature we achieve better universality. In such cases, after performing the correlation analysis on the regular features, we therefore manually fit a polynomial relation between the logarithm of the target feature and the independent feature. Not that the correlation values, however, will still be given between the regular features, and not after applying the logarithm, as this is how the features are fed into the automatic method described in Section~\ref{sec:bivariate}.

A table summarizing all universal relations presented in this section can be found in the Conclusion~\ref{sec:conclusion}.%Appendix~\ref{sec:relation_table}. 
\subsection{Tidal Deformability Relations}
% Figure environment removed
In Figure~\ref{fig:bivariate_0} we show a \AU relation between the normalized tidal deformability $\bar\lambda$ and the normalized moment of inertia $\bar I$. This relation was also previously put forward by Yagi and Yunes~\cite{Yagi365} as part of their \emph{I-Love-Q} relations. The best fit for this relation is given by the function 
\begin{equation}
\bar I = \num{0.01860855} \log \bar\lambda^2 - \num{0.07584032}\log \bar\lambda + \num{0.33375708}\,.
\label{eq:bivariate_0}
\end{equation}
This relation achieves an average relative error of $\num{0.020266944213251526}$. The fact that the relation itself is non-linear is also reflected in the comparatively low correlation value of $\num{0.9108}$ by the Pearson correlation coefficient.

% Figure environment removed
In Figure~\ref{fig:bivariate_1} we show a \AU relation between the effective compactness $\eta$ and the logarithm of the normalized tidal deformability $\log \bar\lambda$. A similar relation was also previously proposed by us in the context of a binary neutron star merger connecting the pre-merger binary tidal deformability to the post-merger effective compactness~\cite{2021PhRvD.104b3005M}. 

This is a case in which the automated approach yields an exponential relation between $\eta$ and $\bar\lambda$, and as discussed above, we manually fit a polynomial relation for $\log \bar \lambda$, yielding the relation 
\begin{equation}
\log \bar \lambda = \num{-0.09253471}\eta^2 -\num{5.42472023} \eta +\num{13.60418765}%\num[exponent-mode=scientific]{1132453.4706765804} e^{\num{-5.80292988}\eta}\,.
\label{eq:bivariate_1}
\end{equation}
This relation achieves an an average relative error of $\num{0.007502394271055736}$. In this case, the originally exponential relation between the two features causes the Pearson correlation coefficient in particular to give a very low correlation value of $-\num{0.7626}$. In comparison, the other correlation measures still assign a fairly high correlation measures, however the Distance Correlation also begins to assign a lower value of $\num{0.9400}$.

% Figure environment removed
In Figure~\ref{fig:bivariate_2} we show a \AU relation between the compactness $C$ and the logarithm of the normalized tidal deformability $\log\bar\lambda$. Such a relation follows directly from the definition of $\bar\lambda$ in terms of the tidal Love number $k_2$, i.e.
\begin{equation}
\bar\lambda = \frac{\lambda}{M^5} = \frac{2}{3} k_2 \frac{R^5}{M ^5} =\frac{2}{3} k_2 C^5\,.
\end{equation}
%an naturally be obtained by combining the relations between normalized $f$-mode frequency and compactness $C$ put forward by Tsui and Leung~\cite{10.1111/j.1365-2966.2005.08710.x}, and the universal relation between $f$-mode and tidal deformability put forward by Chan et al.~\cite{chan2014multipolar}.
The automatic approach again finds an exponential relation between the features $C$ and $\bar\lambda$, and as before, we find that fitting for $\log\bar\lambda$ instead yields the more accurate, universal relations. The manual fit yields the relation 
%The best fit for this relation is given by the function
\begin{equation}
\log \bar \lambda =  \num{46.1226271} C^2 -\num{53.04505671} C + \num{13.63341181}%\num[exponent-mode=scientific]{259755.6187230266} e^{\num{-40.42696543} C}\,.
\label{eq:bivariate_2}
\end{equation}
This relation achieves a relative error of \num{0.019736506644986718}. As before, the regular features have a highly non-linear, exponential relation fore which the Pearson correlation measure assigns a low correlation value, even though we can observe a strong relation.

% % Figure environment removed

% \clearpage
\subsection{Astroseismological Relations}\label{sec:bivariate_osci}
%For the following relations, we will rely on data presented in~\cite{kruger2019fast,Kuan:2021jmk}. 
We here present some of the astroseismological, universal relations we were able to find for the $f$-mode and $g$-mode oscillation frequencies.

% Figure environment removed
In Figure~\ref{fig:bivariate2_0} we show a \AU relation between the compactness $C$ and the normalized $f$-mode frequency $\bar M \omega_f$. This relation was previously put forward by Tsui and Leung~\cite{PhysRevLett.95.151101}. The best fit for this relation is given by the function
\begin{equation}
\bar M \omega_f = \num{0.04212737} \log C^2+ \num{0.22183278} \log C + \num{0.3145776}\,.
\label{eq:bivariate2_0}
\end{equation}
This relation achieves an average relative error of \num{0.01148544725114905}. While the optimal fit is given by a logarithmic relation, visually the relation can generally be considered to be linear. As expected, in this case even the Pearson correlation coefficient assigns a high value, and the other correlation measures also identify strongly correlated features.

% Figure environment removed
In Figure~\ref{fig:bivariate2_1} we show a \AU relation between the normalized moment of inertia $\bar I$ and the normalized $f$-mode frequency $\bar M \omega_f$. This relation follows straight-forwardly by combining the relation by Tsui and Leung~\cite{10.1111/j.1365-2966.2005.08710.x} between the $f$-mode and compactness $C$, with the understanding that the compactness $C$ and effective compactness $\eta$ can often be used interchangeably in such general relativistic relations. However, to our knowledge, this is the first time that this relation is presented explicitly. 

The best fit for this relation is given by the function
\begin{equation}
\bar M \omega_f = \num{1.01634729} \log {\bar I}^2 - \num{0.93359431} \log \bar I + \num{1.53017764}\,.
\label{eq:bivariate2_1}
\end{equation}
This relation achieves an average relative error of \num{0.006879103949206297}. While the best fit is given by a logarithmic relation, it does not appear to be non-linear to an extreme degree. This is reflected by the correlation measures assigned by all correlation measures. However, even here, the Pearson correlation gives these features a comparatively low correlation value.

% Figure environment removed
Figure~\ref{fig:bivariate2_2} shows a \AU relation between the normalized tidal deformability $\bar \lambda$ and the normalized $f$-mode frequency $\bar M \omega_f$. This relation was also previously put forward by Chan et al.~\cite{chan2014multipolar}. The best fit for this relation is given by the function
\begin{equation}
\bar M \omega_f = 0.0003 \log {\bar \lambda}^2 - \num{0.01490679} \log \bar \lambda + \num{0.12746998}\,.
\label{eq:bivariate2_2}
\end{equation}
This relation achieves an average relative error of of \num{0.01425718984920659}. The highly non-linear, logarithmic relation between these features again causes the Pearson correlation coefficient to fail to detect the correlation between these features, and even the Distance Correlation assigns a comparatively small correlation value. 

% Figure environment removed
Figure~\ref{fig:bivariate2_3} shows a \AU relation between the effective compactness $\eta$ and the normalized $f$-mode frequency $\bar M \omega_f$. This relation was also previously put forward by Lau et al.~\cite{lau2010inferring} and Kr\"uger and Kokkotas~\cite{kruger2019fast}. The best fit for this relation is given by the function
\begin{equation}
\bar M \omega_f = \num{0.7134402} \eta^2 + \num{1.18096983} \eta - \num{0.42236486}\,.
\label{eq:bivariate2_3}
\end{equation}
This relation achieves an average relative error of \num{0.006629639167064198}. Visually, this relation again appears to be mostly linear, which is reflected by all correlation measures assigning a high correlation value.

% Figure environment removed
Figure~\ref{fig:bivariate3_0} shows a \AU relation between the average density $\tilde\rho$ and the $f$-mode frequency $\omega_f$. This relation was also previously put forward by Andersson and Kokkotas~\cite{PhysRevLett.77.4134,Andersson98}. The best fit for this relation is given by the function
\begin{equation}
\omega_f = \num{-2.19908806} \tilde\rho^2 + \num{0.98518922} \tilde\rho + \num{0.007272}\,.
\label{eq:bivariate3_0}
\end{equation}
This relation achieves an average relative error of \num{0.03499534511917183}. Again, the fact that this relation appears to be mostly linear is reflected in the fact that all correlation measures assign a fairly high correlation value to these two features.

% Figure environment removed
Figure~\ref{fig:bivariate2_5} shows a \AU relation between the normalized $f$-mode frequency $R \omega_f$ and the logarithm of the normalized $g$-mode frequency $\log \bar M\omega_{g_1}$. This relation was also previously put forward by Kuan et al.~\cite{Kuan:2021jmk}. As was the case for the relations in Equations~\eqref{eq:bivariate_1} and~\eqref{eq:bivariate_2}, the automatic method finds an exponential relation between the features $R \omega_f$ and $\bar M\omega_{g_1}$. As before, we find that manually fitting the relation for the logarithm $\log \bar M \omega_{g_1}$ gives the more accurate relation, yielding %The best fit for this relation is given by the function
\begin{equation}
\log \bar M\omega_{g_1} = \num{16.05167444} \left(R \omega_f\right)^2 -\num{5.32343308} R \omega_f+ \num{5.58908468}%\num{6.0010885033563595} e^{\num{10.3503197} R \omega_f}\,.
\label{eq:bivariate2_5}
\end{equation}
This relation achieves an average relative error of \num{0.004346355091302037}. Even though the automatic method finds an exponential relation between the original features, the non-linearity of the relation in this case is not as extreme. As such, even the Pearson correlation coefficient achieves a fairly high correlation value, however notably lower than the other correlation measures.

\subsection{Quantitative Comparison of Correlation Measures}
% Figure environment removed

We can perform a more quantitative analysis and comparison of the four different correlation measures by considering some specific performance measures commonly used in statistics. To define these, we first introduce a few notions for binary classifiers. We define them here in terms of our use case of identifying universally related neutron star features: A \emph{true positive} is a pair of features that is universally related, and also identified as such by a given correlation measure. The number of true positives is denoted by $\TP$.

A \emph{false positive} is a pair of features that is \emph{not} universally related, but classified as such. The number of false positives is denoted by $\FP$.

A \emph{false negative} is a pair of features that is universally related, but not classified as such. The number of false negatives is denoted by $\FN$.

A \emph{true negative} is a pair of features that is not universally related, and also not classified as such. The number number of true negatives is denoted by $\TN$.

Given these notions, we can now define performance measures that quantify how well our classifiers correctly label pairs of features. \emph{Recall}, or \emph{true positive rate} $\TPR$ is the rate at which the classifier correctly labels universally related pairs of features as universally related. It is given by
\begin{equation}
\TPR = \frac{\TP}{\TP + \FN}\,.
\end{equation}

\emph{Precision}, or \emph{positive predictive value} ($\PPV$), is the rate of pairs of features classified as universally related that are in fact universally related. It is given by
\begin{equation}
\PPV = \frac{\TP}{\TP + \FP}\,.
\end{equation}

Finally, the \emph{fallout}, or \emph{false positive rate} ($\FPR$), is the rate at which not related pairs of features are classified as being universally related. It is given by
\begin{equation}
\FPR = \frac{\FP}{\FP + \TN}\,.
\end{equation}

We can now compute the precision, recall and fallout for each correlation measure at a given classification threshold $\tau$, and compare how these performance develop with $\tau$. Ideally, we would like to achieve high recall, while keeping precision high, and fallout low.
% Figure environment removed

Typically, one considers the precision-recall and ROC-curves for a better understanding on how these quantities evolve with each other. The precision-recall curves plot the maximum precision achieved by a classifier for a required recall, and allow us to understand how accurate a positive prediction (i.e. classification as universal relation) is, given a specific correlation measure and classification threshold. We show the precision-recall curves for each correlation measure, and one combined plot, in Figure~\ref{fig:prec_rec}.  

The ROC (or receiver operating characteristic) curve plots the recall against the fallout of the classifier. This plot allows us to better understand how many incorrectly classified universal relations we should expect for a given recall requirement. The ROC curves for each correlation measure applied to each of the two data sets considered in this work can be found in Figure~\ref{fig:roc}. 

As we can see in all Figures, the standard Pearson correlation measure is outperformed by the other metrics significantly for most of the recall range. The distance correlation measures, in turn, is also outperformed by the mutual information based measures. Both the kernel-based mutual information measure, and the maximal information measure show high precision and low fallout for high recall values, identifying them as the preferred measures for the task of identifying universally related features.

Note that the above analyses were performed by manually labeling all feature pairs in our limited data set as either being universally related or not in order to obtain the true/false positive/negative counts. As such, the exact values for each performance measure will most likely vary with different datasets and labels. However, the difference in behavior of each correlation measure appears to be significant enough to warrant the conclusion drawn above.
\section{Multivariate Correlation Analysis}\label{sec:multivariate}
Until now, we have only considered the functional relation between two features, and tried to find such pairs of features that allow for  \AU  relations across different equations of state. A straightforward extension then, of course, is to look for multivariate relations, i.e. such relations where one predicted/target features is described in terms of a function that depends on more than one explanatory feature.

The field of high-dimensional data analysis is a widely studied field that in particular gained a lot of notoriety in recent times due to the advent of the big data paradigm. While many different approaches, theories and methods have been developed to deal with high dimensionality in data, we will here consider one very prominent method: principal component analysis (PCA). PCA is a dimensionality reduction and feature extraction technique that has been used to great success in various data analysis use cases~\cite{barnett1987origins}. Recently, Soldateschi et al.~\cite{2021A&A...654A.162S} utilized PCA to construct multivariate  \AU  relations for magnetized neutron stars. Here, we will investigate how we can apply PCA in general to identify potential  \AU  relations, and evaluate how well this approach performs on our own data.

\subsection{Finding Multivariate Correlation using PCA}
The general idea behind PCA is to identify the principal directions in which a given set of data varies the most and use this knowledge to reduce the dimensionality of our data: the principal components are independent vectors that give the direction of maximal variance within our data and allow us to construct an alternative, potentially lower dimensional vector space in which we can still express the majority of the information contained within our data.
Through this process, we can get rid of collinearities in our features, or even identify such features that do not cause any notable variance at all.

While the general use case of PCA does not directly match our goal of constructing  \AU  relations, we can make use of the properties of the principal components to potentially find multivariate  \AU  relations: after computing all principal components of our data, we identify those that show a proportionally large contribution by our target feature $F$ (i.e. what is typically called the \emph{loading} if $F$ within the principal component), if any such component exists. Usually, if there are no strong correlations within our data that lead to a large variance for $F$, all principal components will have a comparatively small contribution by $F$. However, in the case of a principal component that has a large contribution by $F$, we might be able to leverage it to construct a  \AU  relation: by projecting the considered features onto the identified principal component and solving for $F$, we potentially obtain a first-order multivariate \AU relations. 

\subsection{General Methodology}\label{sec:multivariate_methodlogy}
We now describe the general methodology we follow for finding multivariate \AU relations using PCA.
\begin{enumerate}
    \item Select a number of explanatory variables $F_1, \ldots, F_n$ and a target feature $F$.
    \item Perform PCA on the feature set 
    \begin{equation}
    \mathbf{F} = \{F_1, \ldots, F_n, F\}\,.
    \end{equation}
    %\item Find principal component $\mathbf{A}$ that explains variance in target feature $F$. This can be done by investigating the loadings of each feature within the principal components and finding one that has a comparatively large loading for $F$.
    \item For each principal component, solve the equation 
    \begin{equation}
    \begin{split}
       &\mathbf{A} \cdot \mathbf{F} = 0 \\
       \Leftrightarrow\qquad &F = \hat a_1 F_1 + \ldots + \hat a_n F_n
    \end{split}
    \label{eq:proj}
    \end{equation}
    where we denote the right hand side as the new \emph{combined feature} $\hat F$
    \begin{equation}
    \hat F = \hat a_1 F_1 + \ldots + \hat a_n F_n \,.
    \label{eq:combined}
    \end{equation}
    \item Evaluate whether there exists a strong correlation between $F$ and a combined feature $\hat F$ using bivariate correlation analysis.
    \item If strong correlation is found, choose a suitable model and fit it for the relation between $F$ and $\hat F$.
\end{enumerate}
In contrast to the bivariate case, this approach cannot be fully automated yet. A lot of guesswork is involved in identifying the principal components from which we can derive suitable combined features. The most straightforward approach for this task is to simply construct the combined feature for all principal components and then perform a bivariate correlation analysis of the target feature $F$ with each found combined feature.

Also, this method will not always yield universal relations: sometimes, there will be no principal component that will suitably explain the variance in the target feature $F$. This might happen in cases where a) $F$ simply does not present much variance across the whole data set, or b) there exist many co-linearities within the selected set of features $\mathbf{F}$. We discuss some cases where the method described above does not yield a \AU relation in Appendices~\ref{app:multivariate_special} and~\ref{app:multivariate_counter}.

\section{Multivariate Universal Relations}\label{sec:multivariate_relations}
We present the results of using PCA to find multivariate \AU relations for neutron stars as described in the previous section. A table summarizing all universal relations presented in this section can be found in the Conclusion~\ref{sec:conclusion}.

\subsection{Multivariate Universal Relations for Tidal Deformability}\label{sec:multivariate_tidal}
We here consider the case where we want to construct a \AU relation for the normalized tidal deformability $\bar\lambda$, using the features $M$, $R$ and $C$. To this end, we perform the principal component analysis on all 4 features using Data set A (cf. Section~\ref{sec:data}). The resulting principal components are given in Table~\ref{tab:multi_example} by means of the loading of each feature within the principal components. A visual representation of the combined feature obtained from each principal component is shown in Appendix~\ref{app:multivariate_details}. 

Performing the bivariate correlation analysis of the combined features derived from each principal component with the target feature $\bar\lambda$ shows that the best correlation is given by Principal Component 0. Since the automated approach finds an exponential relation between $\bar \lambda$ and the combined feature, and we again fit for $\log \bar \lambda$ to obtain a more accurate relation. 
Through our manual fit, we obtain the following \AU relation for the normalized tidal deformability
\begin{equation}
\log \bar \lambda = \num{-0.63455955} \hat F +\num{7.39926917}%\num{9.076890419634398e+27} e^{\num{0.97371172} \hat F}
\label{eq:tidal_relation}
\end{equation}
with 
\begin{equation}
\hat F = \num{3.390534206245} \frac{M}{M_\odot} - \num{5.2414369999999995} \frac{R}{10 km} + \num{4.768246}\frac{C}{0.2}.
\label{eq:combined_tidal}
\end{equation}
This relation is presented in Figure~\ref{fig:multi_fit} and achieves an average relative error of $\num{0.023403790657464087}$. 
Compared to the bivariate relation between the tidal deformability and compactness we presented in Figure~\ref{fig:bivariate_2}, we essentially introduce a linear order correction involving the radius and the mass. While the overall relative error is approximately the same as for the bivariate relation, the multivariate relation remains entirely linear in all independent variables, reducing its sensitivity to potential estimation errors for these quantities.

% Figure environment removed

\begin{table}
\def\arraystretch{1.5}%
\centering
\begin{tabular}{c | r | r | r | r | r}
Component & $M$ & $R$ & $C$ & $\lambda$ \\\hline
0 & \num{-0.48793098} & \num{0.41864877} & \num{-0.57054369}  & \num{0.51101514} \\
1 & \num{ 0.59621943}  & \num{0.7932779}  & \num{-0.03405559} & \num{-0.11862875} \\
2 & \num{ 0.32170168} & \num{-0.09744886} & \num{ 0.41194891}  & \num{0.84694147} \\
3 & \num{ 0.55041238} & \num{-0.43121584} & \num{-0.70966063}  & \num{0.08649222}
\end{tabular}
\caption{Loadings of features in each principal component obtained from performing PCA on the feature set $\mathbf{F} = \{M, R, C, \bar\lambda\}$ on \textbf{Data set A}.}
\label{tab:multi_example}
\end{table}

\medskip
\noindent
\textbf{Relation with Data set B.}
We also perform the same analysis using the data by Kuan et al.~\cite{Kuan:2021jmk}. The principal components obtained from the PCA are listed in Table~\ref{tab:multi_example_Kuan}. The principal components show a similar behavior to the previous examples using \textbf{Data set A}, however we can observe some slight differences caused by the different equations of state used in the data set. 

As before, after performing the bivariate correlation analysis on the combined features derived from each principal component, we find that the combined feature derived from Principal Component 0 shows the best universality. Leveraging this component, we obtain the \AU relation 
\begin{equation}
\log \bar \lambda = \num{-0.93929701} \hat F + \num{6.52146358}
\label{eq:tidal_relation_Kuan2}
\end{equation}
this time with the combined feature
\begin{equation}
\hat F = \num{2.24908701354} \frac{M}{M_\odot} - \num{4.315921} \frac{R}{10 km} + \num{3.5326000000000004}\frac{C}{0.2}\,.
\label{eq:tidal_relation_Kuan}
\end{equation}
The resulting best fit is presented in Figure~\ref{fig:multi_fit_Kuan}. It achieves an average relative error of $\num{0.042847553820507985}$, which is slightly higher than what we achieved for \textbf{Data set A}. We suspect this is caused by some of the outlying neutron star models that are introduced by the larger configuration space considered in \textbf{Data set B}. 

However, the fact remains that our approach for the multivariate correlation analysis yields the same form for the universal relation independent of which data set is used. This is indicative of this approach further generalizing well for different data sets, and that the results presented here are not dependent on the underlying data used for the analysis.

\begin{table}
\def\arraystretch{1.5}%
\centering
\begin{tabular}{c | r | r | r | r | r}
Component & M & R & $C$ & $\bar \lambda$ \\\hline
0 & \num{-0.5142147 } & \num{0.34755964} & \num{-0.59262472} & \num{0.51340187} \\
1 & \num{0.55470435} & \num{0.79888905} & \num{0.14483504} & \num{0.18194005} \\
2 & \num{0.12718581} & \num{-0.35179198}  & \num{0.40576465}  & \num{0.83391919} \\
3 & \num{-0.6416464}  & \num{ 0.3423755}   & \num{0.68056873} & \num{-0.08885448}
\end{tabular}
\caption{Loadings of features in each principal component obtained from performing PCA on the feature set $\mathbf{F} = \{M, R, C, \bar\lambda\}$ on \textbf{Data set B}.}
\label{tab:multi_example_Kuan}
\end{table}

% Figure environment removed

\subsection{Multivariate Astroseismological Relations}\label{sec:multivariate_oscillation}
Andersson and Kokkotas~\cite{PhysRevLett.77.4134,Andersson98} previously proposed a \AU relation linking the average density $\tilde\rho$ to the $f$-mode frequency of a neutron star. We here attempt to apply the same method as above to potentially find corrections to their original astroseismological relation that improve its universality. To this end, we perform the principal component analysis on the features $\omega_f$, $M$, $C$ and $\tilde\rho$, aiming at finding corrections in terms of $M$ and $C$ for the \AU relation.

The best relation is found for the combined feature derived from the fourth principal component found through PCA performed in the feature set $\mathbf{F} = \{M, C, \tilde\rho, \omega_f\}$. . The best fit for the relation between $\omega_f$ and this combined feature is shown in Figure~\ref{fig:multi_fit_fmode}. The best fit shows a quadratic \AU relation for the $f$-mode frequency of the form
\begin{equation}
\omega_f = -0.00033 \hat F^2 + \num{0.012682} \hat F - \num{0.02348942}
\label{eq:relation_fmode}
\end{equation}
with
\begin{equation}
\hat F = \num{2.9800476501500004} \frac{M}{M_\odot} + \num{10.230908}\frac{\tilde\rho}{0.04} - \num{8.39804} \frac{C}{0.2}\,.
\label{eq:combined_fmode}
\end{equation}
This relation achieves an average relative error of $\num{0.015469332922977337}$. When compared to the old relation shown in Figure~\ref{fig:bivariate3_0}, we can clearly observe an improved universality, which is also reflected in the average relative error that is reduced by half. We thus achieve a significant improvement over the existing relation by using our multivariate approach. 

% Figure environment removed

\subsection{Improved Astroseismological Relations for the f-mode Frequency}\label{sec:multivariate_example3}
We next consider another variation on the astroseismological relation we inspected above. This time, instead of introducing mass and compactness as independent variables, we instead only introduce the product $C \tilde\rho$ of compactness and average density as a new independent variable.
Our goal now is therefore to find a \AU relation for $\omega_f$ using the average density $\tilde\rho$ and $C \tilde\rho$. 

In this case, the best relation is found for the combined feature derived from the third principal component found through the PCA performed in the feature set $\mathbf{F} = \{\tilde\rho, C \tilde\rho, \omega_f\}$. 
The best fit for the relation between $\omega_f$ and this combined feature is shown in Figure~\ref{fig:multi_fit_zmode}. The best fit shows a quadratic \AU relation for the $f$-mode frequency of the form
\begin{equation}
\omega_f = 0.0002 \hat F^2 + \num{0.00647675} \hat F + \num{0.00276012}
\label{eq:multivariate_relation_redshift}
\end{equation}
with
\begin{equation}
\hat F = \num{6.9109313199999995}\frac{\tilde\rho}{0.04} - \num{1.71649574} \frac{C \tilde\rho}{0.01}\,. %\num{1.71649574} \frac{z \tilde\rho}{0.02}
\label{eq:combined_feature_imp_zmode}
\end{equation}

When compared to the relation shown in the section above (cf. Figure~\ref{fig:multi_fit_fmode}) we observe an improved universality: the previous relation has an average relative error of $\num{0.015469332922977337}$, whereas the relation with the new combined feature achieves an error of $\num{0.009738598872994107}$. 

Considering that the original relation put forward by Andersson and Kokkotas~\cite{PhysRevLett.77.4134,Andersson98} was inspired by Newtonian gravity, the additional factor in $C \tilde\rho$ could be considered a first order correction to account for general relativity, since
\begin{equation}
\hat F = \num{172.773283} \tilde\rho - \num{171.649574}C \tilde\rho \approx 172 \tilde\rho \left(1-C\right)\,.
\end{equation}
Essentially, this new relation is a stepping-stone between the relation by Andersson and Kokkotas~\cite{PhysRevLett.77.4134,Andersson98}, and other general relativistic \AU relations, such as the one between the $f$-mode frequency $\omega_f$ and the compactness $C$ put forward by Tsui and Leung~\cite{PhysRevLett.95.151101}.

% Figure environment removed


\subsection{Discussion of Results}
As we have demonstrated above, we can utilize the principal components obtained from PCA to construct multivariate \AU relations for neutron stars. Since the relations we construct are, for now, first-order relations, this approach is also suited for finding linear order corrections to suspected \AU relations, allowing an improvement of the accuracy of the \AU relations.

Despite these positive results, our approach here has only been descriptive: while we provide a methodology that can yield multivariate \AU relations, the formal reasons for why this approach works is still not fully clear. Gaining further understanding of the mathematical underpinnings of this approach can allow us to further improve its output, but also better understand its limits. 

For instance, in Appendices~\ref{app:multivariate_special} and~\ref{app:multivariate_counter}, we show some cases where our approach will not yield \AU relations. Sometimes this is caused by the data used, as, ultimately, not all feature combinations will be amenable to \AU relations. Furthermore, specific properties of the used data, such as the existence of strong collinearities with the target feature, can also hinder our approach from producing \AU relations. We currently can only provide intuitive reasons for why our approach does not perform well in such situations, and we hope to obtain a more rigorous understanding through future work. 
\section{Conclusion and Future Work}
In this work, I design corruption-robust algorithms for the Lipschitz contextual search problem. I present the \emph{agnostic checking} technique and demonstrate its effectiveness in designing corruption-robust algorithms. There are several open problems for future research. First, in the algorithm I propose for pricing loss, the schedule for agnostic checks is fixed upfront. Can the learner design an adaptive checking schedule for the pricing loss? Second, this work assumes the learner has knowledge of the Lipschitz constant $L$. Can the learner design efficient no-regret algorithms without knowledge of $L$? 
\bibliographystyle{apsrev4-2}
\bibliography{lit}
\onecolumngrid
\clearpage
\twocolumngrid
\appendix
\begin{comment}
\section{System Architecture}
\label{appendix:architecture}
\system has a novel modularized system architecture with three key components: 
\emph{StreamManager}, 
\emph{TxnManager} and \emph{TxnScheduler}. 
These components are instantiated in each thread locally.
The execution outline of \system is presented in Algorithm~\ref{alg:algo}.
Transactional stream processing is continuous and potentially never ends (Line 1$\sim$8).
The dependency resolution and execution of state transactions are separated into two non-overlapping phases by punctuations~\cite{Tucker:2003:EPS:776752.776780} (Line 2 and 5), which guarantees that no subsequent input event will have a smaller timestamp. 
Effectively, a batch of state transactions is collected during the first phase, and processed during the second phase.

In the first phase (i.e., stream processing phase), 
the \emph{StreamManager} conducts preprocessing for every input event ($e$). Similar to some prior works~\cite{tstream}, state transactions may be issued but not immediately processed during preprocessing (Line 3).
The \emph{pre\_processing} and \emph{post\_processing} functions are exposed as APIs to users.
The \emph{TxnManager} handles dependency resolution (Line 4) among state transactions and insert decomposed operations to construct a \tpg. We discuss the detailed two-phase \tpg construction process in Section~\ref{subsec:construction}.

In the second phase  (i.e., transaction processing phase), 
the \emph{TxnManager} is first involved again to refine (Line 6) the constructed \tpg with further dependency resolution.
The \emph{TxnScheduler} 
schedules operations for concurrent execution based on the constructed \tpg according to the three dimensions of scheduling decisions (Line 7). 
In particular, a scheduling decision model $M$ is instantiated based on the constructed \tpg (Line 14).
\textbf{\circled{1}} Guided by $M$, execution threads adopt an exploration strategy (Section~\ref{subsec:explore}) to explore the constructed \tpg for operations available to be scheduled constrained by dependencies. 
\textbf{\circled{2}} 
During exploration, one or multiple operations may be treated as the 
% basic 
unit of scheduling (Section~\ref{subsec:granularity}). 
Subsequently, \textbf{\circled{3}} every thread executes operation(s) in the unit of scheduling with various abort handling mechanisms (Section~\ref{subsec:abort_handling}).
Only when state transactions are processed (i.e., committed or aborted) can the associated input events be postprocessed (Line 8) by the \emph{StreamManager} based on transaction processing results.
\end{comment}

\begin{comment}
\begin{algorithm}
\footnotesize
    \KwData{$e$ \tcp{Input event}}
    \KwData{$txn_{ts}$ \tcp{State transaction}}
    \KwData{$G$ \tcp{The currently constructed TPG}}
    \While{!finish processing of input streams}{
        \eIf(\tcp*[h]{Phase 1}){\text{$e$ is not a $punctuation$}}{
                $txn_{ts}$ $\gets$ PRE\_Processing($e$)\;
                \textbf{TPG\_Construction}($G$, $txn_{ts}$)\; 
          }(\tcp*[h]{Phase 2}){
                \textbf{TPG\_Refinement}($G$)\; 
                \textbf{TXN\_Scheduling}($G$)\; 
                POST\_Processing()\;
          }
    }
    
    \SetKwFunction{FMain}{TPG\_Construction}
    \SetKwProg{Fn}{Function}{:}{}
    \Fn{\FMain{$G$, $txn_{ts}$}}{
        $O_{1..k}$ $\gets$ \textbf{Partition} $txn_{ts}$\;
        \ForEach{\text{operation $O_{i}$ $\in$ $O_{1..k}$}}{
            \textbf{Identify} its \ld\;
            $G$ $\gets$ $G$ + $O_{i}$ \;
        }
    }
    \SetKwFunction{FMain}{TPG\_Refinement}
    \SetKwProg{Fn}{Function}{:}{}
    \Fn{\FMain{$G$}}{
        \ForEach{\text{vertex $e_{i}$ $\in$ $G$}}{
            \textbf{Identify} its \td, \pd\;
        }
    }
    
    \SetKwFunction{FMain}{TXN\_Scheduling}
    \SetKwProg{Fn}{Function}{:}{}
    \Fn{\FMain{$G$}}{
        $M$ $\gets$ Instantiated with $G$;\tcp{A decision model}
        \While{!finish scheduling of $G$
        }{
          \textbf{\circled{2}} $Scheduling Unit$ $\gets$ \textbf{\circled{1}} \emph{Explore}($G$, $M$)\; 
            \textbf{\circled{3}} \emph{Execute with Abort Handling} ($Scheduling Unit$)\; 
        }
    }
  \caption{Execution Outline of \system}
  \label{alg:algo}
\end{algorithm}
\end{comment}
\end{document}
\typeout{get arXiv to do 4 passes: Label(s) may have changed. Rerun}
