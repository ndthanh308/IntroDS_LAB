% Figure environment removed
\section{Neutron Star Data}\label{sec:data}
In this work we consider non-rotating neutron stars from a wide range of equations of state. We here give a brief description of the origin and shape of the data sets we utilize for our analysis. For a detailed treatment of the computation of the neutron star models we refer to the original work~\cite{Kuan:2021jmk,2021PhRvD.104b3005M}.
\subsection{Data Sets}
For our analyses, we utilize two different data sets that were used in previous publications: we will henceforth call \textbf{Data set A} the data that was put forward in~\cite{2021PhRvD.104b3005M,kruger2019fast}, while \textbf{Data set B} contains the data originally put forward in~\cite{Kuan:2021jmk}. Both data sets contain models of non-rotating stars of different EoSs, providing the values of a wide range of parameters of these neutron stars.

While Data set A only covers a subset of the EoSs considered in Data set B, it contains some additional features of non-rotating neutron stars that we can include in our analysis. For a comprehensive discussion of the EoSs covered in each data set we refer to each respective publication, however Figure~\ref{fig:EoS} gives and overview through the mass-radius relation of each EoS.

The main purpose of utilizing two different data sets is that it allows us to investigate in how far our qualitative observations regarding, e.g., the relative performance of each correlation measure, generalize to different data. To this end, we treat each data set independently, and do not merge the data to obtain one larger data set. By observing the same behavior independently in both data sets increases the confidence that the observations made here also generalize to other data.

%In the following, we give a brief description of the data and the features therein.
% \begin{itemize}
% \item structure of data
% \item how where they used in their respective papers
% \item why are they useful for this current work
% \end{itemize}

\begin{table}[t]
\def\arraystretch{1.5}%
\centering
\begin{tabularx}{0.8\columnwidth}{X | c | c}
Name & Symbol & Data set \\\hline
Gravitational Mass & $\overline{M} = M/M_\odot$ & A, B \\
Radius & $R$ & A, B \\
Square Root of\newline Average Density & $\tilde \rho = \sqrt{M/R^3}$ & A, B \\ 
Compactness & $C = M/R$ & A, B \\
Moment of Inertia & $\bar I = I/M^3$ & A, B \\
Effective Compactness & $\eta = \sqrt{M^3/I}$ & A \\\hline
$f$-mode frequency & $\omega_f = 2 \pi f_2$  & A, B\\
$g$-mode frequency & $\omega_{g_1} = 2 \pi f_{g_1}$ & B \\
Tidal Deformability & $\bar\lambda = \frac{\lambda}{M^5}$ & A, B \\
\end{tabularx}
\caption{Neutron star features considered in this paper. The last column indicates whether these features are available in Data sets A or B.}
\label{tab:features}
\end{table}

\subsection{Neutron Star Features}
The features considered in our analysis are obtained through either the direct integration of the TOV equations, or through first-order perturbation of the non-rotating neutron star models. The formal description on how these features are obtained are presented in the previous publications that introduced this data~\cite{2021PhRvD.104b3005M,Kuan:2021jmk}. We here summarize the properties of these features. Table~\ref{tab:features} gives an overview of all the features mentioned here.

The first group of features is comprised of macroscopic equilibrium features of the computed neutron star models. In a first step, this includes the \emph{gravitational mass} $M$ (typically normalized $\bar M$ = $M/M_\odot$, where $M_\odot$ is the solar mass), the \emph{radius} $R$ and the \emph{compactness} $C = M/R$. In a second step, we here also consider other neutron star features that have been identified in the literature as useful in the construction of \AU relations. This includes the 
%\emph{central energy density} $\rho_c$, which is an initial parameter for the integration of the neutron star model, the 
square-root of the \emph{average density} $\tilde \rho = \sqrt{M/R^3}$, the \emph{moment of inertia} $I$ (typically normalized $\bar I = I/M^3$) and \emph{effective compactness} $\eta = \sqrt{M^3/I} $ of the neutron star.

All of these equilibrium features we try to correlate to various perturbative features that are computed using linear perturbations: this includes the \emph{tidal deformability} $\lambda$ (typically normalized $\bar \lambda = \lambda/M^5$), the \emph{(angular) $f$-mode frequency} $\omega_f$ and the \emph{(angular) $g$-mode frequency} $\omega_{g_1}$ (we here only consider the first $g$-mode frequency for brevity, but keep the given notation to go along with the notation presented in~\cite{Kuan:2021jmk}). To keep in line with a commonly used notion in the literature~\cite{PhysRevLett.77.4134,Andersson98}, we will denote relations involving the latter as \emph{astroseismological} relations.
%are often called \emph{astroseismological} relations in the literature, and we will sometimes adopt this notion throughout this work.
% to its first overtone, or the second $g$-mode, $\omega_{g_2}$ that is also considered in~\cite{Kuan:2021jmk}). 

% \begin{itemize}
% \item more explicit definition of all used features
% \item relevance of said features for observations
% \item clarify used units
% \end{itemize}