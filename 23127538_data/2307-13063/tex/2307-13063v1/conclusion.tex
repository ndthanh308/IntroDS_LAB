\begin{table*}
\centering
\def\arraystretch{2}
\begin{tabularx}{\linewidth}{l | c | X | c | c | c | c}
Type & Features & Form & Avg. rel. Error & Equation & Figure & Reference \\\hline
\multirow{9}{*}{\rotatebox[origin=c]{90}{\Large{Bivariate}}}
& $\bar \lambda$, $\bar I$ & $\bar I = \num{0.01860855} \log \bar\lambda^2 - \num{0.07584032}\log \bar\lambda + \num{0.33375708}$ & $\num{0.020266944213251526}$ & \eqref{eq:bivariate_0} & \ref{fig:bivariate_0} & \cite{Yagi365} \\
& $\bar \lambda$, $\eta$ 	 & $\log \bar \lambda = \num{-0.09253471}\eta^2 -\num{5.42472023} \eta +\num{13.60418765}$ 				& $\num{0.007502394271055736}$ & \eqref{eq:bivariate_1} & \ref{fig:bivariate_1} & \cite{2021PhRvD.104b3005M}\\
& $\bar \lambda$, $C$ & $\log \bar \lambda =  \num{46.1226271} C^2 -\num{53.04505671} C + \num{13.63341181}$ & $\num{0.019736506644986718}$ & \eqref{eq:bivariate_2} & \ref{fig:bivariate_2} & / \\
& $\bar M \omega_f$, $C$ & $\bar M \omega_f = \num{0.04212737} \log C^2+ \num{0.22183278} \log C + \num{0.3145776}$ & $\num{0.01148544725114905}$ & \eqref{eq:bivariate2_0} & \ref{fig:bivariate2_0} & \cite{PhysRevLett.95.151101} \\
& $\bar M \omega_f$, $\bar I$ & $\bar M \omega_f = \num{1.01634729} \log {\bar I}^2 - \num{0.93359431} \log \bar I + \num{1.53017764}$ & $\num{0.006879103949206297}$ & \eqref{eq:bivariate2_1} & \ref{fig:bivariate2_1} & /\\
& $\bar M \omega_f$, $\bar \lambda$ & $\bar M \omega_f = 0.0003 \log {\bar \lambda}^2 - \num{0.01490679} \log \bar \lambda + \num{0.12746998}$ & $\num{0.01425718984920659}$ & \eqref{eq:bivariate2_2} & \ref{fig:bivariate2_2} & \cite{chan2014multipolar}\\
& $\bar M \omega_f$, $\eta$ & $\bar M \omega_f = \num{0.7134402} \eta^2 + \num{1.18096983} \eta - \num{0.42236486}$ & $\num{0.006629639167064198}$ &\eqref{eq:bivariate2_3} & \ref{fig:bivariate2_3} & \cite{lau2010inferring,2021FrASS...8..166K} \\
& $\omega_f$, $\tilde\rho$ & $\omega_f = \num{-2.19908806} \tilde\rho^2 + \num{0.98518922} \tilde\rho + \num{0.007272}$ & $\num{0.03499534511917183}$ & \eqref{eq:bivariate3_0} & \ref{fig:bivariate3_0} & \cite{PhysRevLett.77.4134,Andersson98}\\
& $\bar M \omega_{g_1}$, $R \omega_f$ & $\log \bar M\omega_{g_1} = \num{16.05167444} \left(R \omega_f\right)^2 -\num{5.32343308} R \omega_f+ \num{5.58908468}$ & $\num{0.004346355091302037}$ & \eqref{eq:bivariate2_5} & \ref{fig:bivariate2_5} & \cite{Kuan:2021jmk} \\\hline
\multirow{3}{*}{\rotatebox[origin=c]{90}{\Large{Multivariate\quad}}}
& $\bar\lambda$, $M$, $R$, $C$ & $\log \bar \lambda = \num{-0.63455955} \hat F +\num{7.39926917}$ \newline $\hat F = \num{3.390534206245} \frac{M}{M_\odot} - \num{5.2414369999999995} \frac{R}{10 km} + \num{4.768246}\frac{C}{0.2}$ & $\num{0.023403790657464087}$ & \eqref{eq:tidal_relation} & \ref{fig:multi_fit} & / \\
& $\omega_{f}$, $M$, $\tilde\rho$, $C$ & $\omega_f = -0.00033 \hat F^2 + \num{0.012682} \hat F - \num{0.02348942}$ \newline $\hat F = \num{2.9800476501500004} \frac{M}{M_\odot} + \num{10.230908}\frac{\tilde\rho}{0.04} - \num{8.39804} \frac{C}{0.2}$ & $\num{0.015469332922977337}$ & \eqref{eq:relation_fmode} & \ref{fig:multi_fit_fmode} & / \\
& $\omega_{f}$, $\tilde\rho$, $C\tilde\rho$ & $\omega_f = 0.0002 \hat F^2 + \num{0.00647675} \hat F + \num{0.00276012}$\newline $\hat F = \num{6.9109313199999995}\frac{\tilde\rho}{0.04} - \num{1.71649574} \frac{C \tilde\rho}{0.01}$ & $\num{0.009738598872994107}$ & \eqref{eq:multivariate_relation_redshift} & \ref{fig:multi_fit_zmode} & / \\\hline
\end{tabularx}
\caption{List of all universal relations presented in this work.}
\label{tab:relations}
\end{table*}
\section{Conclusion \& Future Directions}\label{sec:conclusion}
In this work, we discussed the potential of approaching the task of constructing \emph{universal} relations for neutron stars from a statistical data analysis point of view. Instead of relying on physical intuition, our goal was to approach neutron star data using statistical methods only and thus enable a more automated approach to finding \AU relations. 

In a first step, we investigated the suitability of four different correlations measures for identifying pairs of features amenable to bivariate \AU relations. We found that the usual Pearson correlation measure will have difficulties with non-linear relations between features, which has also been observed in the past in the statistical data analysis literature for more general use cases~\cite{clark2013comparison}. Using generalized correlation measures that were explicitly constructed to detect non-linear correlations proved more useful: overall, Mutual Information and Maximal Information both performed best in finding universally related features, and while Distance correlation did not perform as well as the aforementioned ones, it still outperformed Pearson correlation for our use case. % seem to strike a balance between being too permissive, and finding non-linear correlations between features for the specific use-case of universal relations for neutron stars.

In a second step, we also approached the problem of constructing multivariate \AU relations. Inspired by an idea presented in~\cite{2021A&A...654A.162S}, we used the principal components found through PCA to construct a new combined feature that we then related to a initially selected target feature. While this approach is not yet fully automated and requires manual considerations in some steps, our results show that this approach can yield highly accurate, multivariate \AU relations. Our approach works particularly well when we try to find first-order corrections to previously known bivariate relations. For instance, we were able to construct an entirely novel \AU relation that allows us to relate the $f$-mode frequency to the average density and compactness of the neutron stars, significantly improving the error of the relation compared to existing bivariate relations.

In Table~\ref{tab:relations} we give an overview of all universal relations presented in this paper. For each relation, we indicate which features are connected through these relations, their form, and the average relative error achieved through our best fits. We also give references to all corresponding equations and figures in this paper. Finally, if a relation was already presented previously in a different work, we also give a reference to that work.

In a time where theoretical model data for various (astro-)physical objects becomes more widely available, finding useful data analysis tools for the specific use-cases that we are interested in will be an important direction of work that will later enable more comprehensive data exploration. The methods discussed in this paper present a first step into this direction.

For future work, a straightforward extension is the application of the presented methods to even more and different neutron star data. While we have only considered non-rotating neutron stars in this paper, the presented methods should easily apply to other configurations including rotation or magnetic fields. Furthermore, gaining deeper understanding on why and under which constraints the PCA approach will work well can allow us to, in the future, reduce the amount of manual intervention that is still required right now.


