\section{Multivariate Universal Relations}\label{sec:multivariate_relations}
We present the results of using PCA to find multivariate \AU relations for neutron stars as described in the previous section. A table summarizing all universal relations presented in this section can be found in the Conclusion~\ref{sec:conclusion}.

\subsection{Multivariate Universal Relations for Tidal Deformability}\label{sec:multivariate_tidal}
We here consider the case where we want to construct a \AU relation for the normalized tidal deformability $\bar\lambda$, using the features $M$, $R$ and $C$. To this end, we perform the principal component analysis on all 4 features using Data set A (cf. Section~\ref{sec:data}). The resulting principal components are given in Table~\ref{tab:multi_example} by means of the loading of each feature within the principal components. A visual representation of the combined feature obtained from each principal component is shown in Appendix~\ref{app:multivariate_details}. 

Performing the bivariate correlation analysis of the combined features derived from each principal component with the target feature $\bar\lambda$ shows that the best correlation is given by Principal Component 0. Since the automated approach finds an exponential relation between $\bar \lambda$ and the combined feature, and we again fit for $\log \bar \lambda$ to obtain a more accurate relation. 
Through our manual fit, we obtain the following \AU relation for the normalized tidal deformability
\begin{equation}
\log \bar \lambda = \num{-0.63455955} \hat F +\num{7.39926917}%\num{9.076890419634398e+27} e^{\num{0.97371172} \hat F}
\label{eq:tidal_relation}
\end{equation}
with 
\begin{equation}
\hat F = \num{3.390534206245} \frac{M}{M_\odot} - \num{5.2414369999999995} \frac{R}{10 km} + \num{4.768246}\frac{C}{0.2}.
\label{eq:combined_tidal}
\end{equation}
This relation is presented in Figure~\ref{fig:multi_fit} and achieves an average relative error of $\num{0.023403790657464087}$. 
Compared to the bivariate relation between the tidal deformability and compactness we presented in Figure~\ref{fig:bivariate_2}, we essentially introduce a linear order correction involving the radius and the mass. While the overall relative error is approximately the same as for the bivariate relation, the multivariate relation remains entirely linear in all independent variables, reducing its sensitivity to potential estimation errors for these quantities.

% Figure environment removed

\begin{table}
\def\arraystretch{1.5}%
\centering
\begin{tabular}{c | r | r | r | r | r}
Component & $M$ & $R$ & $C$ & $\lambda$ \\\hline
0 & \num{-0.48793098} & \num{0.41864877} & \num{-0.57054369}  & \num{0.51101514} \\
1 & \num{ 0.59621943}  & \num{0.7932779}  & \num{-0.03405559} & \num{-0.11862875} \\
2 & \num{ 0.32170168} & \num{-0.09744886} & \num{ 0.41194891}  & \num{0.84694147} \\
3 & \num{ 0.55041238} & \num{-0.43121584} & \num{-0.70966063}  & \num{0.08649222}
\end{tabular}
\caption{Loadings of features in each principal component obtained from performing PCA on the feature set $\mathbf{F} = \{M, R, C, \bar\lambda\}$ on \textbf{Data set A}.}
\label{tab:multi_example}
\end{table}

\medskip
\noindent
\textbf{Relation with Data set B.}
We also perform the same analysis using the data by Kuan et al.~\cite{Kuan:2021jmk}. The principal components obtained from the PCA are listed in Table~\ref{tab:multi_example_Kuan}. The principal components show a similar behavior to the previous examples using \textbf{Data set A}, however we can observe some slight differences caused by the different equations of state used in the data set. 

As before, after performing the bivariate correlation analysis on the combined features derived from each principal component, we find that the combined feature derived from Principal Component 0 shows the best universality. Leveraging this component, we obtain the \AU relation 
\begin{equation}
\log \bar \lambda = \num{-0.93929701} \hat F + \num{6.52146358}
\label{eq:tidal_relation_Kuan2}
\end{equation}
this time with the combined feature
\begin{equation}
\hat F = \num{2.24908701354} \frac{M}{M_\odot} - \num{4.315921} \frac{R}{10 km} + \num{3.5326000000000004}\frac{C}{0.2}\,.
\label{eq:tidal_relation_Kuan}
\end{equation}
The resulting best fit is presented in Figure~\ref{fig:multi_fit_Kuan}. It achieves an average relative error of $\num{0.042847553820507985}$, which is slightly higher than what we achieved for \textbf{Data set A}. We suspect this is caused by some of the outlying neutron star models that are introduced by the larger configuration space considered in \textbf{Data set B}. 

However, the fact remains that our approach for the multivariate correlation analysis yields the same form for the universal relation independent of which data set is used. This is indicative of this approach further generalizing well for different data sets, and that the results presented here are not dependent on the underlying data used for the analysis.

\begin{table}
\def\arraystretch{1.5}%
\centering
\begin{tabular}{c | r | r | r | r | r}
Component & M & R & $C$ & $\bar \lambda$ \\\hline
0 & \num{-0.5142147 } & \num{0.34755964} & \num{-0.59262472} & \num{0.51340187} \\
1 & \num{0.55470435} & \num{0.79888905} & \num{0.14483504} & \num{0.18194005} \\
2 & \num{0.12718581} & \num{-0.35179198}  & \num{0.40576465}  & \num{0.83391919} \\
3 & \num{-0.6416464}  & \num{ 0.3423755}   & \num{0.68056873} & \num{-0.08885448}
\end{tabular}
\caption{Loadings of features in each principal component obtained from performing PCA on the feature set $\mathbf{F} = \{M, R, C, \bar\lambda\}$ on \textbf{Data set B}.}
\label{tab:multi_example_Kuan}
\end{table}

% Figure environment removed

\subsection{Multivariate Astroseismological Relations}\label{sec:multivariate_oscillation}
Andersson and Kokkotas~\cite{PhysRevLett.77.4134,Andersson98} previously proposed a \AU relation linking the average density $\tilde\rho$ to the $f$-mode frequency of a neutron star. We here attempt to apply the same method as above to potentially find corrections to their original astroseismological relation that improve its universality. To this end, we perform the principal component analysis on the features $\omega_f$, $M$, $C$ and $\tilde\rho$, aiming at finding corrections in terms of $M$ and $C$ for the \AU relation.

The best relation is found for the combined feature derived from the fourth principal component found through PCA performed in the feature set $\mathbf{F} = \{M, C, \tilde\rho, \omega_f\}$. . The best fit for the relation between $\omega_f$ and this combined feature is shown in Figure~\ref{fig:multi_fit_fmode}. The best fit shows a quadratic \AU relation for the $f$-mode frequency of the form
\begin{equation}
\omega_f = -0.00033 \hat F^2 + \num{0.012682} \hat F - \num{0.02348942}
\label{eq:relation_fmode}
\end{equation}
with
\begin{equation}
\hat F = \num{2.9800476501500004} \frac{M}{M_\odot} + \num{10.230908}\frac{\tilde\rho}{0.04} - \num{8.39804} \frac{C}{0.2}\,.
\label{eq:combined_fmode}
\end{equation}
This relation achieves an average relative error of $\num{0.015469332922977337}$. When compared to the old relation shown in Figure~\ref{fig:bivariate3_0}, we can clearly observe an improved universality, which is also reflected in the average relative error that is reduced by half. We thus achieve a significant improvement over the existing relation by using our multivariate approach. 

% Figure environment removed

\subsection{Improved Astroseismological Relations for the f-mode Frequency}\label{sec:multivariate_example3}
We next consider another variation on the astroseismological relation we inspected above. This time, instead of introducing mass and compactness as independent variables, we instead only introduce the product $C \tilde\rho$ of compactness and average density as a new independent variable.
Our goal now is therefore to find a \AU relation for $\omega_f$ using the average density $\tilde\rho$ and $C \tilde\rho$. 

In this case, the best relation is found for the combined feature derived from the third principal component found through the PCA performed in the feature set $\mathbf{F} = \{\tilde\rho, C \tilde\rho, \omega_f\}$. 
The best fit for the relation between $\omega_f$ and this combined feature is shown in Figure~\ref{fig:multi_fit_zmode}. The best fit shows a quadratic \AU relation for the $f$-mode frequency of the form
\begin{equation}
\omega_f = 0.0002 \hat F^2 + \num{0.00647675} \hat F + \num{0.00276012}
\label{eq:multivariate_relation_redshift}
\end{equation}
with
\begin{equation}
\hat F = \num{6.9109313199999995}\frac{\tilde\rho}{0.04} - \num{1.71649574} \frac{C \tilde\rho}{0.01}\,. %\num{1.71649574} \frac{z \tilde\rho}{0.02}
\label{eq:combined_feature_imp_zmode}
\end{equation}

When compared to the relation shown in the section above (cf. Figure~\ref{fig:multi_fit_fmode}) we observe an improved universality: the previous relation has an average relative error of $\num{0.015469332922977337}$, whereas the relation with the new combined feature achieves an error of $\num{0.009738598872994107}$. 

Considering that the original relation put forward by Andersson and Kokkotas~\cite{PhysRevLett.77.4134,Andersson98} was inspired by Newtonian gravity, the additional factor in $C \tilde\rho$ could be considered a first order correction to account for general relativity, since
\begin{equation}
\hat F = \num{172.773283} \tilde\rho - \num{171.649574}C \tilde\rho \approx 172 \tilde\rho \left(1-C\right)\,.
\end{equation}
Essentially, this new relation is a stepping-stone between the relation by Andersson and Kokkotas~\cite{PhysRevLett.77.4134,Andersson98}, and other general relativistic \AU relations, such as the one between the $f$-mode frequency $\omega_f$ and the compactness $C$ put forward by Tsui and Leung~\cite{PhysRevLett.95.151101}.

% Figure environment removed


\subsection{Discussion of Results}
As we have demonstrated above, we can utilize the principal components obtained from PCA to construct multivariate \AU relations for neutron stars. Since the relations we construct are, for now, first-order relations, this approach is also suited for finding linear order corrections to suspected \AU relations, allowing an improvement of the accuracy of the \AU relations.

Despite these positive results, our approach here has only been descriptive: while we provide a methodology that can yield multivariate \AU relations, the formal reasons for why this approach works is still not fully clear. Gaining further understanding of the mathematical underpinnings of this approach can allow us to further improve its output, but also better understand its limits. 

For instance, in Appendices~\ref{app:multivariate_special} and~\ref{app:multivariate_counter}, we show some cases where our approach will not yield \AU relations. Sometimes this is caused by the data used, as, ultimately, not all feature combinations will be amenable to \AU relations. Furthermore, specific properties of the used data, such as the existence of strong collinearities with the target feature, can also hinder our approach from producing \AU relations. We currently can only provide intuitive reasons for why our approach does not perform well in such situations, and we hope to obtain a more rigorous understanding through future work. 