\section{Behavior of WKB Approximate Solutions}
\label{sec:results}

Having developed the WKB approximate solutions, as well as methods by which to find the wavenumber, it is now possible to observe the behavior of the modeled solutions. WKB solutions are compared to numerical results computed using the finite difference method of Neely \cite{neely}. Physical quantities used here are from the 2-D box model of Steele and Taber \cite{steele_lagrange}. These, along with parameters used in wavenumber-finding algorithms, are provided in Table \ref{tab:params}. Mass, resistance and stiffness terms contribute to the impedance according to 
$$Z(x) = j\omega m + r + \frac{s(x)}{j\omega},$$
where $x$ has units mm.

\begin{table}[h!]
    \centering
    \begin{tabular}{|c|c|c|}
        \hline Parameter & Symbol & Value \\
        \hline\hline Mass & $m$ & $1.5 \times 10^{-3}$ g/mm$^2$  \\
        \hline Resistance & $r$ & $2\times10^{-6}$ Ns/mm$^3$ \\
        \hline Stiffness & $s(x)$ & $10 e^{-0.2 x}$ N/mm$^3$ \\
        \hline Scala Height & $h$ & 1 mm \\
        \hline Cochlea Length & $L$ & 35 mm \\
        \hline Fluid Density & $\rho$ & $10^{-3}$ g/mm$^3$ \\ 
        \hline Threshold on $k$ finite difference & $T$ & 0.6 mm$^{-1}$\\
        \hline Iterations of Newton's Method (Algs \ref{alg::no_discon}, \ref{alg::with_discon}) & $M$ &  20 \\ 
        \hline Iterations of Contractive Mappings (Alg \ref{alg::stable}) & $M$ &  20 \\ 
        \hline Points in $x$-space & $N$ & 1024 \\
        \hline Points in $z$-space (finite difference method)& N/A & 16 \\ \hline
    \end{tabular}
    \caption{Parameters used in all simulations. Physical parameters are from Steele and Taber \cite{steele_lagrange}.}
    \label{tab:params}
\end{table}


\subsection{WKB Solutions for the 1-D Box Model}

In Sec \ref{sec:1d}, I derived WKB solutions to the 1-D BVP up to the zeroth (Eqn \ref{WKB1Dpressure0}) and first (Eqn \ref{WKB1Dpressure1}) orders. In Fig \ref{fig:1d}, these solutions are shown in response to a 2 kHz stimulus frequency, and compared to numerical results.

% Figure environment removed

It can be seen that a zeroth-order approximation overestimates the magnitude of the response near the peak, and exhibits more phase accumulation than the numerical solution. On the other hand, the first-order WKB approximation matches the numerical solution well across space in both phase and magnitude. The two orders of solution differ only by a factor of $\sqrt{k}$, which is real-valued for small $x$ explaining the similarity in phase.

The remarkable similarity between the first-order WKB solution and the numerical solution indicates that satisfaction of the WKB assumption (that $k$ varies slowly relative to the model parameters) is valid throughout the 1-D box model. 

\subsection{Long-Wave and Short-Wave Solutions}

 To contextualize findings for the 2-D WKB solutions, it is useful to observe the performance of the long- and short-wave approximate solutions to the 2-D box model. These solutions are valid for regions of small $k$ and large $k$ respectively, but as $k$ is complex-valued and varies non-monotonically across space/frequency (see Fig \ref{fig:loci}) it is not immediately clear in which regions these approximations will best match numerical solutions.

% Figure environment removed

Fig \ref{fig:lwsw} shows the long-wave and short-wave solutions to the 2-D box model alongside a numerical solution. The long-wave response matches the numerical solution well at more basal positions, where the wavenumber is small and real (Fig \ref{fig:loci}), but poorly matches the numerical solutions near or above the peak.

The short-wave solution matches the numerical solution well only in a small spatial range near the peak region. Neither approximation matches the numerical solution past the peak where the slope of the numerical solution becomes smaller. In the context of the root loci (Fig \ref{fig:loci}), neither approximation should be expected to hold well at higher frequencies when the roots approach the negative imaginary axis as the asymptotic forms of the hyperbolic tangent used in their derivation are only valid for real argument. This region is termed the \textit{cutoff region} \cite{watts}.

\subsection{Performance of Wavenumber-Finding Algorithms}

Before comparing 2-D WKB approximations to numerical results, it is first important to assess the methods for determining the wavenumber $k$ in the 2-D case. This is performed by observing velocity responses at the OCC derived from the 2-D WKB approximation of Eqn \ref{duifhuisKING}, using three methods for finding the wavenumber: 1) Alg \ref{alg::no_discon}, an $x$-stepping algorithm that does not account for the discontinuity, amounting to following a root locus as in Fig \ref{fig:loci}, 2) Alg \ref{alg::with_discon}, an $x$-stepping algorithm that does account for discontinuity, via thresholding the finite difference as described above, and 3) Alg \ref{alg::stable}, the stable point method. 

%\par{In all cases the parameters of Steele and Taber are used \cite{steele_lagrange}, along with a 5.5 kHz stimulus. In the $x$-stepping methods, Newton-Raphson is applied for 20 iterations at each location, and resonance is accounted for in method (2) by using a threshold of $T=0.8$. In the stable point method, the iterative algorithm is applied 20 times.}

Fig \ref{fig:disc} contrasts the $x$-stepping methods depending on whether discontinuity is accounted for. The velocity responses show identical behavior up to a position slightly past the peak, where the finite difference in $k$ becomes sufficiently large so that a discontinuity is registered. After this point, the fall-off in velocity amplitude is slower than if the discontinuity were not considered. This slower falloff is seen in the cutoff region of numerical results, owing to the more reasonable choice for $k$ past the peak \cite{steele_lagrange,steele_rootfinding,viergever_Book,deBoer_Rootfinding}. Comparison to numerics is present in the following subsection.


% Figure environment removed


\par{The root loci in the upper-right panel show that for the discontinuous method, the traversal of the locus is halted as the root pattern begins to appear sparser (i.e. faster change in $k$). As described in Sec \ref{sec:implement}, the discontinuous algorithm then assumes a small negative imaginary root (transition shown by the dashed blue arrow), which yields less rapid falloff in the cutoff region than the larger negative imaginary component found by following the locus continuously.}

\par{The bottom-right panel serves to show that the two methods are both correctly converging to roots of $f$ at each given $x$. Using both methods, the value of $f(kh)$ at is less than $10^{-10}$ in magnitude at all $x$ -- this stresses the fact that not all roots lie on the same continuous locus.}

\par{Fig \ref{fig:sp} shows these same results, but now alongside the results obtained via the stable point method (Alg \ref{alg::stable}). These results show similar behavior in velocity magnitude to the continuous $x$-stepping solution, but the phase accumulates more cycles.}

% Figure environment removed


\par{Observation of the root locus and $f(kh)$ for the stable point algorithm reveals strange behavior in the cutoff region. While the stable point method's root locus follows that of the $x$-stepping method for a large range of $x$, it erratically jumps around the complex plane (including to the third quadrant) past the peak. This corresponds to a non-zero value of $f(kh)$ at these positions as well (see the bottom-right panel), showing that the algorithm has not correctly converged to a root of the function.}

\par{This is anecdotal justification of the validity of this method in the long-wave and short-wave regions, but not in the cutoff region -- a drawback of the stable point method. However, the stable point method has certain desirable advantages over longitudinally-stepping algorithms -- it is generally far faster, and does not require dense longitudinal spacing to converge in the long- and short-wave regions.}

\subsection{WKB Solutions for the 2-D Box Model}

In the previous subsection, it was shown that the the wavenumber-finding algorithm that accounts for discontinuities in $k$ both converges to roots across space and yields velocity responses that qualitatively resemble numerical results past the peak. This informs the choice to use this algorithm for comparison to numerical results.

In Sec \ref{sec:2d}, two WKB approximate solutions were presented -- the lower-order solution of Eqn \ref{WKB2Dpressure} and the higher-order solution of Eqn \ref{duifhuisKING}. Fig \ref{fig:2d_num} shows solutions according to both of these equations alongside numerical solutions to the 2-D BVP.

% Figure environment removed

Both approximate solutions resemble the numerical solutions across space, including at positions past the peak where the slope of the response rapidly changes. This contrasts with the approximate solutions derived from the two alternate root-finding methods, as seen in Fig \ref{fig:sp}.

The lower-order solution slightly overestimates tuning at the peak as compared to the higher-order solution, but the solutions are otherwise nearly identical. They differ most significantly from numerical solutions in their phase responses. While they are characteristically similar, both approximate solutions lead the numerical solutions by about 0.1 cycles in the apical range where the phase varies slowly.

It is important to note that these results are sensitive to the choice of threshold $T$ in Alg \ref{alg::with_discon}. A large threshold leads to a registration of a discontinuity at a more apical position. This means that the slope of both the magnitude and phase will change at a more apical location than in the numerical solution. Similarly, a lower value of $T$ will move the discontinuity further basal. For reasonable values of $T$, WKB solutions will be qualitatively similar to numerical solutions in magnitude and phase slope, but they may differ quantitatively due to the shift in the position at which the qualitative tradeoff occurs.

These results show that WKB approximations can match numerical solutions across the length of the cochlea in both 1-D and 2-D box models, save a very small interval between the short-wave and cutoff regions. While many WKB applications do not consider behavior past the peak, the failure of the approximation in this region is due not to the solutions but to the method used for finding the wavenumber. The failure of the stable point algorithm to converge in the cutoff region has been noted before, e.g. in App D of \cite{earhorn}. Still, due to the relative speed of this method's convergence, and its not needing a fine resolution for $x$, it is still an attractive method for many applications and has seen use in many modern works where performance in the cutoff reason has not been critical to model results \cite{earhorn,altoe_2022,Shera_Altoe_2023} 
