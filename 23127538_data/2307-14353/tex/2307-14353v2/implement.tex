\section{Implementation: Solving the Dispersion Relation}
\label{sec:implement}

In the previous sections I have described theoretical underpinnings for WKB solutions to 1-D and 2-D box and tapered box models. In this section, I discuss the challenges involved in implementation of the derived model equations in software. 

The 1-D model poses no such difficult, as the WKB pressure equation and dispersion relation are explicit and in terms of elementary functions, but the dispersion relation of Eqn \ref{dispersion} presents a challenge in the 2-D case. This relation is transcendental, and generally has infinitely many solutions for $k$ in the complex plane. To standardize the language, the solution for $k$ is reframed as a root-finding problem for the function
\begin{equation}
    f(z) = z\tanh{z} - C,
    \label{rootfind}
\end{equation}
where
\begin{equation}
    z=kh,\;\;\;C=-2\rho h j\omega Y_{OC}.
    \label{Cdef}
\end{equation}

At each position and frequency, solutions will exist for multiple values for $k$, but we will select only the most significant of such modes\footnote{This ``one-mode assumption" is worthy of some scrutiny, and the work of Watts \cite{watts,Watts_2000} and Elliott \cite{Elliott_Ni_Mace_Lineton_2013} have covered multi-mode solutions and implications thereof.}. As the velocity is loosely of the form $e^{-jkx}$, the solution should possess a positive real part to correspond to a forward traveling wave. As for the imaginary part, this leads to either damping or amplification of the solution in $x$. Exponential growth is aphysical, meaning that the imaginary part must be negative and the solution for $k$ must lie in the $4^{\text{th}}$ quadrant of the complex plane.

Moreover, of the solutions in this quadrant, the one with the smallest (in magnitude) imaginary part is desired. A more negative imaginary part would lead to more severe exponential damping, so the most significant solution has the least such damping.

In this section, I discuss the properties of the roots of $f$, and the challenges that come in solving for physically reasonable roots. I then describe in detail three algorithms for finding $k$. The performance of these algorithms is discussed in Sec \ref{sec:results}.

\subsection{Root Loci}

Because the function $f$ is continuous, a small variation of $C$ will (mostly) yield a small variation of the root position. Each continuous path traced by the roots with increasing $x$ is called a \textit{root locus}. With realistic impedance functions, the root loci form arcs in the fourth quadrant, traveling from the positive real axis to negative imaginary axis with increasing $x$ \cite{deBoer_Rootfinding}. Fig \ref{fig:loci} shows four such root loci under one set of parameters, where each color corresponds to a different stimulus frequency and each circle is a root at a different $x$ position ($x$-resolution is $7\mu m$). As $x$ increases, the locus diverges from the real line and traverses clockwise towards the negative imaginary axis. At higher frequencies, the arc is broader and arrives at a larger negative imaginary value.

% Figure environment removed

The WKB assumption is that this variation of $k$ in $x$ is slow, so that tracing the continuous arc through the plane (possible with a fine enough resolution in $x$) would give the root of interest. However, with physically realistic parameters, one runs in to multiple issues just past the peak region. In particular, near the location where stiffness and mass cancel, the admittance factor of $C$ varies rapidly. Here, the WKB assumption breaks down, and a tracing of the root locus shows a rapid traversal of the arc near these positions. This can be seen in Fig \ref{fig:loci} where the roots appear sparse along the broad arc of the locus, indicating a much faster change in $k$ than at the denser regions near the real and imaginary axes. In this region, insufficient resolution in $x$ could not capture the continuous but rapid arc of the root locus, and may instead yield convergence to a root in a different locus. This issue can be resolved either by uniformly refining resolution, or refining resolution
close to the resonant position \cite{viergever_Book}. 

\subsection{Continuous Longitudinally-Stepping Algorithm}

The goal is to begin by tracing a single root locus for $f$ through the complex plane. Due to the number of possible roots at a given $x$, canonical root-finding methods can cause trouble. Such methods require an intelligently chosen starting point so as not to converge to the wrong root, or even a saddle point.

\par{In this section, I will describe a class of algorithms for root-finding that step across the longitudinal axis, at each point making an estimate for $k$ informed by the estimate from the previous step \cite{deBoer_Rootfinding,viergever_Book}. Here, $x$ values are quantized. I will write the estimate for $k$ at position $x_n$ as $\hat{k}_n$. As the function $f$ is itself $x$-dependent, I will write $f(z;x_n)$ to refer to $f$ at each position.}

\par{Starting at the very base, we are likely to be in the long-wave region. This motivates the initial approximation of $\hat{k}_1 = k_{lw}(x_1)$. This can be used as the initial value in a standard root-finding algorithm such as Newton-Raphson or the Muller method, which are likely to converge to the correct root.}

\par{Stepping further along in $x$, the long-wave approximation becomes poor. This indicates that we ought not use this initial value forever. As in Fig \ref{fig:loci}, wavenumbers within a single locus at subsequent $x$ locations are likely close to one another, $k_n \approx k_{n+1}$. The intuitive estimate is to use the solution at $x_n$, $\hat{k}_n$, as the starting point for the root-finding method at $x_{n+1}$ to find $\hat{k}_{n+1}$.}

\par{Pseudocode for this algorithm using the Newton-Raphson method in $z$ to compute the wavenumber is presented in Alg \ref{alg::no_discon}. The Newton-Raphson method requires the derivative of $f$, given by
\begin{equation}
    f'(z) = \tanh{z}+z\text{sech}^2{z}.
\end{equation}
Recall that $z = kh$.}

\begin{algorithm}
\begin{algorithmic}
\State $\hat{k}_{ic} \gets k_{lw}(x_1)$ \Comment{Initialize using long-wave $k$}
\For{$n = 1 \rightarrow N$} \Comment{$N$ is the number of steps in $x$ space}
    \State{$z \gets h\hat{k}_{ic}$}
    \For{$m = 1 \rightarrow M$} \Comment {$M$ is the number of Newton-Raphson iterations}
        \State{$z \gets z - \frac{f(z;x_n)}{f'(z;x_n)}$} 
    \EndFor
    \State{$\hat{k}_{n} \gets z/h$}
    \State{$\hat{k}_{ic} \gets k_n$} \Comment{Initial value for next step is current guess for $k$}
\EndFor

\end{algorithmic}
\caption{Continuous longitudinally-stepping root-finding algorithm to determine the wave number at $N$ different $x$ positions, using the Newton-Raphson method.}
\label{alg::no_discon}
\end{algorithm}

This works so long as $k$ is slowly varying, which is precisely the WKB assumption. However, there is a significant problem that occurs near the resonant point where stiffness and mass cancel, creating a rapid change in $k$ (see the broad points of the arcs in Fig \ref{fig:loci})\footnote{\textbf{Note on terminology:} certain modelers have described this point as the ``critical layer," owing its name to a more general theory of critical layer absorption \cite{lighthill_long,lighthill_short}.}. If we were to ignore this feature, we would simply follow the continuous root locus as in Fig \ref{fig:loci}, tending towards solutions for $k$ with large negative imaginary parts. This leads to falloff in the magnitude response that is far more rapid than what is seen in basilar membrane displacement data \cite{Robles_Ruggero_2001,steele_lagrange}.

It is a misconception that this falloff is a result of the WKB method itself failing past the peak. Instead, it is because the roots along the continuous locus do not correspond to dominant modes once their imaginary parts become sufficiently negative. Methods have been developed to counteract this problem by considering a continuous switch-off between dominance of two modes \cite{watts,Watts_2000,Elliott_Ni_Mace_Lineton_2013}, or by discretely swapping the root locus being followed near the resonant position \cite{viergever_Book,steele_rootfinding}. An example of the latter type is discussed in the following subsection.

\subsection{Discontinuous Longitudinally-Stepping Algorithm}

To account for the rapidly increasing imaginary part of the continuously traced roots, one can introduce a discontinuity into the curve of traced roots. To begin, the term $C$ in the root-finding problem (see Eqn \ref{Cdef}) is approximated near the resonant position. Where the stiffness and mass cancel, the admittance is $Y_{OC}\approx 1/R_d$, a real resistance (where the $d$ subscript denotes evaluation near the resonant position), i.e. $C$ is purely negative imaginary. I define a new term $\gamma$:
\begin{equation}
    \gamma = \frac{R_d}{2\rho h \omega_d}\in\mathbb{R}.
\end{equation}
The new problem to solve becomes
\begin{equation}
    z\tanh{z} = \frac{-j}{\gamma}.
    \label{gammaform}
\end{equation}

The solution to this transcendental equation can be approximated using the assumption that $z$ is small in magnitude and lives close to the imaginary axis. This ensures that the chosen value of $k$ will correspond to the dominant mode, falling off less rapidly than what would be found by tracing the continuous root locus. A derivation of this solution, relying on Taylor expansions, is given in brief by Viergever \cite{viergever_Book}\footnote{A more intricate treatment can be found at https://github.com/brian-lance/wkb-derivations}. It yields
\begin{equation}
    k_d \approx \frac{\pi}{2h}\gamma - j \frac{\pi}{2h}(1-\gamma^2),\;\;\; \gamma = \frac{R_d}{2 \rho h \omega_d}.
    \label{kd}
\end{equation}

The discontinuous $x$-stepping algorithm traces the continuous root locus up to some $x_d$ at which it is determined that the WKB assumption is being violated. This can be determined before simulation \cite{viergever_Book} or on the fly by observing the rate of change of the wavenumber (in discrete space, the finite difference) at each step. When the WKB approximation holds, this value should be very small. Picking some threshold $T>0$, the $x$-stepping method is paused once $|\hat{k}_n - \hat{k}_{n-1}|>T$.

After this point, $k_d$ of Eqn \ref{kd} is used as an initial step in the root-finding algorithm. If still $|\hat{k}_n - \hat{k}_{n-1}|>T$, the WKB approximation is violated in this region and the pressure at this position is set equal to the last computed velocity ($p_n = p_{n-1}$). At each subsequent step, it is determined whether $\hat{k}$ satisfies this threshold -- it will eventually do so, at which point we continue the locus-tracing process along this second locus. Pseudocode for this algorithm is presented in Alg \ref{alg::with_discon}.

\begin{algorithm}
\begin{algorithmic}
\State{$\hat{k}_{ic} \gets k_{lw}(x_1)$} \Comment{Initialize using long-wave $k$}

\For{$n = 1 \rightarrow N$} \Comment{$N$ is the number of steps in $x$ space}
    \State{$z \gets h \hat{k}_{ic}$}
    \For{$m = 1 \rightarrow M$} \Comment {$M$ is the number of Newton-Raphson iterations}
        \State{$z \gets z - \frac{f(z)}{f'(z)}$} 
    \EndFor
    \If{$|z/h - k_{ic}| < T$} \Comment{Check for validity of WKB condition}
        \State{$\hat{k}_{n} \gets z/h$}
        \State{$\hat{k}_{ic} \gets k_n$} \Comment{Initial value for next step is current guess for $k$}
    \Else{}
        \State{$\hat{k}_{n} \gets$ NaN} \Comment{Pressure and velocity should not be computed here}
        \State{$\hat{k}_{ic} \gets k_d$} \Comment{Guess for $k$ after the discontinuity}
    \EndIf
\EndFor

\end{algorithmic}
\caption{Discontinuous longitudinally-stepping root-finding algorithm to determine the wave number at $N$ different $x$ positions, accounting for the discontinuity in the wavenumber. }
\label{alg::with_discon}
\end{algorithm}

\subsection{The Stable Point Algorithm}

One alternative to the longitudinally-stepping class of algorithms is a stable point algorithm, in which two distinct relationships between $k$ and $\alpha$ (the pressure-focusing factor) are used (e.g. App D of \cite{earhorn}). The first such relationship is that given in Eqn \ref{focus}, which gives $\alpha$ in terms of $k$. The second, giving $k$ in terms of $\alpha$, can be quickly derived from relating Eqn \ref{webster2Dk} and Eqn \ref{focus}: 
\begin{equation}
    k^2 = \frac{-2j\omega \rho \alpha}{h Z_{OC}(x)}.
    \label{stable}
\end{equation}
Any valid $k$ value must satisfy both equations.

\par{Stable point methods are based on the contractive mapping theorem, which states that repeated application of a contractive function $g$ will converge to a stable point of said function, i.e. a point where $g(x) = x$. Mathematical details are omitted here for the sake of brevity.}

\par{The stable point method for this problem works by starting with the long-wave approximation at every frequency-location pair, $\hat{k} = k_{lw}$. Then, $\hat{k}$ is plugged in to Eqn \ref{focus} to find an the approximate pressure focusing factor $\hat{\alpha}$, and then  $\hat{\alpha}$ is plugged in to Eqn \ref{stable} to find a new wavenumber approximation $\hat{k}$. This is repeated for some number of iterations. Pseudocode for this algorithm is shown in Alg \ref{alg::stable}. 
}

\begin{algorithm}
\begin{algorithmic}
\State{$\hat{k} \gets k_{lw}$} \Comment{Here $\hat{k}$ is a vector with an index for each position}

\For{$m = 1 \rightarrow M$} \Comment {$M$ is the number of stable point iterations}
    \State{$\hat{\alpha} \gets \frac{h \hat{k}}{\tanh{kh}}$}\Comment{Pressure focusing vector update, Eqn \ref{focus}}
    \State{$\hat{k} \gets \sqrt{\frac{2j\omega \rho \hat{\alpha}}{h Z_{OC}}}$} \Comment{Wavenumber update, Eqn \ref{stable}}
    \If{$\mathcal{I}[\hat{k}] < 0 $}
    \State{$\hat{k} \gets -\hat{k}$}\Comment{Ensure the root is for a forward-traveling wave ($\mathcal{I}$ gives the imaginary part)}
    \EndIf
\EndFor

\end{algorithmic}
\caption{Stable point algorithm that updates a vector of $\hat{k}$ approximations by iteratively applying two relationships.}
\label{alg::stable}
\end{algorithm}

\par{This method works under the assumption that it converges to the correct value of $k$, which depends on the properties of the mappings between $\alpha$ and $k$. If the mappings are not (at least locally) contractive, then no convergence is guaranteed. On the other hand, if there are multiple stable points, certain choices of initial conditions may lead to convergence to an undesired $k$. Performance of these three algorithms will be discussed in the following section.}
