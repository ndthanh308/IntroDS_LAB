\section{Cochlear Model}
\label{sec:3d}

The WKB approximation can be applied to any cochlear model described by linear differential equations. In this tutorial, the focus is one popular class of models -- box and tapered box models -- with geometry as shown in Fig \ref{fig:box}. The model is derived by considering the cochlea as uncoiled and containing only two scalae -- scala vestibuli (SV) and scala tympani (ST) -- as Reissner's membrane is assumed to be mechanically invisible. The scalae are separated by an infinitesimally thin plate with the flexible OCC being the only portion of this plate capable of movement.

The cochlea's longitudinal axis ($x$) points towards the apex, terminating at the stapes at $x=0$ and the helicotrema at $x=L$. The transverse axis ($z$) points towards SV with the OCC lying at $z=0$. The cross-sectional area of SV and ST are equal to one another\footnote{This assumption can be relaxed and assymetric scalae can be assumed by finding some ``effective area." Results are largely unchanged.} and vary along the longitudinal axis as $S(x)$. The OCC width varies along the longitudinal axis as $b(x)$. This model simplifies to the common box model when the scala walls are not curved and $S(x)$ and $b(x)$ are constant.

% Figure environment removed

The model can be flattened to 2-D, as is represented geometrically in Fig \ref{fig:box}. This flattening amounts to representation of each quantity as its average over the radial dimension. It appears as a tapered box with height $h(x) = S(x)/b(x)$. Further flattening of the model to 1-D involves representation of all quantities as being only dependent on $x$. This amounts to averaging quantities over transverse space. 


\subsection{Boundary Conditions and Assumptions}
\label{sec:conditions}

The boundary conditions operate on the following assumptions -- 1) fluid does not flow in the normal direction towards or out of the scalae at the walls $z=\pm h$, and at the helicotrema $x=L$, 2) the average pressure at  $x=0$ is some known constant $P_{OW}$, and 3) the OCC is mechanically described by a longitudinally-varying point impedance $Z_{OC}(x)$ (or reciprocally as a point admittance $Y_{OC} = 1/Z_{OC}$). This quantity is complex and frequency-dependent.

It should be noted that the modeled pressure and velocity will vary sinusoidally. Assuming linearity of the model\footnote{The focus of this tutorial is on linear models, but nonlinearity can be implemented in a number of ways (e.g. quasilinear method, formulation of a nonhomogeneous cochlear model, etc.) \cite{Kanis_deBoer_1993,Kanis_deBoer_1996,Talmadge_Tubis_Long_Piskorski_1998,altoe_2022}.}, inputs at a given radian frequency $\omega$ will yield model responses at the same frequency. That is, all quantities will be of the form $C(x,z,\omega)e^{j\omega  t}$.  The time-dependence is identical across all quantities and so it will generally be left implicit. 

The fluid pressure is denoted $P(x,z)$, the longitudinal fluid velocity is denoted $\dot{u}(x,z)$ and the transverse fluid velocity is denoted $\dot{w}(x,z)$. The impedance describes the relationship between transmembrane pressure and transverse displacement at $z=0$. Due to the symmetry of the model, transmembrane pressure $p=2P$. 

In the 2-D tapered box model, the boundary conditions can be written as
\begin{align}
    &\frac{1}{h(0)}\int_{0}^{h(0)} P(0,z)\;dz = P_{OW}, \label{owBV}\;\;\\
    &\frac{\partial P}{\partial x}(L,z) = 0, \label{heliBV}\\
    &\frac{\partial P}{\partial z}(x,h(x)) = 0, \label{wallBV}\\
    &P(x,0)  = \frac{Z_{OC}(x)}{2} \dot{w}(x,0), \label{bvImpedance}
\end{align}
These boundary conditions are equally valid in the 1-D model simply by ignoring dependence on $z$.

An assumption is also made regarding the character of the traveling wave solutions. With the input pressure appearing at the stapes, in the absence of internal reflection, the traveling wave will primarily travel towards the apex where a reflected, basal-traveling wave will be generated at the helicotrema.

For the most part, I will assume that the basal-traveling reflected wave is far smaller than the forward traveling wave. In some models, this is achieved by letting $L\rightarrow \infty$, in which case the WKB solutions will be identical to those arrived at in this tutorial. Basal-traveling waves will be considered in the context of WKB basis functions (Sec \ref{sec:proj}), and this assumption will be removed.

\subsection{Model Equations}

With the geometry described, the model equations can now be developed. The fluid in the scalae is modeled as \textit{incompressible}, \textit{irrotational}, \textit{linear} and \textit{inviscid}. The fluid velocity vector  $\mathbf{v} = (\dot{u}\; \dot{w})^T$ in a volume satisfies the continuity equation, given by
\begin{equation}
    \frac{\partial\rho}{\partial t} + \nabla\cdot(\rho\mathbf{v})= 0,
\end{equation}
where $\rho$ is the fluid density. This represents that within a differential volume, the change in fluid mass in the region is accompanied by an equal and opposite divergence of that fluid into/out of the region.

In an incompressible fluid, the mass of the fluid (and thereby $\rho$) in any region is constant, simplifying the equation to
\begin{equation}
\nabla\cdot\mathbf{v} = 0.
\label{cont}
\end{equation}
An irrotational field is also a conservative field, so the velocity field can be written as the gradient of some scalar field $\phi$. This \textit{velocity potential} thereby satisfies

\begin{equation}
    \nabla\phi = \mathbf{v}.
\end{equation}.

Taking the divergence of both sides and applying Eqn \ref{cont} yields the Laplace equation in velocity potential:
\begin{equation}
    \nabla^2\phi = 0,
\end{equation}
The Navier-Stokes equation in an inviscid, incompressible, linear and irrotational fluid is
\begin{equation}
\rho\frac{\partial \mathbf{v}}{\partial t} + \nabla P =\rho\frac{\partial \nabla \phi}{\partial t} + \nabla P=0.
\label{navier}
\end{equation}
This gives 
\begin{equation}
    P = -\rho \dot{\phi},
    \label{ptophi}
\end{equation}
where the overhead dot indicates a partial derivative in time. Taking the Laplacian of both sides and recalling that $\phi$ satisfies the Laplace equation, we arrive at a Laplace equation in $P$:
\begin{equation}
    \nabla^2 P=0.
\end{equation}
The Laplace equation then also holds for transmembrane pressure $p=2P$. 

Another model equation can be derived for the \textit{average} pressure in a cross-section, i.e. for the 1-D model. Over a small longitudinal cross-section from $x$ to $x + \delta$, transverse fluid displacement occurs at a rate of approximately $b(x) \delta \dot{w}(x)$, as transverse fluid motion is generated only by the motion of the OCC and $b(x) \delta$ is the approximate area of the OCC in this range. 

Longitudinally, fluid enters the region at rate $S(x) \dot{u}(x)$ and exits at rate $S(x+\delta) \dot{u}(x+\delta)$. Recognizing that transverse and longitudinal fluid displacement must be equal and opposite to satisfy mass conservation and letting $\delta \rightarrow 0$ gives
\begin{equation}
    \frac{\partial}{\partial x} [S \dot{u}] = -b\dot{w}.
    \label{consmassHorn}
\end{equation}

This will be written in terms of pressure by simplifying the Navier-Stokes equation (Eqn \ref{navier}) to 1-D,  multiplying it by the cross-sectional area and differentiating in $x$:
\begin{equation}
    \frac{\partial}{\partial x} \bigg[ S\frac{\partial P}{\partial x}\bigg] + \rho \frac{\partial}{\partial x} \bigg[S \frac{\partial \dot{u}}{\partial t}\bigg] =0,
\end{equation}
where I have used the fact that $\dot{u} = \partial \phi/\partial x$.

Replacing the time derivative by product with $j\omega$ and applying Eqn \ref{consmassHorn} gives
\begin{equation}
    \frac{\partial}{\partial x} \bigg[S\frac{\partial P}{\partial x}\bigg] - j\omega\rho b \dot{w} =0.
\end{equation}
To write this entirely in terms of transmembrane pressure $p$, I can use $p=2P$ and the 1-D model's point-impedance boundary condition $p = Z_{OC} \dot{w}$. Doing so and dividing by $S$ gives the Webster horn equation for the 1-D model:

\begin{align}
        \frac{1}{S}\frac{\partial}{\partial x} \bigg[S\frac{\partial p}{\partial x}\bigg] + k^2 p = 0, \label{webster1D}\\
        k^2(x) = \frac{-2 j\omega \rho}{Z_{OC}(x) h(x)}.
        \label{webster1Dk}
\end{align}

This equation can be readily modified to apply to the 2-D model as well. Replacing the 1-D model's pressure $P$ with the 2-D model's pressure averaged across the transverse dimension, $\bar{P}$ (or $\bar{p} = 2\bar{P}$), the derivation holds identically until the final step. In the 2-D model, the boundary condition is that transverse OCC velocity is related to the pressure at $z=0$, not the average pressure. Defining the \textit{pressure-focusing  factor} $\alpha(x) = p(x,0)/\bar{p}$, the 2-D model Webster horn equation is

\begin{align}
        \frac{1}{S}\frac{\partial}{\partial x} \bigg[S\frac{\partial \bar{p}}{\partial x}\bigg] + k_{2D}^2 \bar{p} = 0, \label{webster2D}\\
        k_{2D}^2(x) = \frac{-2 j\omega \rho \alpha(x)}{Z_{OC}(x) h(x)} .
        \label{webster2Dk}
\end{align}

In the box model where $S$ is constant, Eqns \ref{webster1D} and \ref{webster2D} degenerate to wave equations with variable wavenumbers, and Eqn \ref{webster1Dk} is a dispersion relation.