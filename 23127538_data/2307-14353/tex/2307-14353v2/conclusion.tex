\section{Conclusions}
\label{sec:conclusion}


The WKB approximation provides compact, closed-form analytic approximate solutions for 1-D and 2-D cochlear macromechanics models that match numerical solutions well within the entire region of response (all $x$ and $\omega$) save a small region near the resonant position/frequency. These solutions are easily implemented and interpreted, and allow qualitative and quantitative insight into the manner by which physical parameters (impedance variations, scala area, etc.) alter the apical-travelling wave. The solutions also allow (through the method of WKB wave-space projection) for further interpretation of modeled responses as a superposition of apical- and basal-traveling waves, allowing for quantification of intracochlear reflections and otoacoustic emissions.

The forms of the WKB approximate solution for cochlear box models were developed decades ago, so one may reasonably ask: what is the importance of WKB approximate solutions in contemporary times? In fact, many important insights have been gleamed from WKB solutions in recent literature. Below are just a few such contributions from the past three years.

The inclusion of spatially-varying scala dimensions to a 2-D box model as in Eqn \ref{WKB2Dpressure} has been shown by Alto\`e and Shera to be important for the achievement of substantial OCC velocity at the apex in response to stimuli at the base \cite{earhorn}. They analyzed how tapering scala height could introduce an amplification factor that boosts responses at the apex relative to a uniform-height model, resolving losses that occur in the traveling wave as it makes its way to the apex.

Recent micromechanical findings made through optical coherence tomography have also inspired applications of the WKB solutions to the ostensibly macromechanical box model. In particular, motion at the outer hair cell-Deiters cell junction in the organ of Corti appears to move about 90$^\circ$ out of phase with basilar membrane motion within the same longitudinal cross-section. Implementing this as a modification to the impedance term, Alto\`e and Shera have used the WKB solutions as derived in this tutorial to model the impact of such a phenomenon \cite{altoe_2022}. They arrive at an alternative interpretation of cochlear amplification, in which power may be supplied to the fluid rather than directly to the basilar membrane.

Recent work by Sisto \textit{et al.}  used the WKB approximation in studying the level-dependence of the OCC admittance, assumed to arise from outer hair cell motility \cite{sisto_2021,sisto_2023}. Paying special attention to a) the pressure focusing phenomenon described above, and b) the viscosity at the OCC-fluid interface, they have found that substantially level-dependent admittance is not required to obtain the impressive dynamic range of the cochlea.

With much still to learn about the mechanics of the cochlea, the powerful analytic tool offered by the WKB approximation is among the strongest we have due to its interpretability, versatility and simplicity of computation. With the foundations discussed in this tutorial, derivations and implementation details can be modified to tackle contemporary questions as they continue to arise. 

\section*{Acknowledgments}
I would like to thank Dr. Elizabeth S. Olson for providing edits and comments for this tutorial.