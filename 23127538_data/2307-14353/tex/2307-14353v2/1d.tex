\section{WKB Solutions for The 1-D Model}
\label{sec:1d}

The Webster horn equation (Eqn \ref{webster1D}) was studied in the context of cochlear mechanics models as early as 1950 \cite{Peterson_Bogert_1950}. For general values of $S(x)$ and $k(x)$, the PDE may not have a simple closed-form analytic solution, but approximations such as constant $k$ and simple forms for $S$ can be used to yield explicit solutions, generally in terms of Bessel functions \cite{Peterson_Bogert_1950}. In the box model where this simplifies to a wave/transmission line equation with varying wavenumber, an explicit solution is guaranteed but only in terms of retarded Green's functions that are challenging to interpret \cite{Poisson_2004}.

The WKB approximation can be considered by putting the equation into the form of a standard linear ODE:
\begin{equation}
    \frac{\partial^2 p}{\partial x^2} + \frac{S'}{S} \frac{\partial p}{\partial x} + k^2 p = 0,
    \label{hornLinearDE}
\end{equation}
where $\cdot'$ denotes the spatial derivative. Comparing with Eqn \ref{WKBODE}, we have $\epsilon = 1$. 

Plugging in the WKB ansatz (Eqn \ref{WKB_withdelta}) with $\delta = \epsilon = 1$ gives
\begin{equation}
    \exp{\bigg(\sum_{m=0}^\infty C_m \bigg)}\;\bigg[ \sum_{m=0}^\infty C_m'' + \bigg(\sum_{m=0}^\infty C_m' \bigg)^2 + \frac{S'}{S} \sum_{m=0}^\infty C_m' + k^2\bigg] = 0.
\end{equation}
As the exponential term is never 0 and leads every term, I can divide through by it. I choose to keep only terms involving $C_0$ and $C_1$. By the asymptotic assumptions of the WKB approximation $C_1 \ll C_0$, and the terms should decrease as further derivatives are taken so that $C_1 '' \approx 0$ and $(C_1')^2 \approx 0$. This gives
\begin{equation}
    C_0'' + (C_0')^2 + 2C_0'C_1' + \frac{S'}{S} (C_0' + C_1') = -k^2.
\end{equation}

To simplify this approximation further, I make the assumption that the scala area varies slowly relative to its magnitude. That is,
\begin{equation}
    S'\ll S.
    \label{S_assumption}
\end{equation}
 In the box model, this is trivially true as $S$ is constant so that $S' = 0$. In reality, area variations can lead to breakdown of the WKB solution as will be discussed at the end of this section (also see \cite{catastrophe}).

Using Eqn \ref{S_assumption} along with the asymptotic assumptions, I am justified in approximating $S' C_1'/S \approx 0$. The first order WKB ODE is 
\begin{equation}
    C_0'' + (C_0')^2 + 2C_0'C_1' + \frac{S'}{S} C_0' = -k^2.
    \label{WKB1D_ODE1}
\end{equation}


To obtain the zeroth-order approximation, I make a further reduction based in the same asymptotic decay assumptions: I approximate $(C_0')^2 \approx 0$ and $C_0'C_1' \approx 0$. Under these approximations, Eqn \ref{WKB1D_ODE1} reduces in to simply
\begin{equation}
    (C_0')^2 = -k^2.
    \label{WKB1D_ODE0}
\end{equation}
With Eqns \ref{WKB1D_ODE0} and \ref{WKB1D_ODE1}, zeroth- and first-order WKB approximations for the 1-D model can be found.

\subsection{The Zeroth-Order Solution}

The ODE in Eqn \ref{WKB1D_ODE0} is quickly solved by taking the square root of both sides and integrating:
\begin{equation}
    C_0 = \pm j \int_0^x k(\xi)\;d\xi.
    \label{WKB1d_C0}
\end{equation}
The zeroth-order solution is then a superposition of these solutions:
\begin{equation}
    p_0(x) = Ae^{-j\int_0^x k(\xi)\;d\xi} + Be^{j\int_0^x k(\xi)\;d\xi},\;\;\; A,B\in\mathbb{C}.
\end{equation}
This is recognized as a superposition of two traveling waves, with the first summand traveling towards the apex and the second summand traveling towards the base. As stated in Sec \ref{sec:conditions}, the basal-traveling waves due to reflection at the helicotrema are modelled as being negligible, so $B=0$.

Application of the boundary condition at the oval window (Eqn \ref{owBV}) with $B=0$ gives $A=P_{OW}$. The zeroth order solution is then 

\begin{equation}
    p_0(x) = P_{OW}e^{-j\int_0^x k(\xi)\;d\xi}.
    \label{WKB1Dpressure0}
\end{equation}

\subsection{The First-Order Solution}

Plugging in the value for $C_0$ found in Eqn \ref{WKB1d_C0} to the first-order WKB approximate ODE of \ref{WKB1D_ODE1} gives
\begin{equation*}
    \pm j k' - k^2 \pm 2 k C_1' \pm \frac{S'}{S} jk  = -k^2.      
\end{equation*}
Solving for $C_1'$ gives
\begin{equation}
    C_1' = -\frac{k'}{2k} - \frac{S'}{2S}.      
\end{equation}
Integrating both sides give
\begin{equation}
    C_1 = -\frac{1}{2}\log{(Sk)}.       
\end{equation}

Plugging the found values of $C_0$ and $C_1$ into the WKB ansatz gives the first-order WKB approximate solution,
\begin{equation}
    p_1(x) = \frac{A}{\sqrt{S(x)k(x)}}e^{-j\int_0^x k(\xi)\;d\xi} + \frac{B}{\sqrt{S(x)k(x)}}e^{j\int_0^x k(\xi)\;d\xi},\;\;\; A,B\in \mathbb{C}. \label{forback}
\end{equation}
Once again, $B=0$ by the assumption that basal-traveling waves are negligible. $A$ is found by applying the boundary condition at the oval window, and writing $S(0) = S_0, k(0) = k_0$
\begin{equation}
    p_1(x) = P_{OW}\sqrt{\frac{S_0 k_0}{S(x)k(x)}}e^{-j\int_0^x k(\xi)\;d\xi}.
    \label{WKB1Dpressure1}
\end{equation}
In the box model, as $S(0) = S(x)$ for all $x$, the ratio inside the square root simplifies to $k_0/k(x)$. 

Eqns \ref{WKB1Dpressure0} and \ref{WKB1Dpressure1} are explicit formulae for pressure in terms of model parameters -- geometry and wavenumber, found through density, frequency and impedance (Eqn \ref{webster1Dk}). Having solved for pressure, velocity of the OCC in the 1-D model can be determined simply by dividing by the impedance.

\subsection{The Slowly-Varying Wavenumber Approximation}

The term \textit{WKB assumption} is often used to refer to the assumption that the wavenumber varies slowly in space relative to its own magnitude \cite{Zweig_compromise,Zweig_1991}. However, the derivation above for first- and second-order WKB approximations never explicitly made this assumption. Where is the relationship between these two ideas?

For the WKB method to be valid, the terms $C_n$ in the series must decrease monotonically. In particular $|C_1|\ll |C_0|$. Using the box model case for the sake of simplicity (no dependence on $S$), consider this relationship with the values of $C_0$ and $C_1$ from the 1-D model derived above, 
\begin{equation}
    \bigg|-\frac{1}{2}\ln{k}\bigg| \ll \bigg|j\int_0^x k(\xi)\;d\xi\bigg|.
\end{equation}

The first expression can also be written as an integral from $0$ to $x$, and pulling out the constant-modulus factors gives
\begin{equation}
    \frac{1}{2}\bigg|\int_0^x \frac{k'(\xi)}{k(\xi)}\;d\xi\bigg| \ll \bigg|\int_0^x k(\xi)\;d\xi\bigg|.
\end{equation}

This relationship is satisfied if $k$ satisfies 
\begin{equation}
    |k'| \ll |k|^2.
    \label{wkbassume}
\end{equation}
That is, the assumption of slow-varying $k$ implies that the WKB approximation is reasonable. In conjunction with the assumption of Eqn \ref{S_assumption}, the WKB assumption can be written as

\textit{\textbf{WKB Assumption:} The parameters of the model vary slowly relative to their magnitudes.}

Where these conditions are not met, the WKB approximation breaks down. It is important to keep this assumption in mind when observing modeled responses, as they are not likely to match either physical or numerical results in such regions.

Consideration of asymptotic behavior of the cochlea's traveling wave in light of Eqn \ref{wkbassume} is instructive. At positions far basal to the best place, the response is said to be in the \textit{long wave} region. Here, the wavenumber varies slowly in space, so the left-hand term in Eqn \ref{wkbassume} is very small.

At positions near the best place, the wavelength becomes smaller (wavenumber becomes larger) and changes more rapidly in space. This is known as the \textit{short wave} region. Here, the right-hand term in Eqn \ref{wkbassume} is very large. In balance, this assumption may be satisfied across a large portion of the frequency range.

On the other hand, reasonable smooth values for impedance will lead to quickly varying $k$ around the region where stiffness and mass terms of the impedance cancel (see the dispersion relation of Eqn \ref{webster1Dk}). In lossless cases, this leads to an infinite admittance, and with small resistance still yields rapidly varying $k$.

The wavenumber is also inversely proportional to $h(x) = S(x)/b(x)$ (see Eqn \ref{webster1Dk}), the flattened scala height. In the box model, this only trivially contributes to the rate of change of $k$ as it is a constant value, but in the physical cochlea the cross-sectional area tapers across its length. This can contribute significantly to $|k'|$, especially at the base where the scala area decreases approximately exponentially \cite{plassmann}. 

There must be some balancing act between $k$ and $S$ to maintain the validity of the WKB assumption. Thus, modeling the tapering of the scala area may improve satisfaction of the WKB assumption. Zweig and Shera discuss the implications of this balancing act between geometry and OCC impedance in detail, and refer to the failure of models to account for this as the ``cochlear catastrophe" \cite{catastrophe}. It is worth noting, however, that this catastrophe is only notable in the base in response to very low-frequency stimuli, so box models without tapering reasonably satisfy the WKB approximation across most of space and frequency.