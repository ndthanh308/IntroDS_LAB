\section{The WKB Traveling Wave Subspace}
\label{sec:proj}

Secs \ref{sec:1d} and \ref{sec:2d} described derivations of explicit equations for pressure in 1- or 2-D tapered box models via the WKB approximation. These formulae are valid where the model parameters do not change rapidly relative to their magnitudes, and where basal-traveling waves are negligible. However, WKB approximate solutions may not be easily derived for other cochlear models.

Numerical solutions, while more accurate to the dynamics, are generally challenging to interpret in comparison to WKB approximate solutions. This problem arises in, for example, the study of reflections in cochlear models -- with only a numerical solution, how does one separate components of the response that are due to incident waves from those due to reflected waves? This same breakdown may also be challenging in solving alternate cochlear models featuring, for example, nonlinearity. It is in this context that the theory of \textit{cochlear basis functions} was developed \cite{basis}.

A solution to a BVP describing the dynamics of the cochlea in a given model is an infinite-dimensional object, with the modeled OCC velocity living in the Hilbert space $\mathcal{H}$ of smooth functions mapping from $I=[0,L]$ (the interval of $\mathbb{R}$ along which the OCC is modeled to span) to $\mathbb{C}$. 

While the set of exact solutions is a subset of an infinite-dimensional space, they will likely have \textit{qualitatively} similar traveling wave forms for various perturbations to parameters and boundary conditions. Thus, they are likely to be well-approximated as living in a lower-dimensional subspace containing functions that resemble cochlear traveling waves. This motivates the concept of a \textit{traveling wave subspace}.

\subsection{Theory of Basis Wave Projection}

In a pioneering work, Shera and Zweig introduce several sets of basis functions that generate a traveling wave subspace, including the WKB basis functions \cite{basis}. I will develop the 1-D box model WKB basis, but the method is just as well extended to other approximate solutions.  This specification is both for the sake of simplicity and because 1-D box model WKB basis functions are the most commonly seen in literature \cite{basis,oae_moleti,oae_sisto,Talmadge_Tubis_Long_Piskorski_1998,Talmadge_Tubis_Long_Tong_2000,Shera_Tubis_Talmadge_2005,Sisto_Moleti_Shera_2007}.

By Eqn \ref{WKB1Dpressure1} with constant cross-sectional area, the apical-traveling wave is proportional to
\begin{equation}
    W_+ = \sqrt{\frac{1}{k}}e^{-j\int_0^x k(\xi)\;d\xi}.
\end{equation}
The basal-traveling wave has been ignored thus far in this tutorial. However, Eqn \ref{forback} implies that the WKB approximate solution for the basal-traveling wave would differ from $W_+$ only by the sign in the exponential. I define
\begin{equation}
    W_- = \sqrt{\frac{1}{k}}e^{j\int_0^x k(\xi)\;d\xi},\;\;\;x\in I.
\end{equation}

The set $\beta = \{W_+,W_-\}\subset \mathcal{H}$ is linearly independent, and its span, $\mathcal{W} = \text{\textbf{span}}(\beta)$ is a two-dimensional subspace of $\mathcal{H}$ which I will refer to as the \textit{WKB wave-space}. Any function $p \in \mathcal{W}$ can be written as
\begin{equation}
    p(x) = p_+(x) + p_-(x) = \psi_+ W_+(x) + \psi_- W_-(x),\;\;\; x\in I,
    \label{p_proj1}
\end{equation}
where the coefficients $\psi_\pm$ are complex-valued constants. 

One can form a system of two equations in two variables by differentiating either side:
\begin{equation}
    \frac{\partial p}{\partial x} = \psi_+ \frac{\partial W_+}{\partial x} + \psi_- \frac{\partial W_-(x)}{\partial x}.
    \label{p_proj2}
\end{equation}
Solution for the coefficients is neatly written in terms of the Wronskian of $\beta$, which is
\begin{equation}
\mathcal{D} = \textbf{det}\begin{pmatrix}
            W_+ & W_- \\ W'_+ & W'_-
        \end{pmatrix} = 2j.
        \label{wronskian}
\end{equation}
With the Wronskian, the projections onto each basis function can be written as
\begin{align}
    \begin{split}
        p_\pm = \mathcal{P}_\pm [p] &= \psi_\pm W_\pm \\
        &= \frac{\pm W_\pm}{\mathcal{D}} \bigg(\frac{\partial W_\mp}{\partial x} - W_\mp \frac{\partial }{\partial x}\bigg) p \\
        &= \frac{1}{2}\bigg(1 \pm \frac{k'}{2jk^2} \pm \frac{j}{k}\frac{\partial}{\partial x}\bigg) p,
    \end{split}
    \label{projection}
\end{align}
with $\mathcal{P}_\pm$ representing the operators projecting functions in $\mathcal{H}$ onto $W_\pm$\footnote{A more intricate treatment can be found at https://github.com/brian-lance/wkb-derivations}.

Of course, any exact solution to the BVP will not live in $\mathcal{W}$, so the values for $\psi_\pm$ found through this formula will not be constant. Thus, the projections are merely approximations that are best if the derivatives of $\psi_\pm$ are sufficiently small\footnote{A metric has been developed by Mathews and Walker and repeated by Shera and Zweig that quantifies the error through the size of these derivatives \cite{mathews_wkb,basis}. This is done by representing the basis functions as exact solutions to similar BVP $W'' + k^2(x)(1+\epsilon)W = 0$, with smaller $\epsilon$ representing better approximations to the actual BVP.}. 

Having these projection operators, one can formulate a numerical method for determining the apical- and basal-traveling components of any pressure waveform by implementing derivatives as finite differences. The same process can be followed for other basis functions of approximate solutions, such as the short-wave solutions,  long-wave solutions, or the WKB solutions in a tapered box model.

\subsection{Applications to Intracochlear Reflections}

One natural application of the projection described above is the study of reflections in the cochlea. The basal ($+$) and apical ($-$) reflection coefficients can be defined as 
\begin{equation}
    R_\pm(x) = \frac{p_\mp(x)}{p_\pm(x)}.
    \label{refdef}
\end{equation}
In a model of the cochlea where fluid pressure is driven at the stapes, a ``perfectly efficient" cochlea would reflect no energy in the basal direction (this is assumed in the derivations of Sec \ref{sec:1d} and Sec \ref{sec:2d}) and $R_+$ would be 0. In a cochlear model that exhibits some inefficiency, this will be a spatially varying complex-valued quantity. Some reasonable sources of such reflections include roughness in the OCC impedance or nonlinearity.

Conversely, one can consider how basal-traveling waves reflect towards the apex via $R_-$. With a passive cochlea driven at the stapes, this represents ``reflections of reflections." However, it is interesting to consider models where the cochlea is driven from a point along the length of the OCC ($x\neq 0$) \cite{basis,Talmadge_Tubis_Long_Piskorski_1998,Viergever_1986}. This could correspond to mechanical energy sources along the length of the OCC, present in the electromotile outer hair cells (OHCs), which are likely responsible for many forms of OAEs. 

Given that OAEs are measurable when the cochlea is driven at the stapes, there must be some significant portion of energy traveling towards the base without being entirely reflected. Some early modeling work on this topic predicted that the apical reflection coefficient is very large compared to the basal reflection coefficient ($R_+\gg R_-$), so that basal-traveling energy would be significantly reflected before arriving back at the stapes \cite{Viergever_1986,de_Boer_Kaernbach_König_Schillen_1986}. In this formulation, OAEs would have very low magnitudes. It was later argued by Shera and Zweig that the sizes of these quantities are highly dependent on the boundary conditions of the model \cite{basis} -- an important result to keep in mind for the modeling of OAEs.

\subsection{Nonhomogeneous Models and WKB Solutions as a Fundamental Set}

The WKB basis functions may also be used as an analytic tool in finding approximate solutions to related cochlear models. Once again, I will specify to 1-D box models with constant area. Starting with Eqn \ref{webster1D}, the dynamics are governed by a wave equation with spatially-varying wavenumber
\begin{equation}
    \frac{d^2 p}{dx^2} + k^2(x) p = 0,
    \label{complementary_eqn}
\end{equation}
where I have replaced the partial derivatives with ordinary derivatives as time dependence is implicit. This is a second order linear homogeneous ordinary differential equation (ODE) for which $W_{\pm}$ are approximate, linearly independent solutions. If they were truly solutions, $\beta = \{W_+,W_-\}$ would form a \textit{fundamental set} for this ODE, and its general solution would be given by Eqn \ref{p_proj1}.

A corresponding nonhomogeneous ODE to Eqn \ref{complementary_eqn} is given by
\begin{equation}
    \frac{d^2 p}{dx^2} + k^2(x) p = g(x),
    \label{nonhomog}
\end{equation}
where $g$ is some non-zero function defined on $I$ known as the \textit{forcing function}. Eqn \ref{complementary_eqn} is the \textit{complementary equation} (or associated homogeneous equation) for this nonhomogeneous ODE. The theory of linear ODEs tells us that the general solution to Eqn \ref{nonhomog}, $p_{gen}$, can be written as the sum of the general solution to the complementary equation, $p_c$, and any particular solution to Eqn \ref{nonhomog}, $p_p$ \cite{Zill_Wright_Cullen_2013}. That is,
$$p_{gen} = p_c + p_p.$$

The complementary solution $p_c$ is a linear combination of the functions in the complementary equation's fundamental set, which can be approximated by the set of WKB solutions $\beta$. That is,
\begin{equation}
    p_c \approx a_+ W_+ + a_- W_-,\;\;\; a_\pm\in\mathbb{C}.
    \label{compl}
\end{equation}
The theory of variation of parameters \cite{Zill_Wright_Cullen_2013} then gives a particular solution in terms of the functions in the fundamental set and the forcing function:
$$p_p = \frac{W_-}{\mathcal{D}}\int_0^x W_+(\xi) g(\xi)\;d\xi - \frac{W_+}{\mathcal{D}}\int_0^x W_-(\xi) g(\xi)\;d\xi,$$
where $\mathcal{D}$ is once again the Wronskian of the WKB functions already found to be $2j$ in Eqn \ref{wronskian}.

This gives an approximate closed-form general solution for Eqn \ref{nonhomog}:
\begin{align}
    \begin{split}
        p_{gen} &= \bigg[a_+ -\frac{1}{2j}\int_0^x W_-(\xi) g(\xi)\;d\xi \bigg]W_+ + \bigg[ a_-  + \frac{1}{2j}\int_0^x W_+(\xi) g(\xi)\;d\xi\bigg] W_-\\
        &= p_+ W_+ + p_- W_-.
    \end{split}
    \label{gen}
\end{align}
The values of $a_\pm$ are found through the boundary conditions. In particular, if a known pressure is applied at the stapes ($x=0$) creating an initial apical-traveling wave, we would have $a_+= p_0$ (known constant) and $a_-$ = 0. This gives the parameter-free closed-form solution for the stapes-driven nonhomogeneous 1-D model:
\begin{equation}
    p = \bigg[p_0 -\frac{1}{2j}\int_0^x W_-(\xi) g(\xi)\;d\xi \bigg]W_+ + \bigg[\frac{1}{2j}\int_0^x W_+(\xi) g(\xi)\;d\xi\bigg] W_-.
    \label{stapesdrivensln}
\end{equation}

This formulation has broad applications in the modeling of intracochlear reflections and OAEs, where model equations can be manipulated into the form of Eqn \ref{nonhomog} \cite{Talmadge_Tubis_Long_Piskorski_1998,Talmadge_Tubis_Long_Tong_2000}. In these cases, the forcing function $g$ will generally represent sources of reflections such as random perturbations in impedance or nonlinearity. This interpretation is visible in Eqn \ref{stapesdrivensln}, where $g$ can be thought of as a kernel in the integral of the basis function traveling in the opposite direction of that for which it is a coefficient. That is, the size of the apical-traveling component is modulated by the basal-traveling wave weighted by $g$, and vice versa.

In the previous subsection, I discussed the application of projection onto WKB waves to approximating local reflection phenomena (Eqn \ref{refdef}). The application is natural in this analytic treatment as well, given the $p_\pm$ values in Eqn \ref{gen}.

\subsection{Example: Analytic Treatment of Roughness}

There are various applications of WKB basis functions to the study of cochlear phenomena (several described in \cite{Talmadge_Tubis_Long_Piskorski_1998,Talmadge_Tubis_Long_Tong_2000}). In particular, various values of $g$ can be formulated to study different sources of reflection, including nonlinear phenomena (e.g. distortion product otoacoustic emissions). Here, I will provide a representative and important example -- that of applying \textit{roughness} to the cochlea's parameters.

Much work has been done regarding the study of the impact of roughness on the impedance in generating intracochlear reflections \cite{Talmadge_Tubis_Long_Piskorski_1998,Talmadge_Tubis_Long_Tong_2000,Shera_Tubis_Talmadge_2005,oae_sisto,oae_moleti,Sisto_Moleti_Shera_2007}. That is, if the smooth impedance $Z_{s}$ were modified by a small longitudinally varying perturbation,
$$ Z(x) = Z_{s}(x) + \delta Z(x),$$
this would impact the wavenumber of the traveling waves in both directions (Eqns \ref{webster1Dk} in 1-D, \ref{dispersion} in 2-D). One could also model this as a roughening of the wavenumber, where the smooth wavenumber would be $k_s$ and the roughened wavenumber would be
$$k^2 (x)= k_{s}^2(x) + \delta k^2(x).$$
For example, $\delta k^2(x)$ may be modeled as samples from independent identically distributed normal distributions at each $x$. The roughness could also be designed to depend on stimulus frequency, but this dependence will be left implicit as it will not impact the derivations.

Rewriting the wave-equation in terms of the roughened wavenumber, we have
$$\frac{d^2 p}{dx^2} + \big[ k_s^2(x) + \delta k^2(x)\big] p = 0,$$
which is in fact homogeneous and linear. However, $\delta k^2$ is not necessarily differentiable -- in fact, it ought not be as ``rough" implies non-smooth. This precludes use of the WKB approximation in its current form, as the WKB assumption (Eqn \ref{wkbassume}) is not well-posed.

Moving the $\delta k^2$ term to the opposite side gives
$$\frac{d^2 p_{r}}{dx^2} + k_s^2(x)  p = -\delta k^2 p,$$
which is still homogeneous as the right-hand side is proportional to the dependent variable $p$. The strategy is to approximate the right-hand side as a $p$-independent forcing function. If $\delta k^2$ is small, we can consider this right-hand term as a perturbation to the otherwise smooth, complementary response $p_c$ of Eqn \ref{compl}. In the case that an apical-traveling wave of magnitude $p_0$ is induced at the stapes, this gives the approximation 
$$-\delta k^2 p \approx -\delta k^2 p_0 W_+.$$
This is a known $p$-independent function, allowing the ODE to be interpreted as approximately nonhomogeneous. 

That is, it is in the form of Eqn \ref{nonhomog} with $g = -\delta k^2 p_0 W_+$. The roughened pressure solution can be given by substituting this forcing function for $g$ in Eqn \ref{stapesdrivensln}:
\begin{equation}
    p = p_0\bigg[1+ \frac{1}{2j}\int_0^x \delta k^2(
    \xi)W_+(\xi)W_-(\xi) \;d\xi \bigg]W_+ - p_0 \bigg[\frac{1}{2j}\int_0^x \delta k^2(\xi) W_+^2(\xi) \;d\xi\bigg] W_-.
    \label{roughresponse}
\end{equation}
This solution facilitates computation of the reflection coefficients through Eqn \ref{refdef} in terms of the roughness function. 