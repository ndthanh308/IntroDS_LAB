\section{WKB Solutions for The 2-D Model}
\label{sec:2d}

There are a number of methods for arriving at WKB approximate solutions for the 2-D model. The first solution presented in this tutorial is chosen due to its emphasis of the relationship between the 1-D and 2-D models. Under the assumptions of the model (see Sec \ref{sec:3d}), transmembrane pressure satisfies the 2-D Laplace equation:
\begin{equation}
    \frac{\partial^2 p}{\partial x^2} + \frac{\partial^2 p}{\partial z^2} = 0.
\end{equation}
One classical method for solving the Laplace equation is separation of variables, where it is assumed that the transmembrane pressure can be written as a product of a function of only $x$ and a function of only $z$:
$$p(x,z) = \mathcal{X}(x)\mathcal{Z}(z).$$

If separation of variables were satisfied, the solution would a linear combination of eigenfunctions with eigenvalue $k$. These are of the forms
\begin{align}
    p_{k} &= (A\cosh{k x} + B\cosh{k x})(Ce^{jk z} + De^{-jk z}), \label{coshxexpz}\\
    p_{k} &= (Ae^{jk x} + Be^{-jk x})(C\cosh{k z} + D\sinh{k z}).
    \label{coshzexpx}
\end{align}
We know that the $x$-dependence of the solution should have the form of a wave, so $\mathcal{Z}$ should have the form of Eqn \ref{coshzexpx}:
$$\mathcal{Z}(z) = C\cosh{kz} + D\sinh{kz}.$$


Plugging in the boundary condition at the outer wall (Eqn \ref{wallBV}) gives
$$\mathcal{Z}'(h) = k[C\sinh{kh} + D\cosh{kh}] = 0,$$
yielding the relationship $C = -D/\tanh{kh}$. A hyperbolic trigonometric identity gives
\begin{equation}
    \mathcal{Z}(z) = \frac{D}{\sinh{kh}}(\sinh{kh}\sinh{kx} - \cosh{kh}\cosh{kx}) = \frac{D}{\sinh{kh}}\cosh[k(z-h)]. 
\end{equation}
Separation of variables has already been broken, as $h$ depends on $x$, but this nonetheless gives motivation for writing the form of the solution as 
$$p(x,z) = \cosh{[k(z-h)]}\mathcal{X}(x),$$
where $k$ may also vary in $x$. In this form, the solution satisfies the boundary condition at the outer wall. 

Here I will make the first of two WKB approximations by assuming 1) the form of $\mathcal{X}$ is that of Eqn \ref{WKB_withdelta} with $\delta = 1$, and 2) Eqns \ref{S_assumption} and \ref{wkbassume} hold so that $x$ derivatives of $\cosh[k(z-h)]$ are small. The Laplace equation to zeroth order becomes:
$$C_0^2\cosh[k(z-h)]\mathcal{X}(x) + k^2\cosh[k(z-h)]\mathcal{X}(x) = 0,$$
where the second term is the second $z$ derivative of $p$. Just as in the 1-D case, this gives
$$C_0 = \pm j\int_0^x k(\xi)\;d\xi.$$
Assuming that the basal-traveling wave is negligible, the pressure is 
\begin{equation}
    p(x,z) = A(x)\cosh{[k(z-h)]}e^{-j\int_0^x k(\xi)\;d\xi}.
    \label{pwkb2d1}
\end{equation}

A dispersion relation is still needed so that $k$ can be solved for. To do so, recall the boundary condition at $z=0$ (Eqn \ref{bvImpedance}). Pressure and velocity potential are related by $p=-2\rho j\omega \phi$ (Eqn \ref{ptophi}), so the $z$ components of their first derivatives in $z$ give
$$ \dot{w} = \frac{- 1}{2\rho j \omega} \frac{\partial p}{\partial z},$$
or plugging in Eqn \ref{pwkb2d1},
\begin{equation}
    \dot{w} = \frac{- 1}{2\rho j \omega} A(x) k \sinh{[k(z-h)]}e^{-j\int_0^x k(\xi)\;d\xi}.
\end{equation}

At $z=0$, the ratio of velocity and pressure is OCC admittance, $Y_{OC}$.  That is, 
\begin{align*}
    Y_{OC} &= \frac{- A(x) k \sinh{[kh]}e^{-j\int_0^x k(\xi)\;d\xi}}{2\rho j \omega A(x) \cosh{[kh]}e^{-j\int_0^x k(\xi)\;d\xi} } \\
    &= \frac{-k \tanh{[kh]}}{2j\omega \rho},
\end{align*}
giving the dispersion relation
\begin{equation}
    k\tanh{kh} = -2j\omega\rho Y_{OC}.
    \label{dispersion}
\end{equation}
This dispersion relation\footnote{\textbf{A note on terminology: }This equation is often called ``the eikonal equation" in literature due to an analogy to geometric optics, but this language is imprecise. Eikonal equations are a class of differential equations that appear elsewhere in the above derivations, and this algebraic expression is not such an equation. Instead, I will simply use the term \textit{dispersion relation}, as it is more descriptive and precise.} is transcendental and does not possess a unique solution for $k$ (the implications of this will be covered in detail in Sec \ref{sec:implement}). Eqn \ref{dispersion} is also independent of $A$, meaning it will be valid for any approximation of $p$ in the form of Eqn \ref{pwkb2d1}.

Solving Eqn \ref{dispersion} for impedance gives
$$Z_{OC} = j\omega \frac{-2 \rho}{k\tanh{kh}}.$$
This allows for an attractive interpretation of the impact of the impedance on the traveling wave. Due to the leading $j \omega$, this appears similar to a mass (although $k$ is complex so that this is not actually the case). In particular, defining the \textit{effective height} as
\begin{equation}
    h_{e}(k) = \frac{1}{k\tanh{kh}},\;\;\;Z_{OC} = -2j\omega \rho h_{e},
    \label{eqn::heff}
\end{equation}
the impedance is that of a column of fluid with this effective height. 

\subsection{Pressure Focusing}

To determine $A(x)$ in Eqn \ref{pwkb2d1}, recall that the \textit{average} pressure in the 2-D model must satisfy the Webster horn equation (Eqn \ref{webster2D}). The average pressure is 
\begin{align}
    \begin{split}
    \bar{p}(x) &= \frac{1}{h(x)}\int_0^{h(x)} A(x)\cosh[k(z-h)]e^{-j\int_0^x k(\xi)\;d\xi}\;dz\\
    &= \frac{1}{k(x) h(x)}A(x)\sinh[k(x)h(x)]e^{-j\int_0^x k(\xi)\;d\xi}.
    \end{split}
    \label{barp}
\end{align}

The pressure focusing factor  $\alpha = p(x,0)/\bar{p}(x)$ is required to find $k_{2D}$ in Eqn \ref{webster2Dk}, and can now be found using Eqn \ref{pwkb2d1}:
\begin{equation}
    \alpha(x) = \frac{k(x)h(x)}{\tanh{[k(x)h(x)]}}.
    \label{focus}
\end{equation}
This is independent of $A$, meaning it will be valid for approximation of $p$ in the form of Eqn \ref{pwkb2d1}.

Plugging this into Eqn \ref{webster2Dk}, $k_{2D}^2$ can be found to be
$$k_{2D}^2 = \frac{-2j \omega Y_{OC} \rho k h}{h\tanh{[kh]}},$$
but by Eqn \ref{dispersion}, this simplifies directly to 
$$k_{2D}^2 = k^2.$$
In Sec \ref{sec:1d}, zeroth- and first-order WKB approximations for solutions to the Webster horn equation were derived. This gives an approximate formula for average pressure by directly copying Eqn \ref{WKB1Dpressure1}:
\begin{equation}
    \bar{p}(x) = P_{OW} \sqrt{\frac{S_0 k_0}{S(x) k(x)}} e^{-j\int_0^x k(\xi)\;d\xi}.
\end{equation}
Equating this with the earlier expression for $\bar{p}$ in Eqn \ref{barp}, it is finally possible to solve for $A(x)$:
\begin{equation}
    A(x) = P_{OW} \frac{k(x) h(x)}{\sinh[k(x)h(x)]}\sqrt{\frac{S_0 k_0}{S(x) k(x)}}.
\end{equation}

Finally, after this second application of a WKB approximation, a 2-D equation for pressure has been derived:
\begin{equation}
    p(x,z) = P_{OW} \frac{k(x) h(x)}{\sinh[k(x)h(x)]}\sqrt{\frac{S_0 k_0}{S(x) k(x)}} \cosh{[k(z-h)]}e^{-j\int_0^x k(\xi)\;d\xi}.
    \label{WKB2Dpressure}
\end{equation}
In conjunction with the dispersion relation of Eqn \ref{dispersion}, this allows solution for pressure or velocity throughout the scala. 

\subsection{A Higher-Order 2-D Approximation}
\label{sec::higherorder}
The above derivation arrives at Eqn \ref{WKB2Dpressure} through two consecutive applications of the WKB approximation, and neatly piggy-backs off of 1-D results for average pressure. However, this formula is not the only solution referred to in literature as ``the WKB solution" for a 2-D model.

Various alternate approximation methods arrive at the following equation for pressure in a box model:
\begin{equation}
    p(x,z) = P_{OW}\frac{k_0 h}{\cosh{kh}\tanh{k_0 h}} \sqrt{
    \frac{\tanh{k_0h} + k_0h\,\text{sech}^2{k_0h}}{\tanh{kh} + kh\,\text{sech}^2{kh}}} \cosh{[k(z-h)]} e^{-j\int_0^x k(\xi)\;d\xi }
    \label{duifhuisKING}
\end{equation}
(e.g. \cite{steele_lagrange,viergever_Book,duifhuis_moh}). It is clear that this has the form of Eqn \ref{pwkb2d1}, which means that this solution shares the same effective height, dispersion relation and pressure focusing factor as the solution derived above (Eqns \ref{eqn::heff}, \ref{dispersion}, \ref{focus}). 

One derivation involves the solution for $p$ as a formal power series approximation \cite{viergever_Book}, inspired by the physics of surface waves \cite{keller}. It is informative but lengthy, and an outline can be found in App \ref{app:series}\footnote{A more intricate treatment can be found at https://github.com/brian-lance/wkb-derivations}.

A second derivation of this formula follows from considering the Euler-Lagrange equations in a lossless box model (i.e. $Z_{OC}$ purely imaginary) \cite{steele_lagrange}. The derivation in terms of energy conservation is enlightening, but similarly lengthy. An outline of the derivation has been placed in App \ref{app:variational}\footnote{A more intricate treatment can be found at https://github.com/brian-lance/wkb-derivations}. Neither derivation explicitly relies on the WKB approximation, although they do rely on the WKB assumption and feature the characteristic WKB phase term (the integral of the wavenumber). 

While Eqn \ref{duifhuisKING} behaves similarly to Eqn \ref{WKB2Dpressure}, its responses match numerical solutions better in the peak region. On the other hand, Eqn \ref{duifhuisKING} only holds for box models where $h$ is constant, not allowing for the modeling of cochlear tapering. Contemporary work is largely partial to the lower-order approximation of Eqn \ref{WKB2Dpressure} \cite{sisto_2021,sisto_2023,earhorn,altoe_2022,Shera_Altoe_2023}. Differences in the behavior between these solutions will be discussed in Sec \ref{sec:results}. 

\subsection{Long- and Short-Wave Solutions}

It is also instructive to consider the behavior of the solution in the long-wave (small $k$, basal to best place/lower frequency than best frequency) and short-wave (large $k$, near best place/best frequency). These approximations lie in the limiting behavior of the hyperbolic tangent for real argument $a\in\mathbb{R}$: 
$\tanh{a}\approx a$ if $a$ is small and $\tanh{a} \approx 1$ if $a$ is large.

The dispersion relations (Eqn \ref{dispersion}) in the long-wave and short-wave limits are
\begin{align}
    k_{lw}^2 = \frac{-2 j\omega \rho}{Z_{OC} h},\\
    k_{sw} = -2j \omega \rho Y_{OC}.
\end{align}
These are explicit solutions for the wavenumber in these regions, simplifying computation. Notably, $k_{lw}$ is precisely the wavenumber from the 1-D Webster horn equation (Eqn \ref{webster1Dk}). The short-wave solution is not particular to a WKB approximation -- it has been derived in an alternate manner by Ranke and Siebert \cite{Ranke_1950,Siebert_1973}. 

Considering the same limiting behavior for the pressure focusing factor (Eqn \ref{focus}) gives
\begin{align}
    \alpha_{lw} = 1,\\
    \alpha_{sw} = kh.
\end{align}
This reinforces the realization that the long-wave approximation and the 1-D solution are equivalent at $z=0$. The effective height, $h_e$ from Eqn \ref{eqn::heff} also has corresponding long- and short-wave approximations:
\begin{align}
    h_{e,lw} = \frac{1}{hk^2},\\
    h_{e,sw} = \frac{1}{k}.
\end{align}

To visualize the difference between the long-wave, short-wave and WKB approximations, one can observe the effective height as $k$ varies. Fig \ref{fig::he} shows $h_e$ for the three solutions across various values of positive real $k$ with $h=1$ mm. It can be seen that the WKB solution for $h_e$ exhibits a continuous switch-off between the long- and short-wave approximations near the point where these solutions intersect. Behaviors of long- and short-wave velocity responses are discussed in Sec \ref{sec:results}.

% Figure environment removed
