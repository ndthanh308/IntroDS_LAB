\section{Introduction}

Models of 1-D and 2-D macromechanics have offered some of the most significant interpretations of cochlear physiology, both historic and modern. It is intuitive that 3-D models should offer more physically realistic results than 2-D or 1-D models, but this alone implies a ``more the merrier" view of model dimensionality that coincides with quantitative accuracy, but not with frequency of implementation or impact on the field of cochlear mechanics.

In fact, 1-D models are responsible for some of the most important modeling results and interpretations of cochlear mechanics. This includes the existence and character of stapes-driven forward-traveling waves and the presence of a region of negative damping  \cite{Wegel_Lane_1924,Zwislocki_1980,Peterson_Bogert_1950,Zweig_compromise,Zweig_1991,catastrophe,Elliott_Ku_Lineton_2007} as well as intra-cochlear reflections and otoacoustic emissions \cite{Viergever_1986,de_Boer_Kaernbach_König_Schillen_1986,basis,Zweig_1991,Talmadge_Tubis_Long_Piskorski_1998,Talmadge_Tubis_Long_Tong_2000,Shera_Tubis_Talmadge_2005,oae_moleti,oae_sisto}. Qualitative similarity across frequency/space and quantitative similarity in the long-wave region to \textit{in vivo} cochlear responses make 1-D models attractive for the exploration of fundamental macromechanical phenomena. Both implementation and modification of the dynamics to account for features such as nonlinearity and roughness are also far simpler in 1-D models than in 2- or 3-D models.

On the other hand, 2-D models allow for more physical results in the short-wave and cutoff regions of the cochlear response than 1-D models \cite{Siebert_1973,Allen_1977,Allen_Sondhi_1979,neely,steele_rootfinding,viergever_Book}, allowing more complete exploration of potential mode conversion occurring in the traveling wave \cite{watts,Watts_2000,Elliott_Ni_Mace_Lineton_2013}. Moreover, 2-D models are able to capture fluid mechanical properties in the scalae that allow for interpretation of energy flow \cite{lighthill_long,lighthill_short,steele_lagrange}, or the manner by which pressure across scalae is focused at the organ of Corti complex (OCC) to supply energy to the traveling wave\footnote{This concept has been of particular interest in cochlear mechanics since the pioneering work of Olson, who showed that pressure varies rapidly in space within the scala, and is tuned near the OCC \cite{Olson_2001}.} \cite{duifhuis_moh,sisto_2021,sisto_2023,Shera_Altoe_2023}, whereas 1-D models describe only the average pressure across the scalae.

The macromechanics of the cochlea are generally modeled as boundary value problems (BVPs) where the model equations involve partial differential equations (PDEs) without analytically known solutions, such as the Navier-Stokes equation. Such model equations can be tackled using numerics (e.g. the finite element method), or by making sufficient simplifying assumptions so that approximate analytic solutions can be found -- the scala walls are rigid, the fluids are incompressible, etc. 

Techniques based on the Wentzel-Kramers-Brillouin (WKB) approximation, also known as the Liouville-Green (LG) approximation \cite{Liouville_1837,green,mathews_wkb,Dingle_1975,Winitzki_2005}, are among the most popular for achieving approximate, explicit closed-form solutions OCC motion, fluid pressure and fluid velocity in a variety of cochlear models that match numerical solutions well across broad frequency and spatial ranges \cite{Zweig_compromise,Zweig_1991,duifhuis_moh,viergever_Book,de_Boer_Kaernbach_König_Schillen_1986,deBoer_PhysicsReports,deBoer_Energy,sisto_2021,altoe_2022,sisto_2023,Shera_Altoe_2023}. 

Closed-form explicit solutions allow for more easily interpreted model results. While exact solutions have been derived and studied for some cochlear models -- e.g. implicit Green's function solutions \cite{Allen_1977,Allen_Sondhi_1979,Mammano_Nobili_1993} or explicit Fourier transform solutions for box models \cite{deboer_block} -- they are not as simple to qualitatively analyze. 

As an nonexhaustive list, 1-D and 2-D WKB approximate solutions have offered: interpretations of limits on cochlear tuning \cite{Zweig_compromise}; interpretations of intracochlear reflections and OAEs \cite{basis,Talmadge_Tubis_Long_Piskorski_1998,Talmadge_Tubis_Long_Tong_2000,Shera_Tubis_Talmadge_2005,oae_sisto,Sisto_Moleti_Shera_2007,Shera_Altoe_2023}; interpretations of traveling wave mode switching \cite{watts,Watts_2000,Elliott_Ni_Mace_Lineton_2013}; interpretations of the impact of active power generation in the cochlea \cite{Wang_Steele_Puria_2016,sisto_2021,sisto_2023}. Robustness, ease of implementation, interpretability and versatility have earned the WKB approximation its persistence in macromechanics modeling over half of a century.

With the passage of time, foundations of WKB techniques have largely disappeared from cochlear mechanics literature; as with any historical method, derivations, assumptions, implementation details and the implications thereof have become implicit. This efficiency is useful for experienced readers, but creates confusion for newer entrants to the field. In the case of WKB, not only are these objects often missing in contemporary literature, but challenging to find in historic literature as well.

The relevance of the WKB approximation in cochlear models, both historical and contemporary, owes it a foundational exposition. Fundamental understanding of WKB can open the door to adaptations for probing particular questions, with knowledge of its strengths and limitations. As such questions continue to arise with the publication of new data, especially with the advent of optical coherence tomography, this is all the more relevant.

Lastly, the recent passing of Egbert de Boer, Hendrikus Duifhuis and Charles Steele -- three pioneers of the application of the WKB approximation to cochlear mechanics -- suggests a timeliness of such a presentation.

The essence of this tutorial is to present the fundamentals of WKB techniques in linear 1-D and 2-D cochlear mechanics models from an analytic perspective, covering derivations and details of implementation and performance. 

I begin by describing general mathematical details of the WKB approximation agnostic to cochlear applications. This is followed by a description of the 1-D and 2-D BVPs for the box model of the cochlea. Derivations of the 1-D and 2-D WKB solutions to these BVPs follow. I then discuss the theory of the WKB traveling wave subspace (in terms of ``WKB basis functions") most often used in the study of intracochlear reflections and OAEs.

For readers interested in implementation rather than theory, Sec \ref{sec:implement} discusses practical concerns. This includes discussion of several common methods for solving the dispersion relation for 2-D box models, along with details of their performance across frequency and spatial ranges. This is followed by a comparison across methods and to numerical solutions.
