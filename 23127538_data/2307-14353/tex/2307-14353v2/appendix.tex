\appendix
\renewcommand{\theequation}{A.\arabic{equation}}
\setcounter{equation}{0}
\section{The Higher-Order 2-D Model -- A Series Solution Approach}
\label{app:series}

In this appendix, I provide a derivation of the higher-order ``WKB solution" of Eqn \ref{duifhuisKING} from the main text. The following approach is followed by Viergever in his 1980 book \textit{Mechanics of the Inner Ear: A Mathematical Approach} \cite{viergever_Book}. It relies on a transformation of the coordinates of the pressure BVP, and subsequent application of a WKB-adjacent ansatz (but not precisely the WKB method).

The method consists of the following steps:
\begin{enumerate}
    \item{Change the variables of the BVP in pressure so that terms relating to the model parameters appear in the PDE rather than only in the boundary conditions.}
    \item{Write a form for the solution to this new PDE as
    \begin{equation*}
        A(x,\zeta)\cosh{\kappa(x)(H-\zeta)}e^{jKg(x)}.
    \end{equation*}
    That is, assume that the $z$ (here reparameterized as $\zeta$) contribution is hyperbolic and that there is a wave in $x$. The product with arbitrary $A(x,y)$ means this is done without loss of generality.}
    \item{Assume a series solution for $A$ and plug into the ODE to obtain a system of PDEs.}
    \item{Solve for $A$ up to first order, plug back in to the ansatz and undo the change of variables to solve for pressure.}
\end{enumerate}

A detailed outline is presented below, but certain steps feature highly nontrivial computations. For full exposition of these computations, see  https://github.com/brian-lance/wkb-derivations.

\subsection{Setting up the BVP}

The method followed in this section relies on multiple changes of variables and definitions of new parameters. As such, it can be difficult to keep straight the meanings and units of the various variables and parameters at play. Table \ref{tab::vier} serves as a reference for the objects introduced in the derivation.

\begin{table}
    \centering
    \begin{tabular}{|c||c|c|} \hline 
         Symbol & Significance & Units\\ \hline \hline
         $Z_0$ & Arbitrary reference impedance used to the simplify series solution. & Pa$\cdot$s/mm \\ \hline
         $K$ & Reference wavenumber used to simplify the series solution.  & 1/mm \\ \hline
         $f^2(x)$ & $Z_0/Z_{OC}(x)$, used so simplify $x$-dependence of the PDE. & Unitless \\ \hline
         $\zeta$ & $Kz$, Nondimensionalized transverse coordinate. & Unitless \\ \hline
         $H$ & $Kh$, Nondimensionalized scala height. & Unitless \\ \hline
         $Q(x,\zeta)$ &  Pressure written in terms of the nondimensionalized transverse coordinate. & Pa \\ \hline
         $A(x,\zeta)$ & Auxiliary pressure variable that controls the magnitude of pressure at the OCC,& \\ & to be solved for in the simplified BVP. & Pa \\ \hline
         $\kappa(x)$ & Controls the $x$-dependence of transverse pressure variations, & \\ & to be solved for in the simplified BVP. & Unitless\\ \hline
         $g(x)$ & Controls the wavenumber of the traveling wave, & \\ & to be solved for in the simplified BVP. & mm \\ \hline
    \end{tabular}
    \caption{Symbols introduced in the derivation of the model equations in the series solution approach, along with their significance and units.}
    \label{tab::vier}
\end{table}

Viergever begins with the 2-D box model BVP in $P(x,z)$, then performs a change of variables. To start, define a reference impedance $Z_0$ which is some arbitrary constant. We define also $f^2(x) = Z_0/Z_{OC}(x)$ and a reference wavenumber $K^2 = -2j\omega\rho/hZ_0$. Recalling that $P = -2\rho \dot{\phi}$,  the boundary condition at the OCC is
\begin{equation}
    \frac{\partial P}{\partial z} + hK^2f^2(x)P=0,\;\;\;z=0.
\end{equation}
Note that $K$ is not a function of $x$.

Further reparameterizing the $z$ coordinate and defining a reparameterized pressure, $Q$, as 
\begin{equation}
    \zeta = Kz,\;\;H=Kh,\;\;Q(x,\zeta) = P(x,z),
\end{equation}
the BVP in terms of $Q$ is
\begin{align}
    \frac{\partial^2 Q}{\partial x^2} + K^2\frac{\partial^2 Q}{\partial \zeta^2} = 0,\\
    \frac{\partial Q}{\partial \zeta}\bigg|_{\zeta=H} = 0,\\
    \frac{\partial Q}{\partial \zeta}\bigg|_{\zeta=0} + Hf^2(x)Q(x,0) = 0.
\end{align}


\par{The solution to the above PDE is artificially represented in a form resembling what the solutions are expected to be, by intuition about the Laplace equation. In particular, $Q$ is written as
\begin{equation}
    Q(x,\zeta) = A(x,\zeta;K)e^{jKg(x)}\cosh{[\kappa(x)(H-\zeta)].}
    \label{waveishguess}
\end{equation}

 The exponential suggests a traveling wave in $x$, where $A$ modulates the amplitude of this wave. However, this is not actually an assumption of a wave solution -- as $A$, $g$ and $\kappa$ are unknown functions of $x$ (and $\zeta$, for $A$), any function can be represented in this fashion without loss of generality.}

\par{One might wonder why we have chosen to introduce so many new terms into these equations. While this may initially seem to complicate the BVP, it eventually leads to the most mathematically tractable solution method.}
\par{Alongside Table \ref{tab::vier}, it may help to ``look into the future" to see what these newly defined variables will become. The variable $\kappa$ will be found to be the nondimensionalized wavenumber and $g$ will be found to be the integral of the wavenumber. What is intriguing about this method is that we are not \textit{assuming} that the wavenumber will appear, but rather it falls out of the derivation.}
\par{Moreover, the free parameter $K$ will eventually be the variable of our formal power series (similar to $\delta$ from Eqn \ref{WKB_withdelta}).}
 
Plugging this form of $Q$ into the BVP, we can find an equivalent BVP in terms of $A$. Writing $a(x,\zeta) = \kappa(x)(H-\zeta)$ to simplify notation, we arrive at the following PDE and boundary conditions:
\begin{align}
\begin{split}
    K^2&\bigg[(\kappa^2 - g'^2)A\cosh{a} + \frac{\partial^2 A}{\partial \zeta^2}\cosh{a} -2\kappa \frac{\partial A}{\partial \zeta}\sinh{a} \bigg]+\\
    +jK&\bigg[g''A\cosh{a} + 2g'\frac{\partial A\cosh{a}}{\partial x} \bigg] + \frac{\partial^2 A\cosh{a}}{\partial x^2}=0,
    \end{split}
    \label{PDEinA}
\end{align}
\begin{equation}
    \frac{\partial A}{\partial \zeta}\bigg|_{\zeta = 0} -\kappa A(x,0)\tanh{a(x,0)} + Hf^2A(x,0) = 0,
\end{equation}
\begin{equation}
    \frac{\partial A}{\partial \zeta}\bigg|_{\zeta = H} = 0.
\end{equation}

Solving this ODE in the auxiliary pressure $A$ is the new goal. With a solution for $A$, we can find $Q$ and finally $P$.

\subsection{A Series Solution for Auxiliary Pressure}
A formal power series solution in the form
\begin{equation}
    A(x,\zeta;K) = A_0(x) + \sum_{n=1}^\infty \frac{1}{(jK)^n}A_n(x,\zeta)
\end{equation}
is assumed, with monotonic decrease in magnitude of terms and their derivatives in increasing $n$, and allowing for termwise differentiation. This form of the solution is not quite the WKB ansatz, but the logarithm of such a solution with $\delta = jK$. This is motivated by Keller's approach to surface waves on water of non-uniform depth \cite{keller}.

This ansatz is plugged into the PDE for $A$ in Eqn \ref{PDEinA}, resulting in a system of infinitely many PDEs of which we consider only the PDEs including $A_0$ and $A_1$ (justified by the terms and their derivatives being assumed to decrease monotonically). The resulting system of differential equations is
\begin{equation}
    g'^2(x) = \kappa^2(x),
    \label{recurse0}
\end{equation}

\begin{equation}
    \cosh{a}\frac{\partial^2 A_1}{\partial \zeta^2} - 2\kappa\sinh{a}\frac{\partial A_1}{\partial \zeta} =g''A_0\cosh{a} + 2g'\frac{\partial A_0 \cosh{a}}{\partial x}.
    \label{seriespde0}
\end{equation}

Application of boundary conditions gives
\begin{equation}
    \frac{\partial A_1}{\partial \zeta}\bigg|_{\zeta=H} = 0,
    \label{Hboundary}
\end{equation}
\begin{equation}
    \frac{\partial A_1}{\partial \zeta}\bigg|_{\zeta=0} = 0,
\end{equation}
\begin{equation}
    \kappa\tanh{\kappa H} = Hf^2.
    \label{kapparelation}
\end{equation}
The final equation resembles the dispersion relation derived from the WKB method in Sec \ref{sec:2d}.

Eqn \ref{recurse0} is solved by
\begin{equation}
    g(x) = \pm \int_0^x \kappa(\xi)\;d\xi + C
    \label{geqn}
\end{equation}
for arbitrary constant $C$ \footnote{\textbf{Note on terminology:} Eqn \ref{recurse0} is actually an \textit{eikonal equation}, although it is not related to the dispersion relation that is often refered to as ``the eikonal equation" in literature.}. This resembles the characteristic WKB phase term. 

\subsection{Finding a First Approximation for Pressure}

Solving Eqn \ref{seriespde0} is nontrivial, as it contains both $A_0$ and $A_1$. Solution for $A_0$ requires clever substitutions\footnote{This is outlined at https://github.com/brian-lance/wkb-derivations}. I find
\begin{equation}
    A_0 = C(\kappa H + \sinh{\kappa H}\cosh{\kappa H})^{-1/2},
\end{equation}
for arbitrary $C$. 

Theoretically this facilitates solution for $A_n$ for any $n$ as well. On the other hand, the series approximation gives that the higher $n$ terms should be small if $K$ is large relative to its own rate of change (analogous to the WKB assumption).

Ignoring $A_n$ for $n\geq 1$ gives a first approximation for $Q$ by putting $A\approx A_0$. Using Eqn \ref{geqn} for $g$, there are two possible solutions: 
\begin{equation}
    Q_{\pm}(x,\zeta) = C_{\pm}(\kappa H + \sinh{\kappa H}\cosh{\kappa H})^{-1/2} e^{\pm j K \int_{0}^x\kappa(\xi)\;d\xi}\cosh{[\kappa(x)(H-\zeta)]} + O(1/K).
\end{equation}

The reference constant $K$ was defined as $K^2 = -2j\rho\omega/hZ_0$, where $Z_0$ was a \textit{second} reference constant so that $f^2 = Z_0Y_{OC}$. Because $Z_0$ was entirely arbitrary, I am free to choose $Z_0 = -2j\rho\omega h^{-1}$ so that $K=1$ mm$^{-1}$. This also gives $H=1$ mm$^{-1} \times h$ [unitless], $\zeta=1$ mm$^{-1} \times z$ [unitless] and $Q = P$ [Pa].

I define $k = 1$ mm$^{-1} \times \kappa$. In this light, the $\kappa$ relation from Eqn \ref{kapparelation} becomes
\begin{equation}
    k \tanh{k h} = -2j\rho \omega Y_{OC}.
    \label{dispViergever}
\end{equation}
This is \textit{precisely} the dispersion relation derived through the WKB method (Eqn \ref{dispersion}), where $k$ is the wavenumber with units mm$^{-1}$.

The first approximation for pressure with arbitrary constants $C_{\pm}$ is now
\begin{equation}
    P(x,z) = (k h + \sinh{k h}\cosh{k h})^{-1/2}\cosh{[k(x)(h-z)]}\bigg[ C_+ e^{j \int_{0}^x k(\xi)\;d\xi} + C_- e^{- j \int_{0}^x k(\xi)\;d\xi}\bigg].
\end{equation}

To find the constants, the two $x$ boundary conditions are used:
\begin{equation}
    \frac{1}{h}\int_0^h P(0,z)\;dz = P_{OW},\;\;\; \frac{\partial P}{\partial x}\bigg|_{x=L} = 0,
\end{equation}
where $L$ is the length of the cochlea and $P_{OW}$ is the average pressure at the stapes. 

After some computation\footnote{These can be found at https://github.com/brian-lance/wkb-derivations}, assuming that the backwards traveling wave is negligible, we achieve
\begin{equation}
     P(x,z) = \frac{P_{OW} k_0 h}{\cosh{k(x) h} \tanh{k_0 h}}\sqrt{\frac{k_0 h\text{sech}^2 k_0 h + \tanh{k_0 h}}{k(x) h \text{sech}^2 k_0 h + \tanh{k(x) h}}} \cosh{[k(x)(h-z)]} e^{-j\int_0^x k(\xi)\;d\xi}.
\end{equation}
This is precisely Eqn \ref{duifhuisKING}.

\clearpage
\section{The Higher-Order 2-D Model -- The Variational Approach}
\label{app:variational}
\renewcommand{\theequation}{B.\arabic{equation}}
\setcounter{equation}{0}

In this appendix, I provide a derivation of the higher-order ``WKB solution" of Eqn \ref{duifhuisKING} from the main text. The following approach is followed by Steele and Taber \cite{steele_lagrange}. It relies on Lagrangian mechanics, which frames the problem in terms of energy conservation. This offers significant insight into the dynamics of the cochlea.

The method is inspired by Whitman's treatment of waves in fluid \cite{whitman_waves}, and consists of the following steps:
\begin{enumerate}
    \item{Solve the BVP in velocity potential under the assumption that all parameters are constant.}
    \item{Find the Lagrangian of this system.}
    \item{Write the Euler-Lagrange equations for the system in its parameters previously assumed to be constant.}
    \item{Solve these ODEs to find a non-constant form of these parameters, yielding a higher-order approximation.}
\end{enumerate}

\subsection{Solving the Laplace Equation with Constant Parameters}

Step (1) follows from the method of separation of variables. This time, however, the BVP will be solved in terms of velocity potential which also satisfies the Laplace equation. We begin by assuming that the displacement at the basilar membrane is a wave \textit{with only one mode}, and travels only in the base-to-apex direction. That is, transverse displacement $w$ at $z=0$ is 
\begin{equation}
w(x,0) = We^{-jkx},
\end{equation}
where $W$ and $k$ are assumed constant for now. Looking ahead, these will be the variables of our Euler-Lagrange equations in step (3).

Solution for the  was already shown for pressure in Sec \ref{sec:2d}, and with constant $k$ (and ignoring the basal-traveling wave) we arrive again at:
$$\phi(x,z) = A\cosh{k(z-h)}e^{-jkx}$$
The value of (constant) $A$ can be found via application of the boundary condition at $z=0$,
\begin{equation}
    \frac{\partial \phi}{\partial z}\bigg|_{z=0} = j\omega W e^{-jkx}.
\end{equation}
Plugging into the formula for velocity potential gives
$$-A k \sinh {kh}= j\omega W,$$
giving
\begin{equation}
    \phi(x,z)= \frac{-j\omega W}{k\sinh{kh}}\cosh{[k(z-h)]}e^{-jk}.
    \label{phi}
\end{equation}

As little has been done differently from the derivation in Sec \ref{sec:2d} thus far, it is easily shown that the same equations hold for the dispersion relation, effective height and pressure focusing (See Sec \ref{sec::higherorder}). That is, all of Eqns \ref{dispersion} , \ref{eqn::heff} and \ref{focus} hold in this derivation. Having already derived these quantities, the derivation is greatly accelerated.

\subsection{Computing the Lagrangian with Constant Parameters}

The system's (time-averaged) energy can be separated into three components -- the potential energy in the OCC $V$, the kinetic energy in the OCC $T_{OC}$ and the kinetic energy in the fluid $T_f$. We write the impedance at the OCC as a standard linear point-impedance, having a mass $M$, stiffness $S$ and resistance $R$: 
$$Z_{OC} = \frac{p(x,0)}{\dot{w}(x,0)} = j\omega M + R + \frac{S}{j\omega}.$$ 
In general, all of these quantities may be $x$-dependent. Lagrangian mechanics assumes a lossless system \cite{symon_1980}, and thus the impact of resistance will be ignored and simply ``added in" at the end of the derivation. As such, we will define an undamped OCC impedance by $Z_u = j\omega M + \frac{S}{j\omega}$

The potential energy density is given by
\begin{align}
\begin{split}
    V &= \frac{1}{2\pi}\int_{0}^{2\pi} \frac{1}{2}K \mathcal{R}[w(x,0,t)]^2\; d\omega t \\
    &= \frac{1}{2\pi}\int_{0}^{2\pi} \frac{1}{2}K W^2 \cos^2{\omega t}\;d\omega t \\
    &= \frac{KW^2}{4}.
\end{split}
\end{align}
Here, $\mathcal{R}[z]$ denotes the real part of  complex number $z$.

The kinetic energy of the OCC is computed as
\begin{align}
\begin{split}
    T_{OC} &= \frac{1}{2\pi}\int_{0}^{2\pi} \frac{1}{2}M \mathcal{R}[\dot{w}(x,0,t)]^2\; d\omega t \\
    &= \frac{1}{2\pi}\int_{0}^{2\pi} \frac{1}{2} M \omega^2 W^2 \sin^2{\omega t}\;d\omega t \\
    &= \frac{M\omega^2W^2}{4}.
\end{split}
\end{align}

To compute the fluid kinetic energy at position $x$, the two-dimensional fluid velocity must be considered over the whole cross-section in both chambers. After length computations, we arrive at
\begin{align}
\begin{split}
    T_f = &= 2\frac{1}{2\pi}\int_{0}^{2\pi}\int_{0}^{h} \frac{1}{2} \rho (\mathcal{R}[\dot{u}]^2 + \mathcal{R}[\dot{w}]^2)\;dz d\omega t \\
    &= \frac{1}{2}h_{e}\rho \omega^2 W^2.
\end{split}
\end{align}

The total kinetic energy density is the sum of the fluid and membrane contributions. This lets us write the Lagrangian:
\begin{equation}
    \mathcal{L}(k,W) = T-V = \frac{f(k) W^2}{4},
\end{equation}
where
\begin{equation}
    f(k) = 2h_{e}(k)\rho\omega + M\omega^2 - S.
\end{equation}
The last two summands in the above equation resemble the impedance $Z_{u}$, allowing us to write the function $f$ above as 
\begin{equation}
    f(k) = 2\rho \omega^2 h_{e}(k) - j\omega Z_u.
\end{equation}

Guided by the the WKB approximation, we now replace the phase term by defining
\begin{equation}
    \theta = \int_{0}^x k(\xi)\;d\xi.
    \label{thetadef}
\end{equation}
This phase term \textit{would have} appeared if we had performed a first-order WKB approximation in $x$ (see also \cite{Dingle_1975,mathews_wkb}). Under this definition, $\theta' = k$ so we can rewrite the Lagrangian as
\begin{equation}
    \mathcal{L}(\theta,W) = \frac{f(\theta')W^2}{4}.
    \label{truelag}
\end{equation}

\subsection{The Euler-Lagrange Equations}
sub
The Euler-Lagrange equations, derived from Hamilton's principle, are PDEs that relate the Lagrangian to its parameters \cite{symon_1980}. For any parameter of the Lagrangian $\psi$, the corresponding Euler-Lagrange equation is
\begin{equation}
 \frac{\partial \mathcal{L}}{\partial \psi}  - \frac{d}{dx}\frac{\partial \mathcal{L}}{\partial \psi'} = 0.
\end{equation}

In the present case, the Lagrangian parameters are the phase ($\theta$) and amplitude ($W$) of the transverse displacement at the OCC (see Eqn \ref{truelag}). Beginning with $W$, we can quickly see that the Lagrangian has explicit dependence on $W$, but no explicit dependence on $W'$. Thereby, the Euler-Lagrange equation simplifies to
\begin{equation}
    \frac{\partial \mathcal{L}}{\partial W} = 0,
    \label{WEL}
\end{equation}
Solution of this equation simply yields the dispersion relation of Eqn \ref{dispersion}, adding no new information to the problem.

As for $\theta$, we see that $\mathcal{L}$ has explicit dependence on $\theta'$ but no explicit dependence on $\theta$, so
\begin{equation}
    \frac{\partial \mathcal{L}}{\partial \theta} = 0.
\end{equation}
The Euler-Lagrange equation thereby simplifies to
\begin{equation}
    \frac{d}{dx}\frac{\partial \mathcal{L}}{\partial \theta'} = \frac{d}{dx}\frac{\partial \mathcal{L}}{\partial k} =0,
    \label{kEL}
\end{equation}
as $\theta' = k$.

Next is the Euler-Lagrange equation in $k$, Eqn \ref{kEL}, which when combined with Eqn \ref{truelag} yields:
\begin{align}
    \frac{d}{dx} \frac{\partial \mathcal{L}}{\partial k} &= \frac{W^2}{4} \frac{d}{dx} \frac{\partial f}{\partial k} \\
    &= \frac{d}{dx} \frac{W^2}{4}f_k\\
    & = 0.
\end{align}
Integrating both sides gives
\begin{equation}
    W = Cf_k^{-1/2}
    \label{Winf}
\end{equation}
where $C$ is some arbitrary constant.  This equation, which gives the displacement amplitude as a function of $x$, is called the transport equation. The derivative of $f$ with respect to $k$, $f_k$ is
\begin{equation}
    f_k = -2\rho\omega^2 \frac{\tanh{kh} + kh\text{sech}^2\;kh}{k^2\tanh^2\;kh}.
\end{equation}

\subsection{Solving for Velocity and Pressure}

To find $C$ in Eqn \ref{Winf}, we use the known displacement at the oval window (related to the known pressure at the oval window used in the main text). Denote the average displacement at the stapes as $\delta_{st}$. This is related to the $x$-direction motion by averaging $u$ at $x=0$ over the cross-section.

One can solve for the velocity potential in terms of $\delta_{st}$. In short, $C$ is eliminated by finding a formula for $\delta_{st}$ and taking the quotient between it and $\phi$. It is
\begin{equation}
    \phi = -\delta_{st}\frac{\omega h}{\cosh{kh}\tanh{k_0h}}\sqrt{
    \frac{\tanh{k_0h} + k_0h\,\text{sech}^2{k_0h}}{\tanh{kh} + kh\,\text{sech}^2{kh}}} \cosh{[k(z-h)]} e^{-j\theta }.
\end{equation}

Using the relationship between displacement and pressure, the same equation can be written for pressure in terms of average pressure at the oval window $P_{OW}$ to arrive at
\begin{equation}
    p(x,z) = P_{OW}\frac{k_0 h}{\cosh{kh}\tanh{k_0 h}} \sqrt{
    \frac{\tanh{k_0h} + k_0h\,\text{sech}^2{k_0h}}{\tanh{kh} + kh\,\text{sech}^2{kh}}} \cosh{[k(z-h)]} e^{-j\int_0^x k(\xi)\;d\xi},
\end{equation}
precisely Eqn \ref{duifhuisKING}.