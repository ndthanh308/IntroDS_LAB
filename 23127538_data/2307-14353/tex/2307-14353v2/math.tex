\section{The WKB Approximation}

In this section, I will present the mathematical underpinnings of the WKB approximation. These abstract concepts will be applied to cochlear mechanics models specifically in the following sections. 

Consider a homogeneous linear $n^{\text{th}}$ order ordinary differential equation (ODE) of the form
\begin{equation}
    \epsilon\frac{d^n y}{dx} + a_{n-1}(x)\frac{d^{n-1} y}{dx^{n-1}} + \ldots + a_1(x)\frac{dy}{dx} + a_0(x)y = 0,
    \label{WKBODE}
\end{equation}
where the coefficient functions $a_i$, $i=1,2,\ldots,n-1$ are $n$-times continuously differentiable functions on some interval $I\subset \mathbb{R}$, and $\epsilon\in\mathbb{R}$ is presumed to be small relative to the magnitudes of the other coefficient functions. The coefficient functions may be complex-valued.

Consider an ansatz for a solution to Eqn \ref{WKBODE} as the exponential of a formal power series in $\delta\in\mathbb{R}$,
\begin{equation}
    y(x) = \exp{\bigg[\frac{1}{\delta}\sum_{m=0}^{\infty} \delta^{m}C_m(x)}\bigg],
    \label{WKB_withdelta}
\end{equation}
where $C_m$, $m=0,1,2,\ldots$ are $n$-times continuously differentiable functions on $I$ and it is assumed that the series can be differentiated term-wise \cite{Liouville_1837,green,Robnik_Romanovski_2000,Dingle_1975,Winitzki_2005}. So that the series converges\footnote{In reality, the series often \textit{diverges} and terms will begin to increase after a certain order of $m$. This limits the precision of the approximation, but will not come into play in this tutorial as I will never exceed $m=1$. Detailed discussion is presented in \textit{Winitzky, 2005} \cite{Winitzki_2005}.}, $\delta$ should be small and $C_m$ and their derivatives must fall off exponentially in magnitude across the real line with increasing $m$. That is,
\begin{equation}
    \delta^{m}C_{m+1}(x) \ll \delta^{m-1}C_m(x),\;\;m=0,1,2,\ldots.
\end{equation}


The ansatz, when plugged into Eqn \ref{WKBODE}, yields a system of infinitely many ODEs -- one for each $S_m$. The \textit{$M^{\text{th}}$-order WKB approximation} is made by truncating this series up to the $M^{\text{th}}$ term. This is valid so long as all terms at indices higher than $M$ are much smaller than 1 on $I$. That is,
\begin{equation}
    \delta^M S_{M+1}(x) \ll 1.
\end{equation}

In this text, the zeroth- and first-order WKB approximation for two differential equations -- the wave equation with variable wavenumber and the Webster horn equation -- are presented. It is common that the first-order WKB approximation is simply called ``the WKB approximate solution," \cite{mathews_wkb} and unless otherwise specified this custom is followed in the present text. 