\documentclass{article}
\usepackage[utf8]{inputenc}
\usepackage{geometry}
\usepackage{amsmath}
\usepackage{titling}
\usepackage{hyperref}
\usepackage{csquotes}
\usepackage[sorting=none, style=nature]{biblatex}
\usepackage{algpseudocode}
\usepackage{algorithm}
\usepackage{graphicx}
\addbibresource{mybib.bib}

\usepackage{newtxtext,newtxmath}

\title{Foundations of WKB Models for Cochlear Mechanics in 1- and 2-D}
\author{Brian Frost}
\date{Summer 2023}

\pretitle{\begin{center}\Large}
\posttitle{\end{center}}
\preauthor{\begin{center}\small}
\postauthor{\end{center}}
\predate{\begin{center}\footnotesize}
\postdate{\end{center}}
\setlength{\droptitle}{-40pt}

\begin{document}

\maketitle

\begin{abstract}
    Wentzel-Kramers-Brillouin (WKB) models are widely used to explore the mechanics of the cochlea. As opposed to finite element strategies, WKB models are described by interpretable closed-form equations, and can be implemented with relative ease. As a result, these models have maintained relevance in cochlear mechanics for half of a century. Over this time, WKB models have been used to study a variety of phenomena including the limits of frequency tuning, active displacement amplification within the organ of Corti, potential feedforward mechanisms in the cochlea, and otoacoustic emissions. Despite this ubiquity, it is challenging to find rigorous exposition of the formulation, derivation and implementation of WKB models in literature. In this tutorial, I discuss the foundations of the WKB model for cochlear macromechanics in 1-D and 2-D. This  includes mathematical background, rigorous derivations and details of the algorithms used to implement the model in software.  
\end{abstract}

\tableofcontents

\section{Introduction}

\par{Most models of the cochlea, especially those that are not of the finite element type, operate on certain assumptions -- the scala walls are rigid, the fluid is incompressible, etc. These are used to simplify the Navier-Stokes boundary value problem to one that can be solved analytically.}
\par{The Wentzel–Kramers–Brillouin (WKB) approximation, also known as the Liouville-Green (LG) approximation, is among the most popular such assumptions for models of cochlear macromechanics. Its robustness over a large frequency and spatial range, ease of implementation and interpretability make it an attractive complement to the finite element approach.}
\par{With the passage of time, foundations of the WKB model have disappeared from literature; the assumptions and derivations of specific equations, and the implications thereof, have become implicit. This efficiency is useful for experienced readers, but creates confusion for newer entrants to the field. In the case of WKB models, not only are these objects often missing in contemporary literature, but challenging to find in historic literature as well.}
\par{The WKB model's relevance, both historical and contemporary, owes it a foundational exposition. Fundamental understanding of the model can open the door to its adaptation to probe particular questions of interest, with knowledge of its strengths and limitations. As such questions continue to arise with the publication of new data, for example, obtained through optical coherence tomography, this is all the more relevant.}
\par{Lastly, the recent passing of Egbert de Boer and Charles Steele -- both pioneers in the field of WKB models for cochlear macromechanics -- suggests a timeliness of such a presentation.}

\subsection{Outline of the Document}
\par{The essence of this report is to present the fundamentals of 1-D and 2-D WKB modeling from an analytic perspective, covering derivations and details of implementation. I begin by describing the mathematics of the WKB method in the introduction. This is followed by its application to cochlear mechanics by discussing the three-dimensional boundary value problem (BVP) for the box model, from which 2-D WKB models are derived. Derivations of 1-D and 2-D models follow.}
\par{Two non-equivalent derivations of the 2-D WKB model equations are discussed in Secs \ref{sec:variation} and \ref{sec:series}. These methods are, in order:
\begin{enumerate}
    \item{The variational method of Steele and Taber: The authors begin by assuming a single-mode wave solution, having constant magnitude and wavenumber, for organ of Corti complex (OCC) velocity. They then appeal to Lagrangian mechanics to find spatially-dependent wavenumber and amplitude functions that satisfy the Hamilton principle \cite{steele_lagrange}.}
    \item{The series solution of Viergever: After transforming the coordinates of the problem, Viergever assumes a formal power series expansion in the wavenumber and solves for the first order term \cite{viergever_Book}.}

\end{enumerate}
}

\par{For readers who are interested only in implementation rather than derivations, Sec \ref{sec:k} details certain challenges of implementation. In particular, the 2-D WKB model includes an interesting road-block -- a transcendental dispersion relation that has no unique solution for the wavenumber $k$. This dispersion relation (often called ``the eikonal equation") must be numerically solved to determine the single value of $k$ used to compute pressure and velocity. I provide an analysis of this dispersion relation, detailing its behavior in the complex plane and its physically relevant solutions.}
\par{The appendices include the extensive computations involved in each derivation. Also included is a derivation of the more general integral equation for the 3-D box model, and its link to WKB models.}

\subsection{The WKB Assumption and Method}
\par{In this section, I will present the mathematical underpinnings of the WKB model. These more abstract concepts will be applied to cochlear models specifically in the following chapters.}
\par{Consider a homogeneous linear ODE of the form
\begin{equation}
    \epsilon\frac{d^n y}{dx} + a_{n-1}(x)\frac{d^{n-1} y}{dx^{n-1}} + \ldots + a_1(x)\frac{dy}{dx} + a_0(x)y = 0,
    \label{WKBODE}
\end{equation}
where $\epsilon$ is presumed to be small relative to the other coefficient functions.}
\par{Formally, the WKB approximation is based on the ansatz that the solution for $y$ of a linear ODE is of the form
\begin{equation}
    y(x) \sim \exp{\bigg[\frac{1}{\delta}\sum_{n=0}^{\infty} \delta^{n}S_n(x)}\bigg],
    \label{WKB_withdelta}
\end{equation}
and then letting $\delta$ become small \cite{Robnik_Romanovski_2000,Fitzpatrick_2014}. It is assumed that the series can be differentiated term-wise.}
\par{ The ansatz, when plugged into the original ODE, yields a system of infinitely many ODEs -- one for each $S_n$. One can solve these ODEs up to any order and achieve an approximation by keeping a certain number of terms in the series. This method is only valid if the summands in the exponential monotonically decrease -- and quickly -- as $\delta$ approaches 0. That is
\begin{equation}
    \delta^{n}S_{n+1}(x) \ll \delta^{n-1}S_n(x),\;\;\;\delta\rightarrow 0,\;n=0,1,2,\ldots.
\end{equation}
}
\par{Truncating at the $N^{\text{th}}$ term, future terms must be much smaller than 1 in the approximation interval, lest they lead to multiplication by a large factor. That is,
\begin{equation}
    \delta^N S_{N+1}(x) \ll 1,\;\;\;\delta\rightarrow 0.
\end{equation}
}
\subsubsection{Wave-Like Equations}
\par{Of particular interest in cochlear mechanics are wave-like ODEs of the type
\begin{equation}
    \frac{d^2 y}{dx^2} = -k^2(x) y,
\end{equation}
In the case of constant $k$, this is the spatial ODE that appears in the solution of the 1-dimensional wave equation with separable variables, having complex exponentials as solutions.}
\par{Comparison with Eqn \ref{WKBODE} shows that this equation has $\epsilon=1$. It is customary to let $\delta$ approach $\epsilon$, the (relatively) small coefficient, which reduces the ansatz to

\begin{equation}
    y(x) \sim \exp{\bigg[\sum_{n=0}^{\infty} S_n(x)}\bigg].
    \label{WKB}
\end{equation}
}
\subsubsection{A Related Approximation}
\par{It is often stated that the underlying assumption of the WKB method is that the system's parameters do not vary quickly relative to the parameter values themselves -- this is not generally true, but we will show that in the cochlear case the WKB assumption is related (but not precisely equivalent) to: 
\begin{equation}
    \Big|\frac{dk}{dx}\Big| \ll |k(x)|^2,
    \label{wkbassume}
\end{equation}
where $k$ is the wavenumber of the traveling wave\footnote{Historically, the term WKB has referred to the first truncation of the series in Eqn \ref{WKB_withdelta}, in combination with this assumption \cite{mathews_wkb,Dingle_1975,Fitzpatrick_2014}. The series solution is more general and rigorous \cite{Robnik_Romanovski_2000,Fitzpatrick_2014}.}.}

\par{It is interesting to consider asymptotic behavior of the cochlea's traveling wave in light of this inequality. At positions far basal to the best place, the response is said to be in the \textit{long wave} region. Here, the wavelength is large (wavenumber is small) and varies slowly in space. That is, we know that the left-hand term in the above inequality is very small. }

\par{At positions near the best place, the wavelength becomes smaller (wavenumber becomes larger) and changes more rapidly in space. This is known as the \textit{short wave} region. Here, the right-hand term in the inequality is very large. In balance, this assumption may be satisfied in both regions.}

\subsection{Limitations and Strengths of WKB Models}

\par{The WKB assumption has been used to develop 1-, 2- and 3-D cochlear models. While 1-D models can provide some insight due to their simplicity, they do not describe the fluid mechanics in the scalae. On the other hand, 3-D models are both the most physical and the most complicated. The middle ground, 2-D models which include the transverse and longitudinal dimensions of the cochlea, provide interesting insight into motion at the OCC and mechanics of the fluids within the scalae. For example, the concept of \textit{pressure focusing}, or the ratio between the average pressure in the scalae and the pressure at the OCC, is of interest \cite{duifhuis_moh,sisto_2021,sisto_2023}.}

\par{The WKB class of models is \textit{macromechanical}, accounting for motion of the organ of Corti complex (OCC, including the organ of Corti, basilar membrane and tectorial membrane) as one lumped object. In recent years, intra-organ of Corti measurements made with optical coherence tomography have shown that the cochlea's \textit{micromechanics} are important for shaping the stereocilia displacement that leads to neural response. Micromechanical modeling has largely been approached with finite element models, which are attractive in that they can have arbitrarily many degrees of freedom. Even when restricted only to macromechanics, a more exact solution than WKB exists as an integral equation (see Appendix \ref{app:integral}). These limitations lead to the question: why would one use a WKB model?}
\par{WKB is attractive in that it yields an \textit{analytic} solution. In contrast with application of a finite element model, analytic solutions are more easily interpreted, and the effects of parameter variation can be tracked through formulae rather than testing such effects by extensive simulation. It is also the case that micro- and macromechanics are not independent objects, so the WKB model serves as an efficient and illuminating tool for testing the impact of micromechanical hypotheses on the system's macromechanics.}

\pagebreak

\section{The 3-D Box Model Boundary Value Problem}
\label{sec:3d}
\par{The problem is first presented in 3-D, and then the dimension is reduced to yield analogous 1-D and 2-D formulations.}
\par{The geometry of the 3-D model is shown in Fig \ref{fig:box} \textbf{A}. The cochlea is assumed to be a rectangular prism, having no coiling or wall curvature -- this is why it is called a ``box model." The problem is posed as a BVP of the Laplace equation in this box, phrased either in a) the scalar velocity potential, or b) the pressure.}

% Figure environment removed



\subsection{The Laplace Equation in Velocity Potential}
\par{Fluid velocity $\mathbf{v}$ in a volume satisfies the continuity equation, given by
\begin{equation}
    \frac{\partial\rho}{\partial t} + \nabla\cdot(\rho\mathbf{v})= 0,
\end{equation}
where $\rho$ is the fluid density. This says that within a differential volume, the change in fluid mass in the region is accompanied by an equal and opposite divergence of that fluid into/out of the region.}
\par{In an incompressible fluid, the mass of the fluid (and thereby $\rho$) in any region is constant, simplifying the equation to
\begin{equation}
\nabla\cdot\mathbf{v} = 0.
\label{cont}
\end{equation}
An irrotational field is also a conservative field. Under this irrotational assumption, the velocity field can be written as the gradient of some scalar field $\phi$. This \textit{velocity potential} thereby satisfies

\begin{equation}
    \nabla\phi = \mathbf{v}.
\end{equation}.
\par{
Taking the divergence of both sides and applying Eqn \ref{cont} yields the Laplace equation:

\begin{equation}
    \nabla^2\phi = \frac{\partial^2 \phi}{\partial x^2} + \frac{\partial^2 \phi}{\partial y^2} + \frac{\partial^2 \phi}{\partial z^2}= 0,
\end{equation}
where $x$, $y$ and $z$ are the longitudinal, radial and transverse directions, respectively.
}
\subsection{The Laplace Equation in Pressure}
\par{The Navier-Stokes equation in an \textit{inviscid, incompressible, linear, irrotational} fluid is
\begin{equation}
\rho\frac{\partial \mathbf{v}}{\partial t} + \nabla P =\rho\frac{\partial \nabla \phi}{\partial t} + \nabla P=0.
\end{equation}
}
\par{This gives the well-known equation 
\begin{equation}
    P = -\rho \dot{\phi},
\end{equation}
where the overhead indicates a partial time derivative.
}
\par{Taking the Laplacian of both sides and recalling that $\phi$ satisfies the Laplace equation, we arrive at a Laplace equation in $P$:
\begin{equation}
    \nabla^2 P=0.
\end{equation}
}
\subsection{Other Assumptions}
\par{Three other important assumptions are involved in the development of the box model:
\begin{itemize}
\item{\textbf{Symmetry of scalae: }The model contains two fluid chambers; Scala Tympani (ST) and Scala Vestibuli (SV); along with the OCC. Scala Media is not included because Reissner's Membrane is assumed to be mechanically irrelevant. The scalae are assumed to be geometrically identical, leading to a symmetry in the model. This allows us to model fluid only in SV.}

\item{\textbf{Linearity: }Assuming linear behavior, an input wave of the form $Ve^{j\omega t}$ produces a response at the same radian frequency $\omega$ at each location. The time dependence of pressures and velocities will always be present as a factor of $e^{j\omega t}$. It is customary to leave this time dependence as implicit, as its presence everywhere only enlarges expressions.}

\item{\textbf{Dominance of the forward traveling wave: }The solutions for the model equations will always contain both a forward traveling wave and a backward traveling wave, the latter representing the wave that has been reflected from the apex. It is reasonable to assume that the backward traveling wave has no impact on the response. The backward traveling wave plays a role in emissions, but for most purposes can  be ignored.}

\end{itemize}

\subsection{Boundary Conditions and Motion of the Organ of Corti Complex}
\label{sec:conditions}
\par{The boundary conditions operate on the following assumptions -- 1) fluid does not flow in the normal direction towards or out of the scalae at the walls $z=h, y=\pm b/2$, including at the helicotrema $x=L$, 2) the average pressure at  $x=0$ is some known pressure at the oval window $P_{OW}$, and 3) the OCC is mechanically described by a simple point-impedance relationship. }
\par{Fig \ref{fig:box} aids in interpreting the resultant boundary values. Writing the OCC's impedance as $Z_{OC}$, we can write the boundary conditions as:
\begin{align}
    &\frac{1}{hb}\int_{0}^{h} \int_{-b/2}^{b/2}P(0,y,z)\;dy dz = P_{OW},\;\;\\
    &\frac{\partial P}{\partial x}(L,y,z) = 0,\\
    &\frac{\partial P}{\partial y}(x,\pm b/2,z) = 0,\\
    &\frac{\partial P}{\partial z}(x,y,h) = 0,\\
    &P(x,y,0)  = Z_{OC}(x,y) \frac{\partial \phi}{\partial z} (x,y,0) = \dot{w}(x,y,0),
\end{align}
where $\dot{w}$ is the $z$ component of velocity at the OCC.}
\par{We look to reduce the dimension to 1- and 2-D problems where solutions are more easily found, having easily interpreted closed forms. This is in contrast with the more general integral solution presented in App \ref{app:integral}.}

\pagebreak

\section{The 1-D Model}
\label{sec:1d}
\par{Every quantity in a 1-D model is a function of only longitudinal space ($x$), and time. The problem is posed as a 1-D BVP of a wave-like equation in pressure.}
\subsection{Arriving at the Wave-Like Equation}
\par{We refer to the fluid pressure in each fluid chamber as $P(x,t)$, the fluid velocity in the $x$ direction as $\dot{u}(x,t)$, and the transverse velocity of the OCC as $\dot{w}(x,t)$. Say the BM width and scala height are $b$ and $h$, respectively, and the cross-sectional area is $A= bh$.}
\par{The OCC is assumed to span the entire radial width of the cochlea. However, in 1- and 2-D models, we do not consider the radial dimension, so geometrical features of the OCC cannot be implemented as they are in a 3-D treatment.}
\par{By conservation of mass, fluid displaced from a cross-section in the $x$ direction must be accompanied by an equal and opposite displacement of fluid in the $z$ direction. It is only the flexible OCC that can be responsible for $z$-direction fluid displacement, so the mass conservation equation is
\begin{equation}
    A\frac{\partial \dot{u}}{\partial x} = b\dot{w}.
    \label{masscons}
\end{equation}
}

\par{As for the forces present in the system, the linear Navier-Stokes equation for inviscid fluid flow is
\begin{equation}
    \frac{\partial P}{\partial x} - \rho \frac{\partial \dot{u}}{\partial t} = 0.
    \label{forcebalance}
\end{equation}
Under the linearity assumption, the time derivative simply yields a factor of $j\omega$.}
\par{We assume the OCC can be described by a simple impedance at each point in space: $Z_{OC}(x) = p(x)/\dot{w}(x)$ (or admittance $Y_{OC} = 1/Z_{OC}$), where $p$ is the total trans-membrane pressure. This pressure is the difference between the pressures in the two scalae, which is simply $p=2P$ due to the symmetry assumption. 
\par{The equation of the motion of the OCC is thereby $P = Z_{OC}\dot{w}/2$, or
\begin{equation}
    \dot{w} = 2Y_{OC}P.
    \label{impedance1D}
\end{equation}
}
\par{The goal is to combine these equations to obtain a single PDE in $P$. Differentiating Eqn \ref{forcebalance} in $x$ gives
\begin{equation}
    \frac{\partial^2 P}{\partial x^2} - \rho j \omega \frac{\partial \dot{u}}{\partial x} = 0.
    \label{withu}
\end{equation}
Combining this equation with Eqn \ref{masscons} (and using $A=bh$) gives
\begin{equation}
    \frac{\partial^2 P}{\partial x^2} - \rho j \omega h^{-1}\dot{w} = 0.
\end{equation}
}
\par{Using Eqn \ref{impedance1D} yields an equation in only $P$:
\begin{equation}
    \frac{\partial^2 P}{\partial x^2} - \frac{2\rho j \omega}{h Z_{OC}}P = 0.
\end{equation}
}
\par{This equation is similar to a 1-D wave equation, but with an $x$-dependent wavenumber. Using $p = 2P$, we can just as well write this equation in terms of the trans-membrane pressure:
\begin{equation}
    \frac{\partial^2 p}{\partial x^2} +k^2(x) p = 0,
    \label{1dwavelike}
\end{equation}
where
\begin{equation}
    k^2(x) = \frac{-2\rho j\omega}{hZ_{OC}(x)}.
    \label{k1d}
\end{equation}
The equation above is a dispersion relation, as it relates $k$ and $\omega$ through the OCC impedance. It allows us to solve for $k$ in terms of the model's parameters.}
}
\par{To solve the BVP, we consider the boundary conditions on $x$ simplified from those presented in the 3-D model above; at the apex, that is
\begin{equation}
    \frac{\partial p}{\partial x}(L,t) = 0.
    \label{1dapexBV}
\end{equation}
}
\par{At the base, letting the known velocity amplitude at the oval window be $V$, we have
\begin{equation}
    \dot{u}(0,t) = V e^{j\omega t}.
\end{equation}
Then, using Eqn \ref{withu} we can arrive at a boundary condition in $P$:
\begin{align}
    \begin{split}
    \frac{\partial P}{\partial x}(0,t) &= \rho \frac{\partial \dot{u}}{\partial t}(0,t) \\
    &=\rho V j\omega,
    \end{split}
\end{align}
or in terms of the trans-membrane pressure,
\begin{equation}
    \frac{\partial p}{\partial x}(0,t) = 2\rho V j\omega.
    \label{1dbaseBV}
\end{equation}
}
\par{The 1-D model BVP has been determined -- it consists of the wave-like PDE in Eqn \ref{1dwavelike}, and the boundary conditions in Eqns \ref{1dapexBV} and \ref{1dbaseBV}.
}
\par{The one-dimensional wave equation is a constant-coefficient homogeneous second order ODE. Its solutions are of the form $p = e^{\pm jkx}$, or single-mode waves in space. The hitch is that $k$ is itself a function of $x$, determined by Eqn \ref{k1d}. This is why what we have here is not actually a wave equation -- that would only be true if we assume constant $k$.}

\subsection{Assuming Constant $k$}

\par{Assuming constant $k$ is the most extreme of the WKB-style approximations, and does not require any terms be solved for in Eqn \ref{WKB}. I will refer to solution derived using constant $k$ as the ``zero-order WKB solution."}
\par{The solution to the wave equation with constant $k$ is

\begin{equation}
    p_0(x) = C_1e^{-jkx} + C_2 e^{jkx}. 
\end{equation}
Using boundary values, one can solve for $C_1$ and $C_2$ -- these computations can be found in Appendix \ref{app:0orderconsts}. We then ``remember" the $x$ dependence by substituting $k = k(x)$ after solution, arriving at
\begin{equation}
    p_0(x) = -\bigg[\frac{2\rho V \omega}{k_0}\bigg] \frac{e^{-jk(x)x} + e^{jk(x)x - 2jk_L L}}{1-e^{-2jk_L L}},
\end{equation}
where $k_0 = k(0)$ and $k_L = k(L)$.
}
\par{This equation can be simplified by realizing that the $+jkx$ term is the backwards traveling wave. As stated above, we assume this has no impact on the response. For this to be the case, we must have $e^{-jk_L L}\approx 0$. This simplifies both the numerator and the denominator, giving
\begin{equation}
    p_0(x) = -\frac{2\rho V\omega}{k_0} e^{-jk(x)x}.
    \label{1d0order}
\end{equation}
This expression is the zero-order approximation for pressure -- a wave traveling in the $+x$-direction. The $x$-dependence of $k$ appears only in the exponent.}

\subsection{Enter the WKB Approximation}

\par{Achieving a first-order approximation requires application of the WKB method, truncating after the first term in the series. In accordance with Eqn \ref{WKB}, we begin by writing our ansatz:
\begin{equation}
    p(x) = \exp\bigg[\sum_{n=0}^\infty S_n(x)\bigg].
    \label{pwkb}
\end{equation}
Plugging into the ODE in Eqn \ref{1dwavelike} gives
\begin{equation}
    \bigg(\bigg(\sum_{n=0}^\infty S'_n(x)\bigg)^2+\sum_{n=0}^\infty  S''_n(x)\bigg)\exp\bigg[\sum_{n=0}^\infty S_n(x)\bigg] = -k^2(x)\exp\bigg[\sum_{n=0}^\infty S_n(x)\bigg]
\end{equation}
which facilitates division by the exponential to achieve a system of infinitely many second-order ODEs:
\begin{equation}
    \bigg(\sum_{n=0}^\infty S'_n(x)\bigg)^2+\sum_{n=0}^\infty  S''_n(x) = -k^2(x).
\end{equation}
}
\par{The asymptotic assumption is that the terms and their derivatives decrease monotonically. Considering the most significant terms up to second order in $S_0$ and first order in $S_1$ gives
\begin{equation}
    S_0'^2(x) + 2 S_0'(x)S_1'(x) + S_0''(x) = -k^2(x).
    \label{firsteq}
\end{equation}
}
\par{By the asymptotic assumption, the leading term $S_0'^2$ is the most significant. Keeping only this term gives an ODE in only one dependent variable, allowing solution for $S_0$:
\begin{align}
    S_0'^2(x) &= -k^2(x),\\
    S_0'(x) &= \pm j k(x),\\
    S_0(x) &= \pm j \int_{0}^{x} k(\xi)\;d\xi + C,\;\;\;C\in\mathbb{R}.
\end{align}
}

\par{A first order approximation for $p$ is achieved by keeping only this term in the series of Eqn \ref{pwkb}: $p = \exp (S_0)$. The two solutions above for $S_0$ ($+$ and $-$) contribute two linearly independent solutions for $p$. Because the ODE is homogeneous, any superposition of these solutions is also a solution:
\begin{equation}
    p_1(x) = C_1 e^{-j\int_{0}^{x} k(\xi)\;d\xi} + C_2 e^{j\int_{0}^{x} k(\xi)\;d\xi},\;\;\;C_1,C_2\in\mathbb{R}.
    \label{Pfirst}
\end{equation}
Note that if we then set $k$ as constant, we would get \textit{precisely} the zero-order solution!}

\par{A second-order WKB approximation comes from plugging the solution for $S_0$ back in to Eqn \ref{firsteq}, which includes the higher order $S_1$ term. This gives
\begin{equation}
    -k^2(x) \pm 2jk(x)S_1'(x) \pm jk'(x) = -k^2(x).
\end{equation}
Simplification reveals that this is a separable differential equation with solution
\begin{equation}
    S_1(x) = \frac{-1}{2}\ln{k(x)} + C.
\end{equation}
Letting $p = \exp(S_0 + S_1)$ yields the second-order WKB approximation,
\begin{align}
\begin{split}
    p_2(x) &= C_1 e^{-j\int_{0}^{x} k(\xi)\;d\xi}e^{\frac{-1}{2}\ln{k(x)}} + C_2 e^{j\int_{0}^{x} k(\xi)\;d\xi}e^{\frac{-1}{2}\ln{k(x)}}\\
    &=\frac{C_1}{\sqrt{k}} e^{-j\int_{0}^{x} k(\xi)\;d\xi} + \frac{C_2}{\sqrt{k}} e^{j\int_{0}^{x} k(\xi)\;d\xi},\;\;\;C_1,C_2\in\mathbb{R}.
\end{split}
\end{align}
}

\par{These two equations reveal that a significant difference between the first- and second-order approximations is the amplitude-modulating factor of $k^{-1/2}$. If $k$ is constant, this term is simply absorbed into the other unsolved-for constants, again yielding the zero-order approximation.}

\subsection{Explicit Formulae for Pressure}
\par{In Appendix \ref{app:1dWKB}, constants $C_1$ and $C_2$ are found using boundary conditions. Applying again the forward traveling wave assumption gives the first- and second-order model equations for trans-membrane pressure:
\begin{equation}
    p_1(x) = -\bigg[\frac{2\rho V \omega}{k_0}\bigg] e^{-j\int_{0}^{x} k(\xi)\;d\xi}.
    \label{P1Dfirst}
\end{equation}
\begin{equation}
    p_2(x) = -\bigg[\frac{2\rho V \omega}{\sqrt{k_0 k(x)}}\bigg] e^{-j\int_{0}^{x} k(\xi)\;d\xi}.
    \label{P1D}
\end{equation}
Similar model equations for velocity can be found using the identity $\dot{w} = p/Z_{OC}$.}
\par{While one could continue solving for $S_n$ values to achieve arbitrary orders of WKB approximation, there are diminishing returns. It is likely that other second order effects (such as fluid viscosity) play more of a role in the response than additional WKB terms would.}

\par{Analysis of the model equations shows dependence on the wavenumber $k$, itself a function of model parameters, $x$ and $\omega$. Implementation requires solving for $k$ using the \textit{dispersion relation} from Eqn \ref{k1d}. The WKB solution anatomy can be thought of as consisting of two parts: 1) the model equations, and 2) the dispersion relation.}
\par{Solution for $k$ through the dispersion relation includes an ambiguity, as the square function is not invertible. In the 1-D case, it is not difficult to decide the proper root -- $k$ should have a positive real part so that traveling wave is moving forwards.}

\subsection{Comparing the Three Approximations}
\par{Fig \ref{fig:wkb1d} shows velocity responses to a 3 kHz stimulus as a function of $x$ using the zero-, first- and second-order WKB model. Parameters of the model were taken from Steele and Taber \cite{steele_lagrange}. It is clear that even the zero-order model can predict qualities of the cochlear traveling wave: a peak, steep falloff and characteristic accumulation of phase.}

% Figure environment removed

\par{``Upgrading" to first-order increases the peak magnitude, pushes the best place towards the apex and slows phase accumulation. The amplitude falloff is also less rapid. Further upgrade to a second-order approximation makes little qualitative difference, but does reduce the magnitude in the peak region.}
\par{These simulations are consistent with a rough intuition from the WKB method and model equations. As the series is asymptotically decreasing, the $S_1$ component should have less effect on the solution curve than the $S_0$ component. Moreover, the second-order approximation differs from the first-order one only by an amplitude factor, indicating that first- and second-order phase responses should be similar. (As $k$ is complex, this will not be \textit{exactly} true.)}

\subsection{The Slowly-Varying $k$ Approximation}
\par{The term WKB is often used to refer to the idea that the wavenumber is not quickly varying in space relative to its own magnitude (see Eqn \ref{wkbassume}). However, the derivation above for first- and second-order WKB approximations never explicitly made this assumption. Where is the relationship between these two ideas?}

\par{For the WKB method to be valid, the terms $S_n$ in the series must decrease monotonically. In particular $|S_1|\ll |S_0|$, or
\begin{equation}
    \bigg|-\frac{1}{2}\ln{k}\bigg| \ll \bigg|j\int_0^x k(\xi)\;d\xi\bigg|.
\end{equation}
}
\par{The first expression can also be written as an integral from $0$ to $x$, and pulling out the constant-modulus factors gives
\begin{equation}
    \frac{1}{2}\bigg|\int_0^x \frac{k'(\xi)}{k(\xi)}\;d\xi\bigg| \ll \bigg|\int_0^x k(\xi)\;d\xi\bigg|.
\end{equation}
}
\par{This relationship is satisfied if $k$ satisfies Eqn \ref{wkbassume}. That is, the assumption of slow-varying $k$ implies that the WKB assumption is reasonable.}

\pagebreak

\section{The 2-D Model -- The Variational Approach}
\label{sec:variation}
\par{The geometry of the 2-D box model is shown in Fig \ref{fig:box} \textbf{B}. The 3-D BVP in Section \ref{sec:3d} is reduced to a 2-D BVP by assuming the $y$ dependence is negligible.}
\par{The Laplace equations simplify to
\begin{align}
    \nabla^2 \phi = \frac{\partial^2 \phi}{\partial x^2} &+ \frac{\partial^2 \phi}{\partial z^2} = 0\\
    \nabla^2 P =  \frac{\partial^2 P}{\partial x^2} &+ \frac{\partial^2 P}{\partial z^2} = 0,
\end{align}
and the boundary conditions are identical, save we can ignore those at the radial walls.}

\par{In this section, I will describe the method by which Steele and Taber found an approximate solution to this problem \cite{steele_lagrange}. If the reader does not wish to read the derivation, the resultant model equations and dispersion relation can be found in Eqns \ref{kEqn} and \ref{duifhuisKING}.}

\subsection{Overview of the Variational Method}
\par{Steele and Taber solve this BVP using a method based on Lagrangian mechanics that is distinct from the WKB method described above \cite{steele_lagrange}. Their method, motivated by Whitman, is called the \textit{variational method} \cite{whitman_waves}. The procedure is as follows:
\begin{enumerate}
    \item{Solve the BVP under the assumption that all parameters are constant (arriving at the zero-order system).}
    \item{Find the Lagrangian of this system.}
    \item{Write the Euler-Lagrange equations for the system in its parameters previously assumed to be constant.}
    \item{Solve these ODEs to find a non-constant form of these parameters, yielding a better approximation.}
\end{enumerate}
}
\par{The variational method is not the WKB method, but it does employ the same assumption -- namely that the variations in $k$ in $x$ are small relative to $k$ itself.}

\subsubsection{Solving Laplace's Equation with Constant Parameters}

\par{Assuming separable variables, $\phi$ can be written as a product of one function of only $x$ and one function of only $z$ (keeping time-dependence implicit):
\begin{equation}
    \phi(x,z) = X(x)Z(z).
\end{equation}
}

\par{The solution to this PDE is a superposition of products of basis functions. The solutions are of the forms
\begin{align}
    \phi_{\alpha} &= (A\cosh{\alpha x} + B\cosh{\alpha x})(Ce^{j\alpha z} + De^{-j\alpha z}),\\
    \phi_{\beta} &= (Ae^{j\beta x} + Be^{-j\beta x})(C\cosh{\beta z} + D\sinh{\beta z}).
    \label{general2dsols}
\end{align}
}
\par{We begin by assuming that the displacement at the basilar membrane is a wave \textit{with only one mode}, and travels only in the base-to-apex direction. That is, transverse displacement $w$ at $z=0$ is 
\begin{equation}
w(x,0) = We^{-jkx},
\end{equation}
where $W$ is the displacement amplitude and $k$ is the wavenumber. $W$ and $k$ are assumed to vary slowly in $x$ relative to a wavelength. For the zero-order solution, we assume $W$ and $k$ are \textit{constant}}
\par{In terms of velocity potential, the equation above can be written as
\begin{align}
    \dot{w}(x,0,t) &= j\omega We^{-jk} \\
    & = \frac{\partial \phi}{\partial z}(x,0)\\
    & = Z'(0)X(x).
\end{align}
This gives a boundary condition for $Z$ at the BM.}
\par{The assumptions of constant $k$ and $W$ give that $X(x)$ should take the form of a single-mode wave with wavenumber $k$. Observing the general solutions in Eqns \ref{general2dsols}, this corresponds to the $\beta=k$:
\begin{equation}
    \phi(x,z) = (A\cosh{kz} + B\cosh{kz})e^{-jkx},
    \label{arbvelpot}
\end{equation}
Note that the $\exp{+jkx}$ term is not present in the solution due to the forward traveling wave assumption.
}

\par{In Appendix \ref{app:velpotconsts}, I solve for the constants in this equation above using boundary conditions. This gives a formula for $\phi$ free of arbitrary parameters,

\begin{equation}
    \phi(x,z)= \frac{-j\omega W}{k\sinh{kh}}\cosh{[k(z-h)]}e^{-jk}.
    \label{phi}
\end{equation}
 This is the 2-D analogue of the zero-order solution in the 1-D model, as it is arrived at by assuming constant $k$.
}

\subsubsection{Impedance and Effective Height}

\par{An interesting quantity to probe is the impedance at the OCC, defined as the ratio between the pressure and the velocity at $z=0$. Recalling that $P/2 = -\rho \dot{\phi}$, we can write
\begin{align}
    Z_{OC}(x) &= \frac{P(x,0)}{\dot{w}(x,0)}\\
    &=-\frac{2\rho k^{-1}\omega^2 W\coth{kh}e^{-jk}}{j\omega We^{-jk}}\\
    &=-2j\omega \frac{\rho}{k \tanh{kh}}.
\end{align}
}
\par{This looks like a mass, due to the leading $j\omega$. We define the \textit{effective height}, $h_{e}$, as
\begin{equation}
    h_{e}(k) = \frac{1}{k\tanh{kh}},\;\;\;Z_{OC} = -2j\omega \rho h_{e}.
    \label{eqn::heff}
\end{equation}
This is called the effective height because it is precisely the impedance of an $h_{e}$-tall column of fluid.
}
\par{Rearrangement of the equations above yield the \textit{dispersion relation}:
    \begin{equation}
        k\tanh{kh} = -2j\omega\rho Y_{OC}.
        \label{kEqn}
    \end{equation}    
Paired with the model equation (Eqn \ref{phi}), this describes the zero-order 2-D WKB model.}
\par{\textbf{A note on terminology: }This equation is often called ``the eikonal equation" in literature, but this language is imprecise. This is because eikonal equations are a class of differential equations that appear in derivations of the model equations, and this algebraic expression is not such an equation. Instead, I will simply use the term \textit{dispersion relation}, as it is more descriptive and precise.}

\subsection{Computing the Zero-Order Lagrangian}
\par{The variational method works by starting with the zero-order Lagrangian, and then finding the conditions for $k$ and $W$ under which Hamilton's principal is satisfied to first order. To find the Lagrangian density we must find the potential and kinetic energy of the system.}
\par{We write the impedance at the OCC as a standard linear point-impedance, having a mass, stiffness and resistance:
$Z_{OC} = \frac{P(x,0)}{\dot{w}(x,0)} = j\omega M + R + \frac{S}{j\omega}.$ In Appendix \ref{app:lagrangian}, this impedance is used to to compute the time-averaged potential and kinetic energy densities in the OCC ($V$ and $T_{OC}$), and the kinetic energy density in the fluid ($T_f$):
\begin{equation}
    V = \frac{KW^2}{4}, \;\;\; T_{OC} = \frac{M\omega^2W^2}{4},\;\;\; T_f = \frac{1}{2}h_{e}(k)\rho \omega^2 W^2.
    \label{energyeqns}
\end{equation}
}

\par{The total kinetic energy density is the sum of the fluid and membrane contributions. This lets us write the Lagrangian:
\begin{equation}
    \mathcal{L}(k,W) = T-V = \frac{f(k) W^2}{4},
\end{equation}
where
\begin{equation}
    f(k) = 2h_{e}(k)\rho\omega + M\omega^2 - S.
\end{equation}
}
\par{The last two summands in the above equation resemble the impedance $Z_{OC}$, but missing resistance. As such, we will define an undamped OCC impedance by $Z_u = j\omega M + \frac{S}{j\omega}$, allowing
us to write the function $f$ above as 
\begin{equation}
    f(k) = 2\rho \omega^2 h_{e}(k) - j\omega Z_u.
\end{equation}
}
\subsection{Assumptions Prior to Solution}
\par{We run into a problem in the variational method -- application of the Euler-Lagrange equations requires a system in which no frictional forces are present \cite{symon_1980}. We can't completely ignore the resistance of the OCC, but we will assume for this portion of the derivation that \textit{the effect of resistance is negligible}. We will then ``throw it back in" later.}
\par{Our second assumption, guided by the first- and second-order 1-D solutions, is to replace $k$ with
\begin{equation}
    \theta = \int_{0}^x k(\xi)\;d\xi.
    \label{thetadef}
\end{equation}
This phase term \textit{would have} appeared if we had performed a first-order WKB approximation in $x$ (see also \cite{Dingle_1975,mathews_wkb}). This includes $x$ dependence in the phase in a more realistic manner. Under this definition, $\theta' = k$ so we can rewrite the Lagrangian as
\begin{equation}
    \mathcal{L}(\theta,W) = \frac{f(\theta')W}{4}.
    \label{truelag}
\end{equation}
}

\subsection{The Euler-Lagrange Equations}

\par{The Euler-Lagrange equations, derived from Hamilton's principle, are PDEs that relate the Lagrangian to its parameters. For any parameter of the Lagrangian $\psi$, the corresponding Euler-Lagrange equation is
\begin{equation}
 \frac{\partial \mathcal{L}}{\partial \psi}  - \frac{d}{dx}\frac{\partial \mathcal{L}}{\partial \psi'} = 0.
\end{equation}}

\par{In the present case, the Lagrangian parameters are the phase ($\theta$) and amplitude ($W$) of the transverse displacement at the OCC (see Eqn \ref{truelag}). Beginning with $\theta$, we see that $\mathcal{L}$ has no explicit dependence on $\theta$, so
\begin{equation}
    \frac{\partial \mathcal{L}}{\partial \theta} = 0.
\end{equation}
The Euler-Lagrange equation thereby simplifies to
\begin{equation}
    \frac{d}{dx}\frac{\partial \mathcal{L}}{\partial \theta'} = \frac{d}{dx}\frac{\partial \mathcal{L}}{\partial k} =0,
    \label{kEL}
\end{equation}
as $\theta' = k$.
}
\par{As for $W$, we can quickly see that the Lagrangian has no explicit dependence on $W'$. Thereby, the Euler-Lagrange equation simplifies to
\begin{equation}
    \frac{\partial \mathcal{L}}{\partial W} = 0.
    \label{WEL}
\end{equation}
}

\subsection{Solving the Euler-Lagrange Equations}

\par{The simpler of the two equations is Eqn \ref{WEL}, which can be reduced:

\begin{align}
    \frac{\partial \mathcal{L}}{\partial W} & = \frac{\partial}{\partial W} \frac{f(k) W^2}{4}\\
    & = \frac{W}{2} f(k,\omega)\\
    &=0,
\end{align}
implying 
\begin{equation}
    f(k) = 0.
\end{equation}
}
\par{Plugging in the formulae for $f$ and $h_{e}$ gives 
\begin{equation}
    \frac{2\rho\omega^2}{k\tanh{kh}} = j\omega Z_u.
\end{equation}
Calling the undamped admittance $Y_u = 1/Z_u$, we get a dispersion relation
\begin{equation}
    k\tanh{kh} = -2j\rho\omega Y_u.
    \label{ktanhkh}
\end{equation}
}

\par{This is \textit{nearly} the same as the dispersion relation in Eqn \ref{kEqn}, except the $Z_{OC}$ in the first equation includes damping. After ``throwing back in" the effect of resistance, the relation above is identical to Eqn \ref{kEqn}. From this point forward, we will assume resistance is present so that 

\begin{equation}
    f(k) = 2\rho\omega^2 h_{e} - Z_{OC}.
    \label{truef}
\end{equation}
}

\par{Next is the Euler-Lagrange equation in $k$, Eqn \ref{kEL}:
\begin{align}
    \frac{d}{dx} \frac{\partial \mathcal{L}}{\partial k} &= \frac{W^2}{4} \frac{d}{dx} \frac{\partial f}{\partial k} \\
    &= \frac{d}{dx} \frac{W^2}{4}f_k\\
    & = 0.
\end{align}
Integrating both sides gives
\begin{equation}
    W = Cf_k^{-1/2}
    \label{Winf}
\end{equation}
where $C$ is some arbitrary constant.  This equation, which gives the displacement amplitude as a function of $x$, is called the transport equation. The derivative of $f$ with respect to $k$, $f_k$ (derived in Appendix \ref{app:fk}) is
\begin{equation}
    f_k = -2\rho\omega^2 \frac{\tanh{kh} + kh\text{sech}^2\;kh}{k^2\tanh^2\;kh}.
\end{equation}
}

\subsection{Finding Velocity and Pressure}
\par{To find $C$ in Eqn \ref{Winf}, we use the known displacement at the oval window. Denote the average displacement at the stapes as $\delta_{st}$. This is related to the $x$-direction motion by averaging $u$ at $x=0$ over the cross-section.}
\par{In Appendix \ref{app:dispconst}, we solve for the velocity potential in terms of $\delta_{st}$. In short, $C$ is eliminated by finding a formula for $\delta_{st}$ and taking the quotient between it and $\phi$. It is
\begin{equation}
    \phi = -\delta_{st}\frac{\omega h}{\cosh{kh}\tanh{k_0h}}\sqrt{
    \frac{\tanh{k_0h} + k_0h\,\text{sech}^2{k_0h}}{\tanh{kh} + kh\,\text{sech}^2{kh}}} \cosh{[k(z-h)]} e^{-j\theta }.
\end{equation}}
\par{With $C$ eliminated, $W$ can be found by Eqn \ref{Winf}. The form of $\phi$ gives us the model equations for $\phi_x = \dot{u}(x,z,t)$ and $\phi_z = \dot{w}(x,z,t)$, as well as the pressure $P(x,z,t) =-2\rho \dot{\phi}$. These quantities are presented below, using $\Theta = \omega t - \theta$ to make explicit the time-dependence for the sake of completeness:
\begin{equation}
    \dot{u}(x,z,t) = \delta_{st}\frac{\omega kh}{\cosh{kh}\tanh{k_0h}}\sqrt{
    \frac{\tanh{k_0h} + k_0h\,\text{sech}^2{k_0h}}{\tanh{kh} + kh\,\text{sech}^2{kh}}} \cosh{[k(z-h)]} e^{j\Theta},
\end{equation}

\begin{equation}
    \dot{w}(x,z,t) = -\delta_{st}\frac{\omega kh}{\cosh{kh}\tanh{k_0h}}\sqrt{
    \frac{\tanh{k_0h} + k_0h\,\text{sech}^2{k_0h}}{\tanh{kh} + kh\,\text{sech}^2{kh}}} \sinh{[k(z-h)]} e^{j\Theta},
\end{equation}

\begin{equation}
    P(x,y,t) = \delta_{st}\frac{2\rho\omega^2 h}{\cosh{kh}\tanh{k_0h}}\sqrt{
    \frac{\tanh{k_0h} + k_0h\,\text{sech}^2{k_0h}}{\tanh{kh} + kh\,\text{sech}^2{kh}}} \cosh{[k(z-h)]} e^{j\Theta}.
    \label{pressureDisp}
\end{equation}
}
\par{The expression for pressure can also be written in terms of a constant pressure magnitude at the oval window, $P_{OW}$, as is written by Duifhuis \cite{duifhuis_moh}:
\begin{equation}
    P(x,z,t) = P_{OW}\frac{k_0 h}{\cosh{kh}\tanh{k_0 h}} \sqrt{
    \frac{\tanh{k_0h} + k_0h\,\text{sech}^2{k_0h}}{\tanh{kh} + kh\,\text{sech}^2{kh}}} \cosh{[k(z-h)]} e^{j\Theta }.
    \label{duifhuisKING}
\end{equation}
I show that Eqns \ref{pressureDisp} and \ref{duifhuisKING} are equivalent in Appendix \ref{app:duifhuis}.}

\par{Recall the anatomy of the WKB solution, containing a dispersion relation and model equations. With the above equations, one could solve for velocity or pressure only if they could also solve for $k$. Solving the dispersion relation for $k$ is nontrivial, and will be discussed in Sec \ref{sec:k}.}

\subsection{The Long-Wave Limit and Pressure Focusing}
\par{While the equations above may be messy, considering limit behavior in $k$ can give simplified expressions that offer some insight.} \par{For example, consider the long-wave region where $k\rightarrow 0$. In this limit, $\tanh x \rightarrow x$ and $\cosh{x} \rightarrow 1$. This gives
\begin{equation}
    P_{lw}(x,z) = P_{OW}\sqrt{\frac{k_0}{k(x)}}e^{-j\theta}.
\end{equation}
}
\par{This result, which has no $z$-dependence, is similar to the second-order pressure equation from the one-dimensional approximation in Eqn \ref{P1D}.}
\par{Another aspect to consider is that of \textit{pressure focusing}. This is the relationship between the average pressure in an $x$-cross-section of the scala and the pressure at the OCC in that same cross-section. Calculating first the average pressure, we have
\begin{align}
    \bar{P}(x,t) &= \frac{1}{h}\int_0^h P(x,z,t)\;dz \\
    & = P_{OW}\frac{k_0 h}{\cosh{kh}\tanh{k_0 h}} \sqrt{
    \frac{\tanh{k_0h} + k_0h\,\text{sech}^2{k_0h}}{\tanh{kh} + kh\,\text{sech}^2{kh}}}  e^{j\Theta }\frac{1}{h} \int_0^h \cosh{[k(z-h)]}\;dz\\
    &= P_{OW}\frac{k_0 h}{\cosh{kh}\tanh{k_0 h}} \sqrt{
    \frac{\tanh{k_0h} + k_0h\,\text{sech}^2{k_0h}}{\tanh{kh} + kh\,\text{sech}^2{kh}}}  e^{j\Theta }\frac{\sinh{kh}}{kh}.
\end{align}
On the other hand, the pressure at $z=0$ is 

\begin{equation}
    P(x,0,t) = P_{OW}\frac{k_0 h}{\cosh{kh}\tanh{k_0 h}} \sqrt{
    \frac{\tanh{k_0h} + k_0h\,\text{sech}^2{k_0h}}{\tanh{kh} + kh\,\text{sech}^2{kh}}} \cosh{kh} e^{j\Theta }.
\end{equation}
Dividing gives the pressure focusing factor $\alpha(x)$, given by

\begin{equation}
    \alpha(x) = \frac{P(x,0,t)}{\bar{P}(x,t)} = \frac{h k(x)}{\tanh{k(x) h}}.
    \label{alpha}
\end{equation}
}
\par{In the long-wave region, this object approaches unity. This corresponds to the long-wave result above wherein there was no $z$-variation in pressure. In the short-wave region, the $\tanh$ term approaches unity. This means that $\alpha \rightarrow hk$.}

\subsection{A Recap of the Variational Method}
\par{The variational method relies on the WKB assumption to compute the features of the response with constant $k$. The Lagrangian is found, but to employ principles of Lagrangian mechanics, the system is assumed to be lossless (i.e. $R=0$).}
\par{The Euler-Lagrange equations give an approximation of $k$ as a function of $x$. Plugging this form in to the constant-$k$ solution and then inserting resistance into the impedance gives the model equations.}
\par{Can we trust this handling of the resistance? One could argue that the method is reasonable as realistic values of resistance are relatively small, but this argument is insufficient when considering active models where the presence of \textit{negative} resistance is crucial. An empirical but unsatisfying argument is that the model's predictions are quite good anyway.}
\par{In the next chapter, I will show a method that also uses the WKB approximation, but does not assume the absence of resistance. Surprisingly, it yields precisely the same model equations as the variational method presented above.}

\pagebreak

\section{The 2-D Model -- A Series Solution Approach}
\label{sec:series}
\begin{displayquote}
It should be noted that the derivation of the basic formula [...] is based on consideration of energy conservation. The general experience has been that the method is equally well applicable to the case of slightly lossy systems. It is somewhat surprising that, without great problems, it can also be applied to active systems in which energy is supplied by the BM [...]. This suggests that the central formulae [...] may have a deeper lying root.
-- de Boer, 1980 \cite{deBoer_PhysicsReports}.
\end{displayquote}

\par{The following approach is followed by Viergever in his 1980 book \textit{Mechanics of the Inner Ear: A Mathematical Approach} \cite{viergever_Book}. It relies on a transformation of the coordinates of the pressure BVP, and subsequent application of a WKB-adjacent ansatz.}
\par{This method is distinct from that presented above in that it does not rely on energy principles, so there is no issue in including resistance, positive or negative, in the resulting model. Instead, it follows the approach of Keller's modeling of surface waves of non-uniform depth \cite{keller}.  Surprisingly, the derived model equations and dispersion relation are identical to those found by Steele and Taber.}

\par{In this section, I present a derivation of the model equations and dispersion relation following Viergever. If the reader does not wish to read the derivation, the model equations and dispersion relation can be found above in Eqns \ref{kEqn} and \ref{duifhuisKING}.}

\subsection{Outline of the Method}

\par{The method follows the following steps:
\begin{enumerate}
    \item{Change the variables of the BVP in pressure so that terms relating to the model parameters appear in the PDE rather than only in the boundary conditions.}
    \item{Write a form for the solution to this new PDE as
    \begin{equation*}
        A(x,\zeta)\cosh{\kappa(x)(H-\zeta)}e^{jKg(x)}.
    \end{equation*}
    That is, assume that the $z$ (here reparameterized as $\zeta$) contribution is hyperbolic and that there is a wave in $x$. The product with arbitrary $A(x,y)$ means this is done without loss of generality.}
    \item{Assume a series solution for $A$ and plug into the ODE to obtain a system of PDEs.}
    \item{Solve for $A$ up to first order, plug back in to the ansatz and undo the change of variables to solve for pressure.}
\end{enumerate}
}
\par{The process is similar to that followed in the 1-D case. An outline is presented below, with computations presented in the appendices.}

\subsection{Setting up the BVP}

\par{The method followed in this section relies on multiple changes of variables and definitions of new parameters. As such, it can be difficult to keep straight the meanings and units of the various variables and parameters at play. Table \ref{tab::vier} serves as a reference for the objects introduced in the derivation.}

\begin{table}[]
    \centering
    \begin{tabular}{|c||c|c|} \hline 
         Symbol & Significance & Units\\ \hline \hline
         $Z_0$ & Arbitrary reference impedance used to the simplify series solution. & Pa$\cdot$s/mm \\ \hline
         $K$ & Reference wavenumber used to simplify the series solution.  & 1/mm \\ \hline
         $f^2(x)$ & $Z_0/Z_{OC}(x)$, used so simplify $x$-dependence of the PDE. & Unitless \\ \hline
         $\zeta$ & $Kz$, nondimensionalized transverse coordinate. & Unitless \\ \hline
         $H$ & $Kh$, nondimensionalized scala height. & Unitless \\ \hline
         $Q(x,\zeta)$ &  Pressure written in terms of the nondimensionalized transverse coordinate. & Pa \\ \hline
         $A(x,\zeta)$ & Controls the magnitude of pressure at the OCC,& \\ & to be solved for in the simplified BVP. & Pa \\ \hline
         $\kappa(x)$ & Controls the $x$-dependence of transverse pressure variations, & \\ & to be solved for in the simplified BVP. & Unitless\\ \hline
         $g(x)$ & Controls the wavenumber of the traveling wave, & \\ & to be solved for in the simplified BVP. & mm \\ \hline
    \end{tabular}
    \caption{Symbols introduced in the derivation of the model equations in the series solution approach, along with their significance and units.}
    \label{tab::vier}
\end{table}

\par{Viergever begins with the 2-D box model BVP in $P(x,z)$, then performs a change of variables.To start, we define a reference impedance $Z_0$ which is some arbitrary constant. We define also $f^2(x) = Z_0/Z_{OC}(x)$ and a reference wavenumber $K^2 = -2j\omega\rho/hZ_0$. Recalling that $P = -2\rho \dot{\phi}$,  the boundary condition at the OCC is
\begin{equation}
    \frac{\partial P}{\partial z} + hK^2f^2(x)P=0,\;\;\;z=0.
\end{equation}
Note that $K$ is not a function of $x$.}

\par{Further reparameterizing the $z$ coordinate and defining a reparameterized pressure, $Q$, as 
\begin{equation}
    \zeta = Kz,\;\;H=Kh,\;\;Q(x,\zeta) = P(x,z),
\end{equation}
the BVP in terms of $Q$ is
\begin{align}
    \frac{\partial^2 Q}{\partial x^2} + K^2\frac{\partial^2 Q}{\partial \zeta^2} = 0,\\
    \frac{\partial Q}{\partial \zeta}\bigg|_{\zeta=H} = 0,\\
    \frac{\partial Q}{\partial \zeta}\bigg|_{\zeta=0} + Hf^2(x)Q(x,0) = 0.
\end{align}
}

\par{The solution to the above PDE is artificially represented in a form resembling what the solutions are expected to be, by intuition about the Laplace equation. In particular, $Q$ is written as
\begin{equation}
    Q(x,\zeta) = A(x,\zeta;K)e^{jKg(x)}\cosh{[\kappa(x)(H-\zeta)].}
    \label{waveishguess}
\end{equation}

 The exponential suggests a traveling wave in $x$, where $A$ modulates the amplitude of this wave. However, this is not actually an assumption of a wave solution -- as $A$, $g$ and $\kappa$ are unknown functions of $x$ (and $\zeta$, for $A$), any function can be represented in this fashion without loss of generality.}

\par{One might wonder why we have chosen to introduce so many new terms into these equations. While this may initially seem to complicate the BVP, it eventually leads to the most mathematically tractable solution method.}
\par{Alongside Table \ref{tab::vier}, it may help to ``look into the future" to see what these newly defined variables will become. The variable $\kappa$ will be found to be the nondimensionalized wavenumber and $g$ will be found to be the integral of the wavenumber. What is intriguing about this method is that we are not \textit{assuming} that the wavenumber will appear, but rather it falls out of the derivation.}
\par{Moreover, the free parameter $K$ will eventually be the variable of our formal power series (similar to $\delta$ from Eqn \ref{WKB_withdelta}).}
 
\par{Plugging this form of $Q$ into the BVP, we can find an equivalent BVP in terms of $A$. The derivation of the new BVP is shown in Appendix \ref{app:bvpinA}.}
\par{Writing $a(x,\zeta) = \kappa(x)(H-\zeta)$ to simplify notation, we arrive at the following PDE and boundary conditions:
\begin{align}
\begin{split}
    K^2&\bigg[(\kappa^2 - g'^2)A\cosh{a} + \frac{\partial^2 A}{\partial \zeta^2}\cosh{a} -2\kappa \frac{\partial A}{\partial \zeta}\sinh{a} \bigg]+\\
    +jK&\bigg[g''A\cosh{a} + 2g'\frac{\partial A\cosh{a}}{\partial x} \bigg] + \frac{\partial^2 A\cosh{a}}{\partial x^2}=0,
    \end{split}
    \label{PDEinA}
\end{align}
\begin{equation}
    \frac{\partial A}{\partial \zeta}\bigg|_{\zeta = 0} -\kappa A(x,0)\tanh{a(x,0)} + Hf^2A(x,0) = 0,
\end{equation}
\begin{equation}
    \frac{\partial A}{\partial \zeta}\bigg|_{\zeta = H} = 0.
\end{equation}
}
\par{Solving this ODE in $A$ is the new goal. With a solution for $A$, we can find $Q$ and finally $P$.
}

\subsection{A Series Solution for $A$}
\par{Similar to the WKB approach, a formal power series solution in the form
\begin{equation}
    A(x,\zeta;K) = A_0(x) + \sum_{n=1}^\infty \frac{1}{(jK)^n}A_n(x,\zeta)
\end{equation}
is assumed, with monotonic decrease in magnitude of terms and their derivatives in increasing $n$, and allowing for termwise differentiation.}
\par{This form of the solution is not quite the WKB ansatz, but the logarithm of such a solution with $\delta = jK$. This is motivated by Keller's approach to surface waves on water of non-uniform depth \cite{keller}. }
\par{The method that follows is similar to the 1-D WKB solution method. In Appendix \ref{app:pluginA}, I show the result of plugging this ansatz into the PDE for $A$ in Eqn \ref{PDEinA}. The result is a system of infinitely many PDEs, of which we consider only the PDEs including $A_0$ and $A_1$ (justified by the terms and their derivatives being assumed to decrease monotonically).}
\par{The resulting system of differential equations is
\begin{equation}
    g'^2(x) = \kappa^2(x),
    \label{recurse0}
\end{equation}

\begin{equation}
    \cosh{a}\frac{\partial^2 A_1}{\partial \zeta^2} - 2\kappa\sinh{a}\frac{\partial A_1}{\partial \zeta} =g''A_0\cosh{a} + 2g'\frac{\partial A_0 \cosh{a}}{\partial x}.
    \label{seriespde0}
\end{equation}
Application of boundary conditions gives
\begin{equation}
    \frac{\partial A_1}{\partial \zeta}\bigg|_{\zeta=H} = 0,
    \label{Hboundary}
\end{equation}
\begin{equation}
    \frac{\partial A_1}{\partial \zeta}\bigg|_{\zeta=0} = 0,
\end{equation}
\begin{equation}
    \kappa\tanh{\kappa H} = Hf^2.
    \label{kapparelation}
\end{equation}
The final equation resembles the dispersion relation derived from the variational method.}
\par{Eqn \ref{recurse0} is solved by
\begin{equation}
    g(x) = \pm \int_0^x \kappa(\xi)\;d\xi + C
    \label{geqn}
\end{equation}
for arbitrary constant $C$. It is worth noting that this form of the phase's $x$-dependence arrives directly from the derivation, whereas it was simply assumed in the variational method.}

\par{\textbf{Note on terminology:} Eqn \ref{recurse0} is actually an \textit{eikonal equation}, although it is not related to the dispersion relation that is often refered to as ``the eikonal equation" in literature.}

\subsection{Finding a First Approximation for $P$}
\par{Solving Eqn \ref{seriespde0} is nontrivial, as it contains both $A_0$ and $A_1$. In Appendix \ref{app:solveA0}, I solve for the first term $A_0$ using clever substitutions. I find
\begin{equation}
    A_0 = C(\kappa H + \sinh{\kappa H}\cosh{\kappa H})^{-1/2},
\end{equation}
for arbitrary $C$. Theoretically this facilitates solution for $A_n$ for any $n$ as well. On the other hand, the series approximation (as in the WKB method) gives that the higher $n$ terms should be small if $K$ is large relative to its own rate of change.}
\par{Ignoring $A_n$ for $n\geq 1$ gives a first approximation for $Q$ by putting $A\approx A_0$. Using Eqn \ref{geqn} for $g$, there are two possible solutions: 
\begin{equation}
    Q_{\pm}(x,\zeta) = C_{\pm}(\kappa H + \sinh{\kappa H}\cosh{\kappa H})^{-1/2} e^{\pm j K \int_{0}^x\kappa(\xi)\;d\xi}\cosh{[\kappa(x)(H-\zeta)]} + O(1/K).
\end{equation}
}
\par{The reference constant $K$ was defined as $K^2 = -2j\rho\omega/hZ_0$, where $Z_0$ was a \textit{second} reference constant so that $f^2 = Z_0Y_{OC}$. Because $Z_0$ was entirely arbitrary, I am free to choose $Z_0 = -2j\rho\omega h^{-1}$ so that $K=1$ mm$^{-1}$. This also gives $H=1$ mm$^{-1} \times h$ [unitless], $\zeta=1$ mm$^{-1} \times z$ [unitless] and $Q = P$ [Pa].}
\par{I define $k = 1$ mm$^{-1} \times \kappa$. In this light, the $\kappa$ relation from Eqn \ref{kapparelation} becomes
\begin{equation}
    k \tanh{k h} = -2j\rho \omega Y_{OC}.
    \label{dispViergever}
\end{equation}
This is \textit{precisely} the dispersion relation derived through the variational method, where $k$ is the wavenumber with units mm$^{-1}$.}
\par{The first approximation for pressure with arbitrary constants $C_{\pm}$ is now
\begin{equation}
    P(x,z) = (k h + \sinh{k h}\cosh{k h})^{-1/2}\cosh{[k(x)(h-z)]}\bigg[ C_+ e^{j \int_{0}^x k(\xi)\;d\xi} + C_- e^{- j \int_{0}^x k(\xi)\;d\xi}\bigg].
\end{equation}
}

\subsection{Finding the Constants}
\par{To find the constants, the two $x$ boundary conditions are used:
\begin{equation}
    \frac{1}{h}\int_0^h P(0,z)\;dz = P_{OW},\;\;\; \frac{\partial P}{\partial x}\bigg|_{x=L} = 0,
\end{equation}
where $L$ is the length of the cochlea and $P_{OW}$ is the average pressure at the stapes. }
\par{In Appendix \ref{app:pressureconsts}, I compute the constants $C_+$ and $C_-$ using these conditions, achieving

\begin{align}
\begin{split}
    P(x,z) = \frac{P_{OW} k_0 h\cosh{[ k(x)(h-z)]}}{\sinh{ k_0 h}}\sqrt{\frac{ k_0 h + \sinh{ k_0 h}\cosh{ k_0 h}}{ k(x) h + \sinh{ k(x) h}\cosh{ k(x) h}}}\times\\\times\frac{e^{-j\int_{0}^x k(\xi)\;d\xi} + e^{\int_{0}^x k(\xi)\;d\xi -2j\int_{0}^L k(\xi)\;d\xi}}{1+e^{-2j\int_{0}^L k(\xi)\;d\xi}}
\end{split}
\end{align}
}
\par{Assuming that the backwards traveling wave is negligible, $\int_0^L k(\xi)\; d\xi$ must be very large. Asserting this, the expression is simplified:
\begin{equation}
     P(x,z) = \frac{P_{OW} k_0 h\cosh{[ k(x)(h-z)]}}{\sinh{ k_0 h}}\sqrt{\frac{ k_0 h + \sinh{ k_0 h}\cosh{ k_0 h}}{ k(x) h + \sinh{ k(x) h}\cosh{ k(x) h}}}e^{-j\int_0^x k(\xi)\;d\xi}.
\end{equation}
Pulling $\cosh{k_0 h}/\cosh{k h}$ out of the square root symbol yields the form

\begin{equation}
     P(x,z) = \frac{P_{OW} k_0 h}{\cosh{k(x) h} \tanh{k_0 h}}\sqrt{\frac{k_0 h\text{sech}^2 k_0 h + \tanh{k_0 h}}{k(x) h \text{sech}^2 k_0 h + \tanh{k(x) h}}} \cosh{[k(x)(h-z)]} e^{-j\int_0^x k(\xi)\;d\xi}.
\end{equation}
This is precisely Eqn \ref{duifhuisKING}, as derived from the variational method! We also have the same dispersion relation in Eqn \ref{dispViergever}, so both components of the WKB solution are identical.}

\subsection{A Discussion of the Method}
\par{This method was based on a series solution and never assumed any form of the impedance at the OCC. This means that the approximation holds for complex impedances with positive or negative resistance.}
\par{Further, this solution at no point postulates a single-mode wave solution. The assumed solution in Eqn \ref{waveishguess} is motivated by the likelihood of a wave solution, but the wave is multiplied by an arbitrary $A(x,z)$ function that could, in principle, delete the wave character entirely.}
\par{This apparent improved generality relative to the variational method resulted in the same model equations and dispersion relation. This is, in essence, the deeper root that de Boer hinted at.}


\pagebreak

\section{Methods for Finding $k$}
\label{sec:k}
\par{The WKB model poses one computational difficulty -- solving for $k$ in terms of the model parameters. The dispersion relation relating $k$ and the impedance in Eqn \ref{ktanhkh} is transcendental, and generally has infinitely many solutions in the complex plane. To standardize the language, this is rephrased as a root-finding problem to the function
\begin{equation}
    f(z) = z\tanh{z} - C,
    \label{rootfind}
\end{equation}
where
\begin{equation}
    z=kh,\;\;\;C=-2\rho h j\omega Y_{OC}.
    \label{Cdef}
\end{equation}
}
\par{At each position and frequency, we want only \textit{one} of these roots. Which root do we want? As the velocity is loosely of the form $e^{-jkx}$, the solution should possess a positive real part to correspond to a forward traveling wave. As for the imaginary part, this leads to either damping or amplification of the solution in $x$. Exponential growth is aphysical, meaning that the imaginary part must be negative and the solution for $k$ must lie in the $4^{\text{th}}$ quadrant of the complex plane.}

\par{Moreover, of the solutions in this quadrant, the one with the smallest (in magnitude) imaginary part is optimal. A more negative imaginary part would lead to more severe exponential damping, so the most significant solution has the least such damping.}
\par{In this section, I discuss the properties of the roots of $f$, and the challenges that come in trying to solve for physically reasonable roots. I then describe in detail three algorithms for finding $k$.}

\subsection{Qualitative Features of the Roots}

\par{I begin by discussing the roots of this equation in a qualitative sense, observing their behavior in limits and in numerical studies.}
\subsubsection{Asymptotic Behavior}
\par{A long-wave approximation for $k$ can be found by realizing that for small argument, the hyperbolic tangent is approximated by the identity function. The root-finding problem becomes
\begin{equation}
    z_{lw}\tanh{z_{lw}} \approx z_{lw}^2 = C,
\end{equation}
where the $lw$ subscript indicates a long-wave approximation. As $z_{lw} = hk_{lw}$, we have
\begin{equation}
    k_{lw} \approx \sqrt{-2j\rho\omega h^{-1} Y_{OC}}.
\end{equation}
There is always a $\pm$ ambiguity in a square root, but the restriction of positive real part allows for easy discrimination. It should be noted that, as stiffness dominates in this region, the long-wave $k$ is near-purely real. Moreover, it is identical to the 1-D solution for $k$ in Eqn \ref{k1d}.}

\par{The effective height, $h_e$ from Eqn \ref{eqn::heff} also has a corresponding long-wave approximation using the same approximation of the hyperbolic tangent function:
\begin{equation}
    h_{e,lw} \approx \frac{1}{hk^2}.
\end{equation}
}

\par{Similarly, short-wave approximations for $k$ and $h_e$ can be found by realizing that for large argument, the hyperbolic tangent is approximated by 1. This gives
\begin{equation}
    k_{sw} \approx -2j\rho\omega h^{-1} Y_{OC},
\end{equation}
\begin{equation}
    h_{e,sw} \approx \frac{1}{k},
\end{equation}
}

\par{To visualize the difference between these approximations and the exact equation for $h_e$, Fig \ref{fig::he} shows the three solutions as functions of $k$. It shows a continuous switchoff between the two approximations near the point where the long-wave and short-wave solutions intersect.}

% Figure environment removed

\par{The direct proportionality of $k_{sw}$ to the admittance is troublesome, as the impedance eventually reaches a point at which the stiffness and mass cancel. This creates a sharp change in $k$ (and in fact a discontinuity if no resistance is present), violating the WKB assumption of slowly varying $k$ (Eqn \ref{wkbassume}). This problem will be considered when developing root-finding algorithms.}

\subsubsection{Behavior as $x$ Increases}
\par{Viergever and de Boer numerically probed the behavior of $k$ between the long- and short-wave limits limits by carefully tracing the roots of $f$ in the fourth quadrant for different values of $C$ \cite{deBoer_Rootfinding}. Because the function $f$ is continuous, a small variation of $C$ will (mostly) yield a small variation of the root position. Each continuous path traced by the roots with increasing $x$ is called a \textit{root locus}.}
\par{For example, de Boer and Viergever find that with realistic impedance functions, the root loci form arcs in the fourth quadrant, traveling from the positive real axis to negative imaginary axis with increasing $x$. Fig \ref{fig:loci} shows four such root loci under one set of parameters, where each color corresponds to a different stimulus frequency and each circle is a root at a different $x$ position ($x$-resolution is $7\mu m$). As $x$ increases, the locus diverges from the real line and traverses clockwise towards the negative imaginary axis. At higher frequencies, the arc is broader and arrives at a larger negative imaginary component.}

% Figure environment removed

\par{The essence of a WKB model is that this variation of the root $k$ in $x$ is slow, so that tracing the continuous arc through the plane (possible with a fine enough resolution in $x$) would give the root of interest. However, with physically realistic parameters, one runs in to multiple issues just past the peak region. In particular, near the location where stiffness and mass cancel, the admittance factor of $C$ varies rapidly. Here, the WKB assumption breaks down, and a tracing of the root locus shows a rapid traversal of the arc near these positions. This can be seen in Fig \ref{fig:loci}, where the roots appear sparse along the broad arc of the locus. This indicates a much faster change in $k$ than at the denser regions near the real and imaginary axes.}

\par{One issue could come from the fact that insufficient resolution in $x$ could not capture the continuous but rapid arc of the root locus, and may instead yield convergence to a root in a different locus. This issue can be resolved either by uniformly refining resolution, or refining resolution
close to the resonant position \cite{viergever_Book}.}

\par{Less simple is the fact that at higher frequencies, the radius of the locus arc increases, achieving a very large negative imaginary part. This results in a rapid fall-off of the velocity response -- more rapid than what is seen in data or in a finite element model \cite{steele_lagrange}.}

\par{It is a misconception that the WKB approximation should fail past the point of the peak, as the response no longer resembles a wave. Instead, as the root of $f$ approaches the imaginary axis, the not-very-wave-like solution still fits the bill for the WKB approximation, as Fig \ref{fig:loci} suggests. The longitudinal stepping class of algorithms, detailed below, handles these two regions separately.}

\subsection{Longitudinally Stepping Algorithms}
\par{The goal is to begin by tracing a single root locus for $f$ through the complex plane. Due to the number of possible roots at a given $x$, canonical root-finding methods can cause trouble. Such methods require an intelligently chosen starting point so as not to converge to the wrong root, or even a saddle point.}
\par{In this section, I will describe a class of algorithms for root-finding that step across the longitudinal axis, at each point making an estimate for $k$ informed by the estimate from the previous step. These methods were developed and employed mostly by de Boer and Viergever \cite{deBoer_Rootfinding,viergever_Book}. Here, $x$ values are quantized. I will write the estimate for $k$ at position $x_n$ as $\hat{k}_n$. As the function $f$ is itself $x$-dependent, I will write $f(z;x_n)$ to refer to $f$ at each position.}
\par{Starting at the very base, we are likely to be in the long-wave region. This motivates the initial approximation of $\hat{k}_1 = k_{lw}(x_1)$. This can be used as the initial value in a standard root-finding algorithm such as Newton-Raphson or the Muller method, which are likely to converge to the correct root.}
\par{Stepping further along in $x$, the long-wave approximation becomes poor. This indicates that we ought not use this initial value forever. As in Fig \ref{fig:loci}, wavenumbers at subsequent $x$ locations are likely close to one another, $k_n \approx k_{n+1}$. The intuitive estimate is to use the solution at $x_n$, $\hat{k}_n$, as the starting point for the root-finding method at $x_{n+1}$ to find $\hat{k}_{n+1}$.}
\par{Pseudocode for this algorithm using the Newton-Raphson method in $z$ to compute the wavenumber is presented in Alg \ref{alg::no_discon}. The Newton-Raphson method requires the derivative of $f$, given by
\begin{equation}
    f'(z) = \tanh{z}+z\text{sech}^2{z}.
\end{equation}
Recall that $z = kh$.}

\begin{algorithm}
\begin{algorithmic}
\State $\hat{k}_{ic} \gets k_{lw}(x_1)$ \Comment{Initialize using long-wave $k$}
\For{$n = 1 \rightarrow N$} \Comment{$N$ is the number of steps in $x$ space}
    \State{$z \gets \hat{k}_{ic}$}
    \For{$m = 1 \rightarrow M$} \Comment {$M$ is the number of Newton-Raphson iterations}
        \State{$z \gets z - \frac{f(z;x_n)}{f'(z;x_n)}$} 
    \EndFor
    \State{$\hat{k}_{n} \gets z/h$}
    \State{$\hat{k}_{ic} \gets k_n$} \Comment{Initial value for next step is current guess for $k$}
\EndFor

\end{algorithmic}
\caption{Naive longitudinally-stepping root-finding algorithm to determine the wave number at $N$ different $x$ positions, using the Newton-Raphson method.}
\label{alg::no_discon}
\end{algorithm}


\par{This works so long as $k$ is slowly varying, which is precisely the WKB assumption. However, there is a significant problem that occurs near the resonant point where stiffness and mass cancel. As described in the section above, this creates a very sharp change in $k$. If we were to ignore this feature, we would simply follow the continuous root locus as in Fig \ref{fig:loci}, tending towards solutions for $k$ with unrealistically large negative imaginary parts. This leads to falloff in the magnitude response that is far more rapid than what is seen in data. To arrive at more realistic results, we need to come up with a new approximation near this point. }
\par{\textbf{Note on terminology:} certain modelers have described this point as the ``critical layer," owing its name to a more general theory of critical layer absorption \cite{lighthill_long,lighthill_short}.}



\subsubsection{Approximation Near the Discontinuity}
\par{I will present the method for finding $k$ near the resonant position in a manner similar to that developed by Viergever \cite{viergever_Book}. It is distinct from the method of Steele and Miller, but related in that it ascribes a predetermined assumption about $k$ near the resonant position \cite{steele_rootfinding}.}
\par{To begin, the term $C$ in the root-finding problem (see Eqn \ref{Cdef}) is approximated near the resonant position. Where the stiffness and mass cancel, the admittance is $Y_{OC}\approx 1/R_r$ (where the $r$ subscript denotes its evaluation near the resonant position), i.e. $C$ is purely negative imaginary. Following Viergever, I define a new term $\gamma$:
\begin{equation}
    \gamma = \frac{R_r}{2\rho h \omega_d}\in\mathbb{R}.
\end{equation}
The new problem to solve becomes
\begin{equation}
    z\tanh{z} = \frac{-j}{\gamma}.
    \label{gammaform}
\end{equation}
}
\par{The solution to this transcendental equation can be approximated using the assumption that $z$ is small in magnitude and lives close to the imaginary axis. This ensures that the chosen value of $k$ will correspond to the most significant magnitude response, falling off less rapidly than what would be found by tracing the continuous root locus.}

\par{A derivation of this solution, relying on Taylor expansions, is given in Appendix \ref{app:discont}. It yields
\begin{equation}
    k_d \approx \frac{\pi}{2h}\gamma - j \frac{\pi}{2h}(1-\gamma^2),\;\;\; \gamma = \frac{R}{2 \rho h \omega_d}.
\end{equation}
}

\par{The steep slope of $k$ affects the algorithm's convergence in a small interval near the resonant position. It is not generally reasonable to try to analytically determine where the $x$-stepping method breaks down, and is better determined ``on-the-fly."}

\par{Viergever presents a method that blocks out a piece of the $x$ domain, $(x_-,x_+)$, in which not to solve for roots, based on where he assumes the algorithm to break down (near the resonant position). However, he admits that this is an art rather than a science, stating
\begin{displayquote}
It is interesting to notice that a slight shift basalwards of the location at which the LG wavenumber $k$ was prescribed to be discontinuous would have given excellent results between the LG and the finite element results. I have not been able to trace why and how much the location ought to be shifted however \cite{viergever_Book}.
\end{displayquote}
}
\par{Before this blocked out portion, the $x$-stepping method is used as described above. Within this block, no computations are performed, and the velocity is determined here simply by setting it equal to whatever the last computed value of velocity was, $V(x_-)$. As velocity is continuous, this is reasonable so long as $x_+ - x_-$ is small. At $x_+$, he resumes the $x$-stepping algorithm, using the value of $k_d$ computed above for the initial guess.}
\par{Another option is an on-the-fly method, that checks where the method breaks down by evaluating $\hat{k}_n - \hat{k}_{n-1}$ at each step. This difference serves as a first approximation to the derivative of $k$, so when the WKB approximation holds, this value should be very small. Picking some threshold $T>0$, the $x$-stepping method is paused once $|\hat{k}_n - \hat{k}_{n-1}|>T$. This corresponds to Viergever's $x_-$.}
\par{After this point, the above-computed $k_d$ is used as an initial step in the root-finding algorithm. If still $|\hat{k}_n - \hat{k}_{n-1}|>T$, this value is not considered valid, and velocity at this position is set equal to the last computed velocity. We continue to check if $\hat{k}$ satisfies this threshold at subsequent steps -- it will eventually do so, at which point we continue to step in $x$ (corresponding to Viergever's $x_+$). Of course, there is still an art in selecting a proper $T$.}

\begin{algorithm}
\begin{algorithmic}
\State{$\hat{k}_{ic} \gets k_{lw}(x_1)$} \Comment{Initialize using long-wave $k$}

\For{$n = 1 \rightarrow N$} \Comment{$N$ is the number of steps in $x$ space}
    \State{$z \gets \hat{k}_{ic}$}
    \For{$m = 1 \rightarrow M$} \Comment {$M$ is the number of Newton-Raphson iterations}
        \State{$z \gets z - \frac{f(z)}{f'(z)}$} 
    \EndFor
    \If{$|z/h - k_{ic}| < T$} \Comment{Check for validity of WKB condition}
        \State{$\hat{k}_{n} \gets z/h$}
        \State{$\hat{k}_{ic} \gets k_n$} \Comment{Initial value for next step is current guess for $k$}
    \Else{}
        \State{$\hat{k}_{n} \gets$ NaN} \Comment{Pressure and velocity should not be computed here}
        \State{$\hat{k}_{ic} \gets k_d$} \Comment{Guess for $k$ after the discontinuity}
    \EndIf
\EndFor

\end{algorithmic}
\caption{Longitudinally-stepping root-finding algorithm to determine the wave number at $N$ different $x$ positions, accounting for the discontinuity in the wavenumber. }
\label{alg::with_discon}
\end{algorithm}

\subsection{The Stable Point Algorithm}

\par{One alternative to the $x$-stepping class of algorithms is a stable point algorithm, in which two distinct relationships between $k$ and $\alpha$ (the pressure-focusing factor) are used.}

\par{The first such relationship is that given in Eqn \ref{alpha}, which we have already derived from the 2-D WKB solution. The second, derived by Duifhuis, is found by analyzing the wave equation for average pressure in scala vestibuli \cite{duifhuis_moh}. It is his Eqn A8, given by
\begin{equation}
    k^2 = \frac{2j\omega \rho \alpha(x)}{h Z_{OC}(x)}.
    \label{stable}
\end{equation}
A valid $k$ value must satisfy both equations.}

\par{Stable point methods are based on the contractive mapping theorem, which states that repeated application of a contractive function $g$ will converge to a stable point of said function, i.e. a point where $g(x) = x$. Mathematical details are omitted here for the sake of brevity.}

\par{The stable point method for this problem works by starting with the long-wave approximation at every frequency-location pair, $\hat{k} = k_{lw}$. Then, $\hat{k}$ is plugged in to Eqn \ref{alpha} to find an the approximate pressure focusing factor $\hat{\alpha}$, and then  $\hat{\alpha}$ is plugged in to Eqn \ref{stable} to find a new wavenumber approximation $\hat{k}$. This is repeated for some number of iterations. Pseudocode for this algorithm is shown in Alg \ref{alg::stable}. 
}

\begin{algorithm}
\begin{algorithmic}
\State{$\hat{k} \gets k_{lw}$} \Comment{Here $\hat{k}$ is a vector with an index for each position}

\For{$m = 1 \rightarrow M$} \Comment {$M$ is the number of stable point iterations}
    \State{$\hat{\alpha} \gets \frac{h \hat{k}}{\tanh{kh}}$}\Comment{Pressure focusing vector update, Eqn \ref{alpha}}
    \State{$\hat{k} \gets \sqrt{\frac{2j\omega \rho \hat{\alpha}}{h Z_{OC}}}$} \Comment{Wavenumber update, Eqn \ref{stable}}
    \If{$\mathcal{I}[\hat{k}] < 0 $}
    \State{$\hat{k} \gets -\hat{k}$}\Comment{Ensure the root is for a forward-traveling wave ($\mathcal{I}$ gives the imaginary part)}
    \EndIf
\EndFor

\end{algorithmic}
\caption{Stable point algorithm that updates a vector of $\hat{k}$ approximations by iteratively applying two relationships.}
\label{alg::stable}
\end{algorithm}

\par{This method works under the assumption that it converges to the correct value of $k$, which depends on the properties of the mappings. If the mappings are not (at least locally) contractive, then no convergence is guaranteed. On the other hand, if there are multiple stable points, certain choices of initial conditions may lead to convergence to an undesired $k$.}

\par{Anecdotally, the stable point method works well below the peak region. It gives similar results to the $x$-stepping method until the discontinuity is approached, is computed far faster than the $x$-stepping method, and places no restrictions on the resolution of $x$ in the model computations. However, in the region of discontinuity, it does not converge to roots of $f$ and behaves erratically thereafter.}

\par{This fits into the category of methods which fail to predict responses well past the peak. This does not delegitimize the method, but instead incurs a limitation.}

\subsection{Comparison of $k$-Finding Algorithms}
\par{We will now look at velocity responses at the OCC derived from the 2-D WKB model equations, using three methods for finding the wavenumber: 1) an $x$-stepping algorithm that does not account for the discontinuity, amounting to following a root locus as in Fig \ref{fig:loci}, 2) an $x$-stepping algorithm that does account for discontinuity, via thresholding the finite difference as described above, and 3) the stable point method.}

\par{In all cases the parameters of Steele and Taber are used \cite{steele_lagrange}, along with a 5.5 kHz stimulus. In the $x$-stepping methods, Newton-Raphson is applied for 20 iterations at each location, and resonance is accounted for in method (2) by using a threshold of $T=0.8$. In the stable point method, the iterative algorithm is applied 20 times.}

\par{Fig \ref{fig:disc} contrasts the $x$-stepping methods depending on whether resonance is accounted for. The velocity responses show identical behavior up to a position slightly past the peak, where the finite difference in $k$ becomes sufficiently large so that a discontinuity is registered. After this point, the fall-off in velocity amplitude is slower than if resonance were not considered. This slower falloff is more in-line with what is seen in finite element models, owing to the more reasonable choice for $k$ past the peak region \cite{steele_lagrange,steele_rootfinding,viergever_Book,deBoer_Rootfinding}.}


% Figure environment removed



\par{The root loci in the upper-right panel shows that for the discontinuous method, the traversal of the locus is cut off as the root pattern begins to appear sparser (i.e. faster change in $k$). As described above, the discontinuous algorithm then assumes a small negative imaginary root (blue crosses), which yields less rapid falloff than the larger negative imaginary component found by following the locus continuously (red circles).}

\par{The bottom-right panel serves to show that the two methods are both correctly converging to roots of $f$ at each given $x$. Using both methods, the value of $f(kh)$ at is less than $10^{-10}$ in magnitude at all $x$ -- this stresses the fact that not all roots lie on the same continuous locus.}

\par{Fig \ref{fig:sp} shows these same results, but now alongside the results obtained via the stable point method. These results show similar behavior in velocity magnitude to the continuous $x$-stepping solution, but the phase accumulates more cycles.}

% Figure environment removed


\par{Observation of the root locus and $f(kh)$ for the stable point algorithm shows strange behavior past the peak region. While the stable point method's root locus follows that of the $x$-stepping method for a large range of $x$, it erratically jumps around the complex plane (including to the third quadrant) past the peak. This corresponds to a non-zero value of $f(kh)$ at these positions as well, showing that the algorithm has not correctly converged to a root of the function.}

\par{This is anecdotal justification of the validity of this method until the peak region, after which it fails. Due to the relative speed of this method's convergence, and its not needing a fine resolution for $x$, it is still an attractive method for many WKB applications.}



\clearpage

\section{Discussion and Conclusions}

\par{The WKB model, consisting of model equations for pressure and velocity and a dispersion relation, provides an analytic method for exploring cochlear macromechanical responses. In part, it gives compact, closed-form expressions for asymptotic responses, i.e. in the short- and long-wave regions. More generally, informed numerical solution of the dispersion relation allows for description within the entire region of response (all $x$ and $\omega$), save a region near the resonant position/frequency.}
\par{We have shown two distinct methods for deriving the same model equations and dispersion relation. Between them, one common assumption is necessary: the wavenumber varies relatively slowly in space (Eqn \ref{wkbassume}). No assumption about the presence of resistance is necessary, allowing for the modeling of an active cochlea with negative resistance without jeopardizing the model's validity.}

\par{WKB models are responsible for a number of significant contributions to the field of cochlear mechanics. An early study by Zweig \textit{et al.}, for example, used a WKB model to mathematically describe what the authors call a \textit{cochlear compromise}\cite{Zweig_compromise}. This is the idea that the cochlea must both act as a precise frequency analyzer and a low-loss transmission line carrying energy towards the apex.}

\par{As another important example, measured \textit{in vivo} responses can be matched by WKB models if the OCC is subject to active power input \cite{Wang_Steele_Puria_2016, Yoon2011, altoe_2022}. Displacement measurements \textit{in vivo} at the BM and within the OCC have led many researchers to believe that outer hair cell electromotility is responsible for this cycle-by-cycle power generation \cite{Dewey2021} (although this idea is somewhat controversial\cite{marcel2022}), and WKB models which have been modified to include outer hair cell electromotility have shown corroborating results \cite{Yoon2011, Wang_Steele_Puria_2016}.}

\par{WKB models have also enhanced our understanding of otoacoustic emissions (OAEs). While the above derivations have explicitly avoided discussion of waves traveling from apex to base, removing this assumption allows for the development of WKB models that can be used to study internal reflections of waves within the cochlea that may be responsible for OAEs. WKB models with smooth impedance functions generally do not produce significant reflections, although this is in part due to the boundary condition imposed at the stapes rather than the nature of the model assumptions \cite{deBoer_Energy,retrograde}. However, modelers have included ``roughness" to the impedance term (small perturbations in the stiffness as a function of longitudinal distance) in WKB models, and have found that such perturbations act as internal reflection sources, producing OAE spectra similar to what is seen \textit{in vivo}\cite{oae_sisto,oae_moleti}.}

\par{With the model equations and dispersion relation found decades ago, one may reasonably ask: what is the importance of the WKB model in contemporary times? In fact, many important insights have been gleamed from WKB models in recent literature. Below are just a few such contributions from the past three years.}
\par{The inclusion of spatially-varying scala dimensions to a WKB model has been demonstrated by Altoe and Shera \cite{earhorn}. They show that tapering scala height could introduce an amplification factor that boosts responses at the apex relative to a uniform-height model, resolving losses that occur in the traveling wave as it ``makes its way" to the apex.}
\par{A most natural application of the WKB model is to alter the impedance factor. Altoe and Shera have done so in a manner inspired by recent micromechanical findings, including internal motion in the organ of Corti that is 90$^\circ$ out of phase with BM motion \cite{altoe_2022}. Cleverly including this impact in the impedance term, they can use the model equations as derived in the present work to model such a phenomenon. They arrive at an alternative interpretation of cochlear amplification, in which power may be supplied to the fluid rather than directly to the basilar membrane, using a 2-D WKB model that ostensibly does not include micromechanics.}
\par{Sisto \textit{et al} have used a WKB model with special attention paid to a) the pressure focusing phenomenon described above, and b) the viscosity at the OCC-fluid interface to study the impressive dynamic range of the active cochlea\cite{sisto_2021,sisto_2023}. They find that significantly level-dependent admittance is not required to obtain such a dynamic range.}
\par{With much still to learn about the mechanics of the cochlea, the powerful analytic tool offered by the WKB model is among the strongest we have due to its interpretability and simplicity of computation. This contrasts with finite-element models, which can more explicitly model micromechanical phenomena at the cost of difficulty in interpretation and significant computation time. Foundations as described in this document are useful to modelers who may want to adapt, implement, and gain new insights from the powerful WKB model in the future.}




\pagebreak


\appendix
\renewcommand{\thesection}{A}
\section{Appendix -- Extensive Computations}
\setcounter{equation}{0}
\renewcommand{\theequation}{A.\arabic{equation}}
\renewcommand{\thesubsection}{A.\arabic{subsection}}

\par{I have compiled here most of the computations required to arrive at the model equations in the main text. The outlines provided in the main text are sufficient to gain an understanding of the foundations of development, but these detailed computations make more explicit the locations in which assumptions are being applied. These computations, in large part, do not appear in other literature.}

\subsection{The 1-D Model}

\subsubsection{Solving for Constants in the Zero-Order case}
\label{app:0orderconsts}
\par{Begin with the general solution to the wave equation with constant $k$,
\begin{equation}
    p_0(x) = C_1e^{-jkx} + C_2 e^{jkx}, 
\end{equation}

then solve for these constants using the boundary values. The derivative is
\begin{equation}
    p_0'(x) = jk\big(-C_1e^{-jkx} + C_2 e^{jkx}\big), 
\end{equation}
which at $x=0$ gives 
\begin{equation}
    p_0(0) = jk_0(C_2-C_1) = 2\rho V j \omega, 
\end{equation}
where $k_0 = k(0)$. I chose to ``remember" the $x$-dependence of $k$ in evaluating at 0 to ensure that this expression remains constant even after I substitute $k = k(x)$ at the end of the solution.
}
\par{Solving for $C_2$, I have
\begin{equation}
    C_2 = C_1 + \frac{2\rho V \omega}{k_0}.
    \label{a1c2}
\end{equation}
The derivative condition at $L$ is
\begin{equation}
    p_0'(L) = jk_LC_1(-e^{-jk_L L} + e^{jk_L L}) + \frac{2\rho V j\omega k_L}{k_0}e^{jk_L L} = 0, 
\end{equation}
where $k_L = k(L)$. The expression is algebraically solved for $C_1$, and Eqn \ref{a1c2} gives $C_2$:
\begin{align}
    C_1 &= -\bigg[\frac{2 \rho V \omega}{k_0}\bigg] \frac{1}{1-e^{-2jk_L L}}, \\
    C_2 = C_1 + \frac{2\rho V\omega}{k_0} & = -\bigg[\frac{2 \rho V\omega}{k_0}\bigg] \frac{e^{-2jk_L L}}{1-e^{-2jk_L L}},
\end{align}
This gives $p_0$, and plugging in $k=k(x)$ gives
\begin{equation}
    p_0(x) = -\bigg[\frac{2\rho V \omega}{k_0}\bigg] \frac{e^{-jk(x)x} + e^{jk(x)x - 2jk_L L}}{1-e^{-2jk_L L}}.
\end{equation}
}

\subsubsection{Finding Constants in the 1-D WKB Solutions}
\label{app:1dWKB}

\par{I begin with the second-order WKB solution, as it is more complicated. The first-order solution is easily handled after.}
\par{The boundary conditions are on the pressure derivative, but to take the derivative of $p_2$ would require an unwieldy product rule expansion. However, the derivative of the $k^{-1/2}$ term is approximated as 0, as this derivative would be third order in the WKB expansion (recall that the $k^{-1/2}$ term arrives only in the second-order approximation). This gives
\begin{equation}
    p_2'(x) = j\sqrt{k(x)}\Big(-C_1 e^{-j\int_{0}^{x} k(\xi)\;d\xi} + C_2 e^{j\int_{0}^{x} k(\xi)\;d\xi}\Big).
    \label{Pderiv}
\end{equation}
Letting $x=0$, the exponentials go to $1$. Applying the basal boundary condition,
\begin{equation}
    p_2'(0) = j\sqrt{k_0}(C_2-C_1) = 2\rho V j\omega ,
\end{equation}
where $k_0 = k(0)$. This gives $C_2$ in terms of $C_1$ as
\begin{equation}
    C_2 = C_1 + \frac{2\rho V \omega}{\sqrt{k_0}}.
    \label{a1c2}
\end{equation}
}
\par{The apical boundary condition gives
\begin{equation}
    p_2'(L) = j \sqrt{k_L} \Big(-C_1 e^{-j\int_{0}^{L} k(\xi)\;d\xi} + C_2 e^{j\int_{0}^{L} k(\xi)\;d\xi}\Big) = 0,
\end{equation}
where $k_L = k(L)$. Plugging in to Eqn \ref{a1c2} gives
\begin{equation}
    -C_1 e^{-j\int_{0}^{L} k(\xi)\;d\xi} + \bigg(C_1 + \frac{2\rho V \omega}{\sqrt{k_0}}\bigg) e^{j\int_{0}^{L} k(\xi)\;d\xi} = 0.
\end{equation}
Solving for $C_1$ gives both coefficients:
\begin{align}
    C_1 =-\frac{2\rho V \omega}{\sqrt{k_0}} \frac{ 1}{1 - e^{-2j\int_{0}^{L} k(\xi)\;d\xi}},\\
    C_2 = C_1 +\frac{2\rho V \omega}{\sqrt{k_0}} = \frac{2\rho V \omega}{\sqrt{k_0}} \frac{ e^{-2j\int_{0}^{L} k(\xi)\;d\xi}}{1 - e^{-2j\int_{0}^{L} k(\xi)\;d\xi}}.
\end{align}
}
\par{Finally, $p_2$ is
\begin{equation}
    p_2(x) = \frac{2\rho V \omega}{\sqrt{k_0k(x)}}\ \frac{e^{-j\int_{0}^{x} k(\xi)\;d\xi} + e^{j\int_{0}^{x} k(\xi)\;d\xi-2j\int_{0}^{L} k(\xi)\;d\xi}}{1 - e^{-2j\int_{0}^{L} k(\xi)\;d\xi}}.
\end{equation}
}
\par{I can quickly simplify to the first order solution from here, with derivative
\begin{equation}
    p_1'(x) = jk(x)\Big(-C_1 e^{-j\int_{0}^{x} k(\xi)\;d\xi} + C_2 e^{j\int_{0}^{x} k(\xi)\;d\xi}\Big).
\end{equation}
The relationship between this expression and the second-order expression is solely in the multiplying term $k(x)$, as opposed to $\sqrt{k(x)}$.}
\par{Following the exact same formula chain as above, but replacing $\sqrt{k}$ with $k$, 
\begin{align}
    C_1 =-\frac{2\rho V \omega}{k_0} \frac{ 1}{1 - e^{-2j\int_{0}^{L} k(\xi)\;d\xi}},\\
    C_2 = C_1 +\frac{2\rho V \omega}{k_0} = \frac{2\rho V \omega}{k_0} \frac{ e^{-2j\int_{0}^{L} k(\xi)\;d\xi}}{1 - e^{-2j\int_{0}^{L} k(\xi)\;d\xi}}.
\end{align}
Plugging this in to the formula for $p_1$, 
\begin{equation}
    p_1(x) = \frac{2\rho V \omega}{k_0}\ \frac{e^{-j\int_{0}^{x} k(\xi)\;d\xi} + e^{j\int_{0}^{x} k(\xi)\;d\xi-2j\int_{0}^{L} k(\xi)\;d\xi}}{1 - e^{-2j\int_{0}^{L} k(\xi)\;d\xi}}.
\end{equation}
}
\subsection{The Variational Method}
\subsubsection{Solving for Constants in Velocity Potential}
\label{app:velpotconsts}
\par{To solve for the constants $A$ and $B$ in Eqn \ref{arbvelpot} requires use of the boundary conditions from above --  in particular, the condition on $Z'(0)$, and the condition that no fluid may exit through the upper SV wall at $z=h$. These are, respectively,
\begin{align}
    Z'(0)X(x) &= j\omega W e^{-jkx},\\
    Z'(h)X(x) &= 0.
\end{align}
}
\par{Using $X(x) = e^{-jkx}$, the first condition above becomes
\begin{equation}
    Z'(0) = k(A\sinh{kz} + B\cosh{kz}|_{z=0} = j\omega W.
\end{equation}
This gives $B$:
\begin{equation}
    B = \frac{j\omega W}{k}.
\end{equation}
}
\par{To find $A$, the second boundary condition gives
\begin{equation}
    Z'(h) = k(A\sinh{kz} + B\cosh{kz}|_{z=h} = 0,
\end{equation}
so that
\begin{equation}
    A = \frac{-B}{\tanh{kh}} = \frac{-j\omega W}{k\tanh{kh}}.
\end{equation}
}
\par{This describes the velocity potential free of any arbitrary constants:
\begin{equation}
    \phi(x,z) = \frac{-j\omega W}{k}(\coth{kh}\cosh{kz} - \sinh{kz})e^{-jk}.
\end{equation}
}
\par{A more easily understood/manipulated form of this equation can be found by applying hyperbolic trigonometric identities:
\begin{align}
\begin{split}
    \phi(x,z) &= \frac{-j\omega W}{k\sinh{kh}}(\cosh{kh}\cosh{kz} - \sinh{kh}\sinh{kz})e^{-jkx}\\
    &= \frac{-j\omega W}{k\sinh{kh}}\cosh{[k(z-h)]}e^{-jkx}.
\end{split}
\end{align}}

\subsubsection{Computing the Zero-Order Lagrangian}
\label{app:lagrangian}
\par{To compute energy densities, we must make time explicit. Energy is averaged over a single cycle, which is equivalent to integrating in $\omega t$ from $0$ to $2\pi$.}
\par{We write the impedance at the OCC as a standard linear point-impedance, having a mass, stiffness and resistance:
$Z_{OC} = \frac{P(x,0,t)}{\dot{w}(x,0,t)} = j\omega M + R + \frac{K}{j\omega}$. We write the displacement amplitude as $W$. The potential energy density is given by
\begin{align}
\begin{split}
    V &= \frac{1}{2\pi}\int_{0}^{2\pi} \frac{1}{2}K \mathcal{R}[w(x,0,t)]^2\; d\omega t \\
    &= \frac{1}{2\pi}\int_{0}^{2\pi} \frac{1}{2}K W^2 \cos^2{\omega t}\;d\omega t \\
    &= \frac{KW^2}{4}.
\end{split}
\end{align}
Here, $\mathcal{R}[z]$ denotes the real part of  complex number $z$.}

\par{The kinetic energy has two components -- that of the OCC and that of the fluid. The kinetic energy of the OCC is computed as
\begin{align}
\begin{split}
    T_{OC} &= \frac{1}{2\pi}\int_{0}^{2\pi} \frac{1}{2}M \mathcal{R}[\dot{w}(x,0,t)]^2\; d\omega t \\
    &= \frac{1}{2\pi}\int_{0}^{2\pi} \frac{1}{2} M \omega^2 W^2 \sin^2{\omega t}\;d\omega t \\
    &= \frac{M\omega^2W^2}{4}.
\end{split}
\end{align}
}
\par{To compute the fluid kinetic energy at position $x$, the two-dimensional fluid velocity must be considered over the whole cross-section ($z$ from 0 to $h$). There is kinetic energy in both chambers of fluid, so the result for energy in a single chamber must be multiplied by 2. This is
\begin{align}
\begin{split}
    T_f = &= 2\frac{1}{2\pi}\int_{0}^{2\pi}\int_{0}^{h} \frac{1}{2} \rho (\mathcal{R}[\dot{u}]^2 + \mathcal{R}[\dot{w}]^2)\;dz d\omega t \\
    &= \frac{1}{2}\int_{0}^{h} \rho \bigg(\frac{\omega^2 W^2}{k^2\sinh^2{kh}}k^2\cosh^2[k(z-h)] + \frac{\omega^2 W^2}{k^2\sinh^2{kh}}k^2\sinh^2[k(z-h)]\bigg)\;dz \\
    &= \frac{\rho \omega^2 W^2}{2\sinh^2{kh}}\int_{0}^{h} (\cosh^2{[k(z-h)]}+ \sinh^2{[k(z-h)])}\;dz\\
    &= \frac{\rho \omega^2 W^2}{2\sinh^2{kh}}\int_{0}^{h} \cosh[2k(z-h)]\;dz\\
    &= \frac{\rho \omega^2 W^2}{2\sinh^2{kh}} \bigg[\frac{1}{2k}\sinh[2k(z-h)]\bigg|_{0}^{h}\\
    &=\frac{\rho \omega^2 W^2}{4k\sinh^2{kh}}(0-\sinh(-2kh)) \\
    &=\frac{\rho \omega^2 W^2}{4k\sinh^2{kh}}2\sinh{kh}\cosh{kh},
\end{split}
\end{align}
Using the definition of equivalent height, this simplifies  to
\begin{equation}
    T_f = \frac{1}{2}h_{e}\rho \omega^2 W^2.
\end{equation}
This is Steele and Taber's Eqn 8 \cite{steele_lagrange}.}

\subsubsection{Derivative of $f$ in $k$}
\label{app:fk}
\par{The derivative $f_k$ appears in the WKB model equations. It is derived as follows:
\begin{align}
\begin{split}
    f_k &= 2\rho\omega^2\frac{\partial h_{e}}{\partial{k}}\\
    &= 2\rho\omega^2\frac{\partial} {\partial{k}}\frac{1}{k\tanh{kh}}\\
    &=-2\rho\omega^2\frac{\frac{\partial} {\partial{k}}k\tanh{kh}}{k^2\tanh^2{kh}}\\    &=-2\rho\omega^2\frac{\tanh{kh} + kh\,\text{sech}^2{kh}}{k^2\tanh^2{kh}}.
\end{split}
\end{align}
}

\subsubsection{Eliminating the Constant in the Displacement Expression}
\label{app:dispconst}

\par{I denote the average displacement at the stapes as $\delta_{st}$. This is related to the $x$-direction motion by averaging $u$ over the cross-section at $x=0$ . The $x$-direction motion is
\begin{align}
\begin{split}
    u &= \int \dot{u} \;dt\\
    &= \frac{1}{j\omega} \frac{\partial \phi}{\partial x}\\
    &= \frac{1}{j\omega} \frac{\partial}{\partial x}\bigg[ -\frac{j\omega W}{k\sinh{kh}}\cosh{[k(z-h)]}e^{-jkx}\bigg]\\
    &= \frac{1}{j\omega} \bigg[ -\frac{(j\omega W)(-jk)}{k\sinh{kh}}\cosh{[k(z-h)]}e^{-jkx} \bigg]\\
    &= \frac{jW}{\sinh{kh}}\cosh{[k(z-h)]} e^{-jkx}.
\end{split}
\end{align}
where I used the formula for $\phi$ in Eqn \ref{phi} (with constant $k$). Note that the above expression relies on $W$, which is still parameterized by an unknown constant $C$.} 
\par{Then $\delta_{st}$ is the average $u$ at $x=0$:
\begin{align}
\begin{split}
    \delta_{st} &= \frac{1}{h}\int_{0}^{h} u(0,z)\;dz \\
    & = \int_{0}^{h} \frac{jW_0}{h\sinh{k_0h}}\cosh{[k_0(z-h)]}  \;dz\\
    &= \frac{jW_0}{h\sinh{k_0h}} \int_0^h \cosh{[k_0(z-h)]}\;dz\\
    &= \frac{jW_0}{h\sinh{k_0h}} \bigg[\frac{1}{k_0}\sinh{[k_0(z-h)]\bigg|_{z=0}^{h}}\\
    &= \frac{jW_0}{hk_0},
\end{split}
\end{align}
with the 0 subscript indicating evaluation at $x=0$. Steele and Taber define stapes motion as being positive in the outward direction \cite{steele_lagrange}. This introduces a minus sign above. I think this is confusing given the definition of $x$, so this derivation will differ by a minus sign for that reason.
}
\par{The way to get rid of the unknown constant $C$ is to take the quotient between $\phi$, which we are solving for, and the known stapes displacement. Using the formula for $W$ in Eqn \ref{Winf} and the formula for $\phi$, we can write:
\begin{align}
\begin{split}
    \frac{\phi}{\delta_{st}} &= -\frac{j\omega\cosh{[k(z-h)]} We^{-jkx}}{k\sinh{kh}}\frac{hk_0}{jW_0} \\
    &=-\frac{\omega hk_0\cosh{[k(z-h)]} e^{-jkx}}{k\sinh{kh}}\frac{W}{W_0}\\
    &= -\frac{\omega hk_0\cosh{[k(z-h)]} e^{-jkx}}{k\sinh{kh}}\sqrt{\frac{f_{k,0}}{f_k}}\\
    &= -\frac{\omega hk_0}{k\sinh{kh}}\sqrt{
    \frac{\tanh{k_0h} + k_0h\,\text{sech}^2{k_0h}}{k_0^2\tanh^2{k_0h}}}\sqrt{
    \frac{k^2\tanh^2{kh}}{\tanh{kh} + kh\,\text{sech}^2{kh}}} \cosh{[k(z-h)]} e^{-jkx}\\
    &=-\frac{\omega hk_0}{k\sinh{kh}}\frac{k\tanh{kh}}{k_0\tanh{k_0h}}\sqrt{\frac{\tanh{k_0h} + k_0h\,\text{sech}^2{k_0h}}{\tanh{kh} + kh\,\text{sech}^2{kh}}} \cosh{[k(z-h)]} e^{-jkx }\\
    &=-\frac{\omega h}{\cosh{kh}\tanh{k_0h}}\sqrt{
    \frac{\tanh{k_0h} + k_0h\,\text{sech}^2{k_0h}}{\tanh{kh} + kh\,\text{sech}^2{kh}}} \cosh{[k(z-h)]} e^{-jkx }, 
\end{split}
\end{align}
where $f_{k,0}$ is the $k$-derivative of $f$ at $x=0$.}


\subsubsection{Model Equations in Terms of Average Pressure}
\label{app:duifhuis}
\par{The average pressure at the oval window, $P_{OW}$, can be found by averaging Eqn \ref{pressureDisp} over $z$ at $x=0$. Noting that at $x=0$ the expression inside the square root becomes 1, we have

\begin{align}
\begin{split}
    P_{OW} &= \delta_{st} \frac{2 \rho \omega^2 h}{\cosh{k_0 h}\tanh{k_0 h}} \frac{1}{h}\int_{0}^{h} \cosh{[k_0(z-h)]}\;dz\\
    &= \delta_{st} \frac{2 \rho \omega^2 }{\cosh{k_0 h}\tanh{k_0 h}}\bigg[ \frac{1}{k_0}\sinh{[k_0(z-h)]}\bigg|_{z=0}^{h}\\
    &= \delta_{st} \frac{2 \rho \omega^2 \sinh{k_0 h}}{k_0 \cosh{k_0 h}\tanh{k_0 h}} \\
    &= \delta_{st} \frac{2 \rho \omega^2}{k_0}.
\end{split}
\end{align}
Plugging in to Eqn \ref{pressureDisp}, 
\begin{equation}
    P(x,z,t) = P_{OW}\frac{k_0 h}{\cosh{kh}\tanh{k_0 h}} \sqrt{
    \frac{\tanh{k_0h} + k_0h\,\text{sech}^2{k_0h}}{\tanh{kh} + kh\,\text{sech}^2{kh}}} \cosh{[k(z-h)]} e^{-jkx}.
\end{equation}
This is Duifhuis' Eqn A10 \cite{duifhuis_moh}.}







\subsection{The Series Solution Method}

\subsubsection{Deriving the BVP in $A$}
\label{app:bvpinA}
\par{The derivatives of $Q$ in terms of $A$ are
\begin{equation}
    \frac{\partial Q}{\partial \zeta} = \frac{\partial A}{\partial \zeta} e^{jKg}\cosh{a} - \kappa Ae^{jKg}\sinh{a},
\end{equation}
\begin{equation}
    \frac{\partial^2 Q}{\partial \zeta^2} = \frac{\partial^2 A}{\partial \zeta^2} e^{jKg}\cosh{a} 
 -\kappa \frac{\partial A}{\partial \zeta} e^{jKg}\sinh{a} - \kappa \frac{\partial A}{\partial \zeta}e^{jKg}\sinh{a} + \kappa^2Ae^{jKg}\cosh{a},
\end{equation}
\begin{equation}
    \frac{\partial Q}{\partial x} = \frac{\partial A\cosh{a}}{\partial x} e^{jKg} + jKg' Ae^{jKg}\cosh{a},
\end{equation}
\begin{equation}
    \frac{\partial^2 Q}{\partial x^2} = \frac{\partial^2 A\cosh{a}}{\partial x^2} e^{jKg} + 2jKg'e^{jKg}\frac{\partial A\cosh{a}}{\partial x} -K^2g'^2Ae^{jKg}\cosh{a} + jKg''Ae^{jKg}\cosh{a}.
\end{equation}
}
\par{The boundary conditions for $A$ at $\zeta = H$ (where $a=0$) are
\begin{align}
    \frac{\partial Q}{\partial \zeta}\bigg|_{\zeta = H} &= \frac{\partial A}{\partial \zeta}\bigg|_{\zeta = H} e^{jKg} = 0,\\
    \frac{\partial A}{\partial \zeta}\bigg|_{\zeta = H} &= 0.
\end{align}
}
\par{The boundary condition at $\zeta=0$ is
\begin{align}
\begin{split}
        \frac{\partial Q}{\partial \zeta}\bigg|_{\zeta=0} + Hf^2(x)Q(x,0) = \frac{\partial A}{\partial \zeta}\bigg|_{\zeta = 0} e^{jKg}&\cosh{a(x,0)} -\kappa Ae^{jKg}\sinh{a(x,0)} +\\&+ Hf^2A(x,0)e^{jKg}\cosh{a(x,0)} = 0,
\end{split}
\end{align}
which simplifies to
\begin{equation}
        \frac{\partial A}{\partial \zeta}\bigg|_{\zeta = 0} -\kappa A(x,0)\tanh{a(x,0)} + Hf^2A(x,0) = 0.
\end{equation}
}
\par{Finally,the PDE in $A$ is
\begin{align}
\begin{split}
    \frac{\partial^2 Q}{\partial x^2} + K^2 \frac{\partial^2 Q}{\partial \zeta^2} &= \frac{\partial^2 A\cosh{a}}{\partial x^2} e^{jKg} + 2jKg'e^{jKg}\frac{\partial A\cosh{a}}{\partial x} -K^2g'^2Ae^{jKg}\cosh{a} + \\
    &+ jKg''Ae^{jKg}\cosh{a}+K^2\frac{\partial^2 A}{\partial \zeta^2} e^{jKg}\cosh{a} 
 -K^2\kappa \frac{\partial A}{\partial \zeta} e^{jKg}\sinh{a} +\\& + K^2\kappa^2Ae^{jKg}\cosh{a} -K^2\kappa \frac{\partial A}{\partial \zeta}e^{jKg}\sinh{a} = 0.
\end{split}
\label{PDEinAapp}
\end{align}
After some simplification, this is
\begin{align}
\begin{split}
    K^2&\bigg[(\kappa^2 - g'^2)A\cosh{a} + \frac{\partial^2 A}{\partial \zeta^2}\cosh{a} -2\kappa \frac{\partial A}{\partial \zeta}\sinh{a} \bigg]+\\
    +jK&\bigg[g''A\cosh{a} + 2g'\frac{\partial A\cosh{a}}{\partial x} \bigg] + \frac{\partial^2 A\cosh{a}}{\partial x^2}=0.
    \end{split}
\end{align}
}
\subsubsection{Creating the system of PDEs in $A$}
\label{app:pluginA}

\par{Plugging the series ansatz in to the PDE in $A$ (Eqn \ref{PDEinAapp}) gives
\begin{align}
    \begin{split}
    \bigg[(\kappa^2 - g'^2)\cosh{a}\bigg(K^2 A_0 - jKA_1 + \sum_{n=2}^\infty \frac{1}{(jK)^{n-2}}A_n\bigg) +\\ +\cosh{a}\bigg(-jK\frac{\partial^2 A_1}{\partial \zeta^2} + \sum_{n=2}^\infty \frac{1}{(jK)^{n-2}}\frac{\partial^2 A_n}{\partial \zeta^2}\bigg) +\\-2\kappa \sinh{a} \bigg(-jK\frac{\partial A_1}{\partial \zeta} + \sum_{n=2}^\infty \frac{1}{(jK)^{n-2}}\frac{\partial A_n}{\partial \zeta}\bigg) \bigg]+ \\
    +\bigg[ g''\cosh{a}\bigg(jKA_0 + \sum_{n=1}^\infty \frac{1}{(jK)^{n-1}}A_n\bigg) 
    +2g'\sinh{a}\bigg(jKA_0 + \sum_{n=1}^\infty \frac{1}{(jK)^{n-1}}A_n\bigg) + \\
    + 2g'\cosh{a} \bigg(jk\frac{d A_0}{d x} + \sum_{n=1}^\infty \frac{1}{(jK)^{n-1}}\frac{\partial A_n}{\partial x}\bigg)\bigg] + 
    \bigg[ \cosh{a} \bigg(A_0 + \sum_{n=1}^\infty \frac{1}{(jK)^n}A_n\bigg) +\\
     + 2\sinh{a} \bigg( \frac{d A_0}{d x} + \sum_{n=1}^\infty \frac{1}{(jK)^{n}}\frac{\partial A_n}{\partial x}\bigg) 
    + \cosh{a} \bigg( \frac{d^2 A_0}{d x^2} + \sum_{n=1}^\infty \frac{1}{(jK)^{n}}\frac{\partial^2 A_n}{\partial x^2}\bigg) \bigg]=0.
    \end{split}
\end{align}
}
\par{We have to separate this leviathan into its components in powers of $jK$. The single term in $K^2$ gives
\begin{equation}
    (\kappa^2 - g'^2)\cosh{a}A_0 = 0,
\end{equation}
or
\begin{equation}
    g'^2(x) = \kappa^2(x).
    \label{gpkapp}
\end{equation}
}
\par{Next, consider the $jK$ terms:
\begin{equation}
    -(\kappa^2 - g'^2)A_0 -\cosh{a}\frac{\partial^2 A_1}{\partial \zeta^2} + 2\kappa\sinh{a}\frac{\partial A_1}{\partial \zeta} +g''A_0\cosh{a} +2g'\sinh{a}A_0 + 2g'\cosh{a}\frac{d A_0}{dx} = 0.
\end{equation}
With the aid of Eqn \ref{gpkapp}, we can see the first of these summands is 0. This simplifies the expression to
\begin{align}
    \begin{split}
    \cosh{a}\frac{\partial^2 A_1}{\partial \zeta^2} - 2\kappa\sinh{a}\frac{\partial A_1}{\partial \zeta} &= g''A_0\cosh{a} +2g'\sinh{a}A_0 + 2g'\cosh{a}\frac{d A_0}{dx}\\
    &=g''A_0\cosh{a} + 2g'\frac{\partial A_0 \cosh{a}}{\partial x}.
    \end{split}
\end{align}
}
\par{As for the rest of the terms, we have a system of infinitely many equations. Again, by Eqn \ref{gpkapp},the terms leading with $\kappa^2-g'^2$ are 0. The relation is
\begin{align}
\begin{split}
    \cosh{a}\frac{\partial^2 A_n}{\partial \zeta^2} - 2\kappa\sinh{a}\frac{\partial A_n}{\partial \zeta }=(g''\cosh{a} + 2g'\sinh{a})A_{n-1} + 2g'\cosh{a}\frac{\partial A_{n-1}}{\partial x} + \\+ \cosh{a}\frac{\partial A_{n-2}}{\partial x}+2\sinh{a}\frac{\partial A_{n-2}}{\partial x} + \cosh{a}\frac{\partial^2 A_{n-2}}{\partial x^2}\\
    =g''\cosh{a} + 2g'\frac{\partial A_{n-1} \cosh{a}}{\partial x} +\frac{\partial^2 A_{n-2}\cosh{a}}{\partial x^2} ,\;\;\;\;\;\;n\geq 2.
    \end{split}
\end{align}
In theory, this will allow us to solve for any order of approximation desired so long as we can solve for $A_0$. In practice, we will only solve for $A_0$ to make a first approximation.}
\par{Now consider boundary conditions. At $\zeta=H$, all terms must have a zero derivative (trivial for $A_0$ which has no $\zeta$ dependence). That is,
\begin{equation}
    \frac{\partial A_n}{\partial \zeta}\bigg|_{\zeta=H} = 0,\;\;\;n\geq 1.
\end{equation}
}
\par{As for $\zeta=0$, plugging in the series ansatz to the boundary condition gives
\begin{equation}
   (Hf^2-\kappa\tanh{a})A_0 + \sum_{n=1}^\infty \frac{1}{(jK)^{n}}\bigg(\frac{\partial A_n}{\partial \zeta} + (Hf^2-\kappa\tanh{a})A_n\bigg) = 0
\end{equation}
at $\zeta=0$. Recalling that $a(x,0) = \kappa H$, the $A_0$ term gives
\begin{equation}
    \kappa\tanh{\kappa H} = Hf^2.
\end{equation}
As for the other terms, the above relationship simplifies the summands to
\begin{equation}
    \frac{\partial A_n}{\partial \zeta}\bigg|_{\zeta=0} = 0,\;\;\;n\geq 1.
\end{equation}
}

\subsubsection{Solving the PDE for $A_0$}
\label{app:solveA0}
\par{Recalling that $a(x,\zeta) = \kappa(x)(H-\zeta)$, we can note a relation that helps to simplify the left-hand side of the equation:
\begin{align}
    \begin{split}
        \text{sech}\,a\frac{\partial}{\partial \zeta}\bigg[ \cosh^2{a} \frac{\partial A_1}{\partial \zeta} \bigg] &= \text{sech}\,a \cosh^2{a} \frac{\partial^2 A_1}{\partial \zeta^2} + \text{sech}\,a \frac{\partial \cosh^2{a}}{\partial \zeta} \frac{\partial A_1}{\partial \zeta}\\
        &= \cosh{a}\frac{\partial^2 A_1}{\partial \zeta^2} + \text{sech}\,a (2\cosh{a}\sinh{a})\frac{\partial a}{\partial \zeta}\frac{\partial A_1}{\partial \zeta}\\
        &= \cosh{a}\frac{\partial^2 A_1}{\partial \zeta^2} - 2\kappa \sinh{a} \frac{\partial A_1}{\partial \zeta}.
    \end{split}
\end{align}
This is precisely the left-hand side of Eqn \ref{seriespde0}. That is,
\begin{equation}
    \text{sech}\,a\frac{\partial}{\partial \zeta}\bigg[ \cosh^2{a} \frac{\partial A_1}{\partial \zeta} \bigg] = g'' A_0 \cosh{a} + 2g' \frac{\partial A_0 \cosh{a}}{\partial x}.
\end{equation}
}
\par{Multiplying by $\cosh{a}$ on both sides of the equation and integrating from $0$ to $\zeta$,

\begin{align}
    \begin{split}
        \int_{0}^{\zeta}& \frac{\partial}{\partial \xi}\bigg[ \cosh^2{a} \frac{\partial A_1}{\partial \xi} \bigg]\;d\xi = \cosh^2{a} \frac{\partial A_1}{\partial \zeta} \\
        &= \int_{0}^{\zeta} g''A_0\cosh^2{a} + 2g'\cosh{a}\frac{\partial A_0 \cosh{a}}{\partial x} \;d\xi\\
        &= g''A_0 \int_{0}^{\zeta}\cosh^2{a}\;d\xi + 2g' \frac{d A_0}{dx} \int_{0}^{\zeta} \cosh^2{a}\;d\xi  + g' A_0 \int_{0}^{\zeta} 2\cosh{a}\frac{\partial}{\partial x} \cosh{a} \;d\xi\\
        &= \bigg(g''A_0  + 2g' \frac{d A_0}{dx}  + g' A_0\frac{\partial}{\partial x} \bigg)\int_{0}^{\zeta}\cosh^2{a}\;d\xi,
    \end{split}
\label{withint}
\end{align}
where in the last step I have applied the product rule of derivatives in reverse.}
\par{The only integral to compute now is that of the squared hyperbolic cosine. Using the double-argument formula for $\cosh$,
\begin{align}
    \begin{split}
        \int_{0}^{\zeta}\cosh^2{a}\;d\xi &= \frac{1}{2}\int_{0}^{\zeta} \cosh{[2\kappa(x)(H-\xi)]} + 1\;d\xi\\
        &=\bigg[\frac{-1}{4\kappa}\sinh{[2\kappa(x)(H-\xi)]} + \frac{\xi}{2}\bigg|_{\xi=0}^{\zeta}\\
        & =\frac{\zeta}{2} - \frac{1}{2\kappa}\Big[\sinh{[\kappa(x)(H-\xi)]}\cosh{[\kappa(x)(H-\xi)]}\Big|_{\xi=0}^{\zeta}\\
        & =\frac{\zeta}{2} - \frac{1}{2\kappa}[\sinh{a}\cosh{a} - \sinh{\kappa H}\cosh{\kappa H}].
    \end{split}
    \label{cosh2}
\end{align}
}
\par{Plugging this into Eqn \ref{withint} yields a simplified relationship between $A_0$ and $A_1$ with a useful property -- plugging in $\zeta=H$ facilitates solution for only terms involving $A_0(x)$. Because $A_0$ is $\zeta$-independent, this gives full knowledge of $A_0$.}
\par{The boundary condition in Eqn \ref{Hboundary} gives that the left-hand side of Eqn \ref{withint} is 0. Plugging $H$ into Eqn \ref{withint} (using the integral identity in Eqn \ref{cosh2}), 
\begin{equation}
\bigg(g''A_0  + 2g' \frac{d A_0}{dx}  + g' A_0 \frac{d}{d x} \bigg)\bigg(\frac{H}{2} + \frac{1}{2\kappa} \sinh{\kappa H}\cosh{\kappa H}\bigg) = 0.
\label{signnomatter}
\end{equation}
This is a separable first order \textit{ordinary} differential equation for $A_0$ in $x$. To solve it, begin by defining
\begin{equation}
    S(x) = \frac{1}{2\kappa}\sinh{\kappa H}\cosh{\kappa H}.
\end{equation}
}
\par{Eqn \ref{recurse0} gives $g' = \pm\kappa$, so $g'' = \pm\kappa''$. The $\pm$ ambiguity can be seen to be inconsequential by plugging either signed solution into Eqn \ref{signnomatter}. Using these identities and the above-defined $S$, Eqn \ref{signnomatter} becomes
\begin{align}
    \begin{split}
        \bigg(g''A_0  + 2g' \frac{d A_0}{dx} + g' A_0 \frac{d}{d x}\bigg) \bigg(\frac{H}{2} +S \bigg) = 2\kappa\frac{d A_0}{dx}\bigg(\frac{H}{2}  + S\bigg) + A_0 \bigg(\kappa'\bigg(\frac{H}{2}+S\bigg) +\kappa S'\bigg)=0. 
    \end{split}
\end{align}
Separation of variables gives
\begin{equation}
    \frac{d A_0}{dx} A_0^{-1} = -\frac{\kappa'}{2\kappa} - \frac{1}{2}\frac{S'}{S+\frac{H}{2}}.
\end{equation}
Integration of both sides gives
\begin{equation}
    \ln{A_0} = C+\frac{-1}{2}\ln{\kappa} + \frac{-1}{2}\ln{\bigg(S+\frac{H}{2}\bigg)} = C + \ln{\bigg(\bigg(\kappa S + \frac{\kappa H}{2}\bigg)^{-1/2}\bigg)},
\end{equation}
where $C$ is an arbitrary constant.}
\par{Exponentiating both sides and substituting back in for $S$, we get
\begin{equation}
    A_0 = C\bigg(\frac{\kappa H}{2} + \frac{1}{2}\sinh{\kappa H}\cosh{\kappa H}\bigg)^{-1/2} = C(\kappa H + \sinh{\kappa H}\cosh{\kappa H})^{-1/2},
\end{equation}
where $C$ is an arbitrary constant that absorbs the $\sqrt{2}$ term.}

\subsubsection{Finding the Constants for Pressure}

\label{app:pressureconsts}
\par{Computing first the integral of pressure at $x=0$ across $z$ and writing $\kappa(0) = \kappa_0$, 
\begin{align}
\begin{split}
    \frac{1}{h}\int_{0}^h P(0,z) \; dz &= (\kappa_0 h + \sinh{\kappa_0 h}\cosh{\kappa_0 h})^{-1/2}[C_+ + C_-] \int_{0}^h \cosh{[\kappa_0(h-z)]}\;dz\\
    &=(\kappa h + \sinh{\kappa h}\cosh{\kappa h})^{-1/2}[C_+ + C_-] \frac{1}{\kappa_0}\sinh{\kappa_0 h} = P_{OW}.
\end{split}
\end{align}
}
\par{This gives the relationship
\begin{equation}
    C_+ + C_-= P_{OW}\frac{\kappa_0 h(\kappa_0 h + \sinh{\kappa_0 h}\cosh{\kappa_0 h})^{1/2}}{\sinh{\kappa_0 h}} .
    \label{sumrelation}
\end{equation}
}
\par{As for the second condition, we need the derivative of $P$ in $x$. This is no easy task, unless we make the WKB assumption that $|\kappa'|\ll |\kappa|$. In taking the derivative, the three-term product rule will give factors of $k'$ for all terms other than the derived exponential. Thus, this is the only term that matters, giving:
\begin{align}
    \begin{split}
        \frac{\partial P}{\partial x} = \frac{\partial }{\partial x}(\kappa h + \sinh{\kappa h}\cosh{\kappa h})^{-1/2}\cosh{[\kappa(h-z)]}\bigg[ C_+ e^{j \int_{0}^x\kappa(\xi)\;d\xi} + C_- e^{- j \int_{0}^x\kappa(\xi)\;d\xi}\bigg]\\
        = (\kappa h + \sinh{\kappa h}\cosh{\kappa h})^{-1/2} \cosh{[\kappa(h-z)]} \bigg[ j\kappa C_+ e^{j \int_{0}^x\kappa(\xi)\;d\xi} - j\kappa C_- e^{- j \int_{0}^x\kappa(\xi)\;d\xi}\bigg] + \\
        +\frac{-1}{2}\frac{\kappa' h + \kappa' h(\sinh^2{\kappa h} +\cosh^2{\kappa h})}{(\kappa h + \sinh{\kappa h}\cosh{\kappa h})^{3/2}} \bigg[ C_+ e^{j \int_{0}^x\kappa(\xi)\;d\xi} + C_- e^{- j \int_{0}^x\kappa(\xi)\;d\xi}\bigg] + \\
         +(\kappa h + \sinh{\kappa h}\cosh{\kappa h})^{-1/2}\kappa' (h-z)\sin h{[\kappa(h-z)]}\bigg[ C_+ e^{j \int_{0}^x\kappa(\xi)\;d\xi} + C_- e^{- j \int_{0}^x\kappa(\xi)\;d\xi}\bigg]\\
         \approx (\kappa h + \sinh{\kappa h}\cosh{\kappa h})^{-1/2} \cosh{[\kappa(h-z)]} \bigg[ j\kappa C_+ e^{j \int_{0}^x\kappa(\xi)\;d\xi} - j\kappa C_- e^{- j \int_{0}^x\kappa(\xi)\;d\xi}\bigg].
    \end{split}
\end{align}
}
\par{This quantity is $0$ at $x=L$, i.e.
\begin{equation}
    C_+ = C_- e^{-2j\int_{0}^L \kappa(\xi)\;d\xi}.
    \label{prodrelation}
\end{equation}
}
\par{Combining this with Eqn \ref{sumrelation}, 
\begin{align}
    C_+ + C_- = C_- \bigg(1+e^{-2j\int_{0}^L \kappa(\xi)\;d\xi}\bigg) = P_{OW}\frac{\kappa_0 h(\kappa_0 h + \sinh{\kappa_0 h}\cosh{\kappa_0 h})^{1/2}}{\sinh{\kappa_0 h}} ,\\
    C_- = P_{OW}\frac{\kappa_0 h(\kappa_0 h + \sinh{\kappa_0 h}\cosh{\kappa_0 h})^{1/2}}{\sinh{\kappa_0 h}}\frac{1}{1+e^{-2j\int_{0}^L \kappa(\xi)\;d\xi}},\\
    C_+ = C_- e^{-2j\int_{0}^L \kappa(\xi)\;d\xi} =P_{OW}\frac{\kappa_0 h(\kappa_0 h + \sinh{\kappa_0 h}\cosh{\kappa_0 h})^{1/2}}{\sinh{\kappa_0 h}}\frac{e^{-2j\int_{0}^L \kappa(\xi)\;d\xi}}{1+e^{-2j\int_{0}^L \kappa(\xi)\;d\xi}}.
\end{align}
Finally, these are the constants in the pressure equation. This gives the model equation free of arbitrary parameters:
\begin{align}
\begin{split}
    P(x,z) = \frac{P_{OW}\kappa_0 h\cosh{[\kappa(x)(h-z)]}}{\sinh{k_0 h}}\sqrt{\frac{\kappa_0 h + \sinh{\kappa_0 h}\cosh{\kappa_0 h}}{\kappa(x) h + \sinh{\kappa(x) h}\cosh{\kappa(x) h}}}\times\\\times\frac{e^{-j\int_{0}^x \kappa(\xi)\;d\xi} + e^{\int_{0}^x \kappa(\xi)\;d\xi -2j\int_{0}^L \kappa(\xi)\;d\xi}}{1+e^{-2j\int_{0}^L \kappa(\xi)\;d\xi}}
\end{split}
\end{align}
}

\subsection{Finding the Wavenumber Near the Point of Discontinuity}
\label{app:discont}
\par{We know $k$ lives in the fourth quadrant, so we write $z=kh=a-j\beta $ where $a,\beta>0$. I begin by finding the real and imaginary parts of $\tanh{(a -j\beta)}$.}
\par{Beginning with the formula for hyperbolic tangent in terms of the trigonometric tangent, 
\begin{align}
\begin{split}
    \tanh{(a-j\beta)} &= j\tan{j(a-j\beta)}\\
    &=j\frac{\sin{(\beta + ja)}}{\cos{(\beta+ja)}}\\
    &=\frac{j\sin{\beta}\cos{ja} + j\cos{\beta}\sin{ja}}{\cos{\beta}\cos{ja}-\sin{\beta}\sin{ja}}.
\end{split}
\end{align}
}
\par{Similarly, $\cos{jz} = \cosh{z}$ and $\sin{jz} = j\sinh{z}$, so 
\begin{align}
    \begin{split}
        \tanh{(a-j\beta)} &= \frac{j\sin{\beta}\cosh{a} -\cos{\beta}\sinh{a}}{\cos{\beta}\cosh{a}-j\sin{\beta}\sinh{a}}\\ &=\frac{\cos{\beta}\cosh{a}+j\sin{\beta}\sinh{a}}{\cos{\beta}\cosh{a}+j\sin{\beta}\sinh{a}}\frac{j\sin{\beta}\cosh{a} -\cos{\beta}\sinh{a}}{\cos{\beta}\cosh{a}-j\sin{\beta}\sinh{a}}\\ &=\frac{(\cos{\beta}\cosh{a}+j\sin{\beta}\sinh{a})(j\sin{\beta}\cosh{a} -\cos{\beta}\sinh{a})}{\cos^2{\beta}\cosh^2{a}+\sin^2{\beta}\sinh^2{a}}
    \end{split}
\end{align}
}
\par{I will handle the numerator, $N$, and the denominator, $D$, separately. For the numerator, 
\begin{align}
    \begin{split}
        N &=  -(\cos^2{\beta}+\sin^2{\beta})\cosh{a}\sinh{a} + j(\cosh^2{a-\sinh^2a})\cos{\beta}\sin{\beta} \\
        & = -\cosh{a}\sinh{a} + j\cos{\beta}\sin{\beta}\\
        &=\frac{-\sinh{2a} + j\sin{2\beta}}{2}.
    \end{split}
\end{align}
On the other hand, the denominator is
\begin{align}
    \begin{split}
        D &= \cos^2{\beta}\cosh^2{a}+\sin^2{\beta}\sinh^2{a} \\
        & = \frac{1+\cos{2\beta}}{2}\frac{1+\cosh{2a}}{2} + \frac{1-\cos{2\beta}}{2}\frac{\cosh{2a}-1}{2}\\
        &= \frac{1+\cos{2\beta}+\cosh{2a}+\cos{2\beta}\cosh{2a} - 1 +\cos{2\beta}+\cosh{2a}-\cos{2\beta}\cosh{2a}}{4}\\
        &=\frac{\cos{2\beta}+\cosh{2a}}{2}.
    \end{split}
\end{align}
The quotient $N/D$ split into its real and imaginary parts is thereby
\begin{equation}
    \tanh{(a-j\beta)} = \frac{-\sinh{2a}}{\cos{2\beta}+\cosh{2a}} + j\frac{\sin{2\beta}}{\cos{2\beta}+\cosh{2a}}.
\end{equation}
The object of interest is $z\tanh{z}$. The real and imaginary parts of this quantity are
\begin{align}
    \begin{split}
        (a-j\beta)\tanh{(a-j\beta)} &= (a-j\beta)\bigg(\frac{-\sinh{2a}}{\cos{2\beta}+\cosh{2a}} + j\frac{\sin{2\beta}}{\cos{2\beta}+\cosh{2a}}\bigg) \\
        &=\frac{\beta\sin{2\beta}-a\sinh{2a}}{\cos{2\beta}+\cosh{2a}} + j \frac{a\sin{2\beta}+\beta\sinh{2\alpha}}{\cos{2\beta}+\cosh{2a}}
    \end{split}
\end{align}
}
\par{The goal is to solve for $a$ and $\beta$. Equating the real and imaginary parts of both sides of Eqn \ref{gammaform} gives two equations:
\begin{equation}
    a\sinh{2a} - \beta \sin{2\beta} = 0,
    \label{realpart}
\end{equation}
\begin{equation}
    \frac{a\sin{2\beta} + \beta \sinh{2\beta}}{\cosh{2a}+\cos{2\beta}} = \gamma^{-1}.
    \label{impart}
\end{equation}
}
\par{We can solve these for $a$ and $\beta$ under a few assumptions. Numerical studies of the root-finding problem have shown that as $x$ increases, the roots with the smallest magnitude negative imaginary parts tend towards $-\pi j/2$. This motivates the assumption that $0<a\ll 1$ and $\beta = \pi/2 - \epsilon$ with $0<\epsilon \ll 1$.}
\par{Now we solve for $a$ and $\beta$. The following relations and the Maclaurin approximations up to second order are of use: 
\begin{align}
    \sin(\pi-x) = \sin{x},\;&\;\;\cos(\pi-x) = -\cos{x}\\
    \sin{x} \approx x,\;&\;\;\cosh{x}\approx 1+\frac{x^2}{2}.
\end{align}
}
\par{The imaginary part formula (Eqn \ref{impart}) gives a first approximation for $a$:
\begin{align}
\begin{split}
    \frac{1}{\gamma}&=\frac{a\sin{2\beta} + \beta \sinh{2a}}{\cosh{2a}+\cos{2\beta}}\\
    &\approx\frac{a\sin(\pi-2\epsilon) + 2a\beta}{1+2a^2+\cos{(\pi-2\epsilon)}}\\
    &\approx \frac{2a\epsilon + a\pi - 2a\epsilon}{1+2a^2-1}\\
    &= \frac{a\pi}{2a^2} = \frac{\pi}{2a}.
\end{split}
\end{align}
The final approximation is thereby
\begin{equation}
    a = \frac{\pi}{2}\gamma.
    \label{aform}
\end{equation}
}
\par{We can plug this in to the real part equation (Eqn \ref{realpart}) to find a first approximation of $\beta$:
\begin{align}
\begin{split}
    0 & = a\sinh{2a} - \beta \sin{2\beta}\\
    &\approx 2a^2 - \bigg(\frac{\pi}{2}-\epsilon\bigg)\sin{\pi-2\epsilon}\\
    &\approx 2a^2 - 2\epsilon \bigg(\frac{\pi}{2}-\epsilon\bigg)\\
    &\approx 2a^2 - \epsilon\pi,
\end{split}
\end{align}
where I have used the approximation that $\epsilon^2\approx 0$. Plugging in the formula for $a$ (Eqn \ref{aform}),
\begin{align}
\begin{split}
    \epsilon &= 2\frac{\pi^2 \gamma^2}{4}\frac{1}{\pi}\\
    & = \frac{\pi}{2}\gamma^2.
\end{split}
\end{align}
Knowing that $\beta = \pi/2 - \epsilon$,
\begin{equation}
    \beta = \frac{\pi}{2}(1-\gamma^2).
\end{equation}
}
\par{Finally recalling that $z=kh = a-j\beta$, $k_d$ at the point of discontinuity is
\begin{equation}
    k_d \approx \frac{\pi}{2h}\gamma - j \frac{\pi}{2h}(1-\gamma^2),\;\;\; \gamma = \frac{R}{2 \rho h \omega_d}.
\end{equation}
}

\pagebreak
\appendix
\renewcommand{\thesection}{B}
\section{Appendix -- The Integral Formulation in 3-D and 2-D}
\label{app:integral}
\setcounter{equation}{0}
\renewcommand{\theequation}{B.\arabic{equation}}
\renewcommand{\thesubsection}{B.\arabic{subsection}}

\par{In this appendix, I cover a more general solution method to the box model problem in three dimensions. I then relate it back to the two-dimensional problem, and show its connection to the WKB model described in the main text.}

\subsection{The Third Dimension}
\par{We approach the 3-D BVP in pressure. Assuming separable variable (and leaving time-dependence implicit), we write
$$P(x,y,z) = X_k(x)Y_l(y)Z_m(z),$$
where $k$, $l$ and $m$ represent modes of motion in each dimension. The Laplace equation is then
$$X_k''(x)Y_l(y)Z_m(z) + X_k(x)Y_l''(y)Z_m(z) + X_k(x)Y_l(y)Z_m''(z) = 0.$$
Dividing by $X_k(x)Y_l(y)Z_m(z)$ on both sides gives
\begin{equation}
    \frac{X_k''(x)}{X_k(x)} + \frac{Y_l''(y)}{Y_l(y)} + \frac{Z_m''(z)}{Z_m(z)} = 0.
\end{equation}
}
\par{Because each summand is a single-variable function, each of a \textit{different} variable, the only way they can sum to $0$ is if each term is constant. I choose to write these constants as
\begin{align*}
    X_k''(x) &= -k^2 X_k(x),\\
    Y_l''(y) &= -l^2 Y_l(y),\\
    Z_m''(z) &= m^2 Z_m(z),
\end{align*}
where the constants are related by
\begin{equation}
    k^2 + l^2 - m^2 = 0.
\end{equation}
}
\par{I begin by finding the general box model solution as an integral transform, and then relate it to the 2-D WKB solution.}
\subsection{Integral Solution}

\par{The equations in $x$ and $z$ subject to these same boundary conditions are solved in the main text. However, without the assumption of a single-mode wave for $x$, the $x$ spatial frequency can take any real value. The solutions in $x$ and $z$ are
\begin{equation}
    X_k(x) = A_k e^{-jkx},\; k\in \mathbb{R},
\end{equation}
\begin{equation}
    Z_m(z) = C_m \frac{\cosh{[m(z-h)]}}{\cosh{mh}}, \; m^2 = k^2 + l^2.
\end{equation}
The constant $\cosh{mh}$ is divided by so that the right-hand term has unit norm.}
\par{Solving for $Y_l$ requires the boundary condition that $Y'_l(\pm b/2) = 0$. We have
\begin{equation}
    Y_l(y) = D_l e^{j l y} + E_l e^{-j l y}. 
\end{equation}
Differentiating, dividing by $jl$ and plugging in $\pm b/2$ gives the following system of equations:
\begin{align}
    D_l e^{j l b/2} - E_l e^{-j l b/2} = 0,\\
    D_l e^{-j l b/2} - E_l e^{j l b/2} = 0.
\end{align}
As the two exponential functions are linearly independent, this can be true if and only if $D_l = E_l$ for all $l$. This gives a form for $Y_l'(y)$:
\begin{equation*}
    Y_l'(y) = D_l  (e^{j l y} - e^{-j l y}),
\end{equation*}
or
\begin{equation}
    Y_l'(y) = 2 D_l \sin{ly}.
\end{equation}
}
\par{The boundary conditions are only satisfied if $l b/2$ is an integer multiple of $\pi$, so integrating to achieve $Y_l$, 
\begin{equation}
    Y_l(y) = B_n \cos{\frac{2 \pi n y}{b}},\; n = 0,1,2,\ldots
\end{equation}
}
\par{This gives all of $X_k$, $Y_l$ and $Z_m$ for different modes, but we should be careful to recognize which modes are valid -- $k$ can be any real number, while the set of valid $l$ is countable (and we have replaced it with a parameter $n$). Finally, $m$ is defined in terms of $n$ and $l$ by
\begin{equation}
    m^2 = k^2 + \frac{4 \pi^2 n^2}{b^2},
\end{equation}
so we only have two free variables.}

\par{The general solution requires integrating over all possible values of $k$ and summing over all possible values of $n$. Defining $A_k B_n C_m = \gamma(k,n)$, 
\begin{equation}
    P(x,y,z) = \int_{-\infty}^{\infty} \sum_{n=0}^{\infty} \bigg(\gamma(k,n) \frac{\cosh{[m(z-h)]}}{\cosh{mh}}\cos{\frac{2 \pi n y}{b}}\bigg) e^{-jkx}\;dx.
    \label{P3D}
\end{equation}
}
\par{The value of $\gamma(k,z)$ is what determines the particular solution. Generally, this will be influenced by geometric factors, and usually requires assumptions about the behavior of the BM (point-impedance, single-mode motion, etc.). }

\subsection{Flattening the Model -- The 2-D Integral Solution}

\par{To determine a 2-D box model, I attack the problem as one of 2-D separation of variables. Almost every aspect is identical, except 1) the parameter relationship simplifies to $m^2 = k^2$, or $m = k$, and 2) the cosine series in $y$ is simply gone.}

\par{ Making the extra assumption that we are only concerned with forward-travelling waves, we arrive at
\begin{equation}
    P(x,z) = \int_0^\infty \gamma(k)\frac{\cosh{[k(z-h)]}}{\cosh{kh}}e^{-jkx}\;dk.
\end{equation}
This can be identified as a Fourier transform between $x$ and $k$.
}
\par{The vertical velocity $v_z$ and pressure are related by 
\begin{equation}
    \frac{\partial P}{\partial z} = -j\omega \rho v_z,
\end{equation}
so I can write the vertical velocity equation as a Fourier transform as well:
\begin{equation}
    v_z(x,z) = \frac{1}{-j\omega \rho}\int_0^\infty \gamma(k)k\frac{\sinh{[k(z-h)]}}{\cosh{kh}}e^{-jkx}\;dk.
\end{equation}
}
\subsection{Arriving at the Impedance Relation}
\par{We are particularly interested in the pressure and velocity at the OCC, $z=0$, as we have an impedance relationship there. At $z=0$,
\begin{equation}
    P(x,0) = \int_0^\infty \gamma(k)e^{-jkx}\;dk.
\end{equation}
\begin{equation}
    v_z(x,0) = \frac{1}{j\omega \rho}\int_0^\infty \gamma(k)k\tanh{kh}e^{-jkx}\;dk.
\end{equation}
}
\par{For eventual use, I define $W(k) = \gamma(k)k\tanh{kh}$. It is clear from the integral above that this is the inverse Fourier transform of $v_z(x,0)$. I can write the integral for $P$ in terms of $W$ as
\begin{equation}
    P(x,0) = \int_0^\infty W(k)Q(k)e^{-jkx}\;dk,\;\;\;Q(k) = \frac{1}{k\tanh{kh}}.
    \label{qform}
\end{equation}
Knowing the relationship between pressure and velocity at the BM to be $2 P Y_{OC} = v_z$, I can write the integral formula for a 2-D box model:    
\begin{equation}
    -2j\omega \rho Y_{OC} \int_0^\infty W(k)Q(k)e^{-jkx}\;dk = \int_0^\infty W(k)e^{-jkx}\;dk.
    \label{intformappb}
\end{equation}
}
\par{This formula is not reliant on the WKB assumption, and is far more general than that developed in the main text. In fact, a similar form persists in 3-D models or models of other shapes, where $Q(k)$ represents a sort of ``geometric factor." Here, $Q$ is precisely the effective height.}
\subsection{Connection to the WKB Approximation}
\label{app:integralWKB}
\par{The Lagrangian method begins with the assumption that OCC velocity is a wave with a single mode, i.e. not a superposition of terms as in the more general integral formulation. This assumption, when combined with the integral formulation, yields the WKB dispersion relation.}
\par{To see this -- what is the inverse Fourier transform of a wave with a single mode? We know that 
\begin{equation}
w(x) = w_0e^{-jk x} \xrightarrow{\mathcal{F}^{-1}} 
W(\kappa) = w_0\delta(\kappa-k).
\end{equation}
}
\par{Plugging this in to Eqn \ref{intformappb}, 
\begin{equation}
    -2j\omega Y_{OC} w_0\int_0^\infty \delta(\kappa-k)Q(k)e^{-j\kappa x}\;d\kappa  = w_0 \int_0^\infty \delta(\kappa -k)e^{-j\kappa x}\;d\kappa.
\end{equation}
Using the sifting property of the $\delta$ function, this is
\begin{equation}
    -2j\omega Y_{OC} w_0 Q(k)e^{-jkx} = w_0 e^{-jkx}.
\end{equation}
}
\par{Dividing like terms and using the form of $Q(k)$ from Eqn \ref{qform} give the dispersion relation
\begin{equation}
    \frac{1}{Q(k)} = k\tanh{kh} = -2j\omega\rho Y_{OC}.
\end{equation}
This dispersion relation was derived through both 2-D WKB methods shown in the main text, and is characteristic of the 2-D WKB approximation.
}

\pagebreak







\clearpage
\printbibliography
\end{document}
