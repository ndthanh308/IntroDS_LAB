High-level approaches to big data analytics such as Hadoop MapReduce~\cite{WHI2010:OR} or Apache Spark~\cite{ADD2015:PVLDB} are often inspired by bulk synchronous parallelism (BSP)~\cite{VAL1990:CACM} a model of scalable parallel computing. In this context, scalable means that the number of processors of the parallel machines running BSP programs could range from a few to several dozens of thousand cores or more. Bulk Synchronous Parallel ML (BSML)~\cite{LGB2005:ICCS} is a pure functional library for the multi-paradigm language OCaml\footnote{\url{https://ocaml.org}}. BSML embodies the principles of the BSP model, at a higher level than libraries such as the BSPlib library~\cite{HILL1998:BSPLIB} and can easily express patterns~\cite{GG2009:APDCM,LOU2017:PDCAT} (or algorithmic skeletons~\cite{COL1989:BOOK}) of frameworks such as MapReduce or Spark.

\why~\cite{BFM2014:STTT,why3_1_6} is a framework for the deductive verification of programs. It provides a specification and programming language named WhyML which can be used directly or as an intermediate language for other tools to verify C~\cite{KKP2015:FAC}, Java\cite{FM2007:CAV}, Ada or Rust~\cite{DJM2022:IFCEM} programs. The framework itself also provides mini-C and mini-Python front-ends. \why generates verification conditions to be verified by external provers. One of the strength of \why is that it targets a large variety of provers including Alt-Ergo~\cite{CCI2018:SMT}, Z3~\cite{MB2008:TACAS} and CVC5. Correct-by-construction OCaml code can be extracted from WhyML.

Our contributions are the formalization of BSML and its standard library in WhyML and its use in the specification and verification of a scalable parallel function for the maximum prefix sum problem, using map and reduce skeletons.

The remaining of the paper is organized as follows. In Section~\ref{sec:why}, we give an overview of \why and WhyML, including its limitations when dealing with higher-order functions. We introduce functional bulk synchronous parallel programming with BSML in Section~\ref{sec:fbsp}. Section~\ref{sec:bsmlwhy} is devoted to the formalization of the primitives of BSML and its application to the specification and verification of the BSML standard library. We consider the specification, development and verification of a small application: a parallel function that solves the maximum prefix sum problem in Section~\ref{sec:application}. We discuss related work in Section~\ref{sec:related_work} and conclude in Section~\ref{sec:conclusion}.

The set of \why modules is called WhyBSML and is available at \doi{10.5281/zenodo.8166092}.