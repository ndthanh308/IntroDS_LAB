We were able to formalize the primitives of the parallel programming library BSML with WhyML and leverage \why for verifying a large part of the BSML standard library as well as an application written in BSML. We plan to experiment the extracted code more thoroughly and on larger parallel machines with a few thousand cores.

WhyML offers exceptions and references thus allows to write imperative programs. However, such programs cannot be passed as arguments to higher-order functions. It therefore limits the usage of imperative features with BSML as all primitives are higher-order functions. The code outside BSML primitives can be imperative thus the sequencing of BSP super-steps could be imperative. It is also possible to use imperative features to implement pure functions passed as arguments to BSML primitives. Also, it is possible to deal with partial functions as we did with "remove_some". We plan to explore all these possibilities in the future.