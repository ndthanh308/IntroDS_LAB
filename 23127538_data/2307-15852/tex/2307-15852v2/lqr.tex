%%%%%%%%%%%%%%%%%%%%%%%%%%%%%%%%%%%%%%%%%%%%%%%%%%%%%%%%%%%%%%%%%%
In this section, we show that the policy given by Equation \eqref{eq:lqr_policy} is optimal with respect to the LQR problem defined in Section \ref{sec:lqr}. We can write the equation of motion given by Equation \eqref{eq:pendulum_dynamics_lqr} in state-space form, using $G=mgl$ and $H=ml^2$, as follows:
%%%%%%%%%%%%%%%%%%%%%%
\begin{equation}
\frac{d}{dt} 
\begin{bmatrix}
    \theta \\
    \dot{\theta} 
\end{bmatrix}
= 
\underbrace{
\begin{bmatrix}
    0 & 1 \\
    G/H & 0 
\end{bmatrix}
}_{A}
\underbrace{
\begin{bmatrix}
    \theta \\
    \dot{\theta} 
\end{bmatrix}
}_{x}
+
\underbrace{
\begin{bmatrix}
    0 \\
    1/H
\end{bmatrix}
}_{B}
\underbrace{
\begin{bmatrix}
    \tau
\end{bmatrix}
}_{u}
\end{equation}
%%%%%%%%%%%%%%%%%%%%%%

Then, by adapting a solution from \cite{hanks_closed-form_1991}, if we parameterize the weight matrix of the cost function as follows:
%%%%%%%%%%%%%%%%%%%%%%
\begin{equation}
J = \int_0^{\infty}{ \left( x^T 
\underbrace{
\begin{bmatrix}
    a(a-2G) & 0 \\
    0       & b^2 - 2 a H
\end{bmatrix}
}_{Q} x
+
u^T 
\underbrace{
\begin{bmatrix}
    1 
\end{bmatrix}
}_{R} u
\right) dt } 
\end{equation}
%%%%%%%%%%%%%%%%%%%%%%
the optimal cost-to-go is given by:
%%%%%%%%%%%%%%%%%%%%%%
\begin{equation}
J =   x^T 
\underbrace{
\begin{bmatrix}
    b(a-G) & aH \\
    aH       & bH
\end{bmatrix}
}_{S} x
\end{equation}
%%%%%%%%%%%%%%%%%%%%%%
and the optimal feedback policy is given by:
%%%%%%%%%%%%%%%%%%%%%%
\begin{equation}
u = 
- \underbrace{\left[ R^{-1} B^T S\right]}_{K} x
=
-
\underbrace{
\begin{bmatrix}
    a & b \\
\end{bmatrix}
}_{K} x
\label{eq:kx}
\end{equation}
%%%%%%%%%%%%%%%%%%%%%%
This solution can by verified by substituting matrices into the algebraic Riccati equation given by:
%%%%%%%%%%%%%%%%%%%%%%
\begin{equation}
0 = SA + A^T S - SBR^{-1}B^TS + Q
\end{equation}
%%%%%%%%%%%%%%%%%%%%%%
since the problem fits into the framework of the classical infinite horizon LQR result \cite{bertsekas_dynamic_2012}. Then, we can see that the cost function defined in Section \ref{sec:lqr} is a special case, where $Q_{11} = q^2$ and $Q_{22}=0$, leading to the following equations:
%%%%%%%%%%%%%%%%%%%%%%
\begin{align}
q^2 &= a(a-2G) \\
0   &= b^2 -2aH
\end{align}
%%%%%%%%%%%%%%%%%%%%%%
Solving for $a$ and $b$, and retaining the positive solution, leads to the following:
%%%%%%%%%%%%%%%%%%%%%%
\begin{align}
a &= G + \sqrt{G^2+q^2} \\
b &= \sqrt{2aH} = \sqrt{2H \left(G + \sqrt{G^2+q^2} \right)}
\end{align}
%%%%%%%%%%%%%%%%%%%%%%
which, when substituted into Equation \eqref{eq:kx}, is equal to the policy given by Equation \eqref{eq:lqr_policy} in Section \ref{sec:lqr}.