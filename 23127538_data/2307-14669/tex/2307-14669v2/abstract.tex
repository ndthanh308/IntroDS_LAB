\OSF (\osf) logic is a knowledge representation and reasoning language
based on
function-denoting \textit{feature symbols}
and
set-denoting \textit{sort symbols}
ordered in a subsumption lattice.
\osf logic allows the construction of record-like terms that represent
classes of entities and that are themselves ordered in a subsumption
relation. The unification algorithm for such structures provides an
efficient calculus of type subsumption, which has been
applied in computational linguistics and
implemented in constraint logic programming languages
such as LOGIN and LIFE
and automated reasoners such as CEDAR.
This work generalizes
\osf logic to a fuzzy setting. We give
a
flexible definition of a \textit{fuzzy subsumption} relation
which generalizes Zadeh's
inclusion between fuzzy sets. Based on this definition we define a fuzzy
semantics of \osf logic
where sort symbols and \osf terms denote fuzzy sets.
We extend the subsumption relation to \osf terms and prove that it
constitutes a fuzzy partial order
with the property that
two \osf terms are subsumed by one another in the crisp sense
if and only if
their
subsumption degree is greater than 0.
We show how to find the greatest lower bound of two \osf terms by unifying
them and how to compute the
subsumption degree between two \osf terms,
and we provide the complexity of these operations.%
