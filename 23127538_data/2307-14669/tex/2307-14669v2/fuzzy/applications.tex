\paragraph{Application to fuzzy logic programming}%

A possible application of \osf logic with a fuzzy subsumption relation is the
implementation of a logic programming language (like LOGIN \cite{AitKaci1986b}) based
on SLD resolution and fuzzy \osf term unification, i.e., where first-order term
unification is replaced by the unification of \osf terms over a fuzzy sort
subsumption lattice.
Logic programming languages and weaker definitions of unification based on fuzzy
relations such as similarities and proximities have been researched extensively
(e.g.,
\cite{Arcelli2002,Gerla1999,Sessa2002,Iranzo2015b,Iranzo2020,Kutsia2020,AitKaciPasi2020})
and implemented in systems such as \bpl \cite{Iranzo2023} and FASILL
\cite{Iranzo2020fasill}.%

There are several potential advantages to using \osf logic in this context.
First of all, the unification
algorithm for \osf terms takes into account a (fuzzy) sort subsumption relation,
which can result in more efficient computations \cite{AitKaci1986b,Cohn1989}.
Another advantage is the flexibility provided by \osf terms,
which lack a fixed arity and can thus easily represent partial information,
and are moreover simpler to interpret thanks to their use of features rather than
positions to specify arguments \cite{AitKaci1986b}.

\paragraph{Application to similarity-based reasoning}%

Another application of our definition of fuzzy
subsumption relation is similarity-based reasoning.
A possible way to define a fuzzy subsumption relation
is by taking advantage of a given similarity relation $\mathop{\sim}:
\S\times\S\to[0,1]$ between sort symbols in order to enrich a crisp
subsumption relation $\mathop{\isa}\subseteq\S\times\S$ according to the
following intuitive inference rule, which also motivates similarity-based
approaches in logic programming (e.g., \cite{Sessa2002}):
    if   $\s\isa\su$ and $\simop(\su, \s[s'']) = \beta$,
    then $\fisop(\s, \s[s'']) = \beta$.
For example,
    if   $\s[slasher]\isa\s[horror]$
    and  $\simop(\s[horror], \s[thriller]) = \beta$,
    then $\fisop(\s[slasher], \s[thriller]) = \beta$.
This would mean, for example, that if $h$ is an instance of
$\slasher$ with degree $1$, then $h$ must also be an instance of
$\thriller$ with degree greater than or equal to $\beta$.

This approach could be behind the implementation of a fuzzy logic
programming language similar to \bpl \cite{Iranzo2023}, where
the similarity-based
or (proximity-based) first-order
term unification of this language
may be replaced by similarity-based \osf term unification, with the potential
advantages discussed above.

\paragraph{Fuzzy CEDAR}%

Finally, we plan to apply fuzzy \osf logic to define a fuzzy
  version of CEDAR\footnote{%
  More information about the CEDAR project
  can be found at \url{https://cedar.liris.cnrs.fr/},
  including reports, papers, demos and software.},
  a very efficient Semantic Web reasoner based on \osf
  logic \cite{AitKaciAmir2017,AmirAitKaci2017}. One of the
  capabilities supported by CEDAR is the optimization of a query expressed
  as an \osf term according to the knowledge expressed in a given
  ontology, which
  can significantly simplify the original query and reduce the
  instance retrieval
  search space.
  The consistency of the input query against the ontology is also ensured
  during this process, and no answer is provided if the query is
  inconsistent.
  The retrieval is further optimized thanks to a custom RDF triple indexing
  scheme based on \osf sort and attribute information
  \cite{AmirAitKaci2017}.

A fuzzy extension of CEDAR could relax the consistency
  requirement to
  provide approximate answers when the input
  query is inconsistent with the
  \textit{crisp} knowledge expressed by the ontology.
  This would be achieved, for example, by
  enriching the subsumption relation of the given ontology according to a
  similarity relation (with the procedure outlined above) in order to
  obtain a fuzzy subsumption, which would then be taken into
  account during the query optimization phase.%
