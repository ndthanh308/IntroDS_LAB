\section{Fuzzy \osf logic: syntax}%
\label{cha:syntax}

In this section we provide the definitions of fuzzy \osf signature
and of two formal languages
that are used to
represent knowledge with \osf logic: \textit{\osf terms} and \textit{\osf clauses}.
As discussed in the introduction, \osf terms are comparable to the defined
concepts of DLs, and they will be interpreted as fuzzy subsets of a domain
of interpretation. An \osf clause is an equivalent representation that can
be seen as a logical reading of an \osf term, and for which a notion of graded
satisfaction will be defined. In \osf logic, both syntactic
representation are important from an implementation perspective, as \osf
terms are the abstract syntax employed by an user, while \osf clauses
are used in the constraint normalization rules needed for \osf term
unification
\cite{AitKaci1993b}.

As far as syntax is concerned, the \textit{only difference with crisp \osf
logic} is that we consider a fuzzy sort subsumption relation rather than a
crisp one (the most significant differences concern the
  semantics, as will be seen starting from \cref{cha:semantics}). The
  definitions of \osf terms and \osf constraints and the result concerning
  their equivalence are reported from \cite{AitKaci1993b} for
  self-containment.

\begin{definition}[Fuzzy \osf signature]
\label{def:sort_signature}
A \emph{fuzzy \osf signature} is a tuple $\fss$ where
  $\S$ is a set of \emph{sort symbols},
  $\F$ is a set of \emph{feature symbols}, and
  $\ftax$ is a fuzzy finite bounded lattice with least element
    $\bots$ and greatest element $\tops$\footnotemark.
\footnotetext{See \cref{def:fuzzy_lattice} in \cref{app:fuzzy}.}
Elements of $\S$ and $\F$ will also simply be called \emph{sorts} and
\emph{features}, respectively.
The greatest lower bound (GLB) $\s\fmeet\su$ of two sorts $\s$ and $\su$ is
also called their \emph{greatest common subsort}.
\end{definition}

\begin{example}[Fuzzy \osf signature]
\label{ex:fuzzy_osf_signature}
  As an example of a fuzzy \osf signature we may take the set of sorts and
  the fuzzy subsumption relation corresponding to the graph of
  \cref{fig:fuzzy_sub_small}, and $\F = \{ \f[directed\_by], \f[title] \}$
  as the set of features.
\end{example}

\begin{definition}[\osf term \cite{AitKaci1993b}]
\label{def:osf_term}
Let $\V$ be a countably infinite set of variables (or
\emph{coreference tags}, or simply \textit{tags}). Let
$\X\in\V$, $\s\in\S$ and $\fj, \ldots,
\fj[n]\in\F$. An \emph{\osf term} is defined recursively as follows.
\begin{itemize}
    \item A sorted variable $\xs$ is an \osf term.
    \item If $t_1, \ldots, t_n$ are \osf terms, then an attributed sorted
      variable $t = \osfterm$ is an \osf term.
\end{itemize}
We let $\tags(t) \defeq \{ \X \} \cup \bigcup_{1\leq i \leq
n}\tags(t_i)$.
The variable $\X$ is called the \textit{root tag} of $t$ and is denoted
$\rtag(t)$.
\end{definition}



\begin{example}[\osf term]
  \label{ex:osf_term}
  The following \osf term shows how variables can be used to indicate
  coreference. Variables that are not used for coreference may be left
  implicit to improve readability.
    In line with \cite{AitKaci1993b}, we adhere to the convention of
    specifying the sort of each variable at most once, with the implicit
    understanding that other occurrences also refer to the same structure.
  \[
    \s[movie]
    \left(
      \begin{array}{lll}
        \f[title] & \to & \s[string],\\
        \f[directed\_by] & \to & \X[X]:\s[director]
        \left(
          \begin{array}{lll}
            \f[name] & \to & \s[string],\\
            \f[spouse] & \to & \X[Y]
          \end{array}
        \right),\\
        \f[written\_by] & \to & \X[Y]:\s[writer]
        \left(
          \begin{array}{lll}
            \f[spouse] & \to & \X[X]
          \end{array}
        \right)
      \end{array}
    \right).\qedhere
  \]
\end{example}

The definition of \osf terms given above does not rule out the presence of
redundant or even contradictory information (e.g., consider the \osf term
$\s(\f\to \si[0], \f\to\si[0], \f\to\si)$, which is contradictory if
$\si[0]\fmeet\si = \bots$). \osf terms that are well-behaved to this regard
are called
\ifrefer%
  \textit{normal \osf terms}, %
  and the reader is referred to \cite{AitKaci1993b} for their definition.
  \osf terms in normal form are also called \textit{$\psi$-terms} and
  denoted $\psi$, $\psi_i$, and so on.
  For an \osf term $\psi$ in normal form and $\X\in\tags(\psi)$, we let
  $\sort_\psi(\X)$ be the most specific sort $\s$ such that $\X:\s$ appears in
  $\psi$.
  The notation $\X\fto_\psi\Y$ indicates that there is a feature $\f$
  pointing from a subterm of $\psi$ with root tag $\X$ to a subterm of
  $\psi$ with root tag $\Y$.
  We let $\Psi$ denote the set of all normal \osf terms.%

\else
\textit{normal \osf terms} and are defined as follows \cite{AitKaci1993b}.
\begin{definition}[Normal \osf term, or $\psi$-term \cite{AitKaci1993b}]
\label{def:osf_term_normal form}
  An \osf term $t=\osfterm$ is in \textit{normal form} (or \textit{normal})
  if: (i) the root sort $\s$ is different from $\bots$,
  (ii) the features $\fj[1], \ldots, \fj[n]\in\F$ are pairwise distinct,
  (iii) each $t_i$ is in normal form, and
  (iv) for all $\Y\in\tags(t)$, there is at most one occurrence of $\Y$ in
  $t$ such that $\Y$ is the root variable of an \osf term different from
  $\Y:\tops$.

  \osf terms in normal form are also called \textit{$\psi$-terms} and
  denoted $\psi$, $\psi_i$, and so on. For an \osf term $\psi$ in normal
  form and $\X\in\tags(\psi)$, we let
  $\sort_\psi(\X)$ be the most specific sort $\s$ such that $\X:\s$ appears in
  $\psi$.
  The notation $\X\fto_\psi\Y$ indicates that there is a feature $\f$
  pointing from a subterm of $\psi$ with root tag $\X$ to a subterm of
  $\psi$ with root tag $\Y$.
  We let $\Psi$ denote the set of all normal \osf terms.
\end{definition}
\fi


\section{$Q_{\star}^{\prime}$ Constraints}
%
\label{sec:constraints}

\subsection{Individual Constraints}

From the MCMC we obtain posterior samples for $Q_0$, $P_{\text{break}}$ and
$\alpha$ for each system. These samples are converted to $\Q(P_\text{tide})$
using Equation~\ref{eq:q_formalism} evaluated at 30 separate tidal periods
$\log-$uniformly spaced between 1 and 50 days. For each tidal period we apply
the convergence diagnostics described in Section~\ref{sec:convergence} to find
the 2.3, 15.9, 84.1, and 97.7 percentiles of $\Q$ at that tidal period
(corresponding to $\pm1\sigma$ and $\pm2\sigma$ uncertainties) together with the
burn-in steps and cumulative distribution uncertainty.

Figures \ref{fig:ind_const_1} and \ref{fig:ind_const_2} show the KDE estimate
of the posterior distribution of $\Q$ at each tidal period (color-coded heat
map), the 2.3, 15.9, 84.1, and 97.7 percentiles (black dotted or solid curves),
the $50^{\text{th}}$ percentile as the thin horizontal red curve, and the orbital period
(thick vertical black line). The solid parts of the quantile curves, delineated
with vertical red lines, mark the range of periods for which the $\Q$
distribution and quantiles are reliable, i.e. dominated by the data rather than
the priors and not affected by potentially under-estimated uncertainty of the
spin (see Sec. \ref{sec:reliable_constraints} for details).

% Figure environment removed

% Figure environment removed

Figures \ref{fig:burnin_1} and \ref{fig:burnin_2} show the estimated
burn-in periods (see Sec. \ref{sec:convergence}) for the four quantiles of $\Q$
at each tidal period as the curves (solid corresponding to the reliable range
of periods, dotted otherwise), as well as the total number of steps we
accumulated for each system (black area). As that figure demonstrates, for the
tidal periods where a given quantile is deemed reliable (see Sec.
\ref{sec:reliable_constraints}), for all our systems we have accumulated enough
samples to move past the burn-in period.

% Figure environment removed

% Figure environment removed

Finally, Figures \ref{fig:cdfstd_1} and \ref{fig:cdfstd_2} show the estimated
standard deviation of the cumulative distribution function (CDF) at each
quantile estimated as described in Sec. \ref{sec:convergence} from the samples
after burn-in is discarded.

% Figure environment removed

% Figure environment removed

\subsection*{Two-sided limits on $\Q$}

The systems for which we obtain two-sided limits significantly differ between
the orbital period and the spin period of the primary star and significantly
deviate from the spin period of an isolated star given the same mass and age.
For example, in the case of \texttt{KIC11232745}, the orbital period of the
system is 9.6 days, while the spin period is 12.7 days. The lower bound on
$Q_{\star}^{\prime}$ comes from not allowing high dissipation, which would cause
the primary star to synchronize with the orbit. The upper bound on
$Q_{\star}^{\prime}$ is due to the requirement of minimal dissipation such that
the tidal influence on the spin of the star is not negligible compared to the
effect of stellar winds. Given the very steep drop off of the effects of tides
with orbital separation (tidal torque decreases as the fourth power of the
orbital period), at first glance it may appear that such systems should be so
rare that finding several in a sample of 70 binaries would be surprising.
However, two-sided limits on $\Q$ correspond to stars with spin periods
comparable (though not quite equal) to the orbital period. As a result, since
stellar wind torques scale strongly with the spin period (third power of the
spin period), which is itself related to the orbital period, the transition from
fully synchronized systems to systems unaffected by tides is much more gradual
than thinking about tides alone would predict.

\subsection*{Upper Limit on $\Q$}

There are systems where we only obtain an upper bound on $\log_{10}\Q$ (lower
bound on the dissipation). In these cases, the spin period of the primary star
is synchronized with the orbit. Above this limit, the dissipation is not high
enough to maintain a spin-orbit lock and keep the star synchronized with the
orbit. However, reproducing the observed state of the system can tolerate
arbitrarily large dissipation (arbitrarily small $\Q$) since that will just
result in the spin-orbit lock being achieved earlier.

\subsection*{Lower Limit on $\Q$}

Similarly, there are systems with only a lower bound on $\log_{10}\Q$ (an upper
bound on the dissipation). This limit is obtained for systems whose spins are
consistent with that of isolated stars of similar mass and age. The inability to
distinguish the spin from that of an isolated star is generally driven by the
fact that the age is usually the least well constrained parameter.  We stress
that the non-tidal parameters we assume (core-envelope coupling timescale, wind
strength, wind saturation frequency etc.) were tuned to reproduce the observed
spins of isolated stars with well known ages (ones residing in open clusters).
Hence, if we calculate the evolution of a binary under the assumption of
$Q_\star' \rightarrow \infty$, the predicted spin of each star in the binary is
the same as that of an isolated star with the same mass and age. The lower limit
to $Q_\star'$ produced by these systems is thus a statement that if the
dissipation were any larger than this, it would have had a detectable effect on
the spin.

For some systems, the lower bound on $\log_{10}{Q_{\star}^{\prime}}$ can be due
to eccentricity rather than the spin period of the primary star. It may be that
the uncertainty in the age does not allow us to distinguish the spin from that
of an isolated star, but if non-zero orbital eccentricity is detected with high
significance, dissipation must be weak enough to avoid circularizing the system.

\subsection{Reliable Portion of Individual $Q_{\star}^{\prime}$ Constraints}
%
\label{sec:reliable_constraints}

Examining the posterior distributions of $\log_{10}\Q$ in Fig.
\ref{fig:ind_const_1} and \ref{fig:ind_const_2}, it is clear that each
individual binary is sensitive to the dissipation only in a relatively narrow
range of periods. For example, the difference between the median and the 97.7$^{\text{th}}$
percentile of the distribution has a sharp minimum. Sufficiently far away from
that period, the increase of the 97.7$^{\text{th}}$ percentile is limited to large
extent by the fact that we restrict the powerlaw index $\alpha$ in Eq.
\ref{eq:q_formalism} to a maximum value of 5. Similarly, for some systems, the
small quantiles peak at some period, and, away from that, the slope is limited by
the assumption that $\alpha>5$.

In order to select the region where constraints are dominated by the observed
state of the binary rather than the priors, we evaluate the absolute difference
between the highest/lowest quantiles and the median at each tidal period. We
then select the period range where this difference remains within 0.5 dex of its
minimum value. The upper/lower half of the distribution is then only deemed
reliable within that period range.

For the systems for which we obtain a lower bound or two-sided bound on
$\log_{10}\Q$, we must further examine the difference in the measured spin
period and the orbital period. Generally, the statistical uncertainty in
measuring the spin period is small compared to stellar differential rotation. As
described in Sec.  \ref{sec:rotation_periods}, we estimate that difference using
the detection of multiple peaks in the Lomb-Scargle periodogram. However, if the
star spots modulating the lightcurve of a particular star happen to all come
from a narrow rang of latitudes at the time they were observed, our approach
will under-estimate the amount of differential rotation. The result could be a
star which synchronized the spin to the orbit, but for which, due to
differential rotation, the LC modulation happens to have a slightly different
period and the difference appears significant because we under-estimate the
amount of differential rotation. In turn, this will produce an upper limit to
the dissipation (lower limit on $\Q$) that is not justified by the observations.
To compensate for this possibility, we trust only the upper limit to
$\log_{10}\Q$ our analysis gives for systems with measured spin periods within
17\% of the measured orbital period (approximately the amount of differential
rotation between the poles and equator of the Sun) and ignore the lower limit.
Suggestively, the distribution of fractional difference between spin and orbital
period for our systems has a gap between 17\% and 21\%, with most systems below
17\%, perhaps hinting at a switch from a population of synchronized to
non-synchronized systems, though the sample is not large enough to claim this is
statistically significant or to explore possible explanations.

%We selected posterior samples for the orbital and stellar parameters provided
%by \citet{Windemuth_2019} as distribution functions for bayesian analysis.
%\citet{Windemuth_2019} used uniform priors, $U(0,1)$ on the two eccentricity
%components: $e\cos{\omega}, e\sin{\omega}$ in their analysis. We transformed
%these samples to eccentricity using the relation: $e = \sqrt{{e\cos{\omega}}^2
%+ {e\sin{\omega}}}^2$ and sampled from the probability distribution as shown in
%Equations. This methodology requires a correction in the probability
%distribution of eccentricity by multiplying it by a factor $\dfrac{1}{e}$.
%Otherwise, there is a higher preference for non-zero eccentricities in the
%probability distribution. Avoiding this correction can result in a lower limit
%on $Q_{\star}^{\prime}$ for circular systems, which comes from non-zero
%eccentricity rather than synchronization. To correct for this, we visually
%inspect the posterior samples of $e\cos{\omega} \text{ vs } e\sin{\omega}$ for
%each system and identify the system where zero eccentricity is within the
%spread of the two distributions (Figure). Next, from this subset, we identify
%the systems that show high likelihood at low $Q_{\star}^{\prime}$ values. We
%discard the lower limit for these systems and renormalize the probability
%distribution for $Q_{\star}^{\prime}$ at each tidal period.

\subsection{Combined Constraints}
%
As expected, individual constraints are sensitive to a limited range of
frequencies, and most binaries only produce upper or lower limits, but only a
few produce both. As a result, meaningful measurement of $\Q$ can only be
obtained by combining the individual constraints. Furthermore, since we selected
all our binaries to have similar internal structure, it is not unreasonable to
assume they should have similar dissipation.

A common constraint, which agrees with all the constraints obtained by the
individual binary systems, can be found by multiplying the posterior probability
density functions of $\Q(P_\text{tide})$ for the binary systems for each
$P_\text{tide}$, obtained using KDE. This of course is complicated by the
considerations described in Sec. \ref{sec:reliable_constraints}. For a given
system, at a given tidal period, we have already identified whether we trust
both halves of the distribution, just the upper half, just the lower half, or
neither. If neither side of the distribution is reliable, we simply do not
include that system in the product for the given tidal period. If only one side
of the distribution is reliable, we replace the unreliable part of the
probability density with the probability density at the median and re-normalize
before including the distribution in the product for the combined distribution.

One of our binaries -- KIC 7816680 -- produces constraints that are in conflict
with the combined constraint from the other binaries for all tidal periods above
7 days. An outlier binary can have many explanations:
%
\begin{enumerate}
%
    \item There may be additional objects in the system resulting in a very
        different orbital evolution than we calculate assuming just two objects
%
    \item A third object in the system or another star along the same line of
        sight could be contributing extra light to the system. Since W19 models
        the spectral energy distribution assuming just the two binary components
        that could shift the inferred properties of the binary.
%
    \item Spin period measurements sometimes detect an alias of the spin period,
        or perhaps the spot modulations come from the secondary star instead of
        the primary, invalidating our assumption that measured spin is that of
        the brighter star.
%
    \item The stellar evolution models used in the W19 analysis to find binary
        parameter distributions and in our orbital evolution calculations may
        not be applicable. In fact, W19 already report discrepancies for young
        stars and caution discrepancies are also expected for post-main-sequence
        stars, and POET stellar evolution interpolation is unreliable past the
        end of the main sequence as it was never designed to handle such stars.
%
\end{enumerate}

We suspect KIC 7816680 falls in the final category. The primary star has a
maximum likelihood mass of $1.2M_\odot$ and maximum likelihood age of 9.4\,Gyr,
hence the W19 model of this binary involves a post-MS primary, so some amount of
difficulty with stellar evolution models is expected.

Fig. \ref{fig:combined} shows the combined constraint on $\log_{10}\Q$ obtained
through the above procedure, excluding KIC 7816680.

An important question that must be considered is whether the combined
constraints are indeed consistent with all individual constraints. After all, it
is possible to multiply two mutually highly inconsistent distributions and
obtain a combined constraint that is somewhere in the middle that is highly
inconsistent with either distribution. Fig.~\ref{fig:combined_vs_individual}
shows the 2.3, 15.9, 50.0, 84.1, and 97.7 percentiles of the combined
constraints (same as the lines in Fig. \ref{fig:combined}) along with the 2.3
and 97.7 percentiles for each individual system at the tidal period where the
distance between the 2.3 and 97.7 percentiles is smallest. As can be seen in
Fig.~\ref{fig:combined_vs_individual}, there is at least some overlap between
the allowed $\log_{10}\Q$ values for each system and the combined constraint. In
other words, the combined constraint is capable of reproducing the observed
spins of all the binaries in our sample except KIC 7816680.

% Figure environment removed

% Figure environment removed


Finally, we report the constraint normalization rules from
\cite{AitKaci1993b} that are needed to transform an \osf clause into a
solved form. Each rule is of the form
\begin{center}
\osfrulec{\text{Side condition}}{\lab{Rule name}{osf:rn}}
{\text{Premise}~\phi}{\text{Conclusion}~\phi'}\\
\end{center}
and it expresses that, whenever the (optional) side condition holds, the
premise
$\phi$ can be simplified into the conclusion $\phi'$.
The rules of \cref{fig:osf_normalization} are the same as the ones from
\cite{AitKaci1993b}, except that we are considering the GLB operation
$\fmeet$ in a fuzzy lattice rather than an ordinary one. Thanks to
\cref{prop:glbs}, this does not constitute a significant difference as far
as the properties of the normalization procedure are concerned.
An \osf term $\psi$ can be normalized by applying the constraint
normalization rules to $\phi(\psi)$ and translating the result back into an
\osf term.%

\begin{restatable}%
  [{{\osf} clause normalization \cite{AitKaci1993b}}]%
    {theorem}{propclausenormalization}
\label{prop:osf_clause_normalization}
The rules of \cref{fig:osf_normalization} are finite terminating and
confluent (modulo variable renaming). Furthermore, they always result in a
normal form that is either the inconsistent clause or an \osf clause in
solved form together with a conjunction of equality constraints.
\end{restatable}

% Figure environment removed
