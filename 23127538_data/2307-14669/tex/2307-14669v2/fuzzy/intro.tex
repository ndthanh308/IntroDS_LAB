\section{Introduction}%
\label{cha:introduction}

Order-Sorted Feature (\osf) logic is a Knowledge Representation and
Reasoning language that originates in Hassan \ak's work \cite{AitKaci1984}
and, similarly to Description Logics (DLs), it was initially meant as a
formalization of Ron Brachmann's structured inheritance networks
\cite{DLKRHB,AitKaci2007b}.
\osf logic and related formalisms
-- e.g., feature logic \cite{Smolka1988} or the logic of typed feature
structures \cite{Carpenter1992} --
have been applied in computational linguistics \cite{Carpenter1992} and
implemented in constraint logic programming languages such as LOGIN
\cite{AitKaci1986b}, LIFE \cite{AitKaci1993b} and CIL \cite{Mukai1987} and,
more recently, in the very efficient CEDAR Semantic Web reasoner
\cite{AitKaciAmir2017,AmirAitKaci2017}.%

At the core of \osf logic are \textit{sort symbols} denoting conceptual classes such as
$\s[person]$ or $\s[student]$, and a \textit{sort subsumption relation}
that denotes inclusion between classes, comparable to the inclusion axioms
of a DL terminological box. Concept subsumption relations are a
central part of any ontology, and \osf logic and its
implementations rely on graph encoding techniques to perform Boolean
operations on a concept lattice very efficiently
\cite{AitKaciAmir2017,AitKaci1989}.
Besides sort symbols, \osf logic employs \textit{feature symbols} to
describe attributes of objects, like $\f[name]$, $\f[directed\_by]$ or
$\f[written\_by]$.
While feature symbols denote \textit{total functions}, which %
may
appear less versatile than the relational roles of DLs, versions
of \osf logic that support partial functions or relations have also been
defined
\cite{Smolka1988,Carpenter1992,AitKaci2007b}.

Together with \textit{variables}, also named \textit{coreference tags}, sort symbols and
feature symbols can be used to construct record-like structures called
\osf terms that can describe complex concepts, such as the
following, which denotes the class of movies that are written and
directed by the same person:%
\[
  \X[X] : \s[movie]
  \left(
    \begin{array}{lll}
      \f[directed\_by] & \to & \X[Y] : \s[person],\\
      \f[written\_by] & \to & \X[Y]
    \end{array}
  \right).
\]
The unification algorithm for such structures provides an efficient way
to decide whether two \osf terms are subsumed by each other, and, in
general, for finding the most general \osf term that is subsumed by both
terms, thus offering an efficient calculus of partially ordered types
\cite{AitKaci1986b}.%

This work presents a fuzzy generalization of Order-Sorted Feature (\osf)
logic in which both sorts and \osf terms
are interpreted as
\textit{fuzzy subsets} of the domain of an interpretation rather than crisp
subsets, and where the sort subsumption relation is generalized to a
\textit{fuzzy subsumption relation between sorts}, which is then
extended to a fuzzy subsumption between \osf terms.

The idea of a fuzzy generalization of \osf logic
was first introduced in \cite{AitKaciPasi2020}
where a
weaker notion of
first-order term
unification based on similarity relations between
term constructors is introduced that also allows mismatches
between functor arities and argument positions.
While the interpretation of sort symbols (concept names in DL lingo) as
fuzzy subsets of a domain and the notion of fuzzy or graded subsumption
have already been studied extensively within fuzzy DLs,
the generalization of the semantics of \osf logic to a fuzzy setting is
novel.
In particular, the way in which we interpret
fuzzy subsumption relations
departs significantly from their treatment in fuzzy DLs, which relies
on the fuzzy implication operator \cite{Straccia2014,Borgwardt2017}.

In the rest of this section we provide an informal presentation of our
definition of fuzzy sort subsumption and how it is extended to the terms
of \osf logic, and we motivate the development of fuzzy \osf logic by
outlining its possible implementation
as an extension of the CEDAR reasoner
\cite{AitKaciAmir2017,AmirAitKaci2017} that would be capable of providing
approximate solutions in retrieval applications,
and as a fuzzy logic programming language.

\paragraph{Fuzzy sort subsumption}%

Let $\S$ and $\F$ be (finite) sets of \textit{sort symbols}
  and \textit{feature symbols}, respectively, and let us fix
 an interpretation
  $\ii = \osfa$, i.e., a structure with a domain
  $\dom$ and a function $\cdot^\ii$ that provides an interpretation to the
  elements of $\S$ and $\F$\footnote{Fuzzy interpretations will be defined
  properly in \cref{cha:semantics}.}.
  In our fuzzy setting, a sort $\s\in\S$ is interpreted as a
  fuzzy subset\footnotemark{}
  $\s^\ii:\dom\to[0,1]$, while a feature symbol $\f\in\F$ is
  interpreted as a function $\f^\ii:\dom\to\dom$.
  We rely on the minimum t-norm (denoted $\land$) and the maximum
  t-conorm (denoted $\lor$).
  \footnotetext{The definitions of fuzzy sets, fuzzy binary relations and
  fuzzy orders are recalled in \cref{app:fuzzy}.}%


In the crisp setting, a subsumption relation $\mathop{\isa} \subseteq
\S\times\S$ is a binary relation that denotes set inclusion. A natural
fuzzy generalization of this notion is Zadeh's inclusion of fuzzy sets
 \cite{DuboisPrade1980}, according to which the subsumption $\s\isa\su$ has
the following meaning:
\begin{align}
  \label{eq:fuzzy_inclusion}
  \text{if}~\s\isa\su,\text{then}~\forall d\in\dom: \s^\ii(d)\leq\su^\ii(d).
\end{align}
That is, this definition of fuzzy inclusion requires
  that, whenever $d$ is an
instance of $\s^\ii$ with degree
$\s^\ii(d)=\beta\in[0,1]$, then $d$ must
also be an instance of $\su^\ii$ with a degree
\textit{greater than or equal to} $\beta$.
\input{figures/fuzzy_sub_small}

We consider a \textit{fuzzy subsumption relation}
as a way to model a weaker notion of inclusion, where the element $d$ can
be an instance of $\su^\ii$ with a degree
that \textit{may possibly be smaller than} its degree
of membership to $\s^\ii$.
Thus, we define a fuzzy subsumption relation as a fuzzy partial order,
i.e., a function $\fisa: \S^2\to[0,1]$ that associates a subsumption degree
$\beta$ with each pair of sort symbols $\si[0], \si[1]$ (and satisfying
every constraint of \cref{def:fuzzy_poset} in
\cref{app:fuzzy}) and having the following semantics:
\begin{align}
  \label{eq:fuzzy_subsumption}
  \text{if}~\fisop(\si[0], \si[1])=\beta,~\text{then}~\forall d\in\dom:
  \sii[0](d) \land \beta \leq \sii[1](d).
\end{align}
That is, any object $d$ which is an instance of $\sii[0]$ with degree
$\beta_0$ must also be an instance of $\sii[1]$ with a degree $\beta_1$
  that is \textit{greater than or equal to the minimum of}
$\beta_0$ and $\beta$.
Note that \cref{eq:fuzzy_inclusion} is a special case of this equation with
$\beta = 1$.
An example of a fuzzy sort subsumption
relation is given in \cref{fig:fuzzy_sub_small}\footnotemark{}.
\footnotetext{We represent a fuzzy subsumption relation $\fisa$ graphically
  as a weighted directed acyclic graph (DAG), of which $\fisa$ is the
reflexive and transitive closure (see \cref{def:fuzzy_transitive_closure}
in \cref{app:fuzzy}).}

\paragraph{Fuzzy \osf term subsumption and unification}%

Sort symbols are analogous to the primitive or named concepts
of DLs. Complex concepts can be expressed with
\textit{\osf terms},
structures built from \sortcol{sort symbols}, \featcol{\textit{feature
symbols}} (attributes) and \tagcol{variables}. For example, the term
\[\arraycolsep=2pt
  t_1 = \Xj[1]:\s[movie]
  \left(
    \begin{array}{lll}
      \dirby& \to & \Yj[1]:\person,\\
      \f[genre] & \to & \Zj[1]:\thriller
    \end{array}
  \right)
\]
denotes the class of movies directed by some person and whose genre is
thriller.

In this work we will show that
a fuzzy subsumption relation between sort
symbols can be extended to a fuzzy subsumption relation between \osf terms.
For example, consider the fuzzy subsumption relation depicted in
\cref{fig:fuzzy_sub_small}
and consider the term
\[
  \arraycolsep=2pt
  t_2 = \X[X_{2}]:\s[movie]
  \left(
    \begin{array}{lll}
      \f[title] & \to & \X[W_{2}]:\strng,  \\
      \f[genre] & \to & \X[Z_{2}]:\slasher, \\
      \dirby    & \to & \X[Y_{2}]:\director
    \end{array}
  \right).
\]
This term is more specific that $t_1$, as it provides
additional information by
introducing the feature title and by constraining its value
to be of sort $\strng$,
and because it defines more restrictive constraints,
by requiring
that the value of the feature $\dirby$ be of sort $\director$, which is
subsumed by $\person$, and
that the value of the feature $\f[genre]$ be of sort
$\s[slasher]$, which is subsumed by $\s[thriller]$ (with degree
\appdegree).
In this case we say that $t_1$ \textit{subsumes} $t_2$ with degree
$\appdegree$.

One of the reasoning tasks supported by \osf logic is deciding whether a
given term is subsumed by another, or in general finding the most general
term which is subsumed by two given terms. As we will see,
  this problem can be solved by their \textit{unification}, a
  process which aims to combine the constraints expressed by the two terms
  in a consistent way.
For example, $t_2$ is the unifier of term $t_1$ and the following
term:
\[ t_3 = \Xj[3]:\s[movie]
  \left(
    \begin{array}{lll}
      \dirby & \to & \Yj[3]:\s[director],\\
      \f[title] & \to & \X[W_3]:\strng,\\
      \f[genre] & \to & \Zj[3]:\s[horror]
    \end{array}
  \right).
\]
In particular, the value for the feature $\f[genre]$ in $t_2$ must be of
sort $\s[slasher]$, as this sort is subsumed by both $\s[thriller]$
and $\s[horror]$ (the values of $\f[genre]$ in $t_1$ and $t_3$), and it
is the most general one with this property. The unifier $t_2$ is
associated with a unification degree, which depends on the subsumption
degrees of its sorts with respect to the corresponding sorts in $t_1$ and
$t_3$. In this case the unification degree is $0.5$, due to
$\fisop(\slasher, \thriller) = 0.5$.



In this section we explore a few applications of the techniques
introduced in section~\ref{sec:mth}. First we consider the application
of the reweighting technique to an optimization problem. 
Second, we consider the application in Bayesian inference to obtain
the dependence of predictions on the parameters that characterize the
prior distribution.

\subsection{Applications in optimization}
\label{sec:opt}

As an example application of an optimization process we will consider
the probability density function
\begin{equation}
  p_\theta(x) = \frac{1}{\mathcal Z}\exp \left\{ -S(x;\theta) \right\}\,, \qquad \left(  \mathcal Z = \int {\rm d} x\, e^{-S(x:\theta)} \right) \,.
\end{equation}
with
\begin{equation}
  S(x; \theta) = \frac{1}{\theta_1^2+1} \left( x_1^2 + x_1^4 \right) + \frac{1}{2}x_2^2 + \theta_2 x_1x_2\,.
\end{equation}

The shape of $S(x;\theta)$ is inspired in the action of a quantum
field theory in zero dimensions, where $x_1$ and $x_2$ are two fields
with coupling $\theta_2$, while $\theta_1$ is related to the mass of
the field $x_1$.
Expectation values with respect to $p_\theta(x)$ are functions of
the parameters $\theta$.  

% Figure environment removed

As an example we consider the problem of minimizing $\mathbb
E_\theta[x_1^2 + x_2^2]$ (i.e. 
finding the values for $\theta$ that make $\mathbb
E_\theta[x_1^2+x_2^2]$ minimum). 
We have implemented two flavours of Stochastic Gradient Descent (SGD): the first -basic- one, having a constant learning rate, and the second one being the well-known
%both a basic stochastic gradient descent (with constant learning rate) and the
ADAM algorithm \cite{kingma2017adam}. It is worth noting at this
point that as a general concept, SGD implies a stochastic (but
unbiased) evaluation of the gradients of the objective function at
every iteration. While in typical applications in the ML community,
where the task is to fit some dataset, this is done by evaluating the
gradients at different random batches of the data, the present example
is different in that no data is involved. In this case, every
iteration of the SGD evaluates the gradients on the different Monte
Carlo samples used to approximate the objective function  $\mathbb
E_\theta[x_1^2 + x_2^2]$.  

%These algorithms require stochastic evaluations both of the functionand its gradient at arbitrary values of the parameters $\theta$. 
%Here we perform these evaluations via Monte Carlo sampling: we use a
Here we consider a simple implementation of the Metropolis Hastings algorithm in order to
first produce the samples $\{x^{\alpha}\}_{\alpha=1}^N \sim p_\theta(x)$. 
Second, we determine the reweighted expectation value truncated at
first order
\begin{equation}
  \frac{\sum w(x^{\alpha};\tilde\theta) \left[ (x^{\alpha}_1)^2+ (x_2^{\alpha})^2 \right]}{\sum w(x^{\alpha};\tilde \theta)} \approx \bar O + \bar O_i \epsilon_i\,,  
  \qquad \left( w(x^{\alpha};\theta) = e^{S(x^{\alpha};\theta) - S(x^{\alpha};\tilde \theta)}  \right)\,,
\end{equation}
where $\tilde \theta_i =  \theta_i + \epsilon_i$. 
This quantity gives an stochastic estimate of the function value
\begin{equation}
  \bar O = \frac{1}{N}\sum_{i=1}^N [x_1^{\alpha}]^2 + [x_2^{\alpha}]^2\,,
\end{equation}
and its derivatives
\begin{equation}
 \bar O_i \approx \frac{\partial \mathbb E_\theta[x_1^2+x_2^2]}{\partial \theta_i}\,.
\end{equation}

Figure~\ref{fig:sgd} shows the result of the optimization process. 
As the iteration count increases the function is driven to its minima,
while the values of the parameters approach the optimal values
$\theta_1^{\rm opt} = \theta_2^{\rm opt} = 0$. 

It is worth mentioning that in this particular example only $1000$ samples
were used at each step to estimate the loss function and its
derivatives. 
If one decides to use a larger number of samples (say $10^5$), the
value of the parameter $\theta_2$ is determined with a much better precision. 
Note that the direction associated
with $\theta_2$ is much flatter, and therefore its value affects much
less value of the loss function.

\subsection{An application in Bayesian inference}
\label{sec:bayesian}
The purpose of statistical inference is to determine properties of the
underlying statistical distribution of a dataset
$D=\{x_{i},y_{i}\}_{i=1}^{N}$. In many
  cases, the independent variables $x_i$ are fixed, and all the
  stochasticity is captured by the dependent variables $y_i$. As
  such,  
the data is assumed to be sampled from a certain model, specified by
the \textit{likelihood}, 
$p(y|x,\phi)$, which depends on a set of parameters $\phi$.
The Bayesian paradigm attributes a level of confidence to the model by
introducing the \textit{prior} 
$p_{\theta}(\phi)$, \textit{i.e.} an a priori distribution of the
models parameters, where in this context $\theta$ play the role of the
hyper-parameters specifying the prior. Following Bayes' rule, the
\textit{posterior} distribution $p_{\theta}(\phi|D)$ is computed
as\footnote{The normalization factor, 
  $p_{\theta}(D)$, called the evidence, or marginal likelihood, is $\phi$-independent
  and represents the probability distribution of the observed data, given the model.}:
\begin{equation}
  \label{eq:bayes}
  p_{\theta}(\phi|D) \propto p(D|\phi) p_{\theta}(\phi)~.
%  ~~~~
%  p(D|\phi) = \prod_{i=1}^N p(y_i|x_i,\phi)~
\end{equation}
The likelihood of the whole dataset, $p(D|\phi)$, is computed assuming independent data points following a Gaussian distribution:
\begin{equation}
  p(D|\phi) = \prod_{i=1}^{N}\mathcal N(y_{i}|f(x_{i};\phi),\sigma_{i})\,,
  \label{eq:likelihood}
\end{equation}
where $\sigma_i$ are the  uncertainties of the corresponding observations $y_i$ (and assumed here to be given), while the mean of the Gaussian is given by $f(x_i;\phi)$. 
% The posterior in expr.(\ref{eq:bayes}) is the distribution of $\phi$ given the observed data and assumptions.
From a practical standpoint, in addition to the normalization being,
in general, unknown, the usual complexity of the posterior
distribution makes this possibly highly dimensional integral difficult
to compute. The use of Monte Carlo techniques, in particular of the
HMC, is typical in this context.
We focus below on two types of predictions: 1) The variance of the
model parameters $\delta\phi_j^2 = \mathbb{E}_{p_\theta}[\phi_j^2] -
(\mathbb{E}_{p_\theta}[\phi_j])^2$, where $j=1,...,d$, being $d$ the
dimension of $\phi$, and 2) the variance of the output mean $\delta
f_t^2 = \mathbb{E}_{p_\theta}[f_t^2] -
(\mathbb{E}_{p_\theta}[f_t])^2$, where $f_t$ is a shorthand notation
for the output mean $f(x_t;\phi)$, evaluated at a new ``test''
datapoint $x_t$ \footnote{Note that $E_{p_\theta}[f_t]$ is analogous
  to the so-called ``predictive distribution'' of Bayesian inference,
  however here we focus on the expected value of the prediction mean,
  instead of the expected value of the likelihood of $y(x_t)$
  itself.}. 


We are interested in studying the dependence of these quantities on the choice of
hyperparameters $\theta$ that characterize the prior distributions. In
particular we will consider the case of Gaussian priors, and determine
the dependence of our predictions with the width of this Gaussian.

\subsubsection{Model and data set}

We generate a synthetic dataset (cf. Figure \ref{fig:dataset}) by defining the points on an irregular grid in the range $x_i\in[-1.0;1.0]$, such that
\begin{equation}
  y_i = f(x_i;\phi_{\rm true}) + \sigma_i\epsilon~,
\end{equation}
where the mean is a 3rd degree polynomial, $f(x;\phi)=\phi_0+\phi_1x + \phi_2x^2 + \phi_3x^3$, with $\phi_{\rm true} = (1,1,1,1)$; $\epsilon\sim{\cal N}(0,1)$ is sampled from a standard Gaussian, and we consider a heteroscedastic dataset by defining a noise $\sigma_i$ dependent on $x_i$. We adopt the same model in order to  make inference on the parameters $\phi$. 

% Figure environment removed

%The likelihood reads
%\begin{equation}
%  p(D|\phi) = \prod_{i=1}^{N}\mathcal N(y_{i}|f(x_{i},\phi),\sigma_{i})\,,
%\end{equation}
%where $\mathcal N(\mu|\sigma)$ is the usual Gaussian distribution of
%mean $\mu$ and variance $\sigma^2$. As a model we choose a third
%degree polynomial $f(x,\phi) = \phi_{0} + \phi_{1}x + \phi_{2}x^{2}+\phi_{3}x^{3}$. 
%Note that this is the model that was also used to obtain the dataset.

The prior distribution is also chosen as a Gaussian, $\phi\sim {\cal N}(\mu_p,\sigma_p)$. 
For simplicity we choose the priors centered on the ``correct'' values
of the model (i.e. $\mu_p =\phi_{\rm true}$), while we keep the width 
$\sigma_p$ as a hyperparameter to study the dependence on\footnote{This is a simplified setup for the sake of illustration, given the methodological scope of this work. Nonetheless, it is straightforward to apply the method to the situation where we are interested in studying the dependence on both parameters $\mu_p$ and $\sigma_p$ simultaneously, or in general on the joint set of hyperparameters of the model. }.

For any choice of the prior width $\sigma_p$ we can obtain a prediction by
generating $N$ samples $\{\phi^{(\alpha)}\}^N_{\alpha=1}$ according to the distribution
$p_{\theta}(\phi|D)$ computed from \cref{eq:bayes}.  
  
\subsubsection{Reweighting approach}
\label{sec:bayesianhmc}

The reweighting method takes $N$ samples
$\{\phi_{i}^{({\alpha})}\}_{\alpha=1}^{N}$ obtained at
$\sigma_{p}=\sigma_{p}^{*}$ and computes the reweighted average using
$\tilde\sigma_{p}=\sigma_{p}^{*}+\epsilon$ in \cref{eq:rw}.  

For each sample $\phi^{(\alpha)}$, the reweighting factor becomes a polynomial expansion in
$(\sigma-\sigma_{p}^{*})$  
\begin{equation}
  \label{eq:rw bi}
  \tilde w_{\alpha}(\epsilon) = \frac{p_{\mu,\sigma_{p}^{*}+\epsilon}(\phi_{\alpha}|D)}{p_{\mu,\sigma_{p}^{*}}(\phi_{\alpha}|D)}.
\end{equation}
Notice that the zeroth order of \cref{eq:rw bi} is one, such that the zeroth order result corresponds to the usual Monte Carlo point estimate for $\delta\phi_{0}(\sigma_{p}^{*})$.

In order to generate these samples, we used the standard HMC algorithm. 
The equations of motion are
\begin{align}
  &H_{\theta}(\phi,\pi) = \frac{\pi^{2}}{2} - \log(p_{\theta}(\phi|D)),\\
	&\dot\phi_{j} = \pi_{j},\\
	&\dot \pi_{j} = - \frac{1}{\sigma_{p}^{2}}(\phi_{j} - (\mu_p)_j) + \sum_{i=0}^{N}\frac{1}{\sigma_{i}^{2}}\left( y_{i} - f(x_{i},\phi) \right)(x_{i})^{j},
\end{align}
where $\pi=\{\pi_{0},\pi_{1},\pi_{2},\pi_{3}\}$ are the momenta conjugated to $\phi$.
Note that all $\phi$-independent terms can be dropped from the
equations of motion, namely the normalization of $p_{\theta}(\phi|D)$
is not needed.
The eom were solved numerically using a fourth-order symplectic
integrator \cite{OMELYAN2003272} providing a high acceptance rate in
the Metropolis-Hastings step even with a coarse integration.  

The chosen integration step-size was $\varepsilon = 0.001$, while the
trajectory length was uniformly sampled in the interval $[0,100]\times
\varepsilon$\footnote{Due to the quadratic form of the Hamiltonian,
  the phase space of this system is cyclic. The algorithm is ergodic
  only if the trajectory length is
  randomized \cite{RHMC2017}.}.
%Taking into account the conclusions from \cref{sec:nspt}, the average trajectory length is approximately tuned such that the variance is minimized.  

% In the following, all of the Monte Carlo chains correspond to half million thermalized trajectories.

%All the predictions are a function of the hyperparameter $\sigma^*$ and
%we would, generically, be interested in this dependence.
%As for the quantity to study we focus on the uncertainty of the
%average value for $\phi_{i}$, $\delta\phi_{i} =
%\mathbb{E}_{p_{\theta}}[\phi_{i}^{2}]-\mathbb{E}_{p_{\theta}}[\phi_{i}]^{2}$,
%and analogously the uncertainty for the prediction of a new point
%$x_{n}$.  

\subsubsection{Hamiltonian perturbative expansion}

Following the procedure in \cref{sec:nspt}, the Monte Carlo samples
$\{(\tilde\phi_{j})^{\alpha}\}_{\alpha=1}^{N},~j=0,1,2,3$ were
obtained with the modified HMC algorithm for some values of
$\sigma_{p}^{*}$. 
We used the same parameters for the HMC as described in the previous
section. In particular our acceptances were so close to 100\% that any
bias due to the missing accept/reject step is negligible. 
We checked this hypothesis by further performing another simulation
with a coarser value of the integration step and finding completely
compatible results.


\subsubsection{Results}

\begin{table}[t]
  \centering
  \caption{Results for the expansion coefficients of the variance,
    ${\delta\phi^{2}_{j,n}}$ for $\sigma_{p}^{*}=0.3$
    from the reweighting and hamiltonian expansion.}
\scalebox{0.9}{
  \begin{tabular}{cccccccc}
	% \toprule
     & & \multicolumn{6}{c}{$n$} \\\cmidrule{3-8}
     & & 0 & 1 & 2 & 3 & 4 & 5  \\
    \midrule
\multirow{2}{*}{$\delta\phi^{2}_{0,n}$} & RW &    0.00014705(86) &    0.0001384(63) &    -0.000248(29) &     0.000367(62) &     -0.00071(51) &      -0.0003(12)  \\
                  & HAD &   0.00014705(86) &    0.0001365(34) &   -0.0002850(60) &     0.000311(20) &     0.000178(77) &     -0.00115(26)  \\

    \midrule
\multirow{2}{*}{$\delta\phi^{2}_{1,n}$} & RW &     0.01099(15) &       0.0285(12) &      -0.0450(58) &        0.032(13) &         0.04(10) &        -0.61(25)  \\
                  & HAD &     0.01099(15) &      0.02787(69) &      -0.0518(11) &       0.0248(38) &        0.189(16) &       -0.700(46)  \\
    \midrule
\multirow{2}{*}{$\delta\phi^{2}_{2,n}$} & RW &      0.008938(74) &      0.00830(28) &      -0.0283(10) &       0.0850(39) &       -0.234(18) &        0.603(78) \\
                  & HAD &     0.008938(74) &      0.00817(15) &     -0.02789(42) &       0.0849(13) &      -0.2505(44) &        0.726(15)  \\
    \midrule
\multirow{2}{*}{$\delta\phi^{2}_{3,n}$} & RW &     0.03617(59) &      0.1205(51) &       -0.182(24) &        0.050(61) &         0.63(42) &         -4.0(12)  \\
                  & HAD &     0.03617(59) &       0.1177(30) &      -0.2052(42) &        0.020(16) &        1.132(66) &        -4.02(19)  \\
    \bottomrule
    \label{tab:variance phi0}
  \end{tabular}
  }
\end{table}

Here we compare the predictions for the average model parameters
$\phi$ and their dependence on the prior width $\sigma$. In particular
we focus on the variance of the model parameters $\delta\phi^{2}_j$,
since these are the quantities most sentitive to the prior width (i.e. 
very thin priors result in small variance for the model
parameters). We have fixed $\sigma^* = 0.3$, but similar conclusions
are obtained for other values.  

The results of the Monte Carlo average for $\delta\tilde\phi^{2}_i$ and
its derivatives with respect to $\sigma$ are
shown in \cref{tab:variance phi0}. 
Results labeled ``RW'' use the reweighting method, while results
labeled ``HAD'' use the Hamiltonian approach. 

It is obvious that results using the Hamiltonian approach are more
precise:
the uncertainties in the derivatives, $\delta\phi^{2}_{i,n},n\neq 0$, are
smaller for the Hamiltonian approach, despite the statistics being the
same. The difference is larger for higher order derivatives: the
approach based on reweighting struggles to get a signal for the fourth
and fifth derivatives, while the Hamiltonian approach is able to
obtain even the fifth derivative with a few percent precision. 
This fits our expectations (see section~\ref{sec:hamilt-appr-repar}). 
\noindent\newline\newline
On the other hand, for our second quantity of analysis $\delta f_t^2$ (i.e. the variance of the prediction mean), Figure~\ref{fig:ypred} shows the results of the dependence on $\sigma_p$, where we have fixed $x_t=0.5$.

%dependence of the variance of the
%parameter prediction
%\begin{equation}
%  y_{\text{pred}}(x_{n})=\mathbb{E}_{p_{\theta}}[f(x_{n},\phi)] = \int d\phi p_{\theta}(\phi|D) f(x,\phi).
%\end{equation}
%at $x = 0.5$ with respect to $\sigma_P$.
The Hamiltonian approach gives visually results with a reduced variance,
similar to the results presented in table~\ref{tab:variance phi0}.

% Figure environment removed



%%% Local Variables:
%%% mode: latex
%%% TeX-master: "paper"
%%% End:


%!TEX root = ./robust_pe.tex
%\newpage

\section{Robust Policy Evaluation as Policy Optimization}\label{sec_pe_as_po}
This section adopts a policy optimization viewpoint towards policy evaluation, by first formulating the robust policy evaluation problem as a Markov decision problem of the nature.
We then identify a few key structures of the formulated MDP that will prove useful for our subsequent development. 

Consider a MDP of nature, denoted by $\mathfrak{M}$,  defined as follows.
The state space is given by $\cS$, and
the set of possible actions  at any state $s \in \cS$ is given by $ \cD_s \subseteq \Delta_{\cS}^{\abs{\cA}}$ (cf. \eqref{def_ambiguity_set_structure}).
%We will write $\DD = [\DD_{a_1}, \ldots, \DD_{a_{\abs{\cA}}}]$ for any $\DD \in \Delta_{\cS}^{\abs{\cA}}$.
%Equivalently, any possible action $\DD \in \cD_s$ specifies $\abs{\cA}$ elements in $\Delta_{\cS}$,  each denoted as $\DD_{a}$ for every $a \in \cA$.
At state $s \in \cS$, upon making an action $\DD \in \cD_s$, the conditional distribution of the next state $s' \in \cS$ is given by 
\begin{align}\label{transit_kernel_of_nature_mdp}
\mathfrak{P}(s' | s, \DD) \coloneqq  \tsum_{a \in \cA} \sbr{(1-\zeta) \overline{\PP}_{s,a}(s')  + \zeta \DD_{a}(s')} \vartheta(a|s).
\end{align}
%Clearly, the above transition kernel $\mathfrak{P}$ is affine with respect to the action of the nature  $\DD$. 
Finally, the cost function associated with $(s, \DD)$ for any $\DD \in \cD_s$ is given by 
\begin{align*}
\mathfrak{C}(s) \coloneqq - \tsum_{a \in \cA} \vartheta(a|s) c(s,a),
\end{align*}
 and the discount factor is set as $\gamma$.

A non-randomized policy of the nature is denoted as $\pi: \cS \to \cD_s$. 
%It is clear that $\pi$ uniquely determines $\DD^{\pi} \in \cD$ defined in \eqref{eq_cD_set}.
%Let us denote $\DD^{\pi(s)} = \pi(s)$, and 
%Let us define $\DD^{\pi(s)}  = \pi(s)$ for any policy $\pi$.
For any policy $\pi$ and $s \in \cS$, let us define $\DD^{\pi(s)} \equiv \pi(s) \in \cD_s$.
For notational clarity we will sometimes use these two quantities interchangeably. 
We then define
\begin{align}\label{kernel_defined_by_nature_policy}
\PP^{\pi}_{s,a} \coloneqq (1-\zeta) \overline{\PP}_{s,a}  + \zeta \DD^{\pi(s)}_{a}, ~ (s,a) \in \cS \times \cA.
\end{align}
%as the transition kernel of the original planner when the nature's policy is $\pi$.
Consequently, from \eqref{transit_kernel_of_nature_mdp} it holds that 
\begin{align}\label{state_transit_of_nature_given_policy}
\mathfrak{P}(s'|s,  \pi(s)) =  \tsum_{a \in \cA} \sbr{(1-\zeta) \overline{\PP}_{s,a}(s')  + \zeta \DD^{\pi(s)}_{a}(s')} \vartheta(a|s)
= \tsum_{a \in \cA} \PP^{\pi}_{s,a}(s') \vartheta(a|s).
\end{align}
We define the value function of policy $\pi$ as 
\begin{align*}
%\label{eq_def_value_func_nature}
\cV^{\pi} (s) \coloneqq 
\EE \sbr{\tsum_{t=0}^\infty \gamma^t \mathfrak{C}(s_t) \big| s_0 = s, s_{t+1} \sim \mathfrak{P}(\cdot| s_t, \pi(s_t) ) , t \geq 0}, ~~ \forall s \in \cS,
\end{align*}
and the goal of the nature is to find the optimal policy of 
\begin{align}\label{eq_def_optmal_value_nature}
\textstyle
\min_{\pi \in \Pi} \cV^{\pi} (s),
\end{align}
where $\Pi: s \mapsto \cD_s$ is the set of non-randomized stationary policies of the nature.

\begin{lemma}\label{lemma_value_correspondence}
For any $\pi \in \Pi$, we have 
\begin{align}\label{eq_nature_value_as_player_value}
\cV^{\pi}(s) = - V^{\vartheta}_{\PP^{\pi}}(s), ~ \forall s \in \cS,
\end{align}
with $V^{\vartheta}_{\PP^{\pi}}$ defined in \eqref{eq_standard_value_function}.
In addition, let $\cV^*$ denote the optimal value function of \eqref{eq_def_optmal_value_nature}.
Then 
\begin{align}\label{nature_opt_as_robust_value}
\cV^*(s) = - V^{\vartheta}_r(s), ~ \forall s \in \cS,
\end{align}
where $V^{\vartheta}_r$ is defined as in \eqref{eq_def_robust_value}.
\end{lemma}

\begin{proof}
It is clear that the $\cV^{\pi}$ satisfies the following dynamic programming equation 
\begin{align*}
\cV^{\pi} (s) & = \mathfrak{C}(s) + \gamma \tsum_{s' \in \cS} \mathfrak{P}(s'|s,  \pi(s)) \cV^{\pi}(s') \\
 & = - \tsum_{a \in \cA} \vartheta(a|s) c(s,a)
 + \gamma  \tsum_{s' \in \cS} \tsum_{a \in \cA} \PP^{\pi}_{s,a}(s') \vartheta(a|s) \cV^{\pi}(s'), ~ \forall s \in \cS.
\end{align*}
where the last equality follows from \eqref{state_transit_of_nature_given_policy}.
The above relation implies that $- \cV^{\pi}$ is the fixed point of operator 
\begin{align*}
(\cT^{\pi} V)(s) = 
\tsum_{a \in \cA} \vartheta(a|s) c(s,a)
 + \gamma  \tsum_{s' \in \cS} \tsum_{a \in \cA} \PP^{\pi}_{s,a}(s') \vartheta(a|s) V(s'), ~ \forall s \in \cS.
\end{align*}
On the other hand, it is known that $V^{\vartheta}_{\PP^{\pi}}$ is the unique fixed point of $\cT^{\pi}$, from which
we obtain \eqref{eq_nature_value_as_player_value}.
% That is, the value function of the nature's policy $\cV^{\pi}$ corresponds to the negative value function of the policy $\pi$ within $\cM_{\PP^{\pi}}$.
In addition,  Bellman optimality condition of MDP \eqref{eq_def_optmal_value_nature} yields  
\begin{align*}
\cV^*(s) & = \min_{\DD \in \cD_s} \mathfrak{C}(s)  + \gamma \tsum_{s' \in \cS} \mathfrak{P}(s' |s, \DD) \cV^*(s') \\ 
& = \min_{\DD \in \cD_s} -  \tsum_{a \in \cA} \vartheta(a|s) c(s,a) 
+ \gamma \tsum_{s' \in \cS} \tsum_{a \in \cA}\sbr{(1-\zeta) \overline{\PP}_{s,a}(s')  + \zeta \DD_{a}(s')}\vartheta(a|s) \cV^*(s') \\
& = \min_{\PP \in \cP_s} -  \tsum_{a \in \cA} \vartheta(a|s) c(s,a) 
+ \gamma \tsum_{s' \in \cS} \tsum_{a \in \cA} \PP_{a}(s') \vartheta(a|s) \cV^*(s') , ~ \forall s \in \cS.
%\\
%& =  - \tsum_{a \in \cA} \vartheta(a|s) c(s,a) 
%+ \gamma  \tsum_{a \in \cA}   \vartheta(a|s)  \tsum_{s' \in \cS} \min_{\PP_{s,a} \in \cP_{s,a}} \PP_{s,a}(s') \cV^*(s'), ~ \forall s \in \cS.
\end{align*}
Clearly, $-\cV^*$ is the fixed point of operator 
\begin{align}\label{def_robust_ballmen_op}
(\cT V)(s) = \max_{\PP \in \cP_s} \tsum_{a \in \cA} \vartheta(a|s) c(s,a) 
+ \gamma \tsum_{s' \in \cS} \tsum_{a \in \cA} \PP_{a}(s') \vartheta(a|s) V(s'), ~ \forall s \in \cS. 
\end{align}
On the other hand, it is well known that $V^{\vartheta}_{r}$ is the unique fixed point of $\cT V$ \cite{wiesemann2013robust}. 
Consequently we obtain \eqref{nature_opt_as_robust_value}.
%.
%That is, the optimal value function of the nature \eqref{eq_def_value_func_nature} corresponds to the negative robust value function $V^{\pi}_r$ of the policy,
%as both are the (unique) solution of the above dynamic programming equation.
\end{proof}


In view of the above observations, the robust policy evaluation problem \eqref{eq_def_robust_value} can be equivalently solved by solving a Markov decision process \eqref{eq_def_optmal_value_nature}  of nature with finite state space and continuous action space. 
To this end, let us define the following problem:
\begin{align}\label{policy_opt_obj_nature}
\textstyle
\min_{\pi \in \Pi} \cbr{f(\pi) \coloneqq \EE_{s \sim \rho} \sbr{\cV^{\pi}(s)}},
\end{align}
where $\rho$ is any distribution with full support over $\cS$.
Our end goal is to develop efficient methods that can be applied to solve \eqref{policy_opt_obj_nature}.


The state-action value function of the nature, also know as the Q-function, is defined by
\begin{align}\label{def_q_func_nature}
\cQ^{\pi}(s, \DD) & \coloneqq 
\EE \sbr{\tsum_{t=0}^\infty \gamma^t \mathfrak{C}(s_t) \big| s_0 = s, s_1 \sim   \mathfrak{P}(\cdot| s, \DD ), s_{t+1} \sim \mathfrak{P}(\cdot| s_t, \pi(s_t) ), t \geq 1 } . 
%\\
%& =  \mathfrak{C}(s) + \gamma \EE_{s' \sim   \mathfrak{P}(\cdot| s, \DD )} \sbr{\cV^{\pi}(s')} \\
%& = \mathfrak{C}(s) +  
% \gamma \tsum_{s' \in \cS} \tsum_{a \in \cA}\sbr{(1-\zeta) \overline{\PP}_{s,a}(s')  + \zeta \DD_{s,a}(s')}\vartheta(a|s) \cV^{\pi}(s')  \\
%& =  \mathfrak{C}(s) +  
% \gamma  \tsum_{a \in \cA} \vartheta(a|s) 
% \inner{(1-\zeta) \overline{\PP}_{s,a} + \zeta \DD_{s,a}}{\cV^{\pi}} \\
% & = 
% \mathfrak{C}(s) + \gamma \inner{(1-\zeta) \overline{\PP}_s + \zeta \DD}{\cV^{\pi}_{\vartheta, s}}, 
\end{align}
Clearly one also has 
\begin{align}\label{relation_q_and_v}
\cQ^\pi(s,\DD) = \mathfrak{C}(s) + \gamma \EE_{s' \sim \mathfrak{P}(\cdot | s,\DD)} \sbr{\cV^{\pi}(s')}.
\end{align}
%for any $ \DD \in \cD_s$, 
%where in the last equality we define $\cV^{\pi}_{\vartheta, s} = \vartheta(\cdot|s) \otimes \cV^{\pi} \in \RR^{\abs{\cA} \abs{\cS}}$.
We next show that $Q^{\pi}(s, \cdot)$ is indeed an affine function over $\cD_s$, an immediate yet important property that we will exploit in the ensuing development. 

\begin{proposition}\label{prop_q_structure}
For any $\DD \in \cD_s$, we have 
\begin{align*}
\cQ^{\pi}(s, \DD)
=  \mathfrak{C}(s) + \gamma \inner{(1-\zeta) \overline{\PP}_s + \zeta \DD}{\cV^{\pi}_{\vartheta, s}}, 
\end{align*}
where $\cV^{\pi}_{\vartheta, s} \coloneqq \vartheta(\cdot|s) \otimes \cV^{\pi} \in \RR^{ \abs{\cS} \abs{\cA}}$.
%and $\overline{\PP}_s \coloneqq [\overline{\PP}_{s, a_1}, \ldots, \overline{\PP}_{s, a_{\abs{\cA}}}] \in \Delta_{\cS}^{\abs{\cA}}$.
\end{proposition}

\begin{proof}
We have 
\begin{align*}
\cQ^{\pi}(s, \DD)
& =  \mathfrak{C}(s) + \gamma \EE_{s' \sim   \mathfrak{P}(\cdot| s, \DD )} \sbr{\cV^{\pi}(s')} \\
& = \mathfrak{C}(s) +  
 \gamma \tsum_{s' \in \cS} \tsum_{a \in \cA}\sbr{(1-\zeta) \overline{\PP}_{s,a}(s')  + \zeta \DD_{a}(s')}\vartheta(a|s) \cV^{\pi}(s')  \\
& =  \mathfrak{C}(s) +  
 \gamma  \tsum_{a \in \cA} \vartheta(a|s) 
 \inner{(1-\zeta) \overline{\PP}_{s,a} + \zeta \DD_{a}}{\cV^{\pi}} \\
 & = 
 \mathfrak{C}(s) + \gamma \inner{(1-\zeta) \overline{\PP}_s + \zeta \DD}{\cV^{\pi}_{\vartheta, s}},
\end{align*}
which completes the proof.
\end{proof}

%\yan{need to define $d_{\rho}^{\pi}$ within the perf diff lemma}
Our ensuing discussions repeatedly make use of the discounted visitation measure, defined as $d_{\rho}^{\pi}(s) = (1-\gamma) \tsum_{s' \in \cS} \rho(s') \tsum_{t=0}^\infty \gamma^t \mathtt{P}^{\pi}(s_t = s| s_0=s')$ for every $s \in \cS$, where $\mathtt{P}^{\pi}(s_t = s| s_0=s')$ denotes the probability of reaching state $s$, if running $\vartheta$ starting from $s'$ within MDP $\cM_{\PP^\pi}$.
In particular, we write $d_{s}^{\pi}$ when $\rho$ has support $\cbr{s}$. 
We next establish the difference of values for two policies of nature. 

\begin{lemma}\label{lemma_perf_diff}
For a pair of policies $\pi, \pi'$, and any $s\in \cS$,  we have
\begin{align}\label{eq_perf_diff}
\cV^{\pi'}(s) - \cV^{\pi}(s) = \frac{1}{1-\gamma}
\EE_{s' \sim d_{s}^{\pi'}} \sbr{
\cQ^{\pi}(s', \pi'(s')) - 
\cQ^{\pi}(s', \pi(s'))
}
\end{align}
Equivalently, by defining $\phi^{\pi}( s, \pi'(s)) 
%\coloneqq \cQ^{\pi}(s, \pi'(s)) - 
%\cQ^{\pi}(s, \pi(s)) 
\coloneqq \gamma \zeta \inner{\pi'(s) - \pi(s)}{\cV^{\pi}_{\vartheta, s}}$, then 
\begin{align}\label{eq_perf_diff_linearized}
\cV^{\pi'}(s) - \cV^{\pi}(s) = \frac{1}{1-\gamma}
\EE_{s' \sim d_{s}^{\pi'}} \sbr{\phi^{\pi}(s', \pi'(s'))}
\end{align}
\end{lemma}


\begin{proof}
%The proof of \eqref{eq_perf_diff} follows standard steps of performance difference lemma for finite MDPs \cite{lan2021policy, kakade2002approximately} and hence is omitted here.
%\yan{need to expand on this one}
%Let $\xi_(s)$ denote the 
Let $\xi'(s) = \cbr{s_0 = s, \pi'(s_0), s_1, \pi'(s_1), \ldots} $ denote the trajectory generated by $\pi'$ within $\mathfrak{M}$. 
That is 
\begin{align*}
s_{t+1} \sim \mathfrak{P}(\cdot|s_t, \pi'(s_t)),
\end{align*}
or equivalently, in view of \eqref{state_transit_of_nature_given_policy}, that
\begin{align}\label{state_transition_distribution_equivalence}
s_{t+1} \sim \tsum_{a \in \cA} \vartheta(a|s_t) \PP^{\pi'}_{s_t,a} (\cdot) .
\end{align}
We then obtain 
\begin{align*}
\cV^{\pi'}(s) - \cV^{\pi}(s)
& \overset{(a)}{=} \EE_{\xi'(s)} \sbr{\tsum_{t=0}^\infty \gamma^t \rbr{ \mathfrak{C}(s_t) + \gamma \cV^{\pi}(s_{t+1}) - \cV^{\pi}(s_t)}  + \cV^\pi(s_0) }   - \cV^\pi(s)  \\
& \overset{(b)}{=} \EE_{\xi'(s)} \sbr{\tsum_{t=0}^\infty \gamma^t \rbr{ \cQ^{\pi}(s_t, \pi'(s_t)) - \cV^{\pi}(s_t)}  } \\
& \overset{(c)}{=} \frac{1}{1-\gamma} \EE_{s' \sim d_s^{\pi'}} \sbr{\cQ^{\pi}(s', \pi'(s')) - \cV^\pi(s')},
\end{align*}
where $(a)$ follows from the definition of $\cV^{\pi'}(s)$, 
 $(b)$  follows from $s_0 = s$ and \eqref{relation_q_and_v},
and $(c)$ follows from \eqref{state_transition_distribution_equivalence} and the definition of $d_s^{\pi'}$.
Then \eqref{eq_perf_diff} follows  by noting that $\cV^{\pi}(s) = \cQ^\pi(s, \pi(s))$. 
Finally, \eqref{eq_perf_diff_linearized} follows from \eqref{eq_perf_diff} and Proposition \ref{prop_q_structure}.
\end{proof}

Interested readers might find the formulated MDP of nature challenging upon initial examination. 
In particular, as nature has a continuous action space, even evaluating the state-action value function \eqref{def_q_func_nature} seems to be challenging, a crucial quantity for policy improvement. 
It is also unclear whether one should and how to parameterize the policy of nature. 
In the next section, we proceed to develop the first-order robust policy evaluation (FRPE) method that exploits the structural properties established in this section and overcomes the aforementioned difficulties.























