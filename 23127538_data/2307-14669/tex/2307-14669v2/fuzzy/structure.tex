\paragraph{Structure of the paper}%

After recalling its syntax in \cref{cha:syntax},
our fuzzy generalization of \osf logic begins in \cref{cha:semantics},
where we define the interpretation of the syntactic objects of this
language -- namely sorts, features, \osf terms and \osf clauses -- in
structures called fuzzy \osf interpretations, and a special
interpretation called the fuzzy \osf graph algebra, whose elements are
rooted directed labeled graphs called \osf graphs.

In \cref{sec:homomorphisms} we define structure-preserving mappings between
fuzzy \osf interpretations called fuzzy \osf homomorphisms, which are
valuable for proving several results regarding the
satisfiability of \osf clauses in fuzzy interpretations.%

Fuzzy \osf homomorphisms are then employed extensively in
\cref{sec:subsumption}, where we extend the fuzzy sort subsumption
ordering to \osf terms and prove that it constitutes a fuzzy partial order.
We provide similar orderings for \osf clauses and \osf graphs, and
illustrate how
the fuzzy \osf term subsumption ordering is related to its crisp
counterpart.

\cref{sec:unification} is devoted to unification. We show how to compute
the greatest lower bound of two \osf terms in the fuzzy \osf term
subsumption ordering through their unification, and
how this procedure can be used to find the
degree to which two \osf terms are subsumed by each other.
We also discuss the complexity of these computations.%

\cref{app:table} presents a reference table of the symbols used in this
paper.
The basic definitions of fuzzy set theory and fuzzy orders are recalled
in \cref{app:fuzzy} to fix the notation.
The proofs of the main results
are reported in
\ifallproofs
\cref{proofs:sem,proofs:hom,proofs:sub,proofs:unif}.
\else
\cref{proofs:hom,proofs:sub,proofs:unif}.
\fi
