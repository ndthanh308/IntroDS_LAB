\section{Proofs for \cref{sec:subsumption}: \nameref{sec:subsumption}}%
\label{proofs:sub}

\proppreorder*
\begin{proof}[Proof of \cref{prop:preorder}]%
  \hypertarget{proof:proppreorder}{\mbox{}}%
  Let $\ii$ be a fuzzy \osf interpretation.

  For all $d\in\ii$ the identity function on $\ii[d]$ is clearly a
  $1$-morphism, so that $\fapproximates[\ii](d, d) = 1$.

  Now suppose $\fapproximates[\ii](d_0, d_1) = \beta_0$ and
  $\fapproximates[\ii](d_1, d_2) = \beta_1$. We want to prove that
  $\fapproximates[\ii](d_0, d_2) \geq \beta_0 \land \beta_1$. Let $\beta_0
  > 0$ and $\beta_1>0$ (otherwise, the desired result follows immediately)
  and let $\gamma_0:\ii[d_0]\to\ii[d_1]$ and $\gamma_1:\ii[d_1]\to\ii[d_2]$
  be, respectively, a $\beta_0$-morphism and a $\beta_1$-morphism. Note
  that these exist because of \cref{prop:homsel}. Then by \cref{prop:homs}
  $\gamma_1\circ\gamma_0$ is a $\beta_0\land\beta_1$ morphism, and
  $\fapproximates[\ii](d_0,d_2)\geq \beta_0\land\beta_1$ follows by
  \cref{def:endomorphic_approximation}.
\end{proof}

\newcommand{\lemmasymmorphisms}{%
  Let $g_0$ and $g_1$ be two \osf graphs. If
  $\gamma_0: \gg[g_0]\to\gg[g_1]$ is a $\beta_0$-morphism ($\beta_0>0$) and
  $\gamma_1: \gg[g_1]\to\gg[g_0]$ is a $\beta_1$-morphism ($\beta_1>0$)
  such that
  $\gamma_0(g_0) = g_1$
  and
  $\gamma_1(g_1) = g_0$, then:
  \begin{enumerate}
    \item for all $g\in\gg[g_0]$, $g$ and $\gamma_0(g)$ are labeled by the
      same sort;
    \item for all $g\in\gg[g_1]$, $g$ and $\gamma_1(g)$ are labeled by the
      same sort;
    \item $\gamma_0\circ\gamma_1 =  \id[{\dom[{\gg[g_1]}]}]$ and
      $\gamma_1\circ\gamma_0 =  \id[{\dom[{\gg[g_0]}]}]$, which implies
      that $\gamma_0$ and $\gamma_1$ are bijections;
    \item $\gamma_0$ and $\gamma_1$ are 1-morphisms.\ifallproofs\else\qedhere\fi
  \end{enumerate}%
}
\ifallproofs
\begin{lemma}[Morphisms between two graphs]%
  \label{prop:symmorphisms}
  \lemmasymmorphisms{}
\end{lemma}
\begin{proof}[Proof of \cref{prop:symmorphisms}]%
  According to the statement of the proposition, let
  \begin{itemize}
    \item $\gamma_0:\gg[g_0]\to\gg[g_1]$ be a
      $\beta_0$-morphism
      such that $\gamma_0(g_0) = g_1$, and
    \item $\gamma_1:\gg[g_1]\to\gg[g_0]$ be a $\beta_1$-morphism
      such that $\gamma_1(g_1) = g_0$
  \end{itemize}
  with $\beta_0>0$ and $\beta_1$>0.

  Let $g\in\gg[g_1]$, so that $g = w^\gg(g_1)$ for some $w\in \F^*$. Then
  \[
    \begin{array}{lllll}
      \gamma_0(\gamma_1(g)) & = & \gamma_0(\gamma_1(w^\gg(g_1)))
                            & = & \gamma_0(w^\gg(\gamma_1(g_1))) \\
                            & = & w^\gg(\gamma_0(\gamma_1(g_1)))
                            & = & w^\gg(\gamma_0(g_0)) = w^\gg(g_1) = g,
    \end{array}
  \]
  i.e., $\gamma_0\circ\gamma_1 = \id[{\dom[{\gg[g_1]}]}]$.

  Let $\s$ be the sort labeling $g$ and $\su$ be the sort labeling
  $g'\defeq\gamma_1(g)$. By \cref{def:osf_algebra_homomorphism} then
  \begin{align*}
    \s^\gg(g) \land \beta_1 \leq \s^\gg(g')~\text{and}~%
    \su^\gg(g') \land \beta_0 \leq \su^\gg(\gamma_0(g')) = \su^\gg(g).
  \end{align*}
  Note that $\s^\gg(g) = 1$ and $\su^\gg(g') = 1$, so that $\s^\gg(g') > 0$
  and $\su^\gg(g)> 0$. Thus
  \begin{align*}
    \fisop(\su, \s) = \s^\gg(g') > 0~\text{and}~%
    \fisop(\s, \su) = \su^\gg(g) > 0,
  \end{align*}
  so that by antisymmetry of $\fisop$ it follows that $\s = \su$, meaning
  that $g$ and $\gamma_1(g)$ are labeled by the same sort.

  Now let $\s$ be an arbitrary sort, and $g\in\gg[g_1]$ be arbitrary. By
  what we just showed, $g$ and $\gamma_1(g)$ are labeled by the same sort,
  so that $\s^\gg(g) = \s^\gg(\gamma_1(g))$, and thus $\s^\gg(g)\land 1\leq
  \s^\gg(\gamma_1(g))$, i.e., $\gamma_1$ is a 1-morphism.

  In a very similar way it can be shown that
  (i) $\gamma_0$ is a 1-morphism,
  (ii) for all $g\in\gg[g_0]$, $g$ and $\gamma_0(g)$ are labeled by the
  same sort, and (iii)
    $\gamma_1\circ\gamma_0 = \id[{\dom[{\gg[g_0]}]}]$.\qedhere
\end{proof}
\fi

\proppartial*
\begin{proof}[Proof of \cref{prop:partial}]%
  \hypertarget{proof:proppartial}{\mbox{}}%
  \ifallproofs
  By \cref{prop:symmorphisms} and \cref{prop:preorder}.
  \else
  By \cref{prop:preorder} and the following claim.
  \begin{claim}
    \lemmasymmorphisms{}
  \end{claim}
  \fi
\end{proof}

\ifallproofs
\propequivalence*
\begin{proof}[Proof of \cref{prop:equivalence}]
  \hypertarget{proof:equivalence}{\mbox{}}%
  Let $g_0 = G(\psi_0)$ and $g_1 = G(\psi_1)$. By assumption we have that
  $g_0\gequiv g_1$.
  Let $\ii$ be arbitrary and $d\in\dom[\ii]$. Then by
  \cref{prop:homs}, \cref{thm:interpretability} and \cref{def:osf_graph_equivalence}:
  \[
    \begin{array}[b]{lll}
      \denot[\psi_0](d) & = &
      \sup(\{
        \beta%
        \mid \exists~\text{$\beta$-morphism}~
        \gamma:\canon\to\ii~\text{such that}~d=\gamma(g_0)
      \})\\
                        & = & \sup(\{
        \beta%
        \mid \exists~\text{$\beta$-morphism}~
        \gamma:\canon\to\ii~\text{such that}~d=\gamma(g_1)
      \})\\
                        & = & \denot[\psi_1](d).
    \end{array}\qedhere
  \]
\end{proof}
\fi

\ifallproofs
\propequivalenceclause*
\begin{proof}[Proof of \cref{prop:equivalence_clause}]
  \hypertarget{proof:equivalence_clause}{\mbox{}}%
  Let $\ii,\alpha_0\models_\beta\phi_0$ and let $\X$ be the root of
  $\phi_0$ and $\Y$ be the root of $\phi_1$. By
  \cref{prop:equivalence_terms_constraints} then
  $\denot[\psi(\phi_0)](\alpha_0(\X))\geq\beta$, and thus by \cref{prop:equivalence}
  $\denot[\psi(\phi_1)](\alpha_0(\X))\geq\beta$.
  Let $\denot[\psi(\phi_1)](\alpha_0(\X))=\beta_1$. If $\beta_1 = 0$, then $\beta=0$ and
  the conclusion follows trivially. Otherwise,
  by \cref{rem:osf_term_denotation} then there is some $\alpha_1$ such that
  $\denota[\psi(\phi_1)][\alpha_1](\alpha_0(\X))=\beta_1\geq\beta$ and thus
  $\alpha_0(\X) = \alpha_1(\Y)$, and thus by
  \cref{prop:equivalence_terms_constraints} we obtain
  $\ii,\alpha_1\models_\beta\phi_1$. The other direction is analogous.
\end{proof}
\fi

\begin{restatable}%
  [{\protect\hyperlink{proof:transparency}{Equivalence of fuzzy \osf orderings}}]%
  {lemma}{thmtransparency}
\label{thm:transparency}
  If
  the normal \osf terms $\psi$ and $\psi'$
  (with roots $\Y$ and $\X$, respectively, and no common variables),
  the \osf graphs $g$ and $g'$, and
  the rooted solved \osf clauses $\phi_{\Y}$ and $\phi'_{\X}$
  respectively correspond to one another though the syntactic mappings,
  then the following are equivalent, for all $\beta\in(0,1]$.
  \begin{enumerate}
    \item There is a $\beta$-morphism $\gamma:\gg[g]\to\gg[g']$ such that
      $\gamma(g) = g'$.
    \item For all \osf interpretations $\ii$ and $d\in\dom$:
      $\denot[\psi'][\ii](d)\land\beta \leq \denot[\psi][\ii](d)$.
    \item For all \osf interpretations $\ii$ and assignments
      $\alpha$ such that $\ii,\alpha\models_{\beta'}\phi'$, there
      is an assignment $\alpha'$ such that $\alpha'(\X) = \alpha(\X)$ and
      $\ii,\alpha'\models_{\beta'\land\beta}\phi[\X/\Y]$.
    \item For all $g\in\dom[\gg]$: $\denot[\psi'][\gg](g)\land\beta \leq \denot[\psi][\gg](g)$.
  \end{enumerate}
\end{restatable}
\begin{proof}[Proof of \cref{thm:transparency}]%
  \hypertarget{proof:transparency}{\mbox{}}%
  \paragraph{\textbf{(1) implies (2)}}%

  Suppose that there is a $\beta$-homomorphism
  $\gamma:\gg[g]\to\gg[g']$ such
  that $g'= \gamma(g)$. We want to show that for all $\ii$ and $d\in\dom$:
  \[
    \denot[\psi'](d)\land\beta\leq\denot[\psi](d).
  \]
  Let $\ii$ be an arbitrary \osf interpretation, let
  $d\in\dom$ and suppose $\denot[\psi'](d) = \beta' > 0$ (otherwise the desired
  result immediately follows). Then by \cref{thm:interpretability} there is
  a $\beta'$-morphism $\gamma':\canon[\phi']\to\ii$ such that $d =
  \gamma'(G(\psi')) = \gamma'(g')$.
  Note that $\gg[g] = \canon$ and $\gg[g'] = \canon[\phi']$, thus we can
  also write $\gamma:\canon\to\canon[\phi']$.
  Then by \cref{prop:homs} $\gamma'\circ\gamma:\canon\to\ii$ is a
  $\beta\land\beta'$-morphism
  and $d = \gamma'(\gamma(g)) = (\gamma'\circ\gamma)(g) =
  (\gamma'\circ\gamma)(G(\psi))$. By \cref{thm:interpretability} then
  $\denot[\psi](d)\geq\beta\land\beta' = \beta\land\denot[\psi'](d)$.

  \paragraph{\textbf{(2) implies (1)}}

  Suppose that, for any $\ii$ and $d\in\dom$:
  $\denot[\psi'](d)\land\beta\leq\denot[\psi](d)$.
  Then in particular for all $g''\in\dom[\gg]$:
  $\denot[\psi'][\gg](g'')\land\beta \leq \denot[\psi][\gg](g'')$.

  Since $g' = G(\psi')$, then $\denot[\psi'][\gg](g') = 1$, so that
  $\denot[\psi][\gg](g') \geq \beta > 0$. Thus by
  \cref{thm:interpretability} there is a $\beta$-morphism $\gamma:
  \canon\to\gg$ such that $g' = \gamma(G(\psi)) = \gamma(g)$. Note that
  $\gg[g] = \canon$, since $g = G(\phi)$. Then by \cref{prop:homsel}
  $\gamma:\gg[g]\to\gg[\gamma(g)]$, i.e., $\gamma:\gg[g]\to\gg[g']$ as
  desired.

  The following claim will be needed in the next two directions.
  \begin{claim}
  \label{claim:substitution}
    Let $\phi$ be an \osf clause, $\beta\in[0,1]$, $\Y\in\tags(\phi)$ and
    $\X\notin\tags(\phi)$: if $\ii, \alpha\models_{\beta'}\phi$, then
    $\ii,\alpha'\models_{\beta'} \phi[\X/\Y]$, where
    \[
      \alpha'(\Z) =
      \begin{cases}
        \alpha(\Y) & \text{if}~ \Z=\X \\
        \alpha(\Z) & \text{otherwise}.
      \end{cases}
    \]
  \end{claim}
  \paragraph{\textbf{(2) implies (3)}}
  Suppose that $\ii,\alpha\models_{\beta'} \phi'$. Then
  $\denot[\psi'](\alpha(\X))\geq\beta'$ by
  \cref{prop:equivalence_terms_constraints}, so that by assumption (2)
  $\denot[\psi](\alpha(\X))\geq\beta\land\beta'$. %
  If $\beta' = 0$ the desired result follows immediately, so suppose
  otherwise.
  By \cref{rem:osf_term_denotation} %
  and \cref{prop:equivalence_terms_constraints}
  then there is some $\alpha'$ such that $\ii,
  \alpha'\models_{\beta'\land\beta} \phi$ and
  $\alpha(\X) = \alpha'(\Y)$ %
  (recall that $\Y$ is the root tag of
  $\phi$ and $\psi$). But then $\alpha''$ defined by letting
  \[
    \alpha''(\Z) =
    \begin{cases}
      \alpha'(\Y) & \text{if}~ \Z=\X \\
      \alpha'(\Z) & \text{otherwise}
    \end{cases}
  \]
  is such that $\ii, \alpha''\models_{\beta'\land\beta} \phi[\X/\Y]$ by
  \cref{claim:substitution}, and $\alpha''(\X) = \alpha'(\Y) = \alpha(\X)$
  as desired.


  \paragraph{\textbf{(3) implies (2)}}

  Let $\ii$ and $d\in\dom[\ii]$ be arbitrary and we want to show that %
  $\denot[\psi'](d)\land\beta\leq\denot[\psi](d)$.
  Suppose that $\denot[\psi'](d) = \beta' > 0$ (otherwise the result
  follows immediately). By \cref{rem:osf_term_denotation} and
  \cref{prop:equivalence_terms_constraints} then
  there is some $\alpha$ such that
  $d=\alpha(\X)$ and $\ii,\alpha\models_{\beta'}\phi'$.

  By assumption (3) then there is some $\alpha'$ such that
  $\alpha'(\X) = \alpha(\X)$ and
  $\ii,\alpha'\models_{\beta\land\beta'} \phi[\X/\Y]$.
  Let $\alpha''$ be defined by letting
  \[
    \alpha''(\Z) =
    \begin{cases}
      \alpha'(\X) & \text{if}~\Z=\Y\\
      \alpha'(\Z) & \text{otherwise}.
    \end{cases}
  \]
  Then $\ii,\alpha''\models_{\beta\land\beta'} \phi[\X/\Y][\Y/\X]$ by
  \cref{claim:substitution}, i.e.,
  $\ii,\alpha''\models_{\beta\land\beta'} \phi$. Since $\alpha''(\Y) = \alpha'(\X) = \alpha(\X)
  = d$, then by \cref{prop:equivalence_terms_constraints}
  $\denot[\psi](d)\geq\beta\land\beta' = \beta\land\denot[\psi'](d)$ as
  desired.

  \paragraph{\textbf{(2) implies (4)}} Obvious.

  \paragraph{\textbf{(4) implies (1)}} Similar to \textit{\textbf{(2) implies (1)}}.
\end{proof}

\thmtransparencybis*
\begin{proof}[Proof of \cref{thm:transparencybis}]%
  \hypertarget{proof:transparencybis}{\mbox{}}%
  Note that, by \cref{thm:transparency}
  (and the fact that, for any
  $S\subseteq [0,1]$, $\sup(S) = \sup(S\setminus\{ 0 \})$):
  \[
    \begin{array}[b]{ll}
      & \sup(\{ \beta\in[0,1] \mid \gamma(g) = g'~\text{for some $\beta$-homomorphism}~\gamma:\gg[g]\to\gg[g'] \}) \\
      \addlinespace[1mm]
      = & \sup\left(\left\{
      \begin{array}{l|l}
        \multirow{2}{*}{$\beta\in[0,1]$}
        & \forall \ii, \forall d\in\dom:
        (\ii,\alpha\models_{\beta'}\phi') \To (\exists \alpha'. \alpha'(\X)
        = \alpha(\X)\\
                      & \text{and}~\ii,\alpha'\models_{\beta'\land\beta}\phi[\X/\Y]
      \end{array}
  \right\}\right) \\
      \addlinespace[1mm]
      = & \sup(\{ \beta\in[0,1] \mid \forall \ii, \forall d\in\dom: \denot[\psi'][\ii](d)\land\beta \leq \denot[\psi][\ii](d) \}).
    \end{array}\qedhere
  \]
\end{proof}

\begin{restatable}%
  [{\protect\hyperlink{proof:lemmasynsem}{Semantic and syntactic subsumption}}]
  {lemma}{lemmasynsem}
\label{lemma:synsem}
  Let $\psi_0$ and $\psi_1$ be two normal $\osf$ terms. Then,
  for all $\beta\in(0,1]$:
  $\fisop(\psi_0, \psi_1)\geq\beta$ if and only if there are two (normal)
  \osf terms $\psi_0'$ and $\psi_1'$ such that
    $\psi_0 \gequiv \psi_0'$,
    $\psi_1 \gequiv \psi_1'$, and
    $\fsynisop(\psi_0', \psi_1')\geq\beta$.
\end{restatable}
\begin{proof}[Proof of \cref{lemma:synsem}]%
  \hypertarget{proof:lemmasynsem}{\mbox{}}%
  \paragraph{\textbf{$(\bm{\oT})$}} Let $\psi_0'$ and
  $\psi_1'$ be as in the statement of the lemma, with
  $h:\tags(\psi_1')\to\tags(\psi_0')$ witnessing
  $\fsynisop(\psi_0', \psi_1')\geq\beta$.
  By \cref{prop:equivalence} it holds that
  $\denot[\psi_0] = \denot[\psi_0']$ and
  $\denot[\psi_1] = \denot[\psi_1']$ for all $\ii$.
  We show that $\fisop(\psi_0, \psi_1)\geq\beta$ by showing that
      $\forall \ii, \forall d\in\dom[\ii]:
      \denot[\psi_0](d)\land\beta\leq\denot[\psi_1](d)$.

  Let $\ii$ be an arbitrary fuzzy interpretation, let $d\in\dom[\ii]$ and
  suppose $\denot[\psi_0](d) = \denot[\psi_0'](d) = \beta_0$.
  Assume $\beta_0 > 0$, as otherwise the result follows immediately.
  Then by \cref{prop:equivalence_terms_constraints,rem:osf_term_denotation} there is some
  $\alpha_0:\V\to\dom$ such that $\ii, \alpha_0\models_{\beta_0} \phi_0'$, where
  $\phi_0' = \phi(\psi_0')$, $d = \alpha_0(\X[X_0'])$, and $\X[X_0']
  = \rtag(\psi_0')$. Define $\alpha_1:\V\to\dom$ as follows, for
  any $\X\in\V$:
  \[
    \alpha_1(\X) =
    \begin{cases}
      \alpha_0(h(\X)) & \text{if}~\X\in\tags(\psi_1'),\\
      \alpha_0(\X) & \text{otherwise}.
    \end{cases}
  \]
  Then $\ii,\alpha_1\models_{\beta\land\beta_0}\phi_1'$, where $\phi_1' = \phi(\psi_1')$:
  \begin{itemize}
    \item Suppose $\phi_1'$ contains a constraint of the form $\Xj[1]:\si[1]$.
      Since $\phi_1'$ is a rooted solved clause constructed from $\psi_1'$,
      then it must be the case that $\si[1] = \sort_{\psi_1'}(\Xj[1])$. Let
      $\si[0] = \sort_{\psi_0'}(h(\Xj[1]))$ and note that by assumption
      that $\fsynisop(\psi_0', \psi_1')\geq\beta$ we have that
      (a) $\fisop(\si[0],\si[1])\geq\beta$.
      By construction $\phi_0'$ contains a
      constraint of the form $h(\Xj[1]):\si[0]$, and since $\ii,
      \alpha_0\models_{\beta_0}\phi_0'$, then
      (b) $\sii[0](\alpha_0(h(\Xj[1])))\geq\beta_0$.
      Thus
      \[
        \begin{array}{llll}
          \sii[1](\alpha_1(\Xj[1])) & =    & \sii[1](\alpha_0(h(\Xj[1])))                             \\
                                    & \geq & \sii[0](\alpha_0(h(\Xj[1]))) \land \fisop(\si[0],\si[1])  & (\text{\cref{def:osf_algebra}}) \\
                                    & \geq & \beta_0\land\beta. & (\text{by (a) and (b)})
        \end{array}
      \]
    \item Suppose $\phi_1'$ contains a constraint of the form $\X.\f \doteq
      \Y$. Then $\X\fto_{\psi'_1}\Y$ and thus by assumption
      that $\fsynisop(\psi_0', \psi_1')\geq\beta$ we have that
      $h(\X)\fto_{\psi_0'}h(\Y)$, so that $\phi_0'$ must contain a
    constraint of the form $h(\X).\f \doteq h(\Y)$.
    Since $\ii, \alpha_0\models_{\beta_0} \phi_0'$ and $\beta_0>0$, then
    $\f^\ii(\alpha_0(h(\X))) = \alpha_0(h(\Y))$, and thus
    $\f^\ii(\alpha_1(\X)) = \alpha_1(\Y)$, so that
    $\ii,\alpha_1\models_{\beta\land\beta_0} \X.\f \doteq \Y$.
  \end{itemize}
  Let $\X[X_1'] = \rtag(\psi_1')$. By \cref{def:syn_osf_term_subsumption}
  then $h(\X[X_1']) = \X[X_0']$, so that $\alpha_1(\X[X_1']) =
  \alpha_0(h(\X[X_1'])) =\alpha_0(\X[X_0']) = d$. Since
  $\ii,\alpha_1\models_{\beta\land\beta_0} \phi_1'$, then by
  \cref{prop:equivalence_terms_constraints}
  $\denot[\psi_1'](d) =
  \denot[\psi_1](d) \geq \beta_0\land\beta = \denot[\psi_0](d) \land \beta$
  as desired.

  \paragraph{\textbf{$(\bm{\To})$ (Sketch)}}
  Let $\phi_0 = \phi(\psi_0)$ and $\phi_1 =
  \phi(\psi_1)$ and $g_0 = G(\psi_0)$ and $g_1 = G(\psi_1)$.
  In the following, we abbreviate $G(\phi_0(\X))$ as
  $g_0^{\X}$ and $G(\phi_1(\X))$ as $g_1^{\X}$ (in other words, $g_i^{\X}$ is
  the subgraph of $g_i$ rooted at the node corresponding to the variable
  $\X\in\tags(\psi_i)$).

  Since $\fisa(\psi_0,\psi_1)\geq\beta$, then $\fapproximates(g_1,
  g_0)\geq\beta$, so there is a $\beta$-morphism
  $\gamma:\gg[g_1]\to\gg[g_0]$ such that $g_0 = \gamma(g_1)$.
  The following steps almost provide a mapping
  $h:\tags(\psi_1)\to\tags(\psi_0)$ that witnesses $\fsynisop(\psi_0,
  \psi_1)\geq\beta$:
  \begin{enumerate}
    \item for $\Y\in\tags(\psi_1)$, consider
      $g_1^{\Y}\in\gg[g_1]$
      and
      $\gamma(g_1^{\Y})\in\gg[g_0]$;
    \item let $\X$ be the variable in $\psi_0$ such that
      $g_0^{\X} =
      \gamma(g_1^{\Y})$;
      and
    \item define $h(\Y) \defeq \X$.
  \end{enumerate}
  Unfortunately, nothing guarantees that such a variable $\X\in\tags(\psi_0)$
  actually exists, because of cases such as the following:
  \begin{center}
    \begin{tabular}{l}
      $\si[0]\fisa_\beta\si[1]$, $\psi_0 = \X[X_0]:\si[0]$, and
      $\psi_1 = \X[Y_0]:\si[1](\f\to \Y[Y_1]:\tops)$
    \end{tabular}
  \end{center}
  where $\gamma(G(\phi_1(\Yj[1]))) = \gamma(G(\Yj[1]:\tops))$ would be the
  trivial graph (with main sort $\tops$) obtained by applying $\f^\gg$ to
  $G(\psi_0)$ (recall \cref{def:osf_graph_algebra}).
  Clearly $\psi_0\fisa_\beta\psi_1$\footnotemark, but we cannot define an
  $h$ that satisfies \cref{def:syn_osf_term_subsumption}.
  \footnotetext{Note that, for any \osf term $\psi = \osfterm$,
    $\denot[\psi] = \denot[\xs(\fti, \ldots, \fti[n], \f\to\tops)]$, for
    any $\f\in\F$.}%
  This is why the statement of the lemma mentions a term $\psi_0'$ that
  is equivalent to $\psi_0$, where the required subterms of $\psi_0$
  corresponding to such trivial graphs would be made explicit (in the latter
  case $\psi_0'$ would be $\X[X_0]:\si[0](\f\to\X:\tops)$). The details
  concerning the general construction of such $\psi_0'$ %
  are left to the reader.
  In the following we simply assume that $\psi_0$ contains
  the variables needed to define $h$ as above.

  Now we show that the function $h$ witnesses the fact that
    $\fsynisa(\psi_0, \psi_1)\geq\beta$.
    \begin{itemize}
      \item Clearly $h(\rtag(\psi_1)) = \rtag(\psi_0)$, since $\gamma(g_1)
        = g_0$,
        and $h$ has been defined accordingly.
      \item Suppose that $\Yj[0]\fto_{\psi_1}\Yj[1]$.
        According to our definition of $h$:
        \begin{itemize}
          \item let $\Xj[0]$ be the variable in $\psi_0$ such that
            $g_0^{\Xj[0]} = \gamma(g_1^{\Yj[0]}))$,
            so that
            $h(\Yj[0]) = \Xj[0]$; and
          \item let $\Xj[1]$ be the variable in $\psi_0$ such that
            $g_0^{\Xj[1]} = \gamma(g_1^{\Yj[1]})$,
            so that
            $h(\Yj[1]) = \Xj[1]$.
        \end{itemize}
        Since $\Yj[0]\fto_{\psi_1}\Yj[1]$, then
        $g_1^{\Yj[1]} = \f^\gg(g_1^{\Yj[0]})$ and,
        since $\gamma$ is a $\beta$-morphism, then
        $\gamma(g_1^{\Yj[1]}) =
        \gamma(\f^\gg(g_1^{\Yj[0]})) =
        \f^\gg(\gamma(g_1^{\Yj[0]}))$,
        and thus
        $g_0^{\Xj[1]} = \f^\gg(g_0^{\Xj[0]})$.
        By construction of
        $g_0$ this means that
        $\Xj[0]\fto_{\psi_0}\Xj[1]$,
        i.e.,.
        $h(\Yj[0])\fto_{\psi_0}h(\Yj[1])$
        as desired.
      \item Now let $\Y\in\tags(\psi_1)$ be arbitrary.
        According to our definition of
        $h$, let $\X$ be the variable in $\psi_0$ such that
        $g_0^{\X} = \gamma(g_1^{\Y})$,
        so that
        $h(\Y) = \X$.
        Let
        $\si[0] = \sort_{\psi_0}(\X)$
        and
        $\si[1] = \sort_{\psi_1}(\Y)$.
        Since $\gamma$ is a $\beta$-morphism, then
        $\sgg[1](g_1^{\Y}) \land \beta \leq
        \sgg[1](\gamma(g_1^{\Y}))
        = \sgg[1](g_0^{\X})$.
        Note that
        $\sgg[1](g_1^{\Y}) = \fisop(\si[1], \si[1]) = 1$
        and
        $\sgg[1](g_0^{\X}) = \fisop(\si[0], \si[1])$,
        so the previous inequation simplifies to
        $\beta\leq\fisop(\si[0],\si[1])$.

    Since $\Y\in\tags(\psi_1)$ was arbitrary, we have shown that
    $\fisop(\sort_{\psi_0}(h(\Y)), \sort_{\psi_1}(\Y)) \geq\beta$ is
    true for all $\Y\in\tags(\psi_1)$,
    and thus
    $\min(\{
      \fisop(\sort_{\psi_0}(h(\Y)), \sort_{\psi_1}(\Y))
      \mid
      \Y\in\tags(\psi_1)
    \}) \geq\beta$,
    showing indeed that
    $\fsynisop(\psi_0, \psi_1)\geq\beta$.\qedhere
    \end{itemize}
\end{proof}

\newcommand{\bsem}{\beta_{\mathit{sem}}}
\newcommand{\bsyn}{\beta_{\mathit{syn}}}
\propsynsem*
\begin{proof}[Proof of \cref{prop:synsem}]%
  \hypertarget{proof:synsem}{\mbox{}}%
  This is a consequence of \cref{lemma:synsem} and the following fact.
  \begin{claim}
    \label{claim:syndegree}
    Let $\psi_0$ and $\psi_1$ be normal \osf terms. If
    \begin{itemize}
      \item  there exist $\psi_0'$ and $\psi_1'$ such that $\psi_0'\equiv
        \psi_0$ and $\psi_1'\equiv \psi_1$ and
        $\psi_0'\fsynisa_{\beta'}\psi_1'$ with $\beta'>0$, and
      \item  there exist $\psi_0''$ and $\psi_1''$ such that $\psi_0''\equiv
        \psi_0$ and $\psi_1''\equiv \psi_1$ and
        $\psi_0''\fsynisa_{\beta''}\psi_1''$ with $\beta''>0$,
    \end{itemize}
    then $\beta' = \beta''$.
  \end{claim}
  The reason why this holds is that, if $\psi_i$, $\psi_i'$ and $\psi_i''$
  ($i\in\{ 0,1 \}$) are equivalent, then they are essentially the same
  term, except that possibly one term may contain a subterm with a feature
  $\f$ pointing at the sort $\tops$, and this subterm is not present in the other
  equivalent terms (see also \cref{def:osf_graph_equivalence} and the proof
  of \cref{lemma:synsem}). The existence of such trivial subterms
  does not however affect the degree of the syntactic subsumption.

  Let us now prove the statement of the theorem. If $\fisop(\psi_0, \psi_1)
  = \beta$ then by \cref{lemma:synsem}
  there exist $\psi_0'\equiv\psi_0$ and $\psi_1'\equiv\psi_1$ such that
  $\fsynisop(\psi_0', \psi_1')\geq\beta$, i.e.,
  $\psi_0'\fsynisa_{\beta'}\psi_1'$ and $\beta'\geq\beta$.
  Then also $\fsynisop(\psi_0', \psi_1')\geq\beta'$,
  and thus by
  \cref{lemma:synsem} $\fisop(\psi_0, \psi_1)\geq\beta'$, i.e.,
  $\beta'\leq\beta$, and thus $\beta'=\beta$.

  Now suppose that
  there exist $\psi_0'\equiv\psi_0$ and $\psi_1'\equiv\psi_1$ such that
  $\psi_0'\fsynisa_{\beta}\psi_1'$, so that
  $\fsynisop(\psi_0', \psi_1')\geq\beta$.
  Then by \cref{lemma:synsem}
  $\fisop(\psi_0, \psi_1)\geq\beta$. To prove that
  $\fisop(\psi_0, \psi_1) \leq \beta$, let us suppose otherwise, i.e.,
  that $\fisop(\psi_0, \psi_1) = \beta' >\beta$. Then by \cref{lemma:synsem}
  there exist $\psi_0''\equiv\psi_0$ and $\psi_1''\equiv\psi_1$ such that
  $\fsynisop(\psi_0'', \psi_1'')\geq\beta'$, i.e.,
  $\psi_0''\fsynisa_{\beta''}\psi_1''$ and $\beta''\geq \beta' > \beta$.
  But by \cref{claim:syndegree} $\beta'' = \beta$, which is the desired
  contradiction.
\end{proof}

\thmcrispfuzzy*
\begin{proof}[Proof of \cref{thm:crispfuzzy}]%
  \hypertarget{proof:crispfuzzy}{\mbox{}}%
  We employ the following notation for the rest of the proof: for an \osf
  graph $g = \ograph$, we let $\s_g \defeq \lambda_N(\X)$, i.e., $\s_g$ is
  the label of the root of $g$.

  In the rest of the proof we rely on definitions, results and
    notation from crisp \osf logic \cite{AitKaci1993b}. In particular,
    $\isa$ denotes the crisp subsumption ordering on sorts and \osf terms,
    and $\approximates[\jj]$ is the crisp approximation ordering on \osf
  graphs.

  To better distinguish between the crisp and fuzzy contexts in the proof,
  we use $\jj$ for the \textit{crisp} \osf graph algebra
  \cite{AitKaci1993b}, and $\gg$ for the \textit{fuzzy} \osf graph algebra.
  Note that the domains of these two structures and the definition of
  $\f^\jj$ and $\f^\gg$ are the same. The only difference is the
  interpretation of sort symbols, since, for each $\s\in\S$:
    $\s^\jj$ is the set of graphs $g$ such that $\s_g\isa \s$,
    while $\s^\gg$ is a fuzzy set defined by letting $\s^\gg(g) =
      \fisop(\s_{g}, \s)$.
  Note that it follows that $g\in\s^\jj$ if and only if $\s^\gg(g) > 0$.

  Also note that, given a graph $g$, the subalgebra $\jj[g]$ and the fuzzy
  subalgebra $\gg[g]$ have the same domain and feature symbols are
  interpreted in the same way.
  The only difference is again the interpretation of sort symbols, since,
  for each $\s\in\S$:
    $\s^{\jj[g]} = \s^\jj\cap\F^*(g)$ and
    $\s^{\gg[g]} = \s^\gg\fcap\cf[\F^*(g)]$.
  Note that it holds that
  \ifallproofs
  \[
    \begin{array}{lll}
      g'\in\s^{\jj[g]} & \Iff & g'\in\s^\jj~\text{and}~g'\in\F^*(g) \\
                       & \Iff & \s^\gg(g')>0~\text{and}~\cf[\F^*(g)](g') = 1 \\
                       & \Iff & \s^\gg\fcap\cf[\F^*(g)](g')>0 \\
                       & \Iff & \s^{\gg[g]}(g') > 0.
    \end{array}
  \]
  \else
    $g'\in\s^{\jj[g]}$ if and only if $\s^{\gg[g]}(g') > 0$.
  \fi

  Let $g_1 = G(\psi_1)$ and $g_2 = G(\psi_2)$.
  Suppose that $\psi_1\isa\psi_2$, so that $g_2\approximates[{\jj}] g_1$
  \cite{AitKaci1993b}, meaning that
  there is a function $\gamma: \dom[{\jj[g_2]}]\to\dom[{\jj[g_1]}]$ such
  that
  \begin{itemize}
    \item $\gamma(g_2) = g_1$;
    \item $\forall \f\in\F$, $\forall g\in\jj[g_2]$:
                       $\gamma(\f^{\jj}(g)) = \f^{\jj}(\gamma(g))$; and
    \item $\forall \s\in\S$, $\forall g\in\jj[g_2]$:
      if $g\in\s^\jj$, then $\gamma(g)\in\s^\jj$.
  \end{itemize}
  Note that, equivalently, the last condition expresses that $\forall
  \s\in\S$, $\forall g\in\gg[g_2]$:
  \begin{align}
    \label{eq:crispifthen}
    \text{if}~\s^\gg(g)>0,~\text{then}~\s^\gg(\gamma(g))>0.
  \end{align}

  We prove that there is a $\beta> 0$ such that
  $\forall \s\in\S$ and $\forall g\in\gg[g_2]$: $\s^\gg(g)\land\beta\leq
  \s^\gg(\gamma(g))$.

  Note that $\gg[g_2]$ contains a finite number of graphs with root label
  different from $\tops$,
  i.e., at most the ones corresponding to the variables of
  $\phi(\psi_2)$.
  Thus the set
  $B \defeq \{ \fisop(\s_{\gamma(g)}, \s_{g})  \mid \s_{g}\neq\tops, g\in\gg[g_2] \}$
  is finite, so that $\beta \defeq \min(B)$ exists.
  Note that
    $\beta = \min(\{ \fisop(\s_{\gamma(g)}, \s_{g})  \mid
      g\in\gg[g_2] \})$,
      since if $\s_{g} = \tops$, then $\fisop(\s_{\gamma(g)},
      \s_{g}) = 1$ by definition.
      Moreover, for every $g\in\gg[g_2]$, $\fisop(\s_{\gamma(g)},
        \s_{g})>0$:
        indeed $\s^\gg_{g}(g) = \fisop(\s_g, \s_g) = 1 > 0$ so that by
        \cref{eq:crispifthen} $\s_{g}^\gg(\gamma(g)) = \fisop(\s_{\gamma(g)}, \s_{g})
        > 0$.
  It follows that $\beta > 0$. Finally, let $g\in\gg[g_2]$ and $\s\in \S$. Then
  \[
    \begin{array}{lllll}
      \s^\gg(g)\land\beta & = & \fisop(\s_{g}, \s) \land \beta
                  & \leq & \fisop(\s_{g}, \s)\land\fisop(\s_{\gamma(g)},
                  \s_{g})\\
                  & \leq & \fisop(\s_{\gamma(g)}, \s)
                  & = & \s^\gg(\gamma(g))
    \end{array}
  \]
  by transitivity of $\fisop$. Thus $\gamma$ is a $\beta$-morphism $\gamma:
  \gg[g_2]\to\gg[g_1]$ such that $\gamma(g_2) = g_1$. Then
  $\fapproximates(g_2, g_1) = \beta' \geq \beta > 0$, so that by
  \cref{thm:transparencybis} $\fisop(\psi_1, \psi_2) > 0$.

  For the other direction, assume that $\psi_1\fisa_\beta \psi_2$ for some
  $\beta>0$. Then $g_2 \fapprel g_1$, meaning that there is a
  function $\gamma: \dom[{\gg[g_2]}]\to\dom[{\gg[g_1]}]$ such that
  \begin{itemize}
    \item $\gamma(g_2) = g_1$;
    \item $\forall \f\in\F$, $\forall g\in\gg[g_2]$:
                       $\gamma(\f^{\gg}(g)) = \f^{\gg}(\gamma(g))$; and
    \item $\forall \s\in\S$, $\forall g\in\gg[g_2]$:
      $\s^\gg(g)\land\beta\leq \s^\gg(\gamma(g))$.
  \end{itemize}
  Let $\s$ and $g$ be arbitrary and suppose that
  $g\in\s^\jj$,
  so that $\s^\gg(g)>0$. Since
  $\s^\gg(g)\land\beta\leq \s^\gg(\gamma(g))$, then
  $\s^\gg(\gamma(g))>0$,
  and so
  $\gamma(g)\in\s^{\jj}$.
  Thus $\gamma$ is an
  \osf algebra morphism \cite{AitKaci1993b} $\gamma:\jj[g_2]\to\jj[g_1]$
  such that $\gamma(g_2) = g_1$, i.e., $g_2\fapproximates[\jj] g_1$, and
  equivalently $\psi_1\isa\psi_2$ as desired.
\end{proof}
