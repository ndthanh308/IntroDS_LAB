\section{Proofs for \cref{sec:unification}: \nameref{sec:unification}}%
\label{proofs:unif}

\thmalgo*
\begin{proof}[Proof of \cref{thm:algo}]%
  \hypertarget{proof:algo}{\mbox{}}%
  The fact that $\psi = \psi_1\fmeet\psi_2$ is given by
  \cref{thm:fuzzy_osf_term_unification}. Let us assume that $\psi$ is not
  the inconsistent clause, as otherwise $\beta=1$ and the result follows
  immediately.

  For $i\in\{ 1, 2 \}$, let
  $h_i: \tags(\psi_i) \to\tags(\psi)$ be the
  mapping defined by letting
  $h_i(\X) = \Z_{[\X]}$ for all $\X\in\tags(\psi_i)$.
  Then $h_1$ witnesses the syntactic subsumption
  $\psi \fsynisa_{\beta_1} \psi_1$,
  as it satisfies the three conditions of
  \cref{def:syn_osf_term_subsumption}.
  \begin{enumerate}
    \item
      Let $\rtag(\psi_1) = \Xj[1]$.
      The fact that
      $h_1(\rtag(\psi_1)) =h_1(\Xj[1]) = \Z_{[\Xj[1]]} = \rtag(\psi)$
      is clear by the construction of $\psi = \psi(\phi')$ on \cref{line:13}
    and the fact that $\phi'$ is rooted in $\Z_{[\Xj[1]]}$.
    \item
      Suppose that $\X\fto_{\psi_1}\Y$. Then $\phi$ on \cref{line:2}
      contains the constraint $\X.\f \doteq \Y$. Throughout the application of
      the constraint normalization rules of \cref{fig:osf_normalization}
      the variables $\X$ and $\Y$ of this constraint may be substituted
      with other variables, possibly multiple times.
      All of these substitutions are witnessed by equality constraints
      contained in $\phi$ after the loop of
      \cref{line:3,line:4}.
      By the definition of $\osfeq$ on \cref{line:9},
      after the loop of \cref{line:11,line:12} the clause
      $\phi'$ must contain the constraint $\Z_{[\X]}.\f\to \Z_{[\Y]}$.
      Finally, by the construction of $\psi = \psi(\phi')$ on
      \cref{line:13}, it holds that
      $\Z_{[\X]}\fto_{\psi}\Z_{[\Y]}$, i.e.,
      $h_1(\X)\fto_{\psi}h_1(\Y)$, as required.
    \item
      Since $h_1(\X) = \Z_{[\X]}$ for all $\X\in\tags(\psi_1)$, the
      computation of $\beta_1$ on \cref{line:14} is carried out in the same
      way as the definition of the syntactic subsumption degree in
      \cref{def:syn_osf_term_subsumption}.
  \end{enumerate}
  It can be proved analogously that $h_2$ witnesses the syntactic subsumption
  $\psi \fsynisa_{\beta_2} \psi_2$.
  It follows by \cref{prop:synsem} that
  $\fisop(\psi, \psi_1) = \beta_1$
  and
  $\fisop(\psi, \psi_2) = \beta_2$, and, by
  \cref{def:fuzzy_osf_term_unification}, that $\beta=\min(\beta_1,
  \beta_2)$ is the unification degree of $\psi_1$ and $\psi_2$.
\end{proof}
