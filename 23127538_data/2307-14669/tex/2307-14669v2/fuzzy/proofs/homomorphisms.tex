\section{Proofs for \cref{sec:homomorphisms}: \nameref{sec:homomorphisms}}%
\label{proofs:hom}

\newcommand{\homsfirst}{%
  Let $\f\in\F$ and $d\in\dom$. Then
    $\gamma'(\gamma(\f^\ii(d))) = \gamma'(\f^\jj(\gamma(d))) =
    \f^\kk(\gamma'(\gamma(d)))$.
  Now let $\s\in\S$ and $d\in\dom$. Since
    $\s^\ii(d)\land\beta\leq\s^\jj(\gamma(d))$
  and
    $\s^\jj(\gamma(d))\land\beta'\leq\s^\kk(\gamma'(\gamma(d)))$
  then
  $\s^\ii(d)\land\beta\land\beta'\leq\s^\jj(\gamma(d))\land\beta'\leq\s^\kk(\gamma'(\gamma(d)))$.
}
\newcommand{\homssecond}{%
  Since $\gamma$ is a $\beta$-morphism, then $\forall d'\in\dom$:
  $\s^\ii(d')\land\beta\leq\s^\jj(\gamma(d'))$.
  Since $\beta'\leq\beta$, then, $\forall d'\in\dom$:
  $\s^\ii(d')\land\beta'\leq
  \s^\ii(d')\land\beta\leq\s^\jj(\gamma(d'))$.%
}
\newcommand{\homsclaimfirstproof}{%
  \begin{subproof}[Proof of \cref{claim:b}]
    Let $\beta_s\defeq\sup(B)$.
    A few facts used in the rest of the proof:
    \begin{enumerate}
      \item For all $\beta\in B$ and $\beta'\in[0,1]$: if $\beta'\leq \beta$,
        then $\beta'\in B$.
      \item For all $\beta'\in[0,1]$: if $\beta'\notin B$, then $\beta'$ is
        an upper bound of $B$, and thus $\beta_s\leq \beta'$.
      Indeed, suppose $\beta'\notin B$, so let $x_0\in X$ be such that
      $\min(f(x_0), \beta') > g(x_0)$. Let $\beta\in B$ be arbitrary, and
      we want to show that $\beta \leq \beta'$. Suppose that $\beta >
      \beta'$ towards a contradiction. Thus $\min(f(x_0), \beta')\leq
      \min(f(x_0), \beta) \leq g(x_0)$, which is the desired contradiction.
      \item For all $\beta'\in[0,1]$, if $\beta'<\beta_s$, then $\beta'\in B$.
    \end{enumerate}
    Suppose that $\forall x\in X$: $f(x) \leq g(x)$. Then
    $\min(f(x), \beta) \leq g(x)$
    is satisfied for all $\beta\in[0,1]$
    and all $x\in X$, so that $B = [0, 1]$ and $\sup(B) =
    1\in B$.

    Otherwise, let $C \defeq \{ x\in X \mid f(x) > g(x) \}\neq \emptyset$.
    Let $g(C) \defeq \{ g(x) \mid x\in C \}\subseteq[0,1]$. Let $\beta_g
    \defeq \inf(g(C))$. Then
      $\min(f(x), \beta_g)\leq g(x)$ holds
      for all $x\in X$, since:
    \begin{itemize}
      \item This is true for all $x$ such that $f(x)\leq g(x)$.
      \item If $x$ is such that $f(x) > g(x)$, then $x\in C$ and $g(x)\in
        g(C)$, thus $\beta_g\leq g(x)$.
    \end{itemize}
    This means that $\beta_g\in B$.
    Now we show that $\beta_g = \beta_s$:
    \begin{itemize}
      \item Clearly $\beta_g\leq\beta_s$, since $\beta_g\in B$ and $\beta_s =
        \sup(B)$.
      \item Suppose towards a contradiction that $\beta_g<\beta_s$. Then
        there is an $\epsilon>0$ such that $\beta_g+\epsilon<\beta_s$, so that
        $\beta_g+\epsilon\in B$.
        Then there exists an $x_0\in C$ such that $g(x_0)<\beta_g+\epsilon$,
        as otherwise $\beta_g+\epsilon\leq g(x)$ for all $x\in C$, which
        contradicts the fact that $\beta_g$ is the $\inf$ of $g(C)$.
        Then we have that
          $\beta_g\leq g(x_0) < \beta_g+\epsilon<\beta_s$.
        Since $\beta_g+\epsilon\in B$, then
          $\min(f(x_0), \beta_g+\epsilon)\leq g(x_0)$,
        but this is absurd, since $f(x_0) > g(x_0)$ (recall that $x_0\in C$)
        and $\beta_g+\epsilon > g(x_0)$.\qedhere
    \end{itemize}
  \end{subproof}
}
\newcommand{\homsclaimsecondproof}{%
  \begin{subproof}[Proof of \cref{claim:c}]
    Let $\beta_s\defeq\sup(B)$.
    A few facts used in the rest of the proof:
    \begin{enumerate}
      \item For all $\beta\in B$ and $\beta'\in[0,1]$: if $\beta'\leq \beta$,
        then $\beta'\in B$.
      \item For all $\beta'\in[0,1]$: if $\beta'\notin B$, then $\beta'$ is
        an upper bound of $B$, and thus $\beta_s\leq \beta'$.
      Indeed, suppose $\beta'\notin B$, so let $x_0\in X$ and $1\leq
      i_0\leq n$ be such that
      $\min(f_{i_0}(x_0), \beta') > g_{i_0}(x_0)$. Let $\beta\in B$ be arbitrary, and
      we want to show that $\beta \leq \beta'$. Suppose that $\beta >
      \beta'$ towards a contradiction. Thus $\min(f_{i_0}(x_0), \beta')\leq
      \min(f_{i_0}(x_0), \beta) \leq g_{i_0}(x_0)$, which is the desired contradiction.

      \item For all $\beta'\in[0,1]$, if $\beta'<\beta_s$, then $\beta'\in B$.
    \end{enumerate}
    For each $1\leq i \leq n$ let
    $B_i \defeq
      \{ \beta\in[0,1] \mid \forall x\in X: \min(f_i(x), \beta) \leq g_i(x) \}$
    and $\beta_i = \sup(B_i)$. By \cref{claim:b} it holds that $\beta_i\in B_i$
    for each $i$. Also note that $B\subseteq B_i$ for all $1\leq i\leq n$.

    Let $\beta_m = \min(\{ \beta_1, \ldots, \beta_n \})$.
    Note that it holds that
    $\min(f_i(x), \beta_m) \leq g_i(x)$
    for all $1\leq i\leq n$ and all $x\in X$.
    Indeed, for each $1\leq i\leq n$, $\beta_m\leq\beta_i$, and so, for each
    $x\in X$, $\min(f_i(x), \beta_m)\leq\min(f_i(x), \beta_i)\leq g_i(x)$.
    Thus, $\beta_m\in B$, directly implying that $\beta_m\in B_i$ for all
    $1\leq i\leq n$.


    Note that $\beta_s\leq\beta_i$ for all $i$, for suppose otherwise: then
    there is an $i$ such that $\beta_i < \beta_s$, so that there is a
    $\epsilon>0$ such that $\beta_i +\epsilon< \beta_s$ and thus
    $\beta_i+\epsilon\in B$. But then $\beta_i+\epsilon\in B_i$, which
    contradicts the fact that $\beta_i$ is the sup of $B_i$. Thus
    $\beta_s\leq\beta_i$ for all $i$, which means that $\beta_s\leq\min(\{
    \beta_1, \ldots, \beta_n \})=\beta_m$.
    Since $\beta_m\in B$ and $\beta_s = \sup(B)$, then $\beta_m \leq \beta_s$. Thus
    $\beta_m=\beta_s\in B$.\qedhere
  \end{subproof}
}
\newcommand{\homsthird}{%
  Let $B \defeq \{ \beta\in[0,1]\mid \gamma:\ii\to\jj~\text{is
  a}~\beta\text{-morphism} \}$ and $\beta_s = \sup(B)$.
  Note that
  \[
    \begin{array}{lll}
      B & \defeq & \{ \beta\in[0,1]\mid \gamma:\ii\to\jj~\text{is
      a}~\beta\text{-morphism} \}\\
        & = & \{ \beta\in[0,1]\mid \forall \s\in \S,\forall d\in\dom:
        \s^\ii(d)\land \beta\leq\s^\jj(\gamma(d)) \}
    \end{array}
  \]
  and that $B\neq \emptyset$ by assumption. The fact that $\beta_s\in
  B$ and thus $\beta_s$ is the maximum $\beta'$ such that $\gamma$ is a
  $\beta'$-morphism follows from the following
  \ifclaimproofs
  claims.
  \else
  claim.
  \fi
  \ifclaimproofs\begin{claim}
    \label{claim:b}
    Let $f:X\to[0,1]$ and $g:X\to[0,1]$ be functions and
    \[
      B \defeq \{ \beta\in[0,1] \mid \forall x\in X: \min(f(x), \beta)
      \leq g(x) \}.
    \]
    Then $\sup(B) \in B$.
  \end{claim}
  \homsclaimfirstproof{}\fi
  \begin{claim}
    \label{claim:c}
      For each $1\leq i\leq n$, let $f_i: X\to [0,1]$ and $g_i:
      X\to[0,1]$ be functions. Let
      \[
        B \defeq \{ \beta\in[0,1] \mid \forall x\in X, \forall 1\leq i\leq n:
        \min(f_i(x), \beta) \leq g_i(x) \}.
      \]
      Then $\sup(B)\in B$.\ifclaimproofs\else\qedhere\fi
  \end{claim}
  \ifclaimproofs\homsclaimsecondproof{}\fi
}

\prophoms*
\begin{proof}[Proof of \cref{prop:homs}]%
  \hypertarget{proof:homs}{\mbox{}}%
  \ifallproofs
    \begin{enumerate}
      \item \homsfirst{}
      \item \homssecond{}
      \item \homsthird{}
    \end{enumerate}
  \else
  The first two points are easy to show.

  \homsthird{}
  \fi
\end{proof}

\ifallproofs
\prophomsel*
\begin{proof}[Proof of \cref{prop:homsel}]%
  \hypertarget{proof:homsel}{\mbox{}}%
  \begin{enumerate}
    \item Let $d'\in\ii[d]$, i.e., $d' = w^\ii(d)$ for some $w\in\F^*$.
      Then $\gamma(d') = \gamma(w^\ii(d)) =
      w^\jj(\gamma(d)) \in\jj[\gamma(d)]$.
    \item Let $\gamma'$ be a $\beta'$-morphism $\gamma':\ii[d]\to\jj$ such
      that $\gamma'(d) = \gamma(d)$. Let
      $d'\in\ii[d]$, so that $d'=w^\ii(d)$ for some $w\in\F^*$. Then
      $\gamma(d') = \gamma(w^\ii(d)) = w^\jj(\gamma(d))
      =w^\jj(\gamma'(d)) = \gamma'(w^\ii(d)) = \gamma'(d')$.
    \item
      By the previous point
      \[
        \begin{array}{lll}
          B & \defeq & \{ \beta' \mid \exists~\text{$\beta'$-morphism}~\gamma':\ii[d]\to\jj~\text{such that}~\gamma'(d)=\gamma(d) \}\\
            & = & \{ \beta' \mid \gamma:\ii[d]\to\jj~\text{is a $\beta'$-morphism}\}
        \end{array}
      \]
      By \cref{prop:homs} the maximum $\beta_m$ of $B$ exists, and $\gamma$
      is a $\beta_m$-morphism.
      \qedhere
  \end{enumerate}
\end{proof}
\fi

\ifallproofs
\thmextending*
\begin{proof}[Proof of \cref{thm:extending}]%
  \hypertarget{proof:extending}{\mbox{}}%
  Let $\alpha' \defeq \gamma\circ\alpha$.

  Suppose that $\iimod_{\beta_\ii} \X:\s$, i.e., $\s^\ii(\alpha(\X))\geq
  \beta_\ii$. By \cref{def:osf_algebra_homomorphism}, then
  $\s^\jj(\alpha'(\X)) = \s^\jj(\gamma(\alpha(\X))) \geq \s^\ii(\alpha(\X))
  \land \beta \geq \beta_\ii\land \beta$, hence
  $\jj,\alpha'\models_{\beta_\ii\land\beta}\X:\s$.

  Suppose $\iimod_{\beta_\ii}\X.\f \doteq \Y$.
  If $\f^\ii(\alpha(\X)) \neq \alpha(\Y)$ then $\beta_\ii = 0$ and the result
  immediately follows.
  If $\f^\ii(\alpha(\X)) = \alpha(\Y)$ then
  $\f^\jj(\alpha'(\X)) = \f^\jj(\gamma(\alpha(\X)))) =
  \gamma(\f^\ii(\alpha(\X))) =
  \gamma(\alpha(\Y)) = \alpha'(\Y)$ and thus
  $\jj,\alpha' \models_{\beta_\ii\land\beta}\X.\f \doteq \Y$.

  Suppose. $\iimod_{\beta_\ii}\X \doteq \Y$.
  If $\alpha(\X) \neq \alpha(\Y)$ then $\beta_\ii=0$ and the result
  immediately follows.
  If $\alpha(\X) = \alpha(\Y)$ then
  $\alpha'(\X) = \gamma(\alpha(\X))) =
  \gamma(\alpha(\Y)) = \alpha'(\Y)$ and thus
  $\jj, \alpha' \models_{\beta_\ii\land\beta}\X \doteq \Y$.
\end{proof}
\fi

\newcommand{\suppd}{\supp[d]^{\ii}_{\S}}

\thmfinality*
\begin{proof}[Proof of \cref{thm:weak_finality}]%
  \hypertarget{proof:finality}{\mbox{}}%
  For each element $d\in\dom$ we construct an \osf graph
  $\gamma(d)\in\dom[\gg]$.

  Let us start with a few preliminary definitions. Let $d\in\dom$:
  \begin{itemize}
    \item the set of sorts to which $d$ belongs
      to with degree greater than $0$ is denoted\footnotemark
      \[
        \suppd\defeq \{ \s\in\S \mid \s^\ii(d)>0 \};
      \]
    \item the most specific sort to which $d$ belongs to with degree
      greater than $0$ is denoted
        $\s_{d} \defeq \fbigmeet\suppd$.
  \end{itemize}
  We construct a labeled graph $g = (N, E, \lambda_N, \lambda_E)$ with
  $N\subseteq \V$ as follows:
  \begin{itemize}
    \item for each $d\in\dom$ we choose a variable $\X_d\in\V$ to denote
      a node in $g$;
    \item the label of $\X_d$ is defined as $\lambda_N(\X_d) \defeq
      \s_{d}$;
    \item for each $\f\in\F$, if $\f^\ii(d) = d'$, then $g$ contains the
      edge $(\X_d, \X_{d'})$ labeled $\f$.
  \end{itemize}
  \footnotetext{Recall that we assume that $\tax$ is finite, and thus
  $\suppd$ is a finite subset of $\S$.}
  For each $d\in\dom$, let $\beta_d = \min_{\s\in\suppd}\fisop(\s_d, \s)$.
  Note that $\beta_d > 0$ for any $d\in\dom$, and that
  \ifshortproofs
  because the set $\S$ of sorts is finite, then the set $\{ \beta_d\mid
  d\in\dom \} = \{ \min_{\s\in\suppd}\fisop(\s_d, \s) \mid d \in\dom \}$ is also finite.
  \else
  the sets $\S$, $\S\times\S$ and $\pow(\S\times\S)$ are finite. Thus:
  \begin{itemize}
    \item for each $d\in\dom$, the sets $\suppd\subseteq\S$ and
    $E_d\defeq\{ (\s_d, \s) \mid \s\in\suppd \}\subseteq \S\times\S$ are
    finite;
    \item for each $d\in\dom$, the set $F_d\defeq\{ \fisop(\s, \su)
      \mid (\s, \su)\in E_d \} = \{ \fisop(\s_d, \s)
      \mid \s\in\suppd \} \subseteq (0,1]$ is finite;
    \item the set $\{ E_d \mid d\in\dom \}\subseteq \pow(\S\times\S)$ is
      finite, and thus so is the set $\{ F_d \mid d\in \dom \}$;
    \item thus, the set $\{ \beta_d\mid d\in\dom \} =
      \{ \min_{\s\in\suppd}\fisop(\s_d, \s) \mid d \in\dom \} = \{ \min(F_d)
      \mid d\in\dom \}$ is finite.
  \end{itemize}
  \fi
  Let $\beta = \min \{ \beta_d \mid d\in\dom \}>0$ and define
  $\gamma(d)$ to be the maximally connected subgraph of $g$ rooted in
  $\X_d$. It is left to show that $\gamma: \ii\to\gg$ is a $\beta$-homomorphism.
  \begin{itemize}
    \item Let $\f\in\F$ and $d\in\dom$. If $\f^\ii(d) = d'$, then
      $\gamma(d)$ has an edge labeled $\f$ from
      $\X_d$ to $\X_{d'}$, which is the root of $\gamma(d')$. Hence
      $\gamma(\f^\ii(d))  =\f^\gg(\gamma(d))$.
    \item Let $\s\in\S$ and $d\in\dom$.
      If $\s^\ii(d) = 0$, then
      $\s^\gg(\gamma(d)) \geq \beta\land \s^\ii(d)$. Otherwise,
      $\s\in\suppd$ and thus
      $\s^\gg(\gamma(d)) =
      \fisop(\s_d, \s)\geq\min_{\su\in\suppd}\fisop(\s_d,
      \su)=\beta_d\geq\beta\geq \beta\land \s^\ii(d)$.\qedhere
  \end{itemize}
\end{proof}

\ifallproofs
\corcanonicity*
\begin{proof}[Proof of \cref{cor:canonicity}]%
  \hypertarget{proof:canonicity}{\mbox{}}%
  Suppose $\phi$ is satisfiable in some \osf interpretation $\ii$, so let
  $\alpha:\V\to\dom$ and $\beta_\ii>0$ be such that
  $\ii,\alpha\models_{\beta_\ii}\phi$. By
  \cref{thm:weak_finality} let $\gamma:\ii\to\gg$ be a $\beta$-morphism
  for some $\beta>0$. By \cref{thm:extending}
  then $\gg,\gamma\circ\alpha\models_{\beta_\ii\land\beta}\phi$,
  i.e, $\phi$ is satisfiable in $\gg$. The other direction is obvious.
\end{proof}
\fi

\thmextracting*
\begin{proof}[Proof of \cref{thm:extracting}]
  \hypertarget{proof:extracting}{\mbox{}}%
  Let us abbreviate $G(\phi(\X))$ as $g_{\X}$.
  The $\beta$-morphism $\gamma:\canon\to\ii$ is defined as follows. Let
  $g\in\canon$, so that $g = w^\gg(g_{\X})$ for some $w\in \F^*$ and
  $\X\in\tags(\phi)$: we define $\gamma(g) = \gamma(w^\gg(g_{\X})) \defeq
  w^\ii(\alpha(\X))$.
  Note that $\gamma(g_{\X}) = \alpha(\X)$ for each $\X\in\tags(\phi)$.
  First of all, we need to show that $\gamma$ is well-defined. Let
  $g\in\canon$ and note that there are two cases to consider.
  \begin{itemize}
    \item Suppose $g = g_{\X}$ for some $\X\in\osftags(\phi)$ and let $w,
      w'\in\F^*$ and $\Y, \Z\in\osftags(\phi)$ be such that $g =
      w^\gg(g_{\Y}) =
      w'^\gg(g_{\Z})$. We want to prove that $w^\ii(\alpha(\Y)) =
    w'^\ii(\alpha(\Z))$. By construction of $g_{\X}$
    it must be the case that $\phi$ contains constraints of
    shape
    \[
      \begin{array}{lllll}
        \Y.\fj[0] \doteq \Yj[1],  & \Yj[1].\fj[1] \doteq \Yj[2],  & \ldots, & \Yj[n].\fj[n] \doteq \X & \text{and} \\
        \Z.\fjp[0] \doteq \Zj[1], & \Zj[1].\fjp[1] \doteq \Zj[2], & \ldots, & \Zj[m].\fjp[m] \doteq \X
      \end{array}
    \]
    such that $w = \fj[0]\cdot\fj[1]\cdots\fj[n]$ and
    $w' = \fjp[0]\cdot\fjp[1]\cdots\fjp[m]$. Because
      $\ii,\alpha\models_\beta \phi$ and $\beta>0$, then indeed $w^\ii(\alpha(\Y)) =
    w'^\ii(\alpha(\Z))$ as desired.
    \item Suppose $g = G(\Z:\tops)$ for some variable
      $\Z\notin\osftags(\phi)$. Then $g = w'^\gg(g_{\X})$ for some
      $\X\in\osftags(\phi)$ and some $w'\in\F^*$ such that that $g_{\X}$
      and $w'$
      uniquely determine $\Z$ (see \cref{def:osf_graph_algebra}). This
      case now reduces to the previous one.
  \end{itemize}
  We now prove that the function $\gamma$ is indeed a $\beta$-morphism.
  \begin{itemize}
    \item Let $\f\in\F$ and $g\in\canon$, so that $g$ has shape
      $g=w^\gg(g_{\X})$ as before. Let $w' = w\cdot \f\in\F^*$. Then
      $\gamma(\f^{\canon}(g)) = \gamma(\f^{\canon}(w^\gg(g_{\X}))) =
      \gamma(\f^\gg(w^\gg(g_{\X}))) =
      \gamma(w'^\gg(g_{\X})) =
      w'^\ii(\alpha(\X)) =
      \f^\ii(w^\ii(\alpha(\X))) = \f^\ii(\gamma(g))$.

  \item Now let $\s\in\S$ and we want to show that $\s^\ii(\gamma(g)) \geq
    \beta \land \s^\gg(g)$ for all $g\in\canon$. Let
    $g\in\canon$ be arbitrary. There are two cases to consider.
    \begin{itemize}
      \item Suppose $g = G(\Z:\tops)$ for some variable
        $\Z\notin\osftags(\phi)$. If $\s^\gg(g)=0$ then the result holds.
        Otherwise, if $\s^\gg(g)>0$, then $g$ is labeled by a sort $\su$ such
        that $\fisop(\su,\s)>0$, but since $g$ is labeled by $\su=\tops$, then
        $\s =\tops$, so that $\s^\ii(\gamma(g))=
        \tops^\ii(\gamma(g)) = 1$ and the result holds.
      \item Suppose $g = g_{\X}$ for some $\X\in\osftags(\phi)$ and
        let $\su$ be the label of the root of $g$. By construction of
        $g_{\X}$ then
        $\phi$ must contain a constraint of the form $\X:\su$,
        and by the assumption that $\iimodb\phi$ we have that
        $\su^\ii(\alpha(\X)) \geq\beta$.
        By \cref{def:osf_algebra} then we have that
        $\s^\ii(\gamma(g_{\X}))=\s^\ii(\alpha(\X))
        \geq
        \su^\ii(\alpha(\X))\land\fisa(\su, \s) \geq \beta\land\fisa(\su,\s) =
        \beta\land\s^\gg(g)$. \qedhere
    \end{itemize}
  \end{itemize}
\end{proof}

\thminterpretability*
\begin{proof}[Proof of \cref{thm:interpretability}]%
  \hypertarget{proof:interpretability}{\mbox{}}%
  Let $\X$ be the root variable of $\psi$ and let $d\in\dom$.

  Suppose that $\denot[\psi](d)\geq\beta$. By
  \cref{def:osf_term_denotation,rem:osf_term_denotation} then there is some
  $\alpha$ such that $\denota[\psi](d)\geq \beta$ and $d=\alpha(\X)$.
  By \cref{prop:equivalence_terms_constraints} then $\iimodb\phi$, so that
  by \cref{thm:extracting} there is a $\beta$-homomorphism
  $\gamma:\canon\to\ii$ such that $d = \alpha(\X) = \gamma(G(\phi(\X))) =
  \gamma(G(\psi))$.%

  Now suppose $\gamma:\canon\to\ii$ is a $\beta$-homomorphism such that
  $d=\gamma(G(\psi))$.
  By \cref{thm:satisfiability} we know that $\canon,\alpha\models_1 \phi$
  where $\alpha:\V\to\dom[\canon]$ is such that $\alpha(\Y) = G(\phi(\Y))$
  for all $\Y\in\tags(\phi)$, and thus $d = \gamma(G(\psi)) =
  \gamma(G(\phi(\X))) = \gamma(\alpha(\X))$.
  Then by \cref{thm:extending} $\ii, \gamma\circ\alpha \models_\beta \phi$,
  and thus
  by \cref{prop:equivalence_terms_constraints} it holds that
  $\denota[\psi][\gamma\circ\alpha](\gamma(\alpha(\X))) \geq\beta$, i.e.,
  $\denota[\psi][\gamma\circ\alpha](d)\geq \beta$, so that
  $\denot[\psi](d)\geq \beta$.

  Finally, note that by what we just proved
  (and the fact that, for any
  $S\subseteq [0,1]$, $\sup(S) = \sup(S\setminus\{ 0 \})$):
  \[
    \begin{array}[b]{lll}
      \denot[\psi](d)
        &=&
        \sup(\{ \beta\in[0,1]
        \mid \denot[\psi](d)\geq\beta \})\\
        &=&
        \sup(\{ \beta\in(0,1]
        \mid \denot[\psi](d)\geq\beta \})\\
        &=&
        \sup(\{ \beta\in(0,1]
        \mid\exists~\text{$\beta$-morphism}~
        \gamma:\canon\to\ii~\text{such that}~d=\gamma(G(\psi)) \})\\
        &=&
        \sup(\{ \beta\in[0,1]
        \mid\exists~\text{$\beta$-morphism}~
        \gamma:\canon\to\ii~\text{such that}~d=\gamma(G(\psi)) \}).
    \end{array}\qedhere
  \]
\end{proof}
