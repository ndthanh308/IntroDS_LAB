\newcommand{\zo}{\Z_{[\Xj[0]]}}
\newcommand{\zi}{\Z_{[\Xj[1]]}}
\newcommand{\zd}{\Z_{[\Yj[2]]}}
\begin{example}[Fuzzy \osf term unification]
  \label{ex:unif}
  Consider the fuzzy lattice of \Cref{fig:wdag} and
  the \osf terms
  $\psi_1 = \Yj[0] : \s[u] \left(
      \f[f] \to \Yj[1] : \s[v] \left(
        \f[g] \to \Yj[0],
        \f[h]\to \Yj[2]:\s[r] \right) \right)$
  and
  $\psi_2 = \Xj[0] : \s[v] \left(
      \f[f] \to \Xj[1] : \s[u] \left(
        \f[g] \to \Xj[2] : \s[t] \right) \right)$.
  After \cref{line:2,line:3,line:4},
  an application of the rules of \Cref{fig:osf_normalization} to
  $\phi(\psi_1)\osfwith\phi(\psi_2)\osfwith \Xj[0] \doteq \Yj[0]$
  yields the \osf clause in normal form
  \[
    \begin{array}{llllllll}
      \phi & = & \Xj[0]:\s[q]         & \osfwith & \Xj:\s                  & \osfwith & \Yj[2] : \s[r]     & \osfwith \\
           &   & \Xj[0].\f \doteq \Xj & \osfwith & \Xj.\f[g] \doteq \Xj[0] & \osfwith & \Xj.\f[h] = \Yj[2] & \osfwith \\
           &   & \Xj[0] \doteq\Yj[0]  & \osfwith & \Xj[0] \doteq \Xj[2]    & \osfwith & \Xj \doteq \Yj
    \end{array}
  \]
  or an equivalent clause.
  In \cref{line:9,line:10} the set $\osftags(\phi)$ is thus
  partitioned into the equivalence classes
  $[\Xj[0]] = \{ \Xj[0], \Xj[2], \Yj[0] \}$,
  $[\Xj[1]] = \{ \Xj[1], \Yj \}$
  and
  $[\Yj[2]] = \{ \Yj[2] \}$.
  The solved part of $\phi$ is renamed accordingly (\cref{line:11,line:12})
  and translated on \cref{line:13} into the \osf term
  \[
    \psi = \zo:\s[q] (\f\to\zi:\s(\f[g]\to\zo, \f[h]\to\zd:\s[r])).
  \]
  The subsumption degree $\fisa(\psi, \psi_1)$ is then computed on
  \cref{line:14} as
  $\beta_1 =
  \min\{ \fisa(\s[q], \s[u]),\allowbreak
         \fisa(\s, \s[v]),\allowbreak
         \fisa(\s[r], \s[r]) \} = 0.4$.
  In particular,
  $\fisop(\sort_{\psi}(\zi), \sort_{\psi_1}(\Yj[1])) =
  \fisop(\s[s], \s[v]) = 0.4$.
  Similarly, $\fisa(\psi, \psi_2)$ is computed on \cref{line:15} as
  $\beta_2 =
  \min\{ \fisa(\s[q], \s[v]),
         \fisa(\s, \s[u]),
         \fisa(\s[q], \s[t]) \} = 0.5$.
  Overall, the unification degree of $\psi_1$ and $\psi_2$ is thus $0.4$
  (\cref{line:16}).
  For $i\in \{ 1,2 \}$, the mapping $h_i:\tags(\psi_i)\to\tags(\psi)$
witnessing the (syntactic) subsumption $\psi \fsynisa_{\beta_i} \psi_i$
  is defined by letting $h_i(\X) = \Z_{[\X]}$ for all $\X\in\tags(\psi_i)$.
  The unification is depicted in \Cref{fig:exunif}, where the mappings
  $h_1$ and $h_2$ are depicted as arrows relating the nodes of $G(\psi_1)$,
  $G(\psi_2)$ and $G(\psi)$ corresponding to the variables of their
  respective \osf terms.
\end{example}

% Figure environment removed
