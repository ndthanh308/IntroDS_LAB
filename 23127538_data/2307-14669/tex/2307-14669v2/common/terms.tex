\begin{definition}[\osf term \cite{AitKaci1993b}]
\label{def:osf_term}
Let $\V$ be a countably infinite set of variables (or
\emph{coreference tags}, or simply \textit{tags}). Let
$\X\in\V$, $\s\in\S$ and $\fj, \ldots,
\fj[n]\in\F$. An \emph{\osf term} is defined recursively as follows.
\begin{itemize}
    \item A sorted variable $\xs$ is an \osf term.
    \item If $t_1, \ldots, t_n$ are \osf terms, then an attributed sorted
      variable $t = \osfterm$ is an \osf term.
\end{itemize}
We let $\tags(t) \defeq \{ \X \} \cup \bigcup_{1\leq i \leq
n}\tags(t_i)$.
The variable $\X$ is called the \textit{root tag} of $t$ and is denoted
$\rtag(t)$.
\end{definition}



\begin{example}[\osf term]
  \label{ex:osf_term}
  The following \osf term shows how variables can be used to indicate
  coreference. Variables that are not used for coreference may be left
  implicit to improve readability.
    In line with \cite{AitKaci1993b}, we adhere to the convention of
    specifying the sort of each variable at most once, with the implicit
    understanding that other occurrences also refer to the same structure.
  \[
    \s[movie]
    \left(
      \begin{array}{lll}
        \f[title] & \to & \s[string],\\
        \f[directed\_by] & \to & \X[X]:\s[director]
        \left(
          \begin{array}{lll}
            \f[name] & \to & \s[string],\\
            \f[spouse] & \to & \X[Y]
          \end{array}
        \right),\\
        \f[written\_by] & \to & \X[Y]:\s[writer]
        \left(
          \begin{array}{lll}
            \f[spouse] & \to & \X[X]
          \end{array}
        \right)
      \end{array}
    \right).\qedhere
  \]
\end{example}

The definition of \osf terms given above does not rule out the presence of
redundant or even contradictory information (e.g., consider the \osf term
$\s(\f\to \si[0], \f\to\si[0], \f\to\si)$, which is contradictory if
$\si[0]\fmeet\si = \bots$). \osf terms that are well-behaved to this regard
are called
\ifrefer%
  \textit{normal \osf terms}, %
  and the reader is referred to \cite{AitKaci1993b} for their definition.
  \osf terms in normal form are also called \textit{$\psi$-terms} and
  denoted $\psi$, $\psi_i$, and so on.
  For an \osf term $\psi$ in normal form and $\X\in\tags(\psi)$, we let
  $\sort_\psi(\X)$ be the most specific sort $\s$ such that $\X:\s$ appears in
  $\psi$.
  The notation $\X\fto_\psi\Y$ indicates that there is a feature $\f$
  pointing from a subterm of $\psi$ with root tag $\X$ to a subterm of
  $\psi$ with root tag $\Y$.
  We let $\Psi$ denote the set of all normal \osf terms.%

\else
\textit{normal \osf terms} and are defined as follows \cite{AitKaci1993b}.
\begin{definition}[Normal \osf term, or $\psi$-term \cite{AitKaci1993b}]
\label{def:osf_term_normal form}
  An \osf term $t=\osfterm$ is in \textit{normal form} (or \textit{normal})
  if: (i) the root sort $\s$ is different from $\bots$,
  (ii) the features $\fj[1], \ldots, \fj[n]\in\F$ are pairwise distinct,
  (iii) each $t_i$ is in normal form, and
  (iv) for all $\Y\in\tags(t)$, there is at most one occurrence of $\Y$ in
  $t$ such that $\Y$ is the root variable of an \osf term different from
  $\Y:\tops$.

  \osf terms in normal form are also called \textit{$\psi$-terms} and
  denoted $\psi$, $\psi_i$, and so on. For an \osf term $\psi$ in normal
  form and $\X\in\tags(\psi)$, we let
  $\sort_\psi(\X)$ be the most specific sort $\s$ such that $\X:\s$ appears in
  $\psi$.
  The notation $\X\fto_\psi\Y$ indicates that there is a feature $\f$
  pointing from a subterm of $\psi$ with root tag $\X$ to a subterm of
  $\psi$ with root tag $\Y$.
  We let $\Psi$ denote the set of all normal \osf terms.
\end{definition}
\fi
