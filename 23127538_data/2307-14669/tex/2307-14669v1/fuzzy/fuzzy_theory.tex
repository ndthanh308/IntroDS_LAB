\section{Fuzzy set theory and fuzzy orders: notation and definitions}%
\label{app:fuzzy}

We recall the basic definitions of fuzzy set theory
  \cite{DuboisPrade1980} to fix the
notation.
Whenever possible, we use the same notation for fuzzy sets and fuzzy orders as the one
used for ordinary sets and orders, but with the addition of a small dot $(\cdot)$ to avoid
ambiguity. For example, the symbol for the intersection of two crisp sets
is $\cap$, while the one for the intersection of
two fuzzy sets is $\fcap$. We adopt the minimum t-norm (denoted
$\land$) and the maximum t-conorm (denoted $\lor$).

\subsection{Fuzzy sets}%

\begin{definition}[Fuzzy subset]
\label{def:fuzzy_subset}
A \emph{fuzzy subset} $F$ of a (crisp) set $X$ is determined by its
membership function $\mu_F : X \to [0, 1]$.
\end{definition}

\begin{definition}[Intersection and union of fuzzy subsets]
\label{def:intersection_of_fuzzy_subsets}
The \emph{intersection $\fbigcap \mathcal{F}$ of a set $\mathcal{F}$ of
fuzzy subsets of a set $X$} is defined by letting $\mu_{\fbigcap
\mathcal{F}}(x) \defeq \inf(\{ \mu_{F}(x) \mid F\in \mathcal{F} \})$.
The \emph{union $\fbigcup \mathcal{F}$ of a set $\mathcal{F}$ of fuzzy
subsets of a set $X$} is defined by letting $\mu_{\fbigcup \mathcal{F}}(x)
\defeq \sup(\{ \mu_{F}(x) \mid F\in \mathcal{F} \})$.
\end{definition}

\begin{definition}[Support]
\label{def:support}
  The \emph{support} of a fuzzy subset $F$ of $X$ is
  defined as $\supp \defeq \{ x\in X \mid \mu_F(x) > 0 \}$.
\end{definition}

\subsection{Fuzzy Binary Relations}%
\label{sec:fuzzy_binary_relations}

\begin{notation*}
  From now on the membership function of a fuzzy subset $F$ of $X$ will
  simply be written $F:X\to[0,1]$ instead of $\mu_F:X\to[0,1]$.
\end{notation*}

\begin{definition}[Fuzzy binary relation]
\label{def:fuzzy_binary_relation}
  A \emph{fuzzy binary relation} $R$ on a set $X$ is a fuzzy subset of
  $X\times X$, i.e., it is a function $R: X\times X \to [0, 1]$.
\end{definition}

\begin{definition}[Fuzzy preorder]
\label{def:fuzzy_preorder}
A fuzzy binary relation $R$ on a set $X$ is called a \emph{fuzzy preorder} if it
satisfies:
\begin{align*}
  \tag{Fuzzy Reflexivity} \label{eq:fuzzy_reflexivity}
  \forall x\in X,~&R(x, x) = 1,\\
  \tag{Max-Min Transitivity} \label{eq:fuzzy_transitivity}
    \forall x, y, z \in X,~&
    R(x, z) \geq R(x, y) \land R(y, z).
\end{align*}
\end{definition}

\begin{definition}[Fuzzy partial order]
\label{def:fuzzy_poset}
A fuzzy binary relation $R$ on a set $X$ is called a \emph{fuzzy partial order} if it
satisfies
\cref{eq:fuzzy_reflexivity,eq:fuzzy_transitivity} and
\begin{align*}
  \tag{Strong Fuzzy Antisymmetry} \label{eq:fuzzy_antisymmetry}
    \forall x, y \in X,~&
      \text{if}~R(x, y) > 0~\text{and}~R(y, x) > 0,
      \text{then}~x = y.
\end{align*}
The pair $\poset[R]$ is called a \emph{fuzzy partially ordered set} (\emph{fuzzy poset}).
\end{definition}

\begin{definition}[Composition of fuzzy binary relations]
\label{def:fuzzy_composition}
  The \emph{(max-min) composition} of two fuzzy binary relations $R$ and $Q$ on
  a finite set $X$ is the fuzzy binary relation $R\fcomp Q$ defined by the
  membership function
  \[
    {R\fcomp Q}(x, z) \defeq \bigvee_{y\in X}(R(x, y) \land Q(y, z)).
  \]
  The $n$-ary composition of a fuzzy binary relation $R$ with itself is defined by letting
  $R^1 \defeq R$ and $R^n \defeq R\fcomp R^{n-1}$ for $n>1$.
\end{definition}

\begin{definition}[Reflexive and transitive closure of a fuzzy binary relation]
\label{def:fuzzy_transitive_closure}
  The \emph{transitive closure} of a fuzzy binary relation $R$ %
  is defined as $R^\oplus \defeq \fbigcup_{m \geq 1}R^m$.
  The \emph{reflexive and transitive closure} $R^\oast$ of a fuzzy binary
  relation $R$ is obtained by letting $R^\oast(x, y) \defeq 1$ if $x=y$ and
  $R^\oast(x, y) \defeq R^\oplus(x, y)$ otherwise.
\end{definition}

\begin{notation*}
  From now on fuzzy preorders and fuzzy partial orders will be denoted by
  $\fisa$, and infix notation will also be used, i.e.,
  $x\fisa y$ will be written instead of $\fisop(x, y)>0$.%
\end{notation*}

We adopt the definitions of lower bounds and greatest lower bounds from
\cite{Chon2009,Mezzomo2013,Mezzomo2016}.

\begin{definition}[Lower bounds in a fuzzy poset]
\label{def:lower_bounds_in_a_fuzzy_poset}
  Let $\ftax$ be a fuzzy poset and $S \subseteq \S$. The set of
  \emph{(fuzzy) lower bounds of $S$} is defined as
  $\flb{S} \defeq
   \{ \s\in\S \mid \forall \su\in S, \s\fisa \su \}$.
\end{definition}

\begin{definition}[Fuzzy greatest lower bound]
\label{def:fuzzy_glb}
  Let $\ftax$ be a fuzzy poset and $S \subseteq \S$.
  The \emph{greatest lower bound (GLB) of $S$} is the unique
  $\s\in\flb{S}$ such that, for all $\su\in\flb{S}$, $\su\fisa \s$.
  If the GLB of $S$ exists, it is denoted
  $\fbigmeet{S}$, or simply $\s\fmeet\su$ in case $S = \{ \s, \su \}$.
\end{definition}

\begin{definition}[Fuzzy lattice and bounded lattice]
\label{def:fuzzy_lattice}
  A fuzzy poset $(\S, \fisa)$ is a \emph{fuzzy lattice} if every pair of
  elements has a GLB.
  A fuzzy lattice $(\S, \fisa)$ is \textit{bounded} if there are elements
  $\bots, \tops \in \S$ such that, for all $\s\in \S$, $\fisop(\bots, \s) =
  1$ and $\fisop(\s, \tops) = 1$.
\end{definition}

\begin{prop}[Fuzzy and crisp lattices]
\label{prop:glbs}
Let $(\S, \fisa)$ be a fuzzy lattice. Then $(\S, \supp[\fisop])$
 is a (crisp) lattice on $\S$.
Moreover, if $\meet$ is the GLB operation for $(\S, \supp[\fisop])$,
then
$\fbigmeet S = \bigmeet S$
for every subset $S\subseteq \S$.
\end{prop}
\ifallproofs
\begin{proof}
  Let $\isop \defeq \supp[\fisa]$.
  If $(\S, \fisa)$ is fuzzy poset, then $\isa$ is a reflexive,
  antisymmetric and transitive binary relation on $\S$:
  \begin{itemize}
      \item (Reflexivity) Since $\fisop(\s,\s) = 1$ for all $\s\in \S$,
        then $\s\isa\s$ for all $\s\in \S$;
      \item (Antisymmetry) If $\s\isa\s[s']$ and $\s[s']\isa \s$, then
        $\fisop(\s,\s[s']) > 0$ and $\fisop(\s[s'], \s) > 0$, and the
        antisymmetry of $\fisa$ gives $\s = \s[s']$;
      \item (Transitivity) If $\si[1]\isa\si[2]$ and $\si[2]\isa \si[3]$,
        then $\fisop(\si[1],\si[2]) > 0$ and $\fisop(\si[2], \si[3]) > 0$,
        so that max-min transitivity of $\fisa$ gives
        $\fisop(\si[1],\si[3])\geq \min(\fisop(\si[1],\si[2]),
        \fisop(\si[2],\si[3])) > 0$ and thus $\si[1]\isa\si[3]$.
  \end{itemize}
  Let $\lb{S} \defeq \{ \s \in \S \mid \s\isa \s[s'], \forall \s[s']\in S
  \}$ denote the set of lower bounds of $S\subseteq \S$ in
  $(\S, \isa)$,
  and note that $\lb{S} =\flb{S}$ for any $S\subseteq \S$.
  Let $S\subseteq \S$. Then
  \[
    \begin{array}[b]{lll}
      \s = \fbigmeet S & \Iff &
      \text{(i)}~\s\in\flb{S}~\text{and (ii)}~\forall\su\in\flb{S},\su\fisa\s\\
                        & \Iff &
      \text{(i)}~\s\in\lb{S}~\text{and (ii)}~\forall\su\in\lb{S},\su\isa\s\\
                        & \Iff & \s = \bigmeet S.
    \end{array}\qedhere
  \]
\end{proof}
\fi
