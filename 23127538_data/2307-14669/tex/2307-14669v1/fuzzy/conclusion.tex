\section{Conclusion}%
\label{sec:conclusion}

We have defined a fuzzy semantics of \osf logic in which sort
symbols denote fuzzy subsets of a domain of interpretation.
This semantics is based on a fuzzy sort subsumption lattice
$\fisop:\S\times\S\to[0,1]$ whose interpretation generalizes Zadeh's
inclusion of fuzzy sets by requiring that, whenever $\fisop(\s, \su) =
\beta$, then for any interpretation $\ii = \osfa$ and any element $d\in\dom$
the value of $\su^\ii(d)$ must be greater than or equal to the minimum of
$\s^\ii(d)$ and $\beta$.
As argued in the introduction, this notion of fuzzy subsumption may be
applied, for example, in fuzzy logic programming languages based on fuzzy
\osf term unification, or in similarity-based reasoning, where a crisp
subsumption relation $\isop\subseteq\S^2$ may be extended to a fuzzy
subsumption relation $\fisop$ according to a given similarity
$\simop:\S^2\to[0,1]$.%

The generalization to a fuzzy semantics provides \osf logic with
the capability to perform approximate reasoning.
In particular, we have shown how to decide whether two \osf terms are
subsumed by each other, and to which degree, via their unification.
The fact that the same unification procedure as crisp \osf logic can be
used to find the GLB of two \osf terms in the fuzzy subsumption lattice
constitutes a benefit in terms of developing fuzzy \osf logic
reasoners, since it would be possible to take advantage of existing
implementation techniques.%


  There are several avenues for future work. For instance, it would be
  interesting to study the semantics of fuzzy \osf logic under other
  t-norms besides the
  minimum t-norm adopted in this paper. Another direction could consist in
  developing a version of fuzzy \osf logic that supports a more expressive
  language
  that also includes negation and disjunction, and possibly
  in which features are
  interpreted as partial functions rather than total functions.
  Ultimately, the goal of this research would be to
  develop a fuzzy version of the CEDAR reasoner
  \cite{AitKaciAmir2017,AmirAitKaci2017} which is able to provide
  approximate answers to queries posed to a knowledge base by relying on a
  fuzzy sort subsumption relation or a sort similarity relation.%

