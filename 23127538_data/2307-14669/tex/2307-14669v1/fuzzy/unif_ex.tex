\begin{example}[Fuzzy \osf term unification]
  \label{ex:unif}
Consider %
the \osf terms
\[
\psi_0 = \Yj[0] : \s[u]
\left( \f[f] \to \Yj[1] : \s[v] \left( \f[g] \to \Yj[0], \f[h]\to \Yj[2]:\s[r] \right) \right)
\text{ and }
\psi_1 = \Xj[0] : \s[v]
\left( \f[f] \to \Xj[1] : \s[u]
  \left( \f[g] \to \Xj[2] : \s[t] \right)
  \right)
\]
and the corresponding clauses
$\phi(\psi_0) =
\Yj[0] : \s[u]\osfwith \Yj[0].\f[f] \doteq \Yj[1]\osfwith
\Yj[1] : \s[v]\osfwith \Yj[1].\f[g] \doteq \Yj[0] \osfwith
\Yj[1].\f[h] \doteq \Yj[2]\osfwith \Yj[2] : \s[r]$
and
$\phi(\psi_1) = \Xj[0] : \s[v]\osfwith \Xj[0].\f[f] \doteq \Xj[1]\osfwith
\Xj[1] : \s[u]\osfwith
\Xj[1].\f[g] \doteq \Xj[2]\osfwith \Xj[2] : \s[t]$.
An application of the
rules of \Cref{fig:osf_normalization} to
$\phi(\psi_0)\osfwith\phi(\psi_1)\osfwith
\Xj[0] \doteq \Yj[0]$ (with \Cref{fig:wdag} as the fuzzy subsumption) yields
\[
  \begin{array}{llllllll}
    \phi & = & \Xj[0]:\s[q]         & \osfwith & \Xj:\s                  & \osfwith & \Yj[2] : \s[r]               & \osfwith \\
         &   & \Xj[0].\f \doteq \Xj & \osfwith & \Xj.\f[g] \doteq \Xj[0] &
    \osfwith & \Xj.\f[h] = \Yj[2] & \osfwith \\
         &   & \Xj[0] \doteq\Yj[0]  & \osfwith & \Xj[0] \doteq \Xj[2]    & \osfwith & \Xj \doteq \Yj
  \end{array}
\]
or an equivalent clause.
The set
$\osftags(\phi)$
can be partitioned into the equivalence classes
$[\Xj[0]]_{\doteq} = \{ \Xj[0], \Xj[2], \Yj[0] \}$, $[\Xj[1]]_{\doteq} = \{
\Xj[1], \Yj \}$
and
$[\Yj[2]]_{\doteq} = \{ \Yj[2] \}$.
A new tag can be introduced for each class, say
$\Zj[0]$ for $[\Xj[0]]_{\doteq}$,
$\Zj[1]$ for $[\Xj[1]]_{\doteq}$ and
$\Zj[2]$ for $[\Yj[2]]_{\doteq}$.
The unifier of $\psi_0$ and $\psi_1$ can be constructed from $\phi$ using
these new variables\footnote{%
Partitioning the variables of $\phi$ into equivalence classes
and introducing new variables is not strictly necessary, but it allows each
term to maintain its own variable scope.}, resulting in
  $\psi = \Zj[0]:\s[q] (\f\to\Zj:\s(\f[g]\to\Zj[0], \f[h]\to\Zj[2]:\s[r]))$.

A function $h_i:\osftags(\psi_i)\to\osftags(\psi)$ (for $i \in \{0,1\}$) can be
defined by mapping each variable
to the tag associated with
its equivalence class, i.e., by letting
$h_1(\Xj[0]) = h_1(\Xj[2]) = h_0(\Yj[0]) = \Zj[0]$,
$h_1(\Xj[1]) = h_0(\Yj) = \Zj$ and
$h_0(\Yj[2]) = \Zj[2]$.
The subsumption degrees are
$\fisa(\psi, \psi_0) = \min\{ \fisa(\s[q], \s[u]),\allowbreak \fisa(\s,
\s[v]),\allowbreak \fisa(\s[r], \s[r]) \} = 0.4$ and
$\fisa(\psi, \psi_1) = \min\{ \fisa(\s[q], \s[v]), \fisa(\s, \s[u]), \fisa(\s[q], \s[t])
\} = 0.5$ and the unification degree of $\psi_0$ and $\psi_1$
  is thus $0.4$. The unification is depicted in
  \Cref{fig:exunif},
where the functions $h_0$ and $h_1$ are depicted as arrows
relating the nodes of $G(\psi_0)$, $G(\psi_1)$ and $G(\psi)$ corresponding
to the variables of their respective \osf terms.
\end{example}

% Figure environment removed
