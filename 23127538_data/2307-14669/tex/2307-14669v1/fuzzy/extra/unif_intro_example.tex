For example, $t_2$ is the unifier of term $t_1$ and the following
term:
\[ t_3 = \Xj[3]:\s[movie]
  \left(
    \begin{array}{lll}
      \dirby & \to & \Yj[3]:\s[director],\\
      \f[title] & \to & \X[W_3]:\strng,\\
      \f[genre] & \to & \Zj[3]:\s[horror]
    \end{array}
  \right).
\]
In particular, the value for the feature $\f[genre]$ in $t_2$ must be of
sort $\s[slasher]$, as this sort is subsumed by both $\s[thriller]$
and $\s[horror]$ (the values of $\f[genre]$ in $t_1$ and $t_3$), and it
is the most general one with this property. The unifier $t_2$ is
associated with a unification degree, which depends on the subsumption
degrees of its sorts with respect to the corresponding sorts in $t_1$ and
$t_3$. In this case the unification degree is $0.5$, due to
$\fisop(\slasher, \thriller) = 0.5$.
