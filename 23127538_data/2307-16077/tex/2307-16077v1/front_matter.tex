\begin{frontmatter}

\title{Full-field in vivo experimental study of the strains of a breathing  human abdominal wall with intra-abdominal pressure variation}

%proponuje nowy tytuk:
%\title{Full-field \textit{in vivo} deformation  of human abdominal wall}

%\tnotetext[mytitlenote]{Fully documented templates are available in the elsarticle package on \href{http://www.ctan.org/tex-archive/macros/latex/contrib/elsarticle}{CTAN}.}

%% Group authors per affiliation:
%\author{\textcolor{red}{Autorzy do podania w %kolejnosciz afiliacjami}\fnref{myfootnote}}
%\address{Radarweg 29, Amsterdam}
%\fntext[myfootnote]{Since 1880.}

%% or include affiliations in footnotes:
\author[wilis]{Katarzyna Szepietowska}


\author[wilis]{Mateusz Troka} 



\author[gumed]{Monika Lichodziejewska-Niemierko}
\author[gumed]{Michał Chmielewski}
\author[wilis]{Izabela Lubowiecka\corref{mycorrespondingauthor}}
\cortext[mycorrespondingauthor]{Corresponding author}
\ead{lubow@pg.edu.pl}

\address[wilis]{Faculty of Civil and Environmental Engineering, Gda\'nsk University of Technology, Gda\'nsk, Poland}

\address[gumed]{Department of Nephrology, Transplantology and Internal Medicine, Medical University of Gda\'nsk, Gda\'nsk, Poland}

\begin{abstract}

The presented study aims to assess the mechanical behaviour of the anterior abdominal wall based on an \textit{in vivo} experiment on humans. Full-field measurement of abdominal wall displacement during changes of intra-abdominal pressure is performed using a digital image correlation (DIC) system. Continuous measurement in time enables the observation of changes in the strain field during breathing. The understanding of the mechanical behaviour of a living human abdominal wall is important for the proper design of surgical meshes used for ventral hernia repair, which was also a motivation for the research presented below.

The research refers to the strain field of a loaded abdominal wall and presents the evolution of principal strains and their directions in the case of 12 subjects, 8 male and 4 female. Peritoneal dialysis procedure allows for the measurement of intra-abdominal pressure after fluid introduction. 

High variability among patients is observed, also in terms of principal strain direction. Subjects exhibit intra-abdominal pressure of values from 11 to 21 cmH$_2$O. However, the strain values are not strongly correlated with the pressure value, indicated variability of material properties.

 






\end{abstract}


\begin{keyword}
mechanics of abdominal wall \sep Digital Image Correlation \sep \textit{in vivo} measurements \sep strain field  \sep deformation \sep peritoneal dialysis

% principal directions etc, breathing

%\MSC[2010] 00-01\sep  99-00
\end{keyword}

\end{frontmatter}