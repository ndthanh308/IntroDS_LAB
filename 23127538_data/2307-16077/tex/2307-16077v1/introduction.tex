\section{Introduction}

Understanding the mechanical behaviour of the abdominal wall can lead to improvements in surgery, such as ventral hernia repair \citep{junge2001elasticity}, abdominal wall closure \citep{le2020differences} or in deciding the stoma location \citep{tuset2022virtual}. It is known that mismatches between deformation behaviour of the native tissue and of the implanted soft biomedical materials can lead to short and long term complications \citep{mazza2015mechanical}. For instance, in the case of a hernia, the knowledge of  human living abdominal wall deformation characteristics may help  to design surgical meshes mechanically compatible with the tissue and consequently help to improve the repair efficiency \citep{mueller2022mesh}. Another problem is that testing and comparing mechanical properties of available surgical meshes is an important  issue that has not yet been standardised \citep[see protocol propsal by][]{civilini2023reliable}. This  was addressed by \cite{tomaszewska2022combined} where the importance of appropriate test choice and its influence on the identified material properties was shown. Data on the deformation field of the abdominal wall may help to design physical and computational experiments allowing to replicate the physiological loading that the implant will undergo.


Current knowledge on the constitutive behaviour  of  abdominal wall  single components and the mechanical behaviour of the whole abdominal wall has been mainly gained by \textit{ex vivo} studies, showing the anisotropic and hyperelastic behaviour of the abdominal wall depending on the anatomical location \citep[see review by][]{deeken2017mechanical}. \textit{Ex vivo} studies on abdominal muscles mainly focus on passive behaviour \citep{calvo2014determination}, but some research on active muscle behaviour has also been  performed \citep{grasa2016active}. However, it is not clear to what extent the limitations of the \textit{ex vivo} tests restrict capability of such studies to reflect the real behaviour of the living human abdominal wall.

Medical imaging is one of the solutions for collecting data on the  \textit{in vivo} performance of the abdominal wall. \cite{tran2016abdominal} used shear wave elastography to assess the elasticity of the abdominal wall together with  measurements of local stiffness under a low external load. \cite{jourdan2022dynamic} employed dynamic-MRI to study deformation of the abdominal wall muscles during forced breathing, coughing and the Valsalva manoeuvre.

Optical measurements have also been  employed to study the external surface of the abdominal wall in a noninvasive \textit{in vivo} way.   
\cite{szymczak2012investigation} investigated strains on the external surface of the abdominal wall during activities such as bending, stretching and expiration.  Elongation between tacks connecting the surgical mesh to the abdominal wall was obtained in a similar study using X-ray images of subjects in a standing position and  bending to one side, giving information about \textit{in vivo} performance of the surgical mesh \citep{lubowiecka2020vivo}. \cite{breier2017evaluation} used a digital image correlation (DIC) system to investigate strain on the abdominal wall during various movements.
Laser scanning of external abdominal walls was also performed to compare muscle contraction with relaxation    \citep{todros20193d}.  Although the aforementioned studies provide valuable data on the \textit{in vivo} performance of abdominal walls, it is difficult to relate the deformation  with a specific loading state. \cite{song2006elasticity} tracked markers on the human abdominal wall during the measurement of gas inflation pressure in laparosocpic surgery in order to investigate the passive behaviour of the abdominal wall. A similar approach was developed by \cite{simon2015developing} who used photogrametry to investigate a rabbit's abdominal wall. Knowing the deformation  and loading state allows for the identification of the material parameters of the abdominal wall by inverse analysis \cite{simon2017towards}. Nevertheless, not much is yet known about real strain field on the human living abdominal wall. 

In our previous work \citep{in_vivo_abdomen} photogrametric measurements were performed to asses strains on the abdominal wall of subjects undergoing peritoneal dialysis (PD), when intra-abdominal pressure can also be measured. The study was based on photos taken in two states: drained abdominal wall and abdominal wall filled with dialysis fluid, when the abdominal wall is under higher pressure. The drawback of this approach was that only four images were taken from different angles in sequence, which did not allow for a full-field view at a single moment.  In both reference and loaded states, photos were intended to be taken during the exhalation phase. Nevertheless, as shown by  \cite{mikolajowski2022automated} in an elastographic study, breathing  may have an influence on the measurements of the abdominal wall muscles, which are also respiratory muscles. Therefore, in the current study, the DIC system is used to measure deformation of the  abdominal wall during the entire process of dialysis fluid introduction, which enables  capturing the effect of breathing. What is more, DIC allows for a higher resolution strain field  as well as faster measurements and data processing. 

The aim of this study is   \textit{in vivo} investigation of abdominal wall deformation. The presented approach provides novel data on the strain field of the  external  living human abdominal wall surface during   intra-abdominal changes pressure caused by dialysis fluid. What is more, active breathing is included in the analysis.  




