\section{Conclusions}

Presented here has been the deformation of the human living abdominal wall subjected to intra-abdominal  dialysis fluid pressure whilst breathing.
The study concerns \textit{in vivo} tests on human subjects and shows the changes in the strain field due to loading.  The measurements performed during peritoneal dialysis  gives the possibility of linking the deformation with intra-abdominal pressure values.   The shown strain fields are not homogeneous and exhibit high variability between subjects, both in terms of strain values and principal directions during inhalation and exhalation.   


The intention of this study is to advance a better understanding of living human abdominal wall mechanics. 
%Even if the presented research concerns only the passive behaviour of the abdominal wall, the extensive knowledge of its mechanics based on \textit{in vivo} experiments can support the optimisation of surgical strategies and the proper selection or design of the implant used in hernia repairs. 
 The knowledge of  abdominal wall mechanics based on \textit{in vivo} experiments can support the optimisation of surgical strategies for the proper selection and design of the implants used in hernia repairs. The data reported here can be further used to identify the mechanical properties of the human abdominal wall as well as validate numerical models.
 In addition, it may indicate the character of suggested exercises in order to improve mechanical properties of the abdominal wall and reduce the risk of new and recurrent hernia formation.

The high variability of the results suggests the need for a patient specific-approach to hernia repair and other issues concerning  abdominal wall mechanics (e.g. closure). Another route of inquiry that needs to be further considered is uncertainty quantification to include this variability in  simulations. What is more, there is a need to further investigate the active behaviour of the abdominal wall. Although, the surgical mesh reinforce the abdominal wall only in a passive way, the active behaviour of the surrounding muscles may influence the physiological conditions under which the implant functions. 

\section*{Acknowledgements}

We would like to thank the staff of Peritoneal Dialysis Unit Department of Nephrology Transplantology and Internal Medicine Medical University of Gda\'nsk and Fresenius Nephrocare (dr Piotr Jagodzi\'nski, nurses Ms Gra\.zyna Szyszka and Ms Ewa Malek) for their help in accessing the patients, performing PD exchanges and measurements of the IPP.

This work was supported by the National Science Centre (Poland) [grant No. UMO-2017/27/B/ST8/02518]. Calculations were carried out partially at the Academic Computer Centre in Gdansk.
