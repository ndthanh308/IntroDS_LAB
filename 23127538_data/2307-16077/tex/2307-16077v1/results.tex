\section{Results and discussion}

\subsection{Approximated error}

The DIC system enabled obtaining deformation measurements of the abdominal wall with sufficient accuracy. 
Figure \ref{EstimatedError} shows the maps of the approximated error radius of the region of interest  as seen through one of the cameras. The maps show T4, when the deformation peaks. It may be noted that higher errors usually occur  around umbilicus and in the lateral part of the abdomen whose images in the cameras are skewed.  The highest estimated error  was 0.055 mm (D8) in one point. Intraperitoneal adhesions  of subject D7 is another area of higher error.  Subject D9 had  hairs in the mid-line that disturbed  correlation in that area.  


    % Figure environment removed


%The strain results for each subject refer to the four time steps indicated here by T1, T2, T3 and T4. The reference time step, T0 refers to the very beginning of the experiment when no extra load was applied.


% Figure environment removed

% Figure environment removed


% Figure environment removed

\subsection{Displacement}

The shapes and displacements of the abdominal wall  of each subject in the four times steps in reference to the  T0 are shown in Figures \ref{fig_dic2dis},\ref{fig_dic4dis} and \ref{fig_dic8dis}. The profiles of the abdominal wall along its mid-line A-A and along its transverse direction B-B show how the wall deforms under pressure. Here the specific moments T1, T2 and T3, T4 and the reference step T0 are marked with different lines. Thus the deformation resulting from breathing is also visible. Even if the same amount of dialysis fluid is introduced into the abdominal cavity, one can notice different levels of maximum displacement in different subjects. Sometimes, breathing provokes relatively high maximum abdominal wall displacement in relation to dialysis pressure, as observed in subject D4. In that case, the IPP was also quite high, 20 cmH$_2$O (Table \ref{Table_dic_result}).  Obtained pressure values are given for T4 . Most of the subjects increased pressure by around 1 cmH$_2$O during inhalation. A similar observation was  reported by \cite{soucasse2022better} who noted that natural breathing in the supine position was in the 1--6 mmHg range.

\begin{table}[ht]
\begin{tabular}{cccccccccc}
\hline
 &               &   & \multicolumn{3}{c}{$\varepsilon_1$ [-]}   & & \multicolumn{3}{c}{$\varepsilon_2$ [-]}    \\\cline{4-6}\cline{8-10}
subject & pressure                 & & median & \multicolumn{2}{c}{percentile}  & & median & \multicolumn{2}{c}{percentile}  \\\cline{5-6}\cline{9-10}
 &    [cmH$_2$O]              & & & 25th    & 75th   & &  & 25th    & 75th    \\
\hline
    D1     & 11    &       & 0.044 & 0.033 & 0.066 &       & 0.008 & 0.002 & 0.014 \\
    D2     & 15    &       & 0.086 & 0.07  & 0.094 &       & 0.029 & 0.022  & 0.038 \\
   D3     & 11    &       & 0.081 & 0.053 & 0.104 &       & 0.001 & -0.008 & 0.017 \\
    D4     & 21    &       & 0.064 & 0.048 & 0.081 &       & 0.027 & 0.006 & 0.043 \\
    D5     & 12    &       & 0.047 & 0.025 & 0.105 &       & 0.008 & -0.012 & 0.019 \\
    D6     & 15    &       & 0.124 & 0.094 & 0.153 &       & 0.024 & 0.007 & 0.039 \\
    D7     & 21    &       & 0.063 & 0.03  & 0.113 &       & -0.002 & -0.014  & 0.006 \\
    D8     & 16    &       & 0.079 & 0.061 & 0.104 &       & 0.04  & 0.020 & 0.048 \\
    D9     & 20    &       & 0.039 & 0.028 & 0.047 &       & 0.014 & 0.003 & 0.024 \\
    D10    & 12    &       & 0.099 & 0.062 & 0.166 &       & 0.016 & -0.020 & 0.033 \\
    D11    & 18    &       & 0.043 & 0.032 & 0.049 &       & 0.019 & 0.009 & 0.028 \\
    D12    & 10    &       & 0.086 & 0.061 & 0.115 &       & 0.043 & 0.016 & 0.06 \\
\hline
\end{tabular}
\caption{Intra-abdominal pressure and principal strains statistics for each subject (T4)  } \label{Table_dic_result}
\end{table}


\subsection{Principal strains}

Figures \ref{fig_dic1strain}--\ref{fig_dic12strain} show maps of the principal strain $\varepsilon_1$ with its directions  for all the subjects in four time-steps (T).  Detailed information about the principal strains for  T4 is summarised in Table \ref{Table_dic_result} for each subject together with the intra-abdominal pressure value.
The principal strain directions in the case of some subjects (e.g. D8), changed while introducing the dialysis fluid. Principal directions and stains differed in various abdominal wall areas.  This  indicates different material properties dominating in various zones, which may account for the anisotropy of the complex abdominal wall structure. In most subjects,  breathing also affected the principal strain directions. In some parts of the abdominal wall, the directions changed during exhalation (e.g Figure \ref{fig_dic8strain}). It should be noted that sometimes, as in the case of subject D8, the principal direction change was less visible when the cavity was full (compare T1 and T2 vs T3 and T4). The distribution of the first principal direction angle in steps T1--T4 is presented in Figure \ref{fig_hist_angle}. Regarding values, in subjects D1 and D3, the principal direction angle $\alpha$ , close to $90\,^{\circ}$, was dominant, while in the other subjects,  greater variation of principal directions was observed. In the case of subject D9, when the abdominal cavity was filled, the dominating direction of the maximum principal strains changed by $90\,^{\circ}$  in relation to when it was  empty.

The difference in the distribution of principal strains in all subjects during inhaling when the abdominal cavity filled is presented in the Figure \ref{fig_histograms}. The dominating maximum principal strains are presented in Table \ref{tab_mode}. The highest mode values are observed in subjects D2, D3, D6 (female ) and D12 (male  with operated hernia).  In D3 and D6, the values of $\varepsilon_1$ and $\varepsilon_2$ differ most. On the other hand, an opposite situation is observed in case of D1, D4, D5, D7, D9 and D11, who were male subjects. The level of   minimum and maximum strain variability differed in each subject. D9 and D11 had a narrow range of minimum and maximum principal strains, i.e. strain value variability was low. These were the youngest male subjects with similar body-mass indexes (BMI). Female subject D2,  in the same age range, had slightly higher  $\varepsilon_1$ and $\varepsilon_2$ variability.  


%dic 2 \ref{fig_dic2dis}
%dic 3\ref{fig_dic3dis}
%dic4\ref{fig_dic4dis}


%% maps
% Figure environment removed

% Figure environment removed

% Figure environment removed

% Figure environment removed

% Figure environment removed

% Figure environment removed

% Figure environment removed



% Figure environment removed

% Figure environment removed

% Figure environment removed

% Figure environment removed

% Figure environment removed


    

%%
% Figure environment removed




% Figure environment removed

\begin{table}[!ht]
    \centering
    \begin{tabular}{p{0.7cm}p{0.7cm}p{0.7cm}p{0.7cm}p{0.7cm}p{0.7cm}p{0.7cm}p{0.7cm}p{0.7cm}p{0.7cm}p{0.7cm}p{0.7cm}}
    \hline
         D1 & D2 & D3 & D4 & D5 & D6 & D7 & D8 & D9 & D10 & D11 & D12 \\ \hline
        0.02-0.04 & 0.08-0.1 & 0.08-0.1 & 0.06-0.08 & 0.02-0.04 & 0.12-0.14 & 0.02-0.04 & 0.06-0.08 & 0.04-0.06 & 0.06-0.08 & 0.06-0.08 & 0.08-0.1 \\ \hline

            \end{tabular}
            \caption{Most frequent range of $\varepsilon_1$ for  subjects D1--D12 in T4}\label{tab_mode}
\end{table}


%%%%%%%%%%%%%%%%
%%%%%%%%%%%%%%%%%
%%%%%%%%%%%%%%%%%

It may be observed that not only distribution of strain values, but also the distribution of zones with principal directions differed between subjects. Figures \ref{figcontourD1}--\ref{figcontourD12} present isolines  to demostrate what ranges of the mechanical values are typical for various zones. The ranges of the principal Lagrangian strain $\varepsilon_1$, $\varepsilon_2$ values and the direction of their  $\alpha$ angle  on the abdominal walls of each subjects in T1--T4 are shown in the form of  contour maps in Figures \ref{figcontourD1}--\ref{figcontourD12}.  The contour maps contain isolines on the x-y plane of  half of the abdominal wall surface, showing the range of values observed in different areas of the abdominal wall for each subject.  Identifying these ranges can be used when planning abdominal hernia surgery to specific parts of the abdominal wall and selecting surgical implants with appropriate material properties, such as stiffness or anisotropy. In the latter case, the analysis of principal strain directions is also important. 

 

%% contour maps - rysunki


% Figure environment removed

% Figure environment removed

% Figure environment removed

% Figure environment removed

% Figure environment removed

% Figure environment removed

% Figure environment removed

% Figure environment removed

% Figure environment removed

% Figure environment removed

% Figure environment removed

% Figure environment removed



The shapes of the areas separated by the isolines differ depending on the  subject and on the analysed variable. In most subjects (D5, D7, D8, D10--D12), the isolines resemble transversal regions of the abdomen  or in other cases (D1,D4, D6, D7), have a semicircular shape around the central area of the region of interest. Zones parallel to cranio-caudal axis are observed in D3..

 The maximum principal strains on the abdominal surface in subject D1 have a semicircular shape, where higher values of strains are closer to the centre of the area of interest. However, this observation refers only to T2--T4. T1 is more homogeneous. The opposite is observed for the direction of principal strains, which becomes more homogeneous with the increase of fluid pressure. 
 
Zones with specific principal strain  directions often change in size, depending on the pressure. In particular the central zone usually expands. In terms of the principal direction, D6 is divided approximately into two halves - the lower and upper part of the abdomen. The changing direction of principal strains observed on the  abdominal wall surface may indicate that different components of the abdominal wall play different roles during the various stages.


\subsection{Strains along profile lines}

Table \ref{Table_dic_result_lines} shows mean values of strain $\varepsilon_{xx}$ and $\varepsilon_{yy}$ obtained from grid points located in the section lines A--A and B--B (see Figure \ref{fig_dic2dis}c) of each subject. Section A-A runs along the mid-line (y-direction) and section B-B runs in a transverse direction (x-direction), around 3 cm above the umbilicus. The profile lines of the strains along these sections together with principal strain 
 and $\varepsilon_{2}$are presented in Figures \ref{fig_profile_LA}--\ref{fig_profile_TA}.

The relations of $\varepsilon_{xx}$ to $\varepsilon_{yy}$  and principal strains vary between subjects and sometimes vary along a single profile. It may be observed that $\varepsilon_{yy}$ is higher than $\varepsilon_{xx}$   along a longer section of the mid-line in the case of the majority of subjects.  However, $\varepsilon_{xx}$ is higher for mid-lines of D2, D9 and D11,  who were the youngest subjects with the most muscular abdominal walls. Mean $\varepsilon_{xx}$ along the mid-line is higher in those subjects and also in the case of D5 and D12. All of the above had higher ultrafiltration (UF) than the remaining subjects. From the clinical point of view, the high UF (0.7-1L) means higher fluid pressure on the abdominal wall and may pose a risk of hernias. Predominantly longitudinal rather than  transverse higher strains  were  found in the \textit{ex vivo} study by \cite{podwojewski2014mechanical}, though in some individual cases, the opposite was observed. The predominance of the longitudinal direction, with some exceptions, is also  observed in our study. 

\begin{table}[ht]
\begin{tabular}{ccccccc}
\hline
 & &   \multicolumn{2}{c}{transverse line }   & & \multicolumn{2}{c}{midline }    \\\cline{3-4}\cline{6-7}
subject &  &$\varepsilon_{xx}$&$\varepsilon_{yy}$ & & $\varepsilon_{xx}$& $\varepsilon_{yy}$ \\
\hline
D1&  & 0.014 & 0.061   &  & 0.001 & 0.062 \\
D2&  & 0.077 & 0.037   &  &  0.077 &  0.035\\
D3&  &  0.026 & 0.054    &  & -0.006 & 0.090 \\
D4&  &   0.055 &  0.061   &  & 0.035 &  0.064\\
D5&  &  0.042 & 0.077   &  &  0.043 & 0.007 \\
D6&  &  0.031 &  0.143  &  &  0.049 & 0.111\\
D7&  & 0.005 &  0.129   &  &  -0.001 &0.086 \\
D8&  &0.057  &  0.101  &  &0.036 &  0.067\\
D9&  & 0.046 & 0.032   &  &  0.033 & 0.014\\
D10&  & 0.023 &  0.201   &  & 0.052 &  0.083\\
D11&  &  0.030  &  0.042  &  & 0.035 &  0.028\\
D12&  & 0.068 &   0.128 &  &0.058 & 0.056 \\
\hline
\end{tabular}
\caption{Mean strain [-] results in x and y direction along  A--A and B--B at T4 } \label{Table_dic_result_lines}
\end{table}


     % Figure environment removed

     % Figure environment removed

\subsection{Strains in time}

Figure \ref{fig_time} shows strain changes in time between T1--T2 and T3--T4, which both correspond to one breath of air. It may be noticed that in the majority of cases, strains $\varepsilon_1$, $\varepsilon_2$, $\varepsilon_{xx}$, $\varepsilon_{yy}$  increase  in the considered points above the umbilicus (R) and  in the lateral part level with the umbilicus(O) during  inhalation. The exception is subject D7, whose  $\varepsilon_{xx}$ and $\varepsilon_{2}$ decrease. However, the mechanical response of this subject may have been  influenced by strong inter-peritoneal adhesions. This was also the only subject in this study who had a negative median of $\varepsilon_2$  (Table \ref{Table_dic_result}. Similarly,  a single subject had a negative median of $\varepsilon_2$ in our previous  study \cite{in_vivo_abdomen}.

In the  majority of cases, principal strain $\varepsilon_1$ is higher in the mid-line point R than in the lateral point O. In the case of some subjects, this relation changes between stages. In the cases of D5 and D11, the strain in the lateral point O becomes higher in the T3--T4 phase. Conversely, in the cases of  D1, D2, D5, D6, strain  in point O is higher in the early stage T1-T2  . \cite{jourdan2022dynamic} using dynamic-MRI, also observed  that circumferential strains in lateral muscles  were higher than those of the rectus muscle during breathing. The study we present here shows that this may change when the abdominal wall undergoes the higher intra-abdominal pressure of dialysis fluid. 
The relation in the time between $\varepsilon_{xx}$ and $\varepsilon_{yy}$ differs depending on the subject. A possible explanation for this could be the differences in the  contribution of active muscles  during breathing and the passive response of the abdominal wall during the introduction of intra-abdominal pressure. 


 % Figure environment removed

\subsection{Discussion}


The high variability of strains may be observed  in the literature. A comparison of histograms obtained in this study (Figure \ref{fig_histograms} with those of  the previous  study on this subject \citep{in_vivo_abdomen}, reveals similar median values. However, the range of maximum values of $\varepsilon_1$ is higher in the current study, which can be explained by measurement on the inhalation phase and the higher resolution allowing for the observation of more local behaviour.  

 The abdominal wall is composed of various muscles and connective tissues with different fibre orientation. \cite{astruc2018characterization} showed that linea alba and rectus sheath are stiffer in transverse direction than the longitudinal one. An ex vivo study showed the rectus sheath to contribute significantly to passive response of the abdominal wall in \citep{tran2014contribution}. \textit{Ex vivo} studies of passive abdominal wall behaviour under pressure indicated the first principal direction to be along the cranio-caudal axis \citep{le2020differences}. \cite{szymczak2012investigation} obtained similar principal directions in the case of an \textit{in vivo} study of the body bending to one side. The transverse direction of abdominal wall was shown to exhibit lower strains in \citep{szymczak2012investigation,deeken2017mechanical}.  The current study reveals the  variability of principal directions among subjects. Although as in the aforementioned studies, our study reveals generally greater strains in the longitudinal direction, this was not the case with all the studied subjects. These variations in  principal directions may be explained by heterogeneity and variability of the mechanical properties and geometries of individual abdominal walls as well as differences in the active contribution of muscles, which requires further investigation. \cite{pavan2019effects} has already shown in a numerical investigation the importance of including the muscle activity in the mechanical response of the abdominal wall to pressure.  



 
 Variability of the outcomes  indicates the need for a patient-specific approach to the hernia treatment. Variability in principal directions implies that the orientation of applied anisotropic surgical meshes should be personalised. Alignment of orthotropic surgical mesh was shown to be important for the sake of junction forces minimisation \citep{lubowiecka2016preliminary}. High variability can be also addressed in the simulation of abdominal wall and surgical meshes by means of  uncertainty quantification methods \citep{szepietowska2018sensitivity}.

Due to the under-representation of women compared to men, it is hard to draw specific conclusions regarding sex.

This study has several limitations. Firstly, optical measurements were performed only on the  skin  of the abdominal wall. In the case of hernia implants, the knowledge of the strain field in the interior parts of the abdominal wall may be more useful. \cite{podwojewski2014mechanical} showed via an \textit{ex vivo} study that strains on the external surface are statistically twice as high as on the internal surface. Nonetheless, the strain pattern is different.   
Secondly, exhalation and inhalation stages are assessed by the evaluation of  abdominal wall surface deformation. A more detailed approach could be achieved by additional measurements of breathing phases \citep{mikolajowski2022automated}. It should also be noted that the subjects had been undergoing regular peritoneal dialysis for some time. Only in the case of D11, was  the subject being subjected to the very first PD. Other subjects had undergone regular PD from two months to two years prior the measurements, which may have influence on their abdominal wall response.

