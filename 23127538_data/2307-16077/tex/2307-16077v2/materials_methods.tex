\section{Materials and Methods}

\subsection{\textit{In vivo} experiments and Digital Image Correlation}

Twelve subjects, eight male and four female were tested during Peritoneal Dialysis fluid exchange (PD), Figure (\ref{fig_PD}); CAPD-continuous ambulatory peritoneal dialysis - 4 exchanges of PD solution per day or APD -A utomatic peritoneal dialysis - nightly PD performed by cycler with or without fluid left for the day. Their characteristics are included in Table \ref{Table_patient}. Although the patients did not receive any professional physiotherapy/rehabilitation exercises prior to the study, they were informed and encouraged to maintain physical activity and exercise. One can assume that their physical activity was appropriate to their age and gender agreed upon by average resident of Poland/Europe. The subjects suffer from end-stage kidney disease and regularly undergo PD. Thus, the experiments were performed during the dialysis procedure (see \cite{in_vivo_abdomen}). A Digital Image Correlation (DIC) system was applied in the study to register the motion of their abdominal walls.
DIC is an optically-based technique used to measure the evolving full-field 2D or 3D coordinates on the surface of a test piece throughout a mechanical test \citep {DICpractice}. The measured data can be used to calculate, e.g., displacements, strains and other quantities based on changes in the registered geometry of the objects undergoing movement. The system can map the motion and deformation of the tested piece by taking series of images that can replicate the motion of the speckle pattern applied to the specimen. It can  also be used to reconstruct the abdominal wall geometry. A four-camera DIC system Dantec Q-400 was used. This included 4 digital VCXU-23M cameras  with a 2.3 Mpx matrix (resolution: 1920 x 1200 px) and  VS-1620HV lenses(16 mm f/2.0–16) (Figure \ref{fig_stand}).
The stand was situated above the area of interest (front abdominal wall) in a way that enables the correlation of images of the whole abdominal wall (Figure \ref{fig_test}). A random speckle pattern was printed on a 3D printer with the use of a flexible filament in the form of a stamp. The pattern was applied to the abdomen of each subject using  approved  skin colouring paint (Figure \ref{fig_test}). A chequerboard calibration plate (35mm, 9x9 Marked White Plastic) was used to capture the relative positions of the four cameras mounted on two tripods and thus calibrate the system before the  measurement.

\begin{table}[ht]
\begin{tabular}
{p{1cm}p{1cm}p{1cm}p{1.2cm}p{1.2cm}p{1.2cm} p{3.5cm}}

\hline
No & sex & age & height  &weight & BMI  & hernia  \\
 &  & & [m]  &[kg] & [kg/m$^2$] &  \\
\hline
D1          & M            & 78           & 1.63             & 80  & 30.1  & no  \\
D2           & F            & 48           & 1.75             & 66    &21.6    & no  \\
D3           & F            & 73           & 1.60             & 68 & 26.6      & no \\
D4          & M            & 70           & 1.70             & 80.7      &27.9    & no    \\
D5           & M            & 74           & 1.67             & 84     &30.1  & no   \\
D6         & F            & 65           & 1.60             & 67   &26.2   & no   \\

D7         & M            & 88           & 1.72             & 89   &30.1    & no  \\
D8         & M            & 61           & 1.68             & 58   &20.5     & no \\
D9         & M            & 46           & 1.76             & 85   &27.4     & no  \\
D10         & F            & 72           & 1.54             & 61   &25.7    & no  \\
D11         & M            & 36           & 1.76             & 86   &27.8      & no\\
D12         & M            & 56           & 1.74             & 78    &25.8   &  an umbilical hernia repaired with an implant 2 years before the test\\
\hline
\end{tabular}
\caption{Characteristics of the patients } \label{Table_patient}
\end{table} 

\begin{table}[ht]
\begin{tabular}
%{p{1cm}p{1cm}p{1cm}p{2.2cm}p{1.2cm}}
{cccccc}
\hline
No & faceted size  & grid size & approx. grid spacing & frame rate & calibration residuum  \\
 & [px] &  [px] & [mm] &  [Hz] &  [px]    \\

\hline
D1          & 33           & 22      &   8   & 2  & 0.11 \\
D2           & 25            & 19      & 8      & 5  & 0.09   \\
D3           & 29            & 22      &   10    & 5  & 0.09    \\
D4           & 25            & 19      &   8    & 5  & 0.08    \\
D5            & 25            & 19    &    8     & 5  & 0.08   \\
D6         & 29            & 22      &   9    & 5  & 0.09   \\
D7         & 25            & 19      &   9    & 5   & 0.09   \\
D8         & 25            & 19     & 8        & 5  & 0.09  \\
D9         & 25            & 19     &  8    & 5  & 0.09    \\
D10         & 25            & 19    &  8      & 5   & 0.08   \\
D11         & 29            & 22    &   9     & 5   & 0.08  \\
D12         & 29            & 22    &   9     & 5   & 0.08   \\
\hline
\end{tabular}
\caption{DIC test parameters } \label{Table_dic_parameters}
\end{table} 


% Figure environment removed

The motion of the abdomen was registered in  pictures taken throughout the PD procedure, starting with a drained abdominal cavity and finishing with it being filled with two litres  of dialysis fluid \citep {Durand1996}. Then intraperitoneal pressure (IPP) was measured, (Figure  \ref{fig_PD}) with the use of manometer connected to the dialysis bag, as in \citep{PerezDiaz2017}. Due to the ethics issues the authors followed the standard procedure of peritoneal dialysis without any additional actions. That is why the IPP was measured only once during the procedure when the abdominal cavity was filled with the fluid. Additional measurements would imply connecting the subject to a new dialysis bag every time when measured, which might have risen the risk of infection during the procedure. We maintained the same procedure throughout the tests for the safety of the patients. It should be noted that the subjects visit the dialysis centre once a while for a control visit only. The regular peritoneal dialysis is done by the patients at home. The examination of each subject was performed only once within this study.

The collected images were processed using the commercial correlation software Istra 4.7.6.580 to determine the three-dimensional displacement of the  surface of each tested abdominal wall. For  selected parameters used in  the image analysis, see Table \ref{Table_dic_parameters}. DIC measurements have an approximated error radius (\ref{Error_equation})
\begin{equation}
    Err=\sqrt{1/3(V(x)+V(y)+V(z))},
    \label{Error_equation}
\end{equation}
which estimates the uncertainty of the 3D coordinates ($x$, $y$, $z$), where $V$ is the variance.

%------------------
% Figure environment removed

%-------------------
% Figure environment removed

%-----------------
%% Figure environment removed

%-----------------

%% Figure environment removed

The experiments were fully non-invasive and the measurement was contactless. All participants  consented to participate in the study under a protocol approved by the Independent Ethics Committee for Scientific Research at the Medical University of Gdańsk N\textsuperscript{\underline{o}} NKBBN 314/2018.

\subsection{Strain field of abdominal wall outer surface as the basis for mechanical analysis }

The DIC method can measure strains along the surface object where a normal vector $\mathbf{n}$ is determined for each measurement point. This vector defines the local tangent plane. In a standard case, the local x direction lies at the intersection of the z tangent plane with the xz plane. The local y direction is perpendicular to x and $\mathbf{n}$. All x, y values (deformation, strain) always refer to the coordinate system used in each measurement. The strain in software is calculated based on the deformation gradient and the “original” strain calculated from it, thus it is referred to as a Green-Lagrange strain measure. What is more, the deformation at a grid node is a virtual projection of deformations observed in the images of individual cameras following a given node \cite{reu2018dic}.

The postprocessing software of the used DIC system, Istra 4.7.6.58, allows to calculate the displacements of the abdominal wall surface. The system generates a grid of points (see Figure \ref{fig_test}) on the recorded surface and in every grid point the displacements and the strains are calculated. A local Green-Lagrange strain tensor represented by formula  (\ref{GL_strain_tensor})

\begin{equation}
    \boldsymbol{\mathrm{\varepsilon}}=\frac{1}{2}\left( \bm{F}^\top 
    \bm{F} - \bm{I}\right),
    \label{GL_strain_tensor}
\end{equation}
where $\bm{F}$ is the deformation gradient, was used. The eigenevalues and eigenvectors of $\boldsymbol{\mathrm{\varepsilon}}$, i.e. principal strains ($\varepsilon_1$ and $\varepsilon_2$) and directions ($\alpha$), are the quantities on which the analysis is based. $\alpha$ is the angle between transverse direction x and principal direction. Having local tensors, we obtain the local principal strain directions. This lets us to observe their changes within the pressurizing of the abdominal wall in different area of the tested surface.  More about strain calculation via DIC can be found in \cite{reu2018dic}.

Due to the catheter, which covered part of the registration field, only half of the abdominal wall was considered in the study, Figure \ref{fig_test}. The presence of the catheter disturbs the strain field of the abdominal wall, as already presented in (\cite{in_vivo_abdomen}). Four time-steps representing four different deformation states of the abdominal wall were selected for analysis so that two pairs of time-steps corresponding to one inhalation of air were captured at the early and final stages of the experiment, (Figure \ref{czas_T1T4}).  T1 and T2 denote time-steps  at ca. 20\% of the experiment duration, exhalation and inhalation, respectively. T3 and T4 denote time-steps at the final stage (T3 exhalation and T4 inhalation) of the experiment, when the abdominal cavity is filled with 2l of dialysis fluid.  

% Figure environment removed

