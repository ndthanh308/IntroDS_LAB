\documentclass{INTERSPEECH2023}

\usepackage{multirow}
\usepackage{mathtools}
\usepackage{caption,subcaption}

% 2023-01-06 modified by Simon King (Simon.King@ed.ac.uk)  

% **************************************
% *    DOUBLE-BLIND REVIEW SETTINGS    *
% **************************************
% Comment out \interspeechcameraready when submitting the 
% paper for review.
% If your paper is accepted, uncomment this to produce the
%  'camera ready' version to submit for publication.
\interspeechcameraready 


% **************************************
% *                                    *
% *      STOP !   DO NOT DELETE !      *
% *          READ THIS FIRST           *
% *                                    *
% * This template also includes        *
% * important INSTRUCTIONS that you    *
% * must follow when preparing your    *
% * paper. Read it BEFORE replacing    *
% * the content with your own work.    *
% **************************************


\title{Robust Self Supervised Speech Embeddings for Child-Adult Classification in Interactions involving Children with Autism}
\vspace{-10ex}
\name{Rimita Lahiri$^1$, Tiantian Feng$^1$, Rajat Hebbar$^1$, Catherine Lord$^2$, So Hyun Kim$^3$, Shrikanth Narayanan$^1$}
\vspace{-10ex}
%The maximum number of authors in the author list is 20. If the number of contributing authors is more than this, they should be listed in a footnote or the acknowledgement section.
\vspace{-10ex}
\address{
  $^1$Signal Analysis and Interpretation Laboratory, University of Southern California, USA\\
  $^2$Semel Institute of Neuroscience and Human Behavior, University of California, USA\\
  $^3$School of Psychology, Korea University, Korea}
\email{rlahiri@usc.edu}

\begin{document}

\maketitle
 
\begin{abstract}
% 1000 characters. ASCII characters only. No citations.
%Computational modeling of naturalistic conversations has gained a lot of attention in the past few decades because of its potential in rich human behavioral phenotyping. Such approaches using signal processing and machine learning have shown promise in helping with diagnostic and long-term behavioral monitoring in clinical realms including neuro-developmental disorders such as \textit{Autism Spectrum Disorder (ASD)}. 
% Analyzing child speech is more challenging than adult speech due to multiple factors including idiosyncrasies associated with child speech and the scarcity of appropriate datasets for training models. 
We address the problem of detecting who spoke when in child-inclusive spoken interactions i.e., automatic child-adult speaker classification. 
%related to understanding child speech known as child-adult speaker classification (who spoke when). 
Interactions involving children are richly heterogeneous due to developmental differences. The presence of neurodiversity e.g., due to Autism, contributes additional variability. We investigate the impact of additional pre-training with more unlabelled child speech on the child-adult classification performance. %Unlike existing state-of-the-art systems relying primarily on adult speech for training,
We pre-train our model with child-inclusive interactions, following two recent self-supervision algorithms, Wav2vec 2.0 and WavLM, with a contrastive loss objective. We report  $9 - 13\%$  relative improvement over the state-of-the-art baseline with regards to classification F1 scores on two clinical interaction datasets involving children with Autism. 
%Such improvements are considered to be substantial given the rich diversity and variety inherent in child speech. 
We also analyze the impact of pre-training under different conditions by evaluating our model on interactions involving different subgroups of children based on various demographic factors.
 
\end{abstract}
\noindent\textbf{Index Terms}: speech, child-adult classification, self-supervision, autism

\vspace{-2ex}
\section{Introduction}
\label{section:intro}
\vspace{-1ex}

\textit{Autism Spectrum Disorder}~(ASD) is a neuro-developmental disorder, characterized by deficits in social and communicative abilities along with restrictive repetitive behavior~\cite{volden1991neologisms,huemer2010comprehensive}. Individuals with ASD tend to show symptoms of anomalies in language, non-verbal comprehension, expressions and vocal prosody patterns~\cite{kim2014language, sorensen2019cross}. In the United States, the prevalence of ASD in children has steadily increased from 1 in 150~\cite{centers2006mental} in 2002 to 1 in 44 in 2022. It is critical to develop early ASD diagnosis to create timely interventions. One of the most common observation tools supporting ASD diagnostic and intervention efforts includes clinically-administered semi-structured dyadic interactions between the child and a trained clinician~\cite{lord2000autism,grzadzinski2016measuring}. Computational analysis of such interactions provides evidence-driven opportunities for the support of behavioral stratification as well as diagnosis and personalized treatment. 

% shown promises for supporting new, evidence-driven possibilities in support of behavioral stratification as well as diagnosis and personalized treatment. 

% Early diagnosis of ASD in children is critical for intervention. 
%\vspace{-0.5ex}
% Figure environment removed

\par However, with regards to behavioral feature extraction and analysis for these dyadic interactions, prior works have primarily relied on human-annotated data segmentation by speaker labels, which is expensive and time-consuming to obtain, especially for large corpora. Computational modeling of naturalistic conversations has gained a lot of attention in the past few decades because of its potential in rich human behavioral phenotyping. Hence, it is desirable to conduct automatic analysis of these interactions using signal processing and machine learning. Specifically, one fundamental module for supporting automated processing of child-adult interactions is the task of child-adult speech classification i.e., distinguishing the speech regions of the child from those of an interacting adult. Analysis of child speech is more challenging than adult speech because of the wide variability and idiosyncrasies associated with child \cite{lee1999acoustics,bhardwaj2022automatic,GURUNATHSHIVAKUMAR2022101289}. An additional layer of complexity arises while analyzing speech for the clinical domain, as different clinical conditions may lead to unique patterns in language and speech, making it challenging for current computational approaches to capture.


% which are difficult to capture.

% \par Self-supervised algorithms have offered rapid strides in developing solutions for a variety of speech tasks including \textit{Automatic Speech Recognition}~(ASR), speaker diarization, and speaker verification. These algorithms have gained popularity recently both because of their less reliance on labeled data which are difficult to obtain, and due to their ability to produce efficient generic feature representations suitable for a variety of downstream tasks. There are two major categories of self-supervision algorithms: generative algorithms(APC~\cite{chung2019unsupervised}, TERA~\cite{liu2021tera}) and, discriminative algorithms (wav2vec~\cite{schneider2019wav2vec}, HuBERT~\cite{hsu2021hubert}). The pre-training frameworks we choose for the present study \textit{wav2vec 2.0}~(W2V2)~\cite{baevski2020wav2vec} and wavLM~\cite{chen2022wavlm} belong to the latter category. Although these algorithms were primarily designed for ASR, recent works have leveraged them for other downstream tasks such as emotion recognition \cite{wang2021fine}, speaker identification \cite{chen2022wavlm} and speaker diarization \cite{wang2021fine}.


\par Training a robust child-adult classifier is challenging for two main factors: the scarcity of reliably labeled datasets containing child speech and the larger within-class variability due to the changes in child speech based on demographic factors like age, gender, and developmental status including any clinical symptom severity~\cite{lahiri2020learning}. Most recent works addressing the problem of speaker diarization have primarily targeted fine-tuning the pre-trained models by optimizing a supervised objective. 
%So far, the impact of using more unlabelled child speech for pre-training is relatively unexplored for child-adult classification, especially in real world settings such as clinical diagnostic and monitoring sessions.
So far, \textit{Self-Supervised Learning}~(SSL) algorithms are largely under-explored for leveraging unlabelled child speech for developing speaker discriminative embeddings, especially in real-world settings such as clinical diagnostic and monitoring sessions. Specifically, there is a limited understanding of how the performance of these models varies across children with different demographics, including age and gender.

\noindent \textbf{Contributions of this paper:} We address the above questions by evaluating the impact of including more child speech, during pre-training on the child-adult speaker classification. We choose Wav2vec 2.0~(W2V2)-base and WavLM-base+ as the backbone models. The detailed contributions of this work are summarized as:
\begin{itemize}[leftmargin=*]

\item Our work represents one of the first attempts to \textbf{leverage unlabelled child speech in pre-training} for developing speaker discriminative embeddings, especially due to the vast inherent heterogeneity in the data arising from developmental differences.
\item We experimentally substantiate the effectiveness of our method for downstream child-adult speaker classification tasks using W2V2 and WavLM and \textbf{report over  13\% and 9\% relative improvement over the base models in terms of F1 scores} in two datasets, respectively.
% which is crucial in this domain.
\item We also illustrate and analyze the performance of the proposed method among different subgroups of children based on demographic factors. 

\end{itemize}

%in spite of the vast inherent heterogeneity in the data due to developmental differences,  during pre-training on the child-adult speaker classification performance. We choose W2V2-base and WavLM-base-plus as the backbone self-supervised models. %In the first training phase, we use a contrastive loss based objective function to train the base models with unlabelled corpora containing child speech. Next, we use the trained models in the first step to extract pretrained embeddings and use the extracted embeddings for finetuning child-adult classification model with labelled data. 
%While there has been significant progress in leveraging \textit{Self-Supervised Learning}~(SSL) algorithms by fine-tuning for different downstream tasks, to the best of our knowledge, SSL algorithms are underexplored for analysing child-inclusive interactions, especially in the clinical domain. Specifically, there is limited understanding of the impact of additional pre-training for analysing such interactions. Our work addresses these questions and investigates to what extent SSL algorithms can help in understanding child speech for child-adult speaker classification. Additionally, we report the performance of our methods %effect of additional pre-training with child speech 
%among different subgroups of children based on demographic factors, to have a holistic understanding of the robustness of these models under different circumstances.

\vspace{-3ex}
\section{Background}
\vspace{-1.5ex}
\subsection{Self-supervision in speech}
\vspace{-1.5ex}

The need for building speech processing frameworks in low/limited resource scenarios has spurred significant efforts on unsupervised, semi-supervised and weakly supervised learning strategies to reduce reliance on labeled datasets. The success of SSL~\cite{devlin2018bert} in natural language processing, notably due to its generalizability and transferability, has also inspired its adoption within the speech domain. Early studies explored SSL in speech with generative loss~\cite{liu2021tera,ling2020decoar}, while more recent ones have focused on discriminative loss~\cite{baevski2020wav2vec,schneider2019wav2vec} and multi-task learning objectives~\cite{pascual2019learning,ravanelli2020multi}. The current approach in this realm follows a two-step process: first pre-train a model in a self-supervised manner on large amounts of unlabeled data to encode general-purpose knowledge, and next specialize the model on various downstream tasks through fine-tuning. Past studies have reported the efficacy of SSL algorithms by leveraging the pre-trained embeddings on downstream tasks including ASR~\cite{baevski2020wav2vec}, speaker verification~\cite{fan2020exploring}, speaker identification~\cite{chen2022wavlm}, phoneme classification~\cite{chung2019unsupervised}, emotion recognition~\cite{wang2021fine}, spoken language understanding~\cite{wang2021fine}, and TTS~\cite{alvarez2019problem}. %In this work, we report  experimental findings using two recently introduced training algorithms, W2V2~\cite{baevski2020wav2vec} and WavLM~\cite{chen2022wavlm}.

% \par W2V2 pre-training is carried out using masked language modeling following a self-supervision strategy. The model architecture consists of a \textit{Convolutional Neural Network~(CNN)} based feature encoder, a transformer based context network and a quantization module. The CNN encoder transforms the input waveform to latent representations, and contiguous timesteps from these representations are randomly masked and passed through transformers to generate a context representation. The quantization module discretizes the latent representations to form the target representations. The model is then trained to solve a contrastive task that requires identifying the true quantized representations of the masked time steps from a set of incorrect candidates using the following loss function.

% \par WavLM extends the HuBERT~\cite{hsu2021hubert} pre-training strategy by introducing a masked speech denoising and prediction framework. HuBERT uses the same masked language modeling framework as introduced by W2V2, but instead of relying on the quantization module to form the targets, they initially use clustering of MFCC features for the same, the first iteration clusters are fed for assigning second iteration training targets. For WavLM, instead of conventional masked speech modeling, a speech de-noising and prediction module is incorporated to improve model robustness and preserve speaker identity. 

\vspace{-1.5ex}
\subsection{Child-adult classification in the ASD domain}
\vspace{-1ex}

Child-adult classification is among the more difficult tasks within speaker diarization, due to the challenges related to "in the wild" child speech in naturalistic conversational settings including short speaker turns, varied noise sources and a larger fraction of overlapping speech. Early diarization solutions involving child speech
%~(both child-directed and adult-directed) 
used traditional feature representations~(MFCCs, PLPs)~\cite{najafian2016speaker, cristia2018talker}. In \cite{zhou2016speaker}, the authors introduced several methods for processing audio collected from children with autism using a wearable device. Later, deep speech representations, i-vectors~\cite{zhou2016speaker} and x-vectors~\cite{xie2019multi} were studied for this task. A variety of challenges, both from signal processing and limited data availability, have been identified and addressed. In ~\cite{lahiri2020learning}, the authors have proposed an adversarial training strategy to address the large within- and across-age and gender variability due to developmental changes in children. Alternatively, in ~\cite{koluguri2020meta}, pre-trained x-vectors were fine-tuned for child/adult speaker diarization using a meta-learning paradigm, namely prototypical networks. Moreover, the role of the amount of child speech in building deep neural speaker representations was studied in ~\cite{krishnamachari2021developing} and their experimental results confirm that including more child data indeed enhances the task performance in a supervised setup.

\vspace{-3.5ex}
\section{Datasets}
\label{sec:data}
\vspace{-1ex}
Our child-inclusive data come from interactions in a clinical setting, specifically obtained during the administration of two clinical protocols related to developmental disorders. The first protocol is the gold standard \textit{Autism Diagnostic Observation Schedule}~(ADOS)~\cite{lord2000autism}, used for diagnostic purposes. The second protocol is a recently proposed outcome-measure focused instrument \textit{Brief Observation of Social Communication Change (BOSCC)}~\cite{grzadzinski2016measuring} for tracking changes in social and communicative skills during the course of treatment. A typical ADOS session lasts $40-60$ minutes and contains multiple (usually $10-15$) semi-structured activities for addressing specific symptoms related to ASD. Usually these interactions aim to elicit spontaneous responses from children under different circumstances to obtain a diagnostic score for classifying children with and without ASD. A BOSCC session is usually 12 minutes long, consisting of two $2 min$ conversational talk sessions and two $4 min$ play sessions where the child plays with a toy. 

\par In our pre-training experiments, we use a dataset consisting of 369 recordings of unlabelled BOSCC sessions comprising approximately~$100K$ utterances. For the fine-tuning experiments, we use two different corpora, ADOSMod3 and Simons. The ADOSMod3 corpus was collected across 2 clinical sites.
%the \textit{University of Michigan Autism and Communication Disorder Center}~(UMACC) and the \textit{Cincinnati Children’s Medical Center}~(CCHMC). 
These data are from administrations of the ADOS Module-3 designed for verbally fluent children, with a focus on the Social Difficulties and Annoyance and Emotional sub-tasks for this work. The data consist of total 346 sessions collected from $165$ children ($86$ ASD, $79$ Non-ASD).
The Simons corpus used in our study consists of a combination of clinically administered ADOS~($n=6$) and BOSCC~($n=33$) sessions collected across 4 sites and these sessions were labeled by trained annotators to extract speaker timestamps. The details of datasets are reported in Table~\ref{tab:dataset}.
%Unlike the ADOSMod3 corpus, we use the speech data from all activities within the ASD corpus. 


\begin{table}[t!]
\tiny
\begin{center}
\captionsetup{justification=centering}
% \caption{Session-level statistics of child-adult corpora used in this work}
\caption{Session-level statistics of child-adult corpora.}
\vspace{-3ex}
\label{tab:dataset}
\resizebox{0.48\textwidth}{!}{
\begin{tabular}{c c c} 
\hline \hline
\multirow{2}{*}{\textbf{Dataset}} & \textbf{Duration} & \textbf{Child-speaking fraction} \\
& ($mean \pm std$) & ($mean \pm std$) \\
\hline \hline
Pre-training & $14.05 \pm 2.08$ & n.a \\
ADOSMod3 & $3.23 \pm 1.61$ & $0.46 \pm 0.18$ \\
Simons & $19.05 \pm 12.86$ & $0.40 \pm 0.08$ \\
\hline \hline
\end{tabular}}
\end{center}
\vspace{-10ex}
\end{table}

\vspace{-2ex}
\section{System Description}
\label{section:system}


% \subsection{Self Supervision Strategies}

% Past studies have demonstrated the promise of deploying self-supervised algorithms in a variety of downstream tasks like ASR, speaker diarization, speaker verification etc~\cite{yang2021superb,fan2020exploring}. In this section, we study the role of self-supervised pre-training strategies for building a robust deep-learning framework for classifying child-adult speech. Specifically, we report the experimental findings using 2 recently introduced training algorithms, W2V2~\cite{baevski2020wav2vec} and WavLM~\cite{chen2022wavlm}.

% \par W2V2 pre-training is carried out using masked language modeling following a self-supervision strategy. The model architecture consists of a \textit{Convolutional Neural Network~(CNN)} based feature encoder, a transformer based context network and a quantization module. The CNN encoder transforms the input waveform to latent representations, and contiguous timesteps from these representations are randomly masked and passed through transformers to generate a context representation. The quantization module discretizes the latent representations to form the target representations. The model is then trained to solve a contrastive task that requires identifying the true quantized representations of the masked time steps from a set of incorrect candidates using the following loss function.

% \begin{equation}
%     L_{con} = - log {\frac{exp(sim(c_t,q_t)/k)}{\sum_{\tilde{q}\in Q_t}exp(sim(c_t,\tilde{q})/k)}}
% \end{equation}

% Here $sim$ denotes the cosine similarity between the masked contextualized representations and the quantized targets, where $t$ corresponds to the masked timestep and $K$ represents the temperature. In addition to cosine similarity, the objective function incorporates an $L2$ regularization factor and a diversity loss to ensure proper utilization of the codebook entries.

% \par WavLM extends the HuBERT~\cite{hsu2021hubert} pre-training strategy by introducing a masked speech denoising and prediction framework. HuBERT uses the same masked language modeling framework as introduced by W2V2, but instead of relying on the quantization module to form the targets, they initially use clustering of MFCC features for the same, the first iteration clusters are fed for assigning second iteration training targets. For WavLM, instead of conventional masked speech modeling, a speech de-noising and prediction module is incorporated to improve model robustness and preserve speaker identity. Specifically, a subset of the input raw audio is rendered with noise and overlapped with masks and the model aims to reconstruct the original speech of the masked region. This pre-training strategy has shown promise in downstream tasks pertaining to speaker identity as it is expected that the WavLM model learns not only the ASR information by the masked speech prediction, but also the knowledge of non-ASR tasks through the speech denoising modeling.

\vspace{-0.5ex}
\subsection{Pre-training}
\label{section:pretraining}
\vspace{-1.5ex}

Our research aims to adapt the existing self-supervised approaches to the child-adult interaction domain through contrastive learning. Similar to \cite{sachidananda2022calm}, our contrastive learning framework is based on the assumption that neighboring segments from audio samples are highly likely to contain identical information. For instance, it is probable that adjacent audio frames are produced by the same speaker and are expected to contain similar semantic meaning, linguistic content, as well as acoustic characteristics. To elaborate, we define the dataset of audio samples as $N$, where each audio sample is denoted as $x_{i}$. The corresponding neighboring audio segment is represented as $x'_{i}$, and is defined as any audio sample that has a time shift of half a second or less from the original sample $x_{i}$.

As outlined in the previous section, transformer-based models first transform the input speech sample $x$ to intermediate features ${z}$ using the feature encoder $f(\cdot)$ on the basis of CNNs. Subsequently, the transformer encoder $g(\cdot)$ maps the features ${z}$ to contextualized representations ${c}$. Consequently, we can create similar pairs of contextualized representations $c_{i}$ and $c'_{i}$ from the neighboring audio segments $x_{i}$ and $x'_{i}$, with the remaining pairs being considered as negative pairs:
\vspace{-1ex}
\begin{equation}
    \text{Positive Pairs}: c_{i} \approx c'_{i}
\end{equation}
\vspace{-4ex}
\begin{equation}
    \text{Negative Pairs}: c_{i} \neq c_{k} \text{ , } c_{i} \neq c'_{k}, \text{where } i \neq k
\end{equation}

Motivated by SimCLR~\cite{chen2020simple}, we apply the NTXent contrastive loss \cite{sohn2016improved} as the pretraining objective with the adult-child conversational corpora. Given the temperature value $\tau$, the loss function $L_{NTXent}$ for the positive audio pairs $x_{i}$ and $x'_{i}$ within a batch of $B$ input audio is:

\vspace{-5ex}
\begin{equation}
    \scriptsize
    - log {\frac{exp(sim(c_{i}, c'_{i})/\tau)}{\sum^{B}_{\substack{k=0 \\ k\neq i}}exp(sim(z_{i},z_{k})/\tau) + \sum^{B}_{\substack{k=0 }}exp(sim(z_{i},z'_{k})/\tau)}}
%``
\end{equation}

\vspace{-8.5ex}
\subsection{Downstream Classifier Architectures}
\vspace{-1ex}

We use two different neural network models for child-adult speaker classification based on \cite{feng2023trustser}. Both the classifiers include a self-attention based projector module, whereas one of them uses CNNs to capture speaker characteristics and the other uses \textit{Recurrent Neural Netowrks}~(RNN) to model the temporal dependencies present in the signal.

\par The RNN-based classifier consists of a stacked sequence of a \textit{Feed Forward Layer}~(FFL), a bidirectional~\textit{Long Short Term Memory}~(LSTM) layer, a self-attention based projector layer and an output layer comprised of 2 FFLs, separated by a non-linear activation. The CNN classifier architecture is comprised of a weighted feature extraction module, followed by a convolutional module having 3 1D convolutional layers, each with a dropout and a non-linear activation in between, a self-attention based projector layer and an output layer comprised of 2 FFLs, separated by a non-linear activation. For all the experiments we use ~\textit{Rectified Linear Unit~(ReLU)} as the non-linear activation and a dropout ratio of $0.3$.

% \par We also use a self-attention based projector layer to learn speaker-discriminative representations from the RNN/CNN module outputs. This mechanism calculates the compatibility function using a feed-forward network with a single hidden layer. It can be explained as a dot product attention where the keys and the values correspond to the same representation and the query is only a trainable parameter. 
%Given the sequence of output from the RNN/CNN module $H=[h_1, h_2, h_3...h_T]^{tr}$, the self-attention layer output is expressed as,
% \begin{equation}
%     C = softmax(W_c H^T) H
% \end{equation}

% Therefore $C$ is a weighted average of $H$, where the weights are learned from the alignments computed by self-attention mechanism.

\vspace{-2ex}
\section{Experimental Setup}
\vspace{-0.5ex}
\subsection{Child Adult Classification}
\vspace{-1ex}

% The pre-trained base models for W2V2 and WavLM were trained using one CNN-based feature encoder followed by a context encoder of 12 blocks of transformer and all the models were trained predominantly on adult speech~(LibriSpeech corpus:$960 hrs$). Since processing child speech differs from adult speech, 
In this study, we hypothesize leveraging unlabelled child-speech for pre-training can guide models to learn the heterogeneous child speech and interaction patterns, leading to enhanced performance of downstream child-adult speaker classification. Instead of training from scratch, we pre-train the existing W2V2 and WavLM models with additional unlabelled child speech by unfreezing and updating specific transformer layers using a contrastive loss described in Sec~\ref{section:pretraining}. We report the child-adult classification macro F1-score on two labeled child-adult interaction corpus described in section \ref{sec:data}. We report the results in Table~\ref{tab:W2V2results} and Table~\ref{tab:WavLMresults}, where the first row denotes downstream child-adult classification performance using the model solely relying on pre-trained embeddings. The subsequent rows denote downstream task performance using the models pre-trained with additional child speech, where the number indicates the number of trainable transformer layers involved in the pre-training task. 

\begin{table}[t!]
\tiny
\begin{center}
\captionsetup{justification=centering}
\caption{Number of trainable parameters for the pre-training experiments based on unfrozen transformer layers}
\vspace{-3ex}
\label{tab:parameters}
\resizebox{0.48\textwidth}{!}{
%\begin{tabular}{c | c | c | c | c | c} 
\begin{tabular}{c | c | c | c | c}
\hline \hline
%\multirow{2}{*}{\textbf{Experiment}} & \multicolumn{5}{c}{\textbf{Number of unfrozen transformer layers}}\\
\multicolumn{5}{c}{\textbf{Number of unfrozen transformer layers}}\\
\hline
%\cline{2-6}
1 & 2 & 3 & 4 & 5 \\
\hline 
6.2M & 13.5M & 20.8M & 27.1M & 33.8M\\
\hline \hline
\end{tabular}}
\end{center}
\vspace{-9ex}
\end{table}


\vspace{-2ex}
% \subsection{Experiments on ADOSMod3 based on demographic factors}

\subsection{ADOSMod3 Experiments on Demographics}
%\vspace{-0.5ex}
\label{sec:exp_demographic}

\begin{table}[t!]
\tiny
\begin{center}
\captionsetup{justification=centering}
\caption{Child-adult classification F1 score using W2V2. PT corresponds to pre-training and the following number represents the number of layers used for pre-training. }
\label{tab:W2V2results}
\vspace{-2.5ex}
\resizebox{0.48\textwidth}{!}{
 \begin{tabular}{c | c  | c | c | c } \hline \hline
\multirow{2}{*}{\textbf{Model}} & \multicolumn{2}{c|}{\textbf{ADOSMod3}} &  \multicolumn{2}{c}{\textbf{Simons}}\\
\cline{2-5}
& RNN & CNN & RNN & CNN \\
\hline \hline
%x-vector & 63.47 & 0 & 0 & 0 \\
W2V2 - Base  & 67.92 & 70.59 & 63.41 & 64.13 \\
W2V2 - PT1 & 69.31 & 72.41 & 65.19 & 66.28 \\
W2V2 - PT2 & 71.55 & 72.95 & 65.87 & 65.12 \\
W2V2 - PT3 & 72.23 & 74.38 & \textbf{68.81} & 65.44 \\
W2V2 - PT4 & \textbf{74.01} & \textbf{74.89} & 67.63 & \textbf{66.79} \\
W2V2 - PT5 & 72.19 & 74.05 & 65.01 & 65.39 \\
\hline \hline
\end{tabular}}
\end{center}
\vspace{-9ex}
\end{table}



\begin{table}[t!]
\tiny
\begin{center}
\captionsetup{justification=centering}
\caption{Child-adult classification F1 score using WavLM pre-training. PT corresponds to pre-training and the following number represents the number of layers used for pre-training.}
\label{tab:WavLMresults}
\vspace{-2.5ex}
\resizebox{0.48\textwidth}{!}{
\begin{tabular}{c | c  | c | c | c } \hline \hline
\multirow{2}{*}{\textbf{Model}} & \multicolumn{2}{c|}{\textbf{ADOSMod3}} &  \multicolumn{2}{c}{\textbf{Simons}}\\
\cline{2-5}
& RNN & CNN & RNN & CNN \\
\hline \hline
%x-vector & 63.47 & 0 & 0 & 0 \\
WavLM-Base & 72.73 & 73.09 & 71.78 & 70.25 \\
WavLM - PT1 & 74.29 & 74.93 & 72.64 & 71.11 \\
WavLM - PT2 & \textbf{76.66} & 75.81 & \textbf{72.88} & \textbf{72.74} \\
WavLM - PT3 & 75.95 & \textbf{76.37} & 72.31 & 71.09 \\
WavLM - PT4 & 75.18 & 75.92 & 72.01 & 71.59 \\
WavLM - PT5 & 75.48 & 73.17 & 71.47 & 70.17 \\
\hline \hline
\end{tabular}}
\end{center}
\vspace{-12ex}
\end{table}

%\label{sec:exp_demographic}
\vspace{-1.5ex}
% Figure environment removed


Our study also investigates the model performance across age-groups in the ADOSMod3 corpus. Prior works~\cite{lahiri2020learning,lee1999acoustics} have reported age as an important variability factor impacting speech characteristics. Based on this hypothesis, we conduct an experiment by partitioning the ADOSMod3 corpus~($3-13yrs$) into three different age-groups~(Age-group 1: 43-90 months, Age-group 2: 91-118 months, Age-group 3: 119-158 months), such that each group contains equal number of sessions. For each of these groups, we report the child-adult classification F1 score using the pre-trained base models of W2V2 and WavLM and also the best-performing pre-trained models of those two categories.

\par Not only age, analyses of developmental changes in speech have revealed sex ("gender") differences in speech characteristics, especially post puberty~\cite{lee1999acoustics}. In this work, we also report gender-based child-adult speaker classification performance on ADOSMod3 dataset, with recordings from $244$ male and $84$ female individuals. Similar to age-focused experiments, the dataset is partitioned into male and female subsets, and comparisons are drawn between the base model and the best-performing pre-trained models for both W2V2 and WavLM.

\vspace{-1ex}
\subsection{Experimental details}
\vspace{-1ex}

For both the pre-training and fine-tuning experiments, $Adam$ optimizer is used with a batch size of $32$ samples and temperature is set to $0.1$. The number of tunable parameters for the pre-training experiments is reported in Table~\ref{tab:parameters}. The initial learning rate is set to 1e-5 and the models are trained for $30$ epochs with an early stopping callback on validation loss, patience being $5$ epochs. For the downstream child-adult classification task, the model is trained to minimize the binary cross-entropy loss for a maximum of $50$ epochs, while the initial learning rate for this experiment is 2e-4 with a weight decay of 1e-4. For both the datasets, we use 70\% for training, 15\% for validation and 15\% for testing. We use the model checkpoints from HuggingFace~\cite{wolf2020transformers}. We pre-train the models using a single NVIDIA GeForce GPU 1080 Ti and each experiment took less than two days.

\vspace{-4ex}
\section{Results and Discussion}
\label{section:results}

% Figure environment removed
%\vspace{-4.5ex}

\subsection{Classification Evaluation}
\vspace{-1ex}

In this subsection, we analyze the experimental results reported in Table~\ref{tab:W2V2results} and Table~\ref{tab:WavLMresults} to address the following questions:


\noindent \textbf{Does pre-training with more child speech improve the classification?} The results reveal that pre-training with additional child speech improves the child-adult classification F1 score over the base model. This underscores the models' ability to account for the heterogeneity that is inherent in children's speech. It can be observed that WavLM-based pre-trained models show better performance compared to W2V2 across all the experiments. Both the classifiers show comparable performance, with the RNN-based classifier yielding the best score in the majority of the experiments. Among the datasets, the experimental results reveal better F1 scores in ADOSMod3 compared to the Simons corpus. One possible reason might be related to the session recording length difference between the datasets. The average duration of sessions in Simons corpus is much higher than ADOSMod3, resulting in greater potential variability and heterogeneity, which may have degraded the F1 scores. % more child speech in the latter. Since it is more challenging to understand child speech, it possibly becomes a harder task to classify between child and adult speakers in interactions having more child speech, which may have degraded the F1 scores. 


\noindent \textbf{Does pre-training with more transformer layers improve the classification?} It is interesting to note that, while in W2V2-based pre-training, the classification F1 keeps improving by tuning more transformer layers, in the case of WavLM, the performance improvements reach the maximum with tuning fewer transformer layers. One possible explanation is that the WavLM model is trained with an objective function to capture speaker related information, helping the model to achieve the optimum performance with lesser training. However, in both the scenarios, the model performance starts to degrade by adding more than four transformer layers. As these models are designed to provide generalized speech representations, tuning larger portions of these pre-trained models on a relatively smaller dataset might lead to the loss of generalizability, causing the performance to decrease for the classification task. However, our results provide compelling evidence that it is beneficial to adapt the last few transformer layers for the adult-child classification.


\noindent \textbf{Qualitative analysis} We present t-SNE visualizations of pre-trained embeddings for $2$ output classes from two sessions in Figure \ref{fig:tsne}. We plot the embeddings with and without additional pre-training. In both cases, it is evident from the plots that our method increases the discriminative information between them.

\subsection{Result Evaluation based on Demographics}
\vspace{-1ex}
\par For the gender-focused experiments, the relative improvement in F1 scores are $6.39\%$ and $3.14\%$ for the male and female subsets. Possibly due to both inherent speech pattern differences and inherent data distribution biases (see Section~\ref{sec:exp_demographic}), the models yield higher F1 scores in the male population than the female population. %This finding points to a possible future research direction to ensure the model perform fairly across gender. 
For the age-focused experiment, the relative improvements of $9.42\%$, $8.23\%$, and $4.06\%$ are seen for the three age-groups (youngest to oldest). The results imply that it is intrinsically challenging to model children of AG1 and AG2 due to the developing vocal tract behaviors among these ages. As a consequence, adding more children speech in training data provides greater benefits to the model to capture more relevant information, resulting in more improvement in the younger age-group than the older ones.

% The classification results suggest possibly due to the developing vocal tract in young children the challenge is still higher for AG1 and AG2, so adding more children speech in training data helps the model to capture more information helpful for the downstream task, resulting in more improvement in younger age-group categories than older ones.

% Figure environment removed

%  The results reported in Table~\ref{tab:W2V2results} and Table~\ref{tab:WavLMresults} clearly reveal that pre-training with additional child speech improves the child-adult classification F1 score over the base model. This underscores the models' ability to account for the heterogeneity that is inherent in children's speech. It can be discovered that WavLM-based pre-trained models show better performance compared to W2V2 across all our experiments. Both classifiers show comparable performance, with the RNN-based classifier yielding the best score in the majority of the experiments. Among the datasets, the experimental results reveal better F1 scores in ADOSMod3 as compared to Simons corpus. The average duration of sessions in Simons corpus is much higher than ADOSMod3, resulting in more child speech in the latter. Since it is more challenging to understand child speech, possibly it becomes a harder task to classify between child and adult speaker in interactions having more child speech, which may have degraded the F1 scores in Simons corpus. It is interesting to note that, while in W2V2-based pre-training experiments the classification F1 score kept improving by tuning more number of transformer layers, in case of WavLM, an opposite trend of improvement with tuning fewer number of transformer layers was seen. Since WavLM model is already trained using an objective function that is expected to capture speaker related information, the models need lesser training to achieve the optimum performance. 

% \par For the gender-focused experiments, the relative improvement in the F1 score is $6.39\%$ and $3.14\%$ for the male and female subsets, respectively. Possibly due to both inherent speech pattern differences and inherent data distribution biases, the models yield different F1 scores across gender. For the age-focused experiment, relative improvement of $9.42\%$, $8.23\%$ and $4.06\%$ are seen for the three age-groups (youngest to oldest) respectively. The classification results suggest possibly due to the developing vocal tract in young children the challenge is still higher for AG1 and AG2, so adding more children speech in training data helps the model to capture more information helpful for the downstream task, resulting in more improvement in younger age-group categories than older ones.

% \par As qualitative analysis, we present TSNE visualizations of the pre-trained embeddings for $2$ output classes from two sessions in Figure \ref{fig:tsne}. We plot the embeddings with and without additional pre-training. In both cases, it is evident from the plots that our method increases the discriminative information between them.

%\vspace{-0.5ex}

\vspace{-2.5ex}
\section{Conclusion}
\label{section:conclusion}
\vspace{-0.5ex}

Past work has demonstrated the promise of deploying self-supervised algorithms in a variety of downstream tasks like ASR, speaker diarization, and speaker verification~\cite{yang2021superb,fan2020exploring}. In this work, we investigate the utility of additional pre-training with more child speech, even in the presence of the inherent heterogeneity and variability, to improve child-adult speaker classification in clinical recordings involving interactions with children with autism. The experimental results with the proposed models  support our hypothesis of benefiting from incorporating child speech based additional pre-training, across both age and gender dimensions of variability. 

\par In this work, we used the manually-annotated ground truth labels for identifying and evaluating the speech and non-speech regions. In the future, we plan to build a child-adult diarization framework with an integrated \textit{Voice Activity Detection}~(VAD) system to further reduce the need of human effort.  %and to address a more naturalistic use case. 
In addition, we plan to extend this study with an additional emphasis on early vocalization and speech  (from toddlers and infants) in the interaction. Unlike verbally fluent children, toddler speech contains significant amounts of pre-verbal sounds and non-verbal vocalizations, which pose additional challenges for automated processing.

\section{Acknowledgements}
\vspace{-1.5ex}
This work is supported by funds from USC Hearing, Communication and Neuroscience~(HCN) pre-doctoral fellowship, NIH and Simons foundation.
\bibliographystyle{IEEEtran}
\bibliography{main}

\end{document}
