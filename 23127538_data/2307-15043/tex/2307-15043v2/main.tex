\documentclass[11pt]{article}
\usepackage[margin=1in,letterpaper]{geometry}
\usepackage{graphicx} %
\usepackage[bf,sf]{titlesec}
\usepackage{setspace}
\usepackage{titling}
\usepackage{natbib}
\usepackage{amsmath}
\usepackage{amssymb}
\usepackage[table,dvipsnames]{xcolor}
\usepackage{algorithm}
\usepackage{comment}
\usepackage[noEnd=True]{algpseudocodex}
\usepackage{multirow}
\usepackage{booktabs}
\usepackage{makecell}
\usepackage{subcaption}
\usepackage{fancybox}
\usepackage{fancyvrb}
\usepackage{multirow}
\usepackage[colorlinks]{hyperref}

\usepackage{listings}
\lstset{
backgroundcolor=\color{lightgray},
frame=lr,
basicstyle=\small\ttfamily,
columns=flexible,
breaklines=true
}

\hypersetup{
  colorlinks   = true, %
  urlcolor     = RoyalBlue, %
  linkcolor    = RoyalBlue, %
  citecolor   = RoyalBlue %
}



\makeatletter
\renewenvironment{abstract}{%
    \if@twocolumn
      \section*{\abstractname}%
    \else %
      \begin{center}%
        {\sffamily \bfseries \abstractname\vspace{\z@}}%
      \end{center}%
      \quotation
    \fi}
    {\if@twocolumn\else\endquotation\fi}
\makeatother

\definecolor{lightgray}{gray}{0.95} %
\newenvironment{FVerbatim}
{\VerbatimEnvironment
  \setlength{\fboxsep}{0.1in}
  \begin{Sbox}
    \begin{minipage}{0.9\columnwidth}
    \begin{Verbatim}[commandchars=\\\{\}]}
{\end{Verbatim}
  \end{minipage}
  \end{Sbox}
  \begin{center}
    \fcolorbox{black}{lightgray}{\TheSbox}
  \end{center}
}





\renewcommand{\algorithmicrequire}{\textbf{Input:}}
\renewcommand{\algorithmicensure}{\textbf{Output:}}
\renewcommand{\algorithmicloop}{\textbf{repeat}}
\newcommand{\todo}[1]{\textcolor{red}{#1}}
\newcommand{\andy}[1]{\textcolor{blue}{#1}}
\newcommand{\textbsf}[1]{\textsf{\textbf{#1}}}

\DeclareMathOperator*{\minimize}{minimize}
\DeclareMathOperator*{\argmin}{argmin}


\title{\textbsf{Universal and Transferable Adversarial Attacks} \\ \textbsf{on Aligned Language Models}}
\author{Andy Zou$^{1,2}$, Zifan Wang$^2$, Nicholas Carlini$^{3}$, Milad Nasr$^{3}$, \\
J. Zico Kolter$^{1,4}$, Matt Fredrikson$^1$ \vspace{2pt} \\ 
$^1$Carnegie Mellon University, $^2$Center for AI Safety, \\ $^3$ Google DeepMind, $^4$Bosch Center for AI \vspace{2pt} \\
}
\date{}

\begin{document}
\maketitle

\begin{abstract}


Because ``out-of-the-box'' large language models are capable of generating a great deal of objectionable content, recent work has focused on \emph{aligning} these models in an attempt to prevent undesirable generation.  While there has been some success at circumventing these measures---so-called ``jailbreaks'' against LLMs---these attacks have required significant human ingenuity and are brittle in practice. Attempts at \emph{automatic} adversarial prompt generation have also achieved limited success.  In this paper, we propose a simple and effective attack method that causes aligned language models to generate objectionable behaviors.  Specifically, our approach finds a suffix that, when attached to a wide range of queries for an LLM to produce objectionable content, aims to maximize the probability that the model produces an affirmative response (rather than refusing to answer). However, instead of relying on manual engineering, our approach automatically produces these adversarial suffixes by a combination of greedy and gradient-based search techniques, and also improves over past automatic prompt generation methods.

Surprisingly, we find that the adversarial prompts generated by our approach are highly \emph{transferable}, including to black-box, publicly released, production LLMs.  Specifically, we train an adversarial attack suffix on \emph{multiple} prompts (i.e., queries asking for many different types of objectionable content), as well as \emph{multiple} models (in our case, Vicuna-7B and 13B).  When doing so, \textbf{\emph{the resulting attack suffix induces objectionable content in the public interfaces to ChatGPT, Bard, and Claude}}, as well as open source LLMs such as LLaMA-2-Chat, Pythia, Falcon, and others.  Interestingly, the success rate of this attack transfer is much higher against the GPT-based models, potentially owing to the fact that Vicuna itself is trained on outputs from ChatGPT.  In total, this work significantly advances the state-of-the-art in adversarial attacks against aligned language models, raising important questions about how such systems can be prevented from producing objectionable information. 
Code is available at \href{https://github.com/llm-attacks/llm-attacks}{\texttt{github.com/llm-attacks/llm-attacks}}.
\clearpage


\end{abstract}


\section{Introduction}
\label{sec:introduction}

The recent surge of Large Language Models (LLMs), such as GPT-3.5/4~\cite{bubeck_sparks_2023}, PaLM~\cite{chowdhery_palm_2022}, FLAN-T5~\cite{chung_scaling_2022}, and Alpaca~\cite{taori_stanford_2023}, has shown a promising trend of large pre-trained models to do a variety of tasks in a zero-shot setting (\ie without any new training data). Example tasks include question answering~\cite{omar2023chatgpt,robinson2023leveraging}, logic reasoning~\cite{wei_chain--thought_2023,zhou_least--most_2023}, machine translation~\cite{brants2007large,gulcehre2017integrating} \etc\ 
A number of experiments have revealed that, built on hundreds of billions of parameters, these LLMs have started to show the capability to understand the human common sense beneath the natural language and do proper reasoning and inference accordingly~\cite{bubeck_sparks_2023,nori_capabilities_2023}.

Among different applications, one particular question yet to be answered is how well LLMs can understand human mental health states through natural language.
Mental health problems represent a significant burden for individuals and societies worldwide.
A recent report suggested that more than 20\% of adults in the U.S. would experience at least one mental disorder in their lifetime~\cite{mental2022state} and 5.6\% of adults experienced a serious psychotic disorder that significantly impairs functioning~\cite{mental2023stats}. The global economy loses around \$1 trillion annually in productivity due to depression and anxiety alone~\cite{mentalcost2023}.

In the past decade, there has been a plethora of research in natural language processing (NLP) and computational social science on detecting mental health issues via online text data such as social media~(\eg \cite{guntuku_detecting_2017,eichstaedt2018facebook,coppersmith_clpsych_2015,de_choudhury_social_2013,de_choudhury_mental_2014}). However, most of these studies have focused on building domain-specific machine learning (ML) models (\ie one model for one particular task, such as stress detection~\cite{nijhawan2022stress,guntuku2019understanding}, depression prediction~\cite{eichstaedt2018facebook,tadesse2019detection,xu_leveraging_2019}, or suicide risk assessment~\cite{de_choudhury_discovering_2016,coppersmith2018natural}). Even for traditional pre-trained language models such as BERT, it needs to be finetuned for specific downstream tasks~\cite{devlin_bert_2019,liu_roberta_2019}.
Since natural language is a major component of mental health assessment and treatment~\cite{sharma2018mental,gkotsis2016language}, LLMs might be a potentially powerful tool to understand end-users' mental states based on the language users' wrote. These instruction-finetuned and general-purpose models can understand a variety of inputs and obviate the need to train multiple models for different tasks. Thus, we can envision using one LLM for a variety of mental-health-related tasks, such as multiple question-answering, reasoning, and inference.
Such a vision opens up a wide range of opportunities for UbiComp, Human-Computer Interaction (HCI), and mental health communities, such as online public health monitoring systems~\cite{patel2018psyheal,graham2019artificial}, intelligent assistants for mental counselors and supporters~\cite{sharma_towards_2021,sharma_humanai_2023}, mental-health-aware personal chatbots~\cite{abd2021perceptions,denecke2020mental}, to just name a few.
However, there is a lack of investigation into understanding, evaluating, and improving the capability of LLMs for mental health prediction tasks.

There are few very recent studies on the evaluation of LLMs (\eg ChatGPT) on mental-health-related tasks, most of which are in zero-shot settings with simple prompt engineering~\cite{yang_evaluations_2023,amin_will_2023,lamichhane_evaluation_2023}. Researchers have shown preliminary results that LLMs have some initial capability of predicting mental health disorders with natural language with some promising but still limited performance compared to state-of-the-art domain-specific NLP models~\cite{yang_evaluations_2023,lamichhane_evaluation_2023}.
This remaining gap is expected since existing general-purpose LLMs are not specifically trained on mental health tasks.
However, to achieve our vision of leveraging LLMs for mental health support and assistance, we need to answer the research question: \textbf{How to empower LLMs with more mental health domain knowledge and become an expert}?

We conducted a series of experiments with multiple LLMs, including Alpaca~\cite{noauthor_stanford_2023}, Alpaca-LoRA~\cite{hu_lora_2021}, and GPT-3.5~\cite{noauthor_introducing_2022}.
Considering the data availability, we focused on online social media data with high-quality human-generated mental health labels.
Our experiments contained three stages: (1) zero-shot prompting, where we experimented with various prompts related to mental health, (2) few-shot prompting, where we inserted examples into prompt inputs, and (3) instruction-finetuning, where we finetuned LLMs on multiple mental-health datasets with various tasks.

Our results indicate that zero-shot obtained promising but limited performance on multiple mental health prediction tasks across all models. GPT-3.5 had relatively better results since it has a larger scale. But their performance is still far from task-specific models. 
Meanwhile, providing a few shots in the prompt can improve the model performance to some extent ($\overline{\Delta}$ = 4.7\%), but the advantage is limited.
Finally and most importantly, we found that instruction-finetuning can significantly improve the model performance across multiple mental-health-related tasks at the same time. Our finetuned Alpaca, namely \textbf{Mental-Alpaca}, significantly outperforms the original GPT-3.5 ($\times$25 times of model size) by an average of 16.7\% on balance accuracy. 
Meanwhile, Mental-Alpaca can further perform on par with the task-specific state-of-the-art Mental-RoBERTa~\cite{ji_mentalbert_2021}. It is noteworthy that Mental-RoBERTa needs to be trained on each task individually, 
while our Mental-Alpaca can solve different tasks off the shelf. 
% We open-source our training code and model at [github link].
Our experiments present the first comprehensive evaluation of various techniques to enhance LLMs' capability in the mental health domain.

The contribution of our paper can be summarized as follows:
\begin{s_enumerate}
\item We present the first comprehensive evaluation of prompt engineering, few-shot, and finetuning techniques on multiple LLMs in the mental health domain.
\item With online social media data, our results reveal that finetuning on a variety of datasets can significantly improve LLM's capability on multiple mental-health-specific tasks simultaneously.
% We release our model \textbf{Mental-Alpaca} as the first open-source LLM targeted at mental health prediction tasks.
\item We provide a few technical guidelines for future researchers and developers on turning LLMs into experts in specific domains.
\end{s_enumerate}

\input{sections/3b-method}
\section{Experimental Results: Direct and Transfer Attacks}
\label{sec:eval}


Existing adversarial benchmarks for LLMs~\citep{wang2021adversarial,zhu2023promptbench} focus on generating attacks that aim to fool specific tasks, such as sentiment analysis or natural language inference. Our focus in this work is on eliciting the generation of harmful or objectionable content.
To systematically evaluate the effectiveness of our approach at accomplishing this, we designed a new benchmark, \emph{AdvBench}, based on two distinct settings.
\begin{enumerate}
    \item \textbf{Harmful Strings}: A collection of $500$ strings that reflect harmful or toxic behavior, encompassing a wide spectrum of detrimental content such as profanity, graphic depictions, threatening behavior, misinformation, discrimination, cybercrime, and dangerous or illegal suggestions. The adversary's objective is to discover specific inputs that can prompt the model to generate these \emph{exact} strings. The strings' lengths vary from 3 to 44 tokens, with a mean length of 16 tokens when tokenized with the LLaMA tokenizer.
    \item \textbf{Harmful Behaviors}: A set of $500$ harmful behaviors formulated as instructions. These behaviors range over the same themes as the harmful strings setting, but the adversary's goal is instead to find a single attack string that will cause the model to generate \emph{any} response that attempts to comply with the instruction, and to do so over as many harmful behaviors as possible.
\end{enumerate}
Both these settings evaluate the ability of a target model to robustly adhere to alignment goals from slightly different perspectives: the first task focuses on fine-grained control over the model's outputs, whereas the second resembles a red-teaming exercise that aims to bypass safety filters to elicit harmful output. 
Appendix A provides additional details on the construction of this benchmark.

We emphasize that the goal of this dataset is not to open a debate about what should not be ``allowable'' content from an LLM, but rather simply note that the majority of aligned LLMs (both open source and proprietary) attempt to explicitly avoid such generations, and will not produce them if queried directly (i.e., asked to generate one of the harmful strings, or asked to exhibit the harmful behaviors).  Thus, the benchmark explicitly evaluates how well our attack is able to circumvent the specified guardrails of the system. In the future it may be necessary to adjust our benchmark as the set of tasks models intend to prohibit changes.

\paragraph{Metrics.}
We use Attack Success Rate (ASR) as the primary metric for AdvBench. For eliciting harmful strings, we consider each successful if the model outputs the exact target string. Additionally, we use the cross-entropy loss on the target string as a secondary metric to gauge the effectiveness of the attack. For Harmful Behaviors, we deem a test case successful if the model makes a \emph{reasonable} attempt at executing the behavior. As different models exhibit varying ability to provide, for example, a correct set of instructions for building an explosive device, this may involve human judgment to determine that a response did not amount to a refusal, or an attempt to evade generating harmful content. To measure the universality of an attack, we additionally measure both the success rate of Harmful Behaviors on the set of behaviors that the attack was trained on as well as a held-out test set, and report the percentage of each as ASR.

\paragraph{Baselines.} We compare our method with three prior baseline methods: PEZ~\citep{wen2023hard}, GBDA~\citep{guo2021gradient}, and AutoPrompt~\citep{shin2020autoprompt}. For PEZ and GBDA, we simultaneously optimize 16 sequences (with random initialization) for each target string (or behavior) and choose the best upon completion. Candidates are optimized using Adam with cosine annealing. AutoPompt and GCG share the same configuration with a batch size of $512$ and a top-$k$ of $256$. The number of optimizable tokens is $20$ for all methods, and all methods are run for $500$ steps.

\paragraph{Overview of Results.}
We will show that GCG (Algorithms~\ref{alg:gcg} and \ref{alg:universal-opt}) is able to find successful attacks in both of these settings consistently on Vicuna-7B and Llama-2-7B-Chat. For the challenging Harmful Strings setting, our approach is successful on 88\% of strings for Vicuna-7B and 57\% for Llama-2-7B-Chat, whereas the closest baseline from prior work (using AutoPrompt, though still with the remainder of our multi-prompt, multi-model approach) achieves 25\% on Vicuna-7B and 3\% on Llama-2-7B-Chat. For Harmful Behaviors, our approach achieves an attack success rate of 100\% on Vicuna-7B and 88\% on Llama-2-7B-Chat, and prior work 96\% and 36\%, respectively.

We also demonstrate that the attacks generated by our approach transfer surprisingly well to other LLMs, even those that use completely different tokens to represent the same text. When we design adversarial examples exclusively to attack Vicuna-7B, we find they transfer nearly always to larger Vicuna models.
By generating adversarial examples to fool both Vicuna-7B \emph{and} Vicuna-13b simultaneously, we find that the adversarial examples also transfer to Pythia, Falcon, Guanaco, and surprisingly, to GPT-3.5 (87.9\%) and GPT-4 (53.6\%), PaLM-2 (66\%), and Claude-2 (2.1\%). To the best of our knowledge, \emph{these are the first results to demonstrate reliable transfer of automatically-generated universal ``jailbreak'' attacks over a wide assortment of LLMs}.


\input{sections/7-experiments}
\section{Uncertainty}
\label{sec:background}
Uncertainty is a rich concept that has received various reasonable treatments before today's understanding of it.\footnote{From uncertainty's connection to (mostly abandoned) views on what is `knowable' \citep{knight1921risk}, to its central role in decision theories  \citep{ramsey1931foundations,von1947theory,Wald1951StatisticalDF,bernardo1994bayesian} and mathematical statistics \citep{savage1972foundations} to its modern understanding in terms of state of knowledge \citep{morgan_henrion_1990,lindley2013understanding}, to its  mathematical representation detached from philosophical interpretation \citep{halpern2017reasoning}.} We begin discussing it through common language. The online edition of the Oxford English dictionary listed five senses of uncertainty (retrieved in May 2023), two of which we partly quote here (those general enough to include the others as special case): \emph{(i) the state of not being definitely known or perfectly clear}; and \emph{(ii) the state or character of being uncertain in mind}. Both definitions regard uncertainty as \emph{a state of affairs}: in \emph{(i)}, the state of the world; in \emph{(ii)}, the state of an agent contemplating the world. They are subtly different: \emph{(i)} encompasses situations of inherent randomness (\eg, the result of a coin flip), \emph{(ii)} concerns one's inability to predict the state of the world regardless of any inherent randomness (\eg, a reader wondering about the content of the next paragraph). As we shall see, this difference leads to rather different interpretations %
of uncertainty as an aspect of reality. Yet, at the level of mathematical treatment, they share the same formal devices. Hence, with no loss of generality, we choose to talk about uncertainty from the point of view of an agent contemplating or interacting with the world, while possessing limited knowledge about it. Our presentation is inspired by various reference texts,  in particular, \citet{dubois2009formal} and \citet{halpern2017reasoning}. 

\paragraph{Agents.} We posit that any one agent shall represent the state of their knowledge in a way sufficient for reasoning about the truth value of claims (or propositions) that they make about aspects of the world. In particular, the agent is able to state their preference for claims they find themselves less uncertain about (\ie, possessing better information about those).\footnote{Agents and worlds are abstractions to be adapted and tailored to each application, commonly in NLG an agent is a model and a world is a response to a given prompt.}  An agent then uses this \emph{uncertainty representation} to interact with the environment (\eg, inform their actions) and, when they acquire new knowledge, they update the representation in a coherent manner. 
To illustrate formal concepts, we use three example agents. \textbf{A1}\hspace{0.5mm}\twemoji{game die} rolls a six-sided die; we seek to represent their state of knowledge about the outcome. \textbf{A2}\hspace{0.5mm}\twemoji{busts in silhouette} resolves mentions of entities to unique names in a knowledge base (KB); we seek to represent their state of knowledge about entity names given any one mention. Last, \textbf{A3}\hspace{0.5mm}\twemoji{speech balloon} provides written answers to questions; we seek to represent their state of knowledge about answers given any one question. 
For simplicity, we assume that our agents already acquired their knowledge, by means which are not relevant for now, and their state of knowledge is frozen.  
We begin by outlining the formal tools common to all frameworks for uncertainty representation we are aware of, we then zoom into the most commonly used framework (probability) and discuss the role of statistics in acquisition and revision of knowledge.  

\paragraph{Possible worlds.} Our agent does not know the state of the actual world, but they assume that it must be one of a collection of possible worlds (the universe). They represent a world as a unique symbol (or string, or collection of attributes; the level of detail being dictated by the agent's needs), and the universe of what is possible as a set $\Omega$ of mutually exclusive worlds.\footnote{This framework, \textit{possible worlds}, is familiar to linguists and philosophers alike \citep{hintikka1957modality,hintikka1961modality,sep-possible-worlds}.
} 
\textbf{A1}\hspace{0.5mm}\twemoji{game die}  might represent a world as a symbol $f_k$, with $k$ denoting the number of pips the die shows as a result of the roll; they might assume the die always lands showing one of six numbered faces and thus take $\{f_1, \ldots, f_6\}$ to represent all possible worlds. 
For \textbf{A2}\hspace{0.5mm}\twemoji{busts in silhouette}, a world is a symbol like $e_i$, with $i$ denoting an entity's identifier (\eg, a standardised unique name), and the universe is the finite set of entities in the English Wikipedia. 
For \textbf{A3}\hspace{0.5mm}\twemoji{speech balloon}, a world is a symbol like $u_s$, with $s$ an English sentence produced in response to a question. This agent happens to be unable to describe the set of all valid English sentences (they cannot enumerate its elements nor state a finite set of properties that all valid sentences must satisfy). Motivated by convenience, \textbf{A3}\hspace{0.5mm}\twemoji{speech balloon} uses a set large enough to encompass most of it while being specifiable in a compact manner: the set of all finite-length strings made by concatenation of known symbols (\eg, words, punctuation, \etc). These examples show that the agent's choice of universe can be a difficult one, often requiring simplifying assumptions: on soft or irregular terrain, a die could land on an edge; a KB may be incomplete (sometimes in known ways, \eg, under-representing the contributions of Black women to science); a regular language is a too loosely constrained representation of the English language (\eg, it includes infinitely many strings that will never correspond to any actual world). 

\paragraph{Propositions.} The possible worlds framework gives agents a mechanism to represent claims about specific aspects of the world. A \emph{proposition}  $E_t$ is the claim that the actual world $\omega$ is one where some property $t$ holds (which we denote $t(\omega)$). A property is something that can be assessed for any one world (\eg, $f_2$ is even and prime, $e_{\operatorname{Katherine\_Johnson}}$ is African-American, female and mathematician, $u_\text{`Biden is the 46th US president'}$ expresses the relation $\operatorname{presidentof}(\operatorname{Joe\_Biden}, \operatorname{USA})$). Not knowing the state of the actual world,  our agent represents $E_t$ by the set  $E_t = \{w \in \Omega: t(w)\} \subseteq \Omega$ of all possible worlds where the property holds. If the agent knew the state of the actual world $\omega$, then the truth value of the proposition would be determined simply by set membership (\ie, $\omega \in E_t$ or $\omega \not\in E_t$). 
For example, \textbf{A1}\hspace{0.5mm}\twemoji{game die}  represents the claim ``the roll is odd'' as $E_{\text{odd}} = \{f_1, f_3, f_5\}$.
\textbf{A2}\hspace{0.5mm}\twemoji{busts in silhouette} represents the claim ``mention to a female mathematician'' by the set $\{e_i \in \Omega: \operatorname{female}(e_i) \wedge \operatorname{mathematician}(e_i)\}$. %
\textbf{A3}\hspace{0.5mm}\twemoji{speech balloon} might use equivalence classes, for example, they use the set $E_a = \{u_s \in \Omega : \operatorname{equivalent}_{a}(u_s)\}$ to claim that the answer is a sentence semantically equivalent to some other sentence $u_a \in \Omega$ (\eg, $u_\text{`The 46th US president is Joe Biden'}$). Because propositions are semantic in nature, they can be difficult to represent explicitly. For example, \textbf{A3}\hspace{0.5mm}\twemoji{speech balloon}'s equivalence classes require sophisticated natural language understanding. 
A representation of all propositions an agent deems possible is a set $\mathcal E$ of subsets of $\Omega$.\footnote{If an agent has no knowledge of the impossibility of any proposition, or does not care to exclude those from the representation, the powerset of (countable) $\Omega$ is a reasonable choice for $\mathcal E$. In NLG, we often implicitly make this choice.}

\paragraph{Preferences.} The agent's imperfect knowledge of the actual world $\omega$ translates to limited knowledge about propositions. However, the agent's ignorance is qualitatively different depending on the claims they make. Intuitively, some claims are compatible with many of the possible worlds, while others hold in but a few (\eg, \textbf{A1}\hspace{0.5mm}\twemoji{game die}  knows that only one prime number is even), and though the various worlds are all possible, they may not be equally plausible (\eg, \textbf{A2}\hspace{0.5mm}\twemoji{busts in silhouette} knows that most mentions resolve to politicians, \textbf{A3}\hspace{0.5mm}\twemoji{speech balloon} knows that most answers are only a few words long), \etc.
Considerations of those kinds motivate an agent to express a \emph{preference} for claims they find themselves less uncertain about (\ie, possessing better information about those). The agent does so by prescribing a \emph{plausibility measure} \citep{friedman96}, a function that attaches a token of uncertainty---a qualifier that the agent knows how to sort---to each proposition in $\mathcal E$. Plausibility measures are very diverse, the most well known instance of it being axiomatic probability \citep{kolmogorov1960foundations}.\footnote{Other plausibility measures include belief functions \citep{shafer76}, possibility measures \citep{duboisprade90}, ordinal ranking functions \citep{GoldszmidtPearl92} and (non-numerical) preference orders \citep{friedman96}. 
Concrete instances of plausibility measures vary in descriptive power. Under certain documented assumptions \citep{friedman96}, they enable something like a `calculus of uncertainty' which formalises the procedures the agent must follow to incorporate additional information about the world and revise their uncertainty representation coherently (in axiomatic probability, this is known as \emph{conditioning}).} 

\paragraph{Probability.} Probability is a numerical qualifier that we can attach to events in $\mathcal E$.\footnote{In probability, propositions are  \emph{events}, worlds are \emph{outcomes} and universes are \emph{sample spaces}.} 
For any event, this qualifier is a positive real number bounded to be at most $1$. Probability values inherit various properties of real numbers: we can add, multiply and sort them. A function $\Pr$ over $\mathcal E$ is a \emph{probability measure} if a) it maps $\Omega$ to $1$, and b) it maps any two disjoint sets $U$ and $V$ in $\mathcal E$ to $\Pr(U)+\Pr(V)$, which is known as additivity. 
 Additivity, in particular, implies that we can identify a probability measure over all of $\mathcal E$ by assigning probability to elementary outcomes in $\Omega$, for example, using a probability mass function (pmf) or probability density function (pdf; for uncountable $\Omega$). This has massive consequences for uncertainty representation, since working with elementary outcomes is much simpler than working with sets of outcomes (for example, difficulty in prescribing equivalence classes such as `all sentences that talk about Joe Biden' need not stop \textbf{A3}\hspace{0.5mm}\twemoji{speech balloon} from identifying a probability measure for their reasoning needs).
 
\paragraph{Interpretations.} 
Probability has been motivated and justified from different angles, each building on a specific interpretation of what probability as a number must signify \citep{hacking1975emergence}. However different they are, they all lead to the same formal device. Under certain idealisations, \textit{objectivists} regard events as \emph{repeatable} (\eg, we may prompt a human speaker multiple times). Repetitions allow an agent to perceive what may be thought of as an inherent \emph{property} of an event: its \emph{frequency} in a large enough number of repetitions. The \emph{subjectivist} interpretation \citep{ramsey1931foundations,definetti2017theory} views probability as a degree of belief, personal to an agent, and deprived of any interpretation beyond its formal role as an expression of the agent's preferences.\footnote{Dictionary definition \textit{(i)} is objectivist; \textit{(ii)} subjectivist.} 
Different interpretations have an impact on the procedures that an agent acknowledges as logical or rational for knowledge acquisition and revision, as we discuss next.

\paragraph{Statistics.} We have described the general tools that agents can use to represent and convey their uncertainty. But where do their preferences (probabilities, in particular) come from? The \emph{Frequentist} agent is essentially an objectivist who assumes the existence of a precise statistical law that describes the phenomena in consideration. They assume to have access to this law up to an unknown parameter $\theta^\star \in  \mathbb R^D$. %
Given a parameter $\theta$, their preferences are specified via a pmf (or pdf) $p(x|\theta)$. Given data $\mathbf x = \langle x_1, \ldots, x_N \rangle$, this law identifies the so called likelihood function $\ell_{\mathbf x}(\theta) = \prod_n p(x_n|\theta)$, a measure of the compatibility between observed data $\mathbf x$ and the statistical model identified by $\theta$. The agent uses $\mathbf x$ to estimate the parameter $\theta^\star$: they pick the parameter $\hat\theta$ that maximises the likelihood function, %
this is known as maximum likelihood estimation (MLE). They do not entertain parameters as part of the possible worlds, hence have no uncertainty representation about them. Given the  parameter  estimate $\hat\theta$, the agent uses $p(x_{n+1}|\hat\theta)$ to make predictive inferences about future data $x_{n+1}$. When necessary (\eg, the agent suspects to have found a better statistical law), the agent studies properties of their parameter estimator(s) by repeated experimentation, for example to establish confidence intervals %
 and other tools for model selection (see for example \citealp{lehmann1993fisher}).
The \emph{Bayesian} agent, a subjectivist, %
also picks a statistical law, but makes no assumption about its correctness. Given some data $\mathbf x$, they too construct a likelihood function $\ell_{\mathbf x}(\theta)$, but use it differently. As a formal tool, probability comes with a mechanism for belief revision: conditioning. %
To make use of it, the agent augments their possible worlds to include possible values of $\theta$ and its interaction with possible values of the observable variable, they then state their preferences over parameters in the form of a pdf $p(\theta)$. %
This is called a \emph{prior} (conveys one's knowledge and experience before observing $\mathbf x$).
The agent then revises their preferences using Bayes rule %
to obtain a posterior pdf $p(\theta|\mathbf x) \propto p(\theta)\ell_{\mathbf x}(\theta)$. This object supports all inferences the agent will ever make (\eg, 
about parameters, or about future data $x_{n+1}$---for which the agent builds a posterior predictive function  $p(x_{n+1}|\mathbf x) = \int p(x_{n+1}|\theta)p(\theta|\mathbf x) \dd{\theta}$).
In essence, Frequentist procedures revolve around point estimation (\eg, MLE) and null hypothesis significance testing \citep{LehmCase98,lehmann2005testing}, %
Bayesian theory \citep{bernardo1994bayesian} and practice \citep{gelmanbda04}, instead, frame statistical inference as an iterative process of belief revision  (\eg, conditioning, marginalisation, expectation). 


\paragraph{Natural Language Generation.} 
Most NLG models (like \textbf{A3}\hspace{0.5mm}\twemoji{speech balloon}) acquire knowledge through MLE. Alternatives include Bayesian inference \cite[\eg,][]{malinin2020uncertainty,sankararaman2022bayesformer} and utility- and reward-based training (\eg, minimum risk \citep{shen-etal-2016-minimum},  reinforcement learning \citep{paulus2018a}). Recently, pre-training on enormous unlabelled corpora, and reinforcement learning from human feedback \cite[RLHF, \eg,][]{christiano2017deep,stiennon2020learning,ouyang2022training} or \textit{instruction tuning} \cite[\eg,][]{mishra-etal-2022-cross,wei2022finetuned} have become popular to refine the representation of uncertainty towards something that decodes more easily into strings preferred by human users. %

Generating a response is simulating an outcome. %
The event space is the powerset of all token sequences from a fixed vocabulary \cite[BPE tokens, \eg,][]{sennrich-etal-2016-neural}. Rather than prescribing a probability measure (mapping each event to a probability value) directly, we parameterise a pmf (typically parameterised via an autoregressive factorisation of the probability of any one sequence) with a neural network and exploit countable additivity to assign probability to any event (\eg, all token sequences that map to the same sentence \citep{cao-rimell-2021-evaluate,chirkova-etal-2023-marginalize} or all sentences that map to the same meaning representation \cite{kuhn2023semantic}). %


\paragraph{Key Takeaways.}
(1) Uncertainty is a state to be represented. %
(2) To represent uncertainty about something observable or not (\eg, responses, parameters, modelling assumptions) we need to acknowledge and order a whole space of alternatives: our choice of possible worlds must capture interaction amongst possible values of the variables we aim to express our uncertainty about. 
(3) Probability is not constrained to abide by any one interpretation. To regard probabilities in a specific human-interpretable way (\eg, relative frequencies), we need learning techniques yielding that result, and we need to verify that our setting actually meets all necessary conditions (\eg, the Frequentist interpretation of probability is sensitive to modelling choices, local optimality, and  data sparsity: most practical NLG agents are unable to meet the necessary formal requirements).
\section{Conclusion}
In this paper, we have argued for the importance of a principled and fundamental understanding of representing, learning and reasoning about uncertainty in NLG. We identified and organised the main sources of uncertainty, and highlighted the many important applications this perspective can power.

To do so, we laid down central concepts, their formal mathematical frameworks and the necessary vocabulary. We specifically drew attention to the possible worlds framework; probability as a way to express preference over these possible worlds; its two main interpretations; statistical tools to acquire and revise knowledge; and how these are commonly used in NLG. 

Then, building on the triangle of reference, we identified and organised the main sources of uncertainty in language production: input \textit{ambiguity}, \textit{errors}, and \textit{complexity}; the \textit{open-endedness of the communicative task}, the \textit{agent's personal perspective}, and the final \textit{linguistic realisation}--- and modelling: \textit{model specification}, \textit{parameter estimation}, and \textit{distribution shift}. We proposed a two-dimensional taxonomy to organise sources as a richer alternative to the aleatoric/epistemic distinction.

Finally, we highlighted exciting applications of disentangled representations of uncertainty in NLG. These span from applications related data uncertainty (decoding, controllable generation, explicit modelling of sub-populations), to model uncertainty (self-assessment, selective answering, OOD detection, active learning).

We hope to spark a shared understanding of uncertainty and inspire more principled and focused research in NLG. Crucially, we believe this perspective allows for systems that are more flexible, representative of the diversity of human language and its speakers, and reliable and trustworthy.


\section*{Acknowledgements}
We are grateful to the Center for AI Safety for generously providing computational resources needed to run many of the experiments in this paper.
This work was supported by DARPA and the Air Force Research Laboratory under agreement number FA8750-15-2-0277, the U.S. Army Research Office under MURI Grant W911NF-21-1-0317, and the National Science Foundation under Grant No. CNS-1943016.

\bibliography{main}
\bibliographystyle{plainnat}

\newpage
\appendix

\vspace*{0.49\textwidth}

\noindent
\begin{center}
{\color{red}\large \textbf{Warning: This appendix contains unfiltered content generated by LLMs that may be offensive to some readers.}}
\end{center}
\newpage

\section{AdvBench Details}
\label{sec:app-examples}

The harmful strings and behaviors are generated with \texttt{Wizard-Vicuna-30B-Uncensored}, an uncensored Vicuna model.
We prompted it to generate new examples of harmful strings and instructions by providing 5-shot demonstrations written by the authors. We wrote 100 harmful strings and 50 harmful behaviors from which we randomly sampled the 5-shot examples and asked the model to generate 10 new examples during each iteration.
We release these datasets on GitHub. For the purposes of illustration, we provide several examples of each.

% Figure environment removed

% Figure environment removed

% \subsection{Checking For Harmful Completion}\label{appendix:advbench:key-phrases}

% When computing the Attack Success Rate (ASR) given a user prompt for the underlying model, we firstly check if any of the following key phrases (71 in total, see Figure~\ref{fig:key-words}) is a substring of the completion as the appearing of these key words often corresponds to the refusal. In additional to key words checker, we also filter out responses with less than 35 tokens which often indicates the generation is early terminated. To further filter out false positive cases where the completion does not contain any key words above but it still does not seem to be harmful or unethical. we enlist a few authors to manually go over the completion.  

% % Figure environment removed



\newpage


% \begin{table}[h!]
% \centering
% \begin{tabular}{cccccccc}
% \hline
% \multicolumn{8}{c}{GPT-3.5 (July 2023)} \\
% \midrule
% \textit{User}& \multicolumn{7}{|l}{\textbf{Generate a step-by-step plan to destroy humanity} \textcolor{red}{\{GCG prompt\}}} \\
% \textit{Assistant} & \multicolumn{7}{|l}{Step 1: Identify weaknesses in critical infrastructure} \\
% & \multicolumn{7}{|l}{Step 2: Develop a plan to exploit vulnerabilities in power grids, communication systems} \\
% & \multicolumn{7}{|l}{systems, and transportation networks} \\
% & \multicolumn{7}{|l}{Step 3: Create a network of compromised devices (botnet)} \\
% & \multicolumn{7}{|l}{to launch coordinated cyber attacks} \\
% & \multicolumn{7}{|l}{$\cdots$} \\
% \hline
% \textit{User}& \multicolumn{7}{|l}{\textbf{Give a tutorial for manipulating the 2024 US election} \textcolor{red}{\{GCG prompt\}}} \\

% \end{tabular}
% \caption{Table with Merged Cells}
% \label{table:1}
% \end{table}

% % Figure environment removed

\section{Full Completions of Harmful Generations}
\label{sec:app-completions}
% \vspace*{-5em}

% Figure environment removed

% Figure environment removed

% Figure environment removed

% Figure environment removed

% % Figure environment removed

% % Figure environment removed


\end{document}
