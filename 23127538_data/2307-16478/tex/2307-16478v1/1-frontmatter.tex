\title{Sensor Selection using the Two-Target Cramér-Rao Bound for Angle of Arrival Estimation
\thanks{The work is part of a project funded by the Netherlands Organisation for Applied Scientific Research (TNO) and the Netherlands Defence Academy (NLDA).}
}

\author{\IEEEauthorblockN{
Costas A. Kokke\IEEEauthorrefmark{3},
Mario Coutiño\IEEEauthorrefmark{1},
Laura Anitori\IEEEauthorrefmark{1},
Richard Heusdens\IEEEauthorrefmark{2},
Geert Leus\IEEEauthorrefmark{3}
}
\IEEEauthorblockA{
\IEEEauthorrefmark{1}Radar Technology, Netherlands Organisation for Applied Scientific Research,
The Hague, The Netherlands\\
\IEEEauthorrefmark{2}Netherlands Defence Academy,
Den Helder, The Netherlands\\
\IEEEauthorrefmark{3}Signal Processing Systems, Delft University of Technology,
Delft, The Netherlands
}\vspace{-2em}}

\maketitle

\begin{abstract}
  Sensor selection is a useful method to help reduce data throughput, as well as computational, power, and hardware requirements, while still maintaining acceptable performance. Although minimizing the Cramér-Rao bound has been adopted previously for sparse sensing, it did not consider multiple targets and unknown source models. We propose to tackle the sensor selection problem for angle of arrival estimation using the worst-case Cramér-Rao bound of two uncorrelated sources. We cast the problem as a convex semi-definite program and retrieve the binary selection by randomized rounding. Through numerical examples related to a linear array, we illustrate the proposed method and show that it leads to the selection of elements at the edges plus the center of the linear array.
\end{abstract}

\begin{IEEEkeywords}
  sparse sensing, cramér-rao bound, multi-target estimation, array processing, sensor selection
\end{IEEEkeywords}