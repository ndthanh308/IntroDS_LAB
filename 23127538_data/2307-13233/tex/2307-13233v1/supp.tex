\documentclass[onecolumn, amsmath, amssymb, aps]{revtex4-2}


\usepackage{graphicx}% Include figure files
\usepackage{dcolumn}% Align table columns on decimal point
\usepackage{bm}% bold math
\usepackage{braket}
\usepackage{changes}

\usepackage{comment}

\usepackage{hyperref}       % hyperlinks
\hypersetup{
    colorlinks=true,
    linkcolor=blue,
    filecolor=magenta,      
    urlcolor=blue,
}
\usepackage{color,soul}





\setlength {\marginparwidth }{2cm} 
\begin{document}
%\setcounter{table}{0}   %´ÓÁ㿪ʼ±àºÅ
%\setcounter{figure}{0}   %´ÓÁ㿪ʼ±àºÅ
%\setcounter{equation}{0}   %´ÓÁ㿪ʼ±àºÅ
\renewcommand{\thefigure}{S\arabic{figure}}
\renewcommand{\theequation}{S\arabic{equation}}
\renewcommand{\thetable}{S\arabic{table}}


\preprint{APS}



\title{Supplementary Materials: Topological dissipative Kerr soliton combs in a valley photonic crystal resonator}

\author{Zhen Jiang$^{1,2}$} \author{Lefeng Zhou$^{1}$}%
\author{Wei Li$^{3}$}%
\author{Yudong Li$^{3}$}%
\author{Liangsen Feng$^{3}$}%
\author{Tengfei Wu$^{3}$}%
\author{Chun Jiang$^{1}$}  \email{cjiang@sjtu.edu.cn}	
\author{Guangqiang He$^{1,2}$}  \email{gqhe@sjtu.edu.cn}	
\affiliation{%
 $^1$State Key Laboratory of Advanced Optical Communication Systems and Networks, Department of Electronic Engineering, Shanghai Jiao Tong University, Shanghai 200240, China\\
 $^2$SJTU Pinghu Institute of Intelligent Optoelectronics, Department of Electronic Engineering, Shanghai Jiao Tong University, Shanghai 200240, China \\
 $^3$Science and Technology on Metrology and Calibration Laboratory, Changcheng Institute of Metrology $\rm \&$ Measurement, Aviation Industry Corporation of China, Beijing 100095, China
}%
	
\maketitle



\section{Numerical simulation}
We use the commercial software COMSOL Multiphysics to simulate the FWM process in the  $\rm Si_{3}N_{4}$ topological resonator (Fig. 3 in the main text). Here we consider the field profiles of transverse electric (TE) polarization modes. We choose one resonator mode (indexed by the number $\mu=0$) at the frequency of $v_p=370.6$7 THz as the pump mode, where the neighboring two resonator modes ($\mu=1,-1$) are selected as the signal ($v_s=371.13 $THz) and idler ($v_i=370.21$ THz) frequencies, respectively. The third-order nonlinearity of  $\rm Si_{3}N_{4}$ $\chi^{(3)}=2.5\times10^{-19}\mathrm{m}^2/\mathrm{V}^2$ is used to introduce the nonlinear elements. The FWM process conducted by third-order nonlinear polarizations of $\rm Si_{3}N_{4}$ can be described by
\begin{equation}
\begin{aligned}
\mathbf{P}_{p}\left(\omega_{s}+\omega_{i}-\omega_{p}\right)=6\varepsilon_{0}\chi^{(3)}E_{s}E_{i}E_{p}^{*}, \\ 
\mathbf{P}_{s}\left(\omega_{p}+\omega_{p}-\omega_{i}\right)=3\varepsilon_{0}\chi^{(3)}E_{p}E_{p}E_{i}^{*}, \\ 
\mathbf{P}_{i}\left(\omega_{p}+\omega_{p}-\omega_{s}\right)=3\varepsilon_{0}\chi^{(3)}E_{p}E_{p}E_{s}^{*}, \\
\label{eq:S1}
\end{aligned}
\end{equation}	
where $\mathbf{P}_{p,s,i}$ and $E_{p,s,i}$  are the polarizations and electric field of the pump, signal, and idler. In the numerical model, we use sources with $E=E_x+iE_y$ set at the input port of the bus waveguide to emulate right circularly polarized light. And the input power of the pump and signal are set as $|E_p|=10^4$ $\rm {V/m}$ and.
$|E_s|=10^3$ $\rm {V/m}$  respectively, with no input for the idler. As shown in Fig. 3 in the main text, the excitation of idler modes indicates that nonlinear coupling between the pump and signal model at the idler frequency is driven by the nonlinear polarization of electromagnetic models at the pump and signal frequencies. 

\section{Valley topological photonics (VPCs)}
For unperturbed unit cells with $C_6$ lattice symmetry, there exist degenerate Dirac points at the $\rm K$ and $\rm K'$ valleys. The effective Hamiltonian in the vicinity of $\rm K$ ($\rm K'$) point is described as~\cite{s1,s2,s3}

\begin{equation}
H_{K/K'}^{(0)}=v_{D}\bigl(\sigma_{x}\delta k_{x}+\sigma_{y}\delta k_{y}\bigr),
\label{eq:S2}
\end{equation}	
where $v_{D}$ is the group velocity, $\sigma_{x}$ and $\sigma_{y}$ are the Pauli matrices, $\delta\vec{k}=\vec{k}-\vec{k}_{K/K'}$ denotes the deviation of the wavevector. With the distortion of the unit cell ($d_1 \neq d_2$), the Hamiltonian can be rewritten as 
\begin{equation}
H_{K/K'}=v_D\bigl(\sigma_x\delta k_x+\sigma_y\delta k_y\bigr)+v_Dm\sigma_z,
\label{eq:S3}
\end{equation}	
where m denotes the strength of the symmetry-breaking perturbation. The simulated magnetic field profiles $H_z$ at the $\rm K$ and $\rm K'$ valleys for the first (367.01 THz) and second (409.77 THz) bands of the $\rm VPC_2$ are shown in Fig.~\ref{fig:S1}(a). It reveals that the modes at the $\rm K$ and $\rm K'$ valleys show opposite circular polarizations, to be specific, left-handed circular polarization (LCP) and right-handed circular polarization (RCP) respectively. The normalized simulated Berry curvatures are plotted in Fig.~\ref{fig:S1}(b). To characterize the topological properties of VPCs, we calculate the valley Chern numbers of VPCs by~\cite{s1,s4}
\begin{equation}
C_{K/K^{\prime}}=\frac{1}{2\pi}\int_{HBZ}\Omega_{K/K^{\prime}}(\delta\vec{k})dS=\pm1/2,
\label{eq:S4}
\end{equation}	
where $\Omega_{K/K'}$ is the Berry curvature, and this integration region contains half of the Brillouin zone. Therefore, the difference of valley Chern number of the system is calculated as $|C_{K/K'}|=1$, leading to the topological nature of VPCs.
%\vspace{-0.2cm}
% Figure environment removed
%\vspace{-0.6cm}
\section{Topological edge states}
Topologically protected edge states can be observed at the interface between $\rm VPC_1$ and $\rm VPC_2$, these edge states are also called valley kink states~\cite{s5}. Fig.~\ref{fig:S2}(a) shows the calculated dispersion curves of the valley kink state with $k_x>0$, where the red curve denotes the valley kink state. The field distribution for the valley kink states around the interface with $k_x=0.7 (\pi/a)$ is plotted in Fig.~\ref{fig:S2}(b). To verify the robustness of such edge states, we design a “Z” shaped topological waveguide with several random lacking of holes around the interface. The simulated field profile of valley kink states at the frequency of 370 THz along the interface is illustrated in Fig.~\ref{fig:S2}(c). It shows that the light can smoothly pass through the defects and sharp corners, leading to immunity to these defects. We further discuss the linear transmittances of the valley edge states at the “Z” shaped interfaces with or without defects. Two detector dipoles are set to monitor the power of the input and output ports. As shown in Fig.~\ref{fig:S2}(d), an abrupt decrease in the transmittance has not been observed. This behavior also gives clear evidence of robustness against defects. To discuss the localization of valley kink states, we plot the normalized electric field distributions $\rm |E|$ along y axis with a series of frequencies. As depicted in Fig.~\ref{fig:S2}(e), the electric field is concentrated at the interface between two VPCs, which implies the good light confinement of valley kink states. 

% Figure environment removed

\section{Dispersion engineering}
To describe the light propagation of topological edge states, the dispersion characteristics of these states are studied. The propagation constant $\beta$ can also be expanded around frequency $\omega_0$ in a Taylor series:
\begin{equation}
\beta=\beta_0+\beta_1(\omega-\omega_0)+\beta_2\frac{(\omega-\omega_0)^2}{2}+\cdots,
\label{eq:S5}
\end{equation}	
where the expansion term is $\beta_{i}=d^{i}k/d\omega^{i}$ at $\omega=\omega_0$. The first-order term $\beta_1$ is linked to the group velocity of the topological edge state: $\beta_{1}=1/\nu_{g}$, where the group velocity is given by $v_{g}=d\omega/dk$. The second term, $\beta_2$ so-called group velocity dispersion (GVD) can also be used to identify the dispersion characteristics of topological edge states, with normal dispersion ($\beta_2>0$) and anomalous dispersion ($\beta_2<0$). 
% Figure environment removed
Here we calculate the GVD curve of the topological edge state for the wavevector $k_x<0$, using different structure parameters: $d_1=0.3a$, $d_2=0.7a$ and $d_1=0.25a$, $d_2=0.75a$, respectively (Fig.~\ref{fig:S3}). It exhibits anomalous dispersion characteristics around the pump frequency of 370 THz for the structure parameters with $d_1=0.25a$, $d_2=0.75a$. It is worth mentioning that the $\beta_i$ is highly related to the integrated dispersion with $\beta_{1}=2\pi/LD_{1}$ and $\beta_{2}=-2\pi D_{2}/LD_{1}^{3}$, respectively. We can calculate the dispersion $D_2=9.60$ GHz, which is approximately equal to the dispersion of the topological resonator extracted from the simulated transmission. Therefore, the topological resonator can be approximatively considered as a single-ring resonator in the dynamic evolution simulation of Kerr soliton combs.

\section{Coupling between waveguides and topological resonators}
Here we use a straight topological bus waveguide to couple the pump mode into the topological resonator, and then, guide the generated combs from the cavity to the bus waveguide. To assess the couplings between a straight waveguide and a triangular resonator, we design several configurations with a series of gap widths. As shown in Fig.~\ref{fig:S4}(a), the values of the gap width are set from 1 cell to 5 cells. Note that the total Q-factor is given by $1⁄Q=1⁄Q_{in} +1⁄Q_{ex}$, where $Q_{in}$ and $Q_{ex}$ are the intrinsic quality factor and external quality factor respectively. The $Q_{in}$ and $Q_{ex}$ can be calculated as $Q_{in}=\omega/\alpha v_{g}=\omega/\kappa_{in}\approx\omega/(1-e^{-\alpha L})FSR$ and $Q_{ex}=\omega/\kappa_{ex}\approx\omega/(1-|t_{1}|^{2})FSR$, where $\alpha$, $v_g$, and $L$ represent the roundtrip loss, the group velocity, and the roundtrip length of the topological cavity, respectively. The parameters $\kappa_{in}$, $\kappa_{ex}$, and $\kappa$ denote the intrinsic loss, the external loss, and the total energy loss rate of the resonator. 

The Lorentzian fittings of the simulated resonant dips at the pump frequency $v_p$ around 370.6 THz are plotted in Fig.~\ref{fig:S4}(b)-(f). The result shows that the topological resonator with the gap width of 3 cells (Fig.~\ref{fig:S4}(d)) satisfies $\eta=1⁄2$, leading to the critical coupling. For the critical coupling, the external loss is equal to the intrinsic loss with $\kappa_{in}=\kappa_{ex}=1.35\times10^7$ rad/s. Therefore, the intrinsic quality factor and external quality factor are $Q_{in}=Q_{ex}=1.772\times10^5$, leading to a value of a total Q-factor with $Q=8.86\times10^4$. We can also find that the topological resonators are over-coupling ($\eta>1⁄2$) when the gap widths are small than 3 cells. Correspondingly, the topological resonators are under coupling ($\eta<1⁄2$) when the gap widths are large than 3 cells.
% Figure environment removed
\section{Topological resonator modes}
Considering the light propagation behavior of valley kink states in the topological resonator, we simulate the field profiles and energy flow of resonator modes at the pump frequency. We pump the bus waveguide in the opposite direction. As shown in Fig.~\ref{fig:S5}(a), the pump mode excited at the left of the waveguide is coupled into the topological resonator. Its corresponding Poynting vector distributions plotted in Fig.~\ref{fig:S5}(c) reveal that the energy travels clockwise inside the triangular cavity. And then, the energy is coupled out from the cavity to the right of the waveguide. Due to time-reversal symmetry, the pump mode excited in the opposite direction excites an anticlockwise rotation mode inside the cavity (Fig.~\ref{fig:S5}(b), (d)).

% Figure environment removed
\section{Numerical simulation of DKS combs in topological resonators}
Here we theoretically analyze the nonlinear evolution of the pump field in the topological resonator. The Lugiato–Lefever equation (LLE)~\cite{s6} can be concluded from the nonlinear Schrödinger equation (NSE) with boundary conditions of the resonators, where the NSE has the form:
\begin{equation}
\frac{\partial}{\partial z}A^{(m)}+\frac{\alpha}{2}A^{(m)}+i\frac{\beta_{2}}{2}\frac{\partial^{2}}{\partial r^{2}}A^{(m)}=i\gamma\big|A^{(m)}\big|^{2}A^{(m)},
\label{eq:S6}
\end{equation}	
where m denotes the $m$-th roundtrips travel of the light field, $A=A(z,t)$ is the field envelope with the propagation distance $z$ in resonators, $T$ is the fast time variable describing the waveform. $\alpha$, $\beta_{2}$, and $\gamma$ are the roundtrip loss, the dispersion term, and the nonlinear coefficient, respectively. The boundary condition can be written as
\begin{equation}
A^{(m+1)}(0,T)=\sqrt{\Theta}A_i+\sqrt{1-\Theta}\exp(-i\delta_0)A^{(m)}(L,T),
\label{eq:S7}
\end{equation}	
in which $A_i$ is the light field of the pump, $\Theta$ denotes the power coupling coefficient, $\delta_0$ is the detuning of the resonance frequency, and $L$ is length of the topological resonator. $\frac{\partial}{\partial z}A^{(m)}$ takes an approximate form of 
\begin{equation}
\frac{\partial}{\partial z}A^{(m)}(z,T)\bigg|_{z=0}=\frac{A^{(m)}(L,T)-A^{(m)}(0,T)}{L}.
\label{eq:S8}
\end{equation}	

When the light field does not pass through the coupling region between the topological resonator and the straight waveguide, the nonlinear evolution is governed by NSE. Substitute the Eq.~\ref{eq:S8} into Eq.~\ref{eq:S6}, we can get
\begin{equation}
A^{(m)}(L,T)=A^{(m)}(0,T)+L\left(-\frac{\alpha}{2}A^{(m)}(0,T)-\frac{i\beta_{2}}{2}\frac{\partial^{2}}{\partial T^{2}}A^{(m)}(0,T)+i\gamma\Big|A^{(m)}(0,T)\Big|^{2}A^{(m)}(0,T)\Big).\right. 
\label{eq:S9}
\end{equation}	
The NSE reveals how the light field evolves after the propagation length $L$.
Next, we take the coupling region into consideration. In this step, we take two assumptions: first, the power coupling coefficient is small, that is, $\Theta\ll 1$; second, the detuning compared with the FSR is very small, namely $\delta_{0}\ll2\pi$. Therefore, the Eq.~\ref{eq:S7} can be reduced to 
\begin{equation}
A^{(m+1)}(0,T)=\sqrt{\Theta}A_i+\left(1-\frac{\Theta}{2}-i\delta_0\right)A^{(m)}(L,T).
\label{eq:S10}
\end{equation}	

Since the term m is a discrete value, it leads to complexity for derivations. A parameter $t_R$, the so-called slow time variable, to replace the term m. When m increases to $m+1$, it is the equivalent of increasing time $t_R$ (roundtrip time) for the current time $\tau$. We can obtain the relation
\begin{equation}
\frac{\partial}{\partial\tau}A(\tau,T)=\frac{A^{(m+1)}(0,T)-A^{(m)}(0,T)}{t_{R}}.
\label{eq:S11}
\end{equation}	

Substitute the Eq.~\ref{eq:S9} and Eq.~\ref{eq:S10} into Eq.~\ref{eq:S11}, the term can be rewritten as 
\begin{equation}
t_R\frac{\partial}{\partial\tau}A=-\left(\frac{\alpha L+\Theta}{2}+i\delta_0\right)A-iL\frac{\beta_2}{2}\frac{\partial^2}{\partial T^2}A+iL\gamma|A|^2A+\sqrt{\Theta}A_i.
\label{eq:S12}
\end{equation}	
Eq.~\ref{eq:S12} is the first form of LLE. To obtain the second form of the equation, new parameters are introduced to endow the physical meaning of the model. First, the roundtrip time $t_R$ is related to the FSR by $t_R=1⁄FSR$. The normalized losses are defined as
\begin{equation}
\kappa_{in}=\alpha L\cdot FSR,\quad\kappa_{ex}=\Theta\cdot FSR,\quad\kappa=\kappa_{in}+\kappa_{ex},
\label{eq:S13}
\end{equation}	
where $\kappa_{in}$, $\kappa_{ex}$, and $\kappa$ denote the intrinsic loss, the external loss, and the total energy loss rate of the resonator. The parameter $\kappa$ is also related to the linewidth of resonator modes. Then, we can define the normalized detuning as
\begin{equation}
\delta_{0}=\beta_{1}L\big(\omega_{0}-\omega_{p}\big)=\frac{1}{FSR}\delta\omega,
\label{eq:S14}
\end{equation}	
where $\delta\omega$ is the detuning $\omega_{0}-\omega_{p}$. The second-order dispersion is given by
\begin{equation}
\delta_{0}=\beta_{1}L\big(\omega_{0}-\omega_{p}\big)=\frac{1}{FSR}\delta\omega,
\label{eq:S15}
\end{equation}	
Eq.~\ref{eq:S12} can be rewritten as  
\begin{equation}
\frac{\partial}{\partial\tau}A=-\left(\frac{\kappa}{2}+i\delta\omega\right)A+i\pi\cdot FSR\cdot D_{2}\frac{\partial^{2}}{\partial T^{2}}A+iL\cdot FSR\cdot\gamma|A|^{2}A+\sqrt{\frac{\kappa\eta P_{in}}{\hbar\omega}}.
\label{eq:S16}
\end{equation}	

The slow time variable $\tau$ in this equation is applied to replace the spatial coordinates $z$. In our topological resonator, the length of the topological resonator is $L=3l$, where $l=180a$ is the side length of the triangular configuration. The FSR and dispersion $D_2$ are extracted from the simulated transmission (Fig. 2(c) in the main text), with calculated values of $FSR=450$ GHz and $D_2=9.53$ GHz. With the Lorentzian fittings of the simulated resonant dips at the pump frequency $\omega_0$, the external loss and the total energy loss rate of the resonator can be calculated as $\kappa_{ex}=1.35\times10^7$ rad/s and $\kappa=2.7\times10^7$ rad/s. The coupling efficiency is given by $\eta=\kappa_{ex}/(\kappa_{ex}+\kappa_{in})=1/2$. The nonlinear coefficient of the topological resonator is given by $\gamma=\omega_{0}n_{2}/cA_{eff}$, where $n_2$ is the nonlinear index of $\rm Si_{3}N_{4}$.

To confirm the spatial distributions of generated solitons, the relative locations of solitons in the topological resonator are related to the propagating time. Since the length of the resonator $L$ corresponds to the roundtrip time T by $L=v_gT$. The corresponding spatiotemporal evolution of the DKS excitation process (Fig. 4(b) in the main text) is a Fourier transform form of frequency-domain combs. In this transform, we set the coupling corner of the triangular resonator as an initiative point, the range distributions of solitons can be calculated from the range of time solutions (0, $L$) directly. Therefore, the relative locations of solitons can be observed in corresponding spatiotemporal evolution. To get deep insight into spatial distributions of the field inside the resonator, we plot the real-time pulse distributions corresponding to four evolutive states inside the topological resonator (Fig. 5 in the main text). And we also show the spatial intensity distributions with the detuning of $\delta\omega=0$, $\delta\omega=2.26$, and $\delta\omega=9.37$, respectively (Fig.~\ref{fig:S6}).
% Figure environment removed


%\bibliography{ref_supp}
%apsrev4-2.bst 2019-01-14 (MD) hand-edited version of apsrev4-1.bst
%Control: key (0)
%Control: author (8) initials jnrlst
%Control: editor formatted (1) identically to author
%Control: production of article title (0) allowed
%Control: page (0) single
%Control: year (1) truncated
%Control: production of eprint (0) enabled
\begin{thebibliography}{6}%
\makeatletter
\providecommand \@ifxundefined [1]{%
 \@ifx{#1\undefined}
}%
\providecommand \@ifnum [1]{%
 \ifnum #1\expandafter \@firstoftwo
 \else \expandafter \@secondoftwo
 \fi
}%
\providecommand \@ifx [1]{%
 \ifx #1\expandafter \@firstoftwo
 \else \expandafter \@secondoftwo
 \fi
}%
\providecommand \natexlab [1]{#1}%
\providecommand \enquote  [1]{``#1''}%
\providecommand \bibnamefont  [1]{#1}%
\providecommand \bibfnamefont [1]{#1}%
\providecommand \citenamefont [1]{#1}%
\providecommand \href@noop [0]{\@secondoftwo}%
\providecommand \href [0]{\begingroup \@sanitize@url \@href}%
\providecommand \@href[1]{\@@startlink{#1}\@@href}%
\providecommand \@@href[1]{\endgroup#1\@@endlink}%
\providecommand \@sanitize@url [0]{\catcode `\\12\catcode `\$12\catcode
  `\&12\catcode `\#12\catcode `\^12\catcode `\_12\catcode `\%12\relax}%
\providecommand \@@startlink[1]{}%
\providecommand \@@endlink[0]{}%
\providecommand \url  [0]{\begingroup\@sanitize@url \@url }%
\providecommand \@url [1]{\endgroup\@href {#1}{\urlprefix }}%
\providecommand \urlprefix  [0]{URL }%
\providecommand \Eprint [0]{\href }%
\providecommand \doibase [0]{https://doi.org/}%
\providecommand \selectlanguage [0]{\@gobble}%
\providecommand \bibinfo  [0]{\@secondoftwo}%
\providecommand \bibfield  [0]{\@secondoftwo}%
\providecommand \translation [1]{[#1]}%
\providecommand \BibitemOpen [0]{}%
\providecommand \bibitemStop [0]{}%
\providecommand \bibitemNoStop [0]{.\EOS\space}%
\providecommand \EOS [0]{\spacefactor3000\relax}%
\providecommand \BibitemShut  [1]{\csname bibitem#1\endcsname}%
\let\auto@bib@innerbib\@empty
%</preamble>
\bibitem [{\citenamefont {Yang}\ \emph {et~al.}(2020)\citenamefont {Yang},
  \citenamefont {Yamagami}, \citenamefont {Yu}, \citenamefont {Pitchappa},
  \citenamefont {Webber}, \citenamefont {Zhang}, \citenamefont {Fujita},
  \citenamefont {Nagatsuma},\ and\ \citenamefont {Singh}}]{s1}%
  \BibitemOpen
  \bibfield  {author} {\bibinfo {author} {\bibfnamefont {Y.}~\bibnamefont
  {Yang}}, \bibinfo {author} {\bibfnamefont {Y.}~\bibnamefont {Yamagami}},
  \bibinfo {author} {\bibfnamefont {X.}~\bibnamefont {Yu}}, \bibinfo {author}
  {\bibfnamefont {P.}~\bibnamefont {Pitchappa}}, \bibinfo {author}
  {\bibfnamefont {J.}~\bibnamefont {Webber}}, \bibinfo {author} {\bibfnamefont
  {B.}~\bibnamefont {Zhang}}, \bibinfo {author} {\bibfnamefont
  {M.}~\bibnamefont {Fujita}}, \bibinfo {author} {\bibfnamefont
  {T.}~\bibnamefont {Nagatsuma}},\ and\ \bibinfo {author} {\bibfnamefont
  {R.}~\bibnamefont {Singh}},\ }\bibfield  {title} {\bibinfo {title} {Terahertz
  topological photonics for on-chip communication},\ }\href@noop {} {\bibfield
  {journal} {\bibinfo  {journal} {Nature Photonics}\ }\textbf {\bibinfo
  {volume} {14}},\ \bibinfo {pages} {446} (\bibinfo {year} {2020})}\BibitemShut
  {NoStop}%
\bibitem [{\citenamefont {Lu}\ \emph {et~al.}(2017)\citenamefont {Lu},
  \citenamefont {Qiu}, \citenamefont {Ye}, \citenamefont {Fan}, \citenamefont
  {Ke}, \citenamefont {Zhang},\ and\ \citenamefont {Liu}}]{s2}%
  \BibitemOpen
  \bibfield  {author} {\bibinfo {author} {\bibfnamefont {J.}~\bibnamefont
  {Lu}}, \bibinfo {author} {\bibfnamefont {C.}~\bibnamefont {Qiu}}, \bibinfo
  {author} {\bibfnamefont {L.}~\bibnamefont {Ye}}, \bibinfo {author}
  {\bibfnamefont {X.}~\bibnamefont {Fan}}, \bibinfo {author} {\bibfnamefont
  {M.}~\bibnamefont {Ke}}, \bibinfo {author} {\bibfnamefont {F.}~\bibnamefont
  {Zhang}},\ and\ \bibinfo {author} {\bibfnamefont {Z.}~\bibnamefont {Liu}},\
  }\bibfield  {title} {\bibinfo {title} {Observation of topological valley
  transport of sound in sonic crystals},\ }\href@noop {} {\bibfield  {journal}
  {\bibinfo  {journal} {Nature Physics}\ }\textbf {\bibinfo {volume} {13}},\
  \bibinfo {pages} {369} (\bibinfo {year} {2017})}\BibitemShut {NoStop}%
\bibitem [{\citenamefont {Noh}\ \emph {et~al.}(2018)\citenamefont {Noh},
  \citenamefont {Huang}, \citenamefont {Chen},\ and\ \citenamefont
  {Rechtsman}}]{s3}%
  \BibitemOpen
  \bibfield  {author} {\bibinfo {author} {\bibfnamefont {J.}~\bibnamefont
  {Noh}}, \bibinfo {author} {\bibfnamefont {S.}~\bibnamefont {Huang}}, \bibinfo
  {author} {\bibfnamefont {K.~P.}\ \bibnamefont {Chen}},\ and\ \bibinfo
  {author} {\bibfnamefont {M.~C.}\ \bibnamefont {Rechtsman}},\ }\bibfield
  {title} {\bibinfo {title} {Observation of photonic topological valley hall
  edge states},\ }\href@noop {} {\bibfield  {journal} {\bibinfo  {journal}
  {Physical Review Letters}\ }\textbf {\bibinfo {volume} {120}},\ \bibinfo
  {pages} {063902} (\bibinfo {year} {2018})}\BibitemShut {NoStop}%
\bibitem [{\citenamefont {Shalaev}\ \emph {et~al.}(2019)\citenamefont
  {Shalaev}, \citenamefont {Walasik}, \citenamefont {Tsukernik}, \citenamefont
  {Xu},\ and\ \citenamefont {Litchinitser}}]{s4}%
  \BibitemOpen
  \bibfield  {author} {\bibinfo {author} {\bibfnamefont {M.~I.}\ \bibnamefont
  {Shalaev}}, \bibinfo {author} {\bibfnamefont {W.}~\bibnamefont {Walasik}},
  \bibinfo {author} {\bibfnamefont {A.}~\bibnamefont {Tsukernik}}, \bibinfo
  {author} {\bibfnamefont {Y.}~\bibnamefont {Xu}},\ and\ \bibinfo {author}
  {\bibfnamefont {N.~M.}\ \bibnamefont {Litchinitser}},\ }\bibfield  {title}
  {\bibinfo {title} {Robust topologically protected transport in photonic
  crystals at telecommunication wavelengths},\ }\href@noop {} {\bibfield
  {journal} {\bibinfo  {journal} {Nature Nanotechnology}\ }\textbf {\bibinfo
  {volume} {14}},\ \bibinfo {pages} {31} (\bibinfo {year} {2019})}\BibitemShut
  {NoStop}%
\bibitem [{\citenamefont {Gao}\ \emph {et~al.}(2018)\citenamefont {Gao},
  \citenamefont {Xue}, \citenamefont {Yang}, \citenamefont {Lai}, \citenamefont
  {Yu}, \citenamefont {Lin}, \citenamefont {Chong}, \citenamefont {Shvets},\
  and\ \citenamefont {Zhang}}]{s5}%
  \BibitemOpen
  \bibfield  {author} {\bibinfo {author} {\bibfnamefont {F.}~\bibnamefont
  {Gao}}, \bibinfo {author} {\bibfnamefont {H.}~\bibnamefont {Xue}}, \bibinfo
  {author} {\bibfnamefont {Z.}~\bibnamefont {Yang}}, \bibinfo {author}
  {\bibfnamefont {K.}~\bibnamefont {Lai}}, \bibinfo {author} {\bibfnamefont
  {Y.}~\bibnamefont {Yu}}, \bibinfo {author} {\bibfnamefont {X.}~\bibnamefont
  {Lin}}, \bibinfo {author} {\bibfnamefont {Y.}~\bibnamefont {Chong}}, \bibinfo
  {author} {\bibfnamefont {G.}~\bibnamefont {Shvets}},\ and\ \bibinfo {author}
  {\bibfnamefont {B.}~\bibnamefont {Zhang}},\ }\bibfield  {title} {\bibinfo
  {title} {Topologically protected refraction of robust kink states in valley
  photonic crystals},\ }\href@noop {} {\bibfield  {journal} {\bibinfo
  {journal} {Nature Physics}\ }\textbf {\bibinfo {volume} {14}},\ \bibinfo
  {pages} {140} (\bibinfo {year} {2018})}\BibitemShut {NoStop}%
\bibitem [{\citenamefont {Lugiato}\ and\ \citenamefont {Lefever}(1987)}]{s6}%
  \BibitemOpen
  \bibfield  {author} {\bibinfo {author} {\bibfnamefont {L.~A.}\ \bibnamefont
  {Lugiato}}\ and\ \bibinfo {author} {\bibfnamefont {R.}~\bibnamefont
  {Lefever}},\ }\bibfield  {title} {\bibinfo {title} {Spatial dissipative
  structures in passive optical systems},\ }\href@noop {} {\bibfield  {journal}
  {\bibinfo  {journal} {Physical Review Letters}\ }\textbf {\bibinfo {volume}
  {58}},\ \bibinfo {pages} {2209} (\bibinfo {year} {1987})}\BibitemShut
  {NoStop}%
\end{thebibliography}%

 \end{document}