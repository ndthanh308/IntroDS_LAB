\documentclass[prl,twocolumn,floatfix]{revtex4}
\usepackage{graphicx}
\usepackage{graphics}
\usepackage{latexsym}
\usepackage{xcolor}
\usepackage{amsmath, amsthm, amssymb, wasysym}
\usepackage{epsfig}
%\usepackage{hyperref}
\usepackage[normalem]{ulem}
\usepackage{natbib}

\usepackage{xr}
%\usepackage{hyperref}

\def\beq{\begin{equation}}
\def\eneq{\end{equation}}
\def\eps{{\epsilon}} 
\def\ts{{\tilde s}}
\def\nbu
\def\nbuabs{{\left|\nabla u\right|}}
\def\nbut
\def\sg{{\sigma}} 
\def\sgext{{\sigma_{\hbox{ext}} }}
\def\k{{\kappa}}
\def\pe{{\phi_s^{\hbox{eq}}}}
\def\eq{{\hbox{eq}}}
\def\exp{{\hbox{exp}}}
\def\e{{\hbox{e}}}
\def\sin{{\hbox{sin}}}
\def\cos{{\hbox{cos}}}
\def\pbe{{\Phi^{\hbox{eq}}}}
\def\pbc{{\Phi^{\hbox{crit}}}} \def\pbo{{\Phi_{o}}}
\def\pt{{\partial}}
\def\uqo{{\left<U^{o}_{q}U^{o}_{-q}\right>}}
\def\uq*o{{\left<U^{o}_{q^*}U^{o}_{-{q^*}}\right>}}
\def\log{{\hbox{log }}}
\def\ln{{\hbox{ln}}}
\def\crit{{\hbox{crit}}}
\def\bm{{\bf m}}
\def\bx{{\bf x}}
\def\bq{{\bf q}}
\def\bk{{\bf k}}
\def\om{{\omega}}
\def\l{\left}
\def\r{\right}
\def\tq{{\tilde q}}
{\large \def\tCo{{\tilde C_o}}
\def\tsg{{\tilde \sigma}}
\def\Piex{{\Pi_{\hbox{ex}}}}
\def\const{{\text{const.}}}
\def\eff{{\hbox{eff}}}
\def\lb{\label}

\begin{document}

\title{Active fractal networks with stochastic force monopoles and force dipoles unravel subdiffusion of chromosomal loci}

\author{Sadhana Singh and Rony Granek\footnote[1]{Corresponding author. E-mail: rgranek@bgu.ac.il}}

\affiliation{Avram and Stella Goldstein-Goren Department of Biotechnology Engineering, and Ilse Katz Institute for Nanoscale Science and Technology, Ben-Gurion University of The Negev, Beer Sheva 84105, Israel} 

\begin{abstract}
\label{sec:abs} 
We study the Rouse-type dynamics of elastic fractal networks with embedded, stochastically driven, active force monopoles and dipoles, that are temporally correlated. We compute, analytically -- using a general theoretical framework -- and {\it via} Langevin dynamics simulations, the mean square displacement of a network bead. Following a short-time super-diffusive behavior, force monopoles yield anomalous subdiffusion with an exponent identical to that of the thermal system. Force dipoles do not induce subdiffusion, and result in rotational motion of the whole network -- as found for micro-swimmers -- and network collapses beyond a critical force amplitude. The collapse persists with increasing system size, signifying a true first-order dynamical phase transition. We conclude that the observed identical subdiffusion exponents of chromosomal loci in normal and ATP-depleted cells are attributed to active force monopoles rather than force dipoles.
\end{abstract}
\pacs{87.15.ad,87.15.-v,05.40.-a,47.53.+n}

\maketitle
Active matter is attracting increasing interest and spans diverse systems \cite{ramaswamy_2010_anu_rev,active_marchetti_2013,cates_2015_review,active_2015,cates_2016_active_matter,active_fodor_2018}. It has implications for living matter, e.g., moving micro-organisms, {\it in-vivo} intracellular dynamics, or {\it in-vitro} active cytoskeleton components \cite{cells_wilhelm_2008,invivo_robert_2010,cell_toyota_2011,cytoskeleton_mackintosh_2012,cytoplasm_guo_2014,living_matter_fodor2015}. Much effort has been recently devoted to the phase diagram of active systems \cite{cates_2015_review,phase_gonnella_2015}. Early studies aimed to obtain steady-state near-equilibrium dynamics, which allow calculation of the mean square displacement (MSD) of a probe monomer and comparison with experiments \cite{granek_1999,weber_2012,msd_barkai_2012,msd_ghosh_2022,garini_2009_prl,msd_bronshtein2015}.

In one such recent {\it in-vivo} experiment, the motion of different chromosomal loci in bacteria and yeast cells was probed in both standard and ATP-depleted cells \cite{weber_2012}. It was found that in both cell types, chromosomal loci perform anomalous sub-diffusion, MSD$\sim t^{\nu}$, with identical exponents $\nu\simeq 0.4$ yet different amplitudes; the ATP-depleted cells yield a lower amplitude. Telomeres in chromatin of mammalian cells also exhibit subdiffusion with a similar exponent, $\nu\simeq 0.4$ \cite{garini_2009_prl,msd_bronshtein2015}.

These results raise several questions. First, the reason behind the equality of the two exponents in Ref. \cite{weber_2012} is not clear. Rouse dynamics of active linear chains do yield identical exponents, which, however, have a fixed value $1/2$ \cite{winkler_2020}, larger than the experimental value $0.4$. From a different perspective, a simple model that uses the generalized Langevin equation with a delta-correlated active force that does not obey the fluctuation-dissipation theorem (FDT), yields -- for $\nu_{\text{th}}>1/2$ -- the relation $\nu_{\text{ac}}=2\nu_{\text{th}}-1$ between the thermal ($\nu_{\text{th}}$) and active ($\nu_{\text{ac}}$) anomalous exponents \cite{granek_2000}, and $\nu_{\text{ac}}=0$ for $\nu_{\text{th}}\leq 1/2$. Hence, identical active and passive (thermal) exponents are obtained only for normal diffusion, $\nu_{\text{ac}}=\nu_{\text{th}}=1$.

This argument, however,  does not concern with the structure of chromatin and the distribution of active forces within it. In recent years significant insight has been gained into chromatin structure in the interphase stage. The fractal globule model (FGM) \cite{grosberg_1988,grosberg_1993} gained a lot of attention recently due to the remarkable confirmation of its predictions in Hi-C experiments \cite{hc_aiden_2009}. Small deviations from the fractal globule model exponents have been observed \cite{bancaud_2012_fractal,iashina_2019_fractal}. Here we wish to take advantage of the fractal-like structure of chromatin. 

Fractals can be classified into different groups. For example, disordered {\it vs} deterministic fractals and rigid (or semi-rigid) {\it vs} flexible (polymer-like) fractals. 
They are characterized by a few broken dimensions \cite{stauffer_1992}: (i) the mass fractal dimension $d_{f}$ governing the scaling $M(r)\sim r^{d_{f}}$ of the mass $M(r)$ enclosed
in concentric spheres of radius $r$, (ii) the spectral dimension
$d_{s}$ governing the scaling $g(\omega)\sim\omega^{d_{s}-1}$
of the vibrational density of states (DOS) $g(\omega)$ with frequency
$\omega$ \cite{alexander_1982,alexander_1989}, and (iii) the topological dimension
$d_{l}$, that governs the scaling $M(l)\sim l^{d_{l}}$ of the mass
$M(l)$ enclosed in concentric ``spheres" of radius $l$ in the topological space. 
The three broken dimensions, $d_s$, $d_l$, and $d_f$, 
obey the inequalities: $1\leq d_s \leq d_l\leq d_f \leq d$ where $d$ is the Euclidean embedding space dimension.

Considering the FGM, what are the anticipated effects of active forces acting on chromatin within living cells? These forces can arise from various sources \cite{albert_2002, mackintosh_2008,shivshankar_2017,bruinsma_2014}, but it is sufficient to focus on two main categories: force monopoles and force dipoles. Force monopoles are external forces and cannot emerge from within the network. For instance, in prokaryotic cells, they can be exerted {\it via} direct contacts with the cytoplasm and membrane, whereas in eukaryotic cells, the cytoskeleton can exert forces through the nuclear envelope \cite{shivshankar_2017}. On the other hand, force dipoles are internal forces predominantly found in the cytoskeleton, such as the actomyosin system. They might also be present in chromatin, potentially through various enzymes that actively interact with DNA \cite{bruinsma_2014}.

Therefore, we consider these two categories of active forces acting on a fractal network. This generalizes previous works on active linear chains to arbitrary fractals, allowing applications to, e.g., chromatin and active polymeric gels near the percolation (gelation) threshold. We follow the formalism developed in \cite{granek_2005,shlomi_2012_pre,shlomi_2012_prl,granek_2011} for the thermal motion of fractal networks and generalize it here for the active forces. To verify the analytical equations, we perform extensive Langevin dynamics (LD) simulations of the Sierpinski gasket.

{\it Analytical theory}:-- Consider a fractal network of beads connected by harmonic springs where, in addition to the white thermal noise, beads experience stochastic active forces distributed randomly but uniformly over the network. In the case of force monopoles (i.e., forces exerted from the exterior), the force field $\vec{F}({\vec r},t)$ is
\beq
\vec{F}({\vec r},t)=\sum_j \vec f_j(t) \delta(\vec r-\vec r_j),
\eneq
where $\vec{f}_j(t)$ is an active force acting on $j$-th bead. For force dipoles that can be exerted internally, it is given by
\beq
\vec{F}({\vec r},t)=\sum_j \vec f_j(t) \left[\delta(\vec r-\vec r_j)-\delta(\vec r-\vec r_j-\vec\epsilon_{j})\right],
\eneq
where $\vec{\epsilon}_{j}$ is a randomly chosen vector from $j$ to one of its nearest-neighbor beads ($|\vec{\epsilon}_{j}|=b$), and $\vec f_j$ is taken parallel to $\vec{\epsilon}_{j}$. A positive $\vec f_j$ implies an inward (contractile) force dipole, and a negative $\vec f_j$ implies an outward (extensile) one \cite{ramaswamy_2002_prl,ramaswamy_2010_anu_rev,dipoles_intro_de_2015}.

The stochastic active forces are assumed to fluctuate independently of each other and to follow the random telegraph process \cite{gardiner_1991}, where the random variables take two values, $f_0$ and $0$, describing the `on' (probability $p$) and `off' (probability $1-p$) states of the force. The auto-correlation function of the forces thus follows
\beq
\langle \vec f _i(t)\cdot \vec f _j(0)\rangle =f_0^2p^2\hat f_i\cdot\hat f_j+f_0^2\delta_{ij}p(1-p)\e^{-t/\tau},
\lb{force_corr}
\eneq
where $t$ is lag-time and $\tau$ is the force correlation time. We assume that the force directions are isotropically distributed.

To simplify the analytical treatment, we use the scalar elasticity Hamiltonian \cite{elasticity_nakayama_1994,granek_2005} for the displacements $\vec{u}_i=\vec{r}_i-\vec{r}_{0,i}$ about the equilibrium positions $\{\vec{r}_{0,i}\}$, $H\l[\{\vec{u}_i\}\r] = {1\over 2}m\omega_o^2\sum_{<ij>} \left(\vec{u}_i-\vec{u}_j\right)^2$, where $\omega_o$ is the spring self-frequency, $m$ is the bead mass, and $<ij>$ stands for nearest-neighbor pairs connected by springs. The Langevin equations of motion in the overdamped limit, assuming local friction (i.e., a Rouse-type model), are
\begin{equation}
\gamma \frac{d\vec r_i(t)}{dt} =-\nabla_i H+\vec{\zeta}_i(t)+\vec f_i(t)
\label{Langevin1}
\end{equation}
($i=1,...,N$), where $\gamma$ is the local friction coefficient and $\vec{\zeta}_i$ is the thermal white noise obeying FDT \cite{kubo_1966}. Using the normal modes of the fractal network ${\Psi_{\alpha}(\vec{r_i})}$, whose corresponding eigenfrequencies are ${\omega_{\alpha}}$, and transforming Eq.~(\ref{Langevin1}) to the normal mode space, we obtain
\begin{equation}
\frac{d\vec u_{\alpha}}{dt} = -\Gamma_{\alpha}u_{\alpha}+\vec{\zeta}_{\alpha}(t)+ \Lambda_{\alpha}\vec{F}_{\alpha}(t) 
\label{Langevin4}
\end{equation}
Here $\vec u_{\alpha}(t)$ is the amplitude of a normal mode ${\Psi_{\alpha}(\vec{r_i})}$ at time $t$, $\Gamma_{\alpha}=m\omega_{\alpha}^2 \Lambda_{\alpha}$ is the mode relaxation rate, where $\Lambda_{\alpha}=1/\gamma$ is the mode mobility coefficient. $\vec{\zeta}_{\alpha}(t)$ and $\vec{F}_{\alpha}(t)$ are (respectively) the mode-transformed thermal white noise and active noise field; the latter takes the form: (i) $\vec{F}_{\alpha}(t)=\sum_j \delta\vec f_j(t) \Psi_{\alpha}(\vec r_j)$ for force monopoles, and (ii)
$\vec{F}_{\alpha}(t)=\sum_j \delta\vec f_j(t) \left[\Psi_{\alpha}(\vec r_j)-\Psi_{\alpha}(\vec r_j+\vec\epsilon_{j})\right]$ for force dipoles, where $\delta\vec{f}_j(t)=\vec{f}_j(t)-f_0p\hat{f}_j$ are the force fluctuations about their mean.

Next, in the supplemental material (SM), Sec.~S1~\cite{supp}, we calculate the auto-correlation function of the active force field in mode space
\beq
\langle\vec{F}_{\alpha}(t)\vec{F}_{\alpha}(0)\rangle \simeq W_{\alpha}\phi f_0^2p(1-p)\e^{-t/\tau},
\eneq
where $W_{\alpha}=Q\omega^{\xi}$ :  (i) $Q=1$, $\xi=0$, for force monopoles, and (ii) $Q=2 C_0\omega_o^{-2d_s/d_l}$, $\xi=2d_s/d_l$, for force dipoles. Here $\phi$ is the fraction of force monopoles or dipoles in the system, and $C_0$ is a numerical constant related to the mean localization properties of the normal modes \cite{alexander_1982,alexander_1989,stauffer_1992,bunde_1992,bunde_1997}.

Solving the Langevin equation (\ref{Langevin4}) (SM, Sec.~S1~\cite{supp}), expanding the MSD in terms of the normal modes, and performing disorder averaging, we obtain the MSD of an internal network bead, $\langle\left(\vec{r}(t)-\vec{r}(0)\right)^{2}\rangle=\langle\Delta\vec{r}(t)^2\rangle_{\text{th}}+\langle\Delta\vec{r}(t)^2\rangle_{\text{ac}}$ as the sum of a purely thermal contribution, $\langle\Delta\vec{r}(t)^2\rangle_{\text{th}}$, and a purely active contribution, $\langle\Delta\vec{r}(t)^2\rangle_{\text{ac}}$, where
\begin{subequations}
\begin{align}
\begin{split}
& \langle\Delta\vec{r}(t)^2\rangle_{\text{th}} = {1\over N}\sum_{\alpha} {6k_BT\over m \omega_{\alpha}^2}\left(1-\e^{-\Gamma_{\alpha}t}\right),
\label{MSD_total_th}
 \end{split}\\
\begin{split}
 &\langle\Delta\vec{r}(t)^2\rangle_{\text{ac}} = \frac{1}{N}\sum_{\alpha}\frac{2\phi f_0^2p(1-p) W_{\alpha}\Lambda_{\alpha}^2}{\Gamma_{\alpha}\left(\Gamma_{\alpha}+\tau^{-1}\right)}\\
         & \times\left(1 + {\tau^{-1}\over \Gamma_{\alpha}-\tau^{-1}}\e^{-\Gamma_{\alpha}t} -  {\Gamma_{\alpha}\over \Gamma_{\alpha}-\tau^{-1}}\e^{-t/\tau} \right).
\end{split}\label{MSD_total_ac}
\end{align}
\end{subequations}
As shown in Refs.\cite{granek_2005,shlomi_2012_pre,shlomi_2012_prl,granek_2011}, 
the thermal MSD exhibits subdiffusion, $\langle\Delta\vec{r}(t)^2\rangle_{\text{th}}\sim t^{\nu_{\text{th}}}$, with $\nu_{\text{th}}=1-\frac{d_s}{2}$, provided that $d_s<2$.
% Figure environment removed


For a large system such that the frequency spectrum is dense, $\langle\Delta\vec{r}(t)^2\rangle_{\text{ac}}$ can also be evaluated analytically. Using the vibrational DOS $g(\omega)\sim \omega^{d_s-1}$, we can convert the sum in Eq.~(\ref{MSD_total_ac}) to an integral over the frequency, where the lower and upper integration limits determine (respectively) the longest, $\tau_N\sim \tau_0 N^{2/d_s}$, and shortest, $\tau_0=\gamma/(m\omega_0^2)$, relaxation times of the system. Assuming short times such that $t\ll \tau\ll \tau_N$ we find, for both force monopoles and force dipoles, super-diffusive, ballistic-type, behavior,
\beq
\langle\Delta\vec{r}(t)^2\rangle_{\text{ac}}\simeq B_{\text{acs}} t^2\;\;\;   t\ll \tau,
\label{superdiff}\eneq
where the amplitude follows the scaling: (i) for force monopoles, $B_{\text{acs}}\sim \phi f_0^2\tau^{-d_s/2}$ provided that $d_s<2$, and (ii) for force dipoles, given that $d_s/2 + d_s/d_l>1$, we have $B_{\text{acs}}\sim \phi f_0^2\tau^{-1}$. Otherwise, for rare fractal geometries where $d_s/2 + d_s/d_l<1$, we get $B_{\text{acs}}\sim \phi f_0^2\tau^{-d_s/2-d_s/d_l}$.

For times much longer than the force correlation time such that $ \tau \ll t\ll \tau_N$, we find for force monopoles
\beq
\langle\Delta\vec{r}(t)^2\rangle_{\text{ac}}= B_{\text{ac}}t^{\nu_{\text{ac}}},
\label{subdiff}\eneq
where $\nu_{\text{ac}}=1-d_s/2$, that is {\it identical} to the thermal subdiffusion exponent. The active amplitude is $B_{\text{ac}}\sim \phi f_0^2\tau$ such that ${B_{\text{ac}}\over B_{\text{th}}}= {\phi f_0^2\tau p(1-p)\over 3k_BT \gamma}$. In addition, for arrested translation motion, the MSD saturates at a value $\sim N^{2/d_s-1}$, showing a Landau-Peierls-like instability, as in the passive case \cite{burioni_2002,burioni_2004}. Hence, for force monopoles, it is possible to assign an effective temperature $T_{\text{eff}} = T+\phi {f_0}^2\tau p(1-p)/(3\gamma k_B)$. For force dipoles -- since for typical fractals $1-d_s/2-d_s/d_{l}<0$ -- {\it anomalous subdiffusion regime is absent} and the (translation arrested) MSD saturates at a constant, $N$ independent, value.
% Figure environment removed

{\it Simulations}:-- These analytical results are based on the scalar elasticity Hamiltonian, assuming a large system, $N\to\infty$. To verify the sensitivity of our results to these assumptions, we performed Langevin dynamics (LD) simulations of a bead-spring Sierpinski gasket, using different generations where the $n_{th}$ generation is denoted by S$_n$, with $n=1,2,...$. We solve Eq.~(\ref{Langevin1}) with $H=\frac{1}{2}m\omega_0^2\sum_{<ij>}(\vec{r}_i-\vec{r}_j-b \hat{r}_{ij} )^2$
for the Hamiltonian of harmonic springs having an equilibrium distance $b$, where $\hat{r}_{ij}$ is the unit vector of the distance $\vec{r}_i-\vec{r}_j$ between beads $i$ and $j$. For $b=0$, it reduces to the scalar elasticity Hamiltonian discussed above. We follow the position of an internal bead in a Sierpinski gasket and calculate its {\it time-averaged} MSD using the standard method for a stationary noise (For detail, see SM, Sec.~2 \cite{supp}).

Consider first the dynamics of a thermal (passive) network (previously studied \cite{granek_2005,shlomi_2012_pre,shlomi_2012_prl,granek_2011}). We compare the MSD obtained from simulations, for both vanishing and non-vanishing $b$, with the numerical evaluation of Eq.~(\ref{MSD_total_th}) for Sierpinski gasket S$_9$, $N=9843$ (Fig.~S3(a)~\cite{supp}). Anomalous subdiffusion is clearly observed spanning over three decades of time. The fitted exponents to all three curves agree very well with the predicted value (using $d_s=2\ln 3/\ln 5=1.365$ \cite{alexander_1982,rammal_1983}) $\nu_{\text{th}}=1-d_s/2=0.317$ with similar amplitudes except the $b\neq 0$ case, which shows a different amplitude. Comparing the MSD of an internal and a peripheral bead, we find almost identical exponents but somewhat different amplitudes (Fig.~S3(b) \cite{supp}).

{\it Force monopoles}:-- We now turn to an active fractal network with force monopoles. To each bead, we assign an active force of random orientation. In Fig.~S4(a) in SM \cite{supp} and Fig.~\ref{fig_mono_msd}(a) (drift velocity eliminated, see below), we depict the MSD of an internal bead for different values of active force parameters and arrested thermal motion ($T=0$). We observe a crossover at $t\sim \tau$, from short-time ballistic motion, MSD$\sim t^2$, to subdiffusion at intermediate times. In Fig.~S4(a), a crossover to ballistic motion appears at longer times, indicating the presence of a center-of-mass (CM) drift velocity. The latter emerges from the incomplete cancellation of the total force due to the system's finite size. For $N\to\infty$, $\sum_{i\neq j}\hat f_i\cdot\hat f_j=0$; however, in a finite system, this sum may have a residual (random) value, which will nevertheless vanish if we average over many realizations of the force field.

Calculating the MSD about the drift position (Eq.~S57 \cite{supp}), we show in Fig.~\ref{fig_mono_msd}(a) that the intermediate-time subdiffusion regime becomes longer, MSD$\sim t^{\nu_{\text{ac}}}$ with $\nu_{\text{ac}}\simeq 0.32$ as analytically predicted, and for $t \gg \tau_N$ the long-time ballistic motion is replaced with $\sim t$ behavior associated with CM active diffusion. We can also observe the effects of the correlation time $\tau$ and force amplitude $f_0$ on the MSD ballistic and subdiffusion amplitudes confirming our prediction $B_{acs}\propto \phi f_0^2\tau^{-0.682}$ ($t\ll \tau$) and $B_{ac}\propto \phi f_0^2\tau$ ($t\gg \tau$). Similar to the passive case, for $b=0$, theory and simulations match exactly, see Fig.~S4(b) in SM \cite{supp}; for $b=1$, the subdiffusion amplitude somewhat differs, yet the exponent is unaltered.
% Figure environment removed

In Fig.~\ref{fig_mono_msd}(b), we show the combined 
effect of {\it active and thermal} forces for a few sets of parameters, and compare them with the thermal MSD. Subdiffusion regimes with $\nu\simeq 0.32$ can be observed for all cases; in one combination, having a long $\tau$ ($=100 \tau_0$) and small $f_0$ ($=m\omega_0^2 b$), two consecutive such regimes appear, where the first is dominated by thermal motion and is followed by a second one, dominated by activity and having a much larger amplitude.

{\it Force dipoles}:-- Next, we simulate force dipoles randomly assigned to a fraction $\phi=20\%$ of total bonds (unless otherwise specified), where we either prohibit or do not prohibit dipoles having a common bead. In biological networks, in which force dipoles arise from molecular motor ``pairs", the choice of distinct nodes is more physical due to the excluded volume interaction between motors. During the simulations, we update each dipole orientation to remain parallel to $\vec{r}_{ij}$. We estimate the MSD for gasket S$_9$ (Fig.~S6 \cite{supp}) for different force parameters. For times $t \ll \tau$, the MSD shows a  ballistic behavior, $\sim t^2$, which saturates to a constant value at $t \sim \tau$. Both the existence of the early ballistic motion and the {\it absence} of the intermediate subdiffusion regime for the Sirepinski gasket where $1-d_s/2-d_s/d_l=-0.5464<0$, are in accord with the analytical predictions. However, at longer times, a ballistic-like rise in MSD for higher $f_0$ is observed, requiring further analysis. 


To understand this long-time behavior, we calculate the MSD for a smaller generation, S$_4$, $N=42$. 
At longer times, the MSD rises, initially ballistic-like, followed by {\it oscillations}. 
In the movies SM-$1$ to SM-$3$ showing trajectories of a few gasket generations, we can clearly observe a persistent {\it rotational motion} of the objects, either clockwise or anti-clockwise. However, force dipoles cannot generate a net torque since the torque vanishes identically for each dipole. Moving to the CM frame of reference, 
denoting the bead $i$ position by $\vec{R}_i$ and its rotational velocity by $\vec{\Omega}_i$, it follows that $\sum_i \vec{\Omega}_iR_i^2 = 0$, which does {\it not} imply ${1\over N}\sum_i \vec{\Omega}_i = 0$, presenting the mean rotational velocity (SM, Sec.~S2.3. \cite{supp}). We thus pose that the oscillatory long-time dynamics observed in   Fig.~\ref{dipole_multi}(a) are associated mainly with the rotational motion. To confirm this hypothesis, we define a mean square angular displacement (MSAD) associated with the pure rotational motion of a bead $i$, located at mean square distance $\langle R_i^2\rangle$ from the CM, $\langle R_i^2\rangle\langle(\hat{n}(t)-\hat{n}(0))^2 \rangle$, where $\hat{n}(t)$ is the unit vector of $\vec{R}_{i}$. In Fig.~\ref{dipole_multi}(a), we plot the MSAD together with the complete MSD; the inset (linear scale) emphasizes the motion at long times. The overlap of the two oscillatory curves is almost perfect.
% Figure environment removed

We further argue that this rotational motion, whose direction and frequency randomly vary between realizations of the dipole spatial distribution (Fig.~S7(a) \cite{supp}), results from a small residual anisotropy of the realization associated with the finite size of the object. To confirm this, we consider gaskets with increasing generation, i.e., increasing $N$. In Fig.~S7(c) \cite{supp}, we show that the rise from the plateau, marking the onset of oscillations, is postponed to longer times as $N$ increases. Indeed, we expect that dipole realization fluctuations will diminish with increasing system size, leading to smaller angular velocities. A further check is done by simulating a gasket in which force dipoles are assigned to {\it all} bonds rather than just to a fraction of them. Calculating the MSD in this case, the long-term oscillations seen in Fig.~\ref{dipole_multi}(a) effectively disappear completely, see Fig.~\ref{dipole_multi}(b). 

The above behavior for force dipoles does not apply above a {\it critical force} amplitude $f_{0c}\approx m\omega_0^2 b$. For $f_0>f_{0c}$, the gasket {\it collapses} to a random shape maintaining dynamical fluctuations, but with rotational motion effectively arrested (Fig.~S7(d) \cite{supp}). The radius of gyration $R_g$ for contractile dipoles against $f_0$ is shown in Fig.~\ref{rg}(a) for S$_4$, and in Fig.~\ref{fig_chromatin}(a) for generations S$_5$$-$S$_7$. The transition persists and sharpens at larger (higher generation) gaskets, implying a true, activity-induced, first-order phase transition. The collapse may be rationalized as a result of the dynamical ``persistence length" \cite{persistence_RMP_2016,dynamical_persistence_cao_2019,Brahmachari_2023,parameter_cui_2000} -- generalized to account for the spring resistance -- rising above $b$, leading to $f_{0c}\simeq m\omega_0^2 b/\left(1-e^{-\tau/\tau_0}\right)$ (SM, Sec.~S2.4.~\cite{supp}), in agreement with our results for $\tau\gtrsim \tau_0$, Fig.~\ref{fig_chromatin}(b). Contrary to contractile dipoles (Fig.~\ref{rg}(a)), for extensile dipoles $R_g$ rises again at larger forces, Fig.~\ref{rg}(b).  
 
Interestingly, in {\it triangular micro-swimmers} that are subject to force dipoles and hydrodynamic interaction (i.e., Zimm model), rotational motion emerges too \cite{swimmer_rizvi2018,rizvi_2020}. To connect to our study, we show in the SM \cite{supp} (Fig.~S8) that even Rouse dynamics of a triangular bead-spring, with anisotropy of force dipole strength (and random-telegraph fluctuations), is enough to produce rotational motion.

Connecting with the chromatin studies \cite{weber_2012}, we conclude that the observed normal cell dynamics are caused by active force monopoles rather than force dipoles which cannot induce subdiffusion. Assuming that the distance between crosslinks is around the chromatin persistence length $\sim 250$ nm \cite{gero_2002,semiflexible_ghosh_2014,semiflexible_safran_2017} and that $\tau$ and $f_o$ are associated with cytoplasmic motor protein processivity times and forces, leads to $\tau/\tau_0=1-10^4$ and $f_0/(m\omega_0^2 b)=0.5-7$ (SM, Sec.~S2.2.~\cite{supp}). The passive and active MSDs shown in Fig.~\ref{fig_mono_msd}(b) and Fig.~S5 \cite{supp}, using parameters in this range and different fractions of force monopoles $\phi$, yield MSDs with subdiffusion regimes of {\it identical exponents} yet different amplitudes, as observed in chromatin. The MSD amplitude ratio of thermal (passive) to combined thermal+active can vary depending on the combination $\phi f_0^2\tau$, such that our study can reproduce the experimentally measured ratios of ATP-depleted to normal cells \cite{weber_2012}. For {\it S. cerevisiae}, it was found to be $0.12$, similar to the one obtained theoretically for combinations yielding $\phi f_0^2\tau\simeq 100$.  Moreover, with $\nu_{\text{th}}=\nu_{\text{ac}}=1-d_s/2\simeq 0.4$ we have $d_s\simeq 1.2$, associated with a slightly cross-linked chain. Recently, a more specific model for chromatin dynamics \cite{Brahmachari_2023}, based on a linear (non-crosslinked) chain model that is subject to force dipoles, appears to also yield similar passive and active subdiffusion exponents, albeit equal to $1/2$ and on a rather short-time-interval, as anticipated for a linear Rouse chain where $d_s=1$ \cite{alexander_1982}. In contrast, our results show the absence of a true power-law subdiffusion regime when force dipoles alone act in an ideal fractal network. We note that force dipoles, which are also expected to exist in chromatin \cite{bruinsma_2014}, should cause slow rotational motion of the chromosome (as in Fig.~\ref{dipole_multi}), which is yet to be observed. Interestingly, a rotational motion has been observed in cytoplasmic flow and proposed -- in accord with our simulations -- to be caused by actomyosin force dipoles \cite{woodhouse_2012_circulation,cytoplasimic_flow_suzuki_2017}.

\section{Acknowledgments}
SS acknowledges the BGU Kreitmann School postdoctoral fellowship for support. We thank Sam Safran and Omer Granek for useful discussions and the Avram and Stella Goldstein-Goren fund for support.

\bibliography{active.bib}


\pagebreak
\onecolumngrid
\begin{center}
\textbf{\large Supplementary Material for:
``Active fractal networks with stochastic force monopoles and force dipoles unravel subdiffusion of chromosomal loci"}
\end{center}

\setcounter{equation}{0}
\setcounter{figure}{0}
\setcounter{table}{0}
\setcounter{page}{1}
\makeatletter
\renewcommand{\theequation}{S\arabic{equation}}
\renewcommand{\thefigure}{S\arabic{figure}}
%\renewcommand{\bibnumfmt}[1]{[S#1]}
%\renewcommand{\citenumfont}[1]{S#1}
\renewcommand{\thesection}{S\arabic{section}}
\renewcommand{\thesubsection}{\thesection.\arabic{subsection}}
\renewcommand{\thesubsubsection}{\thesubsection.\arabic{subsubsection}}

\setcounter{equation}{0}
\setcounter{page}{1}
\setcounter{secnumdepth}{3}

{\bf S1. Model for analytical analysis:} Following the formalism for the bead spring model \cite{granek_2005}, it is assumed that the networks form a disordered fractal, and the scalar elasticity Hamiltonian \cite{elasticity_nakayama_1994,alexander_1982,granek_2005} of the system is given by
\begin{equation}
  H\l[\{\vec{u}_i\}\r]=\frac{1}{2}m\omega_{0}^2\sum_{<ij>}(\vec{u}_i-\vec{u}_j)^2,
\label{Hamiltonian_ul}
\end{equation}
where $<ij>$ stands for the pair of beads connected by springs, with spring self-frequency $\omega_0$, $m$ is the bead mass, and $\vec{u}_i=\vec{r}_i-\vec{r}_{0,i}$ is the displacement of position vector $\vec{r}_i$ of $i$-th bead about the equilibrium position $\{r_{0,i}\}$. The eigenstates (normal modes) $\Psi_{\alpha}(\vec{r}_i)$ of the Hamiltonian\ (\ref{Hamiltonian_ul}), are solutions of the eigenvalue equation
\beq
\omega_0^2 \sum_{j\in i }\l[\Psi_{\alpha}(\vec{r}_j)-\Psi_{\alpha}(\vec{r}_i)\r]=-\omega_{\alpha}^{2}\Psi_{\alpha}(\vec{r}_i).
\lb{eigen}
\eneq
On a fractal, the normal modes $\Psi_{\alpha}(\vec{r}_i)$ are strongly localized in space, unlike the oscillatory behavior characteristic of uniform networks. A disorder-averaged eigenstate may be defined according to
\beq
\bar\Psi(\omega_{\alpha},|\vec{r}_i-\vec{r}_j|)=N\langle\Psi_{\alpha}(\vec{r}_i)\Psi_{\alpha}(\vec{r}_j)\rangle_{\text{dis}} ~,
\label{dis_ave_eigen}
\eneq
where $\langle ...\rangle_{\text{dis}}$ denotes disorder averaging, i.e., averaging over all realizations of the fractal, keeping the nodes $i$ and $j$ fixed, or averaging, within a given realization, overall different pair of nodes $i$ and $j$ that have the same topological space distance $l$. 
Note that mode normalization implies $\langle\Psi_{\alpha}^2(\vec{r}_i)\rangle_{\text{dis}}=1/N$. It has been shown that $\bar\Psi(\omega_{\alpha},i)$ obeys the following scaling form \cite{stauffer_1992,alexander_1982,alexander_1989} 
\beq 
 \bar\Psi(\omega_{\alpha},l)=f\l[\l(\frac{\omega_{\alpha}}{\omega_o}\r)^{\frac{d_s}{d_l}}l\r],
\label{Psi}
\eneq 
where $f(y)$ is the scaling function. For $y\gg 1$, $f(y)$ is exponentially decaying, and, for
$y\ll 1$, $f(y)\simeq 1-C_0\times y^2$ where $C_0$ is a numerical constant \cite{alexander_1989,bunde_1992,bunde_1997}.

For the dynamical study of fractal network, we followed the Langevin equations described in Refs. \cite{granek_2005,granek_2011,shlomi_2012_prl,shlomi_2012_pre} for a flexible fractal network in the high damping limit for a system in equilibrium. 
In addition to the white thermal noise, beads experience stochastic active forces distributed randomly but uniformly over the network. In the case of force monopoles (i.e., forces exerted from the exterior), the force field $\vec{F}({\vec r},t)$ is
\beq
\vec{F}({\vec r},t)=\sum_j \vec f_j(t) \delta(\vec r-\vec r_j),
\eneq
where $\vec{f}_j(t)$ is an active force acting on $j$-th bead. For force dipoles, that can be exerted internally, it is given by
\beq
\vec{F}({\vec r},t)=\sum_j \vec f_j(t) \left[\delta(\vec r-\vec r_j)-\delta(\vec r-\vec r_j-\vec\epsilon_{ji})\right],
\eneq
where $\vec{\epsilon}_{j}$ is a randomly chosen vector from $j$ to one of its nearest-neighbor nodes ($|\vec{\epsilon}_{j}|=b$), and $\vec f_j$ is taken parallel to $\vec{\epsilon}_{j}$. In the continuum limit of point force dipoles, the limit $\epsilon_{j}\to 0$ should be taken. A positive $\vec f_j$ implies an inward (contractile) force dipole, and a negative $\vec f_j$ implies an outward (extensile) one \cite{ramaswamy_2002_prl,ramaswamy_2010_anu_rev,dipoles_intro_de_2015}.

The stochastic active forces are assumed to fluctuate independently of each other and to follow the random telegraph (on-off) process \cite{gardiner_1991}, where the random variables take two values, $0$ and $f_0$, describing the 'off' and 'on' states of the force. The auto-correlation function of the forces thus follows
\beq
\langle \vec f _i(t)\cdot \vec f _j(0)\rangle =f_0^2p^2\hat f_i\cdot\hat f_j+f_0^2\delta_{ij}p(1-p)\e^{-t/\tau},
\lb{force_corr-1}
\eneq
where $t$ is lag-time, $p$ is the probability for an 'on' state, and $\tau$ is the force correlation time. We assume that the force directions are isotropically distributed,  implying $\sum_{i,j}\hat f_i\cdot\hat f_j=0$ in an infinite network; however, in a finite system this sum may have a residual (random) value. This sum will nevertheless vanish on a finite system if we perform in addition disorder averaging, in which the force field spans over all its realizations. For convenience, we consider the force fluctuations about their mean $f_0 p\hat f _i$. Introducing
\begin{equation}
\delta\vec f _i(t)=\vec f _i(t)-f_0 p\hat f _i~,
\end{equation}
we have
\beq
\langle \delta\vec f _i(t)\cdot \delta\vec f _j(0)\rangle =f_0^2\delta_{ij}p(1-p)\e^{-t/\tau}.
\lb{force_corr}
\eneq
It follows that $\{\vec u_{i}\}$ will stand henceforth for the displacements in bead positions about the mean coordinates, as dictated by the mechanical balance between the mean forces and the network elasticity.

For a Rouse-type fractal network, the set of Langevin equations of motion (same as Eq.~(4) explained in the main text) of an active system are
\begin{equation}
\gamma \frac{d\vec r_i(t)}{dt} =-\nabla_i H + \vec{\zeta}_i(t) +\vec f_i(t), 
\label{Langevin1}
\end{equation}
where $H$ is the scalar elasticity Hamiltonian given in Eq.~(\ref{Hamiltonian_ul}), $\gamma$ is the local friction coefficient ($\gamma=3\pi\eta b$ in case of Stokes drag, where $\eta$ is the solvent viscosity and $b$ is the bead diameter), $\vec{\zeta}_i$ is the thermal white noise obeying Fluctuation-Dissipation (FD) theorem \cite{kubo_1966}, and $\vec{f}_i$ is the active force acting on $i$-th bead. 

Using the normal modes of the fractal network ${\Psi_{\alpha}(\vec{r_i})}$, whose corresponding eigen-frequencies are ${\omega_{\alpha}}$, and transforming Eq.~(\ref{Langevin1}) to the normal mode space, we obtain
\begin{equation}
\frac{d\vec u_{\alpha}}{dt} = -\Gamma_{\alpha}\vec u_{\alpha} + \vec{\zeta}_{\alpha}(t) + \Lambda_{\alpha}\vec{F}_{\alpha}(t),
\label{Langevin4}
\end{equation}
Here $\vec u_{\alpha}(t)$ is the amplitude of a normal mode $\alpha$ at time $t$, $\Gamma_{\alpha}=m\omega_{\alpha}^2\Lambda_{\alpha}$ is the mode relaxation rate, where $\Lambda_{\alpha}$ is the mode mobility coefficient, $\Lambda_{\alpha}=1/\gamma$. 
$\vec{\zeta}_{\alpha}(t)$ is the (mode transformed) thermal white noise that obeys the fluctuation-dissipation theorem
\begin{equation}
\langle\vec{\zeta}_{\alpha}(t)\vec{\zeta}_{\beta}(t^{\prime})\rangle=2
k_BT\Lambda_{\alpha}\delta_{\alpha,\beta}\delta(t-t^{\prime}),
\lb{FD-theorem}\end{equation}
and, the last term, $\vec{F}_{\alpha}(t)$ is the mode transformed active noise fluctuations (about their mean), which takes the form: (i) for force monopoles
\beq
\vec{F}_{\alpha}(t)=\sum_j \delta\vec f_j(t) \Psi_{\alpha}(\vec r_j),
\eneq
and (ii) for force dipoles
\beq
\vec{F}_{\alpha}(t)=\sum_j \delta\vec f_j(t) \left[\Psi_{\alpha}(\vec r_j)-\Psi_{\alpha}(\vec r_j+\vec\epsilon_j)\right].
\eneq

\noindent
{\bf S1.1. Steady-state mode space auto-correlation function:} We first calculate the auto-correlation function of the active force field in mode space:\\
(i) for force monopoles
\beq
\langle\vec{F}_{\alpha}(t)\vec{F}_{\alpha}(0)\rangle=\sum_{i,j}\langle\delta\vec f_i(t)\cdot \delta\vec f_j(0)\rangle \langle \Psi_{\alpha}(\vec r_i)\Psi_{\alpha}(\vec r_j)\rangle_{\text{dis}}~,
\label{force_mono_corr}
\eneq
(ii) for dipoles
\begin{eqnarray}
& & \langle\vec{F}_{\alpha}(t)\vec{F}_{\alpha}(0)\rangle=
\sum_{i,j} \langle\delta\vec f_i(t)\cdot \delta\vec f_j(0)\rangle\nonumber\times\\
& & \left[\langle \Psi_{\alpha}(\vec r_i)\Psi_{\alpha}(\vec r_j) \rangle_{\text{dis}} - \langle \Psi_{\alpha}(\vec r_i)\Psi_{\alpha}(\vec r_j+\vec\epsilon_j) \rangle_{\text{dis}}-\langle \Psi_{\alpha}(\vec r_i+\vec\epsilon_i)\Psi_{\alpha}(\vec r_j) \rangle_{\text{dis}}+\langle \Psi_{\alpha}(\vec r_i+\vec\epsilon_i)\Psi_{\alpha}(\vec r_j+\vec\epsilon_j) \rangle_{\text{dis}} \right].
\label{force_dipoles_corr}
\end{eqnarray}
The averages in Eqs. (\ref{force_mono_corr})-(\ref{force_dipoles_corr}) is performed over both the stochastic evolution of the active forces and disorder:\\
(i) For force monopoles we obtain, after disorder averaging, using Eq.~(\ref{dis_ave_eigen}),
\beq
\langle\vec{F}_{\alpha}(t)\vec{F}_{\alpha}(0)\rangle={1\over N}\sum_{i,j}\langle\delta\vec f_i(t)\cdot \delta\vec f_j(0)\rangle \bar\Psi(\omega_{\alpha},|\vec r_j-\vec r_i|),
\label{Falpha1}\eneq
and after ensemble (or time) averaging over the stochastic forces using Eq.~(\ref{force_corr}), and implementation of the normalization $\bar\Psi(\omega_{\alpha},0)=1$, we get
\beq
\langle\vec{F}_{\alpha}(t)\vec{F}_{\alpha}(0)\rangle =\phi f_0^2p(1-p)\e^{-t/\tau},
\eneq
where $\phi$ is the fraction of force monopoles. Note that Eq.~(\ref{Falpha1}) is
independent of the mode index $\alpha$.\\
(ii) For force dipoles we first use Eq.~(\ref{dis_ave_eigen}) 
\begin{eqnarray}
& & \langle\vec{F}_{\alpha}(t)\vec{F}_{\alpha}(0)\rangle=
{1\over N}\sum_{i,j} \langle\delta\vec f_i(t)\cdot \delta\vec f_j(0)\rangle\nonumber\times\\
& & \left[\bar\Psi(\omega_{\alpha},|\vec r_j-\vec r_i|) - \bar\Psi(\omega_{\alpha},|\vec r_j+\vec\epsilon_j-\vec r_i|)-
\bar\Psi(\omega_{\alpha},|\vec r_j-\vec r_i-\vec\epsilon_i|)+\bar\Psi(\omega_{\alpha},|\vec r_j-\vec r_i+\vec\epsilon_j-\vec\epsilon_i|) \right],
\end{eqnarray}
and after the time (or ensemble) averaging over the stochastic forces, using Eq.~(\ref{force_corr}),
\beq
\langle\vec{F}_{\alpha}(t)\vec{F}_{\alpha}(0)\rangle = 2\phi f_0^2\left[\bar\Psi(\omega_{\alpha},0)-\bar\Psi(\omega_{\alpha},\epsilon)\right]p(1-p)\e^{-t/\tau},
\eneq
where $\phi$ is now the fraction of force dipoles, leading to ($\bar\Psi(\omega_{\alpha},0)=1$) 
\beq
\langle\vec{F}_{\alpha}(t)\vec{F}_{\alpha}(0)\rangle = 2\phi f_0^2\left[1-\bar\Psi(\omega_{\alpha},\epsilon)\right]p(1-p)\e^{-t/\tau}.
\eneq
Note that transforming Eq.~(\ref{Psi}) from topological space to real space, by using $l\sim (r/b)^{d_f/d_l}$, (where  $d_f$ is fractal dimension, 
and $d_l$ topological dimension), implies
\beq \bar\Psi(\omega_{\alpha},r)=f\l[(\omega_{\alpha}/\omega_o)^{d_s/d_l} (r/b)^{d_f/d_l}\r].
\eneq
Using $\epsilon=b$, and the low-frequency expansion for the eigenmodes, we have $\bar\Psi(\omega_{\alpha},\epsilon)\simeq 1- C_0 (\omega_{\alpha}/\omega_o)^{2d_s/d_l}$, leading to
\begin{equation}
\langle\vec{F}_{\alpha}(t)\vec{F}_{\alpha}(0)\rangle \simeq 2C_0(\omega_{\alpha}/\omega_o)^{2d_s/d_l}\phi f_0^2p(1-p)\e^{-t/\tau}.
\end{equation}

To summarize, for both types of active force, we have
\beq
\langle\vec{F}_{\alpha}(t)\vec{F}_{\alpha}(0)\rangle \simeq W_{\alpha}\phi f_0^2 p(1-p)\e^{-t/\tau},
\eneq
where
\beq
W_{\alpha}=Q\, \omega_{\alpha}^{\xi}
\eneq
such that
\beq
Q=\begin{cases} 
    1 & \text{for force monopoles},\\  
     2C_0 \omega_o^{-2d_s/d_l}  & \text{for force dipoles},
     \end{cases}
\eneq
and
\beq
\xi=\begin{cases}
    0 & \text{for force monopoles},\\
    2d_s/d_l & \text{for force dipoles.}
\end{cases}
\eneq

\vspace{0.2cm} 
\noindent
{\bf S1.2. Mode amplitude auto-correlation function:} 
Solving the Langevin equation (\ref{Langevin4}), we obtain the auto-correlation function of the displacement at a steady state such that the absolute times $t_1$, $t_2$ are taken to infinity (or that active forces started to act at time $-\infty$), thereby all transient behavior has completely relaxed. The time difference $t=|t_2-t_1|$ is finite. Hence 
$\langle \vec{u}_{\alpha}(t)\cdot\vec{u}_{\alpha}(0)\rangle\equiv \langle \vec{u}_{\alpha}(t_2)\cdot\vec{u}_{\alpha}(t_1)\rangle$. We obtain
\beq
\langle \vec{u}_{\alpha}(t)\cdot\vec{u}_{\alpha}(0) \rangle = \frac{3k_BT}{m \omega_{\alpha}^2}\e^{-\Gamma_{\alpha}t} + \frac{\phi f_0^2p(1-p) W_{\alpha}\Lambda_{\alpha}^2}{\Gamma_{\alpha}\left(\Gamma_{\alpha}^2-\tau^{-2}\right)} \left[\Gamma_{\alpha}\e^{-t/\tau}  -\tau^{-1}\e^{-\Gamma_{\alpha}t}\right],
\label{mode_corr}
\eneq
and mean square normal mode amplitude is
\beq
\langle \vec{u}_{\alpha}^{\,2} \rangle =\frac{3k_BT}{m \omega_{\alpha}^2} + \frac{\phi f_0^2p(1-p) W_{\alpha}\Lambda_{\alpha}^2}{\Gamma_{\alpha}\left(\Gamma_{\alpha}+\tau^{-1}\right)}.
\label{mode_mean_squre}
\eneq
In Eq.~(\ref{mode_mean_squre}), the first term is the thermal contribution and the second term is the contribution of the stochastic force fluctuations. For $\tau\to\infty$ the latter reduces to
\beq
\langle \vec{u}_{\alpha}^{\,2} \rangle =\frac{3k_BT}{m \omega_{\alpha}^2} + \frac{\phi f_0^2p(1-p) W_{\alpha}} {(m\omega_{\alpha}^2)^2}.
\eneq

\noindent
{\bf S1.3. Mean Square Displacement:} Expanding the two-point correlation function \cite{shlomi_2012_pre} in terms of the exact normal modes $\Psi_{\alpha}(\vec{r})$, and performing disorder averaging, we find the required MSD as the sum of a purely thermal contribution, $\langle\Delta\vec{r}(t)^2\rangle_{\text{th}}$, and a purely active contribution, $\langle\Delta\vec{r}(t)^2\rangle_{\text{ac}}$:
\beq
\langle\Delta\vec{r}(t)^2\rangle\equiv \langle \left(\vec{r}_i(t)-\vec{r}_i(0) \right)^{2} =\langle \left(\vec{u}_i(t)-\vec{u}_i(0) \right)^{2} \rangle =
{2\over N}\sum_{\alpha} \left(\langle \vec{u}_{\alpha}^{\,2} \rangle - \langle \vec{u}_{\alpha}(t)\cdot \vec{u}_{\alpha}(0) \rangle \right),
\label{MSDgeneral}
\eneq
using the result Eq.~(\ref{mode_corr}), and (\ref{mode_mean_squre}) we thus obtain
\beq
\langle \left( \vec{r}(t)-\vec{r}(0) \right)^{2} \rangle = \langle\Delta\vec{r}(t)^2\rangle_{\text{th}}+\langle\Delta\vec{r}(t)^2\rangle_{\text{ac}},
\label{MSD_total}
\eneq
where
\begin{subequations}
\begin{align}
\begin{split}
& \langle\Delta\vec{r}(t)^2\rangle_{\text{th}} = {1\over N}\sum_{\alpha} {6k_BT\over m \omega_{\alpha}^2}\left(1-\e^{-\Gamma_{\alpha}t} \right),
\label{MSD_total_th}
 \end{split}\\
\begin{split}
& \langle\Delta\vec{r}(t)^2\rangle_{\text{ac}} = \frac{1}{N}\sum_{\alpha}\frac{2\phi f_0^2p(1-p) W_{\alpha}\Lambda_{\alpha}^2}{\Gamma_{\alpha}\left(\Gamma_{\alpha}+\tau^{-1}\right)}\left(1 + {\tau^{-1}\over \Gamma_{\alpha}-\tau^{-1}}\e^{-\Gamma_{\alpha}t} -  {\Gamma_{\alpha}\over \Gamma_{\alpha}-\tau^{-1}}\e^{-t/\tau} \right).
\end{split}\label{MSD_total_ac}
\end{align}
\end{subequations}
The first term in Eq.~(\ref{MSD_total}) is the thermal noise contribution to the MSD. As already described in Refs.\cite{granek_2005,shlomi_2012_pre,shlomi_2012_prl,granek_2011}, provided that $d_s<2$ (here, $d_s$ is the spectral dimension), it describes subdiffusion behavior, $\sim t^{\nu_{\text{th}}}$, with $\nu_{\text{th}}<1$. More precisely,
\beq
\langle\Delta\vec{r}(t)^2\rangle_{\text{th}}= B_{\text{th}}\, t^{\nu_{\text{th}}},
\label{thermal_msd}
\eneq
where the exponent $\nu_{\text{th}}=1-\frac{d_s}{2}$ and the amplitude (prefactor) 
\beq
B_{\text{th}}=6C_{\text{th}}\frac{d_s}{\omega_0^{d_s}}\frac{k_BT}{ m^{\frac{d_s}{2}}}   \gamma^{\frac{d_s}{2}-1}.
\eneq
Here, $C_{\text{th}}$ is a numerical prefactor, $C_{\text{th}}=\Gamma[{d_s\over 2}]/(2-d_s)$ in the Rouse model ($\Gamma[x]$ is the Gamma function).

An analytic evaluation of the other, active origin terms, is partly possible for a large system such that the frequency spectrum is dense. Using the DOS $g(\omega)=n_o\omega^{d_s-1}$,
where $n_o=N d_s/\omega_o^{d_s}$ is chosen such that
$\int_0^{\omega_o} d\omega g(\omega)=N$, we have
\beq
\langle\Delta\vec{r}(t)^2\rangle_{\text{ac}} = \frac{2d_s}{\omega_o^{d_s}}\phi f_0^2p(1-p) \int_{\omega_{\text{min}}}^{\omega_o} d\omega\omega^{d_s-1} \frac{ W(\omega)\Lambda(\omega)^2}{\Gamma(\omega)\left(\Gamma(\omega)+\tau^{-1}\right)}\left(1 + \frac{\tau^{-1}}{\Gamma(\omega)-\tau^{-1}}\e^{-\Gamma(\omega)t} -  \frac{\Gamma(\omega)}{\Gamma(\omega)-\tau^{-1}}\e^{-t/\tau}\right).
\label{two_point_integral}
\eneq
Note that the lower and upper integration limits in Eq.~(\ref{two_point_integral}) set the shortest 
$\tau_0=\Gamma(\omega_{0})^{-1}$ and longest relaxation times $\tau_N=\Gamma(\omega_{\text{min}})^{-1}$ where $\omega_{\text{min}}\simeq \omega_o (R_g/b)^{-d_f/d_s}$ of flexible network dynamics. 

Focusing on the intermediate time regime $\tau_0\ll t\ll \tau_N$, and times shorter than the force correlation time such that $\tau_0\ll t\ll\tau\ll \tau_N$, we find super-diffusive, ballistic-type, behavior $\langle\Delta\vec{r}(t)^2\rangle_{\text{ac}}\sim t^2$ for both force types, monopoles and dipoles. More precisely we obtain
\beq
\langle\Delta\vec{r}(t)^2\rangle_{\text{ac}}\simeq B_{\text{acs}} t^2\;\;\;\;\;~~~~~~t\ll \tau,
\eneq
where 
\beq
B_{\text{acs}} = \phi f_0^2p(1-p) {d_s\over
\omega_o^{d_s} }\int_{\omega_{\text{min}}}^{\omega_0} d\omega \; \omega^{d_s-1} {W(\omega)\Lambda(\omega)^2\tau^{-1}\over \Gamma(\omega)+\tau^{-1} }\;.
\label{two_point_integral-1}
\eneq
Setting the lower and upper limits of integration in Eq.~(\ref{two_point_integral-1}) to $0$ and infinity, respectively, and performing the integration we find, {\it provided that} $1-d_s/2-\xi/2>0$,
\beq
B_{\text{acs}} = C_{\text{acs}}\,Q\, p(1-p) \frac{d_s}{\omega_o^{d_s}} \frac{\gamma^{\frac{d_s+\xi-4}{2}}}{m^{\frac{d_s+\xi}{2}}}\phi f_0^2 \tau^{-\frac{d_s+\xi}{2}}, 
\eneq
where $C_{\text{acs}}$ is a numerical factor, $C_{\text{acs}}=\int_{0}^{\infty}d \omega \frac{\omega^{d_s+\xi-1}}{\omega^2+1}= \frac{1}{2}B\left(\frac{d_s+\xi}{2},1-\frac{d_s+\xi}{2}\right)$, and $B(x,y)$ is the Beta function \cite{gradshteyn_2014}. For force monopoles, this implies
\beq
B_{\text{acs}} = C_{\text{acs}} p(1-p) \frac{d_s}{\omega_o^{d_s}} \frac{\gamma^{\frac{d_s}{2}-2}}{m^{\frac{d_s}{2}}}\phi f_0^2 \tau^{-\frac{d_s}{2}} 
\eneq
with $C_{\text{acs}}=\frac{1}{2}B\left(\frac{d_s}{2},1-\frac{d_s}{2}\right)$, hence $B_{\text{acs}}\propto \phi f_0^2 \tau^{-d_s/2}$. In the case of force dipoles, we note that the condition $1-d_s/2-d_s/d_l>0$ is commonly {\it not} obeyed in familiar fractals; in the rare situation where it is obeyed $B_{\text{acs}}\propto \phi f_0^2\tau^{-d_s/2-d_s/d_l}$ for force dipoles. Otherwise, for $1-d_s/2-d_s/d_l<0$ the integral in Eq.~(\ref{two_point_integral-1}) diverges with the upper limit $\omega_0$, and we find
\beq
B_{\text{acs}} = \frac{d_s}{d_s+\xi-2}Q\,p(1-p)\,\omega_o^{\frac{2d_s}{d_l}-2}m^{-1}\gamma^{-1}\phi f_0^2 \tau^{-1} 
\eneq
such that $B_{\text{acs}}\propto \phi f_0^2 \tau^{-1}$. Note that the influence of $\tau$ at times $t\ll \tau$ does not contradict causality, as we describe the fluctuations at steady state where the laboratory time ($t_{tot}$, c.f. Eq.~(\ref{time_ave_msd})) has surpassed $\tau$ many times.

For times much longer than the force correlation time, $t\gg\tau$, Eq.~(\ref{two_point_integral}) can be approximated as
\beq
\langle\Delta\vec{r}(t)^2\rangle_{\text{ac}} \simeq 2\phi f_0^2p(1-p)\tau \frac{d_s}{\omega_o^{d_s}}\int_{\omega_{\text{min}}}^{\omega_0} d\omega\omega^{d_s-1}{W(\omega)\Lambda(\omega)^2\over {\Gamma(\omega) }}\left(1 - \e^{-\Gamma(\omega)t} \right)\;.
\label{two_point_integral_1}
\eneq
{\it Provided that} $1-d_s/2-\xi/2>0$, setting the lower and upper limits of integration to $0$ and infinity (respectively) and scaling of the integral leads to
\beq
\langle\Delta\vec{r}(t)^2\rangle_{\text{ac}}= B_{\text{ac}}t^{\nu_{\text{ac}}},
\eneq
where $\nu_{\text{ac}}=1-d_s/2-\xi/2$, and
\beq
B_{\text{ac}}= 2C_{\text{ac}}~Qp(1-p)\frac{d_s}{\omega_0^{d_s}}\frac{\gamma^{\frac{d_s+\xi-4}{2}}}{m^{\frac{d_s+\xi}{2}}}\phi f_0^2\tau\; . 
\eneq
Here $C_{\text{ac}}$ is a numerical constant, $C_{\text{ac}}=\int_0^{\infty}d\omega~\omega^{d_s+\xi-3}(1-\e^{-\omega^2})=\Gamma(\frac{d_s+\xi}{2})/(2-d_s-\xi)$ where $\Gamma(x)$ is Gamma function \cite{gradshteyn_2014}. Thus, for force monopoles, $\nu_{\text{ac}}=1-d_s/2$, that is identical to the thermal subdiffusion exponent, and
\beq
B_{\text{ac}}= 2C_{\text{ac}}~p(1-p)\frac{d_s}{\omega_0^{d_s}}\frac{\gamma^{\frac{d_s}{2}-2}}{m^{\frac{d_s}{2}}}\phi f_0^2\tau\; ;
\eneq
hence, the ratio of the active to thermal MSD amplitudes is
\beq
\frac{B_{\text{ac}}}{B_{\text{th}}} = p(1-p)\frac{\phi f_0^2\tau}{3k_BT \gamma}.
\eneq
For force dipoles, since as noted usually $1-d_s/2-d_s/d_l<0$, the integral in Eq.~(\ref{two_point_integral_1}) diverges with the upper bound. Therefore an anomalous subdiffusion regime is absent, and the MSD crosses over rapidly to a constant value (see below).

\vspace{0.2cm}
\noindent
{\bf S1.3. Static MSD:} For $t\gg\{\tau_N,\tau\}$ the MSD saturates at
\beq
\langle\Delta\vec{r}(\infty)^2\rangle=2\langle\vec{u}^2\rangle,
\eneq
in which the thermal and active contributions are also additive,
\beq
\langle\vec{u}^2\rangle=\langle\vec{u}^2\rangle_{\text{th}}+\langle\vec{u}^2\rangle_{\text{ac}}~.
\eneq
As previously shown for a thermal (passive) fractal network with $d_s<2$, the thermal contribution exhibits the generalized Landau-Peierls instability $\langle \vec{u}^2\rangle_{\text{th}}\sim N^{{2\over d_s}-1}$ \cite{burioni_2002,burioni_2004}. The complete expression for static thermal MSD is 
\beq
\langle\vec{u}^2\rangle_{\text{th}}= {3d_s\over 2-d_s} {k_BT\over
m\omega_o^{2}}N^{{2\over d_s}-1}.
\label{thermal-static-LP}\eneq
For the active contribution to the static MSD, we use the second term in Eq.~(\ref{mode_mean_squre}) to give the mean square amplitude
\beq
\langle\vec{u}^2\rangle_{\text{ac}}=\phi f_0^2p(1-p) \frac{d_s}{\omega_o^{d_s}} \int_{\omega_{\text{min}}}^{\omega_0} d\omega\, \omega^{d_s-1} \frac{ W_{\alpha} \Lambda_{\alpha}^2}{\Gamma_{\alpha}\left(\Gamma_{\alpha}+\tau^{-1}\right)}~.
\label{MSD-static-active}
\eneq
The integral in Eq.~(\ref{MSD-static-active}) may diverge with the lower limit $\omega_{\text{min}}\to 0$ ($N\to\infty$), where $\omega_{\text{min}}\sim \omega_o N^{-1/d_s}$, giving rise to instability that resembles the Landau-Peirels instability. This divergence occurs if $1-d_s/2-\xi/2>0$, in which case
\beq
\langle\vec{u}^2\rangle_{\text{ac}}=\left({d_s\over 2-d_s-\xi}\right) {Q\over m\omega_o^{2-\xi} \gamma }~\phi f_0^2\tau  p(1-p) N^{ {2-\xi \over d_s}-1}.
\label{MSD-static-active-1}
\eneq
Hence for force monopoles ($\xi=0$) and with $d_s<2$, this yields an {\it identical} divergence with increasing system size similar to the thermal Landau-Peierls instability,
\beq
\langle\vec{u}^2\rangle_{\text{ac}}=\left({d_s\over 2-d_s}\right) {\phi f_0^2\tau  p(1-p)\over m\omega_o^{2} \gamma }~ N^{ {2\over d_s}-1},
\label{MSD-static-active-2}
\eneq
i.e., $\langle \vec{u}^2\rangle_{\text{ac}}\sim N^{{2\over d_s}-1}$. The active noise contribution will thus prevail, provided that the force magnitude and correlation time are sufficiently large. For this (force monopoles) model, we may thus assign {\it an effective temperature}
\beq
T_{\text{eff}}=T+ {\phi f_0^2\tau p(1-p)\over 3k_B\gamma}\;,
\eneq
that can faithfully describe, by simple insertion in the thermal MSD expressions Eqs.\ (\ref{thermal_msd}) and (\ref{thermal-static-LP}), {\it both} the anomalous subdiffusion regime and the static (saturation) MSD, $2\langle\vec{u}^2\rangle$.

Turning to force dipoles since, as already mentioned in typical fractals, we mostly  have $1-d_s/2-d_s/d_l<0$, Eq.~(\ref{MSD-static-active-1}) does not apply, and $\langle\vec{u}^2\rangle_{\text{ac}}$ -- Eq.~(\ref{MSD-static-active}) -- diverges with the upper integration limit ($\omega_0$), becoming essentially independent of $N$. This implies that, surprisingly, $\langle\vec{u}^2\rangle_{\text{th}}$ {\it may dominate} the total static MSD for large enough systems due to the Landau-Peierls instability.

\vspace{0.3cm}
\noindent
{\bf S2. Numerical evaluations and Langevin dynamics simulations for a Sierpinski gasket:} 
Here we apply the general analytical theory for arbitrary fractals, Eqs.~(\ref{MSD_total_th}--\ref{MSD_total_ac}), and Langevin dynamics simulations, to the Sierpinski gasket (Fig.~\ref{fig_schematic}). We construct different generations of the Sierpinski gasket, in which the nodes (vertices) represent beads with identical masses $m$ that are connected by identical harmonic springs with spring constant $m\omega_0^2$ and equilibrium distance $b$, mimicking a bead-spring model of the fractal. We simulate the Sierpinski network dynamics under active forces, monopoles and dipoles, following Langevin dynamics simulations in the high damping limit. The set of Langevin equations of motion follow Eq.~(\ref{Langevin1}) with Hamiltonian $H=\frac{1}{2}m\omega_0^2\sum_{< ij >}(\vec{r}_i-\vec{r}_j-b \hat{r}_{ij} )^2$, where $\hat{r}_{ij}$ is the unit vector of distance $\vec{r}_i-\vec{r}_j$ between beads $i$ and $j$. For $b=0$, this interaction potential reduces to the scalar elasticity Hamiltonian (\ref{Hamiltonian_ul}) studied analytically above and in the main text. The simulations, therefore, will compare this more realistic model predictions to those used in the analytical calculation. 
% Figure environment removed
% Figure environment removed

Without changing the notations (for simplicity), we transform to dimensionless variables as follows: $\vec{r}_i\to \vec{r}_i/b$, $\vec{f}_i\to \vec{f}_i/(m\omega_0^2 b)$ and $\vec{\zeta}_i\to \vec{\zeta}_i/(m\omega_0^2 b)$, and introduce a dimensionless time $t\to t/\tau_0$ where $\tau_0=\gamma/m\omega_0^2$ is the shortest relaxation time, such that the reduced  equation of motions are
\begin{equation}
\frac{d{\vec{r}}_i(t)}{dt}=-\sum_{j\in i}\left[\vec{r}_i-\vec{r}_j- \hat{r}_{ij} \right] + \vec{\zeta}_i(t) + \vec{f}_i(t), 
\label{gle1}
\end{equation}
 with a corresponding FD theorem
\beq
\langle \vec{\zeta}_i(t)\vec{\zeta}_j(t')\rangle=\frac{2k_BT}{m\omega_0^2 b^2}\delta_{ij}\delta(t-t').
\eneq

In the simulations that follow, we have used $\frac{k_BT}{m\omega_0^2 b^2}=1$. For active force, we have taken various values of $f_0$, $\tau$, and $p=0.5$ throughout the numerical analysis. To solve the equation of motions numerically, we have used the three-values Gear predictor-corrector algorithm \cite{allen.1987}. The integration step $\Delta t$ is $0.05\tau_0$. To achieve equilibrium (before data acquisition), we first run the simulations for times much longer (two to three orders of magnitude longer) than the longest relaxation $\tau_N$. After reaching steady-state (e.g., equilibrium for the passive network), the simulations were run sufficiently long to calculate the time-averaged MSD accurately. The time-averaged MSD of an arbitrarily chosen bead (the internal bead shown in fig.~\ref{gen_9}) of the fractal network is calculated using the following familiar expression
\beq
\langle (\vec{r}(t)-\vec{r}(0))^2\rangle = 
\frac{1}{t_{tot}-t}\sum_{t_i}^{t_{tot}-t} [\vec{r}(t_i+t)-\vec{r}(t_i)]^2,
\label{time_ave_msd}
\eneq 
where $t$ is the lag time, and $t_{tot}$ is the time window of data acquisition, obeying $t_{tot}\gg t$. In all figures, presenting the analytical and simulations results; distances are in unit of $b$, forces are in unit of $m\omega_o^2 b$, and times are in unit of $\tau_0=\gamma/m\omega_o^2$. 

% Figure environment removed

% Figure environment removed
% Figure environment removed

\vspace{0.2cm}
\noindent
{\bf S2.1 Force monopoles:} Force monopole directions are randomly picked from a uniform (isotropic) distribution. Thus the total force on an infinite system ($N\to\infty$), and so the drift velocity, would vanish. For a finite network, the average total force on the fractal depends on the actual realization of the forces. If we average on a large number of realizations, the mean total force would again vanish, with a standard deviation $\sim \phi f_0 p \sqrt{N}$. Since the total friction is $N\gamma$, we estimate the typical magnitude of the drift velocity (whose direction is random) as $v_{\text{drift}}\sim \frac{\phi f_0 p}{\gamma \sqrt{N}}$. The effect of this drift velocity can be seen in Fig.~\ref{fig_mono_diff_para}(a) at long times.
To remove the effect of drift velocity, the expression for MSD is modified as following
\beq
\langle (\vec{r}(t)-\vec{r}(0))^2\rangle = \frac{1}{t_{\text{tot}}-t}\sum_{t_i}^{t_{\text{tot}}-t} (\vec{r}(t_i+t)-\vec{r}(t_i)-\langle \vec{r}(t)\rangle)^2),
\label{mono_time_ave_msd}
\eneq
where
$\langle (\vec{r}(t)\rangle = \frac{1}{t_{\text{tot}}-t}\sum_{t_i}^{t_{\text{tot}}-t} \vec{r}(t_i)=\langle v_{\text{drift}}\rangle t$ is the mean displacement due to drift velocity.

\vspace{0.2cm}
\noindent
{\bf S2.2. Application to chromatin:} 
To connect with the chromatin dynamics studies described in Ref.\cite{weber_2012}, we need to establish the parameter regime associated with chromatin. We hypothesize that the chemical distance between crosslinks in chromatin is equal to persistence length $L_p\simeq 250$ nm \cite{gero_2002,semiflexible_ghosh_2014,semiflexible_safran_2017} and that $\tau$ and $f_o$ are associated with cytoplasmic motor protein processivity times and forces. Cytoplasmic typical values are in the range $f_0=1-10 pN$, $\tau=10 ^{-3} - 10$s \cite{parameter_cui_2000,parameter_JOHNSTONE_2020,parameter_fierz_2019,parameters_zhou_2016}. To obtain the dimensionless parameter range applicable to chromatin dynamics, we assume the network segment between two beads behaves as a Gaussian 2D thermal spring with spring constant $m\omega_0^2=\frac{90k_B T L_p^2}{b^4}$ for 
semi-flexible polymers \cite{mackintosh_1995elasticity, mackintosh_2014_RPM}, and the bond length between two beads is $b=L_p$. Taking the solvent viscosity (water) $\eta=1\times10^{-3}$Pa s, we estimate normalization factors for $f_0$ and $\tau$
\beq
\text{Unit of force}= m\omega_0^2b = \frac{90k_BT}{b} = 1.5 pN,
\eneq
and 
\beq
\tau_0= \frac{\gamma}{m\omega_0^2} = \frac{3\pi\eta b^3}{90k_BT} = 4 \times 10^{-4}.
\eneq
Hence, the dimensionless value of force amplitude is in the range $f_0=0.5 - 7$, and the force correlation time is in the range $\tau=1-10000$. 

\vspace{0.2cm}
\noindent 
{\bf S2.3. Simulation with force dipoles:} 
% Figure environment removed
% Figure environment removed
Many active soft-matter systems can produce force dipoles. They emerge in a variety of biological systems such as ``actomyosin", i.e. an actin filament network in which myosin motors are dispersed \cite{drescher_2011}. These internal forces must satisfy the conditions of zero net momentum transfer and zero angular momentum transfer, such that the total force and torque vanish \cite{ramaswamy_2002_prl,ramaswamy_2010_anu_rev}. This implies that for each force acting on a network node, there is an equal and opposite force acting on another node. Thus there is no net force on the system, and so there is no translational motion of the center of mass (CM). Similarly, each force pair contributes zero torque to the system, therefore the net torque on the system vanishes too, i.e.
\begin{equation}
\sum_i \vec{T}_{i} = 0,
\end{equation}
where $\vec{T}_i = \vec{R}_i \times \vec{F}_{\text{tot},i}$ (`$\times$' stand for the vector product) is the torque acting on $i$-th bead, $\vec{R}_i$
is its position vector in the CM frame of reference, and $\vec{F}_{\text{tot},i}$ is the total force acting on it. However, according to the equation of motion, each force is balanced by the friction force, hence
\begin{equation}
\vec{F}_{tot,i} = \gamma \vec{v}_i,
\label{eq_motion_total}
\end{equation}
where $\vec{v}_i$ is the velocity of $i_{th}$ bead. Multiplying both sides of Eq.~(\ref{eq_motion_total}) by $\vec{R}_i$, we can get the torque acting on $i$-th bead,
\begin{equation}
\vec{T}_i =\vec{R}_i \times \vec{F}_{tot,i} = \gamma(\vec{R}_i \times \vec{v}_i),
\end{equation}
Using $\vec{v}_i=\vec{\Omega}_i \times \vec{R}_i$ where $\Omega_i$ is the angular velocity we obtain
\begin{equation}
 \vec{T}_i = \Gamma(\vec{R}_i \times (\vec{\Omega}_i \times \vec{R}_i)),
\end{equation}
The condition of net zero torque on the network therefore implies
\begin{equation}
\sum_i\vec{R}_i \times (\vec{\Omega}_i \times \vec{R}_i) = \sum_i \vec{\Omega}_i |\vec{R}_i|^2 = 0,
\end{equation}
which does {\it not} necessarily imply
\begin{equation}
\sum_i \vec{\Omega}_i = 0\;.
\end{equation}
Hence a net rotational motion of the whole network may exist under the action of active dipole forces. The effect of this rotational motion can be seen in the MSD as a crossover from a constant to oscillatory behavior, whose initial presentation is ballistic-like, see Figs.~\ref{fig_dipole_gen_9} and ~\ref{fig_dipole_multi_si}.
% Figure environment removed

\vspace{0.2cm}
\noindent
{\bf S2.4 Dynamical persistence length in a harmonic potential:} 
The notion of ``dynamical persistence length" emerged in active systems, in analogy to the persistence length of polymers, to describe the motion under forces that exhibit temporal correlations \cite{persistence_RMP_2016}. In the case of stochastic force that follows the random telegraph process, the correlation time $\tau$ can be thought of as ``persistent time" for which the force remains in an `on' state with magnitude $f_0$ and the velocity is kept persistent. For a free particle moving under such active force, one can define a dynamical persistence length, $x_p$, as the distance traveled in a time $\tau$, thereby $x_p=f_0\tau/\gamma$ \cite{dynamical_persistence_cao_2019, Brahmachari_2023,persistence_RMP_2016}.

In the case of a particle bound to a harmonic potential, we generalize the above notion by using the equation of motion 
\beq
\gamma\frac{dx}{dt}= -m\omega_0^2 x + f_0
\eneq
with $x(0)=0$, whose solution is
\beq
x(t) = \frac{f_0}{m\omega_0^2}\left(1-e^{-t/\tau_0}\right),
\eneq
where $\tau_0=\gamma/m\omega_0^2$. Using $t=\tau$ we obtain the desired persistence length
\beq
x_p = \frac{f_0}{m\omega_0^2}\left(1-e^{-\tau/\tau_0}\right)\;,
\eneq
which (obviously) coincides with the free-particle definition for $\tau\ll \tau_0$. The critical force above which one expects collapse of the network is obtained by equating $x_p=b$, leading to
\beq
f_{0c}\simeq \frac{m\omega_0^2 b}{\left(1-e^{-\tau/\tau_0}\right)}.
\eneq
This reduces to the familiar critical force  $f_{0c}\simeq \gamma b/\tau$ for $\tau\ll\tau_0$, and converges to $f_{0c}\simeq m\omega_0^2 b$ for $\tau\gg\tau_0$.

%\bibliography{arxiv_ref.bib}


\end{document}

