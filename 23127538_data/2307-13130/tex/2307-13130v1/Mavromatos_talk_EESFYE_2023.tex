\documentclass{svproc}
%
% RECOMMENDED %%%%%%%%%%%%%%%%%%%%%%%%%%%%%%%%%%%%%%%%%%%%%%%%%%%
%

% to typeset URLs, URIs, and DOIs
\usepackage{url}
\usepackage{amsmath,amssymb,graphicx}
\usepackage{here}
\usepackage[dvipsnames]{xcolor}

\def\UrlFont{\rmfamily}

\begin{document}

\title{String-inspired running-vacuum cosmology, quantum corrections and the current cosmological tensions}
\titlerunning{Quantum Stringy RVM and Tensions}

\author{Nick E. Mavromatos ({\it speaker})\inst{1,2} \and Joan Sol\`a Peracaula\inst{3} \and \\Adri\`a G\'omez-Valent\inst{4}}
\authorrunning{Nick E. Mavromatos et al.}
\tocauthor{Nick E. Mavromatos ({\it speaker}), Joan Sol\`a Peracaula, and Adri\`a G\'omez-Valent}


\institute{National Technical University of Athens, School of Applied Mathematical and Physical Sciences, Department of Physics, 9 Iroon Polytechniou Street, Zografou Campus GR157 80, Athens, Greece, \\
\email{mavroman@mail.ntua.gr}
\and
King's College London, Department of Physics, Strand, London WC2R 2LS, UK \\
\and
Departament de F\'\i sica Qu\`antica i Astrof\'\i sica,  and  Institute of Cosmos Sciences, Universitat de Barcelona, Av. Diagonal 647 E-08028 Barcelona, Catalonia, Spain.
\and
INFN, Sezione di Roma 2, and Dipartimento di Fisica, \\Universit\`a di Roma Tor Vergata,
via della Ricerca Scientifica 1, I-00133 Roma, Italy
}

\maketitle

\begin{abstract}
In the context of a string-inspired running vacuum model (RVM) of cosmology with anomalies and torsion-induced axion-like fields,
we discuss quantum corrections to the corresponding energy density, in approximately de Sitter eras, during which the Hubble parameter $H(t)$ varies very slowly with the  cosmic time $t$.
Such corrections arise either from graviton loops in the corresponding gravitational theory, or from path integration of massive quantum matter fields. They depend logarithmically on $H(t)$, in the form $H^n\, {\rm ln}(H^2)$, $n \in 2Z^+$. In the modern eras, for which the $n=2$ terns are dominant, such corrections may contribute to an alleviation of the currently observed cosmological $H_0$ and structure-growth tensions. In particular, we argue  that such an effect is accomplished for a (dynamically-broken) supergravity-based RVM cosmological model. In the current de-Sitter era, for this case, rather surprisingly, the quantum graviton corrections dominate those due to matter fields, provided the scale of the primordial (pre-RVM inflation) dynamical breaking of local supersymmetry lies near the reduced Planck scale, which is a natural assumption in the context of the model.
\end{abstract}

%% \tableofcontents


\section{Motivation and Summary}\label{sec:intro}

In this talk we shall review the work done in ref.~\cite{gms}, which deals with a phenomenological study of quantum corrections of string-inspired running vacuum models of cosmology.
Our main motivation  is an attempt to resolve, by modifying the standard $\Lambda$CDM Cosmology, the cosmological tensions in the current-epoch data~\cite{tensions,models}.  These are associated with either the measured value of the Hubble parameter today ($H_0$ tension),
using nearby galaxies and comparing with Cosmic Microwave Basckground (CMB) data 
using $\Lambda$CDM fits~\cite{Planck}, or growth-of-structure data.
Although there could be mundane astrophysical or statistical explanations  for these tensions~\cite{freedman}, nonetheless, their persisting nature triggered an enormous theoretical interest, as they might imply true, novel fundamental physics, deviations from the $\Lambda$CDM paradigm~\cite{models}.

One of the cosmological frameworks that offers potential, simultaneous, alleviations of both types of tensions is the running-vacuum-model (RVM) of cosmology, which was initially motivated  by semiqualitative arguments within the renormalization group ~\cite{rvm,rvm2,Solarvm2013,Sola:2015rra} and it has only recently been substantiated in full within the context of explicit quantum field theoretical  (QFT) calculations in curved spacetime\,\cite{qftrvm,qftrvm2}. For a recent review of the modern formulation of the RVM, see \cite{Solarvm2022}.  According to the RVM, the entire evolution~\cite{lima}, but also thermodynamics (and entropy production)~\cite{rvmthermal}, of the Universe can be explained in terms of a smooth running of the cosmological vacuum energy $\rho_{\rm RVM}(t)$ with the cosmic time $t$. Due to {\it general covariance}, the vacuum energy density admits a perturbative expansion on {\it even} powers of the Hubble parameter $H$:\begin{align}\label{rvmener}
\rho_{\rm RVM} (H) = \frac{\Lambda (H^2)}{\kappa^2} = \frac{3}{\kappa^2} \Big( c_0 + \nu\, H^2 + \alpha\, \frac{H^4}{H_I^2} + \dots \Big),
\end{align}
where $\kappa^2 = 8\pi \rm G = M_{\rm Pl}^{-2}$ is the four-dimensional gravitational constant (with G the Newton's constant, and $M_{\rm Pl}=2.435 \times 10^{18}~{\rm GeV}$ the reduced Planck mass), $c_0, \nu, \alpha$ are real constants (and positive in the conventional RVM framework) dimensionless coefficients (with $H_I \sim 10^{-5} \, M_{\rm Pl}$ the inflationary scale, e.g. as measured by Planck Coll.~\cite{Planck}), and the $\dots$ denote higher even powers of $H$.\footnote{Actually there are also terms containing the cosmic time derivative of $H$, $\dot H$, but
such terms can be expressed in terms of the deceleration parameter $q$, as $\dot H = -(1 + q)\, H^2$,
and, since during each cosmological epoch $q$ is approximately constant, these terms can still be expressed in terms of $H^2$ so their effects can be absorbed (in each epoch) in  the coefficient of $H^2$ in \eqref{rvmener}. In detailed phenomenological analyses of RVM, the effects of $\dot H$ can be calculated explicitly~\cite{rvmpheno,rvmstruct,rvmtensions}. }
The constant $3c_0 \kappa^{-2}$ plays the r\^ole of a contribution to the cosmological constant, and the 
combination of today's values $3\kappa^{-2}(c_0  + \nu\, H_0^2 )$
is set to the current value of the observed dark energy density~\cite{Planck}, $\sim 10^{-122} M_{\rm Pl}^4$.

It should be remarked that the entire history of the Universe can be described by the truncation of the expansion \eqref{rvmener} to $H^4$ terms~\cite{lima}, although it should be noted that quantum field theory computations in RVM spacetimes lead to $H^2$ and $H^6$ terms in an explicit way, but not $H^4$~\cite{qftrvm,qftrvm2}. The latter can be produced by condensates of primordial chiral-gravitational-wave tensor modes  in string-inspired RVM models, called from now on stringy RVM (StRVM). The low-energy
gravitational theory stemming from such string theories is characterised by torsion (the r\^ole of which is played by the field strength of the antisymmetric tensor field in the massless gravitational string multiplet) and Chern-Simons (CS) gravitational anomalies~\cite{jackiw}, as discussed in detail in \cite{bms,ms}. The StRVM cosmological evolution is characterised by different phases, in which the coefficient $\nu$ in \eqref{rvmener} is negative during the inflationary era, and positive afterwards. Moreover, in the StRVM $c_0 =0$, since a cosmological constant seems not compatible with either  perturbative (scattering S-matrix)~\cite{smatrix} or non-perturbative string theory (swampland criteria)~\cite{swamp}. In our StRVM, approximate cosmological constant terms may arise either during the early RVM inflationary phase or at late eras, as a result of the formation of appropriate condensates~\cite{bms,ms}, which are {\it metastable} though,  and
eventually disappear. Such a metastability feature of the de Sitter phase seems to be also corroborated by local quantum field theory computations under appropriate vacuum renormalization in an expanding universe within a conventional RVM framework~\cite{qftrvm,qftrvm2}.
The RVM vacuum fluid is characterised by a de-Sitter type equation of state (EoS)~\cite{rvm,rvm2,Solarvm2013,Sola:2015rra,lima}:
$p_{\rm RVM} = - \rho_{\rm RVM}$, which is essentially valid at early Universe epochs, when the vacuum gravitational degrees of freedom are dominant. This is the case of the stringy RVM at early epochs~\cite{ms}.\footnote{We also note for completeness that similar situations, with time-dependent vacuum energy satisfying a de Sitter equation of state, have appeared in the context of some non-critical string models with brane defects~\cite{emndefects}.} In radiation and matter-dominance eras, one should consider the contributions to the energy density of such excitations, $\rho_m$ ($m$=radiation, matter), on top of the above vacuum contributions, which thus lead to a total energy density of the form $\rho_{\rm total} = \rho_{\rm RVM} + \rho_m $. The equations of state of such total contributions coincide approximately with the equation of state of the respective dominant component in the corresponding era. Such a result has been confirmed by detailed quantum field theory computations in RVM spacetime backgrounds~\cite{qfteos}, under appropriate renormalization  of the vacuum energy~\cite{qftrvm,qftrvm2}.

At late eras, the RVM framework is argued~\cite{rvmpheno,tsiapi} to be phenomenologically consistent with the plethora of the cosmological data available today~\cite{Planck}, including cosmic structure formation~\cite{rvmstruct}, but also indicating observable deviations from the $\Lambda$CDM paradigm, due to the $\nu H^2$ term, dominant at the late stages of the cosmological evolution. Fitting the current-era data with the RVM evolution framework leads to the estimate  $0 < \nu = \mathcal O(10^{-3})$. Curiously enough, the same estimate for $\nu$ is obtained by requiring consistency of the RVM framework with the Big-Bang-Nucleosynthesis data~\cite{bbnrvm}, upon assuming that the observed~\cite{Planck} current-era dark energy is of RVM type.
On the other hand, the $H^4$ (and higher order) terms in \eqref{rvmener}, which are dominant during the early eras of the cosmic evolution, drive an RVM inflation without the need for fundamental inflaton fields~\cite{lima,Sola:2015rra}. Moreover, there are claims~\cite{rvmtensions} that a variant of RVM, called RVM type II, allowing for a mild phenomenological dependence of the gravitational constant on the cosmic time $t$,  $\kappa \to \kappa (t)$, can simultaneously alleviate the
$H_0$ and growth tensions observed in late-cosmology data~\cite{tensions}.
As we shall argue in this talk, the StRVM may also provide a simultaneous resolution of both types of tensions, provided quantum corrections are taken into account. Such corrections are a consequence of quantum-gravity- induced logarithmic corrections $H^2{\rm ln}(H^2)$ to the modern era vacuum energy~\cite{mavrophil,ms} and, under appropriate conditions, could dominate, even in modern eras, the corresponding ones induced by appropriately renormalised quantum matter fields in the spacetime background of an expanding universe~\cite{qftrvm,qftrvm2}. However, a more detailed future study is required in order to ascertain whether the two types of RVM contributions (QFT versus stringy or quantum-gravity motivated effects)  are comparable in magnitude and reinforce each other or point towards different directions.

The structure of the talk is the following: in the next section \ref{sec:stringyRVM}, we formulate the StRVM gravitational model and discuss how, upon condensation of primordial chiral GW modes,
one obtains inflation of RVM type, without the need for external inflaton fields. In section \ref{sec:now} we discuss how in the current era, the StRVM, upon inclusion of appropriate quantum corrections, could contribute to a possible alleviation of the current-era $H_0$ tension, as well as tensions associated with galactic growth parameters \cite{gms}.

\section{Stringy Running Vacuum Model of Cosmology}\label{sec:stringyRVM}

The string-inspired cosmology model (StRVM) of \cite{bms,ms} is based on the bosonic part of the massless gravitational multiplet of the (closed sector of the) underlying microscopic superstring theory, which constitutes also the ground state~\cite{string}. This part consists of the fields of graviton (spin-2 symmetric tensor), dilaton (scalar, spin-0 field) and  spin-1 antisymmetric tensor (or Kalb-Ramond (KR)) gauge field, $B_{\mu\nu}=-B_{\nu\mu}$. In (3+1)-dimensions, after string compactification, which we restrict ourselves for the purposes of this talk, the dual to the field strength of the KR field,
$\mathcal H_{\mu\nu\rho}=  \kappa^{-1}\,  \partial_{[\mu}\,B_{\nu\rho]}$ (where
the $[\dots]$ denotes total antisymmetrization of the respective indices), is an axion-like particle, the so-called string-model independent axion~\cite{kaloper,svrcek}. The latter is essentially a Lagrange multiplier field implementing in the path integral a Bianchi identity constraint satisfied  by $\mathcal H_{\mu\nu\rho}$ after its modification to include Green-Schwarz anomaly cancellation terms~\cite{gs}:
\begin{equation}\label{modbianchi2}
 \varepsilon_{abc}^{\;\;\;\;\;\mu}\, {\mathcal H}^{abc}_{\;\;\;\;\;\; ;\mu}
 -  \frac{\alpha^\prime}{32\, \kappa} \, \sqrt{-g}\, R_{\mu\nu\rho\sigma}\, \widetilde R^{\mu\nu\rho\sigma}   =0,
\end{equation}
where where $\alpha^\prime = M_s^{-2}$ is the Regge slope of the string, with $M_s$ the string-mass scale, which is a free parameter in string theory, with $\sqrt{\alpha^\prime} \ne \kappa$ in general.
 The semicolon ; denotes covariant derivative with respect to the standard
Christoffel connection, $R_{\mu\nu\rho\sigma}$ denotes the Riemann tensor with respect to that connection,\footnote{Our conventions and definitions used throughout this work are those of \cite{bms}, that is: signature of metric $(+, -,-,- )$, Riemann Curvature tensor
$R^\lambda_{\,\,\,\,\mu \nu \sigma} = \partial_\nu \, \Gamma^\lambda_{\,\,\mu\sigma} + \Gamma^\rho_{\,\, \mu\sigma} \, \Gamma^\lambda_{\,\, \rho\nu} - (\nu \leftrightarrow \sigma)$, Ricci tensor $R_{\mu\nu} = R^\lambda_{\,\,\,\,\mu \lambda \nu}$, and Ricci scalar $R = R_{\mu\nu}g^{\mu\nu}$.} with $\widetilde R_{\mu\nu\rho\sigma} = \frac{1}{2} \varepsilon_{\mu\nu\lambda\pi} R_{\,\,\,\,\,\,\,\rho\sigma}$ the dual Riemann tensor, and
$\varepsilon_{\mu\nu\rho\sigma} = \sqrt{-g}\,  \epsilon_{\mu\nu\rho\sigma}$, $\varepsilon^{\mu\nu\rho\sigma} =\frac{{\rm sgn}(g)}{\sqrt{-g}}\,  \epsilon^{\mu\nu\rho\sigma}$,
$\epsilon^{0123} = +1$, {\emph etc.},  the gravitationally covariant Levi-Civita tensor densities, totally antisymmetric in their indices.

 The reader should note that the second term in left-hand side of \eqref{modbianchi2} is proportional to the gravitational anomaly~\cite{Eguchi,alvwitt}, which is an exact one-loop result, due to circulation of chiral degrees of freedom in quantum loops.

Ignoring the dilaton, which self-consistently can be set to zero~\cite{bms}, and path-integrating out the $\mathcal H$ field in the string effective action~\cite{string}, which, to $\mathcal O(\alpha^\prime)$ (quadratic order in a derivative expansion) we restrict ourselves here, behaves as a non-propagating field, one obtains the following action~\cite{kaloper,svrcek}
\begin{align}\label{sea3}
S^{\rm eff}_B =\; \int d^{4}x\sqrt{-g}\Big[ -\frac{1}{2\kappa^{2}} R + \frac{1}{2}\, \partial_\mu b \, \partial^\mu b +  \sqrt{\frac{2}{3}}\frac{\alpha^\prime}{96\, \kappa} b(x) R_{\mu\nu\rho\sigma}\, \widetilde R^{\mu\nu\rho\sigma} + \dots \Big],
\end{align}
where the dots $\dots$ denote higher-derivative, terms appearing in the string effective action, that we ignore for our discussion here. The field $b(x)$ is the pseudoscalar Lagrange multiplier which implements the Bianchi constraint \eqref{modbianchi2} in the path integral of the string-inspired (low-energy) effective gravitational theory. It is called gravitational, or KR, or, in modern string-theory language, string-model independent axion~\cite{svrcek}, as it characterises all string theories, independent of their types of compactified geometries.
Because to $\mathcal O(\alpha^\prime)$, one can view $\kappa^{-1} \, \mathcal H_{\mu\nu\rho}$ as a totally antisymmetric contorsion tensor, in the sense that the quadratic $\mathcal H_{\mu\nu\rho}\, \mathcal H^{\mu\nu\rho}$ in the effective low-energy gravitational lagrangian stemming from strings~\cite{string} can be absorbed in a generalised curvature with torsion~\cite{torsion}, one may say that the KR axion in this model has a geometric origin~\cite{mavrophil}.
 In string theory, there are of course many more axions, arising from compactification, which may lead to a rich phenomenology~\cite{arvanitaki} and cosmology~\cite{marsh}.
The model \eqref{sea3} constitutes a CS modification of general relativity (GR)~\cite{jackiw}.
and in general appears in theories with torsion, such as Einstein-Cartan quantum electrodynamics~\cite{kaloper,mavrophil}, with the $b$ field playing the r\^ole of the dual of the totally antisymmetric part of the torsion, as in the string theory case.

In the cosmology model of \cite{bms,ms} it is further assumed that there was a very early phase in the primordial Universe evolution, after the Big Bang, during which the spin 3/2 supersymmetric partners of gravitons (which exist in the superstring case), the gravitinos, $\psi_\mu$, could condense to scalar condensates $\sigma=\langle \overline{\psi}_\mu \, \psi^\mu \rangle \ne 0$, thus acquiring masses close to the Planck scale, and therefore breaking local supersymmetry (supergravity) dynamically~\cite{houston}. In such a scenario, there could be a primordial first hill-top inflation~\cite{ellisinfl}, which is not necessarily slow roll, as it does not lead to observable consequences other than  homogeneity and isotropy, which can then be used to describe quantitatively the transition, via tunnelling~\cite{ms}, of the isotropic and homogeneous system of massless gravitons and KR axions to the RVM second inflationary phase~\cite{ms}, that we now proceed to discuss.

After the dynamical local supersymmetry breaking, and the exit from the first hill-top inflation, gravitational waves can form, as a consequence of a lifting of the degeneracy of the vacua of the double-well potential of the gravitino condensate due to
dynamical percolation effects in the early Universe, which result in asymmetric occupation numbers of the two vacua~\cite{ross}. Such a lifting may lead to the formation of domain walls, whose asymmetric collapse and/or collisions can lead to the formation of primordial {\it chiral} (left-right asymmetric)  gravitational waves (GW). The latter can then condense, leading in turn to non-trivial condensates of the CS gravitational anomaly terms.

If we assume $\mathcal N(t)$ sources of such chiral primordial GW~\cite{mavlorentz}, then it can be shown, following~\cite{lyth}, that the GW induced CS condensate can be estimated as (assuming initially an approximately constant $H$, {\it i.e.} inflation, which we shall argue later that it is a consistent solution of the pertinent cosmic evolution):
 \begin{align}\label{condensateN2}
\langle R_{\mu\nu\rho\sigma} \, \widetilde R^{\mu\nu\rho\sigma} \rangle
=\frac{\mathcal N(t)}{\sqrt{-g}}  \, \frac{1.1}{\pi^2} \,
\Big(\frac{H}{M_{\rm Pl}}\Big)^3 \, \mu^4\, \frac{\dot b(t)}{M_s^{2}} \equiv n_\star \, \frac{1.1}{\pi^2} \,
\Big(\frac{H}{M_{\rm Pl}}\Big)^3 \, \mu^4\, \frac{\dot b(t)}{M_s^{2}}~,
\end{align}
where the overdot denotes derivative with respect to the cosmic time, and $n_\star \equiv \mathcal N(t)/\sqrt{-g}$ is the proper density of sources of GW, which we may assume to be approximately constant during the RVM inflation for simplicity and concreteness. The quantity $\mu$ is the UV cutoff of the effective low-energy theory, which serves as an upper bound for the momenta of the graviton modes that are integrated over in the computation of the condensate \eqref{condensateN2} in  the presence of chiral GW. To arrive at \eqref{condensateN2}, we have assumed isotropy and homogeneity of the string Universe, which can be guaranteed by the first hill-top inflation in the pre-RVM-inflationary scenario of \cite{ms}, as discussed above.\footnote{It should be stressed that the
estimate \eqref{condensateN2} is a valid estimate only in the field theory low-energy limit of the corresponding string theory. Unfortunately, as the estimate relies on the dominant UV physics near the cutoff $\mu$, in the case of microscopic string theory models it is the entire towers of massive string states that contribute, which makes an accurate estimation of the CS anomaly condensate not possible at present, apart from ensuring its non vanishing value.}

The explicit computations of \cite{bms,ms,mavlorentz} have shown that the total vacuum energy density during inflation assumes a RVM form \eqref{rvmener}:
\begin{align}\label{totalenerden}
\rho^{\rm total}_{\rm vac} =  -\frac{1}{2}\, \epsilon \, M_{\rm Pl}^2\, H^2 + 4.3 \times 10^{10} \, \sqrt{\epsilon}\,
\frac{|\overline b(0)|}{M_{\rm Pl}} \, H^4\,,
\end{align}
where we have used the following parametrisation of the approximately constant $\dot b$ axion background during inflation~\cite{bms}:
\begin{align}\label{axionbackgr}
b(t)=\overline{b}(t_0) + \sqrt{2\epsilon} \, H \, (t-t_0) \, M_{\rm Pl}\,,  \quad \overline{b}(t_0) < 0\,,
\end{align}
where $0 < \epsilon < 1 $ is a phenomenological parameter, and the time $t_0$ corresponds to the onset of the RVM inflation, implying that $\overline b(t_0)$ plays the r\^ole of a boundary condition for the KR axion.  In arriving at the estimates of \eqref{totalenerden} we took into account the fact that the assumption on the approximate de Sitter (positive-cosmological-constant) nature of the condensate, so as to lead to an approximately constant $H \simeq H_I$ during the RVM inflation, requires~\cite{bms,ms}:
$|\overline{b}(t_0)| \ge N_e \, \sqrt{2\epsilon} \, M_{\rm Pl}\, = \mathcal O(10^2)\,\sqrt{\epsilon}\, M_{\rm Pl}$, with $N_e= \mathcal O(60-70)$ the number of e-foldings of the RVM inflation.
The EoS of the string-inspired model during this inflationary phase has been explicitly computed in
\cite{ms} and found to coincide with the RVM EoS~\cite{lima,Sola:2015rra}, which justifies {\it a posteriori} calling this model {\it Stringy RVM}.  Because of the RVM form of the vacuum energy density \eqref{totalenerden}, we observe that it is the fourth power $H^4$ of the Hubble parameter, being the dominant one in the early Universe, that drives inflation in this model, which thus is of RVM form, not requiring external inflaton fields for its realization. Notice that any even power of $H$ beyond $2$ would  trigger inflation through that mechanism.  However, in the works \cite{rvmthermal} the structure \eqref{rvmener} (or the generalized one with $H^{2n},  n > 2)$ was assumed phenomenologically only, until such form got substantiated  in the modern QFT formulation of the  RVM  presented in \cite{qftrvm,qftrvm2}, which encompasses the entire cosmic evolution. On the other hand, the explicit $H^4$ power in \eqref{totalenerden} is genuine in the stringy approach~\cite{bms,ms}, and can be accounted for in it. This justifies {\it a posteriori} our analysis on arriving at the RVM inflation from GW-induced anomaly condensates in the StRVM.

From \eqref{condensateN2} and the form of the action \eqref{sea3}, the reader should observe that the condensates breaks the shift symmetry of the axion $b(x)$, leading to a linear potential for this field, of the form encountered in string theory, due to completely different reasons~\cite{silver}. When string-world-sheet effects are taken into account, which lead to periodic modulations of the shift-symmetry-breaking potentials not only of the KR axion $b$, but also of the compactification axions~\cite{svrcek},
one can under appropriate circumstances, explained in \cite{stamou}, arrive at a situation in which
the density of primordial black holes (pBH) produced during the RVM inflation in the StRVM is significantly enhanced, This affects the spectrum of GW in the radiation era, but also leads to significant fractions of such pBH playing the r\^ole of components of dark matter~\cite{pBH}.






\section{Modern Eras: Quantum corrections in Stringy RVM and cosmological tensions}\label{sec:now}

A detailed scenario for the post inflationary eras of the stringy RVM has been described in \cite{bms,ms}, where we refer the interested readers. We only mention here that at the end of the RVM inflation, the decay of the metastable vacuum leads to the generation of chiral fermionic matter and radiation. The chiral matter generates its own gravitational CS terms, but also chiral global anomalies, associated with the gauge sector. The matter-generated gravitational anomalies  {\it cancel} the primordial ones due to the Green-Schwarz mechanism, but the chiral anomalies remain. It is possible then~\cite{bms,ms}, that during the cosmic epoch corresponding to energy scales that correspond to dominance of Quantum Chromodynamics (QCD) effects, instantons of the colour SU$_{\rm c}$(3) gauge group generate periodic potentials, and thus masses, for the KR (and also the other) axions in string models, which could thus play the r\^ole of components of dark matter.

We now remark that, in the context of our stringy RVM model~\cite{ms,mavrophil},
upon making explicit use of methods developed for the prototype primordial supergravity model~\cite{bmssugra,houston}, which characterises the early pre-RVM inflation epochs of StRVM,
and
integrating out graviton quantum fluctuations  in the path integral around approximately (late-epoch) de Sitter cosmological space-times, one obtains logarithmic one-loop corrections to the energy density
of the form
\begin{align}\label{qglncorr}
\delta \rho_{\rm RVM}^{\rm QG}  \propto d_1\, H^2\, {\rm ln}(H^2)\,,
\end{align}
where $H(t)$ is the mildly depending on the cosmic time $t$ Hubble parameter in the approximately de Sitter current epoch, and the constant coefficients $d_1$:
\begin{align}\label{cs}
 d_1 \propto \kappa^2 \mathcal E_0, \,\, {\rm or} \,\,\, d_1 \propto \kappa^2 \mathcal E_0 \, {\rm ln}(\kappa^4 |\mathcal E_0|)\,,
\end{align}
with $|\mathcal E_0|$ a bare (constant) vacuum energy density scale. In supergravity models $\mathcal E_0 < 0$, but in the absence of supersymmetry, as is the case of modern eras we are interested in, $\mathcal E_0$ could be positive. This will play an important r\^ole for the generic phenomenological analysis of the stringy RVM at late epochs~\cite{gms}.

Such logarithmic corrections to the vacuum energy density {\it also} appear in quantum field theories in Robertson-Walker space times, under appropriate renormalization group treatments~\cite{qftrvm,qftrvm2}.
Such matter-generated effects are expected to be present in the current era, since there is still appreciable matter content today ($\Omega_m \simeq 0.26$). It is therefore plausible that they have appreciable effects in
alleviating the cosmological tensions. Naively, one would expect that such matter effects dominate over the quantum gravity effects.
However, upon closer examination and comparison with \eqref{cs}, we see that this is highly dependent on the magnitude of $\mathcal E_0$. For concreteness let us consider the case of
scalar real matter fields of mass $m$, non-minimally coupled to gravity with a parameter $\xi \in  R$~\cite{qftrvm,qftrvm2},
with the conformal theory corresponding to $\xi=1/6$. In this case one obtains for the renormalised current era energy density in an expanding universe  spacetime with Hubble parameter $H(t)$:
\begin{align}\label{qftsc}
\rho_{\rm RVM}^{\rm vac~QFT}  (H) &= \rho_{\rm RVM}^0 + \frac{3\, \nu_{\rm eff} (H) }{\kappa^2} \, \Big(H^2 - H_0^2 \Big), \nonumber \\
\nu_{\rm eff} (H) &\simeq \frac{1}{2\pi} \Big(\xi - \frac{1}{6}\Big) \, (\kappa^2 \, m^2)\,  \, {\rm ln}\Big(\frac{m^2}{H^2}\Big)  \,
\end{align}
The quantity $\rho_{\rm RVM}	^0 $ is the standard current-era RVM energy density, which we associate with the measured value of the cosmological constant through the relation  $\Lambda=8\pi G \rho_{\rm RVM}	^0$~\cite{Solarvm2022}.
The result \eqref{qftsc}, is based on the assumption~\cite{qftrvm,qftrvm2} $|{\rm ln}(m^2/H^2)| \gg 1$. Qualitatively similar expressions characterise the fermion case~\cite{qftrvm2}. However, the reader should notice that the corrections \eqref{qftsc} differ from the quantum graviton logarithmic corrections to the energy density, in that the logarithms
appear in the combination
$\delta \rho_{\rm RVM}^{\rm QFT} \propto (H^2-H_0^2)\,(\kappa^2 \, m^2){\rm ln}\Big(\frac{m^2}{H^2}\Big)$,
which is small in the modern era for which $H^2 \to H_0^2$. Unlike their quantum-gravity counterparts, such corrections cannot be cast in the form $R\,{\rm ln}(\kappa^2 R)$, with $R \sim 12H^2$ the de Sitter Ricci scalar. Comparing $\delta \rho_{\rm RVM}^{\rm QFT}$ with \eqref{qglncorr}, \eqref{cs}, we observe that, depending on the magnitude of the bare cosmological constant $\mathcal E_0$, matter effects could be suppressed compared to their quantum-graviton counterparts, given the smallness of the factor $(\kappa^2 \, m^2)\, (H^2 - H_0^2) \ll (\kappa^2 \, m^2)\, H^2 $ at modern epochs for which $H \to H_0$. Thus, even in modern eras, a non negligible possibility exists that quantum gravity corrections to the energy density of the stringy RVM dominate over quantum-matter-induced effects.
Making the latter assumption, we remark that such corrections can be best parametrised by an effective gravitational model with effective action~\cite{mavrophil,gms}:
\begin{equation}\label{eq:action}
S=-\int d^4x\,\sqrt{-g}\,\left[c_0+ R\, \left(c_1+c_2\ln\left(\frac{R}{R_0}\right)\right)\right]+S_m\,.
\end{equation}
where $R_0=12\, H_0^2$ is the de Sitter curvature scalar today, and $S_m$ is a matter/radiation action. The parameter $c_1 = 1/(2\kappa^2) + \delta c_1$, with $\delta c_1$ describing quantum corrections to the Newton's constant~\cite{gms}. Unitarity of the gravity sector requires $c_1 > 0$ in our conventions
% Figure environment removed
A full phenomenological analysis of this class of models, including fits to all available cosmological data at present (supernovae, CMB, Baryon Acoustic Oscillations and cosmic chronometers) is presented in ref.~\cite{gms}. An interesting case arises when one makes specific use of the broken-supergravity
phase of the StRVM.
On assuming that the supergravity contributions, corresponding to a primordial supersymmetry breaking scale $\sqrt{|f|}$, which contributes to the bare cosmological constant a {\it negative} value $-f^2 < 0$ (as required by supersymmetry~\cite{houston}), are the dominant ones of all the quantum-graviton-induced logarithmic corrections to the energy density from primordial to the current era,
so that the bare cosmological constant scale $\mathcal E_0$ appearing in \eqref{cs} coincides with $-\sqrt{f}< 0$,
we obtain~\cite{ms,mavrophil,gms}: 
\begin{align}\label{c12}
c_1 - c_2\, {\rm ln}(\kappa^2\, H_0^2) &=
\frac{1}{2\kappa^2} \Big[1 + \frac{1}{2}\, \kappa^4 \, f^2 \,\Big(0.083 - 0.049 \, {\rm ln}(3\kappa^4\, f^2)\Big)\Big]\, , \nonumber \\
c_2 &= - 0.0045\, \kappa^2 \, f^2 < 0\,.
\end{align}
We then constrain the constants $c_1,c_2$ from data~\cite{gms}, in order to alleviate the
$H_0$ and growth tensions, with the situation being summarised
in figure \ref{fig:strvmH0}, for the values of the dimensionless parameters $|\frac{c_2}{c_1 + c_2}| =
9 \times 10^{-3} \, \kappa^4 \, f^2 \lesssim \mathcal O(10^{-7})$ and $2\kappa^2\, (c_1 + c_2) = 0.924 \pm 0.017$.
This yields the following estimate on the magnitude of $\sqrt{|f|}$, $\sqrt{|f|} \kappa \sim 10^{-5/4} \lesssim 1$, implying a subplanckian scale for primordial dynamical supergravity breaking, consistent with the transplanckian conjecture.
As follows from \eqref{c12}, such values are also consistent with the perturbative modifications of $c_1$ from the (3+1)-dimensional gravitational constant $1/(2\kappa^2)$, despite the fact that $\kappa^2 H_0^2 = \mathcal O(10^{-122}) \ll 1$. With such scales one can also see~\cite{mavrophil,ms} that the quantum graviton corrections during the RVM inflationary eras are subleading as compared to the $H^4$ terms in the vacuum energy density \eqref{totalenerden} induced by GW condensates, and thus they do not affect our mechanism for inflation discussed in the previous section and in \cite{bms,ms}.




\section*{Acknowledgements} NEM would like to thank the organisers of
the 40th Conference on Recent Developments in High Energy Physics and Cosmology (HEP2023, 5-7 April 2023) of
the Hellenic High Energy Physics Society in the U. of Ioannina (Greece) for their kind invitation to give this plenary talk.
The work of NEM is supported in part by the UK Science and Technology Facilities research Council (STFC) under the research grants ST/T000759/1 and ST/X000753/1. The
work of JSP is funded in part by the projects PID2019-105614GB- C21, FPA2016-76005-C2-1-P (MINECO, Spain), 2021-SGR-249 (Generalitat de Catalunya) and CEX2019-000918-M (ICCUB, Barcelona). The work of AGV is funded by the Istituto Nazionale di Fisica Nucleare (INFN) through the project of the InDark INFN Special Initiative: “Dark Energy and Modified Gravity Models in the light of Low-Redshift Observations” (n. 22425/2020).
N.E.M. and J.S.P also acknowledge participation in the COST Association Action CA18108 ``{\it Quantum Gravity Phenomenology in the Multimessenger Approach (QG-MM)}''.
AGV and JSP take part in the COST Association Action CA21136 “Addressing observational tensions in cosmology with systematics and fundamental physics (CosmoVerse)”.


\begin{thebibliography}{99}

%\cite{Gomez-Valent:2023hov}
\bibitem{gms}
A.~G\'omez-Valent, N.~E.~Mavromatos and J.~Sol\`a Peracaula,
%``Stringy Running Vacuum Model and current Tensions in Cosmology,''
[arXiv:2305.15774 [gr-qc]].
%2 citations counted in INSPIRE as of 17 Jul 2023


%\cite{Verde:2019ivm}
\bibitem{tensions}
L.~Verde, T.~Treu and A.~G.~Riess,
%``Tensions between the Early and the Late Universe,''
Nature Astron. \textbf{3}, 891
%doi:10.1038/s41550-019-0902-0
[arXiv:1907.10625 [astro-ph.CO]].
%753 citations counted in INSPIRE as of 25 Mar 2023
%\cite{Perivolaropoulos:2021jda}
L.~Perivolaropoulos and F.~Skara,
%``Challenges for \ensuremath{\Lambda}CDM: An update,''
New Astron. Rev. \textbf{95} (2022), 101659.
%doi:10.1016/j.newar.2022.101659
%[arXiv:2105.05208 [astro-ph.CO]].
%308 citations counted in INSPIRE as of 25 Mar 2023

%\cite{Abdalla:2022yfr}
\bibitem{models}
E.~Abdalla, \textit{et al.}
%G.~Franco Abell\'an, A.~Aboubrahim, A.~Agnello, O.~Akarsu, Y.~Akrami, G.~Alestas, D.~Aloni, L.~Amendola and L.~A.~Anchordoqui,
%``Cosmology intertwined: A review of the particle physics, astrophysics, and cosmology associated with the cosmological tensions and anomalies,''
JHEAp \textbf{34} (2022), 49-211
%doi:10.1016/j.jheap.2022.04.002
%[arXiv:2203.06142 [astro-ph.CO]].
%281 citations counted in INSPIRE as of 25 Mar 2023
%\cite{DiValentino:2019jae}
E.~Di Valentino, A.~Melchiorri, O.~Mena and S.~Vagnozzi,
%``Nonminimal dark sector physics and cosmological tensions,''
Phys. Rev. D \textbf{101} (2020) no.6, 063502.
%doi:10.1103/PhysRevD.101.063502
%[arXiv:1910.09853 [astro-ph.CO]].
%217 citations counted in INSPIRE as of 25 Mar 2023



%\cite{Planck:2018vyg}
\bibitem{Planck}
N.~Aghanim \textit{et al.} [Planck],
%``Planck 2018 results. VI. Cosmological parameters,''
Astron. Astrophys. \textbf{641} (2020), A6
[erratum: Astron. Astrophys. \textbf{652} (2021), C4].
%doi:10.1051/0004-6361/201833910
%[arXiv:1807.06209 [astro-ph.CO]].
%10129 citations counted in INSPIRE as of 25 Mar 2023

%\cite{Freedman:2017yms}
\bibitem{freedman}
W.~L.~Freedman,
%``Cosmology at a Crossroads,''
Nature Astron. \textbf{1} (2017), 0121.
%doi:10.1038/s41550-017-0121
%[arXiv:1706.02739 [astro-ph.CO]].
%217 citations counted in INSPIRE as of 25 Mar 2023


%\cite{Shapiro:2000dz}
\bibitem{rvm}
I.~L.~Shapiro and J.~Sol\`a,
%``Scaling behavior of the cosmological constant: Interface between quantum field theory and cosmology,''
JHEP \textbf{02} (2002), 006;
%doi:10.1088/1126-6708/2002/02/006
%[arXiv:hep-th/0012227 [hep-th]];
%313 citations counted in INSPIRE as of 25 Mar 2023
 %``On the possible running of the cosmological 'constant',''
Phys. Lett. B \textbf{682} (2009), 105-113.
%doi:10.1016/j.physletb.2009.10.073
%[arXiv:0910.4925 [hep-th]].
%225 citations counted in INSPIRE as of 25 Mar 2023

\bibitem{rvm2}
J.~Sol\`a, 
%Dark energy: a quantum fossil from the inflationary universe?
J.Phys.A {\bf 41} (2008) 164066.
%arXiv:0710.4151 [hep-th],
 %doi:10.1088/1751-8113/41/16/164066.

\bibitem{Solarvm2013}
J.~Sol\`a,
%``Cosmological constant and vacuum energy: old and new ideas''
J.Phys.Conf.Ser. {\bf 453} (2013) 012015;
%arXiv:1306.1527 [gr-qc], doi:10.1088/1742-6596/453/1/012015
AIP Conf.Proc. 1606 (2015) 1, 19-37.
%arXiv: 1402.7049 [gr-qc].


%Sola:2013gha

\bibitem{Sola:2015rra}
%The ΛˉCDMΛˉCDM cosmology: From inflation to dark energy through running Λ
J.~Sol\`a and A. G\' omez-Valent, Int.J.Mod.Phys.D{\bf 24} (2015) 1541003. 
%[arXiv: 1501.03832 [gr-qc]], doi:10.1142/S0218271815410035.

\bibitem{qftrvm}
C.~Moreno-Pulido and J.~Sol\`a Peracaula,
%``Renormalizing the vacuum energy in cosmological spacetime: implications for the cosmological constant problem,''
{\it ibid} \textbf{82} (2022) no.6, 551.
%[arXiv:2201.05827 [gr-qc]], doi:10.1140/epjc/s10052-022-10484-w;
%Equation of state of the running vacuum
Eur.Phys.J.C {\bf 82} (2022) 12, 1137  
% [arXiv:2207.07111 [gr-qc]], doi:10.1140/epjc/s10052-022-11117-y;
%``Running vacuum in quantum field theory in curved spacetime: renormalizing $\rho_{vac}$ without $\sim m^4$ terms,''
{\it ibid.} \textbf{80} (2020) no.8, 692.
%doi:10.1140/epjc/s10052-020-8238-6
%[arXiv:2005.03164 [gr-qc]].
%50 citations counted in INSPIRE as of 25 Mar 2023

\bibitem{qftrvm2} %\cite{Moreno-Pulido:2023ryo}
C.~Moreno-Pulido, J.~Sol\`a Peracaula and S.~Cheraghchi,
%``Running vacuum in QFT in FLRW spacetime: The dynamics of $\rho_{\rm vac}(H)$ from the quantized matter fields,''
Eur.Phys.J.C {\bf 83} (2023) 637. 
 %[2301.05205 [gr-qc]], doi:10.1140/epjc/s10052-023-11772-9.

\bibitem{Solarvm2022}
J.~Sol\`a Peracaula,
%``The cosmological constant problem and running vacuum in the expanding universe,''
Phil. Trans. Roy. Soc. Lond. A \textbf{380} (2022), 20210182. 
%[arXiv:2203.13757 [gr-qc]]
%doi:10.1098/rsta.2021.0182.
%\cite{SolaPeracaula:2022hpd}

%\cite{Perico:2013mna}
\bibitem{lima}
E.~L.~D.~Perico, J.~A.~S.~Lima, S.~Basilakos and J.~Sol\`a,
%``Complete Cosmic History with a dynamical $\Lambda=\Lambda(H)$ term,''
Phys. Rev. D \textbf{88} (2013), 063531;
%doi:10.1103/PhysRevD.88.063531
%[arXiv:1306.0591 [astro-ph.CO]].
%136 citations counted in INSPIRE as of 25 Mar 2023
%\cite{Lima:2013dmf}
J.~A.~S.~Lima, S.~Basilakos and J.~Sol\`a,
%``Expansion History with Decaying Vacuum: A Complete Cosmological Scenario,''
Mon. Not. Roy. Astron. Soc. \textbf{431} (2013), 923-929.
%doi:10.1093/mnras/stt220
%[arXiv:1209.2802 [gr-qc]].
%141 citations counted in INSPIRE as of 25 Mar 2023

\bibitem{rvmthermal} %\cite{Lima:2015mca}
J.~A.~S.~Lima, S.~Basilakos and J.~Sol\`a,
%``Thermodynamical aspects of running vacuum models,''
Eur. Phys. J. C \textbf{76} (2016) no.4, 228;
%doi:10.1140/epjc/s10052-016-4060-6
%[arXiv:1509.00163 [gr-qc]];
%41 citations counted in INSPIRE as of 25 Mar 2023
%\cite{Lima:2014hia}
%``Nonsingular Decaying Vacuum Cosmology and Entropy Production,''
Gen. Rel. Grav. \textbf{47} (2015), 40;
%doi:10.1007/s10714-015-1888-2
%[arXiv:1412.5196 [gr-qc]].
%46 citations counted in INSPIRE as of 25 Mar 2023
%\cite{SolaPeracaula:2019kfm}
J.~Sol\`a Peracaula and H.~Yu,
%``Particle and entropy production in the Running Vacuum Universe,''
Gen. Rel. Grav. \textbf{52} (2020) no.2, 17.
%doi:10.1007/s10714-020-2657-4
%[arXiv:1910.01638 [gr-qc]].
%35 citations counted in INSPIRE as of 25 Mar 2023

%\cite{SolaPeracaula:2018wwm}
\bibitem{rvmpheno}
A.~G\'omez-Valent, J.~Sol\`a   and S. Basilakos,
%``Dynamical vacuum energy in the expanding Universe confronted with observations: a dedicated study,''
JCAP {\bf 01} (2015) 004. 
%[arXiv: 1409.7048 [astro-ph.CO]], doi:10.1088/1475-7516/2015/01/004.
%60 citations counted in INSPIRE as of 25 Mar 2023

%\cite{Gomez-Valent:2014fda}
\bibitem{rvmstruct}
A.~G\'omez-Valent and J.~Sol\`a,
%``Vacuum models with a linear and a quadratic term in H: structure formation and number counts analysis,''
Mon. Not. Roy. Astron. Soc. \textbf{448} (2015), 2810-2821;
%doi:10.1093/mnras/stv209
%[arXiv:1412.3785 [astro-ph.CO]].
%72 citations counted in INSPIRE as of 25 Mar 2023
%\cite{Gomez-Valent:2015pia}
A.~G\'omez-Valent, E.~Karimkhani and J.~Sol\`a,
%``Background history and cosmic perturbations for a general system of self-conserved dynamical dark energy and matter,''
JCAP \textbf{12} (2015), 048.
%doi:10.1088/1475-7516/2015/12/048
%[arXiv:1509.03298 [gr-qc]].
%45 citations counted in INSPIRE as of 25 Mar 2023

%\cite{SolaPeracaula:2021gxi}
\bibitem{rvmtensions}
J.~Sol\`a Peracaula, A.~G\'omez-Valent, J.~de Cruz Perez and C.~Moreno-Pulido,
%``Running vacuum against the $H_0$ and $\sigma_8$ tensions,''
EPL \textbf{134} (2021) no.1, 19001;
%[arXiv:2102.12758 [astro-ph.CO]],
%doi:10.1209/0295-5075/134/19001;
%Running Vacuum in the Universe: Phenomenological Status in Light of the Latest Observations, and Its Impact on the σ88​ and H00​ Tensions
 Universe {\bf 9} (2023) 6, 262. 
 %[arXiv:2304.11157 [astro-ph.CO]], doi:10.3390/universe9060262.
%48 citations counted in INSPIRE as of 25 Mar 2023
 
%\cite{Jackiw:2003pm}
\bibitem{jackiw}
R.~Jackiw and S.~Y.~Pi,
%``Chern-Simons modification of general relativity,''
Phys. Rev. D \textbf{68} (2003), 104012;
%doi:10.1103/PhysRevD.68.104012
%[arXiv:gr-qc/0308071 [gr-qc]].
%585 citations counted in INSPIRE as of 25 Mar 2023
%\cite{Alexander:2009tp}
S.~Alexander and N.~Yunes,
%``Chern-Simons Modified General Relativity,''
Phys. Rept. \textbf{480} (2009), 1-55.
%doi:10.1016/j.physrep.2009.07.002
%[arXiv:0907.2562 [hep-th]].
%521 citations counted in INSPIRE as of 25 Mar 2023



\bibitem{bms}
%\cite{Basilakos:2020qmu}
S.~Basilakos, N.~E.~Mavromatos and J.~Sol\`a Peracaula,
%``Gravitational and Chiral Anomalies in the Running Vacuum Universe and Matter-Antimatter Asymmetry,''
Phys. Rev. D \textbf{101} (2020) no.4, 045001;
%``Quantum Anomalies in String-Inspired Running Vacuum Universe: Inflation and Axion Dark Matter,''
Phys. Lett. B \textbf{803} (2020), 135342.
%doi:10.1016/j.physletb.2020.135342
%[arXiv:2001.03465 [gr-qc]].
%33 citations counted in INSPIRE as of 25 Mar 2023

\bibitem{ms}
N.~E.~Mavromatos and J.~Sol\`a Peracaula,
%``Inflationary physics and trans-Planckian conjecture in the stringy running vacuum model: from the phantom vacuum to the true vacuum,''
Eur. Phys. J. Plus \textbf{136} (2021) no.11, 1152;
%doi:10.1140/epjp/s13360-021-02149-6
%[arXiv:2105.02659 [hep-th]];
%22 citations counted in INSPIRE as of 25 Mar 2023
%\cite{Mavromatos:2020kzj}
%``Stringy-running-vacuum-model inflation: from primordial gravitational waves and stiff axion matter to dynamical dark energy,''
Eur. Phys. J. ST \textbf{230} (2021) no.9, 2077-2110.
%doi:10.1140/epjs/s11734-021-00197-8
%[arXiv:2012.07971 [hep-ph]].
%31 citations counted in INSPIRE as of 25 Mar 2023
%\cite{Mavromatos:2020crd}



\bibitem{smatrix}
%\cite{Fischler:2001yj}
%\cite{Hellerman:2001yi}
S.~Hellerman, N.~Kaloper and L.~Susskind,
%``String theory and quintessence,''
JHEP \textbf{06} (2001), 003;
%doi:10.1088/1126-6708/2001/06/003
%[arXiv:hep-th/0104180 [hep-th]].
%317 citations counted in INSPIRE as of 25 Mar 2023
W.~Fischler, A.~Kashani-Poor, R.~McNees and S.~Paban,
%``The Acceleration of the universe, a challenge for string theory,''
JHEP \textbf{07} (2001), 003.
%doi:10.1088/1126-6708/2001/07/003
%[arXiv:hep-th/0104181 [hep-th]].
%294 citations counted in INSPIRE as of 25 Mar 2023


\bibitem{swamp}
%%\cite{Ooguri:2018wrx}
%%\cite{Ooguri:2006in}
H.~Ooguri and C.~Vafa,
%``On the Geometry of the String Landscape and the Swampland,''
Nucl. Phys. B \textbf{766} (2007), 21-33;
%[arXiv:hep-th/0605264 [hep-th]].
%%804 citations counted in INSPIRE as of 25 Mar 2023
G.~Obied, H.~Ooguri, L.~Spodyneiko and C.~Vafa,
%``De Sitter Space and the Swampland,''
[arXiv:1806.08362 [hep-th]];
%%814 citations counted in INSPIRE as of 25 Mar 2023
%doi:10.1016/j.nuclphysb.2006.10.033
%%\cite{Garg:2018reu}
S.~K.~Garg and C.~Krishnan,
%``Bounds on Slow Roll and the de Sitter Swampland,''
JHEP \textbf{11} (2019), 075;
%doi:10.1007/JHEP11(2019)075
%[arXiv:1807.05193 [hep-th]].
%%541 citations counted in INSPIRE as of 25 Mar 2023
%%\cite{Ooguri:2018wrx}
H.~Ooguri, E.~Palti, G.~Shiu and C.~Vafa,
%``Distance and de Sitter Conjectures on the Swampland,''
Phys. Lett. B \textbf{788} (2019), 180-184;
%doi:10.1016/j.physletb.2018.11.018
%[arXiv:1810.05506 [hep-th]].
%%557 citations counted in INSPIRE as of 25 Mar 2023
For reviews see:
E.~Palti,
%``The Swampland: Introduction and Review,''
Fortsch. Phys. \textbf{67} (2019) no.6, 1900037,
%doi:10.1002/prop.201900037
%[arXiv:1903.06239 [hep-th]],
%679 citations counted in INSPIRE as of 25 Mar 2023
and references therein.

\bibitem{emndefects}
J.~R.~Ellis, N.~E.~Mavromatos and D.~V.~Nanopoulos,
%``Time dependent vacuum energy induced by D particle recoil,''
Gen. Rel. Grav. \textbf{32} (2000), 943-958.
%doi:10.1023/A:1001993226227
%[arXiv:gr-qc/9810086 [gr-qc]].
%49 citations counted in INSPIRE as of 24 Jul 2023

%\cite{Moreno-Pulido:2022upl}
\bibitem{qfteos}
C.~Moreno-Pulido and J.~Sol\`a Peracaula,
%``Equation of state of the running vacuum,''
Eur. Phys. J. C \textbf{82} (2022) no.12, 1137.
%doi:10.1140/epjc/s10052-022-11117-y
%[arXiv:2207.07111 [gr-qc]].
%7 citations counted in INSPIRE as of 25 Mar 2023





%\cite{Papagiannopoulos:2019kar}
\bibitem{tsiapi}
G.~Papagiannopoulos, P.~Tsiapi, S.~Basilakos and A.~Paliathanasis,
%``Dynamics and cosmological evolution in $\Lambda$-varying cosmology,''
Eur. Phys. J. C \textbf{80} (2020) no.1, 55;
%doi:10.1140/epjc/s10052-019-7600-z
%[arXiv:1911.12431 [gr-qc]].
%28 citations counted in INSPIRE as of 25 Mar 2023
%\cite{Tsiapi:2018she}
P.~Tsiapi and S.~Basilakos,
%``Testing dynamical vacuum models with CMB power spectrum from Planck,''
Mon. Not. Roy. Astron. Soc. \textbf{485} (2019) no.2, 2505-2510.
%doi:10.1093/mnras/stz540
%[arXiv:1810.12902 [astro-ph.CO]].
%%22 citations counted in INSPIRE as of 25 Mar 2023





 %\cite{Asimakis:2021yct}
\bibitem{bbnrvm}
P.~Asimakis, S.~Basilakos, N.~E.~Mavromatos and E.~N.~Saridakis,
%``Big bang nucleosynthesis constraints on higher-order modified gravities,''
Phys. Rev. D \textbf{105} (2022) no.8, 8.
%doi:10.1103/PhysRevD.105.084010
%[arXiv:2112.10863 [gr-qc]].
%16 citations counted in INSPIRE as of 25 Mar 2023

%\cite{Mavromatos:2021sew}
\bibitem{mavrophil}
N.~E.~Mavromatos,
%\cite{Mavromatos:2021hai}
%``Torsion in String-Inspired Cosmologies and the Universe Dark Sector,''
Universe \textbf{7} (2021) no.12, 480;
%doi:10.3390/universe7120480
%[arXiv:2111.05675 [hep-th]];
%8 citations counted in INSPIRE as of 14 Apr 2023
%``Geometrical origins of the universe dark sector: string-inspired torsion and anomalies
 %as seeds for inflation and dark matter,''
Phil. Trans. A. Math. Phys. Eng. Sci. \textbf{380} (2022) no.2222, 20210188.
%doi:10.1098/rsta.2021.0188
%[arXiv:2108.02152 [gr-qc]];
%12 citations counted in INSPIRE as of 25 Mar 2023


%\cite{Green:2012oqa}
\bibitem{string}
M.~B.~Green, J.~H.~Schwarz and E.~Witten,
``Superstring Theory Vols. 1 \& 2: 25th Anniversary Edition,'' (Cambridge University Press, 2012)
ISBN 978-1-139-53477-2, 978-1-107-02911-8;
%doi:10.1017/CBO9781139248563;
ISBN 978-1-139-53478-9, 978-1-107-02913-2.
%doi:10.1017/CBO9781139248570


%\cite{Duncan:1992vz}
\bibitem{kaloper}
M.~J.~Duncan, N.~Kaloper and K.~A.~Olive,
%``Axion hair and dynamical torsion from anomalies,''
Nucl. Phys. B \textbf{387} (1992), 215-235.
%doi:10.1016/0550-3213(92)90052-D
%65 citations counted in INSPIRE as of 26 Mar 2023

%\cite{Svrcek:2006yi}
\bibitem{svrcek}
P.~Svrcek and E.~Witten,
%``Axions In String Theory,''
JHEP \textbf{06} (2006), 051.
%doi:10.1088/1126-6708/2006/06/051
%[arXiv:hep-th/0605206 [hep-th]].
%1258 citations counted in INSPIRE as of 26 Mar 2023


%\cite{Green:1984sg}
\bibitem{gs}
M.~B.~Green and J.~H.~Schwarz,
%``Anomaly Cancellation in Supersymmetric D=10 Gauge Theory and Superstring Theory,''
Phys. Lett. B \textbf{149} (1984), 117-122.
%doi:10.1016/0370-2693(84)91565-X
%3015 citations counted in INSPIRE as of 25 Mar 2023






%\cite{Eguchi:1980jx}
\bibitem{Eguchi}
T.~Eguchi, P.~B.~Gilkey and A.~J.~Hanson,
%``Gravitation, Gauge Theories and Differential Geometry,''
Phys. Rept. \textbf{66} (1980), 213.
%doi:10.1016/0370-1573(80)90130-1
%1412 citations counted in INSPIRE as of 26 Mar 2023

\bibitem{alvwitt} L.~Alvarez-Gaume and E.~Witten,
%``Gravitational Anomalies,''
Nucl. Phys. B \textbf{234} (1984), 269.
%doi:10.1016/0550-3213(84)90066-X
%1772 citations counted in INSPIRE as of 28 Apr 2023

%\cite{Hehl:1976kj}
\bibitem{torsion}
F.~W.~Hehl, P.~Von Der Heyde, G.~D.~Kerlick and J.~M.~Nester,
%``General Relativity with Spin and Torsion: Foundations and Prospects,''
Rev. Mod. Phys. \textbf{48} (1976), 393-416;
%doi:10.1103/RevModPhys.48.393.
%1828 citations counted in INSPIRE as of 26 Mar 2023
%\cite{Shapiro:2001rz}
I.~L.~Shapiro,
%``Physical aspects of the space-time torsion,''
Phys. Rept. \textbf{357} (2002), 113.
%doi:10.1016/S0370-1573(01)00030-8
%[arXiv:hep-th/0103093 [hep-th]].
%515 citations counted in INSPIRE as of 26 Mar 2023



%\cite{Arvanitaki:2009fg}
\bibitem{arvanitaki}
A.~Arvanitaki, S.~Dimopoulos, S.~Dubovsky, N.~Kaloper and J.~March-Russell,
%``String Axiverse,''
Phys. Rev. D \textbf{81} (2010), 123530.
%doi:10.1103/PhysRevD.81.123530
%[arXiv:0905.4720 [hep-th]].
%1456 citations counted in INSPIRE as of 26 Mar 2023


%\cite{Marsh:2015xka}
\bibitem{marsh}
D.~J.~E.~Marsh,
%``Axion Cosmology,''
Phys. Rept. \textbf{643} (2016), 1-79.
%doi:10.1016/j.physrep.2016.06.005
%[arXiv:1510.07633 [astro-ph.CO]].
%1367 citations counted in INSPIRE as of 26 Mar 2023

%\cite{Alexandre:2013iva}
\bibitem{houston}
J.~Alexandre, N.~Houston and N.~E.~Mavromatos,
%``Dynamical Supergravity Breaking via the Super-Higgs Effect Revisited,''
Phys. Rev. D \textbf{88} (2013), 125017;
%doi:10.1103/PhysRevD.88.125017
%[arXiv:1310.4122 [hep-th]];
%25 citations counted in INSPIRE as of 26 Mar 2023
%\cite{Alexandre:2014lla}
%``Inflation via Gravitino Condensation in Dynamically Broken Supergravity,''
Int. J. Mod. Phys. D \textbf{24} (2015) no.04, 1541004.
%doi:10.1142/S0218271815410047
%[arXiv:1409.3183 [gr-qc]].
%22 citations counted in INSPIRE as of 26 Mar 2023

%\cite{Ellis:2013zsa}
\bibitem{ellisinfl}
J.~Ellis and N.~E.~Mavromatos,
%``Inflation induced by gravitino condensation in supergravity,''
Phys. Rev. D \textbf{88} (2013) no.8, 085029.
%doi:10.1103/PhysRevD.88.085029
%[arXiv:1308.1906 [hep-th]].
%38 citations counted in INSPIRE as of 26 Mar 2023

%\cite{Lalak:1994qt}
\bibitem{ross}
%\cite{Coulson:1995nv}
%\cite{Lalak:1993bp}
Z.~Lalak, B.~A.~Ovrut and S.~Thomas,
%``Large scale structure as a critical phenomenon,''
Phys. Rev. D \textbf{51} (1995), 5456-5474;
%doi:10.1103/PhysRevD.51.5456
%14 citations counted in INSPIRE as of 26 Mar 2023
Z.~Lalak, S.~Lola, B.~A.~Ovrut and G.~G.~Ross,
%``Large scale structure from biased nonequilibrium phase transitions: Percolation theory picture,''
Nucl. Phys. B \textbf{434} (1995), 675-696;
%doi:10.1016/0550-3213(94)00557-U
%[arXiv:hep-ph/9404218 [hep-ph]].
%29 citations counted in INSPIRE as of 26 Mar 2023
D.~Coulson, Z.~Lalak and B.~A.~Ovrut,
%``Biased domain walls,''
Phys. Rev. D \textbf{53} (1996), 4237-4246.
%doi:10.1103/PhysRevD.53.4237
%86 citations counted in INSPIRE as of 26 Mar 2023




\bibitem{mavlorentz}
%\cite{Mavromatos:2022xdo}
N.~E.~Mavromatos,
``Lorentz Symmetry Violation in String-Inspired Effective Modified Gravity Theories,''
[arXiv:2205.07044 [hep-th]], contribution to {\it 740. WE-Heraeus-Seminar : Experimental Tests and Signatures of Modified and Quantum Gravity Workshop}, invited chapter to Springer book, in press.
%1 citations counted in INSPIRE as of 26 Mar 2023


\bibitem{lyth}  %\cite{Alexander:2004us}
S.~H.~S.~Alexander, M.~E.~Peskin and M.~M.~Sheikh-Jabbari,
%``Leptogenesis from gravity waves in models of inflation,''
Phys. Rev. Lett. \textbf{96} (2006), 081301;
%doi:10.1103/PhysRevLett.96.081301
%[arXiv:hep-th/0403069 [hep-th]].
%254 citations counted in INSPIRE as of 26 Mar 2023
%\cite{Lyth:2005jf}
D.~H.~Lyth, C.~Quimbay and Y.~Rodriguez,
%``Leptogenesis and tensor polarisation from a gravitational Chern-Simons term,''
JHEP \textbf{03} (2005), 016.
%doi:10.1088/1126-6708/2005/03/016
%[arXiv:hep-th/0501153 [hep-th]].
%72 citations counted in INSPIRE as of 26 Mar 2023

%\cite{McAllister:2008hb}
\bibitem{silver}
L.~McAllister, E.~Silverstein and A.~Westphal,
%``Gravity Waves and Linear Inflation from Axion Monodromy,''
Phys. Rev. D \textbf{82} (2010), 046003.
%doi:10.1103/PhysRevD.82.046003
%[arXiv:0808.0706 [hep-th]].
%808 citations counted in INSPIRE as of 27 Mar 2023




%\cite{Mavromatos:2022yql}
\bibitem{stamou}
N.~E.~Mavromatos, V.~C.~Spanos and I.~D.~Stamou,
%``Primordial black holes and gravitational waves in multiaxion-Chern-Simons inflation,''
Phys. Rev. D \textbf{106} (2022) no.6, 063532.
%doi:10.1103/PhysRevD.106.063532
%[arXiv:2206.07963 [hep-th]].
%2 citations counted in INSPIRE as of 27 Mar 2023



%\cite{Carr:2021bzv}
\bibitem{pBH}
%\cite{Carr:2016drx}
B.~Carr, F.~Kuhnel and M.~Sandstad,
%``Primordial Black Holes as Dark Matter,''
Phys. Rev. D \textbf{94} (2016) no.8, 083504;
%doi:10.1103/PhysRevD.94.083504
%[arXiv:1607.06077 [astro-ph.CO]].
%873 citations counted in INSPIRE as of 28 Mar 2023
%\cite{Clesse:2016vqa}
S.~Clesse and J.~Garc\'\i{}a-Bellido,
%``The clustering of massive Primordial Black Holes as Dark Matter: measuring their mass distribution with Advanced LIGO,''
Phys. Dark Univ. \textbf{15} (2017), 142-147.
%doi:10.1016/j.dark.2016.10.002
%[arXiv:1603.05234 [astro-ph.CO]].
%486 citations counted in INSPIRE as of 28 Mar 2023

%\cite{Basilakos:2015yoa}
\bibitem{bmssugra}
S.~Basilakos, N.~E.~Mavromatos and J.~Sol\`a,
%``Starobinsky-like inflation and running vacuum in the context of Supergravity,''
Universe \textbf{2} (2016) no.3, 14.
%doi:10.3390/universe2030014
%[arXiv:1505.04434 [gr-qc]].
%43 citations counted in INSPIRE as of 19 Jul 2023

\end{thebibliography}

\end{document}

