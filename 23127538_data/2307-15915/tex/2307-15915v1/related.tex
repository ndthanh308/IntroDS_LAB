\section{Related Works}
\label{sec:related}

The automation of software vulnerability identification is a topic of great interest for researchers, who continue to develop new techniques to detect vulnerabilities. Approaches range from traditional methods that manually establish vulnerability rules, to machine learning techniques that determine vulnerabilities based on features, and to deep learning that learns vulnerability patterns.

\subsection{Traditional Detection Methods}
Traditional detection methods rely on known vulnerability rules to detect vulnerabilities, using manual code reviews and automated code scanners. Kaur \emph{et al.} introduced five static code analysis tools for vulnerability detection in C/C++ and Java~\cite{KAUR20202023}. Flawfinder~\cite{Flawf8239638:online} is designed to detect vulnerabilities in C/C++ source code, generating a list of vulnerabilities for the program sorted by risk level. RATS~\cite{Googl8714453:online} is a security vulnerability auditing tool for C/C++ source code that can detect issues such as buffer overflows. SPOTBUGS~\cite{SpotB2548343:online} is a program that uses static analysis to identify bugs in Java code, checking for more than 400 bug patterns. PMD~\cite{PMD6397148:online} is an open-source source analysis tool that employs rule sets to find common coding errors, irregular code, and potential vulnerabilities. \textcolor{blue}{Peguero \emph{et al.} analyzed Electron application security, revealing potential front-to-back-end attack escalation. They proposed framework modifications and an IDE plugin for early vulnerability fixes. Their studies confirmed the effectiveness of the plugin, as applications ceased to be exploitable post-fix.\cite{PEGUERO2021100032}}

\subsection{Machine Learning-based Methods}
When integrating various types of conventional hand-crafted features, the challenge remains how to effectively combine these features. Machine learning-based methods can address this issue by performing simple classification on manually extracted features with better performance than traditional methods. Yang \emph{et al.} proposed a deep learning model for just-in-time defect prediction~\cite{7272910}, building a set of expressive features from a set of initial change features using a deep belief network algorithm and constructing a classifier based on the selected features. Chen \emph{et al.} proposed a model~\cite{2018chen} capable of identifying SQL injection vulnerabilities by processing HTTP request text data using word2vec and classifying processed samples with the SVM algorithm. Al-Yaseen \emph{et al.} suggested a multi-level hybrid intrusion detection model that employs support vector machines and extreme learning machines to enhance the efficiency of detecting known and unknown attacks~\cite{10.1016/j.eswa.2016.09.041}. Ren \emph{et al.} introduced DVCMA~\cite{DVCMA}, a software vulnerability detection method based on clustering and model analysis that applies clustering techniques to mine patterns from vulnerability sequences. \textcolor{blue}{Peguero \emph{et al.} analyzes cross-site request forgery vulnerabilities in several server-side JavaScript frameworks. Utilizing automated static analysis, the security efficacy of each is evaluated. Based on these insights, recommendations for more secure application development are provided.\cite{PEGUERO2021100035}}


\subsection{Deep Learning-based Methods}
Manual inspection of source code or manual extraction of features from the source code is time-consuming. Deep learning-based methods can automatically extract vulnerability patterns and classify them based on the input source code's features. Zhan \emph{et al.} proposed ISVSF~\cite{ISVSF}, an intelligent sentence-level vulnerability self-detection framework that considers Java syntax characteristics and adopts sentence-level method representation and pattern exploration. Malhotra \emph{et al.} suggested an improved CNN~\cite{improvecnn}, a modified CNN algorithm that combines CNN-based layers into one and then applies a concatenate algorithm under SVM. Saccente \emph{et al.} introduced Project Achilles~\cite{Achilles}, a Java source code security vulnerability detection tool built upon LSTM RNN models. Lin \emph{et al.} proposed VulEye~\cite{app13020825}, a graph neural network vulnerability detection approach for PHP applications that utilizes program dependence graphs as input and is trained with a graph neural network model containing three stack units. \textcolor{blue}{Zheng \emph{et al.} presented CodeGeeX, a multilingual, 13 billion-parameter model surpassing peers in code generation and translation on HumanEval-X. The model enhances coding efficiency for 83.4\% of users through developed extensions. In September 2022, all associated CodeGeeX resources were publicly released. \cite{zheng2023codegeex} Raymond Li \emph{et al.} presented StarCoder and StarCoderBase, Code LLMs with 15.5B parameters and 8K context length. Trained on The Stack's 1 trillion tokens, StarCoder, a fine-tuned StarCoderBase, outperforms multilingual Code LLMs and Python-specialized models.\cite{li2023starcoder} }