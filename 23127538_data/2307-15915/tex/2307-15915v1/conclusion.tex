\section{Conclusions and Future Research}
\label{sec:conclusion}
As modern software grows in size and complexity, ensuring stability has become a critical concern. In this paper, we introduced JFinder, a novel architecture for Java vulnerability identification based on quad self-attention and a pre-training mechanism. JFinder innovatively combines structural and semantic information through a proposed quad self-attention layer. Experimental results demonstrate that JFinder outperforms all baseline models, achieving a level of performance suitable for industrial use. JFinder's F1 scores are 25\% higher than those of the baselines, and its ACC surpasses them by 5\%. Furthermore, we conducted a case study to investigate whether the model truly understands the root causes of vulnerabilities. 
\textcolor{blue}{Looking forward, we see significant potential in exploring the use of larger language models for pre-training. The pre-training mechanism, an unsupervised learning process, enables the model to learn a wide range of syntactic and semantic patterns before fine-tuning on a specific task. The application of larger language models in this process could enhance our model's understanding of complex code structures and semantics, possibly leading to improved precision and recall in identifying vulnerabilities in Java code.}
















