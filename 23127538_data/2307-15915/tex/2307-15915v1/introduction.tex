\section{Introduction}
\label{sec:introduction}
As modern software continues to increase in functionality, the likelihood of vulnerabilities also grows. These vulnerabilities pose significant risks to cybersecurity, with implications for individuals, the healthcare industry, and industrial production. For individuals, such vulnerabilities can result in the leak of sensitive information, leading to identity theft and fraud. In the healthcare sector, cybersecurity breaches may entail the theft of health information, ransomware attacks on hospitals, and even attacks on implanted medical devices. In industrial production, software vulnerabilities can introduce corresponding vulnerabilities in products reliant on that software, as exemplified by the Log4j2 vulnerability. Apache Log4j2 is susceptible to remote code execution (RCE) attacks\footnote{https://logging.apache.org/log4j/2.x/security.html}, wherein an attacker with the ability to modify logging configuration files can create malicious configurations using the JDBC Appender and a data source referencing a JNDI URI, enabling remote code execution. The Log4j2 vulnerability impacted tens of thousands of products but was swiftly addressed, preventing a catastrophic cyber event.\par
The identification of vulnerabilities is crucial for ensuring system security and timely remediation of security flaws, thus protecting against hacking and data breaches. However, vulnerability detection can be a laborious and challenging process. Researchers have been continuously exploring methods to automate this task. Initially, researchers manually identified features and employed machine learning to ascertain the existence of vulnerabilities, but this approach proved time-consuming. Subsequently, deep learning techniques were utilized to automatically detect vulnerability features and classify them. Many of these methods involve the use of graph neural networks to identify vulnerability patterns, leading to the development of new graph neural network types (e.g., Graph Convolution Networks, Graph Attention Networks, and Graph Autoencoders). Some approaches rely on structural information (e.g., Abstract Syntax Tree, Control Flow Graph, and Data Flow Graph) to construct graphs, while others solely employ semantic information of codes for embedding. However, these methods have not demonstrated satisfactory performance on real software vulnerability datasets and remain unsuitable for industrial application.\par
To address these issues, we propose JFinder, a novel architecture for Java vulnerability identification leveraging structural information with MetaPaths, a quad self-attention layer, and a pre-trained programming language model. We utilize a third-party library to obtain the Abstract Syntax Tree (AST), Control Flow Graph (CFG), and Data Flow Graph (DFG) of source code as structural information. We derive the Code Snippet Sequence (CSS) using a pre-trained model, UniXcoder~\cite{guo2022unixcoder}, a transformer-based language model designed for natural language processing tasks in the software development domain, trained on an extensive corpus of source code and natural language text related to software development. UniXcoder enables the conversion of program language into a feature matrix, providing accurate semantic representations of code snippets as it is trained on program language. JFinder then feeds semantic and structural information into convolutional networks and multilayer perceptrons to predict vulnerability presence. Overall, JFinder uniquely combines semantic and structural information to comprehensively analyze the execution process from multiple perspectives. We have implemented JFinder as an open-source project on GitHub\footnote{https://github.com/WJ-8/JFinder}. We evaluated JFinder on CWE and PROMISE datasets, comprising a total of 20,402 code snippets, of which 7,355 were vulnerable. Experimental results indicate that JFinder achieved outstanding performance on the CWE dataset, with an accuracy rate of 97\%. On the PROMISE dataset, JFinder attained an industrially viable level with F1 scores reaching 0.83. We also conducted case studies with four cases; after vulnerabilities were addressed, JFinder no longer identified these cases as vulnerable. The case study results demonstrated the capacity of JFinder to uncover robust vulnerability patterns and provide insights. Our contributions are threefold: \par
\begin{itemize}
	\item We propose a novel architecture for java vulnerability identification, JFinder, which combines structural information and semantic information from a code snippet. We open the source of JFinder in Github.
	\item We have conducted a large number of experiments and compared them with recent excellent methods. The results show that JFinder outperforms all baseline methods and the results are satisfactory.
	\item We conduct case studies to explore the robustness and intelligence level of JFinder. Experience results show that JFinder understands the meaning of the vulnerabilities in depth.
\end{itemize}
\textbf{Paper Organization.} 
The rest of the paper is organized as follows. Section~\ref{sec:background} presents the recently advanced background knowledge of our approach. Section~\ref{sec:approach} details the design and technical components of the JFinder framework. Section~\ref{sec:implementation} demonstrates our implementation of JFinder. Section~\ref{sec:eval} reports our evaluation results and case studies on JFinder. Section~\ref{sec:related} outlines the most related work.  Section~\ref{sec:conclusion} concludes the paper with a future research discussion.