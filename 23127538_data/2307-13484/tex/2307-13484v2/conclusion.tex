We have presented a comprehensive framework for kernel-based
interpolation of complex-valued functions and frequency response
functions. %
In the complex-valued case, the pseudo-kernel is an additional
ingredient, which can be used to improve the interpolation
accuracy. %
We have introduced the concept of complex/real reproducing kernel
Hibert spaces to reveal the role of the pseudo-kernel and to establish
results on minimum norm interpolation. %
Furthermore, we have proposed a hybrid method, which complements the kernel-interpolant with a
low-order rational function and a new model selection criterion: this extension
is crucial to account for dominant poles in applications.

The capabilities of the rational-kernel method have been illustrated
with several examples, from circuits to frequency response functions
originating from PDE problems. %
In all examples the performance was at least comparable, in some cases
improved, compared to AAA and vector fitting on the same set of
training data.

The kernel method was further linked to complex-valued Gaussian
process regression, which can be used in future work to include noise
and adaptive sampling. %
A generalization to the multivariate case, where, e.g., uncertain parameters are considered as well, 
and comparisons against multivariate AAA \cite{rodriguez2020p} or rational Polynomial Chaos \cite{schneider2023sparse}, would also be of interest. %
