\section{Proofs}

\subsection{Proof of Theorem~\ref{thm:gammaAlpha}}
\label{proof:thmGammaAlpha}

We assume without loss of generality that~$\alpha = 0$ in this
proof---i.e., we consider the case of the Hardy space~$H^2(\Gamma_0)$
on the right
half-plane~$\Gamma_0 = \left\{ s \in \mathbb{C} \,\mid\, \Re[s] >0
\right\}$. %
The general case follows by translation.

The fact that~$H^2(\Gamma_0)$ is an RKHS is well known. %
Indeed, recall the one-sided Paley-Wiener theorem
(see, e.g., Chapter~8 of~\cite{hoffman:1962}): %
for all~$f \in H^2(\Gamma_0)$, there exists a
unique~$\widehat f \in L^2(\Rset_+)$ such that
\begin{equation}\label{equ:PW-repr}
  f(s) \;=\; \frac{1}{\sqrt{2\pi}} \int_0^{+\infty} \widehat f(t)\, e^{-st}\, \dt,
  \qquad \forall s \in \Gamma_0,
\end{equation}
and the mapping $f \mapsto \widehat f$ is a surjective isometry:
$\lVert f \rVert_{H^2(\Gamma_0)} = \lVert \widehat f
\rVert_{L^2(\Rset_+)}$. %
This proves that $H^2(\Gamma_0)$~is a Hilbert space, and a simple
application of the Cauchy-Schwartz inequality for
$s = x+\i y \in \Gamma_0$ yields:
\begin{equation*}
  \left| f(s) \right| \;\le\;
  \frac{1}{2\sqrt{\pi x}} \cdot \lVert \widehat f \rVert_{L^2(\Rset_+)},
\end{equation*}
which proves that the evaluation functionals are continuous
on~$H^2(\Gamma_0)$.

Let us now determine the kernel~$k$ of this RKHS. %
Let $s_0 \in \Gamma_0$ and set $h = k(\cdot, s_0)$. %
Then, for any~$f \in H^2(\Gamma_0)$, the reproduction property
combined with~\eqref{equ:PW-repr} yields:
\begin{equation*}
  \left< f,\, h \right>_{H^2(\Gamma_0)}
  \;=\; f(s_0)
  \;=\; \frac{1}{\sqrt{2\pi}} \int_0^{+\infty} \widehat f(t)\, e^{-s_0t}\, \dt
  \;=\; \left< \widehat f,\, \frac{1}{\sqrt{2\pi}}\, e^{-s_0^* (\cdot)} \right>_{L^2(\Rset_+)},
\end{equation*}
which implies that
$\widehat h = \frac{1}{\sqrt{2\pi}}\, e^{-s_0^* (\cdot)}$ since
$f \mapsto \widehat f$ is an isometric isomorphism. %
The expression of the kernel follows:
\begin{equation}\label{equ:kernel:Gamma0}
  k(s, s_0) \;=\; h(s)
  \;=\; \frac{1}{\sqrt{2\pi}} \int_0^{+\infty} \widehat{h}(t)\, e^{-st}\, \dt
  \;=\; \frac{1}{2\pi \left( s + s_0^* \right)}.
\end{equation}

It remains to show that $k$~is strictly positive definite. %
For any $m \ge 1$ and $s_1, \ldots, s_m \in \Gamma_0$, the kernel
matrix $K_m = \left( k(s_i, s_j) \right)_{1 \le i, j \le m}$ can be
seen as the conjugate Gram matrix of~$h_1, \ldots, h_m$
in~$L^2(\Rset_+)$, where
$h_j(t) = \frac{1}{\sqrt{2\pi}}\, e^{-s_j^* t}$, $t \ge 0$. %
Assume that $s_1$, \ldots, $s_m$ are distinct. %
Then it is well known that the complex
exponentials~$e^{-s_1^* (\cdot)}$, \ldots, $e^{-s_m^* (\cdot)}$ are
linearly independent entire functions on~$\Cset$. %
It follows, using the identity theorem, that $h_1$, \ldots, $h_m$ are
linearly independent as well. %
The kernel matrix $K_m$~is thus invertible and, consequently, positive
definite. %
Therefore $k$ is strictly positive definite.

\begin{remark} 
  The expression of the reproducing kernel is also derived in
  \cite[Theorem 2.12]{Bonyo_2020aa} (for the upper half-plane instead
  of~$\Gamma_0$) using a different approach involving the kernel of
  the Hardy space of the unit disk. %
  Note, however, that the factor~$2\pi$ in the denominator
  of~\eqref{equ:kernel:Gamma0} is missing
  in~\cite[Equation~(2.9)]{Bonyo_2020aa}; %
  the discrepancy comes from a missing factor~$\frac{1}{2\pi}$ in the
  definition of the norm on~$H^p(\Dset)$ on page~14.
\end{remark}


\subsection{Proof of Proposition~\ref{prop:counterexample-dim2}}

Take $H = \left\{ \alpha f_0,\, \alpha \in \Cset \right\}$, where
$f_0:\Xset \to \Cset$ is some fixed function, and define a real inner
product over~$H$ by
$\left< \alpha f_0,\, \beta f_0 \right> := \Re\alpha \cdot \Re\beta +
4\, \Im\alpha \cdot \Im \beta$.  %
Assuming that $f_0 \not\equiv 0$, the resulting space is complex/real
RKHS of dimension two, spanned by~$\left\{ f_0,\, \i f_0 \right\}$. %
($H$ is also a complex vector space of dimension~$1$.)

It not possible to embed~$H$ as a subspace of a complex Hilbert
space~$H_\Cset$ with inner product~$\left< \cdot, \cdot \right>_\Cset$
such that $\left<f, g \right> = \Re \left<f, g \right>_\Cset$ for all
$f, g \in H$. %
To see it, note for instance that $\lVert f_0 \rVert = 1$ while
$\lVert \i f_0 \rVert = 2$.


\subsection{Proof of Proposition~\ref{prop:repr-eval-func}}
\label{proof:propReprEvalFunc}

Let $f \in H$, $s_0 \in \Sset$ and~$a_0 \in \{ \re, \im \}$. Then
\begin{align}
  G_{a_0} \left( f(s_0) \right)
  &\;=\; \left( \Acal f \right)(s_0, a_0)
  \;=\; \left< \Acal f,\, \tilde k\left( \cdot,\, (s_0, a_0)\right) \right>_{\tilde H}\\
  &\;=\; \left< f,\, \Acal^{-1} \left( \tilde k\left( \cdot,\, (s_0, a_0)\right) \right) \right>_H.
    \label{equ:proof:repr-eval-func:2}
\end{align}
Taking $a_0 = \re$, we have thus proved that
$\Re \circ \delta_{s_0} = \left<\, \bm{\cdot}\,,\, \varphi_\re \left(
    \cdot, s_0 \right) \,\right>_H$, where
\begin{equation}
  \varphi_\re \left( \cdot, s_0 \right) = \Acal^{-1} \left( %
    \tilde k\left( \cdot,\, (s_0, \re)\right) %
  \right) \in H
\end{equation}
can be computed as follows:
\begin{align}
  \Re \left[ \varphi_\re \left( s, s_0 \right) \right]
  & \;=\; \left( \Acal\left[ \varphi_\re \left( \cdot, s_0 \right) \right] \right)(s, \re)
    \;=\; \tilde k\left( (s, \re),\, (s_0, \re)\right)
    \;=\; k_{\re\re}(s, s_0),\\
  \Im \left[ \varphi_\re \left( s, s_0 \right) \right]
  & \;=\; \left( \Acal\left[ \varphi_\re \left( \cdot, s_0 \right) \right] \right)(s, \im)
    \;=\; \tilde k\left( (s, \im),\, (s_0, \re)\right)
    \;=\; k_{\im\re}(s, s_0).
\end{align}
The expression of~$\varphi_\im \left( \cdot, s_0 \right)$ is derived similarly
by taking $a_0 = \im$ in~\eqref{equ:proof:repr-eval-func:2}.


\subsection{Proof of Proposition~\ref{prop:dense-subspace}}

In a real or complex RKHS, it is well known that the partial kernel
functions~$k(\cdot,\, s_0)$, $s \in \Sset$, span a dense subset of the
Hilbert space. %
Moreover, recall that the bijection~$\Acal$ defined in
Section~\ref{sec:complex-real-RKHS} is an isometric isomorphism
between~$H$ and a real RKHS $\tilde H$
on~$\tilde\Sset = \Sset \times \{\re,\im\}$, whose kernel~$\tilde k$
can be recovered from~$k$ and~$c$ by inverting
\eqref{prop:repr-eval-func}--\eqref{def:complex-kernels}. %
The claim then follows from the observation that any function
on~$\tilde\Sset$ of the form
\begin{equation*}
  \tilde g \;=\;
  \sum_{i=1}^n \alpha_i\, \tilde k\left( \cdot, (s_i, \re) \right)
  + \sum_{i=1}^n \beta_i\, \tilde k\left( \cdot, (s_i, \im) \right),
\end{equation*}
where $\alpha_1, \beta_1, \ldots, \alpha_n, \beta_n \in \Rset$,
corresponds to the image by~$\Acal$ of
\begin{align*}
  g & \;=\;   \sum_{i=1}^n \alpha_i\, \Acal^{-1}\left( \tilde k\left( \cdot, (s_i, \re) \right) \right)
      + \sum_{i=1}^n \beta_i\, \Acal^{-1} \left( \tilde k\left( \cdot, (s_i, \im) \right) \right)\\
    & \;=\;   \sum_{i=1}^n \alpha_i\, \varphi_\re\left( \cdot, s_i \right)
      + \sum_{i=1}^n \beta_i\, \varphi_\im\left( \cdot, s_i \right)\\
    & \;=\;   \sum_{i=1}^n \gamma_i\, k\left( \cdot, s_i \right)
      + \sum_{i=1}^n \gamma_i^*\, c\left( \cdot, s_i \right),
      \qquad \text{with } \gamma_i = \frac{1}{2} \left( \alpha_i + \i \beta_i \right).
\end{align*}


\subsection{Proof of Theorem~\ref{thm:CR-RKHS-characterization}}

Assume first that~$k$ and~$c$ are the complex kernel and pseudo-kernel
associated to a given complex/real RKHS~$H$. %
Let $\tilde\xi$ denote a zero-mean (e.g., Gaussian) real-valued random
process indexed by~$\Sset$ with covariance function equal to the
kernel~$\tilde k$ of the real RKHS~$\tilde H = \Acal H$, and set
$\xi = \tilde\xi(\cdot,\re) + \i\, \tilde\xi(\cdot,\im)$. %
Then $\xi$ is a complex-valued random process on~$\Sset$, with
covariance function~$k$ and pseudo-covariance function~$c$; %
indeed, for all $s, s_0 \in \Sset$,
\begin{align*}
  \Esp\left( \xi(s)\, \xi(s_0)^* \right)
  & \;=\; \left(
    \tilde k\left( (s,\re),\, (s_0, \re) \right)
    + \tilde k\left( (s,\im),\, (s_0, \im) \right)
    \right)\\
  & \quad + \i\, \left(
    \tilde k\left( (s,\im),\, (s_0, \re) \right)
    - \tilde k\left( (s,\re),\, (s_0, \im) \right)
    \right) \;=\; k(s, s_0),
\end{align*}
and similarly~$\Esp\left( \xi(s)\, \xi(s_0) \right) = c(s, s_0)$. %
It follows readily that $k$~is Hermitian and positive definite,
and that $c$ is symmetric, which proves i) and~ii).

Pick~$s_1, \ldots s_n \in \Sset$, and set
$K_n = \left( k(s_i,s_j) \right)_{1 \le i,j \le n}$ and
$C_n = \left( c(s_i,s_j) \right)_{1 \le i,j \le n}$. %
Then~$K_n$ and~$C_n$ are respectively the covariance and
pseudo-covariance matrix of the random
vector~$Z = \left( \xi(s_1),\, \ldots,\, \xi(s_n) \right)^\tra$, and
thus iv) is precisely the ``only if'' part the following result, due
to~\cite{picinbono:1996}.

\begin{proposition} \label{prop:Picinbono}
  Let $n \in \Nset^*$.  Let $K$ be a complex, Hermitian, positive
  definite matrix of order~$n$, and let $C$ be a complex, symmetric
  matrix of the same size.  Then there exists a complex random
  vector~$Z$ with covariance matrix~$K$ and pseudo-covariance
  matrix~$C$ if, and only if, $K^* - C^\her K^{-1} C$ is positive
  semi-definite.
\end{proposition}

It remains to prove~iii): let $u \in \ker K_n$. %
Then $u^\her K_n u = \Esp\left( \left| u^\her Z \right|^2 \right) = 0$,
therefore $u^\her Z = 0$ almost surely, and as a consequence:
\begin{equation*}
  C_n^* u
  \;=\; \Esp\left( Z Z^\tra \right)^*\, u
  \;=\; \Esp\left( Z^* Z^\her u \right)
  \;=\; \Esp\left( Z^* (u^\her Z)^\her \right)
  \;=\; 0.
\end{equation*}
This completes the proof of~i)--iv).

Conversely, assume now that~$k$ and~$c$ are two functions
from~$\Sset \times \Sset$ to~$\Cset$, such that i)--iv) hold. %
Then it is easy to see that there is a unique
function~$\tilde k: \Sset \times \{ \re, \im \} \to \Rset$ such that
\eqref{eq:complex-kernel}--\eqref{eq:pseudo-kernel} hold, given by
\begin{align*}
  k_{\re\re}(s, s_0) & \;=\; \frac{1}{2}\,  \Re \left( k(s,s_0) + c(s,s_0) \right)\\
  k_{\im\im}(s, s_0) & \;=\; \frac{1}{2}\,  \Re \left( k(s,s_0) - c(s,s_0) \right)\\
  k_{\im\re}(s, s_0) & \;=\; \frac{1}{2}\,  \Im \left( k(s,s_0) + c(s,s_0) \right)
                      \;=\; k_{\re\im}(s_0, s).
\end{align*}
It remains to prove that~$\tilde k$ is positive definite. %
It is easy to see that this is true if, and only if, the
matrices~$K_n$ and~$C_n$ defined above are the covariance and
pseudo-covariance matrices of a complex random vector~$Z$, for any
choice of the points~$s_1, \ldots, s_n \in \Sset$. %
Pick such a set of points, and let $r$ denote the rank of~$K_n$. %
Assume without loss of generality that
\begin{equation}
  \label{eq:Kn-block}
  K_n \;=\;
  \begin{pmatrix}
    K_{11} & K_{12}\\
    K_{12}^\her & K_{22}
  \end{pmatrix},
\end{equation}
with $K_{11}$ a positive definite $r \times r$ matrix. %
Then $K_{22} = K_{12}^\her K_{11}^{-1} K_{12}$ and
\begin{equation}
  \label{eq:Kn-block-bis}
  K_n \;=\; M\,
  \begin{pmatrix}
    K_{11} & 0\\
    0 & 0
  \end{pmatrix}\,
  M^\her,
  \qquad \text{where }
  M \;=\;
  \begin{pmatrix}
    \mathrm{I}_r & 0\\
    K_{12}^\her K_{11}^{-1} & \mathrm{I}_{n-r}
  \end{pmatrix}
\end{equation}
Denote by~$C_{11}$ the upper-left $r \times r$ block in~$C_n$. %
Then it follows from iv) that
$K_{11}^* - C_{11}^\her K_{11}^{-1} C_{11}$ is positive semi-definite,
and thus by Proposition~\ref{prop:Picinbono} there exists a complex
random vector~$Z_1$ of size~$r$ with covariance matrix~$K_{11}$ and
pseudo-covariance matrix~$C_{11}$. %
It is then clear from~\eqref{eq:Kn-block-bis} that $K_n$ is the
covariance matrix of
\begin{equation*}
  Z = M\,
  \begin{pmatrix}
    Z_1\\ 0
  \end{pmatrix}.
\end{equation*}
To complete the proof, it remains to observe that~$C_n$ is the
pseudo-covariance matrix of~$Z$:
\begin{equation}
  \label{eq:Cn-block}
  C_n \;=\; M\,
  \begin{pmatrix}
    C_{11} & 0\\
    0 & 0
  \end{pmatrix}\, M^\tra \;=\; \Esp\left( Z Z^\tra \right),
\end{equation}
which follows from the facts that~$C_n$ is symmetric and that
$\ker K_n \subset \ker C_n^*$, respectively by~ii) and~iii).

\subsection{Proof of Theorem~\ref{thm:interp:cr}} %
Using the bijection~$\Acal$ defined in
Section~\ref{sec:complex-real-RKHS}, the interpolation problem
on~$\Sset$ with complex-valued data~$\left( s_1, y_1 \right)$, \ldots,
$\left( s_n, y_n \right)$ can be reformulated as an interpolation
problem on~$\tilde\Sset = \Sset \times \{\re,\im\}$ with real-valued
data~$\left( (s_1, \re), \Re(y_1) \right)$,
$\left( (s_1, \im), \Im(y_1) \right)$, \ldots,
$\left( (s_n, \re), \Re(y_n) \right)$,
$\left( (s_n, \im), \Im(y_n) \right)$. %
The claim then follows from Theorem~\ref{thm:complexRKHSinterpolation}
using, as in the proof of Proposition~\ref{prop:dense-subspace}, the
fact that $\Acal$ is an isometric isomorphism between~$H$ and the real
RKHS $\tilde H = \Acal(H)$.


\subsection{Proof of Theorem~\ref{thm:hermitian}} %
$i) \Rightarrow ii)$. %
Let $H$ denote a complex/real RKHS on~$\Sset$ with complex kernel~$k$,
such that \eqref{equ:symmetry} holds. %
Let $c$ denote the pseudo-covariance of~$H$. %
Let $s_0 \in \Sset$. %
It follows from Proposition~\ref{prop:dense-subspace} that
\begin{equation*}
  f_\gamma \;=\; \gamma\, k(\cdot, s_0) \,+\, \gamma^*\, c(\cdot, s_0)
\end{equation*}
is in~$H$ for all~$\gamma \in \Cset$. %
Using~\eqref{equ:symmetry}, we see then that
\begin{align*}
  f_\gamma(s^*) & \;=\; \gamma\, k(s^*, s_0) \,+\, \gamma^*\, c(s^*, s_0)\\
         & \;=\; \gamma\, c(s, s_0)^* \,+\, \gamma^*\, k(s, s_0)^* \;=\; f_\gamma(s)^*
\end{align*}
holds for all~$\gamma \in \Cset$. %
This yields in particular that
$c(s,s_0) = k(s^*, s_0)^* = k(s_0, s^*)$, and the claim follows from
the symmetry of~$c$:
\begin{equation*}
  c(s, s_0) = c(s_0, s) = k(s, s_0^*).
\end{equation*}
Note that we have actually proved a little more than~$ii)$: if $i)$
holds, then $ii)$ holds for the \emph{same} complex/real RKHS~$H$. %
Since we will now prove that $ii) \Rightarrow iii) \Rightarrow i)$, it
follows that the complex/real RKHS with complex kernel~$k$ and
pseudo-kernel~$c$ defined by~\eqref{equ:pseudo-kern-symm}, if it
exists, is the only complex/real RKHS with complex kernel~$k$ such
that~\eqref{equ:symmetry} holds.

$ii) \Rightarrow iii)$. %
Let $H$ denote a complex/real RKHS on~$\Sset$ with complex kernel~$k$.
Assume that the pseudo-kernel $c$
satisfies~\eqref{equ:pseudo-kern-symm}. %
Then, for all $s, s_0 \in \Sset$,
\begin{equation*}
  k(s, s_0^*) \;=\; c(s, s_0) \;=\; c(s_0, s) \;=\; k(s_0, s^*).
\end{equation*}

\newcommand \ipC  {\left< \cdot,\, \cdot \right>_{\Cset}}
\newcommand \ipR  {\left< \cdot,\, \cdot \right>_{\Rset}}
\newcommand \truc {\diamond}

$iii) \Rightarrow i)$. %
Let $k$ denote a Hermitian positive definite kernel on~$\Sset$ such
that
\begin{equation}\label{equ:assumpt-iii}
  \forall s,s_0 \in \Sset, \quad k(s, s_0^*) = k(s_0, s^*).
\end{equation}
Let $(H_\Cset, \ipC)$ denote the complex RKHS with kernel~$k$ and let
$\ipR = \Re \ipC$. %
Then, as observed in Remark~\ref{rem:complex-RKHS-subspaces},
$(H_\Cset, \ipR)$ is a complex/real RKHS. %
The associated real and imaginary evaluation kernels, which we
denote by~$\varphi_\re^\truc$ and~$\varphi_\im^\truc$ respectively,
are easily seen to be given by~$\varphi_\re^\truc = k$
and~$\varphi_\im^\truc = \i\, k$, %
and the complex kernel and pseudo-kernel follow:
\begin{equation*}
  k^\truc = \varphi_\re^\truc - \i \varphi_\im^\truc = 2k
  \quad \text{and} \quad
  c^\truc = \varphi_\re^\truc + \i \varphi_\im^\truc = 0.
\end{equation*}

\bigbreak Now let~$H$ denote the subset of all the
functions~$f \in H_\Cset$ that satisfy~\eqref{equ:symmetry}: $H$ is
clearly a real subspace of~$H_\Cset$, and thus $\left( H, \ipR \right)$
is a complex/real RKHS as well. %
Moreover, for any $f \in H$,
\begin{align*}
  \Re\, f(s)
  & \;=\; \Re\, \left\{
    \frac{1}{2}\, \left( f(s) + f(s^*)^* \right) \right\}\\
  & \;=\; \frac{1}{2}\, \Bigl\{
    \left< f,\, \varphi_\re^\truc(\cdot,s) \right>_\Rset
    + \left< f,\, \varphi_\re^\truc(\cdot,s^*) \right>_\Rset
    \Bigr\}\\
  & \;=\; \left< f,\,
    \frac{1}{2}\, \left(
    \varphi_\re^\truc(\cdot,s) + \varphi_\re^\truc(\cdot,s^*) \right)
    \right>_\Rset.
\end{align*}
As a consequence of~\eqref{equ:assumpt-iii}, the function
$s \mapsto \frac{1}{2}\, \left( \varphi_\re^\truc(\cdot,s) +
  \varphi_\re^\truc(\cdot,s^*) \right)$ in this inner product satisfies 
	\begin{align*}
		&\frac{1}{2}\, \left( \varphi_\re^\truc(s_0,s)^* +
  \varphi_\re^\truc(s_0,s^*)^* \right) \\
  	&= \frac{1}{2}\, \left( \varphi_\re^\truc(s,s_0) +
  \varphi_\re^\truc(s^*,s_0) \right)
	\end{align*}  
  is an element of~$H$, which proves that the real evaluation
functional~$\varphi_\re$ of~$\left( H, \ipR \right)$ is given by
\begin{equation*}
  \varphi_\re(s, s_0)
  \;=\; \frac{1}{2}\, \left(
    \varphi_\re^\truc(s,s_0) + \varphi_\re^\truc(s,s_0^*) \right)
  \;=\; \frac{1}{2}\, \left(  k(s,s_0) + k(s,s_0^*) \right).
\end{equation*}
Similarly for the imaginary evaluation functional~$\varphi_\im$:
\begin{equation*}
  \varphi_\im(s, s_0)
  \;=\; \frac{1}{2}\, \left(
    \varphi_\im^\truc(s,s_0) - \varphi_\im^\truc(s,s_0^*) \right)
  \;=\; \frac{\i}{2}\, \left(  k(s,s_0) - k(s,s_0^*) \right).
\end{equation*}
Therefore $\varphi_\re - \i \varphi_\im = k$ is the complex kernel
of~$\left( H, \ipR \right)$, which proves~$i)$.

To prove the remaining assertions, assume that~$i$--$iii)$ hold. %
Let $G$ denote the closed linear span
of~$\left\{ k(\cdot,s_0);\; s_0 \in \Sset \right\}$ over~$\Rset$. %
Then we have $G + \i G = H_\Cset$, and it follows
from~\eqref{equ:assumpt-iii} that~$G \subset H$. %
Observing that
\begin{equation*}
  \i H = \left\{
    f \in H_\Cset \mid \forall s \in \Sset,\, f(s^*) = - f(s)^*
  \right\},
\end{equation*}
we conclude that $H \cap \i H = \{ 0 \}$, therefore $G = H$ and
$H \oplus \i H = H_\Cset$. %


\subsection{Proof of Theorem~\ref{thm:existence-uniqueness-hermit}}
Observe first that, without loss of generality, we can add $m$~extra
data points~$(s_i, y_i)$, for some $m \le n$, in such way that 1) the
points~$s_i \in \Sset$ ($1 \le i \le n+m$) are still distinct, and 2)
for each~$i$ we have $s_j = s_i^*$ and~$y_j = y_i^*$ for some~$j$.

\paragraph{Existence} %
Since $k$ is strictly positive definite, we can find~$\alpha_1$,
\ldots, $\alpha_{n+m} \in \Cset$ such that
$h = \sum_{i=1}^{n+m} \alpha_i k(\cdot, s_i)$ interpolates the
extended data~$(s_1, y_1)$, \ldots, $(s_{n+m}, y_{n+m})$. %
This function $h$ belongs to~$H_\Cset$ but not in general to~$H$. %
Set $g(s) = \frac{1}{2} \left( h(s) + h(s^*)^* \right)$. %
Then $g$ clearly satisfies the symmetry condition ($g(s^*) = g(s)^*$
for all~$s \in \Sset$) and still interpolates the extended
data~$(s_1, y_1)$, \ldots, $(s_{n+m}, y_{n+m})$. %
Moreover, using iii) from Theorem~\ref{thm:hermitian}, we obtain that
\begin{equation*}
  g(s) = \frac{1}{2} \sum_{i=1}^{n+m}
  \left( \alpha_i k(s, s_i) + \alpha_i^* k(s, s_i^*) \right),
\end{equation*}
which shows that~$g \in H_\Cset$, and thus $g \in H$. %
Besides, we easily see using~\eqref{equ:pseudo-kern-symm} that: if
$s_i = s_i^*$ then
\begin{equation}
  \frac{1}{2} \Bigl\{
    \alpha_i k(s, s_i) + \alpha_i^* k(s, s_i^*)
  \Bigr\}
  =
  \gamma_i k(s, s_i) + \gamma_i^* c(s, s_i)
\end{equation}
with $\gamma_i = \frac{1}{2} \alpha_i$, and if $s_j = s_i^*$ with
$i \neq j$ then
\begin{equation} \label{equ:trotro}
  \begin{gathered}
    \frac{1}{2} \Bigl\{
    \bigl( \alpha_i k(s, s_i) + \alpha_i^* k(s, s_i^*) \bigr)
    +
    \bigl( \alpha_j k(s, s_j) + \alpha_j^* k(s, s_j^*) \bigr)
    \Bigr\}\\
    =
    \bigl(
      \gamma_i k(s, s_i) + \gamma_i^* c(s, s_i)
    \bigr) + \bigl(
      \gamma_j k(s, s_j) + \gamma_j^* c(s, s_j)
    \bigr)
  \end{gathered}  
\end{equation}
with $\gamma_i = \frac{1}{2} \left( \alpha_i + \alpha_j^* \right)$
and~$\gamma_j = 0$. %
It follows that $g$ can be rewritten under the
form~\eqref{equ:interpolant:cr}, using the fact that~$\gamma_j = 0$
in~\eqref{equ:trotro} to get rid of the $m$ extra terms. %
Thus $\gamma = \left( \gamma_1, \ldots, \gamma_n \right)^\tra$
solves~\eqref{equ:interp-system:cr}, %
which proves the ``existence'' part of the theorem.

\paragraph{Uniqueness} %
Let $g \in H$ denote a function of the
form~\eqref{equ:interpolant:cr}, where the coefficients~$\gamma_i$ are
such that \eqref{equ:interp-system:cr}~holds. %
Using the property that $c(s, s_i) = k(s, s_i^*)$, any such function
can be rewritten as $g = \sum_{i=1}^{n+m} \alpha_i k(\cdot, s_i)$. %
Moreover, since the $s_i$'s are $n + m$ distinct points in~$\Sset$ and
$k$~is strictly positive definite, the
coefficients~$\alpha_i \in \Cset$ are uniquely determined by the
interpolation conditions: $g(s_i) = y_i$, $1 \le i \le n+m$. %
The first $n$ conditions come directly
from~\eqref{equ:interp-system:cr}, and the $m$ additional conditions
must hold as well by symmetry, since $g \in H$. %

For each~$i$ such that $s_i = s_i^*$, it is easily seen that
$\alpha_i = \gamma_i + \gamma_i^*$ is real, and thus the value
of~$\gamma_i$ is uniquely determined by~$\alpha_i$ and the additional
condition that~$\gamma_i = \gamma_i^*$. %
Similarly, if $s_i = s_j^*$ for some $i, j \le n$, $i \neq j$, then
$\alpha_i = \gamma_i + \gamma_j^*$,
$\alpha_j = \gamma_i^* + \gamma_j$, and therefore $\gamma_i$,
$\gamma_j$ are uniquely determined by $\alpha_i$, $\alpha_j$ and the
condition~$\gamma_i = \gamma_j^*$. %
Finally, if $s_i = s_j^*$ for some $i \le n$ and $j > n$, then
$\alpha_i = \gamma_i$. %
We have thus proved that there is a unique
$\gamma = \left( \gamma_1, \ldots, \gamma_n \right)^\tra$, with the
property that $\gamma_i = \gamma_j^*$ when $s_i = s_j^*$, such that
\eqref{equ:interp-system:cr}~holds.

\bigskip
