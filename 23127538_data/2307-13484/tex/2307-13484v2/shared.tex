\usepackage{lipsum}
\usepackage{amsfonts,bm}
\usepackage{graphicx}
\usepackage{epstopdf}
\usepackage{enumerate}
\usepackage{subcaption}
\usepackage{tikz}
\usepackage{pgfplots}
\usepackage{tudacolors}
\usepackage{siunitx}

\usepackage{multibib}
\newcites{SM}{Additional references}

\ifpdf
  \DeclareGraphicsExtensions{.eps,.pdf,.png,.jpg}
\else
  \DeclareGraphicsExtensions{.eps}
\fi

\def\tit{Rational kernel-based interpolation for complex-valued frequency response functions}
\def\kw{Complex-valued kernel methods, dynamical systems, frequency response function, model selection, rational approximation}

\hypersetup{
  bookmarksnumbered=true,
  pdfauthor={Julien Bect, Niklas Georg, Ulrich Römer, Sebastian Schöps},
  pdftitle={\tit},
  pdfsubject={},
  pdfkeywords={\kw}}

% Add a serial/Oxford comma by default.
\newcommand{\creflastconjunction}{, and~}

% Used for creating new theorem and remark environments
\newsiamremark{remark}{Remark}
\newsiamremark{hypothesis}{Hypothesis}
\crefname{hypothesis}{Hypothesis}{Hypotheses}
\newsiamthm{claim}{Claim}

% Sets running headers as well as PDF title and authors
\headers{Kernel-based interpolation for complex-valued functions}{J. Bect, N. Georg, U. Römer and S. Schöps}

% Title. If the supplement option is on, then "Supplementary Material"
% is automatically inserted before the title.
\title{\tit\thanks{Submitted to the editors DATE.
\funding{The work of N. Georg and U. Römer was funded by the Deutsche Forschungsgemeinschaft (DFG, German Research Foundation) -- RO4937/1-1. The work of N. Georg is also supported by the Graduate School CE within the Centre for Computational Engineering at Technische Universität Darmstadt.}}}

% Authors: full names plus addresses.
\author{
  Julien Bect\footnotemark[3]
\and Niklas Georg\footnotemark[2]\ \footnotemark[4]
\and Ulrich Römer\footnotemark[2]
\and Sebastian Schöps\footnotemark[4]}

\usepackage{amsopn}
\DeclareMathOperator{\diag}{diag}
\DeclareMathOperator{\vect}{span}

\newcommand \Cset {\mathbb C}
\newcommand \Dset {\mathbb D}  % unit disk
\newcommand \Kset {\mathbb K}  % \Rset or \Cset
\newcommand \Nset {\mathbb N}  % non-negative integers
\newcommand \Rset {\mathbb R}
\newcommand \Xset {\mathbb X}
\newcommand \Sset {\mathbb S}

\newcommand \Esp  {\mathsf E}

\newcommand   \Acal {\mathcal{A}}
\renewcommand \i    {\mathrm{i}}    % Complex imaginary unit
\newcommand   \Hol  {\mathrm{Hol}}  % Holomorphic

% To avoid ambiguity, we should avoid the use of \subset to mean "subset of equal".
\renewcommand \subset {\subseteq}

\newcommand \re  {\mathrm{R}}    % Real part
\newcommand \im  {\mathrm{I}}    % Imaginary part
\newcommand \LOO {\mathrm{LOO}}  % Leave-One Out

\newcommand{\tra}{{\mskip -1mu\scriptstyle\mathrm{T}}}
\newcommand{\her}{{\mskip -1mu\scriptstyle\mathrm{H}}}

\newcommand{\dt}{\mathrm{d}t}

\sloppy
