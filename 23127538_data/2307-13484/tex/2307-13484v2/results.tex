We apply the presented approximation techniques to a number of
benchmark functions from different fields. %
We always employ $n$ training points $(\omega_i,f(\i\omega_i)),$ where
the $\omega_i$ are equidistant frequency points in
$[\omega_{\min},\omega_{\max}]$, for simplicity. %
The accuracy of different approximations is then quantified in terms
of the root-mean-square error (RMSE), which is evaluated on a refined
equidistant grid with $201$ points for all numerical examples.

In the following we give a few details on the implementation. %
For AAA \cite{nakatsukasa2018aaa}, we rely on the implementation of
the \texttt{chebfun} toolbox \cite{driscoll2014chebfun}. %
For VF, we employ the \texttt{VectFit3} toolbox
\cite{gustavsen1999rational,gustavsen2006improving,deschrijver2008macromodeling},
where we use complex equidistant starting poles distributed according
to the general recommendation, and always run 30 iterations. %
We apply the ``relaxed non-triviality constraint''
\cite{gustavsen2006improving}, include the constant but not the linear
term, and enforce stable poles. %
The number of complex starting pole pairs is set to the maximum number
of $2\lfloor \frac {n-1}2\rfloor$, which leads to the best results for
the smooth test functions considered. %
For kernel interpolation we consider a separate interpolation of the
real and imaginary part with the squared exponential kernel (SE) and
complex/real interpolation with the Szegö kernel. %
The latter is also considered in combination with an adaptive rational
basis (Sz.-Rat.)\ as described in Section~\ref{sec:hybrid}. %
The implementation is done in \texttt{Matlab} as well, based on the
\texttt{STK} toolbox \cite{bect:2023:stk}. %
To this end, we employ the mapping $\mathcal A$ defined in
\eqref{eq:mappingNonIntrusive} for the complex/real RKHS
interpolation, which allows to realize the implementation based on
real RKHS interpolation on an augmented input space
$\Omega \times\{0,1\}$. %
Note that this approach could be employed with any toolbox for real
RKHS interpolation that provides the option to specify custom kernel
functions. %
The tuning of the hyper-parameters and poles based on the likelihood
function (see Section~\ref{sec:alg}) is carried out using
\texttt{fmincon} in Matlab, i.e., gradient-based optimization (more
precisely an interior point algorithm), which we combine with a
multistart procedure; see Supplementary Material for more details.

\begin{remark}
  By investigating the shape of the likelihood function for a number
  of benchmark problems, we have found that the logarithmic
  reparameterization, discussed in \cite{basak2021numerical} for
  instance, is not beneficial for the parameter $\alpha$. %
  Hence, it is only applied to the scaling parameter~$\sigma$.
\end{remark}


\subsection{Electric circuit (high order rational function)}

We consider in the following a parallel connection of $N$ underdamped
series RLC circuits, as illustrated on the left side in
Figure~\ref{fig:Electric_circuit}. %
The admittance is given as
\begin{equation}
  Y(s) = \sum_{i=1}^N \frac
  {s}{s^2 L_i +s R_i + C_i^{-1}}=\sum_{i=1}^N \frac {c_i}{s-a_i}+\frac
  {c_i^*}{s -a_i^*},\label{eq:admittance}
\end{equation}
where $\Re[a_i] = -\frac{R_i}{2L_i}$ (an explicit representation of
the poles $a_i$ and residues $c_i$ is given in the Supplementary
Material) and we consider the frequency range
$[\SI{10}{kHz}, \SI{25}{kHz}]$. %
First, we assume $N_1=1000$ random series RLC elements, where
$C_i\sim \mathcal U(1,20)\,\SI{}{\micro\farad}$ and
$L_i\sim\mathcal U(0.1,2)\,\SI{}{\milli\henry}$, %
and we assume the resistance $R_i$ to be roughly proportional to the
inductance, with random variations of $\pm 20\%$:
$R_i=L_i(1+\Delta)\, \SI{}{\ohm}(\SI{}{\milli\henry})^{-1}$, where
$\Delta \sim \mathcal U(-0.2, 0.2)$.

Note that for any combination of those parameters, the corresponding
series RLC circuits are underdamped. %
For one particular realization, the distribution of the $2N = 2000$
poles is illustrated in Figure~\ref{fig:Electric_circuit}. %
The corresponding admittance $Y_1(\i\omega)$ is shown in
Figure~\ref{fig:fun_circuit} with dashed black lines. %
We then conduct a convergence study for the particular realization of
the electric circuit, which is shown in
Figure~\ref{fig:ConvStudyCircuit} (top, left). %
We repeat the convergence study for 100 random realizations and depict
the median RMSE at each point in Figure~\ref{fig:ConvStudyCircuit} (top,
right). %
It can be observed that for the considered range of the number of
training points (where $n \le 60 \ll N$) the complex/real Szegö kernel-based interpolation outperforms
AAA and~VF. %
Employing the hybrid algorithm (Sz.-Rat.)\ does not yield an
improvement, but leads to similarly good results.

% Figure environment removed

% Figure environment removed

% Figure environment removed

In our second experiment, we introduce two additional circuit elements
with a very small damping, i.e. we now consider $N_2 = 1002$ and
\begin{align*}
  C_{1001} = \SI{5}{pF,} && L_{1001} =\SI{1}{mH}, && R_{1001}=\SI{0.1}{\ohm},\\
  C_{1002} = \SI{2}{pF}, && L_{1002} =\SI{1}{mH}, && R_{1002}=\SI{0.1}{\ohm}.
\end{align*}
This leads to two additional poles which are closer to the input
domain, as illustrated by the red crosses in
Figure~\ref{fig:Electric_circuit}. %
The corresponding admittance $Y_2(\i \omega)$ differs very little from
$Y_1(\i\omega)$, except for two sharp peaks, as can be seen in
Figure~\ref{fig:fun_circuit}. %
However,
the accuracy of the respective RKHS interpolation is significantly
affected. %
In particular, at the bottom of Figure~\ref{fig:ConvStudyCircuit}, it
can be observed that the convergence order of Szegö kernel
interpolation is significantly reduced. %
By adding the rational basis we are able to mitigate the impact of the two
dominant poles: it exhibits fast convergence and an improvement
w.r.t.\ AAA and~VF can again be observed.


\subsection{PDE-based examples}

% Figure environment removed

In the following, we investigate a number of PDE-based examples. %
We start with the acoustic Helmholtz equation, in particular, the
PAC-MAN benchmark example, introduced in \cite{ziegelwanger2017pac}
which is also included in the platform for benchmark cases in
computational acoustics from the European Acoustics Association
\cite{hornikx2015platform}. %
The model, shown in Figure~\ref{fig:Pacman}, has the PAC-MAN shape with
an opening angle of $30^\circ$ and radius of $\SI{1}{m}$. %
As in \cite[Section 6.1]{ziegelwanger2017pac}, we consider as
excitation a vibration of the surface of the PAC-MAN with cylindrical
modes and observe the radiated field $p_i$ at a point in \SI{2}{m}
distance at an angle of $10^\circ$. %
As in \cite{hornikx2015platform}, the computation was done based on
the implementation of the analytical solution provided in
\cite{ziegelwanger2017pac} by replacing the python module
\texttt{scipy} by \texttt{mpmath} for the computation of higher order
Bessel functions. %
In particular, we set the truncation order to~300. %
The complex acoustic pressure field phasor $p_i$ of the total
sound-field versus the frequency
$f\in[\SI{2000}{\hertz},\SI{4000}{\hertz}]$ is shown in
Figure~\ref{fig:Pacman} (top, right). %
We then conduct a convergence study w.r.t.\ the number of training
points, which is depicted in Figure~\ref{fig:Pacman} (bottom, left). %
It can be observed that the complex/real Szegö kernel-based interpolation
outperforms the alternative approaches in the range up to about 40
training points. %
Adding the rational mean function does not further improve the
accuracy, but does not harm the accuracy either.

% Figure environment removed

Next, we consider an electromagnetic model problem, which is a
demonstration example of CST Microwave Studio \cite{CST_2019aa},
solving the full set of Maxwell equations in the frequency domain. %
The model consists of a waveguide junction with 4 ports, which
contains a small metallic disk and is connected to an external cavity
resonator (see Figure~\ref{fig:WGjunction}). %
The structure is excited at the first port and simulated using the
finite element method in the frequency domain. %
In particular, we set the solver accuracy of the 3rd order solver to
$10^{-6}$ and use a curved mesh with standard settings. %
We employ an initial adaptive mesh refinement at \SI{9}{GHz}, where we
set the scattering parameter 
criterion threshold with 2 subsequent checks to
$10^{-4}$. %
As quantity of interest we consider the scattering parameters on a
frequency range of [\SI{7}{GHz},\,\SI{9}{GHz}] using equidistant
sample points, where we restrict ourself to $S_{21}$ and $S_{41}$ for
brevity, however, the results are qualitatively similar for all four
scattering parameters. %
It can be seen that, the QoIs have a dominant pole at around
\SI{8}{GHz}. %
This causes the purely kernel-based interpolations to be inferior
compared to the rational approximations. %
However, the proposed combination of kernel-based interpolation and
rational approximations leads to satisfactory results, with an
accuracy comparable to that of~AAA and~VF.%

% Figure environment removed

The final test case is a vibroacoustic finite element model, taken
from \cite{Roemer_2021aa} and depicted in
Figure~\ref{fig:Vibroacoustic}. %
A 2D Mindlin plate (vibrating structure $D_s$) is excited by a point
force and strongly coupled to a 3D acoustic domain (air
cavity~$D_f$). %
Then, the response at a point in the fluid is evaluated. %
See \cite{Roemer_2021aa} for more details on the model. %
We consider the frequency response on a frequency interval
$\omega \in [\SI{4500}{\per\second},\SI{5000}{\per\second}]$, shown in
Figure~\ref{fig:Vibroacoustic} (top, right). %
The convergence study, given in Figure~\ref{fig:Vibroacoustic} (bottom),
indicates that the proposed approach usually achieves an accuracy at
least comparable to that of~AAA and~VF, with at certain points an
improvement by about an order of magnitude can be observed. %
It can also be seen that the rational mean function improves the
accuracy at the majority of points compared to the pure Szegö
kernel-based interpolation.%
