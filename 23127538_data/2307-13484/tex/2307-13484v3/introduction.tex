We consider dynamical systems of the form
\begin{equation}
  \label{eq:generalDiscretePDEDynamic}
  \mathbf{M} \ddot{\mathbf{u}}(t) + \mathbf{D} \dot{\mathbf{u}}(t)
  + \mathbf{K}  \mathbf{u}(t) = \mathbf g(t),
\end{equation}
to be endowed with initial conditions and
$\mathbf K, \mathbf D, \mathbf M \in \mathbb R^{n_h\times n_h}$,
$\mathbf{u}(t),\mathbf{g}(t) \in \mathbb R^{n_h}$. %
We are in particular interested in approximating scalar time-dependent
quantities derived from the solution, of the form
\begin{equation}
  \label{eq:generalScalarQuantity}
  f(t) =  \mathbf{j}^\tra \mathbf{u}(t),
  \qquad \mathbf{j} \in \mathbb{R}^{n_h},
\end{equation}
which are commonly used to assess engineering designs. %
System \eqref{eq:generalDiscretePDEDynamic} may stem from a partial
differential equation after spatial discretization with $n_h$ degrees
of freedom. %
In a mechanics context, $\mathbf{K},\mathbf{D},\mathbf{M}$ are
referred to as stiffness, damping and mass matrix, but problems
arising in many areas of science and engineering can be brought into
this form. Our numerical results will cover electromagnetic and
acoustic field problems in particular.  In view of the linearity of
the equation, a frequency domain analysis is often adopted. %
Assuming for simplicity that~$\mathbf{u}$ and~$\dot{\mathbf{u}}$
vanish at~$t = 0$, the (one-sided) Laplace transform
of~\eqref{eq:generalDiscretePDEDynamic}--\eqref{eq:generalScalarQuantity}
with respect to the time variable~$t$ is
\begin{equation}
  \label{eq:generalDiscretePDE}
  \begin{split}
    \left( s^2 \mathbf{M}  + s \mathbf{D} + \mathbf{K} \right)\, \hat{\mathbf{u}}(s)
    &= \hat{\mathbf{g}}(s), \\
    \hat{f}(s) &= \mathbf{j}^\tra \mathbf{\hat{u}}(s),
  \end{split}
\end{equation}
where $s$ denotes the complex frequency variable, also known as the
Laplace variable. %
Assuming a suitably normalized excitation $\hat{\mathbf{g}}(s)$, the
frequency response function is defined as the value
$\omega \mapsto \hat f (\i \omega)$ of~$\hat f$ on the imaginary axis,
where $\omega$ is called the angular frequency, and we are typically
interested more specifically in its value on a certain interval
$\Omega = \left[ \omega_{\min},\omega_{\max} \right] \subseteq
[0,+\infty)$. %
In the following, we omit explicitly indicating frequency domain
variables to simplify the notation.

The location of the poles of $\hat f$ strongly depends on the
properties of $\mathbf{K},\mathbf{D},\mathbf{M}$, see
\cite{tisseur2001quadratic}. %
We assume, in particular, that no pole is placed on the frequency axis
$\i \mathbb{R}$ and that the frequency response function is
holomorphic on the shifted right
half-plane~$\Gamma_\alpha = \{s \in\mathbb C \,\mid
\,\Re[s]>-\alpha\},\,\alpha>0$. The real parts of all poles are strictly negative
for instance if $\mathbf{K},\mathbf{D},\mathbf{M}$ are symmetric
positive definite, see Section 3 of \cite{tisseur2001quadratic}. %
The same holds true if the homogeneous version of
\eqref{eq:generalDiscretePDEDynamic} is stable, in the sense that all
solutions decay exponentially to zero as $t \rightarrow \infty$. %
The holomorphy of response functions has recently been studied also in
the context of partial differential equations, see \cite[Proposition
5.3]{Bonizzoni_2020aa} for instance. %
There, the frequency response map for an acoustic scattering problem
was studied and appropriate damping terms ensured a locally
holomorphic response function, with a negative real part for all
poles\footnote{Because of a different convention
  \cite{Bonizzoni_2020aa} establishes a negative \emph{imaginary} part
  of the eigenvalues}.%

\begin{remark}
  The method introduced in this paper was motivated by problems of the
  form~\eqref{eq:generalDiscretePDEDynamic}, but can be readily
  applied to the approximation of the frequency response function of
  any linear, time-invariant and asymptotically stable dynamical
  system.
\end{remark}


Adopting a data-driven approach, ~\eqref{eq:generalDiscretePDE} must
be solved repeatedly on a set of interpolation/training points
$\omega_i \in \Omega$, with $s_i = \i \omega_i$. %
Numerical efficiency demands a small training set
\begin{equation}
  \left( \omega_i,\, f(\i\omega_i) \right)_{1 \le i \le n},
  \quad
  \mathrm{where}~\omega_i\in\Omega,\, f(\i\omega_i)\in\mathbb C,\,
  i=1, \ldots, n.\label{eq:training_set}
\end{equation} %% Equation could be removed
Hence, there is a need for accurate interpolation in the frequency
domain.

The data-driven approximation of frequency response functions has
attracted considerable interest in the literature, see for instance
\cite{gustavsen1999rational,lataire2016transfer, nakatsukasa2018aaa}
and the references therein. %
Among the numerous available approaches we mention vector fitting
\cite{gustavsen1999rational} and the adaptive Antoulas-Anderson (AAA) method
\cite{nakatsukasa2018aaa} in particular, which are widely used,
state-of-the-art approximation methods.

Vector Fitting (VF) is a rational approximation technique,
specifically tailored to functions in the frequency domain. %
It is based on a representation in terms of partial fractions as
\begin{equation}
  f(\i\omega) \approx \sum_{m=1}^{M} \frac{r_m} {\i\omega-p_m}
  + d +\i\omega h,\label{eq:part_frac}
\end{equation}
where the $M$ poles $p_m$ are relocated in each iteration by solving a
linear least-square problem, see
\cite{gustavsen2006improving,gustavsen1999rational} for details. %
The implementation guarantees that all poles are stable,
i.e. $\mathcal R[p_m]<0$, and are either real or come in
complex-conjugate pairs. %

The AAA method \cite{nakatsukasa2018aaa}
employs the barycentric interpolation
\begin{equation}
  \label{eq:AAA}
  f(\i\omega) \approx  r(\omega)
  = \frac{n(\omega)}{d(\omega)}
  = \frac{\sum_{j\in J}{\frac {w_jf(\i \omega_j)}{\omega-\omega_j}}}{\sum_{j\in J} \frac{w_j}{\omega-\omega_j}},
\end{equation} 
where $J \subseteq \{1, \ldots, n\}$ has cardinality~$m$. %
The rational function in \eqref{eq:AAA} is of type $(m-1,m-1)$, which
can be seen by multiplying both numerator and denominator by
$\prod_{j\in J} (\omega-\omega_j)$. %
Moreover, $r(\omega_j) = f(\i \omega_j)$ for all $j\in J$. %
The weights~$w_j$ and nodes~$\omega_j$, $j\in J$, are determined
adaptively in a two-step procedure, based on linear least squares
problems and a greedy strategy \cite{nakatsukasa2018aaa}.

Other data-driven approaches, related to rational interpolation and
model order reduction are the Loewner framework
\cite{antoulas2017tutorial} and the recent contribution
\cite{nobile2020non}, which employs the Heaviside representation. 
A Bayesian rational Polynomial Chaos-type model has been put forth in \cite{schneider2023sparse} to capture the effect of uncertain parameters, e.g., on frequency response functions. 
A complex-valued version of support vector machine regression has been
presented in \cite{treviso2021multiple}, which is restricted to the
so-called circular case with a single kernel only. %
Complex interpolation with a pair of kernels has been addressed in
\cite{boloix2017widely,picinbono1995widely} and also from a Gaussian
process regression perspective in \cite{boloix2018complex,
  hallemans2022frf}. 

Despite recent progress with complex kernel methods, a general
framework with a complete mathematical background on the underlying
reproducing kernel Hilbert spaces (RKHS) is missing. %
In comparison to parametric rational approximation methods, e.g., AAA
and vector fitting, the kernel/Gaussian process approach is appealing
because of its principled statistical foundations, which allow for
model selection, uncertainty quantification and adaptive sampling. %
Additionally, desired properties of the system, such as stability and
causality, can be ensured during kernel design \cite{hallemans2022frf}. %
Adaptive sampling, in particular, is more involved for Loewner-type approaches. In the standard formulation of AAA, for instance, new support points are chosen from a discrete set of a priori fixed points. %
An exception is the recently introduced Greedy-type adaptive sampling
Loewner approach in \cite{pradovera2023toward}.%
 
In this paper, we introduce a new kernel-based interpolation method
which is well adapted to frequency responses. %
We will put special emphasis on the complex-valued setting and show
that the data are used more efficiently if a dedicated kernel method
is constructed and interpolation of the real and imaginary part
individually is avoided. %
To address problems with a few dominant poles we include a low-order
rational basis into the kernel method and present a new model
selection scheme. %
We compare our rational kernel-based interpolation method against both
AAA and vector fitting and observe an improved or at least comparable
performance for a variety of test cases. %
Finally, the paper develops the required notions of RKHS
and minimum norm interpolation for complex-valued kernel methods in
general.

The material is structured in the following way. %
In \Cref{sec:theory} we introduce the concept of a complex/real
kernel Hilbert space and consider the special case of frequency
response functions as well as the connections to complex-valued
Gaussian process regression. %
\Cref{sec:alg} introduces our new method, which employs a
kernel, a pseudo-kernel and an additional rational basis for capturing
dominant poles. %
Finally, \Cref{sec:numerics} reports several examples from
PDE-based applications, comparing our method to AAA and vector fitting
before conclusions are drawn.

\medbreak\itshape

Nota bene: %
A method sharing some similarities with the one proposed in
Section~\ref{sec:alg} has been published recently in the automatic
control literature \cite{hallemans2022frf}. %
We became aware of it at very late stage in the writing of the present
article. %
After introducing our new method in Section~\ref{sec:alg}, we discuss
similarities and differences in Remark~\ref{rmk:hallemans}.

\medbreak\normalshape
