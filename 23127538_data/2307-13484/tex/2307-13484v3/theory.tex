In order to address kernel-based interpolation of the frequency
response function, we start by recalling basic facts on RKHSs; %
see, e.g., \cite{Paulsen_2016aa} for a comprehensive introduction to
this topic. %
Then, new results establishing the theoretical basis of our method
are stated, the proofs of which can be found in
Appendix~\ref{sec:proofs-appendix}.

\begin{definition}[Complex RKHS] \label{def:complex-rkhs}
  A complex RKHS~$H$ over a non-empty set~$\Sset$ is a complex Hilbert
  space of functions~$\Sset\rightarrow\mathbb C$ such that, for all
  $s \in \Sset$, the evaluation functional
  $\delta_s: H \rightarrow \Cset$, $f \mapsto f(s)$, is continuous.
\end{definition}

The Riesz representation theorem implies that there exists a unique
function $k:\Sset\times \Sset \rightarrow \Cset$, called the
reproducing kernel of~$H$, such that $k(\cdot, s) \in H$ and
\begin{equation} \label{equ:reproduction-property}
  f(s) = \delta_s(f) = \left< f, k(\cdot,s) \right>_H
\end{equation}
for all~$s\in \Sset$ and~$f \in H$, %
where $\left< \cdot,\cdot \right>_H$ denotes the Hermitian inner
product of~$H$. %
Equation~\eqref{equ:reproduction-property} is called the reproduction
property, and it is easily seen that the kernel~$k$ is
\emph{Hermitian} %
(i.e., $k(s, s_0) = k(s_0, s)^*$ for all~$s, s_0 \in \Sset$) and
\emph{positive definite}: for all $n \in \Nset^*$ and all~$(s_1, \alpha_1)$,
\ldots, $(s_n, \alpha_n) \in \Sset \times \Cset$,
\begin{equation}
	\label{eq:kernel_positiveDefinite}
  \sum_{1 \le i, j\le n} \alpha_i^* \alpha_j k(s_i, s_j) \;\ge\; 0.
\end{equation}

\begin{theorem}[Moore-Aronszajn]
	\label{thm:Moore-Aronszajn}
  For any positive definite Hermitian
  kernel~$k: \Sset\times \Sset \rightarrow \Cset$, there exists a
  unique complex Hilbert space~$H$ of functions on~$\Sset$ such that
  the reproduction property holds with reproducing kernel~$k$.
\end{theorem}

Real RKHSs are defined similarly, replacing~$\Cset$ by~$\Rset$ in
Definition~\ref{def:complex-rkhs}: %
in this case $H$~is a real Hilbert space, the reproducing kernel is
symmetric positive definite, and a suitably modified statement of the
Moore-Aronszajn theorem holds as well.

\begin{theorem}[Interpolation] %
  \label{thm:complexRKHSinterpolation} %
  Let $H$ be a real or complex RKHS over~$\Sset$ %
  with kernel $k: \Sset \times \Sset \to \Kset$, %
  where $\Kset = \Rset$ or~$\Cset$ depending on the type of RKHS. %
  Let $n \in \Nset^*$, $s_1, \ldots, s_n \in \Sset$ %
  and $y_1, \ldots, y_n \in \Kset$. %
  Then there exists a function $g \in H$ such that $g(s_i)=y_i$ for
  all $i \in \{ 1, \ldots, n \}$ if, and only if, the system
  \begin{equation}\label{equ:interp-system}
    \begin{bmatrix}
      k(s_1,s_1) &\dots & k(s_1,s_n)\\
      \vdots & \ddots & \vdots\\
      k(s_n,s_1) &\dots & k(s_n,s_n)
    \end{bmatrix}
    \begin{bmatrix}
      \gamma_1\\
      \vdots\\
      \gamma_{n}
    \end{bmatrix}
    = \begin{bmatrix}
      y_1\\\vdots\\y_n
    \end{bmatrix}
  \end{equation}
  admits a solution. %
  Furthermore, for any solution of~\eqref{equ:interp-system},
  $g = \sum_{i=1}^n \gamma_i\, k(\cdot,s_i)$ is the unique interpolant
  of the data $\left( s_1, y_1 \right)$, \ldots,
  $\left( s_n, y_n \right)$ with minimal norm in~$H$.
\end{theorem}
A positive definite kernel is called \emph{strictly} positive definite
if the kernel matrix $K_n=(k(s_i,s_j))_{1\leq i,j\leq n}$ is
invertible (equivalently, if \eqref{eq:kernel_positiveDefinite}~is
strict for all~$\left( \alpha_1, \ldots, \alpha_n \right) \neq 0$)
whenever $s_1,\ldots,s_n$ are distinct points. %
This ensures that \eqref{equ:interp-system} has a unique solution.

We will proceed by introducing several complex RKHS and their
kernels. For $s \in \mathbb C$, let $\Re[s]$ and $\Im[s]$ denote the
real and imaginary part, respectively. %
An important example is the Hardy space $H^2(D)$ on the unit disc $D=\{s \in \Cset: |s| <1 \}$.
This space plays a role in the analysis of the stability of discrete
dynamical systems, see \cite{baratchart1991identification}, for
instance.  
% , which consists of all holomorphic functions in $D$, for which
% $f_r(\theta)=f(r\mathrm{e}^{\i \theta})$ is bounded in the
% $L^2$-norm, as $r\to 1$, see \cite[Chapter 3]{hoffman2007banach}. %
Here, in the context of continuous-time dynamical systems, we are more
interested in the corresponding Hardy space 
\begin{equation}
  H^2(\Gamma_\alpha) \!=\!
  \left\{
    f\in \Hol(\Gamma_\alpha)\!:\!\|f\|_{H^2(\Gamma_\alpha)}
    = \sup_{x>-\alpha}\left(\int_{-\infty}^{\infty} \left| f(x + \i y)^2 \right|\,\mathrm d y \right)^{\frac 1 2} <\infty
  \right\},
\end{equation}
where $\Hol(\Gamma_\alpha)$ denotes the space of holomorphic functions
on~$\Gamma_\alpha$. %
Note, that there is a Banach space isometry between the $H^2$ spaces on disc and half-plane, see \cite[Chapter
8]{hoffman:1962} for details. %
%The kernel of $ H^2(\Gamma_\alpha)$ is \emph{strictly positive
%  definite}, i.e. the kernel matrix
%$K_n=(k(s_i,s_j))_{1\leq i,j\leq n}$ is invertible, as stated in the
%following theorem.
\begin{theorem}\label{thm:gammaAlpha}
  The space $H^2(\Gamma_\alpha)$ is a complex RKHS, with strictly
  positive definite reproducing kernel~$k$ given by
  \begin{equation}\label{equ:szego} 
    k_\alpha\left( s, s_0 \right)
    = \frac{1}{2\pi\, \left( 2\alpha + s + s_0^*\right)},
    \quad s, s_0\in\Gamma_{\alpha}.
  \end{equation}
\end{theorem}

%A proof is given in Appendix~\ref{proof:thmGammaAlpha}. %
Following standard terminology in complex analysis (see, e.g., \cite{krantz:2001:function}), we will refer
to~$k_{\alpha}$ as the \emph{Szegö kernel} for the
domain~$\Gamma_\alpha$. %
Evaluating \eqref{equ:szego} only on the imaginary axis $s=\i \omega$,
the expression simplifies to
\begin{equation}
  k_\alpha\left( \i\omega, \i\omega_0 \right) =
  \frac 1 {2\pi\left( 2\alpha+\i(\omega - \omega_0) \right)},
  \quad \omega, \omega_0 \in \Omega.\label{eq:SzegoFrequency}
\end{equation}

We consider the stable spline kernel
\cite{pillonetto2010new,lataire2016transfer} as another example. This
kernel has been proposed in the time domain to model functions with a
certain smoothness, which additionally incorporate impulse response
stability \cite{pillonetto2010new}. The corresponding kernel for the
frequency domain transfer function has been obtained in
\cite{lataire2016transfer} and reads
\begin{multline}
  \label{eq:stablespline}
  k_\alpha\left( \i\omega, \i\omega_0 \right) =
  \frac{1}{2} \frac{1}{3 \alpha + \i (\omega - \omega_0)} \times \\
  \left( %
    \frac{1}{2 \alpha + \i \omega} + \frac{1}{2 \alpha - \i \omega_0} %
    - \frac{1}{3(3 \alpha + \i \omega)} - \frac{1}{3(3 \alpha - \i \omega_0)} %
  \right).
\end{multline}
Other related kernels can be found in the control literature, see \cite{lataire2016transfer,hallemans2022frf}.


\subsection{Complex/real RKHS interpolation}
\label{sec:complex-real-RKHS}

%% MOTIVATING EXAMPLE:
The frequency response function fulfills the symmetry property
$f^*(s) = f(s^*)$ for all~$s \in \Gamma_\alpha$, since it is the
Laplace transform of a real-valued function. %
We are thus naturally led to cast our interpolation problem not
in~$H^2(\Gamma_{\alpha})$ but in the subset
\begin{equation} \label{equ:H2sym}
  H^2_\mathrm{sym}(\Gamma_{\alpha}) \;=\; \bigl\{
  f\in H^2(\Gamma_{\alpha}):\; \forall s\in\Gamma_\alpha,\; f^*(s)=f(s^*)
  \bigr\}.
\end{equation}
This set of complex-valued functions, however, cannot by endowed with
the structure of a complex RKHS. %
In fact, it is not even a vector space over~$\Cset$: %
indeed, for any $f\in H^2_\mathrm{sym}(\Gamma_\alpha)$ and
$s\in \Gamma_{\alpha}$, we would have
$(\i f)^*(s) = -\i f^*(s) = -\i f(s^*)$ and
$(\i f)^*(s) = (\i f)(s^*) = \i f(s^*)$, %
which is a contradiction if $f(s^*) \neq 0$.

%% DEFINITION
Observing that the subset of~$H^2(\Gamma_{\alpha})$ defined
by~\eqref{equ:H2sym} is a real vector space of complex-valued functions,
we define in the following a new type of function space, which we call
a complex/real RKHS.
%
\begin{definition}[Complex/real RKHS] \label{def:CRrkhs}%
  Let $\Sset$ denote a non-empty set and let $H$ denote a real Hilbert
  space of complex-valued functions on~$\Sset$. %
  We say that $H$ is a complex/real RKHS if the evaluation functionals
  are continuous (i.e., for all $s \in \Sset$, the function
  $\delta_s: H \to \Cset$, $f \mapsto f(s)$, is continuous).
\end{definition}

In the remaining part of this section we will establish general
results related to these
spaces. Section~\ref{sec:complex-real-RKHS-sym} will then present
consequences for the RKHS with the symmetry property
$f^*(s) = f(s^*)$.

\begin{remark} \label{rem:complex-RKHS-subspaces}
  Any complex RKHS~$H$ (such as $H^2(\Gamma_{\alpha})$) can be seen as a
  complex/real RKHS by forgetting the complex structure, i.e., by
  considering $H$ as a real vector space, endowed with the real inner
  product $\left<f, g\right> \mapsto \Re \left( \left< f, g \right>_H \right)$. %
  More generally, any real subspace of~$H$ (such
  as~$H^2_\mathrm{sym}(\Gamma_{\alpha})$), endowed with this inner
  product, is clearly a complex/real RKHS. %
  The converse statement is false, however.
\end{remark}

\begin{proposition} \label{prop:counterexample-dim2}
  There exists a complex/real RKHS of dimension two over the reals
  that is not a real subspace of a complex RKHS.
\end{proposition}

The elements of a complex/real RKHS are complex-valued functions
over~$\Sset$, but can be conveniently represented as real-valued
functions over $\tilde\Sset = \Sset \times \{\re,\im\}$ through the
mapping $\Acal: \Cset^\Sset \rightarrow \Rset^{\tilde\Sset}$ defined
by
\begin{equation} \label{eq:mappingNonIntrusive}
  (\Acal f)(s,a) = G_a(f(s)), 
\end{equation}
where $G_\re(s) = \Re(s)$ and $G_\im(s) = \Im(s)$. %
This mapping defines an isometric isomorphism of real Hilbert spaces
between $H$ and the real vector space
$\tilde H = \Acal H \subset \Rset^{\tilde\Sset}$, endowed with the
image inner product. %
The image space $\tilde H$ is easily seen to be a real RKHS if and
only if~$H$ is a complex/real RKHS: %
this observation will be useful both from a theoretical point of view,
to establish properties of complex/real RKHSs, and from a practical
point of view
(see Section~\ref{sec:numerics}).

\begin{remark}
  Complex/real RKHSs can also been seen a special case of
  vector-valued RKHSs~\cite{burbea:1984:banach,
    micchelli:2005:vector}, through the usual identification
  of~$\Cset$ with~$\Rset^2$.
\end{remark}

The term ``functional'' is used in a loose sense in
Definition~\ref{def:CRrkhs}, since $H$ is a real vector space
while~$\delta_s$ is a complex-valued function. %
Therefore, in contrast with the usual case of complex RKHSs, the
continuous functionals~$\delta_s$, $s \in \Sset$, do not belong to the
topological dual of~$H$. %
The real and imaginary evaluation functions however---namely,
$\Re \circ \delta_s$ and~$\Im \circ \delta_s$---do belong to the
topological dual, and can thus be expressed through inner products.

\begin{proposition} \label{prop:repr-eval-func} %
  Let $H$ be a complex/real RKHS on a set~$\Sset$, and set %
  \begin{equation}
    k_{a\mskip 1mu a_0}(s, s_0) \;=\; \tilde k\left( (s, a),\, (s_0, a_0) \right),
    \qquad s, s_0 \in \Sset,
    \quad a, a_0 \in \{ \re, \im \},
  \end{equation}
  where $\tilde k$ denotes the reproducing kernel of~$\tilde H = \Acal H$. %
  Then, for all $s \in \Sset$, we have
  \begin{equation}
    \delta_s \;=\;
    \underbrace{\left<\, \bm{\cdot}\,,\, \varphi_\re(\cdot,s) \right>_H}_{\Re \circ\, \delta_s}
    \;+\; \i \underbrace{\left<\, \bm{\cdot}\,,\, \varphi_\im(\cdot,s) \right>_H}_{\Im \circ\, \delta_s},
  \end{equation}
  where %
  $\varphi_\re = k_{\re\re} \,+\, \i\, k_{\im\re}$ and %
  $\varphi_\im = k_{\re\im} \,+\, \i\, k_{\im\im}$.
\end{proposition}

This result associates to each complex/real RKHS a
pair~$\left( \varphi_\re, \varphi_\im \right)$ of kernels
$\varphi_{a}:\Sset \times \Sset \to \Cset$, $a \in \{ \re, \im\}$. %
Characterizing admissible choices for this pair of kernels, in the
spirit of Theorem~\ref{thm:Moore-Aronszajn} for complex RKHSs, %
is possible but not convenient. %
Instead, motivated by the connection between complex/real RKHSs and
complex Gaussian processes (to be discussed in
Section~\ref{sec:relation-GPs}), and in particular the work of
Picinbono \cite{picinbono:1996}, we introduce another pair of kernels
as follows.

\begin{definition} \label{def:complex-kernels} %
  Let $H$ denote a complex/real RKHS and let $k_{\re\re}$, $k_{\im\im}$,
  $k_{\re\im}$, $k_{\im\re}$, $\varphi_\re$ and~$\varphi_\im$ be defined as in
  Proposition~\ref{prop:repr-eval-func}. %
  Then we define the \emph{complex kernel}~$k$ of the complex/real
  RKHS as
  \begin{equation}
    \label{eq:complex-kernel}
    k %
    \;=\;  \left( k_{\re\re} + k_{\im\im} \right) %
    \,+\, \i\, \left( k_{\im\re} - k_{\re\im} \right)
    \;=\; \varphi_\re - \i \varphi_\im,
  \end{equation}
  and its \emph{pseudo-kernel} $c$ as:
  \begin{equation}
    \label{eq:pseudo-kernel}
    c %
    \;=\; \left( k_{\re\re} - k_{\im\im} \right) %
    \,+\, \i\, \left( k_{\im\re} + k_{\re\im} \right)
    \;=\; \varphi_\re + \i \varphi_\im.
  \end{equation}
\end{definition}

\begin{proposition}\label{prop:dense-subspace}
  The functions of the form
  $\gamma\, k(\cdot,s_0) + \gamma^*\, c(\cdot, s_0)$, with
  $\gamma \in \Cset$ and $s_0 \in \Sset$, span a dense subset of~$H$.
\end{proposition}

\begin{remark}
  Proposition~\ref{prop:dense-subspace} suggests that the concept of a
  complex/real RKHS, introduced in this article, provides a rigorous
  formalization of the idea of a ``wide-linear complex-valued RKHS''
  (WL-RKHS) proposed in~\cite{boloix2017widely} (see Definition~3.1).
\end{remark}

It can be shown that the complex/real RKHS obtained by forgetting the
complex structure of a complex RKHS with reproducing kernel~$k_0$, as
described in Remark~\ref{rem:complex-RKHS-subspaces}, is the
complex/real RKHS with complex kernel $k = 2 k_0$ and vanishing
pseudo-kernel---which, borrowing terminology from the signal
processing literature \cite{picinbono:1996}, can be called
\emph{circular}. %
The factor~$2$ in the relation between~$k$ and~$k_0$ is the price to
pay for the consistency of Definition~\ref{def:complex-kernels} with
the concepts of covariance and pseudo-covariance functions for complex
Gaussian processes (see Section~\ref{sec:relation-GPs}). %
More generally, we have the following characterization of the set of
admissible $(k, c)$ pairs.

\begin{theorem} \label{thm:CR-RKHS-characterization}\setlength{\parskip}{2pt}%
  For a given complex/real RKHS~$H$, the kernels~$k$ and~$c$
  introduced in Definition~\ref{def:complex-kernels} satisfy the
  following:
  \begin{enumerate}[i) ]
  \item $k$ is complex-valued, Hermitian and positive definite.
  \item $c$ is complex-valued and symmetric.
  \end{enumerate}
  Moreover, for all~$n \ge 1$ and all~$s_1, \ldots s_n \in \Sset$:
  \begin{enumerate}[i) ] \setcounter{enumi}{2}
  \item $\ker K_n \subset \ker C_n^*$ and,
  \item if $K_n$ is positive definite,
    $K_n^* - C_n^* K_n^{-1} C_n$ is positive semi-definite,
  \end{enumerate}
  where $K_n = \left( k(s_i,s_j) \right)_{1 \le i,j \le n}$ and
  $C_n = \left( c(s_i,s_j) \right)_{1 \le i,j \le n}$.

  Conversely, for any pair of functions
  $k, c:\Sset \times \Sset \to \Cset$ that satisfies these four
  properties, there exists a unique complex/real RKHS on~$\Sset$ with
  complex kernel~$k$ and pseudo-kernel~$c$.
\end{theorem}


\begin{theorem}[Interpolation in a complex/real RKHS] %
  \label{thm:interp:cr} %
  Let $H$ denote a complex/real RKHS over~$\Sset$ %
  with complex kernel~$k$ and pseudo-kernel~$c$. %
  Let $n \in \Nset^*$, $s_1, \ldots, s_n \in \Sset$ %
  and $y_1, \ldots, y_n \in \Cset$. %
  Then there exists a function $g \in H$ such that $g(s_i)=y_i$ for
  all $i \in \{ 1, \ldots, n \}$ if, and only if, the system
  \begin{equation}
    \label{equ:interp-system:cr}
    K_n \gamma + C_n \gamma^* = y
  \end{equation}
  admits a solution $\gamma \in \Cset^n$, where
  $K_n = \left( k(s_i,s_j) \right)_{1 \le i,j \le n}$,
  $C_n = \left( c(s_i,s_j) \right)_{1 \le i,j \le n}$,
  and $y = \left( y_1, \ldots, y_n \right)^\tra$. %
  Furthermore, for any solution of~\eqref{equ:interp-system:cr},
  \begin{equation}
    \label{equ:interpolant:cr}
    g = \sum_{i=1}^n \gamma_i\, k(\cdot,s_i)
    + \sum_{i=1}^n \gamma_i^*\, c(\cdot,s_i)
  \end{equation}
  is the unique interpolant of the data $\left( s_1, y_1 \right)$,
  \ldots, $\left( s_n, y_n \right)$ with minimal norm in~$H$.
\end{theorem}

For the usual setting of real or complex RKHSs, strictly positive
definite kernels guarantee that the interpolation
system~\eqref{equ:interp-system} has a solution for any
data~$y_1, \ldots, y_n$. %
This remains true for the system~\eqref{equ:interp-system:cr} in the
case of a complex/real RKHS if the associated real kernel~$\tilde k$
is strictly positive definite on
$\tilde\Sset = \Sset \times \{\re,\im\}$.

%%% FOR A REMARK / SUPPMAT
% Moreover, we have
% \begin{equation} \label{equ:matrix-K}
%   K(x,x') \;=\;
%   \begin{pmatrix}
%     k_{RR}(x,x') & k_{RI}(x,x')\\
%     k_{IR}(x,x') & k_{II}(x,x')
%   \end{pmatrix}.
% \end{equation}

\subsection{Complex/real RKHS with symmetry condition} \label{sec:complex-real-RKHS-sym}

We now characterize, in full generality, the complex/real
RKHSs where a symmetry condition of the form $f^*(s) = f(s^*)$ holds
for all~$f \in H$ and~$s \in \Sset$. %
The following theorem provides a necessary and sufficient condition on~$k$
for such a space to exist and gives the expression of the
corresponding pseudo-kernel. %
The expression appeared previously in \cite[Equations~(48)--(49)]{lataire2016transfer}
for a special type of kernel.

\begin{theorem}\label{thm:hermitian}
  Let $\Sset$ denote a non-empty set, equipped with an involution
  $s \mapsto s^*$ and $k:\Sset \times \Sset \to \Cset$ denote a
  Hermitian positive definite kernel on~$\Sset$. %
  Then the following assertions are equivalent:
  \begin{enumerate}[i) ]
    % 
  \item There exists a complex/real RKHS $H$ on~$\Sset$, with complex
    kernel~$k$, such that
    \begin{equation}
      \label{equ:symmetry}
      \forall f \in H,\; \forall s \in \Sset,\quad f^*(s) = f(s^*).
    \end{equation}
    %
  \item There exists a complex/real RKHS $H$ on~$\Sset$, with complex
    kernel~$k$ and pseudo-kernel $c$ defined by
    \begin{equation}
      \label{equ:pseudo-kern-symm}
      \forall s,s_0 \in \Sset,\quad %
      c(s,s_0) = k(s, s_0^*).
    \end{equation}
    %
  \item $\forall s,s_0 \in \Sset$, $k(s, s_0^*) = k(s_0, s^*)$.
    %
  \end{enumerate}

  \smallbreak
  
  If any (and consequently all) of these assertions holds, then the
  complex/real RKHS~$H$ with complex covariance~$k$ and pseudo
  kernel~\eqref{equ:pseudo-kern-symm} is the unique RKHS on~$\Sset$
  with complex covariance~$k$ such that~\eqref{equ:symmetry} holds. %
  Moreover, denoting by $H_\Cset$ the complex RKHS with kernel~$k$, we
  have $H_\Cset = H \oplus \i H$,
  $H = \left\{ f \in H_\Cset \mid \text{\eqref{equ:symmetry} holds}
  \right\}$ and $\left< f, g \right> = \Re \left< f, g \right>_{H_\Cset}$
  for all $f, g \in H$.
\end{theorem}

It follows from this theorem that $H^2_\mathrm{sym}(\Gamma_{\alpha})$
can be characterized as the complex/real RKHS over~$\Gamma_\alpha$
with complex kernel~\eqref{equ:szego} and pseudo-kernel:
\begin{equation}\label{equ:pseudo-szego} 
  c_\alpha\left( s, s_0 \right) = \frac 1 {2 \pi (2\alpha + s + s_0)},\quad s,\, s_0\in\Gamma_{\alpha}.
\end{equation}

More generally, Theorem~\ref{thm:hermitian} shows that the problem of
minimum-norm interpolation in a complex RKHS, with a symmetry
constraint of the form~\eqref{equ:symmetry}, can be solved by
considering the equivalent problem of minimal-norm interpolation in
the complex/real RKHS with the same complex kernel and the
pseudo-kernel given by~\eqref{equ:pseudo-kern-symm}.
%
In presence of the symmetry condition, even if the complex kernel $k$
is strictly positive definite, $\tilde{k}$ is not and an additional
condition on the data is required to ensure that~\eqref{equ:interp-system}
has a solution.

\begin{theorem}\label{thm:existence-uniqueness-hermit}%
  In the setting of Theorem~\ref{thm:hermitian}, assume that $k$ is
  strictly positive definite, $c$~is given
  by~\eqref{equ:pseudo-kern-symm}, and $s_1, \ldots, s_n \in \Sset$
  are distinct. %
  Then \eqref{equ:interp-system:cr}~has a solution if, and only if,
  $y_j = y_i^*$ for all $i, j$ such that $s_j = s_i^*$. %
  When this holds, there is a unique solution such that
  $\gamma_i = \gamma_j^*$ for all $i, j$ such that~$s_j = s_i^*$.
\end{theorem}


For illustration, we consider the third order rational function
\begin{equation}\label{eq:Frat}
  F_{\mathrm{rat}}(\i\omega)
  = \frac 1 {\i\omega-(-0.1)}
  + \frac {0.5}{\i\omega-(-0.1-0.5\i)}
  + \frac {0.5}{\i\omega-(-0.1+0.5\i)},\;
  \omega\in[0,1],
\end{equation}
which is the Laplace transform of the real-valued
function $t \mapsto e^{-0.1t}\bigl(1+\cos(0.5t)\bigr)$
and thus belongs to
$H^2_\mathrm{sym}(\Gamma_{0.1+\epsilon}) \subset H^2(\Gamma_{0.1+\epsilon})$ for all~$\epsilon > 0$. %
To illustrate the importance of the choice of pseudo-kernel,
we conduct a convergence study in
terms of the root-mean-square error (RMSE) of the approximations,
using equidistant training points
(details on the implementation and selection of
hyper-parameters will be given in the following sections). %
In Figure~\ref{fig:PCov_demo} we demonstrate that choosing a suitable
pseudo-kernel might have a significant impact on the convergence
properties of the (complex/real) RKHS interpolation. %
For the test function~\eqref{eq:Frat}, the pseudo-kernel~\eqref{equ:pseudo-kern-symm}
improves the convergence significantly. %
Note that the test function is a low order rational function which is
here only used to illustrate the impact of the pseudo-kernel. %
Accordingly, rational interpolation techniques as AAA or VF reach
machine accuracy already with $\approx8$ training points and are hence
excluded in the convergence plot for clarity. %
However, it can already be observed that complex/real RKHS
interpolation with the Szegö kernel outperforms the alternative
approach of separate kernel approximations for real and imaginary
part with a Gaussian kernel, as well as polynomial interpolation on
Chebyshev nodes.

% Figure environment removed


\subsection{Relation to Gaussian process interpolation}
\label{sec:relation-GPs}
This section draws connections between minimum norm interpolation in a
RKHS and the posterior mean prediction of a Gaussian process (GP), for
both the complex and complex/real case. GPs are widely used, but to
the authors knowledge this is the first time that the RKHS associated
to any complex GP prediction is characterized. %
Another intention of this section is to make results from the GP
literature available for interpolation with a complex/real RKHS. %
In particular, we are interested in employing statistical methods for
model selection (see, e.g., \cite{petit:2023:parameter} and references
therein)---this will be further developed in
Section~\ref{sec:adap_proc}. %
We consider zero-mean processes in this section, %
for simplicity 
see Remark~\ref{rem:semi-RKHS} below.

\newcommand \kGP {\mathfrak{k}}
\newcommand \cGP {\mathfrak{c}}

Complex GPs are covered for instance in \cite{miller1969complex}. %
A complex GP is a complex process, where the real and imaginary part
considered jointly are a real GP.
% In the terminology of this paper, $\xi$ is a complex GP if
% $\mathcal A \xi = \tilde{\xi}$ is a real GP.
We consider a zero-mean complex-valued random process $\xi$ on
$\Sset$, with covariance function~$\kGP$ and pseudo-covariance
function~$\cGP$:
\begin{align}
  \Esp\left( \xi(s) \xi(s_0)^* \right) & = \kGP(s,s_0), \\
  \Esp \left( \xi(s) \xi(s_0) \right) & = \cGP(s,s_0).
\end{align}

Relying on the mapping~$\Acal$, we can work in a real-valued setting,
i.e., with a real-valued GP $\tilde{\xi}$ indexed
on~$\tilde{\Sset}$. %
In the real-valued case, it is well-known that the conditional mean of
a GP is identical to the minimum-norm interpolant in the RKHS
associated to its covariance function. %
Hence, using~$\Acal$, the conditional mean of a complex GP~$\xi$ is
also identical to a minimum-norm interpolant, but this time in a
complex/real RKHS, the complex kernel~$k$ and pseudo-kernel~$c$ of which are
equal to~$\kGP$ and~$\cGP$ respectively (this follows from
Equations~\eqref{eq:complex-kernel}--\eqref{eq:pseudo-kernel}). %
It is given by Equation~\eqref{equ:interpolant:cr} in general, which
simplifies to
\begin{equation}
  \Esp \left( \xi(s) | y  \right) =  \sum_{i=1}^n \gamma_i k(s,s_i),
  \quad \text{with} \   K_n \gamma  = y,
\end{equation}
if the pseudo-covariance is zero (i.e., in the circular case).

% In \cite{picinbono:1996} it is first shown that the posterior mean
% is widely linear \cite{picinbono1995widely}, which leads to
% \begin{equation}
%    \Esp \left( \xi(s) | y  \right) = (k_{s,n} - c_{s,n} K_n^{-*}C_n^H)P_n^{-*}y + (c_{s,n} - k_{s,n} K_n^{-1}C_n)P_n^{-1}y^*,
%\end{equation}
%where $P_n = K_n^* - C_n^H K_n^{-1}C_n$ and $P_n^{-*}$ denotes the
% complex conjugate of the inverse of $P_n$. The formulas for the
% circular and non-circular case can also be found in
% \cite{boloix2018complex}.\footnote{\color{red} [UR] Link to
% complex/real minimum norm interpolation to be finished}


\begin{remark} \label{rem:real-imaginary-independent-interpolation}
  A common approach to deal with complex data is to
  use GP interpolation for the real and imaginary part separately
  (see, e.g., \cite{Fuhrlander_2020ab}). %
  This corresponds, using notations from
  Proposition~\ref{prop:repr-eval-func}, to $k_{RI} = k_{IR} = 0$, and
  therefore to a complex GP with covariance $k = k_{RR} + k_{II}$ and
  pseudo-covariance $c = k_{RR} - k_{II}$.
\end{remark}

\begin{remark} \label{rem:GP-regression_circular}
  % If $\xi$ is non-circular, i.e., the pseudo-covariance needs to be
  % considered, we can use results from \cite{picinbono:1996} for the
  % posterior mean.
  GP regression with both covariance and pseudo-covariance function
  has also been considered under the name widely linear posterior
  mean. In \cite{picinbono:1996} it is first shown that the posterior
  mean is widely linear \cite{picinbono1995widely}, which leads to
  \begin{equation}
    \Esp \left( \xi(s) | y  \right) = (k_{s,n} - c_{s,n} K_n^{-*}C_n^H)P_n^{-*}y + (c_{s,n} - k_{s,n} K_n^{-1}C_n)P_n^{-1}y^*,
  \end{equation}
  where $P_n = K_n^* - C_n^H K_n^{-1}C_n$ and $P_n^{-*}$ denotes the
  complex conjugate of the inverse of $P_n$. The formulas for the
  circular and non-circular case can also be found in
  \cite{boloix2018complex}.
\end{remark}

\begin{remark} \label{rem:semi-RKHS} %
  In practice, GP models often include a non-zero mean function~$m$,
  usually written as a linear combination
  $m(x) = \sum_{\ell=1}^L \beta_\ell h_\ell(x)$ of known basis functions~$h_\ell$,
  with unknown coefficients~$\beta_\ell$. %
  If the coefficients are estimated by maximum likelihood (as in
  Section~\ref{sec:alg}), the posterior mean of the GP is then equal
  to the interpolant with minimal \emph{semi-norm} in~$G = V + H$,
  where $V = \vect \{ h_1, \ldots h_L \}$ and the semi-norm is
  defined by
  $\left| g \right|_G = \inf_{v \in V} \lVert g - v \rVert_H$.
\end{remark}
