
\section{Data Curation and Pre-Processing}
\label{appsec:dataset_processing}

 To generate an appropriate dataset of reliably labeled bioactivity measurements, we retrieved and reprocessed high-throughput screening data generated by \citet{antonova2018open} as part of a campaign to discover novel chemoprotective antimalarial drug candidates.
 
 The authors established a cell-based phenotypic screening pipeline to identify compounds that inhibit the development of luciferase-expressing liver-stage \textit{Plasmodium falciparum} parasites. After assaying a commercially-available chemical library of \num{538273} of drug-like small molecules in a single-point primary screen, they selected the \num{9963} most promising compounds for a series of confirmatory dose-response screens. Specifically, an 8-point dilution series was used to assess, in duplicate, the potency and efficacy of each compound in the original assay (\textit{Pbluc}). Additionally, the tendency of the assayed compounds to produce false positives and other experimental artifacts was investigated by performing a series of counter-screens that measure hepatic cytotoxicity (\textit{HepG2tox}) and interference with the luciferase-based luminescent readout (\textit{Ffluc}).
 The fact that all bioactivity measurements are (1) generated using biological duplicates and (2) associated with quantitative measures that reflect their likelihood to produce confounding experimental artifacts substantially improves the reliability of the resulting labels.

To facilitate the integration of bioactivity and counter-screen measurements and make the data more amenable to predictive modeling, the $\textsc{IC}_{50}$ values that quantify the concentration at which a molecule produces half of its maximum inhibitory effect were converted to binary labels.
%
Specifically, all compounds with an $\textsc{IC}_{50}\leq\num{1.5} \textrm{\textmu} \textrm{M}$ were denoted as active while all compounds with an $\textsc{IC}_{50}\geq\num{3} \textrm{\textmu} \textrm{M}$ were denoted as inactive, discarding \num{652} compounds with $\num{1.5} \textrm{\textmu} \textrm{M}\leq\textsc{IC}_{50}\leq\num{3} \textrm{\textmu} \textrm{M}$ and assigning qualified $\textsc{IC}_{50}$ values to the appropriate class (see~\Cref{appfig:ic50} for a diagram of the $\textsc{IC}_{50}$ distribution and the applied thresholds). 
% Figure environment removed

In order to integrate information from the \textit{HepG2tox} and \textit{Ffluc} counter-screens and filter out problematic compounds that are likely false positives or risk confounding the evaluation in other ways, the thresholds outlined in \citet{antonova2018open} were applied. In particular, problematic compounds were discarded due to causing hepatotoxicity or assay interference if their respective $\textsc{IC}_{50}$ values met at least one of the criteria outlined in~\Cref{eq:filtering_criteria_a} and~\Cref{eq:filtering_criteria_b}
\begin{align}
    \label{eq:filtering_criteria_a}
    \textit{HepG2tox} \; \textsc{IC}_{50} < 2\cdot \textit{Pbluc}\;\textsc{IC}_{50}&\land\textit{HepG2tox} \; \textsc{IC}_{50} < c_{\text{max}} \\
    \label{eq:filtering_criteria_b}
    \textit{Ffluc} \; \textsc{IC}_{50} < 2\cdot \textit{Pbluc}\;\textsc{IC}_{50}&\land\textit{Ffluc} \; \textsc{IC}_{50} < c_{\text{max}},
\end{align}

where $c_\text{max}$ denotes the maximum concentration a compound was assayed at. These filtering criteria categorized \num{764} compounds as inhibiting hepatocyte viability and \num{446} compounds as interfering with the luminescence readout, including an overlap of \num{49}. Removing these compounds from the dataset results in a total of \num{8150} compounds, of which \num{7301} (90\%) are labeled as inactive and \num{849} (10\%) are labeled as active.
