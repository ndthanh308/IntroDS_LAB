\section{Summary and Conclusions}
\label{sec:discussion}

The objective of early-stage drug discovery is to identify lead compounds that exhibit sufficient evidence of modulating a given disease phenotype---as well as suitable safety profiles---to qualify them for further investigation in in-vivo studies.
Computational techniques that reliably predict the properties of novel molecules in unexplored regions of chemical space have the potential to substantially accelerate this time- and resource-intensive process.
%
Motivated by the practical importance of developing such methods, we derived \textsc{q-savi}, a probabilistic model that allows encoding explicit, problem-informed prior knowledge about the prediction domain into neural network training.

To construct a robust experimental setup and facilitate a practically meaningful evaluation of the proposed method, we carefully pre-processed a high-quality bioactivity dataset and explored different domain-specific statistics to quantify distribution shifts in this setting.
Using these statistics to highlight the limited extent to which commonly used random and scaffold splits are able to induce meaningful covariate and label shifts, we built on two alternative molecular weight- and spectral clustering-based approaches to construct challenging train-test splits.
Leveraging this extrapolative evaluation setup, we demonstrated that using \textsc{q-savi} to provide neural networks with relevant and contextualized information on drug-like chemical space significantly improves both the predictive accuracy and calibration of neural network models, outperforming a range of state-of-the-art self-supervised pre-training, ensembling, and domain adaptation techniques.

The main limitation of the proposed method compared to standard training regimes is its increased computational cost, due to the amortized cost of having to pre-process a suitable context point distribution and the direct cost of having to perform each forward pass over both a mini-batch and a sample of context points.
However, by keeping the size of each context set sample to be roughly comparable to the size of each mini-batch, we found this increase in computational cost to be manageable---especially in comparison to the computational cost of pre-training and fine-tuning related self-supervised methods or deep ensembles.

Promising avenues for future work include an investigation into how using \textsc{q-savi} to specify problem-informed modeling preferences may improve the performance of deep learning algorithms for drug discovery applications that heavily rely on out-of-distribution generalization.
For instance, the approach could be used to construct an acquisition function for an active learning loop to propose structural modifications that optimize the therapeutic properties of an existing lead compound~\citep{nicolaou2007molecular, gomez2018automatic}, as \textsc{q-savi} generates robust predictions and additionally enables researchers to explicitly specify desirable exit vectors.
It may also accelerate the discovery of novel compound classes that exhibit similar pharmacological properties to already explored molecules~\citep{bohm2004scaffold, hu2017recent}, enabling the optimization of certain pharmacokinetic properties or the circumvention of patent restrictions.
More broadly, we hope that this work encourages further research into the utility of probabilistic inference and domain-informed prior distributions over functions for drug discovery and beyond.
%