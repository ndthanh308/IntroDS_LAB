\begin{table}[t]
\small
\centering
\resizebox{\columnwidth}{!}{
\begin{tabular}{lr @{\hspace{1.1\tabcolsep}} rr @{\hspace{1.1\tabcolsep}} r}
\toprule
\multirowcell{2}[-0.5ex][l]{\textbf{BertExt}\\\textbf{length} ($k = \cdot$)} & \multicolumn{2}{c}{\textbf{MIMIC-CXR}} & \multicolumn{2}{c}{\textbf{OpenI}} \\
\cmidrule(lr){2-3}
\cmidrule(lr){4-5}
& R-1 (Prec./Rec.) & $|\hat{\bm{y}}|$ & R-1 (Prec./Rec.) & $|\hat{\bm{y}}|$ \\
\midrule
Fixed ($k = 1$) & 32.7 (\textbf{38.5}/34.2) & 1.0 & \textbf{23.6} (\textbf{24.6}/26.9) & 1.0 \\
\textsc{lr-approx} & 34.5 (35.7/40.0) & 1.4 & 23.5 (23.9/27.2) & 1.1 \\
\textsc{bert-approx} & 35.2 (34.6/42.0) & 1.5 & 23.5 (23.7/27.5) & 1.1 \\
Thresholding & \textbf{36.1} (34.1/\textbf{46.3}) & 1.7 & 23.2 (22.9/\textbf{29.0}) & 1.2 \\
\midrule
$k = |\text{OracleExt}|$ & 36.9 (35.3/44.2) & 1.6 & 24.3 (23.2/29.2) & 1.2 \\
\bottomrule
\end{tabular}
}
\caption{
Comparing strategies for extracting variable-length summaries with BertExt by measuring \textsc{rouge} against the gold summary.
Average summary length $|\hat{\bm{y}}|$ given in sentences.
All methods are tested as guidance signal for GSum in \cref{tab:results}.
}
\label{tab:results-bertex}
\end{table}
