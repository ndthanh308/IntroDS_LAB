\section{Detailed Analysis}
\label{sec:appendix-analysis}
To support replication, this section provides supplementary analysis on the results of the main part.

\subsection{Target Impression Length Distribution and Evaluation by Target Length}
\label{sec:appendix-target-summary-length}
We demonstrated in the main part that variable-length guidance helps to adapt to varying target lengths.
To better interpret this result, we plot the length distribution of target summaries and the ROUGE-1 score by target-length interval in \cref{fig:target-summary-length}.
It can be observed that the length distribution has a long tail with a peak around 4-5 tokens.
Impressions of this length are standard phrasings to indicate that no abnormalities were found (e.g., \emph{``No evidence of acute findings''}).

% Figure environment removed

\subsection{Evaluating GSum in an Oracle Setting}
\label{sec:appendix-gsum-oracle-experiment}
As a supplement to the oracle experiment in \cref{sec:results-gsum-fixed}, we provide all metrics for the three inference settings of GSum in \cref{tab:appendix-gsum-oracle}: (i) automatic fixed-length guidance (i.e., extracted from BertExt with $k=1$), (ii) automatic variable-length guidance but with an oracle length (i.e., BertExt with $k = |\text{OracleExt}(\bm{x}, \bm{y})|$), and (iii) oracle guidance (i.e., $\bm{g} = \text{OracleExt}(\bm{x}, \bm{y})$).

\begin{table}[t]
\small
\centering
\setlength\tabcolsep{2pt}
\begin{tabular}{lrrrrr}
\toprule
\textbf{MIMIC-CXR} & \textbf{R-1} & \textbf{R-2} & \textbf{R-L} & \textbf{BS} & \textbf{Fact.} \\
\midrule
\emph{Guidance signal for GSum}\\
Fixed~\cite{Dou:2021:NAACL} & 46.3 & 32.7 & 44.7 & 57.4 & 46.6 \\
Oracle Length & 51.7 & 36.3 & 49.6 & 61.2 & 52.4 \\
Oracle Length + Content & 58.5 & 42.0 & 56.2 & 66.0 & 60.0 \\
\midrule
% \textbf{OpenI} & \textbf{R-1} & \textbf{R-2} & \textbf{R-L} & \textbf{BS} & \textbf{Fact.} \\
\textbf{OpenI} & & & & & \\
\midrule
\emph{Guidance signal for GSum}\\
Fixed~\cite{Dou:2021:NAACL} & 60.1 & 49.6 & 59.8 & 67.0 & 40.0 \\
Oracle Length & 63.9 & 53.0 & 63.5 & 69.4 & 42.3 \\
Oracle Length + Content & 68.8 & 56.7 & 68.3 & 72.7 & 45.1 \\
\bottomrule
\end{tabular}
\caption{Evaluating GSum in an oracle setting. \emph{Fixed} is reproduced from~\cref{tab:results}.}
\label{tab:appendix-gsum-oracle}
\end{table}


\subsection{BertExt: Evaluating Fixed-length Settings}
\label{sec:appendx-fixed-k-testing}
To evaluate if larger values for $k$ in the fixed-summary length setting would improve the effectiveness of BertExt, we generate summaries for all settings of $k = \{1,...,5\}$.
Analogously, we provide these summaries as guidance signal to GSum.
\cref{tab:appendix-fixed-k-testing} reports the results of this experiment.
While we find that larger settings of $k$ lead to an increase in recall, we see an equally strong drop in precision, both on BertExt and GSum which demonstrates the necessity of variable-length extractive guidance.

\def\SecBertExtStatic{\multicolumn{11}{l}{BertExt with fixed-length summaries} \\\midrule}
\def\SecGsumStatic{\midrule \multicolumn{11}{l}{GSum with fixed-length guidance extracted from BertExt} \\\midrule}
\begin{table}[t]
\small
\centering
\setlength\tabcolsep{3pt}
\resizebox{\linewidth}{!}{
\begin{tabular}{lrrrrrrrrrr}
\toprule
{} & \multicolumn{5}{c}{\textbf{MIMIC-CXR}} & \multicolumn{5}{c}{\textbf{OpenI}} \\
\cmidrule(lr){2-6}
\cmidrule(lr){7-11}
{} & R-1 & R-2 & R-L & B\textsubscript{P} & B\textsubscript{R} & R-1 & R-2 & R-L & B\textsubscript{P} & B\textsubscript{R} \\
\midrule
\SecBertExtStatic
$k=1$ & 32.7 & 18.1 & 30.0 & \textbf{45.2} & 40.1 & \textbf{23.6} & \textbf{7.4} & \textbf{22.6} & \textbf{33.6} & 32.3 \\
$k=2$ & \textbf{34.1} & \textbf{18.6} & \textbf{31.3} & 40.9 & 50.1 & 19.7 & 6.7 & 18.9 & 28.3 & 39.9 \\
$k=3$ & 31.7 & 17.0 & 29.2 & 37.0 & 53.5 & 17.4 & 6.1 & 16.6 & 25.9 & 42.8 \\
$k=4$ & 29.1 & 15.4 & 26.8 & 34.0 & 54.6 & 15.8 & 5.5 & 15.1 & 24.0 & 43.7 \\
$k=5$ & 27.2 & 14.3 & 25.2 & 32.2 & \textbf{54.9} & 15.1 & 5.2 & 14.4 & 23.3 & \textbf{44.1} \\
\SecGsumStatic
$k=1$ & \textbf{46.3} & \textbf{32.7} & \textbf{44.7} & \textbf{64.6} & 52.8 & \textbf{60.1} & \textbf{49.6} & \textbf{59.8} & \textbf{67.0} & \textbf{68.5} \\
$k=2$ & \textbf{46.3} & 30.3 & 44.2 & 58.1 & 58.5 & 54.3 & 43.2 & 53.9 & 61.2 & 66.2 \\
$k=3$ & 44.1 & 27.7 & 41.9 & 53.6 & 59.9 & 54.6 & 43.2 & 54.1 & 61.6 & 67.3 \\
$k=4$ & 42.2 & 26.0 & 40.2 & 50.4 & \textbf{60.2} & 53.5 & 42.1 & 53.1 & 60.1 & 67.5 \\
$k=5$ & 40.8 & 24.6 & 38.8 & 48.3 & 60.1 & 52.7 & 41.3 & 52.2 & 59.5 & 67.5 \\
\bottomrule
\end{tabular}}
\caption{Testing fixed-length summaries ($k \in [1,5]$) for BertExt (first block) and as GSum guidance (second block). Metrics are ROUGE-1/2/L and BERTScore precision~(B\textsubscript{P}) and recall (B\textsubscript{R})}
\label{tab:appendix-fixed-k-testing}
\end{table}


\subsection{Evaluating Guidance Length Prediction}
\label{sec:appendix-oracle-approx}
To predict the length of OracleExt in the variable-length guidance setting, we employ a logistic regression classifier and a BERT-based classifier (cf. \cref{sec:experimental-setup}).
Detailed evaluation results for both classification models are given in \cref{tab:appendix-oracle-approx}.

\begin{table}[t]
\small
\centering
\subfloat[Dataset: MIMIC-CXR]{
\resizebox{\columnwidth}{!}{
\begin{tabular}{lllr}
\toprule
{} & \textbf{LR-Approx} & \textbf{BERT-Approx} & \\
\cmidrule(lr){2-2}
\cmidrule(lr){3-3}
\textbf{Target} & F-1 (Prec./Rec.) & F-1 (Prec./Rec.) & Support \\
\midrule
$k = 0$ & 46.2 (80.0/32.4) & 53.7 (60.0/48.6) & 37 \\
$k = 1$ & 71.1 (63.4/80.9) & 71.7 (68.9/74.9) & 824 \\
$k = 2$ & 39.7 (43.1/36.9) & 46.7 (45.1/48.4) & 512 \\
$k = 3$ & 30.9 (53.3/21.8) & 43.2 (61.5/33.3) & 225 \\
Macro Avg. & 47.0 (59.9/43.0) & 53.9 (58.9/51.3) & 1,598 \\
\midrule
On training set & 52.3 (64.1/47.6) & 62.5 (69.7/58.5) & 122,500 \\
\bottomrule
\end{tabular}}}
\quad
\subfloat[Dataset: OpenI]{
\resizebox{\columnwidth}{!}{
\begin{tabular}{lllr}
\toprule
{} & \textbf{LR-Approx} & \textbf{BERT-Approx} & \\
\cmidrule(lr){2-2}
\cmidrule(lr){3-3}
\textbf{Target} & F-1 (Prec./Rec.) & F-1 (Prec./Rec.) & Support \\
\midrule
$k = 0$ & 77.7 (85.9/70.9) & 84.0 (86.6/81.6) & 103 \\
$k = 1$ & 84.6 (77.2/93.6) & 85.4 (79.8/92.0) & 450 \\
$k = 2$ & 19.8 (36.1/13.7) & 28.4 (39.6/22.1) & 95 \\
$k = 3$ & 15.4 (50.0/9.1) & 8.7 (100.0/4.5) & 22 \\
Macro Avg. & 49.4 (62.3/46.8) & 51.6 (76.5/50.1) & 670 \\
\midrule
On training set & 58.5 (83.3/54.3) & 51.1 (53.0/51.0) & 2,342 \\
\bottomrule
\end{tabular}}}
\caption{
Precision, recall and F1 for length prediction of OracleExt. Scores are provided per class on the test set, and as macro-average for both the training and test set.
Support indicates the number of samples in each class.
}
\label{tab:appendix-oracle-approx}
\end{table}


\subsection{Including the Background Section}
\label{sec:appendix-background-experiment}
To understand to what extent the background section carries important information for summarizing findings to impression, we prepend it to the findings section and retrain all models.
It can be observed that this change improves most abstractive methods on both datasets (\cref{fig:background-experiment}).
For extractive methods results stay largely on par or get worse, indicating that these models do not effectively integrate the background information.

% Figure environment removed

\subsection{Examples of Duplicated Findings and Impressions}
\label{sec:appendix-duplication-examples}
We anecdotally observed a large degree of duplication within MIMIC-CXR which may cause corpus-level inconsistencies (see discussion in~\cref{sec:error-analysis-discussion}).
This section further quantifies the degree of duplication and provides several examples.
Throughout, we only consider instances of \emph{exact} duplication.
Of the 122,500 training reports in MIMIC-CXR, we find that 11.9\% have a findings section occurring more than once.
We present examples of duplicate findings with \emph{different} impressions in~\cref{tab:duplicates}.
In addition, we calculate a \emph{label entropy} over the probabilities that each impression occurs for a given finding.
We posit that duplicate finding-impression pairs may negatively impact model training in two ways.
First, for findings with a high label entropy, the training loss cannot not stabilize (i.e., it is not clear which impression the model should favor).
Second, for findings with a low label entropy, the model may learn a kind of ``majority vote,'' which in turn may render models not sensitive enough to generate useful summaries for slightly different findings.
We leave further investigation of report duplication to future work.

\subsection{Factuality of Additions}
\label{sec:appendix-radnli}
As discussed in~\cref{sec:error-analysis-discussion}, we use RadNLI~\cite{Miura:2021:NAACL} to get a first estimate for the factuality of additions marked by annotators in the error analysis.
RadNLI obtained an accuracy of 77.8\% on a test set of 480 manually labeled sentence pairs in MIMIC-CXR~\cite{Miura:2021:NAACL}, which we consider sufficient for an initial exploration of the factuality of additions.
\cref{tab:radnli-results} presents a breakdown of the RadNLI predictions for all addition spans and models.
It can be seen that the majority of additions is either neutral to the findings section, or entailed by it.
Yet, between 23.4\% and 29.3\% of additions contradict at least one findings sentence, indicating that factuality of radiology report summarization methods can also further be improved.

\subsection{Error Analysis: Responses to \emph{Other} Category}
We analyze the annotators' comments from the \emph{other} error category, and categorize these errors into two-level hierarchy using a bottom-up approach. Our categorization alongside definitions, examples and counts is shown in \cref{tab:app:other}.

\section{Replication Details for Modeling}
\label{sec:appendix-replication-models}
We report hyperparameters of the summarization models in \cref{tab:hyperparameters-summarization}, and for models that predict the length of OracleExt (\textsc{lr-approx}/\textsc{bert-approx}) in \cref{tab:hyperparameters-oracle-approx}.
All models were trained on NVIDIA RTX A6000 GPUs with 48GB of memory.

\begin{table}[t]
\small
\centering
\resizebox{\columnwidth}{!}{
\begin{tabular}{lrrr}
\toprule
\textbf{Model} & \textbf{Entail} & \textbf{Neutral} & \textbf{Contradict}\\
\midrule
BertAbs & 31.9\% & 44.7\% & 23.4\% \\
GSum w/ Thresholding & 34.5\% & 36.2\% & 29.3\% \\
WGSum & 32.0\% & 44.0\% & 24.0\% \\
WGSum+CL & 33.3\% & 41.2\% & 25.5\% \\
\bottomrule
\end{tabular}}
\caption{Factuality of additions in candidates (i.e., spans categorized as ``2a Finding/interpretation''), as per RadNLI~\cite{Miura:2021:NAACL}.}
\label{tab:radnli-results}
\end{table}


\section{Replication Details for Error Analysis}
\label{sec:appendix-replication-error-analysis}

\paragraph{Sample statistics.}
For inclusion in the error analysis, samples were drawn uniformly at random from the official test set of MIMIC-CXR. We compare statistics of the sample with those of the full test set in~\cref{tab:error-analysis-sample-statistics}.

\begin{table}[t]
\small
\centering
\begin{tabular}{lccc}
\toprule
\textbf{Aspect} & \textbf{Full Test Set} & \textbf{Sample} \\
\midrule
Reports & 1,598 & 100 \\
Avg. $|\bm{x}|_{t}$ & 70 {\color{gray} $\pm$ 27.4} & 63 {\color{gray} $\pm$ 20.4} \\
Avg. $|\bm{x}|_{s}$ & 6.2 {\color{gray} $\pm$ 1.9} & 5.7 {\color{gray} $\pm$ 1.6} \\
Avg. $|\bm{y}|_{t}$ & 19 {\color{gray} $\pm$ 15.2} & 18 {\color{gray} $\pm$ 12.4} \\
Avg. $|\bm{y}|_{s}$ & 1.8 {\color{gray} $\pm$ 1.0} & 1.7 {\color{gray} $\pm$ 0.9} \\
Novelty & 69.8\% & 69.7\% \\
CMP & 71.9\% & 70.3\% \\
\bottomrule
\end{tabular}
\caption{Statistics of the MIMIC-CXR test set and the sample used in the error analysis.}
\label{tab:error-analysis-sample-statistics}
\end{table}


\paragraph{Aggregating span-based annotations.}
\label{sec:appendix-span-based-aggregation}
From the three annotations we form a ``gold standard'' as follows: for binary questions we take a majority vote.
For span-based annotations, we first group (partially) overlapping spans, and then take a majority vote within each group.
We provide an example for the majority voting of span-based annotations below. A1, A2, A3, denote annotators, and \texttt{[--eX--]} denotes an error of category X.

\begin{small}
\begin{verbatim}
Tokens:  a   b  c   d   e   f   g   h
A1    : [-e1-]  [-----e2----]
A2    : [-e1-]  [-e1-] [-e2-]
A3    : [-e1-]                 [--e1--]
---------------------------------------
Group :    1          2           3
---------------------------------------
Vote  : [-e1-]         [-e2-]
\end{verbatim}
\end{small}

\paragraph{Inter-annotator agreement (IAA).}
\label{sec:appendix-iaa}
We calculate $F_1$ for span-annotations (\citet{Deleger:2012:AMIA}, categories 1 and 2), and Krippendorffs' Alpha~\cite{Krippendorff:1970:ALPHA} for binary judgments (categories 3 and 4) and report the IAA by category in~\cref{tab:iaa}.
\begin{table*}[t]
\centering
\small
\resizebox{\textwidth}{!}{
\begin{tabular}{llllll}
\toprule
\textbf{Parameter} & \textbf{BertExt} & \textbf{BertAbs} & \textbf{GSum} & \textbf{WGSum} & \textbf{WGSum+CL} \\
\midrule
Training Steps (MIMIC) & 20,000 & 20,000 & 20,000 & 50,000 & 100,000 \\
Training Steps (OpenI) & 20,000 & 20,000 & 20,000 & 20,000 & 20,000 \\
LR (Encoder) & $2\mathrm{e}{-3}$ & $2\mathrm{e}{-4}$ & $2\mathrm{e}{-4}$ & $5\mathrm{e}{-2}$ & $2\mathrm{e}{-4}$ \\
LR (Decoder) & n/a & $2\mathrm{e}{-2}$ & $2\mathrm{e}{-2}$ & $5\mathrm{e}{-2}$ & $5\mathrm{e}{-2}$ \\
Warmup (Encoder) & 10,000 & 20,000 & 20,000 & 8000 & 10,000 \\
Warmup (Decoder) & n/a & 10,000 & 10,000 & 8000 & 7000 \\
Dropout & 0.1 & 0.2 & 0.2 & 0.1 & 0.2 \\
Checkpoint freq. (MIMIC) & 1000 & 2000 & 2000 & 2000 & 2000 \\
Checkpoint freq. (OpenI) & 1000 & 2000 & 2000 & 200 & 200 \\
Decoding & n/a & Beam search & Beam search & Beam search & Beam search \\
Prediction length & n/a & $\ge$ 5 tokens & $\ge$ 5 tokens & $\ge$ 5 tokens & $\ge$ 5 tokens \\
Training GPUs & 3 & 5 & 5 & 4 & 3 \\
Inference GPUs & 1 & 1 & 1 & 1 & 1 \\
Base model & bert-base-uncased & bert-base-uncased & bert-base-uncased & None & \begin{tabular}[x]{@{}l@{}}dmis-lab/biobert-\\base-cased-v1.1\end{tabular} \\
Parameters & 120,512,513 & 180,222,522 & 205,433,914 & 82,260,794 & 221,600,069 \\
\bottomrule
\end{tabular}
}
\caption{Hyperparameters of BertExt/BertAbs~\cite{Liu:2019:EMNLP}, GSum~\cite{Dou:2021:NAACL}, WGSum~\cite{Hu:2021:ACL} and WGSum+CL~\cite{Hu:2022:ACL}. Training steps, warmup and learning rates were adapted as described in \cref{sec:experimental-setup}. Remaining parameters kept as in the original publications.}
\label{tab:hyperparameters-summarization}
\end{table*}

\begin{table}[t]
\centering
\small
\begin{tabular}{lp{0.62\columnwidth}}
\toprule
\textbf{Parameter} & \textbf{Setting} \\\midrule
\emph{LR-Approx} & \\ \midrule
Features & Bag-of-words, unigrams with minimum document-frequency of 5, tf-idf \\
Parameters & 3718 (MIMIC-CXR), 592 (OpenI) \\
Regularization & L2 regularization with strength $C=1$ \\
Solver & SAGA \\
Max. Iterations & 1000 \\\midrule
\emph{BERT-Approx} & \\ \midrule
Checkpoint & \texttt{distilbert-base-cased} \\
Parameters & 65,784,580 \\
Optimizer & Adam \\
Learning rate & $2\mathrm{e}{-5}$ \\
Epochs & 3 \\
Dropout & 0.2 \\
Batch size & 16 \\
Checkpoint freq. & 250 \\
Hardware & 6 GPUs \\ \bottomrule
\end{tabular}
\caption{Hyperparameters for guidance length prediction models.}
\label{tab:hyperparameters-oracle-approx}
\end{table}

\begin{table}[t]
\small\centering
\begin{tabular}{c@{\hspace{1\tabcolsep}}lrr}
\toprule
\textbf{\#} & \textbf{Category} & \textbf{IAA} & \textbf{Count} \\
\midrule
\multicolumn{4}{l}{\emph{Omissions from reference}} \\
1a & Finding/interpretation & 0.64 & 774 \\
1b & Comparison & 0.34 & 236 \\
1c & Ref. to prior report & 0.23 & 43 \\
1d & Communication/followup & 0.83 & 216 \\
\midrule
\multicolumn{2}{l}{Total} & 0.61 & 1269 \\
\addlinespace\multicolumn{4}{l}{\emph{Additions to candidate}} \\
2a & Finding/interpretation  & 0.66 & 718 \\
2b & Comparison  & 0.44 & 155 \\
2c & Ref. to prior report  & 0.08 & 17 \\
2d & Communication/followup  & 0.65 & 72 \\
2e & Contradicting finding & 0.26 & 34 \\
\midrule
\multicolumn{2}{l}{Total} & 0.60 & 996 \\
\addlinespace
3 & Incorrect location & 0.26 & 111 \\
4 & Incorrect severity & 0.41 & 121 \\
\bottomrule
\end{tabular}
\caption{Inter-annotator agreement (IAA) by category and total number of annotations before majority voting.}
\label{tab:iaa}
\end{table}

\begin{table*}[t]
\small\centering
\setlength\tabcolsep{3pt}
\resizebox{\textwidth}{!}{
\begin{tabular}{lp{0.35\textwidth}rcccrp{0.35\textwidth}}
\toprule
\textbf{\#} & \textbf{Finding} & \textbf{Dups.} & \textbf{\%} & $|\bm{y}^*|$ & \textbf{H} & \textbf{Count} & \textbf{Top-5 Impressions} \\
\midrule
\multicolumn{8}{l}{\emph{Most frequent duplicates}} \\\midrule
1 & \multirow[t]{5}{=}{PA and lateral views of the chest provided. There is no focal consolidation, effusion, or pneumothorax. The cardiomediastinal silhouette is normal. Imaged osseous structures are intact. No free air below the right hemidiaphragm is seen.} & 1141 & 0.93 & 26 & 0.12 & 1061 & No acute intrathoracic process.\\
& & & & & & 45 & No acute intrathoracic process\\
& & & & & & 3 & No acute intrathoracic process. \_, MD\\
& & & & & & 3 & No acute intrathoracic process. Specifically, no pneumothorax.\\
& & & & & & 3 & No evidence of pneumonia.\\
\addlinespace
2 & \multirow[t]{5}{=}{Heart size is normal. The mediastinal and hilar contours are normal. The pulmonary vasculature is normal. Lungs are clear. No pleural effusion or pneumothorax is seen. There are no acute osseous abnormalities.} & 1033 & 0.84 & 34 & 0.11 & 974 & No acute cardiopulmonary abnormality.\\
& & & & & & 24 & No evidence of pneumonia.\\
& & & & & & 3 & No radiographic evidence of pneumonia.\\
& & & & & & 2 & No acute cardiopulmonary abnormality. No displaced fracture identified. If there is continued concern for a rib fracture, consider a dedicated rib series.\\
& & & & & & 1 & Improving bibasilar atelectasis and decreasing bilateral effusions.\\
\addlinespace
3 & \multirow[t]{5}{=}{The lungs are clear without focal consolidation. No pleural effusion or pneumothorax is seen. The cardiac and mediastinal silhouettes are unremarkable.} & 753 & 0.61 & 47 & 0.20 & 665 & No acute cardiopulmonary process.\\
& & & & & & 15 & No acute cardiopulmonary process. No focal consolidation to suggest pneumonia.\\
& & & & & & 8 & No pneumonia.\\
& & & & & & 7 & No evidence of pneumonia. No acute cardiopulmonary process.\\
& & & & & & 4 & No acute cardiopulmonary process. No significant interval change.\\
\midrule\multicolumn{8}{l}{\emph{Duplicates with highest impression entropy}} \\\midrule
4 & \multirow[t]{5}{=}{The heart is normal in size. The mediastinal and hilar contours appear within normal limits. There is no pleural effusion or pneumothorax. The lungs appear clear. Bony structures appear within normal limits.} & 25 & 0.02 & 2 & 0.99 & 14 & No evidence of acute cardiopulmonary disease.\\
& & & & & & 11 & No evidence of acute disease.\\
& & & & & & & \\
& & & & & & & \\
& & & & & & & \\
\addlinespace
5 & \multirow[t]{5}{=}{The lungs are clear. There is no pneumothorax. The heart and mediastinum are within normal limits. Regional bones and soft tissues are unremarkable.} & 25 & 0.02 & 2 & 0.94 & 16 & Clear lungs with no evidence of pneumonia.\\
& & & & & & 9 & Clear lungs.\\
& & & & & & & \\
& & & & & & & \\
& & & & & & & \\
\addlinespace
6 & \multirow[t]{5}{=}{The lungs are well expanded and clear. Hila and cardiomediastinal contours and pleural surfaces are normal.} & 23 & 0.02 & 15 & 0.92 & 6 & Normal. No evidence of pneumonia.\\
& & & & & & 2 & No evidence of pneumonia.\\
& & & & & & 2 & Normal chest radiograph.\\
& & & & & & 2 & No pneumonia.\\
& & & & & & 1 & Normal. No evidence of mass.\\
\bottomrule
\end{tabular}}
\caption{Examples of exact duplicates in the training set of MIMIC-CXR. In total, there are 14,596 reports with duplicated findings (11.9\% of the training data). The table shows the number of reports with a given finding (\textbf{Dups.}), the relative frequency in the training set (\textbf{\%}), the number of distinct impressions with this finding ($|\bm{y}^*|$), the entropy over the impression frequencies (\textbf{H}), and the top-5 impressions with their respective \textbf{Count}.}
\label{tab:duplicates}
\end{table*}

\begin{table*}[t]
\small
\centering
\resizebox{\linewidth}{!}{
\begin{tabular}{l p{12em} p{12em} p{12em}  r}
\toprule
\textbf{(Sub-)Category} & \textbf{Description} & \textbf{Example} & \textbf{Explanation} & \textbf{Count} \\
\midrule

\multicolumn{4}{l}{\emph{1. Incorrect findings: the finding in the reference is replaced with a different and incorrect finding.}} & 29 \\ \midrule

Finding &
incorrectness affects the main finding. &
no acute \textcolor{Orange}{intrathoracic} process. &
The reference uses ``\textcolor{Orange}{cardiopulmonary} process'' instead of ``\textcolor{Orange}{intrathoracic} process''. &
21 \\

Past state &
incorrectness affects a past state of the patient. &
\textcolor{Orange}{increased} opacity in the right lung.. &
The reference mentions that the opacity is \textcolor{Orange}{new} and did not exist before.  &
7 \\

Other &
incorrectness affects other aspects.  &
bilateral pleural \textcolor{Orange}{effusions},..., slightly improved... &
The \textcolor{Orange}{improvement} is used to describe a second finding in the reference.   &
1 \\\midrule

\multicolumn{4}{l}{\emph{2. Imprecise findings: the description of the finding or some of its aspects is imprecise compared to the reference.}} & 73 \\ \midrule

Finding &
the description of the finding itself is imprecise compared to the reference.  &
...no \textcolor{Orange}{displaced} fractures are seen. &
The reference uses ``\textcolor{Orange}{acute} fractures'' instead of ``\textcolor{Orange}{displaced} fractures'' (the reference is more general). &
21 \\

Location &
the location of the finding is imprecise.  &
retrocardiac opacity compatible with pneumonia... &
The references specifies the exact location: ``\textcolor{Orange}{Left lower lobe pneumonia}''. &
21 \\

Certainty &
the summary is presented with a different degree of certainty.  &
bilateral middle lobe opacities \textcolor{Orange}{could represent} atelectasis or pneumonia. &
The reference is certain about the finding.  &
9 \\

Repetition &
some findings are repeated. &
unchanged \textcolor{Orange}{bibasilar bronchiectasis} and \textcolor{Orange}{bibasilar bronchiectasis}. &
\textcolor{Orange}{bibasilar bronchiectasis} is mentioned twice. &
6 \\

Count &
the count in the finding is imprise. &
right pleural \textcolor{Orange}{effusion}. &
The reference adds ``Multiloculated'', i.e.,  ``\textcolor{Orange}{Multiloculated} right pleural effusion'' &
2 \\

Size &
the size of the finding is added/omitted/different. &
multiple bilateral pulmonary nodules \textcolor{Orange}{measuring up to 2. 5 cm}. &
The reference omits the size. &
1 \\

Other &
other aspects about the finding are imprecise. &
interval resolution of large right pleural effusion... &
The reference includes other clinical information. &
13 \\\midrule

\multicolumn{4}{l}{\emph{3. Minor/secondary: errors that do not affect the finding.}} & 21 \\\midrule

Limitation &
some limitations of the examination are (not) mentioned.  &
no definite acute cardiopulmonary process. &
The reference adds \textcolor{Orange}{``based on this limited, portable examination''}. &
15 \\

Phone calls &
The time of a telephone call is different.   &
...these findings were discussed with dr. \_ by \_ via \textcolor{Orange}{telephone on \_ at 4 : 45 pm}. &
The reference mentions a different time for the phone call. &
4 \\

Recommendation &
errors related to recommendations.  &
short radiographic follow up is recommended \textcolor{Orange}{within \_ weeks} to document resolution. &
The reference omits \textcolor{Orange}{``within \_ weeks''}.  &
2 \\
\bottomrule
\end{tabular}}
\caption{Bottom-up categorization of errors from the \textit{Other} category with descriptions, examples and counts.}
\label{tab:app:other}
\end{table*}

\clearpage
\clearpage
\section{Annotation Guidelines}
\label{sec:appendix-annotation-guidelines}

\paragraph{Introduction.}
We consider automatic impression generation for English radiology reports of chest imaging examinations. These reports conventionally have three sections (example in \cref{fig:app:report}).

\begin{enumerate}[noitemsep,parsep=0pt,partopsep=0pt]
    \item \textbf{Background.} A description of the exam, patient information, and relevant prior exams.
    \item \textbf{Findings.} A description or itemization of the radiologists' observations based on the radiographs.
    \item \textbf{Impression.} A concise summary of the most important findings, including inferences and any recommendations.
\end{enumerate}

% Figure environment removed

\paragraph{Study setup.}
We are looking to determine typical failures that automatic impression generation systems make. You will be shown a reference impression and four candidate impressions. The reference was written by a radiologist, and the candidates were generated by automatic systems. Your goal is to judge the accuracy of candidate impressions \emph{based on a reference impression}. For each candidate, you will be asked to identify any errors that it may have.

\paragraph{Definition of ``error.''} We define an error as a difference between the candidate and reference. An error can be one of the following:

\begin{enumerate}[noitemsep,parsep=0pt,partopsep=0pt]
\item Omissions
    \begin{enumerate}[(a),noitemsep,topsep=0pt,parsep=0pt,partopsep=0pt]
        \item Omission of finding/interpretation
        \item Omission of comparison describing a change from a previous examination
        \item Omission of reference to prior report while making a comparison
        \item Omission of next steps, recommendation, communications
    \end{enumerate}
\item Additions
    \begin{enumerate}[(a),noitemsep,topsep=0pt,parsep=0pt,partopsep=0pt]
        \item Additional finding/interpretation
        \item Mention a comparison that is not present in reference
        \item Additional reference to prior report while making a comparison
        \item Additional mention of next steps, recommendation, communications
        \item Additional finding/interpretation which contradicts reference
    \end{enumerate}
\item Incorrect location/position of finding
\item Incorrect severity of finding
\item Other difference between candidate and reference (please describe...)
\end{enumerate}
%
\cref{tab:app:guidelines:examples} shows an example for each error category.

\paragraph{Annotating errors as inline annotations.}
You are asked to annotate errors as \emph{inline annotations}. For each error that you identify, first select the error category and then highlight the relevant text snippet with your mouse. This applies the category. If you have to remove an annotation, press on the highlight and use your backspace/delete key (\keystroke{Entf} or \keystroke{\textleftarrow Backspace}). If one of the above categories occurs multiple times, please annotate all of them \emph{separately} (see \cref{fig:app:guidelines:ex1}). Some general guidelines:

\begin{itemize}[noitemsep]
\item A candidate may have multiple errors, so please add all that apply.
\item Some candidates will be the same, so please assign the same errors to all candidates.
\item For additional findings that are plausible, pick \textcolor{purple}{2a. Additional finding/interpretation}. In the context of the full report, these additions may be correct. What this category aims to capture is that the system included information which the radiologist chose not to include. If a finding contradicts the reference, select \textcolor{purple}{2e. Additional finding/interpretation which contradicts reference}.
\item Use \textcolor{purple}{5. Other} freely, especially if you find it difficult to assign any of the above categories. These remarks help us to better understand and characterize potential errors.
\item You can ignore differences in word choice if they are synonymous. Example: \emph{``may reflect developing consolidation''} is equal to \emph{``could represent early consolidation.''}
\end{itemize}
%
Finally, always use your best judgment when assessing the reports. If you are in doubt, you can add any questions/comments about the report or the error categories in the given box.

% Figure environment removed

\subsection*{Corner Cases}
\paragraph{How to annotate ``3. Incorrect location/position of finding'' and ``4. Incorrect severity of finding''?}
Only apply if both reference and candidate mention a finding, \emph{and} when there is a mismatch in severity/location. In the example below, both mention effusion, but the reference does not specify the size of effusion, whereas the candidate states that there are ``small'' effusions. Therefore, apply \textcolor{purple}{4. Incorrect severity of finding}.

\noindent\fbox{%
\parbox{0.98\linewidth}{%
\begin{sanseriffont}
\textbf{Reference:}
interval worsening of now moderate interstitial pulmonary edema. bilateral pleural effusions.\\
\textbf{Candidate:}
interval worsening of now moderate interstitial pulmonary edema. small bilateral pleural effusions.
\end{sanseriffont}
}}

\paragraph{Opacities vs. consolidation.}
Often, opacities are used in place of consolidation and vice versa. In those cases, apply \textcolor{purple}{5. Other} with a comment similar to \textcolor{purple}{``opacities not equal consolidation, but otherwise correct''}.

\noindent\fbox{%
\parbox{0.98\linewidth}{%
\begin{sanseriffont}
\textbf{Reference:}
Improved right lower medial lung peribronchial consolidation. \\
\textbf{Candidate:}
right lower medial lung peribronchial opacities have improved.
\end{sanseriffont}
}}

\paragraph{No acute abnormality vs. COPD.}
Does ``no acute abnormality'' contradict ``COPD''? No, for the purposes of our evaluation, COPD is not an \emph{acute} disease, so this is not contradicting. In the example below, following categories apply: (1) ``COPD'' is missing $\rightarrow$ \textcolor{purple}{1a. Omission of finding/interpretation}, (2) ``opacity is resolved'' $\rightarrow$ \textcolor{purple}{1b. Omission of comparison describing a change from a previous examination}, (3) ``no acute cardiopulmonary abnormality'' $\rightarrow$  \textcolor{purple}{2a. Additional finding/interpretation}.

\noindent\fbox{%
\parbox{0.98\linewidth}{%
\begin{sanseriffont}
\textbf{Reference:}
Left basilar opacity is resolved. COPD. \\
\textbf{Candidate:}
no acute cardiopulmonary abnormality.
\end{sanseriffont}
}}

\paragraph{Misleading grammar or sentence structure.}
In general, disregard grammatical errors. However, please pay attention to any \emph{logical flaws} that arise because of grammar errors or a misleading sentence structure. In the example below, the \emph{``and''} in the candidate implies that both ``bronchiectasis'' and ``peribronchial consolidation'' have improved, whereas the reference only states that the consolidation has improved. In those cases, apply \textcolor{purple}{5. Other} and add a comment similar to \textcolor{purple}{``logical error because of grammar.''}

\noindent\fbox{%
\parbox{0.95\linewidth}{%
\begin{sanseriffont}
\textbf{Reference:}
Bilateral lower lung bronchiectasis with improved peribronchial consolidation\\
\textbf{Candidate:}
bilateral lower lung bronchiectasis and peribronchial consolidation have improved since \_.
\end{sanseriffont}
}}

\begin{table*}[t]
\small
\centering
\resizebox{\textwidth}{!}{
\begin{tabular}{p{10em} p{13em} p{13em} p{12em} }
\toprule
\textbf{Error} & \textbf{Reference} & \textbf{Candidate} & \textbf{Explanation} \\
\midrule

\multicolumn{4}{l}{\emph{Omissions (apply to reference)}}\\\midrule
1a. Omission of finding/interpretation &
New left lower lobe infiltrate \hlomma{and effusion}. &
New left lower lobe infiltrate.	&
Effusion is missing. \\\addlinespace

1b. Omission of comparison describing a change from a previous examination &
\hlommb{In comparison to \_ exam, there is interval near-complete resolution of bilateral pleural effusion.}	&
No evidence of acute cardiopulmonary process. &
Resolution of effusion is not described, therefore the comparison is missing. \\\addlinespace

1c. Omission of reference to prior report while making a comparison &
Increased pulmonary edema \hlommc{compared to \_.}	&
increased pulmonary edema.	&
While the candidate correctly states that the edema has increased, it lacks the reference to the prior report (or the date of it). \\\addlinespace

1d. Omission of next steps / recommendation / communications &
No pneumothorax or pneumomediastinum. \hlommd{Recommend repeat PA and lateral imaging later today to verify these findings.} Otherwise unremarkable chest radiograph. \hlommd{These findings were communicated to Dr. \_ at 11:55 a.m. by telephone by Dr. \_.}	&
No pneumothorax or pneumomediastinum. &
The candidate does not include the followup \emph{(recommend repeat PA)} and the remark about a communication with another doctor \emph{(These findings were communicated [...])}. \\\addlinespace

\midrule
\multicolumn{4}{l}{\emph{Additions (apply to candidate)}}\\
\midrule
2a. Additional finding / interpretation	&
Slight increased hazy opacities at the right lung base which may reflect developing consolidation.	&
slightly increased hazy opacities at the right lung base which may represent \hlomma{atelectasis or} developing consolidation. &
Atelectasis is not mentioned in the reference. This finding is not contradicting the reference. It may be correct in the context of the full report. \emph{Same as 1a, but in the other direction.} \\\addlinespace

2b. Mention a comparison that is not present in reference	&
Mild to moderate pulmonary edema, increased from \_.	&
Mild to moderate pulmonary edema, increased from \_. \hlommb{Stable cardiomegaly.} &
``Stable'' suggests that the state of a finding was compared to a previous examination. This comparison is not made in the reference. \emph{Same as 1b, but in the other direction.} \\\addlinespace

2c. Additional reference to prior report while making a comparison	&
&
&
\emph{Same as 1c, but in the other direction.}\\\addlinespace

2d. Additional mention of next steps / recommendation / communications	&
&
&
\emph{Same as 1d, but in the other direction.}\\\addlinespace

2e. Additional finding / interpretation which contradicts reference &
Unchanged size and position of right-sided hydropneumothorax. &
\hladde{Development of new} right-sided hydropneumothorax &
Unchanged vs. development of new \\\addlinespace

\midrule
\multicolumn{4}{l}{\emph{Incorrect location, Incorrect Severity, Other}}\\
\midrule
3. Incorrect location/position of finding &
New \hlommb{left} lower lobe infiltrate	&
New \hlommb{right} lower lobe infiltrate	&
Left vs. right \\\addlinespace

4. Incorrect severity of finding &
In comparison prior exam, \hlommb{there is near-complete resolution} of bilateral pleural effusion &
In comparison to \_ exam, \hlommb{there is resolution} of bilateral pleural effusion	&
Near complete vs. resolved \\\addlinespace

5. Other &
Slight increased hazy \hlommb{opacities} at the right lung base which may reflect developing consolidation &
Slight increased hazy \hlommb{opacity} at the right lung base which may reflect developing consolidation &
Difference in multiplicity
\\\addlinespace

5. Other &
left picc terminates within the \hlommb{upper} svc.	&
left picc terminates within the \hlommb{proximal} svc. &
Ambiguous location \\\addlinespace

5. Other &
No acute abnormalities identified to explain patient's cough \hlommb{and asthma flare}. &
no acute abnormalities identified to explain patient's cough. &
Asthma flare is a symptom, which was not mentioned in the candidate. \\
\bottomrule
\end{tabular}
}
\caption{Examples for all error categories.}
\label{tab:app:guidelines:examples}
\end{table*}

