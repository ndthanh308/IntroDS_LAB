\clearpage
\clearpage
\section{Annotation Guidelines}
\label{sec:appendix-annotation-guidelines}

\paragraph{Introduction.}
We consider automatic impression generation for English radiology reports of chest imaging examinations. These reports conventionally have three sections (example in \cref{fig:app:report}).

\begin{enumerate}[noitemsep,parsep=0pt,partopsep=0pt]
    \item \textbf{Background.} A description of the exam, patient information, and relevant prior exams.
    \item \textbf{Findings.} A description or itemization of the radiologists' observations based on the radiographs.
    \item \textbf{Impression.} A concise summary of the most important findings, including inferences and any recommendations.
\end{enumerate}

% Figure environment removed

\paragraph{Study setup.}
We are looking to determine typical failures that automatic impression generation systems make. You will be shown a reference impression and four candidate impressions. The reference was written by a radiologist, and the candidates were generated by automatic systems. Your goal is to judge the accuracy of candidate impressions \emph{based on a reference impression}. For each candidate, you will be asked to identify any errors that it may have.

\paragraph{Definition of ``error.''} We define an error as a difference between the candidate and reference. An error can be one of the following:

\begin{enumerate}[noitemsep,parsep=0pt,partopsep=0pt]
\item Omissions
    \begin{enumerate}[(a),noitemsep,topsep=0pt,parsep=0pt,partopsep=0pt]
        \item Omission of finding/interpretation
        \item Omission of comparison describing a change from a previous examination
        \item Omission of reference to prior report while making a comparison
        \item Omission of next steps, recommendation, communications
    \end{enumerate}
\item Additions
    \begin{enumerate}[(a),noitemsep,topsep=0pt,parsep=0pt,partopsep=0pt]
        \item Additional finding/interpretation
        \item Mention a comparison that is not present in reference
        \item Additional reference to prior report while making a comparison
        \item Additional mention of next steps, recommendation, communications
        \item Additional finding/interpretation which contradicts reference
    \end{enumerate}
\item Incorrect location/position of finding
\item Incorrect severity of finding
\item Other difference between candidate and reference (please describe...)
\end{enumerate}
%
\cref{tab:app:guidelines:examples} shows an example for each error category.

\paragraph{Annotating errors as inline annotations.}
You are asked to annotate errors as \emph{inline annotations}. For each error that you identify, first select the error category and then highlight the relevant text snippet with your mouse. This applies the category. If you have to remove an annotation, press on the highlight and use your backspace/delete key (\keystroke{Entf} or \keystroke{\textleftarrow Backspace}). If one of the above categories occurs multiple times, please annotate all of them \emph{separately} (see \cref{fig:app:guidelines:ex1}). Some general guidelines:

\begin{itemize}[noitemsep]
\item A candidate may have multiple errors, so please add all that apply.
\item Some candidates will be the same, so please assign the same errors to all candidates.
\item For additional findings that are plausible, pick \textcolor{purple}{2a. Additional finding/interpretation}. In the context of the full report, these additions may be correct. What this category aims to capture is that the system included information which the radiologist chose not to include. If a finding contradicts the reference, select \textcolor{purple}{2e. Additional finding/interpretation which contradicts reference}.
\item Use \textcolor{purple}{5. Other} freely, especially if you find it difficult to assign any of the above categories. These remarks help us to better understand and characterize potential errors.
\item You can ignore differences in word choice if they are synonymous. Example: \emph{``may reflect developing consolidation''} is equal to \emph{``could represent early consolidation.''}
\end{itemize}
%
Finally, always use your best judgment when assessing the reports. If you are in doubt, you can add any questions/comments about the report or the error categories in the given box.

% Figure environment removed

\subsection*{Corner Cases}
\paragraph{How to annotate ``3. Incorrect location/position of finding'' and ``4. Incorrect severity of finding''?}
Only apply if both reference and candidate mention a finding, \emph{and} when there is a mismatch in severity/location. In the example below, both mention effusion, but the reference does not specify the size of effusion, whereas the candidate states that there are ``small'' effusions. Therefore, apply \textcolor{purple}{4. Incorrect severity of finding}.

\noindent\fbox{%
\parbox{0.98\linewidth}{%
\begin{sanseriffont}
\textbf{Reference:}
interval worsening of now moderate interstitial pulmonary edema. bilateral pleural effusions.\\
\textbf{Candidate:}
interval worsening of now moderate interstitial pulmonary edema. small bilateral pleural effusions.
\end{sanseriffont}
}}

\paragraph{Opacities vs. consolidation.}
Often, opacities are used in place of consolidation and vice versa. In those cases, apply \textcolor{purple}{5. Other} with a comment similar to \textcolor{purple}{``opacities not equal consolidation, but otherwise correct''}.

\noindent\fbox{%
\parbox{0.98\linewidth}{%
\begin{sanseriffont}
\textbf{Reference:}
Improved right lower medial lung peribronchial consolidation. \\
\textbf{Candidate:}
right lower medial lung peribronchial opacities have improved.
\end{sanseriffont}
}}

\paragraph{No acute abnormality vs. COPD.}
Does ``no acute abnormality'' contradict ``COPD''? No, for the purposes of our evaluation, COPD is not an \emph{acute} disease, so this is not contradicting. In the example below, following categories apply: (1) ``COPD'' is missing $\rightarrow$ \textcolor{purple}{1a. Omission of finding/interpretation}, (2) ``opacity is resolved'' $\rightarrow$ \textcolor{purple}{1b. Omission of comparison describing a change from a previous examination}, (3) ``no acute cardiopulmonary abnormality'' $\rightarrow$  \textcolor{purple}{2a. Additional finding/interpretation}.

\noindent\fbox{%
\parbox{0.98\linewidth}{%
\begin{sanseriffont}
\textbf{Reference:}
Left basilar opacity is resolved. COPD. \\
\textbf{Candidate:}
no acute cardiopulmonary abnormality.
\end{sanseriffont}
}}

\paragraph{Misleading grammar or sentence structure.}
In general, disregard grammatical errors. However, please pay attention to any \emph{logical flaws} that arise because of grammar errors or a misleading sentence structure. In the example below, the \emph{``and''} in the candidate implies that both ``bronchiectasis'' and ``peribronchial consolidation'' have improved, whereas the reference only states that the consolidation has improved. In those cases, apply \textcolor{purple}{5. Other} and add a comment similar to \textcolor{purple}{``logical error because of grammar.''}

\noindent\fbox{%
\parbox{0.95\linewidth}{%
\begin{sanseriffont}
\textbf{Reference:}
Bilateral lower lung bronchiectasis with improved peribronchial consolidation\\
\textbf{Candidate:}
bilateral lower lung bronchiectasis and peribronchial consolidation have improved since \_.
\end{sanseriffont}
}}

\begin{table*}[t]
\small
\centering
\resizebox{\textwidth}{!}{
\begin{tabular}{p{10em} p{13em} p{13em} p{12em} }
\toprule
\textbf{Error} & \textbf{Reference} & \textbf{Candidate} & \textbf{Explanation} \\
\midrule

\multicolumn{4}{l}{\emph{Omissions (apply to reference)}}\\\midrule
1a. Omission of finding/interpretation &
New left lower lobe infiltrate \hlomma{and effusion}. &
New left lower lobe infiltrate.	&
Effusion is missing. \\\addlinespace

1b. Omission of comparison describing a change from a previous examination &
\hlommb{In comparison to \_ exam, there is interval near-complete resolution of bilateral pleural effusion.}	&
No evidence of acute cardiopulmonary process. &
Resolution of effusion is not described, therefore the comparison is missing. \\\addlinespace

1c. Omission of reference to prior report while making a comparison &
Increased pulmonary edema \hlommc{compared to \_.}	&
increased pulmonary edema.	&
While the candidate correctly states that the edema has increased, it lacks the reference to the prior report (or the date of it). \\\addlinespace

1d. Omission of next steps / recommendation / communications &
No pneumothorax or pneumomediastinum. \hlommd{Recommend repeat PA and lateral imaging later today to verify these findings.} Otherwise unremarkable chest radiograph. \hlommd{These findings were communicated to Dr. \_ at 11:55 a.m. by telephone by Dr. \_.}	&
No pneumothorax or pneumomediastinum. &
The candidate does not include the followup \emph{(recommend repeat PA)} and the remark about a communication with another doctor \emph{(These findings were communicated [...])}. \\\addlinespace

\midrule
\multicolumn{4}{l}{\emph{Additions (apply to candidate)}}\\
\midrule
2a. Additional finding / interpretation	&
Slight increased hazy opacities at the right lung base which may reflect developing consolidation.	&
slightly increased hazy opacities at the right lung base which may represent \hlomma{atelectasis or} developing consolidation. &
Atelectasis is not mentioned in the reference. This finding is not contradicting the reference. It may be correct in the context of the full report. \emph{Same as 1a, but in the other direction.} \\\addlinespace

2b. Mention a comparison that is not present in reference	&
Mild to moderate pulmonary edema, increased from \_.	&
Mild to moderate pulmonary edema, increased from \_. \hlommb{Stable cardiomegaly.} &
``Stable'' suggests that the state of a finding was compared to a previous examination. This comparison is not made in the reference. \emph{Same as 1b, but in the other direction.} \\\addlinespace

2c. Additional reference to prior report while making a comparison	&
&
&
\emph{Same as 1c, but in the other direction.}\\\addlinespace

2d. Additional mention of next steps / recommendation / communications	&
&
&
\emph{Same as 1d, but in the other direction.}\\\addlinespace

2e. Additional finding / interpretation which contradicts reference &
Unchanged size and position of right-sided hydropneumothorax. &
\hladde{Development of new} right-sided hydropneumothorax &
Unchanged vs. development of new \\\addlinespace

\midrule
\multicolumn{4}{l}{\emph{Incorrect location, Incorrect Severity, Other}}\\
\midrule
3. Incorrect location/position of finding &
New \hlommb{left} lower lobe infiltrate	&
New \hlommb{right} lower lobe infiltrate	&
Left vs. right \\\addlinespace

4. Incorrect severity of finding &
In comparison prior exam, \hlommb{there is near-complete resolution} of bilateral pleural effusion &
In comparison to \_ exam, \hlommb{there is resolution} of bilateral pleural effusion	&
Near complete vs. resolved \\\addlinespace

5. Other &
Slight increased hazy \hlommb{opacities} at the right lung base which may reflect developing consolidation &
Slight increased hazy \hlommb{opacity} at the right lung base which may reflect developing consolidation &
Difference in multiplicity
\\\addlinespace

5. Other &
left picc terminates within the \hlommb{upper} svc.	&
left picc terminates within the \hlommb{proximal} svc. &
Ambiguous location \\\addlinespace

5. Other &
No acute abnormalities identified to explain patient's cough \hlommb{and asthma flare}. &
no acute abnormalities identified to explain patient's cough. &
Asthma flare is a symptom, which was not mentioned in the candidate. \\
\bottomrule
\end{tabular}
}
\caption{Examples for all error categories.}
\label{tab:app:guidelines:examples}
\end{table*}
