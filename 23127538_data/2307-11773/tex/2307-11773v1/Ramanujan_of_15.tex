
% Template for Elsevier CRC journal article
% version 1.2 dated 09 May 2011

% This file (c) 2009-2011 Elsevier Ltd.  Modifications may be freely made,
% provided the edited file is saved under a different name

% This file contains modifications for Procedia Computer Science
% but may easily be adapted to other journals

% Changes since version 1.1
% - added "procedia" option compliant with ecrc.sty version 1.2a
%   (makes the layout approximately the same as the Word CRC template)
% - added example for generating copyright line in abstract

%-----------------------------------------------------------------------------------

%% This template uses the elsarticle.cls document class and the extension package ecrc.sty
%% For full documentation on usage of elsarticle.cls, consult the documentation "elsdoc.pdf"
%% Further resources available at http://www.elsevier.com/latex

%-----------------------------------------------------------------------------------

%%%%%%%%%%%%%%%%%%%%%%%%%%%%%%%%%%%%%%%%%%%%%%%%%%%%%%%%%%%%%%
%%%%%%%%%%%%%%%%%%%%%%%%%%%%%%%%%%%%%%%%%%%%%%%%%%%%%%%%%%%%%%
%%                                                          %%
%% Important note on usage                                  %%
%% -----------------------                                  %%
%% This file should normally be compiled with PDFLaTeX      %%
%% Using standard LaTeX should work but may produce clashes %%
%%                                                          %%
%%%%%%%%%%%%%%%%%%%%%%%%%%%%%%%%%%%%%%%%%%%%%%%%%%%%%%%%%%%%%%
%%%%%%%%%%%%%%%%%%%%%%%%%%%%%%%%%%%%%%%%%%%%%%%%%%%%%%%%%%%%%%

%% The '3p' and 'times' class options of elsarticle are used for Elsevier CRC
%% Add the 'procedia' option to approximate to the Word template
%\documentclass[3p,times,procedia]{elsarticle}
\documentclass[3p,times]{elsarticle}

%% The `ecrc' package must be called to make the CRC functionality available
\usepackage{ecrc}

%% The ecrc package defines commands needed for running heads and logos.
%% For running heads, you can set the journal name, the volume, the starting page and the authors

%% set the volume if you know. Otherwise `00'
\volume{253}

%% set the starting page if not 1
\firstpage{1}

%% Give the name of the journal
\journalname{Journal of Number Theory}

%% Give the author list to appear in the running head
%% Example \runauth{C.V. Radhakrishnan et al.}
\runauth{C. D. Zhang, L. Yang}

%% The choice of journal logo is determined by the \jid and \jnltitlelogo commands.
%% A user-supplied logo with the name <\jid>logo.pdf will be inserted if present.
%% e.g. if \jid{yspmi} the system will look for a file yspmilogo.pdf
%% Otherwise the content of \jnltitlelogo will be set between horizontal lines as a default logo

%% Give the abbreviation of the Journal.  Contact the journal editorial office if in any doubt
\jid{jnt}

%% Give a short journal name for the dummy logo (if needed)
\jnltitlelogo{Journal of Number Theory}

%% Provide the copyright line to appear in the abstract
%% Usage:
%   \CopyrightLine[<text-before-year>]{<year>}{<restt-of-the-copyright-text>}
%   \CopyrightLine[Crown copyright]{2011}{Published by Elsevier Ltd.}
%   \CopyrightLine{2011}{Elsevier Ltd. All rights reserved}
\CopyrightLine{2023}{Published by Elsevier Ltd.}

%% Hereafter the template follows `elsarticle'.
%% For more details see the existing template files elsarticle-template-harv.tex and elsarticle-template-num.tex.

%% Elsevier CRC generally uses a numbered reference style
%% For this, the conventions of elsarticle-template-num.tex should be followed (included below)
%% If using BibTeX, use the style file elsarticle-num.bst

%% End of ecrc-specific commands
%%%%%%%%%%%%%%%%%%%%%%%%%%%%%%%%%%%%%%%%%%%%%%%%%%%%%%%%%%%%%%%%%%%%%%%%%%

%% The amssymb package provides various useful mathematical symbols
\usepackage{amsmath,amssymb}
%% The amsthm package provides extended theorem environments
\usepackage{amsthm}

\theoremstyle{plain}
\newtheorem{theorem}{Theorem}[section]
\newtheorem{lemma}[theorem]{Lemma}    % theorem 和 lemma 环境是 plain 样式
\theoremstyle{definition}
\newtheorem{example}{Example}[section] % example 环境是 definition 样式


%% The lineno packages adds line numbers. Start line numbering with
%% \begin{linenumbers}, end it with \end{linenumbers}. Or switch it on
%% for the whole article with \linenumbers after \end{frontmatter}.
%% \usepackage{lineno}

%% natbib.sty is loaded by default. However, natbib options can be
%% provided with \biboptions{...} command. Following options are
%% valid:

%%   round  -  round parentheses are used (default)
%%   square -  square brackets are used   [option]
%%   curly  -  curly braces are used      {option}
%%   angle  -  angle brackets are used    <option>
%%   semicolon  -  multiple citations separated by semi-colon
%%   colon  - same as semicolon, an earlier confusion
%%   comma  -  separated by comma
%%   numbers-  selects numerical citations
%%   super  -  numerical citations as superscripts
%%   sort   -  sorts multiple citations according to order in ref. list
%%   sort&compress   -  like sort, but also compresses numerical citations
%%   compress - compresses without sorting
%%
%% \biboptions{comma,round}

% \biboptions{}

% if you have landscape tables
\usepackage[figuresright]{rotating}

% put your own definitions here:
%   \newcommand{\cZ}{\cal{Z}}
%   \newtheorem{def}{Definition}[section]
%   ...

% add words to TeX's hyphenation exception list
%\hyphenation{author another created financial paper re-commend-ed Post-Script}

% declarations for front matter

\begin{document}

\begin{frontmatter}
	
	%% Title, authors and addresses
	
	%% use the tnoteref command within \title for footnotes;
	%% use the tnotetext command for the associated footnote;
	%% use the fnref command within \author or \address for footnotes;
	%% use the fntext command for the associated footnote;
	%% use the corref command within \author for corresponding author footnotes;
	%% use the cortext command for the associated footnote;
	%% use the ead command for the email address,
	%% and the form \ead[url] for the home page:
	%%
	%% \title{Title\tnoteref{label1}}
	%% \tnotetext[label1]{}
	%% \author{Name\corref{cor1}\fnref{label2}}
	%% \ead{email address}
	%% \ead[url]{home page}
	%% \fntext[label2]{}
	%% \cortext[cor1]{}
	%% \address{Address\fnref{label3}}
	%% \fntext[label3]{}
	
	\dochead{}
	%% Use \dochead if there is an article header, e.g. \dochead{Short communication}
	%% \dochead can also be used to include a conference title, if directed by the editors
	%% e.g. \dochead{17th International Conference on Dynamical Processes in Excited States of Solids}
	
	\title{Some New Ramanujan’s Modular Equations of Degree 15}
	%% use optional labels to link authors explicitly to addresses:
	%% \author[label1,label2]{<author name>}
	%% \address[label1]{<address>}
	%% \address[label2]{<address>}
	\tnotetext[NSFC]{This work was supported by the National Natural Science Foundation of China (Grant No. 42174037).}	
	
	\author[1,2]{Zhang Chuan-Ding \corref{cor1}}
	\cortext[cor1]{Corresponding author}
	\ead{13607665382@163.net}
	
	
	\author[1,2]{Yang Li}
	\ead{241912296@qq.com}
	
	\address[1]{College of Geography and Environmental Science, Henan University, Zhengzhou 450046, China}
	\address[2]{Henan Technology Innovation Center of Spatio-Temporal Big Data, Henan University, Zhengzhou 450046, China}
	
	
	\begin{abstract}
		%% Text of abstract
		Ramanujan in his notebook recorded two modular equations involving multiplier with moduli of degrees (1,7)  and (1,23).   
		In this paper, we find some new Ramanujan’s modular equations  involving multiplier with moduli of degrees (3,5)  and (1,15), and give concise proofs by employing Ramanujan’s multiplier function equation.
	\end{abstract}
	
	\begin{keyword}
		%% keywords here, in the form: keyword \sep keyword
		Theta functions \sep Modular equations \sep Multiplier
		%% PACS codes here, in the form: \PACS code \sep code
		
		%% MSC codes here, in the form: \MSC code \sep code
		%% or \MSC[2008] code \sep code (2000 is the default)
		
		\MSC[2000] 11F20 \sep 33C05 
		
	\end{keyword}
	
\end{frontmatter}

%%
%% Start line numbering here if you want
%%
%%\linenumbers

%% main text

\section{Introduction}
\label{}

Ramanujan defined modular equations as follows: Suppose that
\begin{equation}\label{eq01}
	n \frac{{}_2 F_1\left(\frac{1}{2}, \frac{1}{2} ; 1 ; 1-\alpha\right)}{{}_2 F_1\left(\frac{1}{2}, \frac{1}{2} ; 1 ; \alpha\right)}=\frac{{}_2 F_1\left(\frac{1}{2}, \frac{1}{2} ; 1 ; 1-\beta\right)}{{}_2 F_1\left(\frac{1}{2}, \frac{1}{2} ; 1 ; \beta\right)},
\end{equation}
holds for some positive integer $n$ or positive rational fraction $n=i/j$ (i and j are coprime). The relation between $\alpha$ and $\beta$ induced by the above equation is called a modular equation of degree $n$ and in such equations, we say that $\beta$ over $\alpha$ is degree $n$. Let
\begin{equation}\label{eq02}
	z_1={}_2 F_1\left(\frac{1}{2}, \frac{1}{2} ; 1 ; \alpha\right) ,\quad z_n={}_2 F_1\left(\frac{1}{2}, \frac{1}{2} ; 1 ; \beta\right),
\end{equation}
the multiplier $m$ of the modular equation is defined by
\begin{equation}\label{eq03}
	m:=\frac{z_1}{z_n}.
\end{equation}

Before proceeding to the main theta function identity of this paper, we shall first
recall certain known theta function identities which we need in the sequel. Throughout
the paper, we assume $|q|<1$. The following is the well-known Jacobi triple product
identity:
\begin{equation}\label{eq04}
	f(a, b):=\sum_{n=-\infty}^{\infty} a^{n(n+1) / 2} b^{n(n-1) / 2}=(-a ; a b)_{\infty}(-b ; a b)_{\infty}(a b ; a b)_{\infty}, \quad|a b|<1,
\end{equation}
the two particular facts of $f(a, b)$ (See \cite{Bruce1991},Entry 22, p36), are as follows
\begin{equation}\label{eq05}
	\varphi(q):=f(q, q)=\left(-q ; q^2\right)_{\infty}^2\left(q^2 ; q^2\right)_{\infty}, \quad \psi(q):=f\left(q, q^3\right)=\left(-q ; q^2\right)_{\infty}\left(q^2 ; q^2\right)_{\infty}
	.
\end{equation}

Again, let the base $q$ in the classical theory of elliptic functions is defined by 
\begin{equation}\label{eq06}
	q=\exp \left(-\pi\frac{{}_2 F_1\left(\frac{1}{2}, \frac{1}{2} ; 1 ; 1-\alpha\right)}{{}_2 F_1\left(\frac{1}{2}, \frac{1}{2} ; 1 ; \alpha\right)}\right),
\end{equation}
we have the following well-known identities (See \cite{Bruce1991},Entry 10 and 11, pp. 122-123)
\begin{equation}\label{eq07}
	z_1=\varphi^2(q),\quad z_n=\varphi^2(q^n),\quad m=\frac{\varphi^2(q)}{\varphi^2(q^n)},
\end{equation}
and
\begin{equation}\label{eq08}
	\frac{\sqrt[4]{q} \psi(q^2)}{\varphi (q)}= \frac{\sqrt[4]{\alpha }}{2},\quad \frac{\varphi (-q)}{\varphi (q)}= \sqrt[4]{1-\alpha },\quad \frac{q^{n/4} \psi (q^{2 n})}{\varphi \left(q^n\right)}= \frac{\sqrt[4]{\beta }}{2},\quad \frac{\varphi \left(-q^n\right)}{\varphi \left(q^n\right)}= \sqrt[4]{1-\beta },
\end{equation}
and
\begin{equation}\label{eq09}
	\frac{\sqrt[8]{q} \psi (q)}{\varphi (q)}= \frac{\sqrt[8]{\alpha }}{\sqrt{2}},\quad \frac{\varphi (-q^2)}{\varphi (q)}= \sqrt[8]{1-\alpha },\quad \frac{q^{n/8} \psi(q^n)}{\varphi \left(q^n\right)}= \frac{\sqrt[8]{\beta }}{\sqrt{2}},\quad \frac{\varphi (-q^{2 n})}{\varphi \left(q^n\right)}= \sqrt[8]{1-\beta }.
\end{equation}

Ramanujan found and recorded the following  modular equations involving multiplier with moduli of degrees 7  and 23,in tha case  degree $(n \bmod 8)=7$, respectively.
\begin{theorem}[See \cite{Bruce1991},Entry 19, pp. 314-315]If $\beta$ has degree 7 over $\alpha$, and $m$ is the multiplier for degree 7 , then
	\begin{equation}\label{eq10}
		(\alpha \beta)^{1 / 8}+\{(1-\alpha)(1-\beta)\}^{1 / 8}=1,
	\end{equation}
	and
	\begin{equation}\label{eq11}
		m-\frac{7}{m}=2\left((\alpha \beta)^{1 / 8}-\{(1-\alpha)(1-\beta)\}^{1 / 8}\right)\left(2+(\alpha \beta)^{1 / 4}+\{(1-\alpha)(1-\beta)\}^{1 / 4}\right),
	\end{equation}
	and
	\begin{equation}\label{eq12}
		\frac{\varphi^2(q)}{\varphi^2(q^7)}-7 \frac{\varphi^2(q^7)}{\varphi^2(q)}=2\left(2 q \frac{\psi(q) \psi(q^7)}{\varphi(q)\varphi(q^7)} -\frac{\varphi (-q^2)\varphi (-q^{14})}{\varphi(q)\varphi(q^7)}\right) \left(2+4 q^2 \frac{\psi(q^2) \psi(q^{14})}{\varphi(q)\varphi(q^7)}+\frac{\varphi (-q)\varphi (-q^7)}{\varphi(q)\varphi(q^7)}\right).
	\end{equation}
\end{theorem}
\begin{theorem}[See \cite{Bruce1991},Entry 15, p.411]
	If $\beta$ is of the 23 degree over $\alpha$,  and $m$ is the multiplier for same degree, then
	
	\begin{equation}\label{eq13}
		(\alpha \beta)^{1 / 8}+\{(1-\alpha)(1-\beta)\}^{1 / 8}+2^{2 / 3}\{\alpha \beta(1-\alpha)(1-\beta)\}^{1 / 24}=1,
	\end{equation}
	and
	\begin{equation}\label{eq14}
		\begin{aligned}
			m-\frac{23}{m} &= 2\left((\alpha \beta)^{1 / 8}-\{(1-\alpha)(1-\beta)\}^{1 / 8}\right)\left(11-13 \cdot 4^{1 / 3}\{\alpha \beta(1-\alpha)(1-\beta)\}^{1 / 24}\right. \\
			& +18 \cdot 2^{1 / 3}\{\alpha \beta(1-\alpha)(1-\beta)\}^{1 / 12}-14\{\alpha \beta(1-\alpha)(1-\beta)\}^{1 / 8} \left.+2^{5 / 3}\{\alpha \beta(1-\alpha)(1-\beta)\}^{1 / 6}\right) .
		\end{aligned}
	\end{equation}
\end{theorem}
Berndt finished the proof of (\ref{eq14}) by the theory of modular forms (See \cite{Bruce1991},Entry 15, pp. 415-416). In this paper, we also find some new Ramanujan’s modular equations involving multiplier with moduli of degrees (3,5) and (1,15).

\section{Some New Ramanujan’s Modular Equations of  degrees 15 and $\frac{5}{3}$}
The beautiful Ramanujan’s modular equations of degrees 15 and $\frac{5}{3}$ (See \cite{Bruce1991},Entry 21, p435): Let $\alpha$ and $\beta$ have degrees $(n_1,n_2)=(3,5)$; or $(1, 15)$, respectively. Then
\begin{equation}\label{eq15}
	(\alpha \beta)^{1 / 8}+\{(1-\alpha)(1-\beta)\}^{1 / 8} \pm\{\alpha \beta(1-\alpha)(1-\beta)\}^{1 / 8} =\sqrt{\frac{1}{2}(1+\sqrt{\alpha \beta}+\sqrt{(1-\alpha)(1-\beta)})},
\end{equation}
where the minus sign is chosen in the first case and the plus sign is selected in the last case.

For modular equations in the form of Russell, we redefine (See \cite{Bruce1998},Entry 21, p435),
\begin{equation}\label{eq16}
	\left\{
	\begin{aligned}
		& P=1+(-1)^{\frac{n1+n2}{8}} \left((\alpha \beta)^{1 / 8}+\{(1-\alpha)(1-\beta)\}^{1 / 8}\right),
		\\
		& Q=4\left((\alpha \beta)^{1 / 8}+\{(1-\alpha)(1-\beta)\}^{1 / 8} +(-1)^{\frac{n1+n2}{8}} \{\alpha \beta(1-\alpha)(1-\beta)\}^{1 / 8}\right),\\
		&R=4(\alpha \beta(1-\alpha)(1-\beta))^{1 / 8} ,
	\end{aligned}\right.
\end{equation}
then, if $\alpha$ and $\beta$ have degrees $(n_1,n_2)=(1, 15)$ (See \cite{Bruce1998},Entry 21, p435),
\begin{equation}\label{eq17}
	P(P^2-Q)+R=0,
\end{equation}
and, if $\alpha$ and $\beta$ have degrees $(n_1,n_2)=(3, 5)$,
\begin{equation}\label{eq18}
	P(P^2+Q)+R=0.
\end{equation}
We can verify that these two modular equations (\ref{eq17}) and (\ref{eq18}) are equivalent formulations of (\ref{eq15}).

\begin{theorem}[new modular equations for degrees $(n_1,n_2)=(1, 15)$] Let $\alpha$ and $\beta$ have degrees $(n_1,n_2)=(1, 15)$, and $m=\frac{z_1}{z_{15}}$ is the multiplier for degree $n=15$, then 
	\begin{enumerate}[(i)]
		\item natural form	
		\begin{equation}\label{eq19}	
			\begin{aligned}
				m-\frac{15}{m} =&2 \left((\alpha \beta)^{1 / 8}-\{(1-\alpha)(1-\beta)\}^{1 / 8}\right) \left[1+3\left((\alpha \beta)^{1 / 8}+\{(1-\alpha)(1-\beta)\}^{1 / 8}\right)\right.\\
				&\left.+3\left((\alpha \beta)^{1 / 4}+\{(1-\alpha)(1-\beta)\}^{1 / 4}\right)+2 (\alpha \beta (1-\alpha)(1-\beta))^{1 / 8}\left(3+(\alpha \beta)^{1 / 8}+\{(1-\alpha)(1-\beta)\}^{1 / 8}\right) \right],
			\end{aligned}
		\end{equation}		
		\item emphatic form		
		\begin{equation}\label{eq20}	
			\begin{aligned}
				m-\frac{15}{m} =&2 \left((\alpha \beta)^{1 / 4}-\{(1-\alpha)(1-\beta)\}^{1 / 4}\right) \left[4\sqrt{\frac{1}{2}(1+\sqrt{\alpha \beta}+\sqrt{(1-\alpha)(1-\beta)})}\right.\\
				&\left.+4-\left((\alpha \beta)^{1 / 4}+\{(1-\alpha)(1-\beta)\}^{1 / 4}\right) \right],
			\end{aligned}
		\end{equation}	
		\item Ramanujan’s theta functions form	
		\begin{equation}\label{eq21}	
			\begin{aligned}
				\frac{\varphi^2(q)}{\varphi^2(q^{15})}-15 \frac{\varphi^2(q^{15})}{\varphi^2(q)}=&2 \left(4 q^4 \frac{\psi(q^2) \psi(q^{30})}{\varphi(q)\varphi(q^{15})}-\frac{\varphi (-q)\varphi (-q^{15})}{\varphi(q)\varphi(q^{15})}\right) \left[4\left(\frac{\varphi (q^2) \varphi (q^{30})}{\varphi (q) \varphi \left(q^{15}\right)}+4 q^8\frac{ \psi (q^4) \psi(q^{60})}{\varphi (q) \varphi \left(q^{15}\right)}\right)\right.\\
				&\left.+4-\left(4 q^4 \frac{\psi(q^2) \psi(q^{30})}{\varphi(q)\varphi(q^{15})}+\frac{\varphi (-q)\varphi (-q^{15})}{\varphi(q)\varphi(q^{15})}\right) \right].
			\end{aligned}
		\end{equation}	
	\end{enumerate}
	
\end{theorem}

\begin{proof} (i).	According to Ramanujan’s multiplier function equation (See \cite{Bruce1991},Entry 24(vi), p. 217), we can write  
	\begin{equation}\label{eq22}
		n \frac{\operatorname{d} \alpha}{\operatorname{d} \beta}=\frac{\alpha(1-\alpha)}{\beta(1-\beta)} m^2 .
	\end{equation}
	
	Let 
	\begin{equation}\label{eq23}
		\begin{aligned}
			&x:=x(t)=(\alpha \beta)^{1 / 8},\\
			&y:= y(t) =\{(1-\alpha)(1-\beta)\}^{1 / 8},
		\end{aligned}
	\end{equation}
	we can get 
	\begin{equation} \label{eq24}
		\begin{aligned}
			&\alpha:=\alpha(t) = \frac{1}{2} \left(1+x^8-y^8+\sqrt{1-2 x^8-2 y^8-2 x^8 y^8+x^{16}+y^{16}}\right),\\
			&\beta:=\beta(t) = \frac{1}{2} \left(1+x^8-y^8-\sqrt{1-2 x^8-2 y^8-2 x^8 y^8+x^{16}+y^{16}}\right),
		\end{aligned}
	\end{equation}
	and using the equations (\ref{eq22}) - (\ref{eq24}), we deduce
	\begin{equation}\label{eq25}
		\frac{n}{m^2} = \frac{\alpha(1-\alpha)  }{\beta(1-\beta)} \frac{\operatorname{d}\beta/\operatorname{d} t}{\operatorname{d}\alpha/\operatorname{d} t}=-\frac{\alpha  y x'(t)+(1-\alpha ) x y'(t)}{\beta  y x'(t)+(1-\beta ) x y'(t)}, 
	\end{equation}
	and	
	\begin{equation}\label{eq26}
		\left(m-\frac{n}{m}\right)^2=-\frac{n \left(y (\alpha +\beta ) x'(t)+x (-\alpha -\beta +2) y'(t)\right)^2}{\left(\alpha  y x'(t)+(1-\alpha ) x y'(t)\right) \left(\beta  y x'(t)+(1-\beta ) x y'(t)\right)}.
	\end{equation}
	
	Again, from the equation (\ref{eq17}), we have equivalent modular equation of degrees 15 in the form of $x$ and $y$, it reads
	\begin{equation}\label{eq27}
		1 - x - x^2 + x^3 - y - 2 x y - x^2 y - y^2 - x y^2 + y^3=0,
	\end{equation}
	from the equation (\ref{eq19}) and (\ref{eq23}), we also have
	\begin{equation}\label{eq28}
		m-\frac{15}{m} =2(x-y) (1 + 3 x + 3 x^2 + 3 y + 6 x y + 2 x^2 y + 3 y^2 + 2 x y^2).
	\end{equation}
	
	Now set
	\begin{equation}\label{eq29}
		x = \frac{1}{2} \left(\frac{1}{t}-\varrho \right),\quad y = \frac{1}{2} \left(\frac{1}{t}+\varrho \right),
	\end{equation}
	using the equation (\ref{eq27}), we obtain
	\begin{equation}\label{eq30}
		t \varrho^2= 1 + t - t^2,
	\end{equation}
	solving the above equation for $\varrho$ and notice that $x<y$, we get
	\begin{equation}\label{eq31}
		\varrho = \frac{\sqrt{1+t-t^2}}{\sqrt{t}},x = \frac{1}{2} \left(\frac{1}{t}-\frac{\sqrt{1+t-t^2}}{\sqrt{t}}\right),\quad y = \frac{1}{2} \left(\frac{1}{t}+\frac{\sqrt{1+t-t^2}}{\sqrt{t}} \right).
	\end{equation}
	
	Employing (\ref{eq31}) and (\ref{eq24}) in (\ref{eq26}) with $n=15$, we deduce
	\begin{equation}\label{eq32}
		\left(m-\frac{15}{m}\right)^2= \frac{\left(1+t-t^2\right) \left(1 + 5 t + 5 t^2 + 3 t^3\right)^2}{t^7},
	\end{equation}
	and employing (\ref{eq31}) in (\ref{eq28}), we obtain 
	\begin{equation}\label{eq33}
		m-\frac{15}{m} =-\frac{\sqrt{1+t-t^2} \left(1 + 5 t + 5 t^2 + 3 t^3\right)}{t^{7/2}}.
	\end{equation}

We can verify that these two equations (\ref{eq32}) and (\ref{eq33}) are equivalent.	So far, we employed the method of parameterization to prove the modular equation of degree 15 which involves the multiplier $m$.
\end{proof}

\begin{proof} (ii).
	Substituting (\ref{eq27}) into (\ref{eq28}),  we obtain
	\begin{equation}\label{eq34}
		\begin{aligned}
			L&=(1 + 3 x + 3 x^2 + 3 y + 6 x y + 2 x^2 y + 3 y^2 + 2 x y^2)-0\\
			&=L-(1 - x - x^2 + x^3 - y - 2 x y - x^2 y - y^2 - x y^2 + y^3)\\
			&=(x + y) (4 (x + y + x y)+4 - x^2 - y^2),
		\end{aligned}
	\end{equation}
	By using the equation (\ref{eq15}), we can write 
	\begin{equation}\label{eq35}
		x+y+xy =\sqrt{\frac{1}{2}(1+\sqrt{\alpha \beta}+\sqrt{(1-\alpha)(1-\beta)})},
	\end{equation}
	Substituting (\ref{eq34}) and (\ref{eq35}) into  (\ref{eq28}),  we arrive at
	\begin{equation}\label{eq36}
		m-\frac{15}{m} =2(x^2-y^2) (4 \sqrt{\frac{1}{2}(1+\sqrt{\alpha \beta}+\sqrt{(1-\alpha)(1-\beta)})}+4 - x^2 - y^2).
	\end{equation}
	which completes the proof.	
\end{proof}	

\begin{proof} (iii).If $\beta$ over $\alpha$ is degree 15, we employ the identity (See \cite{Bruce1991},p. 433 eq. (20.6))
	\begin{equation}\label{eq37}	
		\sqrt{\frac{1}{2}(1+\sqrt{\alpha \beta}+\sqrt{(1-\alpha)(1-\beta)})}=\frac{\sqrt{1-\sqrt{1-\alpha }}}{\sqrt{2}}\frac{\sqrt{1-\sqrt{1-\beta }}}{\sqrt{2}}+\frac{\sqrt{1+\sqrt{1-\alpha }}}{\sqrt{2}}\frac{\sqrt{1+\sqrt{1-\beta }}}{\sqrt{2}},
	\end{equation}
	and translate this identity in form of Ramanujan’s theta functions (See \cite{Bruce1991},Entry 10 and 11, pp. 122-123)
	\begin{equation}\label{eq38}	
		\sqrt{\frac{1}{2}(1+\sqrt{\alpha \beta}+\sqrt{(1-\alpha)(1-\beta)})}=\frac{\varphi (q^2) \varphi (q^{30})}{\varphi (q) \varphi \left(q^{15}\right)}+4 q^8\frac{ \psi (q^4) \psi(q^{60})}{\varphi (q) \varphi \left(q^{15}\right)},
	\end{equation}
	and  use (\ref{eq08}), we get
	\begin{equation}\label{eq39}	
		(\alpha \beta)^{1 / 4}=4 q^4 \frac{\psi(q^2) \psi(q^{30})}{\varphi(q)\varphi(q^{15})},\quad\{(1-\alpha)(1-\beta)\}^{1 / 4}=\frac{\varphi (-q)\varphi (-q^{15})}{\varphi(q)\varphi(q^{15})}.
	\end{equation}
	Substituting (\ref{eq38}) and (\ref{eq39}) into (\ref{eq20}),	which completes the proof.	
\end{proof}	


\begin{theorem}[new modular equations for degrees $(n_1,n_2)=(3, 5)$] Let $\alpha$ and $\beta$ have degrees $(n_1,n_2)=(3, 5)$, and $m=\frac{z_3}{z_{5}}$ is the multiplier for degree $n=\frac{5}{3}$, then 
	\begin{enumerate}[(i)]
		\item natural form	
		\begin{equation}\label{eq40}
			\begin{aligned}
				m-\frac{5}{3m} =&\frac{2}{3}  \left((\alpha \beta)^{1 / 8}-\{(1-\alpha)(1-\beta)\}^{1 / 8}\right) \left[1-3\left((\alpha \beta)^{1 / 8}+\{(1-\alpha)(1-\beta)\}^{1 / 8}\right)\right.\\
				&\left.+3\left((\alpha \beta)^{1 / 4}+\{(1-\alpha)(1-\beta)\}^{1 / 4}\right)+2 (\alpha \beta (1-\alpha)(1-\beta))^{1 / 8}\left(3-(\alpha \beta)^{1 / 8}-\{(1-\alpha)(1-\beta)\}^{1 / 8}\right) \right],
			\end{aligned}
		\end{equation}		
		\item emphatic form		
		\begin{equation}\label{eq41}	
			\begin{aligned}
				m-\frac{5}{3m} =&\frac{2}{3} \left((\alpha \beta)^{1 / 4}-\{(1-\alpha)(1-\beta)\}^{1 / 4}\right) \left[4\sqrt{\frac{1}{2}(1+\sqrt{\alpha \beta}+\sqrt{(1-\alpha)(1-\beta)})}\right.\\
				&\left.-4+\left((\alpha \beta)^{1 / 4}+\{(1-\alpha)(1-\beta)\}^{1 / 4}\right) \right],
			\end{aligned}
		\end{equation}	
		\item 	Ramanujan’s theta functions form	
		\begin{equation}\label{eq42}	
			\begin{aligned}
				\frac{\varphi^2(q^3)}{\varphi^2(q^{5})}-\frac{5}{3} \frac{\varphi^2(q^{5})}{\varphi^2(q^3)}=&\frac{2}{3} \left(\frac{4 q^2 \psi(q^6) \psi (q^{10})}{\varphi (q^3) \varphi \left(q^5\right)}-\frac{\varphi (-q^3) \varphi (-q^5)}{\varphi (q^3) \varphi \left(q^5\right)}\right) \left[4\left(\frac{\varphi (q^6) \varphi (q^{10}))}{\varphi (q^3) \varphi \left(q^5\right)}+\frac{4 q^4 \psi (q^{12})) \psi (q^{20}))}{\varphi (q^3) \varphi \left(q^5\right)}\right)\right.\\
				&\left.-4+\left(\frac{4 q^2 \psi(q^6) \psi (q^{10})}{\varphi (q^3) \varphi \left(q^5\right)}+\frac{\varphi (-q^3) \varphi (-q^5)}{\varphi (q^3) \varphi \left(q^5\right)}\right) \right].
			\end{aligned}
		\end{equation}	
	\end{enumerate}
	
\end{theorem}
The proof of Theorem 2.2 is similar to the proof of the Theorem 2.1, hence we omit the details.


\section*{Acknowledgment}
The authors would like to thank the anonymous referee for many invaluable suggestions. 

%% The Appendices part is started with the command \appendix;
%% appendix sections are then done as normal sections
%% \appendix

%% \section{}
%% \label{}

%% References
%%
%% Following citation commands can be used in the body text:
%% Usage of \cite is as follows:
%%   \cite{key}         ==>>  [#]
%%   \cite[chap. 2]{key} ==>> [#, chap. 2]
%%

%% References with BibTeX database:

\bibliographystyle{elsarticle-num}
\bibliography{Ramanujan_of_15}

%% Authors are advised to use a BibTeX database file for their reference list.
%% The provided style file elsarticle-num.bst formats references in the required Procedia style

%% For references without a BibTeX database:

% \begin{thebibliography}{00}
	
	%% \bibitem must have the following form:
	%%   \bibitem{key}...
	%%
	
	% \bibitem{}
	
	% \end{thebibliography}

\end{document}

%%
%% End of file `ecrc-template.tex'. 