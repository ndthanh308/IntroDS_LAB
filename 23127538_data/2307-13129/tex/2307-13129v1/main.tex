\documentclass[11pt, reqno]{amsart}
\usepackage{amsmath,amsthm,amssymb,enumerate}
\usepackage[all]{xy}
\usepackage{array}
\usepackage{float}
\usepackage{graphicx}

\usepackage[dvipsnames]{xcolor}
\definecolor{citation}{rgb}{0,.40,.80}
\usepackage[colorlinks,linkcolor=citation,citecolor=citation,urlcolor=citation]{hyperref}

\newcommand{\katrina}[1]{{\color{ForestGreen} \sf $\clubsuit\clubsuit\clubsuit$ Katrina: [#1]}}
\newcommand\khedit[1]{{\color{ForestGreen} \sf #1}}
\newcommand{\Katrina}[1]{\footnote{\color{ForestGreen} \sf  Katrina: [#1]}}

\newcommand \surj {\twoheadrightarrow}


\newcommand \C {\mathbb C}
\newcommand \cC {\mathcal C}
\newcommand \F {\mathbb F}
\newcommand \fF {\mathcal F}
\newcommand \gG {\mathcal G}
\newcommand \hH {\mathcal H}
\newcommand \G {\mathbb G}
\newcommand \I {\mathcal I}
\renewcommand \L {\mathcal L}
\newcommand \M {\mathcal M}
\renewcommand \O {\mathcal O}
\renewcommand \P {\mathbb P}
\newcommand \Q {\mathbb Q}
\newcommand \Ql {\mathbb Q_{\ell}}
\newcommand \R {\mathbb R}
\newcommand \W {\mathbb W}
\newcommand \X {\mathcal X}
\newcommand \Z {\mathbb Z}
\newcommand\et{\textrm{\'et}}
\newcommand\xra{\xrightarrow}

\DeclareMathOperator \alb {alb}
\DeclareMathOperator \Aut {Aut}
\DeclareMathOperator \Br {Br}
\DeclareMathOperator \hyp {hyp}
\DeclareMathOperator \CH {CH}
\DeclareMathOperator \charr {char}
\DeclareMathOperator \el {ell}
\DeclareMathOperator \id {id}
\DeclareMathOperator \Hom {Hom}
\DeclareMathOperator \Hilb {Hilb}
\DeclareMathOperator \Supp {Supp}
\DeclareMathOperator \Kt {K3}
\DeclareMathOperator \OO {O}
\DeclareMathOperator \mult {mult}
\DeclareMathOperator \Pic {Pic}
\DeclareMathOperator \NS {NS}
\newcommand \Picbar {\overline{\mathrm{Pic}}{}} 
\DeclareMathOperator \Spec {Spec}
\DeclareMathOperator \QCoh {QCoh}
\DeclareMathOperator \Bl {Bl}
\DeclareMathOperator \Kum {Kum}
\DeclareMathOperator{\Fix}{Fix}
\DeclareMathOperator{\Gal}{Gal}
\DeclareMathOperator{\Id}{Id}
\DeclareMathOperator{\Sp}{Sp}
\newcommand{\Spa}{\Sp(A[2])}


\DeclareMathOperator{\supp}{supp}
\DeclareMathOperator{\sym}{sym}
\DeclareMathOperator{\Sym}{Sym}
\DeclareMathOperator{\im}{im}
\DeclareMathOperator{\sep}{sep}
\newcommand \kbar {{\bar{k}}}
\DeclareMathOperator{\Char}{char}
\DeclareMathOperator{\Span}{Span}
\DeclareMathOperator{\Frac}{Frac}

\DeclareMathOperator{\shF}{\mathcal{F}}
\DeclareMathOperator{\shG}{\mathcal{G}}
\DeclareMathOperator{\shO}{\mathcal{O}}

\newtheorem {thm} {Theorem}[section]
\newtheorem {cor} [thm] {Corollary}
\newtheorem {remark}[thm] {Remark}
\newtheorem {lem} [thm] {Lemma}
\newtheorem {lemma} [thm] {Lemma}
\newtheorem {prop} [thm] {Proposition}
\newtheorem {claim} [thm] {Claim}

\theoremstyle{definition}
\newtheorem {defn} [thm] {Definition}
\newtheorem {setting} [thm] {Setting}
\newtheorem {cons} [thm] {Construction}
\newtheorem {question} {Question}
\newtheorem {defprop} [thm] {Definition--Proposition}
\newtheorem {example}[thm] {Example}
\newtheorem {examples}[thm] {Examples}
\newtheorem {rmk}[thm] {Remark}

\numberwithin{equation}{section}
\renewcommand \labelenumi {(\alph{enumi})}

\hyphenpenalty = 1000

\makeatletter
\newcommand{\xleftrightarrow}[2][]{\ext@arrow 3359\leftrightarrowfill@{#1}{#2}}
\newcommand{\dashto}[2][]{\ext@arrow 0359\rightarrowfill@@{#1}{#2}}
\newcommand{\xdashleftarrow}[2][]{\ext@arrow 3095\leftarrowfill@@{#1}{#2}}
\newcommand{\xdashleftrightarrow}[2][]{\ext@arrow 3359\leftrightarrowfill@@{#1}{#2}}
\def\rightarrowfill@@{\arrowfill@@\relax\relbar\rightarrow}
\def\leftarrowfill@@{\arrowfill@@\leftarrow\relbar\relax}
\def\leftrightarrowfill@@{\arrowfill@@\leftarrow\relbar\rightarrow}
\def\arrowfill@@#1#2#3#4{%
  $\m@th\thickmuskip0mu\medmuskip\thickmuskip\thinmuskip\thickmuskip
   \relax#4#1
   \xleaders\hbox{$#4#2$}\hfill
   #3$%
}
\makeatother


                  
\begin{document}

\author{Katrina Honigs}
\author{Graham McDonald}
\address{Department of Mathematics\\
Simon Fraser University\\
8888 University Drive\\
Burnaby, BC, V5A 1S6\\
Canada}



\title[Theta characteristics and the fixed locus of {$[-1]$}]{Theta characteristics and the fixed locus of~$[-1]$ on some varieties of Kummer type}


\date{}

\begin{abstract}
  We study some combinatorial aspects of the fixed loci of symplectic involutions acting on hyperk\"ahler varieties of Kummer type.

  Given an abelian surface $A$ with a $(1,d)$-polarization $L$, there is an
  isomorphism $K_{d-1}A\cong K_{\hat{A}}(0,\hat{l},-1)$ between a hyperk\"ahler of Kummer type that parametrizes length $d$ points on $A$ and one that parametrizes degree $d-1$ line bundles supported on curves in $|\hat{L}|$, where $\hat{L}$ is a $(1,d)$-polarization on $\hat{A}$.
We examine the bijection this isomorphism gives between isolated points in the fixed loci of $[-1_A]$ when $d$ is odd, which has a combinatorics related to theta characteristics.

Along the way, we give numerical values for
a formula of \cite{KMO} counting the number of components
of a symplectic involution acting on a Kummer-type variety. 
\end{abstract}


\maketitle

\section{Introduction}\label{intro}

The geometry of involutions of hyperk\"ahler varieties is a very active area of study. In this note, we examine some combinatorial aspects of this geometry for hyperk\"ahler
varieties of Kummer type.
In \cite{KMO},
Kamenova, Mongardi, and Oblomkov show that the
fixed locus of any symplectic involution on a Kummer-type variety
is the union of
finitely many even-dimensional components,
which are
themselves hyperk\"ahler varieties of $\mathrm{K3}^{[n]}$ type, and 
give a formula enumerating these components.

The most commonly encountered Kummer varieties are constructed using abelian surfaces.
Given an abelian surface $A$,
the \textit{generalized Kummer variety} $K_n(A)$ is the $2n$-dimensional variety  given by the fiber over $0_A\in A$ of the map 
$\Sigma:\Hilb^{n+1}(A)\to A$, where summation is with respect to the group law on $A$. 
The variety $K_1A$ is the familiar Kummer K3 surface.

The involution on $K_n(A)$ induced by $[-1_A]:A\to A, a\mapsto -a$ is symplectic. All of $K_1(A)$ is fixed by $[-1_A]$, but the fixed locus of its action on the fourfold $K_2A$ has more structure. It consists of a
Kummer K3 surface as well as 36 isolated points:
one whose support consists of $0_A$ and $35$ supported on combinations of three distinct points of $A[2]$ \cite{HasTsc}, \cite{Tari}.

If we fix a polarization $H$ on $A$, we may
also form Kummer-type varieties using $H$-stable sheaves on $A$.
For instance, 
if $A$ has a (symmetric) $(1,d)$-polarization $L$ 
and we take $l=[L]$, the points of the Kummer $2(d-1)$-fold
$K_A(0,l,s)$ parametrize certain
sheaves of rank $0$, N\'eron-Severi class~$l$ and Euler characteristic $s$, i.e.\
rank~ $1$ torsion free sheaves supported  on irreducible curves in the linear system $|L|$.
Generically, these sheaves are line bundles of degree $s+\frac{l^2}{2}=s+d$.
Pullback by $[-1_A]$ also gives a symplectic involution on $K_A(0,l,s)$. 

These two moduli constructions are closely related.
When $A$ has Picard rank $1$, 
there is an isomorphism
given by a Fourier--Mukai transform
that commutes with $[-1_A]$
\begin{equation}\label{psi}
\Psi:K_{d-1}(A)\xra{\sim}  K_{\hat{A}}(0,\hat{l},-1), 
\end{equation}  
where
 $\hat{l}$ is the N\'eron-Severi class of a $(1,d)$-polarization $\hat{L}$ on the dual abelian variety of $A$ \cite{Gulbrandsen,Yoshioka}. 

 The points in $K_{\hat{A}}(0,\hat{l},-1)$ that are fixed by $[-1_A]$ are supported on curves in the two eigenspaces of the linear system $|\hat{L}|$ under the action of $[-1_A]$. 
In the case $d=3$, it was shown in  \cite{Frei_Honigs} that
this distinction partitions
the $36$ isolated points in
the fixed locus of $[-1_A]$ acting on 
$K_{\hat{A}}(0,\hat{l},-1)$ 
into a set of $16$ and of $20$.
 However,
such a partition is not immediately obvious in the fixed locus of $K_2(A)$.
The isomorphism $\Psi$ gives a bijection between these sets of isolated points, but it is not explicitly understood. 

In this paper,
we show multiple ways to distinguish this partition of points.
First, we show a numerical criteria 
in terms of theta characteristics, which are the quadratic forms on $A[2]$ associated to the Weil pairing. The main insight in these results is the application of the work of Gulbrandsen \cite{Gulbrandsen} on the supporting curves of the images of $\Psi$ to this case.

\begin{thm}[Prop.~\ref{determinant}] For $d$ odd,
  let $\xi=(u_1,\ldots,u_{d})\in K_{d-1}A$  be a point
  fixed under the action of $[-1_A]$ that is supported on $d$ distinct points of $A[2]$. Let $q$ be any theta characteristic.
  The value of $q(\xi):=\sum_{i=1}^{d}q(u_i)\in \F_2$ determines which eigenspace of $|\hat{L}|$ contains the supporting curve of $\Psi(\xi)$.
\end{thm}

In the case $d=3$, $K_2A$ contains $35$ isolated points supported on three distinct points of $A[2]$. For $15$ of these points, $q(u_1)+q(u_2)+q(u_3)=0$ and for $20$ of the points, $q(u_1)+q(u_2)+q(u_3)=1$.
This computation, along with the results of \cite{Frei_Honigs}, allows us to determine which of the $36$ isolated fixed points of $[-1_A]$ in $K_2A$
is mapped by $\Psi$ to curves supported in each of the eigenspaces of $|\hat{L}|$.

For any odd $d$, there is also a natural action of the symplectic group of $A[2]$ on the set of points $\xi\in K_{d-1}A$ that are supported on $d$ distinct points of $A[2]$. This action respects the value of $q(\xi)$. In the case $d=3$, we have the following result.

\begin{thm}[Prop.~\ref{twoorbits}]
The action of $\Sp(A[2])$ on the $35$ points $\xi\in K_2A$ that are supported on three distinct points of $A[2]$ has two orbits, which coincide with the two possible values of $q(\xi)$.
\end{thm}

The characterization of the fixed locus of $[-1_A]$ on $K_{\hat{A}}(0,\hat{l},-1)$ in \cite{Frei_Honigs} gives the supporting curves of isolated fixed points. Using this information, we are then able to prove the following result.

\begin{thm}[\S\ref{sec.iso},\S\ref{fourfolds}]
Let $\xi\in K_2A$ be supported on three distinct points of $A[2]$. It is possible to determine the supporting curve of $\Psi(\xi)$ and recover its singular points using only numerical data from the theta characteristics of $A[2]$.
\end{thm}

Our methods also give us information
about singular points of curves in the linear system $|\hat{L}|$
for higher (odd) values of $d$, which we demonstrate by examining some example points of $K_4A$ in \S\ref{onefive}.

Our analysis of theta characteristics
can be visualized using the following table listing the points of $A[2]$, which we call a \textit{Hudson table}:
\[
\begin{tabular}{c|ccc} 
$1$ & $ab'$ & $bc'$ & $ca'$\\
\hline\vrule height 12pt width 0pt
$ac'$ & $a'$ & $c$ & $bb'$\\ 
 $ba'$ &$cc'$  & $b'$ & $a$\\ 
 $cb'$ & $b$ & $aa'$ &$c'$
\end{tabular}
\]
We name this table for its appearance in the classic text of Hudson \cite{Hudson} on the (singular) Kummer surfaces associated to  principally polarized abelian surfaces. There, the table is given in relation to the $(16,6)$ configuration of points and planes.
These configurations have a very rich combinatorics, which is explored in, for instance, Gonzalez-Dorrego \cite{Gonzalez}.
We observe that for any polarization of odd degree, the Hudson table captures information on the theta characteristics and we find it useful in our arguments. Its application in these contexts appears to be novel.

In this paper we also give numerical values and a generating function for the formula in \cite{KMO}
enumerating components of fixed loci on a variety of Kummer type.
The combinatorics of these fixed loci relate to other invariants of these varieties, such as cohomology (cf.~\cite{HasTsc}); these computations may be useful as a reference to researchers working in this~area.

\subsubsection*{Plan of paper}
In \S\ref{BN} we give further background on hyperk\"ahler varieties of Kummer type.
In \S\ref{formula} we give the values for the formula \cite[Theorem~3.9]{KMO}.
The remaining sections are focused on the
results relating to the isomorphism \eqref{psi}:
In \S\ref{group} 
we discuss theta characteristics and how to visualize them in a Hudson table. 
In \S\ref{sec.iso}, we define the
isomorphism \eqref{psi} and prove results about the supporting curves of sheaves in the image of $\Psi$.
In \S\ref{fourfolds}, we review results on
the isolated points in the fixed locus of $[-1_A]$ acting on each
of $K_2(A)$ and $K_{\hat{A}}(0,\hat{l},-1)$ when $d=3$, and then
characterize the supporting curves of the images of these isolated points under $\Psi$.
In \S\ref{sp} we examine the action of the symplectic group $\Spa$ on part of the fixed locus. Finally in \S\ref{onefive},
we give example applications 
of our methods to a $(1,5)$-polarized abelian surface.


\section*{Acknowledgements}
K.H.\ thanks Nils Bruin for helpful conversations at a formative stage of the project and was supported by an NSERC Discovery Grant.
G.M.\ was supported by an NSERC USRA.

\section{Background and notation}\label{BN}

Many of the varieties we consider are constructed using an abelian surface~$A$, which has dual abelian variety $\hat{A}$. We write the involution
$[-1_A]:A\to A, a\mapsto -a$.
In \S\ref{formula}, we will write the group law on $A[2]$ additively as it is convenient there to consider it as isomorphic to $\F_2^4$. In all other sections, we will use multiplicative notation for the group law on $A[2]$, borrowing from the classic text \cite{Hudson}, 
writing the identity point of $A$ as $1_A$. Given a line bundle $L$ on $A$, we denote its associated projectivized linear system by $|L|\cong \P H^0(A, L)$.
For simplicity, we work over the base field~$\C$, but it is possible to consider these constructions over other fields.

Hyperk\"ahler varieties are irreducible holomorphic symplectic varieties with a unique nondegenerate holomorphic two-form. An involution is referred to as ``symplectic'' if it respects this two-form. We encounter two families of these varieties in this paper. Our primary focus is those of Kummer type, which are characterized by  deformation equivalence to a generalized Kummer variety $K_nA$. We also encounter those of $\mathrm{K3}^{[n]}$ type, which are deformation equivalent to the Hilbert scheme of a K3 surface.

One construction for producing varieties of Kummer type is as Albanese fibers of moduli of stable sheaves on an abelian surface. Let $A$ be an abelian surface with a polarization $H$ and a Mukai vector $v=(r,l,s)$.
The moduli space $M_A(v)$ parametrizes
$H$-stable sheaves on $A$ with Mukai vector $v$, that is, rank $r$, N\'eron-Severi class of the determinental line bundle $l$, and Euler characteristic $s$.
If $v$  is
primitive and positive with $v^2\geq 2$ and $H$ is $v$-generic, then
the moduli space of stable sheaves $M_A(v)$ is a non-empty, smooth, projective and irreducible variety of dimension $v^2+2$, where $v^2:=l^2-2rs$.
We refer the interested reader to the papers \cite{Mukaisymp} and \cite{Yoshioka}, or the summary in \cite{Frei_Honigs} for more details on this construction and hypotheses.

In order to construct $K_A(v)$, it is necessary to use the Fourier--Mukai transform whose kernel is the Poincar\'e bundle $P_A$ of $A$, defined as follows:
\begin{align}
  \Phi_{P_A}:D(A)&\to D(\hat{A})\\
  \fF&\mapsto  Rp_{2*} (Lp_1^*(\fF)\otimes^{L} P_A)
\end{align}
The maps $p_1,p_2$ are the projections from $A\times \hat{A}$ to its factors. The functor $\Phi_{P_A}$ gives an equivalence between the bounded derived categories of coherent sheaves of $A$ and $\hat{A}$ \cite{Mukai}.


The Albanese variety of a moduli space $M_A(v)$ constructed as above is $A\times \hat{A}$. It is briefer to show the map to the Albanese torsor:
\begin{align}\label{alb}
\alb: M_A(v)&\to \Pic^l_A\times \Pic^{\hat{l}}_{\hat{A}}
\\ \notag
\fF&\mapsto (\det(\fF),\det(\Phi_P(\fF)))
\end{align}
where 
$\hat{l}$ is the N\'eron-Severi class of $\det(\Phi_P(\fF))$ for $\fF\in M_A(v)$,
which is  the negative of the Poincar\'e dual of $l$ \cite[Prop.~1.17]{MukaiFF}.
We define $K_A(v)$  to be a fiber over the Albanese map. All fibers are isomorphic, so without loss of generality, we may choose the convention that $K_A(v)$ is a fiber over symmetric line bundles, in which case pullback by $[-1_A]$ is a symplectic involution on $K_A(v)$. 

In the case where $v=(1,0,-(n+1))$,
$K_A(v)$ is isomorphic to $K_{n}A$:
$K_A(1,0,-(n+1))$
parametrizes the ideal sheaves of points of length $n+1$ that sum to $1_A$.

The other moduli spaces of this type we will work with have Mukai vectors of the form $v=(0,l,s)$, where $\NS(A)=\Z l$ is the N\'eron-Severi class of a symmetric $(1,d)$-polarization $L$ on $A$.
For any $s\in \Z$,
the fiber of the Albanese morphism \eqref{alb}
on $M_A(0,l,s)$  over
$\{L\}\times \Pic^{\hat{l}}_{\hat{A}}$
is the relative compactified Jacobian
$\overline{\mathrm{Pic}}^{s+d}_{\mathcal{C}/|L|}$ where $\mathcal{C}$ is the tautological family of curves in $|L|$. 
The elements of $\overline{\mathrm{Pic}}^{s+d}_{\mathcal{C}/|L|}$
consist of rank~$1$ torsion-free sheaves of Euler characteristic $s$
supported on the curves of the linear system $|L|$, which have arithmetic genus $d+1$.
When the supporting curves are 
smooth, these sheaves are line bundles of degree $s+d$.

Fix a symmetric line bundle $M\in \Pic^{\hat{l}}(\hat{A})$. 
We may then take $K_A(0,l,s)\subset\overline{\mathrm{Pic}}^{s+d}_{\mathcal{C}/\mathbb{P}|L|}$ to be the fiber of the Albanese map \eqref{alb} over $(L,M)$.
It was shown in \cite[\S 6]{Frei_Honigs} that for any smooth curve $C\in |L|$,
the points in $K_A(0,l,s)$ supported on $C$ are a translation of the fiber of the following map over $1_A$, which is the morphism given by the universal property of the Jacobian:
\begin{align}
 \Pic^0_C&\to A \\\notag
 \O(p-q)&\mapsto p-q
\end{align}
where summation in the target is occuring with the group law of $A$.
This notion extends in a natural way
for singular curves $C$ in $|L|$
using the theory of generalized divisors (cf.~\cite{Kass}).

\section{A formula for the number of fixed loci}\label{formula}

In \cite[Theorem~3.9]{KMO}, the authors give a formula for the number of components of each dimension in the
fixed locus 
of a symplectic involution on a Kummer-type variety.
To find this formula, they consider the fixed locus of $[-1_A]$ acting on $K_nA$ and relate it to
points on $\Sym^{n+1}A$ that are supported on 
combinations of pairs of points $(p,-p)$ for $p\in A$ and tuples of points in $A[2]$ that sum to $0$.
The possible configurations of these supporting points dictate the
number of components of each dimension.
Thus, their  formula depends on the 
the combinatorics problem of
choosing $n$ elements of $A[2]\cong\F_2^4$ that sum to $0$.

In the following formula, the sum is taken over all subsets $I\subseteq \F_2^4$ that sum to $0$, $|I|$ denotes the number of elements of $I$, and
$\binom{n}{k}=0$ if $k>n$ or $k$ is a negative integer. 

\begin{thm}[{{\cite[Theorem 3.9]{KMO} Kamenova, Mongardi, Oblomkov}}]
  Let $X$ be a $2n$-dimensional hyperkahler manifold of Kummer type, and let $\iota$ be a symplectic involution on $X$ that acts nontrivially on $H^3(X)$.
  Define
\begin{equation}\label{original}
  N_m^n:=\sum_{I,\sum I=0}\binom{|I|}{\frac{n-|I|}{2}-m}
\end{equation}  
The fixed locus of $\iota$ consists of $N^{n+1}_m$ connected components of dimension~$2m$. 
\end{thm}

The reference has a typographical error
switching
the top and bottom of the binomial notation in \eqref{original}.

To give a formula for $N_n^m$ that does not involve 
the selection of sets $I\subseteq \F_2^4$, we put it in terms of the number of ways to choose $0\leq i\leq 16$ elements of $\F_2^4$ that sum to $0$, which we call $\gamma_i$. Its values are listed in Table~\ref{gammai}. 
\begin{equation}
	N_m^n=\sum_{i=0}^{16}\gamma_i\cdot\binom{i}{\frac{n-i}{2}-m}
	\label{eq: final_count_formula}
      \end{equation}

The values of $\gamma_i$ are given by the following generating function in the case $r=4$.

\begin{thm}[\cite{OEIS} Song]
  The following is a generating function for number of subsets $I$ of the vector space $\mathbb{F}_2^r$
 so that $|I|=i$ and $\sum_{j\in I}j=0$: 
	\begin{equation}
		(x+1)^{2^{r-1}}\left(\sum\limits_{i=0}^{2^{r-1}}\binom{2^{r-1}}{2i}x^{2i}-(2^{r-1}-1)x\prod\limits_{k=0}^{r-2}\sum\limits_{i=0}^{2^k}\binom{2^k}{2i}x^{2i}\right)
		\label{eq: Jian_song}
	\end{equation}
\end{thm}

In the case $r=4$ we have:
\begin{align}\label{eq: Jian_song_n_4}
&(x+1)^8\left(\sum\limits_{i=0}^8\binom{8}{2i}x^{2i}-7x\prod\limits_{k=0}^2\sum\limits_{i=0}^{2^k}\binom{2^k}{2i}x^{2i}\right)
\\\notag
&=x^{16}+x^{15}+35x^{13}+140x^{12}+273x^{11}+448x^{10}+715x^9\\
&\phantom{{}=x^{16}}+870x^8+715x^7+448x^6+273x^5+140x^4+35x^3+x+1
  \label{eq: poly_expanded}
\end{align}


\begin{table}[H]
  \centering
\caption{Values of $\gamma_i$ }
	\begin{tabular}{|c|c|c|c|c|c|c|c|c|c|c|}
		\hline
		$i$&0&1&2&3&4&5&6&7\\
		\hline 
		$\gamma_i$&1&1&0&35&140&273&448&715\\
          \hline
                \hline
          8&9&10&11&12&13&14&15&16\\
          \hline
          870&715&448&273&140&35&0&1&1\\
          \hline
	\end{tabular}
	\label{gammai}
      \end{table}

\begin{rmk}
There is a symmetry $\gamma_i=\gamma_{16-i}$ because the sum of all elements in $\F_2^4$ is $0$, so any choice of $i$ elements summing to $0$ partitions $\F_2^4$ into two sets that sum to $0$.
\end{rmk}

Now we use the formula \eqref{eq: final_count_formula} and the values in Table~\ref{gammai} to give
values of $N_m^{n+1}$, i.e., 
numbers of components of each dimension of the fixed locus
of a symplectic involution acting on a $2n$-dimensional variety of Kummer type. We show these values up to $n=10$. 

We give the values in two tables, one for each parity of $n$ since, if $n$ is even, only the $\gamma_i$ with $i$ odd appear in the formula $N_m^{n+1}$, and likewise with $i$ even for $n$ odd.
The components of the fixed locus of a symplectic involution 
on a $2n$-dimesional variety of Kummer type
have dimensions $2m$
for $\max\{0, \frac{n+1}{2} - 24\} \leq m \leq \frac{n+1}{2}$ if $n$ is odd, and $\max\{0, \frac{n}{2} - 24\} \leq m \leq \frac{n}{2}$ if $n$ is even since all the binomial coefficients in the formula \eqref{eq: final_count_formula}
will be $0$ outside these bounds. We leave the entries of the tables blank for values of $m$ not within these bounds.

\begin{table}[H]
	\centering
	\caption{Values of $N_m^{n+1}$, for $n$ odd}
	\begin{tabular}{|c|c|ccccccc|}
	\hline 
	&$n$ &&&&&&&\\
		\hline 
$m$:&&$0$&$1$&$2$&$3$&$4$&$5$&$\cdots$\\
		\hline 
        &$1$ 
             &$0$&$1$&&&&&\\
        & $3$ 
             &$140$&$0$&$1$&&&&\\
        &$5$
             &$1008$&$140$&$0$&$1$&&&\\
        &$7$
             &$4398$&$1008$&$140$&$0$&$1$&&\\
        &$9$
             &$14688$&$4398$&$1008$&$140$&$0$&$1$&\\
		&\vdots&\vdots&\vdots&\vdots&\vdots&\vdots&\vdots&$\ddots$\\
		\hline
	\end{tabular}
	\label{tab: even_n}
\end{table}

\begin{table}[H]
	\centering
\caption{Values of $N_m^{n+1}$, for $n$ even}
	\begin{tabular}{|c|c|ccccccc|}
		\hline
		&$n$&&&&&&&\\
		\hline
$m$:&&$0$&$1$&$2$&$3$&$4$&$5$&$\cdots$\\
		\hline 
		&$2$ 
                    &$36$&$1$&&&&&\\
		&$4$
                    &$378$&$36$&$1$&&&&\\
		&$6$
                    &$2185$&$378$&$36$&$1$&&&\\
		&$8$
                    &$8485$&$2185$&$378$&$36$&$1$&&\\
		&$10$
                    &$24453$&$8485$&$2185$&$378$&$36$&$1$&\\
		&\vdots&\vdots&\vdots&\vdots&\vdots&\vdots&\vdots&$\ddots$\\
		\hline
	\end{tabular}
	\label{tab: odd_n}
      \end{table}

For example, $N_0^{n+1}$ is the number of isolated fixed points on a $2n$-fold; $n=48$ is the largest value for which $N_0^{n+1}$ is nonzero. The formula
\eqref{eq: final_count_formula} in this case gives the following:
\[
  N_0^{n+1}=\gamma_{n+1}\binom{n+1}{0}+\gamma_{n-1}\binom{n-1}{1}+
\gamma_{n-3}\binom{n-3}{2}+\cdots,
\]
where $\gamma_{n+1}$ is the number of ways to choose $n+1$ distinct points in $\F_2^4$ that sum to $0$.
In the case of $[-1_A]$ acting on $K_nA$, $\gamma_{n+1}$ is counting the number of fixed points supported on $n+1$ distinct points of $A[2]$.
For any $n$, the component of the fixed locus with the largest dimension is deformation equivalent to one copy of $S^{[\lceil\frac{n}{2}\rceil]}$.

Looking at the above tables, in the $n=1$ case we see that 
in $K_1A$, the entire K3 surface itself is fixed by any symplectic involution. In the $n=2$ case, the the fixed locus of
$K_2A$ has exactly $1$ fixed K3 surface as well as 
$36$ isolated fixed points;
if we consider the symplectic involution to be $[-1_A]$, 
$35$ of those fixed points are supported on three distinct points in $A[2]$.


\section{Theta characteristics on
  abelian surfaces with odd degree polarizations}
\label{group}

In this section, we introduce theta characteristics and Hudson tables.

We begin by giving a brief review of the classical $(16,6)$ configuration of planes and points on the singular Kummer K3 surface associated to a principally polarized abelian surface. In particular, we examine an incidence table of the points and planes. We then show in \S\ref{label} that this table captures more general information about theta characteritics for any abelian surface with an odd degree polarization, and use this combinatorial tool to show some results.

In \S\ref{base} we review the role of theta characteristics in relation to the linear system of the polarization on $A$.

\subsection{The principally polarized case}

In \cite[Ch.~1 \S3]{Hudson},
Hudson describes a $(16,6)$ configuration in the singular Kummer K3 surface $A/[-1]$
associated to a principally polarized abelian surface $A$:
there are $16$ singular points, coming from $A[2]$,
which are in bijection with $16$ planes.
Each plane contains exactly six of the singular points and each point lies on six planes. One such plane is shown in Figure~\ref{fig: singular kummer}.

% Figure environment removed

Hudson names a (non-minimal) set of generators for the group $A[2]$: $1,a,b,c,a',b',c'$. They obey the following multiplication tables:
\[
    \begin{tabular}{>{$}l<{$}|*{6}{>{$}l<{$}}}
    ~ &  a  & b & c  \\
    \hline\vrule height 12pt width 0pt
    a &  1 & c   & b \\
    b &     & 1   & a \\
    c &    &   & 1 \\
    \end{tabular}
\quad\quad\quad
    \begin{tabular}{>{$}l<{$}|*{6}{>{$}l<{$}}}
    ~ &  a'  & b' & c'  \\
    \hline\vrule height 12pt width 0pt
    a' &  1 & c'   & b' \\
    b' &     & 1   & a' \\
    c' &    &   & 1 \\
    \end{tabular}
\]
The table in Figure~\ref{mult}
contains all of the points of $A[2]$
and is also a multiplication table.
We call  it the \emph{Hudson table} of the $(16,6)$ configuration.
The Hudson table records the information of the  $(16,6)$ configuration: 
If we fix a plane and consider the corresponding point in the Hudson table, the six points lying on that plane are precisely those lying in either the same row or the same column. Similarly, if we fix a point of $A[2]$ and consider its entry in the Hudson table, the values lying
in either the same row or the same column correspond to the six planes on which that point lies.

% Figure environment removed

For example, the plane corresponding to the point $1$ contains the points $ac',ba',cb',ab',bc',ca'$, and the point $1$ lies on the planes corresponding to $ac',ba',cb',ab',bc',ca'$.


\subsection{$(1,d)$-polarized surfaces with $d$ odd}\label{label}
Let $L$ be a (symmetric) $(1,d)$-polarization of $A$  
with $d$ odd.
We can generalize the incidence structure of points and planes described in the previous section using theta characteristics on $A[2]$. See Gross and Harris \cite[\S1]{GrossHarris} and Farkas \cite[\S1,2]{Farkas} for modern introductions to the subject. 

Since $d$ is odd, the polarization
$\phi_L:A\to \hat{A}$ restricts to 
an isomorphism $A[2]\cong \hat{A}[2]$ and hence the Weil pairing gives a non-degenerate strictly alternating form $\langle\, ,\,\rangle$ on $A[2]$ as an $\F_2$-vector space.
The quadratic forms associated to the this pairing on $A[2]$ are the
\textit{theta characteristics} of our polarized abelian surface (cf.~\cite[\S1]{BolognesiMassarenti}), which form
a principal homogeneous space over $A[2]$.

From the general theory of such pairings, it is possible to decompose
$A[2]$ into a sum of two maximal isotropic vector spaces $A[2]\cong X\oplus Y$.
Given such a decomposition, any elements both in $X$ or both in $Y$ pair to $0$, and
the function $q:A[2]\to \F_2$ defined by
$q(x\cdot y)=\langle x,y\rangle$ for  $x\in X, y\in Y$ is an even quadratic form. It is possible to choose dual symplectic bases $\{x_i\}_i,\{y_i\}_i$ for $X$ and~$Y$, so that $\langle x_i,y_j\rangle=\delta_{ij}$.

If we label our
dual symplectic bases for $X$ and $Y$ as $\{a,b\}$ and $\{b',a'\}$ and call the associated quadratic form $q_1$,
we find that the theta characteristics have a pleasant visual description in terms of the Hudson table.
The quadratic form $q_1$ evaluates to $0$ and $1$, respectively, on the following sets of points:
\begin{align}\label{even}
  10_1&:=\{1,a,b,c,a',b',c',aa',bb',cc'\} \\6_1&:=\{ab',ac',ba',bc',ca',cb'\}
  \label{odd}
\end{align}
We have chosen notation for the sets based on the number of elements in them and the theta characteristics they are determined by.
The entries in $6_1$ are precisely the elements that are in the same row or column as $1$ in the Hudson table and the entries in $10_1$ are the complement of $6_1$ in $A[2]$.
Labelling one of the theta characteristics as $q_1$ specifies an isomorphism between $A[2]$ and the theta characteristics.
For each $v\in A[2]$, there is a theta characteristic defined as follows:
\begin{align}\label{qv}
  q_v:A[2]&\to \F_2\\\notag
  u&\mapsto q_1(u)+\langle v,u\rangle
\end{align}
If $v$ is among the points in $10_1$, 
$q_v$ is an even theta function, having Arf invariant $0$, and if $v$ is among the points of $6_1$, $q_v$ is odd, having Arf invariant~$1$.
The six points of $A[2]$ that lie in the same row or column of
the Hudson table as $v$ will evaluate to $1$ if $q_v$ is even and $0$ if $q_v$ is odd; we name this set of points $6_v$.
Then $q_v$ will assign the other value to the remaining ten points of $A[2]$, including $v$ itself, which we call $10_v$. 

In the following table, the points in $6_{b'}$ are highlighted and the non-highlighted points are $10_{b'}$. 
\begin{equation}\label{tab: hudson's incidence diagram highlighted}
	\begin{tabular}{c|ccc}
		$1$&$ab'$&\colorbox{pink}{$bc'$}&$ca'$\\
		\hline
		$ac'$&$a'$&\colorbox{pink}{$c$}&$bb'$\\
		\colorbox{pink}{$ba'$}&\colorbox{pink}{$cc'$}&$b'$&\colorbox{pink}{$a$}\\
		$cb'$&$b$&\colorbox{pink}{$aa'$}&$c'$
	\end{tabular}
\end{equation}

The following lemmas will be useful in later sections.

\begin{lemma}\label{sixten}
For any $v\in A[2]$, $v6_v=6_1$ and $v10_v=10_1$.
\end{lemma}

\begin{proof}
  The proof is an immediate consequence of the following claim: for any $u_1,u_2\in A[2]$, if $q_v(u_1)=q_v(u_2)$, then $q_1(vu_1)=q_1(vu_1)$. The proof is a consequence of this claim because the sets $10_v$ and $6_v$ are
  characterized by the fact that $q_v$ is constant on each of them.

For $i=1,2$ we have the following equations:
\[q_v(u_i)=q_1(u_i)+\langle v,u_i\rangle,\quad
q_1(vu_i)+q_1(v)+q_1(u_i)=\langle v,u_i\rangle
\]
Combining these, we have for each $i$ that
$$q_v(u_i)=q_1(u_i)+q_1(vu_i)+q_1(v)+q_1(u_i)=q_1(vu_i)+q_1(v),
$$
proving the claim.
\end{proof}

\begin{lemma}\label{prod}
For any distinct $u,v\in A[2]$, $u$ and $v$ share the same row or column in the Hudson table if and only if $q_1(uv)=1$ (equivalently $uv\in 6_1$). 
\end{lemma}

\begin{proof}
  The points $u$ and $v$ are in the same row or column of the Hudson table if and only if $u\in 6_v$. By Lemma~\ref{sixten}, $u\in 6_v$ if and only if
  $uv\in 6_1$. 
\end{proof}  

\begin{lemma}\label{qform}
  Let $u_1,\ldots,u_n\in A[2]$ so that $u_1\cdots u_n=1$.
  Then $q_v(u_1)+\cdots+q_v(u_n)$ has the same value for any choice of $v\in A[2]$.
\end{lemma}  

\begin{proof}
Applying \eqref{qv}, we have
$$q_1(u_1)+\cdots+q_1(u_n)=q_v(u_1)+\cdots+q_v(u_n)+
\langle v,u_1 \rangle+\cdots+\langle v,u_n \rangle
$$
Using bilinearity, we have 
$\langle v,u_1 \rangle+\cdots+\langle v,u_n \rangle
=\langle v,u_1\cdots u_n \rangle=\langle v,1 \rangle=0$, and thus
$q_1(u_1)+\cdots+q_1(u_n)=q_v(u_1)+\cdots+q_v(u_n)$.
\end{proof}  


\subsection{$2$-torsion base points of a polarization}\label{base}
Let $A$ be an abelian surface with an odd degree symmetric polarization $L$.
The $16$ possible choices of the line bundle $L\otimes P_x$ for $x\in \hat{A}[2]$ correspond to the theta characteristics described in the previous section. 

Following \cite[\S2]{BolognesiMassarenti}, we describe
the eigenspaces of $[-1_a]$ acting on $H^0(A,L\otimes P_x)$ and
the points of $A[2]$ in the base loci of these linear systems. The eigenvalues of the action are $1,-1$.

Fix a line bundle $L\otimes P_x$. It corresponds to a theta characteristic $q_v$.
If $q_v$ is even, then the $+1$-eigenspace
$H^0(A,L\otimes P_x)^+$ is
$\frac{d+1}{2}$-dimensional with
 base locus 
$6_v\subset A[2]$, and
the $-1$-eigenspace $H^0(A,L\otimes P_x)^{-}$ is
$\frac{d-1}{2}$-dimensional with
base locus
$10_v\subset A[2]$. If $q_v$ is odd,
$H^0(A,L\otimes P_x)^-$ is
$\frac{d+1}{2}$-dimensional with base locus $6_v\subset A[2]$ 
and 
$H^0(A,L\otimes P_x)^{+}$ is
$\frac{d-1}{2}$-dimensional with base locus $10_v\subset A[2]$.


\section{Isomorphism of moduli spaces}\label{sec.iso}

In this section, we study the isomorphism 
$\Psi:  K_{d-1}A \xra{\sim} K_A(1,0,-d)$.
In \S\ref{iso} and \S\ref{curve} we briefly define $\Psi$ and then examine the supporting curves of images of elements $\xi\in K_{d-1}A$ under it.
In \S\ref{notiso}, we give criteria that allows us to find points of $\hat{A}[2]$ in these supporting curves.

\subsection{The isomorphism}\label{iso}
Let $A$ be an abelian surface that is $(1,d)$-polarized by a symmetric line bundle $L$ so that $\NS(A)=\Z l$, where $l=[L]$. Let $\hat{l}\in \NS(\hat{A})$ as in \S\ref{BN}.

By work of Gulbrandsen
\cite{Gulbrandsen} and Yoshioka \cite{Yoshioka}, there is an isomorphism
$\Psi:K_{d-1}A\xra{\sim} K_{\hat{A}}(0,\hat{l},-1)$, which is given by a composition as follows:
\begin{align}
  \Psi:  K_{d-1}A &\xra{\sim} K_A(1,0,-d) \xra{\sim} K_A(1,l,0)  \xra{\sim}
                K_{\hat{A}}(0,\hat{l},-1)
\\\notag                                                          
  \xi&\rlap{${}\mapsto \mathcal{I}_{\xi}$}
       \phantom{\xra{\sim} K_A(1,0,-d)}\!
       \llap{$I$}\rlap{${}\mapsto I\otimes L$}
       \phantom{\xra{\sim} K_A(1,l,0)}\,\,\,\,
\llap{$\fF$}{}\mapsto \Phi_P(\fF)[-1] 
\end{align}
The leftmost isomorphism sends a length $d$ point to its ideal sheaf.
Tensoring with the line bundle $L$ gives the middle isomorphism.
The morphism on the right is given by the action of the Fourier--Mukai transform $\Phi_{P_A}$ on the sheaves parametrized by $K_A(1,l,0)$. The images
under $\Phi_{P_A}$
are, \textit{a priori}, complexes of sheaves, but they are 
supported only in index $1$, so shifting the indexing by $-1$ gives a well-defined map. This composition was analyzed in detail in \cite{Gulbrandsen} and it was shown by Yoshioka \cite[Prop.~3.5]{Yoshioka} that it is not just a birational equivalence, but an isomorphism.

Since $L$ is symmetric, $\Psi$ commutes with $[-1_A]$, and so $\Psi$ restricts to an isomorphism on the fixed loci of the action of $[-1_A]$ on both 
$K_{d-1}A$ and $K_{\hat{A}}(0,\hat{l},-1)$. We study the bijection $\Psi$ gives on the isolated points in these fixed loci, which we call  $S$ and $R$, respectively.
\begin{equation}\label{ST}
\xymatrix@R=1em@C=1em{  
  \llap{${\Psi:{}}$}S \ar[r] \ar@{}[d]|{\textstyle\cap}  &R \ar@{}[d]|{\textstyle\cap}\\
K_{d-1}A &K_{\hat{A}}(0,\hat{l},-1)}
\end{equation}
$S$ and $R$ each have cardinality equal to $N_0^d$.
We will often restrict our focus to points in $S$ that are supported on $d$ distinct points. We call this subset $S'$.
In the notation of \S\ref{formula}, the cardinality of $S'$ is equal to
$\gamma_d$.

\subsection{The support of $\Psi(\xi)$}\label{curve}

The points of $K_{\hat{A}}(0,\hat{l},-1)$  parametrize certain rank $1$ torsion-free sheaves (see \S\ref{BN}) supported on curves in the linear system $|\smash{\hat{L}}|$, where $\smash{\hat{L}}\in \smash{\Pic^{\hat{l}}(\hat{A})}$ is a $(1,d)$-polarization on $\smash{\hat{A}}$.
In this section, we follow the approach of \cite[3.1]{Gulbrandsen} (cf.~\cite[11.3]{Polishchukbook})
to describe the points in the supporting curves of the images of $\Psi$.

For any 
$\xi\in K_{d-1}A$
and $x\in \hat{A}$, the fiber of $\Psi(\xi)$ over $x$ is the following:
\[
\Psi(\xi)\otimes k(x)\cong H^1(A,\I_\xi\otimes L\otimes P_x),
\]
where $P_x$ is the line bundle on $A$ corresponding to the point $x$.
The points in the supporting curve of $\Psi(\xi)$ are therefore the following set: 
\begin{equation}
D_{\Psi(\xi)}:=\{x\in\hat{A}\mid H^1(A,\I_{\xi}\otimes L\otimes P_x)\neq 0\}.
\end{equation}
To better understand
the vanishing of 
$H^1(A,\I_\xi\otimes L\otimes P_x)$, we consider the short exact sequence
\[
  0\to \I_{\xi}\to \O_A\to
\O_{\xi}\to 0,
\]
where $\O_{\xi}$ is the structure sheaf associated to $\xi$. When $\xi$ is supported on $d$ distinct points, $\O_{\xi}$ is the direct sum of $d$ skyscraper sheaves.

After tensoring with $L\otimes P_x$ we have
\[
  0\to \I_{\xi}\otimes L\otimes P_x\to L\otimes P_x\to
\O_\xi\otimes L\otimes P_x \to 0.
\]
By Mumford's index theorem, 
$H^1(A,L\otimes P_x)=0$. Therefore,
applying the global sections functor gives the following
long exact sequence:
\begin{align}\notag
  0\to H^0(A,\I_{\xi}\otimes L\otimes P_x)
  \to H^0(A,L\otimes P_x)
  \xra{\hypertarget{star}{(\star)}}
  H^0(A,\O_{\xi}\otimes L\otimes P_x)
\\\label{les}
  {}\to
  H^1(A,\I_{\xi}\otimes L\otimes P_x)\to 0.
\end{align}
This long exact sequence gives us several ways to describe which points $x$ are in $D_{\Psi(\xi)}$.
The cohomology groups $H^1(\I_{\xi}\otimes L\otimes P_x)$ and $H^0(\I_{\xi}\otimes L\otimes P_x)$ have the same dimension, and
$H^0(\I_{\xi}\otimes L\otimes P_x)$ is nonzero exactly when $\xi$ lies on a curve in the linear system $|L\otimes P_x|$. Equivalently, $x\in D_{\Psi(\xi)}$ if and only if \hyperlink{star}{$(\star)$} is not an isomorphism.

\subsection{Criteria for $(\star)$ to not be an isomorphism}\label{notiso}

In this section
we assume that
$d$ is odd, $x\in \hat{A}[2]$ and $\xi\in S'\subset K_{d-1}A$.
Under these assumptions, 
we will give criteria under which \hyperlink{star}{$(\star)$} is not an isomorphism by 
by comparing the action of $[-1_A]^*$ on its domain and codomain.

To state our results, we  need the following definition. Let $q_v$ be a theta characteristic of our polarized abelian surface. 
By  Lemma~\ref{qform}, the following sum in $\F_2$ is independent 
of the choice of $v$:
\begin{equation}
q(\xi):=\textstyle\sum_{u\in \xi} q_v(u).
\end{equation}

For each of the following odd values of $d$, the number of
points $\xi\in S'$ having each value of $q(\xi)$ is listed below:
\begin{equation}\label{zero_one}
\begin{tabular}{c|c|c|c}
  $d$&$\gamma_d$&$\#q(\xi)=0$&
                 $\#q(\xi)=1$\\\hline
	$3$&35&15&20\\
	$5$&273&141&132\\
  $7$&715&355&360\\
  $9$&715&355&360\\
  $11$&273&141&132\\
  $13$&35&15&20\\
\end{tabular}  
\end{equation}

In light of \S\ref{base}, the following result allows us to determine which elements $s\in S$ will have supporting curves $D_{\Psi(s)}$ in each eigenspace of $|\hat{L}|$.

\begin{prop}\label{determinant}
Let $\xi\in S'\subset K_{d-1}A$. 

Let $d\equiv 1\mod 4$.
  If $q(\xi)=0$ then $6_1\subset \Supp(\Psi(\xi))$ and
  if $q(\xi)=1$ then $10_1\subset \Supp(\Psi(\xi))$.

  Let   $d\equiv 3\mod 4$.
  If $q(\xi)=0$ then $10_1\subset \Supp(\Psi(\xi))$ and
  if $q(\xi)=1$ then $6_1\subset \Supp(\Psi(\xi))$.  
\end{prop}

\begin{proof}
We first treat the case where $d\equiv 3\mod 4$. Then $\frac{d+1}{2}$ and $\frac{d-1}{2}$ are even and odd, respectively.

From the discussion in \S\ref{base}, we have that if 
$L\otimes P_x$ is even, the determinant of the action of $[-1_A]^*$ on  $H^0(L\otimes P_x)$ is $-1$. If $L\otimes P_x$ is odd, the determinant of the action of $[-1_A]^*$ on  $H^0(L\otimes P_x)$ is $1$.

The line bundle $L\otimes P_x$ corresponds to a quadratic form $q_v$. For any $z\in A[2]$, the action of $[-1_A]^*$ on the skyscraper $k(z)\otimes L\otimes P_x$ is $(-1)^{q_v(z)}$. The multiplicities of the eigenvalues $1$ and $-1$ of the action of $[-1_A]^*$ on $H^0(A,\O_\xi\otimes L\otimes P_x)$ are
$|\{i\mid q_v(z_i)=0\}|$ and $|\{i\mid q_v(z_i)=1\}|$, respectively.
Thus the determinant of the action of $[-1_A]^*$ on $H^0(A,\O_\xi\otimes L\otimes P_x)$ is given by $(-1)^{q(\xi)}$, which happens to be independent of the choice of $x\in \hat{A}[2]$.

If the determinants of $[-1_A]^*$ acting on $H^0(A,L\otimes P_x)$ and on $H^0(\O_\xi\otimes L\otimes P_x)$ differ, then \hyperlink{star}{$(\star)$} is not an isomorphism. These determinants must differ if
 $q(\xi)=0$ and $L\otimes P_x$ is even or if
 $q(\xi)=1$ and $L\otimes P_x$ is odd, hence our result.

 When  $d\equiv 1\mod 4$, $\frac{d+1}{2}$ and $\frac{d-1}{2}$ are odd and even. The same line of reasoning as above then gives the result.
\end{proof}

\begin{rmk}
  In the case where $d=3$, if $\xi=(u,v,w)\in S$ is supported on three distinct points, then
  $q(\xi)$ is $0$ or $1$ if and only if $q_u,q_v,q_w$ are a syzygetic or azygetic triple, respectively (see \cite[Def.~1.4]{Farkas}).
\end{rmk}  

\begin{rmk}
  The proof of Proposition~\ref{determinant} uses the assumption that $\xi$ is supported on distinct points in computing the action
  of $[-1_A]$ on $H^0(\O_{\xi}\otimes L\otimes P_x)$.
When $\xi$ has fewer supporting points, it would not be correct to compute this action by simply weighting the supporting points according to length.
  For instance, the locus of $K_2A$ fixed by $[-1_A]$ contains a Kummer K3 surface, situated in a way where it contains 
  many points supported on $1$ as well as another choice of point in $A[2]$ with length~$2$ \cite{HasTsc}. However, its image under $\Psi$ is fibered over the eigenspace of $|L|$ that is a pencil of curves with a base locus of $6$ points
  (cf.~\cite{Frei_Honigs}). 
\end{rmk}  


\begin{rmk}\label{singular}
  If the kernel of \hyperlink{star}{$(\star)$} is $1$-dimensional, then $x$ is a smooth point on the curve $\Psi(\xi)$, but if the kernel has a higher dimension $x$ cannot be a smooth point.
Under the criteria in Proposition~\ref{determinant}, the dimension of the kernel of \hyperlink{star}{$(\star)$} is odd,
 and so the supporting points may be smooth. 
However, in cases where \hyperlink{star}{$(\star)$} is not an isomorphism but the domain and codomain have equal determinants, the supporting point $x$ must be singular on $\Psi(\xi)$.
\end{rmk}
  
\section{The fourfolds case}\label{fourfolds}

In this section, we fix the following data: Let $A$ be an abelian surface that is $(1,3)$-polarized by a symmetric line bundle $L$ so that $\NS(A)=\Z l$, where $l=[L]$. We assume that $L$ is associated to an even theta characteristic $q_1$.

We examine the bijection \eqref{ST} in this case.  We first review the elements of the sets $S$ and $R$ and then apply the results of Section~\ref{sec.iso}, which allow us to deduce which curve in $|L|$ supports $\Psi(\xi)$ for each $\xi\in S$.


\subsection{Points of $S$}\label{isok2} 

As summarized in \S\ref{intro}, the points of $S$
consist of one point supported only on $1_A$ and the $35$ points formed from all possible choices of three distinct points in $A[2]$ that sum to $1_A$. Using the notation for $A[2]$ established in \S\ref{group},
we list the $35$ points in the first and third columns of Table~\ref{correspond}, grouped by the two values of $q(\xi)$.

\subsection{Points of $R$}\label{isokl}

An analysis of the points of $K_A(0,l,s)$ fixed by $[-1_A]$ is given in \cite{Frei_Honigs}. We summarize the results on isolated fixed points here.

The points of $K_A(0,l,s)$
that are fixed by $[-1_A]$ correspond to sheaves are supported on curves in the eigenspaces of $|L|$ under the action of $[-1_A]$.
As described in \S\ref{base}, the eigenspaces of $H^0(A,L)$ are $1$ and $2$-dimensional, and these linear eigensystems consist of a single hyperelliptic genus $4$ curve $|L|^-=C$ and a pencil $|L|^+$ of non-hyperelliptic curves.
We have fixed our choice of $L$ so that the intersection of $C$ with $A[2]$ is $10_1$
and every curve in $|L|^+$ passes through all points in $6_1$.

For any choice of $s$, 
the supporting curves of the elements of $K_A(0,l,s)$ fixed by $[-1_A]$ will be the same.
Since the Abel--Jacobi map is surjective and generically one-to-one in degree $4$, it is convenient to find the fixed locus of $K_A(0,l,1)$.

There are $16$ isolated fixed points supported
on $C$.
$15$ of the isolated points are given by the divisors that consist of all possible choices of 
 four distinct points  of $10_1$ that sum to the identity in the group law of $A$, which we list in the second column of Table~\ref{correspond}. The last isolated point may be 
(nonuniquely) represented with the divisor~$4\cdot 1_A$. 

The remaining isolated points are supported on singular curves in $|L|^+$.
For each point $x\in 10_1$, there is a curve we will call $B_x$ in $|L|^+$ that has a nodal singularity at $x$ \cite{Naruki}.
For each point in $10_1$, there are two ways to choose three additional points in
$6_1$ so that they sum to the identity in the group law of $A$. For instance, if we choose $1\in 10_1$, we have
\[
1+ab'+bc'+ca' \quad\text{and}\quad 1+ac'+ba'+cb'
\]
If we consider these sums to be generalized divisors, they correspond to
two sheaves supported on $B_1$ that are fixed by $[-1_A]$.
These two sheaves on each of the ten singular curves $B_x$ gives the remaining $20$ isolated points; the generalized divisors supported on some $B_x$ are listed in the fourth column of Table~\ref{correspond}.

Having fixed our choice of $L$ allows us to identify the points of $A[2]$ on curves in $|L|$ with the points of $\hat{A}[2]$ on curves in $|\hat{L}|$, which we will also call $C$ and $B_x$ for simplicity.
We may obtain divisors representing line bundles of degree~$2$, etc.,
by subtracting $2\cdot 1_A$ from those given here, but to simplify notation we leave them as they are in 
Table~\ref{correspond}.
Thus, we identify these sheaves with the isolated fixed points $R$ in $K_{\hat{A}}(0,\hat{l},-1)$.

\begin{table}
  \centering
\caption{}
\label{correspond}
\begin{tabular}{|p{2.6cm}|p{2.5cm}||p{2.6cm}|p{2.8cm}|}
\hline
$\xi\in S$, $q(\xi)=0$, $D_{\Psi(\xi)}=C$ &
$r\in R$, $D_r=C$&
$\xi\in S$, $q(\xi)=1$, $D_{\Psi(\xi)}=B_x$&
$r\in R$, $D_r=B_x$
  \\\hline
               &  &$(ac',ba',cb')$&$1+ac'+ba'+cb'$\\ 
  &&$(ba',cc',ab')$ &$1+ba'+cc'+ab'$\\ 
$(a,a',aa')$&$1+a+a'+aa'$&&\\ 
$(b,b',bb')$&$1+b+b'+bb'$&$(a,b',ab')$ &$cc'+ca'+ba'+bc'$\\ 
$(c,c',cc')$&$1+c+c'+cc'$&$(a',b,ba')$ &$cc'+cb'+ab'+ac'$\\ 
$(a,b,c)$&$1+a+b+c$&$(a,c',ac')$ &$bb'+ba'+ca'+cb'$\\ 
$(a',b',c')$&$1+a'+b'+c'$&$(a',c,ca')$&$bb'+bc'+ab'+ac'$\\ 
$(aa',bb',cc')$&$1+aa'+bb'+cc'$&$(b,c',bc')$&$aa'+ab'+ca'+cb'$\\ 
                             &&$(b',c,cb')$ &$aa'+ac'+bc'+ba'$\\
$(c',ca',cb')$&$a'+b'+c+cc'$&$(a,bb',cb')$&$c'+ca'+ba'+ac'$\\              
$(a',ab',ac')$&$a+b'+c'+aa'$&$(b,aa',ca')$&$c'+cb'+bc'+ab'$\\              
$(b,cb',ab')$&$a+b'+c+bb'$&$(a',bb',bc')$&$c+ca'+ab'+ac'$\\  
$(b',bc',ba')$&$a'+b+c'+bb'$&$(b',aa',ac')$&$c+cb'+ba'+bc'$\\  
$(c,bc',ac')$&$a+b+c'+cc'$&$(a,cc',bc')$&$b'+ba'+ca'+ab'$\\
$(a,ba',ca')$&$a'+b+c+aa'$&$(c,aa',ba')$&$b'+ac'+bc'+cb'$\\  
$(aa',bc',cb')$&$a+a'+bb'+cc'$&$(a',cc',cb')$&$b+ba'+ac'+ab'$\\              
$(bb',ca',ac')$&$b+b'+aa'+cc'$&$(c',aa',ab')$&$b+bc'+ca'+cb'$\\           
  $(cc',ab',ba')$&$c+c'+aa'+bb'$&$(b,cc',ac')$&$a'+ab'+ba'+cb'$\\
&&$(c,bb',ab')$&$a'+ac'+bc'+ca'$\\
&&$(b',cc',ca')$  &$a+ab'+bc'+ba'$\\
&&$(c',bb',ba')$& $a+ac'+ca'+cb'$\\      
\hline  
\end{tabular}
\end{table}

\subsection{The bijection $\Psi$ and supporting curves}\label{results}

We now consider the bijection \eqref{ST} 
in the case where $d=3$.

Applying Proposition~\ref{determinant} to this case allows us to deduce which eigenspace of $|\hat{L}|$ contains the supporting curve $D_{\Psi(\xi)}$ for each  $\xi\in S$.

\begin{prop}\label{support}
Suppose $\xi\in S'$. If $q(\xi)=0$, then 
  $D_{\Psi(\xi)}=C$, and if $q(\xi)=1$, then
$D_{\Psi(\xi)}=B_x$ for some $x\in 10_1$.

In the unique case where $\xi\in S\setminus S'$, $D_{\Psi(\xi)}=C$.
\end{prop}

\begin{proof}
As discussed in \S\ref{isokl}, for any $r\in R$, $D_r$ is either
$C$ or $B_x$ for $x\in 10_1$.
Since $C\cap A[2]=10_1$ and $B_x\cap A[2]=6_1\cup \{x\}$, the statement in the case where $\xi\in S'$ is an immediate consequence of Proposition~\ref{determinant}.
Since $R$ contains
$16$ points that are supported on $C$ and the images of points in $S'$ have accounted for $15$ of them, 
we conclude the proof.
\end{proof}  

We now remark on how the supporting points in $A[2]$ of elements $\xi\in S'$ are situated in the Hudson table of Figure~\ref{mult}. We will use this characterization of $S'$ in terms of Hudson tables to finish our description of the supporting curves of $\Psi(\xi)$ for each $\xi\in S$.

If $q(\xi)=0$, then
the supporting points are situated in one of two ways. 
First, there are six cases where all three points are in $10_1$.
In these cases, the points are in three distinct rows and columns of the Hudson table.
We illustrate this with $(c,c',cc')$ on the left of \eqref{C_examples}; we have placed stars at points in the support of $(c,c',cc')$ and dots at points that are not in the support.
There are also nine cases where $\xi$ supported at two points in $6_1$ and one point in $10_1$.
In all of these cases, there is one point from the top row, one from the leftmost column, and the third point is their product. We illustrate this with $(ba',bc',b')$ on the right of \eqref{C_examples}.
\begin{equation}\label{C_examples}
\begin{tabular}{c|ccc}
		$\cdot$&$\cdot$&$\cdot$&$\cdot$\\
	\hline
	$\cdot$&$\cdot$&$\ast$&$\cdot$\\
	$\cdot$&$\ast$&$\cdot$&$\cdot$\\
	$\cdot$&$\cdot$&$\cdot$&$\ast$
\end{tabular}  
\quad\text{or}\quad
\begin{tabular}{c|ccc}
			$\cdot$&$\cdot$&$\ast$&$\cdot$\\
			\hline
			$\cdot$&$\cdot$&$\cdot$&$\cdot$\\
			$\ast$&$\cdot$&$\ast$&$\cdot$\\
			$\cdot$&$\cdot$&$\cdot$&$\cdot$
		\end{tabular}
\end{equation}  

If instead $q(\xi)=1$, we again distinguish two possibilities.
There are eighteen cases where $\xi$ has one point in $6_1$ and two points in $10_1$. In these cases, the two points in $10_1$ share a row or column in the Hudson table, but occupy a distinct column or row from the point in $6_1$.
We illustrate one such $\xi$ with $(ba',bb',c')$ on the left of \eqref{Bx_examples}. Otherwise, there are two cases where all three supporting points of $\xi$ are in $6_1$:  either we have 
all three points are either in the leftmost column or in the top row,
i.e.~$(ab',bc',ca')$, which is illustrated on the right in \eqref{Bx_examples}.
\begin{equation}\label{Bx_examples}
\begin{tabular}{c|ccc}
		$\cdot$&$\cdot$&$\cdot$&$\cdot$\\
	\hline
	$\cdot$&$\cdot$&$\cdot$&$\ast$\\
	$\ast$&$\cdot$&$\cdot$&$\cdot$\\
	$\cdot$&$\cdot$&$\cdot$&$\ast$
\end{tabular}  
  \quad\text{or}\quad
\begin{tabular}{c|ccc}
			$\cdot$&$\ast$&$\ast$&$\ast$\\
			\hline
			$\cdot$&$\cdot$&$\cdot$&$\cdot$\\
			$\cdot$&$\cdot$&$\cdot$&$\cdot$\\
			$\cdot$&$\cdot$&$\cdot$&$\cdot$
		\end{tabular}
\end{equation}


\begin{prop}\label{Bx}
Let $\xi\in S'$ such that $q(\xi)=1$. Call its supporting points
$\{u,v,w\}\subset A[2]$.
Then $D_{\Psi(\xi)}=B_x$, where 
$x$ is the unique entry in the Hudson table
that is distinct from $u,v,w$
but shares a row or column with each of $u,v,w$.
Equivalently, $x$ is the unique element of $A[2]$ so that $\{xu,xv,xw\}\subseteq 6_1$.
\end{prop}

\begin{proof}
Let $\xi\in S'$ $q(\xi)=1$.
By Proposition~\ref{support},
$D_{\Psi(\xi)}=B_x$ for some $x\in 10_1$.

Applying the approach of Sections \ref{curve} and \ref{notiso}, we know that $x$ is the unique element of $10_1$
so that \hyperlink{star}{$(\star)$} of \eqref{les} fails to be an isomorphism.

For any $x\in 10_1$,
the eigenvalues of $[-1_A]$ acting on $H^0(A,L\otimes P_x)$ are $1,1,-1$. 
The eigenvalues of $[-1_A]$ acting on
$H^0(A,\O_{\xi}\otimes L\otimes P_x)$ are determined by the quadratic form $q_x$. The multiplicities of $1$ and $-1$ are equal to $|\xi\cap 10_x|$ and
$|\xi\cap 6_x|$. In terms of the Hudson table, $|\xi\cap 6_x|$ is the number of supporting points in $\xi$, other than $x$, that share a row or column with $x$.

From our above analysis of the Hudson tables of $\xi\in S$ with $q(\xi)=1$, we know there is a unique $x\in 10_1$ so that $|\xi\cap 6_x|=3$, and thus the eigenvalues of $[-1_A]$ acting on $H^0(A,\O_{\xi}\otimes L\otimes P_x)$ are $-1$ with multiplicity $3$, so \hyperlink{star}{$(\star)$} cannot be an isomorphism and the result is proved.

The alternate characterization of $x$ in the statement of the proposition is an immediate consequence of Lemma~\ref{prod}.
\end{proof}

\begin{rmk}
The proofs of Propositions~\ref{support} and \ref{Bx} show that we are able to detect all of the $2$-torsion points in the support of the curves $\Psi(\xi), \xi\in S$ via differences in the eigenvalues of $[-1_A]$ acting on the domain and codomain of \hyperlink{star}{$(\star)$}.
In light of Remark~\ref{singular}, these proofs have detected all the singular points in these curves as well.
\end{rmk}  

We illustrate Proposition~\ref{Bx}
in the examples shown in \eqref{Bx_examples}
by placing a bullet at the values of $x$:
\[
\begin{tabular}{c|ccc}
		$\cdot$&$\cdot$&$\cdot$&$\cdot$\\
	\hline
	$\cdot$&$\cdot$&$\cdot$&$\ast$\\
	$\ast$&$\cdot$&$\cdot$&$\bullet$\\
	$\cdot$&$\cdot$&$\cdot$&$\ast$
\end{tabular}  
  \quad\text{or}\quad
\begin{tabular}{c|ccc}
			$\bullet$&$\ast$&$\ast$&$\ast$\\
			\hline
			$\cdot$&$\cdot$&$\cdot$&$\cdot$\\
			$\cdot$&$\cdot$&$\cdot$&$\cdot$\\
			$\cdot$&$\cdot$&$\cdot$&$\cdot$
		\end{tabular}
              \]
In Table~\ref{correspond}, 
we show the points of $S$ side by side with the points of $R$ that are supported on the same curve as their images under $\Psi$.
In the two columns on the right, the points are grouped by the values of $x$ in $B_x$, which may be read off from the first summand of each divisor in the rightmost column.

\begin{rmk}\label{bijection} It is also possible to formulate the specific correspondence between the points of $S$ and $R$ given in Table~\ref{correspond} in terms of the Hudson table. 

Let the supporting points of $\xi\in S'$ be $\{u,v,w\}$.
There is a point $z\in A[2]$ so that $(u,v,w)$ is paired with the $z+zu+zv+zw$ in Table~\ref{correspond}, which is selected in the following way:
If $q(\xi)=1$ then $z\in A[2]$ is the unique point so that $\{zu,zv,zw\}\subseteq 6_1$ (Proposition~\ref{Bx}).
If $q(\xi)=0$ then $z\in A[2]$ is a point such that $\{zu,zv,zw\}\subseteq 10_1$. Although $z$ is not always unique, the divisor $z+zu+zv+zw$ will be the same for any such choice of $z$. 
The two cases illustrated in  \eqref{C_examples} behave a little bit differently. 
In the case where 
$\{u,v,w\}\subseteq 10_1$ are in three distinct rows and three distinct columns in the Hudson table, $z$ is unique, and, in fact, must be $1$. In the case where $u\in 10_1$ and $v,w,\in 6_1$, there are four possible values of $z$. The possible values of $z$ in the examples \eqref{C_examples} are marked with diamonds below:
\[
\begin{tabular}{c|ccc}
		$\diamond$&$\cdot$&$\cdot$&$\cdot$\\
	\hline
	$\cdot$&$\cdot$&$\ast$&$\cdot$\\
	$\cdot$&$\ast$&$\cdot$&$\cdot$\\
	$\cdot$&$\cdot$&$\cdot$&$\ast$
\end{tabular}  
\quad\text{or}\quad
\begin{tabular}{c|ccc}
			$\cdot$&$\cdot$&$\ast$&$\cdot$\\
			\hline
			$\cdot$&$\diamond$&$\cdot$&$\diamond$\\
			$\ast$&$\cdot$&$\ast$&$\cdot$\\
			$\cdot$&$\diamond$&$\cdot$&$\diamond$
\end{tabular}
\]
\end{rmk}

\section{Action of the symplectic group}\label{sp}

Let $A$ be an abelian surface with a symmetric polarization $L$ of odd degree.
As discussed in \S\ref{label}, the Weil pairing gives a nondegenerate, strictly alternating form on $A[2]$. In this section we examine the action of its symplectic group on the set $S'$ defined in the previous section. References for the introductory material in this section are \cite{GrossHarris,Farkas}.

\subsection{Action of $\Spa$ on $S'$}
Let $\Spa$ be the group of all $\F_2$-linear transformations $T:A[2]\to A[2]$ that preserve the Weil pairing. $\Spa$ has order $720$ and is generated by $16$ transvections we may define with each $u\in A[2]$:
\[
T_u(v)=v+\langle v,u\rangle u.
\]  
The group $\Spa$ also acts on theta characteristics of $(A,L)$.
If $q$ is a theta characteristic and $T\in \Spa$, then the theta characteristic $Tq$ is defined by the quality that $Tq(Tv)=q(v)$ for each $v\in A[2]$.

There is a natural action of $\Spa$ on $S'$: if $\xi=(u_1,\ldots,u_d)$, then $T(\xi)=(Tu_1,\ldots,Tu_d)$.
The points $Tu_1,\ldots,Tu_d$ are distinct since $T$ is an isomorphism and 
since addition on $A[2]$ as an $\F_2$ vector space coincides with the group operation of $A$, the elements
$Tu_1,\ldots,Tu_d$ will also sum to~$0$.

Many of our results depend on the interaction between elements $\xi\in S'$ and theta characteristics. The action of $\Spa$ preserves these interactions, as we now show.

\begin{lemma}
Let $\xi\in S'$ and $T\in\Spa$. Then $q(\xi)=q(T\xi)$.
\end{lemma}

\begin{proof}
Let $\xi=(u_1,\ldots,u_d)$.
Pick a theta characteristic $q_v$.
By definition,
$q(\xi)=q_v(u_1)+\cdots+q_v(u_d)$.
Then
$q_v(u_1)+\cdots+q_v(u_d)=
Tq_v(Tu_1)+\cdots+Tq_v(Tu_d)=q(T\xi)$ since we may use any choice of
theta characteristic to determine
$q(T\xi)$. 
\end{proof}

\begin{prop}\label{Tq} Let $\xi\in S'$ and $q_v$ be a theta characteristic corresponding to a line bundle $L\otimes P_x$.
We may construct 
the long exact sequence \eqref{les}, which contains the following map
  \[
H^0(A,L\otimes P_x)
  \xra{\hypertarget{star}{(\star)}}
  H^0(A,\O_{\xi}\otimes L\otimes P_x).
\]
Let $T\in \Spa$
and let $L\otimes P_y$ be the line bundle corresponding to $Tq_v$.
Again, we may construct a long exact sequence \eqref{les} using $L\otimes P_y$ and $T(\xi)$, which will similarly contain the following map:
  \[
H^0(A,L\otimes P_y)
  \xra{\hypertarget{star}{(\star\star)}}
  H^0(A,\O_{T(\xi)}\otimes L\otimes P_y).
\]
The eigenvalues of $[-1_A]$ acting on the domains of 
$(\star)$ and $(\star\star)$ are the same.
The eigenvalues of the codomains are also the same.
\end{prop}

In particular, if $(\star)$ cannot be an isomorphism due to the eigenvalues of $[-1_A]$ occurring with different multiplicities in its domain and codomain, then $(\star\star)$ will also not be an isomorphism.

\begin{proof}
  As described in \S\ref{base}, the eigenvalues of the domains of $(\star)$ and $(\star\star)$ are determined by whether the theta characteristics $q_v$, $Tq_v$ are even or odd. The action $\Spa$ on theta characteristics preserves the Arf invariant
  \cite[Proposition~1.11]{GrossHarris}, and thus we have the result for the domains.

  The multiplicities of the eigenvalues $1$ and $-1$ on the codomain of $(\star)$ are $|\{i\mid q_v(u_i)=0\}|$ and $|\{i\mid q_v(u_i)=1\}|$, respectively. The multiplicities
  on the codomain of $(\star\star)$ are 
$|\{i\mid Tq_v(Tu_i)=0\}|$ and $|\{i\mid Tq_v(Tu_i)=1\}|$.
Since $Tq_v(Tu_i)=q_v(u_i)$, we have the result. 
\end{proof}

\subsection{Action of $\Spa$ in the $d=3$ case}

We now assume that $L$ is a $(1,3)$-polarization and examine the orbits of
$\Spa$ acting on $S'$.

\begin{prop}\label{twoorbits}
The action of $\Spa$ on $S'$ has two orbits, which consist of the $\xi\in S'$ where $q(\xi)=0$ and $q(\xi)=1$, respectively.
\end{prop}

\begin{proof}
  We first examine the elements  $\xi\in S'$ where $q(\xi)=1$.
  As described in Proposition~\ref{Bx}, for each such $\xi$ there is a unique $x\in 10_1$ so that the map \hyperlink{star}{$(\star)$} associated to $\xi$ and $q_x$ is not an isomorphism, which is witnessed by the eigenvalues of $[-1_A]$ acting on the domain and codomain of \hyperlink{star}{$(\star)$}.
  For example, for $(ab',bc',ca')\in S'$, we have $x=1$.

 For any $v\in 10_1$, $T_v(q_1)=q_v$. We may see this from observing that $T_v(10_1)=10_v$ and $T_vq_1(T_v(10_1))=q_1(10_1)$. Thus for each $v\in 10_1$,
 $\Psi(T_v(ab',bc',ca'))$ is supported on the curve $B_v$.

 The composition of transvections $T_c\circ T_b\circ T_a$ fixes $q_v$ for $v\in 10_1$ but exchanges the two sets of points $\{ab',bc',ca'\}$ and $\{ac',ba',cb'\}$. Thus we see that the $20$ points
 $\xi\in S'$ where $q(\xi)=1$ form one orbit under the action of $\Spa$.

 It can be checked directly by successively applying transvections that the remaining points also form an orbit.
\end{proof}  


\section{Some examples in the $(1,5)$-polarized case}\label{onefive}
It is possible to use the methods of sections \ref{curve},\ref{notiso} and \ref{sp}
to analyze the bijection $\Psi:S\to R$ for any odd $d$ and make deductions about singularities at points in $\hat{A}[2]$ occurring in the supporting curves of $R$.
In this section, we show
some consequences of these ideas for a 
few points in $S$ in the case where $d=5$.

Let $A$ be an
abelian surface that is $(1,5)$-polarized by a symmetric line bundle $L$ so that $\NS(A)=\Z l$, where $l=[L]$. We assume that $L$ is associated to an even theta characteristic $q_1$.
In this setting, $K_4A$ and $K_{\hat{A}}(0,l,-1)$ are $8$-folds, and the linear system $|L|$ contains curves of genus $6$. The vector space $H^0(A,L)$ is $5$-dimensional. Its two eigenspaces under the action of $[-1_A]$ are
a net $|L|^+$ and
a pencil $|L|^-$ with base loci $6_1$ and $10_1$, resp.

\begin{example}\label{ex.on}
  Consider the point $\xi=(ab',a',c,b,c')\in S$. Since $q(\xi)=1$, by Proposition~\ref{determinant}, $\Psi(\xi)\supseteq 10_1$.

  Moreover, for any $x\in 10_1$, the eigenvalues of $[-1_A]$ acting on $H^0(A,L\otimes P_x)$ are $1$ with multiplicity $3$ and
  $-1$ with multiplicity $2$. The eigenvalues of
  $[-1_A]$ acting on $H^0(A,\O_{\xi}L\otimes P_x)$ are either
$1$ with mult.\ $1$ and $-1$ with mult.\ $4$
or
$1$ with mult.\ $3$ and $-1$ with mult.\ $2$, so it is possible the supporting curve of $\Psi(\xi)$ is smooth at all the points of $10_1$.

Examining \hyperlink{star}{$(\star)$} for $x\in 6_1$, the eigenvalues of $[-1_A]$ occur with identical multiplicities in the domain and codomain, except when $x=ba'$.
Thus, $\Psi(\xi)$ is in the pencil of curves $|L|^-$ and has a singular point $ba'$.
\end{example}

\begin{example}\label{ex.tw}
  Consider the point $\xi=(ab',cb',a',bb',c')\in S$.
  Since $q(\xi)=0$, $6_1\subseteq \Psi(\xi)$.

  When $x\in 6_1$,
  the eigenvalues of $[-1_A]$ acting on $H^0(\O_{\xi}\otimes L\otimes P_x)$
  are $1$ with mult.\ $1$ and $-1$ with mult.\ $4$
  or $1$ with mult.\ $3$ and $-1$ with mult.\ $2$, which does not exclude the possibility that the points $6_1$ are smooth.

When $x\in 10_1$, the eigenvalues of $[-1_A]$ acting on the domain and codomain of
\hyperlink{star}{$(\star)$} differ when $x=b$ and $x=b'$.

Thus $D_{\Psi(\xi)}$ is contained in the net $|L|^+$. It also passes through the points $b$ and $b'$, and it is singular at both of those points.
\end{example}

\begin{example}\label{ex.th}
  Finally, we consider $\xi=(1,ac',ab',b',c')$. Again, $q(\xi)=0$, so $6_1\subseteq \Psi(\xi)$ and $D_{\Psi(\xi)}$ is contained in the net $|L|^+$.

However, when $x=b'$ or $x=c'$,  
the eigenvalues of $[-1_A]$ acting on $H^0(\O_{\xi}\otimes L\otimes P_x)$
are $1$ with mult.\ $5$, and so
$b'$ and $c'$ are singular points of $D_{\Psi(\xi)}$.
\end{example}


We show Hudson tables below for Examples~\ref{ex.on},\ref{ex.tw},\ref{ex.th},
with the singular points on the supporting curves of $\Psi(\xi)$ marked with bullets. In the rightmost table, two of the entries are marked with both bullets and stars.
\[
\begin{tabular}{c|ccc}
		$\cdot$&$\ast$&$\cdot$&$\cdot$\\
	\hline
	$\cdot$&$\ast$&$\ast$&$\cdot$\\
	$\bullet$&$\cdot$&$\cdot$&$\cdot$\\
	$\cdot$&$\ast$&$\cdot$&$\ast$
\end{tabular}  
\quad\quad
\begin{tabular}{c|ccc}
			$\cdot$&$\ast$&$\cdot$&$\cdot$\\
			\hline
			$\cdot$&$\ast$&$\cdot$&$\ast$\\
			$\cdot$&$\cdot$&$\bullet$&$\cdot$\\
			$\ast$&$\bullet$&$\cdot$&$\ast$
\end{tabular}
\quad\quad
\begin{tabular}{c|ccc}
			$\ast$&$\ast$&$\cdot$&$\cdot$\\
			\hline
			$\ast$&$\cdot$&$\cdot$&$\cdot$\\
  $\cdot$&$\cdot$&
$\ast$\hspace*{-1pt}\llap{\raisebox{1.2pt}{$\scriptscriptstyle\bullet$}}
                   &$\cdot$\\
  $\cdot$&$\cdot$&$\cdot$
                                             &
$\ast$\hspace*{-1pt}\llap{\raisebox{1.2pt}{$\scriptscriptstyle\bullet$}}
\end{tabular}
\]  
Although the curves $D_{\Psi(\xi)}$ for $\xi$ in Examples~\ref{ex.tw},\ref{ex.th} both contain two singular points and have the same value of $q(\xi)$, by Proposition~\ref{Tq} they will not be in the same orbit of $S'$ under the action of $\Spa$. 
So, unlike the $d=3$ case, in the $d=5$ case $S'$ will have more than two orbits. 


%%------{BIBLIOGRAPHY}-----------------------------------------------------------------------------------
\bibliographystyle{alpha}
\bibliography{mainbib}

\end{document}

