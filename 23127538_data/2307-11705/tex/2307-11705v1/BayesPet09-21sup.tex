

% This file is to be used as a template for your submission.
%% Rename this file and replace the text with the text of
%% your manuscript.
%%
%% The standard LaTeX document class "article" is recommended.
%% Use options letterpaper and 12pt.


% Statistical Practice
%
%This section contains articles that are interesting or useful with regard to the practice of statistics, such as:
%
%    case studies that illustrate important lessons and issues involved in the practice of statistics, or that deal with applications having broad appeal;
%    discussions and comparisons of modern statistical methods, focusing on their usefulness in practice;
%    articles that address practical problems at the interface between statistical methodology and areas of application; and
%    articles that present statistical solutions to current industrial problems, with emphasis on methodology.

\documentclass[11pt]{article}

%% This is the recommended preamble for your document.

%% Load BEPress specific settings
%\usepackage{bejournal}

%% The mathptmx package is recommended for Times compatible math symbols.
%% Use mtpro2 or mathtime instead of mathptmx if you have the commercially
%% available MathTime fonts.
%% Other options are txfonts (free) or belleek (free) or TM-Math (commercial)
%\usepackage{mathptmx}

%% Use the graphics package to include figures
\usepackage{graphics}
%\usepackage[dvipdfmx]{graphicx,color}
\usepackage{amsthm}
\usepackage{amsmath}
\usepackage{amssymb}
\usepackage{enumerate}
\usepackage{colortbl}
\usepackage{setspace}

\usepackage{latexsym}
\usepackage{amsfonts}
\usepackage{psfrag}
\usepackage{graphicx}

%% Use natbib with these recommended options
\usepackage[authoryear,comma,longnamesfirst,sectionbib]{natbib}

\begin{document}
\newtheorem{prop}{Proposition}
\newtheorem{conj}{Conjecture}
\title{\textbf{\Large Supplementary material for "Small Sample Estimators for Two-way Capture Recapture Experiments"\protect}}

\author{{\normalsize \scshape{Louis-Paul Rivest} } \vspace{0.5cm}\\
\textit{\normalsize Department of Mathematics and Statistics, Universit\'e Laval}, \\
\textit{\normalsize 1045 avenue de la m\'{e}decine, Quebec, QC, G1V 0A6 Canada}\vspace{0.5cm}\\
{\normalsize \scshape{Mamadou Yauck} } \\
\textit{\normalsize Department of Mathematics, Université du Québec à Montréal}, \\
\textit{\normalsize 201 Avenue du Président-Kennedy, Montreal, QC, H3C 3P8 Canada}\vspace{0.5cm} }

\date{ }
\maketitle

\vspace{.5cm}
%{Corresponding author: Louis-Paul.Rivest@mat.ulaval.ca}

%\footnote{ This work is supported by }

\vspace{0.3cm}
\linespread{1.5}



\singlespacing

\section{R code for the generalized Waring cumulative distribution function}

\begin{verbatim}
pwar<-function(q,a,b,c,lower.tail=TRUE)
#  q is a positive quantile, or a vector of quantiles 
# a b and c are positive real parameters satisfying c>a+b
{
  if(length(q)==1){
    xx<-integrate(function(x) pnbinom(q, size=a, prob=x
                            , lower.tail = lower.tail)
    *dbeta(x, shape1=c-b-a, shape2=b, ncp = 0, log = FALSE),
                  lower = 0, upper = 1)
    xx<-xx$value
  }else{
    xx<-NULL
    for (j in 1:length(q)) {
      xxi<-integrate(function(x) pnbinom(q[j], size=a,
                    prob=x, lower.tail = lower.tail)
      *dbeta(x, shape1=c-b-a, shape2=b, ncp = 0, log = FALSE),
                    lower = 0, upper = 1)
      xx<-c(xx,xxi$value)
    }
  }
  return(xx)
}

\end{verbatim}

\section{R code for the generalized Waring quantile function}

\begin{verbatim}

qwar<-function(p,a,b,c)
#  p is a probability in (0,1)
# a b and c are positive real parameters satisfying c>a+b
{
  if (pwar(0,a,b,c)>p) return(0)
  else {seth<-sqrt(1/(a-.5)+1/(b-.5)+1/(c-b-a-.5)+(c-b-a-.5)/((a-.5)*(a-.5)))
  quan0<-quan1<-round(c+(a-.5)*(b-.5)/(c-b-a-.5)*exp(qnorm(p))*seth,0)
  test<-pwar(quan0,a,b,c)
  if (test>p){
    while (test>p)
    {
      quan1<-quan1-1
      test<-pwar(quan1,a,b,c)
    }
    return(quan1+1)
  }
  else{
    while (test<p)
    {
      quan1<-quan1+1
      test<-pwar(quan1,a,b,c)
    }
    return(quan1)
  }
  }
}
\end{verbatim}

\section{R code for the generalized Waring probability function}

\begin{verbatim}

dwar<-function(y,a,b,c)
#  y is a non negative integer or a vector of
# a b and c are positive real parameters satisfying c>a+b
{
  if(length(y)==1){
    xx<-integrate(function(x) dnbinom(y, size=a, prob=x)
    *dbeta(x, shape1=c-b-a, shape2=b, ncp = 0, log = FALSE),
                  lower = 0, upper = 1)
    return(xx$value)
  }else{
    xx<-NULL
    for (j in 1:length(y)) {
      xxi<-integrate(function(x) dnbinom(y[j], size=a, prob=x)
      *dbeta(x, shape1=c-b-a, shape2=b, ncp = 0, log = FALSE),
                     lower = 0, upper = 1)
      xx<-c(xx,xxi$value)
    }
  }
  return(xx)
}

\end{verbatim}




\end{document}

