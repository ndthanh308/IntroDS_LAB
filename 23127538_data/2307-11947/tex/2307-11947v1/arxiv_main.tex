\documentclass[nohyperref]{article}
\def\usemedgeometry{%
  \usepackage[top=2.54cm,left=3.0cm,right=3.0cm,bottom=2.54cm]{geometry}}
\usemedgeometry


\usepackage{etoolbox}

\newtoggle{arxiv}

\toggletrue{arxiv}

\usepackage[numbers]{natbib}










\usepackage[utf8]{inputenc} %
\usepackage[T1]{fontenc}    %
\usepackage{hyperref}       %
\usepackage{url}            %
\usepackage{booktabs}       %
\usepackage{amsfonts}       %
\usepackage{nicefrac}       %
\usepackage{microtype}      %
\usepackage{xcolor}         %
\usepackage{amsmath}
\usepackage{amssymb}
\usepackage{mathtools}
\usepackage{amsthm}
\usepackage{tabularx}
\usepackage{array}


\newcolumntype{C}[1]{>{\centering\arraybackslash}m{#1}}
\usepackage[capitalize,noabbrev]{cleveref}

\theoremstyle{plain}
\newtheorem{theorem}{Theorem}[section]
\newtheorem{proposition}[theorem]{Proposition}
\newtheorem{lemma}[theorem]{Lemma}
\newtheorem{fact}[theorem]{Fact}
\newtheorem{corollary}[theorem]{Corollary}
\theoremstyle{definition}
\newtheorem{definition}[theorem]{Definition}
\newtheorem{assumption}[theorem]{Assumption}
\theoremstyle{remark}
\newtheorem{remark}[theorem]{Remark}

\usepackage{gary-macros}
%\usepackage[subtle,title=tight]{savetrees} 
%\usepackage[small,compact]{titlesec}
% \usepackage{amsthm}
% \usepackage[english]{babel}
\usepackage{ifthen}
\usepackage{xcolor}
% \usepackage{blindtext}
% \usepackage{algorithm}
% %\usepackage{subfigure}
% \usepackage{graphicx}
\usepackage{amsmath}
% \usepackage[noend]{algpseudocode}
% \usepackage{subfig}
\usepackage{xspace}
\usepackage{amssymb} 
\usepackage{multirow}
\usepackage{url}
\usepackage{hyperref}
% \usepackage{balance}
% \newtheorem{remark}{Remark}
% \usepackage{graphicx}
% \usepackage{footnote}
% \usepackage{comment}
\usepackage{array}
\newcolumntype{P}[1]{>{\centering\arraybackslash}p{#1}}

\usepackage{url}
\def\UrlBreaks{\do\/\do-}

\newcommand{\exclude}[1]{}
\newcommand{\showComments}{yes}
\newcommand{\note}[2]{
    \ifthenelse{\equal{\showComments}{yes}}{\textcolor{#1}{#2}}{}
}
\newcommand{\TODO}[1]{%
  \addcontentsline{tdo}{todo}{\protect{#1}}%
  \note{red}{TODO: #1}
}

\newcommand\numberthis{\addtocounter{equation}{1}\tag{\theequation}}

\newcommand{\frank}[1]{\note{brown}{[WW: #1]}}
\newcommand{\manya}[1]{\note{red}{[MG: #1]}}
\newcommand{\naader}[1]{\note{brown}{[NH: #1]}}
\newcommand{\kayvon}[1]{\note{violet}{[KS: #1]}}
\newcommand{\ying}[1]{\note{green}{[YZ: #1]}}


\newcommand{\psass}{\ensuremath{\mathbin{{=}}\ }}
\newcommand{\addeq}{\ensuremath{\mathbin{{+}{=}}\ }}
\newcommand{\subeq}{\ensuremath{\mathbin{{-}{=}}\ }}
\newcommand{\muleq}{\ensuremath{\mathbin{{\times}{=}}\ }}
\newcommand{\diveq}{\ensuremath{\mathbin{{\divides}{=}}\ }}
\newcommand{\eqeq}{\ensuremath{\mathbin{{=}{=}}\ }}
\newcommand{\todo}[1]{{\color{red} #1}}
\newcommand{\name}{{\sc{OEAINet}}\xspace}
\newcommand{\cc}{{\sc{c$^2$}}\xspace}

\newcommand{\fattree}{{Clos}\xspace}
\newcommand{\fattrees}{{Clos}\xspace}
\newcommand{\LBE}{{\fattree}\xspace}
\newcommand{\SBE}{{Ideal Switch}\xspace}
\newcommand{\OBE}{{Oversub. \fattree}\xspace }
\newcommand{\para}[1]{{\textbf{{#1}}}}
\newcommand{\net}{{{Big-Net}}\xspace}
\newcommand{\MP}{{MP}\xspace}
\newcommand{\allreduce}{{AllReduce}\xspace}
\newcommand{\ata}{{All-to-All}\xspace}
\newcommand{\allgather}{{AllGather}\xspace}
\newcommand{\redsca}{{Reduce-Scatter}\xspace}
\newcommand{\fbd}{{Full-Bisection Domain}\xspace}
\newcommand{\dcd}{{Direct-Connected Domain}\xspace}



\newcommand{\captionvspace}{0em}
\pagestyle{plain}


\newenvironment{CompactItemize}
  {\def\usecounter{\compactify\latexusecounter}
   \begin{itemize}}
  {\end{itemize}\let\usecounter=\latexusecounter}
\date{}
\usepackage{lipsum} % for dummy text
\usepackage{enumitem}
\setlist{nosep} % or \setlist{noitemsep} to leave space around whole list


\newcommand{\footremember}[2]{%
   \footnote{#2}
    \newcounter{#1}
    \setcounter{#1}{\value{footnote}}%
}
\newcommand{\footrecall}[1]{%
    \footnotemark[\value{#1}]%
}

\title{Collaboratively Learning Linear Models with Structured Missing Data}




\author{%
  Chen Cheng\footremember{authorship}{Equal contribution, authors ordered alphabetically by last then first names }\footremember{statdept}{Statistics Department, Stanford University} \\
  \texttt{chencheng@stanford.edu} 
    \and
   Gary Cheng\footrecall{authorship}
   \footremember{eedept}{Electrical Engineering Department, Stanford University}\\
  \texttt{chenggar@stanford.edu} 
  \and
  John Duchi\footrecall{eedept}
  \footrecall{statdept}\\
  \texttt{jduchi@stanford.edu} 
}

\begin{document}


\maketitle


\begin{abstract}



  We study the problem of collaboratively learning least squares estimates for $m$ agents. Each agent observes a different subset of the features---e.g., containing data collected from sensors of varying resolution. Our goal is to determine how to coordinate the agents in order to produce the best estimator for each agent. We propose a distributed, semi-supervised algorithm \textsc{Collab}, consisting of three steps: local training, aggregation, and distribution. Our procedure does not require communicating the labeled data, making it communication efficient and useful in settings where the labeled data is inaccessible. Despite this handicap, our procedure is nearly asymptotically local minimax optimal---even among estimators allowed to communicate the labeled data such as imputation methods. We test our method on real and synthetic data.













\end{abstract}

% Figure environment removed

\section{Introduction}
Automatic 3D reconstruction of clothed humans using image inputs has gained increasing significance due to its potential applications in a wide array of AR/VR scenarios. High-fidelity reconstructions typically depend on sophisticated capture systems, which are developed with dense camera arrays~\cite{collet2015high,joo2015panoptic,joo2018total}, programmable light-stages~\cite{Vlasic2009, guo2019relightables}, and depth sensors~\cite{newcombe2011kinectfusion,DoubleFusion,BodyFusion,dou2016fusion4d,newcombe2015dynamicfusion}. However, stringent capture environments equipped with complex hardware pose significant challenges for consumer-level applications.


In this context, considerable research effort has been dedicated to developing methods that allow for more flexible capture configurations, such as utilizing a few RGB inputs. Among these works, learning implicit functions \cite{iccv2020PIFu, saito2020pifuhd, hong2021stereopifu} has proven effective in achieving highly detailed reconstructions by integrating the advancements of deep neural networks. These methods employ large multi-layer perceptrons (MLPs) to predict the occupancy probability or truncated signed distance function (TSDF) value of every queried 3D point based on its associated local feature, which is extracted from images. They can recover a continuous surface at arbitrary resolutions without topology restrictions.


However, in typical MLP-based implicit networks, the occupancy or TSDF value at each location is solved independently with planar image features, rendering them less capable of addressing challenging cases such as occlusions. Consequently, these methods suffer from generalization and robustness issues, particularly when tackling strong occlusions caused by large motion or multiple interacting humans. 
Some follow-up studies  \cite{zheng2021deepmulticap,zheng2021pamir,huang2020arch} utilize an extra geometric model, SMPL~\cite{Loper2015}, to improve robustness by introducing strong shape priors. 
Their success typically relies on the assumption of geometrical similarity \cite{huang2020arch} between the shape prior and target reconstruction, making them intractable for handling complex cases with loose clothes and sensitive to errors in SMPL model fitting.



%\ping{this paragraph sounds like `TSDF is better than MLP/SMPL, and we use TSDF to solve the problem'. But in Sec 3, we are telling a different story, saying `MLP needs a 3D convolutional encoder'. We need to make these two sections consistent.}\sicong{I think in this paragraph we claim that the TSDF}


%We opt for Trucated Signed Distance Funtion (TSDF) volumetric representations as they are naturally suitable for convolution operations, which have shown remarkable performance for learning hierarchical features on 2D visual perception tasks \cite{SunXLW19}. 
%Meanwhile, TSDF also describes the gradual geometry change around shape surface, which is not reflected by occupancy volume. 

We instead revisit the 3D volumetric representation and resort to 3D convolutional neural networks (CNNs) for feature learning, due to their impressive performance in feature learning and the ability to incorporate spatial context. However, volumetric methods and 3D convolution involve discretization, which might raise concerns regarding whether a discretized volume can preserve subtle geometric details as continuous representations learned in implicit functions. We investigate the relationship between volume resolution and quantization error on synthetic data by converting target mesh objects to TSDF volumes, as shown in Figure~\ref{fig:quantization_error}. We observe that the quantization errors are significantly reduced by increasing volume resolution and become nearly negligible when reaching a relatively high resolution (e.g., 512 or higher). In other words, achieving fine-detailed reconstruction is not supposed to be restricted by the use of volume representations as long as a proper volume resolution is utilized. Therefore, we present a method with high-resolution feature volumes, e.g., 256 and 512, while traditional volumetric methods \cite{varol18_bodynet,gilbert2018volumetric} are often limited to much lower resolutions, such as 32 or 128.



On the other hand, an increase in volume resolution may lead to a cubic growth of memory overhead \cite{8100085}. Reducing memory costs while guaranteeing the granularity of volumetric representations is necessary for pursuing high-quality reconstruction. Thus, we adopt a coarse-to-fine approach and cull away irrelevant voxels to build a sparse high-resolution feature volume. At the coarse level, the network computes an initial TSDF by applying a U-Net with sparse 3D CNN \cite{3DSemanticSegmentationWithSubmanifoldSparseConvNet} on the sparse feature volume, which is carved by a visual hull. Through our experiments, it turns out that more than 95\% of the volume grids are discarded by the visual hull culling, making the sparse 3D CNN efficient. At the fine level, the network focuses on a narrow band near the zero-level set of the initial TSDF and discretizes the narrow band with smaller voxels. By employing this narrow-band culling, we further shrink the sampling space, resulting in a relatively small range of grid numbers (usually 300K--500K in our experiments) even with a high volume resolution of 512. The remaining voxels in the narrow band are associated with features that fuse high-frequency information from the computed normal maps upon the low-frequency shape from the coarse level to compute the TSDF at high resolution. The final mesh is then extracted from the TSDF using the Marching-Cube algorithm ~\cite{Lorensen87marchingcubes}.
% Different from the u-net sturcture to preserve global topology context, we then apply a shallow 3dcnn to compute the final TSDF $D_{final}$ which contain more local geometry detail.




% \ping{this paragraph can be expanded. It is an important contribution and often ignored by other works. stress on the novel idea of regressing blending weights instead of colors}

In addition to geometry, high-quality mesh texture is also a crucial factor contributing to visual appearance. Directly computing a color field in 3D space, as in \cite{iccv2020PIFu}, struggles to capture high-frequency texture details, while the neural radiance field (NeRF) \cite{yu2020pixelnerf} or the DoubleField~\cite{shao2022doublefield} require expensive per-instance optimization and are often unstable for sparse input images. In contrast, we adopt an image-based rendering approach to compute a texture atlas map, which is efficient and widely supported in existing computer graphics tools. 
Specifically, we compute a blending weight at each 3D point on the mesh surface to determine its color as a weighted average of the colors at its image projections. The blending weights can be computed at a relatively coarse resolution, e.g., 512 volume resolution in our case, and leave texture details to the high-resolution images, such as 1K or 2K. Unlike previous methods that generate blurry texturing results under sparse input, our method generalizes well on both synthetic and real data with just a few input views. 
Figure~\ref{fig:teaser} shows two examples reconstructed by our method. Despite the challenging garment, pose, and occlusion, our method recovers faithful shape, normal, and texture on the right.

%with a wide variety of poses and clothing styles, and it is also adaptive to handle input image with arbitrary resolutions.
%\sicong{For this concern we claim that when the resolution of dicretized volume meets certain threshold (which is 256 in our experiment), the quantization error can be neglected.} 



In summary, the main contributions of this paper are as follows:
\begin{itemize}
\vspace{-0.1in}
  \item 
  We revisit the 3D volumetric representation and demonstrate that it can support clothed human reconstruction with equal or even better performance compared to implicit representation. 
  \item 
  We develop a memory and computation-efficient method for high-resolution volumetric reconstruction using sophisticated sparse 3D CNN, coarse-to-fine estimation, and voxel culling by visual hull and narrow bands. 
  \item 
  We introduce a novel method to compute a texture atlas map, which captures rich appearance details from high-resolution input images.
  \item 
  We achieve impressive results on standard benchmark datasets Twindom and MultiHuman, significantly reducing the point-2-surface (P2S) precision to approximately 0.2cm from just six input views, with more than $50\%$ error reduction compared to the state-of-the-art methods, including DoubleField~\cite{shao2022doublefield} and PIFuHD~\cite{saito2020pifuhd}.
\end{itemize}

\section{Mathematical model}
\label{sec:linear-model}
We assume we have $m$ agents that observes a subset of the dimensions of the input data $x \in \R^d$. Each agent $i$ has a ``view'' permutation matrix $\projm_i^\top := \begin{bmatrix} \projm\iplus^\top & \projm\imin^\top \end{bmatrix} \in \R^{d \times d}$. $\projm\iplus \in \R^{d_i\times d}$ describes which feature dimensions the agent sees, and $\projm\imin \in \R^{(d-d_i) \times d}$ describes the dimensions the agent does not see. 
For a feature, label pair $(x, y)$, the $i$-th agent has data $(x\iplus, y)$ where $x\iplus \defeq \projm\iplus x\in \R^\di$.
Each agent has $n$ such observations (independent across agents) denoted as a matrix $X\iplus \in R^{n \times \di}$ and vector $y_i\in\R^n$.
We let $X\imin \in \R^{n \times (d- \di)}$ denote the unobserved dimensions of the input data $x$ drawn for the $i$-th agent, and we let $X_i \in \R^{n \times d}$ denote the matrix of input data $x$ drawn for the $i$-th agent, including the dimensions of $x$ unobserved by the $i$-th agent. To simplify discussions in the following sections, for any vector $v \in \R^d$ we use the shorthand $v\iplus = \projm\iplus v$ and $v\imin = \projm\imin v$. Similarly for any matrix $A \in \R^{d \times d}$ we denote by
\begin{align*}
	A\iplus & = \projm\iplus A \projm\iplus^\top ,
	&  A\imin & =  \projm\imin A \projm\imin^\top ,  \\
	A\ipm & = \projm\iplus A \projm\imin^\top ,
	&  A\imp & =  \projm\imin A \projm\iplus^\top .
\end{align*}
For a p.s.d.\ matrix $A$, we let $\norm{x}_A = \< x, Ax\>$. 

We assume the data from the $m$ agents follow the same linear model. The features vectors $x$ comprising the data matrices $X_1, \ldots, X_m$ are i.i.d.\ with zero mean and covariance $\scov \succ 0$. 
We will assume that each agent has knowledge of $\scov\iplus$---e.g., they have a lot of unlabeled data to use to estimate this quantity. The labels generated follow the linear model
\begin{align*}
	y_i = X_i \param + \noise_i, \qquad \noise_i \simiid\ \normal(0, \sigma^2 I_n).
\end{align*} 
Throughout this work we consider a  fixed ground truth parameter $\param$.


\paragraph{Objectives.} We are interested in proposing a method of using the data of the agents to form an estimate $\hparam$ which minimizes the full-feature prediction error on a fresh sample $x \in \R^d$
\begin{align}\label{eqn:global-test-loss}
	\E_{x}[( \< x, \hparam\> - \<x, \param\>)^2] = \|\hparam - \param
    \|_\scov^2.
\end{align}
We are also interested in forming an estimate $\hparam_i$ which minimizes the missing-feature prediction error of a fresh sample $x\iplus \in \R^{d_i}$ for agent $i$---i.e., $x\iplus = \projm\iplus x$ where $x \in \R^d$ is fresh. Define $T_i \defeq \begin{bmatrix} I_{d_i} & \scov\iplus^{-1} \scov\ipm \end{bmatrix} \projm_i$ and the Schur complement $\Gamma\imin \defeq \scov \backslash \scov\iplus \defeq \scov\imin - \scov\imp \scov\iplus^{-1} \scov\ipm$. The local test error is then
\begin{align}\label{eqn:local-test-loss}
	\E_{x}[( \<x\iplus, \hparam_i\> - \<x, \param\>)^2]= \|\hparam_i - T_i \param \|_{\scov\iplus}^2 +  \norm{\param\imin}_{\Gamma\imin}^2
\end{align}

Here, $\norm{\param\imin}_{\Gamma\imin}^2$ is irreducible error. The role of the operator $T_i$ is significant as $T_i \param$ is the best possible estimator for the $i$th agent\footnote{Maybe surprisingly, $T_i \param$ is better than naively selecting the subset of $\param$ corresponding to the features observed by agent $i$---i.e., $\projm_i \param$. This is because $T_i \param$ leverages the correlations between features.}.
Through this paper, we will also highlight the communication costs of the methods we consider. Recall that we would like our methods to have $o(n)$ communication cost.



\section{Our approach} \label{sec:upper-bounds}


We begin by outlining an approach of solving this problem for general feature distributions. 
The general approach is not immediately usable because it requires some knowledge of $\param$, so we need to do some massaging. 
In \Cref{sec:collab}, we show how to circumvent this issue in the Gaussian feature setting and introduce our method \textsc{Collab}.
Adapting the general approach to other non-Gaussian settings is an open problem, but we discuss some potential approaches in \Cref{sec:discussion}.

\subsection{General approach}
Our solution begins with each agent $i$ performing ordinary least squares on their local data
\begin{align*}
	\est{\param}_i = X_{i+}^\dagger y_i \stackrel{\mathrm{(i)}}{=} (X_{i+}^\top X_{i+})^{-1} X_{i+}^\top y_i,
\end{align*} 
where $A^\dagger$ denotes the Moore–Penrose inverse for a general matrix $A$, and (i) holds whenever $\rank(X_{i+}) \geq d_i$. Because we focus on the large sample asymptotics regime ($n \gg d_i$), (i) will hold with probability $1$.

Then, we aggregate $\hat{\theta}$ using a form of weighted empirical risk minimization parameterized by the positive definite matrices $W_i \in \R^{d_i \times d_i}$
\begin{align} \label{eq:defn-weighted-avg-est-param}
	\est{\param} = \est{\param}(W_1, \cdots, W_m) \defeq \argmin_\param \sum_{i=1}^m \norm{\param\iplus + \scov\iplus^{-1} \scov\ipm \param\imin - \est{\param}_i }_{W_i}^2.
\end{align}
We know by first order stationarity that 
$\est{\param} = \prnbig{\sum_{i=1}^m T_i^\top W_i T_i}^{-1} \prnbig{\sum_{i=1}^m T_i^\top W_i \est{\param}_i}$. 
$\est{\param}$ is a consistent estimate of $\param$ regardless the choice of weighting matrices $W_i$. Furthermore, if the features $X_i$, $\est{\param}$ are Gaussian, $\est{\param}$ is also unbiased. 
We show this result in the Appendix in \Cref{lem:est-param-consistency}.
While \Cref{lem:est-param-consistency} shows that the choice of weighting matrices $W_i$ does not affect consistency, the choice of weighting matrices $W_i$ does dictate the asymptotic convergence rate of the estimator. In the next theorem, we show what the best performing choice of weighting matrices are. 
The proof is in \Cref{sec:proof-upper-bound}.


\newcommand{\gauss}{^{\textup{g}}}
\begin{theorem} \label{thm:upper-bound}
	For any weighting matrices $W_i$, the aggregated estimator $\est{\param} = \est{\param}(W_1, \cdots, W_m)$ is asymptotically normal
	\begin{align*}
		\sqrt{n} \prn{\est{\param} - \param} = \normal(0, C(W_1, \cdots, W_m)),
	\end{align*}
	with some covariance matrix $C(W_1, \cdots, W_m)$. The optimal choice of weighting matrices is
	$$W_i\opt \defeq \scov\iplus (\Ep \brk{x\iplus  \param\imin^\top z\iplus  z\iplus^\top  \param\imin x\iplus^\top} + \sigma^2 \scov\iplus)^{-1} \scov\iplus,$$ where $z\iplus = x\imin - \scov\imp \scov\iplus^{-1} x\iplus$. In particular, 
	for all $W_i$, 
		$C(W_1, \cdots, W_m) \succeq C(W_1\opt, \cdots, W_m\opt) =\prn{\sum_{i=1}^m T_i^\top  W_i\opt   T_i}^{-1}$.
		
\end{theorem}
The main challenge of using \Cref{thm:upper-bound} is in constructing the optimal weights $W_i\opt$, as at face value, they depend on knowledge of $\param$. While we will discuss high level strategies of bypassing this issue in non-Gaussian data settings in \Cref{sec:discussion}, we will currently focus our attention on how we can make use of Gaussianity to construct our estimator \textsc{Collab}.

\subsection{\textsc{Collab} Estimator - Gaussian feature setting}\label{sec:collab}

\SetKwComment{Comment}{/* }{ */}
\begin{algorithm}[t]
\caption{\textsc{Collab} algorithm}\label{alg:collab}
\KwData{$m$ agents with training data $(X_{1+}, y_1), \ldots, (X_{m+}, y_m)$ each with $n$ datapoints}
\For{Each agent $i=1, \ldots, m$ in parallel}{
  Compute $\hparam_i = (X_{i+}^\top X_{i+})^{-1} X_{i+}^\top y_i$\;
  Compute $\hscov_i = \frac{1}{n} X\iplus^\top X\iplus$ or with additional unlabeled data\;
  Compute $R_i = \frac{1}{n}\|X\iplus  \hparam_i - y\|_2^2$\;
  Send $\hparam_i, \hscov_i, R_i$ to central server\;
}
Central server constructs $\hat{W}_i\gauss \defeq \hscov\iplus / R_i$\;
Central server computes $\est{\param}_i\collab = T_i \est{\param}(\hat{W}_1\gauss, \cdots, \hat{W}_m\gauss)$ and distributes them to respective agents\;
\end{algorithm}


If $X_i$ are distributed as $\normal(0, \scov)$, $W_i\opt$ has an explicit closed form as
\begin{align*}
	W_i \opt =W_i\gauss \defeq \frac{\scov\iplus}{\norm{\param\imin}_{\Gamma\imin}^2 + \sigma^2} = \frac{\scov\iplus}{\E_{x, y}[(\< x\iplus, \hparam_i\> - y)^2]},
\end{align*}
where $\Gamma\imin =  \scov\imin - \scov\imp \scov\iplus^{-1} \scov\ipm$ is the Schur complement. 
Recall we assume that each agent has enough unlabeled data to estimate $\scov\iplus$. Furthermore, $\frac{1}{n}\|X\iplus  \hparam_i - y\|_2^2$ is a consistent estimator of $\E_{x, y}[(\< x\iplus, \hparam_i\> - y)^2]$.
Thus, each agent is able to construct estimates of $W_i\gauss$ by computing
\begin{align*}
	\hat{W}_i\gauss \defeq \frac{\scov\iplus}{\frac{1}{n}\|X\iplus  \hparam_i - y\|_2^2}
\end{align*}

Now we construct our global and local \textsc{Collab} estimators defined respectively as
\begin{align}\label{eqn:collab-est}
	\begin{split}
		\est{\param}\collab \defeq \est{\param}(\hat{W}_1\gauss, \cdots, \hat{W}_m\gauss), \qquad\quad
		\est{\param}_i\collab \defeq T_i \est{\param}(\hat{W}_1\gauss, \cdots, \hat{W}_m\gauss).
	\end{split}
\end{align}
We summarize the \textsc{Collab} algorithm in Algorithm~\ref{alg:collab}.
At a high level, $\est{\param}\collab$ is an estimate of $\param$ which also minimizes the full-feature prediction error \eqref{eqn:global-test-loss} and $\est{\param}_i\collab$ minimizes the missing-feature prediction error for agent $i$ \eqref{eqn:local-test-loss}.
Now we show that using the collective ``biased wisdom'' of local estimates $\est{\param}_i$, our collaborative learning approach returns an improved local estimator. The proof is in \Cref{sec:proof-upper-bound-local}.

\begin{corollary}  \label{cor:upper-bound-local}
	Let $X_i \sim \normal(0, \scov)$ and define $C\gauss \defeq (\sum_{i=1}^m T_i^\top  W_i\gauss   T_i)^{-1}$. The global \textsc{Collab} estimator $\est{\param}_i\collab$ and the local $\est{\param}_i\collab$ on agent $i$ are asymptotically normal
	\begin{align*}
		\sqrt{n} \prn{\est{\param}\collab - \param} \cd \normal\left(0, C\gauss \right) \quad \text{and}\quad 
		\sqrt{n} \prn{\est{\param}_i\collab -  T_i \param} \cd \normal\left(0, T_i C\gauss T_i^\top\right).
	\end{align*}
	The following are true
	\begin{itemize}
		\item[(i)] $W_i\gauss$ are the optimal choice of weighting matrices i.e.,particular, $C(W_1, \cdots, W_m) \succeq C(W_1\gauss, \cdots, W_m\gauss) =C\gauss$.
		\item[(ii)] On agent $i$, we have $\sqrt{n}(\est{\param}_i - T_i \param) \cd \normal(0, (W_i\gauss)^{-1})$. The asymptotic variance of $\est{\param}_i$ is larger than that of the \textsc{Collab} estimator $\est{\param}_i\collab$---i.e., $(W_i\gauss)^{-1} \succeq T_i C\gauss T_i^\top$.
	\end{itemize}
\end{corollary}





\subsection{Comparison with state of the art}
\label{sec:comparison}

In this section we intend to answer RQ2 by comparing \dataset with the state-of-the-art while list of libraries generated by Li et al~\cite{10.1109/SANER.2016.52} (RQ2.a) and against LibRadar~\cite{10.1145/2889160.2889178}, a state-of-the-art tool to detect libraries in Android apps (RQ2.b).


\subsubsection{Comparison with state-of-the-art white list}

Since Li et al.'s dataset was generated prior to 2016, it would be unfair to compare both datasets as is, rather than the approaches themselves.
Hence, we applied our approach and only retain libraries that were available before 2016.
In this section, we call our dataset \dataset$_{2016}$

\noindent
\textbf{Comparison dataset:}
To compare with the work of Li et al., we retrieved the package names available in the "libraries" folder of the repository~\cite{commonLibrariesRepo} described in their paper~\cite{LI2019157}.
After removing any duplicates, the resulting list, which we refer to as \textbf{comparison\_dataset}, contains \num{5926} package names.

% Figure environment removed


\noindent
\textbf{Comparison:}
Upon a first examination, we found that the two lists have little in common, as \dataset$_{2016}$ contains \num{9551} package names while \textbf{comparison\_dataset} contains only \num{5926} package names, and the intersection of the two lists is only 66, as shown in Figure~\ref{fig:intersection}.
This suggests that our list is significantly larger and more comprehensive than the state-of-the-art list.
However, it should be noted that this comparison was made using strict one-to-one string matching, so some of our package names might be prefixes of their package names, and vice versa.

As a result, we conducted a follow-up comparison to determine if any of the libraries in \dataset$_{2016}$ might be prefixes of the package names in the \textbf{comparison\_dataset} and vice versa.
Results indicate that 280 of our library names are prefixes of 1636 of their library names.
Additionally, 101 of their library names are prefixes of 194 of our library names.
The total number of common libraries using prefix matching is \num{1722}, as shown in Figure~\ref{fig:intersection}.
These findings show some overlap between the two lists and that \dataset$_{2016}$ covers a larger part of the \textbf{comparison\_dataset}.
However, the overall intersection of the two lists is still relatively small, with only a small proportion of libraries being common to both lists.

To provide further assessment of the \textbf{comparison\_dataset}, a qualitative study was conducted. 
This study aimed to identify potential libraries in the \textbf{comparison\_dataset} that were not included in the package names found in both \dataset$_{2016}$ and the \textbf{comparison\_dataset} with prefix matching (in other words, we want to answer the following question: does the \textbf{comparison\_dataset} contain many libraries that \dataset$_{2016}$ does not?). 
A total of 50 package names were randomly selected. 
We manually searched the Internet to identify any mentions, repositories, or websites that would indicate that they are libraries. 
Results indicate that only 2 out of the 50 package names were found to be actual libraries.
When searching the Internet we found  that these two libraries had dedicated website which were not crawled by our approach.
The 50 package names are available in the project's repository.

We did not further manually check whether the libraries identified by \dataset$_{2016}$ but not by  \textbf{comparison\_dataset} are true libraries or not because, by construction, \dataset only contains "true" libraries.

\highlight{
\textbf{RQ2.a answer:}
Our approach, \dataset$_{2016}$, exhibits a larger coverage of libraries compared to the state-of-the-art dataset by Li Li et al.
Although there is some overlap between the two lists, the overall intersection remains relatively small. 
Our qualitative analysis shows that Li et al. dataset is not reliable, unlike  \dataset.
}

\subsubsection{Comparison with LibRadar}

LibRadar was introduced as a tool to detect third-party libraries in Android apps.
Although its primary purpose is not to generate a dataset of libraries, it can be used to create a list by running it on apps and retaining the libraries identified in the output.
To this end, we executed LibRadar on \num{100000} recent apps (i.e., collected after 2020) and compiled a list of libraries. 
LibRadar successfully completed the analysis for \num{62308} apps, producing a list of \num{12239} unique package names.
However, this list cannot be used as a whitelist since it contains over \num{3000} obfuscated package names (e.g., a.a.a). 
While these package names might represent libraries in apps, they cannot be employed as is for a whitelist, since a.a.a might not be a library in another app. 
Consequently, we cannot directly compare both lists. 

To further investigate which of LibRadar or \dataset would be more useful into filtering libraries in apps, we have randomly selected 100 apps from the \num{62308} previously analyzed apps.
For each of these apps we
\dcircle{1} applied LibRadar to extract the libraries it would detect; and 
\dcircle{2} extracted the package name of each class and used \dataset to check if they were libraries or not.
In total, LibRadar is able to filter 174 non-obfuscated libraries from these 100 apps, whereas \dataset can filter \num{2745}.
The median number of libraries found per app is 1 for LibRadar and 28 for \dataset.
This result shows that LibRadar misses a vast amount of libraries in Android apps that \dataset would not miss.


\highlight{
\textbf{RQ2.b answer:}
Our empirical analysis shows that \dataset covers more non-obfuscated libraries than LibRadar in apps.
}

\newcommand{\nth}{^{(n)}}
\section{Asymptotic Local Minimax Lower Bounds}
\label{sec:lower-bounds}
In this section, we prove asymptotic local minimax lower bounds that show \textsc{Collab} is (nearly) optimal. We work in the partially-fixed-design regime. 
For every sample $x \in \R^d$, $x\iplus\in \R^{d_i}$ is a fixed vector.  We draw $x\imin$ from $\normal(\mu\imin, \Gamma\imin)$ where $\mu\imin$ and $\Gamma\imin$ is the conditional mean and variance of $x\imin$ given $x\iplus$.
Here $\Gamma\imin$ is also the Schur complement. We draw $x\imin$ from $\normal(\mu\imin, \Gamma\imin)$.
The samples $x\iplus \in \R^{d_i}$ comprise the matrices $X\iplus\in \R^{n \times d_i}$. For all $i\in [m]$, we will assume we have an infinite sequence (w.r.t. $n$) of matrices $X\iplus$.
This partially-fixed-design scheme gives the estimators knowledge of the observed features and the distribution of the unobserved features, which is consistent with knowledge that \textsc{Collab} has access to. In this section we fix $\param \in \R^d$.
The corresponding label $y = x\iplus \param\iplus + x\imin \param\imin + \noise$, where $\noise \in \R$ is drawn from i.i.d.\ $\normal(0, \sigma^2 )$. We use $y_j \in \R^n$ to denote its vector form for the agent $j$.
To model the estimator's knowledge about the labels, we will have two observation models---one weaker and one stronger---which we will specify later  when we present our results.
 
For each observation model, we will have two types of results. The first type of result is a minimax lower bound for full-featured data; i.e., how well can estimator perform on a fresh sample without missing features. 
This type of result will concern the full-feature asymptotic local minimax risk
\begin{align*}
    \liminf_{n \to \infty}\minimax_{m, \varepsilon}(\{X\iplus\}_{i\in[m]};\statfamily_n, u) \defeq \liminf_{n \to \infty} \inf_{\bar{\param}} \sup_{\statdist \in \statfamily_n} n
\E_{Z \sim \statdist}  \<u, \bar{\param}(Z, \{X\iplus\}_{i\in[m]}) - \param\>^2.
\end{align*}
We will show that there exists a $B \in \R^{d \times d}$ such that the local minimax risk in the previous display is lower bounded by $u^T B u$ for all $u\in\R^d$. In other words, we have lower bounded the asymptotic covariance of our estimator with $B$ (with respect to the p.s.d.\ cone order). 
The second type of result is an agent specific minimax lower bound; i.e., what is the best prediction error an estimator (for the given observation model) can possibly have on a fresh sample for a given agent.
This type of result will deal with the missing-feature asymptotic local minimax risk, defined as
\begin{align*}
    \liminf_{n \to \infty}\minimax_{m, \varepsilon}^{i+}(\{X\iplus\}_{i\in[m]}; \statfamily_n, u) \defeq \liminf_{n \to \infty}\inf_{\bar{\param}} \sup_{\statdist \in \statfamily_n}
    n \E_{Z \sim \statdist}  \<u, \bar{\param}(Z, \{X\iplus\}_{i\in[m]}) - T_i \param\>^2.
\end{align*}
Similar to the first minimax error definition, we will again show that there exists a $B_i \in \R^{d_i \times d_i}$ such that the local minimax risk we just defined is lower bounded by $u^T B_i u$ for all $u\in\R^{d_i}$.
Recall \eqref{eqn:local-test-loss} for discussion surrounding why $T_i \param$ 
is the right object to compare against. 




\subsection{Weak Observation Model: Access only to local models and features} \label{sec:weak-observation}
Recall the local least squares estimator 
$\est{\param}_i = (X\iplus^\top X\iplus)^{-1} X\iplus^\top y_i$.
Let $P_{\param}^{\est{\param}}$ be a distribution over $\est{\param}_1, \ldots , \est{\param}_m$ induced by $\param$ and $(\noise_1, \ldots, \noise_m) \simiid \normal(0, \sigma^2 I_n)$.
We define the following family of distributions
    $\mc{P}^{\est{\param}}_{n,c} \defeq \{P_{\param'}^{\est{\param}} : \ltwo{\param' - \param} \leq c n^{- 1/2}\}$ 
which defines our observation model. Intuitively, in this observation model, we are constructing a lower bound for estimators which have access to the features $X_{1+}, \ldots, X_{m+}$, the population covariance $\scov$, and access to $\est{\param}_1, \ldots, \est{\param}_m$. In comparison, our estimator \textsc{Collab} only uses $\scov$ and $\est{\param}_1, \ldots \est{\param}_m$.
We present our first asymptotic local minimax lower bound result here. The proof of this result can be found in \Cref{sec:proof-weak-global-lb}.



\begin{theorem}\label{thm:weak-global-lb}
    Recall that $C\gauss \defeq (\sum_{i=1}^m T_i^\top  W_i\gauss   T_i)^{-1}$.
    For all $\in [m]$ and $n$ let the rows of $X\iplus$ be drawn i.i.d.\ from $\normal(0, \scov\iplus)$. Then for all $u\in\R^d$, with probability 1, the full-feature asymptotic local minimax risk for $\mc{P}^{\est{\param}}_{n,c}$ is bounded below as,
    \begin{align*}
        \liminf_{c \to \infty} \liminf_{n \to \infty}\minimax_{m, \varepsilon}(\{X\iplus
        \}_{i\in[m]};\mc{P}^{\est{\param}}_{n,c}, u)  \geq u^\top C\gauss u.
    \end{align*}
    For all $u\in\R^{\di}$, with probability 1, the missing-feature asymptotic local minimax risk for $\mc{P}^{\est{\param}}_{n,c}$ is bounded below as
    \begin{align*}
        \liminf_{c \to \infty} \liminf_{n \to \infty}\minimax_{m, \varepsilon}^{i+}(\{X\iplus
        \}_{i\in[m]};\mc{P}^{\est{\param}}_{n,c}, u) \geq u^\top T_i C\gauss T_i^\top u.
    \end{align*}
\end{theorem}
This exactly matches the upper bound for \textsc{Collab} we presented in \Cref{cor:upper-bound-local}.




\subsection{Strong Observation Model: Access to features and labels}


Define the family of distributions 
    $\mc{P}^y_{n,c} \defeq \{P_{\param'}^y : \ltwo{\param' - \param} \leq c n^{- 1/2}\}$ as the observation model.
Intuitively, in this model, we are constructing a lower bound for estimators having access to all of the features $X_{1+}, \ldots, X_{m+}$ and access to $y_1, \ldots y_m$. This observation model is stronger than the previous observation model because estimators now have access to the labels $y$. We note again that our estimator \textsc{Collab} only uses $\scov$ and $\est{\param}_1, \ldots \est{\param}_m$. The quantities our estimator rely on do not scale with $n$, making our estimator much weaker than other potential estimators in this observation model, as estimators are allowed to depend on $y_i$, which grows in size with $n$.
We present our second asymptotic local minimax lower bound result here, starting with defining the strong local lower bound matrix $
	C\strong := (\sum_{i=1}^m 2\scov/(\norm{\param\imin}_{\Gamma\imin}^2 + \sigma^2))\inv$. 
The proof of this result is in \Cref{sec:proof-strong-global-lb}.
\begin{theorem}\label{thm:strong-global-lb}
    For all $i\in [m]$ and $n$ let the rows of $X\iplus$ be drawn i.i.d.\ from $\normal(0, \scov\iplus)$. Then for all $u\in\R^d$, with probability 1, the full-feature asymptotic local minimax risk for $\mc{P}^{y}_{n,c}$ is bounded below as
    \begin{align*}
        \liminf_{c \to \infty} \liminf_{n \to \infty}\minimax_{m, \varepsilon}(\{X\iplus\}_{i\in[m]};\mc{P}^{y}_{n,c}, u)  \geq u^\top C\strong u .
    \end{align*}
    For all $u\in\R^{\di}$, with probability 1, the missing-feature asymptotic local minimax risk for $\mc{P}^{y}_{n,c}$ is bounded below as
    \begin{align*}
        \liminf_{c \to \infty} \liminf_{n \to \infty}\minimax_{m, \varepsilon}^{i+}(\{X\iplus\}_{i\in[m]};\mc{P}^{y}_{n,c}, u) \geq u^\top T_i C\strong T_i^\top u.
    \end{align*}
\end{theorem}

In view of the lower bound in the strong observation model and that of the weak observation model in Theorem~\ref{thm:weak-global-lb}, it is clear that the lower bound in the strong observation setting is in general smaller as 
\begin{align*}
	\scov - T_i^\top \Sigma\iplus T_i = \Pi_i^\top \begin{bmatrix}
		0 & 0\\
		0 & \Gamma\imin
	\end{bmatrix} \Pi_i \succeq 0,
\end{align*}
which further implies $C\gauss \succeq (\sum_{i=1}^m \scov/(\norm{\param\imin}_{\Gamma\imin}^2 + \sigma^2) )\inv \succeq C\strong$.

We argue that the two lower bounds are comparable in the missing completely at random \cite{Little2019StatisticalAW}. Consider for every agent $i$, each coordinate is missing independently with probability $p$. In this case, $(d_i, \scov\iplus, T_i)$ are i.i.d.\ random triplets parameterized by $p$.

\begin{corollary} \label{cor:lower-bounds-comparable}
	Under the random missingness setup with missing probability $p$, let the eigenvalue of $\Sigma$ be $\lambda_1(\Sigma) \geq \cdots \geq \lambda_d(\Sigma) > 0$ and define its condition number $\kappa = \lambda_1(\Sigma) / \lambda_d(\Sigma)$. Suppose $p \leq \half\kappa^{-1} (1 + \|\param\|_\scov^2 / \sigma^2)^{-1}$, we have the limits $\lim_{m \to \infty} mC\gauss$ and $\lim_{m \to \infty} mC\strong$ exist and
	\begin{align*}
		4\lim_{m \to \infty} mC\strong \succeq \lim_{m \to \infty} m C\gauss \succeq \lim_{m \to \infty} mC\strong.
	\end{align*}
\end{corollary}


\section{Experiments}
% \haizhou{Follow the same way of introduction as we did in Section2.}
% \noindent In this section, we will introduce datasets and experimental setups that we used. Then we evaluate our method, other self-supervised methods, and supervised methods under different distribution shifts (\ie, concept shifts and covariate shifts) under common settings (\ie, transductive, inductive settings). It has to note that we focus on node-level tasks (\eg, node classification) in this work. As for graph-level tasks, we leave it as our future work and some simple experiments can be found in Appendix~\ref{app:graph_classification}. 
In this section, we first introduce the experimental setup including datasets, training, and evaluation protocol in Section~\ref{sec:dataset}~and~\ref{sec:unsupervised}. 
% Next, we present our experimental setup and conduct extensive experiments to evaluate our method in Section~\ref{sec:unsupervised}. 
We then perform an ablation study to demonstrate the effectiveness of each proposed component in Section~\ref{sec:ablation}. 
Additionally, we analyze the impact of important hyper-parameters in Section~\ref{sec:sensitivity}. 
Subsequently, we integrate our method with various encoding models, showcasing the model-agnostic nature of our recipe in Section~\ref{sec:other_models}. 
Finally, we provide some qualitative results such as feature visualization in Section~\ref{sec:vis}.
It is important to note that we focus on node-level tasks (\eg, node classification) in this work. As for graph-level tasks, we leave it as our future work, while some simple experiments are also provided in Appendix~\ref{app:graph_classification}.

\subsection{Datasets}\label{sec:dataset}
There exist some benchmarks for evaluating graph out-of-distribution generalization~\cite{good,ji2022drugood,gds}. 
Among them, GOOD~\cite{good} is the most representative and comprehensive benchmark that curates more diverse graph datasets with diverse tasks, including single/multi-task graph classification, graph regression, and node classification involving more distribution shifts (\ie, concept shifts and covariate shifts). Hence in this work, we follow the evaluation protocol proposed in \cite{good}. Furthermore, we validate the effectiveness of our method in the datasets (\ie, Amazon-Photo, Elliptic) that are used in EERM~\cite{eerm}. The statistics and detailed introduction to these datasets can be found in Table~\ref{tab:dataset} and Appendix~\ref{app:datasets}.

\begin{table*}[htp]
\caption{The descriptions of datasets. ``Domain-Level'' means splitting by graphs, ``Time-Aware'' denotes splitting according to chronological order.``Word'' and ``Degree'' represent splitting according to word diversity and node degree respectively. ``Language'' means splitting by user language, suggesting the prediction should not be impacted by the language the user use. ``University'' denotes splitting according to the domain university, implying that the prediction of webpages should be based on word contents and link connections rather than university features. ``Color'' means that nodes are split according to node differences in covariate shift and color-label correlations in concept shift.}
\label{tab:dataset}
\centering
\begin{tabular}{cccccccc}
\toprule
Datasets     & Network Type        & \#Nodes & \#Edges & \#Attributes &\#Classes& Train/Val/Test Split     & Metric   \\
% Cora         & Artificial Transformation & 2,703   &         &              &         &                      & Accuracy \\
Amazon-Photo\footnotemark
             & Co-purchasing network      & 7,650   & 119,081   & 755          & 10      & Domain-Level         & Accuracy \\
Elliptic\footnotemark  
             & Bitcoin transactions       & 203,769 & 234,355   & 165          & 2       & Time-Aware           & F1-Score \\
GOOD-Cora    & Scientific publications    & 19,793  & 126,842   & 8,710         & 70      & Word/Degree          & Accuracy \\
% GOOD-Arxiv   & arXiv papers               & 169,343 & 2,315,598 & 128          & 40      & Time/Degree          & Accuracy \\
GOOD-Twitch  & Gamer network              & 34,120  & 892,346   & 128          & 2       & Language             & ROC-AUC  \\
GOOD-CBAS    & A BA-house graph           & 700     & 3,962     & 4             & 4       & Color                & Accuracy \\
GOOD-WebKB   & Webpage network            & 617     & 1,138     & 1,703         & 5       & University           & Accuracy \\
\bottomrule
\end{tabular}
\end{table*}
\footnotetext[5]{This dataset is adopted from~\cite{yang2016revisiting}. \cite{eerm} constructs ten graphs with different environment id’s for each graph.} 
\footnotetext[6]{The original is available on \hyperlink{https://www.kaggle.com/ellipticco/elliptic-data-set}{https://www.kaggle.com/ellipticco/elliptic-data-set}}

\subsection{Unsupervised Representation Learning}\label{sec:unsupervised}
\subsubsection{Transductive Setting}~\label{sec:trans}
% \noindent\textbf{Baselines.}\quad We conduct experiments with 12 baselines which consist of three categories: supervised methods and self-supervised generative methods, self-supervised contrastive methods. Specifically, we compare with three supervised baselines: empirical risk minimization~(ERM)~\cite{erm}, invariant risk minimization (IRM)~\cite{irm}, and a recent proposed graph OOD method dubbed EERM~\cite{eerm}. We also compare various unsupervised node-level representation learning methods: three self-supervised generative methods including GAE~\cite{gae}, VGAE~\cite{gae}, GraphMAE~\cite{gmae} and seven self-supervised contrastive methods: DGI~\cite{dgi}, MVGRL~\cite{mvgrl}, GRACE~\cite{grace}, RoSA~\cite{rosa}, BGRL~\cite{bgrl}, COSTA~\cite{costa}, SwAV~\cite{swav}. The descriptions of these methods can be found in Appendix~\ref{app:baselines}.
In this subsection, we focus on validating our proposed algorithm under the transductive setting, where the test nodes will participate in message passing~\cite{gilmer2017neural} during training following~\cite{good}. 

\noindent\textbf{Baselines.} We conduct experiments with 12 baselines from three categories: (i)~supervised methods, including empirical risk minimization~(\textbf{ERM})~\cite{erm}, invariant risk minimization (\textbf{IRM})~\cite{irm}, and a recent proposed graph OOD method \textbf{EERM}~\cite{eerm}; (ii)~self-supervised generative methods including Graph Autoencoder (\textbf{GAE})~\cite{gae}, Variational Graph Autoencoder (\textbf{VGAE})~\cite{gae}, Self-Supervised Masked Graph Autoencoders (\textbf{GraphMAE})~\cite{gmae}; (iii)~self-supervised contrastive methods including Deep Graph Infomax (\textbf{DGI})~\cite{dgi}, Contrastive Multi-View Representation Learning on Graphs (\textbf{MVGRL})~\cite{mvgrl}, Deep Graph Contrastive Representation Learning (\textbf{GRACE})~\cite{grace}, A Robust Self-Aligned Framework for Node-Node Graph Contrastive Learning (\textbf{RoSA})~\cite{rosa}, Bootstrapped Representation Learning on Graphs (\textbf{BGRL})~\cite{bgrl}, Covariance-Preserving Feature Augmentation for Graph Contrastive Learning (\textbf{COSTA})~\cite{costa}, Unsupervised Learning of Visual Features by Contrasting Cluster Assignments (\textbf{SwAV})~\cite{swav}. The detailed descriptions of these baselines can be found in Appendix~\ref{app:baselines}.

\noindent\textbf{Experimental setup.} We use the same graph encoder across different datasets for a fair comparison following~\cite{good}. We use grid search to find other hyper-parameters (\eg, learning rate, epochs) for different methods. For all experiments, we select the best checkpoints for ID and OOD tests according to results on ID and OOD validation sets following~\cite{good}, respectively. Experimental details and hyper-parameter selections are provided in Appendix~\ref{app:hyper}. For evaluating unsupervised methods, a linear classifier will be built on the frozen trained encoder after finishing pre-training. The reported results are the mean performance with standard deviation after 10 runs following~\cite{good}.

\noindent\textbf{Analysis.}\quad Based on the experimental results listed in Table~\ref{tab:trans_concept} and \ref{tab:trans_covariate}, we can draw the following conclusions: firstly, we find strong self-supervised methods (\eg, GRACE, BGRL, COSTA) are more robust to distribution shifts (concept shift in Table~\ref{tab:trans_concept} and covariate shift in Table~\ref{tab:trans_covariate}) compared to supervised methods. For instance, on GOOD-CBAS and GOOD-WebKB datasets, GRACE surpasses the best supervised method by large margins (over 6\% absolute improvement). Interestingly, we find the methods designed for OOD generalization (\ie, IRM) and graph OOD generalization (\ie, EERM) do not attain superior performance than the standard ERM on most of the datasets. For example, EERM shows superior OOD performance compared to ERM in only one experiment, and IRM outperforms ERM in four out of ten experiments across the conducted evaluations. This phenomenon is also observed in \cite{good,ahuja2020empirical,rosenfeld2021risks}, showcasing the challenge of achieving invariant prediction in non-Euclidean graph settings. 

Furthermore, our method surpasses other SOTA self-supervised methods on the OOD test set of all datasets by a considerable margin while achieving comparable performance in the in-distribution test set. For instance, on small datasets such as GOOD-CBAS and GOOD-WebKB, our method outperforms GRACE\footnote{MARIO is built up on GRACE according to our recipe. So, we make a comparison with GRACE here.} by over 2\% absolute accuracy on the OOD test set. On larger datasets such as GOOD-Cora and GOOD-Twitch, our method still outperforms other methods which shows its superiority. For instance, under covariate shift, MARIO surpasses other methods by over 7\% absolute accuracy on the GOOD-Twitch OOD test set. These statistics prove the effectiveness of our design.


\begin{table*}[htp]
\caption{Experimental results of all methods under concept shift. The bold font means the top-1 performance and the underline represents the second performance across the unsupervised methods. 'ID' represents in-distribution test performance and 'OOD' means out-of-distribution test performance. (OOM: out-of-memory on a GPU with 24GB memory)}
\label{tab:trans_concept}
\centering
\scalebox{0.95}{
\begin{tabular}{l|cc|cc|cc|cc|cc}
\toprule
\toprule
\multirow{3}{*}{concept shift} & \multicolumn{4}{c|}{GOOD-Cora}                   & \multicolumn{2}{c|}{GOOD-CBAS} & \multicolumn{2}{c|}{GOOD-Twitch} & \multicolumn{2}{c}{GOOD-WebKB} \\
                           & \multicolumn{2}{c}{word} & \multicolumn{2}{c|}{degree}& \multicolumn{2}{c|}{color}    & \multicolumn{2}{c|}{language}   & \multicolumn{2}{c}{university} \\
                           & ID         & OOD         & ID          & OOD          & ID            & OOD           & ID             & OOD            & ID            & OOD            \\
\midrule
ERM                        & 66.38±0.45 & 64.44±0.18  & 68.60±0.40  & 60.76±0.34   & 89.79±1.39    & 83.43±1.19    & 80.80±1.00     & 56.92±0.92     & 62.67±1.53    & 26.33±1.09     \\
IRM                        & 66.42±0.41 & 64.29±0.31  & 68.57±0.35  & 61.45±0.24   & 89.64±1.21    & 82.29±1.14    & 78.87±1.04     & 59.30±1.79     & 62.67±1.10    & 26.88±1.42     \\
EERM                       & 65.10±0.44 & 62.45±0.19  & 66.95±0.44  & 56.58±0.25   & 79.07±2.12    & 64.50±1.01    & OOM            & OOM            & 62.50±2.01    & 28.07±3.23      \\
\midrule
% Random-Init                & 37.53±1.74 & 32.12±1.24  & 37.82±1.71  & 27.74±1.14   &               &               &                &                & 60.33±2.21    & 27.07±1.70     \\
GAE                        & 60.65±0.89 & 58.00±0.55  & 62.59±1.11  & 53.44±0.80   & 75.28±1.36    & 68.07±2.05    & 81.25±0.81     & 51.51±1.05     & 62.17±3.34    & 25.78±1.85     \\
VGAE                       & 63.19±0.53 & 60.35±0.47  & 61.65±0.66  & 54.28±0.28   & 76.50±0.50    & 59.07±0.56    & 80.46±0.53     & 55.56±4.53     & 62.50±2.38    & 24.40±2.57     \\
GraphMAE                   & \underline{66.44±0.46} & \underline{64.87±0.30}  & 67.95±0.46  & 59.41±0.39   & 89.14±0.89    & 82.93±0.93    & 80.05±0.64     & 59.38±1.49     & 61.83±3.37    & 29.27±2.15     \\
DGI                        & 63.33±0.56 & 60.71±0.49  & 65.93±1.02  & 55.83±0.53   & 91.22±1.47    & 85.00±1.66    & 80.05±0.87     & 59.16±1.88     & 61.83±2.83    & 28.63±1.92      \\
MVGRL                      & OOM        & OOM         & OOM         & OOM          & 88.57±1.15    & 76.50±1.17    & OOM            & OOM            & 62.00±3.79    & 28.26±4.20     \\
GRACE                      & 65.61±0.61 & 63.92±0.44  & \textbf{68.59±0.35}  & 60.15±0.45   & 92.00±1.39    & 88.64±0.67    & \textbf{83.43±0.63}     & \underline{60.45±1.46}     & 64.00±3.43    & \underline{34.86±3.43}  \\
RoSA                       & 64.06±0.67 & 62.44±0.39  & 67.07±0.65  & 57.68±0.44   & 90.78±2.27    & 85.93±2.14    & 82.39±0.42     & 57.45±2.16     & 64.17±4.10    & 32.20±2.15     \\
BGRL                       & 65.18±0.43 & 63.43±0.45  & 66.83±0.80  & 59.63±0.38   & 92.36±1.16    & 87.14±1.60    & 82.52±0.60     & 55.48±1.48     & 63.67±2.33    & 31.47±3.43     \\
COSTA                      & 65.05±0.80 & 62.37±0.45  & 66.76±0.87  & 55.73±0.36   & \underline{93.50±2.62}    & \underline{89.29±3.11}    & 83.15±0.30 & 55.03±3.22     & 61.66±2.58    & 32.39±2.13 \\
% ArCL                       &            &             & 67.64±0.57  & 59.71±0.44   &               &               &                &                & 65.00±3.94    & 35.41±1.97 \\      
SwAV                       & 62.22±0.53 & 59.79±0.53  & 64.65±0.94  & 55.06±0.39   & 89.00±0.79    & 81.72±0.66    & \underline{83.32±0.15}     & 59.69±1.97     & \underline{65.17±3.76}    & 29.36±2.01    \\
\midrule
MARIO                       & \textbf{67.11±0.46} & \textbf{65.28±0.34}  & \underline{68.46±0.40}  & \textbf{61.30±0.28}   & \textbf{94.36±1.21}    & \textbf{91.28±1.10}    & 82.31±0.54     & \textbf{63.33±1.72}     & \textbf{65.67±2.81}    & \textbf{37.15±2.37}     \\
\bottomrule
\end{tabular}}
\end{table*}

\begin{table*}[htp]
\caption{Experimental results of all methods under covariate shift. The bold font means the top-1 performance and the underline represents the second performance across the unsupervised methods. 'ID' represents in-distribution test performance and 'OOD' means out-of-distribution test performance. (OOM: out-of-memory on a GPU with 24GB memory)}
\label{tab:trans_covariate}
\centering
\scalebox{0.95}{
\begin{tabular}{l|cc|cc|cc|cc|cc}
\toprule
\toprule
\multirow{3}{*}{covariate shift} & \multicolumn{4}{c|}{GOOD-Cora}                                   & \multicolumn{2}{c|}{GOOD-CBAS} & \multicolumn{2}{c|}{GOOD-Twitch} & \multicolumn{2}{c}{GOOD-WebKB} \\
                           & \multicolumn{2}{c}{word} & \multicolumn{2}{c|}{degree}& \multicolumn{2}{c|}{color}    & \multicolumn{2}{c|}{language}   & \multicolumn{2}{c}{university} \\
                           & ID         & OOD         & ID          & OOD          & ID            & OOD           & ID             & OOD            & ID            & OOD            \\
\midrule
ERM                        & 70.50±0.41 & 64.69±0.33  & 72.46±0.49  & 55.53±0.50   & 92.00±3.08    & 77.57±1.29    & 70.98±0.41     & 49.35±5.09     & 39.34±1.79    & 14.52±3.14   \\
IRM                        & 70.48±0.26 & 64.53±0.57  & 71.98±0.34  & 53.72±0.46   & 90.86±2.41    & 78.86±1.67    & 69.81±0.95     & 49.11±2.82     & 38.52±3.30    & 13.97±2.80     \\
EERM                       & OOM        & OOM         & OOM         & OOM          & 65.00±2.57    & 57.43±3.60    & OOM            & OOM            & 46.07±4.55    & 27.40±7.65     \\
\midrule
GAE                        & 56.63±0.79 & 48.93±0.93  & 66.30±0.88  & 34.01±0.87   & 73.00±2.16    & 60.86±3.01    & 67.24±1.23     & 47.65±2.49     & 45.08±6.32    & 28.02±6.29    \\
VGAE                       & 62.02±0.66 & 54.12±0.86  & 69.41±0.57  & 44.20±1.29   & 62.29±2.04    & 63.29±1.11    & 66.99±1.43     & \underline{50.48±4.58}     & 48.85±4.68    & 20.87±6.69     \\
GraphMAE                   & 68.14±0.43 & 64.00±0.33  & \textbf{73.36±0.56}  & 53.75±0.55   & 67.28±3.03    & 67.28±1.49    & 68.84±1.20     & 48.02±2.79     & 48.03±4.34    & 30.00±8.09     \\
DGI                        & 60.85±0.75 & 57.03±0.67  & 68.97±0.41  & 41.75±0.88   & 69.57±4.09    & 59.71±3.43    & 68.43±1.05     & 44.83±1.61     & 48.52±5.04    & 21.11±7.50     \\
MVGRL                      & OOM        & OOM         & OOM         & OOM          & 65.00±1.94    & 64.15±0.77    & OOM            & OOM           & \textbf{54.10±5.39}    & 16.59±6.51     \\
GRACE                      & \underline{68.77±0.33} & \underline{64.21±0.41}  & 72.69±0.34  & \underline{56.10±0.63}   & \underline{93.57±1.83}    & \underline{89.29±3.40}    & \underline{71.12±0.87} & 46.21±1.54 & 49.67±5.82    & 28.10±4.68    \\
RoSA                       & 68.19±0.56 & 62.48±0.61  & 71.04±0.62  & 52.72±0.79   & 84.71±4.14    &79.14±3.51     & 70.58±0.36     & 45.83±1.72     & 52.30±4.24    & \underline{34.24±7.92}     \\
BGRL                       & 67.23±0.43 & 61.33±0.36  & 72.11±0.39  & 49.15±0.73   & 89.00±2.56    & 79.86±3.29    & \textbf{71.43±0.53}     & 43.86±0.94     & 51.80±5.55    & 30.32±7.61    \\
COSTA                      & 65.28±0.60 & 60.33±0.53  & 70.65±0.62  & 54.03±0.28   & 92.29±1.59    & 82.71±2.74    & 69.29±1.37     & 49.07±2.13     & 50.49±3.01    & 29.84±4.75   \\
SwAV                       & 63.29±1.01 & 56.98±0.94  & 70.27±0.73  & 43.00±0.52   & 89.57±1.12    & 81.43±1.69    & 69.19±0.93     & 49.37±2.96     & 49.84±4.82    & 30.55±6.72   \\
\midrule
MARIO                       & \textbf{69.99±0.54} & \textbf{65.06±0.34}  & \underline{72.73±0.43}  & \textbf{57.73±0.45}  & \textbf{94.57±2.46}    & \textbf{91.00±2.48}     & 68.31±0.78 & \textbf{57.37±1.37}     & \underline{53.94±3.23}    & \textbf{35.24±4.98}   \\
\bottomrule
\end{tabular}}

\end{table*}

\subsubsection{Inductive Setting}
In this subsection, we conduct experiments under the inductive settings, where the test nodes are kept unseen during training. This setting is more suitable for domain generalization.
% But we think it is more convincing that conduct experiments under inductive settings which means test nodes are unseen during training. This setting is more appropriate for domain generalization.

\noindent\textbf{Baselines:} For GOOD-WebKB and GOOD-CBAS datasets, we adopt ERM, IRM, GraphMAE, and GRACE as our baselines. And for Amazon-Photo and Elliptic datasets, we select ERM, EERM, and GRACE as our baselines.

\noindent\textbf{Experimental setup:} For GOOD-WebKB and GOOD-CBAS datasets, we use the same model configuration in Section~\ref{sec:trans}.
% Besides, we add experiments on Amazon-Photo dataset~\cite{yang2016revisiting} and Elliptic~\cite{elliptic} dataset in this subsection. 
For Amazon-Photo dataset~\cite{yang2016revisiting} and Elliptic~\cite{elliptic} dataset, they consist of many snapshots (training data and testing data use different snapshots) which are naturally inductive. For Amazon-Photo dataset, we use 2-layer GCN~\cite{gcn} as the encoder and for elliptic dataset, we use 5-layer GraphSAGE~\cite{sage} as encoder following~\cite{eerm}.

% Figure environment removed

\noindent\textbf{Analysis:}
According to Figure~\ref{fig:amazon},\ref{fig:elliptic},\ref{fig:ind_con},\ref{fig:ind_cov}, we can draw following conclusions:
firstly, based on Figure~\ref{fig:amazon}, it is evident that our method outperforms other representative supervised and self-supervised methods on all test graphs (T1$\sim$T8). This superiority is reflected in the larger median value of our method compared to others. For instance, MARIO achieves over a 3\% absolute improvement compared to ERM in terms of the mean value of eight median values. Additionally, our method demonstrates higher stability across different random initializations, as indicated by the closer proximity of the first and third quartile values to the median value~(\eg, the difference of first and third quartile values of ERM, EERM, GRACE and MARIO are 4.2, 3.3, 6.7 and 1.0 on T8 respectively which indicates MARIO is much more stable than other methods). Furthermore, our method exhibits consistent performance across different graphs (\eg, The standard deviation of median values on T1$\sim$T8 for ERM, EERM, GRACE, and MARIO are 0.4, 1.1, 1.2, and 0.3, respectively.), indicating its robustness to environmental variations and its ability to extract invariant features: $g(G^e) \approx g(G^{e'})$ for all $e, e' \in \mathcal{E}^\text{train}$. In summary, our method showcases enhanced OOD generalization capabilities.
% $g(G^e)g(G^e^\prime)$ where $any e, e^\prime in \mathcal{E}^{train}$

Secondly, from the results presented in Figure~\ref{fig:elliptic}, we can observe that our method averagely harvests 10.9\% absolute improvement over GRACE and 12.5\% absolute improvement over EERM in terms of F1 scores on Elliptic dataset. This demonstrates the effectiveness of our method in handling distribution shifts and improving performance compared to existing approaches. It is worth noting that GRACE's performance worsens over time, indicating its inability to handle distribution shifts effectively. In contrast, our method consistently achieves better F1 scores, except for T9, which is caused by the dark market shutdown occurred after T7~\cite{elliptic}. The emergence of such an event introduces significant variations in data distributions, which subsequently results in performance degradation for all methods. Indeed, this event serves as an unpredictable external factor that introduces significant challenges for models trained on limited training data. The results indicate that the performance heavily depends on available training data. Nonetheless, our approach outperforms other methods even in such an extreme case. This highlights the effectiveness of our method in addressing distribution shifts and improving generalization performance.

Finally, based on the observations from Figure~\ref{fig:ind_con} and Figure~\ref{fig:ind_cov} MARIO demonstrates the best performances on both ID and OOD test sets for GOOD-WebKB and GOOD-CBAS datasets, under both concept shift and covariate shift. Notably, MARIO outperforms other methods by more than 3\% and 10\% absolute improvement on GOOD-WebKB and GOOD-CBAS, respectively, under covariate shift. We can draw similar conclusions as discussed in Section~\ref{sec:trans}. Even under the inductive setting, our method continues to demonstrate excellent OOD generalization capabilities and achieves comparable or even improved in-distribution test performance. These statistical results further validate the effectiveness of our method in handling distribution shifts and enhancing generalization performance.

Overall, the observations we have made provide strong evidence of the great capacity of our method for handling distribution shifts, validating its effectiveness and potential for real-world applications.



% Figure environment removed

% Figure environment removed


% Figure environment removed


\subsection{Ablation Studies}\label{sec:ablation}
\noindent Table~\ref{tab:aba} provides a detailed analysis of the effect of each component according to our proposed recipe for improving OOD generalization in graph contrastive learning. Let's examine the different variants of our method and their impact on performance.
Specifically, MARIO~(w/o ad) represents MARIO without  adversarial augmentation. MARIO~(w/o cmi) denotes we only maximize the mutual information between positive pairs without considering conditional mutual information. MARIO~(w/o cmi, ad) means a vanilla graph contrastive method that is similar to GRACE. 

From Table~\ref{tab:aba}, we can find MARIO~(w/o cmi) lags far behind MARIO on OOD test set which demonstrates appropriately minimizing the redundant information (\ie, conditional mutual information) is essential to improve OOD generalization of GCL methods. And adversarial augmentation can also boost OOD generalization because it can approximately serve as a supermum operator to learn more invariant features  discussed in Section~\ref{sec:aug}. Based on the analysis of these variants, it is evident that the proposed improvements on data augmentation and contrastive loss in the recipe are both effective in enhancing graph OOD generalization. Each component contributes to the overall performance improvement, and their combination leads to a stronger self-supervised graph learner in terms of graph OOD generalization. 

In short, the findings from Table~\ref{tab:aba} support the rationale behind your proposed recipe and provide empirical evidence of the effectiveness of each proposed component. By incorporating these enhancements, our method achieves superior performance in handling distribution shifts and improving graph OOD generalization in graph contrastive learning.
\begin{table*}[htp]
\caption{Ablation studies for MARIO by masking each component.}
\label{tab:aba}
\centering
\scalebox{0.9}{
\begin{tabular}{l|cc|cc|cc|cc|cc}
\toprule
\toprule
\multirow{3}{*}{concept shift} & \multicolumn{4}{c|}{GOOD-Cora}                       & \multicolumn{2}{c|}{GOOD-CBAS} & \multicolumn{2}{c|}{GOOD-Twitch} & \multicolumn{2}{c}{GOOD-WebKB} \\
                           & \multicolumn{2}{c}{word} & \multicolumn{2}{c|}{degree}& \multicolumn{2}{c|}{color}    & \multicolumn{2}{c|}{language}   & \multicolumn{2}{c}{university} \\
                           & ID         & OOD         & ID          & OOD          & ID            & OOD           & ID             & OOD            & ID            & OOD            \\
\midrule
MARIO                      & \textbf{67.11±0.46} & \textbf{65.28±0.34}  & \textbf{68.46±0.40}  & \textbf{61.30±0.28}      & \textbf{94.36±1.21}  & \textbf{91.28±1.10}    & 82.31±0.54     & \textbf{63.33±1.72}     & \textbf{65.67±2.81}    & \textbf{37.15±2.37}     \\
MARIO(w/o ad)              & 66.23±0.53 & 64.02±0.18  & 67.88±0.38  & 60.46±0.29   & 93.21±1.25    & 90.29±0.91    & 82.42±0.73     & 60.50±1.02     & 64.83±2.83    & 36.51±3.25    \\
MARIO(w/o cmi)             & 65.32±0.60 & 63.51±0.32  & 68.14±0.32  & 61.19±0.34   & 94.15±1.23    & 90.57±1.96    & \textbf{82.51±0.56}     & 61.41±2.63     & 64.50±4.35    & 35.78±2.53     \\
MARIO(w/o cmi, ad)         & 64.67±0.55 & 63.11±0.32  & 67.95±0.65  & 60.01±0.57   & 93.36±1.66    & 89.64±1.73    & 81.90±0.75     & 60.12±1.60     & 64.17±3.67    & 34.13±2.38     \\
\bottomrule
\end{tabular}}
\end{table*}
% & 65.32±0.60 & 63.51±0.32 exchange 64.67±0.55 & 63.11±0.32
% 68.14±0.32       id ood test: 60.95±0.43       ood ood test: 61.19±0.34


\subsection{Sensitivity Analysis}\label{sec:sensitivity}
\noindent In this subsection, we will analyze some important hyper-parameters of our method. We conduct sensitivity analysis on GOOD-WebKB dataset with concept shift, we chose two sensitive hyper-parameters (\ie, the coefficient $\gamma$ of condition mutual information in Equation~\ref{equ:cmi} and the number of prototypes $|C|$ in Equation~\ref{equ:pq}). The coefficient of CMI range in $[0.001, 0.01, 0.1, 0.5, 1]$ and the number of prototypes $|C|$ ranges in $[10, 50, 100, 200, 300]$. From Figure~\ref{fig:sensitivity}, we can observe that $\gamma$ reaches 0.1 and $|C|$ reaches 100 or 200 can achieve the best OOD test accuracy. Both higher and lower values of $\gamma$ result in suboptimal performance. This finding aligns with previous research such as DIB~\cite{dib}, indicating that an appropriate compression level is crucial for achieving optimal performance. Extremely high or low compression values are not ideal. 

Regarding the number of prototypes $|C|$, based on the results shown in Figure~\ref{fig:sensitivity}, it is found that setting $|C|=100$ leads to the best performance in terms of OOD test accuracy. This choice provides a moderate number of pseudo labels, which is beneficial for the learning process. 

Based on the sensitivity analysis, we determined that setting $\gamma=0.1$ and $|C|=100$ on most datasets. These hyperparameter values strike a balance between compression level and the number of prototypes, resulting in improved graph OOD generalization.
% Figure environment removed


\subsection{Integrated with Other Models}\label{sec:other_models}
% Figure environment removed

\begin{table}[htp]
\caption{Results of different learning approaches with different encoding models (\ie, GCN, GraphSAGE, GAT).}
\label{tab:others}
\centering
\scalebox{0.9}{
\begin{tabular}{cc|cc|cc}
\toprule
\toprule
\multirow{3}{*}{Model}& \multirow{3}{*}{Method} & \multicolumn{2}{c|}{GOOD-CBAS} & \multicolumn{2}{c}{GOOD-WebKB} \\
                & & \multicolumn{2}{c|}{color}    & \multicolumn{2}{c}{university} \\
                &   & ID          & OOD         & ID          & OOD            \\
\midrule
\multirow{3}{*}{GCN} 
&ERM               & 89.79±1.39 & 83.43±1.19  &  62.67±1.53 & 26.33±1.09         \\
&GRACE             & 92.00±1.39 & 88.64±0.67  &  64.00±3.43 & 34.86±3.43        \\
&MARIO             & 94.36±1.21 & 91.28±1.10  &  65.67±2.81 & 37.15±2.37        \\ \bottomrule
\multirow{3}{*}{SAGE} 
&ERM               & 95.07±1.51 & 75.14±1.19  & 73.67±2.08  & 46.33±3.42       \\
&GRACE             & 95.29±1.11 & 74.43±2.36  & 70.50±5.06  & 49.54±3.83        \\
&MARIO             & 96.00±1.07 & 76.29±3.01  & 71.00±3.82  & 51.74±4.63        \\ \bottomrule
\multirow{3}{*}{GAT} 
&ERM               & 78.64±3.63 & 72.93±2.64  & 61.33±3.71  & 28.99±2.63        \\
&GRACE             & 84.57±1.79 & 78.36±1.60  & 59.50±2.36  & 35.78±3.26        \\
&MARIO             & 84.93±1.95 & 80.43±1.89  & 62.17±4.78  & 38.17±3.10        \\
\bottomrule
\end{tabular}}
\end{table}



\noindent In the subsection, we demonstrate the model-agnostic nature of the recipe by integrating it with various graph neural network (GNN) models, including GCN, GraphSAGE, and GAT.

From Table~\ref{tab:others}, it can be observed that regardless of the specific GNN model used as the encoder, our method consistently achieves the best performance on the OOD test set. This indicates the effectiveness and robustness of our method across different GNN models.
By achieving superior performance across different GNN models, MARIO demonstrates its versatility and ability to improve the OOD generalization of various graph neural models. This highlights the broad applicability and effectiveness of our recipe in enhancing the performance of different GNN encoders.

Furthermore, we integrate our recipe with other GCL methods in Appendix~\ref{app:other_methods}. The results demonstrate our recipe can boost the OOD generalization ability of various GCL methods which means our recipe can serve as a plug-in for many current classical GCL methods.

% Figure environment removed

\subsection{Visualization}\label{sec:vis}
\subsubsection{Metric Score Curves}
We present metric score curves for ERM and MARIO, including training, ID validation, ID testing, OOD validation, and OOD testing accuracy, in Figure~\ref{fig:curve2}. Notably, MARIO demonstrates superior convergence with approximately 10\% absolute improvement on the OOD test set compared to ERM. Furthermore, MARIO effectively narrows the performance gap between in-distribution and out-of-distribution performance, showcasing its efficacy in enhancing OOD generalization for graph data. More metric score curves can be found in Appendix~\ref{app:curves}.


\subsubsection{Feature Visualization}
In order to assess the quality of learned embeddings, we adopt t-SNE~\cite{tsne} to visualize the node embedding on GOOD-Cora dataset (concept shift in word domain) using random-init of GCN, EERM, GRACE, and MARIO, where different classes have different colors in Figure~\ref{fig:vis}. For clarity, we select eight classes with the largest number of nodes to enhance the informativeness and interpretability of the visualization. We can observe that the 2D projection of node embeddings learned by MARIO has a better separation of clusters, which indicates the model can help learn representative features for downstream tasks. It has to note that we depict both ID nodes and OOD nodes in the same figure. 

Besides, we also separately visualize ID nodes and OOD nodes in the different figures in the Appendix~\ref{app:feature}. And we can find MARIO performs a clearer separation of clusters whether on ID nodes or OOD nodes compared to other methods.



\section{Discussion}
\label{sec: discussion}
\kmsdelete{In this work} We study \kmsreplace{Fairness-Aware PAC learning}{Fair-ERM} in the malicious noise model, and  in some cases allow 
the learner to maintain optimal overall accuracy despite the signal in Group $B$ being almost entirely washed out.
%when we allow learners to use the
%$\PQ$ randomized expansion of the hypothesis class $\mathcal{H}$
In particular we show that different fairness constraints have fundamentally different behavior in the presence of Malicious Noise, in terms of the amount of accuracy loss that a given level of Malicious Noise could cause a fairness-constrained learner to incur. 
The key to achieving our results, which are more optimistic than those in \cite{lampert}, is allowing for improper learners using the (P,Q)-randomized expansions of the given class $\mathcal{H}$.
%We \kmsreplace{present a picture of the}{prove upper and lower bounds on}
%accuracy loss for a range of fairness notions, given \kmsreplace{this simple randomization step.}{learning over $\PQ$.
%In general our results indicate Fair-ERM (given learning over $\PQ$) is more robust than claimed in \cite{lampert}.
The type of smoothness we create by using $\PQ$ seems to be a natural property that is likely shared by many natural hypothesis classes.

Fairness notions are motivated as a response to learned disparities when there is \kmsdelete{data corruption or} systemic error affecting \kmsdelete{the data for}
one group. 
Fairness notions are supposed to mitigate this by ruling out classifiers that have worse performance on a sub-group. 
This can peg both classifiers at a lower level of performance \kmsdelete{(e.g that the lower subgroup)} in order to \emph{motivate} \cite{hardt16} improving the data collection or labelling process to obtain more reliable performance. 
%So in \kmsreplace{some}{a} sense, sensitivity of the fairness notion to poor sub-group performance caused by malicious noise is the \textit{point} of fairness constraints! 
However, it also desirable that fairness constraints perform gracefully when subject to Malicious Noise because fairness constraints will be used in contexts where the data is unreliable and noisy and this might not be known to the learner.
This tension, exposed by our work, motivates 
%a revisiting of fairness notions from first principles approach and trying to axiomatize the 
%desired properties of a fairness intervention a la cryptography and privacy. \footnote{Work in multi-calibration \cite{multicalib} is a viable direction for this problem but it is unclear how 
%that and related notions behave with unreliable data. }
on going work studying the sensitivity level of fairness constraints. 
%If we we are to take a view, if a classifier is deployed 


\bibliography{more-bib}
\bibliographystyle{plainnat}
\newpage
\appendix
\section{Experimental Details}
\subsection{Census Experimental Details}\label{sec:experiment-census-details}
% Figure environment removed


We use the 15 of the 17 features in the ACSTravelTime dataset---which include Age, Educational Attainment, Marital Status, Sex, Disability record, Mobility status, Relationship, etc. More specifically, using the notation from \cite{Ding2021RetiringAN}, we choose to keep the 'AGEP', 'SCHL', 'MAR', 'SEX', 'DIS',
'MIG', 'RELP', 'RAC1P', 'PUMA', 'CIT', 'OCCP', 'JWTR', 'POWPUMA', and 'POVPIP' features. We choose to exclude the State code (ST) and Employment Status of Parents (ESP) as a quick way to bypass low-rank covariance matrix issues. We turn the columns 'MAR', 'SEX', 'DIS', 'MIG', 'RAC1P', 'CIT', 'JWTR' into one-hot vectors. We make use commute time 'JWMNP' as the target variable. We clean our data by making sure AGEP (Age) must be greater than 16, PWGTP (Person weight) must be greater than or equal to 1, ESR (Employment status recode) must be equal to 1 (employed), and JWMNP (Travel time to work) is greater than 0. We normalize our features and targets by centering and dividing by the standard deviation computed from the training data. The California datacenter has access to all of the features. The New York datacenter has access to all categories except 'AGEP'. The Texas datacenter has access to all but 'AGEP', 'SCHL'. The Florida datacenter has access to all but  'AGEP', 'SCHL', 'MAR', 'SEX', and the Illinois datacenter has access to all but 'AGEP', 'SCHL', 'MAR', 'SEX', 'DIS', 'MIG'.


\subsection{Synthetic Experiments}
\label{sec:experiment-synthetic}
% Figure environment removed

We start with a synthetic experiment where we generate $m=30$ agents observing some subset of $d=30$ features. Ten of the agents will have access to random subsets of $20$ of the features. The other twenty agents will have access to random subsets of $15$ of the features. Each agent will have $n$ samples which we vary in this experiment. We sample the features from a $\normal(0, \Sigma)$ distribution. We generate $\Sigma$ by first generating $d$ eigenvalues by sampling $d$ times from a uniform $[0, 1]$ distribution. We randomly select $3$ eigenvalues to multiply by $10$ and use these eigenvalues to populate the diagonal of a diagonal matrix $\Lambda$. Then we use a randomly generated orthogonal matrix $W$ to form $\Sigma \defeq W\Lambda W^T$. We plot a heatmap of $\Sigma$ in \Cref{fig:gauss-cov-heatmap}. For each method that we test, we run $20$ trials to form $95\%$ confidence intervals.

We compare our method \textsc{Collab}, against the Imputation and RW-Imputation methods we outlined in \Cref{sec:experiment-census}. 
After we train each of these methods using the data on our $30$ agents, we measure how well these methods perform in using the features of a test-agent with access to $20$ of the total $30$ features to predict outputs. 
We will also compare our methods against Naive-Local, where we only use the $n$ training datapoints of the $20$ features our test-agent has access to, also described in \Cref{sec:experiment-census}. 
We plot this result in \Cref{fig:gauss-local-pred}.

We also compare our methods in an alternative setting where the test-center of interest has access to all $30$ features. This setup models the setting where we are interested making the best possible predictions from all of the features available. In this experiment, we compare against Naive-Collab, Optimized-Naive-Collab, described in \Cref{sec:experiment-census}. We note that Optimized-Naive-Collab uses fresh labeled samples without any missing features during gradient descent, so in this sense, Optimized-Naive-Collab is more powerful than our method. We plot this result in \Cref{fig:gauss-full-pred}.

We see that reweighting is important; this is why \textsc{Collab} and RW-Imputation outperform the unweighted Imputation method. Our \textsc{Collab} method improves over the Naive-Local approach, meaning that the agents are benefiting from sharing information.
\textsc{Collab} also matches the performance of the RW-Imputation method, despite only needing to communicate the learned parameters of each agent's model, as opposed to all of the data on each agent. The Naive-Collab approaches level out very quickly, likely reflecting the fact that these methods are biased, as the covariance of our underlying data is far from isotropic.

\section{Proofs for Section~\ref{sec:upper-bounds}}

\begin{lemma} \label{lem:est-param-consistency}
	For any positive definite matrices $W_i \in \R^{d_i \times d_i}$, $i=1,2,\dots, m$, the aggregated estimator $\est{\param}$ in Eq.~\eqref{eq:defn-weighted-avg-est-param} is consistent $\est{\param} \cp \param$. In addition, if $X_i \sim \normal(0, \scov)$, we have unbiasedness $\Ep [\est{\param}] = \param$ 	where $\Ep$ is over the random data $X_i$ and noise $\noise_i$.
\end{lemma}

\subsection{Proof of Lemma~\ref{lem:est-param-consistency}}
\label{sec:proof-est-param-consistency}
	For the general case, identify for $\est{\param}_i$, we can write
\begin{align*}
	& \est{\param}_i  = (X\iplus^\top X\iplus)^{-1} X\iplus y_i   = (X\iplus^\top X\iplus)^{-1} X\iplus^\top (X\iplus \param\iplus + X\imin \param\imin + \noise_i) \nonumber \\
	& = \param\iplus +  (X\iplus^\top X\iplus)^{-1}  (X\iplus^\top X\imin \param\imin + X\iplus^\top \noise_i) \nonumber \\
	& =  \param\iplus + \prn{\frac{1}{n}X\iplus^\top X\iplus}^{-1} \prn{\frac{1}{n} X\iplus^\top X\imin \param\imin  + \frac{1}{n} X\iplus^\top \noise_i}.
\end{align*}
The weak law of large numbers implies that $X\iplus^\top X\iplus/n \cp \scov\iplus$, $X\iplus^\top X\imin/n \cp \scov\ipm$ and $ \frac{1}{n} X\iplus^\top \noise_i \cp 0$. Then Slutsky's theorem gives the consistency guarantee
\begin{align*}
	\est{\param}_i \cd \param\iplus + \scov\iplus^{-1} \prn{\scov \ipm \param\imin + 0} = \param\iplus + \scov\iplus^{-1} \scov \ipm \param\imin = T_i \param,
\end{align*}
which is equivalent to $\est{\param}_i \cp T_i \param$. Substituting back into $\est{\param}$, we can obtain again from continuous mapping theorem that
\begin{align*}
	\est{\param} & = \prn{\sum_{i=1}^m T_i^\top W_i T_i}^{-1} \prn{\sum_{i=1}^m T_i^\top W_i \est{\param}_i}\cp \prn{\sum_{i=1}^m T_i^\top W_i T_i}^{-1} \prn{\sum_{i=1}^m T_i^\top W_i T_i \param} = \param.
\end{align*}

Next, we specialize to Gaussian features and show $\est{\param}$ is indeed unbiased in this case. By the tower property, we can write for each local OLS estimator,
\begin{align*}
	& \Ep [\est{\param}_i]  = \Ep [(X_{i+}^\top X_{i+})^{-1} X_{i+} y_i]  = \Ep \brk{\Ep [(X\iplus^\top X\iplus)^{-1} X\iplus^\top (X\iplus \param\iplus + X\imin \param\imin + \noise_i) \mid X\iplus]} \nonumber \\
	& = \param\iplus + \Ep \brk{(X\iplus^\top X\iplus)^{-1} X\iplus^\top  \Ep [ X\imin \mid X\iplus]} \param\imin.
\end{align*}
We want to compute $\Ep [ X\imin \mid X\iplus]$ and the key observation is that with Gaussianity in $X_i$, we have
\begin{align*}
	&\cov (x\imin - \scov\imp \scov\iplus^{-1} x\iplus, x\iplus) = \cov (x\imin, x\iplus) - \scov\imp \scov\iplus^{-1}  \cov (x\iplus, x\iplus) \nonumber \\
	& = \scov\imp - \scov\imp \scov\iplus^{-1} \cdot \scov\iplus = 0,
\end{align*}
and therefore $x\iplus$ is independent of $x\imin - \scov\imp \scov\iplus^{-1} x\iplus$, which further implies that
\begin{align*}
	& \Ep \brk{ X\imin  \mid X\iplus}  = \Ep \brk{ X\iplus \scov\iplus^{-1} \scov\ipm  \mid X\iplus}  + \Ep \brk{X \imin - X\iplus \scov\iplus^{-1} \scov\ipm \mid X\iplus}  =  X\iplus \scov\iplus^{-1} \scov\ipm.
\end{align*} 
Substituting the above property into computing the expectation of local estimates $\est{\param}_i$, it then holds
\begin{align*}
	\Ep [\est{\param}_i] & = \param\iplus + \Ep [(X\iplus^\top X\iplus)^{-1} X\iplus^\top X\iplus \scov\iplus^{-1} \scov\ipm  ] \param\imin  = \param\iplus +  \scov\iplus^{-1} \scov\ipm \param\imin = T_i \param.
\end{align*}
We can then conclude the proof as
\begin{align*}
	\Ep [\est{\param}] = \prn{\sum_{i=1}^m T_i^\top W_i T_i}^{-1} \prn{\sum_{i=1}^m T_i^\top W_i T_i \param} = \param.
\end{align*}

\subsection{Proof of Theorem~\ref{thm:upper-bound}}
\label{sec:proof-upper-bound}
	We first study the central limit theorem for local OLS estimators $\est{\param}_i$. Let the data matrices $X\iplus = [x\iplus^1, \dots, x\iplus^n]^\top$ and $X\imin = [x\imin^1, \dots, x\imin^n]$ and the noise vector $\noise_i = [\noise_i^1, \dots, \noise_i^n]^\top$, we can write out for $\est{\param}_i$ that
\begin{align}
	& \sqrt{n} \prn{\est{\param}_i - T_i \param} = \underbrace{\prn{X\iplus^\top X\iplus/n}^{-1}}_{\mathrm{(I)}}  \cdot \underbrace{\frac{1}{\sqrt{n}} X\ipm^\top \brc{(X \imin - X\iplus \scov\iplus^{-1} \scov\ipm)\param\imin  + \noise_i}}_{\mathrm{(II)}}. \label{eq:local-estimate-clt}
\end{align}
For (II), note that
\begin{align*}
	&  \frac{1}{\sqrt{n}} X\ipm^\top \brc{(X \imin - X\iplus \scov\iplus^{-1} \scov\ipm)\param\imin  + \noise_i}  = \frac{1}{\sqrt{n}} \sum_{k=1}^n x\iplus^j \brc{(x\imin^j - \scov\imp \scov\iplus^{-1} x\iplus^j)^\top \param\imin + \noise_i^j }.
\end{align*}
The summands are independent mean zero random vectors, since 
\begin{align*}
	& \E \brk{x\iplus^j \brc{(x\imin^j - \scov\imp \scov\iplus^{-1} x\iplus^j)^\top \param\imin}}  = \prn{\E \brk{x\iplus^j {x\imin^j}^\top} - \E \brk{x\iplus^j {x\iplus^j}^\top} \scov\iplus^{-1} \scov\ipm } \param\imin \nonumber \\
	& = \prn{\scov\ipm - \scov\iplus \scov\iplus^{-1} \scov\ipm} \param\imin = 0,
\end{align*}
and $\E [x\iplus^j \noise_i^j] = \E [x\iplus^j] \cdot \E [\noise_i^j] = 0$. Denote by $z\iplus^j := x\imin^j - \scov\imp \scov\iplus^{-1} x\iplus^j$ and we can infer from the above display that $x\iplus$ and $z\iplus$ are uncorrelated.
(II) is then asymptotically normal by CLT with limiting covariance (suppressing the superscript $j$ below)
\begin{align}
	& \cov \prn{x\iplus \brc{(x\imin - \scov\imp \scov\iplus^{-1} x\iplus)^\top \param\imin + \noise_i }} = \Ep \brk{x\iplus  \param\imin^\top z\iplus  z\iplus^\top  \param\imin x\iplus^\top} + \Ep \brk{\noise_i^2 x\iplus  x\iplus^\top} \nonumber \\
	& = \Ep \brk{x\iplus  \param\imin^\top z\iplus  z\iplus^\top  \param\imin x\iplus^\top} + \sigma^2 \scov\iplus := Q_i. \label{eq:def-Q-i}
\end{align}

If $X_i$ are Gaussian random vectors, we can additionally have independence between $z\iplus$ and $x\iplus$ by zero correlation. Therefore
\begin{align*}
	& \Ep \brk{x\iplus  \param\imin^\top z\iplus  z\iplus^\top  \param\imin x\iplus^\top}  = \Ep \brk{x\iplus  \param\imin^\top \Ep \brk{z\iplus z\iplus^\top}  \param\imin x\iplus^\top} \nonumber \\
	& = \param\imin^\top  \cov \prn{ x\imin- \scov\imp \scov\iplus^{-1} x\iplus} \param\imin \cdot \Ep \brk{x\iplus x\iplus^\top} = \param\imin^\top  \prn{\scov\imin - \scov\imp \scov\iplus^{-1} \scov\ipm} \param\imin \cdot \scov\iplus   = \norm{\param\imin}_{\Gamma\imin}^2 \scov\iplus,
\end{align*}
and $Q_i =  (\norm{\param\imin}_{\Gamma\imin}^2 + \sigma^2)\scov\iplus$.

We proceed to show $C(W_1, \cdots, W_n) \succeq C\opt$ under general feature distribution $\mc{P}$ and $W_i\opt := \scov\iplus Q_i^{-1} \scov\iplus$. By Slutsky theorem, (I) converges to $\scov\iplus^{-1}$ in probability and we can conclude from Eq.~\eqref{eq:local-estimate-clt} that
\begin{align} \label{eq:asymptotic-normality-est-i}
	\sqrt{n} \prn{\est{\param}_i - T_i \param} \cd \normal\prn{0, \scov\iplus^{-1} Q_i \scov\iplus^{-1}}.
\end{align}
Further from $\est{\param} = \prnbig{\sum_{i=1}^m T_i^\top W_i T_i}^{-1} \prnbig{\sum_{i=1}^m T_i^\top W_i \est{\param}_i}$, it follows that
\begin{align*}
	\sqrt{n} \prn{\est{\param}_i - \param} = \normal(0, C(W_1, \cdots, W_n))
\end{align*}
where
\begin{align}
	& C(W_1, \cdots, W_n) = \prn{\sum_{i=1}^m T_i^\top W_i T_i}^{-1} \cdot \prn{\sum_{i=1}^m T_i^\top W_i {W_i\opt}^{-1} W_i T_i}   \cdot \prn{\sum_{i=1}^m T_i^\top W_i T_i}^{-1}. \label{eq:def-C-function}
\end{align}
With the choice of $W_i = W_i\opt$, we achieve the claimed lower bound for asymptotic covariance as in this case $C(W_1, \cdots, W_m) = \prnbig{\sum_{i=1}^m T_i^\top W_i\opt  T_i }^{-1}$. It thus remains to show
\begin{align*}
	C(W_1, \cdots, W_n) \succeq \prn{\sum_{i=1}^m T_i^\top  W_i\opt T_i}^{-1} = C\opt.
\end{align*}
To prove the above claim, we construct auxiliary matrices $M_i$ as
\begin{align*}
	M_i & = \begin{bmatrix}
		T_i^\top W_i\opt T_i  & T_i^\top W_i T_i \\
		T_i^\top W_i T_i &   T_i^\top W_i  {W_i\opt}^{-1} W_i T_i
	\end{bmatrix}  = \begin{bmatrix} T_i^\top {W_i\opt}^\half  \\  T_i^\top W_i {W_i\opt}^{-\half} \end{bmatrix} \begin{bmatrix} T_i^\top {W_i\opt}^\half  \\  T_i^\top W_i {W_i\opt}^{-\half} \end{bmatrix}^\top \succeq 0.
\end{align*}
Therefore
\begin{align*}
	\sum_{i=1}^m M_i & = \begin{bmatrix}
		{C\opt}^{-1} & \sum_{i=1}^m T_i^\top W_i T_i \\
		\sum_{i=1}^m T_i^\top W_i T_i  & \sum_{i=1}^m   T_i^\top W_i  {W_i\opt}^{-1} W_i T_i
	\end{bmatrix} \succeq 0.
\end{align*}
As the Schur complement is also p.s.d.\, we can conclude with
\begin{align*}
	0 &  \preceq {C\opt}^{-1} - \prn{ \sum_{i=1}^m T_i^\top W_i T_i} \cdot \prn{\sum_{i=1}^m   T_i^\top W_i   {W_i\opt}^{-1} W_i T_i}^{-1} \nonumber \\ & \qquad \cdot \prn{ \sum_{i=1}^m T_i^\top W_i T_i} =  {C\opt}^{-1}  - C(W_1, \cdots, W_n)^{-1}.
\end{align*}


\subsection{Proof of \Cref{cor:upper-bound-local}}
\label{sec:proof-upper-bound-local}
	We first prove (i) and asymptotic normality of $\sqrt{n} (\est{\param}\collab - \param) \cd \normal\left(0, C\gauss \right)$. We point out that Theorem~\ref{thm:upper-bound} is not directly applicable as we use estimated weights that reuse the training data. We claim consistency for $\est{W}\gauss_i \cp W\gauss$, and under this premise, the proof is rather straightforward since we can write
	\begin{align*}
		\sqrt{n} \prn{\est{\param}\collab - \param} = \prn{\sum_{i=1}^m T_i^\top \est{W}_i\gauss T_i}^{-1} \prn{\sum_{i=1}^m T_i^\top \est{W}_i\gauss (\est{\param}_i - T_i\param)}.
	\end{align*}
	With the asymptotic normality established for $\sqrt{n} (\est{\param}_i - T_i \param)$ in Eq.~\eqref{eq:asymptotic-normality-est-i}, Slutsky's theorem and continuous mapping theorem, we can conclude that $\sqrt{n} (\est{\param}\collab - \param) \cd \normal\left(0, C\gauss \right)$. Now it remains to showing $\est{W}\gauss_i \cp W\gauss$, this is from Slutksy's theorem applied to $\est{W}\gauss_i = \est{\Sigma}\iplus / \est{R}_i$ and the weak law of large numbers as follows
	\begin{align*}
		\est{\Sigma}\iplus = \frac{X\iplus^\top X\iplus}{n} \cp \Sigma\iplus, \qquad \est{R}_i = \frac{1}{n}\|X\iplus  \hparam_i - y\|_2^2 \cp \Ep [\norm{x\iplus^\top T_i \param - y_i}_2^2],
	\end{align*}
	where
	\begin{align*}
		& \Ep [\norm{x\iplus^\top T_i \param - y_i}_2^2] = \Ep [\norm{x\iplus^\top \scov\iplus^{-1} \scov\ipm \param\imin - x \imin^\top \param \imin }_2^2] + \sigma^2 \nonumber \\
		& = \norm{\param\imin}_{\cov \prn{x \imin - \scov\imp \scov\iplus^{-1} x\iplus}}^2 + \sigma^2 = \norm{\param\imin}_{\Gamma\imin}^2 + \sigma^2.
	\end{align*}

	We proceed to prove (ii). Applying delta method to the mapping $\param \mapsto T_i \param, \R^d \to \R^{d_i}$ on $\est{\param}(W_1\opt, \cdots, W_m\opt)$ immediately yields the asymptotic normality for $\est{\param}_i\collab$. It only remains to show $T_i C\opt T_i^\top \preceq {W_i\opt}^{-1}$.
	
	Identify ${W_i\opt}^{-1} - T_i C\opt T_i^\top$ as the Schur complement for the block matrix
	\begin{align*}
		M = \begin{bmatrix}
			{W_i\opt}^{-1} & T_i \\
			T_i^\top & {C\opt}^{-1}
		\end{bmatrix},
	\end{align*}
	and it suffices to show $M \succeq 0$. This follows from ${C\opt} = (\sum_{i=1}^m T_i^\top W_i\opt T_i)^{-1}$ and thus
	\begin{align*}
		M & = \begin{bmatrix}
			{W_i\opt}^{-1} & T_i \\
			T_i^\top & \sum_{j=1}^m T_j^\top W_j\opt T_j
		\end{bmatrix} \succeq \begin{bmatrix}
		{W_i\opt}^{-1} & T_i \\
		T_i^\top & T_i^\top W_i\opt T_i 
	\end{bmatrix}   = \begin{bmatrix}
	{W_i\opt}^{-\half} \\
	T_i^\top {W_i\opt}^{\half}
\end{bmatrix} \begin{bmatrix}
	{W_i\opt}^{-\half} \\
	T_i^\top {W_i\opt}^{\half}
\end{bmatrix}^\top \succeq 0.
	\end{align*}
\section{Proofs for Section~\ref{sec:comparison}} \label{proof:upper-bound-comparison}

\subsection{Proof of Theorem~\ref{thm:upper-bound-imputed}} \label{proof:upper-bound-imputed}
The key part of the proof is showing $\est{\param}_i\impute = T_i^\top (T_iT_i^\top)^{-1} \est{\param}_i$. If we can have this claim established, we can make use of the following transformation of the loss function
\begin{align*}
	& \sum_{i=1}^m  \norm{T_i^\top (T_i T_i^\top)^{-1} T_i  \param - \est{\param}_i\impute  }_{W_i}^2  = \sum_{i=1}^m  \norm{T_i^\top (T_i T_i^\top)^{-1} T_i  \param - T_i^\top (T_i T_i^\top)^{-1} \est{\param}_i }_{W_i}^2 \nonumber \\
	& = \sum_{i=1}^m  \norm{ T_i  \param -  \est{\param}_i }_{(T_i T_i^\top)^{-1} T_i W_i T_i^\top (T_i T_i^\top)^{-1}}^2.
\end{align*}
This reduces the optimization problem into the same one in Eq.~\eqref{eq:defn-weighted-avg-est-param} up to weight transformation, and the same lower bound for asymptotic covariance in Theorem~\ref{thm:upper-bound} applies. Hence
\begin{align*}
	C\imputeglb(\alpha_1, \cdots, \alpha_m) \succeq C\opt.
\end{align*}
By taking $W_i= T_i^\top W_i\opt T_i$, we have the transformed weights satisfy
\begin{align*}
	(T_i T_i^\top)^{-1} T_i W_i T_i^\top (T_i T_i^\top)^{-1} = (T_i T_i^\top)^{-1} T_i^\top W_i\opt T_i (T_i T_i^\top)^{-1} = W_i\opt.
\end{align*}
From the optimality condition in Theorem~\ref{thm:upper-bound}, the equality holds under this choice of $W_i$'s.

It then boils down to proving the claim $\est{\param}_i\impute = T_i^\top (T_iT_i^\top)^{-1} \est{\param}_i$. We make use of the following two properties of Moore-Penrose pseudo inverse---for $A \in \R^{d_i \times d}$ of rank $d_i$,
\begin{align*}
	(A^\top A)^\dagger = A^\dagger (A^\dagger)^\top, \qquad A^\dagger = A^\top (AA^\top)^{-1}.
\end{align*} 
Substituting $A=(X\iplus^\top X\iplus)^{\half} T_i$ into the above displays, we then have
\begin{align*}
	& \est{\param}\impute_i  = (T_i^\top X\iplus^\top X\iplus T_i)^\dagger T_i^\top X\iplus^\top y_i \nonumber \\
	& = T_i^\top (X\iplus^\top X\iplus)^{\half} \prn{(X\iplus^\top X\iplus)^{\half} T_i T_i^\top (X\iplus^\top X\iplus)^{\half}}^{-2}   \cdot   (X\iplus^\top X\iplus)^{\half} T_i T_i^\top X\iplus^\top y_i \nonumber \\
	& = T_i^\top (X\iplus^\top X\iplus)^{\half} \prn{(X\iplus^\top X\iplus)^{-\half} (T_i T_i^\top)^{-1} (X\iplus^\top X\iplus)^{-\half}}^{2}   \cdot   (X\iplus^\top X\iplus)^{\half} T_i T_i^\top X\iplus^\top y_i \nonumber \\
	& = T_i^\top (T_i T_i^\top)^{-1} \cdot (X\iplus^\top X\iplus)^{-1} \cdot (T_i T_i^\top)^{-1} \cdot T_i T_i^\top X\iplus^\top y_i \nonumber \\
	& = T_i^\top (T_i T_i^\top)^{-1} \cdot (X\iplus^\top X\iplus)^{-1} X\iplus^\top y_i =  T_i^\top (T_i T_i^\top)^{-1} \est{\param}_i .
\end{align*}
\subsection{Proof of Theorem~\ref{thm:upper-bound-imputed-glb}} \label{proof:upper-bound-imputed-glb}

By a direct calculation, we have
\begin{align*}
	& \est{\param}\imputeglb - \param = \prn{\sum_{i=1}^m \alpha_i T_i^\top X\iplus^\top X\iplus T_i}^{-1} \prn{\sum_{i=1}^m \alpha_i T_i^\top X\iplus^\top y_i} - \param \\
	&  = \prn{\sum_{i=1}^m \alpha_i T_i^\top X\iplus^\top X\iplus T_i}^{-1} \prn{\sum_{i=1}^m \alpha_i T_i^\top X\iplus^\top (X\iplus \param\iplus + X\imin \param\imin +  \noise_i)} - \param \nonumber \\
	& =  \prn{\sum_{i=1}^m \alpha_i T_i^\top X\iplus^\top X\iplus T_i}^{-1} \prn{\sum_{i=1}^m \alpha_i T_i^\top X\iplus^\top (X\iplus \param\iplus + X\imin \param\imin -X\iplus T_i \param +  \noise_i)} \nonumber \\
	& = \prn{\sum_{i=1}^m \alpha_i T_i^\top X\iplus^\top X\iplus T_i}^{-1} \prn{\sum_{i=1}^m \alpha_i T_i^\top X\iplus^\top (X\imin \param\imin -X\iplus \scov\iplus^{-1} \scov\ipm \param\imin +  \noise_i)}.
\end{align*}
Consequently
\begin{align*}
	\sqrt{n} \prn{\est{\param}\imputeglb - \param} = \prn{\sum_{i=1}^m \alpha_i  T_i^\top \cdot \frac{1}{n} X\iplus^\top X\iplus \cdot T_i}^{-1} \cdot \prn{\sum_{i=1}^m \alpha_i T_i^\top \cdot \frac{1}{\sqrt{n}}X\iplus^\top (X\imin \param\imin -X\iplus \scov\iplus^{-1} \scov\ipm \param\imin +  \noise_i)}
\end{align*}
Following the same proof steps applied to Eq.~\eqref{eq:local-estimate-clt} in Appendix~\ref{sec:proof-upper-bound}, we can conclude that
\begin{align*}
	& \sqrt{n} \prn{\est{\param}\imputeglb - \param} \nonumber \\
	& \cd \normal \Bigg(0, \underbrace{\prn{\sum_{i=1}^m \alpha_i  T_i^\top \scov\iplus T_i}^{-1} \prn{\sum_{i=1}^m \alpha_i^2 T_i^\top Q_i T_i} \prn{\sum_{i=1}^m \alpha_i T_i^\top \scov\iplus T_i}^{-1}}_{:= C\imputeglb(\alpha_1, \cdots, \alpha_m)}\Bigg),
\end{align*}
with the same $Q_i$'s as in Eq.~\eqref{eq:def-Q-i}, and with Gaussianity of $X_i$, we also have the explicit form $Q_i =  (\norm{\param\imin}_{\Gamma\imin}^2 + \sigma^2)\scov\iplus$. Note that if $\alpha_i = 1/ (\norm{\param\imin}_{\Gamma\imin}^2 + \sigma^2)$,
\begin{align*}
	C\imputeglb(\alpha_1, \cdots, \alpha_m) = \prn{\sum_{i=1}^m \frac{T_i^\top \scov\iplus T_i}{\norm{\param\imin}_{\Gamma\imin}^2 + \sigma^2}}^{-1} = C\gauss  = C\opt.
\end{align*}
Finally, to show $C\imputeglb(\alpha_1, \cdots, \alpha_m) \succeq C\opt$, we identify from Eq.~\eqref{eq:def-C-function} that
\begin{align*}
	C\imputeglb(\alpha_1, \cdots, \alpha_m) =  C(\alpha_1 \scov_{1+}, \cdots, \alpha_m \scov_{m+}) \succeq C\opt,
\end{align*}
where the last inequality follows from Theorem~\ref{thm:upper-bound}.
\section{Proofs for \Cref{sec:lower-bounds}}
\label{sec:proofs-lower-bounds}


We will use the van Trees inequality to prove our lower bound shown. In particular, we will use a slight modification to Theorem 4 of \cite{Gassiat2014RevisitingTV}, which we state as a corollary below here. 
Throughout this section, we let $\psi: \R^d \to \R^s$ be an absolutely continuous function. The distribution $P_\param$ in the family $\{P_\param\}_{\param \in \R^d}$ is assumed to have density $p_\param$ which satisfies $\int_{\R^d} \ltwo{\nabla p_\param(x)}^2 dx < \infty$.
Let $P_\param^j$ for $j \in [m]$ denote the distribution over either $\tparam_j^n$ or $y_j \in \R^n$.
Let $\finfo_i^n(\param)$ denote the Fisher Information of $P_\param^i$, and let $\finfo^n(\param) = \sum_{i = 1}^m \finfo_i^n(\param)$ denote the Fisher Information of $P_\param$. We note that $P_\param$ is allowed to depend on $n$.


\begin{corollary}[\citet{Gassiat2014RevisitingTV}]\label{thm:van-trees}
 Let $\psi: \R^d \to \R^s$ be an absolutely continuous function such that $\nabla \psi(\param)$ is continuous at $\param_0$. For all $n$, let all distributions $P_\param$ in the family $\{P_\param\}_{\param \in \R^d}$ have density $p_\param$ which satisfies $\int_{\R^d} \ltwo{\nabla p_\param(x)}^2 dx < \infty$. 
 If $\lim_{c \to \infty}\lim_{n \to \infty} \sup_{\ltwo{h} < 1}\finfo^n(\param_0 + ch / \sqrt{n}) / n$ exists almost surely and is positive definite, denote it by $\rho$. Then
 for all sequences $(\hparam_n)_{n \geq 1}$ of statistics $S_n: \mc{X}^n \to \R^s$ and for all $u \in \R^s$
 \begin{align*}
    \liminf_{c \to \infty} \liminf_{n \to \infty} \sup_{\norm{h} < 1}
    \E^n_{\param_0 + \frac{ch}{\sqrt{n}}} \left[ 
        \left\<\sqrt{n} \left(\hparam_n - \psi\left(\param_0 + \frac{ch}{\sqrt{n}}\right)\right), u \right\>^2
    \right]
    \geq 
        u^\top \nabla \psi(\param_0)^\top \rho\inv \nabla \psi (\param_0) u
\end{align*}
\end{corollary}
\begin{proof}
    The main difference between our version of the proof and the one presented in Theorem 4 of \citet{Gassiat2014RevisitingTV} is that we do not assume $\finfo^n = n \finfo$. We also select $\ell(x) = \< u, x\>^2$ in particular. All the steps and notation remain the same except with $n\finfo$ replaced with $\finfo^n$ up until equation (13), which we define with a modified choice of $\Gamma_{c,n}$
    \begin{align*}
        \Gamma_{c,n} \defeq 
        \left(\int_{\mc{B}_{p}([0], 1)}  \nabla\psi(\param_0 + ch /\sqrt{n}) q(h) dh \right)^\top
        \left(
            \frac{1}{c^2} \finfo_q + \frac{1}{n}\int_{\mc{B}_{p}([0], 1)}  \finfo^n(\param_0 + ch /\sqrt{n}) q(h) dh
        \right)\inv\\
        \times \left(\int_{\mc{B}_{p}([0], 1)}  \nabla\psi(\param_0 + ch /\sqrt{n}) q(h) dh \right).
    \end{align*}
   By definition of $\rho$, with probability 1,
    \begin{align*}
        \lim_{c \to \infty} \lim_{n \to \infty} \Gamma_{c, n} = \nabla \psi(\param_0)^\top \rho\inv \nabla \psi (\param_0)
    \end{align*}
\end{proof}


\subsection{Proof of \Cref{thm:weak-global-lb}}
\label{sec:proof-weak-global-lb}
    We will apply \Cref{thm:van-trees} and apply it to two different choices of $\psi$ to get the full feature minimax bound and missing feature minimax bound respectively. 
    For notational simplicty, let $P_\param$ denote the distribution over $\{\tparam_i^n\}_{i \in [m]}$ induced by $\param$. 
    $P_\param$ is in the exponential family, so the conditions of \Cref{thm:van-trees} are satisfied.

    We begin by computing the Fisher Information.
    Let $P_\param^j$ for $j \in [m]$ denote the distribution over
     $\tparam_j^n \in \R^{d_j}$.
    Let $\finfo_i^n(\param)$ denote the Fisher Information of $P_\param^i$, and 
    let $\finfo^n(\param) = \sum_{i=1}^m \finfo_i^n$ denote the Fisher Information of $P_\param$.
    Let $x\iplus$ denote an arbitrary row of $X\iplus$. Let $x\imin$ be drawn from $\normal(\mu\imin(x\iplus), \Gamma\imin)$. 
    Some straightforward calculations tell us 
    $\mu\imin(x\iplus) = \scov\imp \scov\iplus\inv x\iplus$ and $\Gamma\imin = \scov\imin - \scov\imp \scov\iplus\inv\scov\ipm$. From this we can deduce that $\param\imin^T x\imin$ is distributed as $\normal(\mu\imin^T \param\imin, \param\imin^T \Gamma\imin \param\imin)$; we use $\mu\imin$ in place of $\mu\imin(x\imin)$ for simplicity. And $y_i$ is distributed as $P_\param^i$ which is $\normal(\param^T\gamma, \param\imin^T \Gamma\imin \param\imin + \sigma^2 )$ where $\gamma\defeq [ x\iplus^T\projm\iplus, \mu\imin^T \projm\imin]^T$. From this we can deduce that $P_\param^i$ is $\normal\left(J_i \projm_i\param, \beta_i\inv \hscov\iplus\inv\right)$, where $\beta_i\inv \defeq \frac{\param\imin \Gamma\imin \param\imin + \sigma^2}{n}$; let $p_\param^i$ denote its density. 
    We know $\finfo^n(\param) = \sum_{i=1}^m \finfo_i^n(\param)$ due to independence. All that remains is to compute $\finfo_i^n(\param)$.
\begin{align*}
    \finfo_i^n(\param) = \int \nabla_\param \log p_\param^i(z) [\nabla_\param \log p_\param^i(z) ]^T p_\param^i(z) dz.
\end{align*}

We know that for some constant $C$,
\begin{align*}
    \log p_i^\param(z) &= 
    C + \frac{d_i}{2}\log(\beta_i) - \frac{\beta_i}{2} \ltwo{\hscov^\half\iplus\param\iplus + \hscov\iplus^\half \scov\iplus\inv \scov\ipm \param\imin - \hscov\iplus^\half z}^2 .
\end{align*}
Taking derivaties we get that 
\begin{align*}
    \nabla_{\param\iplus}  \log p_i^\param(z) &= 
    -\beta_i \left[ \hscov\iplus \param\iplus + \hscov\iplus \scov\iplus\inv \scov\ipm \param\imin - \hscov\iplus z \right]\\
    \nabla_{\param\imin}  \log p_i^\param(z) &= \left[ -\frac{d_i}{n} + \ltwo{\hscov^\half\iplus\param\iplus + \hscov\iplus^\half \scov\iplus\inv \scov\ipm \param\imin - \hscov\iplus^\half z}^2 \right]\beta_i \Gamma\imin\param\imin\\
    &\quad  + \left[\scov\imp \scov\iplus\inv \hscov\iplus \param\iplus + \scov\imp \scov\iplus\inv \hscov\iplus \scov\iplus\inv \scov\ipm \param\imin- \scov\imp \scov\iplus\inv \hscov\iplus z \right]\beta_i
\end{align*}
Let $b^2 = \ltwo{\hscov^\half\iplus\param\iplus + \hscov\iplus^\half \scov\iplus\inv \scov\ipm \param\imin - \hscov\iplus^\half z}^2$.
Now we compute the expectation over outer products:
\begin{align*}
    &\E[\nabla_{\param\iplus}  \log p_i^\param(z) \nabla_{\param\iplus}  \log p_i^\param(z)^T] = \beta_i \hscov\iplus \\
    &\E[\nabla_{\param\iplus}  \log p_i^\param(z) \nabla_{\param\imin}  \log p_i^\param(z)^T] = \beta_i^2 \hscov\iplus \beta_i\inv \hscov\iplus\inv \hscov\iplus \scov\iplus\inv \scov\ipm
    = \beta_i \hscov\iplus \scov\iplus\inv \scov\ipm\\
    &\E[\nabla_{\param\imin}  \log p_i^\param(z) \nabla_{\param\imin}  \log p_i^\param(z)^T] =\beta_i \scov\imp\scov\iplus\inv \hscov\iplus \scov\iplus\inv \scov\ipm\\
    &\qquad\qquad + \left( 
        \frac{d_i^2}{n^2} + \E[b^2] \frac{2d_i}{n}+ \E[b^4]
    \right)\beta_i^2 \Gamma\imin \param\imin \param\imin^T \Gamma\imin\\
    &\qquad= \beta_i \scov\imp\scov\iplus\inv \hscov\iplus \scov\iplus\inv \scov\ipm + \left( 
        \frac{d_i^2}{n^2} +  \frac{2\beta_i\inv d_i^2}{n}+ \beta_i^{-2}(2d_i + d_i^2)
    \right)\beta_i^2 \Gamma\imin \param\imin \param\imin^T \Gamma\imin
\end{align*}

\begin{align*}
    \finfo_i^n(\param) &= \int \nabla_\param \log p_\param^i(z) [\nabla_\param \log p_\param^i(z) ]^T p_\param^i(z) dz\\
    &= \int \projm_i^T \begin{bmatrix}
        \nabla_{\param\iplus} \log p_\param^i(z)\\
        \nabla_{\param\imin} \log p_\param^i(z)
    \end{bmatrix}
    \begin{bmatrix}
        \nabla_{\param\iplus} \log p_\param^i(z)^T &
        \nabla_{\param\imin} \log p_\param^i(z)^T
    \end{bmatrix}
    \projm_i p_\param^i(z)dz\\
    &=  \frac{n}{\sigma^2 + \param\imin^T\Gamma\param\imin}
    \projm_i^T\begin{bmatrix}
        \hscov\iplus & \hscov\iplus\scov\iplus\inv\scov\ipm\\
        \scov\imp \scov\iplus\inv \hscov\iplus  & \scov\imp\scov\iplus\inv \hscov\iplus \scov\iplus\inv \scov\ipm 
    \end{bmatrix} \projm_i 
    \\
    &\qquad\qquad +
    \projm_i^T
    \begin{bmatrix}
        0 & 0\\
        0 & \left( 
            \frac{d_i^2 \beta_i^2}{n^2} +  \frac{2\beta_i d_i^2}{n}+ 2d_i + d_i^2
        \right)\Gamma\imin \param\imin \param\imin^T \Gamma\imin
    \end{bmatrix}
    \projm_i\\
    &= \frac{n}{\sigma^2 + \param\imin^T\Gamma\param\imin} \left(Q_i + o_n(1)\right)
\end{align*}
The $o_n(1)$ term is due to strong law of large numbers.
From this we know that, with probability 1,
\begin{align*}
    \lim_{c\to \infty}\lim_{n \to \infty} \sup_{\ltwo{h} < 1}\frac{\finfo^n(\param_0 + ch / \sqrt{n})}{n} &= \sum_{i=1}^m  \frac{1}{\sigma^2 + \param\imin^T\Gamma\param\imin} Q_i =: \rho
\end{align*}

Applying \Cref{thm:van-trees} with $\psi \R^d \to \R^d$ as the identity function $\psi(x) = x$ gives the full-feature minimax lower bound. Applying \Cref{thm:van-trees} with $\psi\R^d \to \R^{d_i}$ as $\psi(x) = T_i x$ gives the missing-feature minimax lower bound.





\subsection{Proof of \Cref{thm:strong-global-lb}}
\label{sec:proof-strong-global-lb}
We will apply \Cref{thm:van-trees} and apply it to two different choices of $\psi$ to get the full feature minimax bound and missing feature minimax bound respectively. 
    For notational simplicity, we will use $P_\param$ in place of $P_\param^{y}$.
    $P_\param$ is in the exponential family, so the conditions of \Cref{thm:van-trees} are satisfied.

We begin by computing the Fisher Information.
Let $P_\param^j$ for $j \in [m]$ denote the distribution over $y_j \in \R^n$.
    Let $\finfo_i^n(\param)$ denote the Fisher Information of $\P_\param^i$, and let $\finfo^n(\param) = \sum_{i=1}^m \finfo_i^n(\param)$ denote the Fisher Information of $P_\param$.

    Let $x\kth_i, y\kth_i$ be the $k$th sample from agent $i$.
    We will let $\finfo_i\kth(\param)$ be the fisher information of $y\kth_i$. We know that $\finfo_i^n(\param) = \sum_{k=1}^n\finfo_i\kth(\param)$ by independence. 
    Some straightforward calculations tell us that $x\kth\imin$ is distributed as $\normal(\mu, \Gamma)$ where $\mu = \scov\imp \scov\iplus\inv x\kth\iplus$ and $\Gamma = \scov\imin - \scov\imp \scov\iplus\inv\scov\ipm$. From this we can deduce that $\param\imin^T x\kth\imin$ is distributed as $\normal(\mu^T \param\imin, \param\imin^T \Gamma \param\imin)$. And $y_i\kth$ is distributed as 
     $\normal(\param^T\gamma, \param\imin^T \Gamma \param\imin + \sigma^2 )$ 
    where $\gamma\defeq \projm\iplus^T x\iplus\kth + \projm\imin^T \mu$.
    
    
    Let $\phi \defeq \frac{z - \gamma^T \param}{\sigma^2 +\param\imin^T \Gamma\param\imin}$ and $\Delta \defeq \phi^2 - \frac{1}{\sigma^2 + \param\imin^T \Gamma \param\imin}$. Using $p_\param^{ik}$ denote the density of $x\kth\imin, y\kth_i$, we can calculate the derivative of the log density
    \begin{align*}
        \nabla_{\param\iplus} \log p_\param^{ik}(z) &= \frac{z - \param\iplus^T x\kth\iplus - \param\imin^T \mu}{\sigma^2 + \param\imin \Gamma \param\imin}x\kth\iplus = \phi x\kth\iplus\\
        \nabla_{\param\imin} \log p_\param^{ik}(z) &= \Delta \Gamma \param\imin + \phi \mu.
    \end{align*}

    Using the facts that $\E[\phi]=0$, $\E[\phi^2]=\frac{1}{\sigma^2 + \param\imin\Gamma\param\imin}$, $\E[\phi\Delta]=0$, and $\E[\Delta^2]= \frac{2}{(\sigma^2 + \param\imin\Gamma\param\imin)^2}$, where the expectation is an integral over $z$, we have that 
    \begin{align*}
        &\finfo_i\kth(\param) = \int \nabla_\param \log p_\param^{ik}(z) [\nabla_\param \log p_\param^{ik}(z) ]^T p_\param^{ik}(z) dz\\
        &= \int \projm_i^T \begin{bmatrix}
            \nabla_{\param\iplus} \log p_\param^{ik}(z)\\
            \nabla_{\param\imin} \log p_\param^{ik}(z)
        \end{bmatrix}
        \begin{bmatrix}
            \nabla_{\param\iplus} \log p_\param^{ik}(z)^T &
            \nabla_{\param\imin} \log p_\param^{ik}(z)^T
        \end{bmatrix}
        \projm_i p_\param^{ik}(z)dz\\
        &= \projm_i^T 
        \begin{bmatrix}
            \E[\phi^2] x\iplus\kth (x\iplus\kth)^T & \E[\phi x\iplus\kth (\Delta \Gamma \theta\imin + \phi \mu)^T]\\
            \E[(\Delta \Gamma \theta\imin + \phi \mu) (\phi x\iplus\kth)^T]
            & \E[(\Delta \Gamma \theta\imin + \phi \mu)(\Delta \Gamma \theta\imin + \phi \mu)^T] 
        \end{bmatrix}\projm_i\\
        &= 
        \frac{1}{\sigma^2 + \param\imin\Gamma\param\imin}
        \projm_i^T 
        \begin{bmatrix}
             x\iplus\kth (x\iplus\kth)^T &  x\iplus\kth \mu^T\\
            \mu(x\iplus\kth)^T
            & \mu\mu^T +  \frac{2}{\sigma^2 + \param\imin^T\Gamma\param\imin} \Gamma \param\imin \param\imin^T\Gamma
        \end{bmatrix}\projm_i\\
        &= 
        \frac{1}{\sigma^2 + \param\imin\Gamma\param\imin}
        \projm_i^T
        \begin{bmatrix}
             x\iplus\kth (x\iplus\kth)^T &  x\iplus\kth (x\iplus\kth)^T \scov\iplus\inv \scov\ipm\\
             \scov\imp \scov\iplus\inv x\kth\iplus(x\iplus\kth)^T
            & \scov\imp \scov\iplus\inv x\kth\iplus(x\iplus\kth)^T\scov\iplus\inv \scov\ipm + \frac{2}{\sigma^2 + \param\imin^T\Gamma\param\imin} \Gamma \param\imin \param\imin^T\Gamma
        \end{bmatrix}\projm_i.
    \end{align*}
    From this we can sum over 
    \begin{align*}
       \finfo_i^n(\param)&= \sum_{k=1}^n\finfo_i\kth(\param)\\ 
&= \frac{n}{\sigma^2 + \param\imin^T\Gamma\param\imin}\projm_i^T\begin{bmatrix}
    \hscov\iplus &  \hscov\iplus \scov\iplus\inv \scov\ipm\\
    \scov\imp \scov\iplus\inv \hscov\iplus
   & \scov\imp \scov\iplus\inv\hscov\iplus\scov\iplus\inv \scov\ipm + \frac{2}{\sigma^2 + \param\imin^T\Gamma\param\imin} \Gamma \param\imin \param\imin^T\Gamma
\end{bmatrix} 
    \projm_i\\
        &= \frac{n}{\sigma^2 + \param\imin^T\Gamma\param\imin}\left( Q_i + o_n(1) + \projm_i^T\begin{bmatrix}
                0 & 0\\
                0 & \frac{2}{\sigma^2 + \param\imin^T\Gamma\param\imin} \Gamma \param\imin \param\imin^T\Gamma
            \end{bmatrix}\projm_i
            \right)
    \end{align*}
The $o_n(1)$ term is due to strong law of large numbers. From this we know that, with probability 1
    \begin{align*}
        &\lim_{c\to \infty}\lim_{n \to \infty} \sup_{\ltwo{h} < 1}\frac{\finfo^n(\param_0 + ch / \sqrt{n})}{n} \\&\qquad\qquad= \sum_{i=1}^m \frac{1}{\sigma^2 + \param\imin^T\Gamma\param\imin}\left( Q_i + \projm_i^T\begin{bmatrix}
            0 & 0\\
            0 & \frac{2}{\sigma^2 + \param\imin^T\Gamma\param\imin} \Gamma \param\imin \param\imin^T\Gamma
        \end{bmatrix}\projm_i
        \right) =: \rho
    \end{align*}
    Applying \Cref{thm:van-trees} with $\psi \R^d \to \R^d$ as the identity function $\psi(x) = x$ gives the full-feature minimax lower bound. Applying \Cref{thm:van-trees} with $\psi\R^d \to \R^{d_i}$ as $\psi(x) = T_i x$ gives the missing-feature minimax lower bound. 

    One final transformation remains to get the form of this lower bound to match the one in the theorem statement. We know that from Cauchy-Schwartz that for all $u\in\R^{d - d_i}$
    \begin{align*}
        \frac{u^T \Gamma \param\imin \param\imin^T \Gamma u}{\param\imin^T \Gamma \param\imin} = \frac{(u^T \Gamma^\half \Gamma^\half \param\imin)^2}{\param\imin^T \Gamma \param\imin} \leq u^T \Gamma u.
    \end{align*}
    Using this fact and the definition of $\Gamma$ and $Q_i$ we have that 
    \begin{align*}
        \frac{1}{\sigma^2 + \param\imin^T\Gamma\param\imin}\left( Q_i + \projm_i^T\begin{bmatrix}
            0 & 0\\
            0 & \frac{2}{\sigma^2 + \param\imin^T\Gamma\param\imin} \Gamma \param\imin \param\imin^T\Gamma
        \end{bmatrix}\projm_i \right) \preceq  \frac{2n}{\sigma^2 + \param\imin^T\Gamma\param\imin} \Sigma.
    \end{align*}
    Using this bound gives our final result.





\subsection{Proof of \Cref{cor:lower-bounds-comparable}}
The existence of the limits is a consequence of strong law of large numbers. To further show the inequality in the limit, we note that
\begin{align*}
	& \frac{1}{m} \sum_{i=1}^m \prn{\frac{\scov }{\sigma^2 + \param\imin^T\Gamma\imin \param\imin}  - \frac{T_i^\top \Sigma\iplus T_i}{\sigma^2 + \param\imin^T\Gamma\imin \param\imin}} = \frac{1}{m} \sum_{i=1}^m \frac{\Pi_i^\top \begin{bmatrix}
			0 & 0\\
			0 & \Gamma\imin
		\end{bmatrix} \Pi_i}{\sigma^2 + \param\imin^T\Gamma\imin \param\imin} \nonumber \\
	& \preceq \frac{1}{m} \sum_{i=1}^m \frac{\Pi_i^\top \begin{bmatrix}
			0 & 0\\
			0 & \Gamma\imin
		\end{bmatrix} \Pi_i}{\sigma^2} \preceq \frac{1}{m} \sum_{i=1}^m \frac{\Pi_i^\top \begin{bmatrix}
			0 & 0\\
			0 & \scov\imin
		\end{bmatrix} \Pi_i}{\sigma^2}  \to \frac{p \mathrm{diag}(\scov) + p^2 (\scov - \mathrm{diag}(\scov))}{\sigma^2},
\end{align*}
where the last step holds with probability one by strong law of large numbers. This is true as by our random missing model, $\scov_{ij}$ is not observed with probability $p$ if $i=j$, and $p^2$ if $i \neq j$. We can further derive that
\begin{align*}
	& \frac{p \mathrm{diag}(\scov) + p^2 (\scov - \mathrm{diag}(\scov))}{\sigma^2} \preceq \frac{p\lambda_1(\Sigma) I}{\sigma^2} \preceq \frac{p \kappa \scov}{\sigma^2} \preceq \frac{p\lambda_1(\Sigma) I}{\sigma^2} \nonumber \\
	& \stackrel{\mathrm{(i)}}{\preceq} \frac{p \kappa (\sigma^2 + \norm{\param}_\scov^2)}{\sigma^2} \frac{1}{m} \sum_{i=1}^m \frac{\scov }{\sigma^2 + \param\imin^T\Gamma\imin \param\imin}.
\end{align*} 
In (i), we make use of the fact that
\begin{align*}
	\scov \succeq \Pi_i^\top \begin{bmatrix}
		0 & 0\\
		0 & \Gamma\imin
	\end{bmatrix} \Pi_i
\end{align*} 
and therefore $\norm{\param}_\scov^2 \geq \norm{\param\imin}_{\Gamma\imin}^2$. By our choice of $p \leq \half\kappa^{-1} (1 + \norm{\param}_\scov^2 / \sigma^2)^{-1}$, we can conclude that
\begin{align*}
	\lim_{m \to \infty} \frac{1}{m} \sum_{i=1}^m \frac{T_i^\top \Sigma\iplus T_i}{\sigma^2 + \param\imin^T\Gamma\imin \param\imin} \succeq \lim_{m \to\infty} \frac{1}{m} \sum_{i=1}^m \frac{\scov/2}{\sigma^2 + \param\imin^T\Gamma\imin \param\imin}.
\end{align*}
\end{document}