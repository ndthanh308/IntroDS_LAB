\section{Conclusion}
\label{sec:discussion}

This paper presents an analysis of the impact of heterogeneous user rating behavior and personalized algorithms on the collection of ratings, using data from Piki, a music discovery platform. 
We analyzed the results of a randomized controlled trial (RCT) on the Piki platform.
The users were randomly assigned to different treatment groups, each with a different waiting time before they could "like" a song.
Our research indicates that carefully designing the user interface can influence user behavior in a way that mitigates rating inflation. This leads to more informative ratings and affects downstream personalized recommendations. %
Our work has several implications for platform rating system design and future work. First, and most simply, item quality estimates as a function of ratings should not be viewed as sole functions of item quality; as we show, heterogeneous user behavior and personalized recommendation dynamics also play a large role. Second, platforms should consider several interventions to mitigate inflation. Here, we study rating interface changes, but algorithmic post-processing could also play a role. %

















\textbf{Acknowledgements:} This work is partially sponsored by Meta Research Award, Jacobs Technion-Cornell Institute at Cornell Tech, and by Zuckerman Foundation.









