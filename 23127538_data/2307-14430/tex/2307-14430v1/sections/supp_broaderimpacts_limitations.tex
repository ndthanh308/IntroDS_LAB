\section{Broader Impacts and Limitations}


\paragraph{Broader Impacts} 
As more LMs are developed, a key criteria for their adoption and utility is if they exhibit a wide array of useful capabilities, such as generating harmless content, summarizing essays, and being conversational with the user. While improvements in other parts of the LM development pipeline such as training and architecture are important, many recent advances in building LMs with a wide array of useful capabilities have come from the data itself~\cite{palm2techreport, alpaca, vicuna2023, koala_blogpost_2023, mpt}. Our work is fundamental in investigating how LMs learn and how to select data to learn skills more efficiently. However, we recognize that data selection methods can always be utilized to optimize for particular skills that may be considered malicious or negatively target or exclude specific groups~\cite{bai2022training}. Furthermore, pre-trained LMs have been found to have various biases~\cite{kirk2021bias, nadeem2021stereoset, liang2021towards, bommasani2021opportunities}.

\paragraph{Limitations} The skills graph can either be provided (e.g., using a knowledge graph) or learned. Our work learns the skills graph using Algorithm~\ref{alg:bruteforce_graph} or Algorithm~\ref{alg:approximate_graph}, which requires initial training runs on pairs of skills or each skill, respectively. This can be made more efficient by performing these training runs on a smaller model and for fewer number of steps, but tradeoffs here have yet to be thoroughly investigated. \name also assumes that the ordered skill set is provided; as discussed in sections~\ref{sec:def} and~\ref{sec:skill_recovery}, it is challenging to recover ordered skill sets simply via metadata attributes or embedding clustering. Otherwise, the best way to sample over collections of skills that form a complete or empty graph is random or stratified sampling with no ordering to exploit. Our loss-based clustering approach presented in section~\ref{sec:skill_recovery} demonstrates that grouping by losses can provide an explanation for how skills are defined over data. An important direction for future work is to use such a clustering approach or other unsupervised algorithms in an end-to-end pipeline for skill discovery, skill graph learning, and data selection based on such skills.