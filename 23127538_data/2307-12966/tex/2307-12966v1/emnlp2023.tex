% This must be in the first 5 lines to tell arXiv to use pdfLaTeX, which is strongly recommended.
\pdfoutput=1
% In particular, the hyperref package requires pdfLaTeX in order to break URLs across lines.

\documentclass[11pt]{article}

% Remove the "review" option to generate the final version.
\usepackage{EMNLP2023}

% Standard package includes
\usepackage{times}
\usepackage{latexsym}

% For proper rendering and hyphenation of words containing Latin characters (including in bib files)
\usepackage[T1]{fontenc}
% For Vietnamese characters
% \usepackage[T5]{fontenc}
% See https://www.latex-project.org/help/documentation/encguide.pdf for other character sets

% This assumes your files are encoded as UTF8
\usepackage[utf8]{inputenc}

% This is not strictly necessary and may be commented out.
% However, it will improve the layout of the manuscript,
% and will typically save some space.
\usepackage{microtype}

% This is also not strictly necessary and may be commented out.
% However, it will improve the aesthetics of text in
% the typewriter font.
\usepackage{inconsolata}
\usepackage{url}
\usepackage{amssymb}
\usepackage{graphicx}
\usepackage{booktabs}
\usepackage{CJKutf8}
\usepackage{graphicx}
\usepackage{subfigure}
\usepackage{float}
\usepackage{amsmath}
\usepackage{amssymb}
\usepackage{multirow}
\usepackage{booktabs}
\usepackage{url}
\usepackage{array}
\usepackage{enumitem}
\usepackage{algorithm}
\usepackage{algorithmic}
\usepackage{pifont}
\usepackage{bm}
\usepackage{lipsum}
\usepackage{xcolor}
\usepackage{makecell}
\usepackage{CJKutf8}
\usepackage{tikz}
\usepackage[edges]{forest}
\definecolor{hidden-draw}{RGB}{20,68,106}
\definecolor{hidden-pink}{RGB}{255,245,247}


\newcommand{\cmark}{\ding{51}}%
\newcommand{\xmark}{\ding{55}}%
% If the title and author information does not fit in the area allocated, uncomment the following
%
%\setlength\titlebox{<dim>}
%
% and set <dim> to something 5cm or larger.

\title{Aligning Large Language Models with Human: A Survey}

\author{\textbf{Yufei Wang}, \textbf{Wanjun Zhong}, \textbf{Liangyou Li}, \textbf{Fei Mi},  \textbf{Xingshan Zeng}, \textbf{Wenyong Huang} \\ \textbf{Lifeng Shang}, \textbf{Xin Jiang}, \textbf{Qun Liu} \\
Huawei Noah's Ark Lab
\\
\{wangyufei44,zhongwanjun1,liliangyou,mifei2,zeng.xingshan,wenyong.huang\}@huawei.com \\
\{Shang.Lifeng,Jiang.Xin,qun.liu\}@huawei.com
}

\begin{document}
\maketitle
\begin{abstract}
    

Large Language Models (LLMs) trained on extensive textual corpora have emerged as leading solutions for a broad array of Natural Language Processing (NLP) tasks. 
Despite their notable performance, these models are prone to certain limitations such as misunderstanding human instructions, generating potentially biased content, or factually incorrect (hallucinated) information. Hence, aligning LLMs with human expectations has become an active area of interest within the research community.
This survey presents a comprehensive overview of these alignment technologies, including the following aspects. 
(1) \textbf{Data collection}:  the methods for effectively collecting high-quality instructions for LLM alignment, including the use of NLP benchmarks, human annotations, and leveraging strong LLMs.
(2) \textbf{Training methodologies}: a detailed review of the prevailing training methods employed for LLM alignment. Our exploration encompasses Supervised Fine-tuning, both Online and Offline human preference training, along with parameter-efficient training mechanisms.
(3) \textbf{Model Evaluation}: the methods for evaluating the effectiveness of these human-aligned LLMs, presenting a multifaceted approach towards their assessment.
In conclusion, we collate and distill our findings, shedding light on several promising future research avenues in the field. This survey, therefore, serves as a valuable resource for anyone invested in understanding and advancing the alignment of LLMs to better suit human-oriented tasks and expectations. An associated GitHub link collecting the latest papers is available at~\url{https://github.com/GaryYufei/AlignLLMHumanSurvey}.
\end{abstract}
% \begin{abstract}
% Large Language Models (LLMs), trained on large textual data, have demonstrated superior performance on various Natural Language Processing (NLP) tasks. However, these LLMs do not necessarily follow given instructions and could generate hallucinated facts and biased contents. Recent NLP research community has witnessed rapid growth in aligning these LLMs with human. 
% This survey provides an overview towards these alignment technologies for LLMs. We first focus on how to effectively collect high-quality instructions for alignment via NLP benchmarks, human annotations and existing strong LLMs. We further review the training technologies for LLMs alignment, including Supervised Fine-tuning, Online and Offline human preference training and parameter-efficient training. Next, we show how to comprehensively evaluate these aligned LLMs. Finally, we summarize our findings as well as several promising research directions.
% \end{abstract}


% Figure environment removed

\section{Introduction}
Automatic 3D reconstruction of clothed humans using image inputs has gained increasing significance due to its potential applications in a wide array of AR/VR scenarios. High-fidelity reconstructions typically depend on sophisticated capture systems, which are developed with dense camera arrays~\cite{collet2015high,joo2015panoptic,joo2018total}, programmable light-stages~\cite{Vlasic2009, guo2019relightables}, and depth sensors~\cite{newcombe2011kinectfusion,DoubleFusion,BodyFusion,dou2016fusion4d,newcombe2015dynamicfusion}. However, stringent capture environments equipped with complex hardware pose significant challenges for consumer-level applications.


In this context, considerable research effort has been dedicated to developing methods that allow for more flexible capture configurations, such as utilizing a few RGB inputs. Among these works, learning implicit functions \cite{iccv2020PIFu, saito2020pifuhd, hong2021stereopifu} has proven effective in achieving highly detailed reconstructions by integrating the advancements of deep neural networks. These methods employ large multi-layer perceptrons (MLPs) to predict the occupancy probability or truncated signed distance function (TSDF) value of every queried 3D point based on its associated local feature, which is extracted from images. They can recover a continuous surface at arbitrary resolutions without topology restrictions.


However, in typical MLP-based implicit networks, the occupancy or TSDF value at each location is solved independently with planar image features, rendering them less capable of addressing challenging cases such as occlusions. Consequently, these methods suffer from generalization and robustness issues, particularly when tackling strong occlusions caused by large motion or multiple interacting humans. 
Some follow-up studies  \cite{zheng2021deepmulticap,zheng2021pamir,huang2020arch} utilize an extra geometric model, SMPL~\cite{Loper2015}, to improve robustness by introducing strong shape priors. 
Their success typically relies on the assumption of geometrical similarity \cite{huang2020arch} between the shape prior and target reconstruction, making them intractable for handling complex cases with loose clothes and sensitive to errors in SMPL model fitting.



%\ping{this paragraph sounds like `TSDF is better than MLP/SMPL, and we use TSDF to solve the problem'. But in Sec 3, we are telling a different story, saying `MLP needs a 3D convolutional encoder'. We need to make these two sections consistent.}\sicong{I think in this paragraph we claim that the TSDF}


%We opt for Trucated Signed Distance Funtion (TSDF) volumetric representations as they are naturally suitable for convolution operations, which have shown remarkable performance for learning hierarchical features on 2D visual perception tasks \cite{SunXLW19}. 
%Meanwhile, TSDF also describes the gradual geometry change around shape surface, which is not reflected by occupancy volume. 

We instead revisit the 3D volumetric representation and resort to 3D convolutional neural networks (CNNs) for feature learning, due to their impressive performance in feature learning and the ability to incorporate spatial context. However, volumetric methods and 3D convolution involve discretization, which might raise concerns regarding whether a discretized volume can preserve subtle geometric details as continuous representations learned in implicit functions. We investigate the relationship between volume resolution and quantization error on synthetic data by converting target mesh objects to TSDF volumes, as shown in Figure~\ref{fig:quantization_error}. We observe that the quantization errors are significantly reduced by increasing volume resolution and become nearly negligible when reaching a relatively high resolution (e.g., 512 or higher). In other words, achieving fine-detailed reconstruction is not supposed to be restricted by the use of volume representations as long as a proper volume resolution is utilized. Therefore, we present a method with high-resolution feature volumes, e.g., 256 and 512, while traditional volumetric methods \cite{varol18_bodynet,gilbert2018volumetric} are often limited to much lower resolutions, such as 32 or 128.



On the other hand, an increase in volume resolution may lead to a cubic growth of memory overhead \cite{8100085}. Reducing memory costs while guaranteeing the granularity of volumetric representations is necessary for pursuing high-quality reconstruction. Thus, we adopt a coarse-to-fine approach and cull away irrelevant voxels to build a sparse high-resolution feature volume. At the coarse level, the network computes an initial TSDF by applying a U-Net with sparse 3D CNN \cite{3DSemanticSegmentationWithSubmanifoldSparseConvNet} on the sparse feature volume, which is carved by a visual hull. Through our experiments, it turns out that more than 95\% of the volume grids are discarded by the visual hull culling, making the sparse 3D CNN efficient. At the fine level, the network focuses on a narrow band near the zero-level set of the initial TSDF and discretizes the narrow band with smaller voxels. By employing this narrow-band culling, we further shrink the sampling space, resulting in a relatively small range of grid numbers (usually 300K--500K in our experiments) even with a high volume resolution of 512. The remaining voxels in the narrow band are associated with features that fuse high-frequency information from the computed normal maps upon the low-frequency shape from the coarse level to compute the TSDF at high resolution. The final mesh is then extracted from the TSDF using the Marching-Cube algorithm ~\cite{Lorensen87marchingcubes}.
% Different from the u-net sturcture to preserve global topology context, we then apply a shallow 3dcnn to compute the final TSDF $D_{final}$ which contain more local geometry detail.




% \ping{this paragraph can be expanded. It is an important contribution and often ignored by other works. stress on the novel idea of regressing blending weights instead of colors}

In addition to geometry, high-quality mesh texture is also a crucial factor contributing to visual appearance. Directly computing a color field in 3D space, as in \cite{iccv2020PIFu}, struggles to capture high-frequency texture details, while the neural radiance field (NeRF) \cite{yu2020pixelnerf} or the DoubleField~\cite{shao2022doublefield} require expensive per-instance optimization and are often unstable for sparse input images. In contrast, we adopt an image-based rendering approach to compute a texture atlas map, which is efficient and widely supported in existing computer graphics tools. 
Specifically, we compute a blending weight at each 3D point on the mesh surface to determine its color as a weighted average of the colors at its image projections. The blending weights can be computed at a relatively coarse resolution, e.g., 512 volume resolution in our case, and leave texture details to the high-resolution images, such as 1K or 2K. Unlike previous methods that generate blurry texturing results under sparse input, our method generalizes well on both synthetic and real data with just a few input views. 
Figure~\ref{fig:teaser} shows two examples reconstructed by our method. Despite the challenging garment, pose, and occlusion, our method recovers faithful shape, normal, and texture on the right.

%with a wide variety of poses and clothing styles, and it is also adaptive to handle input image with arbitrary resolutions.
%\sicong{For this concern we claim that when the resolution of dicretized volume meets certain threshold (which is 256 in our experiment), the quantization error can be neglected.} 



In summary, the main contributions of this paper are as follows:
\begin{itemize}
\vspace{-0.1in}
  \item 
  We revisit the 3D volumetric representation and demonstrate that it can support clothed human reconstruction with equal or even better performance compared to implicit representation. 
  \item 
  We develop a memory and computation-efficient method for high-resolution volumetric reconstruction using sophisticated sparse 3D CNN, coarse-to-fine estimation, and voxel culling by visual hull and narrow bands. 
  \item 
  We introduce a novel method to compute a texture atlas map, which captures rich appearance details from high-resolution input images.
  \item 
  We achieve impressive results on standard benchmark datasets Twindom and MultiHuman, significantly reducing the point-2-surface (P2S) precision to approximately 0.2cm from just six input views, with more than $50\%$ error reduction compared to the state-of-the-art methods, including DoubleField~\cite{shao2022doublefield} and PIFuHD~\cite{saito2020pifuhd}.
\end{itemize}

\section{Alignment Data Collection}
\label{instructioncollecting}
Aligning LLMs with human expectations necessitates the collection of high-quality training data that authentically reflects human needs and expectations. For the purposes of this survey, we conceptualize an instruction as $I_k = (x_k, y_k)$, where $x_k$ denotes the instruction input and $y_k$ denotes the corresponding response. This data can be derived from an array of sources, encompassing both human-generated instructions and those generated by strong LLMs.
In this section, we summarize these methods of instruction generation and effective strategies for constructing a composite of diverse training instructions.

% previous researchers collect instructions from these sources and discuss how to effectively mix these instructions training.

\subsection{Instructions from Human}
\label{instructionfromhuman}
Human-provided instructions mainly originate from two main sources: pre-existing human-annotated NLP benchmarks and meticulously hand-crafted instructions. 

\subsubsection{NLP Benchmarks}
An intuitive starting point for data collection involves adapting existing NLP benchmarks into natural language instructions. For instance, Figure~\ref{fig:nlpinstruction} offers an example drawn from the Natural Language Inference task. Works such as PromptSource~\cite{bach-etal-2022-promptsource}, FLAN~\cite{wei2022finetuned,longpre2023flan}, and SuperNaturalInstruction~\cite{wang-etal-2022-super,mishra-etal-2022-cross} are at the forefront of this approach. 
% Figure environment removed
These benchmarks represent a substantial array of \emph{diverse and heterogeneous} NLP tasks, such as dialogue, reasoning tasks and coding tasks, unified under the framework of language instructions. In each NLP benchmark, they engage annotators to craft several natural language templates that smoothly integrate all input data into a sequential text. The objective is to enhance LLMs' capability for multi-task learning across training tasks and foster generalization for unseen tasks. OIG~\cite{OIG} also combines instructions from FLAN-like NLP benchmarks with other types of open-ended instructions, such as how-to, maths and coding instructions. Concurrently, \citet{DBLP:journals/corr/abs-2212-09689} put forth the concept of \emph{Unnatural Instructions}, utilizing LLMs to generate new templates or instances bearing resemblance to the original instructions but with notable variances. Interestingly, the authors discovered that \emph{text-davinci-002} outperforms GPT-3 in responding to these generated instructions, given that GPT-3 often devolved into repetitive or tangential outputs after providing the correct answer. This model of instruction creation is highly scalable and can yield millions of instructions effectively. Further, \citet{wang2023far} demonstrated that FLAN-style instructions considerably enhanced the reasoning capabilities of aligned LLMs.



% One straightforward source is to adapt existing NLP benchmarks into natural language instructions. Figure~\ref{fig:nlpinstruction} shows an example of the Natural Language Inference task. PromptSource~\cite{bach-etal-2022-promptsource}, FLAN~\cite{wei2022finetuned,longpre2023flan} and SuperNaturalInstruction~\cite{wang-etal-2022-super,mishra-etal-2022-cross}  are pioneering research efforts in this direction. For each NLP benchmark, they ask annotators to write multiple natural language templates that could incorporate all inputs information into a sequence of coherent text.  These benchmarks 
% represent large number of \emph{diverse and heterogeneous} NLP tasks as unified language instructions and their goal is to train LLMs to perform better multi-task learning on training tasks and generalization on unseen tasks.~\citet{DBLP:journals/corr/abs-2212-09689} propose \emph{Unnatural Instructions} which leverages LLMs to generate new templates or similar but different instances based on those constructed instructions. Interestingly, the authors find that \emph{text-davinci-002} works better than GPT-3 when answering these generated instructions because GPT-3 often degenerates into repetitions or tangents after producing the correct answer. This paradigm is easy to scale up and can effectively produce millions of instructions. \citet{wang2023far} also find that FLAN-style instructions can effectively improve the reasoning ability of aligned LLMs. 
% However, as many NLP datasets focus on a small and specific skill set, these instructions could be relatively narrow and can not be well suited for complicated needs for real-world applications (e.g., chatting).

\subsubsection{Hand-crafted Instructions}
Constructing instructions from NLP benchmarks could be effective and painless. However, as many NLP datasets focus on a small and specific skill set, which means the resultant instructions are also relatively narrow in scope. Consequently, they may fall short in catering to the complex needs of real-world applications, such as engaging in dynamic human conversation.

To combat the above issues, it is possible to construct instructions via intentional manual annotations. How to effectively design a human-in-the-loop annotation framework becomes the key issue. The Databricks company collects a 15k crowd-sourcing instruction dataset \emph{databricks-dolly-15k}~\cite{DatabricksBlog2023DollyV2} from its employees. 
Those people are instructed to create prompt / response pairs in each of eight different instruction categories, including the seven outlined in~\citet{DBLP:conf/nips/Ouyang0JAWMZASR22}, as well as an open-ended free-form category. Importantly, they are \emph{explicitly} instructed not to use external web information, as well as outputs from generative AI systems.
~\citet{Kopf2023OpenAssistantC} construct the \emph{OpenAssistant} corpus with over 10,000 dialogues using more than 13,000 international annotators. The annotation process includes a) writing initial prompts for dialogue; b) replying as an assistant or user; c) ranking dialogue quality to explicitly provide human preferences. As a result, this corpus can be used for SFT and human preference alignment training for LLMs.
~\citet{Zhang2023ChineseOI} construct high-quality Chinese instructions from existing English instruction datasets. They first translate the English instructions into Chinese, then verify whether these translations are usable. Finally, they hire annotators to 
correct and re-organize the instructions into the {task description, input, output} format in the selected corpus. ShareGPT~\footnote{\url{https://sharegpt.com/}}, which is collected by~\citet{vicuna2023}, is an interesting exploration for crowd-sourcing human-written instructions. It is a website that encourages users to upload and share their interesting ChatGPT/GPT4 conversations. Such a mechanism can effectively collect a large number of diverse and human-written instructions that likely trigger high-quality ChatGPT/GPT4 responses. Popular online QA websites, such as Stack Overflow~\footnote{\url{https://stackoverflow.com/}}, Quora~\footnote{\url{https://www.quora.com/}} and Zhihu~\footnote{\url{https://www.zhihu.com/}}, and large user-generated content databases, such as Wikipedia~\footnote{\url{https://en.wikipedia.org/}}, are all reliable sources to provide high-quality human-written prompts for this purpose.Both ~\citet{ding2023enhancing} and~\citet{DBLP:journals/corr/abs-2304-01196} propose to use these resources as the seed instructions to prompt GPT-3.5 to generate high-quality synthetic multi-turn dialogues.

\subsection{Instructions From Strong LLMs}
\label{llminstruction}
With the emergence of strong closed-source LLMs (e.g., ChatGPT/GPT4), 
it is also feasible to automate the collection process to obtain various types of synthetic instructions (e.g., single-turn, multi-turn, and multilingual instructions) by providing appropriate prompts to these LLMs. The main challenge is how to effectively prompt LLMs to generate diverse and high-quality instructions.   

% Figure environment removed


\subsubsection{Self-Instruction}

\emph{Self-Instruct}~\cite{DBLP:journals/corr/abs-2212-10560} were among the pioneers to automate the instruction collection process.
It employed the in-context learning capability of ChatGPT to generate large-scale instructions from a pre-defined set of human-annotated instructions covering diverse topics and task types, as illustrated in Figure~\ref{fig:selfinstruct}. %Operating within the in-context learning framework, \emph{Self-Instruct} prompts ChatGPT to create new task descriptions and corresponding instructions, building upon a pre-defined set of instruction.
%\citet{DBLP:journals/corr/abs-2212-10560} were the pioneers to introduce the concept of \emph{Self-Instruct}, an innovative method for automatically generating large-scale instructions from a modest pool of human-annotated instructions via the application of ChatGPT. \emph{Self-Instruct} prompts ChatGPT to devise new instructions, as illustrated in Figure~\ref{fig:selfinstruct}. Operating within the in-context learning framework, \emph{Self-Instruct} prompts ChatGPT to create new task descriptions and corresponding instructions, building upon pre-existing instruction sets.
The automatically generated instructions are followed by a quality control filtering process, and this iterative process continues until the desired data volume has been achieved. Interestingly, the researchers discovered that GPT-3~\cite{NEURIPS2020_1457c0d6}, fine-tuned with these instructions, performed better than models fine-tuned using instructions derived from NLP benchmarks SuperNI benchmark~\cite{wang-etal-2022-super} and \emph{User-Oriented Instructions}, as discussed in Section~\ref{instructionfromhuman}).
Several follow-up attempts, such as Aplaca~\cite{alpaca} and its variants~\cite{chinese-llama-alpaca} follow this 
\emph{Self-Instruct} framework.
More subsequent research efforts w.r.t. enhancing instruction diversity, quality, and complexity will be elaborated as follows.
%Following this framework, subsequent research efforts primarily concentrate on enhancing instruction diversity, quality, and complexity.

\iffalse
\citet{DBLP:journals/corr/abs-2212-10560} first propose \emph{Self-Instruct} which automatically collects large-scale instructions from a small set of human-annotated instructions via ChatGPT. \emph{Self-Instruct} prompts ChatGPT to generate new instructions. As shown in Figure~\ref{fig:selfinstruct}, \emph{Self-Instruct}  prompts ChatGPT to generate new tasks description and corresponding instructions based on the pre-existing instructions sets under the in-context learning framework. After quality control filtering, the newly generated instructions are added to the instruction set. This process iterates until the target data volume is reached. They find that GPT-3~\cite{NEURIPS2020_1457c0d6} fine-tuned on these instructions successfully outperforms the ones fine-tuned by 
the NLP benchmarks instructions (i.e., discussed in Section~\ref{nlpbenchmarkinstruction}) on the SuperNI benchmark~\cite{wang-etal-2022-super} and \emph{User-Orientated Instructions}. 
Following this framework, afterward research efforts mainly focus on improving the instruction input or output quality.
\fi


\paragraph{Improving Input Quality}
One limitation of the synthetic instructions from strong LLMs often suffer from diversity issues. For example,~\citet{jentzsch2023chatgpt} find that when prompting to generate jokes, ChatGPT only produces 25 unique joke patterns in thousands of samples. 
To improve the instruction input diversity,~\citet{DBLP:journals/corr/abs-2212-10560} propose different input and output generation strategies for different types of instructions. 
They first prompt ChatGPT to classify generated instruction into \emph{classification tasks} or \emph{non-classification tasks}. Then, they deploy output-first and input-first strategies for \emph{classification tasks} and \emph{non-classification tasks}, respectively.
Others propose to add various external information into the input prompts to enhance diversity and factuality, including Wikipedia Category Keywords~\cite{DBLP:journals/corr/abs-2304-1440}, user-generated questions on the Internet (e.g., Quora, StackOverflow)~\cite{DBLP:journals/corr/abs-2304-01196,gpt4all} and instructions from the SuperNaturalInstruction benchmark~\cite{DBLP:journals/corr/abs-2212-09689}.
\citet{yu2023large} also shows that explicitly adding meta-information (e.g., length, topics, style) into the data generation prompts can effectively remove the bias in the generated synthetic data and improve the diversity of those synthetic data.
Furthermore,~\citet{xu2023wizardlm} propose a novel \emph{Evol-Instruct} framework to obtain complex and difficult instructions gradually. 
Instead of using existing instructions to prompt LLMs to produce new instructions via \emph{in-context learning}, in \emph{Evol-Instruct}, there are five different manually-designed prompts to explicitly instruct LLMs to rewrite the existing simple instructions into complex ones using in-depth methods (i.e., including more information on particular topics) or in-Breadth methods (i.e, improving topics/information coverage). The resulting WizardLM model is ranked top in the MT-Bench~\cite{zheng2023judging} and AlpacaEval~\cite{dubois2023alpacafarm}.
\citet{luo2023wizardcoder} further expand this idea to produce complex code and programming instructions from the simple ones and propose the \emph{WizardCoder} model, which outperforms several strong commercial LLMs, e.g., Anthropic's Claude and Google's Bard.~\citet{gunasekar2023textbooks} propose to generate textbook-like instructions prompted with sufficient background knowledge to promote reasoning and basic algorithmic skills of LLMs. They find that the resulting 1.3B LLMs \emph{phi-1} successfully outperform various much larger LLMs, showing the importance of data quality.

\paragraph{Improving Output Quality}
Aside from the provision of high-quality instruction input, a critical requisite is to skillfully prompt LLMs to yield high-quality responses. The conventional method of enhancing response quality entails appending LLM prompts with additional conditions, encompassing the following facets.

\textbf{(1) Reasoning-Provoking Conditions:} ~\citet{wei2022chain} proposed the Chain-of-Thought (CoT) reasoning approach, which includes preconditions in the LLM prompts and  generation the intermediate reasoning processes for complex problems, thereby assisting LLMs in problem-solving. Inspired by CoT, ~\citet{mukherjee2023orca} developed the Orca model, which learns not only the superficial response text from LLMs, but also captures complex reasoning process signals. Specifically, they guided LLMs to respond to reasoning-intensive FLAN instructions with a series of predefined system prompts (e.g., ``think step-by-step and justify your response''), spurring LLMs (e.g., GPT4) to disclose their reasoning process information. Thanks to these advancements, the Orca model significantly outperformed several powerful open-sourced LLMs.

\textbf{(2) Hand-crafted Guiding Principles:} ~\citet{Sun2023PrincipleDrivenSO} introduced  \emph{self-alignment} framework that incorporates 16 manually devised principle rules into input prompts, thereby steering LLMs towards generating useful, ethical, and reliable responses. To augment the impact of these rules, they employed the Chain-of-Thoughts (CoT) technology~\cite{wei2022chain}, elucidating five examples to coach LLMs in discerning which rules to implement prior to generating actual response contents.

\textbf{(3) Role-playing Conditions:} ~\citet{DBLP:journals/corr/abs-2304-10453} devised a method to generate a set of role profiles using a blend of ChatGPT and manual efforts. They created seed instructions for each role profile and applied \emph{self-instruction} to the combination of role profiles and instructions to obtain nuanced responses from LLMs.~\citet{xu2023expertprompting} proposed a two-stage instruction response framework in which an expert profile is initially generated based on the instructions to be answered, followed by using both the expert profile and actual instructions to prompt LLMs for high-quality responses. 

\textbf{(4) Difficulty-monitoring Conditions:} 
~\citet{Jiang2023LionAD} proposed monitoring the quality of instruction response based on external LLM-based evaluations. They first fine-tune foundational LLMs with instruction data to obtain ``student LLMs''. Then, for each of training instruction, they gather responses from both teacher LLMs (e.g., ChatGPT) and student LLMs and prompted LLMs to conduct pairwise evaluation on the quality of both responses. Instructions are retained only when the student LLMs' response falls short of that from the teacher LLMs.
% Apart from generating high-quality instruction input, it is also necessary to appropriately prompt LLMs to produce high-quality responses. The common way to improve response quality is to prompt LLMs with additional conditions in the following aspects. 
% \textbf{(1) Hand-crafted guiding principles:}
% ~\citet{Sun2023PrincipleDrivenSO} propose a novel \emph{self-alignment} framework that adds 16 manually-designed principle rules into the input prompts to guide LLMs to produce helpful, ethical, and reliable responses. 
% To further enhance the implication of these rules, they apply Chain-of-Thoughts (CoT) technology~\cite{wei2022chain}, with 5 demonstrate examples, to instruct LLMs to determine which rules to use before generating the actual response contents.
% \textbf{(2) Reasoning-provoking condition:}
% \citet{wei2022chain} proposes Chain-of-Thought (CoT) reasoning to prompt LLMs with pre-conditions and asks it to generate intermediate reasoning processes for complex problems, and facilitate LLMs in problem-solving.
% Following CoT, ~\citet{mukherjee2023orca} propose the Orca model that not only learns the surface response text from LLMs but also their complex reasoning process signals. 
% Specifically, they prompt LLMs to respond to reasoning-intensive FLAN instructions with a set of pre-defined system prompts (e.g., `` think step-by-step and justify your response'') that encourage the LLMs (e.g., GPT4) to produce their reasoning process information. With these advances, the Orca model successfully outperforms several strong open-sourced LLMs.
% \textbf{(3) Role-playing condition:}
% ~\citet{DBLP:journals/corr/abs-2304-10453} propose to produce a set of role profiles using a combination of ChatGPT and manual efforts. They then manually build seed instructions for each role profile and apply \emph{self-instruction} over the combination of role profiles and instructions to obtain fine-grained responses from LLMs.
% ~\citet{xu2023expertprompting} propose a two-stage instruction response framework where they first generate an expert profile based on the to-be-answer instructions, then both the expert profile and the actual instructions are used to prompt LLMs to produce the high-quality responses.
% ~\citet{Jiang2023LionAD}
% propose to keep track of instruction response quality based on the external LLM-based evaluation. For each instruction, they collect responses from both teacher LLMs (e.g., ChatGPT) and student LLMs (e.g., LLaMA) and prompt LLMs to evaluate the quality of both responses. The instructions are only retained when the response of student LLMs is worse than the one from teacher LLMs.

\subsubsection{Multi-turn Instructions}
In previous sections, we mainly focus on collecting synthetic single-turn instructions. However, LLMs well aligned with human should be capable to interact with users in a dialogue-based setting. To achieve this goal, some research efforts attempt to collect synthetic multi-turn instructions from strong LLMs.
When aligning LLaMA with human, Vicuna~\cite{vicuna2023} leverage instructions from ShareGPT which is website hosting interesting human-LLMs joint conversations. However, ShareGPT requires large volumes of users to upload their conversations.
~\citet{DBLP:journals/corr/abs-2304-01196} propose a novel Self-Chatting framework where questions from popular QA websites are used as the starting topics, then Chat-3.5 is prompted to chat with itself about this question in a four-turn dialogue.~\citet{DBLP:journals/corr/abs-2303-17760} propose \emph{CAMEL}, a ``role-playing'' framework where a human annotators first provide a topic, then LLMs are separately prompted to be ``AI Users'' and ``AI Assistants'' to discuss about this topic.~\citet{DBLP:journals/corr/abs-2304-07854} take a step further and prompt LLMs to first determine the conversation topic and then ask LLMs to chat with themselves to produce dialogue corpus.~\citet{selfee2023} propose a novel revision-based multi-turn dialogue corpus. Specifically, after instructions and initial responses, they further prompt LLMs to generate feedback and the revised version of responses if necessary. They use this dataset to train the \emph{SelFee} model and show that \emph{SelFee} can effectively improve its own answers when prompted to do so without any external guidance. The UltraLLaMA model~\cite{ding2023enhancing} leverages a wide range of real-world information, including (a) real-world knowledge from LLMs and Wikipedia; (b) various text creation tasks; (c) high-quality textual corpus, to produce initial questions and instructions that guide LLMs to generate diverse and high-quality multi-turn dialogues.

\subsubsection{Multilingual Instructions}
The above-generated instructions or dialogues are mostly based on English. To align LLMs with human who speak other languages, it is urgent and essential to expand the existing English resources into Multilingual ones. One straightforward idea is to translate instruction inputs and outputs into the target languages.
~\citet{DBLP:journals/corr/abs-2304-10453} propose two translation strategies: \emph{(a)} Post-answering which first translates the instruction inputs into the target language and then prompts strong LLMs to answer it. This could potentially preserve the specific culture patterns embedded in the target languages, but the output quality may be low as existing strong LLMs are often English-dominated; \emph{(b)} Post-translating which first prompts strong LLMs to respond the instructions in English, then translate both inputs and outputs. This approach could obtain high-quality output text, but lost the specific culture information.~\citet{bactrian} follow the \emph{Post-answering} strategy to construct instruction data for 52 popular languages using Google-Translate, then use these data to fine-tune LLaMA using the LoRA technology. An alternative solution is to mix several langauges in a multi-turn dialogue. BayLing~\cite{Zhang2023BayLingBC} introduces a set of multi-turn \emph{interactive translation} instructions to simultaneously improve multilingual and instruction-following ability for LLMs. Specifically, each multi-turn instruction is essentially a translation task where users first ask LLMs to translate a sentence to another language, then the users gradually add additional requirements (e.g., could you only use 10 words?). This process naturally connects different languages as well as human preferences with LLMs. We also summarize how to effectively adapt English-oriented LLMs to other languages in Appendix~\ref{otherlanguageLLMs}.



\subsection{Instruction Data Management}
\label{datamanagement}
As discussed above, there are extensive approaches focusing on generating high-quality instructions from different sources. 
Naturally, it becomes critical to effectively manage all of these instruction data in the LLMs alignment.

\paragraph{Instruction Implications}
Several studies focus on the implications of instruction data. ~\citet{DBLP:journals/corr/abs-2304-07854} demonstrate that an increment in the total count of training instructions can be advantageous for standard NLP tasks (e.g., information extraction, classification, Closed QA, summarization). Yet, it bears negligible influence on complex reasoning tasks such as Math, Code, CoT, and Brainstorming. Intriguingly, ~\citet{muennighoff2023scaling} discover that adding approximately 50\% of programming instructions not only leaves unaffected the general conversational performance but also enhances the reasoning prowess of LLMs. In parallel,~\citet{ghosal2023flacuna} observe that integrating FLAN-style instructions with synthetic instructions from ChatGPT/GPT-4 effectively enhances LLMs' reasoning and problem-solving capacity.

\citet{wang2023far} conduct a comprehensive analysis of the impacts of various instructions derived from different sources on factual knowledge, reasoning, coding, multilingual, and open-ended scenarios. They also reveal that instructions pertaining to CoT and Coding are vital for augmenting the reasoning capability of LLMs. Additionally, they ascertain that different instructions can affect different LLM capabilities. Therefore, a composite of all instruction types empowers the corresponding LLMs to reach their better overall performance, hinting at the need for more advanced instruction collection techniques and technologies.

% There are some works that attempt to understand the implication of instruction data.~\citet{DBLP:journals/corr/abs-2304-07854} report that increasing the total number of training instructions is beneficial for common NLP tasks (e.g., information extraction, classification, Closed QA, summarizing), but almost has no effect on complex reasoning tasks (i.e., Math, Code, CoT and Brainstorming). Interestingly,~\citet{muennighoff2023scaling} find that adding around 50\% of programming instructions 
% not only has no negative effect on common chatting performance but also improves the reasoning capability of LLMs. Similarly,~\citet{ghosal2023flacuna} find that mixing the FLAN-style instructions with ChatGPT/GPT-4 synthetic instructions could effectively improve LLMs' reasoning and problem-solving ability. ~\citet{wang2023far} conduct a comprehensive investigation on the implication of different instructions from different sources on factual knowledge, reasoning, coding, multilingual and open-ended scenarios. They also discover that instructions with CoT and Coding are critical for improving LLMs reasoning capability. Furturemore, they find different instructions could improve LLMs' different capabilities. Thus, combining all types of instructions enable the corresponding LLMs to achieve best overall performance, suggesting more advanced instruction collection and technologies.




% ~\citet{DBLP:journals/corr/abs-2304-10453} propose to use both single-turn instructions from strong LLMs (e.g., ChatGPT) and dialogue-based data from ShareGPT, which aligns LLMs with human and to improve their conversational skills. The resulting LLM successfully outperforms several Chinese 7B LLMs. However,~\citet{DBLP:journals/corr/abs-2304-07854} demonstrate that this type of strategy cannot improve the quality of generated outputs under the GPT-4 evaluation framework.


\paragraph{Instruction Quantity}
Another critical question in instruction data management is the optimal quantity of instruction data required for effective LLM alignment. ~\citet{alshikh2023becoming} address this question by introducing a novel early-stopping criterion known as \textbf{IFS}. The premise of \textbf{IFS} rests on the observation that, given an input textual prefix, foundational LLMs typically predict ensuing tokens and generate "continuation-like" outputs, while fully instruction-tuned LLMs interpret the input prefix as questions, thereby generating "answer-like" outputs. \textbf{IFS} is quantified as the proportion of "answer-like" outputs within all its outputs given the instructions. The researchers train an external classifier to discriminate between "continuation-like" and "answer-like" outputs, concluding that LLaMA necessitates approximately 8K instructions to achieve a high IFS score. More instructions could potentially induce a semantic shift in the foundational LLMs.~\citet{zhou2023lima} similarly discern that merely 6K high-quality instructions suffice to align with human preferences.
Motivated by these findings, researchers are investigating high-quality instruction selection.~\citet{cao2023instruction} aim to identify predictive features of high-quality instructions. Initially, they extract representative features from the instruction dataset, then utilize these instructions to fine-tune LLMs. The feature importance is based on the model's performance. Their experiments demonstrate the better performance of LLMs trained on the resultant instructions.
Differently,~\citet{chen2023alpagasus} propose using ChatGPT to directly assess the quality of instructions by assigning scores. They report that the LLM trained on the top 9K instructions notably outperforms those trained on the complete set of 52K Alpaca instructions.
% Another important problem in instruction data management is how much instruction data is succfient for LLMs alignment.~\citet{alshikh2023becoming} provide an answer to this question via a novel early-stopping criteria \textbf{IFS}. The intuition behind \textbf{IFS} is that given an input textual prefix, the foundation LLMs tend to predict the next tokens and produce ``continuation-like''
% outputs, while fully instruction-tuned LLMs should treat the input prefix as questions and tend to produce ``answer-like'' outputs. \textbf{IFS} is defined as the ratio of  ``answer-like'' outputs in all of its outputs when prompting
% the instructions. They train an external classifier to distinguish between ``continuation-like'' and ``answer-like''. They find that LLaMA only requires around 8K instructions to reach high-level of IFS score. More instructions could result in a semantic shift to the foundation LLMs.~\citet{zhou2023lima} also find that only 6K high-quality instructions are sufficient to align with human preferences. Motivated by these findings, there are some research focus on the high-qaulity instruction selection.~\citet{cao2023instruction} propose to find out predictive features for high-quality instructions. They first extract several representative features for the instructions dataset and then use these instructions to fine-tune LLMs. The feature importance is then calculated based on these model performance. The experiments show the superior performance of LLMs trained on the resulting instructions.~\citet{chen2023alpagasus}, on the other hand, propose to leverage ChatGPT to directly evaluate the instruction quality by assigning scores. They find that the LLM trained on the top 9K instructions outperform than the ones train on the full 52k Alpaca instructions.









\noindent The network was written in Python~3.10~\cite{python} using Pytorch~1.13~\cite{pytorch}, and was trained on a single Nvidia Titan A100. 
The seed was set to $42$, and the CUDNN backend was set to deterministic mode.
The network was trained using the ADAM optimizer \cite{adamOpt2015} with a learning rate of $10^{-4}$. The learning rate was halved on a validation loss plateau of more than 3 epochs. The network was run for 60 epochs with batches of 16, meaning that each epoch was roughly 800 iterations. Early stopping was applied for a validation loss plateau of 8 epochs. 
The weights were initialized using the orthogonal scheme \cite{orthoInit2014} with a scaling of $10^{-1}$.
The hyperparameter search was run twice with different seeds (42 and 24), and the best results were chosen.

Multiframes were simulated by randomly sampling homographies for each frame in the dataset, creating different views of the same frame.
The inverse homographies were used to register all of the views toward the original frame, which was set as the pivot frame $\mathcal{I}$. 
The sampled homographies either created a random walk from one side of the temperature map to the other, or a hover above a random point in the frame. 
The overlaps between the different views were randomly set between $60\%$ and $80\%$, similar to a UAV flight scenario, as seen in \cref{sec:results:realdata}.
Homography and frame warping was implemented with the package Kornia~v0.67.

Imperfect registration was simulated by randomly adding perturbations to the inverse homographies: random translation of up to $\pm2$ pixels and noise from the distribution $\mathcal{N}(0, 5\cdot10^-5)$ to the perspective elements of the homography (commonly known as $h_{31}, h_{32}$).
Random horizontal and vertical flips, and $90^\circ$ rotations were applied to the frames before the homography sampling.

The gray-level frames were cropped to $128\times 128$ patches before entering the network. For validation, a constant cropping was applied around the middle of the frame, and no other augmentations were applied.

Random Gaussian noise with $\sigma^2=5$ gray levels and FPN were generated for each frame (\cref{sec:methods:data}). FPN was generated as:
\begin{equation}\label{eq:methods:fpn}
    \begin{bmatrix}
        1 \\ \vdots  \\ 1
    \end{bmatrix}_{h\times1}\cdot
    \begin{bmatrix}
        U[u_{\min}, u_{\max}] \\ \vdots  \\ U[u_{\min}, u_{\max}]
    \end{bmatrix}^T_{1\times w}
\end{equation} where $U$ is uniform distribution. $u_{\min}, u_{\max}$ were chosen as $u_{\min}=0.9, u_{\max}=1.01$. 
The Gaussian noise and FPN were only generated once for each frame and used throughout the entire validation process for reproducibility of results between experiments.

Normalization to range [0,1] was applied to both the temperature map and the gray-level frame.
Throughout the training and validation sets, the maximal and minimal values of the temperature maps and the maximal and minimal values of the gray-level frames were obtained.
The normalization was applied on the temperature maps as:
\begin{equation}\label{eq:methods:norm_t}
    \Bar{X} = \frac{X - X_\text{min}}{X_\text{max}-X_\text{min}}
\end{equation}
where $\Bar{X}$ is the normalized input and $X_\text{min}, X_\text{max}$ are the minimal and maximal temperatures, respectively, over all datasets.
Normalization for the gray-level frames was applied as:
\begin{equation}\label{eq:methods:norm_frame}
    \Bar{I}(\tamb) = \frac{I(\tamb) - I_\text{min}}{I_\text{max}-I_\text{min}}
\end{equation}
where $\Bar{I}$ is the normalized gray-level frame and $I_\text{min},I_\text{max}$ are the minimal and maximal gray levels, respectively, over all datasets.

The following pipeline summarizes the creation of samples for the network.
First, an accurate temperature map is sampled from the dataset. $\nFrames$ homographies are randomly sampled and applied to the temperature map to create an overlapping burst of frames.
The model described in \cref{sec:methods:synthData} is applied to each frame in the burst to turn it into a gray-level frame \cref{eq:methods:estimateFrame}.
The same FPN is applied to all frames in the burst \cref{eq:methods:fpn}, and random noise is applied to each frame in the burst separately.
Finally, normalization is applied to the ambient temperature \cref{eq:methods:norm_t} and overlapping gray-level frames \cref{eq:methods:norm_frame}, and both are passed to the network.


\section{Alignment Evaluation}
\label{eval}
After collecting instructions and training LLMs on these instructions, we finally consider the evaluation for alignment quality. In this section, we will discuss benchmarks used for evaluation in Section~\ref{evalbenchmark} and the evaluation protocols in Section~\ref{evalparadigm}.

\subsection{Evaluation Benchmarks}
\label{evalbenchmark}
There are various benchmarks to evaluate the aligned LLMs. In general, these benchmarks can be categorized into \emph{Closed-set Benchmarks} and \emph{Open-set Benchmarks}. The former type  focuses on evaluating the skills and knowledge of aligned LLMs, while the latter type often concentrates on the open scenarios where there are no standardized answers. 



\subsubsection{Closed-set Benchmarks}
The closed-set benchmarks mostly include testing instances whose possible answers are predefined and limited to a finite set (e.g., multiple choices). We discuss some of the most commonly used benchmarks below. We refer readers to~\citet{chang2023survey} for more comprehensive introduction of LLMs' evaluation benchmarks. 

\paragraph{General Knowledge}
MMLU~\cite{hendrycks2021measuring} is an English-based benchmark to evaluate LLMs knowledge in zero-shot and few-shot settings. It comprehensively includes questions from the elementary level to an advanced professional level from 57 subjects including STEM, the humanities, the social sciences, etc. The granularity and breadth of the subjects make MMLU ideal for identifying LLMs’ blind spots.
There are also several benchmarks attempting in evaluating the general knowledge in Chinese LLMs.
C-MMLU~\cite{li2023cmmlu}, C-Eval~\cite{huang2023ceval}, M3KE~\cite{liu2023m3ke} and AGIEval~\cite{zhong2023agieval} 
are all Chinese counterparts of MMLU that include diverse sets of questions from multiple subjects with different difficulty levels from various Chinese standardized exams, including Chinese college entrance exams, advanced maths competitions and law exams.
The KoLA benchmark~\cite{yu2023kola} is proposed to evaluate the general real-world knowledge of LLMs.

\paragraph{Reasoning}
Reasoning is a fundamental type of human intelligence that are crucial in solving complicated tasks. Interestingly, research find that LLMs have exhibit emergent behaviors, including the reasoning ability, when they are sufficiently large. Thus, there are several benchmarks in evaluating the ability of arithmetic, commonsense, and symbolic reasoning for LLMs. GSM8K~\cite{cobbe2021training} and Maths~\cite{hendrycks2021measuring} are designed to evaluate the arithmetic reasoning ability for LLMs. CSQA~\cite{talmor-etal-2019-commonsenseqa} and StrategyQA~\cite{geva-etal-2021-aristotle} are proposed to evaluate the commonsense reasoning ability which requires the LLMs to use daily life commonsense to infer in novel situations.~\citet{wei2022chain} propose two novel tasks, Last Letter Concatenation and Coin Flip and measure the Symbolic reasoning ability that involves the manipulation of symbols according to formal rules. BBH~\cite{suzgun2022challenging}, a challenging subset of BIG-Bench~\cite{srivastava2023beyond}, focus on evaluating a wide range of reasoning skills, such as Date Understanding, Word Sorting, and Causal Judgement. 

\paragraph{Coding}
HumanEval~\cite{chen2021evaluating}, HumanEval+~\cite{liu2023your}, and MBPP~\cite{austin2021program} are extensively used benchmarks to evaluate the coding skills of LLMs. They encompass a vast collection of Python programming problems and corresponding test cases to automatically verify the code generated by Code LLMs. The DS-1000 benchmark~\cite{Lai2022DS1000} comprises 1,000 distinct data science workflows spanning seven
libraries. It assesses the performance of code generations against test cases and supports two evaluation modes: completion and insertion. 


\subsubsection{Open-ended Benchmarks}
In contrast to the closed-set benchmarks, the responses to open-set benchmarks can be more flexible and diverse, where aligned LLMs are usually given chatting questions or topics that do not have any fixed reference answers. Early attempts of open-ended benchmarks, such as Vicuna-80~\cite{vicuna2023}, Open-Assistant-953~\cite{Kopf2023OpenAssistantC}, User-Instructions-252~\cite{DBLP:journals/corr/abs-2212-10560}, often leverage a small number of syntactic instructions from LLMs as testing instances. All evaluation candidate LLMs are prompted with the same instructions to provide responses, which are then evaluated against human-based or LLMs-based evaluators. However, these types of benchmarks can only provide comparison several LLMs at a time, making it challenging to reveal a fair comparison among a board range of LLMs, as well as incremental updates when new LLMs become available. AlpacaEval~\cite{dubois2023alpacafarm} tackles this issue by reporting the \emph{Win Rate} of the LLMs candidate to 
the reference LLM \emph{text-davinci-003}. Accordingly, LLMs with higher \emph{Win Rate} are generally better than the ones with lower \emph{Win Rate}. MT-Bench~\cite{zheng2023judging} further increases the evaluation difficulty by proposing 80 multi-turn evaluation instances and wishes LLMs could effectively capture context information in previous turns. FLASK~\cite{Ye2023FLASKFL} proposed to provide fine-grained evaluation towards aligned LLMs. FLASK includes  1,700 instances from 120 datasets. Each testing instance is labelled with a set of 12 foundational and essential ``alignment skills'' (e.g., logical thinking, user alignment, etc.). Accordingly, it is straightforward to evaluate LLMs' capabilities on these skills separately. 



\subsection{Evaluation Paradigm}
\label{evalparadigm}
As open-ended benchmarks often do not have reference answers, it is essential to rely on external human or LLMs evaluators. In this section, we will introduce both human- and LLMs-based evaluation paradigm.

\subsubsection{Human-based Evaluation}

Automatic metrics, such as BLUE~\cite{papineni-etal-2002-bleu} and ROUGE~\cite{lin-2004-rouge}, require ground-truth references and have relatively low correlation with human judgments. Thus, they are not feasible for evaluating responses to open-ended questions.
To bridge this gap, human annotators are used to evaluate the quality of open-ended model responses.
~\citet{DBLP:journals/corr/abs-2212-10560,DBLP:journals/corr/abs-2304-1440} propose to evaluate the response quality in an ordinal classification setting where human annotators are instructed to categorize each response into one of the four levels (i.e., acceptable, minor errors, major errors and unacceptable), separately. However, some other research have found that such classification annotation strategy heavily depend on the subjectivity of annotators, which can result in poor inter-rater reliability~\cite{KALPATHYCRAMER20162345}. Accordingly 
~\citet{alpaca} propose to use a pairwise comparison framework for evaluating the output quality of two LLMs systems. Given the instruction inputs and two model outputs, the human annotators are asked to select a better one. Furthermore, to accurately evaluate multiple LLMs,~\citet{zheng2023judging,dettmers2023qlora} further introduce  the Elo rating system which calculates the relative skill levels of players in zero-sum games such as chess games. Specifically, in Elo system, the player scores are updated based on the result of each pairwise comparison and the current player scores.


\subsubsection{LLMs-based Evaluation}
While human evaluations are often of high quality, it could be inefficient and expensive. In addition, the increasing quality of generated text from LLMs makes it more challenging for human annotators to distinguish between human-written and LLM-generated text in the open-ended NLP tasks~\cite{Clark2021AllT}. Given the strong text capability of LLMs, recent studies propose to incorporate LLMs into the output text evaluation in various NLP tasks without additional expensive references and human efforts.~\citet{tang2023not} propose to improve the traditional automatic metrics by increasing the number of references via LLMs-based paraphrasing systems. However, such method still requires one reference for each evaluation instance. In contrast, ~\citet{liu2023gpteval,fu2023gptscore,chen2023exploring,chiang2023can} propose to 
directly use LLMs to evaluate the generated text quality without a single reference in a wide range of Natural Language Generation (NLG) tasks. Specifically, they construct complicated input instructions with tasks background and evaluation rules and prompt LLMs to follow these evaluation instructions to provide scores for output text. There are also some research efforts that propose LLMs-based evaluation framework for specific NLG tasks, including text summarization~\citet{gao2023human}, code generation~\cite{zhuo2023large}, open-ended QA~\cite{bai2023benchmarking} and conversations~\cite{lin2023llm}. Due to the flexibility of prompts, it is also possible to conduct multi-dimensional evaluation towards the generated text~\cite{lin2023llm,fu2023gptscore}.~\citet{min2023factscore,zha2023alignscore} propose to evaluate factual correctness using both closed-sourced and open-sourced LLMs. Similar to human evaluation, there are also research efforts in explicitly prompting LLMs to conduct pairwise comparisons. To compare the capabilities of two LLMs, instead of assigning scores separately,~\citet{dubois2023alpacafarm,zheng2023judging} explicitly to prompt GPT-4 to select the better response for the same instruction inputs. 


\paragraph{LLMs Evaluation Bias}
Despite LLMs achieve impressive consistency with human judgment, 
~\citet{wang2023large} find that such LLM-based evaluation paradigm suffers from a positional bias and those strong LLMs (i.e., GPT-4) tend to assign higher scores to the first appeared candidates. To calibrate such bias, they propose 
to \textbf{a)} repeat the LLM evaluation process multiple times with different candidate ordering and \textbf{b)} explicitly prompt LLMs to provide chain-of-thoughts for the evaluation before assigning the actual score.~\cite{Wu2023StyleOS} find that LLM-based evaluation prefer candidates with factual errors over shorter candidates and candidates with grammatical errors, despite the former one could impose greater danger than the latter ones. To address this bias, they propose a multi-dimensional Elo rating system which separately evaluates the candidates from the perspective of accuracy, helpfulness and language. Such approach allows a more comprehensive understanding towards the candidates quality than previous one-shot evaluation. Concretely,~\cite{zheng2023judging} systematically show the bias LLMs-based evaluation systems. On top of positional and length bias, they also discover Self-enhancement bias which means LLMs favor their own responses than the ones from other sources. To tackle these biases, their solutions include swapping responses, adding few-shot examples and leveraging CoT and references information. 

\paragraph{Evaluation-Specific LLM}
Despite achieving high-quality automatic evaluation results, the above approaches heavily rely on state-of-the-art closed-source LLMs (e.g., GPT-4) which could result in data privacy issues.~\cite{zheng2023judging} propose to train evaluation-specific LLMs. PandaLM~\cite{wang2023pandalm} is such a specialized evaluation LLMs by fine-tuning LLaMA-7B using around 300K high-quality synthetic evaluation instructions generated from GPT-3.5. Specifically, they first collect large volumes of instructions as well as outputs from a diverse range of open-sourced LLMs, such as LLaMA-7B and Bloom-7B. They then prompt GPT-3.5 to analysis and evaluate the quality of a pair of outputs. Their results on human-annotated meta-evaluation shows that, despite bebing much smaller, PandaLM achieves on-par evaluation performance comparing to GPT-3.5 and GPT-4.





\section{Challenges and Future Directions}
\label{challengesdirection}
The development of LLM alignment is still in a rudimentary stage and thus leaves much room for improvement. In this section, we summarize existing important research efforts of aligning LLMs with human in Table~\ref{llmsummary}. Below, we will discuss some of the challenges as well as the corresponding future research directions.



\begin{table*}[!ht]
    \centering
    \resizebox{\linewidth}{!}{
    \begin{tabular}{c|cccccccccc} \toprule
Aligned LLM & Size & Lang. & Initial LLMs & Training & Self Instruction & NLP Benchmarks & Human Annotations & Human Eval & Auto. Benchmark Eval         & LLM Eval                  \\ \midrule
Alpaca~\cite{alpaca} & 7B & EN & LLaMA & SFT & Text-Davinci-003 & \xmark & \xmark & Author Verification & \xmark & \xmark \\
Vicuna~\cite{vicuna2023} & 7B, 13B, 33B & EN & LLaMA & SFT & GPT-3.5 & \xmark & 70K ShareGPT & \xmark & \xmark & \emph{Vicuna-80} \\
GPT4ALL~\cite{gpt4all}     & 6B, 13B & EN & \makecell{LLaMA \\ GPT-J} & SFT & \xmark                & Bloomz-P3      & \makecell{OIG, ShareGPT, Dolly \\ Stack Overflow}                & \xmark     & Common Sense Reasoning & \xmark \\
LLaMA-GPT4~\cite{peng2023instruction} & 7B & EN, CN & LLaMA & SFT & \makecell{Text-Davinci-003 \\ GPT-4} & \xmark & \xmark & \makecell{\emph{User-Instructions-252} \\ Pairwise, AMT} &  Unnatural Instructions & \emph{Vicuna-80} \\
Phoenix~\cite{DBLP:journals/corr/abs-2304-10453} & 7B, 13B & Multilingual & \makecell{LLaMA \\ BLOOMZ} & SFT & \makecell{GPT-3.5 Multilingual \\ and Dialogue Data} & \xmark & ShareGPT & Volunteers & \xmark & GPT-3.5, GPT-4 \\
UltraLLaMA~\cite{ding2023enhancing} & 13B & EN &  LLaMA & SFT & GPT-3.5 Dialogue Data & \xmark & \xmark & \xmark & Truthful QA & \makecell{GPT 3.5 \emph{Vicuna-80} \\ 300 diverse questions} \\
Baize~\cite{DBLP:journals/corr/abs-2304-01196} & 7B, 13B, 30B & EN & LLaMA & Revision, LoRA & GPT-3.5 self-Chat Data & \xmark & Quora Questions & \xmark & \xmark & GPT-4 \\
WizardLM~\cite{xu2023wizardlm} & 7B, 13B, 30B & EN & LLaMA & SFT & \makecell{GPT-3.5, Alpaca \\ Complex Instructions} & \xmark & ShareGPT & \makecell{10 Annotators \\ Pairwise Comparison} & \xmark & GPT-4, \emph{WizedLM-218} \\
WizardCoder~\cite{luo2023wizardcoder} & 15B & EN, Code & StarCoder & SFT & \makecell{GPT-3.5, Code Alpaca \\ Complex Instructions} & \xmark & \xmark & \xmark & \makecell{HumanEval, MBPP \\ HumanEval+, DS-1000} & \xmark \\
OpenChat~\cite{openchat} & 13B & EN & LLaMA & Language & \xmark & \xmark & \makecell{GPT 3.5 \& GPT4 \\ ShareGPT} & \xmark & MMLU & GPT-4 \\
Guanaco~\cite{dettmers2023qlora} & 13B, 33B, 65B & EN & LLaMA & QLoRA & \makecell{Alpaca, SELF-INSTRUCT \\ Unnatural instructions} & FLAN & Chip2 & Elo, \emph{Vicuna-80} & MMLU & \makecell{Elo, \emph{Vicuna-80} \\ \emph{Open-Assistant-953}} \\
MPT-chat~\cite{MosaicML2023Introducing} & 13B, 30B & EN & MPT & SFT & \makecell{GPTeacher, Guanaco \\ Baize Instructions} & \xmark & Vicuna ShareGPT & \xmark & MMLU & GPT4, MT-bench \\
FLACUNA~\cite{ghosal2023flacuna} & 13B & EN & Vicuna & LoRA & Alpaca, Code Alpaca & FLAN & ShareGPT & \xmark & \makecell{MMLU, BBH, DROP \\ CRASS, HumanEval} & GPT 3.5, IMPACT \\
Bactrian-X~\cite{bactrian} & 7B & Multilingual & \makecell{LLaMA \\ BLOOMZ} & LoRA & \makecell{Alpaca \\ Google Translation} & \xmark & \xmark & \xmark & \makecell{XCOPA, XStoryCloze \\ XWinograd, SentimentX} & \makecell{GPT 4 \\ Multilingual \emph{Vicuna-80}}  \\
Ocra~\cite{mukherjee2023orca} & 13B & EN & LLaMA & SFT & \xmark & FLAN & \xmark & \xmark & AGIEval, BBH & \makecell{GPT-4, \emph{Vicuna-80} \\ \emph{WizedLM-218}, \emph{Awesome-164}} \\
Phi-1~\cite{gunasekar2023textbooks} & 350M, 1.3B & EN, Code & Phi-1-base & SFT & \makecell{GPT-3.5 \\ Synthetic Textbook} & \xmark & \makecell{Python, The Stack \\ Stack Overflow} & \xmark & HumanEval & GPT-4 Grading \\
Chinese Alpaca~\cite{cui2023efficient} & 7B, 13B, 33B & EN, CN & Chinese LLaMA & LoRA & \makecell{STEM \\ Org. and Trans.  Alpaca} & pCLUE & \xmark & \xmark & C-Eval & \xmark \\
Lion~\cite{Jiang2023LionAD} & 7B, 13B & EN & LLaMA & SFT & \makecell{Alpaca \\ GPT 3.5 Adv. Instruction} & \xmark & \xmark & \makecell{HHH \\ \emph{User-Instructions-252}} & \xmark & GPT-4, \emph{Vicuna-80} \\
Stable Alignment~\cite{liu2023training} & 7B & EN & Alpaca & SFT & \makecell{GPT-3.5 \\ Social Aligned Instructions} & \xmark & \xmark & \xmark & \xmark & \makecell{GPT-4 \\ HHH, HHH-A} \\
Dromedary~\cite{Sun2023PrincipleDrivenSO} & 65B & EN & LLaMA & SFT & LLaMA-65B, Self-Align & \xmark & \makecell{175 Munnal Examples \\ 16 Principle Rules} & \xmark & TruthfulQA, BBH & GPT-4, \emph{Vicuna-80} \\
Dolly-v2~\cite{DatabricksBlog2023DollyV2} & 3B, 7B, 12B & EN & Pythia & SFT & \xmark & \xmark & \emph{databricks-dolly-15k} & \xmark & \emph{LLM Harness} & \xmark \\
Selfee~\cite{selfee2023} & 7B, 13B & EN & LLaMA & Revision & \makecell{GPT 3.5 Self-Improve \\ Alpaca} & FLAN, Maths, Code & ShareGPT & \xmark & \xmark & GPT-4, \emph{Vicuna-80} \\
\textsc{T\"ulu}~\cite{wang2023far} & 7B, 13B, 30B, 65B & EN & LLaMA & SFT & \makecell{Alpaca, Code Alpaca \\ GPT4-Alpaca, Self-instruct} & FLAN, CoT & \makecell{Dolly, ShareGPT \\ Open Assistant} & \makecell{Acceptability \\ Pairwise Comparison} & \makecell{MMLU, GSM, BBH \\ TydiQA, Codex-Eval} & \makecell{GPT4 on \emph{Vicuna-80}, Koala \\ Open Assistant Benchmarks}\\
Koala~\cite{koala_blogpost_2023} & 13B & EN & LLaMA & Language & Alpaca & \xmark & \makecell{OIG, HC3, Anthropic HH \\ OpenAI WebGPT, Summary} & \makecell{100 AMT Annotators \\ on \emph{Alpaca and Koala Test}} & \xmark & \xmark \\
Bayling~\cite{Zhang2023BayLingBC} & 7B, 13B & Multilingual & LLaMA & SFT & \makecell{Alpaca \\ GPT 3.5  Interactive Translation} & \xmark & ShareGPT & Translation Quality & \makecell{WMT22 Multilingual Translation \\  Lexically Constrained Translation} & \xmark \\
Wombat~\cite{yuan2023rrhf} & 7B & EN & Alpaca & Rank & \makecell{Alpaca \\ ChatGPT Ratings} & \xmark & \emph{Helpful and Harmless} & \xmark & \xmark & GPT-4, \emph{Vicuna-80} \\
Lamini-lm~\cite{DBLP:journals/corr/abs-2304-1440} & 0.7B & EN & T5-Flan & SFT & \makecell{Alpaca \\ Self-instruct} & P3, FLAN & \xmark & Human Rating & \emph{LLM harness} & \xmark \\
\bottomrule
\end{tabular}}
\caption{An overview of popular aligned LLMs, including their Size, supported languages, initial LLMs, alignment training method, alignment data, and alignment evaluation.}
\label{llmsummary}
\end{table*}

\paragraph{Fine-grained Instruction Data Management}
While research on LLMs alignment have been unprecedentedly active, many of these research efforts propose to leverage training instructions from diverse sources, making it challenging to fairly compare among different methods. As discussed in Section~\ref{datamanagement}, there are some interesting findings about the implication of particular instruction dataset. For example, FLAN and programming instructions can improve reasoning capability aligned LLMs~\cite{ghosal2023flacuna} and ShareGPT general performs well across a wide range of benchmarks~\cite{wang2023far}. 
However, there are still many issues in other aspects of instruction data management remaining unclear, including the optimal quality control towards instruction data, optimal instruction training sequence, how to effectively mix-up different instructions. These research efforts could finally enable fine-grained instruction management, allowing researchers and practitioners to construct high-quality instruction data. 





% Thus, it is critical to come up with resource-constrained LLM alignment evaluation framework where certain alignment resources are constrained at a certain level (e.g., maximum 10K instructions, 5 hours training time, etc.), allowing NLP/LLM researchers and practitioners to focus on constructing high-quality alignment components, rather than brute forcing adding up the alignment scale.

\paragraph{LLMs Alignment for non-English Languages} 
Most of existing research in LLMs alignment are English-dominated. While many approaches, such as complex instruction generation~\cite{xu2023wizardlm} and explanation tuning~\cite{mukherjee2023orca}, are language-agnostic, they only explore English-based prompts and it is unclear how well these prompts perform when adapting to other languages, severely hindering the application of LLMs to non-English regions. It is interesting to see \emph{1)} how these alignment technologies perform in various languages, in particular low-resource languages, and \emph{2)} how to effectively transfer the effect of LLMs alignment across different languages.

\paragraph{LLMs Alignment Training Technologies}
As shown in Table~\ref{llmsummary}, most of existing 
aligned LLMs are based on the simple SFT technology. However, SFT does not explicitly incorporate human preference into LLMs. As a result, aligning LLMs solely based on SFT could require a lot more instruction data and training resources. In general, there is a lacking of comprehensive investigation over the effect of various training technologies to incorporate human preference into LLMs. 
Thus, it is critical to come up with resource-constrained LLM alignment training framework where certain alignment resources are given at a certain level (e.g., maximum 10K instructions, 5 hours training time, etc.), allowing researchers and practitioners to verify the effectiveness of various training methods. As increasing number of instruction data have become available, this exploration could further 
promote effective and environmental-friendly LLMs alignment solutions.

\paragraph{Human-in-the-loop LLMs Alignment Data Generation}
Table~\ref{llmsummary} has shown that ShareGPT data has been widely adapted for LLMs alignment. The preliminary analysis in~\citet{wang2023far} also reveal that   ShareGPT  performs consistly well across a wide range of NLP tasks. These results indicate that human is still a key factor in improving LLMs alignment quality. Different from traditional human annotation framework where human provides annotation based on the instructions, ShareGPT is a human-in-the-loop alignment solution where human can freely determine what LLMs should generate. This shows the great potential of human-in-the-loop data generation solution in LLMs alignment. It will be interesting to explore other types of human-in-the-loop solutions to further facilitate LLMs alignment.

\paragraph{Human-LLM Joint Evaluation Framework}
Existing LLM evaluation frameworks either use LLMs for effective evaluation or leverage crowd-sourcing for high-quality evaluation. As shown in~\cite{Wu2023StyleOS,liu2023gpteval}, state-of-the-art LLMs have demonstrated similar or superior evaluation capability in various NLP tasks.
It is feasible to use LLMs as special evaluation annotators and develop LLM-human joint evaluation framework where LLMs and human are assigned with different evaluation tasks based on their own strengths to maintain both efficiency and quality of the evaluation procedure for LLM alignment .




\section{Conclusion}
\label{conclusion}
This survey provides an up-to-date review to recent advances of LLMs alignment technologies. We summarize these research efforts into \emph{Alignment Instruction Collection}, \emph{Alignment Training} and \emph{Alignment Evaluation}.  Finally, we pointed out several promising future directions for LLMs alignment. We hope this survey could provide insightful perspectives and inspire further research in improving LLMs alignment.

% Entries for the entire Anthology, followed by custom entries
\bibliography{anthology,custom}
\bibliographystyle{acl_natbib}

\appendix

\section{Appendix}
\label{sec:appendix}
\begin{comment}
\section{System Architecture}
\label{appendix:architecture}
\system has a novel modularized system architecture with three key components: 
\emph{StreamManager}, 
\emph{TxnManager} and \emph{TxnScheduler}. 
These components are instantiated in each thread locally.
The execution outline of \system is presented in Algorithm~\ref{alg:algo}.
Transactional stream processing is continuous and potentially never ends (Line 1$\sim$8).
The dependency resolution and execution of state transactions are separated into two non-overlapping phases by punctuations~\cite{Tucker:2003:EPS:776752.776780} (Line 2 and 5), which guarantees that no subsequent input event will have a smaller timestamp. 
Effectively, a batch of state transactions is collected during the first phase, and processed during the second phase.

In the first phase (i.e., stream processing phase), 
the \emph{StreamManager} conducts preprocessing for every input event ($e$). Similar to some prior works~\cite{tstream}, state transactions may be issued but not immediately processed during preprocessing (Line 3).
The \emph{pre\_processing} and \emph{post\_processing} functions are exposed as APIs to users.
The \emph{TxnManager} handles dependency resolution (Line 4) among state transactions and insert decomposed operations to construct a \tpg. We discuss the detailed two-phase \tpg construction process in Section~\ref{subsec:construction}.

In the second phase  (i.e., transaction processing phase), 
the \emph{TxnManager} is first involved again to refine (Line 6) the constructed \tpg with further dependency resolution.
The \emph{TxnScheduler} 
schedules operations for concurrent execution based on the constructed \tpg according to the three dimensions of scheduling decisions (Line 7). 
In particular, a scheduling decision model $M$ is instantiated based on the constructed \tpg (Line 14).
\textbf{\circled{1}} Guided by $M$, execution threads adopt an exploration strategy (Section~\ref{subsec:explore}) to explore the constructed \tpg for operations available to be scheduled constrained by dependencies. 
\textbf{\circled{2}} 
During exploration, one or multiple operations may be treated as the 
% basic 
unit of scheduling (Section~\ref{subsec:granularity}). 
Subsequently, \textbf{\circled{3}} every thread executes operation(s) in the unit of scheduling with various abort handling mechanisms (Section~\ref{subsec:abort_handling}).
Only when state transactions are processed (i.e., committed or aborted) can the associated input events be postprocessed (Line 8) by the \emph{StreamManager} based on transaction processing results.
\end{comment}

\begin{comment}
\begin{algorithm}
\footnotesize
    \KwData{$e$ \tcp{Input event}}
    \KwData{$txn_{ts}$ \tcp{State transaction}}
    \KwData{$G$ \tcp{The currently constructed TPG}}
    \While{!finish processing of input streams}{
        \eIf(\tcp*[h]{Phase 1}){\text{$e$ is not a $punctuation$}}{
                $txn_{ts}$ $\gets$ PRE\_Processing($e$)\;
                \textbf{TPG\_Construction}($G$, $txn_{ts}$)\; 
          }(\tcp*[h]{Phase 2}){
                \textbf{TPG\_Refinement}($G$)\; 
                \textbf{TXN\_Scheduling}($G$)\; 
                POST\_Processing()\;
          }
    }
    
    \SetKwFunction{FMain}{TPG\_Construction}
    \SetKwProg{Fn}{Function}{:}{}
    \Fn{\FMain{$G$, $txn_{ts}$}}{
        $O_{1..k}$ $\gets$ \textbf{Partition} $txn_{ts}$\;
        \ForEach{\text{operation $O_{i}$ $\in$ $O_{1..k}$}}{
            \textbf{Identify} its \ld\;
            $G$ $\gets$ $G$ + $O_{i}$ \;
        }
    }
    \SetKwFunction{FMain}{TPG\_Refinement}
    \SetKwProg{Fn}{Function}{:}{}
    \Fn{\FMain{$G$}}{
        \ForEach{\text{vertex $e_{i}$ $\in$ $G$}}{
            \textbf{Identify} its \td, \pd\;
        }
    }
    
    \SetKwFunction{FMain}{TXN\_Scheduling}
    \SetKwProg{Fn}{Function}{:}{}
    \Fn{\FMain{$G$}}{
        $M$ $\gets$ Instantiated with $G$;\tcp{A decision model}
        \While{!finish scheduling of $G$
        }{
          \textbf{\circled{2}} $Scheduling Unit$ $\gets$ \textbf{\circled{1}} \emph{Explore}($G$, $M$)\; 
            \textbf{\circled{3}} \emph{Execute with Abort Handling} ($Scheduling Unit$)\; 
        }
    }
  \caption{Execution Outline of \system}
  \label{alg:algo}
\end{algorithm}
\end{comment}

\end{document}
