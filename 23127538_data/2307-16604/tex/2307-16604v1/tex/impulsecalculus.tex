\subsection{Multivariate impulse spectral space}\label{section:impulseoperator}
In digital systems, we often find nested periodic impulse trains that are e.g. generated by a controller or a by computer program with loops \cite{albers2006hierarchical}. 
For example, consider a message that is periodically sent over a network with period $p_{1} \in \positivrealnumbers$ and each message is divided into $\degreevectorcomponent{2} \in \naturalnumbers$ packets that are sent in a burst with period $p_{2}$. This means the periodic message generates a burst of packets where the burst is itself a periodic impulse train, i.e. we have a nested periodic impulse train. Formally, we describe a nested impulse train by the convolution of \shiftedimpulsenameplural{}.
The result of the convolution of two Diracs impulses is well-known, however, we show their convolution in Lemma \ref{lemma:convolution} to see that this result is also based on the concept of adjoint maps. 
\begin{lemma}[Convolution of Dirac impulses]\label{lemma:convolution}
	Let $a,b \in \realnumbers$. Then $\langle T_{-a} \diracdelta \ast T_{-b} \diracdelta, \phi \rangle = \langle \diracdelta, T_{a+b} \phi \rangle$ which means $\diracdelta(t-a) \ast \diracdelta(t-b) = \diracdelta(t-a-b)$.
\end{lemma}
\begin{proof}
	Let $(f_{k})_{k \in \naturalnumbers}$ be a sequence of test functions approximating the Dirac impulse, i.e. the $f_{k}(\impulsevariable)$ have the properties $f_{k}(\impulsevariable) = 0, |\impulsevariable| \geq 1/k$ and $\int_{-1/k}^{1/k} f_{k}(\impulsevariable) \, d \impulsevariable = 1$ for all $k \in \naturalnumbers$. 
	Then,
	\begin{alignat}{1}
		\langle T_{-a} \diracdelta \ast T_{-b} \diracdelta, \phi \rangle
		&
		=
		\lim_{k \to \infty} \langle T_{-a} f_{k} \ast T_{-b} \diracdelta, \phi \rangle
		\\	
		&
		= \lim\limits_{n \to \infty} \int_{-\infty}^{\infty} (T_{-a} f_{k} \ast T_{-b} \diracdelta)(x) \cdot \phi(x) \, dx
		\\
		&
		=
		\lim\limits_{n \to \infty} \int_{-\infty}^{\infty} \int_{-\infty}^{\infty}  \delta(y - a) \cdot f_{k}(\impulsevariable - y - b) \, dy \cdot \phi(x) \, dx
		\\
		&
		\oset[1.5ex]{\mathclap{\mathrm{Le.} \ref{lemma:translationdirac}}}{=}
		\lim\limits_{n \to \infty} \int_{-\infty}^{\infty}  f_{k}(\impulsevariable - a - b) \cdot \phi(x) \, dx
		\\
		&
		\oset[1.5ex]{\mathclap{\mathrm{Le.} \ref{lemma:translationdirac}}}{=}
		\phi(a + b)
		\\
		&
		=
		\langle \diracdelta, T_{a+b} \, \phi \rangle
	\end{alignat}
\end{proof}
The idea is to describe one of the Dirac impulses as the limit of test functions which is convolved with the other Dirac impulse. 
Applying Lemma \ref{lemma:convolution}, we can describe any $n-$times nested impulse train.
\begin{example}\label{example:nestedtrain}
	Let $\degreevectorcomponent{1},\degreevectorcomponent{2} \in \naturalnumbers$, then a two-times nested impulse train is described by 
	\begin{alignat}{1} 
		& 
		\sum_{\impulseindex_{1} = 0}^{\degreevectorcomponent{1}-1} \amplitudevector{\impulseindex_{1}}\diracdelta(\impulsevariable - \impulseindex_{1} \phasevectorcomponent{1}) 
		\ast 
		\sum_{\impulseindex_{2} = 0}^{\degreevectorcomponent{2}-1} \amplitudevector{\impulseindex_{2}}\diracdelta(\impulsevariable - \impulseindex_{2} \phasevectorcomponent{2})  
		\\
		=
		&
		\sum_{\impulseindex_{1} = 0}^{\degreevectorcomponent{1}-1}  \sum_{\impulseindex_{2} = 0}^{\degreevectorcomponent{2}-1}  \amplitudevector{\impulseindex_{1} }\amplitudevector{\impulseindex_{2} } \diracdelta(\impulsevariable - \impulseindex_{1} \phasevectorcomponent{1}- \impulseindex_{2} \phasevectorcomponent{2} ) 
		\\
		\eqqcolon
		& \,
		\impulsespectraldensityeinsteinmulti{\amplitude}{\impulseindex}{\impulseshift}{\impulsevariable}
	\end{alignat}
\end{example}
\begin{example}[Superposition]
	If two impulse trains occur independently from each other and share the same network, then a receiver in the network observes a superposition of their sent impulse trains.  
	%The superposition or addition of the two signals at every point is called the interference in physics.
	In the impulse algebra, superposition is represented by \impulsegroupoperationname{}:
	\begin{alignat}{1}
		&
		\underbrace{
		\sum_{\impulseindex_{1} = 0}^{\degreevectorcomponent{1} - 1}  \sum_{\impulseindex_{2} = 0}^{\degreevectorcomponent{2} - 1}  \amplitudevector{\impulseindex_{1} }\amplitudevector{\impulseindex_{2} } \diracdelta(\impulsevariable - \impulseindex_{1} \phasevectorcomponent{1}- \impulseindex_{2} \phasevectorcomponent{2} ) 
		}_{\eqqcolon \, {\impulsespectraldensityeinsteinmulti{\amplitude}{\impulseindex}{\impulseshift}{\impulsevariable}}_{1}}
		\nonumber
		\impulsegroupoperation
		\underbrace{
		\sum_{\impulseindex_{3} = 0}^{\degreevectorcomponent{3} - 1}  \sum_{\impulseindex_{4} = 0}^{\degreevectorcomponent{4} - 1}  \amplitudevector{\impulseindex_{3} }\amplitudevector{\impulseindex_{4} } \diracdelta(\impulsevariable - \impulseindex_{3} \phasevectorcomponent{3}- \impulseindex_{4} \phasevectorcomponent{4} ) 
		}_{\eqqcolon \, {\impulsespectraldensityeinsteinmulti{\amplitude}{\impulseindex}{\impulseshift}{\impulsevariable}}_{2}}
		\eqqcolon
		\sum_{\impulseinterferenceindexone = 1}^{2}
		{\impulsespectraldensityeinsteinmulti{\amplitude}{\impulseindex}{\impulseshift}{\impulsevariable}}_{\impulseinterferenceindexone}
	\end{alignat}
\end{example} 
%Generalizing, we can describe an arbitrary superposition of impulses according to the closure of the \impulsespectralspacename{}. 
%We describe impulses that shift other impulses using convolution. Superposed impulses are described by \impulsegroupoperationname{}. The amplitude of an impulse can be scaled by using \impulsespectralspacemultname{}.
%This means we have three basic operations to construct vectors in the impulse algebra: convolution, \impulsegroupoperationname{} and \impulsespectralspacemultname.
%Based on them, we present in the following an algorithm to construct any vector using known matrix operations.

Up to this point, we can conclude that any superposition of periodic impulse trains can be described by a simple addition in a vector space.
Scalar multiplication on the other hand modulates the amplitude of the impulses, while convolution allows any shift in time.
As a result, it is possible to construct any discrete trace of impulses representing any impulse train by well-known algebraic operations.

\subsection{Linear construction of complex multivariate spectral trains}
As mentioned in the last section, let us investigate how any kind of impulse train can be built by a seqence of linear vector operations.
Starting from a simple modulated impulse spectral train, a complex vector of multiple superpositions of impulse trains is constructed.

A periodic sequence of \shiftedimpulsenameplural{} having the shift $\impulseshift \in \realnumbers$, the degree $\impulsedegree \in \naturalnumbers$ (the number of impulses), and the amplitudes $\amplitudevector{\impulseindex}$ for all $\impulseindex \in \{ 1, 2, \ldots, \impulsedegree \}$ is described by the
 \textbf{\impulsespectraltrainname} (\impulsespectraltrainshort)
\begin{alignat}{1}
	\impulsespectraltrain{\impulseshift}{\impulsedegree}{\impulsevariable} 
	&
	\coloneqq
	\sum_{\impulseindex = 0}^{\impulsedegree - 1} 
	\amplitudevector{\impulseindex} \diracdelta(\impulsevariable - \impulseindex \impulseshift)
	\eqqcolon
	\impulsespectraldensityeinsteinmono{\amplitude}{\impulseindex}{\impulseshift}{\impulsevariable}
	\end{alignat}
If $\naturalnumbers \ni \multiperiodicimpulsedimension$ \impulsespectraltrainnameplural{} are nested (see Example \ref{example:nestedtrain} for the case $\multiperiodicimpulsedimension =2 $), we describe their series of impulses by the \textbf{\impulsespectraldensityname} (\impulsespectraldensitynameshort)
\begin{alignat}{1}
	\impulsespectraldensity{\impulseshift}{\impulsedegree}{\impulsevariable}
	&
	\coloneqq
	\impulsespectraldensityeinsteinmonoindexed{\amplitude}{\impulseindex}{\impulseshift}{\impulsevariable}{1} \ast \impulsespectraldensityeinsteinmonoindexed{\amplitude}{\impulseindex}{\impulseshift}{\impulsevariable}{2} \ast \ldots \ast \impulsespectraldensityeinsteinmonoindexed{\amplitude}{\impulseindex}{\impulseshift}{\impulsevariable}{\multiperiodicimpulsedimension}
	\\
	&
	\coloneqq
	\sum_{\impulseindex_{1} = 0}^{\degreevectorcomponent{1} - 1} 
	\amplitudevector{\impulseindex_{1}} \diracdelta(\impulsevariable - \impulseindex_{1}  \phasevectorcomponent{1}) 
	\ast
	\sum_{\impulseindex_{2} = ´0}^{\degreevectorcomponent{2} - 1} 
	\amplitudevector{\impulseindex_{2}} \diracdelta(\impulsevariable - \impulseindex_{2} \phasevectorcomponent{2})  \ast
	 \ldots 
	 \nonumber
	 \\
	 &~~~~~~~~~~~~~~~~~~~~~~~~~~~~~~~~~~~~~~~~~~~~
	 \ast 
	 \sum_{\impulseindex_{\multiperiodicimpulsedimension} = 0}^{\degreevectorcomponent{\multiperiodicimpulsedimension}-1} 
	\amplitudevector{\impulseindex_{\multiperiodicimpulsedimension}} \diracdelta(\impulsevariable - \impulseindex_{\multiperiodicimpulsedimension} \phasevectorcomponent{\multiperiodicimpulsedimension})
	\\
	&
	=
	\sum_{\impulseindex_{1} = 0}^{\degreevectorcomponent{1} - 1} \dots \sum_{\impulseindex_{\multiperiodicimpulsedimension} = 0}^{\degreevectorcomponent{\multiperiodicimpulsedimension}-1}
	\amplitudevector{\impulseindex_{1}} \cdot \ldots \cdot \amplitudevector{\impulseindex_{\multiperiodicimpulsedimension}}  \diracdelta(\impulsevariable - \impulseindex_{1} \phasevectorcomponent{1}  - \ldots - \impulseindex_{\multiperiodicimpulsedimension} \phasevectorcomponent{\multiperiodicimpulsedimension} ) 
	\\
	&
	\coloneqq
	\impulsespectraldensityeinsteinmulti{\amplitude}{\impulseindex}{\impulseshift}{\impulsevariable}
\end{alignat}

The superposition of $\numberofcomponents \in \naturalnumbers$ \impulsespectraldensitynameshort s is described by the \impulsegroupoperationname{}

\begin{alignat}{1}
		\impulseinterference{\impulseshift}{\impulsedegree}{\impulsevariable}
		\coloneqq
	&
	\sum_{\impulseindex_{1,1} = 0}^{\degreevectorcomponent{1,1} - 1} \dots \sum_{\impulseindex_{1,\multiperiodicimpulsedimension} = 0}^{\degreevectorcomponent{1,\multiperiodicimpulsedimension} - 1}
	\amplitudevector{\impulseindex_{1,1}} \cdot \ldots \cdot \amplitudevector{\impulseindex_{1,\multiperiodicimpulsedimension}}  \diracdelta(\impulsevariable - \impulseindex_{1,1} \phasevectorcomponent{1,1}  - \ldots - \impulseindex_{1,\multiperiodicimpulsedimension} \phasevectorcomponent{1,\multiperiodicimpulsedimension} ) 
	\nonumber
	\\
	\impulsegroupoperation
	&
	\nonumber
	\\
	\vdots ~
	\nonumber
	&
	\\
	\impulsegroupoperation
	&
	\sum_{\impulseindex_{\numberofcomponents,1} = 0}^{\degreevectorcomponent{\numberofcomponents,1} - 1} \dots \sum_{\impulseindex_{\numberofcomponents,\multiperiodicimpulsedimension} = 0}^{\degreevectorcomponent{\numberofcomponents,\multiperiodicimpulsedimension} - 1}
	\amplitudevector{\impulseindex_{\numberofcomponents,1}} \cdot \ldots \cdot \amplitudevector{\impulseindex_{\numberofcomponents,\multiperiodicimpulsedimension}}  \diracdelta(\impulsevariable - \impulseindex_{\numberofcomponents,1} \phasevectorcomponent{\numberofcomponents,1}  - \ldots - \impulseindex_{\numberofcomponents,\multiperiodicimpulsedimension} \phasevectorcomponent{\numberofcomponents,\multiperiodicimpulsedimension} ) 
	\\
	=
	&
	\sum_{\impulseinterferenceindexone = 1}^{\numberofcomponents }
	\sum_{\impulseindex_{\impulseinterferenceindexone,1} = 0}^{\degreevectorcomponent{\impulseinterferenceindexone,1} - 1} \dots \sum_{\impulseindex_{\impulseinterferenceindexone,\multiperiodicimpulsedimension} = 0}^{\degreevectorcomponent{\impulseinterferenceindexone,\multiperiodicimpulsedimension} - 1}
	\amplitudevector{\impulseindex_{\impulseinterferenceindexone,1}} \cdot \ldots \cdot \amplitudevector{\impulseindex_{\impulseinterferenceindexone,\multiperiodicimpulsedimension}}  \diracdelta(\impulsevariable - \impulseindex_{\impulseinterferenceindexone,1} \phasevectorcomponent{\impulseinterferenceindexone,1}  - \ldots - \impulseindex_{\impulseinterferenceindexone,\multiperiodicimpulsedimension} \phasevectorcomponent{\impulseinterferenceindexone,\multiperiodicimpulsedimension} ) 
	\\
	\eqqcolon
	&
\sum_{\impulseinterferenceindexone = 1}^{\numberofcomponents}
{	\impulsespectraldensityeinsteinmulti{\amplitude}{\impulseindex}{\impulseshift}{\impulsevariable}}_{\impulseinterferenceindexone}
\end{alignat}
that is called \textbf{\impulseinterferencename} (\impulseinterferencenameshort).
The \impulseinterferencenameshort{} can be written out as a summation of convolved \impulsespectraltrainname s
\begin{alignat}{1}
	\impulseinterference{\impulseshift}{\impulsedegree}{\impulsevariable} 
	&
	=
	\sum_{\impulseinterferenceindexone = 1}^{\numberofcomponents}
	\sum_{\impulseindex_{\impulseinterferenceindexone,1} = 0}^{\degreevectorcomponent{\impulseinterferenceindexone,1} - 1} \dots \sum_{\impulseindex_{\impulseinterferenceindexone,\multiperiodicimpulsedimension} = 0}^{\degreevectorcomponent{\impulseinterferenceindexone,\multiperiodicimpulsedimension} -1}
	\amplitudevector{\impulseindex_{\impulseinterferenceindexone,1}} \cdot \ldots \cdot \amplitudevector{\impulseindex_{\impulseinterferenceindexone,\multiperiodicimpulsedimension}}  \diracdelta(\impulsevariable - \impulseindex_{\impulseinterferenceindexone,1} \phasevectorcomponent{\impulseinterferenceindexone,1}  - \ldots - \impulseindex_{\impulseinterferenceindexone,\multiperiodicimpulsedimension} \phasevectorcomponent{\impulseinterferenceindexone,\multiperiodicimpulsedimension} ) 
	\\
	&
	=	
	\sum_{\impulseinterferenceindexone = 1}^{\numberofcomponents}
	\impulsespectraldensityeinsteinmonoindexed{\amplitude}{\impulseindex}{\impulseshift}{\impulsevariable}{\impulseinterferenceindexone,1} \ast \impulsespectraldensityeinsteinmonoindexed{\amplitude}{\impulseindex}{\impulseshift}{\impulsevariable}{\impulseinterferenceindexone,2} \ast \ldots \ast \impulsespectraldensityeinsteinmonoindexed{\amplitude}{\impulseindex}{\impulseshift}{\impulsevariable}{\impulseinterferenceindexone,\multiperiodicimpulsedimension}
	\nonumber
	\\
	&
	=
	\impulsespectraldensityeinsteinmonoindexed{\amplitude}{\impulseindex}{\impulseshift}{\impulsevariable}{1,1} \ast \impulsespectraldensityeinsteinmonoindexed{\amplitude}{\impulseindex}{\impulseshift}{\impulsevariable}{1,2} \ast \ldots \ast \impulsespectraldensityeinsteinmonoindexed{\amplitude}{\impulseindex}{\impulseshift}{\impulsevariable}{1,\multiperiodicimpulsedimension}
	\nonumber
	\\
	&
	\impulsegroupoperation
	\impulsespectraldensityeinsteinmonoindexed{\amplitude}{\impulseindex}{\impulseshift}{\impulsevariable}{2,1} \ast \impulsespectraldensityeinsteinmonoindexed{\amplitude}{\impulseindex}{\impulseshift}{\impulsevariable}{2,2} \ast \ldots \ast \impulsespectraldensityeinsteinmonoindexed{\amplitude}{\impulseindex}{\impulseshift}{\impulsevariable}{2,\multiperiodicimpulsedimension}
	\nonumber
	\\
	& ~
	\impulsegroupoperation
	\nonumber
	\\
	& ~ \,
	\vdots
	\nonumber
	\\
	&
	\impulsegroupoperation
	\impulsespectraldensityeinsteinmonoindexed{\amplitude}{\impulseindex}{\impulseshift}{\impulsevariable}{\numberofcomponents,1} \ast \impulsespectraldensityeinsteinmonoindexed{\amplitude}{\impulseindex}{\impulseshift}{\impulsevariable}{\numberofcomponents,2} \ast \ldots \ast \impulsespectraldensityeinsteinmonoindexed{\amplitude}{\impulseindex}{\impulseshift}{\impulsevariable}{\numberofcomponents,\multiperiodicimpulsedimension}
	\label{eq:interferenceflattened}
\end{alignat}
Equation \ref{eq:interferenceflattened} can be constructed by matrix manipulations as follows.
Let
\begin{alignat}{1}
	\amplitudematrix
	=
	\begin{pmatrix}
		\amplitudematrixentry{\impulseindex}{1}{1} & \amplitudematrixentry{\impulseindex}{1}{2}  & \dots & \amplitudematrixentry{\impulseindex}{1}{\impulseinterferencedegreetwo} \\
		\amplitudematrixentry{\impulseindex}{2}{1}  & \amplitudematrixentry{\impulseindex}{2}{2}  & \dots & \amplitudematrixentry{\impulseindex}{2}{\impulseinterferencedegreetwo}  \\
		\vdots & \vdots & \ddots & \vdots \\
		\amplitudematrixentry{\impulseindex}{\impulseinterferencedegreeone}{1}  & \amplitudematrixentry{\impulseindex}{\impulseinterferencedegreeone}{2}  & \dots & \amplitudematrixentry{\impulseindex}{\impulseinterferencedegreeone}{\impulseinterferencedegreetwo} \\
	\end{pmatrix}
	~~~
	\shiftmatrix{\degreevector}{\impulsevariable}
	=
 	\begin{pmatrix}
 	 \shiftmatrixentry{\impulseindex}{\impulsevariable}{1}{1} &  \shiftmatrixentry{\impulseindex}{\impulsevariable}{1}{2}& \dots & \shiftmatrixentry{\impulseindex}{\impulsevariable}{1}{\impulseinterferencedegreetwo} \\
 	 \shiftmatrixentry{\impulseindex}{\impulsevariable}{2}{1} &  \shiftmatrixentry{\impulseindex}{\impulsevariable}{2}{2} & \dots &  \shiftmatrixentry{\impulseindex}{\impulsevariable}{2}{\impulseinterferencedegreetwo} \\
 	\vdots & \vdots & \ddots & \vdots \\
 	 \shiftmatrixentry{\impulseindex}{\impulsevariable}{\impulseinterferencedegreeone}{1} &  \shiftmatrixentry{\impulseindex}{\impulsevariable}{\impulseinterferencedegreeone}{2} & \dots &  \shiftmatrixentry{\impulseindex}{\impulsevariable}{\impulseinterferencedegreeone}{\impulseinterferencedegreetwo} \\
 \end{pmatrix}_{\boldsymbol{N}}
\end{alignat}
specify the \textbf{\amplitudematrixname} and the \textbf{\shiftmatrixname} of the \impulseinterferencenameshort{} where $(\amplitudevector{\impulseindex})_{\impulseinterferenceindexone,\impulseinterferenceindextwo} = ( \amplitudevector{1}, \amplitudevector{2}, \dots, \amplitudevector{\numberofrepetitionsvectorcomponent{\impulseinterferenceindexone,\impulseinterferenceindextwo}})$ is the \textbf{\amplitudevectorname},  $\shiftmatrixentry{\impulseindex}{\impulsevariable} {\impulseinterferenceindexone}{\impulseinterferenceindextwo} = ( \diracdelta(\impulsevariable), \diracdelta(\impulsevariable - \phasevectorcomponent{\impulseinterferenceindexone,\impulseinterferenceindextwo}), \dots, \diracdelta(\impulsevariable - (\degreevectorcomponent{\impulseinterferenceindexone,\impulseinterferenceindextwo}-1) \phasevectorcomponent{\impulseinterferenceindexone,\impulseinterferenceindextwo}))$ is the \textbf{\shiftvectorname} and $\degreematrix \in \naturalnumbers^{\impulseinterferencedegreeone \times \impulseinterferencedegreetwo}$ is the \textbf{\degreematrixname} where $\degreematrixentry{\impulseinterferenceindexone}{\impulseinterferenceindextwo}$ is the \degreename{} of the \shiftvectorname{} $\shiftmatrixentry{\impulseindex}{\impulsevariable}{\impulseinterferenceindexone}{\impulseinterferenceindextwo}$.  
Then, we define the  \textbf{\dotproductname} $\matrixdotproduct$ of an \amplitudevectorname{} and an \shiftvectorname{}
\begin{alignat}{1}
	(\amplitudevector{\impulseindex})_{\impulseinterferenceindexone,\impulseinterferenceindextwo}
	\matrixdotproduct
	(\phasevectorcomponent{\impulseindex})_{\impulseinterferenceindexone,\impulseinterferenceindextwo} 
	&
	=
	\begin{pmatrix}
		\amplitudevector{1} \\  \amplitudevector{2} \\ \vdots \\ \amplitudevector{\numberofrepetitionsvectorcomponent{\impulseinterferenceindexone,\impulseinterferenceindextwo}}
	\end{pmatrix}
\matrixdotproduct
	\begin{pmatrix}
	\diracdelta(\impulsevariable) \\ \diracdelta(\impulsevariable - \phasevectorcomponent{\impulseinterferenceindexone,\impulseinterferenceindextwo}) \\ \vdots \\ \diracdelta(\impulsevariable - (\degreevectorcomponent{\impulseinterferenceindexone,\impulseinterferenceindextwo}-1) \phasevectorcomponent{\impulseinterferenceindexone,\impulseinterferenceindextwo})
\end{pmatrix}
\\
&
	\coloneqq
	\begin{pmatrix}
		\amplitudevector{1} \impulsespectralspacemult \diracdelta(\impulsevariable) \\  \amplitudevector{2} \impulsespectralspacemult \diracdelta(\impulsevariable - \phasevectorcomponent{\impulseinterferenceindexone,\impulseinterferenceindextwo})  \\ \vdots \\ \amplitudevector{\numberofrepetitionsvectorcomponent{\impulseinterferenceindexone,\impulseinterferenceindextwo}} \impulsespectralspacemult  \diracdelta(\impulsevariable - (\degreevectorcomponent{\impulseinterferenceindexone,\impulseinterferenceindextwo}-1) \phasevectorcomponent{\impulseinterferenceindexone,\impulseinterferenceindextwo})
	\end{pmatrix}
	\cdot 
	\begin{pmatrix}
		1 \\ 1 \\ \vdots \\ 1_{\degreevectorcomponent{\impulseinterferenceindexone,\impulseinterferenceindextwo}}
	\end{pmatrix}
	\\
	&
	=
	\amplitudevector{1} \impulsespectralspacemult \diracdelta(\impulsevariable) + \amplitudevector{2} \impulsespectralspacemult \diracdelta(\impulsevariable - \phasevectorcomponent{\impulseinterferenceindexone,\impulseinterferenceindextwo}) + \dots + \amplitudevector{\numberofrepetitionsvectorcomponent{\impulseinterferenceindexone,\impulseinterferenceindextwo}} \impulsespectralspacemult  \diracdelta(\impulsevariable - (\degreevectorcomponent{\impulseinterferenceindexone,\impulseinterferenceindextwo}-1) \phasevectorcomponent{\impulseinterferenceindexone,\impulseinterferenceindextwo})
	\\
	&
	=
	\impulsespectraldensityeinsteinmonoindexed{\amplitude}{\impulseindex}{\impulseshift}{\impulsevariable}{\impulseinterferenceindexone,\impulseinterferenceindextwo}
\end{alignat}
and extend it to the \textbf{\matrixdotproductname} of the \amplitudematrixname{} and the \shiftmatrixname{}
\begin{alignat}{1}
	\amplitudematrix \matrixdotproduct \shiftmatrix{\degreevector}{\impulsevariable}  
	&
	= 
		\begin{pmatrix}
		(\amplitudevector{\impulseindex})_{1,1} & (\amplitudevector{\impulseindex})_{1,2} & \dots & (\amplitudevector{\impulseindex})_{1,\impulseinterferencedegreetwo} \\
		(\amplitudevector{\impulseindex})_{2,1} & (\amplitudevector{\impulseindex})_{2,2} & \dots & (\amplitudevector{\impulseindex})_{2,\impulseinterferencedegreetwo} \\
		\vdots & \vdots & \ddots & \vdots \\
		(\amplitudevector{\impulseindex})_{\impulseinterferencedegreeone,1} & (\amplitudevector{\impulseindex})_{\impulseinterferencedegreeone,2} & \dots & (\amplitudevector{\impulseindex})_{\impulseinterferencedegreeone,\impulseinterferencedegreetwo} \\
	\end{pmatrix}
	\matrixdotproduct
	\begin{pmatrix}
		(\phasevectorcomponent{\impulseindex})_{1,1} & (\phasevectorcomponent{\impulseindex})_{1,2} & \dots &(\phasevectorcomponent{\impulseindex})_{1,\impulseinterferencedegreetwo} \\
		(\phasevectorcomponent{\impulseindex})_{2,1} & (\phasevectorcomponent{\impulseindex})_{2,2} & \dots & (\phasevectorcomponent{\impulseindex})_{2,\impulseinterferencedegreetwo} \\
		\vdots & \vdots & \ddots & \vdots \\
		(\phasevectorcomponent{\impulseindex})_{\impulseinterferencedegreeone,1} & (\phasevectorcomponent{\impulseindex})_{\impulseinterferencedegreeone,2} & \dots & (\phasevectorcomponent{\impulseindex})_{\impulseinterferencedegreeone,\impulseinterferencedegreetwo} \\
	\end{pmatrix}
	\\
	&
	=
	\begin{pmatrix}
		(\amplitudevector{\impulseindex})_{1,1}
		\matrixdotproduct
		(\phasevectorcomponent{\impulseindex})_{1,1}   & 	(\amplitudevector{\impulseindex})_{1,2}
		\matrixdotproduct
		(\phasevectorcomponent{\impulseindex})_{1,2}   &  \dots & 	(\amplitudevector{\impulseindex})_{1,\impulseinterferencedegreetwo}
		\matrixdotproduct
		(\phasevectorcomponent{\impulseindex})_{1,\impulseinterferencedegreetwo} 
		\\
		(\amplitudevector{\impulseindex})_{2,1}
		\matrixdotproduct
		(\phasevectorcomponent{\impulseindex})_{2,1}  & 	(\amplitudevector{\impulseindex})_{2,2}
		\matrixdotproduct
		(\phasevectorcomponent{\impulseindex})_{2,2}   &  \dots & 	(\amplitudevector{\impulseindex})_{2,\impulseinterferencedegreetwo}
		\matrixdotproduct
		(\phasevectorcomponent{\impulseindex})_{2,\impulseinterferencedegreetwo} 
		\\
		\vdots & \vdots & \ddots & \vdots 
		\\
		(\amplitudevector{\impulseindex})_{\impulseinterferencedegreeone,1}
		\matrixdotproduct
		(\phasevectorcomponent{\impulseindex})_{\impulseinterferencedegreeone,1} & 	(\amplitudevector{\impulseindex})_{\impulseinterferencedegreeone,2}
		\matrixdotproduct
		(\phasevectorcomponent{\impulseindex})_{\impulseinterferencedegreeone,2}  &  \dots & 	(\amplitudevector{\impulseindex})_{\impulseinterferencedegreeone,\impulseinterferencedegreetwo}
		\matrixdotproduct
		(\phasevectorcomponent{\impulseindex})_{\impulseinterferencedegreeone,\impulseinterferencedegreetwo} 
	\end{pmatrix}
	\\
	&
	=
	\begin{pmatrix}
		(\impulsespectraldensityeinsteinmono{\amplitude}{\impulseindex}{\impulseshift}{\impulsevariable})_{1,1}  & (\impulsespectraldensityeinsteinmono{\amplitude}{\impulseindex}{\impulseshift}{\impulsevariable})_{1,2}  &  \dots & (\impulsespectraldensityeinsteinmono{\amplitude}{\impulseindex}{\impulseshift}{\impulsevariable})_{1,\impulseinterferencedegreetwo}  
		\\
		(\impulsespectraldensityeinsteinmono{\amplitude}{\impulseindex}{\impulseshift}{\impulsevariable})_{2,1}  & (\impulsespectraldensityeinsteinmono{\amplitude}{\impulseindex}{\impulseshift}{\impulsevariable})_{2,2}  &  \dots & (\impulsespectraldensityeinsteinmono{\amplitude}{\impulseindex}{\impulseshift}{\impulsevariable})_{2,\impulseinterferencedegreetwo}
		\\
		\vdots & \vdots & \ddots & \vdots 
		\\
		(\impulsespectraldensityeinsteinmono{\amplitude}{\impulseindex}{\impulseshift}{\impulsevariable})_{\impulseinterferencedegreeone,1}  & (\impulsespectraldensityeinsteinmono{\amplitude}{\impulseindex}{\impulseshift}{\impulsevariable})_{\impulseinterferencedegreeone,2}  &  \dots & (\impulsespectraldensityeinsteinmono{\amplitude}{\impulseindex}{\impulseshift}{\impulsevariable})_{\impulseinterferencedegreeone ,\impulseinterferencedegreetwo}
	\end{pmatrix} 
\end{alignat}
that is called \textbf{\impulsespectralmatrixname}.
It contains all \impulsespectraltrainshort s $(\impulsespectraldensityeinsteinmono{\amplitude}{\impulseindex}{\impulseshift}{\impulsevariable})_{\impulseinterferenceindexone,\impulseinterferenceindextwo} $ of the \impulseinterferencenameshort{} for $\impulseinterferenceindexone \in \{1,2,\dots, \impulseinterferencedegreeone\}$ and $\impulseinterferenceindextwo \in \{1,2,\dots, \impulseinterferencedegreetwo\}$. More precisely, the $\impulseinterferenceindexone$-th row describes an \impulsespectraldensityname{} $\impulsespectraldensityeinsteinmultiindexed{\amplitude}{\impulseindex}{\impulseshift}{\impulsevariable}{\impulseinterferenceindexone}$, so that the convolution of the \impulsespectraltrainshort s of a row result in one \impulsespectraldensitynameshort{} of the \impulseinterferencenameshort{}. Then, the addition of the \impulsespectraldensitynameshort s, i.e. the addition of the rows results in the \impulseinterferencenameshort. We realize this algorithm by two matrix operations $\innerconvolution{}{}$ and $	\impulsespectralinterferenceeinstein{}{}$ for the convolution of the columns and the addition of the rows.

The \textbf{\innerconvolutionname} $\innerconvolution{}{}$ convolves the columns of $\amplitudematrix \matrixdotproduct \shiftmatrix{\degreevector}{\impulsevariable} $.
Let $\unitvector{\impulseinterferenceindextwo} \in \{0,1\}^{\impulseinterferencedegreetwo}$ be a unit vector where the entry of index $\impulseinterferenceindextwo$ is 1 and all others are 0.
 Then,
\begin{alignat}{1}
	\innerconvolution{\amplitudematrix}{\shiftmatrix{\degreevector}{\impulsevariable} }
	&
	\coloneqq
	\begin{pmatrix}
		(((\amplitudematrix \matrixdotproduct \shiftmatrix{\degreevector}{\impulsevariable}  )\cdot \unitvector{1}) \cdot \unitvector{1}) \ast (((\amplitudematrix \matrixdotproduct \shiftmatrix{\degreevector}{\impulsevariable} ) \cdot \unitvector{1}) \cdot \unitvector{2} )\ast \dots \ast (((\amplitudematrix \matrixdotproduct \shiftmatrix{\degreevector}{\impulsevariable}  )\cdot \unitvector{1}) \cdot \unitvector{\impulseinterferencedegreetwo}) \\
		(((\amplitudematrix \matrixdotproduct \shiftmatrix{\degreevector}{\impulsevariable}  )\cdot \unitvector{2}) \cdot \unitvector{1}) \ast (((\amplitudematrix \matrixdotproduct \shiftmatrix{\degreevector}{\impulsevariable}  )\cdot \unitvector{2}) \cdot \unitvector{2}) \ast \dots \ast (((\amplitudematrix \matrixdotproduct \shiftmatrix{\degreevector}{\impulsevariable}  ) \cdot \unitvector{2}) \cdot \unitvector{\impulseinterferencedegreetwo}) 
		\\
		\vdots
		\\
		(((\amplitudematrix \matrixdotproduct \shiftmatrix{\degreevector}{\impulsevariable}  )\cdot \unitvector{\impulseinterferencedegreeone}) \cdot \unitvector{1}) \ast (((\amplitudematrix \matrixdotproduct \shiftmatrix{\degreevector}{\impulsevariable} ) \cdot \unitvector{\impulseinterferencedegreeone}) \cdot \unitvector{2}) \ast \dots \ast (((\amplitudematrix \matrixdotproduct \shiftmatrix{\degreevector}{\impulsevariable} ) \cdot \unitvector{\impulseinterferencedegreeone}) \cdot \unitvector{\impulseinterferencedegreetwo}) \\
	\end{pmatrix}
	\\
	&
	=
	\begin{pmatrix}
		(\impulsespectraldensityeinsteinmono{\amplitude}{\impulseindex}{\impulseshift}{\impulsevariable})_{1,1}  \ast (\impulsespectraldensityeinsteinmono{\amplitude}{\impulseindex}{\impulseshift}{\impulsevariable})_{1,2}  \ast  \ldots \ast (\impulsespectraldensityeinsteinmono{\amplitude}{\impulseindex}{\impulseshift}{\impulsevariable})_{1,\impulseinterferencedegreetwo}  
		\\
		(\impulsespectraldensityeinsteinmono{\amplitude}{\impulseindex}{\impulseshift}{\impulsevariable})_{2,1}  \ast (\impulsespectraldensityeinsteinmono{\amplitude}{\impulseindex}{\impulseshift}{\impulsevariable})_{2,2}  \ast  \ldots \ast (\impulsespectraldensityeinsteinmono{\amplitude}{\impulseindex}{\impulseshift}{\impulsevariable})_{2,\impulseinterferencedegreetwo}
		\\
		\vdots 
		\\
		(\impulsespectraldensityeinsteinmono{\amplitude}{\impulseindex}{\impulseshift}{\impulsevariable})_{\impulseinterferencedegreeone,1}  \ast (\impulsespectraldensityeinsteinmono{\amplitude}{\impulseindex}{\impulseshift}{\impulsevariable})_{\impulseinterferencedegreeone,2}  \ast  \ldots \ast (\impulsespectraldensityeinsteinmono{\amplitude}{\impulseindex}{\impulseshift}{\impulsevariable})_{\impulseinterferencedegreeone ,\impulseinterferencedegreetwo}
	\end{pmatrix} 
%	\\
%	=
%	&
%	\begin{pmatrix}
%		\sum_{\impulseindex_{1,1}=1}^{\degreevectorcomponent{1,1}} \dots \sum_{\impulseindex_{1,\impulseinterferencedegreetwo}=1}^{\degreevectorcomponent{1,\impulseinterferencedegreetwo}} 
%		\amplitudevector{\impulseindex_{1,1}} \cdot \ldots \cdot \amplitudevector{\impulseindex_{1,\impulseinterferencedegreetwo}}
%		\impulsespectralspacemult \diracdelta(\impulsevariable - (\impulseindex_{1,1} - 1) \phasevectorcomponent{1,1} - \ldots - (\impulseindex_{1,\impulseinterferencedegreetwo} - 1) \phasevectorcomponent{1,\impulseinterferencedegreetwo})
%		\\
%		\sum_{\impulseindex_{2,1}=1}^{\degreevectorcomponent{2,1}} \dots \sum_{\impulseindex_{2,\impulseinterferencedegreetwo}=1}^{\degreevectorcomponent{2,\impulseinterferencedegreetwo}} 
%		\amplitudevector{\impulseindex_{2,1}} \cdot \ldots \cdot \amplitudevector{\impulseindex_{2,\impulseinterferencedegreetwo}}
%		\impulsespectralspacemult \diracdelta(\impulsevariable - (\impulseindex_{2,1} - 1) \phasevectorcomponent{2,1} - \ldots - (\impulseindex_{2,\impulseinterferencedegreetwo} - 1) \phasevectorcomponent{2,\impulseinterferencedegreetwo})
%		\\
%		\vdots
%		\\
%		\sum_{\impulseindex_{\impulseinterferenceindexone,1}=1}^{\degreevectorcomponent{\impulseinterferenceindexone,1}} \dots \sum_{\impulseindex_{\impulseinterferenceindexone,\impulseinterferencedegreetwo}=1}^{\degreevectorcomponent{\impulseinterferenceindexone,\impulseinterferencedegreetwo}} 
%		\amplitudevector{\impulseindex_{\impulseinterferenceindexone,1}} \cdot \ldots \cdot \amplitudevector{\impulseindex_{\impulseinterferenceindexone,\impulseinterferencedegreetwo}}
%		\impulsespectralspacemult \diracdelta(\impulsevariable - (\impulseindex_{\impulseinterferenceindexone,1} - 1) \phasevectorcomponent{\impulseinterferenceindexone,1} - \ldots - (\impulseindex_{\impulseinterferenceindexone,\impulseinterferencedegreetwo} - 1) \phasevectorcomponent{\impulseinterferenceindexone,\impulseinterferencedegreetwo})
%	\end{pmatrix}
%	\\
%	=
%	&
%	\begin{pmatrix}
%		(\impulsespectraldensityeinsteinmono{\amplitude}{\impulseindex}{\impulseshift}{\impulsevariable})_{1,\boldsymbol{\impulseinterferenceindextwo}} 
%		\\
%		(\impulsespectraldensityeinsteinmono{\amplitude}{\impulseindex}{\impulseshift}{\impulsevariable})_{2,\boldsymbol{\impulseinterferenceindextwo}} 
%		\\
%		\vdots
%		\\
%		(\impulsespectraldensityeinsteinmono{\amplitude}{\impulseindex}{\impulseshift}{\impulsevariable})_{\impulseinterferencedegreeone,\boldsymbol{\impulseinterferenceindextwo}} 
%	\end{pmatrix}
%	\\
%	=&
%	\innerconvolution(\amplitudematrix \impulsespectralspacemult \shiftmatrix) 
\end{alignat}
which is called \textbf{\impulseconvolutionvectorname} or \textbf{\impulseconvolutionvectornamesf}.
	Furthermore, the \textbf{\innersumname} sums up the rows of the convolved columns and the result is the \impulseinterferencenameshort:

\begin{alignat}{1}
	\innerconvolution{\amplitudematrix}{\shiftmatrix{\degreevector}{\impulsevariable} } \cdot \mathbbm{1}
	&
	\coloneqq
	\begin{pmatrix}
		(\impulsespectraldensityeinsteinmono{\amplitude}{\impulseindex}{\impulseshift}{\impulsevariable})_{1,1}  \ast (\impulsespectraldensityeinsteinmono{\amplitude}{\impulseindex}{\impulseshift}{\impulsevariable})_{1,2}  \ast  \ldots \ast (\impulsespectraldensityeinsteinmono{\amplitude}{\impulseindex}{\impulseshift}{\impulsevariable})_{1,\impulseinterferencedegreetwo}  
		\\
		(\impulsespectraldensityeinsteinmono{\amplitude}{\impulseindex}{\impulseshift}{\impulsevariable})_{2,1}  \ast (\impulsespectraldensityeinsteinmono{\amplitude}{\impulseindex}{\impulseshift}{\impulsevariable})_{2,2}  \ast  \ldots \ast (\impulsespectraldensityeinsteinmono{\amplitude}{\impulseindex}{\impulseshift}{\impulsevariable})_{2,\impulseinterferencedegreetwo}
		\\
		\vdots 
		\\
		(\impulsespectraldensityeinsteinmono{\amplitude}{\impulseindex}{\impulseshift}{\impulsevariable})_{\impulseinterferencedegreeone,1}  \ast (\impulsespectraldensityeinsteinmono{\amplitude}{\impulseindex}{\impulseshift}{\impulsevariable})_{\impulseinterferencedegreeone,2}  \ast  \ldots \ast (\impulsespectraldensityeinsteinmono{\amplitude}{\impulseindex}{\impulseshift}{\impulsevariable})_{\impulseinterferencedegreeone ,\impulseinterferencedegreetwo}
	\end{pmatrix} 
	\cdot 
	\begin{pmatrix}
		1 \\ 1 \\ \vdots \\ 1_{\impulseinterferencedegreeone}
	\end{pmatrix}
	\\
	&
	= 
	\sum_{\impulseinterferenceindexone = 0}^{\impulseinterferencedegreeone}
	{	\impulsespectraldensityeinsteinmulti{\amplitude}{\impulseindex}{\impulseshift}{\impulsevariable}}_{\impulseinterferenceindexone}
	\\
	&
	=
	\impulsespectralinterferenceeinstein{ \hspace{1pt} \amplitudematrix \hspace{-1.5pt} \matrixdotproduct \hspace{-1.5pt} \shiftmatrix{\impulsedegree}{\impulsevariable} \hspace{1pt}}
	\\
	&
	=
	\impulseinterference{\impulseshift}{\impulsedegree}{\impulsevariable} 
\end{alignat}

%jetzt sagen wir wollen M viele davon addieren und erstmal in einstein multi, dann definieren als einstein mono mit 2 indizes und dann sagen wir können das umschreiben als matrix. we can structure these components in a matrix.
%
%To describe a single nested impulse, we present the \impulsespectraldensityname{} based on which we then define the interference as the sum of multiple nested impulses.
%\begin{definition}[\impulsespectraldensitynamecapital]\label{def:impulsespectraldensity}
%	Let $\multiperiodicimpulsedimension \in \naturalnumbers, \degreevector \coloneqq [\degreevectorcomponent{1}, \degreevectorcomponent{2}, \dots, \degreevectorcomponent{\multiperiodicimpulsedimension}] \in \naturalnumberswithinfinityvectors{\multiperiodicimpulsedimension}, \phasevector \coloneqq [\phasevectorcomponent{1}, \phasevectorcomponent{2}, \dots, \phasevectorcomponent{\multiperiodicimpulsedimension}] \in \realnumbersvector{\multiperiodicimpulsedimension}$. The vector
%	\begin{alignat}{1}
%		\impulsespectraldensity{\phasevector}{\degreevector}{\impulsevariable} \coloneqq \sum_{\impulseindex_{1} = 0}^{\degreevectorcomponent{1}} \dots \sum_{\impulseindex_{\multiperiodicimpulsedimension} = 0}^{\degreevectorcomponent{\multiperiodicimpulsedimension}}
%		\amplitudevector{\impulseindex_{1}} \cdot \ldots \cdot \amplitudevector{\impulseindex_{\multiperiodicimpulsedimension}} \impulsespectralspacemult \diracdelta(\impulsevariable - \impulseindex_{1} \phasevectorcomponent{1}  - \ldots - \impulseindex_{\multiperiodicimpulsedimension} \phasevectorcomponent{\multiperiodicimpulsedimension} ) \in \shiftedimpulseset{\impulsevariable}
%	\end{alignat}
%	with $\amplitudevector{\impulseindex_{1}} \cdot \ldots \cdot \amplitudevector{\impulseindex_{\multiperiodicimpulsedimension}} \in \realnumbers$ is called \textbf{\impulsespectraldensityname} (\impulsespectraldensitynameshort). 
%	$\amplitudevector{\impulseindex_{1}} \cdot \ldots \cdot \amplitudevector{\impulseindex_{\multiperiodicimpulsedimension}}$ is called \textbf{\amplitudename} of the SIP $\diracdelta(\impulsevariable - \impulseindex_{1} \phasevectorcomponent{1} - \impulseindex_{2} \phasevectorcomponent{2} - \ldots - \impulseindex_{\multiperiodicimpulsedimension} \phasevectorcomponent{\multiperiodicimpulsedimension} ) $ and $\multiperiodicimpulsedimension$ is the \textbf{\multiperiodicimpulsedimensionname} of $\impulsespectraldensity{\phasevector}{\degreevector}{\impulsevariable} $.
%	As a short-form notation, we define
%	\begin{alignat}{1}
%		\impulsespectraldensityeinsteinmono{\amplitude}{\impulseindex}{\impulseshift}{\impulsevariable}
%		&
%		\coloneqq 
%		\sum_{\impulseindex = 0}^{\impulsedegree}
%		\amplitudevector{\impulseindex} \impulsespectralspacemult \diracdelta(\impulsevariable - \impulseindex \phasevectorcomponent{} )
%		\\
%		\impulsespectraldensityeinsteinmulti{\amplitude}{\impulseindex}{\phase}
%		&
%		\coloneqq 
%		\sum_{\impulseindex_{1} = 0}^{\degreevectorcomponent{1}} \dots \sum_{\impulseindex_{\multiperiodicimpulsedimension} = 0}^{\degreevectorcomponent{\multiperiodicimpulsedimension}}
%		\amplitudevector{\impulseindex_{1}} \cdot \ldots \cdot \amplitudevector{\impulseindex_{\multiperiodicimpulsedimension}} \impulsespectralspacemult \diracdelta(\impulsevariable - \impulseindex_{1} \phasevectorcomponent{1} - \ldots - \impulseindex_{\multiperiodicimpulsedimension} \phasevectorcomponent{\multiperiodicimpulsedimension} )\, .
%	\end{alignat}
%	in case that $\multiperiodicimpulsedimension = 1$ and $\multiperiodicimpulsedimension > 1$.
%\end{definition}
%An \impulsespectraldensitynameshort can be also described by a convolution of $\multiperiodicimpulsedimension$ one-dimensional \impulsespectraldensitynameshorts:
%\begin{alignat}{1}
%	\impulsespectraldensityeinsteinmulti{\amplitude}{\impulseindex}{\phase} = \sum_{\impulseindex_{1} = 0}^{\degreevectorcomponent{1}} \dots \sum_{\impulseindex_{\multiperiodicimpulsedimension} = 0}^{\degreevectorcomponent{\multiperiodicimpulsedimension}}
%	\amplitudevector{\impulseindex_{1}} \cdot \ldots \cdot \amplitudevector{\impulseindex_{\multiperiodicimpulsedimension}} \impulsespectralspacemult \diracdelta(\impulsevariable - \impulseindex_{1} \phasevectorcomponent{1} - \ldots - \impulseindex_{\multiperiodicimpulsedimension} \phasevectorcomponent{\multiperiodicimpulsedimension} )
%\end{alignat}
%
%\begin{definition}[\impulseinterferencenamecapital]\label{def:impulseinterference}
%	The sum of $\numberofcomponents \in \naturalnumbers$ \impulsespectraldensitynameplural{} is called \textbf{\impulseinterferencename} (\impulseinterferencenameshort) and is defined by
%	\begin{alignat}{1}
%		\impulseinterference{\phasevector}{\degreevector}{\impulsevariable}
%		=
%		\sum_{\componentindex = 0}^{\numberofcomponents} \impulsespectraldensityeinsteinmulti{\amplitude}{\impulseindex}{\phase}
%	\end{alignat}
%	\begin{alignat}{1}
%		\impulseinterference{\phasevector}{\degreevector}{\impulsevariable}
%		=
%		&
%		\sum_{\impulseindex_{1,1} = 0}^{\degreevectorcomponent{1,1}} \dots \sum_{\impulseindex_{1,\multiperiodicimpulsedimension} = 0}^{\degreevectorcomponent{1,\multiperiodicimpulsedimension}}
%		\amplitudevector{\impulseindex_{1,1}} \cdot \ldots \cdot \amplitudevector{\impulseindex_{1,\multiperiodicimpulsedimension}} \impulsespectralspacemult \diracdelta(\impulsevariable - \impulseindex_{1,1} \phasevectorcomponent{1,1}  - \ldots - \impulseindex_{1,\multiperiodicimpulsedimension} \phasevectorcomponent{1,\multiperiodicimpulsedimension} ) 
%		\nonumber
%		\\
%		\impulsegroupoperation
%		&
%		\sum_{\impulseindex_{2,1} = 0}^{\degreevectorcomponent{2,1}} \dots \sum_{\impulseindex_{2,\multiperiodicimpulsedimension} = 0}^{\degreevectorcomponent{2,\multiperiodicimpulsedimension}}
%		\amplitudevector{\impulseindex_{2,1}} \cdot \ldots \cdot \amplitudevector{\impulseindex_{2,\multiperiodicimpulsedimension}} \impulsespectralspacemult \diracdelta(\impulsevariable - \impulseindex_{2,1} \phasevectorcomponent{2,1} - \ldots - \impulseindex_{2,\multiperiodicimpulsedimension} \phasevectorcomponent{2,\multiperiodicimpulsedimension} ) 
%		\nonumber
%		\\
%		\impulsegroupoperation 
%		&
%		\nonumber
%		\\
%		\vdots ~
%		\nonumber
%		&
%		\\
%		\impulsegroupoperation
%		&
%		\sum_{\impulseindex_{\numberofcomponents,1} = 0}^{\degreevectorcomponent{\numberofcomponents,1}} \dots \sum_{\impulseindex_{\numberofcomponents,\multiperiodicimpulsedimension} = 0}^{\degreevectorcomponent{\numberofcomponents,\multiperiodicimpulsedimension}}
%		\amplitudevector{\impulseindex_{\numberofcomponents,1}} \cdot \ldots \cdot \amplitudevector{\impulseindex_{\numberofcomponents,\multiperiodicimpulsedimension}}  \impulsespectralspacemult \diracdelta(\impulsevariable - \impulseindex_{\numberofcomponents,1} \phasevectorcomponent{\numberofcomponents,1} -  \ldots - \impulseindex_{\numberofcomponents,\multiperiodicimpulsedimension} \phasevectorcomponent{\numberofcomponents,\multiperiodicimpulsedimension} )
%		\\
%		=
%		&
%		\sum_{\componentindex = 0}^{\numberofcomponents}
%		\sum_{\impulseindex_{\componentindex,1} = 0}^{\degreevectorcomponent{\componentindex,1}} \dots \sum_{\impulseindex_{\componentindex,\multiperiodicimpulsedimension} = 0}^{\degreevectorcomponent{\componentindex,\multiperiodicimpulsedimension}}
%			\amplitudevector{\impulseindex_{\componentindex,1}} \cdot \ldots \cdot \amplitudevector{\impulseindex_{\componentindex,\multiperiodicimpulsedimension}} \impulsespectralspacemult \diracdelta(\impulsevariable - \impulseindex_{\componentindex,1} \phasevectorcomponent{\componentindex,1} - \ldots - \impulseindex_{\componentindex,\multiperiodicimpulsedimension} \phasevectorcomponent{\componentindex,\multiperiodicimpulsedimension} ) 
%	\end{alignat}
%bei den polynomen wird der grad d auch als maximum aller grade gewählt.

%We introduce a matrix notation to construct \impulseinterferencenameshorts from its basic parameters which are \shiftname s, \amplitudename s, and degrees.
%\begin{definition}
%	Let $\impulseinterferencedegreeone,\impulseinterferencedegreetwo \in \naturalnumbers$ and $\impulseinterferenceindexone \in \{1,2,\dots, \impulseinterferencedegreeone\}, \impulseinterferenceindextwo \in \{1,2,\dots, \impulseinterferencedegreetwo\}$.
%	\begin{alignat}{1}
%		(\impulsespectraldensityeinsteinmono{\amplitude}{\impulseindex}{\impulseshift}{\impulsevariable})_{\impulseinterferenceindexone,\impulseinterferenceindextwo} 
%		&
%		\coloneqq 
%		\sum_{\impulseindex_{\impulseinterferenceindexone,\impulseinterferenceindextwo} = 0}^{\degreevectorcomponent{\impulseinterferenceindexone,\impulseinterferenceindextwo}}
%		\amplitudevector{\impulseindex_{\impulseinterferenceindexone,\impulseinterferenceindextwo}} \impulsespectralspacemult \diracdelta(\impulsevariable - \impulseindex_{\impulseinterferenceindexone,\impulseinterferenceindextwo} \phasevectorcomponent{\impulseinterferenceindexone,\impulseinterferenceindextwo} )
%		\\
%		(\impulsespectraldensityeinsteinmono{\amplitude}{\impulseindex}{\impulseshift}{\impulsevariable})_{\impulseinterferenceindexone,\boldsymbol{\impulseinterferenceindextwo}}
%		&
%		\coloneqq 
%		\sum_{\impulseindex_{\impulseinterferenceindexone,1}=1}^{\degreevectorcomponent{\impulseinterferenceindexone,1}} \dots \sum_{\impulseindex_{\impulseinterferenceindexone,\impulseinterferencedegreetwo}=1}^{\degreevectorcomponent{\impulseinterferenceindexone,\impulseinterferencedegreetwo}} 
%		\amplitudevector{\impulseindex_{\impulseinterferenceindexone,1}} \cdot \ldots \cdot \amplitudevector{\impulseindex_{\impulseinterferenceindexone,\impulseinterferencedegreetwo}}
%		\impulsespectralspacemult \diracdelta(\impulsevariable - (\impulseindex_{\impulseinterferenceindexone,1} - 1) \phasevectorcomponent{\impulseinterferenceindexone,1} - \ldots - (\impulseindex_{\impulseinterferenceindexone,\impulseinterferencedegreetwo} - 1) \phasevectorcomponent{\impulseinterferenceindexone,\impulseinterferencedegreetwo})
%		\\
%	(\impulsespectraldensityeinsteinmono{\amplitude}{\impulseindex}{\impulseshift}{\impulsevariable})_{\boldsymbol{\impulseinterferenceindexone},\boldsymbol{\impulseinterferenceindextwo}}
%	&
%	\coloneqq
%			\sum_{\impulseinterferenceindexone = 0}^{\impulseinterferencedegreeone} (\impulsespectraldensityeinsteinmono{\amplitude}{\impulseindex}{\impulseshift}{\impulsevariable})_{\impulseinterferenceindexone,\boldsymbol{\impulseinterferenceindextwo}} 
%	\end{alignat}
%	For each $	(\impulsespectraldensityeinsteinmono{\amplitude}{\impulseindex}{\impulseshift}{\impulsevariable})_{\impulseinterferenceindexone,\impulseinterferenceindextwo} $, we define a vector of amplitudes $\amplitudevector{}_{\impulseinterferenceindexone,\impulseinterferenceindextwo} \coloneqq (\amplitudevector{1}, \amplitudevector{2}, \dots, \amplitudevector{\impulsedegree})_{\impulseinterferenceindexone,\impulseinterferenceindextwo} ^{\intercal}$ and a vector of shifted impulses $\phasevectorcomponent{\impulseinterferenceindexone,\impulseinterferenceindextwo} \coloneqq (\diracdelta(\impulsevariable), \diracdelta(\impulsevariable - \phasevectorcomponent{}), \dots, \diracdelta(\impulsevariable - (\impulsedegree - 1)\phasevectorcomponent{}))_{\impulseinterferenceindexone,\impulseinterferenceindextwo}^{\intercal}$ and their product
%	\begin{alignat}{1}
%		\amplitudevector{}_{\impulseinterferenceindexone,\impulseinterferenceindextwo}  \impulsespectralspacemult \phasevectorcomponent{\impulseinterferenceindexone,\impulseinterferenceindextwo}
%		=
%		\begin{pmatrix}
%			\amplitudevector{1} \\ \amplitudevector{2} \\ \vdots \\ \amplitudevector{\impulsedegree}
%		\end{pmatrix}_{\impulseinterferenceindexone,\impulseinterferenceindextwo} 
%		\impulsespectralspacemult
%		\begin{pmatrix}
%			\diracdelta(\impulsevariable) \\ \diracdelta(\impulsevariable - \phasevectorcomponent{}) \\ \vdots \\ \diracdelta(\impulsevariable - (\impulsedegree - 1)\phasevectorcomponent{})
%		\end{pmatrix}_{\impulseinterferenceindexone,\impulseinterferenceindextwo} 
%		&
%		\coloneqq 
%		\sum_{\impulseindex_{\impulseinterferenceindexone,\impulseinterferenceindextwo} = 0}^{\degreevectorcomponent{\impulseinterferenceindexone,\impulseinterferenceindextwo}}
%		\amplitudevector{\impulseindex_{\impulseinterferenceindexone,\impulseinterferenceindextwo}} \impulsespectralspacemult \diracdelta(\impulsevariable - \impulseindex_{\impulseinterferenceindexone,\impulseinterferenceindextwo} \phasevectorcomponent{\impulseinterferenceindexone,\impulseinterferenceindextwo} ) = 	(\impulsespectraldensityeinsteinmono{\amplitude}{\impulseindex}{\impulseshift}{\impulsevariable})_{\impulseinterferenceindexone,\impulseinterferenceindextwo} 
%	\end{alignat}
%	Then, we define matrices of amplitudes and shifts
%	\begin{alignat}{1}
%		\amplitudematrix 
%		=
%	\begin{pmatrix}
%		\amplitudevector{}_{1,1} & \amplitudevector{}_{1,2} & \dots & \amplitudevector{}_{1,\impulseinterferencedegreetwo} \\
%		\amplitudevector{}_{2,1} & \amplitudevector{}_{2,2} & \dots & \amplitudevector{}_{2,\impulseinterferencedegreetwo} \\
%		\vdots & \vdots & \ddots & \vdots \\
%		\amplitudevector{}_{\impulseinterferencedegreeone,1} & \amplitudevector{}_{\impulseinterferencedegreeone,2} & \dots & \amplitudevector{}_{\impulseinterferencedegreeone,\impulseinterferencedegreetwo} \\
%	\end{pmatrix}
%	~~~~~~~~
%			\shiftmatrix 
%			=
%			\begin{pmatrix}
%			\phasevectorcomponent{1,1} & \phasevectorcomponent{1,2} & \dots &\phasevectorcomponent{1,\impulseinterferencedegreetwo} \\
%			\phasevectorcomponent{2,1} & \phasevectorcomponent{2,2} & \dots & \phasevectorcomponent{2,\impulseinterferencedegreetwo} \\
%			\vdots & \vdots & \ddots & \vdots \\
%			\phasevectorcomponent{\impulseinterferencedegreeone,1} & \phasevectorcomponent{\impulseinterferencedegreeone,2} & \dots & \phasevectorcomponent{\impulseinterferencedegreeone,\impulseinterferencedegreetwo} \\
%		\end{pmatrix}
%	\end{alignat}
%	and their direct product
%	\begin{alignat}{1}
%		\amplitudematrix \impulsespectralspacemult \shiftmatrix
%		&
%		=
%		\begin{pmatrix}
%			\amplitudevector{}_{1,1} & \amplitudevector{}_{1,2} & \dots & \amplitudevector{}_{1,\impulseinterferencedegreetwo} \\
%			\amplitudevector{}_{2,1} & \amplitudevector{}_{2,2} & \dots & \amplitudevector{}_{2,\impulseinterferencedegreetwo} \\
%			\vdots & \vdots & \ddots & \vdots \\
%			\amplitudevector{}_{\impulseinterferencedegreeone,1} & \amplitudevector{}_{\impulseinterferencedegreeone,2} & \dots & \amplitudevector{}_{\impulseinterferencedegreeone,\impulseinterferencedegreetwo} \\
%		\end{pmatrix}
%		\impulsespectralspacemult
%		\begin{pmatrix}
%			\phasevectorcomponent{1,1} & \phasevectorcomponent{1,2} & \dots &\phasevectorcomponent{1,\impulseinterferencedegreetwo} \\
%			\phasevectorcomponent{2,1} & \phasevectorcomponent{2,2} & \dots & \phasevectorcomponent{2,\impulseinterferencedegreetwo} \\
%			\vdots & \vdots & \ddots & \vdots \\
%			\phasevectorcomponent{\impulseinterferencedegreeone,1} & \phasevectorcomponent{\impulseinterferencedegreeone,2} & \dots & \phasevectorcomponent{\impulseinterferencedegreeone,\impulseinterferencedegreetwo} \\
%		\end{pmatrix}
%	\\
%	&
%	\coloneqq
%	\begin{pmatrix}
%		\amplitudevector{}_{1,1}  \impulsespectralspacemult \phasevectorcomponent{1,1} & \amplitudevector{}_{1,2}  \impulsespectralspacemult \phasevectorcomponent{1,2} & \dots & \amplitudevector{}_{1,\impulseinterferencedegreetwo}  \impulsespectralspacemult \phasevectorcomponent{1,\impulseinterferencedegreetwo}
%		\\
%		\amplitudevector{}_{2,1}  \impulsespectralspacemult \phasevectorcomponent{2,1} & \amplitudevector{}_{2,2}  \impulsespectralspacemult \phasevectorcomponent{2,2} & \dots & \amplitudevector{}_{2,\impulseinterferencedegreetwo}  \impulsespectralspacemult \phasevectorcomponent{2,\impulseinterferencedegreetwo}
%		\\
%		\vdots & \vdots & \ddots & \vdots \\
%		\amplitudevector{}_{\impulseinterferencedegreeone,1}  \impulsespectralspacemult \phasevectorcomponent{\impulseinterferencedegreeone,1} & \amplitudevector{}_{\impulseinterferencedegreeone,2}  \impulsespectralspacemult \phasevectorcomponent{\impulseinterferencedegreeone,2} & \dots & \amplitudevector{}_{\impulseinterferencedegreeone,\impulseinterferencedegreetwo}  \impulsespectralspacemult \phasevectorcomponent{\impulseinterferencedegreeone,\impulseinterferencedegreetwo}
%	\end{pmatrix}
%	\\
%	&
%	=
%		\begin{pmatrix}
%			(\impulsespectraldensityeinsteinmono{\amplitude}{\impulseindex}{\impulseshift}{\impulsevariable})_{1,1}  & (\impulsespectraldensityeinsteinmono{\amplitude}{\impulseindex}{\impulseshift}{\impulsevariable})_{1,2}  &  \dots & (\impulsespectraldensityeinsteinmono{\amplitude}{\impulseindex}{\impulseshift}{\impulsevariable})_{1,\impulseinterferencedegreetwo}  
%			\\
%			(\impulsespectraldensityeinsteinmono{\amplitude}{\impulseindex}{\impulseshift}{\impulsevariable})_{2,1}  & (\impulsespectraldensityeinsteinmono{\amplitude}{\impulseindex}{\impulseshift}{\impulsevariable})_{2,2}  &  \dots & (\impulsespectraldensityeinsteinmono{\amplitude}{\impulseindex}{\impulseshift}{\impulsevariable})_{2,\impulseinterferencedegreetwo}
%			\\
%			\vdots & \vdots & \ddots & \vdots 
%			\\
%			(\impulsespectraldensityeinsteinmono{\amplitude}{\impulseindex}{\impulseshift}{\impulsevariable})_{\impulseinterferencedegreeone,1}  & (\impulsespectraldensityeinsteinmono{\amplitude}{\impulseindex}{\impulseshift}{\impulsevariable})_{\impulseinterferencedegreeone,2}  &  \dots & (\impulsespectraldensityeinsteinmono{\amplitude}{\impulseindex}{\impulseshift}{\impulsevariable})_{\impulseinterferencedegreeone ,\impulseinterferencedegreetwo}
%		\end{pmatrix}
%	\\
%	&
%	\eqqcolon		
%	\interferencematrix{\impulseinterferencedegreeone}{\impulseinterferencedegreetwo}{\impulsevariable}
%	\end{alignat}
%	Then, $\innerconvolution \colon \shiftedimpulseset{\impulsevariable}^{\impulseinterferencedegreeone \times \impulseinterferencedegreetwo} \to \shiftedimpulseset{\impulsevariable}^{\impulseinterferencedegreeone} $ convovles the columns of $\interferencematrix{\impulseinterferencedegreeone}{\impulseinterferencedegreetwo}{\impulsevariable}$. Formally,
%	\begin{alignat}{1}
%		&
%		\innerconvolution(	\interferencematrix{\impulseinterferencedegreeone}{\impulseinterferencedegreetwo}{\impulsevariable}) 
%		\\
%		\coloneqq
%		&
%		\begin{pmatrix}
%			(\impulsespectraldensityeinsteinmono{\amplitude}{\impulseindex}{\impulseshift}{\impulsevariable})_{1,1}  \ast (\impulsespectraldensityeinsteinmono{\amplitude}{\impulseindex}{\impulseshift}{\impulsevariable})_{1,2}  \ast  \ldots \ast (\impulsespectraldensityeinsteinmono{\amplitude}{\impulseindex}{\impulseshift}{\impulsevariable})_{1,\impulseinterferencedegreetwo}  
%			\\
%			(\impulsespectraldensityeinsteinmono{\amplitude}{\impulseindex}{\impulseshift}{\impulsevariable})_{2,1}  \ast (\impulsespectraldensityeinsteinmono{\amplitude}{\impulseindex}{\impulseshift}{\impulsevariable})_{2,2}  \ast  \ldots \ast (\impulsespectraldensityeinsteinmono{\amplitude}{\impulseindex}{\impulseshift}{\impulsevariable})_{2,\impulseinterferencedegreetwo}
%			\\
%			\vdots 
%			\\
%			(\impulsespectraldensityeinsteinmono{\amplitude}{\impulseindex}{\impulseshift}{\impulsevariable})_{\impulseinterferencedegreeone,1}  \ast (\impulsespectraldensityeinsteinmono{\amplitude}{\impulseindex}{\impulseshift}{\impulsevariable})_{\impulseinterferencedegreeone,2}  \ast  \ldots \ast (\impulsespectraldensityeinsteinmono{\amplitude}{\impulseindex}{\impulseshift}{\impulsevariable})_{\impulseinterferencedegreeone ,\impulseinterferencedegreetwo}
%		\end{pmatrix} 
%	\\
%	=
%	&
%	\begin{pmatrix}
%		\sum_{\impulseindex_{1,1}=1}^{\degreevectorcomponent{1,1}} \dots \sum_{\impulseindex_{1,\impulseinterferencedegreetwo}=1}^{\degreevectorcomponent{1,\impulseinterferencedegreetwo}} 
%		\amplitudevector{\impulseindex_{1,1}} \cdot \ldots \cdot \amplitudevector{\impulseindex_{1,\impulseinterferencedegreetwo}}
%		\impulsespectralspacemult \diracdelta(\impulsevariable - (\impulseindex_{1,1} - 1) \phasevectorcomponent{1,1} - \ldots - (\impulseindex_{1,\impulseinterferencedegreetwo} - 1) \phasevectorcomponent{1,\impulseinterferencedegreetwo})
%		\\
%		\sum_{\impulseindex_{2,1}=1}^{\degreevectorcomponent{2,1}} \dots \sum_{\impulseindex_{2,\impulseinterferencedegreetwo}=1}^{\degreevectorcomponent{2,\impulseinterferencedegreetwo}} 
%		\amplitudevector{\impulseindex_{2,1}} \cdot \ldots \cdot \amplitudevector{\impulseindex_{2,\impulseinterferencedegreetwo}}
%		\impulsespectralspacemult \diracdelta(\impulsevariable - (\impulseindex_{2,1} - 1) \phasevectorcomponent{2,1} - \ldots - (\impulseindex_{2,\impulseinterferencedegreetwo} - 1) \phasevectorcomponent{2,\impulseinterferencedegreetwo})
%		\\
%		\vdots
%		\\
%		\sum_{\impulseindex_{\impulseinterferenceindexone,1}=1}^{\degreevectorcomponent{\impulseinterferenceindexone,1}} \dots \sum_{\impulseindex_{\impulseinterferenceindexone,\impulseinterferencedegreetwo}=1}^{\degreevectorcomponent{\impulseinterferenceindexone,\impulseinterferencedegreetwo}} 
%		\amplitudevector{\impulseindex_{\impulseinterferenceindexone,1}} \cdot \ldots \cdot \amplitudevector{\impulseindex_{\impulseinterferenceindexone,\impulseinterferencedegreetwo}}
%		\impulsespectralspacemult \diracdelta(\impulsevariable - (\impulseindex_{\impulseinterferenceindexone,1} - 1) \phasevectorcomponent{\impulseinterferenceindexone,1} - \ldots - (\impulseindex_{\impulseinterferenceindexone,\impulseinterferencedegreetwo} - 1) \phasevectorcomponent{\impulseinterferenceindexone,\impulseinterferencedegreetwo})
%	\end{pmatrix}
%	\\
%	=
%	&
%	\begin{pmatrix}
%		(\impulsespectraldensityeinsteinmono{\amplitude}{\impulseindex}{\impulseshift}{\impulsevariable})_{1,\boldsymbol{\impulseinterferenceindextwo}} 
%		\\
%		(\impulsespectraldensityeinsteinmono{\amplitude}{\impulseindex}{\impulseshift}{\impulsevariable})_{2,\boldsymbol{\impulseinterferenceindextwo}} 
%		\\
%		\vdots
%		\\
%		(\impulsespectraldensityeinsteinmono{\amplitude}{\impulseindex}{\impulseshift}{\impulsevariable})_{\impulseinterferencedegreeone,\boldsymbol{\impulseinterferenceindextwo}} 
%	\end{pmatrix}
%	\\
%	=&
%		\innerconvolution(\amplitudematrix \impulsespectralspacemult \shiftmatrix) 
%	\end{alignat}
%	Furthermore, $\innersum \colon \shiftedimpulseset{\impulsevariable}^{\impulseinterferencedegreeone} \to \shiftedimpulseset{\impulsevariable}$ such that 
%	\begin{alignat}{1}
%		\innersum(\innerconvolution(\amplitudematrix \impulsespectralspacemult \shiftmatrix))
%		=
%		\innersum\left(
%		\begin{pmatrix}
%			(\impulsespectraldensityeinsteinmono{\amplitude}{\impulseindex}{\impulseshift}{\impulsevariable})_{1,\boldsymbol{\impulseinterferenceindextwo}} 
%			\\
%			(\impulsespectraldensityeinsteinmono{\amplitude}{\impulseindex}{\impulseshift}{\impulsevariable})_{2,\boldsymbol{\impulseinterferenceindextwo}} 
%			\\
%			\vdots
%			\\
%			(\impulsespectraldensityeinsteinmono{\amplitude}{\impulseindex}{\impulseshift}{\impulsevariable})_{\impulseinterferencedegreeone,\boldsymbol{\impulseinterferenceindextwo}} 
%		\end{pmatrix}
%	\right)
%	\coloneqq 
%	\sum_{\impulseinterferenceindexone = 0}^{\impulseinterferencedegreeone} (\impulsespectraldensityeinsteinmono{\amplitude}{\impulseindex}{\impulseshift}{\impulsevariable})_{\impulseinterferenceindexone,\boldsymbol{\impulseinterferenceindextwo}} 
%	=
%	(\impulsespectraldensityeinsteinmono{\amplitude}{\impulseindex}{\impulseshift}{\impulsevariable})_{\boldsymbol{\impulseinterferenceindexone},\boldsymbol{\impulseinterferenceindextwo}}
%	\end{alignat}
%\end{definition}
%Hence, an \impulseinterferencename{} can be described 
%
%
%We present a matrix description of a multivariate impulse vector based on the weight, shift and degree of the impulse since they provide a finite representation of the vector.
%Based on this matrix, we introduce an operator to construct any multivariate impulse using convolution and linear combinations of impulses.
%We define the matrix in the following and then explain the operator.
%
%
%We present  the convolution of two Dirac distributions $(\tau_a \delta \ast \tau_b\delta)(x)$. 
%
%
%\begin{definition}[Impulse matrix]\label{def:impulsematrix}
%	TODO
%\end{definition}
%\begin{definition}[Impulse constructor]
%	TODO
%\end{definition}
%\begin{theorem}[Impulse constructor]
%	
%	The impulse operator can generate any multivariate impulse.
%\end{theorem}
\subsection{Example of operator}
Let 
\begin{alignat}{2}
	\amplitudematrixentry{\impulseindex}{1}{1} &= \begin{pmatrix}
		2 & 4 & 5
	\end{pmatrix}^{\intercal}
	~~~
	\shiftmatrixentry{\impulseindex}{\impulsevariable}{1}{1} = \begin{pmatrix}
		\diracdelta(\impulsevariable) & 	\diracdelta(\impulsevariable - 5) & 		\diracdelta(\impulsevariable - 10)
	\end{pmatrix}^{\intercal}
	\\
	\amplitudematrixentry{\impulseindex}{1}{2} &= \begin{pmatrix}
		3 & 1 & 8
	\end{pmatrix}^{\intercal}
	~~~
	\shiftmatrixentry{\impulseindex}{\impulsevariable}{1}{2} = \begin{pmatrix}
		\diracdelta(\impulsevariable) & 	\diracdelta(\impulsevariable - 3) & 		\diracdelta(\impulsevariable - 6)
	\end{pmatrix}^{\intercal}
	\\
	\amplitudematrixentry{\impulseindex}{2}{1} &= \begin{pmatrix}
		6 & 2 & 3
	\end{pmatrix}^{\intercal}
	~~~
	\shiftmatrixentry{\impulseindex}{\impulsevariable}{2}{1} = \begin{pmatrix}
		\diracdelta(\impulsevariable) & 	\diracdelta(\impulsevariable - 7) & 		\diracdelta(\impulsevariable - 14)
	\end{pmatrix}^{\intercal}
	\\
	\amplitudematrixentry{\impulseindex}{2}{2} &= \begin{pmatrix}
		9 & 4 & 2
	\end{pmatrix}^{\intercal}
	~~~
	\shiftmatrixentry{\impulseindex}{\impulsevariable}{2}{2} = \begin{pmatrix}
		\diracdelta(\impulsevariable) & 	\diracdelta(\impulsevariable - 4) & 		\diracdelta(\impulsevariable - 8)
	\end{pmatrix}^{\intercal}
\end{alignat}
be \amplitudevectorname s and \shiftvectorname s that are elements of the \amplitudematrixname{} 
\begin{alignat}{1}
	\amplitudematrix 
	= 
	\begin{pmatrix}
		\amplitudematrixentry{\impulseindex}{1}{1} & \amplitudematrixentry{\impulseindex}{1}{2} \\
		\amplitudematrixentry{\impulseindex}{2}{1} & \amplitudematrixentry{\impulseindex}{2}{2}   
	\end{pmatrix}
	=
	\begin{pmatrix}
		\begin{pmatrix}
			2 \\ 4 \\ 5
		\end{pmatrix}
		 & 
		 \begin{pmatrix}
		 	3 \\ 1 \\ 8
		 \end{pmatrix}
	  \\
		  \begin{pmatrix}
		  	6 \\ 2 \\ 3
		  \end{pmatrix}
	  &
	   \begin{pmatrix}
	  	9 \\ 4 \\ 2
	  \end{pmatrix}
	\end{pmatrix}
\end{alignat}
and the \shiftmatrixname
\begin{alignat}{1}
	\shiftmatrix{\impulseindex}{\impulsevariable} 
	= 
	\begin{pmatrix}
		\shiftmatrixentry{\impulseindex}{\impulsevariable}{1}{1} & 	\shiftmatrixentry{\impulseindex}{\impulsevariable}{1}{2}  \\
		\shiftmatrixentry{\impulseindex}{\impulsevariable}{2}{1} & 	\shiftmatrixentry{\impulseindex}{\impulsevariable}{2}{2} 
	\end{pmatrix}
	=
	\begin{pmatrix}
		\begin{pmatrix}
			\diracdelta(\impulsevariable) \\	\diracdelta(\impulsevariable - 5) \\ \diracdelta(\impulsevariable - 10)
		\end{pmatrix}
		&
		\begin{pmatrix}
			\diracdelta(\impulsevariable) \\	\diracdelta(\impulsevariable - 3) \\ \diracdelta(\impulsevariable - 6)
		\end{pmatrix}
		\\
		\begin{pmatrix}
			\diracdelta(\impulsevariable) \\	\diracdelta(\impulsevariable - 7) \\ \diracdelta(\impulsevariable - 14)
		\end{pmatrix}
		&
		\begin{pmatrix}
			\diracdelta(\impulsevariable) \\	\diracdelta(\impulsevariable - 4) \\	\diracdelta(\impulsevariable - 8)
		\end{pmatrix}
	\end{pmatrix}
\end{alignat}
We compute the \dotproductname{} of the \amplitudevectorname s and the \shiftvectorname s to compute the \matrixdotproductname{} of the \amplitudematrixname{} and the \shiftmatrixname:
\begin{alignat}{1}
	\amplitudematrixentry{\impulseindex}{1}{1} \matrixdotproduct \shiftmatrixentry{\impulseindex}{\impulsevariable}{1}{1} 
	&
	=   
	\begin{pmatrix}
		2 \\ 4 \\ 5
	\end{pmatrix}
	\matrixdotproduct 
	\begin{pmatrix}
		\diracdelta(\impulsevariable)  \\	\diracdelta(\impulsevariable - 5) \\	\diracdelta(\impulsevariable - 10)
	\end{pmatrix}
	=
	\begin{pmatrix}
		2 \impulsespectralspacemult \diracdelta(\impulsevariable)  \\	4 \impulsespectralspacemult \diracdelta(\impulsevariable - 5) \\	5 \impulsespectralspacemult\diracdelta(\impulsevariable - 10)
	\end{pmatrix}
	\cdot 
	\begin{pmatrix}
		1 \\ 1 \\ 1
	\end{pmatrix}
	\\
	&
	=
	2 \, \diracdelta(\impulsevariable) + 4 \, \diracdelta(\impulsevariable - 5) + 5  \, \diracdelta(\impulsevariable - 10)
	\\
	&
	=
	(\impulsespectraldensityeinsteinmono{\amplitude}{\impulseindex}{\impulseshift}{\impulsevariable})_{1,1} 
	\\
	\amplitudematrixentry{\impulseindex}{1}{2} \matrixdotproduct \shiftmatrixentry{\impulseindex}{\impulsevariable}{1}{2} 
	&
	=   
	\begin{pmatrix}
		3 \\ 1 \\ 8
	\end{pmatrix}
	\matrixdotproduct 
	\begin{pmatrix}
		\diracdelta(\impulsevariable)  \\	\diracdelta(\impulsevariable - 3) \\	\diracdelta(\impulsevariable - 6)
	\end{pmatrix}
	=
	\begin{pmatrix}
		3 \impulsespectralspacemult \diracdelta(\impulsevariable)  \\	1 \impulsespectralspacemult \diracdelta(\impulsevariable - 3) \\	8 \impulsespectralspacemult\diracdelta(\impulsevariable - 6)
	\end{pmatrix}
	\cdot 
	\begin{pmatrix}
		1 \\ 1 \\ 1
	\end{pmatrix}
	\\
	&
	=
	3 \,  \diracdelta(\impulsevariable) + \diracdelta(\impulsevariable - 3) + 8  \, \diracdelta(\impulsevariable - 6)
	\\
	&
	=
	(\impulsespectraldensityeinsteinmono{\amplitude}{\impulseindex}{\impulseshift}{\impulsevariable})_{1,2} 
	\\
	\amplitudematrixentry{\impulseindex}{2}{1} \matrixdotproduct \shiftmatrixentry{\impulseindex}{\impulsevariable}{2}{1} 
	&
	=   
	\begin{pmatrix}
		6 \\ 2 \\ 3
	\end{pmatrix}
	\matrixdotproduct 
	\begin{pmatrix}
		\diracdelta(\impulsevariable)  \\	\diracdelta(\impulsevariable - 7) \\	\diracdelta(\impulsevariable - 14)
	\end{pmatrix}
	=
	\begin{pmatrix}
		6 \impulsespectralspacemult \diracdelta(\impulsevariable)  \\	2 \impulsespectralspacemult \diracdelta(\impulsevariable - 7) \\	3 \impulsespectralspacemult\diracdelta(\impulsevariable - 14)
	\end{pmatrix}
	\cdot 
	\begin{pmatrix}
		1 \\ 1 \\ 1
	\end{pmatrix}
	\\
	&
	=
	6  \, \diracdelta(\impulsevariable) + 2 \,   \diracdelta(\impulsevariable - 7) + 3  \, \diracdelta(\impulsevariable - 14)
	\\
	&
	=
	(\impulsespectraldensityeinsteinmono{\amplitude}{\impulseindex}{\impulseshift}{\impulsevariable})_{2,1} 
	\\
	\amplitudematrixentry{\impulseindex}{2}{2} \matrixdotproduct \shiftmatrixentry{\impulseindex}{\impulsevariable}{2}{2} 
	&
	=   
	\begin{pmatrix}
		9 \\ 4 \\ 2
	\end{pmatrix}
	\matrixdotproduct 
	\begin{pmatrix}
		\diracdelta(\impulsevariable)  \\	\diracdelta(\impulsevariable - 4) \\	\diracdelta(\impulsevariable - 8)
	\end{pmatrix}
	=
	\begin{pmatrix}
		9 \impulsespectralspacemult \diracdelta(\impulsevariable)  \\	4 \impulsespectralspacemult \diracdelta(\impulsevariable - 4) \\	2 \impulsespectralspacemult\diracdelta(\impulsevariable - 8)
	\end{pmatrix}
	\cdot 
	\begin{pmatrix}
		1 \\ 1 \\ 1
	\end{pmatrix}
	\\
	&
	=
	9  \, \diracdelta(\impulsevariable) + 4   \, \diracdelta(\impulsevariable - 4) + 2 \,  \diracdelta(\impulsevariable - 8)
	\\
	&
	=
	(\impulsespectraldensityeinsteinmono{\amplitude}{\impulseindex}{\impulseshift}{\impulsevariable})_{2,2} 
\end{alignat}
\begin{alignat}{1}
	\amplitudematrix \matrixdotproduct \shiftmatrix{\impulseindex}{\impulsevariable} 
	&
	=
	\begin{pmatrix}
			\amplitudematrixentry{\impulseindex}{1}{1} \matrixdotproduct \shiftmatrixentry{\impulseindex}{\impulsevariable}{1}{1}
			&
				\amplitudematrixentry{\impulseindex}{1}{2} \matrixdotproduct \shiftmatrixentry{\impulseindex}{\impulsevariable}{1}{2}  
		\\
			\amplitudematrixentry{\impulseindex}{2}{1} \matrixdotproduct \shiftmatrixentry{\impulseindex}{\impulsevariable}{2}{1} 
		&
			\amplitudematrixentry{\impulseindex}{2}{2} \matrixdotproduct \shiftmatrixentry{\impulseindex}{\impulsevariable}{2}{2} 
	\end{pmatrix}
\\
	&
	=
	\begin{pmatrix}
		\begin{pmatrix}
			2 \\ 4 \\ 5
		\end{pmatrix} 
		\matrixdotproduct
		\begin{pmatrix}
			\diracdelta(\impulsevariable) \\	\diracdelta(\impulsevariable - 5) \\ \diracdelta(\impulsevariable - 10)
		\end{pmatrix}
		& 
		\begin{pmatrix}
			3 \\ 1 \\ 8
		\end{pmatrix}
		\matrixdotproduct
		\begin{pmatrix}
			\diracdelta(\impulsevariable) \\	\diracdelta(\impulsevariable - 3) \\ \diracdelta(\impulsevariable - 6)
		\end{pmatrix}
		\\
		\begin{pmatrix}
			6 \\ 2 \\ 3
		\end{pmatrix}
		\matrixdotproduct
		\begin{pmatrix}
			\diracdelta(\impulsevariable) \\	\diracdelta(\impulsevariable - 7) \\ \diracdelta(\impulsevariable - 14)
		\end{pmatrix}
		&
		\begin{pmatrix}
			9 \\ 4 \\ 2
		\end{pmatrix}
		\matrixdotproduct
		\begin{pmatrix}
			\diracdelta(\impulsevariable) \\	\diracdelta(\impulsevariable - 4) \\	\diracdelta(\impulsevariable - 8)
		\end{pmatrix}
	\end{pmatrix}
\\
&
=
	\begin{pmatrix}
		\begin{pmatrix}
			2 \impulsespectralspacemult \diracdelta(\impulsevariable)  \\	4 \impulsespectralspacemult \diracdelta(\impulsevariable - 5) \\	5 \impulsespectralspacemult\diracdelta(\impulsevariable - 10)
		\end{pmatrix}
		\cdot 
		\begin{pmatrix}
			1 \\ 1 \\ 1
		\end{pmatrix}
		&
		\begin{pmatrix}
			3 \impulsespectralspacemult \diracdelta(\impulsevariable)  \\	1 \impulsespectralspacemult \diracdelta(\impulsevariable - 3) \\	8 \impulsespectralspacemult\diracdelta(\impulsevariable - 6)
		\end{pmatrix}
		\cdot 
		\begin{pmatrix}
			1 \\ 1 \\ 1
		\end{pmatrix}
		\\
		\begin{pmatrix}
			6 \impulsespectralspacemult \diracdelta(\impulsevariable)  \\	2 \impulsespectralspacemult \diracdelta(\impulsevariable - 7) \\	3 \impulsespectralspacemult\diracdelta(\impulsevariable - 14)
		\end{pmatrix}
		\cdot 
		\begin{pmatrix}
			1 \\ 1 \\ 1
		\end{pmatrix}
		&
		\begin{pmatrix}
			9 \impulsespectralspacemult \diracdelta(\impulsevariable)  \\	4 \impulsespectralspacemult \diracdelta(\impulsevariable - 4) \\	2 \impulsespectralspacemult\diracdelta(\impulsevariable - 8)
		\end{pmatrix}
		\cdot 
		\begin{pmatrix}
			1 \\ 1 \\ 1
		\end{pmatrix}
	\end{pmatrix}
\\
	&
	= 
	\begin{pmatrix}
			2 \, \diracdelta(\impulsevariable) + 4   \, \diracdelta(\impulsevariable - 5) + 5  \, \diracdelta(\impulsevariable - 10)
		&
			3  \, \diracdelta(\impulsevariable) + \diracdelta(\impulsevariable - 3) + 8 \,  \diracdelta(\impulsevariable - 6)
		\\
			6 \,  \diracdelta(\impulsevariable) + 2  \,  \diracdelta(\impulsevariable - 7) + 3 \,  \diracdelta(\impulsevariable - 14) 
			&
			9  \, \diracdelta(\impulsevariable) + 4  \,  \diracdelta(\impulsevariable - 4) + 2  \, \diracdelta(\impulsevariable - 8)
	\end{pmatrix}
\\
&
=
\begin{pmatrix}
		(\impulsespectraldensityeinsteinmono{\amplitude}{\impulseindex}{\impulseshift}{\impulsevariable})_{1,1} 
		&
		(\impulsespectraldensityeinsteinmono{\amplitude}{\impulseindex}{\impulseshift}{\impulsevariable})_{1,2} 
		\\
		(\impulsespectraldensityeinsteinmono{\amplitude}{\impulseindex}{\impulseshift}{\impulsevariable})_{2,1} 
		&
		(\impulsespectraldensityeinsteinmono{\amplitude}{\impulseindex}{\impulseshift}{\impulsevariable})_{2,2} 
\end{pmatrix}
\end{alignat}
Next, we convolve the columns of $	\amplitudematrix \matrixdotproduct \shiftmatrix{\impulseindex}{\impulsevariable} $ as follows

\begin{alignat}{1}
	\innerconvolution{\amplitudematrix}{\shiftmatrix{\degreevector}{\impulsevariable} }
&
=
\begin{pmatrix}
	(((\amplitudematrix \matrixdotproduct \shiftmatrix{\degreevector}{\impulsevariable}  )\cdot \unitvector{1}) \cdot \unitvector{1}) \ast (((\amplitudematrix \matrixdotproduct \shiftmatrix{\degreevector}{\impulsevariable} ) \cdot \unitvector{2}) \cdot \unitvector{1} )\\
	(((\amplitudematrix \matrixdotproduct \shiftmatrix{\degreevector}{\impulsevariable}  )\cdot \unitvector{1}) \cdot \unitvector{2}) \ast (((\amplitudematrix \matrixdotproduct \shiftmatrix{\degreevector}{\impulsevariable}  )\cdot \unitvector{2}) \cdot \unitvector{2})
\end{pmatrix}
\\
&
=
\begin{pmatrix}
	(\impulsespectraldensityeinsteinmono{\amplitude}{\impulseindex}{\impulseshift}{\impulsevariable})_{1,1}  \ast (\impulsespectraldensityeinsteinmono{\amplitude}{\impulseindex}{\impulseshift}{\impulsevariable})_{1,2}  
	\\
	(\impulsespectraldensityeinsteinmono{\amplitude}{\impulseindex}{\impulseshift}{\impulsevariable})_{2,1}  \ast (\impulsespectraldensityeinsteinmono{\amplitude}{\impulseindex}{\impulseshift}{\impulsevariable})_{2,2}  
\end{pmatrix} 
\\
&
=
\begin{pmatrix}
	(2 \diracdelta(\impulsevariable) + 4  \diracdelta(\impulsevariable - 5) + 5 \diracdelta(\impulsevariable - 10))
	\ast
	(3 \diracdelta(\impulsevariable) + \diracdelta(\impulsevariable - 3) + 8 \diracdelta(\impulsevariable - 6))
	\\
	(6 \diracdelta(\impulsevariable) + 2  \diracdelta(\impulsevariable - 7) + 3 \diracdelta(\impulsevariable - 14) )
	\ast
	(9 \diracdelta(\impulsevariable) + 4  \diracdelta(\impulsevariable - 4) + 2 \diracdelta(\impulsevariable - 8))
\end{pmatrix}
\\
&
=
\begin{pmatrix}
	\begin{smallmatrix}
		& 6 \diracdelta(\impulsevariable) &+ 12 \diracdelta(\impulsevariable - 5) &+ 15 \diracdelta(\impulsevariable - 10) \\
		+
		& 2 \diracdelta(\impulsevariable - 3) &+  4 \diracdelta(\impulsevariable - 8) &+  5 \diracdelta(\impulsevariable - 13) \\
		+
		& 16 \diracdelta(\impulsevariable - 6) &+  32 \diracdelta(\impulsevariable - 11) &+  40 \diracdelta(\impulsevariable - 16)
	\end{smallmatrix}
	\\
	\\
	\begin{smallmatrix}
		& 54 \diracdelta(\impulsevariable) & + 18 \diracdelta(\impulsevariable - 7) &+ 27 \diracdelta(\impulsevariable - 14) \\
		+&
		24 \diracdelta(\impulsevariable - 4) &+ 8 \diracdelta(\impulsevariable - 11) &+ 12 \diracdelta(\impulsevariable - 18) \\
		+&
		12 \diracdelta(\impulsevariable - 8) &+ 4 \diracdelta(\impulsevariable - 15) &+ 6 \diracdelta(\impulsevariable - 22)
	\end{smallmatrix}
\end{pmatrix}
\\
&
=
\begin{pmatrix}
	\sum_{\impulseindex_{1,1} = 1}^{3} \sum_{\impulseindex_{1,2} = 1}^{3} \amplitude^{\impulseindex_{1,1}} \amplitude^{\impulseindex_{1,2}} \diracdelta(\impulsevariable - (\impulseindex_{1,1}  - 1) \phasevectorcomponent{1,1} - (\impulseindex_{1,2}  - 1) \phasevectorcomponent{1,2})
	\\
	\sum_{\impulseindex_{2,1} = 1}^{3} \sum_{\impulseindex_{2,2} = 1}^{3} \amplitude^{\impulseindex_{2,1}} \amplitude^{\impulseindex_{2,2}} \diracdelta(\impulsevariable - (\impulseindex_{2,1}  - 1) \phasevectorcomponent{2,1} - (\impulseindex_{2,2}  - 1) \phasevectorcomponent{2,2})
\end{pmatrix}
\end{alignat}
Lastly, we multiply the \impulseconvolutionvectornamesf{} with the one vector to get the \impulseinterferencename{}:
\begin{alignat}{1}
	\impulsespectralinterferenceeinstein{\amplitudematrix \matrixdotproduct \shiftmatrix{\impulsedegree}{\impulsevariable}}
	&
	=
	\innerconvolution{\amplitudematrix}{\shiftmatrix{\degreevector}{\impulsevariable} } \cdot \mathbbm{1}
	\\
	&
	=
	\begin{pmatrix}
		(\impulsespectraldensityeinsteinmono{\amplitude}{\impulseindex}{\impulseshift}{\impulsevariable})_{1,1}  \ast (\impulsespectraldensityeinsteinmono{\amplitude}{\impulseindex}{\impulseshift}{\impulsevariable})_{1,2} 
		\\
		(\impulsespectraldensityeinsteinmono{\amplitude}{\impulseindex}{\impulseshift}{\impulsevariable})_{2,1}  \ast (\impulsespectraldensityeinsteinmono{\amplitude}{\impulseindex}{\impulseshift}{\impulsevariable})_{2,2}  
	\end{pmatrix} 
	\cdot 
	\begin{pmatrix}
		1 \\ 1
	\end{pmatrix}
\\
&
=
 6   \, \diracdelta(\impulsevariable) + 12   \, \diracdelta(\impulsevariable - 5) + 15  \,  \diracdelta(\impulsevariable - 10) 
  \nonumber
 \\
 &
+
 2  \,  \diracdelta(\impulsevariable - 3) +  4  \,  \diracdelta(\impulsevariable - 8) +  5  \,  \diracdelta(\impulsevariable - 13) 
 \nonumber
 \\
 &
+
 16   \, \diracdelta(\impulsevariable - 6) +  32  \,  \diracdelta(\impulsevariable - 11) +  40   \,  \diracdelta(\impulsevariable - 16)
  \nonumber
 \\
 &
 +
  54   \, \diracdelta(\impulsevariable) + 18   \, \diracdelta(\impulsevariable - 7) + 27  \,  \diracdelta(\impulsevariable - 14) 
  \nonumber
 \\
 &
 +
 24  \,  \diracdelta(\impulsevariable - 4) + 8   \, \diracdelta(\impulsevariable - 11) + 12   \, \diracdelta(\impulsevariable - 18) 
  \nonumber
 \\
 &
 +
 12   \, \diracdelta(\impulsevariable - 8) + 4  \,  \diracdelta(\impulsevariable - 15) + 6   \, \diracdelta(\impulsevariable - 22)
	\\
	&
	= 
		\sum_{\impulseindex_{1,1} = 1}^{3} \sum_{\impulseindex_{1,2} = 1}^{3} \amplitude^{\impulseindex_{1,1}} \amplitude^{\impulseindex_{1,2}} \diracdelta(\impulsevariable - (\impulseindex_{1,1}  - 1) \phasevectorcomponent{1,1} - (\impulseindex_{1,2}  - 1) \phasevectorcomponent{1,2})
		\nonumber
	\\
	&
	+
		\sum_{\impulseindex_{2,1} = 1}^{3} \sum_{\impulseindex_{2,2} = 1}^{3} \amplitude^{\impulseindex_{2,1}} \amplitude^{\impulseindex_{2,2}} \diracdelta(\impulsevariable - (\impulseindex_{2,1}  - 1) \phasevectorcomponent{2,1} - (\impulseindex_{2,2}  - 1) \phasevectorcomponent{2,2})
	\\
	&
	=
	\sum_{\impulseinterferenceindexone = 1}^{2}
	\sum_{\impulseindex_{\impulseinterferenceindexone,1} = 1}^{3} \sum_{\impulseindex_{\impulseinterferenceindexone,2} = 1}^{3} \amplitude^{\impulseindex_{\impulseinterferenceindexone,1}} \amplitude^{\impulseindex_{\impulseinterferenceindexone,2}} \diracdelta(\impulsevariable - (\impulseindex_{\impulseinterferenceindexone,1}  - 1) \phasevectorcomponent{\impulseinterferenceindexone,1} - (\impulseindex_{\impulseinterferenceindexone,2}  - 1) \phasevectorcomponent{\impulseinterferenceindexone,2})
	\\
	&
	=
	\impulseinterference{\impulseshift}{\impulsedegree}{\impulsevariable} 
\end{alignat}
We can also directly compute the \impulseinterferencename{} as follows:

\begin{alignat}{1}
	\impulsespectralinterferenceeinstein{\amplitudematrix \matrixdotproduct \shiftmatrix{\impulsedegree}{\impulsevariable}}
	&
	=
	\innerconvolution{\amplitudematrix}{\shiftmatrix{\degreevector}{\impulsevariable} } \cdot \mathbbm{1}
	\\
	&
	=
	\begin{pmatrix}
		(((\amplitudematrix \matrixdotproduct \shiftmatrix{\degreevector}{\impulsevariable}  )\cdot \unitvector{1}) \cdot \unitvector{1}) \ast (((\amplitudematrix \matrixdotproduct \shiftmatrix{\degreevector}{\impulsevariable} ) \cdot \unitvector{1}) \cdot \unitvector{2} )\\
		(((\amplitudematrix \matrixdotproduct \shiftmatrix{\degreevector}{\impulsevariable}  )\cdot \unitvector{2}) \cdot \unitvector{1}) \ast (((\amplitudematrix \matrixdotproduct \shiftmatrix{\degreevector}{\impulsevariable}  )\cdot \unitvector{2}) \cdot \unitvector{2})
	\end{pmatrix}
	\cdot 
	\begin{pmatrix}
		1 \\ 1
	\end{pmatrix}
	\\
	&
	=
	(((\amplitudematrix \matrixdotproduct \shiftmatrix{\degreevector}{\impulsevariable}  )\cdot \unitvector{1}) \cdot \unitvector{1}) \ast (((\amplitudematrix \matrixdotproduct \shiftmatrix{\degreevector}{\impulsevariable} ) \cdot \unitvector{1}) \cdot \unitvector{2} )
	\nonumber
	\\
	&
	+
	(((\amplitudematrix \matrixdotproduct \shiftmatrix{\degreevector}{\impulsevariable}  )\cdot \unitvector{2}) \cdot \unitvector{1}) \ast (((\amplitudematrix \matrixdotproduct \shiftmatrix{\degreevector}{\impulsevariable}  )\cdot \unitvector{2}) \cdot \unitvector{2})
	\\
	&
	=
	\left( \left( 
	\begin{pmatrix}
		\begin{pmatrix}
			2 \\ 4 \\ 5
		\end{pmatrix} 
		\matrixdotproduct
		\begin{pmatrix}
			\diracdelta(\impulsevariable) \\	\diracdelta(\impulsevariable - 5) \\ \diracdelta(\impulsevariable - 10)
		\end{pmatrix}
		& 
		\begin{pmatrix}
			3 \\ 1 \\ 8
		\end{pmatrix}
		\matrixdotproduct
		\begin{pmatrix}
			\diracdelta(\impulsevariable) \\	\diracdelta(\impulsevariable - 3) \\ \diracdelta(\impulsevariable - 6)
		\end{pmatrix}
		\\
		\begin{pmatrix}
			6 \\ 2 \\ 3
		\end{pmatrix}
		\matrixdotproduct
		\begin{pmatrix}
			\diracdelta(\impulsevariable) \\	\diracdelta(\impulsevariable - 7) \\ \diracdelta(\impulsevariable - 14)
		\end{pmatrix}
		&
		\begin{pmatrix}
			9 \\ 4 \\ 2
		\end{pmatrix}
		\matrixdotproduct
		\begin{pmatrix}
			\diracdelta(\impulsevariable) \\	\diracdelta(\impulsevariable - 4) \\	\diracdelta(\impulsevariable - 8)
		\end{pmatrix}
	\end{pmatrix} 
	\cdot 
	\begin{pmatrix}
		1 \\ 0
	\end{pmatrix}
	\right)
	\cdot 
	\begin{pmatrix}
		1 \\ 0
	\end{pmatrix}
	\right)
	\nonumber
	\\
	&
	\ast 
	\left( \left(	
	\begin{pmatrix}
	\begin{pmatrix}
		2 \\ 4 \\ 5
	\end{pmatrix} 
	\matrixdotproduct
	\begin{pmatrix}
		\diracdelta(\impulsevariable) \\	\diracdelta(\impulsevariable - 5) \\ \diracdelta(\impulsevariable - 10)
	\end{pmatrix}
	& 
	\begin{pmatrix}
		3 \\ 1 \\ 8
	\end{pmatrix}
	\matrixdotproduct
	\begin{pmatrix}
		\diracdelta(\impulsevariable) \\	\diracdelta(\impulsevariable - 3) \\ \diracdelta(\impulsevariable - 6)
	\end{pmatrix}
	\\
	\begin{pmatrix}
		6 \\ 2 \\ 3
	\end{pmatrix}
	\matrixdotproduct
	\begin{pmatrix}
		\diracdelta(\impulsevariable) \\	\diracdelta(\impulsevariable - 7) \\ \diracdelta(\impulsevariable - 14)
	\end{pmatrix}
	&
	\begin{pmatrix}
		9 \\ 4 \\ 2
	\end{pmatrix}
	\matrixdotproduct
	\begin{pmatrix}
		\diracdelta(\impulsevariable) \\	\diracdelta(\impulsevariable - 4) \\	\diracdelta(\impulsevariable - 8)
	\end{pmatrix}
\end{pmatrix} 
	\cdot 
	\begin{pmatrix}
		0 \\ 1
	\end{pmatrix}
	\right) 
	\cdot 
	\begin{pmatrix}
		1 \\ 0
	\end{pmatrix}
	\right)
	\nonumber
	\\
	&
	+ \,
	\left(\left( 	
	\begin{pmatrix}
		\begin{pmatrix}
			2 \\ 4 \\ 5
		\end{pmatrix} 
		\matrixdotproduct
		\begin{pmatrix}
			\diracdelta(\impulsevariable) \\	\diracdelta(\impulsevariable - 5) \\ \diracdelta(\impulsevariable - 10)
		\end{pmatrix}
		& 
		\begin{pmatrix}
			3 \\ 1 \\ 8
		\end{pmatrix}
		\matrixdotproduct
		\begin{pmatrix}
			\diracdelta(\impulsevariable) \\	\diracdelta(\impulsevariable - 3) \\ \diracdelta(\impulsevariable - 6)
		\end{pmatrix}
		\\
		\begin{pmatrix}
			6 \\ 2 \\ 3
		\end{pmatrix}
		\matrixdotproduct
		\begin{pmatrix}
			\diracdelta(\impulsevariable) \\	\diracdelta(\impulsevariable - 7) \\ \diracdelta(\impulsevariable - 14)
		\end{pmatrix}
		&
		\begin{pmatrix}
			9 \\ 4 \\ 2
		\end{pmatrix}
		\matrixdotproduct
		\begin{pmatrix}
			\diracdelta(\impulsevariable) \\	\diracdelta(\impulsevariable - 4) \\	\diracdelta(\impulsevariable - 8)
		\end{pmatrix}
	\end{pmatrix} 
	\cdot
	\begin{pmatrix}
		1 \\ 0
	\end{pmatrix}
	\right) 
	\cdot
	\begin{pmatrix}
		0 \\ 1
	\end{pmatrix}
	\right) 
	\nonumber
	\\
	&
	\ast 
	\left(\left( 	
	\begin{pmatrix}
	\begin{pmatrix}
		2 \\ 4 \\ 5
	\end{pmatrix} 
	\matrixdotproduct
	\begin{pmatrix}
		\diracdelta(\impulsevariable) \\	\diracdelta(\impulsevariable - 5) \\ \diracdelta(\impulsevariable - 10)
	\end{pmatrix}
	& 
	\begin{pmatrix}
		3 \\ 1 \\ 8
	\end{pmatrix}
	\matrixdotproduct
	\begin{pmatrix}
		\diracdelta(\impulsevariable) \\	\diracdelta(\impulsevariable - 3) \\ \diracdelta(\impulsevariable - 6)
	\end{pmatrix}
	\\
	\begin{pmatrix}
		6 \\ 2 \\ 3
	\end{pmatrix}
	\matrixdotproduct
	\begin{pmatrix}
		\diracdelta(\impulsevariable) \\	\diracdelta(\impulsevariable - 7) \\ \diracdelta(\impulsevariable - 14)
	\end{pmatrix}
	&
	\begin{pmatrix}
		9 \\ 4 \\ 2
	\end{pmatrix}
	\matrixdotproduct
	\begin{pmatrix}
		\diracdelta(\impulsevariable) \\	\diracdelta(\impulsevariable - 4) \\	\diracdelta(\impulsevariable - 8)
	\end{pmatrix}
	\end{pmatrix} 
	\cdot 
	\begin{pmatrix}
		0 \\ 1
	\end{pmatrix}
	\right) 
	\cdot 
	\begin{pmatrix}
		0 \\ 1
	\end{pmatrix}
	\right)
	\\
	&
	=
	\left(
	\begin{pmatrix}
		2 \\ 4 \\ 5
	\end{pmatrix} 
	\matrixdotproduct
	\begin{pmatrix}
		\diracdelta(\impulsevariable) \\	\diracdelta(\impulsevariable - 5) \\ \diracdelta(\impulsevariable - 10)
	\end{pmatrix}
	\right)
	\ast 
	\left(
	\begin{pmatrix}
		3 \\ 1 \\ 8
	\end{pmatrix}
	\matrixdotproduct
	\begin{pmatrix}
		\diracdelta(\impulsevariable) \\	\diracdelta(\impulsevariable - 3) \\ \diracdelta(\impulsevariable - 6)
	\end{pmatrix}
	\right)
	\nonumber
	\\
	&
	+
	\left(
		\begin{pmatrix}
		6 \\ 2 \\ 3
	\end{pmatrix}
	\matrixdotproduct
	\begin{pmatrix}
		\diracdelta(\impulsevariable) \\	\diracdelta(\impulsevariable - 7) \\ \diracdelta(\impulsevariable - 14)
	\end{pmatrix}
	\right)
	\ast
	\left(
	\begin{pmatrix}
		9 \\ 4 \\ 2
	\end{pmatrix}
	\matrixdotproduct
	\begin{pmatrix}
		\diracdelta(\impulsevariable) \\	\diracdelta(\impulsevariable - 4) \\	\diracdelta(\impulsevariable - 8)
	\end{pmatrix}
	\right)
	\\
	&
	=
	\left(
	\begin{pmatrix}
		2 \impulsespectralspacemult \diracdelta(\impulsevariable)  \\	4 \impulsespectralspacemult \diracdelta(\impulsevariable - 5) \\	5 \impulsespectralspacemult\diracdelta(\impulsevariable - 10)
	\end{pmatrix}
	\cdot 
	\begin{pmatrix}
		1 \\ 1 \\ 1
	\end{pmatrix}
	\right)
	\ast 
	\left(
	\begin{pmatrix}
		3 \impulsespectralspacemult \diracdelta(\impulsevariable)  \\	1 \impulsespectralspacemult \diracdelta(\impulsevariable - 3) \\	8 \impulsespectralspacemult\diracdelta(\impulsevariable - 6)
	\end{pmatrix}
	\cdot 
	\begin{pmatrix}
		1 \\ 1 \\ 1
	\end{pmatrix}
	\right)
	\nonumber
	\\
	&
	+
	\left(
	\begin{pmatrix}
		6 \impulsespectralspacemult \diracdelta(\impulsevariable)  \\	2 \impulsespectralspacemult \diracdelta(\impulsevariable - 7) \\	3 \impulsespectralspacemult\diracdelta(\impulsevariable - 14)
	\end{pmatrix}
	\cdot 
	\begin{pmatrix}
		1 \\ 1 \\ 1
	\end{pmatrix}
	\right)
	\ast
	\left(
	\begin{pmatrix}
		9 \impulsespectralspacemult \diracdelta(\impulsevariable)  \\	4 \impulsespectralspacemult \diracdelta(\impulsevariable - 4) \\	2 \impulsespectralspacemult\diracdelta(\impulsevariable - 8)
	\end{pmatrix}
	\cdot 
	\begin{pmatrix}
		1 \\ 1 \\ 1
	\end{pmatrix}
	\right)
	\\
	&
	=
	\left(
	2 \, \diracdelta(\impulsevariable) + 4   \, \diracdelta(\impulsevariable - 5) + 5  \, \diracdelta(\impulsevariable - 10)
	\right)
	\ast 
	\left(
	3  \, \diracdelta(\impulsevariable) + \diracdelta(\impulsevariable - 3) + 8 \,  \diracdelta(\impulsevariable - 6)
	\right)
	\nonumber
	\\
	&
	+
	\left(
	6 \,  \diracdelta(\impulsevariable) + 2  \,  \diracdelta(\impulsevariable - 7) + 3 \,  \diracdelta(\impulsevariable - 14) 
	\right)
	\ast
	\left(
	9  \, \diracdelta(\impulsevariable) + 4  \,  \diracdelta(\impulsevariable - 4) + 2  \, \diracdelta(\impulsevariable - 8)
	\right)
	\nonumber
	\\
	&
	=
	6   \, \diracdelta(\impulsevariable) + 12  \,  \diracdelta(\impulsevariable - 5) + 15  \,  \diracdelta(\impulsevariable - 10) 
	\nonumber
	\\
	&
	+
	2  \,  \diracdelta(\impulsevariable - 3) +  4   \, \diracdelta(\impulsevariable - 8) +  5  \,  \diracdelta(\impulsevariable - 13) 
	\nonumber
	\\
	&
	+
	16  \,  \diracdelta(\impulsevariable - 6) +  32  \,  \diracdelta(\impulsevariable - 11) +  40   \,  \diracdelta(\impulsevariable - 16)
	\nonumber
	\\
	&
	+
	54  \,  \diracdelta(\impulsevariable) + 18  \,  \diracdelta(\impulsevariable - 7) + 27  \,  \diracdelta(\impulsevariable - 14) 
	\nonumber
	\\
	&
	+
	24  \,  \diracdelta(\impulsevariable - 4) + 8  \,  \diracdelta(\impulsevariable - 11) + 12   \, \diracdelta(\impulsevariable - 18) 
	\nonumber
	\\
	&
	+
	12  \,  \diracdelta(\impulsevariable - 8) + 4  \,  \diracdelta(\impulsevariable - 15) + 6   \, \diracdelta(\impulsevariable - 22)
	\\
	&
	= 
	\sum_{\impulseindex_{1,1} = 1}^{3} \sum_{\impulseindex_{1,2} = 1}^{3} \amplitude^{\impulseindex_{1,1}} \amplitude^{\impulseindex_{1,2}} \diracdelta(\impulsevariable - (\impulseindex_{1,1}  - 1) \phasevectorcomponent{1,1} - (\impulseindex_{1,2}  - 1) \phasevectorcomponent{1,2})
	\nonumber
	\\
	&
	+
	\sum_{\impulseindex_{2,1} = 1}^{3} \sum_{\impulseindex_{2,2} = 1}^{3} \amplitude^{\impulseindex_{2,1}} \amplitude^{\impulseindex_{2,2}} \diracdelta(\impulsevariable - (\impulseindex_{2,1}  - 1) \phasevectorcomponent{2,1} - (\impulseindex_{2,2}  - 1) \phasevectorcomponent{2,2})
	\\
	&
	=
	\sum_{\impulseinterferenceindexone = 1}^{2}
	\sum_{\impulseindex_{\impulseinterferenceindexone,1} = 1}^{3} \sum_{\impulseindex_{\impulseinterferenceindexone,2} = 1}^{3} \amplitude^{\impulseindex_{\impulseinterferenceindexone,1}} \amplitude^{\impulseindex_{\impulseinterferenceindexone,2}} \diracdelta(\impulsevariable - (\impulseindex_{\impulseinterferenceindexone,1}  - 1) \phasevectorcomponent{\impulseinterferenceindexone,1} - (\impulseindex_{\impulseinterferenceindexone,2}  - 1) \phasevectorcomponent{\impulseinterferenceindexone,2})
	\\
	&
	=
	\impulseinterference{\impulseshift}{\impulsedegree}{\impulsevariable} 
\end{alignat}
\section{Impulse bounds}\label{section:operations}
%% Figure environment removed
The calculus of the impulse algebra originates from the unified event bound (UEB) model of \cite{slomka2021beyond}. 
We show how a Heaviside mask can be sampled by a Dirac impulse based on which we can discretize the analysis functions of the UEB model for fast computation.
\begin{definition}[Heaviside mask \cite{slomka2021beyond}]
	Let $a,b,\impulsevariable \in \realnumbers$. The \textbf{Heaviside mask} is defined by $\heavisidemaskone{\impulsevariable}{a}{b} = \heaviside(\impulsevariable - a) \cdot (b - \impulsevariable)$. In distributional short-form notation, we write $\heavisidemaskonedistributional{a}{b} \coloneqq \heavisidemaskone{\impulsevariable}{a}{b}$.
\end{definition}
\subsection{Unified bound discretization}
The UEB function of \cite{slomka2021beyond} applied to a \impulseinterferencename{}
\begin{alignat}{1}
	& \int_{-\infty}^{\infty} \impulseinterference{\impulseshift}{\impulsedegree}{\impulsevariable} \cdot \heaviside(b - \impulsevariable) \cdot \heaviside(\impulsevariable - a) \, d\impulsevariable
	\\
	=
	&
	\int_{-\infty}^{\infty} 
	\sum_{\impulseinterferenceindexone = 1}^{\impulseinterferencedegreeone}
	\sum_{\impulseindex_{\impulseinterferenceindexone,1} = 1}^{\degreevectorcomponent{\impulseinterferenceindexone,1}} \ldots \sum_{\impulseindex_{\impulseinterferenceindexone,\multiperiodicimpulsedimension} = 1}^{\degreevectorcomponent{\impulseinterferenceindexone,\multiperiodicimpulsedimension}}
	\amplitudevector{\impulseindex_{\impulseinterferenceindexone,1},\dots,\impulseindex_{\impulseinterferenceindexone,\multiperiodicimpulsedimension}} 
	\cdot \diracdelta(\impulsevariable - (\impulseindex_{\impulseinterferenceindexone,1} - 1) \phasevectorcomponent{\impulseinterferenceindexone,1}  - \ldots - (\impulseindex_{\impulseinterferenceindexone,\multiperiodicimpulsedimension} - 1) \phasevectorcomponent{\impulseinterferenceindexone,\multiperiodicimpulsedimension}) 
	\nonumber
	\\
	& ~~~~~~~~~~~~~~~~~~~~~~~~~~~~~~~~~~~~~~~~~~~~~~~~
	\cdot   \heaviside(\impulsevariable - a) \cdot \heaviside(b - \impulsevariable) \, d\impulsevariable \label{eq:uebtomist}
\end{alignat}
counts the number of impulses of the \impulseinterferencenameshort{} in the given Heaviside mask $\heavisidemaskonedistributional{a}{b}$.
We observe in Equation \ref{eq:uebtomist} that the Heaviside mask is multiplied by a Dirac impulse within an integral.  
From Lemma \ref{lemma:translationdirac}, we know if a test function is multiplied by a Dirac impulse, then the test function is sampled at the shift of the impulse.
Considering the Heaviside function, we know that it does not have a compact support since its value is 1 for $x > 0$, so that it does not fulfill the properties of a test function.
However, the Dirac impulse has the compact support $\{0\}$, which allows us to apply the Dirac impulse on any infinitely continuous differentiable function \cite{strichartz2003guide} (page 81).
Therefore, if we approximate the Heaviside mask by an infinitely continuous differentiable function, we can apply the sampling property of the Dirac impulse to the Heaviside mask in the integral of Equation \eqref{eq:uebtomist}. 
In other words, the computation of the UEB function is reduced from the evaluation of an integral to the evaluation of a single Heaviside mask. 
To this end, we apply the sequence $(h_{k})$ of functions from Equation \eqref{eq:heavisideapproxproperties} and \eqref{eq:heavisideapprox} that approximates the Heaviside function for $k \to 0$.
% with the properties
%\begin{alignat}{1}\label{eq:heavisideapproxproperties}
%	h_{k}(x) &= 1,~~~ x > k \\
%	h_{k}(x) &= 0, ~~~ x < -k
%\end{alignat}
%and $h_{k}(x) \in C^{\infty}$ for all $k \in \naturalnumbers$. An exemplary approximation is the family of functions
%\begin{equation}\label{eq:heavisideapprox}
%	h_{k}(x) = \left\{\begin{array}{lll}
%		\frac{1}{1 + \operatorname{e}^{\frac{4kx}{x^{2}-k^{2}}}}   &,|x| &< k 
%		\\
%		0  &,x &< - k 
%		\\
%		1  &,x &> k
%	\end{array}
%	\right.
%\end{equation}
%that is depicted in Figure \ref{fig:heavisideapprox}.
Since $h_{k}(x) \in C^{\infty}$, it follows that
% the product of $h_{k}(x)$ and a test function is a test function by theorem 2.6 of \cite{grubb2008distributions} 
$h_{k}(x)$ can  be sampled by the Dirac impulse 
based on which the translation (sampling) of a Heaviside function by a Dirac impulse can be defined.
\begin{lemma}[Translation of Heaviside product]\label{lemma:translationheaviside}
	Let $a,b, \epsilon \in \realnumbers$. 
	Let $(f_{n})$ be a sequence of test functions approximating the Dirac impulse with the properties $	\int_{-1/n}^{1/n}f_n(x) \, dx = 1 $ and $	f_n(x) = 0, x \geq |1/n|$.
	Let $(h_{k})$ be a sequence of $C^{\infty}$-functions that approximate the heaviside function, i.e. $\heaviside(x) = \lim_{k \to 0} h_{k}(x)$ and the $h_{k}(x)$ have the properties described in Equation \eqref{eq:heavisideapproxproperties}. 
	Then, the Heaviside mask is translated by the Dirac impulse as follows:
	\begin{equation}
		\langle T_{-\epsilon} \diracdelta,  \heavisidemaskonedistributional{a}{b} \rangle = \heavisidemaskone{\epsilon}{a}{b}
	\end{equation}
	which means that
	\begin{equation}
		\int_{-\infty}^{\infty} \diracdelta(\impulsevariable - \epsilon) \cdot \heavisidemaskone{\impulsevariable}{a}{b} \, dx = \heavisidemaskone{\epsilon}{a}{b}
	\end{equation}
\end{lemma}
\begin{proof}
	\begin{alignat}{1}
		\langle T_{-\epsilon} \diracdelta,  \heavisidemaskonedistributional{a}{b} \rangle
		&
		=
		\lim\limits_{\substack{n \, \to \infty \\ k \to 0}}
		\int_{-\infty}^{\infty} 
		\underbrace{f_{n}(x - \epsilon)}_{u^\prime} \cdot  \underbrace{ h_{k}(x-a) \cdot h_{k}(b-x) }_{v} 
		\, dx 
		\\
		&
		=
		\lim\limits_{\substack{n \, \to \infty \\ k \to 0}}
		[ F_{n}(x- \epsilon) \cdot h_{k}(x-a) \cdot h_{k}(b-x) ]_{- \infty}^{\infty}
		\nonumber
		\\
		&
		-
		\int_{-\infty}^{\infty} 
		F_{n}(x - \epsilon) \cdot (h_{k}(x-a) \cdot h_{k}(b-x) )'
		\, dx 
		\label{eq:heavisidemaskintegral1}
		\\
		&
		=
		\lim\limits_{\substack{n \, \to \infty \\ k \to 0}}
		1 \cdot 0 \cdot 1 - 0 \cdot 1 \cdot 0 
		-
		\int_{-\infty}^{\infty} 
		F_{n}(x - \epsilon) \, ( h_{k}(x-a) \cdot h_{k}(b-x) )'
		\, dx 
		\\
		&
		=
		\lim\limits_{\substack{n \, \to \infty \\ k \to 0}}
		-
		\left(
		\int_{-\infty}^{- 1/n + \epsilon} 
		\underbrace{F_{n}(x - \epsilon)}_{ = \, 0} \, ( h_{k}(x-a) \, h_{k}(b-x) )'
		\, dx \right.
		\\
		&
		\left.
		+
			\int_{- 1/n + \epsilon}^{ 1/n + \epsilon} 
		F_{n}(x - \epsilon)\, ( h_{k}(x-a) \, h_{k}(b-x) )'
		\, dx 
		\right.
		\\
		&
		\left.
		+
		\int_{1/n + \epsilon}^{ \infty} 
		\underbrace{F_{n}(x - \epsilon)}_{ = \, 1} \, ( h_{k}(x-a) \, h_{k}(b-x) )'
		\, dx 
		\right)
		\\
		&
		=
		\lim\limits_{\substack{n \, \to \infty \\ k \to 0}}
		-
		\left(
			\int_{- 1/n + \epsilon}^{ 1/n + \epsilon} 
		F_{n}(x - \epsilon)\, (h_{k}(x-a) \, h_{k}(b-x)  )'
		\, dx 
		\right.
		\\
		& ~~~~~~~~~~~~
		\left.
		+
		\int_{1/n + \epsilon}^{ \infty} 
		 ( h_{k}(x-a) \, h_{k}(b-x) )'
		\, dx 
		\right)
		\\
		&
		=
		\lim\limits_{k \to 0}
		-
		\int_{\epsilon}^{ \infty} 
		( h_{k}(x-a) \, h_{k}(b-x) )'
		\, dx 
		\\
		&
		=
		\lim\limits_{k \to 0}
		-
		[ h_{k}(x-a) \, h_{k}(b-x) ]_{\epsilon}^{\infty}
		\label{eq:heavisidemaskintegral2}
		\\
		&
		=
		\lim\limits_{k \to 0}
		- ( h_{k}(\infty-a) \, \underbrace{h_{k}(b-\infty)}_{= \, 0} -  h_{k}(\epsilon-a) \, h_{k}(b-\epsilon) )
		\\
		&
		=
		\lim\limits_{k \to 0}
		- ( - \, h_{k}(\epsilon-a) \, h_{k}(b-\epsilon)  )
		\\
		&
		=
		\heaviside(\epsilon - a) \, \heaviside(b - \epsilon) 
	\end{alignat}
%	\begin{alignat}{1}
%	\langle \heavisidemaskonedistributional{a}{b} T_{\epsilon} \diracdelta, \phi \rangle 
%		&
%		=
%		\lim_{k \to \infty}
%		\int_{-\infty}^{\infty} h_{k}(x-a) \cdot h_{k}(b-x) \cdot \delta(x- \epsilon) \cdot \phi(x) \, dx 
%		\\
%		&
%		=
%		\lim_{k \to \infty}
%		\int_{-\infty}^{\infty}  \delta(x-\epsilon) \cdot \underbrace{h_{k}(x-a) \cdot h_{k}(b-x) \cdot \phi(x)}_{\in \mathcal{D}} \, dx 
%		\\
%		&
%		\oset[1.5ex]{\mathclap{\mathrm{Le.} \ref{lemma:translationdirac}}}{=}
%		\lim_{k \to \infty} h_{k}(\epsilon - a) \cdot h_{k}(b - \epsilon) \cdot \phi(\epsilon)
%		\\
%		&
%		=
%		\heaviside(\epsilon - a) \cdot \heaviside(b - \epsilon) \cdot  \phi(\epsilon)
%		\\
%		&
%		= \langle \diracdelta, T_{\epsilon} \heavisidemaskonedistributional{a}{b} \phi \rangle
%	\end{alignat}
%	We note that $h_{k}(x-a) \cdot h_{k}(b-x) \cdot \phi(x) \in \mathcal{D}$ follows from theorem 2.6 of \cite{grubb2008distributions} stating that the product of a $C^{\infty}$-function and a test function is a test function.
\end{proof}
	Dependent on the definition of the Heaviside approximation $h_{k}(x)$, we can replace $\heaviside(\epsilon - a) \cdot \heaviside(b - \epsilon)$ with upper and lower Heaviside functions and hence obtain the Heaviside masks for all combinations of open and closed interval boundaries as this replacement does not change the result of the integral in Equation \ref{eq:heavisidemaskintegral1} and \ref{eq:heavisidemaskintegral2}.
	
	We apply a Heaviside mask to any \impulseinterferencename{} to mask out the amplitudes of impulses that shall not be considered in further analysis. This masking procedure can be computed by an integral as known from the UEB model but there exists also a discrete computation based on the sifting property of the Dirac impulse. We call this discretized version the \heavisidedurationname{} for which we introduce a short-form notation.
	\begin{definition}[\heavisidedurationnamecapital]\label{def:heavisideduration}
		The \textbf{
			\heavisidedurationname} is defined as the discretized unified request bound function of \cite{slomka2021beyond} as follows:
		\begin{equation}
			\heavisideduration{\Sh}{a,b}{} \coloneqq 	
			\amplitudevector{\impulseindex_{\impulseinterferenceindexone,1} \dots \impulseindex_{\impulseinterferenceindexone,\multiperiodicimpulsedimension}} \overline{| \langle \phasevectorcomponent{\impulseindex_{\impulseinterferenceindexone,1}} \dots \phasevectorcomponent{\impulseindex_{\impulseinterferenceindexone,\multiperiodicimpulsedimension}} \rangle  |}_{a,b}
		\end{equation}
		where
		\begin{alignat}{1}
			&
			\amplitudevector{\impulseindex_{\impulseinterferenceindexone,1} \dots \impulseindex_{\impulseinterferenceindexone,\multiperiodicimpulsedimension}} \overline{| \langle \phasevectorcomponent{\impulseindex_{\impulseinterferenceindexone,1}} \dots \phasevectorcomponent{\impulseindex_{\impulseinterferenceindexone,\multiperiodicimpulsedimension}} \rangle  |}_{a,b}
			\\
			\coloneqq 
			&
			\sum_{\impulseinterferenceindexone = 1}^{\impulseinterferencedegreeone}
			\sum_{\impulseindex_{\impulseinterferenceindexone,1} = 1}^{\degreevectorcomponent{\impulseinterferenceindexone,1}} \dots \sum_{\impulseindex_{\impulseinterferenceindexone,\multiperiodicimpulsedimension} = 1}^{\degreevectorcomponent{\impulseinterferenceindexone,\multiperiodicimpulsedimension}}
			\amplitudevector{\impulseindex_{\impulseinterferenceindexone,1},\dots,\impulseindex_{\impulseinterferenceindexone,\multiperiodicimpulsedimension}} 
			\cdot \heaviside(b - (\impulseindex_{\impulseinterferenceindexone,1} - 1) \phasevectorcomponent{\impulseinterferenceindexone,1}  - \dots - (\impulseindex_{\impulseinterferenceindexone,\multiperiodicimpulsedimension} - 1) \phasevectorcomponent{\impulseinterferenceindexone,\multiperiodicimpulsedimension}) 
			\nonumber
			\\
			&
			\cdot \heaviside((\impulseindex_{\impulseinterferenceindexone,1} - 1) \phasevectorcomponent{\impulseinterferenceindexone,1}  + \dots + (\impulseindex_{\impulseinterferenceindexone,\multiperiodicimpulsedimension} - 1) \phasevectorcomponent{\impulseinterferenceindexone,\multiperiodicimpulsedimension} - a) 		
		\end{alignat}
	\end{definition}
	Based on Lemma \ref{lemma:translationheaviside}, we derive in the following the discretized UEB function which is equal to a corresponding \heavisidedurationname{}.
	
\begin{theorem}[Unified event bound function discretization]\label{theorem:discretization}
	Let $\impulseinterference{\impulseshift}{\impulsedegree}{\impulsevariable}$ be a \impulseinterferencename{}.
	The discretized unified event bound function given by theorem 1 in \cite{slomka2021beyond} is equal to a \heavisidedurationname{}:
	\begin{alignat}{1}
		\int_{-\infty}^{\infty} \impulseinterference{\impulseshift}{\impulsedegree}{\impulsevariable} \cdot \heaviside(b - x) \cdot \heaviside(x - a) \, dx
		=
		\heavisideduration{\Sh}{a,b}{} 
	\end{alignat}
\end{theorem}
\begin{proof}
	\begin{alignat}{1}
		&
		\int_{-\infty}^{\infty} \impulseinterference{\impulseshift}{\impulsedegree}{\impulsevariable} \cdot \heaviside(b - \impulsevariable) \cdot \heaviside(\impulsevariable - a) \, d\impulsevariable
		\\
		=
		&
		\int_{-\infty}^{\infty} 
		\sum_{\impulseinterferenceindexone = 1}^{\impulseinterferencedegreeone}
		\sum_{\impulseindex_{\impulseinterferenceindexone,1} = 1}^{\degreevectorcomponent{\impulseinterferenceindexone,1}} \ldots \sum_{\impulseindex_{\impulseinterferenceindexone,\multiperiodicimpulsedimension} = 1}^{\degreevectorcomponent{\impulseinterferenceindexone,\multiperiodicimpulsedimension}}
		\amplitudevector{\impulseindex_{\impulseinterferenceindexone,1},\dots,\impulseindex_{\impulseinterferenceindexone,\multiperiodicimpulsedimension}} 
		\cdot \diracdelta(\impulsevariable - (\impulseindex_{\impulseinterferenceindexone,1} - 1) \phasevectorcomponent{\impulseinterferenceindexone,1}  - \ldots - (\impulseindex_{\impulseinterferenceindexone,\multiperiodicimpulsedimension} - 1) \phasevectorcomponent{\impulseinterferenceindexone,\multiperiodicimpulsedimension}) 
		\nonumber
		\\
		& ~~~~~~~~~~~~~~~~~~~~~~~~~~~~~~~~~~~~~~~~~~~~~~~~
		 \cdot \heaviside(b - \impulsevariable) \cdot \heaviside(\impulsevariable - a) \, d\impulsevariable
		\label{eq:discretization1}
		\\
		=
		&
		\sum_{\impulseinterferenceindexone = 1}^{\impulseinterferencedegreeone}
		\sum_{\impulseindex_{\impulseinterferenceindexone,1} = 1}^{\degreevectorcomponent{\impulseinterferenceindexone,1}} \ldots \sum_{\impulseindex_{\impulseinterferenceindexone,\multiperiodicimpulsedimension} = 1}^{\degreevectorcomponent{\impulseinterferenceindexone,\multiperiodicimpulsedimension}}
				\int_{-\infty}^{\infty} 
	\amplitudevector{\impulseindex_{\impulseinterferenceindexone,1},\dots,\impulseindex_{\impulseinterferenceindexone,\multiperiodicimpulsedimension}} 
		\cdot \diracdelta(\impulsevariable - (\impulseindex_{\impulseinterferenceindexone,1} - 1) \phasevectorcomponent{\impulseinterferenceindexone,1}  - \ldots - (\impulseindex_{\impulseinterferenceindexone,\multiperiodicimpulsedimension} - 1) \phasevectorcomponent{\impulseinterferenceindexone,\multiperiodicimpulsedimension}) 
		\nonumber
		\\
		& ~~~~~~~~~~~~~~~~~~~~~~~~~~~~~~~~~~~~~~~~~~~~~~~~
		\cdot \heaviside(b - \impulsevariable) \cdot \heaviside(\impulsevariable - a) \, d\impulsevariable
		\label{eq:discretization2}
		\\
		=
		&
		\sum_{\impulseinterferenceindexone = 1}^{\impulseinterferencedegreeone}
		\sum_{\impulseindex_{\impulseinterferenceindexone,1} = 1}^{\degreevectorcomponent{\impulseinterferenceindexone,1}} \ldots \sum_{\impulseindex_{\impulseinterferenceindexone,\multiperiodicimpulsedimension} = 1}^{\degreevectorcomponent{\impulseinterferenceindexone,\multiperiodicimpulsedimension}}
		\amplitudevector{\impulseindex_{\impulseinterferenceindexone,1},\dots,\impulseindex_{\impulseinterferenceindexone,\multiperiodicimpulsedimension}} 
		\cdot \heaviside(b - (\impulseindex_{\impulseinterferenceindexone,1} - 1) \phasevectorcomponent{\impulseinterferenceindexone,1}  - \ldots - (\impulseindex_{\impulseinterferenceindexone,\multiperiodicimpulsedimension} - 1) \phasevectorcomponent{\impulseinterferenceindexone,\multiperiodicimpulsedimension}) 
		\nonumber
		\\
		&		~~\,~~~~~~~~~~~~~~~~~~~~~~~~~~~~~~~~~~~~~~
		\cdot \heaviside((\impulseindex_{\impulseinterferenceindexone,1} - 1) \phasevectorcomponent{\impulseinterferenceindexone,1}  + \ldots + (\impulseindex_{\impulseinterferenceindexone,\multiperiodicimpulsedimension} - 1) \phasevectorcomponent{\impulseinterferenceindexone,\multiperiodicimpulsedimension} - a) 		
		\label{eq:discretization3}
		\\
		=
		&~
		\amplitudevector{\impulseindex_{\impulseinterferenceindexone,1} \ldots \impulseindex_{\impulseinterferenceindexone,\multiperiodicimpulsedimension}} \overline{| \langle \phasevectorcomponent{\impulseindex_{\impulseinterferenceindexone,1}} \ldots \phasevectorcomponent{\impulseindex_{\impulseinterferenceindexone,\multiperiodicimpulsedimension}} \rangle  |}_{a,b}
		\\
		=
		&
		\heavisideduration{\Sh}{a,b}{} 
	\end{alignat}
\end{proof}
From Equation \eqref{eq:discretization1} to \eqref{eq:discretization2} we apply Fubini's theorem \cite{tao2009analysis} to switch the summation and the integral as the integral is bounded from above which is discussed in details in lemma 1 of \cite{slomka2021beyond}. 
From Equation \eqref{eq:discretization2} to \eqref{eq:discretization3} we apply the sampling property of the Dirac impulse to the Heaviside mask which is shown in Lemma \ref{lemma:translationheaviside}.
We note that discretization means that the integral over the variable $\impulsevariable$ is substituted by a summation of Heaviside masks as demonstrated in the following.
\begin{example}[Unified bound discretization]\label{example:unifiedbounddiscretization}
	Let $4 \, \diracdelta(\impulsevariable - 1) \impulsegroupoperation 5 \, \diracdelta(\impulsevariable - 2) \impulsegroupoperation  6 \,\diracdelta(\impulsevariable - 3)$ be a Dirac train. 
	Then, the sum of the amplitudes of impulses occuring in the interval $[2,4]$ is computed by the UEB function as follows
	\begin{alignat}{1}
		&
		\int_{-\infty}^{\infty} (4 \, \diracdelta(\impulsevariable - 1) \impulsegroupoperation 5 \, \diracdelta(\impulsevariable - 2) \impulsegroupoperation  6 \,\diracdelta(\impulsevariable - 3))\cdot \heaviside(\impulsevariable - 2) \cdot \heaviside(4 - \impulsevariable) \, d\impulsevariable
		\label{eq:exampleunifiedbounddiscretization1}
		\\
		=
		&
		\int_{-\infty}^{\infty} 4 \, \diracdelta(\impulsevariable - 1) \, \heaviside(\impulsevariable - 2) \, \heaviside(3 - \impulsevariable)  \impulsegroupoperation 5 \, \diracdelta(\impulsevariable - 2) \, \heaviside(\impulsevariable - 2) \, \heaviside(4 - \impulsevariable)
		\nonumber
		\\
		&~~~~
		\impulsegroupoperation  6 \, \diracdelta(\impulsevariable - 3) \,  \heaviside(\impulsevariable - 2) \, \heaviside(4 - \impulsevariable) \, d\impulsevariable
		\label{eq:exampleunifiedbounddiscretization2}
		\\
		=
		&
		\int_{-\infty}^{\infty} 4 \, \diracdelta(\impulsevariable - 1) \, \heaviside(\impulsevariable - 2) \, \heaviside(4 - \impulsevariable) \, d\impulsevariable + \int_{-\infty}^{\infty} 5 \, \diracdelta(\impulsevariable - 2) \, \heaviside(\impulsevariable - 2) \, \heaviside(4 - \impulsevariable) \, d\impulsevariable
		\nonumber
		\\
		+ 
		&
		\int_{-\infty}^{\infty} 6 \, \diracdelta(\impulsevariable - 3) \,  \heaviside(\impulsevariable - 2) \, \heaviside(4 - \impulsevariable) \, d\impulsevariable
		\label{eq:exampleunifiedbounddiscretization3}
		\\
		\oset[1.5ex]{\mathclap{\mathrm{Le.} \, \ref{lemma:translationheaviside}}}{=}
		& ~
		4 \, \heaviside(1 - 2) \, \heaviside(4 - 1) + 5 \, \heaviside(2 - 2) \, \heaviside(4 - 2) + 6 \, \heaviside(3 - 2) \, \heaviside(4 - 3)
		\label{eq:exampleunifiedbounddiscretization4}
		\\
		= 
		&~
		4 \cdot 0 \cdot 1 + 5 \cdot 1 \cdot 1 + 6 \cdot 1 \cdot 1 = 11
	\end{alignat}
	We observe that the integral in Equation \ref{eq:exampleunifiedbounddiscretization1} is solved by distributing the integral to each Dirac impulse in the series in Equation \ref{eq:exampleunifiedbounddiscretization3}. Then, the sampling property of the Dirac impulse is applied to sample the Heaviside mask in Equation \ref{eq:exampleunifiedbounddiscretization4}.
	Note that the discretization of the UEB function to the \heavisidedurationname{} occurs from Equation \ref{eq:exampleunifiedbounddiscretization3} to \ref{eq:exampleunifiedbounddiscretization4} by applying Lemma \ref{lemma:translationheaviside}.
	Furthermore, all the parameters to compute Equation \ref{eq:exampleunifiedbounddiscretization4} are already known: the amplitudes and the shifts of the impulses and the interval $[2,3]$ of the Heaviside mask. Therefore, we can skip the computation steps from Equation \ref{eq:exampleunifiedbounddiscretization1} to \ref{eq:exampleunifiedbounddiscretization3} and directly compute the \heavisidedurationname{} in Equation \ref{eq:exampleunifiedbounddiscretization4}.
\end{example}


\subsection{Maximum and minimum bound on the Heaviside distribution}
To find the maximal or minimal density of impulses, we formulate an extreme value problem on \impulseinterferencename s. 
This allows us to compute worst- and best-cases in a performance analysis of a computing system or a network which can be e.g. the maximum or minimum computational (execution time) or network load (bits). 
By modeling computational or network load as the amplitude of the impulse, we can represent an extreme value problem of computational or network load by an extreme value problem of the \heavisidedurationname{}. Therefore, we assume that the amplitudes of all impulses are positive real numbers.
\begin{problem}[Local extrema of \heavisidedurationname]\label{problem:localextrema}
	Given is an 
	\impulseinterferencename{} $\impulseinterference{\impulseshift}{\impulsedegree}{\impulsevariable}$
	 and a distance $\intervalduration \in \realnumbers$. The problem is to find values for $\impulsevariable^{+},\impulsevariable^{-} \in [\localextremevalueintervalleft,\localextremevalueintervalright - \intervalduration]$ with $\localextremevalueintervalleft,\localextremevalueintervalright \in \realnumbers, \localextremevalueintervalleft \leq \localextremevalueintervalright$, such that
	\begin{alignat}{1}
		\heavisideduration{\Sh}{\impulsevariable^{+},\impulsevariable^{+} + \intervalduration}{} \geq \heavisideduration{\Sh}{\impulsevariable, \impulsevariable + \intervalduration}{} \geq \heavisideduration{\Sh}{\impulsevariable^{-}, \impulsevariable^{-} + \intervalduration}{} ~~~~~~\forall \impulsevariable \in [\localextremevalueintervalleft,\localextremevalueintervalright-\intervalduration]
	\end{alignat}
\end{problem}
To solve Problem \ref{problem:localextrema}, let us consider the extreme value theorem by Bolzano \cite{keisler2013elementary}. It states that a continuous real-valued function attains a maximum and minimum in a closed interval $[\localextremevalueintervalleft,\localextremevalueintervalright - \intervalduration]$.
However, the \heavisidedurationname{} is not a continuous but a discontinuous function, so that we cannot apply Bolzano's theorem to ensure the existence of extreme values of the \heavisidedurationname{}. Hence, we need to show the existence of a maximum and minimum in $[\localextremevalueintervalleft,\localextremevalueintervalright - \intervalduration]$.
%TODO hier die s_m ersetzen druch shiftmatrixentry. s also grenze ersetzen.
\begin{lemma}[Existence of extreme values]\label{lemma:existenceextremevalues}
	Let $\localextremevalueintervalleft,\localextremevalueintervalright \in \realnumbers$ with $\localextremevalueintervalleft \leq \localextremevalueintervalright$, $\intervalduration \in \nonnegativerealnumbers$ and $\impulseinterference{\impulseshift}{\impulsedegree}{\impulsevariable}$ be a \impulseinterferencename{}.
	If 
	\begin{equation}
		\forall \impulseinterferenceindexone \in \naturalnumbersuntil{\impulseinterferencedegreeone}, \impulseinterferenceindextwo \in \naturalnumbersuntil{\impulseinterferencedegreetwo} \colon \degreevectorcomponent{\impulseinterferenceindexone,\impulseinterferenceindextwo} \in \naturalnumbers \setminus \{\infty\} \lor \phasevectorcomponent{\impulseinterferenceindexone,\impulseinterferenceindextwo} \in \realnumbers \setminus \{0\} \label{eq:extremevalueassumption}
	\end{equation}
	then, there exist values $\impulsevariable^{+},\impulsevariable^{-} \in [\localextremevalueintervalleft,\localextremevalueintervalright]$, such that 
	\begin{alignat}{1}
		\heavisideduration{\Sh}{\impulsevariable^{+},\impulsevariable^{+} + \intervalduration}{} \geq \heavisideduration{\Sh}{\impulsevariable, \impulsevariable + \intervalduration}{} \geq \heavisideduration{\Sh}{\impulsevariable^{-}, \impulsevariable^{-} + \intervalduration}{} ~~~~~~\forall \impulsevariable \in [\localextremevalueintervalleft,\localextremevalueintervalright - \intervalduration]
	\end{alignat}
holds.
\end{lemma}
\begin{proof}	
	Assume that 	$\exists \impulseinterferenceindexone \in \naturalnumbersuntil{\impulseinterferencedegreeone}, \impulseinterferenceindextwo \in \naturalnumbersuntil{\impulseinterferencedegreetwo} \colon \degreevectorcomponent{\impulseinterferenceindexone,\impulseinterferenceindextwo} = \infty \land \phasevectorcomponent{\impulseinterferenceindexone,\impulseinterferenceindextwo} = 0$. This means that infinitely many impulses occur at a certain point. The \heavisidedurationname{} then sums up the positive amplitudes of infinitely many impulses which is infinite. Hence, a maximum does not exists.
	
	Now, assume the opposite statement shown in Equation \eqref{eq:extremevalueassumption}.
	If $ \phasevectorcomponent{\impulseinterferenceindexone,\impulseinterferenceindextwo} \in \realnumbers \setminus \{0\}$, then the $(\impulseinterferenceindexone, \impulseinterferenceindextwo)$-th \impulsespectraltrainname{} generates finitely many impulses in the interval $[\localextremevalueintervalleft ,\localextremevalueintervalright]$ as they are separated by a non-zero phase $\phasevectorcomponent{\impulseinterferenceindexone,\impulseinterferenceindextwo}$. 
	If $\degreevectorcomponent{\impulseinterferenceindexone,\impulseinterferenceindextwo} \in \naturalnumbers \setminus \{\infty\} $, then the $(\impulseinterferenceindexone,\impulseinterferenceindextwo)$-th \impulsespectraltrainname{} generates finitely many impulses over time, hence, also finitely many impulses in $[\localextremevalueintervalleft ,\localextremevalueintervalright ]$. 
	As there are $\impulseinterferencedegreeone \cdot \impulseinterferencedegreetwo \in \naturalnumbers$ many \impulsespectraltrainname s, the \impulseinterferencename{} generates finitely many impulses in $[\localextremevalueintervalleft ,\localextremevalueintervalright ]$. 
	Furthermore, the impulses have a finite and positive amplitude. 
	%The sum of the amplitudes of finitely many impulses of some interval $[s ,\localextremevalueintervalright + \intervalduration]$ is finite. Hence,  $\heavisideduration{\Sh}{\impulsevariable, \impulsevariable + \intervalduration}{} $ is finite.
	If $k \in \naturalnumbers$ is the number of impulses in $[\localextremevalueintervalleft ,\localextremevalueintervalright ]$ and $a^{+}$ is the maximum amplitude of the impulses in $[\localextremevalueintervalleft ,\localextremevalueintervalright ]$, then the \heavisidedurationname{} is bounded by 
	$$0 \leq \heavisideduration{\Sh}{\impulsevariable, \impulsevariable + \intervalduration}{} \leq k\cdot a^{+}$$ 
	since the minimum amplitude is 0. %and their may be no impulses counted by the \heavisidedurationname.
	For increasing $\impulsevariable$ from $\localextremevalueintervalleft$ to $\localextremevalueintervalright - \intervalduration$, an impulse occurring at $\impulsevariable_{1} \in [\localextremevalueintervalleft,\localextremevalueintervalright - \intervalduration]$ changes the value of the \heavisidedurationname{} at most twice: When it is included for $ \impulsevariable < \impulsevariable_{1} < \impulsevariable + \intervalduration $ and when it is excluded from the \heavisidedurationname{} for $ \impulsevariable_{1} < \impulsevariable $.
	As the \heavisidedurationname{} is otherwise constant,
	it follows that the cardinality of its image in $[\localextremevalueintervalleft, \localextremevalueintervalright]$ is upper bounded by $2k$.  A finite image of bounded values implies that the  \heavisidedurationname{} $ \heavisideduration{\Sh}{\impulsevariable, \impulsevariable + \intervalduration}{} $ has a maximum and minimum in $[\localextremevalueintervalleft,\localextremevalueintervalright]$.
\end{proof}
After showing the existence of the extreme values, we formalize the maximum and minimum \heavisidedurationname{}.
\begin{definition}[Maximum and minimum Heaviside duration]\label{def:maxminheavisideduration}
	Let $\localextremevalueintervalleft,\localextremevalueintervalright \in \realnumbers$ with $\localextremevalueintervalleft \leq \localextremevalueintervalright$, $\intervalduration \in \nonnegativerealnumbers$ and  $\impulseinterference{\impulseshift}{\impulsedegree}{\impulsevariable}$ be a \impulseinterferencename{}.
	Then,
	\begin{alignat}{1}
		\heavisideduration{\Sh}{\localextremevalueintervalleft, \localextremevalueintervalright,\intervalduration}{+} \coloneqq \max_{\impulsevariable \in [\localextremevalueintervalleft, \localextremevalueintervalright- \intervalduration]}\{ \heavisideduration{\Sh}{\impulsevariable, \impulsevariable + \intervalduration}{} \} 
		=
		\max_{\impulsevariable \in [\localextremevalueintervalleft, \localextremevalueintervalright - \intervalduration]}\{
					\amplitudevector{\impulseindex_{\impulseinterferenceindexone,1} \dots \impulseindex_{\impulseinterferenceindexone,\multiperiodicimpulsedimension}} \overline{| \langle \phasevectorcomponent{\impulseindex_{\impulseinterferenceindexone,1}} \dots \phasevectorcomponent{\impulseindex_{\impulseinterferenceindexone,\multiperiodicimpulsedimension}} \rangle  |}_{\impulsevariable,\impulsevariable + \intervalduration} \}
				\\
		\heavisideduration{\Sh}{\localextremevalueintervalleft,\localextremevalueintervalright,\intervalduration}{-} \coloneqq \min_{\impulsevariable \in [\localextremevalueintervalleft, \localextremevalueintervalright- \intervalduration]}\{ \heavisideduration{\Sh}{\impulsevariable, \impulsevariable + \intervalduration}{} \} 
				=
		\min_{\impulsevariable \in [\localextremevalueintervalleft, \localextremevalueintervalright- \intervalduration]}\{
		\amplitudevector{\impulseindex_{\impulseinterferenceindexone,1} \dots \impulseindex_{\impulseinterferenceindexone,\multiperiodicimpulsedimension}} \overline{| \langle \phasevectorcomponent{\impulseindex_{\impulseinterferenceindexone,1}} \dots \phasevectorcomponent{\impulseindex_{\impulseinterferenceindexone,\multiperiodicimpulsedimension}} \rangle  |}_{\impulsevariable,\impulsevariable + \intervalduration} \}
	\end{alignat}
	are called \textbf{\maximumheavisidedurationname} and \textbf{\minimumheavisidedurationname}.
\end{definition}
As the maximum and minimum \heavisidedurationname{} are local extrema of the \heavisidedurationname{} on intervals of length $\intervalduration$, we call them the \textbf{interval transformation} of the \heavisidedurationname.
\subsection{Maximum and minimum bounds as functions}\label{section:analysisofextremevalues}
The \heavisidedurationname{}  based on the Heaviside distribution abstracts from the concrete definition of its values at jump discontinuities. Indeed, the left- and right-continuous versions of the \heavisidedurationname{} belong to the same distribution \cite{strichartz2003guide}, as discussed in Equation \eqref{eq:heavisideintegralzero} of Section \ref{section:heavisidefunctionanddiracdelta}. 
Therefore, we can work on this abstraction level as long as concrete function values at the jump discontinuities are not considered.
If we want to compute the extrema of the \heavisidedurationname{} in an interval $[\localextremevalueintervalleft,\localextremevalueintervalright]$, and if e.g. a jump occurs at $\localextremevalueintervalleft$ or $\localextremevalueintervalright$, then the left- or right-continuity may decide whether the extremum occurs at the lower or upper point of the jump. The continuity also impacts the analysis to compute the extrema.
To describe these analyses, we present the upper and lower Heaviside function of \cite{slomka2021beyond} that define whether the upper or lower point of the jump is the function value.
\begin{definition}[Upper and lower Heaviside functions]
	The \textbf{upper Heaviside function} and \textbf{lower Heaviside function} are defined by
	\begin{alignat}{1}
		\upperheaviside(\impulsevariable) \coloneqq 
		\left\{ 
		\begin{array}{ll}
			1,& \impulsevariable \geq 0 \\
			0,& \impulsevariable < 0
		\end{array}
		\right.
		\\
		\lowerheaviside(\impulsevariable) \coloneqq 
		\left\{ 
		\begin{array}{ll}
			1,& \impulsevariable > 0 \\
			0,& \impulsevariable \leq 0
		\end{array}
		\right.
	\end{alignat}
\end{definition}
Based on these two functions, we concretize the \heavisidedurationname{} and its maxima and minima in Table \ref{table:listofsymbols}.
The analysis of the extrema is presented. 
\begin{theorem}[Analysis of maximum \heavisidedurationname]\label{theorem:analysisofmaximumheavisideduration} 
	Let $\localextremevalueintervalleft,\localextremevalueintervalright \in \realnumbers$ with $\localextremevalueintervalleft \leq \localextremevalueintervalright$, $\intervalduration \in \nonnegativerealnumbers$ and  $\impulseinterference{\impulseshift}{\impulsedegree}{\impulsevariable}$ be a \impulseinterferencename{}.
	Let $\setofimpulsetimepoint$ be the set of points at which impulses of $\impulseinterference{\impulseshift}{\impulsedegree}{\impulsevariable}$  occur and let $\setofimpulsetimepointminusintervalduration \coloneqq \{ a - \intervalduration \, | \, a \in \setofimpulsetimepoint \}$ be the set of impulse points subtracted by the interval duration $\intervalduration \in \nonnegativerealnumbers$.
	Let
	\begin{alignat}{1}\label{eq:testpointsmax1}
		\testsetleft &=  ([\localextremevalueintervalleft, \localextremevalueintervalright] \cap \setofimpulsetimepoint) \cup \{\localextremevalueintervalright\}
		\\
		\testsetright &= ([\localextremevalueintervalleft + \intervalduration, \localextremevalueintervalright + \intervalduration] \cap \setofimpulsetimepoint) \cup \{\localextremevalueintervalleft + \intervalduration\}\label{eq:testpointsmax2}
	\end{alignat}
	Then, the \maximumheavisidedurationname{} is computed by
	\begin{alignat}{1}
		\heavisidedurationupperupper{\Sh}{\intervalduration}{+} 
		&
		=
		 \max_{\impulsevariable \in \testsetleft}\{ \heavisidedurationupperupper{\Sh}{\impulsevariable, \impulsevariable + \intervalduration}{} \} 
		=
		 \max_{\impulsevariable \in \testsetleft}\{
		\amplitudevector{\impulseindex_{\impulseinterferenceindexone,1} \dots \impulseindex_{\impulseinterferenceindexone,\multiperiodicimpulsedimension}} \overline{| \langle \phasevectorcomponent{\impulseindex_{\impulseinterferenceindexone,1}} \dots \phasevectorcomponent{\impulseindex_{\impulseinterferenceindexone,\multiperiodicimpulsedimension}} \rangle  |}_{\overline{\impulsevariable},\overline{\impulsevariable + \intervalduration}} \}
		\\
		\heavisidedurationupperlower{\Sh}{\intervalduration}{+} 
		&
		=
		\max_{\impulsevariable \in \testsetleft}\{ \heavisidedurationupperlower{\Sh}{\impulsevariable, \impulsevariable + \intervalduration}{} \} 
		=
		\max_{\impulsevariable \in \testsetleft}\{
	\amplitudevector{\impulseindex_{\impulseinterferenceindexone,1} \dots \impulseindex_{\impulseinterferenceindexone,\multiperiodicimpulsedimension}} \overline{| \langle \phasevectorcomponent{\impulseindex_{\impulseinterferenceindexone,1}} \dots \phasevectorcomponent{\impulseindex_{\impulseinterferenceindexone,\multiperiodicimpulsedimension}} \rangle  |}_{\overline{\impulsevariable},\underline{\impulsevariable + \intervalduration}} \}
		\\
		\heavisidedurationlowerupper{\Sh}{\intervalduration}{+} 
		&
		=
		\max_{\impulsevariable \in \testsetright}\{ \heavisidedurationlowerupper{\Sh}{\impulsevariable - \intervalduration, \impulsevariable }{} \} 
		=
		\max_{\impulsevariable \in \testsetright }\{
	\amplitudevector{\impulseindex_{\impulseinterferenceindexone,1} \dots \impulseindex_{\impulseinterferenceindexone,\multiperiodicimpulsedimension}} \overline{| \langle \phasevectorcomponent{\impulseindex_{\impulseinterferenceindexone,1}} \dots \phasevectorcomponent{\impulseindex_{\impulseinterferenceindexone,\multiperiodicimpulsedimension}} \rangle  |}_{\underline{\impulsevariable - \intervalduration},\overline{\impulsevariable }} \}
	\end{alignat}
\end{theorem}
\begin{proof}
	Let $\impulsevariable \in \realnumbers$ be a point.
	Let $\impulsevariable^{*} = \min \{ \impulsevariable' \in \testsetleft \, | \, \impulsevariable' \geq \impulsevariable \}$. By definition of $\impulsevariable^{*}$, there does not exist an impulse in the interval $(\impulsevariable, \impulsevariable^{*})$. It follows that no impulse is excluded when shifting the \heavisidedurationname s $\heavisidedurationupperupper{\Sh}{\impulsevariable, \impulsevariable + \intervalduration}{} $ and $\heavisidedurationupperlower{\Sh}{\impulsevariable, \impulsevariable + \intervalduration}{} $ to $\heavisidedurationupperupper{\Sh}{\impulsevariable^{*}, \impulsevariable^{*} + \intervalduration}{} $ and $\heavisidedurationupperlower{\Sh}{\impulsevariable^{*}, \impulsevariable^{*} + \intervalduration}{} $, since the impulse at $\impulsevariable^{*}$ is included by these \heavisidedurationname s.
	This means the \heavisidedurationname{} can only increase from $\impulsevariable$ to $\impulsevariable^{*}$. Formally,
	\begin{alignat}{1}
		\heavisidedurationupperupper{\Sh}{\impulsevariable, \impulsevariable + \intervalduration}{} \leq \heavisidedurationupperupper{\Sh}{\impulsevariable^{*}, \impulsevariable^{*} + \intervalduration}{}
		\\
		\heavisidedurationupperlower{\Sh}{\impulsevariable, \impulsevariable + \intervalduration}{} \leq \heavisidedurationupperlower{\Sh}{\impulsevariable^{*}, \impulsevariable^{*} + \intervalduration}{}
	\end{alignat}
	which means the \heavisidedurationnameplural{} in the intervals $[\impulsevariable, \impulsevariable + \intervalduration]$ and $[\impulsevariable, \impulsevariable + \intervalduration)$ are respectively smaller or equal to the \heavisidedurationnameplural{} in the intervals $[\impulsevariable^{*} + \intervalduration, \impulsevariable^{*} + \intervalduration]$ and $[\impulsevariable^{*} + \intervalduration, \impulsevariable^{*} + \intervalduration)$.
	Therefore, $		 \max_{\impulsevariable \in [\localextremevalueintervalleft, \localextremevalueintervalright] \cap \setofimpulsetimepoint}\{ \heavisidedurationupperupper{\Sh}{\impulsevariable, \impulsevariable + \intervalduration}{} \}$ and respectively $		 \max_{\impulsevariable \in [\localextremevalueintervalleft, \localextremevalueintervalright] \cap \setofimpulsetimepoint}\{ \heavisidedurationupperlower{\Sh}{\impulsevariable, \impulsevariable + \intervalduration}{} \} $ is found at some $\impulsevariable = \impulsevariable^{*} \in \testsetleft$.
	
	Now, consider $\impulsevariable^{*} = \max \{ \impulsevariable' \in \testsetright \, | \, \impulsevariable' \leq \impulsevariable + \intervalduration \}$.
	By definition of $\impulsevariable^{*}$, there does not exist an impulse in the interval $(\impulsevariable^{*} + \intervalduration, \impulsevariable + \intervalduration)$. It follows that no impulse is excluded when shifting the \heavisidedurationname{} $\heavisidedurationlowerupper{\Sh}{\impulsevariable - \intervalduration, \impulsevariable }{} $ to $\heavisidedurationlowerupper{\Sh}{\impulsevariable^{*} - \intervalduration, \impulsevariable^{*}}{} $ since the impulse at $\impulsevariable^{*}$ is included by the \heavisidedurationname{}.
	This means the \heavisidedurationname{} can only increase from $\impulsevariable + \intervalduration$ to $\impulsevariable^{*}$. Formally,
	\begin{alignat}{1}
		\heavisidedurationlowerupper{\Sh}{\impulsevariable - \intervalduration, \impulsevariable }{} \leq \heavisidedurationlowerupper{\Sh}{\impulsevariable^{*} - \intervalduration, \impulsevariable^{*}}{}
	\end{alignat}
	Therefore, $		 \max_{\impulsevariable \in [\localextremevalueintervalleft, \localextremevalueintervalright] \cap \setofimpulsetimepoint}\{ \heavisidedurationlowerupper{\Sh}{\impulsevariable - \intervalduration, \impulsevariable}{} \}$  is found at some $\impulsevariable = \impulsevariable^{*} \in \testsetright$.
	\end{proof}

\begin{theorem}[Analysis of minimum \heavisidedurationname]\label{theorem:analysisofminimumheavisideduration} 
	Let $\localextremevalueintervalleft,\localextremevalueintervalright \in \realnumbers$ with $\localextremevalueintervalleft \leq \localextremevalueintervalright$, $\intervalduration \in \nonnegativerealnumbers$ and  $\impulseinterference{\impulseshift}{\impulsedegree}{\impulsevariable}$ be a \impulseinterferencename{}.
	Let $\setofimpulsetimepoint$ be the set of values at which impulses of $\impulseinterference{\impulseshift}{\impulsedegree}{\impulsevariable}$ occur and let $\setofimpulsetimepointminusintervalduration \coloneqq \{ a - \intervalduration \, | \, a \in \setofimpulsetimepoint \}$ be the set of impulse points subtracted by the interval duration $\intervalduration \in \nonnegativerealnumbers$.
	Let 
	\begin{alignat}{1}
		\testsetleft &=  ([\localextremevalueintervalleft, \localextremevalueintervalright] \cap \setofimpulsetimepoint) \cup \{\localextremevalueintervalleft\}
		\\
		\testsetright &= ([\localextremevalueintervalleft + \intervalduration, \localextremevalueintervalright + \intervalduration] \cap \setofimpulsetimepoint) \cup \{\localextremevalueintervalright + \intervalduration\}
	\end{alignat}
Then, the \minimumheavisidedurationname{} is computed by
\begin{alignat}{1}
	\heavisidedurationlowerlower{\Sh}{\intervalduration}{-} 
	&
	=
	\min_{\impulsevariable \in \testsetleft}\{ \heavisidedurationlowerlower{\Sh}{\impulsevariable, \impulsevariable + \intervalduration}{} \} 
	=
	\min_{\impulsevariable \in \testsetleft}\{
	\amplitudevector{\impulseindex_{\impulseinterferenceindexone,1} \dots \impulseindex_{\impulseinterferenceindexone,\multiperiodicimpulsedimension}} \overline{| \langle \phasevectorcomponent{\impulseindex_{\impulseinterferenceindexone,1}} \dots \phasevectorcomponent{\impulseindex_{\impulseinterferenceindexone,\multiperiodicimpulsedimension}} \rangle  |}_{\underline{\impulsevariable},\underline{\impulsevariable + \intervalduration}} \}
	\\
	\heavisidedurationlowerupper{\Sh}{\intervalduration}{-} 
	&
	=
	\min_{\impulsevariable \in \testsetleft}\{ \heavisidedurationlowerupper{\Sh}{\impulsevariable, \impulsevariable + \intervalduration}{} \} 
	=
	\min_{\impulsevariable \in \testsetleft}\{
	\amplitudevector{\impulseindex_{\impulseinterferenceindexone,1} \dots \impulseindex_{\impulseinterferenceindexone,\multiperiodicimpulsedimension}} \overline{| \langle \phasevectorcomponent{\impulseindex_{\impulseinterferenceindexone,1}} \dots \phasevectorcomponent{\impulseindex_{\impulseinterferenceindexone,\multiperiodicimpulsedimension}} \rangle  |}_{\underline{\impulsevariable},\overline{\impulsevariable + \intervalduration}} \}
	\\
	\heavisidedurationupperlower{\Sh}{\intervalduration}{-} 
	&
	=
	\min_{\impulsevariable \in \testsetright}\{ \heavisidedurationlowerupper{\Sh}{\impulsevariable - \intervalduration, \impulsevariable }{} \} 
	=
	\min_{\impulsevariable \in \testsetright}\{
	\amplitudevector{\impulseindex_{\impulseinterferenceindexone,1} \dots \impulseindex_{\impulseinterferenceindexone,\multiperiodicimpulsedimension}} \overline{| \langle \phasevectorcomponent{\impulseindex_{\impulseinterferenceindexone,1}} \dots \phasevectorcomponent{\impulseindex_{\impulseinterferenceindexone,\multiperiodicimpulsedimension}} \rangle  |}_{\overline{\impulsevariable - \intervalduration},\underline{\impulsevariable }} \}
\end{alignat}
\end{theorem}
\begin{proof}
	Let $\impulsevariable \in \realnumbers$ be a value.
	Let $\impulsevariable^{*} = \max \{ \impulsevariable' \in \testsetleft \, | \, \impulsevariable' \leq \impulsevariable \}$. By definition of $\impulsevariable^{*}$, there does not exist an impulse in the interval $(\impulsevariable^{*}, \impulsevariable)$. It follows that no impulse is included when shifting the \heavisidedurationname s $\heavisidedurationlowerlower{\Sh}{\impulsevariable, \impulsevariable + \intervalduration}{} $ and $\heavisidedurationlowerupper{\Sh}{\impulsevariable, \impulsevariable + \intervalduration}{} $ to $\heavisidedurationlowerlower{\Sh}{\impulsevariable^{*}, \impulsevariable^{*} + \intervalduration}{} $ and $\heavisidedurationlowerupper{\Sh}{\impulsevariable^{*}, \impulsevariable^{*} + \intervalduration}{} $, since the impulse at $\impulsevariable^{*}$ is not included by these \heavisidedurationname s.
	This means the \heavisidedurationname{} can only decrease from $\impulsevariable$ to $\impulsevariable^{*}$. Formally,
	\begin{alignat}{1}
		\heavisidedurationlowerlower{\Sh}{\impulsevariable, \impulsevariable + \intervalduration}{} \geq \heavisidedurationlowerlower{\Sh}{\impulsevariable^{*}, \impulsevariable^{*} + \intervalduration}{}
		\\
		\heavisidedurationlowerupper{\Sh}{\impulsevariable, \impulsevariable + \intervalduration}{} \geq \heavisidedurationlowerupper{\Sh}{\impulsevariable^{*}, \impulsevariable^{*} + \intervalduration}{}
	\end{alignat}
	Therefore, $\min_{\impulsevariable \in [\localextremevalueintervalleft,\localextremevalueintervalright]}\{ \heavisidedurationlowerlower{\Sh}{\impulsevariable, \impulsevariable + \intervalduration}{} \} $ and respectively $\min_{\impulsevariable \in [\localextremevalueintervalleft,\localextremevalueintervalright]}\{ \heavisidedurationlowerupper{\Sh}{\impulsevariable, \impulsevariable + \intervalduration}{} \}$ is found at some $\impulsevariable = \impulsevariable^{*} \in \testsetleft$.
	
	Now, consider $\impulsevariable^{*} = \min \{ a \in \testsetright \, | \, a \geq \impulsevariable + \intervalduration \}$.
	By definition of $\impulsevariable^{*}$, there does not exist an impulse in the interval $(\impulsevariable + \intervalduration, \impulsevariable^{*})$. It follows that no impulse is included when shifitng the \heavisidedurationname{} $\heavisidedurationupperlower{\Sh}{\impulsevariable, \impulsevariable + \intervalduration}{} $ to $\heavisidedurationupperlower{\Sh}{\impulsevariable^{*} - \intervalduration, \impulsevariable^{*}}{} $ since the impulse at $\impulsevariable^{*}$ is excluded by the \heavisidedurationname{}.
	This means the \heavisidedurationname{} can only decrease from $\impulsevariable + \intervalduration$ to $\impulsevariable^{*}$. Formally,
	\begin{alignat}{1}
		\heavisidedurationupperlower{\Sh}{\impulsevariable, \impulsevariable + \intervalduration}{} \geq \heavisidedurationupperlower{\Sh}{\impulsevariable^{*} - \intervalduration, \impulsevariable^{*}}{}
	\end{alignat}
	Therefore, $\min_{\impulsevariable \in [\localextremevalueintervalleft,\localextremevalueintervalright]}\{ \heavisidedurationupperlower{\Sh}{\impulsevariable - \intervalduration, \impulsevariable}{} \}$ is found at some $\impulsevariable = \impulsevariable^{*} \in \testsetright$.
\end{proof}


\subsection{Graph of the maximum bound}
The interval transformation describes the \maximumheavisidedurationname{} for a single distance $\intervalduration$. 
We showed that the transformation is computed by a maximum over a finite set of \heavisidedurationnameplural{}.
However, if we want to compute the graph of the \maximumheavisidedurationname{} of the interval $[\localextremevalueintervalleft,\localextremevalueintervalright]$, then we have to find all intervals $[x_{1},x_{2}] \subseteq [\localextremevalueintervalleft,\localextremevalueintervalright]$ that result in the \maximumheavisidedurationname{} and \minimumheavisidedurationname{}.

The approach is to determine all interval length $x_{2} - x_{1}$ that are required to find all function values of the interval transformation and then to apply Theorem \ref{theorem:analysisofmaximumheavisideduration} and \ref{theorem:analysisofminimumheavisideduration} to compute the graph of the interval transformation. We begin by defining the set of \distancesetname{} in the interval $[a,b]$.
\begin{definition}[\distancesetnamecapital]
	Let
	 $\impulseinterference{\impulseshift}{\impulsedegree}{\impulsevariable}$
	be a \impulseinterferencenameshort{} and $a,b \in \realnumbers$ with $a \leq b$. 
	Let $\setofimpulsetimepoint$ be the set of values at which impulses of $\impulseinterference{\impulseshift}{\impulsedegree}{\impulsevariable}$ occur. Then,
	\begin{equation}
		\distanceset{\setofimpulsetimepoint}{a}{b} = \big\{ |x_{1} - x_{2}|  ~\big| ~  x_{1},x_{2} \in \setofimpulsetimepoint \cup \{a,b\} ~ \land ~ x_{1},x_{2} \in [a,b] \big\}
	\end{equation}
	describes the set of absolute differences of \shiftnameplural{} of $\impulseinterference{\impulseshift}{\impulsedegree}{\impulsevariable}$ in the interval $[a,b]$ and their absolute differences to the interval boundaries $a$ and $b$. $\distanceset{\setofimpulsetimepoint}{a}{b} $ is called  \textbf{\distancesetname} of $[a,b]$.
\end{definition}
In Section \ref{section:analysisofextremevalues}, we derived the finite set of test values to compute the local extrema of a given interval $[a,b]$ and some $\intervalduration$. To complete the interval transformation, we show that the required $\intervalduration$ to find the local extrema in $[a,b]$ are equal to the \distancesetname{} in $[a,b]$.
\begin{lemma}[Piece-wise constant \heavisidedurationname]\label{lemma:piecewiseconstant}
	Let
	$\impulseinterference{\impulseshift}{\impulsedegree}{\impulsevariable}$
	be a \impulseinterferencenameshort{} and $a,b \in \realnumbers$ with $a \leq b$. 
	Let $\setofimpulsetimepoint$ be the set of values at which impulses of $\impulseinterference{\impulseshift}{\impulsedegree}{\impulsevariable}$ occur. 
	Let $\distanceset{\setofimpulsetimepoint}{a}{b} $ be the  \distancesetname{} of $[a,b]$.
	Let $x^{*} \in \setofimpulsetimepoint \cap [a,b]$ and $\intervalduration_{1}, \intervalduration_{2} \in \distanceset{\setofimpulsetimepoint}{a}{b} $ with $\intervalduration_{1} < \intervalduration_{2}$, such that $x^{*}  + \intervalduration_{1} \in [a,b]$ and $x^{*} + \intervalduration_{2} \in [a,b]$ are two consecutive impulse shifts, i.e. an impulse occurs at $x^{*}  + \intervalduration_{1}$ and another impulse occurs at  $x^{*} + \intervalduration_{2}$ and there does not exist an impulse occurring in the interval $(x^{*} + \intervalduration_{1}, x^{*} + \intervalduration_{2})$.
	Then,
	\begin{alignat}{1}
		\forall \intervalduration \in (\intervalduration_{1}, \intervalduration_{2}) \colon
		\heavisidedurationupperupper{\Sh}{x^{*}, x^{*} + \intervalduration_{1}}{} 
		&
		=
		\heavisidedurationupperupper{\Sh}{x^{*}, x^{*} + \intervalduration}{} 
		\nonumber
		\\
		&
		<
		\heavisidedurationupperupper{\Sh}{x^{*}, x^{*}+ \intervalduration_{2}}{} 
		\\
		\forall \intervalduration \in ( \intervalduration_{1}, \intervalduration_{2}) \colon
		\heavisidedurationlowerupper{\Sh}{x^{*}, x^{*} + \intervalduration_{1}}{} 
		&
		=
		\heavisidedurationlowerupper{\Sh}{x^{*}, x^{*} + \intervalduration}{} 
		\nonumber
		\\
		&
		<
		\heavisidedurationupperlower{\Sh}{x^{*}, x^{*}+ \intervalduration_{2}}{} 
		\\
		\forall \intervalduration \in (\intervalduration_{1}, \intervalduration_{2}) \colon
		\heavisidedurationupperlower{\Sh}{x^{*}, x^{*}+ \intervalduration_{1}}{} 
		&
		<
		\heavisidedurationupperlower{\Sh}{x^{*}, x^{*}+ \intervalduration}{} 
		\nonumber
		\\
		&
		=
		\heavisidedurationupperlower{\Sh}{x^{*}, x^{*}+ \intervalduration_{2}}{} 
	\end{alignat}
	which means that the \heavisidedurationname{} is a piece-wise constant and monotonically increasing function. 
\end{lemma}
\begin{proof}
	Consider the intervals $[x^{*},x^{*} + \intervalduration_{1}]$, $(x^{*},x^{*} + \intervalduration_{1}]$ and $[x^{*},x^{*} + \intervalduration_{1})$.
	The sum of the amplitudes of impulses occurring in these intervals are computed by the \heavisidedurationnameplural{} $\heavisidedurationupperupper{\Sh}{x^{*}, x^{*}+ \intervalduration_{1}}{} $, $\heavisidedurationlowerupper{\Sh}{x^{*}, x^{*}+ \intervalduration_{1}}{} $ and $\heavisidedurationupperlower{\Sh}{x^{*}, x^{*}+ \intervalduration_{1}}{} $, called the first, second and third \heavisidedurationname{} in this proof.
	Note that the impulse at $x^{*} + \intervalduration_{1}$ is included by the first and second and excluded by the third \heavisidedurationname.
	This means that
	\begin{equation}
		\forall \intervalduration \in (\intervalduration_{1}, \intervalduration_{2}) \colon
		\heavisidedurationupperlower{\Sh}{x^{*}, x^{*}+ \intervalduration_{1}}{} 
		<
		\heavisidedurationupperlower{\Sh}{x^{*}, x^{*}+ \intervalduration}{} 
	\end{equation}
	since the impulse is included as soon as $\intervalduration > \intervalduration_{1}$.
	However, by assumption, there does not occur an impulse in the interval $(x^{*} + \intervalduration_{1}, x^{*} + \intervalduration_{2})$. 
	Hence, the \heavisidedurationnameplural{} $\heavisidedurationupperupper{\Sh}{x^{*}, x^{*}+ \intervalduration}{} $, $\heavisidedurationlowerupper{\Sh}{x^{*}, x^{*}+ \intervalduration}{} $ and $\heavisidedurationupperlower{\Sh}{x^{*}, x^{*}+ \intervalduration}{}$  are constant $\forall \intervalduration \in (\intervalduration_{1}, \intervalduration_{2})$. 
	This implies for the third \heavisidedurationname{} that
	\begin{equation}\label{eq:thirdheavisideduration}
		\forall \intervalduration \in (\intervalduration_{1}, \intervalduration_{2}) \colon
		\heavisidedurationupperlower{\Sh}{x^{*}, x^{*}+ \intervalduration_{1}}{} 
		<
		\heavisidedurationupperlower{\Sh}{x^{*}, x^{*}+ \intervalduration}{} 
		= 
				\heavisidedurationupperlower{\Sh}{x^{*}, x^{*}+ \intervalduration_{2}}{} 
	\end{equation}
	Furthermore, as there occurs an impulse at $x^{*} + \intervalduration_{2}$, it follows that the first and second \heavisidedurationname{} increase by the amplitude of that impulse, formally, 
	\begin{alignat}{1}\label{eq:firstheavisideduration}
		\heavisidedurationupperupper{\Sh}{x^{*}, x^{*} + \intervalduration}{} 
		&
		<
		\heavisidedurationupperupper{\Sh}{x^{*}, x^{*}+ \intervalduration_{2}}{} 
		\\
		\heavisidedurationlowerupper{\Sh}{x^{*}, x^{*} + \intervalduration}{} 
		&
		<
		\heavisidedurationupperlower{\Sh}{x^{*}, x^{*}+ \intervalduration_{2}}{}  \label{eq:secondheavisideduration}
	\end{alignat}
	From \eqref{eq:thirdheavisideduration}, \eqref{eq:firstheavisideduration} and \eqref{eq:secondheavisideduration}, it follows that the \heavisidedurationname{} is constant $\forall \intervalduration \in (\intervalduration_{1}, \intervalduration_{2})$ and its function value discontinuously increases at \shiftnameplural{} of impulses such as $x^{*}+ \intervalduration_{1}$ and $x^{*}+ \intervalduration_{2}$.
\end{proof}
The left- or right-continuity at jump discontinuities depends on the choice of the Heaviside mask.
Based on the piece-wise constant property of the \heavisidedurationname{}, we present the computation of the graph of the interval transformation for an interval $[a,b]$.
\begin{theorem}[Graph of interval transformation]\label{theorem:graphofintervaltransformation}
	Let
	$\impulseinterference{\impulseshift}{\impulsedegree}{\impulsevariable}$
	be a \impulseinterferencenameshort{} and $a,b \in \realnumbers$ with $a \leq b$. 
	Let $\setofimpulsetimepoint$ be the set of \shiftnameplural{} of $\impulseinterference{\impulseshift}{\impulsedegree}{\impulsevariable}$ occur. 
	Let $\distanceset{\setofimpulsetimepoint}{a}{b} $ be the  \distancesetname{} of $[a,b]$.
	Then, the interval transformation of the interval $[a,b]$ is computed by
	\begin{alignat}{1}
		\intervaltransformationupperupper{+}{a}{b} 
		&
		=
		\left\{ \left(\intervalduration, \heavisidedurationupperupper{\Sh}{\intervalduration}{+}\right) \big| \intervalduration \in \distanceset{\setofimpulsetimepoint}{a}{b} \right\}
		\label{eq:graph1}
		\\
		\intervaltransformationupperlower{+}{a}{b} 
		&
		=
		\left\{ \left(\intervalduration, \heavisidedurationupperlower{\Sh}{\intervalduration}{+}\right) \big| \intervalduration \in \distanceset{\setofimpulsetimepoint}{a}{b} \right\}
		\label{eq:graph2}
		\\
		\intervaltransformationlowerupper{+}{a}{b} 
		&
		=
		\left\{ \left(\intervalduration, \heavisidedurationlowerupper{\Sh}{\intervalduration}{+}\right) \big| \intervalduration \in \distanceset{\setofimpulsetimepoint}{a}{b} \right\}
		\label{eq:graph3}
		\\
		\intervaltransformationlowerlower{-}{a}{b} 
		&
		=
		\left\{ \left(\intervalduration, \heavisidedurationlowerlower{\Sh}{\intervalduration}{-}\right) \big| \intervalduration \in \distanceset{\setofimpulsetimepoint}{a}{b} \right\}
		\label{eq:graph4}
		\\
		\intervaltransformationlowerupper{-}{a}{b} 
		&
		=
		\left\{ \left(\intervalduration, \heavisidedurationlowerupper{\Sh}{\intervalduration}{-}\right) \big| \intervalduration \in \distanceset{\setofimpulsetimepoint}{a}{b} \right\}
		\label{eq:graph5}
		\\
		\intervaltransformationupperlower{-}{a}{b} 
		&
		=
		\left\{ \left(\intervalduration, \heavisidedurationupperlower{\Sh}{\intervalduration}{-}\right) \big| \intervalduration \in \distanceset{\setofimpulsetimepoint}{a}{b} \right\}
		\label{eq:graph6}
	\end{alignat}
\end{theorem}
\begin{proof}
	By Lemma \ref{lemma:piecewiseconstant}, the \heavisidedurationname{} is constant and only increases at shifts of $\impulseinterference{\impulseshift}{\impulsedegree}{\impulsevariable}$. 
	This means, if 		$X(a,b) =	\setofimpulsetimepoint \cup \{a,b\} \cap [a,b] $ is the set of shifts with $a$ and $b$ in $[a,b]$, then all function values of the \heavisidedurationname{} in the interval $[a,b]$ are covered by  the set 
	\begin{alignat}{1}
		\heavisideduration{\Sh}{[a,b]}{}
		&
		=
		\{ \heavisideduration{\Sh}{x_{1},x_{2}}{} \, \big| \, x_{1},x_{2} \in X(a,b) \}
		\\
		&
		=
		\{ \heavisideduration{\Sh}{x_{1},x_{1} + x_{2} - x_{1}}{} \, \big| \, x_{1},x_{2} \in X(a,b) \}
		\\
		&
		=
		\{ \heavisideduration{\Sh}{x_{1},x_{1} + \intervalduration}{} \, \big| \, x_{1} \in X(a,b), \intervalduration \in \distanceset{\setofimpulsetimepoint}{a}{b}, x_{1} + \intervalduration \leq b \}
		\\
		&
		=
		\bigcup_{\intervalduration \in \distanceset{\setofimpulsetimepoint}{a}{b}} \{ \heavisideduration{\Sh}{x_{1},x_{1} + \intervalduration}{} \, \big| \, x_{1} \in X(a,b), x_{1} + \intervalduration \leq b \}
	\end{alignat}
	which means we can describe the function values of the \heavisidedurationname{} as a function of $\intervalduration$. For a given $\intervalduration$, we can now compute a local extremum by 
	\begin{alignat}{1}
		\bigcup_{\intervalduration \in \distanceset{\setofimpulsetimepoint}{a}{b}} \max\{ \heavisideduration{\Sh}{x_{1},x_{1} + \intervalduration}{} \, \big| \, x_{1} \in X(a,b), x_{1} + \intervalduration \leq b \}
		\\
		\bigcup_{\intervalduration \in \distanceset{\setofimpulsetimepoint}{a}{b}} \min\{ \heavisideduration{\Sh}{x_{1},x_{1} + \intervalduration}{} \, \big| \, x_{1} \in X(a,b), x_{1} + \intervalduration \leq b \}
	\end{alignat}
	But this is exactly computed by $\heavisidedurationupperupper{\Sh}{\intervalduration}{+}$, $\heavisidedurationupperlower{\Sh}{\intervalduration}{+}$, $\heavisidedurationlowerlower{\Sh}{\intervalduration}{-}$ and $\heavisidedurationlowerupper{\Sh}{\intervalduration}{-}$, by Theorem \ref{theorem:analysisofmaximumheavisideduration} and \ref{theorem:analysisofminimumheavisideduration} .
	Similarly, we have 
	\begin{alignat}{1}
		\heavisideduration{\Sh}{[a,b]}{}
		&
		=
		\{ \heavisideduration{\Sh}{x_{1},x_{2}}{} \, \big| \, x_{1},x_{2} \in X(a,b) \}
		\\
		&
		=
		\{ \heavisideduration{\Sh}{x_{2} + x_{1} - x_{2}, x_{2} }{} \, \big| \, x_{1},x_{2} \in X(a,b) \}
		\\
		&
		=
		\{ \heavisideduration{\Sh}{x_{2} - \intervalduration,x_{2} }{} \, \big| \, x_{2} \in X(a,b), \intervalduration \in \distanceset{\setofimpulsetimepoint}{a}{b}, x_{2} - \intervalduration \geq a \}
		\\
		&
		=
		\bigcup_{\intervalduration \in \distanceset{\setofimpulsetimepoint}{a}{b}} \{ \heavisideduration{\Sh}{x_{2} - \intervalduration,x_{2}}{} \, \big| \, x_{2} \in X(a,b), x_{2} - \intervalduration \geq a \}
	\end{alignat}
	in case of $\heavisidedurationlowerupper{\Sh}{\intervalduration}{+}$ and $\heavisidedurationupperlower{\Sh}{\intervalduration}{-}$.
	As $\heavisideduration{\Sh}{[a,b]}{}$ includes all function values of the \heavisidedurationname{} in $[a,b]$, it follows that Equation \eqref{eq:graph1} to \eqref{eq:graph6} compute the whole graph.
\end{proof}

If the \heavisidedurationname{} is a periodic function, i.e. if $\heavisideduration{\Sh}{\impulsevariable, \impulsevariable + \intervalduration}{} = \heavisideduration{\Sh}{\impulsevariable + np , \impulsevariable + \intervalduration + np}{} $ where $p \in \positivrealnumbers$ is the period of the function and $n \in \naturalnumbers$, then we can find the global maximum and minimum of the \heavisidedurationname{} in the interval $[\localextremevalueintervalleft,\localextremevalueintervalright]$ where $\localextremevalueintervalright - \localextremevalueintervalleft = p$, because
\begin{equation}
	\forall \impulsevariable \in [\localextremevalueintervalleft, \localextremevalueintervalleft + p]\colon \heavisideduration{\Sh}{\impulsevariable , \impulsevariable + \intervalduration }{} = \heavisideduration{\Sh}{\impulsevariable + np , \impulsevariable + \intervalduration + np}{}
\end{equation}
so that all the different function values of the \heavisidedurationname{} already occur in one period.




%wenn wir das interval um eine periode erhöhen, verdoppelt sich der funktionswert. wir können also eine impulsdichte aufstellen, die sich periodisch wiederholt und durch das integral oder die diskretisierung wieder die eriegnisse periodisch zählt. jetzt muss man noch herleiten, wie man die heaviside duration zu einer dichte umwandeln kann.

This means we can limit the number of interval durations $\intervalduration$ required to compute the global maximum and minimum of a \heavisidedurationname{}. 
Generalizing, if the \heavisidedurationname{} is periodic or if the number of impulses of the \impulseinterferencenameshort{} is finite, then the interval transformation finds a global extremum based on a finite set of interval durations.
We note that if the number of impulses is infinite and if the \heavisidedurationname{} is not periodic, then we cannot find the global extrema.
%\subsection{Impulse count}
%In the last sections, we have built the mathematical structure to describe impulses. 
%Now, we present methods to analyze them. 
%A basic function to analyze impulses is to observe the number of impulses that occur in a given time interval $[a,b]$ with $a,b \in \realnumbers$. 
%This can be mathematically realized by the integral of the Dirac impulse. 
%More precisely, the integral $\int_{-\infty}^{\infty} \diracdelta(\impulsevariable) \, d \impulsevariable = 1$ (see \eqref{eq:integraldiracisone} in the Appendix) models the fact that an impulse is always observed in an infinite interval $[-\infty, \infty]$.
%If the time interval $[a,b]$ is bounded, i.e. $a$ and $b$ are not infinite, then we may observe the impulse $\diracdelta(\impulsevariable - \epsilon)$ dependent on the point $\epsilon \in \realnumbers$ at which it occurs:
%If $\epsilon \in [a,b]$, then the impulse occurs and is hence observed in $[a,b]$. Indeed, if $\epsilon \in (a,b)$, then $\int_{a}^{b} \diracdelta(\impulsevariable- \epsilon) \, d \impulsevariable = 1$ (see \eqref{eq:integraldiracboundedone}), so that the observed impulse is also modeled by the integral to be 1.
%
%However, if $\epsilon = a$ or $\epsilon = b$, then $\int_{a}^{b} \diracdelta(\impulsevariable- \epsilon) \, d \impulsevariable = 0$ (see \eqref{eq:integraldiracboundedzero}), so that this mathematical model does not represent the occurrence of the impulse at the boundary of the time interval.
%To model these boundary cases, we present a different approach based on the product of two Heaviside functions, called the Heaviside mask, that looks like a rectangle function with jumps at $a$ and $b$.
%This mask nullifies an impulse if it does not occur in $[a,b]$ and otherwise it is counted.
%As the Heaviside function has not a defined value at its jump, we define the upper and lower Heaviside functions, so that we can precisely apply it to model intervals.
%
%sagen das wir impulse zählen wollen. sagen, dass integralgrenzen keine ausgrenzung erluaben und daher Slomka die Heaviside Maske eingeführt hat. dann wird das kürzer.
%
%\begin{definition}[Upper and lower Heaviside functions]
%	The \textbf{upper Heaviside function} and \textbf{lower Heaviside function} are defined by
%	\begin{alignat}{1}
%		\upperheaviside(\impulsevariable) \coloneqq 
%		\left\{ 
%		\begin{array}{ll}
%			1,& \impulsevariable \geq 0 \\
%			0,& \impulsevariable < 0
%		\end{array}
%		\right.
%		\\
%		\lowerheaviside(\impulsevariable) \coloneqq 
%		\left\{ 
%		\begin{array}{ll}
%			1,& \impulsevariable > 0 \\
%			0,& \impulsevariable \leq 0
%		\end{array}
%		\right.
%	\end{alignat}
%\end{definition}
%Based on these two functions, we define four different Heaviside masks to model the four possible ways of open and closed intervals. Moreover, if there are other parameters shifting the point at which the impulse occurs, then we may want to check whether the shifted impulse is in the interval $[a,b]$. 
%To this end, we present the Heaviside vector mask. It is based on the one-dimensional mask from \cite{slomka2021beyond}.
%
%heaviside window nach diskretisierung. duration mask
%
%\begin{definition}[Heaviside mask]\label{def:heavisidemask}
%	Let $a,b\in \realnumbers$ and $\boldsymbol{\impulsevariable} \in \realnumbersvector{n}$ where $n \in \naturalnumbers$. Then,
%	\begin{alignat}{1}
%		\heavisidemaskoneupperupper{\boldsymbol{\impulsevariable}}{a}{b} 
%		&
%		\coloneqq \prod_{i = 1}^{n} \upperheaviside(\impulsevariable_{i} - a) \cdot \upperheaviside(b - \impulsevariable_{i}) = 
%		\left\{ 
%		\begin{array}{ll}
%			1,& \forall i \in \{1,2,\dots, n\}\colon \impulsevariable_{i} \in [a,b]  \\
%			0,& \exists i \in \{1,2,\dots, n\}\colon \impulsevariable_{i} \notin [a,b]
%		\end{array}
%		\right.
%		\\
%		\heavisidemaskoneupperlower{\boldsymbol{\impulsevariable}}{a}{b} 
%		&
%		\coloneqq \prod_{i = 1}^{n} \upperheaviside(\impulsevariable_{i} - a) \cdot \lowerheaviside(b - \impulsevariable_{i}) = 
%		\left\{ 
%		\begin{array}{ll}
%			1,& \forall i \in \{1,2,\dots, n\}\colon \impulsevariable_{i} \in [a,b)  \\
%			0,& \exists i \in \{1,2,\dots, n\}\colon \impulsevariable_{i} \notin [a,b)
%		\end{array}
%		\right.
%		\\
%		\heavisidemaskonelowerupper{\boldsymbol{\impulsevariable}}{a}{b} 
%		&
%		\coloneqq \prod_{i = 1}^{n} \lowerheaviside(\impulsevariable_{i} - a) \cdot \upperheaviside(b - \impulsevariable_{i}) = 
%		\left\{ 
%		\begin{array}{ll}
%			1,& \forall i \in \{1,2,\dots, n\}\colon \impulsevariable_{i} \in (a,b]  \\
%			0,& \exists i \in \{1,2,\dots, n\}\colon \impulsevariable_{i} \notin (a,b]
%		\end{array}
%		\right.
%		\\
%		\heavisidemaskonelowerlower{\boldsymbol{\impulsevariable}}{a}{b} 
%		&
%		\coloneqq \prod_{i = 1}^{n} \lowerheaviside(\impulsevariable_{i} - a) \cdot \lowerheaviside(b - \impulsevariable_{i}) = 
%		\left\{ 
%		\begin{array}{ll}
%			1,& \forall i \in \{1,2,\dots, n\}\colon \impulsevariable_{i} \in (a,b)  \\
%			0,& \exists i \in \{1,2,\dots, n\}\colon \impulsevariable_{i} \notin (a,b)
%		\end{array}
%		\right.
%	\end{alignat}
%\end{definition}
%Using the Heaviside mask, we can precisely describe whether impulses occur in open or closed intervals by multiplying a multidimensional Heaviside mask to the Dirac impulse. To the best of our knowledge, Lemma \ref{lemma:impulsecount} is a contribution of this report as we could not find such a manipulation of Dirac impulses by Heaviside masks in the literature.
%Furthermore, it discretizes the continuous approach of counting events over time from \cite{slomka2021beyond} by the sifting property of the Dirac impulse.
%\begin{lemma}[Discrete Heaviside mask]\label{lemma:impulsecount}
%	Let $\impulsevariable \in \realnumbers$, $\boldsymbol{\impulsevariable},\boldsymbol{\tau} \in \realnumbersvector{n}$ where $n \in \naturalnumbers$ such that $\boldsymbol{\impulsevariable} = (\impulsevariable + \tau_{1}, \impulsevariable + \tau_{2},\dots,\impulsevariable+ \tau_{n})^{\intercal}$. Let $a,b \in \realnumbers$. Then,
%	\begin{alignat}{1}
%		\diracdelta(\impulsevariable - \epsilon) \cdot \heavisidemaskoneupperupper{\boldsymbol{\impulsevariable} }{a}{b} \coloneqq \int_{-\infty}^{\infty} \diracdelta(\impulsevariable - \epsilon) \cdot \prod_{i = 1}^{n}\heavisidemaskoneupperupper{\impulsevariable + \tau_{i}}{a}{b} \, d\impulsevariable = \prod_{i = 1}^{n} \heavisidemaskoneupperupper{\epsilon + \tau_{i}}{a}{b} \label{eq:impulsecount1}
%		\\
%		\diracdelta(\impulsevariable - \epsilon) \cdot \heavisidemaskoneupperlower{\boldsymbol{\impulsevariable} }{a}{b} \coloneqq  \int_{-\infty}^{\infty} \diracdelta(\impulsevariable - \epsilon) \cdot \prod_{i = 1}^{n} \heavisidemaskoneupperlower{\impulsevariable+ \tau_{i}}{a}{b} \, d\impulsevariable = \prod_{i = 1}^{n} \heavisidemaskoneupperlower{\epsilon+ \tau_{i}}{a}{b}
%		\\
%		\diracdelta(\impulsevariable - \epsilon) \cdot \heavisidemaskonelowerupper{\boldsymbol{\impulsevariable} }{a}{b} \coloneqq  \int_{-\infty}^{\infty} \diracdelta(\impulsevariable - \epsilon) \cdot \prod_{i = 1}^{n} \heavisidemaskonelowerupper{\impulsevariable+ \tau_{i}}{a}{b} \, d\impulsevariable = \prod_{i = 1}^{n} \heavisidemaskonelowerupper{\epsilon+ \tau_{i}}{a}{b}
%		\\
%		\diracdelta(\impulsevariable - \epsilon) \cdot \heavisidemaskonelowerlower{\boldsymbol{\impulsevariable} }{a}{b} \coloneqq  \int_{-\infty}^{\infty} \diracdelta(\impulsevariable - \epsilon) \cdot \prod_{i = 1}^{n} \heavisidemaskonelowerlower{\impulsevariable+ \tau_{i}}{a}{b} \, d\impulsevariable = \prod_{i = 1}^{n} \heavisidemaskonelowerlower{\epsilon+ \tau_{i}}{a}{b}
%	\end{alignat}
%\end{lemma}
%\begin{proof}
%	This proof applies distribution theory to show that the sifting property of the Dirac impulse is applied to the multidimensional Heaviside, see Appendix \ref{appendix:heavisidemask} for a detailed explanation of the theory.
%	Let $\boldsymbol{\chi} \in \realnumbersvector{k}$ for $k \in \naturalnumbers$, $x \in \realnumbers$ and $\boldsymbol{x} = (x - \chi_{1}, x - \chi_{2}, \dots, x- \chi_{k})^{\intercal}$.
%	Then,
%	\begin{alignat}{1}
%		&
%		\prod_{i = 1}^{k} \heaviside(x - \chi_{i}-a) \cdot \heaviside(b- x - \chi_{i}) \cdot \delta(x-\epsilon) 
%		\\
%		\coloneqq
%		&
%		\lim_{n \to \infty}	\langle \prod \tau_{a}h_{n} \cdot \tau_{-b} h_{n} \tau_{\epsilon}\delta, \varphi \rangle
%		\\
%		=
%		&
%		\lim_{n \to \infty}
%		\int_{-\infty}^{\infty} \prod_{i = 1}^{k}  h_{n}(x -\chi_{i}-a) \cdot h_{n}(b-x-\chi_{i}) \delta(x-\epsilon) \cdot \varphi(x) \, dx 
%		\\
%		&
%		=
%		\lim_{n \to \infty}
%		\int_{-\infty}^{\infty}  \delta(x-\epsilon) \cdot \underbrace{\prod_{i = 1}^{k} h_{n}(x-\chi_{i}-a) \cdot h_{n}(b-x-\chi_{i}) \cdot \varphi(x)}_{\in \mathcal{D}} \, dx 
%		\\
%		&
%		\oset[1.5ex]{\mathclap{\eqref{eq:diractranslation}}}{=}
%		\lim_{n \to \infty} \prod_{i = 1}^{k}  h_{n}(\epsilon -\chi_{i}- a) \cdot h_{n}(b - \epsilon-\chi_{i}) \cdot \varphi(\epsilon)
%		\\
%		&
%		=
%		\prod_{i = 1}^{k}  \heaviside(\epsilon-\chi_{i} - a) \cdot \heaviside(b - \epsilon-\chi_{i}) \cdot  \varphi(\epsilon)
%	\end{alignat}
%	
%\end{proof}
%
%\begin{comment}
%	Next, we generalize the impulse count to the count of an impulse density. From a mathematical point of view, we apply the multidimensional Heaviside mask to the impulse density which is distributed to each individual impulse, so that we can re-use the impulse count.
%	For one-dimensional Heaviside masks, this accumulation of Dirac impulses introduced by \cite{slomka2021beyond} is called unified event bound and is renamed to event count in this report.
%	\begin{theorem}[Event count \cite{slomka2021beyond}]\label{theorem:unifiedeventbound}
%		Let $\multivariateimpulsedensity_{\periodvector}^{\numberofrepetitionsvector} (\impulsevariable)  = 	a_{\repetitionindex_{1}, \dots, \repetitionindex_{\numberofcomponents}} \diracdelta^{ \repetitionindex_{1}, \dots, \repetitionindex_{\numberofcomponents}}_{\period_{1},\dots, \period_{\numberofcomponents}}$ be a multivariate impulse density as defined by Definition \ref{def:multivariateimpulsevectorspace}. Let $a,b \in \realnumbers$ with $a \leq  b$. 
%		Then, the \textbf{event count}, which is the number of events occurring in the interval $[a,b)$, can be counted by
%		\begin{alignat}{1}
%			\eventcountofinterval{\task}{a}{b} = \int_{-\infty}^{\infty} \sum_{\componentindex = 1}^{\numberofcomponents} \sum_{\repetitionindex = 0}^{\numberofrepetitionsvectorcomponent{\componentindex}-1} \diracdelta(\impulsevariable - \offsetvectorcomponent{\componentindex} - \repetitionindex\periodvectorcomponent{\componentindex}) \cdot \heavisidemaskone{\impulsevariable}{a}{b} \, d\impulsevariable \label{eq:unifiedeventbound1}
%		\end{alignat}
%		There are four possible event counts according to the four possible masks: $\upperuppereventcountofinterval{\task}{a}{b}$, $\upperlowereventcountofinterval{\task}{a}{b}$, $ \loweruppereventcountofinterval{\task}{a}{b}$ and $\lowerlowereventcountofinterval{\task}{a}{b}$.
%	\end{theorem}
%	\begin{proof}
%		The proof is shown in theorem 1 of \cite{slomka2021beyond}.
%	\end{proof}
%	One can imagine the calculation of the integral as follows: As $\impulsevariable$ increases over time, the Heaviside mask is shifted over the time line. 
%	If an event occurs at $\impulsevariable$, i.e. if $\impulsevariable= \offsetvectorcomponent{\componentindex} + \repetitionindex\periodvectorcomponent{\componentindex}$ and if $\impulsevariable \in [a,b)$, then $\heavisidemaskoneupperlower{\impulsevariable}{a}{b} = 1$, which means that the event occurs in the interval $[a,b)$. Hence, the event count of the interval $[a,b)$ is incremented by 1.
%	But this actually means that we only need to consider the points $\impulsevariable = \offsetvectorcomponent{\componentindex} + \repetitionindex\periodvectorcomponent{\componentindex}$ in order to find out whether events occur in the interval $[a,b)$. 
%	In this way, we would discretize the infinite set of points $\impulsevariable \in \realnumbers$ to a discrete set of event points which would improve the computation of the event count.
%	Now, let us consider the sifting property $\int_{-\infty}^{\infty} \diracdelta(\impulsevariable - a) f(\impulsevariable) \, d \impulsevariable = f(a)$ of the Dirac impulse.
%	It returns the function value of $f$ at the shift of the Dirac impulse.
%	In case of Equation \eqref{eq:unifiedeventbound1} this means the Heaviside mask is evaluated at the event point.
%	Therefore, the sifting property of the Dirac impulse already provides us with a discrete computation of the event count which extends the HeRTA framework with a fast computation.
%	\begin{theorem}[Discrete event count]\label{theorem:discreteeventcount}
%		Let $\task = (\periodvector, \offsetvector, \executiontimevector, \deadlinevector)_{\numberofrepetitionsvector}$ be a multiperiodic task as defined by Definition \ref{def:task}.
%		Let $a,b \in \realnumbers$ with $a \leq  b$. 
%		Then, the event count of an interval defined by $a$ and $b$ is computed by
%		\begin{alignat}{1}
%			\eventcountofinterval{\task}{a}{b} 
%			&
%			=
%			\int_{-\infty}^{\infty} \sum_{\componentindex = 1}^{\numberofcomponents} \sum_{\repetitionindex = 0}^{\numberofrepetitionsvectorcomponent{\componentindex}-1} \diracdelta(\impulsevariable - \offsetvectorcomponent{\componentindex} - \repetitionindex\periodvectorcomponent{\componentindex}) \cdot \heavisidemaskone{\impulsevariable}{a}{b} \, d\impulsevariable
%			\\
%			&
%			=
%			\sum_{\componentindex = 1}^{\numberofcomponents} \sum_{\repetitionindex = 0}^{\numberofrepetitionsvectorcomponent{\componentindex}-1}  \heavisidemaskone{\offsetvectorcomponent{\componentindex} + \repetitionindex\periodvectorcomponent{\componentindex}}{a}{b}
%		\end{alignat}
%		There are four possible event counts according to the four possible masks: $\upperuppereventcountofinterval{\task}{a}{b}$, $\upperlowereventcountofinterval{\task}{a}{b}$, $ \loweruppereventcountofinterval{\task}{a}{b}$ and $\lowerlowereventcountofinterval{\task}{a}{b}$.
%	\end{theorem}
%	\begin{proof}
%		The proof applies two properties of the event count function. First, the integral and the summation can be exchanged since the event count of any finite interval is finite, see \eqref{eq:shortdiscreteeventcount1} to \eqref{eq:shortdiscreteeventcount2}. Second, the sifting property of the Dirac impulse is applied to Heaviside mask of each event, see \eqref{eq:shortdiscreteeventcount2} to \eqref{eq:shortdiscreteeventcount3}. Formally,
%		\begin{alignat}{1}
%			\eventcountofinterval{\task}{a}{b}
%			&
%			=
%			\int_{-\infty}^{\infty} \sum_{\componentindex = 1}^{\numberofcomponents} \sum_{\repetitionindex = 0}^{\numberofrepetitionsvectorcomponent{\componentindex} - 1} \diracdelta(\impulsevariable - \offsetvectorcomponent{\componentindex} - \repetitionindex \periodvectorcomponent{\componentindex}) \cdot \heavisidemaskone{\impulsevariable}{a}{b} \, d \impulsevariable \label{eq:shortdiscreteeventcount1}
%			\\
%			&
%			=
%			\sum_{\componentindex = 1}^{\numberofcomponents} \sum_{\repetitionindex = 0}^{\numberofrepetitionsvectorcomponent{\componentindex} - 1} \int_{-\infty}^{\infty} \diracdelta(\impulsevariable - \offsetvectorcomponent{\componentindex} - \repetitionindex \periodvectorcomponent{\componentindex}) \cdot \heavisidemaskone{\impulsevariable}{a}{b} \, d \impulsevariable \label{eq:shortdiscreteeventcount2}
%			\\
%			&
%			=
%			\sum_{\componentindex = 1}^{\numberofcomponents} \sum_{\repetitionindex = 0}^{\numberofrepetitionsvectorcomponent{\componentindex} - 1}  \heavisidemaskone{\offsetvectorcomponent{\componentindex} + \repetitionindex \periodvectorcomponent{\componentindex}}{a}{b} \label{eq:shortdiscreteeventcount3}
%		\end{alignat}
%		The detailed proof is shown in Appendix \ref{appendix:discreteeventcount}.
%	\end{proof}
%	This discretization does not only provide a faster computation of the event count but also connects the HeRTA framework with classical real-time scheduling theory. In the following, we derive the demand bound function of \cite{baruah1990algorithms} as a discrete summation of Heaviside masks.
%	Inhalt...
%\end{comment}
%
%
%Hier das Zeitfenster definieren mit T mit 2 Strichen.
%
%
%
%Based on the discrete Heaviside mask, the count of a whole series of Dirac impulses is also discretized which discretizes the unified event bound function of \cite{slomka2021beyond}.
%\begin{theorem}[Heaviside series]\label{theorem:densitycount}
%	Let $\impulsevariable \in \realnumbers$, $\boldsymbol{\impulsevariable},\boldsymbol{\tau} \in \realnumbersvector{n}$ where $n \in \naturalnumbers$ such that $\boldsymbol{\impulsevariable} = (\impulsevariable - \tau_{1}, \impulsevariable - \tau_{2},\dots,\impulsevariable- \tau_{n})^{\intercal}$.
%	Let $\multivariateimpulsedensity_{\periodvector}^{\numberofrepetitionsvector} (\impulsevariable)  = 	a_{\repetitionindex_{1}, \dots, \repetitionindex_{\numberofcomponents}} \diracdelta^{ \repetitionindex_{1}, \dots, \repetitionindex_{\numberofcomponents}}_{\period_{1},\dots, \period_{\numberofcomponents}}$ be a multivariate impulse density as defined by Definition \ref{def:multivariateimpulsevectorspace}. Let $a,b \in \realnumbers$.
%	Then,  the \textbf{density count} of time interval $[a,b], [a,b), (a,b]$ and $(a,b)$ is
%	\begin{alignat}{1}
%		\upperuppereventcountofinterval{\multivariateimpulsedensity}{a}{b} 
%		&
%		\coloneqq 
%		\sum_{\repetitionindex_{1} = 1}^{\numberofrepetitionsvectorcomponent{1}} \dots \sum_{\repetitionindex_{\numberofcomponents} = 1}^{\numberofrepetitionsvectorcomponent{\numberofcomponents}} a_{\repetitionindex_{1},\dots,\repetitionindex_{\numberofcomponents}}\cdot  
%		\prod_{i = 1}^{k} \heavisidemaskoneupperupper{(\repetitionindex_{1}-1)\periodvectorcomponent{1} + \dots + (\repetitionindex_{\numberofcomponents}-1)\periodvectorcomponent{\numberofcomponents}+ \tau_{i}}{a}{b}
%		\\
%		\upperlowereventcountofinterval{\multivariateimpulsedensity}{a}{b} 
%		&
%		\coloneqq 
%		\sum_{\repetitionindex_{1} = 1}^{\numberofrepetitionsvectorcomponent{1}} \dots \sum_{\repetitionindex_{\numberofcomponents} = 1}^{\numberofrepetitionsvectorcomponent{\numberofcomponents}} a_{\repetitionindex_{1},\dots,\repetitionindex_{\numberofcomponents}}\cdot  
%		\prod_{i = 1}^{k} \heavisidemaskoneupperlower{(\repetitionindex_{1}-1)\periodvectorcomponent{1} + \dots + (\repetitionindex_{\numberofcomponents}-1)\periodvectorcomponent{\numberofcomponents} + \tau_{i}}{a}{b}
%		\\
%		\loweruppereventcountofinterval{\multivariateimpulsedensity}{a}{b} 
%		&
%		\coloneqq 
%		\sum_{\repetitionindex_{1} = 1}^{\numberofrepetitionsvectorcomponent{1}} \dots \sum_{\repetitionindex_{\numberofcomponents} = 1}^{\numberofrepetitionsvectorcomponent{\numberofcomponents}} a_{\repetitionindex_{1},\dots,\repetitionindex_{\numberofcomponents}}\cdot  
%		\prod_{i = 1}^{k} \heavisidemaskonelowerupper{(\repetitionindex_{1}-1)\periodvectorcomponent{1} + \dots + (\repetitionindex_{\numberofcomponents}-1)\periodvectorcomponent{\numberofcomponents} + \tau_{i}}{a}{b}
%		\\
%		\lowerlowereventcountofinterval{\multivariateimpulsedensity}{a}{b} 
%		&
%		\coloneqq 
%		\sum_{\repetitionindex_{1} = 1}^{\numberofrepetitionsvectorcomponent{1}} \dots \sum_{\repetitionindex_{\numberofcomponents} = 1}^{\numberofrepetitionsvectorcomponent{\numberofcomponents}} a_{\repetitionindex_{1},\dots,\repetitionindex_{\numberofcomponents}}\cdot  
%		\prod_{i = 1}^{k} \heavisidemaskonelowerlower{(\repetitionindex_{1}-1)\periodvectorcomponent{1} + \dots + (\repetitionindex_{\numberofcomponents}-1)\periodvectorcomponent{\numberofcomponents} + \tau_{i}}{a}{b}		
%	\end{alignat}
%	which is the sum of weights of impulses occuring in these intervals, respectively.
%\end{theorem}
%\begin{proof}
%	The proof applies the impulse count of Lemma \ref{lemma:impulsecount} to each impulse of the density. We show the proof for $\upperuppereventcountofinterval{\multivariateimpulsedensity}{a}{b}$. The other three cases are shown in the same way.
%	\begin{alignat}{1}
%		&
%		\sum_{\repetitionindex_{1} = 0}^{\numberofrepetitionsvectorcomponent{1}-1} \dots \sum_{\repetitionindex_{\numberofcomponents} = 0}^{\numberofrepetitionsvectorcomponent{\numberofcomponents}-1}
%		a_{\repetitionindex_{1}, \dots, \repetitionindex_{\numberofcomponents}} \cdot \diracdelta(\impulsevariable - (\repetitionindex_{1}-1) \period_{1} - \dots - (\repetitionindex_{\numberofcomponents}-1) \period_{\numberofcomponents}) 
%		\cdot \prod_{i = 1}^{k} \heavisidemaskoneupperupper{\impulsevariable + \tau_{i}}{a}{b}
%		\\
%		\oset[1.5ex]{\eqref{eq:impulsecount1}}{=}
%		&
%		\sum_{\repetitionindex_{1} = 0}^{\numberofrepetitionsvectorcomponent{1}-1} \dots \sum_{\repetitionindex_{\numberofcomponents} = 0}^{\numberofrepetitionsvectorcomponent{\numberofcomponents}-1} 
%		\int_{-\infty}^{\infty} a_{\repetitionindex_{1},\dots,\repetitionindex_{\numberofcomponents}}\cdot  \diracdelta(\impulsevariable -( \repetitionindex_{1}-1) \period_{1} - \dots - (\repetitionindex_{\numberofcomponents}-1) \period_{\numberofcomponents})  \cdot \prod_{i = 1}^{k}\heavisidemaskoneupperupper{\impulsevariable + \tau_{i}}{a}{b} \, d\impulsevariable
%		\\
%		\oset[1.5ex]{\eqref{eq:impulsecount1}}{=}
%		&
%		\sum_{\repetitionindex_{1} = 0}^{\numberofrepetitionsvectorcomponent{1}-1} \dots \sum_{\repetitionindex_{\numberofcomponents} = 0}^{\numberofrepetitionsvectorcomponent{\numberofcomponents}-1} 
%		a_{\repetitionindex_{1},\dots,\repetitionindex_{\numberofcomponents}} \cdot \prod_{i = 1}^{k}\heavisidemaskoneupperupper{( \repetitionindex_{1}-1) \period_{1} + \dots + (\repetitionindex_{\numberofcomponents}-1) \period_{\numberofcomponents} + \tau_{i}}{a}{b} \, d\impulsevariable
%	\end{alignat}
%\end{proof}
%\subsection{Maximum impulse count}
%
%lokales maximum, globales wenn periodisch.
%
%% jede unified event bound funktion is kompakt wenn \impulsevariable die variable ist. aber eine menge von endlichen werten lässt sich nicht beschränken. daher mit countability und periodizität.
%In the analysis of impulse densities we are often interested in computing the highest or lowest density of impulses. 
%More precisely, we investigate the impulse density for time intervals in which the density reaches a maximum or minimum count. 
%For a given interval duration $\intervalduration \in \nonnegativerealnumbers$, we want to find the time interval $[\impulsevariable, \impulsevariable + \intervalduration]$ that maximizes the impulse count that is formalized in the following.
%\begin{definition}[Maximum impulse count]
%	Let $\impulsevariable \in \realnumbers$, $\boldsymbol{\impulsevariable},\boldsymbol{\tau} \in \realnumbersvector{n}$ where $n \in \naturalnumbers$ such that $\boldsymbol{\impulsevariable} = (\impulsevariable - \tau_{1}, \impulsevariable - \tau_{2},\dots,\impulsevariable- \tau_{n})^{\intercal}$.
%	Let $\multivariateimpulsedensity_{\periodvector}^{\numberofrepetitionsvector} (\impulsevariable)  = 	a_{\repetitionindex_{1}, \dots, \repetitionindex_{\numberofcomponents}} \diracdelta^{ \repetitionindex_{1}, \dots, \repetitionindex_{\numberofcomponents}}_{\period_{1},\dots, \period_{\numberofcomponents}}$ be a multivariate impulse density as defined by Definition \ref{def:multivariateimpulsevectorspace}. Let $\impulsevariable \in \realnumbers$ and $\intervalduration \in \nonnegativerealnumbers$.
%	Then,  the \textbf{maximum impulse count} of time interval $[\impulsevariable, \impulsevariable + \intervalduration], [\impulsevariable, \impulsevariable + \intervalduration,\impulsevariable, \impulsevariable + \intervalduration), (\impulsevariable, \impulsevariable + \intervalduration]$ and $(\impulsevariable, \impulsevariable + \intervalduration)$ is defined by
%	\begin{alignat}{1}
%		\upperuppereventbound{\multivariateimpulsedensity}{\intervalduration} \coloneqq \max_{\impulsevariable \in \realnumbers} \{ \upperuppereventcountofinterval{\multivariateimpulsedensity}{\impulsevariable}{\impulsevariable + \intervalduration}  \}
%		\\
%		\upperlowereventbound{\multivariateimpulsedensity}{\intervalduration} \coloneqq \max_{\impulsevariable \in \realnumbers} \{ \upperlowereventcountofinterval{\multivariateimpulsedensity}{\impulsevariable}{\impulsevariable + \intervalduration}  \}
%		\\
%		\loweruppereventbound{\multivariateimpulsedensity}{\intervalduration} \coloneqq \max_{\impulsevariable \in \realnumbers} \{ \loweruppereventcountofinterval{\multivariateimpulsedensity}{\impulsevariable}{\impulsevariable + \intervalduration}  \}
%		\\
%		\lowerlowereventbound{\multivariateimpulsedensity}{\intervalduration} \coloneqq \max_{\impulsevariable \in \realnumbers} \{ \lowerlowereventcountofinterval{\multivariateimpulsedensity}{\impulsevariable}{\impulsevariable + \intervalduration}  \}
%	\end{alignat}
%\end{definition}
%This means we have to find the point $\impulsevariable^{+} \in \realnumbers$ that provides the maximum impulse count of all intervals $[\impulsevariable, \impulsevariable + \intervalduration]$.
%Consequently, we have to investigate infinitely many points $\impulsevariable$ since the maximum can be located at any time interval $[\impulsevariable, \impulsevariable + \intervalduration]$ of which there exists infinitely many due to $\impulsevariable \in \realnumbers$.
%
%However, the maximum of an infinite set of impulse counts does not necessarily exist since we cannot determine an upper bound on infinitely many impulse counts.
%To ensure that the maximum exists, we show in the following that we can reduce the search of the maximum from an infinite to a finite set of points $\impulsevariable$.
%This will imply that we compute a finite set of impulse counts of which a maximum exists.
%The method that computes the maximum impulse count on a finite set of density counts is called \textbf{$\intervalduration$-transformation} since it transforms a function of points (density count) to a function of interval durations (maximum impulse count). In the following two sections, we derive the $\intervalduration$-transformation.
%\subsubsection{Countable density count}
%\begin{lemma}[Countable $\intervalduration$-transformation]\label{lemma:countabledeltatransformation}
%	Let $\impulsevariable \in \realnumbers$, $\boldsymbol{\impulsevariable},\boldsymbol{\tau} \in \realnumbersvector{n}$ where $n \in \naturalnumbers$ such that $\boldsymbol{\impulsevariable} = (\impulsevariable - \tau_{1}, \impulsevariable - \tau_{2},\dots,\impulsevariable- \tau_{n})^{\intercal}$.
%	Let $\multivariateimpulsedensity_{\periodvector}^{\numberofrepetitionsvector} (\impulsevariable)  = 	a_{\repetitionindex_{1}, \dots, \repetitionindex_{\numberofcomponents}} \diracdelta^{ \repetitionindex_{1}, \dots, \repetitionindex_{\numberofcomponents}}_{\period_{1},\dots, \period_{\numberofcomponents}}$ be a multivariate impulse density as defined by Definition \ref{def:multivariateimpulsevectorspace}. Let $\impulsevariable \in \realnumbers$ and $\intervalduration \in \nonnegativerealnumbers$.
%	Let $\boldsymbol{\epsilon}$ be the set of points at which impulses occur.
%	Then,  the \textbf{maximum impulse count} can be computed based on a countable set of density counts:
%	\begin{alignat}{1}
%		\upperuppereventbound{\multivariateimpulsedensity}{\intervalduration} = \max_{\impulsevariable \in \boldsymbol{\epsilon}} \{ \upperuppereventcountofinterval{\multivariateimpulsedensity}{\impulsevariable}{\impulsevariable + \intervalduration}  \}
%		\\
%		\upperlowereventbound{\multivariateimpulsedensity}{\intervalduration} = \max_{\impulsevariable \in \boldsymbol{\epsilon}} \{ \upperlowereventcountofinterval{\multivariateimpulsedensity}{\impulsevariable}{\impulsevariable + \intervalduration}  \}
%		\\
%		\loweruppereventbound{\multivariateimpulsedensity}{\intervalduration} = \max_{\impulsevariable \in \boldsymbol{\epsilon}} \{ \loweruppereventcountofinterval{\multivariateimpulsedensity}{\impulsevariable}{\impulsevariable + \intervalduration}  \}
%		\\
%		\lowerlowereventbound{\multivariateimpulsedensity}{\intervalduration}  = \max_{\impulsevariable \in \boldsymbol{\epsilon}} \{ \lowerlowereventcountofinterval{\multivariateimpulsedensity}{\impulsevariable}{\impulsevariable + \intervalduration}  \}
%	\end{alignat}
%\end{lemma}
%\begin{proof}
%	The proof is shown in Appendix \ref{appendix:countabledeltatransformation}.
%\end{proof}
\newpage
\appendix
\section{Notation}\label{appendix:notation}
The linear space of impulses requires a lot of composed summations and convolutions. Therefore, we introduce some short-form notations. First, we apply the well-known Einstein notation \cite{einstein1916foundation}
\begin{equation}
	\sum_{\impulseindex = 0}^{\impulsedegree - 1} c_{\impulseindex}x^{\impulseindex} = c_{0}x^{0} + c_{1}x^{1} + c_{2}x^{2} + c_{3}x^{3} + \ldots + c_{\impulsedegree - 1}x^{\impulsedegree - 1} \eqqcolon c_{\impulseindex}x^{\impulseindex}
\end{equation}
for a short-form notation of summations. We have 
\begin{alignat}{1}
	\impulsespectraldensityeinsteinmono{\amplitude}{\impulseindex}{\impulseshift}{\impulsevariable} 
	=
	\sum_{\impulseindex = 0}^{\impulsedegree - 1} \amplitudevector{\impulseindex} \diracdelta(\impulsevariable - \impulseshift_{\impulseindex})
	=
	\sum_{\impulseindex = 0}^{\impulsedegree - 1} 
	\amplitudevector{\impulseindex} \diracdelta(\impulsevariable - \impulseindex \cdot \impulseshift)
\end{alignat}
If multiple summations of the form $\impulsespectraldensityeinsteinmono{\amplitude}{\impulseindex}{\impulseshift}{\impulsevariable} $ are convolved, we represent their convolution by the enclosing angles $\langle \rangle$ as follows
\begin{alignat}{1}
	\impulsespectraldensityeinsteinmulti{\amplitude}{\impulseindex}{\impulseshift}{\impulsevariable} = \sum_{\impulseindex_{1} = 0}^{\degreevectorcomponent{1} - 1} 
	\amplitudevector{\impulseindex_{1}} \diracdelta(\impulsevariable - \impulseindex_{1} \phasevectorcomponent{1}) 
	\ast
	\ldots 
	\ast 
	\sum_{\impulseindex_{\impulseinterferenceindexone} = 0}^{\degreevectorcomponent{\impulseinterferenceindexone} - 1} 
	\amplitudevector{\impulseindex_{\impulseinterferenceindexone}} \diracdelta(\impulsevariable - \impulseindex_{\impulseinterferenceindexone} \phasevectorcomponent{\impulseinterferenceindexone}) 
\end{alignat}
Note that the enclosing angles $\langle \rangle$ to represent convolution are to be distinguished from the notation $\langle , \rangle$ to represent a distribution. Moreover, greek and latine letters are used to describe scalars and vectors, respectively. Further notations are listed in Table \ref{table:listofsymbols}.
%\renewcommand{\arraystretch}{1.3}
%\renewcommand{\cellalign}{cl}
%

%\begin{table}
%	\begin{center}
%		\begin{tabular}{l  l } 
%			Symbol & Meaning  \\ [0.5ex] 
%			\hline\hline
%			$\naturalnumbers$, $\naturalnumberswithzero$,$\naturalnumbersuntil{n}$
%			&
%			\makecell{set of natural numbers, natural numbers with zero, \\ natural numbers $\{1,2,\dots n\}$}
%			\\
%			\hline
%			$\realnumbers, \realnumbersvector{n}$
%			&
%			set of real numbers, $n$-dimensional vectors
%			\\
%			\hline
%			\impulseshift 
%			&
%			shift
%			\\
%			\hline
%			$\amplitude $
%			&
%			amplitude
%			\\
%			\hline
%			$\impulsedegree $
%			&
%			degree
%			\\
%			\hline
%			$\shiftedimpulse{\impulseindex}{\impulsevariable}$ 
%			&
%			\makecell{Dirac impulse function $\diracdelta(\impulsevariable - \impulseindex s)$ of variable $\impulsevariable$ \\ shifted by $\impulseindex \cdot s$}
%			\\ 
%			\hline
%			$\shiftedimpulseset{\impulsevariable}$ & set of all shifted Dirac impulse functions of variable $\impulsevariable$  \\
%			\hline
%			$\impulsegroupoperation$ &  addition operator on two impulse functions \\
%			\hline
%			$\impulsegroup{\impulsevariable}  = (\shiftedimpulseset{\impulsevariable}, \impulsegroupoperation)$ &  abelian group of impulse functions $\shiftedimpulseset{\impulsevariable}$ and group operation $\impulsegroupoperation$ \\
%			\hline
%			$\impulsespectralspacemult$ &  multiplication operator on a real number and an impulse function \\
%			\hline
%			$\impulsespectralspace{\impulsevariable} =  (\shiftedimpulseset{\impulsevariable}, \impulsegroupoperation, \impulsespectralspacemult) $& \makecell{\impulsespectralspacename{} with vector addition $\impulsegroupoperation$ \\ and scalar multiplication $\impulsespectralspacemult$ } \\
%			\hline
%			$\impulsespectraltrain{\impulseshift}{\impulsedegree}{\impulsevariable}$
%			&
%			\makecell{		\impulsespectraltrainname{} (\impulsespectraltrainshort): series of impulse functions \\ where each is impulse is multiplied an amplitude $\amplitude$}
%			\\
%			\hline
%			$\impulsespectraldensity{\impulseshift}{\impulsedegree}{\impulsevariable}$ & \impulsespectraldensityname{} (\impulsespectraldensitynameshort): convolution of $\impulseinterferencedegreetwo \in \naturalnumbers$ \impulsespectraltrainshort s
%			\\
%			\hline
%			$\impulseinterference{\impulseshift}{\impulsedegree}{\impulsevariable}$
%			&
%			\impulseinterferencename{} (\impulseinterferencenameshort): series of $\impulseinterferencedegreeone \in \naturalnumbers$ \impulsespectraldensitynameshort s
%			\\
%			\hline
%			$\amplitudematrix$, 
%			$\amplitudematrixentry{\impulseindex}{\impulseinterferenceindexone}{\impulseinterferenceindextwo}$
%			&
%			\amplitudematrixname,
%			\amplitudevectorname{} at matrix entry $(\impulseinterferenceindexone, \impulseinterferenceindextwo)$
%			\\
%			\hline 
%			$\shiftmatrix{\impulseindex}{\impulsevariable}$,
%			$\shiftmatrixentry{\impulseindex}{\impulsevariable}{\impulseinterferenceindexone}{\impulseinterferenceindextwo} $
%			&
%			\shiftmatrixname{}, 
%			\shiftvectorname{} at matrix entry $(\impulseinterferenceindexone, \impulseinterferenceindextwo)$
%			\\
%			\hline
%			$\amplitudematrixentry{\impulseindex}{\impulseinterferenceindexone}{\impulseinterferenceindextwo} \matrixdotproduct 
%			\shiftmatrixentry{\impulseindex}{\impulsevariable}{\impulseinterferenceindexone}{\impulseinterferenceindextwo} $
%			&
%			\dotproductname{} of an \amplitudevectorname{} and an \shiftvectorname{}
%			\\
%			\hline
%			$\amplitudematrix \matrixdotproduct  \shiftmatrix{\impulseindex}{\impulsevariable}$
%			&
%			\matrixdotproductname{} 
%			\\
%			\hline
%			$\impulsespectraldensityeinsteinmono{\amplitude}{\impulseindex}{\impulseshift}{\impulsevariable}$
%			&
%			\impulsespectraltrainshort{} in Einstein notation
%			\\
%			\hline
%			$\impulsespectraldensityeinsteinmulti{\amplitude}{\impulseindex}{\impulseshift}{\impulsevariable}$
%			&
%			\impulsespectraldensitynameshort{} in Einstein notation
%			\\
%			\hline
%			$\innerconvolution{}{}$
%			&
%			\makecell{\innerconvolutionname{} that convolves \\ the  columns of the two input matrices}
%			\\
%			\hline
%			$\innerconvolution{\amplitudematrix}{\shiftmatrix{\degreevector}{\impulsevariable} }$
%			&
%			\makecell{\impulseconvolutionvectorname{} resulting from the \\ \innerconvolutionname{} of the \amplitudematrixname{} \\ and the \shiftmatrixname{}}
%			\\
%			\hline 
%			$\impulsespectralinterferenceeinstein{\amplitudematrix \matrixdotproduct \shiftmatrix{\impulsedegree}{\impulsevariable}}$
%			&
%			\impulseinterferencenameshort{} as a result of the dot product $\innerconvolution{\amplitudematrix}{\shiftmatrix{\degreevector}{\impulsevariable} } \cdot \boldsymbol{1}$
%			\\
%			\hline
%			$	\amplitudevector{\impulseindex_{\impulseinterferenceindexone,1} \dots \impulseindex_{\impulseinterferenceindexone,\multiperiodicimpulsedimension}} \overline{| \langle \phasevectorcomponent{\impulseindex_{\impulseinterferenceindexone,1}} \dots \phasevectorcomponent{\impulseindex_{\impulseinterferenceindexone,\multiperiodicimpulsedimension}} \rangle  |}_{a,b}$
%			&
%			\makecell{	\heavisidedurationname{}: sum of amplitudes of \\ shifted impulse functions in the interval from a to b}
%			\\
%			\hline
%			$\heavisideduration{\Sh}{a,b}{} $
%			&
%			short-form of \heavisidedurationname
%			\\
%			\hline
%			\makecell{$\heavisidedurationupperupper{\Sh}{a, b}{}$, $\heavisidedurationupperlower{\Sh}{a, b}{}$ \\ 				$\heavisidedurationlowerupper{\Sh}{a, b}{}$}
%			&
%			\heavisidedurationname{} of the interval $[a,b]$,$[a,b)$,$(a,b]$
%			\\
%			\hline
%			$\heavisideduration{\Sh}{\intervalduration}{+} $, $\heavisideduration{\Sh}{\intervalduration}{-} $
%			&
%			\maximumheavisidedurationname{} and \minimumheavisidedurationname{}
%			\\ 
%			\hline
%			\makecell{$\heavisidedurationupperupper{\Sh}{\intervalduration}{+} $, 				$\heavisidedurationupperlower{\Sh}{\intervalduration}{+} $, 	\\			$\heavisidedurationlowerupper{\Sh}{\intervalduration}{+} $}
%			&
%			\makecell{\maximumheavisidedurationname{} of all intervals $[\impulsevariable, \impulsevariable +\intervalduration]$, \\ $[\impulsevariable, \impulsevariable +\intervalduration)$, $(\impulsevariable, \impulsevariable +\intervalduration]$, $\impulsevariable \in \realnumbers$}
%			\\
%			\hline
%			\makecell{$\heavisidedurationupperupper{\Sh}{\intervalduration}{-} $, $\heavisidedurationupperlower{\Sh}{\intervalduration}{-} $, \\ $\heavisidedurationlowerupper{\Sh}{\intervalduration}{-} $}
%			&
%			\makecell{\minimumheavisidedurationname{} of all intervals $[\impulsevariable, \impulsevariable +\intervalduration]$, \\ $[\impulsevariable, \impulsevariable +\intervalduration)$, $(\impulsevariable, \impulsevariable +\intervalduration]$ $\impulsevariable \in \realnumbers$}
%		\end{tabular}
%	\end{center}
%	\caption{List of symbols}
%	\label{table:listofsymbols}
%\end{table}
\renewcommand{\arraystretch}{1.3}
\renewcommand{\cellalign}{cl}
\begin{longtable}{| p{.27\textwidth} | p{.73\textwidth}  |} 	
	\hline
	Symbol & Meaning  \\ [0.5ex]
	\hline\hline
	$\naturalnumbers$, $\naturalnumberswithzero$,$\naturalnumbersuntil{n}$
	&
	\makecell{set of natural numbers, natural numbers with zero, \\ natural numbers $\{1,2,\dots n\}$}
	\\
	\hline
	$\realnumbers, \realnumbersvector{n}$
	&
	set of real numbers, $n$-dimensional vectors
	\\
	\hline
	\impulseshift 
	&
	shift
	\\
	\hline
	$\amplitude $
	&
	amplitude
	\\
	\hline
	$\impulsedegree $
	&
	degree
	\\
	\hline
	$\shiftedimpulse{\impulseindex}{\impulsevariable}$ 
	&
	\makecell{Dirac impulse function $\diracdelta(\impulsevariable - \impulseindex s)$ of variable $\impulsevariable$ \\ shifted by $\impulseindex \cdot s$}
	\\ 
	\hline
	$\shiftedimpulseset{\impulsevariable}$ & set of all shifted Dirac impulse functions of variable $\impulsevariable$  \\
	\hline
	$\impulsegroupoperation$ &  addition operator on two impulse functions \\
	\hline
	$\impulsegroup{\impulsevariable}  = (\shiftedimpulseset{\impulsevariable}, \impulsegroupoperation)$ &  abelian group of impulse functions $\shiftedimpulseset{\impulsevariable}$ and group operation $\impulsegroupoperation$ \\
	\hline
	$\impulsespectralspacemult$ &  multiplication operator on a real number and an impulse function \\
	\hline
	$\impulsespectralspace{\impulsevariable} =  (\shiftedimpulseset{\impulsevariable}, \impulsegroupoperation, \impulsespectralspacemult) $& \makecell{\impulsespectralspacename{} with vector addition $\impulsegroupoperation$ \\ and scalar multiplication $\impulsespectralspacemult$ } \\
	\hline
	$\impulsespectraltrain{\impulseshift}{\impulsedegree}{\impulsevariable}$
	&
	\impulsespectraltrainname{} (\impulsespectraltrainshort): series of impulse functions where each is impulse is multiplied an amplitude $\amplitude$
	\\
	\hline
	$\impulsespectraldensity{\impulseshift}{\impulsedegree}{\impulsevariable}$ & \impulsespectraldensityname{} (\impulsespectraldensitynameshort): convolution of $\impulseinterferencedegreetwo \in \naturalnumbers$ \impulsespectraltrainshort s
	\\
	\hline
	$\impulseinterference{\impulseshift}{\impulsedegree}{\impulsevariable}$
	&
	\impulseinterferencename{} (\impulseinterferencenameshort): series of $\impulseinterferencedegreeone \in \naturalnumbers$ \impulsespectraldensitynameshort s
	\\
	\hline
	$\amplitudematrix$, 
	$\amplitudematrixentry{\impulseindex}{\impulseinterferenceindexone}{\impulseinterferenceindextwo}$
	&
	\amplitudematrixname,
	\amplitudevectorname{} at matrix entry $(\impulseinterferenceindexone, \impulseinterferenceindextwo)$
	\\
	\hline 
	$\shiftmatrix{\impulseindex}{\impulsevariable}$,
	$\shiftmatrixentry{\impulseindex}{\impulsevariable}{\impulseinterferenceindexone}{\impulseinterferenceindextwo} $
	&
	\shiftmatrixname{}, 
	\shiftvectorname{} at matrix entry $(\impulseinterferenceindexone, \impulseinterferenceindextwo)$
	\\
	\hline
	$\amplitudematrixentry{\impulseindex}{\impulseinterferenceindexone}{\impulseinterferenceindextwo} \matrixdotproduct 
	\shiftmatrixentry{\impulseindex}{\impulsevariable}{\impulseinterferenceindexone}{\impulseinterferenceindextwo} $
	&
	\dotproductname{} of an \amplitudevectorname{} and an \shiftvectorname{}
	\\
	\hline
	$\amplitudematrix \matrixdotproduct  \shiftmatrix{\impulseindex}{\impulsevariable}$
	&
	\matrixdotproductname{} 
	\\
	\hline
	$\impulsespectraldensityeinsteinmono{\amplitude}{\impulseindex}{\impulseshift}{\impulsevariable}$
	&
	\impulsespectraltrainshort{} in Einstein notation
	\\
	\hline
	$\impulsespectraldensityeinsteinmulti{\amplitude}{\impulseindex}{\impulseshift}{\impulsevariable}$
	&
	\impulsespectraldensitynameshort{} in Einstein notation
	\\
	\hline
	$\innerconvolution{}{}$
	&
	\innerconvolutionname{} that convolves the  columns of the two input matrices
	\\
	\hline
	$\innerconvolution{\amplitudematrix}{\shiftmatrix{\degreevector}{\impulsevariable} }$
	&
	\impulseconvolutionvectorname{} resulting from the \innerconvolutionname{} of the \amplitudematrixname{} and the \shiftmatrixname{}
	\\
	\hline 
	$\impulsespectralinterferenceeinstein{\amplitudematrix \matrixdotproduct \shiftmatrix{\impulsedegree}{\impulsevariable}}$
	&
	\impulseinterferencenameshort{} as a result of the dot product $\innerconvolution{\amplitudematrix}{\shiftmatrix{\degreevector}{\impulsevariable} } \cdot \boldsymbol{1}$
	\\
	\hline
	$	\amplitudevector{\impulseindex_{\impulseinterferenceindexone,1} \dots \impulseindex_{\impulseinterferenceindexone,\multiperiodicimpulsedimension}} \overline{| \langle \phasevectorcomponent{\impulseindex_{\impulseinterferenceindexone,1}} \dots \phasevectorcomponent{\impulseindex_{\impulseinterferenceindexone,\multiperiodicimpulsedimension}} \rangle  |}_{a,b}$
	&
	\heavisidedurationname{}: sum of amplitudes of shifted impulse functions in the interval from a to b
	\\
	\hline
	$\heavisideduration{\Sh}{a,b}{} $
	&
	short-form of \heavisidedurationname
	\\
	\hline
	$\heavisidedurationupperupper{\Sh}{a, b}{}$, $\heavisidedurationupperlower{\Sh}{a, b}{}$, $\heavisidedurationlowerupper{\Sh}{a, b}{}$
	&
	\heavisidedurationname{} of the interval $[a,b]$,$[a,b)$,$(a,b]$
	\\
	\hline
	$\heavisideduration{\Sh}{\intervalduration}{+} $, $\heavisideduration{\Sh}{\intervalduration}{-} $
	&
	\maximumheavisidedurationname{} and \minimumheavisidedurationname{}
	\\ 
	\hline
	$\heavisidedurationupperupper{\Sh}{\intervalduration}{+} $, 				$\heavisidedurationupperlower{\Sh}{\intervalduration}{+} $, 	$\heavisidedurationlowerupper{\Sh}{\intervalduration}{+} $
	&
	\maximumheavisidedurationname{} of all intervals $[\impulsevariable, \impulsevariable +\intervalduration]$, $[\impulsevariable, \impulsevariable +\intervalduration)$, $(\impulsevariable, \impulsevariable +\intervalduration]$, $\impulsevariable \in \realnumbers$
	\\
	\hline
	$\heavisidedurationupperupper{\Sh}{\intervalduration}{-} $, $\heavisidedurationupperlower{\Sh}{\intervalduration}{-} $,  $\heavisidedurationlowerupper{\Sh}{\intervalduration}{-} $
	&
	\minimumheavisidedurationname{} of all intervals $[\impulsevariable, \impulsevariable +\intervalduration]$, $[\impulsevariable, \impulsevariable +\intervalduration)$, $(\impulsevariable, \impulsevariable +\intervalduration]$ $\impulsevariable \in \realnumbers$    \label{table:listofsymbols}
	\\ \hline \caption{List of symbols}
\end{longtable}