\documentclass[aps,prd,reprint,nofootinbib,twocolumn]{revtex4-1}
\pdfoutput=1
\usepackage[english]{babel}
\usepackage{amsmath}
\usepackage{amsfonts}
\usepackage{graphicx}
\usepackage[colorlinks=true, allcolors=blue]{hyperref}
\usepackage{cancel}
\usepackage[export]{adjustbox}



\newtheorem{theorem}{Theorem}[section]
\newtheorem{definition}{Definition}[section]
\newtheorem{lemma}{Lemma}[section]
\newtheorem{corollary}{Corollary}[section]
\newtheorem{proposition}{Proposition}[section]
\newtheorem{conjecture}{Conjecture}[section]
\newtheorem{remark}{Remark}[section]
\newtheorem{example}{Example}[section]
\newtheorem{problem}{Problem}[section]
\newtheorem{assumption}{Assumption}[section]
\newtheorem{conditions}{Conditions}[section]
\newtheorem{property}{Property}[section]


\def\dr#1{\textcolor{red}{DR: #1}}
\def\yl#1{\textcolor{blue}{YL: #1}}


\newcommand{\VSI}{Van Swinderen Institute for Particle Physics and Gravity,\\ University of Groningen,
Nijenborgh 4, 9747 AG Groningen, The Netherlands}

\newcommand{\mac}{Department of Physics and Astronomy,
Macalester College,
Saint Paul MN 55105-1899, U.S.A.}

\begin{document}
\title{Hybrid Goldstone Modes from the Double Copy Bootstrap}
%\title{Extended Goldstone theories from the Double Copy}
\author{Yang Li}
\author{Diederik Roest}
\affiliation{\VSI}
\author{Tonnis ter Veldhuis }
\affiliation{\mac}

\begin{abstract}
\noindent
We perform a systematic classification of scalar field theories whose amplitudes admit a double copy formulation and identify two building blocks at 4-point and 13 at 5-point. Using the 4-point blocks as bootstrap seeds, this naturally leads to a single copy theory that is a gauged NLSM. Moreover, its double copy includes a novel theory that can be written in terms of Lovelock invariants of an induced metric, and includes Dirac-Born-Infeld and the special Galileon in specific limits. The amplitudes of these Goldstone modes have two distinct soft behaviour regimes, corresponding to a hybrid of non-linear symmetries.
\end{abstract}

\maketitle

\section{Introduction}

\noindent
The double copy framework manifests a remarkable connection between the unique (at lowest order in derivatives) interacting theories of spin-1 and spin-2: the amplitudes of GR can be written as the squares of specific color-dual kinematic factors that define Yang-Mills \cite{Bern:2008qj, Bern:2010ue}. This connection has its origin in open-closed string duality \cite{Kawai:1985xq} and is closely related to the scattering equations approach of  \cite{Cachazo:2013gna,Cachazo:2013hca}. The double copy has since been extended to include supersymmetric theories and loop level, as reviewed in \cite{Bern:2019prr,Bern:2022wqg}. Moreover, scalar field theories with enhanced soft limits \cite{Cheung:2014dqa} (that generalise the Adler zero and hence can be seen as Goldstone theories) have also been formulated in the double copy framework, see e.g.~\cite{Cheung:2017pzi}. 

A natural question regards the uniqueness of the double copy factors: are higher-derivative corrections encoded in other factors? For the color-dual factors of YM, there is a single additional possibility at 3-point (while at 4-point, there are already 8 different tensorial structures \cite{Bern:2017tuc, Carrasco:2019yyn}) that generates the unique $F^3$ correction. Using this as a seed interaction, double copy compatibility at 4-point then implies the inclusion of a $F^4$ term. Moreover, following the same logic at 5-point requires the further quartic term $D^2 F^4$ \cite{Carrasco:2022lbm}. It was conjectured to go up to all derivatives, leading to a UV complete series that is part of the bosonic open string amplitude \cite{Carrasco:2022lbm,Elvang:2018dco}.

A related result was found very recently for higher-derivatives to an specific scalar field theory, the NLSM \cite{Brown:2023srz}. Again, higher-derivative corrections to the 4-point seed interactions were found to be constrained by higher-point consistency. The only known theory that satisfies these constraints at all order (apart from the NLSM itself) is Z-theory, again with an infinite tower of derivatives\footnote{Similar infinite series were also found in generalisations of the KLT kernel of the bi-adjoint scalar theory in \cite{Chen:2023dcx}; it would be interesting to investigate how these relate to the current results.} \cite{Carrasco:2016ldy}.

In this letter, we perform a related analysis for scalar field theories with Goldstone modes. The crucial difference with \cite{Brown:2023srz} is that we do not restrict ourselves to 4-point contact interactions. As we will show, this allows for a unique additional exchange interaction. Similarly, we classify all possible double copy scalar seeds at $n=5$ and find 29 independent structures, of which generate 13 physical amplitudes. 

Using a linear combination of the two 4-point seed interactions, we then employ the bootstrap procedure to construct theories for Goldstone modes with a hybrid character: while the entire theory has a particular soft degree $\sigma_{\rm min}$, it contains a subsector that is defined by having $\sigma_{\rm max} = \sigma_{\rm min} +1$ instead. We present examples of a single and a double copy: a gauged version of the NLSM with $\sigma_{\rm min} = 0$ and a particular higher-derivative extension of DBI with $\sigma_{\rm min} = 2$. As the latter is formulated in terms of Lovelock invariants, we refer to it as DBI-Lovelock. 

In contrast to the results of \cite{Carrasco:2022lbm, Brown:2023srz}, we find no need for infinite sets of quartic higher-derivative corrections; in this sense, our results are more akin to the extended DBI theory \cite{Cachazo:2014xea, Low:2020ubn, Kampf:2021tbk}. We  provide an interpretation for this difference in the conclusion, and outline further implications and generalisations.

\section{BCJ Representations}

\noindent
We will start with a short recap of the double copy approach towards amplitudes (see e.g.~\cite{Bern:2019prr,Bern:2022wqg} for more extensive reviews). Throughout we will highlight the role of the permutation group $S_n$ and phrase the DC in terms of representation requirements. 

The double copy or BCJ approach \cite{Bern:2008qj,Bern:2010ue} has identified a number of field theories, famously including GR and YM, whose amplitudes can be rewritten in terms of a sum over trivalent diagrams:
 \begin{align}
     A_n = \sum_{\text{trivalent}}\frac{N\tilde{N}}{D} \,.
     \label{double copy amplitude}
 \end{align}
In total there are $(2n-5)!!$ inequivalent such diagrams at multiplicity $n$. These always include a number (equal to $\tfrac18 n!$ for $n \geq 4$) of half-ladder diagrams, which can be seen as the backbone of the double copy approach, while from $n \geq 6$ onwards there are also additional topologies; for instance, at six-point there are also snowflake diagrams. 
 
Turning to the summand in \eqref{double copy amplitude}, the denonimator in the above sum consists of the propagators for every diagram. The numerator instead is the product of two so-called BCJ factors that encode the characteristics of the particles in the scattering process. 

An important and arguably the simplest example is given by the colour factors that consist of products of structure constants $f_{abc}$. At multiplicity $n$, these are given by a product of $n-4$ structure constants:
 \begin{align}
  N_{abc \ldots} = f_{ab}{}^{x_1} f_{x_1c}{}^{x_2} f_{x_2 \ldots} \ldots \,. \label{structure-constants}
\end{align}
When viewed as a representation of the permutation group $S_n$, the above factor satisfies the nested commutator structure \cite{deNeeling:2022tsu}
\begin{equation}
-N_{abcd\dots}=N_{bacd\dots}=N_{c[ab]d\dots}=N_{d[[ab]c]\dots}=\ldots
\label{Jacobi id}
\end{equation}
At four-point this is simply the Jacobi identity for the group structure; we will refer to the above as generalised Jacobi identity.  Moreover, the colour factors are even or odd under reflection,
\begin{equation}
N_{abcd\dots}=(-)^nN_{\dots dcba} \,.
\label{reflection inva}
\end{equation}
We have identified which irreps of $S_n$ the above constraints correspond to, as listed in table~\ref{table:BCJ irrep} (see Appendix A for details). Note that the dimensions of these irreps adds up to $(n-2)!$ in every case; in other words, allowing for an arbitrary tensor $N_{abc..}$ subject to these constraints leaves $(n-2)!$ different components in these specific irreps.

\begin{table}
\centering
\begin{tabular}{||c | |c|c||} 
 \hline
$n$ & BCJ factors \\ [0.5ex] 
 \hline \hline
 4 &  $[2,2]$  \\ 
 \hline
 5 &  $[3,1,1]$ \\
 \hline
 6 & $[4,2]$, $[3,1,1,1]$, $[2,2,2]$ \\
 \hline
 7  & $[5, 1, 1]$, $[4, 2, 1]$, $[3, 3, 1]$, $[3, 2, 1, 1]$, $[2, 2, 1, 1, 1]$ \\
 \hline
 8 &  \begin{tabular}{@{}c@{}} $[6,2]$, $[5,2,1]$, $[5,1,1,1]$, $[4,4]$, $[4,3,1]$, $2\times [4,2,2]$,  \\ $[4,2,1,1]$, $[4,1,1,1,1]$, $2\times[3,3,2,2]$,  \\ $[3,2,2,1]$, $[3,2,1,1,1]$, $[2,2,2,2]$, $[2,2,1,1,1,1]$\end{tabular}\\
 \hline
\end{tabular}
    \caption{\it BCJ-compatible  irreps of $S_n$.}
\label{table:BCJ irrep}
\end{table}

Instead of structure constants, we will be interested in BCJ factors that only contain Mandelstam variables. These are relevant for scalar field theories: in single scalar field theories, the particles only carry momentum information and thus Mandelstam invariants. Moreover, in multi-scalar field theories such as the NLSM with higher-derivative corrections, the colour information factorises and thus one of the two BCJ factors again only involves Mandelstam variables. 

The possibilities can be phrased in terms of $S_n$ representations. The set of $n$ external momenta forms the so-called standard irrep $[n-1,1]$ with dimension $n-1$ (the sum of all momenta spans the singlet $[n]$ and vanishes due to momentum conservation). Lorentz invariants then consist of inner products of momenta and live in the irrep $[n-2,2]$ with dimension $\tfrac12 n (n-3)$; these correspond to the Mandelstam invariants. Note that the symmetric product of the standard irrep leads to two additional irreps, but again these vanish due to momentum conservation and the on-shell nature of external particles. Moreover, we work in general dimensions and hence are not affected by Gram determinant considerations that reduce the number of independent Mandelstam variables.

The above approach reduces the classification of scalar BCJ factors to a representation theory problem\footnote{It can also be phrased as a representation theory of the cyclic group instead, with all cyclic invariants generating an overcomplete basis of BCJ factors \cite{Bonnefoy:2021qgu}.} of $S_n$: to know how many factors there are at a given multiplicity $n$ and at a given order $p$ in Mandelstam variables, one simply calculates the symmetric product of $p$ irreps $[n-2,2]$ and decomposes this into $S_n$ irreps. A comparison with the BCJ-required irreps of table \ref{table:BCJ irrep} then directly gives the number of possible scalar BCJ factors at this order. 

\begin{table}
\centering
\begin{tabular}{||c | |c|c||} 
 \hline
$n$ & BCJ gauge parameters \\ [0.5ex] 
 \hline \hline
 4 &  $[1,1,1,1]$  \\ 
 \hline
 5 &  $[2,2,1]$ \\
 \hline
 6 &  $[3, 2, 1]$, $[3, 1 ,1 ,1]$\\
 \hline
 7  &  \begin{tabular}{@{}c@{}} $[4, 3]$, $[4, 2, 1]$, $[4, 1, 1, 1]$, $[3, 2, 2]$, $[3, 2, 1, 1]$,\\ $[3, 1, 1, 1, 1]$, $[2, 2, 2, 1]$ \end{tabular}\\
 \hline
\end{tabular}
    \caption{\it BCJ-compatible gauge parameter irreps of $S_n$. }
\label{table:BCJ gauge}
\end{table}

Not all BCJ factors will contribute to the amplitude; some solutions $N$ will give a vanishing contribution to \eqref{double copy amplitude}, independent of the choice for $\tilde N$. Interestingly, these gauge solutions can also be characterised by representation theory: all BCJ factors that can be written as the product of Mandelstam variables with a specific $S_n$ irrep drop out of the amplitude. The first example surfaces at 4pt and reads
 \begin{align}
 N_{abcd} = s_{ab} G_{abcd} \,, \label{gauge-4pt}
 \end{align}
where the convention $s_{i\dots j}=(p_i+\cdots + p_j)^2$ is adopted, and $G$ is a fully anti-symmetric tensor and hence lives in the $[1,1,1,1]$. We have listed the analogous irrep requirements at higher multiplicies in table \ref{table:BCJ gauge}.

\section{BCJ seed classification}

\noindent
We will now proceed to construct BCJ factors at lower multiplicities. Remarkably, representation theory allows for a systematic classification to all orders in Mandelstam variables\footnote{At three-point it is impossible to construct BCJ factors for scalars only, as there are no Mandelstam invariants; relatedly, three-point operators with derivatives can always be reformulated as four-point operators and hence give vanishing three-point amplitudes.} at 4- and 5-points. 
 
At {\bf four-point}, the required BCJ irrep is the window of $S_4$ with dimension $2$. The Mandelstam invariants in this case actually live in the same irrep. Therefore there is naturally a linear combination of Mandelstam invariants that satisfies the BCJ constraints. An explicit construction shows that it is given by 
 \begin{align}
    N_4^{(1)}=s_{bc} - s_{ac} \,.
    \label{4pt_win_1}
 \end{align}
Moreover, at quadratic order in Mandelstam, the symmetric product of two window irreps decomposes into $[2,2] + [4] + [1,1,1,1]$  and hence there is another BCJ factor for four-point at this order. It takes the form
 \begin{align}
      N_4^{(2)}=s_{ab} (s_{bc} - s_{ac}) \,.
     \label{4pt_win_2}
 \end{align}
Note that it has the same structure as the previous one, multiplied by the propagator of the corresponding half-ladder diagram. We will see similar structures at higher multiplicities later on. We will refer to the linear and quadratic solutions as exchange and contact factors, respectively.

At higher orders, there are new solutions to the generalised Jacobi. However, these are always of the form of one of the two above building blocks, multiplied by Mandelstam expressions that are separately invariant (and hence can be used to construct additional solutions to the BCJ conditions). The number of invariants at every order is given by the Taylor coefficients of the Molien series\footnote{Note that this coincides with the Hilbert series for the case of invariant polynomial rings.} \cite{benson_1993,derksen2015computational,Henning:2017fpj}
 \begin{align}
    H_4^{\rm Inv}(x)=\frac{1}{(1-x^2)(1-x^3)} \,.
    \label{4-pt Hilbert series}
 \end{align}
This amounts to the statement that all invariants can be written as arbitrary powers of two primary invariants:
 \begin{align}
   I_4^{(2)} =s_{ab} s_{bc} + s_{ac}s_{bc}+s_{ab}s_{ac} \,, \quad  I_4^{(3)} = s_{ab} s_{ad} s_{ac} \,.
   \label{4pt invariant Y and X}
 \end{align}
Moreover, the number of window irreps at every order in Mandelstam is generated by 
 \begin{align}
   H_4^{\rm BCJ}(x) = (x+x^2) H^{\rm Inv}_4(x) \,.
 \end{align}
All window solutions are therefore either \eqref{4pt_win_1} or \eqref{4pt_win_2} multiplied by an invariant, as also found in \cite{Carrasco:2019yyn}.

Turning to gauge parameters, the Molien series for the relevant irrep $[1,1,1,1]$ is given by
 \begin{align}
   H_4^{\rm Gauge}(x) = x^3 H^{\rm Inv}_4(x) \,,
 \end{align}
which generates the number of gauge parameters at one order higher. Indeed, it turns out that the combination
 \begin{align}
  2N_4^{(2)} I_4^{(2)} - 3 N_4^{(1)} I_4^{(3)} \,,
  \label{4pt gauge factor}
 \end{align}
is of the form \eqref{gauge-4pt} and drops out of the amplitude \eqref{double copy amplitude} for any BCJ factor $\tilde N$. The number of physical BCJ parameters is therefore given by
 \begin{align}
     H_4^{\rm BCJ}(x) - x H_4^{\rm Gauge}(x) = \frac{x}{1-x^2} + \frac{x^2}{(1-x^2)(1-x^3)} \,,
 \end{align}
generated by the linear or quadratic seed solutions \eqref{4pt_win_1} and \eqref{4pt_win_2} multiplied by quadratic and/or cubic invariants\footnote{This general solution includes the 4-point Abelian Z-theory \cite{Carrasco:2016ldy} for a specific tuning of its coefficients, as suggested by \cite{Elvang:2018dco,CarrilloGonzalez:2019fzc,Brown:2023srz}.}.

At {\bf five-point}, the story is similar but more complicated. The 5-point Hilbert series for invariants is given by 
\begin{equation}
   H_5^{\rm Inv} (x) = (1+x^6+x^7+x^8+x^9+x^{15}) / D_5(x) \,,
\end{equation}
with the denominator given by
 \begin{align}
     D_5(x) = (1-x^2)(1-x^3)(1-x^4)(1-x^5)(1-x^6) \,.
 \end{align}
Each factor in the denominator corresponds to a primary invariant, and each term in the numerator to a secondary invariant; for example, $(1-x^2)$ represents the contribution from a quadratic primary invariant, whereas $x^6$ represents a sextic secondary invaraint. The difference is that primary invariants can appear at any power to form new invariants, while there can only be a single secondary invariant. The latter restriction is due to relations between products of invariants, referred to as syzygies \cite{benson_1993,derksen2015computational}.

The BCJ irreps, instead, are given by $[3,1,1]$ corresponding to the ``hook'' Young tableau. The Molien series for this is\footnote{These numbers were found up to order 12 in \cite{Carrasco:2021ptp}, which also includes explicit expressions for the factors. The five maps \textcircled{$\mathfrak{a}$}, \textcircled{$\mathfrak{r}$}, \textcircled{$\mathfrak{p}$}, \textcircled{$\mathfrak{h}$}, \textcircled{$\mathfrak{s}$} of \cite{Carrasco:2021ptp} are polynomial realizations of tensor product plus projection onto a certain representation.}
 \begin{align}
   H_5^{\rm BCJ}(x) = ( & x^3 + 2 x^4 + 4 x^5 + 5 x^6 + 6 x^7 + 6 x^8 + 5 x^9 \nonumber\\
   & + 4 x^{10} + 2 x^{11} + x^{12}) / D_5(x) \,,
 \end{align}
and are thus given by the numerator structures multiplied by primary invariants. The former can contain secondary primaries, however; we have checked that this is the case for 7 of these irreps. We therefore conclude that there are 29 independent hooks (of powers from three to ten in Mandelstam) that generate all the solutions of Jacobi identity at 5-pt; all other hooks are products of the independent hooks with (primary and secondary) invariants. 

The gauge parameters in this case are generated by the irrep $[2,2,1]$; the number of such irreps at every order is generated by 
 \begin{align}
     H_5^{\rm Gauge}= (  &x^2 + x^3 + 3 x^4 + 3 x^5 + 3 x^6 + 4 x^7 + 4 x^8 \nonumber\\
     & + 3 x^9 + 3 x^{10} + 3 x^{11} + x^{12} + x^{13} ) / D_5(x) \,.
 \end{align}
However, in this case the number of distinct resulting gauge parameters (at one order higher) is somewhat smaller and given by 
\begin{align}
    &(x^3 + x^4 + 2 x^5 + 3 x^6 + 2 x^7 + 3 x^8 + 4 x^9 + 3 x^{10} \nonumber\\ 
 & + 2 x^{11}+2 x^{12} + x^{13}) / D_5(x) \,.
\end{align}
This difference comes about as some BCJ parameters can be split into Mandelstam times $[2,2,1]$ in multiple ways. The resulting number of physical BCJ parameters at every order can then be written as a sum of fractions with positive coefficients,
 \begin{align}
    H_5^{\rm Phys}(x)= &
    ( x^4+2 x^5+2 x^6+4 x^7+3 x^8) / D_5(x)  \nonumber\\
    &+ \frac{(x^9+x^{10})}{(1-x^2)(1-x^4)(1-x^5)(1-x^6)} \,,
\end{align}
but this decomposition is not unique. Modulo the primary invariants of the denominators, this series consists out of 14 different hook structures. However, one of these can be written in terms of a secondary invariant, implying that there are 13 independent hook structures that can be used as 5-point seed interactions. 

\section{BCJ bootstrap}

\noindent 
From {\bf six-point} on, a systematic classification of BCJ factors becomes more complicated. The 6-point Molien series for invariants is 
\begin{align}
    H_6^{\rm Inv}(x) =&  (1 + 2 x^5 + 5 x^6 + 7 x^7 + 9 x^8 + 11 x^9 + 13 x^{10} \nonumber \\
    & + 14 x^{11}+ 21 x^{12} + 24 x^{13} + 28 x^{14} + 32 x^{15} \nonumber \\
    & + 26 x^{16} + 22 x^{17} +  13 x^{18} + 7 x^{19} + 3 x^{20} \nonumber \\
    & + x^{21} + x^{22})/ D_6(x) \,,
 \end{align}
 in terms of the following denominator containing the primary invariants:
  \begin{align}
    D_6(x)=& (1 - x^2) (1 - x^3)^2 (1 - x^4)^3 (1 - x^5)^2 (1 - x^6) \,. 
\end{align}
Thus there are nine primary invariants and 239 secondary ones. However, deriving the explicit forms of these primaries and secondaries is problematic due to complicated relations between invariants (with syzygies of syzygies~\cite{benson_1993,Hilbert1890berDT}). 

The three BCJ irreps have the following joint Molien series:
\begin{align}
    H_6^{\rm BCJ}(x) =& (x + 3 x^2 + 8 x^3 + 18 x^4 + 36 x^5 + 71 x^6 \nonumber \\
    & + 122 x^7 + 197 x^8 + 292 x^9 + 394 x^{10} \nonumber \\
    &+ 498 x^{11} + 581 x^{12} + 630 x^{13} + 635 x^{14} \nonumber \\
    & + 596 x^{15} + 518 x^{16} + 415 x^{17} + 309 x^{18} \nonumber \\
    & + 206 x^{19} + 123 x^{20} + 66 x^{21} \nonumber \\
    & + 28 x^{22} + 10 x^{23} + 3 x^{24}) / D_6(x) \,. 
\end{align}
However, not all of these are independent, but instead can be written in terms of secondary invariants; to determine this, one would need to construct both the secondary invariants and the BJC irreps to these orders. We will not attempt such a general classification, and only list the number of physical and gauge parameters at lowest orders in table~\ref{table:6-point data}.


\begin{table}
\centering
\begin{tabular}{||c | |c | c | c| | c| | c ||} 
 \hline
$p$ & ${\rm BCJ}$ & ${\rm Phys}$ & ${\rm Gauge}$ & ${\rm Inv}$ & Bootstrap \\ [0.5ex] 
 \hline\hline
 1 &  1 &  1 & 0 & 0 & 0\\
 \hline
 2 &  3 &  3  & 0 & 1 & 1\\
 \hline
 3 &  9 &  8 & 1 &  2 & 2 \\
 \hline
 4 &  23 & 18 &  5 &  4  &  3\\
 \hline
 5 &  54 & 38 & 16 & 6 &  8\\
 \hline
 6 &  121 & 79 & 42 &  13 &  24\\
 \hline
 7 &  246 & 151 & 95 & 19 & 53 \\
 \hline
\end{tabular}
    \caption{\it The number of 6-point BCJ-compatible factors (split into physical and gauge parameters) and invariants to $\mathcal{O}(s^7)$. The last column lists the number of factors that are compatible with the BCJ bootstrap.}
\label{table:6-point data}
\end{table}

Instead, we will focus on the subset of 6pt interactions that follow from 4pt seeds; in other words, we will require that they are BCJ bootstrappable from 4-point seed interactions, similar to \cite{Brown:2023srz}. This implies in particular that at singular channels such as $s_{abc}\rightarrow 0$, the amplitude should factorize into two 4-point amplitudes. In turn, this implies that a BCJ factor of ${\cal O}(s^p)$ factorizes as 
\begin{align}
    \lim_{s_{abc}\rightarrow 0}N_6^{(p)} = \sum_q c_q N_4^{(p-q)}(abcx) N_4^{(p+q)}(xdef)  \,,
\end{align}
where $x$ denotes the internal leg. As the lowest 4-point factor is linear in Mandelstam, the first BCJ factor that can be bootstrapped is quadratic. Beyond that, we find up to quartic:
\begin{align}
    \lim_{s_{abc}\rightarrow 0}N_6^{(2)}&=(s_{ac}-s_{bc}) (s_{de}-s_{df}) \,, \notag \\
    \lim_{s_{abc}\rightarrow 0}N_6^{(3)}&=(s_{ac}-s_{bc})  s_{ab} (s_{de}-s_{df})\nonumber\\
    &\hspace{7mm}+(s_{ac}-s_{bc})  (s_{de}-s_{df})s_{ef} \,, \notag \\
    \lim_{s_{abc}\rightarrow 0}N_6^{(4)}&=(s_{ac}-s_{bc})s_{ab}  (s_{de}-s_{df})s_{ef} \,.
\end{align}
Imposing this BCJ bootstrap fixes $N_6^{(2)}$ uniquely, while in other cases it still leaves some free parameters which can be seen as contact interactions that are separately BCJ compatible. We provide an overview of these numbers in table~\ref{table:6-point data}.

Note that the quartic 6-point expression can be interpreted as a product of two quadratic 4-point ones, while this could have also contained the product of two linear 4-points ones with a single quadratic invariant.  We have verified that this behaviour extends up to septic: $N_6^{(2)}$ up to $N_6^{(6)}$ only decompose into $N_4^{(1)} (I_4^{(3)})^n$ and $N_4^{(2)} I$, where $I$ is an invariant of arbitrary power. It would be interesting to understand why this is the case, and whether it extends to higher order.


\section{Single copy: the gauged NLSM}
\label{sec:gauged NLSM}

\noindent
The systematic classification of BCJ seed factors at 4-point and the corresponding bootstrapped ones at 6- and higher-point allows for the construction of novel theories that feature interactions with different soft limits. As a first illustration, we will propose a single copy theory for an adjoint Goldstone scalar field, with amplitudes generated by the product of a colour factor with a linear combination of the different elementary solutions involving Mandelstam. Moreover, we will use the requirement of $\sigma_{\rm min} =0$ as a guiding principle. This will result in a gauged version of the chiral NLSM with symmetry breaking $G \times G \rightarrow G$, with additional interactions due to gluon exchange.

At four-point this theory is generated by $C_4 \times (N_4^{(1)} + N_4^{(2)})$, and therefore has two different contributions to the amplitudes. The corresponding Lagrangian is 
 \begin{align}
     \mathcal{L}_4 = - \tfrac12 (D \phi)^2 + \tfrac16 f^2 \phi^2 (D \phi)^2 - \tfrac14 F^2 \,,
 \end{align}
up to this order. Moving to six-order, we consider the sketchy form $C_6 \times (N_6^{(2)} + N_6^{(3)} + N_6^{(4)})$. While the quadratic BCJ factor is unique, that is not the case for the cubic and quartic ones. To unambiguously determine the theory, we impose the soft limit $\sigma_{\rm min} = 0$ at cubic and $\sigma_{\rm max} = 1$ at quartic order; as promised in the introduction, this theory has different soft degrees with $\sigma_{\rm max} = \sigma_{\rm min} +1$. In this way, we obtain a unique BCJ factor for each order, with e.g.
 \begin{align}
     N_6^{(2)} =& \left(s_{ac}-s_{bc}\right) \left(s_{de}-s_{df}\right)\nonumber\\
     &+\tfrac{1}{2} s_{abc} \left(s_{ae}-s_{af}-s_{be}+s_{bf}\right) \,.
     \end{align}
In order to generate these amplitudes, we have to extend the Lagrangian of the gauged NLSM with the following (somewhat sketchily) terms\footnote{This gauged NLSM was also considered in \cite{Carrasco:2022sck}, whose analysis also includes gluonic external states and concluded that $D^2 F^4$ are needed. We only consider pion scattering instead.}:
 \begin{align}
    \mathcal{L}_6 = \mathcal{L}_4 + \tfrac{1}{45} f^4 \phi^4 (D \phi)^2 - 2f^2 F^3 + \tfrac{1}{6}f^2\phi^2 F^2 \,.
 \end{align}
Moving to higher multiplicities, we conjecture that this pattern continues. For instance, at 8-point, one can take the amplitude generated by BCJ factors of the form $C_8 \times (N_8^{(3)} + .. + N_8^{(6)})$. The corresponding Lagrangian will include all terms above, and possibly new ones such as $F^4$ and $F^2\phi^4$ which contribute to the amplitude corresponding to the second highest BCJ numerator $N_8^{(5)}$.
 
Note that all terms are gauge covariant, and will thus result in $\sigma_{\rm min} =0$ amplitudes. Moreover, the amplitude with most derivatives, which is the two-derivative part, reduces to the NLSM and thus has $\sigma_{\rm max} = 1$. Thus one should think of this theory as the NLSM with subleading terms included. These are dictated by a combination of non-linear symmetries (e.g.~the structure of two-derivative terms), gauge invariance (i.e.~the covariant derivatives) and BCJ consistency (e.g.~the $F^3$ and $F^2 \phi^2$).
 
\section{Double copy: DBI-Lovelock}
\label{sec:extended Goldstone}

\noindent
Instead of pursuing the gauged NLSM to higher multiplicities, we turn to a specific double copy theory that involves a single scalar field and that is fully determined by two different non-linear symmetries, with BCJ compatibility arising as a result. 

We again start from the full classification at 4-point, where the independent BCJ factors for a single scalar field are linear and/or quadratic in Mandelstam. The combination of linear with quadratic yields DBI with quartic operator $(\partial \phi)^4$. In order to retain the $\sigma_{\rm min} = 2$ generalised Adler zero, one must add $(\partial \phi)^{2n}$ with specific coefficients at every order \cite{Cheung:2014dqa}. The full DBI theory is then given by the measure of the metric
 \begin{align}
    g_{\mu\nu} = \eta_{\mu\nu} + \partial_\mu \phi \partial_\nu \phi \,,
    \label{special metric}
 \end{align}
which can be seen as a brane-induced metric and is covariant under non-linear 5D Poincare symmetries \cite{deRham:2010eu}
\begin{align}
  \delta \phi = c_{\mu}x^{\mu} + c_{\mu}\phi \partial^{\mu} \phi \,.
\end{align}
The product of two quadratic 4-point factors, instead, yields the special Galileon theory, with operator $(\partial\phi)^2\left([\Pi]^2-[\Pi^2]\right),\;\Pi_{\mu\nu}=\partial_{\mu}\partial_{\nu}\phi$, $[\dots]={\rm Tr}[\dots]$, and the non-linear symmetry \cite{Hinterbichler:2015pqa}
 \begin{align}
  \delta \phi = s_{\mu\nu}x^{\mu}x^{\nu} \,,
  \end{align}
resulting in the soft degree $\sigma_{\max} = 3$.

Can these be combined into a single, extended Goldstone theory that is BCJ-compatible? We will provide evidence that the answer to this question is affirmative. The defining property, similar to DBI and SG, will be the soft limit: all interactions are required to have at least $\sigma_{\rm min} =2$. This is naturally satisfied when taking curvature invariants of the metric \eqref{special metric}; a general EFT of this kind would therefore be
 \begin{align}
  \mathcal{L} = \sqrt{-g} [ c_0 + c_1 R + c_2 R^2 + \ldots ] \,.
 \end{align}
However, this does not display the $\sigma_{\rm max} = 3$ scaling in any limit. In order to ensure that the highest-derivative terms have the SG scaling, one needs to restrict to the specific Lovelock invariants at every order; these have the special property of being degenerate and hence do not generate any corrections with more than two derivatives on a given field \cite{Lovelock:1971yv}. When evaluated at the induced metric \eqref{special metric}, the Lovelock invariants become total derivatives:
 \begin{align}
  R^n &\equiv \delta_{\alpha_1\beta_1\dots\alpha_n\beta_n}^{\mu_1\nu_1\dots\mu_n\nu_n}\prod_{i=1}^n R^{\alpha_r\beta_r}{}_{\mu_r\nu_r} = \partial_\mu j^{(n)\mu}\,.
 \end{align}
For illustration, the currents for $n=1,2$ are given by
\begin{align}
 j^{(1)\mu} = &\frac{2}{1+(\partial \phi)^2} \left(\phi^\mu[\Pi]  - \phi_\nu\Pi^{\nu\mu} \right) \,, \nonumber \\
 j^{(2)\mu} = & \frac{4}{\left(1+(\partial \phi)^2\right)^2} (2 \phi^\mu [\Pi^3]-3 \phi^\mu [\Pi^2][\Pi]  \nonumber \\
  + & \phi^\mu [\Pi]^3+3 \phi_\nu \Pi^{\nu\mu} [\Pi^2] -6 \phi_\kappa \Pi^\kappa_\lambda \Pi^\lambda_\nu \Pi^{\nu\mu}  \nonumber \\
  + & 6 \phi_\kappa \Pi^\kappa_\nu \Pi^{\nu\mu} [\Pi] -3 \phi_\nu \Pi^{\nu\mu} [\Pi]^2 ) \,, 
\end{align}
with $\phi_\mu=\partial_\mu \phi$. At lowest order in $\phi$ these become
 \begin{align}
 \sqrt{-g} R^n 
  &\simeq (\partial\phi)^2\delta_{\mu_1\dots\mu_n}^{\nu_1\dots\nu_n}\prod_{r=1}^n (\partial^{\mu_r} \partial_{\nu_r} \phi) \,,
 \end{align}
up to an overal constant (for any other than the Lovelock combinations, there would be leading $(\partial^2 \phi)^n$ instead).
Note that these are exactly the SG invariants; indeed, both the Lovelock and the SG invariants trivialise in sufficiently low dimensions $D$.

We can therefore tune the Lovelock coefficients to have a theory that propagates a single scalar field, has $\sigma_{\rm min}=2$ for all amplitudes (ensuring that it is DBI plus higher-derivative corrections) and moreover the higher-derivative amplitude has $\sigma_{\rm max} = 3$ (such that it asymptotes to the SG). This {\it DBI-Lovelock} theory is therefore defined to all orders (and in all dimensions) uniquely by its non-linear symmetries and associated soft degrees.

We have checked that the amplitudes resulting from the above theory can be written in terms of a linear combination of BCJ factors, $N_4^{(2)} \times (N_4^{(1)} + N_4^{(2)})$ and $N_6^{(4)} \times (N_6^{(2)} + N_6^{(3)} + N_6^{(4)})$.
%\yl{Uniformally, $c_{N_6^{(2)}}/c_{N_6^{(3)}}=\tfrac32$, and $c_{N_6^{(2)}}/c_{N_6^{(4)}}=18$ for single- and double-copy.} 
Therefore they are BCJ compatible, at least up to this order. Moreover, the specific linear combination of quadratic, cubic and quartic 6-point terms that are needed for the gauged NLSM and the DBI-Lovelock theory are identical.

\section{Conclusion}


Purely based on representation theory of $S_n$, we have classified the different structures that are BCJ-compatible: at 4- and 5-point, there are two and 13 BCJ factors, respectively. 

Focussing on the former, we have identified a new sector of scalar field theories that can be BCJ-bootstrapped, over and beyond the analysis of \cite{Brown:2023srz}: the inclusion of the linear 4-point seed interaction introduces lower- (instead of higher-) derivative corrections to the NLSM. These allow for the construction of a gauged NLSM. BCJ compatibility then requires specific $F^3$ and $\phi^2 F^2$ terms. Moreover, the two 4-point seeds can be double copied into an extension of the special Galileon theory with lower-derivative corrections, which can be phrased in terms of Lovelock invariants of the DBI metric.

We have demonstrated the bootstrap construction of our two example theories explicitly at 6-point\footnote{Naturally, it would be very interesting to verify that this behaviour persists at higher-point; we leave this for future work.}.  In both cases, the theories are strongly constrained by two soft degrees with $\sigma_{\rm max} = \sigma_{\rm min} + 1$: this plays a crucial role in uniquely determining the BCJ factors. Our theories therefore appear to be closely related to the so-called extended DBI theory\footnote{The same reference also includes the combination of gluon exchange with cubic interactions of bi-adjoint scalars; this theory can be understood along similar lines as the extended DBI theory.} \cite{Cachazo:2014xea}, which involves the NLSM and DBI in specific limits. Indeed this theory can also be phrased in terms of BCJ factors \cite{Low:2020ubn}, and only has a finite number contributing to n-point amplitudes. Moreover, the on-shell constructability of this kind of theories follows from the graded soft theorem proposed in \cite{Kampf:2021tbk}. 

Based on the current results, it thus appears there is a fundamental difference between the higher-derivative corrections of \cite{Carrasco:2022lbm, Chen:2023dcx,Brown:2023srz} on the one hand, and hybrid Goldstone theories such as extended DBI, the gauged NLSM and DBI-Lovelock on the other hand: the latter category does not lead to a possibly infinite set of higher-derivative corrections at given multiplicity. We expect that this difference arises due to the special nature of the highest-derivative terms in these theories: these have a softer degree $\sigma_{\rm max}$, are related by the non-linear symmetry and are separately BCJ compatible. In contrast, the leading terms of the higher-derivative corrections $F^3$ and $F^4$ in \cite{Carrasco:2022lbm} (at vanishing gauge coupling, i.e.~of the form $(\partial A)^n$) do not have a non-linear symmetry (beyond Abelian gauge symmetry) and are not separately BCJ compatible. A similar discussion applies to the scalar BCJ bootstrap \cite{Brown:2023srz} and the KLT kernel bootstrap \cite{Chen:2023dcx}, whose 4-point seed structures do not include the linear exchange interaction \eqref{4pt_win_1}.  

Finally, the general analysis of this paper suggest a number of novel theories beyond the two examples that we outlined. At four-point, one can instead take the product $N_4^{(1)} \times (N_4^{(1)} + N_4^{(2)})$ leading to a combination of DBI with gravitational interactions; moreover, this can be extended to have an $SO(N)$ flavour along the lines of \cite{deNeeling:2022tsu}. More generally, one can consider the case where both BCJ factors have multiple terms of different order in Mandelstam. At first sight one might expect this to lead to a theory with three soft degree sectors; it would be interesting to investigate how this relates to the graded soft theorem of \cite{Kampf:2021tbk} that only allows for $\sigma_{\min}$ and $\sigma_{\rm max}$ to differ by one. We leave this question to future research.


\section*{Acknowledgements}

\noindent
The authors would like to thank Tom\'{a}\v{s} Brauner, John Joseph Carrasco, Dijs de Neeling and Karol Kampf for very valuable discussions.

\appendix 

\section{Representation theory approach}
\label{app:Jacobi representation}

\noindent
In the following, we briefly outline the method to obtain the decomposition of the BCJ representation into irreps, as given for 4- to 8-point in tables~\ref{table:BCJ irrep} and \ref{table:BCJ gauge}.

The decomposition of the BCJ representation into irreps is calculated via the characters of $S_n$. One straightforward approach is to solve the Jacobi identity in terms of a basis and to calculate the transformation matrices of each conjugacy class (as well as their traces). One then uses the standard character method of finite groups to calculate the multiplicities of irreps in the reducible BCJ representation. The character tables of $S_n$ can be found in \cite{character}.

Moreover, the decomposition of the symmetrised tensor product of $S_n$ irreps can be obtained in a parallel way. The only difference is that ${\rm trace}_i$ is obtained by Newton's identity of symmetric polynomials:
\begin{equation}
    h_r(\rho_i)=\sum_{j=1}^r\frac{1}{r}h_{r-j}(\rho_i)P_j(\rho_i) \,,
\end{equation}
where $\rho_i$ is the representative of a certain conjugacy class, $h_r(\rho_i)$ is the complete homogeneous symmetric function of eigenvalues corresponding to the matrix $\rho_i$, and $P_j$ is the $j$-th power sum 
\begin{align}
    P_j(\rho_i)=\sum_{k=1}^{\frac{1}{2}n(n-3)}\lambda_k^j
\end{align}
of the eigenvalues $(\lambda_1,\dots,\lambda_{\frac{1}{2}n(n-3)})$ of the transformation matrix $M_{[n-2,2]}^{\rho_i}$.

For the $r$-th power symmetric tensor product of $[n-2,2]$-irreps, one calculates $h_r(\rho_i)$ for all conjugacy classes iteratively. One then solves a set of linear equations, {\em i.e.}, $h_r(\rho_i)=\sum_{j}a_j\chi_j(\rho_i)$, $j$ sum over all irreps, and each linear equation is about one conjugacy class $\rho_i$. Finally, one obtains the multiplicity $a_j$ of irreps in the symmetric tensor product.



\bibliography{BCJNLSM.bib}

\end{document}