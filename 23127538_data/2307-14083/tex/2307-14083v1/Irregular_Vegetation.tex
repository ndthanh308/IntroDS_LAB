\documentclass[preprintnumbers,superscriptaddress,pre,twocolumn]{revtex4-2}
\usepackage{stackengine}
\usepackage{amsmath}
\usepackage[sc]{mathpazo}
\usepackage{graphicx,subfigure}
\usepackage{tabularx, booktabs}
%\usepackage{subcaption}
\usepackage{geometry}
\usepackage{tikz}
\geometry{
	a4paper,
	total={170mm,257mm},
	left=20mm,
	top=20mm,
}
\newcommand{\JFdel}[1]{\textcolor{red}{#1}}
\newcommand{\JFadd}[1]{\textcolor{blue}{#1}}

\begin{document}
	
	\title{Self-organized criticality explains the emergence of irregular vegetation patterns in semi-arid regions}
	
	\author{Jean-François de Kemmeter}
	\affiliation{Department of Mathematics and naXys, Namur Institute for Complex Systems, University of Namur, Rue Graf\'e 2, B5000 Namur, Belgium}
	\affiliation{Department of Mathematics, Florida State University, 1017 Academic Way, Tallahassee, FL 32306, United States of America}
	
	\author{Adam Byrne}
	\affiliation{School of Mathematics and Statistics, University College Dublin, Belfield, Dublin 4, Ireland}
	
		\author{Amy Dunne}
	\affiliation{School of Mathematics and Statistics, University College Dublin, Belfield, Dublin 4, Ireland}
	
	\author{Timoteo Carletti}
	\affiliation{Department of Mathematics and naXys, Namur Institute for Complex Systems, University of Namur, Rue Graf\'e 2, B5000 Namur, Belgium}
	
	\author{Malbor Asllani}
	\affiliation{Department of Mathematics, Florida State University, 1017 Academic Way, Tallahassee, FL 32306, United States of America}
	

	
\begin{abstract}
Vegetation patterns in semi-arid areas manifest either through regular or irregular vegetation patches separated by bare ground. Of particular interest are the latter structures, that exhibit a distinctive power-law distribution of patch sizes. While a Turing-like instability mechanism can explain the formation of regular patterns, the emergence of irregular ones still lacks a clear understanding. To fill this gap, we present a novel self-organizing criticality mechanism driving the emergence of irregular vegetation patterns in semi-arid landscapes. The model integrates essential ecological principles, emphasizing positive interactions and limited resources. It consists of a single-species evolution equation with an Allee-logistic reaction term and a nonlinear diffusion one accounting for self-segregation. The model captures an initial mass decrease due to resource scarcity, reaching a predictable threshold. Beyond this threshold and due to local positive interactions that promote cooperation, vegetation self-segregates into distinct clusters. Numerical investigations show that the distribution of cluster sizes obeys a power-law with an exponential cutoff, in accordance with empirical observations found in literature.  The study aims to establish a foundation for understanding self-organizing criticality in vegetation patterns, advancing the understanding of ecological pattern formation.
\end{abstract}	

\date{\today}
\maketitle 
	

%\section{Introduction}
%

\textbf{Introduction}\label{sec:I} Nature exhibits various forms and shapes of order, spanning from the collective flight of birds in flocks~\cite{bialek2012statistical} to the synchronized flashing of fireflies~\cite{firefly_sync}. Self-organization has long been recognized as the fundamental principle driving the emergence of such captivating patterns~\cite{ball_self-made_1999, murray_mathbio,cross_pattern_2009}. Exploring how these collective behaviors and patterns arise from the interactions among the system basic units has been a vibrant research field for a long time. Notably, the past two decades have witnessed a growing interest in understanding the formation of vegetation patterns in semi-arid ecosystems~\cite{Rietkerk, Bonachela,klausmeier1999regular,zhang2020regular,scanlon2007positive}.
Remarkably, even in harsh environmental conditions, plants manage to survive by clustering together in spots or stripes. Recent studies have classified vegetation patterns into two main categories: regular and irregular patterns \cite{klausmeier1999regular}. Regular patterns exhibit a characteristic length scale that represents a spatial order, whereas irregular vegetation patterns consist of clusters of vegetation of diverse sizes, separated by bare areas~\cite{klausmeier1999regular}. Although a lack of organization is observed at short scales in irregular patterns, an overall global-level order emerges. Indeed, investigations conducted in various geographical areas have revealed that the distribution of cluster sizes in irregular vegetation patterns can be well described by a power-law behavior, often exhibiting an exponential cutoff~\cite{scanlon2007positive,kefi2007spatial}.
While the mechanism behind regular patterns is thought to arise from a Turing-like instability~\cite{Rietkerk, Bonachela, meron_vegetation_2019}, the understanding of the causes underlying irregular vegetation patterns remains limited. Although a previous study~\cite{scanlon2007positive} identified the significant role of positive interactions between neighboring plants through an agent-based approach, a comprehensive analytical understanding of the underlying mechanisms is still missing. 

To address this knowledge gap, we propose in this paper a novel mechanism of \emph{self-organized criticality} (SOC), that governs the evolution dynamics of irregular patterns. Self-organized criticality has proved successful in explaining emergent phenomena characterized by power-law scaling of cluster sizes in various scenarios, such as avalanches in the sand-pile model~\cite{BTW}, forest-fire dynamics~\cite{forest_fire}, and the spread of infections in epidemics~\cite{rhodes_critical_1997}. SOC models are distinguished by their critical state, wherein system dynamics reach a critical point as a specific dynamical variable, such as mass~\cite{BTW} or energy~\cite{Zhang_energy}, surpasses a certain threshold, instead of relying on a model parameter.
We begin by considering fundamental principles that describe individual efforts to survive against internal and external factors, as well as their dispersal in the spatial domain while considering limited resources and social affinity. Based on these considerations, we derive a reaction-diffusion equation that governs the temporal evolution of vegetation density in the spatial domain. To address the challenges of survival in a harsh arid environment, we utilize a modified logistic equation with an Allee effect~\cite{allee_1932, courchamp2008allee}. This phenomenon dictates that a species can only persist if its local population exceeds a specific threshold, otherwise leading to extinction. Notably, the nonlinear diffusion model developed in this study is reminiscent of earlier nonlinear random walk processes introduced by the authors~\cite{Asllani_PRL, Carletti_PRR, Siebert_2022, DeKemmeter}. In a previous work~\cite{DeKemmeter}, we demonstrated that due to the heterogeneous structure of the spatial network, individual interactions lead to spontaneous self-segregation, resulting in vacant habitats. In contrast, the present study highlights the emergence of asymmetrical interactions among neighboring individuals as the trigger for this phenomenon. Fundamentally, the segregation arises from the propensity individuals possess for occupying the same patch or area.
In essence, the process can be summarized as follows: the system is initialized above the SOC threshold for the segregation to occur but below the Allee threshold of survivability, with the dispersion rate set higher than the birth/death rates. Starting below the survivability threshold, akin to an arid landscape, the mass steadily decreases until it reaches a critical value. Interestingly, during this process, the mass segregates into clusters separated by vegetation-free areas, as faster diffusion enables accumulation in patches with densities locally surpassing the threshold for survivability. Once slower reaction dynamics come into play, the density maximizes at the carrying capacity. Thus, counterintuitively with what expected, the vegetation will survive emphasizing the role of the self-segregation process representing the positive interaction between individuals. Analysis of cluster size distributions reveals a power-law behavior with an exponential cutoff at larger sizes.
In conclusion, our model effectively captures the primary dynamics of vegetation in arid landscapes, aligning remarkably well with empirical observations conducted \emph{in situ}~\cite{scanlon2007positive,kefi2007spatial}.

%The paper is organized as follows. In section~\ref{sec:ILM}, we introduce the non-linear reaction-diffusion process governing the spatio-temporal evolution of the vegetation densities. In section~\ref{sec:III}, we show how this process leads to the emergence of irregular clusters separated by empty areas. We then analyze in section~\ref{sec:IV} the distribution of the cluster sizes. We summarize our findings and conclude in section~\ref{sec:V}.  


%%%%%%%%%%%%%%%%%%%%%%%%%%%%%%%%%%%%%%%%%%%%%%%%%%%%%%%%%%%%%%%%%%%%%%%%%%%%%%%%%%%%%%%%%%%%%%%%



\textbf{The individual-based model and the mean-field limit}\label{sec:II} Let start by considering the spatial domain where the interactions between the agents occur to be a two-dimensional square support of unit length subdivided into $\Omega = L\times L$ spatial compartments or squared patches of equal area, labeled $v_{i}$ for $i=1,\cdots,\Omega$. 
%% Limited carrying capacity: plants and vacancies
We assume each patch contains the same limited amount of abiotic resources available to the plants, where by \textit{plant} we refer to an autotrophic organism of the same species. As a consequence, each patch has a limiting carrying capacity $N$, that sets the maximal number of individuals the patch can host simultaneously. The number of plants within patch $v_i$ is denoted by $0\leq n_i\leq N$ and thus $N-n_i$ quantifies the vacancies, i.e., the additional number of plants the patch $v_i$ might host. 
%%
Since plants are considered indistinguishable from each other, we denote by $X_i$ any of such individuals within patch $v_i$. Similar criterion applies to the vacancies that we indicate with $E_i$.
%% Dynamics
Throughout this paper we will assume that individuals will interact with each other both within each patch and between adjacent ones. The dynamics at a purely local level will capture the basics of the ecological phenomena such as birth and death processes relevant to the particular framework of the vegetation in (semi-)arid regions. In particular, it is known that all living beings in this harsh environmental conditions, are affected by a strong Allee effect \cite{courchamp2008allee}. On the other side the individuals are allowed to interact also at the inter-patch level with the individuals of the neighbor compartments. Such an interaction is the clue point of this paper and to anticipate it in few words, it will describe the asymmetric positive interaction between individual of different patches taking at the same time into account the finite carrying capacity of each site of the two-dimensional lattice. 

We start by representing the dynamics within an arbitrary patch $v_i$ through the following set of chemical-like reactions
\begin{subequations}
	\begin{align}
	X_{i} + E_{i} &\xrightarrow{r_1} 2 E_{i} \label{2AE-3A},\\
	2 X_{i} + E_{i} &\xrightarrow{r_2} 3X_i \label{AE-2E} 
	\end{align}
	\label{eq:3part}
\end{subequations}
where each reaction occurs respectively with a probability proportional to $r_1$ and $r_2$, $0<r_1<r_2$, and denotes the death and birth of individuals due to cooperation, respectively.
Eq.~\eqref{2AE-3A} tells us that individuals have a natural trend of dying unless they cooperate with each other, as modeled by Eq.~\eqref{AE-2E}. In fact, there are many examples of ecological systems in which cooperation between agents is crucial for the population to survive and to oppose to the challenges imposed by the finite carrying capacity. The latter encapsulates, in the most general description, not simply limited resources but all other factors with a negative impact on the growth and survivability of the species, such as the presence of predators, intra- or inter-species competition, lack of potential mating partners and so on~\cite{allee_1932,stephens1999allee,courchamp1999inverse,dennis2002allee,taylor2005allee,courchamp2008allee}. This phenomenon is known as the Allee effect and it is widespread in the framework of ecological dynamical systems.  Many mechanisms have been proposed to show how species optimize their chances to survive~\cite{kindvall1998individual,kanarek2015overcoming,Asllani_Allee} and in this paper we will show how self-segregation processes can be crucial for vegetation survival in arid landscapes and for spatially shaping their power-law clustering.
 
    
We now turn our attention to the spatial dimension of our model. For simplicity, we will assume that agents move between nearest neighbor sites, provided that there is enough available place to host them. This dynamics has been previously described by using again the language of chemical reaction and further generalized in Ref.~\citep{Asllani_PRL} in the form $X_{i} + E_{j} \xrightarrow{\delta} E_{i} + X_{j}$
where the sites $i$ and $j$ are nearest neighbors and the reaction takes place with a probability proportional to $\delta>0$. Clearly, the microscopic process of the migration of individuals from one site to another preserves the number of total agents involved in the closed system. %Using again the law of mass action, one could expect that the evolution of the density $\rho_i$ depends proportionally on the coupling term $\rho_i(1-\rho_j)$, thus reflecting the amount of crowding of hosting site. 
Nevertheless, such a term does not capture the positive feedback mechanisms known to drive the growth of vegetation population. According to such feedback, the new plants are more likely to settle in areas in which the vegetation density is already large enough~\cite{scanlon2007positive,meron_vegetation_2019}. In this sense, the dispersion of the vegetation in the spatial domain will act as a trade-off between the positive interactions between neighbor individuals and the finite carrying capacity. Based on these premises, we modify the transport process accordingly:
\begin{equation}
\begin{split}
X_{i} + aX_{j} + E_{j} &\xrightarrow{\delta} E_{i} + 2X_{j},\\
\end{split}
\label{eq:diffpart}
\end{equation}
with $a\geq 1$. Let us observe that to balance the two sides of the reaction~\eqref{eq:diffpart} one must require $a=1$, but as we will show in the following, in order for the irregular patterns to emerge we must assume $a>1$. This requirement implies an asymmetry in the interaction between individuals of adjacent sites, a property of the dynamics that is inspired by the ecological literature~\cite{Weiner} and we will refer to as \emph{size-asymmetric interaction}. In other words, individuals of the patch $v_i$ will perceive an higher number of individuals than those really available in the hosting site $v_j$. 

To cast such individuals-based rules in a single mathematical framework, we will make use of the master equation 
\begin{equation}
\frac{ d \mathrm{P}(\mathbf{n},t) }{dt} =  \sum_{\mathbf{n'} \neq \mathbf{n}} 
\mathrm{T}(\mathbf{n} \vert \mathbf{n'}) \mathrm{P}(\mathbf{n'},t)
-\mathrm{T}(\mathbf{n'} \vert \mathbf{n}) \mathrm{P}(\mathbf{n},t)
\label{eq:MEQ}
\end{equation}
which provides a detailed probabilistic description of the dynamics starting from the microscopic setting. Here $\mathbf{n}(t) = \left(n_1\left(t\right),n_2\left(t\right),\cdots,n_{\Omega}\left(t\right)\right)$ is the state vector, with $n_i(t)$ the number of individuals within node $v_i$ at time $t$, and $\mathrm{P}(\mathbf{n},t)$ is the probability that the system will be in such a state {at time $t$}. Furthermore $\mathrm{T}(\mathbf{n'} \vert \mathbf{n})$ denotes the transition probability, per time unit, from state $\mathbf{n}$ to state $\mathbf{n}'$ and the summation in~\eqref{eq:MEQ} extends over all the states different from $\mathbf{n}$. Details of the master equation and explicit derivation for the transition probability are given in the Appendix~\ref{app:ME}. Here we summarize such transition probabilities. For the aspatial reactions, one has
\begin{subequations}
	\begin{align}
&T(n_i-1 \vert n_i) = \frac{r_1}{\Omega} \dfrac{n_i}{N}\Big(1-\dfrac{n_i}{N}\Big),\\
&T(n_{i}+1 \vert n_{i}) = \frac{r_2}{\Omega}\dfrac{n_i}{N}\dfrac{n_i-1}{N}\Big(1-\dfrac{n_i}{N}\Big),
	\end{align}
\label{eq:BirthDeath}
\end{subequations}
for the birth and death dynamics, respectively. 
Similarly, provided sites $v_i$ and $v_j$ are connected, the transition from $v_i$ to $v_j$ is
\begin{equation}
	T(n_{i}-1,n_{j}+1 \vert n_i, n_j) 
	= \frac{\delta}{k \Omega} \frac{n_i}{N} {\Big(\frac{n_j}{N}\Big)}^a {\Big(1-\frac{n_j}{N}\Big)},
	\label{eq:ratesDiff}
\end{equation}	
with $k$ the number of neighbors per site, i.e., $k=4$ in the present setting. A similar transition is obtained for agents moving from node $v_j$ to node $v_i$.

The master equation \eqref{eq:MEQ} is generally impossible to treat analytically, for this reason we will look for a mean-field formalism which describes deterministically the evolution of densities of individuals and we assume to deal with a continuum domain. A detailed derivation can be found in the Appendix~\ref{app:ME}, let us here recall that the standard approach is to consider the time evolution of the averaged number $\langle n_i\rangle$ of agents within the site $v_i$ in the limit $N \rightarrow +\infty$ and then take the continuum limit in which the mesh space goes to zero. This procedure leads to the following partial differential equation for the time evolution of the vegetation density $\rho\equiv \rho(\mathbf{x},t)$ at point $\mathbf{x}=(x,y)$ and time $t$: 
\begin{equation}
\frac{\partial \rho}{\partial t}=
r\, f(\rho)
+ D \, \Big[
g\Delta \rho - \rho \Delta g	
\Big]\ .
\label{eq:PDE}
\end{equation}
Here $D>0$ represents the diffusion coefficient, $\Delta$ the Laplace operator, $f(\rho)=\rho(1-\rho)(\rho-A)$ the Allee reaction term with $r>0$ the growth rate and $0<A<1$ the Allee coefficient \footnote{In Appendix \ref{app:ME} we show how to determine the expressions for the coefficients $D, A$ and $r$ as a function of the rates hereby introduced. More precisely, the diffusion coefficient is obtained as the limit $D = \lim_{\Omega\rightarrow \infty} \delta/\Omega$. Similarly, the growth rate is obtained as the limit $r = \lim_{\Omega\rightarrow \infty} r_2/\Omega$. Finally $A=r_1/r_2$.}. The function $g(\rho)=\rho^a(1-\rho)$ captures in a compact form the nonlinear interacting terms between individuals of neighbor sites. Throughout this paper we consider periodic boundary conditions in the square continuous domain $\mathcal{R}=[0,1]\times [0,1]$. Let us observe that if $r=0$ the total mass is conserved (see Appendix~\ref{app:ME}). \newline


% Figure environment removed

%%%%%%%%%%%%%%%%%%%%%%%%%%%%%%%%%%%%%%%%%%%%%%%%%%%%%%%%%%%%%%%%%%%%%%%%%%%%%%%%%%%%%%%%%%%%%%%%


\textbf{Self-segregation process as a self-organized criticality mechanism}\label{sec:III} The primary objective of this study is to demonstrate that the mean-field model described by the nonlinear reaction-diffusion equation \eqref{eq:PDE} can elucidate the emergence of spatial vegetation patterns characterized by a power-law distribution of cluster sizes. As a preview of our findings, we will establish that the irregular patterns arise when the total mass of the system reaches a critical value, a characterizing feature of SOC processes. Let us first observe that any uniformly distributed density $\rho(\mathbf{x},t)=\beta$ with $\beta={0,A,1}$ represents a stationary solution of \eqref{eq:PDE}. The stability of these states can be determined by analyzing the linear evolution of the perturbation $\delta \rho(\mathbf{x},t)$, governed by the equation
\begin{equation}
\frac{\partial \delta \rho}{\partial t} = f'(\beta)\delta \rho + D\left[g(\beta)-\beta g'(\beta)\right] \Delta \delta \rho\, .
\label{eqA:Stab}
\end{equation}
It can be readily verified that $f'(\beta)<0$ when $\beta={0,1}$ and $f'(\beta)>0$ when $\beta=A$, indicating the bistable nature of the Allee model. By seeking solutions of the form $\delta \rho \sim \sum_{\mathbf{k}} e^{\lambda_{\mathbf{k}}t}e^{i\mathbf{k}\cdot \mathbf{x}}$, we obtain the dispersion relation
\begin{equation}
\lambda_{\mathbf{k}}=f'(\beta)-D \left[g(\beta)-\beta g'(\beta)\right]|\mathbf{k}|^2\, ,
\label{eq:disp_rel}
\end{equation}
where $|\mathbf{k}|^2=k_1^2+k_2^2$ is the module of the vector $\mathbf{k}$. Notably, since the latter is positive, the fixed point $\beta=A$ is unstable. In fact, there will always exist a finite interval (e.g., near the origin) for which $\lambda_{\mathbf{k}} > 0$. Conversely, the other homogeneous fixed points $\beta={0,1}$ are stable as long as the effective diffusion coefficient $D_{\mathit{eff}}=D\left[g(\beta)-\beta g'(\beta)\right]$ remains positive, or equivalently, $g(\beta) - \beta g'(\beta) \geq 0$. Observe that this condition is satisfied for $\beta={0,1}$, but it does not contribute to the problem of pattern formation because it will lead to a global extinction or to a fully occupied domain. Conversely, the dynamics stemming from the unstable state $\beta=A$ do not guarantee the emergence of any spatial pattern organized into separate clusters. Clearly, Eq.~\eqref{eq:PDE} displays other fixed points, whose existence and stability will be addressed in the following.

In order to investigate the occurrence of irregular patterns in the system, we adopt an alternative approach based on slow-fast dynamics. Specifically, we consider the limit $r/D\rightarrow 0$, where the fast dynamics is solely governed by the nonlinear diffusion process. In general, diffusion processes tend to homogenize the spatial distribution of mass. However, as previously observed, under certain conditions, the effective diffusion coefficient can become negative ($D_{\mathit{eff}}<0$). Negative diffusion exhibits the opposite effect of homogenization, leading to the accumulation and localization of mass within the spatial domain~\cite{karpov_negative_1995,argyrakis_negative_2009}.
Motivated by this insight, we first note that, in contrast to the full reaction-diffusion equation, the nonlinear diffusion operator vanishes for every uniform state $\rho(\mathbf{x})=\beta$, i.e., $\beta >0$. At this stage, we can ascertain the critical value $\beta_c$ of the average node density $\beta$ at which the equilibrium $\rho(\mathbf{x})=\beta$ undergoes instability due to diffusion. It can be easily shown (see Appendix~\ref{app:stabilityMatrix}), that this critical value is given by
\begin{equation}
\beta_c=\frac{a-1}{a}\, .
\label{eq:beta_c}
\end{equation}
This formula justifies the choice of $a > 1$ for the problem to have a physical meaning, in particular the asymmetry in the interactions between adjacent sites allows for the self-segregation to occur, i.e., the existence of densities such that $0 < \beta \leq \beta_c $. Also the stronger the asymmetry in the interactions, the larger is the total mass (neglecting the reaction) for which the system can self-organize i.e., $\lim_{a\rightarrow+\infty} \beta_c \rightarrow 1$. Any uniform state with $\beta < \beta_c$ becomes unstable, while it remains stable otherwise. Upon instability, due to mass conservation, a redistribution of mass is expected to occur. However, in the case of self-organization through self-segregation, this redistribution can only take place in the form of clusters, hereby referred to connected subregions of homogeneous mass, separated by empty patches. For a cluster $\mathcal{C}_s$ to be independently stable, the local density must satisfy $\beta_{\mathit{loc}}=\frac{1}{\vert \mathcal{C}_s \vert }\int_{\mathcal{C}_s} \rho (\mathbf{x}) \text{d}\mathbf{x} >\beta_c$, where $\vert \mathcal{C}_s \vert$ is the area of the cluster $\mathcal{C}_s$. Once the (fast) diffusion creates a stable uniform cluster, the (slow) reaction comes into play by maximizing the cluster density to unity if $\beta_{loc}>A$ or reducing it to zero otherwise. In addition, considering the physics of the problem, each cluster adheres to no flux boundary conditions, denoted as $\nabla \rho(\mathbf{x})=0$. This approach additionally uncovers the presence of heterogeneous (stable) fixed points in the function $\rho(\mathbf{x})$. A more comprehensive and rigorous proof is provided in Appendix~\ref{app:fixedpoints}.
This initial discovery holds great significance in our analysis, as it unveils a fundamental insight: the stability of a specific state relies not on model parameters, but rather on the size, specifically the total mass, of the state itself~\footnote{Note that the dependence of $\beta_c$ on the parameter $a$ is irrelevant for the problem because, by definition, $a>1$. Therefore, there will always exist a critical threshold on the average density $\beta$.}. This property aligns perfectly with the concept of self-organized criticality (SOC), which pertains to the inherent self-organization of a system when it reaches a critical threshold of a globally defining observable, such as mass or energy~\cite{BTW, Zhang_energy}.

% Figure environment removed

Numerous intriguing phenomena manifest in SOC processes near the tipping point. Without delving further into this direction, our focus will be on exploring the implications of self-segregation on vegetation patterns. Referring to Fig.~\ref{fig:SOC}, panel $(a)$, we consider a slow-fast dynamics setting, i.e. $r/D \ll 1$, within a spatial mesh of $500 \times 500$. Initially, the total vegetation mass is randomly distributed throughout the spatial domain, yet always below the Allee threshold and above the critical value $\beta_c$. The strong diffusion tends to initially homogenize the mass, causing it to decrease since it remains below the survivability threshold. As expected in SOC dynamics, when the global observable (density in our case) reaches the critical value $\beta=\beta_c$, a change of behaviour occurs, leading to an overall increase in mass across most lattice sites. The remarkable {and conterintuitive} aspect is that the species manage to survive in the asymptotic state, despite the fact that the {initial} density at each site is below the Allee parameter, as indicated by the initial trend. The rationale behind this fact is that, as the average density decreases further, a new phenomenon emerges: self-segregation. Driven by the negative $D_{\mathit{eff}}$, the mass rapidly accumulates and localizes in different disconnected subregions of the domain $\mathcal{R}$. If the density of the new clusters surpasses both the critical values of self-segregation and Allee ($\beta_{\mathit{loc}} > \beta_c, A$), the species will survive in those particular clusters and eventually reach a full carrying capacity $\beta_{\mathit{loc}}=1$, as illustrated in Fig.~\ref{fig:SOC}, panel $(a)$. The benefit of self-segregation for plant survivability is systematically investigated in Fig.~\ref{fig:SOC}, panel $(b)$, where various initial density values $\beta_0=\int_{\mathcal{R}}\, \rho(\mathbf{x},0)\mathrm{d}\mathbf{x}$ are considered. In all cases, the species survive beyond intuitive expectations. Particularly, in the interval $\beta\in[\beta_c, A]$, the diffusion has a homogenizing effect by reducing the initial perturbation, consequently slowing down the fast dynamics of the diffusion component. Consequently, the final equilibrium density is lower than the initial density. However, this outcome is an artefact of the deterministic mean-field approach utilized here. In a real scenario, the presence of external or demographic noise acts as a permanent perturbation (forcing) term, preventing a substantial decrease in the final density compared to the critical value $\beta_c$. Stochastic simulations were performed using the Gillespie algorithm, as depicted by the blue dashed line (and corresponding shaded blue region) in Fig. \ref{fig:SOC} (b), thereby substantiating our claim.

%%%%%%%%%%%%%%%%%%%%%%%%%%%%%%%%%%%%%%%%%%%%%%%%%%%%%%%%%%%%%%%%%%%%%%%%%%%%%%%%%%%%%%%%%%%%%%%%


\textbf{Power-law distribution in the self-segregation vegetation patterns}\label{sec:IV} The self-segregation process we have introduced in this work, bridges the formation of irregular vegetation patterns to SOC processes which are renown for their power-law distributed clusters' size. This is the case for instance for the original sandpile model or the forest-fire one where the size of all possible avalanches generated has a scale-free distribution \cite{BTW,forest_fire}. Based on this intuition, we have carried out a large number of independent simulations of Eq.~\eqref{eq:PDE} for different initial values of the density $\beta_0$ approaching the critical value of the system, i.e., the value beyond which the system transits to unstable, i.e., patterns will occur. Although in the slow-fast setting this value is expected to be close to $\beta_c$, the critical point for the self-segregation diffusion, it cannot be in general exactly calculated since we uniformly initialize the system far from the known fixed points of the reaction-diffusion system. In Fig.~\ref{fig:PowerLaw}, panels $(a_1)$ and $(a_2)$, we show patterns with clusters of varying sizes for two different {values of the initial density $\beta_0$}. Analyzing statistically the distribution of the size of the clusters obtained as equilibrium states of~\eqref{eq:PDE}, in Fig.~\ref{fig:PowerLaw} $(b)$ we show the cumulative distribution $\mathbb{P}(S \geq s) $ as a function of the clusters size $s$. It can readily {be} observed that they fit very well to a power-law function with {almost} the same critical exponent $\alpha$ and are characterized by different values of exponential cutoffs {that depend on} the initial density $\beta_0$. In summary, 
\begin{equation}
\mathbb{P}(S \geq s) \sim s^{-\alpha}e^{s\,\xi(\beta_0)},
\end{equation}
where the function $\xi(\beta_0)$ vanishes when $\beta_0$ equals $\hat{\beta}_c$, the value for which a perfect power-law relation is observed. Inspired by similar scenarios as for the Ising model~\cite{sethna_statistical_2021} or the percolation processes~\cite{ding2014numerical}, we set $\xi(\beta_0)\sim {\vert \hat{\beta}_c - \beta_0 \vert}^{\gamma}$ yielding to a second critical exponent $\gamma>0$ which describes the transition to a genuine power-law. It is noteworthy that our model is characterized by two (independent) critical exponents. In particular, the exponent $\gamma$ signifies the independence of the exponential cutoff from the size of the system. In Fig.~\ref{fig:PowerLaw} $(b)$ we have shown with a red dashed line and colored solid lines, respectively, for the power-law and the exponential cutoff, the best fit to the empirical critical exponents $\alpha \approx 0.83$ and $\gamma \approx 1.85$; let us observe that both fits agree well with the data except for small values of $s$. The latter issue is due to the finite resolution of the numerical simulations.  Let us notice that being the latter {two parameters} independent from the values of the initial system mass, this suggests that our model belongs to a broader universality class, typical of systems where power-law distributions emerge~\cite{sethna_statistical_2021}. A compact way to illustrate this is by plotting $s^\alpha \mathbb{P}(S \geq s)$ as a function of $s\,{\vert \hat{\beta}_c - \beta_0 \vert}^{\gamma}$; the different curves now collapse onto a single one, known in the litterautre as the universal curve (see inset of panel $(c)$). In conclusion, we assert that while the universal power-law distribution of patch sizes is solely driven by the self-segregation process (see Appendix~\ref{app:stabilityMatrix}), the reaction component is vital for accurately describing the resilience of vegetation in harsh arid conditions.\newline

%%%%%%%%%%%%%%%%%%%%%%%%%%%%%%%%%%%%%%%%%%%%%%%%%%%%%%%%%%%%%%%%%%%%%%%%%%%%%%%%%%%%%%%%%%%%%%%%


\textbf{Conlusions}\label{sec:V} In this study, we presented a novel dynamical model that addresses the emergence of irregular vegetation patterns in semi-arid landscapes, characterized by a power-law distribution of spatial patch sizes. Our model incorporates essential ecological principles necessary for the survival of plant populations in such environments. By considering positive interactions between individuals and accounting for limited resources dictated by the harsh arid conditions, we develop a self-consistent mathematical formalism. The model encompasses a single-species evolution equation, specifically targeting the vegetation density, with a local reaction term based on the Allee-logistic function. This choice allows us to emphasize the importance of positive interactions among individuals within the population. To capture the spatial dynamics, we introduce a nonlinear diffusion term that serves to mimic the phenomenon of self-segregation. The latter process assumes a critical role in initiating pattern formation and establishing a mechanism of self-organized criticality. Within this framework, we observe an initial decrease in mass, driven by insufficient resource availability, until a threshold, that can  be analytically predicted, is reached. Beyond this threshold, we observe the spatial organization of vegetation into distinct clusters characterized by higher densities, thus fostering cooperative behaviors among individuals. Consequently, clusters with densities surpassing the Allee threshold shape the final pattern. Numerical investigations confirm that the distribution of cluster sizes aligns perfectly with empirical observations, following a power-law function with an exponential cutoff. Notably, we find that the critical exponents associated with this distribution remain independent of the initial mass of the system. This model establishes a foundation for understanding the self-organizing criticality mechanism underlying power-law distributions in vegetation patterns, paving the way for new directions in ecological pattern formation research.


\textbf{Acknowledgments} JFDK is supported by a FNRS Aspirant Fellowship under the Grant FC38477. Part of the results were obtained using the computational resources provided by the “Consortium des Equipements de Calcul Intensif” (CECI), funded by the Fonds de la Recherche Scientifique de Belgique (F.R.S.-FNRS) under Grant No. 2.5020.11 and by the Walloon Region.

\bibliographystyle{apsrev4-2}%{abbrv}%{unsrtnat}%{plainnat}%{abbrvnat}%{apsrev4-2}
\bibliography{biblio}% Produces the bibliography via BibTeX.


\clearpage
\onecolumngrid
\appendix
\section{Mathematical details of the individual-based modeling}\label{app:ME}

\subsection{Exact transition rates}
The aim of this subsection is to derive the transition probabilities given by Eq.~\eqref{eq:BirthDeath}-\eqref{eq:ratesDiff} from the corresponding reactions provided in Eq.~\eqref{eq:3part} and~\eqref{eq:diffpart} and reproduced below for convenience.
\begin{subequations}
\begin{align}
X_{i} + E_{i} \xrightarrow{r_1} 2 E_{i} 
\quad &\Longrightarrow \quad T(n_{i}-1 \vert n_i) = \frac{r_1}{\Omega} \frac{n_i}{N}\frac{N-n_i}{N}\label{eqA:rateDeath}, \\
2 X_{i} + E_{i} \xrightarrow{r_2} 3X_i  
\quad &\Longrightarrow \quad T(n_{i}+1 \vert n_{i}) = \frac{r_2}{\Omega} \frac{n_i}{N}\frac{n_i-1}{N}\frac{N-n_i}{N} \label{eqA:rateBirth},\\
X_{i} + aX_{j} + E_{j} \xrightarrow{\delta} E_{i} + 2X_{j} 
\quad &\Longrightarrow \quad T(n_{i}-1,n_{j}+1 \vert n_i, n_j) = \frac{\delta}{\Omega k} \frac{n_i}{N} g\Big(\frac{N-n_j}{N}\Big),\\
X_{j} + aX_{i} + E_{i} \xrightarrow{\delta} E_{j} + 2X_{i}  
\quad &\Longrightarrow \quad T(n_{i}+1,n_{j}-1 \vert n_i, n_j) = \frac{\delta}{\Omega k} \frac{n_j}{N} g\Big(\frac{N-n_i}{N}\Big)\, ,
\end{align}
\label{eqA:rates}
\end{subequations}
where $g(\rho)=\rho^a(1-\rho)$. 

The derivation is similar to the one discussed in~\cite{mckane2004stochastic}. As a preliminary step, let us remind how to compute the probability $\mathcal{P}(X=k,E=\ell)$ to pick without reinsertion $k\leq n$ letters $X$ and $\ell\leq N-n$ letters $E$ in an urn that contains $n$ letters $X$ and $N-n$ letters $E$. As a first step, let us determine the probability to pick $k$ consecutive letters $X$ follows by $\ell$ consecutive letters $E$. This probability reads:
\begin{equation}
\Big(\frac{n}{N}\frac{n-1}{N-1}\cdots\frac{n-k+1}{N-k+1}\Big)
\Big(\frac{N-n}{N-k}\frac{N-n-1}{N-k-1}\cdots\frac{N-n-\ell+1}{N-k-\ell+1}\Big)
= \frac{n!}{(n-k)!} \frac{(N-k-\ell)!}{N!} \frac{(N-n)!}{(N-n-\ell)!}.
\end{equation}
The probability $\mathcal{P}(X=k,E=\ell)$ is then obtained by multiplying the above expression by the number of distinct configurations obtained upon permutations of X's and E's, which is given by the binomial coefficient $\binom{k+\ell}{\ell}$. Overall, we obtain:
\begin{equation}
\mathcal{P}(X=k,E=\ell) = \frac{\binom{N-n}{\ell}\binom{n}{k}}{\binom{N}{k+\ell}}.
\end{equation}
It follows that for~\eqref{eqA:rateBirth}, for instance, the probability to pick two agents $X$ and one vacancy $E$ within node $v_i$ is given by $3 \dfrac{n_i}{N}\dfrac{n_i-1}{N-1}\dfrac{N-n_i}{N-2}$. Denoting by $p_2$ (resp. $p_1$) the probability that the reaction will be of type~\eqref{eqA:rateBirth} (resp.~\eqref{eqA:rateDeath}) and using the fact that node $v_i$ is selected with probability $1/\Omega$, the transition rate corresponding to reaction ~\eqref{eqA:rateBirth} is given by:
\begin{equation}
\begin{split}
T(n_i+1 \vert n_{i}) &\, \sim \, 3 r_2 \frac{p_2}{\Omega}\frac{n_i}{N}\frac{n_i-1}{N-1}\frac{N-n_i}{N-2} \, \sim \, 3 r_2 \frac{p_2}{\Omega} \frac{N}{N-1} \frac{N}{N-2} \frac{n_i}{N}\frac{n_i-1}{N}\frac{N-n_i}{N}\\
&\equiv \frac{r_2'}{\Omega} \frac{n_i}{N}\frac{n_i-1}{N}\frac{N-n_i}{N},
\end{split}
\end{equation}
with $r_2'$ proportional to $r_2$. Without affecting the results in the paper, one can ommit the $'$ notation in the above reaction rate (which amounts to relabel $r_2'$ into $r_2$).
Similarly, one obtains:
\begin{equation}
\begin{split}
T(n_i-1 \vert n_{i}) &\, \sim \, 2 r_1 \frac{p_1}{\Omega} \frac{n_i}{N}\frac{N-n_i}{N-1} \, \sim \, 2 r_1 \frac{p_1}{\Omega} \frac{N}{N-1} \frac{n_i}{N}\frac{N-n_i}{N}\\
&\equiv \frac{r_1'}{\Omega} \frac{n_i}{N}\frac{N-n_i}{N},
\end{split}
\end{equation}
with $r_1'$ proportional to $r_1$. As before, one can ommit the $'$ notation. With probability $1-p_1-p_2$, the reaction will correspond to the displacement of an agent between neighboring sites. Since the probability to select node $v_i$ and one of its neighbors $v_j$ is given by $\frac{1}{\Omega k}$ with $k=4$ (each node has four nearest neighbors), we obtain:
\begin{equation}
\begin{split}
&T(n_{i}-1,n_{j}+1 \vert n_i, n_j) \, \sim \, \delta (1-p_1-p_2) \frac{1}{\Omega k} \frac{n_i}{N} g\Big(1-\frac{n_j}{N}\Big) \equiv \frac{\delta'}{\Omega k} \frac{n_i}{N} g\Big(1-\frac{n_j}{N}\Big),\\
&T(n_{i}+1,n_{j}-1 \vert n_i, n_j) \, \sim \, \delta (1-p_1-p_2) \frac{1}{\Omega k} \frac{n_j}{N} g\Big(1-\frac{n_i}{N}\Big) \equiv \frac{\delta'}{\Omega k} \frac{n_j}{N} g\Big(1-\frac{n_i}{N}\Big).
\end{split}
\label{eqA:ratesDiff}
\end{equation}
Again, one can relabel $\delta'$ into $\delta$.

One can thus rewrite the master equation as follows:
\begin{equation}
\begin{split}
\frac{ d \mathrm{P}(\mathbf{n},t) }{dt}=
  &\sum_{i}
  \Big[
		 T(n_i \vert n_i+1)\mathrm{P}(n_i+1)
 		+  T(n_i \vert n_i-1)\mathrm{P}(n_i-1)
 		-  T(n_i+1 \vert n_{i})\mathrm{P}(n_i)
 		-  T(n_i-1 \vert n_{i})\mathrm{P}(n_i)
 \Big]\\
 + &\sum_{i} \sum_{j \in \mathcal{N}(i)}
 \Big[
 	T(n_i,n_j \vert n_i+1,n_j-1) P(n_i+1,n_j-1)
 	+ T(n_i,n_j \vert n_i-1,n_j+1) P(n_i-1,n_j+1)
 \Big]\\
 - &\sum_{i} \sum_{j \in \mathcal{N}(i)}
 \Big[
 	 T(n_i-1,n_j+1 \vert n_i, n_j) P(n_i, n_j)
 	+ T(n_i+1,n_j-1 \vert n_i, n_j) P(n_i, n_j)
 \Big],
\end{split}
\label{eq:MEQ2}
\end{equation}
where $\mathcal{N}(i)$ denotes the set of (nearest) neighbors of node $i$. For the sake of clarity we only highlighted the entries corresponding to the site(s) involved in the reaction. For instance $\mathrm{P}(n_i-1)$ is the probability that the state of the system at time $t$ is given by $\mathbf{n'}= (n_1,n_2,\cdots,n_{i-1},n_i-1,n_{i+1},\cdots,n_\Omega)$.

\subsection{Details of the averaging method and mass conservation}
Let us now denote by $\langle n_i \rangle$ the average number of agents within node $v_i$, where the average is perfomed over all the stochastic realizations of the system.
Starting from the master equation (\ref{eq:MEQ2}), the time evolution of $\langle n_i \rangle$ is given by:
\begin{equation}
\begin{split}
\frac{d \langle n_i \rangle}{d\tau } = 
\Bigr \langle T(n_i+1\vert n_i) \Bigr \rangle
- \Bigr \langle T(n_i-1\vert n_i) \Bigr \rangle
+ \sum_{j \in \mathcal{N}(i)} \Bigr \langle T(n_i+1, n_j-1\vert n_i, n_j) \Bigr \rangle
- \sum_{j \in \mathcal{N}(i)} \Bigr \langle T(n_j+1, n_i-1\vert n_i, n_j) \Bigr \rangle.
\end{split}
\end{equation}
%
Let us then substitute the transition probabilities by their expressions given in Eq.~\eqref{eqA:rates} and let us take the limit $N\rightarrow + \infty$. Upon rescaling of the time $t=\frac{\tau}{N}$, one finds:
\begin{equation}
\begin{split}
\frac{d \langle \frac{n_i}{N} \rangle}{dt} = 
- \frac{r_1}{\Omega} \Bigr \langle \frac{n_i}{N} \Bigr \rangle \Bigr \langle 1-\frac{n_i}{N} \Bigr \rangle 
+ \frac{r_2}{\Omega} {\Bigr \langle \frac{n_i}{N} \Bigr \rangle}^2 \Bigr  \langle 1-\frac{n_i}{N} \Bigr \rangle 
+ \frac{\delta}{\Omega} \sum_{j \in \mathcal{N}(i)} \frac{1}{k}\Bigr \langle \frac{n_j}{N} \Bigr \rangle g\Big( \Bigr \langle \frac{n_i}{N} \Bigr \rangle \Big)
- \frac{\delta}{\Omega} \sum_{j \in \mathcal{N}(i)} \frac{1}{k}\Bigr \langle \frac{n_i}{N} \Bigr \rangle  g\Big(\Bigr \langle \frac{n_j}{N} \Bigr \rangle \Big)\, .
\end{split}
\end{equation}  
Recalling the definition $\rho_i = \lim_{N\rightarrow +\infty} \Bigr \langle \frac{n_i}{N} \Bigr \rangle$, one obtains:
\begin{equation}
\begin{split}
\dot{\rho_i}=  \frac{r_2}{\Omega} \rho_i (1-\rho_i)(\rho_i - r_1/r_2)
+ \frac{\delta}{\Omega k} \sum_{j \in \mathcal{N}(i)} [\rho_j g(\rho_i)-\rho_ig(\rho_j)], ~i=1,\cdots,\Omega.
\label{eq:ODEs}
\end{split}
\end{equation}
The reaction part is a cubic polynomial in $\rho_i$, modeling the Allee effect. One immediately sees that any state in which sites either are fully occupied, i.e., $\rho_i^*=1$, or fully empty, i.e., $\rho_i^*=0$ will be a fixed point of the system. There are in total $2^\Omega$ of such fixed points, each of them being (locally) stable as shown in Appendix~\ref{app:stabilityMatrix}. %\JF{[So far, the computation is restricted to the case $b\geq 1, a>1$]}. 
Taking the continuum limit, i.e., sending $\Omega \rightarrow + \infty$ while keeping the size of the domain fixed, leads to the following partial differential equation governing the spatio-temporal evolution of the vegetation density $\rho$ 
\begin{equation}
\frac{\partial \rho}{\partial t}=
r\, \rho(1-\rho)(\rho-A) 
+ D \, \Big[
g\Delta \rho - \rho \Delta g	
\Big]\, ,
\label{eqA:PDE}
\end{equation}
where we have defined the positive and bounded quantities $r=\lim_{\Omega\rightarrow \infty}r_2/\Omega$ and $D=\lim_{\Omega\rightarrow \infty}\delta/\Omega$ and we assume $0<A:=r_1/r_2<1$.
%Periodic boundary conditions translate into $\rho(x=0,t)=\rho(x=1,t)$ and $\rho(y=0,t) = \rho(y=1,t)$
In the above expression, $\rho \equiv \rho(\mathbf{x},t)$ is defined on the square domain $\mathcal{R}=[0,1] \times [0,1]$ with periodic boundary conditions, $\mathbf{x}\equiv (x,y)$ and $\Delta \equiv \frac{\partial }{\partial x^2} + \frac{\partial }{\partial y^2}$.
The nonlinear diffusion (second term of the r.h.s.) preserves the total mass $M=\int_\mathcal{R} \rho(\mathbf{x},t) d\mathbf{x}$. Indeed, by assuming $r=0$, one has:
\begin{equation}
\frac{d M}{dt} = D \int_\mathcal{R} (g\Delta \rho - \rho \Delta g) d\mathbf{x}=0,
\end{equation}
as follows upon integrating by parts and using the assumption of periodic boundary conditions.


\section{Fixed points of the system and their stability}\label{app:fixedpoints}
In this appendix, we assume $g(x)=x^a(1-x)$ with $a>1$ and investigate the stability of the fixed points of the ODE system given by~\eqref{eq:ODEs}, namely (upon relabelling),
\begin{equation}
\dot{\rho_i}= r \rho_i (1-\rho_i)(\rho_i - A) + \frac{D}{4} \sum_{j \in \mathcal{N}(i)} [\rho_j g(\rho_i)-\rho_ig(\rho_j)], ~i=1,\cdots,\Omega \, .
\end{equation}
Let us first consider the homogeneous fixed points of the system. We will denote by $\rho_i^*$ the stationary density within site $v_i$. There are in total three distinct homogeneous states, corresponding to $\rho_i^*=\{0,1,A\}$ for all $i$. A linear stability analysis, see hereafter for more details, shows that the homogeneous state $\rho_i^*=A$ is unstable while the two others are (locally) stable.

Any configuration in which stationary sites' densities are equal to $0$ or $1$ will be a fixed point of the system. There are $2^\Omega$ of such fixed points, including the two homogeneous states $\rho_i^*=\{0,1\}$  for all $i$. To determine their local stability, let us compute the Jacobian matrix $\mathbf{J}$ of the system. The $(i,j)$ element of this matrix is given by:
\begin{equation}
J_{ij} = \frac{A_{ij}}{4}\Big(g(\rho_i)-\rho_i g'(\rho_j)\Big) + \delta_{ij}\Big[f'(\rho_i)+ \sum_l \frac{A_{il}}{4} \Big(\rho_l g'(\rho_i)-g(\rho_l)\Big)\Big],
\end{equation}
with $A_{ij}=1$ if nodes $i$ and $j$ are nearest neighbors ($A_{ij}=0$ otherwise).
Since $g(x)=x^a(1-x)$ (with $a>1$), it follows that $g(0)=0=g(1)$ and thus:
\begin{equation}
J_{ij}(\rho_k^*\in \{0,1\}) = -\frac{A_{ij}}{4} \rho_i^* g'(\rho_j^*) + \delta_{ij}\Big[f'(\rho_i^*)+ \sum_l \frac{A_{il}}{4}\, \rho_l^* g'(\rho_i^*)\Big],
\end{equation}
with $g'(x) = x^{a-1}\big[a-x(a+1)\Big]$. In particular, one has $g'(0)=0$ and $g'(1)=-1$. 

Let $i\in\{1,\cdots,\Omega\}$ be arbitrarily fixed. If $\rho_i^*=0$, then $J_{ij}=\delta_{ij}f'(0)$ for all $j=1,\cdots,\Omega$,  while if $\rho_i^*=1$, $J_{ij} = -\frac{A_{ij}}{4}g'(\rho_j^*) + \delta_{ij}\Big[f'(1)- \sum_l \frac{A_{il}}{4}\, \rho_l^*\Big]$, for all $j = 1,\cdots,\Omega$. Let us observe that $\sum_{j\neq i} \vert \frac{A_{ij}}{4}g'(\rho_j^*) \vert = \sum_{j\neq i} \frac{A_{ij}}{4}\rho_j^*$. By Gershgorin's theorem, we know that all the eigenvalues fall within the union of discs centerd at $J_{ii}$ and of radius $R_i = \sum_{j\neq i}\vert J_{ij} \vert$. Since $f'(0)<0$,  we thus deduce that all the eigenvalues lie in the complex half plane and hence the fixed point is stable.

\section{Self-segregation in the slow-fast limit}\label{app:stabilityMatrix}
In this section, we consider the limit $r\rightarrow 0$. In this case, the dynamical system boils down to the following equation:
\begin{equation}
\dot{\rho_i}= \sum_{j} \mathcal{L}_{ij} [\rho_j g(\rho_i)-\rho_ig(\rho_j)], ~i=1,\cdots,\Omega \, ,
\label{eqA:ODEs}
\end{equation}
with $\mathcal{L}_{ij} = \frac{A_{ij}}{k}-\delta_{ij}$. Following a linear stability analysis, see Appendix~\ref{app:fixedpoints}, we obtain that the homogeneous state $\rho_i^*=\beta$ is stable if and only if $\beta > \beta_c$, with
\begin{equation}
\beta_c = \frac{a-1}{a}\, .
\end{equation}  
Below this critical threshold, empty nodes emerge, as shown in Fig.~\ref{fig:Diffusion} where we report the stationary densities for $g(x)=x^2(1-x)$ and an average density $\beta=0.3$ (left panel) and $\beta=0.4$ (right panel). Further numerical analysis, see Fig.~\ref{fig:PLDiffusion}, indicates that the distribution of the cluster sizes is well-described by a power-law with an exponential cut-off, suggesting that the segregation of vegetation in clusters is driven by the diffusion.

% Figure environment removed 

% Figure environment removed 

 
\end{document}







