\section{Introduction}\label{sec:intro}
The RISC-V Instruction Set Architecture (ISA)\footnote{\url{https://riscv.org}} 
is a novel open Reduced
Instruction Set Computer (RISC) ISA, which is often  
used for embedded systems. While RISC ISAs innately have a
smaller attack surface than Complex Instruction Set Computer (CISC) ISAs, many
of them run critical systems, including industrial control systems 
%(ICS) 
or
cyber-physical systems,
%(CPS), 
whose failure may have dramatic consequences
(including environmental disasters and loss of human lives).
Using a novel open ISA brings several benefits. Its novelty brings
security advantages by taking past security failures into experience. Even more
important is the open status, where trust in the architecture relies on
community review. This also enables national independence in microchip supplies,
a very important feature as target systems may be strategical, and export
restrictions become more common.

While most RISC-V architectures offer a satisfying level of security compared to
similar classes of systems, they will increasingly become the target to complex
attacks as their relevance in the industrial and strategical field increases.
Eventually, state-backed attackers are deemed to attack them.
In order to anticipate this threat, security researchers face the challenge to
anticipate potential vulnerabilities and imagine suitable protection mechanisms.
Code-Reuse Attacks (CRA), and specifically those based on Jump-Oriented Programming (JOP), are
among the most complex attacks to realize, but also to prevent. They can be very
powerful when successful, as they can allow the attacker to run an arbitrary
sequence of instructions within the corrupted application. In this article we
adopt the attacker's point of view and try to perform a JOP attack, with the
intent of (1) better understanding the vulnerabilities of RISC-V systems, and
(2) ultimately designing better countermeasures to prevent these attacks.

\emph{Contributions.} We summarize our contributions as follows:
\begin{itemize}
\item a first analysis of vulnerabilities to JOP attacks on RISC-V architecture;
\item a description of new dispatcher gadgets enabling JOP attacks to bypass modern mitigations on RISC-V architecture;
\item a demonstration of feasibility by implementing and testing a stealth JOP attack on a vulnerable RISC-V application.
\end{itemize}

\emph{Outline.}
Section~\ref{sec:background} introduces code-reuse attacks, countermeasures
against them and the limitations of the latter. Section~\ref{sec:riscv-jop}
briefly describes the \mbox{RISC-V} ISA attack surface regarding JOP attacks, and
introduces a new kind of gadgets, increasing gadget availability above the level
of ROP attacks. Section~\ref{sec:experiment} describes an attack we developed
against a vulnerable RISC-V application using techniques described in previous
sections, and how we use them to reach a new property for CRA, stealth.
Section~\ref{sec:related} compares our approach to other efforts related to
Jump-Oriented Programming and RISC-V security. Finally,
Section~\ref{sec:conclusion} provides a conclusion.
