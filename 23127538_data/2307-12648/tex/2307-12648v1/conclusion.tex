\section{Conclusion}\label{sec:conclusion}

%\todoog[inline]{Anticipating security vulnerabilities for RISC-V systems
%in order to identify and prevent possible attacks is an important challenge.
%Building attacks is a necessary step to test application's vulnerabilities, 
%as adversary actors (\textit{black hat} hackers) will eventually find them 
%out. In this article, we contributed by expeding the feasability and 
%practicability of JOP attacks, allowing for more extensive security testing.}
%
%\todoog[inline]{Pas sur a 100\% de l'angle d'attaque. Donnez-moi votre avis.}
%
%Athough they exist for more than a decade and have the ability to overcome 
%the shadow stacks, JOP attacks are not widespreads, mainly because of the 
%difficult to implement them. While a few tools are able to find functional 
%JOP gagdet in RISC-V, none is able to actually build a full gadget chain.
%In this article, we presented the ADG, a new variant of the dispatcher gadget
%which increases greatly the number of available functional JOP gadgets, 
%making them at least as common as ROP gadgets, thus increasing greatly the 
%feasability of JOP attacks on software running on RISC-V chips.
%
%We demonstrated this approach on Mangoose, an application commonly used in 
%embedded critical systems. After adding a single memory vulnerability, we 
%were able to take control of in order to perform an adversary action 
%(sending a private key to a remote attacker). Thank to the large number of
%functional gadget available through the use of the ADG, we were able to make 
%the attack stealthy, be restoring the nominal behaviour of the application
%after the attack is completed -- a property that we did not found in previous
%Code Reuse Attacks.
%
%New challenges to increase practicability of JOP attacks include assistance in 
%gagdet finding and even automated chain building. There is very impressive 
%state of the art in ROP chain building~\cite{Vishnyakov21}, that would be a good 
%basis to buid up automated testing frameworks for RISC-V applications' 
%vulnerabilities to JOP.
%
%\tododgp[inline]{My proposition mixing all the text above.
%Note: there are also some text corrections.}

Anticipating security vulnerabilities for RISC-V systems in order to identify
and prevent possible attacks is an important challenge.
Building attacks is a necessary step to test platforms and detect application
vulnerabilities, as adversary actors (\textit{black hat} hackers) will
eventually find them out.
In this article, we contribute by demonstrating the feasibility and
a practical way to realize {jump-oriented programming} (JOP) attacks, allowing for
more extensive security testing.

We have introduced a new variant of dispatcher gadget, the \emph{autonomous
dispatcher gadget} (ADG), which greatly improves the RISC-V JOP attack surface
by enabling the use of ROP gadgets.
While its rigorous validation against a CVA6 implementing a shadow stack is left as future work,
we are convinced that it will be able to bypass shadow stack mitigation.
%while~--- theoretically\tododgp{maybe worth precising ``theoretically, because we could validate it against a CVA6 implementing a shadow stack'', no?} --- still bypassing the
%shadow stack mitigation.
%\todonk{In general, a bad idea to state smth not established in scientific paper. 
%I proposed a more careful wording}

We have demonstrated a JOP attack on a RISC-V platform using a real world
application, the Mongoose web server,  commonly used in embedded
critical systems.
After adding a single memory vulnerability, we 
were able to take control of the application in order to perform an adversary
action, sending a private key to a remote attacker.
Thanks to the large number of functional gadgets available through the use of
the ADG, we were able to make the attack stealthy, by restoring the nominal
behavior of the application after the attack is completed~--- a property that
we did not found in previous code-reuse attacks.

New challenges to increase practicability of JOP attacks include assistance in 
gagdet finding and even automated chain building.
There is a very impressive body of research on ROP chain
building~\cite{Vishnyakov21}, that would be a good basis to build up automated
testing frameworks for RISC-V application vulnerabilities to JOP.
Likewise, studies like the one presented in this article will enable the
development of better and more efficient countermeasures for the RISC-V
architecture against JOP attacks and enhance {control-flow integrity}
%(CFI) 
in general.
