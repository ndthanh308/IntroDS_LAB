%  LaTeX support: latex@mdpi.com 
%  For support, please attach all files needed for compiling as well as the log file, and specify your operating system, LaTeX version, and LaTeX editor.

%=================================================================
\documentclass[universe,review,submit,pdftex,moreauthors]{Definitions/mdpi} 

%--------------------
% Class Options:
%--------------------
%----------
% journal
%----------
% Choose between the following MDPI journals:
% acoustics, actuators, addictions, admsci, adolescents, aerobiology, aerospace, agriculture, agriengineering, agrochemicals, agronomy, ai, air, algorithms, allergies, alloys, analytica, analytics, anatomia, animals, antibiotics, antibodies, antioxidants, applbiosci, appliedchem, appliedmath, applmech, applmicrobiol, applnano, applsci, aquacj, architecture, arm, arthropoda, arts, asc, asi, astronomy, atmosphere, atoms, audiolres, automation, axioms, bacteria, batteries, bdcc, behavsci, beverages, biochem, bioengineering, biologics, biology, biomass, biomechanics, biomed, biomedicines, biomedinformatics, biomimetics, biomolecules, biophysica, biosensors, biotech, birds, bloods, blsf, brainsci, breath, buildings, businesses, cancers, carbon, cardiogenetics, catalysts, cells, ceramics, challenges, chemengineering, chemistry, chemosensors, chemproc, children, chips, cimb, civileng, cleantechnol, climate, clinpract, clockssleep, cmd, coasts, coatings, colloids, colorants, commodities, compounds, computation, computers, condensedmatter, conservation, constrmater, cosmetics, covid, crops, cryptography, crystals, csmf, ctn, curroncol, cyber, dairy, data, ddc, dentistry, dermato, dermatopathology, designs, devices, diabetology, diagnostics, dietetics, digital, disabilities, diseases, diversity, dna, drones, dynamics, earth, ebj, ecologies, econometrics, economies, education, ejihpe, electricity, electrochem, electronicmat, electronics, encyclopedia, endocrines, energies, eng, engproc, entomology, entropy, environments, environsciproc, epidemiologia, epigenomes, est, fermentation, fibers, fintech, fire, fishes, fluids, foods, forecasting, forensicsci, forests, foundations, fractalfract, fuels, future, futureinternet, futurepharmacol, futurephys, futuretransp, galaxies, games, gases, gastroent, gastrointestdisord, gels, genealogy, genes, geographies, geohazards, geomatics, geosciences, geotechnics, geriatrics, grasses, gucdd, hazardousmatters, healthcare, hearts, hemato, hematolrep, heritage, higheredu, highthroughput, histories, horticulturae, hospitals, humanities, humans, hydrobiology, hydrogen, hydrology, hygiene, idr, ijerph, ijfs, ijgi, ijms, ijns, ijpb, ijtm, ijtpp, ime, immuno, informatics, information, infrastructures, inorganics, insects, instruments, inventions, iot, j, jal, jcdd, jcm, jcp, jcs, jcto, jdb, jeta, jfb, jfmk, jimaging, jintelligence, jlpea, jmmp, jmp, jmse, jne, jnt, jof, joitmc, jor, journalmedia, jox, jpm, jrfm, jsan, jtaer, jvd, jzbg, kidneydial, kinasesphosphatases, knowledge, land, languages, laws, life, liquids, literature, livers, logics, logistics, lubricants, lymphatics, machines, macromol, magnetism, magnetochemistry, make, marinedrugs, materials, materproc, mathematics, mca, measurements, medicina, medicines, medsci, membranes, merits, metabolites, metals, meteorology, methane, metrology, micro, microarrays, microbiolres, micromachines, microorganisms, microplastics, minerals, mining, modelling, molbank, molecules, mps, msf, mti, muscles, nanoenergyadv, nanomanufacturing,\gdef\@continuouspages{yes}} nanomaterials, ncrna, ndt, network, neuroglia, neurolint, neurosci, nitrogen, notspecified, %%nri, nursrep, nutraceuticals, nutrients, obesities, oceans, ohbm, onco, %oncopathology, optics, oral, organics, organoids, osteology, oxygen, parasites, parasitologia, particles, pathogens, pathophysiology, pediatrrep, pharmaceuticals, pharmaceutics, pharmacoepidemiology,\gdef\@ISSN{2813-0618}\gdef\@continuous pharmacy, philosophies, photochem, photonics, phycology, physchem, physics, physiologia, plants, plasma, platforms, pollutants, polymers, polysaccharides, poultry, powders, preprints, proceedings, processes, prosthesis, proteomes, psf, psych, psychiatryint, psychoactives, publications, quantumrep, quaternary, qubs, radiation, reactions, receptors, recycling, regeneration, religions, remotesensing, reports, reprodmed, resources, rheumato, risks, robotics, ruminants, safety, sci, scipharm, sclerosis, seeds, sensors, separations, sexes, signals, sinusitis, skins, smartcities, sna, societies, socsci, software, soilsystems, solar, solids, spectroscj, sports, standards, stats, std, stresses, surfaces, surgeries, suschem, sustainability, symmetry, synbio, systems, targets, taxonomy, technologies, telecom, test, textiles, thalassrep, thermo, tomography, tourismhosp, toxics, toxins, transplantology, transportation, traumacare, traumas, tropicalmed, universe, urbansci, uro, vaccines, vehicles, venereology, vetsci, vibration, virtualworlds, viruses, vision, waste, water, wem, wevj, wind, women, world, youth, zoonoticdis 
% For posting an early version of this manuscript as a preprint, you may use "preprints" as the journal. Changing "submit" to "accept" before posting will remove line numbers.

%---------
% article
%---------
% The default type of manuscript is "article", but can be replaced by: 
% abstract, addendum, article, book, bookreview, briefreport, casereport, comment, commentary, communication, conferenceproceedings, correction, conferencereport, entry, expressionofconcern, extendedabstract, datadescriptor, editorial, essay, erratum, hypothesis, interestingimage, obituary, opinion, projectreport, reply, retraction, review, perspective, protocol, shortnote, studyprotocol, systematicreview, supfile, technicalnote, viewpoint, guidelines, registeredreport, tutorial
% supfile = supplementary materials

%----------
% submit
%----------
% The class option "submit" will be changed to "accept" by the Editorial Office when the paper is accepted. This will only make changes to the frontpage (e.g., the logo of the journal will get visible), the headings, and the copyright information. Also, line numbering will be removed. Journal info and pagination for accepted papers will also be assigned by the Editorial Office.

%------------------
% moreauthors
%------------------
% If there is only one author the class option oneauthor should be used. Otherwise use the class option moreauthors.

%---------
% pdftex
%---------
% The option pdftex is for use with pdfLaTeX. Remove "pdftex" for (1) compiling with LaTeX & dvi2pdf (if eps figures are used) or for (2) compiling with XeLaTeX.

%=================================================================
% MDPI internal commands - do not modify
\firstpage{1} 
\makeatletter 
\setcounter{page}{\@firstpage} 
\makeatother
\pubvolume{1}
\issuenum{1}
\articlenumber{0}
\pubyear{2023}
\copyrightyear{2023}
%\externaleditor{Academic Editor: Firstname Lastname}
\datereceived{ } 
\daterevised{ } % Comment out if no revised date
\dateaccepted{ } 
\datepublished{ } 
%\datecorrected{} % For corrected papers: "Corrected: XXX" date in the original paper.
%\dateretracted{} % For corrected papers: "Retracted: XXX" date in the original paper.
\hreflink{https://doi.org/} % If needed use \linebreak
%\doinum{}
%\pdfoutput=1 % Uncommented for upload to arXiv.org

%=================================================================
% Add packages and commands here. The following packages are loaded in our class file: fontenc, inputenc, calc, indentfirst, fancyhdr, graphicx, epstopdf, lastpage, ifthen, float, amsmath, amssymb, lineno, setspace, enumitem, mathpazo, booktabs, titlesec, etoolbox, tabto, xcolor, colortbl, soul, multirow, microtype, tikz, totcount, changepage, attrib, upgreek, array, tabularx, pbox, ragged2e, tocloft, marginnote, marginfix, enotez, amsthm, natbib, hyperref, cleveref, scrextend, url, geometry, newfloat, caption, draftwatermark, seqsplit
% cleveref: load \crefname definitions after \begin{document}
%\usepackage{mathenv}


\newcommand{\MADGRAPH}{\sc{MadGraph} \rm}
%\renewcommand{\POWHEG}{\sc{POWHEG} \rm}

\newcommand{\mathe}{\mathrm{e}}
\newcommand{\mathi}{\mathrm{i}}
%---------------------------------------------------------------------
%\newcommand{\PT}    {{\rm P}_{\!\!\scriptscriptstyle\rm T}}
%\newcommand{\ET}    {{\rm E}_{\scriptscriptstyle\rm T}}
%\newcommand{\MET}   {\mbox{$\raisebox{.3ex}{$\not$}\ET$}}
%\newcommand{\MET}   {\mbox{$\not \!\! E_T$}}
%\newcommand{\met}{\mbox{\ensuremath{\slash\kern-.7emE_{T}}}}
\newcommand{\qqbar} {q\bar{q}}
%\newcommand{\ppbar} {p\bar{p}}
\newcommand{\ttbar} {t\bar{t}}
\newcommand{\bbar}  {b\bar{b}}
\newcommand{\bbbar} {b\bar{b}}
\newcommand{\ccbar} {c\bar{c}}
\newcommand{\ipb}   { {\rm pb}^{-1} }
\def        \gev    {{\rm GeV/}c{\rm ^2}}
\def        \tev    {{\rm TeV/}c{\rm ^2}}
\def        \gevc   {{\rm GeV/}c}
\def        \dzero  {D\O~}
\def        \st     {{\it single top}}
\def        \St     {{\it Single top}}
\def        \tggy     {{\it taggabilit\'e}}
\def        \Tggy     {{\it Taggabilit\'e}}
\newcommand{\comphep}   {C\rm{omp}\sc{hep}}
\newcommand{\herwig}    {\sc{herwig}}
\newcommand{\pythia}    {\sc{pythia} \rm}
\newcommand{\toprex}    {\sc{TopRex}}
\newcommand{\alpgen}    {\sc{alpgen}}
\newcommand{\singletop} {\rm{SingleTop}}
\newcommand{\reco}      {\sc{reco}}


\newcommand{\relmu}{\mbox{Re} (\hat{\mu}_\mathrm{t})}
\newcommand{\imd}{\mbox{Im} (\hat{d}_\mathrm{t})}
\newcommand{\dphi}{$\left|\Delta \phi_{\ell^+\ell^-}\right|$}



\newcommand{\lumione}    {36 pb$^{-1}$ }
\newcommand{\lumitwo}    {2.3 fb$^{-1}$}
\newcommand{\lumithree}  {4.9 fb$^{-1}$}
\newcommand{\lumifour}   { 20fb$^{-1}$ }
\newcommand{\lumifive}   { Xfb$^{-1}$ }


\providecommand{\ee}{$ee$}
\newcommand{\emu}{$e\mu$}
\newcommand{\mumu}{$\mu\mu$}


\newcommand{\mtw}{$m_T^W$}


\newcommand{\POWHEG} {\textsc{Powheg}\xspace}
\newcommand{\PYTHIA} {\textsc{Pythia}\xspace}
\newcommand{\HERWIG} {\textsc{Herwig}\xspace}
\newcommand{\TAUOLA} {\textsc{TAUOLA}\xspace}

\newcommand{\ie}{{\it i.e.}}
\newcommand{\eg}{{\it e.g.}}
\newcommand{\delphes}{{\sc Delphes \rm }}
\newcommand{\madgraph}{{\sc MadGraph \rm }}
\newcommand{\madanalysis}{{\sc MadAnalysis \rm }}
\newcommand{\feynrules}{{\sc Feyn\-Rules \rm }}




%\newcommand{\PTREL}{\ensuremath{p_{\mathrm{T}}^{rel}}\xspace}

\newcommand{\PTREL}{p$_{T}^{rel}$}

\newcommand{\PTHAT}{\ensuremath{\hat{p}_{\mathrm{T}}}}
\newcommand{\STAMu}{STA-muon\xspace}
\newcommand{\STAMus}{STA-muons\xspace}
\newcommand{\DT}{\ensuremath{\Delta\mathrm{T}}\xspace}
\newcommand{\DXY}{\ensuremath{\Delta\mathrm{XY}}\xspace}
\newcommand{\MV}{\ensuremath{\alpha}}
\newcommand{\JPTCut}{20\xspace}


\newcommand{\invpb}{pb$^{-1}$}
\newcommand{\invfb}{pb$^{-1}$}

\def\met       {$\not\!\!E_T$}
\def\metjes    {$\not\!\!E_T^{\rm JES}$}

\def\bb{$b\bar{b}$}
\def\cc{$c\bar{c}$}
\def\qcdbb{$\mathrm{qcd} \rightarrow b\bar{b}$}
\def\qcdcc{$\mathrm{qcd} \rightarrow c\bar{c}$}
\def\tt{$t\bar{t}$}
\def\zqq{$Z \rightarrow q\bar{q}$}
\def\zcc{$Z \rightarrow c\bar{c}$}
\def\zbb{$Z \rightarrow b\bar{b}$}
\def\zbbmu{$Z \rightarrow b\bar{b} \rightarrow \mu X$}
\def\zccmu{$Z \rightarrow c\bar{c} \rightarrow \mu X$}
\def\zuds{$Z \rightarrow u\bar{u}/d\bar{d}/s\bar{s}$}
\def\wbb{$\mathrm{W} + \mathrm{b}\overline{\mathrm{b}}\;$}
\def\wcc{$\mathrm{W} + \mathrm{c}(\overline{\mathrm{c}})\;$}
\def\ttdilep{$t\bar{t} \rightarrow \ell^+\ell^-\nu_{\ell}\bar{\nu_{\ell}} b\bar{b}$}
\def\ttljets{$t\bar{t} \rightarrow \ell\bar{\nu_{\ell}}jjb\bar{b}$}
\def\ttbar{$t\bar{t}$}
\def\ttbarmu{$t\bar{t} \rightarrow b\bar{b} \rightarrow \mu X$}
\def\dilep{$\ell^+\ell^-\nu_{\ell}\bar{\nu}_{\ell} b\bar{b}$}
\def\ljets{$\ell\bar{\nu}_{\ell} jjb\bar{b}$}
\def\gev{~GeV/$c\;$}
\def\gevZ{~GeV/$c$}
\def\mum{~$\mu$m$\;$}
\def\pt{p$_T\;$}
\def\ptZ{$p_T$}
\def\Et{$E_T\;$}
\def\ET{$E_T\;$}
\def\EtZ{$E_T$}
\def\etajet{$|\eta({\mbox jet})|\;$}
\def\etajetZ{$|\eta({\mbox jet})|$}
\def\ip{$IP\;$}
\def\ipZ{$IP$}
\def\dca{$dca\;$}
\def\prob{${\cal P}_{jet}\;$}
\def\probZ{${\cal P}_{jet}$}
\def\dr{$\Delta R\;$}
\def\pscat{$p_{scat}\;$}
\def\pscatZ{$p_{scat}$}
\def\sip{${\cal S}_{IP}\;$}
%\def\ptrel{$p_{T}^{rel}\;$}
\def\ptrelZ{$p_{Trel}$}
\def\tag{${\cal P}_{jet}^+<$}
%=================================================================
% Please use the following mathematics environments: Theorem, Lemma, Corollary, Proposition, Characterization, Property, Problem, Example, ExamplesandDefinitions, Hypothesis, Remark, Definition, Notation, Assumption
%% For proofs, please use the proof environment (the amsthm package is loaded by the MDPI class).

%=================================================================
% Full title of the paper (Capitalized)
\Title{Single-top quark physics at the LHC: from precision measurements to rare processes and top quark properties}

% MDPI internal command: Title for citation in the left column
\TitleCitation{Single-top quark physics at the LHC: from precision measurements to rare processes and top quark properties}

% Author Orchid ID: enter ID or remove command
\newcommand{\orcidauthorA}{0000-0002-8298-7560} % Add \orcidA{} behind the author's name
\newcommand{\orcidauthorB}{0000-0002-2939-5646} % Add \orcidB{} behind the author's name

% Authors, for the paper (add full first names)
\Author{J\'er\'emy Andrea $^{1}$\orcidA{}, Nicolas Chanon $^{2}$\orcidB{}*}

%\longauthorlist{yes}

% MDPI internal command: Authors, for metadata in PDF
\AuthorNames{J\'er\'emy Andrea, Nicolas Chanon}

% MDPI internal command: Authors, for citation in the left column
\AuthorCitation{Andrea, J.; Chanon, N.}
% If this is a Chicago style journal: Lastname, Firstname, Firstname Lastname, and Firstname Lastname.

% Affiliations / Addresses (Add [1] after \address if there is only one affiliation.)
\address{%
$^{1}$ \quad Universit\'e de Strasbourg, CNRS, IPHC UMR7178, 67000 Strasbourg, France; jeremy.andrea@cern.ch\\
$^{2}$ \quad Univ. Lyon, Univ. Claude Bernard Lyon 1, CNRS/IN2P3, IP2I Lyon, F-69622, Villeurbanne, France; nicolas.pierre.chanon@cern.ch}

% Contact information of the corresponding author
\corres{Correspondence: nicolas.pierre.chanon@cern.ch}

% Current address and/or shared authorship
%\firstnote{Current address: Affiliation 3.} 
%\secondnote{These authors contributed equally to this work.}
% The commands \thirdnote{} till \eighthnote{} are available for further notes

%\simplesumm{} % Simple summary

%\conference{} % An extended version of a conference paper

\preto{\abstractkeywords}{\nolinenumbers}

% Abstract (Do not insert blank lines, i.e. \\) 
%\abstract{A single paragraph of about 200 words maximum. For research articles, abstracts should give a pertinent overview of the work. We strongly encourage authors to use the following style of structured abstracts, but without headings: (1) Background: place the question addressed in a broad context and highlight the purpose of the study; (2) Methods: describe briefly the main methods or treatments applied; (3) Results: summarize the article's main findings; (4) Conclusions: indicate the main conclusions or interpretations. The abstract should be an objective representation of the article, it must not contain results which are not presented and substantiated in the main text and should not exaggerate the main conclusions.}
\abstract{Since the first measurements of single-top quark production at the Tevatron, a tremendous progress has been made at the LHC in the understanding of the main production mechanisms and of the associated production of a top quark with a neutral boson. In this article, we review the measurements of the three main production mechanisms ($t$-channel, $s$-channel and $tW$ production), and of the associated production with a photon, a $Z$ boson, or a Higgs boson. Differential cross sections are measured for several of these processes. The top quark properties that can be measured in single-top quark processes are scrutinized, such as $Wtb$ couplings and couplings of the top quark with neutral bosons, as well as the $W$ boson and top quark polarizations. Perspectives for further measurements at the LHC Run 3 and at the HL-LHC are sketched in conclusions.}

% Keywords
\keyword{top quark; single-top quark production; top quark properties; ATLAS; CMS; LHC} 

% The fields PACS, MSC, and JEL may be left empty or commented out if not applicable
%\PACS{J0101}
%\MSC{}
%\JEL{}

%%%%%%%%%%%%%%%%%%%%%%%%%%%%%%%%%%%%%%%%%%
% Only for the journal Diversity
%\LSID{\url{http://}}

%%%%%%%%%%%%%%%%%%%%%%%%%%%%%%%%%%%%%%%%%%
% Only for the journal Applied Sciences
%\featuredapplication{Authors are encouraged to provide a concise description of the specific application or a potential application of the work. This section is not mandatory.}
%%%%%%%%%%%%%%%%%%%%%%%%%%%%%%%%%%%%%%%%%%

%%%%%%%%%%%%%%%%%%%%%%%%%%%%%%%%%%%%%%%%%%
% Only for the journal Data
%\dataset{DOI number or link to the deposited data set if the data set is published separately. If the data set shall be published as a supplement to this paper, this field will be filled by the journal editors. In this case, please submit the data set as a supplement.}
%\datasetlicense{License under which the data set is made available (CC0, CC-BY, CC-BY-SA, CC-BY-NC, etc.)}

%%%%%%%%%%%%%%%%%%%%%%%%%%%%%%%%%%%%%%%%%%
% Only for the journal Toxins
%\keycontribution{The breakthroughs or highlights of the manuscript. Authors can write one or two sentences to describe the most important part of the paper.}

%%%%%%%%%%%%%%%%%%%%%%%%%%%%%%%%%%%%%%%%%%
% Only for the journal Encyclopedia
%\encyclopediadef{For entry manuscripts only: please provide a brief overview of the entry title instead of an abstract.}

%%%%%%%%%%%%%%%%%%%%%%%%%%%%%%%%%%%%%%%%%%
% Only for the journal Advances in Respiratory Medicine
%\addhighlights{yes}
%\renewcommand{\addhighlights}{%

%\noindent This is an obligatory section in “Advances in Respiratory Medicine”, whose goal is to increase the discoverability and readability of the article via search engines and other scholars. Highlights should not be a copy of the abstract, but a simple text allowing the reader to quickly and simplified find out what the article is about and what can be cited from it. Each of these parts should be devoted up to 2~bullet points.\vspace{3pt}\\
%\textbf{What are the main findings?}
% \begin{itemize}[labelsep=2.5mm,topsep=-3pt]
% \item First bullet.
% \item Second bullet.
% \end{itemize}\vspace{3pt}
%\textbf{What is the implication of the main finding?}
% \begin{itemize}[labelsep=2.5mm,topsep=-3pt]
% \item First bullet.
% \item Second bullet.
% \end{itemize}
%}

%%%%%%%%%%%%%%%%%%%%%%%%%%%%%%%%%%%%%%%%%%
\begin{document}

%%%%%%%%%%%%%%%%%%%%%%%%%%%%%%%%%%%%%%%%%%
%\setcounter{section}{-1} %% Remove this when starting to work on the template.
\section{\label{sec:singletopProd}Introduction}


After the discovery of the top quark~\cite{CDF:discovery,D0:discovery} at the Fermilab Tevatron, the CERN LHC era opened up many possibilities for investigating top quark processes. 
Both at the LHC and at the Tevatron, the processes with the largest cross sections for producing top quarks in proton-proton or proton-antiproton collisions are the \ttbar\ production modes. 
In addition to the \ttbar\ production, which arises from Quantum Chromodynamics (QCD) interactions, top quarks can be singly produced through electroweak interactions: this leads to the so-called single-top quark channels. 
The single-top quark production features many interesting properties owing to the $V-A$ structure of the electroweak interaction. 
It shows specific sensitivities to parton density functions (PDF), to the $V_{tb}$ matrix element of the CKM matrix and $Wtb$ coupling beyond the SM, or top quark polarization, just to mention a few examples. 
Measuring inclusive cross section and differential cross sections for single-top quark processes constitutes as well an interesting test of perturbative QCD (pQCD). 
The associated production of a single-top quark with a boson allows to probe the coupling between the top quark and bosons, in a complementary way with the associated production of a boson with a \ttbar~pair.

Three mains production modes for single-top quark signatures can be distinguished: the production via the exchange of a virtual $W$ boson in the $t$- and $s$-channels, and the associated production with a $W$ boson (tW production). 
The corresponding diagrams at leading order (LO) in pQCD are presented in Fig.~\ref{fig:stopdiag}.

% Figure environment removed

The first observation of single-top electroweak production in the $t$-channel happened at the Tevatron~\cite{D0:tchandiscovery,CDF:tchandiscovery}. The p-$\bar{\mathrm{p}}$ collisions at the Tevatron were providing a unique opportunity for measuring the $s$-channel, since the initial state for this process features a light quark and a light antiquark, taken from the valence partons in the proton and antiproton. As of today, the $s$-channel was observed only at the Tevatron~\cite{CDF:2014uma} and remains to be observed at the LHC, although evidences for this process were already reported at 8 TeV~\cite{ATLAS:2015jmq} and 13 TeV~\cite{ATLAS:2022wfk}.
At the LHC, the largest cross-sections at $\sqrt{s}=$13 TeV (the center-of-mass energy of Run 2) are obtained for the $t-$channel ($214.2^{+4.1}_{-2.6}$ pb at NNLO with MCFM~\cite{Campbell:2020fhf}), followed by the $tW$ production ($79.3^{+2.9}_{-2.8}$ pb at NLO+NNLL~\cite{Kidonakis:2021vob}), and then only the $s-$channel ($10.3^{+0.4}_{-0.4}$ pb at NLO with Hathor v2.1~\cite{Aliev:2010zk,Kant:2014oha}). 

At the LHC, many new processes involving single-top quarks were measured in p-p collisions. The $t$-channel production has been measured multiple times, and is investigated in great details by measuring differential cross sections. Most of the top quark properties probed with the single-top production channels are measured using the $t$-channel production, since it yields the largest cross section at the LHC among all production mechanisms. The LHC was also able to observe the associated $tW$ production, for which the differential cross sections are even measured. This channel is particularly interesting since at next-to-LO (NLO) in pQCD, it features an interference with the \ttbar\ process. Understanding this interference in details is still an active topic in the field. 

In addition, single-top quarks can be produced in association with neutral bosons. Those processes yield relatively low cross sections, however the additional boson decay in the lepton channel is experimentally providing an useful handle for measuring couplings or searching for new physics. This class of processes covers the production of a single-top quark with a photon ($t\gamma$), with a $Z$ boson ($tZ$) or with a Higgs boson ($tH$). For each of these processes, the single-top quark can be produced either via the $t$-channel, $tW$ production or $s$-channel, with the boson emitted from a quark line or from a $W$ boson exchange. The $t\gamma$ process was observed recently, and $tZ$ processes are now measured differentially and used for property measurements. %The coupling of the top quark to the associated boson can indeed be probed using those processes, even if their cross section is smaller than the cross section for the production of a boson with a \ttbar pair. 
The $tH$ processes constitute a special case, since they are searched for at the same time as the $t\bar{t}H$ production. The $tH$ cross section is so small that there was not yet an evidence for this processes. However, it is already considered in several analyses because of  its unique sensitivity to the sign of the top quark Yukawa coupling.

The cross sections for all SM top quark processes measured at ATLAS are compared with theory predictions in Fig.~\ref{fig:ATLAS_TopCrossSection}. The cross sections for single-top production $t$+X are generally lower than that of top pair production $t\bar{t}$+X, which constitute a large background to single-top quark searches. 

A summary of the cross section for single-top quark processes measured at CMS is compared with theory predictions, as a function of the center-of-mass energy, in Fig.~\ref{fig:ATLAS_TopCrossSection}. It can be observed that the cross section for the $s$-channel process is not growing as fast as that of the $t$-channel process as a function of the energy, which makes the search for the $s$-channel more difficult with recent LHC runs. The production of a single-top quark associated with a photon or a $Z$ boson yield cross sections lower than that of the $t$-channel, $s$-channel or $tW$ production.

% Figure environment removed

% Figure environment removed

If we turn now to top quark property measurements, it is interesting to notice first that the cross sections for single-top quark production are proportional to the square of $|V_{tb}|$. It is therefore possible to determine $|V_{tb}|$ from the measurement of single-top quark cross sections. If one assumes that $|V_{td}|, |V_{td}| << |V_{tb}|$, the $|V_{tb}|$ matrix element can be extracted from:
\begin{linenomath}
\begin{equation}
|V_{tb}| = \sqrt{ \sigma_{st}/\sigma_{st} ({theo, |V_{tb}|=1})}, \label{equ:vtb}
\end{equation}
\end{linenomath}
where $\sigma_{st}$  is the measured cross-section and $\sigma_{st}^{theo, V_{tb}=1}$ the expected cross section for $|V_{tb}|=1$. Equation $\ref{equ:vtb}$ also assumes that no new physics effect modifies the $V-A$ structure of the $tWb$ interaction vertex.

In single-top quark processes, the $Wtb$ vertex appears in the top quark production and in its decay, while in \ttbar\ production it appears twice in top quark decay. Therefore, the Lorentz structure of the $Wtb$ coupling can be investigated in details using decay information. 
The single-top quark production is also sensitive to the CP property of the $Wtb$ vertex (much more difficult to measure in \ttbar~production, where the CP property is probed preferentially in the top quark - gluon coupling). The $W$ boson polarization and top quark polarization can also be probed. 
For all of these properties, the $t$-channel process is usually employed as a probe because of its large cross section. Within the $t$-channel, but also in the associated production with a boson, modern tools such as the standard model (SM) effective field theory are employed to parametrize deviations from the standard model in an almost model-independent way.

The outline of this review is as follows. In section~\ref{SingleTopCrossSections}, after a brief note on the generation of single-top quark processes, the measurements of the three main single-top quark production mechanisms are presented: $t$-channel, $tW$ production and $s$-channel. The section~\ref{SingleTopBoson} will discuss the measurements of single-top quark production in association with a neutral boson (a photon, a $Z$ boson or a Higgs boson). Eventually, top quark property measurements with single-top quark production will be reviewed in section~\ref{SingleTopProperties}, with a focus on $V_{tb}$, the $W$ boson and top quark polarization, the structure of the $Wtb$ vertex and the interpretation in term of the SM effective field theory. Conclusions of this review will be given in section~\ref{Conclusions}.


\section{\label{SingleTopCrossSections}Precise and differential measurements of single-top quark processes}


\subsection{\label{t-channel}The $t$-channel process: the large statistics production mode}

\subsubsection{Features of the $t$-channel process}

The so called $t$-channel production mode owns the largest cross section among all single-top quark production modes. Top quarks produced in the $t$-channel are accompanied with a high $p_T$ light quark predominantly produced in the forward region of the detector ($|\eta| > 2.5$), and of a low $p_T$ b-quark often failing the minimum jet $p_T$ requirements in the analysis, and thus remaining experimentally invisible. Feynman diagrams are presented in Fig.~\ref{T-channel:Diagrams}.

% Figure environment removed

The initial $b$ quark is accounted for in the theoretical calculation according to the 5 flavour scheme (5FS) or the 4 flavour scheme (4FS), as shown in the Feynman diagrams of Fig.~\ref{T-channel:Diagrams}. Either one considers a $b$ quark PDF within the proton (5FS); or one can consider that the proton is made only light-flavor quarks in the sea (4FS), in which case the $b$-quark arises from virtual gluons. In the 5FS, the uncertainty associated to PDF can be relatively large, because $b$-quark PDF are not necessarily well known. On the other hand, the 4FS calculations usually suffer from a higher sensitivity to QCD renormalization and factorization scales. The decision whether to employ the 4FS or the 5FS is particularly important for the $t$-channel signature, where the additional jet (the so called recoiling or spectator jet) is relatively forward and its pseudorapidity ($\eta$) distribution is sensitive to the PDF.
It has been observed that the $\eta$ distribution of the recoiling jet ($\eta(j')$) in data is actually better described using the 4FS, while inclusive cross section calculations are more accurately described with the 5FS.

With single-top quark production in the $t-$channel (as well in the $s-$channel), one of the incoming light quarks can be a valence quark of the proton, depending whether a top or an antitop quark is being produced. This leads to a larger cross section for top quark production ($134.2^{+2.6}_{-1.7}$ pb at $\sqrt{s}=$13 TeV, calculated at NNLO with MCFM~\cite{Campbell:2020fhf}) than for antitop quark production ($80.0^{+1.8}_{-1.4}$ pb). 




\subsubsection{Experimental techniques for the $t$-channel measurement}

The $t-$channel process was the first single-top production mode observed at the LHC~\cite{bib:CMSFirstSingleTop,ATLASFirstSingleTop}, thanks to its large cross section and its manageable signal-over-background ratio. For this channel, there exists also clear discriminating observables between signal and background, such as the $\eta(j')$ distribution.
Most of the $t$-channel analyses share a lot of common, either on the event selection, background estimation, separation of the signal from the backgrounds, or in the signal extraction.
The following paragraphs are aiming at providing a general description of the analysis methods applied in $t$-channel measurements, and are also valid to a large extent for the other single-top quark measurements discussed in this paper.

The top quark decays at almost 100\% to a $W$ boson and a b-quark, therefore top quark decays can be either leptonic ($t\rightarrow b l \nu$) or hadronic ($t\rightarrow b q q'$). The hadronic decay of top quarks produced in the single-top $t$-channel leads to a signature with several jets, thus suffering from an overwhelming QCD multijet background. For this reason, only the leptonic decay of the top quark is usually studied. The experimental signature for the analyses presented here is targeting the leptonic decay products from the $W$ bosons: a charged lepton (electron or muons potentially arising from tau lepton decay), and the presence of a significant missing transverse energy \met\ originating from a neutrino. Leptons are accompanied with a mainly forward light-quark jet and a b-quark jet arising from the top quark decay.

The data sample considered is usually selecting events with a trigger requiring at least one lepton with a large $p_T$ and isolated from hadronic activity. The usage of b-quark identification ("b-tagging") at trigger level was investigated in early analyses~\cite{tchanBtagTrigger}, but found to add a significant complexity for a limited gain, especially with increasing luminosity. 
In a short summary, the event selection typically requires the presence of only one high $p_T$ charged lepton (electron or muon with $p_T>20$ GeV), a significant missing transverse energy (\met$>$ 30 GeV), the presence of at least two high $p_T$ jets ($p_T>$30 GeV), one of them being identified as arising from a b-quark and the other  failing this requirement while being possibly detected in the forward region ($|\eta(j')|<4.7$).

%As in several single lepton signatures, 
The backgrounds can be classified as arising from two main sources: events containing a charged lepton produced from a boson decay (referred to as prompt lepton), and events with hadronic objects mis-identified as prompt leptons. Prompt leptons standing away from large hadronic activities, and non-prompt leptons being surrounded by hadrons, a very powerful mean of rejecting the overwhelming QCD multijet background consists in requiring the charged lepton to be isolated. An isolation variable is constructed by summing the hadronic energy around charged leptons and required to be small. As the modeling of non-prompt background is hardly well simulated, non-prompt backgrounds are usually estimated directly from the data, leading possibly to large systematic uncertainties. 
This estimate is performed, for instance, by inverting the lepton isolation requirement, thus enriching the events in QCD multijet processes. The shape of a distribution of interest is then used as a data-driven estimate of the non-prompt lepton background. 

The major prompt lepton background events are chiefly arising from \ttbar\ production with semi-leptonic decays, where jets are not well reconstructed or are not passing the b-tagging requirements. The \ttbar\ process has been extensively studied; precise measurements have been confronted to theoretical predictions. This process is very well described by the state of the art Monte-Carlo (MC) generators, such that single-top measurements rely on simulation to describe \ttbar\ kinematics, while its normalization is usually estimated or constrained from data.

The associated production of a single $W$ boson with additional jets, referred to as "$W$+jets" in the following, constitutes the second main source of background events. The $W+jets$ processes have been measured at the LHC, and the event kinematics shows a good agreement between data and MC predictions. However, the kinematics of $W$+jets process varies slightly depending on the flavor of the additional jets. For this reason, several analyses actually split the $W+jets$ simulation into $W$+b, c or light jets, measuring the normalization if each contribution separately.

Finally, other subdominant processes after selection are contributing to the background events, such as the Drell-Yan production when one of the two leptons is not reconstructed or not passing the lepton selection. These processes are usually estimated from simulations.

The $t$-channel analyses benefited from the rise of the LHC profile likelihood method~\cite{plr2,CMS-NOTE-2011-005} to, simultaneously, estimate the background contributions and constrain the systematic uncertainties from the data. 
Background normalizations are constrained in the fit, possibly using control regions enriched in background events, usually defined by jet and b-tagged jets multiplicities. 
For instance, the \ttbar\ background can be controlled by fitting events with at least 3 jets and 2 b-tagged jets ($3j,2t$).
The $W$+jets background can be controlled with events containing 2 jets and no b-tagged jet ($2j0t$), using the distribution in the transverse mass of the $W$ boson ($m_T(W)$), showing a broad resonance for $W$ bosons, as shown in Fig.~\ref{fig:t-channel_mTW_BDT_CMS}.
The signal events are mainly expected in the region defined by asking for 2 jets and among which one b-tagged jet ($2j1t$). 
The signal is extracted from a combined fit in the ($3j,2t$), ($2j0t$) and the ($2j1t$) regions.
Discriminating observables in each of these regions are fitted together with common nuisance parameters representing the systematic uncertainties. 

Several  distributions can be used to discriminate signal from backgrounds. In the early versions of analyses, the most obvious observables used were the pseudo-rapidity of the recoiling jet or the reconstructed top-quark mass. 
In the most precise measurements, the discriminating variables in the ($3j,2t$) and ($2j1t$) regions are constructed from multivariate analyses, such as Boosted Decision Trees (BDT) or Neural Networks (NN), using various kinematic observables as input. An example is shown in Fig.~\ref{fig:t-channel_mTW_BDT_CMS}.
In the latest published cross section measurement at 13 TeV~\cite{Sirunyan_2020_TOP_17_023}, the BDT are trained using input variables related to the the absolute value of the pseudorapidity of the untagged jet, $|\eta(j')|$, the reconstructed top quark mass, the transverse $W$ boson mass, $m_T(W)$, the distance in $\eta-\phi$ space between the b-tagged and the untagged jet, $\Delta R(b, j')$, the absolute difference in pseudo-rapidity between the b-tagged jet used to reconstruct the top quark and the selected lepton, $|\Delta \eta(b, l)|$. 

% Figure environment removed

Thanks to the large amount of integrated luminosity collected at the LHC, the uncertainties  related to the $t$-channel measurements are no longer statistically dominated. 
Remarkably, one can even select a relatively pure sample of $t$-channel events by applying a stringent requirement on the BDT discriminants, as illustrated in 
Fig.\ref{fig:tcahdiff_cossthetastar} (taken from~\cite{Sirunyan_2020_TOP_17_023}), showing the distribution in the cosine of the top quark polarization angle $\cos{\theta^*}$ in background-enriched region (low $BDT_{t-ch}$ requirement) and in a signal-enriched region (high $BDT_{t-ch}$ requirement). The sample can be vastly enriched in signal events while still providing a large event yield.


% Figure environment removed


The main sources of systematic uncertainties impacting $t$-channel measurements can be summarized as follows:
\begin{itemize}
	\item Integrated luminosity: typically of a few percents (depending on the dataset),
	\item Signal and background modeling from SM theory predictions: uncertainties in the modeling of signal acceptance and in the modeling of distributions used as discriminant observables are usually a major source of systematic uncertainty in top quark physics. This includes renormalization and factorization scale variations (accounting for missing higher order contributions in pQCD), parton-shower and hadronization, PDF, the choice of the matching scheme between fixed order predictions and the parton shower, the choice of the flavour scheme (4FS or 5FS), and MC statistics. These uncertainties are treated by generating various MC samples, or including various event weights in the MC samples, with generation parameters varied up and down. The same uncertainties are also included for most backgrounds which are estimated from simulations. 
	\item Data-driven background estimate: jet mis-identified as leptons being poorly desceibed in the simulation, the non-prompt lepton background is directly estimated from data. 
	Usually, these estimations are complicated and rather imprecise: it is rare to lower the relative systematic uncertainty below 30\%, before any constrains from the fit.
	\item Simulation-to-data corrections: several corrections (so-called scale factors) to reconstructed objects are applied on the simulation to improve its agreement with the data. These corrections are derived from dedicated analyses estimating the associated systematic uncertainties. The corrections are typically related to trigger and lepton selection, jet energy scale and resolution, and b-tagging. 
\end{itemize}

In the most recent analyses, the statistical uncertainty provides a small contribution to the total uncertainty (less than 5\%). 
The relative size of the systematic uncertainties depends on the analysis strategy; for instance, the choice of the discriminating observables matters. 
The use of $|\eta_{j'}|$ distribution naturally leads to large uncertainties on the jet energy scale and resolution (up to about 5\%), since controlling such corrections in the forward part of the detector is difficult. 
Using a multivariate discriminant allows to significantly reduce the jet energy scale and resolution uncertainties to a couple of percents. 
Another large source of systematic uncertainty is arising from the signal modeling, which can be lowered by performing a fiducial measurement, as described in section~\ref{sec:fiducial}. Fiducial measurements are performed within a generator-level acceptance to avoid the extrapolation from the visible phase space to the full process phase space, thus reducing the modeling uncertainties.

\subsubsection{Summary of the latest measurements of $t$-channel inclusive cross section}

A summary of the latest measurements of the cross section for $t$-channel production at $\sqrt{s}=$7, 8 and 13 TeV from the ATLAS and CMS collaborations can be seen on Tab.\ref{tab:singletop_t-channel_measurements}, where the combinations made by the LHC$top$WG are also shown when available. 
Fig.\ref{fig:tchan_RelativUncertEvolution} shows the relative total uncertainty of the $t$-channel cross section $\Delta \sigma_{t-chan}/\sigma_{t-chan}$ as a function of the integrated luminosity accumulated at the different center-of-mass energies. 
One can observe that, rapidly, the statistical uncertainty becomes a sub-dominant source of uncertainty, precision measurements becoming therefore dominated by systematic uncertainties. 
One can also observe an increase of the total uncertainty with $\sqrt{s}$. Comparing the most precise result measured at 7 TeV~\cite{JHEP05.2019.088} with the published measurement at 13 TeV~\cite{Sirunyan_2020_2}, it becomes clear that while several systematic uncertainties related to detector effects and background estimate have decreased, there is an increase of all the theory uncertainties related to the signal modeling.
While experimental systematic uncertainties can be reduced further, a significant improvement in the total precision of the inclusive $t$-channel cross section requires as well an effort on the signal modeling involving the theory community. 
The most precise 13 TeV measurement has been recently released, as a conference note, by the ATLAS collaboration~\cite{ATLAS-CONF-2023-026}. The $t$-channel cross section has also been measured recently at 5.02 TeV by the ATLAS collaboration~\cite{ATLAS-CONF-2023-033}.


\begin{table}[H] 
\caption{\label{tab:singletop_t-channel_measurements}
Summary of the most recent and precise $t$-channel cross sections from the ATLAS and CMS collaborations, and their combinations for 7 and 8 TeV.}
\begin{tabularx}{\textwidth}{CCC}
\toprule
 	& \textbf{Cross section (pb)}	& \textbf{$\Delta \sigma_{t-chan}/\sigma_{t-chan}$}\\
\midrule
7 TeV   								   &                                        &   \\      

 ATLAS~\cite{PhysRevD.90.112006}               &         $ 68 \pm 2 \pm 8 \pm 1$                              &  0.122 \\    
  
 CMS~\cite{tchanBtagTrigger}           &          $67.2\pm 3.7 \pm 4.6 \pm 1.5$                   & 0.091 \\     

 Combination~\cite{JHEP05.2019.088}                       &           $ 67.5 \pm 2.4 \pm 5.5 \pm 1.1  $              &  0.090\\      

\midrule
8 TeV                             &             								& \\    

 ATLAS~\cite{Aaboud_2017}                &            $89.6 \pm 1.2 ^{+6.8}_{-5.9} \pm 1.7$        &      0.076                  \\    

 CMS.~\cite{Khachatryan_2014}            &           $83.6 \pm 2.3 \pm 7.1 \pm 2.2$                   & 0.093                \\     

 Combination~\cite{JHEP05.2019.088}                      &           $87.7 \pm 1.1 \pm 5.5 \pm 1.5$                 & 0.066 \\      

\midrule
13 TeV                     &                  \\   

 ATLAS~\cite{Aaboud_2017_2}                                   &            $247 \pm 6 \pm 45 \pm5$                     & 0.185          \\    

 ATLAS~\cite{ATLAS-CONF-2023-026}                      &            $221 \pm 13$                     & 0.059        \\    

 CMS~\cite{Sirunyan_2020_2}                                &           $207 \pm 2 \pm 30 \pm 5$                   & 0.147            \\     
\bottomrule
\end{tabularx}
\end{table}



% Figure environment removed  


\subsubsection{Measurements of fiducial and differential cross sections\label{sec:fiducial} }

The most recent measurements of the $t$-channel process, benefiting from the large statistics of the LHC Run 2, are able to investigate differential cross sections~\cite{Sirunyan_2020_TOP_17_023}. Differential cross sections are providing precious information on the theory modeling, and can also be used to constrain the parameters of the Effective Field Theories. Differential measurements are of critical importance for reaching a deeper understanding of the $t$-channel process and to search for any deviations from SM predictions.

The distributions measured at reconstructed level are unfolded to theoretically well-defined observables, by correcting for detector and acceptance effects. 
The basic principle consists in determining corrections from simulation to infer the "true" top quark properties, by accounting for the signal acceptance induced by the selection, for the detector resolution and the efficiencies. The unfolded distributions can be compared in a robust way with theoretical predictions. 
Two fundamental unfolding levels are usually defined in top quark physics:
\begin{itemize}
	\item \bf Parton level: \rm corresponds to the generated on-shell top-quarks, after QCD radiative corrections.
	\item \bf Particle level: \rm corresponds to (pseudo-)top quarks reconstructed from the simulated particles after QED and QCD radiation, particle decays and hadronization, with a dedicated algorithm. 
\end{itemize}

With the definitions adopted in~\cite{Sirunyan_2020_TOP_17_023}, measurements unfolded to parton and particle level are confronted with NLO theory predictions for various observables like the top-quark $p_T$ and rapidity $y$, $\cos{\theta^*}$ or $W$ boson $p_T$. Beyond the differential cross sections, the ratios of the cross section $\sigma_t$ to $\sigma_{t+\bar{t}} $ is also measured. This observable is sensitive to the PDFs. 
Fig.\ref{fig:tchan_unfolded} presents examples of differential cross sections and cross section ratios. The measurements are showing a good agreement between data and SM predictions, validating our understanding of electroweak interactions in the production of single-top quarks.

% Figure environment removed

The so-called fiducial cross section is defined at particle level, and is less sensitive than the inclusive cross section to the systematic uncertainties arising from signal modeling. 
In inclusive cross section measurements, the number of signal events are measured in the visible phase space at reconstructed level, defined by the detector acceptance and selection efficiencies. The observed number of events is then extrapolated to the full phase space based on simulations. This extrapolation induces a large systematic uncertainty related to the modeling of signal events in the simulation. 
By contrast, the measurement of the fiducial cross section is performed in the visible phase space, and extrapolated to the fiducial phase space volume defined as close as possible to the phase space of the selected data set. 
Fiducial single-top $t$-channel cross section has been measured~\cite{Aaboud_2017} at $\sqrt{s}=$8 TeV and led to a reduction of about 2\% points of systematic uncertainties related to the QCD scale and the NLO matching. This results not only in a significant improvement of the precision, it constitutes as well a robust way to compare data with theory calculations.



\subsection{The $tW$ process, and its interplay with the $t\bar{t}$ process}

\subsubsection{Introduction to the $tW$ process}

The $tW$ process features a top quark produced in association with a $W$ boson, either initiated by a gluon and a b-quark (in the 5FS, see Fig.~\ref{tWchannnel:diagram}), or with the b-quark being produced by gluon splitting (in the 4FS). 
Because the PDFs for bottom and antibottom quarks in the proton in the 5FS is assumed to be the same, the predicted cross section for $tW^{-}$ and $\bar{t}W^{+}$ is identical at LO (and almost identical at higher order)~\cite{Kidonakis:2021vob}.

% Figure environment removed

There is some degree of overlap between the $tW$ process and the \ttbar\ process, since the $tW$ production at NLO in pQCD features resonant diagrams which are interfering with LO diagrams of \ttbar\ production.
The NLO corrections to the production of $tW$ include $tWb$ processes, where the $Wb$ system can also arise from the decay of an on-shell top quark. 
Examples of LO Feynman diagrams for $tWb$ processes are shown in Fig.\ref{fig:tWbdiag}. 
Since the cross section for \ttbar\ production is much higher than that of $tW$ production, these corrections are very large. As as result, there is an ambiguity in the way the $tW$+1 jet processes are defined. A similar situation is occurring in the flavour changing neutral current (FCNC) processes~\cite{universe8110609}.

% Figure environment removed

The definition of the $tW$ process therefore relies on the treatment of this interference and presents challenges at theoretical level, depending on the choice of suppressing the interference to define independent simulation samples for the $tW$ process at NLO, or including it in the simulation in a consistent way between $tW$ and \ttbar\ process.
Two methods exist to suppress the interference~\cite{tWIsolating}. In the diagram removal (DR) method, the resonant \ttbar\ diagrams are excluded at the level of the matrix element calculation. In the diagram subtraction (DS) approach, the \ttbar\ resonant contributions are removed from the cross-sections calculation by the mean of counter terms. Thus, a comparison of the DR and DS prediction gives an estimation of the importance of the interference terms and its treatment, which is found to be small for the usual kinematic selection applied~\cite{WtbAguilar}.



\subsubsection{Measurements of the $tW$ process}

The $tW$ process has not been measured at the Tevatron, its cross section being small at the Tevatron center-of-mass energy in $p-\bar{p}$ collisions. 
The measurement of the $tW$ process is more challenging than that of the $t$-channel, since a very large background from \ttbar\ events mimics the signal with almost the same experimental signature.
The ATLAS and CMS collaborations presented evidence for this process in the dilepton channel at 7 TeV~\cite{ATLAS:2012bqt,CMS:2012pxd}, while the inclusive cross section was measured at 8 TeV~\cite{ATLAS:2015igu,CMS:2014fut} and 13 TeV~\cite{ATLAS:2016ofl,CMS:2018amb}. The measurements at 13 TeV performed with a larger collected data sample allowed as well to measure the differential cross sections~\cite{ATLAS:2017quy,CMS:2022ytw} for the first time. 
Lately, the $tW$ process was measured in the more difficult lepton+jet channel at ATLAS using 8 TeV collisions~\cite{ATLAS:2020cwj} and at CMS using 13 TeV collisions~\cite{CMS:2021vqm}.

The dilepton decay channel for the $tW$ process refers to processes where one lepton arises from the top quark decay through $Wb$ and another lepton is produced by the associated $W$ boson decay. We describe here features of the ATLAS~\cite{ATLAS:2017quy} and CMS~\cite{CMS:2022ytw} analyses measuring differential cross sections at 13 TeV with the dilepton channel, where the leptons refer to electrons or muons. 
Nominal SM predictions for the $tW$ process use the DR scheme. 
For this analysis, the main background contribution after event selection is the $t\bar{t}$ process in the dilepton decay channel, amounting to nearly 80\% of the event yield after selection. The signal region is defined with exactly one reconstructed jet being tagged as a b jet (so-called $1j1b$ region), to remove contribution from doubly resonant diagrams. 
In general, a selection on the transverse missing energy does not need to be applied (among recent measurements, the ATLAS 13 TeV inclusive cross section measurement~\cite{ATLAS:2016ofl} is an exception), this variable is used to reconstruct kinematic quantities and provided as input to machine learning techniques. 
The Fig.~\ref{tWchannnel:categories} shows the number of events after selection, sorted in bins of the number of jets and b jets. 
Two (three) regions defined depending on the number of jets and b jets are used to measure the inclusive cross section at ATLAS (CMS), with dedicated BDTs.

% Figure environment removed

The inclusive cross section at 13 TeV is measured to be $\sigma_{tW} =$ 79.2 $\pm$ 0.9 (stat) $^{+7.7}_{-8.0}$ (syst) $\pm$ 1.2 (lumi) pb at CMS using 138 fb$^{-1}$~\cite{CMS:2022ytw}, and $\sigma_{tW} =$ 94 $\pm$10 (stat.) $^{+28}_{-22}$ (syst.) $\pm$ 2 (lumi.) pb at ATLAS using 3.2 fb$^{-1}$~\cite{ATLAS:2016ofl}, in agreement with SM theory predictions. The dominant systematic uncertainty is the jet energy scale, followed by the background normalization and the theory uncertainties on $tW$ process modeling. 
%\textbf{FIXME: agreement with NLO?}

Both at ATLAS and CMS, the $1j1b$ region is used to extract the differential cross sections. 
In the ATLAS analysis, an additional selection is applied on the output of the BDT to increase the separation between signal and backgrounds for the differential measurement. In CMS a veto on additional loose jets is also applied. The data is corrected for detector effects and compared to theoretical predictions, for instance the invariant mass of the dilepton and b jet in Fig.~\ref{tWchannnel:mllb}. 
In most of the measured bins, data and simulation agree within less than 1$\sigma$, however more data is needed to discriminate between the different ways of modeling the signal.

% Figure environment removed

Measuring the $tW$ process in the lepton+jets channel, which targets a final state where one of the two $W$ boson decays leptonically and the other hadronically, is a challenging task, owing to the prominent $t\bar{t}$ and $W$+jets backgrounds arising from the selection. 
Machine learning methods are used to enhance the signal over background ratio, a NN at ATLAS~\cite{ATLAS:2020cwj} and a BDT at CMS~\cite{CMS:2021vqm}. 
ATLAS extracts the signal using a two-dimensional distribution in the NN output and the invariant mass of the hadronically-decaying $W$, in events with at least three jets (among which one b jet). 
CMS employs the BDT outputs in three regions, whether there are two, three or four jets in the event (among which one b jet).  
The analyses lead to an evidence for $tW$ process in the lepton+jets channel using 8 TeV data at ATLAS, and an observation using 13 TeV data at CMS. The measured inclusive cross sections are in agreement with the SM predictions, and the precision is already dominated by systematic uncertainties.
The main systematic uncertainties arise from jet energy scale, background normalization and $t\bar{t}$ or $tW$ modeling. 

The lepton+jets analysis shows that more difficult channels are nowadays used to measure the $tW$ process. One of the next steps would be to scrutinize the tails of kinematic distributions by using boosted jet tagging, allowing to access highly boosted regions.
Differential distributions are measured with the dilepton channel and will be investigated more differentially in the future. 
Despite having a smaller crosss section than the $t$-channel, the $tW$ process could also be used to measure SM parameters.
Similarly to measurements performed in the $t$-channel, measuring charge ratios would be interesting since they are sensitive to PDFs; it would require separating top from antitop contributions in $tW$ production with advanced techniques like the Matrix Element Method~\cite{Brochet:2018pqf}. 

\subsubsection{Understanding the interference betweem $tW$ and \ttbar\ processes}

While the above-mentioned measurements of $tW$ process are designed to minimize the interference with $t\bar{t}$ process by mostly selecting events with only one b-jet, a recent analysis at ATLAS~\cite{ATLAS:2018ivx} is targeting a phase-space with two b jets where the interference effect is maximized. 
This analysis employs a variable defined as an invariant mass of a lepton and a b jet, as a proxy for the top quark mass. Since there is an ambiguity in the way to assign leptons and b jets to a given top quark, a particular choice is made:
\begin{linenomath}
\begin{equation}m_{b\ell}^{minimax} = min \Big( max(m_{b_1\ell_1}, m_{b_2\ell_2}),max(m_{b_1\ell_2}, m_{b_2\ell_1}) \Big),
\end{equation}
\end{linenomath}
where this variable is defined such that at LO, $m_{b\ell}^{minimax} < \sqrt{m_t^2-m_W^2}$. The cross section above this value has increased sensitivity to the interference between single and double resonant contributions.

Events are selected if there are two leptons and two jets satisfying a tight b-tagging criterion, with a veto on further leptons using a loose requirement (which suppresses backgrounds arising from $t\bar{t}$ associated with heavy flavor jets). 
The analysis is measuring the normalized differential cross section in a phase-space at generator level as close as possible to the reconstructed level, as a function of the $m_{b\ell}^{minimax}$ observable. 

The data is compared to simulation at particle level in Fig.~\ref{tWchannnel:bb4l_datamc} after background subtraction and correction for detector effects. 
The simulation sample matching the best the data on the whole $m_{b\ell}^{minimax}$ range includes both $tW$ and $t\bar{t}$ and their interference with POWHEG~\cite{Jezo:2016ujg}. Samples featuring interference suppression with  the DR or DS scheme do not reproduce the data at large values of $m_{b\ell}^{minimax}$. 

% Figure environment removed

In the future, one can expect new measurements to probe the nature of the interference in more depth. The lepton+jet final state could be scrutinized as well for this purpose.





\subsection{The challenging $s$-channel}

The final state for top quark production in the $s$-channel is similar to that of the $t$-channel in Section~\ref{t-channel}, except that the top quark is now produced with a $b$ or $\bar{b}$ quark in the final state instead of a light quark (in the 5FS). The process occurs through the exchange of a time-like $W$ boson instead of a space-like $W$ boson, as shown in Fig.~\ref{S-channel:Diagrams}. 
The virtual $W$ boson has to be far away from its resonant mass to produce a top-quark, and this highly suppresses the corresponding cross section, which makes the observation of the $s$-channel very challenging. 
The top quark is more likely to be produced with central b jets than with a forward light jet. 

% Figure environment removed


The $s$-channel process was observed at the Tevatron~\cite{CDF:2014uma}, using 9.7 fb$^{-1}$ of proton-antiproton collisions collected at D0 and CDF at $\sqrt{s}=1.96$ TeV. This process remains to be observed at the LHC. 
At CDF, the lepton+jets channel and \met+jets channel were used, while the lepton+jets channel was used at D0. Multivariate techniques are employed to identify the b jets and to reduce the contribution of background processes. 
Events are classified in categories depending on the number of jets, and the number and quality of b jets. 
Multivariate discriminants are built to extract the $s$-channel cross section using a bayesian statistical technique. 
The result is $\sigma_{s} = 1.29^{+0.26}_{-0.24}$ pb, in agreement with the SM prediction of $\sigma = 1.05 \pm0.06$ pb at approximative NNLO with NNLL accuracy~\cite{Kidonakis:2010tc} at the Tevatron. 
The $s$-channel process was observed at 6.3 $\sigma$ at the Tevatron. 
The Fig.~\ref{fig:s-channel_SoverB} shows the measured cross section for each channel at the Tevatron and their combination.

% Figure environment removed

Proton-antiproton collisions at the Tevatron provided mainly a quark and an antiquark in the initial state of the while the LHC does not. Furthermore, the b-quark content in the proton is larger at the LHC. As a result, the ratio of the $s$-channel to the $t$-channel cross section decreases from the Tevatron to the LHC. 
Furthermore, the ratio of signal to background is quite favorable at the Tevatron relative to the LHC. 
As a matter of fact, when the energy in the p--p center of mass increases, this search becomes more difficult: the quark luminosity increases at a lower pace than the gluon luminosity when increasing the center-of-mass energy. As an example, the ratio of the $s$-channel to $t\bar{t}$ cross section changes from 2.1\% at 8 TeV to 1.2\% at 13 TeV~\cite{ATLAS:2022wfk}. 

For all these reasons, searches for the $s$-channel are very challenging at the LHC. 
A first search at ATLAS using 8 TeV data obtained a significance of 1.3$\sigma$~\cite{ATLAS:2014hvq}, quickly superseded by a new search using the Matrix Element Method (MEM) on the same dataset, leading to an observed significance of 3.2$\sigma$~\cite{ATLAS:2015jmq}. CMS analyzed Run 1 data using the 7 and 8 TeV datasets, resulting in an observed significance of 2.5$\sigma$~\cite{CMS:2016xoq}. 
Recently, ATLAS performed a search using the same analysis techniques with the MEM as in their 8 TeV paper, analyzing Run 2 data at 13 TeV~\cite{ATLAS:2022wfk}. Despite the unfavorable signal to background ratio at 13 TeV compared to 8 TeV, a similar observed significance of 3.3$\sigma$ was achieved. 

Since the ATLAS result is the latest, with the largest observed significance, and the only one published using 13 TeV data, we give details on this analysis in what follows. 
The lepton+jet channel is analyzed, with one electron or muon having $p_T > 30$ GeV and at least two jets with $p_T > 25$ GeV. Events from multijet production are reduced by requiring \met>$35$ GeV and $m_{T,W}>30$ GeV. In the signal region, exactly two jets are required, and both of them must be b-tagged. A validation region is targeting the $W$+jets process, where one of the jets must fail the b-tag requirement. Events are also validated using two regions enriched in \ttbar\ process, with three or four jets, among which two must be b-tagged. 
The  normalization for multijet production is estimated from data, while the other background processes are taken from simulation. 
A dedicated method, the MEM, is employed to reduce further the backgrounds. The MEM consists in calculating the probability for the event to be compatible with hypotheses for signal and backgrounds using exact calculations at LO in pQCD. Hypotheses for the $s$-channel, $t$-channel, \ttbar\ production, and $W$ boson production are considered. A likelihood is built by combining these hypotheses, and the less likely events are discarded. The shape of the likelihood distribution in the signal region is then used to extract the $s$-channel cross section. The postfit distribution is shown in Fig.~\ref{fig:s-channel_ATLASresult} left, and after background subtraction in Fig.~\ref{fig:s-channel_ATLASresult} right. 
The measured cross section is $\sigma = 8.2\pm 0.6$(stat)$^{+3.4}_{-2.8}$(syst) pb, in agreement with the SM prediction of $\sigma_{SM}=10.32^{+0.40}_{-0.36}$ pb.

% Figure environment removed

As a conclusion, the observation of the $s$-channel remains to be achieved at the LHC. A result for a CMS analysis using Run 2 data is desired. 
The analysis is already systematics-dominated, therefore new techniques should be employed to reduce the uncertainties. 
A simultaneous fit using signal and control regions could be used to constrain further the background contributions. 
Involved analysis techniques beyond the MEM, like DNNs, could also improve the significance. 
While still with an unfavorable ratio of signal to background relative to the Tevatron, the searches should be pursued at the LHC with Run 3 data and at the HL-LHC to claim an observation. 




\section{\label{SingleTopBoson}Associated production of a single-top quark with a neutral boson}



\subsection{A newcomer: associated production of a single-top quark with a photon ($t\gamma$) }

The production of a photon in association with a top quark ($t\gamma$) is a rare process, accessible at the LHC. The cross section predicted at NLO in pQCD with Madgraph5\_aMC@NLO is $ 2.95 \pm 0.13 \mathrm{(scale)} \pm 0.03 \mathrm{(pdf)}$ pb~\cite{CMS:2018hfk} with a requiring the photon $p_T$ to be greater than 10 GeV, before top quark decay, in the 5FS. The cross section is dominated by $t$-channel diagrams with radiation of a photon ($t\gamma q$), featuring a forward jet due to the electroweak nature of the $t$-channel. 
Measuring the $t\gamma$ processes extends the landscape of measured top quark processes and is an experimental challenge owing to its low cross section. 
The $t\gamma$ final states are also a powerful tool to constrain the FCNC~\cite{universe8110609}. Together with $t\bar{t}\gamma$ processes, it can be used to constrain the top-$\gamma$ coupling. 
Examples of Feynman diagrams are shown in Fig.~\ref{Tgammaq:Diagrams}. The photon can be emitted in the initial state, final state or in the top quark decay. 

%Discussion 4FS and 5FS? (just as in t-channel)\\

% Figure environment removed

The searches for the $t\gamma q$ process led to an evidence at CMS~\cite{CMS:2018hfk} using 36 fb$^{-1}$ of Run 2 LHC data, and an observation at ATLAS with the full Run 2 dataset~\cite{ATLAS:2023qdu}. 
Special care is needed in single-top $t$-channel MC samples to remove photons produced in the parton shower, since they could be double-counted with photons produced at matrix element level in the $t\gamma q$ signal samples. 
There is also some freedom in the signal definition: photons arising from top quark decay are treated as background in ATLAS analysis~\cite{ATLAS:2023qdu}. 
The dominant backgrounds contain prompt leptons and photons, like $t\bar{t}\gamma$ and $W+\gamma$ processes, and processes involving jets or electrons misidentified as photons (hereafter denominated as "fake photons"). 
A control region is defined to measure $t\bar{t}\gamma$ background. The $W+\gamma$ process also benefits from a control region in ATLAS analysis. The fake photon backgrounds are twofold, either arising from the misidentification of an electron as a photon, or of a jet as a photon. In the ATLAS analysis, both of them are estimated with dedicated methods from data, while in CMS analysis, only the backgrounds made of jets misidentified as photons is estimated from data. 
To maximize the sensitivity to the signal, the signal extraction is performed by constructing a discriminant with a BDT (CMS) and a deep NN (ATLAS). 
Both analyses are making use of the forward jet to discriminate the signal against the backgrounds, including its pseudorapidity as an input variable to the machine learning algorithms. The discriminants are shown in Fig.~\ref{Tgammaq:discriminants}. 

% Figure environment removed

The observed (expected) significance obtained is 4.4$\sigma$ (3.0$\sigma$) at CMS~\cite{CMS:2018hfk} and 9.1$\sigma$ (6.7$\sigma$) at ATLAS~\cite{ATLAS:2023qdu}. The $t\gamma$ processes will soon provide enough statistics for differential cross section measurement and property measurement at the LHC Run 3. 



\subsection{A path towards top-$Z$ coupling: single-top quark production with a Z boson ($tZ$)}

The first process observed for single-top quark production in association with a neutral boson was actually the single-top quark production with a Z boson ($tZ$), thanks to the large datasets made available at the LHC. 
In general, the $tZ$ processes refer to the production of a single-top quark in association with a $Z$ boson, including the interferences between on-shell and off-shell $\gamma^{*}$ or $Z$ bosons. 
%At the moment, only the leptonic decay of the associated boson and of the top quark was used for such measurements at ATLAS and CMS, leading to a three-lepton signature. 
Similarly to the $t\gamma$ process, the process with the largest cross section is provided in the $t$-channel ($tZq$).

The Feynman diagrams for $tZq$ production at LO in pQCD can be seen on Fig.~\ref{fig:tZqDiagram}. The inclusive $tZq$ cross section predicted at NLO in the SM, as calculated with MG5\_aM@NLO, is 800 fb $^{+6.1\%}_{-7.4\%}$~\cite{Aaboud_2018_tZq1}. Because of its clear signature and interesting signal-to-background ratio, the $tZq$ process is measured in the three-lepton channel. The cross section for $tZq$ production in the three-lepton decay channel, calculated at NLO with MG5\_aMC@NLO and including a dilepton invariant mass cut of $m_{\ell\ell}$ > 30 GeV, is 94.2$^{+1.9}_{-1.8}$ (QCD scale) $\pm$ 2.5 (PDF) fb~\cite{Sirunyan_2018_tZq2}.

% Figure environment removed

The $tZq$ production has several interesting features. 
Similarly as for the $t$-channel process without an associated $Z$ boson, the top and antitop quarks from the $tZq$ production are strongly polarized, making this process an excellent probe for studying $t-Z$ couplings, in particular in the context of the EFT measurements.
It is also sensitive to triple gauge couplings $WWZ$, in a complementary manner with the diboson production. Both are potentially sensitive to physics beyond the SM.

Data is selected with a combination of single lepton or double leptons triggers. 
Events are selected events if they contain three well identified and isolated leptons (electrons or muons possibly arising from $\tau$ lepton decays).
A pair of same-flavor opposite-charge leptons, compatible with a $Z$ boson decay, is then required. 
Because the $tZq$ process is a $t$-channel process, it contains a light jet preferentially produced at large $|\eta|$, a b-tagged jet arising from the top quark decay and missing transverse energy arising from the neutrino from the $W$ boson decay.

Similarly to other analyses presented in this review, the signal is extracted from signal and control regions defined by the number of jets and b-tagged jets. 
The first signal region requires $N_j=2$ or $N_j=3$ with $N_b=1$ (so-called $2j1b$ and $3j1b$ regions). These regions contain most of the signal with the $WZ$+jets process as the dominant background, and with contributions from other diboson processes. 
For larger jet and b-tagged jet multiplicities ($N_j\geq3$, $N_b\geq 2$), the dominant background source is arising from $t\bar{t}Z$ events, with a contamination from $tt+W,H$ processes. 
A control region with $N_b=0$ allows to constrain on diboson events.

Other sources of background are arising from non-prompt lepton events in $t\bar{t}$ or $Z$+jets processes. 
While backgrounds presenting three prompt leptons are estimated from simulations, and  constrained from data in the likelihood fit, events containing at least one non-prompt lepton are not well described by simulations and are therefore more difficult to estimate. 
In the CMS observation paper~\cite{Sirunyan_2019tZqOBs}, the analysis uses a fully data-driven technique where the probabilities for measuring a non-prompt lepton are measured from a region where one lepton fails the lepton isolation. 
The ATLAS observation paper~\cite{Aad_2020_tZqObs} however uses a semi data-driven technique, where the normalization of the non-prompt background is estimated from data in control regions, and the kinematic distributions are determined from simulations of \ttbar+$tW$ and $Z$+jets events, by replacing b-jets with non-prompt leptons and accounting for the needed corrections.

The discriminating variables used in the fit are based on multivariate discriminants (BDT or NN), which include kinematic variables related to the reconstructed $Z$ bosons or top quarks, the pseudo-rapidity of the spectator jet $|\eta_{j'}|$, dijet invariant mass, or kinematic variables related to the lepton from the $W$ decay. Examples of NN output distributions from ATLAS~\cite{Aad_2020_tZqObs} can be found in Fig.~\ref{fig:tZqATLASNN}. 


% Figure environment removed
 

The most recent inclusive $tZq$ cross sections measured at ATLAS\cite{Aad_2020_tZqObs} and CMS~\cite{JHEP_TOP_20_10} are found to be compatible with the SM. The precision is still dominated by statistical uncertainties, although in the case of the CMS results, systematic and statistical uncertainties are almost of the same level.


Thanks to the large integrated luminosity provided by the LHC, it is now possible to measure differential cross sections for $tZq$ production~\cite{JHEP_TOP_20_10}. 
This analysis follows a different approach to extract the signal. A multi-class NN is used to separate the $tZq$ process from the $t\bar{t}Z$, $WZ$ and other $t+X$ processes. The signal region is then sub-divided based on the bins of the observables of interest, at detector level. The NN score of the $tZq$ node is used to extract the signal in each bin. Similarly, the NN score of the $t\bar{t}Z$ node is used to constrain the $t\bar{t}Z$ background. An unfolding procedure infers the particle- or parton-level distributions. Examples of differential cross section measurements for $p_T(Z)$ and $p_T(t)$ at parton level, and $|\eta(j')|$ and $\cos{\theta^*_{pol.}}$ at particle level are shown in Fig.~\ref{fig:tZq_diffDistrib}. The $\cos{\theta^*_{pol.}}$variable is the cosine of  polariation angle of the top quark, defined as:
\begin{linenomath}
\begin{equation}
\cos{\theta^*_{pol.}} = \frac{\vec{p}(q'^*)\cdot \vec{p}(l_t^*)}  { | \vec{p}(q'^*) || \vec{p}(l_t^*)|  }   
\end{equation} 
\end{linenomath}
with $\vec{p}(q'^*)$ and $\vec{p}(l_t^*)$ the three momenta of the light jet and of the lepton from the top quark decay. A good agreement between data and predictions is observed. This very promising publication is the first differential measurements of a rare single-top process, and can serve as basis for future studies. In particular, it gives a clear procedure to perform a differential measurement, with an interesting signal extraction  based on a multi-class discriminant.

% Figure environment removed

Eventually, the first measurement for the $tWZ$ process led to an evidence at CMS~\cite{CMS-PAS-TOP-22-008} (presented for now as a conference note). This very rare process can be seen as a $tZ$ production in the $tW$ channel. The analysis techniques are similar with those of the $tZq$ analysis, using a multi-class NN, with a multi-lepton signature targeted. This result opens up a new era for the measurement of top quark processes associated with multi-bosons.




\subsection{\label{SingleTopTHQ}The $tH$ processes: companions for the top quark Yukawa coupling}

\subsubsection{Introduction on the $tH$ processes}

Among the processes involving a top quark and a boson in the final state, the $tH$ processes are produced with the lowest cross section predicted in the SM, of approximately 71 fb and 16 fb at NLO for the $t$-channel and $tW$ associated production with $\sqrt{s}=$13 TeV~\cite{LHCHiggsCrossSectionWorkingGroup:2016ypw}. 
The $t$-channel $tHq$ processes share many properties with the $t\gamma q$ and the $tZq$ processes, noticeably their modeling in the 4FS or 5FS scheme, and the production of an associated quark in the forward direction. The Feynman diagrams for the production of $tHq$ are depicted in Fig.~\ref{THq:Diagrams}.
The $tHW$ production is also considered in the analyses.

% Figure environment removed

The search for the $tH$ processes is traditionally performed in association with the search for or the measurement of the Higgs boson in the $t\bar{t}H$ production mode, whose cross section is larger than the cross section of the $tHq$ process by a factor 10, as shown in Fig.~\ref{THq:HiggsXSvsEnergy}. 
The amplitude for $tHq$ production features the interesting property of showing an interference between diagrams where the Higgs boson is emitted from a top quark line, or from W boson exchange. This property makes the measurement of the $tHq$ process appealing, since it gives access to the sign of the Yukawa coupling of the top quark. If the sign of the Yukawa coupling $\kappa_t$ is negative, the interference becomes constructive, for instance increasing the cross section by a factor of approximately 12 if $\kappa_t = -1$~\cite{Farina:2012xp}. 
The $tH$ final states, in an equal footing with the $tZ$ and the $t\gamma$ final states, are also used in the searches for FCNC~\cite{universe8110609}. 

% Figure environment removed


\subsubsection{Searches for the $tH$ processes}

The early searches for the $tH$ processes at 8 TeV~\cite{CMS:2015nrd} were trying to measure directly the $tH$ production, while the $t\bar{t}H$ process was treated as a background. It was however realized that by varying the value of the top quark Yukawa coupling, the cross section for both the $tH$ and $t\bar{t}H$ processes would be modified in correlated way. Nowadays, the searches for the $tHq$ process are performed in a combined measurement with the $t\bar{t}H$ process, either targeting the measurement of the top quark Yukawa coupling, or measuring simultaneously the cross section for the $tH$ and the $t\bar{t}H$ processes.


Branching ratios for the Higgs boson decay are shown in Fig.~\ref{THq:HiggsXSvsEnergy}. The ATLAS and CMS analyses aiming at measuring $t\bar{t}H+tH$ processes are targeting the main decay modes of the Higgs boson: $H\rightarrow\gamma\gamma$; $H\rightarrow WW$, $H\rightarrow ZZ$ and $H\rightarrow \tau\tau$ (grouped under the naming of "multilepton final state" since $W$, $Z$, $\tau$ and associated top quarks can decay leptonically); and to a lesser extent, $H\rightarrow b \bar{b}$ (suffering from a lack of available luminosity to achieve similar sensitivity as the other channels). We present here the methodology and the latest results.

The analysis of the $H\rightarrow\gamma\gamma$ decay channel with Run 2 data at ATLAS~\cite{ATLAS:2022tnm} and CMS~\cite{CMS:2021kom} is following a similar strategy as for the measurement of the other production mechanisms of the Higgs boson. 
The small $H\rightarrow\gamma\gamma$ branching fraction (close to 0.2\% at $m_H=125$ GeV) is compensated by the excellent resolution of the electromagnetic calorimeters (the effective mass resolution on the Higgs boson is close to 1.5 GeV, depending on the analysis categories). The background processes involving jets reconstructed as photons are reduced using photon isolation and information on the shape of the electromagnetic energy deposit, with sequential criteria at ATLAS and a multivariate method at CMS. Several event classes are constructed, targeting specifically a given production mechanism. For each event class targeting the $tH$ processes, the background is reduced by the means of a BDT discriminant, which is subsequently fitted with a smoothly falling function. 
In the latest versions of the analysis~\cite{ATLAS:2022tnm,CMS:2021kom}, several subcategories are built to target specifically the $t\bar{t}H$ and $tH$ processes in kinematic bins, and the fit is interpreted in the so-called "simplified template cross section" framework (STXS)~\cite{Berger:2019wnu}. The STXS framework is a convention to provide results in kinematic bins at particle level within a defined acceptance for each Higgs boson production mechanism. 
In the CMS analysis, a category at reconstructed level is targeting specifically $tH$ in the leptonic channel, and a DNN discriminant is used to improve separation between $t\bar{t}H$ and $tH$. Using this category, together with many reconstructed-level event classes in a simultaneous fit, the cross section for the $tH$ processes at STXS level is quoted to be $6.3 ^{+3.4}_{-3.7}$ times the SM expectation (in the so-called "maximal merging scenario", where less STXS categories are used at particle level than in the "minimal merging scenario"). 
In the ATLAS analysis, four reconstructed categories targeting $tH$ processes are defined, among which two categories are targeting specifically the $tHq$ processes with either a positive or a negative top quark Yukawa coupling (defined using the output of a NN), and one category is targeting the $tHW$ process, with the remaining category gathering events with low-score of the BDT for $tHq$ and $t\bar{t}H$. At STXS level, the cross section for the $tH$ processes is found to be $2.1 ^{+4.2}_{-3.1}$ times the standard model expectation.

Using the multilepton channel, CMS~\cite{CMS:2020mpn} reported measurements of the cross section for $t\bar{t}H$ and $tH$ production simultaneously with Run 2 data. This analysis is using multiple final states; for leptonic top decay: same-sign $2\ell+0\tau_h$ (with $\ell=e,\mu$ and $\tau_h$ refers to a hadronically decaying $\tau$), $3\ell+0\tau_h$, $2\ell+1\tau_h$ (same-sign and opposite sign), $1\ell+2\tau_h$, $4\ell+0\tau_h$, $3\ell+1\tau_h$ and $2\ell+2\tau_h$; for hadronic top decay $1\ell+1\tau_h$ and $0\ell+2\tau_h$. 
The sensitivity arises mainly from the channels same-sign $2\ell+0\tau_h$, $3\ell+0\tau_h$ and $1\ell+2\tau_h$. In those main categories, the analysis employs a multi-class DNN providing discriminants for $t\bar{t}H$ and $tH$ separately, while using simpler BDTs in the other categories. In the same-sign $2\ell+0\tau_h$, $2\ell+1\tau_h$ channels, categories are further divided according to the lepton flavour and whether the number of b jets is larger or smaller than 2.
The jet faking lepton background is estimated with a data-driven method by relaxing lepton identification criteria in a region enriched in multijet events. The background arising from mismeasurement of the lepton charge is estimated with $Z\rightarrow ee$ events. The dominant background is arising from $t\bar{t}W$ and $t\bar{t}Z$ processes, estimated from simulation. The background arising from conversion of leptons in the detector is estimated from simulation. 
The signal is extracted using bins in the multivariate discriminants. Several control regions with $3\ell$ and $4\ell$ final states are also used in the fit. Two parameters of interest are measured: the signal strength $\mu$ for $t\bar{t}H$ and for $tH$ processes. 
The signal strength for $tH$ production is found to be $5.7 \pm 2.7$(stat) $\pm 3.0$(syst). Additionally, a 2-dimensional distribution of the likelihood as a function $\mu_{t\bar{t}H}$ and $\mu_{tH}$ is measured, as shown in Fig.~\ref{THq:CMSmuTTHvsTH}. 

% Figure environment removed

A few results arising from combinations of final states are also reported. 
A dedicated CMS analysis targeting specifically $tH$ at 13 TeV using 36 fb$^{-1}$ of Run 2 data~\cite{CMS:2018jeh}, employed the $H\rightarrow\gamma\gamma$ and multilepton final states, as well as the final state $H\rightarrow b\bar{b}$ in the $VH$ production mode with single lepton decay of the associated boson. The multilepton analysis is using simpler techniques than those previously described~\cite{CMS:2021kom,CMS:2020mpn}, and trained multivariate methods with $tH$ processes as signal. The $H\rightarrow\gamma\gamma$ analysis reinterprets the content of $t\bar{t}H$ categories of a previous analysis. The $H\rightarrow b\bar{b}$ analysis brings little sensitivity and will not be described here. The combined measurement results in an observed limit on the cross section for $tH$ production of 1.94 pb at 95\% CL, in the SM hypothesis. Results for hypotheses with negative top quark Yukawa coupling are also reported.
Eventually, grand combinations at the occasion of the celebration of 10 years after the discovery of the Higgs boson were performed at CMS~\cite{CMS:2022dwd} and ATLAS~\cite{ATLAS:2022vkf}, including many final states. Categories targeting specifically the $tH$ processes are taken from the $H\rightarrow\gamma\gamma$ channel at ATLAS, and $H\rightarrow\gamma\gamma$ and multilepton channel at CMS. The CMS combination reports a measured signal strength of $\mu_{tH} = 6.05 ^{+2.66}_{-2.42}$.

Studies estimating the sensitivity to the $tH$ processes at the HL-LHC were expecting a relative uncertainty of 90\% on the $tH$ signal strength in the SM hypothesis~\cite{CMS:2018qgz} (with $t\bar{t}H$ signal strength floating), however these studies were based on early projections and would need to updated with latest ATLAS and CMS results.



\subsubsection{Probing the sign of the top quark Yukawa coupling}

With the ATLAS and CMS $H\rightarrow\gamma\gamma$ analyses~\cite{ATLAS:2022tnm,CMS:2021kom}, with the CMS multilepton analysis~\cite{CMS:2020mpn}, as well as with the earlier CMS combination~\cite{CMS:2018jeh}, all of them including categories targeting specifically the $tH$ processes, it becomes feasible to access the sign of the top quark Yukawa coupling, thanks to the interference between Feynman diagrams involving Higgs boson coupling to the top quark and with the $W$ boson. The modifier $\kappa_t$ of the top quark Yukuawa coupling in the SM, $y_{t,SM}$, is defined as $\kappa_t = y_{t}/y_{t,SM}$. Furthermore, since a similar interference is also present in $H\rightarrow\gamma\gamma$ decay between the top quark loop and the $W$ boson loop, further sensitivity is gained in this channel. Sensitivity to the positive values remains dominated by the $t\bar{t}H$ process in direct measurements, and by the $gg\rightarrow H$ process (involving a top quark loop) in indirect measurements because of their larger cross section. 

The Fig.~\ref{THq:KappaTopScans} shows the likelihood fit value as a function of the $\kappa_t$ parameter. The best fit value is found to be positive and close to 1, while a second minimum of the likelihood is found at a value close to -1. As shown in the left of Fig.~\ref{THq:KappaTopScans}, including the parameterization of the gluon fusion mechanism as a function of $\kappa_t$ in the likelihood provides more weight to the positive value of $\kappa_t$. On the contrary, when only the $tH$ and $t\bar{t}H$ processes are included, more sensitivity is gained on the sign of $\kappa_t$. Values outside of $0.65 < \kappa_t < 1.25$ in the first case, and $0.87 < \kappa_t < 1.20$  in the second case are excluded at 95\% CL by $H\rightarrow\gamma\gamma$ analysis at ATLAS. The CMS multilepton analysis results in $-0.9 < \kappa_t < -0.7$ or $0.7 < \kappa_t < 1.1$ at 95\% CL.

% Figure environment removed

Projections for the measurement of the top quark Yukawa coupling at the HL-LHC are reported by CMS~\cite{CMS:2022dwd} without emphasis on a possible negative coupling. A precision on the order of 3-4\% on $\kappa_t$ would be achievable, while a precision on the order of 10\% is achieved today~\cite{CMS:2022dwd,ATLAS:2022vkf}. 

The $tH$ processes, together with the $t\bar{t}H$ process, can also be used to set constraints on a CP-odd top quark Yukawa coupling. Such measurements were performed at ATLAS with the $H\rightarrow\gamma\gamma$~\cite{ATLAS:2020ior} and $H\rightarrow b \bar{b}$~\cite{ATLAS:2023cbt} channels, and at CMS with the $H\rightarrow\gamma\gamma$~\cite{CMS:2020cga} and multilepton~\cite{CMS:2022dbt} channels. Since the $t\bar{t}H$ process has a larger cross section than the $tH$ processes, most of the sensitivity will come from the former, and these measurements will not be described here.







\section{\label{SingleTopProperties}Discovery potential of property measurements and interpretations}

The large statistics of single-top events and the high precision obtained in single-top measurements allow for the measurements of top quark properties, which can be seen as a test of the SM or a search for physics beyond the SM. 
Despite single-top quark production yields a lower cross section than that of \ttbar\ production, the production of single-top quarks and their subsequent decay to $Wb$ involve twice the $Wtb$ vertex, simultaneously in the top quark production and decay. This interesting feature can be used to measure several interconnected properties: the couplings of the $Wtb$ vertex including the CKM matrix element $|V_{tb}|$, the $W$ polarization and top quark polarization. Precision measurements of the $Wtb$ couplings can be expressed in term of CP-even and CP-odd effective couplings or within the Effective Field Theory (EFT). Apart from the $|V_{tb}|$ measurement, which can be inferred from the single-top cross section, the general experimental strategy for measuring all other properties consists in performing various angular analyses of the top quark decay, choosing suitable angular distributions to measure the parameters of interest. 
Additional couplings can be probed within the EFT, including four-fermion couplings, and couplings between the top quark and neutral bosons. 

This review will not discuss the measurement of top quark mass using single-top $t$-channel nor the tests of CPT symmetry comparing top and antitop quark masses in single-top events (for a recent result, see~\cite{CMS:2021jnp}), since the precision is not yet at the required level for competing with \ttbar\ measurements.
This section will cover the other above-mentioned top quark properties using single-top quark as a probe, which are reaching a precision similar to or better than that achieved in \ttbar\ measurements. 



\subsection{\label{VtbSection}Measurement of the CKM matrix element $|V_{tb}|$}

Because the $V_{tb}$ CKM matrix element is close to unity in the SM, the measurement of $V_{tb}$ is particularly intriguing, and its study is an excellent way to better understand the SM and search for signs of new physics. 
The measurement of the $V_{tb}$ CKM matrix element is strongly related to the electroweak nature of the single-top production. 
The cross section for single-top production can be used to test the unitarity of the CKM matrix. Assuming the values of $|V_{td}|$ and $|V_{ts}|$ to be much smaller than that of $|V_{tb}|$, the measured single-top cross section can be used to determine $|V_{tb}|$ according to the formula~\cite{ATLAS:2019hhu}:
\begin{linenomath}
\begin{equation}
|f_{L_V}V_{tb}| = \sqrt{ \frac{\sigma^{meas}}{\sigma^{theo}} },
\end{equation}
\end{linenomath}
with $\sigma^{meas}$ the measured cross section, $\sigma^{th}$ the SM theoretical cross section assuming  $|V_{tb}|=1$, and $f_{L_V}$ an anomalous form factor (of the kind vectorial left-handed, as in the SM) which can be different from 1 in new physic models. 
Such a method was used to re-interpret several single-top cross sections at 7 and 8 TeV. Their combinations, including ATLAS and CMS results for $t$-channel, $tW$ production and $s$-channel, was performed in the context of the LHC$top$WG, and lead to the most precise $|f_{L_V}V_{tb}|$ measurement up to now, as shown in Fig.~\ref{fig:tchan_vTB}. One can note that the $t$-channel measurements are dominating the combination. 
The latest $t$-channel measurement at 13 TeV~\citep{ATLAS-CONF-2023-026} improves over this  combination by approximately 30\% in precision, with $|V_{tb}|=1.014 \pm 0.031$ reported.


% Figure environment removed

It is possible to release the assumption that $|V_{td}|$ and $|V_{ts}|$ are negligible compared to $|V_{tb}|$. 
Such a method has also been pursued, consisting in measuring $|V_{tb}|$, $|V_{td}|$ and $|V_{ts}|$ in a model independent way using single-top $t$-channel enriched events~\cite{CMS:2020vac}. The main principle of the analysis relies in considering several single-top $t$-channel signals, according to the presence of a $tWb$ vertex in single-top production ($ST_{b, q}$), in top quark decay ($ST_{q, b}$), or in both ($ST_{b, b}$). Several signal regions, based on the jet and b-tagged jets multiplicities, can be defined and fitted simultaneously. Further discrimination between $ST_{b, q}$, $ST_{q, b}$ and $ST_{b, b}$ is obtained using kinematic and angular properties of the involved processes, using the fact that: 1) PDFs are different for each of them, and 2) the presence of an additional b jet from gluon splitting can affect top quark reconstruction. Using the constraint of CKM unitarity ($|V_{tb}|^2 + |V_{td}|^2 +|V_{tb}|^2$), a precision similar to that of the combination~\cite{ATLAS:2019hhu} is achieved~\cite{CMS:2020vac} with an integrated luminosity of 35.9 fb$^{-1}$ of 13 TeV proton-proton collisions. The method allows to perform also the measurements under the constraints of BSM scenarios. Results are found to be compatible with previous measurements and with the SM predictions.


\subsection{$W$ boson polarization fractions}

The V -- A structure of the electroweak theory, together with the mass of the particles involved, determines the fractions of longitudinal, left-handed and right-handed $W$ boson polarization (sometimes called helicity fractions), denoted respectively $F_0$, $F_L$ and $F_R$. Predictions for these fractions computed at NNLO in pQCD are~\cite{Czarnecki:2010gb}: $F_0 = 0.687 \pm 0.005$, $F_L = 0.311 \pm 0.005$ and  $F_R = 0.0017 \pm 0.0001$. Experimentally, the fractions can be measured within the $W$ rest frame where the $W$ boson arises from leptonic top decay, using the angle $\theta^{*}$ defined as the angle between the direction of the charged lepton and the reversed direction of the b quark. The differential decay rate writes:
\begin{linenomath}
\begin{equation}
\frac{1}{\Gamma}\frac{d\Gamma}{dcos \theta^{*}} = \frac{3}{4} (1-cos^2 \theta^{*}) F_0 + \frac{3}{8} (1-cos \theta^{*})^2 F_L  + \frac{3}{8} (1+cos \theta^{*})^2 F_R,
\end{equation}
\end{linenomath}
with $F_0+F_L+F_R=1$. 
The differential decay rate as a function of $cos \theta^{*}$ is illustrated in Fig.~\ref{SingletopEFTprop:HelicityFractions}.

% Figure environment removed

The fractions are obtained from a fit of the $cos \theta^{*}$ distribution to the data. 
The $W$ boson polarization fractions have been measured at CDF and D0~\cite{CDF:2012dup} with a precision on $F_0$ of the order of 10-15\%, using $t\bar{t}$ decay. 
At the LHC, the observation of single-top production in the $t$-channel and its large cross section offers the possibility of measuring the polarization fractions in single-top decay in addition to $t\bar{t}$ decay. 
The fractions were measured at 8 TeV with CMS~\cite{CMS:2014uod} as: $F_L = 0.298 \pm 0.028 (stat) \pm 0.032 (syst)$, $F_0 =0.720 \pm 0.039(stat) \pm 0.037(syst)$, and $F_R =-0.018 \pm 0.019(stat) \pm 0.011(syst)$. 
The precision achieved with single-top measurements justifies its inclusion in ATLAS and CMS combination of 8 TeV results~\cite{CMS:2020ezf}, leading to $F_0 = 0.693 \pm 0.014$, $F_L = 0.315 \pm 0.011$ and $F_R = -0.008 \pm 0.007$. 
The 7 TeV results were obtained by analyzing $t\bar{t}$ and were not considered since they were expected to bring negligible improvement. 

ATLAS also employed the "generalized helicity fractions and phases" formalism~\cite{Boudreau:2013yna}, by the means of amplitude decomposition in several angular distributions in the top quark rest frame. 
Among the parameters measured, the transverse polarization fraction using single-top decays at 7 and 8 TeV~\cite{ATLAS:2015ryj, ATLAS:2017rcx} yields $F_T = F_L + F_R = 0.30 \pm 0.05$~\cite{ATLAS:2017rcx} as best result. ATLAS also measured the phase between amplitudes for longitudinally and transverse $W$ bosons recoiling against left-handed b-quarks~\cite{ATLAS:2015ryj,ATLAS:2017rcx}, giving no sign of CP-violation. From this formalism, left- and right-handed fractions could in principle be calculated. 

\subsection{$Wtb$ effective couplings and interpretation in the SM effective field theory}

The $Wtb$ effective couplings were also measured, either at CMS as extracted from the $W$ boson polarization fractions~\cite{CMS:2014uod} or measured directly~\cite{CMS:2016uzc}; or at ATLAS by analyzing the single-top amplitudes~\cite{ATLAS:2015ryj, ATLAS:2017rcx} or measuring various angular asymmetries~\cite{ATLAS:2017ygi}. The lagrangian describing $Wtb$ effective couplings reads~\cite{Aguilar-Saavedra:2006qvv}:
\begin{linenomath}
\begin{equation}
\label{WtbCouplingsLagrangian}
L_{Wtb}  = -\frac{g}{\sqrt{2}} \bar{b} \gamma^{\mu} (V_L P_L + V_R P_R) t W_{\mu}^{-} -\frac{g}{\sqrt{2}} \bar{b} \frac{i \sigma^{\mu\nu} q_{\nu}}{m_W}  (g_L P_L + g_R P_R)  t W_{\mu}^{-} + h.c.
\end{equation}
\end{linenomath}
with $V_L$, $V_R$ the vectorial left-handed and right-handed $Wtb$ couplings, and $g_L$, $g_R$ the tensorial left-handed and right-handed $Wtb$ couplings (sometimes called respectively $f_V^L$, $f_V^R$, $f_T^L$, $f_T^R$ depending on the convention~\cite{CMS:2016uzc}). In the SM at LO in pQCD, $V_L = V_{tb}$ while $V_R=g_L=g_R=0$. The couplings $V_R, g_L, g_R$ are complex and can be CP-odd if their imaginary part is non-zero. The $V_{tb}$ CKM matrix element is inferred from single-top cross section measurement, as discussed in section~\ref{VtbSection}. 

The analyses have moderate sensitivity to the right-handed vectorial coupling and left-handed tensorial coupling. With a simultaneous fit of both parameters, ATLAS reports $| V_R / V_L | < 0.37$ and $ | g_L / V_L | < 0.29 $ at 95\% CL~\cite{ATLAS:2017rcx}, and CMS reports $f_V^R < 0.16$ and $f_T^L < 0.057 $ at 95\% CL~\cite{CMS:2016uzc} including also $f_V^L$ in the fit using inclusive cross section information. 
The best sensitivity on the $Wtb$ couplings is obtained on the $g_R$ coefficient. ATLAS obtained with a simultaneous fit $-0.12 < Re(g_R/V_L)< 0.17$ and $-0.07 < Im(gR /V_L) < 0.06$ at 95\% CL~\cite{ATLAS:2017rcx}. If using single-top cross section information and assuming null imaginary part, CMS obtained $|Re(f_T^R)| < 0.046 $. 
These results can be compared with the combination of 8 TeV $W$ boson polarization fraction (including $t\bar{t}$ channels)~\cite{CMS:2020ezf}: $-0.11 < Re(V_R) < 0.15$, $-0.08 < Re(g_L) < 0.05$ and $-0.04 < Re(g_R) < 0.02$. 
Since the imaginary part of $g_R$ cannot be accessed easily from the \ttbar\ process and would need a dedicated analysis~\cite{Aguilar-Saavedra:2006qvv}, the single-top measurements such as~\cite{ATLAS:2015ryj,ATLAS:2017rcx} are irreplaceable. 


The results obtained in the effective coupling formalism can be translated into the modern framework of the Standard Model Effective Field Theory~\cite{Buckley:2015lku}, adding all operators respecting gauge invariance to the SM lagrangian. The $Wtb$ couplings considered in Eq.~\ref{WtbCouplingsLagrangian} ($V_L$, $V_R$, $g_L$, $g_R$) are respectively related to the following four dimension 6 operators:
\begin{linenomath}
\begin{equation}
O_{\phi q}^{(3)} = \frac{c_{\phi q}^{(3)}}{\Lambda^2} i (\phi^{\dagger} \overleftrightarrow{D_{\mu}}^I \phi) (\bar{q}\gamma^{\mu} \tau^I q) ,
\end{equation}
%\qquad V_L = V_{tb} + c_{\phi q}^{(3)} \frac{v^2}{\Lambda^2}
\begin{equation}
O_{\phi t b} = \frac{c_{\phi t b}}{\Lambda^2} (\phi^{\dagger} \overleftrightarrow{D_{\mu}}^I \phi) (\bar{t}\gamma^{\mu} \tau^I b),
\end{equation}
%\qquad V_R = \frac{1}{2} c_{\phi t b} \frac{v^2
\begin{equation}
O_{tW} = \frac{c_{tW}}{\Lambda^2} (\bar{q}\sigma^{\mu\nu} \tau^I t) \widetilde{\phi} W_{\mu\nu}^I,
\end{equation}
\begin{equation}
O_{bW} = \frac{c_{bW}}{\Lambda^2} (\bar{q}\sigma^{\mu\nu} \tau^I b) \phi W_{\mu\nu}^I,
\end{equation}
\end{linenomath}
using notations from~\cite{Grzadkowski:2010es}. 
Results from the combination of $W$ boson polarization at 8 TeV are~\cite{CMS:2020ezf}: $-3.48 < Re(c_{\phi t b}) < 5.16$, $-0.48 < Re(c_{tW}) < 0.29$ and $-0.96 < Re(c_{bW})< 0.67$. A translation from the best measurement of $Im(g_R)$~\cite{ATLAS:2017rcx} to the EFT formalism using~\cite{Buckley:2015lku} gives $-0.82 < Im(c_{bW}) < 0.70$. 

\subsection{Top quark polarization}

Recently, by an analysis of the top quark polarization, ATLAS measured directly the coefficient $Im(c_{tW})$ for the first time~\cite{ATLAS:2022vym}, using the full Run 2 dataset at 13 TeV. 
Because of parity conservation in QCD, top quarks in $t\bar{t}$ production are unpolarized, while top quarks are produced mostly polarized in single-top production. The polarization vector $\vec{P}$ is defined with components $P_i = 2 < S_i >$ where $S_i$ is the top quark spin along the $i$ direction~\cite{Aguilar-Saavedra:2014eqa}, in the top quark rest frame where the $z'$ direction is defined as the $W$ boson direction, the $x'$ direction is defined as the spectator quark direction projected on the transverse plane while the $y'$ axis completes the direct  basis. The values of the polarization vectors for single-top and antitop quarks produced in the $t$-channel are close to respectively: $(-0024,0,0.965)$ and $(-0.073,0,-0.957)$ at NNLO in pQCD~\cite{ATLAS:2022vym}. 
The top quark polarization can be extracted from angular distributions of top decay products defined in the top quark rest frame, which are given by the general formula:
\begin{linenomath}
\begin{equation}
\frac{1}{\Gamma}\frac{d\Gamma}{dcos \theta_{X}} = \frac{1}{2}(1+\alpha_X P_X cos \theta_X),
\end{equation}
\end{linenomath}
where $\theta_X$ is the angle between the top quark spin axis and the direction of motion of the chosen decay $X$, $\alpha_X$ is the spin analyzing power associated to $X $ and $P_X$ the top quark degree of polarization along the direction of $X$. 
The measurement of top quark polarization in~\cite{ATLAS:2022vym} is performed using the angular distributions related to the charged lepton (shown to have the largest spin analyzing power,  close to 1) arising from top decay and projected on the previously defined directions. 
If top quark polarization had already been measured previously at the LHC along the $z$ direction (for instance in~\cite{ATLAS:2017ygi}), the measurement~\cite{ATLAS:2022vym} is the most precise and also includes $x'$ and $y'$ directions. It leads respectively for top and antitop quarks to: $P_x'=0.01 \pm 0.18$, $P_y'=-0.029 \pm 0.027$, $P_z'=0.91 \pm 0.10$ and $P_x'=-0.02 \pm 0.20$, $P_y'=-0.007 \pm 0.051$, $P_z'=-0.79 \pm 0.16$. The polarizations along the directions $x'$ and $z'$ are also reported in Fig.~\ref{SingletopEFTprop:Polarization}. 
Using the same angular distributions, ATLAS reports $-0.9 < Re(c_{tW}) < 1.4$ and $-0.8 < Im(c_{tW}) <  0.2$ at 95\% CL. 

% Figure environment removed


\subsection{Discussion on other couplings with single-top quark measurements in the SM effective field theory}

If anomalous couplings measurements in single-top quark processes are primarily interesting for $Wtb$ couplings, other couplings are also actively measured: the coupling between heavy quarks and light quarks, the couplings between heavy quarks and neutral bosons or between heavy quarks and leptons. 
The discussion in this section excludes the FCNC (for a review, see~\cite{universe8110609}). 

In general, single-top production with a boson can help to constrain the coupling between top quarks and neutral bosons. 
The top-Z (resp. top-Higgs) coupling is impacting single-top quark produced in association with a $Z$ boson (resp. a Higgs boson).  
The top-gluon coupling impacts the $tW$ channel (since $tW$ channel LO diagrams feature one gluon in the initial state) and any production channel considered at NLO, where gluons can be emitted from top quarks. 
The process of single-top production accompanied with a photon has just been observed and could be used in the near future for measuring the top-$\gamma$ coupling. 
It has also been emphasized that the $tZq$ and $tHq$ processes can be greatly impacted by some of these couplings~\cite{Degrande:2018fog}. 
However, the cross sections for processes of single-top production in association with bosons ($t+V$) are lower than that top pairs produced in association with bosons ($t\bar{t}+V$), therefore analyses of $t+V$ final states are generally swamped by $t\bar{t}+V$ backgrounds. As a consequence, measuring the $tZ$, $tH$ or $tg$ couplings requires, for consistency, to include the modeling of the anomalous couplings both in $t+V$ and $t\bar{t}+V$ simulation samples, which will help in constraining the couplings. 
It is difficult to disentangle what is the exact contribution of single-top production to the sensitivity in these couplings. We will therefore limit ourselves to give some examples where the contribution of single-top processes is explicitly included. 
Generic searches for measuring top quark couplings in the multi-lepton lepton final state are defining many event classes to target explicitly a great number of EFT operators impacting $tZq$ and $tHq$ processes~\cite{CMS:2020lrr}: 9 operators involving two quarks and one or more bosons (among which some impacting the $Wtb$ vertex, considered at production level only), as well as 7 operators involving two heavy quarks and two leptons. An updated analysis~\cite{CMS-PAS-TOP-22-006} involving more operators remains to be published. 
The top gluon coupling has been considered in~\cite{CMS:2020lrr} by including its impact on gluon radiation at LO. 
Measurement of the EFT operators in $t\bar{t}Z+tZq$ final states~\cite{CMS:2021aly} include 5 operators involving two quarks and one or more bosons (including $Wtb$ vertex) and uses machine learning to maximize sensitivity. 
The Yukawa coupling is measured in $t\bar{t}H$ analyses by including its impact on $tHq$, as discussed in section~\ref{SingleTopTHQ}. 

A recent measurement of the $t$-channel process using full Run 2 data at ATLAS~\cite{ATLAS-CONF-2023-026} (to be published) sets constraints on the coupling between light and heavy quarks (the $C_{q,Q}^{(1,3)}$ coefficient within the SMEFT framework), in a competitive manner with global fits reinterpreting LHC data. 

More such analyses targeting EFT measurements are expected in the future. 
The LHC$top$WG, together with the LHC EFT WG, are working on prescriptions towards the combination of direct top quark EFT measurements.




\section{\label{Conclusions}Conclusions}

After more than 10 years of data-taking with the LHC, the understanding of the physics involving single-top quark processes has undergone a spectacular change. Before the LHC, only the $s$-channel production mode had been discovered and could be studied. 
Nowadays, differential cross sections for the $t$-channel and $tW$ channel production modes are measured. The $t$-channel is commonly used for top quark property measurements, from the structure of the $Wtb$ vertex, to $W$ boson and top quark polarization, not mentioning top quark mass measurement. The $tW$ channel is employed to probe delicate interference effects with $t\bar{t}$. The $s$-channel process remains to be observed at the LHC, and the first evidences show that this can be achieved soon. 
The cross sections measured for single-top quark production in the $t$-channel, $tW$ channel and $s$-channel at ATLAS and CMS are compared with theory predictions in Fig.~\ref{fig:tXsummaryXsec}.

% Figure environment removed

Processes involving a single-top quark with an associated $Z$ boson are measured differentially, and are used top probe various couplings within the standard model effective field theory framework. The associated production with a photon has now been observed. The $tH$ processes are already used to probe the sign of the top quark Yukawa coupling, however it has not been observed yet. 
The cross sections measured at ATLAS and CMS for single-top quark production associated with a $\gamma$ or a $Z$ boson are compared with theory predictions in Fig.~\ref{fig:tBosonsummaryXsec}.

% Figure environment removed

The Run 3 of the LHC is ongoing, with a center-of-mass energy of 13.6 TeV. One can expect measurements for all of the processes discussed in this review to be repeated at this unprecedented energy, to probe the agreement between the data and the standard model predictions. 
With the luminosity accumulating (around 140 fb$^{-1}$ collected at separately at ATLAS and CMS at Run 2, and 200 fb$^{-1}$ expected at Run 3), even rarer processes can be reached. 
There is good hope that the $s$-channel process could be observed at the LHC Run 3. 
The measurement of $tWZ$ is the first of its kind, featuring a single-top quark accompanied with two bosons, and it could be inverstigated in more depth. 
Along the same lines, increased luminosity will allow to measure differential cross sections for the $t\gamma$ process. The search for the $tH$ processes will continue, although its observation might be postponed to the HL-LHC. 

Beyond these extensions of the already engaged single-top quark program, new possibilities can be explored. 
Using boosted top quarks with jet substructure is one of them (already used for \ttbar\ measurements~\cite{CMS:2021vhb} or in Ref.~\cite{CMS-PAS-TOP-22-008}). 
The production process for three top quarks is sometimes classified as belonging to the realm of the single-top quark physics; it is included as a small background in the measurement of the four top quark process~\cite{ATLAS:2023ajo,CMS:2023ftu}, and would deserve a direct search. 
single-top quark production with a combination of two $W$, $Z$ or photons could be measured, beyond $tWZ$; and there high sensitivity to effective field theory couplings could be fruitfuly employed. Some studies also show that the production of single-top quarks in vector boson fusion would also be an interesting, rare, process to be searched for~\cite{Degrande:2018fog}. 
In general, the program of measuring the top quark couplings within the effective field theory is still at its infancy. One can hope that the couplings where the single-top quark area is relevant will be measured systematically at the LHC Run 3 and at the HL-LHC, for instance the $Wtb$ couplings and especially their possible CP violation, as well as the top quark couplings to the gauge bosons. Combination with measurements of other top quark production modes, or with electroweak and Higgs boson measurements, will certainly be a must and could lead to the observation of statistical deviations pointing to physics beyond the SM.






%%%%%%%%%%%%%%%%%%%%%%%%%%%%%%%%%%%%%%%%%%
\vspace{6pt} 

%%%%%%%%%%%%%%%%%%%%%%%%%%%%%%%%%%%%%%%%%%
%% optional
%\supplementary{The following supporting information can be downloaded at:  \linksupplementary{s1}, Figure S1: title; Table S1: title; Video S1: title.}

% Only for journal Methods and Protocols:
% If you wish to submit a video article, please do so with any other supplementary material.
% \supplementary{The following supporting information can be downloaded at: \linksupplementary{s1}, Figure S1: title; Table S1: title; Video S1: title. A supporting video article is available at doi: link.}

% Only for journal Hardware:
% If you wish to submit a video article, please do so with any other supplementary material.
% \supplementary{The following supporting information can be downloaded at: \linksupplementary{s1}, Figure S1: title; Table S1: title; Video S1: title.\vspace{6pt}\\
%\begin{tabularx}{\textwidth}{lll}
%\toprule
%\textbf{Name} & \textbf{Type} & \textbf{Description} \\
%\midrule
%S1 & Python script (.py) & Script of python source code used in XX \\
%S2 & Text (.txt) & Script of modelling code used to make Figure X \\
%S3 & Text (.txt) & Raw data from experiment X \\
%S4 & Video (.mp4) & Video demonstrating the hardware in use \\
%... & ... & ... \\
%\bottomrule
%\end{tabularx}
%}

%%%%%%%%%%%%%%%%%%%%%%%%%%%%%%%%%%%%%%%%%%
%\authorcontributions{For research articles with several authors, a short paragraph specifying their individual contributions must be provided. The following statements should be used ``Conceptualization, X.X. and Y.Y.; methodology, X.X.; software, X.X.; validation, X.X., Y.Y. and Z.Z.; formal analysis, X.X.; investigation, X.X.; resources, X.X.; data curation, X.X.; writing---original draft preparation, X.X.; writing---review and editing, X.X.; visualization, X.X.; supervision, X.X.; project administration, X.X.; funding acquisition, Y.Y. All authors have read and agreed to the published version of the manuscript.'', please turn to the  \href{http://img.mdpi.org/data/contributor-role-instruction.pdf}{CRediT taxonomy} for the term explanation. Authorship must be limited to those who have contributed substantially to the work~reported.}

\funding{This research received no external funding.}

%\institutionalreview{In this section, you should add the Institutional Review Board Statement and approval number, if relevant to your study. You might choose to exclude this statement if the study did not require ethical approval. Please note that the Editorial Office might ask you for further information. Please add “The study was conducted in accordance with the Declaration of Helsinki, and approved by the Institutional Review Board (or Ethics Committee) of NAME OF INSTITUTE (protocol code XXX and date of approval).” for studies involving humans. OR “The animal study protocol was approved by the Institutional Review Board (or Ethics Committee) of NAME OF INSTITUTE (protocol code XXX and date of approval).” for studies involving animals. OR “Ethical review and approval were waived for this study due to REASON (please provide a detailed justification).” OR “Not applicable” for studies not involving humans or animals.}

%\informedconsent{Any research article describing a study involving humans should contain this statement. Please add ``Informed consent was obtained from all subjects involved in the study.'' OR ``Patient consent was waived due to REASON (please provide a detailed justification).'' OR ``Not applicable'' for studies not involving humans. You might also choose to exclude this statement if the study did not involve humans.

%Written informed consent for publication must be obtained from participating patients who can be identified (including by the patients themselves). Please state ``Written informed consent has been obtained from the patient(s) to publish this paper'' if applicable.}

%\dataavailability{We encourage all authors of articles published in MDPI journals to share their research data. In this section, please provide details regarding where data supporting reported results can be found, including links to publicly archived datasets analyzed or generated during the study. Where no new data were created, or where data is unavailable due to privacy or ethical restrictions, a statement is still required. Suggested Data Availability Statements are available in section ``MDPI Research Data Policies'' at \url{https://www.mdpi.com/ethics}.} 

% Only for journal Nursing Reports
%\publicinvolvement{Please describe how the public (patients, consumers, carers) were involved in the research. Consider reporting against the GRIPP2 (Guidance for Reporting Involvement of Patients and the Public) checklist. If the public were not involved in any aspect of the research add: ``No public involvement in any aspect of this research''.}

% Only for journal Nursing Reports
%\guidelinesstandards{Please add a statement indicating which reporting guideline was used when drafting the report. For example, ``This manuscript was drafted against the XXX (the full name of reporting guidelines and citation) for XXX (type of research) research''. A complete list of reporting guidelines can be accessed via the equator network: \url{https://www.equator-network.org/}.}

%\acknowledgments{In this section you can acknowledge any support given which is not covered by the author contribution or funding sections. This may include administrative and technical support, or donations in kind (e.g., materials used for experiments).}

\conflictsofinterest{The authors declare no conflict of interest.} 


%%%%%%%%%%%%%%%%%%%%%%%%%%%%%%%%%%%%%%%%%%
%% Optional
%\appendixtitles{no} % Leave argument "no" if all appendix headings stay EMPTY (then no dot is printed after "Appendix A"). If the appendix sections contain a heading then change the argument to "yes".
%\appendixstart


%%%%%%%%%%%%%%%%%%%%%%%%%%%%%%%%%%%%%%%%%%
\begin{adjustwidth}{-\extralength}{0cm}
%\printendnotes[custom] % Un-comment to print a list of endnotes

\reftitle{References}
%\dfrac{•}{•}
% Please provide either the correct journal abbreviation (e.g. according to the “List of Title Word Abbreviations” http://www.issn.org/services/online-services/access-to-the-ltwa/) or the full name of the journal.
% Citations and References in Supplementary files are permitted provided that they also appear in the reference list here. 

%=====================================
% References, variant A: external bibliography
%=====================================
\bibliography{SingleTopReviewUniverse}

%=====================================
% References, variant B: internal bibliography
%=====================================
%\begin{thebibliography}{999}
% Reference 1
%\bibitem[Author1(year)]{ref-journal}
%Author~1, T. The title of the cited article. {\em Journal Abbreviation} {\bf 2008}, {\em 10}, 142--149.
% Reference 2
%\bibitem[Author2(year)]{ref-book1}
%Author~2, L. The title of the cited contribution. In {\em The Book Title}; Editor 1, F., Editor 2, A., Eds.; Publishing House: City, Country, 2007; pp. 32--58.
% Reference 3
%\bibitem[Author3(year)]{ref-book2}
%Author 1, A.; Author 2, B. \textit{Book Title}, 3rd ed.; Publisher: Publisher Location, Country, 2008; pp. 154--196.
% Reference 4
%\bibitem[Author4(year)]{ref-unpublish}
%Author 1, A.B.; Author 2, C. Title of Unpublished Work. \textit{Abbreviated Journal Name} %year, \textit{phrase indicating stage of publication (submitted; accepted; in press)}.
% Reference 5
%\bibitem[Author5(year)]{ref-communication}
%Author 1, A.B. (University, City, State, Country); Author 2, C. (Institute, City, State, Country). Personal communication, 2012.
% Reference 6
%\bibitem[Author6(year)]{ref-proceeding}
%Author 1, A.B.; Author 2, C.D.; Author 3, E.F. Title of presentation. In Proceedings of the Name of the Conference, Location of Conference, Country, Date of Conference (Day Month Year); Abstract Number (optional), Pagination (optional).
% Reference 7
%\bibitem[Author7(year)]{ref-thesis}
%Author 1, A.B. Title of Thesis. Level of Thesis, Degree-Granting University, Location of University, Date of Completion.
% Reference 8
%\bibitem[Author8(year)]{ref-url}
%Title of Site. Available online: URL (accessed on Day Month Year).
%\end{thebibliography}

% If authors have biography, please use the format below
%\section*{Short Biography of Authors}
%\bio
%{\raisebox{-0.35cm}{% Figure removed}}
%{\textbf{Firstname Lastname} Biography of first author}
%
%\bio
%{\raisebox{-0.35cm}{% Figure removed}}
%{\textbf{Firstname Lastname} Biography of second author}

% For the MDPI journals use author-date citation, please follow the formatting guidelines on http://www.mdpi.com/authors/references
% To cite two works by the same author: \citeauthor{ref-journal-1a} (\citeyear{ref-journal-1a}, \citeyear{ref-journal-1b}). This produces: Whittaker (1967, 1975)
% To cite two works by the same author with specific pages: \citeauthor{ref-journal-3a} (\citeyear{ref-journal-3a}, p. 328; \citeyear{ref-journal-3b}, p.475). This produces: Wong (1999, p. 328; 2000, p. 475)

%%%%%%%%%%%%%%%%%%%%%%%%%%%%%%%%%%%%%%%%%%
%% for journal Sci
%\reviewreports{\\
%Reviewer 1 comments and authors’ response\\
%Reviewer 2 comments and authors’ response\\
%Reviewer 3 comments and authors’ response
%}
%%%%%%%%%%%%%%%%%%%%%%%%%%%%%%%%%%%%%%%%%%
\PublishersNote{}
\end{adjustwidth}
\end{document}

