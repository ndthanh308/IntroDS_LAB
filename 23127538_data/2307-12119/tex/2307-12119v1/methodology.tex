\section{Thermal Estimation Considering Variability}
\label{sec:method}

\subsection{Overview}
 

% Figure environment removed
        

Our approach fundamentally uses the Green's functions to compute the full-chip thermal profile. These can either be theoretically calculated, or empirically obtained by applying a unit impulse power source to the center of the chip, and measuring the corresponding temperature rise.
To consider the feedback effect of leakage power on temperature, Sarangi et al.~\cite{lightsim} derived a leakage-aware Green's function that captures this temperature-dependence of leakage power.
However, there are two effects that have not been considered in the use of these functions for full-chip temperature estimation: 
\begin{enumerate}[wide, labelindent=0pt]
\item The first effect is that the conductivity of silicon is also temperature-dependent, and has a non-negligible effect on temperature.
To capture this effect, we derive a novel modified Green's function, that not only captures the temperature-dependence of leakage power, but also captures the temperature-dependence of conductivity.

\item  The second unmodeled effect is process variation. Process variation poses multiple challenges in the use of Green's function-based methods.
The first challenge is in deriving the Green's function itself considering the effects of process variation as well.
{In a previous version of this work~\cite{varsim}, we had used the expected value of the leakage power map in place of the baseline leakage power map, to simplify the derivation of the modified temperature-dependent Green's function (note that the Green's function itself had captured the baseline variability, and this was not approximated; the approximation was in computing the temperature-dependent part of the leakage power). 
In the current work, we get rid of this approximation, and work out the complete expression.}

The second challenge is in computing the full-chip thermal profile using the derived Green's function. Temperature estimation using the Green's function relies on its \textit{shift-invariance}, which means that a power applied to any location of the chip will cause the same temperature rise, irrespective of the location of the power source on the chip (aside from the boundary effects, which need to be handled separately). This stops being true when process variation is considered. Hence, we split the Green's function into random and deterministic components. Then, we propose a novel way of combining the two components to generate the full-chip temperature profile. The deterministic component of the full-chip temperature profile is computed the regular way, using a convolution operation. The random component is location-dependent and hence is computed using the Hadamard product.

\end{enumerate}

We provide an overview of our proposed approach in Figure~\ref{fig:flow}. 
Table~\ref{tab:glossary} lists the terms used in our derivations.


Thus, our contributions can be summarized as follows:
\begin{enumerate}[wide, labelwidth=!, labelindent=0pt]

\item [\one] We derive an analytical expression for a novel leakage aware Green's function accounting for 
variability and 
the dependence 
of leakage power and conductivity on temperature. This Green's function has two components -- a shift-invariant component that accounts for the temperature dependence of conductivity and leakage power, and a shift-variant component that accounts for process variation.
We derive separate modified Green's functions for the steady-state (Section~\ref{sec:steady}) and the transient case (Section~\ref{sec:tran}). Furthermore, we describe the use of the transient Green's function to compute the temperature profile for a time-varying source in Section~\ref{sec:truetran}.


\item [\two] We calculate the full-chip variability-aware thermal profile in the presence of leakage as 
well as dynamic power by convolving the shift-invariant modified Green's functions with the dynamic power map and multiplying the shift-varying Green's functions with the baseline leakage power map considering process variation.


\end{enumerate}


\begin{table}[!hbtp]
\centering
\caption{Glossary \label{tab:glossary}}
	\begin{tabular}{c p{7cm}}
	\toprule
	\bf{Symbol} & \bf{Meaning} \\
	\midrule
	$\Pleak$ 	& Leakage power at ambient temperature considering variability\\
	$\beta$	 	& Temperature dependence of leakage power\\
	$\alpha$ 	& Temperature dependence of conductivity\\
	$\kappa$ 	& Conductivity of silicon\\
	$T$		 	& Temperature\\
	$\T$		& Temperature rise above ambient temperature\\
	$f_{sp_0}$	& Green's function without considering leakage and temperature-dependent conductivity $= f_{silic_0} + \phi $\\
	$\E(X)$		& Expected value of $X$\\
	$\fourier$	& Fourier transform operator \\
	$\hankel$		& Hankel transform operator\\
	$x,y$			& Spatial coordinates\\
	$u,v$			& Fourier frequency domain variables\\	
	$h$			& Hankel variable\\	
	$t$			& Time\\
	$C$			& Thermal capacitance\\
	$f_{leaksp}^k$ & Leakage aware Green's functions considering temperature dependence of conductivity\\
	$k_0'$ 		& Nominal conductivity of silicon at ambient temperature\\
	\bottomrule
	\end{tabular}
\end{table}

 
\subsection{Variation of the Thermal Conductivity with Temperature}
The conductivity of silicon varies with temperature according to the following relation~\cite{yuK}:
\begin{equation}
\kappa = k_{0} \left({T \over 300}\right)^{-\eta}, 
\label{eqn:kvsT}
\end{equation}
where $k_{0}$ is the conductivity of silicon at 300K, $T$ is the temperature in Kelvin, and $\eta$ is a material-dependent constant. 
In the operating range of ICs ($40 - 100$\celsius), we can linearize Equation~\ref{eqn:kvsT}:
\begin{equation}
\label{eqn:linkvsT}
\kappa(T) = {k_{0}}'(1-c\Delta T), 
\end{equation}
where $k_0'$ is the nominal conductivity of silicon at the ambient temperature, and $c$ is a constant. 

Next, we vary the conductivity of silicon and observe the change in the Green's function $f_{sp}$. We then obtain an empirical relation between the Green's function and conductivity (using HotSpot~\cite{hotspot6}):

 
\begin{equation}
\label{eqn:fspvsk}
f_{sp}(\kappa) = f_{sp_0}(1 - c'(\kappa(T) - k_0')), 
\end{equation}

where, $f_{sp_0}$ is the baseline Green's function when the temperature dependence of conductivity is not considered (the variation in conductivity because of random dopant fluctuations is captured in $f_{sp_0}$), and $c'$ is a constant.
Combining Equations~\ref{eqn:linkvsT} and \ref{eqn:fspvsk}, we obtain a relation for the Green's function that captures the dependence of conductivity on temperature.
\begin{equation}
\label{eqn:fspvsT}
f_{sp}(T) = f_{sp_0}(1 + \alpha\Delta T),
\end{equation}
where $\alpha$ is a constant that captures how much the Green's function varies because of a change in conductivity with temperature. 

\subsection{Modified Green's Functions for the Steady-State}
\label{sec:steady}
\subsubsection{Formulating the equation for the modified Green's function considering all effects}
Next, we derive the Green's function considering temperature-dependent conductivity, as well as temperature-dependent leakage power. We start with
an approach that is similar to that adopted by Sarangi et al.~\cite{lightsim} while also incorporating the effects of process variation and
the temperature dependence of conductivity.


The total power consumption ($P$) is the sum of the dynamic power ($P_{dyn}$) and the leakage power ($P_{leak}$). Using 
Equation~\ref{eqn:Pleak}, we get:
\begin{equation}
\label{eqn:Pt}
P = P_{dyn} + \Pleak(1 + \beta \Delta T).
\end{equation}

We assume that the dynamic power dissipation in the chip is initially zero. This implies that any temperature rise above the ambient temperature is because of leakage. From Equation~\ref{eqn:Green}, we have (subscript $_0$ refers to the initial state):
\begin{equation}
\label{eqn:T0}
T_0 = f_{sp_0} \star \Pleak
\end{equation}

Now, let us apply a unit impulse (Dirac delta function) dynamic power source to the center of the chip. Using the results in Equations~\ref{eqn:Green},
 \ref{eqn:fspvsT}, and \ref{eqn:Pt}  we get the updated temperature, $T_f$:
\begin{equation}
\label{eqn:Tf}
T_f = f_{sp_0}(1 + \alpha \Delta T) \star \left(\delta(x,y) +  \Pleak(1 + \beta \Delta T)\right)
\end{equation}
where $x,y$ are the spatial coordinates.

 
Next, we find out the increase in temperature because of the unit power source considering process variation and the temperature-dependent effects. We also use the property that the convolution of a function and a delta function is the function itself. We then arrive at:
\begin{equation}
\begin{split}
\label{eqn:Tsub}
\T &= T_f - T_0 \\
 &= f_{sp_0}(1 + \alpha\T)  + f_{sp_0}(1 + \alpha\T) \star \Pleak(1 + \beta \T) \\& 
 - f_{sp_0} \star \Pleak
\end{split}
\end{equation}
 

We need to solve for the temperature rise, $\T$, here. To convert the convolution operation into multiplication, {we compute the Fourier transform on both sides and} apply the property that the Fourier transform of the convolution of two functions is equal to the product of their individual Fourier transforms.
We compute the Fourier transform of both sides of Equation~\ref{eqn:Tsub} to 
arrive at 
Equation~\ref{eqn:FTf}. 
\begin{equation}
\begin{split}
\label{eqn:FTf}
\fourier(\T) &= \left( \fourier(f_{sp_0}) + \alpha \fourier(f_{sp_0}\T)\right) + \left(\fourier(f_{sp_0}) + \alpha \fourier(f_{sp_0}\T)\right) \times\\
			  &\left(\fourier(\Pleak) + \beta\fourier(\Pleak \T)\right) - \fourier(f_{sp_0}) \fourier(\Pleak)\\		  
			 &=  \underbrace{\fourier(f_{sp_0})}_{I} + \shadedboxb{\underbrace{\alpha \fourier(f_{sp_0}\T)}_{II}} +  \shadedboxc{\underbrace{\beta \fourier(f_{sp_0})\fourier({\Pleak \T})}_{III}}\\ & {+} 
\shadedboxd{\underbrace{\alpha \fourier(f_{sp_0}\T)\fourier(\Pleak)}_{IV}}
 + \shadedboxe{\underbrace{\alpha \beta\fourier(\Pleak \T)\fourier(f_{sp_0}\T)}_{V}}
\end{split}
\end{equation}

In Equation~\ref{eqn:FTf}, {$\fourier$ is the Fourier transform operator,} term $I$ is the baseline Green's function, term $II$ corresponds to the increase in 
temperature because of the temperature-dependence of conductivity, term $III$ is the increase in temperature because 
of the temperature-dependence of leakage power, term $IV$ corresponds to the compounded 
effect of the baseline leakage 
power and the temperature-dependence of the conductivity, and term $V$ is the 
increase in temperature because of the compounded effects of  temperature-dependent conductivity and leakage. The last term here is small because each of the temperature-dependent variables (conductivity/leakage power) by itself do  not cause a large enough change in the other variable to result in a large temperature change. Hence, we 
neglect this term. 


\subsubsection{Reducing the bottleneck term (Term II)}

The most difficult term to compute in the above equation is $\fourier(f_{sp_0}\T)$. Let $G(u,v) =  \fourier(f_{sp_0}\T)$. 
For this we make use of Lemma I described below:
\subsubsection*{Lemma I}: $G(u,v) =  \fourier(f_{sp_0}\T) = \fourier (\T) g_{sp_0}$, where $g_{sp_0} = \left( f_{sp_0} - \kappa + f_{sp_0}(0,0)\right)$

Interested readers may refer to the proof of Lemma 1 in the appendix. 	   

\noindent Using Lemma I in Equation~\ref{eqn:FTf}, we have:

 

\begin{equation}
\begin{split}
\label{eqn:expanded1}
 \fourier(\T) &= \fourier(f_{sp_0}) + \alpha g_{sp_0} \fourier(\T) 
 +\beta \fourier(f_{sp_0})\fourier(\Pleak \T)\\&
 +\alpha g_{sp_0} \fourier(\Pleak)\fourier(\T)
 \end{split}
 \end{equation}
 




 
\subsubsection{Separating the random and deterministic terms}	
The modified Green's function considering variability (solution of Equation~\ref{eqn:expanded1}) is not shift-invariant because of process variation (terms $III$ and $IV$), which means that we will not be able to directly convolve the modified Green's function with a power profile without a loss of accuracy.
To overcome this limitation, we split the modified Green's function  into two components: a deterministic component $f_{leaksp}^{det}$, which is shift-invariant and is obtained by assuming the variability to be zero (replacing the baseline leakage power profile with its mean value), and a random component $f_{leaksp}^{rand}$, which is shift-variant and accounts for all the variation in leakage power.


To arrive at the respective expressions for the modified Green's functions, we first split the variable leakage power, $\Pleak$, into two components -- a constant equal to its mean ($\mu$) and a randomly varying part $P_{leak_0}^{var}$. Thus, $\Pleak = \shadedboxb{\mu + P_{leak_0}^{var}}$. Using this in Equation~\ref{eqn:expanded1}, we get:
\begin{equation}
\begin{split}
\label{eqn:expanded2}
 \fourier(\T) &= \fourier(f_{sp_0}) {+} \alpha g_{sp_0} \fourier(\T) 
 {+}\beta \fourier(f_{sp_0})\fourier\bigl(\shadedboxb{(\mu + P_{leak_0}^{var}}) \T\bigr)\\&
 +\alpha g_{sp_0} \fourier\left(\shadedboxb{\mu + P_{leak_0}^{var}}\right)\fourier(\T)\\
 &= \fourier(f_{sp_0}) + \alpha g_{sp_0} \fourier(\T)
 +\mu \beta \fourier(f_{sp_0})\fourier(\T) \\&
 +\beta \fourier(f_{sp_0})\fourier(P_{leak_0}^{var} \T) 
  +\alpha g_{sp_0} \fourier(\mu)\fourier(\T)\\&
 +\alpha g_{sp_0} \fourier(P_{leak_0}^{var})\fourier(\T)\\
 \end{split}
 \end{equation}
 
 
 
 Now, the temperature profile  $\T$ is itself composed of a deterministic and a variable part, $\T = \T^{det} + \T^{var}$.
 The deterministic part can be obtained by assuming the variability to be zero, $P_{leak_0}^{var} = 0$.
 Applying this to Equation~\ref{eqn:expanded2}, we arrive at Equation~\ref{eqn:expanded3}:
  \begin{equation}
\begin{split}
\label{eqn:expanded3}
 \fourier(\T^{det}) &=\fourier(f_{sp_0}) + \alpha g_{sp_0} \fourier(\T^{det})
 +\mu \beta \fourier(f_{sp_0})\fourier(\T^{det})+ \\&
  \alpha g_{sp_0} \fourier(\mu)\fourier(\T^{det})\\
 \end{split}
 \end{equation}


  \begin{equation}
\begin{split}
\label{eqn:detsol}
& \fourier(\T^{det})  =\fourier(f_{leaksp}^{det}) = 
 \frac{\fourier(f_{sp_0})}{1-\alpha g_{sp_0}(1+\fourier(\mu)) -\mu \beta \fourier(f_{sp_0})}
 \end{split}
 \end{equation}
 
 Next, we use an equation similar to Equation~\ref{eqn:leakvar2} for $\fourier(P_{leak_0}^{var} \T)$. We get:
 
   \begin{equation}
\begin{split}
\label{eqn:randTsplit}
  \fourier(P_{leak_0}^{var} \T) &= \fourier(\T)\bigl(\underbrace{P_{leak_0}^{var} -  P_{leak_0}^{var}(\infty,\infty) + P_{leak_0}^{var}(0,0) }_{=Q_{leak_0}^{rand}}\bigr)\\
  &= \fourier(\T) Q_{leak_0}^{rand}
  \end{split}
 \end{equation} 
 
 Substituting Equation~\ref{eqn:randTsplit} and  $\T = \T^{det} + \T^{var}$ in Equation~\ref{eqn:expanded2} and simplifying, we finally get:
 
%%%%%% TODO: REMOVED in reviews, BRING BACK
% \begin{equation}
% \begin{split}
%\label{eqn:expanded4}
% \fourier(\T^{det} + \T^{var}) =&\fourier(f_{sp_0}) + \alpha g_{sp_0} \fourier(\T^{det} + \T^{var})
% \\&
% +\mu \beta \fourier(f_{sp_0})\fourier(\T^{det} + \T^{var}) +\\&
% \beta \fourier(f_{sp_0})\fourier(\T^{det} + \T^{var})Q_{leak_0}^{rand}  \\&
%  +\alpha g_{sp_0} \fourier(\mu)\fourier(\T^{det} + \T^{var})\\&
% +\alpha g_{sp_0} \fourier(P_{leak_0}^{var})\fourier(\T^{det} + \T^{var})\\
% \end{split}
% \end{equation}

 \begin{equation}
\begin{split}
\label{eqn:expanded5}
 \fourier(\T^{det})&\left(\shadedboxc{1-  \alpha\left(1+\fourier(\mu)\right)g_{sp_0}
 -\mu \beta \fourier(f_{sp_0})} - \right.\\&
 \left.\beta \fourier(f_{sp_0})Q_{leak_0}^{rand} 
-\alpha \fourier(P_{leak_0}^{var})g_{sp_0} \right) +  \\
%...
 \fourier(\T^{var})&\left(1-  \alpha(1+\fourier(\mu))g_{sp_0}
 -\mu \beta \fourier(f_{sp_0}) \right.\\&- 
 \beta \fourier(f_{sp_0})Q_{leak_0}^{rand} 
-\left.\alpha \fourier(P_{leak_0}^{var})g_{sp_0}\right)  \\ & = \shadedboxc{\fourier(f_{sp_0})}
 \end{split}
 \end{equation}

Now, we substitute the expression for $\fourier(f_{sp_0})$ from Equation~\ref{eqn:expanded3} in Equation~\ref{eqn:expanded5}. After cancelling the common terms (shaded/green ones) we get: 
 \begin{equation}
\begin{split}
\label{eqn:expanded6}
 \fourier(\T^{var})&\left(1{-}  \alpha(1{+}\fourier(\mu))g_{sp_0}
 {-}\mu \beta \fourier(f_{sp_0}) \right.{-} 
 \beta \fourier(f_{sp_0})Q_{leak_0}^{rand} \\&
-\left.\alpha \fourier(P_{leak_0}^{var})g_{sp_0}\right) \\ & =  \fourier(\T^{det})\left( 
 \beta \fourier(f_{sp_0})Q_{leak_0}^{rand}+\alpha \fourier(P_{leak_0}^{var})g_{sp_0}\right) \\
 \end{split}
 \end{equation}
 
  \begin{equation}
\begin{split}
\label{eqn:randsol}
 &\fourier(\T^{var}) = \fourier(f_{leaksp}^{rand})\\ & = \fourier(\T^{det})\times   \frac{\left(  \beta \fourier(f_{sp_0})Q_{leak_0}^{rand} + \alpha \fourier(P_{leak_0}^{var})g_{sp_0}\right) }{\splitfrac{\Bigl(1-  \alpha\Bigl(1+\fourier(\mu)+\fourier(P_{leak_0}^{var})\Bigr)g_{sp_0}
  \Bigr.}{\Bigl.-\mu \beta \fourier(f_{sp_0}) -
 \beta \fourier(f_{sp_0})Q_{leak_0}^{rand} \Bigr)}}
 \end{split}
 \end{equation}
 
The deterministic part of the Green's function remains the same for every variation profile. Hence, it needs to be computed once for a given chip only. The random part needs to be recomputed every time we get a new variation map. 


\subsubsection{Full-chip steady-state thermal profile}
The total temperature profile is a sum of the random and deterministic components. The standard approach to compute the deterministic thermal profile is to convolve the deterministic Green's function (Equation~\ref{eqn:detsol}) with the respective power profile. 
The random component of the thermal profile is not shift-invariant and cannot be computed using the convolution operation, since it is location-dependent. Hence, we compute the Hadamard product of the random component with the leakage power profile and scale it by the total dynamic power applied to the chip. This is an approximation that we justify empirically after conducting exhaustive experiments.

 
Thus the total thermal profile is given by:
   \begin{equation}
\begin{split}
\label{eqn:detrandsol}
 \T = f_{leaksp}^{det} \star P_{dyn} + f_{leaksp}^{rand}* P_{leak_0}^{var}* \sum_{i=1}^{n}\sum_{j=1}^{n} P_{dyn(i,j)} 
 \end{split}
 \end{equation}
where $*$ represents the Hadamard product, $\star$ represents the convolution operation, and $n$ represents the number of grid points in the chip in one direction. 

 
 
\subsection{Modified Green's function for the Transient Case}
\label{sec:tran}
Next, we look at the temporal evolution of temperature.
Because of the complexity involved in obtaining the transient solution, we do not split the transient Green's function into shift-variant and shift-invariant components. 

The basic transient Green's function equation is given by \cite{lightsim}:
\begin{equation}
\label{eqn:Ttran}
\T = f_{sp} \star P - C f_{sp}\star \frac{\partial \T}{\partial t},
\end{equation}
where $C$ is the thermal capacitance. 

Proceeding in the same manner as the steady-state solution, and looking at Equation~\ref{eqn:leakvar2}, we arrive at the following equation for the transient case:

\begin{equation}
\begin{split}
\label{eqn:FTftran}
\fourier&(\T)  =\fourier(f_{sp_0}) + \alpha g_{sp_0} \fourier(\T) 
 +\beta \fourier(f_{sp_0})\fourier(\Pleak \T)+\\&
 \alpha g_{sp_0} \fourier(\Pleak)\fourier(\T)
 {-} C \bigl(\fourier(f_{sp_0}) {+} 
\alpha \underbrace{\fourier(f_{sp_0}\T)}_{G(u,v)}\bigr) \fourier \left(\frac{\partial \T}{\partial t}\right)\\
&= \fourier(f_{sp_0}) + \alpha g_{sp_0} \fourier(\T) 
 +\beta \fourier(f_{sp_0})\fourier(\Pleak \T)+\\&
 \alpha g_{sp_0} \fourier(\Pleak)\fourier(\T) {-} C \bigg(\fourier(f_{sp_0}) {+}
\alpha g_{sp_0}\fourier(\T) \bigg) \fourier \left(\frac{\partial \T}{\partial t}\right)\\
\end{split}
\end{equation}


The first three terms on the RHS correspond to the steady-state temperature profile, $\T_{ss}$. 
Thus we have:
\begin{equation}
\begin{split}
\fourier(\T) =& \fourier(\T_{ss})- C \bigl(\fourier(f_{sp_0}) + 
\shadedbox{\alpha \fourier(\T)g_{sp_0} } \bigr) \fourier \left(\frac{\partial \T}{\partial t}\right)\\
=& \fourier(\T_{ss})- C \fourier(f_{sp_0})\left(\frac{\partial \fourier(\T)}{\partial t}\right) \\& -\shadedbox{\alpha C 
 \fourier(\T)g_{sp_0}    \left(\frac{\partial \fourier(\T)}{\partial t}\right)}\\
\end{split}
\end{equation}
The shaded term is of the form $\fourier(\T)\frac{\partial \fourier(\T)}{\partial t}$, making the solution complex.


We separate the partial derivative term next:
\begin{equation}
\begin{split}
 \frac{\partial \fourier(\T)}{\partial t}& = -\frac{\fourier(\T) - \fourier(\T_{ss})}{C\fourier(f_{sp_0}) + 
\alpha C \fourier(\T)g_{sp_0}  }
\end{split}
\end{equation}

Separating the variables and replacing partial derivatives with total derivatives (since there is only one variable):
\begin{equation}
\begin{split}
 \frac{C\fourier(f_{sp_0}) + 
\alpha C \fourier(\T)g_{sp_0}}{\fourier(\T) - \fourier(\T_{ss})}d \fourier(\T)& = -d t
\end{split}
\end{equation}
Integrating on both sides:
\begin{equation}
\label{eqn:traniter}
\begin{split}
&C\fourier(f_{sp_0}) ln(\fourier(\T) - \fourier(\T_{ss})) + 
\alpha C g_{sp_0}(\fourier(\T) - \fourier(\T_{ss})) +\\& \alpha C g_{sp_0}\fourier(\T_{ss})ln(\fourier(\T) {-} \fourier(\T_{ss}))  = -(t {+} C_1)\\
\\
&ln(\fourier(\T) - \fourier(\T_{ss}))\bigl(C\fourier(f_{sp_0}) + \alpha C g_{sp_0}\fourier(\T_{ss})\bigr)\\& + 
\alpha C g_{sp_0}(\fourier(\T) - \fourier(\T_{ss})) = -(t + C_1)\\
\end{split}
\end{equation}
We need to solve for $\fourier(\T)$ here.
We observe that the last term on the LHS is small since both $\alpha$ and $C$ are small numbers. Hence, we ignore this term.

Thus we have:
\begin{equation}
\label{eqn:tranlneqn}
\begin{split}
ln(\fourier(\T) {-} \fourier(\T_{ss}))\bigl(C\fourier(f_{sp_0}) {+} \alpha C g_{sp_0}\fourier(\T_{ss})\bigr) = -(t {+} C_1)\\
\end{split}
\end{equation}

Taking the exponential on both sides and simplifying:
\begin{equation}
\label{eqn:tranexpeqn}
\begin{split}
\fourier(\T) &= \fourier(\T_{ss}) + e^{-\frac{t+C_1}{C\fourier(f_{sp_0}) + \alpha C g_{sp_0}\fourier(\T_{ss})}} \\
&=\fourier(\T_{ss}) + k_1 e^{-\frac{t}{C\fourier(f_{sp_0}) + \alpha C g_{sp_0}\fourier(\T_{ss})}}
\end{split}
\end{equation}
where $k_1$ is a constant.

To compute $k_1$, we see that at $t=0$, the temperature rise $\T$ is zero. Substituting these in Equation~\ref{eqn:tranexpeqn}, we get $k_1 = -\fourier(\T_{ss})$.

Thus the final transient leakage and variation-aware Green's function is given by:
\begin{equation}
\begin{large}
\label{eqn:tranGreen}
\boxed{\fourier(\T) {=} f_{leaksp}^{tran} {=} \fourier(\T_{ss}) \bigl(1 - e^{-\frac{t}{C\fourier(f_{sp_0}) + \alpha C g_{sp_0}\fourier(\T_{ss})}}\bigr) }\\
\end{large}
\end{equation}


\subsection{Full-chip Transient Thermal Profile for Time-varying Sources}
\label{sec:truetran}
Next, we use the transient Green's function to compute the full-chip thermal profile corresponding to a time-varying power profile.
Note that the transient Green's function is a 3D function of space as well as time. 

We observe that the thermal response decays to under 1\% of its peak value within $5~ms$ after a power source is removed. Thus, we conclude that incorporating the thermal response corresponding to the power sources in the last $5~ms$ only should be sufficient to attain a reasonable accuracy. In the general case, we need to consider the power sources from the last $k$ time instants.


Let $P(t_i)$ denote the
instantaneous power profile at time $t_i$, and let $\T(t_i)$ be the temperature profile. 
The Green's function is obtained by applying a power source of unit magnitude and $1~ms$ width at $t=0$ at the center of the chip. The resulting thermal response is measured for the entire chip at intervals of 1~ms, resulting in a 3D tensor. 

We start with an initial temperature $T_0$, and at every time instant, we either increase or decrease the temperature depending on the change in power values in each time interval. To determine the amount of change in temperature, we compute a difference of the power profiles at time instant $t_i$ and $t_{i-1}$ and convolve this with the corresponding leakage-aware Green's function sampled at $t_i$. We do this for each of the last $k$ time steps, and sum the effects up. Beyond $k$ time steps, the power values in the past would have achieved steady-state and their effects would already have been considered.
Let us first describe our full-chip transient estimation approach mathematically for a 2D chip without leakage. Without any loss of generality, let us assume the initial temperature $T_0$ to be zero.


Let the Green's function at time instant $t_0$ be denoted by $f_{sp}(x,y,t_0)$. Let the power dissipation profile be $P(x,y,t)$. 

Let us apply a power source at $t=0$.

At $t=0$, the temperature rise is given by:
\begin{equation}
\label{eqn:tranapprox}
\begin{split}
\T(t_1) &= {fsp(t_1)} \star \left({P(t_1) - P(t_0)}\right)\\
&= {fsp(t_1)} \star {(P(t_1) - 0)}\\
\T(t_2) &= {fsp(t_1)} \star \left({P(t_2) - P(t_1)}\right) + {fsp(t_2)} \star \left({P(t_1) -  0}\right)\\
\T(t_3) &= {fsp(t_1)} \star \left({P(t_3) {-} P(t_2)}\right) {+}
 {fsp(t_2)} {\star} \left({P(t_2) {-} P(t_1)}\right)\\& + {fsp(t_3)} \star \left({P(t_1) -  0}\right)\\
...\\
\T(t_n) &= {fsp(t_1)} \star ({P(t_n) {-} P(t_{n-1})}) + {fsp(t_2)} \star \left(P(t_{n-1}) {-}\right. 
\\& \left. P(t_{n-2})\right) + ... +{fsp(t_{5})} \star({P(t_{n-5}) {-} P(t_{n-6})}) + \\& {fsp(t_{6})} \star({P(t_{n-6}) {-} P(t_{n-7}}) + ... + {fsp(t_{n})} \star \\& {(P(t_1) {-} P(t_0))}\\
\\
\end{split}
\end{equation}

Now, beyond $5~ms$, we assume that the step response saturates. 

Thus 
$fsp(t_{5}) = fsp(t_{6}) = fsp(t_{7}) .... =fsp(t_{\infty})$. 

Equation~\ref{eqn:tranapprox} reduces to:
\begin{footnotesize}
\begin{equation}
\begin{split}
\T(t_n) &\approx {fsp(t_1)} \star ({P(t_n) {-} P(t_{n-1})}) {+} {fsp(t_2)} \star \\
&\left({P(t_{n-1}) {-} P(t_{n-2})}\right)  + ... +{fsp(t_{\infty})} \star \\ & \bigl(\underbrace{{(P(t_{n-5}) {-} P(t_{n-6}) {+} P(t_{n-6}) {-} P(t_{n-7})  {+ ... +} 
P(t_1) {-} P(t_0))}}_{=P(t_{n-5})}\bigr)
\end{split}
\end{equation}
\end{footnotesize}

We can see that all terms cancel each other in the third convolution term, and only $P(t_{n-5})$ remains. \\
Thus we can calculate
the transient thermal profile by considering the last 5 time instants only.
To incorporate leakage, we simply need to substitute the \emph{modified leakage aware} Green's functions in place of the basic Green's function.


\subsection{Thermal Estimation at the Edges and Corners}
In the Green's function approach, the edges and corners have to be handled separately. The standard procedure to do so is to use an analogy with the method of images from 
electromagnetics (also used in \cite{powerblur2014}). In this approach, the power matrix is extended to twice its size and padded with mirror image sources on the other side of the boundary at an equal distance from the edge.
To compute the full-chip thermal profile, we convolve the modified deterministic Green's functions with the 
dynamic power profile and multiply the random Green's function with the baseline leakage power profile (Equation~\ref{eqn:detrandsol}). 





