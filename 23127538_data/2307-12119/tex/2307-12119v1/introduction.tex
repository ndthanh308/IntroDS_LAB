\section{Introduction}
\label{sec:intro}
The demand for high-performance computing as well as applications in fields such as machine learning, big data analytics, IoT, and edge computing has led to increased power densities in modern-day chips.
The resultant temperature rise has several harmful effects that include an increase in leakage power and a disproportionate 
decrease in reliability. 
Hence, thermal simulation is now one of the most critical
steps in the overall semiconductor design flow. It is typically a long and time-consuming process, that has to be repeated several times for a multitude of use cases in the design cycle.

To make matters worse, process variation has increasingly been leading to large deviations in 
electrical and thermal parameters of transistors, thereby leading to a high degree of unpredictability in key circuit parameters such as the timing delay and leakage power consumption. 
With ongoing device scaling, handling and mitigating process variation continues to become increasingly critical. Process variation affects all major architectural design decisions. 

Sadly, thermal modeling in chips with process variation is extremely complex and slow; till date no fast and efficient solutions have been 
proposed; researchers still rely on traditional Finite Element Method (FEM) and Finite Difference Method (FDM) analysis.

There is however a {\bf strong need} for fast thermal simulation methods in this space.
Many architectural techniques have been proposed with the aim of mitigating the adverse
effects of process variation such as functional unit level body biasing and retiming. However, to effectively incorporate such schemes, an accurate estimate of the impact of variation is needed. This requires extensive thermal simulations for a wide range of power plans.
Similarly, while floorplanning or cell placement, a large number of optimization strategies need to be quickly evaluated for a range of process variation scenarios~\cite{elseviermcm}. Leakage power has a strong temperature dependence and is heavily influenced by process variation as well. The chip temperature itself is dependent on leakage power, resulting in a cyclic effect. Thus leakage has a significant impact on the temperature of modern-day chips, often contributing to half of the total temperature rise. 
\textbf{Thus, a proper design optimization requires performing a thorough thermal evaluation through
simulations on chips having a significant range of process, conductivity and temperature variation, while correctly incorporating leakage power.} Hence, there is a need for a \textit{fast} thermal simulator in this space. A disclaimer is due -- at different stages of the
design process the designer has different degrees of information. Nevertheless, there is still a need for ultra-fast thermal simulation
because designers always seek the most productive design choices with the information that they have at that stage.
For example, at the architectural level, in the product planning stage, just a broad idea of the power consumption is available. Hence, the designer uses high level core power and variation models here, which still need to factor in the effects of temperature. After RTL signoff, placement and routing, synthesis and layout, progressively more accurate power numbers are available at each stage, and any change introduced at any level requires further modeling updates to guide the optimal design choices. This exercise needs to be done for every DCVS corner. After tapeout in the post silicon stage, the exact dynamic power numbers and a concrete idea of the process variation is available. 
Thus at every stage of the design process, each simulation produces crisp, exact data, and thousands of such simulations are run with different power dissipation (variation) values for a range of use cases. 
This Monte Carlo simulation yields a distribution that helps determine the final chip design. 
This is the standard practice widely used across the semiconductor industry.
 
Unfortunately, existing {\bf architectural thermal simulators do not consider the effects of process variation}. Ignoring process variation could lead to failure of the device after fabrication~\cite{sapatnekarvar}. Additionally, most simulators fail to factor in the temperature dependence of conductivity, leading to significant errors in thermal estimation. Prior work has shown that ignoring the temperature dependence of conductivity can result in an error of up to 5\celsius~\cite{isac}. Furthermore, traditional thermal simulators are based on the costly finite element and finite difference methods, making them slow and limiting the scope of design space exploration. 
On top of that, most existing thermal simulators consider the effects of leakage power by iterating through the leakage-temperature feedback loop, increasing their runtime several times. Since leakage power is indispensable in modern-day chips, it is essential for today's thermal simulators to naturally consider the effects of leakage power as part of the core modeling methodology and avoid iterative computations. 

Consequently, fast thermal estimation that takes variability into account has hitherto remained an open problem.
To solve this problem, we propose a thermal simulation methodology, \fname, in this paper. \textbf{\fname is a novel Green's function-based analytical thermal simulator, that inherently captures the effects of both process variation as well as leakage power, without running costly iterations.} Our main contributions can be summarized as follows:

\begin{enumerate}[wide, labelwidth=!, labelindent=0pt]
\item \fname considers the impact of process variation as well as the temperature dependence of conductivity. To the extent of our knowledge, no existing technique has done this. 

\item Our method is based on Green's functions (impulse response of a power source), which are known to be very fast methods~\cite{3dsim,sapatnekar}. However, Green's functions in thermal estimation rely on the shift-invariance of the impulse response, which ceases to be true when process variation is considered. We propose a novel way of overcoming this limitation by splitting the Green's function into a shift-invariant component and a shift-variant component.

\item We mathematically derive a novel, modified leakage-aware Green's function that incorporates in itself the impact of temperature-dependent conductivity and leakage power. This modified Green's function is directly used at runtime with the dynamic power profile to obtain an accurate estimate of the full-chip thermal profile considering all the desired effects. 
Since our method is analytical, it is also extremely fast. 
\item Green’s function-based methods have great potential, since their
accuracy is not dependent on the grid size, making them faster. However,
their applicability is limited by a lack of solutions for modern
EDA problems. Thus our work contributes to an ecosystem, where
more researchers can propose faster Green’s function-based solutions
for newer problems.
\end{enumerate}

Using an analytical Green's function-based approach, we obtain 
a several orders of 
magnitude speedup over state-of-the-art approaches, while keeping the maximum error within \error.
\textbf{Thus by leveraging the underlying physics of heat transfer and process variation, we are able to increase the accuracy of thermal simulations by considering all the critical temperature-affecting phenomena, while simultaneously achieving a very high simulation speed.} The speed advantage becomes even more pronounced when thermal simulations need to be repeated thousands of times, as is done in a typical design cycle. 

The rest of the paper is organized as follows.
We describe some background information for our work in Section~\ref{sec:backgnd}. The current state-of-the-art literature
is described in Section~\ref{sec:related}. We describe our modeling methodology in Section~\ref{sec:method}. Section~\ref{sec:eval} describes the evaluation of our proposed method and the corresponding results. We finally conclude in Section~\ref{sec:conc}.


