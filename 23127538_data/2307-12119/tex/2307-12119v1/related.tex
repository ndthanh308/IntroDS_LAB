\section{Related Work}
\label{sec:related}
{Thermal modeling has been a focus area of the EDA industry in the last two decades, and hence researchers have extensively worked on various aspects of this problem, such as 2D and 3D ICs~\cite{hotspot, pod, systemc,deepoheat,pathania3dttp}, smartphones and other mobile devices, thermal-aware DNN accelerators~\cite{coskun2023} .
 
Process variation per se has been widely studied along with techniques for mitigating its pernicious effects. 
However, very rarely has the effect of process variation on temperature been looked at.
The works that do consider the effects of process variation on temperature, often do so by neglecting the temperature dependence of leakage power~\cite{jaffari}. We demonstrate that considering the effects of process variation on leakage power, but neglecting its temperature dependence may result in a 4 to 6\celsius error.


Prior works have also established the importance of modeling the temperature-dependence of the conductivity of silicon as well, and proposed methods to tackle the problem. However, such approaches do not consider process variation. Since both of these effects have never been considered simultaneously before, we look at each of these effects in related work separately.

\subsection{Effects of Process Variation on Temperature}
Varipower~\cite{varipower} models power variability at the architectural functional unit level by performing circuit-level Monte Carlo simulations incorporating parameter variation. However, it does not model the effects of variability on temperature. 

Humenay et al.~\cite{skadronvar} recognized the challenges imposed by systematic variation in ensuring homogeneous performance across cores. They demonstrated a large variation in power, temperature and performance across cores because of core-to-core systematic variation.

Jaffari and Anis~\cite{jaffari} statistically calculated the expected value of temperature considering the impact of variability. They first obtain the leakage-converged temperature iteratively without considering variation and then statistically compute the effect of parameter variation. They use their technique to iteratively update the computed power and temperature to estimate the full-chip power   and the probability density function of the temperature distribution. However, a significant limitation of their technique is that it is iterative, 
making it extremely slow ($\approx 158s$ for a $50\times50$ grid), 121X slower than 
\fname. 
Juan et al. \cite{juan} use a linear regression-based model to train and predict the maximum temperature in a 
3D IC in the presence of variability. They use measured values of leakage power for training. They demonstrate that 3D ICs are much more susceptible to variation, as compared to 2D ICs. However, 
learning-based methods are very sensitive to input data and do not generalize well when test conditions change. Additionally, their method captures the maximum temperature only, and not the complete thermal profile.

Shafique et al.~\cite{henkelvar} propose a variability-aware dark silicon management technique in which the cores to be throttled are determined on their workload patterns while accounting for the temperature map and variability. They propose a complex heuristic for predicting the temperature profile considering process variation. They superpose the impact of variability-affected leakage power on the estimated temperature map for a given thread-to-core mapping.
This heuristic attempts to manually approximate the underlying logic that is captured well   in our modified Green's function based approach. Our proposed method achieves this in a precise and efficient mathematical manner using the convolution operations. In comparison to our method, their technique is inexact, unnecessarily complex and slow. 

{Srinivasa et al.~\cite{pvsmartphone} demonstrate using measurements that because of process variation, smartphones of the same model may show a variation of upto 10-12\% in energy and performance.}



\subsection{Modeling the Temperature Dependence of Conductivity}
Yang et al.~\cite{isac} propose a temporally and spatially adaptive thermal analysis technique that accounts for the temperature dependence of conductivity. However, they do not consider leakage power.

Li et al.~\cite{isacvar} calculate the leakage power in the presence of variation, and use this as an input to the ISAC thermal modeling tool~\cite{isac}. They update leakage power in each time step and   iteratively proceed towards convergence. Ultimately, they use this augmented tool to study process variation in network-on-chips. This approach is iterative, and would require a large number of iterations to get to accurate variation-aware leakage-converged temperature values. For the sake of
comparison, we implement a similar approach using the HotSpot thermal modeling tool~\cite{hotspot5}, and demonstrate several orders of magnitude speedup using our method over such approaches.

Ziabari et al.~\cite{ziabari} consider the temperature-dependence of conductivity by using a lookup table to store Green's functions with different conductivities. At runtime, they iteratively update the Green's function until the temperature profile converges. They, however, do not model leakage or process variation.
In comparison, our approach encompasses the effects of leakage power, variability in leakage, and temperature-dependent conductivity analytically without requiring costly iterations.

Köroğlu and Pop~\cite{insulators} propose high thermal conductivity insulators for effective thermal management in 3D ICs.

There is an extensive body of work in fast transient thermal modeling as well (~\cite{MLtran,comet}). These works focus on fast runtime thermal estimation, 2.5 and 3D stacked packages, and thermal models for TSVs but do not take process variation into account.

{He et al.~\cite{he} propose a novel polynomial chaos for modeling uncertainty at the architectural level using mixed integer programming.}

{Chittamuru et al.~\cite{pvnoc} demonstrate the sensitivity of photonic networks-on-Chip (NoCs) to thermal and process variation and propose a robust framework to overcome its impact on reliability.
}


\textbf{However, process variation, along with the variation of conductivity with temperature, has 
never been considered before in a leakage-aware thermal simulation tool. }
