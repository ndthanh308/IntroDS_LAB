\section{Evaluation}
\label{sec:eval}


\subsection{Setup}
We use an augmented version of the thermal modeling tool Hotspot~\cite{hotspot6} to carry out 
simulations with variable leakage power and conductivity. The scripts to invoke HotSpot has been written in R. 
We run all our HotSpot 
simulations on an Intel i7-7700 4-core CPU running Ubuntu 16.04 with 16 GB of RAM. 
We implemented and tested our proposed algorithm in Matlab on a Windows 8 desktop 
with an Intel i7-2600S processor and 8 GB of RAM.
We discretized the chip into a $64\times64$ grid. The parameters of the modeled chip are given in Table~\ref{tab:param}.

\begin{table}[!htb]
\small
 \begin{center}
 \caption{Parameters of the chip~\cite{hotspot6}\label{tab:param}}
 \begin{tabular}{p{5cm}l}
\toprule
Parameter & Value \\
\midrule
No. of grid points per layer, $n$		 & $64\times64$ \\
$\beta$ 	 & 0.0275  \\
Die size & 100 mm$^2$ \\
Die thickness & 0.15 mm \\
Nominal Silicon conductivity & 130 W/m-K \\
TIM thickness & 0.02 mm \\
TIM conductivity & 4 W/m-K \\
Spreader thickness & 3.5 mm\\
Spreader conductivity & 400 W/m-K \\
Ambient temperature & 318.15K \\
\bottomrule
\end{tabular}
\end{center}
\end{table}


\subsubsection*{Error Metric}
We use the mean absolute error and the percentage error relative to the maximum temperature \textbf{rise} (calculated temperature minus the ambient temperature) as the error metric. Other thermal modeling tools often report errors relative to the absolute maximum temperature in the die, which under-represents the error~\cite{powerblur2014, isac}.



\subsection{Calibration of the Setup}
To calibrate our HotSpot setup, we use the commercial CFD software Ansys Icepak. It is an industry-standard 
tool widely used for high-accuracy thermal simulations. We model an identical layout in HotSpot and Icepak 
and compare the temperature values obtained from the two tools. We find that the normalized temperature 
values obtained using both of these tools conform well (within 1.5\%).

\subsection{Steady-State Results}

\begin{table}[!htbp]
  \begin{tabular}{p{4.3cm}|p{1.5cm}|p{1.9cm}}
    \toprule
    {Simulator} &
      {Considering $\kappa(T)$} &
      {Without considering $\kappa(T)$} \\
      \midrule
    Hotspot\footnotemark & $18~minutes$ & $4~s$   \\
    3D-ICE & -- & $1.36~s$   \\
    Icepak & $15~minutes$ & $15~minutes$  \\
    Jaffari et al.\cite{jaffari} & -- & $158~s$ \\
    \rowcolor{lblue}
    \textbf{\fname}& \textbf{$2.9~ms$} & \textbf{$2.9~ms$}  \\	  
    \rowcolor{vlblue}
    {~~\fname det. Green's func. {(Offline)}}& \textbf{$0.55~ms$} & \textbf{$0.55~ms$}  \\	  
    \rowcolor{vlblue}
    {~~\fname rand. Green's func. {(Offline)}}& \textbf{$1.6~ms$} & \textbf{$1.6~ms$}  \\	  
    \rowcolor{vlblue}
        {~~\fname full-chip (Online)}& \textbf{$0.74~ms$} & \textbf{$0.74~ms$}  \\	  

    \bottomrule
  \end{tabular}
  \footnotesize{\\1. To model temperature-dependent conductivity, detailed thermal modeling is done in HotSpot, since the properties of each block are different. \\2. HotSpot, 3D-ICE and Icepak do not consider variability}\\
  \caption{Speed of the studied simulators  \label{tab:exectime}}	
  	\end{table}


\begin{table*}[!htbp]
\centering
  \caption{Errors in various scenarios}	
  \label{tab:error}
  \setlength\tabcolsep{3pt} 
  \begin{tabular}{p{4.1cm}|ccc|ccc}
    \toprule
    \multirow{2}{*}{Effects considered} &
      \multicolumn{3}{c|}{Test Case 1 (Alpha21264)} &
      \multicolumn{3}{c}{Test Case 2 } \\
      & {Max. Temp. (K)} & {Max. Deviation (K)} & {Percent Deviation} & {Max. Temp. (K)} & {Max. Deviation (K)} & {Percent Deviation} \\
      \midrule
	\cmnt{5}No effects&				341.36	& 	\br{6.77} 	&	22.6	&372.90 & 9.01	&14.1\\
	\cmnt{3}Rand.-cond., Cond.(T)  &	341.86	&	\br{6.27}	&	20.9	&377.59&4.32& 6.8\\
	\cmnt{8}Leakage-var&					344.04	&	\dblue{4.09}	&	13.6	&375.29&6.62& 11.6\\
	\cmnt{7}Leakage(T)&			344.30 	& 	\dblue{3.83} 	&	12.8 	&374.56& 7.35&12.2\\
	\cmnt{6}Rand.-cond., Cond.(T), Leakage(T)&	344.91& \dblue{3.22}&10.7& 378.39 &3.52 &5.8\\
	\cmnt{2}Leakage-var, Leakage(T)&	347.43	&	0.70	&	2.33		&377.16&4.75&7.5\\
	\cmnt{4}Cond.(T), Leakage-var, Leakage(T)&	348.12&	0.01&	0.03&381.36 & 0.55&0.86\\
	\cmnt{1}Rand.-cond., Cond.(T), Leakage-var, Leakage(T)&	348.13& -- & -- &381.91&--&--\\
	\hline
	\fname 			&	\textbf{347.58}&	\textbf{0.55} &\textbf{1.8}& \textbf{379.31}& \textbf{2.60}&\textbf{4.1}\\
    \bottomrule
  \end{tabular}	
  \footnotesize{\\Leakage(T) = temperature-dep. leakage, Leakage-var = variability in leakage, Rand.-cond. = random conductivity, cond.(T) = temperature-dep. conductivity}\\
\end{table*}


% Figure environment removed

% Figure environment removed

% Figure environment removed




\subsubsection{Modified Green's functions for the steady-state}
The first step in our proposed method is to compute the modified Green's functions that we have derived in our work. 
To do so, we first need the baseline leakage power map considering variation at ambient temperature. To obtain the variation-aware baseline leakage power maps, we use the popular 
variation modeling tool, \textit{Varius}~\cite{varius}. Using a similar approach, we model the randomness in conductivity values because of random dopant fluctuations. 

Next, we obtain the corresponding baseline Green's functions using a modified version of HotSpot. We apply a unit impulse power source to the center of the chip and obtain the baseline Green's 
function considering the variability in conductivity. 
We then use Equation~\ref{eqn:detsol} to obtain the deterministic part of the modified Green's function accounting for the 
effects of temperature-dependent conductivity and leakage power. Our approach takes $0.55~ms$ to compute the deterministic modified Green's function.
Next, we compute the random part of the Green's function using Equation~\ref{eqn:randsol}. This takes a further $1.86~ms$. This part needs to be computed once for a chip.


\subsubsection{Full-chip steady-state thermal simulations}
At runtime, the dynamic power profile is mirrored 
and the full-chip thermal profile is computed using the calculated Green's functions using Equation~\ref{eqn:detrandsol}. This step takes an 
additional $0.74~ms$. Thus the total time taken by our algorithm is $2.89~ms$ (online time = $0.74~ms$ + deterministic Green's function computation = $0.55~ms$ + random Green's function computation = $1.6~ms$). The mean absolute error is limited to 
2\% (as demonstrated by multiple test cases, Table~\ref{tab:steadyres}).


To validate our proposal, we adopt the following approach: the leakage power obtained from Varius is added to the dynamic power profile, and HotSpot is invoked 
iteratively. After each iteration, we update the leakage power and conductivity 
values based on the current temperature. 
We keep iterating until the temperature values converge. HotSpot supports modeling of variable conductivity only when detailed 3D modeling is enabled, since different conductivity values for different blocks result in a change of the parameters of the differential equation from block to block. As a result, HotSpot requires $18~min$  to compute the final temperature. If we do not model variable conductivity, the simulation completes within 4s. Thus, our method provides a $370000\times$ speedup over HotSpot in steady-state thermal simulation.

\noindent \textbf{Test Case 1 [Real floorplan]:} We validate our approach using the floorplan of the 
Alpha21264 processor. The 
power values are taken from the \textit{ev6} test case of HotSpot. The leakage and dynamic power profile and 
the corresponding temperature profiles are shown in Figure~\ref{fig:results}.
We can see in Figure~\ref{fig:calctemp} that the calculated thermal profile matches the actual thermal 
profile very well (mean absolute error = $0.36$\celsius, i.e., within 2\%).
%\noindent \textbf{Test Case 2 [Real floorplan]:}
%\red{Next we validate our method on the floorplan of the processor Gainestown.
%The total power in this case is $57.3~W$, while the maximum temperature rise is 17.7\celsius. The mean absolute error compared to HotSpot is 0.53\celsius. }

\noindent \textbf{Test Case 2 [Stress testing]:} In this case, multiple dynamic power sources are applied to 
different 
locations on the chip. The total dynamic power is $8~W$. Although the total power applied is 
lower than test case 1, the power density of the sources is much higher, resulting in a higher maximum temperature. In this case, too, the temperature obtained 
using our algorithm 
matches the actual value very well, with a mean absolute error limited to 0.7\celsius (1.1\%).

\noindent \textbf{Test Case 3 [Variance testing]:} In this case, we apply the same dynamic power as test case 1, but the baseline leakage power has a higher variance. The calculated thermal profile is shown in Figure~\ref{fig:calctc3}, while the corresponding thermal profile obtained from HotSpot is shown in Figure~\ref{fig:acttc3}. This test case is a limit study. 
We see that here, too, the calculated and the actual thermal profiles match closely. The mean absolute error is 0.61\celsius for a maximum temperature rise of 32.3\celsius (1.9\%), while the maximum temperature is 77.3\celsius. 
The maximum error at the hotspot location is $2.8$\celsius. 
In Figure~\ref{fig:igntc3}, we show the temperature profile if the random component of the solution is ignored (Equation~\ref{eqn:detrandsol}). We see that the errors are larger in this case, and the main hotspot location is missed. The mean absolute error upon ignoring the random effects is 1.2\celsius (3.7\%), while the maximum error is 5.9\celsius (18.3\%). Thus, we are able to lower the error in modeling process variation-aware thermal profile by up to 52\% using our proposed approach. Moreover, our proposed method helps in capturing the location of the hotspot much more accurately, which would otherwise get missed.

\noindent \textbf{Test Case 4 [Variance + Stress testing]:} In this case, we have a uniform power applied to the entire chip. The total power dissipated is $204.8~W$. In this case, the mean absolute error using our proposed approach is $1.5$\celsius for a maximum temperature rise of $77$\celsius (1.9\%). The mean absolute error upon ignoring the shift-varying Green's function goes up to $3$\celsius (3.9\%).

\noindent \textbf{Test Case 5 [Variance + Stress testing]:} In this case, the alternate grid points have power sources applied to them. The total power dissipated is $51.2~W$. Our proposed method results in a mean absolute error of $0.6$\celsius for a maximum temperature rise of $29.2$\celsius (2\%). 

The steady state test cases are summarized in Table~\ref{tab:steadyres}.
\begin{table}[!htbp]
  \begin{tabular}{l|c|c|c}
    \toprule
    {Test Case} & {Total power ($W$)} & Max. temp. (\celsius) & Mean abs. error (\celsius) \\
      \midrule
    Test case 1 & 48.9 & 74.9 & 0.36\celsius (1.2\%)   \\
    {Test case 2}& 8.0 & 108.8 & 0.70\celsius (1.1\%)   \\	  
    Test case 3 & 48.9 & 77.3 & 0.61\celsius (1.9\%)   \\
    Test case 4 & 204.8 & 122.1 & 1.45\celsius (1.9\%)   \\
    Test case 5 & 51.2 & 74.2 & 0.59\celsius (2\%)  \\    
    \bottomrule
  \end{tabular}
  \caption{Steady-state test cases summary \label{tab:steadyres}}	
  	\end{table}
  	
  	
\subsection{Transient Results}


\subsubsection{Modified transient Green's functions}
We obtain the modified transient Green's function using the leakage-aware steady-state Green's function as the starting point. 
Figure~\ref{fig:Greentrancalc} shows the temporal evolution of the calculated transient Green's function using Equation~\ref{eqn:tranGreen} at the center of the chip. The estimation error is less than 3\% at all times. We compute the modified Green's function at 100 time instants between 0 and $10~ms$.
Our algorithm takes $0.12~s$ to compute the temperature profiles for the 100 time steps (Table~\ref{tab:exectimetran}).

\begin{table}[!tbp]
  \begin{tabular}{l|c}
    \toprule
    {Simulator} &
      {Time } \\
      \midrule
    Hotspot & $18-20~minutes$    \\
    \rowcolor{lblue}
    \textbf{\fname}& \textbf{$0.29~s$ (150 time steps)}   \\	  
    \rowcolor{vlblue}
    {~~\fname modified Green's func.}& \textbf{$120~ms$} (100 time steps)  \\	  
    \rowcolor{vlblue}
    {~~\fname full chip step response}& \textbf{$70~ms$} (100 time steps)   \\	  
    \rowcolor{vlblue}
        {~~\fname full chip time varying temp.}& \textbf{$290~ms$} (150 time steps) \\	  

    \bottomrule
  \end{tabular}
  \footnotesize{\\1. To model temperature-dependent conductivity, detailed thermal modeling is done in HotSpot, since the properties of each block are different. \\2. HotSpot, 3D-ICE and Icepak do not consider variability}\\
  \caption{Speed of existing simulators  \label{tab:exectimetran}}	
  	\end{table}

% Figure environment removed


\subsubsection{Full chip transient thermal profile}
Next, we use the derived transient Green's function to obtain the transient thermal profile for the floorplan of Alpha21264, corresponding to the power profile in Figure~\ref{fig:dynpower}.
The error at all times was observed to be less than 5\% with a simulation time of $70~ms$ for 100 time steps. 
The corresponding computed transient thermal profiles at $0.5~ms$, $1~ms$, $2~ms$, and $5~ms$ are shown in Figure~\ref{fig:Ttranalphacalc}. The corresponding actual thermal profiles obtained from a modified version of Hotspot are shown in Figure~\ref{fig:Ttranalphaact}.


% Figure environment removed

% Figure environment removed

\subsubsection{Full chip time-varying transient thermal profile}
\noindent \textbf{Test Case 1 [Real floorplan]:} 
We randomly vary the power profile in Figure~\ref{fig:dynpower} every $1~ms$ and obtain the thermal profile for this time-varying power input using HotSpot as well as our proposed method (Equation~\ref{eqn:tranapprox}). We simulate the temperature until $15~ms$ at intervals of $0.1~ms$. Using our algorithm, we needed $290~ms$ to compute the thermal profile for 150 time steps ($\approx 2~ms$ per time step), while HotSpot took nearly 20 minutes ($4000\times$ speedup). The average error in our case was 3.8\%.
The evolution of the dynamic power profile at the hottest location is shown in Figure~\ref{fig:powtimevar}, while the corresponding thermal profile is shown in Figure~\ref{fig:trantimevar}.

% Figure environment removed
% Figure environment removed
\noindent \textbf{Test Case 2 [Stress testing]:}		
Next, we randomly vary the power profile in test case 2 of the steady-state every $1~ms$ and observe the evolution of temperature over 30 time steps. The power profile at one of the locations and the corresponding thermal profiles are given in Figure~\ref{fig:trantimevarfigtc2}. Our algorithm takes $59~ms$ to compute the thermal profile for 30 time steps at intervals of $1~ms$ each ($2~ms$ per time step). In contrast, HotSpot takes over 15 minutes to compute the same thermal profile ($15000\times$ speedup). 
These results are summarized in Table~\ref{tab:exectimetran}.

\subsection{Analysis of the Results}
We have established through a wide range of test cases that our proposed method provides fast as well as accurate solutions for the effects considered. Next, we analyze the importance of modeling each of the individual effects --  temperature-dependent leakage, variation in leakage, temperature-dependent conductivity, and variation in conductivity profiles. 

To do so, we sequentially consider a subset of these effects while ignoring the rest of the effects. This helps us determine the individual thermal contribution of each of these effects in the cases we have modeled.

Table~\ref{tab:error} summarizes the results obtained in various scenarios. Figure~\ref{fig:error} graphically represents this error. We see that not accounting for any kind of variability leads to a temperature estimation error of up to 22\%. 
If we ignore the temperature dependence of conductivity but consider variability and temperature dependence 
of leakage, the error varies from 2 to 7.5\%. 
Ignoring variations in the conductivity profile leads to $<1\%$  error in thermal estimation. Thus in thermal 
modeling, the effects of random variations in the conductivity profile can be safely ignored. The need to model temperature-dependent conductivity would depend on the accuracy requirement of the design. However, considering 
variability in leakage power along with the temperature dependence of leakage power is 
\emph{absolutely essential} to achieve a meaningful simulation accuracy.

\subsubsection*{Comparison with state-of-the-art approaches}
Since there is no state-of-the-art work in thermal modeling that considers variability as well as temperature-dependent leakage, 
we compare our results against the modified version of HotSpot. Our algorithm provides a $4000-370000\times$ speedup 
over HotSpot, while maintaining the error within 4\% in all cases. Table~\ref{tab:exectime} summarizes the simulation 
speed of various tools for the steady-state.

\subsubsection*{Memory and Energy Analysis}
Since we use compact thermal models in our work, our method consumes a lot less memory compared to the existing simulators. For the steady-state, our method uses 4MB memory only, compared to 360MB for HotSpot. Similarly, in the transient case, we use approximately 85MB, compared to the 900 MB used by HotSpot. The high memory consumption in HotSpot is primarily because finite difference methods involve matrices with a very large number of nodes, whereas
Green's function methods need just as many nodes as the required temperature resolution.

Owing to its analytical nature and ultra-high speed, we outperform state-of-the-art methods in terms of energy consumption as well.

