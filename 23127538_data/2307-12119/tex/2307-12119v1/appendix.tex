\section{Hankel Transform and its Properties}
The Hankel transform is a mathematical transform that decomposes any function $f(r)$ into an infinite number of Bessel functions of the first kind. It is defined as:
\begin{equation}
\label{eqn:hankeldefapp}
\hankel(f(r)) = H(s) = \int_0^\infty f(r) \bessel_0(sr)r dr,
\end{equation}
where $\bessel_0$ is the Bessel function of the first kind of order 0, and $\hankel$ denotes the Hankel transform operator.

The inverse Hankel transform of $H(s)$ is defined as: 
\begin{equation}
\label{eqn:invhankeldef}
\hankel^{-1}(H(s)) = f(r) = \int_0^\infty H(s) \bessel_0(sr)s ds,
\end{equation}
where $\hankel^{-1}$ denotes the inverse Hankel transform operator.



\subsection*{Theorem I: The Hankel transform of a radially symmetric function in polar coordinates is equivalent to its 2D Fourier transform.}
\subsection*{Proof:}

The 2D Fourier transform is given by:
\begin{equation}
\fourier(f(r)) = F(u,v) = \int_{-\infty}^\infty\int_{-\infty}^\infty f(x,y) e^{-j(ux+vy)}dx dy,
\end{equation}
Let $x = rcos\theta$ and $y  = rsin\theta$.
Let $u = \rho cos\phi$ and $v = \rho sin\phi$

Substituting these in the above equation:
\begin{equation}
F(\rho, \phi) = \int_{0}^\infty\int_{-\pi}^\pi f(r,\theta)e^{-ir\rho cos(\phi - \theta)}rdrd\theta
\end{equation}


If $f(r)$ is radially symmetric, it is independent of the angle. Thus we can rewrite the above equation as:
\begin{equation}
F(\rho, \phi) = \int_{0}^\infty rf(r)dr\int_{-\pi}^\pi e^{-ir\rho cos(\phi - \theta)}d\theta
\end{equation}

Using the  definition of the zeroth-order Bessel function:
\begin{equation}
J_0(x) = \frac{1}{2\pi}\int_{-\pi}^\pi e^{-ix cos(\phi - \theta)}
\end{equation}

Using this in the equation above, we get:
\begin{equation}
F(\rho) = \fourier (f(r)) = 2\pi\int_{0}^\infty f(r)J_0(\rho r)rdr
\end{equation}
which is the same is $2\pi$ times the Hankel transform of order 0.
Thus:
\begin{equation}
F(\rho) = \fourier (f(r)) = 2\pi\hankel (f(r))
\end{equation}


%\subsection*{Lemma I: }
\subsubsection*{Lemma I}: $G(u,v) =  \fourier(f_{sp_0}\T) = \fourier (\T) g_{sp_0}$, where $g_{sp_0} = \left( f_{sp_0} - \kappa + f_{sp_0}(0,0)\right)$
\subsection*{Proof:}

We use the zero order Hankel transform on $G(u,v)$ to reduce the 2D Fourier transform to a 1D Hankel transform. We denote the variables in polar coordinates by the $~\tilde{}~$ operator. Next, we apply integration by parts:
\begin{equation}
\begin{split}
\label{eqn:hankel1}
H(h) &= \hankel(\tilde{f}_{sp_0}\tilde{\T})
= \int_0^\infty(\tilde{f}_{sp_0}\tilde{\T})J_0(hr)r dr\\
&= \tilde{f}_{sp_0} {\int_0^\infty}\tilde{\T} J_0(hr)r dr {-} {\int_0^\infty} \tilde{f}'_{sp_0}dr  {\int_0^\infty} \tilde{\T} J_0(hr)r dr \\
&= \tilde{f}_{sp_0} \hankel(\tilde{\T}) - \int_0^\infty \tilde{f}'_{sp_0}dr \times \hankel(\tilde{\T}) \\
&=  \hankel(\tilde{\T})\left(\tilde{f}_{sp_0} - \int_0^\infty \frac{\partial{\tilde{f}_{sp_0}}}{\partial{r}} dr \right)\\
&=  \hankel(\tilde{\T})\left(\tilde{f}_{sp_0} - \tilde{f}_{sp_0}(\infty) + \tilde{f}_{sp_0}(0) \right)
\end{split}
\end{equation}
where $'$ denotes the derivative with respect to $r$.

Let $\tilde{f}_{sp_0}(\infty) = \kappa$.
The equivalent expression in the Cartesian coordinates becomes:
\begin{equation}
\begin{split}
\label{eqn:leakvar2}
G(u,v) 
%&=  
%f_{sp_0} \fourier(\T) -\int_{{-}\infty}^\infty\!\!\int_{{-}\infty}^%\infty \!\!- f_{sp_0}(\infty,\infty) + f_{sp_0}(0,0) \fourier(\T)  \\
	   &= \fourier(f_{sp_0}\T) = \fourier (\T)\underbrace{\left( f_{sp_0} - \kappa + f_{sp_0}(0,0)\right)}_{= g_{sp_0}} \\
	   &=\fourier (\T) g_{sp_0}
\end{split}
\end{equation}
{where $\fourier$ is the Fourier transform operator.}
