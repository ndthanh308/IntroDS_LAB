The unprecedented accuracy of convolutional neural networks (CNNs) across a broad range of AI tasks has led to their widespread deployment in mobile and embedded settings. In a pursuit for high-performance and energy-efficient inference, significant research effort has been invested in the design of FPGA-based CNN accelerators. In this context, single computation engines constitute a popular design approach that enables the deployment of diverse models without the overhead of fabric reconfiguration. Nevertheless, this flexibility often comes with significantly degraded performance on memory-bound layers and resource underutilisation due to the suboptimal mapping of certain layers on the engine's fixed configuration. 
In this work, we investigate the implications in terms of CNN engine design for a class of models that introduce a pre-convolution stage to decompress the weights at run time. We refer to these approaches as \textit{on-the-fly}. 
This paper presents \tool, a novel CNN inference system that counteracts the limitations of existing CNN engines. The proposed framework comprises a novel CNN hardware architecture that introduces a weights generator module that enables the on-chip on-the-fly generation of weights, alleviating the negative impact of limited bandwidth on memory-bound layers. We further enhance \tool with an automated hardware-aware methodology that tailors the weights generation mechanism to the target CNN-device pair, leading to an improved accuracy-performance balance.
Finally, we introduce an input selective processing element (PE) design that balances the load between PEs in suboptimally mapped layers. Quantitative evaluation shows that the proposed framework yields hardware designs that achieve an average of 2.57$\times$ performance efficiency gain over highly optimised GPU designs for the same power constraints and up to 3.94$\times$ higher performance density over a diverse range of state-of-the-art FPGA-based CNN accelerators.
