%%
%% This is file `sample-acmsmall.tex',
%% generated with the docstrip utility.
%%
%% The original source files were:
%%
%% samples.dtx  (with options: `acmsmall')
%% 
%% IMPORTANT NOTICE:
%% 
%% For the copyright see the source file.
%% 
%% Any modified versions of this file must be renamed
%% with new filenames distinct from sample-acmsmall.tex.
%% 
%% For distribution of the original source see the terms
%% for copying and modification in the file samples.dtx.
%% 
%% This generated file may be distributed as long as the
%% original source files, as listed above, are part of the
%% same distribution. (The sources need not necessarily be
%% in the same archive or directory.)
%%
%% Commands for TeXCount
%TC:macro \cite [option:text,text]
%TC:macro \citep [option:text,text]
%TC:macro \citet [option:text,text]
%TC:envir table 0 1
%TC:envir table* 0 1
%TC:envir tabular [ignore] word
%TC:envir displaymath 0 word
%TC:envir math 0 word
%TC:envir comment 0 0
%%
%%
%% The first command in your LaTeX source must be the \documentclass command.
\documentclass[acmsmall]{acmart}
%% NOTE that a single column version is required for 
%% submission and peer review. This can be done by changing
%% the \doucmentclass[...]{acmart} in this template to 
%% \documentclass[manuscript,screen]{acmart}
%% 
%% To ensure 100% compatibility, please check the white list of
%% approved LaTeX packages to be used with the Master Article Template at
%% https://www.acm.org/publications/taps/whitelist-of-latex-packages 
%% before creating your document. The white list page provides 
%% information on how to submit additional LaTeX packages for 
%% review and adoption.
%% Fonts used in the template cannot be substituted; margin 
%% adjustments are not allowed.
%%
%% \BibTeX command to typeset BibTeX logo in the docs
\AtBeginDocument{%
  \providecommand\BibTeX{{%
    \normalfont B\kern-0.5em{\scshape i\kern-0.25em b}\kern-0.8em\TeX}}}

%% Rights management information.  This information is sent to you
%% when you complete the rights form.  These commands have SAMPLE
%% values in them; it is your responsibility as an author to replace
%% the commands and values with those provided to you when you
%% complete the rights form.
\setcopyright{acmcopyright}
\copyrightyear{2022}
\acmYear{2022}
\acmDOI{XXXXXXX.XXXXXXX}


%%
%% These commands are for a JOURNAL article.
\acmJournal{TODAES}
\acmVolume{37}
\acmNumber{4}
\acmArticle{111}
\acmMonth{8}

%% extra imports
\usepackage{soul}
\usepackage{xcolor}
\usepackage{xspace}
\usepackage{booktabs}
\usepackage[flushleft]{threeparttable} % for notes below table
\usepackage{bm}
\usepackage{wrapfig}
\usepackage{multirow}
\usepackage{multicol}
\usepackage{amsmath}
\usepackage{amsfonts}
\usepackage{enumitem}
\usepackage{color}
\usepackage{xcolor}
\usepackage{soul}
\usepackage{extarrows}
\usepackage{gensymb}
\usepackage{natbib}
\usepackage{microtype}
\usepackage{graphicx}
\usepackage{subcaption}
\usepackage{stfloats} % position two-col figs
\usepackage[bottom]{footmisc}
\newcommand*\circled[1]{\tikz[baseline=(char.base)]{
            \node[shape=circle,fill,inner sep=1.2pt] (char) {\textcolor{white}{#1}};}}
\newcommand{\tablefontsize}[0]{
\fontsize{8pt}{10pt}\selectfont
}
\usepackage[ruled,linesnumbered]{algorithm2e}
% for multiple inputs in algorithms
\newlength\mylenin
\newcommand\myinput[1]{%
\settowidth\mylenin{\KwIn{}}%
\setlength\hangindent{\mylenin}%
\hspace*{\mylenin}#1\\}

\let\oldnl\nl% Store \nl in \oldnl
\newcommand{\nonl}{\renewcommand{\nl}{\let\nl\oldnl}}% Remove line number for one line

\newlength\mylenout
\newcommand\myoutput[1]{%
\settowidth\mylenout{\KwOut{}}%
\setlength\hangindent{\mylenout}%
\hspace*{\mylenout}#1\\}
\usepackage{algpseudocode}
\DeclareMathOperator*{\argmin}{argmin}
\usepackage{tikz}

%%
%% Submission ID.
%% Use this when submitting an article to a sponsored event. You'll
%% receive a unique submission ID from the organizers
%% of the event, and this ID should be used as the parameter to this command.
%%\acmSubmissionID{123-A56-BU3}

%%
%% For managing citations, it is recommended to use bibliography
%% files in BibTeX format.
%%
%% You can then either use BibTeX with the ACM-Reference-Format style,
%% or BibLaTeX with the acmnumeric or acmauthoryear sytles, that include
%% support for advanced citation of software artefact from the
%% biblatex-software package, also separately available on CTAN.
%%
%% Look at the sample-*-biblatex.tex files for templates showcasing
%% the biblatex styles.
%%

%%
%% The majority of ACM publications use numbered citations and
%% references.  The command \citestyle{authoryear} switches to the
%% "author year" style.
%%
%% If you are preparing content for an event
%% sponsored by ACM SIGGRAPH, you must use the "author year" style of
%% citations and references.
%% Uncommenting
%% the next command will enable that style.
%%\citestyle{acmauthoryear}

\newif\ifcomment
\commenttrue
\commentfalse 
            
\ifcomment
\definecolor{stelios_colour}{RGB}{191, 232, 255}
\newcommand{\stelios}[1]{\sethlcolor{stelios_colour}\hl{[Stelios: #1]}}

\definecolor{javier_colour}{RGB}{255, 204, 204}
\newcommand{\javier}[1]{\sethlcolor{javier_colour}\hl{[Javier: #1]}}

\newcommand{\blue}[1]{\textcolor{blue}{#1}}
\newcommand{\red}[1]{\textcolor{red}{#1}}

\else
\newcommand{\stelios}[1]{}
\newcommand{\javier}[1]{}
\newcommand{\blue}[1]{\textcolor{black}{#1}}
\newcommand{\red}[1]{\textcolor{black}{#1}}
\fi

\newcommand{\rev}[1]{\textcolor{black}{#1}}

\newcommand{\tool}{unzipFPGA\xspace}

\hyphenation{unzipFPGA}

%%
%% end of the preamble, start of the body of the document source.
\begin{document}

%%
%% The "title" command has an optional parameter,
%% allowing the author to define a "short title" to be used in page headers.
\title{Mitigating Memory Wall Effects in CNN Engines with On-the-Fly Weights Generation}

%%
%% The "author" command and its associated commands are used to define
%% the authors and their affiliations.
%% Of note is the shared affiliation of the first two authors, and the
%% "authornote" and "authornotemark" commands
%% used to denote shared contribution to the research.
% \author{Ben Trovato}
% \authornote{Both authors contributed equally to this research.}
% \email{trovato@corporation.com}
% \orcid{1234-5678-9012}
% \author{G.K.M. Tobin}
% \authornotemark[1]
% \email{webmaster@marysville-ohio.com}
% \affiliation{%
%   \institution{Institute for Clarity in Documentation}
%   \streetaddress{P.O. Box 1212}
%   \city{Dublin}
%   \state{Ohio}
%   \country{USA}
%   \postcode{43017-6221}
% }

\author{Stylianos I. Venieris}
\authornote{Both authors contributed equally to this research.}
\email{s.venieris@samsung.com}
\orcid{1234-5678-9012}
\affiliation{%
  \institution{Samsung AI Center, Cambridge}
  \country{UK}
}
\author{Javier Fernandez-Marques}
\authornotemark[1]
\email{j1.fernandez@samsung.com}
\orcid{1234-5678-9012}
\affiliation{%
  \institution{Samsung AI Center, Cambridge}
  \country{UK}
}
\author{Nicholas D. Lane}
\orcid{1234-5678-9012}
\affiliation{%
  \institution{Samsung AI Center, Cambridge \& University of Cambridge}
  \country{UK}
}
\email{nic.lane@samsung.com}

%%
%% By default, the full list of authors will be used in the page
%% headers. Often, this list is too long, and will overlap
%% other information printed in the page headers. This command allows
%% the author to define a more concise list
%% of authors' names for this purpose.
\renewcommand{\shortauthors}{Venieris and Fernandez-Marques, et al.}

%%
%% The abstract is a short summary of the work to be presented in the
%% article.
% \begin{abstract}
%   A clear and well-documented \LaTeX\ document is presented as an
%   article formatted for publication by ACM in a conference proceedings
%   or journal publication. Based on the ``acmart'' document class, this
%   article presents and explains many of the common variations, as well
%   as many of the formatting elements an author may use in the
%   preparation of the documentation of their work.
% \end{abstract}

%%
%% The code below is generated by the tool at http://dl.acm.org/ccs.cfm.
%% Please copy and paste the code instead of the example below.
%%
\begin{CCSXML}
<ccs2012>
   <concept>
       <concept_id>10010520.10010521.10010542.10010543</concept_id>
       <concept_desc>Computer systems organization~Reconfigurable computing</concept_desc>
       <concept_significance>500</concept_significance>
       </concept>
   <concept>
       <concept_id>10010147.10010257.10010293.10010294</concept_id>
       <concept_desc>Computing methodologies~Neural networks</concept_desc>
       <concept_significance>500</concept_significance>
       </concept>
 </ccs2012>
\end{CCSXML}

\ccsdesc[500]{Computer systems organization~Reconfigurable computing}
\ccsdesc[500]{Computing methodologies~Neural networks}

%%
%% Keywords. The author(s) should pick words that accurately describe
%% the work being presented. Separate the keywords with commas.
\keywords{neural networks, hardware accelerator, weights generation}

% \received{20 February 2007}
% \received[revised]{12 March 2009}
% \received[accepted]{5 June 2009}

\begin{abstract}
    \begin{abstract}
Graph Neural Networks (GNNs) have proven to be effective in processing and learning from graph-structured data.
However, previous works mainly focused on understanding single graph inputs while many real-world applications require pair-wise analysis for graph-structured data (e.g., scene graph matching, code searching, and drug-drug interaction prediction).
To this end, recent works have shifted their focus to learning the interaction between pairs of graphs.
Despite their improved performance, these works were still limited in that the interactions were considered at the node-level, resulting in high computational costs and suboptimal performance.
To address this issue, we propose a novel and efficient graph-level approach for extracting interaction representations using co-attention in graph pooling. 
Our method, Co-Attention Graph Pooling (CAGPool), exhibits competitive performance relative to existing methods in both classification and regression tasks using real-world datasets, while maintaining lower computational complexity.

\end{abstract}
\end{abstract}

%%
%% This command processes the author and affiliation and title
%% information and builds the first part of the formatted document.
\maketitle

\section{Introduction}

% Figure environment removed

Reinforcement Learning from Human Feedback (RLHF) has recently been used to great effect to align pretrained large language models (LLMs) to human preferences, optimizing for desirable qualities like harmlessness and helpfulness~\citep{bai2022training} and achieving state-of-the-art results across a variety of natural language tasks~\citep{openai2023gpt4}. %RLHF approaches fundamentally rely on collecting pairs of LLM outputs $(o_1, o_2)$ from a shared prompt $p$, with a human indicating which output in each pair is better on a specified attribute.
% A fundamental component of RLHF is a preference model derived from human labels, typically formatted as pairs of LLM outputs $(o_1, o_2)$ generated from a shared prompt $p$.

A standard RLHF procedure fine-tunes an initial unaligned LLM using an RL algorithm such as PPO~\citep{schulman2017proximal}, optimizing the LLM to align with human preferences. %\violet{not sure whether we need to provide this detail in the intro, especially this has nothing to do with our contribution.} % i feel like this context is useful later when e.g. explaining that context distillation is SFT
RLHF is thus critically dependent on a reward model derived from human-labeled preferences, typically \textit{pairwise preferences} on LLM outputs $(o_1, o_2)$ generated from a shared prompt $p$. % and labeled by humans. 

However, collecting human pairwise preference data, especially high-quality data, may be expensive and time consuming at scale. To address this problem, approaches have been proposed to obtain labels without human annotation, such as Reinforcement Learning from AI Feedback (RLAIF) and context distillation. 

\iffalse
raising the question of whether we can generate high-quality data for RLHF without using human labeling. %accurately-labeled preference pairs $(o_1, o_2)$
%, motivating model alignment approaches that aim to generate accurately-labeled preference pairs $(o_1, o_2)$ without human involvement. 
Two major categories of such approaches are . 
\fi

RLAIF approaches (e.g.,~\citet{bai2022constitutional}) simulate human pairwise preferences by scoring $o_1$ and $o_2$ with an LLM (Figure \ref{fig:rlcd_differences} center); the scoring LLM is often the same as the one used to generate the original pairs $(o_1, o_2)$. Of course, the resulting LLM pairwise preferences will be somewhat noisier compared to human labels. However, this problem is exacerbated by using the same prompt $p$ to generate both $o_1$ and $o_2$, causing $o_1$ and $o_2$ to often be of very similar quality and thus hard to differentiate (e.g., Table~\ref{tab:rlaif_bad_example}). Consequently, training signal can be overwhelmed by label noise, yielding lower-quality preference data. 

% While it avoids human labeling efforts, it has weakness. First, LLM preference labels will naturally be somewhat noisier compared to human labels. Furthermore, since the same prompt $p$ is used to generate both $o_1$ and $o_2$, their quality is often very similar and hard to differentiate (See Table~\ref{tab:rlaif_bad_example}). As a result, training signals can be overwhelmed by label noise, yielding lower-quality preference data. 

Meanwhile, context distillation methods (e.g., \citet{sun2023principle}) create more training signal by modifying the initial prompt $p$. 
%to create more significant training signal. 
The modified prompt $p_+$ typically contains additional context encouraging a \textit{directional attribute change} in the output $o_+$ (Figure \ref{fig:rlcd_differences} right). However, context distillation methods only generate a single output $o_+$ per prompt $p_+$, which is then used for supervised fine-tuning, losing the pairwise preferences which help RLHF-style approaches to 
%rather than using a RLHF-style preference model to 
derive signal from the contrast between outputs. 
Multiple works have observed that RL approaches using preference models for pairwise preferences can substantially improve over supervised fine-tuning by itself when aligning LLMs~\citep{ouyang2022training,dubois2023alpacafarm}. 

% conduct alignment by running supervised fine-tuning on model outputs $o_+$ generated from a modified prompt $p_+$. $p_+$ typically contains additional context encouraging desirable attributes (Figure \ref{fig:rlcd_differences} right), such as in \citet{sun2023principle}. However, multiple works have observed that RLHF-style approaches can substantially improve over supervised fine-tuning by itself when aligning LLMs~\citep{ouyang2022training,dubois2023alpacafarm}. 

Therefore, while both RLAIF and context distillation approaches have already been successfully applied in practice to align language models, we posit that it may be even more effective to combine the key advantages of both. That is, we will use RL with \textit{pairwise preferences}, while also using modified prompts to encourage \textit{directional attribute change} in outputs. %In particular, we will adapt the RLAIF data generation process with two different prompts rather than a single $p$, modifying both prompts similarly to context distillation. %\violet{this motivation is a little unexciting. I think we can more specifically discuss the potential benefits of our approach, like the benefits from RL: exploration/data generation; benefits from contrast. I don't think we get too much benefits from context distillation since we switched to the RL framework.} 

Concretely, we propose \oursfull{} (\ours{}). 
\ours{} generates preference data as follows. Rather than producing two i.i.d.\ model outputs $(o_1, o_2)$ from the same prompt $p$ as in RLAIF, \ours{} creates two variations of $p$: a \textit{positive prompt} $p_+$ similar to context distillation which encourages directional change toward a desired attribute, and a \textit{negative prompt} $p_-$ which encourages directional change \textit{against} it (Figure \ref{fig:rlcd_differences} left). We then generate model outputs $(o_+, o_-)$ respectively, and automatically label $o_+$ as preferred---that is, \ours{} automatically ``generates'' pairwise preference labels by construction. %, without further post hoc labeling.\violet{should make it clearer that our approach `generates' labels by construction} 
We then follow the standard RL pipeline of training a preference model followed by PPO. 

Compared to RLAIF-generated preference pairs $(o_1, o_2)$ from the same input prompt $p$, there is typically a clearer difference in the quality of $o_+$ and $o_-$ generated using \ours{}'s directional prompts $p_+$ and $p_-$, which may result in less label noise. %which may result in better training signal for the preference model. 
That is, intuitively, \ours{} exchanges having examples be \textit{closer to the classification boundary} for much more \textit{accurate labels} on average. Compared to standard context distillation methods, on top of leveraging pairwise preferences for RL training, \ours{} can derive signal not only from the positive prompt $p_+$ which improves output quality, but also from the negative prompt $p_-$ which degrades it. %\ours{} is not learning to imitate $o_+$, but to distill the \textit{contrast} between $o_+$ and $o_-$. 
Positive outputs $o_+$ don't need to be perfect; they only need to contrast with $o_-$ on the desired attribute while otherwise following a similar style.

% \todo{discuss our method and why intuitively it may be better.}

We evaluate the practical effectiveness of \ours{} through both human and automatic evaluations on three tasks, aiming to improve the ability of LLaMA-7B~\citep{touvron2023llama} to generate harmless outputs, helpful outputs, and high-quality story outlines. %\ours{} outperforms both RLAIF and context distillation baselines in pairwise comparisons on 
As shown in Sec. \ref{sec:experiments}, \ours{} substantially outperforms both RLAIF and context distillation baselines in pairwise comparisons when simulating preference data with LLaMA-7B, while still performing equal or better when simulating with LLaMA-30B. 
%On all three tasks, \ours{} substantially outperforms both RLAIF and context distillation baselines in pairwise comparisons---by a margin of at least 9\% and often more than 30\%---validating our method's efficacy. 
We will release all code at a later date, although in any case \ours{} is fairly easy to implement by modifying any reference RLAIF codebase. %We release all code at \todo{github link}.
\section{Background}
\subsection{Parallel Strategy}
\label{subsec:background:parallel-strtegy}
\paragraph{Pipeline parallelism~(PP)} In PP, each worker~(machine or GPU) holds a subset of model layers. Adjacent layers on different workers need to transfer activations in the forward propagation~(FP) step and gradients in the backward propagation~(BP) step. 
\paragraph{Data parallelism~(DP)} In DP, each worker holds a replica of the whole model and partitions training samples. In each iteration, each worker computes gradients and synchronizes them with the other workers using all-reduce collective communication~(CC). All workers will have the same model parameters after the synchronization step.
\paragraph{Tensor parallelism~(TP)} In TP, each worker holds a replica of training samples and partitions within model layers. In each iteration, each worker computes its local outputs in FP and its local gradients in BP. To synchronize outputs and gradients, all workers will perform all-reduce CC in FP and BP steps according to the partition scheme.
\paragraph{Fully sharded data parallelism~(FSDP)} FSDP partitions optimizer states, parameters and gradients of the model into separate workers. During the FP and BP step of each iteration, FSDP performs an all-gather CC to obtain the complete parameters for the relevant layer, respectively. After computing the gradients, FSDP conducts a reduce-scatter CC to distribute the global gradients among the workers.

\subsection{Manual Parallelism}
MP refers to the parallel methods in which human experts design and optimize the parallel strategies. Representative MP methods include Megatron-LM~\citep{narayanan_efficient_2021}, Mesh-TensorFlow~\citep{shazeer_mesh-tensorflow_2018}, and GSPMD~\citep{xu_gspmd_2021}. Megatron-LM manually designs TP and PP strategies for training Transformer-based models and exhibits superior efficiency. Mesh-TensorFlow and GSPMD require human effort to designate and tune the intra-layer parallel strategy. These methods rely on expert design and have little flexibility, challenging their automatic application to other models.

\subsection{Automatic Parallelism}
\paragraph{Inter-layer-only AP or intra-layer-only AP} For inter-layer-only AP, GPipe~\citep{huang_gpipe_2019} and vPipe~\citep{zhao_vpipe_2022} employ a balanced partition algorithm and a dynamic layer partitioning middleware to partition pipelines, respectively. For intra-layer-only AP, OptCNN~\citep{jia_exploring_2018}, TensorOpt~\citep{cai_tensoropt_2022}, and Tofu~\citep{wang_supporting_2019} employ dynamic programming methods to optimize DP and TP strategies together. FlexFlow~\citep{jia_beyond_2019} and Automap~\citep{schaarschmidt_automap_2021} use the Monte Carlo method to find the optimal DP and TP strategy. Colossal-Auto~\citep{liu_colossal-auto_2023} utilizes integer programming techniques to generate intra-layer parallelism and activation checkpointing strategies without optimizing inter-layer parallelism. All these methods optimize only one category of parallel strategies.


\paragraph{Inter- and intra-layer AP} PipeDream~\citep{narayanan_pipedream_2019}, DAPPLE~\citep{fan_dapple_2021}, and PipeTransformer~\citep{he_pipetransformer_2021} use dynamic programming to determine optimal strategies for both DP and PP. DNN-partitioning~\citep{tarnawski_efficient_2020} adopts integer and dynamic programming to explore DP and PP strategies. Piper~\citep{tarnawski_piper_2021} and Alpa~\citep{zheng_alpa_2022} adopt a parallel method considering DP, TP, and PP.
Galvatron~\citep{miao_galvatron_2022} uses dynamic programming to determine DP, TP, and FSDP strategies in a single pipeline stage. As for PP, it partitions stages and determines micro-batch size using naive greedy algorithms. All these methods are hierarchical, which will result in sub-optimal solutions.



\begin{wrapfigure}{R}{0.5\textwidth}
    \vspace{-0.4cm}
    \centering
    % \fbox
    {
    % Figure removed
    \vspace{-0.6cm}
    }
    \captionsetup{font=small,labelfont=bf}
    \caption{\footnotesize Overview of \tool's design flow.}
    \label{fig:design_flow}
     \vspace{-0.2cm}
\end{wrapfigure}

\section{\lowercase{unzip}FPGA's Design Flow}
\label{sec:design_flow}

Our framework aims to enhance the performance of hardware CNN engines, while maintaining a high level of abstraction for deep learning developers. Fig.~\ref{fig:design_flow} shows a high-level view of \tool's design flow, comprising two software components: \textit{1)}~the \textit{OVSF Model Converter} and \textit{2)}~the \textit{Optimiser}.

As a starting point, the deep learning expert provides the CNN model, expressed in PyTorch, and the target FPGA platform. The \textit{Converter} processes the supplied CNN architecture and derives an OVSF variant, by transforming the conventional convolutional layers into OVSF convolutional (OVSF-CONV) layers. This step entails \textit{i)}~the replacement of weight filters with a trainable linear combination of OVSF bases, followed by \textit{ii)}~the selection of each layer's compression ratio $\rho$. Next, the OVSF model is passed to the \textit{Trainer}, where the model gets trained using the supplied training set.

The \textit{Optimiser} accepts the trained OVSF CNN and a given FPGA platform and, uses them to populate the CNN \textit{Performance Model} and the \textit{Resource Constraints}, respectively. Importantly, the \textit{Optimiser} navigates the hardware configuration space considering resource allocations between the CNN engine and the weights generator. Upon completion, the design space exploration (\textit{DSE}) stage yields the highest performing configuration of \tool's architecture for the given CNN-device pair and the system is deployed on the FPGA.


\section{CNN Engine Design for On-the-Fly Weights}
\label{sec:arch}

This section starts by reviewing the hardware architecture of a conventional CNN engine, similar to the ones in \cite{shidiannao2915isca,eyeriss2017jssc,cascadecnn2018fpl}. Then, it provides a series of design requirements to enable on-the-fly weights generation for CNNs and presents our techniques for achieving them.


\subsection{Conventional CNN Engine Design}
Fig.~\ref{fig:conventional_cnn_engine} illustrates a typical CNN engine design. The accelerator consists of an array of processing elements (PEs) to perform matrix multiplications and convolutions, one input and one output activations buffer, and a weights buffer. From an operational perspective, the CNN layers are scheduled sequentially, with pipelining applied between I/O communication and computation to hide the off-chip memory transfer latency.

\textbf{Processing Engine:}
To execute layers of various shapes and types, the core processing engine comprises an array of PEs for the execution of block matrix multiply (GEMM). Each PE contains a scalable dot-product circuit with configurable number of multiply-accumulate (MAC) units.
By translating convolutions into matrix multiplication, the engine can process both CONV and FC layers. To this end, a CONV layer with $N_{\text{in}}$ $H$$\times$$W$ input activations, $N_{\text{out}}$ output channels, $K$$\times$$K$ filters, $p$ padding and $S$ stride involves the multiplication between an $R$$\times$$P$ activations matrix and a $P$$\times$$C$ weights matrix to produce an $R$$\times$$C$ output matrix, with $R$$=$$\left\lceil \frac{H + 2p - K}{S}+1 \right\rceil \left\lceil \frac{W + 2p - K}{S} + 1  \right\rceil$, $P$$=$$N_{\text{in}} K^2 $ and $C$$=$$N_{\text{out}}$.

\begin{wrapfigure}{R}{0.5\textwidth}
    % \vspace{-0.45cm} ,
    \centering
    % \fbox
    {
    % Figure removed
    \vspace{-0.6cm}
    }
    \captionsetup{font=small,labelfont=bf}
    \caption{A conventional CNN engine.}
    \label{fig:conventional_cnn_engine}
     \vspace{-0.2cm}
\end{wrapfigure}

\textbf{Design-time Parametrisation:}
The CNN engine can be scaled based on the workload characteristics and the resources of the target FPGA. As such, it is parametrised with respect to the parameter tuple $\left<T_R,T_P,T_C\right>$. Each parameter determines the tile sizes for each matrix dimension $\left<R,P,C\right>$, the number of PEs ($T_C$) and the MAC units within each PE ($T_P$).

\textbf{Operational Flow:}
To produce a {\small $T_R$$\times$$T_C$} output tile, {\small $\left\lceil \frac{P}{T_P}\right\rceil$} tiles from the activations and weights matrices are processed and accumulated sequentially. 
A common mapping strategy (Fig.~\ref{fig:conventional_cnn_engine}) ties $T_P$ to the MACs per PE to exploit the parallelism within each $T_P$-wide dot product, and $T_C$ to PEs to parallelise the dot products at each output column. Overall, the rows of the {\small $T_R$$\times$$T_P$} activations tile are processed in a pipelined manner to maximise throughput. This is equivalent to an output stationary dataflow~\cite{shidiannao2915isca,eyeriss2017jssc,cascadecnn2018fpl}, which minimises the memory accesses for the output activations by caching partial sums on-chip. Nonetheless, \tool is adaptable to other dataflows with minimal modifications.

\textbf{The Data Movement Bottleneck:}
From a data movement perspective, this approach requires the transfer of {\small $\left\lceil \frac{P}{T_P}\right\rceil$} tiles of size {\small $T_R$$\times$$T_P$} for the inputs, {\small $\left\lceil \frac{P}{T_P}\right\rceil$} tiles of size {\small $T_P$$\times$$T_C$} for the weights, and one tile of size {\small $T_R$$\times$$T_C$} for the outputs. To produce all the output tiles, all the data movements are performed {\small $\left\lceil \frac{R}{T_R}\right\rceil \left\lceil \frac{C}{T_C}\right\rceil$} times. 
In spite of the compute-bound CONV layers, the external memory bandwidth often becomes the bottleneck in CNN inference. This is primarily manifested in cases where: \textit{i)}~a resource-rich FPGA device is targeted. In this case, a large and powerful processing engine is instantiated and the speed of feeding it with new data constrains the performance; \textit{ii)}~the CNN layers have a large amount of weights, either due to large kernel sizes or number of filters. This case often occurs in deeper CNN layers, which are typically of significant width. As such, the weights cannot be stored on-chip and multiple memory transactions have to be issued, putting pressure on the available bandwidth; \textit{iii)}~high-dimensional input and output activations have to be transferred, increasing excessively the bandwidth requirements. This typically occurs in earlier layers of a CNN, where the feature maps dimensions are still large. 

As such, there is an emerging need for relieving the data movement burden and address the impact of hitting the memory wall. In this context, \textit{on-the-fly weights generation} can be an enabling factor in extracting higher performance and making more cost-effective use of hardware CNN engines.


% \stelios{TODO: Outline in which cases this becomes the bottleneck to motivate on-the-fly weights. See the following.}
% \stelios{1) Fast processing engine $\rightarrow$ memory becomes the bottleneck, 2) large amount of weights, 3) large amount of inputs/outputs where freeing bandwidth resources can speed up the execution.}
% \stelios{These necessitate the need for removing the burden of transferring weights. We propose on-the-fly weights generation. But on-the-fly weights without carefully designed hardware support defeats the purpose.}


% Nonetheless, \tool is adaptable to other dataflows.


\subsection{Devising a Hardware CNN Weights Generator}

% Figure environment removed

The objective is to minimise the data movement of CNNs by dynamically constructing the model weights using only on-chip resources. %To do so, the underlying hardware engine needs to generate at run time the weights of a given layer in a timely manner. 
Importantly, on-the-fly weights generation needs to take place at run time, in a timely and per-layer fashion, since each CNN layer is a standalone independent schedulable unit. Moreover, the amount of computational and memory resources assigned to the weights generator ought to be balanced with the rest of the CNN engine to maximise the throughput of the accelerator while sustaining high resource utilisation. To that end, bringing forth on-the-fly weights generation requires devising two major components:
% Efficiently mapping on-the-fly weights generation algorithms to hardware requires devising two major components: a data blocking methodology, which enables a fixed-size weights generator to sustain high throughput for layers with various shapes; and a microarchitecture that achieves fast weights generation with low resource usage while being compatible with existing CNN engine designs. 

% \begin{wrapfigure}{R}{0.45\textwidth}
%     \centering
%     \vspace{-0.6cm}
%     {
%     % Figure removed
%     }
%     \vspace{-0.4cm}
%     \captionsetup{font=small,labelfont=bf}
%     \caption{\footnotesize Example of TiWGen. With a tile size of $M$, each tile is generated in $\lceil T_P T_C/M \rceil \cdot \rho K^2$ cycles.}
%     \vspace{-0.55cm}
%     \label{fig:weights_gen_tiling}
% \end{wrapfigure}


\subsubsection{Tiled Weights Generation}
Our novel insight is that, to be able to generate weights for layers of various dimensions, there is a need for a tiling method on top of the weights generation process. We denote our proposed tiling method by TiWGen. As shown in Fig.~\ref{fig:unzipfpga_cnn_engine}, TiWGen divides each $T_P$$\times$$T_C$ weights tile into subtiles of size $M$, with $M$ being uniform across the CNN's layers. Tiling on top of the weights generation method makes the dataflow of diverse layers identical to each other. With this approach, the value of $M$ becomes independent of the CNN architecture and is solely bound by the resources allocated to the weights generator. As such, $M$ exposes a tunable trade-off between weights generation speed and resource consumption.

\SetArgSty{textnormal} % normal text in if-conditions
\setlength{\textfloatsep}{0pt}% Remove \textfloatsep
\begin{algorithm}[!t]	
%	\algsetup{linenosize=\tiny}
	\footnotesize
%	\scriptsize
	\SetAlgoLined
	\LinesNumbered
	\DontPrintSemicolon
	
	% inputs
	\KwIn{Layer's weights matrix shape:  $P \times C = N_{\text{in}}K^2 \times N_{\text{out}}$}
	\nonl
	\myinput{Row and column tile sizes $T_P$ and $T_C$}
	\nonl
	\myinput{$\alpha$ values with $\alpha \in \mathbb{R}^{N_\text{in}N_\text{out}\left\lceil \rho K^2 \right\rceil}$}
	
	% outputs
	\KwOut{Weights matrix $\boldsymbol{W}$}
	
    \For(\Comment{\textit{tiles loop} - \textbf{\#} \textbf{\texttt{PIPELINE}}}){$t\gets1$ \KwTo $\left\lceil \frac{P}{T_P} \right\rceil$$\cdot$$\left\lceil \frac{C}{T_C} \right\rceil$}{
	   \For(\Comment{\textit{subtiles loop} - \textbf{\#} \textbf{\texttt{PIPELINE}}}){$i\gets1$ \KwTo $\left\lceil \frac{T_P T_C}{M} \right\rceil$}{
    	    $\text{subtile}^t_i \leftarrow \mathbf{0}$ \;
    	    \For(\Comment{\textit{basis vectors loop} - \textbf{\#} \textbf{\texttt{PIPELINE}}}){$j\gets1$ \KwTo $\rho K^2$}{
    	       % \textbf{\#} \textbf{\texttt{CNN-WGen}}: \textit{pipeline}\;
    	        \For(\Comment{\textbf{\#} \textbf{\texttt{UNROLL}}}){$k\gets1$ \KwTo $M$}{
    	            $incr_k \leftarrow \text{vec}_j(k) \cdot \alpha_k$ \hspace{0.37in}\Comment{\textit{Multiplier array}} \;
    	            $\text{subtile}^t_i(k) \leftarrow \text{subtile}^t_i(k) + incr_k$ \hfill\Comment{\textit{Adder array}}\;
    	        }
    	    }
    	    $\text{tile}^t \leftarrow \text{UpdateTile}(\text{tile}^t,\text{subtile}^t_i)$\;
    	}
    	$\boldsymbol{W} \leftarrow \text{UpdateMatrix}(\boldsymbol{W}, \text{tile}^t)$\;
	}
	
		
	\caption{\footnotesize 
	Generation of a layer's weights using TiWGen
	}
	\label{alg:tiled_weights_gen}	
\end{algorithm}


Alg.~\ref{alg:tiled_weights_gen} describes the internal workings of TiWGen. Initially, the $P$$\times$$C$ weights matrix is partitioned into {\small $\left\lceil \frac{P}{T_P}\right\rceil$} tiles of size $T_P$$\times$$T_C$, with each tile processed sequentially (line~1). Next, each tile is divided into $\left\lceil \frac{T_P T_C}{M} \right\rceil$ subtiles~(line 2). After all basis vectors of the current subtile have been processed (lines~4-9), the associated part of the output tile is updated (line~10) and the algorithm proceeds to the next subtile. When all subtiles of a tile have been generated, the weights matrix is updated (line~12) and the algorithm continues to the next iteration until all weights tiles have been constructed.  

\rev{\textbf{Applicability to Other Dataflows.}~Although the presented instance of TiWGen focuses on output stationary dataflows, our method can be applied to hardware designs that employ other dataflows. The main modifications comprise \textit{i)}~the order the generated weights and \textit{ii)}~the required generation rate. For instance, considering Google’s TPU~\cite{tpu2017isca} which is the most widely used systolic array for CNN inference, the accelerator adopts a weight-stationary dataflow. In this case, as the tile of the weight matrix is reused for several cycles, the OVSF generator would have to generate weights in longer periods compared to output stationary dataflow and the resource allocation would be automatically adjusted at the DSE stage accordingly.}


\subsubsection{Weights Generator Microarchitecture}
With the design objectives and constraints of Section~\ref{sec:design_flow} in mind, we propose a microarchitectural unit, called \texttt{CNN-WGen}, which is placed within the CNN engine (Fig.~\ref{fig:unzipfpga_cnn_engine}) and is responsible for generating the weights in an orderly manner and feeding them to the processing engine. Fig.~\ref{fig:hw_weights_gen} illustrates the design. As shown, the unit consists of: \textit{i)}~a \textit{vector compute datapath} comprising two vector units (multiplier and adder arrays), \textit{ii)}~the \textit{Alpha buffer} storing the $\alpha$ values, and \textit{iii)}~the \textit{OVSF generator} that is responsible for outputting the $M$-sized basis vector subtiles as dictated by the TiWGen scheme.

\textbf{Mapping Strategy.}
To efficiently map and perform the TiWGen loops, \texttt{CNN-WGen} employs loop optimisation techniques, annotated in Alg.~\ref{alg:tiled_weights_gen}. Namely, loop pipelining and unrolling are employed to customise the computation patterns and on-chip memory reuse of the weights generator. Pipelining is applied on the three outer loops over tiles (line~1), subtiles (line~2) and basis vectors (line~4), and unrolling on the inner loop that processes the $M$-sized subtile (line~5). To unroll the innermost loop, \texttt{CNN-WGen} employs two $M$-wide vector units that perform $M$-parallel multiplications and additions, respectively. In this manner, tuning $M$ can balance the parallelism-resource usage trade-off of the module.

% Figure environment removed


\textbf{Parametrised Vector Compute Datapath.}
As shown in Fig.~\ref{fig:hw_weights_gen}, the vector arithmetic units must have a fixed size that complies with the resource constraints of the target FPGA and namely the available DSP blocks. The multiplier array is connected to the \textit{OVSF generator} and the \textit{Alpha buffer}. For the i-th subtile (line~3 in Alg.~\ref{alg:tiled_weights_gen}), the former produces $\rho K^2$ basis vectors of size $M$, while the latter outputs the associated $\alpha$ coefficients, both of which are forwarded to the multiplier array in a pipelined manner.
All $M$ elements are processed in parallel by the $M$-wide vector units, leading to the vectorised unrolling of the inner loop on line~5 of Alg.~\ref{alg:tiled_weights_gen}. The adder array processes the output of the multiplier array by accumulating the $\rho K^2$ partial results. Finally, when TiWGen proceeds to the next subtile (\textit{i.e.}~next iteration of the loop on line~2), the control unit (CU in Fig.~\ref{fig:hw_weights_gen}) resets the accumulators' state.
Overall, the vector compute datapath is design-time configurable with respect to parameter $M$ which controls the sizing of the vectors units and balances in this way the performance-resource usage trade-off of \texttt{CNN-WGen}. The design space exploration of $M$ is discussed in Section~\ref{sec:dse}.

% the design-time configurable parameter $M$ determines the size of the vector units and controls in this way the performance-resource trade-off of \texttt{CNN-WGen}. The tuning of $M$ is exposed to the DSE to determine how many resources should be allocated to \texttt{CNN-WGen}.

\textbf{Memory Customisation in Alpha Buffer.}
TiWGen dictates that each subtile contains weights from $N_f$ distinct $K$$\times$$K$ filters. To sustain \texttt{CNN-WGen}'s throughput, an equal number of $\alpha$s have to be fetched in parallel from the \textit{Alpha buffer}. To accomplish this, we design the \textit{Alpha buffer} as a unified buffer with customised memory organisation and addressing.
Each layer contains $N_{\text{in}} N_{\text{out}} \left\lceil \rho_l K_l^2 \right\rceil$ distinct $\alpha$ values. 
As such, the \textit{Alpha buffer} is broken down to $N_P^{\text{Alpha}}$$=$$N_f$ independent multi-bank sub-buffers, with a depth of $D^{\text{Alpha}}$ (Eq.~(\ref{eq:alpha_buff_depth})) to accommodate $N_L$ layers.
% 
% \vspace{-1mm}
\begin{equation}
    \footnotesize
    % \resizebox{0.725\linewidth}{!}{
    N_{f} = \left\lceil \frac{\min (T_P,M)}{K_\text{max}^2} \right\rceil \left\lfloor \frac{M}{T_P} \right\rfloor + \text{mod}(M,T_P) \left\lceil \frac{M}{K_\text{max}^2} \right\rceil
    % }
    \vspace{-1mm}
    \label{eq:filters_per_subtile}
\end{equation}
\begin{equation}
    \footnotesize
    % \resizebox{0.725\linewidth}{!}{
    D^{\text{Alpha}} = \overbrace{\sum_{l=1}^{N_L}}^{\text{for each layer}} \frac{\overbrace{N_{\text{in}}^l N_{\text{out}}^l \left\lceil \rho_l K_l^2 \right\rceil}^{\text{no. of $\alpha$ values}}}{N_P^{\text{Alpha}}} \quad \text{ ( \textit{Buffer depth} )}
    % }
    \label{eq:alpha_buff_depth}
    \vspace{-0.1cm}
    % \vspace{-0.2cm}
\end{equation}
where $N_L$ is the number of layers, $N_{\{\text{in},\text{out}\}}^l$ the l-th layer's number of input/output channels and $\rho_l$ the compression ratio. Finally, the outputs of the sub-buffers are concatenated and connected to the multiplier array to provide concurrent access to $N_f$ coefficients. Finally, if the number of $\alpha$ coefficients exceeds the available on-chip memory, the remaining coefficients are transferred from the off-chip memory.

\textbf{Rate Matching in OVSF Generator.}
Following TiWGen, the basis vectors are processed in a blocked manner with a tile size of $M$. This approach leads to two pipelined loops over the $\left\lceil \frac{T_P T_C}{M} \right\rceil$ subtiles (line~2) and the $\rho K^2$ basis vectors (line~4) and the unrolled loop of processing the $M$-element subtile with the vector units (line~5). To produce the i-th subtile, the \textit{OVSF generator} feeds the compute datapath with $\rho K^2$ basis vectors that are tiled as dictated by TiWGen's parameter $M$.

In order not to straggle the operation of \texttt{CNN-WGen}, the \textit{OVSF generator} has to match the rate of the vector units by feeding them with $M$ bits/cycle. A conventional design involves statically laying out the tiled vectors into a single buffer, with $M$ ports and a depth equal to the number of reads per tile (\textit{i.e.}~\#basis vectors$\times$\#subtiles).
However, such a monolithic design would impose significant overheads as the basis vectors would have to be replicated either in the same address (\textit{e.g.}~when $M$$>$$K^2$) or in multiple addresses (\textit{e.g.}~storing rotated versions as required by different subtiles). This leads to inefficient utilisaiton of the on-chip memory due to excessive replication.

An alternative approach that would avoid the basis vector replication involves the instantiation of a $K^2$-deep OVSF memory with each location storing one $K^2$-bit vector. Such a design requires significantly lower amount of storage and provides an access rate of 1 vector/cycle by reading the appropriate address. Nonetheless, to obtain the $M$-bit subtile from the $K^2$-bit vectoc, complex multiplexer selection circuitry has to be instantiated. This approach can affect the maximum clock rate or add latency cycles to such an extent that any throughput gains would be outweighed.

To alleviate these limitations when mapping TiWGen's tiling scheme, a custom \textit{OVSF generator} was developed. The top-level diagram of the \textit{OVSF generator} is shown in Fig.~\ref{fig:hw_weights_gen}. It is composed of three main components: the \textit{OVSF FIFO}, a \textit{basis vector aligner} and the \textit{output register}. By introducing a FIFO for the OVSF vectors in combination with a \textit{basis vector aligner}, the \textit{OVSF generator} introduces a rate-matching mechanism that sustains the processing rate of the \textit{vector compute datapath} while efficiently utilising the on-chip memory. The generator performs a different operation depending on the values of $M$ and $K^2$ of layer $i$ (Fig.~\ref{fig:hw_weights_gen}).

Initially, the \textit{OVSF FIFO} stores the \mbox{{\small ($K_i^2K_i^2$)}-bit} basis vectors. %\footnote{Based on OVSF theory, an $\mathbb{R}^{K^2}$ base requires $K^2$ components.} 
The current vector is read from the FIFO into the top register. 
If {\small $M$$\le$$K_i^2$}, % (top left), 
the $M$ least significant bits (LSBs) are outputted to the \textit{vector compute datapath}. At the same time, the basis vector is processed by the \textit{basis vector aligner}, which performs an $M$-bit left circular shift and writes the rotated vector to the \textit{OVSF FIFO}.
If {\small $M$$>$$K_i^2$}, % (top right), 
the basis vector is self-concatenated {\small $\left\lfloor \frac{M}{K_i^2} \right\rfloor$} times and written to the output's LS part. Simultaneously, the {\small $\text{mod}(M,K_i^2)$} LSBs of the basis vector are written to the output's MSBs and the constructed vector is passed to the compute datapath. Here, the aligner performs a left circ-shift of \mbox{{\small $K_i^2$-$\text{mod}(M,K_i^2)$}} bits and writes the result to the FIFO.

With this approach, when the basis vectors are read again out of the OVSF FIFO after $\rho K_i^2$ cycles (\textit{i.e.}~in the next iteration of the loop on line~2), they are correctly aligned to directly match TiWGen's tiling pattern. For instance, after the generation of the red-striped subtile in Fig.~\ref{fig:unzipfpga_cnn_engine}, the FIFO-read basis vectors will be correctly aligned in order to generate the blue-striped subtile without the need for costly selection logic or redundant storage. 
For CNNs with multiple filter sizes, the \textit{basis vector aligner} is instantiated with as many circ-shift options. As the distinct filter sizes are known \textit{a priori}, only the required shifting logic is inserted, avoiding expensive generic multiplexers, and the appropriate per-layer bit-shift is selected at run time.

Overall, the proposed design offers two main benefits. First, it alleviates the redundant replicated storage of basis vectors and avoids the hardware cost of partitioning multiplexers that would require excessive LUTs usage. Second, it provides the necessary bandwidth to the \textit{vector compute datapath} while efficiently utilising the on-chip memory through the \textit{OVSF FIFO} and the resource-efficient aligner design.
As the values of $K_i^2$ for each layer and $M$ are known at design time and after the DSE phase respectively, the \textit{OVSF generator} can be statically instantiated at compile time. % with the appropriate design.

\begin{wrapfigure}{R}{0.45\textwidth}
    \centering
    \vspace{-0.6cm}
    % \fbox
    {
    % Figure removed
    }
    \vspace{-0.6cm}
    \captionsetup{font=small,labelfont=bf}
    \caption{\footnotesize \tool's input selective PE array for CNNs.}
    \vspace{-0.2cm}
    \label{fig:new_pe_design}
\end{wrapfigure}

\vspace{0.4cm}
\subsection{Input Selective PEs for Counteracting Underutilisation}
\label{sec:pe_design}
% \vspace{-0.2em}

One key limitation of existing CNN engines is that, when processing compute-bound layers, the layer dimensions often do not match the fixed processing engine configuration, leading to underutilisation of the computational resources and severe performance penalties~\cite{latency2017fpl,alamo2020tcad,caffeine2019tcad,maestro2020micro}.
Such a scenario can be observed when mapping a layer with $N_\text{out}$$=$$64$ channels (\textit{i.e.}~$C$$=$$64$) on an engine with 128 PEs (\textit{i.e.}~$T_C$$=$$128$). In this case, the PEs would remain idle 50\% of the time, halving the attainable performance.

To alleviate this, we propose \textit{input selective PEs}, a design that enables existing PEs to perform load-balancing through inter-PE work-stealing in a resource-efficient manner.
Fig.~\ref{fig:new_pe_design} shows \tool's input selective PEs. The initial PE is augmented with registers and switches. However, not all PEs have the same components; only the PEs that remain \textit{underutilised} even for a single layer are further equipped with a compact switch that selects the inputs to the dot-product circuit. 
In addition to the normal flow of data, these switches enable each PE to send its weight to its bottom neighbour. As highlighted in dark blue at the bottom PE of Fig.~\ref{fig:new_pe_design}, the switch on the left of the PE selects its input from \rev{two options: \textit{i)}~under normal operation, the PE is fed with the weight written by \texttt{CNN-WGen} in the weights buffer $\bigl(\protect\circled{1}\bigr)$; \textit{ii)}~in the absence of this weight (\textit{e.g.}~due to a mismatch between $C$ and $T_C$), the PE is fed with the weight passed by the adjacent PE $\bigl(\protect\circled{3}\bigr)$.}  
% either \textit{i)}~the weight written by \texttt{CNN-WGen} in the weights buffer \rev{$\bigl(\protect\circled{1}\bigr)$} or, \textit{ii)}~in the absence of this weight, the weight passed by the adjacent PE \rev{$\bigl(\protect\circled{3}\bigr)$}.
In the second case, the weights are propagated along the PE array \rev{$\bigl(\protect\circled{3}\bigr)$} so that a different weight is used by each augmented PE in each cycle. Moreover, the Input Buffer (Fig.~\ref{fig:unzipfpga_cnn_engine}) is reorganised accordingly to provide parallel access to multiple rows.




Effectively, this design works as a load-balancing mechanism that partially unrolls the $T_R$ dimension and thus distributes the work more evenly among the PEs. By restricting connectivity to adjacent units and enhancing only the underutilised PEs, the additional circuitry is low-overhead and delivers up to 20\% higher performance on compute-bound layers.


\section{Design Space Exploration}
\label{sec:dse}

Based on its parametrisation of the processing engine, buffer sizes and weights generator, \tool defines a particular architectural design space. To estimate the performance and resource usage of different configurations, an analytical modelling framework has been developed. 
At a high-level, the key decisions for yielding a high-performance configuration of the system are: the allocation of the on-chip resources between the CNN engine and the weights generator and, the sizes of the activations buffers.
The design-time tunable parameters comprise \textit{1)}~$M$ that determines the TiWGen's tile size and the size of \texttt{CNN-WGen}'s vector units, \textit{2)}~tile sizes $T_C$ and $T_P$ that determine the number of PEs and MACs per PE, respectively, and $T_R$ affecting the size of the activations buffers.

\subsection{Performance Model}
\label{sec:perf_model}

The workload of a CNN with $N_L$ layers is represented as a sequence of {\small $W_i$$=$$\left<R_i,P_i,C_i\right>$} \textit{workload tuples} with $i~\in~\{1,...,N_L\}$.
Given a design point {\small $\sigma$$=$$\left<M,T_R,T_P,T_C\right>$}, the \texttt{CNN-WGen}'s runtime for generating the i-th layer's weights required 
to compute a {\small $(T_R\times T_C)$} 
output tile is given by
\vspace{-0.1cm}
%
\begin{equation}
    \footnotesize
    \vspace{-1mm}
    % \resizebox{0.7\linewidth}{!}{
    t_{\texttt{CNN-WGen}}^i(\sigma, W_i) = \left\lfloor \rho \cdot l \right\rfloor \cdot \left\lceil \frac{T_P \cdot T_C}{M} \right\rceil \cdot \left\lceil \frac{P_i}{T_P} \right\rceil
    % }
    \vspace{-0.5mm}
    \label{eq:wgen_exec_time}
\end{equation}
% 
where $\rho$ and $l$ are the OVSF ratio and basis length, respectively, and with one factor for each of the pipelined loops in Algorithm~\ref{alg:tiled_weights_gen}.
With $\alpha$ values transferred upfront and the OVSF method generating all weights on-chip, the off-chip memory transfers involve only the input/output activations
% 
\begin{equation}
    \footnotesize
    % \resizebox{0.45\textwidth}{!}{
    t_{\text{mem in}}^i(\sigma, W_i) = \frac{T_R \cdot P \cdot WL}{bw_\text{in}}, \quad t_{\text{mem out}}^i(\sigma, W_i) = \frac{T_R \cdot T_C \cdot WL}{bw_\text{out}}
    % }
    \label{eq:transfer_times}
\end{equation}
% 
where $WL$ is the adopted wordlength, and $bw_{\{\text{in},\text{out}\}}$ are the memory bandwidths for transferring inputs/outputs.


With $T_C$ and $T_P$ dimensions unrolled, computing an output tile %by the accelerator's CNN engine 
requires the pipelined processing of $\frac{P_i}{T_P}$ tiles for each of the $T_R$ rows. Hence, the processing engine's runtime for each output tile is estimated as {\small $t_{\text{eng}}^i(\sigma, W_i) = T_R \left\lceil \frac{P_i}{T_P} \right\rceil$}. With the input selective PEs, the runtime is refined as
% 
\begin{equation}
    \footnotesize
    % \resizebox{0.45\textwidth}{!}{
    t_{\text{eng}^*}^i (\sigma, W_i) = \left( T_C-C_i + \left\lceil \frac{T_R \cdot C_i - (T_C-C_i) \cdot (C_i+1)}{T_C} \right\rceil \right) \cdot \left\lceil \frac{P_i}{T_P} \right\rceil
    % }
    \label{eq:cnn_engine_exec_time_enhanced}
\end{equation}
% 
where dimension $T_R$ is partially unrolled by processing rows of $T_R$ through the underutilised PEs.

Overall, the accelerator forms a pipeline of three coarse stages:
the concurrent input transfer and weights generation, the CNN engine processing and the output transfer. 
In this context, the initiation interval of the architecture is given by the maximum initiation interval of the three-stage pipeline, calculated as
% 
% \vspace{-1mm}
\begin{equation}
    \footnotesize
    \vspace{1mm}
    % \resizebox{0.91\linewidth}{!}{
    II^i(\sigma, W_i) = \max\left( \max\left(t_{\text{mem in}}^i, t_{\texttt{CNN-WGen}}^i \right), t_{\text{eng}^*}^i, t_{\text{mem out}}^i \right)
    % }
    \label{eq:initiation_interval}
\end{equation}
% 
As such, the total runtime for layer $i$ is given by 
\mbox{{\footnotesize $t_\text{total}^i(\sigma, W_i)=II^i(\sigma, W_i) \left\lceil \frac{R_i}{T_R} \right\rceil \left\lceil \frac{C_i}{T_C} \right\rceil$}}.
Thus, for a CNN with $N_L$ layers, the workload tuple is {\footnotesize $W$$=$$\left<W_i ~|~ \forall i \in \{1, ..., N_L \} \right>$} and the throughput in inferences per sec (inf/s) is estimated as \mbox{{\footnotesize $T(\sigma, W) = 1/\sum\limits_{i=1}^{N_L} t_\text{total}^i (\sigma, W_i)$}}.

\subsection{Resource Consumption Model}
The primary factor that constrains the mapping of a CNN engine on a given platform is resource availability. Each candidate configuration has a corresponding resource consumption. We define the \textit{feasible space} of our model as the set of configurations that satisfy all the platform-specific resource constraints. 
In our context, the main design constraints are the DSPs and on-chip RAM blocks of the target FPGA. Assuming that all MAC operators are mapped to DSPs, the values of {\small $
\left<M,T_P,T_C\right>$} are constrained as \mbox{{\footnotesize $D_\text{MAC} \times (M + T_P T_C) \leq D_\text{fpga}$}},
with {\footnotesize $D_{\text{fpga}}$} the available DSPs and {\footnotesize $D_\text{MAC}$} the DSPs/MAC. We consider 16-bit fixed-point precision, where {\footnotesize $D_\text{MAC}$$=$$1$} on the evaluated Xilinx FPGAs.
%In our case, 16-bit fixed-point precision is used, where {\footnotesize $D_\text{MAC}$$=$$1$} on the evaluated Xilinx FPGAs.


% From a resource perspective, the main design constraints are the DSPs and on-chip RAM blocks of the target FPGA. Assuming that all MAC operators are mapped to DSPs, the values of {\small $
% \left<M,T_P,T_C\right>$} are constrained as {\footnotesize $D_\text{MAC} \times (M + T_P T_C) \leq D_\text{fpga}$},
% with {\footnotesize $D_{\text{fpga}}$} the available DSPs and {\footnotesize $D_\text{MAC}$} the DSPs/MAC. In our case, 16-bit fixed-point precision is used, where {\footnotesize $D_\text{MAC}$$=$$1$} on the evaluated Xilinx FPGAs.

In terms of on-chip RAM, the accelerator has the I/O and Alpha buffers with wordlength $WL$ and the binary OVSF FIFO, with a total capacity requirement as given by Eq.~(\ref{eq:onchip_ram}).
% 
% \vspace{-1mm}
\begin{equation}
    % \footnotesize
    \vspace{1mm}
    % \resizebox{0.91\linewidth}{!}{
    \left(2(T_R T_P + T_R T_C) + D^{\text{Alpha}}N_P^{\text{Alpha}}\right) WL + K_\text{max}^2K_\text{max}^2 \leq C_\text{fpga}
    % }
    \label{eq:onchip_ram}
\end{equation}
% 
where the factor of 2 accounts for double-buffering and $C_\text{fpga}$ is the on-chip RAM capacity of the target device.

To further estimate the consumption of look-up tables (LUTs), we used a set of place-and-route measurements and fitted linear regression models as a function of \tool's tunable parameters.
% 
Overall, we formally capture the %on-chip 
resource consumption of a design point $\sigma$ by means of vector $\textbf{\textit{rsc}}(\sigma)$ that holds the utilised amount of DSPs, BRAMs and LUTs. Similarly, we denote the FPGA resource vector by $\textbf{\textit{rsc}}_\text{Avail.}$. 

\subsection{Configuration Optimisation Framework}
To yield the highest performing design for the given CNN-FPGA pair, we cast the DSE task as a constrained optimisation problem that aims to determine the values of the configurable parameters $\left< M, T_R, T_P, T_C \right>$ that achieve the highest performance for the target CNN and available hardware resources. Formally, we express this setup as
% 
\begin{equation}
    \footnotesize
    \min\limits_{\sigma=\left<M,T_R,T_P,T_C\right>} T\left(\sigma, W \right) \quad \text{s.t.} \quad \textbf{\textit{rsc}}(\sigma) \leq \textbf{\textit{rsc}}_\text{Avail.}
    \label{eq:opt}
\end{equation}
%
where $T$, $\textbf{\textit{rsc}}$ and $\textbf{\textit{rsc}}_{\text{Avail.}}$ are the throughput in inferences per second (inf/s), the resource consumption of the current design point $\sigma$ and the resource vector of the target platform, respectively.
% Given a CNN-FPGA pair, we perform exhaustive search, exploring different resource allocations between \texttt{CNN-WGen} and the processing engine. 
Given a CNN-FPGA pair, we perform exhaustive search for different resource allocations between \texttt{CNN-WGen} and the processing engine. 
Designs that violate the resource constraints are pruned as infeasible to accelerate the exploration.

\section{Deriving Lightweight OVSF Models}
\label{sec:light_ovsf_models}

Having presented \tool's hardware architecture, its strategy for mapping OVSF models on the accelerator and its design space exploration process, we now describe important challenges for constructing efficient OVSF models. The main challenges comprise: \textit{i)}~extracting correctly-sized filters from OVSF codes, \textit{ii)}~selecting a subset of OVSF vectors to meet a given OVSF ratio for each layer, and \textit{iii)}~setting the per-layer OVSF ratios themselves. Section~\ref{sec:OVSF_issues} discusses our approach to \textit{i)} and \textit{ii)}, while Section~\ref{sec:hw_aware_ratios} introduces our novel hardware-aware scheme for tuning OVSF ratios.

\subsection{Practical Considerations to Train OVSF Models}
\label{sec:OVSF_issues}
Unlike standard CNNs, architectures using OVSF codes do not learn convolutional filters directly. Instead, they learn weighting coefficients for each OVSF code representing a filter. However, despite their simplicity as a straight drop and replacement option for standard convolutional layers, the nature of OVSF codes and the filter generation process, present two fundamental challenges: \textit{1)}~OVSF codes are of power-of-two length: This constrains the generation of filters with all $N_{\text{out}}$, $N_{\text{in}}$, and $K$ being power-of-two integers. While this might be reasonable for the input and output channel dimensions, it prevents the construction of $3$$\times$$3$ filters, which are ubiquitous in modern CNN architectures; and \textit{2)}~choosing a subset of basis: Model compression is only achieved when OVSF ratio $\rho<$ $1$, which raises the question of \textit{which bases to choose} from the total $L$ available for OVSF codes of length $L$. Intuitively, their should be an optimal subset of basis for a given $\rho<$ $1$ that allows the learning of more expressive filters.

For \textit{1)}, we consider between \textit{i)}~utilising the first $\lfloor \rho \cdot K^2\rceil$ codes and \textit{ii)}~iteratively discarding OVSF codes based on their associated scalar $\alpha$ until the target compression ratio $\rho$ is reached. Compared to \textit{ii)}, with \textit{i)} we have a simpler optimization objective at the expense of potentially limiting the expressivity of OVSF filters. For \textit{2)}, we consider \textit{i)}~extracting a $3$$\times$$3$ crop from a $4$$\times$$4$ filter and \textit{ii)}~learning a mapping to a $3$$\times$$3$ filter by means of an average pooling layer. Similarly to the first pair of solutions, \textit{i)} represents a simpler training stage at the expense of a reduced effective field over the OVSF basis when constructing $3$$\times$$3$ filter. In Sec.~\ref{sec:basis_selection_and_3x3_extraction}, we compare both pairs of approaches for the above challenges.

%Training OVSF-base models require the explicit generation of OVSF filters in a layer as part of the inference stage. This can slow down training, especially in very large models.
In certain scenarios, a pre-trained model with standard convolutions might be available or can be trained very cheaply. In such cases, the formulation in Eq.~(\ref{eq:vector_eq}) could be reinterpreted as a minimisation problem and regress the set of $\pmb{\alpha}^*_i$ that minimise the difference w.r.t the standard filter $\hat{f}_{i}$ as \mbox{{\footnotesize$\pmb{\alpha}^{*}$$=$$\argmin_{\alpha}\left\|f - \hat{f} \right \|_2^2$}}, which can be implemented as a 2-layer MLP regression stage. We leverage this strategy when training OVSF models on ImageNet. More details are provided in Section \ref{sec:training_scheme}. %The \textit{Converter}, as in Fig.~\ref{fig:design_flow} transforms a CNN with normal convolutions into its OVSF counterpart following the same regression procedure.


% Figure environment removed

\subsection{Hardware-Aware Tuning of OVSF Ratios}
\label{sec:hw_aware_ratios}

A critical component of \tool is the \textit{OVSF Ratios Selection} module of the \textit{OVSF Model Converter} (Fig.~\ref{fig:design_flow}).
In the original work~\cite{unzipfpga2021fccm}, the OVSF ratios (\textit{i.e.}~$\rho$ for each layer) were manually selected in a coarse, per-block manner, with the objective to reach a given compression ratio while minimising accuracy degradation. For example, as detailed in Section~\ref{sec:training_scheme}, to achieve a compression of 50\% in model size, denoted by OVSF50, the hand-tuned ratios were set by the tuple $[1.0, 0.5, 0.5, 0.5]$, indicating the OVSF ratio for each of the four blocks comprising a ResNet. Lower ratios were assigned to deeper layers as these contain a larger portion of the model parameters and are known to be more resilient to compression. The first CONV layer in the network remains untouched (\textit{i.e.}~not OVSF) as it has been shown to be less resilient to approximations including quantisation~\cite{alizadeh2018a}. To reach higher compression ratios with minimal accuracy drop, more involved tuning is required. This can be observed through the OVSF25 variant which achieves 75\% compression with diverse ratios of $[1.0, 0.4, 0.25, 0.125]$. Nonetheless, this process neither takes into account the impact of ratio values on hardware performance nor is automated, requiring elaborate tuning.

To alleviate this, we introduce a hardware-aware autotuning scheme for selecting the OVSF ratios of a given model on a target device. 
The key insight behind our method is that, for layers that are either compute- or memory-bound, we can allow the weights generation stage to consume more cycles by using more OVSF vectors (\textit{i.e.}~using a higher OVSF ratio) \textit{without} affecting the processing speed. As such, \texttt{CNN-WGen} will output a better approximation of the layer's weights, increasing the model's expressivity and potentially improving accuracy. With reference to our performance model (Section~\ref{sec:perf_model}), this case occurs for layer $i$ when either the processing engine's runtime ($t^i_{\text{eng}*}$) or the off-chip memory transfers ($t^i_{\text{mem in}}$ or $t^i_{\text{mem out}}$)  dominate the initiation interval in Eq.~(\ref{eq:initiation_interval}). This allows us to allocate more cycles for $t^i_{\texttt{CNN-WGen}}$ by using a higher OVSF ratio and obtaining a better approximation of the weights.

With this insight, Fig.~\ref{fig:hw_aware_tune_flow} presents our hardware-aware autotuning scheme is as follows. As a first step, we run \tool's design flow (Fig.~\ref{fig:design_flow}) using the OVSF25 ratios (\textit{e.g.}~$[1.0, 0.4, 0.25, 0.125]$ for ResNet) and derive the corresponding accelerator configuration $\bigl(\circled{1}\bigr)$. Next, we perform a bottleneck analysis of each layer's mapping on the accelerator that indicates which stage dominates the initiation interval $\bigl(\circled{2}\bigr)$, \textit{i.e.}~whether it is memory-bound (either input or output activation transfer), compute-bound or weights-generation-bound. For the layers where \texttt{CNN-WGen} is \textit{not} the bounding factor, we iteratively \stelios{Typically, we converged at 5 to 6 iterations in the examined models. We can report how much extra time that is, i.e. the tuning/search overhead - given the journal's theme, this might be expected. Else, we can leave it for the revised version after the reviews.} increase the OVSF ratios up to the point where the bottleneck does not shift to the weights generation stage $\bigl(\circled{3}\bigr)$. 
This leads to a more balanced pipelining of each layer, hence increasing accuracy by better utilising the instantiated accelerator. At the end of this process, the converged set of OVSF ratios are passed as the output of the \textit{OVSF Ratios Selection} module (Fig.~\ref{fig:design_flow}) and the rest of \tool's flow is run $\bigl(\circled{4}\bigr)$. As such, the model is retrained and the design space exploration is rerun with the new OVSF ratios $\bigl(\circled{5}\bigr)$, and the final model-accelerator pair are deployed on the target FPGA. Despite the additional retraining step, we note that the training protocol, \textit{i.e.}~all hyperparameters, remains the same throughout, without the need for further tuning.

\stelios{This paragraph is new.}
In step $\circled{3}$, the candidate values of ratio $\rho^l$ for the $l$-th layer lie in $\left\{ L^l/n ~|~ \forall \, n \in [1,L^l] \right\}$, where $L^l=N^l_{\text{in}}K^lK^l$ is the layer's code length (Sec.~\ref{sec:background_cnn_ovsf}). As such, there are $|L^l|$ candidate OVSF ratio values per OVSF layer. Overall, for a CNN with $N_L$ OVSF layers, there is a total of $\prod_{l=1}^{N_L} |L^l|$ possible OVSF ratio combinations. With an increase in either the model's depth or an OVSF layer's width, an enumerative exhaustive search quickly becomes computationally intractable. To alleviate this, we perform a parallel search for all OVSF layers, starting from the OVSF25 configuration.
At the first iteration, we first calculate the \textit{throughput ratio} between the weights-generation stage and the bottleneck stage of each layer and set the respective OVSF ratio to the closest feasible value. Then, we search the neighbouring candidate ratios until we find the maximum value that does not turn the weights-generation stage into the bottleneck. Throughout our experiments, this process lead to an average of 5 iterations to converge to the final hardware-aware set of ratios across the examined models.

Table~\ref{tab:hw_aware_tune_example} illustrates the impact of our scheme, denoted by \texttt{hw-aware-autotuning}, using ResNet18 on Z7045 for varying bandwidth availability. In the most bandwidth-constrained case (1.1~GB/s), OVSF25 is memory-bound, with all layers being limited by the transfer of the input feature maps. Our method exploits severe memory-boundedness and selectively increases the OVSF ratios, leading to an accuracy improvement of 1.2pp over OVSF25 with no sacrifice of the processing speed. For medium bandwidth levels (2.2~GB/s), a number of OVSF25 layers become compute-bound. If we naively set all OVSF ratios to 1.0 (shown as \texttt{uniform-1.0}), several layers become bound by the weights generation stage. Instead, with our bottleneck-guided method, the weights are more accurately generated while no change occurs to the boundedness of each layer. This results in achieving the same throughput as OVSF25, but with a 1.1pp increase in accuracy. Finally, in the high-bandwidth case (4.4~GB/s), our method introduces a 0.3pp accuracy gain, without affecting the hardware performance.
\stelios{Can comment more on the selection of ratios. Specifically, point out that our method picks the exact OVSF ratio values depending on the margin of each layer from changing bounding factor, e.g. there is room for higher ratios for some middle layers (0.333 in 4.4 GB/s) and less room in latter layers (0.25 ratios).}

Overall, our OVSF ratio selection method incorporates three features: \textit{1)}~it is fine-grained by allowing for different per-layer OVSF ratios within each block. As such, we obtain finer-grained control over the accuracy-compression trade-off; \textit{2)}~it bounds the accuracy drop from below by means of an informed initialisation of the ratio values. By starting from the OVSF25 ratios (\textit{i.e.} our most lightweight setting), we guarantee the accuracy's lowest bound and, by allowing only increases in OVSF ratios, we ensure that these would only potentially contribute accuracy gains; and \textit{3)}~it is hardware-aware as it is guided by the bottleneck analysis of each layer's processing.





\begin{table}[t]
    \normalsize
    % \tablefontsize
    \centering
    \captionsetup{font=small,labelfont=bf}
    \captionof{table}{\footnotesize Different OVSF ratio selection methods with respect to accuracy and bottleneck stage for ResNet18.}
    \vspace{-0.1cm}
    \resizebox{1.0\linewidth}{!}{
    \begin{tabular}{c|c|c|l|cccccccccccccccccccc}
   
        \toprule
         Memory & OVSF Ratio & Accuracy & & \multicolumn{20}{c}{Layer ID} \\
         Bandwidth & Selection Method & (\%) & & L0 & L1 & L2 & L3 & L4 & L5 & L6 & L7 & L8 & L9 & L10 & L11 & L12 & L13 & L14 & L15 & L16 & L17 & L18 & L19 \\ 
        \midrule

        % 1.1 GB/s
        % \multirow{6}{*}{\parbox{1.9cm}{\centering ResNet18}} 
        \multirow{6}{*}{1.1 GB/s} & \multirow{2}{*}{OVSF25} & \multirow{2}{*}{67.3} & Bound & 
        IFM & IFM & IFM & IFM & IFM & IFM & IFM & IFM & IFM & IFM & IFM & IFM & IFM & IFM & IFM & IFM & IFM & IFM & IFM & IFM \\
        % \cline{5-24} 
        & & & OVSF Ratio & 1.0 & 1.0 & 1.0 & 1.0 & 1.0 & 0.4 & 0.4 & 1.0 & 0.4 & 0.4 & 0.25 & 0.25 & 1.0 & 0.25 & 0.25 & 0.125 & 0.125 & 1.0 & 0.125 & 0.125 \\
        
        \cline{3-24} & \multirow{2}{*}{\texttt{uniform-1.0}} & \multirow{2}{*}{N/A} & Bound & 
        IFM & IFM & IFM & IFM & IFM & IFM & IFM & IFM & IFM & IFM & IFM & IFM & IFM & IFM & IFM & IFM & IFM & IFM & IFM & IFM \\
        % \cline{5-24} 
        & & & OVSF Ratio & 
        1.0 & 1.0 & 1.0 & 1.0 & 1.0 & 1.0 & 1.0 & 1.0 & 1.0 & 1.0 & 1.0 & 1.0 & 1.0 & 1.0 & 1.0 & 1.0 & 1.0 & 1.0 & 1.0 & 1.0
        \\
        
        \cline{3-24} & \multirow{2}{*}{\texttt{hw-aware-autotuning}} & \multirow{2}{*}{68.5} & Bound & 
        IFM & IFM & IFM & IFM & IFM & IFM & IFM & IFM & IFM & IFM & IFM & IFM & IFM & IFM & IFM & IFM & IFM & IFM & IFM & IFM \\
        % \cline{5-24} 
        & & & OVSF Ratio & 
        1.0 & 1.0 & 1.0 & 1.0 & 1.0 & 1.0 & 1.0 & 1.0 & 1.0 & 1.0 & 1.0 & 0.5 & 0.5 & 0.5 & 0.5 & 0.5 & 0.5 & 0.5 & 0.5 & 0.25 \\
        \cline{2-24}

        % 2.2 GB/s
         \multirow{6}{*}{2.2 GB/s} & \multirow{2}{*}{OVSF25} & \multirow{2}{*}{67.3} & Bound & 
        IFM & IFM & IFM & IFM & IFM & C & C & IFM & C & C & C & C & IFM & C & C & C & C & IFM & C & C \\
        % \cline{5-24} 
        & & & OVSF Ratio & 1.0 & 1.0 & 1.0 & 1.0 & 1.0 & 0.4 & 0.4 & 1.0 & 0.4 & 0.4 & 0.25 & 0.25 & 1.0 & 0.25 & 0.25 & 0.125 & 0.125 & 1.0 & 0.125 & 0.125 \\
        
        \cline{3-24} & \multirow{2}{*}{\texttt{uniform-1.0}} & \multirow{2}{*}{N/A} & Bound & 
        IFM & IFM & IFM & IFM & IFM & C & C & IFM & C & C & C & C & IFM & C & C & C & W & IFM & W & W \\
        % \cline{5-24} 
        & & & OVSF Ratio & 
        1.0 & 1.0 & 1.0 & 1.0 & 1.0 & 1.0 & 1.0 & 1.0 & 1.0 & 1.0 & 1.0 & 1.0 & 1.0 & 1.0 & 1.0 & 1.0 & 1.0 & 1.0 & 1.0 & 1.0
        \\
        
        \cline{3-24} & \multirow{2}{*}{\texttt{hw-aware-autotuning}} & \multirow{2}{*}{68.4} & Bound & 
        IFM & IFM & IFM & IFM & IFM & C & C & IFM & C & C & C & C & IFM & C & C & C & C & IFM & C & C \\
        % \cline{5-24} 
        & & & OVSF Ratio & 
        1.0 & 1.0 & 1.0 & 1.0 & 1.0 & 1.0 & 1.0 & 1.0 & 1.0 & 1.0 & 0.5 & 0.5 & 0.5 & 0.5 & 0.5 & 0.5 & 0.5 & 0.5 & 0.5 & 0.25 \\
        \cline{2-24}

        % 4.4 GB/s
        % \multirow{6}{*}{\parbox{1.9cm}{\centering ResNet18}} 
        \multirow{6}{*}{4.4 GB/s} & \multirow{2}{*}{OVSF25} & \multirow{2}{*}{67.3} & Bound & 
        IFM & C & C & C & C & C & C & OFM & C & C & C & C & IFM & C & C & C & C & IFM & C & C \\
        % \cline{5-24} 
         & & & OVSF Ratio & 1.0 & 1.0 & 1.0 & 1.0 & 1.0 & 0.4 & 0.4 & 1.0 & 0.4 & 0.4 & 0.25 & 0.25 & 1.0 & 0.25 & 0.25 & 0.125 & 0.125 & 1.0 & 0.125 & 0.125 \\
        
        \cline{3-24} & \multirow{2}{*}{\texttt{uniform-1.0}} & \multirow{2}{*}{N/A} & Bound & 
        IFM & C & C & C & C & C & C & OFM & C & C & W & W & IFM & W & W & W & W & IFM & W & W \\
        % \cline{5-24} 
        & & & OVSF Ratio & 
        1.0 & 1.0 & 1.0 & 1.0 & 1.0 & 1.0 & 1.0 & 1.0 & 1.0 & 1.0 & 1.0 & 1.0 & 1.0 & 1.0 & 1.0 & 1.0 & 1.0 & 1.0 & 1.0 & 1.0
        \\
        
        \cline{3-24} & \multirow{2}{*}{\texttt{hw-aware-autotuning}} & \multirow{2}{*}{67.6} & Bound & 
        IFM & C & C & C & C & C & C & OFM & C & C & C & C & IFM & C & C & C & C & IFM & C & C \\
        % \cline{5-24} 
        & & & OVSF Ratio & 
        1.0 & 1.0 & 1.0 & 1.0 & 1.0 & 1.0 & 1.0 & 1.0 & 1.0 & 0.333 & 0.333 & 0.5 & 0.333 & 0.333 & 0.333 & 0.25 & 0.25 & 0.25 & 0.25 & 0.25 \\
        
        % \cline{2-3} \cline{5-5} \cline{7-7} \cline{9-9}
        % & \multirow{2}{*}{Iterative} & Crop &  & 94.1 & & 93.6 & & \textbf{93.6} \\
        % & & Adaptive &  & 94.0 & & 93.8 & & 92.3 \\ 
        % \midrule \midrule
                                   
        % \multirow{4}{*}{\parbox{1.9cm}{\centering ResNet18$^\dagger$ 91.3\% 0.27M}} & \multirow{2}{*}{Sequential} & Crop & \multirow{4}{*}{0.48} & 90.8 & \multirow{4}{*}{0.25} & 90.8 & \multirow{4}{*}{0.15} & 88.3 \\
        %                           & & Adaptive &  & 91.1 & & 91.2 & & 88.5\\ \cline{2-3} \cline{5-5} \cline{7-7} \cline{9-9}
        %                           & \multirow{2}{*}{Iterative} & Crop &  & 91.1 & & 91.3 & & \textbf{91.4}\\
        %                           & & Adaptive &  & 91.2 & & 91.4 & & 91.0\\ \midrule \midrule
                                   
        % \multirow{4}{*}{\parbox{1.9cm}{\centering ResNet34 93.9\% 21.3M }} & \multirow{2}{*}{Sequential} & Crop & \multirow{4}{*}{37.7} & 94.1 &  \multirow{4}{*}{17.6} & 93.9 & \multirow{4}{*}{7.2} & 93.4  \\
        %                           & & Adaptive &  & 94.3& & 94.0 & & 93.4 \\ \cline{2-3} \cline{5-5} \cline{7-7} \cline{9-9}
        %                           & \multirow{2}{*}{Iterative} & Crop &  & 94.1& & 93.8 & & \textbf{94.3}\\
        %                           & & Adaptive &  & 93.8 & & 93.7 & & 93.2\\ \midrule \midrule

        % \multirow{4}{*}{\parbox{1.9cm}{\centering ResNet34$^\dagger$ 92.1\% 0.46M}} & \multirow{2}{*}{Sequential} & Crop & \multirow{4}{*}{0.82} & 92.3 & \multirow{4}{*}{0.43} & 91.4 & \multirow{4}{*}{0.26} & 89.3 \\
        %                           & & Adaptive &  & 92.2 & & 91.5 & & 89.2\\ \cline{2-3} \cline{5-5} \cline{7-7} \cline{9-9}
        %                           & \multirow{2}{*}{Iterative} & Crop &  & 92.3& & 91.8 & & \textbf{92.2}\\
        %                           & & Adaptive &  & 92.4 & & 91.7 & & 91.7\\

    \bottomrule
    \multicolumn{20}{l}{* IFM: Memory-bound w.r.t. input feature maps | OFM: Memory-bound w.r.t. output feature maps | C: Compute-bound | W: Weights Generation-bound.} \\
    \end{tabular}
    }
    
    \label{tab:hw_aware_tune_example}
\end{table}

% \begin{table}[t]
%     \normalsize
%     % \tablefontsize
%     \centering
%     \captionsetup{font=small,labelfont=bf}
%     \captionof{table}{\footnotesize Comparison of different OVSF ratio selection methods with respect to accuracy and bottleneck stage.}
%     \vspace{-0.1cm}
%     \resizebox{1.0\linewidth}{!}{
%     \begin{tabular}{c|c|c|c|l|cccccccccccccccccccc}
   
%         \toprule
%         Model & Bandwidth & OVSF Ratio & Accuracy & & \multicolumn{20}{c}{Layer ID} \\
%         Arch. &  & Selection Method & (\%) & & L0 & L1 & L2 & L3 & L4 & L5 & L6 & L7 & L8 & L9 & L10 & L11 & L12 & L13 & L14 & L15 & L16 & L17 & L18 & L19 \\ 
%         \midrule

%         % 1.1 GB/s
%         % \multirow{6}{*}{\parbox{1.9cm}{\centering ResNet18}} 
%         & \multirow{6}{*}{1.1 GB/s} & \multirow{2}{*}{OVSF25} & \multirow{2}{*}{67.3} & Bound & 
%         IFM & IFM & IFM & IFM & IFM & IFM & IFM & IFM & IFM & IFM & IFM & IFM & IFM & IFM & IFM & IFM & IFM & IFM & IFM & IFM \\
%         % \cline{5-24} 
%         & & & & OVSF Ratio & 1.0 & 1.0 & 1.0 & 1.0 & 1.0 & 0.4 & 0.4 & 1.0 & 0.4 & 0.4 & 0.25 & 0.25 & 1.0 & 0.25 & 0.25 & 0.125 & 0.125 & 1.0 & 0.125 & 0.125 \\
        
%         \cline{3-25} & & \multirow{2}{*}{\texttt{uniform-1.0}} & \multirow{2}{*}{N/A} & Bound & 
%         IFM & IFM & IFM & IFM & IFM & IFM & IFM & IFM & IFM & IFM & IFM & IFM & IFM & IFM & IFM & IFM & IFM & IFM & IFM & IFM \\
%         % \cline{5-24} 
%         & & & & OVSF Ratio & 
%         1.0 & 1.0 & 1.0 & 1.0 & 1.0 & 1.0 & 1.0 & 1.0 & 1.0 & 1.0 & 1.0 & 1.0 & 1.0 & 1.0 & 1.0 & 1.0 & 1.0 & 1.0 & 1.0 & 1.0
%         \\
        
%         \cline{3-25} & & \multirow{2}{*}{\texttt{hw-aware-autotuning}} & \multirow{2}{*}{68.5} & Bound & 
%         IFM & IFM & IFM & IFM & IFM & IFM & IFM & IFM & IFM & IFM & IFM & IFM & IFM & IFM & IFM & IFM & IFM & IFM & IFM & IFM \\
%         % \cline{5-24} 
%         & & & & OVSF Ratio & 
%         1.0 & 1.0 & 1.0 & 1.0 & 1.0 & 1.0 & 1.0 & 1.0 & 1.0 & 1.0 & 1.0 & 0.5 & 0.5 & 0.5 & 0.5 & 0.5 & 0.5 & 0.5 & 0.5 & 0.25 \\
%         \cline{2-25}

%         % 2.2 GB/s
%         \multirow{6}{*}{\parbox{1.9cm}{\centering ResNet18}} & \multirow{6}{*}{2.2 GB/s} & \multirow{2}{*}{OVSF25} & \multirow{2}{*}{67.3} & Bound & 
%         IFM & IFM & IFM & IFM & IFM & C & C & IFM & C & C & C & C & IFM & C & C & C & C & IFM & C & C \\
%         % \cline{5-24} 
%         & & & & OVSF Ratio & 1.0 & 1.0 & 1.0 & 1.0 & 1.0 & 0.4 & 0.4 & 1.0 & 0.4 & 0.4 & 0.25 & 0.25 & 1.0 & 0.25 & 0.25 & 0.125 & 0.125 & 1.0 & 0.125 & 0.125 \\
        
%         \cline{3-25} & & \multirow{2}{*}{\texttt{uniform-1.0}} & \multirow{2}{*}{N/A} & Bound & 
%         IFM & IFM & IFM & IFM & IFM & C & C & IFM & C & C & C & C & IFM & C & C & C & W & IFM & W & W \\
%         % \cline{5-24} 
%         & & & & OVSF Ratio & 
%         1.0 & 1.0 & 1.0 & 1.0 & 1.0 & 1.0 & 1.0 & 1.0 & 1.0 & 1.0 & 1.0 & 1.0 & 1.0 & 1.0 & 1.0 & 1.0 & 1.0 & 1.0 & 1.0 & 1.0
%         \\
        
%         \cline{3-25} & & \multirow{2}{*}{\texttt{hw-aware-autotuning}} & \multirow{2}{*}{68.4} & Bound & 
%         IFM & IFM & IFM & IFM & IFM & C & C & IFM & C & C & C & C & IFM & C & C & C & C & IFM & C & C \\
%         % \cline{5-24} 
%         & & & & OVSF Ratio & 
%         1.0 & 1.0 & 1.0 & 1.0 & 1.0 & 1.0 & 1.0 & 1.0 & 1.0 & 1.0 & 0.5 & 0.5 & 0.5 & 0.5 & 0.5 & 0.5 & 0.5 & 0.5 & 0.5 & 0.25 \\
%         \cline{2-25}

%         % 4.4 GB/s
%         % \multirow{6}{*}{\parbox{1.9cm}{\centering ResNet18}} 
%         & \multirow{6}{*}{4.4 GB/s} & \multirow{2}{*}{OVSF25} & \multirow{2}{*}{67.3} & Bound & 
%         IFM & C & C & C & C & C & C & OFM & C & C & C & C & IFM & C & C & C & C & IFM & C & C \\
%         % \cline{5-24} 
%         & & & & OVSF Ratio & 1.0 & 1.0 & 1.0 & 1.0 & 1.0 & 0.4 & 0.4 & 1.0 & 0.4 & 0.4 & 0.25 & 0.25 & 1.0 & 0.25 & 0.25 & 0.125 & 0.125 & 1.0 & 0.125 & 0.125 \\
        
%         \cline{3-25} & & \multirow{2}{*}{\texttt{uniform-1.0}} & \multirow{2}{*}{N/A} & Bound & 
%         IFM & C & C & C & C & C & C & OFM & C & C & W & W & IFM & W & W & W & W & IFM & W & W \\
%         % \cline{5-24} 
%         & & & & OVSF Ratio & 
%         1.0 & 1.0 & 1.0 & 1.0 & 1.0 & 1.0 & 1.0 & 1.0 & 1.0 & 1.0 & 1.0 & 1.0 & 1.0 & 1.0 & 1.0 & 1.0 & 1.0 & 1.0 & 1.0 & 1.0
%         \\
        
%         \cline{3-25} & & \multirow{2}{*}{\texttt{hw-aware-autotuning}} & \multirow{2}{*}{67.6} & Bound & 
%         IFM & C & C & C & C & C & C & OFM & C & C & C & C & IFM & C & C & C & C & IFM & C & C \\
%         % \cline{5-24} 
%         & & & & OVSF Ratio & 
%         1.0 & 1.0 & 1.0 & 1.0 & 1.0 & 1.0 & 1.0 & 1.0 & 1.0 & 0.333 & 0.333 & 0.5 & 0.333 & 0.333 & 0.333 & 0.25 & 0.25 & 0.25 & 0.25 & 0.25 \\
        
%         % \cline{2-3} \cline{5-5} \cline{7-7} \cline{9-9}
%         % & \multirow{2}{*}{Iterative} & Crop &  & 94.1 & & 93.6 & & \textbf{93.6} \\
%         % & & Adaptive &  & 94.0 & & 93.8 & & 92.3 \\ 
%         % \midrule \midrule
                                   
%         % \multirow{4}{*}{\parbox{1.9cm}{\centering ResNet18$^\dagger$ 91.3\% 0.27M}} & \multirow{2}{*}{Sequential} & Crop & \multirow{4}{*}{0.48} & 90.8 & \multirow{4}{*}{0.25} & 90.8 & \multirow{4}{*}{0.15} & 88.3 \\
%         %                           & & Adaptive &  & 91.1 & & 91.2 & & 88.5\\ \cline{2-3} \cline{5-5} \cline{7-7} \cline{9-9}
%         %                           & \multirow{2}{*}{Iterative} & Crop &  & 91.1 & & 91.3 & & \textbf{91.4}\\
%         %                           & & Adaptive &  & 91.2 & & 91.4 & & 91.0\\ \midrule \midrule
                                   
%         % \multirow{4}{*}{\parbox{1.9cm}{\centering ResNet34 93.9\% 21.3M }} & \multirow{2}{*}{Sequential} & Crop & \multirow{4}{*}{37.7} & 94.1 &  \multirow{4}{*}{17.6} & 93.9 & \multirow{4}{*}{7.2} & 93.4  \\
%         %                           & & Adaptive &  & 94.3& & 94.0 & & 93.4 \\ \cline{2-3} \cline{5-5} \cline{7-7} \cline{9-9}
%         %                           & \multirow{2}{*}{Iterative} & Crop &  & 94.1& & 93.8 & & \textbf{94.3}\\
%         %                           & & Adaptive &  & 93.8 & & 93.7 & & 93.2\\ \midrule \midrule

%         % \multirow{4}{*}{\parbox{1.9cm}{\centering ResNet34$^\dagger$ 92.1\% 0.46M}} & \multirow{2}{*}{Sequential} & Crop & \multirow{4}{*}{0.82} & 92.3 & \multirow{4}{*}{0.43} & 91.4 & \multirow{4}{*}{0.26} & 89.3 \\
%         %                           & & Adaptive &  & 92.2 & & 91.5 & & 89.2\\ \cline{2-3} \cline{5-5} \cline{7-7} \cline{9-9}
%         %                           & \multirow{2}{*}{Iterative} & Crop &  & 92.3& & 91.8 & & \textbf{92.2}\\
%         %                           & & Adaptive &  & 92.4 & & 91.7 & & 91.7\\

%     \bottomrule
%     \multicolumn{16}{l}{* IFM: Memory-bound w.r.t. input feature maps | OFM: Memory-bound w.r.t. output feature maps | C: Compute-bound | W: Weights Generation-bound.} \\
%     \end{tabular}
%     }
    
%     \label{tab:hw_aware_tune_example}
% \end{table}


\begin{table}[t]
% 	\vspace{-0.2cm}
	\centering
% 	\vspace{-0.2cm}
        \captionsetup{font=small,labelfont=bf}
	\caption{\footnotesize FPGA platforms used for evaluation.}
	\vspace{-0.2cm}
	\resizebox{0.65\linewidth}{!}{%
		\begin{tabular}{l l r l l r}
			\toprule
			\multicolumn{1}{l}{\multirow{1}{*}{Platform}} & Processor %& FPGA 
			& \multicolumn{1}{c}{\multirow{1}{*}{LUTs}} & Flip-Flops & DSPs & \multicolumn{1}{l}{\multirow{1}{*}{BRAM}} \\
			\midrule
			Zynq 7045 & Arm Cortex A9 %& Kintex-7 
			& 218,600 & 437,200 & 900 & 2.40 MB \\
			UltraScale+ ZU7EV & Arm Cortex A53 %& Artix-7 
			& 230,000 & 461,000 & 1,728 & 4.75 MB \\
			\bottomrule
		\end{tabular}%
	}
	\label{tab:fpgas}
    \vspace{0.2cm}
\end{table}

\section{Evaluation}
\label{sec:eval}


\subsection{Experimental Setup}
\label{sec:exp_setup}

In our experiments, we target two widely used FPGA platforms with varied computational capabilities and memory resources (Table~\ref{tab:fpgas}): the Xilinx ZC706 board mounting the mid-tier Zynq Z7045 and the Xilinx ZCU104 board with the more resource-rich Zynq UltraScale+ ZU7EV. The two platforms are based on the Xilinx Zynq-7000 SoC and UltraScale+ MPSoC architectures, respectively, integrating a dual-core Arm Cortex A9 CPU and a quad-core Arm Cortex A53 CPU, respectively, alongside an FPGA fabric on the same chip. Our hardware designs were synthesised and placed-and-routed with Xilinx Vivado HLS and Vivado Design Suite (v2019.2) and run on both boards, with operating clock frequencies of 150~MHz for ZC706 and 200~MHz for ZCU104, respectively. The achieved clock frequency is currently constrained by the technology of the target device and the use of HLS, which relies on the vendor's toolchain and does not allow for low-level optimisations to shorten the critical path.

The corresponding Arm CPU was used to set up the transactions with the off-chip memory, launch the execution of inference and measure the end-to-end performance of each design. \tool provides support for both custom fixed-point and floating-point precisions. For the evaluation, 16-bit fixed-point precision was used, following the practice of the FPGA works we compare with. The available off-chip memory bandwidth was controlled by using a different number of memory ports and amount of word packing, spanning from 1.1 GB/s (1$\times$) to 13.4 GB/s (12$\times$).

\subsubsection{\textbf{Benchmarks}} 
\label{sec:benchmarks}
We evaluate on CNNs of varying depth, workload and memory footprint. Each CNN has been selected to impose a different design challenge. In particular, we target the widely used family of residual networks~\cite{DBLP:journals/corr/HeZRS15} and map variants of different depths to evaluate the scalability of our design. Concretely, we use ResNet18, ResNet34 and ResNet50 on the ImageNet dataset. In addition to image classification, ResNet models are also found as backbone of other tasks including object detection~\cite{detector2019iros}, super-resolution~\cite{mobisr2019mobicom} and semantic segmentation~\cite{deeplab2018tpami}. We also target SqueezeNet1.1~\cite{iandola2016squeezenet}, to assess \tool's efficacy on a highly optimised network for resource-constrained devices.

\subsubsection{\textbf{Basis Selection and \bm{$3$$\times$$3$} extraction}}
\label{sec:basis_selection_and_3x3_extraction}
The proposed on-the-fly formulation using OVSF codes allows for different strategies for: \textit{i)}~selecting which basis to use when $\rho$$<$$1$; and, \textit{ii)}~extracting $3$$\times$$3$ filters from \emph{true} OVSF filters that are restricted to be of shape $K$$\times$$K$ with $K$ being a power-of-two. In ~\ref{sec:OVSF_issues} we presented two solutions for each of aforementioned considerations. %We therefore consider two approaches to tackle the aforementioned considerations. 
%For i), we consider between \textit{1)}~utilising the first $\lfloor \rho \cdot K^2\rceil$ codes and \textit{2)}~iteratively discarding OVSF codes based on their associated scalar $\alpha$ until the target compression ratio $\rho$ is reached. For ii), we consider \textit{1)}~extracting a $3$$\times$$3$ crop from a $4$$\times$$4$ filter and \textit{2)}~learning a mapping to a $3$$\times$$3$ filter by means of an average pooling layer.
Table~\ref{tab:ovsf_cifar10} shows our analysis of the different approaches on CIFAR-10 with ResNet18/34. For the basis selection strategy \textit{i)}, iteratively dropping bases consistently yields higher-accuracy models. For \textit{ii)}, as the models become more compact (\textit{e.g.}~for OVSF50/25), cropping achieves higher accuracy compared to the average pooling approach.  
Thus, we leverage these findings to inform the parametrisation for ImageNet for the rest of the evaluation.

\subsubsection{\textbf{Training Scheme}}
\label{sec:training_scheme}
We have developed \tool's offline flow on top of \textit{PyTorch} (1.5). To derive the OVSF models, we modified the official \textit{PyTorch}-based ResNet by replacing all $3$$\times$$3$ convolutional layers within residual blocks with their OVSF counterparts. In all our experiments, we employed pre-trained ImageNet models from \textit{torchvision} (0.6.0). After a regression stage that transforms standard models into OVSF ones, the models were fine-tuned for 30 epochs using an Adam optimiser~\cite{kingma2014adam} and learning rate decay every 10 epochs. For each given model, we trained two OVSF variants following different distributions of ratios $\rho$ for layers in each of the four residual blocks. First, OVSF50 with ratios=$[1.0,0.5,0.5,0.5]$; and OVSF25 with ratios=$[1.0,0.4,0.25,0.125]$. 
We follow the same procedure and ratios for SqueezeNet's \textit{Fire} modules.

\subsubsection{\textbf{Baselines}}
\label{sec:baselines}

We introduce two highly optimised single computation engines executing: \textit{a)}~the vanilla CNN and \textit{b)}~pruned variants. For \textit{b)}, we use a state-of-the-art method~\cite{Molchanov_2019} which applies channel pruning based on the first-order Taylor approximation contribution of each filter to the model's loss. This process is carried out iteratively until a target compression ratio is reached. We refer to a pruned model that keeps 82\% of the filters as Tay82 and follow the same naming scheme for other ratios.
The baseline architecture comprises the conventional CNN engine design shown in Fig.~\ref{fig:conventional_cnn_engine}, with the weights transferred from the off-chip memory into the {\small $T_P$$\times$$T_C$} weights buffer, if they do not fit on-chip. Both \textit{a)} and \textit{b)} are parametrised with tile sizes {\small $\left<T_R,T_P,T_C \right>$} and roofline modelling~\cite{cnnroofline2015fpga} is used to obtain the highest throughput configuration for the target \mbox{CNN-FPGA pair}. 


\subsection{Performance Comparison}
\label{sec:perf_comparison}
This section analyses the performance of the proposed framework with respect to both our optimised baselines and existing FPGA work.


\begin{table}[t]
    \normalsize
    % \tablefontsize
    \centering
    \captionsetup{font=small,labelfont=bf}
    \captionof{table}{\footnotesize Impact on accuracy for i)~each basis selection strategy and ii)~method to extract $3$$\times$$3$ filters from $4$$\times$$4$ OVSF filters. Models trained on CIFAR-10, with ResNet18/34 adapted for this dataset adn the much smaller variants ($\dagger$) proposed in~\cite{DBLP:journals/corr/HeZRS15}. Performing an iterative drop of bases, as opposed to taking the first $\lfloor \rho \cdot K^2\rceil$, consistently results in better models. As model size is reduced, taking a $3$$\times$$3$ crop from a $4$$\times$$4$ filter performed better than using an average pooling stage.}
    \vspace{-0.1cm}
    \resizebox{0.65\linewidth}{!}{
    \begin{tabular}{c|c|c|cc|cc|cccc}
   
        \toprule
        Model Arch. & Basis & Filters & \multicolumn{2}{c|}{OVSF100} & \multicolumn{2}{c|}{OVSF50} & \multicolumn{2}{c}{OVSF25}\\
        (baseline) & Strategy & to $3$$\times$$3$ & Param. & Acc. & Param. & Acc. & Param. & Acc.\\ \midrule
        
        \multirow{4}{*}{\parbox{1.9cm}{\centering ResNet18 93.2\% 11.2M}} & \multirow{2}{*}{Sequential} & Crop & \multirow{4}{*}{19.7} & 93.9 &  \multirow{4}{*}{9.1} & 93.7 & \multirow{4}{*}{3.6} & 92.9 \\
                                   & & Adaptive &  & 93.7 & & 93.8 & & 93.0 \\ \cline{2-3} \cline{5-5} \cline{7-7} \cline{9-9}
                                   & \multirow{2}{*}{Iterative} & Crop &  & 94.1 & & 93.6 & & \textbf{93.6} \\
                                   & & Adaptive &  & 94.0 & & 93.8 & & 92.3 \\ \midrule \midrule
                                   
        \multirow{4}{*}{\parbox{1.9cm}{\centering ResNet18$^\dagger$ 91.3\% 0.27M}} & \multirow{2}{*}{Sequential} & Crop & \multirow{4}{*}{0.48} & 90.8 & \multirow{4}{*}{0.25} & 90.8 & \multirow{4}{*}{0.15} & 88.3 \\
                                  & & Adaptive &  & 91.1 & & 91.2 & & 88.5\\ \cline{2-3} \cline{5-5} \cline{7-7} \cline{9-9}
                                  & \multirow{2}{*}{Iterative} & Crop &  & 91.1 & & 91.3 & & \textbf{91.4}\\
                                  & & Adaptive &  & 91.2 & & 91.4 & & 91.0\\ \midrule \midrule
                                   
        \multirow{4}{*}{\parbox{1.9cm}{\centering ResNet34 93.9\% 21.3M }} & \multirow{2}{*}{Sequential} & Crop & \multirow{4}{*}{37.7} & 94.1 &  \multirow{4}{*}{17.6} & 93.9 & \multirow{4}{*}{7.2} & 93.4  \\
                                  & & Adaptive &  & 94.3& & 94.0 & & 93.4 \\ \cline{2-3} \cline{5-5} \cline{7-7} \cline{9-9}
                                  & \multirow{2}{*}{Iterative} & Crop &  & 94.1& & 93.8 & & \textbf{94.3}\\
                                  & & Adaptive &  & 93.8 & & 93.7 & & 93.2\\ \midrule \midrule

        \multirow{4}{*}{\parbox{1.9cm}{\centering ResNet34$^\dagger$ 92.1\% 0.46M}} & \multirow{2}{*}{Sequential} & Crop & \multirow{4}{*}{0.82} & 92.3 & \multirow{4}{*}{0.43} & 91.4 & \multirow{4}{*}{0.26} & 89.3 \\
                                  & & Adaptive &  & 92.2 & & 91.5 & & 89.2\\ \cline{2-3} \cline{5-5} \cline{7-7} \cline{9-9}
                                  & \multirow{2}{*}{Iterative} & Crop &  & 92.3& & 91.8 & & \textbf{92.2}\\
                                  & & Adaptive &  & 92.4 & & 91.7 & & 91.7\\
                                   
    \bottomrule
    \end{tabular}
    }
    
    \label{tab:ovsf_cifar10}
\end{table}

\begin{table}[t]
    \tablefontsize
    \centering
    \captionsetup{font=small,labelfont=bf}
    \caption{\footnotesize Accuracy and number of parameters for ResNet34 models on ImageNet following different compression schemes. Performance measured on ZC706 at different memory bandwidths.}
    \vspace{-0.1cm}
    \resizebox{0.65\columnwidth}{!}{
    \begin{tabular}{l c c c c}
        \toprule
        Model & Compression & Params & Accuracy & Performance (inf/sec) \\
        Arch. & Method & (millions) & (\%) & ($1\times$, $2\times$, $4\times$) \\%, $12\times$)\\
        \midrule
        ResNet34 & - & 21.8 & 73.3 & (8.6, 16.8, 28.7) \\ %, 36.0)\\
        \midrule
        ResNet34 & Tay82 & 17.4 & $72.7$ & (10.7, 21.0, 35.6) \\ %, 44.3) \\
        ResNet34 & Tay72 & 15.1 & $71.9$ & (13.3, 25.8, 44.0) \\ %, 54.7) \\
        ResNet34 & Tay56 & 9.4 & $67.8$ & (18.3, 36.3, 63.8) \\ %, 77.6) \\
        ResNet34 & Tay45 & 6.3 & $63.1$ & (21.8, 43.4, 79.8) \\ %, 97.9) \\
        \midrule
        ResNet34 & OVSF50 & 17.2 & $72.8$ & (18.1, 21.8, 31.1) \\ %, 33.3) \\
        ResNet34 & OVSF25 & 7.2 & $71.5$ & (18.4, 27.3, 33.5) \\ %, 33.7) \\
        \midrule
        ResNet34 & Tay82+OVSF50 & 13.2 & $71.1$ & (18.6, 30.0, 37.3) \\ %, 38.3) \\
        ResNet34 & Tay82+OVSF25 & 6.7 & $70.6$ & (18.8, 31.0, 38.9 )\\ %, 40.5) \\
        ResNet34 & Tay72+OVSF50 & 11.9 & $70.3$ & (18.8, 32.0, 40.2) \\ %, 41.3) \\
        ResNet34 & Tay72+OVSF25 & 4.9 & $68.9$ & (18.9, 33.3, 42.0) \\ %, 43.3) \\
        \bottomrule
    \end{tabular}
    }
    \label{tab:accResultsResnet34}
\end{table}

\begin{table}[t]
    \tablefontsize
    \centering
    \captionsetup{font=small,labelfont=bf}
    \caption{\footnotesize Accuracy and number of parameters for ResNet18 models on ImageNet following different compression schemes. Performance measured on ZC706 at different memory bandwidths.}
    \vspace{-0.1cm}
    \resizebox{0.65\columnwidth}{!}{
    \begin{tabular}{l c c c c}
        \toprule
        Model & Compression & Params & Accuracy & Performance (inf/sec) \\
        Arch. & Method & (millions) & (\%) & ($1\times$, $2\times$, $4\times$) \\ %, $12\times$)\\
        \midrule
        ResNet18 & - & 11.7 & 69.8  & (12.0, 23.5, 40.1)\\ %, 54.5) \\
        \midrule
        ResNet18 & Tay88 & 9.1 & $68.8$ & (14.3, 28.0, 46.4)\\ %, 61.9) \\
        ResNet18 & Tay82 & 7.9 & $67.3$ & (14.3, 27.8, 45.4)\\ %, 61.5) \\
        ResNet18 & Tay72 & 6.0 & $64.8$ & (18.2, 35.3, 57.6)\\ %, 77.3) \\
        ResNet18 & Tay56 & 3.7 & $58.3$ & (23.8, 47.3, 82.2)\\ %, 99.5) \\
        \midrule
        ResNet18 & OVSF50 & 9.1 & $69.2$ & (19.4, 33.8, 49.9)\\ %, 51.9) \\
        ResNet18 & OVSF25 & 4.1 & $67.3$ & (19.4, 34.8, 51.0)\\ %, 52.7) \\
        \midrule
        ResNet18 & Tay82+OVSF50 & 6.3 & $66.2$ & (24.5, 43.2, 57.9)\\ %, 58.5) \\
        ResNet18 & Tay82+OVSF25 & 2.8 & $64.4$ & (24.5, 43.6, 59.7)\\ %, 61.2) \\
        \bottomrule
    \end{tabular}
    }
    \label{tab:accResultsResnet18}
    % \vspace{-0.1cm}
\end{table}




\subsubsection{\textbf{Comparison with Optimised Baselines}}
\label{sec:baseline_comparison}
Tables~\ref{tab:accResultsResnet34} and~\ref{tab:accResultsResnet18} show the achieved validation set accuracy and actual performance of each design as measured on ZC706 under varying bandwidth budget. Across bandwidths (1$\times$/2$\times$/4$\times$ where 4$\times$ is the 4.5 GB/s peak measured bandwidth on ZC706), \tool's OVSF50 and OVSF25 designs outperform the faithful baseline by {\small 2.1$\times$/1.3$\times$/1.1$\times$} and {\small 2.1$\times$/1.6$\times$/1.2$\times$} respectively for ResNet34, and by {\small 1.6$\times$/1.6$\times$/1.24$\times$} and {\small 1.4$\times$/1.5$\times$/1.3$\times$} respectively for ResNet18. As bandwidth availability increases, the baseline becomes less memory-bound and the performance gap closes. 
Table~\ref{tab:accResultsSqueezenet} shows the comparison of \tool with the faithful baseline for SqueezeNet on ZU7EV with peak measured bandwidth of 13.4~GB/s (12$\times$). Both OVSF50 and OVSF25 designs yield increasing throughput gains as the bandwidth becomes more restricted, with OVSF25 sustaining over 57\% speedup for up to 4$\times$ bandwidth.
Under 1$\times$ bandwidth, OVSF25 offers minimal additional gains. This is because, below a compression ratio, even though the memory needs are further reduced, activations begin to dominate I/O, and hence further weights reduction does not provide significant benefits. Activations compression techniques~\cite{eyeriss2017jssc,scnn2017isca} can be orthogonally combined to obtain further gains.

\rev{Based on our evaluation using SqueezeNet, we observe that computation can take place fast due to its lighter workload. As such, the attainable performance depends on how rapidly we can feed the CNN Engine with new inputs. Specifically, for the 4$\times$ bandwidth configuration, all layers of SqueezeNet are memory-bound. On the other hand, at 12$\times$ bandwidth, 88\% of the layers become compute-bound. As such, when there is restricted or medium availability of memory bandwidth, \tool significantly improves performance through our weights generation approach, with 78\%, 74\% and 55\% higher throughput for the 1$\times$, 2$\times$ and 4$\times$ bandwidth configurations, respectively (Table~\ref{tab:accResultsSqueezenet}). This improvement gradually decreases as the available bandwidth increases, with 15\% gain at 12$\times$ bandwidth.}


\begin{table}[t]
    \vspace{-0.2cm}
    \small
    \centering
    \captionsetup{font=small,labelfont=bf}
    \caption{\footnotesize Comparing \tool with faithful baseline on SqueezeNet on ImageNet. Performance measured on the UltraScale+ ZCU104 platform at different memory bandwidths.}
    \vspace{-0.1cm}
    \resizebox{0.65\linewidth}{!}{
    \begin{tabular}{l c c c c}
        \toprule
        Model & Compression & Params & Accuracy & Performance (inf/sec) \\
        Arch. & Method & (millions) & (\%) & ($1\times$, $2\times$, $4\times$, $12\times$)\\
        \midrule
        SqueezeNet & - & 1.24 & 58.2 & (72.9, 145.2, 290.4, 687.4) \\
        % SqueezeNet & - & 1.24 & 58.2 & (54.7, 108.9, 217.8, 515.6) \\
        \midrule
        SqueezeNet & OVSF50 & 1.07 & 57.6 & (129.8, 252.9, 452.1, 792.1) \\
        SqueezeNet & OVSF25 & 0.86 & 57.1 & (129.8, 252.9, 456.8, 800.6) \\
        % SqueezeNet & OVSF50 & 1.07 & 57.6 & (97.4, 189.7, 339.1, 594.1) \\
        % SqueezeNet & OVSF25 & 0.86 & 57.1 & (97.4, 189.7, 342.6, 600.5) \\
        \bottomrule
    \end{tabular}
    }
    \vspace{0.2cm}
    \label{tab:accResultsSqueezenet}
\end{table}

\textbf{Comparison with Pruned Baselines.} Compared to the pruned baselines, \tool's OVSF models are more resilient at high compression ratios while resulting in similar accuracy at lower compression ratios. Informed by the analysis in Table~\ref{tab:ovsf_cifar10}, OVSF models are trainined to extract a $3$$\times$$3$ from a $4$$\times$$4$ and, to iteratively discard OVSF basis until the target compression ratio $\rho$ for each layer is reached, as first discussed in Sec.~\ref{sec:basis_selection_and_3x3_extraction}. In terms of throughput, \tool delivers faster processing at more constrained bandwidths. Concretely, ResNet34-OVSF50 is 80\% faster than Tay82 at 1$\times$ bandwidth, with less than 1 percentage point (pp) accuracy drop. Despite being almost identical in terms of model size and accuracy, Tay82's approach, which prioritises the pruning of layers with the least accuracy impact, leads to the pruning of mostly compute-bound layers when targeting ResNet34. On the other hand, ResNet34-OVSF50 compresses more effectively memory-bound layers, leading to significantly higher throughput at low bandwidths.
A similar pattern is observed for ResNet18. At higher compression ratios, ResNet34-OVSF25 yields 3.7 pp higher accuracy than Tay56, despite using 25\% fewer parameters.

To explore the benefits of combining \tool's OVSF execution scheme with channel pruning, we derive, train and map on \tool Tay-OVSF models. 
This results in competitive lightweight models that are not attainable through pruning alone. For instance, ResNet18 with Tay82+OVSF25 is 25\% smaller than ResNet18-Tay56 and achieves 6.1 pp higher accuracy, while achieving 34.6\% and 23.5\% higher throughput over ResNet18-Tay72 with less than 0.5 pp accuracy drop. 


\begin{table*}[t]
	\centering
	\captionsetup{font=small,labelfont=bf}
    \caption{\footnotesize Comparison with prior FPGA work on ResNet18 (4.03 GOps), ResNet34 (7.40 GOps) \& SqueezeNet (0.78 GOps).}
	\vspace{-0.2cm}
	\resizebox{1.0\linewidth}{!}{
	\setlength\tabcolsep{10pt} % default value: 6pt
		\begin{threeparttable}
			%\vspace{-0.1cm}
			\small
			%	\footnotesize
			%	\normalsize
			%	\huge
			%	\large
			%\resizebox{0.5\textwidth}{!}{
			%	\resizebox{\linewidth}{!}{
			%	\resizebox{0.5\textwidth}{!}{
			\begin{tabular}{@{}l l l| l l| l l l l@{}}
				\toprule
				Comparison with: 
				& \multicolumn{2}{l|}{\textbf{Compiler-based Design}}
				& \multicolumn{2}{l|}{\textbf{Compression-based Design}} 
				& \multicolumn{1}{l}{\textbf{Light-CNN-tailored Design}}
				& \multicolumn{2}{l}{\textbf{Multi-Accelerator Designs}}
				\\
				& ResNet18~\cite{snowflake2017compiling} 
				& \begin{tabular}[l]{@{}l@{}} \tool: \\ ResNet18* \end{tabular}
				& \begin{tabular}[l]{@{}l@{}} Sparse ResNet34~\cite{sparsecnnaccel2019fccm} \\ using Deep Compression  \end{tabular}
				& \begin{tabular}[l]{@{}l@{}} \tool: \\ ResNet34* \end{tabular} 
				& SqueezeNet~\cite{lightopu2020fpga} 
				& \multicolumn{2}{l}{SqueezeNet~\cite{maximising2017isca}} 
				& \begin{tabular}[l]{@{}l@{}} \tool:\\ SqueezeNet* \end{tabular}
				\\
				\cmidrule{7-8}
				\midrule
				FPGA  
				& Z7045 
				& Z7045 
				& Z7045 
				& Z7045
				& K325T
				% & \begin{tabular}[l]{@{}l@{}} Kintex-7 \\ 325T \end{tabular}
				& V485T 
				% & \begin{tabular}[l]{@{}l@{}} Virtex-7 \\ 485T \end{tabular}
				& V690T
				% & \begin{tabular}[l]{@{}l@{}} Virtex-7 \\ 690T \end{tabular} 
				& ZU7EV 
				\\
				Clock (MHz) & 250 & 150 & 166	& 150 & 200 & 170 & 170 & 200 \\
				Precision & 16b fixed & 16b fixed & 16b fixed & 16b fixed & 8b fixed & 16b fixed & 16b fixed & 16b fixed \\
				DSPs$^\dagger$ & 900 & 900 & 900 & 900 & 840 & 2800 & 3600 & 1728 \\
				Logic Capacity & 218.6 kLUTs & 218.6 kLUTs & 218.6 kLUTs & 218.6 kLUTs & 203.8 kLUTs & 303.6 kLUTs & 433.2 kLUTs & 230.0 kLUTs \\
				%	Fixed-point DSPs* & 220 & 900 & 220 & 900 \\
				Block RAM & 2.40 MB & 2.40 MB & 2.40 MB & 2.40 MB & 1.95 MB & 4.52 MB & 6.46 MB & 4.75 MB \\
				{\color{black}DSP Util.$^\dagger$} & 28.4\% & 100\% & 56.8\% & 100\% & 83.8\% & 80\% & 80\% & 100\% \\
				
				\begin{tabular}[t]{@{}l@{}} 
					inf/s
				\end{tabular} 
				& 21.38
				& 49.90
				& 27.84
				& 31.1
				& 420.90
				& 913.40
				& 1173.00
				& 792.20
				\\
				\begin{tabular}[t]{@{}l@{}} 
					inf/s/DSP$^\dagger$
				\end{tabular} 
				& 0.0237
				& 0.0576
				& 0.0309
				& 0.0369
				& 0.2505
				& 0.3260
				& 0.3258
				& 0.4584
				\\
				\begin{tabular}[t]{@{}l@{}} 
					inf/s/Logic
				\end{tabular} 
				& 0.0978
				& 0.2282 % 0.2372
				& 0.1273
				& 0.1422 % 0.1521
				& 2.0652
				& 3.0085
				& 2.7077
				& 3.444
				\\
			
				\bottomrule
				
			\end{tabular} 
			\begin{tablenotes}
				\small
				\item * using OVSF50, ** batch size = 1, $\dagger$ 18$\times$18, 19$\times$18 and 25$\times$18 DSP configurations, inf/s/DSP is adjusted based on precision for fair comparison (multiplied by 0.5 for 8b).
			\end{tablenotes}
			
		\end{threeparttable}
	}
	\vspace{-0.25cm}
	\label{tab:comparison_table}
\end{table*}

\begin{table*}[t]
	\centering
	\captionsetup{font=small,labelfont=bf}
    \caption{\footnotesize Comparison with prior FPGA work on ResNet50 (8.41 GOps).}
	\vspace{-0.2cm}
	\resizebox{1.0\linewidth}{!}{
	\setlength\tabcolsep{2pt} % default value: 6pt
		\begin{threeparttable}
			%\vspace{-0.1cm}
			\small
			%	\footnotesize
			%	\normalsize
			%	\huge
			%	\large
			%\resizebox{0.5\textwidth}{!}{
			%	\resizebox{\linewidth}{!}{
			%	\resizebox{0.5\textwidth}{!}{
			\begin{tabular}{l l l | l l l l l l l l l l}
				%\hline
				\toprule
				Comparison with: 
				& \multicolumn{4}{l}{\textbf{Compiler-based Designs}}
				& \multicolumn{2}{l}{\textbf{CNN-to-FPGA Toolflows}}
				& \multicolumn{1}{l}{\textbf{CNN-tailored Designs}} 
				& \multicolumn{1}{l}{\textbf{Overlay Designs}}
				& \multicolumn{1}{l}{\textbf{Cloud-based Designs}}
				& \begin{tabular}[l]{@{}l@{}} \textbf{Interconnect-aware} \\ \textbf{Designs} \end{tabular}
				& \begin{tabular}[l]{@{}l@{}} \textbf{Full-stack-optimised} \\ \textbf{Designs} \end{tabular}
				\\
				
				& Snowflake~\cite{snowflake2017iscas} 
				& \begin{tabular}[l]{@{}l@{}} \tool: \\ ResNet50* \end{tabular}
				& xDNN~\cite{xdnn2020xilinx}
				& DNNVM~\cite{dnnvm2019tcad}
				& \multicolumn{2}{l}{ALAMO~\cite{alamo2020tcad}}
				
				& ResNetAccel~\cite{residaccel2017iscas} 
				& FTDL~\cite{ftdl2020dac}
				& Cloud-DNN~\cite{clouddnn2019fpga}
				& \begin{tabular}[l]{@{}l@{}} Scaling the \\ Cascades~\cite{scaling_the_cascades2019fpl}
				\end{tabular}
				& Full-Stack~\cite{fullstack2021tnnls}
				& \begin{tabular}[l]{@{}l@{}} \tool:\\ ResNet50* \end{tabular}
				\\
				\cmidrule{6-7}
				\midrule
				FPGA  
				& Z7045 
				& Z7045 
				& VU9P 
				& ZU9
				& Arria 10 GX1150
				% & \begin{tabular}[l]{@{}l@{}} Kintex-7 \\ 325T \end{tabular}
				& Stratix 10 GX2800
				& Arria 10 GX1150 
				% & \begin{tabular}[l]{@{}l@{}} Virtex-7 \\ 485T \end{tabular}
				& VU125
				% & \begin{tabular}[l]{@{}l@{}} Virtex-7 \\ 690T \end{tabular} 
                & VU9P
                & VU37P
                & Arria 10 GX1150
                & ZU7EV 
				\\
				Clock (MHz) & 250 & 150 & 500 & 500 & 240 & 150 & 300 & 650 & 125 & 650 & 200 & 200 \\
				Precision & 16b fixed & 16b fixed & 8b fixed & 8b fixed & 16b fixed & 16b fixed & 16b fixed & 16b fixed & 16b fixed & 8b fixed & 8b fixed & 16b fixed \\
				DSPs$^\dagger$ & 900 & 900 & 6840 & 2520 & 3036 & 11,520 & 3036 & 1200 & 3036 & 9024 & 3036 & 1728 \\
				Logic Capacity & 218.6 kLUTs & 218.6 kLUTs & 1182.0 kLUTs & 274.0 kLUTs & 427.2 kALMs & 933.0 kALMs & 427.2 kALMs & 716.0 kLUTs & 1182 kLUTs & 1304 kLUTs & 427.2 kALMs & 230.0 kLUTs \\
				%	Fixed-point DSPs* & 220 & 900 & 220 & 900 \\
				Block RAM & 2.40 MB & 2.40 MB & 9.48 MB & 4.01 MB & 6.60 MB & 28.62 MB & 6.60 MB & 11.075 MB & 43.23 MB & 42.61 MB & 6.60 MB & 4.75 MB \\
				{\color{black}DSP Util.$^\dagger$} & 28.4\% & 100\% & 100\% & 83.8\% & 80\% & 80\% & 56.8\% & 100\% & 80.2\% & 95\% & 97\% & 100\% \\
				
				\begin{tabular}[t]{@{}l@{}} 
					inf/s
				\end{tabular} 
				& 17.7
				& 28.18
				& 153.57
				& 80.95
				& 71.38
				& 77.55
				& 33.93
				& 151.22
				& 71.94
				& 766
				& 197.23
				& 71.71
				\\
				\begin{tabular}[t]{@{}l@{}} 
					inf/s/DSP$^\dagger$
				\end{tabular} 
				& 0.0196
				& 0.0313
				& 0.0112
				& 0.016
				& 0.0235
				& 0.0067
				& 0.0111
				& 0.1260
				& 0.0105
				& 0.0424
				& 0.0324
				& 0.0415
				\\
				\begin{tabular}[t]{@{}l@{}} 
					inf/s/Logic
				\end{tabular} 
				& 0.0809
				& 0.1289
				& 0.0649
				& 0.1477 
				& 0.1671
				& 0.0831
				& 0.0794
				& 0.2112
				& 0.0608
				& 0.5874
				& 0.4616
				& 0.3117
				\\

				\bottomrule
				
			\end{tabular} 
			\begin{tablenotes}
				\normalsize
				\item * using OVSF50, ** batch size = 1, $\dagger$ 18$\times$18, 19$\times$18 and 25$\times$18 DSP configurations, inf/s/DSP is adjusted based on precision for fair comparison (multiplied by 0.5 for 8b).
			\end{tablenotes}
			
		\end{threeparttable}
	}
	\vspace{0.2cm}
	\label{tab:resnet50_comparison_table}
\end{table*}


\subsubsection{\textbf{Comparison with Existing FPGA Designs}}
\label{sec:fpga_comparison}

To assess the performance of the proposed framework with respect to existing FPGA work, we perform a number of comparison with a broad range of state-of-the-art works that optimise CNN inference from different aspects. These span accelerators that aggressively apply compiler techniques~\cite{snowflake2017compiling,snowflake2017iscas,xdnn2020xilinx,dnnvm2019tcad}, the highest performing FPGA-based accelerators for sparse~\cite{sparsecnnaccel2019fccm} and lightweight CNNs~\cite{lightopu2020fpga}, a multi-accelerator design that addresses PE underutilisation for SqueezeNet~\cite{maximising2017isca}, a state-of-the-art CNN-to-FPGA toolflow~\cite{alamo2020tcad}, an optimised overlay architecture~\cite{ftdl2020dac}, a highly customised accelerator for residual networks~\cite{residaccel2017iscas}, a cloud-optimised framework~\cite{clouddnn2019fpga}, a CNN accelerator designed in an interconnect-aware manner~\cite{scaling_the_cascades2019fpl} and an accelerator that applies full-stack optimisations~\cite{fullstack2021tnnls}. 

Table~\ref{tab:comparison_table} lists the performance results for ResNet18/34 and SqueezeNet. On Z7045, \tool achieves 2.33$\times$ and 1.12$\times$ higher throughput than \cite{snowflake2017compiling} and \cite{sparsecnnaccel2019fccm}, respectively. For SqueezeNet, our design delivers 1.83$\times$ and 1.67$\times$ higher performance density in inf/s/DSP and inf/s/Logic than Light-OPU~\cite{lightopu2020fpga}. 
Compared to the multi-accelerator design~\cite{maximising2017isca} that also addresses the PE underutilisation, \tool yields 1.40$\times$ higher inf/s/DSP and 1.14$\times$-1.27$\times$ higher inf/s/Logic despite having the same (V48T-based design~\cite{maximising2017isca}) or 36\% lower (V690T-based design~\cite{maximising2017isca}) on-chip memory budget.

The original ResNet50 reaches 76.15\% accuracy with a model size of 25.56M parameters. With \tool's ResNet50-OVSF50 variant improves accuracy to 76.23\% while having 10\% fewer parameters (22.84M).
Table~\ref{tab:resnet50_comparison_table} presents the measured performance results for ResNet50. On Z7045, \tool outperforms Snowflake by 1.59$\times$ in inf/s. Compared with designs on larger devices, our design achieves higher performance density (inf/s/DSP) by 3.69$\times$, 2.58$\times$, 1.76$\times$-6.16$\times$, 3.17$\times$ and 3.94$\times$ over xDNN, DNNVM, ALAMO, ResNetAccel and Cloud-DNN. The overlay-based FTDL reaches higher inf/s/DSP and 1.47$\times$ lower inf/s/Logic, but targets a platform with 2.33$\times$ larger on-chip memory and 2$\times$ higher bandwidth, both of which substantially reduce the off-chip memory accesses and the associated latency. Similarly, compared to the interconnect-aware design of \cite{scaling_the_cascades2019fpl}, \tool reaches 97.8\% of its inf/s/DSP, despite using a platform with 8.9$\times$ smaller on-chip memory. Finally, \tool outperforms the full-stack-optimised accelerator of \cite{fullstack2021tnnls} by 1.28$\times$ in inf/s/DSP.

\textbf{Discussion.}
Based on the presented evaluation, \tool consistently outperforms a wide range of FPGA-based accelerator designs, in spite of their diverse designs.
As such, our framework delivers an average throughput gain of 2.23$\times$ (2.05$\times$ geo. mean) over designs that aggressively apply compiler optimisations on fixed accelerators~\cite{snowflake2017compiling,snowflake2017iscas,xdnn2020xilinx,dnnvm2019tcad} and, at the same time, achieves an average inf/s/DSP gain of 2.5$\times$ (2.41$\times$ geo. mean) over highly customised CNN-tailored designs~\cite{lightopu2020fpga, residaccel2017iscas} and 3.94$\times$ over the cloud-optimised mapping of Cloud-DNN.
A notable comparison is with the sparse CNN accelerator for ResNet34 presented in~\cite{sparsecnnaccel2019fccm}, with \tool achieving 12\% throughput gain. It should be noted that the sparse CNN accelerator applies Deep Compression~\cite{deepcompression2015iclr} to sparsify the target CNN, employs a specialised dataflow and modifies the underlying PEs in order to extract high performance. 
In contrast, \tool improves the performance of CNN engines while affecting neither the selected dataflow nor the internal design of the PEs, and still delivers 12\% higher throughput than the sparse CNN accelerator. 

\begin{table}[b]
\rev{
	\centering
    \vspace{0.3cm}
    \captionsetup{font=small,labelfont=bf}
	\caption{\rev{\footnotesize Resource usage breakdown of \tool's designs.}}
    \vspace{-0.2cm}
	\resizebox{0.65\linewidth}{!}{%
		\begin{tabular}{l l l l l l}
			\toprule
			\multicolumn{1}{l}{\multirow{1}{*}{\textbf{Design Config.}}} & \textbf{Platform} & \textbf{Resource Type} & \multicolumn{1}{c}{\multirow{1}{*}{\textbf{\texttt{CNN-WGen}}}} & \textbf{CNN Engine} \\
			\midrule
			\multirow{2}{*}{ResNet18-OVSF50} & \multirow{2}{*}{ZC706} 
			& DSPs & 7.5\% & 92.5\% \\
            & & LUTs & 1\% & 74\% \\
	          \midrule		
            \multirow{2}{*}{ResNet34-OVSF50} & \multirow{2}{*}{ZC706}  
			& DSPs & 11.3\% & 88.7\% \\
            & & LUTs & 3\% & 75\% \\
            \midrule
            \multirow{2}{*}{ResNet50-OVSF50} & \multirow{2}{*}{ZC706}  
			& DSPs & 11.1\% & 88.9\% \\
            & & LUTs & 3\% & 75\% \\
			\bottomrule
		\end{tabular}%
	}
	\label{tab:rsc_usage}
    % \vspace{0.2cm}
    }
\end{table}

\subsubsection{\textbf{Resource Usage}}
\label{sec:rsc_usage}
We select \tool and baseline designs with up to 1-pp accuracy drop and compare their post place-and-route resource usage on Z7045, reported in [DSPs, BRAM, LUTs] tuples for 4$\times$ bandwidth.
For ResNet34, the faithful baseline consumes {\small [99\%,83\%,77\%]}, Tay82 {\small [99\%,79\%,77\%]}, OVSF50 {\small [100\%,81\%,78\%]} and Tay82+OVSF50 {\small [100\%,87\%,81\%]}. For ResNet18, the faithful baseline {\small [78\%,99\%,70\%]}, Tay88 {\small [78\%,99\%,69\%]}, OVSF50 {\small [100\%,87\%,75\%]} and Tay82+OVSF50 {\small [100\%,83\%,80\%]}. For ResNet50, OVSF50 on ZU7EV consumes {\small [100\%,87\%,78\%]}. Finally, the input selective PE mechanism adds a minimal LUTs overhead of less than 7\%. \rev{We further report the breakdown of resource consumption between \texttt{CNN-WGen} and the CNN Engine in Table~\ref{tab:rsc_usage}. We observe that, using our performance model, the DSE stage is able to balance the allocation of DSPs between the two modules. Moreover, the LUT overhead of the weights generator is minimal compared to the CNN Engine, providing a beneficial trade-off.}



% Figure environment removed

\subsection{Sensitivity to Off-Chip Memory Bandwidth}
\label{sec:mem_bandwidth_eval}
Fig.~\ref{fig:bw_sensitivity} shows the impact of varying off-chip memory bandwidth over performance on the two target platforms. The figure compares the speedup of \tool and the Tay82 baseline over the vanilla baseline when varying the external memory bandwidth from 1$\times$ to 12$\times$. The bandwidth's impact is most prominent on the larger ZU7EV, where the performance gains are sustained higher across 1$\times$-4$\times$. In the case of the mid-tier Z7045, we observe a sharper drop in the speedup as the bandwidth increases. This is due to the more limited computational resources of Z7045, which makes most CNN layers compute-bound. In contrast, the abundance of computational resources on ZU7EV makes the CNN layers more memory-bound. For instance, at 4$\times$ bandwidth (4.5 GB/s), the vanilla ResNet18 baseline yields DSP utilisation of 71\% on the compute-bound Z7045 and 53\% on the more memory-bound ZU7EV. In this case, \tool significantly improves both cases by mapping ResNet18-OVSF25 with 89\% and 71\% DSP utilisation.
As a result, \tool sustains its gains across a wider range of bandwidths and outperforms Tay82, until the bandwidth-abundant case (12$\times$) where computational resources become the critical factor. In this case, Tay82's lower number of operations due to pruning leads to higher performance.

Across the designs, the input selective PEs contribute an additional speedup of up to 20\%, with varying gains depending on the CNN-FPGA pair and the available bandwidth. For ResNet34-OVSF25 on ZU7EV, disabling this mechanism leads to 0/13.9/3.3/5.9\% lower throughput for the four bandwidths, with a similar pattern observed for the rest of the CNNs. Our input selective PEs effectively improve the performance of suboptimally mapped layers in compute-bound settings, whereas no gain is obtained for the most bandwidth-constrained case (1$\times$) where the designs are severely memory-bound, limiting further improvements through higher PE utilisation.


\subsection{Impact of Input Selective PEs}
\label{sec:input_sel_pes_eval}
Here, we evaluate the impact of input selective PEs on the achieved performance.
This is investigated by implementing \tool's selected hardware design for each of the benchmark CNNs with and without the input selective PEs and comparing the achieved performance, measured on the two target FPGA platforms. When the input selective PEs are omitted, we call the designs \textit{ablated}. 
Table~\ref{tab:input_sel_pes_eval} presents the achieved performance gains between the two designs.


\begin{table}[t]
    % \vspace{-0.2cm}
    \small
    \centering
    \captionsetup{font=small,labelfont=bf}
    \caption{Ablation study of input selective PEs.}
    \vspace{-0.1cm}
    \resizebox{0.5\linewidth}{!}{
    \begin{tabular}{l l l r l c}
        \toprule
        \multicolumn{1}{l}{Model} & & FPGA & \multicolumn{2}{c}{Input Selective PEs} & Performance \\
        \multicolumn{1}{l}{Arch.} & & Platform & \multicolumn{1}{c|}{without} & with & Gain \\
        \midrule
        
        & OVSF50 & Z7045 & 49.9 inf/s & 49.9 inf/s & 1.00$\times$ \\
        \multicolumn{1}{l}{\multirow{1}{*}{ResNet18}} & OVSF25 & Z7045 & 50.4 inf/s & 51.0 inf/s & 1.01$\times$ \\
        \cmidrule{2-6}
        & OVSF50 & ZU7EV & 124.1 inf/s & 124.1 inf/s & 1.00$\times$ \\
        & OVSF25 & ZU7EV & 135.2 inf/s & 135.2 inf/s & 1.00$\times$ \\
        
        \midrule
        
        & OVSF50 & Z7045 & 25.4 inf/s & 31.1 inf/s & 1.22$\times$ \\
        \multicolumn{1}{l}{\multirow{1}{*}{ResNet34}} & OVSF25 & Z7045 & 33.5 inf/s & 33.3 inf/s & \phantom{1}0.6\% \\
        \cmidrule{2-6}
        & OVSF50 & ZU7EV & 81.1 inf/s & 81.1 inf/s & 1.00$\times$ \\
        & OVSF25 & ZU7EV & 72.4 inf/s & 88.0 inf/s & 1.21$\times$ \\
        
        \midrule
        
        & OVSF50 & Z7045 & 23.7 inf/s & 27.0 inf/s & 1.14$\times$ \\
        \multicolumn{1}{l}{\multirow{1}{*}{ResNet50}} & OVSF25 & Z7045 & 23.7 inf/s & 28.1 inf/s & 1.18$\times$ \\
        \cmidrule{2-6}
        & OVSF50 & ZU7EV & 63.1 inf/s & 71.7 inf/s & 1.13$\times$ \\
        & OVSF25 & ZU7EV & 68.5 inf/s & 77.8 inf/s & 1.13$\times$ \\
        
        \midrule
        
        % & OVSF50 & Z7045 & inf/s &  inf/s & $\times$ \\
        % \multicolumn{1}{c}{\multirow{1}{*}{SqueezeNet}} & OVSF25 & Z7045 & inf/s &  inf/s & $\times$ \\
        \multicolumn{1}{l}{\multirow{1}{*}{SqueezeNet}} & OVSF50 & ZU7EV & 724.2 inf/s & 792.2 inf/s & 1.09$\times$ \\
        & OVSF25 & ZU7EV & 731.4 inf/s & 800.6 inf/s & 1.09$\times$ \\
        \midrule
        \midrule
        Average & & & & & 1.12$\times$ \\
        Geo. Mean & & & & & 1.11$\times$ \\
        \bottomrule
    \end{tabular}
    }
    \vspace{0.2cm}
    \label{tab:input_sel_pes_eval}
\end{table}

The PE-enhancing mechanism contributes varying throughput gains, yielding up to 22\% faster inference and an average improvement of 12\% (11\% geo. mean). 
For ResNet18, the ablated designs already sustain high DSP utilisation with ResNet18-OVSF50 and -OVSF25, reaching 90\% and 86.5\% of the theoretical peak performance of Z7045 and ZU7EV, respectively.
On the other hand, the ablated ResNet34-OVSF50 design on Z7045 achieves only 69.6\% of the theoretical peak throughput. Similarly, the ablated ResNet34-OVSF25 design on ZU7EV achieves 77.5\% of the theoretical performance. In both cases, the input selective PEs are able to substantially increase the DSP utilisation, with the enhanced CNN engines achieving 85.1\% and 94.2\% of the peak performance, respectively.

A similar effect is observed for ResNet50 and SqueezeNet. The ablated designs yield 73.8\% of the theoretical peak performance for both OVSF50 and OVSF25 on Z7045, and 76.7\% and 83.3\% for OVSF50 and OVSF25, respectively, on ZU7EV. In this case, our input selective PEs are able to improve the DSP utilisation, achieving 84.1\% and 87.4\% of the peak throughput on Z7045 for OVSF50 and OVSF25, respectively, and 87.2\% and 94.7\% for OVSF50 and OVSF25, respectively, on ZU7EV. Finally, for SqueezeNet, the input selective PEs improve the measured throughput from 73.2\% to 80.1\% of the peak performance for OVSF50 and from 73.9\% to 80.9\% for OVSF25 on ZU7EV. As such, enhancing a CNN engine's PEs with our proposed input selectivity technique alleviates the resource underutilisation due to the diverse layer shapes within a CNN. In the cases where our technique is estimated to provide minimal gains (\textit{i.e.} $<$5\%) and its usage is not justified, \tool opts for omitting it to save LUT resources.


% 691.2
% 518.4
% 270


% Figure environment removed


\subsection{Hardware-Aware vs. Manual Tuning of OVSF Ratios}
\label{sec:eval_hw_aware_ratios}

Next, we evaluate the effectiveness of our hardware-aware tuning of OVSF ratios in yielding designs with improved accuracy-performance trade-off. To this end, we compare against two ratio selection methods: \textit{i)}~\texttt{uniform-$\rho$} which uses the same ratio $\rho$ across all layers, with the exception of the first CONV layer. This baseline represents a brute-force approach of setting the OVSF ratios; and \textit{ii)}~\texttt{manual-OVSF50} and \texttt{manual-OVSF25} which use the manually selected ratios detailed in Section~\ref{sec:training_scheme} to achieve 50\% and 75\% reduction in model size from the original model. This baseline constitutes an optimised hand-engineered method. We perform the comparison by implementing ResNet18 and ResNet34 using both the hardware-aware and the baseline flows for different bandwidth availability and comparing the achieved performance, measured on Z7045.

Figure~\ref{fig:hw_autotune_resnet34_low_bw} shows the achieved accuracy and execution time measured on the target FPGA and depicts how our method, denoted by \texttt{hw-aware autotuning}, yields Pareto-optimal designs that were previously unattainable. For ResNet34, our method sustains the same performance as the fast OVSF25 design across all memory bandwidths. However, it additionally improves OVSF25's accuracy by 0.8pp, thus outputting design that are within 1pp of the original model's accuracy (72.3\% for all three bandwidths vs 73.3\% for the vanilla ResNet34). At the same time, it is consistently faster than the coarse \texttt{uniform-0.5}. 
We obtain similar results for ResNet18, with the same processing speed as OVSF25 and accuracy gains in the range of 0.3pp-1.2pp (0.86pp average gain across bandwidths) over OVSF25. Across all cases, the \texttt{uniform-$\rho$} baselines were either unnecessarily slow (\texttt{uniform-0.5}) or low in accuracy (\texttt{uniform-0.25}), further advocating for a principled method of selecting the OVSF ratios.

By exploiting the bounding factor of each layer, our hardware-aware scheme selectively allows for a longer weights generation stage without affecting the processing speed. As such, we can obtain a better approximation of the weights and sustain high throughput. As shown through our experiments, the hardware-aware methodology yielded competitive designs, performing either better or in par even against highly optimised hand-tuned configurations (OVSF50 and OVSF25).
\stelios{Added this.}


\subsection{Comparison with Embedded GPU}
\label{sec:gpu}

With the majority of CNNs deployed for inference on embedded and mobile devices, our evaluation focuses on the embedded space. In power-constrained applications, the main metrics of interest comprise: \textit{1)}~the absolute power consumption and \textit{2)}~the energy efficiency in performance-per-watt. In this respect, we investigate the energy efficiency of \tool in relation to the widely used high-performance NVIDIA Jetson TX2 platform. In all cases, for \tool we use the OVSF50 variant with less than 1-pp accuracy drop.

For the performance evaluation on TX2, we use NVIDIA TensorRT as supplied by the JetPack~4.1.1 package. TensorRT is run with the NVIDIA cuDNN library and 16-bit half-precision floating-point arithmetic (FP16), which enables the highly optimised execution of layers. Across all platforms, each CNN is run 100 times to obtain the average throughput. Furthermore, power measurements for the GPU and FPGAs are obtained via a power monitor on the corresponding board. In all cases, we subtract the average idle power from the measurement to obtain the power due to benchmark execution. The idle power of the FPGA platforms is measured at the board level with no design programmed in the FPGA fabric, so that the clock tree power and the power leakage of the chip are also included in the run-time power due to benchmark execution.
Across all experiments, we used a batch size of 1, as is typical in mobile and embedded settings.

% Figure environment removed

Tegra X2 mounts a 256-core GPU with native support for FP16 arithmetic which supports a range of operating modes with different clock frequencies and power consumption. To perform a fair comparison with respect to energy efficiency in terms of performance-per-watt, we configure the GPU with the maximum energy efficiency mode (Max-Q) which sets the frequency of the GPU at 850~MHz and configures all components of TX2 to achieve the best power-throughput trade-off. Fig.~\ref{fig:gpu_comparison} presents the conducted comparison. \tool achieves an energy efficiency improvement over TX2 of up to 5.32$\times$ in inf/s/W with an average of 2.57$\times$ (2.31$\times$ geo. mean) across the benchmarks. As a result, \tool consistently demonstrates significant gains in performance-per-watt across the benchmarks over highly optimised embedded GPU implementations.


\section{conclusion}

Our work embeds the social scientific construct of anti-democratic attitudes into a social media AI objective function. We demonstrate that the survey instruments from prior work on this construct can be adapted into prompts for a large language model (LLM), producing a \textit{democratic attitude model}. Through a series of three studies, we found that social media feeds that integrate this democratic attitude model can significantly reduce partisan animosity without compromising user engagement levels. This \textit{societal objective function} method presents a novel strategy for translating social science theory to algorithmic objectives, which opens up new possibilities to encode societal values in social media AIs.

\bibliographystyle{ACM-Reference-Format}
\bibliography{references}

\end{document}
\endinput
%%
%% End of file `sample-acmsmall.tex'.
