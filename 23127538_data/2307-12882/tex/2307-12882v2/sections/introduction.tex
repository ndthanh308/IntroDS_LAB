\section{Introduction}
\label{sec:intro}

Food waste constitutes a significant environmental issue, contributing substantially to environmental degradation and hindering the efficient use of resources \cite{Stenmarck2016}. This issue is particularly pronounced within the context of college campuses. For instance, \universityname has reported an average monthly accumulation of over 40 tons of food waste generated on campus.

% Figure environment removed
Various interventions have been developed and implemented to address this issue, ranging from information-based strategies to technological solutions and policy changes \cite{reynolds2019consumption}. For example, students at Baldwin Wallace University developed ``Campus Plate'', a platform that facilitates the collection and distribution of surplus food from dining services and campus events \cite{campusplate}. Leanpath proposed a system that utilizes a weight sensor and a camera to track and analyze food waste data, offering comprehensive insights via a cloud-based dashboard \cite{leanpath}. Alipay Ant Forest introduced a gamified mobile phone application, incentivizing users to finish their meals by rewarding the ``virtual green energy'' that can be used to adopt a real tree in the desert \cite{alipay}. Additionally, social media has been leveraged as an effective tool for spreading knowledge about food waste, with studies showing its impact on raising awareness of waste generation \cite{young2017can}.


Inspired by these successful endeavors, we present \systemname, a system that leverages advanced technologies and human-centered computing to motivate sustainable behavior change. Our approach lies in the integration of a data storytelling dashboard and a mobile web application, together arousing awareness amongst the campus community about food waste issues and cultivating sustainable food-saving habits. The system was deployed in a two-week food-saving campaign at \shortuniversityname which attracted over 200 participants including students, faculties, and staffs. A post-study survey with 53 respondents validated the effectiveness of the system in heightening awareness of food waste issues and fostering food-saving behavior on campus. This system provides a blueprint for other institutions aiming to reduce food waste and foster sustainability within their communities through the strategic use of data visualization and human-centered design.