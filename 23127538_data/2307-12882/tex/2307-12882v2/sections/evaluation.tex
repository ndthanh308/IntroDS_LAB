% Figure environment removed

\section{Evaluation}
\label{sec:evaluation}

We conducted a post-campaign survey for a comprehensive evaluation of the effectiveness of our system and the campaign. To encourage user feedback, we offered extra five coupons worth 50 Hong Kong dollars ($\sim$\$6.5) through a lucky draw for survey respondents. We received responses from 53 participants. The questionnaire consists of 7 single-answer Likert scale questions (Figure \ref{fig:eva_likert}), 1 multiple-answer question (Figure \ref{fig:eva_issue}), and 1 open-ended question. Our usability-related questions were informed by the Post-Study System Usability Questionnaire (PSSUQ), a widely used tool for evaluating the usability of systems \cite{lewis2002psychometric}.

\subsection{Dashboard Evaluation (Task 1)}\label{dashboard_eva}
Question 1 (Q1) asked whether the large-screen data storytelling dashboard increased users' awareness of campus food waste. Over 60\% of respondents agreed with its role in heightening their awareness. However, about a quarter of respondents reported ``N/A'', meaning they have not noticed the dashboard, suggesting that its placement could be optimized for better visibility.

\subsection{Web Application Evaluation (Task 2)}

We evaluated the usability of our mobile web application through Q2, with over 90\% finding the application user-friendly. Figure \ref{fig:eva_issue} displays the responses to Q3, revealing that approximately 70\% of users experienced no technical issues. However, the prevalent issue among the remaining respondents was the difficulty in uploading photos, indicating a potential area for enhancement, such as implementing a front-end image compressor.


We further asked whether users felt encouraged to save food and use the application. Almost 90\% of respondents in Q4 reported a sense of accomplishment when earning a badge, suggesting it may motivate ongoing food-saving actions.

\subsection{Campaign Evaluation (Task 1 + Task 2)}

We aimed to measure the impact of the campaign on changing user behavior and promoting food-saving habits through Q5 and Q6. In Q5, more than 80\% of respondents agreed that the campaign heightened their awareness of food waste issues. Moreover, approximately 65\% of respondents in Q6 reported improved meal planning and waste reduction during and after the campaign, highlighting the campaign's effectiveness in instigating behavioral change for a more sustainable campus.

We also explored the potential of conducting similar future food-saving campaigns in Q7 and Q8. The response to Q7 underscored a strong participant interest in future campaigns, and Q8 revealed the majority's willingness to involve their friends in these initiatives. This enthusiasm suggests promising potential for expanding these campaigns, inviting a larger and more diverse participant group on campus.

\subsection{User suggestions}

To gain further insights into the user experience and campaign effectiveness, we offered an open-ended section in Q9 for participants to provide their personal thoughts. We conducted a thematic analysis of these comments, identifying common areas of improvement suggested by users, which are listed below.


% Figure environment removed
\begin{itemize}
    \item \textbf{Reminder System Enhancements} Eleven users expressed forgetting to record their meal, indicating a need for a more robust reminder system. \textit{``Add notification, sometimes forget to record''} and to add \textit{``reminder notification''} were common suggestions. Incorporating these features could help users maintain a consistent routine of logging their meals.
    \item \textbf{Diversification and Flexibility of Rewards} Eight respondents advocated for a more varied range of rewards. Suggestions included coupons for cafes, cash coupons for groceries, and more flexibility in rewards overall.
    \item \textbf{Increased Campaign Awareness} Five respondents suggested improving campaign promotion and awareness, echoing the issue mentioned earlier in Section \ref{dashboard_eva}. \textit{``More eye-catching poster and promotion''} and \textit{``more information to students''} were some of the suggestions put forth. Enhancing awareness efforts could attract more participants and further elevate the campaign's impact.
    \item \textbf{Campaign Duration and Rules Clarity} Three respondents found campaign rules unclear and felt the campaign duration was too short. One user candidly shared, \textit{``The badges are too difficult for 5 days in a row because some students may not be able to eat at school every day.''} Extending the campaign and clearly communicating the rules could make it more inclusive and achievable for all participants.
\end{itemize}
