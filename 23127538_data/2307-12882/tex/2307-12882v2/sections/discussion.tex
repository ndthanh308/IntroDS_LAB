\section{Discussion}

\subsection{Significance}

Our study showed positive outcomes, with the data storytelling dashboard and web application successfully raising food waste awareness and promoting behavior changes within a campus environment. The dashboard notably heightened awareness for 60\% of the respondents, highlighting the potential of persuasive techniques for promoting sustainability among university community members, who can act as significant change-makers due to their exposure to innovative ideas. User-friendliness and the absence of technical issues in the web application were also acknowledged with 90\% and 70\% positive feedback, respectively. The gamification feature in the web application was impactful, motivating nearly 90\% of the respondents, which suggests playful competition can be an effective strategy for promoting sustainable behavior in a community.

The broader impact of our campaign was evident with more than 80\% of respondents reporting increased food waste awareness and about 65\% confirming improved meal planning and waste reduction practices. This demonstrates the campaign's success in not only raising awareness but also instigating tangible behavior change. Furthermore, respondents' high interest in future campaigns and their willingness to promote these initiatives highlight the power of community involvement in sustainability campaigns.

\subsection{Lessons Learned}

For future organizers planning sustainability initiatives in a campus setting, we suggest several considerations to optimize their outcomes. 
Firstly, when designing the computing system, it is crucial to incorporate cultural context to create a more relevant and effective system that resonates with the local community. For example, it is recommended to carefully consider and identify the major components of food waste in the local food culture when creating an image dataset for image segmentation model training. Additionally, it is important to ensure that the visual elements used for emotion elicitation align with the local conventions and cultural norms. 
Secondly, due to the busy nature of campus life, particularly for students with rigorous schedules, an efficient yet user-friendly reminder system is crucial to maintain consistent user engagement~\cite{wu2018pulse}. For instance, the web application could request users' routine meal times and consent during registration. Consequently, personalized reminder emails aligning with meal times can be sent to enhance the likelihood of engagement.
Thirdly, while the success of the gamification elements in our campaign underscores their potential for boosting engagement, it's vital to tailor these elements to suit diverse user preferences, ensuring broader participation. For example, users making incremental progress in waste reduction could be awarded badges, even if they haven't completed all tasks. This approach could encourage sustained progress and ensure broader participation.
Lastly, while an emotionally engaging data storytelling dashboard is effective, its location in a busy campus environment is crucial. Strategic placement in high-traffic locations or even on the food ordering machine could ensure its visibility, thereby maximizing engagement among campus members. 

Additionally, while our campaign successfully engaged a portion of the campus community, the number of participants represented only about one percent of the total campus population. This highlights the importance of exploring the potential for integrating this system into broader policy changes or system-level interventions as outlined in Reynolds et al.~\cite{reynolds2019consumption}. This integration may significantly expand the system's influence and sustain long-term food-saving behavior beyond the duration of the campaign. 
For instance, future work could explore partnerships with campus dining services to integrate the system into their operations, such as incorporating food waste tracking into the meal ordering process. Additionally, collaborations with campus administration could lead to policy changes that incentivize food-saving behavior, such as discounts for students who consistently demonstrate low food waste. By integrating our system into broader institutional practices and policies, we could reach a larger audience and have a more substantial impact on food waste reduction at the campus level.

In summary, our study demonstrates an effective approach to fostering sustainable change in a campus setting by leveraging information-based and technological interventions. While we've seen promising results, continuous refinement based on user feedback and exploration of strategies for long-term engagement and policy integration are crucial. The insights from this project can inform future efforts in similar contexts, contributing to a larger goal of promoting global sustainability.
