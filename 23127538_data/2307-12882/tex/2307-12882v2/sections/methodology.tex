
% Figure environment removed
\section{Method}
\label{sec: method}

In this section, we provide a thorough overview of \systemname, which consists of a data storytelling dashboard designed to raise public consciousness and a mobile web application built to promote user participation. Following this, we present a campus-wide food-saving campaign that incorporates \systemname. Collectively, these elements form a multifaceted intervention strategy, aiming to foster sustainable food consumption behaviors.

\subsection{Data Storytelling Dashboard}

The architecture of the dashboard consists of four main modules, as shown in Figure \ref{fig:dashboard_arch}:
\begin{itemize}
    \item A video processing pipeline that processes food waste information captured from video clips.
    \item A database to store structured data.
    \item A back-end system that handles data requests from the front-end via application programming interfaces (APIs).
    \item A front-end data storytelling website designed for data visualization and analytics, targeting the public audience.
\end{itemize}


The video processing pipeline analyzes the surveillance footage captured by the tray return station cameras in a canteen on campus. It aims to estimate the quantities of three types of food waste, including rice, meat, and vegetables. These categories were chosen in accordance with the food culture of the region where \shortuniversityname is located, Hong Kong, where rice is a staple food and meat and vegetables are common components of meals. This pipeline leverages several pre-trained deep-learning models.
Firstly, an object detection model trained with the YOLO architecture \cite{redmon2016yolo} is utilized to extract tray images from the video clips. Subsequently, an image segmentation model trained with the DeepLabv3 architecture \cite{chen2018deeplab} is employed to segment and identify the various types of food waste present. Finally, the areas occupied by each type of food waste on each tray are stored in JSON format within a MongoDB database \cite{mongo}.

For the back-end, we utilized Flask \cite{flask}, a Python-based framework. The back-end interacts with the front-end to retrieve daily and monthly aggregated statistics. To ensure efficiency, we designed a scheduler that runs on a daily basis. This scheduler triggers various statistical algorithms and caches the most recent aggregate data in the database for quick retrieval. Additionally, we developed a regression model for estimating the weight of food waste based on the pixel area. The dependent variable in this analysis is the weight data obtained from the university database, while the calculated pixel area serves as the independent variable.

% Figure environment removed

% Figure environment removed

To enhance the emotional impact of the front-end, we referred to academic research on emotion elicitation in data stories \cite{Lan21smile, lan2022negative} and incorporated some relevant design heuristics into our dashboard design. Using Vue.js \cite{vue}, we developed two pages that aim to evoke an emotional response from the audience.
The first page, updated daily, displays the severity and type of food waste using data visualization and graphic illustrations (Figure \ref{fig:dashboard_ui1}). 
We visually depict the extent of waste distribution using 100 bowls, a symbolism deeply rooted in the local food culture, where each bowl stands for 1\% of the total meals.
By presenting timely and relatable information, this page seeks to resonate with the audience and capture their attention.
The second page, updated monthly, showcases a series of bar charts illustrating the changing trend of food waste on campus (Figure \ref{fig:dashboard_ui2}). This provides a broader perspective on the issue and allows users to track progress over time.
To make the dashboard visually appealing on large screens, we incorporated animations to enhance the presentation of the data visualization components with the two pages seamlessly transitioning in an infinite loop.
In addition to providing data stories, we aimed to prompt action from the audience. Therefore, each page includes food-saving tips such as ``consider your appetite before ordering'', ``choose the `less rice' option'', ``kindly ask the staff to give you less food'', and ``bring a lunch box to pack excess food''. By incorporating these tips, we encourage users to take practical steps toward reducing food waste.

\subsection{Mobile Web Application}
We developed a mobile web application to encourage self-tracking and reflective behaviors~\cite{luo2021foodscrape}, where we employed gamified design to engage participants.
The web application's architecture, shown in Figure \ref{fig:app_arch}, consists of four main modules:
\begin{itemize}
    \item A database for user data storage.
    \item A storage service for user-uploaded files.
    \item A back-end system offering functions via application programming interfaces (APIs).
    \item A front-end web application on mobile phones providing various user features.
\end{itemize}

% Figure environment removed

Like the dashboard, the web application's database is MongoDB-based. We utilize AWS Simple Storage Service (S3) \cite{s3} for file storage. The back-end interacts with the file storage system and database to facilitate various functions. It allows users to register, log in, upload photos and food completion percentages, view historical data, and engage in gamification activities.

% Figure environment removed

For the front-end, we prioritized a web-based application over native iOS or Android applications due to its advantages in cross-platform compatibility, maintenance efficiency, and accelerated development cycle. We utilized Vue.js to build a versatile web application, as displayed in Figure \ref{fig:app_ui}.
The Overview page (Figure \ref{fig:app_ui}-A) presents a comprehensive summary of users' self-recorded food-saving actions. In this context, food-saving actions refer to the process of users uploading photos of their finished meals and self-reporting the completeness score for three types of ingredients. These ingredient types align with the food waste categories displayed on the storytelling dashboard, namely rice, vegetables, and meat. Initially, we also explored the integration of computer vision technology to automatically quantify leftovers on a plate, drawing inspiration from the video processing algorithm employed in the data storytelling dashboard. However, our pilot study revealed that this approach could lead to laggy responses and potential inaccuracies, compromising user experience. As a result, we eventually decided to rely on the honesty of user-reported values from our community members. The data panel (Figure \ref{fig:app_ui}-A1) at the top displays individual user's average completion statistics across different types of food. A ring chart on the left visually encodes the completion percentage, using arcs of varying lengths to represent different completion rates. This panel also juxtaposes user statistics with average completion values of all registered users, enabling users to compare their performance against the broader campus community. The middle section of the page exhibits the array of badges earned by users. At the bottom, users can review their past submissions along with the corresponding completion ring chart (Figure \ref{fig:app_ui}-A2). A conveniently located camera button at the bottom right allows users to navigate swiftly to the Record page (Figure \ref{fig:app_ui}-B) and log their food-saving actions post-meal.
The Record page (Figure \ref{fig:app_ui}-B) guides users to take a photo of their finished meals and input their completion score of each type of food. Each submission of this record is considered a food-saving action and contributes to the user's progress towards earning badges and final rewards.
The Badge page (Figure \ref{fig:app_ui}-C) displays badges earned, supplemented by a progress bar (Figure \ref{fig:app_ui}-C1) that indicates the remaining effort required to attain each badge. This page also showcases the number of users who have earned each badge, fostering a competitive spirit and motivating users to strive for these badges.
Lastly, the History page (Figure \ref{fig:app_ui}-D) provides users with access to their record list, offering a retrospective view of their food-saving journey.

% Figure environment removed

The web application integrates a gamification feature to maintain user engagement through competition and goal-setting. Gamification has proven to be effective in increasing user engagement and promoting behavior change \cite{deterding2011game}. As shown in Figure \ref{fig:badges}, this long-term, habit-forming challenge is made more interesting by setting different levels of goals in the campaign, including attempt, persistence, quantity, and quality. Each time users achieve a goal, they earn a badge accordingly. The multi-tiered badge system creates a playful atmosphere by encouraging participants to make meaningful choices on their own \cite{nicholson2015recipe}.
The Attempt badge is awarded upon the user's first successful submission. The Persistence badge requires users to record actions for several consecutive days, encouraging a daily food-saving habit. Quality and Quantity badges motivate users to consistently finish meals instead of making several random attempts for rewards. To claim the final reward, users must earn all of the badges.

\subsection{Campaign Organization}
To generate positive social impact and evaluate the real-world effectiveness of \systemname, we incorporated our system into a two-week food-saving campaign named ``Save Food, Win Free Meals'' in March and April 2023. Participants who acquired all the badges were rewarded with a coupon that is worth approximately \$6.5 from a local healthy food store, sufficient for one free meal. The campaign timeline commenced with a pre-registration period from March 13 to March 19 to engage potential participants. The campaign officially took flight from March 20 to April 3, spanning over two weeks.

To promote participation, we developed promotional materials, comprising printed posters and digital content for social media platforms. With the aid of the Sustainable Office at \shortuniversityname, we employed various channels to publicize the campaign, as illustrated in Figure \ref{fig:promotion}. This included sharing campaign details on \shortuniversityname's official Instagram with about 20k followers, posting the event on the university event calendar, featuring the poster and real-time data storytelling dashboard on digital signages on campus, and displaying posters at 10 high-visibility locations on campus, particularly around the canteens.

% Figure environment removed

Our campaign successfully attracted 220 users who registered on the web application. Throughout the campaign, users logged a total of 811 food-saving actions, and 51 users won the final award.