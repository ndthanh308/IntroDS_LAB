

\section{Objectives}
\label{sec: backg}

Our study is motivated by a comprehensive review on food waste reduction interventions, which categorizes applied interventions into three types: information-based, technological solutions, and policy/system/practice change \cite{reynolds2019consumption}. Information-based interventions are designed to raise awareness and stimulate reflection on the food waste problem. Technological solutions employ technologies to track food waste data and promote food-saving behavior. Policy/system/practice change interventions advocate for larger-scale changes to the way food is managed and consumed.

While policy/system/practice change interventions often require substantial institutional support and changes to existing infrastructure, which may not be feasible within the scope of our current project, we strategically chose to focus on the first two intervention types, information-based and technological solutions. These interventions align well with our campus environment, where we can leverage existing technological infrastructure and the inherent openness of the university community to new information and behavioral changes.
Therefore, the primary objective of our system is to foster food-saving habits of the university community members and contribute to building a sustainable campus by leveraging these two intervention types. To achieve this goal, we have divided it into two actionable tasks:

\noindent\textbf{Task 1: Raising awareness about the severity of campus food waste}:
In alignment with the information-based intervention strategy, we propose a large-screen data storytelling dashboard that vividly depicts the food waste information on campus. The dashboard is designed to provide insights to the audience and stimulate reflection on the existing food waste problem. Consequently, the interface needs to be visually appealing yet capable of eliciting negative emotions to stir environmental consciousness about food wastage.

\noindent\textbf{Task 2: Encouraging action to reduce food waste at every meal}:
In line with the technological solutions intervention strategy, we propose a mobile web application that motivates users to record their food waste reduction efforts and track the changes in their behavior throughout a food-saving campaign, fostering long-term food conservation habits. To enhance engagement, we incorporate gamification elements and reward active participants.

As depicted in Figure \ref{fig:overview}, our proposed system comprises two main components: a data storytelling dashboard on a large screen and a mobile web application. The dashboard presents data-driven insights to raise food waste awareness among the HKUST community. Concurrently, the gamified web application and the food-saving campaign motivate participants to reduce food wastage in exchange for rewards. The individual efforts are then collectively displayed on the dashboard, creating a feedback loop that encourages continued food waste reduction. This approach aligns with our ultimate objective of cultivating a sustainable and smart campus environment.

