\documentclass{ecai}  % use option [doubleblind] for double blind submission and hiding the authors section

\usepackage{graphicx}
\usepackage{latexsym}

%\ecaisubmission      % inserts page numbers. Use only for submission of paper.
                      % Do NOT use for camera-ready version of paper.

% \paperid{440}        % paper id for double blind submission





\usepackage{balance} % for balancing columns on the final page

%%%%%%%%%%%%%%%%%%%%%%%%%%%%%%%%%%%%%%%%%%%%%%%%%%%%%%%%%%%%%%%%%%%%%%%%
% \usepackage[colorlinks,linkcolor=red]{hyperref}
\usepackage{booktabs} 
% \usepackage[ruled,norelsize,algo2e]{algorithm2e}
\usepackage[ruled]{algorithm2e} %定义算法
% \usepackage{algorithmic}
\usepackage{multirow}
\usepackage{multicol}
% \usepackage{amsfonts,amssymb, amsmath} %%大写空心粗体字命令 \mathbb 和欧拉字体命令 \mathfrak
\usepackage{bm} %公式粗体
\usepackage{stfloats}


\usepackage[figuresright]{rotating}
% 定义短下划线
% \usepackage{ulem}
% \usepackage[marginal]{footmisc}
% \renewcommand{\thefootnote}{} %引入宏包,使得脚注首行不缩进。

% 脚注每页重新编号
\usepackage{perpage}
\MakePerPage{footnote}

\usepackage{enumitem}
\usepackage{float}
\usepackage{booktabs}

\usepackage{diagbox}
\usepackage{graphicx}

\let\comment\undefined
\usepackage{changes}
\usepackage{enumitem}
\usepackage{float}
\usepackage{array}
% \usepackage[justification=centering]{caption} % 全局图片标题居中对齐
% \usepackage{subfigure}
% \usepackage[subfigure]{tocloft}
\usepackage{subfigure}
% \usepackage{tablefootnote}
\usepackage[bottom]{footmisc}

\usepackage{caption}
\usepackage{hyperref}
% \usepackage{amssymb}
\usepackage{amsfonts,amssymb, amsmath} %%大写空心粗体字命令 \mathbb 和欧拉字体命令 \mathfrak
% \usepackage{geometry}

\newtheorem{definition}{Definition}


\begin{document}

\begin{frontmatter}

\title{Learning to Collaborate by Grouping: a Consensus-oriented Strategy for Multi-agent Reinforcement Learning}

\author[A]{\fnms{Jingqing}~\snm{Ruan}}
\author[B]{\fnms{Xiaotian}~\snm{Hao}}
\author[C]{\fnms{Dong}~\snm{Li}\thanks{Corresponding Author. Email: lidong106@huawei.com}} 
\author[D]{\fnms{Hangyu}~\snm{Mao}} 


\address[A]{Institute of Automation, Chinese Academy of Sciences}
\address[B]{Tianjin University}
\address[C]{Noah’s Ark Lab, Huawei}
\address[D]{SenseTime Research}


\begin{abstract}
Multi-agent systems require effective coordination between groups and individuals to achieve common goals. However, current multi-agent reinforcement learning (MARL) methods primarily focus on improving individual policies and do not adequately address group-level policies, which leads to weak cooperation. 
To address this issue, we propose a novel {\it Consensus-oriented Strategy} (CoS) that emphasizes group and individual policies simultaneously. 
Specifically, CoS comprises two main components: (a) the vector quantized group consensus module, which extracts discrete latent embeddings that represent the stable and discriminative group consensus, and (b) the group consensus-oriented strategy, which integrates the group policy using a hypernet and the individual policies using the group consensus, thereby promoting coordination at both the group and individual levels. Through empirical experiments on cooperative navigation tasks with both discrete and continuous spaces, as well as google research football, we demonstrate that CoS outperforms state-of-the-art MARL algorithms and achieves better collaboration, thus providing a promising solution for achieving effective coordination in multi-agent systems.
\end{abstract}

\end{frontmatter}



The meteoric advancement of machine learning and artificial intelligence technologies has enabled the construction of neural networks that effectively emulate the complex computations of the human brain. These deep learning models have found utility in a wide range of applications, such as computer vision, natural language processing, autonomous driving, and more. With the growing complexity and sophistication of these neural network models, the computational requirements, particularly for 32-bit operations, have exponentially increased. This heightened computational demand necessitates the exploration of more efficient alternatives, such as 16-bit operations.

However, the shift to 16-bit operations is riddled with challenges. A common standpoint within the research community argues that 16-bit operations are not ideally suited for neural network computations. This belief is mainly attributable to concerns related to numerical instability during the backpropagation phase, especially when popular optimizers like Adam are employed. This instability, more pronounced during the optimizer-mediated backpropagation process rather than forward propagation, can negatively impact the performance of 16-bit operations and compromise the functioning of the neural network model. Current optimizers predominantly operate on 32-bit precision. If these are deployed in a 16-bit environment without appropriate hyperparameter fine-tuning, the neural network models encounter difficulties during learning. This issue is particularly evident in backward propagation, which heavily relies on the optimizer. Confronted with these challenges, the objective of this research is to conduct an exhaustive investigation into the feasibility and implementation of 16-bit operations for neural network training. We propose and evaluate innovative strategies aimed at reducing the numerical instability encountered during the backpropagation phase under 16-bit environments. A significant focus of this paper is also dedicated to exploring the future possibilities of developing 16-bit based optimizers. One of the fundamental aims of this research is to adapt key optimizers such as Adam to prevent numerical instability, thereby facilitating efficient 16-bit computations. These newly enhanced optimizers are designed to not only address the issue of numerical instability but also leverage the computational advantages offered by 16-bit operations, all without compromising the overall performance of the neural network models. Through this research, our intention goes beyond improving the efficiency of neural network training; we also strive to validate the use of 16-bit operations as a dependable and efficient computational methodology in the domain of deep learning. We anticipate that our research will contribute to a shift in the prevalent perceptions about 16-bit operations and will foster further innovation in the field. Ultimately, we hope our findings will pave the way for a new era in deep learning research characterized by efficient, high-performance neural network models.
The matter of numerical precision in deep learning model training has garnered substantial attention in recent years. A milestone study by Gupta et al. \cite{Gupta2015} was among the first to explore the potential of lower numerical precision in deep learning, emphasizing the critical balance between computational efficiency and precision. They suggested that with careful implementation, lower precision models can be as effective as their higher precision counterparts while requiring less computational and memory resources. Building upon these findings, significant advancements have been made in utilizing 16-bit operations for training convolutional neural networks. Courbariaux et al. \cite{Courbariaux2016} pioneered a novel technique to train neural networks using binary weights and activations, significantly reducing memory and computational demands. Similarly, Micikevicius et al. \cite{Micikevicius2017} stressed the imperative transition from 32-bit to 16-bit operations in deep learning, given the memory and computational constraints associated with training increasingly complex neural networks. Their research demonstrated that half-precision computations offer a more memory and computation-efficient alternative without compromising the model's performance. Despite these advancements, numerical instability during backpropagation remains a persistent challenge. Bengio et al. \cite{Bengio1994} illustrated the difficulties encountered in learning long-term dependencies with gradient descent due to numerical instability. Such findings underscore the need for innovative solutions to mitigate this widespread issue. One critical component to addressing this challenge lies in the development of effective optimizers. Adam, an optimizer introduced by Kingma and Ba \cite{kingma2014adam}, is known to face issues of numerical instability when employed in lower-precision environments. Another research by Yun et al. \cite{yun2023defense}, provided a comprehensive theoretical analysis, focusing on the performance of pure 16-bit floating-point neural networks. They introduced the concepts of floating-point error and tolerance to define the conditions under which 16-bit models could approximate their 32-bit counterparts. Their results indicate that pure 16-bit floating-point neural networks can achieve similar or superior performance compared to their mixed-precision and 32-bit counterparts, offering a unique perspective on the benefits of pure 16-bit networks. Lastly, in the context of efficient neural network training, Han et al. \cite{Han2015} proposed a three-stage pipeline that significantly reduces the storage requirements of the network. Their work forms an integral part of the broader discussion on efficient neural network training, further reinforcing the relevance of our research on 16-bit operations. Through our investigation, we aim to contribute to this body of work by presenting a novel approach to addressing numerical instability issues associated with 16-bit precision in the realm of deep learning model training.
% \input{2-problem-setup}
% \input{2-preliminary}
\section{Methods}
\label{sec:methods}
We introduce our experiment design with LMMs on multiple mental health prediction task setups, including zero-shot prompting (Sec.~\ref{sub:methods:zero-shot}), few-shot prompting (Sec.~\ref{sub:methods:few-shot}), and instruction finetuning (Sec.~\ref{sub:methods:finetuning}). These setups are model-agnostic, and we will present the details of language models and datasets employed for our experiment in the next section.

\subsection{Zero-shot Prompting}
\label{sub:methods:zero-shot}
The language understanding and reasoning capability of LLMs have enabled a wide range of applications without the need for any domain-specific data, but only providing appropriate prompts~\cite{kojima_large_2022,wei2021finetuned}.
Therefore, we start with prompt design for mental health tasks in a zero-shot setting.

The goal of prompt design is to empower a pre-trained general-purpose LLM to achieve good performance on tasks in the mental health domain. We propose a general zero-shot prompt template ($\textit{Prompt}_{ZS}$) that consists of four parts:
\begin{equation}
    \textit{Prompt}_{ZS} = \textit{TextData} + \textit{Prompt}_{\textit{Part1-S}} + \textit{Prompt}_{\textit{Part2-Q}} + \textit{OutputConstraint}
\label{eq:prompt-zs}
\end{equation}
where \textit{TextData} is the online text data generated by end-users. \textit{Prompt}$_{\textit{Part1-S}}$ provides specifications for a mental health prediction target. \textit{Prompt}$_{\textit{Part2-Q}}$ poses the question for LLMs to answer. And \textit{OutputConstraint} controls the output of models (\eg ``Only return yes or no'' for a binary classification task).

We propose several design strategies for \textit{Prompt}$_{\textit{Part1-S}}$, as shown in the top part of Table~\ref{tab:prompt_design}: (1) \textbf{Basic}, which leaves it as blank; (2) \textbf{Context Enhancement}, which provides more social media context about the \textit{TextData}; (3) \textbf{Mental Health Enhancement}, which inserts mental health concept by asking the model to act as an expert. (4) \textbf{Context \& Mental Health Enhancement}, which combines both enhancement strategies by asking the model to act as a mental health expert under the social media context.

As for \textit{Prompt}$_{\textit{Part2-Q}}$, we mainly focus on two categories of mental health prediction targets: (1) predicting critical mental states, such as stress or depression, and (2) predicting high-stake risk actions, such as suicide. We tailor the question description for each category. Moreover, for both categories, we explore binary and multi-class classification tasks
\footnote{We also conduct an exploratory case study on mental health reasoning tasks. Please see more details in Sec.~\ref{sub:results:reasoning}.}.
Thus, we also make small modifications based on specific tasks to ensure appropriate questions (see Sec.~\ref{sec:implementation} for our mental health tasks). The bottom part of Table~\ref{tab:prompt_design} summarizes the mapping.

\renewcommand{\arraystretch}{1}
\begin{table}[]
\centering
\caption{Prompt Design for Mental Health Recognition Tasks. Prompt Part 1 aims to provide a better specification for LLMs and Part 2 poses the questions for LLMs to answer. For Part 1, we propose three strategies: context enhancement, mental health enhancement, and the combination of both. As for Part 2, we design different content for multiple mental health problem categories and prediction tasks. For each part, we propose two to three versions to improve its variation.}
\label{tab:prompt_design}
\label{tab:my-table}
\resizebox{1\textwidth}{!}{
\begin{tabular}{ccl}
\thickhlinespace
 \multicolumn{2}{c}{\textbf{Strategy}} &
  \multicolumn{1}{c}{\textbf{Prompt Part 1 - S}} \\
\thickhlinespace 
\multicolumn{2}{c}{Basic} &
  \makecell[l]{$\bullet$ \{ blank \} } \\ \hlinespace
\multicolumn{2}{c}{Context Enhancement} &
  \makecell[l]{$\bullet$ This person wrote this paragraph on social media.\\$\bullet$ Consider this post on social media to answer the question.} \\ \hlinespace
\multicolumn{2}{c}{Mental Health Enhancement} &
  \makecell[l]{$\bullet$ As a psychologist, read the post on social media and answer the question.\\ $\bullet$ If you are a psychologist, read the post on social media and answer the question.} \\ \hlinespace
\multicolumn{2}{c}{Context \& Mental Health Enhancement} &
  \makecell[l]{ $\bullet$ This person wrote this paragraph on social media. As a psychologist, read the post on social media \\ \xspace\xspace\xspace\xspace and answer the question.\\ $\bullet$ This person wrote this paragraph on social media. As a psychologist, consider the mental well-being\\\xspace\xspace\xspace\xspace condition expressed in this post, read the post on social media, and answer the question.} \\
\thickhlinespace
\multicolumn{1}{c}{\textbf{Category}} &
  \multicolumn{1}{c}{\textbf{Task}} &
  \multicolumn{1}{c}{\textbf{Prompt Part 2 - Q}} \\
\thickhlinespace
\multirow{2}{*}{\makecell[c]{\\Mental state\\ (\eg stressed, \\ depressed)}} &
  \makecell[c]{Binary classification\\ (\eg yes or no)} &
  \makecell[l]{$\bullet$ Is the poster {[}stressed{]}?\\$\bullet$ Is the poster of this post {[}stressed{]}?\\$\bullet$ Determine if the poster of this post is {[}stressed{]}.} \\  \clinespace{2-3}
 &
  \makecell[c]{Multi-class classification\\ (\eg multiple levels)} &
  \makecell[l]{$\bullet$ Which level is the person {[}stressed{]}?\\$\bullet$ How {[}stressed{]} is the person?\\ $\bullet$ Determine how {[}stressed{]} the person is.} \\ \hlinespace
\multirow{2}{*}{\makecell[c]{\\Critical risk action\\ (\eg suicide)}} &
  \makecell[c]{Binary classification\\ (\eg yes or no)} &
  \makecell[l]{$\bullet$ Does the poster want to {[}suicide{]}?\\$\bullet$ Is the poster likely to {[}suicide{]}?\\$\bullet$ Determine if the poster of this post want to {[}suicide{]}.} \\  \clinespace{2-3}
 &
  \makecell[c]{Multi-class classification\\ (\eg multiple levels)} &
  \makecell[l]{$\bullet$ Which level of {[}suicide{]} risk does the person have?\\$\bullet$ How {[}suicidal{]} is the person?\\$\bullet$ Determine which level of {[}suicide{]} risk does the person have.}\\
\thickhlinespace
\end{tabular}
}
\end{table}
\renewcommand{\arraystretch}{1.0}

For both \textit{Prompt}$_{\textit{Part1-S}}$ and \textit{Prompt}$_{\textit{Part2-Q}}$, we propose several versions to improve its variability. We then evaluate these prompts on multiple LLMs on different datasets and compare their performance.

\subsection{Few-shot Prompting}
\label{sub:methods:few-shot}

In order to provide more domain-specific information, researchers have also explored few-shot prompting with LLMs by providing few-shot demonstrations to support in-context learning (\eg \cite{agrawal2022large,dang2022prompt}). Note that these few examples are used solely in prompts, and the model parameters remain unchanged. The intuition is to present a few ``examples'' for the model to learn domain-specific knowledge \textit{in situ}.
In our setting, we also test this strategy by adding additional randomly sampled [$\textit{Prompt}_{ZS} - \textit{label}$] pairs. The design of the few-shot prompt ($\textit{Prompt}_{FS}$) is straightforward:
\begin{equation}
    \textit{Prompt}_{FS} = [\textit{Sample Prompt}_{ZS} - \textit{label}]_{M} + \textit{Prompt}_{ZS}
\label{eq:prompt-fs}
\end{equation}
where $M$ is the number of prompt-label pairs and is capped by the input length limit of a model. Note that both the $\textit{Sample Prompt}_{ZS}$ and $\textit{Prompt}_{ZS}$ follow Eq.~\ref{eq:prompt-zs} and employ the same design of \textit{Prompt}$_{\textit{Part1-S}}$ and \textit{Prompt}$_{\textit{Part2-Q}}$ to ensure consistency.

\subsection{Instruction Finetuning}
\label{sub:methods:finetuning}

In contrast to the few-shot prompting strategy in Sec.~\ref{sub:methods:few-shot}, the goal of this strategy is closer to the traditional few-shot transfer learning, where we further train the model with a small amount of domain-specific data (\eg \cite{huang2022large,xu2021raise,liu_large_2023}).
We experiment with multiple finetuning strategies.

\subsubsection{Single-dataset Finetuning}
\label{subsub:methods:finetuning:single-dataset}
Following most of the previous work in the mental health field~\cite{yang_evaluations_2023,coppersmith_clpsych_2015,de_choudhury_discovering_2016}, we first conduct basic finetuning on a single dataset (the training set). This finetuned model can be tested on the same dataset (the test set) to evaluate its performance and different datasets to evaluate its generalizability.

\subsubsection{Multi-dataset Finetuning}
\label{subsub:methods:finetuning:multi-dataset}
From Sec.~\ref{sub:methods:zero-shot} to Sec.~\ref{subsub:methods:finetuning:single-dataset}, we have been focusing on one single mental health dataset $D$.
More interestingly, we further experiment with finetuning across multiple datasets simultaneously. Specifically, we leverage instruction finetuning to enable LLMs to handle multiple tasks in different datasets~\cite{brown_language_2020}.

It is noteworthy that such an instruction finetuning setup differs from the state-of-the-art mental-health-specific models (\eg Mental-RoBERTa~\cite{ji_mentalbert_2021}). The previous models are finetuned for a specific task, such as depression prediction or suicidal ideation prediction. Once trained on task A, the model becomes specific to task A and is only suitable for solving that particular task. In contrast, we finetune LLMs on several mental health datasets, employing diverse instructions for different tasks across these datasets in a single iteration. This enables them to handle multiple tasks without additional task-specific finetuning.

% It is noteworthy that such an instruction finetuning setup differs from the state-of-the-art mental-health-specific models (\eg Mental-RoBERTa~\cite{ji_mentalbert_2021}). The previous models are finetuned for a specific task, such as depression prediction or suicidal ideation prediction, or multiple predefined tasks in the multi-task setting. Once the model is trained on task(s) A, it becomes task-A-specific and only aims to solve task A. In contrast, we finetune LLMs on multiple mental health datasets with various instructions for different tasks across multiple datasets in a single round, empowering them to work on multiple tasks without the need for further task-specific finetuning.

For both single- and multi-dataset finetuning, we follow the same two steps:
\begin{align}
\begin{split}
\text{Step 1:} & \text{ Finetune with } [\textit{Prompt}_{ZS} - \textit{label}]_{\sum\limits^{I} N_{D_{i-train}}} \\
\text{Step 2:} & \text{ Test with } [\textit{Prompt}_{ZS}]_{\sum\limits^{I} N_{D_{i-test}}}
\end{split}
\label{eq:prompt-ft}
\end{align}
where $N_{D_{i-train}}$/$N_{D_{i-test}}$ is the total size of the training/test dataset $D_i$, $I$ represents the set of datasets used for finetuning, and $i$ indicates the specific dataset index ($i \in I, |I| \ge 1$). Both $\textit{Prompt}_{ZS\textit{-train}}$ and $\textit{Prompt}_{ZS\textit{-test}}$ follow Eq.~\ref{eq:prompt-zs}. Similar to the few-shot setup in Eq.~\ref{eq:prompt-fs}, they employ the same design of \textit{Prompt}$_{\textit{Part1-S}}$ and \textit{Prompt}$_{\textit{Part2-Q}}$.

% In addition, we also experiment with multiple variations:
% (1) \textbf{Data size variation}, where we sample a subset of each training dataset $D_{i-train}$ with a smaller $N'_{D_{i-train}}$ to explore the effect of data size on finetuning performance.
% (2) \textbf{Prompt variation}, where we randomly sample \textit{Prompt}$_{\textit{Part1-S}}$ from Table~\ref{tab:prompt_design} during the training to explore the effect of prompt diversity on finetuning performance.


\setlength{\tabcolsep}{2.8pt}
\setlength{\abovecaptionskip}{0.5pt}
\setlength{\belowcaptionskip}{0.5pt}
\begin{table*}[!ht]
\vspace{-6pt}
    \centering
    \caption{End steps of various methods on d-CN.}
    \begin{tabular}{ccccccc}
    \toprule
      \#agents & CoS & MAPPO & VDN & QMIX & RODE & ROMA \\ \hline
     4 & \textbf{19.8} $\pm$ \scriptsize{1.1} & 21.2 $\pm$ \scriptsize{0.7} & 24.6 $\pm$ \scriptsize{0.4} & 24.3 $\pm$  \scriptsize{0.3} & 23.6 $\pm$ \scriptsize{1.2} & 24.8 $\pm$ \scriptsize{0.9} \\ 
     6 & \textbf{23.1} $\pm$ \scriptsize{0.9} & 23.2 $\pm$ \scriptsize{0.8} & 24.9 $\pm$ \scriptsize{0.6} & 24.7 $\pm$ \scriptsize{0.5} & 24.9 $\pm$ \scriptsize{0.8} & 24.9 $\pm$ \scriptsize{0.6} \\ 
    10 & 25 $\pm$ \scriptsize{0} & 25 $\pm$ \scriptsize{0} & 25 $\pm$ \scriptsize{0} & 25 $\pm$ \scriptsize{0} & 25 $\pm$ \scriptsize{0} & 25 $\pm$ \scriptsize{0} \\   
    \bottomrule
    \end{tabular}
    \label{tab:endstep-d-CN}
\end{table*}
% \vspace{-26pt}

% \vspace{-10pt}

\section{Experimental Evaluations}
% \vspace{-10pt}

We evaluate the effectiveness of our algorithm in three environments including both discrete and continuous action space: discrete Cooperative Navigation (d-CN), continuous Cooperative Navigation (c-CN), and Google Research Football (GRF), illustrated as Figure~\ref{fig:draw}.


\vspace{-10pt}
% Figure environment removed
\vspace{-20pt}
\subsection{Experimental Settings}

\paragraph{\textbf{d-CN and c-CN}.}
We modify the classic Cooperative Navigation (CN) implemented in the multi-agent particle world~\cite{lowe2017multi} to a more challenging environment, which requires more collaboration among agents.
We initialize the CN world with $n$ landmarks and $2*n$ agents with random locations at the beginning of each episode.
Each agent can only observe its velocity, position, and displacement from other agents and landmarks. 
The final objective is to occupy all the landmarks and each landmark contains two agents.
The reward function can be formulated as $r_t = -0.1 + 3*single + 10*double$. The variables $single$ and $double$ denote the number of landmarks that are occupied by only one and two agents, which corresponds to the reward $3$ and $10$, respectively. 
$-0.1$ is the step punishment.
Obviously, the game can reach the maximum reward while each landmark contains two agents, and can be over. 
d-CN denotes that the world has a discrete action space $\mathcal{A}_d$ including five actions \textit{[up, down, left, right, stop]}.
c-CN represents that the world has a continuous action space $\mathcal{A}_c$.
We set the length of each episode as $25$ time steps.

\paragraph{\textbf{GRF}.}
GRF~\cite{kurach2020google} is a realistically complicated and dynamic multi-agent testbed. 
Agents should have a division of labor and plan to coordinate the time and location to complete the scoring.
In the experiments, we control left-side players except for the goalkeeper while the right-side players are built-in
bots controlled by the game engine. 
Here, each player has 19 actions to control,
including the standard move actions and different ball-kicking techniques.
The observation contains the positions and moving directions of the ego-agent, other agents, and the ball.
We use the Floats wrappers to represent the state that contains a 115-dimensional vector.
The rewards include the SCORING reward $\{-1, +1\}$, and
the CHECKPOINT reward, which is the shaped reward that specifically addresses the sparsity of SCORING. 
The detailed descriptions of GRF can be found in Appendix~\ref{app:grf}.

\paragraph{\textbf{Baselines}.}
We compared our results with several baselines as
follows. {VDN} and {QMIX} are state-of-the-art (SOTA) value factorization approaches, with which it is difficult to obtain coordinated behaviors.
{MAPPO} is the multi-agent competitive SOTA algorithm extended from PPO by setting the sharing actor and a centralized critic. 
{MADDPG} and {MATD3} are classic continuous control algorithms.
% \textbf{LDSA} 
{ROMA} and {RODE} are role-based grouping algorithms.
Note that VDN, QMIX, ROMA, and RODE are designed for the discrete action space, while MADDPG and MATD3 are designed for the continuous action space.


\subsection{Does CoS Perform Better?}

For validating our CoS, we conduct empirical experiments on d-CN, c-CN, and GRF.
The benchmarks we choose include discrete and continuous action spaces, realistically complicated and stochastic worlds.
All experiments are repeated for 10 runs with different random seeds.





% \begin{table*}[!ht]
%     \centering
%     \caption{End steps of various methods. The episode limit is set to 25. The '-' denotes the dismatch action space.}
%     \begin{tabular}{c|c|cccccccc}
%     \toprule
%         ~ & \#agents & CoS & MAPPO & VDN & QMIX & MADDPG & MATD3 & RODE & ROMA \\ \hline
%         ~ & 4 & \textbf{19.8} & 21.2 & 24.6 & 24.3 & - & - & 23.6 & 24.8 \\ 
%         d-CN & 6 & \textbf{23.1} & 23.2 & 24.9 & 24.7 & - & - & 24.9 & 24.9 \\ 
%         ~ & 10 & 25 & 25 & 25 & 25 & - & - & 25 & 25 \\ \hline
%         ~ & 4 & \textbf{19.6} & 20.3 & - & - & 22.3 & 22.1 & - & - \\ 
%         c-CN & 6 & \textbf{22.1} & 23.8 & - & - & 24.3 & 24.2 & - & - \\ 
%         ~ & 10 & 25 & 25 & - & - & 25 & 25 & - & - \\ 
%     \bottomrule
%     \end{tabular}
%     \label{tab:endstep}
% \end{table*}



% Figure environment removed
% \protect\footnotemark. 
% \footnotetext{Note that we don't conduct the experiments with RODE and ROMA on c-CN because RODE and ROMA is only suitable for the discrete action space as reported in the original paper.}



\subsubsection{Performance on d-CN.}
We conduct the experiments across $4,6$ and $10$ agents in d-CN with discrete action space as shown in Figure~\ref{fig:exp-cn}(a-c). 
CoS substantially gets a better average reward than all the baselines, indicating that the group consensus strategy increasingly enhances the superiority 
of our method.


Moreover, as shown in Table~\ref{tab:endstep-d-CN}, we report the mean end steps of these methods after testing 100 episodes, and the episode limit is set to 25. 
CoS basically completes the task faster with the least number of steps.
These results show that the group consensus strategy of Cos can group agents with different intentions toward various landmarks, which boosts the training performance.


\subsubsection{Performance on c-CN.}
As shown in Figure~\ref{fig:exp-cn}(d-f), the results in continuous space also exhibit better performance. 
Obviously, our method has a quicker convergence speed and a smaller variance than others, which indicates the decision of CoS to consider both global and individual guidance can further exploit the underlying properties and facilitate cooperative behaviors among agents.

In Table~\ref{tab:endstep-c-CN}, our method still get the least number of steps. Unfortunately, all the methods have failed with 10 agents because of the stochasticity and difficulty of the environment. It will be an interesting direction to study how to obtain the optimal solution in such complicated scenarios in the fastest time steps.

 % \vspace{-12pt}
\setlength{\tabcolsep}{5.8pt}
\setlength{\abovecaptionskip}{0.5pt}
\setlength{\belowcaptionskip}{0.5pt}
\begin{table}[!ht]
% \vspace{-16pt}
    \centering
    \caption{End steps of various methods on c-CN.}
    \begin{tabular}{ccccccc}
    \toprule
      \#agents & CoS & MAPPO & MADDPG & MATD3 \\ \hline
    4 & \textbf{19.6} $\pm$ 2.1  & 20.3 $\pm$ 2.3  & 22.3 $\pm$ 1.2  & 22.1 $\pm$ 1.6   \\ 
    6 & \textbf{22.1} $\pm$ 1.4  & 23.8 $\pm$ 1.5   & 24.3  $\pm$ 1.3 & 24.2 $\pm$ 1.3   \\ 
    10 & 25 $\pm$ 0  & 25 $\pm$ 0   & 25 $\pm$ 0  & 25 $\pm$ 0   \\ 
\bottomrule
    \end{tabular}
    \label{tab:endstep-c-CN}
\end{table}

 % \vspace{-15pt}


% Figure environment removed




% \textbf{}
\vspace{-20pt}

\subsubsection{Performance on GRF.}

Further, we conduct experiments in google research football, a more dynamic and complicated benchmark to validate the effectiveness of our proposed CoS, shown in Figure~\ref{fig:exp-grf}.
Specifically, we choose three popular scenarios, including \textit{academy\_3\_vs\_1\_with\_keeper} (3vs1), \textit{academy\_counterattack\_easy} (c\_easy), and \textit{academy\_coun-terattack\_hard} (c\_hard).

In Figure~\ref{fig:exp-grf}(a-c), we observed that CoS consistently obtains higher performance than all the baselines in different scenarios of GRF, indicating that our method is robust and effective in complex and dynamic environments. Moreover, this performance improvement in GRF demonstrates that our approach excels at handling stochasticity, as the dynamic grouping strategy can extract more underlying details and make informed high-level decisions.  
% The additional results and the detailed descriptions of GRF can be found in Appendix~\ref{app:grf}.
Additionally, we report the average end steps of CoS in the three scenarios, shown in Figure~\ref{fig:grf-end}.
We test the trained model for 100 episodes and count the average end steps of every algorithm.
% CoS can complete the football game in the fewest steps compared to all baselines, which further demonstrates the superiority of our proposed algorithm.
Specifically, we tested the trained model for 100 episodes and counted the average end steps of each algorithm. Notably, CoS completed the football game in the fewest steps compared to all the baselines, further validating the superiority of our proposed algorithm.


% \subsection{Is Grouping Effective?}


\subsection{Does the components of CoS Really Work?}

% Figure environment removed



To evaluate the effectiveness of the components in CoS, we conduct the ablation study with the following configurations.
\begin{itemize}
    \item \textit{CoS(ours)}: The proposed framework.
    \item \textit{CoS w/o GGP}: Remove the group-guided policy.
    \item \textit{CoS w/o GCP}: Remove the group consensus policy.
    % \item \textit{CoS+GCP(MLP)}: Replace hypernet in GCP with MLP.
\end{itemize}


As shown in Figure~\ref{fig:exp-abl}, the ablation study of 4 and 6 agents on d-CN shows the components in the CoS really work. The performance degradations of \textit{CoS w/o GGP} and \textit{CoS w/o GCP} show that the combination of these two modules is necessary, mainly due to two aspects. (1) GCP utilizes the extracted group consensus embeddings to make high-level decisions, which is conducive to long-term utility. (2) GGP perceives the individual observation from the true world, which can exploit underlying dynamics to make accurate decisions. Therefore, the fine-grained combination of these two modules
can achieve forward-looking guidance to specialize in a certain subtask and rational individual decision-making. 

% Moreover, the blue line represents the results of \textit{CoS+GCP(MLP)}, which indicate it is more effective to use the hypernet to make global guidance in CoS.
% From the perspective of training, the hypernet can well coordinate the potentially conflicting parameters in a unified space through feeding fed with different group consensus embeddings. 
% At the same time,  the hyper-network can also capture the higher-order interaction among group consensus embeddings from another perspective.

\subsection{How Well Does the Group Consensus Embedding Perform?}


To investigate the discriminative power of the CoS model's group consensus embeddings, we visualize the embeddings of several scenarios on two tasks collected in the later stages of the training process. The goal is to show whether CoS can distinguish different groups and achieve separate consensus among every group, aiding in the collaboration of intra-groups in the decision-making process.

Due to the assumption that $K$ agents can group into $K$ groups maximally, the number of group clusters is the same as the number of agents in each scenario. As shown in Figure~\ref{fig:exp-vis1}, the 3D visualizations demonstrate that CoS can consistently produce distinct group embeddings in all the scenarios. 
We showcase the visualizations obtained by applying the T-SNE technique to the group consensus embeddings saved from the last 5000 training steps.
Each cluster in the plots represents a specific group. Each point in the cluster corresponds to a vector in the group codebook. The distinguishable border highlights the ability of CoS to learn consistent group consensus representations.
% The plots show that each group forms a separate cluster, indicating that the model's learned embeddings are effective in discriminating between different groups.
% Additionally, we provide a 2D projection of the same visualizations, which makes the clusters more easily distinguishable.


Overall, these results demonstrate that our proposed method can facilitate intra-group collaboration and decision-making by producing effective group consensus embeddings that can discriminate between different groups in the decision-making process.


% To demonstrate whether CoS can discriminate different groups and achieve separate consensus among every group, we visualize the group consensus embeddings of several scenarios on two tasks collected in the later stages of the training process.
% Due to the assumption that $K$ agents can group into $K$ groups maximally, there are the same group clusters as the agents in each scenario.
% As shown in Figure~\ref{fig:exp-vis1}, CoS can consistently obtain the discriminative group embeddings on all the scenarios, which further indicates that our proposed method can facilitate the collaboration of intra-groups in the decision-making process conditioned on them.



% % Figure environment removed



% Figure environment removed
 \vspace{-16pt}



% % Figure environment removed




\subsection{Has CoS Learned to Group?}

To assess CoS's ability to group and its impact on the collaborative behavior of agents, we conducted some visualizations and analyses. First, at time step $t=20$ during one testing episode, we visualize the grouping effect, as shown in Figure~\ref{fig:d-CN-vis2-1}, which illustrates that agents grouped by the same number in yellow exhibit similar behavior, indicating that CoS has learned to divide the agents into groups and promote collaboration.

To further examine the effect of grouping on performance, we evaluate the average rewards of 100 episodes at different checkpoints with varying numbers of groups, as shown in Figure~\ref{fig:d-CN-vis2-2}. Our results show that in the scenario with 10 agents, dividing them into 5 groups achieves the highest utility, consistent with the environmental setting. This finding highlights the importance of an effective grouping mechanism and demonstrates the capability of CoS to optimize the group sizes based on the task requirements.

% To show whether CoS has learned to group and the grouping effect, we give some visualizations. As shown in Figure~\ref{fig:d-CN-vis2-1}, we visualize one time step ($t=20$) at the testing process and find that agents grouping with the same number in yellow have the same intention to cover a landmark, which indicates CoS has the ability to divide the groups and promote collaboration. Numerically, we count the average rewards of 100 episodes on the checkpoints with different numbers of groups and give a histogram as shown in Figure~\ref{fig:d-CN-vis2-2}. The results show that in the scenario of 10 agents, dividing into 5 groups reaches the highest utility, which is consistent with the environmental setting. It also demonstrates the importance of the proper grouping mechanism.

% Figure environment removed
 \vspace{-16pt}












\section{Conclusions}
% In this paper, we propose CoS, a consensus-oriented strategy to promote multi-agent collaboration. First, CoS leverages the vector quantized variational autoencoder to extract the distinguishable and stable group consensus embeddings.
% Furthermore, the embeddings are used to assist global and individual decisions through the proposed group consensus policy and group-guided policy, respectively.
% Empirically, the performance on three benchmarks demonstrates the significance of CoS, the ablation study shows the necessity of combing two policies, and the visualizations further validate the effectiveness of CoS. In the future, it is an interesting direction to investigate universal group embeddings across different tasks.




In this paper, we propose CoS, a novel consensus-oriented strategy to promote multi-agent collaboration. First, CoS leverages the vector quantized variational autoencoder to extract the distinguishable and stable group consensus embeddings. Furthermore, the embeddings are used to assist global and individual decisions through the proposed group consensus policy and group-guided policy, respectively. Our empirical results on three benchmarks show that CoS significantly outperforms existing state-of-the-art methods. Additionally, the ablation study demonstrates the necessity of combining the two policies, and the visualizations further validate the effectiveness of CoS in promoting collaboration among agents. In the future, it is an interesting direction to investigate universal group embeddings across different tasks.






% \input{}

\clearpage


\bibliography{ecai}

\normalsize 
\newpage % 添加一页
% \newgeometry{left=2.5cm,right=2.5cm,top=2.5cm,bottom=2.5cm}
\onecolumn % 切换为单列模式
\setlength{\baselineskip}{20pt} % 设置行间距为20pt
% \setlength{\textwidth}{5in} % 设置页宽为6英寸
% \lipsum[7-ethics] % 新添加页的内容
% 
 
% \setcounter{tocdepth}{2}
% \centering \LARGE
 \section*{\centering \LARGE Ethical Statement}

Our submission involves the development and evaluation of a cooperative multi-agent reinforcement learning method, which has potential implications in various applications, such as multi-player games, traffic signal control, and sensor networks. We have taken steps to ensure that our research does not infringe upon personal data privacy or contribute to any unethical practices such as those related to policing or military applications.

We use publicly available datasets for our experiments and have taken appropriate measures to ensure that any personal data has been anonymized or otherwise rendered non-identifiable. Additionally, we have implemented measures to ensure that our model does not infer personal information or engage in any discriminatory practices. Furthermore, we do not promote the use of our work for any unethical purposes, such as for surveillance or military purposes.

In summary, we recognize the importance of ethical considerations in machine learning and data mining and have taken steps to ensure that our work adheres to ethical principles and does not contribute to any negative societal impacts. We remain committed to ethical research practices and encourage others to do the same.
% \restoregeometry % 恢复原始的页边距

\appendix
\setcounter{tocdepth}{2}
\onecolumn
{\centering\section{Appendix}}
\label{sec:app}

% 11111

\subsection{Network Architecture}
\label{app:net}

There is a summary of all the neural networks used in our framework about the network structure, layers, and activation functions.
% list all the network architecture
\begin{table}[h!]
    \centering
    \begin{tabular}{ccccc}
       \toprule
       ~ & Network Structure &  Layers & Hidden Size & Activation Functions  \\
       \midrule
       History Encoder & CausalSelfAttention & 1 &  -  & ReLu \\
       History Decoder & MLP & 2 &  64  & ReLu   \\
       Group Codebook &   nn.Embedding & 1 & [N, 32] & None \\
       Feature Extractor & CNN+FiLMedBlock & - & - & ReLu \\
       Group-Guided Policy &   Hyper-MLP & 2 & 32 & Tanh \\
       Group Consensus Policy  &   MLP+RNN+MLP & 3 & [64]+[64]+[32] & Tanh \\
       Policy Critic & MLP & 2 & 64 & Tanh\\
       \bottomrule
    \end{tabular}
   \caption{The Summary for Network Architecture}
   \label{tab:net}
%   \vspace{-2em}
\end{table}

Please note that '$-$' denote that we refer readers to check the open source repository CausalSelfAttention, see \href{https://github.com/sachiel321/Efficient-Spatio-Temporal-Transformer}{https://github.com/sachiel321/Efficient-Spatio-Temporal-Transformer}.


\subsection{Parameter Settings}
\label{app:para}

There are our hyper-parameter settings for the training of Vector Quantized Group Consensus and Group Consensus-oriented Strategy, shown in Table~\ref{tab:para1} and Table~\ref{tab:para2}, respectively.

\vspace{10pt}

\begin{minipage}{\textwidth}
\begin{minipage}[t]{0.5\textwidth}

\centering
% \begin{table}[h!]
    % \centering
    \begin{tabular}{cc}
       \toprule
       Description &  Value  \\
       \midrule
       optimizer & $AdamW$  \\
       lr & $5*10^{-4}$  \\
        group embedding size  &   $32$ \\
        context length         &    $1$ \\
        model type        &     $state \ only$ \\
        attention head &    $4$ \\
        actor embedding size &    $32$ \\
        group codebook $K$ &    $agent \ number$ \\
        encoder coefficient $\beta$ &    $1$ \\
       \bottomrule
    \end{tabular}
\makeatletter\def\@captype{table}\makeatother\caption{Hyper-parameters of Vector Quantized Group Consensus}
\label{tab:para1}
\end{minipage}
\begin{minipage}[t]{0.5\textwidth}
\centering
% \begin{table}[h!]
%     \centering
    \begin{tabular}{cc}
       \toprule
       Description &  Value \\
       \midrule
       optimizer       &    Adam   \\
        $\alpha$        &    $10^{-4}$  \\
        $\beta_1$       &    $0.9$  \\
        $\beta_2$       &    $0.999$  \\
        $\varepsilon$-greedy $\varepsilon$   &    $10^{-5}$  \\
        clipping $\epsilon$       &   $0.2$ \\
        seed             &   $[0,10)$  \\
        number of process &  $64$  \\
        lr             &     $10^{-4}$  \\
        eval interval  &     $400000$  \\
        eval episodes  &     $100$   \\
        jump & $0.3$ \\
       \bottomrule
    \end{tabular}

   
\makeatletter\def\@captype{table}\makeatother\caption{Hyper-parameters of Group Consensus-oriented Strategy}
\label{tab:para2}
% \end{table}
\end{minipage}
\end{minipage}

By the way, 
we refer readers to the source code in the supplementary to check the detailed hyper-parameters.

\subsection{The Pseudo-code of CoS}
\label{app:algo}

The main procedures of our proposed CoS are summarized as Algorithm~\ref{algo:cos}.
% Our ultimate goal is to obtain the~\namefp~ $\{\pi^i\}_{i=1}^N$.
% % and the intermediate goal is to get the optimal graph generation policy $\rho$. 
% The~\namegg~$\rho$ is an intermediate for accessing
% a graph structure that guides the order of decisions among agents to achieve a high degree of coordination. 
% The graph generator is a pluggable module that is replaceable while solving DAGs. Note that the DAG is necessary because we need an execution structure that can determine a clear sequence of the actions, and therefore it should be directed rather than circular. The policy solver is a universal module, which, in general, one can choose from a diverse set of cooperative MARL algorithms~\cite{lowe2017maddpg,yu2021mappo,iqbal2019maac}.



\begin{algorithm}[ht!]
\caption{The pseudo-code of CoS.}
\label{algo:cos}
\textbf{Ensure} vector quantized group consensus $\rho=\{q, \{e^j\}_{j=1}^K, p\}$;\\
\textbf{Ensure} group consensus policy $\pi^{gc^i}$, group-guided policy $\pi^{gg^i}$ for agent $i$, and critic $Q_{\pi}$;\\
\textbf{Initialize}~the parameters $\theta$ 
for the vector quantized group consensus, the parameter $\phi^{gc^i}, \phi^{gg^i}$ for the group consensus policy and group-guided policy, where $i=1,...,N$, and
$\varphi$ for the critic network;\\
\textbf{Initialize}~jump\_rate=0.3;\\
\For {each episode}{
Initial state;\\
 \For{ each timestep}{
Compute history encoder output $z_e(\tau^i)=q_{\theta}(\cdot|\tau^i)$;\\
Quantize $z_e$ as $z_q$ with the group codebook $\bm{e}$;\\
Obtain the group consensus embedding $e^j$;\\
Reconstruct the input $\hat \tau^i = p_{\theta}(\cdot|z^q)$;\\
Compute the group-guided action $a^{{gg}^i_j}=\pi^{gg^i}(\cdot | o^i, e^j)$;\\
Obtain action for agent $i$ as $u^i=a^{gg^i_j}$;\\
\If{ $\#episodes > jump\_rate*total\_number$}{
Compute the group consensus action $a^{gc^i_j} = \pi^{gc^i}(\cdot|e^j)$;\\
Obtain action for the agent $i$ as $u^i=a^{gc^i_j}+a^{gg^i_j}$; \\
// Note: the sum of actions for continuous action space; the sum of action distributions for discrete action space;
}
Obtain the joint action $\bm{u}=\{u^1,u^2,...,u^N\}$;\\
Receive reward $r_t$ and observe next state $\{o_t^i\}_{i=1}^N$;\\
}
// Training \\
Update the group consensus policy with Equation~(\ref{eq:gcp});\\ 
Update the group-guided policy with Equation~(\ref{eq:ggp});\\ 
Update the vector quantized group consensus with Equation~(\ref{eq:vqgc});\\
Update the critic network with TD error.
}

\end{algorithm}




\subsection{The Detailed Description of GRF}
\label{app:grf}

\paragraph{\textbf{Observations}}
The environment exposes the raw observations as Table~\ref{tab:grf-info}.
We use the \textit{Simple115StateWrapper}\footnote{We refer the reader to:\url{https://github.com/google-research/football} for details of encoded information.} as the simplified representation of a game state encoded with 115 floats. 





\begin{sidewaystable}[!ht]
    \centering
    \caption{The detailed descriptions of the information included in the raw observation of GRF.}
\label{tab:grf-info}
    \begin{tabular}{cc|cc}
    \toprule
        \textbf{Information} & \textbf{Descriptions} & \textbf{Information} & \textbf{Descriptions} \\ \hline
        ~ & position of ball & ~ & ~ \\ 
        ~ & direction of ball & ~ & controlled player index \\ 
        Ball & rotation angles of ball & Controlled player & designated player index \\ 
        ~ & owned team of ball & ~ & active action \\ 
        ~ & owned player of ball & ~ & ~ \\ \hline
        ~ & position of players in left team & ~ & position of players in right team \\ 
        ~ & direction of players in left team & ~ & direction of players in right team \\ 
        Left Team & tiredness factor of players & Right Team & tiredness factor of players \\ 
        ~ & numbers of players with yellow card & ~ & numbers of players with yellow card \\ 
        ~ & whether a player got a red card & ~ & whether a player got a red card \\ 
        ~ & roles of players & ~ & roles of player \\ \hline
        ~ & goals of left and right teams & ~ & ~ \\ 
        Match state & left steps & Screen & rendered screen \\ 
        ~ & current game mode & ~ & ~ \\ \bottomrule
    \end{tabular}
\end{sidewaystable}






\paragraph{\textbf{Actions}}
The number of actions available to an individual agent can be denoted as $|\mathcal{A}| = 19$.
The standard move actions (in $8$ directions) include $\mathcal{A}_{move} = \{Top, Bottom, Left, Right,Top-Left, Top-Right, Bottom-Left, Bottom-Right\}$.
Moreover, the actions represent different ways to kick the ball is 
$\mathcal{A}_{kick} = \{Short Pass, High Pass,Long Pass, Shot,Do-Nothing,Sliding,Dribble,Stop-Dribble,Sprint, Stop-Moving, Stop-Sprint\}$.

\paragraph{\textbf{Rewards}}
The reward function mainly includes two parts. The first is $SCORING$, which corresponds to the natural reward where the team obtains $+1$ when scoring a goal and $-1$ when losing one to the opposing team. The second part is $CHECKPOINT$, which is proposed to address the issue of sparse rewards. It is encoded with  domain
knowledge by an additional auxiliary reward contribution.



\paragraph{\textbf{Scenarios}}
We conduct the experiments on the following scenarios.
\begin{itemize}
    \item academy\_3\_vs\_1\_with\_keeper: Three of our players try to score from the edge of the box, one on each side, and the other at the center. Initially, the player at the center has the ball and is facing the defender. There is an opponent keeper.
    \item academy\_counterattack\_easy: 4 versus 1 counter-attack with keeper; all the remaining players of both teams run back towards the ball.
    \item academy\_counterattack\_hard: 4 versus 2 counter-attack with keeper; all the remaining players of both teams run back towards the ball. 
\end{itemize}


\end{document}
%%%%%%%%%%%%%%%%%%%%%%%%%%%%%%%%%%%%%%%%%%%%%%%%%%%%%%%%%%%%%%%%%%%%%%
