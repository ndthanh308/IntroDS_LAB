\section{Conclusion}
\noindent In this work, we propose a model-agnostic recipe called MARIO (Model-Agnostic Recipe for Improving OOD Generalization) to address the challenges of distribution shifts in graph contrastive learning. Specifically, this recipe mainly aims to address the drawbacks of the main components (\ie, view generation and representation contrasting) in graph contrastive learning while facing distribution shifts motivated by invariant learning and information bottleneck principles. To the best of our knowledge, this is the first work that investigates the OOD generalization problem of graph contrastive learning specifically for node-level tasks. We conduct substantial experiments to show the superiority of our method on various real-world datasets with diverse distribution shifts. This research contributes to bridging the gap in understanding and addressing distribution shifts in graph contrastive learning, providing valuable insights for future research in this area. 
