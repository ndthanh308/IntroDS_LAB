\section{Proof of CMI Lower Bound\label{app:cmi}}
In this section, we provide a theoretical justification of why Equation~\ref{equ:con} is a lower bound of CMI. And some justifications are borrowed from~\cite{ma2021conditional,nguyen2010estimating}. Firstly, we present the following lemmas which will be used. 
\subsection{Fundamental Lemmas}
\begin{lemma}
    Let $U$ and $V$ be two random variables whose sample spaces are $\mathcal{U}$ and $\mathcal{V}$, $f: (\mathcal{U}\times\mathcal{V}) \rightarrow \mathbb{R}$ be a mapping function, and $\mathbb{P}$ and $\mathbb{Q}$ be the probability measures on $\mathcal{U}\times\mathcal{V}$, we can obtain:
\begin{equation}
D_{\mathrm{KL}}(\mathbb{P} \| \mathbb{Q})=\sup _f \mathbb{E}_{(u, v) \sim \mathbb{P}}[f(u, v)]-\mathbb{E}_{(u, v) \sim \mathbb{Q}}\left[e^{f(u, v)}\right]+1
\label{equ:kl}
\end{equation}
\label{lem:kl}
\end{lemma}
\emph{Proof.} The second-order functional derivative of the above function is $-e^{f(u, v)} \cdot d \mathbb{Q}$. This negative term means Equation~\ref{equ:kl} has a supreme value. Through setting the first-order functional derivative as zero $d \mathbb{P}-e^{f(u, v)} \cdot d \mathbb{Q}=0$, we can get the optimal mapping function $f^*(u, v)=\log \frac{d \mathbb{P}}{d \mathbb{Q}}$. Rewrite the Equation~\ref{equ:kl} with $f^*$:
\begin{equation}
\mathbb{E}_{\mathbb{P}}\left[f^*(u, v)\right]-\mathbb{E}_{\mathbb{Q}}\left[e^{f^*(u, v)}\right]+1=\mathbb{E}_{\mathbb{P}}\left[\log \frac{d \mathbb{P}}{d \mathbb{Q}}\right]=D_{\mathrm{KL}}(\mathbb{P} \| \mathbb{Q})
\end{equation}

\begin{lemma}
    Let $U$, $V$, and $Y$ be three random variables whose sample spaces are $\mathcal{U}$, $\mathcal{V}$ and $\mathcal{Y}$, $f: (\mathcal{U}\times\mathcal{V} \times \mathcal{Y}) \rightarrow \mathbb{R}$ be a mapping function, and $\mathbb{P}$ and $\mathbb{Q}$ be the probability measures on $\mathcal{U}\times\mathcal{V}\times\mathcal{Y}$, we can obtain:
\begin{equation}
\begin{aligned}
    D_{\mathrm{KL}}(\mathbb{P} \| \mathbb{Q})=\sup _f & \mathbb{E}_{(u, v, y) \sim \mathbb{P}}[f(u, v, y)]\\
                                                      & -\mathbb{E}_{(u, v, y) \sim \mathbb{Q}}\left[e^{f(u, v, y)}\right]+1
\end{aligned}
\label{equ:kl2}
\end{equation}
\label{lem:kl2}
\end{lemma}
\emph{Proof.} The second-order functional derivative of the above function is $-e^{f(u, v, y)} \cdot d \mathbb{Q}$. This negative term means Equation~\ref{equ:kl2} has a supreme value. Through setting the first-order functional derivative as zero $d \mathbb{P}-e^{f(u, v, y)} \cdot d \mathbb{Q}=0$, we can get the optimal mapping function $f^*(u, v, y)=\log \frac{d \mathbb{P}}{d \mathbb{Q}}$. Rewrite the Equation~\ref{equ:kl2} with $f^*$:
\begin{equation}
\begin{aligned}
    \mathbb{E}_{\mathbb{P}}\left[f^*(u, v, y)\right]-\mathbb{E}_{\mathbb{Q}}\left[e^{f^*(u, v, y)}\right]+1 & =\mathbb{E}_{\mathbb{P}}\left[\log \frac{d \mathbb{P}}{d \mathbb{Q}}\right]\\ & =D_{\mathrm{KL}}(\mathbb{P} \| \mathbb{Q})
\end{aligned}
\end{equation}

\subsection{Results based on Lemma~\ref{lem:kl}}
\begin{lemma}
    \begin{equation}
\begin{aligned}
\text{Weak-CMI}&(U ; V \mid Y)\\ & =D_{\mathrm{KL}}\left(P_{U, V} \| \mathbb{E}_{P_Y}\left[P_{U \mid Y} P_{V \mid Y}\right]\right) \\
& =\sup _f \mathbb{E}_{(u, v) \sim P_{U, V}}[f(u, v)]\\
& \quad\quad\quad -\mathbb{E}_{(u, v) \sim \mathbb{E}_{P_Y}\left[P_{U \mid Y} P_{V \mid Y}\right]}\left[e^{f(u, v)}\right]+1
\end{aligned}
\end{equation}
\label{lem:weak_cmi}
\end{lemma}
\emph{Proof.} Let $\mathbb{P}$ be the joint distribution $P_{U,V}$ and $\mathbb{Q}$ be expectation on the product of marginal distribution $\mathbb{E}_{P_Y}\left[P_{U \mid Y} P_{V \mid Y}\right]$ in Lemma~\ref{lem:kl}.

\begin{lemma}
    \begin{equation}
    \begin{aligned}
&\sup _f \mathbb{E}_{\left(u, v_1\right) \sim \mathbb{P},\left(u, v_{2: n}\right) \sim \mathbb{Q}^{\otimes(n-1)}}\left[\log \frac{e^{f\left(u, v_1\right)}}{\frac{1}{n} \sum_{j=1}^n e^{f\left(u, v_j\right)}}\right] \\ & \leq D_{\mathrm{KL}}(\mathbb{P} \| \mathbb{Q})
\end{aligned}
\end{equation}
\label{lem:kl_long}
\end{lemma}
\emph{Proof.} $\forall f$, we can draw:
\begin{equation}
\begin{aligned}
D_{\mathrm{KL}}& (\mathbb{P} \| \mathbb{Q})  =\mathbb{E}_{\left(u, v_{2: n}\right) \sim \mathbb{Q}^{\otimes(n-1)}}\left[D_{\mathrm{KL}}(\mathbb{P} \| \mathbb{Q})\right] \\
& \geq \mathbb{E}_{\left(u, v_{2: n}\right) \sim \mathbb{Q}^{\otimes(n-1)}}\left[\mathbb{E}_{\left(u, v_1\right) \sim \mathbb{P}}\left[\log \frac{e^{f\left(u, v_1\right)}}{\frac{1}{n} \sum_{j=1}^n e^{f\left(u, v_j\right)}}\right] \right.\\
    &\quad\quad \left. -\mathbb{E}_{\left(u, v_1\right) \sim \mathcal{Q}}\left[\frac{e^{f\left(u, v_1\right)}}{\frac{1}{n} \sum_{j=1}^n e^{f\left(u, v_j\right)}}\right]+1\right] \\
& =\mathbb{E}_{\left(u, v_{2: n}\right) \sim \mathbb{Q} \otimes(n-1)}\Bigg[\mathbb{E}_{\left(u, v_1\right) \sim \mathbb{P}}\left[\log \frac{e^{f\left(u, v_1\right)}}{\frac{1}{n} \sum_{j=1}^n e^{f\left(u, v_j\right)}}\right]\\ & \quad\quad\quad\quad\quad\quad\quad\quad\quad -1+1\Bigg] \\
& =\mathbb{E}_{\left(u, v_1\right) \sim \mathbb{P},\left(u, v_{2: n}\right) \sim \mathbb{Q} \otimes(n-1)}\left[\log \frac{e^{f\left(u, v_1\right)}}{\frac{1}{n} \sum_{j=1}^n e^{f\left(u, v_j\right)}}\right] .
\end{aligned}
\end{equation}
In detail, the first line always exists because $D_{\mathrm{KL}}(\mathbb{P} \| \mathbb{Q})$ is a constant. The second line comes from Lemma~\ref{lem:kl}. And because $\left(x, y_1\right)$ and $\left(x, y_{2: n}\right)$ can be interchangeable when they are all sampled from $\mathbb{Q}$, we can obtain the result in the third line. In conclusion, since the inequality works for $\forall f$, we can obtain Lemma~\ref{lem:kl_long}

\subsection{Results based on Lemma~\ref{lem:kl2}}
\begin{lemma}
    \begin{equation}
\begin{aligned}
\text{CMI}(U ; V \mid Y)&=\mathbb{E}_{P_Y}\left[D_{\mathrm{KL}}\left(P_{U, V \mid Y} \| P_{U \mid Y} P_{V \mid Y}\right)\right] \\
& =D_{\mathrm{KL}}\left(P_{U, V, Y} \| P_Y P_{U \mid Y} P_{V \mid Y}\right) \\
& =\sup _f \mathbb{E}_{(u, v, y) \sim P_{U, V, Y}}[f(u, v, y)]\\ &\quad -\mathbb{E}_{(u, v, y) \sim P_Y P_{U \mid Y} P_{V \mid Y}}\left[e^{f(u, v, y)}\right]+1 \\
&
\end{aligned}
\end{equation}
\label{lem:cmi}
\end{lemma}

\subsection{Proving $\text { Weak-CMI }(U;V \mid Y) \leq \text{CMI}(U;V \mid Y)$}
\begin{proposition}
    $\text { Weak-CMI }(U;V \mid Y) \leq \text{CMI}(U;V \mid Y).$
\end{proposition}
\emph{Proof.} According to Lemma~\ref{lem:weak_cmi},
\begin{equation}
\label{pro:weak}
\begin{aligned}
\operatorname{Weak-CMI}&(U ; V \mid Y)\\  & =\sup _f \mathbb{E}_{(u, v) \sim P_{U, V}}[f(u, v)]\\ & \quad\quad -\mathbb{E}_{(u, v) \sim \mathbb{E}_{P_Y}\left[P_{U \mid Y} P_{V \mid Y}\right]}\left[e^{f(u, v)}\right]+1 \\
& =\sup _f \mathbb{E}_{(u, v, y) \sim P_{U, V, Y}}[f(u, v)]\\ & \quad\quad -\mathbb{E}_{(u, v, y) \sim P_Y P_{U \mid Y} P_{U \mid Y}}\left[e^{f(u, v)}\right]+1
\end{aligned}
\end{equation}
When the equality for $\operatorname{Weak-CMI}$ holds, we assume the function as $f_1^*(u, v)$. And let $f_2^*(u, v, y)=f_1^*(u, v)$ which means $\forall y \sim P_Y$, $f_2^*(u, v, y)$ will not change. Then, we can get:
\begin{equation}
\begin{aligned}
\operatorname{Weak-CMI}&(U ; V \mid Y) \\  &=\mathbb{E}_{(u, v, y) \sim P_{U, V, Y}}\left[f_1^*(u, v)\right] \\ & \quad\quad -\mathbb{E}_{(u, v, y) \sim P_Y P_{U \mid Y} P_{V \mid Y}}\left[e^{f_1^*(u, v)}\right]+1 \\
& =\mathbb{E}_{(u, v, y) \sim P_{U, V, Y}}\left[f_2^*(u, v, y)\right]\\ & \quad\quad -\mathbb{E}_{(u, v, y) \sim P_Y P_{U \mid Y} P_{U \mid Y}}\left[e^{f_2^*(u, v, y)}\right]+1
\end{aligned}
\label{equ:weak}
\end{equation}

Comparing Equation~\ref{equ:weak} with Lemma~\ref{lem:cmi}, we can conclude $\text { Weak-CMI }(U;V \mid Y) \leq \text{CMI}(U;V \mid Y)$.

\subsection{Showing the Equation~\ref{equ:con} is a lower bound of CMI}
\begin{proposition}
    We restate the Equation~\ref{equ:con} in the main text, and call it as the estimate of CMI (CMIE):
\begin{small}
\begin{equation}
\begin{aligned}
\text{CMIE} & =\sup _f \underset{{y \sim P_Y}}{\mathbb{E}}\left[\mathbb{E}_{\left(u_i, v_i\right) \sim P_{U, V \mid y} \otimes n}\left[\log \frac{e^{f\left(u_i, v_i\right)}}{\frac{1}{n} \sum_{j=1}^n e^{f\left(u_i, v_j\right)}}\right]\right] \\
& \leq D_{\mathrm{KL}}\left(P_{U, V} \| \mathbb{E}_{P_Y}\left[P_{U \mid Y} P_{V \mid Y}\right]\right)\\ & =\text { Weak-CMI }(U ; V \mid Y) \leq \text{CMI}(U ; V \mid Y)
\end{aligned}
\end{equation}
\end{small}
\end{proposition}
\emph{Proof.} By defining $\mathbb{P}=P_{U,V}$ and $\mathbb{Q}=\mathbb{E}_{P_Y}\left[P_{U \mid Y} P_{V \mid Y}\right]$ we can obtain:
\begin{equation}
\begin{aligned}
\mathbb{E}_{\left(u, v_1\right) \sim \mathbb{P},\left(u, v_{2: n}\right) \sim \mathbb{Q} \otimes(n-1)}\left[\log \frac{e^{f\left(u, v_1\right)}}{\frac{1}{n} \sum_{j=1}^n e^{f\left(u, v_j\right)}}\right]= \\
\mathbb{E}_{y \sim P_Y}\left[\mathbb{E}_{\left(u_i, v_i\right) \sim P_{U, V \mid y} \otimes_n}\left[\log \frac{e^{f\left(u_i, v_i\right)}}{\frac{1}{n} \sum_{j=1}^n e^{f\left(u_i, v_j\right)}}\right]\right]
\end{aligned}
\end{equation}
Combined with Lemma~\ref{lem:kl_long}, we can deduce:
\begin{equation}
\begin{aligned}
\sup _f \mathbb{E}_{y \sim P_Y}\left[\mathbb{E}_{\left(u_i, v_i\right) \sim P_{U, V \mid y}{ }^{\otimes n}}\left[\log \frac{e^{f\left(u_i, v_i\right)}}{\frac{1}{n} \sum_{j=1}^n e^{f\left(u_i, v_j\right)}}\right]\right] \leq \\
D_{\mathrm{KL}}\left(P_{U, V} \| \mathbb{E}_{P_Y}\left[P_{U \mid Y} P_{V \mid Y}\right]\right)
\end{aligned}
\end{equation}
Through Proposition~\ref{pro:weak} that $\text { Weak-CMI }(U;V \mid Y) \leq \text{CMI}(U;V \mid Y)$, we can draw the conclusion that CMIE is a lower bound of CMI. 