%%
%% This is file `sample-authordraft.tex',
%% generated with the docstrip utility.
%%
%% The original source files were:
%%
%% samples.dtx  (with options: `authordraft')
%% 
%% IMPORTANT NOTICE:
%% 
%% For the copyright see the source file.
%% 
%% Any modified versions of this file must be renamed
%% with new filenames distinct from sample-authordraft.tex.
%% 
%% For distribution of the original source see the terms
%% for copying and modification in the file samples.dtx.
%% 
%% This generated file may be distributed as long as the
%% original source files, as listed above, are part of the
%% same distribution. (The sources need not necessarily be
%% in the same archive or directory.)
%%
%% Commands for TeXCount
%TC:macro \cite [option:text,text]
%TC:macro \citep [option:text,text]
%TC:macro \citet [option:text,text]
%TC:envir table 0 1
%TC:envir table* 0 1
%TC:envir tabular [ignore] word
%TC:envir displaymath 0 word
%TC:envir math 0 word
%TC:envir comment 0 0
%%
%%
%% The first command in your LaTeX source must be the \documentclass command.
% \documentclass[sigconf,review, anonymous, balance=false]{acmart}
\documentclass[sigconf,balance=false]{acmart}


%% NOTE that a single column version may required for 
%% submission and peer review. This can be done by changing
%% the \doucmentclass[...]{acmart} in this template to 
%% \documentclass[manuscript,screen]{acmart}
%% 
%% To ensure 100% compatibility, please check the white list of
%% approved LaTeX packages to be used with the Master Article Template at
%% https://www.acm.org/publications/taps/whitelist-of-latex-packages 
%% before creating your document. The white list page provides 
%% information on how to submit additional LaTeX packages for 
%% review and adoption.
%% Fonts used in the template cannot be substituted; margin 
%% adjustments are not allowed.
% \settopmatter{printacmref=false} % Removes citation information below abstract
% \renewcommand\footnotetextcopyrightpermission[1]{} % removes footnote with conference information in first column
% \pagestyle{plain} % removes running headers

\input{CUSTOM-commands-and-imports.tex}

%%
%% \BibTeX command to typeset BibTeX logo in the docs
\AtBeginDocument{%
  \providecommand\BibTeX{{%
    \normalfont B\kern-0.5em{\scshape i\kern-0.25em b}\kern-0.8em\TeX}}}

 % \setcopyright{none}
\copyrightyear{2023}
\acmYear{2023}
\setcopyright{rightsretained}
\acmConference[UIST '23]{The 36th Annual ACM Symposium on User Interface
Software and Technology}{October 29-November 1, 2023}{San Francisco, CA,
USA}
\acmBooktitle{The 36th Annual ACM Symposium on User Interface Software and
Technology (UIST '23), October 29-November 1, 2023, San Francisco, CA, USA}
\acmDOI{10.1145/3586183.3606806}
\acmISBN{979-8-4007-0132-0/23/10}

%% Rights management information.  This information is sent to you
%% when you complete the rights form.  These commands have SAMPLE
%% values in them; it is your responsibility as an author to replace
%% the commands and values with those provided to you when you
%% complete the rights form.
%\setcopyright{2023}
% \copyrightyear{2018}
% \acmYear{2018}
% \acmDOI{XXXXXXX.XXXXXXX}
%\setcopyright{none}
% \renewcommand\footnotetextcopyrightpermission[1]{} % removes footnote with conference information in first column



%% These commands are for a PROCEEDINGS abstract or paper.
% \acmConference[Conference acronym 'XX]{Make sure to enter the correct
%   conference title from your rights confirmation emai}{June 03--05,
%   2018}{Woodstock, NY}
%
%  Uncomment \acmBooktitle if th title of the proceedings is different
%  from ``Proceedings of ...''!
%
%\acmBooktitle{Woodstock '18: ACM Symposium on Neural Gaze Detection,
%  June 03--05, 2018, Woodstock, NY} 
% \acmPrice{15.00}
% \acmISBN{978-1-4503-XXXX-X/18/06}


%%
%% Submission ID.
%% Use this when submitting an article to a sponsored event. You'll
%% receive a unique submission ID from the organizers
%% of the event, and this ID should be used as the parameter to this command.
%%\acmSubmissionID{123-A56-BU3}

%%
%% For managing citations, it is recommended to use bibliography
%% files in BibTeX format.
%%
%% You can then either use BibTeX with the ACM-Reference-Format style,
%% or BibLaTeX with the acmnumeric or acmauthoryear sytles, that include
%% support for advanced citation of software artefact from the
%% biblatex-software package, also separately available on CTAN.
%%
%% Look at the sample-*-biblatex.tex files for templates showcasing
%% the biblatex styles.
%%

%%
%% For managing citations, it is recommended to use bibliography
%% files in BibTeX format.
%%
%% You can then either use BibTeX with the ACM-Reference-Format style,
%% or BibLaTeX with the acmnumeric or acmauthoryear sytles, that include
%% support for advanced citation of software artefact from the
%% biblatex-software package, also separately available on CTAN.
%%
%% Look at the sample-*-biblatex.tex files for templates showcasing
%% the biblatex styles.
%%

%%
%% The majority of ACM publications use numbered citations and
%% references.  The command \citestyle{authoryear} switches to the
%% "author year" style.
%%
%% If you are preparing content for an event
%% sponsored by ACM SIGGRAPH, you must use the "author year" style of
%% citations and references.
%% Uncommenting
%% the next command will enable that style.
%%\citestyle{acmauthoryear}



%%
%% end of the preamble, start of the body of the document source.
\begin{document}

%%
%% The "title" command has an optional parameter,
%% allowing the author to define a "short title" to be used in page headers.
%\title{\olio: A Data-driven Semantic Search Interface}
\title{\olio: A Semantic Search Interface for Data Repositories}
%% VS: Alternative title - A Semantic Search Interface for Supporting Analytical Inquiry and Data Exploration
%% VS: Alternative wordy title - A Semantic Search Interface for Dynamic and Pre-authored Content in Data Repositories

%% The "author" command and its associated commands are used to define
%% the authors and their affiliations.
%% Of note is the shared affiliation of the first two authors, and the
%% "authornote" and "authornotemark" commands
%% used to denote shared contribution to the research.
\author{Vidya Setlur}
\affiliation{%
  \institution{Tableau Research}
  \city{Palo Alto}
  \country{USA}}
\email{vsetlur@tableau.com}

\author{Andriy Kanyuka}
\affiliation{%
  \institution{Tableau Software}
  \city{Vancouver}
  \country{Canada}}
\email{akanyuka@tableau.com}

\author{Arjun Srinivasan}
\affiliation{%
  \institution{Tableau Research}
  \city{Seattle}
  \country{USA}}
\email{arjunsrinivasan@tableau.com}

%%
%% By default, the full list of authors will be used in the page
%% headers. Often, this list is too long, and will overlap
%% other information printed in the page headers. This command allows
%% the author to define a more concise list
%% of authors' names for this purpose.
\renewcommand{\shortauthors}{Setlur, et al.}

%%
%% The abstract is a short summary of the work to be presented in the
%% article.
%To enable reusing existing visualizations without requiring users to learn a new language or framework, previous research has examined how to obtain a semantic understanding of a chart by deconstructing its visual representation into reusable components such as encodings. However, existing deconstruction approaches are limited because they primarily focus on style, not layout. In this paper, we investigate how to identify multiple factors that can jointly determine the layout of a range of rectangle-based charts, representing diverse designs and styles. We adopt a mixed-initiative approach to extract the axes and legends and deconstruct a chart's layout into four semantic components: mark groups, spatial relationships, data encodings, and graphical constraints. Based on the deconstruction results, we design a wizard assistant interface to guide the authors through a series of steps to specify how these components map to their data. On 150 rectangle-based SVG charts representing diverse structures, Mystique achieves 90\% accuracy for axis and legend extraction and 96\% accuracy for chart deconstruction. In a chart reproduction study, participants could easily reuse existing charts on new datasets. We discuss the current limitations of Mystique and future research directions.
To facilitate the reuse of existing charts, previous research has examined how to obtain a semantic understanding of a chart by deconstructing its visual representation into reusable components, such as encodings. However, existing deconstruction approaches primarily focus on chart styles, handling only basic layouts. In this paper, we investigate how to deconstruct chart layouts, focusing on rectangle-based ones as they cover not only 17 chart types but also advanced layouts (e.g., small multiples, nested layouts). We develop an interactive tool, called Mystique, adopting a mixed-initiative approach to extract the axes and legend, and deconstruct a chart's layout into four semantic components: mark groups, spatial relationships, data encodings, and graphical constraints. Mystique employs a wizard interface that guides chart authors through a series of steps to specify how the deconstructed components map to their own data. On 150 rectangle-based SVG charts, Mystique achieves above 85\% accuracy for axis and legend extraction and 96\% accuracy for layout deconstruction. In a chart reproduction study, participants could easily reuse existing charts on new datasets. We discuss the current limitations of Mystique and future research directions.

%%
%% The code below is generated by the tool at http://dl.acm.org/ccs.cfm.
%% Please copy and paste the code instead of the example below.
%%
\begin{CCSXML}
<ccs2012>
   <concept>
       <concept_id>10003120.10003145</concept_id>
       <concept_desc>Human-centered computing~Visualization</concept_desc>
       <concept_significance>500</concept_significance>
       </concept>
   <concept>
       <concept_id>10002951.10003317.10003371</concept_id>
       <concept_desc>Information systems~Specialized information retrieval</concept_desc>
       <concept_significance>300</concept_significance>
       </concept>
   <concept>
       <concept_id>10003120.10003121.10003124.10010870</concept_id>
       <concept_desc>Human-centered computing~Natural language interfaces</concept_desc>
       <concept_significance>500</concept_significance>
       </concept>
 </ccs2012>
\end{CCSXML}

\ccsdesc[500]{Human-centered computing~Visualization}
\ccsdesc[300]{Information systems~Specialized information retrieval}
\ccsdesc[500]{Human-centered computing~Natural language interfaces}

%%
%% Keywords. The author(s) should pick words that accurately describe
%% the work being presented. Separate the keywords with commas.
\keywords{Hybrid search, question and answering, exploratory search, design search, federated querying, dynamic and static content, visualizations, curated data sources.}

%% A "teaser" image appears between the author and affiliation
%% information and the body of the document, and typically spans the
%% page.
\begin{teaserfigure}
  % Figure removed
  \caption{Examples of various semantic search scenarios supported in \olio{}. (A) \textit{Q\&A}. For an input query, \textit{``How has the trend of movie budgets changed over time for different genres?,''} \olio{} detects that it is a Q\&A search with an analytical intent, `trend.' A curated data source, `movies,' is the top-scored match to the query, and the system generates a multivariate line chart  response. A generated text summary describes the visualization as shown. Pre-authored visualization content is also displayed as thumbnails below the generated response as additional information. (B) \textit{Exploratory Search}. \olio{} identifies the input query, \textit{``elections,''} as a keyword search query and shows pre-authored visualizations with text content pertaining to `elections.' (C) \textit{Design Search}. The query, \textit{``treemap stocks''} is identified as a search of all content containing treemap visualizations pertaining to `stocks.' \olio{} returns a set of relevant pre-authored visualizations for the query and displays them as thumbnails. The thumbnails are linked to the actual visualizations if the user desires to continue with their analytical workflow.}
  \Description{}
  \label{fig:teaser}
\end{teaserfigure}


% \received{20 February 2007}
% \received[revised]{12 March 2009}
% \received[accepted]{5 June 2009}

%%
%% This command processes the author and affiliation and title
%% information and builds the first part of the formatted document.
\maketitle


% Figure environment removed



The understanding of 3D scene geometry is essential for many down-stream applications.  In robotics, it allows for accurate manipulation and motion planning considering the surrounding environment.  In the field of augmented reality, it allows for better mapping and rendering to bridge the virtual world to the real world.  With smartphones and robots that are equipped with high quality depth sensors, the task of 3D scene reconstruction is becoming feasible in these domains. 
%
These depth sensors allow for accurate reconstruction of the observed parts of the scene. However, to reconstruct the unseen parts, we must use prior information conditioned on the observed information. The missing information in the input image combined with the diversity in shapes, sizes, and depth distribution of the household objects presents a major challenge for scene reconstruction in-the-wild. 
%
In this paper, we study this problem in a general setting, where the goal is to reconstruct a complex scene with multiple novel objects, given only one RGB-D image of the scene.  
%



We present our method Rotate-Inpaint-Complete (\ours{}), which predicts both the 3D geometry and the texture of the unseen parts of the scene in the input image by leveraging the inpainting capabilities of large visual-language models.
%
Given an RGB-D image of a scene, first we generate novel views (RGB and depth images) by rotating and then projecting the input scene. Then we use a surface-aware masking method to select regions in the image to allow us to inpaint utilizing the powerful 2D inpainting capabilities of \dalle{}~\cite{ramesh2022hierarchical} for exposing the potential object geometry not visible in the input image. 
%
Finally, we optimize the depth images using the input depth values and occlusion boundaries and normals estimated from the inpainted images. These inpainted and completed novel RGB-D views provide the reconstructed scene geometry as a fused pointcloud with associated textures.
To mitigate the object hallucination and spatial inconsistency of predictions from \dalle{}, we integrate algorithmic features such as filtering inpainting outputs and enforcing consistency across viewpoints into our method that play a crucial role for generalizable, yet accurate and robust scene reconstruction.

%-------------------------------------------------------

We demonstrate our method on cluttered scenes with unseen household objects and categories. Through a series of rigorous quantitative experiments, we show that our approach outperforms baseline methods in settings where no training data is available.

% \subsection{Statement of Contributions}
In short, the contributions of this paper can be summarized as follows. \textit{i)} We present an integrated approach for scene completion of unseen objects under occlusion and clutter, by solving the problem through novel view inpainting and 2D to 3D scene lifting. \textit{ii)} We develop a method for selectively inpainting regions in the novel views of the input scene that enables synthesis of consistent 2D geometry. \textit{iii)} We train a 2D to 3D lifting method on the YCB-V~\cite{xiang2018posecnn} dataset and demonstrate the generalization capability on novel household objects and categories which is crucial for maintaining the generalization capability of our integrated scene reconstruction method.


\begin{table}[t]
\caption{Types and percentages of charts composed of lines, circles, pies, arcs, rectangles, and other marks in the Beagle dataset~\cite{battle2018beagle}.}
    \centering
    {\small
    \setlength\tabcolsep{1.5pt}
    \begin{tabular}{lp{0.725\linewidth}r}
    \toprule
     Mark   & Chart  & Percentage\\
     \midrule
      Rectangle & bar chart~(histogram), grouped bar chart, stacked bar chart, diverging bar chart (pyramid chart), Marimekko chart, heatmap, bullet chart, treemap, waffle chart, waterfall chart, range chart, gantt chart, matrix chart, cartogram, calendar chart & 32.85\%\\
      Line  &  line graph, parallel coordinates, Kagi chart &  30.51\%\\
      Pie &  pie chart, donut chart & 16.50\%\\
      Circle &  scatter plot, bubble plot, dot plot, circle packing & 14.96\%\\
      Others & geographic map, area chart, stream graph, chord chart, hexbin plot, Sankey diagram, Voronoi diagram, word cloud, sunburst chart, boxplot, network diagram, contour plot, radial plot & 5.18\%\\
     \bottomrule
    \end{tabular}
    }
    \label{tbl:lineCircleRect}
\end{table}

\section{Related Work}   
\subsection{\revise{Chart Reuse} Approaches}\label{sec:2.1}
\revise{To create new charts, previous studies on visualization designers' practices~\cite{bigelow_reflections_2014,walny_data_2019} suggest that it is more natural to change existing graphics than to start from scratch.}
Templates are generally recognized as a user-friendly way to create charts, especially for beginners. In traditional template-based systems, templates are created by system developers and they usually suffer from limited expressivity and quantity. 
Previous research thus has investigated how to turn existing visualizations into reusable templates without involving developers. For example, D3 Deconstructor \cite{harper_deconstructing_2014,harper_converting_2017} works on basic charts created using D3.js; iVoLVER \cite{mendez_ivolver_2016} extracts data from charts and updates them with new data; %However, they both support only basic chart types. 
Ivy \cite{mcnutt_integrated_2021} supports turning JSON-based declarative specifications into parameterized templates; 
% Mystique focuses on SVG charts, and does not require the examples to be created using a specification language. 
Chen~\etal~\cite{chen_towards_2020} use deep learning to extract timelines from infographics as templates; and Chartreuse \cite{cui_mixed-initiative_2022} supports reusing infographics bar chart templates. 

Overall, D3 Deconstructor and Chartreuse are the closest work to Mystique. D3 Deconstructor only takes charts created using D3 \cite{bostock_d3_2011}, which have the source data embedded, and Chartreuse primarily works on Microsoft PowerPoint graphics assets. In contrast, Mystique works on visualizations in the general SVG format, does not require access to underlying data, \revise{and supports more advanced layouts}. 
% We elaborate on the differences between Mystique and these tools in~\cref{sec:comparison}.

% Similar to Chartreuse, Mystique adopts a mixed initiative approach to visualization reuse, the primary difference lies in the types of visualizations. Chartreuse primarily works on Microsoft PowerPoint graphics assets and designs with highly customized glyphs. While the variation of glyph design is rich, the layout and structure of the visualizations are simple: Chartreuse only works on infographics bar charts. Mystique handles real-world SVGs with diverse and complex structures such as nested hierarchy and layouts resulting from multi-variate datasets even though it focuses on a rectangle mark.

% Mystique focuses on charts composed of rectangle marks, with an emphasis on layouts and nested structures. 

\subsection{Chart Understanding and Deconstruction}
Making a visualization example reusable requires understanding and deconstructing visualizations.  
Various automated or semi-automated methods have been proposed to detect marks \cite{ying_glyphcreator_2022,chen_towards_2020} as well as axes and legends \cite{shukla_recognition_2008, choudhury_scalable_2016}, classify chart types \cite{savva_revision_2011,shukla_recognition_2008}, and extract data \cite{jung_chartsense_2017,harper_converting_2017,harper_deconstructing_2014,masson2023chartdetective} and visual encodings \cite{poco_reverse-engineering_2017,poco_extracting_2018,harper_deconstructing_2014,harper_converting_2017,cui_mixed-initiative_2022}. Due to the vast space of visualization examples, these methods typically narrow the scope by focusing on specific glyph or chart types. 

\bpstart{Mark Detection}~Many approaches assume that input visualizations are in a raster image format, where object detection is essential. For example, GlyphCreator \cite{ying_glyphcreator_2022} focuses on circular glyphs, and uses deep learning to perform object and bounding box detection. Similarly, visual elements in timeline infographics can be identified using deep learning \cite{chen_towards_2020}. OCR is typically used to recognize text elements \cite{poco_reverse-engineering_2017}. Since our input format is SVG, mark detection is not necessary.

\bpstart{Axis and Legend Detection}~Simple heuristics \cite{shukla_recognition_2008,poco_extracting_2018} or supervised learning \cite{poco_reverse-engineering_2017} can be used to extract
axes and legends.
%Poco and Heer  detected axis and legend byclassifying text role into categories such as legend title, legend label, and axis label using SVM. 
However, these methods can still be error-prone. Since it is relatively easy to indicate where the axes and legends are, some tools expect users to provide such information \cite{poco_extracting_2018}. Mystique uses heuristics to find axes and legends, and provides a user interface for authors to correct potential mistakes.

\bpstart{Data Extraction} ~Previous work also addressed extracting data values from visualization images \cite{savva_revision_2011,jung_chartsense_2017} or vector graphics \cite{harper_converting_2017,harper_deconstructing_2014, masson2023chartdetective}. In Mystique, we demonstrate that a chart can be effectively reused without recovering the original data. Thus, data extraction is not necessary. 

\bpstart{Extraction of Visual Encoding and Spatial Arrangements} Inferring a visual encoding concerns the identification of relevant visual channel, data type, and potentially scale type. For glyphs with regular shapes (e.g., rectangles), visual encodings can be inferred using heuristics by combining information from mark type and axis \cite{poco_reverse-engineering_2017}. For custom glyphs (e.g., those used in infographics), \revise{sometimes the positions are not strictly encoded by data, but instead determined by specific spatial relationships or constraints. In these cases, current approaches usually classify charts into a predefined set of spatial arrangements~\cite{cui_mixed-initiative_2022,chen_towards_2020}. In Mystique, we break down the spatial arrangement of a chart into semantic components to handle more complex layouts.}

% \subsection{Mark and Chart Type Classification}

\bpstart{Chart Type Classification} Previous work also tackled the chart type classification problem. Most approaches are based on a simple chart taxonomy that roughly corresponds to mark types. For example, Revision \cite{savva_revision_2011} classifies chart images into 10 categories using SVM: area, bar, line, map, Pareto, pie, radar, scatter plot, table, and Venn diagram. This taxonomy is used in subsequent neural network-based methods \cite{poco_reverse-engineering_2017,jung_chartsense_2017}. In this work, we decided not to classify mark or chart types because such taxonomies are inadequate to capture the richness and variations of visualization design. Instead, we deconstruct charts into finer-grained semantic components.
\section{Identifying Search Scenarios for Data Repositories}
\label{sec:formative}

To better understand the types of search tasks people would find useful when searching over data repositories, we conducted a series of interviews.
We sought to collect a broad perspective from users spanning different backgrounds (e.g., programmers vs. non-programmers) and roles (e.g., visualization designers, consultants, casual viewers, or consumers).
We recruited $14$ participants \change{(7 females, 7 males)}, including seven visualization designers or consultants, three product managers involved in the design of visualization repositories, and four software engineers and designers. Participants had working experience with visualization repositories for tools like Tableau (e.g., Tableau Public), Microsoft Power BI (e.g., Power BI Partner Showcase), D3 (e.g., D3's Observable Example Gallery), and general experience searching for visualizations on Google. 


Interviews were conducted remotely and lasted 30-45 minutes. We asked participants about their backgrounds (e.g., their job descriptions, visualization repositories they use actively) and then asked them to share their experience, including the scenarios in which they search data or visualization repositories, current limitations, and areas for improvement in terms of the search experience, and metadata they find most relevant during visualization search.
We qualitatively analyzed the session transcripts and used an affinity diagramming approach to iteratively group similar comments (e.g., comments referring to searching for visualizations with a specific title or by an author, comments referring to using chart type as part of the search query).
We combined these groups under broader clusters of different scenarios search is used in as well as the most relevant search querying features.
Below, we summarize the key findings from our formative interviews in terms of the user goals and metadata features most relevant to \change{search in the context of data repositories containing both datasets and pre-authored charts}.

\subsection{Search Scenarios}

We identified three key user goals or scenarios for search in the context of data repositories.

\begin{itemize}[leftmargin=.15in]
    \item \textbf{Question \& Answering (Q\&A).}
    One common goal echoed by participants, particularly those who worked with organization-specific repositories hosting several data sources, was to leverage search to answer analytic questions.
    This goal is similar to information lookup~\cite{hearst2009search} in the broader web search context where user queries map to brief and discrete pieces of information (e.g., entities, dates, computed values).
    However, with data repositories, participants wanted to issue analytic questions (e.g., \textit{``What are sales trends across regions?,''} \textit{``highest covid cases by country''}) and get an appropriate response containing visualization and/or text generated from the available data sources.
    % Users' goal in this context is to find or retrieve a known item or a set of items given very specific criteria.
    % Some exemplars of targeted search queries include \textit{``\#makeovermonday,"} \textit{``mbostock force,"} or \textit{``Neil Richards"}.
    % Participants stated this was one of their most common use cases for search, especially in community-driven repositories hosting D3 and Tableau visualizations.
    % For instance, one of the visualization designers said \textit{``I often search for authors I know of or use a combination of author names and partial titles of the viz if I remember them. The ability to search for other authors and their work is integral to community initiatives."}
    % Another visualization designer (also a Tableau Zen Master), said \textit{``Most of my searches end up being of this [Q\&A search] type. I mostly remember who made something but not what it was called."}
    % Six participants (four designers/consultants, two product managers) commented that besides author names, hashtags (e.g., \#vizoftheday) on platforms like Tableau Public were another popular means to perform Q\&A search.
  
   
    \vspace{.5em}
   \item \textbf{Exploratory Search.}
    In line with the notion of exploratory search in web search~\cite{marchionini2006exploratory}, participants wanted to leverage data repositories to learn about a topic through available charts and data.
    % In this case, the users' primary goal is to see popular visualizations about a topic or learn about a topic through charts and their underlying data.
    Examples of exploratory search queries include \textit{``NFL drafts,''} \textit{``USA covid trends,''} or \textit{``Fifa world cup.''}
    Such queries are typically open-ended and do not provide refined filtering criteria beyond the topic itself. For instance, one participant (a visualization consultant) referred to exploratory search as one of his prominent goals during the initial stages of customer interactions.
    He highlighted the example of searching for visualizations on \textit{``private equity dashboards"} on Tableau Public during his recent interaction with a client at an investment firm.
    Describing her use cases for search, another participant (a visualization designer) alluded to exploratory search as one of her frequent search goals, stating \textit{``I often use search to see a few examples of what people create and to hunt for data sources about a topic."}
    
    \vspace{.5em}
    \item \textbf{Design Search.}
    The ability to find visualizations based on design features (e.g., chart type, color) was another popular use case for search, especially among the seven participants who were designers/consultants or novice visualization authors.
    Design search query examples include \textit{``sunburst chart,"} \textit{``bar and line combination chart,"} or \textit{``map with icons."}
    Based on anecdotes shared by the participants, this type of search is typically performed when users are looking for learning resources (e.g., a novice D3 developer looking for examples of force-directed layouts created with D3, a Tableau user trying to create a bespoke visualization like a Sankey diagram) or trying to understand design practices and find inspiration for their own work (e.g., using searches like \textit{``maps with a dark background''} to find examples of charts with specific color constraints).
    % , looking for examples of brushing-and-linking between specific chart types through a query like \textit{``dashboard with map and line chart linked"}).
    
\end{itemize}

% \noindent{}Note that these goals are not exhaustive.
% For instance, another goal that came up during the interviews was ``Question Answering," which was inspired by participants' experience with asking questions like \textit{``hottest months in Seattle?"} or \textit{``covid cases in California"} in Google and seeing a textual and/or visual response.
% However, we do not list this as a key search goal since this was only echoed by three participants and spans beyond the scope of a core visualization search problem on a public platform (e.g., to generate responses, the system would need a module to interpret a question, permission to access the data underlying hosted charts, map concepts from question to the data sources, and finally perform required calculations to compute an answer).
\noindent{}Note that these scenarios are neither exhaustive nor mutually exclusive.
For instance, \change{three participants mentioned ``targeted search'' as another scenario, where the intent was to retrieve a specific chart or dataset that the users knew existed in the repository.
However, we do not explicitly call this scenario out as it would inherently be supported by any search system that supports exploratory search (e.g., users can include specific and precise terms during exploratory search to retrieve the desired content). Furthermore,} 
queries like \textit{``sales by state and segment as a heatmap"} or \textit{``maps showing covid trends,"} combine Q\&A and design search, and design and exploratory search, respectively.
We also asked participants to rank the scenarios in terms of frequency/importance.
The responses, however, were fairly mixed and there was no single primary goal or a specific ordering of scenarios that stood out across participants.

Thus, rather than being a definitive and ordered set, the three scenarios listed above are primarily intended to serve as guidance for broad categories of user tasks to keep in mind when designing search systems for data repositories.

Besides understanding \textit{when} and \textit{why} people use search in data repositories (i.e., the above scenarios), to design and implement an effective search system, we also wanted to identify \textit{what} information people find most relevant while searching and browsing visualizations.
To this end, combining the participants' comments and search documentation for platforms like Observable~\cite{Observable:metadata} and Tableau Server~\cite{tableau_server_search}, we curated a list of the most prominent metadata fields that we focus on in our prototype.
These fields include the visualization title and description, the chart type (e.g., `bar chart,' `map,' `heatmap'), graphical encodings such as mark type, the visualization author, and the chart's creation date.



\subsection{Design Considerations}
Combining the feedback from the formative interviews with guidelines and findings from prior work on visualization search (e.g.,~\cite{viegas2007manyeyes,hoque2019searching,sechler2017sightline}), web and image search interfaces (e.g.,~\cite{hearst2009search,silverstein1999analysis,marchionini2006exploratory}), and NLIs for visual analysis (e.g.,~\cite{tory2019mean,shen2021towards,srinivasan2021collecting}).

\pheading{DC1. Support a unified experience that supports all three search scenarios.}
As we discussed the different search scenarios during the formative study, participants noted that they would ideally want the same interface and modality to perform the different tasks.
Thus, one consideration for us while building \olio{} was to design a seamless experience that supported a common input modality (NL) and blended Q\&A (a task commonly performed on data source collections) with exploratory and design search (tasks commonly performed with pre-authored visualization repositories).

\pheading{DC2. Support linguistic variations in queries.}
Both prior work on NLIs for visualization (e.g.,~\cite{tory2019mean,setlur2019inferencing,srinivasan2021collecting}) and web search (e.g.,~\cite{silverstein1999analysis,barr2008linguistic}) has shown that people use a variety of phrasings in search queries to accomplish the same goal.
Even during our interviews, participants used linguistically varied examples while discussing the same goal (e.g., ``\textit{What are sales trends across regions?}'' vs.~``\textit{sales by region over time}'').
Accommodating such user behavior, a second design consideration for \olio{} was that the system should support a variety of query formats - terse keywords as well as queries phrased as questions or sentence fragments, with an understanding of analytical intent relevant to data repositories in either case.

\pheading{DC3. Show textual responses and provide guidance for Q\&A queries.}
When discussing Q\&A scenarios, we asked participants about the types of visualizations they would expect for different queries.
During these conversations, in line with prior research on information lookup on the web~\cite{marchionini2006exploratory}, participants noted that besides charts, it may be valuable to ``\textit{provide a text response to a text query},'' suggesting the inclusion of complementary text along with a generated chart.
To this end, we noted that given a Q\&A query, \olio{} should not only select an appropriate data source and generate a chart but also text content that leverages the chart to help answer the input query.
Furthermore, since Q\&A queries can map to multiple data sources and users may not be aware of the available data source and fields, the system should guide users to ask questions (e.g., via query suggestions) and provide metadata information on the relevant data sources (e.g., available data fields and values to query).

\pheading{DC4. Provide visual summaries and filtering options for search results.}
One struggle that was echoed by several participants was that current visualization search systems do not provide an easy way to comprehend and sift through results beyond manual inspection.
To overcome this limitation with current systems, we noted that the system should provide visual summaries and support dynamic filtering~\cite{ahlberg1994visual} to help people get an overview, organize, and create meaningful facets of the visualization search results.


\section{\olio}

\olio~is designed as an interface that supports semantic search behavior by dynamically generating visualization responses and pre-authored visualizations from data repositories. Below, we describe \olio{'s} interface through a brief usage scenario and subsequently detail the key system components and implementation.

\subsection{Interface}


% Figure environment removed

% Figure environment removed


The interface initially shows a landing screen that displays a sampling of data sources available for Q\&A search (Figure~\ref{fig:start-screen}). A user can hover over a data source thumbnail and view its corresponding metadata information (\textbf{DC3}). The user then types a search query, \textit{``housing prices usa''} in the input text box (Figure~\ref{fig:interface}A). The system detects that token `usa' is a geographic location and searches for a relevant data source in its data repository. \olio~finds the housing data source to be a match, and a map is dynamically generated as a Q\&A response to the query (Figure~\ref{fig:interface}B). In addition, as part of exploratory search, the query tokens are used as keywords to match any pre-authored visualizations (\textbf{DC1}). A grid of thumbnails is displayed to serve as a preview to the user for browsing and exploration (Figure~\ref{fig:interface}C). Each thumbnail is hyperlinked to its corresponding visualization file that the user can choose to peruse in more detail or download to their local machine. The title, author name, and creation date of the visualization are displayed below each thumbnail to provide additional context. Scented widgets~\cite{willett2007scented} appear on the right side of the exploratory search panel to support faceted browsing of the pre-authored visualizations (Figure~\ref{fig:interface}D). The user can narrow down the search results by simultaneously applying one or more filters, namely, author name, visualization type, and the creation date (\textbf{DC4}).



\subsection{System Overview}
% Figure environment removed

\olio~is implemented as a web-based application using Python and a Flask backend connected to a Node.js frontend. We leverage Elasticsearch~\cite{elasticsearch}, an open-source Java search engine that is designed to be distributive, scalable, and with near real-time query execution performance. \change{As a result, \olio{} can scale to a large number of data repositories for indexing and search.} Figure~\ref{fig:overview} illustrates a high-level depiction of the system’s architecture, with the following main components: query classifier, parser, semantic search framework, Q\&A module, and the general search module.

A repository of curated data sources is included in the system for Q\&A search. The data sources could be any tabular CSV file, but for the purpose of this prototype, we include \change{eight} data sources across a variety of familiar topics such as sales~\cite{superstore,coffeesales}, sports~\cite{rgriffin_2018}, world events~\cite{covid-ca}, entertainment~\cite{bansal_2021}, and civic issues~\cite{us-crimes, housing}. \change{The datasources varied in the number of attributes as well as their cardinality, including 4-20 columns and $\sim$300-28,000 rows.}

% COVID-19~\cite{covid-ca}, college admissions, movies~\cite{bansal_2021}, coffee sales~\cite{coffeesales}, Superstore~\cite{superstore}, Olympics~\cite{rgriffin_2018}, housing~\cite{housing}, and crimes in the U.S.~\cite{us-crimes}.


\subsection{Data Repositories and Metadata}
\label{sec:metadata}
Unlike traditional document search, data sources and visualizations tend to be text-sparse, with limited searchable text content. Hence, \olio~augments the data repositories with additional metadata and semantics that helps the system's understanding and interpretation of the search queries. Specifically, attributes and values in the data sources are linked to ontological concepts, including synonyms (e.g., `film' and `movie')~\cite{thesaurus} and related terms (e.g., `theft,' `burglary,' and `crime')~\cite{word2vec}. The system includes a small hierarchy of hypernyms and hyponyms, from Wordnet~\cite{wordnet}, whose depth typically ranges up or down to two hierarchical levels (e.g., $[`beverage,' `drink'] \rightarrow [`espresso,' `cappuccino']$).
The metadata also includes data types (i.e., `text,' `date,' `Boolean,' `geospatial,' `temporal,' and ‘numeric’) and attribute semantics, such as currency type (e.g., United States Dollar). This information could also be inferred using existing data pattern matching techniques~\cite{pytheus,adelfio:2013,potterswheel}. The metadata also identifies attributes that are measures (i.e., attributes that can be measured, aggregated, or used for mathematical operations) and dimensions (i.e., fields that cannot be aggregated except as count). This final set of metadata information is then added to the semantic search framework.

The pre-authored content is a set of $75,000$ visualizations sourced from Tableau Public~\cite{tableau2023public}, a free community-based platform. The topics of the visualizations are reflective of that demographic of users and include themes such as natural calamities, health, world events, financial news, entertainment, and sports, for example.

Given the XML visual specification of the Tableau workbooks, the system traverses the DOM structure and indexes any text metadata that can be extracted from the visualizations, similar to techniques described in~\cite{d3deconstruction}. Extracted metadata includes the visualization title, caption, tags, description, author name, and profile, the visualization marks encoded in the visualization, and the visualization type. To support design search for recognizing visualization types mentioned in the search query (\textbf{DC2}), we include a general list of visualization types and their linguistic variants in the semantic search framework, as shown in Figure~\ref{fig:viztypes}.

% Figure environment removed


While we focused on CSV data sources and Tableau visualizations, the architecture for \olio~is extensible to include any new or additional data repositories, including D3 and Vega-lite charts, and knowledgebase articles, for example.

\noindent We now describe the rest of \olio's system components in detail.


\subsection{Query Classifier}
 \olio~takes as input an NL search query that is passed to the \textit{query classifier}. The classifier supports federated query search~\cite{Shokouhi2011FederatedS}, which is the process of distributing a query to multiple search repositories and combining results into a single, consolidated search result. Thus, for users, it appears as if they were interacting with a single search instance (\textbf{DC1}). In this context, a user can search \olio~over heterogeneous data repositories (i.e., both data sources and visualizations) without having to change or modify how they structure the query input. The query classifier passes the search tokens to a \emph{parser} and the \emph{data source search index} (which is part of the semantic search framework) and determines if ~\olio~needs to generate a Q\&A search to dynamically generate visualization responses, or simply general search that supports both exploratory and design searches. Algorithm 1 describes the query classification process. At a high level, the query classifier passes the query tokens to the parser (line 7) to determine if the query contains any analytic intents such as aggregation, correlation, temporal, or geospatial expression (refer to Section \ref{sec:parser} for more details). The query classifier also passes the query tokens to the semantic search framework (refer to Section \ref{sec:semanticsearch} for more details) to determine if the query tokens match fields in any of the data sources (e.g., `prices' $\rightarrow$ \texttt{Price} in the housing data source) and the normalized match score is greater than a predetermined threshold (line 10). In practice, we found that $fieldMatch = 2$ and $normMatch = .3$ provided a reasonable threshold for relevant data source matches. If both conditions, i.e., the presence of an analytical intent and the match score meets the threshold criteria, then Q\&A search is first invoked to dynamically generate visualization responses to the given query (line 13); else, general search is invoked to return pre-authored content from the data repository (line 16).

\begin{algorithm}
\caption{Classifies the search behavior based on whether the query contains an analytical intent and there is a match on one or more of the curated data sources in \olio.}
 \flushleft
\begin{algorithmic}[1]
\Function{QueryClassifier}{$query$}
    \LeftComment*{Boolean to check if there is an analytical intent in query}
    \State $hasAnalyticalIntent \leftarrow False$
    \LeftComment*{Boolean to check if there is a data source match}
    \State $hasDSMatch \leftarrow False$
    \LeftComment*{Contains the match scores for $query$ and each data source, $ds$}
    \State $dsScores \leftarrow getDSScores(query, ds)$
    \LeftComment*{Contains the normalized match scores for $query$ and each data source, $ds$}
    \State $normScores \leftarrow norm(dsScores)$ 
    \LeftComment*{Predetermined thresholds set for field match in $ds$ and $normScores$}
    \State $fieldMatch$, $normMatch$
     \LeftComment{Check if the parsed query contains an analytical intent}    
    \If{$(parseForAnalyticalIntent(query)$}
        \State $hasAnalyticalIntent \leftarrow True$  
    \EndIf
    \LeftComment{Check if the query tokens match fields in $ds$ and normalized match score to $ds$ is greater than a pre-determined thresold}     
    \If{$(dsScores['fields'] > fieldMatch)$ \textbf{and} $(normScores > normMatch))$}
        \State $hasDSMatch \leftarrow True$  
    \EndIf
    \LeftComment{If $query$ has an analytical intent and contains tokens matching a $ds$, invoke Q\&A search before general search, else just invoke general search.}
    \If{($hasAnalyticalIntent$ \textbf{and} $hasDSMatch$)}
        \State $invokeQ\&ASearch(query, ds)$
    \EndIf
        \State $invokeGeneralSearch(query)$
\EndFunction
\end{algorithmic}
\end{algorithm}

% Figure environment removed

\subsection{Parser}
\label{sec:parser}
The parser removes stopwords (e.g., `a', `the') and conjunctions / disjunctions (e.g., `and,' `or') from the search query and extracts a list of n-gram tokens (e.g., ``\textit{Seattle house prices}'' $\rightarrow$ [Seattle], [house], [prices], [house prices], [Seattle house prices], etc.). The parser employs a Cocke-Kasami-Younger (CKY) parsing algorithm ~\cite{cocke1969programming,kasami1966efficient,younger1967recognition} and generates a dependency tree to understand relationships between words in the query. 

The input to the underlying CKY parser is a context-free grammar with production rules augmented with both syntactic and semantic predicates to detect the following analytical intents in the search query:

\begin{tight_itemize}
\item \textbf{Grouping}. Partition the data into categories. E.g., `by' a data attribute.
\item \textbf{Aggregation}. Group values of multiple rows of data together to form a single value
based on a mathematical operation. E.g., `average,' `median,' `count,' `distinct count.'
\item \textbf{Correlation}. Statistical measure of the strength of the relationship between two data attributes (measures). E.g., `correlate,' `relate.'
\item \textbf{Filters and limits}. Finite sets of operators that return a subset of the data attribute's domain. E.g., `filter to,' ‘at least,’ ‘between,’ ‘at most.’ Limits are also a finite set of operators akin to filters that return a subset of the attribute’s domain, restricting up to n rows. E.g., `top,' `bottom.'
\item \textbf{Temporal}. Time and date expressions containing temporal tokens and phrases. E.g., `over time,' `year,` `in 2020', `when.'
\item \textbf{Geospatial}. Geospatial expressions referring to location and place. E.g., `in Canada,'  `by location,' `where.'
\end{tight_itemize}

To help with detecting data attributes and values along with the intents, the parser has access to the set of curated data sources and their metadata. The parser then compares the n-grams to available data attributes looking for both syntactic (e.g., misspellings) and semantic similarities (e.g., synonyms) using the Levenshtein distance~\cite{yujian2007normalized} and the Wu-Palmer similarity score~\cite{wu1994verbs}, respectively (\textbf{DC2}). If the parser detects one or more of the aforementioned analytical intents, it returns the intent(s) along with its corresponding data attributes and values to the query classifier. 


\subsection{Semantic Search Framework}
\label{sec:semanticsearch}
The semantic search framework primarily comprises two phases: indexing and searching content and metadata in the data repositories. This two-phase process applies to content in the data repositories, i.e., both the curated data sources and visualization content. Figure~\ref{fig:semanticsearchframework} illustrates the pipeline of the semantic search framework.

\subsubsection{Indexing}
The indexing phase creates indices for each of the data repositories (data sources and visualization content) along with their metadata to support federated search in \olio~(\textbf{DC1}).

Given a data source and visualization content with associated metadata (i.e., attributes, data values, chart type, author name), each file is represented as a document vector, $x_i$, where:
\useshortskip
\begin{equation}
    \mathcal{X} = \{x_1, x_2, ..., x_n\} 
\end{equation}

We also store n-gram string tokens from these document vectors to support partial and exact matches in the system (\textbf{DC2}):
\useshortskip
\begin{equation}
\mathcal{S} = \{s_1, s_2, ...s_n\}
\end{equation}

where $s_i = \varepsilon(x_i)$ for some encoder, $\varepsilon$ that converts the document vectors into a collection of string tokens of cardinality $n$. The original vectors $\mathcal{X}$ and encoded tokens $\mathcal{S}$ are stored in the semantic search engine index by specifying the \emph{mapping} of the content, i.e., defining the type and format of the fields in the index. \olio~stores the text as keywords in the index, supporting exact-value search, fuzzy matching to handle typos and spelling variations, and n-n-grams for phrasal matching. A scoring algorithm, tokenizers, and filters are specified as part of the search index \emph{settings} to determine how the matched documents are scored with respect to the input query and the handling of tokens, such as the adding of synonyms from a thesaurus, removal of stopwords (e.g., `a,' 'the,' for') and duplicate tokens, and converting tokens to lowercase. The complete configuration specification is provided in supplementary material.
%Figure~\ref{fig:indexmappings} shows a JSON snippet for how we specify the mappings and settings for the search index.  

% % Figure environment removed

\subsubsection{Search}
\label{sec:search}
Conceptually, the search phase has two steps: retrieval and ranking. Given an input query, $q$, that is represented as a query vector, $\hat{q}$ with query tokens $q_1, q_2, ..., q_j$; we encode the vector into string tokens, $\hat{s} = \varepsilon(\hat{q})$ using the same encoder, $\varepsilon$ from the indexing phase. The search process retrieves the most relevant $r$ document vectors, $\mathcal{R} = \{x_{1}, x_{2}, ... x_{r}\}$ as candidates based on the amount of overlap between the query string token set $\hat{s}$ and the document string tokens in $\{s_1, s_2, ..., s_n\}$. More specifically, the scoring function $r_{max}$ maximizes search relevance by computing:
\useshortskip
\begin{equation}
    \{x_1, x_2, ..., x_r\} = {r_{max}}_{i \in \{1, 2, ..., n\}} |\hat{s} \cap s_i |
\end{equation}

\noindent\olio~then ranks the vectors in the candidate search result set, $\mathcal{R}$ based on $BM25$ scoring~\cite{manning2008introduction} with respect to the query vector, $\hat{q}$. BM25 is essentially a bag-of-words retrieval scoring function that ranks documents based on the query terms appearing in each document, regardless of their proximity within the document. It is a preferred metric for computing similarities between vectors as the method corrects for variations in vector magnitudes resulting from uneven-length documents~\cite{manning2008introduction}. Given  $\hat{q}$, the BM25 score of a document vector, $x_i$ is:

\useshortskip
\begin{equation}
     {BM25}(\hat{q}, x_i) = \Sigma^{n}_{i=1}IDF(q_j).\frac{f(q_j, x_i).(k_1 + 1)}{f(q_j, x_i) + k_1. (1 - b + b . \frac{|x_i|}{avgdl})}
\end{equation}

where $f(q_j, x_i)$ is the number of times that  $q_{j}$ occurs in the document vector, $x_i$ and $avgdl$ is the average document vector length in the search index. $k_{1}$ and $b$ are constants to further optimize the scoring function. In practice, we have found that $k_1 \in [1.2,2.0]$ and $b = 0.75$ tend to provide reasonable ranking behavior. The Inverse Document Frequency, $IDF$, measures how often a term occurs in all of the documents and ranks unique terms in documents higher. It is computed as:

\useshortskip
\begin{equation}
IDF = ln(1 + \frac{(docCnt - f(q_j) + 0.5)}{f(q_j) + 0.5}
\end{equation}

where $docCnt$ is the total number of documents that have a value for the given query token, $q_j$ and $f(q_j)$ is the number of documents that contains the $i^{th}$ query term.

The $BM25$ scoring function sorts the vectors in descending order of normalized $BM25$ scores, $b \in [0,1]$, i.e., the higher the score, the higher the rank, creating the final ranked search result set,  $\mathcal{T}$, ranked based on the minimum difference between the query and each of the document vectors:

\useshortskip
\begin{equation}
    \mathcal{T}_{i \in \{i_i, i_2,...,i_r\}} = min(|\hat{q} - x_i|)
\end{equation}

% % Figure environment removed


The search request is then passed to the Elasticsearch server to compute Equations 3 and 4 and the system returns a ranked result set of either data sources (used for Q\&A) or visualization content used for both exploratory and design search scenarios.


\subsection{Q\&A Module}


% Figure environment removed


The \emph{Q\&A module} interprets the analytical intent expressed in the input search queries and dynamically generates visualization responses based on the list of top-matched data source(s) returned from the semantic search framework, as described in Section~\ref{sec:semanticsearch}. The module accepts tabular CSV datasets for the top-matched data source(s) as input, and all the visualizations in the tool are created using Vega-Lite~\cite{satyanarayan2016vega} and D3~\cite{2011-d3}. 

The interface and functionality for Q\&A search in \olio~is similar to that of NLIs for visual analysis~\cite{datatone,eviza,orko} with a few extensions that are inherent to the Q\&A behavior in the context of semantic search. For instance, the interface displays text showing a match (if any), to one or more data sources, along with a drop-down menu of the matched data sources  (\textbf{DC3}). A visualization is rendered based on attributes, values, and the analytical intent in the query, along with a text summary describing the visualization (refer to Figure~\ref{fig:teaser}A). A user can peruse the drop-down list of other data source alternatives, along with their corresponding percentage match scores (as computed in Section~\ref{sec:search}), and choose to switch to another data source in the drop-down list as shown in Figure~\ref{fig:qa-special-cases}A. In cases where there is a match to a data source for the query, but the tokens in the query do not resolve to valid attributes and values within the data source, \olio~displays suggested queries for the data source (\textbf{DC3}), shown in Figure~\ref{fig:qa-special-cases}B. These query suggestions are generated using a template-based approach presented by Srinivasan and Setlur~\cite{srinivasan2021snowy} that is based on a combination of attributes from the data source and data interestingness metrics.

The visualization generation process for Q\&A search supports three encoding channels (\texttt{x}, \texttt{y}, \texttt{color}) and four mark types (\texttt{bar}, \texttt{line}, \texttt{point}, and \texttt{geoshape}). These marks and encodings support the dynamic generation of bar charts, line charts, scatterplots, and maps that cover the range of analytic intents described in Section~\ref{sec:parser}. \olio~selects the default visualization using a simplified version of the Show Me system~\cite{mackinlay2007show}, employing similar rules to determine mark types based on the mappings between the visual encodings and attribute data types (e.g., showing a scatterplot if two quantitative attributes are mapped to the \texttt{xy}-channels and showing a line chart if a temporal attribute is visualized on the \texttt{x}-axis with a quantitative attribute on the \texttt{y}-axis).

Finally, \olio~displays a dynamic text summary describing the generated visualization (\textbf{DC3}). While template-based approaches~\cite{kim:2021,fasciano-lapalme-1996-postgraphe,mittal:1995} are viable options for the summary generation process, we chose to employ a large language model (LLM)-based approach~\cite{chatgpt} to explore its capabilities and better understand its limitations.
We initially attempted to pass the chart data as-is to ChatGPT to generate a description. However, we found the model was oftentimes generating wrong statistics or even hallucinating depending on the data domain context. To overcome these challenges but still provide an eloquent description, we instead opted for a combined approach of using both basic statistical computations and an LLM.

% The input to the LLM model is a prompt containing a statistical description returned with the generated visualization specification.
Specifically, the input to ChatGPT is a prompt containing a statistical description that is extracted from the generated visualization using a set of heuristics defined in prior data insight recommendation tools~\cite{demiralp2017foresight,cui2019datasite,srinivasan2018augmenting}.
For instance, for bar charts, we identify the min/max and average values; for scatterplots, we compute the Pearson's correlation coefficient~\cite{freedman2007statistics}, and so on. Consider the search query, \textit{``sales by region,''} which results in a bar chart displaying \texttt{Sales} across four \texttt{Region}s. An example of the statistical description, $keyStats$ from this bar chart is:
\begin{verbatim}
Region: Central has a minimum value of $220 for Sales
Region: South has the maximum value of $240 for Sales
Average Sales across Region is: $230
\end{verbatim}

\noindent The corresponding prompt to ChatGPT then becomes \textit{Rephrase the following input more eloquently: \escape{n}`\$\{keyStats\}\escape{n}'}, which ultimately generates the text summary: \textit{``The Sales in Central Region had the lowest value of \$220,  while South Region had the highest value of \$240. The average Sales across all Regions was \$230.}

\subsection{General Search Module}
% Figure environment removed

The \emph{general search module} displays thumbnails of pre-authored visualization content along with information such as title and date. The thumbnail images are hyperlinked to the corresponding Tableau Public workbook URLs if users choose to download or analyze the visualization in more detail. The module enables two types of searches: exploratory and design (\textbf{DC1}). Exploratory search returns visualization results based on keyword matches (\textbf{DC2}) in the input search query (e.g., \textit{``world population''} in Figure~\ref{fig:exploratory-search-examples}). Design search is a special form of exploratory search that returns visualization results specifically for keywords containing tokens referring to visualization types, their synonyms, and related concepts (e.g., \textit{``covid correlations''}) (\textbf{DC2}). Figure~\ref{fig:design-search-examples} shows examples of design search results in \olio.

% Figure environment removed
\section{Results \& Analysis} 
\label{sec:eval}
This section presents the user study results.
Unless otherwise indicated, results are based on the full set of participants.


\subsection{Solving times}
\label{sec:eval_solving_time}
This subsection addresses \textbf{RQ1:} \emph{How long do human users take to solve different types of \captchas?}
Figure~\ref{fig:captcha_solving_time} shows the the distribution of solving times for each \captcha type.
We observed a small number of extreme outliers where the participant likely switched to another task before returning to the study.
We therefore filtered out the highest $50$ solving times per \captcha type, out of $1,000$ total. %

For reCAPTCHA, the selection between image- or click-based tasks is made dynamically by Google.
Whilst we know that $85\%$ and $71\%$ of participants (easy and hard setting) were shown a click-based \captcha, the exact task-to-participant mapping is not revealed to website operators.
We therefore assume that the slowest solving times correspond to image-based tasks.
\todo{This seems to be entirely conjecture. Can you clarify this process. A comment on HotCRP is fine.}
After disambiguation, click-based reCAPTCHA had the lowest median solving time at $3.7$ seconds.
Curiously, there was little difference between easy and difficult settings.

The next lowest median solving times were for distorted text \captchas.
As expected, simple distorted text \captchas were solved the fastest.
Masked and moving versions had very similar solving times.
For hCAPTCHA, there is a clear distinction between easy and difficult settings.
The latter consistently served either a harder image-based task or increased the number of rounds.
However, for both hCAPTCHA settings, the fastest solving times are similar to those of reCAPTCHA and distorted text.
Finally, the game-based and slider-based \captchas generally yielded higher median solving times, though some participants still solved these relatively quickly (e.g., $< 10$~seconds).

With the exception of reCAPTCHA (click) and distorted text, we observed that solving times for other types have a relatively high variance. 
Some variance is expected, especially since these results encompass all input modalities across both direct and contextualized settings. 
However, \emph{relative differences in variances} indicate that, while some types of \captchas are consistently solved quickly, most have a range of solving times across the user population.
The full statistical analysis of our solving time results is presented in Appendix~\ref{sec:eval_statistical}.

\todo{I think a violin plot would be preferable to the box plots.}

% Figure environment removed

% Figure environment removed

\subsection{Preferences analysis}
\label{sec:eval_preferences}
This subsection addresses \textbf{RQ2:} \emph{What \Captcha types do users prefer?}
Figure~\ref{fig:pref_res} shows participants' \captcha preference responses after completing the solving tasks.
The \captcha types are sorted from most to least preferred by overall preference score, which is calculated by summing the numeric scores.
Since easy and difficult settings of hCAPTCHA are visually indistinguishable, we could only ask participants for one preference.

As expected, participants tend to prefer \captchas with lower solving times.
For example, reCAPTCHA (click) has the lowest median solving time and the highest user preference.
However, surprisingly, this trend does not seem to hold for game-based and slider-based \captchas, since these received some of the highest preference scores, even though they typically took longer than other types.
\todo{If there is room, it would be interesting to see a linear regression of these items per captcha.}
This suggests that factors beyond solving time could be contributing to participants' preference scores.
Notably, no single \captcha type is either universally liked or disliked.
For example, even the top-rated click-based reCAPTCHA, was rated 1 or 2 by $18.9\%$ of participants.
Similarly, over $31.0\%$ rated hCAPTCHA 4 or 5, although it had the lowest overall preference score.



% Figure environment removed



\subsection{Direct vs.\ contextualized setting}
\label{sec:eval_context}
This subsection addresses \textbf{RQ3:} \emph{Does experimental context affect solving time?}
Figure~\ref{fig:b_v_ub} shows histograms of \captcha solving times for participants in the direct vs.\ contextualized settings.
In every case except one, the mean solving time is lower in the direct setting.
In most cases, the distribution from the contextualized setting has more participants with longer solving times, i.e., a longer tail.

The largest statistically significant difference is in reCAPTCHA (easy click), where the mean solving time grows by $1.8$ seconds ($57.5\%$).
Second is Arkose (rotation), where it grows by $10$ seconds ($56.1\%$).
Across all \captcha types, the average increase from direct to contextualized is $26.7\%$.
Similarly, the mean solving time for reCAPTCHA (easy image) increased by $63.6\%$ in the contextualized setting showing the largest increase. 
However this was not statistically significant.
This is likely due to the skew of participants in direct and contextualized versions receiving image-challenges, which is controlled by Google.
Easy images were shown to 8.9\% of contextualized and to 17.2\% of direct setting participants, while hard images were shown to $25.5\%$ and $30\%$ respectively, resulting in different population sizes.

On the other hand, hCAPTCHA (difficult), which has the highest median solving time overall, showed no significant difference in mean solving time between direct and contextualized settings.
This may be attributable to the difficulty of solving this type of \captcha, regardless of the setting.

Results of Kruskal-Wallis tests confirm that there are statistically significant differences in mean solving times for all \captcha types ($p < 0.001$) except Geetest, reCAPTCHA (image) and hCAPTCHA (difficult).
While there were several potential confounding factors in our study, these results suggest that experimental context can have a significant impact on participants' \captcha solving times, and must therefore be taken into account in the design of future user studies.

\subsection{Effects of demographics}
\label{sec:eval_dem}
This subsection addresses \textbf{RQ4:} \emph{Do demographics affect solving time?}
We analyzed how demographic characteristics in our study correlate with \captcha solving times.
For some characteristics, such as education and gender, we did not observe large differences in \captcha solving times (see Figures~\ref{fig:gender_vs_time} and~\ref{fig:education_vs_time} in Appendix~\ref{sec:dem_appendix}).


\subsubsection{Effects of age}
\label{sec:eval_age}

Figure~\ref{fig:age_vs_time} shows the effect of participants' age on solving time.
The green line is the average solving time for each age, and the red line is a linear fit minimizing mean square error.
For all types, except reCAPTCHA (easy image), there is a trend of younger participants having lower average solving times.
This agrees with prior results~\cite{Bursztein} and is especially noticeable in hCAPTCHA, Arkose (selection), and Geetest.

\subsubsection{Effects of device type}
\label{sec:eval_device_type}
% Figure environment removed


Figure~\ref{fig:device_type_vs_time} shows the effect of device type.
Although there are some differences in median between device types for certain \captcha types, the Kruskal-Wallis test shows that the differences in means are mostly not statistically significant.
The only statistically significant differences are in distorted text \captchas ($p < 0.02$) and reCAPTCHA (hard click) ($p < 0.01$), where participants who used computers had a lower mean solving time compared to those using phones.
Interestingly, we found a statistically significant difference between participants who used physical keyboards and those who used touch input for the simple and masked distorted text \captchas ($p < 0.02$), as well as reCAPTCHA (hard click) ($p < .001$), reCAPTCHA (easy click) ($p < .05$), and Arkose (selection) ($p < .003$).
We found no statistically significant difference in mean solving times for moving distorted text. %

\subsubsection{Effects of typical Internet use}
% Figure environment removed

Figure~\ref{fig:internet_use_vs_time} shows the relationship between participants' self-reported dominant Internet usage patterns and their \captcha solving times. 
The Kruskal-Wallis test shows some initial evidence for statistically significant differences between participants who use the Internet primarily for work and those who use it for other purposes ($p < 0.05$).
The former were typically slower than the latter in 8 out of 12 \captchas. %
However, some categories do not have a sufficient number of participants, thus further investigation is recommended.


\subsection{Accuracy of \Captchas}
\label{sec:eval_accuracy}
Table~\ref{tab:bots_vs_humans} contrasts our measured human solving times and accuracy against those of automated bots reported in the literature.
Interestingly, these results suggest that bots \emph{can} outperform humans, both in terms of solving time and accuracy, across all these \captcha types.
As mentioned in Section~\ref{sec:study_limitations}, our decision to use unmodified real-world \captchas means we only have accuracy results for a subset of \captcha types (e.g., neither Geetest nor Arkose provide accuracy information).
For the same reason, our accuracy results also include participants who only partially completed the study. 


\taggedpara{reCAPTCHA:} The accuracy of image classification was 81\% and 81.7\% on the easy and hard settings respectively.
Surprisingly, the difficulty appeared not to impact accuracy.

\taggedpara{hCAPTCHA:} The accuracy was 81.4\% and 70.6\% on the easy and hard settings respectively.
This shows that, unlike reCAPTCHA, the difficulty has a direct impact on accuracy.

\taggedpara{Distorted Text:} We evaluated \emph{agreement} among participants as a proxy for accuracy.
As each individual \captcha was served to three separate participants, we measured agreement between any two or more participants.
We also observed that agreement increases dramatically (20\% on average) if responses are treated as case insensitive, as shown in Table~\ref{tab:accuracy}.


\begin{table}[h!]
\centering
\footnotesize
\caption{Humans vs.\ bot solving time (seconds) and accuracy (percentage) for different \captcha types.}
\label{tab:bots_vs_humans}
\begin{tabularx}{\columnwidth}{l l l l l}
\toprule
      &       \multicolumn{2}{c}{\textbf{Human}}                    & \multicolumn{2}{c}{\textbf{Bot}}  \\\cmidrule(lr){2-3}\cmidrule(lr){4-5}
\textbf{\Captcha Type}      &  Time                  &   Accuracy       &  Time                  &  Accuracy        \\ \midrule
reCAPTCHA (click) &   3.1-4.9                             &  71-85\%               & 1.4 \cite{Sivakorn2016}      & 100\% \cite{Sivakorn2016}              \\ \midrule
Geetest           &  28-30                            &  N/A                   & 5.3~\cite{Haiqin2019}            & 96\%  ~\cite{Haiqin2019}               \\ \midrule
Arkose            &  18-42                            &  N/A                   & N/A                              & N/A                                    \\ \midrule
Distorted Text    &  9-15.3                            &  50-84\%               & \textless{}1 ~\cite{Zi20}       & 99.8\% \cite{Goodfellow}              \\ \midrule
reCAPTCHA (image) &  15-26                              &  81\%                  & 17.5  \cite{Hossen2020}        & 85\% \cite{Hossen2020}                 \\ \midrule
hCAPTCHA          &  18-32                            &  71-81\%               & 14.9 \cite{Hossen2021}           & 98\% \cite{Hossen2021}                 \\ \bottomrule
\end{tabularx}

\vspace{\floatsep}
\caption{Agreement for distorted text \captchas.}
\label{tab:accuracy}
\begin{tabularx}{\columnwidth}{X c c}
\toprule
      & Average Agreement & Average Agreement (case insensitive) \\
	   \midrule
Simple & 84\%             & 93\%                                                                      \\
Masked & 50\%             & 73\%                                                                      \\
Moving & 62\%             & 90\%                                                                      \\
\midrule
Total  & 65\%             & 85\%                                                                      \\
\bottomrule
\end{tabularx}
\end{table}


\section{Measuring User Abandonment}
\label{sec:eval_abandonment}
This subsection addresses \textbf{RQ5:} \emph{Does experimental context influence abandonment?}
Upon completion, we observed that the number of \captchas solved during our study exceeded what would be expected based on the number of participants who completed the study.
We hypothesized that this was due to participants starting but not completing the study.
To measure this behavior, we conducted a second user study that collected timestamps between \captchas, regardless of whether the entire study was completed.
We measured: (1) how many participants started the task; (2) how many abandoned the task when solving a \captcha; and (3) if so, at which task and \captcha.

This abandonment-focused study consisted of four groups, each with $100$ unique participants.
Two groups were presented with the  direct setting and the other two with the contextualized setting (see Section~\ref{sec:B_vs_UB_SD}).
We hypothesized that the amount of compensation might also impact abandonment, so we doubled the compensation for one of the groups in each setting.
The studies were run sequentially to avoid prospective participants simply picking the higher-paying study.

We summarize the key findings below, and present the full results in Tables~\ref{tab:unbiased_75},~\ref{tab:unbiased_150},~\ref{tab:biased_30}, and~\ref{tab:biased_60} in Appendix~\ref{sec:appendix-abandonment}.
Out of a total of $574$ participants who started the study, $174$ abandoned prior to completion (i.e., $30\%$ abandonment rate).
Several observations can be made:
First, in the direct setting, $25\%$ of the participants who ultimately abandoned the study did so before solving the first \captcha, but this rose to nearly $50\%$ in the contextualized setting.
Second, doubling the pay halved the abandonment rate for the contextualized setting (as expected), but increased it by $50\%$ in the direct setting.
Third, participants in the contextualized setting were $120\%$ more likely to abandon than those in the direct setting.
Fourth, in the contextualized setting, participants at the higher compensation level solved \captchas faster than those at the lower compensation level ($21.5\%$ decrease in average solving time across all \captcha types).
Interestingly, in the direct setting, participants at the higher compensation level solved \captchas \emph{slower} than those at the lower compensation level ($27.4\%$ \emph{increase} in average solving time across all \captcha types).
Finally, some \captcha types (e.g., Geetest) exhibited higher rates of abandonment than others.

This initial investigation strongly motivates the need for further exploration of \captcha-induced abandonment.
Although we studied the impact of compensation and experimental context, there may be other reasons behind abandonment, such as: \captcha type, \captcha difficulty, and expected duration of study.
Nevertheless, the trend of average users' unwillingness to solve a \captcha during account creation (even for monetary compensation) is a relevant finding for websites that choose to protect account creation (and/or account access) using \captchas.

\section{Discussion}
\label{sec:discussion}

Besides user feedback on \olio{'s} support for different search scenarios, the study also helped identify high-level themes and user behaviors pertaining to the semantic search paradigm.

\pheading{Hybrid results facilitate a fluid and analytical search experience.}
When talking about the utility of the presented idea, participants particularly appreciated the \textit{complementary nature} of the dynamically generated content and the pre-authored visualizations.
Noting the benefits of each component, P9, for instance, said, ``\textit{if there's a question that can be answered using a data source then dynamically generated content like this is going to save a lot of time... But when I'm looking for inspiration, of course, that's not the best way, and what I would look for is work by actual people so definitely I see applications for both. None of them are mutually exclusive, and I was able to utilize both of them.}''
% Along similar lines, P8 compared \olio{'s} user experience to their prior search experience with Tableau and said ``having \textit{the ability to search by the dataset topic and then get related visualizations around them, I see that as a very highly desirable feature if I compare that to what I do in Tableau Public.}''
P2 viewed pre-authored content as a fail-safe for cases when there is no dynamic content stating, ``\textit{even if you don't have a dataset that's directly relevant to your query, if there are visualizations, then they come up immediately, which I really appreciate.}''
We also observed that the combination of the two content types encouraged participants to introspect on the data and findings more closely.
For example, P11 issued a query, \textit{``compare movies by genre''} that generated a bar chart from one of the available data sources, depicting that the \texttt{Action} genre has the highest number of movies.
However, she found a similar chart in the pre-authored set that showed a different result and correspondingly started inquiring about the data source, what dates it covered, if certain movies were excluded, etc.

\pheading{The link (or the lack thereof) between the dynamic chart and the pre-authored content should be more apparent.}
Although participants understood the differences between the two types of results, some participants were initially confused that the dynamic and pre-authored content did not stem from the same data source.
P7 alluded to this initial confusion about the visual layout of the page, stating that ``\textit{the page kind of creates a hierarchy that is difficult to break. I thought that there was the data I'm looking at at the top was getting visualized in different ways at the bottom, and that was that.}'' P2 suggested adding a button above the filters in the interface (Figure~\ref{fig:interface}D) to toggle the pre-authored results to only those that are created using the same data source as the dynamic result.
% When contrasted with the positive feedback on the value of hybrid content in the preceding point,
This feedback suggests that for the semantic search experience to be effective, systems like \olio{} should explore interface designs that clearly depict the relationship between content types, providing users the option to update the content ad-hoc.

\pheading{The inclusion of dynamically generated content changes user expectations.}
We noticed an intriguing change in the querying pattern for some participants (P3, P5, P7, P11) as they became familiar with the tool and experienced dynamic content as part of the results.
Specifically, once the system generated charts for a few queries, they switched from treating \olio{} as a search tool using keyword-style queries as input to more of an NLI, issuing imperative system commands like \textit{``Show me a chart of tuition cost by region''} and \textit{``Display examples of treemaps showing stock market data.''} 
While \olio{'s} query parsing logic was able to accommodate most phrasing variations, there were cases where the system no longer met the participants' expectations.
For instance, P3 issued a query, \textit{``show examples of charts displaying sales by state''} and \olio{} returned a map and bar chart for the Superstore data sources as part of its dynamic content along with other pre-authored charts matching the search query.
However, P3 was confused by this result as he expected the system to understand the phrase \textit{`show examples'} and ignore the data source search and dynamic chart rendering altogether.
When asked about the change in their querying patterns during the session, multiple participants (P3, P11) commented that it was a combination of \olio{} initially exhibiting an understanding of well-formed natural language utterances and their recent exposure to a slew of conversational interaction experiences through language models like ChatGPT.
Such mismatches in the system's functionality (supporting search) and the user's expectation (conversational interaction with an agent) could lead to errors in a larger scale setting, however, and should be clarified through a combination of interface techniques and system guidance.

\pheading{Textual descriptions should provide structure and contextual information.}
Participants' reactions to the system-generated descriptions were lukewarm at best, with only four participants (P4, P7, P9, and P10) commenting on them during the study.
During their comments, participants noted that the text was helpful in that it re-iterated the key facts from the chart, making it easy to interpret the chart, particularly when it was very dense with overlapping marks (e.g., a multi-series line chart or a scatterplot).
However, participants felt that ``\textit{text structure is too verbose}'' (P7) and ``\textit{lacks contextual information about what it means for a value to be high or low},'' (P11) minimizing its overall utility.
Such comments suggest that future systems investigating text generation in the context of data repository search should not only focus on the mapping between the generated text and chart, but also on the structure and degree of external information in the text itself.
\section{Limitations and Future Work}

Semantic search interfaces for data repositories hold promise for helping a user navigate and explore the growing amount of visualizations and analytical assets available. While \olio~received positive feedback as a research probe, research exploring semantic search for data repositories is still in its infancy. We identify various themes that highlight the challenges and opportunities for supporting semantic search that are unique to data repositories.

\pheading{Search precision depends on the availability and curation quality of data sources.} \change{Similar to other semantic search experiences~\cite{kaufmann:2006,klein-manning-2003-accurate,KOLOMIYETS20115412}, Q\&A search utilizes a small set of curated datasets to address analytical intents with focused responses.} However, an important aspect of search precision, especially for dynamically generated responses, is having access to high-quality, curated data sources with well-understood semantics. However, there is often a disconnect between environments where users publish content and downstream applications like search that consume the content. Participants echoed this challenge with P10 stating, ``\textit{this is a great interface and experience but will have to overcome the data garbage problem at scale}.''
Authors tend to perform some amount of curation during the publishing process but often are not provided sufficient tools to annotate, tag, or enrich their content. The process of curation is often tedious and time-consuming. More research should explore techniques (both semi-automated and automated)~\cite{potterswheel,wrangler,datacleaning:survey} to reduce the friction while curating content in data repositories; this includes the de-duplication of similar or near-similar content and the suggestion of topics and tags to help with content discoverability and faceting. \change{Future work should also explore techniques to help with data curation, such as employing LLMs for metadata enrichment, incorporating entity recognition, synonyms, and relational extraction to help automate curation for Q\&A support.}


\pheading{Incorporating additional analytical assets and metadata.} \olio~\\currently searches over pre-authored singleton visualizations. Future extensions should consider expanding the repertoire of analytical assets to include dashboards, data tables, and computational notebooks~\cite{Observable,jupyter}. These forms of content have interesting implications for interpreting analytical intent, Q\&A, and design search beyond data source and visualization repositories. Further, combining data repositories with document repositories could provide additional searchable metadata to improve search precision and for generating contextually relevant summaries alongside the results.

\pheading{Need for scaffolding to orient the user.}
Semantic search interfaces support new techniques for information seeking but with the added complexity of determining the type of queries and understanding the search results. Guidance and scaffolding may need to be provided as users search across multiple data repositories of content. While \olio~displays metadata for the available data sources along with query suggestions to guide a user toward a successful search, additional scaffolding could improve sensemaking and exploration. Recent work has explored data-driven autocompletion for helping users formulate targeted Q\&A-type queries~\cite{sneakpique} and integrate contextual query suggestions within a person’s sensemaking environment~\cite{interweave}. An interesting research direction would be to explore data scaffolds across different types of search, each unique in its own way, in the context of a semantic search system.

\pheading{Explore new search paradigms and modalities.} \olio~indexes available textual content in the data repositories. However, akin to image search, content-based search~\cite{cbr:2000} that leverages \emph{visual} features could improve recall of sparse text content, particularly for design search. Reverse image search~\cite{visualsearchpinterest} addresses the challenge for a user to guess at keywords and terms to return a specific result that they may have in mind. Exploring reverse visualization search, wherein a user provides a sample visualization or sketch to discover content related to the sample visualization image, could support richer forms of expressing design search goals. In addition to new search paradigms, other modalities, and platforms should be explored. Mobile devices, for example, generate large amounts of sensor footprints (e.g., GPS, motion sensors) and user activity data that are often missing from their desktop counterparts~\cite{franti2005mobile}. These new sources of implicit and explicit user feedback are valuable for discovering actionable content which is both situationally and contextually relevant to the user. Further, voice and touch modalities could open new possibilities for query formulation and browsing content in the data repositories. 


\pheading{Trust and provenance.} Trust is an important issue, and users would benefit from information that communicates the provenance of data sources used to generate the visualization responses, along with the ranking of pre-authored content. Exploring the inclusion of explanations for the search results could lead to increased transparency and understanding of the system behavior~\cite{ramos:2020}. There are additional challenges in an enterprise context; data and visualization content may be private to certain teams and organizations due to the sensitivity of the data (e.g., a human resources department or the current revenue forecast of a business). More work needs to explore ways to support built-in data privacy for indexing and searching of content within these organizational boundaries.


\pheading{Exploring the utility of LLMs for search.} Due to their ease of use and their fluent text-generative capabilities, LLMs are garnering attention for search and conversational interfaces~\cite{meyer:2022}. We explored the use of ChatGPT to generate a summary of the dynamically generated visualization response for Q\&A. The model does have limitations in the types of summaries it can generate (as described in Section~\ref{sec:discussion}) and challenges around higher-order numeracy reasoning~\cite{frieder2023mathematical}. Custom-trained GPT models could potentially bridge this gap in higher-order analytical reasoning if they can be trained on the data repositories employed in a semantic search system. In addition to summary generation, other utilities for these custom LLMs could explore automatic metadata generation from data repositories to enrich sparse searchable text content. Understanding the quality and accuracy of the generated text both for metadata ingestion and summary generation\change{, and comparing the resulting search experience to that of \olio{}}, are important research directions to pursue as future work.


% \begin{itemize}
  %  \item Publishing support (de-duplication, topic/tag recommendation)
 %   \item Result browsing experience (VizSummaries, summarizing not just content but interaction, form-factor considerations, example-based recommendation to support serendipitous discovery)
  %  \item Implementing features like autocomplete (loop in SneakPique). Highlight how collecting data through a design probe like \olio~can help with this.
   % \item Incorporating more metadata
    %\item Current system: individual charts, future work: dashboard search
   % \item Reverse image search + multimodal search.
 %   \item talk about GPT for summary generation
% \end{itemize}




\section{Conclusion}
\label{con}

We improved the overall code coverage by applying a DDQN agent 
that guided a TCN generator network. The DDQN agent was trained
based on the code coverage performance of generated test cases
to predict the next HTML tag in the test case. Our experiments 
have demonstrated how a TCN model can be used
to generate HTML test cases that discover basic blocks not
discovered by the underlying generation based fuzzer. Furthermore,
the results were improved by utilizing a DDQN agent that guided
the TCN model while generating test cases. 

Overall, the proposed system was able to improve the total code
coverage by up 18.9\% compared to the generation based fuzzer
used to generate the baseline. This highlights that the proposed
system is able to improve an existing fuzzer.

% \begin{enumerate}
%     \item proposed improving fuzzing design
%     \item combination of two deep learning models improve the overall outcome
% \end{enumerate}


%%
%% The acknowledgments section is defined using the "acks" environment
%% (and NOT an unnumbered section). This ensures the proper
%% identification of the section in the article metadata, and the
%% consistent spelling of the heading.
%\begin{acks}
% Acknowledgments here.
% \end{acks}

%\clearpage
%%
%% The next two lines define the bibliography style to be used, and
%% the bibliography file.
\bibliographystyle{ACM-Reference-Format}
\bibliography{main}

\end{document}
\endinput
%%
%% End of file.