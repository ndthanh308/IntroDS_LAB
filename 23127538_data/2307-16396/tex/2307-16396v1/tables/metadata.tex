\begin{table*}[ht!]
\centering
\resizebox{\textwidth}{!}{%
\begin{tabular}{@{}lll@{}}
\toprule
\textbf{Category} &
  \textbf{Feature (\#Participants)} &
  \textbf{Description} \\ \midrule
Visualization Features &
  Chart Type (X) &
  The name commonly used to refer to a visualization (e.g., sankey, treemap, lollipop chart, histogram). \\
 &
  Colors (X) &
  \begin{tabular}[c]{@{}l@{}}The salient colors in a visualization (e.g., the color scheme used for the marks) or its background\\ (e.g., dark versus light).\end{tabular} \\
 &
  Interactivity (X) &
  \begin{tabular}[c]{@{}l@{}}The type of interaction(s) supported in a visualization. This is particularly relevant for examples\\ with multiple visualizations (e.g., selecting one chart to highlight/update other charts in a dashboard,\\ expanding/collapsing links in a node-link diagram).\end{tabular} \\
 &
  Responsiveness (X) &
  \begin{tabular}[c]{@{}l@{}}A field indicating if a visualization is designed to adapt to varying screen sizes and devices\\ (e.g., do the axes or mark types update when a chart is viewed on a phone instead of a desktop).\end{tabular} \\
 &
  Encodings and Marks (X) &
  \begin{tabular}[c]{@{}l@{}}Lower-level visualization specification details that can be used to understand the chart's construction\\ (e.g., the mark/shape used such as bar, point, etc. or the graphical encoding channels such as row,\\ column, size, color etc. to which data is mapped).\end{tabular} \\ \midrule
Data Features &
  Topics (X) &
  \begin{tabular}[c]{@{}l@{}}A high-level word or phrase describing the topic or phenomenon represented by the visualization\\ (e.g., covid19,  olympic games, US presidential elections, housing rates).\end{tabular} \\
 &
  Data Fields (X) &
  Titles of data attributes or fields displayed in a visualization (e.g., sales, region, profit). \\
 &
  Datasource (X) &
  \begin{tabular}[c]{@{}l@{}}A reference to the source from which a visualization's data originated (e.g., imdb database, \\ Our World In Data).\end{tabular} \\ \midrule
\begin{tabular}[c]{@{}l@{}}Platform/Community\\ Features\end{tabular} &
  Date (X) &
  The date on which the visualization was posted on the platform. \\
 &
  Hashtags (X) &
  \begin{tabular}[c]{@{}l@{}}Platform-specific tags associated with a visualization. These hashtags could refer to a visualization's\\ design or data topic (e.g., \#sankey, \#covid), or platform specific events (e.g., \#makeovermonday,\\ \#ironviz, \#vizoftheday on Tableau Public).\end{tabular} \\
 &
  Viz Popularity (X) &
  \begin{tabular}[c]{@{}l@{}}A metric representing the viewer engagement level with a visualization (e.g., view count, download\\ count, number of times a chart is favorited).\end{tabular} \\
 &
  Author Popularity (X) &
  \begin{tabular}[c]{@{}l@{}}A metric representing an author's reputation in terms of the number of people that follow an author or \\ based on a cumulative popularity of the content they have authored (e.g., number of times an author's\\ visualizations have been favorited).\end{tabular} \\ \bottomrule
\end{tabular}%
}
\vspace{.1em}
\caption{Metadata table caption.}
\label{tab:metadata}
\end{table*}