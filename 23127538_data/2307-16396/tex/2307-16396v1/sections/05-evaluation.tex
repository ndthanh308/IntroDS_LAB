\section{Preliminary User Study}

% \begin{itemize}
%     \item Give participants three search tasks/scenarios (e.g., give an image for design search, give a scenario for exploratory, give a full thumbnail for targeted).
%     \item Have open-ended phase.
%     \item When results are incorrect, see if participants can adjust weights to get better results. Use these adjustments to discuss trade-offs between metadata types and potential considerations for an ``advanced search" interface.
%     \item Observe if and what role highlighting entity type in the query plays.
% \end{itemize}

Using \olio{} as a design probe, we conducted a preliminary user study to qualitatively assess the overarching idea of combining dynamically generated visualizations with pre-authored charts when searching data repositories. \change{Note that while a comparison of \olio{} with other systems would be helpful in identifying their relative strengths and weaknesses, the current state-of-the-art semantic search engines~\cite{Bing,google,ding:2005} focus on web documents rather than data repositories. In addition, existing visualization search systems~\cite{hoque2019searching,sechler2017sightline} focus on a subset of search functionality supported in \olio{}. Removing individual components for an ablation study would be challenging due to \olio{}'s unified hybrid search behavior. However, \olio{} does implement industry-standard performant recommendations like BM25 scoring and ElasticSearch indexing.}


\subsection{Participants and Setup}

We recruited $11$ participants (P1-P11, \change{6 males and 5 females}) through a mailing list at a data analytics software company.
Based on self-reporting by the participants, five participants frequently searched for data or visualization content on data repositories, four participants had some experience with searching data repositories but did so infrequently, and two participants had minimal experience with search in the context of data and visualizations.

All sessions were conducted remotely via the Cisco WebEx video conferencing software~\cite{webex}.
The prototype was hosted on a local server running on the experimenter’s laptop\footnote{2.4 GHz MacBook Pro running macOS Ventura 13.2.1 set to a resolution of 3072 $\times$ 1920.}.
Participants were granted control over the experimenter’s screen during the session, and all studies followed a think-aloud protocol.
The audio, video, and on-screen actions were recorded for all sessions with permission from the participants.

\subsection{Procedure}

Sessions lasted between 39-60 minutes (mean: 46 min.) and were organized as follows:

\pheading{Introduction} [$\sim$10min]: After providing an overview of the study goal, the experimenter asked participants about their job roles and prior experience with search, particularly in the context of data and visualization. The participants were provided a brief introduction to \olio{'s} interface, highlighting the four key components listed in Figure~\ref{fig:interface}. \change{Consistent with Jeopardy-style evaluations of prior NLIs for visualization~\cite{datatone}, to} avoid biasing participants, we did not provide any explicit training or queries and instead allowed participants to implicitly discover the system through the study tasks.

\pheading{Task Phase} [$\sim$25min]: Participants were asked to perform four tasks: one task corresponding to each search goal listed in Section~\ref{sec:formative} and a fourth open-ended task where participants were allowed to freely explore the available data sources and pre-authored visualizations.

For the \textit{Q\&A} task, participants were asked to use one or more of the available data sources for a Jeopardy-style fact~\cite{datatone} about college admissions and a directed analysis question of ``listing 1-3 insights on differences between movie genres.''

For \textit{exploratory} search, participants were asked to use \olio{} to explore the topics of elections and colleges in the US. Participants were encouraged to use any search terms and phrase queries however they saw fit.

To assess \olio{'s} support for \textit{design} search, participants were given two images and were asked to search for similar examples using the tool. 
The images included a treemap showing stock data and a choropleth map of US states with an overlaid pie chart showing product sales data.

\pheading{Debrief} [$\sim$10min]: Sessions concluded with a semi-structured interview discussing the overall experience and utility of the underlying idea, support for different search goals, and areas for improvement.

\subsection{Results}

Overall, participants noted that the semantic search paradigm was useful and could help accomplish their search goals in the context of data repositories. Below, we detail participant feedback and usage behavior with respect to the three search goals listed in Section~\ref{sec:formative}.

\pheading{Q\&A.}
All participants successfully completed the two Q\&A tasks and generally appreciated the system's ability to interpret different phrasing variations (e.g., \textit{``tuition across us regions,''} \textit{``compare movie genres,''} \textit{``What were covid cases across countries?'\''}).
P9, for instance, said, ``\textit{I think the system did better than what I would expect in terms answering questions even though my questions were not good enough to begin with.}''
Participants also used the system's ability to dynamically generate visualizations for in-place data exploration.
For example, P3 issued a query, \textit{``What are movie budgets by genre?''} that resulted in a bar chart showing average \texttt{Budget} by \texttt{Genre}.
Then, using the metadata tooltip (Figure~\ref{fig:start-screen}), he inspected other fields to notice the \texttt{Gross} field and issued a query to visualize both \texttt{Budget} and \texttt{Gross}.
P6, P7, and P9 also exhibited a similar behavior on multiple instances suggesting that the dynamic content promoted a state of analytic flow.
Participants also appreciated the ability to view and choose from matched data sources (Figure~\ref{fig:qa-special-cases}A).
Specifically, participants commented that \olio{} provided the freedom to ``\textit{get more with less}'' by supporting keyword-based or open-ended queries to retrieve multiple data sources instead of focusing on well-phrased queries that were optimized to match a single data source.
Proposing an improvement to the current interface, however, P10 suggested that instead of rendering a visualization by default, when there are multiple data source matches, the system could allow first choosing a data source and then rendering the chart to save computation resources at scale.


\pheading{Exploratory search.}
Participants commented that \olio{} returned appropriate sets of pre-authored charts during open-ended exploratory searches (e.g., `elections,' `olympics winners,' `covid trends').
However, we noticed that the quality of search results deteriorated when queries went beyond keywords and included additional information such as location (e.g., `election results in Maryland'), subjective concepts (e.g., `safest cities in the us'), or metadata properties like `popularity' that were not included in our chart corpus (e.g., `popular NBA charts').
Although there were mixed reactions to the quality of search results for exploratory scenarios, all participants appreciated the form and function of the dynamic filtering widgets (Figure~\ref{fig:interface}D), commenting they ``\textit{loved it}'' (P8, P11) and asked ``\textit{why these [dynamic filtering widgets] don't exist in all systems today?}'' (P10).
Participants predominantly used filters to facet the search results (e.g., choosing chart types to focus on a subset of results or specifying a time range to focus on recent results).
On two occasions, participants (P6, P11) also leveraged the filters to chronologically compare search results.
When exploring the topic of `us elections,' for instance, P11 used the date range slider widget to focus on charts created during the 2020 elections to those created using the 2016 elections.

\pheading{Design search.}
All participants successfully completed the two design search tasks except for P1 and P5, who found only one of the two required charts.
Overall, participants were very positive about the system's support for searching for visualizations by design features, with P8 stating, \textit{``I would love to have this in Tableau Public today.''}
Similarly, P7 also noted, \textit{``if all we do from this system is enable this search by design [in other visualization repositories], I'd argue that it can solve a lot of challenges for chart authors and especially for someone new to visualization tools.''}
In terms of user behavior, as we expected, some participants (6 out of 11) did not recollect `treemap' as a chart type and instead used the data topic, the closest chart type they could think of, or the mark type (e.g., `stock heatmap,' `finance group blocks,' `square chart').
However, since \olio{} inspects the content of the chart (e.g., titles) as well as design features (e.g., chart type, mark type), the system was able to return relevant charts as some of its top results.
In combination with the chart type filter widget, this feature enabled participants to reliably find example charts even when they could not precisely describe them.
Participants also successfully used a variety of phrasings to find the combined map and pie chart (e.g., \textit{``examples of piemaps,''} \textit{``pie+map,''} \textit{``show me charts with sales on a map with overlaid pie charts''}).
% That said, the use of analytic intents to find charts did not work as effectively during the study.
% For instance, to find the combined map and pie chart, P2 issued the query `sales by state by type' with the assumption that the system could interpret that query included two categorical fields and a numeric field and would suggest the required chart.
% However, the titles overrode the 