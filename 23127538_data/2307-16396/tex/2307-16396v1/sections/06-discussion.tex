\section{Discussion}
\label{sec:discussion}

Besides user feedback on \olio{'s} support for different search scenarios, the study also helped identify high-level themes and user behaviors pertaining to the semantic search paradigm.

\pheading{Hybrid results facilitate a fluid and analytical search experience.}
When talking about the utility of the presented idea, participants particularly appreciated the \textit{complementary nature} of the dynamically generated content and the pre-authored visualizations.
Noting the benefits of each component, P9, for instance, said, ``\textit{if there's a question that can be answered using a data source then dynamically generated content like this is going to save a lot of time... But when I'm looking for inspiration, of course, that's not the best way, and what I would look for is work by actual people so definitely I see applications for both. None of them are mutually exclusive, and I was able to utilize both of them.}''
% Along similar lines, P8 compared \olio{'s} user experience to their prior search experience with Tableau and said ``having \textit{the ability to search by the dataset topic and then get related visualizations around them, I see that as a very highly desirable feature if I compare that to what I do in Tableau Public.}''
P2 viewed pre-authored content as a fail-safe for cases when there is no dynamic content stating, ``\textit{even if you don't have a dataset that's directly relevant to your query, if there are visualizations, then they come up immediately, which I really appreciate.}''
We also observed that the combination of the two content types encouraged participants to introspect on the data and findings more closely.
For example, P11 issued a query, \textit{``compare movies by genre''} that generated a bar chart from one of the available data sources, depicting that the \texttt{Action} genre has the highest number of movies.
However, she found a similar chart in the pre-authored set that showed a different result and correspondingly started inquiring about the data source, what dates it covered, if certain movies were excluded, etc.

\pheading{The link (or the lack thereof) between the dynamic chart and the pre-authored content should be more apparent.}
Although participants understood the differences between the two types of results, some participants were initially confused that the dynamic and pre-authored content did not stem from the same data source.
P7 alluded to this initial confusion about the visual layout of the page, stating that ``\textit{the page kind of creates a hierarchy that is difficult to break. I thought that there was the data I'm looking at at the top was getting visualized in different ways at the bottom, and that was that.}'' P2 suggested adding a button above the filters in the interface (Figure~\ref{fig:interface}D) to toggle the pre-authored results to only those that are created using the same data source as the dynamic result.
% When contrasted with the positive feedback on the value of hybrid content in the preceding point,
This feedback suggests that for the semantic search experience to be effective, systems like \olio{} should explore interface designs that clearly depict the relationship between content types, providing users the option to update the content ad-hoc.

\pheading{The inclusion of dynamically generated content changes user expectations.}
We noticed an intriguing change in the querying pattern for some participants (P3, P5, P7, P11) as they became familiar with the tool and experienced dynamic content as part of the results.
Specifically, once the system generated charts for a few queries, they switched from treating \olio{} as a search tool using keyword-style queries as input to more of an NLI, issuing imperative system commands like \textit{``Show me a chart of tuition cost by region''} and \textit{``Display examples of treemaps showing stock market data.''} 
While \olio{'s} query parsing logic was able to accommodate most phrasing variations, there were cases where the system no longer met the participants' expectations.
For instance, P3 issued a query, \textit{``show examples of charts displaying sales by state''} and \olio{} returned a map and bar chart for the Superstore data sources as part of its dynamic content along with other pre-authored charts matching the search query.
However, P3 was confused by this result as he expected the system to understand the phrase \textit{`show examples'} and ignore the data source search and dynamic chart rendering altogether.
When asked about the change in their querying patterns during the session, multiple participants (P3, P11) commented that it was a combination of \olio{} initially exhibiting an understanding of well-formed natural language utterances and their recent exposure to a slew of conversational interaction experiences through language models like ChatGPT.
Such mismatches in the system's functionality (supporting search) and the user's expectation (conversational interaction with an agent) could lead to errors in a larger scale setting, however, and should be clarified through a combination of interface techniques and system guidance.

\pheading{Textual descriptions should provide structure and contextual information.}
Participants' reactions to the system-generated descriptions were lukewarm at best, with only four participants (P4, P7, P9, and P10) commenting on them during the study.
During their comments, participants noted that the text was helpful in that it re-iterated the key facts from the chart, making it easy to interpret the chart, particularly when it was very dense with overlapping marks (e.g., a multi-series line chart or a scatterplot).
However, participants felt that ``\textit{text structure is too verbose}'' (P7) and ``\textit{lacks contextual information about what it means for a value to be high or low},'' (P11) minimizing its overall utility.
Such comments suggest that future systems investigating text generation in the context of data repository search should not only focus on the mapping between the generated text and chart, but also on the structure and degree of external information in the text itself.