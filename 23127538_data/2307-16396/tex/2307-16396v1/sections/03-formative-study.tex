\section{Identifying Search Scenarios for Data Repositories}
\label{sec:formative}

To better understand the types of search tasks people would find useful when searching over data repositories, we conducted a series of interviews.
We sought to collect a broad perspective from users spanning different backgrounds (e.g., programmers vs. non-programmers) and roles (e.g., visualization designers, consultants, casual viewers, or consumers).
We recruited $14$ participants \change{(7 females, 7 males)}, including seven visualization designers or consultants, three product managers involved in the design of visualization repositories, and four software engineers and designers. Participants had working experience with visualization repositories for tools like Tableau (e.g., Tableau Public), Microsoft Power BI (e.g., Power BI Partner Showcase), D3 (e.g., D3's Observable Example Gallery), and general experience searching for visualizations on Google. 


Interviews were conducted remotely and lasted 30-45 minutes. We asked participants about their backgrounds (e.g., their job descriptions, visualization repositories they use actively) and then asked them to share their experience, including the scenarios in which they search data or visualization repositories, current limitations, and areas for improvement in terms of the search experience, and metadata they find most relevant during visualization search.
We qualitatively analyzed the session transcripts and used an affinity diagramming approach to iteratively group similar comments (e.g., comments referring to searching for visualizations with a specific title or by an author, comments referring to using chart type as part of the search query).
We combined these groups under broader clusters of different scenarios search is used in as well as the most relevant search querying features.
Below, we summarize the key findings from our formative interviews in terms of the user goals and metadata features most relevant to \change{search in the context of data repositories containing both datasets and pre-authored charts}.

\subsection{Search Scenarios}

We identified three key user goals or scenarios for search in the context of data repositories.

\begin{itemize}[leftmargin=.15in]
    \item \textbf{Question \& Answering (Q\&A).}
    One common goal echoed by participants, particularly those who worked with organization-specific repositories hosting several data sources, was to leverage search to answer analytic questions.
    This goal is similar to information lookup~\cite{hearst2009search} in the broader web search context where user queries map to brief and discrete pieces of information (e.g., entities, dates, computed values).
    However, with data repositories, participants wanted to issue analytic questions (e.g., \textit{``What are sales trends across regions?,''} \textit{``highest covid cases by country''}) and get an appropriate response containing visualization and/or text generated from the available data sources.
    % Users' goal in this context is to find or retrieve a known item or a set of items given very specific criteria.
    % Some exemplars of targeted search queries include \textit{``\#makeovermonday,"} \textit{``mbostock force,"} or \textit{``Neil Richards"}.
    % Participants stated this was one of their most common use cases for search, especially in community-driven repositories hosting D3 and Tableau visualizations.
    % For instance, one of the visualization designers said \textit{``I often search for authors I know of or use a combination of author names and partial titles of the viz if I remember them. The ability to search for other authors and their work is integral to community initiatives."}
    % Another visualization designer (also a Tableau Zen Master), said \textit{``Most of my searches end up being of this [Q\&A search] type. I mostly remember who made something but not what it was called."}
    % Six participants (four designers/consultants, two product managers) commented that besides author names, hashtags (e.g., \#vizoftheday) on platforms like Tableau Public were another popular means to perform Q\&A search.
  
   
    \vspace{.5em}
   \item \textbf{Exploratory Search.}
    In line with the notion of exploratory search in web search~\cite{marchionini2006exploratory}, participants wanted to leverage data repositories to learn about a topic through available charts and data.
    % In this case, the users' primary goal is to see popular visualizations about a topic or learn about a topic through charts and their underlying data.
    Examples of exploratory search queries include \textit{``NFL drafts,''} \textit{``USA covid trends,''} or \textit{``Fifa world cup.''}
    Such queries are typically open-ended and do not provide refined filtering criteria beyond the topic itself. For instance, one participant (a visualization consultant) referred to exploratory search as one of his prominent goals during the initial stages of customer interactions.
    He highlighted the example of searching for visualizations on \textit{``private equity dashboards"} on Tableau Public during his recent interaction with a client at an investment firm.
    Describing her use cases for search, another participant (a visualization designer) alluded to exploratory search as one of her frequent search goals, stating \textit{``I often use search to see a few examples of what people create and to hunt for data sources about a topic."}
    
    \vspace{.5em}
    \item \textbf{Design Search.}
    The ability to find visualizations based on design features (e.g., chart type, color) was another popular use case for search, especially among the seven participants who were designers/consultants or novice visualization authors.
    Design search query examples include \textit{``sunburst chart,"} \textit{``bar and line combination chart,"} or \textit{``map with icons."}
    Based on anecdotes shared by the participants, this type of search is typically performed when users are looking for learning resources (e.g., a novice D3 developer looking for examples of force-directed layouts created with D3, a Tableau user trying to create a bespoke visualization like a Sankey diagram) or trying to understand design practices and find inspiration for their own work (e.g., using searches like \textit{``maps with a dark background''} to find examples of charts with specific color constraints).
    % , looking for examples of brushing-and-linking between specific chart types through a query like \textit{``dashboard with map and line chart linked"}).
    
\end{itemize}

% \noindent{}Note that these goals are not exhaustive.
% For instance, another goal that came up during the interviews was ``Question Answering," which was inspired by participants' experience with asking questions like \textit{``hottest months in Seattle?"} or \textit{``covid cases in California"} in Google and seeing a textual and/or visual response.
% However, we do not list this as a key search goal since this was only echoed by three participants and spans beyond the scope of a core visualization search problem on a public platform (e.g., to generate responses, the system would need a module to interpret a question, permission to access the data underlying hosted charts, map concepts from question to the data sources, and finally perform required calculations to compute an answer).
\noindent{}Note that these scenarios are neither exhaustive nor mutually exclusive.
For instance, \change{three participants mentioned ``targeted search'' as another scenario, where the intent was to retrieve a specific chart or dataset that the users knew existed in the repository.
However, we do not explicitly call this scenario out as it would inherently be supported by any search system that supports exploratory search (e.g., users can include specific and precise terms during exploratory search to retrieve the desired content). Furthermore,} 
queries like \textit{``sales by state and segment as a heatmap"} or \textit{``maps showing covid trends,"} combine Q\&A and design search, and design and exploratory search, respectively.
We also asked participants to rank the scenarios in terms of frequency/importance.
The responses, however, were fairly mixed and there was no single primary goal or a specific ordering of scenarios that stood out across participants.

Thus, rather than being a definitive and ordered set, the three scenarios listed above are primarily intended to serve as guidance for broad categories of user tasks to keep in mind when designing search systems for data repositories.

Besides understanding \textit{when} and \textit{why} people use search in data repositories (i.e., the above scenarios), to design and implement an effective search system, we also wanted to identify \textit{what} information people find most relevant while searching and browsing visualizations.
To this end, combining the participants' comments and search documentation for platforms like Observable~\cite{Observable:metadata} and Tableau Server~\cite{tableau_server_search}, we curated a list of the most prominent metadata fields that we focus on in our prototype.
These fields include the visualization title and description, the chart type (e.g., `bar chart,' `map,' `heatmap'), graphical encodings such as mark type, the visualization author, and the chart's creation date.



\subsection{Design Considerations}
Combining the feedback from the formative interviews with guidelines and findings from prior work on visualization search (e.g.,~\cite{viegas2007manyeyes,hoque2019searching,sechler2017sightline}), web and image search interfaces (e.g.,~\cite{hearst2009search,silverstein1999analysis,marchionini2006exploratory}), and NLIs for visual analysis (e.g.,~\cite{tory2019mean,shen2021towards,srinivasan2021collecting}).

\pheading{DC1. Support a unified experience that supports all three search scenarios.}
As we discussed the different search scenarios during the formative study, participants noted that they would ideally want the same interface and modality to perform the different tasks.
Thus, one consideration for us while building \olio{} was to design a seamless experience that supported a common input modality (NL) and blended Q\&A (a task commonly performed on data source collections) with exploratory and design search (tasks commonly performed with pre-authored visualization repositories).

\pheading{DC2. Support linguistic variations in queries.}
Both prior work on NLIs for visualization (e.g.,~\cite{tory2019mean,setlur2019inferencing,srinivasan2021collecting}) and web search (e.g.,~\cite{silverstein1999analysis,barr2008linguistic}) has shown that people use a variety of phrasings in search queries to accomplish the same goal.
Even during our interviews, participants used linguistically varied examples while discussing the same goal (e.g., ``\textit{What are sales trends across regions?}'' vs.~``\textit{sales by region over time}'').
Accommodating such user behavior, a second design consideration for \olio{} was that the system should support a variety of query formats - terse keywords as well as queries phrased as questions or sentence fragments, with an understanding of analytical intent relevant to data repositories in either case.

\pheading{DC3. Show textual responses and provide guidance for Q\&A queries.}
When discussing Q\&A scenarios, we asked participants about the types of visualizations they would expect for different queries.
During these conversations, in line with prior research on information lookup on the web~\cite{marchionini2006exploratory}, participants noted that besides charts, it may be valuable to ``\textit{provide a text response to a text query},'' suggesting the inclusion of complementary text along with a generated chart.
To this end, we noted that given a Q\&A query, \olio{} should not only select an appropriate data source and generate a chart but also text content that leverages the chart to help answer the input query.
Furthermore, since Q\&A queries can map to multiple data sources and users may not be aware of the available data source and fields, the system should guide users to ask questions (e.g., via query suggestions) and provide metadata information on the relevant data sources (e.g., available data fields and values to query).

\pheading{DC4. Provide visual summaries and filtering options for search results.}
One struggle that was echoed by several participants was that current visualization search systems do not provide an easy way to comprehend and sift through results beyond manual inspection.
To overcome this limitation with current systems, we noted that the system should provide visual summaries and support dynamic filtering~\cite{ahlberg1994visual} to help people get an overview, organize, and create meaningful facets of the visualization search results.

