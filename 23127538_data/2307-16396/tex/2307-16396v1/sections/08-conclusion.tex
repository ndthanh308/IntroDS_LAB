\section{Conclusion}
 In this paper, we explore how we can support data sensemaking and exploration in a semantic search paradigm designed specifically for data repositories. We introduce \olio, a research probe that realizes semantic search behavior through three types of searches: Q\&A, exploratory, and design. The system implements a novel semantic search framework that leverages analytical intent derived from the user's query, along with searchable metadata and content to provide a hybrid set of dynamically generated visualization responses with pre-authored visualizations. A preliminary evaluation of \olio~indicates that users find the system helpful for supporting a range of both targeted and open-ended data exploration activities. As people continue to actively explore data and author visualizations, there will be an increasing amount of searchable analytical content made available in these data repositories. The ability to support more expressive ways to utilize the content for a wide range of search goals will become especially important. This work provides interesting opportunities for managing and interacting with data beyond search; data curation and enrichment, along with novel modalities for exploring more varieties of content can further scaffold analytical discovery and insights.