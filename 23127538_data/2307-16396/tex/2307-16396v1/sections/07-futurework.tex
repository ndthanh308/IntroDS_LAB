\section{Limitations and Future Work}

Semantic search interfaces for data repositories hold promise for helping a user navigate and explore the growing amount of visualizations and analytical assets available. While \olio~received positive feedback as a research probe, research exploring semantic search for data repositories is still in its infancy. We identify various themes that highlight the challenges and opportunities for supporting semantic search that are unique to data repositories.

\pheading{Search precision depends on the availability and curation quality of data sources.} \change{Similar to other semantic search experiences~\cite{kaufmann:2006,klein-manning-2003-accurate,KOLOMIYETS20115412}, Q\&A search utilizes a small set of curated datasets to address analytical intents with focused responses.} However, an important aspect of search precision, especially for dynamically generated responses, is having access to high-quality, curated data sources with well-understood semantics. However, there is often a disconnect between environments where users publish content and downstream applications like search that consume the content. Participants echoed this challenge with P10 stating, ``\textit{this is a great interface and experience but will have to overcome the data garbage problem at scale}.''
Authors tend to perform some amount of curation during the publishing process but often are not provided sufficient tools to annotate, tag, or enrich their content. The process of curation is often tedious and time-consuming. More research should explore techniques (both semi-automated and automated)~\cite{potterswheel,wrangler,datacleaning:survey} to reduce the friction while curating content in data repositories; this includes the de-duplication of similar or near-similar content and the suggestion of topics and tags to help with content discoverability and faceting. \change{Future work should also explore techniques to help with data curation, such as employing LLMs for metadata enrichment, incorporating entity recognition, synonyms, and relational extraction to help automate curation for Q\&A support.}


\pheading{Incorporating additional analytical assets and metadata.} \olio~\\currently searches over pre-authored singleton visualizations. Future extensions should consider expanding the repertoire of analytical assets to include dashboards, data tables, and computational notebooks~\cite{Observable,jupyter}. These forms of content have interesting implications for interpreting analytical intent, Q\&A, and design search beyond data source and visualization repositories. Further, combining data repositories with document repositories could provide additional searchable metadata to improve search precision and for generating contextually relevant summaries alongside the results.

\pheading{Need for scaffolding to orient the user.}
Semantic search interfaces support new techniques for information seeking but with the added complexity of determining the type of queries and understanding the search results. Guidance and scaffolding may need to be provided as users search across multiple data repositories of content. While \olio~displays metadata for the available data sources along with query suggestions to guide a user toward a successful search, additional scaffolding could improve sensemaking and exploration. Recent work has explored data-driven autocompletion for helping users formulate targeted Q\&A-type queries~\cite{sneakpique} and integrate contextual query suggestions within a person’s sensemaking environment~\cite{interweave}. An interesting research direction would be to explore data scaffolds across different types of search, each unique in its own way, in the context of a semantic search system.

\pheading{Explore new search paradigms and modalities.} \olio~indexes available textual content in the data repositories. However, akin to image search, content-based search~\cite{cbr:2000} that leverages \emph{visual} features could improve recall of sparse text content, particularly for design search. Reverse image search~\cite{visualsearchpinterest} addresses the challenge for a user to guess at keywords and terms to return a specific result that they may have in mind. Exploring reverse visualization search, wherein a user provides a sample visualization or sketch to discover content related to the sample visualization image, could support richer forms of expressing design search goals. In addition to new search paradigms, other modalities, and platforms should be explored. Mobile devices, for example, generate large amounts of sensor footprints (e.g., GPS, motion sensors) and user activity data that are often missing from their desktop counterparts~\cite{franti2005mobile}. These new sources of implicit and explicit user feedback are valuable for discovering actionable content which is both situationally and contextually relevant to the user. Further, voice and touch modalities could open new possibilities for query formulation and browsing content in the data repositories. 


\pheading{Trust and provenance.} Trust is an important issue, and users would benefit from information that communicates the provenance of data sources used to generate the visualization responses, along with the ranking of pre-authored content. Exploring the inclusion of explanations for the search results could lead to increased transparency and understanding of the system behavior~\cite{ramos:2020}. There are additional challenges in an enterprise context; data and visualization content may be private to certain teams and organizations due to the sensitivity of the data (e.g., a human resources department or the current revenue forecast of a business). More work needs to explore ways to support built-in data privacy for indexing and searching of content within these organizational boundaries.


\pheading{Exploring the utility of LLMs for search.} Due to their ease of use and their fluent text-generative capabilities, LLMs are garnering attention for search and conversational interfaces~\cite{meyer:2022}. We explored the use of ChatGPT to generate a summary of the dynamically generated visualization response for Q\&A. The model does have limitations in the types of summaries it can generate (as described in Section~\ref{sec:discussion}) and challenges around higher-order numeracy reasoning~\cite{frieder2023mathematical}. Custom-trained GPT models could potentially bridge this gap in higher-order analytical reasoning if they can be trained on the data repositories employed in a semantic search system. In addition to summary generation, other utilities for these custom LLMs could explore automatic metadata generation from data repositories to enrich sparse searchable text content. Understanding the quality and accuracy of the generated text both for metadata ingestion and summary generation\change{, and comparing the resulting search experience to that of \olio{}}, are important research directions to pursue as future work.


% \begin{itemize}
  %  \item Publishing support (de-duplication, topic/tag recommendation)
 %   \item Result browsing experience (VizSummaries, summarizing not just content but interaction, form-factor considerations, example-based recommendation to support serendipitous discovery)
  %  \item Implementing features like autocomplete (loop in SneakPique). Highlight how collecting data through a design probe like \olio~can help with this.
   % \item Incorporating more metadata
    %\item Current system: individual charts, future work: dashboard search
   % \item Reverse image search + multimodal search.
 %   \item talk about GPT for summary generation
% \end{itemize}



