\section{Introduction}
User expectations of search interfaces are evolving. Search engines are increasingly expected to answer questions along with providing contextually relevant content that help address a searcher's goal~\cite{crook:2018}. Existing keyword-based search methods are mostly designed for content retrieval. Their main underlying drawback is limited support for structured query types that generally expect focused and specific responses. Natural language (NL) question \& answering (Q\&A) interfaces, on the other hand, support more fact-finding inquiry but do not support content or document discovery and retrieval. To bridge the gap between these two contrasting search paradigms, a hybrid approach called \emph{semantic search}~\cite{Guha2003SemanticS} applies user intent and the meaning (i.e., semantics) of words and phrases to determine the right content that might not be present immediately in the text (the keywords themselves) but is closely tied to what the searcher wants~\cite{bhagdev:2008}. The information retrieval technique goes beyond simple keyword matching by using information such as entity recognition, word disambiguation, and relationship extraction to interpret the searcher's intent in the queries. For example, keyword search can find documents with the query, ``French press,'' while queries such as ``How do I make quickly make strong coffee?'' or ``manual coffee brewing methods'' are better served by semantic search to produce targeted responses.

With an increase in the number of data repositories on the web, including structured data in the form of relational databases, files, and knowledge graphs, there is a plethora of information that supports the blend of generating responses to fact-finding questions with document retrieval~\cite{galhotra:2020}. Along similar lines, data repositories and visualization tools such as Observable~\cite{Observable}, Tableau Public~\cite{tableau2023public}, and Microsoft Power BI Partner Showcase~\cite{powerbi} host hundreds or thousands of visualizations representing a wide range of datasets, making them rich platforms for knowledge sharing and consumption. Search plays a pivotal role in these repositories, providing people the ability to winnow in on content they are interested in (e.g., charts on a specific topic, charts showing data trends and bespoke visualizations such as Sankey diagrams, or charts authored by a particular person). Current search systems tend to rely on document-retrieval techniques to provide relevant search results for a given query. However, the challenge with data repositories lies in the sparseness of searchable text within them; data sources and charts often have limited text information in the form of titles, captions, and textual data values, for example. There is a need to explore alternative ways to index and search for content based on this limited availability of textual information.



Another challenge is that current search features for data repositories offer limited expressivity in specifying search queries, restricting users to predominantly perform keyword search for content based on the visualizations’ titles and authors. In contrast, other contemporary search interfaces such as general web search, image and video search, and social networking sites enable users to find and discover content through a rich combination of textual content (e.g., keywords or topics covered in a website), visual features within the content (e.g., looking for images with a specific background color), dates (e.g., only viewing videos from the recent week), geographic locations (e.g., limiting search to certain zip codes or cities), and even different types of media (e.g., searching for similar images through features like reverse image search). %VS: This feels repetitive. This gap in expressivity offered during visualization search compared to search in other contexts can prevent people from freely specifying what they are interested in or lead to irrelevant search results, both of which could ultimately discourage the use of these data repositories altogether.

Designing expressive search interfaces for data repositories requires gaining a deeper empirical understanding of people's search requirements, given the current limitations of these systems. For instance, what goals do people have in mind when using search in the context of data repositories? How do people formulate their search queries? Is text alone a sufficient modality for search? If not, what are complementary/alternative modalities to consider? What supporting metadata do people want to query for or use to filter the search results?

\pheading{Contributions.} To explore these research questions, we first conducted a set of formative user elicitation interviews with $14$ participants who regularly search for visualizations or are involved in the design of search interfaces within mainstream visualization tools and data repositories. Findings from the interviews identified search scenarios specific to content exploration for data repositories and motivated the design and implementation of \olio\footnote{The word \textit{Olio} is defined as `a miscellaneous collection', reflecting the hybrid mix of search content displayed in the interface~\cite{mw:olio}}, an interface that supports semantic search behavior by dynamically generating visualization responses and pre-authored visualizations for data repositories. Specifically, the interface implements three search scenarios on a semantic search framework: \emph{Q\&A search} by interpreting analytical intent over a set of curated data sources, \emph{exploratory search} using document-based information retrieval methods on existing indexed visualization content, and \emph{design search} by leveraging visualization metadata for the content (Figure~\ref{fig:teaser}). The interface also supports facet-driven browsing to prune the search results by author name, time range, and visualization type. Employing \olio~as a design probe, we conducted a qualitative study with $11$ participants to gain feedback on the implemented metadata and querying features, identify system design and implementation challenges, and better understand user behavior.

The study confirmed that the semantic search paradigm supports the different data repository search goals. We observed that the ability to perform both Q\&A and search for pre-authored content facilitated a fluid analytic search experience but raised new questions about user expectations from search systems and the style of user interaction.
Lastly, from our observational data and participant feedback, we highlight promising directions for future work on visualization search interfaces, including better support for creating curated data sources, the need for scaffolding \change{(i.e., support to help with the discoverability of features or functionality in a user interface~\cite{sneakpique})} and building trust in the system behavior and exploring additional search paradigms and modalities.


