\def\cvprPaperID{5546}
\def\confName{ICCV}
\def\confYear{2023}

\def\paperTitle{CTVIS: Consistent Training for Online Video Instance Segmentation}

\def\authorBlock{
    Kaining Ying$ ^{1,2}\thanks{KY (email: {$\tt kaining.ying.cv@gmail.com$}) and QZ contributed equally to this work. This work was done %by KY during his visit 
    when KY, QZ, WM, YZ were visiting 
     Zhejiang University.}  $     \quad
    Qing Zhong$ ^{4}\footnotemark[1]  $     \quad
    Weian Mao$^{4}$  \quad
    Zhenhua Wang$ ^{3}\thanks{Correponding authors.}  $  \quad
    Hao Chen$ ^{1}\footnotemark[2]  $ \quad
    \\ 
    Lin Yuanbo Wu$ ^{5}  $ \quad 
     Yifan Liu$ ^{4}  $ \quad
    Chengxiang Fan$ ^{1}  $ \quad
    Yunzhi Zhuge$ ^{4}  $ \quad
    Chunhua Shen$ ^{1}  $
    
    \\[.2cm]
    \normalsize $ ^1 $ Zhejiang University  \qquad
    \normalsize $ ^2 $ College of Computer Science and Technology, Zhejiang University of Technology \\
    \normalsize $ ^3 $ College of Information Engineering, Northwest A\&F University  \\
    \normalsize $ ^4 $ The University of Adelaide, Australia \qquad
    \normalsize $ ^5 $ Swansea University, UK \\
    \small \href{https://github.com/KainingYing/CTVIS}{https://github.com/KainingYing/CTVIS}
    % \small {$\tt kaining.ying.cv@gmail.com$}
    
    %\small {$\tt \{qing.zhong,weian.mao,yunzhi.zhuge\}@adelaide.edu.au$} \\
    %\small {$\tt zhenhuawang@nwafu.edu.cn$}, 
    %\small {$\tt \{haochen.cad,fanchengxiang\}@zju.edu.cn$}, 
    %\small {$\tt l.y.wu@swansea.ac.uk$} 
}

% Compilation vars
\newif\ifreview\newcommand{\review}{\reviewtrue}
\newif\ifarxiv\newcommand{\arxiv}{\arxivtrue}
\newif\ifcamera\newcommand{\cameraready}{\cameratrue}
\newif\ifrebuttal\newcommand{\rebuttal}{\rebuttaltrue}

\newcommand{\red}[1]{\textcolor{RubineRed}{#1}}
\newcommand{\blue}[1]{\textcolor{blue}{#1}}
\newcommand{\green}[1]{\textcolor{Green}{#1}}
\newcommand{\orange}[1]{\textcolor{Orange}{#1}}


\newcommand{\ykn}[1]{\textbf{\color[]{KY: #1}}}

% \textbf{\textcolor{red}{KY: Modify this paragraph.}}


\arxiv % \review OR \arxiv OR \cameraready

\pdfoutput=1
\documentclass[10pt,twocolumn,letterpaper]{article}
\input{cvpr_header}
\begin{document}
%% TITLE
\title{\paperTitle}
\author{\authorBlock}

\twocolumn[{
\renewcommand\twocolumn[1][]{#1}%
\maketitle
\begin{center}
 \centering
 \captionsetup{type=figure}
 % Figure removed
% \vspace{-34pt}

% \caption{Despite their remarkable ability to generate plausible images from text descriptions, diffusion models fail to be faithful to multiple concepts in the input text. We identify the issues causing this pitfall, and propose a training-free method to fix them. We propose two new loss functions, attention segregation loss and attention retention loss, that only require test time optimization to drive the diffusion process and produce substantially improved generation results. We can note from these results that our method captures all key concepts in the input prompt as opposed to baseline Stable Diffusion \cite{rombach2022high}.}
	\caption{
	Towards Multi-Large-Language-Models Interchangeable Assistance with Human Feedback System Overall Framework.
In \textit{Section A)}, we elucidate the operational workflow of the Chat Research Paper System. The process commences when a user submits a query concerning a specific research paper. The system initially retrieves the PDF of the paper from an online repository or, alternatively, from local storage if available. The document is then segmented into multiple portions which undergo an embedding process using OpenAI Embeddings, generating distinct query and document embeddings. Following this, a k-nearest neighbors algorithm is employed to match these embeddings, facilitating the extraction of the most relevant segments of the document in relation to the user’s query. These pertinent segments, along with instructions and the user's query, are then submitted to the Large Language Model (LLM) for processing.
In \textit{Section B)}, we expound on the policy-oriented aspect of the system, which dynamically adjusts its approach based on human feedback, particularly when ambiguous queries or unsatisfactory responses from the LLM are encountered. Users have the option to escalate the level of assistance rendered by the system. At the entry level, the system employs an efficient matching algorithm and integrates a local summarization layer for document embeddings. Ascending to the intermediate level, GPT-3 is engaged to perform chunk-level filtering and summarization. At the most advanced stage, the system relaxes token window limitations and utilizes GPT-3-16k to conduct multi-page retrieval and refinement. During this phase, the embeddings are generated using the cutting-edge Davini003 model, and a robust recommendation system is employed to craft a comprehensive prompt that queries GPT-4 for sophisticated reasoning. Additionally, all intermediate outputs are retained in a memory buffer for potential utility in addressing subsequent queries.
	}
 \label{fig:overall}
\end{center}
}]

\maketitle
%%



Recent work has shown that when both the chart and caption emphasize the same aspects of the data, readers tend to remember the doubly-emphasized features as takeaways; when there is a mismatch, readers rely on the chart to form takeaways and can miss information in the caption text. Through a survey of 280 chart-caption pairs in real-world sources (e.g., news media, poll reports, government reports, academic articles, and Tableau Public), we find that captions often 
do not emphasize the same information in practice, which could limit how effectively readers take away the authors' intended messages.
Motivated by the survey findings, we present \toolname{}, an interactive tool that highlights visually prominent chart features as well as the features emphasized by the caption text along with any mismatches in the emphasis. The tool implements a time-series prominent feature detector based on the Ramer-Douglas-Peucker algorithm and a text reference extractor that identifies time references and data descriptions in the caption and matches them with chart data.
This information enables authors to compare features emphasized by these two modalities, quickly see mismatches, and make necessary revisions. A user study confirms that our tool is both useful and easy to use when authoring charts and captions.
%!TEX root = ../main.tex

\section{Introduction}
\label{sec:intro}

Climate change and the decline of species richness are severe challenges that influence the living conditions of humans around the world.
Especially the dramatic loss of insects~\cite{hallmann2017more,wagner2021insect} plays a crucial role in many ecological processes that affect agriculture and others.
Hence, monitoring insect species populations becomes more important nowadays to better understand insect decline and long-term trends in species distributions.
Furthermore, there are about one million named species on our planet~\cite{stork2018many}, making manual counting of individuals unrealistic.
Consequently, automated monitoring of insects is inevitably required to infer abundance estimations across larger regions.
One possible way is to use camera traps to collect images of insects that computer vision algorithms can then process to recognize the depicted species automatically.

In this paper, we focus on nocturnal insects, mainly nocturnal moths (Lepidoptera).
Even for this subset, there exist hundred thousands of different species worldwide and depending on the habitat, species lists can be narrowed down based on the study region.
For example, image datasets containing hundreds of moth species from Ecuador and Costa Rica are publicly available and can directly be used for evaluating fine-grained recognition algorithms~\cite{Rodner15:FRD}.
Here, we are interested in monitoring moth species in Central Europe.
We present datasets of moth images we have collected so far and our analysis of algorithms for insect localization and species classification.

% Figure environment removed

Our work is part of a larger project called AMMOD\footnote{\scriptsize{AMMOD = \textbf{A}utomated \textbf{M}ultisensor Station for \textbf{M}onitoring \textbf{o}f Bio\textbf{d}iversity (\url{https://ammod.de/})}}, which aims at developing self-sustaining multi-sensor stations for monitoring species diversity~\cite{Waegele22:TAM}.
One component of these stations is a light-based camera trap for nocturnal insects, called the \emph{moth~scanner}~\cite{Radig2021:AVL,Korsch21_DLP}.
It is a non-invasive monitoring system for automatically gathering images at nighttime.
A UV-LED lamp illuminates a white planar surface to attract the insects that land on this surface.
A high-resolution camera takes an image of the whole surface every two minutes.
Our prototype is shown in Figure~\ref{fig:prototype}.

With this setup, we can collect large-scale datasets of nocturnal insects over a long period that can then be used to develop and evaluate appropriate fine-grained species recognition algorithms.
The moth scanner takes several hundred images during one night, and within five months, we collected more than \num{27000} images with our prototype.
In this paper, we refer to the resulting dataset as the \emph{nocturnal insects dataset~(NID)}, and more details are given in Section~\ref{sec:dataset}.
Note that this dataset is supposed to be extended over time as our system will be in operation within the following years.
We plan to maintain multiple sensor stations in parallel at different locations.
Hence, it has the potential to become a valuable source for large-scale learning and continuous learning within a fine-grained domain.

Besides its impact on research in fine-grained recognition, our developments for automated visual monitoring of nocturnal insects are beneficial for ecologists.
Until now, insect monitoring is mainly done by hand and supported by citizen scientists who manually take images of individual insects in their gardens. 
Previously, we published an image dataset of nocturnal moths captured manually by citizen scientists, called \emph{\mbox{EU-Moths}} dataset at a local workshop~\cite{Korsch21_DLP}.
This paper also includes a dataset description and our baseline results for insect localization and species classification.
There are two reasons for this.
First, we want to announce this dataset to a broader audience interested in fine-grained recognition because it can directly be used for algorithm development and evaluation.
Second, we want to highlight the challenges for recognition algorithms that arise when processing automatically captured camera trap images compared to manually taken images with hand-held cameras.

In general, our paper aims to promote the application of moth species identification as a fine-grained visual recognition problem.
We underpin this with existing datasets, results of baseline algorithms, and a light-based camera trap setup that will be used during the following years to automatically collect further large-scale image data.
We believe that research on automated visual identification of hundreds to thousands of different nocturnal moth species can have a major impact on developing fine-grained recognition algorithms in general, and we, therefore, want to share our insights and datasets with the community.

%The rest of the paper is structured as follows. 
%After a short review of related work in Section~\ref{sec:related_work}, we describe the two abovementioned datasets containing images of nocturnal moths in Section~\ref{sec:dataset}.
%The algorithms we applied to both datasets are described in Section~\ref{sec:methods} and we present the achieved results in Section~\ref{sec:results}. 
%We discuss challenges of processing automatically captured images with light-based camera traps in Section~\ref{sec:challenges} that are also important to consider for similar projects, followed by conclusions in Section~\ref{sec:conclusions}.

%\todo{REWRITE from here}

%Before the classification can be performed, we need to perform a detection of the insects.
%At this stage, the application of the state-of-the-art detection models like SSD~\cite{liu2016ssd} or YOLO~\cite{redmon2016you} is an obvious step.
%On the other hand, these models are computationally expensive and other light-weight methods like the MCC blob detector~\cite{bjerge2021automated} are more suitable for the application in the field.
%Unfortunately, to evaluate and compare different detection methods suitable benchmark datasets are missing.

%In this paper, we present a new dataset collected with the help of our prototype.
%In the period of five months, we captured over \num{27000} images in suburban area in Middle Germany.
%For bootstrapping and first evaluations we annotated a subset of these images with bounding boxes for the captured insects.
%The image data and the annotations will be soon publicly available.

%As a first baseline for insect detection task, we evaluated two different methods on the data and present these results further in our paper.
%First, we used a well-established Deep Learning detection model capable of identifying multiple objects in an image, namely the single-shot MultiBox detector (SSD)~\cite{liu2016ssd}.
%As a light-weight alternative that can be easily deployed directly at the computationally limited hardware of the moth scanner, we developed and evaluated a multi-step blob detection algorithm.
% \todo{Edge Computing as buzzword? EdgeAI may be wrong here?}
%First, it reduces the power consumption due to the reduction of computations.
%Further, applying the detection directly at the moth scanner, we can drastically reduce the amount of data that needs to be transmitted  when the system will gather data autonomously in the field.
%The algorithm is closely related to the blob detection method proposed by Bjerge~\etal\cite{bjerge2021automated} but mitigates some of the method's limits.
%We present the idea and the improvement in more detail in Sect.~\ref{sec:methods}.

%\section{Related Work}
\label{sec:related}

\begin{table*}
\small
\caption{Pareto-optimal \textit{YOLOBench} models on 3 datasets and 3 hardware platforms. Shown are the best models in terms of mAP$_{50-95}$ under a given latency threshold (max. latency). For each model, the scaling parameters are given (d33w25 means depth factor $=0.33$ and width factor $=0.25$), corresponding input resolution of the models is indicated in brackets.}
\vspace{1mm}
\label{tab:pareto_table} 
\begin{tabularx}{\linewidth}{lXXXXXX}
\toprule
{HW/max.} & {VOC} & {VOC} & {SKU-110k} & {SKU-110k} & {WIDERFACE} & {WIDERFACE} \\ %
{latency} & {model} & {mAP$_{50-95}$} & {model} & {mAP$_{50-95}$} & {model} & {mAP$_{50-95}$} \\ %
\midrule
{Nano/0.1 sec} & {YOLOv7} & {0.657} & {YOLOv8} & {0.567} & {YOLOv7} & {0.336}\\
{} & {d1w5 (288)} & {} & {d1w25 (480)} & {} & {d1w25 (480)}\\
\midrule
{VIM3/0.05 sec} & {YOLOv6l} & {0.620} & {YOLOv6s} & {0.556} & {YOLOv6m} & {0.318}\\
{} & {d67w25 (416)} & {} & {d33w25 (480)} & {} & {d67w25 (480)}\\
\midrule
{Raspi4/0.5 sec} & {YOLOv6l} & {0.667} & {YOLOv4} & {0.569} & {YOLOv7} & {0.336}\\
{} & {d67w5 (384)} & {} & {d1w25 (480)} & {} & {d1w25 (480)}\\
\bottomrule
\end{tabularx}
\end{table*}

% Figure environment removed

There has been a tremendous amount of progress in efficient object detection in recent years pushing the accuracy-latency frontier, including architectures like YOLOv7 \cite{wang2023yolov7}, YOLOv6-3.0 \cite{li2023yolov6}, DAMO-YOLO \cite{xu2022damo}, RTMDet \cite{lyu2022rtmdet}, RT-DETR \cite{rtdetr} and PP-YOLOE \cite{xu2022pp}. These works oftentimes improve upon state-of-the-art latency-accuracy trade-off, providing comparisons of several generations of detectors on the COCO dataset. Benchmarks of different model families are also provided by framework developers, such as MMYOLO \cite{mmyolo2022} and Ultralytics \cite{ultralytics}. Additionally, there exist third-party benchmarks of several architectures from the YOLO series on server-grade and embedded GPUs as well as specialized accelerators \cite{stereo_labs,opencv_yolo_benchmarking,feng2022benchmark,zhu2022performance}. We have identified a few limitations of the existing efficient detector benchmarks that have served as motivation for \textit{YOLOBench}:

\begin{itemize}
    \item Comparisons of different YOLO versions are frequently done either by using a proxy metric for the actual latency like MAC count and number of parameters or by reporting latency values on server-grade GPUs, neither of which is directly indicative of latency on embedded devices,
    \item Accuracy metrics are usually reported on the COCO dataset, which could be considered too large-scale with respect to actual practical use cases,
    \item Some architecture parameters (like input resolution) are often considered to be fixed in detector benchmarking, while it is known that they serve as important factors in optimal CNN scaling \cite{tinynet},
    \item Different YOLO variations being compared to one another are typically trained with different training codebases, training techniques (loss functions, data augmentations), and hyperparameter values, making it hard to disentangle the contribution of the training pipeline improvements vs. better architecture design. 
\end{itemize}

To address these issues, we have conducted a thorough accuracy and latency benchmarking of state-of-the-art YOLO detector versions in controlled, fixed conditions to study the impact of backbone and neck design proposed by several YOLO model families.

\section{Methods}
\label{sec:method}

CTVIS builds upon Mask2Former \cite{mask2former}, which is an effective image instance segmentation model (briefly reviewed in Section~\ref{sec:mask2former})\footnote{
%We would like to
Note that CTVIS can be easily combined with other query-based instance segmentation models \cite{idol, detr, deformabledetr} with minor modifications.}. Our CTVIS is motivated by the inference of typical online VIS methods introduced in Section~\ref{sec:inference}. 
Then we detail our consistent training method in Section~\ref{sec:ct}. Finally, Section~\ref{sec:pseudo} presents our goal-oriented pseudo-video generation technique for training VIS models with sparse image-level annotations.

% We closely follow the Notations in MinVIS
\subsection{Brief Overview of Mask2Former} 
\label{sec:mask2former}
Mask2Former \cite{mask2former} composed of three main components: an \emph{image encoder} $\mathcal{E}$ (consist of a backbone and a pixel decoder), a \emph{transformer decoder} $\mathcal{T}$ and a \emph{prediction head} $\mathcal{P}$. Given an input image $I\in \mathbb{R}^{H \times W \times 3}$, $\mathcal{E}$ extracts a set of feature maps $\bm{F}=\mathcal{E}(I)$, where $\bm{F} = \{ F_0 \cdots F_{-1}\}$ is a sequence of multi-scale feature maps, and $F_{-1}$ is the final output of the $\mathcal{E}$ with $1/4$ resolution of $I$. The $N$ raw query embeddings $\hat{Q} \in \mathbb{R}^{N \times C}$ are learnable parameters, where $N$ is a large enough number of outputs and $C$ is the number of channels. Then, $\mathcal{T}$ takes both $\bm{F}$ and $\hat{Q}$ to iteratively refine query embeddings, and consequently outputs $Q \in \mathbb{R}^{N \times C}$. Finally, the prediction head outputs the segmentation masks $M$ and the classification scores $O$. For classification, $O=\mathcal{C}(Q) \in \mathbb{R}^{N \times K}$, where $K$ is the number of object categories. For  segmentation, the masks $M \in \mathbb{R}^{N \times H/4 \times W/4}$ are generated with $M = \sigma(Q \ast F_{-1})$, where $\ast$ denotes the convolution operation and $\sigma(\cdot)$ is the sigmoid function.

\noindent\textbf{Our Modification.} Because CTVIS employs instance embeddings to associate instances during inference, we add a  head (a few MLP layers) to compute the instance embeddings $E \in \mathbb{R}^{N \times C}$ based $Q$. 
% The entire process can be summarized as
% \begin{equation}
% \label{eq:mask2former}
%     O, M, E = Mask2Former(I).
% \end{equation}

\subsection{Inference of CTVIS}
\label{sec:inference}
CTVIS leverages Mask2Former\cite{mask2former} to process each frame 
%(\ie Equation~\eqref{eq:mask2former}) 
and introduces an external memory bank\cite{idol, masktrackrcnn} to store the states of previously detected instances, including classification scores, segmentation masks and instance embeddings. 
% gets the corresponding classification scores, segmentation masks and instance embeddings for each frame. 
% Specially, CTVIS makes instance association frame by frame and introduces an external memory bank to store the states of previously detected instances, including classification scores, segmentation masks and instance embeddings. 
To ease presentation, we assume that CTVIS has already processed $T$ frames out of an input video of $L$ frames, and there are $N$ predicted instances with $N$ instance embeddings $\bold{d}_i \in \mathbb{R}^C$ in the current frame. The memory bank stores for the previous $T$ frames $M$ detected instances, each of which has multiple temporal instance embeddings $\{ \bold{e}^t_j \in \mathbb{R}^C  \}^T_{t=1}$ and a momentum-averaged instance embedding $\hat{\bold{e}}_j^T$, which is computed according to the similarity-guided fusion \cite{sgf}: 
\begin{gather}
    \label{eq:sgf}
    \hat{\bold{e}}^T_j=(1-\beta^T) \hat{\bold{e}}^{T-1}_j+\beta^T \bold{e}^T_j \text {, } \\
    \beta^T=\max \left\{0, \frac{1}{T-1} \sum_{k=1}^{T-1} \Psi_d\left(e^T_j, e^{T-k}_j\right)\right\} , 
\end{gather}

% Figure environment removed 

\noindent where $\Psi_d$ denotes the cosine similarity. % This momentum type brings more flexblity for instance embedding fusion. 
We refer the reader to \cite{sgf} for more details. Next, for each instance $i$ detected in the current frame, we compute its bi-softmax similarity \cite{qdtrack} with respect to the previously detected instance $j$ using


\begin{equation}
\label{equ:bio_softmax}
    f_{i,j}=
    0.5 \cdot \left[\frac{\exp \left(\hat{\mathbf{e}}_j^T \cdot \mathbf{d}_i\right)}{\sum_k \exp \left(\hat{\mathbf{e}}_k^T \cdot \mathbf{d}_i\right)}+\frac{\exp \left(\hat{\mathbf{e}}_j^T \cdot \mathbf{d}_i\right)}{\sum_{l} \exp \left(\hat{\mathbf{e}}_j^T \cdot \mathbf{d}_l\right)}\right] %\cdot 
 %   / 2. 
\end{equation}

Finally, we find the ``best''  
instance ID for $i$ with
\begin{equation}
\hat{j}=\arg \max f_{i,j}, \forall j \in\{1,2, \ldots, M\}.
\end{equation}
If $f_{i,\hat{j}} > 0.5$, we believe that newly detected instance $i$ and instance $\hat{j}$ in the memory bank correspond to the identical target. Otherwise, we initiate a new instance ID in the memory bank. When all frames are processed, the memory bank contains a certain number of instances, each of which takes a classification score list $\{c_i^t\}_{t=1}^{L}$ and a mask list $\{m_i^t\}_{t=1}^{L}$ (recall that $L$ denotes the number of frames). For each instance $i$, we calculate its video-level classification score by averaging the frame-level scores of the object. 
% $\boldsymbol{c_i}$ as follows: $\boldsymbol{c_i} = \Sigma_{t=1}^L c_i^t$. We get the $\{(\boldsymbol{c_i}, \{m_i^t\}_{t=1}^{L})\}_{i=1}^S$ as final outputs. 

% \subsection{Constructing CIs via Consistent Training}
\subsection{Consistent Learning}
\label{sec:ct}

A reliable matching of instances (\ie using Equation~\eqref{equ:bio_softmax}) across time is required to track instances successfully. Hence the extraction of highly discriminative embeddings of objects is of great importance. 
We argue that the discrimination of instance embeddings extracted with recent models \cite{idol, stc} is still inadequate, especially for videos involving object-occlusion, shape-transformation and fast-motion. One main reason is that mainstream contrastive learning methods build CIs (\ie $\{\mathbf{v},\mathbf{k}^+,\mathbf{k}^-\}$) from the reference frame only, which results in the comparison of the anchor embedding against instantaneous instance embeddings in $\mathbf{k}^+$ and $\mathbf{k}^-$. Such embeddings are typically less discriminative and contain noise, which prevents training from learning robust representations. To address this, our CTVIS leverages a memory bank to store MA embeddings, thus supporting contrastive learning from more stable representations. Here our insight is to align the embedding comparison of training with that of inference (such that the two comparisons are consistent). Figure~\ref{fig:main} sketches our CTVIS, which processes the training video frame-by-frame. For an arbitrary frame $t$, CTVIS involves three steps: a) it first takes the Mask2Former and Hungarian matching to compute the instance embeddings, and to match them with GT (highlighted by red, green and purple); b) Then, it builds CIs using MA embeddings within the memory bank, and performs contrastive learning with CIs; and c) It updates the memory bank with noise (\eg the embedding of the \emph{cat} is deliberately added to the memory of the \emph{dog}), which serves the learning from the next frame.

\noindent\textbf{Forward passing and GT assignment.} As shown in Figure~\ref{fig:main}~(a), we first feed the current frame $t$ into Mask2Former to compute the embeddings for queries. Then we employ Hungarian matching to find an optimal match between the decoded instances and the ground truth (GT), such that each GT instance is assigned one instance embedding. Note that Hungarian matching relies on the costs calculated for all (\emph{Decoded-Instance}, \emph{GT-Instance}) pairs. Essentially, each cost measures the similarity between a pair of instances based on their labels and masks.

\noindent\textbf{Construct CIs.} 
After GT assignment, we build the contrastive items for each GT instance using a memory bank. The memory bank stores all detected instances of previous $t-1$ frames, each associated with 1) a series of instance embeddings extracted at different times, and 2) its MA embedding computed by Equation~\eqref{eq:sgf}. 
% To clarify, we only  show the contrastive item of the person in Figure~\ref{fig:main}(b), we select the instance embeddings of the person at current frames as query embedding $v$. For the positive embedding, we select the momentum-averaged embedding of person from the memroy bank
In order to prepare the CIs $\{\mathbf{v}, \mathbf{k}^+, \mathbf{k}^-\}$ for instance $i$ (termed as the \emph{anchor}, \eg the person in Figure~\ref{fig:main}~(a)) at the $t$-th frame, the instance embedding extracted from this frame is used as the anchor embedding $v$.
%
For the positive embedding, we pick from the memory bank the MA embedding of instance $i$.
%
The negative embeddings $\mathbf{k}^-$ include the major negative embeddings and the supplementary negative embeddings. We use the MA embeddings of other instances in the memory bank as the major negative embeddings. We also sample the background query embeddings of previous $t - 1$ frames to form the supplement negative embeddings. Taking as inputs the created CIs, we compute the contrastive loss with
\begin{equation}
\label{eq:loss_embed}
\begin{aligned}
    \mathcal{L}_{\text {emb}} & =-\log \frac{\exp \left(\mathbf{v} \cdot \mathbf{k}^{+}\right)}{\exp \left(\mathbf{v} \cdot \mathbf{k}^{+}\right)+\sum\nolimits_{\mathbf{k}^{-}} \exp \left(\mathbf{v} \cdot \mathbf{k}^{-}\right)} \\
    & =\log \left[1+\sum\nolimits_{\mathbf{k}^{-}} \exp \left(\mathbf{v} \cdot \mathbf{k}^{-}-\mathbf{v} \cdot \mathbf{k}^{+}\right)\right].
\end{aligned}
\end{equation}
As shown in Figure~\ref{fig:main} (c), training with $\mathcal{L}_{\text {emb}}$ pulls the embeddings of positive instances close to the anchor embedding, while pushing the negative embeddings away from it.

\noindent\textbf{Update memory bank.} After computing the $\mathcal{L}_{\text{emb}}$ for each instance in frame $t$, we need to update the memory bank, such that the updated version can be taken to build CIs for frame $t+1$.
%
Unlike the inference stage, for training we can get the ground truth ID of each instance so as to update the memory bank with their embeddings extracted from frame $t$.
%
In comparison, inference can fail to track instances across time (\ie the ID switch issue), especially for complicated scenarios. To alleviate this, we introduce noise to the update of the memory bank, which compels the contrastive learning to tackle the switch of instance IDs.
%
Specifically, each disappeared instance (\eg the dog) in frame $t$ will have a little chance to receive an embedding of other instances (\eg the cat, which is randomly picked from all available instances) in the same frame, which is called the \emph{noise}. 
%
% As illustrated in Figure~\ref{fig:main}~(c), the dog disappeared in frame $t$, and a new instance of cat presents.
%
If the generated random value exceeds a threshold (\eg 0.05), as illustrated in Figure~\ref{fig:main}~(c), we use the noise as the embedding of the disappeared instance at frame $t$. Finally, the MA embeddings are updated for all instances using Equation~\eqref{eq:sgf}. Due to the low similarity between the disappeared instance and the noise, such an update has quite a limited impact on the MA embedding of the instance, which is reidentified later. Indeed, training with noise is able to reduce the chance of ID switch, as demonstrated by the fish example in Figure~\ref{fig:video}. 

\noindent\textbf{Loss.} After processing all frames, The $\mathcal{L}_{\text {emb}}$ values of all CIs are averaged to obtain $L_{\text {emb}}$.
The total training loss is
\begin{equation}
L_{\text{total}} = \lambda_{\text{emb}}L_{\text{emb}} + \lambda_{\text{cls}} L_{\text{cls}} + \lambda_{\text{ce}} L_{\text {ce}} + \lambda_{\text{dice}} L_{\text{dice}},
\end{equation}
where $\lambda$ denotes loss weight. $L_{\text {cls}}$, $L_{\text {ce}}$ and $L_{\text {dice}}$ supervise the per-frame segmentation as suggested in \cite{mask2former}.

% when the ID of an instance changes to another instance in a complicated scene, most current methods always accumulate errors; to ease this issue, we introduce noise training, which directly simulates this situation during the construction of CI. As illustrated in Figure~\ref{fig:main}, the dog disappeared in the third frame, but a new instance of the cat appeared, and we added the cat's embedding to the external memory bank of the dog. Due to the low similarity between the instance embeddings of cats and dogs, it will have little impact on the MA embedding of further dogs that appear in the following frames. As shown in the video scene on the right of Figure~\ref{fig:video}, the wrong instances are corrected to original trajectories through noise training. 

% Here we describe how to build CIs via consistent training. Following the inference pipeline, we build the CIs frame by frame with updating the memory bank. For each frame (expect the first frame),  we will build CIs for each instances. 

% The consistent training aims at constructing CIs frame by frame following the inference stage introduced in Section~\ref{sec:inference}. As shown in Figure~\ref{fig:main}, we sample $T$ temporally adjcent frames as train video $\{ I_t\}_{t=1}^T$. 
% The first step is feeding each frame into a weight-shared Mask2Former and then matching the output with the corresponding ground truth via Hungarian matching. The output of each input frame $I_t$ comprises classification scores, segmentation masks and instance embeddings, formulated as $\{ O_t, M_t, E_t\}$. Then we calculate the pair-wise matching cost, considering both class prediction and the similarity of predicted and the ground truth masks. Next, we use Hungarian matching to assign one predicted instance to each ground truth instance. Specially, for each ground truth instance $j$ at frame $I_t$, we have a matched predicted instance embedding $\mathbf{e}^t_j$. 

% Then we construct contrastive items, each of which consists of anchor/positive/negative embeddings, for each ground truth instances at each frames. 
% % Contrastive items are composed of three parts: anchor embedding, positive embedding and negative embedding. Assume we contruct the contrastive item for the instance $i$ at the $I_t$, we build 
% Here we detail the construction of the CI of the GT instance $i$ at the frame $I_t$ (e.g. the dog at the $I_5$ shown in Figure~\ref{fig:main}). As shown in Figure~\ref{fig:main} (b), the memory bank stores the status of instances of previous frames. For each instance ID in the memory bank, we can get corresponding momentum-averaged embedding from the instance embeddings via Equation~\ref{eq:sgf}. We sample the instance embedding $e^t_j$ at the current frame as the anchor embedding $v$. For the positive embedding, we choose the momentum-averaged embedding of the same instance ID from the memory bank. The negative embeddings $k^-$ of contrastive item compose of two parts: major negative embeddings and supplement negative embeddings. And we select the momentum-averaged embeddings of other instance IDs as major negative embeddings. Furthermore, we sample the background embedding of previous $t - 1$ frames as supplement negative embeddings. Finally, we compute the contrastive loss upon each contrastive item as follows:
% \begin{equation}
% \label{eq:loss_embed}
% \begin{aligned}
%     \mathcal{L}_{\text {embed}} & =-\log \frac{\exp \left(\mathbf{v} \cdot \mathbf{k}^{+}\right)}{\exp \left(\mathbf{v} \cdot \mathbf{k}^{+}\right)+\sum_{\mathbf{k}^{-}} \exp \left(\mathbf{v} \cdot \mathbf{k}^{-}\right)} \\
%     & =\log \left[1+\sum_{\mathbf{k}^{-}} \exp \left(\mathbf{v} \cdot \mathbf{k}^{-}-\mathbf{v} \cdot \mathbf{k}^{+}\right)\right].
% \end{aligned}
% \end{equation}
% As shown in Figure~\ref{fig:main} (c), the $\mathcal{L}_{\text {embed}}$ enforces the embeddings of same instances while draws embeddings of different embeddings far away. 

% In addition, when the ID of an instance changes to another instance in a complicated scene, most current methods always accumulate errors; to ease this issue, we introduce noise training, which directly simulates this situation during the construction of CI. As illustrated in Figure~\ref{fig:main}, the dog disappeared in the third frame, but a new instance of the cat appeared, and we added the cat's embedding to the external memory bank of the dog. Due to the low similarity between the instance embeddings of cats and dogs, it will have little impact on the MA embedding of further dogs that appear in the following frames. As shown in the video scene on the right of Figure~\ref{fig:video}, the wrong instances are corrected to original trajectories through noise training. 

% Finally, we get the final contrastive loss by average on all contrastive items. Besides, we get the loss final loss as follows: $$

% For instance, a contrastive item is generated for contrastive learning that matches the ground truth (GT) on each frame. This contrastive item is mainly composed of three parts:
% \begin{itemize}
% \item The anchor, which is the instance embedding of the current frame
% \item The positive and negative samples, which are the instances matched with GT by calculating the momentum-averaged instance  embedding of the current frame through the similarity-guided fusion mentioned in Section~\ref{sec:inference}
% \item Other embeddings that do not match GT, which are directly treated as negative samples
% \end{itemize}

% When the ID of an instance changes to another instance in a complicated scene, most current methods will always be wrong; thus, we directly simulate this situation in the training phase. As illustrated in Figure~\ref{fig:main}, the dog disappeared in the third frame, but a new instance of the cat appeared, and we added the cat's embedding to the external memory bank $B_{dog}$ of the dog. Due to the low similarity between the instance embeddings of cats and dogs, it will have little impact on the MA embedding of further dogs that appear in the following frames. As shown in the video scene on the right of Figure~\ref{fig:video}, the wrong instances are corrected to original trajectories through noise training.

\vspace{-3mm}
\subsection{Learning from Sparse Annotation}
\label{sec:pseudo}

% Figure environment removed 

We now elaborate on our pseudo-video and mask generation technique, which enables the training of VIS models when only sparse annotations (\eg image data) are available. We take a few widely applied image-augmentation methods, including \emph{random rotation}, \emph{random crop} and \emph{copy\&paste} on source image to create pseudo-videos and the associated instance masks. Note that the pseudo-videos are created by no means to approximate real ones. Instead, they are taken to mimic the movement of targets in reality. 

\noindent\textbf{Rotation.} 
As shown in the first row of Figure~\ref{fig:augs}, the rotation augmentation rotates the source images with several random angles (e.g., $ [-15, 15]$ ) to introduce subtle changes between frames of the pseudo-videos. 

\noindent\textbf{Crop.} 
The rotation augmentation cannot alter the shapes and magnitudes of instances. However, instances deform or/and enter/exit the visible field due to the movement introduced either by the target or the camera. To address this, we apply random crop augmentation to the image, which allows the generated videos to mimic the zooming in/out effect of the camera lens and the shifting of targets. The second and the third rows of Figure~\ref{fig:augs} present two examples of \emph{crop-zoom} and \emph{crop-shift}, respectively. The pseudo-videos generated by such augmentations cover a large proportion of targets' movements.

\noindent\textbf{Copy and Paste.} 
As mentioned earlier, the trajectories of instances in pseudo-videos created by the augmentations share the identical motion direction. To incorporate the relative motion between instances, we also employ the \emph{copy\&paste} augmentation\cite{copypaste}, which copies the instances from another image in the dataset and pastes them into  random locations within the source image. Note that the pasting positions of an instance are typically different across time, which brings the relative motion between different instances (as shown in the fourth row of Figure~\ref{fig:augs}).
% As shown in Figure~\ref{fig:augs}, this operation brings relative between several instances.

% \noindent\textbf{Merge All.} Suppose we want to generate a pseudo-video of T frames. Given an input image $I_{des}$, we randomly select another image from datasets as $I_{src}$. Then we parallelly make the copy\&paste $T$ times each of which copy\&pastes the instances of $I_{src}$ into the $I_{des}$. We get the output as N

% which selects two images, ${img_k}$ and ${img_r}$, from the dataset by random, and applies the aforementioned augmentations to them simultaneously. Subsequently, a subset of the pixel values of the instance from ${img_r}$, exceeding a certain threshold, is selected and pasted onto ${img_k}$. Finally, the necessary modifications to the ground-truth annotation are made, the fully occluded object is removed, and the mask of the partially occluded mask and bounding box is updated. As shown in the fourth row of Figure~\ref{fig:augs}, the copy\&paste augmentation\cite{copypaste}


\section{Conclusions}
\label{sec:conclusions}
Organ segmentation is a fundamental task in the medical field. The volumetric data that characterize CT and MRI acquisitions make, however, the segmentation task computationally expensive. On the one hand, 2D CNNs provide a low latency solution unable to capture inter-slice information, on the other hand, 3D CNNs extract three-dimensional features at the price of high computation costs and risk of overfitting. Moreover, popular 2.5D multi-view fusion methods train three separate networks where the features of the orthogonal planes are learned independently, despite being part of the same volume. In SSH-UNet this is addressed by imposing weight sharing between convolutions so that only one network needs to be trained and multi-view features are collaboratively learned.
In this work, we introduced a novel approach for the segmentation of volumetric medical data. Inspired by works in the field of Video Action Recognition we interpret the slices of a volume as the frame of a video. Given a 2D backbone, to re-integrate the information between features belonging to adjacent slices we leverage the power of a shifting mechanism inspired by the TSM module. Spatio-temporal modeling, declined on pseudo-3D operators, despite being well-known in the Video Understanding field was never used before in the medical image analysis to extract and mingle multi-slice features. Our network, by using a 2D convolution with weight sharing mechanism and slice shift, can extract 3D features keeping low computational complexity.
In comparison to other popular state-of-the-art methods, SSH-UNet achieves an accuracy of \textbf{87.28\%} on the AMOS validation providing the smallest model in terms of parameters (6.48M) compared to the best network which has $+1.6\%$ improve in accuracy but $\times5$ increase in parameters.





{\small
\bibliographystyle{ieee_fullname}
\bibliography{11_references}
}
\newpage

\renewcommand{\thefigure}{S\arabic{figure}}
\renewcommand{\thetable}{S\arabic{table}}

\appendix
\label{sec:appendix}

\section{Latency measurements.}
\label{sec:appendix_A}

Details regarding hardware platforms used to collect latency measurements are outlined in Table \ref{tab:benchmarking_hardware}. Figures \ref{fig:nano_timing_corr} and \ref{fig:raspi4_timing_corr} show the difference of latency value distributions between devices computed for the full initial \textit{YOLOBench} architecture space consisting of $\sim$1000 models. While generally good correlation is observed between model inference latencies on different devices (see also Figure \ref{fig:timing_corr_matrix}), notably latency values measured on Khadas VIM3 NPU differ significantly from latency values on other devices. That is, for models with roughly the same latency on Jetson Nano GPU or Raspi4 ARM CPU, the difference in VIM3 NPU latency could be up to several times.  This difference between NPU values from other common GPU/CPU-based platforms highlights the necessity to develop hardware-aware architecture design and search methods. The difference in the NPU benchmark is also reflected in the structure of model Pareto frontiers (Figs. \ref{fig:voc_pareto}, \ref{fig:coco_pareto}, \ref{fig:sku_pareto}) and the performance of zero-cost predictors in identifying Pareto-optimal models (Figs. \ref{fig:zc_pareto_voc}, \ref{fig:zc_sku_parero_front}). 

\section{\textit{YOLOBench} Pareto frontiers for different datasets.}
\label{sec:appendix_B}

\textit{YOLOBench} Pareto frontiers for SKU-110k, WIDER FACE, and COCO datasets are shown in Figs. \ref{fig:sku_pareto}, \ref{fig:wider_pareto}, \ref{fig:coco_pareto}, correspondingly. Note that while mAP$_{50-95}$ values for VOC, SKU-110k, and WIDER FACE datasets are obtained by fine-tuning COCO pre-trained weights (all trained at 640x640 image resolution) on multiple image resolutions considered in \textit{YOLOBench} (11 values from 160 to 480 with a step of 32), the mAP$_{50-95}$ values on the COCO dataset are obtained by directly evaluating pre-trained COCO weights, without fine-tuning on the corresponding target image resolutions. This corresponds to the situation of deployment of pre-trained COCO weights without any additional training.

Table \ref{tab:pareto_table_full} shows the identified Pareto-optimal YOLO models on 3 different datasets and 4 hardware platforms under several latency thresholds. It can be noted that under the same latency threshold on a given hardware platform, the optimal YOLO model family and input image resolution are typically dataset-dependent.

Figures \ref{fig:scaling_raspi4} and \ref{fig:scaling_vim3} show the statistics of architecture scaling parameters (width factor, depth factor, image resolution) in Pareto-optimal models on Raspberry Pi4 CPU and VIM3 NPU, respectively. Although some differences are observed between devices and datasets (in particular depth factor distributions), there is a general trend in all computed Pareto fronts where a variation in depth/width factors is observed at higher resolutions, and resolution is reduced when the depth/width factors (especially the width factor) already have low values.

\begin{table*}[ht!]
\caption{Details on hardware platforms and corresponding runtimes used for benchmarking.}
\vspace*{3mm}
\label{tab:benchmarking_hardware}
\resizebox{\linewidth}{!}{
\begin{tabular}{l|c|c|c|c}
% \begin{tabularx}{\linewidth}{lsssss}
\toprule
{\bf } & {\bf Raspberry Pi 4 Model B} & {\bf Jetson Nano (NVIDIA)} & {\bf Khadas VIM3} & {\bf Lambda tensorbook}\\
\midrule
{CPU} & {\vtop{\hbox{\strut Quad Core Cortex-A72,}\hbox{\strut 64-bit SoC @1.8GHz}}} & {\vtop{\hbox{\strut Quad Core Cortex-A57 MPCore,}\hbox{\strut 64-bit SoC @1.43GHz}}} & {\vtop{\hbox{\strut Quad Core Cortex-A73 @2.2Ghz,}\hbox{\strut Dual Core Cortex-A53 @1.8Ghz}}} & {\vtop{\hbox{\strut Intel$^{\circledR}$ Core\texttrademark i7-10875H CPU}\hbox{\strut @ 2.30GHz}}} \\
\midrule
{Memory} & {4GB LPDDR4-3200 SDRAM} & {\vtop{\hbox{\strut 4 GB 64-bit LPDDR4,}\hbox{\strut 1600MHz 25.6 GB/s}}} & {4GB LPDDR4/4X} & {64GB DDR4 SDRAM}\\
\midrule
{AI-chip} & {-} & {\vtop{\hbox{\strut NVIDIA Maxwell GPU,}\hbox{\strut 128 NVIDIA CUDA$^{\circledR}$ cores}}} & {\vtop{\hbox{\strut Custom NPU}\hbox{\strut INT8 inference up to 1536 MAC}}} & {NVIDIA RTX 2080 Super Max-Q} \\
%\midrule
%{Interface} & {USB 3.0/2.0} & {1 x4(PCIe Gen2)} & {USB-3.0/PCIe Gen2}& {1 x4(PCIe Gen2)} \\
\midrule
{Ops}  & {-}& {472 GFLOPs} & {5.0 TOPS}& {-}\\
%\midrule
%{TDP} & {2.7-6.4 W}& {5W / 10W}& {24-30W}& {CPU: 45 W}\\
\midrule 
%{First Party OS} & {Raspbian 32/64 bit} & {Ubuntu 64 bit} & {Ubuntu 64 bit}& {Ubuntu 64 bit}\\
%\midrule
{Framework/runtime} & {TensorFlow Lite (FP32, XNNPACK backend)}& { ONNX Runtime (FP32, GPU)} & {AML NPU SDK (INT16)} & {OpenVINO (CPU, FP32)}\\
\bottomrule
\end{tabular}}
\end{table*}
 
\section{Performance of zero-cost accuracy predictors on \textit{YOLOBench}.}
\label{sec:appendix_C}

The performance of zero-cost accuracy predictors used in neural architecture search \cite{abdelfattah2021zero} is empirically evaluated on \textit{YOLOBench} models on VOC and SKU-110k. Table \ref{tab:zc_table_full} shows the Kendall-Tau scores and Pareto-optimal model prediction recall values obtained by a variety of zero-cost predictors. The zero-cost predictor values are computed using a randomly sampled batch of test set data with batch size $= 16$ (the used batch was the same for all ZC metrics). MAC count and the number of parameters are computed for models in evaluation mode, with normalization layers fused into preceding convolutions (if possible), and RepVGG-style blocks \cite{ding2021repvgg} also fused, if present in the model. Hence, the performance of MAC and parameter counts might slightly differ if computed for models in training mode. Most predictors perform poorly and are outperformed by the MAC count baseline, except for the NWOT score (in particular the pre-activation version of it). The good performance of NWOT can be also observed in Fig. \ref{fig:ZC_scatter}, where scatter plots of fine-tuned model mAP$_{50-95}$ vs. zero-cost predictor value are shown for a few predictors. Some predictors (notably parameter count, ZiCo, and Zen-score) can be observed to produce very close values for subsets of models with significantly different accuracy. This is an indication of the fact that these predictors perform poorly in estimating accuracy differences in models when the underlying architecture is fixed, but the input image resolution is varied. 

We also test the performance of a training-based predictor on \textit{YOLOBench} which is the mAP$_{50-95}$ values of models trained on a representative dataset (VOC) from scratch for 100 epochs. This predictor sets a strong baseline to be outperformed by training-free predictors, as it is generally found to perform well on a variety of datasets (see Fig. \ref{fig:voc_scratch_corr}), including datasets from different visual domains (e.g. SKU-110k).

We further look into the robustness of the results obtained with the pre-activation NWOT estimator. Since this zero-cost estimator does not require computing the loss function, the main parameters that could influence its performance are the exact batch of data sampled, the batch size, and the dataset split (training or test data) used to sample the batch. Figure \ref{fig:nwot_robust} shows the global Kendall-Tau scores achieved with pre-activation NWOT on VOC \textit{YOLOBench} models with different batches sampled, different batch sizes and different data splits used. There is an observed variance in performance depending on the sampled batch, which is higher when the test set data are used (with an absolute difference of up to $0.05$ in global Kendall-Tau score). Notably, scores computed on training set data (with augmentations) performed better on average compared to test set data, and performance is observed to decrease with increasing batch size. Furthermore, Table \ref{tab:nwot_robust_hohead} shows the mean and standard deviation of Kendall-Tau scores for the standard and pre-activation versions of NWOT on VOC \textit{YOLOBench} models computed on 5 different batches of size 16. We also estimate the performance of the mean predictor values averaged over the 5 sampled batches, which is expectedly found to outperform predictors computed on single batches. Moreover, we compute the pre-activation NWOT scores for all layers in YOLO models except the ones contained in detection heads. This is motivated by the fact that the larger distances between binary activation codes in NWOT are meant to correlate with better performance for the feature extraction layers (e.g. layers in the backbone and neck of YOLO), not the last layers used to compute model predictions. We find an overall performance boost in terms of Kendall-Tau scores for the case when the NWOT score is computed only for the layers in the backbone and neck (Table \ref{tab:nwot_robust_hohead}).

\section{Pareto-optimal model prediction using training-free proxies.}
\label{sec:appendix_D}

We evaluate the training-free accuracy predictors (and the training-based one, VOC training from scratch) for the task of predicting Pareto-optimal models. That is, if one computes the ZC values for each model and determines the Pareto set of models in the ZC value-real latency two-dimensional space, we want to estimate how many models in that Pareto set are going to also be present in the actual Pareto frontier (computed in the two-dimensional mAP$_{50-95}$-latency space). Two metrics are of importance here: recall (how many of actual mAP-latency Pareto-optimal models are captured by a ZC-based Pareto set) and precision (how many of ZC-based Pareto set models are actually Pareto optimal in the real mAP-latency space). Additionally, one could consider the first $N$ ($N=1, 2, 3,...$) ZC-based Pareto sets to expand the set of potential model candidates. We look at how precision and recall values change with $N$ for a few well-performing predictors (NWOT, pre-activation NWOT, MAC count, and VOC training from scratch) with latency values taken from different target devices. 

Recall values for several zero-cost predictors for Pareto models on Jetson Nano GPU and VOC dataset are shown in Fig. \ref{fig:ZC_pareto_VOC_nano}. Corresponding precision values for a few well-performing predictors on 3 different HW platforms are shown in Fig. \ref{fig:ZC_pareto_VOC_precision}. Recall values for these best-performing predictors on the SKU-110k dataset are shown in Fig. \ref{fig:zc_sku_parero_front}.

A different way to evaluate the predictors on \textit{YOLOBench} is to treat models with the same architectures but different input image resolutions as identical data points. That is, if a certain architecture is predicted by ZC-based Pareto front to be optimal on a certain resolution, we count that as a correct prediction if that same architecture on a different resolution is found to be really Pareto-optimal. Such a way to evaluate ZC performance stems from the fact that in practice one typically wishes to predict the most promising architectures, not necessarily the particular optimal image resolution (since that architecture would be pre-trained with a certain fixed resolution, e.g. 640x640 on a dataset like COCO for further fine-tuning on the target dataset). Recall and precision values for such an evaluation protocol for the VOC dataset are shown in Figs. \ref{fig:ZC_pareto_VOC_nores}, \ref{fig:ZC_pareto_VOC_nores_precision}.

We also evaluate the performance of the best training-free predictor (pre-activation NWOT) in predicting Pareto-optimal models, when the latency values used are different from actual latency measurements, but either are computed via a latency proxy like MAC count or measurement on another device. Note that in the case of MAC count as a latency predictor, the whole Pareto-frontier computation process is zero-cost: the approximation for mAP is given by the pre-activation NWOT score, the approximation for latency by MAC count. One might wonder how such a fully zero-cost approach performs in practice. Figures \ref{fig:ZC_pareto_VOC_latproxy} and \ref{fig:ZC_pareto_VOC_latproxy_prec} show the recall and precision values when accuracy predictor is taken to be pre-activation NWOT and latency predictors are varied from MAC count to latencies from other (proxy) devices. Interestingly, MAC count is found to perform relatively well in terms of recall, specifically for Raspberry Pi 4 CPU. Notably, none of the latency proxies work well to predict Pareto-optimal models on VIM3 NPU. Also, perhaps not surprisingly, using Intel CPU latency measurements works well to predict Pareto-optimal models on Raspberry Pi 4 CPU, but does not significantly outperform MAC count.

Finally, we test the pre-activation NWOT accuracy estimator to predict potentially well-performing models out of a set of YOLO models we generated with different CNN backbones from the \texttt{timm} package \cite{rw2019timm}. We have computed the NWOT-latency Pareto set for YOLO-PAN-C3 models with \texttt{timm} backbones on input images of 480x480 resolution, with latency measured on Raspberry Pi 4 ARM CPU (TFLite, FP32). The neck structure (PAN-C3) for each of the candidate models was taken to be that of YOLOv5s and the detection head to be that of YOLOv8 (same as for all \textit{YOLOBench} models), with Hardswish activations in the neck and head, and activation function(s) in the backbone kept the same as originally implemented in \texttt{timm}. Table \ref{tab:nwot_timm_pareto} shows examples of predicted Pareto-optimal models (a subset of the full NWOT-latency Pareto set). Based on these observations, we have selected FBNetV3-D as a potential backbone of a YOLO model to be trained on the COCO dataset and compared it to a reference YOLOv8 model in a similar latency range (YOLOv8s). 

Table \ref{tab:timm_coco_full} shows COCO minival mAP$_{50-95}$ and inference latency results for a YOLO-FBNetV3-D-PAN-C3 model trained on the COCO dataset for 300 epochs and profiled on 640x640 input resolution on Raspi4 CPU with TFLite. We observe that the choice of activation function significantly affects TFLite model inference latency, so for a more fair comparison we also train and profile a Hardswish-based version of YOLOv8s in addition to its default SiLU-based version. While we observe a significant reduction in inference latency with a negligible mAP drop shifting from SiLU to Hardswish, the FBNetV3-based model still outperforms YOLOv8s-HSwish. Furthermore, we train and profile a ReLU-based version of YOLO-FBNetV3-D-PAN-C3 (with activation functions in the backbone kept to be those of the original backbone, i.e. Hardswish, but neck and detection head activations replaced with ReLU) and observe further latency improvements at the cost of $\sim 0.56\%$ drop in mAP$_{50-95}$. However, this model is still found to outperform YOLOv8s in terms of both accuracy and latency (see Table \ref{tab:timm_coco_full}). Furthermore, we train the same models for 500 epochs with a batch size of $256$, which is found to achieve better results on COCO minival and test (Table \ref{tab:timm_coco}). Although we could not exactly reproduce COCO minival mAP results for YOLOv8s reported by Ultralytics \cite{ultralytics}, we find that the FBNetV3-based model outperforms both our YOLOv8s mAP results as well as those of Ultralytics, with lower latency on Raspberry Pi 4 CPU. The COCO minival mAP$_{50-95}$ values reported in Table \ref{tab:timm_coco} were obtained using \texttt{pycocotools} \cite{pycocotools} (with IoU threshold for NMS $=0.6$ and object confidence threshold for detection $=0.001$), and mAP values on test-dev were obtained using the same evaluation parameters by submitting to the competition server \cite{coco_test}. More details on the performance comparison of models on COCO test-dev are shown in Table \ref{tab:timm_coco_test}. 

%\clearpage

\begin{table*}
\small
\caption{Pareto-optimal \textit{YOLOBench} models on 3 datasets and 4 hardware platforms. Shown are the best models in terms of mAP$_{50-95}$ under a given latency threshold (max. latency). For each model, the scaling parameters are given (d33w25 means depth factor $=0.33$ and width factor $=0.25$), corresponding input resolution of the models is indicated in brackets.}
\vspace{1mm}
\label{tab:pareto_table_full} 
\begin{tabularx}{\linewidth}{lXXXXXX}
%\begin{tabularx}{\linewidth}{lssssss}
\toprule
{HW/max.} & {VOC} & {VOC} & {SKU-110k} & {SKU-110k} & {WIDERFACE} & {WIDERFACE} \\ %& {WIDER} & {WIDER} \\
{latency} & {model} & {mAP$_{50-95}$} & {model} & {mAP$_{50-95}$} & {model} & {mAP$_{50-95}$} \\ %& {model} & {mAP} \\
\midrule
{Nano/0.5 sec} & {YOLOv8} & {0.726} & {YOLOv7} & {0.593} & {YOLOv7} & {0.382}\\
{} & {d67w1 (448)} & {} & {d1w75 (480)} & {} & {d1w75 (480)} & {}\\
{Nano/0.3 sec} & {YOLOv7} & {0.701} & {YOLOv7} & {0.589} & {YOLOv7} & {0.369}\\
{} & {d1w5 (480)} & {} & {d1w5 (480)} & {} & {d1w5 (480)}\\
{Nano/0.1 sec} & {YOLOv7} & {0.657} & {YOLOv8} & {0.567} & {YOLOv7} & {0.336}\\
{} & {d1w5 (288)} & {} & {d1w25 (480)} & {} & {d1w25 (480)}\\
\midrule
{VIM3/0.3 sec} & {YOLOv8} & {0.726} & {YOLOv7} & {0.593} & {YOLOv7} & {0.382}\\
{} & {d67w1 (448)} & {} & {d1w75 (480)} & {} & {d1w75 (480)} & {}\\
{VIM3/0.1 sec} & {YOLOv6l} & {0.669} & {YOLOv8} & {0.567} & {YOLOv6m} & {0.350}\\
{} & {d67w5 (384)} & {} & {d1w25 (480)} & {} & {d33w5 (480)}\\
{VIM3/0.05 sec} & {YOLOv6l} & {0.620} & {YOLOv6s} & {0.556} & {YOLOv6m} & {0.318}\\
{} & {d67w25 (416)} & {} & {d33w25 (480)} & {} & {d67w25 (480)}\\
\midrule
{Intel/0.08 sec} & {YOLOv8} & {0.719} & {YOLOv7} & {0.593} & {YOLOv7} & {0.382}\\
{} & {d1w75 (416)} & {} & {d1w75 (480)} & {} & {d1w75 (480)} & {}\\
{Intel/0.04 sec} & {YOLOv7} & {0.701} & {YOLOv7} & {0.589} & {YOLOv7} & {0.369}\\
{} & {d1w5 (480)} & {} & {d1w5 (480)} & {} & {d1w5 (480)}\\
{Intel/0.02 sec} & {YOLOv6l} & {0.682} & {YOLOv6l} & {0.576} & {YOLOv6l} & {0.346}\\
{} & {d6w5 (448)} & {} & {d33w5 (480)} & {} & {d33w5 (480)}\\
\midrule
{Raspi4/3 sec} & {YOLOv8} & {0.719} & {YOLOv7} & {0.593} & {YOLOv7} & {0.382}\\
{} & {d1w75 (416)} & {} & {d1w75 (480)} & {} & {d1w75 (480)} & {}\\
{Raspi4/1 sec} & {YOLOv7} & {0.701} & {YOLOv7} & {0.589} & {YOLOv7} & {  0.369}\\
{} & {d1w5 (480)} & {} & {d1w5 (480)} & {} & {d1w5 (480)}\\
{Raspi4/0.5 sec} & {YOLOv6l} & {0.669} & {YOLOv4} & {0.569} & {YOLOv7} & {0.336}\\
{} & {d67w5 (384)} & {} & {d1w25 (480)} & {} & {d1w25 (480)}\\
\bottomrule
\end{tabularx}
\end{table*}

\begin{table*}
\small
\caption{Performance of training-free accuracy predictors on \textit{YOLOBench} models and two datasets (VOC and SKU-110k, from COCO-pretrained weights) compared to using mAP$_{50-95}$ of models trained from scratch on the VOC dataset as a predictor.}
\vspace{2mm}
\label{tab:zc_table_full}
\begin{tabularx}{\linewidth}{l|XXX|XXX}
% \begin{tabularx}{\linewidth}{lssssss}
\toprule
{} & \multicolumn{3}{c|}{\text{VOC, mAP$_{50-95}$}} & \multicolumn{3}{c}{\text{SKU-110k, mAP$_{50-95}$}}\\
\midrule
{Predictor metric} & {global $\tau$} & {top-15\% $\tau$} & {\%Pareto pred. (GPU)} & {global $\tau$} & {top-15\% $\tau$} & {\%Pareto pred. (GPU)}\\
\midrule    
{GraSP} & {-0.011} & {-0.068} & {0.062} & {0.040} & {0.032} & {0.025}\\
{Plain} & {0.029} & {0.069} & {0.015} & {-0.388} & {-0.176} & {0.025}\\
{JacobCov} & {0.095} & {-0.078} & {0.015} & {0.541} & {0.136} & {0.025}\\
{ZiCo} & {0.195} & {0.016} & {0.015} & {0.115} & {0.081} & {0.025}\\
{Zen} & {0.255} & {0.092} & {0.062} & {0.146} & {0.121} & {0.050}\\
{GradNorm} & {0.262} & {0.173} & {0.015} & {-0.331} & {-0.072} & {0.025}\\
{Fisher} & {0.280} & {0.156} & {0.015} & {-0.380} & {-0.096} & {0.025}\\
{L2 norm} & {0.326} & {0.090} & {0.015} & {0.189} & {0.118} & {0.025}\\
{SNIP} & {0.336} & {0.217} & {0.015} & {-0.290} & {-0.059} & {0.025}\\
{\#params} & {0.399} & {0.372} & {0.031} & {0.256} & {0.119} & {0.050}\\
{SynFlow} & {0.558} & {0.227} & {0.062} & {0.512} & {0.254} & {0.100}\\
{MACs} & {0.739} & {0.520} & {0.123} & {0.604} & {0.314} & {0.125}\\
{NWOT} & {0.756} & {0.622} & {0.262} & {0.703} & {0.321} & {\bf{0.200}}\\
{NWOT (pre-act)} & {{\bf 0.827}} & {{\bf 0.623}} & {{\bf 0.292}} & {{\bf 0.765}} & {{\bf 0.406}} & {{\bf 0.200}}\\
\midrule
{VOC training} & {0.847} & {0.665} & {0.369} & {0.739} & {0.374} & {0.425}\\
{from scratch (mAP$_{50-95}$)} &  {} & {} & {} & {}\\
% {mAP$_{50-95}$} & {} & {} & {} & {}\\
\bottomrule
\end{tabularx}
\end{table*}

% Figure environment removed

% Figure environment removed

% Figure environment removed

% Figure environment removed

% Figure environment removed

% Figure environment removed

% Figure environment removed

% Figure environment removed


% Figure environment removed

% Figure environment removed

% Figure environment removed

\begin{table*}
  \small
  \begin{center}
    \caption{Mean and standard deviation of the global Kendall-Tau scores for NWOT metrics computed for 5 different randomly sampled batches of size 16 on VOC \textit{YOLOBench} models. The metric denoted as "no head" was computed only for the layers contained in the neck and backbone of YOLO models, not in the detection head. The second column shows Kendall-Tau scores for prediction with the mean ZC metric values averaged over the 5 batches.}
    \label{tab:nwot_robust_hohead}
    \vspace*{2mm}
    \begin{tabularx}{\linewidth}{X|X|X} % <-- Alignments: 1st column left, 2nd middle and 3rd right, with vertical lines in between
      \hline
      {ZC metric} & {global $\tau$} & {global $\tau$ (prediction with mean ZC value)}\\
      % {} & {} & {ms} \\
      \hline
      {NWOT} & {0.7839 (0.0159)} & {0.7895} \\
      \hline
      {NWOT (pre-act)} & {0.8402 (0.0191)} & {0.8486} \\
      \hline
      {NWOT (pre-act, no head)} & {\textbf{0.8472 (0.0194)}} & {\textbf{0.8570}} \\
      \hline
    \end{tabularx}
  \end{center}
\end{table*}

% Figure environment removed

% Figure environment removed

% Figure environment removed

% Figure environment removed


% Figure environment removed

% Figure environment removed

% Figure environment removed



\begin{table*}
\centering
\caption{Example YOLO-PAN-C3 models with \texttt{timm} backbones identified in the NWOT-latency Pareto frontier, with pre-activation NWOT score computed on the VOC dataset. Latency values are measured on Raspberry Pi 4 ARM CPU with TFLite (FP32), batch size 1.}
\label{tab:nwot_timm_pareto}
%\begin{tabular}{c|c|c}
\begin{tabularx}{\textwidth}{l|X|X|X}
\toprule
% Model name & Input resolution & GMACs & NWOT (pre-act) \\
Model name & Input resolution & Raspi4 CPU latency, sec & NWOT (pre-act) \\
\midrule
\texttt{yolo\_pan\_efficientnet\_b4} & { 480} & { 1.72} & { 511.84 }  \\
\texttt{yolo\_pan\_tf\_efficientnet\_b4\_ap} & { 480} & { 1.71} & { 511.77 }  \\
\texttt{yolo\_pan\_gc\_efficientnetv2\_rw\_t} & { 480} & { 1.41} & { 508.73 }  \\
\texttt{yolo\_pan\_tf\_efficientnet\_lite4} & { 480} & { 1.08} & { 506.67 }  \\
\texttt{yolo\_pan\_fbnetv3\_d} & { 480} & { 0.71} & { 502.71 }  \\
\texttt{yolo\_pan\_tf\_efficientnet\_lite1} & { 480} & { 0.61} & { 493.48 }  \\
\texttt{yolo\_pan\_efficientnet\_lite1} & { 480} & { 0.61} & { 493.32 }  \\
\texttt{yolo\_pan\_mobilenetv2\_110d} & { 480} & { 0.54} & { 480.92 }  \\
\texttt{yolo\_pan\_mobilenetv2\_075} & { 480} & { 0.45} & { 480.14 }  \\
\texttt{yolo\_pan\_tf\_mobilenetv3\_large\_075} & { 480} & { 0.45} & { 468.85 }  \\
\texttt{yolo\_pan\_mobilenetv2\_035} & { 480} & { 0.37} & { 457.41 }  \\
\texttt{yolo\_pan\_tf\_mobilenetv3\_small\_minimal\_100} & { 480} & { 0.36} & { 451.10 }  \\
% \texttt{yolo\_pan\_efficientnet\_b6} & 416 & 265.67 & 520.07 \\
% \texttt{yolo\_pan\_mixnet\_xxl} & 480 & 234.37 & 518.63 \\
% \texttt{yolo\_pan\_regnetz\_d8\_evos} & 384 & 221.56 & 515.61 \\
% \texttt{yolo\_pan\_fbnetv3\_g} & 480 & 157.49 & 515.17 \\
% \texttt{yolo\_pan\_fbnetv3\_g} & 448 & 129.21 & 509.66 \\
% \texttt{yolo\_pan\_fbnetv3\_g} & 416 & 120.60 & 506.66 \\
% \texttt{yolo\_pan\_fbnetv3\_d} & 480 & 111.20 & 502.71 \\
% \texttt{yolo\_pan\_mobilenetv2\_120d} & 448 & 100.48 & 500.79 \\
% \texttt{yolo\_pan\_fbnetv3\_g} & 384 & 95.99 & 500.18 \\
% \texttt{yolo\_pan\_fbnetv3\_d} & 448 & 91.22 & 497.67 \\
% \texttt{yolo\_pan\_fbnetv3\_g} & 352 & 88.62 & 496.58 \\
% \texttt{yolo\_pan\_fbnetv3\_d} & 416 & 85.15 & 494.13 \\
% \texttt{yolo\_pan\_fbnetv3\_g} & 320 & 81.24 & 492.63 \\
% \texttt{yolo\_pan\_mobilenetv2\_120d} & 384 & 74.64 & 491.50 \\
% \texttt{yolo\_pan\_fbnetv3\_d} & 384 & 67.77 & 488.27 \\
% \texttt{yolo\_pan\_fbnetv3\_g} & 288 & 61.56 & 484.83 \\
% \texttt{yolo\_pan\_fbnetv3\_g} & 256 & 55.41 & 480.02 \\
% \texttt{yolo\_pan\_mobilenetv2\_120d} & 288 & 47.85 & 475.86 \\
% \texttt{yolo\_pan\_tf\_efficientnet\_b1\_ap} & 288 & 44.74 & 472.46 \\
% \texttt{yolo\_pan\_tf\_efficientnet\_b1\_ns} & 288 & 44.74 & 472.46 \\
\bottomrule
% \end{tabular}
\end{tabularx}
\end{table*}

\begin{table*}
  \small
  \begin{center}
    \caption{COCO test mAP values and inference latency on Raspberry Pi 4 CPU (TFLite with XNNPACK backend, FP32) for YOLOv8s vs. a model identified from the NWOT-latency Pareto frontier (YOLO-FBNetV3-D-PAN). For mAP values, the mean and standard deviation over three random seeds are shown. For inference time, mean and standard deviation of inference time over 5
    runs (each one 100 iterations) are shown.} 
    \label{tab:timm_coco_test}
    \vspace*{2mm}
    \begin{tabularx}{\linewidth}{X|X|X|X|X|X|X|X} % <-- Alignments: 1st column left, 2nd middle and 3rd right, with vertical lines in between
      \hline
      {Model} & {AP$^{test}_{50-95}$} & {AP$^{test}_{50}$} & {AP$^{test}_{75}$} & {AP$^{test}_{S}$} & AP$^{test}_{M}$ & AP$^{test}_{L}$ & {Latency, ms}\\
      % {} & {} & {ms} \\
      \hline
      {YOLOv8s} & {43.17\% (0.12\%)} & {60.53\% (0.09\%)} & {46.5\% (0.08\%)} & {\textbf{22.7\%} (0.14\%)} & {47.13\% (0.17\%)} & {57.0\% (0.22\%)} & {1476.09 (1.49)} \\
      \hline
      {YOLOv8s-HSwish} & {42.90\% (0.0\%)} & {60.3\% (0.0\%)} & {46.30\% (0.0\%)} & {22.46\% (0.09\%)} & {47.0\% (0.08\%)} & {56.39\% (0.08\%)} & {1381.62 (7.34)} \\
      \hline
      {YOLO-FBNetV3-D-PAN} & {\textbf{43.87\%} (0.05\%)} & {\textbf{61.53\%} (0.09\%)} & {\textbf{47.23\%} (0.05\%)} & {22.67\% (0.19\%)} & {\textbf{47.87\%} (0.05\%)} & {\textbf{58.36\%} (0.12\%)} & {\textbf{1355.21} (9.93)} \\
      \hline
    \end{tabularx}
  \end{center}
\end{table*}


\begin{table*}
  \small
  \begin{center}
    \caption{COCO minival mAP and inference latency on Raspberry Pi 4 CPU (TFLite with XNNPACK backend, FP32) for YOLOv8s vs. a model identified from the NWOT-latency Pareto frontier (YOLO-FBNetV3-D-PAN). Mean and standard deviation of inference time over 5 runs (each one 100 iterations) are shown. The input image resolution used was 640x640, batch size $=1$ for latency measurements. Models were trained for 300 epochs, with batch size $=64$.} %All models trained for 300 epochs from random initialization, benchmarked on 640x640 input resolution.}
    \label{tab:timm_coco_full}
    \vspace*{2mm}
    \begin{tabularx}{\linewidth}{X|X|X} % <-- Alignments: 1st column left, 2nd middle and 3rd right, with vertical lines in between
      \hline
      {Model} & {COCO mAP$^{val}_{50-95}$} & {Raspberry Pi 4 ARM CPU latency, ms}\\
      % {} & {} & {ms} \\
      \hline
      {YOLOv8s} & {43.64\%} & {1476.09 (1.49)} \\
      \hline
      {YOLOv8s-HSwish} & {43.55\%} & {1381.62 (7.34)} \\
      \hline
      {YOLO-FBNetV3-D-PAN} & {44.63\%} & {1355.21 (9.93)} \\
      \hline
      {YOLO-FBNetV3-D-PAN-ReLU} & {44.07\%} & {1344.50 (8.06)} \\
      \hline
    \end{tabularx}
  \end{center}
\end{table*}

%\ifarxiv \clearpage \renewcommand{\thefigure}{S\arabic{figure}}
\renewcommand{\thetable}{S\arabic{table}}

\appendix
\label{sec:appendix}

\section{Latency measurements.}
\label{sec:appendix_A}

Details regarding hardware platforms used to collect latency measurements are outlined in Table \ref{tab:benchmarking_hardware}. Figures \ref{fig:nano_timing_corr} and \ref{fig:raspi4_timing_corr} show the difference of latency value distributions between devices computed for the full initial \textit{YOLOBench} architecture space consisting of $\sim$1000 models. While generally good correlation is observed between model inference latencies on different devices (see also Figure \ref{fig:timing_corr_matrix}), notably latency values measured on Khadas VIM3 NPU differ significantly from latency values on other devices. That is, for models with roughly the same latency on Jetson Nano GPU or Raspi4 ARM CPU, the difference in VIM3 NPU latency could be up to several times.  This difference between NPU values from other common GPU/CPU-based platforms highlights the necessity to develop hardware-aware architecture design and search methods. The difference in the NPU benchmark is also reflected in the structure of model Pareto frontiers (Figs. \ref{fig:voc_pareto}, \ref{fig:coco_pareto}, \ref{fig:sku_pareto}) and the performance of zero-cost predictors in identifying Pareto-optimal models (Figs. \ref{fig:zc_pareto_voc}, \ref{fig:zc_sku_parero_front}). 

\section{\textit{YOLOBench} Pareto frontiers for different datasets.}
\label{sec:appendix_B}

\textit{YOLOBench} Pareto frontiers for SKU-110k, WIDER FACE, and COCO datasets are shown in Figs. \ref{fig:sku_pareto}, \ref{fig:wider_pareto}, \ref{fig:coco_pareto}, correspondingly. Note that while mAP$_{50-95}$ values for VOC, SKU-110k, and WIDER FACE datasets are obtained by fine-tuning COCO pre-trained weights (all trained at 640x640 image resolution) on multiple image resolutions considered in \textit{YOLOBench} (11 values from 160 to 480 with a step of 32), the mAP$_{50-95}$ values on the COCO dataset are obtained by directly evaluating pre-trained COCO weights, without fine-tuning on the corresponding target image resolutions. This corresponds to the situation of deployment of pre-trained COCO weights without any additional training.

Table \ref{tab:pareto_table_full} shows the identified Pareto-optimal YOLO models on 3 different datasets and 4 hardware platforms under several latency thresholds. It can be noted that under the same latency threshold on a given hardware platform, the optimal YOLO model family and input image resolution are typically dataset-dependent.

Figures \ref{fig:scaling_raspi4} and \ref{fig:scaling_vim3} show the statistics of architecture scaling parameters (width factor, depth factor, image resolution) in Pareto-optimal models on Raspberry Pi4 CPU and VIM3 NPU, respectively. Although some differences are observed between devices and datasets (in particular depth factor distributions), there is a general trend in all computed Pareto fronts where a variation in depth/width factors is observed at higher resolutions, and resolution is reduced when the depth/width factors (especially the width factor) already have low values.

\begin{table*}[ht!]
\caption{Details on hardware platforms and corresponding runtimes used for benchmarking.}
\vspace*{3mm}
\label{tab:benchmarking_hardware}
\resizebox{\linewidth}{!}{
\begin{tabular}{l|c|c|c|c}
% \begin{tabularx}{\linewidth}{lsssss}
\toprule
{\bf } & {\bf Raspberry Pi 4 Model B} & {\bf Jetson Nano (NVIDIA)} & {\bf Khadas VIM3} & {\bf Lambda tensorbook}\\
\midrule
{CPU} & {\vtop{\hbox{\strut Quad Core Cortex-A72,}\hbox{\strut 64-bit SoC @1.8GHz}}} & {\vtop{\hbox{\strut Quad Core Cortex-A57 MPCore,}\hbox{\strut 64-bit SoC @1.43GHz}}} & {\vtop{\hbox{\strut Quad Core Cortex-A73 @2.2Ghz,}\hbox{\strut Dual Core Cortex-A53 @1.8Ghz}}} & {\vtop{\hbox{\strut Intel$^{\circledR}$ Core\texttrademark i7-10875H CPU}\hbox{\strut @ 2.30GHz}}} \\
\midrule
{Memory} & {4GB LPDDR4-3200 SDRAM} & {\vtop{\hbox{\strut 4 GB 64-bit LPDDR4,}\hbox{\strut 1600MHz 25.6 GB/s}}} & {4GB LPDDR4/4X} & {64GB DDR4 SDRAM}\\
\midrule
{AI-chip} & {-} & {\vtop{\hbox{\strut NVIDIA Maxwell GPU,}\hbox{\strut 128 NVIDIA CUDA$^{\circledR}$ cores}}} & {\vtop{\hbox{\strut Custom NPU}\hbox{\strut INT8 inference up to 1536 MAC}}} & {NVIDIA RTX 2080 Super Max-Q} \\
%\midrule
%{Interface} & {USB 3.0/2.0} & {1 x4(PCIe Gen2)} & {USB-3.0/PCIe Gen2}& {1 x4(PCIe Gen2)} \\
\midrule
{Ops}  & {-}& {472 GFLOPs} & {5.0 TOPS}& {-}\\
%\midrule
%{TDP} & {2.7-6.4 W}& {5W / 10W}& {24-30W}& {CPU: 45 W}\\
\midrule 
%{First Party OS} & {Raspbian 32/64 bit} & {Ubuntu 64 bit} & {Ubuntu 64 bit}& {Ubuntu 64 bit}\\
%\midrule
{Framework/runtime} & {TensorFlow Lite (FP32, XNNPACK backend)}& { ONNX Runtime (FP32, GPU)} & {AML NPU SDK (INT16)} & {OpenVINO (CPU, FP32)}\\
\bottomrule
\end{tabular}}
\end{table*}
 
\section{Performance of zero-cost accuracy predictors on \textit{YOLOBench}.}
\label{sec:appendix_C}

The performance of zero-cost accuracy predictors used in neural architecture search \cite{abdelfattah2021zero} is empirically evaluated on \textit{YOLOBench} models on VOC and SKU-110k. Table \ref{tab:zc_table_full} shows the Kendall-Tau scores and Pareto-optimal model prediction recall values obtained by a variety of zero-cost predictors. The zero-cost predictor values are computed using a randomly sampled batch of test set data with batch size $= 16$ (the used batch was the same for all ZC metrics). MAC count and the number of parameters are computed for models in evaluation mode, with normalization layers fused into preceding convolutions (if possible), and RepVGG-style blocks \cite{ding2021repvgg} also fused, if present in the model. Hence, the performance of MAC and parameter counts might slightly differ if computed for models in training mode. Most predictors perform poorly and are outperformed by the MAC count baseline, except for the NWOT score (in particular the pre-activation version of it). The good performance of NWOT can be also observed in Fig. \ref{fig:ZC_scatter}, where scatter plots of fine-tuned model mAP$_{50-95}$ vs. zero-cost predictor value are shown for a few predictors. Some predictors (notably parameter count, ZiCo, and Zen-score) can be observed to produce very close values for subsets of models with significantly different accuracy. This is an indication of the fact that these predictors perform poorly in estimating accuracy differences in models when the underlying architecture is fixed, but the input image resolution is varied. 

We also test the performance of a training-based predictor on \textit{YOLOBench} which is the mAP$_{50-95}$ values of models trained on a representative dataset (VOC) from scratch for 100 epochs. This predictor sets a strong baseline to be outperformed by training-free predictors, as it is generally found to perform well on a variety of datasets (see Fig. \ref{fig:voc_scratch_corr}), including datasets from different visual domains (e.g. SKU-110k).

We further look into the robustness of the results obtained with the pre-activation NWOT estimator. Since this zero-cost estimator does not require computing the loss function, the main parameters that could influence its performance are the exact batch of data sampled, the batch size, and the dataset split (training or test data) used to sample the batch. Figure \ref{fig:nwot_robust} shows the global Kendall-Tau scores achieved with pre-activation NWOT on VOC \textit{YOLOBench} models with different batches sampled, different batch sizes and different data splits used. There is an observed variance in performance depending on the sampled batch, which is higher when the test set data are used (with an absolute difference of up to $0.05$ in global Kendall-Tau score). Notably, scores computed on training set data (with augmentations) performed better on average compared to test set data, and performance is observed to decrease with increasing batch size. Furthermore, Table \ref{tab:nwot_robust_hohead} shows the mean and standard deviation of Kendall-Tau scores for the standard and pre-activation versions of NWOT on VOC \textit{YOLOBench} models computed on 5 different batches of size 16. We also estimate the performance of the mean predictor values averaged over the 5 sampled batches, which is expectedly found to outperform predictors computed on single batches. Moreover, we compute the pre-activation NWOT scores for all layers in YOLO models except the ones contained in detection heads. This is motivated by the fact that the larger distances between binary activation codes in NWOT are meant to correlate with better performance for the feature extraction layers (e.g. layers in the backbone and neck of YOLO), not the last layers used to compute model predictions. We find an overall performance boost in terms of Kendall-Tau scores for the case when the NWOT score is computed only for the layers in the backbone and neck (Table \ref{tab:nwot_robust_hohead}).

\section{Pareto-optimal model prediction using training-free proxies.}
\label{sec:appendix_D}

We evaluate the training-free accuracy predictors (and the training-based one, VOC training from scratch) for the task of predicting Pareto-optimal models. That is, if one computes the ZC values for each model and determines the Pareto set of models in the ZC value-real latency two-dimensional space, we want to estimate how many models in that Pareto set are going to also be present in the actual Pareto frontier (computed in the two-dimensional mAP$_{50-95}$-latency space). Two metrics are of importance here: recall (how many of actual mAP-latency Pareto-optimal models are captured by a ZC-based Pareto set) and precision (how many of ZC-based Pareto set models are actually Pareto optimal in the real mAP-latency space). Additionally, one could consider the first $N$ ($N=1, 2, 3,...$) ZC-based Pareto sets to expand the set of potential model candidates. We look at how precision and recall values change with $N$ for a few well-performing predictors (NWOT, pre-activation NWOT, MAC count, and VOC training from scratch) with latency values taken from different target devices. 

Recall values for several zero-cost predictors for Pareto models on Jetson Nano GPU and VOC dataset are shown in Fig. \ref{fig:ZC_pareto_VOC_nano}. Corresponding precision values for a few well-performing predictors on 3 different HW platforms are shown in Fig. \ref{fig:ZC_pareto_VOC_precision}. Recall values for these best-performing predictors on the SKU-110k dataset are shown in Fig. \ref{fig:zc_sku_parero_front}.

A different way to evaluate the predictors on \textit{YOLOBench} is to treat models with the same architectures but different input image resolutions as identical data points. That is, if a certain architecture is predicted by ZC-based Pareto front to be optimal on a certain resolution, we count that as a correct prediction if that same architecture on a different resolution is found to be really Pareto-optimal. Such a way to evaluate ZC performance stems from the fact that in practice one typically wishes to predict the most promising architectures, not necessarily the particular optimal image resolution (since that architecture would be pre-trained with a certain fixed resolution, e.g. 640x640 on a dataset like COCO for further fine-tuning on the target dataset). Recall and precision values for such an evaluation protocol for the VOC dataset are shown in Figs. \ref{fig:ZC_pareto_VOC_nores}, \ref{fig:ZC_pareto_VOC_nores_precision}.

We also evaluate the performance of the best training-free predictor (pre-activation NWOT) in predicting Pareto-optimal models, when the latency values used are different from actual latency measurements, but either are computed via a latency proxy like MAC count or measurement on another device. Note that in the case of MAC count as a latency predictor, the whole Pareto-frontier computation process is zero-cost: the approximation for mAP is given by the pre-activation NWOT score, the approximation for latency by MAC count. One might wonder how such a fully zero-cost approach performs in practice. Figures \ref{fig:ZC_pareto_VOC_latproxy} and \ref{fig:ZC_pareto_VOC_latproxy_prec} show the recall and precision values when accuracy predictor is taken to be pre-activation NWOT and latency predictors are varied from MAC count to latencies from other (proxy) devices. Interestingly, MAC count is found to perform relatively well in terms of recall, specifically for Raspberry Pi 4 CPU. Notably, none of the latency proxies work well to predict Pareto-optimal models on VIM3 NPU. Also, perhaps not surprisingly, using Intel CPU latency measurements works well to predict Pareto-optimal models on Raspberry Pi 4 CPU, but does not significantly outperform MAC count.

Finally, we test the pre-activation NWOT accuracy estimator to predict potentially well-performing models out of a set of YOLO models we generated with different CNN backbones from the \texttt{timm} package \cite{rw2019timm}. We have computed the NWOT-latency Pareto set for YOLO-PAN-C3 models with \texttt{timm} backbones on input images of 480x480 resolution, with latency measured on Raspberry Pi 4 ARM CPU (TFLite, FP32). The neck structure (PAN-C3) for each of the candidate models was taken to be that of YOLOv5s and the detection head to be that of YOLOv8 (same as for all \textit{YOLOBench} models), with Hardswish activations in the neck and head, and activation function(s) in the backbone kept the same as originally implemented in \texttt{timm}. Table \ref{tab:nwot_timm_pareto} shows examples of predicted Pareto-optimal models (a subset of the full NWOT-latency Pareto set). Based on these observations, we have selected FBNetV3-D as a potential backbone of a YOLO model to be trained on the COCO dataset and compared it to a reference YOLOv8 model in a similar latency range (YOLOv8s). 

Table \ref{tab:timm_coco_full} shows COCO minival mAP$_{50-95}$ and inference latency results for a YOLO-FBNetV3-D-PAN-C3 model trained on the COCO dataset for 300 epochs and profiled on 640x640 input resolution on Raspi4 CPU with TFLite. We observe that the choice of activation function significantly affects TFLite model inference latency, so for a more fair comparison we also train and profile a Hardswish-based version of YOLOv8s in addition to its default SiLU-based version. While we observe a significant reduction in inference latency with a negligible mAP drop shifting from SiLU to Hardswish, the FBNetV3-based model still outperforms YOLOv8s-HSwish. Furthermore, we train and profile a ReLU-based version of YOLO-FBNetV3-D-PAN-C3 (with activation functions in the backbone kept to be those of the original backbone, i.e. Hardswish, but neck and detection head activations replaced with ReLU) and observe further latency improvements at the cost of $\sim 0.56\%$ drop in mAP$_{50-95}$. However, this model is still found to outperform YOLOv8s in terms of both accuracy and latency (see Table \ref{tab:timm_coco_full}). Furthermore, we train the same models for 500 epochs with a batch size of $256$, which is found to achieve better results on COCO minival and test (Table \ref{tab:timm_coco}). Although we could not exactly reproduce COCO minival mAP results for YOLOv8s reported by Ultralytics \cite{ultralytics}, we find that the FBNetV3-based model outperforms both our YOLOv8s mAP results as well as those of Ultralytics, with lower latency on Raspberry Pi 4 CPU. The COCO minival mAP$_{50-95}$ values reported in Table \ref{tab:timm_coco} were obtained using \texttt{pycocotools} \cite{pycocotools} (with IoU threshold for NMS $=0.6$ and object confidence threshold for detection $=0.001$), and mAP values on test-dev were obtained using the same evaluation parameters by submitting to the competition server \cite{coco_test}. More details on the performance comparison of models on COCO test-dev are shown in Table \ref{tab:timm_coco_test}. 

%\clearpage

\begin{table*}
\small
\caption{Pareto-optimal \textit{YOLOBench} models on 3 datasets and 4 hardware platforms. Shown are the best models in terms of mAP$_{50-95}$ under a given latency threshold (max. latency). For each model, the scaling parameters are given (d33w25 means depth factor $=0.33$ and width factor $=0.25$), corresponding input resolution of the models is indicated in brackets.}
\vspace{1mm}
\label{tab:pareto_table_full} 
\begin{tabularx}{\linewidth}{lXXXXXX}
%\begin{tabularx}{\linewidth}{lssssss}
\toprule
{HW/max.} & {VOC} & {VOC} & {SKU-110k} & {SKU-110k} & {WIDERFACE} & {WIDERFACE} \\ %& {WIDER} & {WIDER} \\
{latency} & {model} & {mAP$_{50-95}$} & {model} & {mAP$_{50-95}$} & {model} & {mAP$_{50-95}$} \\ %& {model} & {mAP} \\
\midrule
{Nano/0.5 sec} & {YOLOv8} & {0.726} & {YOLOv7} & {0.593} & {YOLOv7} & {0.382}\\
{} & {d67w1 (448)} & {} & {d1w75 (480)} & {} & {d1w75 (480)} & {}\\
{Nano/0.3 sec} & {YOLOv7} & {0.701} & {YOLOv7} & {0.589} & {YOLOv7} & {0.369}\\
{} & {d1w5 (480)} & {} & {d1w5 (480)} & {} & {d1w5 (480)}\\
{Nano/0.1 sec} & {YOLOv7} & {0.657} & {YOLOv8} & {0.567} & {YOLOv7} & {0.336}\\
{} & {d1w5 (288)} & {} & {d1w25 (480)} & {} & {d1w25 (480)}\\
\midrule
{VIM3/0.3 sec} & {YOLOv8} & {0.726} & {YOLOv7} & {0.593} & {YOLOv7} & {0.382}\\
{} & {d67w1 (448)} & {} & {d1w75 (480)} & {} & {d1w75 (480)} & {}\\
{VIM3/0.1 sec} & {YOLOv6l} & {0.669} & {YOLOv8} & {0.567} & {YOLOv6m} & {0.350}\\
{} & {d67w5 (384)} & {} & {d1w25 (480)} & {} & {d33w5 (480)}\\
{VIM3/0.05 sec} & {YOLOv6l} & {0.620} & {YOLOv6s} & {0.556} & {YOLOv6m} & {0.318}\\
{} & {d67w25 (416)} & {} & {d33w25 (480)} & {} & {d67w25 (480)}\\
\midrule
{Intel/0.08 sec} & {YOLOv8} & {0.719} & {YOLOv7} & {0.593} & {YOLOv7} & {0.382}\\
{} & {d1w75 (416)} & {} & {d1w75 (480)} & {} & {d1w75 (480)} & {}\\
{Intel/0.04 sec} & {YOLOv7} & {0.701} & {YOLOv7} & {0.589} & {YOLOv7} & {0.369}\\
{} & {d1w5 (480)} & {} & {d1w5 (480)} & {} & {d1w5 (480)}\\
{Intel/0.02 sec} & {YOLOv6l} & {0.682} & {YOLOv6l} & {0.576} & {YOLOv6l} & {0.346}\\
{} & {d6w5 (448)} & {} & {d33w5 (480)} & {} & {d33w5 (480)}\\
\midrule
{Raspi4/3 sec} & {YOLOv8} & {0.719} & {YOLOv7} & {0.593} & {YOLOv7} & {0.382}\\
{} & {d1w75 (416)} & {} & {d1w75 (480)} & {} & {d1w75 (480)} & {}\\
{Raspi4/1 sec} & {YOLOv7} & {0.701} & {YOLOv7} & {0.589} & {YOLOv7} & {  0.369}\\
{} & {d1w5 (480)} & {} & {d1w5 (480)} & {} & {d1w5 (480)}\\
{Raspi4/0.5 sec} & {YOLOv6l} & {0.669} & {YOLOv4} & {0.569} & {YOLOv7} & {0.336}\\
{} & {d67w5 (384)} & {} & {d1w25 (480)} & {} & {d1w25 (480)}\\
\bottomrule
\end{tabularx}
\end{table*}

\begin{table*}
\small
\caption{Performance of training-free accuracy predictors on \textit{YOLOBench} models and two datasets (VOC and SKU-110k, from COCO-pretrained weights) compared to using mAP$_{50-95}$ of models trained from scratch on the VOC dataset as a predictor.}
\vspace{2mm}
\label{tab:zc_table_full}
\begin{tabularx}{\linewidth}{l|XXX|XXX}
% \begin{tabularx}{\linewidth}{lssssss}
\toprule
{} & \multicolumn{3}{c|}{\text{VOC, mAP$_{50-95}$}} & \multicolumn{3}{c}{\text{SKU-110k, mAP$_{50-95}$}}\\
\midrule
{Predictor metric} & {global $\tau$} & {top-15\% $\tau$} & {\%Pareto pred. (GPU)} & {global $\tau$} & {top-15\% $\tau$} & {\%Pareto pred. (GPU)}\\
\midrule    
{GraSP} & {-0.011} & {-0.068} & {0.062} & {0.040} & {0.032} & {0.025}\\
{Plain} & {0.029} & {0.069} & {0.015} & {-0.388} & {-0.176} & {0.025}\\
{JacobCov} & {0.095} & {-0.078} & {0.015} & {0.541} & {0.136} & {0.025}\\
{ZiCo} & {0.195} & {0.016} & {0.015} & {0.115} & {0.081} & {0.025}\\
{Zen} & {0.255} & {0.092} & {0.062} & {0.146} & {0.121} & {0.050}\\
{GradNorm} & {0.262} & {0.173} & {0.015} & {-0.331} & {-0.072} & {0.025}\\
{Fisher} & {0.280} & {0.156} & {0.015} & {-0.380} & {-0.096} & {0.025}\\
{L2 norm} & {0.326} & {0.090} & {0.015} & {0.189} & {0.118} & {0.025}\\
{SNIP} & {0.336} & {0.217} & {0.015} & {-0.290} & {-0.059} & {0.025}\\
{\#params} & {0.399} & {0.372} & {0.031} & {0.256} & {0.119} & {0.050}\\
{SynFlow} & {0.558} & {0.227} & {0.062} & {0.512} & {0.254} & {0.100}\\
{MACs} & {0.739} & {0.520} & {0.123} & {0.604} & {0.314} & {0.125}\\
{NWOT} & {0.756} & {0.622} & {0.262} & {0.703} & {0.321} & {\bf{0.200}}\\
{NWOT (pre-act)} & {{\bf 0.827}} & {{\bf 0.623}} & {{\bf 0.292}} & {{\bf 0.765}} & {{\bf 0.406}} & {{\bf 0.200}}\\
\midrule
{VOC training} & {0.847} & {0.665} & {0.369} & {0.739} & {0.374} & {0.425}\\
{from scratch (mAP$_{50-95}$)} &  {} & {} & {} & {}\\
% {mAP$_{50-95}$} & {} & {} & {} & {}\\
\bottomrule
\end{tabularx}
\end{table*}

% Figure environment removed

% Figure environment removed

% Figure environment removed

% Figure environment removed

% Figure environment removed

% Figure environment removed

% Figure environment removed

% Figure environment removed


% Figure environment removed

% Figure environment removed

% Figure environment removed

\begin{table*}
  \small
  \begin{center}
    \caption{Mean and standard deviation of the global Kendall-Tau scores for NWOT metrics computed for 5 different randomly sampled batches of size 16 on VOC \textit{YOLOBench} models. The metric denoted as "no head" was computed only for the layers contained in the neck and backbone of YOLO models, not in the detection head. The second column shows Kendall-Tau scores for prediction with the mean ZC metric values averaged over the 5 batches.}
    \label{tab:nwot_robust_hohead}
    \vspace*{2mm}
    \begin{tabularx}{\linewidth}{X|X|X} % <-- Alignments: 1st column left, 2nd middle and 3rd right, with vertical lines in between
      \hline
      {ZC metric} & {global $\tau$} & {global $\tau$ (prediction with mean ZC value)}\\
      % {} & {} & {ms} \\
      \hline
      {NWOT} & {0.7839 (0.0159)} & {0.7895} \\
      \hline
      {NWOT (pre-act)} & {0.8402 (0.0191)} & {0.8486} \\
      \hline
      {NWOT (pre-act, no head)} & {\textbf{0.8472 (0.0194)}} & {\textbf{0.8570}} \\
      \hline
    \end{tabularx}
  \end{center}
\end{table*}

% Figure environment removed

% Figure environment removed

% Figure environment removed

% Figure environment removed


% Figure environment removed

% Figure environment removed

% Figure environment removed



\begin{table*}
\centering
\caption{Example YOLO-PAN-C3 models with \texttt{timm} backbones identified in the NWOT-latency Pareto frontier, with pre-activation NWOT score computed on the VOC dataset. Latency values are measured on Raspberry Pi 4 ARM CPU with TFLite (FP32), batch size 1.}
\label{tab:nwot_timm_pareto}
%\begin{tabular}{c|c|c}
\begin{tabularx}{\textwidth}{l|X|X|X}
\toprule
% Model name & Input resolution & GMACs & NWOT (pre-act) \\
Model name & Input resolution & Raspi4 CPU latency, sec & NWOT (pre-act) \\
\midrule
\texttt{yolo\_pan\_efficientnet\_b4} & { 480} & { 1.72} & { 511.84 }  \\
\texttt{yolo\_pan\_tf\_efficientnet\_b4\_ap} & { 480} & { 1.71} & { 511.77 }  \\
\texttt{yolo\_pan\_gc\_efficientnetv2\_rw\_t} & { 480} & { 1.41} & { 508.73 }  \\
\texttt{yolo\_pan\_tf\_efficientnet\_lite4} & { 480} & { 1.08} & { 506.67 }  \\
\texttt{yolo\_pan\_fbnetv3\_d} & { 480} & { 0.71} & { 502.71 }  \\
\texttt{yolo\_pan\_tf\_efficientnet\_lite1} & { 480} & { 0.61} & { 493.48 }  \\
\texttt{yolo\_pan\_efficientnet\_lite1} & { 480} & { 0.61} & { 493.32 }  \\
\texttt{yolo\_pan\_mobilenetv2\_110d} & { 480} & { 0.54} & { 480.92 }  \\
\texttt{yolo\_pan\_mobilenetv2\_075} & { 480} & { 0.45} & { 480.14 }  \\
\texttt{yolo\_pan\_tf\_mobilenetv3\_large\_075} & { 480} & { 0.45} & { 468.85 }  \\
\texttt{yolo\_pan\_mobilenetv2\_035} & { 480} & { 0.37} & { 457.41 }  \\
\texttt{yolo\_pan\_tf\_mobilenetv3\_small\_minimal\_100} & { 480} & { 0.36} & { 451.10 }  \\
% \texttt{yolo\_pan\_efficientnet\_b6} & 416 & 265.67 & 520.07 \\
% \texttt{yolo\_pan\_mixnet\_xxl} & 480 & 234.37 & 518.63 \\
% \texttt{yolo\_pan\_regnetz\_d8\_evos} & 384 & 221.56 & 515.61 \\
% \texttt{yolo\_pan\_fbnetv3\_g} & 480 & 157.49 & 515.17 \\
% \texttt{yolo\_pan\_fbnetv3\_g} & 448 & 129.21 & 509.66 \\
% \texttt{yolo\_pan\_fbnetv3\_g} & 416 & 120.60 & 506.66 \\
% \texttt{yolo\_pan\_fbnetv3\_d} & 480 & 111.20 & 502.71 \\
% \texttt{yolo\_pan\_mobilenetv2\_120d} & 448 & 100.48 & 500.79 \\
% \texttt{yolo\_pan\_fbnetv3\_g} & 384 & 95.99 & 500.18 \\
% \texttt{yolo\_pan\_fbnetv3\_d} & 448 & 91.22 & 497.67 \\
% \texttt{yolo\_pan\_fbnetv3\_g} & 352 & 88.62 & 496.58 \\
% \texttt{yolo\_pan\_fbnetv3\_d} & 416 & 85.15 & 494.13 \\
% \texttt{yolo\_pan\_fbnetv3\_g} & 320 & 81.24 & 492.63 \\
% \texttt{yolo\_pan\_mobilenetv2\_120d} & 384 & 74.64 & 491.50 \\
% \texttt{yolo\_pan\_fbnetv3\_d} & 384 & 67.77 & 488.27 \\
% \texttt{yolo\_pan\_fbnetv3\_g} & 288 & 61.56 & 484.83 \\
% \texttt{yolo\_pan\_fbnetv3\_g} & 256 & 55.41 & 480.02 \\
% \texttt{yolo\_pan\_mobilenetv2\_120d} & 288 & 47.85 & 475.86 \\
% \texttt{yolo\_pan\_tf\_efficientnet\_b1\_ap} & 288 & 44.74 & 472.46 \\
% \texttt{yolo\_pan\_tf\_efficientnet\_b1\_ns} & 288 & 44.74 & 472.46 \\
\bottomrule
% \end{tabular}
\end{tabularx}
\end{table*}

\begin{table*}
  \small
  \begin{center}
    \caption{COCO test mAP values and inference latency on Raspberry Pi 4 CPU (TFLite with XNNPACK backend, FP32) for YOLOv8s vs. a model identified from the NWOT-latency Pareto frontier (YOLO-FBNetV3-D-PAN). For mAP values, the mean and standard deviation over three random seeds are shown. For inference time, mean and standard deviation of inference time over 5
    runs (each one 100 iterations) are shown.} 
    \label{tab:timm_coco_test}
    \vspace*{2mm}
    \begin{tabularx}{\linewidth}{X|X|X|X|X|X|X|X} % <-- Alignments: 1st column left, 2nd middle and 3rd right, with vertical lines in between
      \hline
      {Model} & {AP$^{test}_{50-95}$} & {AP$^{test}_{50}$} & {AP$^{test}_{75}$} & {AP$^{test}_{S}$} & AP$^{test}_{M}$ & AP$^{test}_{L}$ & {Latency, ms}\\
      % {} & {} & {ms} \\
      \hline
      {YOLOv8s} & {43.17\% (0.12\%)} & {60.53\% (0.09\%)} & {46.5\% (0.08\%)} & {\textbf{22.7\%} (0.14\%)} & {47.13\% (0.17\%)} & {57.0\% (0.22\%)} & {1476.09 (1.49)} \\
      \hline
      {YOLOv8s-HSwish} & {42.90\% (0.0\%)} & {60.3\% (0.0\%)} & {46.30\% (0.0\%)} & {22.46\% (0.09\%)} & {47.0\% (0.08\%)} & {56.39\% (0.08\%)} & {1381.62 (7.34)} \\
      \hline
      {YOLO-FBNetV3-D-PAN} & {\textbf{43.87\%} (0.05\%)} & {\textbf{61.53\%} (0.09\%)} & {\textbf{47.23\%} (0.05\%)} & {22.67\% (0.19\%)} & {\textbf{47.87\%} (0.05\%)} & {\textbf{58.36\%} (0.12\%)} & {\textbf{1355.21} (9.93)} \\
      \hline
    \end{tabularx}
  \end{center}
\end{table*}


\begin{table*}
  \small
  \begin{center}
    \caption{COCO minival mAP and inference latency on Raspberry Pi 4 CPU (TFLite with XNNPACK backend, FP32) for YOLOv8s vs. a model identified from the NWOT-latency Pareto frontier (YOLO-FBNetV3-D-PAN). Mean and standard deviation of inference time over 5 runs (each one 100 iterations) are shown. The input image resolution used was 640x640, batch size $=1$ for latency measurements. Models were trained for 300 epochs, with batch size $=64$.} %All models trained for 300 epochs from random initialization, benchmarked on 640x640 input resolution.}
    \label{tab:timm_coco_full}
    \vspace*{2mm}
    \begin{tabularx}{\linewidth}{X|X|X} % <-- Alignments: 1st column left, 2nd middle and 3rd right, with vertical lines in between
      \hline
      {Model} & {COCO mAP$^{val}_{50-95}$} & {Raspberry Pi 4 ARM CPU latency, ms}\\
      % {} & {} & {ms} \\
      \hline
      {YOLOv8s} & {43.64\%} & {1476.09 (1.49)} \\
      \hline
      {YOLOv8s-HSwish} & {43.55\%} & {1381.62 (7.34)} \\
      \hline
      {YOLO-FBNetV3-D-PAN} & {44.63\%} & {1355.21 (9.93)} \\
      \hline
      {YOLO-FBNetV3-D-PAN-ReLU} & {44.07\%} & {1344.50 (8.06)} \\
      \hline
    \end{tabularx}
  \end{center}
\end{table*} \fi

\end{document}
