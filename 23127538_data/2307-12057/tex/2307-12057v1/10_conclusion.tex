\section{Limitations}

While we have proposed a highly flexible system, there are certain limitations to be noted. One such limitation pertains to the upgrading mechanism, which is currently reliant on human feedback. This dependence may affect the user experience. To address this, one potential solution is the integration of a sentiment analyzer that gauges the sentiment of the responses generated by the large language models (LLMs). There are instances when LLMs may fail to generate an appropriate response due to a mismatch between the local document and the user query. In such cases, LLMs might return a response like “The text provided is not related to the query”. By employing sentiment analysis, the system can discern this as a negative sentiment from the LLM and consequently escalate the query to a higher level of assistance automatically.
Another limitation arises due to the restrictive context window sizes of the LLMs, which are typically 4k or 8k tokens. This constraint can result in the system's failure to adequately address complex queries that necessitate a broader understanding of the context or the synthesis of information from multiple documents. One way to mitigate this limitation is by increasing the context window size. To this end, ALiBi, a linear-biased attention mechanism, could be integrated into the system to allow for an adjustable maximum token length at the interface level.
This proactive adaptation, through sentiment analysis and the incorporation of mechanisms like ALiBi, can potentially lead to a more fluid and effective interaction, enhancing both the system’s capabilities and the user experience. Furthermore, these adaptations emphasize the importance of the system's ability to recognize its limitations and make automatic adjustments in real-time to meet the demands of complex queries.

\section{Conclusion}

In this study, we have presented an intricate system that harnesses the capabilities of Large Language Models (LLMs) to solve complex queries, particularly in the context of retrieving and synthesizing information from scientific papers. Through a series of innovations, including sophisticated embedding methods, a novel key reference matching algorithm, and a policy system that employs varying levels of assistance, our system achieves remarkable flexibility and adaptability.
However, it is essential to recognize that with the sheer complexity and evolving nature of natural language processing, there is no one-size-fits-all solution. The limitations of the system, such as reliance on human feedback for upgrading assistance levels and the challenges posed by restrictive context window sizes of LLMs, were candidly acknowledged. We further explored potential improvements, such as the integration of sentiment analysis to autonomously escalate queries to higher levels of assistance, and employing ALiBi to permit an adjustable maximum token length.
Going forward, it is evident that as the field of natural language processing continues to evolve, systems like ours will need to continually adapt and innovate. Not only must they address the existing challenges but also stay ahead of the curve in incorporating emerging technologies. Such progression is vital in ensuring that these systems remain effective and relevant in catering to the ever-increasing demands for sophisticated information retrieval and synthesis.
In conclusion, this study represents a significant step towards building a dynamic, adaptable, and powerful system for handling complex queries within scientific literature. It serves as a basis for further research and development in optimizing LLMs for specialized tasks and, in the broader sense, contributes to the advancement of natural language processing applications in academia and beyond.

\label{sec:conclusion}


\section{Acknowledgement}
This paper serves as a technical report for the final project conducted within the scope of the University of Adelaide Course, COMP SCI 4817 - Applied Natural Language Processing Honours. We would like to express our gratitude to Dr. Lingqiao Liu, the course coordinator, for providing guidance and support throughout the duration of this course. Additionally, we extend our appreciation to my principal supervisor of the Honours program, Dr. YiFan Liu, for valuable discussions and insights that contributed to the development of this work.
