%%
%% This is file `sample-sigplan.tex',
%% generated with the docstrip utility.
%%
%% The original source files were:
%%
%% samples.dtx  (with options: `sigplan')
%% 
%% IMPORTANT NOTICE:
%% 
%% For the copyright see the source file.
%% 
%% Any modified versions of this file must be renamed
%% with new filenames distinct from sample-sigplan.tex.
%% 
%% For distribution of the original source see the terms
%% for copying and modification in the file samples.dtx.
%% 
%% This generated file may be distributed as long as the
%% original source files, as listed above, are part of the
%% same distribution. (The sources need not necessarily be
%% in the same archive or directory.)
%%
%% Commands for TeXCount
%TC:macro \cite [option:text,text]
%TC:macro \citep [option:text,text]
%TC:macro \citet [option:text,text]
%TC:envir table 0 1
%TC:envir table* 0 1
%TC:envir tabular [ignore] word
%TC:envir displaymath 0 word
%TC:envir math 0 word
%TC:envir comment 0 0
%%
%%
%% The first command in your LaTeX source must be the \documentclass command.
\documentclass[sigconf]{acmart}%review,anonymous]
\usepackage{multirow}
\usepackage{booktabs}
\usepackage{colortbl}
\usepackage{enumitem}
\usepackage{soul} 
\usepackage{color} 
\usepackage{xcolor}
\definecolor{bestresult}{RGB}{221,238,211}
\definecolor{Green}{RGB}{112,173,71}
\definecolor{Red}{RGB}{237,125,49}
\definecolor{Purple}{RGB}{112,48,160}
\definecolor{Blue}{RGB}{101,165,223}
\definecolor{DarkBlue}{RGB}{68,114,196}
\definecolor{Yellow}{RGB}{255,192,0}
\usepackage{listings}
\lstset{ %
language=python,                % choose the language of the code
basicstyle=\footnotesize,       % the size of the fonts that are used for the code
numbers=left,                   % where to put the line-numbers
numberstyle=\footnotesize,      % the size of the fonts that are used for the line-numbers
stepnumber=1,                   % the step between two line-numbers. If it is 1 each line will be numbered
numbersep=5pt,                  % how far the line-numbers are from the code
backgroundcolor=\color{white},  % choose the background color. You must add \usepackage{color}
showspaces=false,               % show spaces adding particular underscores
showstringspaces=false,         % underline spaces within strings
showtabs=false,                 % show tabs within strings adding particular underscores
frame=single,           % adds a frame around the code
tabsize=2,          % sets default tabsize to 2 spaces
captionpos=b,           % sets the caption-position to bottom
breaklines=true,        % sets automatic line breaking
breakatwhitespace=false,    % sets if automatic breaks should only happen at whitespace
escapeinside={\%*}{*)}          % if you want to add a comment within your code
}

%% NOTE that a single column version is required for 
%% submission and peer review. This can be done by changing
%% the \doucmentclass[...]{acmart} in this template to 
%% \documentclass[manuscript,screen,review]{acmart}
%% 
%% To ensure 100% compatibility, please check the white list of
%% approved LaTeX packages to be used with the Master Article Template at
%% https://www.acm.org/publications/taps/whitelist-of-latex-packages 
%% before creating your document. The white list page provides 
%% information on how to submit additional LaTeX packages for 
%% review and adoption.
%% Fonts used in the template cannot be substituted; margin 
%% adjustments are not allowed.
%%
%% \BibTeX command to typeset BibTeX logo in the docs
%%\AtBeginDocument{%
%%  \providecommand\BibTeX{{%
%%   \normalfont B\kern-0.5em{\scshape i\kern-0.25em b}\kern-0.8em\TeX}}}

%% Rights management information.  This information is sent to you
%% when you complete the rights form.  These commands have SAMPLE
%% values in them; it is your responsibility as an author to replace
%% the commands and values with those provided to you when you
%% complete the rights form.
 %\setcopyright{acmcopyright}
%% \copyrightyear{2018}
%% \acmYear{2018}
%% \acmDOI{XXXXXXX.XXXXXXX}

%% These commands are for a PROCEEDINGS abstract or paper.
%% \acmConference[Conference acronym 'XX]{Make sure to enter the correct
%%   conference title from your rights confirmation emai}{June 03--05,
 %%  2018}{Woodstock, NY}
%
%  Uncomment \acmBooktitle if th title of the proceedings is different
%  from ``Proceedings of ...''!
%
%\acmBooktitle{Woodstock '18: ACM Symposium on Neural Gaze Detection,
%  June 03--05, 2018, Woodstock, NY} 
%\acmPrice{15.00}
%\acmISBN{978-1-4503-XXXX-X/18/06}


%%
%% Submission ID.
%% Use this when submitting an article to a sponsored event. You'll
%% receive a unique submission ID from the organizers
%% of the event, and this ID should be used as the parameter to this command.
\acmSubmissionID{2459}

%%
%% For managing citations, it is recommended to use bibliography
%% files in BibTeX format.
%%
%% You can then either use BibTeX with the ACM-Reference-Format style,
%% or BibLaTeX with the acmnumeric or acmauthoryear sytles, that include
%% support for advanced citation of software artefact from the
%% biblatex-software package, also separately available on CTAN.
%%
%% Look at the sample-*-biblatex.tex files for templates showcasing
%% the biblatex styles.
%%

%%
%% The majority of ACM publications use numbered citations and
%% references.  The command \citestyle{authoryear} switches to the
%% "author year" style.
%%
%% If you are preparing content for an event
%% sponsored by ACM SIGGRAPH, you must use the "author year" style of
%% citations and references.
%% Uncommenting
%% the next command will enable that style.
%%\citestyle{acmauthoryear}

%%
%% end of the preamble, start of the body of the document source.
\setcopyright{none}
\settopmatter{printacmref=false} % Removes citation information below abstract
\renewcommand\footnotetextcopyrightpermission[1]{} % removes footnote with conference information in first column
\pagestyle{plain}

\newcommand{\qin}[1]{{\color{red}{\bf Qin}: {#1}}}

\begin{document}

%%
%% The "title" command has an optional parameter,
%% allowing the author to define a "short title" to be used in page headers.
%\title{Visual Captioning at Will: Describing Visual Contents with the Text Style Extracted from a Few Examples}
\title{Visual Captioning at Will: Describing Images and Videos \\Guided by a Few Stylized Sentences}
%%Arbitrary Stylized Visual Captioning: Few-Shot Text Style Extraction and Cross-Modal Alignment}

%%
%% The "author" command and its associated commands are used to define
%% the authors and their affiliations.
%% Of note is the shared affiliation of the first two authors, and the
%% "authornote" and "authornotemark" commands
%% used to denote shared contribution to the research.

\author{Dingyi Yang}
\authornote{This work was completed during the author's internship at Alibaba Group.}
\email{yangdingyi@ruc.edu.cn}
\affiliation{%
  \institution{School of Information,\\ Renmin University of China}
  \state{Beijing}
  \country{China}
  \postcode{43017-6221}
}

\author{Hongyu Chen}
\email{yinchen.chy@alibaba-inc.com}
\affiliation{%
  \institution{Alibaba Group}
  \state{Beijing}
  \country{China}
}

\author{Xinglin Hou}
\email{ xingli.hxl@alibaba-inc.com}
\affiliation{%
  \institution{Alibaba Group}
  \state{Beijing}
  \country{China}
}
\author{Tiezheng Ge}
\email{ tiezheng.gtz@alibaba-inc.com}
\affiliation{%
  \institution{Alibaba Group}
  \state{Beijing}
  \country{China}
}
\author{Yuning Jiang}
\email{ mengzhu.jyn@alibaba-inc.com}
\affiliation{%
  \institution{Alibaba Group}
  \state{Beijing}
  \country{China}
}


\author{Qin Jin}
\authornote{Corresponding Author.}
\email{qjin@ruc.edu.cn}
\affiliation{%
  \institution{School of Information, \\Renmin University of China}
  \state{Beijing}
  \country{China}
  \postcode{43017-6221}
}



%%
%% By default, the full list of authors will be used in the page
%% headers. Often, this list is too long, and will overlap
%% other information printed in the page headers. This command allows
%% the author to define a more concise list
%% of authors' names for this purpose.
%\renewcommand{\shortauthors}{Trovato and Tobin, et al.}

%%
%% The abstract is a short summary of the work to be presented in the
%% article.
\begin{abstract}
  Stylized visual captioning aims to generate image or video descriptions with specific styles, making them more attractive and emotionally appropriate. One major challenge with this task is the lack of paired stylized captions for visual content, so most existing works focus on unsupervised methods that do not rely on parallel datasets. However, these approaches still require training with sufficient examples that have style labels, and the generated captions are limited to predefined styles. To address these limitations, we explore the problem of Few-Shot Stylized Visual Captioning, which aims to generate captions in any desired style, using only a few examples as guidance during inference, without requiring further training. We propose a framework called FS-StyleCap for this task, which utilizes a conditional encoder-decoder language model and a visual projection module. Our two-step training scheme proceeds as follows: first, we train a style extractor to generate style representations on an unlabeled text-only corpus. Then, we freeze the extractor and enable our decoder to generate stylized descriptions based on the extracted style vector and projected visual content vectors. During inference, our model can generate desired stylized captions by deriving the style representation from user-supplied examples. Our automatic evaluation results for few-shot sentimental visual captioning outperform state-of-the-art approaches and are comparable to models that are fully trained on labeled style corpora. Human evaluations further confirm our model’s ability to handle multiple styles.
  
  %Stylized visual captioning aims to generate image or video descriptions with specific styles, making them more attractive and emotionally appropriate. One major challenge with this task is the lack of paired stylized captions for visual content. As a result, most existing works focus on unsupervised methods that do not rely on parallel datasets. However, these approaches still require training with numerous examples that have style labels, and the generated captions are limited to predefined styles. To overcome these limitations, we investigate the problem of \textbf{F}ew-\textbf{S}hot \textbf{Styl}iz\textbf{e}d Visual \textbf{Cap}tioning. This task aims to generate captions with any desired styles, using only a few examples as guidance, without requiring further training.
  %aims to generate captions with arbitrary style guided by a few examples, and not requiring further training.
  %Compared to previous methods, our approach is not limited to predefined styles, does not require a large amount of labeled style corpus, and does not need additional training processes to handle new styles. 
  %To tackle this problem, we propose a framework called \textbf{FS-StyleCap}, which utilizes a conditional encoder-decoder language model and a visual projection module. Our two-step training scheme proceeds as follows: first, we train a style extractor using an unlabeled text-only corpus to enable text style extraction. Then, we freeze the extractor and enable our decoder to generate stylized descriptions based on the extracted style vector and projected visual content vectors. During inference, our model generates any desired stylized captions by simply extracting text style from user-supplied examples. %As verified by human evaluations, our model is able to handle multiple styles. 
  %Compared to previous methods, our approach is not limited to predefined styles, and does not need additional training processes . 
  %With training on unlabeled corpus, our model is able to generate captions with multiple styles, guided by only 1-100 stylized examples. 
\end{abstract}

%%
%% The code below is generated by the tool at http://dl.acm.org/ccs.cfm.
%% Please copy and paste the code instead of the example below.
%%

\begin{CCSXML}
<ccs2012>
 <concept>
  <concept_id>10010520.10010553.10010562</concept_id>
  <concept_desc>Computer systems organization~Embedded systems</concept_desc>
  <concept_significance>500</concept_significance>
 </concept>
 <concept>
  <concept_id>10010520.10010575.10010755</concept_id>
  <concept_desc>Computer systems organization~Redundancy</concept_desc>
  <concept_significance>300</concept_significance>
 </concept>
 <concept>
  <concept_id>10010520.10010553.10010554</concept_id>
  <concept_desc>Computer systems organization~Robotics</concept_desc>
  <concept_significance>100</concept_significance>
 </concept>
 <concept>
  <concept_id>10003033.10003083.10003095</concept_id>
  <concept_desc>Networks~Network reliability</concept_desc>
  <concept_significance>100</concept_significance>
 </concept>
</ccs2012>
\end{CCSXML}

%% \ccsdesc[500]{Computer systems organization~Embedded systems}
%% \ccsdesc[300]{Computer systems organization~Redundancy}


\ccsdesc[500]{Computing methodologies~Natural language generation}
%% \ccsdesc[100]{Networks~Network reliability}

%%
%% Keywords. The author(s) should pick words that accurately describe
%% the work being presented. Separate the keywords with commas.
\keywords{Stylized visual captioning, Few-shot learning}

%% A "teaser" image appears between the author and affiliation
%% information and the body of the document, and typically spans the
%% page.
\iffalse
\begin{teaserfigure}
  % Figure removed
  \caption{Seattle Mariners at Spring Training, 2010.}
  \Description{Enjoying the baseball game from the third-base
  seats. Ichiro Suzuki preparing to bat.}
  \label{fig:teaser}
\end{teaserfigure}
\fi

%%
%% This command processes the author and affiliation and title
%% information and builds the first part of the formatted document.
\settopmatter{printfolios=true}
\maketitle





The problem of the presence or absence of phase transition is central in statistical mechanics. To prove the existence of phase transition, the standard idea is to define a notion of contour and use \textit{Peierls' argument} \cite{Peierls.1936}. In the usual Ising model \cite{Ising_25}, particles of the system interact only with their nearest-neighbors. On ferromagnetic long-range Ising models \cite{Anderson_Yuval_69}, there is interaction between each pair of spins in the lattice. The Hamiltonian of the model is given formally by
\begin{equation*}
    H(\sigma) = - \sum_{x,y\in \Z^d}J_{xy}\sigma_x\sigma_y,
\end{equation*}
where $J_{xy}=J|x-y|^{-\alpha}$, $J>0$, $\alpha > d$. It is well-known that the Peierls' argument in dimension 2 implies phase transition for Ising models with nearest-neighbors or long-range interactions when $d\geq 2$, using correlation inequalities. For the unidimensional lattice, it was known that short-range models do not present phase transition. In the long-range case, a different behavior was expected depending on the exponent $\alpha$ (see \cite{Kac_Thompson_69}), but the problem was challenging since contours were first created as multidimensional objects.

In dimension $d=1$, phase transition was proved first in 1969 by Dyson \cite{Dyson.69}, for $\alpha \in (1,2)$, by proving phase transition in an auxiliary model and then using correlation inequalities. In 1982, Fr{\"o}hlich and Spencer \cite{Frohlich.Spencer.82} introduced a notion of one-dimensional contours and then applied the Peierls' argument to show phase transition for the critical value $\alpha = 2$. These contours were inspired by the multiscale techniques previously introduced to study the Berezinskii-Kosterlitz-Thouless transition in two-dimensional continuous spin systems \cite{FS81}. Later, Cassandro, Ferrari, Merola and Presutti  \cite{Cassandro.05} extended the contour argument previously available for $\alpha=2$ to exponents $\alpha\in (3-\frac{\ln 3}{\ln 2}, 2)$, with the additional restriction that the nearest-neighbor interaction is strong, i.e.,  ${J(1)\gg 1}$; this restriction was removed for a subclass of interactions in \cite{Bissacot.Endo.18}. Further results were obtained using contour arguments, such as the decay of correlations, cluster expansions, phase transition with random interactions, etc; some references with these results are \cite{ Cassandro.Merola.Picco.17, Cassandro.Merola.Picco.Rozikov.14, Imbrie.82, Imbrie.Newman.88, Johansson.91}. 

In the multidimensional setting ($d\geq 2$), Ginibre, Grossmann, and Ruelle, in \cite{Ginibre.Grossmann.Ruelle.66}, proved the phase transition for $\alpha > d+1$, using an enhanced version of Peierls' argument and the usual contours. Park proposed a different notion of contour for long-range systems in \cite{Park.88.I, Park.88.II}, extending the Pirogov-Sinai theory available for short-range interactions assuming $\alpha > 3d+1$, although he can also consider Potts models with his methods. Some results in the literature suggest that truly long-range effects appear only when $d < \alpha \leq d+1$, see for instance, \cite{Biskup_Chayes_Kivelson_07}. Recently, Affonso, Bissacot, Endo and Handa \cite{Affonso.2021}, inspired by the ideas from Fr{\"o}hlich and Spencer in \cite{FS81, Frohlich.Spencer.82}, introduced a version of multiscale multidimensional contour and proved phase transition by a contour argument in the whole region $\alpha > d$. They can consider long-range Ising models with deterministic decaying fields, first introduced in the context of nearest-neighbor interactions in \cite{Bissacot_Cioletti_10}. For these models, the lack of analyticity of the free energy does not imply phase transition since these models have the same free energy as the models with zero field. It is expected that fields decaying slowly imply uniqueness. In this setting, a contour argument is useful for proofs of phase transitions as well for uniqueness, some papers with models with deterministic decaying fields are \cite{Aoun_Ott_Velenik_23, Bissacot_Cass_Cio_Pres_15, Bissacot.Endo.18, Cioletti_Vila_2016}.

The Random Field Ising model (RFIM) \cite{Imry.Ma.75} is the nearest-neighbor Ising model with an additional external field acting on each site $(h_x)_{x\in\Z^d}$ that is a family of i.i.d. Gaussian random variable with mean 0 and variance 1. Formally, the Hamiltonian of the model is given by
\begin{equation*}
    H(\sigma) = - \sum_{\substack{x,y\in \Z^d \\|x-y|=1}}J\sigma_x\sigma_y  - \varepsilon\sum_{x\in\Z^d}h_x\sigma_x,
\end{equation*}
where $J>0$, $\varepsilon>0$, $\alpha > d$ and $d \geq 1$. A detailed account of the history of the phase transition problem for this model, as well as detailed proofs, was given in \cite{Bovier.06}. Here we present a brief overview.

During the 1980s, the question of the specific dimension where phase transition for the RFIM should happen attracted much attention and was a topic of heated debate. Two convincing arguments were dividing the physics community. One of them, due to Imry and Ma \cite{Imry.Ma.75}, was a non-rigorous application of the Peierls' argument together with the use of the isoperimetric inequality. The key idea of Peierls' argument is to define a notion of contour and calculate the energy cost of "erasing" each contour, i.e., the energy cost of flipping all spins inside the contour. When there is no external field, that energy necessary to flip the spins in a region $A\subset \Z^d$ is of the order of the boundary $|\partial A|$. When we add an external field, we get an extra cost depending on this field. Imry and Ma argued that this cost should be approximately $\sqrt{|A|}$, which is smaller than $|\partial A|$ for all regions only when $d\geq 3$, so this should be the region where phase transition occurs. The other argument, due to Parisi and Sourlas \cite{Parisi.Sourlas.79}, based on dimensional reduction, predicted that the $d$-dimensional RFIM would behave like the $d-2$-dimensional nearest-neighbor Ising model, therefore presenting phase transition only when $d\geq 4$. 

The question was settled by two celebrated papers showing that Imry and Ma's prediction was correct. First, in 1988, Bricmont and Kupiainen \cite{Bricmont.Kupiainen.88} showed that there is phase transition almost surely in $d\geq3$, for low temperatures and variance $\varepsilon$ small enough. Their proof uses a rigorous renormalization group analysis for the short-range case and it is considered involved. Still, they claimed that the result works for any model with a suitable contour representation and centered sub-gaussian external field. Later on, Aizenman and Wehr \cite{Aizenman.Wehr.90} proved uniqueness for $d\leq 2$. For detailed proofs of these results, we refer the reader to \cite{Bovier.06} (see also \cite{Berretti.85, Camia.18, Frohlich.Imbre.84,  Klein.Masooman.97} for more uniqueness results). 

Recently, Ding and Zhuang, see \cite{Ding2021}, provided a simpler proof of the phase transition, not using RGM. And in  \cite{Ding.Liu.Xia.22}, Ding, Liu and Xia proved that if $\beta_c(d)$ is the critical inverse of the temperature of the Ising model with no field, for all $\beta>\beta_c(d)$ there exists a critical value $\varepsilon_0(d, \beta)$ such that the RFIM with $\varepsilon \leq \varepsilon_0$ presents phase transition. 

In the present paper, we are considering a long-range Ising model with a random field, whose Hamiltonian is given formally by
\begin{equation*}
    H(\sigma) = - \sum_{x,y\in \Z^d}J_{xy}\sigma_x\sigma_y - \varepsilon\sum_{x\in\Z^d}h_x\sigma_x,
\end{equation*}
where $J_{xy}=J|x-y|^{-\alpha}$, $J, \varepsilon>0$, $\alpha > d$ and $h_x\in\mathbb{R}$, $d\geq 3$.
Until now, the only known result in the long-range setting is for the one-dimensional long-range Ising model with a random field, by Cassandro, Orlandi, and Picco \cite{Cassandro.Picco.09}. They used the contours of \cite{Cassandro.05} to show the phase transition for the model when $\alpha\in (3-\frac{\ln 3}{\ln 2}, \frac{3}{2})$, under the assumption $J(1) \gg 1$. We stress that, as remarked by Aizenman, Greenblatt, and Lebowitz \cite{Aizenman_Greenblatt_Lebowitz_2012}, although their argument does not work for the whole region for the exponent $\alpha$, the phase transition holds for values close to the critical value $\alpha=3/2$, since by the Aizenman-Wehr theorem we know that there is uniqueness for $\alpha>3/2$.

The argument from Ding and Zhuang in \cite{Ding2021}, for $d\geq3$, involves controlling the probability of a bad event, which is closely related to controlling the quantity $$\sup_{\substack{0\in A\subset\Z^d \\ A \text{ connected }}}\frac{\sum_{x\in A}h_x}{|\partial A|},$$ known as the greedy animal lattice normalized by the boundary. The greedy animal lattice normalized by the size, instead of the boundary, was extensively studied for general distributions of $(h_x)_{x\in\Z^d}$, see \cite{Cox_Gandolfi_Griffin_Kesten_93, Gandolfi_Kesten_94, Hammond_06, Martin_02}. When we normalize by the boundary, an argument by Fisher, Fr\"{o}hlich and Spencer \cite{FFS84} shows that the expected value of the greedy animal lattice is constant. In dimension $d=2$, the expected value is not finite, see \cite{Ding.Wirth.20}. The supremum is taken over connected regions containing the origin since the interiors of the usual Peierls contours are of this form.


For the long-range model, the interior of contours is not necessarily connected. In fact, long-range contours may have considerably large diameters with respect to their size, so their interiors can be very sparse. To avoid this, we define contours, strongly inspired by the $(M,a,r)$-partition in \cite{Affonso.2021}, using a multiscaled procedure that assures that the contours have no cluster with small density.  With them, we generalize the arguments by Fisher-Fr\"{o}hlich-Spencer \cite{FFS84}, and prove that the expected value of the greedy animal lattice is constant, even considering regions not necessarily connected in the supremum. Then, we prove the phase transition for $d\geq 3$. The main result of this paper is the following.
\begin{theorem*}Given $d\geq 3$, $\alpha>d$, there exists $\beta_c\coloneqq\beta(d, \alpha)$ and $\varepsilon_c\coloneqq\varepsilon(d, \alpha)$ such that, for $\beta >\beta_c$ and $\varepsilon\leq \varepsilon_c$, the extremal Gibbs measures $\mu_{\beta, \varepsilon}^+$ and $\mu_{\beta, \varepsilon}^-$ are distinct, that is, $\mu_{\beta, \varepsilon}^+ \neq \mu_{\beta, \varepsilon}^-$ $\mathbb{P}$-almost surely. Therefore the long-range random field Ising model presents phase transition.
\end{theorem*}

This paper is divided as follows. In Section 2, we define the model and the contours, and suitable generalizations to the constructions in \cite{Affonso.2021} are introduced.  In Section 3, we define two bad events of the external field and prove that they occur with a small probability.  In Section 4, we present the proof of the phase transition.
\section{Related Works}
% Figure environment removed

%\subsection{Stylized Visual Captioning}
\noindent \textbf{Stylized Visual Captioning.} 
Stylized Visual Captioning \citep{mathews2016senticap,bin2021multi} aims to generate descriptions of images or videos that are both visually relevant and stylistically accurate. Previous works in this area can be categorized into supervised and unsupervised methods. Supervised approaches \cite{mathews2016senticap,you2018image,shuster2019engaging,li2021similar,bin2021multi,li2022taking} involve constructing paired stylized captions for visual content. However, building large-scale paired datasets can be laborious, and these methods are limited to certain styles in parallel datasets.
%\citet{bin2021multi} propose a model that mines various perspectives within a video and generate a description for each perspective, including the visual aspect, language style, and perception pattern. \citet{li2022taking}  propose a large-scale emotion and logic driven multilingual dataset for video paragraph captioning task. 

Therefore, most existing works focus on unsupervised methods. \citet{mathews2018semstyle} propose to generate semantic terms for visual content, and transform them into descriptions with various styles. Several approaches, including StyleNet \cite{gan2017stylenet}, Factual \cite{chen2018factual}, DLN \cite{chen2019unsupervised}, and MSCap \cite{guo2019mscap}, suggest different architectures for learning style-dependent matrices that capture style-related information.
%\citet{guo2019mscap} propose an adversarial learning framework with a style dependent generator, a caption discriminator and a style classifier to improve the overall performance. 
MemCap \cite{zhao2020memcap} and Senti-Trans \cite{wu2023sentimental} incorporate style knowledge to generate stylized descriptions.
%\citet{zhao2020memcap} introduce a memory module which captures style knowledge within content-related and style-related parts. \citet{wu2023sentimental} integrate both content and sentiment information from multiple modalities and incorporates prior sentimental knowledge to generate sentimental video descriptions. 
\citet{tan2022detach} propose to detach several text style representations for specific styles, and then attach them to image content to generate stylized captions. Recently, some works start to leverage the power of large language models. \citet{zeng2023conzic} propose a sampling-based methods to generate captions that incorporates controllable signals like predefined sentiments. \citet{nukrai2022capdec} and \citet{gu2022can} utilize the CLIP embedding space \cite{radford2021clip} and additional stylized training data to perform stylized captioning tasks, achieving impressive results. Although these unsupervised methods have shown effective results, they still require sufficient samples with desired style labels, and additional training is necessary to accommodate new styles. These limitations motivate us to explore the problem of few-shot stylized visual captioning. 

%\subsection{Few-shot Text Style Transfer}
\noindent \textbf{Few-shot Text Style Transfer.} 
%Most text style transfer researches fall into supervised or unsupervised approaches, which will be limited by pre-specific style labels. 
Recently, some works have started to explore the problem of few-shot text style transfer. This task does not require style labels during training, but instead uses a small number of labeled stylized examples as guidance during inference.
\citet{xu2020variational} propose to train a variational auto-encoder on unlabeled text, learning a text representation that features a controllable portion, which is restricted to lie on a k-dimensional simplex. To perform transfer, the controllable dimensions are manipulated with a basis vector that corresponds most strongly to the target style. 
Similarly, \citet{riley2020textsettr} fine-tune the T5 encoder \cite{raffel2020exploring} to extract a style vector from input text. This vector can then be employed to modify the latent representation in order to generate text with the target style. 
\citet{reif2021recipe}, \citet{luo2023prompt} and \citet{suzgun2022prompt} employ prompting-based methods. They utilize large language models (LLMs) and design specific prompts to rewrite texts in various styles. However, these methods usually require paired prompts that represent the relations and differences between different styles, which makes the process less flexible. Additionally, they face the problem of hallucinations, making them less reliable \cite{reif2021recipe} when compared to trained methods. To address these concerns, we utilize pre-trained models and fine-tune them to tackle our specific task, leveraging the benefits of LLMs and avoiding the drawbacks of prompt-based techniques at the same time.

\section*{Methods}
%Why AutoPet dataset
\label{S: Methods}
\subsection*{Data Acquisition}
%Why CT

Two of the most common volumetric modalities are CT and MRI. 
While MRI often focuses on soft tissue analysis and brain imaging, CT is a common choice in the clinical routine due to its acquisition time and broad field of use. As we aim to generate models to segment any anatomy utilizing various sources, we start by selecting a dataset that acts as a solid basis for full-body label aggregation.
% Why AuotPET
The recently published AutoPET dataset~\cite{gatidis2022whole} is a PET-CT dataset that perfectly fits our requirements since nuclear medicine often requires full-body CT scans to track therapy. In addition to the full-body CTs, this dataset might enable future multi-modal segmentation tasks~\cite{xue2021multi, marinov2023mirror} due to the separate PET domain and lesion annotations. Future multimodal tasks could make use of the provided anatomical structures, which, however, is not the focus of this work.

%Which volumes have been selected? 
%Why did we choose a subset?
We select a subset of $566$ CTs of the AutoPET dataset. The selection criterion is based on similar slice thickness in the axial dimension leading to a homogeneous dataset. Furthermore, we make sure that the images show important regions of interest. Our region of interest starts from the head and ends slightly below the hip which includes all thoracic and abdominal organs. The chosen subset consists of the CTs which contain between $336$ and $400$ slices in the original dicom files. We exclude CTs with fewer slices as these tend to show an insufficient subpart of the body contradicting the desired full-body dataset. CT images containing more than 400 slices tend to include more irrelevant content. %background and anatomical structures outside of our region of interest such as the legs. 
This leaves us with a homogeneous dataset of size $566$. Our final DAP Atlas dataset does deviate from this selection, as we filter out implausible predictions, in a final post-processing step leaving us with $533$ CT images. The filter criteria will be described in detail later. In the following, we will refer to the DAP Atlas dataset as the dataset containing $533$ images. 

%Description of the subset
Our DAP Atlas is similar to AutoPET regarding the age and gender distributions as well as pathological findings. We show a descriptive analysis of the dataset regarding the aforementioned dimensions in Fig.~\ref{fig:descriptive_statistics}. 

% Figure environment removed

%Maybe we need a table showing average 

\subsection*{Knowledge Acquisition}


 % Figure environment removed
% % Figure environment removed


%Where do the individual labels come from
The dataset aggregates multiple sources of anatomical segmentation knowledge which we differentiate into public knowledge which is present in the form of publicly available datasets and private knowledge which are private datasets available to us. Besides the segmentation knowledge, we leverage rule-based knowledge which is derived from anatomical textbook knowledge and represents what could be described as the common sense of a radiologist. These rules contain for instance, which anatomical structures are possible in which part of the human body. We display the DAP Atlas construction workflow in Fig.~\ref{fig:merging} and discuss the details in the following.

A large amount of labels of the dataset is derived from public segmentation knowledge which is present in fragmented form through publicly available datasets. These contain annotations of organs of interest on CT images showing parts of the human body. We extract this knowledge by training neural networks on these public datasets, which learn to predict the labels of the respective datasets. Typically, training a neural network is a data and task-specific problem and requires finetuning a large set of hyperparameters which is impracticable for our desired applications as we intend to train a vast number of models on various heterogeneous datasets. To overcome this problem, we use the nnU-Net~\cite{isensee2021nnu}, a framework that automatically configures a U-Net and adapts the training procedure to the data at hand. This Auto-ML framework provides segmentation results surpassing several more complex works without any of their additional engineering overhead. We employ standard nnU-Nets and train them on publicly available datasets. The used publicly available datasets are shown along with their obtained label category in Fig.~\ref{Tab:Label_Usage}. After training, these networks are used to carry over the extracted knowledge by predicting the learned labels into our full-body selected DAP Atlas CT images. We describe the used datasets and the merging procedure in the following. %and show more detailed information about the used datasets and the source of each label in the supplementary information. 

%TODO: Include pathological labels?

% \begin{table}[]
% \centering
% \begin{tabular}{|l||r|r|r|r|r|r|r|r|r|r|r|r|r|r|}
% \hline
%                & \multicolumn{1}{l|}{P} & \multicolumn{1}{l|}{TS} & \multicolumn{1}{l|}{V} & \multicolumn{1}{l|}{RS} & \multicolumn{1}{l|}{Hip} & \multicolumn{1}{l|}{A} & \multicolumn{1}{l|}{C} & \multicolumn{1}{l|}{ATM} & \multicolumn{1}{l|}{P} & \multicolumn{1}{l|}{ST} & \multicolumn{1}{l|}{Abd} & \multicolumn{1}{l|}{Rule} & \multicolumn{1}{l|}{BCA} & \multicolumn{1}{l|}{HN} \\ \hline \hline
% Musculature    & 0                      & 10                      & 0                      & 0                       & 0                        & 0                      & 0                      & 0                        & 0                      & 0                       & 0                        & 2                         & 1                        & 0                       \\ \hline
% Tissues        & 2                      & 0                       & 0                      & 0                       & 0                        & 0                      & 0                      & 0                        & 0                      & 0                       & 0                        & 0                         & 4                        & 0                       \\ \hline
% Digestive      & 9                      & 9                       & 0                      & 0                       & 0                        & 0                      & 0                      & 0                        & 0                      & 1                       & 1                        & 0                         & 0                        & 0                       \\ \hline
% Urinary        & 3                      & 3                       & 0                      & 0                       & 0                        & 0                      & 0                      & 0                        & 0                      & 0                       & 0                        & 0                         & 0                        & 0                       \\ \hline
% Endocrine      & 4                      & 2                       & 0                      & 0                       & 0                        & 2                      & 0                      & 0                        & 0                      & 0                       & 0                        & 0                         & 2                        & 0                       \\ \hline
% Reproductive   & 2                      & 0                       & 0                      & 0                       & 0                        & 0                      & 1                      & 0                        & 0                      & 0                       & 0                        & 0                         & 0                        & 0                       \\ \hline
% Nervous        & 1                      & 1                       & 0                      & 0                       & 0                        & 0                      & 0                      & 0                        & 0                      & 0                       & 0                        & 0                         & 0                        & 0                       \\ \hline
% Immune         & 2                      & 1                       & 0                      & 0                       & 0                        & 1                      & 0                      & 0                        & 0                      & 0                       & 0                        & 0                         & 0                        & 0                       \\ \hline
% Vertebras      & 0                      & 24                      & 24                     & 0                       & 0                        & 0                      & 0                      & 0                        & 0                      & 0                       & 0                        & 0                         & 0                        & 0                       \\ \hline
% Ribs           & 0                      & 24                      & 0                      & 24                      & 0                        & 0                      & 0                      & 0                        & 0                      & 0                       & 0                        & 1                         & 1                        & 0                       \\ \hline
% Bones          & 2                      & 10                      & 0                      & 0                       & 1                        & 0                      & 0                      & 0                        & 0                      & 0                       & 0                        & 1                         & 0                        & 1                       \\ \hline
% Cardiovascular & 1                      & 11                      & 0                      & 0                       & 0                        & 2                      & 0                      & 0                        & 0                      & 2                       & 1                        & 0                         & 1                        & 11                      \\ \hline
% Vessels        & 0                      & 2                       & 0                      & 0                       & 0                        & 0                      & 0                      & 0                        & 0                      & 0                       & 0                        & 0                         & 0                        & 0                       \\ \hline
% Respiratory    & 0                      & 6                       & 0                      & 0                       & 0                        & 0                      & 0                      & 1                        & 1                      & 1                       & 0                        & 1                         & 0                        & 0                       \\ \hline
% Visual         & 0                      & 0                       & 0                      & 0                       & 0                        & 0                      & 0                      & 0                        & 0                      & 0                       & 0                        & 2                         & 0                        & 0                       \\ \hline 
%                & \multicolumn{1}{l|}{}  & \multicolumn{1}{l|}{}   & \multicolumn{1}{l|}{}  & \multicolumn{1}{l|}{}   & \multicolumn{1}{l|}{}    & \multicolumn{1}{l|}{}  & \multicolumn{1}{l|}{}  & \multicolumn{1}{l|}{}    & \multicolumn{1}{l|}{}  & \multicolumn{1}{l|}{}   & \multicolumn{1}{l|}{}    & \multicolumn{1}{l|}{}     & \multicolumn{1}{l|}{}    & \multicolumn{1}{l|}{}   \\ \hline
% Used Labels    & 26                     & 103                     & 24                     & 24                      & 1                        & 5                      & 1                      & 1                        & 1                      & 4                       & 2                        & 7                         & 9                        & 12                      \\ \hline
% \end{tabular}
% \caption{Overview of the different source datasets from which the DAP Atlas dataset is derived. Within this table we cluster the individual labels according to their anatomical system. We show a full table with individual labels in the supplementary material. On the bottom row we show the number of labels in the DAP Atlas dataset which are influenced by the respective source dataset. The sum of columns cannot be calculated, because predictions of the same label are fused. The abbreviations for the datasets are as follows: \textbf{P}:Pediatric~\cite{jordan2022pediatric}, \textbf{TS}: Total Segmentator~\cite{wasserthal2022totalsegmentator}, \textbf{V}: Verse~\cite{sekuboyina2021verse}, \textbf{RS}:RibSeg~\cite{yang2021ribseg}, \textbf{Hip}: Pelvis CT~\cite{liu2021deep}, \textbf{A}:Amos~\cite{ji2022amos}, \textbf{C}:MAL Cervix~\cite{landman2015miccai}, \textbf{ATM}: ATM Airway Tree Modelling~\cite{zhang2023multi}, \textbf{PARSE}: Pulmonary Artery Segmentation Challenge~\cite{kuanquan_wang_2022_6361906}, \textbf{ST}:SegThor~\cite{lambert2020segthor}, \textbf{Abd}: CT50 Abdomen~\cite{ma2021abdomenct}, \textbf{Rule}: Rule based derived label, \textbf{BCA}: Body composition analysis model~\cite{koitka2021fully}, \textbf{HN}: Private Head-neck dataset. \textbf{Self} refers to the self-annotated labels.}
% \label{Tab:Label_Usage}
% \end{table}

% Figure environment removed

%Brief description of the datasets
\begin{itemize}
    \item \textbf{Pediatric~\cite{jordan2022pediatric}:} This dataset consists of $359$ chest-abdomen-pelvis and abdomen-pelvis CT images of patients between the age of $5$ and $16$ years. It provides $29$ anatomical structures annotated by experts. Patients were selected based on random clinical indications from the university clinic of Children's Wisconsin.
    \item \textbf{Total Segmentator~\cite{wasserthal2022totalsegmentator}:} The TotalSegmentator dataset is large and diverse with $1024$ CT images of different body parts with labels for $104$ anatomical structures. The dataset was collected by randomly sampling from the PACs systems of multiple sites. Its annotation is based on an interactive semi-automatic approach. Here, models are first trained on a few manual annotations. These models infer predictions on unlabeled scans which are lastly refined by an expert. This cycle repeats with an ever-increasing number of training images.
    \item \textbf{SegThor~\cite{lambert2020segthor}:} A dataset consisting of $60$ thoracic CTs collected at the Henri Becquerel Center. The patients were selected based on lung cancer or Hodkin's lymphoma diagnosis. The CTs contain annotations for four organs at risk whose tissues must remain intact during radiation therapy. The annotations of the dataset are provided by an experienced radiotherapist. 
    \item \textbf{CT50Abdomen~\cite{ma2021abdomenct}:} The dataset is part of the CT1k Abdomen datasets extension in which the authors provide 50 abdominal CT images with previously less annotated structures such as the adrenal glands. Annotations are provided by multiple junior annotators and checked by senior radiologists.
    \item \textbf{MAL Cervix~\cite{landman2015miccai}:} This dataset is part of the Beyond the Cranial Vault challenge. It consists of $30$ training and $20$ testing abdominal CT images acquired via a full bladder drinking protocol and annotated by a trained radiation oncologist. It focuses on the digestive and reproductive systems of female cervical cancer patients.
    \item \textbf{Amos~\cite{ji2022amos}:} A diverse dataset with $500$ CT images collected from different scanners and sites covering 15 abdominal organ categories. The selection of patients relates to abdominal tumors or abnormalities examinations. Annotations rely on a combination of junior and senior radiologist labor.
    \item \textbf{RibSeg~\cite{yang2021ribseg}:} The RibSeg dataset consists of $490$ CT Scans taken from publicly available RibFrac~\cite{jin2020deep} dataset. The authors use a semi-automatic morphology-based segmentation approach based on thresholding, point cloud segmentation, and morphological operations. They check the proposed segmentations by hand and refine them if necessary. 
    \item \textbf{Verse ~\cite{sekuboyina2021verse}:} A large dataset for vertebra segmentation. It consists of two subsets and has a total of $374$ CT scans of $355$ patients from multiple detectors and sites with voxel-wise annotations for individual vertebras. Segmentations have been performed semi-automatically with initial proposals being generated by an in-house pipeline. The proposals are refined by a team of trained medical students and experts and finally approved by a radiologist with more than $30$ years of experience. 
    \item \textbf{ATM~\cite{zhang2023multi}:} This dataset establishes a benchmark for Airway Tree Modelling by providing $500$ chest CT scans from different sites and includes scans of healthy patients, patients with pulmonary diseases, and even noisy COVID-19 CTs. Annotations of the pulmonary airways were performed by a team of three experts with each radiologist having more than five years of experience.
    \item \textbf{PARSE~\cite{kuanquan_wang_2022_6361906}:} The PARSE dataset is part of the Pulmonary Artery Segmentation Challenge and contains a total of $203$ CT images from $203$ patients which have been diagnosed with pulmonary nodular diseases. The CTs were generated using devices from two different manufacturers, with data collected from four distinct sites. Each of the images has been annotated by five experts with each expert having at least five years of experience in the field. 
    \item \textbf{Pelvic CT~\cite{liu2021deep}:} A large-scale dataset that focuses on the segmentation of pelvic bone structures such as hip bones or sacrum. It consists of $1184$ CT images collected from different source datasets combining images from multiple sites, scanners, and even metal artifacts. The labeling was conducted by a team of junior and senior radiologists.
    
\end{itemize}

%Include Labels
% \begin{table*}[t]
    \centering
    \small
    \begin{tabular}{lccccccc}
    \toprule
    Source & Series & Labels & Label Domain & Spacing & Slices & IoU \\
    \midrule
         Pediatric~\cite{jordan2022pediatric} & $ 359 $ & 29 & Full Body & [$  0.52 \pm 0.11 , 0.52 \pm 0.11 , 1.76 \pm 0.52  $]  & $ 306.6 \pm 245.4 $ & \todo{-}\\
         
         Total Segmentator~\cite{wasserthal2022totalsegmentator} & $ 1024 $ & 104 & Full Body & [$  1.5 \pm 0.0 , 1.5 \pm 0.0 , 1.5 \pm 0.0  $]  & $ 259.0 \pm 130.3 $ & \todo{-} \\

         SegTHOR~\cite{lambert2020segthor} &  $ 60 $ & 4 & Thoracic & [$  1.00 \pm 0.09 , 1.00 \pm 0.09 , 2.39 \pm 0.23  $]  & $ 184.7 \pm 30.35 $ & \todo{-} \\
         
         CT50Abdomen~\cite{ma2021abdomenct} & $ 50 $ & 13 & Abdominal & [$  0.81 \pm 0.07 , 0.81 \pm 0.07 , 2.63 \pm 0.52  $]  & $ 95.88 \pm 9.000 $& \todo{-}\\

         MAL Cervix~\cite{landman2015miccai} & $ 50 $ & 4 & Reproductive & \todo{-}  &  \todo{-} & \todo{-}\\

         Amos~\cite{ji2022amos} & $ 500 $ & 5 & Abdominal & \todo{-}  & \todo{-} & \todo{-}\\
         
         RibSeg~\cite{yang2021ribseg} & $ 490 $ & 1 & Ribs & [$  0.74 \pm 0.07 , 0.74 \pm 0.07 , 1.13 \pm 0.14  $]  & $ 359.5 \pm 59.06 $ & \todo{-}\\

         Verse~\cite{sekuboyina2021verse} & $374$ & 28 & Spine & [$  0.79 \pm 0.23 , 0.79 \pm 0.23 , 1.29 \pm 0.65  $]  & $ 444.8 \pm 348.6 $ & \todo{-}\\
         
         ATM~\cite{zhang2023multi} & $ 500 $ & 1 & Respiratory & \todo{-}  & \todo{-} & \todo{-} \\
         
         PARSE~\cite{kuanquan_wang_2022_6361906} & $ 203 $ & 1 & Vessels & [$  0.67 \pm 0.07 , 0.67 \pm 0.07 , 0.99 \pm 0.01  $]  & $ 301.4 \pm 31.22 $ & \todo{-} \\
         
         Pelvic CT~\cite{liu2021deep} & $ 1184 $ & \todo{-} & Bones & \todo{-}  & \todo{-} & \todo{-} \\
         
        
        
        
         \bottomrule
    \end{tabular}
    \caption{Comparison of considered CT datasets for datasets used for label aggregation regarding size, number of labels, label domain, volume spacing, number of slices and segmentation performance of a nnUNet~\cite{isensee2021nnu} in 5-fold cross validation.}
    \label{tab:ch5_datasets_overview}
\end{table*}


%Third source of knowledge: Own annotations, BCA and Alex's Labels
Besides the previously described dataset, we leverage non-publicly available datasets and models. One of the models is the body composition analysis model~\cite{koitka2021fully} which differentiates between different types of tissues. From this model, we obtain labels such as \textit{fat} or the general class \textit{muscles}. In total, we extract $9$ labels from the body composition model source. 
A second private source dataset consisting of $104$ diverse head and neck contrast CT images from four different source cohorts~\cite{giske2011local, stoiber2017analyzing, bejarano2019longitudinal, bejarano2018head, clark2013cancer}.
%The total of $104$ images have been annotated by medical students with a focus on anatomical structures required for the. 
This dataset focuses on the diagnosis and treatment of oropharyngeal or hypopharyngeal head and neck cancer and has been annotated by medical students. It provides fine-grained classes for head neck vessels and bone structures. We train a standard nnU-Net on $86$ images to extract the dataset knowledge and add $12$ unique, previously unavailable labels from this dataset, mostly vessels in the head-neck region. 

%Second source of truth: Rules and Remapping
After obtaining the labels from the different nnU-Net predictions, we use anatomically derived rules to refine the current predictions and generate $7$ additional labels. An intuitive example for a new label that can be derived from the combination of obtained labels and medical common sense is the skull. It can be derived from a thresholding procedure obtained by the bone window present in CT images. Bones typically lead to CT values between 350 and 3000 Hounsfield Units (HU) which serve as the described thresholds. The obtained set of voxels can be restricted to the area above the C5 vertebra which previously was obtained. Finally, we remove already predicted vertebras from the thresholded voxels which leaves us with an accurate mask for the skull. We furthermore exploit the behavior of the neural network predictions which have only been trained on parts of the anatomy and typically confuse structures that look similar in the CT images. Common systematic errors are to predict gonads as the eyeballs or colon as the nasal cavity. We exploit these systematic mistakes and remap the produced labels according to the location within the human body. By employing these simple rules we add $7$ additional labels. 


% Figure environment removed


\subsection*{Knowledge Aggregation}
In order to aggregate the predictions of the individual models, we define a common labeling scheme %which is derived from established medical nomenclature standards.  
to which we map the obtained masks. Since some of these labels present multiple versions of the same anatomical structure, such as the class \textit{aorta} which is present in Total Segmentator, Amos, and SegThor it is necessary to combine these predictions. 
%Might need to delete this sentence
%We check the label quality of the models for the individual predictions on randomly sampled volumes and exclude predictions of suboptimal quality. 
Unless stated otherwise, we merge the predictions of models trained on the different source datasets into a single mask which is the union of all individual masks. 
This procedure is simple and stable. It also helps in aggregating masks of the same anatomical structures which are only predicted within certain regions of the human body on which the respective models have been trained. An example of this behavior can be found in the mask for the class \textit{aorta} predicted by the SegThor~\cite{lambert2020segthor} model. While the aorta spans outside the thorax, this model only predicts it within the thorax region. Only by merging the mask for \textit{aorta} of this model with additional masks from other models leaves us with a full mask for the aorta spanning over the entire anatomy. Thus merging these labels combines the knowledge present in different parts of the human body into a single, unified anatomy which is the goal of this work. 

When integrating the different anatomical structures into the Atlas labeling scheme, we aggregate them according to their anatomical hierarchical level from course to fine starting from general tissues such as \textit{muscles} or \textit{fat}. On top, we gradually add the different organs and finally fine-grained vessel structures such as \textit{Pulmonary Arteries}. During the aggregation process, we employ basic anatomical knowledge to improve individual predictions on the fly. One of these operations is that we split the predicted voxels into left and right clusters for paired organs such as the hip bones, kidneys or adrenal glands and resolve conflicts. Furthermore, we restrict predictions based on previous labels or eliminate non-largest connected components if it is deemed appropriate. %Furthermore, we employ basic morphological operations such morphological growing exclude predictions of structures which are likely to be confused such are adjacent bone structures. Examples for this are hip bones and the two femurs

%Volle List der Labens angeben

\subsubsection*{From Aggregated Predictions to a Unified Dataset}
\label{SS: Label Aggregation}
After integrating the labels into the common DAP Atlas CT volumes, it is an integrated dataset, but the different masks are still predictions of models which were trained on heterogeneous source datasets and thus generate heterogeneous masks. To unite these different, integrated masks into a single seamless dataset, we perform one iteration of self-training. The benefits of this procedure are four-fold: As previously mentioned, we bring the labels which originate from datasets of different resolutions into the common Atlas resolution leading to a truly seamless integration. A second reason to perform self-training is to eliminate non-systematic random noise. The network receives consistent feedback from consistent predictions, while noisy predictions are non-systematic. This observation is a well-known fact in image classification~\cite{liu2020early} which states that before the memorization of training data, networks tend to ignore noisy predictions and focus on consistent feedback. We make sure to not overfit the network on the dataset by closely monitoring training and validation losses. A third reason, to perform self-training is to distill the fragmented knowledge into a single model capable to predict the entire anatomy, this massively decreases the necessary time to predict the anatomy, since it reduces the inference time from $n$ expert models to a single model. Finally, self-training hampers the exact reconstruction of private data from expert models which were directly trained on private source datasets. 

%Beschreibung von V1
We generate the first version of the DAP Atlas dataset by applying the obtained unified anatomical model on the selected Atlas target volumes. While the overall label quality is good, we notice certain patterns which were repeatedly done wrong and with which the networks seemed to struggle. These systematic exceptions are the confusion of voxels that belong to paired structures such as the left and right kidney or adjacent vertebrae. Further, we observe implausible predictions of structures within body regions that are not possible, e.g. colon being predicted outside the abdomen. Finally, we observe structures belonging to the reproductive system to be predicted for the wrong sex. 
These errors are relatively easy to correct by once again applying anatomical rules. To address these, we propose Algorithm~\ref{Alg: Post-Processing}. 
We furthermore use two sets of rules to filter out implausible prediction: During Algorithm ~\ref{Alg: Rib counting}: We exclude predictions leading to different orderings induced by median points and minimum points of the ribs. Additionally, we examine the normal vector of the hyperplane during Algorithm~\ref{Alg: Left-Right-Splitting} and exclude predictions leading to hyperplanes that deviate too much from the axial directions. This reduces the number of images in the Atlas dataset from $566$ to $533$ CTs.

After applying Algorithm~\ref{Alg: Post-Processing} to the raw labels, we receive the final version of the dataset, which is rated as very impressive by a consulted radiologist. We describe the extensive validation procedure of the dataset in Section \textit{Technical Validation}. 

While the dataset is convincing, we acknowledge that the performance of the developed anatomical model is dependent on Algorithm~\ref{Alg: Post-Processing} which is undesirable as it requires the availability of anchor predictions which may not always be available for arbitrary CTs. As an additional contribution besides the dataset, we develop a more robust, model based on the available Atlas Knowledge which is more suitable for a clinical environment in which the model has to process arbitrary CT Volumes. We will refer to the previous model used to generate the dataset as the Atlas dataset model (V1) and the novel model as the Atlas prediction model (V2).

\subsubsection*{Developing a Prediction Model from the Atlas Dataset}
The goal of the Atlas prediction model is to eliminate the need for post-processing which is impractical within a clinical setting in which the model should be able to deliver convincing results on arbitrary CT volumes. When examining the different steps of Algorithm~\ref{Alg: Post-Processing}, we notice two steps that are easy to address algorithmically: sex-based consistency and non-largest connected component suppression as defined in Algorithms~\ref{Alg: Sex-based consistency} and~\ref{Alg: Non largest CC supression} respectively, as these methods simply suppress predictions and do not rely on other anchor predictions such as Algorithm~\ref{Alg: Left-Right-Splitting}. We thus aim to develop a training procedure that eliminates the need for left-right splitting (Algorithm~\ref{Alg: Left-Right-Splitting}), area-restrictions (Algorithm~\ref{Alg: Area Restriction}), and rib counting (Algorithm~\ref{Alg: Rib counting}).

To tackle these challenges we develop a custom training strategy for the Atlas prediction model. First, we apply Algorithm ~\ref{Alg: Post-Processing} during the aggregation phase of the individual expert models to maximize the agreement with the desired output which has been approved by experts. Next, we observe that due to the large number of classes, the standard nnU-Net~\cite{isensee2021nnu} learning rate schedule is suboptimal as it closely follows a linear learning rate schedule allocating approximately the same number of epochs for small and large learning rates. We find that the proposed task is more difficult than most standard segmentation tasks and thus increase the number of training epochs from $1000$ to $5000$. Finally, we fine-tune the network for another $1000$ epochs with a fixed learning rate of $0.001$ and without the standard mirror augmentation. This allows the network to focus on the improvements on smaller structures and helps to mitigate the right-left and rib confusion. We show a comparison of the raw output of the Atlas dataset model, the post-processed volume, and the raw output of the Atlas prediction model in Fig.~\ref{fig:V1_V2_comparison}. As it can be seen, the output of the robust model has a large agreement with the post-processed predictions of the first model without relying on Algorithm~\ref{Alg: Post-Processing}. We analyze this behavior and find that the vast majority of predicted structures have an agreement of more than $90\%$ IoU between the post-processed V1 Model and the raw V2 predictions.

Besides the DAP Atlas dataset, we also release the robust segmentation model which can be used to perform inference without post-processing. It furthermore tends to perform better for out-of-distribution tasks which are common within a clinical setting. We examine this behavior in Section \textit{Technical Validation}.

% Figure environment removed

%We examine the same 25 volumes and compare these to previously discussed limitations. 

%TODO Do the comparison The discussion and the figure

%Besides the proposed dataset generated by the previously described procedure and approved by experts, we also release a second version of the dataset which has been generated by the robust model, while there is no substantial difference among the two datasets as they have a mean IOU of almost $90\%$ we do not perform the same costly evaluation procedure twice. We generally recommend the usage of the approved standard V1 dataset and the robust models for custom inference without post-processing, as it tends to perform better for out-of-distribution tasks. We examine this behaviour in Section~\ref{S: Technical Validation}.


\section{Experimental Evaluations}\label{sec:experiment}

\textbf{Implementation.}
We implement \puma\ on top of SecretFlow~\citep{spu} in \textrm{C++} and Python. SecretFlow compiles a high-level Flax code to secure computation protocols, which are then executed by our designed cryptographic backends, and we encode the floating-ponit values as $64$-bit integers in ring $\mathbb{Z}_{2^{64}}$ with $18$-bit fractional part. 
Our experiments are run on 3 Alibaba Cloud ecs.g7.8xlarge servers with 32 vCPU and 128GB RAM each. The CPU model is Intel Xeon(Ice Lake) Platinum 8369B CPU @ 2.70GHz. We evaluate \puma\ on Ubuntu 20.04.6 LTS with Linux kernel 5.4.0-144-generic. Our bandwidth is about 5Gbps and round trip time is about 1ms. %\cheng{Describe fixed point parameters: scale, share bits.}

\textbf{Models \& Datasets.}
We evaluate \puma\ on seven NLP models: Bert-Base, Roberta-Base, and Bert-Large~\citep{bert}; GPT2-Base, GPT2-Medium, and GPT2-Large~\citep{gpt}; and LLaMA-7B~\citep{touvron2023llama}. We measure the Bert performance for three NLP tasks over the datasets of Corpus of Linguistic Acceptability (CoLA), Recognizing Textual Entailment (RTE), Stanford Question Answering Dataset (QNLI) from GLUE benchmarks~\citep{wang2018glue}, and GPT2 performance on Wikitext-103 V1~\citep{merity2016pointer}.

\textbf{Baseline.}
We compare \puma\ to the most similar prior work \mpcformer~\citep{li2023mpcformer}. But for fair comparison, we have the following considerations:
\romannumeral1) As \mpcformer\ neither supports loading pretrained transformer models nor implements LayerNorm faithfully\footnote{ As \mpcformer~does not support loading pre-trained Transformer models, we did an experiment in plaintext Bert-Base that replaced LayerNorm with BatchNorm  as \mpcformer~did. This  resulted in a significant drop in the MCC score for CoLA task from $0.616$ to $-0.020$. On the contrary, \puma~achieves an MCC score of $0.613$. }, we cannot achieve meaningful secure inference results using their framework.
Therefore, we compare our secure Transformer models inference performance to that of plaintext (floating-point) to show our precision guarantee.
\romannumeral2) \mpcformer\ with \textit{Quad} approximations (for both $\gelu$ and $\softmax$) requires retraining the  modified models. As \puma\ does not require retraining, we compare our cost to that of \mpcformer\ without \textit{Quad} approximations. Also, we re-run \mpcformer~in our environment.



\subsection{Precision}\label{sec:accuracy}

% Figure environment removed

%\begin{table}
\centering
\caption{Performance on GLUE benchmark of Bert-Base, Roberta-Base, and Bert-Large on CoLA, RTE, and QNLI, Matthews correlation is reported for CoLA. Accuracy is reported for other datasets.}\label{table:bertacc}
\begin{tabular}{c|ccc|ccc|ccc}
\hline \hline
 Model & \multicolumn{3}{c|}{Bert-Base} & \multicolumn{3}{c|}{Roberta-Base} & \multicolumn{3}{c}{Bert-Large} \\ \hline
 TASK & CoLA & RTE & QNLI & CoLA & RTE & QNLI & CoLA & RTE & QNLI \\ \hline
CPU & $0.616$     & $0.700$      & $0.916$     & $0.629$ & $0.805$ & $0.920$  & $0.686$   & $0.755$ & $0.922$ \\
\puma   & $0.613$     & $0.700$     & $0.916$     & $0.618$ & $0.805$ & $0.918$ & $0.690$ & $0.747$ & $0.918$ \\ \hline \hline
\end{tabular}
\end{table}

\begin{table}[]
    \centering
    \caption{Perplexity of GPT2-Base, GPT2-Medium, and GPT2-Large on Wikitext-103 V1.}
    \label{tab:gpot2ppl}
    \begin{tabular}{c|c|c|c}
    \hline \hline
      Model & GPT2-Base & GPT2-Medium & GPT2-Large \\ \hline
      CPU & $16.284$ & $12.536$ & $10.142$ \\
      \puma & $16.284$ & $12.540$ & $10.161$ \\
      \hline \hline
    \end{tabular}
    
\end{table}

We compare our secure model 
inference performance to that of plaintext (floating-point) in Figure~\ref{fig:performance} to show our precision guarantee.

In Figure~\ref{fig:bert-base}-\ref{fig:bert-large}, we show the Matthews correlation/accuracy of plaintext and \puma\ on the Bert-Base, Roberta-base, and Bert-Large. We observe that the accuracy achieved by \puma~ matches the accuracy of the plaintext Flax code. Specifically, the accuracy difference does
not exceed $0.011$ over all datasets. 

Moreover, in Figure~\ref{fig:gpt2}, we also compare our perplexity on dataset Wikitext-103 V1 with the plaintext baseline on models GPT2-Base, GPT2-Medium, and GPT2-Large. The results are similar and the perplexity differences do not exceed $0.02$ over all models.

The above accuracy and perplexity advantages experimentally validate that our protocols are numerically precise. 

\subsection{Inference cost}\label{sec:efficiency}
\begin{table}[h]
    \centering
    \caption{Costs of Bert-Base, Roberta-Base, and Bert-Large for one sentence of length $128$. Time is in seconds and Communication (Comm. for short) is in GB, which is the same for the following tables.}\label{tab:costbert}
    \begin{tabular}{c|cc|cc|cc}
    \hline \hline
       Model & \multicolumn{2}{c|}{Bert-Base} & \multicolumn{2}{c|}{Roberta-Base} & \multicolumn{2}{c}{Bert-Large} \\ \hline
       Costs & Time & Comm. & Time & Comm. & Time & Comm. \\ \hline
       \mpcformer & $55.320$ & $12.089$ & $57.256$ & $12.373$ & $141.222$ & $32.577$ \\
       \puma & $33.913$ & $10.773$ & $41.641$ & $11.463$ & $73.720$ & $27.246$ \\
       \cellcolor{mygray} Improv. & \cellcolor{mygray} $1.631\times$ & \cellcolor{mygray} $1.122\times$ & \cellcolor{mygray} $1.375\times$ & \cellcolor{mygray} $1.079\times$ & \cellcolor{mygray} $1.916\times$ & \cellcolor{mygray} $1.195\times$ \\
       \hline \hline
    \end{tabular}
    \vspace{-0.2cm}
\end{table}

\begin{table}[]
    \centering
    \caption{Costs of GPT2-Base, GPT2-Medium, and GPT2-Large. The input sentence is of length $32$, all of the costs are for generating $1$ token.}\label{tab:costgpt2}
    \begin{tabular}{c|cc|cc|cc}
    \hline \hline
       Model & \multicolumn{2}{c|}{GPT2-Base} & \multicolumn{2}{c|}{GPT2-Medium} & \multicolumn{2}{c}{GPT2-Large} \\ \hline
       Costs & Time & Comm. & Time & Comm. & Time & Comm. \\ \hline
       \mpcformer & $34.889$ & $4.999$ & $73.078$ & $11.766$ & $129.095$ & $22.522$  \\
       \puma & $15.506$ & $3.774$ & $30.272$ & $7.059$ & $54.154$ & $11.952$ \\
       \cellcolor{mygray} Improv. & \cellcolor{mygray} $2.250\times$ & \cellcolor{mygray} $1.325\times$ & \cellcolor{mygray} $2.414\times$ & \cellcolor{mygray} $1.667\times$ & \cellcolor{mygray} $2.383\times$ & \cellcolor{mygray} $1.884\times$ \\
       \hline \hline
    \end{tabular}
    \vspace{-0.2cm}
\end{table}

In this subsection, we compare \puma's inference cost to that of \mpcformer. 
We evaluate  three Bert models (Bert-Base, Roberta-Base, and Bert-Large) and three GPT2 models (GPT2-Base, GPT2-Medium, and GPT2-Large).
The costs are for processing one input sentence: \romannumeral1) For Bert models the input sentence is of length $128$. \romannumeral2) GPT2 models input one length-32 sentence and generate $1$ new word. 

On the 3 Bert models in Table~\ref{tab:costbert}, \puma\ is  $1.375\sim 1.916\times$ faster than  \mpcformer, and is $1.079\sim 1.195\times$ more communication-efficient. For the GPT2 models in Table~\ref{tab:costgpt2}, \puma\ is $2.250\sim 2.414\times$ faster than \mpcformer, and is $1.325\sim 1.884\times$ more communication-efficient. 
    
We observe that \puma's improvements increase as the model size grows, particularly for the GPT2 models. This trend is because our specialized optimizations are more effective when processing large-scale evaluations.



\subsection{Scalability}\label{sec:scala}

In this subsection, we measure the costs of evaluating \puma\ on Bert-Base and GPT2-Base models for varying-length inputs, and varying-length outputs (only for GPT2-Base). We also compare our costs to those of \mpcformer~to demonstrate our improvements.





\begin{table}[]
    \centering
    \caption{Costs of Bert-Base and GPT2-Base for different input length (denoted as \#Input). The input lengths for Bert-Base and GPT2-Base are respective $\{64, 128, 256, 512\}$ and $\{16, 32, 64, 128\}$. GPT2-Base generates $1$ token.}\label{tab:costbertinput}
    \begin{tabular}{cc|cc|cc|cc|cc}
    \hline \hline
       \multicolumn{2}{c|}{\#Input} & \multicolumn{2}{c|}{$64 / 16$} & \multicolumn{2}{c|}{$128 / 32$} & \multicolumn{2}{c|}{$256 / 64$} & \multicolumn{2}{c}{$512 / 128$}  \\ \hline
       \multicolumn{2}{c|}{Costs} & Time & Comm. & Time & Comm. & Time & Comm. & Time & Comm. \\ \hline
       \multirow{3}{*}{Bert}& \mpcformer & $46.428$ & $4.750$ & $85.887$ & $9.673$ & $196.372$ & $23.443$ & $582.787$ & $68.069$ \\
       & \puma & $24.345$ & $1.627$ & $42.525$ & $3.591$ & $87.561$ & $8.668$ & $212.600$ & $23.439$\\
       & \cellcolor{mygray} Improv. & \cellcolor{mygray} $1.907\times$ & \cellcolor{mygray} $2.919\times$ & \cellcolor{mygray} $2.020\times$ & \cellcolor{mygray} $2.694\times$ & \cellcolor{mygray} $2.243\times$ & \cellcolor{mygray} $2.705\times$ & \cellcolor{mygray} $2.741\times$ & \cellcolor{mygray} $2.904$ \\
       \hline
       \multirow{3}{*}{GPT2}& \mpcformer & $34.522$ & $3.767$ & $42.615$ & $4.516$ & $60.451$ & $6.281$ & $105.028$ & $11.225$  \\
       & \puma & $20.692$ & $0.625$ & $29.248$ & $1.258$ & $40.968$ & $2.607$ & $74.529$ & $5.611$\\
       &\cellcolor{mygray} Improv. & \cellcolor{mygray} $1.668\times$ & \cellcolor{mygray} $6.027\times$ & \cellcolor{mygray} $1.457\times$ & \cellcolor{mygray} $3.590\times$ & \cellcolor{mygray} $1.476\times$ & \cellcolor{mygray} $2.409\times$ & \cellcolor{mygray} $1.409\times$ & \cellcolor{mygray} $2.001\times$\\
       \hline \hline
    \end{tabular}
\end{table}
\textbf{Input Length Evaluation.}
Table~\ref{tab:costbertinput} shows our costs on varying-length inputs, we evaluate Bert-Base on the inputs of length $\{64, 128, 256, 512\}$, and GPT2-Base on the inputs of length $\{16, 32, 64, 128\}$.
For Bert-Base, \puma\ is $1.720\sim 2.282\times$ faster, and for GPT2-Base, \puma\ is $1.550\sim 2.686\times$ faster. Unlike the observations in Section~\ref{sec:efficiency}, our efficiency gains decrease with increasing input sizes in GPT2, and \puma\ requires more communication when the input length is greater than 64. This phenomenon is attributed to the interesting fact: To directly support pre-trained plaintext models, \puma\ strictly follows the plaintext model format that only accept token ids as input, so \puma\ has to compute the one-hot vectors from token ids in an MPC way. On the other hand, \mpcformer\ uses modified models that accept one-hot vectors as input, so the one-hot function could be computed at the client side in plaintext. Nevertheless, \puma\ remains faster than \mpcformer.

%\begin{table}[]
    \centering
    \caption{Costs of GPT2-small for generating different output tokens (denoted as \#Output), the input length is set as $32$.}\label{tab:costgpt2tokens}
    \begin{tabular}{c|cc|cc|cc|cc}
    \hline \hline
       \#Output & \multicolumn{2}{c|}{2} & \multicolumn{2}{c|}{4} & \multicolumn{2}{c|}{8} & \multicolumn{2}{c}{16}  \\ \hline
       Costs & Time & Comm. & Time & Comm. & Time & Comm. & Time & Comm. \\ \hline
       \mpcformer & $72.833$ & $7.676$ & $132.644$ & $13.998$ & $252.796$ & $26.648$ & $494.509$ & $51.972$ \\
       \puma & $53.191$ & $2.549$ & $111.457$ & $5.167$ & $215.352$ & $11.115$ & $457.994$ & $24.917$ \\
       Improv. & $1.369\times$ & $3.011\times$ & $1.190\times$ & $2.709\times$ & $1.174\times$ & $2.397\times$ & $1.080\times$ & $2.086\times$ \\
       \hline \hline
    \end{tabular}
\end{table}

\begin{wrapfigure}{r}{0.4\textwidth}
    % Figure removed
    \caption{Runtime of GPT2-Base for generating different number of output tokens, the input length is of length $32$.} 
    \label{fig:gptwoutcosts}
\end{wrapfigure}

\textbf{Output Length Evaluation.}
Fig~\ref{fig:gptwoutcosts} presents our costs on varying-length outputs for GPT2-Base, and compares our costs to those of \mpcformer. Our improvements in runtime range from $1.279\sim 2.700\times$ respectively.
As more output tokens are generated, both costs increase in a linear way, this is because each output token must be input back into the model to generate the next token, increasing the required one-hot embedding costs. We should emphasize
again that although the time costs might be close for long outputs, \puma\ could achieve a similar accuracy as plaintext models while \mpcformer\  could not. 


\begin{table}[]
    \centering
    \caption{Costs of the secure inference of LLaMA-7B, \#Input denotes the length of input sentence and \#Output denotes the number of generated tokens.}\label{tab:llama7b}
    \begin{tabular}{c|cc|cc|cc}
    \hline \hline
       (\#Input, \#Output) & \multicolumn{2}{c|}{$(4,1)$} & \multicolumn{2}{c|}{$(8,1)$} & \multicolumn{2}{c}{$(8,2)$} \\ \hline
       Costs & Time & Comm. & Time & Comm. & Time & Comm. \\ \hline
       \puma & $122.004$ & $0.907$ & $200.473$ & $1.794$ & $364.527$ & $3.857$ \\
       \hline \hline
    \end{tabular}
    \vspace{-0.2cm}
\end{table}

\textbf{Scale to LLaMA-7B in Five Minutes.}
We evaluated the large language model LLaMA-7B using \puma\ under 3 Alibaba Cloud
ecs.r7.32xlarge servers, each has 128 threads and 1TB RAM, with 20GB bandwidth, 0.06ms round-trip-time. 
As shown in Table~\ref{tab:llama7b}, \puma\ can support the secure inference of large language model LLaMA-7B with reasonable costs. For example, given an input sentence of 8 tokens, \puma\ can output one token in around $346.126$ seconds with communication costs of $1.865$ GB. To our knowledge, this is the first time that LLaMA-7B has been evaluated using MPC.


%Llama-7B, LAN=(20GB, 0.06ms), 128 threads, input length=8, output=1 token, costs: 346.126s, 2002213760 bytes
\section{Results of RL active flow control}\label{sec:Results}

In this section, we discuss the converge of the RL algorithms for the three FM and PM cases (\S\ref{subsec:Convergence}) and evaluate their drag reduction performance (\S\ref{Result_drag_reduction}). A parametric analysis of the effect of NARX memory length is presented (\S\ref{subsec:Nfs}) and the isolated effect of including past actions as observations during the RL training and control (\S\ref{subsec:past_actions}). Studies of reward function (\S\ref{subsec:Rewards_Study}), sensor placement (\S\ref{subsec:Sensor_study}) and generalisability to Reynolds number changes (\S\ref{subsec:Res}) are presented, followed by a comparison of SAC and TQC algorithms (\S\ref{subsec:SACvsTQC}). 

\subsection{Convergence of learning}\label{subsec:Convergence}

We perform RL with the maximum entropy TQC algorithm to discover control policies for the three cases shown in figure \ref{fig:Case_Demo}, which maximise the net-power-saving reward function given by \req{eq: PowerR}. During the learning stage, each episode (1 DNS simulation) corresponds to $200$ non-dimensional time units.  To accelerate learning, $65$ environments run in parallel.


Figure \ref{fig:Learning_Curve} shows the learning curves of the three cases.  Table \ref{tab:LearningConvergence} shows the number of episodes needed for convergence and relevant parameters for each case.
It can be observed from the curve of episode reward that the RL agent is updated after every 65 episodes, i.e. $1$ iteration, where the episode reward is defined as 
\begin{equation}
R_{ep} = \sum_{k=1}^{N_k} r_{k},
\label{eq:Epi_R}
\end{equation}
where $k$ denotes the $k^{th}$ RL step in one episode and $N_k$ is the total number of samples in one episode.
The root mean square (RMS) value of the drag coefficient, $C_D^{RMS}$, at the asymptotic regime of control, is also shown to demonstrate convergence, defined as 
$C_D^{RMS} = \sqrt { (\mathcal{D}(\langle C_D\rangle_{env}))^2 }$,
where the operator $\mathcal{D}$ detrends the signal with a $9^{th}$-order polynomial and removes the transient part, and $\langle ~ \rangle_{env}$ denotes the average value of parallel environments in a single iteration. 

% Figure environment removed

\begin{table}
  \begin{center}
\def~{\hphantom{0}}
  \begin{tabular}{lcccccc}
    
      Environment  & Algorithm  &  $N_{c}$ & $R_{ep,c}$ & (Layers, Neurons) & $N_{fs}$ & Number of Inputs \\ 
       FM-Static   & TQC & $325$ & $37.72$ & (3,512) & $0$ & $64p_t+2a_{t-1}$\\
       PM-Static   & TQC & $1235$ & $21.87$ & (3,512) & $0$ & $64p_t+2a_{t-1}$\\
       PM-Dynamic  & TQC & $715$ & $34.35$ & (3,512) & $27$ & $N_{fs} (64p_t+2a_{t-1})$\\
  \end{tabular}
  \caption{Number of episodes $N_{c}$ required for RL convergence in different environments. The episode reward $R_{ep,c}$ at the convergence point, the configuration of NN and the dimension of inputs are presented for each case. $N_{fs}$ is the finite-horizon length of past actions-measurements.}
  \label{tab:LearningConvergence}
  \end{center}
\end{table}

In figure \ref{fig:Learning_Curve}, it can be noticed that in the FM environment, RL converges after approximately $325$ episodes ($5$ iterations) to a   {nearly} optimal policy using a static   {feedback} controller. As will be shown in \S\ref{Result_drag_reduction}, this policy is globally optimal since the vortex shedding is fully attenuated and the jets converge to zero mass flow actuation, thus recovering the unstable base flow and the minimum drag state.  However, with the same static   {feedback} controller in a PM environment (POMDP), the RL agent fails to discover the   {nearly} optimal solution, requiring around $1235$ episodes for convergence but only obtaining a relatively low episode reward.
Introducing a dynamic   {feedback} controller in the PM environment, the RL agent convergences to a near-optimal solution in 735 episodes. The dynamic   {feedback} controller trained by RL achieves a higher episode reward (34.35) than the static   {feedback} controller in the PM case (21.87), which is close to the FM case (37.72). The learning curves illustrate that using a finite horizon of past actions-measurements ($N_{fs} = 27$) to train a dynamic   {feedback} controller in the PM case improves learning in terms of speed of convergence and accumulated reward achieving nearly optimal performance with only wall pressure measurements. 


\subsection{Drag reduction with dynamic RL controllers} \label{Result_drag_reduction}

% Figure environment removed

The trained controllers for the cases shown in figure \ref{fig:Case_Demo} are evaluated to obtain the results shown in figure \ref{fig:TQC_FMPM}.   {Evaluation tests are performed for 120 non-dimensional time units to show both transient and asymptotic dynamics of the closed-loop system.}
Control is applied at $t=0$ with the same initial condition for each case, i.e. steady vortex shedding with average drag coefficient $\langle C_{D0}\rangle \approx 1.45$ (baseline without control). Consistent with the learning curves, the difference in control performance in the three cases can be observed both from the drag coefficient $C_D$ and the actuation $Q_1$.
  {The drag reduction is quantified by a ratio $\eta$ using the asymptotic time-averaged drag coefficient with control $C_{Da} = \langle C_{D}\rangle_{t \in [80,120]}$, the drag coefficient $C_{Db}$ of the base flow (details presented in Appendix \ref{App:BaseFlow}), and the baseline time-averaged drag coefficient without control $\langle C_{D0}\rangle$, as
\begin{equation}
\eta = \frac{\langle C_{D0}\rangle - C_{Da}}{\langle C_{D0}\rangle - C_{Db}} \times 100\%.
\label{eq:drag_reduction}
\end{equation}}

\begin{itemize}

\item {\bf FM-Static:} With a static   {feedback} controller trained in a full-measurement environment, a drag reduction of $\eta = 101.96\%$ is obtained with respect to the base flow (steady unstable fixed point; maximum drag reduction). This indicates that an RL controller informed with full-state information can entirely stabilise the vortex shedding and cancel the unsteady part of the pressure drag.

\item {\bf PM-Static:} A static/memoryless controller in a partial-measurement environment leads to performance degradation and a drag reduction of   {$\eta = 56.00\%$} in the asymptotic control stage, i.e. after $t=80$, compared to the performance of ``FM-Static''. This performance loss can also be observed from the control actuation curve, as $Q_1$ oscillates with a relatively large fluctuation in ``PM-Static'' while it stays about zero in the ``FM-Static'' case. 
The discrepancy between FM and PM environments using a static   {feedback} controller reveals the challenge of designing a controller with a POMDP environment. The RL agent cannot fully identify the dominant dynamics with only partial measurements on the   {downstream} surface of the bluff body, resulting in sub-optimal control behaviour.

\item{\bf PM-Dynamic:} With a dynamic   {feedback} controller (NARX model presented in \S\ref{subsec:PM_Dynamic}) in a partial-measurement environment, the vortex shedding is stabilised and the dynamic   {feedback} controller achieves   {$\eta = 97.00\%$} of the maximum drag reduction after time $t=60$. Although there are minor fluctuations in the actuation $Q_1$, the energy spent in the synthetic jets is significantly lower compared to the ``PM-Static'' case. Thus, a dynamic   {feedback} controller in PM environments can achieve nearly optimal drag reduction, even if the RL agent only collects information from pressure sensors on the   {downstream} surface of the body. The improvement in control indicates that the POMDP due to the PM condition of the sensors can be reduced to an approximate MDP by training a dynamic   {feedback} controller with a finite horizon of past actions-measurements. Furthermore, high-frequency action oscillations, which can be amplified with static   {feedback} controllers, are attenuated in the case of dynamic   {feedback} control. These encouraging and unexpected results support the effectiveness and robustness of model-free RL control in practical flow control applications, in which sensors can only be placed on a solid surface/wall.

\end{itemize}


% Figure environment removed

In figure \ref{fig:Contour}, snapshots of the velocity magnitude   {$|\boldsymbol{u}| = \sqrt{u^2+v^2}$} are presented for ``Baseline'' without control, ``PM-Static'', ``PM-Dynamic'' and ``FM-Static'' control cases. Snapshots are captured at $t=100$ in the asymptotic regime of control. A vortex-shedding structure of different strengths can be observed in the wake of all three controlled cases. In ``PM-Static'', the recirculation area is lengthened compared to the baseline flow, corresponding to base pressure recovery and pressure drag reduction. A longer recirculation area can be noticed in ``PM-Dynamic'' due to the enhanced attenuation of vortex shedding and pressure drag reduction. The dynamic   {feedback} controller in the PM case renders a $326.22\%$ increase of recirculation area with respect to the baseline flow, while only a $116.78\%$ increase is achieved by a static   {feedback} controller. The ``FM-Static'' case has the longest recirculation area, and the vortex shedding is almost fully stabilised, which is consistent with the drag reduction shown in figure \ref{fig:TQC_FMPM}.

% Figure environment removed

Figure \ref{fig:Obs} presents first- and second-order base pressure statistics for the baseline case without control and PM cases with control. In figure \ref{fig:Obs}(a), the time-averaged value of base pressure, $\overline{p}$, demonstrates the base pressure recovery after control is applied. Due to flow separation and recirculation, the time-averaged base pressure is higher at the middle of the   {downstream surface}, which is retained with control. The base pressure increase is directly linked to pressure drag reduction, which quantifies the control performance of both static and dynamic   {feedback} controllers. Up to $49.56\%$ of pressure increase at the centre of the   {downstream surface}  is obtained in the ``PM-Dynamic'' case, while only $21.15\%$ can be achieved by a static   {feedback} controller. In figure \ref{fig:Obs}(b), the base pressure RMS is shown. For the baseline flow, strong vortex-induced fluctuations of the base pressure can be noticed around the top and bottom   {on the downstream surface} of the bluff body. In the ``PM-Static'' case, the RL controller   {partially suppresses} the vortex shedding, leading to a sub-optimal reduction of the pressure fluctuation. The sensors close to the top and bottom corners are also affected by the synthetic jets, which change the RMS trend for the two top and bottom measurements. In the ``PM-Dynamic'' case,  the pressure fluctuations are nearly zero for all the measurements on the   {downstream surface}, highlighting the success of vortex shedding suppression by a dynamic RL controller in a PM environment.

% Figure environment removed

The differences between static and dynamic controllers in PM environments are further elucidated in figure \ref{fig:Action_analysis} by examining  the time series of pressure differences $\Delta p_t$ from surface sensors (control input) and control actions $a_{t-1}$ (output). The pressure differences are calculated from sensor pairs at $y=\pm y_{sensor}$, where $y_{sensor}$ is defined in Eq. \req{eq:Probe_base}. For $N=64$, there are 32 time series of $\Delta p_t$ for each case. 
%
During  the initial stages of control ($t \in [0,11]$), the control actions are similar  for the two PM cases and they deviate for $t>11$, resulting in discernible control performance at the asymptotic regime. 
At the initial stages, the controllers operate in nearly anti-phase to $\Delta p_t$, in order to eliminate the antisymmetric pressure component due to vortex shedding. The inability of the static controller to have a frequency dependent amplitude (and phase), manifests as well through the amplification of high frequency noise. For $t>11$, the static feedback controller continues to operate in nearly anti-phase to the pressure difference, resulting in partial stabilisation of unsteadiness. However, the dynamic feedback controller adjusts its phase and amplitude significantly, which attenuates the antisymmetric fluctuation of base pressure and drives $\Delta p_t$ to near zero. 

% Figure environment removed

Figure \ref{fig:ContourComparision} shows instantaneous vorticity contours for PM-Dynamic and PM-Static cases, showing both the similarities and discrepancies between the two cases. At $t=2$, flow is expelled from the bottom jet for both cases, generating a clockwise vortex, termed V1. This V1 vortex, shown in black, works against the primary counter-clockwise vortex labelled as P1, depicted in red, emerging from the bottom surface. At $t=5.5$, a secondary vortex, V2, forms from the jets to oppose the primary vortex shedding from the top surface (labelled as P2). 
%
 At $t=13$, the suppression of the two primary vortices near the bluff body is evident in both cases, indicated by their less tilted shapes compared to the previous time instances. At $t=13$, the PM-Dynamic adjusted the phase of the control signal, which corresponds to a marginal action at this time instance at figure \ref{fig:Action_analysis}. Consequently, no additional counteracting vortex is formed in PM-Dynamic. However, in the PM-Static scenario, the jets generate a third vortex, labelled V3, which emerges from the top surface. This corresponds to a peak in the action of the PM-Static controller at this time. The inability of the PM-Static controller to adapt the amplitude/phase of the input/output behaviour results in suboptimal performance.

\subsection{Horizon of the finite-history sufficient statistic}\label{subsec:Nfs}

A parametric study on the horizon of the finite history in NARX (equation \req{eq:NARX}), i.e. the number of frames stacked $N_{fs}$, is presented in this section. Since the NARX model uses a finite horizon of past actions-measurements in  \req{eq:Sufficient_statistic}, the horizon of the finite history affects the convergence of the approximation \citep{yu_near_2008}. This approximation affects the optimisation during the learning of RL because it determines whether the RL agent can observe sufficient information to converge to an optimal policy. 

Since vortex shedding is the dominant instability to be controlled, the choice of $N_{fs}$ should intuitively link to the timescale of the vortex shedding period. The ``frames'' of observations are obtained every RL step ($0.5$ time units), while the vortex shedding period is $t_{vs}\approx6.85$ time units. Thus, $N_{fs}$ is rounded to integer values for different numbers of vortex shedding periods, as shown in table \ref{tab:Frame_Stack}.


% Figure environment removed

\begin{table}
  \setlength{\tabcolsep}{12pt}
  \begin{center}
\def~{\hphantom{0}}
  \begin{tabular}{ccc}
      Number of  & Non-dimensional &  History length \\
      VS periods &    time units          &  ($N_{fs}$)         \\ [3pt]
      \hline
       0.5   & 3.43 & 7 \\
       1   & 6.85 & 14 \\
       2  & 13.70 & 27 \\
       3 & 20.55 & 41\\
       4 & 27.40 & 55\\
       5 & 34.25 & 68\\
  \end{tabular}
  \caption{Correspondence between the number of vortex shedding (VS) periods and frame stack (history) length in samples $N_{fs}$. The RL control step size is $t_a =0.5$, and $N_{fs}$ is rounded to an integer.}
  \label{tab:Frame_Stack}
  \end{center}
\end{table}

The results of time-averaged drag coefficients $\langle C_{D}\rangle$ after control and the average episode rewards $\langle R_{ep}\rangle$ in the final stage of training are presented in figure \ref{fig:Frame_Stack}. As $N_{fs}$ increases from 0 to 27, the performance of RL control improves, resulting in a lower $\langle C_{D}\rangle$ and a higher $\langle R_{ep}\rangle$. $N_{fs}=2$ is specially examined because the latent dimension of the vortex shedding limit cycle is 2. However, the control performance with $N_{fs}=2$ is marginally improved to the one with $N_{fs}=0$, i.e. a static   {feedback} controller. This result indicates that the horizon consistent with the vortex shedding dimension is not long enough for the finite horizon of past action measurements. The optimal history length to achieve stabilisation of the vortex shedding   {in PM environments} is 27 samples, which are equivalent to 13.5 convective time units or $\sim 2$ vortex shedding periods. 

With $N_{fs}=41$ and $N_{fs}=55$, the drag reduction and episode rewards drop slightly compared to $N_{fs}=27$. The decline in performance is non-negligible as $N_{fs}$ increases further to 68. This decline shows that excessive inputs to the neural networks (see table \ref{tab:LearningConvergence}), may impede training because more parameters need to be tuned or larger neural networks need to be trained. 

\subsection{Observation sequence with past actions}\label{subsec:past_actions}

Past actions (exogenous terms in NARX) facilitate reducing a POMDP to an MDP problem, as discussed in \S\ref{subsec:PM_Dynamic}. In the near-optimal control of a PM environment using a dynamic   {feedback} controller with inputs $\left( o_t, o_{t-1}, ..., o_{t-N_{fs}} \right)$, a sequence of observations $o_t = \left \{ p_t, a_{t-1}\right \}$ at step $t$ is constructed to include pressure measurements and actions. In the FM environment, due to the introduction of one-step delayed action due to the first-order-hold interpolation given by \req{eq:FOH_action}, the inclusion of the past action along with the current pressure measurement, meaning $o_t = \left \{ p_t, a_{t-1} \right \}$, is required even when the sensors are placed in the wake and cover the wavemaker region. 

Figure \ref{fig:ActionInObs} presents the control performance for the same environment with and without past actions included.
In the FM case, there is no apparent difference between RL control with $o_t = \left \{ p_t, a_{t-1} \right \}$ or $o_t = \left \{ p_t \right \}$, which indicates that the inclusion of the past action is negligible to the performance. This is the case when the RL sampling frequency is sufficiently faster than the timescale of the vortex shedding dynamics. 
In PM cases, if exogenous action terms are not included in the observations but only the finite history of pressure measurements is used, the RL control fails to converge to a near-optimal policy, with only   {$\eta = 67.45\%$}  drag reduction. With past actions included, the drag reduction of the same environment increases up to   {$\eta = 97.00\%$}. 

The above results show that in PM environments, sufficient statistics cannot be constructed only from the finite history of measurements. Missing state information needs to be reconstructed by both state-related measurements and control actions. 

% Figure environment removed

\subsection{Reward study}
\label{subsec:Rewards_Study}

In \S\ref{Result_drag_reduction}, a power-based reward function given by \req{eq: PowerR} has been implemented, and stabilising controllers can be learned by RL, as shown. In this section, RL control results with other forms of reward functions (introduced in \S\ref{subsec:Reward}) are provided and discussed.

% Figure environment removed

The control performance of RL control with the different reward functions is evaluated based on the drag coefficient $C_D$ shown in figure \ref{fig:Reward_Study}. Static   {feedback} controllers are trained in FM environments, and dynamic   {feedback} controllers are trained in PM environments. In FM cases, control performance is not sensitive to the choice of reward function (power or force-based).  
In PM cases, the discrepancies between RL-step time-averaged and instantaneous rewards can be observed in the asymptotic regime of control. The controllers with both rewards (power or force-based) achieve nearly optimal control performance, but there is some unsteadiness in the cases using instantaneous rewards due to slow statistical convergence of the rewards and limited correlation to the partial observations.

All four types of reward functions studied in this work achieve nearly optimal drag reduction around $100\%$. However, the energy-based reward (``PowerR'') offers an intuitive reward design, attributable to its physical properties and the dimensionally consistent addition of the constituent terms of the reward function. Further enhancing its practicality, since the power of the actuator can be directly measured, it avoids the necessity for hyperparameter tuning, as in the force-based reward. Additionally, the results show similar performance with both time-averaged between RL steps and instantaneous rewards, avoiding the necessity for faster sampling for the calculation of the rewards. This choice of reward function can be extended to various RL flow control problems and can be beneficial to experimental studies.


\subsection{Sensor configuration study with partial measurements}\label{subsec:Sensor_study}

% Figure environment removed

In the PM environment, the configuration of sensors (number and location on the downstream surface) may also affect the information contained in the observations and thus control performance. 
Control results of drag coefficient $C_D$ for different sensor configurations in PM-dynamic cases are presented in figure \ref{fig:Sensor_config}. In the configuration with $N = 2$, two sensors are placed at $y=\pm 0.25$, and for $N = 1$, only one sensor is placed at $y = 0.25$. Other configurations are consistent with equation \req{eq:Probe_base}. 

The $C_D$ curves in figure \ref{fig:Sensor_config} show that, as the number of sensors is reduced from 64 to 2, RL control achieves the same level of performance with minor discrepancies due to randomness in different learning cases. However, if RL control uses observations from only one sensor at $y = 0.25$, performance degradation can be observed in the asymptotic stage with 19.79\% on average less drag reduction. The sub-figure presents the relationship between the number of sensors and asymptotic drag coefficient $\langle C_D \rangle$. These results indicate a limit on sensor configuration for the use of the NARX-modeled controller to stabilise the vortex shedding. 

% Figure environment removed

To understand the cause of performance degradation in the $N=1$ case, the pressure measurements from two sensors in both baseline and PM-Dynamic cases are presented in figure \ref{fig:Pressure2Sensors}. In the baseline case, two sensors are placed at the same location as the $N=2$ case ($y=\pm 0.25$) only for observations. It can be observed that the pressure measurements from two sensors are anti-symmetric since they are placed symmetrically on the downstream surface.
In the PM-Dynamic case, the NARX controller is used, and control is applied at $t=0$. In this closed-loop system, the anti-symmetric relationship between two sensors (from the symmetric position) is broken by the control actuation, and no correlation is evident. This can be seen during the transient dynamics, e.g. in $t \in [0,10]$. Therefore, when the number of sensors is reduced to $N=1$ by removing one sensor from the $N=2$ case, the dynamic feedback from the removed sensor cannot be fully reflected by the remaining sensor in the closed-loop system. This loss of information affects the fidelity of the control response to the dynamics of the sensor-removing side, causing suboptimal drag reduction in the $N=1$ scenario.

It should be noted that the configuration of 64 sensors is not necessary for control, as $N = 2$ or $N = 16$ also achieves nearly optimal performance. The number of sensors $N = 64$ in PM-Static environments is used for comparison with the FM-Static configuration (Eq. \ref{eq:Probe_wake}), which eliminates the effect from different input dimensions between two static cases. Also, 64 sensors sufficiently cover the downstream surface of the bluff body to avoid missing spatial information. 
The optimal configuration of sensors can be tuned with optimisation techniques such as \cite{paris_robust_2021}, but the results in figure \ref{fig:Sensor_config} indicate that RL adapts with nearly optimal performance to non-optimised sensor placement in the present environment.

\subsection{Performance of RL controllers to unseen $Re$} \label{subsec:Res}

% Figure environment removed

The RL controller is tested at different Reynolds numbers, in order to examine its generalisability to environment changes. The controllers have been trained at $Re=100$ with both FM and PM conditions, and tested at $Re= 80, 90, 100, 110, 120, 150$. The controllers were further trained at $Re=150$, denoted as continual learning (CL), and tested again at $Re=150$. 

As shown in figure \ref{fig:Res}, in both ``PM-Dynamic'' and ``FM-Static'' cases, the RL controllers are able to reduce drag by $\eta=64.68\%$ in the worst case, when $Re$ is close to the training point at $Re=100$, i.e. the test cases with $Re= 80, 90, 100, 110, 120$. 
However, when applying the controllers trained at $Re=100$ to an environment at $Re=150$, the drag reduction drops to $\eta=41.98\%$ and $\eta = 74.04\%$ in PM-Dynamic and FM-Static cases, respectively.

Performing CL at $Re=150$, the drag reduction is improved to $\eta = 78.07\%$ in PM-Dynamic after 1105 training episodes while $\eta = 88.13\%$ in FM-Static after 390 episodes, with the same RL parameters as the training at $Re=100$.
Overall, the results of these tests indicate that the RL-trained controllers can achieve significant drag reduction in the vicinity of the training point (i.e. $\pm\%20$ $Re$ change). If the test point is far from the training point, a CL procedure can be implemented to achieve nearly optimal control.

\subsection{TQC vs SAC}\label{subsec:SACvsTQC}

% Figure environment removed

Control results with TQC and SAC are presented in figure \ref{fig:TQCvsSAC} in terms of $C_D$. TQC shows a more robust control performance. In the case of FM, SAC might demonstrate a slightly more stable transient behaviour attributed to the fact that the quantile regression process in TQC introduced complexity to the optimisation process. Both controllers achieved an identical level of drag reduction in the FM case. 

However, in the context of the PM cases, it is observed that TQC outperforms SAC in drag reduction with both static and dynamic   {feedback} controllers. For static   {feedback} control, TQC achieved an average drag reduction of   {$\eta = 56.00\%$}, compared to the   {$\eta = 46.31\%$}  reduction achieved by SAC. The performance under dynamic   {feedback} control conditions is more compelling, where TQC fully reduced the drag, achieving   {$\eta = 97.00\%$}  of drag reduction, reverting it to a near-base-flow scenario. In contrast, SAC managed to achieve an average drag reduction of   {$\eta = 96.52\%$}.

The fundamental mechanism for updating Q-functions in RL involves selecting the maximum expected Q-functions among possible future actions. This process, however, can potentially lead to overestimation of certain Q-functions \citep{hasselt_double_2010}. In POMDP, this overestimation bias might be exacerbated due to the inherent uncertainty arising from the partial-state information. Therefore, the Q-learning-based algorithm, when applied to POMDPs, might be more prone to choosing these overestimated values, thereby affecting the overall learning and decision-making process.

As mentioned in \S\ref{subsec:SACTQC}, the core benefit of TQC under these conditions can be attributed to its advanced handling of the overestimation bias of rewards. By constructing a more accurate representation of possible returns, TQC provides a more accurate Q-function approximation than SAC. This process of modulating the probability distribution of the Q-function assists TQC in managing the uncertainties inherent in environments with only partial-state information. In this case, TQC can adapt more robustly to changes and uncertainties, leading to better performance in both static and dynamic feedback control tasks.
\section{Conclusion}\label{sec:conclusion}

This paper presents our empirical domain knowledge distillation framework using ChatGPT and discusses our observations from the framework application experiments in the autonomous driving domain. The key finding is that: 1) with proper design of prompt engineering and execution flow, fully automated domain knowledge (in the ontology format) distillation is possible. However, due to the randomness in the response and the butterfly effect, the quality of fully automated distillation results is not guaranteed. To address this, we develop a web-based assistant to enable manual supervision and early intervention at runtime. We hope our findings and tools inspire future research toward revolutionizing the engineering processes of knowledge-based systems across domains.



\iffalse
% Figure environment removed
\fi
\bibliographystyle{ACM-Reference-Format}
\bibliography{refs}
%\appendix
%\begin{comment}
\section{System Architecture}
\label{appendix:architecture}
\system has a novel modularized system architecture with three key components: 
\emph{StreamManager}, 
\emph{TxnManager} and \emph{TxnScheduler}. 
These components are instantiated in each thread locally.
The execution outline of \system is presented in Algorithm~\ref{alg:algo}.
Transactional stream processing is continuous and potentially never ends (Line 1$\sim$8).
The dependency resolution and execution of state transactions are separated into two non-overlapping phases by punctuations~\cite{Tucker:2003:EPS:776752.776780} (Line 2 and 5), which guarantees that no subsequent input event will have a smaller timestamp. 
Effectively, a batch of state transactions is collected during the first phase, and processed during the second phase.

In the first phase (i.e., stream processing phase), 
the \emph{StreamManager} conducts preprocessing for every input event ($e$). Similar to some prior works~\cite{tstream}, state transactions may be issued but not immediately processed during preprocessing (Line 3).
The \emph{pre\_processing} and \emph{post\_processing} functions are exposed as APIs to users.
The \emph{TxnManager} handles dependency resolution (Line 4) among state transactions and insert decomposed operations to construct a \tpg. We discuss the detailed two-phase \tpg construction process in Section~\ref{subsec:construction}.

In the second phase  (i.e., transaction processing phase), 
the \emph{TxnManager} is first involved again to refine (Line 6) the constructed \tpg with further dependency resolution.
The \emph{TxnScheduler} 
schedules operations for concurrent execution based on the constructed \tpg according to the three dimensions of scheduling decisions (Line 7). 
In particular, a scheduling decision model $M$ is instantiated based on the constructed \tpg (Line 14).
\textbf{\circled{1}} Guided by $M$, execution threads adopt an exploration strategy (Section~\ref{subsec:explore}) to explore the constructed \tpg for operations available to be scheduled constrained by dependencies. 
\textbf{\circled{2}} 
During exploration, one or multiple operations may be treated as the 
% basic 
unit of scheduling (Section~\ref{subsec:granularity}). 
Subsequently, \textbf{\circled{3}} every thread executes operation(s) in the unit of scheduling with various abort handling mechanisms (Section~\ref{subsec:abort_handling}).
Only when state transactions are processed (i.e., committed or aborted) can the associated input events be postprocessed (Line 8) by the \emph{StreamManager} based on transaction processing results.
\end{comment}

\begin{comment}
\begin{algorithm}
\footnotesize
    \KwData{$e$ \tcp{Input event}}
    \KwData{$txn_{ts}$ \tcp{State transaction}}
    \KwData{$G$ \tcp{The currently constructed TPG}}
    \While{!finish processing of input streams}{
        \eIf(\tcp*[h]{Phase 1}){\text{$e$ is not a $punctuation$}}{
                $txn_{ts}$ $\gets$ PRE\_Processing($e$)\;
                \textbf{TPG\_Construction}($G$, $txn_{ts}$)\; 
          }(\tcp*[h]{Phase 2}){
                \textbf{TPG\_Refinement}($G$)\; 
                \textbf{TXN\_Scheduling}($G$)\; 
                POST\_Processing()\;
          }
    }
    
    \SetKwFunction{FMain}{TPG\_Construction}
    \SetKwProg{Fn}{Function}{:}{}
    \Fn{\FMain{$G$, $txn_{ts}$}}{
        $O_{1..k}$ $\gets$ \textbf{Partition} $txn_{ts}$\;
        \ForEach{\text{operation $O_{i}$ $\in$ $O_{1..k}$}}{
            \textbf{Identify} its \ld\;
            $G$ $\gets$ $G$ + $O_{i}$ \;
        }
    }
    \SetKwFunction{FMain}{TPG\_Refinement}
    \SetKwProg{Fn}{Function}{:}{}
    \Fn{\FMain{$G$}}{
        \ForEach{\text{vertex $e_{i}$ $\in$ $G$}}{
            \textbf{Identify} its \td, \pd\;
        }
    }
    
    \SetKwFunction{FMain}{TXN\_Scheduling}
    \SetKwProg{Fn}{Function}{:}{}
    \Fn{\FMain{$G$}}{
        $M$ $\gets$ Instantiated with $G$;\tcp{A decision model}
        \While{!finish scheduling of $G$
        }{
          \textbf{\circled{2}} $Scheduling Unit$ $\gets$ \textbf{\circled{1}} \emph{Explore}($G$, $M$)\; 
            \textbf{\circled{3}} \emph{Execute with Abort Handling} ($Scheduling Unit$)\; 
        }
    }
  \caption{Execution Outline of \system}
  \label{alg:algo}
\end{algorithm}
\end{comment}
\end{document}
\endinput
%%
%% End of file `sample-sigplan.tex'.
