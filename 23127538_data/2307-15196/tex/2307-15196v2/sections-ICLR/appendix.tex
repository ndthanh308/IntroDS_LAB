\newpage
\section{Additional Preliminaries}
\subsection{SGDM Formulations}
Recall \Cref{def:SGDM} where we defined SGDM as following: \begin{align}\label{equ:sgdm-2}
		\vm_{k+1} = \beta_{k}\vm_{k} + (1-\beta_{k})\vg_{k}, \qquad
		\vx_{k+1} = \vx_{k} - \eta_{k}\vm_{k+1}.
\end{align}
Our formulation is different from various SGDM implementations (e.g. pytorch \citep{paszke2019pytorch}) which admit the following form:
\begin{definition}[Standard formulation of SGD with momentum]\label{def:SGDM_standard}
Given a stochastic gradient $\bar{\vg}_k\sim\cG_\sigma(\bar{\vx}_k)$, SGD-Standard with the hyperparameter schedule $(\gamma_k,\mu_k,\tau_k)$ momentum updates the parameters $\bar{\vx}_k\in\RR^d$ from $(\bar{\vm}_0,\bar{\vx}_0)$ as
	\begin{align}
		\bar{\vm}_{k+1} &= \mu_{k}\bar{\vm}_{k} + (1-\tau_{k})\bar{\vg}_{k} \\
		\bar{\vx}_{k+1} &= \bar{\vx}_{k} - \gamma_{k}\bar{\vm}_{k+1}
	\end{align}	
 where $\tau_k\in [0,1)$.
\end{definition}

Notice that with $\tau_k=0$, $\mu_k = \beta$ and $\gamma_k=\gamma$, the standard formulation recovers \Cref{equ:SGDM-intro} that has commonly appeared in the literature. The standard formulation also recovers \Cref{equ:sgdm-2} with $\mu_k = \tau_k=\beta_k$ and $\eta_k=\gamma_k$. Actually, as we will show in the following lemma, these formulations are equivalent up to hyperparameter transformations. 
\begin{lemma}[Equivalence of SGDM and SGD-Standard]\label{lem:SGDM_standard}
    Let $\alpha_k$  be the sequence that $\alpha_0=1$ and 
    \[\alpha_{k+1} = \frac{\alpha_k}{\alpha_k(1-\tau_k)+\mu_k}.\]
    Then $0<\alpha_{k+1}\leq \frac{1}{1-\tau_k}$. For parameters schedules and initialization
    \[\beta_k =\frac{\alpha_{k+1}}{\alpha_k}\mu_k, \eta_k = \frac{\gamma_k}{\alpha_{k+1}}, \vm_0 = \bar{\vm}_0, \vx_0 = \bar{\vx}_0\] then $\bar{\vx}_k$ and $\vx_k$ follow the same distribution. $\vx_k$ is by \Cref{def:SGDM} and $\bar{\vx}_k$ by \Cref{def:SGDM_standard}.
\end{lemma}
\begin{proof}
    We shall prove that given $\vx_k=\bar{\vx_k}$,  $\vg_k=\bar{\vg}_k$ and 
    $\vm_k = \alpha_k\bar{\vm}_k$, there is and $\vm_{k+1} = \alpha_{k+1}\bar{\vm}_{k+1}$ and $\bar{\vx}_{k+1}=\vx_{k+1}$. This directly follows that $\alpha_{k+1}(1-\tau_{k})=1-\beta_k$, and
    \begin{align*}
        \bar{\vm}_{k+1} & = \mu_{k}\bar{\vm}_{k} + (1-\tau_{k})\bar{\vg}_{k}\\
        & = \frac{\mu_{k}}{\alpha_k}\vm_{k} + (1-\tau_{k})\vg_{k}\\
        & = \frac{1}{\alpha_{k+1}}(\beta_k\vm_{k} + \alpha_{k+1}(1-\tau_{k})\vg_{k})\\
        & = \frac{1}{\alpha_{k+1}}\vm_{k+1}.\\
        \bar{\vx}_{k+1} & = \bar{\vx}_{k} - \gamma_{k}\bar{\vm}_{k+1}\\
        & = \vx_{k} - \eta_k \vm_{k+1}.
    \end{align*}  
    Then we can prove the claim by induction that $\vx_k$ and $\bar{\vx}_k$ are identical in distribution. More specifically,
    we shall prove that the CDF for $(\vx_k,\vm_k,\vg_k)$ at $(\vx,\vm,\vg)$ is the same as the CDF for $(\bar{\vx}_k,\bar{\vm}_k,\bar{\vg}_k)$ at $(\vx,\frac{\vm}{\alpha_k},\vg)$.
    For $k=0$, the claim is trivial. If the induction premise holds for $k$, then let $(\vx_{k+1},\vm_{k+1},\vg_{k+1})$ be the one-step iterate in \Cref{equ:sgdm-2} from $(\vx_k,\vm_k,\vg_k)=(\vx,\vm,\vg)$ and $(\bar{\vx}_{k+1},\bar{\vm}_{k+1},\bar{\vg}_{k+1})$ be the one-step iterate in \Cref{def:SGDM_standard} from $(\bar{\vx}_k,\bar{\vm}_k,\bar{\vg}_k)=(\vx,\frac{\vm}{\alpha_k},\vg)$, then we know from above that    $\alpha_{k+1}\bar{\vm}_{k+1}=\vm_{k+1}$ and $\bar{\vx}_{k+1}=\vx_{k+1}$, and therefore the conditional distribution for $(\vx_{k+1},\vm_{k+1},\vg_{k+1})$ is the same as that for $(\bar{\vx}_{k+1},\alpha_{k+1}\bar{\vm}_{k+1},\bar{\vg}_{k+1})$. Integration proves the claim for $k+1$, and thereby the induction is complete.
\end{proof}

Therefore, in the rest part of the appendix, we will stick to the formulation of SGDM as \Cref{equ:sgdm-2} unless specified otherwise.

\subsection{Descent Lemma for SGD and SGDM}

In \Cref{lem:descent}, given the SGD updates $\{z_k\}$ with constant learning rate $\eta$, we 
 have the gradient descent lemma 
\begin{align}\label{eq:gdl1}      &\E[\Loss(\vz_{k+1})|\vz_k] - \Loss(\vz_{k})  =\\
     &\qquad\underbrace{ - \eta \norm{\nabla \Loss(\vz_{k})}^2}_\text{descent force} + \underbrace{\frac{1}{2}(\sigma\eta)^2\tr((\nabla^2\Loss)\mSigma(\vz_k))}_\text{noise-induced} + \underbrace{\frac{1}{2}\eta^2(\nabla\Loss^\top(\nabla^2\Loss)\nabla\Loss(\vz_k))}_\text{curvature-induced}+o(\eta^2, (\sigma\eta)^2).
    \end{align}

We can get a similar result for the SGDM updates $\{x_k\}$, namely, when $1-\beta\ll\eta$,
\begin{align}\label{eq:gdl2}
        \E\Loss(\vx_{k+1}) & = \E\Loss(\vx_{k}) - \eta \E\norm{\nabla \Loss(\vx_{k})}^2 \\
        &+ \frac{1}{2}\eta^2\E(\nabla\Loss^\top(\nabla^2\Loss)\nabla\Loss(\vx_k)+\frac{1}{2} (\sigma\eta)^2\tr((\nabla^2\Loss)\mSigma(\vx_k)))\\
        & + \frac{\eta\beta}{1-\beta}(S_{k+1}-S_k) + \eta^2\E [\vm_k^\top\nabla^2\Loss\nabla\Loss(\vx_k)] + o(\eta^2/(1-\beta),(\sigma\eta)^2/(1-\beta)).
    \end{align}
    where $S_k = \E[\nabla\Loss(\vx_k)^\top \vm_k - \frac{\eta}{2}\vm_k^\top \nabla^2\Loss(\vx_k)\vm_k]$.
    Notice that beside the terms in the SGD updates, additions are an $O(\eta/(1-\beta))$-order telescoping term and an $O(\eta^2)$ extra impact term. The impact of the telescoping term is large for per-step updates but is limited across the whole trajectory, so the difference between SGDM and SGD trajectory is actually depicted by the extra term $\eta^2\E [\vm_k^\top\nabla^2\Loss\nabla\Loss(\vx_k)]$. In the settings where the extra term is bounded, we can bound the difference between training curves of SGDM and SGD.
\begin{proof}[Brief Proof for \cref{lem:descent}]
    We assume $\norm{\nabla^n\Loss}$ is bounded ($\norm{\nabla^n\Loss}_\infty<C_n$) for any order of derivatives $n=0,1,2,3\cdots$ and the trajectories are bounded ($\E\norm{\vz_k}^m,\E\norm{\vx_k}^m\leq C_m$) to simplify the reasoning. In the SGD case, $z_{k+1}-z_k=-\eta g_k$ has moments
    \begin{align*}
        \E\norm{z_{k+1}-z_k}^{2m} & = \eta^{2m} \E\norm{g_k}^{2m}\\
        & \leq \eta^{2m} 2^{2m} \E (\norm{\nabla\Loss(z_k)}^{2m} + \sigma^{2m}\norm{\vv_k}^{2m})\\
        & = O(\eta^{2m} + (\sigma\eta)^{2m}).
    \end{align*} 
    Similarly,
    \begin{align*}
        \E\norm{x_{k+1}-x_k}^{2m} & = \eta^{2m} \E\norm{m_{k+1}}^{2m}\\
        & = \eta^{2m} \E\norm{\beta^{k+1} m_0 + \sum_{s=0}^{k}\beta^{k-s}(1-\beta) g_{s}}^{2m}\\
        & \leq \eta^{2m}2^{2m} \E\left(\beta^{2m(k+1)} \norm{m_0
        }^{2m}+\left(\sum_{s=0}^{k}\beta^{k-s}\right)^{2m-1}\left(\sum_{s=0}^{k}\beta^{k-s}(1-\beta)^{2m}\norm{g_{s}}^{2m}\right)\right)\\
        & = O(\eta^{2m} + (\sigma\eta)^{2m}).
    \end{align*} 
    Therefore $\E\norm{z_{k+1}-z_k}^3\leq \sqrt{\E\norm{z_{k+1}-z_k}^6} = o(\eta^2,(\sigma\eta)^2)$ and $\E\norm{x_{k+1}-x_k}^3\leq \sqrt{\E\norm{x_{k+1}-x_k}^6} = o(\eta^2,(\sigma\eta)^2)$.
    Taylor expansion gives
 \begin{align*}
    \E\Loss(z_{k+1})|z_k & = \Loss(z_k) + \E\nabla\Loss(z_k)^\top (z_{k+1}-z_k) + \frac{1}{2} \E(z_{k+1}-z_k)^\top\nabla^2\Loss(z_k)(z_{k+1}-z_k)+o(\eta^2,(\sigma\eta)^2).
 \end{align*}
 Expansion yields \Cref{eq:gdl1}. Similarly,
  \begin{align*}
    \E\Loss(x_{k+1}) & = \E[\Loss(x_k) + \nabla\Loss(x_k)^\top (x_{k+1}-x_k) + \frac{1}{2} (x_{k+1}-x_k)^\top\nabla^2\Loss(x_k)(x_{k+1}-x_k)]+o(\eta^2,(\sigma\eta)^2)\\
    & = \E[\Loss(x_k) - \eta\nabla\Loss(x_k)^\top m_{k+1} + \frac{\eta^2}{2} m_{k+1}^\top\nabla^2\Loss(x_k)m_{k+1}]+o(\eta^2,(\sigma\eta)^2)
 \end{align*}
 Since $m_{k+1} = g_k + \frac{\beta}{1-\beta}(m_k-m_{k+1})$, 
   \begin{align*}
    \E\Loss(x_{k+1}) 
    & = \E[\Loss(x_k) - \eta\nabla\Loss(x_k)^\top g_k - \eta\nabla\Loss(x_k)^\top \frac{\beta}{1-\beta}(m_k-m_{k+1})\\
    & + \frac{\eta^2}{2} m_{k+1}^\top\nabla^2\Loss(x_k)(g_k + \frac{\beta}{1-\beta}(m_k-m_{k+1}))]+o(\eta^2,(\sigma\eta)^2)\\
    & = \E[\Loss(x_k) - \eta\nabla\Loss(x_k)^\top g_k - \frac{\eta\beta}{1-\beta}(\nabla\Loss(x_k)^\top m_k-\nabla\Loss(x_{k+1})^\top m_{k+1})\\
    & + \frac{\eta\beta}{1-\beta}(m_{k+1}^\top\nabla^2\Loss(x_k)(x_k-x_{k+1}) + O(\eta^2,\sigma^3\eta^2))\\
    & + \frac{\eta^2}{2} m_{k+1}^\top\nabla^2\Loss(x_k)(g_k + \frac{\beta}{1-\beta}(m_k-m_{k+1}))]+o(\eta^2,(\sigma\eta)^2)\\
    & = \E[\Loss(x_k) - \eta\nabla\Loss(x_k)^\top g_k - \frac{\eta\beta}{1-\beta}(\nabla\Loss(x_k)^\top m_k-\nabla\Loss(x_{k+1})^\top m_{k+1})\\
    & + \frac{\eta^2}{2} (g_k + \frac{\beta}{1-\beta}(m_k-m_{k+1}))^\top\nabla^2\Loss(x_k)(g_k + \frac{\beta}{1-\beta}(m_k+m_{k+1}))]+\frac{o(\eta^2,(\sigma\eta)^2)}{1-\beta}.
 \end{align*}
 Expansion yields \Cref{eq:gdl2}.
\end{proof}

\section{The dynamics of SGDM in $O(1/\eta)$ time}

In this section we will prove \Cref{thm:weak-appro-main} as the first part of our main result. First we shall take a closer examination at our update rule in \Cref{def:sgdm}. Let $\beta_{s:t}=\prod_{r=s}^t \beta_r$ be the product of consecutive momentum hyperparameters for $s\leq t$, and define $\beta_{s:t}=1$ for $s>t$, we get the following expansion forms.
\begin{lemma}\label{lem:unrollm}
    \begin{align*}
        \vm_k & =\beta_{0:k-1} \vm_0 + \sum_{s=0}^{k-1}\beta_{s+1:k-1}(1-\beta_{s}) \vg_{s}\\
        \vx_k & = \vx_0 - \sum_{s=0}^{k-1}\eta_s\beta_{0:s}\vm_0 - \sum_{s=0}^{k-1}\sum_{r=s}^{k-1}\eta_r\beta_{s+1:r}(1-\beta_{s})\vg_s\\
    \end{align*}
\end{lemma}
\begin{proof}
    By expansion of \Cref{dqu:sgdm-1}.
\end{proof}
%The update rule is
%\[\vx_{k}=\vx_{k-1}-\eta_k\beta_{1:k} \vm_0 -\eta_k \sum_{s=0}^{k-1}\beta_{s+2:k}(1-\beta_{s+1}) \vg_{s}.\]
%Or equivalently,

%\[\vx_{k}=\vx_{0}-\sum_{s=1}^k\eta_s\beta_{1:s} \vm_0 -\sum_{s=0}^{k-1}\sum_{t=s+1}^k\eta_t \beta_{s+2:t}(1-\beta_{s+1}) \vg_{s}.\]
Let the coefficients $c_{s,k}=\sum_{r=s}^{k-1}\eta_r\beta_{s+1:r}(1-\beta_{s})$ so that $\vx_k=\vx_0 - \frac{\beta_0 c_{0:k}}{1-\beta_0}\vm_0 -\sum_{s=0}^{k-1} c_{s,k}\vg_s$. Notice that $c_{s,k}$ is increasing in $k$ but it is always upper-bounded as $c_{s,k}\leq c_{s,\infty}=\sum_{r=s}^{\infty}\eta_r\beta_{s+1:r}(1-\beta_{s})\leq (\sup_{t}\eta_t)\sum_{r=s}^{\infty}(\sup_t\beta_t)^{r-s}=\frac{\sup_{r}\eta_r}{1-\sup_t\beta_t}$. Furthermore, if for any $t$, $\beta_t$ and $\eta_t$ are about the same scale, then we should expect that the difference $c_{s,\infty}-c_{s,k}$ is exponentially small in $k$. Then we can hypothesize that the trajectory $\vx_k$ should be close to a trajectory of the following form:
\[\tilde{\vx}_k=\tilde{\vx}_0 - \frac{\beta_0 c_{0:\infty}}{1-\beta_0}\vm_0 -\sum_{s=0}^{k-1} c_{s,\infty}\tilde{\vg}_s\]
which is exactly a SGD trajectory given the learning rate schedule $c_{s,\infty}$.

To formalize the above thought, we define the averaged learning rate schedule
\begin{equation}\label{equ:averaged-lr}
    \bar{\eta}_k= c_{k,\infty}=\sum_{s=k}^\infty \eta_s  \beta_{k+1:s}(1-\beta_{k})
\end{equation}
And the coupled trajectory
\begin{align}\label{equ:coupled-traj}
    \vy_{k} & =\vx_{k}-\frac{\bar{\eta}_k\beta_k}{1-\beta_k}\vm_k\\
    & = \vx_0 - \frac{\beta_0 \bar{\eta}_0}{1-\beta_0}\vm_0 -\sum_{s=0}^{k-1} \bar{\eta}_s\vg_s.
\end{align}
Then there is the following transition:
\begin{align}
    \vy_{k} =\vy_{k-1}-\bar{\eta}_{k-1}\vg_{k-1}
\end{align}
$\vy_k$ has an interesting geometrical interpretation as the endpoint of SGDM if we cut-off all the gradient signal from step $k$. E.g., $\vx_\infty$ if we set $\vg_{k+1}=\vg_{k+2}=\cdots = 0$ for and update the SGDM from ($\vx_k$,$\vm_k$).
%Then \[\vy_k =\vx_{k}-\frac{\bareta_{k+1}}{1-\beta_{k+1}}(\beta_{1:k+1} \vm_0 +\sum_{s=0}^{k-1}\beta_{s+2:k+1}(1-\beta_{s+1})\vz_{s}). \]
%And
%\[\vy_{k} =\vy_{k-1}-\bareta_{k}g_{k-1}?. \]

\iffalse
\begin{definition}[Order-1 Approximation]
Fix any initialization $\vx^\eta_0=\vy^\eta_0=\vx$, $\vx_t^\eta$ and $\vy_t^\eta$ are order-$c$ approximation of one another in time $T$ when $\eta\to 0$, if for any time $0\leq t\leq T$ and any function $g$ of polynomial growth, there is constant $K(\vx,t)$ independent of $\eta$ such that
\[\E |g(\vx_t^\eta)-g(\vy_t^\eta)|\le K(\vx,t)\eta.\]
\end{definition}
\begin{lemma}
$\vy_k$  is an order-1 approximation of $\vx_k$.
\end{lemma}
\begin{lemma}
$\vy_k$  is an order-1 approximation of $\barvy_k$, where $\barvy_k$ is updating through iterating
\[\barvy_{k+1}=\barvy_{k} -\eta(\nabla \Loss(\barvy_k) + \sqrt{\Xi}\sigma_{\vxi_k}(\barvy_k))\]
\end{lemma}
\begin{lemma}
$\barvy_k$  is an order-1 approximation of $\bbarvy_{k\eta}$, where $\bbarvy_{t}$ is given by SDE
\[\dd\bbarvy_t=\nabla \Loss(\bbarvy_t)\dd t + \sqrt{\eta}\sigma(\bbarvy_t)\dd W_t\]
\end{lemma}
\begin{theorem}
    SGDM $\vx_k$ and SGD $\barvx_k$ are order-$(1-\alpha)$ weak approximations of each other in $k=O(1/\eta)$ time.
\end{theorem}

\begin{theorem}
    $\vx_k$  is an order-1 approximation of $\bvx_{k\eta}$, where $\bvx_{t}$ is given by SDE
\[\dd\bvx_t=\nabla \Loss(\bvx_t)\dd t + \sqrt{\eta}\sigma(\bvx_t)\dd W_t\]
\end{theorem}
\fi
\subsection{$\alpha\in [0,1)$}
In this section we provide the proof for \Cref{thm:weak-appro-main}. Specifically, when $\alpha\in[0,1)$ is the index for the control of the learning rate schedule (\Cref{def:hpschedule}), we hope to show that $\vy_k$ is close to a SGD trajectory $\vz_k$, defined by the following for $k\geq 0$:
\begin{align}
    &\vz_0 = \vx_0\\
    &\vh_{k} \sim \cG_\sigma(\vz_{k})\\
    &\vz_{k+1} =\vz_{k}-\bar{\eta}_{k}\vh_{k}
\end{align}

%Assume $\eta$ and $\beta$ are not time-varying with $\beta=1-\lambda\eta^\alpha$. Let $\vy_k =  \vx_k - \frac{\eta\beta}{1 - \beta}\vm_{k}$, we have that 
%\begin{align*}
 %   \vy_{k} = \vy_{k-1} - \eta (\nabla \Loss(\vx_{k-1})+  \vsigma_{\vxi_{k-1}}(\vx_{k-1}))
%\end{align*}
%\begin{align*}
 %   \vy_{k} = \vy_{k-1} -  (\eta\nabla \Loss(\vx_{k-1})+  \sqrt{\eta}\vsigma_{\vxi_{k-1}}(\vx_{k-1}))
%\end{align*}
%Also define 
%\[\barvy_{k}=\barvy_{k-1} -\eta(\nabla \Loss(\barvy_{k-1}) + \vsigma_{\vxi_{k-1}}(\barvy_{k-1})), \quad \barvy_0=\vy_0\]

Specifically, we use $\vy_k$ as a bridge for connecting $\vx_k$ and $\vz_k$. Our proof consists of two steps.
\begin{enumerate}
    %\item momentum SGD $\vx_k,\vy_k$ is a order-1 weak approximation for SGD: $\barvy_k$;
    \item We show that $\vx_k$ and $\vy_k$ are order-$(1-\alpha)/2$ weak approximations of each other.  
    \item We show that $\vz_k$ and $\vy_k$ are order-$(1-\alpha)/2$ weak approximations of each other. 
\end{enumerate}
Then we can conclude that $\vx_k$ and $\vz_k$ are order-$(1-\alpha)/2$ weak approximations of each other, which states our main theorem \Cref{thm:weak-appro-main}.

%\begin{definition}[Coupled Trajectory]
%	Given $\{\vx_k\}_{k=0}^K$, define $\{\vy_k\}_{k=0}^K$ such that $\vy_0=\vx_0$ and $\vy_i$ is written as
%	\begin{equation}
%		\vy_k = \vx_k - \frac{\eta\beta}{1-\beta}\vm_k = \vy_{k-1} - \eta(\nabla\Loss(\vx_{k-1}) + \sigma_{\xi_{k-1}}(\vx_{k-1}))
%	\end{equation}
%	\label{def:coupled_trajectory}
%\end{definition}

First, we give a control of the averaged learning rate $\bar{\eta}_k$ (\Cref{equ:averaged-lr}) with the following lemma.
\begin{lemma}
    Let $(\eta_k,\beta_k)$ be a learning rate schedule scaled by $\eta$ (\Cref{def:hpschedule}) with index $\alpha$, and $\bar{\eta}_k$ be the averaged learning rate (\Cref{equ:averaged-lr}), there is 
    \begin{align*}
        & \bar{\eta}_k\leq \frac{\lambda_{\max}\eta_{\max}}{\lambda_{\min}}\eta\\
        & \frac{\bar{\eta}_k\beta_k}{1-\beta_k}\leq \frac{\lambda_{\max}\eta_{\max}}{\lambda^2_{\min}}\eta^{1-\alpha}.\\
    \end{align*}
\end{lemma}
\begin{proof}
    By \Cref{def:hpschedule} we know $\beta_k \in [1-\lambda_{\max}\eta^\alpha,1-\lambda_{\min}\eta^\alpha]$ and $\eta_k\leq \eta_{\max}\eta$. Therefore
    \begin{align*}
      \bar{\eta}_k&=\sum_{s=k}^\infty \eta_s  \beta_{k+1:s}(1-\beta_{k})\\
      & \leq \sum_{s=k}^\infty \eta_{\max}\eta(1-\lambda_{\min}\eta^\alpha)^{s-k} \lambda_{\max}\eta^\alpha\\
      & = \frac{\lambda_{\max}\eta_{\max}}{\lambda_{\min}}\eta.\\
      \frac{\bar{\eta}_k\beta_k}{1-\beta_k} & \leq \frac{\lambda_{\max}\eta_{\max}}{\lambda_{\min}}\eta\cdot \frac{1}{1-\beta_k}\\
      & \leq \frac{\lambda_{\max}\eta_{\max}}{\lambda^2_{\min}}\eta^{1-\alpha}.
    \end{align*}
\end{proof}
\subsubsection{Step 1}
For the first step we are going to proof the following \Cref{thm:weakappro-1}. For notation simplicity we omit the superscript $\eta$ for the scaling when there is no ambiguity.
\begin{theorem}[Weak Approximation of Coupled Trajectory]\label{thm:weakappro-1}
    With the learning rate schedule $(\eta_k,\beta_k)$ scaled by $\eta$ with index $\alpha\in [0,1)$. Assume that the NGOS $\cG_\sigma$ satisfy \Cref{assume:ngos} and the initialization $\vm_0$ satisfy \Cref{ass:init-bound}. For any noise scale $\sigma\leq \eta^{-1/2}$, let $(\vx_k,\vm_k)$ be the SGDM trajectory and $\vy_k$ be the corresponding coupled trajectory (\Cref{equ:coupled-traj}). 
	%Assume:
	%\begin{enumerate}
	%	\item $L$-Lipschitz Gradient: $\|\nabla\Loss(\vx_s)-\nabla\Loss(\vx_k)\| \leq L\|\vx_s - \vx_k\|$ for all time steps $s$
  	%	\item Bounded trace of covariance: $\tr(\mSigma(\vx_s))\leq C$ for all time steps $s$
	%\end{enumerate}
	Then, the coupled trajectory $\vy_k$ is an order-$\gamma$ weak approximation (\Cref{def:weak_approx}) to $\vx_k$ with $\gamma = (1-\alpha)/2$.
	\label{thm:coupled_weak_approx}
\end{theorem}

For notations, let $\nabla_k = \nabla\Loss(\vx_k)$ so $\vg_k = \nabla_k + \sigma\vv_k$ is the stochastic gradient sampled from the NGOS. Also by \Cref{assume:ngos}, as $\mSigma^{1/2}$ is bounded, let $C_{\tr}>0$ be the constant that $\tr\mSigma(\vx)\leq C_{\tr}$ for all $\vx\in\R^d$. As $\nabla\Loss$ is Lipschitz, let $L>0$ be the constant that $\norm{\nabla\Loss(\vx)-\nabla\Loss(\vy)}\leq L\norm{\vx-\vy}$ for all $\vx,\vy\in\R^d$.  As $\mSigma^{1/2}$ is Lipschitz and bounded, let $L_{\Sigma}>0$ be the constant that $|\tr{\mSigma(\vx)-\mSigma(\vy)}|\leq L_{\Sigma}\norm{\vx-\vy}$ for all $\vx,\vy\in\R^d$

To show the result it suffices to bound $\E\norm{\vx_k}^{2m}$ and $\E\norm{\vm_k}$ for any $m\geq 0$ and $k=O(\eta^{-1})$. As the gradient noises are scaled with variance $O(\sigma^2) = O(\eta^{-1})$ so that may dominate the expansion \Cref{lem:unrollm}. We will show that $\E\norm{\vm_k}=O(\eta^{(\alpha-1)/2})$ is the correct scale so we still have $\E\norm{\vx_k}^{2m}=O(1)$. An useful inequality we will use often in our proof is the Gr\"{o}nwall's inequality
\begin{lemma}[Gr\"{o}nwall's Inequality]\label{lem:gronwall-discrete}
    For non-negative sequences $\{f_i,g_i,k_i\in\R\}_{i\geq 0}$, if for all $t>0$
    \[f_t\leq g_t + \sum_{s=0}^{t-1} k_s f_s\]
    then $f_t\leq g_t + \sum_{s=0}^{t-1} k_sg_s \exp(\sum_{r=s+1}^{t-1} k_r)$.
\end{lemma}

Next follows the proofs.

\begin{lemma} 
	We can bound the expected norm of \begin{align}\E\norm{\vm_k}^2  \leq 12\frac{\lambda^2_{\max}}{\lambda^2_{\min}}\sup_{\tau=1,...,k}\E(\|\nabla_\tau\|)^2 + (4+8\frac{L\eta_{\max}}{\lambda_{\min }}+\eta^{\alpha-1})K_0\label{equ:expe-norm-m}\end{align}
	\label{lem:exp_norm_m}
    with constant $K_0 = \frac{2\lambda^2_{\max}}{\lambda_{\min}}C_{\tr} + 3\E \norm{\vm_0}^2=\frac{2\lambda^2_{\max}}{\lambda_{\min}}C_{\tr}+3C_2$ ($C_2$ by \Cref{ass:init-bound}), for a small enough learning rate $\eta<\left(\frac{\lambda_{\min }}{4L\eta_{\max}}\right)^{\frac{1}{1-\alpha}}$. 
\end{lemma}

\begin{proof}

To bound $\E\norm{\vm_k}^2$, we unroll the momentum by \Cref{lem:unrollm} and write

\begin{align}
	 \vm_k & =\beta_{0:k-1} \vm_0 + \sum_{s=0}^{k-1}\beta_{s+1:k-1}(1-\beta_{s}) \vg_{s}\label{eq:unrolled_momentum} 
\end{align}

Define
\begin{align*}
	\tilde{\vm}_k &= \beta_{0:k-1} \vm_0 + \sum_{s=0}^{k-1}\beta_{s+1:k-1}(1-\beta_{s})\nabla_s.\label{eq:unrolled_momentum-2} 
\end{align*}
Then $\E \vm_k = \E \tilde{\vm}_k$, and
\begin{align*}
	\vm_k-\tilde{\vm}_k &= \sum_{s=0}^{k-1}\beta_{s+1:k-1}(1-\beta_{s})\sigma\vv_{s}
\end{align*}


Let constant $K_0 =2C_{\tr}+3C_2$, we can write
\begin{align}
    \E\norm{\vm_k}^2&\leq \E(\norm{\tilde{\vm}_k}+\norm{\vm_k-\tilde{\vm}_k})^2\\
    & \leq 2\E\norm{\tilde{\vm}_k}^2+ 2\E\norm{\vm_k-\tilde{\vm}_k}^2\\
    & = 2\E \norm{\tilde{\vm}_k}^2+ 2 \sum_{s=0}^{k-1}(\beta_{s+1:k-1})^2(1-\beta_{s})^2 \sigma^2\E \tr\mSigma(\vx_s)\\
    & \leq 2\E \norm{\tilde{\vm}_k}^2+ 2\sum_{s=0}^{k-1} (1-\lambda_{\min}\eta^\alpha)^{2(k-s-1)}\lambda^2_{\max}\eta^{2\alpha} \sigma^2\E \tr\mSigma(\vx_s)\\
    & \leq 2\E \norm{\tilde{\vm}_k}^2+ \eta^{\alpha-1}\frac{2\lambda^2_{\max}}{\lambda_{\min}}C_{\tr}\\
    & \leq 2\E \norm{\tilde{\vm}_k}^2+ \eta^{\alpha-1}K_0.
    \label{eq:exp_mnorm_sq}
\end{align}
\iffalse
Hence, 
\begin{align}
	\E\norm{\vm_k} &\leq \sqrt{\E\norm{\vm_k}^2}\\
		& \leq \sqrt{\norm{\E\vm_k}^2 + \frac{1}{\eta}\frac{1-\beta}{1+\beta}C}\\
		& \leq \norm{\E\vm_k}  +\sqrt{\frac{1}{\eta}\frac{1-\beta}{1+\beta}C}
\end{align}
\fi

On the other hand, we know
\begin{align*}
    \tilde{\vm}_k &= \beta_{0:k-1} \vm_0 + \sum_{s=0}^{k-1}\beta_{s+1:k-1}(1-\beta_{s})\nabla_s
\end{align*}
We use the Lipschitzness of $\nabla$ to write 
\begin{align*}
	\|\nabla_s - \nabla_k\| & \leq L\|\vx_s - \vx_k\|\leq L\sum_{\tau=s}^{k-1}\eta_\tau\norm{\vm_{\tau+1}},  	
\end{align*}
then we can write
\begin{align*}
 \E \norm{\tilde{\vm}_k}^2 & \leq \E\left(  \beta_{0:k-1} \norm{\vm_0} + \sum_{s=0}^{k-1}\beta_{s+1:k-1}(1-\beta_{s})(\|\nabla_k\| + L\sum_{\tau=s}^{k-1}\eta_\tau\norm{\vm_{\tau+1}})\right)^2 \\
 &  \leq \E\left( \norm{\vm_0} +  \frac{\lambda_{\max}}{\lambda_{\min}}\|\nabla_k\| + L\eta\eta_{\max}\sum_{\tau=0}^{k-1}(1-\lambda_{\min}\eta^{\alpha})^{k-1-\tau}
 \|\vm_{\tau+1}\|\right)^2\\
 & \leq  3\E (\norm{\vm_0})^2 +  3\frac{\lambda^2_{\max}}{\lambda^2_{\min}}\E(\|\nabla_k\|)^2 + 3L^2\eta^2\eta_{\max}^2\left(\sum_{\tau=0}^{k-1}(1-\lambda_{\min}\eta^{\alpha})^{k-1-\tau}\right)\cdot\\&\qquad\left(\sum_{\tau=0}^{k-1}(1-\lambda_{\min}\eta^{\alpha})^{k-1-\tau}\E\|\vm_{\tau+1}\|^2\right)\\
 & \leq K_0+3\frac{\lambda^2_{\max}}{\lambda^2_{\min}}\E(\|\nabla_k\|)^2 + \frac{3L^2\eta_{\max}^2\eta^{2-2\alpha}}{\lambda_{\min }^2}\sup_{\tau=1\cdots {k}}\E\|\vm_\tau\|^2.
\end{align*}
Then by \Cref{eq:exp_mnorm_sq}, we know 
\[\sup_{\tau=1,...,k} \E\|\vm_\tau\|^2\leq2\sup_{\tau=1,...,k}\E \norm{\tilde{\vm}_\tau}^2+ \eta^{\alpha-1}K_0,\] so
\begin{align}
\sup_{\tau=1,...,k} \E\norm{\tilde{\vm}_\tau}^2 & \leq K_0+3\frac{\lambda^2_{\max}}{\lambda^2_{\min}}\sup_{\tau=1,...,k}\E(\|\nabla_\tau\|)^2 + \frac{6L^2\eta_{\max}^2\eta^{1-\alpha}}{\lambda_{\min }^2}K_0\\
&\qquad + \frac{6L^2\eta_{\max}^2\eta^{2-2\alpha}}{\lambda_{\min }^2}\sup_{\tau=1,\cdots, {k}}\E\|\tilde{\vm}_\tau\|^2
\end{align}
Choose $\eta$ small enough so that $\frac{6L^2\eta_{\max}^2\eta^{2-2\alpha}}{\lambda_{\min }^2}<\frac{1}{2}$, we know
\begin{align}
\sup_{\tau=1,...,k} \E\norm{\tilde{\vm}_\tau}^2 & \leq 6\frac{\lambda^2_{\max}}{\lambda^2_{\min}}\sup_{\tau=1,...,k}\E(\|\nabla_\tau\|)^2 + (2+2\frac{\sqrt{3}L\eta_{\max}}{\lambda_{\min }})K_0.
\end{align}
So
\begin{align}
    \E\norm{\vm_k}^2 & \leq 2\E \norm{\tilde{\vm}_k}^2+ \eta^{\alpha-1}K_0 \\
    & \leq 12\frac{\lambda^2_{\max}}{\lambda^2_{\min}}\sup_{\tau=1,...,k}\E(\|\nabla_\tau\|)^2 + (4+8\frac{L\eta_{\max}}{\lambda_{\min }}+\eta^{\alpha-1})K_0.
\end{align}
\end{proof}

\iffalse
\begin{lemma}
	For $\eta$ small enough, we can bound 
	$$\sup_{\tau=0,...,k} \norm{\E\vm_k}  \leq 2\|\vm_0\| + 2\sup_{\tau=0,...,k} \norm{\E\nabla_\tau} + \sqrt{2LC}.$$\label{lem:norm_exp_m}
\end{lemma}

\begin{proof}From \ref{eq:unrolled_momentum} we know
\begin{align*}
    \E\vm_k &= \beta^k\vm_0 + \sum_{s=0}^{k-1} \beta^{k-s-1}(1-\beta)\nabla_s  \label{eq:unrolled_momentum2}
\end{align*}
%We replace $\nabla_{s,i}$ with $\nabla_{s,i} - \E \nabla_{k,i} + \E \nabla_{k,i}$. 
We use the Lipschitzness of $\nabla$ to write 
\begin{align*}
	\|\nabla_s - \nabla_k\| &\leq L\eta(k-s)\sup_{\tau=s,...,k-1}\|\vm_\tau\|  	
\end{align*}
If we substitute this into \Cref{eq:unrolled_momentum2}, then we can write
$$ \E\vm_k = \beta^k\vm_0 + \sum_{s=0}^{k-1} \beta^{k-s-1}(1-\beta)(\|\nabla_k\| + L\eta(k-s)\sup_{\tau=s,...,k-1} \|\vm_\tau\|) $$
Then, we can bound the coefficients of the summation:
$$ \sum_{s=0}^{k-1} \beta^{k-s-1} (1-\beta)\eta(k-s) \leq \frac{\eta}{1-\beta} = \lambda^{-1}\eta^{1-\alpha} $$
\begin{align}
\norm{\E\vm_k} &\leq \|\E\vm_0\| + \sum_{s=0}^{k-1} \beta^{k-s-1}(1-\beta)\norm{\E\nabla_k}+ \sum_{s=0}^{k-1} \beta^{k-s-1} (1-\beta) L \eta (k-s) \sup_{\tau=s,...,k-1} \E\|\vm_\tau\|	\\
& \leq \|\vm_0\| + \norm{\E\nabla_k} + \frac{L\eta}{1-\beta}\sup_{\tau=0,...,k-1} \E\|\vm_\tau\|
\end{align}
Then by \ref{lem:exp_norm_m}, we know $ \E\|\vm_k\| \leq \|\E \vm_k\| + \sqrt{\frac 1\eta \frac{1-\beta}{1+\beta} C}$, so
\begin{align}
\sup_{\tau=0,...,k} \norm{\E\vm_k} & \leq \|\vm_0\| + \sup_{\tau=0,...,k} \norm{\E\nabla_\tau} + \frac{L\eta}{1-\beta}\sup_{\tau=0,...,k} \|\E\vm_\tau\| + L\sqrt{\frac{\eta}{1-\beta^2}C}
\end{align}
Choose $\eta$ so that $\frac{L\eta}{1-\beta}=\frac{L}{\lambda}\eta^{1-\alpha}<\frac{1}{2}$, we know
\begin{align}
\sup_{\tau=0,...,k} \norm{\E\vm_k} & \leq 2\|\vm_0\| + 2\sup_{\tau=0,...,k} \norm{\E\nabla_\tau} + \sqrt{2LC}.
\end{align}
\end{proof}
\fi

\begin{lemma}
There is a function $f$, irrelevant to $\eta$, of polynomial growth in $\vm_0$ and $\vx_0$,   such that 
\[\sup_{k=0,...,\lfloor T/\eta \rfloor} \E\norm{\nabla_k}^2 \leq f(\vm_0,\vx_0,T)\]\label{lem:gradient_norm}
when $\eta<\min(\left(\frac{\lambda_{\min}^3}{12L\lambda_{\max}^2\eta_{\max}}\right)^{\frac{1}{1-\alpha}},1)$
\end{lemma} 
\begin{proof}
By the Lipschitzness of $\nabla\Loss$, we know
\begin{align}
\|\nabla_k\| & \leq \|\nabla_0\| + L\norm{\vx_{k}-\vx_0}\\
& \leq \|\nabla_0\| + L\norm{\vy_{k}-\vy_0} + L\norm{\vx_{0}-\vy_0} + L\norm{\vx_{k}-\vy_k} \label{eq:normnablak}
\end{align}
Observe that $\|\vx_k - \vy_k\| = \frac{\bar{\eta}_k\beta_k}{1-\beta_k}\|\vm_k\|\leq \frac{\lambda_{\max}\eta_{\max}}{\lambda_{\min}^2}\eta^{1-\alpha}\|\vm_k\|$. Let $d_k=\E\|\vy_k-\vy_0\|^2$. As a result, we can write
\begin{align}
\E\|\nabla_k\|^2 & \leq 3\E\left(\|\nabla_0\|  +\frac{L\lambda_{\max}\eta_{\max}}{\lambda_{\min}^2}\eta^{1-\alpha}\|\vm_0\|\right)^2+3\left(\frac{L\lambda_{\max}\eta_{\max}}{\lambda_{\min}^2}\eta^{1-\alpha}\right)^2\E(\|\vm_k\|^2) + 3L^2 d_k.\label{equ:grad-bound-1}
\end{align}
Choose $\eta$ such that $\eta^{1-\alpha}<\frac{\lambda_{\min}^3}{12L\lambda_{\max}^2\eta_{\max}}$ and let the function \begin{align}
    K_1(\vm_0,\vx_0) & =3\E\left(\|\nabla\Loss(\vx_0)\|  +\frac{\lambda_{\min}}{12\lambda_{\max}}\|\vm_0\|\right)^2 \\ &\qquad + \left(\frac{\lambda_{\min}}{6\lambda_{\max}}\right)^2(3+6\frac{L\eta_{\max}}{\lambda_{\min }})K_0+\left(\frac{L\eta_{\max}}{4\lambda_{\min}}\right)K_0
\end{align}
where recall $K_0$ is the constant from \Cref{lem:exp_norm_m}. Plug \Cref{equ:expe-norm-m} into \Cref{equ:grad-bound-1} gives
\begin{align}
\E\|\nabla_k\|^2 & \leq K_1 +\frac{1}{2}\sup_{\tau=1,...,k}\E(\|\nabla_\tau\|)^2+ 3L^2 d_k.
\end{align}
therefore 
\begin{align}\label{equ:lya-1}\E\|\nabla_k\|^2  \leq 2K_1 + 6L^2\sup_{\tau=1,...,k}  d_\tau.
\end{align}
Now we need to bound $d_k$ by iteration.
\begin{align}
d_{k+1} &= d_{k} +\E\|\vy_{k+1}-\vy_k\|^2 + 2\E(\vy_{k+1}-\vy_k)^\top(\vy_k-\vy_0)\\
&= d_k + \bar{\eta}_k^2 \E\norm{\nabla_k}^2 + \bar{\eta}_k^2\sigma^2 \E\tr(\mSigma(\vx_k)) + 2 \E(\nabla_k)^\top(\vy_k-\vy_0)\\
&\leq d_k + \frac{\lambda^2_{\max}}{\lambda^2_{\min}}\eta_{\max}^2\eta^2 \E\norm{\nabla_k}^2 + \frac{\lambda^2_{\max}}{\lambda^2_{\min}}\eta_{\max}^2\eta C_{\tr} + 2 \frac{\lambda_{\max}}{\lambda_{\min}}\eta_{\max}\eta\sqrt{d_k\E\norm{\nabla_k}^2}
\end{align}
Let function $K_2(\vm_0,\vx_0) = \frac{\lambda^2_{\max}}{\lambda^2_{\min}}\eta_{\max}^2 C_{\tr} +2\frac{\lambda^2_{\max}}{\lambda^2_{\min}}\eta_{\max}^2K_1(\vm_0,\vx_0) +2 \frac{\lambda_{\max}}{\lambda_{\min}}\eta_{\max}\frac{K_1(\vm_0,\vx_0)}{\sqrt{6}L}$, and $\bar{d}_k=\sup_{\tau=1,...,k} d_{\tau}$. We know
\begin{align}
\sqrt{d_k\E\norm{\nabla_k}^2} & \leq \sqrt{6L^2}d_k + \frac{K_1}{\sqrt{6L^2}},\\
\bar{d}_{k+1} &\leq \bar{d}_k + K_2\eta + 6L^2\frac{\lambda^2_{\max}}{\lambda^2_{\min}}\eta_{\max}^2\eta^2 \bar{d}_k +  2 \frac{\lambda_{\max}}{\lambda_{\min}}\eta_{\max}\eta(\sqrt{6}L\bar{d}_k)\\
& \leq (1+6L \frac{\lambda_{\max}\eta_{\max}}{\lambda_{\min}}\eta+6L^2\frac{\lambda^2_{\max}\eta_{\max}^2}{\lambda^2_{\min}}\eta^2)\bar{d}_k + \eta K_2
\end{align}
 Let $\kappa = 1+6L \frac{\lambda_{\max}\eta_{\max}}{\lambda_{\min}}\eta+6L^2\frac{\lambda^2_{\max}\eta_{\max}^2}{\lambda^2_{\min}}\eta^2$, then by Gr\"{o}nwall's inequality \Cref{lem:gronwall-1}, 
\begin{align}
\bar{d}_k   
    & \leq \eta K_2 (1 + \sum_{s=0}^{k-1} \kappa \exp^{\kappa s})\\
    & \leq K_2 + K_2 T \exp^{1+6L \frac{\lambda_{\max}\eta_{\max}}{\lambda_{\min}}T+6L^2\frac{\lambda^2_{\max}\eta_{\max}^2}{\lambda^2_{\min}}T}.
\end{align}
Plugging into \Cref{equ:lya-1} finished the proof.
\end{proof}
\begin{lemma}
    There is a function $f$, irrelevant to $\eta$, of polynomial growth in $\vm_0$ and $\vx_0$,   such that 
    \[\E\norm{\vm_k}^2  \leq \eta^{\alpha-1} f(\vm_0,\vx_0,T)\]
    for all $k\leq \frac{T}{\eta}$.\label{lem:boun-mome}
\end{lemma}
\begin{proof}
    The result directly follows from \Cref{lem:exp_norm_m} and \Cref{lem:gradient_norm}.
\end{proof}
\begin{lemma}
    %Assume that conditions 1 and 2 of \Cref{thm:coupled_weak_approx} hold.
    %Then 
    There exists function $h$ that, for all $k\leq \frac{T}{\eta}$,
    \begin{align*} 
    	\E[(1+\|\vx_k\|^2)^{m}] &\leq h(\vm_0, \vx_0,m,T), \\
    	\E[(1+\|\vy_k\|^2)^{m}] &\leq h(\vm_0, \vx_0,m,T), \\
    	\E[\eta^{m}\|\vg_k\|^{2m}] &\leq h(\vm_0, \vx_0,m,T),
    \end{align*}
	and that $h$ is irrelevant to $k$ and $\eta$. 
	%\snote{Kaifeng: can you take a look at this? The constant depends on $\|\vm_0\|$ too.}
	\label{lem:small_higher_order}
\end{lemma}

\begin{proof}

    We use the fact that $(a+b)^m \leq 2^{m-1}(a^m + b^m)$ from the Jensen inequality for $a,b>0$ and $m\geq 1$. Furthermore by Young's inequality $a^\alpha b^{\beta}\leq \frac{\alpha a^{\alpha+\beta}+ \beta b^{\alpha+\beta}}{\alpha+\beta}\leq a^{\alpha+\beta}+b^{\alpha+\beta}$ for $\alpha,\beta>0$.
	%First, we show that $\E[\eta^{m}\|\vg_k\|_2^{2m}] \leq f_3(\vm_0, \vx_0,m)(1 + \|\vx_k\|)^{2m}$. 
	%We can bound $G^{2m}$.
	\begin{align}
		\norm{\vg_k}^{2m} &\leq (\|\nabla_k\| + \sigma\|\vv_k\|)^{2m} \\
        & \leq 2^{2m-1} ( \|\nabla_k\|^{2m} + \eta^{-m}\|\vv_k\|^{2m})\\ 
		\|\nabla_k\|^{2m} &\leq (\|\nabla_0\| + L\|\vx_k - \vx_0\|)^{2m} \\
		&\leq 2^{2m-1}((\|\nabla_0\| + L\|\vx_0\|)^{2m} + L^{2m}\|\vx_k\|^{2m})
  \label{moments-bound-1}
	\end{align}
    %Moveover we again define the sequence
    %$\tilde{\vm}_k = \beta_{0:k-1} \vm_0 + \sum_{s=0}^{k-1}\beta_{s+1:k-1}(1-\beta_{s})\nabla_s$ and $\hat{\vm}_k = \vm_k-\tilde{\vm_k}$ which are the momentum counting the two components of the gradients, then there is
    %\[\tilde{\vm}_{k+1} = \beta_{k}\tilde{\vm}_{k} + (1-\beta_{k})\nabla_{k},\]
    %\[\hat{\vm}_{k+1} = \beta_{k}\hat{\vm}_{k} + (1-\beta_{k})\sigma\vv_k.\]
	The term $\E\norm{\vv_k}^{2m}$ is bounded by \Cref{assume:ngos}. Now we need to show the bounds on $\E[(1+\|\vx_k\|^2)^{m}] $ and $\E[(1+\|\vy_k\|^2)^{m}] $. Specifically we shall prove $\E[(1+\|\vy_k\|^2+\eta^{2-2\alpha}\|\vm_k\|^2)^{m}] \leq h(\vm_0, \vx_0,m,T)$. 

    Let 
    \begin{align*}
        \delta_k & = \bar{\eta}_k^2 \norm{\vg_{k}}^2 - 2\bar{\eta}_k \dotp{\vg_{k}}{\vy_{k}} 
         +\eta^{2-2\alpha}(\norm{\beta_k\vm_k+(1-\beta_k)\vg_k}^2-\norm{\vm_k}^2),
    \end{align*}
    then there is
	\begin{align*}	&\qquad\E[(1+\|\vy_{k+1}\|^2+\eta^{2-2\alpha}\|\vm_{k+1}\|^2)^m|\vy_{k},\vm_k]\\
   & = \E [(1+\|\vy_{k} - \bar{\eta}_k\vg_{k}\|^2+
   \eta^{2-2\alpha}\norm{\beta_k\vm_k+(1-\beta_k)\vg_k}^2)^m|\vy_{k},\vm_k]\\
        & = \E [(1+\|\vy_{k}\|^2+\eta^{2-2\alpha}\|\vm_k\|^2+\delta_k)^m|\vy_{k},\vm_k]\\
        &\leq (1+\norm{\vy_k}^2+\eta^{2-2\alpha}\|\vm_k\|^2)^m + m(1+\norm{\vy_k}^2+\eta^{2-2\alpha}\|\vm_k\|^2)^{m-1} \E[\delta_k|\vy_k,\vm_k] \\&+  \E[\delta_k^2\sum_{i=0}^{m-2} \binom{m}{i+2}\delta_k^{i}(1+\norm{\vy_k}^2+\eta^{2-2\alpha}\|\vm_k\|^2)^{m-2-i}|\vy_{k},\vm_k]\\
        &\leq (1+\norm{\vy_k}^2+\eta^{2-2\alpha}\|\vm_k\|^2)^m + m(1+\norm{\vy_k}^2+\eta^{2-2\alpha}\|\vm_k\|^2)^{m-1} \E[\delta_k|\vy_k,\vm_k] \\& +  2^m\E[\delta_k^2 
        (|\delta_k|^{m-2}+(1+\norm{\vy_k}^2+\eta^{2-2\alpha}\|\vm_k\|^2)^{m-2})|\vy_{k},\vm_k].
        \end{align*}
	%$\|\vy_{k-1}\|^{2m}$ is small by the inductive hypothesis. Using the assumption, we can write
	%\begin{align*}
	%	\|\vg_{k-1}\|^{2m} \leq f_3(\vm_0,\vx_0,m)(1+\|\vx_{k-1}\|^{2m})
	%\end{align*}
    %We know	from \Cref{lem:gradient_norm} that $\sup_{k=0,...,\lfloor T/\eta \rfloor} \E\norm{\nabla_k}^2 \leq f(\vm_0,\vx_0,T)$ for some function $f$. 
    Then there exists constant $K_3$ such that
    \begin{align*}	
        \E[\delta_k|\vy_k,\vm_k] & =\bar{\eta}_k^2 \norm{\nabla_{k}}^2 + \bar{\eta}_k^2\sigma^2 \E\tr\mSigma(\vx_k) - 2\bar{\eta}_k \dotp{\nabla_{k}}{\vy_{k}} 
         +\eta^{2-2\alpha}(\norm{\beta_k\vm_k+(1-\beta_k)\vg_k}^2-\norm{\vm_k}^2)\\
        & \leq \E_{|\vy_k}\frac{\lambda^2_{\max}\eta^2_{\max}}{\lambda^2_{\min}}( 2\eta^2(\|\nabla_0\| + L\|\vx_0\|)^{2} + 2\eta^2L^{2}\|\vx_k\|^{2} + \eta C_{\tr})\\      
    &\quad +2\frac{\lambda_{\max}\eta_{\max}}{\lambda_{\min}}\eta (\|\nabla_0\| + L\|\vx_k\|+L\| \vx_0\|)\norm{\vy_k}\\
    &\quad + \eta^{2-2\alpha}(\frac{1-\beta_k}{1+\beta_k}\norm{\nabla_k}^2 + (1-\beta_k)^2\sigma^2\E\tr\mSigma(\vx_k))\\
        & \leq (1+\E[\norm{\vx_k}^2|\vy_k,\vm_k]+\norm{\vy_k}^2)\eta K_3.\\ 
    \end{align*}
    Note that 
    \begin{align*}
        \norm{\vx_k}^2 & = \norm{\vy_k+\bar{\eta}_k\beta_k(1-\beta_k)^{-1}\vm_k}^2\\
        & \leq 2\norm{\vy_k}^2 + \frac{2\eta_{\max}^2\lambda_{\max}^2}{\lambda_{\max}^4}\eta^{2-2\alpha}\norm{\vm_k}^2
    \end{align*}
    Then we can write $\E[\delta_k|\vy_k,\vm_k]\leq K_4\eta(1+\norm{\vy_k}^2+\eta^{2-2\alpha}\norm{\vm_k}^2)$ for some constant $K_4$. Futhermore, expansion gives for some constant $K_4,K_5,K_6,k_7$,
    \begin{align*}
        \E[\delta_k^2|\vy_k] & \leq K_4 \E(\eta \norm{\vx_k}^2+\eta\norm{\vv_k}^2+\eta\norm{\vy_k}^2+\eta^{1/2}\dotp{\vv_k}{\vy_k}+\eta^{3/2-\alpha}\dotp{\vm_k}{\vv_k}+\eta^{3/2-\alpha}\dotp{\vx_k}{\vv_k})^2\\
        &\leq K_5 \eta(1+\norm{\vy_k}^2+\eta^{2-2\alpha}\norm{\vm_k}^2).\\
        \E[|\delta_k|^m|\vy_k] & =K_6 \E(\eta \norm{\vx_k}^2+\eta\norm{\vv_k}^2+\eta\norm{\vy_k}^2+\eta^{1/2}\dotp{\vv_k}{\vy_k}+\eta^{3/2-\alpha}\dotp{\vm_k}{\vv_k}+\eta^{3/2-\alpha}\dotp{\vx_k}{\vv_k})^m \\
        &\leq K_7 \eta^{m/2}(1+\norm{\vy_k}^2+\eta^{2-2\alpha}\norm{\vm_k}^2).
    \end{align*}
    Therefore taking expecation with respect to all, we know 
    \[\E[(1+\|\vy_{k+1}\|^2+\eta^{2-2\alpha}\|\vm_{k+1}\|^2)^m]\leq (1+mK_4\eta + 2^m(K_5+K_7)\eta)\E(1+\|\vy_{k}\|^2+\eta^{2-2\alpha}\|\vm_{k}\|^2)^m\]
    Therefore 
    \begin{align*}\E(1+\|\vy_{k}\|^2+\eta^{2-2\alpha}\|\vm_{k}\|^2)^m & \leq e^{(mK_4\eta + 2^m(K_5+K_7)\eta)k} \E(1+\|\vy_{0}\|^2+\eta^{2-2\alpha}\|\vm_{0}\|^2)^m\\
    &\leq e^{mK_4T + 2^m(K_5+K_7)T}3^m(1+\E\|\vy_{0}\|^{2m}+\eta^{2-2\alpha}\E\|\vm_{0}\|^{2m}).
    \end{align*}
    And clearly $\E\|\vy_{0}\|^{2m}\leq 2^m \E\|\vx_{0}\|^{2m} + (\frac{2\lambda_{\max}\eta_{\max}}{\lambda_{\min}^2})^m \eta^{m-m\alpha}\E\|\vm_{0}\|^{2m}$. Therefore we finished bounding the moments of $\vy_k$ and $\eta^{\frac{1-\alpha}{2}}\vm_k$.
    Then for $\vx_k$, we can write
	$$ \|\vx_{k}\|^{2m} \leq 2^{2m}\left(\|\vy_{k}\|^{2m} + (\frac{2\lambda_{\max}\eta_{\max}}{\lambda_{\min}^2})^m \eta^{m-m\alpha} \|\vm_{k}\|^{2m}\right) $$
    And for $\vg_k$ with \Cref{moments-bound-1} we are able to finish the proof.
	%Then, the only term remaining to bound is $\|\vm_{k-1}\|^{2m}$. 
	%\begin{align*}
		%\|\vm_{k-1}\|^{2m} &= \|\beta^{k-1}\vm_0 + \sum_{s=0}^{k-2}\beta^{k-s-1}(1-\beta)\vg_s\|^{2m} \leq (\beta^{k-1}\|\vm_0\| + G)^{2m}
	%\end{align*}
	%We bounded $G^{2m}$ above.
	%$\sup\|\vx_\tau\|$ is small by the inductive hypothesis, so we are done.
\end{proof}
% As a result, we can write 
%\begin{align}
%\norm{\E\vm_k} &\leq \|\vm_0\| + \mathcal O(1) + \frac{L\eta}{1-\beta}\sup_{\tau=0,...,k-1} \E\|\vm_\tau\|
%\end{align}
\begin{proof}[Proof for \Cref{thm:coupled_weak_approx}]
	We expand the weak approximation error for a single $k$. There is $\theta\in[0,1]$ and $\vz=\theta \vx_k+(1-
 \theta)\vy_k$ such that
	\begin{align*}
		|\E h(\vx_k) - \E h(\vy_k)| &\leq \E[|\langle \nabla h(\vz), \vx_k-\vy_k \rangle|] \\
		&\leq \E[\|\nabla h(\vz) \| \|\vx_k-\vy_k\|] \\
		&\leq K(\vx_0, \vm_0,T) \E[\|\vx_k-\vy_k\|]
	\end{align*}
	where $x_0$ is the initialization and the first line follows from the mean value theorem. $K(x_0)$ has polynomial growth, and the last line follows from the assumption that the test function and its derivatives have polynomial growth combined with \Cref{lem:small_higher_order}.
	From the definition of the coupled trajectory, we can write
	\begin{align*}
		\E \|\vx_k - \vy_k \| \leq \frac{\lambda_{\max}\eta_{\max}}{\lambda^2_{\min}}\eta^{1-\alpha} \E \|\vm_k\|	
	\end{align*}
	Then, we apply \Cref{lem:boun-mome} together to write

 \iffalse
	\begin{align*}
		\E\|\vm_k\| &\leq \sqrt{\frac 1\eta \frac{1-\beta}{1+\beta} C} + \|\vm_0\| + \frac{L\eta}{1-\beta}\sup_{\tau=0,...,k-1} \E\|\vm_\tau\| + \mathcal O(1) \\
		&\leq \frac{L\eta}{1-\beta}\sup_{\tau=0,...,k-1} \E\|\vm_\tau\| + \mathcal O(\eta^{(\alpha-1)/2})
	\end{align*}

    Choose $\eta$ small enough such that $\eta<(\frac{\lambda}{2L})^{1/(1-\alpha)}$,
    \begin{align*}
		\sup_{\tau=0,...,k-1} \E\|\vm_\tau\|  &\leq\frac{L\eta}{1-\beta}\sup_{\tau=0,...,k-1} \E\|\vm_\tau\| + \mathcal O(1) \\
		&\leq \frac{L\eta}{1-\beta}\sup_{\tau=0,...,k-1} \E\|\vm_\tau\| + \mathcal O(\eta^{(\alpha-1)/2})
	\end{align*}
 
	Applying \Cref{lem:exp_norm_m} repeatedly to bound the supremum gives us that $\sup_{\tau=0,...,k-1} \E\|\vm_\tau\| = \norm{\E\nabla_k} + \mathcal O(\eta^{(\alpha-1)/2})$.
	As a result,
 \fi
	$$  \E\|\vm_k\|\leq\sqrt{\E\norm{\vm_k}^2}  \leq \eta^{\frac{\alpha-1}{2}} f(\vm_0,\vx_0,T) $$
	which implies that
	$$|\E h(\vx_k) - \E h(\vy_k)| \leq \eta^{(1-\alpha)/2} \frac{\lambda_{\max}\eta_{\max}}{\lambda^2_{\min}}f(\vm_0,\vx_0,T)K(\vx_0, \vm_0,T).$$
\end{proof}
\subsubsection{Step 2}
%\runzhe{define $\Delta$ and $\tilde\Delta$ here}
In this section we are comparing the trajectory $\vy_k$ with a SGD trajectory $\vz_k$. To avoid notation ambiguity, denote $\vg(\vx)\sim\cG_\sigma(\vx)$ to be the stochastic gradient sampled at $\vx$. . Recall that (with $\vy_0=\vx_{0}-\frac{\bar{\eta}_0\beta_0}{1-\beta_0}\vm_k$ and $\vz_0=\vx_0$)
\begin{align*}
\vy_{k} =\vy_{k-1}-\bar{\eta}_{k-1}\vg(\vx_{k-1})  \\
\vz_{k} =\vz_{k-1}-\bar{\eta}_{k-1}\vg(\vz_{k-1})  \\
\end{align*}
The only difference in the iterate is that the stochastic gradients are taken at close but different locations of the trajectory. Therefore to study the trajectory difference, we adopt the method of moments proposed in~\citet{li2019stochastic}.

We start by defining the one-step updates for the coupled trajectory and for SGD. 
\begin{definition}[One-Step Update of Coupled Trajectory]\label{def:one_step_coupled}
    The one-step update for the coupled trajectory $\vDelta$ can be written as 
    \begin{align*}
		\vDelta(\vy, \vm,C) &= -\eta\vg(\vx) = -\eta\vg\left(\vy + C \eta^{1-\alpha}\vm\right)
    \end{align*}
\end{definition}

\begin{definition}[One-Step Update of SGD]\label{def:one_step_sgd}
    The one-step update for SGD $\tilde\vDelta$ can be written as 
    \begin{align*}
		\tilde\vDelta(\vy) &= -\eta\vg(\vy) 	
	\end{align*}
\end{definition}

\begin{lemma}[Close Moments]\label{lem:moment-diff}
	Let $\Delta$ be the one-step update for the coupled trajectory and $\tilde\Delta$ be the one-step update for SGD. Then for any $C\in [0,\frac{\lambda_{\max}\eta_{\max}}{\lambda^2_{\min}}]$, there is function $j(\vx_0,\vm_0,T)$ of polynomial growth in $\vx_0$ and $\vm_0$, independent of $\eta$, such that
	$$ \left|\E [\tilde\Delta_{(i)}(\vy) -\Delta_{(i)}(\vy,\vm_k,C)]\right| 
	\leq \eta^{\frac{3-\alpha}{2}} j(\vx_0,\vm_0,T) $$$$ \left|\E [\tilde\Delta_{(i)}\tilde\Delta_{(j)}(\vy) -\Delta_{(i)}\Delta_{(j)}(\vy,\vm_k,C)]\right| 
	\leq  \eta^{\frac{3-\alpha}{2}} j(\vx_0,\vm_0,T) $$for all $k\in [0,T/\eta]$.
\end{lemma}

\begin{proof}
%We first compute the moments of the updates $\Delta$ and $\tilde\Delta$. 
%Applying Lemma 5 of the SME paper gives us
%\begin{align*}
%	\E[\tilde\Delta_{(i)}(\vy)] &= -\eta\partial_{(i)} L(\vy) \\
%	\E[\tilde\Delta_{(i)}(\vy)\tilde\Delta_{(j)}(\vy)] &= \eta^2\partial_{(i)}L(\vy)\partial_{(j)} L(\vy) + \eta^2\sigma^2\mSigma(\vy)_{(i,j)} = O(\eta) \\
%	\E\left[\prod_{j=1}^3 |\tilde\Delta_{i_j}(\vy)|\right] &= O(\eta^3) 
%\end{align*}
%So, we can write
We can write
\begin{align*}
	\left|\E[\E [\tilde\Delta_{(i)}(\vy) -\Delta_{(i)}(\vy)\mid \vm ]]\right| &= \eta \left| \E[\partial_{(i)} \Loss(\vy) -  \partial_{(i)} \Loss\left(\vy + C\eta^{1-\alpha} \vm\right)] \right| \\
	&\leq LC\eta^{2-\alpha} \E[\|\vm\|]
\end{align*}
where the second step uses the Lipschitzness of the loss gradient. The proof can be completed by noting $\E[\|\vm\|] = O(\eta^{(\alpha-1)/2})$ (\Cref{lem:boun-mome}).
\begin{align*}
	\left|\E[\E [\Delta_{(i)}(\vy)\Delta_{(j)}(\vy) -\tilde\Delta_{(i)}(\vy)\tilde\Delta_{(j)}(\vy)\mid \vm ]]\right| &= \eta^2 \partial_i\Loss\partial_j \Loss(\vy) + \eta^2\sigma^2\mSigma_{ij}(\vy)  \\ & - \E\left[\eta^2\partial_i\Loss\partial_j \Loss\left(\vy + C\eta^{1-\alpha}\vm\right) - \eta^2\sigma^2 \mSigma_{ij}(\vy + C\eta^{1-\alpha}\vm)\right]\\
 &\leq \eta^{2-\alpha} C(L_\Sigma +\eta L) \E[\|\vm\|]
\end{align*}
where $L_\Sigma$ is the smoothness of $\mSigma$. Again by \Cref{lem:boun-mome} we obtain the desired result.
\end{proof}

\begin{lemma}
	%Let $\vDelta(\vy, \vm)$ be the single step update for the coupled trajectory (\Cref{def:coupled_trajectory}), as defined below.
	%\begin{align*}
	%	\vDelta(\vy, \vm) &= -\eta\vg(\vx) = -\eta\vg\left(\vy + \frac{\eta\beta}{1-\beta}\vm\right) 	
	%\end{align*}
	%Then, 
    For any $m\geq 1$, $C\in [0,\frac{\lambda_{\max}\eta_{\max}}{\lambda^2_{\min}}]$,
	\begin{align*}
            \E[\|\vDelta(\vy_k, \vm_k)\|_2^{2m}] \leq \eta^{m} h(\vm_0, \vx_0,m,T)
	\end{align*}
    for all $k\leq T/\eta$.\label{lem:bounded_coupled_updates}
\end{lemma}
\begin{proof}
	Pulling $\eta^{2m}$ out of the update and applying \Cref{lem:small_higher_order} completes the proof.
\end{proof}



The below lemma is an analog to Lemma 27 of \citet{li2019stochastic}, Lemma C.2 of \citet{li2021validity}, and Lemma B.6 of \citet{malladi2022sdes}.
It shows that if the update rules for the two trajectories are close in all of their moments, then the test function value will also not change much after a single update from the same initial condition. 
\begin{lemma}
	Suppose $u\in G^{\lceil\gamma\rceil + 1}$ for $\gamma\geq 0$. Assume Conditions 1 and 2 from \Cref{thm:one_step_to_many} hold. Then, there exists a function $K\in G$ independent of $\eta$ such that
	$$\left|\E[u(\vx + \Delta(\vx,k))] - \E[u(\vx + \tilde\Delta(\vx,k))]\right| \leq K(\vx)\eta^{\gamma+1} $$
	\label{lem:one_step}
%\runzhe{We need to expand $u$ so R is the $(p+1)$-th order remainder. Write the proof formally for all $p$}
%\runzhe{$p=2$ is what we needed at last}
\end{lemma}
\begin{proof}

Since $u \in G^{\lceil\gamma\rceil + 1}$, we can find $K_0 \in G$ such that $u(\vx)$ is bounded by $K_0(\vx)$ and so are all the partial derivatives of $u$ up to order $4$.
By Taylor's Theorem with Lagrange Remainder, for all $1 \le j \le N$, $1 \le k \le N$ we have
\begin{align*}
    u(\vx + \vDelta(\vx, k)) - u(\vx + \tilde{\vDelta}(\vx, k))
    &= \underbrace{\sum_{1\leq i\leq D} \frac{\partial u(\vx)}{\partial x_i} \left(\Delta_i(\vx,k) - \tilde\Delta_i(\vx, k)\right)}_{A} \\
    &+ \underbrace{\frac{1}{2} \sum_{1\leq i_1, i_2\leq D}\frac{\partial^2 u(\vx)}{\partial x_{i_1}\partial x_{i_2}} \left(\Delta_{i_1}(\vx,k)\Delta_{i_2}(\vx,k) - \tilde\Delta_{i_1}(\vx, k)\tilde\Delta_{i_2}(\vx,k)\right)}_B \\
    &+ R_j - \tilde{R}_j
\end{align*}
where the remainders $R_j$, $\tilde{R}_j$ are
\begin{align*}
    R_j &:=
    \frac{1}{3!} \sum_{1 \le i_1, i_2, i_3 \le D} \frac{\partial^3 u(\vx + a \vDelta(\vx, k))}{\partial x_{i_1} \partial x_{i_2}\partial x_{i_3}} \Delta_{i_1}(\vx, k) \Delta_{i_2}(\vx, k) \Delta_{i_3}(\vx, k). \\
    \tilde{R}_j &:=
    \frac{1}{3!} \sum_{1 \le i_1, i_2, i_3 \le D} \frac{\partial^3 u(\vx + a \tilde\vDelta(\vx, k))}{\partial x_{i_1} \partial x_{i_2}\partial x_{i_3}} \tilde\Delta_{i_1}(\vx, k) \tilde\Delta_{i_2}(\vx, k)\tilde\Delta_{i_3}(\vx, k).
\end{align*}
for some $a, \tilde{a} \in [0, 1]$.
By Condition 1 of \Cref{thm:one_step_to_many}, 
\begin{align*}
\E[A + B] \leq (D+D^2/2) K_0(\vx)K_1(\vx)\eta^{\gamma+1}
\end{align*}

Now let $\kappa_0, m$ be the constants so that $K_0(\vx)^2 \le \kappa_0^2 (1 + \normtwosm{\vx}^{2m})$.

For $R_j$, by Cauchy-Schwarz inequality we have
\begin{align*}
    \E[R_j] 
    &\le \frac{1}{3!}\left(\sum_{i_1, i_2, i_3}
        \E\left[\abs{\frac{\partial^3 u(\vx + a \vDelta(\vx, k))}{\partial x_{i_1} \partial x_{i_2}\partial x_{i_3}}}^2\right]
    \right)^{1/2}
    \cdot
    \left(\E[\|\vDelta(\vx, k)\|_2^6]\right)^{1/2} \\
    &\le \frac{1}{3!}\left(D^{3/2} K_0(\vx + a\vDelta(\vx,k))\right)\cdot 
    \E[\|\vDelta(\vx, k)\|_2^3] \\
    &\le \frac{1}{3!}\left(D^{3/2} K_0(\vx + a\vDelta(\vx,k))\right)\cdot 
    \eta^{1+\gamma}K_2(\vx)
\end{align*}

\iffalse
\begin{align*}
    \E[R_j]
    &\le \frac{1}{2} \left(
        \sum_{i_1, i_2}
        \E\left[\abs{\frac{\partial^2 u(\vx + a \vDelta(\vx, k))}{\partial x_{i_1} \partial x_{i_2}}}^2\right]
    \right)^{1/2}
    \cdot
        \E[\|\vDelta(\vx, k)\|_2^{4}]^{1/2} \\
    &\le \frac{1}{2} \left(D^2 \cdot K_0 (\vx + a \vDelta(\vx, k))\right)
    \cdot \E[\|\vDelta(\vx, k)\|_2^{2}]\\
    &\le \frac{1}{2} \left(D^2 \cdot K_0 (\vx + a \vDelta(\vx, k))\right)
    \cdot \eta^{1+\gamma} K_1(\vx),
\end{align*}
\fi

where the last line uses Condition 2 of \Cref{thm:one_step_to_many} and $K_2$ is a function of polynomial growth. 
%\red{TODO fix problem with power on $\eta$ in last line...}
For $K_0(\vx + a \vDelta(\vx, k))$, we can bound its expectation by
\begin{align*}
    \E[K_0(\vx + a \vDelta(\vx, k))]
    &\le \kappa_0 \E\left[1 + \normtwosm{\vx + a \vDelta(\vx, k)}^{2m}\right]^{1/2} \\
    &\le \kappa_0 \left( 1 + 2^{2m-1} \E[\normtwosm{\vx}^{2m} + \E\normtwosm{\vDelta(\vx, k)}^{2m}] \right)^{1/2} \\
    &\le \kappa_0 \left( 1 + 2^{2m-1} (\normtwosm{\vx}^{2m} + C_{2m} \eeta^{\gamma+1} (1 + \normtwosm{\vx}^{2m}))  \right)^{1/2}.
\end{align*}
%\runzhe{the constant $C_{2m}$ does appear in \cref{thm:one_step_to_many}. Define it somewhere}
where $C_{2m}$ arises from the constants in the polynomial growth function $K_2(\vx)$.
We use \Cref{lem:bounded_coupled_updates} between the second and third lines. 
Combining this with our bound for $\E[R_j]$ proves that $\frac{1}{\eta^{\gamma+1}}\E[R_j]$ is uniformly bounded by a function in $G$:
\begin{align*}
    \E[R_j] \le
    \eta^{\gamma +1} \cdot 
    \frac{1}{(\lceil\gamma\rceil+1)!} \cdot D^4 \cdot 
    \kappa_0 \left( 1 + 2^{2m-1} (\normtwosm{\vx}^{2m} + C_{2m} \eeta^{\gamma+1} (1 + \normtwosm{\vx}^{2m}))  \right)^{1/2}
    \cdot K_1(\vx).
\end{align*}

An analogous argument bounds $\E[\tilde{R}_j]$. Thus, the entire Taylor expansion and remainders are bounded as desired.
	
\end{proof}





\subsubsection{Main Result}
\begin{theorem}
	Let $T>0$, $\eta_{\text{thr}}>0$, and $\eta\in(0,\eta_{\text{thr}}]$, so $N=\lfloor T/\eta\rfloor$. Let $\gamma=\frac{1-\alpha}{2}> 0$ and $p\in \mathbf{Z}^+$. 
	Let $\vDelta$ be the single step update for the coupled trajectory $\vy_k$ and $\tilde\vDelta$ be the single step update for SGD $\vz_k$ (\Cref{def:one_step_coupled,def:one_step_sgd}).
	Suppose the following conditions hold:
	\begin{enumerate}
		\item \textbf{Close moments}: %For $s=1,...,\lceil\gamma\rceil$
         For $s=1,2$,  $$ \left| \E\prod_{j=1}^s \Delta_{i_j}(x) - \E\prod_{j=1}^s \tilde\Delta_{i_j}(x)\right| \leq K_1(x)\eta^{\gamma+1} $$
		\item \textbf{Small higher order moments}: $$ \E\prod_{j=1}^{3} |\Delta_{(i_j)}(x)| \leq K_2(x)\eta^{\gamma+1} $$
		\item \textbf{Bounded trajectory}: For each $m\geq 1$, the $2m$-moment of $\vx_k$ is uniformly bounded for all $k<T$ and $\eta$.
		\item \textbf{Lipschitz Gradients}: For all $\vx,\vy$, $\nabla L(\vx) - \nabla L(\vy) \leq L\|\vx-\vy\|$	
		\end{enumerate}
    Then $\vy_k$ and $\vz_k$ are order-$\gamma$-weak approximations of each other.
\label{thm:one_step_to_many}
\end{theorem}
\begin{proof}
	%Let $\vz_k$ be the SGD trajectory and $\vy_k$ be the coupled trajectory (\Cref{def:coupled_trajectory}).
	%Define $\Delta$ and $\tilde{\Delta}$ to be the coupled trajectory and SGD one-step updates, respectively:
	%\begin{align}
	%	\Delta(\vy) &= -\eta\vg(\vx) = -\eta\vg\left(\vy + \frac{\eta\beta}{1-\beta}\vm\right) \\
		%\tilde\Delta(\vy) &= -\eta\vg(\vy)
	%\end{align}
	Let $\hat\vy_{j,k}$ be the trajectory defined by following the coupled trajectory for $j$ steps and then standard SGD for $k-j$ steps.
	So, $\hat\vy_{j,j+1} = \vy_j + \tilde\Delta(\vy_j)$ and $\hat\vy_{j+1,j+1} = \vy_j + \Delta(\vy_j)$.
	Let $h$ be the test function with at most polynomial growth.
	Then, we can write
	\begin{equation}
		|\E[h(\vz_k) - \E[h(\vy_k)]| = \sum_{j=0}^{k-1} (\E[h(\hat\vy_{j+1,k})] - \E[h(\hat\vy_{j,k})])
	\end{equation}
	Define $u(\vy, s, t) = \E_{\hat\vy\sim\mathcal{P}(\vy, s, t)} [h(\hat\vy_t)]$, where $\mathcal{P}(\vy, s, t)$ is the distribution induced by starting at $\vy$ at time $s$ and following the SGD trajectory until time $t$.
	Then,
	\begin{equation}
		|\E[h(\vz_k) - \E[h(\vy_k)]| \leq \sum_{j=0}^{k-1} |\E[u(\hat\vy_{j+1,j+1}, j+1, k)] - \E[u(\hat\vy_{j,j+1}, j+1, k)]|
	\end{equation}
	Define $u_{j+1} = u(\vy, j+1, k)$.
	Then,
	\begin{equation}
		|\E[h(\vz_k) - \E[h(\vy_k)]| \leq \sum_{j=0}^{k-1} | \E[u_{j+1}(\vy_j + \tilde\Delta(\vy_j))] - \E[u_{j+1}(\vy_j + \Delta(\vy_j))]|
	\end{equation}
	\Cref{lem:one_step} shows that
	\begin{equation}
		| \E[u_{j+1}(\vy_j + \tilde\Delta(\vy_j))] - \E[u_{j+1}(\vy_j + \Delta(\vy_j))]| \leq \E[K(\vy_j)\eta^{\gamma+1}]
	\end{equation}
	where $K(\vy_j) = \kappa_1(1+\|\vy_j\|_2^{2m})$.
	Then,
	\begin{align*}
		|\E[h(\vz_k) - \E[h(\vy_k)]| &\leq \eta^{\gamma+1} \sum_{j=0}^{k-1} \E[\kappa_1 (1+\|\vy_j\|_2^{2m})] \\
		&\leq \eta^{\gamma+1} \sum_{j=0}^{k-1} \kappa_1(1+f(\vx_0, \vm_0)) \\
		&\leq \kappa_1(1+f(\vx_0, \vm_0)) T\eta^{\gamma}	
	\end{align*}
	where we use the fact that $\E[\|\vy_j\|_2^{2m}] \leq f(\vx_0, \vm_0)$ (\Cref{lem:small_higher_order}) and $\E[\|\vz_j\|_2^{2m}]\leq f(\vx_0, \vm_0)$ (Similarly applying \Cref{lem:small_higher_order} when $\beta_k=0$).
\end{proof}
\begin{proof}[Proof for \Cref{thm:weak-appro-main}]
By \Cref{lem:moment-diff} the two trajectories satisfy the close moments condition. By \Cref{lem:small_higher_order} they satisfy bounded trajectory condition. The Lipschitz Gradients 
condition is given in \Cref{assume:ngos} and now we show the two trajectories have small higher order moments: for
$m\geq 3$ there is for $\vv_k\sim\cG_\sigma(\vx+C\eta^{1-\alpha}\vm_k)$,
\begin{align*}\E\norm{\Delta(\vx,\vm_k,C)}^{m} & =\eta^m\E\norm{\nabla\Loss(\vx+C\eta^{1-\alpha}\vm_k)+\sigma\vv_k}^m\\
& \leq \eta^m2^m (\norm{\nabla\Loss(\vx)}+LC\eta^{1-\alpha}\norm{\vm_k})^m + \eta^{m/2}2^m\E\norm{\vv_k}^m = O(\eta^{m/2}).\\
\end{align*}
Therefore the higher order moments are of order $O(\eta^{3/2})$.

From \Cref{thm:coupled_weak_approx}, we know $\vx_k$ and $\vy_k$ are  order-$(1-\alpha)/2$-weak approximations of each other, and from \Cref{thm:one_step_to_many} we know $\vz_k$ and $\vy_k$ are  order-$(1-\alpha)/2$-weak approximations of each other. Then we conclude that $\vx_k$ and $\vz_k$ are order-$(1-\alpha)/2$-weak approximations of each other.
\end{proof}

\iffalse
If $\beta=1-\lambda\eta^\alpha$,
\begin{align}
	\E[x_{k,i} - y_{k,i}] = (\lambda\eta^{1-\alpha} - \eta) \E[m_{k,i}]	
\end{align}

To bound $\E[m_{k,i}]$, we unroll the momentum and write 
\begin{align}
	\vm_k &= \beta^k\vm_0 + \sum_{s=0}^{k-1} \beta^{k-s-1}(1-\beta)\vg_s \label{eq:unrolled_momentum} \\
	\E\vm_k &= \beta^k\E\vm_0 + \sum_{s=0}^{k-1} \beta^{k-s-1}(1-\beta)\nabla_s  \label{eq:unrolled_momentum2}
\\
	\E m_{k,i} &= \beta^k \E m_{0,i} + \sum_{s=0}^{k-1} \beta^{k-s-1}(1-\beta)\nabla_{s,i} 
\end{align}

We replace $\nabla_{s,i}$ with $\nabla_{s,i} - \E \nabla_{k,i} + \E \nabla_{k,i}$. 
We use the Lipschitzness of $\nabla$ to write 
\begin{align*}
	\|\nabla_s - \nabla_k\| &\leq L\eta(k-s)\sup_{\tau=s,...,k-1}\|\vm_\tau\| \\
	\E[\nabla_{s,i} - \nabla_{k,i}] &\leq \frac{1}{\sqrt{P}} L\eta(k-s)\sup_{\tau=s,...,k-1}\|\vm_\tau\|     	
\end{align*}
where $P$ is the length of $\nabla$ (i.e., the number of parameters).

If we substitute this into \Cref{eq:unrolled_momentum2}, then we can write
$$ \E\vm_k = \beta^k\E\vm_0 + \sum_{s=0}^{k-1} \beta^{k-s-1}(1-\beta)(\|\nabla_k\| + L\eta(k-s)\sup_{\tau=s,...,k-1} \|\vm_\tau\|) $$
We first bound the coefficients in the last term to be $\mathcal O(\eta^{1-\alpha})$.
$$ \sum_{s=0}^{k-1} \beta^{k-s-1} (1-\beta)\eta(k-s) \leq \frac{\eta}{1-\beta} = \mathcal{O}(\eta^{1-\alpha}) $$
Then, we bound $\E \|\vm_k\|^2$:
\begin{align}
        \E\norm{\vm_k}^2&= \norm{\E\vm_k}^2+ \E\norm{\vm_k-\E\vm_k}^2\\
        & = \norm{\E\vm_k}^2+\beta^{2k}\E\norm{\vm_0-\E\vm_0}^2 + \sum_{s=0}^{k-1} \beta^{2(k-s-1)}(1-\beta)^2\E\norm{\vsigma_s}^2\\
        & \leq \norm{\E\vm_k}^2 + \E\norm{\vm_0}^2 + \frac{1}{\eta}\frac{1-\beta}{1+\beta}\sum_{s=0}^{k=1}\tr(\mSigma_s\mSigma_s^\top) \label{eq:exp_mnorm_sq}
\end{align}

Since $\tr(\mSigma_s\mSigma_s^\top)$ is bounded for all time steps $s$, we drop the summation and just write $\tr(\mSigma\mSigma^\top)$. To handle the first term:
\begin{align}
\norm{\E\vm_k} &\leq \|\E\vm_0\| + \sum_{s=0}^{k-1} \beta^{k-s-1}(1-\beta)\norm{\E\nabla_k}+ \sum_{s=0}^{k-1} \beta^{k-s-1} (1-\beta) L \eta (k-s) \sup_{\tau=s,...,k-1} \E\|\vm_\tau\|	\\
& \leq \|\E\vm_0\| + \norm{\E\nabla_k} + \frac{L\eta}{1-\beta}\sup_{\tau=0,...,k-1} \E\|\vm_\tau\|
\end{align}
We use this with \Cref{eq:exp_mnorm_sq} to bound $\E\norm{\vm_k}$.
\begin{align}
	\E\norm{\vm_k} &\leq \sqrt{\E\norm{\vm_k}^2}\\
		& \leq \sqrt{\norm{\E\vm_k}^2 + \E\norm{\vm_0}^2 + \frac{1}{\eta}\frac{1-\beta}{1+\beta}\tr(\mSigma\mSigma^\top)}\\
		& \leq \norm{\E\vm_k} + \sqrt{\E\norm{\vm_0}^2} +\sqrt{\frac{1}{\eta}\frac{1-\beta}{1+\beta}\tr(\mSigma\mSigma^\top)}\\
		& \leq \norm{\E\nabla_k} + 2\sqrt{\E\norm{\vm_0}^2} +\sqrt{\frac{1}{\eta}\frac{1-\beta}{1+\beta}\tr(\mSigma\mSigma^\top)}+\frac{L\eta}{1-\beta}\sup_{\tau=0,...,k-1} \E\|\vm_\tau\|
\end{align}
From this, we can conclude that $\sup_{\tau=0,...,k-1} \E\|\vm_\tau\| = \norm{\E\nabla_k} + \mathcal O(\eta^{(\alpha-1)/2})$, which also implies that $\norm{\E\vm_k}\leq 2\norm{\E\nabla_k} + \mathcal O(1)$. 

Finally, we can bound $\E\|\nabla_k\|$. Let $d_k = \E\|\vy_k-\vy_0\|^2$, so 
\begin{align}
d_{k+1} &= d_{k} +\E\|\vy_{k+1}-\vy_k\|^2 + 2\E(\vy_{k+1}-\vy_k)^\top(\vy_k-\vy_0)\\
&\leq d_k + \eta^2 \E\norm{\nabla_k}^2 + \eta \tr(\mSigma\mSigma^\top) + 2 \eta\sqrt{d_k}\E\norm{\nabla_k}
\end{align}
\begin{align}
\|\nabla_k\| & \leq \|\nabla_0\| + L\norm{\vx_{k}-\vx_0}\\
& \leq \|\nabla_0\| + L\norm{\vy_{k}-\vy_0} + L\norm{\vx_{0}-\vy_0} + L\norm{\vx_{k}-\vy_k} \label{eq:normnablak}
\end{align}
Then, we note that $\|x_k - y_k\| = \frac{\eta\beta}{(1-\beta)}\|\vm_k\|$. So
\begin{align}
\E\|\nabla_k\| & \leq \E\|\nabla_0\| + L\sqrt{d_k} +\frac{\eta\beta}{1-\beta}(\E\|\vm_0\|+\E\|\vm_k\|)\\
& \leq O(1) + L\sqrt{d_k} + O(\eta^{1-\alpha})\E\|\nabla_k\|
\end{align}
From \Cref{eq:normnablak},
\begin{align}
\E\|\nabla_k\|^2 & \leq (O(1) + L\sqrt{d_k})^2
\end{align}
plugin

\begin{align}
d_{k+1} &\leq d_k + \eta^2 \E\norm{\nabla_k}^2 + \eta \tr(\mSigma\mSigma^\top) + 2 \eta\sqrt{d_k}\E\norm{\nabla_k}\\
&\leq (1+\eta^2L^2+2\eta L)d_k + O(\eta)
\end{align}
 Then by Gronwall's inequality, $d_k$ is bounded by $O(1)$ and so is $\E\|\nabla_k\|$ for $k=O(1/\eta)$. The proof is completed by noting $\|\E\nabla_k\| \leq \E\|\nabla_k\|$
 \fi
\subsection{$\alpha\geq 1$}\label{discus}
Following the idea of Stochastic Modified Equations \citep{li2019stochastic}, the limiting distribution can be described by the law of the solution $\bvx_t$ to an SDE under brownian motion $\bvw_t$
\begin{equation}
\dd \bvx_t = -\lambda_t \nabla\Loss(\bvx_t) \dd t + \lambda_t \mSigma^{1/2}(\bvx_t) \dd \bvw_t\label{equ:sgd-sde-1}
\end{equation}
for some rescaled learning rate schedule $\lambda_t$ that $\bar{\eta}_k\to \lambda_{k\eta}$ in the limit.

When $\alpha=1$, the limit distribution of SGDM becomes
\begin{align}
\dd \bvx_t & = \lambda_t/\gamma_t \dd \bvm_t -\lambda_t \nabla\Loss(\bvx_t) \dd t + \lambda_t \mSigma^{1/2}(\bvx_t) \dd \bvw_t\\
\dd \bvm_t & = -\gamma_t  \bvm_t \dd t + \gamma_t \nabla\Loss(\bvx_t) \dd t - \gamma_t \mSigma^{1/2}(\bvx_t) \dd \bvw_t\label{equ:sgd-sde-2}
\end{align}
with similarly $\gamma_t$ being the limit $(1-\beta_k)/\eta\to \gamma_{k\eta}$.
Here $\bvm_t$ is the rescaled momentum process that induces a non-negligible impact on the original trajectory $\bvx_t$.

Furthermore, when $\alpha>1$, if we still stick to following $O(1/\eta)$ steps for any $\eta$, then the dynamics of trajectory will become trivial. In the limit $\eta\to 0$, as $\vm_k-\vm_0=O(k\eta^{\alpha})=O(T\eta^{\alpha-1})\to 0$, the trajectory has limit $\vx_k = \vx_0 - \sum_{i=0}^{k-1}\eta_i \vm_0$ for all $k=O(1/\eta)$. This is different from the case $\alpha\leq 1$ where there is always a non-trivial dynamics in $\vx_k$ for the same time scale $k=O(1/\eta)$, regardless of the $\alpha$ index. We can think of the phenomenon by considering the trajectory of SGDM on a quadratic loss landscape, and in this case the SGDM behaves like a Uhlenbeck-Ornstein process. When $\alpha<1$, the direction of $\vx$ has a mixing time of $O(1/\eta)$ while the direction of $\vm$ has a shorter mixing time of $O(1/\eta^{\alpha})$, while when $\alpha>1$, both mixing time of $\vx$ and $\vm$ mixes at a time scale of $O(1/\eta^{\alpha})$, so in $O(1/\eta)$ steps the trajectory is far from any stationary states.

Therefore in this regime we should only consider the case where $\vm_0=0$ to avoid the trajectory moving too far. By rescaling $\vm$ and considering $O(\eta^{-\frac{1+\alpha}{2}})$ steps, we would spot non-trivial behaviours of the SGDM trajectory. In this case the SGDM have a different tolerance on the noise scale $\sigma$.
\section{The dynamics of SGDM in $O(1/\eta^2)$ time}\label{sec:app_long_horizon}
In this section we will present results that characterizes the behaviour of SGD and SGDM in $O(1/\eta^2)$ time. We call this setting the slow SDE regime in accordance with previous works~\citep{gu2023why}.
\subsection{Slow SDE Introduction}
There are a line of works that discusses the slow SDE regime that emerges when SGD is trapped in the manifold of local minimizers.
The phenomenon was introduced in~\citet{blanc2020implicit} and studied more generally in~\citet{li2021happens}.
In these works, the behavior of the trajectory near the manifold, found to be a sharpness minimization process for SGD, is thought to be responsible for the generalization behavior of the trajectory.

The observations in these works is that SGD should mimic a Uhlenbeck-Ornstein process along the \emph{normal} direction of the manifold of minimizers. Each stochastic step in the normal direction contributes a very small movement in the tangent space. Over a long range of time, these small contributions accumulate into a drift.

To overcome the theoretical difficulties in analyzing these small contributions,~\citet{li2021happens} analyzed a projection $\Phi$ applied to the iterate that maps a point near the manifold to a point on the manifold. $\Phi(X)$ is chosen to be the limit of gradient flow when starting from $X$. Then it is observed that when the learning rate is small enough, $\Phi(X)\approx X$, and the dynamics of $\Phi(X)$ provides an SDE on the manifold that marks the behaviour of SGD in this regime.
\subsection{Slow SDE Preliminaries}
\subsubsection{The Projection Map}
We consider the case where the local minimizers of the loss $\cL$ form a manifold $\Gamma$ that satisfy certain regularity conditions 
as \Cref{assump:manifold}. In this section, we fix a neighborhood $O_\Gamma$ of $\Gamma$ that is an attraction set under $\nabla\cL$, and define  $\phi(\bx,t) = \bx - \int_0^t\nabla\cL(\phi(\bx,s))\diff s$ and $\Phi(\bx) := \lim_{t\to\infty}\phi(\bx,t)$. $\Phi(\vx)$ is well-defined for all $\vx\in O_\Gamma$ as indicated by \Cref{assump:neighborhood}. We call $\Phi$ the gradient projection map.

The most important property of the projection map is that its gradient is always orthogonal to the direction of gradient. We ultilize the following lemma from previous works.
\begin{lemma}[\citet{li2021happens} Lemma C.2]\label{lem:gra-proj-1}
    For all $\vx\in O_\Gamma$ there is
    $\partial\Phi(\vx)\nabla\Loss(\vx)=0$.
\end{lemma}

For technical simplicity, we choose compact set $K\subset O_\Gamma$ and only consider the dynamics of $\vx_k$ within the set $K$. Formally, for any dynamics $\vx_k$ with $\vx_0\in \mathring{K}$, define the exiting stopping time $\tau=\min_k\{k>0:\vx_{k+1}\not\in K\}$, and the truncated process $\hatvx_k=\vx_{\min(k,\tau)}$; for any dynamics $\bvx_t$ in continuous time with $\bvx_0\in \mathring{K}$, define the exiting stopping time $\tau=\inf\{t>0:\bvx_{t}\not\in \mathring{K}\}$, and the truncated process $\hatbvx_t=\bvx_{\min(t,\tau)}$.

There are a few regularity conditions Katzenberger proved in the paper~\citet{katzenberger1991solutions}:
\begin{lemma}\label{lem:katzen-results}
There are the following facts.
\begin{enumerate}
\item If the loss $\Loss$ is smooth, then $\Phi$ is third-order continuously differentiable on $O_\Gamma$.
\item For the distance function $d(\vx,\Gamma)=\min_{\vy\in \Gamma\bigcap K}\norm{\vy-\vx}$, there exists a positive constant $C_K$ that 
\[\norm{\Phi(\bvx)-\bvx}\leq  C_K d(\bvx, \Gamma)\]
for any $\bvx\in K$.
\item There exists a Lyaponuv function $h(\bvx)$ on $K$ that
\begin{itemize}
    \item $h:K\to [0,\infty)$ is third-order continuously differentiable and $h(\bvx)=0$ iff $\bvx\in \Gamma$.
    \item For all $\bvx\in K$, $h(\bvx)\leq c\dotp{\nabla h(\bvx)}{\nabla \Loss(\bvx)}$ for some constant $c>0$.
    \item $d^2(\bvx, \Gamma)\leq c' h(\bvx)$ for some constant $c'>0$.
\end{itemize}
\end{enumerate}
\end{lemma}
\subsubsection{The Katzenberger Process}
We recap the notion of \emph{Katzenberger processes} in~\citet{li2021happens} and the characterization of the corresponding limiting diffusion based on Katzenberger's theorems~\citep{katzenberger1991solutions}.

\begin{definition}[Uniform metric]\label{def: uniform metric}
The \emph{uniform metric} between two functions $f,g:[0,\infty)\to \RR^D$ is defined to be $d_U(f,g) = \sum_{T=1}^\infty 2^{-T} \min\{1, \sup_{t\in[0,T)} \|f(t) - g(t)\|_2\}$.
\end{definition}

For each $n\in\NN$, let $A_n:\RR_+\to\RR_+$ be a non-decreasing functions with 
$A_n(0)=0$, and $\{Z_n(t)\}_{t\geq0}$ be a $\RR^\Xi$-valued stochastic process.
In our context of SGD, given loss function $\Loss:\RR^D\to\RR$, noise function $\sigma:\RR^D\to 
\RR^{D\times \Xi}$ and initialization $\xinit\in U$, we call the following stochastic process~\eqref{eq:katzenberger_process} a \emph{Katzenberger process} 
\begin{align}\label{eq:katzenberger_process}
    X_n(t) =\xinit + \int_0^t \sigma(X_n(s)) \diff Z_n(s)  - \int_0^t 
    \nabla \Loss(X_n(s)) \diff A_n(s)
\end{align}
if as $n\to \infty$ the following conditions are satisfied:
    \begin{enumerate}
        \item $A_n$ increases infinitely fast, i.e., $\forall
        \epsilon>0,\inf_{t\geq 0} (A_n(t+\epsilon)- A_n(t))\to\infty$;

        \item $Z_n$ converges in distribution to 
       $Z$ in uniform metric.
    \end{enumerate}

\begin{theorem}[Adapted from Theorem B.7 in \citet{li2021happens}]
\label{thm:previous_thm}
Given a  Katzenberger process $\{X_n(\cdot)\}_{n\in\NN}$, if SDE \eqref{eq:limiting_SDE_general} has a global solution $Y$ in $U$ with $Y(0)=\Phi(\xinit)$, then for any $t>0$, $X_n(t)$ converges in distribution to 
    $Y(t)$ as $n\to\infty$.
    
    \vspace{-0.6cm}
    \begin{align}\label{eq:limiting_SDE_general}
        Y(t) &= \Phi(\xinit) 
        + \int_0^t \partial\Phi(Y(s)) \sigma(Y(s)) \diff Z(s)\notag\\
        &\qquad + \int_0^t \frac{1}{2}\sum_{i,j\in [D]}\sum_{k,l\in[\Xi]}\partial^2_{ij}\Phi(Y(s))\sigma_{ik}(Y(s))\sigma_{jl}(Y(s))] \diff [Z]_{kl}(s).
    \end{align} 
    \vspace{-0.6cm}
    
\end{theorem}
We note that the global solution always exists if the manifold $\Gamma$ is compact. For the case 
 where $\Gamma$ is not compact, we introduce a compact neighbourhood of $\Gamma$ and a stopping time later for our result.
 Our formulation is under the original framework of \citet{katzenberger1991solutions} and the proof in \citet{li2021happens} can be easily adapted to \Cref{thm:previous_thm}.

\subsubsection{The C\`{a}dl\`{a}g Process}
A c\`{a}dl\`{a}g process is a right continuous process that has a left limit everwhere.
For real-value c\`{a}dl\`{a}g semimartingale processes $X_t$ and $Y_t$, define $X_{t-}=\lim_{s\to t-}X_s$, and $\int_a^b X_s dY_s$ to be the process $\int_{a+}^{b+} X_{s-} dY_s$ for interval $a<b$. That is in the integral we do not count the jump of process $Y_s$ at $s=a$ but we count the jump at $s=b$. Then it's easy to see that
\begin{itemize}
    \item $\int_a^b X_s \diff Y_s + \int_b^c X_s \diff Y_s = \int_a^c X_s \diff Y_s$ for $a<b<c$.
    \item $Z_t = \int_0^t X_s \dd Y_s$ is a c\`{a}dl\`{a}g  semimartingale if both $X_s$ and $Y_s$ are c\`{a}dl\`{a}g  semimartingales.
\end{itemize}


By a extension to higher dimensions, let $$\dd[X]_t = \dd (X_t X_t^\top) -  X_{t} (\dd X_t)^\top - (\dd X_{t}) X_t^\top$$ and  $$\dd[X, Y]_t = \dd (X_t Y_t^\top) -  X_{t} (\dd Y_t)^\top - (\dd X_{t}) Y_t^\top$$,

we know $[X]_t$ and $[X,Y]_t$ are actually matrix-valued processes with $\Delta [X]_t = (\Delta X_t) (\Delta X_t)^\top$ and $\Delta [X, Y]_t = (\Delta X_t) (\Delta Y_t)^\top$.

The generalized Ito's formula applies to a c\`{a}dl\`{a}g  semimartingale process 
(let $\partial^2 f(X)[M] = \sum_{i,j} M_{ij}\partial_{ij}f(X) $ for any matrix $M$) is given as
\[\dd f(X_t) = \dotp{\partial f(X_t)} {\dd X_t} + \frac{1}{2}\partial^2 f(X_t) [d[X]_t] + \Delta f(X)_t - \dotp{\partial f(X_{t-})}{\Delta X_t} - \frac{1}{2}\partial^2 f(X_{t-})[\Delta X_t \Delta X_t^\top].\]

And integration by part
\[\dd (XY)_t = X_t (dY_t) + (dX_t) Y_t + d[X,Y]_t.\]

These formulas will be useful in our proof of the main theorem.

\subsubsection{Weak Limit for  C\`{a}dl\`{a}g Processes}
As we are showing the weak limit for a c\`{a}dl\`{a}g process as the solution of an SDE, the following theorem is useful. We use $\mathcal{C}([0,T],\R^{d})$ to denote the set of c\`{a}dl\`{a}g functions $X:[0,T]\to \R^{d}$. 
\begin{theorem}[Theorem 2.2 in \citet{kurtz1991weak}]\label{thm:weak-limit}
    For each $n$, let $\bvx^{(n)}_t$ be a processes with path in $\mathcal{C}([0,T],\R^{d\times m})$ and let $\bvy^{(n)}_t$ be a semi-martingale with sample path in $\mathcal{C}([0,T],\R^{m})$ respectively. Define function $h_\delta(r)=(1-\delta/r)^+$ and $\tilde{Y}^{(n)}_t = Y^{(n)}_t-\sum_{s\leq t}h_\delta(|\Delta Y^{(n)}_s|)\Delta Y^{(n)}_s$ be the process with reduced jumps. Then $\tilde{Y}^{(n)}_t$ is also a semi-martingale. If the expected quadratic variation of $\tilde{Y}^{(n)}_t$ is bounded uniformly in $n$, and
    $(X_n,Y_n)\to (X,Y)$ in distribution under the uniform metric (\Cref{def: uniform metric}) of $\mathcal{C}([0,T],\R^{d\times m}\times \R^{m})$, then
    \[(X_n,Y_n,\int X_n \dd Y_n)\to (X,Y, \int X \dd Y)\]
    in distribution under the uniform metric of $\mathcal{C}([0,T],\R^{d\times m}\times \R^{m}\times \R^{d})$.    
\end{theorem}
Therefore if $X^{(n)}$ is the solution of an SDE $X^{(n)}_t=X_0 +Z^{(n)}_t + \int_0^t F^{(n)}(X^{(n)}_s) \dd Y^{(n)}_s $, and if the tuple $(F^{(n)}(X_s),Y^{(n)}_s,Z^{(n)}_t)$ converges for all $X_s$ to the process $(F(X_s),Y_s,0)$, then by the above theorem we know the processes $(X^{(n)}, Y^{(n)},Z^{(n)})$ are relative compact, and the solution to the SDE $X_t = X_0 + \int_0^t F(X_s) \dd Y_s$ is the limit of $X^{(n)}_t$, as stated rigorously in Theorem 5.4, \citet{kurtz1991weak}. This will be the main tool in finding the limiting dynamics of SGDM.

\subsection{The Main Results}
We provide a more formal version of \Cref{thm:slow-sde-main}.
\begin{theorem} \label{thm:slow-sde-formal}Fix a compact set $k\subset O_\Gamma$, an initialization $\vx_0\in K$ and $\alpha\in(0,1)$. Consider the SGDM trajectory $(\vx_k^{(n)})$ with hyperparameter schedule $(\eta_k^{(n)}, \beta_k^{(n)})$ scaled by $\eta^{(n)}$, noise scaling $\sigma^{(n)}=1$ and initialization $(\vx_0,\vm_0)$ satisfy \Cref{ass:init-bound}; SGD trajectory $(\vz_k^{(n)})$ with learning rate schedule $(\eta_k^{(n)})$, noise scaling $1$ and initialization $\vz_0=\vx_0$. Furthermore the hyperparameter schedules satisfy \Cref{ass:conve-hpschedule,ass:finite-var}. 
Define the process $\bvx^{(n)}_t= \vx_{\lfloor t/(\eta^{(n)})^2\rfloor}-\phi(\vx_0,t/\eta^{(n)})+\Phi(\vx_0)$ and $\bvz^{(n)}_t= \vz_{\lfloor t/(\eta^{(n)})^2\rfloor}-\phi(\vz_0,t/\eta^{(n)})+\Phi(\vz_0)$, and stopping time $\tau_n=\inf\{t>0: \bvx^{(n)}_t\not\in K\}, \psi_n=\inf\{t>0: \bvz^{(n)}_t\not\in K\}$. Then the processes $(\bvx_{t\wedge\tau_n}^{(n)},\tau_n)$ and $(\bvz_{t\wedge\psi_n}^{(n)},\psi_n)$ are relative compact in $\mathcal{C}([0,T],\R^d)\times[0,T]$ with the uniform metric and they have the same unique limit point $(\bvx_{t\wedge\tau},\tau)$ such that $\bvx_{t\wedge\tau}\in \Gamma$ almost surely for every $t>0$, $\tau=\inf\{t>0: \bvx_t\not\in K\}$, and 
\[\bvx_t = \Phi(\vx_0)+\int_0^t\lambda_t \partial\Phi(X_s)\mSigma^{1/2}(X_s)\dd W_s +\int_0^t\frac{\lambda_t^2}{2}\partial^2\Phi(X_s)[\mSigma(X_s)]\dd s.\]
\end{theorem}

Note that for a sequence $\vx_k$ we are considering the sequence
$\bvx_t= \vx_{\lfloor t/(\eta^{(n)})^2\rfloor}-\phi(\vx_0,t/\eta^{(n)})+\Phi(\vx_0)$ after rescaling time $k=\lfloor t/(\eta^{(n)})^2\rfloor$. This adaptation is due to technical difficulty that the limit of $\vx^{(n)}_k$ will lie on the manifold $\Gamma$ for any $t>0$, but $\vx^{(n)}_0=\vx_0\not\in \Gamma$, so the limiting process will be continuous anywhere but zero. For the process $\bvx^{(n)}_t= \vx_{\lfloor t/(\eta^{(n)})^2\rfloor}-\phi(\vx_0,t/\eta^{(n)})+\Phi(\vx_0)$ however, we know $\bvx^{(n)}_t\to \vx_{\lfloor t/(\eta^{(n)})^2\rfloor}$ for any $t>0$ and  $\bvx^{(n)}_0= \Phi(\vx_{0})\in\Gamma$, thereby we make the limit a continuous process while preseving the same limit for all $t>0$.

\subsubsection{The limit of SGD}
    Let us reparametrize the SGD process $\vz_k$. Define the learning rate $\lambda^{(n)}_t = \frac{\eta^{(n)}_{\lfloor t/(\eta^{(n)})^2\rfloor}}{\eta^{(n)}}$, and the SGD iterates is given by
\begin{align}
    \vg_{k} &= \nabla \Loss(\vx_{k}) + \mSigma^{1/2}(\vx_k)\vxi_{k},\\
    \vz_{k+1} &= \vz_{k} - \eta^{(n)}_{k} \vg_k.\label{equ:sgd-sde-1}
\end{align}
Here we reparameterize the noise $\vv_k=\mSigma^{1/2}(\vx_k)\vxi_k$ so that $\vxi_k$ are independent, $\E\vxi_k=0$ and $\E\vxi_k\vxi_k^\top = \mI$. We also assume that the constants $C_m$ in \Cref{assume:ngos} are defined here as $\E\norm{\vxi_k}^m\leq C_m$. 

Now define the stochastic process $A_n(t) = \sum_{k=0}^{\lfloor t/(\eta^{(n)})^2\rfloor-1} \eta_k^{(n)}$ and $Z_n(t)=\sum_{k=0}^{\lfloor t/(\eta^{(n)})^2\rfloor-1} \eta_k^{(n)}\vxi_k$, then \Cref{equ:sgd-sde-1} can be written as a stochastic integral as
\begin{align}
    \bvz^{(n)}_t &= \vz_{0} -\int_0^t \nabla \Loss(\bvz^{(n)}_t) \dd A_n(t) + \mSigma^{1/2}(\bvz^{(n)}_t)\dd Z_n(t) .\label{equ:sgd-sde-2}
\end{align}
with $\bvz^{(n)}_t = \vz_{\lfloor t/(\eta^{(n)})^2\rfloor}$. Then we can characterize its limiting dynamics.
\begin{theorem}\label{thm:sgd-slow-sde-main}
    The process $\bvz^{(n)}_t$ is a Katzenberger process, and for any $t>0$, $\bvz^{(n)}_t$ converges in distribution to $\bvz_t$ as $n\to\infty$ that
    \[\bvz_t = \Phi(\vz_0) + \int_0^t\lambda_t \partial\Phi(\bvz_s)\mSigma^{1/2}(\bvz_s)\dd W_s +\int_0^t\frac{\lambda_t^2}{2}\partial^2\Phi(\bvz_s)[\mSigma(\bvz_s)]\dd s.\]
\end{theorem}
\begin{proof}
    First we show that $\bvz^{(n)}_t$ is a Katzenberger process. Note that
    
    \begin{itemize}
        \item $A_n$ increases infinitely fast: by \Cref{ass:conve-hpschedule}, for all $s<t$, \[A_n(t)-A_n(s) =  \sum_{k=\lfloor s/(\eta^{(n)})^2\rfloor}^{\lfloor t/(\eta^{(n)})^2\rfloor-1} \eta_k^{(n)}\to \sum_{k=\lfloor s/(\eta^{(n)})^2\rfloor}^{\lfloor t/(\eta^{(n)})^2\rfloor-1} \eta^{(n)} \lambda_{k(\eta^{(n)})^2},
        \]
        \[\sum_{k=\lfloor s/(\eta^{(n)})^2\rfloor}^{\lfloor t/(\eta^{(n)})^2\rfloor-1} \eta^{(n)} \lambda_{k(\eta^{(n)})^2}\geq (\frac{s-t}{\eta^{(n)}}-\eta^{(n)})(\inf_{t\in [0,T]}\lambda_t)\to \infty\]
        as $\eta^{(n)}\to 0$.
        \item $Z_n(t)$ converges to $Z(t)$ that there is a brownian motion $W_t$ and  
        \[Z(t)=\int_0^t \lambda_s \dd W_s.\]
        This is shown with the standard central limit theorem. Let $W_n(t) = \int_0^t \frac{\dd Z(t)}{\lambda_s}$ be the normalized martingale. By the standard central limit theorem (for instance Theorem 4.3.2~\citet{whitt2002stochastic}), $W_n(t)-W_n(s)$ has a limit as a gaussian distribution with variance
        $\sum_{k=\lfloor s/(\eta^{(n)})^2\rfloor}^{\lfloor t/(\eta^{(n)})^2\rfloor-1}(\frac{\eta_k^{(n)}}{\lambda_{k(\eta^{(n)})^2}})^2\to (t-s)$ by \Cref{ass:conve-hpschedule}. Then $W_n(t)$ converges to a bronwian motion $W_t$ by Levy's characterization.
    \end{itemize}
    Therefore $\bvz^{(n)}_t$ is a Katzenberger process, and by \Cref{thm:previous_thm} its limit is given by
    \[\bvz_t = \Phi(\vz_0) + \int_0^t\lambda_t \partial\Phi(\bvz_s)\mSigma^{1/2}(\bvz_s)\dd W_s +\int_0^t\frac{\lambda_t^2}{2}\partial^2\Phi(\bvz_s)[\mSigma(\bvz_s)]\dd s.\]
\end{proof}
\subsubsection{SGDM when $\alpha<1$}
For the SGDM setting, we wish to extract the scale from the hyperparameters to make notations clear. Therefore we define 
\begin{align*}
    \lambda^n_t & = \frac{\eta^{(n)}_k}{\eta^{(n)}} \\
    \gamma^n_t & = \frac{1-\beta^{(n)}_k}{(\eta^{(n)})^\alpha},\\
    k &= \lfloor t/(\eta^{(n)})^2\rfloor.
\end{align*}


Then the original process
\begin{align}
\vm^{(n)}_{k+1} & = \beta^{(n)}_{k}\vm^{(n)}_{k} + (1-\beta^{(n)}_{k})\vg^{(n)}_{k}\\
		\vx^{(n)}_{k+1} & = \vx^{(n)}_{k} - \eta^{(n)}_{k}\vm^{(n)}_{k+1}
\end{align}
can be rewritten into a SDE formulation. Similarly, we reparameterize the noise $\vv_k=\mSigma^{1/2}(\vx_k)\vxi_k$ so that $\vxi_k$ are independent, $\E\vxi_k=0$ and $\E\vxi_k\vxi_k^\top = \mI$.
Let the processes $Z^n_t = \eta^2 \lfloor t/\eta^2 \rfloor$,  and $Y^n_t = \eta \sum_{i=1}^{\lfloor t/\eta^2\rfloor} \vxi_i$. By the previous convention, the process can be rewritten as (let $\eta=\eta^{(n)}$)
\begin{align}\label{equ:sgdm-sde-main}
    \begin{cases}
    \diff X^n_t & = - \frac{ \lambda_t}{\eta} (M^n_t \diff Z^n_t + \eta^2 \diff M^n_t) \\
     & = - \frac{ \lambda_t }{\eta} ((1-\gamma_t\eta^\alpha)M^n_t \dd Z_t^n+\gamma_t\eta^\alpha \nabla \Loss(X^n_t))\diff Z^n_t - \gamma_t \lambda_t\eta^\alpha\mSigma^{1/2}(X^n_t)\diff Y^n_t \\
    \diff M^n_t & = - \frac{\gamma_t}{\eta^{2-\alpha}} ( M^n_t  - \nabla \Loss(X^n_t)) \diff Z^n_t + \frac{\gamma_t}{\eta^{1-\alpha}}\mSigma^{1/2}(X^n_t)\diff Y^n_t
    \end{cases}
\end{align}
Then $X_t^n = \vx^{(n)}_{\lfloor t/\eta^2\rfloor}$ and  $M_t^n = \vm^{(n)}_{\lfloor t/\eta^2\rfloor}$.

Rewriting the second line gives
\begin{equation}
 M^n_t \diff Z^n_t = -\gamma_t^{-1}\eta^{2-\alpha} \diff M^n_t + 
\nabla \Loss(X^n_t) \diff Z^n_t + 
\eta \mSigma^{1/2}(X^n_t)\diff Y^n_t
\end{equation}
So
\begin{equation}
 dX^n_t = -\lambda_t\eta \diff M^n_t  +(\lambda_t/\gamma_t)\eta^{1-\alpha} \diff M^n_t -\lambda_t \eta^{-1}
\nabla \Loss(X^n_t) \diff Z^n_t -\lambda_t \mSigma^{1/2}(X^n_t)\diff Y^n_t
\end{equation}


Now consider the Ito's formula applied to the gradient projected process $ \Phi(X^n_t)$. Fix $K\subset O_\Gamma$ be a compact neighbourhood of the manifold $\Gamma$. Since $\Phi(X)$ is only defined for $X\in O_\Gamma$, we take an arbitrary regular extension of $\Phi$ to the whole space. Fix time horizon $T>0$, and let $\tau_n=\min(\inf\{t>0:X^n_{t}\not\in \mathring{K}\},T)$ to be the exiting time for the compact set $K\subset O_\Gamma$ we have chosen earlier. Since $X^n_0\in \mathring{K}$, we know $\tau_n\geq \eta^2$. We use $\chi^n_t = \mathbf{1}[t<\tau_n]$ to denote the indicator process of the stopping time $\tau_n$. For any c\`{a}dl\`{a}g semi-martingale $X_t$, $X_{t\wedge \tau_n}$ is a c\`{a}dl\`{a}g semi-martingale that $\diff X_{t\wedge \tau_n} = \chi^n_t\diff X_t$.

For simplicity we omit the superscript $n$ unless necessary. Ito's formula on $\Phi(X_t)$ gives
\begin{align*}
     \dd \Phi(X_t) & = \partial \Phi(X_t) \dd X_t + \frac{1}{2}\partial^2 \Phi(X_t) [\dd[X]_t] + \diff\delta\\
     & = \partial \Phi(X_t)((-\lambda_t\eta+ \lambda_t\eta^{1-\alpha}/\gamma_t)\diff M_t - \lambda_t \eta^{-1}\nabla\Loss(X_t)dZ_t -\lambda_t \mSigma^{1/2}(X_t)dY_t)\\
     & + \frac{1}{2}\partial^2 \Phi(X_t) [\dd[X]_t] +\diff\delta
\end{align*}
where $\delta$ are error terms as $\delta(0)=0$ and
\begin{align*}\diff\delta &= \Delta \Phi(X_t) -  
\partial\Phi(X_{t-})
\Delta X_t - \frac{1}{2}\partial^2\Phi(X_{t-})[\Delta X_t(\Delta X_t)^\top]\\
\end{align*}
For $X_t\in K$, there is always $\partial \Phi(X_t)\nabla\Loss(X_t)=0$  by \Cref{lem:gra-proj-1}, so  consider the following process $\Phi_t$: $\Phi_0=\Phi(X_0)$ and 
\begin{align}
     \dd \Phi_t  & =  \partial \Phi(X_t)((-\lambda_t\eta+ \lambda_t\eta^{1-\alpha}/\gamma_t)\diff M_t -\lambda_t \mSigma^{1/2}(X_t)dY_t) + \frac{1}{2}\partial^2 \Phi(X_t) [\dd[X]_t] +\diff\delta
\end{align}
Then $\Phi_{t\wedge \tau_n} = \Phi(X_{t\wedge \tau_n})$. 

In addition, direct calculations gives
\begin{align*}
        [Z]_t & = \eta^2 Z_t,\\
        [Y]_t & = \eta^2 \sum_{i=1}^{\lfloor t/\eta^2\rfloor} \xi_i \xi_i^\top,  \E[Y]_t = Z_t\mI\\
        [Y,Z]_t & = \eta^2 Y_t,\\
        \diff [M]_t & = \gamma_t^2\eta^{2\alpha-2}(M_t-\nabla\Loss(X_t))(M_t-\nabla\Loss(X_t))^\top \diff Z_t + \gamma_t^2\eta^{2\alpha-2}\mSigma^{1/2}(X_t) d[Y]_t \mSigma^{1/2}(X_t) \\
        & - \gamma_t^2\eta^{2\alpha-1}(M_t-\nabla\Loss(X_t))dY_t^\top \mSigma^{1/2}(X_t) - \gamma_t^2\eta^{2\alpha-1} \mSigma^{1/2}(X_t)dY_t (M_t-\nabla\Loss(X_t))^\top,\\
         \diff [M,Z]_t & =  - \gamma_t\eta^\alpha ( M_t  - \nabla \Loss(X_t)) \diff Z_t + \gamma_t\eta^{\alpha+1}\mSigma^{1/2}(X_t)\diff Y_t\\
        d[X]_t & = \lambda_t^2 (M_tM_t^\top  dZ_t + \eta^2 d[M]_t + M_td[Z,M]_t +  d[M,Z]_t M_t^\top)\\
        d[M,X]_t & =   -  \lambda_t (- \gamma_t \eta^{\alpha-1} ( M_t  - \nabla \Loss(X_t))M_t^\top \diff Z_t + \gamma_t\eta^{\alpha}\mSigma^{1/2}(X_t)\diff Y_t M_t^\top) + \eta \diff [M]_t)\\
\end{align*}
\subsubsection{Control of the velocity processes}
%\begin{assumption}
%   The noise process $\xi_i$ and i.i.d. with $\E \xi_i=0$ and $\E\xi_i\xi_i^\top = \mI$. Furthermore, $\xi_i$ is almost surely bounded, e.g. with probability 1,
%   $\norm{\xi_i}\leq B_{\xi}$.
%\end{assumption}
%In this way, the processes $m^n_{t\wedge \tau_n}$ is almost surely bounded for each $n$. 
Notice that our process $X^n_t\in K$ is bounded for $t<\tau_n$. Therefore following regularies, for any continuous function $f:K\to \R$, $f(X^n_{t})$ is bounded $t<\tau_n$. Also, notice that $M^n_{t\wedge \tau_n}$ has bounded moments:
\begin{lemma}
    There exists constants $C^n_m$ such that $\E\norm{M^n_{t\wedge \tau_n}}^m\leq C^n_m$.
\end{lemma}
\begin{proof}
    This follows trivially from the iterate $\vm^{(n)}_{k+1}  = \beta^{(n)}_{k}\vm^{(n)}_{k} + (1-\beta^{(n)}_{k})\vg^{(n)}_{k}$,
    \begin{align*}
        \E\norm{\vm^{(n)}_{k+1}}^m &\leq 2^m \norm{\vm^{(n)}_{k}}^m + 2^m \norm{\vg^{(n)}_{k}}^m\\
        & \leq 2^m \norm{\vm^{(n)}_{k}}^m + 4^m \norm{\nabla\Loss(\vx^{(n)}_k)}^m + 4^m \norm{\vv^{(n)}_k}^m.
    \end{align*}
    The term $\norm{\nabla\Loss(\vx^{(n)}_k)}^m\leq \sup_{\vx\in K}\norm{\nabla\Loss(\vx)}^m$ is bounded and $\norm{\vv^{(n)}_k}^m$ is bounded by \Cref{assume:ngos}. Therefore the lemma follows from the Gr\"{o}nwall inequality \Cref{lem:gronwall-discrete}.
\end{proof}
%\begin{assumption}
%    The momentum decay rate is bounded: $\inf_{t\in [0,T]}\lambda_t\geq \lambda_{\min}>0$ and $\sup_{t\in [0,T]}\lambda_t<\lambda_{\max}=1$. The rate schedule $\gamma_t$ is upper bounded $\sup_{t\in [0,T]}\gamma_t\leq \gamma_{\max}<\infty$. moreover $\gamma_t$ and $\lambda_t$ are cadlag and have finite variation.
%\end{assumption}

\begin{lemma}
    For all stopping time $t$ and function $f>0$, $\int_0^t f(M_s,X_s) \diff Z_s\leq \int_0^t f(M_s,X_s) \diff s $, and $\int_0^t f(M_s,X_s) \diff Z_s\geq \int_0^t f(M_s,X_s) \diff s  - \eta^2 f(M_t,X_t)$.
\end{lemma}
\begin{proof}
    Note that the processes $M_s$ and $X_s$ only changes at jumps at $s=k\eta^2$, the result followed directly from the definition of $Z_s$.
\end{proof}
Next follows some facts are are useful in proving the theorems.
\begin{lemma}[\citet{katzenberger1991solutions} lemma 2.1 ]\label{lem:gronwall-1}
    Let $f,g:\mathbf{R}\to [0,\infty)$ be functions that $g$ is non-decreasing and $g(0)=0$. Assume for constant $C>0$ and all $t>0$
    \[0\leq f(t)\leq C+\int_0^t f(s-)\diff g(s).\]
    Then $f(t)\leq Ce^{g(t)}$ for all $t\geq 0$.    
\end{lemma}

\begin{proof}
    In this case $g(t)\geq 0$. Expansion gives
    \begin{align*}
        f(t) & \leq C(1+\sum_{n=1}^\infty \int_0^t \int_0^{s_1-}\int_0^{s_2-}\cdots \dd g(s_3) \dd g(s_2) \dd g(s_1)\\
        &\leq C(1+\sum_{n=1}^\infty \frac{1}{n!}g^n(t))\\
        & = C e^{g(t)}.
    \end{align*}
\end{proof}
We may also encounter the case where $g$ is negative. Another form of Gronwall inequality is useful here.
\begin{lemma}[Gr\"{o}nwall]\label{lem:gronwall-2}
  Let $f,g:\mathbf{R}\to [0,\infty)$ be functions that $g(0)=0$. Assume for constant $C>0$ and all $0<s<t<T$
    \[ f(t)\leq f(s)- C\int_s^t f(r-)\diff r +g(t)-g(s).\]
    Then $f(t)\leq e^{-C t}(f(0)+g(t)) + C \int_0^t e^{-C(t-s)}(g(t)-g(s))\dd s$ for all $0\leq s\leq T$.   
\end{lemma}

We need a form of Gronwall's inequality with our uncountinuous process $Z_t$
\begin{lemma}\label{lem:gronwall-3}
     Let $f:\R\to [0,\infty)$ be a non-decreasing function and $g:\R\times \R\to [0,\infty)$ be non-negative. Assume for constant $C>0$ and all $0<s<t<T$
    \[ f(t)\leq f(s)- C\int_s^t f(r-)\diff Z^n_r +g(t,s).\]
    Then $f(t)\leq e^{-C' Z_t^n}(f(0)+g(t,0)) + C\int_0^t e^{-C'(Z_t^n-Z_s^n)}g(t,s)\dd Z_s^n$ for all $0\leq s\leq T$ with $C' = \eta^{-2}\log(1+C\eta^2)$.
\end{lemma}
\begin{lemma}
    $\int_0^t e^{C Z_s^n} dZ_s^n = \frac{\eta^2}{e^{C \eta^2}-1} (e^{CZ_t^n}-1)$.
\end{lemma}
\begin{proof}
    Directly calculation that $\sum_{i=0}^n c^i = \frac{1-c^{n+1}}{1-c}$ for $c=e^{C\eta^2}$.
\end{proof}
\begin{proof}[Proof of \Cref{lem:gronwall-3}]
    multiplying both sides with $e^{-C'(Z_t^n-Z_s^n)}$ and integration yields the result.
\end{proof}
Specifically for $g(t,s)=Z_t^n-Z_s^n$, the bound can be further simlified.
\begin{lemma}\label{cor:gronwall-3}
    $\int_0^t e^{-C(Z_t^n-Z_s^n)}(Z_t^n - Z_s^n)\dd Z_s^n \leq C^{-2}$.
\end{lemma}
\begin{proof}
    \[\int_0^t e^{-C(Z_t^n-Z_s^n)}(Z_t^n - Z_s^n)\dd Z_s^n \leq \int_0^{\infty} e^{-C z} z \dd z =C^{-2}.\]
\end{proof}
Another useful theorem is the Doob's martingale inequality.
\begin{lemma}\label{lem:Doob-1}
    Let $X_t$ be a martingale for $t\in[0,T]$ whose sample path is almost surely right-continuous. Then for any $C>0$, and $p\geq 1$,
    \[\Pr[\sup_{t\in [0,T]}|X_t|\geq C]\leq \frac{\E |X_T|^p}{C^p}.\]
    Furthermore, integration with $p=2$ gives
    \[\E \sup_{t\in [0,T]}|X_t|\leq 2\sqrt{\E |X_T|^2}.\]
\end{lemma}
\iffalse
\begin{lemma}
    For all $t\in [0,T]$,
        \[\sup_n \int_0^{t} \eta^{-\alpha}\E\norm{m_{s\wedge \tau_n}^n}^2 \dd s< + \infty.\]
\end{lemma}
\begin{proof}
    Consider the energy function $G^n_t(X^n_t,M^n_t) = \Loss(X^n_t) + \frac{\gamma_t}{2}(\eta^{1-\alpha} - \lambda_t\eta)\norm{M^n_t}^2$,  there is
    \begin{align*}\diff G^n_t(X^n_t,M^n_t) & = \dotp{\nabla\Loss(X^n_t)}{\diff X^n_t} + \frac{1}{2}\tr{\nabla^2\Loss(X^n_t)\diff [X^n_t]}\\
    & + \Delta \Loss(X_t) -  
    \partial\Loss(X_{t-})
    \Delta X_t - \frac{1}{2}\partial^2\Loss(X_{t-})[\Delta X_t(\Delta X_t)^\top]\\
    & + \frac{1}{2}\eta^{1-\alpha}\norm{M^n_t}^2 \frac{\dd \gamma_t}{\dd t} \dd t + \gamma_t \eta^{1-\alpha}\dotp{M_t^n}{\ddM_t^n} + \frac{\gamma_t}{2}\eta^{1-\alpha}\tr \dd[M_t^n]\\
    & \leq -\eta\lambda_t\gamma_t\dotp{\nabla\Loss(X^n_t)}{ \ddM_t^n} +  \frac{1}{2}\tr{\nabla^2\Loss(X^n_t)\diff [X^n_t]}\\
    & + \Delta \Loss(X_t) -  
    \partial\Loss(X_{t-})
    \Delta X_t - \frac{1}{2}\partial^2\Loss(X_{t-})[\Delta X_t(\Delta X_t)^\top]\\
    & + \frac{1}{2}\eta^{1-\alpha}\norm{M^n_t}^2 \frac{\dd \gamma_t}{\dd t} \dd t - \gamma_t\lambda_t \eta^{-1} \norm{M_t}^2\dd Z^n_t + \frac{\gamma_t}{2}\eta^{1-\alpha}\tr \dd[M_t^n]\\
    & + \gamma_t\lambda_t \dotp{M_t}{\mSigma^{1/2}(X_t)\dd Y_t}.\\
    & \leq - \frac{1}{2}C\tr\diff [X^n_t]- \gamma_t\lambda_t \eta^{-1} \norm{M_t}^2\dd Z^n_t + C \norm{\Delta X_t^n}^3\\
    & + \frac{1}{2}\eta^{1-\alpha}\norm{M^n_t}^2 \frac{\dd \gamma_t}{\dd t} \dd t  + \frac{\gamma_t}{2}\eta^{1-\alpha}\tr \dd[M_t^n]\\
    & + \gamma_t\lambda_t \dotp{M_t}{\mSigma^{1/2}(X_t)\dd Y_t}.\\
    \end{align*}
    
\end{proof}
\fi
\begin{lemma}\label{lem:momentum-1}
    For all $t>0$ and $\alpha\in [0,1)$, 
    \begin{itemize}
    \item $\lim_{n\to\infty}\eta^{\beta}\E\sup_{t\in [0,T]}\norm{M_{t\wedge\tau_n}^n}^2=0$ for any $\beta>1-\alpha$. \\
    moreover, $\lim_{n\to\infty}\eta^{\beta/2}\E\sup_{t\in [0,T]}\norm{M_{t\wedge\tau_n}^n}=0$
    \item $\lim_{n\to\infty}\eta^{\beta}\E\int_0^{t\wedge \tau_n} \norm{M_{t\wedge\tau_n}^n}^2 \dd Z^n_s=0$ for any $\beta>0$.\\
    moreover, $\sup_n \E\int_0^{t\wedge \tau_n} \norm{M_{t\wedge\tau_n}^n}^2 \dd Z^n_s<\infty$.
\end{itemize}
For $\alpha\in (0,1)$,
\begin{itemize}
    \item $\lim_{n\to\infty}\eta^{\beta} \E \int_0^{t\wedge \tau_n} \norm{M_s^n}^3 \dd Z^n_s=0$ for any $\beta>0$.
    \end{itemize}
\end{lemma}
\begin{proof}
     Ito's formula on $\norm{M_t^n}^2$ gives 
      \begin{align*}
        d\norm{M_t^n}^2 & = 2\dotp{M_t^n}{dM_t^n} +  \tr d[M^n]_t \\
        & = - 2\gamma_t\eta^{\alpha-2} ( \norm{M^n_t}^2  - (M^n_t)^\top\nabla \Loss(X^n_t)) \diff Z^n_t + 2\gamma_t\eta^{\alpha-1}  (M^n_t)^\top\mSigma^{1/2}(X^n_t)\diff Y^n_t  \\
        & +  \gamma_t^2\eta^{2\alpha-2}\norm{M^n_t-\nabla\Loss(X^n_t)}^2 \diff Z^n_t + \gamma_t^2\eta^{2\alpha-2} \tr (\mSigma(X^n_t)\dd[Y^n]_t)  \\
        & - 2\gamma_t^2\eta^{2\alpha-1}\tr ((M^n_t-\nabla\Loss(X^n_t))^\top\mSigma^{1/2}(X^n_t)\dd Y^n_t) \\
        & = \eta^{2\alpha-2} \dd W_t^n   - 2\gamma_t\eta^{\alpha-2} ( \norm{M^n_t}^2  - (M^n_t)^\top\nabla \Loss(X^n_t)) \diff Z^n_t\\
        & +  \gamma_t^2\eta^{2\alpha-2}\norm{M^n_t-\nabla\Loss(X^n_t)}^2 \diff Z^n_t + \gamma_t^2\eta^{2\alpha-2} \tr (\mSigma(X^n_t))\dd Z^n_t .
    \end{align*}
    Let $W^n_t$ be the martingale that $W_0^n=0$ and
    \begin{align*}\dd W^n_t & = 2\gamma_t\eta^{1-\alpha}  (M^n_t)^\top\mSigma^{1/2}(X^n_t)\diff Y^n_t + \gamma_t^2 \tr (\mSigma(X^n_t) (\dd[Y^n]_t - \mI\dd Z^n_t))\\
    & - 2\gamma_t^2\eta(M^n_t-\nabla\Loss(X^n_t))^\top\mSigma^{1/2}(X^n_t)\dd Y^n_t.
    \end{align*}


    When $\alpha>0$, take the constant $C_1 = 2\sup_{X\in K}\norm{\nabla\Loss(X)}(2+\norm{\nabla\Loss(X)}) $ and $C_2 = \sup_{X\in K} \tr\mSigma(X) + C_1$. Since $\eta^{-\alpha}\norm{M_t^n}\leq (\eta^{-1}+\eta^{1-2\alpha}\norm{M_t^n}^2)/2$, for any $t>s>0$ there is
    \begin{align*}\eta^{\beta}\norm{M_{t\wedge \tau_n}^n}^2  & \leq  \int_{s\wedge \tau_n}^{t\wedge \tau_n}  (- 2\lambda_{\min}\eta^{\beta+\alpha-2} \norm{M^n_r}^2  + C_1\eta^{\beta+\alpha-2}\norm{M^n_r} +  C_1\eta^{\beta+2\alpha-2}(\norm{M^n_r}^2+1))\diff Z^n_r \\
    & + \eta^{\beta}\norm{M_{s\wedge \tau_n}^n}^2 + \eta^{\beta+2\alpha-2}\int_{s\wedge \tau_n}^{t\wedge \tau_n} \lambda_r^2 \tr (\mSigma(X^n_r)) \dd Z^n_r + \eta^{\beta+2\alpha-2}(W^n_{t\wedge \tau_n} - W^n_{s\wedge \tau_n})\\
    & \leq  \eta^{\beta}\norm{M_{s\wedge \tau_n}^n}^2 + \int_{s\wedge \tau_n}^{t\wedge \tau_n}  (- 2\lambda_{\min}\eta^{\alpha-2}+\frac{1}{2} C_1 \eta^{-1}+C_1 \eta^{2\alpha-2} )\eta^{\beta}\norm{M^n_r}^2 \diff Z^n_r\\
    & +\eta^{2\alpha+\beta-3} C_2 (Z_{t\wedge \tau_n}^n-Z_{s\wedge \tau_n}^n) + \eta^{2\alpha+\beta-2}(W^n_{t\wedge \tau_n} - W_{s\wedge \tau_n}^n)\\
    \end{align*}
    By the Doob's inequality \Cref{lem:Doob-1}, 
    \begin{align*} W^{s,n}_{t}  & = W_{t\wedge \tau_n}^n-W^n_{s\wedge \tau_n} \\ & =  
    \int_{s\wedge \tau_n}^{t\wedge \tau_n}2\gamma_r\eta^{1-\alpha}  (M^n_r)^\top\mSigma^{1/2}(X^n_r)\diff Y^n_r + \gamma_r^2 \tr (\mSigma(X^n_r)(\dd[Y^n]_r - \mI\dd Z^n_r))\\
    & - 2\gamma_r^2\eta(M^n_r-\nabla\Loss(X^n_r))^\top\mSigma^{1/2}(X^n_r)\dd Y^n_r
    \end{align*}
    is a martingale, so there is a universal constant $A$ that
    \begin{align*}
        \E\sup_{r\in [s,t]}|W_r^{s,n}| & \leq 2\sqrt{\E |W_t^{s,n}|^2}\\
        & \leq 2A\sqrt{Z^n_{t\wedge \tau_n}-Z^n_{s\wedge \tau_n}+\eta^{2-2\alpha}\int_{s\wedge \tau_n}^{t\wedge \tau_n}\norm{M_r^n}^2 \dd Z_r^n}\\
        & \leq 2A\eta^{-1}(Z^n_{t\wedge \tau_n}-Z^n_{s\wedge \tau_n}+\eta^{2-2\alpha}\int_{s\wedge \tau_n}^{t\wedge \tau_n}\norm{M_r^n}^2 \dd Z_r^n)
    \end{align*}
    Therefore let $K^n_t = \eta^{\beta}\E\sup_{s\in[0,t]}\norm{M^n_{s\wedge\tau_n}}^2$,
    \begin{align*}
        K^n_t\leq K^n_s + \int_s^t \kappa_n K_r^n \dd Z_r^n+ (2A+ C_2) \eta^{2\alpha+\beta-3}(Z_{t}^n-Z_{s}^n)
    \end{align*}
    Here $\kappa_n= - 2\lambda_{\min}\eta^{\alpha-2}+\frac{1}{2} C_1 \eta^{-1}+C_1 \eta^{2\alpha-2}+2A\eta^{-1}$, then when $\eta\to 0$ eventually $\kappa_n<0$. Then by the Gronwall inequality \Cref{lem:gronwall-3}, with $\kappa'_n=\eta^{-2}\log(1+|\kappa_n|\eta^2)$,
    \begin{align*}
        K^n_t & \leq \eta^{2\alpha+\beta-3}(2A+ C_2)[e^{-\kappa'_n Z_t^n}Z_t^n + \int_0^t |\kappa_n|e^{-\kappa'_n (Z_t^n-Z_s^n)}(Z_t^n-Z_s^n) \dd Z_s^n]\\
    \end{align*}
    By \Cref{cor:gronwall-3}, \[K^n_t \leq \eta^{2\alpha+\beta-3}(2A+ C_2)(e^{-\kappa'_n Z_t^n}Z_t^n + |\kappa_n|(\kappa'_n)^{-2})\]
    Taking the limit gives $\lim_{n\to\infty} K_t^n=\tilde{O}(\eta^{3\alpha+\beta-1})=0$.

    For $\alpha=0$, there is
     \begin{align*}
        \eta^2 d\norm{M_t^n}^2 & = \dd W_t^n   - 2\gamma_t( \norm{M^n_t}^2  - (M^n_t)^\top\nabla \Loss(X^n_t)) \diff Z^n_t\\
        & +  \gamma_t^2\norm{M^n_t-\nabla\Loss(X^n_t)}^2 \diff Z^n_t + \gamma_t^2 \tr (\mSigma(X^n_t) )\dd Z^n_t\\
        & \leq \dd W_t^n-\gamma_t\norm{M^n_t}^2 \diff Z^n_t +  \norm{\nabla\Loss(X^n_t)}^2 \diff Z^n_t + \tr (\mSigma(X^n_t) )\dd Z^n_t.
    \end{align*}
    For $K^n_t = \eta^{\beta}\E\sup_{s\in[0,t]}\norm{M^n_{s\wedge\tau_n}}^2$, let $\iota_n = -\lambda_{\min}\eta^{-2} + 2A\eta^{-1}$ and some constant $C_3$, there is
    \[K^n_t\leq K^n_s + \int_s^t \iota_n K_r^n \dd Z_r^n + C_3 \eta^{\beta-3}_n (Z^n_{t}-Z^n_{s})\]
    Similarly by \Cref{lem:gronwall-3} and \Cref{cor:gronwall-3}, $\lim_{n\to \infty}K_t^n = O(\eta^{\beta-1})=0$.
    
    For any jump $\Delta M_t^n$, and $f(M_t^n)=\norm{M_t^n}^3$ there is $\theta\in [0,1]$ that $M = \theta M_{t-}^n + (1-\theta)M_t^n$ and
    \begin{align*}
         & \Delta f(M_t^n) - \dotp{\partial f(M_{t-}^n)}{M_t^n} - \frac{1}{2}\partial^2 f(M_{t-}^n)[\Delta M_t^n (\Delta M_t^n)^\top] \\
         = & \frac{1}{6}\partial^3 f(M)[\Delta M_t^n,\Delta M_t^n,\Delta M_t^n ] \\
         = & -\frac{1}{2} \frac{\dotp{M}{\Delta M_t^n}^3}{\norm{M}^3} + \frac{3}{2}\frac{\dotp{M}{\Delta M_t^n}}{\norm{M}}\\
         & \leq 2\norm{\Delta M_t^n}^3.
    \end{align*}
    Ito's formula on $\norm{M_t^n}^3$ gives
    \begin{align*}
        d\norm{M_t^n}^3 & \leq 3\norm{M_t^n} \dotp{M_t^n}{dM_t^n} + \frac{3}{2}(\norm{M_t^n} \dd \tr[M_t^n] + \norm{M_t^n}^{-1} \tr (M_t^n(M_t^n)^\top d[M_t^n])) + 2\norm{\Delta M_t^n}^3\\
    \end{align*}
    At $t=k\eta^2$, there is a jump for the process $M_t^n$ as $\Delta M_t^n=\gamma_t\eta^\alpha(-M^n_{t-}+\nabla\Loss(X^n_{t-})+\mSigma^{1/2}(X^b_{t-})\xi_k)$, so for constant $C_4=12\sup_{X\in K}(\norm{\nabla\Loss(X)}+1+\norm{\mSigma^{1/2}(X)})^3$
    \begin{align*}
        \E\norm{\Delta M_{t\wedge\tau_n}^n}^3\leq C_4\eta^{3\alpha-2}(\norm{M^n_{t\wedge\tau_n}}^3+1)\dd Z_{t\wedge\tau_n}^n
    \end{align*}
    When $\alpha>0$, let $J_t^n = \eta^{\beta}\E \norm{M_t^n}^3$, as $\eta^{\beta+\alpha-2}\norm{M_t^n}\leq \eta^{3\beta/2+\alpha-2}\norm{M_t^n}^2 + \eta^{\beta/2+\alpha-2}$,there is
    \begin{align*}
        J_t^n \leq J_s^n - \int_s^t (3\lambda_{\min}\eta^{\alpha-2}- \eta^{\alpha-2+\beta/2}C_5)J_r^n \dd Z_r^n + C_5\eta^{\alpha-2+\beta/2}(Z^n_t - Z^n_s)
    \end{align*}
    where $C_5=3C_4+3(\sup_{t\leq T}|\lambda_t|)(1+\sup_{X\in K}\max(\norm{\nabla \Loss(X)}))$ is some universal constant. By \Cref{lem:gronwall-3} and \Cref{cor:gronwall-3}, there is
    \begin{align*}
        J_t^n \leq C_5 Z^n_t O(\eta^{\beta/2}).
    \end{align*}
    And the conclusion follows.
\end{proof}
\begin{lemma}
    There exist a universal constant $C$ such that
    \begin{itemize}
        \item $\norm{\Delta M_{t\wedge \tau_n}^n}\leq C\eta^{\alpha}(\norm{M_{t\wedge \tau_n}^n}+1)$.
        \item $\norm{\Delta X_{t\wedge \tau_n}^n}\leq C\eta(\norm{M_{t\wedge \tau_n}^n}+\eta^{\alpha}).$
    \end{itemize}
\end{lemma}
\begin{proof}
    Direct from the iterations \Cref{equ:sgdm-sde-main}.
\end{proof}
\begin{lemma}\label{lem:remn-term}
    When $n\to \infty$ (or $\eta\to 0$),  $\delta(t\wedge \tau_n)\to 0$ weakly for all $\alpha\in (0,1)$ and $t\in [0,T]$.
\end{lemma}
We wish to generalize the result a little bit.
\begin{lemma}\label{lem:remn-term-1}
    For any function $f\in C^3(K)$, as $n\to 0$,
   \[ \E\sup_{t\in [0,\tau_n\wedge T]} \sum_{s\in [0,t]}\left|\Delta f(X^n_s) -  
\partial f(X^n_{s-})
\Delta X^n_s - \frac{1}{2}\partial^2f (X^n_{s-})[\Delta X^n_s, \Delta X^n_s]\right|\to 0\]
\end{lemma}
\begin{proof}
    $\Delta X^n_s$ and $\Delta f(X^n_s)$ are non-zero at times $s=k\eta^2$ for $k\in\Z^+$. By the mean value theorem, there is $\theta\in [0,1]$ that
    \begin{align*} & |\Delta f(X^n_s) -\partial f(X^n_{s-})
\Delta X^n_s - \frac{1}{2}\partial^2f (X^n_{s-})[\Delta X^n_s, \Delta X^n_s]| \\
 = & \frac{1}{6} |\partial^3 f (\theta X^n_{s-} + (1-\theta)X^n_{s})[\Delta X^n_s, \Delta X^n_s,\Delta X^n_s]|\\
 \leq & \frac{1}{6} (\sup_{X\in K}\norm{\partial^3 f (X)}_F) \norm{\Delta X^n_s}^3.
 \end{align*}
Here the norm is defined for tensors as $\norm{\partial^3 f (X)}_F = \sqrt{\sum_{ijk}(\partial_i\partial_j\partial_k f (X))^2}$. The first term $\sup_{X\in K}\norm{\partial^3 f (X)}_F$ is a constant independent of $n$ (as $K$ is compact). Notice that 
\[\Delta X^n_s=-\lambda_s\eta (M_{s-}^n + \Delta M_s^n) =-\lambda_s\eta M_{s}^n.\]
Therefore there exists a constant $C=\frac{1}{6} (\sup_{X\in K}\norm{\partial^3 f (X)}_F)(\sup_{t\in [0,T]}(\lambda_t)^3)$ that 
\begin{align*}
& \E\sup_{t\in [0,\tau_n\wedge T]} \sum_{s\in [0,t]}\left|\Delta f(X^n_s) -  
\partial f(X^n_{s-})
\Delta X^n_s - \frac{1}{2}\partial^2f (X^n_{s-})[\Delta X^n_s, \Delta X^n_s]\right|\\
\leq & C \cdot  \E \sup_{t\in [0,\tau_n\wedge T]} \eta^3 \sum_{s=k\eta^2\in [0,t]}\norm{M_s^n}^3\\
\leq & C \cdot  \E \eta \int_0^{T\wedge\tau_n}\norm{M_s^n}^3 \diff Z^n_t.
\end{align*}

From \Cref{lem:momentum-1} we know $\eta \int_0^{T\wedge\tau_n}\E\norm{M_s^n}^3 \diff Z^n_t\to 0$, so the proof is done.
\end{proof}
\begin{proof}[Proof for \Cref{lem:remn-term}]
    The result follows by applying \Cref{lem:remn-term-1} to every coordinate of $\Phi(X_t^n)$.
\end{proof}
\begin{lemma} For $\alpha\in (0,1)$,
    $\lim_{n\to\infty}\eta^{\beta}\E\int_0^{t\wedge \tau_n} \norm{M_{t\wedge\tau_n}^n}^2 \dd Z^n_s=0$ for any $\beta>-\alpha$.\\
    moreover, $\sup_n \eta^{-\alpha}\E\int_0^{t\wedge \tau_n} \norm{M_{t\wedge\tau_n}^n}^2 \dd Z^n_s<\infty$\label{lem:m-bound-tight}
\end{lemma}
\begin{proof}
    Consider the energy function $G(X_t,M_t)=2\frac{\gamma_t}{\lambda_t}\Loss(X_t) + \eta^{1-\alpha} \norm{M_t}^2$, there is
    \begin{align*}
        \E G(X_t,M_t) & = \E G(X_0,M_0) + \E\int_0^t 2\frac{\gamma_t}{\lambda_t}\nabla\Loss(X_t) dX_t +2 d(\frac{\gamma_t}{\lambda_t})(\Loss(X_t)+\Delta\Loss(X_t)) \\
        & + \frac{\gamma_t}{\lambda_t}  \nabla^2\Loss(X_t) d[X]_t +d\delta\\
         & - 2\gamma_t\eta^{-1} ( \norm{M^n_t}^2  - (M^n_t)^\top\nabla \Loss(X^n_t)) \diff Z^n_t +   \\
        & +  \gamma_t^2\eta^{\alpha-1}\norm{M^n_t-\nabla\Loss(X^n_t)}^2 \diff Z^n_t + \gamma_t^2\eta^{\alpha-1} \tr (\mSigma(X^n_t) \dd[Y^n]_t)  \\
        & = G(X_0,M_0) +  A_t - \int_0^t 2\gamma_t\eta^{-1}  \E\norm{M^n_t}^2  \diff Z^n_t    \\
        & +   \int_0^t\gamma_t^2\eta^{\alpha-1}\E\norm{M^n_t-\nabla\Loss(X^n_t)}^2 \diff Z^n_t + \gamma_t^2\eta^{\alpha-1} \tr (\mSigma(X^n_t) \dd[Y^n]_t) 
    \end{align*}
    Here $A_t$ is some uniformly bounded process. Multiply both sides by $\eta^{1-\alpha}$ gives 
    \begin{align*}
        \int_0^t 2\gamma_t\eta^{-\alpha}  \E\norm{M^n_t}^2  \diff Z^n_t & \leq \eta^{1-\alpha}\E G(X_0,M_0) + \eta^{1-\alpha} A_t \\
        & +   \int_0^t\gamma_t^2\E\norm{M^n_t-\nabla\Loss(X^n_t)}^2 \diff Z^n_t + \gamma_t^2 \tr (\mSigma(X^n_t) \dd[Y^n]_t).
    \end{align*}
    From \Cref{lem:momentum-1} we know the right-hand-side is uniformly bounded in $n$, and the conclusion follows.
\end{proof}

\subsubsection{Convergence to the Manifold}
We wish to show that the process $X^n_{t\wedge\tau_n}\to \Phi^n_{t\wedge\tau_n}$ as $n\to\infty$ for any $t>0$. 
First we need to show that as the learning rate $\eta\to 0$, there is a distance function $d$ that $d(\Phi_t, \Gamma)\to 0$ weakly as a stochastic process. 
\begin{lemma}\label{lem:conver-to-mani}
    As $n\to \infty$, $d(\Phi_t, \Gamma)\to 0$ weakly for all $\alpha\in (0,1)$.
\end{lemma}
\begin{proof}
    By \Cref{lem:katzen-results}, we need to prove that for all $T>0$, $\sup_{t\leq T} h(X_{t\wedge \tau_n})\to 0$ in probability.

    Ito's formula on $h(X_{t\wedge \tau_n})$ gives for some process $\delta\to 0$ ($n\to \infty$)
    \begin{align*}
        dh(X_t) & = \dotp{\nabla h(X_t)}{dX_t} + \frac{1}{2} \nabla^2 h(X_t) d[X]_t + d\delta \\
        & = \dotp{\nabla h(X_t)}{-\lambda_t\eta \diff M^n_t  +(\lambda_t/\gamma_t)\eta^{1-\alpha} \diff M^n_t -\lambda_t \eta^{-1}
\nabla \Loss(X^n_t) \diff Z^n_t -\lambda_t \mSigma^{1/2}(X^n_t)\diff Y^n_t}\\
&+ \frac{1}{2} \nabla^2 h(X_t) d[X]_t + d\delta\\
    \end{align*}
    Let the process 
    \[dS_t = \dotp{\nabla h(X_t)}{-\lambda_t\eta \diff M^n_t  +(\lambda_t/\gamma_t)\eta^{1-\alpha} \diff M^n_t -\lambda_t \mSigma^{1/2}(X^n_t)\diff Y^n_t}
+ \frac{1}{2} \nabla^2 h(X_t) d[X]_t + d\delta,\] there is
\begin{align*}
    dh(X_t) & = dS_t -\lambda_t\eta^{-1}\dotp{\nabla h(X_t)}{\nabla\Loss(X_t)}dZ_t\\
    & \leq dS_t - \lambda_{\min} c^{-1} \eta^{-1} h(X_t)dZ_t
\end{align*}
Therefore by \Cref{lem:gronwall-3}, for $\kappa = \eta^{-2}\log(1+\lambda_{\min} c^{-1} \eta )$, there is
\[h(X_t)\leq e^{-\kappa Z_t}S_t + (\lambda_{\min} c^{-1} \eta^{-1})\int_0^t e^{-\kappa (Z_t-Z_s)}(S_t-S_s) dZ_s\]
Clearly $e^{-\kappa Z_t}S_t\to 0$. Furthermore we have for 
\[A_t = \int_0^{t\wedge\tau_n} (-\lambda_t\eta +(\lambda_t/\gamma_t)\eta^{1-\alpha})\dotp{\nabla h(X_t)}{\diff M^n_t}\]
\[B_t = \int_0^{t\wedge\tau_n} \dotp{\nabla h(X_t)}{ -\lambda_t \mSigma^{1/2}(X^n_t)\diff Y^n_t}\]
\[C_t = \int_0^{t\wedge\tau_n}\frac{1}{2} \nabla^2 h(X_t) d[X]_t + d\delta\]
we know $S_t = A_t+B_t +C_t$ .

First, we show $\eta^{-1}\int_0^te^{-\kappa (Z_t-Z_s)}(A_t-A_s)dZ_s\to 0$. as $A_t-A_s\leq \int_s^t K [\eta^{-1}(\norm{M_r} + 1)dZ_r + dY_r]$ for some constant $K$, we know by \Cref{cor:gronwall-3} and \Cref{lem:momentum-1},
\[\eta^{-2}\sup_r\norm{M_r}\int_0^te^{-\kappa (Z_t-Z_s)}(Z_t-Z_s)dZ_s\leq \eta^{-2}{\kappa}^{-2}\sup_r\norm{M_r} \leq \eta^{2}\sup_r\norm{M_r}\to 0\]
and since $\E\sup_{r}\norm{Y_t-Y_s}\leq 2\sqrt{\E \norm{Y_t}^2}\leq K t$, there is
$\E\sup_n\eta^{-1}|\int_0^te^{-\kappa (Z_t-Z_s)}(Y_t-Y_s)dZ_s|\leq \frac{Kt}{\kappa\eta}\to 0$.

Next, we show $\eta^{-1}\int_0^te^{-\kappa (Z_t-Z_s)}(B_t-B_s)dZ_s\to 0$. $B_t-B_s$ is a martingale so by Doob's inequality, $\E\sup_{r}|B_t-B_r|\leq 2\sqrt{\E |B_t|^2}\leq K t$ for some constants $K$. Therefore $\E\sup_n\eta^{-1}|\int_0^te^{-\kappa (Z_t-Z_s)}(B_t-B_s)dZ_s|\leq \frac{Kt}{\kappa\eta}\to 0$.

Finally, there exists constant $K$ such that $\eta^{-1}\int_0^te^{-\kappa (Z_t-Z_s)}(C_t-C_s)dZ_s \leq \eta^{-1}\int_0^te^{-\kappa (Z_t-Z_s)}K(Z_t-Z_s)dZ_s \leq \frac{K}{\eta\kappa^2}\to 0$ by \Cref{cor:gronwall-3}. Therefore we concludes the proof by showing that $h(X_t)\to 0$.
\end{proof}
\subsection{Averaging}
\begin{lemma}\label{lem:ave-1}
    $\lim_{n\to\infty} \eta^{\beta}\E \sup_{t\leq T}|\int_0^{t\wedge\tau_n}\partial^2\Phi(X_s^n)[\nabla\Loss(X_s^n)(M_s^n)^\top]\dd Z_s^n|= 0$ for any $\beta>-1$.
\end{lemma}
To prove the result we need another lemma.
\begin{lemma}\label{lem:ave-2}
    $\lim_{n\to\infty} \eta^{\beta}\E \sup_{t\leq T}|\int_0^{t\wedge\tau_n}\norm{\nabla\Loss(X_s^n)}^2\dd Z_s^n|= 0$ for any $\beta>-1$.
\end{lemma}
\begin{proof}
    Use Ito on $\dotp{\nabla\Loss(X_t)}{M_t}$, there is
    \begin{align*}
         d\dotp{\nabla\Loss(X_t)}{M_t} & = -\lambda_t \partial^2\Loss(X_t)[ \eta^{-1} M_t M_t^\top dZ_t+ \eta M_t dM_t^\top] + \dotp{\Delta\nabla\Loss(X_t) }{\Delta M_t}\\
         & +\gamma_t\nabla\Loss(X_t)[\eta^{-2+\alpha}(\nabla\Loss(X_t) - M_t) dZ_t +\eta^{-1+\alpha}\mSigma^{1/2}(X_t) dY_t ].
    \end{align*}
    Therefore we know there exists constant $K$ such that
    \begin{align*}
        \E\int_0^{t\wedge\tau_n}\norm{\nabla\Loss(X_s)}^2 dZ_s & \leq K\eta^{2-\alpha}\E(\dotp{\nabla\Loss(X_{t\wedge\tau_n})}{M_{t\wedge\tau_n}}-\dotp{\nabla\Loss(X_0)}{M_0})\\
        & + \eta^{1-\alpha}K\E\int_0^{t\wedge\tau_n} \norm{M_s}^2 dZ_s + K\eta^{3}\E\sum_{\Delta M_t\neq 0} \norm{ M_t}^2\\
        & + \E\int_0^{t\wedge\tau_n}\dotp{\nabla\Loss(X_t)}{M_t}dZ_t. 
    \end{align*}
    The first four terms vanishes when multiplied $\eta^{\beta}$ for $\beta>-1$ by \Cref{lem:m-bound-tight}. Note the last term
    \begin{align*}
        \int_0^{t\wedge\tau_n}\dotp{\nabla\Loss(X_t)}{M_t}dZ_t & = \int_0^{t\wedge\tau_n}\dotp{\nabla\Loss(X_t)}{-\eta \lambda_t^{-1} dX_t -\eta^2 dM_t}
    \end{align*}
    so 
    \[\lambda_t^{-1}\dotp{\nabla\Loss(X_t)}{ dX_t}=d(\lambda_t^{-1}\Loss(X_t))- (\Delta\lambda_t^{-1})(\Delta \Loss(X_t)+\Loss(X_t)) -\frac{1}{2}\lambda_t^{-1}\partial^2\Loss(X_t)[d[X]_t] - d\delta.\] 
    $\lambda_t^{-1}\dotp{\nabla\Loss(X_t)}{ dX_t}$ is clearly a bounded process given the bounded variation of $\lambda_t^{-1}$ and boundedness of $X_{t\wedge \tau_n}$. Thereby we finished the proof.
\end{proof}
\begin{proof}[Proof of \Cref{lem:ave-1}]
    Let $f(X_s)=\partial^2\Phi(X_s)\nabla\Loss(X_s)$.
    
    Ito's formula on $\partial^2\Phi(X_s)[\nabla\Loss(X_s)(M_s)^\top] = f(X_s)M_s$ gives
    \begin{align*}
        d(f(X_s)M_s) & =df(X_s)M_s + f(X_s) dM_s  + \Delta f(X_s) \Delta M_s\\ 
        & = df(X_s) M_s + \Delta f(X_s) \Delta M_s\\
        & -\gamma_s\eta^{\alpha-2} f(X_s)(M_s-\nabla \Loss(X_s))dZ_s + \gamma_s\eta^{\alpha-1} f(X_s)\mSigma^{1/2}(X_s) dY_s .
    \end{align*}
    Therefore there is constant $K$ such that
    \begin{align*}
        & \int_0^t f(X_s) M_s dZ_s\leq \int_0^t f(X_s)\nabla\Loss (X_s)dZ_s\\
   & + K(\int_0^t \eta^{2-\alpha}df(X_s)M_s + \eta f(X_s)\mSigma^{1/2}(X_s) dY_s+\eta^{2-\alpha}\sum \Delta f(X_s) \Delta M_s)
    \end{align*}
    We know that $\eta^{2-\alpha}\sum \Delta f(X_s) \Delta M_s=O(\eta)$ and $\sup_t\int_0^t\eta f(X_s)\mSigma^{1/2}(X_s) dY_s = O(\eta)$. Expansion gives $\int_0^t \eta^{2-\alpha}df(X_s)M_s = O(\eta)$ by \Cref{lem:m-bound-tight}. Finally by \Cref{lem:ave-2} we obtain the desired result.
\end{proof}
Let $\phi_t =  \lambda_t/\gamma_t-\eta^\alpha\lambda_t$.
\begin{lemma}\label{lem:aver-1}
$\int_{c_1}^{c_2} [\eta^{1-\alpha}\phi_t \partial\Phi(X_t) \dd M_t+\phi_t\lambda_t(\eta^{-\alpha} - \lambda_t)\partial^2\Phi(X_t) [M_t M_t^\top]\dd Z_t \to 0$ as $\eta\to 0$.
\end{lemma}
\begin{proof}
     Ito's formula on $\eta^{1-\alpha}\phi_t\partial\Phi(X_t)M_t$ gives
    \begin{align*}
        \dd(\phi_t\partial\Phi(X_t)M_t) &=  \dd(\phi_t) \partial\Phi(X_t)M_t + (\Delta \phi_t)\Delta (\partial\Phi(X_{t-}) M_t)+ \phi_t \partial\Phi(X_t) \dd M_t\\
        &+\phi_t \partial^2\Phi(X_t) [M_t\dd X_t^\top] +\phi_t \Delta\partial\Phi(X_{t})\Delta  M_t+\dd\delta\\
    \end{align*}
    where \[\dd \delta = \phi_t (\Delta\partial\Phi(X_t) - \partial^2\Phi(X_{t-})\Delta X_t)M_t\]
    We know for $\alpha\in (0,1)$,  by \Cref{lem:momentum-1}
    \begin{itemize}
        \item $\eta^{1-\alpha}\delta\to 0$ as $|\delta_t| \leq C \int_0^t \norm{M_s}^3 \dd Z_s$.
        \item $\int \eta^{1-\alpha}\dd(\phi_t) \partial\Phi(X_t)M_t\to 0$ as $\int \dd(\phi_t)<\infty$ by \Cref{ass:finite-var} and $\sup_t \eta^{1-\alpha} \norm{M_t}\to 0$.
        \item $ \sum\eta^{1-\alpha}(\Delta \phi_t)\Delta (\partial\Phi(X_{t-}) M_t)\to 0$ as $\sum_t \Delta \phi_t<\infty$ by \Cref{ass:finite-var} and $\int \eta^{1-\alpha} \norm{M_t}\to 0$.
        \item $\int \eta^{1-\alpha}\phi_t \partial^2\Phi(X_t) [M_t\dd X_t^\top] +\int \eta^{-\alpha}\phi_t\lambda_t \partial^2\Phi(X_t) [M_t M_t^\top]\dd Z_t\to 0$
        \item $\sum \eta^{1-\alpha}\phi_t \Delta\partial\Phi(X_{t})\Delta M_t -\int \phi_t \lambda_t\gamma_t \partial^2\Phi(X_t)[M_t,M_t-\nabla\Loss(X_t)]\dd Z_t \to 0$
        \item $\eta^{1-\alpha} \phi_t \partial\Phi(X_t)M_t\to 0$
    \end{itemize}
    
    Therefore adding them up we obtain the result.
    %$$\[ -\eta^{1-\alpha}\dd(\phi_t\partial\Phi(X_t)M_t)+\eta^{1-\alpha}\phi_t \partial^2\Phi(X_t) [M_t\dd X_t^\top] + \eta^{1-\alpha}\phi_t \partial\Phi(X_t) \dd M_t+\eta^{1-\alpha}\phi_t \partial^2\Phi(X_{t})[\dd [X_t,M_t]]\to 0\]
\end{proof}

\begin{lemma}
    $|\int_0^{t\wedge \tau_n}\partial^2 \Phi(X_s) [\dd[X]_s]-\int_0^{t\wedge \tau_n}\lambda_t^2\partial^2 \Phi(X_s) [M_tM_t^\top]\dd Z_t|\to 0$.\label{lem:aver-2}
\end{lemma}
\begin{proof}
    This  follows directly by the expansion of $[X]_t$ and \Cref{lem:momentum-1}.
\end{proof}
\begin{lemma}
$\sup_{c_1,c_2}\left|\int_{c_1}^{c_2}\frac{\lambda^2_t/\gamma_t}{\eta^\alpha} \partial^2\Phi(X_t)[M_t,M_t]\dd Z_t  - \int_{c_1}^{c_2}\frac{\lambda_t^2}{2}\partial^2\Phi(X_t)[\mSigma(X_t)]\dd t\right|\to 0$ as $\eta\to 0$.\label{lem:aver-3}
\end{lemma}
\begin{proof}
 let $A_t =\phi_t \partial^2\Phi(X_t)[M_t,M_t]$ for some schedule $\phi_t$, then for some uniformly bounded process $B_t$,
 \begin{align}\dd A_t & = d(\phi_t) (\partial^2\Phi(X_t)[M_t,M_t]+\Delta \partial^2\Phi(X_t)[M_t,M_t]) + \eta^{3\alpha-2}dB_t \\
 &- \frac{\phi_t\lambda_t}{\eta}\partial^3\Phi(X_t)[M_t,M_t,M_t] \diff Z_t + \frac{\gamma^2_t\phi_t}{\eta^{2-2\alpha}}\partial^2\Phi(X_t)[\mSigma(X_t)^{1/2}\dd [Y]_t\mSigma(X_t)^{1/2}]\\
 &  -  \frac{2\phi_t\gamma_t}{\eta^{2-\alpha}}\partial^2\Phi(X_t)[ M_t,M_t]\diff Z_t+\frac{2\phi_t\gamma_t}{\eta^{2-\alpha}}\partial^2\Phi(X_t)[ \nabla L(X_t), M_t]\diff Z_t \\
 & + \frac{2\phi_t\gamma_t}{\eta^{1-\alpha}}\partial^2\Phi(X_t)[\mSigma^{1/2}(X_t)\diff Y_t,M_t]\\
 & - \frac{2\phi_t\gamma_t^2}{\eta^{1-2\alpha}}\partial^2\Phi(X_t)[(M_t-\nabla\Loss(X_t))dY_t^\top \mSigma^{1/2}(X_t)]\\
 & + \gamma_t^2\eta^{2\alpha-2}  \phi_t\partial^2\Phi(X_t)[(M_t-\nabla\Loss(X_t))(M_t-\nabla\Loss(X_t))^\top]\diff Z_t.
 \end{align}
 Multiply both sides by $\eta^{2\alpha-2}$, by \Cref{lem:ave-1,lem:ave-2} we know $\int \partial^2\Phi(X_t)[\nabla\Loss(X_t)M_t^\top]\diff Z_t$ and $\int \partial^2\Phi(X_t)[\nabla\Loss(X_t)\nabla\Loss(X_t)^\top]\diff Z_t$ converges to 0. Bound the Martingale $W_t$ that
 \[dW_t = 2\phi_t\gamma_t\eta^{1-\alpha}\partial^2\Phi(X_t)[\mSigma^{1/2}(X_t)\diff Y_t,M_t] -2\phi_t\gamma_t^2\eta\partial^2\Phi(X_t)[(M_t-\nabla\Loss(X_t))dY_t^\top \mSigma^{1/2}(X_t)],\]
 Doob's inequality alongside with \Cref{lem:momentum-1} shows $W_t\to 0$. Then we know with $\phi_t=\lambda^2_t/\gamma_t^2$ that 
 \[\sup_{c_1,c_2}\left|\int_{c_1}^{c_2}\frac{\lambda^2_t/\gamma_t}{\eta^\alpha} \partial^2\Phi(X_t)[M_t,M_t]\dd Z_t  - \int_{c_1}^{c_2}\frac{\lambda_t^2}{2}\partial^2\Phi(X_t)[\mSigma(X_t)]\dd t\right|\to 0\]
\end{proof}
Finally we are ready to show the limiting dynamics as

\begin{theorem}\label{thm:sgdm-slow-sde-main}
    For any $t>0$, $(X^{n}_{t\wedge \tau_n},\tau_n)$ converges in distribution to $(X_{t\wedge \tau},\tau)$ that $\tau=\inf\{t>0:X_t\not\in K\}$, and that
    \[X_t = \Phi(\vx_0) + \int_0^t\lambda_t \partial\Phi(X_s)\mSigma^{1/2}(X_s)\dd W_s +\int_0^t\frac{\lambda_t^2}{2}\partial^2\Phi(X_s)[\mSigma(X_s)]\dd s.\]
\end{theorem}
\begin{proof}
    Recall the process $\Phi_t$ as
    \begin{align*}
     \dd \Phi_t  & =  \partial \Phi(X_t)((-\lambda_t\eta+ \lambda_t\eta^{1-\alpha}/\gamma_t)\diff M_t -\lambda_t \mSigma^{1/2}(X_t)dY_t) + \frac{1}{2}\partial^2 \Phi(X_t) [\dd[X]_t] +\diff\delta
    \end{align*}
    Therefore we know
    \begin{align*}
        X_{t\wedge \tau_n} & = \Phi(\vx_0) + (X_{t\wedge \tau_n} - \Phi_{t\wedge \tau_n}) + \delta_t \\
        & + \int_0^{t\wedge \tau_n} \partial \Phi(X_t)((-\lambda_t\eta+ \lambda_t\eta^{1-\alpha}/\gamma_t)\diff M_t -\lambda_t \mSigma^{1/2}(X_t)dY_t) + \frac{1}{2}\partial^2 \Phi(X_t) [\dd[X]_t].
    \end{align*}
    By \Cref{lem:conver-to-mani} we know the process $X_t-\Phi_t$ weakly converges to zero. By \Cref{lem:remn-term} we know $\delta_t\to 0$. By \Cref{lem:aver-1,lem:aver-2,lem:aver-3} we know
    \[|\int_0^{t\wedge \tau_n} \partial \Phi(X_t)((-\lambda_t\eta+ \lambda_t\eta^{1-\alpha}/\gamma_t)\diff M_t + \frac{1}{2}\partial^2 \Phi(X_t) [\dd[X]_t] -  \int_0^{t\wedge \tau_n}\frac{\lambda_t^2}{2}\partial^2\Phi(X_s)[\mSigma(X_s)]\dd [Y]_s|\to 0.\]
    We know the process $Y^n_t$ and $Z^n_t$ are of bounded quadratic variation. Furthermore by the central limit theorem $Y^n_t\to W_t$ where $W_t$ is a Brownian motion, and $Z^n_t\to t$. By the law of large numbers we know $[Y^n]_t\to t$. Additionally, the process $X_t^n$, $Z^n_t$ and $Y^n_t$ always share jumps at the same locations. This implies we can write
    \[X^n_{t} = X_0 + P^n_{t} + \int_0^{t} F_n(X^n_s)\dd Y^n_{s} + G_n(X^n_s)\dd [Y^n]_{s} + H_n(X^n_s) \dd Z^n_s.\]
    Notice that the tuple $(P^n_{t} ,Y^n_{t},[Y^n]_{t},Z^n_t,F_n(X_t),G_n(X_t),H_n(X_t),\tau_n)$ converges in the uniform metric to $(0,W_t,t,t,F(X_t),G(X_t),H(X_t),\tau(X_t))$ for any process $X_t$, then by \Cref{thm:weak-limit}, the limit of $X^n_t$ can be denoted by
    \[X_{t} = X_0 +\int_0^{t} F(X_s)\dd W_s + G(X_s)\dd s + H(X_s) \dd s.\]
    Plugging in the above results gives the limit
    \[X_t = \Phi(\vx_0) + \int_0^t\lambda_t \partial\Phi(X_s)\mSigma^{1/2}(X_s)\dd W_s +\int_0^t\frac{\lambda_t^2}{2}\partial^2\Phi(X_s)[\mSigma(X_s)]\dd s.\] 
\end{proof}
\begin{proof}[Proof for \Cref{thm:slow-sde-formal}]
    The result is a natural corollary of \Cref{thm:sgd-slow-sde-main} and \Cref{thm:sgdm-slow-sde-main}.
\end{proof}
