This section reviews some foundational concepts of human behavior including   
perception, cognition, decision-making and actuation. Motivated by Boyd's OODA loop from his writing 
on combat operational process~\cite{Boy87}, 
the OODA loop consists of four stages: observe, orient, decide and act. 
\emph{Observe} is the ability of humans to perceive and sense the environment around them. In Section~\ref{subsec:visual}, some basic theories of human perception (primarily visual) are surveyed. \emph{Orient} is the ability of humans to process the information coming from their sensory sub-systems, understand it, and make predictions based on such understanding.  This ability is provided by the cognitive processes of the human. In Section~\ref{subsec:cognitive}, we discuss briefly the theories of human cognition. 
\emph{Decide} is the ability of humans to create course of actions based on goals, utility or payoff functions, and available information. Fundamental concepts in human decision-making are surveyed in Section~\ref{subsec:decide}.  Finally, \emph{act} is the execution of plans, which is briefly discussed in Section~\ref{subsec:act}. 

\subsection{Perception} 
\label{subsec:visual}
A fundamental framework for building models of human perception is Signal Detection Theory (SDT)~\cite{green1966signal}, which is is one of most successful quantitative theories of human performance~\cite{wickens2002elementary}. 
It is an experimental framework in which the human participants are asked to detect the 
signal within a noisy input stimulus.  The participant's reaction
time and accuracy are recorded, and are used to build a model of sensitivity and response bias. 

In human perception modeling, one of the central questions that researchers seek to answer is how
efficiently can a human operator find a specific area of information on a visual display, 
a task also known as
\emph{visual search}.  A noisy stimulus, in the context of visual search, can be the presence of
items of distractions on a display with the target item.  Examples of using the SDT framework
for building models of human visual search include~\cite{Ver01,Eck00}, and numerous other works which can
be found in the survey paper~\cite{Wol98}. 

A notable work in human visual attention is the feature integration theory of Treisman \&
Gelade~\cite{Tre80}, which attempted to answer the question about the serial and/or parallel characteristics of the visual process.  Their model hypothesizes that feature search is mostly parallel while
other visual processes are serial.  In contrast, the work of Bundeson in~\cite{Bun90} contains a model with a strictly parallel architecture.  The parallel argument can be traced back to skepticism about distinguishing the serial processes from the parallel process in works such as~\cite{Tow90}.
Some early theories on the mental effort of human visual attention and processing can be found in~\cite{kahneman1973attention} where the performance is dependent on a single pool of undifferentiated resources.  This work later progressed into theory of multiple-resources~\cite{Wic91} (MRT), where attention and processing resources are divided into certain differentiated pools.  

%MRT became the predominant basis to develop state-of-art workload simulation and predictive tools. 


\subsection{Cognition}
\label{subsec:cognitive}
A theory of cognition describes human cognitive
processes such as reasoning, problem solving, decision-making, language processing, memory, and learning. 
The development of a theory of cognition is a multi-disciplinary effort that spans 
linguistics, artificial intelligence, psychology, and sociological, neurological and behavior sciences. 
A comprehensive review of cognition is beyond the scope of this paper.  
Instead, we survey the efforts to unify and codify cognitive principles into a common framework for 
building computational models of human cognition, reasoning and learning.  
These common frameworks, also referred to as cognitive architectures, 
are collections of structural and mechanistic descriptions 
of ``what human cognitive behaviors have in common"~\cite{lehman1996gentle}.  
Cognitive architecture models are the realization of the unified theories of 
human cognition and reasoning in the form of computer programs that 
simulate underlying cognitive processes of the human, and predict various cognitive outcomes or phenomena.  
They are based on both
fundamental theories about the human cognitive process and assumptions
about the domain being modelled such as language acquisition, helicopter pilot
training, expert behaviors, or implicit learning. 
One of the earliest cognitive models  
was the EPAM (Elementary Perceiver and Memorizer)~\cite{Sim64} by Simon
et al.  It is a quantitative model based on the verbal learning of nonsense
syllables, which predicts the rate of verbal learning as a function of
parameters such as familiarity, meaningfulness, and similarity. 
More recently, the trend has been to unify the theories of the cognitive processes, and then
formalize them as a basis for a programming language
environment which allows the end-users to write programs that are executable
models of the human mind.  


\subsection{Decision Making} 
\label{subsec:decide}

Models of human decision making range from 
game theory in economics~\cite{Von07},  where each agent minimizes or maximizes a utility function,
to sub-optimal descriptive models (behavioral economics)~\cite{Tve92}, where the model tries to reproduce the
heuristics of actual human decisions based on various human idiosyncrasies such as bounded
rationality~\cite{Gig02} and over-confidence in their knowledge~\cite{Gig91}.  
Other works on informal heuristics of human decision making include~\cite{Gig96,Gig99,Pay93}.
In a similar vein, the notion of satisficing decision-making, which originated in Herbert Simon's 1947 
work on administrative theory and refined over the years in subsequent works such as~\cite{simon1955behavioral},
relaxes the notion of the maximization of utility to simplify decision making under large
uncertainty and poorly defined utility function~\cite{reverdy2017satisficing}.  
In the satisficing decision-making framework, the limited informational and 
computational capacities of the human are taken into account. 
A decision is made if a simple payoff function exceeds a certain threshold value, which captures the process of making decisions of unknown optimality that are ``good enough'' for a given problem.

From the field of psychology, the lens model from Brunswik's 1952 work~\cite{Bru52} 
became an important conceptual framework for modeling human decision-making.  
The main idea behind the lens model is that one can quantitatively capture 
a human judgment of a criterion using a finite set of cues or observations related to the criterion. 
Examples include a manager making a judgment on the future success of a candidate (the criterion) based on
the behavior that the candidate exhibits during an interview (the cues), or a financial analyst
making a prediction on the future movements of a stock (the criterion) based on current social media news (the cues). 
The typical lens model consists of two halves. Each half is a multi-linear regression model. The right half 
captures the relations between the cues and the human judgment.  
The left half captures the relations between the cues and the actual outcome of the criterion.  
A meta-survey of the lens model can be found in~\cite{Kar08}. 


\subsection{Actuation} 
\label{subsec:act} 
Actuation refers to the motor outputs of the human operator. The typical
motor outputs that are modeled in human factors research include simple tasks
such as pointing, dragging, clicking, gazing, steering, etc. Other forms of
actuation such as verbal or non-verbal body movements are not described here.
A fundamental model of actuation is the Hyman-Hick law~\cite{Hym53,Hic52}, 
which hypothesized that the difficulty of a motor task is proportional to
the entropy of the situation. More commonly, the Hyman-Hick law is known in the 
context of decision-making i.e. 
more choices equates to higher entropy which 
means the person will take longer to make a decision.  

While Hyman-Hick established an 
information-theoretic foundation for the performance modeling of motor tasks, it 
did not become the dominant modeling
framework. Instead the most ubiquitous modeling framework in quantifying human
motor performance is Fitts' law~\cite{Fit54,Mac92}, which relates the difficulty of a
task to the amount of information that needs to be transmitted over a
hypothetical communication channel of limited capacity. The time performance of the motor task is 
linearly correlated with the amount of information transmitted by the task.  
Consider the pointing task in which a person is required to move a
finger to a target on the screen.  The signal for this task is the distance $D$ to
the target, and the noise is the width $W$ of the target. A model capturing the difficulty of 
the task would be the entropy of the task i.e. $\log{\frac{D}{W}}$.  
Fitt's model has been generalized to other tasks beyond
finger pointing such as  mouse, trackball, stylus, touchpad, helmet-mounted sights, eye tracker and joy stick~\cite{Sou04}, and
also generalized to predict errors in motor tasks~\cite{Wob08}.  

Beyond Fitt's law, a more recent work by Accot et al.~\cite{Accms} extended the
informational-theoretic framework to cover complex motor tasks that are more
trajectory based.  Examples of these motor tasks include steering, drawing
curves, navigating through nested menus, moving in a 3D world, which becomes
more relevant as interfaces move beyond the simple point and click paradigm.

% First we examine some of the latent states of a human  
% such as workload, and situational awareness.  These latent states are useful for predicting 
% performance of a HMS but may not 
% have any singular physiological manifestation in the human. 
% Due to the complexity of a ``human system'', these latent variables 
% cannot be measured directly nor are they simple functions of measurable physiological indicators.  