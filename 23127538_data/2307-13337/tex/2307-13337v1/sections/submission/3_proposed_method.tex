\section{Proposed method}
\label{sec:proposed_method}


\input{sections/submission/figures/method-obs}
\subsection{Preliminaries}
\label{subsec:preliminaries}

To reduce the heavy computations of convolutional layers in neural networks,
the input feature (activation) and weight of each convolutional layer are quantized to low-bit values~\cite{cai2017deep,choi2018pact,jung2019learning,gholami2021survey}.
In this work, as the features of SR network are more sensitive to quantization, we focus more on quantizing features.
The input feature of the $i$-th convolutional layer $x_i\in\mathbb{R}^{B\times C\times H\times W}$, where $B, C, H,$ and $W$ denote the dimension of input batch, channel, height, and width, a quantization operator $Q(\cdot)$ quantizes the feature $x_i$ with bit-width $b$:
\begin{equation}\label{eqn:method-quantmodule-pre}
    Q(x_i) = \text{Int}({ \frac{\text{clip}(x_i, \alpha_l, \alpha_u) - \alpha_l}{s} }) \cdot s + \alpha_l,
\end{equation} 
where $ \text{clip}(\cdot, \alpha_l, \alpha_u)$ truncates the input into the range of $[\alpha_l, \alpha_u]$ and $s = \frac{\alpha_u - \alpha_l}{2^b-1}$.
After truncation, the truncated feature is scaled to the integer range of bit-width $b$, $[0, 2^b-1]$.
Then the scaled feature in the integer range is rounded to integer values with Int$(\cdot)$, and it is rescaled to range $[\alpha_l, \alpha_u]$.
To obtain better quantization ranges for SR networks, 
range parameters $\alpha_l, \alpha_u$ for each layer are generally learned through quantization-aware training~\cite{Li2020pams,zhong2022ddtb}.
Since the rounding function is not differentiable, a straight-through estimator (STE)~\cite{bengio2013estimating} is used to train the range parameters in an end-to-end manner.
We initialize $\alpha_u$ and $\alpha_l$ as the $j$-th and $100-j$-th percentile value of feature averaged among the training data.
$j$ is set as 1 in our experiments to avoid outliers from corrupting the quantization range.
Similarly, to quantize the weight of the $i$-th convolutional layer $w^i$, quantization operator $Q(\cdot)$ is used.
However, instead of setting range parameters as learnable parameters, 
$\alpha_l, \alpha_u$ for weights are fixed as the $j$-th and $100-j$-th percentile of weights.



\subsection{Distribution mismatch in SR networks}
\label{subsec:motivation}

Quantization unfriendliness of SR networks is from the diverse feature (activation) distributions, as reported in previous studies~\cite{Li2020pams,hong2022daq,zhong2022ddtb}, mainly due to the absence of batch normalization layers in SR networks.
Existing SR quantization methods address this issue by employing one~\cite{Li2020pams} or two~\cite{zhong2022ddtb} learnable quantization range parameters for each convolutional layer feature.
However, despite that the quantization-aware training process aims to find the optimal range for each feature, it fails to account for the channel-wise and input-wise variance in distributions.
As illustrated in Figure~\ref{fig:method-motiv}, where notable discrepancies exist between layer-wise and channel-wise distributions, quantization grids are needlessly allocated to regions with minimal feature density.
This mismatch in inter-channel distributions leads to performance degradation when quantizing SR networks.
In the following sections, we introduce a new quantization-aware training scheme to address the distribution mismatch problem.






\subsection{Cooperative variance regularization}
\label{subsec:var-reg}

\input{sections/submission/figures/method-loss}
Instead of focusing on finding a better quantization range parameter capable of accommodating the diverse feature distributions, our approach aims to regularize the distribution diversity beforehand.
Obtaining an appropriate quantization range for a feature with low variance is an easier task compared to that of high variance.
In this work, we define the overall mismatch of a feature distribution with the standard deviation, 
\begin{equation}
    M(x_i) \coloneqq \sigma(x_i),
\label{eq:mismatch}
\end{equation}
where $\sigma(\cdot)$ calculates the standard deviation of the feature.
Thus, variance regularization can be directly applied to the feature to be quantized $(x_i)$, which is formulated as follows:
\begin{equation}
    \mathcal{L}_{V}(x_i) = \lambda_V \cdot M(x_i),
\label{eq:varianceloss}
\end{equation}
where 
$\lambda_V$ is the hyperparameter that denotes the weight of regularization. 
The overall $\mathcal{L}_{V} = \sum_{i}^{\# \;\text{layers}}{ \mathcal{L}_{V}(x_i)}$ is obtained by summing over all quantized convolutional layers. 
The variance regularization loss can be used in line with the reconstruction loss, which is originally used in the general quantization-aware training process.
The optimization of parameter $\theta^t$ is formulated as follows:
\label{subsec:loss}
\begin{equation}
    {\theta}^{t+1} = \theta^{t} - \alpha^{t} (\nabla_\theta \mathcal{L}_R(\theta^t) + \nabla_\theta \mathcal{L}_V(\theta^t) ), 
\end{equation}
where $\nabla_\theta \mathcal{L}_R(\theta^t)$ the gradient from the original reconstruction loss and $\nabla_\theta \mathcal{L}_V(\theta^t)$ denotes the gradient from variance regularization loss and  $\alpha^t$ denotes the learning rate. 
Updating the network to minimize the variance regularization loss will reduce the quantization error of each feature.

However, then a question arises, \textit{does reducing the quantization error of each feature lead to improved reconstruction accuracy}?
The answer is, according to our observation in Figure~\ref{fig:method-conflict}, not necessarily.
During the training process, the variance regularization loss can collide with the original reconstruction loss.
However, we want to avoid the conflict between two losses, in other words, minimize the variance as long as it does not hinder the reconstruction loss.
Thus, we determine whether the two losses are cooperative or not by examining the sign of the gradients of each loss.
If the signs of the gradients are equal, then the parameter is updated in the same direction by two losses. 
By contrast, if the sign values are inverse, the two losses restrain each other, thus we only employ the reconstruction loss.
In summary, we leverage variance regularization for parameters that the gradients have the same sign value as that from the reconstruction loss.
Our optimization can be formulated as follows:
\begin{equation}
    \theta^{t+1} = 
    \begin{cases}
    \theta^{t} - \alpha^{t} (\nabla_\theta \mathcal{L}_R(\theta^t) + \nabla_\theta \mathcal{L}_V(\theta^t) ),  & \quad \nabla_\theta \mathcal{L}_R(\theta^t) \cdot \nabla_\theta \mathcal{L}_V(\theta^t)\geq0, 
    \\
    \theta^{t} - \alpha^{t} (\nabla_\theta \mathcal{L}_R(\theta^t) ), & \quad \nabla_\theta \mathcal{L}_R(\theta^t) \cdot \nabla_\theta \mathcal{L}_V(\theta^t)<0.
    \end{cases}
    \label{eq:cooperativeloss}    
\end{equation} 
This allows the network to reduce the quantization error cooperatively with the reconstruction error.


\subsection{Distribution offsets}
\label{subsec:offsets}
\input{sections/submission/figures/method-motivation}

The variance of the distribution can be reduced to a certain extent via variance regularization in Section~\ref{subsec:var-reg}.
However, since regularization is applied only when it is cooperative with the SR reconstruction, the gap between distributions remains.
In this section, we explore the remaining gap between distributions.
First, as visualized in Figure~\ref{fig:method-motiv}, we observe that the distribution gap is larger (and more critical) in the channel dimension compared to the image dimension.


Also, we find that this extent of the channel-wise gap in the distribution is different for each layer of the SR network, as shown in Figure~\ref{fig:method-offset}.
Some layers (Figure~\ref{fig:method-offset-b}) exhibit a larger mismatch in the distribution deviation, while others (Figure~\ref{fig:method-offset-c}) show a larger mismatch in the distribution average.
The quantization errors of the layer with a large mismatch in distribution mean can be decreased by shifting the channel-wise feature.
On the other hand, the mismatch in layers with large divergence in distribution deviation can be reduced by scaling each channel-wise distribution for overall similar distributions.

Since channel-wise shifting and scaling incur computational overhead and not all layers are in need of additional shifting and scaling (Figure~\ref{fig:method-offset-a}), we selectively apply offset scaling/shifting to layers that can maximally benefit from it.
The standards for our selection are derived by feeding a patch of images to the 32-bit pre-trained network and calculating the mismatch in average/deviation of each layer.
Given the $i$-th feature statistics $\hat{x}^i$ from the pre-trained network, the mismatch of $i$-th convolutional layer is formulated as follows:
\begin{equation}
\begin{aligned}
    M_\mu^i \coloneqq \sigma(\mu_{c}(\hat{x}_i)) \quad \text{and} \quad M_\sigma^i \coloneqq \sigma(\sigma_{c}(\hat{x}_i)),
\label{eq:indicators}
\end{aligned}
\end{equation}
where $\mu_c(\cdot)$ and $\sigma_c(\cdot)$ respectively calculate the channel-wise mean and standard deviation of a feature and $\sigma(\cdot)$ calculates the standard deviation.
After all $M_\mu^i$s and $M_\sigma^i$s  $(i=1,\cdots,\#\text{layers})$ are collected,
we apply additional scaling offsets to top-p layers with high $M_\sigma^i$ value and shifting offsets to top-p layers with high $M_\mu^i$ value.
The shifting and scaling process for feature $x^i$ of the $i$-th convolutional layer is formulated as follows:
\begin{equation}
\begin{aligned}
    x_i^* = x_i + S_\mu, \quad \text{if} \; M_\mu^i \in \text{top-}p([M_\mu^1, \cdots, M_\mu^{\# \text{layers}}]),
    \label{eq:offsets-b}
\end{aligned}
\end{equation}
\begin{equation}
    x_i^* = x_i \cdot S_\sigma, \quad \text{if} \; M_\sigma^i \in \text{top-}p([M_\sigma^1, \cdots, M_\sigma^{\# \text{layers}}]),
\label{eq:offsets-w}
\end{equation}
where $S_\mu, S_\sigma \in \mathbb{R}^{C}$ are learnable parameters, top-$p(\cdot)$ constructs a set that contains values greater than the $100(1-p)$-percentile value of the given set.  
$p$ is the hyperparameter that determines the ratio of layers to apply distribution offsets, which we set to $0.3$ in our experiments.
Moreover, both offsets $S_\mu$ and $S_\sigma$ are quantized to low-bit, 4-bit in our experiments, to minimize the computational overhead.
Consequently, the offsets additionally incur only $0.02\%$ overhead to the network storage size for EDSR.
The offsets further relieve distribution mismatch with minimal overhead.


\subsection{Overall training}


In this section, 
we formally describe our algorithm: a simple variation of the model-based algorithm called {\em Monotonic Value Propagation} proposed by \citet{zhang2020reinforcement}. 
%
We present the full procedure in Algorithm~\ref{alg:main}, and point out several key ingredients. 
%
\begin{itemize}
	\item {\em Optimistic updates using upper confidence bounds (UCB).} The algorithm implements the optimism principle in the face of uncertainty 
by adopting the frequently used UCB-based framework \citep{azar2017minimax,jin2018q}. 
More specifically, the learner maintains upper estimates for both the value and Q-function, 
by calculating the following optimistic Bellman equation backward (from $h=H,\ldots,1$): 
%
\begin{subequations}
\begin{align}
Q_{h}(s,a)\, & \leftarrow\,\min\big\{\widehat{r}_{h}(s,a)+\langle\widehat{P}_{s,a,h},V_{h+1}\rangle+b_{h}(s,a),H\big\}, 
	\label{eq:Qh-UCB-informal}\\
V_{h}(s)\, & \leftarrow\, \max\nolimits_{a}Q_{h}(s,a). 
\end{align}
\end{subequations}
%
Here, $Q_{h}$ (resp.~$V_h$) is the running estimate for the Q-function (resp.~value function), 
		$\widehat{r}_{h}(s,a) \in \mathbb{R}$ is an estimate of the mean reward at $(s,a,h)$, 
$\widehat{P}_{s,a,h}\in \mathbb{R}^S$ indicates an estimate of the transition probability vector from $(s,a,h)$, 
whereas $b_{h}(s,a)\geq 0$ is some suitably chosen bonus term that compensates for the uncertainty.   



	\item {\em Monotonic bonus functions.} Another crucial step in order to ensure near-optimal regret lies in careful designs of the data-driven bonus terms $\{b_h(s,a)\}$ in \eqref{eq:Qh-UCB-informal}. 
		Here, we adopt the monotonic bonus function for MVP originally proposed in \citet{zhang2020reinforcement}, 
		to be made precise in \eqref{eq:update1}. 
		Compared to the bonus function in $\mathtt{Euler}$~\citep{zanette2019tighter} and $\mathtt{UCBVI}$~\citep{azar2017minimax},  the monotonic bonus form has a cleaner structure that effectively avoid large lower order terms. In order to enable variance-aware regret, we also need to keep track of the empirical variance of the (stochastic) immediate rewards.


	\item {\em An epoch-based procedure and a doubling trick.} 
%
		A key step of our algorithm is to update the empirical transition kernel and empirical rewards in an epoch-based fashion. 
		More concretely, the whole learning process is divided into several consecutive epochs via a simple doubling rule.
		That is, once the number of visits to a $(s,a,h)$-tuple reaches a power of 2, we end the current epoch,  reconstruct the empirical transition kernel and rewards using data from this epoch (cf.~lines~\ref{line:r-hsa-update} and \ref{line:P-hsa-update} of Algorithm~\ref{alg:main}), compute the Q-function and value function using the newly updated transition kernel and rewards (cf.~\eqref{eq:updateq}), and then start a new epoch. In each epoch, the learned policy is induced by the optimistic Q-function estimate computed based on the empirical transition kernel of the {\em current} epoch. 




\end{itemize}




\begin{remark}[Doubling batch]
We note that a doubling update rule has also been used in the original MVP \citep{zhang2020reinforcement}. 
A major difference between our modified version and the original one is that: when the visitation count for some $(s,a,h)$ reaches $2^i$ for some integer $i$, we only use the second half of the samples (i.e., the $\{2^{i-1}+j\}_{j=1}^{2^{i-1}}$-th samples) to compute the empirical model, whereas the original MVP makes use of all the $2^i$ samples. This step is crucial for decoupling statistical dependence.

\end{remark}







\begin{algorithm}[h]
	\DontPrintSemicolon
\caption{Monotoinic Value Propagation (MVP)~\citep{zhang2020reinforcement}\label{alg:main}}
%\begin{algorithmic}[ht]
	\textbf{input:} state space $\mathcal{S}$, action space $\mathcal{A}$, horizon $H$, total number of episodes $K$, confidence parameter $\delta$, 
	%$\mathcal{L}=\{1,2,\ldots, 2^{\log_2K}\}$, 
	$c_1=\frac{460}{9}$, $c_2 = 2\sqrt{2}$ and $c_3=\frac{544}{9}$.  \\
%\textbf{Initialize:} . \\
	\textbf{initialization: } for all $(s,a,s',h)\in \mathcal{S}\times \mathcal{A}\times\mathcal{S}\times [H]$, set $\theta_h(s,a)\leftarrow 0$, $\kappa_h(s,a)\leftarrow 0$, $\overline{N}_h(s,a,s')\leftarrow 0$, $N_h(s,a,s')\leftarrow 0$, $n_h(s,a)\leftarrow 0$, $Q_h(s,a)\leftarrow H-h+1$, $V_h(s)\leftarrow H-h+1$. \\
	\For{$k=1,2,...$} {
		Set $\pi^k$ such that $\pi_h^k(s) = \arg\max_{a}Q_h(s,a)$ for all $s\in \mathcal{S}$ and $h\in [H]$. {\color{blue}\tcc{policy iterate.}}
		\For {$h=1,2,...,H$} {
			Observe $s_{h}^k$, 
			take action $ a_h^k= \arg\max_{a}Q_h(s_h^k,a)$, 
%	%	\STATE{ Receive reward $r_h^k$ (receive $c_h^k$ and set $r_h^k = -c_h^k$ for the cost case) and  observe $s_{h+1}^k$.}
			receive  $r_h^k$,  observe $s_{h+1}^k$. \label{line:choose_action} 
			{\color{blue}\tcc{sampling.}}
			$(s,a,s')\leftarrow (s_h^k,a_h^k,s_{h+1}^k)$. \\
			Update $\overline{N}_h(s,a) \leftarrow  \overline{N}_h( s,a )+1$, $N_h(s,a,s') \leftarrow   N_h(s,a,s')+1$, $\theta_h(s,a)\leftarrow \theta_h(s,a)+r_h^k$, $\kappa_h(s,a)\leftarrow \kappa_h(s,a)+(r_h^k)^2$. \\
		{\color{blue}\tcc{perform updates using data of this epoch.}}
		\If{$\overline{N}_h(s,a)\in \{1,2,\ldots, 2^{\log_2K}\}$ \label{line:rp_update_start} }   {
			$n_h(s,a)\leftarrow \sum_{\widetilde{s}}N_h(s,a,\widetilde{s})$. 
			{\color{blue}\tcp{number of visits to $(s,a,h)$ in this epoch.}}
			$\widehat{r}_h(s,a)\leftarrow \frac{\theta_h(s,a)}{n_h(s,a)}$. \label{line:r-hsa-update}
			{\color{blue}\tcp{empirical rewards of this epoch.}} 
			$\widehat{\sigma}_h(s,a)\leftarrow \frac{\kappa_h(s,a)}{n_h(s,a) }  $. 
			{\color{blue}\tcp{empirical squared rewards of this epoch.}}
			$\widehat{P}_{s,a,h}(\widetilde{s}) \leftarrow  \frac{N_h(s,a,\widetilde{s})}{n_h(s,a)}$ for all $\widetilde{s} \in \mathcal{S}$.  \label{line:P-hsa-update}
			{\color{blue}\tcp{empirical transition for this epoch.}}
			%
			Set TRIGGERED = TRUE, 
			and $\theta_h(s,a)\leftarrow 0$, $\kappa_h(s,a)\leftarrow 0$,  $N_h(s,a,\widetilde{s})\leftarrow 0$  for all $\widetilde{s}\in \mathcal{S}$. 
%		\ENDIF \label{line:rp_update_end}
		}
		}
		{\color{blue}\tcc{optimistic Q-estimation using empirical model of this epoch.}}
		\If {\textnormal{TRIGGERED= TRUE}} {
			Set TRIGGERED = FALSE, and $V_{H+1}(s)\leftarrow 0$ for all $s\in \mathcal{S}$. \\
			\For{$h=H,H-1,...,1$} {
				%
				\For{$(s,a)\in \mathcal{S}\times \mathcal{A}$} {

					%\vspace{-3ex}
					\begin{align} 
						\vspace{-3ex}
						b_h(s,a) &\leftarrow c_1 \sqrt{\frac{   \mathbb{ V}(\widehat{P}_{s,a,h} ,V_{h+1}) \log \frac{1}{\delta}  }{ \max\{n_h(s,a),1 \} }}+c_2 \sqrt{\frac{\big(\widehat{\sigma}_h(s,a)- (\widehat{r}_h(s,a))^2 \big)\log \frac{1}{\delta}}{\max\{n_h(s,a),1\}}} \qquad\qquad\qquad\qquad\qquad\qquad\nonumber\\
						&\qquad\qquad\qquad +c_3\frac{H\log \frac{1}{\delta}}{ \max\{n_h(s,a) ,1\}  },  \label{eq:update1}  \\
						Q_h(s,a) &\leftarrow \min\big\{    \widehat{r}_h(s,a)+\langle \widehat{P}_{s,a,h}, V_{h+1} \rangle +b_h(s,a)    ,H\big\},\,
						V_{h}(s) \leftarrow \max_{a}Q_h(s,a).
						\label{eq:updateq}
					\end{align}
					\vspace{-3ex}
				}
			}
			%
		}
	}
%\end{algorithmic}
\end{algorithm}






















\begin{comment}

\begin{algorithm}
	\DontPrintSemicolon
\caption{Monotoinic Value Propagation (MVP)~\citep{zhang2020reinforcement}\label{alg:main}}
\begin{algorithmic}[ht]
\STATE{\textbf{Input:} number of states $S$, number of actions $A$, horizon $H$, total number of episodes $K$, confidence parameter $\delta$;}
\STATE{\textbf{Initialize:} $\mathcal{L}=\{1,2,\ldots, 2^{\log_2(K)}\}$; $c_1=\frac{460}{9}, c_2 = 2\sqrt{2},  c_3=\frac{544}{9}$;}
\STATE{\textbf{Initialization: } $\theta_h(s,a)\leftarrow 0, \kappa_h(s,a)\leftarrow 0, \overline{N}_h(s,a,s')\leftarrow 0, N_h(s,a,s')\leftarrow 0, n_h(s,a)\leftarrow 0, Q_h(s,a)\leftarrow H-h+1, V_h(s)\leftarrow H-h+1, \forall (s,a,s',h)\in \mathcal{S}\times \mathcal{A}\times\mathcal{S}\times [H]$;}
\FOR {$k=1,2,...$}
		\FOR {$h=1,2,...,H$}
		\STATE{ Observe $s_{h}^k$;}
		\STATE{ Take action $ a_h^k= \arg\max_{a}Q_h(s_h^k,a)$;} \label{line:choose_action}
	%	\STATE{ Receive reward $r_h^k$ (receive $c_h^k$ and set $r_h^k = -c_h^k$ for the cost case) and  observe $s_{h+1}^k$.}
 \STATE{ Receive reward $r_h^k$  and  observe $s_{h+1}^k$.}
		\STATE{ Set $(s,a,s')\leftarrow (s_h^k,a_h^k,s_{h+1}^k)$;.}
		\STATE{ Set $\overline{N}_h(s,a) \leftarrow  \overline{N}_h( s,a )+1$, \ \,$N_h(s,a,s') \leftarrow   N_h(s,a,s')+1$, $\theta_h(s,a)\leftarrow \theta_h(s,a)+r_h^k$, $\kappa_h(s,a)\leftarrow \kappa_h(s,a)+(r_h^k)^2$.}
		\STATE{ \verb|\\| \emph{Update empirical rewards and transition probability}}
		\IF {$\overline{N}_h(s,a)\in \mathcal{L}$}  \label{line:rp_update_start}
  \STATE{Set $\widehat{r}_h(s,a)\leftarrow \frac{\theta_h(s,a)}{\sum_{\widetilde{s}'}N_h(s,a,\widetilde{s}')}$;}
  \STATE{Set $\widehat{\sigma}_h(s,a)\leftarrow \frac{\kappa_h(s,a)}{\sum_{\widetilde{s}'}N_h(s,a,\widetilde{s'}) } - (\widehat{r}_h(s,a))^2 $}
		\STATE Set $\widehat{P}_{s,a,h,\widetilde{s}} \leftarrow  \frac{N_h(s,a,\widetilde{s})}{\sum_{\widetilde{s}'}N_h(s,a,\widetilde{s}')}$ for all $\widetilde{s} \in \mathcal{S}$.
		%			\ENDFOR
  \STATE{}
		\STATE{ Set $n_h(s,a)\leftarrow \sum_{\widetilde{s}'}N_h(s,a,\widetilde{s}')$;}
		\STATE{ Set TRIGGERED = TRUE.}
  \STATE{$\theta_h(s,a)\leftarrow 0, \kappa_h(s,a)\leftarrow 0,  N_h(s,a,\widetilde{s})\leftarrow 0, \forall \widetilde{s}\in \mathcal{S}$;}
		\ENDIF \label{line:rp_update_end}
		\ENDFOR
		\STATE{ \verb|\\| \emph{Update $Q$-function}}
		\IF {TRIGGERED}
  \STATE{$V_{H+1}(s)\leftarrow 0,\forall s\in \mathcal{S}$;}
		\FOR{$h=H,H-1,...,1$}
		\FOR{$(s,a)\in \mathcal{S}\times \mathcal{A}$}
		%		  \vspace{-3ex}
		\STATE 	 {		%\vspace{-0.5cm} 	
			%					\vspace{-0.5cm}
			%			\simon{Maybe move to the line $N(s,a) \in \mathcal{L}$ so it's clear for the reader that we update the policy only in the set.}
			Set
			\begin{align} 
			~~~~~~~~~~~~~&b_h(s,a)\leftarrow c_1 \sqrt{\frac{   \mathbb{ V}(\widehat{P}_{s,a,h} ,V_{h+1}) \log(\frac{1}{\delta})  }{ \max\{n_h(s,a),1 \} }}+c_2 \sqrt{\frac{(\widehat{\sigma}_h(s,a)- (\widehat{r}_h(s,a))^2 )\log(\frac{1}{\delta})}{\max\{n_h(s,a),1\}}}+c_3\frac{H\log(\frac{1}{\delta})}{ \max\{n_h(s,a) ,1\}  },  \label{eq:update1}  \\
			%\\ 	\hspace{-20ex} 	&  \left\{\begin{array}{l}   Q_h(s,a)\leftarrow \min\{    \widehat{r}_h(s,a)+\widehat{P}_{s,a,h} V_{h+1} +b_h(s,a)    ,H\} \quad  \text{for reward case} \\ Q_h(s,a)\leftarrow \max\{\min \left\{ \widehat{r}_h(s,a) + \widehat{P}_{s,a,h}V_{h+1}+b_h(s,a), 0 \right\} ,-H\} \quad \text{for cost case}\label{eq:updateq}\end{array}\right.
   \hspace{-20ex} 	&    Q_h(s,a)\leftarrow \min\{    \widehat{r}_h(s,a)+\widehat{P}_{s,a,h} V_{h+1} +b_h(s,a)    ,H\}\label{eq:updateq}
			\\ & V_{h}(s) \leftarrow \max_{a}Q_h(s,a).\nonumber
			\end{align}
			\vspace{-3ex}
		}
		\ENDFOR
		\ENDFOR
		\STATE{ Set TRIGGERED = FALSE}
		\ENDIF
		\ENDFOR
\end{algorithmic}
\end{algorithm}

\end{comment}



Algorithm~\ref{algo-odm} summarizes the overall pipeline for our framework, ODM.
We follow the common practice~\cite{Li2020pams,zhong2022ddtb} to use $\mathcal{L}_1$ loss and $\mathcal{L}_\text{SKT}$ loss for the reconstruction loss as follows:
\begin{equation}
    \mathcal{L}_R = \mathcal{L}_1 + \lambda \mathcal{L}_{SKT}, 
\label{eq:overallloss}
\end{equation}
where
$\mathcal{L}_1$ loss indicates the $l_1$ distance between the reconstructed image and the ground-truth HR image, and $\mathcal{L}_\text{SKT}$ loss is the $l_2$ distance between the structural features of the quantized network and the 32-bit pre-trained network.
The structural features are obtained from the last layer of the high-level feature extractor.
The balancing weight $\lambda$ is set as $1000$ in our experiments.
Also, the weight $\lambda_V$ to balance $\mathcal{L}_V$ and $\mathcal{L}_R$ in Eq.~\ref{eq:varianceloss} is set differently depending on the mismatch severeness of the SR architecture. 
We provide detailed settings in the supplementary materials.




