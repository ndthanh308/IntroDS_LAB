\documentclass[runningheads]{llncs}

% ---------------------------------------------------------------
% Include basic ECCV package
 \usepackage[dvipsnames,svgnames,table]{xcolor} % added here before usepackage eccv, due to collision
 
% TODO REVIEW: Insert your submission number below by replacing '*****'
% TODO FINAL: Comment out the following line for the camera-ready version
% \usepackage[review,year=2024,ID=2121]{eccv}
% TODO FINAL: Un-comment the following line for the camera-ready version
\usepackage{eccv}

% OPTIONAL: Un-comment the following line for a version which is easier to read
% on small portrait-orientation screens (e.g., mobile phones, or beside other windows)
% \usepackage[mobile]{eccv}


% ---------------------------------------------------------------
% Other packages

% Commonly used abbreviations (\eg, \ie, \etc, \cf, \etal, etc.)
\usepackage{eccvabbrv}

% Include other packages here, before hyperref.
\usepackage{graphicx}
\usepackage{booktabs}


% The "axessiblity" package can be found at: https://ctan.org/pkg/axessibility?lang=en
\usepackage[accsupp]{axessibility}  % Improves PDF readability for those with disabilities.


% ---------------------------------------------------------------
% Hyperref package

% It is strongly recommended to use hyperref, especially for the review version.
% Please disable hyperref *only* if you encounter grave issues.
% hyperref with option pagebackref eases the reviewers' job, but should be disabled for the final version.
%
% If you comment hyperref and then uncomment it, you should delete
% main.aux before re-running LaTeX.
% (Or just hit 'q' on the first LaTeX run, let it finish, and you
%  should be clear).

% TODO FINAL: Comment out the following line for the camera-ready version
% \usepackage[pagebackref,breaklinks,colorlinks,citecolor=eccvblue]{hyperref}
% TODO FINAL: Un-comment the following line for the camera-ready version
\usepackage{hyperref}

% Support for ORCID icon
\usepackage{orcidlink}

% added packages
\usepackage{url}
\usepackage{times}
\usepackage{epsfig}
\usepackage{amssymb}
\usepackage{amsmath} % in math_commands
\usepackage{bm} % in math_commands
\usepackage{subcaption}
\usepackage{graphbox}
\usepackage{paralist}
\usepackage{amsfonts}

% table
\usepackage{makecell}
\usepackage{tabularx}
\usepackage{multirow}
\usepackage{pifont}
\usepackage[para,online,flushleft]{threeparttable}

% packages for algorithms
\usepackage{algorithm}
\usepackage{setspace}
\usepackage[noend]{algpseudocode}

% packages for color
\usepackage{color}

% packages for equations
\usepackage{stmaryrd} %for shortarrow

\usepackage[normalem]{ulem}

% definitions 
\def\mX{{\bm{X}}}
\def\mW{{\bm{W}}}
\def\mI{{\bm{I}}}
\def\gQ{{\mathcal{Q}}}
\def\gP{{\mathcal{P}}}
\def\vv{{\bm{v}}}

\def\Plus{\texttt{+}}
\def\Minus{\texttt{-}}
\def\Equal{\texttt{=}}




\begin{document}

% ---------------------------------------------------------------
% TODO REVIEW: Replace with your title
% \title{Overcoming Distribution Mismatch \\ in Quantizing Image Super-Resolution Networks}
\title{Overcoming Distribution Mismatch\texorpdfstring{\\}{} in Quantizing Image Super-Resolution Networks}

% TODO REVIEW: If the paper title is too long for the running head, you can set
% an abbreviated paper title here. If not, comment out.
\titlerunning{ODM}

% TODO FINAL: Replace with your author list. 
% Include the authors' OCRID for the camera-ready version, if at all possible.
\author{Cheeun Hong\inst{1}\orcidlink{0009-0009-3480-748X} \and
Kyoung Mu Lee\inst{1,2}\orcidlink{0000-0001-7210-1036}
}

% TODO FINAL: Replace with an abbreviated list of authors.
\authorrunning{C.~Hong and K.M.~Lee}

% First names are abbreviated in the running head.
% If there are more than two authors, 'et al.' is used.

% TODO FINAL: Replace with your institution list.
\institute{
Dept. of ECE \& ASRI, \email{\{cheeun914, kyoungmu\}@snu.ac.kr} \and
IPAI, Seoul National University
}


\maketitle
\begin{abstract}
Graph Neural Networks (GNNs) have proven to be effective in processing and learning from graph-structured data.
However, previous works mainly focused on understanding single graph inputs while many real-world applications require pair-wise analysis for graph-structured data (e.g., scene graph matching, code searching, and drug-drug interaction prediction).
To this end, recent works have shifted their focus to learning the interaction between pairs of graphs.
Despite their improved performance, these works were still limited in that the interactions were considered at the node-level, resulting in high computational costs and suboptimal performance.
To address this issue, we propose a novel and efficient graph-level approach for extracting interaction representations using co-attention in graph pooling. 
Our method, Co-Attention Graph Pooling (CAGPool), exhibits competitive performance relative to existing methods in both classification and regression tasks using real-world datasets, while maintaining lower computational complexity.

\end{abstract}

\section{Introduction}
Deep learning models have been widely used in many applications.
For example, BERT~\citep{devlin_bert_2019}, GPT-3~\citep{brown_language_2020}, and T5~\citep{raffel_exploring_2020} achieved state-of-the-art~(SOTA) results on different natural language processing~(NLP) tasks. 
For computer vision~(CV), Transformer-like models such as ViT~\citep{dosovitskiy_image_2021} and Swin Transformer~\citep{liu_swin_2021} deliver excellent accuracy performance upon multiple tasks. 


At the same time, training deep learning models has been a critical problem troubling the community due to the long training time, especially for those large models with billions of parameters~\citep{brown_language_2020}. 
In order to enhance the training efficiency, researchers propose some manually designed parallel training strategies~\citep{narayanan_efficient_2021,shazeer_mesh-tensorflow_2018,xu_gspmd_2021}. 
However, selecting, tuning, and combining these strategies require extensive domain knowledge in deep learning models and hardware environments. With the increasing diversity of modern hardware architectures~\cite{flynn_very_1966,flynn_computer_1972} and the rapid development of deep learning models, these manually designed approaches are bringing heavier burdens to developers. 
Hence, \emph{automatic parallelism} is introduced to automate the parallel strategy searching for training models.


There are two main categories of parallelism in deep learning models: inter-layer parallelism~\citep{huang_gpipe_2019,narayanan_pipedream_2019,narayanan_memory-efficient_2021,fan_dapple_2021,li_chimera_2021,lepikhin_gshard_2021,du_glam_2022,fedus_switch_2022} and intra-layer parallelism~\citep{li_pytorch_2020,narayanan_efficient_2021,rasley_deepspeed_2020,fairscale_authors_fairscale_2021}. 
Inter-layer parallelism partitions the model into disjoint sets on different devices without slicing tensors. 
Alternatively, intra-layer parallelism partitions tensors in a layer along one or more axes and distributes them across different devices.


Current automatic parallelism techniques focus on optimizing strategies within these two categories. However, they treat these two categories separately. 
Some methods~\citep{zhao_vpipe_2022,jia_exploring_2018,cai_tensoropt_2022,wang_supporting_2019,jia_beyond_2019,schaarschmidt_automap_2021,liu_colossal-auto_2023} overlook potential opportunities for inter- or intra-layer parallelism, the others optimize inter- and intra-layer parallelism hierarchically and sequentially~\citep{narayanan_pipedream_2019,fan_dapple_2021,he_pipetransformer_2021,tarnawski_efficient_2020,tarnawski_piper_2021,zheng_alpa_2022}. 
As a result, current automatic parallelism techniques often fail to achieve the global optima and instead become trapped in local optima. 
Therefore, a unified inter- and intra-layer approach is needed to enhance the effectiveness of automatic parallelism.


This paper aims to find the optimal parallelism strategy while simultaneously considering inter- and intra-layer parallelism. 
It enables us to search in a more extensive strategy space where the globally optimal solution lurk. 
However, unifying inter- and intra-layer parallelism in automatic parallelism brings us two challenges. 
Firstly, to adopt a unified perspective on the inter- and intra-layer automatic parallelism, we should not formalize them with separate formulations as prior works. Therefore, how can we express these parallelism strategies in a unified formulation? 
Secondly, previous methods take a long time to obtain the solution with a limited strategy space. Therefore, how can we ensure that the best solution can be obtained in a reasonable time while expanding the strategy space?


To solve the above challenges, we propose UniAP. For the first challenge, UniAP adopts the mixed integer quadratic programming~(MIQP)~\citep{lazimy_mixed_1982} to search for the globally optimal parallel strategy automatically. 
It unifies the inter- and intra-layer automatic parallelism in a single MIQP formulation. 
For the second challenge, our complexity analysis and experimental results show that UniAP can obtain the globally optimal solution in a significantly shorter time.


The contributions of this paper are summarized as follows: 
\begin{itemize}
    \item We propose UniAP, the first framework to unify inter- and intra-layer automatic parallelism in model training.
    \item The optimal parallel strategies discovered by UniAP exhibit scalability on training throughput and strategy searching time.
    \item The experimental results show that UniAP speeds up model training on four Transformer-like models by up to 1.70$\times$ and reduces the strategy searching time by up to 16$\times$, compared with the SOTA method.
\end{itemize}

% Figure environment removed

\section{Related Works}
\subsection{Density-aware Dehazing Methods}

In recent years, several methods \cite{zhang2021hierarchical,deng2020hardgan,guo2022image,yang2022self,wang2021haze,yi2022two,yeperceiving} have attempted to improve the dehazing performance by enabling the network to perceive haze density.

\subsubsection{Density-awareness via estimating T-map} Haze density is influenced by several factors and is inversely proportional to T-map, so some methods learn density information by estimating T-map. Lou et al. \cite{lou2020integrating} predict a T-map first for nighttime image dehazing. Zhang et al. \cite{zhang2021hierarchical} estimate a low-resolution T-map and then jointly input the feature map and the estimated T-map to a Laplacian pyramid decoder to achieve a restored image. Yang et al. \cite{yang2022self} propose a semi-supervised method that does not require paired data. The method estimates T-map, scattering coefficient, and depth to reconstruction hazy images and restores clear images. However, these methods require additional labeled data and might be inaccurate due to the complexity of practical scenes \cite{li2017aod}.
% The Haze-Aware Feature Distillation (HARD) module is designed in \cite{deng2020hardgan}, which introduces two dual-channel maps to describe the atmospheric brightness and pixel-wise spatial information of each feature channel respectively, and calculates a haze aware map through an InstanceNorm followed by a sigmoid layer. Finally it fuses the above three factors.

% Figure environment removed

\subsubsection{Density-awareness via extracting density features directly} Research works \cite{deng2020hardgan,chen2020unsupervised,yeperceiving} directly learn haze density information without estimating a T-map. Deng et al. \cite{deng2020hardgan} design a Haze-Aware Representation Distillation (HARD) module to extract global brightness and a haze-aware map. Chen et al. \cite{chen2020unsupervised} propose an attention mechanism based on dark channel prior to describe haze concentration. However, not estimating the T-map would result in a lack of a comparator to measure density. Generating intermediate results and using the information contained therein can address this issue.

\subsection{Dehazing Methods with Intermediate Results}
Considering the difficulty of recovering images directly from the haze input, dehazing methods \cite{bai2022self,chen2021desmokenet,yeperceiving,hong2022uncertainty} which generate intermediate results (or one result) inside the network to facilitate the dehazing process are proposed. Bai et al. \cite{bai2022self} first generate a reference image by a deep pre-dehazer, and then develop a progressive feature fusion module to fuse the hazy and reference features, which achieves high metrics on several datasets. Chen et al. \cite{chen2021desmokenet} first remove light and thick smoke by a Smoke Remove Network (SRN) to gain a coarse output, which is concatenated with the original input and fed to a Pixel Compensation Network (PCN) to recover the missing pixels in the thick smoke. Hong et al. \cite{hong2022uncertainty} propose an Uncertainty-Driven Dehazing Network. In this method, intermediate results are together generated with uncertainty maps for uncertainty features extraction. Ye et al. \cite{yeperceiving} also pre-generate a pseudo-haze-free image. The hazy input and the pseudo-haze-free image are concatenated to estimate a Density Encoding Matrix describing the relationship between haze density and absolute position and mixed up to the following deep layers.

Despite the above methods extracting feature from intermediate results, they do not fully consider the differences between these results and the haze inputs, especially the differences in haze density. Simple concatenation \cite{chen2021desmokenet,bai2022self} or linear summation \cite{yeperceiving} might lead the networks to rely on the uncertain learning process and lose the capture of information about the differences between the two images. In addition, the lack of a targeted design that addresses the relationship between the intermediate results and the original input leads the extracted features not fine enough and limits the dehazing performance.

Our DFR-Net improves on the aforementioned methods by exploring and refining density features through the utilization of density differences between a generated proposal image and the hazy input, thereby achieving an awareness of haze density and superior dehazing performance.


\section{Proposed method}
\label{sec:proposed_method}


\input{sections/submission/figures/method-obs}
\subsection{Preliminaries}
\label{subsec:preliminaries}

To reduce the heavy computations of convolutional layers in neural networks,
the input feature (activation) and weight of each convolutional layer are quantized to low-bit values~\cite{cai2017deep,choi2018pact,jung2019learning,gholami2021survey}.
In this work, as the features of SR network are more sensitive to quantization, we focus more on quantizing features.
The input feature of the $i$-th convolutional layer $x_i\in\mathbb{R}^{B\times C\times H\times W}$, where $B, C, H,$ and $W$ denote the dimension of input batch, channel, height, and width, a quantization operator $Q(\cdot)$ quantizes the feature $x_i$ with bit-width $b$:
\begin{equation}\label{eqn:method-quantmodule-pre}
    Q(x_i) = \text{Int}({ \frac{\text{clip}(x_i, \alpha_l, \alpha_u) - \alpha_l}{s} }) \cdot s + \alpha_l,
\end{equation} 
where $ \text{clip}(\cdot, \alpha_l, \alpha_u)$ truncates the input into the range of $[\alpha_l, \alpha_u]$ and $s = \frac{\alpha_u - \alpha_l}{2^b-1}$.
After truncation, the truncated feature is scaled to the integer range of bit-width $b$, $[0, 2^b-1]$.
Then the scaled feature in the integer range is rounded to integer values with Int$(\cdot)$, and it is rescaled to range $[\alpha_l, \alpha_u]$.
To obtain better quantization ranges for SR networks, 
range parameters $\alpha_l, \alpha_u$ for each layer are generally learned through quantization-aware training~\cite{Li2020pams,zhong2022ddtb}.
Since the rounding function is not differentiable, a straight-through estimator (STE)~\cite{bengio2013estimating} is used to train the range parameters in an end-to-end manner.
We initialize $\alpha_u$ and $\alpha_l$ as the $j$-th and $100-j$-th percentile value of feature averaged among the training data.
$j$ is set as 1 in our experiments to avoid outliers from corrupting the quantization range.
Similarly, to quantize the weight of the $i$-th convolutional layer $w^i$, quantization operator $Q(\cdot)$ is used.
However, instead of setting range parameters as learnable parameters, 
$\alpha_l, \alpha_u$ for weights are fixed as the $j$-th and $100-j$-th percentile of weights.



\subsection{Distribution mismatch in SR networks}
\label{subsec:motivation}

Quantization unfriendliness of SR networks is from the diverse feature (activation) distributions, as reported in previous studies~\cite{Li2020pams,hong2022daq,zhong2022ddtb}, mainly due to the absence of batch normalization layers in SR networks.
Existing SR quantization methods address this issue by employing one~\cite{Li2020pams} or two~\cite{zhong2022ddtb} learnable quantization range parameters for each convolutional layer feature.
However, despite that the quantization-aware training process aims to find the optimal range for each feature, it fails to account for the channel-wise and input-wise variance in distributions.
As illustrated in Figure~\ref{fig:method-motiv}, where notable discrepancies exist between layer-wise and channel-wise distributions, quantization grids are needlessly allocated to regions with minimal feature density.
This mismatch in inter-channel distributions leads to performance degradation when quantizing SR networks.
In the following sections, we introduce a new quantization-aware training scheme to address the distribution mismatch problem.






\subsection{Cooperative variance regularization}
\label{subsec:var-reg}

\input{sections/submission/figures/method-loss}
Instead of focusing on finding a better quantization range parameter capable of accommodating the diverse feature distributions, our approach aims to regularize the distribution diversity beforehand.
Obtaining an appropriate quantization range for a feature with low variance is an easier task compared to that of high variance.
In this work, we define the overall mismatch of a feature distribution with the standard deviation, 
\begin{equation}
    M(x_i) \coloneqq \sigma(x_i),
\label{eq:mismatch}
\end{equation}
where $\sigma(\cdot)$ calculates the standard deviation of the feature.
Thus, variance regularization can be directly applied to the feature to be quantized $(x_i)$, which is formulated as follows:
\begin{equation}
    \mathcal{L}_{V}(x_i) = \lambda_V \cdot M(x_i),
\label{eq:varianceloss}
\end{equation}
where 
$\lambda_V$ is the hyperparameter that denotes the weight of regularization. 
The overall $\mathcal{L}_{V} = \sum_{i}^{\# \;\text{layers}}{ \mathcal{L}_{V}(x_i)}$ is obtained by summing over all quantized convolutional layers. 
The variance regularization loss can be used in line with the reconstruction loss, which is originally used in the general quantization-aware training process.
The optimization of parameter $\theta^t$ is formulated as follows:
\label{subsec:loss}
\begin{equation}
    {\theta}^{t+1} = \theta^{t} - \alpha^{t} (\nabla_\theta \mathcal{L}_R(\theta^t) + \nabla_\theta \mathcal{L}_V(\theta^t) ), 
\end{equation}
where $\nabla_\theta \mathcal{L}_R(\theta^t)$ the gradient from the original reconstruction loss and $\nabla_\theta \mathcal{L}_V(\theta^t)$ denotes the gradient from variance regularization loss and  $\alpha^t$ denotes the learning rate. 
Updating the network to minimize the variance regularization loss will reduce the quantization error of each feature.

However, then a question arises, \textit{does reducing the quantization error of each feature lead to improved reconstruction accuracy}?
The answer is, according to our observation in Figure~\ref{fig:method-conflict}, not necessarily.
During the training process, the variance regularization loss can collide with the original reconstruction loss.
However, we want to avoid the conflict between two losses, in other words, minimize the variance as long as it does not hinder the reconstruction loss.
Thus, we determine whether the two losses are cooperative or not by examining the sign of the gradients of each loss.
If the signs of the gradients are equal, then the parameter is updated in the same direction by two losses. 
By contrast, if the sign values are inverse, the two losses restrain each other, thus we only employ the reconstruction loss.
In summary, we leverage variance regularization for parameters that the gradients have the same sign value as that from the reconstruction loss.
Our optimization can be formulated as follows:
\begin{equation}
    \theta^{t+1} = 
    \begin{cases}
    \theta^{t} - \alpha^{t} (\nabla_\theta \mathcal{L}_R(\theta^t) + \nabla_\theta \mathcal{L}_V(\theta^t) ),  & \quad \nabla_\theta \mathcal{L}_R(\theta^t) \cdot \nabla_\theta \mathcal{L}_V(\theta^t)\geq0, 
    \\
    \theta^{t} - \alpha^{t} (\nabla_\theta \mathcal{L}_R(\theta^t) ), & \quad \nabla_\theta \mathcal{L}_R(\theta^t) \cdot \nabla_\theta \mathcal{L}_V(\theta^t)<0.
    \end{cases}
    \label{eq:cooperativeloss}    
\end{equation} 
This allows the network to reduce the quantization error cooperatively with the reconstruction error.


\subsection{Distribution offsets}
\label{subsec:offsets}
\input{sections/submission/figures/method-motivation}

The variance of the distribution can be reduced to a certain extent via variance regularization in Section~\ref{subsec:var-reg}.
However, since regularization is applied only when it is cooperative with the SR reconstruction, the gap between distributions remains.
In this section, we explore the remaining gap between distributions.
First, as visualized in Figure~\ref{fig:method-motiv}, we observe that the distribution gap is larger (and more critical) in the channel dimension compared to the image dimension.


Also, we find that this extent of the channel-wise gap in the distribution is different for each layer of the SR network, as shown in Figure~\ref{fig:method-offset}.
Some layers (Figure~\ref{fig:method-offset-b}) exhibit a larger mismatch in the distribution deviation, while others (Figure~\ref{fig:method-offset-c}) show a larger mismatch in the distribution average.
The quantization errors of the layer with a large mismatch in distribution mean can be decreased by shifting the channel-wise feature.
On the other hand, the mismatch in layers with large divergence in distribution deviation can be reduced by scaling each channel-wise distribution for overall similar distributions.

Since channel-wise shifting and scaling incur computational overhead and not all layers are in need of additional shifting and scaling (Figure~\ref{fig:method-offset-a}), we selectively apply offset scaling/shifting to layers that can maximally benefit from it.
The standards for our selection are derived by feeding a patch of images to the 32-bit pre-trained network and calculating the mismatch in average/deviation of each layer.
Given the $i$-th feature statistics $\hat{x}^i$ from the pre-trained network, the mismatch of $i$-th convolutional layer is formulated as follows:
\begin{equation}
\begin{aligned}
    M_\mu^i \coloneqq \sigma(\mu_{c}(\hat{x}_i)) \quad \text{and} \quad M_\sigma^i \coloneqq \sigma(\sigma_{c}(\hat{x}_i)),
\label{eq:indicators}
\end{aligned}
\end{equation}
where $\mu_c(\cdot)$ and $\sigma_c(\cdot)$ respectively calculate the channel-wise mean and standard deviation of a feature and $\sigma(\cdot)$ calculates the standard deviation.
After all $M_\mu^i$s and $M_\sigma^i$s  $(i=1,\cdots,\#\text{layers})$ are collected,
we apply additional scaling offsets to top-p layers with high $M_\sigma^i$ value and shifting offsets to top-p layers with high $M_\mu^i$ value.
The shifting and scaling process for feature $x^i$ of the $i$-th convolutional layer is formulated as follows:
\begin{equation}
\begin{aligned}
    x_i^* = x_i + S_\mu, \quad \text{if} \; M_\mu^i \in \text{top-}p([M_\mu^1, \cdots, M_\mu^{\# \text{layers}}]),
    \label{eq:offsets-b}
\end{aligned}
\end{equation}
\begin{equation}
    x_i^* = x_i \cdot S_\sigma, \quad \text{if} \; M_\sigma^i \in \text{top-}p([M_\sigma^1, \cdots, M_\sigma^{\# \text{layers}}]),
\label{eq:offsets-w}
\end{equation}
where $S_\mu, S_\sigma \in \mathbb{R}^{C}$ are learnable parameters, top-$p(\cdot)$ constructs a set that contains values greater than the $100(1-p)$-percentile value of the given set.  
$p$ is the hyperparameter that determines the ratio of layers to apply distribution offsets, which we set to $0.3$ in our experiments.
Moreover, both offsets $S_\mu$ and $S_\sigma$ are quantized to low-bit, 4-bit in our experiments, to minimize the computational overhead.
Consequently, the offsets additionally incur only $0.02\%$ overhead to the network storage size for EDSR.
The offsets further relieve distribution mismatch with minimal overhead.


\subsection{Overall training}


In this section, 
we formally describe our algorithm: a simple variation of the model-based algorithm called {\em Monotonic Value Propagation} proposed by \citet{zhang2020reinforcement}. 
%
We present the full procedure in Algorithm~\ref{alg:main}, and point out several key ingredients. 
%
\begin{itemize}
	\item {\em Optimistic updates using upper confidence bounds (UCB).} The algorithm implements the optimism principle in the face of uncertainty 
by adopting the frequently used UCB-based framework \citep{azar2017minimax,jin2018q}. 
More specifically, the learner maintains upper estimates for both the value and Q-function, 
by calculating the following optimistic Bellman equation backward (from $h=H,\ldots,1$): 
%
\begin{subequations}
\begin{align}
Q_{h}(s,a)\, & \leftarrow\,\min\big\{\widehat{r}_{h}(s,a)+\langle\widehat{P}_{s,a,h},V_{h+1}\rangle+b_{h}(s,a),H\big\}, 
	\label{eq:Qh-UCB-informal}\\
V_{h}(s)\, & \leftarrow\, \max\nolimits_{a}Q_{h}(s,a). 
\end{align}
\end{subequations}
%
Here, $Q_{h}$ (resp.~$V_h$) is the running estimate for the Q-function (resp.~value function), 
		$\widehat{r}_{h}(s,a) \in \mathbb{R}$ is an estimate of the mean reward at $(s,a,h)$, 
$\widehat{P}_{s,a,h}\in \mathbb{R}^S$ indicates an estimate of the transition probability vector from $(s,a,h)$, 
whereas $b_{h}(s,a)\geq 0$ is some suitably chosen bonus term that compensates for the uncertainty.   



	\item {\em Monotonic bonus functions.} Another crucial step in order to ensure near-optimal regret lies in careful designs of the data-driven bonus terms $\{b_h(s,a)\}$ in \eqref{eq:Qh-UCB-informal}. 
		Here, we adopt the monotonic bonus function for MVP originally proposed in \citet{zhang2020reinforcement}, 
		to be made precise in \eqref{eq:update1}. 
		Compared to the bonus function in $\mathtt{Euler}$~\citep{zanette2019tighter} and $\mathtt{UCBVI}$~\citep{azar2017minimax},  the monotonic bonus form has a cleaner structure that effectively avoid large lower order terms. In order to enable variance-aware regret, we also need to keep track of the empirical variance of the (stochastic) immediate rewards.


	\item {\em An epoch-based procedure and a doubling trick.} 
%
		A key step of our algorithm is to update the empirical transition kernel and empirical rewards in an epoch-based fashion. 
		More concretely, the whole learning process is divided into several consecutive epochs via a simple doubling rule.
		That is, once the number of visits to a $(s,a,h)$-tuple reaches a power of 2, we end the current epoch,  reconstruct the empirical transition kernel and rewards using data from this epoch (cf.~lines~\ref{line:r-hsa-update} and \ref{line:P-hsa-update} of Algorithm~\ref{alg:main}), compute the Q-function and value function using the newly updated transition kernel and rewards (cf.~\eqref{eq:updateq}), and then start a new epoch. In each epoch, the learned policy is induced by the optimistic Q-function estimate computed based on the empirical transition kernel of the {\em current} epoch. 




\end{itemize}




\begin{remark}[Doubling batch]
We note that a doubling update rule has also been used in the original MVP \citep{zhang2020reinforcement}. 
A major difference between our modified version and the original one is that: when the visitation count for some $(s,a,h)$ reaches $2^i$ for some integer $i$, we only use the second half of the samples (i.e., the $\{2^{i-1}+j\}_{j=1}^{2^{i-1}}$-th samples) to compute the empirical model, whereas the original MVP makes use of all the $2^i$ samples. This step is crucial for decoupling statistical dependence.

\end{remark}







\begin{algorithm}[h]
	\DontPrintSemicolon
\caption{Monotoinic Value Propagation (MVP)~\citep{zhang2020reinforcement}\label{alg:main}}
%\begin{algorithmic}[ht]
	\textbf{input:} state space $\mathcal{S}$, action space $\mathcal{A}$, horizon $H$, total number of episodes $K$, confidence parameter $\delta$, 
	%$\mathcal{L}=\{1,2,\ldots, 2^{\log_2K}\}$, 
	$c_1=\frac{460}{9}$, $c_2 = 2\sqrt{2}$ and $c_3=\frac{544}{9}$.  \\
%\textbf{Initialize:} . \\
	\textbf{initialization: } for all $(s,a,s',h)\in \mathcal{S}\times \mathcal{A}\times\mathcal{S}\times [H]$, set $\theta_h(s,a)\leftarrow 0$, $\kappa_h(s,a)\leftarrow 0$, $\overline{N}_h(s,a,s')\leftarrow 0$, $N_h(s,a,s')\leftarrow 0$, $n_h(s,a)\leftarrow 0$, $Q_h(s,a)\leftarrow H-h+1$, $V_h(s)\leftarrow H-h+1$. \\
	\For{$k=1,2,...$} {
		Set $\pi^k$ such that $\pi_h^k(s) = \arg\max_{a}Q_h(s,a)$ for all $s\in \mathcal{S}$ and $h\in [H]$. {\color{blue}\tcc{policy iterate.}}
		\For {$h=1,2,...,H$} {
			Observe $s_{h}^k$, 
			take action $ a_h^k= \arg\max_{a}Q_h(s_h^k,a)$, 
%	%	\STATE{ Receive reward $r_h^k$ (receive $c_h^k$ and set $r_h^k = -c_h^k$ for the cost case) and  observe $s_{h+1}^k$.}
			receive  $r_h^k$,  observe $s_{h+1}^k$. \label{line:choose_action} 
			{\color{blue}\tcc{sampling.}}
			$(s,a,s')\leftarrow (s_h^k,a_h^k,s_{h+1}^k)$. \\
			Update $\overline{N}_h(s,a) \leftarrow  \overline{N}_h( s,a )+1$, $N_h(s,a,s') \leftarrow   N_h(s,a,s')+1$, $\theta_h(s,a)\leftarrow \theta_h(s,a)+r_h^k$, $\kappa_h(s,a)\leftarrow \kappa_h(s,a)+(r_h^k)^2$. \\
		{\color{blue}\tcc{perform updates using data of this epoch.}}
		\If{$\overline{N}_h(s,a)\in \{1,2,\ldots, 2^{\log_2K}\}$ \label{line:rp_update_start} }   {
			$n_h(s,a)\leftarrow \sum_{\widetilde{s}}N_h(s,a,\widetilde{s})$. 
			{\color{blue}\tcp{number of visits to $(s,a,h)$ in this epoch.}}
			$\widehat{r}_h(s,a)\leftarrow \frac{\theta_h(s,a)}{n_h(s,a)}$. \label{line:r-hsa-update}
			{\color{blue}\tcp{empirical rewards of this epoch.}} 
			$\widehat{\sigma}_h(s,a)\leftarrow \frac{\kappa_h(s,a)}{n_h(s,a) }  $. 
			{\color{blue}\tcp{empirical squared rewards of this epoch.}}
			$\widehat{P}_{s,a,h}(\widetilde{s}) \leftarrow  \frac{N_h(s,a,\widetilde{s})}{n_h(s,a)}$ for all $\widetilde{s} \in \mathcal{S}$.  \label{line:P-hsa-update}
			{\color{blue}\tcp{empirical transition for this epoch.}}
			%
			Set TRIGGERED = TRUE, 
			and $\theta_h(s,a)\leftarrow 0$, $\kappa_h(s,a)\leftarrow 0$,  $N_h(s,a,\widetilde{s})\leftarrow 0$  for all $\widetilde{s}\in \mathcal{S}$. 
%		\ENDIF \label{line:rp_update_end}
		}
		}
		{\color{blue}\tcc{optimistic Q-estimation using empirical model of this epoch.}}
		\If {\textnormal{TRIGGERED= TRUE}} {
			Set TRIGGERED = FALSE, and $V_{H+1}(s)\leftarrow 0$ for all $s\in \mathcal{S}$. \\
			\For{$h=H,H-1,...,1$} {
				%
				\For{$(s,a)\in \mathcal{S}\times \mathcal{A}$} {

					%\vspace{-3ex}
					\begin{align} 
						\vspace{-3ex}
						b_h(s,a) &\leftarrow c_1 \sqrt{\frac{   \mathbb{ V}(\widehat{P}_{s,a,h} ,V_{h+1}) \log \frac{1}{\delta}  }{ \max\{n_h(s,a),1 \} }}+c_2 \sqrt{\frac{\big(\widehat{\sigma}_h(s,a)- (\widehat{r}_h(s,a))^2 \big)\log \frac{1}{\delta}}{\max\{n_h(s,a),1\}}} \qquad\qquad\qquad\qquad\qquad\qquad\nonumber\\
						&\qquad\qquad\qquad +c_3\frac{H\log \frac{1}{\delta}}{ \max\{n_h(s,a) ,1\}  },  \label{eq:update1}  \\
						Q_h(s,a) &\leftarrow \min\big\{    \widehat{r}_h(s,a)+\langle \widehat{P}_{s,a,h}, V_{h+1} \rangle +b_h(s,a)    ,H\big\},\,
						V_{h}(s) \leftarrow \max_{a}Q_h(s,a).
						\label{eq:updateq}
					\end{align}
					\vspace{-3ex}
				}
			}
			%
		}
	}
%\end{algorithmic}
\end{algorithm}






















\begin{comment}

\begin{algorithm}
	\DontPrintSemicolon
\caption{Monotoinic Value Propagation (MVP)~\citep{zhang2020reinforcement}\label{alg:main}}
\begin{algorithmic}[ht]
\STATE{\textbf{Input:} number of states $S$, number of actions $A$, horizon $H$, total number of episodes $K$, confidence parameter $\delta$;}
\STATE{\textbf{Initialize:} $\mathcal{L}=\{1,2,\ldots, 2^{\log_2(K)}\}$; $c_1=\frac{460}{9}, c_2 = 2\sqrt{2},  c_3=\frac{544}{9}$;}
\STATE{\textbf{Initialization: } $\theta_h(s,a)\leftarrow 0, \kappa_h(s,a)\leftarrow 0, \overline{N}_h(s,a,s')\leftarrow 0, N_h(s,a,s')\leftarrow 0, n_h(s,a)\leftarrow 0, Q_h(s,a)\leftarrow H-h+1, V_h(s)\leftarrow H-h+1, \forall (s,a,s',h)\in \mathcal{S}\times \mathcal{A}\times\mathcal{S}\times [H]$;}
\FOR {$k=1,2,...$}
		\FOR {$h=1,2,...,H$}
		\STATE{ Observe $s_{h}^k$;}
		\STATE{ Take action $ a_h^k= \arg\max_{a}Q_h(s_h^k,a)$;} \label{line:choose_action}
	%	\STATE{ Receive reward $r_h^k$ (receive $c_h^k$ and set $r_h^k = -c_h^k$ for the cost case) and  observe $s_{h+1}^k$.}
 \STATE{ Receive reward $r_h^k$  and  observe $s_{h+1}^k$.}
		\STATE{ Set $(s,a,s')\leftarrow (s_h^k,a_h^k,s_{h+1}^k)$;.}
		\STATE{ Set $\overline{N}_h(s,a) \leftarrow  \overline{N}_h( s,a )+1$, \ \,$N_h(s,a,s') \leftarrow   N_h(s,a,s')+1$, $\theta_h(s,a)\leftarrow \theta_h(s,a)+r_h^k$, $\kappa_h(s,a)\leftarrow \kappa_h(s,a)+(r_h^k)^2$.}
		\STATE{ \verb|\\| \emph{Update empirical rewards and transition probability}}
		\IF {$\overline{N}_h(s,a)\in \mathcal{L}$}  \label{line:rp_update_start}
  \STATE{Set $\widehat{r}_h(s,a)\leftarrow \frac{\theta_h(s,a)}{\sum_{\widetilde{s}'}N_h(s,a,\widetilde{s}')}$;}
  \STATE{Set $\widehat{\sigma}_h(s,a)\leftarrow \frac{\kappa_h(s,a)}{\sum_{\widetilde{s}'}N_h(s,a,\widetilde{s'}) } - (\widehat{r}_h(s,a))^2 $}
		\STATE Set $\widehat{P}_{s,a,h,\widetilde{s}} \leftarrow  \frac{N_h(s,a,\widetilde{s})}{\sum_{\widetilde{s}'}N_h(s,a,\widetilde{s}')}$ for all $\widetilde{s} \in \mathcal{S}$.
		%			\ENDFOR
  \STATE{}
		\STATE{ Set $n_h(s,a)\leftarrow \sum_{\widetilde{s}'}N_h(s,a,\widetilde{s}')$;}
		\STATE{ Set TRIGGERED = TRUE.}
  \STATE{$\theta_h(s,a)\leftarrow 0, \kappa_h(s,a)\leftarrow 0,  N_h(s,a,\widetilde{s})\leftarrow 0, \forall \widetilde{s}\in \mathcal{S}$;}
		\ENDIF \label{line:rp_update_end}
		\ENDFOR
		\STATE{ \verb|\\| \emph{Update $Q$-function}}
		\IF {TRIGGERED}
  \STATE{$V_{H+1}(s)\leftarrow 0,\forall s\in \mathcal{S}$;}
		\FOR{$h=H,H-1,...,1$}
		\FOR{$(s,a)\in \mathcal{S}\times \mathcal{A}$}
		%		  \vspace{-3ex}
		\STATE 	 {		%\vspace{-0.5cm} 	
			%					\vspace{-0.5cm}
			%			\simon{Maybe move to the line $N(s,a) \in \mathcal{L}$ so it's clear for the reader that we update the policy only in the set.}
			Set
			\begin{align} 
			~~~~~~~~~~~~~&b_h(s,a)\leftarrow c_1 \sqrt{\frac{   \mathbb{ V}(\widehat{P}_{s,a,h} ,V_{h+1}) \log(\frac{1}{\delta})  }{ \max\{n_h(s,a),1 \} }}+c_2 \sqrt{\frac{(\widehat{\sigma}_h(s,a)- (\widehat{r}_h(s,a))^2 )\log(\frac{1}{\delta})}{\max\{n_h(s,a),1\}}}+c_3\frac{H\log(\frac{1}{\delta})}{ \max\{n_h(s,a) ,1\}  },  \label{eq:update1}  \\
			%\\ 	\hspace{-20ex} 	&  \left\{\begin{array}{l}   Q_h(s,a)\leftarrow \min\{    \widehat{r}_h(s,a)+\widehat{P}_{s,a,h} V_{h+1} +b_h(s,a)    ,H\} \quad  \text{for reward case} \\ Q_h(s,a)\leftarrow \max\{\min \left\{ \widehat{r}_h(s,a) + \widehat{P}_{s,a,h}V_{h+1}+b_h(s,a), 0 \right\} ,-H\} \quad \text{for cost case}\label{eq:updateq}\end{array}\right.
   \hspace{-20ex} 	&    Q_h(s,a)\leftarrow \min\{    \widehat{r}_h(s,a)+\widehat{P}_{s,a,h} V_{h+1} +b_h(s,a)    ,H\}\label{eq:updateq}
			\\ & V_{h}(s) \leftarrow \max_{a}Q_h(s,a).\nonumber
			\end{align}
			\vspace{-3ex}
		}
		\ENDFOR
		\ENDFOR
		\STATE{ Set TRIGGERED = FALSE}
		\ENDIF
		\ENDFOR
\end{algorithmic}
\end{algorithm}

\end{comment}



Algorithm~\ref{algo-odm} summarizes the overall pipeline for our framework, ODM.
We follow the common practice~\cite{Li2020pams,zhong2022ddtb} to use $\mathcal{L}_1$ loss and $\mathcal{L}_\text{SKT}$ loss for the reconstruction loss as follows:
\begin{equation}
    \mathcal{L}_R = \mathcal{L}_1 + \lambda \mathcal{L}_{SKT}, 
\label{eq:overallloss}
\end{equation}
where
$\mathcal{L}_1$ loss indicates the $l_1$ distance between the reconstructed image and the ground-truth HR image, and $\mathcal{L}_\text{SKT}$ loss is the $l_2$ distance between the structural features of the quantized network and the 32-bit pre-trained network.
The structural features are obtained from the last layer of the high-level feature extractor.
The balancing weight $\lambda$ is set as $1000$ in our experiments.
Also, the weight $\lambda_V$ to balance $\mathcal{L}_V$ and $\mathcal{L}_R$ in Eq.~\ref{eq:varianceloss} is set differently depending on the mismatch severeness of the SR architecture. 
We provide detailed settings in the supplementary materials.







% % Figure environment removed


% % Figure environment removed

% % Figure environment removed

% % Figure environment removed

% % Figure environment removed

% % Figure environment removed

% Figure environment removed


% Figure environment removed

% Figure environment removed


\subsection{Implementation Details}


\paragraph{Network.} In order to disentangle shape and color latent information within the hashgrids, we split the single hash table in the NeRF network architecture of Instant-NGP~\cite{mueller2022instant} into two: a density grid $\mathcal{G}^{\sigma}$ and a color grid $\mathcal{G}^c$, with the same settings as the original density grid in the open-source PyTorch implementation torch-ngp~\cite{torch-ngp}. We do this to make it possible to make fine-grained edits of one to one of the color or geometry properties without affecting the other. The rest of the network architecture remains the same, including a sigma MLP $f^\sigma$ and a color MLP $f^c$. For a spatial point $\mathbf{x}$ with view direction $\mathbf{d}$, the network predicts volume density $\sigma$ and color $c$ as follows:
\begin{align}
    \sigma, \mathbf{z} &= f^\sigma(\mathcal{G}^{\sigma}(\mathbf{x})) \\
    c &= f^c(\mathcal{G}^c(\mathbf{x}),\mathbf{z},\mathrm{SH}(\mathbf{d}))
\end{align}
where $\mathbf{z}$ is the intermediate geometry feature, and $\mathrm{SH}$ is the spherical harmonics directional encoder~\cite{mueller2022instant}. The same as Instant-NGP's settings, $f^\sigma$ has 2 layers with hidden channel 64, $f^c$ has 3 layers with hidden channel 64, and $\mathbf{z}$ is a 15-channel feature.

We compare our modified NeRF network with the vanilla architecture in the Lego scene of NeRF Blender Synthetic dataset\cite{mildenhall2020nerf}. We train our network and the vanilla network on the scene for 30,000 iterations. The result is as follows:
\begin{itemize}
    \item Ours: training time 441s, PSNR 35.08dB
    \item Vanilla: training time 408s, PSNR 34.44dB
\end{itemize}
We observe slightly slower runtime and higher quality for our modified architecture, indicating that this modification causes negligible changes.

\paragraph{Training.}
% We use Instant-NGP\fcite{NGP} as our editing framework backbone to achieve real-time editing preview. 
We select Instant-NGP~\cite{mueller2022instant} as the NeRF backbone of our editing framework.
Our implementations are based on the open-source PyTorch implementation torch-ngp~\cite{torch-ngp}. All experiments are run on a single NVIDIA RTX 3090 GPU. Note that we make a slight modification to the original network architecture. Please refer to the supplementary material for details.

During the pretraining stage, we set $\msymbol{weight_pretrain_color}=\msymbol{weight_pretrain_sigma}=1$ and the learning rate is fixed to $0.05$. During the finetuning stage, we set $\msymbol{weight_train_color} = \msymbol{weight_train_depth} = 1$ with an initial learning rate of 0.01. 
% The bit field mask of the editing space is filled so that the editing space can be fully sampled during training. 
Starting from a pretrained NeRF model, we perform 50-100 epochs of local pretraining (for about 0.5-1 seconds) and about 50 epochs of global finetuning (for about 40-60 seconds). The number of epochs and time consumption can be adjusted according to the editing type and the complexity of the scene. Note that we test our performance in the absence of tiny-cuda-nn~\cite{tiny-cuda-nn} which achieves superior speed to our backbone, which indicates that our performance has room for further optimization.
% Note that the training speed is evaluated when tiny-cuda-nn is not enabled.

\paragraph{Datasets.}
We evaluate our editing in the synthetic\Skip{lego, chair, and ship from} NeRF Blender Dataset~\cite{mildenhall2020nerf}, and the real-world captured \Skip{family and truck from}Tanks and Temples~\cite{Knapitsch2017} and \Skip{, and scan83 from} DTU~\cite{jensen2014large} datasets. We follow the official dataset split of the frames for the training and evaluation.


% Figure environment removed

% Figure environment removed

% Figure environment removed

\subsection{Experimental Results}
\label{sec-results}
% \paragraph{Comparisons of rendering quality between teacher and student network.} 


\paragraph{Qualitative NeRF editing results.} 
We provide extensive experimental results in all kinds of editing categories we design, including bounding shape (\cref{fig-bbox,fig-bbox-elf}), brushing (\cref{fig-brush}), anchor (\cref{fig-anchor}), and color (\cref{fig-teaser}). Our method not only achieves a huge performance boost, supporting instant preview at the second level but also produces more visually realistic editing appearances, such as shading effects on the lifted side in \cref{fig-brush} and shadows on the bumped surface in \cref{fig-neumesh}. Besides, results produced by the student network can even outperform the teacher labels, \eg in \cref{fig-bbox-elf} the $F^t$ output contains floating artifacts due to view inconsistency. As analyzed in \cref{sec-train}, the distillation process manages to eliminate this. We also provide an example of object transfer (\cref{fig-bbox-baby}): the bulb in the Lego scene (of Blender dataset) is transferred to the child's head in the family scene of Tanks and Temples dataset.
% \Skip{
% We evaluate our method on all the editing types we design, \ie bounding shape, brushing and anchor, respectively:
% \begin{itemize}
%     \item Bounding shape editing. As shown in \cref{fig-bbox}, we scale the warning light on the top of the Lego model, shorten the chair leg, \zjs{TBD}, and provides plausible results.
%     \item Brushing and color editing. As shown in \cref{fig-brush}, our method edits the scene according to the user's paintings (\ie a cross sign on the chair back, a heart shape on the car logo, and \zjs{TBD}). Note that our brushing method supports simultaneous geometry lifting, as shown in the ``cross'' example. Due to our shading preservation strategy in HSL space, the edited surface can contain realistic visual effects (see the shading effects of the lifted surface).
%     \item Anchor editing. As shown in \cref{fig-anchor}, our method edits the scene according to the anchor points (\ie ship's bow, bulldozer's shovel and \zjs{TBD}) and the stretching direction. The edited geometry has consistent appearance with the anchored area.
% \end{itemize}
% }

% Figure environment removed

% Figure environment removed

\paragraph{Comparisons to baselines.} Existing works have strong restrictions on editing types, which focus on either geometry editing or appearance editing, while ours is capable of doing both simultaneously. Our brushing and anchor tools can create user-guided out-of-proxy geometry structures, which no existing methods support. We make comparisons on color and texture painting supported by NeuMesh~\cite{neumesh} and Liu \etal~\cite{liu2021editing}. 

\cref{fig-neumesh} illustrates two comparisons between our method and NeuMesh~\cite{neumesh} in scribbling and a texture painting task. Our method significantly outperforms NeuMesh, which contains noticeable color bias and artifacts in the results. In contrast, our method even succeeds in rendering the shadow effects caused by geometric bumps.

\cref{fig-neumesh-mic} illustrates the results of the same non-rigid blending applied to the Mic from NeRF Blender\cite{mildenhall2020nerf}. It clearly shows that being mesh-free, We have more details than NeuMesh\cite{neumesh}, unlimited by mesh resolution.

\cref{fig-editnerf} shows an overview of the pixel-wise editing ability of existing NeRF editing methods and ours. Liu \etal~\cite{liu2021editing}'s method does not focus on the pixel-wise editing task and only supports textureless simple objects in their paper. Their method causes an overall color deterioration within the edited object, which is highly unfavorable. This is because their latent code only models the global color feature of the scene instead of fine-grained local features. Our method supports fine-grained local edits due to our local-aware embedding grids.

% \yq{describe the difference}

% \paragraph{Artistic applications (a comic on NeRF).} Based on the four example tools we implemented, we created a comic \textit{Bob the Bulb} (Fig. \ref{fig-comic}) to show the potential applications of our editing method. This might be the first artwork created with NeRF.

% \subsection{Experiments on Bounding Shape Editing}
% \subsection{Experiments on Brush Editing}
% \subsection{Experiments on Anchor Editing}
% \subsection{Experiments on Color Shape Editing}
% \subsection{Real-world Example: NeRF-Rendered Comic}


\subsection{Ablation Studies}
\label{sec-ablation}
\paragraph{Effect of the two-stage training strategy.} To validate the effectiveness of our pretraining and finetuning strategy, we make comparisons between our full strategy (3\textsuperscript{rd} row), finetuning-only (1\textsuperscript{st} row) and pretraining-only (2\textsuperscript{nd} row) in \cref{fig-ab_pre}. Our pretraining can produce a coarse result in only 1 second, while photometric finetuning can hardly change the appearance in such a short period. The pretraining stage also enhances the subsequent finetuning, in 30 seconds our full strategy produces a more complete result. However, pretraining has a side effect of local overfitting and global degradation. Therefore, our two-stage strategy makes a good balance between both and produces optimal results.

\paragraph{MLP fixing in the pretraining stage.} In \cref{fig-ab_fix}, we validate our design of fixing all MLP parameters in the pretraining stage. The result confirms our analysis that MLP mainly contains global information so it leads to global degeneration when MLP decoders are not fixed.


We proposed a machine-learning based method to approximate diagonal as well as non-diagonal elements of the Hessian of a molecule. The representation used is specific for every internal coordinates, and takes explicitly into account the bond order, which is sensible because a single point DFT calculation is computationally considerably less expensive that the explicit calculation of the Hessian.
We trained our ML model on a relatively small dataset (subset of QM7) of less than 7000 molecules. The Hessian was computed at the B3LYP/cc-pVDZ level of theory. 
The agreement between ML and DFT was satisfactory. In particular, the calculated MAPE for bond stretching force constant was below 2\%, and was particularly small for bonds involving hydrogen atoms because they point outwards and are less affected by the chemical environment. The MAPE for bending and torsion was of 5\% and 10\%, respectively. 
From the ML model trained on QM7 we were also able to predict the Hessian of some molecules representative of the QM9 dataset. The Hessian predicted in internal coordinates was then transformed into the mass-weighted Cartesian Hessian, the diagonalization of which yields the harmonic vibrational frequencies and normal modes, that can be compared to the ones calculated  explicitly from DFT.

High frequency vibrations and normal modes were predicted accurately, while lower frequency ones were not. This behaviour is analogous to the IR spectroscopy theory, where stretchings and bendings can be identified accurately, while torsion and delocalized vibrations are more difficult to be interpreted.

The approximate Hessian obtained with ML is computational inexpensive and can be used as an initial Hessian guess for geometry optimization, or in the context of alchemical geometry relaxation \cite{Domenichini2020,domenichini2022alchemical, shiraogawa2022exploration,shiraogawa2023optimization}. 
A good starting Hessian may speed up the convergence of the geometrical optimization. An in detail assessment of the performance of the ML Hessian proposed is not yet provided, but should carefully take into account many parameters on which the optimization depends, \textit{e.g.} the type of molecule, the initial geometry, the optimization algorithm, and the Hessian update scheme.



This work was supported in part by the Fundamental Research Funds for the Central Universities; NSFC under Grants (62103372, 62088101, 62233013); the Key Research and Development Program of Zhejiang Province (2021C03037); Zhejiang Lab (121005-PI2101); Information Technology Center and State Key Lab of CAD\&CG, Zhejiang University. 


% ---- Bibliography ----
%
% BibTeX users should specify bibliography style 'splncs04'.
% References will then be sorted and formatted in the correct style.
%
\bibliographystyle{splncs04}
\bibliography{main}

\clearpage
\title{Supplementary Material for\texorpdfstring{\\}{} ``Overcoming Distribution Mismatch\texorpdfstring{\\}{} in Quantizing Image Super-Resolution Networks''}
\titlerunning{ODM}
\author{Cheeun Hong\inst{1}\orcidlink{0009-0009-3480-748X} \and
Kyoung Mu Lee\inst{1,2}\orcidlink{0000-0001-7210-1036}}
\authorrunning{C.~Hong and K.M.~Lee}
\institute{
Dept. of ECE \& ASRI, \email{\{cheeun914, kyoungmu\}@snu.ac.kr} \and
IPAI, Seoul National University
}
\maketitle

\renewcommand{\thetable}{S\arabic{table}}
\renewcommand{\thesection}{S\arabic{section}}
\renewcommand{\thefigure}{S\arabic{figure}}
\renewcommand{\theequation}{S\arabic{equation}}


In this supplementary material, we present additional experimental results in \Cref{sec:sup-experiments}; additional ablation study in \Cref{sec:sup-ablation}; additional analyses in \Cref{sec:sup-analyses}.


\section{Additional Experiments}
\label{sec:sup-experiments}
Along with the evaluations on SR networks of scale $\times4$ discussed in the main manuscript, we also assess our framework on networks of scale $\times2$.
As shown in Table~\ref{tab:sup-scale2}, our framework outperforms existing SR quantization methods in terms of both PSNR and SSIM, demonstrating its effectiveness on scale $\times2$ SR networks.
Specifically, the PSNR gain on Set5 is 0.28 dB on EDSR, 0.74 dB on RDN, and 0.25 dB on SwinIR.
These results confirm that our framework is effective for both CNN- and Transformer-based SR networks of scale 2.
\begin{table}[ht]
\centering
\setlength{\tabcolsep}{1.2mm}
\caption{ \textbf{Quantitative comparisons on SR networks of scale $\times2$} 
}
\makebox[\linewidth]{\scriptsize
    \scalebox{1.25}{
    \begin{tabular}{l c cc cc cc cc}
        \toprule
        \multirow{2}{*}{Model} & \multirow{2}{*}{Bit} & 
        \multicolumn{2}{c}{Set5} & \multicolumn{2}{c}{Set14} & \multicolumn{2}{c}{B100} & \multicolumn{2}{c}{Urban100} \\
        \cmidrule(lr){3-4} \cmidrule(lr){5-6} \cmidrule(lr){7-8} \cmidrule(lr){9-10}& 
         & PSNR & SSIM & PSNR & SSIM & PSNR & SSIM & PSNR & SSIM \\
        \midrule
        EDSR   &32 & 37.93 & 0.960 & 33.46 & 0.916 & 32.10 & 0.899 & 31.71 & 0.925 \\
        \midrule
        EDSR-PAMS &2 &35.30&0.946 &31.63&0.899 &30.66&0.879 &28.11&0.875\\
        EDSR-DAQ  &2 &36.82&0.955 &32.50&0.908 &31.34&0.891 &29.85&0.905\\
        EDSR-DDTB &2 &37.25&0.958 &32.87&0.911 &31.67&0.893 &30.34&0.910\\
        EDSR-ODM (Ours)&2 &\textbf{37.53}& \textbf{0.958} & \textbf{33.06} & \textbf{0.913} & \textbf{31.81} & \textbf{0.895} & \textbf{30.81} & \textbf{0.915} \\
        \midrule
        RDN   &32 & 38.05 & 0.961 & 33.59 & 0.917 & 32.20 & 0.900 & 32.13 & 0.927 \\
        \midrule
        RDN-PAMS &2 &35.45&0.946 &31.67&0.899 &30.69&0.879 &28.14&0.874\\
        RDN-DAQ & 2 &37.23&0.957 &32.84&0.910 &31.66&0.893 &30.46&0.908\\
        RDN-DDTB &2 &36.76&0.955 &32.54&0.908 &31.44&0.890 &29.77&0.903\\
        RDN-ODM (Ours)&2 &\textbf{37.50}& \textbf{0.958} & \textbf{33.03} & \textbf{0.913} & \textbf{31.80} & \textbf{0.895} & \textbf{30.57} & \textbf{0.913} \\
        \midrule
        SwinIR  &32 & 38.14 & 0.961 & 33.86 & 0.921 & 32.31 & 0.901 & 32.76 & 0.934 \\
        \midrule
        SwinIR-PAMS &2 & 35.38 & 0.947 & 31.63 & 0.899 & 30.65 & 0.880 & 28.07 & 0.873\\
        SwinIR-DAQ  &2 & 34.98 & 0.943 & 31.38 & 0.896 & 30.47 & 0.876 & 27.83 & 0.869\\
        SwinIR-DDTB &2 & 37.17 & 0.957 & 32.78 & 0.911 & 31.42 & 0.888 & 30.24 & 0.908\\
        SwinIR-ODM (Ours)&2 &\textbf{37.42}& \textbf{0.958} & \textbf{33.03} & \textbf{0.913} & \textbf{31.79} & \textbf{0.895} & \textbf{30.76} & \textbf{0.914} \\
        \bottomrule
    \end{tabular}
}
}
\label{tab:sup-scale2}
\end{table}





Furthermore, we compare our method with existing methods by training each method for 300K iterations.
The results in \Cref{tab:sup-ep300} show that the gains achieved by our approach for 60K iterations reported in the main manuscript are maintained.
Our framework still achieves more than a 0.37 dB gain over other methods for Set5 when trained for extended iterations.
\begin{table}[ht]
\centering
\setlength{\tabcolsep}{1.2mm}
\caption{ \textbf{Quantitative comparisons of SR quantization methods with 300K iterations} 
}
\makebox[\linewidth]{\scriptsize
    \scalebox{1.25}{
    \begin{tabular}{l c cc cc cc cc}
        \toprule
        \multirow{2}{*}{Model} & \multirow{2}{*}{Bit} & 
        \multicolumn{2}{c}{Set5} & \multicolumn{2}{c}{Set14} & \multicolumn{2}{c}{B100} & \multicolumn{2}{c}{Urban100} \\
        \cmidrule(lr){3-4} \cmidrule(lr){5-6} \cmidrule(lr){7-8} \cmidrule(lr){9-10}& 
         & PSNR & SSIM & PSNR & SSIM & PSNR & SSIM & PSNR & SSIM \\
        \midrule
        EDSR   &32 & 32.10&0.894 &28.58&0.781 &27.56&0.736 &26.04&0.785\\
        \midrule
        EDSR-PAMS &2 &30.05&0.852 &27.17&0.744 &26.70&0.705 &24.09&0.707\\
        EDSR-DAQ &2 & 31.11&0.874 &27.98&0.765 &27.14&0.720 &24.96&0.745\\
        EDSR-DDTB &2 &31.19&0.878 &27.97&0.767 &27.14&0.723 &25.01&0.749\\
        EDSR-ODM (Ours)&2 &\textbf{31.56}& \textbf{0.884} & \textbf{28.15} & \textbf{0.771} & \textbf{27.30} & \textbf{0.728} & \textbf{25.24} & \textbf{0.758} \\
        \bottomrule
    \end{tabular}
}
}
\label{tab:sup-ep300}
\end{table}





Also, to ensure a fair comparison with DAQ~\cite{hong2022daq}, we follow their settings and apply our method to EDSR of 32 residual blocks with 256 channel dimensions.
As shown in \Cref{tab:sup-fulledsr}, our method outperforms DAQ even though DAQ employs a channel-wise quantization function, whereas our method utilizes a more efficient layer-wise function.
\begin{table}[ht]
\centering
\setlength{\tabcolsep}{1.2mm}
\caption{ \textbf{Quantitative comparisons on EDSR (full) of scale $\times4$} which consists of 32 residual blocks and 256 channel dimensions. For a fair comparison with DAQ, our model (ODM$^*$) is trained for 300K iterations.
}
\makebox[\linewidth]{\scriptsize
    \scalebox{1.25}{
    \begin{tabular}{l c cc cc cc cc}
        \toprule
        \multirow{2}{*}{Model} & \multirow{2}{*}{Bit} & 
        \multicolumn{2}{c}{Set5} & \multicolumn{2}{c}{Set14} & \multicolumn{2}{c}{B100} & \multicolumn{2}{c}{Urban100} \\
        \cmidrule(lr){3-4} \cmidrule(lr){5-6} \cmidrule(lr){7-8} \cmidrule(lr){9-10}& 
         & PSNR & SSIM & PSNR & SSIM & PSNR & SSIM & PSNR & SSIM \\
        \midrule
        EDSR-full & 32 & 32.46 & 0.897 & 28.80 & 0.788 & 27.72 & 0.742 & 26.64 & 0.803 \\
        \midrule
        EDSR-full-DAQ & 2 & 32.05 & 0.890 & 28.53 & 0.778 & 27.50 & 0.733 & 25.97 & 0.781 \\
        EDSR-full-ODM$^*$ (Ours)&2& \textbf{32.15} & \textbf{0.893} & \textbf{28.56} & \textbf{0.781} & \textbf{27.57} & \textbf{0.737} & \textbf{26.04} & \textbf{0.786}\\
        \bottomrule
    \end{tabular}
    }
}
\label{tab:sup-fulledsr}
\end{table}


Moreover, we compare our method with a fully-quantized SR network, EDSR-FQSR~\cite{Wang2021fully}, in which all layers and also the skip connections are quantized.
For a fair comparison, we also quantize all convolutional layers and the skip connections.
The results in \Cref{tab:sup-fully} show that our ODM outperforms FQSR, indicating that our approach is also effective when the network is fully quantized.
\begin{table}[!ht]
\centering
\caption{ \textbf{Quantitative comparisons on EDSR with fully quantized method.} 
S.C. refers to the bit-width of skip-connections.
}
\renewcommand{\arraystretch}{1.2}
\setlength{\tabcolsep}{0.8mm}
\makebox[\linewidth]{\scriptsize
    \scalebox{1.2}{
    \begin{tabular}{cl cc cc cc cc cc}
        \toprule
        \multirow{2}{*}{Scale}& \multirow{2}{*}{Model} & \multirow{2}{*}{Bit} & \multirow{2}{*}{S.C.} & 
        \multicolumn{2}{c}{Set5} & \multicolumn{2}{c}{Set14} & \multicolumn{2}{c}{B100} & \multicolumn{2}{c}{Urban100} \\
        \cmidrule(lr){5-6} \cmidrule(lr){7-8} \cmidrule(lr){9-10} \cmidrule(lr){11-12}& 
         & & & PSNR & SSIM & PSNR & SSIM & PSNR & SSIM & PSNR & SSIM \\
        \midrule
         & EDSR &32 & 32 & 32.10&0.894 &28.58&0.781 &27.56&0.736 &26.04&0.785\\
        % \midrule
        \cmidrule(lr){2-12}
        $\times4$ & EDSR-FQSR & 4 & 8 & 30.93 & 0.870 & 27.82 & 0.761 & 27.07 & 0.715 & 24.93 & 0.744\\
        & EDSR-ODM (Ours) & 4 & 8 & \textbf{31.99} & \textbf{0.890} & \textbf{28.42} & \textbf{0.777} & \textbf{27.47} & \textbf{0.733} & \textbf{25.70} & \textbf{0.775}\\
        \midrule
        & EDSR &32 & 32 & 37.93 & 0.960 & 33.46 & 0.916 & 32.10 & 0.899 & 31.71 & 0.925\\
        \cmidrule(lr){2-12}
        $\times2$ & EDSR-FQSR & 4 & 8 & 37.04 & 0.951 & 32.84 & 0.908 & 31.67 & 0.889 & 30.65 & 0.911\\
        & EDSR-ODM (Ours) & 4 & 8 & \textbf{37.86} & \textbf{0.960} & \textbf{33.42} & \textbf{0.916} & \textbf{32.08} & \textbf{0.898} & \textbf{31.71} & \textbf{0.924}\\
        \bottomrule
    \end{tabular}
}}
\label{tab:sup-fully}
\end{table}




Additionally, along with SwinIR, as demonstrated in the main manuscript, we also apply our method to a more recent, large Transformer-based model, HAT~\cite{chen2023activating} ($\sim$10M parameters).
The results in Table~\ref{tab:sup-hat} indicate that our method can be effectively applied to Transformer models.
\begin{table}[ht]
\centering
\setlength{\tabcolsep}{1.2mm}
\caption{ \textbf{Quantitative comparisons on HAT of scale $\times{4}$} 
}
\makebox[\linewidth]{\scriptsize
    \scalebox{1.25}{
    \begin{tabular}{l c cc cc cc cc}
        \toprule
        \multirow{2}{*}{Model} & \multirow{2}{*}{Bit} & 
        \multicolumn{2}{c}{Set5} & \multicolumn{2}{c}{Set14} & \multicolumn{2}{c}{B100} & \multicolumn{2}{c}{Urban100} \\
        \cmidrule(lr){3-4} \cmidrule(lr){5-6} \cmidrule(lr){7-8} \cmidrule(lr){9-10}& 
         & PSNR & SSIM & PSNR & SSIM & PSNR & SSIM & PSNR & SSIM \\
        \midrule
        HAT   &32 & 32.92&0.905 &29.15 &0.796 &27.97&0.751 &27.87&0.835\\
        \midrule
        HAT-PAMS &2 &30.30&0.859 &27.35&0.750 &26.78&0.710 &24.23&0.713\\
        HAT-DAQ  &2 &30.21&0.856 &27.29&0.747 &26.74&0.708 &24.14&0.708\\
        HAT-DDTB &2 &31.23&0.878 &27.96&0.766 &27.16&0.724 &25.08&0.751\\
        HAT-ODM (Ours) &2 &\textbf{32.06}& \textbf{0.891} & \textbf{28.56} & \textbf{0.780} & \textbf{27.53} & \textbf{0.736} & \textbf{26.10} & \textbf{0.787} \\
        \bottomrule
    \end{tabular}
}
}
\label{tab:sup-hat}
\end{table}




\section{Additional Ablation Study}
\label{sec:sup-ablation}
In this section, we present an ablation study on the hyperparameters of our framework.
First, we conduct an ablation study on the percentile $j$, which is used to initialize quantization range clipping parameters.
The results in \Cref{tab:sup-ablation-p} show that when $j$=100, meaning the quantization range is not clipped and is determined by the maximum value, accuracy severely degrades.
This hints that clipping is important for performance and that using the max function does not serve as an effective initialization policy.

Also, we analyze the impact of the gradient balance terms $\lambda_R$ and $\lambda_M$, which balances the gradient of the reconstruction loss and the mismatch regularization loss, and initial learning rate $\beta^0$.
The results in \Cref{tab:sup-ablation-m,tab:sup-ablation-r} support our choice of $\lambda_R$=1, $\lambda_M$=1e-5, and $\beta^0$=1e-4.
\begin{table}[!ht]
\caption{
    \textbf{Ablation on hyperparameters} on EDSR $\times4$ (2-bit)
}
\centering
\setlength{\tabcolsep}{0.8mm}
    \newcommand{\h}{0.135\linewidth}
    \subfloat[Ablation on j\label{tab:sup-ablation-p}]{
        \resizebox*{!}{\h}{
            \begin{tabular}{lc ccc}
                \toprule
                $j$ & Set5 & Set14 & B100 & U100 \\
                \midrule
                100 & 29.63&26.86&26.53&23.82\\
                \textbf{99} & 31.50&28.14&27.27&25.17\\ 
                95 & 31.51&28.13&27.27&25.18\\ 
                \bottomrule
            \end{tabular}
        }
    }
    \subfloat[Ablation on $\lambda_M$ \label{tab:sup-ablation-r}]{
        \resizebox*{!}{\h}{
            \begin{tabular}{lc ccc}
                \toprule
                $\lambda_M$ & Set5 & Set14 & B100 & U100 \\
                \midrule
                2e-5 & 31.40 & 28.07 & 27.22 & 25.09\\
                \textbf{1e-5} & 31.50 & 28.14 & 27.27 & 25.17\\ 
                1e-6 & 31.48 & 28.15 & 27.28 & 25.18\\ 
                \bottomrule
            \end{tabular}
        }
    }
    \subfloat[Ablation on $\beta^0$ \label{tab:sup-ablation-m}]{
        \resizebox*{!}{\h}{
            \begin{tabular}{lc ccc}
                \toprule
                $\beta^0$ & Set5 & Set14 & B100 & U100 \\
                \midrule
                1e-3 & 30.82 & 27.72 & 27.01 & 24.63\\
                \textbf{1e-4} & 31.50 & 28.14 & 27.27 & 25.17\\ 
                1e-5 &  31.17 & 27.96 & 27.14 & 24.93\\ 
                \bottomrule
            \end{tabular}
        }
    }
\label{tab:sup-ablation}
\end{table}

Furthermore, we compare our weight clipping correction scheme with the commonly used quantization scheme for weights, LSQ~\cite{esser2019learned}.
The results in \Cref{tab:sup-ablation-lsq} validate the effectiveness of our clipping correction approach.
\begin{table}[!ht]
    \renewcommand{\arraystretch}{1.2}
    \setlength{\tabcolsep}{1.2mm}
    \centering
    \aboverulesep=0ex
    \belowrulesep=0ex
    \caption{\textbf{Comparison of WCC with LSQ} on EDSR $\times4$ (2-bit)
    }
    \resizebox{0.85\linewidth}{!}{
        \begin{tabular}{l| cc}
            \toprule
            Method & Set5 (PSNR/SSIM) & Urban100 (PSNR/SSIM) \\
            \midrule
            LSQ + Cooperative MR & 31.26 / 0.877 & 25.02 / 0.746\\
            WCC + Cooperative MR (Ours) & 31.50 / 0.882 & 25.17 / 0.755\\ 
            \bottomrule
        \end{tabular}
    }
    \label{tab:sup-ablation-lsq}
\end{table}


\section{Additional Analyses}
\label{sec:sup-analyses}
\subsection{Complexity Analysis}
\label{subsec:sup-complexity}
We provide additional complexity analysis on SwinIR in \Cref{tab:sup-complexity}.
The results show that our method achieves superior SR performance with minimal or no computational overhead in terms of model storage size and bitOPs.
BitOPs for SwinIR are calculated for processing a 64$\times$64 input patch.
\begin{table}[!ht]
    \centering
    \setlength{\tabcolsep}{1.4mm}
    \renewcommand{\arraystretch}{1.2}
    \aboverulesep=0ex
    \belowrulesep=0ex
    \caption{ 
        \textbf{Computational complexity comparison} on SwinIR of scale $\times4$
    }
    \resizebox{0.7\textwidth}{!}{
        \begin{tabular}{l|c|cc|cc}
            \toprule
            Model & Bit & Storage size & BitOPs & PSNR & SSIM \\
            \midrule
            SwinIR & 32 & 929.6K & 5.071T & 32.44 & 0.898 \\
            \midrule
            SwinIR-PAMS & 2 & 160.2K & 1.100T & 29.48 & 0.834\\
            SwinIR-DAQ  & 2 & 160.1K & 1.176T & 29.10 & 0.824 \\
            SwinIR-DDTB & 2 & 160.4K & 1.100T & 31.01 & 0.873 \\
            SwinIR-ODM (Ours) & 2 & \textbf{160.4K} & \textbf{1.100T} & \textbf{31.44} & \textbf{0.880} \\
            \bottomrule
        \end{tabular}
    }
\label{tab:sup-complexity}    
\end{table}

\subsection{Distribution mismatch}
For the generalizability of the observed distribution mismatch problem of the main manuscript, we analyze the variance of features in SR networks across a set of images.
As reported in \Cref{tab:sup-analyses-mismatch}, the classification network (ResNet20) exhibits much less image-wise and channel-wise variance compared to SR networks (EDSR, RDN).
This suggests that the distribution mismatch problem is particularly severe in SR networks. 
\begin{table}[!ht]
    \setlength{\tabcolsep}{1.4mm}
    \caption{
    {\textbf{Average variance in feature.}} The metrics are measured on DIV2K validation set for SR networks and ImageNet validation set for the classification network.} 
    \centering
    \resizebox{0.85\linewidth}{!}{
            \begin{tabular}{llc ccc}
                \toprule
                Task & Model & Image-wise Variance & Channel-wise Variance \\
                \midrule
                Image super-resolution & EDSR ($\times4$) & 15.08 & 40.29\\
                Image super-resolution & RDN ($\times4$) & 6.40 & 58.14 \\
                Image classification & ResNet-20 & 0.04 & 0.09\\
                \bottomrule
            \end{tabular}
    }
    \label{tab:sup-analyses-mismatch}
\end{table}


Moreover, we analyze the feature mismatch after quantization-aware training (QAT) using different methods.
To track feature similarity, we compute the variance between feature distribution statistics (mean and standard deviation) within the benchmark dataset.
As shown in \Cref{tab:sup-analyses-mismatch-reduction}, the feature mismatch is effectively reduced with our framework.
\begin{table}[!ht]
    \renewcommand{\arraystretch}{1.2}
    \setlength{\tabcolsep}{1.4mm}
    \centering
    \aboverulesep=0ex
    \belowrulesep=0ex
    \caption{
        {\textbf{Feature mismatch after quantization-aware training.}} The metrics are measured on DIV2K validation set for SR networks and ImageNet validation set for the classification network.
    } 
    \resizebox{0.65\textwidth}{!}{
    \begin{tabular}{l|ccc}
        \toprule
        Similarity metric & Var[mean] & Var[std] & Avg. Mismatch\\
        \midrule
        Before QAT & 0.28 & 3.60 & 2.41e+02 \\
        \midrule
        After QAT w/ \textbf{ODM} & \textbf{0.25} & \textbf{1.02} & \textbf{1.37e+02}\\
        After QAT w/ PAMS & 0.29 & 2.34 & 4.04e+02 \\
        After QAT w/ DDTB & 2.54 & 1.67 & 5.46e+05\\
        After QAT w/ DAQ & 2.14 & 6.59  & 6.10e+06\\
        \bottomrule
    \end{tabular}
    }
    \label{tab:sup-analyses-mismatch-reduction}
\end{table}



% We provide additional analysis supporting the choice of distance from the quantization grid as a measure of distribution mismatch in Eq.~(2) of the main manuscript.
We provide additional analysis supporting the choice of distance from the quantization grid as a measure of distribution mismatch in \cref{eq:mismatch} of the main manuscript.
We note that QAT is a process searching for a quantization grid that best fits the discrepant input distributions. 
If the average distance of each feature from the quantization grid is small, it implies that most features are aligned with the grid, indicating a low distribution mismatch.
According to \Cref{tab:sup-analyses-mismatch-reduction}, it is verified that using distance as the mismatch measure reduces the variance in feature distribution statistics across test images.



\subsection{Cooperative Regularization}
\label{subsec:sup-cooperative}
In the main manuscript, we emphasized the importance of \textit{cooperatively} using mismatch regularization and reconstruction losses.
For the cooperative update, we weigh the gradient of mismatch regularization using the cosine similarity with the gradient of the reconstruction loss.
The weighing term is formulated as $0.5 \cdot (cos(\vv_a, \vv_b)+1)$.
Here, we present results using the weighing term as $cos(\vv_a, \vv_b)$ following Du~\etal~\cite{du2018adapting} and that of $u(cos(\vv_a, \vv_b))$ where $u(\cdot)$ is the unit step function.
The results in \Cref{tab:sup-cooperative} show that all these functions that alleviate the conflict between the two losses achieve high reconstruction accuracy, indicating that the general cooperative property is the key to performance gain.
\begin{table}[!ht]
    \renewcommand{\arraystretch}{1.2}
    \setlength{\tabcolsep}{1.2mm}
    \centering
    \aboverulesep=0ex
    \belowrulesep=0ex
    \caption{\textbf{Different formulation for cooperative behaviour} on EDSR $\times4$ (2-bit)
    }
    \resizebox{0.85\linewidth}{!}{
        \begin{tabular}{l| cc}
            \toprule
            Formulation & Set5 (PSNR/SSIM) & Urban100 (PSNR/SSIM) \\
            \midrule
            $cos(\vv_a, \vv_b)$~\cite{du2018adapting} & 31.46 / 0.881 & 25.14 / 0.755 \\
            $u(cos(\vv_a, \vv_b))$ & 31.49 / 0.883 & 25.15 / 0.756\\
            $0.5 \cdot (cos(\vv_a, \vv_b)+1)$ (Ours) & 31.50 / 0.882 & 25.17 / 0.755\\ 
            \bottomrule
        \end{tabular}
    }
    \label{tab:sup-cooperative}
\end{table}


\subsection{More Visualizations}
\label{subsec:sup-visualizations}
For better comprehension, we provide additional results of the effect of our loss after training in \Cref{fig:sup-dist}.
% Along with the results in Figure~{5} of the main manuscript, these results show that our loss term updates the activation distributions to a further quantization-friendly state while mostly preserving the high-density values of the original distribution.
Along with the results in \Cref{fig:exp-dist} of the main manuscript, these results show that our loss term updates the activation distributions to a further quantization-friendly state while mostly preserving the high-density values of the original distribution.
\newcommand{\sidecaption}[2][\empty]% #1=caption, #2=image
{\bgroup% use local registers
  \sbox0{#2}% measure image
  \rotatebox[origin=Bl]{90}{\parbox{\ht0}{\subcaption[position=above]{#1}}}%
  \usebox0
\egroup}
% Figure environment removed





Moreover, we provide visualizations of layer-wise mismatch in weights for different SR networks in \Cref{fig:sup-weight}.
According to the visual results, the weight distributions of different layers have a similar mean (\ie, near 0), but exhibit varying minimum and maximum values.
This motivates us to use a layer-wise different policy for determining the weight quantization range.
\input{sections/figures_sup/sup-weight}

\subsection{Training Time}
\label{subsec:sup-traintime}

Although our framework primarily aims to achieve an accurate quantized SR network in which the inference cost is reduced via quantization, we also provide comparisons on the training time.
According to \Cref{tab:sup-traintime}, our training scheme requires a shorter training time than DDTB and DAQ.
Although our training incurs slightly more time overhead compared to PAMS, the gains in test accuracy compensate for this additional training cost.
\begin{table}[!ht]
    \centering
    \setlength{\tabcolsep}{1.2mm}
    \caption{
    \textbf{Training time} of QAT methods on SR networks. The training time is measured by running the experiment on a single RTX 2080Ti GPU.
    % \vspace{-2mm}
    } 
    % 똑같이 맞춰주기? to ImageNet validation set?
    \resizebox{0.8\linewidth}{!}{
        \begin{tabular}{l cccc}
            \toprule
            Method & EDSR-PAMS & EDSR-DAQ & EDSR-DDTB & EDSR-ODM (Ours) \\
            \midrule
            Time (hours) & 1.5 & 3.7 & 2.5 & 2.4 \\ 
            \bottomrule
        \end{tabular}
    }
    \label{tab:sup-traintime}
\end{table}



\subsection*{License of the Used Assets}
\begin{itemize}
    \item[$\bullet$] DIV2K~\cite{agustsson2017ntire}  dataset is publicly available for academic research purposes.
    \item[$\bullet$] Set5~\cite{bevilacqua2012low}, Set14~\cite{ledig2017photo}, BSD100~\cite{martin2001database}, Urban100~\cite{huang2015single} datasets are made available at \href{https://github.com/jbhuang0604/SelfExSR}{https://github.com/jbhuang0604/SelfExSR}.
\end{itemize}




\end{document}