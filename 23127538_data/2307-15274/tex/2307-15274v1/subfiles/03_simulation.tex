%%%%%%%%%%%%%%%%%%%%%%%%%%%%%%%%%%%%%% SIMULATION %%%%%%%%%%%%%%%%%%%%%%%%%%%%%%%%%%%%%%%%%%

\section{Simulations} \label{subsec:simulation}

To supplement discussions, we illustrate the proposed model through numerical and microscopic traffic simulations.

\subsection{Particle Simulations} \label{subsubsec:particlesimulation}

We compared numerically simulated distributions of $\hat{m}$ with the theoretical distributions of $\hat{m}$.

\subsubsection{Method}

In Julia 1.8.5, the number of probe footprints was modelled as a series of particles with independent uniform linear motion along a road segment. In this experiment, the emergence of binomial distributions (Equation \ref{eq:yathe2}) was assumed trivial. The Distributions.jl package was used to generate statistical distributions under the following two scenarios: scenario 1 ($d = 300$ and $t = 4$) and scenario 2 ($d = 40$ and $t = 1$). In each scenario, $m \in \{1, 2, 4, 8\}$ and $S \sim g(s)$ as shown in Figure \ref{fig:figure3}a. We performed one million simulations using Equation \ref{eq:draso} for each combination of scenarios and $m$.

\subsubsection{Results} \label{subsubsec:results1}

Table \ref{tab:table1} exhibits the descriptive statistics of simulations and theory, while Figure \ref{fig:figure9} shows the histograms of simulated $\hat{m}$ and theoretical PDFs of $\hat{m}$ calculated by Equation \ref{eq:hnmnmi}. The simulation results showed a good match in descriptive statistics between simulated values and theoretical values.

As seen in Figure \ref{fig:figure9}, $\hat{m}$ distributes around $m$, but the PDFs are not necessarily line-symmetric with respect to $\hat{m} = m$.
The PDFs approached normal distributions as $m$ increased.

%% The AAS Journal's LaTeX/AASTeX table creator: https://authortools.aas.org/LATEX/make-latex.html
\begin{deluxetable}{cccccc}[H]
\caption{Descriptive Statistics of $\hat{m}$ in Simulations and Theory}
\label{tab:table1}
\tablenum{1}
\tablehead{\colhead{Scenario} & \colhead{$m$} & \colhead{Item} & \colhead{E[$\hat{m}$]} & \colhead{Var[$\hat{m}$]} & \colhead{CV[$\hat{m}$]}}
%% All data must appear between the \startdata and \enddata commands
\startdata
1 & 1 & Simulated & 1.000 & 0.019 & 0.137 \\
 &  & Theoretical & 1 & 0.019 & 0.137 \\
 & 2 & Simulated & 2.000 & 0.037 & 0.097 \\
 &  & Theoretical & 2 & 0.037 & 0.097 \\
 & 3 & Simulated & 4.000 & 0.075 & 0.068 \\
 &  & Theoretical & 4 & 0.075 & 0.068 \\
 & 4 & Simulated & 8.000 & 0.150 & 0.048 \\
 &  & Theoretical & 8 & 0.149 & 0.048 \\
2 & 1 & Simulated & 1.000 & 0.088 & 0.297 \\
 &  & Theoretical & 1 & 0.088 & 0.297 \\
 & 2 & Simulated & 2.000 & 0.177 & 0.210 \\
 &  & Theoretical & 2 & 0.177 & 0.210 \\
 & 3 & Simulated & 4.000 & 0.353 & 0.148 \\
 &  & Theoretical & 4 & 0.353 & 0.149 \\
 & 4 & Simulated & 7.999 & 0.706 & 0.105 \\
 &  & Theoretical & 8 & 0.706 & 0.105 \\
\enddata
\end{deluxetable}

% Multiple figures
% Figure environment removed

\subsection{Regressions on Microscopic Traffic Simulations} \label{subsubsec:microscopic trafficsimulation}

To develop traffic volume estimation models, $\hat{m}$ is compared to known traffic volumes to calibrate the ratio of traffic volumes to the estimated probe traffic volumes. Here, we briefly illustrate how the theoretical variance of $\hat{m}$ can improve traffic volume estimation accuracy through a fitter model.

\subsubsection{Method}

%% The AAS Journal's LaTeX/AASTeX table creator: https://authortools.aas.org/LATEX/make-latex.html
\begin{deluxetable}{cccccc}[H]
\tablecaption{Traffic Volume Data Used in Regressions}
\tablenum{2}
\label{tab:table2}
\tablehead{\colhead{Site number} & \colhead{Modelled speed distribution} & \colhead{ADT} & \colhead{$m$} & \colhead{$d$} & \colhead{VMR[$\hat{m}$]}\\ 
\colhead{} & \colhead{(m/s)} & \colhead{} & \colhead{} & \colhead{(m)} & \colhead{}  } 
%% All data must appear between the \startdata and \enddata commands
\startdata
4945 & $N(26.82,5.00)$ & 763 & 107 & 14 & 0.916 \\
4953 & $N(26.82,5.00)$ & 27 & 4 & 53 & 0.028 \\
4961 & $N(26.82,5.00)$ & 19 & 3 & 64 & 0.035 \\
4977 & $N(26.82,5.00)$ & 34 & 5 & 56 & 0.025 \\
4985 & $N(26.82,5.00)$ & 618 & 87 & 52 & 0.030 \\
5001 & $N(26.82,5.00)$ & 751 & 105 & 62 & 0.033 \\
5025 & $N(26.82,5.00)$ & 706 & 99 & 25 & 0.092 \\
5057 & $N(26.82,5.00)$ & 83 & 12 & 27 & 0.059 \\
5065 & $N(26.82,5.00)$ & 275 & 39 & 46 & 0.066 \\
5081 & $N(26.82,5.00)$ & 52 & 7 & 56 & 0.025 \\
5089 & $N(26.82,5.00)$ & 38 & 5 & 35 & 0.116 \\
5097 & $N(13.41,5.00)$ & 183 & 26 & 41 & 0.019 \\
5113 & $N(26.82,5.00)$ & 110 & 15 & 10 & 1.682 \\
5121 & $N(13.41,5.00)$ & 539 & 75 & 41 & 0.019 \\
5129 & $N(13.41,5.00)$ & 353 & 49 & 62 & 0.008 \\
5145 & $N(26.82,5.00)$ & 448 & 63 & 20 & 0.341 \\
5185 & $N(26.82,5.00)$ & 668 & 94 & 53 & 0.028 \\
5201 & $N(13.41,5.00)$ & 96 & 13 & 21 & 0.086 \\
5217 & $N(26.82,5.00)$ & 277 & 39 & 39 & 0.111 \\
5225 & $N(26.82,5.00)$ & 232 & 32 & 66 & 0.035 \\
9193 & $N(26.82,5.00)$ & 48 & 7 & 57 & 0.026 \\
9201 & $N(26.82,5.00)$ & 8 & 1 & 64 & 0.035 \\
9209 & $N(26.82,5.00)$ & 63 & 9 & 17 & 0.578 \\
9233 & $N(26.82,5.00)$ & 289 & 40 & 52 & 0.030 \\
9249 & $N(26.82,5.00)$ & 216 & 30 & 7 & 2.831 \\
9257 & $N(26.82,5.00)$ & 820 & 115 & 57 & 0.026 \\
9281 & $N(13.41,5.00)$ & 226 & 32 & 46 & 0.014 \\
9289 & $N(26.82,5.00)$ & 446 & 62 & 25 & 0.092 \\
9297 & $N(26.82,5.00)$ & 52 & 7 & 70 & 0.030 \\
9305 & $N(26.82,5.00)$ & 135 & 19 & 49 & 0.045 \\
9313 & $N(26.82,5.00)$ & 221 & 31 & 50 & 0.039 \\
9321 & $N(26.82,5.00)$ & 121 & 17 & 41 & 0.101 \\
9329 & $N(26.82,5.00)$ & 41 & 6 & 28 & 0.059 \\
9353 & $N(26.82,5.00)$ & 46 & 6 & 31 & 0.090 \\
\enddata
\tablecomments{Speed distributions had been truncated at $s = 0$ and $s = 60$. The modelled speed distributions did not necessarily match the actual speed distribution at each site in Hudspeth County.}
\end{deluxetable}

We used PTV Vissim 11.00-02 to microscopically simulate vehicular traffic on one-lane straight road links 300 m in length. There were 730 links, each carrying 500 vehicles over 24 h. Vehicle speed distributions were $N(26.82, 5.00)$ for 365 links and $N(13.41, 5.00)$ for the remaining 365 links \citep{OPPENLANDER-1963}. The recording interval was $t = 1$.

Thirty-four average annual traffic volumes (AADTs) of fewer than 1,000 recorded in Hudspeth County, Texas, in 2021 \citep{TXDOT-2022} were used as the ground truth for average daily traffic (ADT) (Table \ref{tab:table2}). The probe penetration rate was assumed to be 2 \% everywhere, and the observation period was seven days; therefore, $m \approx (\mathrm{ADT}) \times 0.02 \times 7$ in this experiment. A random integer between 5 and 70 was set as $d$ at each location. Table \ref{tab:table2} summarises modelled speed distribution, ADT, $m$, $d$, and theoretical VMR[$\hat{m}$] (Equation \ref{eq:ghvry}) by site.

Based on $m$, virtual probes were randomly assigned to vehicles that had already been simulated. For instance, at site 5121, one of the 365 links simulating the speed of $N(13.41, 5.00)$ was randomly chosen to represent this location. Within the link, a virtual cordon ($d = 41$) was set at a random location, and trajectories from 75 randomly chosen vehicles were captured within the virtual cordon to compute $\hat{m}$ using Equation \ref{eq:draso}. This procedure was performed at all the sites in each trial.

To estimate ADTs based on $\hat{m}$, the experiment assumed that two out of the 34 ADTs were known (e.g., when traffic counters existed) and calculated the ratios of ADT to $\hat{m}$ at these sites. In each trial, regression models were created using Julia's GLM.jl package for all possible combinations (i.e., ${}_{34} \mathrm{ C }_2 = 561$) of “known” ADT counts. For every combination of known ADTs, an ordinary least square (OLS) regression and a linear regression with weighted least square (WLS) using the reciprocal of VMR[$\hat{m}$] (Equation \ref{eq:ghvry}) were performed while assuming the intercept was 0.

We performed 2,023 trials. In each trial, MAPE was calculated among the 32 estimated ADTs (i.e., excluding the pair used to develop each regression model) against the ground truth ADTs with every possible combination of “known” ADTs. The mean MAPE of the 561 combinations was considered as the average MAPE of the trial. If our model is correct, linear regressions on observed $\hat{m}$ inversely weighted by VMR[$\hat{m}$] should estimate the traffic volumes better than OLSs do.

\subsubsection{Results} \label{subsubsec:results2} 

The OLS and WLS resulted in the the mean average coefficients of determination were $R^2 = 0.970$ for OLS and $R^2 = 0.986$ for WLS. The mean average MAPE of OLS was 0.097, whereas that of WLS was 0.086. Although both methods estimated ADTs well, the results should be interpreted relatively. Figure \ref{fig:figure10} shows sorted differences in average MAPE between OLS and WLS, where positive values indicate improvements in WLS (i.e., average MAPE of WLS subtracted from that of OLS). The WLS yielded a better average MAPE than OLS in 2,001 out of the 2,023 (98.91 \%) trials. As we had predicted, the results exemplified the efficacy of Equation \ref{eq:ghvry} for developing a traffic volume estimation model.

% Figure environment removed
