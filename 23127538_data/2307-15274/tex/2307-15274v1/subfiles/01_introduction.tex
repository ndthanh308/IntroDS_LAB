\section{Introduction} \label{sec:intro}

Traffic volume is a fundamental element of transportation engineering \citep{GREENSHIELDS-1934}, urban planning, real estate valuation, air pollution models \citep{LURIAetal-1990, OKAMOTOetal-1990}, wildlife protection \citep{SEILERandHELLDIN-2006}, and marketing \citep{ALEXANDERetal-2005}. Traffic counts are typically performed at fixed locations using equipment such as pneumatic tubes, loop coils, radars, ultrasonic sensors, video cameras, and light detection and ranging (LiDAR) systems \citep{ZHAOetal-2019}. While conventional traffic counts are believed to have acceptable precision, traffic counts at fixed locations are constrained in space, time, and budget. For this reason, average annual daily traffic (AADT), which is one of the basic traffic metrics in traffic engineering, is often estimated based on 24- or 48-hour traffic counts with temporal adjustments \citep{JESSBERGERetal-2016, KRILE-2016, RITCHIE-1986}. Nevertheless, this scalability constraint still places transportation professionals in a leash. For example, researchers have pointed out a lack of reliable traffic volume data in substantive road safety analyses \citep{CHENetal-2019, ELBASYOUNY-2010, MITRAandWASHINGTON-2012, ZAREIandHELLINGA-2022}. 

To maximise the value of limited numbers of traffic counts, extensive research efforts have been devoted to developing traffic volume estimation methods focused on calibration and its accuracy. Such approaches include travel demand modelling \citep{ZHONGandHANSON-2009}, spatial kriging \citep{SELBYandKOCKELMAN-2013}, support vector machines \citep{SUNandDAS-2015}, linear and logistic regressions \citep{APRONTIetal-2016}, geographically weighted regression \citep{PULUGURTHAandMATHEW-2021}, locally weighted power curves \citep{CHANGandCHEON-2019}, and clustering \citep{SFYRIDISandAGNOLUCCI-2020}.

\subsection{Probe Data in Traffic Volume Estimation} \label{subsec:probedataintrafficvolumeestimation}

With the advancements in information technology, expectations for traffic volume availability have increased. In the United States, for example, the Highway Safety Improvement Program (HSIP) asks state departments of transportation to prepare traffic volume data even on low-volume roads \citep{FHWA-2016}. As mobile devices compatible with global navigation satellite systems (GNSSs) have spread throughout our daily lives, opportunities to estimate traffic volumes based on passively collected location data have gained industry attention \citep{CACERESetal-2008, HARRISONetal-2020}. Road agencies have started exploring the feasibility of using probe data to estimate traffic volumes \citep{CODJOEetal-2020, FISHetal-2021, KRILEandSLONE-2021, MACFARLANE-2020, ZHANGetal-2019} because probe volumes and traffic volumes tend to be positively correlated. In proprietary products providing AADT estimations, reports have found negative correlations between true traffic volumes and estimation accuracy as measured by percentage errors \citep{BARRIOSandCASBURN-2019, ROLL-2019, SCHEWELetal-2021, TSAPAKISetal-2020, TSAPAKISetal-2021a, TURNERetal-2020, YANGetal-2020}.

Machine learning methods have become popular calibration tools for traffic volume estimation by using probe location data. For instance, \cite{MENGetal-2017} and \cite{ZHANetal-2017} applied spatio-temporal semi-supervised learning and an unsupervised graphical model, respectively, to taxi trajectories in Chinese cities to estimate citywide traffic volumes. With a Maryland probe dataset, \cite{SEKULAetal-2018}, for example, showed that neural networks could significantly improve estimation accuracy. In Kentucky, \cite{ZHANGandCHEN-2020} used annual average daily probes (AADP) and betweenness centrality to estimate AADTs across the state. Using random forest, they found that an AADP of 53 was the lower threshold for having a mean absolute percentage error (MAPE) of less than 20 \% to 25 \%. Machine learning methods, including support vector machines and gradient boosting, provide practical solutions for calibrating high-dimensional data and improving estimation accuracy \citep{SCHEWELetal-2021}.

\subsection{Types of Probe Data} \label{subsec:typefofprobedata}

Figure \ref{fig:figure1} illustrates different types of probe data: point data (Figure \ref{fig:figure1}a) and line data (Figure \ref{fig:figure1}b). Point data refer to data that contain information to identify a point location (e.g., geographic coordinates) on a surface, such as the Earth’s ellipsoid. Location data are usually first recorded and stored as point data. In contrast, line data, also called trajectories, paths, or routes, consist of a series of point data of an entity connected chronologically \citep{MARKOVICetal-2019}. Conventional traffic counts require information on passing objects over a cross-section at a fixed location. With probe data, one can count the number of probes passing through a specific location based on trajectories reconstructed from point data (e.g., GPS Exchange Format (GPX)) when the point data meet all of the following conditions:

\begin{itemize}
    \item Each probe has a pseudonym (e.g., device identifier).
    \item Each point data has a timestamp in the ordinal scale or a higher level of measurement.
    \item The recording interval is small enough to determine a route.
\end{itemize}

% Figure environment removed

In other words, data that meet these conditions have less anonymity because one can track each probe’s locations and time simultaneously \citep{DEMONTJOYEetal-2013}. In fact, all of the aforementioned studies used line data of probes to estimate traffic volumes. However, some point data are unsuitable for the precise reconstruction of line data. Sparsely recorded probe data are an example \citep{SUNetal-2013}. In addition, agencies might not be able to obtain detailed line data in which they can identify a probe’s geographic coordinates and timestamps at once, depending on privacy regulations and data providers’ policies.

To relax such constraints on probe data availability in traffic volume estimation, this paper presents a method for estimating probe traffic volumes using passively collected point location data without route reconstruction. In addition, we describe the exact distribution of the unbiased estimator with which one can assess the estimation precision. On the other hand, we will hardly tap into detailed calibration methods against known traffic volumes, which ultimately influences traffic volume estimation accuracy. In the following sections, we derive analytical relationships between traffic variables and estimated probe counts with example calculations. Numerical and microscopic traffic simulations further demonstrate the conformity of the model. Finally, we discuss the characteristics, limitations, applications, and opportunities of the model.