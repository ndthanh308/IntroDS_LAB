%%%%%%%%%%%%%%%%%%%%%%%%%%%%%%%%%%%%%% IMPLICATION OF THE MODEL %%%%%%%%%%%%%%%%%%%%%%%%%%%%%%%%%%%%%%%%%%

\section{Implication of the Model} \label{subsec:implicationofthemodel}

This paper documented the distribution of $\hat{m}$ and estimated probe counts based on the point probe location data recorded at fixed intervals. The final section discusses the model’s implications regarding theory, applications, and opportunities.
\newpage
\subsection{Model Characteristics} \label{subsec:modelcharacteristics}

Practitioners can use $\hat{m}$ as an unbiased estimator of probe traffic volumes in any timeframes. The more probes are present, the more closely the distribution of $\hat{m}$ can be approximated by a normal distribution.
The estimation precision measured as CV[$\hat{m}$] is inversely proportional to the square root of actual probe volume $m$, roughly proportional to recording interval $t$, and roughly inversely proportional to cordon length $d$ (Equation \ref{eq:hokeg}). In other words, the higher the probe volume, the more precise the volume estimates are likely to be, while the degree of marginal improvement decreases as the traffic volume increases. A lower probe speed also tends to result in better precision when other conditions remain the same. In reality, the speed distribution $g(s)$ can change along with $d$ unless $S$ truly follows uniform motion; therefore, the theoretical optimal cordon length $d$ should be considered a suggestion rather than a perfect means of optimisation. Therefore, it is a reasonable strategy to set the longest possible $d$ that fits the road segment that carries a single traffic probe volume when an analyst does not know the probe data recording interval $t$ or the speed distribution $g(s)$.

However, the relationship between $d$ and CV[$\hat{m}$] is not always monotonic. Depending on the recording interval and speed distribution, there is a local optimum cordon length $d$ that maximises the precision of $\hat{m}$ estimation (Figure \ref{fig:figure8}a). Although the authors are unaware of the exact data processing methods used in proprietary traffic volume estimation software, the estimation precision is likely to improve by setting an optimal cordon length $d$ in these products if they inherently rely on probe point data with speed information.

In developing a traffic volume estimation model, calibration among $\hat{m}$ is required to convert the values into traffic volume estimates. Because probes, in reality, are not likely to be distributed homogeneously among road users, this procedure ultimately determines traffic volume estimation accuracy. In this process, modellers can use the theoretical variance to effectively weight $\hat{m}$ (Figure \ref{fig:figure10}).

Knowledge of how the distribution emerges can improve the traffic volume estimation models, as shown in Figure \ref{fig:figure8}. Our method equips modellers with the capability to incorporate even low traffic volumes into their calibration models, as the distribution of $\hat{m}$ cannot always be approximated by a normal distribution when the actual probe volume is low. The theoretical PDF of the estimated probe traffic volume allows modellers or analysts to perform interval estimation on $m$. Depending on the calibration model, probe traffic volume estimates with confidence intervals (CIs) can also be used to improve the calibration accuracy against known traffic volumes.

Practitioners should be aware of some other elements when applying the proposed method to probe point location data. First, spatial characteristics should be considered when drawing virtual cordons. For example, a modeller must pay attention to grade-separated facilities, tunnels, crosswalks, sidewalks, and cell phone location data from flying objects. Sometimes, probe data need to be coded to avoid capturing location data from unintended road users, as we truncated the high speed in our example calculation.

With real traffic, the variance of $\hat{m}$ can become larger than the theoretical variance because GNSS is not free from systematic and random errors \citep{MARKOVICetal-2019}. Although centimetre-level positioning is available with some GNSS \citep{CHOYetal-2015}, most GNSS argumentations are associated with horizontal errors varying up to 3-15 meters \citep{MERRYandBETTINGER-2019, ZANDBERGENandBARBEAU-2011}. As a result, speed measurement is also associated with some errors \citep{GUIDOetal-2014}. Because speed distribution plays a crucial role in estimating traffic volumes in the proposed method, it is essential to make an effort to reduce speed bias \citep{AHSANIetal-2019} in the data acquisition process.

\subsubsection{Limitations} \label{subsec:limitations}

Although we addressed the theoretical aspect of traffic volume estimation from probe point data, the proposed model is not free from limitations in practical settings. One limitation is that our model assumes i.i.d. uniform motion as the speed among the probes. As actual traffic conditions may not necessarily suit these assumptions, modellers and analysts should recognise this limitation. With careful selection of cordon locations (both spatially and temporally), biases from these assumptions can be weakened. 

In traffic volume estimation, another limitation of the model is that the PDF formulation (Equation \ref{eq:hnmnmi}) of $\hat{m}$ includes the true probe volume $m$ itself. Although this does not prevent the computation of $\hat{m}$ (Equation \ref{eq:draso}) or VMR[$\hat{m}$] (Equation \ref{eq:ghvry}), this recursion is sometimes not ideal, because the probe volume is usually estimated when the probe volume $m$ is unknown. In this context, this study is descriptive and may not be a silver bullet for issues that some readers might have expected to solve. Nevertheless, the theoretical elements of the estimated probe traffic volume still contribute to the lineage of traffic volume estimation research in that we described how the distribution of $\hat{m}$ emerges.

\subsection{Applications} \label{subsec:applications}

The proposed method can contribute to various aspects of traffic volume estimation. First, it allows agencies to use marginal point probe data without pseudonyms or granular timestamps. For example, they can enhance the quality of traffic volume estimation by utilising sparsely recorded probe data, which would have been ignored without our method. Depending on how much marginal probe point data are available compared with the line data already available, probe location data without pseudonyms can be a sleeping lion.

Furthermore, the model predicts “the economy of scale”, encompassing probe data valuation. A higher recording frequency ($\because$ Equation \ref{eq:hokeg}) and homogeneity make the traffic volume estimation more precise and accurate, respectively. As a result, probe location data with a high recording frequency and homogeneity are more valuable for traffic volume estimation. Thus, agencies could perform cost-benefit analyses based on the specific goals they want to achieve.

Another economy of scale arises from the synergistic effect of acquiring traffic counts at fixed locations. Probe traffic volumes can be used to estimate traffic volumes at many locations. This fact does not smear the importance of fixed-location traffic counts, because it is impossible to calibrate the values against traffic volumes without ground truths. A higher density of reliable traffic count data from conventional devices can enhance the proposed method by providing additional calibration points. Therefore, governments investing in continuous traffic monitoring infrastructures can expect an even larger return on investment (ROI) than expected.

As reported by \cite{TURNER-2021}, the evaluation of big data quality and valuation has been of concern among transportation professionals, as machine learning models can quickly become black boxes for data users, including decision-makers. In addition to the data availability enhancement, the distribution of $\hat{m}$ can be used to calculate the valuation of probe point data. From Equation \ref{eq:hokeg}, it, for example, may be reasonable to formulate the value of point probe data somewhat inversely proportional to the data recording interval $t$.

\subsection{Opportunities} \label{subsec:opportunities}

The proposed technique can positively impact society, as transportation systems are woven into daily human activities. On a global scale, traffic volume estimations based on probe point data can positively impact agencies and nations with limited financial and human resources \citep{LORDetal-2003, YANNISetal-2014}. In particular, the method will be useful for low-volume rural roads, where traditional or passive traffic recording tools may not be cost efficient \citep{DAS-2021}. Because remote highways tend to have long uninterrupted segments, drawing long virtual cordons would help transportation professionals estimate probe traffic volumes quite precisely. Such traffic volume information along rural highways can be used to develop safety performance functions (SPFs) more thoroughly and continuously than ever before \citep{TSAPAKISetal-2021b}.

Because traffic volume estimation using probe data is in its infancy, there are many research opportunities in this field. Future research related to traffic volume estimation from probe point data would include the relaxation of the i.i.d. and uniform motion constraints in the distribution of $\hat{m}$, the development of universal indices to describe the homogeneity of probe data, a framework to evaluate the transferability of the data, cost-benefit analyses of probe location data, and real-time crash hotspot identification.

Our model paves the way for unleashing probe point data as a means of social good. In the 1940s, \cite{GREENSHIELDS-1947} analysed traffic using a series of aerial photographs taken at fixed intervals. Decades later, we have opportunities to improve the quality of transportation through “snapshots” of probes recorded at a fixed interval but with unprecedented scalability. Interorganizational collaborations, including cooperation between the public and private sectors, will be crucial in bringing technology to life to tackle various societal challenges.