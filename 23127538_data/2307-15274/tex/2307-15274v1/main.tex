%% Use LuaLateX as a compiler

%% Beginning of file 'sample631.tex'
%%
%% Modified 2022 May  
%%
%% This is a sample manuscript marked up using the
%% AASTeX v6.31 LaTeX 2e macros.
%%
%% AASTeX is now based on Alexey Vikhlinin's emulateapj.cls 
%% (Copyright 2000-2015).  See the classfile for details.

%% AASTeX requires revtex4-1.cls and other external packages such as
%% latexsym, graphicx, amssymb, longtable, and epsf.  Note that as of 
%% Oct 2020, APS now uses revtex4.2e for its journals but remember that 
%% AASTeX v6+ still uses v4.1. All of these external packages should 
%% already be present in the modern TeX distributions but not always.
%% For example, revtex4.1 seems to be missing in the linux version of
%% TexLive 2020. One should be able to get all packages from www.ctan.org.
%% In particular, revtex v4.1 can be found at 
%% https://www.ctan.org/pkg/revtex4-1.

%% The first piece of markup in an AASTeX v6.x document is the \documentclass
%% command. LaTeX will ignore any data that comes before this command. The 
%% documentclass can take an optional argument to modify the output style.
%% The command below calls the preprint style which will produce a tightly 
%% typeset, one-column, single-spaced document.  It is the default and thus
%% does not need to be explicitly stated.
%%
%% using aastex version 6.3
\documentclass[]{aastex631}

%% The default is a single spaced, 10 point font, single spaced article.
%% There are 5 other style options available via an optional argument. They
%% can be invoked like this:
%%
%% \documentclass[arguments]{aastex631}
%% 
%% where the layout options are:
%%
%%  twocolumn   : two text columns, 10 point font, single spaced article.
%%                This is the most compact and represent the final published
%%                derived PDF copy of the accepted manuscript from the publisher
%%  manuscript  : one text column, 12 point font, double spaced article.
%%  preprint    : one text column, 12 point font, single spaced article.  
%%  preprint2   : two text columns, 12 point font, single spaced article.
%%  modern      : a stylish, single text column, 12 point font, article with
%% 		  wider left and right margins. This uses the Daniel
%% 		  Foreman-Mackey and David Hogg design.
%%  RNAAS       : Supresses an abstract. Originally for RNAAS manuscripts 
%%                but now that abstracts are required this is obsolete for
%%                AAS Journals. Authors might need it for other reasons. DO NOT
%%                use \begin{abstract} and \end{abstract} with this style.
%%
%% Note that you can submit to the AAS Journals in any of these 6 styles.
%%
%% There are other optional arguments one can invoke to allow other stylistic
%% actions. The available options are:
%%
%%   astrosymb    : Loads Astrosymb font and define \astrocommands. 
%%   tighten      : Makes baselineskip slightly smaller, only works with 
%%                  the twocolumn substyle.
%%   times        : uses times font instead of the default
%%   linenumbers  : turn on lineno package.
%%   trackchanges : required to see the revision mark up and print its output
%%   longauthor   : Do not use the more compressed footnote style (default) for 
%%                  the author/collaboration/affiliations. Instead print all
%%                  affiliation information after each name. Creates a much 
%%                  longer author list but may be desirable for short 
%%                  author papers.
%% twocolappendix : make 2 column appendix.
%%   anonymous    : Do not show the authors, affiliations and acknowledgments 
%%                  for dual anonymous review.
%%
%% these can be used in any combination, e.g.
%%
%% \documentclass[twocolumn,linenumbers,trackchanges]{aastex631}
%%
%% AASTeX v6.* now includes \hyperref support. While we have built in specific
%% defaults into the classfile you can manually override them with the
%% \hypersetup command. For example,
%%
%% \hypersetup{linkcolor=red,citecolor=green,filecolor=cyan,urlcolor=magenta}
%%
%% will change the color of the internal links to red, the links to the
%% bibliography to green, the file links to cyan, and the external links to
%% magenta. Additional information on \hyperref options can be found here:
%% https://www.tug.org/applications/hyperref/manual.html#x1-40003
%%
%% Note that in v6.3 "bookmarks" has been changed to "true" in hyperref
%% to improve the accessibility of the compiled pdf file.
%%
%% If you want to create your own macros, you can do so
%% using \newcommand. Your macros should appear before
%% the \begin{document} command.
%%

\newcommand{\vdag}{(v)^\dagger}
\newcommand\aastex{AAS\TeX}
\newcommand\latex{La\TeX}
\usepackage{amssymb}
\usepackage{amsthm}
\usepackage{upgreek}
\usepackage{physics}
\usepackage{textcomp, gensymb}
\usepackage{float}
\usepackage{svg}
\usepackage[normalem]{ulem}
\useunder{\uline}{\ul}{}
\usepackage{enumerate}
\newtheorem{innercustomgeneric}{\customgenericname}
\providecommand{\customgenericname}{}
\newcommand{\newcustomtheorem}[2]{%
  \newenvironment{#1}[1]
  {%
   \renewcommand\customgenericname{#2}%
   \renewcommand\theinnercustomgeneric{##1}%
   \innercustomgeneric
  }
  {\endinnercustomgeneric}
}

\newcustomtheorem{customthm}{Theorem}
\newcustomtheorem{customlemma}{Lemma}
\newcustomtheorem{customcorollary}{Corollary}

%% Reintroduced the \received and \accepted commands from AASTeX v5.2
%\received{March 1, 2021}
%\revised{April 1, 2021}
%\accepted{\today}

%% Command to document which AAS Journal the manuscript was submitted to.
%% Adds "Submitted to " the argument.
%\submitjournal{PSJ}

%% For manuscript that include authors in collaborations, AASTeX v6.31
%% builds on the \collaboration command to allow greater freedom to 
%% keep the traditional author+affiliation information but only show
%% subsets. The \collaboration command now must appear AFTER the group
%% of authors in the collaboration and it takes TWO arguments. The last
%% is still the collaboration identifier. The text given in this
%% argument is what will be shown in the manuscript. The first argument
%% is the number of author above the \collaboration command to show with
%% the collaboration text. If there are authors that are not part of any
%% collaboration the \nocollaboration command is used. This command takes
%% one argument which is also the number of authors above to show. A
%% dashed line is shown to indicate no collaboration. This example manuscript
%% shows how these commands work to display specific set of authors 
%% on the front page.
%%
%% For manuscript without any need to use \collaboration the 
%% \AuthorCollaborationLimit command from v6.2 can still be used to 
%% show a subset of authors.
%
%\AuthorCollaborationLimit=2
%
%% will only show Schwarz & Muench on the front page of the manuscript
%% (assuming the \collaboration and \nocollaboration commands are
%% commented out).
%%
%% Note that all of the author will be shown in the published article.
%% This feature is meant to be used prior to acceptance to make the
%% front end of a long author article more manageable. Please do not use
%% this functionality for manuscripts with less than 20 authors. Conversely,
%% please do use this when the number of authors exceeds 40.
%%
%% Use \allauthors at the manuscript end to show the full author list.
%% This command should only be used with \AuthorCollaborationLimit is used.

%% The following command can be used to set the latex table counters.  It
%% is needed in this document because it uses a mix of latex tabular and
%% AASTeX deluxetables.  In general it should not be needed.
%\setcounter{table}{1}

%%%%%%%%%%%%%%%%%%%%%%%%%%%%%%%%%%%%%%%%%%%%%%%%%%%%%%%%%%%%%%%%%%%%%%%%%%%%%%%%
%%
%% The following section outlines numerous optional output that
%% can be displayed in the front matter or as running meta-data.
%%
%% If you wish, you may supply running head information, although
%% this information may be modified by the editorial offices.
%\shorttitle{AASTeX v6.3.1 Sample article}
%\shortauthors{Schwarz et al.}
%%
%% You can add a light gray and diagonal water-mark to the first page 
%% with this command:
%% \watermark{text}
%% where "text", e.g. DRAFT, is the text to appear.  If the text is 
%% long you can control the water-mark size with:
%% \setwatermarkfontsize{dimension}
%% where dimension is any recognized LaTeX dimension, e.g. pt, in, etc.
%%
%%%%%%%%%%%%%%%%%%%%%%%%%%%%%%%%%%%%%%%%%%%%%%%%%%%%%%%%%%%%%%%%%%%%%%%%%%%%%%%%
%\graphicspath{{./}{figures/}}
%% This is the end of the preamble.  Indicate the beginning of the
%% manuscript itself with \begin{document}.

\begin{document}

\title{On the Distribution of Probe Traffic Volume Estimated from Their Footprints}

%% LaTeX will automatically break titles if they run longer than
%% one line. However, you may use \\ to force a line break if
%% you desire. In v6.31 you can include a footnote in the title.

%% A significant change from earlier AASTEX versions is in the structure for 
%% calling author and affiliations. The change was necessary to implement 
%% auto-indexing of affiliations which prior was a manual process that could 
%% easily be tedious in large author manuscripts.
%%
%% The \author command is the same as before except it now takes an optional
%% argument which is the 16 digit ORCID. The syntax is:
%% \author[xxxx-xxxx-xxxx-xxxx]{Author Name}
%%
%% This will hyperlink the author name to the author's ORCID page. Note that
%% during compilation, LaTeX will do some limited checking of the format of
%% the ID to make sure it is valid. If the "orcid-ID.png" image file is 
%% present or in the LaTeX pathway, the OrcID icon will appear next to
%% the authors name.
%%
%% Use \affiliation for affiliation information. The old \affil is now aliased
%% to \affiliation. AASTeX v6.31 will automatically index these in the header.
%% When a duplicate is found its index will be the same as its previous entry.
%%
%% Note that \altaffilmark and \altaffiltext have been removed and thus 
%% can not be used to document secondary affiliations. If they are used latex
%% will issue a specific error message and quit. Please use multiple 
%% \affiliation calls for to document more than one affiliation.
%%
%% The new \altaffiliation can be used to indicate some secondary information
%% such as fellowships. This command produces a non-numeric footnote that is
%% set away from the numeric \affiliation footnotes.  NOTE that if an
%% \altaffiliation command is used it must come BEFORE the \affiliation call,
%% right after the \author command, in order to place the footnotes in
%% the proper location.
%%
%% Use \email to set provide email addresses. Each \email will appear on its
%% own line so you can put multiple email address in one \email call. A new
%% \correspondingauthor command is available in V6.31 to identify the
%% corresponding author of the manuscript. It is the author's responsibility
%% to make sure this name is also in the author list.
%%
%% While authors can be grouped inside the same \author and \affiliation
%% commands it is better to have a single author for each. This allows for
%% one to exploit all the new benefits and should make book-keeping easier.
%%
%% If done correctly the peer review system will be able to
%% automatically put the author and affiliation information from the manuscript
%% and save the corresponding author the trouble of entering it by hand.

\correspondingauthor{Gulshan Noorsumar}

\author[0000-0002-2165-747X]{Kentaro Iio}
\affiliation{% \\ 
Imazucho Otomo \\ Takashima, Shiga 520-1635, Japan}

\author[0000-0002-6718-4508]{Gulshan Noorsumar}
\affiliation{Department of Engineering Sciences, University of Agder \\ 
Jon Lilletuns vei 9 \\ 4879 Grimstad, Norway}
\email{$gulshan.noorsumar@uia.no$}

\author[0000-0002-7434-6886]{Dominique Lord}
\affiliation{Zachry Department of Civil and Environmental Engineering, Texas A\&M University \\
3127 TAMU \\
College Station, Texas 77843-3127, United States}

\author[0000-0003-2404-5409]{Yunlong Zhang}
\affiliation{Zachry Department of Civil and Environmental Engineering, Texas A\&M University \\
3127 TAMU \\
College Station, Texas 77843-3127, United States}

%% Note that the \and command from previous versions of AASTeX is now
%% depreciated in this version as it is no longer necessary. AASTeX 
%% automatically takes care of all commas and "and"s between authors names.

%% AASTeX 6.31 has the new \collaboration and \nocollaboration commands to
%% provide the collaboration status of a group of authors. These commands 
%% can be used either before or after the list of corresponding authors. The
%% argument for \collaboration is the collaboration identifier. Authors are
%% encouraged to surround collaboration identifiers with ()s. The 
%% \nocollaboration command takes no argument and exists to indicate that
%% the nearby authors are not part of surrounding collaborations.

%% Mark off the abstract in the ``abstract'' environment. 

\begin{abstract}
Collecting traffic volume data is a vital but costly piece of transportation engineering and urban planning. In recent years, efforts have been made to estimate traffic volumes using passively collected probe data that contain spatiotemporal information. However, the feasibility and underlying principles of traffic volume estimation based on probe data without pseudonyms have not been examined thoroughly. In this paper, we present the exact distribution of the estimated probe traffic volume passing through a road segment based on probe point data without trajectory reconstruction. The distribution of the estimated probe traffic volume can exhibit multimodality, without necessarily being line-symmetric with respect to the actual probe traffic volume. As more probes are present, the distribution approaches a normal distribution. The conformity of the distribution was demonstrated through numerical and microscopic traffic simulations. Theoretically, with a well-calibrated probe penetration rate, traffic volumes in a road segment can be estimated using probe point data with high precision even at a low probe penetration rate. Furthermore, sometimes there is a local optimum cordon length that maximises estimation precision. The theoretical variance of the estimated probe traffic volume can address heteroscedasticity in the modelling of traffic volume estimates.
\end{abstract}

%% Keywords should appear after the \end{abstract} command. 
%% The AAS Journals now uses Unified Astronomy Thesaurus concepts:
%% https://astrothesaurus.org
%% You will be asked to selected these concepts during the submission process
%% but this old "keyword" functionality is maintained in case authors want
%% to include these concepts in their preprints.

\keywords{Traffic Volume, AADT, Probe Data, Point Data, Telematics, Privacy Protection}

%% We recommend that authors also use the natbib \citep
%% and \citet commands to identify citations.  The citations are
%% tied to the reference list via symbolic KEYs. The KEY corresponds
%% to the KEY in the \bibitem in the reference list below. 

% !TEX root = ../AttackGraphBasedRiskAnalysis.tex
% !TEX spellcheck = en_US
% !TEX encoding = UTF-8 Unicode

\section{Introduction}\label{sec: intro}

Traditional cities are becoming smarter. 
One of the core smart city concepts is smart mobility, which has attracted considerable attention from security researchers due to the emergence of smart vehicles and V2X communication that have given rise to novel cybersecurity threats.

Over the last decade, several trends have contributed to the automotive and railway threat landscape. 
First, sophisticated features in smart vehicles come with a higher volume of lines of code, aggravating testability and auditing and increasing the likelihood and severity of vulnerabilities. 
Second, (wireless) communication interfaces in smart vehicles come with a higher volume of external peripheral devices that can connect to smart vehicles, hence increasing the attackers' access point options, and also with a higher volume of connections, hence increasing the risk of malicious interactions. 
Finally, a higher volume of connections between smart vehicles comes with a higher volume of exchanged data, which in most cases is personal and, therefore, immensely valuable. In other words, more data is generated and needs to be considered and protected.

Graphical security modeling is a widely-used and well-established approach for representing and analyzing threat landscapes that examine vulnerabilities of systems and organizations. 
One of the primary strengths of graphical security models is that they allow for the inclusion of user-friendly visual elements with formal semantics and algorithms, enabling both qualitative and quantitative analyses. 
Over the last couple of decades, security researchers have been progressively focusing on graphical security modeling, which has gradually evolved into a valuable tool for the assessment of risks in real-life systems, such as automotive and railway environments.

Threat landscapes include (1) malicious actions of an attacker, whose goal is to harm or damage one or more assets of a system or organization, and (2) countermeasures for either preventing or mitigating such malicious actions. 
The first \emph{tree-based approach} for graphical security modeling was the \emph{threat logic trees}, which was introduced by Weiss in 1991~\cite{weiss1991}, thereby motivating the development of several subsequent frameworks, such as attack trees, which are still considered one of the most important and favored tools for the assessment of risks to date.

In all tree-based approaches, the modeling process begins with identifying a feared event, which is shown as a root node, and continues with the refinement of the attack steps, resulting in a tree model.
However, tree structures are limited to only one path between a pair of nodes. 
In other words, with tree structures, each refined node can only have one parent node. 
This limitation is addressed by the \emph{directed acyclic graph (DAG) structure}, which enables refined nodes to have multiple parent nodes. 
As a result, DAG structures can provide a higher level of detail, but they can also come with a higher level of complexity, which can nevertheless be dealt with modularization, thereby allowing the model to be subdivided into loosely-coupled, independent, and interchangeable parts that can be studied individually and in parallel. 
Finally, while the one-to-many relationship between nodes in tree structures results in a linear analysis of the threat landscape, the many-to-many relationship between nodes in DAG structures can theoretically result in an exponential analysis.
However, the complexity is kept small in practice due to the acyclic structure, and the threat landscape analysis is eventually possible.

Ensuring the security of systems is not a static process that is over after going through once.
The conditions are constantly changing, on the one hand attackers and their capabilities are evolving, and on the other hand, systems themselves are being extended and evolving.
To effectively perform the necessary continuous security management, it is necessary to know not just the threat landscape but to be able to understand the consequences and impacts if attacks are performed successfully.
Hence, it is necessary to continuously perform a risk analysis to identify the potential exposure.
Nowadays, risk management is primarily done using large tables filled with a lot of information and use cases.
Large tables only offer limited visibility, as it is challenging to maintain a comprehensive overview of risks.
With numerous rows and columns, it becomes difficult to identify trends and patterns or prioritize risks effectively.
Furthermore, managing risk can be a tedious and time-consuming process.
Updating and maintaining tables with evolving risks and mitigation measures can require significant effort, especially when dealing with a complex system or multiple risk factors.
This gets even harder when dealing with large tables that often fail to provide the necessary context and connections between different risks.
Additionally, analyzing and interpreting data from large tables can be daunting. 
It may require specialized tools or skills to extract meaningful insights from the extensive amount of information presented in the table format.
Large tables may further lack the flexibility to accommodate changing risk scenarios or evolving requirements. 
Modifying or updating the table structure to incorporate new risks or factors can be cumbersome and may hinder agility in risk management.
With numerous cells and data entries, there is also an increased risk of errors, inaccuracies, or inconsistencies in the large table. 
These issues can undermine the reliability and integrity of the risk management process.

We propose a graphical solution for the risk management process to mitigate these disadvantages of tables.
A visual representation can enhance the understanding and communication of complex risk information and make it easier to identify patterns, trends, and relationships among risks, facilitating effective decision-making.
Complex risk data is further simplified by presenting it in a clear and concise manner.
Understanding  the relationships, dependencies, and interactions between various risk elements is necessary to understand the overall risk landscape.
Visual representations of the entire risk landscape provide this overview, allowing for the identification of interdependencies, hotspots, or areas of high vulnerability.
Graphical solutions can also aid in developing and evaluating risk mitigation strategies. 
By visually representing the potential consequences and effectiveness of different mitigation measures, decision-makers can make more informed choices and allocate resources more efficiently.
Furthermore, it allows for the exploration of different risk scenarios. 
By manipulating variables or parameters within the visual representation, it becomes possible to assess the potential impact of various risk factors and evaluate the effectiveness of different response strategies.
Additionally, as graphical solutions can be more adaptable to changing requirements and evolving risks, they allow for easier updates and modifications, enabling risk management processes to be more responsive and agile.

Consequently, we believe a graphical solution for the risk assessment process improves the maintenance of risk scenarios and facilitates accessibility to different stakeholders, including non-technical audiences.
However, the existing graphical solutions are momentarily used to describe the threat landscape.
Which, of course, is helpful for the risk management process but not sufficient to represent the entire risk management process.
Therefore, motivating us to define a new graphical method for risk assessment by extending existing graphical methods for depicting the threat landscape.
Besides ways to depict attack vectors, their probability, and countermeasures, our method includes a way to depict the consequences of attack vectors and the impact level, enabling us to calculate a risk value.

The remainder of the paper is structured as follows:
After the introduction,~\cref{sec: related work} discusses the related work.
Our definition of attack graphs is given in~\cref{sec: attack graphs}.
The necessary adjustments to use these attack graphs are presented in~\cref{sec: attack graph risk assessment}, including an example of how the risk assessment is performed in our project.
\cref{sec: applicability of attack graphs to risk management standards} validates our defined method by combing it with the risk assessment processes of ISO/SAE 21434~\cite{21434} and CLC/TS 50701~\cite{50701} respectively.
The scalability and practicality are evaluated in~\cref{sec: evaluation}.
Finally,~\cref{sec: conclusion} concludes this paper.

% \newpage
\section{Theory} \label{sec:theory}

This section describes the problem, provides our findings with proofs, and offers illustrative examples. We adhere to the International System of Units throughout the paper unless stated otherwise.

\subsection{Problem Statement} \label{subsec:problemstatement}

We define a “probe” as a device that records its position as point data in the Earth's spatial reference system (e.g., geographic coordinates). For instance, a smartphone and connected vehicle can be a probe\footnote{By this definition, the number of probes does not necessarily match the number of vehicles (e.g., multiple smartphones on a vehicle).}. Let $m\ \{ m \in \mathbf{Z}^{nonneg}\}$ and $\hat{m}$ denote the number of probes passing through a unit segment during an observation period and its estimator, respectively. We present the distribution of $\hat{m}$ under the following conditions.

Assume that each probe traverses the Earth's surface at a speed of $S$ $\{ S \in \mathbf{R}^{+}\}$ m/s, where $S$ is an independent and identically distributed (i.i.d.) continuous random variable. We denote the realised value of $S$ as $s$ $\{ s \in \mathbf{R}^{+}\}$. Let $g(s)$ $\{ g(s) \in \mathbf{R}^{nonneg} \mid 0 \leq g(s); \displaystyle \int_{0}^{\infty} g(s)\mathrm{d}s = 1 \}$ be the probability density function (PDF) of the probe speed population. The population of $s$ is a hypothetical infinite group of $s$. Therefore, the possibility that multiple probes are carried by one vehicle at the same $s$ is already considered in $g(s)$ as a part of the distribution. All probes share the same data recording interval $t$ $\{ t \in \mathbf{R}^{+}\}$ s. In a uniform motion, each probe records its position as point data (i.e., “\textit{footprints}”) at an interval of $t$ s. The probe speed is recorded during this process. Probe identifiers $i$ or detailed timestamps are not necessarily recorded, but data points have at least nominal information to identify a recorded time range of interest (e.g., a label of “July 2023”). We assume that there are no errors or failures in the positioning or recording.

After the probe point location data are recorded, an analyst draws a $d$-m virtual cordon ($\{ d \in \mathbf{R}^{+}\}$) over the data measured along the road segment of interest. This spatial data cropping procedure makes each probe record its first location in the virtual cordon at a uniformly distributed random timing within any nonnegative seconds less than $t$ after a probe enters the cordon. The analyst may extract data within the time range of interest, as needed. The virtual cordon will contain $n\ \{ n \in \mathbf{Z}^{nonneg}\}$ data points at a speed of $s_a$ where $a\ \{ a \in \mathbf{Z}^{nonneg}\mid a \leq n\}$ is a record identifier. Figure \ref{fig:figure2} shows an example of a virtual cordon containing eight data points. Although the figure distinguishes the two probes, this work does not assume that analysts have information to identify individual probes.

% Figure environment removed

\subsection{Unbiased Estimator of $m$} \label{subsec:lemma1}

\begin{customlemma}{1}\label{lma:lemma1}\normalfont
    If we define $\hat{m}$ as
    
    \begin{equation} \label{eq:draso}
        \hat{m} = \frac{t}{d} \sum_{a=1}^{n}s_{a}, \forall m, d, t, n, s
    \end{equation}
    
    \noindent
    $\hat{m}$ is an unbiased estimator of the true probe traffic volume $m$ (Equation \ref{eq:kgbnho}).
    
    \begin{equation} \label{eq:kgbnho}
        \mathrm{E}[\hat{m}] = m, \forall m
    \end{equation}
    
\end{customlemma}

\begin{quote}
    \begin{proof}
        Because uniform motion is assumed, $s_i = s_a$ within any probe and $s_it$ is the distance along a cordon the $i$th probe traverses in $t$ s. Using $n_i$ as the number of data points within a cordon from the probe, Equation \ref{eq:draso} can be reduced to
        
        \begin{equation} \label{eq:uthnbgn}
            \hat{m} = \displaystyle \frac{t s_i n_i}{d}
        \end{equation}
        
        \noindent
        for the $i$th probe. In Equations \ref{eq:draso} and \ref{eq:uthnbgn}, $n_i$ can be broken down into $n_i = \tilde{n}_i + K_i$ where $\tilde{n}_i$ $\lbrace{\tilde{n} \in \mathbf{Z}^{nonneg}}\rbrace$ is the minimum number of data points that could be recorded in the virtual cordon. It is calculated with the floor function as
          
        \begin{equation} \label{eq:kdtfg2}
            \tilde{n}_i = \left\lfloor\frac{d}{s_i t}\right\rfloor
        \end{equation}
        
        \noindent
        Here, $K_i$ is a Bernoulli random variable representing the number of additional data points per probe $\lbrace{K \in \mathbf \{0, 1\}}\rbrace$ observed in addition to $\tilde{n}_i$ data points. Because uniform motion is assumed and a probe leaves its first record in the cordon at a random time at $t$ s or before once the probe enters the cordon. Naturally, an additional data point is recorded at the probability of the fractional part of $d/(s_it)$. When we define the fractional part as $p_i$ $\lbrace{p \in \mathbf{R}^{nonneg} \mid 0 \leq p < 1}\rbrace$,
        
        \begin{equation} \label{eq:yathe2}
            p_i = \frac{d}{s_it} \mod 1
        \end{equation}
    
        Because $K_i$ follows the Bernoulli distribution $Ber(p_i)$, its expected value $\mathrm{E}[K_i]$ is $p_i$. From Equations \ref{eq:uthnbgn}, \ref{eq:kdtfg2}, and \ref{eq:yathe2}, $\mathrm{E}[\hat{m}]$, the expected value of $\hat{m}$, is
        
        \begin{equation} \label{eq:yatlhh2}
            \mathrm{E}[\hat{m}] = \frac{s_it}{d} \left[ \left\lfloor\frac{d}{s_it}\right\rfloor + \left( \frac{d}{s_it} \mod 1 \right) \right] = 1
        \end{equation}
        
        \noindent
         when $m = 1$. Accordingly, $\mathrm{E}[\hat{m}] = m$ for any $m$. Therefore, $\hat{m}$ is an unbiased estimator of $m$.
    \end{proof}
\end{quote}

\subsubsection{Example 1} \label{subsub:example1}

We assume $d = 100$ and $t = 1$ in Figure \ref{fig:figure2}. The expected number of records within the segment from probe B ($s_i = 30$) is $100/(30 \cdot 1) \approx 3.333$; therefore, at least three records are observed (i.e., $\tilde{n} = 3$). Since it is impossible to observe 3.333 records, one more record is observed at a probability of approximately 0.333 (i.e., $p_i \approx 0.333$). In Figure \ref{fig:figure2}, $m = 2$, $\mathrm{E}[\hat{m}] = 2$ and $\hat{m} = 1.9$. If the cordon had contained the data points only from probe A, $m = 1$, $\mathrm{E}[\hat{m}] = 1$ and $\hat{m} = 1$. If the cordon had included the data points only from probe B, $m = 1$, $\mathrm{E}[\hat{m}] = 1$ and $\hat{m} = 0.9$.

\subsection{Variance of $\hat{m}$} \label{subsec:lemma2}

\begin{customlemma}{2}\label{lma:lemma2}\normalfont
    When we denote the variance of $\hat{m}$ as $\mathrm{Var}[\hat{m}]$:
    
    \begin{equation} \label{eq:efrus}
         \mathrm{Var}[\hat{m}]=\frac{mt^2}{d^2}\int_0^{\infty} b(s, d, t)g(s)\mathrm{d}s
    \end{equation}
    
    \noindent
    where
    
    \begin{equation} \label{eq:bahos}
        b(s, d, t) = s^2 p(1-p) = s^2 \left( \frac{d}{st} \mod 1 \right) \left[1- \left( \frac{d}{st} \mod 1 \right)\right]
    \end{equation}
    
\end{customlemma}

\begin{quote}
    \begin{proof}
         The variance of $\hat{m}$ originates from the discreteness of the number of recorded data points, namely, the Bernoulli random variable $K$. From Equation \ref{eq:uthnbgn} and the multiplication rule of probability, $\mathrm{Var}[\hat{m} \mid S = s_i]$ is proportional to the variance of the Bernoulli distribution $p(1-p)$ multiplied by the scaling factor $st/d$ raised to a power of 2. Because $S \sim g(s)$, integrating $s^2 t^2 p(1-p)g(s)/d^2$ over $s$ gives the variance of $\hat{m}$ per probe. Because $S$ is assumed to be i.i.d.,  $\mathrm{Var}[\hat{m}] \propto m$ from the additivity of variances.
    \end{proof}
\end{quote}

\subsubsection{Example 2} \label{subsub:example2}

Hereafter, we use a finite mixture of normal distributions by \cite{PARKetal-2010} as $g(s)$ unless stated otherwise. The speed distribution had been fitted to data collected on Interstate Highway 35 (I-35) in Texas. It is comprised of four normal distributions $N(\upmu, \upsigma^2)$ defined by $\upmu = (27.042, 24.000, 9.394, 4.294)$, $\upsigma = (1.831, 4.797, 3.167, 1.686)$, $w = (0.647, 0.223, 0.055, 0.074)$, and $\displaystyle \sum w_j = 1$ where $\upmu \ \{ \upmu \in \mathbf{R}\}$ is a tuple (i.e., a finite ordered list) of mean speed in m/s, $\upsigma \ \{ \upsigma \in \mathbf{R}^{nonneg}\}$ is a tuple of standard deviation in m/s before truncation, and $w \ \{w \in \mathbf{R}^{nonneg}\, |\, w \leq 1 \}$ defines the proportions of the normal distributions within the mixture. The distribution was truncated at $s$ = 0 and $s$ = 40. The resulting $g(s)$ is a mixture of four truncated normal distributions, defined by the following equations (Figure \ref{fig:figure3}a):

\begin{equation} \label{eq:bruxl}
    g(s \mid \upmu, \upsigma, 0, 40) =
    \begin{cases}
    \displaystyle \sum\limits_{j=1}^4
    w_j \uppsi(s \mid \upmu_j, \upsigma_j, 0, 40)
    , & 0 < s \leq 40 \\
    0, & \text{otherwise}
    \end{cases}
\end{equation}

\noindent
where $\upalpha < \upbeta$, $0 < \upsigma$, and

\begin{equation} \label{eq:trunc}
    \uppsi(x \mid \upmu, \upsigma, \upalpha, \upbeta) =
    \frac{\upphi \displaystyle \left( \frac{x-\upmu}{\upsigma} \right)}
    {\upsigma \displaystyle \left[ \Phi \displaystyle \left( \frac{\upbeta-\upmu}{\upsigma} \right) - \Phi \displaystyle \left(\frac{\upalpha-\upmu}{\upsigma}\right) \right] } \\
\end{equation}

\begin{equation} \label{eq:iojmh}
    \upphi(x) = \displaystyle \frac{e^{\displaystyle \left( \frac{-x^2}{2} \right) }}{\sqrt{2\uppi}}
\end{equation}

\begin{equation} \label{eq:jkniy}
    \Phi(x) = \frac{1}{2} \displaystyle \left[1 + \operatorname*{erf}\displaystyle \left(\frac{x}{\sqrt{2}}\right) \right]\\
\end{equation}

Assuming $d = 300$ and $t = 4$, Figure \ref{fig:figure3}b displays $4^2/300^2 \cdot b(s, 300, 4)$, the variance in the estimated probe traffic volume as a function of $s$ (Equation \ref{eq:bahos}). If $S$ were uniformly distributed between 0 and 40 (i.e., $S \sim U(0, 40]$), the area under the function in Figure \ref{fig:figure3}b would have been proportional to the variance of the estimated probe traffic volume (i.e., $\mathrm{Var}[\hat{m} \mid S = s_i]$). Here,  we want to weigh $4^2/300^2 \cdot b(s, 300, 4)$ by $g(s)$ because $S \sim g(s)$. The operation gives Figure \ref{fig:figure3}c, where the area under the function, 0.019, is the theoretical variance of $\hat{m}$ from a probe (Equation \ref{eq:efrus}).

% Multiple figures
% Figure environment removed

\subsection{Shape of $\hat{m}$} \label{subsec:theorem1}

\begin{customthm}{1}\label{theorem:theorem1}\normalfont
    Let $u$ be a nonnegative integer $\lbrace{u \in \mathbf{Z}^{nonneg}}\rbrace$ that operationally substitutes $\tilde{n}$. With the previously defined variables and a function, the PDF of $\hat{m}$ is given as $f(\hat{m}; m)$:
    
    \begin{equation} \label{eq:hnmnmi}
        f(\hat{m}; m) = f'^{*m}(\hat{m})
    \end{equation}
    
    \noindent
    where $f'^{*m}(\hat{m})$ denotes $m$-fold self-convolution of $f'(\hat{m})$. The function $f'(\hat{m})$ is defined as
    
    \begin{equation} \label{eq:rojoc}
        f'(\hat{m})=\displaystyle \sum_{u=0}^{\infty}\sum_{k=0}^{1} h(\hat{m}; t, d, u, k) 
    \end{equation}
    
    \noindent
    where
    
    \begin{equation} \label{eq:husan}
        h(\hat{m}; t, d, u, k)  =
        \begin{cases}
        \displaystyle g\left(\frac{d\hat{m}}{t(u+k)}\right)
        \frac{p^{k}(1 - p)^{1-k}d}{t (u+k)}
        , & (u = 0 \land k \neq 0) \lor \left( u \neq 0 \land \displaystyle \frac{u+k}{u+1} < \hat{m} \leq \displaystyle \frac{u+k}{u} \right)\\
        0, & \text{otherwise}
        \end{cases}
    \end{equation}
    
\end{customthm}

\begin{quote}
    \begin{proof}
        From Equations \ref{eq:kdtfg2} and \ref{eq:yathe2}, $s$ uniquely determines $\tilde{n}$ and $p$ once $d$ and $t$ are determined. In addition, any single $s$ has a mutually exclusive set of $k$ as the outcome of a Bernoulli trial. In Equation \ref{eq:uthnbgn}, $\hat{m}$ is a linear function of $s$ with slope $t (\tilde{n}+k)/d$. Because the probe speed $S$ is i.i.d., the summation of all relative frequencies for possible occurrences of $\tilde{n}$ and $k$ by $\hat{m}$ gives the PDF of $\hat{m}$; therefore, the PDF of $\hat{m}$ contains the joint probability function $g(s)p^{k}(1 - p)^{1-k}$. In Equations \ref{eq:rojoc} and \ref{eq:husan}, $u$ substitutes for $\tilde{n}$. Let $x$ be a nonnegative real number $\lbrace{x \in \mathbf{R}^{nonneg}}\rbrace$ and $\updelta$ be an infinitesimal interval. The probability that $\hat{m}$ takes a value in the interval $(x, x + \updelta ]$ is calculated by integrating the PDF of $\hat{m}$ over the interval. From Equation \ref{eq:uthnbgn}, $m = 0$ when $u + k = 0$; otherwise, the interval of $s$ corresponding to $(x, x + \updelta]$ is $(s, s+ \updelta']$ = $\displaystyle \left(dx/\left[t(u+k)\right], dx/\left[t(u+k)\right] + \updelta d/\left[t(u+k)\right] \right]$, where $dx/\left[t(u+k)\right]$ is $s$ as a function of $\hat{m}$  and $d/\left[t(u+k)\right]$ is the reciprocal of the slope of $\hat{m}$ as a function of $s$ (e.g., Figure \ref{fig:figure4}). However, the interval of $s$ must be constant regardless of $\hat{m}$ in the PDF of $\hat{m}$ because $\hat{m}$ results from $S$, but not vice versa. Therefore, the joint probability of $u$ and $k$, in fact, must be multiplied by $d/\left[t(u+k)\right]$, which is the reciprocal of the slope of $s$ as a function of $\hat{m}$. When $S$ is i.i.d., $\hat{m}$ is also i.i.d. (Equation \ref{eq:uthnbgn}). Hence, the PDF of $\hat{m}$ emerges as an $m$-fold self-convolution of the PDF where $m = 1$ (Equation \ref{eq:hnmnmi}).  
    \end{proof}
\end{quote}

\begin{customcorollary}{1}\label{crlry:corollary1}\normalfont
    As $m$ approaches infinity, the shape of $f(\hat{m}; m)$ converges to that of a normal distribution:
    
    \begin{equation} \label{eq:oihnmybn}
        \lim_{{m \to \infty}} f(\hat{m}; m) = N \left(m, \frac{mt^2}{d^2} \int_0^{\infty} b(s,d,t)g(s)\mathrm{d}s\right)
    \end{equation}
\end{customcorollary}

\begin{quote}
    \begin{proof}
        Because $\hat{m}$ is i.i.d., Equation \ref{eq:oihnmybn} is derived from the classical central limit theorem on lemmata \ref{lma:lemma1} and \ref{lma:lemma2}.
    \end{proof}
\end{quote}

\subsubsection{Example 3} \label{subsub:example3}

Assuming $d = 300$ and $t = 4$, Figure \ref{fig:figure4} plots Equation \ref{eq:uthnbgn} (i.e., when $m = 1$). The combinations of $\tilde{n}$ and $k$ fall into an infinite periodic pattern along the $s$-axis because $\tilde{n}$ increases towards infinity as $s$ approaches 0. Because $S \sim g(s)$, we want to take the relative frequency of speed and each $k$ by multiplying the probability mass function (PMF) of $Ber(p)$ by $g(s)$. After this operation, we obtain the overall frequency of the combination of $\tilde{n}$ and $k$ by $s$ (Figure \ref{fig:figure5}).

% One figure
% Figure environment removed

% One figure
% Figure environment removed

From Figure \ref{fig:figure4}, it is apparent that the density in an interval of $\hat{m}$ can emerge from more than one combination of $\tilde{n}$ and $k$, which have different slopes for $\hat{m}$ with respect to $s$. Therefore, an infinitesimal interval of $\hat{m}$ can have different cardinalities of the frequencies projected from the $s$-axis; thus, we must consider the cardinality of $\hat{m}$. For example, the length of an infinitesimal interval of $\hat{m}$ corresponding to any interval between $s = 25$ and $s = 37.5$ in Figure \ref{fig:figure4} is 50 \% longer when $k = 1$ than when $k = 0$. Because we are interested in the PDF of $\hat{m}$, we must normalise the value using the cardinality of $\hat{m}$. This operation can be performed by dividing the relative frequency given the combination of $\tilde{n}$ and $k$ by each slope $t(\tilde{n}+k)/d$ before summation. Equation \ref{eq:rojoc} results in the PDF in Figure \ref{fig:figure6} in this example.

% One figure
% Figure environment removed

\subsection{Optimum Cordon Length} \label{subsec:theorem2}

Equation \ref{eq:efrus} tells that $d$ determines $\mathrm{Var}[\hat{m}]$ when $t$ and $g(s)$ are already fixed.  Considering that $d$ is often the only parameter that an analyst can control, the art of estimation error minimisation lies in setting a good cordon length $d$. That said, how long should $d$ be in what conditions? Modelling the relationships between $\hat{m}$ and the other variables gives us a hint on choosing a good cordon length $d$. % If an analyst pursues precision, 

\begin{customcorollary}{2}\label{crlry:corollary2}\normalfont
    Let $\max(d)$ denote the maximum feasible $d$ within a given segment. When $\max(d)$ exists, there can be a cordon length $d$ shorter than $\max(d)$ that minimises the precision of estimating $m$. Such $d$ can be sought by $\displaystyle \operatorname*{argmin}_{0 < d \leq \max(d)} obj(d)$ where $obj(d)$ is an objective function such as the variance-to-mean ratio (VMR)
    
    \begin{equation} \label{eq:ghvry}
        \mathrm{VMR}[\hat{m}] = 
        \frac{\mathrm{Var}[\hat{m}]}{\mathrm{E}[\hat{m}]} = \frac{t^2}{d^2}\int_0^{\infty} b(s,d,t)g(s)\mathrm{d}s
    \end{equation}
    
    \noindent
    or the coefficient of variation (CV)
    
    \begin{equation} \label{eq:hokeg}
        \mathrm{CV}[\hat{m}] = 
        \frac{ \sqrt{\mathrm{Var}[\hat{m}]} }{\mathrm{E}[\hat{m}]} = \frac{t}{d} \sqrt{\frac{1}{m} \int_0^{\infty} b(s,d,t)g(s)\mathrm{d}s}
    \end{equation}
\end{customcorollary}

\begin{quote}
    \begin{proof}
    Assume that Corollary 2 is false. In other words, assume that $\mathrm{CV}[\hat{m}]$ always monotonically decreases as $d$ increases. When $m = 1$, $t = 4$ and $S \sim g(s)$ defined by Equations \ref{eq:bruxl}-\ref{eq:jkniy}, $\mathrm{CV}[\hat{m}] = 0.310$ when $d = 150$ whereas $\mathrm{CV}[\hat{m}] = 0.230$ when $d = 110$. Because there is a counterexample to the assumed proposition that Corollary 2 is false, Corollary 2 is true.
    \end{proof}
\end{quote}

\subsubsection{Example 4} \label{subsub:example}

This example provides graphical descriptions of the proof of Corollary 2. Figure \ref{fig:figure7} displays an example\footnote{Figure \ref{fig:figure7} plots the function given this specific combination of $t$ and $g(s)$ and will look different with a different set of input values.}: $b(s, d, 4)g(s)$ and $4^2/d^2 \cdot b(s, d, 4)g(s)$ as functions of $s$ and $d$ when $S \sim g(s)$. In Figure \ref{fig:figure7}a, $b(s, d, t)g(s)$ has a periodic pattern along the $d$-axis. Figure \ref{fig:figure7}b is an extension of Figure \ref{fig:figure3}c to the $d$-axis, where $b(s, d, t)g(s)$ is scaled by $t^2/d^2$ to plot Equation \ref{eq:ghvry} when $m = 1$). Because $\mathrm{VMR}[\hat{m}]$ is inversely proportional to $d^2$, a larger $d$ tends to result in a better precision in $\hat{m}$. This is intuitive considering $\mathrm{Var}[\hat{m}]$ arises from the discreteness of the observed number of records. The ratio of the additional number of records $K$, a Bernoulli random variable, to the total number of records $n$ decreases as the cordon captures more data points, owing to a larger $d$.

However, VMR$[\hat{m}]$ or CV$[\hat{m}]$ does not always exhibit a monotonic decrease over $d$. As seen in Figure \ref{fig:figure8}a, the non-monotonicity of $\mathrm{CV}[\hat{m}]$ as a function of $d$ indicates the potential existence of $d$ that locally minimises the VMR or CV when the maximum $d$ exists. When some road geometry dictates the maximum $d$ to be 150 m (e.g., a 150-m road segment immediately bounded by intersections, beyond which traffic volumes can be different) in the condition of Figure \ref{fig:figure8}a, it would be better to set 110-m $d$ ($CV = 23.048\ \%$) than trying to set 150-m $d$ ($CV = 30.999\ \%$). Figure \ref{fig:figure8}b plots CV$[\hat{m}]$ as a function of $t$ when $d = 300$. CV$[\hat{m}]$ tends to increase as $t$ increases, but this relationship is not always monotonic.

% Multiple figures
% Figure environment removed

% Multiple figures
% Figure environment removed


%%%%%%%%%%%%%%%%%%%%%%%%%%%%%%%%%%%%%% SIMULATION %%%%%%%%%%%%%%%%%%%%%%%%%%%%%%%%%%%%%%%%%%

\section{Simulations} \label{subsec:simulation}

To supplement discussions, we illustrate the proposed model through numerical and microscopic traffic simulations.

\subsection{Particle Simulations} \label{subsubsec:particlesimulation}

We compared numerically simulated distributions of $\hat{m}$ with the theoretical distributions of $\hat{m}$.

\subsubsection{Method}

In Julia 1.8.5, the number of probe footprints was modelled as a series of particles with independent uniform linear motion along a road segment. In this experiment, the emergence of binomial distributions (Equation \ref{eq:yathe2}) was assumed trivial. The Distributions.jl package was used to generate statistical distributions under the following two scenarios: scenario 1 ($d = 300$ and $t = 4$) and scenario 2 ($d = 40$ and $t = 1$). In each scenario, $m \in \{1, 2, 4, 8\}$ and $S \sim g(s)$ as shown in Figure \ref{fig:figure3}a. We performed one million simulations using Equation \ref{eq:draso} for each combination of scenarios and $m$.

\subsubsection{Results} \label{subsubsec:results1}

Table \ref{tab:table1} exhibits the descriptive statistics of simulations and theory, while Figure \ref{fig:figure9} shows the histograms of simulated $\hat{m}$ and theoretical PDFs of $\hat{m}$ calculated by Equation \ref{eq:hnmnmi}. The simulation results showed a good match in descriptive statistics between simulated values and theoretical values.

As seen in Figure \ref{fig:figure9}, $\hat{m}$ distributes around $m$, but the PDFs are not necessarily line-symmetric with respect to $\hat{m} = m$.
The PDFs approached normal distributions as $m$ increased.

%% The AAS Journal's LaTeX/AASTeX table creator: https://authortools.aas.org/LATEX/make-latex.html
\begin{deluxetable}{cccccc}[H]
\caption{Descriptive Statistics of $\hat{m}$ in Simulations and Theory}
\label{tab:table1}
\tablenum{1}
\tablehead{\colhead{Scenario} & \colhead{$m$} & \colhead{Item} & \colhead{E[$\hat{m}$]} & \colhead{Var[$\hat{m}$]} & \colhead{CV[$\hat{m}$]}}
%% All data must appear between the \startdata and \enddata commands
\startdata
1 & 1 & Simulated & 1.000 & 0.019 & 0.137 \\
 &  & Theoretical & 1 & 0.019 & 0.137 \\
 & 2 & Simulated & 2.000 & 0.037 & 0.097 \\
 &  & Theoretical & 2 & 0.037 & 0.097 \\
 & 3 & Simulated & 4.000 & 0.075 & 0.068 \\
 &  & Theoretical & 4 & 0.075 & 0.068 \\
 & 4 & Simulated & 8.000 & 0.150 & 0.048 \\
 &  & Theoretical & 8 & 0.149 & 0.048 \\
2 & 1 & Simulated & 1.000 & 0.088 & 0.297 \\
 &  & Theoretical & 1 & 0.088 & 0.297 \\
 & 2 & Simulated & 2.000 & 0.177 & 0.210 \\
 &  & Theoretical & 2 & 0.177 & 0.210 \\
 & 3 & Simulated & 4.000 & 0.353 & 0.148 \\
 &  & Theoretical & 4 & 0.353 & 0.149 \\
 & 4 & Simulated & 7.999 & 0.706 & 0.105 \\
 &  & Theoretical & 8 & 0.706 & 0.105 \\
\enddata
\end{deluxetable}

% Multiple figures
% Figure environment removed

\subsection{Regressions on Microscopic Traffic Simulations} \label{subsubsec:microscopic trafficsimulation}

To develop traffic volume estimation models, $\hat{m}$ is compared to known traffic volumes to calibrate the ratio of traffic volumes to the estimated probe traffic volumes. Here, we briefly illustrate how the theoretical variance of $\hat{m}$ can improve traffic volume estimation accuracy through a fitter model.

\subsubsection{Method}

%% The AAS Journal's LaTeX/AASTeX table creator: https://authortools.aas.org/LATEX/make-latex.html
\begin{deluxetable}{cccccc}[H]
\tablecaption{Traffic Volume Data Used in Regressions}
\tablenum{2}
\label{tab:table2}
\tablehead{\colhead{Site number} & \colhead{Modelled speed distribution} & \colhead{ADT} & \colhead{$m$} & \colhead{$d$} & \colhead{VMR[$\hat{m}$]}\\ 
\colhead{} & \colhead{(m/s)} & \colhead{} & \colhead{} & \colhead{(m)} & \colhead{}  } 
%% All data must appear between the \startdata and \enddata commands
\startdata
4945 & $N(26.82,5.00)$ & 763 & 107 & 14 & 0.916 \\
4953 & $N(26.82,5.00)$ & 27 & 4 & 53 & 0.028 \\
4961 & $N(26.82,5.00)$ & 19 & 3 & 64 & 0.035 \\
4977 & $N(26.82,5.00)$ & 34 & 5 & 56 & 0.025 \\
4985 & $N(26.82,5.00)$ & 618 & 87 & 52 & 0.030 \\
5001 & $N(26.82,5.00)$ & 751 & 105 & 62 & 0.033 \\
5025 & $N(26.82,5.00)$ & 706 & 99 & 25 & 0.092 \\
5057 & $N(26.82,5.00)$ & 83 & 12 & 27 & 0.059 \\
5065 & $N(26.82,5.00)$ & 275 & 39 & 46 & 0.066 \\
5081 & $N(26.82,5.00)$ & 52 & 7 & 56 & 0.025 \\
5089 & $N(26.82,5.00)$ & 38 & 5 & 35 & 0.116 \\
5097 & $N(13.41,5.00)$ & 183 & 26 & 41 & 0.019 \\
5113 & $N(26.82,5.00)$ & 110 & 15 & 10 & 1.682 \\
5121 & $N(13.41,5.00)$ & 539 & 75 & 41 & 0.019 \\
5129 & $N(13.41,5.00)$ & 353 & 49 & 62 & 0.008 \\
5145 & $N(26.82,5.00)$ & 448 & 63 & 20 & 0.341 \\
5185 & $N(26.82,5.00)$ & 668 & 94 & 53 & 0.028 \\
5201 & $N(13.41,5.00)$ & 96 & 13 & 21 & 0.086 \\
5217 & $N(26.82,5.00)$ & 277 & 39 & 39 & 0.111 \\
5225 & $N(26.82,5.00)$ & 232 & 32 & 66 & 0.035 \\
9193 & $N(26.82,5.00)$ & 48 & 7 & 57 & 0.026 \\
9201 & $N(26.82,5.00)$ & 8 & 1 & 64 & 0.035 \\
9209 & $N(26.82,5.00)$ & 63 & 9 & 17 & 0.578 \\
9233 & $N(26.82,5.00)$ & 289 & 40 & 52 & 0.030 \\
9249 & $N(26.82,5.00)$ & 216 & 30 & 7 & 2.831 \\
9257 & $N(26.82,5.00)$ & 820 & 115 & 57 & 0.026 \\
9281 & $N(13.41,5.00)$ & 226 & 32 & 46 & 0.014 \\
9289 & $N(26.82,5.00)$ & 446 & 62 & 25 & 0.092 \\
9297 & $N(26.82,5.00)$ & 52 & 7 & 70 & 0.030 \\
9305 & $N(26.82,5.00)$ & 135 & 19 & 49 & 0.045 \\
9313 & $N(26.82,5.00)$ & 221 & 31 & 50 & 0.039 \\
9321 & $N(26.82,5.00)$ & 121 & 17 & 41 & 0.101 \\
9329 & $N(26.82,5.00)$ & 41 & 6 & 28 & 0.059 \\
9353 & $N(26.82,5.00)$ & 46 & 6 & 31 & 0.090 \\
\enddata
\tablecomments{Speed distributions had been truncated at $s = 0$ and $s = 60$. The modelled speed distributions did not necessarily match the actual speed distribution at each site in Hudspeth County.}
\end{deluxetable}

We used PTV Vissim 11.00-02 to microscopically simulate vehicular traffic on one-lane straight road links 300 m in length. There were 730 links, each carrying 500 vehicles over 24 h. Vehicle speed distributions were $N(26.82, 5.00)$ for 365 links and $N(13.41, 5.00)$ for the remaining 365 links \citep{OPPENLANDER-1963}. The recording interval was $t = 1$.

Thirty-four average annual traffic volumes (AADTs) of fewer than 1,000 recorded in Hudspeth County, Texas, in 2021 \citep{TXDOT-2022} were used as the ground truth for average daily traffic (ADT) (Table \ref{tab:table2}). The probe penetration rate was assumed to be 2 \% everywhere, and the observation period was seven days; therefore, $m \approx (\mathrm{ADT}) \times 0.02 \times 7$ in this experiment. A random integer between 5 and 70 was set as $d$ at each location. Table \ref{tab:table2} summarises modelled speed distribution, ADT, $m$, $d$, and theoretical VMR[$\hat{m}$] (Equation \ref{eq:ghvry}) by site.

Based on $m$, virtual probes were randomly assigned to vehicles that had already been simulated. For instance, at site 5121, one of the 365 links simulating the speed of $N(13.41, 5.00)$ was randomly chosen to represent this location. Within the link, a virtual cordon ($d = 41$) was set at a random location, and trajectories from 75 randomly chosen vehicles were captured within the virtual cordon to compute $\hat{m}$ using Equation \ref{eq:draso}. This procedure was performed at all the sites in each trial.

To estimate ADTs based on $\hat{m}$, the experiment assumed that two out of the 34 ADTs were known (e.g., when traffic counters existed) and calculated the ratios of ADT to $\hat{m}$ at these sites. In each trial, regression models were created using Julia's GLM.jl package for all possible combinations (i.e., ${}_{34} \mathrm{ C }_2 = 561$) of “known” ADT counts. For every combination of known ADTs, an ordinary least square (OLS) regression and a linear regression with weighted least square (WLS) using the reciprocal of VMR[$\hat{m}$] (Equation \ref{eq:ghvry}) were performed while assuming the intercept was 0.

We performed 2,023 trials. In each trial, MAPE was calculated among the 32 estimated ADTs (i.e., excluding the pair used to develop each regression model) against the ground truth ADTs with every possible combination of “known” ADTs. The mean MAPE of the 561 combinations was considered as the average MAPE of the trial. If our model is correct, linear regressions on observed $\hat{m}$ inversely weighted by VMR[$\hat{m}$] should estimate the traffic volumes better than OLSs do.

\subsubsection{Results} \label{subsubsec:results2} 

The OLS and WLS resulted in the the mean average coefficients of determination were $R^2 = 0.970$ for OLS and $R^2 = 0.986$ for WLS. The mean average MAPE of OLS was 0.097, whereas that of WLS was 0.086. Although both methods estimated ADTs well, the results should be interpreted relatively. Figure \ref{fig:figure10} shows sorted differences in average MAPE between OLS and WLS, where positive values indicate improvements in WLS (i.e., average MAPE of WLS subtracted from that of OLS). The WLS yielded a better average MAPE than OLS in 2,001 out of the 2,023 (98.91 \%) trials. As we had predicted, the results exemplified the efficacy of Equation \ref{eq:ghvry} for developing a traffic volume estimation model.

% Figure environment removed


%%%%%%%%%%%%%%%%%%%%%%%%%%%%%%%%%%%%%% IMPLICATION OF THE MODEL %%%%%%%%%%%%%%%%%%%%%%%%%%%%%%%%%%%%%%%%%%

\section{Implication of the Model} \label{subsec:implicationofthemodel}

This paper documented the distribution of $\hat{m}$ and estimated probe counts based on the point probe location data recorded at fixed intervals. The final section discusses the model’s implications regarding theory, applications, and opportunities.
\newpage
\subsection{Model Characteristics} \label{subsec:modelcharacteristics}

Practitioners can use $\hat{m}$ as an unbiased estimator of probe traffic volumes in any timeframes. The more probes are present, the more closely the distribution of $\hat{m}$ can be approximated by a normal distribution.
The estimation precision measured as CV[$\hat{m}$] is inversely proportional to the square root of actual probe volume $m$, roughly proportional to recording interval $t$, and roughly inversely proportional to cordon length $d$ (Equation \ref{eq:hokeg}). In other words, the higher the probe volume, the more precise the volume estimates are likely to be, while the degree of marginal improvement decreases as the traffic volume increases. A lower probe speed also tends to result in better precision when other conditions remain the same. In reality, the speed distribution $g(s)$ can change along with $d$ unless $S$ truly follows uniform motion; therefore, the theoretical optimal cordon length $d$ should be considered a suggestion rather than a perfect means of optimisation. Therefore, it is a reasonable strategy to set the longest possible $d$ that fits the road segment that carries a single traffic probe volume when an analyst does not know the probe data recording interval $t$ or the speed distribution $g(s)$.

However, the relationship between $d$ and CV[$\hat{m}$] is not always monotonic. Depending on the recording interval and speed distribution, there is a local optimum cordon length $d$ that maximises the precision of $\hat{m}$ estimation (Figure \ref{fig:figure8}a). Although the authors are unaware of the exact data processing methods used in proprietary traffic volume estimation software, the estimation precision is likely to improve by setting an optimal cordon length $d$ in these products if they inherently rely on probe point data with speed information.

In developing a traffic volume estimation model, calibration among $\hat{m}$ is required to convert the values into traffic volume estimates. Because probes, in reality, are not likely to be distributed homogeneously among road users, this procedure ultimately determines traffic volume estimation accuracy. In this process, modellers can use the theoretical variance to effectively weight $\hat{m}$ (Figure \ref{fig:figure10}).

Knowledge of how the distribution emerges can improve the traffic volume estimation models, as shown in Figure \ref{fig:figure8}. Our method equips modellers with the capability to incorporate even low traffic volumes into their calibration models, as the distribution of $\hat{m}$ cannot always be approximated by a normal distribution when the actual probe volume is low. The theoretical PDF of the estimated probe traffic volume allows modellers or analysts to perform interval estimation on $m$. Depending on the calibration model, probe traffic volume estimates with confidence intervals (CIs) can also be used to improve the calibration accuracy against known traffic volumes.

Practitioners should be aware of some other elements when applying the proposed method to probe point location data. First, spatial characteristics should be considered when drawing virtual cordons. For example, a modeller must pay attention to grade-separated facilities, tunnels, crosswalks, sidewalks, and cell phone location data from flying objects. Sometimes, probe data need to be coded to avoid capturing location data from unintended road users, as we truncated the high speed in our example calculation.

With real traffic, the variance of $\hat{m}$ can become larger than the theoretical variance because GNSS is not free from systematic and random errors \citep{MARKOVICetal-2019}. Although centimetre-level positioning is available with some GNSS \citep{CHOYetal-2015}, most GNSS argumentations are associated with horizontal errors varying up to 3-15 meters \citep{MERRYandBETTINGER-2019, ZANDBERGENandBARBEAU-2011}. As a result, speed measurement is also associated with some errors \citep{GUIDOetal-2014}. Because speed distribution plays a crucial role in estimating traffic volumes in the proposed method, it is essential to make an effort to reduce speed bias \citep{AHSANIetal-2019} in the data acquisition process.

\subsubsection{Limitations} \label{subsec:limitations}

Although we addressed the theoretical aspect of traffic volume estimation from probe point data, the proposed model is not free from limitations in practical settings. One limitation is that our model assumes i.i.d. uniform motion as the speed among the probes. As actual traffic conditions may not necessarily suit these assumptions, modellers and analysts should recognise this limitation. With careful selection of cordon locations (both spatially and temporally), biases from these assumptions can be weakened. 

In traffic volume estimation, another limitation of the model is that the PDF formulation (Equation \ref{eq:hnmnmi}) of $\hat{m}$ includes the true probe volume $m$ itself. Although this does not prevent the computation of $\hat{m}$ (Equation \ref{eq:draso}) or VMR[$\hat{m}$] (Equation \ref{eq:ghvry}), this recursion is sometimes not ideal, because the probe volume is usually estimated when the probe volume $m$ is unknown. In this context, this study is descriptive and may not be a silver bullet for issues that some readers might have expected to solve. Nevertheless, the theoretical elements of the estimated probe traffic volume still contribute to the lineage of traffic volume estimation research in that we described how the distribution of $\hat{m}$ emerges.

\subsection{Applications} \label{subsec:applications}

The proposed method can contribute to various aspects of traffic volume estimation. First, it allows agencies to use marginal point probe data without pseudonyms or granular timestamps. For example, they can enhance the quality of traffic volume estimation by utilising sparsely recorded probe data, which would have been ignored without our method. Depending on how much marginal probe point data are available compared with the line data already available, probe location data without pseudonyms can be a sleeping lion.

Furthermore, the model predicts “the economy of scale”, encompassing probe data valuation. A higher recording frequency ($\because$ Equation \ref{eq:hokeg}) and homogeneity make the traffic volume estimation more precise and accurate, respectively. As a result, probe location data with a high recording frequency and homogeneity are more valuable for traffic volume estimation. Thus, agencies could perform cost-benefit analyses based on the specific goals they want to achieve.

Another economy of scale arises from the synergistic effect of acquiring traffic counts at fixed locations. Probe traffic volumes can be used to estimate traffic volumes at many locations. This fact does not smear the importance of fixed-location traffic counts, because it is impossible to calibrate the values against traffic volumes without ground truths. A higher density of reliable traffic count data from conventional devices can enhance the proposed method by providing additional calibration points. Therefore, governments investing in continuous traffic monitoring infrastructures can expect an even larger return on investment (ROI) than expected.

As reported by \cite{TURNER-2021}, the evaluation of big data quality and valuation has been of concern among transportation professionals, as machine learning models can quickly become black boxes for data users, including decision-makers. In addition to the data availability enhancement, the distribution of $\hat{m}$ can be used to calculate the valuation of probe point data. From Equation \ref{eq:hokeg}, it, for example, may be reasonable to formulate the value of point probe data somewhat inversely proportional to the data recording interval $t$.

\subsection{Opportunities} \label{subsec:opportunities}

The proposed technique can positively impact society, as transportation systems are woven into daily human activities. On a global scale, traffic volume estimations based on probe point data can positively impact agencies and nations with limited financial and human resources \citep{LORDetal-2003, YANNISetal-2014}. In particular, the method will be useful for low-volume rural roads, where traditional or passive traffic recording tools may not be cost efficient \citep{DAS-2021}. Because remote highways tend to have long uninterrupted segments, drawing long virtual cordons would help transportation professionals estimate probe traffic volumes quite precisely. Such traffic volume information along rural highways can be used to develop safety performance functions (SPFs) more thoroughly and continuously than ever before \citep{TSAPAKISetal-2021b}.

Because traffic volume estimation using probe data is in its infancy, there are many research opportunities in this field. Future research related to traffic volume estimation from probe point data would include the relaxation of the i.i.d. and uniform motion constraints in the distribution of $\hat{m}$, the development of universal indices to describe the homogeneity of probe data, a framework to evaluate the transferability of the data, cost-benefit analyses of probe location data, and real-time crash hotspot identification.

Our model paves the way for unleashing probe point data as a means of social good. In the 1940s, \cite{GREENSHIELDS-1947} analysed traffic using a series of aerial photographs taken at fixed intervals. Decades later, we have opportunities to improve the quality of transportation through “snapshots” of probes recorded at a fixed interval but with unprecedented scalability. Interorganizational collaborations, including cooperation between the public and private sectors, will be crucial in bringing technology to life to tackle various societal challenges.

%%%%%%%%%%%%%%%%%%%%%%%%%%%%%%%%%%%%%% ACKNOWLEDGEMENTS %%%%%%%%%%%%%%%%%%%%%%%%%%%%%%%%%%%%%%%%%%
\section*{Acknowledgements}
We would like to express our sincere gratitude to Traf-IQ, Inc. for providing the first author with access to microscopic traffic simulation software. The first author would like to express gratitude to Dr. Daniel Romero at the University of Agder for his valuable advice in the field of statistics.

%%%%%%%%%%%%%%%%%%%%%%%%%%%%%%%%%%%%%% FUNDING SOURCE DECLARATION %%%%%%%%%%%%%%%%%%%%%%%%%%%%%%%%%%%%%%%%%%
\section*{Funding Source Declaration}
This research was funded in part by the A.P. and Florence Wiley Faculty Fellow provided by the College of Engineering at Texas A\&M University.

%%%%%%%%%%%%%%%%%%%%%%%%%%%%%%%%%%%%%% REFERENCES %%%%%%%%%%%%%%%%%%%%%%%%%%%%%%%%%%%%%%%%%%
\bibliography{sample631}{}
\bibliographystyle{chicago}

\end{document}

% End of file.