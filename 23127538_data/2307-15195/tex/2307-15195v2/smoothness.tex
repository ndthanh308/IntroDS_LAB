\section{$C^1$-smoothness of irrational Arnold tongues}
\label{sec-measure}
In this section, we prove that all Arnold tongues that correspond to irrational rotation numbers are $C^1$-smooth at critical circle maps. This result does not require an inequality on eigenvalues from Theorem \ref{th-smoothness} and applies to all irrational rotation  numbers.

The proof is inspired by, and partly based on, the results of L. Slammert, see \cite{Slammert}.
Recall that $\depsr$ is the Banach manifold of $\bbR/\bbZ$-preserving analytic maps defined in a strip of wdth $\eps$.
%%The tangent spaces to  $\mathcal D_\eps$ at all its points are identified with the space of analytic vector fields in the strip of width $\eps$.
The Banach submanifold $\dcrepsr$ of $\depsr$ consists of cubic critical circle maps with $f'(0)=f''(0)=0$, and its tangent space is formed by analytic vector fields in the strip of width $\eps$ with $v'(0)=v''(0)=0$. 

\begin{definition}
 The (-1)-measure of a smooth circle map $f$ is any measure that satisfies
 $$\int \phi(x)d\mu = \int f'(f^{-1}(x))\phi(f^{-1}(x)) d\mu$$ for any continuous test function $\phi$.
\end{definition}
Note, that for circle diffemorphisms this is equivalent to the definition of a (-1)-measure used by Slammert in \cite{Slammert}:
$$
\int \xi (f(x)) \frac 1{f'(x)} d\mu = \int \xi d\mu,
$$
if we use test functions of the form  $\xi(x) = \phi(f^{-1}(x))f'(f^{-1}(x))$.
Slammert attributes the following result to Douady and Yoccoz (see \cite{dMPugh}):
\begin{theorem}
\label{th-1meas}
 A circle diffeomorphism with an irrational rotation number has a unique (-1)-measure.
\end{theorem}
This measure might be singular for Liouville rotation numbers. If  $f$ is conjugate to a rotation via a circle diffeomorphism $h$ (for instance, if the rotation number of $f$ is Herman), then it is easy to check that its (-1)-measure $\mu_f$ is absolutely continuous with density $(h'(x))^2$.

We can synthesize the following from \cite[Theorem 2.8]{Slammert}:
\begin{theorem}
  Let $g$ be a circle diffeomorphism with an irrational rotation number $\alpha$, and let $\mu$ be the (-1)-measure of $g$. Then the 
    condition $\int v|_{g^{-1}(x)} d\mu =0$ defines the tangent space to $\{\rot f=\alpha\}$ at $g$.
  \end{theorem}

Note, that
if $\alpha$ is a Herman number (see \cite{Yoccoz2002} for the definition of the class of Herman numbers $\cH$),
we can prove this directly without referring to \cite{Slammert}.
Recall that all circle diffeomorphisms with Herman rotation numbers are analytically conjugate to  rotations \cite{Yoccoz2002}; as shown in \cite{GY}, the conjugacy depends analytically on the diffeomorphism in the Banach submanifold $\{\rot f=\alpha\}$. Let $h$ be the conjugacy, $g(x) = h^{-1}(h(x)+\alpha)$. Consider the probability measure $\mu$ with density proportional to  $(h'(x))^2$.  Then $\mu$ satisfies
\begin{equation}
\label{eq-1}
 \int \phi(x) d\mu = \int (g)'(g^{-1}(x)) \phi(g^{-1}(x)) d\mu.
\end{equation}
 The tangent space to $\{\rot f=\alpha\}$ at $g$ consists of vector fields of the form $$v = \frac{d}{d\xi} (id+\eps w) g(id + \eps w)^{-1} = w|_{g(x)} - g'(x) w$$ where $w$ is analytic. For any such vector field, we have
$$\int v|_{g^{-1}(x)} d\mu= \int w - g'(g^{-1}(x)) w(g^{-1}(x)) d\mu = \int w d\mu - \int w d\mu =0,$$
the second identity is due to \eqref{eq-1}. Thus the codimension-1 condition $$\int v|_{g^{-1}(x)} d\mu =0$$ defines the tangent space to $\{\rot f=\alpha\}$ at $g$.


\smallskip

We will prove the following two theorems:
\begin{theorem}
\label{th-measure}
 A cubic critical circle map $f$ with an irrational  rotation number has a unique (-1)-measure $\mu_f$. The tangent space to the  local analytic manifold $\{\rot g = \alpha, g\in \dcrepsr\}$, at the critical circle map $f$ coincides with $\{v\in T \dcrepsr\mid \int v|_{f^{-1}(x)} d\mu_f=0\}$.

 If $g_n\to g$ are critical circle maps with irrational rotation numbers, then the corresponding (-1)-measures weakly converge.
\end{theorem}

\begin{theorem}
\label{th-C1}
 For any sequence $f_n\to f$ where $f_n\in \depsr$ are circle diffeomorphisms and $f$ is a cubic critical circle map with $f'(0)=f''(0)=0$, if $\rot f_n=\rot f = \alpha\in \bbR\setminus \bbQ$, then (-1)-measures $\mu_n$ of $f_n$ converge to the (-1)-measure $\mu_f$ of $f$. Consequently, the tangent spaces to the local analytic manifolds $\{\rot g = \alpha\}$ at $f_n$ converge to  the space given by  $ \int v|_{f^{-1}(x)} d\mu_f =0$ which intersects $T\dcrepsr$ on the tangent space to the local analytic manifold $\{\rot g = \alpha, g\in \dcrepsr\}$, at $f$.
\end{theorem}

\noindent
The convergence of the tangent spaces has an immediate corollary:
\begin{corollary}
  \label{cor:C1}

Consider a $C^\omega$-smooth family $f_{\mu}$, $\mu\in \bbR^n$, defined on a semi-neigh\-bor\-hood $\mu_1\in (-\eps, 0]$, $\mu_k \in (-\eps,\eps)$, such that $f_{\mu}$ is an analytic circle diffeomorphism for $\mu_1<0$ and a cubic critical circle map for $\mu_1=0$. Assume that $\frac {\partial f}{\partial \mu_2}>0$ on the circle whenever $\mu_1=0$. Then for each irrational rotation number $\alpha$, the Arnold $\alpha$-tongue  is given by a function $\mu_2=\mu_2(\mu_1, \mu_3, \dots, \mu_n)$ that is at least $C^1$-smooth at $0$.
  \end{corollary}

%
% For each irrational $\mu$ and for each triple $(F, H_a, G)$, $a<0$, there exists a unique $b$ so that $\mathcal p (F+b, H_a, G)$ has rotation number $\mu$. This defines surfaces $\rot (F, H_a, G) = \mu$ --- continuous graphs of functions defined on the  codimension-1 subspace of triples with  $F(0)=0$. Prove they are $C^1$ smooth surfaces with border. It is sufficient to prove this for circle diffeomorphisms instead of triples.

% Indeed, complexify everything (take any $a\in (\bbC,0)$ and remove assumption of $\bbR$-preservation for $F$). For each rotation number $\mu = \lim p_n/q_n$, consider the sequence of complex-analytic surfaces --- graphs of $(F, H_a, G) \to b_n$ given by $(\hat{ \mathcal P} (F+b_n, H_a, G))^{q_n} (0)=p_n$, defined on codim-1 subspace of triples with $F(0)=0$. $a\notin \bbR$ NOT CLEAR For each complex one-parameter family $T_\delta$ of triples with $F(0)=0$, we  get a sequence of complex-analytic functions $b_n(\delta)$ with values in $\bbC$.

% Clearly, $p_n/q_n$-surfaces defined by $f^{q_n}(0)=p_n$ converge in $C$ to the surface $\rot (F, H_a, G) = \alpha$: this follows from continuity of rotation number. Convergence is uniform in $C^\omega$ away from critical maps. Estimate derivatives near critical maps.

% Let $f_\delta$ be a family of circle diffeomorphisms, $f_0$ is critical.
% Let $b_n(\delta)$ be a number such that $(f_\delta+b_n)^q_n(0)=p_n$, and let $b(\delta)$ be such that $\rot(f_\delta + b) = \alpha$. Let $f_0$ be critical.
%
% Let $x_k = x_k(\delta)$ be the orbit of zero. Since $(f+b_n)^{q_n}(0)=p_n$ for all $\delta$, we get
% $$(b_n)'_\delta (1 + f'_{x_{q_{n}-1}} + (f^2)'|_{x_{q_{n}-2}} +\dots )  + f_\delta|_{q_{n}-1}  + f_\delta|_{x_{q_{n}-2}} f'_{x_{q_{n}-1}} + f_\delta|_{x_{q_{n}-2}}  (f^2)'|_{x_{q_{n}-2}} + \dots =0;$$
%
% $$(b_n)'_\delta  = -\frac  {f_\delta|_{q_{n}-1}  + f_\delta|_{x_{q_{n}-2}} f'_{x_{q_{n}-1}} + f_\delta|_{x_{q_{n}-3}}  (f^2)'|_{x_{q_{n}-2}} + ...}{(1 + f'_{x_{q_{n}-1}} + (f^2)'|_{x_{q_{n}-2}} +\dots )}.$$
%
% This is a weighted sum of $f'_\delta$ and thus bounded by $\max$ and $\min$ of $f'_\delta$. This implies that $p_n/q_n$ curves have uniformly bounded derivatives.
%
% Note that $(b_n)'_\delta$ is the integral of $f_\delta$ over the atomic probability measure  $\mu_n(f_\delta)$ supported on the orbit of zero. Below (see next subsection) we prove that any weakly convergent subsequence of such measures converges to an atomless measure

\begin{proof}[Simultaneous proof of Theorem \ref{th-measure} and Theorem \ref{th-C1}]
$\-$ \\
   \textbf{Step 1: Partial limits of tangent spaces are given by (-1)-measures of critical circle maps.}

   In assumptions of Theorem \ref{th-C1}, choose a weakly convergent subsequence $\mu_{n_k}$ from $\mu_n$. Let $\mu$ be the limit measure.
% For any orbit $x_k$ of $f$, by considering measures $\mu$ and $\mu_\delta$ of small neighborhoods of $x_k$ and by using \eqref{eq-1}, one can conclude that $\mu_\delta$-measure of a sufficiently small neigborhood of $x_0$ is small for all $\delta$, and thus $\mu$ is atomless.

Due to the weak convergence of measures, \eqref{eq-1} implies that
\begin{equation}
\label{eq-1-crit}
 \int \phi(x) d\mu = \int f'(f^{-1}(x)) \phi(f^{-1}(x)) d\mu
\end{equation}  for any continuous test function $\phi$. Indeed, the left-hand sides of \eqref{eq-1} converge to the left-hand side of \eqref{eq-1-crit}. In the right-hand side, the difference between the integrals of $(f_{n_k})'(f_{n_k}^{-1}(x))\phi(f_{n_k}^{-1}(x)) $ with respect to  $\mu_{n_k}$ and $\mu$ is small due to the weak convergence, and for any fixed  continuous test function, the distance between
$(f_{n_k})'(f_{n_k}^{-1}(x))\phi(f_{n_k}^{-1}(x))$ and $f'(f^{-1}(x)) \phi(f^{-1}(z)) $ is small for large ${n_k}$.


We conclude that as $k\to \infty$, the tangent spaces to $\{\rot g = \alpha\}$ at $f_{n_k}$ have a limit, namely a codimension-1 space of vector fields with  $\int v|_{f^{-1}(x)} d\mu =0$, where $\mu$ is a (-1)-measure for $f$.

\noindent
\textbf{Step 2: relation of (-1)-measures for critical circle maps to tangent spaces of $\{\rot f=\alpha\}$. }

We now prove that this limit space intersects the space of critical circle maps on the tangent space of the condition $\{\rot f=\alpha\}$.
By real {\it a priori} bounds,  all critical circle maps with $\{\rot g=\alpha \}$ are quasiconformally conjugate to $f$.
We are now going to use a stronger version of this statement, derived from the fact that the set $\{\rot g=\alpha\}\cap \dcrepsr$ lies inside a complex-analytic Banach manifold $W\subset\dcreps$, which is a stable manifold of the renormalization operator constructed in \cite{Ya4}. As shown in \cite{Ya4}, the maps in $W$ are quasiconformally conjugate in a neighborhood of $\RR/\ZZ$, with the conjugacy varying analytically on $W$. 
%
%Indeed, consider the family $f_\eps$ of critical circle maps with $\rot f=\eps\in \bbR\setminus \bbZ$, and consider the maps $z\mapsto P(z,\eps)$ that take $f^n(0)$ to $f^n_\eps(0) $. This is a holomorphic motion, and Mane-Sad-Sullivan lemma implies that it extends to the holomorphic motion of a neighborhood of $\bbR/\bbZ$ that is quasiconformal on $z$, with the Beltrami differential $\mu_\eps$ holomorphic on $\eps$ as an element of $L^\infty$. \fixme{TODO Krushkal states this, Quasiconformal Reflections across Polygonal Lines --- check}Note that $\mu_\eps$ is invariant under $f$. The derivative $w$ of this motion with respect to $\eps$ at $\eps=0$ is thus well-defined due to Ahlfors-Bers theorem \cite{}, is Holder continuous and satisfies $w_{\bar z}=\mu'|_{\eps=0}$;  we have $ f'_\eps = w|_{f(x)} - f'(x) w$.
The tangent space to $\{\rot g=\alpha\}\cap \dcrepsr$, which is the real slice of $W$,  consists of the vector fields of the form $v =  w|_{f(x)} - f'(x) w$ where $w$ is
a real-symmetric quasiconformal vector field varying analytically with $g$.
Hence,
%%%a solution of $w_{\bar z}=\mu'$. For any such vector field, we have
$$\int v|_{f^{-1}(x)} d\mu =  \int w - f'(f^{-1}(x)) w(f^{-1}(x)) d\mu = \int w d\mu - \int w d\mu =0.$$
Thus, the tangent space to $\{\rot g = \alpha\}\cap \dcrepsr$ lies in the linear subspace given by the condition $\int v|_{f^{-1}(x)} d\mu = 0$, and, by the dimension count, coincides with it.
%%thus $\mu$ defines the tangent space to $\{\rot g = \alpha\}$ in the space of vector fields with $v'(0)=v''(0)=0$.

\noindent
\textbf{Step 3: uniqueness of (-1)-measures for critical circle maps.}

Let us prove the first statement of Theorem \ref{th-measure}:  a critical circle map  $f$ has a unique (-1)-measure. This  also implies that the limit $\mu$ does not depend on the sequence $f_{n_k}$ and thus completes the proof of Theorem \ref{th-C1}.

%For a critical circle map $f$,  one (-1)-measure can be constructed as above using any sequence of diffeomorphisms $f_n\to f$.
Suppose that we have two  different (-1)-measures $\mu$ and $\tilde \mu$. Combine them with such coefficients $c_1,\;c_2$ that  $$\int v|_{f^{-1}(x)} d(c_1 \mu + c_2 \tilde \mu) =0$$ for all vector fields in $T_f\dcrepsr$. For the codimension $1$ tangent subspace  to $\{\rot g = \alpha\}$ this is automatic. For the remaining direction, this can be achieved by an appropriate choice of the constants.
%For the stable distribution, this follows from the properties of $\mu, \tilde \mu$, and for the one-dimensional unstable direction, we should choose appropriate $c_1, c_2$.
The following lemma applies to the charge $(c_1 \mu + c_2 \tilde \mu)$ and thus implies that $\mu = \tilde \mu$.

\begin{lemma}
 Any charge $\nu$ with $\int v(f^{-1}(z)) d\nu=0$ for any vector field $v\in T_f\dcrepsr$ is zero.
\end{lemma}
\begin{proof}
 For any vector field $v$, choose large $n$ and let $$\tilde v(z) = v(z) + a_1 \sin n z + a_2 \cos n z$$ where $a_1, a_2$ are chosen so that $\tilde v'(0)=\tilde v''(0)=0$.
 Then $\tilde v \in T_f\dcrepsr$, and $ a_1, a_2\to 0$ as $n\to \infty$. We have $$\left|\int v(f^{-1}(z)) d\nu\right| =$$ $$=\left|\int \tilde v(f^{-1}(z)) d\nu - a_1 \int \sin n(f^{-1}(z)) d\nu - a_2 \int \cos n(f^{-1}(z)) d\nu\right|\le$$ $$ \le c|a_1| + c|a_2| \to 0 \text{ where }c=\int 1 d\nu <\infty.$$ This implies $\int v(f^{-1}(z)) d\nu =0$ for any $v$, thus $\nu=0$.
\end{proof}

This completes the proof of  Theorem \ref{th-C1}.

\noindent
\textbf{Step 4: convergence of (-1)-measures for critical circle maps}

It remains to prove the first statement of Theorem \ref{th-measure}: if $g_n\to g$ are critical circle maps, the corresponding measures weakly converge. Indeed,  any weak limit of a sequence of (-1)-measures for $f_n$ is a (-1)-measure for $f$, and since such measure is unique, the statement follows.
\end{proof}



We conclude by proving the following lemma which will be useful to us in what follows, and is interesting in its own right.
\begin{lemma}
\label{lem-noatoms}
The (-1)-measure for a critical circle map $f$ with irrational rotation number cannot have atoms.
%%at images and preimages of the critical point.
\end{lemma}
\begin{proof}
Fix $p$ and apply the definition of the (-1)-measure  for continuous test functions that approximate $\chi([p-\frac 1n, p+\frac 1n])$, then let $n\to \infty$. We get that $\mu_f(\{p\}) = \mu_f(\{f(p)\})\cdot f'(p)$. Iterating this relation, we get $\mu_f(\{p\}) = \mu_f(\{f^k(p)\})\cdot (f^k)'(p)$ for any $k>0$, $p\in \bbR/\bbZ$.

If $p$ is a preimage of a critical point, $f^k(p)=0$, this implies $\mu_f(\{p\}) = 0$.

Otherwise we get that
\begin{equation}
\label{eq-meas-series}
1\ge \mu_f ( \{p, f(p), f^2(p), \dots\}) = \mu_f(\{p\}) \cdot \left( 1+\sum_{k=1}^{\infty} \frac{1}{f^{(k)}(p)}\right).
                     \end{equation}
Let us prove that this series diverges.
%%%if $p$ is an image of a critical point.
Indeed, set $$A_n(p)=[p,f^{q_n}(p)],\text{ and } B_n(p)=[f^{-q_n}(p),f^{2q_n}(p)].$$
By real {\it a priori} bounds \ref{th-realbounds}, $f^{q_n}(B_n(p))$ contains a $C=C(f)$-scaled neighborhood of $f^{q_n}(A_n(p))$, and $f^{q_n}(A_n(p))$ is $C(f)$-commenurable with $A_n(p)$ (the value of $C$ becomes universal for large enough $n$). Furthermore, since $q_n$ is a first return iterate, the orbit $B_n(p)\mapsto\cdots\mapsto f^{q_n}(B_n(p))$ passes through each critical point of $f$ at most three times. Hence, by one-dimensional Koebe Distortion Theorem, the iterate $f^{q_n}|_{A_n(p)}$ decomposes into a universally bounded number of power maps and a universally bounded number of maps with $K(f)$-bounded distortion (again, $K$ becomes universal for large $n$). Hence, there exists $c=c(f)$ (also, universal for large $n$) such that $(f^{q_n}(p))'<c$. 


%Indeed, if $f^{k}(0)=p$, then $p\in I_k$. Suppose that $q_n$ is sufficiently large so that $q_n< q_{n+1}-k$. Then  $f^{q_n}$ has bounded distortion on $f^k(I_n)$ due to Theorem \ref{th-realbounds} b, and takes it to the adjacent interval $f^{k+q_n}(I_n)$ of the partition $\mathcal P_n$, thus $(f^{q_n}(p))'< c$ for a universal $c$ due to Theorem \ref{th-realbounds} a.

Since this holds for any $n$, the series diverges, thus $\mu_f(\{p\})=0$.
\end{proof}


% The following lemma describes how the measures $\mu_f$ change under renormalization.
% \begin{lemma}
%  We have $\int d\mathcal R|_{f}v (\mathcal Rf^{-1}(x)) d\mu_{\mathcal R f}=0$ if and only if $\int v (f^{-1}(x)) d\mu_{f}=0$.
% \end{lemma}
% \begin{proof}
%  Lift a measure $\mu_{\mathcal R f}$ to a real line and transfer it to the union $I\cup f^{q_{n}}(I)$ where $I=[c, f^{q_n}(c)]$ using $\Psi$:
%  $ \mu (A) := \int_{\Psi(A)} \Psi'(f^{-1}(x)) d\mu_{\mathcal Rf}$.
%  This new measure satisfies
%  $ \int \phi(x) d\mu = \int (f^{q_{n+1}})'\cdot ((f^{q_{n+1}})^{-1}(x)) \phi((f^{q_{n+1}})^{-1}(x)) d\mu$ since the lift of $\mathcal Rf$ equals $\Psi f^{q_{n+1}}\Psi$.
%
%  Now, extend this measure to the circle using $\mu(A) : =  \int_{f(A)} f'(f^{-1}(x)) d\mu$. This new measure is well-defined due to \eqref{} and is an (-1)-measure due to the definition. Due to uniqueness of the (-1)-measure, it coincides with $\mu_f$.
%
%  We conclude that $\int_{\Psi(A)} \Psi'(f^{-1}(x)) d\mu_{\mathcal Rf} \int 1 d\mu_f$.
%
%  Due to the computation above,
%  $\int d\mathcal R|_{f}v (\mathcal Rf^{-1}(x)) d\mu_{\mathcal R f}=\int \Psi' $  TODO
% \end{proof}



%


