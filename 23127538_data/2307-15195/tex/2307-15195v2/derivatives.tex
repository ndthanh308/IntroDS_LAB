\section{Uniform estimates on stable manifolds far from critical circle maps}
\label{sec-derivatives}

Let $\hat {\mathcal  V}_\alpha\subset  D$ be a germ of the manifold $\{\rot p(\hat f) = \alpha\}$. In this section we give some geometric estimates on $\hat\cV_\alpha$ outside of a neighborhood of critical circle maps.
%The previous section implies an estimate on the first derivatives of functions that define $\hat {\mathcal  V}$.
%Here we estimate the derivatives of these functions far from the space $D_0$ corresponding to critical circle maps.

%Consider the Banach manifold  $L_{0}\subset \mathcal P^{\bbC}$ of triples $(F, H_a, G)$ that satisfy $F(0)=0$. Fix an irrational number $\alpha$ and  consider the map $\hat q_\alpha \colon L_0\to \bbR$ given by $(F, H_a, G)\mapsto b$ where $b\in \bbR$ is chosen so that $\rot (p(F+b, H_a, G)) = \alpha$.

 %Note that the projection of $K_{a_1, a_0}$ to $L_0$ is $L_0\cap K_{a_1, a_0}$.

% Take a triple $\hat f$ with $a<0$ and  $\rot \hat f = \alpha_0$.
%Let $V+W$ be some splitting of $T\mathcal P$ with $\dim W=1$ such that $W$ is transversal to $d|_{\hat f}\mathcal V_{\alpha_0}$. 
%For $\alpha\approx \alpha_0$, let  $\hat {\mathcal  V}_{\alpha}$ be given by $(x, \hat Q_\alpha(x))$ in the coordinates $V, W$, where $Q_{\alpha}\colon V\to W$.

%\begin{lemma}
%\label{lem-deriv-diff}
 %For  any  Herman number $\alpha\in \mathcal H$, the map $\hat Q_\alpha$ is analytic.

% If $\alpha\in \mathcal B_C$ and the orbit of $\alpha$ under the Gauss map only accumulates to Herman numbers,  for any  $g\in K_{a_1, a_0}$,  for any $n$, derivatives  $\hat Q_\alpha^{(n)}$ are bounded.
 
%\end{lemma}
%\begin{proof}
% Denote $q_\alpha \colon f\mapsto b: \rot (f+b)=\alpha$. Then we have  $\hat q_\alpha(F, H_a, G) = q_{\alpha} (p (F, H_a, G))$. Since  the projection $p$ is analytic and has uniformly bounded derivatives, it suffices to show that the map $q_\alpha \colon f\mapsto b: \rot (f+b)=\alpha$ is analytic and has uniformly bounded derivatives on $p(K_{[a_1, a_0]}\cap L_0)$.


%Note that for small $a_1$, all  $f\in p(K_{[a_1, a_0]})$ are analytic circle diffeomorphisms that are univalent in the strip of width $\mu:=(0.5\sqrt{a_0})$. Indeed, elements of each triple $\hat f\in K_{[a_1, a_0]}$ are well-
%defined in a $2 \eps$-neighborhood of $\bbR$, and the critical points of $p\hat  f$ are outside the strip $G^{-1} (\{\Im z \in [-\sqrt{a_0}, \sqrt{a_0}]\})$.

%The first statement follows from the Risler's theorem; Risler's theorem implies that for a Herman number $\alpha$, the condition $\{\rot f = \alpha\}$ is analytic for a circle diffeomoprhism $f$, and since the tangent plane to $\{\rot f = \alpha\}$ is transversal to $d/dx$, the map $q_\alpha$ is analytic.

%Suppose that the second statement fails. Fix $n, C$, $a_1$, and $a_0$, and find a sequence of circle diffeomorphisms $f_m$ with $\rot f_m=\alpha_m$ at which  $q_{\alpha_m}^{(n)}\to \infty$ as $m\to \infty$. They are univalent in the strip of width $0.5\sqrt{a_0}$. Thus the same holds for the limit $f=\lim_{m\to \infty} f_m$ with the rotation number $\alpha_0=\lim \alpha_n$, $\alpha_0\in \mathcal B_C\cap \mathcal H$ since $\mathcal B_C$ is compact and the orbit of $\alpha$ only accumulates to $\mathcal H$.

%We have found a single map $f\in K_{a_1, a_0}$ such that in its neighborhood,  $q_{\alpha_m}^{(n)}\to \infty$.
%We will obtain a contradiction in a sequence of lemmas below.

%Complexify the Banach manifold $\mathcal P_\eps$ of triples to form $\mathcal P^{\bbC}_\eps$ by removing conditions that $F, G$ preserve $\bbR$.
%Consider the Banach manifold  $L_{0}^\bbC \subset \mathcal P^{\bbC}$ of triples $(F, H_a, G)$ that satisfy $F(0)=0$. Fix an irrational number $\alpha$ and  consider the map $\hat q_\alpha \colon L_0^\bbC \to \bbC$ given by $(F, H_a, G)\mapsto b$ where $b\in \bbC$ is chosen so that $p(F+b, H_a, G)$ is analytically conjugate to $R_\alpha$ in some strip that contains the real axis. The graph of $\hat q_\alpha$ in $D = L_0\times \bbC$ is  $\hat Q_\alpha\}$.


% Consider a biholomorphism $\xi$ of the circle that uniformizes the circle diffomorphism $f+b(f)$; it exists due to the Yoccoz's theorem.  The map $\xi$ induces an analytic map of a neighborhood of $f+b(f)$ to a neighborhood of the rotation  $R_\alpha$, namely a conjugacy with $\xi$, that preserves the rotation number. The image of the graph $Q_\alpha$ of $q_\alpha$ is a surface $\{\rot f = \alpha\}$ in $D_{diff}$, which is analytic due to Risler's Theorem \ref{Ris}. This implies that  $Q_\alpha$ is an analytic surface; since the derivative of $q_\alpha$ is bounded, $q_\alpha$ is an analytic function.

We start with the following result that was essentially proved  in \cite{GY},  but was only formulated for the Arnold's family, see Corollary 1.9.



Denote  $\alpha_{-1} =1$, $\alpha_k = G^k(\alpha)$, where $G(x)$ is the Gauss map, $G(x)  = \{1/x\}$. Recall that the Yoccoz-Brjuno function is
$$
\Phi(\alpha) = \sum_{j=0}^{\infty} \alpha_{-1} \alpha_0  \dots \alpha_{j-1} \log \frac{1}{\alpha_j} 
$$ 
and the set of Brjuno rotation numbers $\mathcal B\subset \bbR$ consists of $\alpha\notin\bbQ$ for which  $\Phi(\alpha)$ is finite. 
For any sequence $s_n\to 0$, let $\mathcal B_{\{s_n\}}\subset \mathcal B$ be the set of numbers $\alpha=[a_1, a_2, \dots]$ such that 
$$
\sum_{j=n}^{\infty} \alpha_{-1} \alpha_0  \dots \alpha_{j-1} \log \frac{1}{\alpha_j} \le s_n.
$$
In \cite{GY}, we defined an analytic renormalization operator $$\mathcal R\colon \mathcal D_\nu \to \mathcal D_\nu$$ and proved its hyperbolicity on Brjuno rotations. For any $\alpha\in \mathcal B$, let $R_\alpha(z)=z+\alpha$ be the corresponding rotation and let $\mathcal V_\alpha\subset \mathcal D_\nu$ be the local stable manifold of $\mathcal R$ at $R_\alpha$.


\begin{lemma}
\label{lem-deriv-diff-rot}
 For any $\nu>0$ and any sequence $\{s_n\}, s_n\to  0$, there exists $\delta>0$ such that the surfaces $\mathcal V_\alpha\subset \mathcal D_\nu$ for $\alpha\in \mathcal B_{\{s_n\}}$ are graphs of analytic functions over the open ball $U_\delta(0) \subset V_0 = \{f(0)=0\}$. They consist of maps that are conjugate to $R_\alpha$ in  a strip of width $\Pi_{0.5\nu}$.

%  Over the ball of radius $\delta/2$ centered at zero, these surfaces are  the graphs of functions with uniformly bounded $n$-th derivatives;  the bound only depends on $\nu, C, n$.
\end{lemma}
\begin{proof}
Risler's theorem implies that for any Brjuno number $\alpha$, the set  $\mathcal V_\alpha$ is a local analytic submanifold of codimension 1 at $R_\alpha$ that consists of maps that are conjugate to $R_\alpha$.

%In Lemma 5.5 \cite{GY}, we proved that for a small $\kappa=\kappa(C, \nu)$, if $f\in \mathcal V_\alpha$ is $\kappa$-close to the rotation $R_\alpha$, then $f$ is conjugate to $R_\alpha$ in  a strip of width $\Pi_{0.4\nu}$.  In Lemma 5.8, we used this to show that relative boundaries of analytic surfaces $\mathcal V_\alpha$ cannot belong to $\kappa$-neighborhood of rotations, and concluded that $V_\alpha$ are graphs over the open ball of radius $\kappa$ in the space $V_0 = \{f(0)=0\}$.  This lemma was formulated for a particular sequence $\alpha_n$ of periodic rotation numbers, but is proved in the same way for any $\alpha\in \mathcal B_C$, since  we know that $\mathcal V_\alpha$ is analytic. This completes the proof.

The proof of Theorem 1.5 in \cite{GY} shows that the statement of Lemma \ref{lem-deriv-diff-rot} holds if $\nu  > c_1 \Phi(\alpha)+c_2$ for large universal $c_1, c_2$. Indeed, for $\alpha = [a_1, a_2, \dots]$, the analytic surface $\cV_\alpha$ was constructed as a limit of the sequence of analytic surfaces $\cV_n$ that correspond to periodic continued fractions of the form $[a_1, a_2, \dots, a_n, a_1, a_2, \dots, a_n, \dots]$, and Lemma 5.8 shows that all these surfaces for all $\alpha\in \mathcal B_{s_n}$ are graphs over the same ball in $V_0$.  This implies that all $\cV_\alpha$ are graphs over the same ball in $V_0$ if $\nu > c_1 \Phi(\alpha)+c_2$. Moreover, they consist of maps that are conjugate to $R_\alpha$ in $\Pi_{0.7\nu}$ due to Lemma 5.5. 

The reduction to the case of an arbitrary $\nu$ is the same as in Sec. 7 of \cite{GY}. Namely, set $\alpha_k=G^k(\alpha)$; use a renormalization $\mathcal R_{\tilde \eps, m}$ of a high order $m=m(\{s_n\})$ to take the $\delta$-neighborhood of the rotation $R_\alpha$ to a $\delta'$-neighborhood of rotation $R_{\alpha_k}$ in a space $\mathcal D_{\tilde \eps}$, where  $\tilde \eps > c_1 \Phi(\alpha_k)+c_2$. Due to the above,  $\mathcal V_{\alpha_k}\cap U_{\delta'}(R_{\alpha_k})$ are analytic submanifolds that  project to one and the same ball in $V_0$ and consist of maps that are analytically conjugate to $\mathcal R_{\alpha_k}$ in $
\Pi_{0.7\tilde \eps}$. For large $\tilde \eps$, this implies that the conjugacy is close to the identity.

In \cite{GY}, we further define $\mathcal V_\alpha$ as a preimage of $\mathcal V_{\alpha_k} $ under $\mathcal R_{\tilde \eps, m}$.
The construction of renormalization implies that the maps  $f\in \mathcal V_\alpha\cap U_\delta(R_\alpha)$ for small $\delta$ are analytically conjugate to $R_\alpha$, with conjugacy defined in $\Pi_{0.5\eps} $ and close to the identity.  This estimate on the conjugacy implies a uniform  estimate on the slope of $T \mathcal V_\alpha$.  Thus $\mathcal V_\alpha$, $\alpha\in \mathcal B_{\{s_n\}}$,  project to one and the same ball in $V_0$.


% Over the ball of radius $\delta/2$, the $n$-th derivatives of $\mathcal V_\alpha$ are bounded, due to Cauchy estimates; the bound only depends on $n, \nu, C$.

\end{proof}
Roughly speaking, this lemma shows that $\cV_\alpha$, $\alpha\in \mathcal B_{\{s_n\}}$, form an analytic foliation near $R_\alpha$: the union  $(\bigcup_{\alpha \in \mathcal B_{\{s_n\}}} \cV_\alpha)$ is a union of graphs of analytic functions over the same codimension-1 ball.


Now, we prove that the same statement holds in the space of triples in a neighborhood of any triple $\hat f$ with Herman rotation number.
We use the conjugacy of $p(\hat f)$ with rotation to reduce this statement to the previous lemma.

% Then we will use the conjugacy to rotation to prove the statement for any circle map $f$ with Herman rotation number.
\begin{lemma}
\label{lem-local-bound-deriv}

Consider  an $\bbR$-preserving triple $\hat f\in D$ such that $p$ is defined on its neighborhood and $p(\hat f)$ is a circle diffeomorphism with $\rot (p(\hat f)) = \alpha_0 \in \mathcal H$.



 %Consider a circle diffeomorphism $f\in \mathcal D_\nu$, $\rot f =\alpha_0 \in \mathcal H$, $f\in \mathcal D_{\nu}$. Let $d\mathcal D_\nu$ be represented as $V+W$ where the dimension of $W$ is $1$ and $W$ is transversal to $\mathcal V_{\alpha_0}$ at $f$. 
 
 Then for any $\{s_n\}, s_n\to 0$, such that $\alpha_0\in \mathcal B_{\{s_n\}}$, there exists $\delta$ such that whenever  $\alpha\in\mathcal B_{\{s_n\}}$ satisfies $|\alpha-\alpha_0|<\delta$, there exist local analytic submanifolds $\hat {\mathcal V}_\alpha \subset  D$ near $\hat f$. They consist of triples $\hat g$ such that $p(\hat g)$ is a map of an annulus that is conjugate to $R_\alpha$ in a strip of width bounded from below.


 Let $V+W$ be some splitting of $T D$ with $\dim W=1$ such that $W$ is transversal to $T\hat {\mathcal V}_{\alpha_0}$ at $\hat f$.
 Then all $\hat {\mathcal V}_\alpha$, $|\alpha-\alpha_0|<\delta$, can be represented as graphs in $\hat f + V+W$ over one and the same ball $U_{\tau}(0)$ in $V$.
 Moreover, if $\hat {\mathcal  V}_{\alpha}$ is represented as a graph $(x, Q_\alpha(x))$ in the coordinates $V, W$, where $Q_{\alpha}\colon V \to W$, then for any $n$, the derivatives  $Q_\alpha^{(n)}$ are bounded on $U_{\tau/2}(0)$ uniformly on $\alpha$.
\end{lemma}

\begin{proof}
%Let $\hat f = (F,H_a, G)$. In this proof, we  consider the lift of $F$ from $\bbC/\bbZ$ to $\bbC$. Slightly abusing the notation, we will use the same letter $F$ for the lift.

\noindent
\textbf{Construction of $\hat  {\mathcal V}_{\alpha}$.}

%Let $f=p(\hat f)$. Since $f$ is an analytic circle diffeomoprhism, it extends univalently to  a certain strip  $\Pi_\mu$, and the same holds for projections of all $\delta$-close triples if $\delta$ is sufficiently small.

%Note that for small $a_1$, all  $f\in p(K_{[a_1, a_0]})$ are analytic circle diffeomorphisms that are univalent in the strip of width $$\mu:=(0.5\sqrt{a_0}).$$ Indeed, elements of each triple $\hat f\in K_{[a_1, a_0]}$ are well-defined in a $2 \eps$-neighborhood of $\bbR$, and the critical points of $p(\hat  f)$ are outside the strip $$G^{-1} (\{\Im z \in [-\sqrt{a_0}, \sqrt{a_0}]\}).$$

%Thus the first statement follows from the Risler's theorem applied to $f\in \mathcal D^{diff}_\mu$. Indeed, Risler's theorem implies that for a Herman number $\alpha$, the condition $\{\rot g =\rot f\}$ is analytic, thus $\hat {\mathcal V_\alpha}= \{\rot p(\hat g) = \rot p(\hat f\}) $ is an analytic condition in the space of triples; since the tangent plane to $\hat {V_{\alpha_0}}$ at $f$ is transversal to $W$, the map $Q_\alpha$ is analytic.

Let $f=p(\hat f)$. Due to Yoccoz's theorem \cite{Yoccoz2002}, $f$ is conjugate to $R_{\alpha_0}$ via some analytic circle diffeomorphism $h$; let $L_h$ be a conjugacy with $h$, $f=L_h R_\alpha$. Fix $\mu>0$ and consider analytic codimension-1 submanifolds $\mathcal V_\alpha\subset \mathcal D_\mu$, $\alpha\in \mathcal B_{\{s_n\}}$, near $R_{\alpha_0}$. Due to the previous lemma, for some $\tau$, all $\mathcal V_\alpha$ are graphs over the $\tau$-ball $U_\tau(0)$ in $V_0$, and thus the relative boundaries of $\mathcal V_\alpha$ are uniformly detached from $R_\alpha$. Also,  all maps in $\mathcal V_\alpha \cap U_{\tau}(R_\alpha)$ are conjugate to the rotation $R_{\alpha}$ in $\Pi_{0.5\mu}$. In what follows, we trim $\mathcal V_\alpha$ to $\mathcal V_\alpha \cap U_{\tau}(R_\alpha)$.

Note that the tangent spaces of $\mathcal V_\alpha$ at rotations are parallel, given by $\int_{\bbR/\bbZ} v dx =0$. Since $\mathcal V_\alpha$ are graphs of bounded functions, their second derivatives are bounded in $U_{\tau/2}(0)$ and thus their tangent spaces $T\mathcal V_\alpha$ are close to $T\mathcal V_{\alpha_0}$ at $R_{\alpha_0}$ for $\alpha\approx \alpha_0$ and small $\tau$.

%Trim $\mathcal V_\alpha$ to these graphs over $U_\delta(0)$ and c
Consider $\hat {\mathcal V}_{\alpha} = p^{-1}(L_h (\mathcal V_{\alpha}))\cap U_{\delta}(\hat f)$. These are analytic submanifolds, since $p$ is analytic and  $L_h$ is an analytic invertible operator. They are non-degenerate and have codimension 1, since $dp$ does not take $TD$ into $T\mathcal V_\alpha$ (indeed, the tangent vector field to the family of triples  $(F+\eps, H_a, G)$ is a unit vector field that does not belong to $T\mathcal V_\alpha$). On  $\hat {\mathcal V}_\alpha$,  projections of all triples are conjugate to $R_{\alpha}$ on some fixed strip $\Pi_\nu\subset h^{-1}(\Pi_{0.5\mu})$ that depends on $f$ only. Finally, tangent spaces of $\hat {\mathcal V}_\alpha$ in a neighborhood of $\hat f$ are close to the tangent space of $\hat {\mathcal V}_{\alpha_0}$ at $\hat f$, since $T\mathcal V_{\alpha}$ is  close to $T|_{R_{\alpha_0}}\mathcal V_{\alpha_0}$. 


\noindent

\textbf{Boundaries of $\hat {\mathcal V}_\alpha$ and the domain of $Q_\alpha$}

If $\delta$ was chosen sufficiently small, the relative boundaries of $\hat {\mathcal V}_{\alpha}$ belong to the boundary of $U_\delta(\hat f)$: indeed, the relative boundaries of $\mathcal V_\alpha$ are detached from $R_\alpha$, and thus the relative boundaries of $p^{-1}(L_h (\mathcal V_{\alpha}))$ are detached from $\hat f$.  After trimming by $U_\delta(\hat f)$ with sufficiently small $\delta$, all relative boundaries will belong to the boundary of $U_\delta(\hat f)$.

Tangent spaces to $\hat{\mathcal V}_{\alpha}$ at $\hat g\approx \hat f$ are uniformly transversal to $W$ since they are close to $T|_{\hat f} \mathcal V_{\alpha_0}$, which is transversal to $W$.   Thus for $|\alpha-\alpha_0|<\delta'$ with small $\delta'$, all $\hat{\mathcal V}_{\alpha}$ are graphs over one and the same subset of $V$, i.e. $Q_\alpha$ have same domain of definition $U_\tau(0)\subset V$.

%Recall that if a point of a relative boundary of trimmed submanifold $\hat {\mathcal V_\alpha}$ is located inside $B_\delta$, then it is  mapped under $L_h p$  to the boundary of $\mathcal V_\alpha$ that is detached from $R_\alpha$. Due to continuity of $L_h p$, is detached from $\hat f$. Due to the bound on $(Q_\alpha)'$, we conclude that  for $\alpha\approx \alpha_0$,  $\mathcal V_{\alpha}$ are graphs over the one and the same ball in $V$, i.e. $Q_\alpha$ are all defined on the same ball in $V$.  From now on, we restrict $Q_\alpha$ to this ball.

\noindent
% \textbf{Bound on $Q_\alpha$.}
%
% Now, let us prove that $Q_\alpha$ are uniformly bounded on the ball chosen above. Recall, that the projection $p$ is only defined for $a,G$ in a certain neighborhood of $(0, \text{Id})$, thus for each triple $\hat g\in \hat V_\alpha$, the second and the third components $a$, $G$ of the triple $\hat g$ are bounded. Now, we will estimate the lift to $\bbC$ of the map $p(\hat g)$ for $\hat g\in \mathcal V_\alpha$, and thus the lift to $\bbC$ of the first component of the triple $\hat g$. This will imply the bound on $Q_\alpha$. Note that if the $F$-component of the vector field that generates $W$ is zero, then the first component of $Q_\alpha(x)$ is zero and there is nothing to prove.
%
% Assume $\rot f=\alpha_0\in (0,1)$.  For $\alpha\approx \alpha_0$ and any $g$ that belongs to $p(\mathcal V_\alpha)$, the value $g(x), x\in \bbR$, must belong to $\Pi_{\nu}$ for any $g\in V_\alpha$ (otherwise the conjugacy of $g$ to the rotation on $\Pi_\nu$ is not possible). This provides an estimate on $\Im g(x)$. Further, the image of $[x-i\nu, x+i\nu]$ under $g$ cannot be to the right from $[x-i\nu+1, x+i\nu+1]$ (or to the left from $[x-i\nu-1, x+i\nu-1]$), otherwise the rotation number of $g$ is greater than one (smaller than $-1$). This provides an estimate on $\Re g(x)$. Since we estimated $g = p(\hat g)=F\circ \pi \circ H_a\circ G$ and we know that $a,G$ belong to the bounded domain, we get an estimate on the first component  $F$ of the triple $\hat g$.  Fixing any point  $x\in \bbR$ where the $F$-component of the vector field generating $W$ is not zero, we can estimate $Q_\alpha$.

% \noindent
\textbf{Derivatives of $Q_\alpha$.}

Since the graphs of $Q_\alpha$ are located in a small neighborhood of $\hat f$, $Q_\alpha$ are bounded in $U_\tau(0)$.
 Cauchy estimates imply that on $U_{\tau/2}(0)$ all functions $Q_\alpha$, $\alpha\approx \alpha_0$, have uniformly bounded derivatives. 
\end{proof}

%Let $\mathcal V_\alpha$ be represented as a graph of the function $q_\alpha$ in $V+W\sim d\mathcal D_\nu$.
%The value of $q_\alpha$ is defined modulo one; however, we can lift these maps to $\bbC$ assuming $\dist (q_\alpha(0), 0)<1$. For the lifts, we get the following result.
%\begin{lemma}
% For small $\delta$, $q_\alpha|_{U_\delta(0)}$ are uniformly bounded for all $\alpha\approx\alpha_0$.
%\end{lemma}
%\begin{proof}

%\end{proof}
%The previous two lemmas, together with Cauchy estimates, imply that in a neighborhood of $f$, all functions $q_\alpha$, $\alpha\approx \alpha_0$, have uniformly bounded derivatives. This contradicts our choice of $f$. The contradictions proves Lemma \ref{lem-deriv-diff}.

%\end{proof}

