\section{Introduction}
Arnold tongues are one of the most familiar images in one-dimensional dynamics (Figure~\ref{fig-tongues}).
Consider the two real parameter family of standard (or Arnold) maps
$$f_{a,b}(z)=z+a+\frac{b}{2\pi}\sin 2\pi z \mod\bbZ,\; a,b\in[0,1].$$
% Figure environment removed
They are analytic diffeomorphisms of the circle $\RR/\ZZ$ for $b\in[0,1)$, and analytic homeomorphisms with a single cubic critical point (critical circle maps) for $b=1$. The rotation number $\rot (f_{a,b})$ is a non-decreassing function of $a$. For each fixed $b>0$, the graph of the  function
  $$a\mapsto\rot(f_{a,b})$$
  is a ``devil's staircase'': a continuous non-decreasing curve with flat ``steps'' at rational heights. By definition, Arnold $\alpha$-tongue is the set of parameters $(a,b)$ which corresponds to a rotation number $\alpha$. For a rational $\alpha=p/q$, this set is bounded by two curves (graphs over the $b$-axis) which correspond to the algebraic condition of having a periodic orbit with combinatorial rotation number $p/q$ and the unit multiplier. The boundaries of a rational tongue are thus algebraic. They meet at the point $(p/q,0)$ corresponding to the rigid rotation by $p/q$, cutting out a sharp tongue-looking shape. For an irrational $\alpha$, the tongue is a curve -- a continuous graph over the second coordinate. The  question of smoothness of irrational tongues is deep and fundamental; it is the central theme of this paper.



In 2001, L.~Slammert \cite{Slammert} proved $C^1$-smoothness of \emph{open} irrational Arnold tongues (that is, without the endpoints $b=1$ corresponding to critical circle maps). His argument rests on the theorem of Douady and Yoccoz  that for a circle diffeomorphism $f$ there exists a unique  {\it (-1)-measure} $\lambda$ which satisfies the invariance law
$$\int\xi(f(x))\frac{1}{f'(x)}d\lambda=\int\xi d\lambda\text{ for a test function }\xi.$$
Slammert used these measures to describe the tangent bundle of an irrational tongue.

 Much stronger results can be obtained by imposing arithmetic conditions on $\alpha$. The  strongest of them is due to E.~Risler \cite{Risler} who showed that open Arnold tongues corresponding to Herman numbers $\alpha$ are analytic curves. The Herman class (defined by Yoccoz \cite{Yoccoz2002})  consists of rotation numbers $\alpha$ such that if $\rot f = \alpha$ for an analytic  circle diffeomorphism $f$, then it is analytically conjugate to the rotation by $\rot f$.
Risler also showed that in a larger Brjuno class of rotation numbers, the curves are locally analytic for sufficiently small values of $b$, and, furthermore, form a foliation by analytic curves over $0\leq b<b_0$ given a Brjuno condition with a uniform rate of convergence (see \cite{Risler} for the details). Using quasiconformal surgery, similar results were later obtained in the standard family in \cite{FaGe}.


 
%Recall that the rotation number of an orientation-preserving circle homeomorphism $f\colon \bbR/\bbZ\to \bbR/\bbZ$ is defined as follows:
%$$
%\rot f = \lim_{n\to \infty}\frac{F^n(x)}{n} \mod 1,
%$$
%where $F\colon \bbR\to \bbR$ is a lift of $f$ to the real line.

%\begin{definition}
% 
%\end{definition}

  Note that the above results do not address the question of the degree of  smoothness of Arnold tongues at the ends $b=1$, which is quite subtle. Indeed, the above proofs break down when circle maps develop critical points.
  In \cite{LlaveLuque}, De la Llave and Luque
  %considered several analytic families of analytic circle maps that contain critical circle maps. One example is the 
%Arnold's maps
%$$ f_{a,b} (z)= z+a + \frac {b}{2\pi} \sin 2\pi z\operatorname{ mod }\ZZ,$$
%which   are diffeomorphisms of the circle for $b\in[0,1)$, and are homeomorphisms with a single cubic critical point for $b=1$ (critical circle maps).
   performed numerical experiments on the smoothness of Arnold tongues at  $b=1$.
  Based on the results of these experiments, they conjectured that Arnold tongues that correspond to Diophantine rotation numbers are finitely smooth at the ends. In particular, they conjectured that the tongue that corresponds to the golden ratio is $C^2$-smooth at the endpoint. De la Llave and Luque suggested an explanation of this fact that involves the behaviour of a conjectural renormalization operator on a neighborhood of critical circle maps. 

In the same way as for the standard family, given any parametric family $f_\mu$, $\mu\in\bbR^n$ of circle homeomorphisms, the Arnold $\alpha$-tongue can be defined as the set of parameters $\mu $ for which $\rot(f_\mu)=\alpha$. The results of Slammert and Risler, as well as the conjectures of De la Llave and Luque suitably translate into this general setting.

Recently, in \cite{GY}, we developed a new approach to results of \cite{Risler} by constructing an analytic renormalization operator for which Brjuno rotations form a hyperbolic invariant set with a codimension-one stable foliation. Furthermore, the analytic stable submanifolds coincide with analytic conjugacy classes of rotations. For an analytic family of diffeomorphisms which crosses an $\alpha$-leaf of this foliation transversally, the Arnold $\alpha$-tongue will be an analytic codimension-one surface, implying the above quoted results of Risler. In the present paper, we define a novel renormalization framework, which combines the renormalization of diffeomorphisms and renormalization of critical circle maps developed by the second author (see \cite{Ya3,Ya4} and references therein).

The new renormalization operator has two hyperbolic horseshoes: the one consisting of rigid rotations with a single unstable direction as in \cite{GY} and another ``nontrivial'' one consisting of critical circle maps with two unstable directions. Arnold tongues in this picture lie in the stable-unstable manifolds, whose smoothness at critical circle maps is quantified in terms of the Lyapunov exponents of the second horseshoe. This is, roughly, what was pictured in \cite{LlaveLuque}. We give  estimates of the expansion factors of renormalization of critical circle maps, and use this to give a lower bound on the smoothness of closed Arnold tongues.

Of course, renormalization of diffeomorphisms we defined in \cite{GY} cannot be directly extended to maps with critical points. The method we use to put the two horseshoes under one roof is new and will likely be useful in other contexts. Another notable step in our construction is extension of the results on existence, uniqueness, and general properties of (-1)-measures to maps with critical points. This is a necessary part of our proofs, and also allows us to extend Slammert's result to closed, rather than open, Arnold $\alpha$-tongues for {\it all} irrational $\alpha$. But it is also of an independent interest: (-1)-measures are a useful technical tool in the study of dynamics, but also provide a new description of the stable tangent bundle of renormalization. In \S~\ref{sec-unifhyp} we use (-1)-measures to give a new proof of renormalization expansion, with explicit bounds.
Another useful tool developed in this paper is a theorem on the smoothness of stable-unstable manifolds in a general Banach space setting, which has been previously missing in the literature.


%  In this paper we justify this explanation, and give a lower bound $k>1$ on the smoothness of Arnold tongues in terms of the Lyapunov exponents of a renormalization operator.
  
%%  \begin{remark}
%%    \label{rem-slammert}
%%    The $C^1$-smoothness of \emph{open} irrational Arnold tongues (that is, without the endpoints corresponding to critical circle maps) was established by L.~Slammert in \cite{Slammert}. It turns out that the same technique can be used to show $C^1$ smoothness of closed Arnold tongues, as we demonstrate in \S\ref{sec-measure} (see Corollary~\ref{cor:C1}). Our proof of this weaker result does not use any assumptions on Lyapunov exponents of the action of renormalization.
%%    Instead, we derive it from a theorem about invariant (-1)-measures of critical circle maps, which is interesting in its own right. 
  %%%  \end{remark}

Let us proceed with formulating our main results. 
Let $\deps$ be the set of analytic maps $f\colon \bbC/\bbZ \to \bbC/\bbZ$ that have bounded analytic continuations to the strip of width $\eps$ around $\bbR/\bbZ$. Equipped with the sup-norm in this strip, $\deps$ is a complex Banach manifold. Its real slice $\depsr$ consists of circle-preserving maps. Let $\dcreps \subset \deps$ be the Banach submanifold of analytic maps  $f\colon \bbC/\bbZ \to \bbC/\bbZ$  with $f'(0)=0$, $f''(0)=0$, $f'''(0)> 0$, that extend to the strip of width $\eps$ around $\bbR/\bbZ$. Its real slice $\dcrepsr=\dcreps\cap \depsr$ consists of {\it critical circle maps} -- analytic circle homeomorphisms with a single cubic critical point at the origin. 

In \cite{Ya3}, the second author constructed the cylinder renormalization operator $\mathcal R_{cyl}$ on an open neighborhood of $\dcrepsr$ in $\dcreps$ for sufficiently large $\eps$, and showed that it is hyperbolic, with one unstable direction, on an invariant horseshoe-like set $\mathcal I$. This result is formulated below, see Theorem \ref{th-horseshoe}.

In \cite{GY}, we defined a renormalization operator $\rdif$ on a neighborhood of rotations in  the space $\deps$ for sufficiently large $\eps$ (the space of analytic diffeomorphisms of an annulus), and proved its hyperbolicity at Brjuno rotations, with a single unstable direction corresponding to the rotation angle. The construction was motivated by Risler's result \cite{Risler}, and by  Yoccoz's result \cite{Yoc} on linearizations of circle diffeomorphisms.

In both cases, the renormalization of a circle map was defined to be the first return map to  a fundamental domain $[0, f^n(0)]$ for a suitably chosen $n$, in a certain analytic chart on $[0, f^n(0)] / f^n \sim  \bbR/\bbZ$. The key to either construction lied in the choice of a specific analytic chart. 
Let us generally say that a smooth mapping $\cR$ of $\depsr$ to itself is a renormalization operator if for any circle map $f$ in its domain there exists $N$ such that the circle map $\cR f$ is conjugate, via an analytic circle map, to a first-return map under $f$ to a quotient of the fundamental domain $[0, f^{N}(0)] / f^N$.


\begin{theorem}
\label{th-operator}
For a sufficiently large $\eps$, there exists a Banach manifold  $D$, a codimension-1 submanifold $D_0\subset D$, a projection $p \colon D\to \deps$ (defined on a certain subset of $D$) such that $p(D_0)=\dcreps$, and an operator $\mathcal R \colon D\to D$ such that the following holds:
 \begin{itemize}
  \item $\mathcal R$ is an analytic operator with compact derivative on its domain, and preserves $D_0$;
  \item $\mathcal R$ commutes with the projection: $p \mathcal R = \mathcal R p$ on the respective domains of definition;
  \item $\mathcal R$ contracts on fibers $p^{-1} f$;
  \item The restriction of $p \mathcal R p^{-1}$ to circle homeomorphisms  is a renormalization operator, in the sense described above;
  \item There exists a set $\Lambda \subset D_0$ invariant under $\mathcal R$ such that $p \Lambda = \mathcal I$, $\mathcal R$ is uniformly hyperbolic on a set $\Lambda\subset D_0$, with two-dimensional unstable subspaces at any trajectory in $\Lambda$. These unstable spaces intersect $D_0$ on one-dimentional subspaces.
  
 \end{itemize}


\end{theorem}


 The Banach manifold $D$ will be the space of triples described in Sec. \ref{sec-triples} below; a similar construction  was used before in \cite{GorYa}.

%
% \begin{theorem}
% Let $\Lambda$ be a compact invariant set of an analytic operator $R$, $U$ be its neighborhood. Let $\{x_k\}$ be an orbit of $R$ that stays in $U$, $x_0=x$, and let the eigenvalues of $R$ along $\{x_k\}$ be $\lambda_1>\lambda_2>1>\dots$, with eigenvectors $\xi_1,  \dots$.  Let $(M_k, x_k)$ be the sequence of  codimension-1 continuous local manifolds at $x_k$ such that $R M_n = M_{n+1}$.
%
% Let $C_n$ be an unstable cone around $dR^n|_{x} \xi_1$. Assume that for each $y\in M_n$, the manifold $M_n$ intersects $y+C_n$ at $y$ only.
%
% Let $y_k\to x$ and assume that $M_0$ is $r$-differentiable at $y_k$, $r<\log \lambda_2/\log\lambda_1$. Then the $r$-derivatives of $M_0$ at $y_k$ converge as $y_k\to x$.
%
%  Let $A_k$ be an infinite sequence of linear operators in the Banach space $B$ such that their infinite product has two unstable eigenvectors: $\|A_k A_{k-1}\dots A_1  \xi_1 \|> C_1 \lambda_1^k \xi_1 $,   $\|A_k A_{k-1}\dots A_1  v \|> C_2 \lambda_2^k v $ for $v \in \langle \xi_1, \xi_2\rangle$, and $\|A_k A_{k-1}\dots A_1  v \|< C_3 \mu^k v $ for $\mu<1$ and $v\in V$ such that $V + \langle \xi_1, \xi_2\rangle=B$.
%  Let $R_k$ be (nonlinear) analytic operators on a Banach manifold $M$ correcponding to $B$ such that for some $c$, $R_k'(x_k)=A_k$ and $|R_k(x)-A_k'(x-x_k)|<c |x-x_k|^2$ for all $|x-x_k|<\delta$. Let $M_n$ be the sequence of codimension-1 manifolds such that $R_n M_n = M_{n+1}$ and $M_n$ are uniformly  transversal (TODO) to $A_k \xi_1$. Then $M_n$ are at least $ C^k$ smooth with $k>\log \lambda_1/\log \lambda_2$.

% \end{theorem}
% We will apply this lemma to the (compact) part $\Lambda$ of the horseshoe $\mathcal I$ that corresponds to bounded-type rotation numbers.
% The sequence of manifolds $\{\rot f = \alpha_n\}$ will serve as $M_n$.
%  Clearly, for irrational $\alpha$, the tangent space to $\{\rot f = \alpha\}$ at diffeomorphisms does not contain any vectors $v$ with $\inf_{S^1} v >0$. This means that it does not intersect the unstable cone field as defined in Lemma 9.3 in \footnote{M.Yampolsky, Hyperbolicity of renormalization of critical circle maps, 2002}. If this cone field corresponds to the largest eigenvalue of $\mathcal R$, we get the following corollary.


Using this result, we will prove the following conditional result on the smoothness of Arnold tongues.
\begin{definition}
Let $R$ be a smooth operator in the Banach space. Suppose that for some orbit $g_n = R^n g$, $R$ has an invariant direction field $l_{g_n}\in T_{g_n}D$, that is, $d|_{g_n}R \, l_{g_n} = l_{g_{n+1}}$. We will say that \emph{the maximal (resp. minimal) expansion rate} of $R$ along $l_{g_n}$ is
$$\lambda_{max}=\limsup_{n\to \infty} \sup_{m\ge 0} \sqrt[n]{\|dR^n|_{l_{g_m}}\|}; \qquad \lambda_{min}=\liminf_{n\to \infty} \inf_{m\ge 0} \sqrt[n]{\|dR^n|_{l_{g_m}}\|}.$$
\end{definition}
The definition of the expansion rate is close to the definition of the Lyapunov exponent. Note also that if $g_n$ is a $q$-periodic orbit, then both expansion rates coincide with the root of order $q$ of the modulus of the eigenvalue of $R^q$ that corresponds to the eigenvector contained in $l_{g_0}$.


Let $f_\mu$ be an analytic family of analytic self-maps of $\bbC/\bbZ$ defined  in a neighborhood of $\bbR/\bbZ$ for $\mu_j\in (\mathbb C^n, 0)$.  Suppose that for real $\mu_j$ and $\mu_1<0$, $f_\mu$ is a diffeomorphism of $\bbR/\bbZ$. Suppose that   for $\mu_1=0$, the maps $f_\mu$ are cubic critical circle maps with $0$ as a critical point. Suppose that $f'_{\mu}(0)$ has a zero of order  1 at $\mu_1=0$: when $\mu_1$ encircles zero, $f'_\mu(0)$ makes one turn around zero.
Finally, assume that $\frac {\partial f}{\partial \mu_2}>0$ on the circle whenever $\mu_1=0$.
An example to keep in mind is
the Arnold family, normalized as
\begin{equation}
  \label{eq:arnoldfamily}
  f_{\mu_1, \mu_2} (z)= z+\mu_2 + \left(\mu_1+\frac {1}{2\pi}\right) \sin 2\pi z.
  \end{equation}

\begin{theorem}
\label{th-smoothness}
Let $g\in \mathcal I$ be a cubic critical circle map with $\rot g=\alpha$ where $\alpha$ is of a bounded type. Let $\hat g \in p^{-1}(g)$  belong to $\Lambda$. Suppose that the renormalization operator  $\mathcal R\colon D\to D$ has an invariant direction field $\xi_{\mathcal R^n \hat g}\subset E^u_{\mathcal R^n \hat g}$, such that   $\xi_{\mathcal R^n \hat g}$ is uniformly transversal to $D_0$. Let $\lambda_1$ be the minimal expansion rate along the direction field $E^u_{\mathcal R^k \hat g} \cap D_0$, and let $\lambda_2$ be the maximal expansion rate along $\xi_{\mathcal R^k \hat g}$. Assume $\lambda_1>\lambda_2$.

Then in any $C^\omega$-family $f_{\mu}$,  $\mu = (\mu_1,\dots, \mu_n)\in \bbR^n$,  as above with $\rot f_{0} = \alpha$, the Arnold $\alpha$-tongue  is given by a function $\mu_2=\mu_2(\mu_1, \mu_3, \dots, \mu_n)$ that is at least $C^k$-smooth at $0$, where $k=\lfloor \frac{\log\lambda_1}{\log \lambda_2} \rfloor$.
\end{theorem}
Since $\mathcal R$ contracts on fibers of $p$ that project to $D_0$, we will see that $\lambda_1,\lambda_2$ only depend on $\alpha$ and not on $g$, $\hat g$. 

\begin{remark}
 The Arnold tongues for $\mu_1<0$ are analytic curves due to  Risler's theorem \cite{Risler,GY}. This result shows that they are at least finitely smooth at $\mu_1=0$. It is an open question whether these curves are analytic up to $\mu_1=0$.
\end{remark}

\begin{remark}
\label{rem-transversality}
 The condition $df/d\mu_2>0$ can be replaced by the condition that the vector field $df/d\mu_2$ is transversal to the surface $\{rot f = \alpha\}$ for $\mu=0$, see Sec. \ref{sec-stableunstable}.
\end{remark}


\begin{remark}
 For a rotation number $\alpha$ with a periodic continued fraction,  the assumptions of this theorem are equivalent to the condition that for some $g\in \Lambda$ with $\rot g = \alpha$, $d \mathcal R^q|_{\hat g}$ has two distinct  unstable eigenvalues, and the one with the larger absolute value corresponds to $E^u_{\hat g}\cap D_0$ (that is, to critical circle maps).
 
 Remark \ref{rem-periodic} shows that this assumption always holds true; however, for some periodic orbits,  $1< \frac{\log \lambda_1}{\log\lambda_2} <2$ and the conclusion of the theorem does not guarantee  smoothness of Arnold tongues higher than $C^1$.  
\end{remark}



In hyperbolic dynamics, it is standard to expect that if a hyperbolic map has a gap $(\lambda_2, \lambda_1)\subset \bbR$ in its spectrum, then the corresponding invariant manifolds are smooth, and the degree of smoothness is at least $\lfloor \log \lambda_1/\log \lambda_2 \rfloor$. This, together with Theorem~\ref{th-operator}, would imply Theorem~\ref{th-smoothness}, since the codimension-1 manifolds $\{\rot f = \text{const}\}$ form a ``weakly stable'' invariant foliation that corresponds to $\lambda_2$ and all stable eigenvalues of $\mathcal R$, and the corresponding spectral gap is $(\lambda_2, \lambda_1)$. In a Banach space setting, such statements were proven in \cite{ElBialy}.
However, \cite{ElBialy} requires either global estimates on the nonlinearity of the operator or the existence of smooth cut-off functions (that are scarce in Banach spaces), so we cannot refer to these results.
We  prove the smoothness directly in Theorem \ref{th-Banach-op}, see \S~\ref{sec-stableunstable}. We do not need to prove the existence of the invariant  manifold; we study the smoothness of the existing invariant manifold. As a result, we only use hyperbolicity of the operator, with no further assumptions.


% The previous lemma immediately implies this theorem at points of the Arnold tongue that are in a small neihborhood of a horseshoe; otherwise, we will have to renormalize our family several times before we apply it.

The inequality for expansion rates required in Theorem \ref{th-smoothness} holds for the golden ratio rotation number. Indeed, the two top unstable eigenvalues of $\mathcal R$ at the period-two periodic orbit that corresponds to the critical map $f$ with the golden ratio rotation number were estimated rigorously numerically in \cite[Theorem 3.2 and p.19]{Mest}. They equal $\gamma_1\approx -2.83$ (corresponding to critical maps) and $\gamma_2\approx 1.66$.
Since $\frac{\log 2.83}{\log 1.66} = 2.05$, this agrees with the numerical results of \cite{LlaveLuque} on $C^2$-smoothness of Arnold tongues that correspond to the golden ratio rotation number.  

%%% Figure removed
%%\hfil
%% Figure removed
% Figure environment removed

Also, this inequality holds for orbits in $\Lambda$ that correspond to rotation numbers with a sufficiently high type, due to the next result. Thus the inequality holds for the two cases that are believed to be most extreme: high-type rotation numbers (with large denominators of continued fraction convergents), and the golden ratio (with the smallest possible denominators of continued fraction convergents).  It is an open question whether the inequality holds for all Herman numbers.

Consider the Gauss map $G(x)=\{\frac 1 x\}$ and set
$$\alpha_{n}=G^n(\alpha), a_n=[1/\alpha_{n}]$$
when defined.
Numbers $a_n$ are the coefficients of the continued fractional expansion of $\alpha$, and we have $$\alpha = \cfrac{1}{a_0 + \cfrac{1}{a_1+\dots}},$$
which we will abbreviate as 
$$\alpha=[a_0,a_1,\ldots].$$
We also consider the continued fraction convergents
$$\frac{p_n}{q_n}=[a_0,\ldots,a_{n-1}]$$
and write $\alpha \sim \{p_n/q_n\}$ for the correspondence between the number $\alpha$ and its continued fractional convergents.
\begin{proposition}
\label{prop-exponents}
For a real number $\alpha = [a_1,a_2,\dots]$ with $a_k>n$, let $g\in \mathcal I$ have rotation number $\alpha$ and let $\hat g \in p^{-1}g$  belong to $\Lambda$.

Then the minimal expansion rate $\lambda_1$ of the renormalization operator $\mathcal R\colon D\to D$ along the distribution $E^u_{\hat g}\cap D_0$ tends to infinity as $n\to \infty$. Also, $\mathcal R$ has an invariant line field in $E^u_{R^n (\hat g)}$ transversal to $D_0$, and the corresponding maximal expansion rate $\lambda_2 $ remains bounded as $n\to \infty$.
\end{proposition}

This, together with the above, immediately implies the corollary.



\begin{corollary}
\label{cor-smoothness-htype}
 Let $f_\mu$, $\mu = (\mu_1,\dots, \mu_n)\in \bbR^n$ be a family of analytic circle maps as above. Suppose that $\alpha=[a_1,a_2,\dots]$ with $n<a_k<m$. For sufficiently large $n$, the Arnold $\alpha$-tongue in the family $f_\mu$ is (at least) $C^k$ smooth, where $k=k(n)$ tends to infinity as $n\to \infty$.
\end{corollary}

 One can extend this result to rotation numbers such that in the corresponding  sequence $a_k$, the numbers with $a_k<n $ occur sufficiently rarely.


\begin{theorem}
\label{th-smoothness-htype2}
Let $f_\mu$, $\mu = (\mu_1,\dots, \mu_n)\in \bbR^n$ be a family of analytic circle maps as above. For any $\delta\ge 0$, for sufficiently large $K$,   if  $\alpha=[a_1,a_2,\dots]$ is of bounded type and for all $m$, for sufficiently large $n$, we have $$\frac{\#\{k \mid a_k<K, m\le k<m+n\}}{n}\le \delta,$$ then the Arnold $\alpha$-tongue in the family $f_\mu$ is (at least) $C^k$ smooth, where $k=k(K, \delta)$ tends to infinity as $K\to \infty$.
\end{theorem}

Let us conclude the introduction with a brief guide to the layout of the paper. In \S~\ref{sec-cylren} we recall the basic setting of cylinder renormalization for critical circle maps. In \S~\ref{sec-measure} we construct (-1)-measures for critical circle maps and prove an extension of Slammert's result on smoothness of open Arnold tongues to the endpoints. In \S~\ref{sec-triples} we describe a Banach manifold setting in which renormalization horseshoes of diffeomorphisms and critical circle maps can be combined, and prove Theorem~\ref{th-operator}. In \S~\ref{sec-derivatives} we give geometric estimates for the stable foliation of renormalization away from critical circle maps. Section~\ref{sec-stableunstable} contains the statement of a general result on the smoothness of stable-unstable manifolds in a Banach space setting (Theorem \ref{th-Banach-op}) and the proof of Theorem \ref{th-smoothness} modulo Theorem \ref{th-Banach-op}. In \S~\ref{sec-Banach-op} we prove Theorem~\ref{th-Banach-op}. Finally, \S~\ref{sec-unifhyp} contains a new construction of the expanding direction of renormalization of critical circle maps, with explicit estimates on the expansion rate as needed to derive Proposition~\ref{prop-exponents} and Theorem~\ref{th-smoothness-htype2}.
