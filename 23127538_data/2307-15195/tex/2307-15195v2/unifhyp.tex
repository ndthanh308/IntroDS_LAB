\section{Uniform hyperbolicity and expansion rates of the renormalization operator}
\label{sec-unifhyp}
In this section we present a new construction of the expanding direction of renormalization, different
from the approach taken in \cite{Ya3} and \cite{GorYa}. The proof we present uses (-1)-measures constructed in \S~\ref{sec-measure}, and 
%Uniform hyperbolicity for cylinder renormalization was proved in \cite{Ya3}. However, the proof contains a minor mistake, namely in the construction of unstable cones for $\mathcal R_{cyl}$, in the estimate on the expansion rate.
provides explicit estimates on the expansion rate of the renormalization operator as needed for Proposition \ref{prop-exponents} and Theorem \ref{th-smoothness-htype2}.


Recall that the cylinder renormalization $\cren f$  of a critical circle map $f$ is the first return map to a fundamental domain of the map  $f^{q_{N}}$, in the straightening chart.
In this section, it will be convenient for us to use $n=N$ as a parameter; we will write $\crenn f$ to denote the cylinder renormalization on the fundamental crescent of the iterate $f^{q_n}$. 
%%%We will omit $n$ if this does not lead to a confusion.
Here and below we assume that $n$  is even; for odd $n$, all proofs are analogous, but the straightening map reverses the orientation on the circle, etc.

Recall that  $\Psi=\Psi_n$ is the straightening coordinate  used in the cylinder renormalization: $\Psi$  is defined $C\cup f^{q_n}(C)$ where $C$ is a crescent-shaped fundamental domain of $f^{q_n}$ joining fixed points of $f^{q_n}$, and $\Psi$ conjugates $f^{q_n}$ to $z\to z-1$ in $C\cup f^{q_n}(C)$. We may and will assume that $f^{q_n}(C)$ contains $[f^{q_n+q_{n+1}}(0), f^{q_{n+1}}(0)]$, and $\Psi(0)=0$.
%From now on, we assume that $C$ contains zero (we can always modify $C$ using iterates of $f^{q_n}$ in a neighborhood of $(0, f^{-q_n}(0))$ where $f^{q_n}$ is univalent). Then $\Psi(0)=0$.


Take a vector field $v\in T\mathcal D^{cr}_{\eps}$. Consider the family $f_a=f+av$. Let $P_a$ be the first-return map to $[f_a^{q_n+q_{n+1}}, f^{q_{n+1}}_a(0)]$. The map $\mathcal R_{cyl} f_a$ for small $a$ coincides with $\Psi_a P_a \Psi_a^{-1}$, where $\Psi_a$ is the straightening chart for $f_a^{q_n}$ as above.
 We have
 \begin{multline}\label{eq-iter}d|_f\mathcal R_{cyl, n} v =\\ \frac d{da} \mathcal R_{cyl, n} f_a =  \Psi'_a  |_{P\Psi^{-1}} + \Psi'|_{P\Psi^{-1}} \cdot P'_a|_{\Psi^{-1}(x)}  - (\Psi P \Psi^{-1})'(x)  \cdot\Psi'_a |_{\Psi^{-1}(x)} = \\ \Psi'_a  |_{\Psi^{-1}(\mathcal R_{cyl}f(x)) }   - (\mathcal R_{cyl}f )' (x)\cdot  \Psi'_a |_{\Psi^{-1}(x)}+ \Psi'|_{P\Psi^{-1}(x)} \cdot P'_a|_{\Psi^{-1}(x)}.
   \end{multline}
  In the first summand, we choose the representative of  $(\mathcal R_{cyl}f)(x)\in \mathcal \bbR/\bbZ$ in $\bbR$ that belongs to  $\Psi([f^{q_n+q_{n+1}}(0), f^{q_{n+1}}(0)])$.
    

Since the map $P$ equals either $f^{q_{n+1}}$ or $f^{q_{n+1}+q_n}$, we have the following two expressions for  $P'_a|_{\Psi^{-1}(x)}.$  Let $y=\Psi^{-1}(x)$.
\begin{itemize}
\item On the arc of the circle where $P(y)  = f^{q_{n+1}}(y)$, we have
\begin{equation}
 \label{eq-sum1}
 P'_a(y)  =  \sum_{l=1}^{q_{n+1}} (f^{q_{n+1}-l})'(y_{l}) v(y_{l-1}),
\end{equation}
 where $y_l = f^l (y)$;
 \item On the arc of the circle where $P(y)  = f^{q_n+q_{n+1}}(y)$, we have    \begin{equation}
\label{eq-sum2}
P'_a(y) = \sum_{l=1}^{q_n+q_{n+1}} (f^{q_n+q_{n+1}-l})'(y_{l}) v(y_{l-1}).
\end{equation}
% the term that corresponds to $l=q_{n+1}$ in the sum is $1 \cdot v(x_{q_{n+1}-1})$.
%Analogous formula with $q_n+q_{n+1} $ instead of $q_{n+1}$ holds for the part of the circle where $Pf=f^{q_n+q_{n+1}}$.
\end{itemize}

Recall that  (-1)-measure $\mu_f$ of a smooth circle map $f$ is a probability measure such that for any continuous test function $\phi$, we have
\begin{equation}
 \label{eq-measure}
 \int \phi(x) d\mu_{f} = \int \left. [f'\phi ]\right|_{f^{-1}(x)} d\mu_{f}.
\end{equation}
 In Section \ref{sec-measure}, we proved that any cubic critical circle map has a unique  (-1)-measure $\mu_f$. Also, we proved that the stable distribution for $\cren$ is given by the condition $L_f(v)=0$ where linear functionals $L_f$ on $T\mathcal D^{cr}_\eps$ are given by  $ L_f v = \int v|_{f^{-1}(z)} d\mu_f$. Here and below, integrals are computed along the circle $\bbR/\bbZ$.
 We will prove the following.

\begin{lemma}
\label{lem-mu-zero}
The sum of the first two summands $$\Xi(x) = \Psi'_a  |_{\Psi^{-1}(\crenn f(x)) }   - (\crenn f )' (x)\cdot \Psi'_a |_{\Psi^{-1}(x)}$$ in \eqref{eq-iter} satisfies $$ L_{\crenn f} (\Xi) = \int  \Xi|_{(\crenn f)^{-1}(x)} d\mu_{\crenn f} =0.$$

\end{lemma}
\begin{proof}

%%We conclude that 
%%$$\int _{\mathcal Rf[0,1]} \Xi|_{(\mathcal Rf)^{-1}(z)} d\mu_{\mathcal Rf} > 
%\int _{[0,1]} \psi(z) d\mu_{\mathcal Rf}- \int_{\mathcal Rf [0,1]}  (\mathcal Rf)'|_{(\mathcal Rf)^{-1}(z)}  \phi({(\mathcal Rf)^{-1}(z)}) =0.$$

%
%Lift $\mathcal R f$ and the measure $\mu_{\mathcal Rf}$ to the line $\bbR$, preserving the same notation. 
%We will use the expression $\mathcal Rf = \Psi f^{q_{n+1}} \Psi^{-1}$ on the interval $[0,1]$, and thus replace the target integral with $\int _{\mathcal Rf[0,1]} \Xi|_{(\mathcal Rf)^{-1}(x)} d\mu_{\mathcal Rf}$. 

 Recall that by Lemma \ref{lem-noatoms}, the (-1)-measure $\mu_{f}$ has no atoms.
 %an atom at zero. Indeed, let $\mu_{f}(\{0\})=a$ and  use this formula for a sequence of continuous functions $\phi_n$ that approximate $\chi_{[-\frac 1n, \frac 1n]}$, then the limit of the left-hand side is $a$, and the limit of the right-hand side is zero, since $f'(0)=0$. We conclude that $a=0$. One can prove that (-1)-measures of critical maps with irrational rotation numbers cannot have atoms, but we will not need this fact here.
 Thus  the formula \eqref{eq-measure} holds true for any function $\phi$ that has a single jump discontinuity.
Let $\phi =\Psi'_a |_{\Psi^{-1}(x)} $; this function is continuous on $$I=\Psi([f^{q_n+q_{n+1}}(0), f^{q_{n+1}}(0)]) = [(\crenn f)(0)-1, (\crenn f)(0)] $$ and has a single jump discontinuity at $\crenn f(0)\in \bbR/\bbZ$.
Let $x\in I$. We will write $(\crenn f)^{-1}(x)\in \mathbb R$ for the lift of the point     $(\crenn f)^{-1}(x)\in \mathbb R/\mathbb Z$ that belongs to $I$.
Then the first summand of $\Xi$, computed at a point $(\crenn f)^{-1}(x)$, equals $\phi(x)$. Also, the second summand of $\Xi$, computed at a point $(\crenn f)^{-1}(x)$, is  $\left. [(\crenn f)' \cdot   \phi ]\right|_{(\crenn f)^{-1}(x)}. $ We get
\begin{multline*}\int \Xi|_{(\crenn f)^{-1}(x)} d\mu_{\crenn f} = \\ \int  \phi(x) d\mu_{\crenn f}- \int \left. [ (\crenn f)' \phi ]\right|_{(\crenn f)^{-1}(x)}d\mu_{\crenn f} =0\end{multline*}
due to \eqref{eq-measure} applied to $\mu_{\crenn f}$.
%It remains to prove that the first integral is greater than $\int _{\mathcal Rf ([0,1])} \phi(z) d\mu $, i.e. $\Phi'_a|_{\Phi^{-1}(z+1)} > \Phi'_a|_{\Phi^{-1}(z)}$ for $z\in \mathcal Rf[0,1]\setminus [0,1]$. Indeed, since we have 
%$\Psi(f^{q_n}(z)) = \Psi(z)-1,$
%$\Psi'_a|_{f^{q_n(z)}} + \Psi' (f^{q_n})'_a = \Psi'_a$, and thus $\Psi'_a|_{f^{q_n(z)}} < \Psi'_a|_z$ for any $z$. We conclude that 
%$$\int _{\mathcal Rf[0,1]} \Xi|_{(\mathcal Rf)^{-1}(z)} d\mu_{\mathcal Rf} > 
%\int _{[0,1]} \psi(z) d\mu_{\mathcal Rf}- \int_{\mathcal Rf [0,1]}  (\mathcal Rf)'|_{(\mathcal Rf)^{-1}(z)}  \phi({(\mathcal Rf)^{-1}(z)}) =0.$$
%This completes the proof.
\end{proof}




%
% We get the following corollary:
%\begin{corollary}
%\label{cor-deriv-estim}
% For any critical map $f$ with irrational rotation number, for any $x\in I_n$, we have $$(f^{q_{n+1}-l})' (f^l(x)) > c(f) \frac{M_n}{  |J|} \text{ if } 0<l<q_{n+1}$$  and $$ (f^{q_{n}-l})' (f^l(x)) > c(f) \frac{M_n}{  |J|} \text{ if } 0<l<q_{n}$$ where $J=f^l (I_n) = [f^{l+q_n}(0), f^{l}(0)]$ and $c(f)$ does not depend on $n$.
%
%These bounds are  universal on the invariant horseshoe: for $f\in \mathcal I$, we can replace $c(f)$ by a universal constant $c$.
%
%\end{corollary}
%\begin{proof}
%Let $0<l<q_{n+1}$, and apply Theorem  \ref{th-realbounds}(b) for $i= l, j = q_{n+1}$.
%We get $(f^{q_{n+1}-l})' (f^l(x)) > \frac{1}{C_0} \frac{|f^{q_{n+1}}(I_n) |}{| f^l (I_n) |}$ for $x\in I_n$.
%
%The denominator is $|J|$.
%Since the interval $f^{q_{n+1}}(I_n)$ covers $I_{n+1}$, commensurability of adjacent intervals $I_n$ and $I_{n+1}$ of the partition $\mathcal P_n$  (Theorem \ref{th-realbounds}(a))  implies  $|f^{q_{n+1}}(I_n)|> M_n/C_0$.  We conclude that $(f^{q_{n+1}-l})' (f^l(0)) > \frac{1}{C_0^2} \frac{M_n }{| J|}$.
%
%Similarly, let $0<l<q_n$ and apply  Theorem  \ref{th-realbounds}(b) for $i= l, j = q_{n}$. We get $(f^{q_{n}-l})' (f^l(x)) > \frac{1}{C_0} \frac{|f^{q_{n}}(I_n) |}{| f^l (I_n) |}$ for $x\in I_n$. The denominator is $|J|$. The interval $f^{q_{n}}(I_n)$ belongs to $\mathcal P_n$, thus it is commensurable with the adjacent interval $I_{n}$. We get    $|f^{q_{n}}(I_n)|>M_n/C_0$, and the statement follows.
%
%\end{proof}




Now we will estimate the value of the linear functional $L_{\mathcal R_{cyl}f}$ on $d\mathcal R_{cyl} v$ for the unit vector field $v=1$, thus the expansion rate of the renormalization operator.
\begin{theorem}
\label{th-dR-estim}
 For any map $f\in \mathcal I$, we have an estimate
 \begin{equation}
\label{eq-dR-estim}
 L_{\crenn f}( d|_f\crenn \, 1 ) \ge \frac{c}{|J_{n}|}
 \end{equation}
where $c$ is a universal constant,  and $J_n$ is the shortest interval of the partition $\mathcal P_n$ that has the form $J_n = f^l(I_n)$, $0<l<q_{n+1}$.
 \end{theorem}
\begin{proof}
 Lemma \ref{lem-mu-zero} implies that $ L_{\mathcal R_{cyl, n}f} (\Xi)=0$; it remains to estimate the integral of the last summand in \eqref{eq-iter}. Since the interval  $ [f^{q_{n+1}+q_n}(0), f^{q_{n+1}}(0)]$ covers $I_{n+1} $ and is contained in $I_n\cup I_{n+1}$, Theorem \ref{th-realbounds}(a) implies that it is commensurable with $I_n$. The map $\Psi$ takes this interval to the inverval $[\crenn f(0)-1, \mathcal R_{cyl}f(0)] $ of length one; since $f(0)\neq 0$ on $\overline {\mathcal I}$, the point  $\crenn f(0)-1$ is detached from $0$ and $-1$, and Lemma \ref{lem-Psi-estim} implies that the distortion of the map $\Psi$ on $[f^{q_{n+1}+q_n}(0), f^{q_{n+1}}(0)]$ is uniformly bounded for $f\in \mathcal I$.  Thus $$| \Psi'|_{P\Psi^{-1}(x)}| >\frac c{M_n},$$ where $M_n=|I_n|$. It remains to prove the following:

 \begin{proposition}
 \label{prop-Pa}
In the family $f_a=f+a$, we have   $ P'_a(y) > c\frac{M_n}{|J_n|}$ for $y\in  [f^{q_{n+1}+q_n}(0), f^{q_{n+1}}(0)]$.
 \end{proposition}
% \begin{proposition}
% \label{prop-Pa}
%We have   $ P'_a(y) > c\frac{M_n}{|J_{n+1}|}$ for any $y = \Psi^{-1}(x)\in [0, f^{-q_n}]$ such that $P(y) = f^{q_{n}+q_{n+1}}(y)$.
% \end{proposition}
 \begin{proof}

 If $P(y) = f^{q_{n+1}}(y)$, then $y\in [f^{q_{n}+q_{n+1}}(0), 0]\subset I_n$. Hence  $f^l(y)\in f^l(I_{n})$ and
 $$(f^{q_{n+1}-l})' (f^l(y)) >c \frac{|[f^{q_{n+1}+q_n}(0), f^{q_{n+1}}(0)]|}{|f^l(I_n)|}$$
 due to Theorem \ref{th-realbounds}b. The denominator is $|J_n|$, the numerator is commensurable with $M_n$ as explained above. Thus $(f^{q_{n+1}-l})' (f^l(y)) >c\frac{M_n}{|J_n|}$, and since other summands of the sum \eqref{eq-sum1} are positive, the statement follows.

 If $P(y) = f^{q_n+q_{n+1}}(y)$, then $y\in [0, f^{q_{n+1}}(0)]$. We have  $f^{q_n}(y)\in I_n$, thus  $f^{q_n+l}(y)\in f^l(I_{n})$ and $$(f^{q_{n+1}-l})' (f^{l+q_n}(y)) >c\frac{[f^{q_{n+1}+q_n}(0), f^{q_{n+1}}(0)]|}{|f^l(I_n)|}> \tilde c\frac{M_n}{|J_n|}.$$ Since other summands of the sum \eqref{eq-sum2} are positive, the statement follows.

%
%  Since $y\in [0, f^{-q_n}(0)]$, we have $z = f^{q_n}(y) \in I_n$.    Due to the fact that $f^{q_{n+1}-l}$ has bounded distortion on $f^l(I_n)$ (Theorem \ref{th-realbounds} (b)), we get $$(f^{q_{n+1}-l})' (f^{l+q_n}(y)) = (f^{q_{n+1}-l})' (f^l(z)) > c \frac{|f^{q_{n+1}}(I_n)|}{|f^l(I_n)|}.$$
% The denominator is $|J_n|$. Note that $I_n, I_{n+1}$ are adjacent intervals in $\mathcal P_n$, and thus commensurable (Theorem \ref{th-realbounds} (a));  since $f^{q_{n+1}}(I_n)\supset I_{n+1}$, this derivative is at least $c\frac{M_n}{|J_n|}.$
%
%  The term $(f^{q_{n+1}-l})' (f^{l+q_n}(y)) $ belongs to the sum \eqref{eq-sum2}, and since all its summands are positive, the proposition is proved in the case when $P(y)=f^{q_{n+1}+q_n}(y)$.
%
%  Suppose that $P(y) = f^{q_{n+1}}(y)$.
%  Note that in \eqref{}, $(f^{q_{n+1}-l})' (f^{l+q_n}(y))$
%
%
%
%  Consider three cases.
%
%  \textbf{Case 1.} Suppose that $l \le q_{n+1}-q_n $. Distortion estimate on $f^{q_{n+1}-q_n-l}$ on $f^l(I_n)$ (Theorem \ref{th-realbounds} (b)) implies that  $$(f^{q_{n+1}-q_n-l})' (f^{l+q_n}(y)) = (f^{q_{n+1}-q_n-l})' (f^l(z)) > c \frac{|f^{q_{n+1}-q_n}(I_n)|}{|J_n|}.$$ Since the interval  $f^{q_{n+1}-q_n}(I_n)$ is adjacent to $I_{n+1}$ in the partition $\mathcal P_n$, and $I_{n+1}$ is in turn adjacent to $I_n$, the numerator is commensurable with $M_n$. We have estimated from below one of the summands of the sum \eqref{eq-sum1}, and since all summands are positive, the statement follows.
%
% \textbf{Case 2.} Suppose that $l>q_{n+1}-q_n$, $l\le 2q_{n+1}-2q_n$. Then $\mathcal P_n$ contains adjacent intervals $J_n = f^l(I_n),$  $J_n^1 = f^{l+q_n-q_{n+1}}(I_{n+1})$, and $J_n^2 = f^{l+q_n-q_{n+1}}(I_{n})$. We have   $f^{q_n}(y)\in I_n$, thus  the point $z=f^{l+2q_n-q_{n+1}}(y)$ belongs to $J_n^2$, and distortion estimates imply  $$ (f^{2q_{n+1}-2q_n-l})' (f^{l+2q_n-q_{n+1}}(y)) = (f^{2q_{n+1}-2q_n-l})' (z)  > c \frac{|f^{q_{n+1}-q_n}(I_{n})|}{|J_{n}^2|}.$$ The numerator is commensurable with $M_n$ as in Case 1. Since  $J_n, J_n^1, J_n^2$ are adjacent intervals in $\mathcal P_n$, they are commensurable, and the statement follows.
%
% \textbf{Case 3.} Suppose that  $l> 2q_{n+1}-2q_n$. Since $l<q_{n+1}$, this is only possible if $q_{n+1}<2q_n$ i.e. $a_{n+1}=1,$ $q_{n+1}=q_n+q_{n-1}$.
%
% If $y\in I_{n+1}$, then
%
% If $y \in f^[q_{n+1}-q_n}(I_n) $, then

 \end{proof}
% 
% \begin{proposition}
% \label{prop-Pa2}
%  If $\rot f=\rho = [a_1, \dots, ]$ and $a_{n+1}=1$, then we have  $ P'_a(y) > c\frac{M_n}{|J_n|}$ for all $y = \Psi^{-1}(x)\in I_n$.
% \end{proposition}
%\begin{proposition}
%\label{prop-meas}
%There exists a universal constant $c>0$ such that if $\rot f=\rho = [a_1, \dots, ]$ and $a_{n+1}>1$, then $$\mu_{\mathcal R_{cyl} f} (\mathcal R_{cyl}f (\Psi (\{y\in I_n  \mid P(y) = f^{q_{n+1}}(y)\}))>c.$$
%\end{proposition}
%If $a_{n+1}=1$, the required estimate now follows from Proposition \ref{prop-Pa2}: we get  $\Psi'|_{P\Psi^{-1}(x)} P'_a|_{\Psi^{-1}(x)} > \frac{c}{|J_n|}$ for all $x\in [-1, 0]$, thus the value of $L_{\mathcal R_{cyl}f}$ on this expression is bounded below by $\frac{c}{|J_n|}$.
%
%If $a_{n+1}>2$, the required estimate follows from Propositions \ref{prop-Pa} and \ref{prop-meas}, since the value of the positive function $(\Psi'|_{P\Psi^{-1}(\mathcal R_{cyl}f)^{-1}(x)}P'_a|_{\Psi^{-1}(\mathcal R_{cyl}f)^{-1}(x)}$  is bounded below by $\frac{c}{|J_n|}$ on the set of measure at least $c$.
%
%It remains to prove the propositions.

%
% \begin{proof}[Proof of Proposition \ref{prop-Pa}]
%  Let the shortest interval $J_n$ be given by  $J_n = f^l(I_n)$.
% Since  $P(y) = f^{q_{n+1}}(y)$, we have that  $P'_a(y)$ given by \eqref{eq-sum1}.  Due to Corollary \ref{cor-deriv-estim},  we have $$(f^{q_{n+1}-l})'(f^l ( y))>c \frac{M_n}{|J_n|}. $$ Since other summands in the sum \eqref{eq-sum1} are positive, $P'_a (y)>  c \frac{M_n}{|J_n|}$.
% \end{proof}
%
% \begin{proof}[Proof of Proposition \ref{prop-Pa}]
%  Let the shortest interval $J_{n+1}$ be given by  $J_{n+1} = f^l(I_{n+1})$.
% Since  $P(y) = f^{q_{n}}(y)$, we have that  $P'_a(y)$ given by \eqref{eq-sum2}.  Due to Corollary \ref{cor-deriv-estim},  we have $$(f^{q_{n}-l})'(f^l ( y))>c \frac{M_n}{|J_n|}. $$ Since other summands in the sum \eqref{eq-sum1} are positive, $P'_a (y)>  c \frac{M_n}{|J_n|}$.
% \end{proof}
%
% \begin{proof}[Proof of Proposition \ref{prop-Pa2}]
% The estimate on $P'_a$ from Proposition \ref{prop-Pa} is valid if $y\in [f^{q_n}(0), f^{-q_{n+1}}(0)]$. Suppose that $y\in [f^{-q_{n+1}}(0), 0]$. Since $a_{n+1}=1$, we have  $q_{n+2}=q_n+q_{n+1}$. Then  $y\in I_{n+2}$ or $y\in f^{q_n}(I_{n+1})$.
%
% If $y\in I_{n+2}$, apply the second statement of Lemma \ref{lem-intervals} to $n+2$: we get $(f^{q_{n+2}-l})' (f^l(y))\ge c \frac{M_{n+2}}{|f^l(I_{n+2})|}$.  The denominator is smaller than $|f^l(I_n)|=|J_n|$ since $I_{n+2}\subset I_n$, and the numerator is commensurable with $M_n$ since $I_n$, $I_{n+1}$ are adjacent in $\mathcal P_n$ and $I_{n+1}, I_{n+2}$ are adjacent in $\mathcal P_{n+1}$.
% We conclude that $$(f^{q_{n}+q_{n+1}-l})' (f^l(y)) = (f^{q_{n+2}-l})' (f^l(y))  \ge c \frac{M_{n}}{|J_n|},$$ and since other summands of \eqref{eq-sum2} are positive, the statement follows
%
% If $y\in f^{q_n}(I_{n+1})$, apply the first statement of Lemma  \ref{lem-intervals} to $n+1$ and $\tilde J = f^{l+q_n}(I_{n+1})$. We get  $(f^{q_{n+2}-l})' (f^l(y))\ge c \frac{M_{n+1}}{|\tilde J|}$.
% Again, $M_{n+1}$ is commensurable with $M_n$; since $\tilde J$ is adjacent to $f^l(I_{n+2})$ in the partition $\mathcal P_{n+1}$ and thus commensurable with it, we get the same estimate as in the previous case.
% This completes the proof.
% \end{proof}
%
% \begin{proof}[Proof of Proposition \ref{prop-meas}]
% Since  $\rho=[a_1, a_2, \dots]$ with $a_{n+1}\ge 2$, we have $\rot \mathcal R_{cyl} f \in (0, 1/2)$.
% Let us prove that there exists a uniform constant $c>0$ such that for any $f\in \mathcal I$ with $\rot f\in (0, 1/2)$, for any $x\in \bbR/\bbZ$, we have $\mu_f([x, f(x)])<1-c$. Indeed, apply \eqref{eq-measure} with a continuous $\phi$ that approximates $\chi_{(x, f(x))}$. Since intervals $(x, f(x))$ and $(f(x), f^2(x))$ do not intersect, we get $\mu_f((x, f(x))) = \int_{(f(x), f^2(x))} f'|_{f^{-1}(t)} d\mu < (1-\mu_f((x,f(x)))) \cdot C$ where $C =\sup_{f\in \mathcal I, z\in \bbR/\bbZ} f'(z)$ is finite due to compactness of $\overline{\mathcal I}$. This implies the statement.
%
% In particular, the measure $\mu_{\mathcal R_{cyl} f}$ of the interval $(\mathcal R_{cyl} f)  \Psi([f^{-q_{n+1}}(0), 0])$ is at most $1-c$, and the measure of its complement is at least $c$. Since $\{y \in I_n \mid P(y) = f^{q_{n+1}}(y)\} = [f^{-q_{n+1}}(0), 0]$, this implies the statement.
% \end{proof}

%   \begin{itemize}
%   \item Case 2A: Suppose that $P(y) = f^{q_{n+1}+q_n}(y)$ and $l\le q_n$.
%   \end{itemize}
% Since $y\in I_n$ and $f^{q_{n+1}}(y) \notin I_n$, we have $z= f^{q_{n+1}}(y) \in I_{n-1}$. Apply Corollary \ref{cor-deriv-estim} to $n-1$ and the point $z\in I_{n-1}$: we get $$(f^{q_n-l})'(f^{l+q_{n+1}} ( y)) = (f^{q_n-l})'(f^l ( z))>c \frac{M_{n-1}}{|J|}, $$ where $J=f^l(I_{n-1})$. Note that $J, J_n \in \mathcal P_{n-1}$ are adjacent intervals and  $I_n, I_{n-1}\in \mathcal P_{n-1}$ are also adjacent. Commensurability of these intervals (Theorem \ref{th-realbounds}(b)) implies $(f^{q_n-l})'(f^{l+q_{n+1}} ( y)) >\tilde c \frac{M_n}{|J_n|} $. Since this is one of the summands of \eqref{eq-sum2} and all summands are positive, we have  $P'_a(y) > \tilde c \frac{M_n}{|J_n|} $.
% \begin{itemize}
% \item Case 2B: Suppose that $P(y) = f^{q_{n+1}+q_n}(y)$ and $l> q_n$.
% \end{itemize}
% Apply Corollary \ref{cor-deriv-estim} for $J=f^{-q_n}(J_n)=f^{l-q_n}(I_n)$. Since $f^{l-q_n}(y)\in J$, we get $$(f^{q_{n+1}-(l-q_n)})'(f^{l-q_{n}} ( y)) >c \frac{M_n}{|J|}.$$ Since $J$ is an interval of the partition $\mathcal P_n$ adjacent to $J_n$, these two intervals are commensurable, which implies  $(f^{q_{n+1}+q_n-l})'(f^{l-q_{n}} ( y)) > c\frac{M_n}{|J_n|}$, and thus  $P'_a(y)>c\frac{M_n}{|J_n|}$.



The proof of Theorem~\ref{th-dR-estim} is thus completed.
\end{proof}



%
% To estimate the first summand, we will use the renormalization operator in the space of pairs and its hyperbolicity.
%
% Following \cite{Ya1}, we consider the space of real analytic cubic critical commuting pairs $(\eta, \xi)$ of interval homeomorphisms that satisfy the following:
% \begin{itemize}
%  \item $\eta, \xi$ are real analytic orientation-preserving interval maps $\eta
% \colon  I_\eta \to  \eta(I_\eta), \xi \colon I_\xi \to \xi(I_\xi)$,
% \item  $I_\eta = [0, \xi(0)], I_\xi = [\eta(0), 0]$;
% \item  Both $\eta$ and $\xi$ have homeomorphic $C^3$-smooth extensions to interval neighborhoods of their respective domains  which commute, $\eta \circ \xi = \xi \circ \eta$;
% \item $\xi \circ \eta(0)\in I_\eta$;
% \item $\eta'(x)  \neq 0,  \xi'(y)\neq 0$, for all $x \in I_\eta \setminus {0}$, and all $y \in I_\xi \setminus {0}$;
% \item $\xi, \eta$ can be  decomposed near zero as $\eta = h_1 \circ Q \circ H_1$, and $\xi = h_2\circ Q \circ H_2$, where $h_{1,2}, H_{1,2}$ are real analytic
% diffeomorphisms and $Q(x) = x^3$.
% %\item $\xi$ analytically extends to an interval (a, b) ∋ 0 with ξ(a, b) ⊃ [η(0), ξ(0)], and has a single critical point 0 in this interval.
% \end{itemize}
%
% Consider the pair $(\eta^r \circ \xi, \eta) $ where $r$ is chosen so that $0\in [\eta^r \xi(0), \eta^{r+1}\xi(0)]$. The renormalization $\mathcal R_{pairs}(\eta, \xi)$ is a pair $(\eta^r \circ \xi, \eta) $ restricted to intervals $I_\xi,[0,  \eta^r \xi(0)]$ respectively  and rescaled by $1/|I_\xi|$.
%
% Renormalization operator leaves invariant the set of Epstein pairs $\mathcal E$, see \cite{Ya2} for the definition of the Epstein class. The attractor of the renormalization operator $\mathcal R_{pairs}$ described by the following result belongs to $\mathcal E$.
%
% \begin{theorem}[\cite{Ya2}, Theorem A]
% \label{th-renorm-pairs}
%  There exists an $\mathcal R_{pairs}$-invariant set $\tilde \Lambda$ of commuting pairs with irrational rotation numbers with the following properties.
%
%  The action of $\mathcal R_{pairs}$ on $\tilde \Lambda$ is bijective. Moreover, there
% is a one-to-one correspondence
% $i \colon \tilde \Lambda \to \Sigma$
% such that if $\zeta = i^{-1}(. . . , r_{-k}, . . . , r_{-1}, r_0, r_1, . . . , r_k, . . . )$, then $\rho(\zeta) = [r_0, r_1, . . . , r_k, . . . ]$ and
% thus the action of $\mathcal R_{pairs}$ on $\tilde \Lambda$ is conjugate to the shift:
% $i \circ \mathcal R_{pairs} \circ i^{-1} = \sigma$.
%
%
% The set $\tilde \Lambda$ is pre-compact in Caratheodory topology, its closure $\mathcal A$
% is the attractor for the renormalization operator:
% $\mathcal R_{pairs}^n (\zeta) \to \mathcal A$, for all $\zeta \in \mathcal E$ with $\rho(\zeta) \in \bbR\setminus\bbQ$,
% where the convergence is understood in the sense of Caratheodory topology.
%
% More precisely, for any pair $\zeta' \in \mathcal A$ with $\rho(\zeta) = \rho(\zeta')$ we have
% $dist(\mathcal R_{pairs}^n \zeta, \mathcal R_{pairs}^n \zeta') \to 0$
% for the $C^0$-distance between the analytic extensions of the renormalized pairs on an open neighborhood of the origin.
% \end{theorem}
% Since renormalizations of $C^3$-smooth commuting pairs converge to the Epstein class \cite{dF1}, the last statement is true for pairs outside the Epstein class.
%
%  Let $\tilde \Lambda_{\le K}$ be the  part of $\mathcal A$ that corresponds to sequences $\{a_n\}\in \Sigma$ with $a_1\le K$. Due to Lemma 7.6 of \cite{Ya3}, for each pair from $\mathcal E$ of finite height, thus for each pair  $(\eta, \xi) \in \tilde \Lambda_{\le K}$,  the map $\eta$ has a pair of complex conjugate repelling fixed points $p^\pm$ that admit the fundamental crescent $C$ bounded by curves $\gamma, \eta(\gamma)$ that join $p^-$ to $p^+$. Let  $\Psi_\eta$ be the uniformizing chart for $C/\eta$.
%
%  In the next lemma and below, we denote by $d_v\Psi_f$ the derivative of the perturbed Douady coordinate $\Psi$ along the vector field $v$, i.e. its derivative with respect to $t$ in the family $f+tv$.  Also, $B_\eps(V)$ is an $\eps$-neighborhood of a set $V$. Compactness of $\tilde \Lambda_{\le K}$ implies the following.
%
% \begin{lemma}
%  \label{lem-pairs-estim}
% For each $K$,  there exists a neighborhood $U_{\le K}$ of $\tilde \Lambda_{\le K}$ in the Caratheodory topology and a constant $c=c(K)$ such that for any pair $\zeta=(\eta, \xi) \in U_{\le K}$, the map $\eta$ admits a fundamental crescent, and the corresponding uniformizing map $\Psi_\eta$ satisfies the following estimates:  $\frac 1c < (\Psi_\eta)'<c$ on $\Psi_{\eta}^{-1}([-1, 1])$ and $$|d_v \Psi_\eta|  \le c |v|_{C(B_\eps([-1,0])}$$ on $\Psi_{\eta}^{-1}([-1, 1])$.
%  \end{lemma}
% \begin{proof}
%  Compactness arguments imply that for some $\eps$, for all $(\eta, \xi)\in \tilde \Lambda_{\le K}$, the map $\eta$ analytically extends to an $\eps$-neighborhood of $[-1,0]$ and the values $|\Psi_\eta|$ are uniformly bounded on an $\eps$-neighborhood of $\Psi_{\eta}^{-1}([-1, 1])\times [-\eps, \eps]$.  For the component $\eta$ of each pair  $(\eta, \xi)\in \tilde \Lambda_{\le K}$, we consider its $\delta$-neighborhood (in the space of analytic maps on $\eps$-neighborhoods of $[-1,0]$) where this chart survives and is still uniformly bounded on $\Psi_{\eta}^{-1}([-1, 1])\times[-\eps, \eps]$. Then in a $\delta/2$-neighborhood of $\eta$, we have   $c<(\Psi_\eta)'<C$, and for any vector field $v$,  $d_v \Psi_\eta  \le c(\eta) |v|_{C(B_\eps([-1,0]))}$ due to Cauchy estimates.   Finally, we find an open  subcover of $\tilde \Lambda_{\le K}$. We get a neighborhood $U_{\le K}$ of $\tilde \Lambda_{\le K}$ with uniform estimates on $\Psi'_{\eta}$ and $d_v\Psi_\eta$ as required.
%  \end{proof}
%
%
%
% This implies the following.
%
% \begin{lemma}
% \label{lem-Psi-estim2}
% Fix $K\in \bbN$. There exists a constant $c=c(K)>0$ such that if for a critical circle map $f$, for some $n$, the pair  $\mathcal R_{pairs}^n (z+1, f) $ belongs to the neighborhood $U_{\le K}$ from the previous lemma, we have the following estimates for the corresponding perturbed Douady coordinate $\Psi$ that conjugates $f^{q_n}$ to $z\mapsto z-1$:
% \begin{enumerate}
% \item $\Psi'>c/M_n$  in $\Psi^{-1}[-1, 1]$ and
%
% \item  for any positive vector field $v$ on a neighborhood of $I_n$, we have  $$ d_v \Psi_f> - \frac{c}{M_n} \|v\|_{B_{\eps/M_n} (I_n)}$$ on $\bbR/\bbZ$.
% \end{enumerate}
% \end{lemma}
% \begin{proof}
% Recall that $\Psi$ is a uniformization of a cylinder $C/f^{q_n}$: $\Psi f_a^{q_n}(z)  = \Psi(z)-1$, and $\Psi(0)=0$.
%  Represent $\Psi$ as a composition of the linear map $L\colon z\to z/M_n$ that takes  $[f^{q_n}(0), 0]$ to the interval $[-1, 0]$ and the uniformization $\tilde \Psi$ of a cylinder $L(C/   ( Lf^{q_n}L^{-1} )) $. Thus
%  $$\Psi' = \tilde \Psi' L' = \frac 1M \tilde \Psi',$$ and
%  in any family $f_a$,
%  $$
% \frac{d}{da} \Psi =  \frac{d}{da} (\tilde \Psi \circ L) = \frac{d}{da} \tilde \Psi + \tilde \Psi' \cdot \frac{d}{da} L.
%  $$
% The map  $\tilde \Psi$ uniformizes  the cylinder $L(C/   ( Lf^{q_n}L^{-1} )) $. Note that this  cylinder  is a fundamental cylinder for a map $\eta = Lf^{q_n}L^{-1} $. This map is the first component of the commuting pair $\zeta = (\eta, \xi) =\mathcal R^n_{pairs} (z+1, f)$.
%
% Since the pair  $\zeta=\mathcal R^n_{pairs} (z+1, f)$ belongs to $U_{\le K}$, its perturbed Douady coordinate $\tilde \Psi$ is estimated in Lemma \ref{lem-pairs-estim}.  We get that $\tilde \Psi'$ is positive and uniformly bounded (above and below), and $$\tilde \Psi'_a \le c |Lv|_{C(B_{\eps}(-1, 0))} \le \frac{c}{M_n} |v|_{C(B_{\eps/M_n}I_n)}.$$
%
%  It remains to prove that $\frac{d}{da} L$ is positive on the real axis for any positive vector field. This will imply the lower bound on $\Psi'_a$.
%
% Since $Lz = z/M_n $, the derivative  $L'_a$ equals $-(M_n)'_a / (M_n)^2$. In turn, $(M_n)'_a = \frac d{da} ( -f^{q_{n}}(0))<0$. Thus $L'_a$ is positive.
%
% \end{proof}
%
%
% \begin{lemma}
% \label{lem-lowtype}
%  Suppose that $f\in \mathcal I$ and  $\rot f=\rho=[a_1, a_2,\dots]$. For any $K$,  there exists a constant $c(K)$ such that for sufficiently large $n$, we have the following: if $a_n\le K$, then  $$|d|_f \mathcal R^n_{cyl} \, 1|> \frac{c(K)}{M_{s}^3}$$ where $s$ is a number such that  $\mathcal R^n$ is a first-return map to $ I_s$.
% \end{lemma}
% \begin{proof}
% Construct a neighborhood $U_{\le K}$ as in Lemma \ref{lem-pairs-estim}.
%  Let $n$ be sufficiently large so that $f\in \mathcal I$, $a_n\le K$ implies $\mathcal R_{pairs}^{n} f\in U_{\le K}$; this is possible due to Theorem \ref{th-renorm-pairs} and compactness of $\mathcal I$.
%
% Let $z=0$ and $v(z)=1$ in \eqref{eq-iter} and let us estimate $(d\mathcal R_{cyl}^{n}|_f v)(0)$ from below. Since $0$ is a critical point of the map $\Psi f^{q_{n}} \Psi^{-1}$, we have  $(\Psi f^{q_{n}} \Psi^{-1})'(0)=0$, thus the second summand in \eqref{eq-iter} is zero when $z=0$.
%
% Applying Lemma \ref{lem-Psi-estim2}, we get that  the first summand  in \eqref{eq-iter} is bounded from below by  $-\frac {c(K)}{M_{s}}$.
%
%
% Finally, the sum contains  a summand
% $$\Psi' \cdot f^{q_{n+1}-1}(f(0)) v(0) \ge \frac{c(K)}{M_{s}} \cdot \frac{c}{M_{s}^2} v(0) = \frac{c_1(K)}{M_{s}^3}.$$
% Here we estimate $\Psi'$ using Lemma \ref{lem-Psi-estim}, and the estimate on $(f^{q_{n+1}-1})'$ is due to Lemma \ref{lem-deriv-estim} (b). All summands of the sum are positive. We conclude that $$(d|_f\mathcal R_{cyl}^{n}) v(0) \ge \frac {c_1 (K)}{M_{s}^3}  - \frac{c_2(K)}{M_{s}} .$$
%
%  Since Lemma \ref{lem-intervals} provides us with the exponential estimate on $M_s^{-1}$, the statement follows.
%
% \end{proof}
%
%
%
%
% To prove uniform expansion  of the renormalization operator for rotation numbers of unbounded type, we will need a new way to estimate $\Psi'_a $.
%
% Let $\tilde \Lambda_{\infty}$ be the subset of $\mathcal A$ given by $a_1=\infty$. Let  $\tilde \Lambda_{>N}\supset \tilde \Lambda_{\infty}$ be the subset of $\mathcal A$ that corresponds to the set of rotation numbers with $a_1>N$. The first components of the pairs from $\Lambda_{\infty}$ are  quadratic parabolic maps (see Lemma 2.13 from \cite{Ya2}). Let $P_N$ be the image of the projection of $\Lambda_{>N}$ to the parabolic maps: each pair $(\eta, \xi) $ is mapped to the parabolic map $\eta(x) - \min_{I_\eta} |\eta(x)-x|$.  Note that $P_N$ is located in a small neighborhood of  $\Lambda_{\infty}$, and thus only contains quadratic parabolic maps. Also, compactness of $\mathcal A$ implies that  maps from $ P_N$, rescaled to maps of $[0,1]$, form a sequentially compact set (with Caratheodori convergence on compact sets).
% \begin{lemma}
% \label{lem-parab-neigh}
%  For each $Q$ there exists $K$ such that  for some $\eps$, any critical pair $\zeta\in \Lambda_{>K}$ satisfies the following. Let  $\zeta =(\eta, \xi)$. Set $m = \min_{I_{\eta}} (\eta(x)-x)$.
%
%  (a) There exist  at least $Q$ intervals of the form $[\eta^k(0), \eta^{k+1}(0)]$ with $0< k<1/\rot \zeta$ that have length no more than $4m$.
%
%  (b) For some universal constants $c,C$, we have $0<c<\Psi'_\eta<C$  in $[\eta(0), \eta^{-1}(0)]$ and  for any positive vector field $v$, $$ |(d (\Psi_{\eta} )v |< \frac{c}{m} \|v\|_{\Pi_{\eps}}$$ on $\bbR/\bbZ$.
% \end{lemma}
% \begin{proof}
% (a) Consider the projection $f$ of $\zeta$ onto parabolic maps defined by $f=\eta - \min_{I_\eta}|\eta(x)-x|$. Let  $p$ be its parabolic point.
% For each $\delta$, find $l$ such that $f'<1+\delta$ inside $[p-l, p+l]$. Due to compactness of $P_Q$, we can choose $l$ uniformly on $f\in \mathcal P_Q$.  Since $\eta$ projects to $f$, we have $\eta'=f'$.
%
% Also, if $\eta$ is close to $f$, then $\min |\eta(x)-x|$ is attained inside this interval $[p-l, p+l]$; due to compactness of $\mathcal P_Q$, the required distance between $\eta$ and $f$ can be chosen uniformly on $f\in P_Q$, i.e. the above estimates will hold for any $\eta\in \Lambda_{>M}$ for large $M$. Finally,   the interval $[p-l, p+l]$ will contain the interval of length $\min_{I_{\eta}} (\eta(x)-x)$ that has the form $[\eta(x),x]$ and thus at least  $\log_{1+\delta} 2$ its images under $\eta$  of the  form $[\eta^{l}(x), \eta^{l+1}(x)]$ will have length at most $2m$. Since each interval of the form $[\eta^k(0), \eta^{k+1}(0)]$ belongs to the union of subsequent intervals of this form, the estimate follows.
%
% (b) Note that for each pair $\zeta = (\eta, \nu)$ in the closure of the horseshoe $\tilde \Lambda$ with $a_1\neq \infty$, the perturbed Douady coordinate $\Psi$ is defined, bounded on a certain neighborhood of $[]$, and has bounded distortion, and extends analytically to a certain neighborhood of $\nu$.  Cauchy estimates imply that we can cover this pair $\zeta $ with a neighborhood where  $d\Psi (v)<c(\zeta)< c(\zeta)/m$. It remains to cover points with $a_1=\infty$ by such neighborhoods and to extract a finite subcover using compactness.  From now on, we only consider pairs in $\tilde \Lambda_{>Q}$ with sufficiently large $Q$.
%
%  Recall that perturbed Douady coordinates of $\zeta\in \Lambda_{>Q}$ tend to repelling Douady coordinates of parabolic pairs from $\Lambda_{\infty}$, and the latter set is compact. This implies the  uniform estimate on $\Psi_\eta$.
%
% Consider the pair $(\eta,\xi)\in \Lambda_{>Q}$ which is projected to a parabolic map $f\in P_Q$; recall that  $\eta=f+m$.
% Let us prove that the perturbed Douady coordinate  $\Psi=\Psi_h$ is defined for any holomorphic map $h$ (not necessarily preserving the real axis) that is $m/2$ close to $\eta$, i.e. $\sup_{I_\eta \times [-\eps, \eps]} |h-\eta|<m/2$. Since there is a uniform estimate on $\Psi_h |_{\Psi_h^{-1}[-1-\eps, 1+\eps] \times [-\eps, \eps]}$, Cauchy estimates imply the required estimate on $d_v \Psi_{\eta}$.
%
% Indeed, for a parabolic map  $f\in P_Q$, fix $\eps$ so that  the repelling Douady coordinate is defined in a neighborhood of  ${\Psi_f^{-1}[-1-2\eps, 1+2\eps] \times [-2\eps, 2\eps]}$. Due to compactness, we can choose $\eps$ uniformly.
%
% For each $f\in P_Q$, let $p$ be its parabolic point; for $h\sim f$, let $A_1, A_2$ be the fixed points of $h$ that bifurcate from $p$.
% Let  $\alpha=\frac{1}{2\pi i} \log h'(A_2)$.
% Due to \cite[Theorem 2.1]{Inou-Shi},  there exists a neighborhood of $f$ such that for any $h$ in this neighborhood, if $|\arg (-\alpha)|<\pi/4$,  then the perturbed Douady coordinate $\Psi_h$ for $h$ is defined  in  ${\Psi_f^{-1}[-1-\eps, 1+\eps]\times [-\eps, \eps]}$; moreover, $\Psi_h$ tends to $\Psi_f$  on this domain as $h\to f$.   Since we have $A_1-A_2 = -4\pi i \alpha / f''(p)+o(\alpha)$, the condition on $\alpha$ will be satisfied if the fixed points $A_1, A_2$  satisfy $\arg (A_1-A_2) -\pi/2<\pi/5$. Due to compactness, the size of the required neighborhood of $f$ can be chosen uniformly on $f\in P_Q$.
%
% It remains to prove that if $h$ is in the $m/2$-neighborhood of $\eta=f+m$ and $m$ is small, then the  fixed points of $h$ satisfy $\arg (A_1-A_2) -\pi/2<\pi/5$. Indeed, let $h-f-m=\xi$, $\xi(z)<m/2$ on a neighborhood of $\bbR/\bbZ$. Then  $\arg(\xi(z)+m)<\pi/6 $. Also,   inside the  angles $|\arg z| <3\pi/10$ and $|\arg z-\pi/2|<3\pi/10$, we have $\arg (f(z)-z) = \arg(az^2+\dots ) < 6\pi/10$ for small $z$. Thus $f(z)+m+h(z)-z$ is nonzero in these angles and the fixed points $A_1, A_2$ must belong to their complement $|\arg z -\pi/2|<\pi/5$ if $h\approx f$. Thus implies the estimate for small $m$. Due to compactness of $P_Q$, we can choose the estimate on $m$ uniformly on $f\in P_Q$. \end{proof}
%
% Let $J_n$ be the shortest interval of the partition $\mathcal P_n$.
%
% \begin{lemma}
% \label{lem-htype1}
%  Suppose that $f\in \mathcal I$ and  $\rot f=\rho=[a_1, a_2,\dots]$. For sufficiently large $K$, for sufficiently large $n$, we have the following: if $a_n>K$, then for $v(z)=1$ we have $$|d|_f \mathcal R^n_{cyl} v|>\frac{c}{|J_s|}$$ where $c$ is a universal constant and $s$ is a number such that  $\mathcal R^n$ is a first-return map to $ I_s$.
% \end{lemma}
% \begin{proof}
%  For some $Q$ to be chosen later, we will find $K$ using Lemma \ref{lem-parab-neigh} and note that the statement of Lemma \ref{lem-parab-neigh} survives in a small neighborhood $V$ of $\Lambda_{>K}$. Let $n$ be sufficiently large so that $a_n>K$ implies $\mathcal R_{pairs}^{n-1} f\in V$; this is possible due to Theorem \ref{th-renorm-pairs}.
%
% For any $n$,  let $m_n$ be the length of the shortest interval of the form $f^k(I_l)$ that belongs to  $I_{n-1}$, for $k\le q_{n+1}$.
%
% Let $v=1, z=0$ in \eqref{eq-iter}. Thus the two statements of Lemma \ref{lem-parab-neigh} imply the following.
%
%
% (a) Let $m_n$ be the length of the shortest interval of the form $f^k(I_l)$ in the partition $\mathcal P_n$ that belongs to  $I_{n-1}$. Note that the restriction of the partition $\mathcal P_{n}$ to $I_{n-1}$ is a linear rescaling of the partition of $[0,1]$ by intervals $ [\eta^k(0), \eta^{k+1}(0)]$.   The first statement of Lemma \ref{lem-parab-neigh} implies that the partition $\mathcal P_n$ includes at least $Q$ intervals in $I_{n-1}$ that are shorter than $4 m_n$.
%
% (b)  Similarly to Lemma \ref{lem-Psi-estim2},  the Douady chart $\Psi$ is a composition of a linear map $L\colon z\to z/M_{n-1}$ and the Douady chart $\tilde \Psi$ for the critical pair $(\eta,\xi)=\mathcal R_{pairs}^{n-1}(f, z+1)$; We have $d_a\Psi v = d_a\tilde \Psi (Lv) + d_a L $. The second summand is positive, due to the same reason as in Lemma \ref{lem-Psi-estim2}. For the first summand,  $d_a\tilde \Psi'_a (Lv) < \frac{c}{ m} \cdot |Lv|$ due to Lemma \ref{lem-parab-neigh}.
%
% Since the restriction of the partition $\mathcal P_{n}$ to $I_{n-1}$ is a linear rescaling of the partition of $[0,1]$ by intervals $ [\eta^k(0), \eta^{k+1}(0)]$, we have $m = m_n/M_n$.
% Taking into account that $L$ multiplies the norm of the vector field by $1/M_n$, we get $\Psi'_a > -c/m_n$.
%
% Also, $\Psi'=1/M_n \cdot \tilde \Psi'>c/M_n$.
%
%
% Now we are ready to estimate $\eqref{eq-iter}$.  The second summand in \eqref{eq-iter} is zero for $z=0$. The first summand is at least $ -c/m_n$ due to our estimate on $\Psi'_a$.
%
% Lemma \ref{lem-deriv-estim} implies that each interval of length $4m_n$ adds a term to the sum in  \eqref{eq-iter} that is at least  $(f^{q_{n+1}-l})' (f^l(0)) > c M_n / 4m_n$. We will have at least  $Q$ such intervals, thus at least $Q$ summands of this form. Also, the shortest interval in the partition $\mathcal P_n$ adds a term that is at least $c M_n / |J_n|$.
% Since $\Psi'>c/M_n$, we get  $$| d|_f\mathcal R_{cyl}^n v  |> (Q-1) \frac{c_1}{4m_n} + \frac{c_2}{|J_n|}.$$ We use $(Q-1) $ instead of  $Q$ to account for  the case if $J_n$ itself belongs to $I_{n-1}$ and is counted as one of the $Q$ intervals that are shorter than $4m_n$.
%
%
%  Since $c_1, c_2$ are universal constants, we may and will choose $Q$ in the beginning of the proof to guarantee that the sum of the first two summands is positive. This completes the proof.
% \end{proof}
%
%
%
%The previous lemma shows that $d\mathcal R_{cyl}^n = d\mathcal R_{cyl, n}$ uniformly expands at an exponential rate, since $|J_n|$ exponentially decrease (Lemma \ref{lem-intervals}).
Now, we are ready to complete the proof of uniform hyperbolicity of $\mathcal R_{cyl}$ by providing an unstable invariant cone field.

% \begin{lemma}
% \label{lem-expansion}
% For any $\lambda>1$, there exists a choice of constant $N$ in the definition of  $\mathcal R_{cyl, N}$ such that $|d\mathcal R_{cyl, N}^n| > \lambda^n$ at any point of the invariant horseshoe $\mathcal I$.
% \end{lemma}
% \begin{proof}
% For any fixed $\lambda>1$ choose $K$ and large  $n$ such that if $f\in \mathcal I$ satisfies $\rot f = [a_1, \dots]$ with  $a_m\ge K$ for some $m>n$, then Lemma \ref{lem-htype1} implies $|d|_f \mathcal R_{cyl, m}^n \, 1|>\lambda^n$.
%
% Using this $K$, fix a neighbborhood $U_{\le K}$ of $\tilde \Lambda_{\le K}$ as in Lemma \ref{lem-pairs-estim} and choose  $N>n$ such that Lemma \ref{lem-lowtype} implies   $|d|_f \mathcal R_{cyl, N}^n \, 1|>\lambda^n$ whenever  $a_N\le K$.
%
% Finally, for any $f\in \mathcal I$, Lemma \ref{lem-lowtype} will apply if $a_{N}\le K$  and Lemma \ref{lem-htype1} will apply if $a_{N}>K$. The estimate on $|d|_f \mathcal R_{cyl, N} |$ follows.
% \end{proof}

%\begin{lemma}
%For some $c$, for sufficiently large $n$, we have $|\mathcal R_{cyl, n} 1 - L_{\mathcal R^n f} (dR^n 1)| < c  |L_{\mathcal R^n f} (dR^n 1)|$. 
%\end{lemma}
%\begin{proof}
%Let us denote $s_k =  L_{\mathcal R^k f} v \colon  v $ and  $w_k = \mathcal R_{cyl, k} 1 - L_{\mathcal R_{cyl, k} f} (dR_{cyl,k} 1) $; not that $w_k$ belong to the stable distribution.  Then the right-hand side is $s_1\dots s_n$.  Represent the left-hand side as the sum $d\mathcal R_{cyl, n-1} w_1 + d\mathcal R_{cyl, n-2} s_1w_2 \cdot  + \mathcal R_{cyl, n-3} s_1s_2w_3+\dots +  s_1s_2\dots s_{n-1}w_n$. Since $\mathcal R_{cyl}$ contracts on the stable distribution and $w_k$ are bounded, we get  $|\mathcal R_{cyl, n} 1 - L_{\mathcal R^n f} (dR^n 1)| < C (\tau^{n-1} + \tau^{n-2}s_1 + \dots + \tau s_1\dots s_{n-2} + s_1\dots s_{n-1})$.  
%
%WHY  may and will choose $N$ so that $\tau/s_n < 0.5$
%\end{proof}

\begin{theorem}
\label{th-cones}
 There exists a choice of $N$ in $\crenN\equiv\cren$ and a universal constant $c<1$  such that the cone field $$\mathcal C_f=\{v\in \mathcal D^{cr}_\eps \mid
\, |L_f(v) |>c\|v -  L_f(v) \|\}$$ defined on a neighborhood of $\mathcal I$ is invariant under $d\cren$ and vectors in the cones are uniformly expanding under $d\cren$.
\end{theorem}
Since linear functionals $L_f$ depend continuously on $f$  in weak topology, this cone field is continuous.
\begin{proof}
Choose $\lambda>1$.

Note that in Theorem \ref{th-dR-estim}, the interval $J_n$ is the smallest out of $q_{n+1}-2$ non-intersecting intervals, thus  $|J_n|<\frac 1{q_{n+1}-2}$. The denominators $q_{n}$ increase at a uniform exponential rate, thus we can choose $N$ so that $$\|d|_f\crenN \, 1\|>c\frac{1}{|J_n|}>\lambda\text{ for all  }f\in \mathcal I.$$


Increase $N$ if needed so that  for all $f\in \mathcal I$, the operator $d|_f\crenN$ uniformly contracts on a stable distribution: for any $w\in T \mathcal D^{cr}_{\eps}$ with $L_f(w)=0$, we have  $\|d_f\crenN w\|< \tau \|w\| $ for a universal  $\tau<1$. This is possible due to \cite[Theorem 6.4]{Ya4}.

From now on, we will write $\cren$ instead of $\crenN$ and $d\cren$ instead of $d|_f \cren$ for shortness.

 Note that we have $$L_{\cren f} (d\cren\xi)  = (d\cren \, 1) \cdot  L_f (\xi)$$  for any vector field $\xi\in T|_f \mathcal D^{cr}_{\eps}$, since this holds both for $\xi=1$ and for any $\xi\in \mathrm{ Ker}\,  L_f$.


Take $v\in T_f \mathcal D^{cr}_{\eps}$, and let  $k= L_f(v)$, then the vector field $v -  k=w$ belongs to the stable distribution, $L_f(w) =0$. Suppose that $v$ belongs to the cone  $\mathcal C_f$, i.e.  $c\|w\| <|k|$.
 Let us prove that $d\cren v$ belongs to the cone $\mathcal C_{\cren f}$ if $c$ was properly chosen. Indeed, $$|L_{\cren  f} (d\cren  v)| = \|d\cren  \, 1\| \cdot |L_f (v)| = \|d\cren  \, 1\| \cdot |k| $$ and $$\|d\cren v \|  = \|d\cren (w+k)\|\le\\ \tau \|w\| + |k|\cdot  \|d\cren  \, 1\|, $$ thus
 \begin{multline*}
\|d\cren v -L_{\cren  f} d\cren  v \|  \le\tau \|w\| + 2|k|\cdot \|d\cren  \, 1\|  \le \tau \frac {|k|}c + 2|k|\cdot \|d\cren  \, 1\|.
 \end{multline*}

It remains to find $c$ so that  $$\|d\cren  \, 1\| \cdot |k|> \tau |k| +2 c |k|\cdot  \|d\cren  \, 1\|.$$
 Since  $\|d\cren  \, 1\|>\lambda>1>\tau$, the inequality holds for small universal $c$.


 Now, let us show that the vectors in the cones are uniformly expanded. We use Theorem~\ref{th-dR-estim} again to choose $n$ so that
 $$\|d\cren^n \, 1\| = \|d{\mathcal R}_{\text{cyl}, nN} \, 1\|\gg 1+1/c\text{ for all }f\in \mathcal I.$$  Then for any $v\in \mathcal C_f$, using the representation $v=w+k$ again, we have $$\|d\cren ^n v\|> |k|\cdot  \|d\cren ^n \, 1\| - \tau \|w\|> |k|(  \|d\cren ^n\, 1\| -\tau/c)$$ and $\|v\|< |k|+\|w\| < |k| (1+1/c)$, thus $\|d\cren ^n v\| > C \|v\|$ for $C>1$, which implies uniform expansion.
\end{proof}




Standard techniques now imply uniform hyperbolicity of $\cren $:
\begin{itemize}
\item  the unstable distribution is constructed as an intersection of images of unstable cones;
\item This intersection may only be generated by one vector at each point  due to contraction in the transversal direction, and the resulting distribution  is unstable since it belongs to unstable cones;
\item the distribution depends continuously on a point since this is true for images of cones under $d\cren ^k$ for each $k$;
\item uniform transversality of stable and unstable distribution follows from the fact that the unstable vector $v$ at each point belongs to the unstable cone:  $|L_f(v) |>c\|v -  L_f(v) \|$, so if $v$ is normalized by $\|v\|=1$, then $L_f(v)$ is bounded away from zero.
\end{itemize}

Below we obtain finer estimates on the minimal expansion rate along the unstable direction of $\cren $, as required for Proposition \ref{prop-exponents} and Theorem \ref{th-smoothness-htype2}.
The next lemma shows that the partition $\mathcal P_n$ contains an interval $f(I_n)$ of length $\sim M_n^3$, which provides an upper estimate on the length of the shortest interval in $\mathcal P_n$.

\begin{lemma}
\label{lem-intervals2}
 For any critical map $f$ with irrational rotation number,  $|f(I_n)| <  c(f) M_n^{3}$ where $c(f)$ does not depend on $n$.

This bound is  universal on the invariant horseshoe.
\end{lemma}
\begin{proof} Since $|I_n|\to 0$ with $n$ and any cubic critical map has the Taylor series expansion $c_0+c_1x^3+\dots $ at zero,
for any critical map $f$  we have $|f(I_n)| < c(f) |I_n|^3 = c(f) M_n^3$ where $c(f)$ does not depend on $n$.

On the invariant horseshoe $\mathcal I$, there is a uniform estimate on $c(f)$ due to compactness of $\overline {\mathcal I}$.

\end{proof}



 Now we are ready to provide finer estimates on the expansion rates of the renormalization operator. Recall that $\delta\in(0,1)$ is a constant that satisfies   $M_n<C\delta^n$ for any $f\in \mathcal I$, see  Theorem \ref{th-intervals}.
 
 

Let $M_{m, n}$ be the length of the interval $[0, g^{{q_n}}(0)]$ that corresponds to the map $g=\mathcal R_{cyl, N}^m f$.

\begin{lemma}
\label{lem-Lyap-lowtype}
  For any irrational number  $\rho$, if $f\in \mathcal I$ has rotation number $\rho$, then  the minimal expansion rate of $\mathcal R_{cyl, N}$ over its unstable distribution $\eta_f$ along the orbit of $f$  is bounded below: $$\lambda_1\ge \left(\liminf_{n\to\infty} \inf_{m\ge 0} \sqrt[n]{\frac 1 {M_{m,n}^3}}\right)^N.$$ In particular,  $$\lambda_1\ge  \delta^{-3N}.$$ 
\end{lemma}
\begin{proof}
The minimal expansion rate can be computed using any vector in the unstable cones, since the difference of such vectors (properly rescaled) belongs to the stable distribution. Thus $$\lambda_1 = \liminf_{n\to \infty} \inf_{m\ge 0}\sqrt[n]{\|d \cren^n|_{\cren^m f} 1\|}.$$   Let $g=\cren^m f$. This norm is estimated below by the value of the linear functional $L_{\cren^n  g}$ on the vector field $d \cren^n \, 1$, since $L_{\cren g}$ is given by the integral with respect to the probability measure. Due to Corollary \ref{lem-intervals2}, we have $|J_n| \le c|M_n|^3$, thus Theorem \ref{th-dR-estim} implies  $$\|d\cren^n|_g \,1\| > \frac{c}{M_{m, Nn}^3}\text{ for all }n.$$ This implies the first statement.

Due to Theorem \ref{th-intervals}, we have $M_{m,n}< C\delta^{n}$. This implies the second statement.
\end{proof}

\begin{remark}[Periodic rotation numbers]
\label{rem-periodic}
It is easy to prove that for any periodic orbit of $\cren $, there exists a limit $\lim_{n\to\infty} \sqrt[n]{M_{m,n}} $ that is uniform with respect to $m$. Thus  $\liminf_{n\to\infty}\inf_m$ can be replaced with the limit in the previous estimate.

Due to Lemma \ref{lem-distributions}, the maximal expansion rate along the second unstable direction of $\mathcal R$ is $$\lambda_2 = \limsup_{n\to \infty} \sup_{m\ge 0} \sqrt[n]{(\Psi'_{m, n}(0))^2}$$ where $\Psi_{m,k}$ is the Douady coordinate for the $n$-th renormalization of $g=\cren^m f$. The estimate on the distortion of $\Psi$ from Lemma \ref{lem-Psi-estim} implies that $$\lambda_2 \le \left( \limsup_{n\to\infty} \sup_{m\ge 0} \sqrt[n]{M_{m,nN}^{-2}}\right)^N.$$
Again, for a periodic orbit we can replace $\limsup$ by a limit. So the inequality  $\lambda_1>\lambda_2$ in Proposition \ref{prop-exponents} will be satisfied for any periodic point of $\cren $.

 However, the previous estimates only guarantee the fraction of logarithms $\log \lambda_1/\log\lambda_2 =  3/2<2$, and we cannot conclude using Theorem \ref{th-smoothness} that the corresponding Arnold tongues are more than $C^1$ smooth. This agrees with the result of \cite{LlaveLuque}: for some periodic rotation numbers, numerical experiments showed that Arnold tongues are less than $C^2$ smooth.
\end{remark}



\begin{lemma}
\label{lem-Lyap-htype}
Let $\rho = [a_1, a_2, \dots].$  For any $A$, $\delta\ge 0$, for any  sufficiently large $K$,   if $\rho=[a_1, a_2,\dots]$ is such that $$ \quad \quad \frac{\#\{k \mid a_k<K, m\le k<m+n\}}{n}\le \delta $$
 for all $m$ and for all sufficiently large $n$, then for any $f\in \mathcal I$ with rotation number $\rho$, the minimal expansion rate of $\mathcal R_{cyl, N}$ over its unstable distribution $\eta_f$ along the orbit of $f$ is  $\lambda_1>A^N$.
\end{lemma}
\begin{proof}
Again, the minimal expansion rate can be computed using any vector in the unstable cones; we will use $v=1$.
Fix $A$, $\delta$.

% Let $n_k\to \infty$ be given by $a_{n_k}>K$.
% Due to Lemma \ref{lem-htype1}, for sufficiently large $n$,$K$, we have  $|d\mathcal R^{n_k} \, 1| > \frac{c}{|J_{n_k \cdot N}| }$ whenever $a_s>K$;  here   $J_s$ is the shortest interval of the form $f^l(I_s) $, $l<q_{s+1}$, on the circle.


By increasing $K$, we can guarantee that for any irrational $\rho$ and any large $n$, if at most the $\delta$-proportion of terms  $a_j$, $m\le j\le m+n N$, are smaller than $K$, then the denominators $q_{nN}$ of the continued fractional expansion of $G^m(\rho)$ grow quicklier than any given geometric progression. In particular, for any large $n$ and sufficiently large $K$, we have $q_{nN+1}> A^{nN}/c+2$ where $c$ is the same as in Theorem \ref{th-dR-estim}. Let $g=\cren^m f$, and apply Theorem \ref{th-dR-estim} to $g$. Since $J_{nN}$ is the shortest out of $q_{nN+1}-2$ non-intersecting intervals $g^l(I_{nN})$, $0<l<q_{nN+1}$, on the circle, we have  $|J_{nN}|<1/(q_{nN+1}-2) < c/A^{nN} $.
Thus for sufficiently large $n$, for all $m$, we have $\|d\cren^n|_g \, 1\| >  A^{nN}$ due to Theorem \ref{th-dR-estim}.

%It remains to obtain a lower estimate for any $n$.
%Since $\cren $ expands on the unstable cones and $v=1$ belongs to these cones, we have  $|d_f\mathcal R^{n}_{cyl} \, 1| > |d_f\mathcal R^{n_k}_{cyl} \, 1| > ( (2A)^{\frac 1{1-\delta}})^{n_k}$ where $n_k$ is the largest number with $a_{n_k}>K$ and $n_k<n$. The estimate on the proportion of terms with  $a_n<K$ implies that $n_k>n-n\delta $. Thus $|d_f\mathcal R^{n}_{cyl}\, 1| > |d_f\mathcal R^{n_k}_{cyl} v| >  (2A)^n$.


This implies the required estimate on the minimal expansion rate.


\end{proof}

Due to Lemma \ref{lem-distributions}, the minimal expansion rate along the unstable distribution of $\mathcal R|_{D_0}$ coincides with that of $\cren $, while the maximal expansion rate along the other unstable direction of $\mathcal R$ is bounded by the maximal value of  $(\Psi'_N(0))^{2}$ over $\overline{ \mathcal I}$. Thus Lemma \ref{lem-Lyap-htype} implies Proposition \ref{prop-exponents}
%%%(for $\delta=0$) 
and Theorem \ref{th-smoothness-htype2}.
