
\section{Cylinder renormalization.}
\label{sec-cylren}
\subsection{Cylinder renormalization for cubic critical circle maps}
\label{sec-crit}
We are going to briefly recall the cylinder renormalization operator 
$\cren$ on $\mathcal D^{cr}_\eps$ constructed in \cite{Ya3} by the second author.
%%Here we describe the explicit construction of $\mathcal R_{cyl}$ and formulate the results on its hyperbolicity.
As before, $p_n/q_n$ will stand for the continued fractional convergents of $\rot f$.

\subsection*{Construction of the cylinder renormalization}

% The renormalizable part of the space $\mathcal D^{cr}$ is divided into countably many open domains $S_n$; let $n=n(f)$ be the number of the domain that contains $f$. $\mathcal R_{cyl}$ will be analytic in $S_n$, the union $\mathcal U=\cap S_n$ is the open set in Theorem \ref{}.

In \cite{Ya3}, \cite{Ya4}, the second author showed that  for any $f\in \dcrepsr$ there exists $N(f)$ such that for all $n\geq N(f)$, the iterate $f^{q_n}$ has two complex conjugate repelling fixed points joined by a crescent-like fundamental domain $C_f$ for the iterate $f^{q_n}$. The iterate $f^{q_n}$ glues the boundary curves of the crescent $C_f$ (minus the two repelling points at the ends) into a Riemann surface whose conformal type is $\CC/\ZZ$. 
%$N\in\NN$ and constructed a horseshoe-like set $\mathcal I \subset \mathcal D^{cr}_\eps$  such that for some its neighborhood $\mathcal U \subset \mathcal D^{cr}_\eps$, for any $f\in \mathcal U$ with $\rot f\sim p_n/q_n$, the map  $f^{q_N}$ has two complex conjugate hyperbolic repellors joined by a crescent-like fundamental domain $C$ of $f^{q_n}$.
The number $N(f)$ can be chosen uniformly on bounded subsets of $\dcrepsr$. Furthermore, if a critical circle map $f$ has a fundamental crescent $C_f$ for an iterate $f^{q_n}$, as above, then the same is true for non circle-preserving analytic maps in an open neighborhood of $f$ in  $\dcreps$.

Let us fix a bounded open set $\cU^\RR\subset\dcrepsr$ on which  $N=N(f)$ can be chosen uniformly. 
The fundamental crescent $C_f$ for the iterate $f^{q_N}$ can be chosen to move holomorphically with $f$ in an open neighborhood $\cU\supset \cU^\RR$ in the space $\dcreps$ of non circle-preserving maps, and to contain $0$ in its interior. There is a unique real-symmetric uniformizing coordinate $\Psi_f$ which sends the quotient  $C_f/f^{q_N}$ to $\CC/\ZZ$ with $\Psi_f(0)=0$. 
%For any $f\in \mathcal U$, consider a crescent $C$ that joins hyperbolic fixed points of $f^{q_N}$ and contains zero. The quotient space $C/f^{q_N}$ is  a cylinder; it is normalized by $\Psi \colon \Pi \to \bbC/\bbZ$.
The map $\Psi_f$  depends analytically on $f\in \mathcal U$, and conjugates $f^{q_N}$ to the shift $z\to z+1$ (for odd $n$) and to the shift $z\to z-1$ (for even $n$)  in $C \cup f^{q_N}(C)$. Note that when $f\in\cU^\RR$ and the $N$-th digit of the continued fraction of $f$ is sufficiently large, the map $f^{q_N}$ is near-parabolic, and we can think of $\Psi_f$ as the perturbed Fatou coordinate of $f^{q_N}$ (see e.g. \cite{Sh}).

Let $P$ be the first-return map to $C$ under the action of $f$, defined in a neighborhood of $C\cap \bbR/\bbZ$.
By definition,
$$\cren f := \Psi_f P \Psi_f^{-1}.$$
%where a post-composition with a shift is added to $\Psi$ if necessary to guarantee that the critical point of $\mathcal R_{cyl} f$ is at zero.
%This map is restricted to the strip of width $\eps$ around the real axis.
% Clearly, the same holds in a small neighborhood $\mathcal U$ of $\mathcal I$, since $\Psi$ depends analytically on $f$.
%
%  We will use the following estimate.
% \begin{lemma}
% \label{lem-Psi-estim}
%  For any $c$ and sufficiently large $C$, one can choose $N$ such that for sufficiently small semi-neighborhood $\mathcal U$ of $\mathcal I$, $\Psi_f'(0)>c$ for $f\in \mathcal U$, $\Pi_\eps \subset \Psi_f (\Pi_\eps) $, and the distortion of $\Psi_f$ is bounded by $C$.
% \end{lemma}


% Note that the above construction does not provide an analytic operator on a neighborhood of the closure of $\mathcal I$. Indeed, the closure of $\mathcal I$ contains parabolic circle maps. The chart $\Psi_{cyl}$ is not analytic on $f$ in a neighborhood of such map.


\noindent
 We are ready to formulate \cite[Theorem 3.7]{Ya4}.
\begin{theorem}[\cite{Ya4}]
  \label{th-horseshoe}
  There exists a choice of an open bounded set $\cU\subset \dcreps$ as above and a corresponding choice of $N$ such that the following holds.
  With these choices, $\cren$ is an analytic operator $\cU\to \dcreps$, which is real-symmetric, and preserves $\dcrepsr$.  This operator has a hyperbolic invariant set $\cI\subset \dcrepsr\cap \cU$ with one-dimensional unstable direction.
  The set $\cI$ is pre-compact in the sense of uniform convergence. There is a map
$\iota\colon  {\mathcal I}\to \bbN^{\bbZ}$   that takes any $f\in {\mathcal I}$ to a bi-infinite sequence such that the positive part of this sequence coincides with the continued fractional expansion of $\rot f$, and $\iota$ conjugates $\cren|_{{\mathcal I}}$ to the $N$-th power of the Bernoulli shift.
\end{theorem}
The following theorem describes the stable foliation of this hyperbolic set.
\begin{theorem}{\cite[Theorem 3.8]{Ya4}}
  \label{th-convergence}
  If $f,g\in\dcrepsr$ and  $\rot f=\rot g$ is irrational, then
  $\dist( \cren^n f, \cren^n g) \to 0$
  at an eventually uniform geometric rate. Furthermore, all limit points of the sequence $\{\cren^n f\}$ lie in the closure of $\cI$.
\end{theorem}
An analogue of this result for the formalism of critical commuting pairs was earlier established by de Faria and de Melo \cite{dFdM2} under the assumption that the rotation number is of bounded type.
%
% The set $\mathcal I$ is not compact. The critical circle maps that belong to its closure are the ones with a parabolic periodic orbit of period $N$ on $\bbR/\bbZ$. In \cite{Ya4}, the second author considered a suitable Banach manifold with parabolic critical maps on its border, a ``blow-up'' of $\mathcal D^{cr}_\eps$ at parabolic critical points, and extended $\mathcal R_{cyl}$ to this manifold by continuity. The resulting renormalization operator has a hyperbolic set $\overline {\mathcal I}$, the closure of $\mathcal I$ in the new manifold; the dynamics on $\overline {\mathcal I}$ is conjugate to the shift by $N$ on the space of bi-infinite sequences of symbols $\{0,1,\dots, \infty\}.$ \fixme{looks like we don't need this result}


% A pair of $C^k$-smooth interval maps $(\nu, \xi)$ of the intervals $I_{\nu} = [0, \xi(0)], I_\xi = [0, \nu(0)]$ that extend to $C^k$-smooth commuting maps in neighborhoods of $I_\nu, I_\xi$ and satisfies $\xi(0)\subset I_\nu$ is called a commuting pair. We say that this is a quadratic critical commuting pair if $0$ is the only critical point of $\nu, \xi$ and this point is quadratic.
% Consider the set of irrational rotation numbers of bounded type:  $a_n < C$.
% The renormalization operator $\mathcal R$ is defined on the set of such pairs in the following way: $\mathcal R (\nu, \xi)$ is the linear rescaling of $(\xi, \nu^k\xi)$ with the factor $-I_{\nu}$,  where $k$ is such that $\nu^k \xi (I_\xi)$ contains $0$.

% Let $\mathcal E$ be the Epstein class: the space of inverval maps that extend to a three-fold cover of a disc onto a two-slit plane $\bbR \setminus (\bbR\setminus [-c, c])$ that contains this disc. Renormalizations of smooth critical maps converge to the Epstein class (Guerino, de Melo; Yampolsky, Gorbovickis).  In \cite{Ya}\footnote{M.Yampolsky, Renormalization horseshoe for critical circle maps} the following theorem was proved.
% \begin{theorem}
%  There exists an $\mathcal R_{cyl}$- forward invariant set $\mathcal I \subset\mathcal D^{cr}$ that consists of consisting of pairs with irrational rotation numbers with the following properties. The action of $\mathcal R$ on $\mathcal I$ is topologically conjugate via $i\colon \mathcal I \to \Sigma$ to the two-sided shift $\sigma$ on the space $\Sigma$ of infinite sequences with symbols $1,2,\dots, \infty$, and if $i(\zeta) = (\dots, r_{-1}, r_0, r_1, \dots)$, then $\rho(\zeta)=[r_0, r_1, \dots]$.  The set $\mathcal I$ is pre-compact, its closure $\mathcal A \subset \mathcal E$ is the horseshoe attractor for the renormalization. That is, for any $\zeta\subset \mathcal E$ with irrational rotation number we have $\mathcal R^n \zeta \to \mathcal A$ in Caratheodory topology. Moreover, for any two pairs $\zeta, \zeta'$ with $\rho (\zeta)=\rho(\zeta')$ we have $dist(\mathcal R^n \zeta, \mathcal R^n \zeta')\to 0$.
% \end{theorem}
% This theorem shows that the codimension-1 local stable foliation of $\mathcal I$ at any its point is the set $Q_\lambda\subset \mathcal D^{cr}$ of critical maps with same rotation number.
We will also use the following estimate. Here and below we assume that the crescent $f^{q_{N}}(C)$ contains $0$, and $\Psi_f(0)=0$.
 \begin{lemma}
 \label{lem-Psi-estim}
 For any $\eps, K$ and any sufficiently large $C$, one can choose $\cU$ and  $N$ as above so that
 for any $f\in \cU$,   we have  $\Psi_f'(0)>K$ and $\Pi_{K\eps} \subset \Psi_f (\Pi_\eps) $. 
 
 Also, for any $\delta>0$,  for sufficiently large $C$, the distortion of $\Psi_f$ is bounded by $C$ on the curvilinear rectangle  $\Psi^{-1}_f([a,a+1]\times[-i\eps, i\eps] )$ with $-1+\delta<a<\delta $: $$\frac 1C<\left|\frac{\Psi_f'(k)}{\Psi'_f(l)}\right|<C\text{ for }k, l\in [a, a+1]\times[-i\eps, i\eps]$$ and $$\left|\frac{\Psi''(k)}{\Psi'(k)}\right| < C\Psi'(0).$$
 \end{lemma}
The proof is by a reference to the Koebe Distortion Theorem and an upper estimate on the length of the interval $\Psi^{-1}[0,1]$, see \cite[p.20]{Ya4}.



%%\fixme{Please check the new statement.}

\subsection{Real {\it a priori} bounds}


Let $I_n = [ f^{q_n}(0), 0]$. Recall that the dynamical partition $\mathcal P_n$ of the map $f$ is formed by the intervals $$f^l(I_n), 0\le l<q_{n+1}\text{, and }f^l(I_{n+1}), 0\le l < q_n.$$ The intervals of $\cP_n$ cover the circle and have disjoint interiors.
For further reference, let us formulate a statement of real {\it a priori} bounds for smooth critical circle maps; see \cite{dFdM1} for the proofs.
%%will use real apriori bounds to estimate the lengths of the intervals in the partition $\mathcal P_n$.
 \begin{theorem}[Theorem 3.1 \cite{dFdM1} Parts  (a), (e)]
\label{th-realbounds}
There exists a universal constant $C_0>1$ such that for every $C^3$-smooth critical circle map $f$ with a single critical point $0$, there exists $N_1=N_1(f)$ such that for any $n\geq N_1$ the following holds.
\begin{enumerate}
\item[(a)]
  Any two adjacent intervals $I,J$ of the partition $\mathcal P_n$ are $C_0$-commensurable:
  $$1/{C_0}<|I|/|J|< C_0;$$

\item [(b)] For any $0<i\le j \le q_{n+1}$, the distortion of the restriction of $f^{j-i}$ to $f^{i}(I_n)$ is bounded by $C_0$.
\end{enumerate}
The number $N_1$ can be chosen uniformly on a $C^2$-precompact set of critical circle maps with irrational rotation numbers.
\end{theorem}
%%%It is a well-known fact that both bounds are universal on the invariant horseshoe $\mathcal I$. \fixme{add some reasonable explanation, E.g. dFdM1 Proposition 3.8 involves arguments for the case of negative Schwartzian.}



Real {\it a priori} bounds  imply the following geometric estimate on $|I_n|$:
\begin{theorem}[Theorem 3.1  \cite{dFdM1} Part (b)]
  \label{th-intervals}
  There exists $\delta\in(0,1)$ such that for any $C^3$-smooth critical circle map $f$ with an irrational rotation number $\rho$ there exists $N_2=N_2(f)$ such that for all $n\geq N_2$ all 
  %%%For each irrational rotation number $\rho$, there exists $\delta_\rho>0$ such that for any analytic cubic critical map $f$ with $\rot (f) = \rho$,
  intervals of the partition $\mathcal P_n$ are shorter than $\delta^n$.

  The number $N_2$ can be chosen uniformly on a $C^2$-precompact set of critical circle maps with irrational rotation numbers.
%%% These estimates are uniform on $\mathcal I$.
\end{theorem}

Let us note that since maps in $\cI$ are infinitely many times backward renormalizable, for all of them one can set $N_1=N_2=1$ in the above statements.
