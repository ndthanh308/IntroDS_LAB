\section{General statement on smoothness of stable-unstable manifolds and proof of  Theorem \ref{th-smoothness}}
\label{sec-stableunstable}

In this section, we formulate a general version of the theorem on the smoothness of invariant manifolds of hyperbolic operators with a two-dimensional unstable bundle, and use this result to prove Theorem \ref{th-smoothness}. The proof of Theorem \ref{th-Banach-op} will be given in Section \ref{sec-Banach-op}.

 
\begin{theorem}[Smoothness of stable-unstable  invariant manifolds for operators in Banach spaces]
\label{th-Banach-op}
 Consider a real Banach space $D$, a linear closed codimension-1 space $D_0$ and open half-spaces $D_{-}$ and $D_+$ such that $D = D_0 \cup D_- \cup D_+$. Consider a sequence of smooth operators $\{R_n\}$, $$R_n \colon D\to D\text{ with }R_n D_0\subset D_0;$$ choose a point $x_0$ and set $\{x_n\}$ to be its orbit, $x_{n+1} = R_n x_n$. Fix $k\in \bbN$ and for a  certain $\delta>0$, assume  that each $R_n$ is defined and $k$ times differentiable on a semi-neighborhood $B_\delta(x_n)\cap (D_0 \cup D_-)$ and has bounded derivatives up to the order $k$, with bounds independent on $n$.


 For  a splitting $D = \xi_0 + \eta_0 + l_0$ where $$\dim \xi_0 = \dim \eta_0=1,\; l_0+\eta_0 = D_0,$$ we  set $$\xi_{n+1} = dR_n|_{x_n} \xi_n, \; \eta_{n+1} = dR_n|_{x_n} \eta_n,\; l_{n+1} = dR_n|_{x_n} l_n.$$
 Suppose that $l_n$ is a stable distribution for $\{R_n\}$ and $\xi_n+\eta_n$ is its unstable distribution.
Suppose that $\xi_n$ is uniformly transversal to $D_0$,  the minimal expansion rate $\lambda_1$ of $ \{R_n\}$ along $\eta_n$ is larger than the  maximal expansion rate $\lambda_2$ along $\xi_n$, and moreover, $$k<  \frac {\log \lambda_1}{\log \lambda_2}.$$ 

Suppose that there exist codimension-1 local manifolds $M_n$ with border  that pass through $x_n$, such that  $(R_n M_n )\cap B_{\delta}(x_0)\subset  M_{n+1}$. Suppose that $M_n\cap D_0$ is a stable foliation for $R_n$, $T(M_n\cap D_0)=l_n$. Let   $M_n$ be given by a function $$Q_n \colon  (l_n+\xi_n) \to \eta_n;$$ suppose that $Q_n$  is well-defined with uniformly bounded derivative in $ (l_n+\xi_n) \cap (D_0\cup D_-)\cap B_\delta(x_n) $ where $\delta$ does not depend on $n$. Suppose that $Q_n $ is $k$-differentiable in $ (l_n+\xi_n)\cap D_- \cap B_\delta(x_n)$, and for any $\delta_1<\delta$, all its derivatives up to the order $k$ are uniformly (on $n$) bounded on $D_-\cap (l_n+\xi_n)\cap (B_\delta(D_0)\setminus B_{\delta_1}(D_0))$.

 Then $Q_n^{(l)}(x)$ with $l\le k $ has a limit as $x\to D_0$, thus $Q_n(x)$ is $k$ times continuously differentiable.
\end{theorem}

\subsection*{Assumptions of Theorem \ref{th-Banach-op} hold under the assumptions of  Theorem \ref{th-smoothness}}

Choose the space of triples $D_{\eps, \eps'}$ that satisfies Lemma \ref{lem-distributions}, and let $D$ be the space of $\bbR$-preserving triples $D^{\bbR}_{\eps, \eps'}$. Let $D_0 = \{a=0\}$, $D_- = \{a<0\}$, $D_+ = \{a>0\}$. Set $R_n=\mathcal R$. Let $x_0=\hat g$, take $\eta_n$, $\xi_n$ to be the unstable distributions $\eta_{\hat g_n}$ and $\xi_{\hat g_n}$ respectively. Take $l_n$ to be the stable distibution of $\mathcal R$ at $\mathcal R^n (\hat g) = \hat g_n$; we will also use the notation $l_{\hat g}$ for the stable distribution of $\mathcal R$ at $\hat g$. 

Since the orbit $\{\hat g_n\}$ belongs to a compact set $$\Lambda_c = \{f\in \Lambda, \rot f\text{ is of type bounded by }c\},$$ we conclude that an analytic operator $\mathcal R$ is defined and has uniformly bounded derivatives of each order in  $B_\delta(\hat g_n) \cap (D_0\cup D_-)$ for a small $\delta$.

The inequality on $\lambda_1, \lambda_2$ and the transversality condition is included in the assumptions of Theorem \ref{th-smoothness}. It remains to establish the required properties of the invariant manifolds $$M_n = \{\hat f \in D \mid \rot p(\hat f) = \rot p(\hat g_n) \} \cap B_\delta( \hat g_n)$$ for sufficiently small $\delta$.

Recall that we denote $\mathcal V_\alpha = \{\rot f=\alpha\}\subset \mathcal D_\eps$.
We will not restrict ourselves to the sequence $\hat g_n$,  we will rather establish the required properties for all the manifolds $\hat {\mathcal V}_{\alpha} = p^{-1}(\mathcal V_\alpha)\subset D $ with $\alpha$ of type bounded by $c$. Note that  $M_n=\hat {\mathcal V}_{\rot (p(\hat g_n))} \cap B_\delta(\hat g_n)$.

   Due to  Theorem \ref{th-C1}, the manifolds $\mathcal V_\alpha$ are smooth in $\mathcal D_\eps$ up to $\mathcal D_{\eps}^{cr}$. Thus $\hat {\mathcal V}_{\alpha}$  are smooth in $D$ up to $D_0$. Theorem \ref{th-C1} also  implies that the tangent space to  $\hat {\mathcal V}_\alpha \cap D_0$ at the critical triple $\hat g\in D_0$ is the preimage under $dp$ of the stable subspace of  $\mathcal R_{cyl}\colon \mathcal D_{\eps}^{cr}\to \mathcal D_{\eps}^{cr}$, thus coincides with the stable subspace of $\mathcal R \colon D\to D$. This implies that for each $\hat g\in \Lambda$, $T_{\hat g}\hat {\mathcal V}_{\alpha}$ is transversal  to the unstable direction $\eta_{\hat g}$ of $\mathcal R$. Thus, $\hat {\mathcal V}_{\alpha}\cap B_\delta(\hat g)$ can be represented as a graph $(x,Q_{\hat g}(x))$ of a smooth function $Q_{\hat g}\colon (l_{\hat g } + \xi_{\hat g})\to \eta_{\hat g}$ defined in some neighborhood of $0$. 

Let us prove that all $Q_{\hat g}$ are defined in a $\delta'$-semi-neighborhood of $\hat g\in \Lambda_c$ for a certain $\delta'$, and have uniformly bounded derivatives.  
 
Recall that for bounded-type $\alpha$,  $\hat {\mathcal V}_\alpha \cap D_-$ is an analytic codimension-1 manifold at each its point, due to Risler's theorem and analyticity of $p$. Also, $\hat {\mathcal V}_\alpha \cap D_0$ is an analytic codimension-1 manifold at each its point, namely a leaf of the stable foliation of $\mathcal R_{cyl}$ (see \cite[Theorem 3.8]{Ya4}). Thus the boundary of the domain of the corresponding function $Q_{\hat g}$ consists of points $x$ such that either $(x, Q_{\hat g}(x))\in \partial B_\delta(\hat g)$, or  $\|dQ_{\hat g}\|=\infty$. 
 The next lemma shows that $Q_{\hat g}$ have bounded derivatives on a certain small semi-neighborhood of $D_0$.  Thus $Q_{\hat g}$ are defined on one and the same neighborhood of $\hat g$.
 

\begin{lemma}
 For each $c$, there exists $\delta$ and $K$ such that  if  $\hat g\in \Lambda_c$, if  $$(x, Q_{\hat g}(x)) \subset B_\delta(\hat g),$$ then  $\|dQ_{\hat g}|_{x}\|<K$.
\end{lemma}
\begin{proof}
 Suppose that for some $K$, this statement fails for any $\delta$. Due to the compactness of $\Lambda_c$, there exists a sequence of critical triples $\hat g_{k} \to \hat g\in \Lambda_c$, non-critical triples $\hat h_k\to \hat g$ with the same rotation numbers $\rho_k$, and vector fields $v_k\in l_k+ \xi_k\subset T|_{\hat g_k}D$  such that $\|v_k\|=1$, and the differentials $dQ_{\hat g_k}$ at $\hat h_k$ satisfy $dQ_{\hat g_k} v_k=C_k \eta_k$ with $|C_k| \ge K$. Due to the definition of $Q_{\hat g_k}$, the vector field  $v_k + dQ_{\hat g_k} v_k$ is tangent to the manifold $\{\rot p(\hat f) = \rot p(\hat g_k)\}$ at $\hat h_k$.

 Let $g=p(\hat g), g_k=p(\hat g_k), h_k=p(\hat h_k)$ be the corresponding circle maps.
 Let $\mu_k$, $\tilde \mu_k$, and $\mu$  be the (-1)-measures for $h_k, g_k, g$.  Then   $\mu_k\to \mu$ and $\tilde \mu_k \to \mu$ weakly due to Theorems \ref{th-measure} and \ref{th-C1}.
 
 Since $v_k+dQ_{\rho_k} v_k$ is tangent to $\{\rot p(\hat f) = \text{const}\}$ at $\hat h_k$,  we have
 \begin{equation}
 \label{eq-v-1}
 \int (dp)v_k|_{h_k^{-1}} d\mu_k +\int (dp)(dQ_{\rho_k} v_k)|_{h_k^{-1}} d\mu_k  =0.
 \end{equation} 
 The first summand is uniformly bounded since $\|v_k\|=1$, $\|dp\| $ is uniformly bounded on a compact set $\Lambda_c$, and $\mu_k$ is a probability measure. Thus  $\int (dp)dQ_k v_k|_{h_k^{-1}} d\mu_k $ is uniformly bounded. On the other hand, $dQ_k v_k = C_k \eta_k$ with $|C_k|\ge K$, and $ (dp)\eta_k = \eta_{g_k}$ is the unstable direction for $\mathcal R_{cyl} $ at $g_k$.   Since $\eta_{g_k}$ is uniformly transversal to the stable distribution, the values of the linear functional $L_{g_k} $ on $\eta_k$ are bounded away from zero, i.e. $|\int \eta_k|_{g_k^{-1}} d\tilde \mu_k |$ is bounded away from zero.  Since $\tilde \mu_k\to\mu$, $\mu_k\to\mu$, this implies an upper bound on $K$. 
 
 Thus the statement of the lemma holds for sufficiently large $K$. 
\end{proof}
\begin{remark}
One can prove that the statement holds for any $K$, i.e. the first derivatives of $Q_{\hat f}$ uniformly tend to zero as $(x, Q_{\hat f}(x))\to D_0$. Indeed, due to Lemma \ref{lem-weakstable}, $l_{k}+\xi_{k}$ coincides with the subspace given by $\int (dp)v|_{g_k^{-1}} \tilde d\mu_{k}=0  $, and since $v_k\in l_k+\xi_k$ and both $g_k$ and $h_k$ tends to $g$, we have $\lim_{k\to \infty } \int (dp)v_k|_{h_k^{-1}} d\mu_k =  \lim_{k\to\infty} \int (dp)v_k|_{g_k^{-1}} d\tilde \mu_{k} =0$. Since  the first summand in \eqref{eq-v-1} tends to zero, the second summand also tends to zero,  and thus $|C_k|>K$ is not possible for any $K$. 
\end{remark}

For negative $a_1, a_0$, let the set $K_{a_0, a_1}$ of triples $(F, H_a, G)$ be given by $a_1< a<a_0$.
It remains to prove the uniform estimate on higher derivatives of $Q_n$ on the set $D_-\cap (l_n+\xi_n)\cap B_\delta(D_0)\setminus B_{\delta_1}(D_0)$, i.e. in $K_{a_0, a_1}$ with small $a_0, a_1$.

The following lemma completes the reduction.
\begin{lemma}
\label{lem-qQ}
For any $0>a_0>a_1$, the  functions $Q_n\colon (l_n+\xi_n)\to \bbR$ have uniformly bounded derivatives on $ (l_n+\xi_n)\cap K_{a_0, a_1}$.
\end{lemma}
\begin{proof}
Assume the contrary: suppose that derivatives $Q_k^{(s)}$ have unbounded norms at some points $\hat h_k$. Consider a larger space $D_{\tilde \eps, \tilde \eps'}$ of triples with slightly smaller domain of definition $\tilde \eps<\eps, \tilde \eps'<\eps'$. Clearly, the corresponding derivatives still have unbounded norms. Extracting a convergent subsequence from $\hat h_k$ in this new space,  we get the limit $\hat h=\lim_{k\to \infty} \hat h_{n_k}\in D_{\tilde \eps, \tilde \eps'} \cap K_{[a_0, a_1]}$. Note that  $\rot p(\hat h) =\lim \alpha_{n_k}$ has bounded type. Consider the splitting $T D = V+W$ where $V$ is the limit of $l_{n_k}+\xi_{n_k}$ and $W$ is the limit of $\eta_{n_k}$. Let $q_{n_k}\colon V\to W$ be  the functions whose graphs coincide with $M_{n_k}$; they also have unbounded derivatives of order $s$, since $l_{n_k}+\xi_{n_k} \to V$ and $\eta_{n_k}\to W$.


This contradicts Lemma \ref{lem-local-bound-deriv}, applied to the triple $\hat h\in D_{\tilde \eps, \tilde \eps'}$.  
\end{proof}


\subsection*{Assertions of Theorem \ref{th-smoothness} follow from the assertions of Theorem \ref{th-Banach-op}}

We will only use the fact that all vector fields $d  f_\mu/d\mu_2$ for $\mu_1=0$ are transversal to the surface $\{\rot f = \text{const}\}$. This clearly follows from the assumptions, since the positive vector field  $v=df_{\mu}/d\mu_2>0$ cannot satisfy $\int v|_{f^{-1}(z)}d \mu_f =0$ for a (-1) -measure $\mu_f$  and thus cannot be tangent to the surface $\{\rot f=\text{const}\}$.  

First, let us reduce the general statement to the case when the critical map  $f_0$ is close to $g\in \mathcal I$. Indeed, results of \cite{Ya3} imply that $\tilde f_\mu = \mathcal R_{cyl}^n f_\mu$ is well-defined and close to $\mathcal I$. In brief, let $\mathcal R_{\text{pairs}}$ be the renormalization of  commuting pairs. Then $\mathcal R_{\text{pairs}}^n(f_\mu)$ is a rescaled pair $(f^{q_n}_\mu, f^{q_{n+1}}_\mu)$. Due to \cite{Ya3}, for $\mu=0$, for large $n$, this pair is close to a commuting pair from the attractor of $\mathcal R_{\text{pairs}}$. In particular, $f^{q_n}_\mu$ with small $\mu$ admits a fundamental crescent that joins hyperbolic repelling points of $f^{q_n}_\mu$, and the corresponding linearizing chart $\Psi$ is defined. So $\tilde f_\mu = \cren^n f_\mu$ is well-defined and close to $\mathcal I$.
%%\fixme{I made another correction to have $\mathcal R_{pairs}$ only appear in the reference -- this way we have no obligation to define it. Misha, could you have a look again? }


Replacing $f_\mu$ with $\tilde f_\mu$, we may and will assume that $f_{\mu}$ is close to $g\in \mathcal I$ and thus $f_0$ belongs to the stable manifold  $W_g^s$ of $g$ for $\cren$. 

The transversality condition is preserved since $\cren$ preserves the foliation $\{\rot f =\text{const}\}$ in $D_0$ and has a transversal unstable direction. Also, the condition $\rot f_\mu = \rot g$ is equivalent to $\rot \tilde f_\mu = \rot \cren^n g$ for $\mu\approx 0$, hence we do not change the Arnold tongue when we replace $f_\mu$ with $\tilde f_\mu$. 

Lemma \ref{lem-lift} provides us with an analytic family  $\hat f_\mu\subset D$ such that $p(\hat f_\mu) = h_\mu f_\mu h_{\mu}^{-1}$; for $\mu_1=0$ (i.e. when $f_\mu$ is critical), we have $h_\mu=id$ and $\hat f_\mu=j(f_\mu)$. Note that  the vector field $d(h_\mu f_\mu h_\mu^{-1})/d\mu_2$ is  still transversal to the surface $\{\rot f=\alpha\}$ since the conjugacy is an invertible analytic operator that takes tangent spaces to these surfaces to tangent spaces, thus the condition of transversality is preserved.
Thus  $d \hat f_\mu/d\mu_2 \in TD$  is transversal to the surface given by $$\{\rot (p(F, H_a, G))=\alpha\},$$ since it projects under $d p$ to a vector  $d(h_\mu f_\mu h_\mu^{-1})/d\mu_2$ that is transversal to $\{\rot f=\alpha\}$.


%We may  assume that $\hat f$ belongs to a $\delta$-neighborhood of $\hat g$, where $\delta$ is provided by Theorem \ref{th-Banach-op}. 




%Now reduce the general statement of Theorem \ref{th-smoothness} to the particular case of  $f$ close to the invariant horseshoe $\mathcal I$, with the assumption that all vector fields $d (\tilde f_\mu)/(d\mu_2)$ are transversal to the surface $\{\rot f = const\}$.

Since $f_0\in W^s_g$ for $\cren$, the construction of the stable foliation of $\mathcal R$ (Lemma \ref{lem-distributions}) implies that $j(f_0)=\hat f_0$ belongs to the stable manifold for $\mathcal R$ that contains $\hat g$, thus  for any $\delta$, we can find $n$ such that  $\mathcal R^n \hat f_0$ is $\delta$-close to $\mathcal R^n \hat g\in \Lambda$.
  Note that   $d (\mathcal R^n \hat f_\mu)/d\mu_2 \in TD$  is still  transversal to the surface  $$\{\rot (p(F, H_a, G))=\rot p(\mathcal R^n \hat g)\}$$ for $\mu_1=0$ since $\mathcal R^n$ preserves transversality to its invariant manifolds.

%If the statement of Theorem \ref{th-smoothness}  holds for the family $\tilde f_a = \cren^n f_a$ that passes through the critical map $\mathcal R^n_{cyl} f$, then it also holds for the initial family, since the condition $\{\rot f_\mu=\alpha\}$ on $\mu$ is equivalent to the condition $\{\mathcal R^n f_\mu  = \alpha_n\}$.

%It is easy to verify that the new family $\mathcal R^n f_\mu$  satisfies all assumptions of Theorem \ref{th-smoothness} except $d (\mathcal R^n f_\mu)/(d\mu_2)>0$. Note that since $df_{\mu}/d\mu_2>0$, this vector field cannot be tangent to the surface $\{\rot f=const\}$ since it cannot satisfy $\int v(f^{-1})d\mu $ for a (-1)-measure $\mu$.   Hence the new vector field $d (\mathcal R^n f_\mu)/(d\mu_2) = d\mathcal R^n (df_{\mu}/d\mu_2)$ is not tangent to this surface as well.  This completes the reduction.



Theorem \ref{th-Banach-op} implies that the function $Q_0$ that defines the surface $\{(F, H_a, G) \in D \mid \rot p(F, H_a, G)=\rot p(\mathcal R^n \hat g)\}$ is $k$ times continuously differentiable at $\mathcal R^n (\hat f_\mu)$ for $\mu_1=0$ and small $\mu_2, \dots, \mu_n$. 

%The spaces $l_n, \eta_n$ project under $dp$ to the unstable and stable distributions $\tilde l_n, \tilde \eta_n$  of $\cren|_{D_0}$ at $f$; $\xi_n$ also projects to a certain one-dimensional distribution $\tilde \xi_n$. We conclude that the function $q_0$ that defines a surface $\{\rot f=\alpha\}$ in $\tilde \xi +\tilde l_n, \tilde\nu_n$ coordinates is also $k$ times continuously differentiable.


The Implicit Function Theorem, applied to $Q_0$ and $\mu \mapsto (\mathcal R^n \hat f_\mu)$, implies that the function $\mu_2(\mu_1, \mu_3\dots, \mu_n)$ defined implicitly by $$\{\mu\mid \rot p(\mathcal R^n \hat f_\mu)=\rot p(\mathcal R^n \hat g)\}$$ is $k$ times continuously differentiable at any point $(0,\mu_3, \dots, \mu_n)$ sufficiently close to zero. Since $\rot p(\mathcal R^n \hat f_\mu)=\rot p(\mathcal R^n \hat g)$ is equivalent to  $\rot f_\mu = \rot p(\hat f_\mu)=\rot (\hat g)=\alpha$ in a small neighborhood of $\mu=0$, this function defines the Arnold $\alpha$-tongue in the initial family $f_\mu$, which completes the proof.



