\section{Proof of Theorem \ref{th-Banach-op} on the smoothness of stable-unstable manifolds.}
\label{sec-Banach-op}
\begin{proof}[Proof of Theorem \ref{th-Banach-op}]
  $\;$

  \smallskip
  \noindent
\textbf{Step 1. Coordinates on $TD$ and notation.}

We are going to make an appropriate choice of the vectors  $\tl\xi_n$, $\tl\eta_n$ which generate the direction fields $ \xi_n, \eta_n$. 
Namely, for a small value $\eps>0$ to be fixed later (in Lemma \ref{lem-DnBn}), we choose $\tl\eta_n$ so that  $$d |_{x_n} R_n \tl\eta_{n} =k \tl\eta_{n+1}\text{ with }|k|\ge \lambda_1-\eps$$
 and  $\|\tl \eta_n\|$ are bounded away from zero and infinity.  Hence the ratio of the standard norm and the coordinate norm on the one-dimensional subspace $\eta_n$ that corresponds to the basis $\tl\eta_n$ is bounded. 
 This choice is possible due to the fact that  $\lambda_1$ is the minimal expansion rate along $\eta_n$.  

Set $L_n = \xi_n  + l_n $. Similarly, we  will choose vectors  $\tl\xi_n$ that generate the spaces $\xi_n$, and the norms on $l_n$ uniformly equivalent to the standard norm,  so that $$\|(d|_{x_n} R_n)|_{L_n}\| \le \lambda_2+\eps$$ with respect to the induced norm on $L_n =  \xi_n  + l_n $. The ratio of the standard norm and this norm on $L_n$ is bounded. The choice is possible  due to the fact that  $\lambda_2>1$ is the maximal expansion rate along $\xi_n$ and $l_n$ is stable.

Let $\pi_s^n$ be the projection onto $L_n$ along $\tl\eta_n$ and let $\pi_u^n$ be the projection onto $\tl\eta_n$ along $L_n$.
Let $d_u$ be the derivative along $\tl\eta_n$ and $d_s$ be the derivative along $L_n$.
We will use the representation $T_zD =  \eta_n + L_n$ for tangent vectors at each point $z\in M_n$, and we will also use the notation  $(a,b) $ for a point $x_n + a + b\tl\eta_n$ in a neighborhood of $x_n$, where $a\in L_n$, $b\in \bbR$.

Recall that  $Q_n\colon L_n\to \bbR$ is the function defined on a neighborhood of zero in $L_n$ such that its graph $(y, Q_n(y))$  coincides with $M_n$.


\medskip
\noindent
\textbf{Step 2: Choosing neighborhoods.}

\begin{lemma}
\label{lem-estim}
For any $\eps>0$, there exists $\mu>0$ such that for all $n$, in a $\mu$-neighborhood $U_n$ of zero  in $L_n$, for  $y\in U_n$, $z=(y, Q_n(y))$, we have:
\begin{enumerate}
\item $ \|d_s \pi_s R_n|_{z}\|<\lambda_2+2\eps $;


%  \item $\|d \pi_s R|_z\|<(1+\eps)\|d\pi_s R|_{f_n}\|$;
%  \item the tangent space to $M_n$ at $z$ stays in the cone on which $\|dR|_z\|<(1+\eps)\|d\pi_s R|_z\|$;
 \item  $\|d_u(\pi_s R_n)|_{z}\|<\eps$; $\|d_s( \pi_u R_n)|_{z} \|<\eps$;
 \item   $d\pi_u R_n|_{z} \tl\eta_n = \nu_n \tl\eta_{n+1}$ where $|\nu_n| > \lambda_1-2\eps$.
\end{enumerate}
\end{lemma}
\begin{proof}
With the choice of $\tl\eta_n$ as above, the matrix of $dR_n|_{x_n}$ in $(\tl\eta_n, L_n)$ coordinates in the domain and $(\tl\eta_{n+1}, L_{n+1})$ coordinates in the image is block-diagonal. In particular,  $$d_s \pi_u R_n|_{x_n} =0,\; d_u(\pi_s R_n)|_{x_n} =0,\; \|d_s (\pi_s R|_{x_n}) \| \le \lambda_2+\eps,$$ and $$d_u (\pi_u R|_{x_n}) \tl\eta_n = \nu_n \tl\eta_{n+1},\;|\nu_n| \ge \lambda_1-\eps.$$ Since we have uniform estimates on derivatives of $R_n$ on a $\delta$-neighborhood of $\{x_n\}$, the required estimates hold in $\mu$-neighborhoods of $x_n$ for a certain $\mu$ independent of $n$.

\end{proof}

% We will also fix $c$ so that $\|d^{(m)}(R)|_{z}\|<c$ for $m\le k$ in all these neighborhoods. Uniform choice of $c$ is possible due to compactness of $\{g\in \Lambda, \rot g \in \mathcal B_C\}$. Finally, fix $K$ so that in a neighborhood of $\mathcal I$, $\|dQ_n|_{z}\|<K$ for all $n$. \fixme{explain why possible}

\noindent
\textbf{Step 3. Recurrent relation between higher-order derivatives along orbits of $ R_n$.}

We have for $y\in U_n$ such that $R_n(y, Q_n(y))\in M_{n+1}$:
% $$
% Q_{n}(y) = \frac{1}{\lambda_1} Q_{n+1} (\pi_s R(y, Q_n(y))) - \frac{1}{\lambda_1} (\pi_u R(y, Q_n(y)) - \lambda_1 Q_n(y))
% $$
% Indeed, this is equivalent to
$$ \pi_u R_n(y, Q_n(y)) =  Q_{n+1} (\pi_s R_n(y, Q_n(y)))$$
since  a point $R_n(y, Q_n(y)) =(\pi_s R_n(y, Q_n(y)),  \pi_u R_n(y, Q_n(y)))$ belongs to $M_{n+1}=\{(y, Q_{n+1}(y))\}$.

Now, we differentiate this equality $m\le k$ times with respect to $y$ and separate away the terms that involve $m$-th derivatives of $Q_n$ and $Q_{n+1}$. We get:
\begin{multline}
\label{eq-R}
d_u(\pi_u R_n)Q^{(m)}_n(y) + \left[ d_u d_s \pi_u R_n Q^{(m-1)}_n(y)+\dots +d_s^{(m)} \pi_u R_n \right] = \\= Q_{n+1}^{(m)}|_{ (\pi_s R_n(y, Q_n(y)))} \cdot (d \, \pi_s R_n(y, Q_n(y)))^m  +  dQ_{n+1} d_u(\pi_s R_n)Q_n^{(m)}(y) +\\+ \left[ dQ_{n+1} d_u d_s \pi_s R_n Q_n^{(m-1)}(y))) + \dots\right].
\end{multline}
Here $Q^{(m)}$ is an $m$-linear form on $L_n$, it is defined on $m$-tuples of vectors from $L_n$, and $ {(d \, \pi_s R_n(y, Q_n(y)))^m}$ applies the operator  $d \, \pi_s R_n(y, Q_n(y))$ to each vector of the $m$-tuple.

This relates $Q^{(m)}_n(y)$ to  $Q^{(m)}_{n+1}$ at a point $\pi_s R_n(y, Q_n(y))$.

We inroduce the following notation:
\begin{itemize}
 \item $D_n$ acts on tuples of $m$ vectors in $TD$ by
 $$D_n :=  {(d \, \pi_s R_n (y, Q_n(y)))^m}$$
(this operator takes a tuple of $m$ vectors $v_1,\dots, v_m$ in $L_n$ to the tuple $d \, \pi_s R_n(y, Q_n(y))) v_1$, $d \, \pi_s R_n(y, Q_n(y))) v_2,$ $\dots $, where all derivatives are computed at $(y, Q_n(y))$),
\item  $ B_n\colon \tl \eta_n  \to  \tl\eta_{n+1} $ is given by $$B_n:= - dQ_{n+1}  d_u(\pi_s R_n)+   d_u \pi_u R_n$$ at $(y, Q_n(y))$;
\item $A_n$ is the sum of all components in the square brackets in \eqref{eq-R}; this sum involves derivatives of $ R_n$ and lower derivatives of $Q_n$, $Q_{n+1}$.
\end{itemize}


Now,  \eqref{eq-R} turns into the relation
  $$B_n Q^{(m)}_{n} (y)  =  Q^{(m)}_{n+1}|_{\pi_s R_n(y, Q_n(y)) } D_n +A_n.$$


% Now consider this relation for larger and larger $n$, in the (decreasing) domain in $L$ for which $R^n(x, F(x))$ still belongs to $M$ and to the domain where $F$ is analytic with bounded derivatives. We will later see that this domain shrinks to $L_1$.

\noindent
\textbf{Step 4. Inverting $B_n$ and iterating the recurrent relation.}


The proof of Theorem \ref{th-smoothness} is by induction on $k$. Namely, we will prove that derivatives $Q_m^{(s)}$ have limits as $(y, Q_m(y))$ tends to $x\in D_0$ and are bounded on $M_n$ uniformly on $n$. The base $k=1$ is included in the assertions of the theorem. Suppose that derivatives of $Q_n$ up to order $m-1$ are bounded uniformly on $n$ and have limits as $(y, Q_m(y))\to x\in D_0$.
\begin{lemma}
\label{lem-An}
 For sufficiently small $\eps>0$,  the terms $A_n$ are bounded by the same constant for all $n$ and have limits as $(y, Q_n(y))\in M_n$ tends to $x\in D_0$.
\end{lemma}
\begin{proof}
 This holds since $A_n$ is a combination of lower derivatives of $Q_n$ (that are uniformly bounded and have limits due to the inductive statement) and derivatives of $R_n$ that are continuous and  uniformly bounded in a $\delta$-neighborhood of $\{x_n\}$.
\end{proof}

The constant $\eps>0$ will be chosen to satisfy the following. 

\begin{lemma}
\label{lem-DnBn}
 Suppose that the norms of $dQ_n$ in $U_n$ are bounded by $K$.  Then   $\|D_n\|\le (\lambda_2+\eps+2K\eps)^m$ and $\|B_n-\nu_n\|< 2K\eps$ for some  $|\nu_n| > \lambda_1-2\eps$.

 For sufficiently small $\eps$, $B_n$ is invertible and $\|D_n\|\cdot \|B_n^{-1}\|<1$.
\end{lemma}
\begin{proof}
The first statement follows from Lemma \ref{lem-estim} since $$d \, \pi_s R_n(y, Q_n(y))) = d_s \pi_s R_n + d_u \pi_s R_n \cdot dQ_n$$ and its norm is thus bounded by $\lambda_2+\eps+2K\eps$.

The second estimate is a direct corollary of Lemma \ref{lem-estim} and a bound on $dQ_n$.


For small $\eps$,  $B_n$ is invertible and $\|B_n^{-1}\|< (\lambda_1-2\eps-2K\eps)^{-1}$, thus $\|D_n\|\cdot \|B_n^{-1}\|<1$ for small $\eps$ since $\lambda_2^m < \lambda_1$.
\end{proof}


We get
$$Q^{(m)}_{n} (y)  =  B_n^{-1} Q^{(m)}_{n+1}|_{ (\pi_s R_n(y, Q_n(y)))} D_n +B_n^{-1}A_n.$$

% Put
% $$A_n =  \sum_{s<m} Q_n^{(s)}\alpha_s.$$

%  If we have a bound on  $Q^{(l)}$ (which will be the case if $\pi_s R^n(x, F(x))$ is in the domain of analyticity for $F$), this term tends to zero uniformly as $n\to \infty$ in the shrinking domains as above, since $(\lambda_2+\eps)^l\lambda_1^{-1}<1$ due to the choice of $\eps$;





Suppose that $L_n$-projections of images of the point $(y, Q_0(y))$ under $ R_1, R_2R_1, \dots, R_{n}\dots R_2 R_1$ stay in the domains $U_n$. Iterating the relation between $Q^{(m)}_n$ and $Q^{(m)}_{n+1}$, we get
\begin{multline}
\label{eq-recur}
 Q^{(m)}_{0} (y)  =  B_1^{-1} B_2^{-1} \dots B_n^{-1} Q^{(m)}_{n+1}D_n\dots  D_2 D_1 + \\+ \dots + B_1^{-1}B_2^{-1}A_2 D_1+  B_1^{-1}A_1
\end{multline}
where $Q^{(m)}_{n+1}$ is computed at a point $\pi_s R_nR_{n-1}\dots R_1(y, Q_{0}(y))$.


\medskip
\noindent
\textbf{Step 5. End of the proof.}

\begin{lemma}
\label{lem-delta1}
For sufficiently small $\delta>0$, there exists $\delta_1>0$, $\delta_1<\delta$, with the following property. For any $x\in M_0\cap D_0$ that belongs to $B_{\delta/2}(x_0)$, for any  $z\in M_0\cap D_-$ with $\dist(z, x)<\delta_1$, there exists $N$ such that $N$ iterates of $z$ under $R_1, R_2R_1, \dots$ stay in neighborhoods $B_\delta(x_k)$, and $R_N\dots R_1(z) $ is outside $B_{\delta_1}(D_0)$.
\end{lemma}
\begin{proof}
Select a point $w\in M_0\cap D_0$ such that $(z-w)\in \xi_0+\eta_0$; this is possible since $TM_0\cap D_0 = l_0$ is transversal to $\xi_0+\eta_0$ at $x_0$, thus in its neighborhood. 

Let $z-w = a \tl \xi_0+b\tl\eta_0$, then $\dist (z, D_0)=a$ in our metric. 
Since  $\|dQ_0\|$ is bounded, the fraction $b:a$ is uniformly bounded for all points $\in B_\delta(x_0)$ and thus the fraction $\dist(z, w): \dist(z, D_0)$ is bounded by some constant $C$ that does not depend on $z$. 

Since $\xi_n$  is an unstable distribution for $\{R_n\}$,  we have $$\| dR_{n}\dots R_1  (z-w)\| > c(1+\rho)^n \dist (z, D_0) > c_1 (1+\rho)^n \dist (z, w)$$ for some $\rho>0$. Since $\|dR_k\|$ are uniformly bounded, we have  $$\|d R_n\dots R_1 (z-w)\|< A^n \|z-w\|.$$
Due to the uniform estimate on $R''_n$, for sufficiently small $\delta$, this implies $$\dist (R_k \dots R_1z, D_0)> \tilde c (1+\rho/2)^n \dist (z,w)$$ and   $$ \dist ( R_n\dots R_1 z, R_n\dots R_1 w) < (2A)^n \dist (z, w)  $$  while $R_n\dots R_1 z$ stays in $B_{\delta}(x_n)$.

Since $ w$ is close to $x_0$ and belongs to the stable manifold $M_0\cap D_0$ of $\{R_n\}$, we have $\dist (R_n \dots R_1 w, x_n) \to 0 $; since $x\in B_{\delta/2}(x_0)$ and $z\in B_{\delta_1}(x)$, we can choose  $\delta_1$ so that the  future orbit of $w$   stays in $B_{3\delta/4}(x_n) $. 

Now, at least $K = \lfloor \log_{2A} \frac{\delta/4}{\dist(z, w)} \rfloor $ images of $z $ under $\{R_n\}$ stay in $B_{\delta/4}(R_n\dots R_1 w)$ and thus in $B_{\delta}(x_n)$. Also, the iterate of order  $N= \lfloor \log_{1+\rho/2} \frac{\delta_1}{\tilde c \cdot  \dist (z, w)}\rfloor +1$ is outside $B_{\delta_1}(D_0) $; note that $N\ge 1$ since $z$ itself belongs to $B_{\delta_1}(D_0)$.  Choose  $\delta_1>0$ such that $ \log_{2A}(\delta/4)>2\log_{1+\rho/2}(\delta_1/\tilde c)$. Then we have $K\ge 2(N-1)\ge N$. Finally, $N $ iterates of $z$  belong to $B_{\delta}(x_n)$, while the $N$-th iterate is outside $B_{\delta_1}(D_0)$. This completes the proof.

\end{proof}
% Passing to a subsequence, we may and will assume that projections of $N$ iterates of $(y_N, Q(y_N))$ stay in respective  neighborhoods $U_n$, and the projection of an $N$-th iterate $R_N R_{N-1} \dots R_1 (y_N, Q(y_N))$ is at a distance $d$, $\delta/ (2|\lambda_2|)<d< \delta$, from  zero in $L_n$. This follows from the estimate on the norm of $\pi_s R$.

% Fix $a_0$ so that $|a|>|a_0|$ outside $\delta / (2\lambda_2)$-neighborhoods of zero in $l_n$. This is possible since $a$ is detached from zero on these neighborhoods.

Consider $(y_s, Q_0(y_s))\to x\in D_0\cap M_0$, and apply this lemma to  $z=(y_s, Q_0(y_s))$. Fix the corresponding $\delta_1$, and recall that $Q^{(m)}_n$ are uniformly bounded on   $B_{\delta}(D_0)\setminus B_{\delta_1}(D_0)$.  
For  $y=y_s$, use the formula \eqref{eq-recur} with the number of iterates $n+1=N$ provided by Lemma \ref{lem-delta1}; note that $N$ depends on $s$ and tends to infinity as $s\to \infty$. The  relation \eqref{eq-recur} applies since iterates of $(y_s, Q_0(y_s))$ stay in $B_\delta(x_n)$. Due to the choice of $N$, we have  $$\pi_s R_nR_{n-1}\dots R_1(y_s, Q_{0}(y_s))\in B_{\delta}(D_0)\setminus B_{\delta_1}(D_0).$$ Thus the first term in \eqref{eq-recur} tends to zero as $s\to \infty$ since $Q^{(m)}_{n+1}$ is bounded  and $\|D_n\| \cdot \|B_n^{-1}\|<1$ (Lemma \ref{lem-DnBn}). 


The remaining terms in  \eqref{eq-recur} decrease at a uniform geometric rate  because all $A_n$-s are bounded by the same constant (Lemma \ref{lem-An}) and $\|B_k^{-1}\| \cdot \|D_k\|<1$. Each term has a limit as $s\to \infty$. Thus the sum of these terms converges as $s\to \infty$  to the sum of limits of all  the terms.  
%Thus for any $\nu>0$, one can find $N$  such that for all $y_s$,  $$\sum_{k>N} \|B_1^{-1}\dots B_{k}^{-1} A_k D_{k-1}\dots D_1\|<\nu.$$ Also, $$\sum_{k\le N} \|B_1^{-1}\dots B_{k}^{-1} A_k D_{k-1}\dots D_1\|$$ has a limit as $y_n\to x$ since all of its terms have limits. 

We conclude
that as $n\to \infty$, the sequence $Q^{(m)}_{0} (y_s)$ has a limit. Since arguments work for any $m\le k$, we conclude that  $M_0$ is $k$-smooth.
Repeating the same arguments for $(y_s, Q_n(y_s)) \to x_n$, we show that all $Q_n^{(m)}$ are $k$-smooth. 

To complete the induction step, it remains to provide a uniform bound on $Q_n^{(m)}$ in small neighborhoods of $x_n$ with size independent on $n$. The above arguments work in the $\delta_1$-neighborhood of any point $x\in B_{\delta/2}(x_n)\cap D_0\cap M_n$.  Both the first term of \eqref{eq-recur} and the sum of the remaining terms are uniformly bounded due to the same estimates as above, thus $Q_n^{(m)}$ are uniformly bounded in  $\delta_1$-neighborhoods of $D_0\cap B_{\delta/2}(x_n)\cap M_n$. Also, outside $B_{\delta_1}(D_0)$, the derivatives $Q_n^{(m)}$ are uniformly bounded due to the assumptions. Thus they are uniformly bounded in $ B_{\delta/2}(x_n)$. This completes the proof.

\end{proof}
