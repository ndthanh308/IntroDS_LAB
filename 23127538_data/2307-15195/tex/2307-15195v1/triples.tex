\section{Renormalization in the space of triples}
\label{sec-triples}
In this section, we define a Banach space setting for renormalization in which the results for homeomorphisms and critical circle maps can be combined, and prove Theorem~\ref{th-operator}.
\subsection{Space of triples}
Recall that $\Pi_\eps$ is a strip of width $\eps$ around $\bbR/\bbZ$ in $\bbC/\bbZ$, and let
$$\bar \Pi_\eps=[-0.5,0.5]\times[-i\eps, i\eps]\subset \bbC.$$
Set
\begin{equation}
  \label{eq-triples1}
  H_{a}(z) = z^3 + az.
  \end{equation}

We start with a helpful lemma:
\begin{lemma}
\label{lem-pi-exist}
Let $\eps>0$ and consider a closed set $\Omega^\bbR$ of analytic maps $G\colon \bar\Pi_\eps\to \bbC$ such that  $G(0)=0$,  $G$ is $\bbR$-preserving, $\max_{\bar \Pi_{\eps}}|G''|$ is uniformly bounded, $G'|_{[-0.5,0.5]}$ is uniformly bounded away from zero.

For sufficiently small $\delta$ and a small neighborhood $\Omega\supset \Omega^{\bbR}$, we can choose  $\tau_1, \tau_2>0$, the domains  $K = K_{a, G}\subset H_a G (\bar  \Pi_\eps)$, and  maps $\pi_{a,G} \colon K\to \bbC$ with the following properties.

\begin{itemize}
\item For any $G\in \Omega$,  for any  $|a|<\delta$, the map $\pi_{a, G}$ is a biholomorphism between $K$ and $\pi_{a, G}(K)$, continuous on $\overline  K$.  It depends analytically on $a,G$.

\item There exists $0<\tau_1<\eps, 0<\tau_2$ such that for any $G\in \Omega$, $|a|<\delta$, the map $\pi_{a,G}\circ H_{a} \circ G$ is defined in $\bar  \Pi_{\tau_1}$ and commutes with the unit translation. The projection of the domain $$\pi_{a,G}\circ H_a\circ G(\bar \Pi_{\tau_1}) \subset \bbC$$ to $\bbC/\bbZ$ contains $\Pi_{\tau_2}$.


 %?\item For $G\in \Omega^{\bbR}$, $K_{0, G}$ contains a strip around $[(G(-0.5))^3, (G(0.5))^3]$, and the size of this strip is uniformly bounded away from zero.

% \item There exists $\nu>0$ such that for any $G\in \Omega$,  for any  $|a|<\delta$, the image of $\bar  \Pi_\eps$ under the map  $\pi_{a,G}\circ H_{a} \circ G$ (whenever defined) covers $\Pi_{\nu}$.

\item  If $G\in \Omega^{\bbR}$ and $a\in \bbR$, the map $\pi_{a,G}$ is $\mathbb R$-preserving.

%\item There exists $\nu$, $\delta$ such that for $|a|<\delta$, $G\in \Omega$,    we can set $\nu=0.9\eps$.

\item   If $H_{a} \circ G$ commutes with the unit translation, then $\pi_{a,G}=\id$.

\end{itemize}

\end{lemma}

\begin{proof}
Choose small $\nu$ and consider the image of the left side and the right side of $\bar \Pi_\nu$  under $H_a \circ G$. Join endpoints of these curves by segments. If $\nu=\nu(\Omega^{\bbR})>0$  is sufficiently small,  $a$ is sufficiently close to zero, and $G\in \Omega$ is sufficiently close to $\Omega^{\bbR}$, the bounds on $G', G''$ imply that  these curves bound a curvilinear rectangle $K$ with non-self-intersecting boundary and  $K\subset H_a \circ G(\bar  \Pi_\eps)$. 

%Note that for $G\in \Omega^\bbR, a=0, $ this domain $K$ is a strip around $[G(-0.5)^3, G(0.5)^3]$ of width at least $\nu/C$ with uniformly bounded $C$.

The map $H_a\circ G ( G^{-1}\circ H^{-1}_a(z)+1)$  takes the left side of $K$ to its right side. Consider a smooth map $\Xi$ that depends analytically on $a, G$, takes $\bar \Pi_\nu$ to $K$ and conjugates  $z\to z+1$ to $H_a\circ G ( G^{-1}\circ H^{-1}_a(z)+1)$. This map  $\Xi$  is a  linear homotopy between the restrictions of the map $ H_a\circ G$ to the left and to the right boundary of $\bar \Pi_\nu$:
$$
\Xi (x+iy)= H_a G (-0.5+iy)\cdot  (0.5-x) + H_a G (0.5+iy) \cdot (0.5+x).
$$
Note that if $G$ preserves the real line and $a$ is real, then $\Xi$ preserves the real line.

Let us verify that the fraction  $\left| \Xi_{\bar z} / \Xi_z\right|$ is bounded away from 1:  we have
$$
\begin{aligned}
\Xi_x =& H_a G (0.5+ iy) - H_a G(-0.5+iy),   \\
-i\Xi_y =&  (H_a G)'(-0.5+iy) \cdot (0.5-x) + (H_a G)'(0.5+iy)\cdot  (0.5+x)
\end{aligned}
$$
and
$$
\frac {\Xi_{\bar z}} {\Xi_z} =  \frac{\Xi_x+i\Xi_y}{\Xi_x-i\Xi_y}
$$
Due to the bound on $G'$, the values $H_0G(0.5)$, $-H_0G(-0.5)$, $(H_0G)'(\pm 0.5)$  are positive, uniformly bounded and bounded away from zero for $G\in \Omega^{\bbR}$. Due to our bound on $G''$, we can choose small $\Omega\supset \Omega^{\bbR}$,  $\nu$, and $\delta$, so that  both $\Xi_x$ and $-i\Xi_y$ are close to $\bbR^+$, uniformly bounded and bounded away from zero for all $|a|<\delta, (x,y)\in \bar \Pi_\nu$. Thus the fraction  $\left| \Xi_{\bar z} / \Xi_z\right|$ is  uniformly  bounded away from 1.

Define a Beltrami differential $\mu$ on the plane by extending $ \Xi_{\bar z} / \Xi_z$ to the horizontal strip of width $\nu$ periodically, and by setting $\mu=0$ outside the strip. Let $\psi$ be a solution of the Beltrami equation $\psi_{\bar z}/\psi_z =\mu$ with the Beltrami differential $\mu$, normalized by $\psi(0)=0, \psi(1)=1, \psi(\infty)=\infty$; it exists due to the Ahlfors-Bers-Boyarski theorem \cite{Ahl-B}.
Set $\pi_{a, G} = \psi\circ \Xi^{-1}$. By construction, $\pi_{a,G}$ is an analytic univalent map in $K$ that extends continuously to its boundary and conjugates $$H_a\circ G ( G^{-1}\circ H^{-1}_a(z)+1)$$ to  the unit translation; thus  $\pi_{a,G}\circ H_{a} \circ G$ commutes with the unit translation on its domain. 
If $G\in \Omega^{\bbR}$, then $\Xi$ is $\mathbb R$-symmetric, thus the solution of the Beltrami equation is $\mathbb R$-symmetric. We conclude that in this case, $\pi_{a,G}$ preserves the real axis.

Let us prove that the map  $\pi_{a, G}$ depends analytically on $a,G$. Indeed, $\Xi $  (thus $\mu$) depends on $a,G$ analytically by construction, and $\psi$ depends on $a,G$ analytically due to the Ahlfors-Bers-Boyarski theorem. It is easy to see that an analytic compositional difference of two (non-analytic) maps depends analytically on a parameter if both maps depend analytically on a parameter.
 This implies the statement.


If $\delta$, $\Omega\supset\Omega^{\bbR}$ are sufficiently small, there exists $\tau_1>0$ such that $H_a G (\bar  \Pi_{\tau_1}) \subset K$ for $|a|<\delta,G\in \Omega$. Thus $\pi_{a,G} H_a G$ is well-defined in $\bar \Pi_{\tau_1}$.

{\it A posteriori}, we know that the solution of the Beltrami equation $$\psi = \pi_{a,G}\circ\Xi$$ is smooth, with non-degenerate differential in $\bar \Pi_{\nu}$. Since it depends analytically on $G$ in $\Omega$, the norm of its differential is bounded from below on $\bar \Pi_{\nu}$ for $G\in \Omega^{\bbR}$. The set $\Xi^{-1} (H_0G(\bar \Pi_{\tau_1}))\subset \bar \Pi_{\nu}$ contains a strip around $[-0.5, 0.5]$ of width uniformly bounded from below, thus the projection of $\pi_{0,G}(H_0G(\bar \Pi_{\tau_1})) = \psi(\Xi^{-1} (H_0G(\bar \Pi_{\tau_1}))$ to $\bbC/\bbZ$  contains a strip around $\bbR/\bbZ$ of width uniformly bounded from below. Hence we can fix $\tau_2>0$ so that for small $\delta$ and $\Omega\supset \Omega^{\bbR}$, for $|a|<\delta, G\in \Omega$, the projection of $\pi_{a, G}H_aG(\bar \Pi_{\tau_1})$ to $\bbC/\bbZ$ contains $\Pi_{\tau_2}$.






%For $G\in \Omega^{\bbR}$, the map $\pi_{a,G}$ is a biholomorphism defined in a strip $K$ around $[G(-0.5)^3, G(0.5)^3]$ of width bounded below due to construction.
 

%For $a=0, G=\id$, and small $\eps$, we have $H_a G (\bar \Pi_{\eps})\subset K$, thus  the map $\pi_{0,\id}H_0 G$ is defined in $\bar \Pi_{\eps}$. Thus for all close $a,G$, the map $\pi_{a,G}H_a G$ is defined in $\bar \Pi_{0.9\eps}$, and we can choose $\eta=0.9\eps$.


 If $H_a G$ commutes with the unit translation, then $H_a\circ G ( G^{-1}\circ H^{-1}_a(z)+1)$ coincides with the unit translation on the left side of the rectangle $K$, then $\Xi=\psi=\id$ and thus $\pi_{a,G}=\id$.


\end{proof}
% We now set
% \begin{equation}
%   \label{eq-triples2}
%   s = \pi_{0,\id}\circ H_0,
% \end{equation}
%   where $\pi_{a,G}$ is as in Lemma~\ref{lem-pi-exist}.
%

\begin{definition}
  \label{def-triples}
The Banach manifold $D=D_{\eps, \eps'}$ in Theorem \ref{th-operator} is the space of triples of maps $(F, H_{a}, G)$ where
\begin{itemize}
 \item $F\colon \bbC/\bbZ\to \bbC/\bbZ$ is conformal on $\Pi_{\eps'} \subset \bbC/\bbZ$;
 \item $G\colon \bbC\to \bbC$ satisfies $G(0)=0, G'(0)=1$ and is  conformal in a neighborhood of $\bar \Pi_\eps$.
\end{itemize}
The values of $\eps, \eps'$ will be chosen later. With the uniform norm in the domains of $F$, $G$ specified above,  $D$ is an analytic Banach manifold. Let $D_0\subset D$ be given by $a=0$.


We will say that the triple is $\bbR$-preserving if its $F$, $G$-components preserve the real line and $a$ is real.
\end{definition}

%%Now, construct a projection of $D$ to the space $\mathcal D_\eps$ of conformal maps in $\Pi_\eps$.


% These four curves If $\eps =\eps(C)$  is sufficiently small,  and $a$ is sufficiently close to zero, estimates on $G'$ near $\pm 1/2$ imply that  these curves bound a curvilinear rectangle $K$ with non-self-intersecting boundary.





Using Lemma~\ref{lem-pi-exist}, we define the projection $p\colon D\to \mathcal D_\tau$.
We set
$$p \colon \hat f = (F, H_{a}, G) \mapsto F \circ \pi_{a, G} \circ H_{a} \circ G.$$
We will use the same notation $p(\hat f)$ for an analytic extension of $p(\hat f)$ to any larger strips. Note that the projection $p$ is not defined on the whole space $D$; its domain depends on $\tau$. 

%Note that if $G$ is close to identity and $a$ is close to zero, then the domain of $p(\hat f)$ is only slightly smaller than  $\Pi_\eps$: indeed,  $\pi_{a,G}\circ G$ is close to $s=\pi_{0, \id}\circ G$, and $F$ is defined on $s(\Pi_\eps)$. This motivates the choice of the domain for $F$ in the definition of the space of triples.

The next lemmas  show that we can lift critical circle  maps and analytic families of circle maps that approach critical circle  maps to the space of triples.
\begin{lemma}
\label{lem-lift-cr}
For any $\eps, \eps'>0$, there exists an analytic map $$j\colon f\to \hat f =(F, H_0, G),\; j\colon \mathcal D_{\eps}^{cr} \to D_{\eps, \eps'}$$ such that $p(j(f))=f$ on the domain of $j$, and $j$ is  defined for all critical circle maps $f\in \mathcal D_{\eps}^{cr, \bbR}$ that  have no preimages of the critical value $f(0)$ in $\Pi_\eps$ except zero.
\end{lemma}
\begin{proof}
 Set $F (z)= z+f(0)$, $G(z) = \sqrt[3]{f(z)-f(0)}$. For $G$, we choose the branch of the cubic root that preserves the real line for critical circle maps $f\in \mathcal D_{\eps}^{cr, \bbR}$, which defines the branch unambiguously in a neighborhood of $\mathcal D_{\eps}^{cr, \bbR}$ in $\mathcal D_{\eps}^{cr}$.    Set $j( f) = (F, H_0, G)$. Then $G$ is well-defined in a strip $\Pi_{\eps}$ since it contains no preimages of $f(0)$ under $f$ except zero, the  map $H_0\circ G=f(z)-f(0)$ commutes with the shift by $1$, and thus $\pi_{0, G}=\id$. We get $F\circ \pi_{0, G} \circ H_0 \circ G = f$.

 We have $G(0)=0$; to guarantee $G'(0)=1$, we will have to replace $G$ by $C\cdot G$ for an appropriate $C$. This replaces $\pi_{0, G}$ by $z\mapsto z/C^3 $ and does not affect the equality  $F\circ \pi_{0, G} \circ H_0 \circ G = f$.
\end{proof}

\begin{lemma}
\label{lem-lift}
 Let $f_\mu$ be an analytic family of self-maps of $\bbC/\bbZ$ defined for $\mu_j\in (\mathbb C^k, 0)$, $z\in \Pi_\eps$. Suppose that $f_\mu^{-1}(f_\mu(0))=0$ in $\Pi_{\eps}$.  Suppose that for real $\mu_j$ and $\mu_1<0$, $f_\mu$ is a diffeomorphism of $\bbR/\bbZ$. Suppose that   for $\mu_1=0$, the maps $f_\mu$ are cubic critical circle maps with $0$ as a critical point. Assume that $f'_{\mu}(0)$ has a zero of order  1 at $\mu_1=0$: when $\mu_1$ encircles zero, $f'_\mu(0)$ makes one turn around zero.

 Then for some $\delta$, for all $\mu$ with $0\le \mu_1<\delta$ and $\dist (\mu, 0)<\delta$, we can define homeomorphisms  $h_\mu$ and triples $F_\mu, H_{a(\mu)}, G_\mu$ such that $F_\mu$ is holomorphic everywhere,  $G_\mu$ is a holomorphic map defined in $\bar  \Pi_{\eps}$, all these objects depend analytically on $\mu$, and $$h_\mu^{-1} f_\mu h_\mu=p(F_\mu, H_\mu, G_\mu).$$
 
 For $\mu_1=0$ and real $\mu_j$, the triple $(F_\mu, H_{a(\mu)}, G_\mu)$ coincides with $j(f)$ from the previous lemma, and $h_\mu=\id$.
\end{lemma}
\begin{proof}
Slightly abusing the notation, we will keep the notation $f_\mu$ for a lift $f_\mu\colon \bar \Pi_\eps\to \bbC$.

We need to satisfy $$h_\mu F_\mu \pi_{a(\mu),G_\mu} H_{a(\mu)} G_\mu h_\mu^{-1}(z)  = f_\mu(z)\text{ for }G_\mu(0)=0, G'_\mu(0)=1.$$ First, we will satisfy these equalities without the requirement $G_\mu'(0)=1$, and then we will make the final adjustments.

Since $f_\mu$ for $\mu_1=0$ is a cubic critical map, for small $\mu_1\neq 0$, $f_{\mu}$ has a pair of quadratic critical points $c_1, c_2$ close to zero, given by $(f_\mu)'(c_{1,2})=0$.
 When $\mu_1$ makes one turn around zero,  these critical points make half of a turn each and exchange; this follows from the fact that $(f_\mu)'(0)$ has a simple root at $\mu_1=0$. Namely, it is clear that the pair of critical points depends analytically on $\mu$ when $\mu_1\neq 0$, thus we only need to count the number of turns; since $$(c_1)'_{\mu_1} = -(f''_{z\mu_1}/f''_z)|_{c_1}  \sim \text{const}/ c_1,$$ we have $c_1\sim \sqrt{\mu_1}$ and this number is 1/2.

 Let  images of $c_1, c_2$ under $f$ be $d_1, d_2$. Then $d_1$, $d_2$ make $3/2$ of the turn each around $f(0)$ when $\mu_1$ makes 1 turn around zero, and they also exchange, since $$f_\mu(c_1)-f_\mu(0) \sim f'_z(c_1) c_1 \sim \mu_1^{3/2}.$$  Also, they are complex conjugate for real $\mu$.

 Both maps $F_\mu, h_\mu$ will be translations. We set $$F_\mu h_\mu (z)=z+(d_1+d_2)/2;$$ note that this composition depends analytically on $\mu$ (the translations $F_\mu, h_\mu$ themselves will be chosen later). Then the critical values of the map $(F_\mu h_\mu)^{-1}f_\mu$ are symmetric with respect to $0$ and purely imaginary for real $\mu$. Also, they make $3/2$  of the turn each around $0$ when $\mu_1$ makes 1 turn around zero. The difference between these points is still $d_1-d_2$.

  Set $$A(\mu)= \left(-\frac{27(d_1-d_2)}{4}\right)^{2/3};$$ we choose the branch of the cubic root that is real for $\mu_1\in \bbR_{-}$ (that is, when $d_1, d_2$ are purely imaginary).  Then the critical values of $H_A$, namely $\pm \frac{2a}{3}\sqrt{-a/3}$, coincide with the critical values $\pm \frac{(d_1-d_2)}{2}$ of  $(F_\mu h_\mu)^{-1}f_\mu$. Since $d_1-d_2$ makes $3/2$ turns around the origin as $\mu_1$ makes one turn around the origin, $A$ depends analytically on $\mu$ and has a simple zero at $\mu_1=0$.

Consider the 3-valued map  $H_A^{-1}(F_\mu h_\mu)^{-1}f_\mu$. It is defined in $\bar \Pi_{\eps}$.  For each $\mu$, it does not branch at critical points of $f_\mu$ since critical values of $H_A$ and $(F_\mu h_\mu)^{-1}f_\mu$ coincide. It might branch at other preimages of critical values of $f_\mu$, but our assumptions imply that the strip $\Pi_{\eps}$ does not contain such points.  Also, it is bounded on $\bar \Pi_\eps$. Thus this formula for each $\mu$ defines three distinct holomorphic functions in $\bar\Pi_\eps$ that depend analytically on $\mu$. We choose the one that is real for real $\mu$, $\mu_1<0$, and let it be equal to $G_\mu h_\mu$. Finally, we choose the translation  $h_\mu$ so that $G_\mu (0)=0$. Since $F_\mu h_\mu$ was chosen above, this uniquely defines $F_\mu$.

Note that for $\mu_1=0$,  we get $$A=0,  \; F_\mu h_\mu (z)=z+f_{\mu}(0),\; G_{\mu}(z) = H_0^{-1}(f_\mu(z)-f_\mu(0))\text{, and }h_\mu=\id,$$
so for critical maps in the family $f_\mu$, the lift coincides with $j(f_\mu)$ from Lemma \ref{lem-lift-cr}.


We now have  $h_\mu F_\mu H_{A(\mu)} G_\mu h_\mu^{-1}(z)  = f_\mu(z)$, all maps depend analytically on $\mu$, and $G_{\mu}(0)=0$.  
Since $H_{A(\mu)}\circ G_\mu$ is a circle map, $\pi_{A(\mu),G_\mu}=\id$ (see the last statement of Lemma \ref{lem-pi-exist}).

It remains to adjust our triple to achieve $G_\mu'(0)=1$.
Note that  $G_\mu'(0)\neq 0$ for $\mu_1=0$ since $f$ has a cubic critical point and $H_0^{-1}(z)=\sqrt[3]{z}$, so $G_\mu'(0)=C_\mu$ is a nonzero analytic function for small $\mu$.
Since $$H_A \circ Cz = C^3 H_{A/C^2}z,$$ we can make this adjustment while preserving analytic dependence of $A, F, G$ on $\mu$: we replace $G_\mu$ by $\frac{1}{C_\mu}G_\mu$, $A$ by $A/C_\mu^2$, $F_\mu$ does not change, and $\pi_{a,G} $ becomes $z\mapsto z\cdot C_\mu^3$.  This completes the proof.
\end{proof}



\subsection{Renormalization for triples}

Recall that the renormalization operator $\cren \colon \mathcal U\to \mathcal D_{\eps}^{cr, \bbR}$ leaves invariant the horseshoe-like set $\mathcal I$. Consider a small open neighborhood $\mathcal V$ of $\mathcal I$  in the space $\mathcal D_\eps$ so that the function $q_N(f)$ extends to  $\mathcal V$ as a locally constant function. Now $q_N$ is defined even for non-$\bbR$-preserving $f\in\mathcal V$.

If $\mathcal V$ is sufficiently small, then for each map $f\in \mathcal V$,  there exists a fundamental crescent $C=C_f$ and the perturbed Fatou coordinate $\Psi=\Psi_f$ with $\Psi(0)=0$ that conjugates $f^{q_N}$ to the unit translation  in $f^{q_N}(C)\cup C$, and is real-symmetric for real-symmetric $f$.  Thus the operator $\cren$ extends to $\mathcal V$. Moreover, due to the compactness of  $\bar  {\mathcal I}$, all maps $\cren f$ for $f\in \mathcal V$ extend analytically to the same strip $\Pi_c$ around the real axis, that is, $\cren \colon \mathcal V\to \mathcal D_{c}$ is well-defined.

The construction of renormalization $\cren \colon \mathcal U\to \mathcal D_{\eps}^{cr}$ for critical circle maps motivates the following definition of the renormalization in the space of triples, also used in \cite{GorYa}.
\begin{definition}
\label{def-R}
The renormalization in the space of triples $D$ is defined in the following way: for any triple $(F, H_a, G)\in p^{-1}(\mathcal V)$, the triple $\mathcal R (F, H_a, G)=(\tilde F, H_{\tilde a}, \tilde G)$ is given by 
 $$\tilde G = \Psi'(0)  G \Psi^{-1} \text{ restricted to } \bar \Pi_\eps,$$
  $\tilde a = a (\Psi'(0))^2$ so that $$H_{\tilde a} = \Psi'(0)^3 \cdot H_a (\Psi'(0))^{-1},$$
  and
 $$\tilde F = \cren f  \circ (\pi_{\tilde a, \tilde G} H_{\tilde a} \tilde G)^{-1}  \text{ restricted to } \Pi_{\eps'}, $$
where $f = p(F, H_a, G)$.
 \end{definition}
Since $ \cren f =\Psi \circ P \circ \Psi^{-1}$ where $P$ is an iterate of $f = F \pi_{a,G}H_a  G $, the composition that gives $ \cren f  $ will end with terms  $H_a G \Psi^{-1} = \Psi'(0)^{-3} H_{\tilde a} \tilde G$. Since $\pi_{a,G}$ are bijections, the inverse map  in the definition of $\tilde F$ does not produce any branching and $\tilde F$ is well-defined.

 Note that $\tilde F$ commutes with the unit translation as the compositional difference of two maps $\cren f$ and  $\pi_{\tilde a, \tilde G} H_{\tilde a} \tilde G_a$ which both commute with the unit translation.

Recall that the construction of the map $(a,G)\to \pi_{a,G}$, and thus the construction of $p$, depends on the set $\Omega^{\bbR}$ of maps $G$ with certain uniform estimates. The resulting operator $\mathcal R$ will only be defined on a certain subset of the space $D$ of triples where the estimates on the $G$-components hold true. This set  $\Omega^{\bbR}$ will be fixed in Lemma \ref{lem-horseshoe}.

\begin{lemma}
\label{lem-projections}
 For any triple $\hat f$ such that $p(\hat f)\in \mathcal V$, we have $p (\mathcal R \hat f) = \cren(p(\hat f))$ whenever the left-hand side is defined, i.e. the renormalization operator $\mathcal R$  projects to the renormalization operator $\cren$ on $\mathcal V$.
\end{lemma}
\begin{proof}
Indeed,
$$ p  ( \mathcal R (F, H_a, G)) = \tilde F \circ \pi_{\tilde a, \tilde G} H_{\tilde a} \tilde G = \cren f =\cren (p (F, H_a, G)).$$

\end{proof}

Recall that the construction of the renormalization operator $\cren$ depends on the choice of the iterate $q_N$.
In the assumptions of Lemma \ref{lem-pi-exist}, let   $U_{\delta, \Omega}$ be the set of triples $(F,H_a, G)$ such that $0\le |a|\le \delta,\; G\in \Omega$.

 \begin{lemma}
\label{lem-compact}
For sufficiently small  $\eps>0$ and a suitable choice of $\eps'>0$ in the definition of the space of triples $D$, for a set  \begin{multline}
\Omega^{\bbR} = \{G\colon \bar \Pi_{\eps}\to \bbC, G(0)=0, G'(0)=1, G\text{ preserves } \bbR  \mid \\  \sup_{\bar  \Pi_\eps}|G''|\le C, \min_{[-0.5, 0.5]}G'\ge 1/C\}\end{multline} with any sufficiently large $C$,
%for sufficiently large $N$ in the definition of $\cren$, the renormalization operator $\mathcal R\colon D\to D$  is analytic in  $p^{-1}(\mathcal V)$ (wherever defined).
 we can choose its neighborhood $\Omega\supset \Omega^{\bbR}$, $N$,  and a small $\delta$ such that the renormalization operator $\mathcal R\colon D\to D$  is defined, analytic, and has a compact differential on  the ``truncated fibers'' $U_{ \delta,  \Omega} \cap p^{-1}(\mathcal V)$ of the projection $p$.

The images of truncated fibers  $\mathcal R(U_{\Omega,\delta} \cap p^{-1}(\mathcal V))$ are precompact, and the projection $p \colon \mathcal R(U_{\Omega,\delta} \cap p^{-1}(\mathcal V)) \to \mathcal D_{\eps}$ is defined.

Finally, the set $\mathcal R(U_{\Omega^{\bbR},0} \cap p^{-1}(\mathcal V))$ is contained in $U_{\Omega^{\bbR},0}$.

%The set $\Omega^{\bbR}$ can be chosen arbitrarily large: for any compact set of $\mathbb R$-preserving maps in the space of analytic maps in $\bar  \Pi_{\eps}$, $\Omega^{\bbR}$ can be chosen to contain it.
\end{lemma}
\begin{proof}


Note that for any $f\in \mathcal V$, the renormalized map  $\cren f$ is defined in a certain strip  $\Pi_{c}$; choose $\eps<c/2$, then the map $\cren f$ is defined in $\Pi_{2\eps}$. We will use this $\eps$  in the definition  of $D$. The value of $\eps'$ will be chosen later.


Since the map $\pi_{a,G}$ is analytic on $a, G$, the projection $p$ is analytic. Since the uniformizing coordinate  $\Psi=\Psi_f$  is defined in $\mathcal V$ and  analytic on $f$, the operator $\mathcal R$ is defined and analytic on the intersection of the domain of $\pi_{a,G}$ with $p^{-1}(\mathcal V)$.

Now let us check that its differential is compact.
Lemma \ref{lem-Psi-estim} implies that  for sufficiently large $N$, the map   $\tilde G= \Psi'(0) G \Psi^{-1}$ is defined in a rectangle  $\{-1<\Re z <1, -2\eps<\Im z<2\eps\}$ that covers $\bar  \Pi_\eps$. It remains to show that for a suitable choice of $N$,  $\eps'$, and $\Omega$, the map $\tilde F$ also extends analytically to a domain that contains $\Pi_{\eps'}$ in its interior. This will imply both compactness statements of the lemma.

Let $(\tilde F, H_{\tilde a}, \tilde G) = \mathcal R(F, H_a, G)$.
Note that   $$\tilde G'(z) = \Psi'(0) G'|_{\Psi^{-1}(z)} \frac{1}{\Psi'(\Psi^{-1}(z))}$$ and
\begin{equation}
 \label{eq-G}
 \tilde G''(z) = \Psi'(0) G''|_{\Psi^{-1}(z)} \frac{1}{\Psi'(\Psi^{-1}(z))^2} - \Psi'(0) G'|_{\Psi^{-1}(z)} \frac{\Psi''|_{\Psi^{-1}(z)} }{\Psi'(\Psi^{-1}(z))^3}.
\end{equation}

 
Let $$\dis \Psi = \sup_{\mathcal V}\sup_{ a, b\in \Psi^{-1}(\bar \Pi_{\eps})}\left(\frac{\Psi'(a)}{\Psi'(b)},  \frac{\Psi''(a) }{\Psi'(a)^2}\right)$$ be the maximal distortion of $\Psi$; due to Lemma \ref{lem-Psi-estim}, this is finite and independent of $N$. Consider any $C$ with $C>2 (\dis \Psi)^2$, and use it in the definition of  $\Omega^{\bbR}$.

Choose $N$ so that $\Psi'(0)> 2 C \cdot \dis \Psi$; this is possible due to Lemma \ref{lem-Psi-estim}.  
Due to the normalization $G(0)=0$ and $G'(0)=1$, we conclude that if $G\in \Omega^{\bbR}$, then in $\bar  \Pi_{\eps}$,   $$|\tilde G'|> (\dis \Psi)^{-1} \cdot |G'|_{\Psi^{-1}(z)}| > (\dis \Psi)^{-1} (1-\frac{C \cdot \dis \Psi}{|\Psi'(0)|}) > 0.5 (\dis \Psi)^{-1} > \frac{1}{C}$$ and $$|\tilde G''|<C\frac{(\dis \Psi)^2}{ |\Psi'(0)|}+ (\dis \Psi)^2 (1 + \frac{C \cdot \dis \Psi}{|\Psi'(0)|}) < 0.5 (\dis \Psi)^2 + 1.5 (\dis \Psi)^2<C.$$

Thus $G\in \Omega^{\bbR}$ implies $\tilde G\in \Omega^{\bbR}$, and we have proved the last statement of the lemma. 

Use Lemma \ref{lem-pi-exist} to define $\pi_{a,G}$, choose $\delta,  \Omega\supset \Omega^{\bbR}$, and  $\eps'=\tau_2/2$ such that for all $G\in \Omega$, we have $\pi_{0, G}H_0 G(\bar \Pi_{\eps})\supset \Pi_{2\eps'} $. We will use this $\eps'$ in the definition of the space of triples  $D=D_{\eps, \eps'}$.

Since  $\cren f = \tilde F\circ \pi_{\tilde a,\tilde G}\circ H_{\tilde a}\circ \tilde G$ is defined in $\Pi_{2\eps}$, we conclude that $\tilde F$ analytically extends to $\pi_{\tilde a,\tilde G}\circ H_{\tilde a}\circ \tilde G(\bar  \Pi_{\eps}) \supset \Pi_{2\eps'}$.

% Lemma \ref{lem-pi-exist} (applied to $2\eps$ instead of $\eps$) implies that for large $N$ and small $\delta$,  for $|a|<\delta$,  the map  $\pi_{\tilde a,\tilde G}$ is well-defined; also, the domain of $\pi_{\tilde a, \tilde G}\circ H_{\tilde a}\circ \tilde G$ includes $\Pi_{1.8 \eps}$.

%On the other hand, for any $f\in \mathcal V$, the renormalized map  $\cren f$ is defined in a certain strip  $\Pi_{c}$; if $\eps<c/2$, the map $\cren f$ is defined in $\Pi_{2\eps}$.
%Since  $\cren f = \tilde F\circ \pi_{\tilde a,\tilde G}\circ H_{\tilde a}\circ \tilde G$, we conclude that for large $N$ and $0<\tilde a<\delta$, the map  $\tilde F$ analytically extends to $s(\Pi_{1.8\eps})$.

Since we restrict $\tilde F, \tilde G$  to smaller domains $\bar \Pi_\eps, \Pi_{\eps'}$ in the definition of $\mathcal R$, this operator $\mathcal R$ has compact derivative. We have also proved that the set $\mathcal R(U_{\Omega, \delta}\cap p^{-1}(\mathcal V))$ is pre-compact.


The above arguments  imply that for sufficiently small $\Omega\supset \Omega^{\bbR}$ and $\delta$, on the image of $\hat f\in U_{\Omega, \delta}\cap p^{-1}(\mathcal V)$ under $\mathcal R$, the map $\pi_{\tilde a, \tilde G} $ is defined and $p(\mathcal R \hat f)=\pi_{\tilde a,\tilde G}\circ H_{\tilde a}\circ \tilde G$ is well-defined in a certain strip $\Pi_{\tau_1}$ around $\bbR/\bbZ$. This map extends to the strip $\Pi_\eps$ since it equals $\cren f$. Thus $p\colon \mathcal R(U_{\Omega, \delta}\cap p^{-1}(\mathcal V)) \to \mathcal D_{\eps}$ is defined. This completes the proof. 
\end{proof}

\begin{lemma}
\label{lem-fibers}
In the assumptions of Lemma \ref{lem-compact}, the operator $\mathcal R$ in the space of triples uniformly contracts on truncated fibers $p^{-1}(f) \cap U_{\Omega, 0}$ of the projection $p$ for critical maps $f\in \mathcal V\cap \mathcal D^{cr}_\eps$.

In more details, for any set $\Omega^{\bbR}$ as in Lemma \ref{lem-compact} we can choose its neighborhood $\Omega$, $N$, and constants $0<\lambda<1$, $c$, such that for any two triples $(F_1, H_0, G_1)$ and $(F_2, H_0, G_2)$ in $U_{\Omega, 0}$  with $$p(F_1, H_0, G_1) = p(F_2, H_0, G_2)\in \mathcal V\cap \mathcal D^{cr}_{\eps},$$  we have
$$\dist (\mathcal R^n (F_1, H_0, G_1), \mathcal R^n (F_2, H_0, G_2)) <c \lambda^n \dist((F_1, H_0, G_1), (F_2, H_0, G_2)). $$
\end{lemma}
\begin{proof}

%The statement on the domains of %image of each fiber $\mathcal R(p^{-1}f)$ follows from the fact that $\mathcal R$ increases domains of $F, G$, see the previous lemma.

The estimate \eqref{eq-G} applied to $G_1, G_2$ and Lemma \ref{lem-Psi-estim} implies  that by choosing sufficiently large $N$ in the definition of $\cren$, we can guarantee that \begin{equation}\label{eq-G''}\dist (\tilde G_{1}'', \tilde G_{2} '') \le \mu_0\dist_{C^2} (G_1, G_2)\end{equation} for some $\mu_0<1$. Indeed, the first summands in \eqref{eq-G} are uniformly small, and in the second summands, $G_1'-G_2'$ is computed on a uniformly small neighborhood of zero and multiplied by a bounded function. Since $G_1'(0)=G_1'(0)=1$, we can estimate $G_1'-G_2'$ on a small neighborhood of zero in terms of $G_1''-G_2''$ and get \eqref{eq-G''}.

Since  we only consider triples normalized by  $G(0)=0$ and $G'(0)=1$, this implies that $G$-components of the triples with the same projection approach each other exponentially quickly.

Since $G$-components of $\mathcal R^n(F_{1,2}, H_0, G_{1,2})$ approach each other exponentially quickly, the same holds for the corresponding maps $\pi_{0,G_{1,2}}$ due to the fact that the dependence of $\pi_{0,G}$ on $G$ is analytic, thus Lipshitz on $\Omega^{\bbR}$.  Since the triples have the same projection, their $F$-components approach each other exponentially quickly. This implies the statement.



%
% ; the chart $\Psi$ is the same for both cases since it is determined by the projection $p(F_1, H_0, G_1) = p(F_2, H_0, G_2)$ to the space of critical maps. We have
% $$
% \sup_{\Pi_{\eps}} |G_{1\,  new} -  G_{2 \, new}  |= \sup_{\Pi_\eps} |(\Psi'(0))^{-1}  (G_1 \Psi - G_2 \Psi)| \le \Psi'(0))^{-1}  \sup_{\Pi_\eps} |G_1 - G_2|
% $$
% provided that $\Psi$ is an expanding map, $\Pi_\eps \subset \Psi(\Pi_\eps)$.
% Due to \cite{}, on a neighborhood of $\mathcal I$, the map $\Psi$ is expanding and $\Psi'(0)$ is uniformly bounded away from $1$ on $\mathcal I$ due to Lemma \ref{lem-Psi-estim}. Thus the distance between the third components of the triples exponentially decreases. Therefore that the distance between the corresponding maps $\pi_{0, G}$ exponentially decreases since $(0, G)\mapsto \pi_{(0,G)}$ is Lipshitz . Since the triples $\mathcal R^n(F_1, H_0, G_1)$ and $\mathcal R^n (F_2, H_0, G_2)$ have the same projections, the distance between their first components also exponentially decreases. This implies the first statement.
%
% Note that $G_{1,new}$ belongs to a compact uniformly bounded set of maps that extend to a uniformly large strip  $\Psi^{-1}(\Pi_\eps)$. Given $f$, the map $G\mapsto F : p(F, H_0, G)$ is continuous due to a continuity of $G\to \pi_{(0,G)}$ and the continuity of the compositional difference, thus its image is compact.
%
% If $N$ is chosen to be sufficiently large, $\Psi'(0)$ is large (Lemma \ref{lem-Psi-estim}) and thus $G_{1,new}$ is close to identity. This implies the last statement.
\end{proof}


Now we can lift $\mathcal I$ to the horseshoe-like set $\Lambda$ in the space of triples. In this lemma, we will fix the choice of $\eps, \eps'$ in the definition of $D$. We will also fix the choice of $\pi_{a,G}$, and thus complete the construction of the projection $p$.

\begin{lemma}
\label{lem-horseshoe}
There exists a choice of $\eps, \eps'$ in the definition of $D$, a choice of the map $(a,G)\to \pi_{a,G}$, and a set $\Lambda \subset D_0$ such that $\Lambda$ is invariant under $\mathcal R$, the operator $\mathcal R$ is an analytic operator with compact derivative on a neighborhood of $\Lambda$, and the projection $p\colon \Lambda \to {\mathcal I}$ is a homeomorphism that conjugates the dynamics of $\cren|_{\mathcal I}$ to the dynamics of $\mathcal R|_{\Lambda}$.

All triples in $\Lambda$ are $\bbR$-preserving.
\end{lemma}
\begin{proof}
For each critical circle map $f\in \mathcal I$, construct its lift $j(f) = (F,H_a, G)$ to $D$ using Lemma \ref{lem-lift-cr}: namely, set $$G(z) = \sqrt[3]{f(z)-f(0)}/C\text{, and }F(z) = z+f(0).$$ Then   $p((F, H_0,G))=f$.

All resulting maps $G$ for all $f\in \mathcal I$ are $\bbR$-preserving. Choose $\eps$ small, so that  all these maps   extend to $\bar \Pi_\eps$ univalently with uniform estimates on $\max_{\bar \Pi_\eps} |G''|$, $\min G'|_{[-0.5, 0.5]}$; consider  $C$ so that the set $\Omega^{\bbR}$ from Lemma \ref{lem-compact} contains all $G$-components of all lifts of critical maps  $f\in \mathcal I$. Decrease $\eps$ if needed to satisfy assumptions of Lemma \ref{lem-compact}, and use this lemma to choose $\eps'$. These values of $\eps, \eps'$ will be used in the definition of the space $D$. Finally, fix $\delta$ and $N$ so that assertions of Lemma \ref{lem-compact} and Lemma \ref{lem-fibers} hold true.

The space of triples $D$ is now fixed, and the projection $p$ is defined.

For each $f\in \mathcal I$, consider critical circle maps $f_{-n}\in \mathcal I$ such that $$\cren^n f_{-n}=f;$$ note that this uniquely defines $f_{-n}$.

For each $f_{-n}$, consider  truncated fibers of the projection, $p^{-1}(f_{-n})\cap U_{\Omega,0}$. Due to the choice of $\Omega^{\bbR}$, these sets are non-empty for all $g\in \mathcal I$: every such set contains a lift $j( f_{-n})$ of $f_{-n}$.

Since $\mathcal R$ contracts on truncated fibers (Lemma \ref{lem-fibers}), the diameters of the sets $\mathcal R^n(p^{-1} f_{-n}\cap U_{\Omega, 0}) \subset p^{-1}(f)$ tend to zero. All these sets are non-empty since they contain $\mathcal R^n (j( f_{-n}))$. Since each set is precompact and components of the triples in these sets extend to larger strips (Lemma \ref{lem-compact}), there exists a unique accumulation point $i( f)$ of these sets, and $F,G$- components of the triple $i(f)$ are defined in strips larger than $\Pi_\eps$, $\Pi_{\eps'}$ respectively. By continuity, the projection $p$ is defined on $i(f)$ and $p(i(f))=f$.

The construction implies that $$i(\cren f) = \mathcal R i( f),$$ thus the set $\Lambda = i(\mathcal I)$  is an invariant set for $\mathcal R$, and the dynamics of $\mathcal R|_{\Lambda}$ is conjugate to the dynamics of $\cren|_{\mathcal I}$ via the continuous map $p$.


Since  $i(f)$ is the limit point of $ \bbR$-preserving triples $\mathcal R^n(j( f_{-n}))$,  it is $\bbR$-preserving.

All maps $\mathcal R^n(p^{-1}(f_{-n})\cap U_{\Omega^{\bbR},0})$ belong to $U_{\Omega^{\bbR},0}$ due to the last statement of Lemma \ref{lem-compact}. In particular, $\Lambda\subset U_{\Omega^\bbR, 0}$.   Thus Lemma \ref{lem-compact} implies that $\mathcal R$ is an analytic operator with a compact differential on a neighborhood of $\Lambda$.


It remains to prove that $i$ is continuous, that is, the projection $p$ is a homeomorphism between $ \Lambda$ and $\mathcal I$.


Let $f\in \mathcal I$, $\delta>0$; show that the lifts of sufficiently close points of $\mathcal I$ are $\delta$-close to $i(f)$.  Let $\lambda, c$ be the same as in Lemma \ref{lem-fibers}; choose $n$ so that $c\lambda^n\cdot 2C<\delta/3$. For a map $f_{-n}$, take its neighborhood $U$ in $\mathcal I$, and let $g\in U$. Construct the lifts $j( g), j( f_{-n})$ using Lemma \ref{lem-lift}; clearly, the lifts are close in the space of triples if $U$ is small enough. Due to Lipshitz estimates on $\mathcal R$, we also have that $\mathcal R^n (j( g))$ and $\mathcal R^n(j( f_{-n}))$ are $\delta/3$-close if $U$ is small enough.

Note that the diameters of the sets $$\mathcal R^{n} (U_{\Omega,0}\cap p^{-1}(g))\text{ and }\mathcal R^{n} (U_{\Omega,0}\cap p^{-1}(f_{-n}))$$ are smaller than $c\lambda^n \cdot \diam (U_{\Omega,0})$ due to Lemma \ref{lem-fibers}, and the latter is smaller than $\delta/3$ due to the choice of $n$. Also, the closure of the first set contains both $\mathcal R^n(j( g))$ and $i(\cren^ng)$, and the closure of the second set contains both $\mathcal R^n(j( f_{-n}))$ and $i(f)$.
Since $\mathcal R^n (j( g))$ and $\mathcal R^n(j( f_{-n}))$ are $\delta/3$-close, we conclude that $i(f)$ and $i(\cren^ng)$ are $\delta$-close for $g\in U$. This completes the proof.


%Consider the set $A_C=\{\max_{\Pi_{2\eps}}|G''|<C\}$ in the space of analytic maps of an $\eps$-neighborhood of $\Pi_{\eps}$. For each $f$, consider the map $L_f \colon A_C \to D$ given by $L_f \colon G \to (F, H_0, G)$ with $F = f \circ (\pi_{0, G}H_0 G)^{-1}$.


\end{proof}
The previous lemmas provide the description of the Banach space $D$, its projection $p$, and the operator $\mathcal R$, required for Theorem \ref{th-operator}, with the exception of the last property: uniform hyperbolicity of $\mathcal R$ with two unstable directions.
The next lemma completes the proof of Theorem \ref{th-operator} and provides an explicit formula for the maximal expansion rate along the second one-dimensional unstable distribution of $\mathcal R$.

\begin{lemma}
\label{lem-distributions}
 Along each orbit in $\Lambda$, the operator $\mathcal R$ is hyperbolic, with two-dimensional unstable distribution which intersects $TD_0$ on a one-dimensional distribution $\eta_{\hat f}$. The operator $\mathcal R$ is uniformly hyperbolic on $\Lambda$.

 The stable distribution of $\mathcal R$ is a full preimage under $(dp)$ of the stable distribution of $\cren$. 

The distribution  $(dp)\eta_{\hat f}$ is  the unstable distribution of $\cren$.
At the orbit of the triple $\hat f\in \Lambda$,  the minimal expansion rate $\lambda_1$ of $\mathcal R$ along $\eta_{\hat f}$ coincides with the minimal expansion rate of $\cren$ along its unstable distribution $(dp)\eta_{\hat f}$ at the orbit of $p(\hat f)$.

Set
\begin{equation}\label{eq-lambda2}\lambda_2 =\limsup_{n\to +\infty} \sup_{m\ge 0} \sqrt[n]{\Psi'_{m, n}(0))^2}
\end{equation}
 where $\Psi_{m, n}$ is the uniformizing chart that corresponds to the iterate $\mathcal R^n $ at $\mathcal R^m \hat f$; if $\lambda_1>\lambda_2$, then $\mathcal R$ has an invariant one-dimensional distribution $\xi_{\hat f}$ transversal to $D_0$ along the orbit of $\hat f$, and the maximal expansion rate along this distribution coincides with $\lambda_2$.

Both one-dimensional distributions $\eta_{\hat f}$ and $\xi_{\hat f}$ are generated by  vector fields that belong to $TD^{\bbR}$. 

\end{lemma}
\begin{proof}
Roughly speaking, since $\mathcal R$ contracts on the fibers of the projection $p$ and $\mathcal R|_{D_0}$ projects, via $p$, to the hyperbolic operator $\cren$ on $\mathcal D^{cr}_{\eps}$, the operator $\mathcal R$ must be hyperbolic on $D_0$ with one unstable direction that projects to the unstable direction of $\cren$ and has the same expansion rate. Since $\mathcal R^n$ takes $a$ to $(\Psi'_{m,n}(0))^2 a$, it expands in the direction transversal to $D_0=\{a=0\}$, thus it must have one more unstable direction in $D$ with maximal expansion rate $\lambda_2$ as above.

Below we provide detailed proofs of these statements, by constructing the stable foliation and unstable invariant cone fields for $\mathcal R$.

 \textbf{Stable distribution of $\mathcal R$}


 Let $W^s_f\subset \mathcal D_{\eps}^{cr}$ be the local stable manifold of $f\in \mathcal I$ under $\cren$. We will prove that an analytic manifold $p^{-1}(W^s_f)$ is a stable manifold of $i(f)$. Consider the lift $j ({W^s_f})$ constructed as in Lemma \ref{lem-lift-cr}.  Then $p^{-1} (W^s_f)$ is a saturation of $j ({W^s_f})$ by the fibers of $p$, and one of these fibers  passes through both $j( f)$ and $i(f)\in \Lambda$. Since $\mathcal R$ uniformly contracts on the truncated fibers and both $i(f)$, $j(f)$ belong to $U_{\Omega, 0}$, it is sufficient to prove that $p^{-1} (W^s_f)$ is a locally stable manifold of its point $j(f)$.
 
 Let $v$ be any vector of unit length such that $(dp)v\in TW^s_f$. We prove that $v$ is uniformly contracted under the action of $d\mathcal R$. Replacing $\mathcal R$ with its iterate, we may and will assume that $\|d\cren^n|_{W^s_f}\| <\tau^n $ and $\|d\mathcal R^n|_{p^{-1}(f)}\|<\mu^n$ for all $n=1, 2, \dots$ and some $\mu, \tau\in(0,1)$. Let us show that the contraction rate of $d\mathcal R$ on $v$ is at least $\max(\mu, \tau)$.
 

There exists a uniform $k$ such that for each $v\in T_{\hat f} D$, we can construct a vector $u$ with the same projection, $(dp)u=(dp) v$, such that $\|u\|<k \|(dp)v\|$; indeed, we can take $u = (dj)(dp)v$.

 Take  $c\gg 2k\|dR\|$ and consider the cones $$\mathcal C^s_{\hat f} = \{\|v \| > c \|(dp)v \| \mid v \in T_{\hat f}D\}$$ ``along'' the fibers of the projection $p$ for all $\hat f\in \Lambda$.   Then  for any vector field in this cone, we have a representation $v=u +w$ where $w\in \Ker (dp)$ and $u$ is as above, $\|u\|\le k\|(dp)v\|\le \frac kc \|v\|$. Thus $$\|w\|\le \|v\|+\|u\| \le \|v\|(1+k/c),$$ and  $$  \|dR v\| \le  \|d\mathcal R   w\| + \|d\mathcal R u\| \le  \mu \|v\|(1+k/c) + \|d\mathcal R\| \cdot  \|v\|\cdot  \frac kc. $$ Hence for any $\eps$, for sufficiently large $c$, we have $\|d\mathcal R v\|< (\mu+\eps)\|v\| $: the contraction rate in the cones  $\mathcal C_{\hat f}^s$ is only slightly bigger than $\mu$.

Now, show that  $\|d\mathcal R^n v\|< C (\max (\tau, \mu) +\eps)^n \|v\|$. If all vectors $d\mathcal R^n v$ for all $n$ stay in the corresponding cones $\mathcal C_{\mathcal R^n \hat f}^s$, the statement follows from the above estimate. Otherwise, consider a block $n, n+1, \dots, n+m$ such that $d\mathcal R^n v$ does not belong to the cone $\mathcal C^s_{\mathcal R^n \hat f}$, but the vectors  $d\mathcal R^{n+1} v, \dots, d\mathcal R^{n+m} v$ do. It is sufficient to prove an estimate for each (finite or infinite) block of this form. 

Since $d\mathcal R^n v $ is not in the cone, we have $\|d\mathcal R^n v\|<c   \|(dp)d\mathcal R^n v \|< \tau^n \|dp(v)\| < C \tau^n \cdot \|v\|$ and the statement holds for $n$. Further, $$\|d\mathcal R^{n+j} v\| < (\mu+\eps)^{j-1} \cdot \|d\mathcal R\| \cdot \|d\mathcal R^n v\| < \tilde C\tau^n (\mu+\eps)^{j-1}$$ for $j=1, \dots, m$, thus the estimate holds for the block $n, n+1, \dots, n+m$.

 
 
This produces the codimension-1 stable distribution for $\mathcal R$ in $D_0$.

\textbf{One-dimensional unstable distribution for $\mathcal R|_{D_0}$.}


 We will construct unstable cones for this operator. Let $f\in \mathcal I$, $\hat f = i(f)\in \Lambda$. Let $TW^u_f$ be  the unstable distribution for $\cren$. Let $p_{u}\colon T\mathcal D_{\eps}^{cr}\to T\mathcal D_{\eps}^{cr} $ be the projection onto $TW^u_f$  along the stable distribution $TW^s_f$ , and define uniformly bounded linear functionals $L_{\hat f}$ by  $L_{\hat f}= p_u \circ dp$. Let $u_{\hat f} \in  (dj)(TW^u_f)$ be the vectors that belong to the lift of the distributions  $TW^u_f$  to the space $D_0$ constructed as in Lemma \ref{lem-lift-cr}, and normalize them so that $L_{\hat f} u_{\hat f}=1$; note that the norms $\|u_{\hat f}\|$ are uniformly bounded.

 Then $\Ker L_{\hat f}$ is a stable distribution for $\mathcal R$, thus replacing $\mathcal R$ with its iterate we can achieve $$\|d\mathcal R|_{\Ker\, L_{\hat f}}\|<\lambda_3<1;$$ $TW^u_f$  is the unstable distribution for $\cren$, thus  replacing $\mathcal R$ with its iterate we can guarantee that $$L_{\mathcal R {\hat f} }(d\mathcal R u_{\hat f}) =p_u (d\cren (dp) u_{\hat f})  >\lambda_1\gg 1.$$

 Now, it is easy to check that the cones  $$\mathcal C^{uu}_{\hat f} = \{|L_{\hat f}v|>c\|v\|\}$$ are invariant unstable cones for $\mathcal R$ for sufficienly small uniformly chosen $c$. Note that these cones project to the unstable cones of $\cren$.

 Indeed, let $v$ belong to this cone, and represent it as $v = L_{\hat f}(v) \cdot u_{\hat f}+w $ where $w$ belongs to $\Ker L_{\hat f}$. Then $$\|w\|\le \|v\| + |L_{\hat f}(v) |\cdot \|u_{\hat f}\| \le  |L_{\hat f}(v) | (\frac 1c + \|u_{\hat f}\|), $$    $$|L_{\mathcal R{\hat f}} (d\mathcal R v)| = |L_{\hat f}(v) \cdot L_{\mathcal R{\hat f}}d\mathcal R u_{\hat f}|\ge   |L_{\hat f} (v) | \cdot \lambda_1$$ and $$c\|d\mathcal Rv\| = c\| L_{\hat f}(v) \cdot d \mathcal R u_{\hat f} +d\mathcal R w \|  \le  c|L_{\hat f}(v)| \cdot (\|d\mathcal R\|\cdot \|u_{\hat f}\| +\lambda_3 \left(\frac 1c + \|u_{\hat f}\|)\right) ,$$ hence the cones $\mathcal C^{uu}_{\hat f}$ are invariant for small uniform $c$. Also, since   $\|d\mathcal R^nv\|  \ge  |L_{\hat f}(v)| \cdot \| d \mathcal R^n u_{\hat f} \| - \lambda_3^n (\frac 1c + \|u_{\hat f}\|))$,  by using this inequality for sufficiently large iterate of $\mathcal R$ we conclude that high iterates of $\mathcal R$ uniformly expand in the cones.


 

Now, the unstable distribution $\eta_{\hat f}$ is constructed as the intersection of images of the unstable cones $\mathcal C^{uu}_{\hat f}$.
 Since the projections of the cones (and their iterates) to $T\mathcal D^{cr}_{\eps}$ are unstable cones (and their iterates) for $\cren$,  the unstable distribution for $\mathcal R$ projects to the unstable distribution for $\cren.$

 Since in the unstable cones, we have both lower and upper estimates on $\|v\|$ in terms of  $L_{\hat f} (v)$, and thus in terms of $\|(dp)v\|$, the minimal expansion rates along these distributions coincide.


\textbf{Two-dimensional unstable distribution for $\mathcal R$.}

Let $\tilde \eta_{\hat f}$ be unit vectors that belong to the distribution $\eta_{\hat f}$. 
Let $v\in T_{\hat f} D$, ${\hat f}\in D_0$,  be represented as a sum $v=x \cdot \tilde \eta_{\hat f} + y \cdot \frac{\partial }{\partial a}+z $, where $z\in T W^s_{\hat f}$ and $\frac{\partial }{\partial a}$ is a change of the coordinate $a$ in the space of triples. Recall that $a$ is multiplied by $\Psi'(0)^2$ under $\mathcal R$. Due to Lemma \ref{lem-Psi-estim}, by replacing $\mathcal R$ with its iterate we can guarantee that $\Psi'(0)^2>\mu >1$. Thus $$d \mathcal R (\frac{\partial }{\partial a}) = k \tilde \eta_{\hat f} + l \frac{\partial }{\partial a} +m $$ where $|l|>\mu$. Since $\|dR\|$ is uniformly bounded, there is a uniform bound $\max(|k|, |l|, \|m\|) <K$. Also, $\|d\mathcal R z\| < t \|z\|$ for $t<1$, due to the uniform contraction on $ W^s_{\hat f}$. Now, it is easy to check that
$$\mathcal C^u_{\hat f} = \{v=x \cdot \tilde \eta_{\hat f} + y \cdot\frac{\partial }{\partial a}+z \mid   |x|+c|y|>\|z\|\}$$
is a family of  unstable invariant cones for $d\mathcal R$ if $ c$ is chosen sufficiently big.  This  implies that $d\mathcal R$  has  a two-dimensional unstable distribution transversal to $D_0$ that  contains $\eta_{\hat f}$.


In more detail, assume $d\mathcal R \tilde \eta_{\hat f} = \lambda_{\hat f}  \tilde \eta_{\mathcal R{\hat f}}$ where $|\lambda_{\hat f}|>1$.
  The image of the vector $v=x \tilde \eta_{\hat f}+y \frac{\partial }{\partial a}  +z$ has coordinates  $$\tilde x = \lambda_{\hat f} x+yk,\; \tilde y = ly, \; \tilde z = d\mathcal R z +  y m .$$

  Now, $$\|d\mathcal R z + y m\| <t \|z\|+K|y| < t (|x|+c|y|)+K|y| $$ and we also have
  $$|\lambda_{\hat f} x + y k|+c|l y| > |\lambda_{\hat f}|\cdot | x|+ |y| (c\mu-K) .$$ For a large $c$, the second quantity is bigger and thus the cone field is invariant.

  It remains to prove that the vectors in the cone are uniformly expanded. Indeed, for large $c$ and small $\delta>0$,
\begin{eqnarray*}
  |\lambda_{\hat f} x+yk| + c| l y|+ \frac \delta 2\|d\mathcal R z + y m\|  &>&  (1+\delta) (|x|+c|y|)\\&>& \left(1+\frac \delta 3\right)\left(|x|+c|y| +\frac \delta 2\|z\|\right) ,
\end{eqnarray*}
  that is, the vectors are uniformly  stretched in the norm $|x|+c|y| + \frac{\delta}{2} \|z\|$.
  This implies the existence of the two-dimensional unstable distribution $l_{\hat f}$ that uniformly expands under the action of $\mathcal R$ and depends continuously on a point $\hat f\in \Lambda$. Thus $\mathcal R$ is  uniformly hyperbolic on $\Lambda$.

 \textbf{Second one-dimensional unstable distribution for $\mathcal R$.}

  If $\lambda_1>\lambda_2$, the existence of the unstable distribution uniformly transversal to $D_0$ follows from standard results on operators in  $\bbR^2$. In more detail, one can choose bases in the two-dimensional unstable distributions $l_{\hat f}, l_{\mathcal R\hat f}, l_{\mathcal R^2 \hat f}, \dots $ for $\mathcal R$ so that the first basis vector belongs to the distribution  $\eta_{\mathcal R^n \hat f}$ and its length is uniformly bounded above and below,  while  the second basis vector has a unit projection to $\frac{\partial}{\partial a}$ and its length is uniformly bounded above and below. Then the matrices of $d\mathcal R^n|_{l_{\mathcal R^m \hat f}}$ in these bases have the form $\begin{pmatrix}\lambda_{m,n} & \tau_{m, n} \\ 0 & \mu_{m,n}\end{pmatrix}$ with 
  $$\frac 1C \|d\mathcal R^n|_{\eta_{\mathcal R^m \hat f }}\|<\lambda_{m,n} < C \|d\mathcal R^n|_{\eta_{\mathcal R^m \hat f }}\|\text{ and  }\mu_{m, n} = (\Psi_{m, n}'(0))^2.$$

Due to the definition of $\lambda_1, \lambda_2$, there exists $n$ such that for all $m$, $\lambda_{m, n}>  (\lambda_1-\eps)^n >(\lambda_2+\eps)^n > \mu_{m, n}$. Fix $n$ and increase the lengths of  second basis vectors in $l_{\mathcal R^k \hat f}$ so that $$\tau_{m, n} < (\lambda_1-\eps)^n - (\lambda_2+\eps)^n < \lambda_{m, n}-\mu_{m,n};$$ this is possible since $\|d\mathcal R^n\|$, and thus $\tau_{m,n}$, are uniformly bounded. 
 

Now it is easy to see that  the cones $|x|<|y|$ in $l_{\mathcal R^{m}\hat f}$ are invariant  under $(d\mathcal R)^{-n }$, thus under $d\mathcal R^{-nk}$ for all $k$ (note that $\mathcal R$ is invertible on $l_{\mathcal R^m \hat f}$). Consider their images under $d\mathcal R^{-kn}$ in $l_{\hat f}$. The intersection of this sequence of embedded cones contains vectors $v$ such that their images under $d\mathcal R^{kn}$ stay in $|x|<|y|$, and thus $\|d\mathcal R^{kn} v\|<c \mu_{0,kn}\|v\|$; since $\lambda_{m,n} \gg \mu_{m,n}$, this vector $v\in l_{\hat f}$ is unique up to rescaling. Its images  under $d\mathcal R$ provide us with an invariant one-dimensional distribution in $l_{\mathcal R^k \hat f}$ uniformly transversal to $\eta_{\hat f}$, such that the maximal expansion rate along this distribution equals $$\underset{n\to +\infty}{\limsup} \sup_{m\ge 0} \sqrt[n]{\mu_{m, n}} =  \lambda_2.$$
  

\textbf{One-dimensional unstable distributions are generated by real vector fields}

Note that the above constructions of unstable cone fields  work equally well in the space of $\bbR$-preserving triples $D^{\bbR}$. In particular, $\eta_f$ is generated by an  $\bbR$-preserving vector field, and $j$ lifts $\bbR$-preserving maps to $\bbR$-preserving triples. Thus the resulting unstable direction fields are generated by some vectors from $TD^{\bbR}$.  




 \end{proof}

The following lemma removes the assumption $\lambda_1>\lambda_2$ using the properties of $(-1)$-measure; it will not be used in the proofs of our main results, since  the assumption $\lambda_1>\lambda_2$ is still necessary to prove smoothness of Arnold tongues. 
\begin{lemma}
\label{lem-weakstable}
The statements of the previous lemma hold true without the assumption $\lambda_1>\lambda_2$: the one-dimensional unstable distribution $\xi_{\hat f}$ exists for all $\hat f\in \Lambda$ and belongs to the codimension-1 subspace $L_{\hat f} v=0$ given by
$$L_{\hat f} v =  \int_{\bbR/\bbZ} (dp)v|_{(p(\hat f))^{-1}(z)} d\mu_{p(\hat f)}=0.$$ 
\end{lemma}
\begin{proof}
Note that  codimension-1 spaces $\{L_{\hat f} v=0\}$ are preimages under $p$ of the spaces $\{L_f v=0\}$ in $\mathcal D_{\eps}$. Recall that for circle diffeomorphisms, the condition $L_f v=0$ defines tangent spaces to $\{\rot f=\alpha\}$, thus $\mathcal R$ takes $\{L_{\hat f} v=0\}$ to $\{L_{\mathcal R\hat f} v=0\}$ if $p(\hat f)$ is a circle diffeomorphism. The limit transition, possible due to Theorem \ref{th-C1}, implies that the same holds if $p(\hat f)$ is a cubic critical circle map, in particular for all  $\hat f \in \Lambda$. Also, Theorem \ref{th-C1} implies that $\{L_{\hat f}v=0\}$ intersects $D_0$ on the stable subspace of $\mathcal R$, thus is uniformly transversal to $\eta_{\hat f}$.

The previous lemma implies that $\mathcal R$ has a two-dimensional unstable  distribution $l_{\hat f}$. The intersection of $l_{\hat f}$ with $\{L_{\hat f} v=0\}$ is a one-dimensional invariant distribution $\xi_{\hat f}$ that is uniformly transversal to $\eta_{\hat f}$. Since $\mathcal R$ multiplies the projection of any vector to $d/da$ by $\Psi'(0)$, the maximal expansion rate along $\xi_{\hat f}$ is the same as in the previous lemma. This completes the proof.
\end{proof}


