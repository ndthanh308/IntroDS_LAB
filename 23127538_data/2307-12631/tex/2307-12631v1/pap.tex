% ****** Start of file apssamp.tex ******
%
%   This file is part of the APS files in the REVTeX 4.2 distribution.
%   Version 4.2a of REVTeX, December 2014
%
%   Copyright (c) 2014 The American Physical Society.
%
%   See the REVTeX 4 README file for restrictions and more information.
%
% TeX'ing this file requires that you have AMS-LaTeX 2.0 installed
% as well as the rest of the prerequisites for REVTeX 4.2
%
% See the REVTeX 4 README file
% It also requires running BibTeX. The commands are as follows:
%
%  1)  latex apssamp.tex
%  2)  bibtex apssamp
%  3)  latex apssamp.tex
%  4)  latex apssamp.tex
%
\documentclass[twocolumn, prl, aps, superscriptaddress, longbibliography,showpacs,amsmath,amssymb,floatfix
]{revtex4-1}

\usepackage{graphicx}% Include figure files
\usepackage{dcolumn}% Align table columns on decimal point
\usepackage{bm}% bold math
\usepackage{graphicx}                                               
\usepackage{amssymb}
\newtheorem{theorem}{Theorem}
\usepackage{amsmath}
\usepackage{epsfig}
\usepackage{xcolor}
\usepackage{tabu}
\usepackage{mathtools}
\usepackage[colorlinks,linkcolor=blue,anchorcolor=blue,citecolor=blue,urlcolor=blue]{hyperref}
\usepackage{physics}
\usepackage{float}
\usepackage{diagbox}
\usepackage{inputenc}
%\usepackage{hyperref}% add hypertext capabilities
%\usepackage[mathlines]{lineno}% Enable numbering of text and display math
%\linenumbers\relax % Commence numbering lines

%\usepackage[showframe,%Uncomment any one of the following lines to test 
%%scale=0.7, marginratio={1:1, 2:3}, ignoreall,% default settings
%%text={7in,10in},centering,
%%margin=1.5in,
%%total={6.5in,8.75in}, top=1.2in, left=0.9in, includefoot,
%%height=10in,a5paper,hmargin={3cm,0.8in},
%]{geometry}

\begin{document}

\title{Fate of localization in coupled free chain and disordered chain}
\author{Xiaoshui Lin}
\affiliation{CAS Key Laboratory of Quantum Information, University of Science and Technology of China, Hefei, 230026, China}
\author{Ming Gong}
\email{gongm@ustc.edu.cn}
\affiliation{CAS Key Laboratory of Quantum Information, University of Science and Technology of China, Hefei, 230026, China}
\affiliation{Synergetic Innovation Center of Quantum Information and Quantum Physics, University of Science and Technology of China, Hefei, Anhui 230026, China}
\affiliation{Hefei National Laboratory, University of Science and Technology of China, Hefei 230088, China}

\date{\today}% It is always \today, today,
             %  but any date may be explicitly specified

\begin{abstract}
It has been widely believed that almost all states in one-dimensional (1d) disordered systems with short-range hopping and uncorrelated random potential are localized. 
Here, we consider the fate of these localized states by coupling between a disordered chain (with localized states) and a free chain (with extended states), showing that states in the overlapped and un-overlapped regimes exhibit totally different localization behaviors, which is not a phase transition process. 
In particular, while states in the overlapped regime are localized by resonant coupling, in the un-overlapped regime of the free chain, significant suppression of the localization with a prefactor of $\xi^{-1} \propto t_v^4/\Delta^4$ appeared, where $t_v$ is the inter-chain coupling strength and $\Delta$ is the energy shift between them. 
This system may exhibit localization lengths that are comparable with the system size even when the potential in the disordered chain is strong.
We confirm these results using the transfer matrix method and sparse matrix method for systems $L \sim 10^6 - 10^9$.
These findings extend our understanding of localization in low-dimensional disordered systems and provide a concrete example, 
which may call for much more advanced numerical methods in high-dimensional models. 
\end{abstract}

\maketitle

Anderson localization (AL), which describes the phenomenon that the disorder totally suppresses the diffusion of the system, has attracted a great deal of attention for many decades \citep{anderson_absence_1958, Lee1985Disordered, Evers2008Anderson, Lagendijk2009Fifty, Roati2008Anderson, Segev2013Anderson}.
It has been found that the spatial dimension plays an essential role in AL \citep{Mott1961Theory, Abrahams1979Scaling, Ishii1973Localization, Thouless1972Relation, Thouless1973Localization}.
In the one-dimensional (1d) tight-binding model with random potential, the localization length is given by \citep{Thouless1981Conductivity, Heinrichs2002Localization}
\begin{equation}
\xi_0^{-1}(E) = {v^2 \over 8 t^2 - 2 E^2} ={V^2 \over 96 t^2 - 24 E^2},
\label{eq-thouless-expression}
\end{equation} 
where $v^2=\langle v_i^2\rangle$ is the variance of the potential $v_i \in U[-V/2, V/2]$, with $V$ being the disorder strength, $t$ is the hopping strength between the neighboring sites and $E$ is the eigenvalue (see Eq. 14 in Ref. \citep{Thouless1981Conductivity} for more details ). 
When the system size $L$ is much larger than $\xi_0$, $L \gg \xi_0$, localization of wave functions can be observed and the system is in the localized phase without conductance. 
In the presence of weak disorder, $V \ll t$, we have $|E| < 2t$, thus all states should be localized with $\xi_{0}^{-1}>0$. 
For example, when $V \sim 0.1t$, $\xi_0 \sim 10^4$, which can be easily confirmed by numerical simulation.
The fate of the states in 1d systems will be changed fundamentally with incommensurate potentials \citep{Biddle2009Localization, Biddle2010Predicted, Ganeshan2015Nearest, Wang2020one-dimensional}, long-range correlated disorders \citep{Izrailev1999Localization,  Moura1998Delocalization, Dietz2011Microwave,Delande2014Mobility, Johnston1986Localization} and many-body interactions \citep{Bordia2016Coupling, Thiery2018Many, Chiew2023Stability, Leonard2023Probing, Hyunsoo2023Many-Body}. 

% Figure environment removed

While the physics of disordered models have been widely discussed \citep{Ishii1973Localization, Brouwer1998Delocalization, Heinrichs2002Localization, Evers2008Anderson}, the fate of localization by the coupling of extended and localized states are much lesser investigated. 
We are interested in this issue for the reason of the dilemma that (I) the random uncorrelated potential induces localization for the extended states in 1d systems \citep{Abrahams1979Scaling}; 
and (II) the hybridization between localized and extended states may lead to delocalization  \citep{Mott1987Mobility}.
The interplay between these two mechanisms may lead to different physics. 
In this work, we propose a coupled disordered model (Fig. \ref{fig-model}) to address this problem.
Our model is constructed from one free chain (with all states extended) and one disordered chain (with all states localized). Two major conclusions have been established: 
(1) While all states exhibit localization in the presence of inter-chain coupling, their localization length exhibits a distinct difference in the overlapped spectra and 
un-overlapped spectra, which is not a phase transition process; (2) The localization length for states in the un-overlapped regime 
of the free chain is significantly suppressed given by 
\begin{equation}
\xi^{-1}(E) \simeq {t_\text{v}^4 V^2 \over (96 t^2 - 24 (E - \Delta)^2) \Delta^4} = {t_\text{v}^4 \over \Delta^4} \xi_0^{-1}(E - \Delta), 
\label{eq-loc-length}
\end{equation}
in the limit when $\Delta \gg |V|$. 
Here $t_\text{v}$ is the inter-chain coupling, and $\Delta$ is the energy shift between the free and disordered chains. 
Thus localization is greatly suppressed when $t_\text{v} \ll \Delta$.
For instance, when $t_\text{v} = 0.1t$ and $\Delta =10t$, the localization length can be suppressed by eight orders of magnitude.
We examine the above conclusions using the transfer matrix method and the sparse matrix method with system sizes  $L\sim 10^6-10^9$.
Our results show that the inter-chain coupling, disorder strength, and energy shift are the three major factors influencing the localization length of states in the un-overlapped regime.
In the regime when $\xi \gtrsim L$, we can understand the localization of wave functions with the following general theorem based on the large amount of researches; see review articles \citep{Ishii1973Localization, Beenakker1997Random-matrix, Belitz1994Anderson, Evers2008Anderson}. 

% Figure environment removed

\textit{Theorem}: In 1d disordered systems with short-range hopping and uncorrelated random potential, almost all states are localized in the thermodynamic limit ($L \rightarrow \infty$). 

The exact proof of this theorem is still a great challenge at the present stage; 
however, it can be understood intuitively from the observation by Mott \textit{et al} \citep{Mott1961Theory}, the argument by Thouless \citep{thouless1974electrons}, and the scaling argument by Abrahams \textit{et al}, in which the $\beta$ function is always negative \citep{Abrahams1979Scaling}. 
This theorem is also addressed by the celebrated Dorokhov-Mello-Pereyra-Kumar equation \citep{Beenakker1997Random-matrix} and the non-linear sigma model
\citep{efetov1983supersymmetry}.
Here, not all states are localized because, in the 1d model with off-diagonal random potential, the state with $E = 0$ is extended while all the other states are localized \citep{George1976Extended}.
The mathematicians have great interest in this problem and have proved this theorem with random potentials \citep{Carmona1987Anderson, Kunz1980Sur, Frohlich1983Absence, Pastur1980Spectral}, showing of no continuous spectra for extended states. 
This theorem will play a determinative role for localization when $\xi \gtrsim L$, which is beyond the capability of the numerical simulations. 

\textit{Physical model and methods}:
We consider the following coupled disordered model (see Fig. \ref{fig-model})
\begin{equation}
    H = H_0 + H_1 + \sum_{m,\sigma} t_\text{v} a_{m,\sigma}^{\dagger}a_{m,\bar{\sigma}},
    \label{eq-model}
\end{equation}
where $H_\sigma = \sum_{m} t_{\sigma} a_{m,\sigma}^{\dagger}a_{m+1,\sigma} + \mathrm{h.c.} + V_{m,\sigma}a_{m,\sigma}^{\dagger}a_{m,\sigma}$, with $H_0$ for the free chain and $H_1$ for the disordered chain. 
Here $V_{m, 0} = \Delta$ is the energy shift of the free chain $H_0$ and 
$V_{m, 1} \in U[- V/2, V/2]$ is the random potential in the disordered chain $H_1$, and $\bar{\sigma} = 1-\sigma \in \{ 0, 1\}$.
When $t_\text{v} = 0$, this model is reduced to a free chain with all states extended, and a disordered chain with all states localized.
By changing $\Delta$, the energy spectra of these two chains can be un-overlapped (Fig. \ref{fig-wave-function} (a)) or overlapped (Fig. \ref{fig-wave-function} (b)).
By the theorem, all states should be localized in the presence of inter-chain coupling (with $t_v \ne 0$). The fundamental question is what are the quantitative differences between the states in the overlapped regime and the un-overlapped regime of the coupled model during localization? 

We apply the transfer matrix and sparse matrix methods, whose available size is $L \sim 10^{6} - 10^{9}$, to understand the localization of wave functions in these two regimes.
In the transfer matrix method \citep{Hoffmanbook}, the Lyapunov exponent $\gamma(E) = \xi(E)^{-1}$ is defined as the smallest positive eigenvalue of the matrix 
\begin{equation}
    \Gamma(E) = \lim_{L\rightarrow \infty} \frac{1}{2L}\ln(T_1^{\dagger} \dots T_L^{\dagger}T_L\dots T_1),
    \label{eq-gammaE}
\end{equation}
where $T_i$ is the transfer matrix at the $i$-th site. 
From the Oseledets ergodic theorem \citep{Oseledets1968Multiplicative, Raghunathan1979Proof}, the above multiplication of transfer matrices is converged when $L \rightarrow \infty$.
When $\gamma(E) \neq 0$, the state with eigenvalue $E$ is localized. 
In the sparse matrix method, we use the shift-invert method \citep{Pietracaprina2018Shift, Luitz2015Many-body} to obtain about $N_E = 20$ eigenstates $|\psi_{E_i}\rangle$ with eigenvalues $E_i$ around a given $E$ and define the averaged inverse participation ratio (IPR) as   \citep{Evers2008Anderson} 
\begin{equation}
    \langle \text{IPR}\rangle_E = {1\over N_E} \sum_{i=1}^{N_E} \sum_{m=0}^{2L-1}|\psi_{E_i}(m)|^4. 
    \label{eq-iprE}
\end{equation}
For the extended state, $\langle \text{IPR}\rangle_E \propto L^{-1}$ and for the localized state, $\langle \text{IPR}\rangle_E$ is finite.
Furthermore, we can define the fractal dimension $\tau_2(E, L) = -\ln(\langle \text{IPR}\rangle_E)/\ln(L)$ and its limit $\tau_2(E) = \lim_{L\rightarrow\infty} \tau_2(E, L)$.
We have $\tau_2(E) = 0$ for localized states, $\tau_2(E) = 1$ for extended states, and $0<\tau_2(E)<1$ for critical states, respectively \citep{Hiramoto1989Scaling, Evers2008Anderson, Lin2022General}.
We note that the IPR should be proportional to the Lyapunov exponent $\gamma(E)$ for an exponentially localized state $\psi_m \sim e^{-|m|/\xi}$, with $\text{IPR} \propto \xi^{-1} = \gamma$, in the limit $L \gg \xi$. 

\textit{Physics in the overlapped and un-overlapped regimes}:
Although all states are expected to be localized in our model, the effect of inter-chain coupling in the overlapped regime and un-overlapped regime should be different, leading to distinct localization behavior.
We consider two different cases, which are shown in Fig. \ref{fig-wave-function} (a) and (b). 
In the first case, the spectra in the two chains are un-overlapped to avoid the resonant coupling between the extended and localized states; while in the second case, resonant coupling is induced in their overlapped regime. Furthermore, we present their typical wave functions in these regimes in Fig. \ref{fig-wave-function} (c) and (d) with $t_\text{v}=0.1$.
The results show that the wave functions in the un-overlapped regime of the disordered chain are exponentially localized with localization length around unity (see the state with $E=0$ in Fig. \ref{fig-wave-function} (c) and (d)).
The wave functions in the overlapped regime are also exponentially localized, however, with localization length $\xi \sim 0.05L$, which is much larger than the localized states in the un-overlapped regime of the disordered chain.
Strikingly, the wave functions in the un-overlapped regime of the free chain are extended even when the system size $L=10^6$. 
Similar features can be found when $L$ is increased to $L \sim 10^9$ for smaller $t_\text{v}$. This seems to contradict with the general theorem. This dilemma is the major concern of this work. 

Next, we investigate the asymptotic behavior of the wave function using the transfer matrix method.
In Fig. \ref{fig-localization-length} (a) and (b), we present the Lyapunov exponent $\gamma(E)$ against the energy $E$ for a system with size $L=10^9$.
When $t_\text{v} = 0$, the states in the free chain are extended and the states in the disordered chain are localized.
When $t_\text{v} = 0.1$, all states become localized with $\gamma(E)>L^{-1}$.
However, there are three distinct energy regimes for $\gamma(E)$, corresponding to the overlapped and un-overlapped regimes.
In the overlapped regime, we have $\gamma(E) \sim 10^{-4}$, while in the un-overlapped regime, we have $\gamma(E) \sim 10^{0}$ (disordered chain) or $\gamma(E) \sim 10^{-7}$ (free chain). 
These distinct behaviors are unique features of the coupled disorder model, which should not be regarded as some kind of phase transition between extended and localized states; see below. 

% Figure environment removed

We further characterize the three energy regimes using IPR and fractal dimension $\tau_2(E, L)$, with results presented in Fig. \ref{fig-localization-length} (c) and (d).
It is found that $\tau_2(E, L) \rightarrow 0$ for states in the overlapped regime and un-overlapped regime of the disordered chain, indicating localization. 
However, $\tau_2(E, L) \rightarrow 1$ for states in the un-overlapped regime of the free chain, contradicting the previous theorem at first sight.
This contradiction is actually a finite-size effect since $L=2^{18} \sim 10^5 <\xi\sim 10^7$.
Thus, it is expected that all these states will be localized from the general theorem in the thermodynamic limit ($L \gg \xi$, or equivalently, $L\gamma(E) \gg 1$).
This also clarifies the previous disagreement presented in Fig. \ref{fig-wave-function}.

\textit{Origin of the suppressed localization length}:
The above results raise some fundamental questions that need to be addressed much more carefully. 
In the overlapped regime, the localized states and extended states are coupled through resonant coupling because their energy is close to each other.
From perturbation theory \cite{anderson_absence_1958}, all the higher-order terms will become important, leading to significant modification of the wave functions for localization.
In the un-overlapped regime of the free chain, wave function localization is greatly suppressed by a different mechanism.

To this end, we first consider the localization in the following minimal model 
\begin{eqnarray}
    H = H_0 + (t_\text{v} a_{m,0}^{\dagger}a_{m,1} + \mathrm{h.c.}) + V_{m,1} a_{m,1}^{\dagger}a_{m,1}.
    \label{eq-minimal-model}
\end{eqnarray}
As compared with Eq. \ref{eq-model}, here we set $t_1 = 0$ and $t_0 = t = 1$ (see Fig. \ref{fig-full-localized-chain} (a)). 
When $t_\text{v}=0$, the eigenstates of component $\sigma=1$ are fully localized at one site with eigenvalue distributed in the interval $[-V/2, V/2]$.
On the other hand, the eigenstates of component $\sigma=0$ are $\psi(m) \propto e^{i k m}$ with energy spectra in $[\Delta - 2t, \Delta + 2t]$.
Thus, the energy spectra of each component are well separated when $|\Delta| > |V/2| + 2t$. We focus on the physics in this regime (with $t_\text{v} \neq 0$) for the suppressed localization. 

The central problem is to verify the major result of Eq. \ref{eq-loc-length}.
We employ the sparse matrix method to examine the physic in Eq. \ref{eq-minimal-model}, and the results for $E = \Delta$ are presented in Fig. \ref{fig-full-localized-chain}.
It is found that $\langle \text{IPR} \rangle_E \propto t_\text{v}^4$ with intermediate $t_\text{v}$ with $\xi < L$.
In the small inter-chain coupling limit, the IPR will be saturated to $L^{-1}$ when $\xi \gtrsim L$ due to the finite size effect.
Thus the relation of $\text{IPR} \propto t_\text{v}^4$ always holds in the thermodynamical limit.
In Fig. \ref{fig-full-localized-chain} (c), we also examine the dependence of IPR as a function of energy shift $\Delta$, finding that $\text{IPR} \propto \Delta^{-4}$ in the large $\Delta$ limit. 
However, when $t_\text{v}/\Delta \ll 1$ we have $\xi \gtrsim L$ and saturation of IPR is found again for the same reason as Fig. \ref{fig-full-localized-chain} (b). Combining these two power-law dependence will yields $\xi^{-1} \propto (t_\text{v}/\Delta)^4$ in Eq. \ref{eq-loc-length}, using IPR$\propto 1/\xi$. 

To more accurately describe localization length as a function of eigenvalue $E$, 
we then derive an effective Hamiltonian as  
\begin{eqnarray}
H_{\text{eff}} = \sum_m t a_{m+1,0}^{\dagger}a_{m,0} + \mathrm{h.c.} + \sum_m (\Delta + W_m) a_{m,0}^{\dagger}a_{m,0},
\label{eq-couple-full-localized} 
\end{eqnarray}
where $W_m =  t_\text{v}^2/(\Delta - V_{m,1})$ for states with $E \sim \Delta$. A direct calculation shows that
\begin{eqnarray}
   \langle W_m \rangle &&= {1\over V}\int_{-V/2}^{V/2} W_m dV_{m,1}= \frac{t_\text{v}^2}{V}\ln(\frac{2|\Delta|/V + 1}{ 2|\Delta|/V -1}), \nonumber \\
   \langle W_m W_n \rangle &&= \frac{4t_\text{v}^4}{4\Delta^2 - V^2} \delta_{m,n},
\end{eqnarray} 
with $\langle \cdot \rangle$ represents its disorder averaged value. The variance of $W_m$ can be written as 
\begin{equation}
   v^2 = \langle W_m^2 \rangle - \langle W_m \rangle^2 = t_\text{v}^4(\frac{4}{4\Delta^2-V^2} 
 - f(\Delta, V)),
\end{equation}
with $f(\Delta, V) = (\frac{1}{V}\ln(\frac{2|\Delta|/V + 1}{ 2|\Delta|/V -1}))^2$.
It yields  
\begin{equation} 
v^2 = \frac{ t_\text{v}^4 V^2}{12\Delta^4} + \frac{11 t_\text{v}^4 V^4}{360 \Delta^6} + \mathcal{O}(\Delta^{-6}),
\end{equation}
when $\Delta \gg V$. The leading term yields the localization length in Eq. \ref{eq-loc-length} with a suppressed prefactor of  $t_\text{v}^4/\Delta^4$, which is numerically confirmed in Fig. \ref{fig-full-localized-chain} in a large system. This completes the proof of Eq. \ref{eq-loc-length}. 

% Figure environment removed

The previous conclusion is based on the minimal model with $t_1=0$.
We then move to examine the effect of hopping $t_1$ in the disordered chain, which is expected to extend the wave functions and hence suppressed the localization length in the free chain. 
Thus it is crucial to ask to what extent $t_1$ can influence the localization length $\xi$.
To this end, we fix $\Delta$ and $t_\text{v}$ and change the value of $t_1$, and the results of IPR against $t_1$ are presented in Fig. \ref{fig-hopping}.
We find that the IPR is slightly decreased with the increasing of $t_1$, indicating that $t_1$ is not the essential term for the localization of the free chain.
Therefore, we expect that Eq. \ref{eq-loc-length} serves as a good approximation for the localization length even with finite $t_1$, which accounts for the excellent agreement between the numerical and theoretical results in Fig. \ref{fig-localization-length} (a) and Fig. \ref{fig-loc-length-compare}. 
Finally, we present the relation between the localization length as a function of energy $E$ in a single disordered chain and in a coupled disordered model in Fig. \ref{fig-loc-length-compare}, which further confirms the empirical formulas of Eq. \ref{eq-thouless-expression} and Eq. \ref{eq-loc-length}. In Fig. \ref{fig-localization-length} (b), we have used 
\begin{equation}
\xi^{-1}(E) = (\frac{4}{4\Delta^2-V^2} 
 - f(\Delta, V))\frac{t_\text{v}^4}{8t^2 - 2(E-\Delta)^2},
\label{eq-fullloclength}
\end{equation}
which yields Eq. \ref{eq-loc-length} in the large $\Delta$ limit. We point out that for finite $\Delta$, the higher-order term $\mathcal{O}(\Delta^{-4})$ in $f(\Delta, V)$ is important. Therefore, while the localization length in the overlapped and un-overlapped spectra exhibits distinct behaviors with all wave functions being localized, it is not a phase transition process. 

% Figure environment removed

% Figure environment removed

\textit{Conclusion and discussion}: 
In this work, we present a coupled disordered model by coupling a disordered chain with a free chain, where the localization lengths in the overlapped and un-overlapped regimes differ by several orders of magnitude. 
In the overlapped regime, the states from the free chain are localized by resonant coupling between the localized and extended chains.
However, in the un-overlapped regime of the free chain, while the states are still localized by the general theorem, they exhibit suppressed localization with a prefactor of $ t_\text{v}^4 /\Delta^4$. 
We find that the inter-chain coupling, disorder strength, and energy shift play a leading role in localization, yet the effect of intra-chain hopping $t_1$ in the disordered chain is not significant.
The results presented in this work are in 1d disordered models, and extending this research to higher dimensional models \citep{Tarquini2017Critical, Jani2005Mean-field, Avishai2002New} and many-body models \citep{Nandkishore2015Many-body, Alet2018Many-body, Abanin2019Manybody} is also intriguing, in which we expect the overlapped regime and un-overlapped regime will also exhibit totally different behaviors \citep{Lin2023Singleparticle}.
Furthermore, for a system with a large localization length in the higher-dimensional models, it may call for much more advanced numerical methods. 

These results can be readily confirmed in the states-of-art experiments with ultracold atoms \citep{Roati2008Anderson,gross2017quantum, darkwah2022probing, Mandel2003Coherent, Yang2017Spin-dependent}, in which the two chains can be realized by the hyperfine states. The inter-chain coupling can be realized by Raman coupling and their potential shift is a natural consequence of detuning and Zeeman field. 
In these systems, the wave functions in each chain can be independently realized in the limit $t_v \sim 0$, and their localization can be measured individually using the time-of-flight imaging technique. In recent years, AL in disordered systems has been an important direction in ultracold atoms and huge progress has already been achieved \citep{Roati2008Anderson,  Skipetrov2008Anderson, Kohlert2019Observation, Alex2021Interaction, Wang2022Observation, white2020observation, Dikopoltsev2022Observation} and we expect the experimental confirmation of these results can provide perspicuous evidences for the dilemma of (I) and (II). 

Finally, it is necessary to emphasize that the disordered potential (with short-range correlation) has totally different features from the incommensurate potential.
In the coupled free chain and the incommensurate chain, without the guarantee of the general theorem, one can realize a critical phase in the overlapped spectra \cite{Lin2022General}, in which the overlapped and un-overlapped spectra also exhibit distinct behaviors in localization. 
The similar critical phase by coupling of extended and localized states in the Floquet model with incommensurate potential has also been presented by Roy {\it et al} in Ref. \citep{Roy2018Multifractality}. Here, we present a much-simplified model, which can be solved analytically in the limiting condition, in the hope that these intriguing results to be found in the more complicated coupled many-body models and coupled random matrices \cite{Lin2023Model}. 

\textit{Acknowledgments}: 
This work is supported by the National Natural Science Foundation of China (NSFC) with No. 11774328, and Innovation Program for Quantum Science and Technology (No. 2021ZD0301200 and No. 2021ZD0301500).

\bibliography{ref}

\end{document}


