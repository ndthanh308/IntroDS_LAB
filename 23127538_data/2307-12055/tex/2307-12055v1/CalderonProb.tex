\documentclass[11pt,leqno,twoside]{amsart}

%\documentclass[leqno]{amsart}
\usepackage{fancyvrb}
\usepackage{fancyhdr}
\usepackage{etoolbox}
%\usepackage{Tikz, tcolorbox}
%\usepackage{enumerate}
%\usepackage{tikz}
\usepackage{pgfplots}
%\pgfplotsset{compat=1.18, width = 10cm}
%\usepackage{axis}

%\usepackage{cancel}

\usepackage{amssymb}
\usepackage{placeins}
%\usepackage[pagewise]{lineno}\linenumbers
%\usepackage{german}
\usepackage[english]{babel}
\usepackage{amssymb,amsthm,amsmath,eucal,mathrsfs}
\usepackage{bm}


\setlength{\textwidth}{16cm}

\setlength{\textheight}{21.6cm}

\hoffset=-55pt

\newcommand*{\QEDA}{\hfill\ensuremath{\blacksquare}}%
\newcommand*{\QEDB}{\hfill\ensuremath{\square}}%

%%%%%%%%%%%%%%%%%%%%%%%%%%%%%%%%%%%%
%%%%%%%%%%%%%%%%%%%%%%%%%%%%%%%%%%%%
\usepackage{amsmath}
\usepackage{amsfonts}
\usepackage{float}
\usepackage{amsmath}
\usepackage{amssymb}
\usepackage{graphicx}
%\usepackage[utf8]{inputenc}
%\usepackage[new]{old-arrows}
\usepackage{lscape}
\usepackage{amstext}
\usepackage{amsthm}
\usepackage{color}
\usepackage{float}
%\usepackage[lite]{mtpro2}
%\usepackage{ulem}
%\usepackage{cancel}
%\usepackage{geometry}
\usepackage{mathrsfs}
\usepackage{epsfig}
\usepackage{url}
\usepackage{fancyhdr}
\usepackage{pspicture}
%\usepackage{wrapfig}
\usepackage{graphicx}
%\usepackage{multirow}
%\usepackage{lmodern}
%%%%%%%%%%%%%%%%%%%%%%%%%%%%%%%%%
%%%%%%%%%%%%%%%%%%%%%%%%%%%%%%%%%%%%
%%%%%%%%%%%%%%%%%%%%%%%%%%%%%%%%%%%%







%\usepackage[latin1]{inputenc}  

%\usepackage[T1]{fontenc}       

%\usepackage[ngerman]{babel}

%\usepackage{tikz, subfigure, xcolor} 

%\usepackage{pgfplots}
%\usepackage{Comment}
\usepackage{cite}

\usepackage{amsmath,verbatim}

\usepackage{amsthm}

\usepackage{amssymb}

\usepackage{amsfonts}

%\usepackage{showkeys}

\DeclareRobustCommand{\rchi}{{\mathpalette\irchi\relax}}
\newcommand{\irchi}[2]{\raisebox{\depth}{$#1\chi$}}


%\usepackage{cancel}


\usepackage{dsfont}



\usepackage{hyperref}



\newtheorem{theorem}{Theorem}[section]

\newtheorem{lemma}[theorem]{Lemma}

\newtheorem{proposition}[theorem]{Proposition}

\newtheorem{corollary}[theorem]{Corollary}

\newtheorem{assumption}[theorem]{Assumption}

%\newtheorem{claim}[theorem]{Claim}
\newtheorem{claim}{Claim}
\newtheorem{Hypotheses}{Hypotheses}
%

\theoremstyle{definition}
\newtheorem{definition}[theorem]{Definition}
\newtheorem{remark}[theorem]{Remark}

\newtheorem{example}{Example}[section]

%\newenvironment{Example}{\begin{example}}{\hfill\qed\end{example}}



\numberwithin{equation}{section}



%Mathematische Symbole

\newcommand{\diag}{\mathrm{diag}}      % diagonal

\newcommand{\dist}{\mathrm{dist}}      % distance

\newcommand{\diam}{\mathrm{diam}}      % diameter

\newcommand{\codim}{\mathrm{codim}}    % co-dimension

\newcommand{\supp}{\mathrm{supp}}      % support

\newcommand{\clo}{\mathrm{clo}}        %closure

\newcommand{\beweisqed}{\hfill$\Box$\par} % Beweisende

\newcommand{\Span}{\mathrm{span}}       % span

\renewcommand{\Im}{{\ensuremath{\mathrm{Im\,}}}} %Imagin????rteil nicht alsfraktur

\renewcommand{\Re}{{\ensuremath{\mathrm{Re\,}}}} %Realteil nicht als fraktur

\renewcommand{\div}{\mathrm{div}\,}    %div anstatt geteilt







\providecommand{\norm}[1]{\lVert#1\rVert} %Norm

\providecommand{\abs}[1]{\lvert#1\rvert} % absolut value

%\providecommand{\au}[1]{\underline{a}} % length of the edges



\DeclareMathOperator{\Hom}{Hom}

\DeclareMathOperator{\End}{End} 

\DeclareMathOperator{\Imt}{Im} 

\DeclareMathOperator{\Ran}{Ran} 

\DeclareMathOperator{\ran}{Ran} 

\DeclareMathOperator{\Rank}{Rank} 

\DeclareMathOperator{\sign}{sign} 

\DeclareMathOperator{\sgn}{sign} 

\DeclareMathOperator{\tr}{tr}

\DeclareMathOperator{\Ker}{Ker}

\DeclareMathOperator{\Dom}{Dom}

\DeclareMathOperator{\dom}{dom}

\DeclareMathOperator{\grad}{grad}

\DeclareMathOperator{\Gr}{Gr}





\DeclareMathOperator{\dd}{\mbox{d}} %differential

\DeclareMathOperator{\ddx}{\mbox{dx}} %dx

\DeclareMathOperator{\ddy}{\mbox{dy}} %dx



\DeclareMathOperator{\Sm}{\mathfrak{S}} %coefficient matrix

\DeclareMathOperator{\Cm}{\mathfrak{C}} %coefficient matrix

\DeclareMathOperator{\Jm}{\mathfrak{J}} %coefficient matrix

\DeclareMathOperator{\jm}{\mathfrak{j}} %coefficient matrix







\newcommand{\au}{\underline{a}}

\newcommand{\bu}{\underline{b}}

%mathematische Schriften

\newcommand{\HH}{\mathbb{H}}

\newcommand{\Q}{\mathbb{Q}}

\newcommand{\R}{\mathbb{R}}

\newcommand{\T}{\mathbb{T}}

\newcommand{\C}{\mathbb{C}}

\newcommand{\Z}{\mathbb{Z}}

\newcommand{\N}{\mathbb{N}}

\newcommand{\bP}{\mathbb{P}}

\newcommand{\PP}{\mathbb{P}}

\newcommand{\1}{\mathbb{I}}

\newcommand{\E}{\mathbb{E}}

\newcommand{\EE}{\mathsf{E}}







\newcommand{\fQ}{\mathfrak{Q}}

\newcommand{\fH}{\mathfrak{H}}

\newcommand{\fR}{\mathfrak{R}}

\newcommand{\fF}{\mathfrak{F}}

\newcommand{\fB}{\mathfrak{B}}

\newcommand{\fS}{\mathfrak{S}}

\newcommand{\fD}{\mathfrak{D}}

\newcommand{\fE}{\mathfrak{E}}

\newcommand{\fG}{\mathfrak{G}}

\newcommand{\fP}{\mathfrak{P}}

\newcommand{\fT}{\mathfrak{T}}

\newcommand{\fC}{\mathfrak{C}}

\newcommand{\fM}{\mathfrak{M}}

\newcommand{\fI}{\mathfrak{I}}

\newcommand{\fL}{\mathfrak{L}}

\newcommand{\fW}{\mathfrak{W}}

\newcommand{\fK}{\mathfrak{K}}

\newcommand{\fA}{\mathfrak{A}}

\newcommand{\fV}{\mathfrak{V}}



\newcommand{\fa}{\mathfrak{a}}

\newcommand{\fb}{\mathfrak{b}}

\newcommand{\fh}{\mathfrak{h}}

\newcommand{\ft}{\mathfrak{t}}

\newcommand{\fs}{\mathfrak{s}}

\newcommand{\fv}{\mathfrak{v}}

\newcommand{\fq}{\mathfrak{q}}

%%%%%%%%%%%%%%%%%%%%%%%%%%%%%%%%

%Kaliegraphie

%%%%%%%%%%%%%%%%%%%%%%%%%%%%%%%%

\newcommand{\cA}{{\mathcal A}}

\newcommand{\cB}{{\mathcal B}}

\newcommand{\cC}{{\mathcal C}}

\newcommand{\cD}{{\mathcal D}}

\newcommand{\cE}{{\mathcal E}}

\newcommand{\cF}{{\mathcal F}}

\newcommand{\cG}{{\mathcal G}}

\newcommand{\cH}{{\mathcal H}}

\newcommand{\cI}{{\mathcal I}}

\newcommand{\cJ}{{\mathcal J}}

\newcommand{\cK}{{\mathcal K}}

\newcommand{\cL}{{\mathcal L}}

\newcommand{\cM}{{\mathcal M}}

\newcommand{\cN}{{\mathcal N}}

\newcommand{\cO}{{\mathcal O}}

\newcommand{\cP}{{\mathcal P}}

\newcommand{\cQ}{{\mathcal Q}}

\newcommand{\cR}{{\mathcal R}}

\newcommand{\cS}{{\mathcal S}}

\newcommand{\cT}{{\mathcal T}}

\newcommand{\cU}{{\mathcal U}}

\newcommand{\cV}{{\mathcal V}}

\newcommand{\cX}{{\mathcal X}}

\newcommand{\cW}{{\mathcal W}}

\newcommand{\cZ}{{\mathcal Z}}

%Doppelt definieret symbole

\newcommand{\Me}{{\mathcal M}}

\newcommand{\He}{{\mathcal H}}

\newcommand{\Ke}{{\mathcal K}}

\newcommand{\Ge}{{\mathcal G}}

\newcommand{\Ie}{{\mathcal I}}

\newcommand{\Ee}{{\mathcal E}}

\newcommand{\De}{{\mathcal D}}

\newcommand{\Qe}{{\mathcal Q}}

\newcommand{\We}{{\mathcal W}}

\newcommand{\Be}{{\mathcal B}}



\newcommand{\lpso}{L^p_{{\sigma}}(\Omega)}



\newcommand\restr[2]{{% we make the whole thing an ordinary symbol
  \left.\kern-\nulldelimiterspace % automatically resize the bar with \right
  #1 % the function
  \vphantom{\big|} % pretend it's a little taller at normal size
  \right|_{#2} % this is the delimiter
  }}





% David's macros

\newcommand{\cf}{\emph{cf.}}

\newcommand{\ie}{{\emph{i.e.}}}

\newcommand{\eg}{{\emph{e.g.}}}

\usepackage{eucal}

\newcommand{\verts}{{\mathcal V}}

%%%%%%%%%%%%%%%%%%%%%%%%%%%%%%%%%%%%%%%%%%%%%%%%%%%%%%
\date{\today}   

 

%\title[Photo-acoustic inversion using plasmonics]{Photo-acoustic inversion using plasmonic contrast agents:\\ The full Maxwell model}

\title[The Calderón problem revisited]{The Calderón problem revisited:\\
Reconstruction with resonant perturbations}


\author[Ghandriche and Sini]{Ahcene Ghandriche  $^*$ and Mourad Sini$^{\ddag}$}
\thanks{$^*$ Nanjing Center for Applied Mathematics, Nanjing, 211135, People’s Republic of China. Email: gh.hsen@njcam.org.cn}
\thanks{$^{\ddag}$ RICAM, Austrian Academy of Sciences, Altenbergerstrasse 69, A-4040, Linz, Austria. Email: mourad.sini@oeaw.ac.at. This author is partially supported by the Austrian Science Fund (FWF): P 30756-NBL}


\begin{document}

\subjclass[2010]{35R30, 35C20}
\keywords{Inverse problems, Neumann-to-Dirichlet map, Newtonian operator, Acoustic imaging, Asymptotic expansions, Spectral theory, Lippmann-Schwinger equation, droplets.}
\maketitle



\begin{abstract}
The original Calderón problem consists in recovering the potential (or the conductivity) from the knowledge of the related Neumann to Dirichlet map (or Dirichlet to Neumann map). Here, we first perturb the medium by injecting small-scaled and highly heterogeneous particles. Such particles can be bubbles or droplets in acoustics or nanoparticles in electromagnetism. They are distributed, periodically for instance, in the whole domain where we want to do reconstruction. Under critical scales between the size and contrast, these particles resonate at specific frequencies that can be well computed. 
Using incident frequencies that are close to such resonances, we show that
\bigskip

\begin{enumerate}
\item the corresponding Neumann to Dirichlet map of the composite converges to the one of the homogenized medium. In addition, the equivalent coefficient, which consist in the sum of the original potential and the effective coefficient, is negative valued with a controlable amplitude.
\bigskip

\item as the equivalent coefficient is negative valued,  then we can linearize the corresponding Neumann to Dirichlet map using the effective coefficient's amplitude.
\bigskip

\item from the linearized Neumann to Dirichlet map, we reconstruct the original potential using explicit complex geometrical optics solutions (CGOs).

\end{enumerate}
\end{abstract}


\bigskip

\section{Introduction and statement of the results}

\subsection{Introduction}
~\\

The original Calderón problem stated in the acoustic framework reads as follows. Let $n:=c^{-1}$ be the index of refraction, where $c$ stands for the acoustic sound speed. In turn, this speed of sound is given by $c:=\sqrt{\dfrac{k}{\rho}}$ where $\rho$ is the mass density and $k$ is the bulk modulus. In the time-harmonic regime, the propagation of the acoustic waves is modelled by:
\begin{align}\label{EquaKg-introdution}
\begin{cases}  
\left( \Delta  + \omega^{2} \, n^2(\cdot) \right) \, p^{f} = 0 \quad \text{in} \quad  \Omega,  \\ 
\qquad \quad \qquad \, \partial_{\nu} p^{f} \; =  f \;\;\;  \text{on} \quad \partial \Omega. 
\end{cases}
\end{align}
where $p^f$ is the acoustic pressure generated by the applied source $f$. The Neumann to Dirichlet (NtD) operator $\Lambda_c$ corresponds to any $f \in H^{-\frac{1}{2}}(\partial \Omega)$, the trace on $\partial \Omega$ of the induced pressure $p^{f}$, i.e. $\Lambda_c(f) := p^{f}_{|_{\partial \Omega}}$.
The Calderón problem consists in recovering the sound speed $c$ from the knowledge of the NtD map $\Lambda_c$. According to the model $(\ref{EquaKg-introdution})$, the mass density $\rho$ is assumed to be a constant, while the bulk modulus $k$ is variable in a smooth domain $ \Omega$. We assume $k$ to be a $W^{1, \infty}(\Omega)$ \footnote{This condition can be replace by an $L^{\infty}$-regularity.} and positive function and $\Omega$ of class $C^2$. In addition, we assume that $(\ref{EquaKg-introdution})$ has a unique solution, i.e. $\omega^2$ is not an eigenvalue of $-n^{-2}\Delta$ with zero Neumann boundary condition on $\partial \Omega$. 
\bigskip

The Calderón problem was the object of an intensive study since the early 80's. The reader can see the following references for more information \cite{colton2019inverse}, \cite{Isakov-book}, \cite{Ramm-book} and \cite{Uhlmann-Review}. A model of particular interest is the EIT (Electrical Impedance Tomography) problem, also called Calderón's problem, which consists in identifying the conductivity $\gamma$ using Cauchy data $(u_{|_{\partial \Omega}}, \gamma \nabla u \cdot \nu_{|_{\partial \Omega}})$
 of the solution of equation $\nabla \cdot \gamma \nabla u=0$, in $\Omega \subset \mathbb{R}^3$, where $\nu$ is the outward unit normal vector to $\partial \Omega$. The uniqueness question of this problem is reduced to the construction of the so-called complex geometrical optics solutions (in short CGO's), see \cite{JSGU}, where $\gamma$ is a positive 
$C^2$-smooth function. The regularity of $\gamma$ is reduced to $C^{\frac{3}{2}+\epsilon}, \epsilon >0$,
 in \cite{Brown-1996}, then to $\mathbb{W}^{\frac{3}{2}, \infty}$  in \cite{paivarinta2003complex} and to $\mathbb{W}^{\frac{3}{2}, p}, p>6$ 
 in \cite{Brown-Torres-2003}. Finally, in \cite{Caro-Rogers-2016} and \cite{Haberman-Tataru} this condition is reduced to $W^{1, \infty}$
 and then to $W^{1, 3}$
 in \cite{Haberman-2015}. The corresponding Calderón problem in the 2D-setting was solved in \cite{nachman1996global}. In  \cite{bukhgeim2008recovering} the author shows, for the Schrödinger equation given by $\Delta u + \mathfrak{q} \, u =0$, in $\Omega \subset \mathbb{R}^{2}$, the uniqueness of a reconstruction of the potential $\mathfrak{q}(\cdot) \in \mathbb{L}^{p}\left( \Omega \right) ,  p > 2,$ from the Cauchy data, i.e. $\left( u_{|_{\partial \Omega}} ; \partial_{\nu} u_{|_{\partial \Omega}} \right)$, see \cite[Theorem 3.5]{bukhgeim2008recovering}. In \cite{nachman1996global}, we find a justification of the uniquely determination, from the knowledge of Dirichlet-to-Neumann map, of the coefficient $\gamma$ of the elliptic equation\footnote{The substitution $\tilde{u} = \sqrt{\gamma} \, u$ in $\nabla \cdot \left( \gamma \, \nabla u \right) = 0$ yields $\Delta \tilde{u} + \mathfrak{q} \, \tilde{u} = 0$, with $\mathfrak{q} = - \dfrac{1}{\sqrt{\gamma}}\, \Delta \left( \sqrt{\gamma} \right)$. 
 %The previous substitution removes the unknown coefficient from a higher order term to the lower order term.
 } $\nabla \cdot \left( \gamma \, \nabla u \right) = 0$ in a two dimensional domain. In \cite{Chanillo}, it is proved that if for all cubes $\bm{Q} \subset \mathbb{R}^{n}$ the condition on the smallness of $
\underset{\bm{Q}}{Sup} \; \left\vert \bm{Q} \right\vert^{\frac{2 \, p \, - \, n}{n \, p}} \; \left\Vert \mathfrak{q} \right\Vert_{\mathbb{L}^{p}(\bm{Q})}, \quad p > \frac{(n-1)}{2}, $
is satisfied, or $\mathfrak{q} \in \mathbb{L}^{p}$ with $ p > \frac{n}{2}$, then the Dirichlet-to-Neumann map determines the potential $\mathfrak{q}$. Let us also cite  \cite{MPS-2018, Caro-Garcia} regarding Dirac-type singular potentials. For more details, and without being exhaustive we  refer the readers to the following works \cite{Alessandrini-1988, Alessandrini-1990, paivarinta2003complex, brown1996global, Brown1997UniquenessIT, JSGU, KV, nachman1988, salo2008calderon} and the references therein. Let us mention, however, that apart from few works, like \cite{nachman1988}, where we find a reconstruction algorithm, most of these works are devoted to unique identifiability questions or stability estimates.
\bigskip

In this work, we propose a different approach for solving re-constructively this problem. We are inspired by our recent works on using resonant contrast agents for solving inverse problems appearing in some imaging modalities, ultrasound, optics or photo-acoustic imaging modalities, \cite{ghandriche2022mathematical, AhceneMouradMaxwell, ghandriche2022simultaneous, AlexBubbles, SW-2022, SSW-2023}. In those works, we use the measurements created after injecting single contrast agents (acoustic bubbles or nano-particles) as follows:
\bigskip

\begin{enumerate}
\item In the time-harmonic regime, we recover the induced resonances (as the Minnaert or plasmonic ones) from which, we could recover the wave speeds (or related coefficients), see \cite{ghandriche2022mathematical, AhceneMouradMaxwell, AlexBubbles}.
\bigskip

\item In the time-domain regime, we recover the internal values of the travel time function. From the Eikonal equation, we extract the values of the speed, see \cite{ghandriche2022simultaneous, SW-2022, SSW-2023}. 

\end{enumerate}

In those works, we use contrast agents injected in isolation. This means, for each single injected agent we collect the generated measurements. However, it is of importance to emphasize that we measure only on one single point. In terms of dimensionality, this is advantageous. 
\bigskip

In the current work, we inject the contrast agents all at once and then collect the measurements for multiple incident waves. In short, we collect the NtD mapping after injecting the collection of contrast agents all at once. With such measurements, we propose an approach how to do the reconstruction of the sound speed $c(\cdot)$.  This approach is divided into two steps:
\begin{enumerate}

\item In the first step, we show that the NtD map generated by the coefficient $n^{2}(\cdot)$ and the collection of contrast agents converges to the one generated by a sum of $n^{2}(\cdot)$ and an effective coefficient. This effective coefficient is negative values and one can tune its amplitude. The negativity of the effective coefficient, which is key, is due to the resonant character of the injected contrast agents. Therefore, we can tune these injected agents so that the sum of $n^{2}(\cdot)$ and the effective coefficients is negative valued with a controllable amplitude. 
\bigskip

\item From the effective NtD map, we reconstruct the coefficient $n^{2}(\cdot)$. To do so, we show that, due to the negativity of the effective coefficients, mentioned above, we can linearize the  effective NtD map. Finally, from this linearized map, we derive an explicit formula to recover $n^{2}(\cdot)$ in terms of (explicit) CGO-solutions. 
\end{enumerate}
\bigskip

To go more in details, let us take as contrast agents droplets, which are bubbles filled in with water, having the following properties. They are modelled as $D_j, j=1, ..., M,$ of the form $D_j=z_j + a\; B$ with $B$ as a smooth domain containing the origin and maximum radius as unity, such that $D = \underset{j=1}{\overset{M}{\cup}} D_{j}$. Their mass density $\rho_j$ are equal and estimated as $\rho_{j} = \rho_{0} \neq 1$, for $1 \leq j \leq M$, with $\rho_{0}$ is a constant independent on the parameter $a$, while their bulk modulus are very small and of order 
\begin{equation}\label{ScaleBulk}
k_{j} = k_{0} \; a^2,
\end{equation}
with $k_0$ being as fixed constant independent of $a$.
The maximum radius $a$ of this droplet is of  order micrometer, therefore we take $$a \ll 1. $$ 
\bigskip
We introduce the Newtonian operator $N_{D_j}^{\phi}: \mathbb{L}^{2}(D_j)\rightarrow \mathbb{L}^2(D_j)$ with the image in $\mathbb{H}^2(D_j)$ given by the expression 
\begin{equation}\label{DefNPO}
N_{D_j}^{\phi}(f)(x):=\int_{D_j} \phi(x;y) \; f(y) \; dy ,\; x \in D_{j},
\end{equation}
where $\phi(\cdot;\cdot)$ is the fundamental  solution of the interior Neumann problem for the Laplace's equation, such that 
\begin{align}\label{phi-harmonic}
\begin{cases}  
\Delta \phi = - \, \delta \quad \text{in} \quad  \Omega  \\ 
\partial_{\nu} \phi = \;\;\; 0 \;\;\;  \text{on} \;\;  \partial \Omega 
\end{cases}.
\end{align}
 This operator is self-adjoint and compact, therefore it enjoys a positive sequence of eigenvalues $\left\{ \lambda^{D_j}_{n} \right\}_{n \in \mathbb{N}}$ and they scale as $\lambda^{D_j}_{n} =a^2\; \lambda^{B}_{n}$. We fix any $n_0 \in \mathbb{N}$; $j \in \{1; \cdots ; M \}$ and we consider the eigenvalue $\lambda^{D_{j}}_{n_0}$. The incident frequency $\omega$ that we use, in this acoustic model, is taken of the form:
 \begin{equation}\label{cn0}
\frac{\omega^2}{\omega_0^2} = 1 - \frac{ c_{n_{0}} \; a^h}{k_{0}}
 \end{equation}
with $c_{n_{0}} \in \mathbb{R}$, \begin{equation} \label{c-n-0} c_{n_{0}}<0,\end{equation} is a parameter which is independent of $a$. The quantity $\omega_0$ is defined as 
\begin{equation*}
\omega_0:=\sqrt{\frac{k_j}{\lambda^{D_j}_{n_0}}}=\sqrt{\frac{k_0}{\lambda^{B}_{n_0}}},
\end{equation*}
 where the last equality is a consequence of the eigenvalues scales and $(\ref{ScaleBulk})$.
\bigskip

The droplets are distributed periodically inside $\Omega$.
We are concerned with the case where we have the number $M$ of droplets of the order 
\begin{equation}\label{M-}
M \sim a^{h-1}, \;  a \ll 1 \quad \text{with} \quad 0 \leq h<1,
\end{equation}
and, then, the minimum distance between the droplets is 
\begin{equation*}
d := \underset{i \neq j \atop 1 \leq i , j \leq M}{\min} \left\vert z_{i} - z_{j} \right\vert \sim a^{\frac{(1-h)}{3}},\;  a \ll 1 \quad \text{with} \quad 0 \leq h<1,   
\end{equation*}
as $M\sim d^{-3}$.


\bigskip



\subsection{From the original NtD map $\Lambda_D$ to the effective NtD map $\Lambda_P$}
~\\

Let $v^{g}(\cdot)$  to be solution of 
\begin{align}\label{Equavf}
\begin{cases}  
\left( \Delta  + \omega^{2} \, n^{2}(\cdot) +  \omega^{2} \,  \dfrac{\rho_{1}}{k_{1}} \, \underset{D}{\chi} \right) \, v^{g} = 0 \quad \text{in} \quad  \Omega,  \\ 
\qquad \quad \;\, \qquad \qquad \qquad \; \quad \partial_{\nu} v^{g}  =  g \quad  \text{on} \quad \partial \Omega. 
\end{cases}
\end{align}
Multiplying $(\ref{Equavf})$ by $p^{f}(\cdot)$, solution of $(\ref{EquaKg-introdution})$, and integrating over $\Omega$, we obtain 
\begin{equation*}
\langle \Lambda_0\left( f \right);  g \rangle_{\mathbb{H}^{\frac{1}{2}}(\partial \Omega) \times \mathbb{H}^{-\frac{1}{2}}(\partial \Omega)} =  \langle \nabla v^{g}; \nabla p^{f} \rangle_{\mathbb{L}^{2}(\Omega)}  - \omega^{2} \, \langle  n^{2} \, v^{g};  p^{f} \rangle_{\mathbb{L}^{2}(\Omega)} - \omega^{2} \, \dfrac{\rho_{1}}{k_{1}} \, \langle  v^{g}; p^{f} \rangle_{\mathbb{L}^{2}(D)}, 
\end{equation*}
where $\Lambda_0$ is the NtD map defined from $\mathbb{H}^{-\frac{1}{2}}\left( \partial \Omega \right)$ to $\mathbb{H}^{\frac{1}{2}}\left( \partial \Omega \right)$ by  
\begin{equation*}
\langle \Lambda_0 \left( f \right) ; g \rangle_{\mathbb{H}^{\frac{1}{2}}(\partial \Omega) \times \mathbb{H}^{-\frac{1}{2}}(\partial \Omega)}  :=  \int_{\partial \Omega} p^{f}(x) \, g(x) \, d\sigma(x).
\end{equation*}
Hence, if we set $\Lambda_{D}$ to be the NtD map of the background after injecting a cluster of droplets, we end up with the coming formula
\begin{equation}\label{Equa1Lambda}
\langle \Lambda_{D}(f); g \rangle_{\mathbb{H}^{\frac{1}{2}}(\partial \Omega) \times \mathbb{H}^{-\frac{1}{2}}(\partial \Omega)} - \langle \Lambda_0 \left( f \right) ; g \rangle_{\mathbb{H}^{\frac{1}{2}}(\partial \Omega) \times \mathbb{H}^{-\frac{1}{2}}(\partial \Omega)} = \omega^{2} \, \dfrac{\rho_{1}}{k_{1}} \; \langle v^{g}; p^{f} \rangle_{\mathbb{L}^{2}(D)},
\end{equation}
where 
\begin{equation*}
\langle \Lambda_{D}(f); g \rangle_{\mathbb{H}^{\frac{1}{2}}(\partial \Omega) \times \mathbb{H}^{-\frac{1}{2}}(\partial \Omega)} :=  \langle \nabla v^{g};  \nabla p^{f} \rangle_{\mathbb{L}^{2}(\Omega)} - \omega^{2} \, \langle n^{2} \, v^{g};  p^{f} \rangle_{\mathbb{L}^{2}(\Omega)}.
\end{equation*}
In a similar way, we define $u^{g}(\cdot)$ to be solution of 
\begin{align}\label{EquaWf}
\begin{cases}  
\left( \Delta  + \omega^{2} \, n^{2}(\cdot) - P^{2} \right) \, u^{g} \; = 0 \quad \text{in} \quad  \Omega,  \\ 
\qquad \qquad \qquad \qquad \partial_{\nu} u^{g} \; =  g \;\;\;  \text{on} \quad \partial \Omega. 
\end{cases}
\end{align}
Here
\begin{equation}\label{DefP2cn0}
P^{2} := \frac{ - \, k_{0} \; \left( \langle 1; \overline{e}_{n_{0}} \rangle_{\mathbb{L}^{2}(B)} \right)^{2}}{\lambda_{n_{0}}^{B} \; c_{n_{0}}}, 
\end{equation}
where $\overline{e}_{n_{0}}$ is the eigenfunction associated to the eigenvalue $\lambda_{n_{0}}^{B}$ related to the Newtonian operator, given by $(\ref{DefNPO})$, defined in the domain $B$. Multiplying $(\ref{EquaWf})$ by $p^{f}(\cdot)$, solution of $(\ref{EquaKg-introdution})$, and integrating over $\Omega$, we obtain 
\begin{equation*}
\langle \Lambda_0 \left( f \right) ; g \rangle_{\mathbb{H}^{\frac{1}{2}}(\partial \Omega) \times \mathbb{H}^{-\frac{1}{2}}(\partial \Omega)} = \langle \nabla u^{g}; \nabla p^{f} \rangle_{\mathbb{L}^{2}(\Omega)} - \omega^{2} \, \langle  n^{2} \, u^{g}; p^{f} \rangle_{\mathbb{L}^{2}(\Omega)} + P^{2} \, \langle u^{g}; p^{f} \rangle_{\mathbb{L}^{2}(\Omega)}.
\end{equation*}
Hence, if we set $\Lambda_{P}$ to be the NtD map of the equivalent background, we obtain  
\begin{equation}\label{Equa2Lambda}
\langle \Lambda_{P}(f); g \rangle_{\mathbb{H}^{\frac{1}{2}}(\partial \Omega) \times \mathbb{H}^{-\frac{1}{2}}(\partial \Omega)} - \langle \Lambda_0 \left( f \right) ; g \rangle_{\mathbb{H}^{\frac{1}{2}}(\partial \Omega) \times \mathbb{H}^{-\frac{1}{2}}(\partial \Omega)} =  - \, P^{2} \, \langle  u^{g}; p^{f} \rangle_{\mathbb{L}^{2}(\Omega)},
\end{equation}
where 
\begin{equation*}
\langle \Lambda_{P}(f); g \rangle_{\mathbb{H}^{\frac{1}{2}}(\partial \Omega) \times \mathbb{H}^{-\frac{1}{2}}(\partial \Omega)} := \langle \nabla u^{g}; \nabla p^{f} \rangle_{\mathbb{L}^{2}(\Omega)} - \omega^{2} \, \langle  n^{2} \, u^{g}; p^{f} \rangle_{\mathbb{L}^{2}(\Omega)}.
\end{equation*}
From $(\ref{Equa1Lambda})$ and $(\ref{Equa2Lambda})$, we see that
\begin{equation}\label{Lambda-d--Lambda-P}
\langle \Lambda_{D}(f); g \rangle_{\mathbb{H}^{\frac{1}{2}}(\partial \Omega) \times \mathbb{H}^{-\frac{1}{2}}(\partial \Omega)} - \langle \Lambda_{P}(f); g \rangle_{\mathbb{H}^{\frac{1}{2}}(\partial \Omega) \times \mathbb{H}^{-\frac{1}{2}}(\partial \Omega)} = \omega^{2} \, \dfrac{\rho_{1}}{k_{1}} \, \langle v^{g};  p^{f} \rangle_{\mathbb{L}^{2}(D)} \, + \, P^{2} \, \langle  u^{g};  p^{f} \rangle_{\mathbb{L}^{2}(\Omega)}.
\end{equation}
In the sequel, we prove that when $M$ is large, or $a$ is small, the perturbed medium, after injecting a cluster of $M$ droplets, behaves like the equivalent background. In other words, the map $\Lambda_{D}$ converges to $\Lambda_{p}$.
\begin{theorem}\label{principal-Thm}
 We have the following convergence
\begin{equation*}
\langle \Lambda_{D}(f); g \rangle_{\mathbb{H}^{\frac{1}{2}}(\partial \Omega) \times \mathbb{H}^{-\frac{1}{2}}(\partial \Omega)}  \underset{a \rightarrow 0}{\longrightarrow} \langle \Lambda_{P}(f); g \rangle_{\mathbb{H}^{\frac{1}{2}}(\partial \Omega) \times \mathbb{H}^{-\frac{1}{2}}(\partial \Omega)},
\end{equation*}
uniformly in terms of $ \left( f, g \right) \in \mathbb{H}^{-\frac{1}{2}}\left(\partial \Omega \right) \times  \mathbb{H}^{-\frac{1}{2}}\left(\partial \Omega \right)$. 
Precisely, we have the following rate \footnote{The $W^{1, \infty}$-regularity of $k$, and hence $n$, is used to derive the rate in (\ref{energy-D}). The $L^\infty(\Omega)$-regularity is enough to derive the convergence ( without rates).} 
\begin{equation}\label{energy-D}
\left\Vert \Lambda_{D} - \Lambda_{P} \right\Vert_{\mathcal{L}(\mathbb{H}^{-\frac{1}{2}}(\partial \Omega), \mathbb{H}^{\frac{1}{2}}(\partial \Omega))}  \lesssim  a^{\frac{( 1 - h ) ( 1 - \delta )}{3 (3-\delta)}} \, P^{2},\; a \ll 1,
\end{equation}
%\underset{M \rightarrow +\infty}{\longrightarrow}
where $\delta$ a sufficiently small but arbitrarily positive number. 
\end{theorem}

\begin{remark}
Two comments are in order.
     \begin{enumerate}
         \item[] 
         \item Since $M \sim a^{h-1}$ and $\delta$ is very small, we can rewrite $(\ref{energy-D})$ as: 
         \begin{equation*}
\left\Vert \Lambda_{D} - \Lambda_{P} \right\Vert_{\mathcal{L}(\mathbb{H}^{-\frac{1}{2}}(\partial \Omega), \mathbb{H}^{\frac{1}{2}}(\partial \Omega))} 
 \lesssim  M^{\frac{(\delta - 1)}{3 (3-\delta)}} \, P^{2},\; M\gg 1.
\end{equation*}
We can choose $M$, i.e. $a$, such that \begin{equation}\label{M-P}
M^{\frac{(\delta - 1)}{3 (3-\delta)}} \, P^{2} \ll 1.
\end{equation}
\item[] 
\item The parameter $\delta$ in $(\ref{energy-D})$ is linked to the $\mathbb{L}^{3-\delta}(\Omega)$-integrability of the fundamental solution $\phi(\cdot;\cdot)$, solution of $(\ref{phi-harmonic})$. Additional details can be found in \textbf{Lemma \ref{LemmaG=phi+Remainder}} and its associated proof given in \textbf{Subsection \ref{SubsectionProofLemma2.3}}.
         
\item[] 
     \end{enumerate}
\end{remark}
As we assume to know the NtD map $\Lambda_D$, for $M$ large, the previous theorem suggests the following result. 
\begin{corollary}  The NtD map $\Lambda_P$ is approximately known.
\end{corollary}

The proof of Theorem \ref{principal-Thm} is based on the point-interaction approximation, or the so-called Foldy-Lax approximation. We first approximate the left part in (\ref{energy-D}) by a linear combination of elements of a vector which is solution of an algebraic system. This algebraic system captures the multiple scattering between the injected droplets through an interaction matrix where the interaction coefficients, that are also called scattering coefficients, are all positive due to the choice made in (\ref{cn0}) of the sign of $c_{n_0}$. To prove the invertibility of this algebraic system, uniformly of the large number $M$ of droplets, we first justify the invertibility of the related continuous integral equation and then, we show, with quite tedious computations, that the algebraic equation is 'a discrete form' of this continuous integral equation.

\begin{remark} 
Two remarks are in order.
\begin{enumerate}

\item In $(\ref{DefP2cn0})$, we take the constant $c_{n_{0}} < 0$ and the parameter $P^{2}$ such that  
\begin{equation*}
P^{2} > \omega_{0}^{2} \; \left\Vert n^{2} \right\Vert_{\mathbb{L}^{\infty}(\Omega)} := P_{\min}, 
\end{equation*}
where $\omega_{0}^{2} = \dfrac{k_{0}}{\lambda^{B}_{n_{0}}}$.
This is possible if we choose the parameter $c_{n_0}$ to satisfies\footnote{We assume that we have an a priori information on $\underset{y \in \Omega}{Inf} \left\vert k(y) \right\vert$.}
\begin{equation*}
- \, \rho^{-1} \, \underset{y \in \Omega}{Inf} \left\vert k(y) \right\vert \, \left( \langle 1 ; \overline{e}_{n_{0}} \rangle_{\mathbb{L}^{2}(B)} \right)^{2} \, < \, c_{n_{0}} \, < \, 0 \quad \text{and} \quad c_{n_{0}} \rightarrow 0^{-}.  
\end{equation*}
We recall that the parameter $c_{n_0}$ appears in (\ref{cn0}) and we have (\ref{DefP2cn0}). The coefficient $c_{n_0}$ is taken small, and hence $P$ large, but satisfies (\ref{M-P}).
\bigskip

\item The parameter $h$ appearing in (\ref{cn0}) models how dilute, or dense, is the distribution of the injected droplets in $\Omega$. If $h$ is close to $1$, we have a dense distribution and when $h$ is close to $0$ we have a light distribution.
\end{enumerate}
\end{remark}

\bigskip

\subsection{The linearization of $\Lambda_P$}
~\\

\begin{theorem}\label{THMLinearization}
  We have the following linearisation of $\Lambda_P$, in the $\mathbb{H}^{\frac{1}{2}}(\partial \Omega)$ sense,
\begin{equation}\label{ASMTV}
\Lambda_P(f) - q^{f} = \omega^{2} \, \gamma \left( \bm{W}^{q^{f}} \right)  + \mathcal{O}\left( \left\Vert f \right\Vert_{\mathbb{H}^{-\frac{1}{2}}(\partial \Omega)} \, \frac{1}{P^{4}}  \right),
\end{equation}
where $f \in \mathbb{H}^{-\frac{1}{2}}(\partial \Omega)$, \, $q^{f}$ is solution of 
\begin{align}\label{Equaqf}
\begin{cases}  
\left( \Delta - P^{2}  \right) q^{f}  = 0 \quad \, \text{in} \quad  \Omega,  \\ 
\qquad \; \, \quad \partial_{\nu} q^{f}  = f \quad  \text{on} \quad \partial \Omega, 
\end{cases}
\end{align} 

and $\bm{W}^{q^{f}}$ satisfies

\begin{align*}\label{ASKD-intro}
\begin{cases}  
\left(  \Delta -  P^{2} \, I \right) \bm{W}^{q^{f}} \, = - \, n^{2} \, q^{f} \quad \text{in} \quad  \Omega,  \\ 
\qquad \qquad \, \partial_{\nu} \bm{W}^{q^{f}} = 0  \quad \quad \; \quad  \text{on} \quad \partial \Omega. 
\end{cases}
\end{align*}
\end{theorem}
Therefore knowing $\Lambda_P(f)$ allows us to construct $\bm{W}^{q^{f}}$, for $f \in \mathbb{H}^{-\frac{1}{2}}(\partial \Omega)$.
\medskip

The proof of Theorem \ref{THMLinearization} is based on the observation that the solution operator (i.e. the Lippmann-Schwinger operator) of the problem (\ref{EquaWf}) can be seen as the one of the problem (\ref{Equaqf}) plus a 'small' perturbation. The smallness of this perturbation permits us to justify the related linearization. The arguments of the analysis are based on the spectral and scaling properties of the Newtonian operator of the solution operator of (\ref{Equaqf}) via Calderon-Zygmund type estimates.  

\subsection{Construction of $c(\cdot)$ from the linearization of $\Lambda_P$}
~\\

The next theorem describes a way how we can reconstruct the sound speed from the linearized part of $\Lambda_P$.
\begin{theorem}\label{THMReconstruction}
For every  $l := \left(l_{1};l_{2};l_{3} \right) \in \mathbb{Z}^{3}$, we choose 
\begin{equation}\label{Existencexi}
\xi = \frac{P^{2+\varsigma} \, \left\vert l \right\vert^{2+\varsigma}}{\sqrt{2} \, \sqrt{l^{2}_{2} + l^{2}_{3}}} \; \begin{pmatrix}
- i \left(l_{2}^{2} + l_{3}^{2} \right) \\
- \left\vert l \right\vert \, l_{3} + i \, l_{1} \, l_{2}  \\
- \left\vert l \right\vert \, l_{2} + i \, l_{1} \, l_{3} 
\end{pmatrix},
\text{with} \;\; \varsigma \in \mathbb{R}^{+}.
\end{equation}
Hence,
\begin{equation}\label{normxi}
\left\vert \xi \right\vert = P^{2+\varsigma} \, \left\vert l \right\vert^{3+\varsigma}.
\end{equation}

We set $q^{f}(\cdot):=q^{l, \xi}$ the function defined by 
\begin{equation*}\label{Defvf}
q^{l, \xi}(x) := e^{i \, \xi \cdot x} \, \left( e^{i \,  x \cdot l} \, + r_{1}(x) \right), \quad x \in \mathbb{R}^{3},
\end{equation*}
and $r_{1}(\cdot)$ is such that
\begin{equation}\label{DiffEquar1Intro}
\left(\Delta + 2 \, i \, \xi \cdot \nabla - P^{2} \right) r_{1}(x) = \left(\left\vert l \right\vert^{2} + P^{2} \right) \; e^{i \, x \cdot l}, \quad \text{in} \quad \Omega.
\end{equation}

In the same manner we set $q^{g}(\cdot):=q^{\xi}(\cdot)$ to be the function defined by 
\begin{equation*}\label{Defvg}
q^{\xi}(x) := e^{- i \, \xi \cdot x} \left( 1 + r_{2}(x) \right), \quad x \in \mathbb{R}^{3},
\end{equation*}
where $r_{2}(\cdot)$ is such that 
\begin{equation}\label{DiffEquar2Intro}
\left(- \Delta + 2 \, i \, \xi \cdot \nabla + P^{2} \right) r_{2}(x) = - \, P^{2}  \; , \quad \text{in} \quad \Omega.
\end{equation}

Then we have the following approximate reconstruction formula:  
\begin{equation}\label{Reconstruction-formula}
n^{2}(x)   =   \left(2 \, \pi \right)^{-3} \; \sum_{\ell \in \mathbb{Z}^{3}} \langle \bm{W}^{q^{l, \xi}} ; \partial_{\nu} q^{\xi} \; \rangle \;\; e^{i \, \ell \cdot x}  + \mathcal{O}\left( P^{-\varsigma} \right),
\end{equation}
in the $L^2(\Omega)$ sense.
\end{theorem}

The justification of the existence and uniqueness of solutions corresponding to the problems $(\ref{DiffEquar1Intro})$ and $(\ref{DiffEquar2Intro})$ can be found in \cite[Section 3.2]{salo2008calderon}. More precisely, in \cite[Theorem 3.7]{salo2008calderon} the result is proved first for the free case equation, i.e. equation of the form $\left(\Delta + 2 \, i \, \xi \cdot \nabla \right) r = f$, where $r(\cdot)$ is a correction term and $f$ is a source data. Then, in \cite[Theorem 3.8]{salo2008calderon} the general case, i.e. equation of the form $\left(\Delta + 2 \, i \, \xi \cdot \nabla + \textbf{q} \right) r = f$, where $\textbf{q}$ is a potential, was proved under the conditions 
$\xi \cdot \xi = 0 \quad \text{and} \quad \left\vert \xi \right\vert \geq \max\left(C_{0} \, \left\Vert \textbf{q} \right\Vert_{\mathbb{L}^{\infty}(\Omega)}; 1 \right)$, where $C_{0}$ is a constant depending on the domain $\Omega$ and the space dimension.
These nicely rederived estimates are proved originally in the seminal work \cite[Theorem 1.1, Proposition 2.1]{JSGU}. 

The key observation here is that these CGOs are solutions of fully explicit equations, see (\ref{DiffEquar1Intro}) and (\ref{DiffEquar2Intro}),  which make the representation in (\ref{Reconstruction-formula}) re-constructive.


\bigskip

The remaining parts of the paper are organized as follows. In Section \ref{Linearization-step}, we discuss and justify the linearization step and in Section \ref{Construction-C} we deal with the reconstruction of $n^{2}(\cdot)$ from the linearized NtD map. The justification of the effective NtD is stated in Section \ref{effective-NtD}. This choice is taken as this step is, technically, the most involved part. Section \ref{SectionInjective} is devoted to the proof of the invertibility of the algebraic system related to the derivation of the effective NtD map. Finally, we postpone several technical steps to be developed and justified in Section \ref{Appendix} stated as an appendix.


\section{Linearization of $\Lambda_P$ - Proof of \textbf{Theorem \ref{THMLinearization}}}\label{Linearization-step}
~~\\
Let $u^{f}(\cdot)$ be the solution of the following Lippmann-Schwinger Equation (L.S.E in short) 
\begin{equation}\label{UfLSE}
u^{f}(x) - \omega^{2} \, N^{p}\left(n^{2} \, u^{f} \right)(x) \, = q^{f}(x), \quad x \in \Omega,  
\end{equation}
where $q^{f}(\cdot)$ is solution of $(\ref{Equaqf})$ and $G_{p}(\cdot,\cdot)$ satisfies
\begin{align*}\label{Green's-Kernel-with-P}
\begin{cases}  
\left( \Delta - P^{2}  \right) G_{p}(\cdot,\cdot) = - \, \delta \quad \text{in} \quad  \Omega,  \\ 
\qquad \; \quad \partial_{\nu} G_{p}(\cdot,\cdot) = 0 \qquad  \text{on} \quad \partial \Omega. 
\end{cases}
\end{align*}
In effortless manner we can check that $u^{f}(\cdot)$, solution of $(\ref{UfLSE})$, is also solution of $(\ref{EquaWf})$. Moreover, by an induction process on the L.S.E, given by $(\ref{UfLSE})$, we prove that 
\begin{equation}\label{AV}
u^{f}(x) - q^{f}(x) = \omega^{2} \, \gamma N^{p}\left(n^{2} \,  q^{f} \right)(x) + \sum_{j \geq 2} \left(K_{j} \overset{j}{\otimes} \left( n^{2} \right) \right)(x), \quad x \in \partial \Omega, 
\end{equation}
where 
\begin{eqnarray*}
%\left(K_{1} \overset{1}{\otimes} \left( n^{2} \right) \right)(x) &:=& \omega^{2} \, \int_{\Omega} G_{p}(x,y) \, n^{2}(y) \, q^{f}(y)  \, dy = \omega^{2} \, \gamma N^{p}\left(n^{2} \,  q^{f} \right)(x) \\
\left(K_{2} \overset{2}{\otimes} \left( n^{2} \right) \right)(x) %&:=& \left( \omega^{2} \right)^{2} \, \int_{\Omega} G_{p}(x,y) \, n^{2}(y) \,  \int_{\Omega}G_{p}(y,z_{1}) \, n^{2}(z_{1}) \, q^{f}(z_{1}) dz_{1} \, dy \\ 
&=& \left( \omega^{2} \right)^{2} \, \gamma N^{p}\left( n^{2} \,  N^{p}\left( n^{2} \, q^{f} \right)\right)(x) \\
\left(K_{3} \overset{3}{\otimes} \left( n^{2} \right) \right)(x) %&:=& (\omega^{2})^{3} \,
% \int_{\Omega} G_{p}(x,y) \, n^{2}(y) \,  N^{p}\left( n^{2} \, N^{p} \left(n^{2} \,  q^{f} \right) \right)(y) \, dy \\
& = & (\omega^{2})^{3} \, \gamma N^{p}\left(n^{2} \,  N^{p}\left( n^{2} \, N^{p}\left(n^{2} \, q^{f} \right) \right)\right)(x) \\
& \vdots & \\
\left(K_{j} \overset{j}{\otimes} \left( n^{2} \right) \right) &:=& \left( \omega^{2} \right)^{j} \,  \int_{\Omega} \cdots \int_{\Omega}  G_{p} \cdots G_{p} \, n^{2} \cdots  n^{2} \, q^{f} \, dy_{1} \cdots dy_{j},
\end{eqnarray*}
with $N^{p}(\cdot)$ is the Newtonian operator, defined from $\mathbb{L}^{2}(\Omega)$ to $\mathbb{H}^{2}(\Omega)$, by  
\begin{equation}\label{DefNp1752}
N^{p}\left( f \right)(x) := \int_{\Omega} G_{p}(x,y) \, f(y) \, dy,
\end{equation}
and $\gamma(\cdot)$ is the trace operator defined from $\mathbb{H}^{s}(\Omega)$ to $\mathbb{H}^{s-\frac{1}{2}}(\partial \Omega)$, $s\geq \frac{1}{2}$, with $\Omega$ a smooth domain. The coming lemma is useful to study the convergence of the previous series with respect to the $\mathbb{H}^{\frac{1}{2}}(\partial \Omega)$-norm. 
\begin{lemma}\label{LemmaNp}
The Newtonian operator given by $(\ref{DefNp1752})$ admits the following estimations, 
\begin{equation}\label{NormNewtonian}
\left\Vert N^{p} \right\Vert_{\mathcal{L}\left(\mathbb{L}^{2}(\Omega); \mathbb{L}^{2}(\Omega) \right)} = \mathcal{O}\left( \frac{1}{P^{2}} \right),
\end{equation}
and 
\begin{equation}\label{TraceNormNewtonian}
\left\Vert \gamma N^{p} \right\Vert_{\mathcal{L}\left(\mathbb{L}^{2}(\Omega); \mathbb{H}^{\frac{1}{2}}(\partial \Omega) \right)} = \mathcal{O}\left( \frac{1}{P} \right).
\end{equation}
\end{lemma}
\begin{proof}
See \textbf{Subsection \ref{AS2331}}. 
\end{proof}
For the convergence of the series given into $(\ref{AV})$, we have
\begin{equation}\label{DirectSeries}
\left\Vert \cdots \right\Vert_{\mathbb{H}^{\frac{1}{2}}(\partial \Omega)}  \leq \sum_{j \geq 2} \left\Vert K_{j} \overset{j}{\otimes} \left( n^{2} \right) \right\Vert_{\mathbb{H}^{\frac{1}{2}}(\partial \Omega)}.
\end{equation}
Now, we estimate the terms appearing in the previous series. 
\begin{enumerate}
\item For $j=2$, 
\begin{eqnarray*}
\nonumber
\left\Vert K_{2} \overset{2}{\otimes} \left( n^{2} \right) \right\Vert_{\mathbb{H}^{\frac{1}{2}}(\partial \Omega)} &=& \left( \omega^{2} \right)^{2} \,\left\Vert \gamma N^{p}\left(n^{2} \,  N^{p}\left( n^{2} \, q^{f} \right) \right)\right\Vert_{\mathbb{H}^{\frac{1}{2}}(\partial \Omega)} \\ %\nonumber
%& \leq & \left( \omega^{2} \right)^{2} \, \left\Vert \gamma N^{p} \right\Vert_{\mathcal{L}} \; \left\Vert n^{2} \right\Vert_{\mathbb{L}^{\infty}(\Omega)} \; \left\Vert N^{p}\left( n^{2} \, q^{f} \right) \right\Vert_{\mathbb{L}^{2}(\Omega)} \\
& \leq & \left( \omega^{2} \right)^{2} \, \left\Vert \gamma N^{p} \right\Vert_{\mathcal{L}(\mathbb{L}^{2}(\Omega);\mathbb{H}^{\frac{1}{2}}(\partial \Omega))} \; \left\Vert n^{2} \right\Vert^{2}_{\mathbb{L}^{\infty}(\Omega)} \; \left\Vert N^{p} \right\Vert_{\mathcal{L}(\mathbb{L}^{2}(\Omega);\mathbb{L}^{2}(\Omega))} \; \left\Vert q^{f}  \right\Vert_{\mathbb{L}^{2}(\Omega)}.
\end{eqnarray*}
\item For $j=3$, 
\begin{eqnarray*}
\left\Vert K_{3} \overset{3}{\otimes} \left( n^{2} \right) \right\Vert_{\mathbb{H}^{\frac{1}{2}}(\partial \Omega)} &=& (\omega^{2})^{3} \, \left\Vert \gamma N^{p}\left(n^{2} \,  N^{p}\left( n^{2} \, N^{p}\left(n^{2} \, q^{f} \right) \right)\right) \right\Vert_{\mathbb{H}^{\frac{1}{2}}(\partial \Omega)} \\ 
%& \leq & (\omega^{2})^{3} \, \left\Vert \gamma N^{p} \right\Vert_{\mathcal{L}} \; \left\Vert n^{2} \right\Vert_{\mathbb{L}^{\infty}(\Omega)} \; \left\Vert N^{p}\left( n^{2} \, N^{p}\left( n^{2} \, q^{f} \right) \right) \right\Vert_{\mathbb{L}^{2}(\Omega)}  \\
& \leq & (\omega^{2})^{3} \, \left\Vert \gamma N^{p} \right\Vert_{\mathcal{L}(\mathbb{L}^{2}(\Omega);\mathbb{H}^{\frac{1}{2}}(\partial \Omega))} \; \left\Vert n^{2} \right\Vert^{3}_{\mathbb{L}^{\infty}(\Omega)} \; \left\Vert N^{p} \right\Vert^{2}_{\mathcal{L}(\mathbb{L}^{2}(\Omega);\mathbb{L}^{2}(\Omega))} \; \left\Vert q^{f}  \right\Vert_{\mathbb{L}^{2}(\Omega)}.  %\\
%& \overset{(\ref{NormNewtonian})}{\lesssim} & \left\Vert C_{eff} \right\Vert^{6}_{\mathbb{L}^{\infty}(\Omega_{1})} \; \frac{1}{p^{8}} \, \left\Vert  v^{f}  \right\Vert^{2}_{\mathbb{L}^{2}(\Omega_{1})}  \int_{\partial \Omega_{1}} \, \left\Vert G_{p}(x,\cdot)    \right\Vert^{2}_{\mathbb{L}^{2}(\Omega_{1})}  \, d\sigma(x).
\end{eqnarray*}
\item For an arbitrary $j$, by induction, we can prove that 
\begin{equation}\label{EstimationKj}
\left\Vert K_{j} \overset{j}{\otimes} \left( n^{2} \right) \right\Vert_{\mathbb{H}^{\frac{1}{2}}(\partial \Omega)} \leq \; \bm{\Xi} \;  \, \left\Vert n^{2} \right\Vert_{\mathbb{L}^{\infty}(\Omega)} \; \left( \omega^{2} \, \left\Vert  N^{p} \right\Vert_{\mathcal{L}(\mathbb{L}^{2}(\Omega);\mathbb{L}^{2}(\Omega))} \; \left\Vert n^{2} \right\Vert_{\mathbb{L}^{\infty}(\Omega)} \right)^{j-1},
\end{equation}
where 
\begin{equation}\label{ConstantXi}
\bm{\Xi} := \omega^{2}  \, \left\Vert  q^{f}  \right\Vert_{\mathbb{L}^{2}(\Omega)} \left\Vert  \gamma N^{p} \right\Vert_{\mathcal{L}(\mathbb{L}^{2}(\Omega);\mathbb{H}^{\frac{1}{2}}(\partial \Omega))}. 
\end{equation}
\end{enumerate}
Therefore, by going back to $(\ref{DirectSeries})$ and using the estimation $(\ref{EstimationKj})$, we obtain
\begin{eqnarray}\label{||uf-vf||}
\nonumber
\left\Vert \cdots \right\Vert_{\mathbb{H}^{\frac{1}{2}}(\partial \Omega)} & \leq & \sum_{j \geq 2} \; \bm{\Xi} \; \left\Vert n^{2} \right\Vert_{\mathbb{L}^{\infty}(\Omega)} \; \left( \omega^{2} \; \left\Vert N^{p} \right\Vert_{\mathcal{L}(\mathbb{L}^{2}(\Omega);\mathbb{L}^{2}(\Omega))}  \; \left\Vert n^{2} \right\Vert_{\mathbb{L}^{\infty}(\Omega)} \right)^{j-1} \\ &=& \bm{\Xi} \; \; \left\Vert n^{2} \right\Vert^{2}_{\mathbb{L}^{\infty}(\Omega)} \;  \omega^{2} \; \left\Vert N^{p} \right\Vert_{\mathcal{L}(\mathbb{L}^{2}(\Omega);\mathbb{L}^{2}(\Omega))} \; \sum_{j \geq 0}  \left( \omega^{2} \; \left\Vert n^{2} \right\Vert_{\mathbb{L}^{\infty}(\Omega)} \left\Vert N^{p} \right\Vert_{\mathcal{L}(\mathbb{L}^{2}(\Omega);\mathbb{L}^{2}(\Omega))} \right)^{j}.
\end{eqnarray}
Under the condition 
\begin{equation}\label{CdtCvgS}
    \omega^{2} \; \left\Vert n^{2} \right\Vert_{\mathbb{L}^{\infty}(\Omega)} \left\Vert N^{p} \right\Vert_{\mathcal{L}(\mathbb{L}^{2}(\Omega);\mathbb{L}^{2}(\Omega))} < 1, 
\end{equation}
the previous series converges. Now, because $\left\Vert N^{p} \right\Vert_{\mathcal{L}(\mathbb{L}^{2}(\Omega);\mathbb{L}^{2}(\Omega))} = \mathcal{O}\left(P^{-2} \right)$, see $(\ref{NormNewtonian})$, then with $P$ large enough; knowing that $\omega^{2} \; \left\Vert n^{2} \right\Vert_{\mathbb{L}^{\infty}(\Omega)} $ is a bounded term, we deduce that the condition $(\ref{CdtCvgS})$ is satisfied. In addition, from $(\ref{||uf-vf||})$, we have
\begin{eqnarray}\label{DSuf-vf}
\nonumber
\left\Vert \cdots \right\Vert_{\mathbb{H}^{\frac{1}{2}}(\partial \Omega)} &=& \mathcal{O}\left( \bm{\Xi} \; \left\Vert N^{p} \right\Vert_{\mathcal{L}(\mathbb{L}^{2}(\Omega);\mathbb{L}^{2}(\Omega))} \right) \\ \nonumber & \overset{(\ref{ConstantXi})}{=} & \mathcal{O}\left(\left\Vert N^{p} \right\Vert_{\mathcal{L}(\mathbb{L}^{2}(\Omega);\mathbb{L}^{2}(\Omega))} \, \left\Vert  q^{f}  \right\Vert_{\mathbb{L}^{2}(\Omega)} \, \left\Vert  \gamma N^{p}\right\Vert_{\mathcal{L}(\mathbb{L}^{2}(\Omega);\mathbb{H}^{\frac{1}{2}}(\partial \Omega))} \right) \\ & \overset{Lemma \; \ref{LemmaNp}}{=} & \mathcal{O}\left(\left\Vert  q^{f}  \right\Vert_{\mathbb{L}^{2}(\Omega)} \, \frac{1}{P^{3}} \right).
\end{eqnarray}
The coming lemma is important to get an estimation of $\left\Vert \cdots \right\Vert_{\mathbb{H}^{\frac{1}{2}}(\partial \Omega)}$, with respect to the data $f$ and the parameter $P$. 
\begin{lemma}\label{Lemma22}
The function $q^{f}(\cdot)$, solution of $(\ref{Equaqf})$, satisfies: 
\begin{equation}\label{Equa214}
\left\Vert q^{f} \right\Vert_{\mathbb{L}^{2}(\Omega)} = \mathcal{O}\left( \left\Vert f \right\Vert_{\mathbb{H}^{-\frac{1}{2}}(\partial \Omega)} \, \frac{1}{P}  \right).
\end{equation}
\end{lemma}
\begin{proof}
For $x \in \Omega$, the solution $q^{f}(\cdot)$ can be represented as follows 
\begin{equation*}
q^{f}(x) = \int_{\partial \Omega} G_{p}(x,y) \, f(y) \, d\sigma(y) =: SL^{p}\left( f \right)(x).
\end{equation*}
Multiplying the previous equation by $\overline{q^{f}}(\cdot)$ and integrating over $\Omega$, gives us: 
\begin{eqnarray*}
\left\Vert q^{f} \right\Vert^{2}_{\mathbb{L}^{2}(\Omega)}  =   \langle q^{f} ; SL^{p}\left( f \right) \rangle_{\mathbb{L}^{2}(\Omega)} &=& 
\langle f ; \gamma N^{p} \left( q^{f} \right) \rangle_{\mathbb{H}^{-\frac{1}{2}}(\partial \Omega) \times \mathbb{H}^{\frac{1}{2}}(\partial \Omega)}  \\
& \leq & \left\Vert f \right\Vert_{\mathbb{H}^{-\frac{1}{2}}(\partial \Omega)} \, \left\Vert \gamma N^{p} \left( q^{f} \right) \right\Vert_{\mathbb{H}^{\frac{1}{2}}(\partial \Omega)} \\ & \leq & \left\Vert f \right\Vert_{\mathbb{H}^{-\frac{1}{2}}(\partial \Omega)} \, \left\Vert \gamma N^{p} \right\Vert_{\mathcal{L}(\mathbb{L}^{2}(\Omega);\mathbb{H}^{\frac{1}{2}}(\partial \Omega))}  \, \left\Vert q^{f} \right\Vert_{\mathbb{L}^{2}(\Omega)}.
\end{eqnarray*}
Then, 
\begin{equation*}
\left\Vert q^{f} \right\Vert_{\mathbb{L}^{2}( \Omega)} = \mathcal{O}\left( \left\Vert f \right\Vert_{\mathbb{H}^{-\frac{1}{2}}(\partial \Omega)} \, \left\Vert \gamma N^{p} \right\Vert_{\mathcal{L}(\mathbb{L}^{2}(\Omega);\mathbb{H}^{\frac{1}{2}}(\partial \Omega))}  \right) \overset{(\ref{TraceNormNewtonian})}{=} \mathcal{O}\left( \left\Vert f \right\Vert_{\mathbb{H}^{-\frac{1}{2}}(\partial \Omega)} \, \frac{1}{P}  \right).
\end{equation*}
This concludes the proof of \textbf{Lemma \ref{Lemma22}}. 
\end{proof}
Using $(\ref{Equa214})$, the estimation $(\ref{DSuf-vf})$ becomes
\begin{equation*}
\left\Vert \cdots \right\Vert_{\mathbb{H}^{\frac{1}{2}}(\partial \Omega)} =  \mathcal{O}\left(\left\Vert  f  \right\Vert_{\mathbb{H}^{-\frac{1}{2}}(\partial \Omega)} \, \frac{1}{P^{4}} \right).
\end{equation*}
Hence, from $(\ref{AV})$, we get 
\begin{equation}\label{ASMTV0521}
u^{f}(x) - q^{f}(x) = \omega^{2} \, \gamma \left( \bm{W}^{q^{f}} \right) (x) + \mathcal{O}\left(\left\Vert  f  \right\Vert_{\mathbb{H}^{-\frac{1}{2}}(\partial \Omega)} \, \frac{1}{P^{4}} \right), \quad x \in \partial \Omega,
\end{equation}
where $\bm{W}^{q^{f}} = N^{p}\left(n^{2} \,  q^{f} \right)$ is the function satisfying
\begin{align}\label{ASKD}
\begin{cases}  
\left(  \Delta -  P^{2} \, I \right) \bm{W}^{q^{f}} \, = - \, n^{2} \, q^{f} \quad \text{in} \quad  \Omega,  \\ 
\qquad \qquad \, \partial_{\nu} \bm{W}^{q^{f}} = 0  \quad \quad \; \quad  \text{on} \quad \partial \Omega. 
\end{cases}
\end{align}
Because on the boundary $\partial \Omega$, we have $u^{f} = \Lambda_{p}\left(\partial_{\nu} u^{f} \right) \overset{(\ref{EquaWf})}{=} \Lambda_{p}\left( f \right)$ and by plugging it into $(\ref{ASMTV0521})$ we derive $(\ref{ASMTV})$. This concludes the proof of \textbf{Theorem \ref{THMLinearization}}.
\section{Construction of $c(\cdot)$ from the linearized part of $\Lambda_p$ - Proof of \textbf{Theorem \ref{THMReconstruction}}}\label{Construction-C}
~\\
From the previous section, we deduce that measuring $u^{f}(\cdot) - q^{f}(\cdot)$ means measuring, approximately, $\bm{W}^{q^{f}}(\cdot)$, on the boundary $\partial \Omega$. We set $q^{g}(\cdot)$ to be the solution of 
\begin{align}\label{Equavg}
\begin{cases}  
\left( \Delta -  P^{2} \, I \right) q^{g} \, = 0 \quad \text{in} \quad  \Omega,  \\ 
\qquad \, \quad \quad \partial_{\nu} q^{g} = g  \;\;\;  \text{on} \quad \partial \Omega. 
\end{cases}
\end{align}
Multiplying the first equation of $\left( \ref{ASKD} \right)$ with $q^{g}$, solution of $(\ref{Equavg})$, and integrating over $\Omega$, we get:
\begin{equation*}
\langle \nabla \bm{W}^{q^{f}}; \nabla q^{g} \rangle_{\mathbb{L}^{2}(\Omega)} + P^{2} \, \langle  \bm{W}^{q^{f}}; q^{g} \rangle_{\mathbb{L}^{2}(\Omega)} = \langle n^{2} \, q^{f}; q^{g} \rangle_{\mathbb{L}^{2}(\Omega)}.
\end{equation*}
Moreover, by multiplying $\left( \ref{Equavg} \right)$ with $\bm{W}^{q^{f}}$, solution of $(\ref{ASKD})$, and integrating over $\Omega$, we get  
\begin{equation*}
\langle \nabla \bm{W}^{q^{f}}; \nabla q^{g} \rangle_{\mathbb{L}^{2}(\Omega)} + P^{2} \, \langle  \bm{W}^{q^{f}}; q^{g} \rangle_{\mathbb{L}^{2}(\Omega)}  = \langle \bm{W}^{q^{f}};g \rangle_{\mathbb{H}^{\frac{1}{2}}(\partial \Omega) \times \mathbb{H}^{-\frac{1}{2}}(\partial \Omega)}.
\end{equation*}
Then, by subtracting the two previous equations we end up with 
\begin{equation}\label{Equavfvg}
\langle \bm{W}^{q^{f}};g \rangle_{\mathbb{H}^{\frac{1}{2}}(\partial \Omega) \times \mathbb{H}^{-\frac{1}{2}}(\partial \Omega)} = \langle n^{2} \, q^{f}; q^{g} \rangle_{\mathbb{L}^{2}(\Omega)}, \;  \quad \forall \, \left( f, g \right) \in \mathbb{H}^{-\frac{1}{2}}(\partial \Omega) \times \mathbb{H}^{-\frac{1}{2}}(\partial \Omega).
\end{equation}
Knowing that $\bm{W}^{q^{f}}$ can be measured, on the boundary $\partial \Omega$, and $g$ is a data function, we deduce that the L.H.S is a known term. The goal is then to reconstruct $\, n^{2}(\cdot) \,$, inside $\Omega$. To achieve this, we start by fixing $\eta \in \mathbb{R}^{3}$ and choosing $\xi \in \mathbb{C}^{3}$ such that
\begin{equation}\label{cdtxixi}
\xi \cdot \xi = 0. 
\end{equation}
We set $q^{f}(\cdot)$ the function defined by 
\begin{equation}\label{Defvf}
q^{f}(x) := e^{i \, \xi \cdot x} \, \left( e^{i \,  x \cdot \eta} \, + r_{1}(x) \right), \quad x \in \mathbb{R}^{3},
\end{equation}
where $\xi$ is chosen such that\footnote{For every fixed $\eta \in \mathbb{R}^{3}$, we choose $\xi \in \mathbb{C}^{3}$ such that $(\ref{cdtxixi})$ and $(\ref{cdtetaxi})$ will be fulfilled. Such $\xi$ exists, see $(\ref{Existencexi})$.} 
\begin{equation}\label{cdtetaxi}
\eta \cdot \xi = 0,
\end{equation}
and $r_{1}(\cdot)$ is such that 
\begin{equation}\label{DiffEquar1}
\left(\Delta + 2 \, i \, \xi \cdot \nabla - P^{2} \right) r_{1}(x) = \left(\left\vert \eta \right\vert^{2} + P^{2} \right) \; e^{i \, x \cdot \eta}, \quad \text{in} \quad \Omega.
\end{equation}
Observe that the R.H.S is depending on $\eta$ and $P$,  then  $r_{1}(\cdot)$ will also depends on both $\eta$ and $P$. Later, to mark this dependence, we note $r_{1,\eta,p}(\cdot)$ instead of $r_{1}(\cdot)$. Thanks to \cite[Theorem 3.8]{salo2008calderon}, we know that under the condition 
\begin{equation}\label{Cdtxi}
\left\vert \xi \right\vert \geq \max\left( C_{0} \, P^{2} ; 1 \right) = \, C_{0} \, P^{2}, 
\end{equation}
where the last equality is a consequence of the fact that $P \gg 1$,  and $C_{0}$ is a constant depending on $\Omega$, the equation $(\ref{DiffEquar1})$
has a solution $r_{1,\eta,p}(\cdot) \in \mathbb{H}^{1}(\Omega)$ satisfying 
\begin{equation}\label{Estimationr1}
\left\Vert r_{1,\eta,p} \right\Vert_{\mathbb{L}^{2}(\Omega)} \leq  \frac{C_{0}}{\left\vert \xi \right\vert} \; \left(\left\vert \eta \right\vert^{2} + P^{2} \right) \;  \left\vert \Omega \right\vert^{\frac{1}{2}} \quad \text{and} \quad \left\Vert \nabla r_{1,\eta,p} \right\Vert_{\mathbb{L}^{2}(\Omega)} \leq  C_{0} \; \left(\left\vert \eta \right\vert^{2} + P^{2} \right) \;  \left\vert \Omega \right\vert^{\frac{1}{2}}.
\end{equation} 
In the same manner we set $q^{g}(\cdot)$ to be the function defined by 
\begin{equation}\label{Defvg}
q^{g}(x) := e^{- i \, \xi \cdot x} \left( 1 + r_{2}(x) \right), \quad x \in \mathbb{R}^{3},
\end{equation}
where $r_{2}(\cdot)$ is such that 
\begin{equation}\label{DiffEquar2}
\left(- \Delta + 2 \, i \, \xi \cdot \nabla + P^{2} \right) r_{2}(x) = - \, P^{2}  \; , \quad \text{in} \quad \Omega.
\end{equation}
Because the R.H.S is depending on $P$, the solution $r_{2}(\cdot)$ will also depends on  $P$. Later, to mark this dependence, we note $r_{2,p}(\cdot)$ instead of $r_{2}(\cdot)$.
Again, thanks to \cite[Theorem 3.8]{salo2008calderon}, we know that under the condition $(\ref{Cdtxi})$, the equation $(\ref{DiffEquar2})$
has a solution $r_{2,p}(\cdot) \in \mathbb{H}^{1}(\Omega)$, satisfying 
\begin{equation}\label{Estimationr2}
\left\Vert r_{2,p} \right\Vert_{\mathbb{L}^{2}(\Omega)} \leq \frac{C_{0}}{\left\vert \xi \right\vert} \;  P^{2}  \;  \left\vert \Omega \right\vert^{\frac{1}{2}} \quad \text{and} \quad \left\Vert \nabla r_{2,p} \right\Vert_{\mathbb{L}^{2}(\Omega)} \leq  C_{0} \;  P^{2}  \;  \left\vert \Omega \right\vert^{\frac{1}{2}}.
\end{equation} 
Now, we take unit vectors $\omega_{1}$ and $\omega_{2}$ in $\mathbb{R}^{3}$ such that $\{ \omega_{1}; \omega_{2}; \eta \}$ is an orthogonal set. In addition, we choose $\xi = s \left(\omega_{1} + i \, \omega_{2} \right)$, so that $\left\vert \xi \right\vert = s \, \sqrt{2}$ and $\xi \cdot \xi = 0$. Using the fact that $P \gg 1$ and taking the parameter $s$ sufficiently large, such that $(\ref{Cdtxi})$ will be satisfied, we reduce the estimation of the $\mathbb{L}^{2}(\Omega)-$norm of $r_{1,\eta,p}(\cdot)$ and $r_{2,p}(\cdot)$ to
\begin{equation}\label{Estimationrj}
\left\Vert r_{1,\eta,p} \right\Vert_{\mathbb{L}^{2}(\Omega)} = \mathcal{O}\left( \frac{P^{2}}{s} \right) \quad \text{and} \quad \left\Vert r_{2,p} \right\Vert_{\mathbb{L}^{2}(\Omega)} = \mathcal{O}\left( \frac{P^{2}}{s} \right).
\end{equation} 
Next, by taking the product between $q^{f}(\cdot)$, given by $(\ref{Defvf})$, and $q^{g}(\cdot)$, given by $(\ref{Defvg})$, we obtain 
\begin{equation}\label{CGOvfvg}
\left( q^{f} \cdot q^{g} \right)(x) = e^{i \,  x \cdot \eta} + r_{1,\eta ,p}(x) + e^{i \,  x \cdot \eta} \, r_{2,p}(x) + r_{1, \eta, p}(x) r_{2,p}(x),
\end{equation}
and we would like to choose the solution in such a way that $\left( q^{f} \cdot q^{g} \right)(\cdot)$ is close to $e^{i \, \cdot \cdot \eta}$, since the functions $\left\{ e^{i \, \cdot \cdot \eta} \right\}$ form a dense set, see \cite[Theorem 1.1]{VI}, in $\mathbb{L}^{1}(\Omega)$. By going back to $(\ref{Equavfvg})$, we have 
\begin{equation*}
\langle \bm{W}^{q^{f}};g \rangle_{\mathbb{H}^{\frac{1}{2}}(\partial \Omega) \times \mathbb{H}^{-\frac{1}{2}}(\partial \Omega)} = \int_{\Omega} \, n^{2}(x) \; q^{f}(x) \; q^{g}(x) \; dx 
 \overset{(\ref{CGOvfvg})}{=}  \int_{\Omega} \, n^{2}(x) \; e^{i \,  x \cdot \eta}  \; dx + Error(\eta ,p),
\end{equation*}
where
\begin{equation*}
Error(\eta ,p) := \int_{\Omega} \, n^{2}(x) \;  r_{1, \eta ,p}(x)  \; dx + \int_{\Omega} \, n^{2}(x) \;  e^{i \,  x \cdot \eta} \, r_{2, p}(x) \; dx + \int_{\Omega} \, n^{2}(x) \;  r_{1, \eta , p}(x) r_{2, p}(x) \; dx,
\end{equation*}
which can be estimated as
\begin{equation*}
\left\vert Error(\eta ,p) \right\vert %& \leq & \left\vert \int_{\Omega} \,   n^{2}(x) \; r_{1, \eta , p}(x) \, dx \right\vert + \left\vert \int_{\Omega} \, n^{2}(x) \; e^{i \,  x \cdot \eta} \, r_{2, p}(x) \, dx \right\vert + \left\vert \int_{\Omega} \, n^{2}(x) \; r_{1, \eta ,p}(x) r_{2, p}(x)  \; dx \right\vert \\ \nonumber
 \leq  \left\Vert n^{2} \right\Vert_{\mathbb{L}^{\infty}(\Omega)} \left[ \left\Vert r_{1, \eta ,p} \right\Vert_{\mathbb{L}^{2}(\Omega)} \, \left\vert \Omega \right\vert^{\frac{1}{2}} + \left\Vert r_{2, p} \right\Vert_{\mathbb{L}^{2}(\Omega)} \, \left\vert \Omega \right\vert^{\frac{1}{2}} +  \left\Vert r_{1, \eta ,p} \right\Vert_{\mathbb{L}^{2}(\Omega)} \, \left\Vert r_{2, p} \right\Vert_{\mathbb{L}^{2}(\Omega)} \right],
\end{equation*}
which, based on $(\ref{Estimationr1})$ and $(\ref{Estimationr2})$, can be reduced to
\begin{equation}\label{GzGz}
\left\vert Error(\eta ,p) \right\vert \lesssim  \frac{\left( \left\vert \eta \right\vert^{2} + P^{2} \right)}{\left\vert \xi \right\vert}  \overset{(\ref{Estimationrj})}{=}  \mathcal{O}\left( \frac{P^{2}}{s}  \right) =  \mathcal{O}\left( \frac{P^{2}}{\left\vert \xi \right\vert}  \right).
\end{equation}
Moreover, based on its construction, see $(\ref{Defvf})$, the function $q^{f}(\cdot)$ depends on $\eta$ and this implies the dependency of $\bm{W}^{q^{f}}(\cdot)$ with respect to $\eta$. We mark explicitly this dependence and we write:
\begin{equation}\label{FC}
\langle \bm{W}^{q^{f}}_{\eta} ; g \rangle_{\mathbb{H}^{\frac{1}{2}}(\partial \Omega) \times \mathbb{H}^{-\frac{1}{2}}(\partial \Omega)} - Error(\eta ,p) =  \int_{\Omega} \, n^{2}(x) \; e^{i \,  x \cdot \eta}  \; dx,  
\end{equation}
which is valid in $\Lambda_{\eta} := \left\{ \xi \in \mathbb{C}^{3},\quad \text{such that} \quad \left\vert \xi \right\vert \gg 1 \text{,} \quad \xi \cdot \xi = 0 \quad \text{and} \quad \xi \cdot \eta = 0 \right\},$
where $\eta$ is fixed in $\mathbb{R}^{3}$. The set $\Lambda_{\eta}$ is not empty, see $(\ref{Existencexi})$. By restricting $\eta$ to $\mathbb{Z}^{3}$, i.e. $\eta = - \ell$ with $\ell \in \mathbb{Z}^{3}$, we rewrite $(\ref{FC})$ as 
\begin{equation}\label{FCk}
\langle \bm{W}^{q^{f}}_{-\ell};g \rangle_{\mathbb{H}^{\frac{1}{2}}(\partial \Omega) \times \mathbb{H}^{-\frac{1}{2}}(\partial \Omega)} - Error(-\ell, p) =  \int_{\Omega} \, n^{2}(x) \; e^{- \, i \,  x \cdot \ell}  \; dx = \left( 2 \pi \right)^{3} \, \mathcal{F}\left(n^{2} \, \underset{\Omega}{\chi}\right)(\ell),  
\end{equation}
where $\mathcal{F}(\cdot)$ is the 3D-Fourier transform operator\footnote{We recall that we have 
\begin{equation*}
\mathcal{F}(f)(\ell) := \left( 2 \, \pi \right)^{-3} \, \int_{\mathbb{R}^{3}} f(x) \, e^{- i \, \ell \cdot x} \; dx, \quad \ell \in \mathbb{Z}^{3}.
\end{equation*}
}. Now, thanks to \cite[Theorem 2.3]{salo2008calderon}, we know that  
\begin{equation*}
n^{2}(x) = \sum_{\ell \in \mathbb{Z}^{3}} \mathcal{F}\left(n^{2} \, \underset{\Omega}{\chi}\right)(\ell) \; e^{i \, \ell \cdot x}, \quad \, x \in \Omega,
\end{equation*} 
with convergence in the $\mathbb{L}^{2}(\Omega)$-norm. Then, by gathering the previous expression and $(\ref{FCk})$, we end up with
\begin{equation}\label{NR0821}
n^{2}(x)  =   \left(2 \, \pi \right)^{-3} \; \sum_{\ell \in \mathbb{Z}^{3}} \int_{\partial \Omega} \bm{W}^{q^{f}}_{-\ell}(x) \; g(x) \; d\sigma(x)    \;\; e^{i \, \ell \cdot x} + \textbf{Error(x,p)},
\end{equation}
in the $L^2(\Omega)$ sense, 
where $\textbf{Error(x,p)}$ is a trigonometric series given by 
\begin{equation*}
\textbf{Error(x,p)} := - \left(2 \, \pi \right)^{-3} \; \sum_{\ell \in \mathbb{Z}^{3}}  Error(-\ell , p) \; e^{i \, \ell \cdot x}, \quad x \in \Omega.
\end{equation*}
Next, we estimate the $\mathbb{L}^{2}(\Omega)$ norm of $\textbf{Error($\cdot$ ,p)}$. We have, 
\begin{equation*}
\left\Vert \textbf{Error($\cdot$ ,p)} \right\Vert_{\mathbb{L}^{2}(\Omega)} \lesssim  \sum_{\ell \in \mathbb{Z}^{3}} \left\vert  Error(-\ell , p) \right\vert \overset{(\ref{GzGz})}{\lesssim}  \sum_{\ell \in \mathbb{Z}^{3}} \frac{\left( \left\vert \ell \right\vert^{2} + P^{2} \right)}{\left\vert \xi \right\vert}.
\end{equation*}
At this stage, we recall that for every fixed $\ell \in \mathbb{Z}^{3}$, we choose $\xi \in \mathbb{C}^{3}$ such that 
\begin{equation*}
\xi \cdot \xi = 0, \;\; \ell \cdot \xi = 0 \;\; \text{and} \;\; \left\vert \xi \right\vert \gg 1.
\end{equation*}
Such $\xi$ exists,  see $(\ref{Existencexi})$. Without loss of generality, we take $\xi$ satisfying $(\ref{normxi})$, hence $\left\vert \xi \right\vert = P^{2+\varsigma} \, \left\vert \ell \right\vert^{3+\varsigma}$, with $\varsigma \in \mathbb{R}^{+}$. Then, 
\begin{equation*}
\left\Vert \textbf{Error($\cdot$ ,p)} \right\Vert_{\mathbb{L}^{2}(\Omega)} \lesssim  \sum_{\ell \in \mathbb{Z}^{3}} \frac{\left( \left\vert \ell \right\vert^{2} + P^{2} \right)}{P^{2+\varsigma} \, \left\vert \ell \right\vert^{3+\varsigma}} = P^{-2-\varsigma} \sum_{\ell \in \mathbb{Z}^{3}} \frac{1}{\, \left\vert \ell \right\vert^{1+\varsigma}} + P^{-\varsigma} \sum_{\ell \in \mathbb{Z}^{3}} \frac{1}{ \left\vert \ell \right\vert^{3+\varsigma}}.
\end{equation*}
After that, we use the convergence of the two previous series to reduce the last estimation to
\begin{equation*}
\left\Vert \textbf{Error($\cdot$ ,p)} \right\Vert_{\mathbb{L}^{2}(\Omega)} = \mathcal{O}\left( P^{-\varsigma} \right).
\end{equation*}
Hence, $(\ref{NR0821})$ becomes, 
\begin{equation*}\label{NR0836}
n^{2}(x)  =   \left(2 \, \pi \right)^{-3} \; \sum_{\ell \in \mathbb{Z}^{3}} \langle \bm{W}^{q^{f}}_
{-\ell};g \rangle_{\mathbb{H}^{\frac{1}{2}}(\partial \Omega) \times \mathbb{H}^{-\frac{1}{2}}(\partial \Omega)}    \;\; e^{i \, \ell \cdot x} +\mathcal{O}\left( P^{-\varsigma} \right), 
\end{equation*}
in the $L^2(\Omega)$ sense. This ends the proof of \textbf{Theorem \ref{THMReconstruction}}.   





\section{Proof of Theorem \ref{principal-Thm}}\label{effective-NtD}
This section is divided into four subsections. The goal of the first subsection is to extract the dominant term of  
\begin{equation*}
\bm{I_1} := \omega^{2} \, \dfrac{\rho_{1}}{k_{1}} \, \langle v^{g}; p^{f} \rangle_{\mathbb{L}^{2}(D)}, 
\end{equation*}
where we prove that 
\begin{equation*}
\bm{I}_{1}  =  \omega^{2} \, \dfrac{\rho_{1}}{k_{1}}  \, \sum_{j=1}^{M} \, p^{f}(z_{j}) \, \int_{D_{j}}   v^{g}_{j}(x) \,  dx + Error, 
\end{equation*}
see $(\ref{I1-formula-Intro})$. In the second subsection we derive and we justify the invertibility of the discrete  algebraic system satisfied by the vector $\left( \int_{D_{j}}   v^{g}_{j}(x) \,  dx \right)_{j=1,\cdots,M}$, contained in $\bm{I_1}$, see $(\ref{ASEqua1})$ and \textbf{Lemma \ref{MTR}}. The third subsection consists in writing down the L.S.E, satisfied by $u^{g}(\cdot)$, where $u^{g}(\cdot)$ is the function appearing in 
\begin{equation*}
\bm{I_2} := - \, P^{2} \, \langle u^{g}; p^{f} \rangle_{\mathbb{L}^{2}(\Omega)}, 
\end{equation*}
see $(\ref{NewL.S.E})$. Then, we prove that the discrete algebraic system can be approximated by the continuous L.S.E, see $(\ref{maxY-Y})$. The goal of the last subsection fall in with the justification of the convergence of $\bm{I_1}$ to $\bm{I_2}$ for large number of droplets, i.e.  $M \gg 1$.
\medskip
\newline
To avoid making this section heavy and cumbersome, we have noted six lemmas without proofs. The proof of each lemma can be found in \textbf{Section \ref{SectionInjective}} and \textbf{Section \ref{Appendix}}.
\subsection{Extraction of the dominant term of \textbf{$\boldmath{I_1}$}\hfill} 
We set 
\begin{equation*}
\bm{I}_{1} := \omega^{2} \, \dfrac{\rho_{1}}{k_{1}} \, \, \langle v^{g}; p^{f} \rangle_{\mathbb{L}^{2}(D)} = \omega^{2} \, \dfrac{\rho_{1}}{k_{1}}  \, \sum_{j=1}^{M} \, \int_{D_{j}}   v^{g}_{j}(x) \, p^{f}(x) \, dx,
\end{equation*}
where $v^{g}(\cdot)$ satisfies $(\ref{Equavf})$, $p^{f}(\cdot)$ is solution of $(\ref{EquaKg-introdution})$ and we have used the notation $v^{g}_{j} := v^{g}_{|_{D_{j}}},$ for $j=1,\cdots,M$. In addition, as the coefficients $n^{2}(\cdot)$ is $W^{1, \infty}$-regular, then $p^f$, which is in $H^1(\Omega)$, enjoys a $W^{2, \infty}$-interior regularity. Based on this, we use Taylor expansion near the centres, $z_j$, to get 
\begin{equation*}
\bm{I}_{1}  =  \omega^{2} \, \dfrac{\rho_{1}}{k_{1}}  \, \sum_{j=1}^{M} \, p^{f}(z_{j}) \, \int_{D_{j}}   v^{g}_{j}(x) \,  dx + \omega^{2} \, \dfrac{\rho_{1}}{k_{1}} \, \sum_{j=1}^{M} \, \int_{D_{j}}   v^{g}_{j}(x) \, \int_{0}^{1} (x-z_{j}) \cdot \nabla p^{f}(z_{j}+t(x-z_{j})) \, dt \, dx.
\end{equation*}
We estimate the last term as 
\begin{eqnarray*}
%J_{1} & := & \omega^{2} \, \dfrac{1}{k_{1}} \, \sum_{j=1}^{M} \, \int_{D_{j}}   v^{f}_{j}(x) \, \int_{0}^{1} (x-z_{j}) \cdot \nabla p^{g}(z_{j}+t(x-z_{j})) \, dt \, dx \\
\left\vert J_{1} \right\vert %& \lesssim & %a^{-2} \, \sum_{j=1}^{M} \left\vert \int_{D_{j}}   v^{f}_{j}(x) \, \int_{0}^{1} (x-z_{j}) \cdot \nabla K^{g}(z_{j}+t(x-z_{j})) \, dt \, dx \right\vert \\
& \lesssim & a^{-2} \, \sum_{j=1}^{M} \left\Vert v^{g}_{j} \right\Vert_{\mathbb{L}^{2}(D_{j})} \; \left\Vert \int_{0}^{1} ( \cdot - z_{j}) \cdot \nabla p^{f}(z_{j}+t(\cdot - z_{j})) \, dt \,  \right\Vert_{\mathbb{L}^{2}(D_{j})} \\
& \leq & a^{-2} \, \left\Vert v^{g} \right\Vert_{\mathbb{L}^{2}(D)} \; \left( \sum_{j=1}^{M}  \left\Vert \int_{0}^{1} ( \cdot - z_{j}) \cdot \nabla p^{f}(z_{j}+t(\cdot - z_{j})) \, dt \,  \right\Vert^{2}_{\mathbb{L}^{2}(D_{j})} \right)^{\frac{1}{2}} \\
& = & \mathcal{O}\left(  a^{-1} \; \left\Vert v^{g} \right\Vert_{\mathbb{L}^{2}(D)} \;  \left\Vert \nabla p^{f} \,  \right\Vert_{\mathbb{L}^{2}(D)} \right).
\end{eqnarray*}
Moreover, based on $(\ref{EquaKg-introdution})$ we deduce that $\nabla p^{f}(\cdot)$ can be expressed as
\begin{equation}\label{ETVKT}
\nabla p^{f}(x) = \underset{x}{\nabla} \int_{\partial \Omega} G(x,y) \, f(y) \, d\sigma(y) = \langle \nabla G(x,\cdot) ; f \rangle_{\mathbb{H}^{\frac{1}{2}}(\partial \Omega) \times \mathbb{H}^{-\frac{1}{2}}(\partial \Omega)}, \quad x \in \Omega. 
\end{equation}
Then, 
\begin{equation*}
\left\Vert \nabla p^{f} \,  \right\Vert_{\mathbb{L}^{2}(D)} \leq \left\Vert  f \right\Vert_{\mathbb{H}^{-\frac{1}{2}}(\partial \Omega)} \; \left[ \int_{D}  \left\Vert \nabla G(x,\cdot) \right\Vert^{2}_{\mathbb{H}^{\frac{1}{2}}(\partial \Omega)} \,  dx \right]^{\frac{1}{2}}, 
\end{equation*}
and, using the smoothness, in $D$, of the function $x \rightarrow \left\Vert \nabla G(x,\cdot) \right\Vert_{\mathbb{H}^{\frac{1}{2}}(\partial \Omega)}$, we end up with the following estimation:
\begin{equation*}\label{Add-proof-1}
\left\Vert \nabla p^{f} \,  \right\Vert_{\mathbb{L}^{2}(D)} = \mathcal{O}\left( a^{\frac{3}{2}} \; \left\Vert f \right\Vert_{\mathbb{H}^{-\frac{1}{2}}(\partial \Omega)} \right).
\end{equation*} 
Hence, 
\begin{equation*}
J_{1} = \mathcal{O}\left( a^{\frac{1}{2}} \; \left\Vert f \right\Vert_{\mathbb{H}^{-\frac{1}{2}}(\partial \Omega)} \; \left\Vert v^{g} \right\Vert_{\mathbb{L}^{2}(D)} \right).
\end{equation*}
The following lemma gives us an a priori estimate satisfied by $v^{g}(\cdot)$. 
\begin{lemma}\label{ADZ-Lemma} 
We have the following a priori estimate 
\begin{equation}\label{AprioriEstvf}
\left\Vert v^{g} \right\Vert_{\mathbb{L}^{2}(D)}   \lesssim  a^{\frac{3}{2}-h} \; \left\Vert g \right\Vert_{\mathbb{H}^{-\frac{1}{2}}(\partial \Omega)}. 
\end{equation}
\end{lemma}
\begin{proof}
See \textbf{Subsection \ref{Proof-A-Priori-Estimate}}.
\end{proof}
Thanks to the previous lemma, we reduce the estimation of $\bm{I}_{1}$ to: 
\begin{equation}\label{I1-formula-Intro}
\bm{I}_{1}  =  \omega^{2} \, \dfrac{\rho_{1}}{k_{1}} \, \sum_{j=1}^{M} \, p^{f}(z_{j}) \, \int_{D_{j}}   v^{g}_{j}(x) \,  dx + \mathcal{O}\left(  a^{2-h} \; \left\Vert f \right\Vert_{\mathbb{H}^{-\frac{1}{2}}(\partial \Omega)} \; \left\Vert g \right\Vert_{\mathbb{H}^{-\frac{1}{2}}(\partial \Omega)} \right). 
\end{equation}
The goal of the coming subsection is to derive the algebraic system satisfied by the vector $ \left( \int_{D_{j}}   v^{g}_{j}(x) \,  dx \right)_{j=1,\cdots,M}$ and justify its invertibility.

\subsection{Algebraic system}
We start with the following L.S.E,
\begin{equation}\label{L.S.E.vf}
v^{g}(x) - \omega^{2} \, \dfrac{\rho_{1}}{k_{1}} \, \int_{D} G(x,y) \, v^{g}(y) \, dy = S(x), \quad x \in D,
\end{equation}
where $S$ is solution of 
\begin{align}\label{EquaSg}
\begin{cases}  
\Delta S + \omega^{2} \, n^{2}(\cdot) \, S = 0 \quad \text{in} \quad  \Omega,  \\ 
\qquad \quad \qquad \, \partial_{\nu} S = g \;\;\;  \text{on} \quad \partial \Omega. 
\end{cases}
\end{align}
and $G(\cdot,\cdot)$ is the Green's kernel solution of
\begin{align}\label{EquaGKernel}
\begin{cases}  
\Delta G + \omega^{2} \, n^{2}(\cdot) \, G = - \, \delta \quad \text{in} \quad  \Omega,  \\ 
\qquad \quad \qquad \, \partial_{\nu} G = \quad 0 \;\;\;  \text{on} \quad \partial \Omega. 
\end{cases}
\end{align}
The coming lemma, on the decomposition of the Green's kernel $G(\cdot;\cdot)$, is useful for the next step.
\begin{lemma}\label{LemmaG=phi+Remainder}
The Green's kernel $G(\cdot,\cdot)$, solution of $(\ref{EquaGKernel})$, admits the following decomposition: 
\begin{equation}\label{G=phi+Remainder}
G(x,y) = \phi(x,y) + \mathcal{R}(x,y), \quad x \neq y,
\end{equation} 
where $\phi(\cdot,\cdot)$ is the fundamental solution of $(\ref{phi-harmonic})$ and the remainder term  $\mathcal{R}$ satisfies 
\begin{align}\label{RemainderPDE}
\begin{cases}  
\Delta \mathcal{R} + \omega^{2} \, n^{2}(\cdot) \, \mathcal{R} = - \, \omega^{2} \, n^{2}(\cdot) \, \phi \quad \text{in} \quad  \Omega,  \\ 
\qquad \qquad   \quad \partial_{\nu} \mathcal{R} = 0 \;\;\;  \text{on} \quad \partial \Omega.
\end{cases}
\end{align}  
In addition, for an arbitrarily and sufficiently small positive $\delta$, the fundamental solution $\phi(\cdot,y)$ is in $\mathbb{L}^{3-\delta}(\Omega)$ and $\mathcal{R}(\cdot,y)$ is an element in $\mathbb{L}^{\infty}(\Omega)$. 
\end{lemma}
\begin{proof}
See \textbf{Subsection \ref{SubsectionProofLemma2.3}}.
\end{proof}
For $x \in D_{m}$, we rewrite $(\ref{L.S.E.vf})$ as 
\begin{equation}\label{Equa1432}
\left( I - \omega^{2} \, \dfrac{\rho_{1}}{k_{1}} \, N_{D_{m}}^{\phi} \right)(v^{g}_{m})(x) - \omega^{2} \, \dfrac{\rho_{1}}{k_{1}} \, \sum_{j \neq m}^{M} \int_{D_{j}} G(x,y) \, v^{g}_{j}(y) \, dy = S_{m}(x) + \omega^{2} \, \dfrac{\rho_{1}}{k_{1}} \, \int_{D_{m}} \mathcal{R}(x,y) \, v^{g}_{m}(y) \, dy, 
\end{equation}
where $S_{m} = S_{|_{D_{m}}}$, $\mathcal{R}(\cdot,\cdot)$ is solution of $(\ref{RemainderPDE})$ and $N_{D_{m}}^{\phi}$ is the Newtonian operator defined, from $\mathbb{L}^{2}(D_{m})$ to $\mathbb{L}^{2}(D_{m})$, by $(\ref{DefNPO})$. In both sides of $(\ref{Equa1432})$, successively, we multiply by $\dfrac{k_{1}}{\omega^{2} \, \rho_{1}}$, we take the inverse operator of $\left( \dfrac{k_{1}}{\omega^{2} \, \rho_{1}} \, I -  N_{D_{m}}^{\phi} \right)$ and integrate over $D_{m}$, the obtained equation, to get:
\begin{eqnarray}\label{Equa1451}
\nonumber
\int_{D_{m}} v_{m}^{g}(x) \; dx \; &-& \;  \sum_{j = 1 \atop j \neq m}^{M} \int_{D_{m}} W_{m}(x) \, \int_{D_{j}} G(x;y) \;  v_{j}^{g}(y) \; dy \; dx = \dfrac{k_{1}}{\omega^{2} \, \rho_{1}} \; \int_{D_{m}} W_{m}(x) \, S_{m}(x) \, dx \\ &+& \int_{D_{m}} W_{m}(x) \, \int_{D_{m}} \mathcal{R}(x,y) \, v^{g}_{m}(y) \, dy \, dx , 
\end{eqnarray}
where $W_{m}(\cdot)$ is solution of
\begin{equation*}\label{EquaWm3D}
\dfrac{k_{1}}{\omega^{2} \, \rho_{1}} \; W_{m}(x) -  N_{D_{m}}^{\phi}\left( W_{m} \right)(x) = 1, \quad x \in D_{m}.
\end{equation*} 
Next, to derive the desired algebraic system, we expand in the equation $(\ref{Equa1451})$ the Green's kernel $G(\cdot,\cdot)$ and the source term $S(\cdot)$, near the centres, to obtain:
\begin{equation}\label{ASEqua1510}
\int_{D_{m}} v_{m}^{g}(x) \; dx \; -  \; \alpha_{m} \; \sum_{j = 1 \atop j \neq m}^{M} G(z_{m};z_{j}) \; \int_{D_{j}} v_{j}^{g}(x) \; dx = \dfrac{k_{1}}{\omega^{2} \, \rho_{1}} \; \alpha_{m} \; S(z_{m}) + Rest_{m},
\end{equation}
where $\alpha_{m}$ is the scattering coefficient given by 
\begin{equation}\label{Defalpha}
\alpha_{m} := \int_{D_{m}} W_{m}(x) \, dx,
\end{equation}
and 
\begin{eqnarray}\label{DefRestm}
\nonumber
Rest_{m} &:=&  \sum_{j = 1 \atop j \neq m}^{M} \int_{D_{m}} W_{m}(x)  \int_{0}^{1} (x-z_{m}) \cdot \nabla G(z_{m}+t(x-z_{m});z_{j}) \; dt \; dx \, \int_{D_{j}} \, v_{j}^{g}(y) \, dy  \\ \nonumber 
&+&  \sum_{j = 1 \atop j \neq m}^{M} \int_{D_{m}} W_{m}(x) \int_{D_{j}} \int_{0}^{1} (y-z_{j}) \cdot \nabla G(x;z_{j}+t(y-z_{j})) \; dt \, v_{j}^{g}(y) \, dy \; dx \\ \nonumber 
&+& \dfrac{k_{1}}{\omega^{2} \, \rho_{1}} \; \int_{D_{m}} W_{m}(x) \, \int_{0}^{1} (x-z_{m}) \cdot \nabla S_{m}(z_{m}+t(x-z_{m})) dt \, dx  \\
&+& \int_{D_{m}} W_{m}(x) \, \int_{D_{m}} \mathcal{R}(x,y) \, v^{g}_{m}(y) \, dy \, dx. 
\end{eqnarray}
Without loss of generalities, we can assume, in $(\ref{ASEqua1510})$, that $\alpha_{m} = \alpha,$ for $m=1,\cdots,M$. Then, we obtain: 
\begin{equation}\label{ASEqua1}
\frac{1}{\alpha} \; \int_{D_{m}} v_{m}^{g}(x) \; dx \; - \;   \sum_{j = 1 \atop j \neq m}^{M} G(z_{m};z_{j}) \; \int_{D_{j}} v_{j}^{g}(x) \; dx = \dfrac{k_{1}}{\omega^{2} \, \rho_{1}} \; S(z_{m}) + \frac{Rest_{m}}{\alpha}.
\end{equation}
The next lemma ensures the invertibility of the previous algebraic system. 
\begin{lemma}\label{MTR}
The algebraic system given by $(\ref{ASEqua1})$ is invertible.
\end{lemma}
\begin{proof}
See \textbf{Section \ref{SectionInjective}}.    
\end{proof}





\subsection{The L.S.E satisfied by \textbf{$u^{g}(\cdot)$}}\label{SubSection43} 
We start by multiplying both sides of $(\ref{ASEqua1})$ with $\omega^{2} \,  \dfrac{\rho_{1}}{k_{1}}$ and, then, setting \footnote{In the equation $\alpha  = \overline{\alpha} \, a^{1-h}$, the term $a^{1-h}$ comes from the estimation of $ \alpha $, see \textbf{Lemma \ref{EstimationRealphaImalpha}}.} 
\begin{equation}\label{DefYj}
Y_{m} := \dfrac{\omega^{2} \, \rho_{1}}{\alpha \; k_{1}}  \; \int_{D_{m}} v_{m}^{g}(x) \; dx
\end{equation}
with $\alpha = \overline{\alpha} \, a^{1-h}$ where $\overline{\alpha} \sim 1$, with respect to the parameter $a$, to get 
\begin{equation}\label{0820}
Y_{m} -  \sum_{j \neq m} G(z_{m};z_{j}) \; \overline{\alpha} \, a^{1-h} Y_{j} = S(z_{m})   + \dfrac{\omega^{2} \, \rho_{1}}{k_{1}}  \;  \frac{Rest_{m}}{\alpha}.
\end{equation}
Without keeping the errors terms, we rewrite the previous algebraic system as
\begin{equation}\label{1924}
Y_{m} -  \sum_{j = 1 \atop j \neq m}^{M} G(z_{m};z_{j}) \; \overline{\alpha} \, a^{1-h} \, Y_{j} =   S(z_{m}).
\end{equation}
We set the following L.S.E, 
\begin{equation}\label{NewL.S.E}
Y(z) -  \overline{ \alpha } \, \int_{\Omega} G(z;y) \, Y(y) \, dy =  S(z), \quad z \in \Omega, 
\end{equation}
where $G(\cdot,\cdot)$ is solution of $(\ref{EquaGKernel})$ and $S(\cdot)$ is solution of $(\ref{EquaSg})$. We need the following lemma. 
\begin{lemma}\label{ExistenceYEstimation}
There exists one and only one solution $Y(\cdot)$ of the L.S.E $(\ref{NewL.S.E})$, and it satisfies the estimates
\begin{equation}\label{Lemma6.4YS}
\left\Vert Y \right\Vert_{\mathbb{H}^{1}(\Omega)} \lesssim \left\Vert S \right\Vert_{\mathbb{H}^{1}(\Omega)}. %\quad \text{and} \quad \left\Vert \nabla Y \right\Vert_{\mathbb{L}^{2}(\Omega)} \lesssim \left\Vert S \right\Vert_{\mathbb{H}^{1}(\Omega)}. 
\end{equation}
\end{lemma}
\begin{proof}
The equation $(\ref{NewL.S.E})$ is invertible from $\mathbb{L}^{2}(\Omega)$ to $\mathbb{L}^{2}(\Omega)$\ and this gives us the estimation 
\begin{equation}\label{KS1}
 \left\Vert Y \right\Vert_{\mathbb{L}^{2}(\Omega)} \lesssim \left\Vert S \right\Vert_{\mathbb{L}^{2}(\Omega)} \le \left\Vert S \right\Vert_{\mathbb{H}^{1}(\Omega)}.   
\end{equation}
Now, by taking the $\mathbb{H}^{1}(\Omega)$-norm in both sides of $(\ref{NewL.S.E})$, we get: 
\begin{equation*}
    \left\Vert Y \right\Vert_{\mathbb{H}^{1}(\Omega)}  \leq \left\Vert S \right\Vert_{\mathbb{H}^{1}(\Omega)} + \left\vert \overline{\alpha} \right\vert \; \left\Vert N \left( Y \right) \right\Vert_{\mathbb{H}^{1}(\Omega)}, 
\end{equation*}
and using the continuity of the Newtonian operator, from $\mathbb{L}^{2}(\Omega)$ to $\mathbb{H}^{1}(\Omega)$ , we obtain
\begin{equation*}
    \left\Vert Y \right\Vert_{\mathbb{H}^{1}(\Omega)}  \lesssim \left\Vert S \right\Vert_{\mathbb{H}^{1}(\Omega)} +  \left\Vert Y \right\Vert_{\mathbb{L}^{2}(\Omega)} \overset{(\ref{KS1})}{\lesssim} \left\Vert S \right\Vert_{\mathbb{H}^{1}(\Omega)}. 
\end{equation*}
This ends the proof of \textbf{Lemma \ref{ExistenceYEstimation}}. 
\end{proof}
\begin{remark}
    The function $S(\cdot)$, solution of $(\ref{EquaSg})$, can be represented as a single layer potential with density function given by $g$, i.e. $S = SL(g)$. Then, from $(\ref{Lemma6.4YS})$, we obtain 
\begin{equation*}
    \left\Vert Y \right\Vert_{\mathbb{H}^{1}(\Omega)} \lesssim \left\Vert SL(g) \right\Vert_{\mathbb{H}^{1}(\Omega)}, 
\end{equation*}
and using the continuity of the single layer operator, from $\mathbb{H}^{-\frac{1}{2}}(\partial \Omega)$ to $\mathbb{H}^{1}(\Omega)$, we end up with the following estimation: 
\begin{equation}\label{Yfctg}
    \left\Vert Y \right\Vert_{\mathbb{H}^{1}(\Omega)} \lesssim \left\Vert g \right\Vert_{\mathbb{H}^{-\frac{1}{2}}(\partial \Omega)}. 
\end{equation}
\end{remark}
\bigskip
Remember that the set of droplets $D_j$'s is distributed periodically inside $\Omega$. We split $\Omega$ as $\Omega := \underset{j=1}{\overset{M}{\cup}} \Omega_{j}$ where each $\Omega_j:=\Omega\cap K_j$ with $K_j$ is cube of center $z_j$ and volume $a^{1-h}$.
% Figure environment removed
We rearrange the splitting of the domain $\Omega$ as $\Omega = \Omega_{cube} \cup \Omega_{r}$, with 
\begin{equation*}
\Omega_{cube} = \overset{M}{\underset{j=1}{\cup}}
\Omega_{j} \;\; \text{and} \;\; \Omega_{r} = \underset{j-1}{\overset{\aleph}{\cup}} \Omega_{j}^{\star}, \;\;\;\; M = M(a), \; \aleph = \aleph(a) \in \mathbb{N}, 
\end{equation*}
where $\Omega_{j}$'s are cubes located strictly within the interior of the domain $\Omega$. Each $\Omega_j$ contains one $D_j$ while the $\Omega_j^{\star}$'s do not contain any, see Figure \ref{Fig1}.  We observe that $\Omega_{cube} \subsetneq \Omega$.
\bigskip


Based on the introduced notations, in particular as $\vert \Omega_j\vert=a^{1-h}$, \footnote{We have $\vert \Omega_j\vert=a^{1-h}$ and $\vert \Omega^{\star}_j\vert\sim a^{1-h}$ but, as these $\Omega^{\star}_j$'s intersect $\partial \Omega$, we cannot necessarily replace $\sim$ with $=$.} we can rewrite $(\ref{NewL.S.E})$ as 
\begin{equation}\label{LSE=RHS+AI+AII}
Y(z_{m}) -  \sum_{j = 1 \atop j \neq m}^{M} G(z_{m};z_{j}) \, \overline{ \alpha } \, a^{1-h}  \, Y(z_{j}) =  S(z_{m}) + \overline{ \alpha } \,  \bm{\tilde{I}} ,
\end{equation}
where 
\begin{equation*}
\bm{\tilde{I}} :=  \int_{\Omega} G(z_{m};y) \, Y(y) \, dy - \, \sum_{j = 1 \atop j \neq m}^{M} G(z_{m};z_{j})    \, Y(z_{j}) \, \left\vert \Omega_{j} \right\vert. 
\end{equation*}
To estimate $\bm{\tilde{I}}$, we first consider the term $$\int_{\Omega} G(z_{m};y) \, Y(y) \, dy:=\int_{\Omega_{cube}} G(z_{m};y) \, Y(y) \, dy+\int_{\Omega_r} G(z_{m};y) \, Y(y) \, dy$$ and show that the second term is negligible.
Indeed, regarding the domains $\Omega_{j}^{\star}$, which are not necessarily  cubes, they have the no empty intersection property  with $\partial \Omega$, i.e. $\partial \Omega_{n}^{\star} \cap \partial \Omega \neq \{ \emptyset \}$, for $1 \leq n \leq \aleph$. Since each $\Omega_{j}$ has volume equals to $a^{1-h}$, and then its maximum radius is of order $a^{\frac{1}{3}(1-h)}$ then intersecting surfaces with $\partial \Omega$ has volume of order $a^{\frac{2}{3}(1-h)}$. As the volume of $\partial \Omega$ is of order one, we conclude that the number of such cubes will not exceed the order $a^{-\frac{2}{3}(1-h)}$. Hence, the volume of $\Omega_{r}$ will not exceed the order $a^{\frac{1}{3}(1-h)} \underset{a \rightarrow 0}{\rightarrow} 0$, i.e. 
\begin{equation}\label{VolOmegar}
    \left\vert \Omega_{r} \right\vert = \mathcal{O}\left( a^{\frac{1}{3}(1-h)} \right).
\end{equation}
Observe, due to \textbf{Lemma \ref{ExistenceYEstimation}}, that we have\footnote{We recall from \textbf{Lemma \ref{LemmaG=phi+Remainder}} that $G(z_{m};\cdot) \in \mathbb{L}^{3-\delta}(\Omega)$, with $z_{m}$ fixed, where $\delta$ is an arbitrarily and sufficiently small positive number.}
\begin{equation*}
\left\vert \int_{\Omega_r} G(z_{m};y) \, Y(y) \, dy \right\vert  \leq  \Vert Y \Vert_{\mathbb{L}^{2}(\Omega_{r})} \left\Vert G(z_{m};\cdot) \right\Vert_{\mathbb{L}^{2}(\Omega_{r})} \leq  \Vert Y \Vert_{\mathbb{L}^{2}(\Omega)} \; \left\vert \Omega_{r} \right\vert^{\frac{1-\delta}{3-\delta}} \; \left\Vert G(z_{m};\cdot) \right\Vert^{2}_{\mathbb{L}^{3-\delta}(\Omega_{r})},
\end{equation*}
and using the fact that $G(z_{m};\cdot) \in \mathbb{L}^{3-\delta}(\Omega_{r})$, hence $\left\Vert G(z_{m};\cdot) \right\Vert^{2}_{\mathbb{L}^{3-\delta}(\Omega_{r})} \sim 1$, we reduce the previous estimation to
\begin{equation}
\left\vert \int_{\Omega_r} G(z_{m};y) \, Y(y) \, dy \right\vert  \lesssim  \Vert Y \Vert_{\mathbb{L}^{2}(\Omega)} \; \left\vert \Omega_{r} \right\vert^{\frac{1-\delta}{3-\delta}} 
 \overset{(\ref{VolOmegar})}{\underset{(\ref{Yfctg})}{=}}  \mathcal{O}\left( \left\Vert g \right\Vert_{\mathbb{H}^{-\frac{1}{2}}(\partial \Omega)} \; a^{\frac{(1-h) \, (1 - \delta)}{3(3-\delta)}} \right),
\end{equation}
which tends to zero with $a$. Therefore,
\begin{equation}\label{tildeI=I+err}
\bm{\tilde{I}}=\bm{I} + \mathcal{O}\left( \left\Vert g \right\Vert_{\mathbb{H}^{-\frac{1}{2}}(\partial \Omega)}  \; a^{\frac{(1-h) \, (1 - \delta)}{3(3-\delta)}}  \right), \quad \text{as} \quad a \ll 1,    
\end{equation}
where 
\begin{equation}
\bm{I} := \sum_{\ell = 1}^{M} \int_{\Omega_{\ell}} G(z_{m};y) \, Y(y) \, dy - \sum_{j = 1 \atop j \neq m}^{M} G(z_{m};z_{j})  \, Y(z_{j}) \, \left\vert \Omega_{j} \right\vert.
\end{equation}

Let us now estimate $\bm{I}$. We write
\begin{eqnarray*}
\bm{I}
&=& \sum_{\ell = 1 \atop \ell \neq m}^{M} \int_{\Omega_{\ell}} \left[ G(z_{m};y)  Y(y) - G(z_{m};z_{\ell})  Y(z_{\ell})\right] \, dy +  \int_{\Omega_{m}} G(z_{m};y) \,  Y(y) \, dy,
\end{eqnarray*}
and, by using Taylor expansion for the function $G(z_{m},\cdot) \, Y(\cdot)$ near the point $z_{\ell}$, we get: 
\begin{eqnarray*}
\bm{I} &=&  \sum_{\ell = 1 \atop \ell \neq m}^{M}   \int_{\Omega_{\ell}} \int_{0}^{1} G(z_{m};z_{\ell} + t (y-z_{\ell})) \nabla Y\left(z_{\ell} + t (y-z_{\ell}\right) \cdot (y-z_{\ell}) \, dt  \, dy \\ 
&+&\sum_{\ell = 1 \atop \ell \neq m}^{M}  \int_{\Omega_{\ell}} \int_{0}^{1} Y\left(z_{\ell} + t (y-z_{\ell}) \right) \nabla G(z_{m};z_{\ell} + t (y-z_{\ell})) \cdot (y-z_{\ell})  dt   dy + \int_{\Omega_{m}} G(z_{m};y)  Y(y) dy. 
\end{eqnarray*}
From \textbf{Lemma \ref{LemmaG=phi+Remainder}}, we know that $G(x,y) = \phi(x,y) + \mathcal{R}(x,y)$, for $x \neq y$, where the dominant part $\phi(\cdot,\cdot)$ has a singularity at most of order 1, i.e. $\phi(x,y) \sim \frac{1}{\left\vert x - y \right\vert}$, for $x \neq y$, see \textbf{ Remark \ref{RemarkSAFS}}. In the sequel, we keep only the dominant part of $G(\cdot,\cdot)$. More precisely, we have
\begin{equation}\label{SG}
\left\vert G(x,y) \right\vert \lesssim \frac{1}{\left\vert x - y \right\vert} \quad \text{and} \quad \left\vert \nabla G(x,y) \right\vert \lesssim \frac{1}{\left\vert x - y \right\vert^{2}}, \quad \text{for} \quad x \neq y.
\end{equation}

By taking the modulus in both sides and 
using $(\ref{SG})$, we deduce: 
\begin{eqnarray*}
\left\vert \bm{I} \right\vert & \lesssim & a^{\frac{(1-h)}{3}}
\sum_{\ell = 1 \atop \ell \neq m}^{M}   \int_{\Omega_{\ell}} \int_{0}^{1} \frac{1}{ \left\vert z_{m} - z_{\ell} - t (y-z_{\ell}) \right\vert} \left\vert \nabla Y\left(z_{\ell} + t (y-z_{\ell}\right) \right\vert  dt  \, dy  \\ \nonumber
&+& a^{\frac{(1-h)}{3}} \sum_{\ell = 1 \atop \ell \neq m}^{M}  \int_{\Omega_{\ell}} \int_{0}^{1}  \frac{\left\vert Y\left(z_{\ell} + t (y-z_{\ell}) \right) \right\vert}{ \left\vert z_{m} - z_{\ell} - t (y-z_{\ell})\right\vert^{2}} \, dt  \, dy + \int_{\Omega_{m}} \left\vert G(z_{m};y) \, \right\vert \left\vert Y(y) \, \right\vert dy.
\end{eqnarray*}
Using the fact that $\left\vert \Omega_{\ell} \right\vert \sim a^{1-h}$, with $1 \leq \ell \leq M$, we obtain:
\begin{equation*}
\left\vert \bm{I} \right\vert  \lesssim  a^{\frac{5(1-h)}{6}}
\sum_{\ell = 1 \atop \ell \neq m}^{M}    \frac{\left\Vert \nabla Y  \right\Vert_{\mathbb{L}^{2}(\Omega_{\ell})}}{ \left\vert z_{m} - z_{\ell}  \right\vert} + a^{\frac{5(1-h)}{6}} \sum_{\ell = 1 \atop \ell \neq m}^{M} \frac{\left\Vert  Y  \right\Vert_{\mathbb{L}^{2}(\Omega_{\ell})}}{ \left\vert z_{m} - z_{\ell}  \right\vert^{2}}  +  \left\Vert Y \, \right\Vert_{\mathbb{L}^{2}(\Omega_{m})} \;  \left\Vert G(z_{m};\cdot) \right\Vert_{\mathbb{L}^{2}(\Omega_{m})}.
\end{equation*}
By making the use of Cauchy-Schwartz inequality we get  
\begin{eqnarray*}
\left\vert \bm{I} \right\vert  & \lesssim & a^{\frac{5(1-h)}{6}} \; \left\Vert \nabla Y  \right\Vert_{\mathbb{L}^{2}(\Omega)} \; \left(
\sum_{\ell = 1 \atop \ell \neq m}^{M}    \frac{1}{ \left\vert z_{m} - z_{\ell}  \right\vert^{2}} \right)^{\frac{1}{2}}    + a^{\frac{5(1-h)}{6}} \; \left\Vert Y  \right\Vert_{\mathbb{L}^{2}(\Omega)} \; \left(
\sum_{\ell = 1 \atop \ell \neq m}^{M}    \frac{1}{ \left\vert z_{m} - z_{\ell}  \right\vert^{4}} \right)^{\frac{1}{2}} \\ &+&  \left\Vert  Y  \right\Vert_{\mathbb{L}^{2}(\Omega)} \; \left( \int_{\Omega_{m}} \frac{1}{\left\vert z_{m} - y  \, \right\vert^{2}} \, dy \right)^{\frac{1}{2}}. 
\end{eqnarray*}
To comupte the sums appearing on the R.H.S, we use the fact that 
\begin{equation}\label{SID}
    \sum_{j = 1 \atop j \neq m}^{M} \left\vert z_{i} - z_{j} \right\vert^{-k} = \begin{cases}
			\mathcal{O}\left(d^{-3}\right), & \text{for $k < 3$}\\
            \mathcal{O}\left(d^{-k}\right), & \text{for $k > 3$}
		 \end{cases},
\end{equation}
see \cite[Section 3.3]{Ammari_2019} for more details. Hence, 
\begin{equation}\label{AI3RHS}
\left\vert \bm{I} \right\vert  \lesssim  a^{\frac{5(1-h)}{6}}  \; \left\Vert \nabla Y  \right\Vert_{\mathbb{L}^{2}(\Omega)} \; d^{-\frac{3}{2}} + a^{\frac{5(1-h)}{6}} \; \left\Vert  Y  \right\Vert_{\mathbb{L}^{2}(\Omega)} \; d^{-2} + \left\Vert  Y  \right\Vert_{\mathbb{L}^{2}(\Omega)} \; \left( \int_{\Omega_{m}} \frac{1}{\left\vert z_{m} - y \right\vert^{2}}  dy \right)^{\frac{1}{2}}.
\end{equation}
To achieve the estimation of $\bm{I}$, we set and we estimate the third term appearing on the R.H.S
\begin{equation*}
\bm{I_3} = \int_{B(z_{m};r)}  \frac{1}{\left\vert z_{m} - y \right\vert^{2}} \, dy + \int_{\Omega_{m} \setminus B(z_{m};r)}  \frac{1}{\left\vert z_{m} - y \right\vert^{2}} \,  dy,
\end{equation*}
where $B(z_{m};r)$ is the ball of center $z_{m}$ and radius $r$, where $r$ is such that $r \in \bm{I_4} := \left[ 0 ; \frac{\sqrt{3}}{2} \, a^{\frac{(1-h)}{3}} \right]$.
Then, 
\begin{eqnarray*}
\bm{I_3} & \lesssim &  \int_{0}^{r} \int_{\partial \, B(z_{m};s)}  \frac{1}{\left\vert z_{m} - y \right\vert^{2}} \, d\sigma(y) \, ds +  \frac{1}{r^{2}} \; \left\vert \Omega_{m} \setminus B(z_{m};r) \right\vert  \\
& = &  \int_{0}^{r} \, \frac{1}{s^{2}} \, \left\vert \partial \, B(z_{m};s) \right\vert  \,  ds +  \frac{1}{r^{2}} \; \left\vert \Omega_{m} \setminus B(z_{m};r) \right\vert  =  \frac{8 \, \pi \, r}{3} +  \frac{1}{r^{2}} \; a^{1-h}  \leq   \; \underset{r \in \bm{I_4}}{\max} \, \rho(r,a),
\end{eqnarray*}
where 
\begin{equation*}
\rho(r,a) :=    \frac{8 \, \pi}{3} \, r +  \frac{1}{r^{2}} \; a^{1-h}.  
\end{equation*}
We have $\underset{r \in \bm{I_4}}{\max} \, \rho(r,a) = \rho(r_{sol},a)$, where $r_{sol}$ is such that $\partial_{r} \rho(r_{sol},a) = 0$. Straightforward computations, gives us $r_{sol} =  \left( \frac{3}{4 \, \pi} \, a^{1-h} \right)^{\frac{1}{3}}$. Consequently, 
\begin{equation*}
\underset{r \in \bm{I_4}}{\max} \, \rho(r,a) = \left( 48 \, \pi^{2} \right)^{\frac{1}{3}} \, a^{\frac{(1-h)}{3}}.  
\end{equation*}
Hence, 
\begin{equation}\label{EstimationAI3}
\bm{I_3}  =  \mathcal{O}\left( a^{\frac{(1-h)}{3}} \right). 
\end{equation}
%\end{enumerate}
Finally, by gathering $(\ref{EstimationAI3})$ and  $(\ref{AI3RHS})$ and using the fact that $d \sim a^{\frac{1-h}{3}}$, we end up with 
\begin{equation*}
\bm{I} \lesssim a^{\frac{(1-h)}{3}} \; \left\Vert \nabla Y \right\Vert_{\mathbb{L}^{2}(\Omega)} + a^{\frac{(1-h)}{6}} \; \left\Vert  Y \right\Vert_{\mathbb{L}^{2}(\Omega)} \overset{(\ref{Yfctg})}{\lesssim} a^{\frac{(1-h)}{6}} \; \left\Vert g \right\Vert_{\mathbb{H}^{-\frac{1}{2}}(\partial \Omega)}.
\end{equation*}
Hence, using $(\ref{tildeI=I+err})$, we get
\begin{equation}\label{EstimationAI}
\bm{\tilde{I}} = \mathcal{O}\left( a^{\frac{(1 - \delta) \, (1-h)}{3(3-\delta)}} \; \left\Vert g \right\Vert_{\mathbb{H}^{-\frac{1}{2}}(\partial \Omega)} \right).
\end{equation}
Finally, by going back to $(\ref{LSE=RHS+AI+AII})$ and making use of the estimation $(\ref{EstimationAI})$, we obtain 
\begin{equation}\label{1918}
Y(z_{m}) -  \alpha \, \sum_{j = 1 \atop j \neq m}^{M} G(z_{m};z_{j}) \, Y(z_{j})  =  S(z_{m}) + \mathcal{O}\left( \overline{\alpha} \, a^{\frac{(1 - \delta) (1-h)}{3(3-\delta)}} \; \left\Vert g \right\Vert_{\mathbb{H}^{-\frac{1}{2}}(\partial \Omega)} \right).
\end{equation}
Taking the difference between $(\ref{1924})$ and $(\ref{1918})$ gives us the following  algebraic system:
\begin{equation*}
\left( Y_{m} - Y(z_{m}) \right) -  \sum_{j = 1 \atop j \neq m}^{M} G(z_{m};z_{j}) \; \overline{\alpha} \, a^{1-h} \, \left( Y_{j} - Y(z_{j}) \right) = \mathcal{O}\left( \overline{\alpha} \, a^{\frac{(1 - \delta) (1-h)}{3(3-\delta)}}  \; \left\Vert g \right\Vert_{\mathbb{H}^{-\frac{1}{2}}(\partial \Omega)} \right).
\end{equation*}
Consequently, 
\begin{equation*}
\left\vert Y_{m} - Y(z_{m}) \right\vert = \mathcal{O}\left( \overline{\alpha} \, a^{\frac{(1 - \delta)(1-h)}{3(3-\delta)}} \; \left\Vert g \right\Vert_{\mathbb{H}^{-\frac{1}{2}}(\partial \Omega)} \right), \quad m=1,\cdots,M.
\end{equation*}
In particular 
\begin{equation}\label{maxY-Y}
\underset{1 \leq m \leq M}{\max} \left\vert Y_{m} - Y(z_{m}) \right\vert = \mathcal{O}\left( \overline{ \alpha } \, a^{\frac{(1-h)(1 - \delta)}{3(3-\delta)}} \; \left\Vert  g \right\Vert_{\mathbb{H}^{-\frac{1}{2}}(\partial \Omega)} \right).
\end{equation}
The previous estimation confirm the convergence of the discrete algebraic system to the continuous L.S.E.
\subsection{Finishing the proof of Theorem \ref{principal-Thm}}
Now, we set $\bm{J}$ to be: 
\begin{equation}\label{DefJ}
\bm{J} := \omega^{2} \, \frac{\rho_{1}}{k_{1}} \,  \langle v^{g}; p^{f} \rangle_{\mathbb{L}^{2}(D)} + P^{2} \, \langle u^{g}; p^{f} \rangle_{\mathbb{L}^{2}(\Omega)} = \sum_{j=1}^{M} \left[\omega^{2} \, \frac{\rho_{1}}{k_{1}} \,   \langle v^{g}_{j}; p^{f} \rangle_{\mathbb{L}^{2}(D_{j})} + P^{2} \,  \langle u^{g}_{j}; p^{f} \rangle_{\mathbb{L}^{2}(\Omega_{j})} \right].  
\end{equation}
Then, by using the Taylor expansion for the function $p^{f}(\cdot)$ near the centres, we get: 
\begin{equation}\label{Equa0729}
\bm{J} =  \sum_{j=1}^{M} p^{f}(z_{j}) \, \left[ \omega^{2} \, \frac{\rho_{1}}{k_{1}}   \, \int_{D_{j}} v^{g}_{j}(x) \, dx + P^{2} \,  \int_{\Omega_{j}} u^{g}_{j}(x) \, dx \right] + \bm{Err_J},
\end{equation}
where 
\begin{eqnarray*}
\bm{Err_J} &:=& \omega^{2} \, \frac{\rho_{1}}{k_{1}} \,  \sum_{j=1}^{M} \int_{D_{j}} v^{g}_{j}(x) \, \int_{0}^{1} (x - z_{j}) \cdot \nabla p^{f}(z_{j} + t ( x - z_{j}) ) \, dt \, dx \\ &+& P^{2} \, \sum_{j=1}^{M} \int_{\Omega_{j}} u^{g}_{j}(x)  \, \int_{0}^{1} (x - z_{j}) \cdot \nabla p^{f}(z_{j} + t ( x - z_{j}) ) \, dt  \, dx,  
\end{eqnarray*}
which can be estimated, by recalling that $\rho_{1} \sim 1; \, k_{1} \sim a^{2}$ and $\left\vert \Omega_{j} \right\vert \sim a^{1-h}$, as
\begin{eqnarray*}
\left\vert \bm{Err_J} \right\vert 
& \lesssim & a^{-1} \, \sum_{j=1}^{M} \left\Vert v^{g}_{j} \right\Vert_{\mathbb{L}^{2}(D_{j})} \, \left\Vert  \nabla p^{f} \right\Vert_{\mathbb{L}^{2}(D_{j})} + a^{\frac{(1-h)}{3}} \, P^{2} \, \sum_{j=1}^{M} \left\Vert u^{g}_{j} \right\Vert_{\mathbb{L}^{2}(\Omega_{j})}  \, \left\Vert \nabla p^{f} \right\Vert_{\mathbb{L}^{2}(\Omega_{j})} \\ 
& \lesssim & a^{-1} \, \left\Vert v^{g} \right\Vert_{\mathbb{L}^{2}(D)} \, \left\Vert  \nabla p^{f} \right\Vert_{\mathbb{L}^{2}(D)} + a^{\frac{(1-h)}{3}} \, P^{2} \, \left\Vert u^{g} \right\Vert_{\mathbb{L}^{2}(\Omega)}  \, \left\Vert \nabla p^{f} \right\Vert_{\mathbb{L}^{2}(\Omega)} \\
& \overset{(\ref{Add-proof-1})}{\lesssim} & a^{\frac{1}{2}} \, \left\Vert v^{g} \right\Vert_{\mathbb{L}^{2}(D)} \, \left\Vert f \right\Vert_{\mathbb{H}^{-\frac{1}{2}}(\partial \Omega)} + a^{\frac{(1-h)}{3}} \, P^{2} \, \left\Vert u^{g} \right\Vert_{\mathbb{L}^{2}(\Omega)}  \, \left\Vert \nabla p^{f} \right\Vert_{\mathbb{L}^{2}(\Omega)}.
\end{eqnarray*}
Next, we estimate $\left\Vert \nabla p^{f} \right\Vert_{\mathbb{L}^{2}(\Omega)}$. To do this, we recall from $(\ref{ETVKT})$ that
\begin{equation*}
\nabla p^{f}(x) = \langle \nabla G(x,\cdot) ; f \rangle_{\mathbb{H}^{\frac{1}{2}}(\partial \Omega) \times \mathbb{H}^{-\frac{1}{2}}(\partial \Omega)}. 
\end{equation*}
Then, 
\begin{equation*}
\left\Vert \nabla p^{f} \,  \right\Vert_{\mathbb{L}^{2}(\Omega)} \leq \left\Vert  f \right\Vert_{\mathbb{H}^{-\frac{1}{2}}(\partial \Omega)} \; \left[ \int_{\Omega} \left\Vert \nabla G(x,\cdot) \right\Vert^{2}_{\mathbb{H}^{\frac{1}{2}}(\partial \Omega)} \,  dx \right]^{\frac{1}{2}} = \mathcal{O}\left( \left\Vert  f \right\Vert_{\mathbb{H}^{-\frac{1}{2}}(\partial \Omega)} \right). 
\end{equation*}
Then, by using $(\ref{AprioriEstvf})$, we have
\begin{equation*}
\left\vert \bm{Err_J} \right\vert \lesssim  \left\Vert f \right\Vert_{\mathbb{H}^{-\frac{1}{2}}(\partial \Omega)} \, \left[ \left\Vert g \right\Vert_{\mathbb{H}^{-\frac{1}{2}}(\partial \Omega)} \, a^{2-h}  + a^{\frac{(1-h)}{3}} \, P^{2} \, \left\Vert u^{g} \right\Vert_{\mathbb{L}^{2}(\Omega)} \right].
\end{equation*}
The following lemma is needed to finish with the estimation of $\bm{Err_J}$.
\begin{lemma}\label{LZ-Lemma}
The function $u^{g}(\cdot)$, solution of $(\ref{EquaWf})$, satisfies
\begin{equation*}
\left\Vert u^{g} \right\Vert_{\mathbb{L}^{2}(\Omega)} \lesssim \frac{1}{P} \; \left\Vert g \right\Vert_{\mathbb{H}^{-\frac{1}{2}}\left( \partial \Omega \right)}. 
\end{equation*}
\end{lemma}
\begin{proof}
See \textbf{Subsection \ref{Lemma1020}}.
\end{proof}
Then, 
\begin{equation*}
\left\vert \bm{Err_J} \right\vert  \lesssim a^{\frac{(1-h)}{3}} \, P \,  \left\Vert g \right\Vert_{\mathbb{H}^{-\frac{1}{2}}(\partial \Omega)} \,    \left\Vert f \right\Vert_{\mathbb{H}^{-\frac{1}{2}}(\partial \Omega)}.
\end{equation*}
Going back to $(\ref{Equa0729})$, we obtain  
\begin{equation*}
\bm{J} =  \sum_{j=1}^{M} p^{f}(z_{j}) \, \left[ \omega^{2} \, \frac{\rho_{1}}{k_{1}} \, \int_{D_{j}} v^{g}_{j}(x) \, dx + P^{2} \,  \int_{\Omega_{j}} u^{g}_{j}(x) \, dx \right]  + \mathcal{O}\left( a^{\frac{(1-h)}{3}} \, P \, \left\Vert f \right\Vert_{\mathbb{H}^{-\frac{1}{2}}(\partial \Omega)}  \, \left\Vert g \right\Vert_{\mathbb{H}^{-\frac{1}{2}}(\partial \Omega)} \right). 
\end{equation*}
Remark that $u^{g}(\cdot)$ is solution of the L.S.E given by $(\ref{NewL.S.E})$, i.e. $u^{g}(\cdot) = Y(\cdot)$, in $\Omega$. Using this, we obtain
\begin{equation*}
\int_{\Omega_{j}} u^{g}_{j}(x) \, dx =  \int_{\Omega_{j}} Y(x) \, dx =  Y(z_{j}) \, \left\vert \Omega_{j} \right\vert +  \int_{\Omega_{j}} \int_{0}^{1} (x-z_{j}) \cdot \nabla Y(z_{j}+t(x-z_{j})) \, dt \, dx. 
\end{equation*}
Then, 
\begin{eqnarray}\label{J2T}
\nonumber
\bm{J} &=&  \sum_{j=1}^{M} p^{f}(z_{j}) \, \left[ \omega^{2} \, \frac{\rho_{1}}{k_{1}}  \, \int_{D_{j}} v^{f}_{j}(x) \, dx + P^{2} \, Y(z_{j}) \, \left\vert \Omega_{j} \right\vert  \right] \\ \nonumber
&+&  \sum_{j=1}^{M} p^{f}(z_{j}) \, \left[ P^{2} \,     \int_{\Omega_{j}} \int_{0}^{1} (x-z_{j}) \cdot \nabla Y(z_{j}+t(x-z_{j})) \, dt \, dx \right] \\ &+&  \mathcal{O}\left( a^{\frac{(1-h)}{3}} \, P \, \left\Vert f \right\Vert_{\mathbb{H}^{-\frac{1}{2}}(\partial \Omega)}  \, \left\Vert g \right\Vert_{\mathbb{H}^{-\frac{1}{2}}(\partial \Omega)} \right). 
\end{eqnarray}
We estimate the second term on the R.H.S, as 
\begin{eqnarray}\label{0539}
\nonumber
\left\vert \cdots \right\vert & \lesssim & P^{2}  \sum_{j=1}^{M} \left\vert p^{f}(z_{j}) \right\vert \, \left\vert      \int_{\Omega_{j}} \int_{0}^{1} (x-z_{j}) \cdot \nabla Y(z_{j}+t(x-z_{j})) \, dt \, dx \right\vert \\ \nonumber
& \leq & P^{2} \,  \left( \sum_{j=1}^{M} \left\vert p^{f}(z_{j}) \right\vert^{2} \right)^{\frac{1}{2}} \, \left( \sum_{j=1}^{M} \left\vert \int_{\Omega_{j}} \int_{0}^{1} (x-z_{j}) \cdot \nabla Y(z_{j}+t(x-z_{j})) \, dt \, dx \right\vert^{2} \right)^{\frac{1}{2}} \\
&=& \mathcal{O}\left( P^{2} \, \left( \sum_{j=1}^{M} \left\vert p^{f}(z_{j}) \right\vert^{2} \right)^{\frac{1}{2}} \,  a^{\frac{5}{6}(1-h)}  \, \left\Vert \nabla Y \right\Vert_{\mathbb{L}^{2}(\Omega)} \right).
\end{eqnarray}
At this stage, we need first to estimate $\overset{M}{\underset{j=1}{\sum}}  \left\vert p^{f}(z_{j}) \right\vert^{2}$. To achieve this, we recall that $p^{f}(\cdot)$ is solution of $(\ref{EquaKg-introdution})$ and we set $\tilde{p}^{f}(\cdot)$ to be solution of 
\begin{align}\label{Equatildepf}
\begin{cases}  
\Delta \, \tilde{p}^{f} = 0 \quad \text{in} \quad  \Omega,  \\ 
\partial_{\nu} \tilde{p}^{f} =  f \;\;\;  \text{on} \quad \partial \Omega. 
\end{cases}
\end{align}
Now, by subtracting $(\ref{EquaKg-introdution})$ from $(\ref{Equatildepf})$, we get 
\begin{align*}
\begin{cases}  
\left( \Delta + \omega^{2} \, n^{2} \right) \, \left( p^{f} - \tilde{p}^{f} \right) = - \, \omega^{2} \, n^{2} \, \tilde{p}^{f}  \quad \text{in} \quad  \Omega,  \\ 
\qquad \qquad \;\; \partial_{\nu} \left( p^{f} - \tilde{p}^{f} \right) =  0 \;\;\;  \text{on} \quad \partial \Omega. 
\end{cases}
\end{align*}
Its solution takes the following form 
\begin{equation*}
    \left( p^{f} - \tilde{p}^{f} \right)(z) = \omega^{2} \; \int_{\Omega} G(z,y) \, n^{2}(y) \, \tilde{p}^{f}(y) \, dy, \quad z \in \Omega. 
\end{equation*}
By taking the modulus we get
\begin{equation}\label{pf-tildepf}
   \left\vert \left( p^{f} - \tilde{p}^{f} \right)(z) \right\vert \leq \omega^{2} \; \left\Vert n^{2} \right\Vert_{\mathbb{L}^{\infty}(\Omega)} \; \left\Vert G(z,\cdot) \right\Vert_{\mathbb{L}^{2}(\Omega)} \; \left\Vert \tilde{p}^{f} \right\Vert_{\mathbb{L}^{2}(\Omega)} \lesssim \left\Vert \tilde{p}^{f} \right\Vert_{\mathbb{L}^{2}(\Omega)},  
\end{equation}
where the last estimation is a consequence of the $\mathbb{L}^{2}(\Omega)$-integrability of the Green's kernel $G(z,\cdot)$, uniformly on $z \in \Omega$, and the boundedness of $\omega^{2} \; \left\Vert n^{2} \right\Vert_{\mathbb{L}^{\infty}(\Omega)}$. In addition, we use the fact that $(\ref{Equatildepf})$ is a well posed problem to derive 
\begin{equation}\label{WPP}
\left\Vert \tilde{p}^{f} \right\Vert_{\mathbb{L}^{2}(\Omega)} \lesssim \left\Vert f \right\Vert_{\mathbb{H}^{-\frac{1}{2}}(\partial \Omega)}. 
\end{equation}
Then, by gathering $(\ref{pf-tildepf})$ and $(\ref{WPP})$, we obtain:
\begin{equation}\label{MMTT}
   \left\vert \left( p^{f} - \tilde{p}^{f} \right)(z) \right\vert  \lesssim \left\Vert f \right\Vert_{\mathbb{H}^{-\frac{1}{2}}(\partial \Omega)}.  
\end{equation}
For $\tilde{p}^{f}(\cdot)$ using its harmonicity in $\Omega$, we deduce that 
\begin{equation*}
    \tilde{p}^{f}(z_{j}) = \frac{1}{\left\vert \mathcal{B}_{j} \right\vert} \; \int_{\mathcal{B}_{j}} \tilde{p}^{f}(x) \; dx, 
\end{equation*}
where $\mathcal{B}_{j}$ is the largest ball centred at $z_{j}$ and contained in the cube $\Omega_{j}$. We observe that, for $1 \leq j \leq M$, we have $\left\vert \mathcal{B}_{j} \right\vert = \left\vert \mathcal{B}_{j_{0}} \right\vert \sim a^{1-h} \sim M^{-1}$. Then, using Hölder inequality, we deduce that: 
\begin{equation}\label{LAM}
    \left\vert \tilde{p}^{f}(z_{j}) \right\vert \leq \left\vert \mathcal{B}_{j} \right\vert^{-\frac{1}{2}} \; \left\Vert \tilde{p}^{f} \right\Vert_{\mathbb{L}^{2}( \mathcal{B}_{j})}. 
\end{equation}
Therefore, 
\begin{equation*}
    \sum_{j=1}^{M} \left\vert p^{f}(z_{j}) \right\vert^{2} = \sum_{j=1}^{M} \left\vert \tilde{p}^{f}(z_{j}) + \left( p^{f}(z_{j}) -  \tilde{p}^{f}(z_{j}) \right) \right\vert^{2}   \lesssim  \sum_{j=1}^{M} \left\vert \tilde{p}^{f}(z_{j}) \right\vert^{2} + \sum_{j=1}^{M} \left\vert  p^{f}(z_{j}) -  \tilde{p}^{f}(z_{j})  \right\vert^{2}. 
\end{equation*}
By making the use of $(\ref{LAM})$ and $(\ref{MMTT})$, we obtain: 
\begin{eqnarray*}
    \sum_{j=1}^{M} \left\vert p^{f}(z_{j}) \right\vert^{2} & \lesssim & \sum_{j=1}^{M} \left\vert \mathcal{B}_{j} \right\vert^{-1} \; \left\Vert \tilde{p}^{f} \right\Vert^{2}_{\mathbb{L}^{2}( \mathcal{B}_{j})} + M \; \left\Vert f \right\Vert^{2}_{\mathbb{H}^{-\frac{1}{2}}(\partial \Omega)} \\
    & \lesssim & \left\vert \mathcal{B}_{j_{0}} \right\vert^{-1} \;  \left\Vert \tilde{p}^{f} \right\Vert^{2}_{\mathbb{L}^{2}( \overset{M}{\underset{j=1}{\cup}} \mathcal{B}_{j} )} + M \; \left\Vert f \right\Vert^{2}_{\mathbb{H}^{-\frac{1}{2}}(\partial \Omega)}. 
\end{eqnarray*}
As $\left\vert \mathcal{B}_{j_{0}} \right\vert_{-1}  \sim M$ and $\overset{M}{\underset{j=1}{\cup}} \mathcal{B}_{j} \subset \Omega$, we obtain:
\begin{equation}\label{PfMf}
    \sum_{j=1}^{M} \left\vert p^{f}(z_{j}) \right\vert^{2}  \lesssim  M \;  \left(   \left\Vert \tilde{p}^{f} \right\Vert^{2}_{\mathbb{L}^{2}( \Omega )} +  \left\Vert f \right\Vert^{2}_{\mathbb{H}^{-\frac{1}{2}}(\partial \Omega)} \right) \overset{(\ref{WPP})}{\lesssim} M \;    \left\Vert f \right\Vert^{2}_{\mathbb{H}^{-\frac{1}{2}}(\partial \Omega)}. 
\end{equation}
We continue with our estimation of $(\ref{0539})$ , using $(\ref{PfMf})$, 
\begin{equation*}
\left\vert \cdots \right\vert  \lesssim P^{2} \,  M^{\frac{1}{2}} \, \left\Vert f \right\Vert_{\mathbb{H}^{-\frac{1}{2}}(\partial \Omega)} \, a^{\frac{5}{6}(1-h)}  \, \left\Vert \nabla Y \right\Vert_{\mathbb{L}^{2}(\Omega)} 
\overset{(\ref{Yfctg})}{=} \mathcal{O}\left(  P^{2} \, a^{\frac{(1-h)}{3}}  \, \left\Vert f \right\Vert_{\mathbb{H}^{-\frac{1}{2}}(\partial \Omega)} \, \left\Vert g \right\Vert_{\mathbb{H}^{-\frac{1}{2}}(\partial \Omega)} \right).
\end{equation*}
Then, $\bm{J}$ becomes
\begin{equation*}\label{HAJ}
\bm{J} = \sum_{j=1}^{M} p^{f}(z_{j}) \, \left[ \omega^{2} \, \frac{\rho_{1}}{k_{1}} \, \int_{D_{j}} v^{g}_{j}(x) \, dx + P^{2} \, Y(z_{j}) \, \left\vert \Omega_{j} \right\vert  \right] + \mathcal{O}\left( a^{\frac{(1-h)}{3}} \, P^{2} \, \left\Vert f \right\Vert_{\mathbb{H}^{-\frac{1}{2}}(\partial \Omega)}  \, \left\Vert g \right\Vert_{\mathbb{H}^{-\frac{1}{2}}(\partial \Omega)} \right). 
\end{equation*}
To see how the parameter $P^{2}$ is related to the scattering coefficient $\alpha$, we set the following lemma.  
\begin{lemma}\label{EstimationRealphaImalpha}
The scattering coefficient $\alpha$, given by $(\ref{Defalpha})$, admits
the following estimation: 
\begin{equation}\label{EstimationRealpha}
\alpha  = - \; P^{2} \; a^{1-h}  + \mathcal{O}\left( a \right), 
\end{equation}
where $0 < h < 1$, and
\begin{equation*}
P^{2} := \frac{ - \, k_{0} \; \left( \langle 1; \overline{e}_{n_{0}} \rangle_{\mathbb{L}^{2}(B)} \right)^{2}}{\lambda_{n_{0}}^{B} \; c_{n_{0}}}  \sim 1, 
\end{equation*}
with respect to the parameter $a$. 
\end{lemma}
\begin{proof}
See \textbf{Subsection \ref{ESC}}. 
\end{proof}
Knowing that $\left\vert \Omega_{j} \right\vert = a^{1-h}$ and using $(\ref{EstimationRealpha})$, we deduce that $P^{2} \, Y(z_{j}) \, \left\vert \Omega_{j} \right\vert = - \, \alpha \, Y(z_{j})$. In addition, from $(\ref{DefYj})$, we have
\begin{equation*}
\omega^{2} \, \frac{\rho_{1}}{k_{1}}  \, \int_{D_{j}} v_{j}^{g}(x) \; dx =  \alpha  \, Y_{j}.
\end{equation*}
Hence, we end up with the coming formula
\begin{equation*}
\bm{J} =  \sum_{j=1}^{M} p^{f}(z_{j}) \, \alpha  \, \left[ Y_{j}  -  Y(z_{j}) \right] + \mathcal{O}\left( a^{\frac{(1-h)}{3}} \, P^{2} \, \left\Vert f \right\Vert_{\mathbb{H}^{-\frac{1}{2}}(\partial \Omega)}  \, \left\Vert g \right\Vert_{\mathbb{H}^{-\frac{1}{2}}(\partial \Omega)} \right). 
\end{equation*}
By taking the modulus in both sides of the previous inequality, we get:  
\begin{eqnarray*}
\left\vert \bm{J} \right\vert & \lesssim & \left\vert \alpha \right\vert \, M^{\frac{1}{2}}  \, \underset{1 \leq j \leq M}{\max} \left\vert Y_{j}  -  Y(z_{j}) \right\vert \; \left( \sum_{j=1}^{M} \left\vert p^{f}(z_{j}) \right\vert^{2} \right)^{\frac{1}{2}} +  a^{\frac{(1-h)}{3}} \, P^{2} \, \left\Vert f \right\Vert_{\mathbb{H}^{-\frac{1}{2}}(\partial \Omega)}  \, \left\Vert g \right\Vert_{\mathbb{H}^{-\frac{1}{2}}(\partial \Omega)} \\
& \overset{(\ref{PfMf})}{\lesssim} & \left\vert \alpha \right\vert \, M \, \underset{1 \leq j \leq M}{\max} \left\vert Y_{j}  -  Y(z_{j}) \right\vert \; \left\Vert f \right\Vert_{\mathbb{H}^{-\frac{1}{2}}(\partial \Omega)} +  a^{\frac{(1-h)}{3}} \, P^{2} \, \left\Vert f \right\Vert_{\mathbb{H}^{-\frac{1}{2}}(\partial \Omega)}  \, \left\Vert g \right\Vert_{\mathbb{H}^{-\frac{1}{2}}(\partial \Omega)}.
\end{eqnarray*}
Noticing that $M \sim a^{h-1}$, using the fact that $ \alpha  \sim P^{2} \, a^{1-h}$, see  $(\ref{EstimationRealpha})$ and taking into account the estimation derived in $(\ref{maxY-Y})$, we obtain: 
\begin{eqnarray*}
\left\vert \bm{J} \right\vert  & \lesssim &  a^{\frac{(1-h) (1 - \delta)}{3(3-\delta)}} \, P^{2} \, \left\Vert f \right\Vert_{\mathbb{H}^{-\frac{1}{2}}(\partial \Omega)}  \, \left\Vert g \right\Vert_{\mathbb{H}^{-\frac{1}{2}}(\partial \Omega)} +  a^{\frac{(1-h)}{3}}  \, P^{2} \, \left\Vert f \right\Vert_{\mathbb{H}^{-\frac{1}{2}}(\partial \Omega)}  \, \left\Vert g \right\Vert_{\mathbb{H}^{-\frac{1}{2}}(\partial \Omega)} \\ 
& = & \mathcal{O}\left( a^{\frac{(1-h) (1 - \delta)}{3(3-\delta)}} \, P^{2} \, \left\Vert f \right\Vert_{\mathbb{H}^{-\frac{1}{2}}(\partial \Omega)}  \, \left\Vert g \right\Vert_{\mathbb{H}^{-\frac{1}{2}}(\partial \Omega)} \right).
\end{eqnarray*}
Now, by gathering $(\ref{Lambda-d--Lambda-P})$ and $(\ref{DefJ})$, we obtain:   
\begin{equation*}
\left\vert \langle \left( \Lambda_{D} - \Lambda_{P} \right)(f); g \rangle_{\mathbb{H}^{\frac{1}{2}}(\partial \Omega) \times \mathbb{H}^{-\frac{1}{2}}(\partial \Omega)} \right\vert \lesssim a^{\frac{(1-h) (1 - \delta)}{3(3-\delta)}} \, P^{2} \, \left\Vert f \right\Vert_{\mathbb{H}^{-\frac{1}{2}}(\partial \Omega)}  \, \left\Vert g \right\Vert_{\mathbb{H}^{-\frac{1}{2}}(\partial \Omega)}.
\end{equation*}
This suggest, 
\begin{equation*}
\left\Vert \Lambda_{D} - \Lambda_{P} \right\Vert_{\mathcal{L}(\mathbb{H}^{-\frac{1}{2}}(\partial \Omega);\mathbb{H}^{\frac{1}{2}}(\partial \Omega))} \lesssim a^{\frac{(1-h) (1 - \delta)}{3(3-\delta)}} \, P^{2}.
\end{equation*}
This proves $(\ref{energy-D})$ and ends the proof of \textbf{Theorem \ref{principal-Thm}}.


\section{Proof of Lemma \ref{MTR}}\label{SectionInjective}
This section is divided into two subsections. In the first one we show that the continuous integral equation is invertible, by transforming the satisfied PDE into an integral equation involving  positive operators. In the second subsection we relate the algebraic system that we desire to invert, see  $(\ref{Equa1138})$, to the continuously invertible integral equation.  
\subsection{Invertibility of the continuous integral equations}\label{ICIE}
We observe that $u^{g}(\cdot)$, solution of $(\ref{EquaWf})$, can be represented as a solution of the following integral equation  
\begin{equation}\label{WIE}
    u^{g} + N^{\phi}_{\Omega}\left( \boldsymbol{\alpha} \, u^{g} \right) = \tilde{g},
\end{equation}
where $N^{\phi}_{\Omega}$ is the Newtonian operator defined by $(\ref{DefNPO})$ and $\boldsymbol{\alpha}(\cdot) :=  \left(P^{2} - \omega^{2} \, n^{2}(\cdot) \right)$. Observe that, as $P^{2} \gg 1$, we have $\boldsymbol{\alpha}(\cdot) > 0$.  After multiplying both sides of the previous equation with $\boldsymbol{\alpha}$, we obtain
\begin{equation}\label{InjIE}
   \boldsymbol{\alpha} \, u^{g} + \boldsymbol{\alpha} \, N^{\phi}_{\Omega}\left( \boldsymbol{\alpha} \, u^{g} \right) = \boldsymbol{\alpha} \, \tilde{g}.
\end{equation}
Next, we set $\gimel_{1}(\cdot;\cdot)$ to be the bilinear form defined by 
\begin{equation*}
    \gimel_{1}(u;v) := \langle \boldsymbol{\alpha} \, u; v \rangle_{\mathbb{L}^{2}(\Omega)} + \langle \boldsymbol{\alpha} \, N^{\phi}_{\Omega}\left( \boldsymbol{\alpha} \, u \right) ; v \rangle_{\mathbb{L}^{2}(\Omega)}, 
\end{equation*}
and, we set $\gimel_{2}(\cdot)$ to be the linear form defined by  
\begin{equation*}
    \gimel_{2}(v) := \langle \boldsymbol{\alpha} \, \tilde{g}; v \rangle_{\mathbb{L}^{2}(\Omega)}.
\end{equation*}
It is clear that $\gimel_{2}(\cdot)$ is a continuous linear form. For $\gimel_{1}(\cdot,\cdot)$, using the continuity of the Newtonian operator, we can prove that  $\gimel_{1}(\cdot,\cdot)$ is a continuous bilinear form. In addition, from the positivity of the Newtonian operator, we have 
\begin{equation*}
    \langle u ; \boldsymbol{\alpha} \, N^{\phi}_{\Omega}\left( \boldsymbol{\alpha} \, u \right) \rangle_{\mathbb{L}^{2}(\Omega)} \geq c_{0} \; 
 \left( \underset{\Omega}{\min}\left\vert \boldsymbol{\alpha} \right\vert \right)^{2} \; \left\Vert u \right\Vert^{2}_{\mathbb{L}^{2}(\Omega)},
\end{equation*}
where $c_{0}$ is the positivity constant of the  Newtonian operator. This allows us to obtain: 
\begin{equation*}
  \gimel_{1}(u,u) \geq \underset{\Omega}{\min}\left\vert \boldsymbol{\alpha} \right\vert \; \left( 1 + c_{0} \; \underset{\Omega}{\min}\left\vert \boldsymbol{\alpha} \right\vert \right) \;  \left\Vert u \right\Vert^{2}_{\mathbb{L}^{2}(\Omega)},
\end{equation*}
which proves the coercivity of $\gimel_{1}(\cdot,\cdot)$. Thanks to Lax-Milgram theorem, see \cite[Corollary 5.8]{brezis}, we deduce the existence and uniqueness of the solution to $(\ref{InjIE})$, in $\mathbb{L}^{2}(\Omega)$. In other words, the integral equation given by $(\ref{WIE})$ is invertible. Moreover the function $u^{g}(\cdot)$, solution of $(\ref{WIE})$, admits the following estimation:
\begin{equation}\label{Ss51}
    \left\Vert u^{g} \right\Vert_{\mathbb{L}^{2}(\Omega)} \leq \frac{\left\Vert \boldsymbol{\alpha} \right\Vert_{\mathbb{L}^{\infty}(\Omega)} }{ \underset{\Omega}{\min}\left\vert \boldsymbol{\alpha} \right\vert \; \left( 1 + c_{0} \; \underset{\Omega}{\min}\left\vert \boldsymbol{\alpha} \right\vert \right)} \; \left\Vert \tilde{g} \right\Vert_{\mathbb{L}^{2}(\Omega)} \lesssim \frac{1}{P^{2}} \; \left\Vert \tilde{g} \right\Vert_{\mathbb{L}^{2}(\Omega)}.
\end{equation}

\subsection{Relating the algebraic system to the continuous integral equation}
From $(\ref{0820})$, we have 
\begin{equation}\label{Equa1138}
Y_{m} -  \sum_{j \neq m} G(z_{m};z_{j}) \; \overline{\alpha} \, a^{1-h} Y_{j} = S(z_{m})   + \dfrac{\omega^{2} \, \rho_{1}}{k_{1} \, \alpha}  \; Rest_{m}.
\end{equation}
where $Y_{m}$ is defined by $(\ref{DefYj})$ and $Rest_{m}$ is given by $
(\ref{DefRestm})$. We have seen that $\alpha = - \, P^{2} \, a^{1-h} + \mathcal{O}\left( a \right)$, see $(\ref{EstimationRealpha})$. Then, by keeping its dominant part and using the fact that $\left\vert \Omega_{j} \right\vert = a^{1-h}$, for $1 \leq j \leq M$, we deduce that $\alpha$ can be approximated by $\alpha = - \, P^{2} \, \left\vert \Omega_{j} \right\vert$.  Hence, we rewrite the previous equation as 
\begin{eqnarray*}
Y_{m} \; + \; \sum_{j = 1 \atop j \neq m}^{M} G(z_{m};z_{j}) \; P^{2} \, \left\vert \Omega_{j} \right\vert \; Y_{j} \; &=&  S(z_{m}) + \dfrac{\omega^{2} \, \rho_{1}}{k_{1} \, \alpha}  \; Rest_{m} \\
Y_{m} \; + \; P^{2} \; \sum_{j = 1 \atop j \neq m}^{M} \int_{\Omega} G(z_{m};z_{j}) \,  \rchi_{\Omega_{j}}(x) \; Y_{j} \; dx &=& S(z_{m}) + \dfrac{\omega^{2} \, \rho_{1}}{k_{1} \, \alpha}  \; Rest_{m}.
\end{eqnarray*}
Multiplying the two sides of the previous equation with $\rchi_{\Omega_{m}}(\cdot)$ and summing up with respect to the index $m$, we get:
\begin{eqnarray*}
\sum_{m=1}^{M} \rchi_{\Omega_{m}}(\cdot) Y_{m} \; + \; P^{2} \; \sum_{m=1}^{M} \, \rchi_{\Omega_{m}}(\cdot) \sum_{j = 1 \atop j \neq m}^{M} \int_{\Omega} G(z_{m};z_{j}) \,  \rchi_{\Omega_{j}}(x) \; Y_{j} \; dx &=&  \sum_{m=1}^{M} \, \rchi_{\Omega_{m}}(\cdot) S(z_{m}) \\
&+& \dfrac{\omega^{2} \, \rho_{1}}{k_{1} \, \alpha}  \; \sum_{m=1}^{M} \rchi_{\Omega_{m}}(\cdot) Rest_{m}.
\end{eqnarray*}
which can be rewritten using the notations
\begin{equation}\label{DefYDefS}
    \boldsymbol{Y}(\cdot) := \sum_{m=1}^{M} \rchi_{\Omega_{m}}(\cdot) Y_{m} \;\; \text{and} \;\; \boldsymbol{S}(\cdot) := \sum_{m=1}^{M} \rchi_{\Omega_{m}}(\cdot) S(z_{m}) \;\; \text{and} \;\; \boldsymbol{R}(\cdot) := \sum_{m=1}^{M} \rchi_{\Omega_{m}}(\cdot) Rest_{m},
\end{equation}
as
\begin{equation}\label{DHB}
\boldsymbol{Y}(\cdot) \; + \; P^{2} \; \sum_{m=1}^{M} \, \rchi_{\Omega_{m}}(\cdot) \sum_{j = 1 \atop j \neq m}^{M} \int_{\Omega} G(z_{m};z_{j}) \,  \rchi_{\Omega_{j}}(x) \; Y_{j} \; dx = \boldsymbol{S}(\cdot) + \dfrac{\omega^{2} \, \rho_{1}}{k_{1} \, \alpha}  \; \boldsymbol{R}(\cdot). 
\end{equation}
The goal of the next lemma is to prove that the second term on the L.H.S converges, in $\mathbb{L}^{1}(\Omega)$, to a function which belongs to the range of the Newtonian operator. 
\begin{lemma}\label{Lemma51}
We have the following estimation
    \begin{equation}\label{LHSRHS}
      \left\Vert  N\left( \boldsymbol{Y} \right)(\cdot) - \sum_{m=1}^{M} \, \rchi_{\Omega_{m}}(\cdot) \sum_{j = 1 \atop j \neq m}^{M} \int_{\Omega} G(z_{m};z_{j}) \,  \rchi_{\Omega_{j}}(x) \; Y_{j} \; dx   \right\Vert_{\mathbb{L}^{1}(\Omega)} \lesssim a^{\frac{1}{6}(1-h)} \, \left\Vert \boldsymbol{Y} \right\Vert_{\mathbb{L}^{2}(\Omega)},  
    \end{equation}
    where $N$ is the Newtonian operator defined by 
    \begin{equation*}
    N(f)(x) := \int_{\Omega} G(x,y) \, f(y) \, dy.
    \end{equation*}
\end{lemma}
\begin{proof}
See \textbf{Subsection \ref{Prooflemma51}}. 
\end{proof}
Thanks to the previous lemma, we rewrite $(\ref{DHB})$ as    
\begin{equation}\label{IEI} 
\left( I  \; + \; P^{2} \, N \right) \, \left( \boldsymbol{Y} \right) (\cdot) =  \boldsymbol{S}(\cdot) +  \boldsymbol{r}(\cdot), 
\end{equation}
where 
\begin{equation*}
\boldsymbol{r}(\cdot) := \dfrac{\omega^{2} \, \rho_{1}}{k_{1} \, \alpha}  \; \boldsymbol{R}(\cdot) + P^{2} \, \left[ N\left( \boldsymbol{Y} \right) (\cdot)  - \sum_{m=1}^{M} \, \rchi_{\Omega_{m}}(\cdot) \sum_{j = 1 \atop j \neq m}^{M} \int_{\Omega} G(z_{m};z_{j}) \,  \rchi_{\Omega_{j}}(x) \; Y_{j} \; dx  \right], 
\end{equation*}
with
\begin{equation}\label{estimationr}
    \left\Vert \boldsymbol{r} \right\Vert_{\mathbb{L}^{2}(\Omega)} =  \mathcal{O}\left(  a^{\frac{(1-h)}{6}} \, P^{2} \, \left\Vert \boldsymbol{Y} \right\Vert_{\mathbb{L}^{2}(\Omega)}  \right). 
\end{equation}


We can check that $\boldsymbol{Y}(\cdot)$ is solution of $(\ref{EquaWf})$, with source term given by $\boldsymbol{S}(\cdot) +  \boldsymbol{r}(\cdot)$. We have proved in \textbf{Subsection \ref{ICIE}} the existence and uniqueness of solution of $(\ref{EquaWf})$, which is equivalent to the invertibility of $(\ref{IEI})$, i.e. $\boldsymbol{Y}(\cdot)$ exist. In addition, thanks to $(\ref{Ss51})$, the solution $\boldsymbol{Y}(\cdot)$ satisfies 
\begin{equation*}
    \left\Vert \boldsymbol{Y} \right\Vert_{\mathbb{L}^{2}(\Omega)}  \lesssim \frac{1}{P^{2}} \; \left\Vert \boldsymbol{S} + \boldsymbol{r} \right\Vert_{\mathbb{L}^{2}(\Omega)} \overset{(\ref{estimationr})}{\lesssim} \frac{1}{P^{2}} \; \left\Vert \boldsymbol{S}  \right\Vert_{\mathbb{L}^{2}(\Omega)} +  a^{\frac{1}{6}(1-h)} \; \left\Vert \boldsymbol{Y} \right\Vert_{\mathbb{L}^{2}(\Omega)}.
\end{equation*}
Hence, as $h<1$ and $a\ll 1$, we deduce that
\begin{equation*}
    \left\Vert \boldsymbol{Y} \right\Vert_{\mathbb{L}^{2}(\Omega)}  \lesssim  \frac{1}{P^{2}} \; \left\Vert \boldsymbol{S}  \right\Vert_{\mathbb{L}^{2}(\Omega)}.
\end{equation*}
This implies the injectivity of $(\ref{IEI})$. In addition, it is known that any injective linear map between two finite dimensional vector spaces of the same dimension is surjective. This proves the surjectivity and, consequently, the bijectivity of $(\ref{IEI})$. Hence, we have also the invertibility of the algebraic system $(\ref{ASEqua1})$. In addition, as by construction, 
\begin{equation*}
\Vert \boldsymbol{Y} \Vert^2_{\mathbb{L} ^{2}(\Omega)} = \sum_{m=1}^{M} \vert Y_m\vert^2\vert \Omega_m\vert = \left\vert \Omega_{m_{0}} \right\vert \; \sum_{m=1}^{M} \left\vert Y_{m} \right\vert^{2},     
\end{equation*}
similarly for $\boldsymbol{S}$,  with the $\Omega_m$'s having equivalent volumes, we deduce the estimate 
\begin{equation*}
    \left( \sum_{m=1}^{M} \left\vert Y_{m} \right\vert^{2} \right)^{\frac{1}{2}} \lesssim  \frac{1}{P^{2}} \;  \left( \sum_{m=1}^{M} \left\vert S(z_{m}) \right\vert^{2} \right)^{\frac{1}{2}}.
\end{equation*} 
This concludes the proof of \textbf{Lemma \ref{MTR}}. 






\section{Appendix}\label{Appendix}

This section is organised as follows. We start by proving \textbf{Lemma \ref{LemmaNp}} related to the smallness of the Newtonian operator $N^{p}$ with respect to the parameter $P$. Next, it is important to first examine the proof of \textbf{Lemma \ref{LemmaG=phi+Remainder}}, on the analysis of the Green's kernel decomposition $G(\cdot;\cdot) = \phi(\cdot;\cdot) + \mathcal{R}(\cdot;\cdot)$, before moving on to the proof of \textbf{Lemma \ref{ADZ-Lemma}}, giving us an a priori estimation satisfied by the acoustic field $v^{g}(\cdot)$. Then, we continue with the proof of \textbf{Lemma \ref{LZ-Lemma}}. Later, we examine the proof of \textbf{Lemma \ref{EstimationRealphaImalpha}}, which gives us an estimation of the scattering coefficient $\alpha$.  Finally, we conclude this section by proving \textbf{Lemma \ref{Lemma51}}.


 
\subsection{Proof of Lemma \ref{LemmaNp}}\label{AS2331}
From the spectral theory, we have
\begin{equation*}
\left\Vert N^{p} \right\Vert_{\mathcal{L}\left(\mathbb{L}^{2}(\Omega);\mathbb{L}^{2}(\Omega)\right)} = \left\Vert \mathcal{R}(P^{2};\Delta) \right\Vert_{\mathcal{L}\left(\mathbb{L}^{2}(\Omega);\mathbb{L}^{2}(\Omega)\right)} \leq \frac{1}{\dist\left(P^{2};\sigma(\Delta) \right)},
\end{equation*} 
where $\sigma(\Delta)$ stands for the spectrum of the Laplacian operator in $\mathbb{L}^{2}(\Omega)$. It is known that $\sigma(\Delta) := \left\{ \mu_{n} \right\}_{n \geq 1}$ such that $ 0 > \mu_{1} > \mu_{2} > \mu_{3} > \cdots \rightarrow - \infty$. Hence, we get $\dist\left(P^{2};\sigma(\Delta) \right) > P^{2}$. Consequently, 
\begin{equation*}
\left\Vert N^{p} \right\Vert_{\mathcal{L}\left(\mathbb{L}^{2}(\Omega);\mathbb{L}^{2}(\Omega)\right)}  \leq \frac{1}{P^{2}}.
\end{equation*}
This proves $(\ref{NormNewtonian})$. To prove $(\ref{TraceNormNewtonian})$, we start by remarking that %proceed into two steps.
%\begin{enumerate}
%\item Estimation of $\left\Vert N^{p} \right\Vert_{\mathcal{L}\left(\mathbb{L}^{2}(\Omega); \mathbb{H}^{1}(\Omega) \right)}$, \\
for an arbitrary function $f \in \mathbb{L}^{2}(\Omega)$, the function $N^{p}(f)$ satisfies the problem 
\begin{align*}
\begin{cases}  
\left(\Delta  - P^{2} \, I \right) N^{p}(f)  = - \, f \quad \text{in} \quad  \Omega,  \\ 
\qquad \quad \quad \partial_{\nu} N^{p}(f) = \quad 0 \;\;\; \,  \text{on} \quad \partial \Omega. 
\end{cases}
\end{align*}
Multiplying both sides of the first equation by $N^{p}(f)$ and integrating in $\Omega$, we get
\begin{eqnarray*}
\left\Vert \nabla N^{p}(f) \right\Vert^{2}_{\mathbb{L}^{2}(\Omega)} & \leq & P^{2} \, \left\Vert N^{p}(f) \right\Vert^{2}_{\mathbb{L}^{2}(\Omega)} + \left\Vert f \right\Vert_{\mathbb{L}^{2}(\Omega)} \; \left\Vert N^{p}(f) \right\Vert_{\mathbb{L}^{2}(\Omega)} \\
& \leq & P^{2} \, \left\Vert N^{p} \right\Vert^{2}_{\mathcal{L}\left(\mathbb{L}^{2}(\Omega);\mathbb{L}^{2}(\Omega)\right)} \, \left\Vert f \right\Vert^{2}_{\mathbb{L}^{2}(\Omega)} + \left\Vert f \right\Vert^{2}_{\mathbb{L}^{2}(\Omega)} \; \left\Vert N^{p} \right\Vert_{\mathcal{L}\left(\mathbb{L}^{2}(\Omega);\mathbb{L}^{2}(\Omega)\right)}.
\end{eqnarray*}
Hence,
\begin{equation}\label{EstimationnablaN}
\left\Vert \nabla N^{p} \right\Vert^{2}_{\mathcal{L}\left(\mathbb{L}^{2}(\Omega);\mathbb{L}^{2}(\Omega)\right)} 
 \leq  P^{2} \, \left\Vert N^{p} \right\Vert^{2}_{\mathcal{L}\left(\mathbb{L}^{2}(\Omega);\mathbb{L}^{2}(\Omega)\right)}  +  \left\Vert N^{p} \right\Vert_{\mathcal{L}\left(\mathbb{L}^{2}(\Omega);\mathbb{L}^{2}(\Omega)\right)} \overset{(\ref{NormNewtonian})}{=} \mathcal{O}\left( \frac{1}{P^{2}} \right).
\end{equation}
Then, 
\begin{equation*}
\left\Vert N^{p} \right\Vert_{\mathcal{L}\left(\mathbb{L}^{2}(\Omega);\mathbb{H}^{1}(\Omega)\right)} := \left[ \left\Vert N^{p} \right\Vert^{2}_{\mathcal{L}\left(\mathbb{L}^{2}(\Omega);\mathbb{L}^{2}(\Omega)\right)} + \left\Vert \nabla N^{p} \right\Vert^{2}_{\mathcal{L}\left(\mathbb{L}^{2}(\Omega);\mathbb{L}^{2}(\Omega)\right)} \right]^{\frac{1}{2}}, 
\end{equation*}
which, using $(\ref{NormNewtonian})$ and $(\ref{EstimationnablaN})$, becomes $\left\Vert N^{p} \right\Vert_{\mathcal{L}\left(\mathbb{L}^{2}(\Omega);\mathbb{H}^{1}(\Omega)\right)} = \mathcal{O}\left( \dfrac{1}{P} \right)$ 
%\label{NormNewtonianH1}
%\begin{equation*}
%\left\Vert N^{p} \right\Vert_{\mathcal{L}\left(\mathbb{L}^{2}(\Omega);\mathbb{H}^{1}(\Omega)\right)} = \mathcal{O}\left( \frac{1}{p} \right),
%\end{equation*}
and, by taking the trace operator we end up with the following estimation
\begin{equation*}
\left\Vert \gamma N^{p} \right\Vert_{\mathcal{L}\left(\mathbb{L}^{2}(\Omega); \mathbb{H}^{\frac{1}{2}}(\partial \Omega) \right)} = \mathcal{O}\left( \frac{1}{P} \right).
\end{equation*}
This proves $(\ref{TraceNormNewtonian})$ and ends the proof of \textbf{Lemma \ref{LemmaNp}}. 




\subsection{Proof of Lemma \ref{LemmaG=phi+Remainder}}\label{SubsectionProofLemma2.3}
We split our proof into two steps. The first one consists in estimating the singularity of the dominant part of $G(\cdot,\cdot)$, and in the second one we deal with the remainder term.
\begin{enumerate}
\item Estimation of the scale of $\phi$. \newline
We recall  that $\phi$ is solution of 
\begin{align*}
\begin{cases}  
\Delta \phi = - \, \delta \quad \text{in} \quad  \Omega_{a},  \\ 
\partial_{\nu} \phi =  \;\; 0 \quad \text{on} \quad \partial \Omega_{a}, 
\end{cases}
\end{align*}  
where $\Omega_{a} := z + a \, \Omega_{1}$, and $\Omega_{1}$ is bounded connected domain such that $\left\vert \Omega_{1} \right\vert \sim 1$. Now, let $f$ be an arbitrary function, then
\begin{equation}\label{IRS}
u(x) := \int_{\Omega_{a}} \phi(x,y) f(y) \, dy, 
\end{equation}
is a solution of 
\begin{align*}
\begin{cases}  
- \Delta u =  f \quad \text{in} \quad  \Omega_{1},  \\ 
\partial_{\nu} u =  0 \;\; \quad \text{on} \quad \partial \Omega_{1}. 
\end{cases}
\end{align*}
By scaling the previous PDE, we get:  
\begin{align}\label{ScaledPDE}
\begin{cases}  
- \Delta \tilde{u} =  a^{2} \, \tilde{f} \quad \text{in} \quad  \Omega,  \\ 
\partial_{\nu} \tilde{u} =  0 \quad \text{on} \quad \partial \Omega. 
\end{cases}
\end{align}
Moreover, by scaling the integral solution representation, see $(\ref{IRS})$, we obtain:  
\begin{equation}\label{IRSscaled}
\tilde{u}(x) = a^{3} \; \int_{\Omega_{1}} \tilde{\phi}(x,y) \tilde{f}(y) \, dy. 
\end{equation}
In addition, the integral equation solution of the scaled PDE, see $(\ref{ScaledPDE})$, is 
\begin{equation}\label{IRSscaledPDE}
\tilde{u}(x) = a^{2} \; \int_{\Omega_{1}} \phi^{\star}(x,y) \tilde{f}(y) \, dy,
\end{equation}
where $\phi^{\star}$ is solution of 
\begin{align*}
\begin{cases}  
\Delta \phi^{\star} = - \, \delta \quad \text{in} \quad  \Omega_{1},  \\ 
\partial_{\nu} \phi^{\star} =  \;\; 0 \quad \text{on} \quad \partial \Omega_{1}. 
\end{cases}
\end{align*}
By gathering $(\ref{IRSscaled})$ and $(\ref{IRSscaledPDE})$, we deduce 
\begin{equation*}
a^{3} \; \int_{\Omega_{1}} \tilde{\phi}(x,y) \tilde{f}(y) \, dy = a^{2} \; \int_{\Omega_{1}} \phi^{\star}(x,y) \tilde{f}(y) \, dy, \qquad \forall \;\; f. 
\end{equation*}
This implies the following scale 
\begin{equation}\label{SFS}
\tilde{\phi}(\cdot,\cdot) = a^{-1} \; \phi^{\star}(\cdot,\cdot).
\end{equation}
\begin{remark}\label{RemarkSAFS}
In the singularity analysis point of view, we deduce, from $(\ref{SFS})$, that
\begin{equation}\label{phisin}
\phi(x,y) \sim \frac{1}{\left\vert x - y \right\vert}, \quad x \neq y.
\end{equation}  
\end{remark}
\item Estimation of the remainder term. 
\newline
By multiplying both sides of $(\ref{RemainderPDE})$ by $\phi(\cdot,\cdot)$, solution of $( \ref{phi-harmonic})$, integrating by parts, using the fact that $\partial_{\nu} \mathcal{R}_{\big|_{\partial \Omega}} = \partial_{\nu} \phi_{\big|_{\partial \Omega}} = 0$, $\Delta \phi = - \delta$, in $\Omega$, and multiplying by $n^{2}(\cdot)$, we obtain:
\begin{equation}\label{RIEII}
\left( I - \omega^{2} \; n^{2}(\cdot) \; N^{\phi} \right)\left(n^{2} \, \mathcal{R}(\cdot,y) \right)(x) = \omega^{2} \, n^{2}(x) \,  \int_{\Omega} n^{2}(t) \, \phi(x,t) \, \phi(t,y) \, dt. 
\end{equation} 
Based on \textbf{Remark \ref{RemarkSAFS}}, we have the following estimation on the R.H.S, 
\begin{equation*}
\left\vert \omega^{2} \, n^{2}(x) \, \int_{\Omega} n^{2}(t) \, \phi(x,t) \, \phi(t,y) \, dt \right\vert \overset{(\ref{phisin})}{ \lesssim}  \int_{\Omega} \frac{1}{\left\vert t - x \right\vert} \, \frac{1}{\left\vert t - y \right\vert} \, dt  \overset{(\star)}{\leq}  c_{2}(\Omega) \, \left\vert x - y \right\vert + c_{2}(\Omega) = \mathcal{O}\left( 1 \right),
\end{equation*}
where $(\star)$ is justified in
\cite[Lemma 4.1]{Valdivia}. This implies that the term on the R.H.S of $(\ref{RIEII})$ is in $\mathbb{L}^{p}(\Omega), \; \forall \, p \in [1,+\infty]$. Knowing that, for any bounded domain $\Omega \subset \mathbb{R}^{3}$, the Newtonian operator $N^{\phi}$ is a bounded operator from  $\mathbb{L}^{p}(\Omega)$ onto $\mathbb{W}^{2,p}(\Omega)$, see \cite{colton2019inverse}, we deduce that, for fixed $y \in \Omega$, the term $n^{2}(\cdot) \, \mathcal{R}(\cdot,y)$ is in $  \mathbb{W}^{2, p}(\Omega), \; \forall \, p \in [1,+\infty]$. In particular, $\mathcal{R}(\cdot,y) \in \mathbb{L}^{\infty}(\Omega)$. 
\end{enumerate}
This concludes the proof of \textbf{Lemma \ref{LemmaG=phi+Remainder}}.

\subsection{Proof of Lemma \ref{ADZ-Lemma}}\label{Proof-A-Priori-Estimate}
We start by recalling, from $(\ref{L.S.E.vf})$, that $v^{g}(\cdot)$ is solution of:
\begin{equation}\label{L.S.E.vf.Lemma}
v^{g}(x) - \omega^{2} \, \frac{\rho_{1}}{k_{1}} \; \int_{D} G(x,y) \, v^{g}(y) \, dy = S(x), \quad x \in D.
\end{equation}
In the sequel, we divide the proof into two steps. 
\begin{enumerate}
\item[]
\item The case of one droplet. 
Using the decomposition $(\ref{G=phi+Remainder})$, of the Green's kernel $G(\cdot,\cdot)$, we rewrite $(\ref{L.S.E.vf.Lemma})$ as  
\begin{equation*}
v^{g}(x) - \omega^{2} \, \frac{\rho_{1}}{k_{1}} \; \int_{D} \phi(x,y) \, v^{g}(y) \, dy = S(x) + \omega^{2} \, \frac{\rho_{1}}{k_{1}} \; \int_{D} \mathcal{R}(x,y) \, v^{g}(y) \, dy.
\end{equation*}
Next, we denote by $\left(\lambda_{n}; e_{n} \right)$ the eigensystem associated to the Newtonian operator $N_{D}^{\phi}$ in $\mathbb{L}^{2}(D)$. Then, after taking the inner product with respect to $e_{n}$ and the square modulus in both sides of the previous equation, we get
\begin{eqnarray*}
%\langle v^{f}; e_{n} \rangle &=& \frac{k_{1}}{\left( k_{1} - \omega^{2} \;  \lambda_{n} \right)} \, \left[ \langle S; e_{n} \rangle + \frac{\omega^{2}}{k_{1}} \; \langle \int_{D} \mathcal{R}(\cdot,y) \, v^{f}(y) \, dy; e_{n} \rangle \right] \\
\left\vert \langle v^{g}; e_{n} \rangle \right\vert^{2} & \leq & \frac{2 \, k_{1}^{2}}{\left\vert k_{1} - \omega^{2} \; \rho_{1} \, \lambda_{n} \right\vert^{2}} \, \left[ \left\vert \langle S; e_{n} \rangle \right\vert^{2} + \left\vert \frac{\omega^{2} \, \rho_{1}}{k_{1}}  \right\vert^{2} \, \left\vert \langle \int_{D} \mathcal{R}(\cdot,y) \, v^{g}(y) \, dy; e_{n} \rangle \right\vert^{2} \right].
\end{eqnarray*}
Then, by summing up with respect to the index $n$ and taking into account the relations $(\ref{DispersionEqua})$ and $(\ref{ScaleBulk})$ we obtain 
\begin{equation*}
\left\Vert v^{g} \right\Vert^{2}_{\mathbb{L}^{2}(D)}  \lesssim a^{-2h} \; \left[ \left\Vert  S \right\Vert^{2}_{\mathbb{L}^{2}(D)} + a^{-4} \, \left\Vert  \int_{D} \mathcal{R}(\cdot,y) \, v^{g}(y) \, dy \right\Vert^{2}_{\mathbb{L}^{2}(D)} \right].
\end{equation*}
We estimate the second term on the R.H.S, as 
\begin{equation*}
\bm{R_1} := \left\Vert  \int_{D} \mathcal{R}(\cdot,y) \, v^{g}(y) \, dy \right\Vert^{2}_{\mathbb{L}^{2}(D)} 
%\\ \nonumber
%&=& \int_{D} \left\vert  \int_{D} \mathcal{R}(x,y) \, v^{f}(y) \, dy \right\vert^{2} \; dx \\ \nonumber
 \leq  \int_{D} \int_{D} \left\vert \mathcal{R}(x,y) \right\vert^{2} \, dy  \; dx \; \left\Vert v^{g} \right\Vert^{2}_{\mathbb{L}^{2}(D)}.
\end{equation*}
From \textbf{Lemma \ref{LemmaG=phi+Remainder}}, we know that $\mathcal{R}(\cdot,y) \in \mathbb{L}^{\infty}(\Omega)$. Then, $\int_{D} \int_{D} \left\vert \mathcal{R}(x,y) \right\vert^{2} \, dy  \; dx \lesssim a^{6},$
%\begin{equation*}
%\int_{D} \int_{D} \left\vert \mathcal{R}(x,y) \right\vert^{2} \, dy  \; dx \lesssim a^{6},
%\end{equation*}
and, 
\begin{equation}\label{Rvf}
\bm{R_1} := \left\Vert  \int_{D} \mathcal{R}(\cdot,y) \, v^{g}(y) \, dy \right\Vert^{2}_{\mathbb{L}^{2}(D)} \lesssim a^{6} \; \left\Vert v^{g} \right\Vert^{2}_{\mathbb{L}^{2}(D)}.
\end{equation}
Then, 
\begin{equation*}
\left\Vert v^{g} \right\Vert^{2}_{\mathbb{L}^{2}(D)}  \lesssim a^{-2h} \; \left\Vert  S \right\Vert^{2}_{\mathbb{L}^{2}(D)} + a^{2 - 2 \, h} \, \left\Vert  v^{g} \right\Vert^{2}_{\mathbb{L}^{2}(D)}, 
\end{equation*}
which 
%under the condition 
%\begin{equation}\label{CdtSing}
%1 + 2 \, \delta - 2 h > 0, 
%\end{equation}
becomes, 
\begin{equation*}
\left\Vert v^{g} \right\Vert_{\mathbb{L}^{2}(D)}  \lesssim a^{-h} \; \left\Vert  S \right\Vert_{\mathbb{L}^{2}(D)} = \mathcal{O}\left( a^{\frac{3}{2}-h} \right). 
\end{equation*}
\item[]
\item The case of multiple droplets. 
From $(\ref{L.S.E.vf.Lemma})$, by taking $x \in D_{m}$, we get:
\begin{equation*}
v^{g}_{m}  = \left( I - \frac{\omega^{2} \, \rho_{1}}{k_{1}}  \; N_{D_{m}}^{\phi} \right)^{-1} \left[ S_{m}  +  \frac{\omega^{2} \, \rho_{1}}{k_{1}} \;  \sum_{j=1 \atop j \neq m}^{M} \int_{D_{j}} G(\cdot,y) \, v^{g}_{j}(y) \, dy +  \frac{\omega^{2} \, \rho_{1}}{k_{1}} \;  \int_{D_{m}} \mathcal{R}(\cdot,y) \, v^{g}_{m}(y) \, dy \right].
\end{equation*}
By taking the $\mathbb{L}^{2}(D_{m})$-norm, using $(\ref{Rvf})$ and 
\begin{equation*}
 \left\Vert \left( I - \dfrac{\omega^{2} \, \rho_{1}}{k_{1}} \; N_{D_{m}}^{\phi} \right)^{-1} \right\Vert_{\mathcal{L}(\mathbb{L}^{2}(D_{m});\mathbb{L}^{2}(D_{m}))} \lesssim a^{-h},   
\end{equation*}
proved for the case of one droplet, we get: 
\begin{equation*}
\left\Vert v^{g}_{m} \right\Vert_{\mathbb{L}^{2}(D_{m})}   \lesssim  a^{-h} \; \left\Vert S_{m} \right\Vert_{\mathbb{L}^{2}(D_{m})} + a^{-2-h} \;  \sum_{j=1 \atop j \neq m}^{M} \left\Vert \int_{D_{j}} G(\cdot,y) \, v^{g}_{j}(y) \, dy \right\Vert_{\mathbb{L}^{2}(D_{m})}.
\end{equation*}
We estimate the second term on the R.H.S. 
\begin{eqnarray*}
\bm{R_2} &:=& \sum_{j=1 \atop j \neq m}^{M} \left\Vert \int_{D_{j}} G(\cdot,y) \, v^{g}_{j}(y) \, dy \right\Vert_{\mathbb{L}^{2}(D_{m})} \\
& \overset{(\ref{SG})}{\lesssim} & a^{3} \; \sum_{j=1 \atop j \neq m}^{M}  \frac{1}{\left\vert z_{m} - z_{j} \right\vert} \; \left\Vert v^{g}_{j} \right\Vert_{\mathbb{L}^{2}(D_{j})}  \lesssim  a^{3} \; \left[ \sum_{j=1 \atop j \neq m}^{M}  \frac{1}{\left\vert z_{m} - z_{j} \right\vert^{2}} \right]^{\frac{1}{2}} \; \left\Vert v^{g} \right\Vert_{\mathbb{L}^{2}(D)}. 
\end{eqnarray*}
Using $(\ref{SID})$ we reduce the previous estimation to:  
\begin{equation*} 
\bm{R_2} \lesssim  a^{3} \; d^{-1} \; \left\Vert v^{g} \right\Vert_{\mathbb{L}^{2}(D)} = \mathcal{O}\left( a^{\frac{8+h}{3}}  \; \left\Vert v^{g} \right\Vert_{\mathbb{L}^{2}(D)} \right).
\end{equation*}
Then, 
\begin{eqnarray*}
\left\Vert v^{g}_{m} \right\Vert_{\mathbb{L}^{2}(D_{m})}  & \lesssim & a^{-h} \; \left\Vert S_{m} \right\Vert_{\mathbb{L}^{2}(D_{m})} +  a^{\frac{2-2h}{3}}  \; \left\Vert v^{g} \right\Vert_{\mathbb{L}^{2}(D)} %\\
%\left\Vert v^{f}_{m} \right\Vert^{2}_{\mathbb{L}^{2}(D_{m})}  & \lesssim & a^{-2h} \; \left\Vert S_{m} \right\Vert^{2}_{\mathbb{L}^{2}(D_{m})} +  a^{\frac{2(2-2h)}{3}}  \; \left\Vert v^{f} \right\Vert^{2}_{\mathbb{L}^{2}(D)}.
\end{eqnarray*}
By taking the square, summing up with respect to the index $m$ and using the fact that $M \sim a^{h-1}$, we get, as $h<1$, 
\begin{eqnarray*}
\left\Vert v^{g} \right\Vert^{2}_{\mathbb{L}^{2}(D)}  & \lesssim & a^{-2h} \; \left\Vert S \right\Vert^{2}_{\mathbb{L}^{2}(D)} +  a^{\frac{(1-h)}{3}} \; \left\Vert v^{g} \right\Vert^{2}_{\mathbb{L}^{2}(D)} \\
\left\Vert v^{g} \right\Vert^{2}_{\mathbb{L}^{2}(D)}  & \lesssim & a^{-2h} \; \left\Vert S \right\Vert^{2}_{\mathbb{L}^{2}(D)}.
\end{eqnarray*}
Finally, 
\begin{equation}\label{Equa1001}
\left\Vert v^{g} \right\Vert_{\mathbb{L}^{2}(D)}   \lesssim  a^{-h} \; \left\Vert S \right\Vert_{\mathbb{L}^{2}(D)} = \mathcal{O}\left( a^{\frac{3}{2}-h} \right). 
\end{equation}
\end{enumerate} 
In addition, we can check, from $(\ref{EquaSg})$, that $S(\cdot)$ can be represented as a single layer with respect to the source data $g$, i.e. $S(\cdot) = SL(g)(\cdot)$. This implies, using the previous derived estimation, 
\begin{equation*}
\left\Vert v^{g} \right\Vert_{\mathbb{L}^{2}(D)}   \lesssim  a^{-h} \; \left\Vert SL(g) \right\Vert_{\mathbb{L}^{2}(D)}.
\end{equation*}
Next, we have 
\begin{eqnarray*}
    \left\Vert SL(g) \right\Vert_{\mathbb{L}^{2}(D)} := \left[\int_{D} \left\vert \int_{\partial \Omega} G(x,y) \, g(y) \, dy \right\vert^{2} \; dx \right]^{\frac{1}{2}} &=& \left[\int_{D} \left\vert \langle  G(x,\cdot)  ; g \rangle_{\mathbb{H}^{\frac{1}{2}}(\partial \Omega) \times \mathbb{H}^{-\frac{1}{2}}(\partial \Omega)} \right\vert^{2} \; dx \right]^{\frac{1}{2}} \\
    & \leq & \left\Vert g \right\Vert_{\mathbb{H}^{-\frac{1}{2}}(\partial \Omega)}  \; \left[\int_{D} \left\Vert G(x,\cdot)  \right\Vert^{2}_{\mathbb{H}^{\frac{1}{2}}(\partial \Omega)} \; dx \right]^{\frac{1}{2}}.
\end{eqnarray*}
As the function inside the integral is smooth, we end up with the following estimation:
\begin{equation*}
    \left\Vert SL(g) \right\Vert_{\mathbb{L}^{2}(D)}  \leq  \left\Vert g \right\Vert_{\mathbb{H}^{-\frac{1}{2}}(\partial \Omega)}  \; a^{\frac{3}{2}}.
\end{equation*}
The previous estimation combined with $(\ref{Equa1001})$ gives us
\begin{equation}
\left\Vert v^{g} \right\Vert_{\mathbb{L}^{2}(D)}   \lesssim  a^{\frac{3}{2}-h} \; \left\Vert g \right\Vert_{\mathbb{H}^{-\frac{1}{2}}(\partial \Omega)}. 
\end{equation}
This ends the proof of \textbf{Lemma \ref{ADZ-Lemma}}.    


\subsection{Proof of Lemma \ref{LZ-Lemma}}\label{Lemma1020}
The function $u^{g}(\cdot)$, solution $(\ref{EquaWf})$, satisfies the following integral equation 
\begin{equation*}
u^{g}(x) = \omega^{2} \, N^{p}\left( n^{2} \, u^{g} \right)(x) +  SL^{p}\left( g \right)(x),
\end{equation*}
where $SL^{p}$ is the Single-Layer operator. Then, by taking the $\mathbb{L}^{2}(\Omega)$-norm in both sides of the previous equation, we obtain:
\begin{eqnarray*}
\left\Vert u^{g} \right\Vert_{\mathbb{L}^{2}(\Omega)} & \leq & \omega^{2} \;  \left\Vert N^{p} \right\Vert_{\mathcal{L}(\mathbb{L}^{2}(\Omega);\mathbb{L}^{2}(\Omega))} \; \left\Vert n^{2} \right\Vert_{\mathbb{L}^{\infty}(\Omega)} \; \left\Vert  u^{g}  \right\Vert_{\mathbb{L}^{2}(\Omega)} + \left\Vert SL^{p}\left( g \right) \right\Vert_{\mathbb{L}^{2}(\Omega)} \\
\left\Vert u^{g} \right\Vert_{\mathbb{L}^{2}(\Omega)} & \overset{(\ref{CdtCvgS})}{\leq} & \left\Vert SL^{p}\left( g \right) \right\Vert_{\mathbb{L}^{2}(\Omega)}.
\end{eqnarray*} 
Next, we estimate $\left\Vert SL^{p}\left( g \right) \right\Vert_{\mathbb{L}^{2}(\Omega)}$. To achieve this, straightforward computations give us,  
\begin{eqnarray*}
\left\Vert SL^{p}\left( g \right) \right\Vert_{\mathbb{L}^{2}(\Omega)} & \leq & \left\Vert  g  \right\Vert_{\mathbb{H}^{-\frac{1}{2}}(\partial \Omega)} \; \left[ \int_{\Omega} \left\Vert G_{p}(x,\cdot) \right\Vert^{2}_{\mathbb{H}^{\frac{1}{2}}(\partial \Omega)} dx \, \right]^{\frac{1}{2}} \\
& \leq & \left\Vert  g  \right\Vert_{\mathbb{H}^{-\frac{1}{2}}(\partial \Omega)} \; \left\Vert \gamma \, N^{p} \right\Vert_{\mathcal{L}\left( \mathbb{L}^{2}(\Omega); \mathbb{H}^{\frac{1}{2}}(\partial \Omega) \right)} \overset{(\ref{TraceNormNewtonian})}{=} \mathcal{O}\left( \frac{1}{P} \left\Vert  g  \right\Vert_{\mathbb{H}^{-\frac{1}{2}}(\partial \Omega)} \right).
\end{eqnarray*}
Hence, 
\begin{equation*}
\left\Vert u^{g} \right\Vert_{\mathbb{L}^{2}(\Omega)} \lesssim  \frac{1}{P} \; \left\Vert  g  \right\Vert_{\mathbb{H}^{-\frac{1}{2}}(\partial \Omega)}. 
\end{equation*}
This concludes the proof of \textbf{Lemma \ref{LZ-Lemma}}.



\subsection{Proof of Lemma $\ref{EstimationRealphaImalpha}$}\label{ESC}
We know that, for $m$ fixed, 
\begin{equation*}
\alpha := \int_{D_{m}} W_{m}(x) \, dx  =   \int_{D_{m}} \left( \dfrac{k_{1}}{\omega^{2} \, \rho_{1}} \; I -  N_{D_{m}}^{\phi} \right)^{-1}\left( 1 \right)(x) \; dx.
\end{equation*}
By expanding the constant function 1 over the basis of the Newtonian operator $N_{D_{m}}^{\phi}$ we obtain 
\begin{equation}\label{DispersionEqua}
\alpha  =  \sum_{n} \; \langle 1 ; e_{n} \rangle_{\mathbb{L}^{2}(D_{m})} \; \int_{D_{m}} \left( \dfrac{k_{1}}{\omega^{2} \, \rho_{1}} \; I -  N_{D_{m}}^{\phi} \right)^{-1}\left( e_{n} \right)(x) \; dx  =  \sum_{n} \; \left( \langle 1 ; e_{n} \rangle \right)^{2} \;  \frac{\omega^{2} \, \rho_{1}}{\left( k_{1} - \omega^{2} \, \rho_{1} \, \lambda_{n}\right)}. 
\end{equation}
We chose $\omega^{2}$ such that 
\begin{equation}\label{closeres}
    \left\vert k_{1} - \omega^{2} \, \rho_{1} \, \lambda_{n} \right\vert = \begin{cases}
			a^{2+h}, & \text{if $n = n_{0}$}\\
            a^{2}, & \text{otherwise}
		 \end{cases}
\end{equation}
Then,
\begin{equation*}\label{alpha=n0+Remainder}
\alpha  =  \left( \langle 1 ; e_{n_{0}} \rangle_{\mathbb{L}^{2}(D_{m})} \right)^{2} \;  \;  \frac{\omega^{2} \, \rho_{1}}{\left( k_{1} - \omega^{2} \, \rho_{1} \, \lambda_{n_{0}}\right)} + \sum_{n \neq n_{0}} \; \left( \langle 1 ; e_{n} \rangle_{\mathbb{L}^{2}(D_{m})} \right)^{2} \; \;  \frac{\omega^{2} \, \rho_{1}}{\left( k_{1} - \omega^{2}  \, \rho_{1} \, \lambda_{n}\right)}  
\end{equation*}
We estimate the second term on the R.H.S as
\begin{equation*}
\left\vert T_{R.H.S} \right\vert  \lesssim  \sum_{n \neq n_{0}} \frac{\left\vert \langle 1 ; e_{n} \rangle_{\mathbb{L}^{2}(D_{m})} \right\vert^{2}}{ \left\vert k_{1} - \omega^{2} \, \rho_{1} \, \lambda_{n} \right\vert} \overset{(\ref{closeres})}{\lesssim}  a^{-2} \; \left\Vert 1 \right\Vert^{2}_{\mathbb{L}^{2}(D_{m})}  = \mathcal{O}\left( a \right).
\end{equation*}
Hence, 
\begin{equation}\label{alphabtreim}
\alpha  = \left( \langle 1 ; e_{n_{0}} \rangle_{\mathbb{L}^{2}(D_{m})} \right)^{2} \; \frac{\omega^{2} \, \rho_{1}}{\left( k_{1} - \omega^{2} \; \rho_{1} \, \lambda_{n_{0}}  \right)}  + \mathcal{O}\left( a \right). 
\end{equation}
We chose $\omega$ solution of the coming  dispersion equation 
\begin{equation*}\label{DispersionEquation}
k_{1} - \omega^{2} \; \rho_{1} \, \lambda_{n_{0}} =  c_{n_0} \, a^{2+h}, \quad c_{n_{0}} \in \mathbb{R}, \quad  c_{n_{0}} \, \sim 1.
\end{equation*}
By solving the previous quadratic equation we obtain 
\begin{equation*}
\omega^{2} = \frac{k_{1} - c_{n_{0}} \; a^{2+h}}{\rho_{1} \, \lambda_{n_{0}}}.
\end{equation*}
And, by plugging the previous relation into $(\ref{alphabtreim})$, we obtain
\begin{equation*}
\alpha = \frac{\left( k_{1} - c_{n_{0}} \; a^{2+h} \right)}{\lambda_{n_{0}} \; c_{n_{0}} \; a^{2+h}} \; \left( \langle 1 ; e_{n_{0}} \rangle_{\mathbb{L}^{2}(D_{m})} \right)^{2}  + \mathcal{O}\left( a \right).
\end{equation*}
Knowing that $\langle 1 ; e_{n_{0}} \rangle_{\mathbb{L}^{2}(D_{m})} = a^{\frac{3}{2}} \, \langle 1 ; \overline{e}_{n_{0}} \rangle_{\mathbb{L}^{2}(B)}$ and using the fact that $k_{1} = a^{2} \, k_{0};\, \lambda_{n_{0}} = a^{2} \, \lambda_{n_{0}}^{B}$, we rewrite the previous equation like
\begin{equation*}
\alpha = \frac{\left( k_{0} - c_{n_{0}} \; a^{h} \right)}{\lambda_{n_{0}}^{B} \; c_{n_0} } \; \left( \langle 1 ; \overline{e}_{n_{0}} \rangle_{\mathbb{L}^{2}(B)} \right)^{2} \; a^{1-h} + \mathcal{O}\left( a \right) = \frac{k_{0}}{\lambda_{n_{0}}^{B} \; c_{n_0} } \; \left( \langle 1 ; \overline{e}_{n_{0}} \rangle_{\mathbb{L}^{2}(B)} \right)^{2} \; a^{1-h} + \mathcal{O}\left( a \right).
\end{equation*}
We set $P^{2}$ to be the scaled dominant part of $\alpha$, i.e. 
\begin{equation*}\label{DefP2}
P^{2} := \frac{- \, k_{0} \; \left( \langle 1 ; \overline{e}_{n_{0}} \rangle_{\mathbb{L}^{2}(B)} \right)^{2}}{\lambda_{n_{0}}^{B} \; c_{n_0} }, 
\end{equation*}
and we end up with the following formula  
\begin{equation*}
\alpha  = - \, P^{2} \, a^{1-h} +\mathcal{O}\left(a \right). 
\end{equation*}
This concludes the proof of \textbf{Lemma $\ref{EstimationRealphaImalpha}$}.





\subsection{Proof of Lemma \ref{Lemma51}}\label{Prooflemma51}
We compute the $\mathbb{L}^{1}(\Omega)$-norm of the difference between the R.H.S and the L.H.S appearing in $(\ref{LHSRHS})$,  
\begin{equation*}
     \left\Vert \cdots \right\Vert_{\mathbb{L}^{1}(\Omega)}  :=  \int_{\Omega} \left\vert \int_{\Omega} G(y,x) \, \boldsymbol{Y}(x) \; dx - \sum_{m=1}^{M} \, \rchi_{\Omega_{m}}(y) \sum_{j = 1 \atop j \neq m}^{M} \int_{\Omega} G(z_{m};z_{j}) \,  \rchi_{\Omega_{j}}(x) \; Y_{j} \; dx  \right\vert \, dy.
\end{equation*}
In contrary to \textbf{Subsection \ref{SubSection43}}, where the cutting of $\Omega$ onto $\underset{j=1}{\overset{M}{\cup}} \Omega_{j}$ and $\underset{j=1}{\overset{\aleph}{\cup}} \Omega_{j}^{\star}$ was of capital importance to derive the exact dominant term related to $\int_{\Omega_{j}} u^{g}(x) \, dx$, for $1 \leq j \leq M$. Here, we need only to estimate functions  (not to extract dominant term) defined in $\Omega$, thus involving both $\underset{j=1}{\overset{M}{\cup}} \Omega_{j}$ and $\underset{j=1}{\overset{\aleph}{\cup}} \Omega_{j}^{\star}$.  Because, for every $1 \leq j \leq M$ and $1 \leq k \leq \aleph$, we have $\left\vert \Omega_{j} \right\vert \sim a^{1-h} \sim \left\vert \Omega_{k}^{\star} \right\vert$, we do not need to specify, in our comping computations, if we are dealing with $\left\{ \Omega_{j} \right\}_{j=1}^{M}$ or $\left\{ \Omega_{j}^{\star} \right\}_{j=1}^{\aleph}$. Moreover, to write short, we use the notation $\Omega_{j}$ for the domains  $\Omega_{j}^{\star}$. Then, 
    \begin{eqnarray*}
    \left\Vert \cdots \right\Vert_{\mathbb{L}^{1}(\Omega)}    & = & \int_{\Omega} \left\vert \sum_{m=1}^{M} \, \left( \int_{\Omega} G(y,x) \, \boldsymbol{Y}(x) \; dx -   \sum_{j = 1 \atop j \neq m}^{M} \int_{\Omega} G(z_{m};z_{j}) \,  \rchi_{\Omega_{j}}(x) \; Y_{j} \; dx \right) \rchi_{\Omega_{m}}(y) \right\vert \, dy \\
        & = & \sum_{m=1}^{M} \, \int_{\Omega_{m}} \left\vert  \int_{\Omega} G(y,x) \, \boldsymbol{Y}(x) \; dx -   \sum_{j = 1 \atop j \neq m}^{M} \int_{\Omega} G(z_{m};z_{j}) \,  \rchi_{\Omega_{j}}(x) \; Y_{j} \; dx  \right\vert \, dy.
    \end{eqnarray*}
Using the definition of $\boldsymbol{Y}(\cdot)$, see  $(\ref{DefYDefS})$, and the triangular inequality we rewrite the previous equation as 
    \begin{eqnarray*}
        \left\Vert \cdots \right\Vert_{\mathbb{L}^{1}(\Omega)} %& = & \sum_{m=1}^{M} \, \int_{\Omega_{m}} \left\vert  \int_{\Omega} G(y,x) \,  \rchi_{\Omega_{m}}(x) Y_{m} \; dx + \sum_{j = 1 \atop j \neq m}^{M} \int_{\Omega} \left( G(y,x)  -   G(z_{m};z_{j}) \right) \, \rchi_{\Omega_{j}}(x) \; Y_{j} \; dx  \right\vert \, dy \\
        & \leq & \sum_{m=1}^{M} \left\vert Y_{m} \right\vert \, \int_{\Omega_{m}} \int_{\Omega_{m}} \left\vert G(y,x) \right\vert \; dx \, dy \\ &+& \sum_{m=1}^{M} \, \int_{\Omega_{m}} \left\vert   \sum_{j = 1 \atop j \neq m}^{M} Y_{j} \; \int_{\Omega_{j}} \left( G(y,x)  -   G(z_{m};z_{j}) \right)   \; dx  \right\vert \, dy. 
    \end{eqnarray*}
Using Taylor expansion for the function $G(\cdot;\cdot)$, near the centres, we get 
\begin{equation*}
    G(y,x)  -   G(z_{m};z_{j}) = \int_{0}^{1} (x-z_{j}) \cdot \nabla G(z_{m};z_{j}+t(x-z_{j})) \, dt + \int_{0}^{1} (y-z_{m}) \cdot \nabla G(z_{m}+t(y-z_{m});x) \, dt. 
\end{equation*}
We plug the previous expansion into the previous estimation, use $(\ref{SG})$ and  the fact that $M \sim a^{h-1}$ to reduce the previous estimation to:
    \begin{eqnarray*}
      %  r & \lesssim & \sum_{m=1}^{M} \left\vert Y_{m} \right\vert \, \int_{\Omega_{m}} \int_{\Omega_{m}}  \frac{1}{\left\vert y - x \right\vert} \; dx \, dy + \sum_{m=1}^{M} \, \int_{\Omega_{m}}  \sum_{j = 1 \atop j \neq m}^{M} \left\vert Y_{j} \right\vert \; \int_{\Omega_{j}} \int_{0}^{1} \frac{\left\vert x-z_{j} \right\vert}{\left\vert z_{j}+t(x-z_{j}) - z_{m} \right\vert^{2}} dt   \; dx   \, dy \\ % &+& \sum_{m=1}^{M} \, \int_{\Omega_{m}}    \sum_{j = 1 \atop j \neq m}^{M} \left\vert Y_{j} \right\vert  \int_{\Omega_{j}}  \int_{0}^{1}  \frac{\left\vert y-z_{m} \right\vert}{\left\vert z_{m}+t(y-z_{m}) - x \right\vert^{2}} \, dt  \; dx   \, dy \\
        \left\Vert \cdots \right\Vert_{\mathbb{L}^{1}(\Omega)} & \lesssim & \underset{m}{\max} \int_{\Omega_{m}} \int_{\Omega_{m}}  \frac{1}{\left\vert y - x \right\vert} \; dx \, dy  \, \sum_{m=1}^{M} \left\vert Y_{m} \right\vert +  \underset{j}{\max} \int_{\Omega_{j}} \left\vert x-z_{j} \right\vert    \; dx  \, \sum_{j = 1 \atop j \neq m}^{M} \left\vert Y_{j} \right\vert \; \frac{1}{\left\vert z_{j} - z_{m} \right\vert^{2}} \\  %   \\ &+& a^{1-h} \, M \,  \underset{m}{\max} \int_{\Omega_{m}}  \left\vert y-z_{m} \right\vert \; dy \;  \sum_{j = 1 \atop j \neq m}^{M} \left\vert Y_{j} \right\vert \, \frac{1}{\left\vert z_{m} - z_{j} \right\vert^{2}}. 
  %  \end{eqnarray*}
%We use the fact that $M \sim a^{h-1}$ and, on the R.H.S, we remark that the second term and the third term are the same, to reduce the above estimation to:  
% \begin{eqnarray*}
 %     r & \lesssim & \sum_{m=1}^{M} \left\vert Y_{m} \right\vert \, \int_{\Omega_{m}} \int_{\Omega_{m}}  \frac{1}{\left\vert y - x \right\vert} \; dx \, dy + \, \underset{j}{\max} \int_{\Omega_{j}} \left\vert x-z_{j} \right\vert    \; dx  \, \sum_{j = 1 \atop j \neq m}^{M} \left\vert Y_{j} \right\vert \; \frac{1}{\left\vert z_{j} - z_{m} \right\vert^{2}}  \\
      & \lesssim &  \left\vert Y \right\vert \, M^{\frac{1}{2}} \, \underset{m}{\max} \int_{\Omega_{m}} \int_{\Omega_{m}}  \frac{1}{\left\vert y - x \right\vert} \; dx \, dy + \underset{j}{\max} \int_{\Omega_{j}} \left\vert x-z_{j} \right\vert    \; dx \, \left\vert Y \right\vert \, \left( \sum_{j = 1 \atop j \neq m}^{M} \frac{1}{\left\vert z_{j} - z_{m} \right\vert^{4}} \right)^{\frac{1}{2}}.
 \end{eqnarray*}   
 Moreover, we have the following estimations: 
 \begin{equation*}
     \underset{m}{\max} \int_{\Omega_{m}} \int_{\Omega_{m}}  \frac{1}{\left\vert y - x \right\vert} \; dx \, dy = \mathcal{O}\left( a^{\frac{5}{3}(1-h)} \right) \;\, \text{and} \;\, \underset{j}{\max} \int_{\Omega_{j}} \left\vert x-z_{j} \right\vert \; dx = \mathcal{O}\left( a^{\frac{4}{3}(1-h)} \right).
 \end{equation*}
 Then, 
 \begin{equation*}
    \left\Vert \cdots \right\Vert_{\mathbb{L}^{1}(\Omega)}  \lesssim   \left\vert Y \right\vert \,\, a^{\frac{7}{6}(1-h)} + a^{\frac{4}{3}(1-h)} \, \left\vert Y \right\vert \, d^{-2} = \mathcal{O}\left(  a^{\frac{2}{3}(1-h)} \, \left\vert Y \right\vert  \right) \overset{(\ref{DefYDefS})}{=} \mathcal{O}\left(  a^{\frac{1}{6}(1-h)} \, \left\Vert \boldsymbol{Y} \right\Vert_{\mathbb{L}^{2}(\Omega)}  \right).
 \end{equation*}
This concludes the proof of \textbf{Lemma \ref{Lemma51}}. 









\begin{thebibliography}{10}

%\bibitem{AA}
%A. Alsenafi, A. Ghandriche and M. Sini, 
%\newblock{Estimation of the eigenvalues and the integral of the eigenfunctions of the Newtonian  operator. Preprint.}
%%%%%%%%%%%%%%%%%%%%%%%%%%%%%%%%%%%%%%%%

\bibitem{Alessandrini-1988}
G. Alessandrini,
\newblock{Stable determination of conductivity by boundary measurements. Appl. Anal. 27 (1988), no. 1-3, 153-172.}

%%%%%%%%%%%%%%%%%%%%%%%%%%%%%%%%%%%%%%%%%

\bibitem{Alessandrini-1990}
G. Alessandrini,
\newblock{Singular solutions of elliptic equations and the determination of conductivity by boundary measurements. J. Differential Equations 84 (1990), no. 2, 252-272.}

%%%%%%%%%%%%%%%%%%%%%%%%%%%%%%%%%%%%%%%%%%%%%%

%\bibitem{AHMAD2015563}
%B. Ahmad, D. P. Challa, M. Kirane and M. Sini,
%\newblock{The equivalent refraction index for the acoustic scattering by many small obstacles: With error estimates},
%\newblock{Journal of Mathematical Analysis and Applications, volume 424, number 1, pages 563-583, 2015.}

%%%%%%%%%%%%%%%%%%%%%%%%%%%%%%%%%%%%%%%%%%%%%%%


%%%%%%%%%%%%%%%%%%%%%%%%%%%%%%%%%%%%%%

%Cancel
%\bibitem{AhnDyaRae99}
%J.~F.~Ahner, V.~V.~Dyakin, V.~Ya.~Raevskii and R.~Ritter,
%\newblock{On series solutions of the magnetostatic integral equation,}
%\newblock{Zh. Vychisl. Mat. Mat. Fiz, number 4, volume 39, pages 630-637, 1999.}

%%%%%%%%%%%%%%%%%%%%%%%%%%%%%%%%%%%

%Cancel
%\bibitem{PhysRevA.82.055802}
%A. Akyurtlu and A.G. Kussow, 
%\newblock{Relationship between the Kramers-Kronig relations and negative index of refraction},
%\newblock{Phys. Rev. A, volume 82, American Physical Society, 2010.}

%%%%%%%%%%%%%%%%%%%%%%%%%%%%%%%%%%%%%%%%%%

%\bibitem{alsenafi2022foldy}
%A. Alsenafi, A. Ghandriche and M. Sini, 
%\newblock{The Foldy-Lax approximation is valid for nearly resonating frequencies, Zeitschrift für angewandte Mathematik und Physik, Volume 74, 2022.}



%%%%%%%%%%%%%%%%%%%%%%%%%%%%%%%%%%%%%%%%


%\bibitem{amirov2014integral}
%A. Kh. Amirov,
%\newblock{Integral geometry and inverse problems for kinetic equations, De Gruyter, 2014.}

%%%%%%%%%%%%%%%%%%%%%%%%%%%%%%%%%%%%%%%%%%%%

%\bibitem{Habib-book}
%H. Ammari,
%\newblock{An introduction to mathematics of emerging biomedical imaging,}
%\newblock{ Springer-Verlag, Volume 62, 2008.}


%%%%%%%%%%%%%%%%%%%%%%%%%%%%%%%%%%%

\bibitem{Ammari_2019}
H. Ammari, D. P. Challa,  A. P. Choudhury and M. Sini,   
\newblock{The point-interaction approximation for the fields generated by contrasted bubbles at arbitrary fixed frequencies,}
\newblock{Journal of Differential Equations, volume 267, number 4, pages 2104-2191, 2019.}


%%%%%%%%%%%%%%%%%%%%%%%%%%%%%%%%

%Cancelled
%\bibitem{Ammari-Li-Zou}
%Habib Ammari, Bowen Li and Jun Zou, 
%\newblock{Super-resolution in Recovering Embedded Electromagnetic Sources in High Contrast Media,}
%\newblock{SIAM Journal on Imaging Sciences, volume 13, number 3,
%1467-1510, 2020.}

%%%%%%%%%%%%%%%%%%%%%%%%%%%%%%%%%%%
%\Cancelled
%\bibitem{Ammari-Zhang} 
%Habib Ammari and Hai Zhang,  
%\newblock{Super-resolution in high-contrast media,
%Proceedings of the Royal Society A: Mathematical, Physical and Engineering Sciences, volume 471, number 2178, 2015.}
%%%%%%%%%%%%%%%%%%%%%%%%%%%%%%%%%%%%%

%\bibitem{ammari2018super}
%H. Ammari, Y. T. Chow and J. Zou,
%\newblock{Super-resolution in imaging high contrast targets from the perspective of scattering coefficients},
%\newblock{Journal de Math{\'e}matiques Pures et Appliqu{\'e}es, volume 111, pages 191--226, Elsevier, 2018.}

%%%%%%%%%%%%%%%%%%%%%%%%%%%%%%%%%%%%%

%\bibitem{ammari2019subwavelength}
%H. Ammari, A. Dabrowski, B. Fitzpatrick, P. Millien and M. Sini,
%\newblock{Subwavelength resonant dielectric nanoparticles with high refractive indices},
%\newblock{Mathematical Methods in the Applied Sciences, volume 42, number 18, pages 6567--6579, Wiley Online Library, 2019.}


%%%%%%%%%%%%%%%%%%%%%%%%%%%%%%%%%%%%%

%Cancel
%\bibitem{Amrouche-Houda}
%C. Amrouche, N. El Houda Seloula. 
%\newblock{$\mathbb{L}^{p}$-Theory for vector potentials and Sobolev's inequalities for vector fields. Application to the Stokes equations with pressure boundary conditions. 2011. hal-00686230.}


%%%%%%%%%%%%%%%%%%%%%%%%%%%%%%%%%%%%%%

%\Cancelled
%\bibitem{amrouche1998vector}
%Ch\'erif Amrouche, Christine Bernardi, Monique Dauge and Vivette Girault, 
%\newblock{Vector potentials in three-dimensional non-smooth domains, Mathematical Methods in the Applied Sciences, volume 21,
%number 9, 1998.}

%%%%%%%%%%%%%%%%%%%%%%%%%%%%%%%%%%%%
%\bibitem{Anderson}
%J. M. Anderson, D. Khavinson and V. Lomonosov,
%\newblock{Spectral properties of some integral operators arising in potential theory, The Quarterly Journal of Mathematics, volume 43,
%number 4, pages 387--407, Oxford University Press, 1992.}
%%%%%%%%%%%%%%%%%%%%%%%%%%%%%%%%%%%%%

%\bibitem{anikonov1997uniqueness}
%Yu. E. Anikonov and V. G. Romanov, 
%\newblock{On uniqueness of determination of a form of first degree by its integrals along geodesics, Walter de Gruyter, 1997.}

%%%%%%%%%%%%%%%%%%%%%%%%%%%%%%%%%%%%%%%

%\Cancelled
%\bibitem{Ali}
%Ali Bouzekri and Mourad Sini, 
%\newblock{Foldy-Lax approximation of the electromagnetic fields generated by anisotropic inhomogeneities in the mesoscale regime with complements for the perfectly conducting case, arXiv, 2019.}

%%%%%%%%%%%%%%%%%%%%%%%%%%%%%%%%%%%%%%%

%Cancel
%\bibitem{B:2014}
%G.~Bal.
%\newblock Hybrid inverse problems and redundant systems of partial differential
%  equations.
%\newblock In {Inverse problems and applications}, volume 615 of {\em
%  Contemp. Math.}, pages 15--47. Amer. Math. Soc., Providence, RI, 2014.

%%%%%%%%%%%%%%%%%%%%%%%%%%%%%%%%
%Cancel
%\bibitem{B-B-M-T}
%G.~Bal, E.~Bonnetier, F.~Monard, and F.~Triki.
%\newblock Inverse diffusion from knowledge of power densities.
%\newblock {\em Inverse Probl. Imaging}, 7(2):353-375, 2013.

%%%%%%%%%%%%%%%%%%%%%%%%%%%%%%%%%

%\bibitem{B-U:2010}
%G. Bal and G. Uhlmann,
%\newblock Inverse diffusion theory of photoacoustics.
%\newblock{\em Inverse Problems}, 26 (2010). 085010.

%%%%%%%%%%%%%%%%%%%%%%%%%%%%%%%%%%%%%%%

%\bibitem{B-E-K-S:2018}
% A. Beigl, P. Elbau, K. Sadiq, O. Scherzer, 
%\newblock{ Quantitative photoacoustic imaging in the acoustic regime using SPIM.}
%\newblock{ Inverse Problems 34 (2018), no. 5, 054003, 5 pp.}

%%%%%%%%%%%%%%%%%%%%%%%%%%%%%%%%%%

%\bibitem{B-G-S:2016}
%Z. Belhachmi, T. Glatz, O. Scherzer,
%\newblock{ A direct method for photoacoustic tomography with inhomogeneous sound speed.}
%\newblock{ Inverse Problems 32 (2016), no. 4, 045005, 25 pp.}

%%%%%%%%%%%%%%%%%%%%%%%%%%%%%%

\bibitem{Brown-1996}
R. Brown,
\newblock{Global uniqueness in the impedance-imaging problem for less regular conductivities
SIAM J. Math. Anal., 27 (1996), pp. 1049-1056.}

%%%%%%%%%%%%%%%%%%%%%%%%%%%%%%
%\bibitem{caflisch1985wave}
%R. E. Caflisch, M. J. Miksis, G. C. Papanicolaou and L. Ting, 
%\newblock{Wave propagation in bubbly liquids at finite volume fraction, Journal of Fluid Mechanics,
%vol. 160, 1--14, Cambridge University Press, 1985.}

%%%%%%%%%%%%%%%%%%%%%%%%%%%%%%%%%%%%%%%%%%%%%%


\bibitem{brezis}
H. Brezis, 
\newblock{Functional analysis, Sobolev spaces and partial differential equations, 2010, Springer Science \& Business Media.}


%%%%%%%%%%%%%%%%%%%%%%%%%%%%%%%%%%%
%\Cancelled
%\bibitem{BCS}
%A. Buffa, M. Costabel and D. Sheen,
%\newblock{On traces for $\bm{H(curl,\Omega)}$ in Lipschitz domains},
%\newblock{Journal of Mathematical Analysis and Applications,
%Volume 276, Issue 2, 2002, Pages 845-867.}

%%%%%%%%%%%%%%%%%%%%%%%%%%%%%%%%%%%%%%%%%%

\bibitem{brown1996global}
R. M. Brown,
\newblock{Global uniqueness in the impedance-imaging problem for less regular conductivities, SIAM Journal on Mathematical Analysis, volume 27, number 4, pages 1049-1056, 1996.}

%%%%%%%%%%%%%%%%%%%%%%%%%%%%%%%%%%%%%%%%%%
\bibitem{Brown-Torres-2003}
R. Brown, R. Torres
\newblock{Uniqueness in the inverse conductivity problem for conductivities with $3/2$ derivatives in $L^p$
, $p>2n$ 
J. Fourier Anal. Appl., 9 (2003), pp. 563-574.}

%%%%%%%%%%%%%%%%%%%%%%%%%%%%%%%%%%%%%%%%%%
\bibitem{Brown1997UniquenessIT}
R. M. Brown and G. Uhlmann,
\newblock{Uniqueness in the inverse conductivity problem for nonsmooth conductivities in two dimensions, Communications in Partial Differential Equations, volume 22, pages 1009-1027, 1997.}

%%%%%%%%%%%%%%%%%%%%%%%%%%%%%%%%%%%%%%%%%%

\bibitem{bukhgeim2008recovering}
A. L. Bukhgeim,
\newblock{Recovering a potential from Cauchy data in the two-dimensional case, Walter de Gruyter, 2008.}

%%%%%%%%%%%%%%%%%%%%%%%%%%%%%%%%%%%%%%%%%%
% Cancel
%\bibitem{CZ}
%A. P. Calder{\'o}n and A. Zygmund,
%\newblock{Singular integral operators and differential equations},
%\newblock{American Journal of mathematics, volume 79, number = 4, pages 901-921, 1957, JSTOR.}

%%%%%%%%%%%%%%%%%%%%%%%%%%%%%%

%\bibitem{carton2007cours}
%H. Carton,
%\newblock{Cours de calcul diff\'{e}rentiel, Herman, 2007.}

%%%%%%%%%%%%%%%%%%%%%%%%%%%%%%%%%%%%%%%%%%

%\bibitem{carslaw1947conduction}
%H. S. Carslaw and J. C. Jaeger,
%\newblock{Conduction of heat in solids, 1947.}

%%%%%%%%%%%%%%%%%%%%%%%%%%%%%%%%%%%%%%%%%%

%\bibitem{cartan1945theorie}
%H. Cartan,
%\newblock{Th{\'e}orie du potentiel newtonien: {\'e}nergie, capacit{\'e}, suites de potentiels, Bulletin de la Soci{\'e}t{\'e} Math{\'e}matique de France, volume 73, pages 74--106, 1945.}

%%%%%%%%%%%%%%%%%%%%%%%%%%%%%%%%%%%%%%%%%%

%\bibitem{challa2022corrigendum}
%D. P. Challa and M. Sini,
%\newblock{Corrigendum: On the justification of the Foldy-Lax Approximation for the Acoustic Scattering by Small Rigid Bodies of Arbitrary Shapes, Multiscale Modeling \& Simulation, SIAM, volume 20, number 2, 882--892, 2022.}

%%%%%%%%%%%%%%%%%%%%%%%%%%%%%%%%%%%%%%%%%%

\bibitem{challa2020characterization}
D. P. Challa, A. Mantile and M. Sini, 
\newblock{Characterization of the acoustic fields scattered by a cluster of small holes, Asymptotic Analysis, volume 118, number 4, pages 235--268, IOS Press, 2020.}

%%%%%%%%%%%%%%%%%%%%%%%%%%%%%%%%%%%%%%%%%%%

\bibitem{Chanillo}
S. Chanillo,
\newblock{A Problem in Electrical Prospection and a n-Dimensional Borg-Levinson Theorem, American Mathematical Society, number 3, pages 761--767, volume 108, 1990.}

%%%%%%%%%%%%%%%%%%%%%%%%%%%%%%%%%%%%%%%

\bibitem{Caro-Rogers-2016}
P. Caro and K. Rogers,
\newblock{Global uniqueness for the Calderón problem with Lipschitz conductivities. Forum Mathematics, Pi, volume 4, 2016.}

%%%%%%%%%%%%%%%%%%%%%%%%%%%%%%%%%%%%%%%%%%

\bibitem{Caro-Garcia}
P. Caro and A, Garcia,
\newblock{Scattering with critically-singular and $\delta$-shell potentials. Comm. Math. Phys. volume 379, no. 2, pages 543--587, 2020.}

%%%%%%%%%%%%%%%%%%%%%%%%%%%%%%%%%%%%%%




%\Cancelled
%\bibitem{Ma2019MathematicalAN}
%Ma Chupeng, Z. Yongwei and J. Zou, 
%\newblock{Mathematical and numerical analysis of a nonlocal Drude model in nanoplasmonics, ArXiv, 2019.}

%%%%%%%%%%%%%%%%%%%%%%%%%%%%%%%%%%%%

%\bibitem{choy2015effective}
%T. C. Choy,
%\newblock{Effective medium theory: principles and applications, volume 165, Oxford University Press, 2015.}

%%%%%%%%%%%%%%%%%%%%%%%%%%%%%%%%%%%%

\bibitem{colton2019inverse}
D. Colton and R. Kress,
\newblock{Inverse acoustic and electromagnetic scattering theory, 93, 2019, Springer Nature.}


%%%%%%%%%%%%%%%%%%%%%%%%%%%%%%%%%%%%%%%
%Cancel
%\bibitem{Costabel}
%M. Costabel, 
%\newblock{Some historical remarks on the positivity of boundary integral operators. In: Boundary element analysis. vol. 29 of Lect. Notes Appl. Comput. Mech. Springer, Berlin; 2007. p. 1–27.}

%%%%%%%%%%%%%%%%%%%%%%%%%%%
%\Cancelled
%\bibitem{Costabel.D.K}
%M. Costabel, E. Darrigrand and E.H. Koné,
%\newblock{Volume and surface integral equations for electromagnetic scattering by a dielectric body, Journal of Computational and Applied Mathematics, Volume 234, Issue 6,
%2010, Pages 1817-1825.}

%%%%%%%%%%%%%%%%%%%%%%%%
%Cancel
%\bibitem{C-A-B:2007}
%B. T. Cox, S. R. Arridge, and P. C. Beard,
%\newblock{Photoacoustic tomography with a limited-aperture planar sensor and a reverberant cavity,}
%\newblock{\em Inverse Problems,} 23, pp. S95-S112, 2007.


%%%%%%%%%%%%%%%%%%%%%%%%%%%%%%%%%%%%%%%
%Cancel
%\bibitem{cutzach1998existence}
%P. M. Cutzach and C. Hazard, 
%\newblock{Existence, uniqueness and analyticity properties for electromagnetic scattering in a two-layered medium},
%\newblock{Mathematical methods in the applied sciences, volume 21, number 5, pages 433--461, 1998.}

%%%%%%%%%%%%%%%%%%%%%%%%%%%%%%%%%%%%%%

\bibitem{AlexBubbles}
A. Dabrowski, A. Ghandriche and M. Sini,
\newblock{Mathematical analysis of the acoustic imaging modality using bubbles as contrast agents at nearly resonating frequencies, Inverse Problems and Imaging, vol. 15, num. 3, pages 555-597, 2021.}

%%%%%%%%%%%%%%%%%%%%%%%%%%%%%%%%%%%%%
%Cancel
%\bibitem{Dyakin-Rayevskii}
%V.V. Dyakin and V.Ya. Rayevskii,
%\newblock{Investigation of an equation of electrophysics, U.S.S.R Computational Mathematics and Mathematical Physics, Volume 30, Number 1, Pages 213-217, 1990.}

%%%%%%%%%%%%%%%%%%%%%%%%%%%%%%%%%%%%%%%%ù
%Cancel
%\bibitem{Dautry-Lions}
%R. Dautray and J. L. Lions,
%\newblock{Mathematical Analysis and Numerical Methods for Science and Technology Volume 3 Spectral Theory and Applications.}
%\newblock{1st ed. Berlin, Heidelberg: Springer Berlin Heidelberg, 2000.}

%%%%%%%%%%%%%%%%%%%%%%%%%%%%%%%%%%%%%%
%\Cancelled
%\bibitem{Di_Fratta_2016}
%Giovanni Di Fratta,
%\newblock{The Newtonian potential and the demagnetizing factors of the general ellipsoid},
%\newblock{Proceedings of the Royal Society A: Mathematical, Physical and Engineering Sciences, volume 472, number 2190, 2016.}

%%%%%%%%%%%%%%%%%%%%%%%%%%%%%%%%%%
%Cancel
%\bibitem{engheta2006metamaterials}
%N. Engheta and R. W. Ziolkowski,
%\newblock{Metamaterials: physics and engineering explorations, John Wiley \& Sons, 2006.}

%%%%%%%%%%%%%%%%%%%%%%%%%%%%%%%%%%%%%%%%%%%%
%\bibitem{Evans}
%L.C. Evans,
%\newblock{Partial Differential Equations, American Mathematical Society, 2010.}

%%%%%%%%%%%%%%%%%%%%%%%%%%%%%%%%%%%%%%%%%%%%
%Cancel
%\bibitem{FHR}
%D. Finch, M. Haltmeier and Rakesh,
%\newblock{Inversion of spherical means and the wave equation in even dimensions,}
%\newblock{SIAM Journal on Applied Mathematics, V. 68, n. 2, pp. 392-412, 2007.}

%%%%%%%%%%%%%%%%%%%%%%%%%%%%%%%%%%%%%%%%%%%%
%Cancel
%\bibitem{Foias1978}
%C. Foias and R. Temam,
%\newblock{Remarques sur les \'equations de Navier-Stokes stationnaires et les ph\'enom\`enes successifs de bifurcation},
%\newblock{Annali della Scuola Normale Superiore di Pisa-Classe di Scienze, Scuola normale superiore, Number 1, pages 29-63, Volume 5, 1978.}

%%%%%%%%%%%%%%%%%%%%%%%%%%%%%%%%%%%%%%%%%%%
%Cancel
%\bibitem{friedman1980mathematical}
%M. J. Friedman,
%\newblock{Mathematical study of the nonlinear singular integral magnetic field equation. \bf{I}.}
%\newblock{SIAM Journal on Applied Mathematics, volume 39, number 1, pages 14-20, 1980.}

%%%%%%%%%%%%%%%%%%%%%%%%%%%%%%%%%%%%%%%%%%%%%%%%
%Cancel
%\bibitem{friedman1981mathematical}
%M. J. Friedman,
%\newblock{Mathematical study of the nonlinear singular integral magnetic field equation. \bf{III}.}
%\newblock{SIAM Journal on Mathematical Analysis, volume 12, number 4, pages 536-540, 1981.}
  
%%%%%%%%%%%%%%%%%%%%%%%%%%%%%%%%%%%
%Cancel
%\bibitem{10.2307/2008286}
%M. J. Friedman and J. E. Pasciak,
%\newblock{Spectral Properties for the Magnetization Integral Operator, Mathematics of Computation, number = 168, pages  447-453, volume 43, 1984.}

%%%%%%%%%%%%%%%%%%%%%%%%%%%%%%%%%%

%\bibitem{Gelfand1967GeneralizedFV}
%I. M. Gel'fand and G. E. Shilov, 
%\newblock{Generalized Functions, Volume 1: Properties and Operations, American Mathematical Monthly, volume 377, pages 1026, 1967.}

%%%%%%%%%%%%%%%%%%%%%%%%%%%%%%%%%%%%%
%Cancel
%\bibitem{Ahcene-Mourad-IICM}
%A. Ghandriche and M. Sini,
%\newblock{An Introduction To The Mathematics Of The Imaging Modalities Using Small Scaled Contrast Agents, arXiv, physics.optics, 2020.}


%%%%%%%%%%%%%%%%%%%%%%%%%%%%%%%%%%%%%

\bibitem{AhceneMouradMaxwell}
A. Ghandriche and M. Sini,
\newblock{Photo-acoustic inversion using plasmonic contrast agents: The full Maxwell model, Journal of Differential Equations, vol. 341, pages 1-78, 2022.}


%%%%%%%%%%%%%%%%%%%%%%%%%%%%%%%%%%%%%%%%%%

\bibitem{ghandriche2022mathematical}
A. Ghandriche and M. Sini,
\newblock{Mathematical analysis of the photo-acoustic imaging modality using resonating dielectric nano-particles: The 2D TM-model},
\newblock{Journal of Mathematical Analysis and Applications, volume 506, number 2, pages 125658, Elsevier, 2022.}

%%%%%%%%%%%%%%%%%%%%%%%%%%%%%%%%

\bibitem{ghandriche2022simultaneous}
A. Ghandriche and M. Sini,
\newblock{Simultaneous Reconstruction of Optical and Acoustical Properties in Photo-Acoustic Imaging using plasmonics, arXiv, 2022.}



%%%%%%%%%%%%%%%%%%%%%%%%%%%%%%%%%%

%\bibitem{gilbarg2001elliptic}
%D. Gilbarg and N. S. Trudinger,
%\newblock{Elliptic partial differential equations of second order, 2001, Springer.}

%%%%%%%%%%%%%%%%%%%%%%%%%%%%%%%%%%%%%%%%%%%%%%%%

%\bibitem{Hahner}
%P. Hähner,
%\newblock{A periodic Faddeev-Type solution operator, Journal of Differential Equations, Volume 128, Pages 300-308,
%1996.}
%%%%%%%%%%%%%%%%%%%%%%%%%%%%%%%%%%

%\bibitem{henrot2006extremum}
%A. Henrot,
%\newblock{Extremum problems for eigenvalues of elliptic operators, Springer Science \& Business Media, 2006.}


%%%%%%%%%%%%%%%%%%%%%%%%%%%%%%%%%%%%%%%%%


%\Cancelled
%\bibitem{girault2012finite}
%Vivette Girault and Raviart Pierre-Arnaud,   
%\newblock{Finite element methods for Navier-Stokes equations: theory and algorithms, volume 5, Springer Science \& Business Media, 2012.}

%%%%%%%%%%%%%%%%%%%%%%%%%%%%%%%%%%%%%%%
%\Cancelled
%\bibitem{li2012time}
%Y. Huang and J. Li, 
%\newblock{Time-Domain Finite Element Methods for Maxwell's Equations in Metamaterials, 2012, Springer Berlin Heidelberg.}
%%%%%%%%%%%%%%%%%%%%%%%%%%%%%%%%%%%%%%%%
\bibitem{Haberman-Tataru}
B. Haberman and D. Tataru,
\newblock{Uniqueness in Calderón problem with Lipschitz conductivities
Duke Math. J., 162 (2013), pp. 497-516.}

\bibitem{Haberman-2015}
B. Haberman,
\newblock{Uniqueness in Calderón's problem for conductivities with unbounded gradient
Comm. Math. Phys., 340 (2015), pp. 639-659.}
%%%%%%%%%%%%%%%%%%%%%%%%%%%%%%%%%%%%%
%\Cancelled
%\bibitem{Sini-India}
%Manas Kar and Mourad Sini,
%\newblock{Reconstruction of interfaces using CGO solutions for the Maxwell equations, Journal of Inverse and Ill-posed Problems, vol. 22, no. 2, 2014, pp. 169-208.} 

%%%%%%%%%%%%%%%%%%%%%%%%%%%%%%%%%%%%%%
%\Cancel
%\bibitem{lgeleyen2010}
%İ, Gölgeleyen,
%\newblock{An integral geometry problem along geodesics and a computational approach},
%\newblock{number 2, pages 91-112, volume 18, Analele Ştiinţifice ale Universităţii “Ovidius" Constanţa. Seria: Matematică, 2010.}

%%%%%%%%%%%%%%%%%%%%%%%%%%%%%%%%%%%%


\bibitem{VI}
V. Isakov,
\newblock{Completeness of products of solutions and some inverse problems for PDE, Journal of Differential Equations, Volume 92, Pages 305-316, 1991.}

\bibitem{Isakov-book}
V. Isakov,
\newblock{Inverse Problems for Partial Differential Equations
(second edition), Springer, New York (2006).}




%%%%%%%%%%%%%%%%%%%%%%%%%%%%%%%%%

%\bibitem{kalmenov2011boundary}
%T. Kalmenov and D. Suragan,
%\newblock{A boundary condition and spectral problems for the Newton potential, Modern aspects of the theory of partial differential equations, pages 187--210, Springer, 2011.}

%%%%%%%%%%%%%%%%%%%%%%%%%%%%%%%%%%%%

%\bibitem{kellogg}
%O. D. Kellogg,
%\newblock{Foundations of potential theory,
%  volume 31, Courier Corporation, 1953.}

%%%%%%%%%%%%%%%%%%%%%%%%%%%%%%%%%%%%

\bibitem{KV}
R. Kohn and M. Vogelius, 
\newblock{Determining conductivity by boundary measurements, Communications on Pure and Applied Mathematics, VL 37, 1984.}

%%%%%%%%%%%%%%%%%%%%%%%%%%%%%%%%%%%%%%%%%%

%\bibitem{Kirsch-Scherzer}
%A. Kirsch and O. Scherzer, 
%\newblock{Simultaneous reconstructions of absorption density and wave speed with photoacoustic measurements,} \newblock{SIAM J. Appl. Math. 72 (2012), no. 5, 1508-1523.}


%%%%%%%%%%%%%%%%%%%%%%%%%%%%%ù%%%%
%\Cancelled
%\bibitem{Khavinson_2014}
%D. Khavinson and E. Lundberg,
%\newblock{A Tale of Ellipsoids in Potential Theory, Notices of the American Mathematical Society, volume 61, number 02, 2014.}

%%%%%%%%%%%%%%%%%%%%%%%%%%%%%%%%%%%%%%%%%%%%%%
%Cancel
%\bibitem{Kirsch}
%A. Kirsch,
%\newblock{An Integral Equation for Maxwell's Equations in a Layered Medium with an Application to the Factorization Method},
%\newblock{Journal of Integral Equations and Applications, volume 9, number 3, pages 333 - 358,
%2007.}

%%%%%%%%%%%%%%%%%%%%%%%%%%%%%%%%%%%%%%%%%%

%\bibitem{Kroger}
%P. Kröger,
%\newblock{Upper bounds for the Neumann eigenvalues on a bounded domain in euclidean space, Journal of Functional Analysis,
%Volume 106, Issue 2, 353-357, 1992.}



%%%%%%%%%%%%%%%%%%%%%%%%%%%%%%%%%%%%%%%%%%


%\bibitem{K-K:2010} 
%P. Kuchment and L. Kunyansky,
%\newblock{Mathematics of thermoacoustic and photoacoustic tomography,} 
%\newblock{in Handbook of Mathematical Methods in Imaging,
%O. Scherzer, ed., Springer-Verlag,} pp. 817-866, 2010.

%%%%%%%%%%%%%%%%%%%%%%%%%%%%%%%%%%%%%%%%%%%

%\bibitem{KuchmentKunyansky} 
%P. Kuchment and L. Kunyansky,  
%\newblock{Mathematics of thermoacoustic tomography, \,
%European Journal of Applied Mathematics,}
%\newblock{Volume 19, Number 02, 2008.}


%%%%%%%%%%%%%%%%%%%%%%%%%%%%%%%%%%%%%%
%\Cancelled
%\bibitem{Mitrea}
%D. Mitrea, M. Mitrea and J. Pipher, 
%\newblock{Vector potential theory on nonsmooth domains in $\mathbb{R}^{3}$ and applications to electromagnetic scattering. The Journal of Fourier Analysis and Applications 3, 131–192 (1997).}

%%%%%%%%%%%%%%%%%%%%%%%%%%%%%%%%%%%%%%

%\bibitem{lavrent1967certain}
%M. M. Lavrent'ev and Yu. E. Anikonov, 
%\newblock{A certain class of problems in integral geometry},
%\newblock{Doklady Akademii Nauk, volume 176, number 5, pages 1002--1003, Russian Academy of Sciences, 1967.}

%%%%%%%%%%%%%%%%%%%%%%%%%%%%%%%%%%%%%%

%\bibitem{LANDAU}
%L.J. Landau, 
%\newblock{Ratios of Bessel Functions and Roots of $\alpha \, \begin{LARGE}
%\textbf{J}_{\nu}
%\end{LARGE}\left( x \right) + x \, \begin{LARGE}
%\textbf{J}^{\prime}_{\nu}
%\end{LARGE}\left( x \right) = 0$},
%\newblock{Journal of Mathematical Analysis and Applications, volume  240, number 1, pages 174-204, 1999.}

%%%%%%%%%%%%%%%%%%%%%%%%%%%%%%%%%%%%%%

%\bibitem{lax2002functional}
%P. D. Lax,
%\newblock{Functional analysis, volume 55, John Wiley \& Sons, 2002.}
%%%%%%%%%%%%%%%%%%%%%%%%%%%%%%%%%%%%%%%

\bibitem{MPS-2018}
A. Mantile, A. Posilicano and M. Sini,
\newblock{Uniqueness in inverse acoustic scattering with unbounded gradient across Lipschitz surfaces. J. Differential Equations 265 (2018), no. 9, 4101-4132.}

%%%%%%%%%%%%%%%%%%%%%%%%%%%%%%%%%%%%%%

%\bibitem{McLean}
%W. McLean, 
%\newblock{Strongly elliptic systems and boundary integral equations, Cambridge university press, 2000.}
%%%%%%%%%%%%%%%%%%%%%%%%%%%%%%%%%



%\bibitem{moskow2008convergence}
%S. Moskow amd J. C. Schotland,
%\newblock{Convergence and stability of the inverse scattering series for diffuse waves, Inverse Problems, volume 24, number 6, IOP Publishing, 2008}

%%%%%%%%%%%%%%%%%%%%%%%%%%%%%%%%%%%%%%%%%%%%%

\bibitem{nachman1996global}
A. I. Nachman,
\newblock{Global uniqueness for a two-dimensional inverse boundary value problem,
Annals of Mathematics, pages 71--96, JSTOR, 1996.}

%%%%%%%%%%%%%%%%%%%%%%%%%%%%%%%%%%%%%%%%%%

\bibitem{nachman1988}
A. I. Nachman,
\newblock{Reconstructions From Boundary Measurements, Annals of Mathematics, number 3, volume 128, pages 531--576, 1988.}
 




%\bibitem{N-S:2014}
%W. Naetar and O. Scherzer,
%\newblock{Quantitative photoacoustic tomography with piecewise constant material parameters,}
%\newblock{\em SIAM J.
%Imag. Sci.,} V. 7, pp. 1755-1774, 2014.

%%%%%%%%%%%%%%%%%%%%%%%%%%%%%%
%Cancel
%\bibitem{Natterer}
%F. Natterer,
%\newblock{The Mathematics of Computerized Tomography,}
%\newblock{Society for Industrial and Applied Mathematics, \; 2001}.


%%%%%%%%%%%%%%%%%%%%%%%%%%%%%%%%%%%%
%Cancel
%\bibitem{Neil}
%P. V. O'Neil,
%\newblock{Beginning partial differential equations, 2011,  John Wiley \& Sons.}

%%%%%%%%%%%%%%%%%%%%%%%%%%%%%%%%%%%%%%%%%%

\bibitem{paivarinta2003complex}
L. P{\"a}iv{\"a}rinta, A. Panchenko and G. Uhlmann, 
\newblock{Complex geometrical optics solutions for Lipschitz conductivities, Revista Matematica Iberoamericana, vol. 19, num. 1, pages 57--72, 2003.}


%%%%%%%%%%%%%%%%%%%%%%%%%%%%%%%%%%%%
%\Cancelled
%\bibitem{powell2011calculating}
%Philip D. Powell,
%\newblock{Calculating Determinants of Block Matrices, arXiv, 2011.}  

%%%%%%%%%%%%%%%%%%%%%%%%%%%%%%%
%\bibitem{P-P-B:2015}
%A. Prost, F. Poisson and E. Bossy.
%\newblock{ Photoacoustic generation by gold nanosphere: From linear to nonlinear thermoelastic in the 
%long-pulse illumination regime}
%\newblock arkiv:1501.04871v4

%%%%%%%%%%%%%%%%%%%%%%%%%%%%%%%%%%%%%%%%%
%\Cancelled

%\bibitem{Ritter}
%Stefan Ritter,
%\newblock{On the computation of Lam\`e functions, of eigenvalues and eigenfunctions of some potential operators,}
%\newblock{Z. f. angew. Math. u. Mech. 78, pages 66-72, 1998.}

%%%%%%%%%%%%%%%%%%%%%%%%%%%%%%%%%%%%%%%%%ù
%Cancel
%\bibitem{Stein}
%E. M. Stein,
%\newblock{Singular integrals and differentiability properties of functions, Princeton Univ, 1970.}


%%%%%%%%%%%%%%%%%%%%%%%%%%%%%%%%%%%%%%%%%ù


%Cancel
%\bibitem{Raevskii1994}
%V. Ya. Raevskii,
%\newblock{Some properties of the operators of potential theory and their application to the investigation of the basic equation of electrostatics and magnetostatics,} 
%\newblock{Theoretical and Mathematical Physics, Volume 100, Number 3, pages 1040-1045, 1994.}
%%%%%%%%%%%%%%%%%%%%%%%%%%%%%%%%%%%%%%
\bibitem{Ramm-book}
A. G. Ramm,
\newblock{Inverse Problems. Mathematical and Analytical Techniques with Applications to Engineering Springer, New York (2005).}


%%%%%%%%%%%%%%%%%%%%%%%%%%%%%%%%%%%%%%

%\bibitem{romanovBook}
%V. G. Romanov,
%\newblock{Inverse problems of mathematical physics, De Gruyter, 2018.}

%%%%%%%%%%%%%%%%%%%%%%%%%%%%%%%%%%%%%%%
%Cancel
%\bibitem{romanov1978integral}
%V. G. Romanov, 
%\newblock{Integral geometry on geodesics of an isotropic Riemannian metric,}
%\newblock{Doklady Akademii Nauk, vol. 241, number 2, pages 290--293, Russian Academy of Sciences, 1978.}

%%%%%%%%%%%%%%%%%%%%%%%%%%%%%%%%%%%%%%

%\bibitem{romanov2009smoothness}
%V. G. Romanov,
%\newblock{On smoothness of a fundamental solution to a second order hyperbolic equation},
%\newblock{Siberian Mathematical Journal, Springer,
%volume 50, number 4, pages 700--705, 2009.}

%%%%%%%%%%%%%%%%%%%%%%%%%%%%%%%%%%%%%%%

%\bibitem{romanov2013integral}
%V. G. Romanov, 
%\newblock{Integral geometry and inverse problems for hyperbolic equations,
%  volume 26, Springer Science \& Business Media, 2013.}

%%%%%%%%%%%%%%%%%%%%%%%%%%%%%%%%%%%%%%

%\bibitem{ruzhansky2016isoperimetric}
%M. Ruzhansky and D. Suragan,
%\newblock{Isoperimetric inequalities for the logarithmic potential operator, Journal of Mathematical Analysis and Applications, volume 434, number 2, pages 1676--1689, Elsevier, 2016.}

%%%%%%%%%%%%%%%%%%%%%%%%%%%%%%%%%%%%%%%%

\bibitem{salo2008calderon}
M. Salo,
\newblock{Calder{\'o}n problem, Lecture Notes, Citeseer, 2008.}


%%%%%%%%%%%%%%%%%%%%%%%%%%%%%%%%%%%%%%

%\bibitem{salo}
%M. Salo,
%\newblock{Stability for solutions of wave equations with $C^{1, 1}$ coefficients, arXiv preprint math/0611457, 2006.}

%%%%%%%%%%%%%%%%%%%%%%%%%%%%%%%%%%%%%%%%

%\bibitem{S:2010}
%O. Scherzer,
%\newblock{Handbook of Mathematical Methods in Imaging,}
%\newblock {\em Springer-Verlag}, 2010.
%%%%%%%%%%%%%%%%%%%%%%%%%%%%%%%%%%%%%%%%%%
\bibitem{SSW-2023}
S. Senapati, M. Sini and H. Wang,
\newblock{Recovering both the wave speed and the source function in a time-domain wave equation by injecting high contrast bubbles, arXiv:2304.08869, 2023.}


%%%%%%%%%%%%%%%%%%%%%%%%%%%%%%%%%%%%%%%%%%

%\bibitem{smith}
%H. F. Smith,
%\newblock{A parametrix construction for wave equations with $C^{1,1}$ coefficients, Annales de l'institut Fourier, vol. 48, num. 3, 797--835, 1998.}
%%%%%%%%%%%%%%%%%%%%%%%%%%%%%%%%%%%%%%%%%
\bibitem{SW-2022}
M. Sini and H. Wang,
\newblock{The inverse source problem for the wave equation revisited: a new approach. SIAM J. Math. Anal. 54 (2022), no. 5, 5160-5181.}

%%%%%%%%%%%%%%%%%%%%%%%%%%%%%%%%%%%%%%%%%%
%\bibitem{S-U:2009}
%P. Stefanov and G. Uhlmann,
%\newblock{Thermoacoustic tomography with variable sound speed,} 
%\newblock {\em Inverse Problems}, 25, 075011, 2009.

%%%%%%%%%%%%%%%%%%%%%%%%%%%%%%%%%%%%%%%%%

\bibitem{JSGU}
J. Sylvester and G. Uhlmann,
\newblock{A Global Uniqueness Theorem for an Inverse Boundary Value Problem, Annals of Mathematics, number 1, pages 153--169, Annals of Mathematics, volume 125, JSTOR, 1987.}

\bibitem{Uhlmann-Review}
G. Uhlmann,
\newblock{Inverse problems: seeing the unseen
Bull. Math. Sci., 4 (2014), pp. 209-279.}


%%%%%%%%%%%%%%%%%%%%%%%%%%%%%%%%%%%%

%\bibitem{Triki-Vauthrin:2017}
%F. Triki and M. Vauthrin, 
%\newblock{Mathematical modelization of the Photoacoustic effect generated by the heating of metallic nanoparticles, arXiv, 2017.}

%%%%%%%%%%%%%%%%%%%%%%%%%%%%%%%%%%%%%%%

%\bibitem{Temme}
%N. M. Temme,
%\newblock{Special functions: An introduction to the classical functions of mathematical physics, John Wiley \& Sons, 1996.}


%%%%%%%%%%%%%%%%%%%%%%%%%%%%%%%%%%%%%%%


%\bibitem{watson}
%G. N. Watson,
%\newblock{A treatise on the theory of Bessel functions, volume 3, The University Press,1922.}


%%%%%%%%%%%%%%%%%%%%%%%%%%%%%%%%%%%%%%%%%

%\bibitem{wlcx}
%C. H. Wilcox,
%\newblock{Multiparameter acoustic imaging in the born approximation, Mathematical methods in the applied sciences, vol. 5, num. 1, 276--291, Wiley Online Library, 1983.}

%%%%%%%%%%%%%%%%%%%%%%%%%%%%%%%%%%%%%%

%\bibitem{wilcox1977spectral}
%C. H. Wilcox,
%\newblock{Spectral and asymptotic analysis of acoustic wave propagation},
%\newblock{ booktitle: Boundary Value Problems for Linear Evolution Partial Differential Equations,
%385--473, Springer, 1977.}


%%%%%%%%%%%%%%%%%%%%%%%%%%%%%%%%%%%%%%%%%%

\bibitem{Valdivia}
N. Valdivia,
\newblock{Uniqueness in inverse obstacle scattering with conductive boundary conditions, Applicable Analysis, volume  83, number 8, pages 825-851, 2004.}

%%%%%%%%%%%%%%%%%%%%%%%%%%%%%%%%%%%%

 



\end{thebibliography}








\end{document}