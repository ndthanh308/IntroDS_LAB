% https://en.wikipedia.org/wiki/Wikipedia:Statistics
% https://en.wikipedia.org/wiki/Special:MostRevisions
% https://en.wikipedia.org/wiki/Wikipedia:Edit_warring
% https://en.wikipedia.org/wiki/Wikipedia:Editing_policy
% https://en.wikipedia.org/wiki/Help:Page_history

Since the dawn of Wikipedia, users have made $1.7 \times 10^9$ edits to its pages. Its most revised entry contains 56,713 revisions, all documented in the page history.\footnote{According to \href{https://stats.wikimedia.org/\#/all-wikipedia-projects/contributing/user-edits/normal|table|2001-01-01~2023-05-01|(page\_type)~content*non-content|monthly}{Wikimedia Statistics} and \href{https://en.wikipedia.org/wiki/Special:MostRevisions}{wiki Special}.} In such an active community, conflicts inevitably occur. Editors can begin competing to override each other's contributions, causing dysfunctional \textit{edit warrings}.\footnote{\href{https://en.wikipedia.org/wiki/Wikipedia:Edit\_warring}{https://en.wikipedia.org/wiki/Wikipedia:Edit\_warring}} To help regulate the environment, an editing policy is in force, aiming at making edits constructive and improving quality.\footnote{\href{https://en.wikipedia.org/wiki/Wikipedia:Editing_policy}{https://en.wikipedia.org/wiki/Wikipedia:Editing\_policy}}

Edits, revisions and policies are key concepts in incremental processing, where a model must rely on partial input to generate partial output. Incrementality can help optimise reactivity, naturalness, quality and realism in interactive settings \citep{iu-restart}. This is particularly relevant in dialogue models whose NLU components need to operate on incoming input, \textit{e.g.}~while performing NER, slot filling or disfluency detection, or doing simultaneous translation. 

Local ambiguities in the linguistic input and transient mistakes by the model can result in wrong partial hypotheses, so that the ability to \textit{revise}, by \textit{editing} previous outputs, is desirable \citep{tapir}. Beyond monitoring the occurrence of edits, it is also beneficial to have a \textit{policy} regulating when and which revisions should be made, reducing the occurrence of undesirable edits. Existing literature using consolidated incremental evaluation metrics falls short in capturing relevant nuances of the incremental behaviour in terms of revisions.  

% Figure environment removed

In this work, we propose an evaluation methodology for revision policies in incremental sequence labelling. A constructed example is shown in Figure~\ref{fig:contructed-example}, with revisions indicated in the right column. Specifically, our contributions to address the identified evaluation gap are: A formalisation of revision policy in incremental sequence labelling, characterising types of edits and of revisions (\S \ref{sec:formalisation}-\ref{sec:characterisation}); a proposal of specialised evaluation metrics for revision policies, accompanied by a discussion on the desired behaviour of incremental processors (\S\ref{sec:metrics}-\ref{sec:ideal}); and a demonstration of our methodology with an analysis of the revision policy in three sequence labelling Transformer-based models (\S \ref{sec:analysis}).\footnote{Our implementation is available at \url{https://github.com/briemadu/inc-eval-revisions} with accompanying documentation on how to run the evaluation for other models.}

