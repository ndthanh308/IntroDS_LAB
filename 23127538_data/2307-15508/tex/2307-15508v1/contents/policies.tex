To perform good revisions, a model must decide \textit{when} to recompute or revise. For that decision, both a \textit{revision policy} and a \textit{recomputation policy} can be generally defined as:

\begin{equation}
    \pi: \text{IC} \rightarrow [0,1] \hspace{0.8cm} \pi(\text{IC}_t) = \Pr(r | \text{IC}_t )
\end{equation}


It gives the probability of performing a revision or recomputation $r$, respectively, given the state of the incremental chart at time $t$.\footnote{It is also possible to make the policy dependent only in a portion of the $IC$, as done \textit{e.g.}~by \citet{tapir}.} When $\Pr(r | \text{IC}_t ) > \tau$, where $\tau$ is a threshold hyperparameter, a revision/recomputation is performed. If the revisions are not a mere consequence of full recomputations, the model must then also decide \textit{what} and \textit{how} to edit. 
