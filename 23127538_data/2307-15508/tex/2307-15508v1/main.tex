% This must be in the first 5 lines to tell arXiv to use pdfLaTeX, which is strongly recommended.
\pdfoutput=1
% In particular, the hyperref package requires pdfLaTeX in order to break URLs across lines.

\documentclass[11pt]{article}

% Remove the "review" option to generate the final version.
%\usepackage[review]{acl}
\usepackage{acl}

% Standard package includes
\usepackage{times}
\usepackage{latexsym}

% For proper rendering and hyphenation of words containing Latin characters (including in bib files)
\usepackage[T1]{fontenc}
% For Vietnamese characters
% \usepackage[T5]{fontenc}
% See https://www.latex-project.org/help/documentation/encguide.pdf for other character sets

% This assumes your files are encoded as UTF8
\usepackage[utf8]{inputenc}

% This is not strictly necessary, and may be commented out.
% However, it will improve the layout of the manuscript,
% and will typically save some space.
\usepackage{microtype}

% This is also not strictly necessary, and may be commented out.
% However, it will improve the aesthetics of text in
% the typewriter font.
\usepackage{inconsolata}


% If the title and author information does not fit in the area allocated, uncomment the following
%
%\setlength\titlebox{<dim>}
%
% and set <dim> to something 5cm or larger.


% Added packages
\usepackage{graphicx}
\usepackage{enumerate}
\usepackage[colorinlistoftodos]{todonotes}
\usepackage{booktabs}
\usepackage{multirow}
\usepackage{hyperref}
\usepackage{caption}
\usepackage{subcaption}
\usepackage{amssymb}
\usepackage{array}
\usepackage{nicematrix}
\usepackage{comment}
\usepackage{nicematrix}
%\usepackage{colortbl}


\title{The Road to Quality is Paved with Good Revisions: A Detailed Evaluation Methodology for Revision Policies in Incremental Sequence Labelling}



\author{Brielen Madureira$^{\mathbf{1}}$ \hspace{10mm}  Patrick Kahardipraja$^{\mathbf{1}}$  \hspace{10mm}  David Schlangen$^{\mathbf{1, 2}}$ \\
$^{\mathbf{1}}$Computational Linguistics, Department of Linguistics, University of Potsdam, Germany \\
$^{\mathbf{2}}$German Research Center for Artificial Intelligence (DFKI), Berlin, Germany \\
  \texttt{\{madureiralasota,kahardipraja,david.schlangen\}@uni-potsdam.de}}

\begin{document}
\maketitle
\begin{abstract}
Incremental dialogue model components produce a sequence of output prefixes based on incoming input. Mistakes can occur due to local ambiguities or to wrong hypotheses, making the ability to revise past outputs a desirable property that can be governed by a policy. In this work, we formalise and characterise edits and revisions in incremental sequence labelling and propose metrics to evaluate revision policies. We then apply our methodology to profile the incremental behaviour of three Transformer-based encoders in various tasks, paving the road for better revision policies.
\end{abstract}

\section{Introduction}
\label{sec:intro}
% Figure environment removed

\section{Introduction}
Automatic 3D reconstruction of clothed humans using image inputs has gained increasing significance due to its potential applications in a wide array of AR/VR scenarios. High-fidelity reconstructions typically depend on sophisticated capture systems, which are developed with dense camera arrays~\cite{collet2015high,joo2015panoptic,joo2018total}, programmable light-stages~\cite{Vlasic2009, guo2019relightables}, and depth sensors~\cite{newcombe2011kinectfusion,DoubleFusion,BodyFusion,dou2016fusion4d,newcombe2015dynamicfusion}. However, stringent capture environments equipped with complex hardware pose significant challenges for consumer-level applications.


In this context, considerable research effort has been dedicated to developing methods that allow for more flexible capture configurations, such as utilizing a few RGB inputs. Among these works, learning implicit functions \cite{iccv2020PIFu, saito2020pifuhd, hong2021stereopifu} has proven effective in achieving highly detailed reconstructions by integrating the advancements of deep neural networks. These methods employ large multi-layer perceptrons (MLPs) to predict the occupancy probability or truncated signed distance function (TSDF) value of every queried 3D point based on its associated local feature, which is extracted from images. They can recover a continuous surface at arbitrary resolutions without topology restrictions.


However, in typical MLP-based implicit networks, the occupancy or TSDF value at each location is solved independently with planar image features, rendering them less capable of addressing challenging cases such as occlusions. Consequently, these methods suffer from generalization and robustness issues, particularly when tackling strong occlusions caused by large motion or multiple interacting humans. 
Some follow-up studies  \cite{zheng2021deepmulticap,zheng2021pamir,huang2020arch} utilize an extra geometric model, SMPL~\cite{Loper2015}, to improve robustness by introducing strong shape priors. 
Their success typically relies on the assumption of geometrical similarity \cite{huang2020arch} between the shape prior and target reconstruction, making them intractable for handling complex cases with loose clothes and sensitive to errors in SMPL model fitting.



%\ping{this paragraph sounds like `TSDF is better than MLP/SMPL, and we use TSDF to solve the problem'. But in Sec 3, we are telling a different story, saying `MLP needs a 3D convolutional encoder'. We need to make these two sections consistent.}\sicong{I think in this paragraph we claim that the TSDF}


%We opt for Trucated Signed Distance Funtion (TSDF) volumetric representations as they are naturally suitable for convolution operations, which have shown remarkable performance for learning hierarchical features on 2D visual perception tasks \cite{SunXLW19}. 
%Meanwhile, TSDF also describes the gradual geometry change around shape surface, which is not reflected by occupancy volume. 

We instead revisit the 3D volumetric representation and resort to 3D convolutional neural networks (CNNs) for feature learning, due to their impressive performance in feature learning and the ability to incorporate spatial context. However, volumetric methods and 3D convolution involve discretization, which might raise concerns regarding whether a discretized volume can preserve subtle geometric details as continuous representations learned in implicit functions. We investigate the relationship between volume resolution and quantization error on synthetic data by converting target mesh objects to TSDF volumes, as shown in Figure~\ref{fig:quantization_error}. We observe that the quantization errors are significantly reduced by increasing volume resolution and become nearly negligible when reaching a relatively high resolution (e.g., 512 or higher). In other words, achieving fine-detailed reconstruction is not supposed to be restricted by the use of volume representations as long as a proper volume resolution is utilized. Therefore, we present a method with high-resolution feature volumes, e.g., 256 and 512, while traditional volumetric methods \cite{varol18_bodynet,gilbert2018volumetric} are often limited to much lower resolutions, such as 32 or 128.



On the other hand, an increase in volume resolution may lead to a cubic growth of memory overhead \cite{8100085}. Reducing memory costs while guaranteeing the granularity of volumetric representations is necessary for pursuing high-quality reconstruction. Thus, we adopt a coarse-to-fine approach and cull away irrelevant voxels to build a sparse high-resolution feature volume. At the coarse level, the network computes an initial TSDF by applying a U-Net with sparse 3D CNN \cite{3DSemanticSegmentationWithSubmanifoldSparseConvNet} on the sparse feature volume, which is carved by a visual hull. Through our experiments, it turns out that more than 95\% of the volume grids are discarded by the visual hull culling, making the sparse 3D CNN efficient. At the fine level, the network focuses on a narrow band near the zero-level set of the initial TSDF and discretizes the narrow band with smaller voxels. By employing this narrow-band culling, we further shrink the sampling space, resulting in a relatively small range of grid numbers (usually 300K--500K in our experiments) even with a high volume resolution of 512. The remaining voxels in the narrow band are associated with features that fuse high-frequency information from the computed normal maps upon the low-frequency shape from the coarse level to compute the TSDF at high resolution. The final mesh is then extracted from the TSDF using the Marching-Cube algorithm ~\cite{Lorensen87marchingcubes}.
% Different from the u-net sturcture to preserve global topology context, we then apply a shallow 3dcnn to compute the final TSDF $D_{final}$ which contain more local geometry detail.




% \ping{this paragraph can be expanded. It is an important contribution and often ignored by other works. stress on the novel idea of regressing blending weights instead of colors}

In addition to geometry, high-quality mesh texture is also a crucial factor contributing to visual appearance. Directly computing a color field in 3D space, as in \cite{iccv2020PIFu}, struggles to capture high-frequency texture details, while the neural radiance field (NeRF) \cite{yu2020pixelnerf} or the DoubleField~\cite{shao2022doublefield} require expensive per-instance optimization and are often unstable for sparse input images. In contrast, we adopt an image-based rendering approach to compute a texture atlas map, which is efficient and widely supported in existing computer graphics tools. 
Specifically, we compute a blending weight at each 3D point on the mesh surface to determine its color as a weighted average of the colors at its image projections. The blending weights can be computed at a relatively coarse resolution, e.g., 512 volume resolution in our case, and leave texture details to the high-resolution images, such as 1K or 2K. Unlike previous methods that generate blurry texturing results under sparse input, our method generalizes well on both synthetic and real data with just a few input views. 
Figure~\ref{fig:teaser} shows two examples reconstructed by our method. Despite the challenging garment, pose, and occlusion, our method recovers faithful shape, normal, and texture on the right.

%with a wide variety of poses and clothing styles, and it is also adaptive to handle input image with arbitrary resolutions.
%\sicong{For this concern we claim that when the resolution of dicretized volume meets certain threshold (which is 256 in our experiment), the quantization error can be neglected.} 



In summary, the main contributions of this paper are as follows:
\begin{itemize}
\vspace{-0.1in}
  \item 
  We revisit the 3D volumetric representation and demonstrate that it can support clothed human reconstruction with equal or even better performance compared to implicit representation. 
  \item 
  We develop a memory and computation-efficient method for high-resolution volumetric reconstruction using sophisticated sparse 3D CNN, coarse-to-fine estimation, and voxel culling by visual hull and narrow bands. 
  \item 
  We introduce a novel method to compute a texture atlas map, which captures rich appearance details from high-resolution input images.
  \item 
  We achieve impressive results on standard benchmark datasets Twindom and MultiHuman, significantly reducing the point-2-surface (P2S) precision to approximately 0.2cm from just six input views, with more than $50\%$ error reduction compared to the state-of-the-art methods, including DoubleField~\cite{shao2022doublefield} and PIFuHD~\cite{saito2020pifuhd}.
\end{itemize}

\section{Motivation}
\label{sec:motivation}
\section{Motivation}
\label{sec:motivation}

IGNORE THIS FILE, WILL DO IN INTRO


\section{Related Literature}
\label{sec:litreview}
\begin{table*}[ht!]
\centering
\footnotesize

\begin{NiceTabular}{m{7cm} m{4cm} m{3.5cm}}
\CodeBefore
\rowcolors{2}{white}{gray!25}
\Body
    \toprule
        latency, quality, stability & simultaneous translation &  \citet{arivazhagan-etal-2020-translation} \citet{ma-etal-2020-simuleval}
        \\ \midrule    
        quality, responsiveness, robustness, stability & speech recognition and \newline  diarization & \citet{baumann-etal-2009-assessing} \newline \citet{addlesee-etal-2020-comprehensive} \\
        \midrule
        similarity, timing, diachronic & general & \citet{baumann-incremental} \\ 
        \midrule
        fluency, latency, quality, recovery capabilities, timing & simultaneous interpreting \newline (MT and speech synthesis) & \citet{baumann-etal-2014-towards} \\ 
        \midrule
        decisiveness, monotonicity, stability, timeliness & POS tagging & \citet{beuck-etal-2011-decision} \\
        \midrule
        amount of predicted information, connectedness, delay, inclusiveness, monotonicity, quality & parsing & \citet{beuck-inc-parsing, beuck2013predictive} \citet{kohn-menzel-2014-incremental} \\
        \midrule
        cognitive aspects, efficiency &  
        neural coreference resolution & \citet{grenander-etal-2022-sentence} \\
        \midrule
        jumpiness, position & reference resolution & \citet{schlangen-etal-2009-incremental} \\
        \midrule 
        accuracy, integration, representational similarity & sequence-to-sequence & \citet{ulmer-etal-2019-assessing}  \\
        \midrule
        consistency, diminishing returns, interruptibility, monotonicity, preemptability, (recognisable) quality & anytime algorithms & \citet{zilberstein1996using} \\
    \bottomrule

    \end{NiceTabular}%
    \caption{Overview of relevant properties for incremental evaluation in various tasks.}
    \label{table:properties}
\end{table*}



Revisability is in the nature of incremental processing: Hypothesis revision is a necessary operation to correct mistakes and build up a high-quality final output \citep{iu-restart}. Still, there is a trade-off between requiring that later modules handle a processor's revisions and buying stability by reducing some of its incrementality, which makes the concept of \textit{hypothesis stability} very relevant \citep{baumann-etal-2009-assessing}. \citet{beuck-etal-2011-decision} argue that performing revisions should not take as long as the initial processing, so as to retain the advantages of incremental processing. They propose two strategies: Allowing revisions only within a fixed window or limiting their types. Empirically determining how often a model changes the output is an aspect of their analysis we also rely on.

 The restart-incremental paradigm was investigated for Transformer-based sequence labelling by \citet{madureira-schlangen-2020-incremental} and \citet{kahardipraja-etal-2021-towards}; recently, adaptive policies were proposed to reduce the computational load \citep{kaushal-etal-2023-efficient,tapir}. \citet{rohanian-hough-2021-best} and \citet{chen-etal-2022-teaching} explored adaptation strategies to use Transformers for incremental disfluency detection. In simultaneous translation, where policies are a central concept \citep{zheng-etal-2020-simultaneous,zhang-etal-2020-learning-adaptive}, the restart-incremental approach is in use and revisions are studied \citep{arivazhagan-etal-2020-translation,sen-etal-2023-self}. 

 Sequence labelling is a staple of various incremental linguistic tasks possibly used in dialogue systems, like SRL \citep{konstas-etal-2014-incremental}, POS-tagging \citep{beuck-etal-2011-decision}, dialogue act segmentation \citep{manuvinakurike-etal-2016-toward}, disfluency detection \citep{Hough-2015} and dependency parsing \citep{honnibal-johnson-2014-joint}.

\paragraph{\textbf{Revision Categorisation and Prediction}} Approaches to categorise the properties of revisions or edits exist in various areas. \citet{faigley1981analyzing} examine the effects and causes of revisions in writing, providing a taxonomy on whether revisions change meaning and bring new information. \citet{afrin-litman-2018-annotation} classify revision quality by whether they improve student essays. \citet{anthonio-etal-2020-wikihowtoimprove} categorise revisions and edits in WikiHow in terms of what they cause to the text. Wikipedia's edits have also been classified according to factuality and fluency \citep{bronner-monz-2012-user} and intents \citep{rajagopal-etal-2022-one}. Other typologies and taxonomies have been proposed for translation revisions \citep{fujita-etal-2017-consistent} and multilingual NLG revision operations \citep{callaway-2003-multilingual}.

\citet{vaughan-mcdonald-1986-model} outline three phases of the revision process in NLG: Recognition, editing and re-generation. Revision rules have been applied for incremental summarisation by \citet{robin-1996-evaluating}. Non-incremental revision learning models also exist, relying on revision rules for dependency parsing \citep{attardi-ciaramita-2007-tree} or classification in POS-tagging \citep{nakagawa-etal-2002-revision}. Predicting stability and accuracy of hypotheses is a relevant task \citep{selfridge-etal-2011-stability}, which allows to distinguish hypotheses that will survive and are thus more reliable \citep{baumann-etal-2009-assessing}.

\paragraph{\textbf{Incremental Evaluation}} Table \ref{table:properties} presents an overview of relevant properties for incremental evaluation. In their seminal work, \citet{baumann-incremental} define three general categories of metrics for incremental processing: \textit{similarity}, \textit{timing} and \textit{diachronic}, which can be employed in incremental sequence labelling. They are suitable for capturing \textit{e.g.}~instability (edit overhead), quality of prefixes (correctness) and lag (correction time). \citet{kaushal-etal-2023-efficient} propose streaming exact match, comparing prefixes with the final gold standard. While these metrics capture instability and correctness of output prefixes, we lack a standard way to evaluate the quality of the performed revisions. We thus complement their work by proposing fine-grained metrics focusing on revisions and recomputations. 


\section{Evaluation Methodology}
\label{sec:eval}
\section{Evaluation}
\label{sec:eval}

In this section, we first present our evaluation methodology and then describe performance results.

\subsection{Evaluation Methodology}
\label{ssec:eval-method}

\para{Network traces:} To understand the performance of our video enhancement approaches under diverse scenarios, we collect network traces over QUIC from WiFi, 3G, 4G, and 5G networks, shown in Table~\ref{tab:network_traces}. \kj{We use \textit{net-export}~\cite{net-export} in Chrome to collect the QUIC-related packets while watching Youtube videos. We especially identify packet loss in QUIC by capturing \textit{LOSS\_RETRANSMISSION} and \textit{PTO\_RETRANSMISSION} on the transmission type of the packet. Meanwhile, we measure the downlink throughput using iperf from an Azure server located in the central U.S. to a local client over the Internet. The 3G, 4G and 5G traces include static and walking scenarios. We also move the local client randomly to add mobility to the WiFi traces.}

% \kj{We also accumulate network traces from WiFi, 3G, 4G, and 5G over QUIC to show that our system is also indispensable in modern protocols. We use \textit{net-export}~\cite{net-export} in Chrome to collect the QUIC-related packets. We especially identify packet loss in QUIC by capturing \textit{LOSS\_RETRANSMISSION} and \textit{PTO\_RETRANSMISSION} on the transmission type of the packet. We run iperf to measure TCP/UDP throughput in January 2023.}

% We use 3G, 4G, and 5G traces from the existing works~\cite{mao2017neural,5G-measurement2}. We collect WiFi traces by running iperf from an Azure server to a local client over the Internet. The Azure server is located in the central U.S. and we move the local client randomly to add mobility for WiFi traces.
% We collect LEO traces in the StarLink network. We have access to a StarLink RV ground station in the west coast of the U.S.

% Figure environment removed

\begin{table}
  \centering
  \resizebox{0.5\columnwidth}{!}{%
  \begin{tabular}{c|c|c|c|c}
    \toprule
    & 3G & 4G & 5G & WiFi \\
    \midrule
    Amount & 45 & 62 & 53 & 68 \\
    Avg. Duration (s) & 322 & 317 & 302 & 309 \\
    Avg. Throughput (Mbps) & 7.5 & 21.6 & 36.4 & 82.3 \\
    Avg. Packet loss rate (\%) & 0.9 & 1.3 & 1.6 & 0.5 \\
    % Source & ~\cite{mao2017neural} & ~\cite{5G-measurement2} & ~\cite{5G-measurement2} & self-collected \\
    \bottomrule
  \end{tabular}
  }
  \vspace{10pt}
  \caption{Network traces}
  \label{tab:network_traces}
  % \vspace{-10pt}
\end{table}

\para{Video datasets:} We use the video dataset from NEMO for the evaluation. We choose videos from the top ten popular categories~\cite{medium-report} on YouTube: 'Product review', 'How-to', 'Vlogs', 'Game play', 'Skit', 'Haul', 'Challenges', 'Favorite', 'Education', and 'Unboxing'. From each category, we select five videos from distinct creators which support 4K at 30fps and are at least 5 minutes long. Then, four of them are distributed to the training dataset and the other belongs to the testing dataset. For adaptive streaming, we transcode them into multiple bitrate versions using the VP9 codec as per Wowza’s recommendation~\cite{wowza-recommendation}: \{512, 1024, 1600, 2640, 4400\} kbps at \{240, 360, 480, 720, 1080\}p resolutions. The GOP size is 120 (4 sec). 

\para{Performance metrics:} We use raw 1080p videos as a reference for measuring PSNR. We quantify the quality of recovered and super-resolved video frames using two widely used video quality metrics: SSIM and PSNR. Higher SSIM and PSNR values indicate better video quality. We quantify the performance of our system using QoE. A higher QoE indicates better video streaming for users. 


\subsection{DNN Performance} 

First, we evaluate the DNN performance in terms of video quality.

\para{DNN performance of video recovery: } Figure~\ref{fig:rc_dnn_quality} compares the video quality of simply reusing the previous video frame, predicting the video frame without the binary point code, and predicting using our binary point code. We use these schemes to predict the next 5, 10, 20, and 50 frames and calculate the average video quality. As we can see, video recovery without the binary point code yields 4-9dB PSNR improvement and 0.03-0.13 SSIM improvement over simple frame reuse; and the binary point code further increases PSNR by 6-12dB and increases SSIM by 0.04-0.17. The result shows the effectiveness of our binary point code. As we increase the number of future frames to predict, the prediction quality  using our recovery model degrades gracefully. 
% \zhaoyuan{Additionally, we evaluate the performance on the Macbook Air, which conducts warping at 1080p resolution. As a result, it shows superior performance compared to the iPhone 12, with an improvement of XXdB in PSNR and XX in SSIM.}

Figure~\ref{fig:vis_recovery} further illustrates the visualization of our video recovery results. As we can see, our recovery model can learn the motion movement between two consecutive frames and the recovered frames can closely resemble the ground truth frames. In addition, there is often a large difference between the previews frame and the current frame, and it can also be seen in this visualization that our model generates very reasonable predictions in regions where no reference can be found.

% Figure environment removed

Figure~\ref{fig:rc_cl_dnn_quality} further shows the partial video recovery results. We receive and decode video frames in a WiFi network environment. In this setting, many frames can only be partially decoded and our recovery model can recover these corrupted frames. We fill the decoded part of the frame into the recovered frame for all of the schemes such that the overall video quality is higher than the whole frame prediction. As we can see, our recovery without the binary point code yields 0.6-5dB PSNR improvement and 0.01-0.04 SSIM improvement over reusing the previous frame. The binary point code further increases PSNR by 4-8.5dB and increases SSIM by 0.04-0.06, respectively. 

Moreover, the gap between our video recovery without the binary point code and reusing the previous frame becomes larger because  $I_{part}$ allows the network to get an accurate hint to infer the content of the current frame. Similarly, the gap between the performance of our recovery with the binary point code and the other algorithms increases a lot compared to Figure~\ref{fig:rc_dnn_quality} likely because the model learns a stronger association between RGB frame content and binary point code in the successfully decoded part and better utilizes the learned binary code to generate predictions in the missing part.

Figure~\ref{fig:vis_conceal} plots the visualization of our error concealment results (\ie, recovery from partially corrupted frames). The recovered frames are also very similar to the original video frames. These results demonstrate the effectiveness of our video recovery for both completely lost or partially corrupted frames. 

% Figure environment removed

\para{DNN Performance of video super-resolution: } Figure~\ref{fig:sr_dnn_quality} compares the performance of our video super-resolution with upsampling. As we can see, our SR improves the PSNR and SSIM by 1.2dB, 1.1dB, 1dB, and 1.3dB; 0.015, 0.01, 0.007, and 0.008 at 240p, 360p, 480p, and 720p, respectively. The lower resolution video frames yield a higher improvement, as expected. Figure~\ref{fig:vis_sr} plots the visualization of video super-resolution results. Our proposed super-resolution algorithm delivers stable video frame quality improvement at all resolutions. % This confirms that our proposed super-resolution algorithm, which can run in real time on a mobile device, can consistently support the ABR algorithm to make better video streaming strategies. % this doesn't use ABR yet



% Figure environment removed

% Figure environment removed

\subsection{System Performance}

In this section, we evaluate the system performance in terms of video QoE. Note that we downscale the throughput for all network traces so that their throughput falls into the range between the highest and lowest video bit rates.  %  because adaptive streaming does not deliver any benefits at very high throughput. 
The average downscaled throughput among all the network traces is around 1-2Mbps. % To evaluate the performance under a lossy network environment, we use tc-netem as a Linux tool to impose packet loss on the network traffic using a 2-state Gilbert loss model, where the loss probability of the next packet depends on the previous state. % To emulate packet burst losses, we use the probability model, $Prob_{n} = p * Prob_{n-1} + (1 - p) * Random$, where each successive probability depends on a probability $p$ on the last one.

% \subsubsection{Video Recovery}\mbox{}\\

\para{QoE performance of video recovery: }To evaluate the QoE performance of video recovery, we consider three schemes: (i) without recovery model, (ii) without recovery-aware ABR, and (iii) our approach. Note that (ii) means we still perform video recovery for lost or late frames but select the bitrate without taking into account the benefits and cost of video recovery.  

Figure~\ref{fig:rc_only_qoe} shows the QoE performance of recovery-only schemes across different network traces. We make the following observations. First, video recovery alone improves the average QoE by 6.3\%, 11.2\%, 14.2\%, and 9.6\% in 3G, 4G, 5G, and WiFi, respectively, because it can recover lost and late frames such that the rebuffering time can be effectively reduced. 

Second, our recovery-aware algorithm improves over without recovery by 8.6\%, 18.3\%, 22.8\%, and 14.5\% in 3G, 4G, 5G, and WiFi, respectively, and improves over recovery alone by 2.2\%, 6.4\%, 7.5\%, and 4.5\% in 3G, 4G, 5G, and WiFi, respectively because it is aware of the usage of recovery for the next frames such that the bitrate can be chosen more wisely to maximize the system QoE. 

Third, comparing the results across different types of networks, we observe 5G enjoys the largest improvement because more frames require video recovery as we will show next.

% Figure environment removed 

Figure~\ref{fig:throughput} shows the average downscaled throughput of different network traces. We see a large fluctuation in 5G traces. Figure~\ref{fig:rc_percentage} reports the percentage of recovered frames. As 5G has the largest throughput fluctuation, many video frames are not received in time and require video recovery. Meanwhile, even 4G and WiFi see close to 10\% or more video frames that require video recovery. These numbers suggest that video recovery is important due to challenging network conditions. Figure~\ref{fig:sample_thrp} further shows a sample time series of throughput. We find that the scheme without recovery cannot sustain a good QoE when the throughput varies a lot. Recovery alone has a more stable QoE but sometimes gets below 0 due to the rebuffering overhead. Our approach always chooses the bitrate that yields the best QoE. 

% Figure environment removed

Table~\ref{tab:qoe_rc_frames} reports the average QoE of the recovered video frames only. Video recovery alone improves the QoE for the recovered frames by 1.26 - 10.65. The improvement comes mostly from reduced rebuffering time. 
% [XXX: double check; add break down of the improvement from 3 terms in QoE] 
Incorporating recovery-aware ABR further increases the QoE by 0.25 - 1.4. 
% [XXX: do we select a higher or lower rate in RC-aware ABR]

\para{QoE performance without FEC under lossy networks: } Figure~\ref{fig:rc_lossy_qoe} shows the QoE performance of recovery-only schemes across different network traces. Under this setting, we do not enable FEC for loss recovery. For (i), we reuse the last frame when a video frame is late or lost. For (ii) and (iii), our recovery model recovers both lost frames and late frames. Under a lossy network environment, we observe that video recovery alone improves the average QoE by 58.9\%, 74.3\%, 82.7\%, and 70.6\% in 3G, 4G, 5G, and WiFi, respectively. Our approach improves over that without recovery by 71.8\%, 90.8\%, 110\%, and 84.3\% in 3G, 4G, 5G, and WiFi, respectively, and improves over recovery alone by 8.1\%, 9.5\%, 14.6\%, and 8\% in 3G, 4G, 5G, and WiFi, respectively. We find that the improvement of our approach over baselines increases a lot under the lossy network environment because reusing the previous frames is not effective under many consecutive lost/late frames (as shown in Figure~\ref{fig:rc_dnn_quality}), which is more likely under a lossy environment. However, our recovery model can recover many frames with little degradation so its QoE performance is much better. 

% Figure environment removed

\para{QoE performance with FEC under lossy networks: } So far, we disable FEC in our algorithm. Next we further jointly optimize FEC and video recovery. Figure~\ref{fig:rc_fec_lossy_qoe} compares our algorithm but disable FEC (w/o FEC) with all other schemes with FEC, where the amount of FEC is determined based on our lookup table. We offline build separate lookup tables that map the packet loss rates to desired FEC levels for different schemes. Our joint optimization yields 51\%, 68\%, 83\%, and 72\% improvement over no recovery in 3G, 4G, 5G, and WiFi, respectively. The corresponding improvements over recovery alone are 13\%, 41\%, 48\%, and 31\%, respectively. Also, it outperforms no FEC by 1.2, 1.15, 1.3, and 1.23 in QoE, respectively. These results show that (i) FEC plays an important role under lossy network conditions, (ii) the desired amount of FEC depends on the recovery and ABR algorithms, and (iii) each component in our recovery model (\ie, recovery alone, recovery-aware, and joint optimization of FEC and recovery) is beneficial. 

% Without FEC performs worst and acquires negative average QoEs under different network traces. 

% We further adaptively Our recovery model can still get benefits under lossy networks because we can recover the lost frames without retransmission to avoid much rebuffering overhead. Figure~\ref{fig:qoe_redundant} indicates that there is always a peek where we can get the best QoE with the combination of FEC and our recovery model. To this end, we build a lookup table to get the best FEC redundant ratio under different packet loss rates and different network traces. Figure~\ref{fig:rc_lossy_qoe} shows the QoE performance across different network traces. 

% \subsubsection{Video Super-Resolution}\mbox{}\\

\para{QoE performance of video super-resolution: }Figure~\ref{fig:sr_only_qoe} compares the QoE of our super-resolution with (i) without SR, (ii) SR alone using our model, and (iii) NEMO~\cite{yeo2020nemo}.  Our SR-aware approach significantly outperforms all the other algorithms. Its improvement over (i), (ii), and (iii) are 18\%, 21\%, 22\%, and 19\%; 4.5\%, 6.5\%, 7.1\%, and 4.5\%; 0.7\%, 3.8\%, 4.5\%, and 2.7\% in 3G, 4G, 5G, and WiFi, respectively. SR alone brings 12\%-14\% improvement, and SR-aware ABR further brings 4\%-7\% improvement, which shows the importance of joint design of ABR and SR. % It is interesting that the improvement of SR-aware ABR is so large. [XXX: zoom in what video rates are selected. do we select higher or lower rates?] Our approach out-performs NEMO, the state of the art because XXX. 

% Figure environment removed

% \subsubsection{Video Recovery and Super-Resolution}\mbox{}\\

\para{QoE performance of video recovery and super-resolution: } Figure~\ref{fig:sr_rc_qoe} compares the QoE of (i) without SR or recovery, (ii) SR and recovery alone, (iii) NEMO, and (iv) our final algorithm. Our algorithm out-performs (i), (ii), and (iii) by 23.7\%, 32.2\%, 37.1\%, and 29\%; 5.9\%, 10\%, 11.9\%, and 8.4\%; 4.7\%, 13.2\%, 17.4\%, and 10.9\% in 3G, 4G, 5G, and WiFi, respectively. It can be found that both SR and Recovery play a significant effect, and combined with our enhancement aware ABR strategy, our method achieves the best performance. It out-performs NEMO by 4.7\%-17.4\%, and even SR and Recovery alone out-perform NEMO in all cases except 3G, because NEMO does not have recovery and has to reuse the previous frames for late or lost video frames. 3G is better for NEMO due to fewer lost/late video frames.

\zhaoyuan{
\subsection{System Latency and Resource Usage}

\para{System latency: } At the start of video streaming, we establish TCP and QUIC transmission sessions. The binary code, with a size of 1KB, can be encapsulated into a single TCP packet. Consequently, the TCP latency for each frame is expected to be approximately equivalent to the round-trip time (RTT). Given that the decoding and model inference processes of a frame can occur simultaneously with the receiving process of subsequent frames, the total latency can be viewed as the sum of the decoding time and the duration required for neural enhancement or recovery. The decoding time of 240p, 360p, 480p, 720p, and 1080p videos is 1.8, 2.3, 2.9, 4.1, and 6.2ms on the iPhone 12, respectively. Our model adds an additional 22ms for both enhancement and recovery, regardless of the video resolution. This results in a total latency of under 33 ms, demonstrating real-time processing capability in our system.

\para{CPU usage and energy consumption: } We also measure the CPU utilization and energy consumption with and without our model. We only evaluate the neural video recovery because both video recovery and enhancement share a similar model structure and exhibit identical inference time. This similarity implies comparable CPU usage and energy consumption between the two models. Without DNN processing, iPhone 12's CPU utilization is 28\% and the energy consumption is 0.04J per frame. Under 20\% frame losses, the corresponding numbers are 37\% and 0.05J, and under 100\% frame losses, they are 68\% and 0.07J. Consequently, with each frame undergoing neural recovery or enhancement, the expected battery life decreases from 13.2 hours to 7.5 hours.

}


\section{Architecture Profiling}
\label{sec:analysis}
\subsection{Performance Indicators}\label{subsection:performance_indicators_definition}
We use three categories of indicators to analyze the performance of a scenario definition:

\textbf{Inefficiency Rate}: it is the difference in fuel spent between a traffic scenario in which the aircraft comply to DAA resolution advisories (RA), and the analogous scenario in which each vehicle follows its optimal path, without observation of traffic separation rules. In case of vertical deviations, a higher fuel rate is required for climb maneuvers, a lower fuel rate in descent maneuver, which increase the net total for the mission. The resulting value of this indicator is the average value over all traffic configurations in a scenario set. 

\textbf{Loss of Separation (LoS) Rate}: despite there being several ways of defining traffic separation, we examine just the simplest one, which is checking whether or not the vehicles are separated by at least a fixed \emph{minimum distance}. In order to include both DAIDALUS and ACAS sXu in the same tables, we use one separation distance from each one, respectively: 4,000 ft, which is the Horizontal Miss Distance, or HMD, used in DO-365B \cite{DO_365} to define the so-called \emph{Hazard Alert Zone} (HAZ), associated to the DAA Well-Clear (DWC) concept of separation; and 2,000 ft, used in DO-396 \cite{DO_396}, that defines the Loss of Well-Clear (LoWC) event in relation to large UAVs or manned aircraft. These indicators will denote the rate of scenarios, in a scenario set, where the distance between any aircraft pair fell below the afore mentioned threshold values. 

\textbf{Timeout Rate}: this indicates, in a scenario set, the rate of scenarios where any aircraft exceeded a maximum time without reaching its destination point. As pointed out in section \ref{section:scenario_definitions}, this phenomenon occurs because DAA Resolution Advisories (RAs) cause long chains of maneuvers that extend beyond the energy/fuel allowance of the vehicle, due to shortcomings in coordination. We use a time threshold of 1,000 seconds but, in practice, the timeout would be determined by the energy/fuel capacity of the vehicle.

\textbf{Scenario Computing Time}: the time needed to simulate a single scenario instance, in a single core of an Intel Xeon CPU, discounted the fact that multiple scenario instances can be run in parallel in a multi-core CPU. In our simulated scenarios, the DAA algorithm is called at least each 2 seconds, for each aircraft, but when the aircraft is in avoidance mode, that can happen more often. In the case of ACAS sXu, the requirement of receiving various messages to update a single track contribute to result in multiple calls per simulated second.


\subsection{Scenario Specifications and Labels}
In this study, a scenario specification is defined by features such as: the DAA algorithm used, the dimensionality (2-D or 3-D), if it uses extrinsic priorities or not, the target separation parameter, and possibly other features. The scenario labels used in this section encode these attributes:
\begin{itemize}
\item \texttt{dai\_ip\_2d\_4k}: DAIDALUS without extrinsic priorities, 2-D maneuvering and regular 4 kft Horizontal Miss Distance (HMD);
\item \texttt{dai\_ep\_2d\_4k}: similar to the above, with extrinsic priorities;
\item \texttt{sxu\_ip\_2d\_2k}: ACAS sXu with intrinsic priorities only, 2-D maneuvering and regular 2 kft LoWC threshold;
\item \texttt{sxu\_ep\_2d\_2k}: similar to the above, with extrinsic priorities;
\item \texttt{dai\_ep\_3d\_4k}: similar to \texttt{dai\_ep\_2d\_4k}, with 3-D maneuvering;
\item \texttt{dai\_ep\_2d\_2k}: similar to \texttt{dai\_ep\_2d\_4k}, with HMD reduced to 2 kft.
\end{itemize}

And there are other features and labels that will be mentioned below as needed.

\subsection{Performance Analysis}
The analysis of selected scenario specifications is summarized in table~\ref{table:performance_analysis}. The first notorious observation in this table is the effect of extrinsic priorities to decrease inefficiency. With regards to safety indicators, their effect is mixed, and we have to observe each case separately. In the case of DAIDALUS, priorities decreased the 2 kft LoS rate and, most drastically, the timeout rate, while increased the 4 kft LoS rate. In the case of ACAS sXu, priorities increased both LoS rate indicators, but decreased the timeout rate drastically. Based on our rule of thumb assessment, it can be said that DAIDALUS works better with extrinsic priorities, while ACAS sXu works better without them. We conjecture that the following reasons explain this fact: i) that DAIDALUS has more built-in symmetries than ACAS sXu; ii) that ACAS sXu has already built-in priority rules for multi-aircraft encounters, and extrinsic priorities may contradict with them.

\begin{table}[h]
    \caption{Summary of closed-loop performance indicators per scenario.}
    \label{table:performance_analysis}
    \begin{center}
    \begin{tabularx}{\columnwidth}{|p{0.20\columnwidth}|p{0.11\columnwidth}|p{0.09\columnwidth}|p{0.09\columnwidth}|p{0.1\columnwidth}|p{0.105\columnwidth}|}
        \hline
    Scenario spec. & Ineffici-ency rate & LoS rate 4~kft & LoS rate 2~kft & Timeout rate & Scenario comp. time (s)\\
    \hline
    \texttt{dai\_ip\_2d\_4k} & 9.71\% & 1.4E-2 & 6.5E-5 & 1.6E-2 & 6.5E-2\\
    \texttt{dai\_ep\_2d\_4k} & 4.83\% & 2.4E-2 & 4.1E-5 & 0 & 5.1E-2\\
    \texttt{sxu\_ip\_2d\_2k} & 20.7\% & 8.7E-1 & 4.1E-2 & 1.6E-3 & 7.8E+1\\
    \texttt{sxu\_ep\_2d\_2k} & 10.9\% & 9.0E-1 & 2.0E-1 & 1.8E-5 & 7.8E+1\\
    \texttt{dai\_ep\_3d\_4k} & 4.38\% & 8.9E-6 & 0 & 0 & 7.4E-2\\
    \texttt{dai\_ep\_2d\_2k} & 1.3\% & 9.0E-1 & 1.1E-1 & 0 & 3.9E-2\\
    \hline
    \end{tabularx}
    \end{center}
\end{table}

According to a line of reasoning, it would be expected, that, in the more efficient scenarios, the aircraft fly closer to each other and, therefore, there should be a higher probability of losing separation. But this is not the only principle at play, because, if the aircraft perform deviations with the least extra distance, while keeping separation, they stay less in the air and decrease the total number of conflicts. This becomes more understandable when we compare \texttt{dai\_ep\_2d\_4k} with \texttt{dai\_ep\_3d\_4k}, where the latter achieved a small advantage in efficiency, but a huge one in safety. \texttt{dai\_ep\_3d\_4k} is capable of shortening the total distances, but has a residual cost associated to vertical maneuvers, where the climb maneuvers spend fuel at higher rates. 

It cannot escape from observation that DAIDALUS performed much better than ACAS sXu in almost all indicators. In our opinion, it would be reasonable to expect that ACAS sXu would not excel in the LoS rates, especially that of 4 kft, because its first protection criterion is 2 kft, as a built-in feature. However, with smaller protection volumes, the deviations should be smaller and, by this reasoning, its expected inefficiency would be lower than that of DAIDALUS. But our results show otherwise when we compare the cases of DAIDALUS with those of ACAS sXu. The only case in which ACAS sXu obtained an advantage was for the LoS rates comparison between \texttt{sxu\_ip\_2d\_2k} and \texttt{dai\_ep\_2d\_2k}, which have the same separation target. In any case, the Los rate obtained for ACAS sXu meets the performance requirement of ASTM F3442 \cite{ASTM}, which uses the definition of LoWC Ratio (LR), which is the ratio between the LoS rate of 2 kft shown in table~\ref{table:performance_analysis}, with DAA active, and corresponding LoS rate with DAA inactive, the latter being 0.785 according  to our simulations. Thus, the resulting LR scores for ACAS sXu here are 0.052 and 0.253, respectively for the two ACAS sXu specs, which are well below the value of 0.4 from \cite{ASTM}, and consistent with the performance analysis of \cite{DO_396}. We conjecture that these scores would be lower in a future 3-D scenario spec of a ACAS sXu, by following the same improvement obtained with DAIDALUS. 

\subsection{Possible approximations to the closed-loop behavior}
Trying to alleviate the heavy computational load to simulate large numbers of different traffic configurations, in this multi-aircraft, closed-loop setup, we considered some approximated solutions, such as the use of Deep Neural Networks to emulate the ACAS Xu/sXu behavior, in the lines followed by \cite{Julian2018,Bak2022}. However, the existing solutions that we found available were developed for just one intruder aircraft, so they were not suitable for our study. Another possibility would be the exploitation of symmetry transformations \cite{Sibai2020}, however the history-dependent nature of the ACAS Xu/sXu algorithms, associated to the present closed-loop setup, make this possibility unpractical. Thus, we started exploring simpler ways of deducing closed-loop behavior without having to perform the full simulation of a scenario. So far, we tried to analyze correlations between measures of open-loop maneuvers and the closed loop performance. The features that we explored are:
\begin{itemize}
    \item \textbf{Distance flown until the end of the first deviation maneuver} ($\overline{M/D}$): we consider the total distance flown until a ``Clear-of-Conflict'' (CoC) event happens, that is, after one or more divergent maneuvers start in a scenario instance, we stop the scenario when the first divergent maneuver of any aircraft finishes and that individual aircraft is clear of conflict, the moment from which some decision must be made on how to continue the mission. We count the total number of maneuvers started, and divide it by the sum of the flown distances, across all scenario instances in an execution set associated to a scenario spec.
    \item \textbf{Average angle deviation maneuver} ($\overline{\alpha}$): using the same stopping rule of above, we account the angle difference between the heading angles of the aircraft at the beginning of the divergent maneuver and at the stopping moment. 
\end{itemize}
For each of these measures, we ran the 122,416 traffic configuration instances with the open-loop stopping rule. Here, all the scenario specifications are 2-dimensional, and we use abbreviated lables to achieve a better display in the graph legends. Namely, the scenario specifications in this subsection are defined as:
\begin{itemize}
    \item \texttt{D1}: DAIDALUS without extrinsic priorities and with deterministic sensor data;
    \item \texttt{D2}: DAIDALUS with extrinsic priorities and deterministic sensor data;
    \item \texttt{D3}: DAIDALUS with extrinsic priorities and Sensor Uncertainty Mitigation (SUM). This is a design feature \cite{Narkawicz2018} to mitigate uncertainty in sensor data, as for example, to determine the position of an intruder aircraft;
    \item \texttt{D4}: DAIDALUS with its Horizontal Miss Distance (HMD) set to 200 ft (the standard is 4,000 ft) and uncertain sensor data;
    \item \texttt{X1}: ACAS sXu without extrinsic priorities;
    \item \texttt{X2}: ACAS sXu with extrinsic priorities;
    \item \texttt{X3}: ACAS sXu with scenario downscaled to speed of 43 knots and cell radius of 1 km.
\end{itemize}
We used the values of $\overline{M/D}$ and $\overline{\alpha}$ obtained for each of the specs above as inputs to a linear regressor of inefficiency, as defined in section~\ref{subsection:performance_indicators_definition}, and generated a plot with the pairs of (true, predicted) values from this regression, as shown in fig.~\ref{fig:inefficiency_regression}. The trend line in the figure, which depicts the regressor, seems to represent a strong correlation, which is confirmed by the value of $R^2$. When considering each of the regression inputs separately, we obtain $R^2=0.73$ for $\overline{M/D}$ and $R^2=0.77$ for $\overline{\alpha}$, which show that they contribute with approximately equal predicting power. 
% Figure environment removed

We performed a similar analysis for the LoS indicators, but we found very little correlation, as $R^2=0.11$ for the 2~kft LoS indicator. Nevertheless, it can be concluded that these open-loop measurements are a good proxy for the closed-loop inefficiency, with the advantage that the predictor discounts the bias that may have been introduced by the closed-loop mission management system, which is not part of the DAA specification. A rough estimate for the computing time saved is of 68\% in the inefficiency case.      


\section{Conclusion}
\label{sec:conclusion}
\section{Conclusion and Future Work}
In this work, I design corruption-robust algorithms for the Lipschitz contextual search problem. I present the \emph{agnostic checking} technique and demonstrate its effectiveness in designing corruption-robust algorithms. There are several open problems for future research. First, in the algorithm I propose for pricing loss, the schedule for agnostic checks is fixed upfront. Can the learner design an adaptive checking schedule for the pricing loss? Second, this work assumes the learner has knowledge of the Lipschitz constant $L$. Can the learner design efficient no-regret algorithms without knowledge of $L$? 

\section*{Acknowledgements}
We thank the anonymous reviewers for their valuable comments and suggestions. We also thank \citet{kaushal-etal-2023-efficient} for a conversation on this topic at EACL, in particular about the locally valid hypotheses.


\bibliography{anthology,custom}

\end{document}
