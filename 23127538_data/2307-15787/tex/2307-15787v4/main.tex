
\documentclass[11pt, a4paper]{amsart}

\usepackage[utf8]{inputenc}

\usepackage{xcolor}
\definecolor{darkgreen}{rgb}{0,0.5,0}
\usepackage[
%       draft,
        colorlinks, citecolor=darkgreen,
        backref,
        pdfauthor={S. Gajovi\'c, J.S. M\"uller},
        pdftitle={Computing $p$-adic heights on hyperelliptic curves}
]{hyperref}
\usepackage[hyphenbreaks]{breakurl}

\setlength{\parindent}{0mm}
\setlength{\parskip}{1ex plus 0.5ex minus 0.5ex}
\addtolength{\hoffset}{-1cm}
\addtolength{\textwidth}{2cm}
\addtolength{\voffset}{-1cm}
\addtolength{\textheight}{1cm}



\usepackage{fancyhdr}
\usepackage{tikz,enumerate,caption,stmaryrd,amsfonts,amssymb,systeme,comment}
\usetikzlibrary{matrix,positioning,arrows,shapes,chains,calc,automata}
\usetikzlibrary{decorations.pathreplacing}
\usepackage{amssymb}
\usepackage{amsmath}
\usepackage{mathtools}
\usepackage{amsthm}
\usepackage{tikz-cd}
\usepackage[OT2,T1]{fontenc}
\usepackage{url}
\DeclareSymbolFont{cyrletters}{OT2}{wncyr}{m}{n}
\DeclareMathSymbol{\Sha}{\mathalpha}{cyrletters}{"58}


\usepackage{cleveref}
\crefformat{section}{§#2#1#3}
\usepackage{colonequals}
\newcommand{\R}{\mathbb{R}}
\newcommand{\PP}{\mathbb{P}}
\newcommand{\Q}{\mathbb{Q}}
\newcommand{\Qp}{\mathbb{Q}_p}
\newcommand{\C}{\mathbb{C}}
\newcommand{\calC}{\mathcal{C}}
\newcommand{\calU}{\mathcal{U}}
\newcommand{\calX}{\mathcal{X}}
\newcommand{\calD}{\mathcal{D}}
\newcommand{\Z}{\mathbb{Z}}
\newcommand{\N}{\mathbb{N}}
\newcommand{\F}{\mathbb{F}}
\newcommand{\X}{\mathcal{X}}
\renewcommand{\O}{\mathcal{O}}
\newcommand{\p}{\mathfrak{p}}
\newcommand{\q}{\mathfrak{q}}
\newcommand{\hdr}{\operatorname{H^1_{dR}}}
\newcommand{\hrig}{\operatorname{H^1_{rig}}}
\newcommand{\Span}{\operatorname{Span}}
\newcommand{\hdrh}{\operatorname{H^{1,0}_{dR}}}
\newcommand{\hmw}{\operatorname{H^1_{MW}}}
\newcommand\om{\underline{\omega}}
%\newcommand{\omeg}{\underline{\omega}}



\newcommand{\Dom}{\operatorname{Dom}}
\newcommand{\codim}{\operatorname{codim}}
\newcommand{\Sel}{\operatorname{Sel}}
\newcommand{\tors}{\operatorname{tors}}


\newcommand{\aff}{\operatorname{aff}}
\newcommand{\Pic}{\operatorname{Pic}}
\newcommand{\Div}{\operatorname{Div}}
\newcommand{\dv}{\operatorname{div}}
\newcommand{\reg}{\operatorname{reg}}
\newcommand{\Res}{\operatorname{Res}}
\newcommand{\Reg}{\operatorname{Reg}}
\newcommand{\ext}{\operatorname{ext}}
\newcommand{\im}{\operatorname{im}}
\newcommand{\tr}{\operatorname{tr}}
\newcommand{\id}{\operatorname{id}}
\newcommand{\Spec}{\operatorname{Spec}}
\newcommand{\spec}{\operatorname{sp}}
\newcommand{\Sym}{\operatorname{Sym}}
\newcommand{\Proj}{\operatorname{Proj}}
\newcommand{\supp}{\operatorname{supp}}
\newcommand{\ord}{\operatorname{ord}}
\newcommand{\rk}{\operatorname{rk}}
\newcommand{\Frob}{\operatorname{Frob}}
\newcommand{\holo}{\operatorname{holo}}
\newcommand{\anti}{\operatorname{anti}}
\newcommand{\Norm}{\operatorname{Norm}}
\newcommand{\Gal}{\operatorname{Gal}}
\newcommand{\rad}{\text{rad}}
\newcommand{\ch}{\operatorname{char}}
\renewcommand{\sp}{\operatorname{sp}}
\renewcommand{\O}{\mathcal{O}}

\newtheorem{theorem}{Theorem}[section]
\newtheorem{proposition}[theorem]{Proposition}
\newtheorem{corollary}[theorem]{Corollary}
\newtheorem{lemma}[theorem]{Lemma}
\newtheorem{cond}{Condition}
\newtheorem{ass}{Assumption}
\newtheorem{conj}[theorem]{Conjecture}
\newtheorem{alg}[theorem]{Algorithm}
\theoremstyle{remark}
\newtheorem{remark}[theorem]{Remark}
\newtheorem{definition}[theorem]{Definition}
\newtheorem{example}[theorem]{Example}


\numberwithin{equation}{section}

\newcommand{\sg}{{}}
\newcommand{\sm}{{}}

\author[Stevan Gajovi\'c]{Stevan Gajovi\'c}
\address{Stevan Gajovi\'c, Charles University, Faculty of Mathematics and Physics, Department of Algebra, Sokolov\-sk\' a 83, 186~75 Praha~8, Czech Republic}
\email{gajovic@karlin.mff.cuni.cz}

\author[J.~Steffen M\"uller]{J.~Steffen M\"uller}
\address{J.~Steffen M\"uller, University of Groningen, Faculty of Science and Engineering, Algebra, Nijenborgh~9, 9747~AG Groningen, Netherlands}
\email{steffen.muller@rug.nl} 
\title[Computing $p$-adic heights on hyperelliptic curves]{Computing $p$-adic heights on
hyperelliptic curves}
%\author{Stevan Gajovi\'c, J.\thinspace{}Steffen M\"uller}
\date{}




\begin{document}

\maketitle

\begin{abstract}
We describe an algorithm to compute the local Coleman--Gross $p$-adic height 
  at $p$ on a hyperelliptic curve. Previously, this was only possible using an algorithm
due to Balakrishnan and Besser, which was limited to odd degree. While we follow
  their general strategy, our
algorithm is significantly faster and simpler and works for both odd and even degree. 
  We discuss a precision analysis and an implementation in SageMath.
  Our work has several applications, also discussed in this article. 
  These include various versions of the 
  quadratic Chabauty method, and numerical evidence 
  for a $p$-adic version of the conjecture of Birch and Swinnerton--Dyer
  in cases where this was not previously possible.
\end{abstract}

%\tableofcontents


%%%%%%%%%%%%%%%%%%%%%%%%%%%%%%%%%%%%%%%%%%%%%%%%%%%%%%%%%%%%%%%%%%%%%%%%%%%%%%%%%%%%%%%%%%%%%%%%%%%%%%%%%%%%%%%%%%%%%%%%%%%%%%%%%%%%%

\section{Introduction} \label{sec:introduction}

Algorithms for the computation of $p$-adic heights on Jacobian varieties 
curves have recently played a crucial
part in explicit methods for the computation of integral and
rational points on curves via the quadratic Chabauty
method. Such heights also appear in
$p$-adic analogues of the conjecture of Birch and Swinnerton--Dyer.
In this work, we restrict to the case of the Jacobian $J$
of a hyperelliptic curve $$X\colon y^2=f(x)\,,\quad f\in
\Z[x]\;\;\text{squarefree}$$
over $\Q$ and a prime $p$ of good reduction for $X$. 
Denote by
$h\colon J(\Q)\times J(\Q)\to \Q_p$ the $p$-adic height pairing constructed
by Coleman and Gross~\cite{Coleman-Gross-Heights} (depending on various
choices).
Then $h$ is bilinear, and is a sum of local height pairings $h_v(\cdot,\cdot)$
for each finite prime $v\in \Z$. These are defined on pairs of divisors on
$X\otimes \Q_v$ of degree~0 with disjoint support, and the construction 
depends on whether $v\ne p$ or $v=p$.
The former are defined in terms of intersection theory on arithmetic
surfaces and can be computed using algorithms discussed in \cite{Hol12,
Mul14, Raymond-David-Steffen}.
Our work focuses on the latter; these are given
in terms of $p$-adic analysis. 
More precisely, one has $h_p(D_1,D_2)=\int_{D_2}\omega_{D_1}$, where
$\omega_{D_1}$ is a certain differential of the third kind on $X\otimes \Q_p$ whose residue divisor is $D_1$, and  the integral is a Coleman integral
\cite{ColemanI}, \cite{Coleman-deShalit}.



When $X$ has an odd degree model over $\Q_p$, one can compute $h_p$
using work of 
Balakrishnan and Besser~\cite{BBHeights}. 
Based on their strategy, we describe  in Sections~\ref{sec:Algorithm-computations}, \ref{sec:Even-two-infinities}  and~\ref{sec:affine}
an algorithm  to compute $h_p$ that does not
require $X$ to have an odd degree model over $\Q_p$, and
is much simpler and faster than the one from~\cite{BBHeights}  (see the
timings in~\S\ref{subsec:timings}).  
We use it 
\begin{itemize}
  \item to obtain a vast speed-up of the quadratic Chabauty method for
rational points in some situations (see~\S\ref{subsec:qcrat});
\item    to develop a new
quadratic Chabauty method for integral points on even degree
hyperelliptic curves over number fields that is simpler than previous
instances of this method
(see~\S\ref{subsec:intro-QC-details} and~\cite{LinQC}) and crucially relies
on our new algorithm;
    \item to give numerical evidence for  the
$p$-adic Birch and Swinnerton--Dyer conjecture from~\cite{BMS16}
in cases where this was not previously possible
(see~\S\ref{subsec:pbsd} and Section~\ref{S:pbsd}).
\end{itemize}

We now sketch our method.
Assume that $X$ is given by an even degree model (we will see in Section \ref{sec:Algorithm-computations} that this is sufficient), and, for simplicity, that $f$ is monic. Denote by $\infty_-$ and
$\infty_+$ the two points at infinity, namely 
  $\infty_{\pm} = (1:\pm 1:0)$ on the model of $X$ given by the closure of~$y^2=f(x)$ in the weighted
  projective plane $\PP_{1,g+1,1}$.
Since the local heights are bi-additive, it suffices to compute $h_p(P-Q,R-S)$
for distinct points $P,Q,R,S$.
As in~\cite{BBHeights} we assume,  for simplicity and for implementation
reasons (see Remark~\ref{R:extensions})
that these points are all defined over $\Q_p$.

Our method crucially depends on the existence of a nontrivial
degree-0 divisor supported at infinity, namely $D_\infty\colonequals \infty_- - \infty_+$. 
We show in Sections~\ref{sec:Even-two-infinities} and~\ref{sec:affine} that the computation of $h_p(P-Q,R-S)$ can be reduced essentially to the 
computation of Coleman integrals of the differentials 
$$\omega_i\colonequals \frac{x^idx}{y},\quad i=0,\ldots,2g\,. $$
For $P-Q=D_\infty$, this reduction is straightforward, since the residue
divisor of $\omega_g$ is $D_\infty$ (see
Section~\ref{sec:Even-two-infinities}).
Using a change of variables and further reductions to divisors of a specific shape,
we also reduce the computation of the relevant integrals $\int
\omega_{P-Q}$ for most other divisors $P-Q$ to the computation of $\int
\omega_g$ (see Section~\ref{sec:affine}).
Coleman integrals of the differentials $\omega_i$ 
can be computed easily using Balakrishnan's algorithm
\cite{Jen-Even-Degree-CI}, which is implemented in {\tt SageMath}~\cite{Sage}.

No nontrivial degree~0 divisor supported at $\infty$ exists when $f$ has odd degree.
Instead, the authors of~\cite{BBHeights}  compute Coleman
integrals of differentials of the third kind  using a clever but 
more complicated 
approach that involves high precision computations in 
local coordinates of points over high degree extensions of
$\Q_p$.
Our algorithm in fact extends to odd degree, by transforming to an even degree model. 
We give some timings in~\S\ref{subsec:implementation}; these show that our algorithm
is indeed faster than the one from~\cite{BBHeights}.  

A {\tt SageMath}-implementation of our algorithm is discussed
in~\cref{subsec:implementation} and can be obtained 
from~{\url{https://github.com/StevanGajovic/heights_above_p}}. 
We give a precision analysis in Section~\ref{sec:prec-p-adic-heights}.
For our algorithm, we assume that the points $P,Q,R,S$ satisfy a mild condition
(see \sg{Condition 1 in Definition~\ref{def:conds}}, and see~\S\ref{subsec:weakening} on how this 
requirement may be weakened).

%%%%%%%%%%%%%%%%%%%%%%%%%%%%%%%%%%%%%%%%%%%%%%%%%%%%%%%%%%%%%%%%%%%%%%%%
\subsection{Application~I: Quadratic Chabauty for rational points}\label{subsec:qcrat}
  The quadratic Cha\-bauty method for rational points was introduced by
  Balakrishnan and Dogra in~\cite{Jen-Netan-QCRP1}. In its simplest form,
  it  can be used to
  compute the rational points on certain curves $X/\Q$ of genus $g>1$
  having Mordell--Weil rank
  $r\colonequals \rk J(\Q) =g$ and Picard number $\rk\mathrm{NS}(J)>1$, 
  where $J$ is the Jacobian variety of $X$. 
The idea is to fix a prime $p$ of good reduction such that $J(\Q)$ has finite index in $J(\Q_p)$
and to use quadratic relations in the image
of the abelian logarithm $\log\colon J(\Q)\otimes \Q_p\to
\mathrm{H}^0(X_{\Q_p},\Omega^1)^\vee$ coming from $p$-adic heights to write down a locally analytic function $\rho\colon X(\Q_p)\to \Q_p$ which has only
finitely many zeroes and vanishes along $X(\Q)$. This is similar to the
method of Chabauty--Coleman, which uses linear relations and requires
$r<g$.
  The approach of Balakrishnan and Dogra fits into Kim's non-abelian
  Chabauty framework  \cite{Kim2005MFG} and \cite{Kim2009Selmer}, which
  aims to generalize Chabauty--Coleman using $p$-adic Hodge theory
  and arithmetic fundamental groups.
  Their results have  been
  generalized~\cite{Jen-Netan-QCRP2}, reinterpreted~\cite{EL19, BMS21} and 
  applied to several explicit examples (see for instance~\cite{QC13, BBBLMTV19, AABCCKW}).

To apply the quadratic Chabauty method in practice, one needs to 
solve for the $p$-adic height pairing
as a bilinear pairing. 
Two methods for this step are discussed in~\cite[\S3.3, \S3.5.4]{BDMTV2}: The first one uses Neko\-v\'{a}\v{r}'s construction and the machinery of~\cite{QC13}, but it requires that $X$ has ``enough'' rational points. 
The second method uses that 
the global $p$-adic height $h$ can be written a linear combination of products of
Coleman integrals of holomorphic forms as in~\cite{BBM1} (see
also~\S\ref{subsec:intro-QC-details} below). To find the coefficients, we need to evaluate these products and the height pairing $h$ in sufficiently many (pairs of) points. For this purpose we can use the techniques of the present article.

For instance, the Atkin--Lehner quotient
$$X_0^+(107)\colon y^2 = x^6 + 2x^5 + 5x^4 + 2x^3 - 2x^2 - 4x - 3$$
has precisely the six rational points $(\pm1, \pm1), \infty_{\pm}$.
In~\cite{BDMTV2}, this is proved using quadratic Chabauty for the prime
$p=61$. Code for this computation can be found
at~{\url{https://github.com/steffenmueller/QCMod}}.
The fairly large prime $p=61$ was chosen for two reasons: First,
it is expected that $\rho$ vanishes in further $p$-adic points
in addition to the rational ones. To show that these 
are not rational, one can use the Mordell--Weil
sieve~\cite{Bruin-Stoll-MW-Sieve}, but for this to have a reasonable chance of
success, the prime $p$ needs to be chosen carefully.
Second, to apply the algorithm from~\cite{BBHeights}, a prime $p$ such that  $X\otimes \Q_p$ has
an odd degree model over $\Q_p$ is required. 
The first condition is also satisfied for $p=7$, but the second one is not. We use the methods of Sections~\ref{sec:Algorithm-computations}, \ref{sec:Even-two-infinities}, and \ref{sec:affine} to compute  $X_0^+(107)(\Q)$ via a combination of the quadratic Chabauty method and the Mordell--Weil sieve as in~\cite{Jen-Netan-QCRP1, QC13, BDMTV2} for $p=7$. 
On 
a single core of a 4-core 2.6 GHz Intel i7-6600U CPU with 8GB RAM, the
entire computation took 47 seconds, in contrast with about~40
minutes for $p=61$ using the \texttt{Magma}~\cite{Magma} implementation of
the algorithm from \cite{BBHeights}. 

Recent work of Duque-Rosero, Hashimoto and Spelier~\cite{DRHS} makes the
geometric quadratic Chabauty method due to Edixhoven and Lido~\cite{EL19}
explicit by reinterpreting it in terms of Coleman--Gross $p$-adic heights. The work described in
the present article can also be used to apply their algorithm in practice.


\subsection{Application~II: Quadratic Chabauty for integral
points}\label{subsec:intro-QC-details}

Prior to the work of Balakrishnan--Dogra, a quadratic Chabauty method to
compute the integral points on hyperelliptic curves  $X$ of odd degree had been developed by
Balakrishnan, Besser and the
second author in~\cite{BBM1}.
This method also requires that $r=g$ and that the closure of
$J(\Q)$ has finite index in $J(\Q_p)$, but has no assumption on
$\rk\mathrm{NS}(J) $.
The main result of~\cite{BBM1} is the existence of a nontrivial locally
analytic function $\rho\colon \calU(\Z_p)\to \Q_p$, where
$\mathcal{U}=\Spec\Z[x,y]/(y^2-f(x))$, and a finite set $T\subset \Q_p$, both computable, such that $\rho$ takes values in $T$ on 
$\calU(\Z)$.
The construction of $\rho$ is fairly technical: it uses Besser's $p$-adic Arakelov theory, an extension of local Coleman--Gross heights to pairs of  
divisors with common support, and double Coleman integrals from tangential base points.
It was further extended in \cite{BBBM-QCNF} to $\O_K$-integral points on certain hyperelliptic curves $X$ of odd degree over number fields $K$. 
The explicit methods based on this approach developed in~\cite{BBM1, BBM2,
BBBM-QCNF} require the computation of local and global $p$-adic heights,
and hence our algorithm leads to a considerable speed-up of these methods. However, these methods are all limited to curves with a global odd degree model.

We develop in~\cite{LinQC} a quadratic Chabauty
method for integral points on even degree hyperelliptic curves over $\Q$
and more general number fields. Using the present article, this can be made explicit,
and we discuss two examples in~\cite{LinQC}, including an example over
$\Q(\sqrt{7})$ that is not amenable to any other existing technique (as far
as we can tell). The idea is to 
decompose the linear functional $\lambda(P)\colonequals
h(\infty_--\infty_+, P)$ into local terms and to write it as a sum of
single Coleman integrals.
A noteworthy feature of this new quadratic Chabauty
method is its simplicity; no double integrals, $p$-adic Arakelov theory,
tangential base points or
$p$-adic Hodge theory are required. 











%%%%%%%%%%%%%%%%%%%%%%%%%%%%%%%%%%%%%%%%%%%%%%%%%%%%%%%%%%%%%%%%%%%%%%%%
\subsection{Application~III: $p$-adic BSD}\label{subsec:pbsd}
Another
application of our algorithm is the numerical verification of Conjecture~\ref{pbsd}, a $p$-adic Birch and Swinnerton--Dyer
conjecture for modular abelian varieties of $\mathrm{GL}_2$-type with good ordinary reduction, in examples.
So far, this has only been done for elliptic curves and for
Jacobians of hyperelliptic curves that have an odd degree model over $\Q_p$
(see~\cite{BMS16}). Our implementation makes it possible to drop this
condition.
We discuss how to numerically verify Conjecture~\ref{pbsd} and give
a worked example (for $J_0^+(67)$ and $p=11$) in Section~\ref{S:pbsd}. 


\subsection*{Acknowledgements}
We thank 
Jennifer Balakrishnan, Alex Best, Francesca Bianchi, Martin L\"udtke and Michael Stoll for many valuable comments and suggestions, and Amnon Besser, Bas Edixhoven, Sachi Hashimoto, Enis Kaya, Kiran Kedlaya, Guido Lido, David Rohrlich, and Jaap Top for helpful discussions. 
We thank the anonymous referees for various particularly helpful remarks and suggestions for
improvement.
We acknowledge support from DFG through DFG-Grant MU 4110/1-1. 
In addition. S.M. was supported by NWO Grant VI.Vidi.192.106, and S.G. was
supported by a guest postdoc fellowship at the Max Planck Institute for
Mathematics in Bonn, by Czech Science Foundation GAČR, grant 21-00420M, and
by Charles University Research Centre program UNCE/SCI/022 during various
stages of this project. 
Part of this research was done during a visit of S.G. to Boston University,
partially supported by a Diamant PhD Travel Grant; S.G. wants to thank
Boston University for their hospitality.
Some of the research in this article forms part of S.G.'s PhD
thesis~\cite{Gajovic-thesis}, written at the University of Groningen under
S.M.'s supervision.


\section{Coleman--Gross heights} \label{sec:Coleman-Gross}
Following~\cite[Section~2]{BBHeights}, we introduce the $p$-adic height
pairing constructed by Coleman and Gross in \cite{Coleman-Gross-Heights}. Let $p$ be a prime number and let $K$ be a number
field. Let $X/K$  be a 
smooth projective geometrically integral  curve of genus $g>0$, with good reduction at all
primes above $p$, and let $J/K$ denote its Jacobian. 
For a non-archimedean place $v$ of $K$ we write $K_v$ for the completion of $K$ at
$v$, $\O_v$ for its ring of integers and $\pi_v$ for a uniformizer, and we denote
$X_v\colonequals X\otimes K_v$.

The $p$-adic height depends on the  
following data, which we fix:
\begin{enumerate}[(a)]
  \item 
A continuous idèle class character
$$\ell\colon \mathbb{A}_K ^{\ast}/K^{\ast}\longrightarrow \mathbb{Q}_p$$
such that the local characters $\ell_\p$ induced by $\ell$, for $\p\mid p$, do not vanish 
on $\O^\ast_{K_\p}$.
\item 
For each $\p\mid p$ a choice of a subspace $W_\p \subset \hdr(X_\p/K_\p)$
    complementary to the space of holomorphic forms.
\end{enumerate}

	
\begin{remark}\label{rmk:ramified-unramified-over-p}
  The condition in (a) means that the local character $\ell_{\p}$ is ramified,
  which we assume. We require this to define the
  local component of the height using Coleman integration, see Definition
  \ref{def:local-height-above-p}. It is also possible to define the local
  height if the local character $\ell_{\p}$ is unramified, using
  intersection theory exactly as for primes $\q\nmid p$. See also \cite[\S2.2]{BBBM-QCNF}. 
  We use this greater generality in~\cite{LinQC}.
\end{remark}

For any place $\q\nmid p$, we have $\ell_\q(\O_{K_\q}^{\ast})=0$ for continuity reasons, which implies
that $\ell_\q$ is completely determined by $\ell_\q(\pi_\q)$. On the other hand, for
$\p\mid p$, we can decompose 
\begin{equation}\label{trace-on-O_K}
\begin{tikzcd}
 \O_{K_\p}^{\ast}  \arrow{r}{\log_{\p}}  \arrow{rd}{\ell_\p} 
  & K_\p \arrow{d}{t_\p} \\
    & \mathbb{Q}_p
\end{tikzcd}\,,
\end{equation}
where $t_\p$ is a $\mathbb{Q}_p$-linear map. By Condition (a),
it is then
possible to extend $\log_\p$ to 
\begin{equation}\label{trace}
\begin{tikzcd}
 K_\p^{\ast}  \arrow{r}{\log_{\p}}  \arrow{rd}{\ell_\p} 
  & K_\p \arrow{d}{t_\p} \\
    & \mathbb{Q}_p\,.
\end{tikzcd}
\end{equation}
From now on, we fix the branch of the logarithm at $\p\mid p$ to be the one above.

Coleman--Gross define the  height pairing on $J$ by first constructing, for all finite primes $v$
of $K$, local height pairings
$h_v(D_1, D_2)\in\Q_p$ for divisors of degree zero $D_1,D_2\in \Div^0(X_v)$  with disjoint support. These have the property that for $D_1,D_2\in \Div^0(X)$ with disjoint support,
only finitely many $h_v(D_1, D_2)\colonequals h_v(D_1\otimes K_v, D_2\otimes K_v)$ are nonzero.
Hence it makes sense to define 
$$
h(D_1,D_2)\colonequals \sum_{v\sm{\text{ finite}}} 
 h_v(D_1,D_2)\,.
$$
By~\cite[Sections~1,~5]{Coleman-Gross-Heights}, 
the pairing $h$ induces a bilinear pairing
$$
h \colon J(K)\times J(K)\to \Q_p\,.
$$
The local height pairings $h_v$ for $v\nmid p$ are symmetric, whereas the local height pairing $h_{\p}$ for $\p\mid p$ is symmetric if and only if $W_\p$ is isotropic with respect to the cup product
pairing~\eqref{eq:cup-prod} by~\cite[Proposition 5.2]{Coleman-Gross-Heights}.

We  now recall the construction of the local pairings $h_v$.
Since the local terms $h_v$ depend only on $X_v$, we ease  notation as follows:
we write $k=K_v$, $\O = \O_v$, $\pi=\pi_v$, and $C=X_v$.
Let $\chi=\ell_v \colon k^{\ast}\longrightarrow\mathbb{Q}_p$.

\begin{proposition}{\cite[Proposition 1.2]{Coleman-Gross-Heights}}\label{P:htaway}
  If $p\in \O^\ast$, then there exists a unique function $h_v(D_1, D_2)$, defined for
all $D_1,D_2 \in \Div^0(C)$ with disjoint support, which is continuous, symmetric, bi-additive,
and takes values in $\mathbb{Q}_p$, such that for all $f\in k(C)^{\ast}$  we have
  $$h_v(\dv(f),D_2)=\chi(f(D_2))\,,\quad \text{if  }\supp(\dv(f))\cap
  \supp(D_2)=\emptyset\,.$$
\end{proposition} 
We briefly review the construction of $h_v$ in the situation of Proposition~\ref{P:htaway}.
Let $\calC/\O$ be a regular model of $C$ and let $(- \cdot-)$ be the ($\Q$-valued)
intersection multiplicity on $\calC$. Let $\calD_1$ and $\calD_2$ be extensions of
$D_1$ and $D_2$ to $\calC$ such that $(\calD_i\cdot V)=0$
for all vertical divisors $V$ on $\calC$. Then
\begin{equation*}\label{eq:intmult}
  h_v(D_1,D_2)=\chi(\pi)\cdot (\calD_1\cdot \calD_2)\,.
\end{equation*}
Up to a constant, this is equal to the local N\'eron symbol at $v$, discussed in~\cite[Sections~2,~3]{Gross-local-heights}, \sm{which is a local summand of the canonical real-valued height pairing}. It does not depend on the choice of model or on the choices of $\mathcal{D}_1,\mathcal{D}_2$.
\sm{Algorithms to compute this pairing in practice are discussed in  \cite{Hol12,
Mul14, Raymond-David-Steffen}. The {\tt Magma}-command {\tt LocalIntersectionData} uses an implementation of the algorithm in~\cite{Mul14} to compute $h_v$ for all $v\nmid p$; it requires the computation of a regular model using {\tt Magma}'s {\tt RegularModel}-package. A {\tt Magma}-implementation from the algorithm from~\cite{Raymond-David-Steffen}, which does not require $X$ to be hyperelliptic, but currently only works when $K=\Q$, can be obtained from {\url{https://github.com/emresertoz/neron-tate}}.}

The remaining case is when $p\notin\O^\ast$, which we assume from now on
~\footnote{Colmez~\cite{Col98} has shown that there is a uniform
analytic
construction of $h_v$ for all finite primes $v$.}. In particular, $C$ has good reduction.
\begin{definition}\label{diffdef}
A meromorphic differential on $C/k$ is called 
  \begin{itemize}
    \item of the \textit{second kind} if all its residues are~0;
    \item of the \textit{third kind} if it has at most simple poles with residues in $\mathbb{Z}$.
    %(and recall that there are finitely many poles?)} if it is holomorphic except possibly at finitely many points and if it has at most simple poles with residues in $\mathbb{Z}$\,.
  \end{itemize}
\end{definition}
%Recall the Hodge filtration of $\hdr(C/k)$:
%\[0\rightarrow \mathrm{H}^0(C,\Omega^1_{C/k}) \rightarrow \hdr(C/k) \rightarrow H^1(C,\O_C) \rightarrow 0.\]
%Denote by $\hdrh(C/k)$ the image of $\mathrm{H}^0(C,\Omega^1_{C/k})$ in $\hdr(C/k)$.
Let $W=W_v$ be the subspace of $\hdr(C/k)$ chosen above, complementary to
\sm{the image of the holomorphic differentials in $\hdr(C/k)$.}
%$\hdrh(C/k)$. 
 
\begin{remark}\label{rmk:W-Unit-Root}
When $C$ has good ordinary reduction, there is a canonical choice of
  the complement $W$. Namely, Frobenius acts on $\hdr(C/k)$ with $2g$
  eigenvalues, of which exactly $g$ are $p$-adic units in the ordinary
  case; the eigenspace corresponding to all of them is called the  {\it
  unit root subspace}, which is automatically  isotropic with respect to
  the cup product \sm{(see for instance~\cite[Theorem~3.1]{Iov00})}.
  %See \cite[Theorem 3.1(v)]{Noriko-g-eigenvalues-Frobenius} for the hyperelliptic case. \sg{We couldn't find a reference in general, maybe to ask the referee?}
\end{remark}

Differentials of the form $df$ for $f\in k(X)$ are called {\it exact}; they are of the second kind,
whereas differentials of the form $df/f$ for $f\in k(X)^{\ast}$ are called {\it logarithmic}; they are of the third kind.
We denote by $T(k)$ the group of differentials on $C$ of the third kind and by
$T_l(k)$ the group of logarithmic differentials.
The \textit{residue divisor homomorphism}
$$\Res\colon T(k)\rightarrow \Div^0(C), \hspace{5mm} \Res(\omega)=\sum_{P\in \sg{C(\bar{k})}}\Res_P(\omega)P$$
is surjective.
%and induces an exact sequence 
%$$0\rightarrow \hdrh(C/k)\rightarrow T(k)/T_l(k)\rightarrow J(k)\rightarrow 0\,.$$
\sm{We now use the space $W$ to associate a unique differential of the
third kind to a given residue divisor.

Let $\psi\colon T(k)/T_l(k)\rightarrow \hdr(C/k)$ denote 
the canonical homomorphism from
\cite[Proposition~2.5]{Coleman-Gross-Heights}. For our purposes, it
suffices to know that it satisfies Proposition~\ref{prop:Cup-of-psi} below;
we describe a generalization of $\psi$ \sg{in the proof of Proposition~\ref{prop:psi(omega_g)}.}} 
%\begin{lemma}\label{L:psi} \cite[Proposition~2.5]{Coleman-Gross-Heights}
%There is a (canonical) homomorphism 
%$$\psi\colon T(k)/T_l(k)\rightarrow \hdr(C/k)$$  such that the following diagram commutes:
%      $$
%\begin{tikzcd}
%0 \arrow{r} & \hdrh(C/k) \arrow{d}{=} \arrow{r} & T(k)/T_l(k) \arrow{d}{\psi} \arrow{r} & J(k) \arrow{d}{\log_J} \arrow{r} & 0 \\
%  0 \arrow{r} &  \mathrm{H}^0(C,\Omega^1 _C) \arrow{r} & \hdr(C/k)
%  \arrow{r} & \mathrm{H}^1(C,\O_C) \arrow{r} & 0\,.
%\end{tikzcd}$$
%\end{lemma}
We extend $\psi$ to a linear map on all meromorphic differentials $\omega$ on $C/k$ as follows: Write 
$$\omega = \sum_{i=1}^n
  a_i\mu_i+\eta\,,$$ where $a_i\in \overline{k}$, $\mu_i\in T(k)$ and $\eta$ is a differential
  of the second kind. Then we set
  $$\psi(\omega)\colonequals\sum_{i=1}^n a_i\psi(\mu_i)+[\eta]\in
  \hdr(C/k)\,.$$
  In particular, $\psi$ maps a differential of the second kind to its class
in $\hdr(C/k)$.
%Via $\psi$ we may fix a unique differential of the third kind with a prescribed residue divisor, depending on the choice of complementary subspace $W$.
\begin{proposition}\cite[Proposition~3.2]{Coleman-Gross-Heights}\label{prop:Unique-differential-3rd-kind}
For any divisor $D$ of degree 0 on $C$, there is a  unique differential form $\omega_D$ of the third
  kind such that
$$\Res(\omega_D)=D, \hspace{5mm} \psi(\omega_D)\in W\,.$$
Hence the choice $W$ induces a section $\Div^0(C)\longrightarrow T(k)$ of the residue
  divisor homomorphism, given by $D\mapsto \omega_D$. If $D=\dv(f)$ is principal, then $\omega_D=\frac{df}{f}$.
\end{proposition} 
Using Proposition~\ref{prop:Unique-differential-3rd-kind}, we can now
define the central object of this article.

\begin{definition}\label{def:local-height-above-p}
Let $D_1$ and $D_2$ be two divisors on $C$ of degree zero with disjoint support. The {\it local $p$-adic height pairing} is given by
$$h_v(D_1,D_2)\colonequals t_v\left(\int_{D_2}\omega_{D_1}\right),$$ 
  where $t_v$ is the linear map introduced in~\eqref{trace-on-O_K} and $\int_{D_2}\omega_{D_1}$ is a Coleman integral. 
\end{definition} 
\sm{The Coleman integral $\int^Q_P\omega$ between two points $P,Q$ such that $\omega$ has no pole at $P$ or $Q$ is given by termwise integration of the convergent power series expansion of $\omega$ around $P$ if $P$ and $Q$ have the same reduction; such integrals are called~\textit{tiny}. 
Coleman~\cite{ColemanI} and Coleman--de Shalit~\cite{Coleman-deShalit} extended this to general $P$ and $Q$ by analytic continuation along Frobenius. Coleman integrals along divisors are then defined by additivity. If $D$ is a divisor of degree~0 and $\omega$ is holomorphic on $C$, then the integral $\int_D\omega$ can also be defined using the abelian logarithm on $J(\Q_p)$.}

The {\it (algebraic) cup product pairing} $\hdr(C/k)\times \hdr(C/k)\longrightarrow k$ is given by 
\begin{equation}\label{eq:cup-prod}
  ([\mu_1],[\mu_2])\mapsto [\mu_1]\cup[\mu_2]\colonequals \sum_{P\in
  C(\bar{k})} \Res_P\left(\mu_2\int\mu_1\right),
\end{equation}
where $\mu_1$ and $\mu_2$ are differentials of the second kind. It
defines a canonical non-degenerate alternating form on $\hdr(C/k)$.

\begin{proposition}{\cite[Proposition 5.2]{Coleman-Gross-Heights}}\label{prop:height-properties}
The local height pairing $h_v(D_1, D_2)$ is continuous, bi-additive and satisfies 
$$ h_v(\dv(f),D_2)=\chi(f(D_2))\,.$$
Furthermore, it is symmetric if and only if the subspace $W$ of $\hdr(C/k)$ is isotropic with respect to the cup product pairing.
\end{proposition} 

%Finally, we discuss an alternative expression for the map $\psi$ due to Besser, which is more suitable for explicit computations than the one given above.
Finally, we discuss an expression for the map $\psi$ due to Besser, which
is suitable for explicit computations.

\begin{definition}
For a meromorphic form $\omega$ and a form of the second kind $\rho$ on
  $C$, we define the \textit{local symbol} at a point $A\in C(\bar{k})$ by
$$\langle \omega,\rho\rangle _A\colonequals -\Res_A\left(\omega\int\rho\right),$$
where $\int\rho$ is the function $Q\mapsto\int_{Z}^{Q}\rho$, for some fixed
  point $Z\in C(\bar{k})$.
We define the \textit{global symbol} $\langle \omega,\rho\rangle $ as
$$\langle \omega,\rho\rangle \colonequals \sum_{A}\langle \omega,\rho\rangle _A,$$
where the sum is taken over all points in $C(\bar{k})$ where $\omega$ or $\rho$ have a singularity.
\end{definition} 
We note that although the local symbol depends on
the choice of the point $Z$, the definition of the global symbol does not depend on $Z$. 
In Section~\ref{sec:affine}, Step
(ii), we use the following statement. 
\begin{proposition}{\cite[Proposition 4.10]{Besser-Syntomic-2}}\label{prop:Cup-of-psi}
Let $\omega$ and $\rho$ be as above. Then
%$$\langle \omega,\rho\rangle =\psi(\omega)\cup \psi(\rho)\,.$$
  $$\langle \omega,\rho\rangle =\psi(\omega)\cup\sm{ \rho}\,.$$ 
%\sg{Maybe we wanted to keep it general for our application which we now need to change? Do we need the full generality, if this makes sense?}
\end{proposition} 
An important application of this result is the following independence result.
\begin{corollary}\label{cor:psi-commutes-with-isomoprhisms}
  Let $\tau\colon C\to C'$ be an isomorphism of curves and let $\omega'$ be a
 differential of the third kind on $C'$. Then we have
  $$\psi(\tau^{*}\omega')=\tau^{*}(\psi(\omega'))\,.$$
\end{corollary}
\begin{proof}
  This follows from Proposition~\ref{prop:Cup-of-psi}
  and~\cite[Lemma~3.8]{Besser-padic-Arakelov}.
\end{proof}

\begin{corollary}\label{cor:height-independence}
  Suppose that $\tau\colon C\to C'$ is an isomorphism of curves. Let $W$
  (respectively $W'$) be a complementary subspace of $\hdr(C/k)$
  (respectively $\hdr(C'/k)$) such that \sm{$\tau^*(W') = W$}.
  Let $h_p$ (respectively $h'_p$) be the local height on $C$ with respect to $W$
  (respectively on $C'$ with respect to $W'$). Suppose that $D_1,D_2\in \Div^0(C)$ have disjoint support. Then we have
$$
h_p'(\tau_*(D_1), \tau_*(D_2)) = h_p(D_1,D_2)\,.
$$
\end{corollary}

\begin{remark}\label{R:other_heights}
  The $p$-adic height should be viewed as a $p$-adic analogue of the 
classical (N\'eron--Tate) canonical height, which
takes values in $\R$. As for the latter, there 
are several constructions of $p$-adic height pairings on abelian varieties taking values
in $\Q_p$: for instance by Mazur and Tate \cite{Mazur-Tate-p-adic-heights}, and by Schneider
\cite{Schneider-p-adic-heights1}.
  Nekov\'{a}\v{r}~\cite{Nek93} has also given a motivic construction of
  $p$-adic heights
  on fairly general Galois representations. When the abelian
variety in question is the Jacobian of a curve with good ordinary reduction at $p$,
  these constructions are all equivalent to the construction of Coleman and
  Gross,
  which we prefer to work with due to its simplicity. 
  \end{remark}

  %%%%%%%%%%%%%%%%%%%%%%%%%%%%%%%%%%%%%%%%%%%%%%%%%%%%%%%%%%%%%%%%%%%%%%%%
  \subsection{Wide open spaces}\label{subsec:}
\sm{
For a point $P\in C(\Q_p)$, we denote 
the \textit{residue disc} that contains $P$ by 
\begin{equation}\label{res_disc}
D(P)\colonequals\{Q\in C(\Q_p)\,\colon  P\equiv Q\pmod{p}\}\,.
\end{equation}
\textit{Affine discs} are residue discs that do not contain any point at infinity.

  A rigid analytic subspace $\calU$ of the analytification of $C/k$ is called a
  \textit{wide open space} if there are closed discs $D_1,\ldots,D_n$ of radius
  $r_i<1$, each
  contained inside a residue disc, such that $\calU$ is the complement of
  $D_1\cup\ldots\cup D_n$. See~\cite[Section~2]{Besser-Syntomic-2} for more
  information on wide open spaces.

  For a rigid analytic form $\omega$ on $C/k$, we define the
  \textit{residue of $\omega$ along $D_i$} to be the residue along a
  sufficiently small 
  annulus around $D_i$ that is contained in $\calU$ (with respect to a
  suitable local parameter on the annulus). In particular, this
  residue is~0 if $\omega$ 
  has trivial residue at all points in the residue disc containing $D_i$.

  We say that a rigid analytic form on $C/k$
  is \textit{essentially of the second kind} if its residue along each
  $D_i$ is trivial.
  Coleman's integration theory extends to wide open spaces; we refer
  to~\cite[Theorem~4.1]{BBHeights} for its properties. 
  In particular, the
  notion of tiny integrals extends to integrals along annuli in $\calU$
  around a disc $D_i$ if the differential has trivial residue along $D_i$.
}
\section{Precomputations}\label{sec:precomp}
%We first have to compute some data that does not depend on the points $P,Q,R,S$.
Let \sm{$p>2$ be a prime and let $f\in\Z_p[x]$ be of degree $\ge 3$ without repeated roots, so that $C\colon y^2=f(x)$ is a hyperelliptic curve  over $\Q_p$ with good reduction.
Our algorithm to compute local heights (see Algorithm~\ref{alg:hp} below) assumes that the following data is available:
\begin{itemize}
  \item A basis of $\hdr(C/\Q_p)$. 
  \item The action of Frobenius on $\hdr(C/\Q_p)$.
  \item The cup product matrix on $C$.
  \item A basis for  a subspace $W \subset \hdr(C/\Q_p)$, 
    complementary to the space of holomorphic forms and isotropic with respect to the cup product pairing.
\end{itemize}
Recall that $\omega_0,\ldots,\omega_{g-1}$ form a  basis of
the holomorphic differentials on $C$, where  we denote
$\omega_i\colonequals \frac{x^idx}{y}$ for each $i\ge 0$. 
If $\deg(f)$ is odd, then the classes $[\omega_0],\ldots,[\omega_{2g-1}]$ form a basis of 
$\hdr(C/\Q_p)$. We can then compute the action of Frobenius on this basis using
Kedlaya's algorithm \cite{Kedlaya-MW-reduction}, and the cup product matrix can be computed using local integrals, see~\cite[\S5.1]{BBHeights}. An algorithm to compute a basis for the unit root subspace $W$ (when $p$ is ordinary) is discussed in~\cite[\S6.1]{BBHeights}.

We now discuss how to compute these objects when $\deg(f)=2g+2$ is even, which we shall assume for the remainder of this section. We also assume that $f$ is monic, which suffices for the situation discussed in Section~\ref{sec:Algorithm-computations} below.
%We assume that there exists a point $P_0\in C(\Q_p)$ such that $\ord_p(y(P_0))=0$. In this case, we can apply a transformation (see Step \eqref{ints} in \S\ref{subsec:affine}) \sg{Section~\ref{sec:affine}} that does not
%change the local height by Corollary~\ref{cor:height-independence} and results in a model $C\colon y^2=f(x)$ where $f$ \sg{$f\in\Z_p[x]$} is monic of even degree. 
In particular, the points at infinity, denoted
$\infty_+$, $\infty_-$, are in $C(\Q_p)$. }


%\sg{If $p$ is a prime of good ordinary reduction for $C$, and if it is not specified otherwise, we will assume that $W$ is the unit-root subspace and we will compute its basis (up to a desired precision) in Proposition \ref{prop:unit-root-subspace}. If $p$ is not ordinary, we will assume that the space $W$ with its basis is given in advance. We note that in our implementation in {\tt SageMath}, we also compute the basis of the complement of the holomorphic differentials with respect to the symplectic basis. See more in \cref{subsec:subspace}.}

%\begin{enumerate}[(I)]
%  \item\label{basis} Extend $\omega_0,\ldots,\omega_{g-1}$ to a basis of $\hdr(C/\Q_p)$. 
%  \item\label{frob} Compute the action of Frobenius on $\hdr(C/\Q_p)$.
%  \item\label{cup} Compute the cup product matrix on $C$.
%  \end{enumerate}
%\end{alg}
%We consider each Step~\eqref{basis}--\eqref{cup} in turn.}

\subsection{Computing a basis of $\hdr(C/\Q_p)$}\label{basis}
\sm{We extend the classes of the differentials $\omega_0,\ldots,\omega_{g-1}$ to a basis of $\hdr(C/\Q_p)$ in
Lemma \ref{lm:dR-from-MW} below;} first we briefly discuss the cohomological background.
Denote the Weierstrass points on the reduction $\overline{C}$ of $C$ modulo $p$ by $W_1,\ldots,W_{2g+2}\in \overline{C}(\overline{\F_p})$. 
%and let
%\[
%\mathcal{W}_1=(a_1,0),\ldots,\mathcal{W}_{2g+2}=(a_{2g+2},0)\in C(\overline{\Q_p})
%\] 
%be the lifts of $W_1,\ldots,$ $W_{2g+2}$, respectively, to the Weierstrass
%points of $C$. %Denote by $F$ the splitting field of $f$ over $\Q_p$.
Consider the set
\[
V\colonequals \{W_1,\ldots,W_{2g+2},\infty_-,\infty_+\}\subset
\overline{C}(\overline{\F}_p)\,.
%\;\;\; \mathcal{V}\colonequals \{\mathcal{W}_1,\ldots,\mathcal{W}_{2g+2}\}\subset C(\overline{\Q}_p)\,.
\]

Let $U=\Spec\F_p[x,y,1/y]/(y^2-f(x))$ be the affine curve
$\overline{C}\setminus V$. 
\sm{For each $P\in V$, we choose a disc $D_P$ of sufficiently large  radius contained in the residue disc reducing to $P$. Similar
to~\cite{BBHeights}, we consider a wide open space $\calU$ obtained
by removing the discs $D_P$ for all $P\in V$.
Then $U$ is the reduction of $\calU$.
By~\cite[Proposition 4.8]{Besser-Syntomic-2}, there is an exact sequence:
}

%\sg{Is $\C_p$, in the end, OK as all other spaces are regarded as $\Q_p$ vector spaces, should we tensor them with $\C_p$? More importantly, $V$ is defined over $\F_P$ and $\mathcal{V}$ over $\Q_p$, so I think that the indexing in both sequences is wrong. The first sequence is over $\Q_p$, so the last term should be $\bigoplus_{\substack{P\in \mathcal{V}}}\C_p$, and the second sequence is over $\F_p$, so the last term should be $\bigoplus_{\substack{P\in V}}\C_p$. Also, we should also replace the definition of $\mathcal{V}$ by adding points at infinity because they play a role in this exact sequence: $\mathcal{V}\colonequals \{\mathcal{W}_1, \ldots,\mathcal{W}_{2g+2}, \infty_-, \infty_+\}$.}

\begin{equation}\label{eq:Besser-Exact-Sequence}
\begin{tikzcd}
    \displaystyle  0 \arrow[r] & \hdr(C/\Q_p) \arrow[r] &
  \hdr(\calU) \arrow[r, "\Res"] & \bigoplus_{\substack{P\in
  V}}\C_p\,,
\end{tikzcd}
\end{equation} 
%\sm{changed $\mathcal{V}$ to $V$ and the final two $\Q_p$'s to $\C_p$ (see commented out version). Any reason not to do it?}
\sm{where for $P\in V$, the $P$-component of the map $\Res$ is the residue
  along $D_P$.

The work of Baldassarri and Chiarellotto~\cite{Baldassarri-Chiarellotto}
implies that we have
$\hdr(C/\Q_p)\simeq \hrig(\overline{C})$, where
$\hrig(\overline{C})$ is the first rigid cohomology of $\overline{C}$.
Similarly,~\cite{Baldassarri-Chiarellotto} also shows that
$\hrig(U)\simeq \hdr(\calU)$. In rigid cohomology, the exact sequence
becomes } 
\begin{equation}\label{eq:Cohomology-Exact-Sequence}
\begin{tikzcd}
\displaystyle  0 \arrow[r] & \hrig(\overline{C}) \arrow[r] &
  \hrig(U) \arrow[r, "\Res"] & \bigoplus_{\substack{P\in
  V}}\C_p\,,
\end{tikzcd}
\end{equation} 
\sm{
where for $P\in V$, the $P$-component of $\Res$ is the residue $\Res_P$ at
$P$ (see~\cite[Theorem 3.11]{Tuitman-P1-two}).}

By
  \cite[Proposition 1.10]{Berthelot-isomorphism-rigid-MW}, $\hrig(U)$ is isomorphic to  the first Monsky-Washnitzer
    cohomology group $\hmw(U)$ of $U$. \sm{For the construction of the latter see~\cite{Marius-MW-Cohomology}; it has the advantage of
    being amenable to computations.
Let $\hmw(U)^{\pm}$ be the $\pm$ component of $\hmw(U)$ with respect to the hyperelliptic involution. 
By~\cite[\S3.2]{Harrison-Even-Deg-MW}, $\{[\omega_0], \ldots,[\omega_{2g}]\}$
is a basis of $\hmw(U)^-$ and $\{ [\mu_0],\ldots, [\mu_{2g+1}]\}$ is a basis of $\hmw(U)^+$ 
where $\mu_i \colonequals \frac{x^idx}{2y^2}$.
Hence we have}
\begin{equation}\label{eq:H1MW-even-basis}
\hrig(U)\simeq\Span([\omega_0],\ldots,[\omega_{2g}])\oplus\Span([\mu_0],\ldots,[\mu_{2g+1}])\,,
\end{equation}
    \sm{where $\Span$ denotes the $\Q_p$-linear span.}


  %\colon H^1 _{\mathrm{rig}}(U)\to \bigoplus_{\substack{P\in \mathcal{V}}}F\oplus \Q_p \oplus \Q_p$ is defined by
%\[
%\omega\mapsto (\Res_{\mathcal{W}_1}(\omega),\ldots,\Res_{\mathcal{W}_{2g+2}}(\omega), \Res_{\infty_-}(\omega), \Res_{\infty_+}(\omega)).
%\]
%\begin{equation}\label{eq:Cohomology-Exact-Sequence}
%\begin{tikzcd}
%\displaystyle  0 \arrow[r] & \hrig(\overline{C}) \arrow[r] &
%  \hrig(U) \arrow[r, "\Res"] & \bigoplus_{\substack{P\in
%  V}}\C_p\,,
%\end{tikzcd}
%\end{equation} 


Since $\omega_i$ is holomorphic on $C(\overline{\Q}_p)\setminus \{\infty_-,\infty_+\}$ for all $i\geq 0$, and holomorphic on $C$ for $0\leq i \leq g-1$, we have
\begin{itemize}
\item $\Res(\omega_i) = (0,\ldots,0,0,0)$, for $0\leq i \leq g-1$;
\item $\Res(\omega_g) =  (0,\ldots,0,1,-1)$, as we will see in Proposition \ref{prop:residue-of-wg};
\item $\Res(\omega_i) =  (0,\ldots,0,c_i,-c_i)$ for each $g+1\leq i\leq 2g$ and some  $c_i\in \Q_p$.
\end{itemize}

On the other hand, we compute that for $1\leq j\leq 2g+2$, $0\leq i\leq
2g+1$, we have
\[
\Res_{(a_j,0)}(\mu_i)=\Res_{(a_j,0)}\left(\dfrac{x^i}{f'(x)}\dfrac{dy}{y}
\right)= \dfrac{a_j^i}{f'(a_j)}\,,
\]
\sm{where the $a_j\in \overline{\Q_p}$ are the roots of $f$ and $(a_j,0)$ reduces to $W_j$.}

\begin{lemma}\label{lm:positive-cohomology-misses-kernel}
We have $\ker(\Res)\cap \Span(\mu_0,\ldots,\mu_{2g+1}) = \{0\}$.
\end{lemma}
\begin{proof}
Assume that there are $\gamma_0,\ldots,\gamma_{2g+1}\in \Q_p$ such that
  \[\Res(\gamma_0\mu_0+\cdots+\gamma_{2g+1}\mu_{2g+1})=(0,\ldots,0,0,0)\,.\]
  It follows that for any $1\leq j\leq 2g+2$ we have
\[
\sum_{i=0}^{2g+1}\gamma_i \dfrac{a_j^i}{f'(a_j)}=0 \implies
  \sum_{i=0}^{2g+1}\gamma_i a_j^i=0\,.
\]
In other words, the polynomial $\sum_{i=0}^{2g+1}\gamma_i x^i$ has at least $2g+2$ distinct zeros $a_1,\ldots,a_{2g+2}$, hence, it is the constant zero polynomial, so $\gamma_0=\cdots=\gamma_{2g+1}=0$.
\end{proof}

Lemma~\ref{lm:positive-cohomology-misses-kernel} and the discussion
preceding it imply: 
%\sg{Again a question about $\C_p$. Also, maybe when we remove the extra text it would be easier to follow, but I find going back and forth between $\F_p$, $\Q_p$ and $\C_p$ a bit confusing.}
\begin{corollary}\label{cor:exact-sequence-dr-cohomology-computation}
There is an exact sequence
\begin{equation*}\label{eq:cohomology-dR-MWminus-exact-sequence}
\begin{tikzcd}
\displaystyle    0 \arrow[r] & \hdr(C/\Q_p) \arrow[r] &
  \Span(\omega_0,\ldots,\omega_{2g}) \arrow[r, "\Res"] &
  \mathbb{C}_p\oplus \mathbb{C}_p\,,
  %\mathbb{Q}_p\oplus \mathbb{Q}_p\,,
\end{tikzcd}
\end{equation*} 
where $\Res(\omega)=(\Res_{\infty_-}\omega, \Res_{\infty_+}\omega)$.
\end{corollary}


We now use Corollary \ref{cor:exact-sequence-dr-cohomology-computation}  to
construct an explicit basis of $\hdr(C/\Q_p)$. For $0\leq i \leq g-1$, let
$\eta_i\colonequals \omega_i$ and for $g\leq i \leq 2g-1$, let $\eta_i
\colonequals\omega_{i+1} -
c_i\omega_g$, with $c_i$ as in (iii). 

\begin{lemma}\label{lm:dR-from-MW}
The classes of the differentials $\eta_0,\ldots,\eta_{2g-1}$ form a basis of $\hdr(C/\Q_p)$.
\end{lemma}
\begin{proof}
From the above, it follows that $\Span(\eta_0,\ldots,\eta_{2g-1})\subset \ker(\Res)$. Since the classes of the $\eta_i$ are independent in $\hdr(C/\Q_p)$ by \eqref{eq:H1MW-even-basis}, the claim follows by comparing dimensions. 
\end{proof}
%We denote $\omeg\colonequals (\omega_0,\ldots,\omega_{2g})$, $\et\colonequals (\eta_0,\ldots,\eta_{2g-1})$.
%and the subspaces $\langle \omeg\rangle,\langle \et\rangle \subset \hdr(C/\Q_p)$ generated by their classes

\subsection{The action of Frobenius on $\hdr(C/\Q_p)$}\label{frob}
An algorithm for the computation of the matrix $\Phi=(f_{i,j})\in
\Q_p^{(2g+1)\times(2g+1)}$ of Frobenius acting on $\Span([\omega_0], \ldots,[\omega_{2g}])
\subset \hmw(U)^-$
is given in~\cite{Harrison-Even-Deg-MW}. \sm{More precisely, this is a
lift of Frobenius to a rigid analytic morphism that maps the wide open space
$\calU$ to a space with the same cohomology.}
Hence the matrix of Frobenius on $\hdr(C/\Q_p)$ with respect to the basis $[\eta_0],\ldots,[\eta_{2g-1}]$, which we denote by $\Frob$, is 
$$\Frob=\begin{pmatrix} 
f_{0,0} & \ldots & f_{0,g-1} &  f_{0,g+1}-c_{g+1}f_{0,g} & \ldots & f_{0,2g}-c_{2g}f_{0,g} \\
\vdots & \ddots & \vdots & \vdots & \ddots & \vdots\\
f_{g-1,0} & \ldots & f_{g-1,g-1}  & f_{g-1,g+1}-c_{g+1}f_{g-1,g} & \ldots & f_{g-1,2g}-c_{2g}f_{g-1,g}\\
f_{g+1,0} & \ldots & f_{g+1,g-1}  & f_{g+1,g+1}-c_{g+1}f_{g+1,g} & \ldots & f_{g+1,2g}-c_{2g}f_{g+1,g}\\
\vdots & \ddots & \vdots & \vdots  & \ddots & \vdots\\
f_{2g,0} & \ldots & f_{2g,g-1} & f_{2g,g+1}-c_{g+1}f_{2g,g} & \ldots & f_{2g,2g}-c_{2g}f_{2g,g}
\end{pmatrix}.$$

\subsection{The cup product matrix}\label{cup} 
Since all $\eta_i$ are holomorphic on $U$, 
the cup product matrix on $C$, which we denote by $M=(m_{ij})_{i,j}$, is given by 
\[
m_{ij}=\Res_{\infty_+}\left(\eta_j\int\eta_i\right)+\Res_{\infty_-}\left(\eta_j\int\eta_i\right) =2\Res_{\infty_+}\left(\eta_j\int\eta_i\right)\,.
\]
The final equality holds because if $(s_x,s_y)$ are local coordinates of one point at infinity, for example induced by $s_x=\frac{1}{x}$, then $(s_x,-s_y)$ are local coordinates of the other one, and $\eta_j\int\eta_i$ is an even function in $y$. 

\subsection{Isotropic complementary subspaces}\label{subsec:subspace}
\sm{Recall that part of the data required to construct the local height $h_p$ is a 
choice of a subspace $W \subset \hdr(C/\Q_p)$ 
    complementary to the space of holomorphic forms. In applications, we
    often find it convenient to work with symmetric height pairings, so in view of Proposition~\ref{prop:height-properties}, we want $W$ to be isotropic with respect to the cup product pairing. Suppose we have computed the $\eta_i$, the matrix $\Phi$ and the cup product matrix $M$.}

One possible choice of isotropic complementary subspace $W$ is a subspace 
whose basis, together with $[\eta_0],\ldots,[\eta_{g-1}]$,
forms a symplectic basis. \sm{Such a basis can be computed easily from $\eta_0,\ldots,\eta_{g-1}$ and the matrix $M$ using linear algebra.}

Alternatively, if $p$ is ordinary, we can compute the unit root
subspace $W$ \sm{(see Remark~\ref{rmk:W-Unit-Root})} to any desired precision using the
following result. We denote by $\phi$ the lift of Frobenius to (a certain wide open
subspace of) $C$ constructed by
Kedlaya~\cite{Kedlaya-MW-reduction} and
Harrison~\cite{Harrison-Even-Deg-MW}, and we denote by $\phi^*$ the action of Frobenius 
on functions and differentials. 
\begin{proposition}\label{prop:unit-root-subspace}{\cite[Proposition 6.1]{BBHeights}}
\sg{When $p$ is a prime of good ordinary reduction, the elements } ${\phi^*}^n(\eta_g), \ldots , {\phi^*}^n(\eta_{2g-1})$
form a basis of the unit root subspace modulo $p^n$.
\end{proposition}

\begin{remark}\label{unit-root}
For some applications, for instance, to quadratic Chabauty, we can take $W$ to be any
  complementary subspace isotropic with respect to the cup product.
 In the good ordinary case, the
  global Coleman--Gross height with respect to the unit root subspace is of
  special importance due to its appearance in the 
  $p$-adic Birch and Swinnerton--Dyer conjecture formulated in~\cite{BMS16}.
  Hence we need to
  compute $p$-adic heights with respect to the unit
  root subspace to gather empirical evidence for this conjecture. See
  Section~\ref{S:pbsd}.
\end{remark}
\section{Computing local heights above $p$ on even degree hyperelliptic curves}\label{sec:Algorithm-computations}

%\sg{\subsection{Setup}\label{subsec:setup}}
As in the previous section, we let 
$p>2$ be prime and we let $C\colon y^2=f(x)$\sg{, where $f\in\Z_p[x]$} be a hyperelliptic curve with good reduction over $\Q_p$. In \cite{BBHeights},  Balakrishnan and Besser
introduce an algorithm to compute the local height $h_p$ on $C$ with respect to
the unit root subspace when $\deg(f)$ is odd. Here we partially follow
their strategy, but modify the key steps to significantly simplify and
speed up the algorithm %\sm{removed: to compute $h_p$}
when $\deg(f)$ is even. 

\sg{Suppose that there exists a point $P_0\in C(\Q_p)$ such that $\ord_p(y(P_0))=0$. In this case, we can apply a transformation (see Equation~\eqref{eq:change-of-variables} in \sg{Section~\ref{sec:affine}}) that does not
change the local height by Corollary~\ref{cor:height-independence} and
results in a model $C\colon y^2=f(x)$ where $f\in\Z_p[x]$ is monic of even degree. }
%\sg{I would swap the following two sentences: In particular, the points at infinity, denoted
%$\infty_+$, $\infty_-$, are in $C(\Q_p)$. From now on, until the rest of Section~\ref{sec:Algorithm-computations}, we assume that $C\colon y^2=f(x)$ where $f\in\Z_p[x]$ and $p$ is a prime of good reduction for $C$.\\} 
We will assume that we are in this situation from now on.
%In particular, the points at infinity, denoted $\infty_+$, $\infty_-$, are in $C(\Q_p)$. \sg{I would move the following sentence to later. It is not necessary to be here.} Recall that for each $i\geq 0$, we denote $\omega_i\colonequals \frac{x^idx}{y}$; then  $\omega_0,\ldots,\omega_{g-1}$ form a  basis of the holomorphic differentials on $C$. 
\sm{
As in the previous section, we fix a complementary subspace
$W\subset \hdr(C/\Q_p)$, isotropic with respect to the cup product pairing,
and a continuous homomorphism $\chi\colon
\Q_p^{\ast}\longrightarrow\mathbb{Q}_p$, obtained from a continuous idèle class character.
In this situation, $\chi=\log_p$ is the Iwasawa branch of the logarithm, determined by $\log_p(p)=0$, and the linear map $t_\p$ in~\eqref{trace-on-O_K}  is a scalar multiple of the identity map. In the following, we assume for ease of notation that $t_\p$ is the identity map, and we call $\chi$ the (local component of) the \textit{cyclotomic id\`ele class character}.}
%even though we state Step~\eqref{ints} of Algorithm~\ref{alg:hp} more generally. We do this to emphasize that in theory, it can be more general, but for practical reasons we focus on this case - probably rephrase this better! 

%\sg{If $p$ is a prime of good ordinary reduction for $C$, and if it is not specified otherwise, we will assume that $W$ is the unit-root subspace and we will compute its basis (up to a desired precision) in Proposition \ref{prop:unit-root-subspace}. If $p$ is not ordinary, we will assume that the space $W$ with its basis is given in advance. We note that in our implementation in {\tt SageMath}, we also compute the basis of the complement of the holomorphic differentials with respect to the symplectic basis. See more in \cref{subsec:subspace}.}

\subsection{Reduction using bi-additivity}\label{subsec:reduction-bi-additivity}

Any divisors $D_1, D_2\in \Div^0(C)$ with disjoint support can be written as 
$$D_1=\sum_{i=1}^n (P_i-Q_i), \hspace{5mm} D_2=\sum_{j=1}^m (R_j-S_j)\,$$
for pairwise distinct points $P_i,Q_i,R_j,S_j\in C(\overline{\Q}_p)$. For
simplicity (and in our implementation), we will assume
that all these points are already defined over $\Q_p$, but the results below also hold in greater generality.
Using the bi-additivity of the local height pairing, we can reduce the computation of
$h_p(D_1,D_2)$ (with respect to $W$ and $\chi$) to the computation of $h_p(P_i-Q_i,R_j-S_j)$ for all $i,j$. Hence, to
compute $p$-adic heights, it suffices to compute $h_p(P-Q,R-S)$ for distinct points $P,Q,R,S\in C(\Q_p)$. %from now on, we focus on computing $h_p(P-Q,R-S)$.}
%Our method for computing $h_p(P-Q,R-S)$ uses the following strategy,
%already applied by Balakrishnan and Besser for odd degree.
\sm{In Algorithm~\ref{alg:hp}, we introduce a method to compute
$h_p(P-Q,R-S)$ that essentially follows the strategy
applied by Balakrishnan and Besser for odd degree.} 


%We first describe some necessary precomputations in \S\ref{subsec:Curve_Part}, then we discuss how to find a suitable complementary subspace $W$ in~\S\ref{subsec:subspace}.} \sg{These steps do not depend on the choice of the points $P,Q,R,S\in C(\Q_p)$. However, the remaining steps depend on the nature of the points, and also we have some mild restrictions about the points which we can plug in into our algorithm which we now explain.}
%\sm{Add more about the subsections? But we do this in the intro already.}



\subsection{Conditions on points}\label{subsec:conditions-points} 
%IMPORTANT: We will abuse the notation to make our exposition simpler, namely, we will call the points $\infty_{\pm}$ points at infinity, and all the other points affine points. So, our notation of the affine points is with respect to the given model.

%For practical purposes, we introduce the following conditions on points. 
%We will indicate in the respective parts of the article which of these conditions we assume. 
%We assume throughout that the points $P,Q,R,S$ are pairwise distinct.

Let $\iota\colon C\rightarrow C$ denote the hyperelliptic involution. 
\sm{
Recall that for $P\in C(\Q_p)$ we denote by $D(P)$ the residue disc containing it.
\begin{definition}\label{def:conds} We say that 
   distinct points $P_1,P_2,P_3,P_4\in C(\Q_p)$ satisfy \textit{Condition~1} if
  $$\{D(P_1), D(P_2), D(\iota(P_1)), D(\iota(P_2))\}\cap
  \{D(P_3),D(P_4)\} = \emptyset.$$
  We say that they satisfy \textit{Condition~1'} if 
  $$\{\iota(P_1),\iota(P_2)\}\cap \{P_3,P_4\} = \emptyset.$$ 
\end{definition}

Note that Condition~1 is strictly stronger than Condition~1'.
Our strategy for computing $h_p(P-Q, R-S)$, which imitates the strategy from \cite{BBHeights}, relies on the decomposition of divisors of degree zero into the symmetric and antisymmetric part with respect to $\iota$, see for instance Lemma \ref{L:symm-antisymm}. 
This strategy requires that Condition~1' is satisfied for the points $P,Q,R,S$.

%Therefore, we need to keep this condition even in theory, but this is the smallest less general condition that what is required by the original definition of Coleman--Gross - that the points are distinct (even though there is an extension of the definition ADD REF).

In the following, we assume that Condition~1 is satisfied for the points $P,Q,R,S$ for practical purposes; concretely, for the computation of Step~\eqref{ints} of Algorithm~\ref{alg:hp}. If Condition~1 were not satisfied, then at least one of the endpoints of the resulting Coleman integral in Section~\ref{sec:Even-two-infinities} would belong to a residue disc at infinity. Since
Balakrishnan's implementation \cite{Jen-Even-Degree-CI} of even degree Coleman integration in {\tt SageMath} assumes that this is not the case, we would not be able to compute this integral. Similarly, in Section~\ref{sec:affine}, we use a change of variables to compute a certain Coleman integral, which would lead to the same issue. We explain in \cref{subsec:weakening} how to circumvent this issue in theory, thus weakening Condition~1 to Condition~1', but we have not attempted to implement this.}

%We assume that Condition~1 is satisfied for the points $P,Q,R,S$ for practical purposes. Namely, for Step~\eqref{ints} of Algorithm~\ref{alg:hp},  we use a change of variables to compute a certain Coleman integral, see \eqref{eq:change-of-variables}. If  Condition~1 were not satisfied, then  at least one endpoint of the resulting Coleman integral would belong to the residue discs at infinity, and since Balakrishnan's implementation \cite{Jen-Even-Degree-CI} of even degree Coleman integration in {\tt SageMath} assumes that this is not the case, we would not be able to compute this integral. See also {Section~\ref{sec:Even-two-infinities}}, where we run into the same issue.  We explain in \cref{subsec:weakening} how to circumvent this issue in theory, thus weakening Condition~1 to Condition~1', but we have not attempted to implement this.}

In fact, Balakrishnan's implementation of the Coleman integral 
   $\int_{R}^{S} \omega$ assumes that
    $\omega$ is a differential that has no poles at $R$ and $S$ and
    $D(R)= D(S)$ or the following two conditions are satisfied:
\begin{itemize}
\item[(a)] neither $D(R)$ nor $D(S)$  is a disc at infinity and
\item[(b)] if $D(R)$  is a  Weierstrass disc, then $R$ is a  Weierstrass point
  (and the same for $D(S)$).
\end{itemize} 
   %(note that if $\omega$ has poles in a residue disc of $R$ or $S$, the tiny integrals (Coleman integrals whose endpoints $R$ and $S$ satisfy $D(R)=D(S)$) \sm{explain this. maybe also say more about how they're computed?} depend on the chosen branch of logarithm).
   So $R$ and $S$ need to satisfy the following Condition~2 if we want to carry out Step~\eqref{ints} of Algorithm \ref{alg:hp} directly. 
%, where we consider the case $P-Q=\infty_- -\infty_+$, as then if $R$ and $S$ do not satisfy Condition 1, it implies that they belong to the residue disc of points at infinity,and we cannot compute the desired integrals in this case too.  Thus, for our current implementation in {\tt SageMath} we assume Condition~2 on the points, however in theory, we could weaken it to the following condition.
%even though Lemma \ref{L:change} and its proof are correct and Coleman integrals still well defined (unless the points do not satisfy Condition 1 bellow)
%In fact, it assumes Condition 2 (see below) for the endpoints. 

\sg{

\begin{definition}
  We say that two distinct affine points $P_1, P_2\in C(\Q_p)$ satisfy \textit{Condition~2} if they are in the same residue disc or if \[\ord_p(y(P_1)),\ord_p(y(P_2))\in\{0,\infty\}.\]
\end{definition}

In fact, Step~\eqref{ints} of Algorithm \ref{alg:hp} assumes that $P$ and $Q$ also satisfy Condition~2; see Remark~\ref{stepivcond2}.
%There we apply the change of variables \eqref{eq:change-of-variables}, which includes divisions by $y(P)$ and $y(Q)$. If Condition~2 were not satisfied for $P$ and $Q$, this could result in a 
%non-integral model, which is not suitable for our algorithms.
%of another hyperelliptic curve, and we want to avoid this and to work only with integral models. Hence, we want to assume that $p\nmid y(P)y(Q)$, and that gives $\ord_p(y(P)),\ord_p(y(Q))\leq 0$. On the other hand, when Condition 1 is satisfied, and $\ord_p(y(P)),\ord_p(y(Q))< 0$, then after applying the change of variables \eqref{eq:change-of-variables}, the new points would be in Weierstrass discs of the new curve $C'$ and then we reduce to the previous case, which we have already explained how we can remove.
However, it turns out that it is possible to remove Condition~2 while keeping Condition~1. We discuss how to do this in~\S\ref{subsec:weierstrass}.
Therefore, our implementation in {\tt SageMath} does not assume Condition~2. Nevertheless, since we believe that it simplifies the exposition, we will assume it for both $P,Q$ and for $R,S$.
%Nevertheless, we choose to assume Condition 2 in our description  of the algorithm below, in order to simplify the exposition.  

For convenience, we summarize our running assumptions here:
\begin{ass}\label{assumptions}
\hfill
\begin{enumerate}
\item $P,Q,R,S$ satisfy Condition~1.
\item $P,Q$ satisfy Condition~2.
\item $R,S$ satisfy Condition~2.
\end{enumerate}
\end{ass}
}

%\sg{Maybe we should once again reconsider adding the case $D(R)=D(S)$ or $D(P)=D(Q)$, also look at Remark 8.2.}



\subsection{The algorithm}
%Steps~\eqref{basis}--\eqref{W} of Algorithm~\ref{alg:hp} do not depend on the points
%$P,Q,R,S\in C(\Q_p)$ and only have to be completed once. We describe these in~\cref{subsec:Curve_Part}. 
\sm{We now present our algorithm to compute the local height $h_p$. We first state it, and then we discuss the various steps in Sections~\ref{sec:Even-two-infinities} and~\ref{sec:affine}.}

\begin{alg} \label{alg:hp}  {\bf (Computation of the local height)}
\hfill

\sm{ {\bf Input:} 
\begin{itemize}
    \item A hyperelliptic curve $C\colon y^2=f(x)$ over $\Q_p$ of good reduction. 
    \item Four distinct points $P,Q,R,S\in C(\Q_p)$ satisfying Assumption~\ref{assumptions}.
    \item Differentials $\eta_i$ as in~\S\ref{basis}.
    \item The matrix of Frobenius $\Phi$ as in~\S\ref{frob}.
    \item The cup product matrix $M$ as in~\S\ref{cup}.
    \item A basis $\kappa_0,\ldots,\kappa_{g-1}$ for an isotropic complementary subspace $W$ as in~\S\ref{subsec:subspace}.
\end{itemize}
{\bf Output:} The local height $h_p(P-Q,R-S)$ with respect to $W$ and the local component of the cyclotomic id\`ele class character.}
\begin{enumerate}[(i)]
  \item\label{om'} Find some differential $\omega'$ such that $\Res(\omega')=P-Q$.
  \item\label{psiom'} Compute $\psi(\omega')$.
\item\label{coeffs}  Compute the unique coefficients $u_0,\ldots,u_{g-1}\in
  \Q_p$ 
    such that
   \sg{$\omega\colonequals \omega' -\sum^{g-1}_{i=0}u_i\eta_i$} satisfies $\Res(\omega)=P-Q$ and $\psi(\omega)\in W$.
\item\label{ints}  Compute the Coleman integrals $\int_S ^R \omega'$ and \sg{$\int_S ^R \eta_i$} for
  $i=0,\ldots,g-1$ and return $h_p(P-Q,R-S) =\int^R_S\omega$. 
\end{enumerate}
\end{alg}
%\begin{remark}\label{rmk:trace-scalar}
%%Recall that the linear map $t_\p$ in Step~\eqref{ints} is determined by $\chi$ via Equation~\eqref{trace}. Since we assume that $k=\Q_p$, the same equation also induces a branch $\log_p$ of the $p$-adic logarithm. In this situation, $\chi=\log_p$ is the Iwasawa branch of the logarithm, determined by $\log_p(p)=0$, and $t_\p$ is a scalar multiple of the identity map. In the following, we assume for ease of notation that it is the identity map. 
%\end{remark}
%In contrast, Steps~\eqref{om'} --\eqref{ints} depend on the points we plug in to compute heights. 

\sg{{\it Step \eqref{coeffs}} only requires linear algebra, and it is the same for any distinct points $P,Q,R,S\in C(\Q_p)$ satisfying Assumption~\ref{assumptions}.
%(we simply use a formula for a change of basis from the computed basis of
%$\hdr(C)$ in Lemma~\ref{lm:dR-from-MW} to a given basis of $\hdrh(C)\oplus
%W$). \sm{commented out, since I find it confusing.}
\begin{lemma}\label{lm:represent-psi-in-mixed-basis}
Let $\omega'$ be a differential of the third kind such that $\Res(\omega')=P-Q$. Using a base change formula, we express
\[
\psi(\omega')=u_0\eta_0+\cdots+u_{g-1}\eta_{g-1}+u_g\kappa_0+\cdots + u_{2g-1}\kappa_{g-1}
\]
and set
\[
\omega\colonequals\omega'-(u_0\eta_0+\cdots+u_{g-1}\eta_{g-1})\,.
\]
 Then $\Res(\omega) = P-Q$ and $\psi(\omega)\in W$.
\end{lemma}}



We will distinguish two cases: first, we assume
in Section~\ref{sec:Even-two-infinities} that 
one of the divisors is $\infty_- - \infty_+$. This case has no analogue in
odd degree, and we found no simple way to adapt the strategy employed
in~\cite{BBHeights} to this case. Instead, we develop a new approach that
turns out to be faster; the main task is to compute the matrix of Frobenius
on Monsky--Washnitzer cohomology,
which we need anyway for Coleman integration. In fact, 
this case is particularly interesting for two reasons. Namely,
the heights in our new quadratic Chabauty
algorithm for integral points (see~\cite{LinQC}) are precisely of this
form. Moreover, we can reduce the case where all four points are 
affine to it; see Section~\ref{sec:affine}, especially Remark~\ref{R:aff_inf}.


We will not consider the case when one divisor contains one point at infinity
in its support and the other divisor contains the other point in its support, see 
\sg{Assumption~\ref{assumptions}} . When there is only one point at infinity among
these points, we can reduce to the cases already considered,
see~\S\ref{subsec:inf}.

\begin{comment}
For practical purposes, we introduce the following conditions on points. We will
indicate in the respective parts of the article which of these conditions we assume. 
We assume throughout that the points $P,Q,R,S$ are pairwise distinct.
For a point $P\in C(\Q_p)$, we denote 
the \textit{residue disc} that contains $P$ by 
$$D(P)\colonequals\{Q\in C(\Q_p)\,\colon  P\equiv Q\pmod{p}\}\,.$$
\textit{Affine discs} are residue discs that do not contain any point at infinity.
Let $\iota\colon C\rightarrow C$ denote the hyperelliptic involution. 

%\begin{cond}\label{cond2}
%  The points $P_1,P_2,P_3,P_4$ 
%  satisfy $$\{D(P_1), D(P_2), D(\iota(P_1)), D(\iota(P_2))\}\cap
%  \{D(P_3),D(P_4)\} = \emptyset.$$
%\end{cond}

\begin{cond}\label{cond3}
  The affine points $P_1$ and $P_2$ are  in the same residue disc or \[\ord_p(y(P_1)),\ord_p(y(P_2))\in\{0,\infty\}.\]
\end{cond}


 \begin{remark}\label{rmk:cond3} An algorithm to compute the integral
   $\int_S^R \omega$ is given in \cite{Jen-Even-Degree-CI} when
\begin{itemize}
\item[(a)] $R$ and $S$ are in the same residue disc or
\item[(b)] the residue discs of $R$ and $S$ are not Weierstrass discs or discs at
  infinity (the only exception is that the
  point belonging to a Weierstrass disc is actually a Weierstrass point),
\end{itemize} 
   and $\omega$ is a differential that has no poles at $R$ and $S$ (note that if $\omega$ has poles in a residue disc of $R$ or $S$, the tiny integrals depend on the chosen branch of logarithm).
   So $R$ and $S$ need to satisfy Condition~2 if we want to carry out Step~\eqref{ints} directly. 
However, it is possible to remove this condition, and our implementation
   does not assume it; see~\S\ref{subsec:weierstrass}. 
Nevertheless, we choose to assume Condition 2 in our description
 of the algorithm below, in order to simplify the exposition.  
 \end{remark}
\end{comment}

\sg{\section{One divisor supported at $\infty$}\label{sec:Even-two-infinities}}

\sg{Suppose that Assumption~\ref{assumptions} holds for $R,S, \infty_-,\infty_+$.}
%; in this case, it is sufficient to consider only Condition 2, and we will allow $D(R)=D(S)$ - we even implemented this.}
%Suppose that $R,S\in C(\Q_p)$ are affine such that 
%Condition 1 is satisfied for $\infty_-,\infty_+, R,S$ or we have $D(R)=D(S)$. 
\footnote{We explain in \S\ref{subsec:weierstrass} how to remove the assumption that $R$ or $S$ belongs to a Weierstrass disc.}
We now explain how to compute $h_p(\infty_- - \infty_+,R-S)$ using Steps \eqref{om'}--\eqref{ints}.


{\it Step \eqref{om'}} is solved by the following result.
\begin{proposition}\label{prop:residue-of-wg}
Let $\omega'=\omega_g$. Then $\Res(\omega')=\infty_- -\infty_+$.
\end{proposition}

\begin{proof}
  Recall that $\omega_g$ is holomorphic on $C(\overline{\Q}_p)\setminus \{\infty_-,\infty_+\}$. At $\infty_{\pm}$, we can take the uniformizer $t=\frac{1}{x}$, and these two points are distinguished by the function $\frac{x^{g+1}}{y}$, since
\[
\dfrac{x^{g+1}}{y}\left(\infty_+\right)=1, \hspace{2mm} \dfrac{x^{g+1}}{y}\left(\infty_-\right)=-1.
\]
Differentiating the relation $tx=1$ gives $\dfrac{dx}{x}=-\dfrac{dt}{t}$. Upon rewriting
$$\dfrac{x^gdx}{y}=\dfrac{x^{g+1}}{y}\dfrac{dx}{x}=-\dfrac{x^{g+1}}{y}\dfrac{dt}{t}\,,$$
we see that 
 \sg{$\Res_{\infty_+}\omega_g=-1$ and $\Res_{\infty_-}\omega_g=1$}.
\end{proof}

{\it Step \eqref{psiom'}.}
In the spirit of \cite[Algorithm 4.8]{BBHeights}, we define the differential  $$\alpha=\,\sg{\phi^*(\omega_g)-p\omega_g\,}.$$
The strategy employed in~\cite{BBHeights} is to first compute $\psi(\omega') $ and then
deduce $\psi(\alpha)$ from this. Here, we do the opposite, using the following result.

\begin{proposition}\label{P:alphasecond}\hfill
\begin{itemize}
\item[(a)] The differential $\alpha$ is holomorphic at both $\infty_{\pm}$.
\item[(b)] \sg{The
  differential $\alpha$ is essentially of the second kind. More precisely,
    $\alpha$ is holomorphic at non-Weierstrass points, and has  residue~0
    along the discs $D_P$ for all Weierstrass points $P$.} 
    %along a sufficiently small annulus containing Weierstrass points is zero.}
\end{itemize} 
\end{proposition}

\begin{proof}
To prove (a) we only consider one point at infinity, say $\infty_-$; the
  proof is the same for the other one. The action of Frobenius on the
  uniformizer $t=\frac{1}{x}$ is 
\[
\phi^* t=\phi^*\left(\dfrac{1}{x}\right)=\dfrac{1}{x^p}=t^p\,.
\]
We know that $\dfrac{x^gdx}{y}=\dfrac{dt}{t}+A(t) dt $, where $A(t)\in \Q_p[[t]]$.
Hence
\[
\phi^*\left(\dfrac{x^gdx}{y}\right)=p\dfrac{dt}{t}+pt^{t-1}A(t^p)dt\,.
\]
It follows that
\[
\alpha=(pt^{p-1}A(t^p)-pA(t))dt
\]
is holomorphic at $\infty_-$. 

For (b), we first show that $\alpha$ is holomorphic at non-Weierstrass
  points. We already know that \sg{$\omega_g$} is holomorphic at every point
  $P=(a,b)\in C(\overline{\Q}_p)$ with $b\neq 0$. The function $t=x-a$ is a uniformizer at $P$, thus $dx=dt$. We compute 
\begin{equation}\label{eq:Frob-of-wg}
 \sg{\phi^*(\omega_g)}=\phi^*\left(\dfrac{x^gdx}{y}\right)=\dfrac{px^{pg+p-1}dx}{y^p}\displaystyle\sum_{i\geq
  0}\dbinom{-\frac{1}{2}}{k}\dfrac{(f(x^p)-f(x)^p)^i}{y^{2pi}}\,,    
\end{equation}
so that  \sg{$\ord_P\frac{\phi^*(\omega_g)}{dt}=\ord_P\frac{\phi^*(\omega_g)}{dx}\geq 0$}.
Hence the $t$-adic valuation of $\alpha$ is nonnegative.

A uniformizer at a Weierstrass point $P=(a,0)$ is given by $t=y$. Then $x$ is an
  even function of $t$ as we now explain. Let $x=\sum_{n\geq 0} e_nt^n$ for
  some $e_n\in\overline{\Q}_p$, then, inductively equating powers of $t$ on
  both sides of the equation $t^2=f(x)$, we see that $f(e_0)=0$, and then for odd $n$, $f'(e_0)e_n=0$. Since $f'(e_0)\neq 0$, it follows that $e_n=0$ for odd $n$.
We now prove that the same holds for the expansion of $\alpha$ in $t$. Using the relation
$$\dfrac{dx}{y}=\dfrac{2dy}{f'(x)}$$
it is clear that 
$$p\dfrac{x^gdx}{y}=p\dfrac{2x^gdy}{f'(x)}$$
has only even powers of $t$. For the other term of $\alpha$, we
  re-express~\eqref{eq:Frob-of-wg} using $dy$:
$$\phi^*\left(\dfrac{x^gdx}{y}\right)=\dfrac{2px^{pg+p-1}dy}{f'(x)y^{p-1}}\displaystyle\sum_{i\geq 0}\dbinom{-\frac{1}{2}}{k}\dfrac{(f(x^p)-f(x)^p)^i}{y^{2pi}},$$
from which we see that all powers of $y$ are even. Hence there are no odd
  powers of $t$ in the expansion of $\alpha$ and therefore \sm{$\alpha$ has
  trivial residue along $D_P$. }
  %\sg{if this is correct:
  %the residue along any disc containing $P$ is zero, so} (b) follows.
\end{proof}

\begin{comment}
   \sg{Maybe in the beginning, we want to recall that  $\hdr(\calU)\simeq \hdr(U)$ and $H^1 _{\mathrm{rig}}(\overline{C})\simeq \hdr(C)$, so that the sequence~\eqref{eq:Cohomology-Exact-Sequence} can be seen as a sequence (for some reason the next command didn't work in the comment, so I put it below the comment) as from~\cite[Proposition~4.8]{Besser-Syntomic-2}.
}
\begin{equation}\label{eq:Cohomology-Exact-Sequence-over-QP}
\begin{tikzcd}
\displaystyle  0 \arrow[r] & \hdr(C) \arrow[r] &
  \hdr(\calU) \arrow[r, "\Res"] & \bigoplus_{\substack{P\in  V}}\C_p\,,
\end{tikzcd}
\end{equation} 
\end{comment}
 



Recall that \sg{$\phi^*\omega_g=\sum_{i=0}^{2g}f_{g,i}\omega_i$}. 
\begin{proposition}\label{prop:psi(omega_g)}
\sm{
%  The coordinate vector of $\psi(\alpha)$ with respect to the basis $[\eta_0],\ldots,[\eta_{2g-1}]$ is $$\begin{pmatrix}
%f_{g,0} & \cdots & f_{g,g-1} & f_{g,g+1} \cdots & f_{g,2g}
%  \end{pmatrix}^t$$ and 
  We have
\begin{equation*}\label{Psi-Omega-From-Alpha}
\psi(\omega_g)=(\Frob-pI)^{-1}\cdot \begin{pmatrix}
f_{g,0} & \cdots & f_{g,g-1} & f_{g,g+1} \cdots & f_{g,2g}
  \end{pmatrix}^t
\end{equation*}}
with respect to the basis $[\eta_0],\ldots, [\eta_{2g-1}]$ of $\hdr(C)$.
\end{proposition}
\begin{proof}
Harrison's extension \cite{Harrison-Even-Deg-MW} of Kedlaya's algorithm
  \cite{Kedlaya-MW-reduction} shows that there is a certain overconvergent
  function $v$ on $\calU$
  %$C\backslash(\mathcal{V}\cup \{\infty_-,\infty_+\})$ (when we look at the MW cohomology, we look at $C\backslash(\mathcal{V}\cup \{\infty_-,\infty_+\})$, so I don't know if we can use this argument with residues at $\infty_-$ but it seems to me that we should be allowed to use it) 
  such that we have 
\[
\alpha=\sg{\phi^*(\omega_g)-p\omega_g}=\sum_{0\leq i\leq 2g, i\neq g}f_{g,i}\omega_i + (f_{g,g}-p)\omega_g + dv\,.
\]
Recall that $\eta_i=\omega_i$ for $0\leq i<g$ and $\eta_i
=\omega_{i+1} -
c_i\omega_g$ for $g\leq i \leq 2g-1$ and some $c_i\in\Q_p$, so we can further write
\[
\alpha=\sum_{0\leq i< 2g}f_{g,i}\eta_i+\nu\omega_g+dv\,,
\]
for some $\nu\in\Q_p$. Since $\alpha$ is holomorphic at $\infty_-$  by Proposition~\ref{P:alphasecond}(a), $\eta_i$ are holomorphic, and $dv$ has all residues zero, we obtain that $\nu=0$ and
$\alpha=\sum_{0\leq i< 2g}f_{g,i}\eta_i+dv$. 
  Thus, we have that
\begin{equation}\label{alphahdrU}
 \sg{[\alpha]=\sum_{0\leq i< 2g}f_{g,i}[\eta_i]\in \hdr(\calU). }
\end{equation}

\sg{We now use an extension of the map $\psi$ to rigid analytic forms due to
Besser.
By~\cite[Proposition~4.8]{Besser-Syntomic-2}, the
sequence~\eqref{eq:Besser-Exact-Sequence} is split exact, so we obtain
a canonical projection ${\bf p}\colon \hdr(\calU)\to \hdr(C)$ which is the
identity on the image of $\hdr(C)$. For a rigid analytic form $\omega$
that is holomorphic on $\calU$, we define $\psi(\omega)\colonequals
\bf{p}([\omega])$. This extends the map $\psi$, see~\cite[Remark~4.12]{Besser-Syntomic-2}.

Recall that $\alpha$ is holomorphic along $\calU$
and essentially of the second kind on $C$ by Proposition~\ref{P:alphasecond}. Then the class of $\alpha$ in $\hdr(\calU)$ lies in the image of $\hdr(C)$ by
exactness of~\eqref{eq:Besser-Exact-Sequence}. Hence~\eqref{alphahdrU} immediately implies that this class is the image of $\sum_{0\leq i< 2g}f_{g,i}[\eta_i]\in \hdr(C)$. By the extended definition of $\psi$, we have that
$$\psi(\alpha)=\begin{pmatrix}
f_{g,0} & \cdots & f_{g,g-1} & f_{g,g+1} \cdots & f_{g,2g}
\end{pmatrix}^t.$$ 
with respect to the basis $[\eta_0],\ldots, [\eta_{2g-1}]$ of $\hdr(C)$.
The construction of the splitting
in~\cite[Proposition~4.8]{Besser-Syntomic-2} implies that the map $\psi$ satisfies
\begin{equation}\label{phipsi}
  \psi\circ \phi^*={\phi^*}\circ \psi\,.
\end{equation}
Therefore, $\psi(\alpha)=(\Frob - pI)\psi(\omega_g)$, which finishes the proof.
}

  
%Now, since $\alpha$ has a pole of infinite order at Weierstrass points, we need to use the construction by Besser to extend the domain of the map $\psi$ to $\hmw(C\backslash(\mathcal{V}\cup \{\infty_-,\infty_+\}))$, see \cite[Section 4]{Besser-Syntomic-2} and, in particular, \cite[Remark 4.12]{Besser-Syntomic-2}. Then, the split exact sequence from \cite[Proposition 4.8]{Besser-Syntomic-2} implies that the map $\psi\colon\hmw(C\backslash(\mathcal{V}\cup \{\infty_-,\infty_+\}))\rightarrow \hdr(C)$ is simply a projectio
%  \sm{ Hence, by the discussion before the corollary, we obtain that}
%$$\psi(\alpha)=\begin{pmatrix}
%f_{g,0} & \cdots & f_{g,g-1} & f_{g,g+1} \cdots & f_{g,2g}
%\end{pmatrix}^t.$$    
%  The result follows from this and from~\eqref{phipsi}.
\end{proof}

%\begin{remark}\label{rmk:even-degree-no-odd-analogue-of-psi-infinities}
%We found no straightforward way to adapt the algorithm
%  from~\cite{BBHeights} for Step \eqref{psiom'}. Hence we had to develop a
%  completely different approach for this task. \end{remark}

%\sg{Just to mention that we say this twice, here and at the beginning of page 14, for me this is fine, but I wanted to check if you agree.} 

\begin{comment}
\sg{Then we don't need this part, i.e., Lemma~\ref{lm:represent-psi-in-mixed-basis}} and Remark~\ref{rmk:computing_the_correct_omega_D}.

If $p$ is good ordinary and $W$ is the unit root subspace, we use the following for Step \eqref{coeffs}.
\begin{lemma}\label{lm:represent-psi-in-mixed-basis}
Suppose that $p$ is good ordinary. Let
\[
\psi(\omega')=u_0\eta_0+\cdots+u_{g-1}\eta_{g-1}+u_g{\phi^*}^n(\eta_g)+\cdots + u_{2g-1}{\phi^*}^n(\eta_{2g-1})
\]
and set
\[
\omega\colonequals\omega'-(u_0\eta_0+\cdots+u_{g-1}\eta_{g-1})\,.
\]
 Then $\Res(\omega) = \infty_- -\infty_+$ and $\psi(\omega)\in W \pmod{p^n}$, where $W$ is the unit root subspace.
\end{lemma}
\begin{proof}
This follows from Propositions \ref{prop:unit-root-subspace} and \ref{prop:residue-of-wg}.
\end{proof}

\sg{
\begin{remark}\label{rmk:computing_the_correct_omega_D}
In Lemma~\ref{lm:represent-psi-in-mixed-basis}, we have that $\omega'=\omega_g$. But the same principle works when we want to compute $\omega_D$ (up to desired precision) for any degree zero divisor $D$. The only change is that 
  %in conclusion, 
  we obtain $\Res(\omega) = D$  instead of $\Res(\omega) = \infty_-
  -\infty_+$ (which we have when we consider $\omega'=\omega_g$). See Step
  \eqref{coeffs} in  \sg{Section~\ref{sec:affine}}. Here I refer to later where I
  refer to here, I hope that is OK.
\end{remark}
}\sm{Sorry, I don't understand the final sentence.}
\end{comment}

\sg{Using Lemma~\ref{lm:represent-psi-in-mixed-basis}, we compute $u_0,\ldots,u_{g-1}\in \Q_p$ such that for 
$\omega=\omega_g-(u_0\eta_0+\cdots+u_{g-1}\eta_{g-1})$, we have $\Res(\omega) = \infty_- -\infty_+$ and $\psi(\omega)\in W$.}

\begin{corollary}\label{hpinfformula}
The local  height pairing $h_p$ with respect to $\ell_p$ and $W$ satisfies 
\[
h_p(\infty_- - \infty_+,R-S) = \int_S^R \omega_g -
  (u_0\int_S^R\eta_0+\cdots+u_{g-1}\int_S^R\eta_{g-1})\,,
\]
where $u_0,\ldots,u_{g-1}$ are defined in Lemma \ref{lm:represent-psi-in-mixed-basis}.
\end{corollary}
%\sm{Shouldn't the following be moved to earlier, e.g. where we discuss how
%to compute $W$?} \sg{Yes, we can remove this paragraph.}
%We can instead construct a complementary subspace $W$ whose
%basis, together with $\eta_0,\ldots,\eta_{g-1}$, forms a symplectic basis. 
%This only requires the cup product matrix $M$ and linear algebra, starting
%with the basis $\eta_0,\ldots,\eta_{2g-1}$.

{\it Step \eqref{ints}.} 
Using \cite{Jen-Even-Degree-CI}, we compute 
%\sg{here as well $\eta_i\rightarrow\eta_i$}
\[
\int_S^R \omega_g,\;\; u_0\int_S^R\eta_0+\cdots+u_{g-1}\int_S^R\eta_{g-1}\,.
\]
\sm{By Corollary~\ref{hpinfformula}, this finishes the computation of the
local height $h_p(\infty_- - \infty_+,R-S)$.}
Note that for the computation of the integrals above we needed to assume
that $R$ and $S$ belong to the same residue disc or to \sg{non-Weierstrass} affine discs.
Also note that we do not need the cup product matrix to compute
$h_p(\infty_--\infty_+, R-S)$.



\sm{We emphasize that the only step in computing $h_p(\infty_--\infty_+, R-S)$ that depends on the points $R,S\in\Q_p$ is the computation of Coleman integrals with endpoints $R$ and $S$. Hence, if we want to compute several different local heights of the type $h_p(\infty_--\infty_+, R-S)$ (for example, for linear quadratic Chabauty in \cite{LinQC}), it is more efficient to complete all previous steps once and use them as input.}

\sg{\section{The affine case}\label{sec:affine}}


Our goal is to compute $h_p(P-Q,R-S)$ for four distinct affine points $P,Q,R,S\in
C(\Q_p)$. 
\sm{To this end, we describe Steps~\eqref{om'} to 
\eqref{ints} of Algorithm~\ref{alg:hp} in this case.} 
We assume that $P,Q,R,S$ satisfy \sg{Assumption~\ref{assumptions}}. 
As explained in~\S\ref{subsec:implementation}, our implementation in
\texttt{SageMath} assumes only Condition 1 for $P,Q,R,S$. 

We first use the same trick as in \cite{BBHeights}; we write a degree zero divisor as a sum of a symmetric and an antisymmetric one. 

\begin{lemma}\label{L:symm-antisymm}
\begin{align*} 
  h_p(P-Q,R-S) = &\dfrac{1}{2}\log_p\left(\dfrac{x(R)-x(P)}{x(R)-x(Q)}\cdot \dfrac{x(S)-x(Q)}{x(S)-x(R)}\right)\\ 
  &+ \dfrac{1}{2}h_p(P-\iota(P),R-S)-\dfrac{1}{2}h_p(Q-\iota(Q),R-S).
\end{align*}
\end{lemma}
\begin{proof}
From 
\[
\dv\left(\dfrac{x-x(P)}{x-x(Q)}\right)=P+\iota(P)-Q-\iota(Q)\,,
\]
we find
\[
P-Q=\dfrac{1}{2}\dv\left(\dfrac{x-x(P)}{x-x(Q)}\right)+\dfrac{1}{2}(P-\iota(P))-\dfrac{1}{2}(Q-\iota(Q))\,.
\]
Recall from Proposition \ref{prop:height-properties} that we have 
\[
h_p\left(\dv\left(\dfrac{x-x(P)}{x-x(Q)}\right),
  R-S\right)=\log_p\left(\dfrac{x(R)-x(P)}{x(R)-x(Q)}\cdot \dfrac{x(S)-x(Q)}{x(S)-x(P)}\right)\,.
\]
The claim follows from additivity. 
\end{proof}
Condition~1 for  $P,Q,R,S$ implies that the right hand side in
Lemma \ref{L:symm-antisymm} is defined. 
 However, the much weaker condition $\{x(R),x(S)\}\cap \{x(P), x(Q)\} =
\emptyset$ actually suffices. This is precisely Condition~1'. We will explain in~\S\ref{subsec:weakening} how to
weaken Condition 1 to Condition~1' in such a way that
Lemma \ref{L:symm-antisymm} continues to hold (though we did not implement
this). 
\sm{In fact, Lemma \ref{L:symm-antisymm} is the reason why we cannot weaken Condition
1', even in theory. }
%\sg{do we want to add: even in theoretical description?}. 


From now on, we will restrict attention to $h_p(P-\iota(P), R-S)$,
where $P$ is a non-Weierstrass point.
In this case, {\it Step \eqref{om'}}  follows from the following result,
which can be viewed as an explicit version of \cite[Proposition
5.13]{BBHeights}.
\begin{proposition}\label{prop:antisymmetric-residue-differential}
The differential form 
$$\omega'=\dfrac{y(P)}{x-x(P)}\dfrac{dx}{y}$$ 
satisfies $\Res(\omega')=P-\iota(P)$.
\end{proposition}

\begin{proof}
To prove this, we use that the differential $\dfrac{dx}{y}$ is holomorphic on $C$. The only possible poles of $\omega'$ are at $P$ and $\iota(P)$. 
At these points, $t=x-x(P)$ is a uniformizer, so
$$\omega'=\dfrac{y(P)}{y}\dfrac{dt}{t},$$
which immediately gives $\Res_P(\omega')=1$ and $\Res_{\iota(P)}(\omega')=-1$.
\end{proof}

For {\it Step \eqref{psiom'}}  we follow and improve \cite[\S5.2]{BBHeights}.
Let $\omega'$ be a form as in Proposition \ref{prop:antisymmetric-residue-differential} and write
$\psi(\omega')=\sum_{i=0}^{2g-1}u_i\eta_i$. Then Proposition~\ref{prop:Cup-of-psi} implies
$$\langle\omega',\eta_j \rangle=\psi(\omega')\cup
[\eta_j]=\sum_{i=0}^{2g-1}u_i([\eta_i]\cup[\eta_j])\,.$$
Recalling that $M$ is the cup product matrix, we have
  \begin{equation}\label{cupsymbol}
\begin{pmatrix}
u_0 & u_1 & \cdots & u_{2g-1}
\end{pmatrix}^t =-M^{-1}\begin{pmatrix}
    \langle \omega',\eta_0\rangle  &
   \langle \omega',\eta_1\rangle  &
   \cdots &
    \langle \omega',\eta_{2g-1}\rangle 
\end{pmatrix}^t.
  \end{equation}
By the same argument as in \cite[Proposition 5.12]{BBHeights}, we have
\[
\langle\omega',\eta_j\rangle=-\int_{\iota(P)}^P\eta_j-\Res_{\infty_+}\left(\omega'\int\eta_j\right)-\Res_{\infty_-}\left(\omega'\int\eta_j\right)\,,
\]
since $\infty_{\pm}\notin\supp(\Res(\omega'))$.
Because $P$ and $\iota(P)$ (as well as $Q$ and $\iota(Q)$) are endpoints of
Coleman integrals, we use our assumption that
Condition~2 is satisfied for $P$ and, by symmetry, $Q$.
We can simplify this
further as follows:

\begin{lemma}\label{L:zero-residue-even}
\sg{Let $\omega'$ be a form as in Proposition \ref{prop:antisymmetric-residue-differential}.} We have $\Res_{\infty_+}\left(\omega'\int\eta_j\right)=\Res_{\infty_-}\left(\omega'\int\eta_j\right)=0$ for $0\leq j\leq 2g-1$.
\end{lemma}
\begin{proof}
It suffices to prove the statement for $\infty_+$ and for $j\geq g$.
  Consider the uniformizer $t=\frac{1}{x}$ at $\infty_+$. Then we have $\ord_t(y)=-g-1$. Using $dx=-t^{-2}dt$ we compute
\[
\ord_t\left(\dfrac{\omega'}{dt}\right)=\ord_t\left(-\dfrac{y(P)}{t^{-1}-x(P)}\dfrac{t^{-2}}{y}\right)=
  -2-(-1+(-g-1))=g,\quad \text{and}\] 

\[\ord_t\left(\int\eta_{j}\right)
  =\ord_t\left(\int \dfrac{t^{-j-1}t^{-2}dt}{t^{-g-1}}\right)= \ord_t\left(\int  t^{g-j-2}dt\right)=g-j-1.\]

Thus 
\[
\ord_t\left(\dfrac{\omega'\int\eta_{j}}{dt}\right)=2g-1-j\,,
\]
so $\omega'\int\eta_j$ is holomorphic at $\infty_+$ because $j\leq 2g-1$.
\end{proof}

\begin{remark}\label{R:zero-residue-odd}
We can prove a similar statement in the odd degree case. Namely, in the
  setting of \cite[Proposition 5.12]{BBHeights}, we also have
  $\Res_{\infty}\left(\omega'\int\eta_j\right)=0$ for $0\leq j\leq 2g-1$ \sg{and for the same definition of $\omega'$}.
  This can be used to simplify the computation of the global symbols
  in~\cite[\S5.2]{BBHeights}.
\end{remark}

\sg{We compute $u_0,\ldots,u_{g-1}\in \Q_p$ in Step \eqref{coeffs} as in
Lemma~\ref{lm:represent-psi-in-mixed-basis}.}

The main task of {\it Step \eqref{ints}} is the computation of the integral 
$$\int_S^R
\dfrac{y(P)}{x-x(P)}\dfrac{dx}{y}\,.$$ 
If $R$ and $S$ are in the same residue disc, then the integral
is tiny and can be computed directly. In the remaining cases, it can be computed using a change of variables that translates the desired
integral to the integral $\int \omega_g$ 
considered in Section~\ref{sec:Even-two-infinities}.
Namely, consider
\begin{align}\label{eq:change-of-variables}
  \tau\colon C&\to C'\colon
  y'^2=\dfrac{1}{y(P)^2}x'^{2g+2}f\left(x(P)+\frac{1}{x'}\right)\\
  (x,y)&\mapsto(x',y')\colonequals\left(\dfrac{1}{x-x(P)},\dfrac{-y}{y(P)(x-x(P))^{g+1}}\right)\,.\nonumber
\end{align}

\sg{Since $C'$ is defined by an even degree model, and since
$\tau(P)=\infty_-\in C'$ and $\tau(\iota(P))=\infty_+\in C'$, the
differential $\frac{y(P)}{x-x(P)}\frac{dx}{y}$ is mapped to
$\frac{x'^gdx'}{y'}$. This is also consistent with the fact that $\Res(\frac{y(P)}{x-x(P)}\frac{dx}{y})=P-\iota(P)$ and $\Res(\frac{x'^gdx'}{y'})=\infty_- - \infty_+$.}

\begin{lemma}\label{L:change}
Denote $R'=\tau(R)$, $S'=\tau(S)$. Then we have 
  %\sg{here as well $\omega_i\rightarrow\eta_i$}
\[
  h_p(P-\iota(P),R-S) = \int_{S'}^{R'} \omega_g - u_0\int_S^R \eta_0 -
  \cdots - u_{g-1}\int_S^R \eta_{g-1}.
\]
\end{lemma}
\begin{proof}
  This follows from 
\begin{equation}\label{Integral:ChangeOfVariables}
\int_S^R \dfrac{y(P)}{x-x(P)}\dfrac{dx}{y} = 
\int_{S'}^{R'} \dfrac{x'^gdx'}{y'}\,, 
\end{equation}
for which we use the change of variables formula for
  Coleman integrals  (see \cite[Section~2]{Coleman-deShalit}.
\end{proof}
Lemma~\ref{L:change} reduces Step \eqref{ints} to Step \eqref{ints}
from \sg{Section~\ref{sec:Even-two-infinities}}. 

\begin{remark}\label{stepivcond2}
\sm{
In Lemma \ref{L:change}, the points $R, S\in
C(\Q_p)$, and $R',S'\in C'(\Q_p)$ are endpoints of Coleman integrals. 
Hence we
require Condition 2 \sg{for $R,S$}, and we require 
Condition
\sg{1} for $P,Q,R,S$. The division by $y(P)$ in the
definition of the map $\tau$ makes the assumptions $p\nmid y(P)$ and $p\nmid y(Q)$
necessary, since otherwise $C'$ could have non-integral coefficients, so we keep Condition 2 also for $P,Q$. }
\end{remark}

\sm{For convenience, we summarize our algorithm to compute
$h_p(P-Q, R-S)$, where all points are affine. This makes
Algorithm~\ref{alg:hp} more precise in this case.}
%\sg{Now we state our algorithm to compute $h_p(P-Q, R-S)$, where none of the points are points at infinity. Here, as in Section~\ref{sec:Even-two-infinities}, we assume that all necessary input from Algorithm~\ref{alg:hp} is given. 
\sg{
\begin{alg} \label{alg:hp-affine}  {\bf (Computation of $h_p(P-Q, R-S)$ for
  affine $P,Q,R,S$)}
\hfill

 {\bf Input:} The same input as for Algorithm~\ref{alg:hp}, with all points
      $P,Q,R,S$ affine and satisfying Assumption~\ref{assumptions}.

{\bf Output:} The local height $h_p(P-Q, R-S)$.
\begin{enumerate}
%\item Reduce the computation to computing $h_p(P-\iota(P), R-S)$ and
    %$h_p(Q-\iota(Q), R-S)$. From now on, assume that we compute
    %$h_p(P-\iota(P), R-S)$. \sm{Commented out, since this is not really an
    %algorithmic step at this point.}
\item Find a differential $\omega'$ such that $\Res(\omega')=P-\iota(P)$. Define  $\omega'$ as  in Proposition~\ref{prop:antisymmetric-residue-differential}.
  \item\label{affine:psi} Compute $\psi(\omega')$.
\item  Compute the unique coefficients $u_0,\ldots,u_{g-1}\in
  \Q_p$ 
    such that
   \sg{$\omega\colonequals \omega' -\sum^{g-1}_{i=0}u_i\eta_i$} satisfies $\Res(\omega)=P-Q$ and $\psi(\omega)\in W$.
\item  Compute the Coleman integrals $\int_S ^R \omega'$ and \sg{$\int_S ^R \eta_i$} for
  $i=0,\ldots,g-1$ and compute $h_p(P-\iota(P),R-S) =\int^R_S\omega'-\sum^{g-1}_{i=0}u_i\int^R_S\eta_i$.
\item Similarly, compute $h_p(Q-\iota(Q),R-S)$.
\item\label{affine:final-formula} Return
%$$h_p(P-Q,R-S)=$$
$$\dfrac{1}{2}\left(h_p(P-\iota(P),R-S)-h_p(Q-\iota(Q),R-S) +
    \log_p\left(\dfrac{x(R)-x(P)}{x(R)-x(Q)}\cdot \dfrac{x(S)-x(Q)}{x(S)-x(P)}\right)\right).$$
\end{enumerate}
\end{alg}}


\begin{remark}\label{R:BB-nonholo-integral}
Alternatively, one could potentially compute $\int^R_S\omega'$ using 
\cite[Algorithm 4.8]{BBHeights}. Define $\alpha\colonequals
  \phi^*\omega'-p\omega'$ and $\beta$ to be a differential whose residue
  divisor is $R-S$. Then the desired integral can be expressed as 
  \begin{equation}\label{E:4.8}
\int_S^R\omega'=\dfrac{1}{1-p}\cdot \left(\psi(\alpha)\cup\psi(\beta)+\sum_{P\in\mathcal{P}}\Res_P\left(\alpha\int\beta\right)-\int_{\phi(S)}^S\omega-\int_{R}^{\phi(R)}\omega\right),
  \end{equation}
where $\mathcal{P}$ is the subset of $C(\overline{\Q}_p)$ consisting of the
Weierstrass points and the
poles of $\alpha$. See Equation~(14) in~\cite{BBHeights}.
While this formula is only proved there for odd degree hyperelliptic curves, the extension
to even degree is immediate. This leads to a more general, but more complicated algorithm.
\end{remark}


\begin{remark}\label{R:aff_inf}
We can also completely reduce the affine case
  to the infinite case discussed in Section~\ref{sec:Even-two-infinities}. 
  Recall that it suffices to compute heights of the form $h_p(P-\iota(P), R-S)$.
  \sg{Using the same notation and the map $\tau\colon C\rightarrow C'$ as in Lemma
  \ref{L:change}, and the model-independence of local heights (see Corollary
  \ref{cor:height-independence}), this amounts to computing $h_p(\infty_- - \infty_+, R'-S')$
  on $C'$.}
\end{remark}



\section{Precision analysis}\label{sec:prec-p-adic-heights}
\sm{
  In this section we analyze the precision of our algorithms.}
  The algorithms in \S\ref{basis} and \S\ref{cup}  depend on the precision of local
coordinates. In both cases, we need to compute residues of differentials at
points at infinity. The computations in~\S\ref{cup} clearly require more precision than
\S\ref{basis}. We want to compute
$\Res_{\infty_-}\left(\eta_j\int\eta_i\right)$ for $0\leq i,j\leq 2g-1$.
Recall that $t=\frac{1}{x}$ is a uniformizer at $\infty_-$ and that for $g\leq i\leq 2g-1$ we have
$\eta_i=\left( \frac{1}{t^{i-g+2}}+\cdots\right)dt$. The term that requires the largest precision is 
\[
\Res_{\infty_-}\left(\eta_{2g-1}\int\eta_{2g-1}\right)=\left( \frac{1}{t^{g+1}}+\cdots\right)dt\cdot \left( \frac{1}{gt^{g}}+\cdots\right).
\]
Hence we need sufficient $t$-adic precision to compute the coefficient of
$t^{g-1}$ of $\eta_{2g}$, which amounts to computing the first $2g+1$ coefficients of $y=y(t)$. 

Using Harrison's extension of Kedlaya's algorithm~\cite{Harrison-Even-Deg-MW}, we can
compute the matrix of Frobenius acting on $[\omega_0],\ldots,[\omega_{2g}]\subset \hmw(U)^-$ to any desired precision $p^n$. We assume
that we have already computed the basis differentials
$\eta_0,\ldots,\eta_{2g-1}$ in~\S\ref{basis} to the same
precision $p^n$. Then it is clear that the results of~\S\ref{frob}
and~\S\ref{subsec:subspace} are correct to
precision $p^n$ as well. 

\sm{We now analyze precision for Algorithm}~\ref{alg:hp}.
Step \eqref{om'} is exact, so there is no loss of precision. 
%\sg{Now, I reordered and slightly rephrased the rest of this section.}
\sg{
Step \eqref{coeffs} only involves a linear change of variables by Lemma~\ref{lm:represent-psi-in-mixed-basis}, and the loss of precision in this step is bounded by the valuation of the determinant of the corresponding matrix, which can be precomputed.

%a matrix whose determinant is not divisible by $p$ (ADD WHY? Is this true at all, are we sure that the basis of $W$ has $p$-integral coefficients? But even if we lose precision, this can again be computed in advance, as we know both bases).  

The precision in Step \eqref{ints} depends on the precision of the Coleman integrals and the coefficients computed in \eqref{coeffs}. We compute the integrals using Algorithm 4 from \cite{Jen-Even-Degree-CI}. As stated in loc. cit., the same precision estimates as in~\cite{BBK} apply. Briefly:
\begin{itemize}
\item[(a)] By~\cite[Proposition~18]{BBK}, the tiny integrals in~\cite[Equation~(3.2)]{Jen-Even-Degree-CI} are computed correctly to $\min\{n,k+1-\lfloor\log_p(k+1)\rfloor$ digits of precision, where the points are accurate to $n$ digits and we truncate the local expansions of the differentials at precision $t^k$.

\item[(b)] As in~\cite{BBK}, the evaluation of the functions $h_i$ at the
  endpoints does not lead to any loss in precision. Indeed, according to
    Harrison's algorithm, the functions $h_i$ are finite sums of terms
    $c_{i,j}x^iy^j$, where $i\in \Z, i\geq 0$, $j\in \Z$. If the $c_{i,j}$
    are computed modulo $p^n$, then $h_i(P)$ is correct to at least
    the same precision.
\item[(c)] By~\cite[Proposition~19]{BBK}, if we denote $m'\colonequals \ord_p\det(\Phi-I)$, and the previous computations are correct modulo $p^n$, then the Coleman integrals are computed correctly modulo $p^{n-m'}$.
\end{itemize}
\begin{remark}
    By~\cite[Section~3]{Harrison-Even-Deg-MW}, a working precision of $n+\lfloor\log_p(2n)\rfloor+1$ digits suffices to compute both the matrix of Frobenius and the functions $h_i$ to precision $p^n$.
\end{remark}

Step \eqref{psiom'} is the only step where we need to distinguish cases. We first deal with the approach from \sg{Section~\ref{sec:Even-two-infinities}}. Suppose that the matrix of Frobenius and the differentials $\eta_0,\ldots,\eta_{2g-1}$ of
$\hdr$ are correct modulo $p^n$. Denoting  $m\colonequals \ord_p\det(\Frob-pI)$, we compute $\psi(\omega')$ in terms of $\eta_0,\ldots,\eta_{2g-1}$ correctly modulo $p^{n-m}$.

It remains to analyze the precision for Step \eqref{psiom'} in \sg{Section~\ref{sec:affine}}; more precisely, this is Step~\eqref{affine:psi} in Algorithm~\ref{alg:hp-affine}). We first compute the global symbols $\langle
\omega',  \eta_j\rangle = -\int_{\iota(P)}^P\eta_j$.
and then apply~\eqref{cupsymbol}.
Hence, if all computations so far are correct modulo $p^n$, and we denote $m''=\ord_p\det(M)$, where $M$ is the cup product matrix, the coordinates of $\psi(\omega')$ in Step \eqref{psiom'} are computed accurately modulo $p^{n-m''}$.

All three matrices $\Frob-pI$, $\Phi-I$, and $M$ depend only on the curve. Thus, we may precompute their determinants and predict which starting
precision is necessary to reach a specified target precision.
For the approach discussed in Remark~\ref{R:aff_inf}, we also need to
compute $\det(\Frob-pI)$ and $\det(\Phi-I)$ for the curve $C'$, which depends on the point $P$.

In addition, Step~\eqref{affine:final-formula} in Algorithm~\ref{alg:hp-affine} requires an evaluation of $\log_p$, but since $P,Q,R,S$ satisfy Assumption~\ref{assumptions}, this does not lead to any loss of precision. 

\begin{remark}\label{R:pg}
Our precision analysis assumes that the matrix of Frobenius
on $\hmw(U)$ is $p$-adically integral. 
As shown in~\cite[Lemma~3.4, Conclusion]{Harrison-Even-Deg-MW} this is the
case when $p>g$, so we assume this, as
in \cite{Jen-Even-Degree-CI}, whenever needed.
\end{remark}

\begin{remark}\label{R:precspecial}

  In the precision analysis above, we assume Assumption~\ref{assumptions}. As discussed
  in~\S\ref{subsec:weierstrass} and \S\ref{subsec:inf},
  Condition~2 is not necessary if we assume Condition~1, and in order to reduce 
  to the situation treated above, only symmetry and 
tiny integrals are required.
  Hence, we do not need an additional
  precision analysis in this more general situation.
\end{remark}
}

\begin{comment}
Now we need to distinguish cases. We first deal with the approach from \sg{Section~\ref{sec:Even-two-infinities}}.
For Step \eqref{psiom'}, suppose that the matrix of Frobenius and the differentials $\eta_0,\ldots,\eta_{2g-1}$ of
$\hdr$ are correct modulo $p^n$. Denoting  $m\colonequals \ord_p\det(\Frob-pI)$, we compute $\psi(\omega')$ in terms of $\eta_0,\ldots,\eta_{2g-1}$ correctly modulo $p^{n-m}$. Step \eqref{coeffs} is accurate to the same precision as Step \eqref{psiom'}. 
The precision in Step \eqref{ints} depends on the precision of the Coleman integrals and the coefficients computed in \eqref{coeffs}. We compute the integrals using Algorithm 4 from \cite{Jen-Even-Degree-CI}. As stated in loc. cit., the same precision estimates as in~\cite{BBK} apply. Briefly:
\begin{itemize}
\item[(a)] By~\cite[Proposition~18]{BBK}, the tiny integrals in~\cite[Equation~(3.2)]{Jen-Even-Degree-CI} are computed correctly to $\min\{n,k+1-\lfloor\log_p(k+1)\rfloor$ digits of precision, where the points are accurate to $n$ digits and we truncate the local expansions of the differentials at precision $t^k$.

\item[(b)] As in~\cite{BBK}, the evaluation of the functions $h_i$ at the
  endpoints does not lead to any loss in precision. Indeed, according to
    Harrison's algorithm, the functions $h_i$ are finite sums of terms
    $c_{i,j}x^iy^j$, where $i\in \Z, i\geq 0$, $j\in \Z$. If the $c_{i,j}$
    are computed modulo $p^n$, then $h_i(P)$ is correct to at least
    the same precision.
\item[(c)] By~\cite[Proposition~19]{BBK}, if we denote $m'\colonequals \ord_p\det(\Phi-I)$, and the previous computations are correct modulo $p^n$, then the Coleman integrals are computed correctly modulo $p^{n-m'}$.
\end{itemize}
\begin{remark}
    By~\cite[Section~3]{Harrison-Even-Deg-MW}, a working precision of $n+\lfloor\log_p(2n)\rfloor+1$ digits suffices to compute both the matrix of Frobenius and the functions $h_i$ to precision $p^n$.
\end{remark}

\begin{remark}\label{R:pg}
Our precision analysis assumes that the matrix of Frobenius
on $\hmw(U)$ is $p$-adically integral. 
As shown in~\cite[Lemma~3.4, Conclusion]{Harrison-Even-Deg-MW} this is the
case when $p>g$, so we assume this, as in 
in \cite{Jen-Even-Degree-CI}, whenever needed.
\end{remark}

It remains to analyze the precision for \sg{Section~\ref{sec:affine}}. In addition to the above,
Step \eqref{om'} requires an evaluation of $\log_p$, but  since $P,Q,R,S$ satisfy Condition
2, this does not lead to any loss of precision. 
Furthermore, Steps \eqref{coeffs} and
\eqref{ints} are analogous to Steps~\eqref{coeffs} and~\eqref{ints} in \sg{Section~\ref{sec:Even-two-infinities}}, so we only focus on Step \eqref{psiom'}.
In Step \eqref{psiom'} we first compute the global symbols $\langle
\omega',  \eta_j\rangle = -\int_{\iota(P)}^P\eta_j$.
and then apply~\eqref{cupsymbol}.
Hence, if all computations so far are done modulo $p^n$, and we denote $m''=\ord_p\det(M)$, where $M$ is the cup product matrix, the coordinates of $\psi(\omega')$ in Step \eqref{psiom'} are computed accurately modulo $p^{n-m''}$.

All three matrices $\Frob-pI$, $\Phi-I$, and $M$ depend only on the curve. Thus, we may precompute their determinant and predict which starting
precision is necessary to reach a specified target precision.
For the approach discussed in Remark~\ref{R:aff_inf}, we also need to
compute $\det(\Frob-pI)$ and $\det(\Phi-I)$ for the curve $C'$, which depends on the point $P$. 

\begin{remark}\label{R:precspecial}

  In the precision analysis above, we assume Conditions~2
  and~3. As discussed
  in~\S\ref{subsec:weierstrass} and \S\ref{subsec:inf},
  Condition~2 is not necessary, and in order to reduce 
  to the situation treated above, only symmetry and 
tiny integrals are required.
  Hence we do not need an additional
  precision analysis in this more general situation.
\end{remark}
\end{comment}


\section{Implementation}\label{subsec:implementation}
We have implemented Algorithm~\ref{alg:hp} in {\tt SageMath}. For simplicity,
we assume that $p>g$, whenever needed (see Remark~\ref{R:pg}),
and that the branch $\log_p$ is the
branch determined by $\log_p(p)=0$. This suffices for our applications
discussed in the present article and in~\cite{LinQC}.
{\tt SageMath} is particularly suitable for our approach because there are
existing {\tt SageMath}-implementations of 
\begin{itemize}
\item Harrison's extension of Kedlaya's algorithm to
    compute $\hmw(U)$ and the action of Frobenius on $\hmw(U)^-$
    (see~\cite{Harrison-Even-Deg-MW}) and
\item Balakrishnan's algorithm from~\cite{Jen-Even-Degree-CI} to compute
  Coleman integrals
\end{itemize}
when $C$ is a monic even degree hyperelliptic curve; 
  available from~\url{https://github.com/jbalakrishnan/AWS}.
 In particular, the command \verb+coleman_integrals_on_basis+ computes the integrals
$\int_S^R\eta_0,\ldots,\int_S^R\eta_{2g}$ efficiently if $R$ and $S$ are $\Q_p$-points in
affine discs. 
This made it easy to implement the algorithm outlined 
in Sections~\ref{sec:Even-two-infinities} and~\ref{sec:affine}
for points $P,Q,R,S\in C(\Q_p)$
that satisfy \sg{Assumption~\ref{assumptions}}. Recall that by~\S\ref{subsec:weierstrass} we can drop Condition~2 \sg{for both pairs $(P,Q)$ and $(R,S)$} provided
Condition~1 is satisfied. 


We require Condition~1 because we use Lemma~\ref{L:change} for
Step \eqref{ints}
of Section~\ref{sec:affine}, since it is much simpler than the one based on
Equation~\eqref{E:4.8}. The latter would require
an implementation of a generalization of the rather complicated Algorithm~4.8 from~\cite{BBHeights} to even
degree, which we did not attempt. In particular, the computation of the residues in~\eqref{E:4.8} requires
computations to high precision in annuli with Laurent series that have an essential
singularity and computations over extensions of $\Q_p$ that can have degree as large as
$p$, or even larger.
While this would lead to a more general implementation than our present one, we expect it to be  
significantly slower.
    We have also implemented the approach discussed in
    Remark~\ref{R:aff_inf} in {\tt SageMath}. For a single  height computation,
    its performance is similar to that of the implementation discussed in
    the present subsection. However, when we need to compute several local
    heights on the same curve, then the approach from
    Remark~\ref{R:aff_inf} requires precomputations for several different
    curves that show up in this approach, making it less efficient.



In practical applications, such as quadratic Chabauty or the computation of the $p$-adic
regulator, we often have some freedom in choosing representatives, so that an algorithm
for points satisfying Condition~1 usually suffices. 



%%%%%%%%%%%%%%%%%%%%%%%%%%%%%%%%%%%%%%%%%%%%%%%%%%%%%%%%%%%%%%%%%%%%%%%%
\subsection{Timings}\label{subsec:timings}
To illustrate the performance of our algorithm we present timings in
Table~\ref{tab:timings} by computing $h_p(P-Q, R-S)$ for the following curves $C_g$ of respective genus
$g$ and various primes $p$:
\begin{align*}
  C_2&\colon y^2=x^5+(x-1)(x-4)(x-9)(x-16)\\
    &P=(1,1),\, Q=(4,2^5),\,R=(9,3^5),\,S=(16,4^5) \in C_2(\Q_p)\\
  C_3&\colon y^2=x^7+(x-1)^2(x-4)^2(x-9)(x-16)\\ 
    &P=(1,1),\, Q=(4,2^7),\,R=(9,3^7),\,S=(16,4^7) \in C_3(\Q_p)\\
  C_4&\colon y^2=x^9+(x-1)^2(x-4)^2(x-9)^2(x-16)^2\\ 
    &P=(1,1),\, Q=(4,2^9),\,R=(9,3^9),\,S=(16,4^9) \in C_4(\Q_p)\\
  C_{17}&\colon y^2=x^{35}+(x-1)^2(x-4)^2(x-9)^2(x-16)^2\\
    &P=(1,1),\, Q=(4,2^{35}),\,R=(9,3^{35}),\,S=(16,4^{35}) \in C_{17}(\Q_p)
\end{align*}
For example, we find that on $C_2$ we have
\[h_7(P-Q,R-S)= 5\cdot7 + 4\cdot7^3 + 6\cdot7^4 + 3\cdot7^5 + 3\cdot7^6 +
4\cdot7^7 + 4\cdot7^8 + 6\cdot7^9 + O(7^{10})\,.
\]

\begin{table}[t]
\begin{tabular}{| l | c | c | c | c | }
\hline
  Curve & $p$ & Precision & Our time & \cite{BBHeights} \\
\hline 
$C_2$ & 7 & 10 & 2s & 7s  \\ 
$C_2$ & 7 & 300 & 11m &  \\ 
$C_2$ & 503 & 10 & 4m & 19h \\ 
$C_3$ & 11 & 10 & 6s & 28s \\
$C_4$ & 23 & 20 & 2m & 46m \\
$C_{17}$ & 11 & 7 & 14m & \\ \hline
\end{tabular}
  \caption{Timings}\label{tab:timings}
\end{table}
The examples were computed on a single core of a 4-core 2.6 GHz Intel
i7-6600U CPU with 8GB RAM, running {\tt SageMath~9.8} on {\tt  Xubuntu 20.04}.
Whenever possible, we compare our
implementation against the implementation of 
the algorithm 
from~\cite{BBHeights} in \texttt{SageMath} (see
  \url{https://github.com/jbalakrishnan/AWS}).
We found that our implementation performed better in all examples. 
It can handle fairly large genera, primes and
precisions, whereas the \texttt{SageMath}-implementation  
of the algorithm from~\cite{BBHeights} is limited to genus at most~4 and
did not finish the computation of $h_{7}(P-Q,R-S)$ on $C_2$ to $300$ digits
of precision in one week.
Code for these timings can be found at~{\url{https://github.com/StevanGajovic/heights_above_p}}. 



\begin{remark}\label{R:extensions}
  The main reason for assuming that our points are defined over $\Q_p$ is
  that for even degree hyperelliptic curves, no implementation of Coleman
  integration over proper extensions of $\Q_p$ is currently available. Once such
  an implementation becomes available, it should be possible to extend 
  our algorithm without major problems.
    For instance, Best~\cite{Alex-ColemanIntegration-Unramified-Superelliptic} has implemented Coleman integration for superelliptic
    curves $y^n=f(x)$ with $\gcd(n,\deg(f))=1$ in {\tt Julia}, extending the approach of
    Balakrishnan--Bradshaw--Kedlaya~\cite{BBK} for odd degree hyperelliptic curves. Notably, his implementation
    works for unramified extensions of $\Q_p$. The special case of an extension to
    $\gcd(n,\deg(f))\ne 1$ would allow us to extend our algorithm in practice to points
    $P,Q,R,S$ defined over unramified extensions of $\Q_p$.
    Moreover, in forthcoming work, Best, Kaya and Keller will extend the general
    Coleman integration algorithm from~\cite{BalakrishnanTuitman} and give
    a {\tt Magma}-implementation, extending the one
for Coleman integrals over $\Q_p$ available 
    from~{\url{https://github.com/jtuitman/Coleman}}.
\end{remark}


\subsection{Implementation - special cases}\label{subsec:implementation_special_cases}

\sg{We now explain how to compute $h_p(P-Q, R-S)$ when Condition 2 is not
satisfied for either $(P,Q)$ or $(R,S)$}. Recall that we still assume that $P,Q,R,S\in C(\Q_p)$ satisfy Condition 1.

%%%%%%%%%%%%%%%%%%%%%%%%%%%%%%%%%%%%%%%%%%%%%%%%%%%%%%%%%%%%%%%%%%%%%%%%
\subsubsection{Weierstrass and infinite discs}\label{subsec:weierstrass}
We can deal with the case that $R$ or $S$ belongs to a Weierstrass disc as
follows. Suppose that $D(R)$ contains a Weierstrass point $T$ and that
$\omega$ has no singularity at $T$. This is satisfied in our applications of Algorithm~\ref{alg:hp}.
Then
$\int_S^R \omega= \int_S^T \omega + \int_T^R \omega$. The second integral is tiny,
   so we can compute it. Since we integrate odd forms, we can also compute the first integral, using $\int_S^T \omega=\frac{1}{2}\int_S^{\iota(S)} \omega$ (see \cite[Lemma 16]{BBK}). 
 We could therefore replace 
  Condition~2 by the weaker condition that
   $\ord_p(y(R))\geq 0$ and $\ord_p(y(S))\geq 0$.

By symmetry, it follows that we can compute $h_p(P-Q,R-S)$ if
$\ord_p(y(P)),\ord_p(y(Q))\geq 0$ or if  $\ord_p(y(R)),\ord_p(y(S))\geq 0$.
Lemma \ref{L:symm-antisymm} implies that it is enough to compute heights of
the  type $h_p(P-\iota(P),R-\iota(R))$. Condition 1 implies that
it is not possible to have both $\ord_p(y(P))<0$ and $ \ord_p(y(R)) <0$ at
the same time. Hence, without loss of generality, the only remaining case
is when we have $\ord_p(y(P))\ge 0$, $\ord_p(y(R)) <0$, and $R\in D(\infty_-)$. Then
\begin{multline*}
h_p(P-\iota(P),R-\iota(R))=h_p(P-\iota(P),\infty_- - \infty_+)\\ +
  h_p(P-\iota(P), R - \infty_-) -  h_p(P-\iota(P), \iota(R) - \infty_+)\,.
\end{multline*}
We find $h_p(P-\iota(P),\infty_- - \infty_+)$ using the method from
Section~\ref{sec:Even-two-infinities}. The integrals that show up in Step
\eqref{psiom'} and Step \eqref{ints} when computing $h_p(P-\iota(P), R -  \infty_-)$ and $h_p(P-\iota(P), \iota(R) - \infty_+)$ are tiny, so we can  compute all heights on the right hand side.

 Hence we only have to
assume Condition 1 when the points $P,Q,R,S$ are affine. 
We have implemented this approach in \texttt{SageMath}. 
%%%%%%%%%%%%%%%%%%%%%%%%%%%%%%%%%%%%%%%%%%%%%%%%%%%%%%%%%%%%%%%%%%%%%%%%
\subsubsection{Points at infinity}\label{subsec:inf}
When exactly one of the points $P,Q,R,S$ (satisfying Condition 1) is a point at infinity, we compute
  the local height using a similar strategy. Namely, we have
  \begin{align*}
    h_p(\infty_- - Q,R-S)
     =&\dfrac{1}{2}\left(\log_p\left(\dfrac{x(S)-x(Q)}{x(R)-x(Q)}\right)\right.
     \\&+ \left.h_p(\infty_- - \infty_+,R-S)-  h_p(Q-\iota(Q),R-S) \right).
  \end{align*}
The case $P=\infty_+$ is analogous, since $h_p(\infty_+ - Q,R-S)=h_p(\infty_- -\iota(Q),\iota(R)-\iota(S))$. 
We have also implemented this case in \texttt{SageMath}.

%%%%%%%%%%%%%%%%%%%%%%%%%%%%%%%%%%%%%%%%%%%%%%%%%%%%%%%%%%%%%%%%%%%%%%%%
\subsubsection{Weakening Condition~1 to Condition~1'}\label{subsec:weakening}
We now explain how Condition~1 may be weakened. However, we have
not implemented this, so in our implementation 
we assume Condition~1.

Assume that the points $P,Q,R,S$ do not satisfy Condition~1, but do
satisfy Condition~1'.
%\begin{cond}\label{cond1}
%The points $P_1,P_2,P_3,P_4$ satisfy $\{\iota(P_1),\iota(P_2)\}\cap
%  \{P_3,P_4\} = \emptyset$. 
%\end{cond}
By Remark \ref{R:aff_inf}, it is enough to consider the approach in
Section~\ref{sec:Even-two-infinities}, where Condition~1
is required to perform Step~\eqref{ints}. 
One possible approach is to use the existing {\tt Magma} implementation of the algorithm
for Coleman integrals on general 
curves due to Balakrishnan and
    Tuitman~\cite{BalakrishnanTuitman}. This algorithm directly computes Coleman integrals with
    respect to a basis of $\hdr(C/\Q_p)$. However, it currently cannot be used
    to compute integrals of differentials that are not essentially of the second kind; in particular,
    we cannot compute integrals of the form $\int^R_S\omega_g$ directly.  Instead, we could
    follow the strategy from~\cite{BBHeights} to carry out Step~\eqref{ints}, by using \sg{the differential $\alpha=\phi^{*}(\omega_g)-p\omega_g$. which is essentially of the second kind.}  Hence
    $\int_R^S \alpha$ can be computed using the existing {\tt Magma} implementation even when $R$ or $S$ belong to $D(\infty_-)$
    or $D(\infty_+)$.
    We obtain the required integral from
    \[
    \int_S^R\omega_g=\dfrac{1}{1-p} \left(\int_S^R\alpha - \int_{\phi(S)}^S\omega_g - \int_R^{\phi(R)}\omega_g\right).
    \]
The integrals  \sg{$ \int_{\phi(S)}^S\omega_g$} and \sg{$\int_R^{\phi(R)}\omega_g$} are tiny integrals,
  hence they can be computed easily. 

  \begin{remark}\label{R:} 
As remarked in Step \eqref{ints} of Section~\ref{sec:affine}, if the
endpoints of a Coleman integral are in the same residue disc, we can
compute it as a tiny integral. Hence it would be also possible to weaken
Condition~1 by allowing that $R$ and $S$ (or by symmetry, $P$ and
$Q$) are in the same residue disc. But as this is a minor improvement
comparing to weakening Condition~1 to Condition~1' and
was not necessary for our applications, we have not implemented this.
  \end{remark}




\section{Numerical evidence for $p$-adic BSD}\label{S:pbsd}

For elliptic curves $E/\Q$ of good ordinary reduction at a prime $p$,
Mazur--Tate--Teitelbaum propose 
\footnote{In fact, Mazur--Tate--Teitelbaum also give conjectures for bad
semistable reduction; a supersingular conjecture is due to Bernardi and
Perrin--Riou~\cite{BPR93}.} 
a $p$-adic conjecture of Birch and
Swinnerton--Dyer type in~\cite{MTT86}. It relates the special value of the 
analytic $p$-adic $L$-function $L_p(E,s)$ of
Mazur and Swinnerton--Dyer~\cite{MSD74} to the $p$-adic regulator
defined via the canonical $p$-adic height of Mazur--Tate from \cite{Mazur-Tate-p-adic-heights}. They also give numerical
evidence for their conjecture; for further evidence see~\cite{SW13}.
Together with Balakrishnan and Stein, the second author has
extended this conjecture
in~\cite{BMS16} to the case of an abelian variety $A/\Q$ with good ordinary
reduction at a prime $p$ such that $A$ is modular
of $\mathrm{GL}_2$-type, associated to $d=\dim(A)$ conjugate newforms
$f_1,\ldots, f_d \in S_2(\Gamma_0(N))$. 

To formulate the conjecture, we recall some notation
from~\cite{BMS16}.
Let $\Reg_p(A/\Q)$ denote the $p$-adic regulator of $A(\Q)$, defined via
the canonical Mazur--Tate height and the cyclotomic idèle class character,
normalized as in~\cite[Definition~3.3]{BMS16}.
Let $L_p(A,s)\colonequals L_p(f_1,s)\cdots L_p(f_d,s)$, where the
Mazur--Swinnerton--Dyer $p$-adic $L$-functions $L_p(f_i,
s)$ are normalized with respect to Shimura periods that
satisfy~\cite[Theorem~2.4]{BMS16}. 
Let $K$ denote the (totally real)
field generated by the Hecke eigenvalues $a_{p,i}$ of the $f_i$ and fix an embedding
$\sigma\colon K\hookrightarrow\C$ and a prime $\p\mid p$ of $K$. Let $\alpha_i$ be the unit root of 
$x^2-\sigma(a_{p,i})x+p \in K_\p[x]$ and define 
the $p$-adic multiplier $\epsilon_p(A)\colonequals \prod_{i=1}^d(1-\alpha^{-1}_i)^2$.  
Finally, let $c_v$ denote the
    Tamagawa number at a finite place $v$ of $\Q$, $\Sha(A/\Q)$ the
    Shafarevich--Tate
group of $A/\Q$ and $A(\Q)_{\tors}$ the torsion subgroup of $A(\Q)$.
Then a slightly rewritten and specialized version
of~\cite[Conjecture~1.4]{BMS16} reads as follows:

\begin{conj}\label{pbsd}
Let $A/\Q$ be a modular abelian variety of  $\mathrm{GL}_2$-type, with
good ordinary reduction at a prime $p$.
Then the Mordell--Weil rank $r$ of $A/\Q$ equals
      $\ord_{s=1}L_p(A,s)$ and we have
      \begin{equation*}\label{eq-pbsd}{L}_p^*(A,1)
        =\frac{\epsilon_p(A)}{\log_p(1+p)^r}
\cdot\frac{ |\Sha(A/\Q)|
      \cdot \Reg_{p}(A/\Q) \cdot \prod_v  c_v
}{|A(\Q)_{\tors}|\cdot|A^{\vee}(\Q)_{\tors}|},\end{equation*} where
  $L_p^*(A,1)$
is the leading coefficient of $L_p(A,s)$ at $s=1$. \end{conj}

We now discuss how to gather empirical evidence for Conjecture~\ref{pbsd}. The $p$-adic
$L$-functions $L_p(f_i,s)$ can be approximated using an algorithm due to
Pollack--Stevens (see~\cite{PS11}) based on overconvergent modular
symbols. 
{\tt SageMath} contains an implementation of this algorithm;
currently, it requires that the prime $p$ splits in $K$. 
A forthcoming {\tt Magma}
implementation of this algorithm that does not have this restriction is due
to Keller.
The Tamagawa numbers can be computed using work of van Bommel~\cite{vB22}.
 In~\cite{KS22, KS} Keller and Stoll announce algorithms to compute
 $|\Sha(A/\Q)|$ and the verification that $|\Sha(A/\Q)|=1$ for a
 number of geometrically simple abelian surfaces.
However, if we cannot compute $|\Sha(A/\Q)|$, we can still
try to compute all other quantities in Conjecture~\ref{pbsd}
and test whether the quotient
\begin{equation*}\label{}
  \frac{{L}_p^*(A,1)\cdot \log_p(1+p)^r\cdot |A(\Q)_{\tors}|\cdot|A^{\vee}(\Q)_{\tors}|}
{\epsilon_p(A)
      \cdot \Reg_{p}(A/\Q) \cdot \prod_v  c_v
}
\end{equation*}
is an integer and whether it agrees with what we expect $|\Sha(A/\Q)|$ to
be.

Suppose that $A$ is the Jacobian of a smooth projective geometrically
integral curve $X/\Q$, satisfying the
conditions of Conjecture~\ref{pbsd}.
By \cite[Theorem 3.3.1]{Coleman-bi-extension-p-adic-heights}, the
canonical Mazur--Tate height equals the Coleman--Gross height with respect
to the unit root subspace (both taken with respect to the same idèle
class character).
If $X$ is a hyperelliptic curve given by an odd degree model over $\Q_p$,
then one can compute $\Reg_p(A/\Q)$ using the algorithm of
Balakrishnan--Besser from \cite{BBHeights} (and {\tt Magma}'s {\tt
LocalIntersectionData})
Using this approach, evidence for Conjecture~\ref{pbsd} is discussed
in~\cite[Section~4]{BMS16} for the Jacobians of~16 genus-2 curves having
rank~2 and for all good ordinary primes $p<100$ such that the curve has an odd degree model
over $\Q_p$.

Using our Algorithm~\ref{alg:hp}, we can gather empirical evidence for 
Conjecture~\ref{pbsd} for hyperelliptic curves $X/\Q$ with modular
Jacobian of GL$_2$-type, and good ordinary primes $p$ such that $X$ has no odd degree model
over $\Q_p$. 

\begin{example}\label{E:67bsd}
 Consider the Atkin--Lehner quotient $X\colonequals X_0^+(67)$,
given by the equation 
  \begin{equation*}\label{67eqn}
X\colon y^2 = x^6+4x^5+2x^4+2x^3+x^2-2x+1\,,
  \end{equation*}
and let $A$ be its Jacobian.
  It was shown in~\cite{FLSSSW01} that
  \begin{equation}\label{tametc}
    \mathrm{rk}(A/\Q)=2,\quad c_v=1\text{ for all }v,\quad
    |A(\Q)|_{\mathrm{tors}}=1
  \end{equation}
 and that the full (classical) Birch and Swinnerton--Dyer conjecture
holds numerically for $A/\Q$  if and only if $|\Sha(A/\Q)|=1$. The latter was recently 
verified by Keller and Stoll~\cite{KS22, KS}.
In~\cite{BMS16}, Conjecture~\ref{pbsd} was verified numerically for the
primes $p=7,13,17,19,37,41,43,47,59,61,73,79,83$. These are all good
ordinary primes $p<100$ such that $X$ has a quintic model over $\Q_p$.

  We now discuss the verification of Conjecture~\ref{pbsd} for $p=11$.
  The Fourier coefficients of the newforms corresponding to $A$ lie in
  $K=\Q(\sqrt{5})$. 
  Since $p=11$ splits over $K$, we can use the {\tt SageMath} implementation of
  the overconvergent algorithm of Pollack and Stevens to compute
  $L_{11}^*(A,1)$. According to~\cite[Table 4.4]{BMS16}, this involves a 
  normalization factor $\delta^+$, which is independent of $p$. In the present example, we have $\delta^+=1/4$ by~\cite[Table~4.4]{BMS16}. We find that
  \begin{equation}\label{l11}
    L_{11}^*(A,1) = 10 + 11 + 3\cdot 11^2 + 10\cdot 11^3 + O(11^4)\,.
  \end{equation}
  The $11$-adic multiplier is
  \begin{equation}\label{eps11}
  \epsilon_{11}(A)=9 + 8\cdot 11 + 11^2 + 5\cdot 11^3 + O(11^4)\,.
  \end{equation}
  It remains to compute $\Reg_{11}(A/\Q)$. The Mordell--Weil group $A(\Q)$
  is generated by $P_1 = [(0,1) - \infty_-]$ and $P_2 = [(0,1)-(0,-1)]$.
  Since $ x^6+4x^5+2x^4+2x^3+x^2-2x+1$ has no root in
  $\F_{11}$, we cannot use the algorithm from~\cite{BBHeights}.
  Instead, we apply Algorithm~\ref{alg:hp} to compute the $11$-adic heights
  $h(P_1,P_1), h(P_1,P_2)$ and $h(P_2,P_2)$; we have 
$$
  \Reg_{11}(A/\Q) = h(P_1,P_1)h(P_2,P_2)-h(P_1,P_2)^2\,.
$$
  To do so, we find suitable multiples of $P_1$ and $P_2$ 
that have representatives
  satisfying the
  conditions of Algorithm~\ref{alg:hp} and of the algorithm
  in~\cite{Mul14}.
  For instance, $3P_1$ and $10P_1$ have representatives $D_{1}$ and $D'_1$,
  respectively, such that 
  \begin{itemize}
    \item $D_1$ and $D'_1$ have disjoint support;
    \item $D_1\otimes \Q_{11}$ is of the form $Q_1+Q_2-R_1-R_2$, where
      $Q_1,Q_2\in X(\Q_{11})$ are in affine discs and $R_{1}, R_2$ are the
      points with $x$-coordinate~1; 
    \item $D'_1\otimes \Q_{11}$ is of the form $Q'_1+Q'_2-R'_1-R'_2$, where
      $Q'_1,Q'_2\in X(\Q_{11})$ are in affine discs and $R'_1, R'_2$ are the
      points with $x$-coordinate~2; 
    \item the tuples $(Q_i,R_i, Q'_j, R'_j)$ satisfy Condition~1
      for all $i,j\in \{1,2\}$.
  \end{itemize}
  Hence we may compute $h_{11}(D_1,D'_1)$ using Algorithm~\ref{alg:hp} and
  $h_{q}(D_1,D'_1)$ for $q\ne p$ using {\tt Magma}'s {\tt 
LocalIntersectionData}, and we find
$$
h(P_1, P_1) = \frac{1}{30}(h_{11}(D_1,D'_1) -\log_{11}(2)
-2\log_{11}(3)+2\log_{11}(5)+\log_{11}(17)-\log_{11}(317))\,.
$$
We compute $h(P_1,P_2)$ and $h(P_2,P_2)$ using the same strategy and obtain
$$
\Reg_{11}(A/\Q)=6\cdot 11^2 + 3\cdot 11^3 + 7\cdot 11^4 + O(11^5)\,.
$$
Together with~\eqref{tametc},~\eqref{l11} and~\eqref{eps11}, this suffices
to show that Conjecture~\ref{pbsd} holds to 4 digits of precision. 
In fact, we verified the conjecture to 10 digits of precision, and we
did the same for the primes $p=29,31,53,71,89$.
The code for these computations can be found
at~{\url{https://github.com/StevanGajovic/heights_above_p}}. 



\end{example}





\bibliographystyle{alpha}
\bibliography{References}

\end{document}

\begin{comment}
\begin{corollary}
Let $[\alpha]\in \hdr(C/\Q_p)$ be the class of $\alpha$. Recall that $\phi^*\omega_g=\sum_{i=0}^{2g}f_{0,i}\omega_i$. Then the coordinate vector of $[\alpha]$ with respect to the basis $[\eta_0],\ldots,[\eta_{2g-1}]$ is $\begin{pmatrix}
f_{0,0} & \cdots & f_{0,g-1} & f_{0,g+1} \cdots & f_{0,2g}
\end{pmatrix}^t$ and
\begin{equation*}\label{Psi-Omega-From-Alpha}
\psi(\omega')=(\Frob-pI)^{-1}[\alpha]\,.  
\end{equation*}
\end{corollary}
\begin{proof}
Harrison's extension \cite{Harrison-Even-Deg-MW} of Kedlaya's algorithm \cite{Kedlaya-MW-reduction} shows that there is a certain overconvergent function $v$ such that we have 
\[
\alpha=\phi^*(\omega')-p\omega'=\sum_{0\leq i\leq 2g, i\neq g}f_{0,i}\omega_i + (f_{0,g}-p)\omega_g + dv\,.
\]
Since $\alpha$ is of the second kind by Proposition~\ref{P:alphasecond}, we obtain
$\alpha=\sum_{0\leq i< 2g}f_{0,i}\eta_i+dv$ and by the comment after Lemma~\ref{L:psi}, we have
$       \psi(\alpha)=[\alpha]\in \hdr(C/\Q_p)$.
Combining this with $\psi(\phi^*\omega')={\phi^*}\cdot \psi(\omega')$, the result follows. 
\end{proof}
\end{comment}
