Medical image registration involves estimating the optimal spatial transformation to align the structures of interest in a pair of fixed and moving images.
The choice of spatial transformation depends on the specific application and can be categorized as either rigid/affine or non-rigid/deformable.
In rigid/affine registration, all spatial coordinates are transformed using the same rigid/affine matrix.
On the other hand, non-rigid/deformable registration employs independent transformations for individual local regions of spatial coordinates.
Both types of registration are of great importance to many medical imaging tasks.
Rigid registration is commonly used when the rigid body assumption holds.
For example, it is used to align a structural scan---\emph{e.g.}, magnetic resonance image~(MRI) or computer tomography~(CT)---with a functional scan---\emph{e.g.}, functional magnetic resonance image~(fMRI) or positron emission tomography~(PET)---of the same patient for attenuation correction~\citep{hofmann2008mri} or interpretation of functional activities~\citep{studholme2000accurate}.
On the other hand, deformable image registration~(DIR) is often used in cases where more complex, spatially varying deformations are needed.
Examples of such applications include constructing deformable templates for a patient
cohort~\hbox{\citep{christensen1996deformable, ganser2004deformable}}
or registering atlases to a patient image for multi-atlas
segmentation~\citep{reed2009automatic, cabezas2011review,
aljabar2009multi}.

% Figure environment removed

Traditionally, image registration has been accomplished by iteratively solving an optimization problem (\emph{e.g.}, demons~\citep{vercauteren2009diffeomorphic}, LDDMM~\citep{beg2005computing}, SyN~\citep{avants2008symmetric}, DARTEL~\citep{ashburner2007fast}, and Elastix~\citep{klein2009elastix}).
These methods are well-established and supported by strong mathematical theory. However, they can be computationally expensive and slow in practice, as the optimization problem must be solved for each individual pair of moving and fixed images. Several review papers have covered traditional medical image registration methods extensively~\citep{maintz1998survey, hill2001medical, shams2010survey, fluck2011survey, sotiras2013deformable, oliveira2014medical, viergever2016survey}.
Interested readers can refer to these references for more information on these methods. In the last decade, deep learning-based methods have shown promise in improving the accuracy and efficiency of image registration.
Unlike traditional methods, deep learning-based methods train a general network by optimizing a global objective function on a training dataset.
Then in the testing phase, the trained network is directly applied to each image pair with the fixed network weights, resulting in a significant speedup compared to traditional methods.
Initially, ResNet-like network architectures~\citep{he2016deep} were explored, which consist of a convolutional encoder and a multilayer perceptron~\citep{wu2013unsupervised, miao2016cnn}.
During the training process, ground truth transformations have to be provided for direct supervision.
In rigid/affine transformations, the ground truth is
represented as a transformation matrix; while a dense displacement field is often used for deformable registration.
Subsequently, with the introduction of spatial transformer networks~\citep{jaderberg2015spatial} and the success of U-Net~\citep{ronneberger2015u} in medical imaging applications, learning-based deformable registration methods adopted an encoder-decoder design in either supervised~\citep{yang2017quicksilver, rohe2017svf} or unsupervised~\citep{vos2017end, li2018non, balakrishnan2019voxelmorph, kim2021cyclemorph, chen2022transmorph} training schemes. These methods typically output a high-resolution dense deformation field. On the other hand, learning-based rigid/affine registration methods continue to adopt encoder-only networks~\citep{miao2016cnn, hu2018label, de2019deep,chen2021learning, chen2022transmorph, mok2022affine}, with the output being the rigid or affine parameters. While there are papers that provide general reviews of learning-based registration methods~\citep{fu2020deep, chen2021deep, xiao2021review, zou2022review}, it is important to note that these reviews may not be fully up-to-date due to the rapid advancement of the field of deep learning. Recent advancements, including learning-based similarity metrics and regularizers, novel network architectures, and innovative evaluation metrics and uncertainty estimation methods, have demonstrated promising potential for medical image registration. This paper provides a timely review of learning-based methods in medical image registration, highlighting the latest technologies that have been proposed and discussing their respective characteristics and applications. In addition, we investigate and formally define registration uncertainty for deep learning-based image registration and address the appropriate evaluation metrics for these methods that have been overlooked in previous review papers. For simplicity, we refer to deep learning-based methods as learning-based methods throughout the paper.

In this survey paper, we analyze over 250 articles on learning-based
medical image registration. As depicted in Fig.~\ref{fig:paper_count},
the focus is primarily on recent advancements proposed in the last
five years. Our search covers well-established medical imaging
journals, such as Medical Image Analysis, IEEE~Transactions on Medical
Imaging, Medical Physics, and NeuroImage, as well as conference
proceedings related to medical imaging and image registration, such as
MICCAI, IPMI, WBIR, CVPR, ECCV, ICCV, and NeurIPS. The remainder of
the paper is organized as follows:
Section~\ref{sec:Fund_Img_Reg}~offers a brief overview of the
fundamentals of learning-based image registration.
Section~\ref{sec:loss}~explores widely-used loss functions for
learning-based registration methods which resemble objective
functions in traditional methods, and discusses other novel loss functions
enabled by deep learning. Section~\ref{sec:net_arch}~investigates
network architectures developed for medical image registration,
with a focus on recent developments.  Section~\ref{sec:uncertainty}~delves
into methods for estimating registration uncertainty in learning-based
registration.  Section~\ref{sec:Eval_Metric}~considers appropriate
evaluation metrics for learning-based methods and examines methods for
quantifying the regularity of generated deformation fields.
Section~\ref{sec:Application}~summarizes recent applications of
learning-based registration in medical imaging. Finally,
Section~\ref{sec:future_persp}~discusses current challenges and
provides future perspectives for deep learning in medical image
registration.
