% \section{Registration Evaluation Metrics}
Manual correspondences are usually regarded as the gold standard for evaluating the performance of a registration algorithm~\citep{peter2021uncertainty}.
Landmark correspondences are the most frequently used type, although surfaces or lines may also serve as manual correspondences.
The evaluation of registration performance using landmark correspondences is relatively simple for rigid and affine transformations, since these transformations can be expressed as matrix multiplication and the ground truth transformation can be determined through several pairs of manual landmark correspondences.
In contrast, determining the parameters of deformable transformations requires dense manual landmark correspondences, which are typically not obtainable.
Even in cases where manual landmark correspondences are available, they are often restricted to highly selective intensity features~\citep{castillo2009framework} and neglect regions with homogeneous intensities.
Therefore, validating deformable registration algorithms is still considered a non-trivial task~\citep{viergever2016survey}.
In current literature, the performance of deformable registration algorithms is most commonly evaluated in terms of accuracy and regularity.
\subsection{Accuracy Measures}
When the manual landmark correspondences are available, the accuracy of the transformation can be evaluated by target registration error~(TRE),
\begin{equation}
    \text{TRE}_{\text{forward}} = \sum_{i=1}^{N} ||T_{\text{forward}}(l_m^i) - l_f^i||_k,
    \label{eqn:tre_forward}
\end{equation}
where $T_{\text{forward}}$ is the estimated forward transformation that takes the moving image to the fixed image; $l_m^i$ and $l_f^i$ are the $i^{\,\text{th}}$ pair of landmarks in the moving and fixed image, and $k\in\{1,2\}$ denoting either the $\ell^1$-norm or $\ell^2$-norm.
Both $l_m^i$ and $l_f^i$ as well as the warped moving landmark $T(l_m^i)$ can be non-integer locations.
Note that we used the forward transformation $T_{\text{forward}}$ in Eqn.~\ref{eqn:tre_forward}, but it is more common in practice to estimate the transformation $T_{\text{backward}}$ that maps the fixed image to the moving image.
In order to generate the warped image, $T_{\text{backward}}^{-1}$ can be applied in place of $T_{\text{forward}}$.
Both $T_{\text{forward}}$ and $T_{\text{backward}}$ are mappings from integer locations to non-integer locations.
The difference between these two schemes is manifested when rendering the warped image.
When $T_{\text{forward}}$ is applied to $I_m$, integer locations are mapped to non-integer locations, which necessitates interpolating scattered data~\citep{crum2007methods,zhuang2008atlas}.
On the other hand, $T_{\text{backward}}^{-1}$ maps non-integer locations back to integer locations.
Thus, rendering the warped image only requires interpolating the moving image, which is defined on a regular grid.
For algorithms that only output $T_{\text{backward}}$, TRE can be computed as
\begin{equation}
    \text{TRE}_{\text{backward}} = \sum_{i=1}^{N} || l_m^i - T_{\text{backward}}(l_f^i) ||_k.
    \label{eqn:tre_backward}
\end{equation}

Landmark correspondences can also be generated using artificial deformations~\citep{bauer2021generation,sdika2008fast,ger2017accuracy}. Different from manual landmark correspondences, artificial deformation can produce dense correspondences that are not limited to regions with highly selective intensity features.
However, the performance of algorithms on artificial deformation may not accurately reflect their actual performance due to the discrepancy between the artificial and real deformations~\citep{obeidat2016comparison, pluim2016truth}.
To overcome this issue, many works have been focused on generating deformations that are more akin to those observed in practical applications.
For instance, \citet{lobachev2021evaluating}~proposed a pipeline for simulating sectioning-induced deformation fields.
\citet{vlachopoulos2015selecting}~used a thin-plate kernel spline model to simulate lung deformations arising from respiration. Biomechanical simulation~\citep{fu2021deformable, teske2017construction} and phantoms~\citep{wu2019characterization,ayyalusamy2021performance} are other techniques used to generate artificial deformations.

In situations where manual landmark correspondences are not available, surrogate measures are used to evaluate accuracy.
The most straightforward measures of this kind include absolute intensity differences and the root-mean-square intensity difference between the warped image and the fixed image.
Other similarity measures such as mutual information, structural similarity index~(SSIM) can also be used.
When anatomic labels are available, evaluating the overlaps between the warped and fixed label images is a popular technique. The Dice coefficient and Jaccard Index are examples of such measures.
However, \citet{rohlfing2011image}~demonstrated that by simply reordering the voxels from the moving image based on the intensity values ranking without any geometrical constraints, one can achieve significantly better performance compared with the state-of-the-art registration algorithms in most of the surrogate measures.
They concluded that surrogate measures might still be useful to detect inaccurate registrations but many times they do not provide sufficient positive evidence for accurate registrations.
Only the overlap of sufficiently local labels among the surrogate measures was found to distinguish between reasonable and poor registrations in their experiments.

Label surface distances from segmentation maps offers an alternative to overlap measures.
\citet{dalca2019unsupervised}~converted segmentation maps into signed distance functions to approximate the distance between the fixed and warped surfaces.
They also showed that incorporating a similar surface distance loss during network training enhanced the surface alignment of anatomical structures.
\citet{cheng2020cortical}~used the mean minimum distance~(MMD), computed as the average Euclidean distance between manually defined surface points of anatomical structures and their corresponding nearest points on the warped surface, to measure the discrepancy between the surfaces.
Additionally, the Hausdorff distance has been extensively used~\citep{hering2022learn2reg}.

Previous studies \citep{lotfi2013improving, sokooti2016accuracy} have explored the use of machine learning algorithms for quantifying registration errors.
Compared to manual landmark correspondences, those methods provide dense error estimations that can be easily visualized.
More recently, several deep learning techniques have been employed, offering a speed advantage over traditional machine learning algorithms, especially when a graphical processing unit~(GPU) is available~\citep{sokooti2021hierarchical}.
Most of these methods were trained to predict the registration errors between a fixed and a warped image inputs.
During training, artificial deformations are used to produce the warped image and the accuracy of these methods were validated using manual landmark correspondences~\citep{eppenhof2018error,sokooti2021hierarchical}.
Additionally, these techniques can also be applied to inter-modality registration tasks by incorporating an extra image synthesis step~\citep{bierbrier2023towards}.

\subsection{Regularity Measures}
Given the challenge of acquiring dense manual landmark correspondences and the aforementioned limitation of surrogate measures, the regularity of the transformations is often used alongside accuracy measures to obtain a more comprehensive understanding of the transformations. 
The underlying assumption is that accurate transformations should be spatially smooth.
Particularly, transformations that fold the space result in physically un-realistic anatomy structures, which usually indicate errors. 
For continuous transformations, their Jacobian determinant $|J|$ must be positive everywhere to avoid folding of space.
This concept is extended to digital transformations where the number or the percentage of voxels with non-positive Jacobian determinant $|J|\leq0$ are reported to measure the irregularity~\citep{meng2022enhancing, liu2022coordinate, mok2022unsupervised, dey2022contrareg, chen2022deformer, mok2020fast, jia2021learning, wu2022nodeo}.
For a 3D transformation $T(x,y,z) = [T_x, T_y, T_z]$, the Jacobian is defined as
\begin{equation}
    J = \begin{vmatrix}
        \frac{\partial T_x}{\partial x} & \frac{\partial T_x}{\partial y} & \frac{\partial T_x}{\partial z} \\[4pt]
        \frac{\partial T_y}{\partial x} & \frac{\partial T_y}{\partial y} & \frac{\partial T_y}{\partial z} \\[4pt]
        \frac{\partial T_z}{\partial x} & \frac{\partial T_z}{\partial y} & \frac{\partial T_z}{\partial z} \\
        \end{vmatrix}.
\end{equation}

% Figure environment removed

\citet{ashburner1999high} considered the transformations to be locally affine, and the Jacobian determinant could be computed using singular value decomposition.
More generally, the Jacobian of a dense nonlinear transformation is estimated through numerical approximation using finite difference methods.
In a recent study,~\citet{liu2022finite} showed that when approximating the Jacobian using forward or backward difference, it is implicitly assumed that the digital transformations are linearly interpolated on a {tetrahedra} mesh grid.
{They found that the Jacobian determinant, when approximated using central difference, exhibits checkerboard and self-intersection problems.
Consequently, it consistently underestimates non-diffeomorphic spaces in a transformation.
In Fig.~4 of~\citet{liu2022finite}, the authors show examples of the checkerboard problem and the self-intersection problem in 2D. 
In both cases, the central difference approximated Jacobian determinants are positive, but the underlying transformations introduce folding of space~(under the assumption that the digital transformations are piecewise linear).
To address these issues, \citet{liu2022finite}~proposed the
non-diffeomorphic volume measurement, which uses a combination of forward and backward differences to measure the {folding implied by a displacement field based on the volume of folded tetrahedra in the mesh grid.}}
%volume of the folded tetrahedra mesh grid.}
Importantly, \citet{liu2022finite}~showed that the Jacobian determinant, when approximated using central difference, results in the checkerboard and self-intersection problems. Consequently, it consistently underestimates non-diffeomorphic spaces.
Figure~\ref{fig:central_difference} shows examples of the checkerboard problem and the self-intersection problem in 2D.
In both cases, the central difference approximated Jacobian determinants are positive, but the underlying transformations introduce folding of space~(under the assumption that the digital transformations are piecewise linear).
They conclude that for a 2D transformation, four unique finite difference approximations of $|J|$'s must be positive to ensure the entire domain is invertible and free of folding; in 3D, ten unique finite differences approximations of $|J|$'s are required to be positive.
Note that their method is closely related to simplex counting~\citep{holland2011nonlinear} used in deformation-based volumetric change estimation.
Because of the issues associated with central difference approximation of $|J|$'s,~\citet{liu2022finite} recommend using non-diffeomorphic volume to accurately reflect the non-diffeomorphism introduced by transformations.

The logarithm of the Jacobian determinant is also an important measure, especially for applications where it requires the volume of the underlying anatomy to be preserved~\citep{rohlfing2003volume,jian2022weakly}. The logarithm is used to symmetrically weight local expansion and compression~\citep{rohlfing2003volume,lange2020symmetric}.
In recent works such as \citep{hering2022learn2reg,chen2022transmorph, dey2022contrareg,chen2022deformer,song2022cross}, the standard deviation of the logarithm of the Jacobian determinant has been used to quantify the smoothness of the displacement field. Additionally, the statistical distribution of the logarithm of the Jacobian determinant can be used as a visualization tool to reveal differences between registration algorithms~\citep{leow2007statistical, lange2020symmetric}.

Similar to many surrogate measures, the regularity of the transformations can detect inaccurate transformations, but by itself, it is insufficient to provide adequate positive evidence for accurate transformations. For example, the identity transformation would be deemed a perfectly regularized transformation, but it would not provide a meaningful registration.
